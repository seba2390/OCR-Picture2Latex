\documentclass[10pt,conference]{IEEEtran}
%\IEEEoverridecommandlockouts
% The preceding line is only needed to identify funding in the first footnote. If that is unneeded, please comment it out.
%\renewcommand{\baselinestretch}{1.01821}

\usepackage{cite}
\usepackage{amsmath,amsthm,amsfonts,amssymb,amsbsy,esint,cancel}

%\usepackage{algorithmic}
\usepackage{graphicx}
\usepackage{xcolor}

\usepackage{textcomp,gensymb,bbm,bm,dsfont,upgreek}

\usepackage[caption=false,font=footnotesize]{subfig}
%\usepackage[font=small,labelfont=bf]{caption}

\usepackage{url} % Improves handling and breaking URLs.
\usepackage[bookmarks=false,hidelinks]{hyperref} % colorlinks=false,citecolor=black

\usepackage[normalem]{ulem} % https://tex.stackexchange.com/questions/179691/removing-underline-from-journal-title-when-using-hyperref

\usepackage{tikz}
\tikzset{every picture/.style={font issue=\footnotesize},
	font issue/.style={execute at begin picture={#1\selectfont}}
}
\usepackage{pgfplots}
\pgfplotsset{compat=newest}
\usetikzlibrary{backgrounds,plotmarks}
\usetikzlibrary{positioning,calc} % for coordinate calculation
\usetikzlibrary{arrows,arrows.meta}
\usepackage{ulem} % zum unter- und durchstreichen

% https://tex.stackexchange.com/questions/118173/how-to-write-ceil-and-floor-in-latex
\usepackage{mathtools}
\DeclarePairedDelimiter\ceil{\lceil}{\rceil}
\DeclarePairedDelimiter\floor{\lfloor}{\rfloor}

\usepackage[capitalize]{cleveref}
	\Crefname{figure}{Fig.}{Fig.}
	\Crefname{section}{Sec.}{Sec.}
	\Crefname{subsection}{Sec.}{Sec.}
	\Crefname{prop}{Proposition}{Proposition}
	\Crefname{lemma}{Lemma}{Lemma}
	\Crefname{equation}{}{}
	\Crefname{footnote}{Footnote}{Footnote}

%\usepackage{balance} % for IEEE two-column \balance

\newcommand{\figref}[1]{Fig.~\ref{#1}}
\newcommand{\tblref}[1]{Table~\ref{#1}}
\newcommand{\secref}[1]{Section~\ref{#1}}
\renewcommand{\eqref}[1]{Equation~(\ref{#1})}

\def\availableat{\url{url-published-on-acceptance}}

\newcommand{\todo}[1]{{\color{red} TODO: {#1}}}
\newcommand{\newstuff}[1]{{\color{red} CHECK: {#1}}}
%\newcommand{\todo}[1]{{}}

\newcommand{\ckp}[2]{$CK_{#1}P_{#2}$}
\newcommand{\cext}{\ckp{8}{16}$ext$}
\newcommand{\cfin}{$F_{CK_{X}P_{Y}}$}
\newcommand{\cray}{\ckp{8}{8}$ray$}
\newcommand{\csin}{$SK_{8}P_{8}$}
\newcommand{\casin}{$SK_{combined}$}
%\renewcommand{\cext}{$SK_{8}K_{8}P_{8}$}
\newcommand{\ckpnl}[2]{$CK_{#1}P_{#2}nl$}


\makeatletter
\newcommand{\Spvek}[2][r]{%
	\gdef\@VORNE{1}
	\left(\hskip-\arraycolsep%
	\begin{array}{#1}\vekSp@lten{#2}\end{array}%
	\hskip-\arraycolsep\right)}

\def\vekSp@lten#1{\xvekSp@lten#1;vekL@stLine;}
\def\vekL@stLine{vekL@stLine}
\def\xvekSp@lten#1;{\def\temp{#1}%
	\ifx\temp\vekL@stLine
	\else
	\ifnum\@VORNE=1\gdef\@VORNE{0}
	\else\@arraycr\fi%
	#1%
	\expandafter\xvekSp@lten
	\fi}
\makeatother

\begin{document}

\newcommand\OurPaperTitle{%
%Passive Load Modulation for RFID or Backscatter Communication: Instantaneous Achievable Rates and Channel Capacity%
%Passive Load Modulation for RFID or Backscatter Communication: Achievable Rates and Capacity%
%Channel Capacity and Practically Achievable Rates of Passive Load Modulation for RFID or Backscatter Communication%
%Achievable Rates of Load Modulation in AWGN for RFID or Ambient Backscatter%
%Achievable Rates and Channel Capacity of Ambient Backscatter or RFID Load Modulation in AWGN%
%Achievable Rates and Channel Capacity of Load Modulation for RFID or Ambient Backscatter%
%Achievable Rates and Channel Capacity of Load Modulation for RFID or Ambient Backscatter%
%Load Modulation in RFID or Ambient Backscatter: Achievable Rates and Channel Capacity in AWGN%
%Load Modulation in RFID or Backscatter: Achievable Rates and Channel Capacity in AWGN%
%Load Modulation for Backscatter Communication: Achievable Rates and Channel Capacity in AWGN%
%Load Modulation for Backscatter Communication: On Achievable Rates and Channel Capacity%
%On Achievable Rates and the Channel Capacity of Load Modulation for Backscatter Communication%
%On the Channel Capacity of Load Modulation for Backscatter Communication%
%Channel Capacity of Backscatter Load Modulation with Arbitrarily Adaptive Passive Load%
%Channel Capacity of Backscatter Communication With Arbitrary Passive Load Modulation%
%The Channel Capacity of Arbitrary Passive Load Modulation for Backscatter Communication%
The Channel Capacity of General Complex-Valued Load Modulation for Backscatter Communication%
}
\title{\OurPaperTitle}
%
\author{
\IEEEauthorblockN{Gregor Dumphart, Johannes Sager, and Armin Wittneben} 
\IEEEauthorblockA{\textit{Wireless Communications Group, D-ITET, ETH Zurich, Switzerland}\\
Email: dumphart@nari.ee.ethz.ch, sagerj@student.ethz.ch, wittneben@nari.ee.ethz.ch}}

\maketitle

\begin{abstract}
%... klassische Theorie nicht anwendbar ... Abbildung auf bekanntes Problem ...
%
This paper studies achievable information rates of backscatter communication systems where the tag performs load modulation with a freely adaptable passive termination. We find that the complex phasor of the tag current is constrained to a disk and that the capacity problem can therefore be described with existing results on peak-power-limited quadrature channels. This allows us to state the channel capacity and the capacity-achieving distribution of the load impedance, which is described by non-concentric circles in the right half-plane. For the low-SNR case (SNR \(<\) 4.8 dB) we find that channel capacity is achieved by a purely reactive load with Cauchy-distributed reactance. The exposition is based on a system model that abstracts all relevant classes of backscatter communication systems, including RFID. To address practicality, we construct a symbol alphabet that allows for a near-capacity information rate of more than 6 bit per load-switching period at reasonably high SNR. We also find that the rate hardly decreases when typical value-range constraints are imposed on the load impedance.
%The capacity result applies for additive white Gaussian noise at the receiver and is obtained by combining steady-state circuit theory with the information theory of peak-power constrained channels. 
%For the backscatter case, the ambient signal is assumed to be either unmodulated or a decodable digital-modulation signal with constant envelope.
%The ambient signal is thereby assumed to be an unmodulated AC signal.
%Thereby, proper utilization of both load resistance and reactance allows for practically achievable rates in excess of 6 bit per channel use (i.e. per load-switching period).
\end{abstract}

\newcommand\OurKeywords{%
load modulation, backscatter communication, RFID, ambient backscatter, channel capacity, achievable rate, low energy communication, passive communication}
% 

\begin{IEEEkeywords}
\OurKeywords
\end{IEEEkeywords}

\section{Introduction}
\label{sec:intro}
Backscatter communication (BC) via load modulation allows simple passive tags to communicate with essentially zero transmit power and no transmit amplifier. This is achieved by modulating the termination load of the tag antenna in order to affect the reflection of an incident field (possibly an ambient field). This technique found widespread use in radio-frequency identification (RFID) and smart cards \cite{Finkenzeller2015} and is a promising approach to ultra-low-energy communication in the Internet of Things (IoT) \cite{HuynhCST2018}. The high data rate requirements of many IoT applications have recently prompted interest in backscatter modulation beyond binary \cite[Tab.~III]{HuynhCST2018}, e.g. 16-QAM \cite{KimionisNAT2021} or QPSK \cite{WangMercierISSCC2020}, together with error-correcting codes \cite{HuynhCST2018}. %An overview of modulation in ambient backscatter communication is given in .

From the perspective of communication theory, it is natural to ask for the channel capacity of a BC link, i.e. the maximum achievable information rate.
The existing research literature contains only a few related investigations. For example \cite{ZhaoACCESS2018} addresses the calculation of the channel capacity of binary load modulation in ambient backscatter communication (ABC) for various cases of the ambient signal modulation.
The work in \cite{KimSPAWC2017} concerns the maximization of ABC network capacity in terms of redundancy and reflection coefficient (for BPSK, QPSK, and 16-QAM alphabets) in a WiFi setting with OFDM. The focus of \cite{FuschiniAPL2008} is on the effect of the propagation environment on the Euclidean symbol distances and the resulting bit error rate with PSK and ASK for RFID load modulation.
%and diversity to combat fading in \cite{LiTVT2019}.
The literature lacks a complete description of the channel capacity and the capacity-achieving transmit scheme of BC load modulation, which would provide a crucial guideline for the design of practical systems with near-optimal rates \cite{Tse2005}.

This paper describes for the first time the channel capacity of load modulation in the general case of a  freely adaptable passive load. In this case, the load impedance can take on any complex value with non-negative real part for the duration of every symbol period. %which is the criterion for passive loads .
This is a generalization of specific modulation schemes such as QPSK, where the load takes values from a finite alphabet. The results and insights promise useful implications for practical BC systems.

This paper contains the following specific contributions:
\begin{itemize}
\item We develop a signal and noise model that abstracts all major classes of load-modulated single-tag BC links.
%, namely ambient, bistatic, and monostatic links.
\item Based thereon, we study the physical constraints on the tag-side transmit signal, arising from the passive nature of the tag. We find that the transmit current phasor $i \in \bbC$ must lie in a certain disk in the right half-plane. %because it follows from the load impedance via a M\"obius transform.
\item We discover that this disk constraint allows to solve the channel capacity problem at hand with existing theory on peak-power-constrained quadrature AWGN channels.
\item The capacity result is stated and discussed. We identify special cases in which the result even applies to ABC.
\item The capacity-achieving distribution of the transmit current and of the load impedance are characterized in detail. For the low-SNR case we show that a purely reactive load with Cauchy-distributed reactance achieves capacity.
\item We construct a finite symbol alphabet that approximates the capacity-achieving distribution. It yields near-capacity data rates, even if several symbols are unrealizable due to implementation constraints on the load.
\end{itemize}

This paper does not address the tag power consumption or aspects of the energy harvesting circuit. Specific channel models and multi-user interference are also out of scope.

% We assume that (i) the reception is subject to additive white Gaussian noise (AWGN), (ii) the load impedance is piecewise constant over time and just changes at the switching instants, (iii) ideal channel codes with a very large block length are used, and (iv) the signal transients can be ignored (the investigation in \cite[Appendix~E]{Dumphart2020} shows that this is a sensible assumption).

%%%%% it does not require the transmitting tag to provide any physical transmit power. To the best of our knowledge, the state of the art in load modulation considers only schemes with one bit \cite{Finkenzeller2015} or two bit \cite{WangMercierISSCC2020} per symbol (i.e. per load switching period) and the employed error-correcting codes are rudimentary.
%For the sake of improving the data rate of future load modulation systems, this work investigates the maximum achievable information rate when an arbitrary passive load impedance can be realized at the tag. We consider \ToDo{a magneto-inductive RFID load modulation system where the receiving reader observes the voltage change over the reader coil} due to the presence of the loaded tag. The observation is assumed to be subject to complex-valued additive white Gaussian noise (AWGN).

%%%%%We find that the received signal point (without noise) lies within a disc on the complex plane, with real-valued center $\DiskRadius$ and radius $\DiskRadius$. Any signal point within this disk can be realized with a passive tag load. Hence, the channel capacity of this system (which, by assumption, is subject to AWGN) equals the channel capacity of the complex-valued AWGN channel subject to a peak-power constraint, which has been investigated and analyzed in \cite{ShamaiTIT1995}. Hence, we shall make use of the knowledge contained in \cite{ShamaiTIT1995}, as it is the key to maximizing load modulation data rates.

%\subsubsection*{Related Work}

%... A printed millimetre-wave modulator and antenna array for backscatter communications at gigabit data rates \cite{KimionisNAT2021}.

%... passive long-range communication from ambient LoRa \cite{PengACM2018}.

%A circuit implementation for a backscatter IoT tag using 4-PSK modulation, which is capable of communicating with commodity WiFi transceivers, was presented in \cite{WangMercierISSCC2020}.

%... efficiency of RFID load modulation under practical conditions in \cite{FuschiniAPL2008}.

%... channel modeling and code design for near-field passive RFID in \cite{BarberoTC2014}.

%Netscatter \cite{Hessar2019}.

%Optimum modulation and coding \cite{KimSPAWC2017}.

\subsubsection*{Paper Structure}
\Cref{sec:model} describes the employed system model and \Cref{sec:constraint} the special transmit-side constraints. \Cref{sec:capacity} states the channel capacity, the associated distributions, and a familiar upper bound. \Cref{sec:practical} addresses practical modulation aspects and \Cref{sec:summary} concludes the paper.

\subsubsection*{Notation}
For a random variable $x$, the probability density function (PDF) is denoted as $f_x(x)$. For simplicity, we do not use distinct random variable notation.
% $\bbZ$, $\bbR$, $\bbR_+$, $\bbC$

\section{System Model}
\label{sec:model}
Before studying the information theory of load-modulated BC, we first have to establish an adequate system model. Our approach is based on the circuit models in \Cref{fig:SystemModelCircuit}, which are inspired by \cite{Finkenzeller2015}. They describe the different classes of tag-to-receiver BC links as listed in \cite[Fig.~2]{HuynhCST2018}.
%We develop a system model based on circuit descriptions of the communication systems of interest. We aim for a system model that allows to study the achievable information rate of all cases of interest with the same notation.
In each case, the left-hand circuit is a tag that modulates information via an adaptive passive load. We employ a symbol time index $n \in \bbZ$ and denote the load impedance $Z\L[n] \in \bbC$. It must fulfill $\Re(Z\L[n]) \geq 0$ at all times because the load is passive \cite[Sec.~4.1]{Pozar2004}.
The tag current phasor $i\Tx[n] \in \bbC$ depends on $Z\L[n]$. The tag antenna impedance is $R\Tx + jX\Tx$, however its reactance $X\Tx$ is canceled by the serial $-X\Tx$ element (resonance). The right-hand circuit is an information receiver that measures a voltage phasor $v[n] \in \bbC$. The tag and receiver circuits are coupled via the mutual impedance $Z\RxTx \in \bbC$, which encapsulates all aspects of the propagation channel.

\begin{figure}[!ht]
\centering
\subfloat[ambient or bistatic backscatter]{\centering
\resizebox{.9\columnwidth}{!}{\begin{tikzpicture}

\node (0,0) {\includegraphics[width=80mm]{SystemAmbientBS_noTermBox.pdf}};

%\put (-100, 47) {$Z\Term[n]$}

\put (-60, 37) {$v\Ind\Tx\! = v\Amb\Tx$}
\put (3, 38) {$v\Amb\Rx \! + v\N[n]$}
%\put (23, 38) {$v\N[n]$}
%\put (53, 38) {$v\Amb\Rx\! - \! Z\RxTx \iTx[n]$}
\put (62, 38) {$-Z\RxTx\,\iTx[n]$}

\put (-94, 35) {$-X\Tx$}

\put (115, -5) {$v[n]$}
\put (-131, -3) {$Z\L[n]$}

\put (-20, -15) {$X\Tx$}
\put (-3, -15) {$X\Rx$}

\put (-21, 10) {$R\Tx$}
\put (-4, 10) {$R\Rx$}

\put (-52, -25) {$\iTx[n]$}

\put(32,-6){$\begin{array}{l}
\mathrm{\scriptstyle not\hphantom{.}shown}\!:\\[-0.7mm]
\mathrm{\scriptstyle source\hphantom{.}of}\\[.3mm]
v\Tx\Amb, v\Rx\Amb\end{array}$}

%$\substack{\mathrm{not\hphantom{-}shown:}\\\mathrm{source\hphantom{-}of}\\v\Tx\Amb, v\Rx\Amb}$

\end{tikzpicture}}
\label{fig:SystemModel_AmbientBackscatter}%
}\\[5mm]
\subfloat[monostatic backscatter (e.g., RFID)]{\centering
\resizebox{.9\columnwidth}{!}{\begin{tikzpicture}

\node (0,0) {\includegraphics[width=80mm]{SystemPassiveRFID_noTermBox.pdf}};

%\put (-110, 47) {$Z\Term[n]$}

\put (-60, 37) {$v\Ind\Tx \! = Z\RxTx\,i\Rx$}
\put (23, 38) {$v\N[n]$}
\put (62, 38) {$-Z\RxTx\,\iTx[n]$}

\put (-94, 35) {$-X\Tx$}

%\put (124, -8) {$\begin{array}{c} + \\[2mm] v[n] \\[2mm] - \end{array}$}
\put (115, -5) {$v[n]$}
\put (66, -5) {$i\Rx$}
\put (-131, -3) {$Z\L[n]$}

\put (-20, -15) {$X\Tx$}
\put (-3, -15) {$X\Rx$}

\put (-20, 10) {$R\Tx$}
\put (-3, 10) {$R\Rx$}

\put (-52, -25) {$\iTx[n]$}

\end{tikzpicture}}
\label{fig:SystemModel_RfidBackscatter}%
}%\\
%\subfloat[inductive RFID link]{\centering
%\resizebox{.9\columnwidth}{!}{\begin{tikzpicture}

\node (0,0) {\includegraphics[width=80mm]{figures/SystemInductiveRFID_noTermBox.pdf}};

%\put (-103, 47) {$Z\Term[n]$}

.\put (-104, 33) {$C\Res$}

\put (42, 38) {$v\N[n]$}

%\put (116, -8) {$\begin{array}{c} + \\[2mm] v[n] \\[2mm] - \end{array}$}
\put (115, -5) {$v[n]$}
\put (65, -5) {$i\Rx$}
\put (-131, -3) {$Z\L[n]$}

\put (-4, -6) {$M$}

\put (-18, -25) {$L\Tx$}
\put (7, -25) {$L\Rx$}

\put (-18, 10) {$R\Tx$}
\put (7, 10) {$R\Rx$}

\put (-52, -25) {$\iTx[n]$}

\end{tikzpicture}
}
%\label{fig:SystemModel_RfidInductive}%
%}
\caption{Circuit descriptions of different classes of BC links. In each case, a load-modulating passive tag (left) transmits to an information receiver (right). %The voltmeter is the information sink.
In (b) the information receiver is also the power source (cf. current $i\Rx$).}
\label{fig:SystemModelCircuit}
\end{figure}

The circuit \Cref{fig:SystemModel_AmbientBackscatter} describes both ambient and bistatic backscatter links. These paradigms differ only in the assumptions regarding the voltages
$v\Amb\Tx , v\Amb\Rx \in \bbC$
that are induced by an extrinsic electromagnetic field, generated by some source. In ambient backscatter they are random modulated signals from an ambient source, but in the bistatic case they are unmodulated signals from a dedicated source \cite{HuynhCST2018}. % We consider them constant for the duration of a code word.
In either case, $v\Amb\Tx$ is the crucial cause for any electrical activity at the tag while $v\Amb\Rx$ is receive-side interference.

The monostatic case in \Cref{fig:SystemModel_RfidBackscatter} does not assume any extrinsic source. Instead, the information receiver is the system's power source (e.g., an RFID reader) and the crucial tag-side induced voltage $v\Ind\Tx = Z\RxTx\, i\Rx$ is due to the source current $i\Rx$. A prominent example of monostatic BC is inductive RFID, where $X\Tx, X\Rx, Z\RxTx$ are determined by inductances and where $-X\Tx$ is realized by a resonance capacitor.

%Here the induced tag voltage $v\Ind\Tx$ (whose value is left unspecified) is a consequence of the ambient electromagnetic field. The very similar circuit in \Cref{fig:SystemModel_RfidBackscatter} describes a passive RFID system; here $v\Ind\Tx$ is modeled as dependent source. Specifically, $v\Ind\Tx = Z\RxTx i\Rx$ is induced by the electromagnetic field that was generated by the current $i\Rx$ through the driven RFID reader antenna. A special case thereof is the inductive RFID system in \ToDo{fig:SystemModel\_RfidInductive} where the antenna reactances $X\Tx = \omega L\Tx$, $X\Rx = \omega L\Rx$ are determined by inductances $L\Tx, L\Rx$ and the mutual impedance $Z\RxTx = j\omega M$ is determined by the mutual inductance $M$ with $M^2 \leq L\Tx L\Rx$. In particular, the induced tag voltage is $v\Ind\Tx = j\omega M \cdot i\Rx$ and the resonance capacitance $C\Res$ is chosen such that $\f{1}{j\omega C\Res} = -jX\Tx$ at the resonance frequency $\omega = 2\pi f$.

We assume that $Z\L[n]$ is piecewise constant over time and that it changes instantaneously at the symbol switching instants. We neglect any signal transients which result for $i\Tx$ and $v$. This is a meaningful assumption if the symbol duration is significantly larger than the time constants of the circuits. Our previous work \cite[Appendix~E]{Dumphart2020} showed that transients do not deteriorate the receive processing of load-modulated signals and, when anticipated, can even improve the SNR.
%We consider the steady state (assume that transients have decayed to an extent where they have no appreciable impact).

The noise voltage sequence $v\N[n]$ is white Gaussian noise $v\N[n] \iid \calCN(0,\sigma^2)$ with variance $\sigma^2$, a well-established model for thermal noise \cite{Tse2005}. The samples are statistically independent and identically distributed (iid) for different $n$.
% Could also model sky noise, interference, ...)}.
%Recall that this is a shorthand notation for $\Re(w[n]),\Im(w[n]) \sim \calN(0,\f{\sigma^2}{2})$.

%We note that inductive RFID (\Cref{fig:SystemModel_RfidInductive}) is just a special case of monostatic backscatter (\Cref{fig:SystemModel_RfidBackscatter}) with ...

A basic circuit analysis yields the tag current expression
\begin{align}
\iTx[n] &= \f{v\Ind\Tx}{R\Tx + Z\L[n]} %\, , &
%\Re(Z\L[n]) &\geq 0\ \forall n 
\label{eq:Current} \, .
\end{align}
The receive voltage in the ambient backscatter case is given by
$v[n] = -Z\RxTx\,\iTx[n] + v\Amb\Rx + v\N[n]$.
In the monostatic backscatter case,
$v[n] = -Z\RxTx\,\iTx[n] + (R\Rx + jX\Rx) i\Rx + v\N[n]$.
%
To unify these different cases within the same system model, we consider a phase rotation
$e^{j\alpha} = \f{(v\Ind\Tx)^*}{|v\Ind\Tx|}$, a specific receive signal compensation,
and other transformations:
\begin{align}
i[n] &:= e^{j\alpha} \, \iTx[n]
= \f{|v\Ind\Tx|}{R\Tx} \f{1}{1 + z[n]}
\label{eq:TxSignalDef} \, , \\
z[n] &:= Z\L[n] \, / \, R\Tx
\label{eq:NormalizedZ} \, , \\
w[n] &:= -e^{j\alpha} \, v\N[n]
\label{eq:NoiseDef} \, , \\
y[n] &:= -e^{j\alpha} \!\left( v[n] - v\big|_{\iTx = 0, v\N = 0}  \right)
\label{eq:ObservationDef} \, .
\end{align}
The noise $w[n]$ maintains the statistics of $v\N[n]$. The unitless $z[n]$ is the normalized load impedance. The transformation from $v$ to $y$ in \Cref{eq:ObservationDef} could be practically realized via interference cancellation, calibration, and channel estimation. For the ambient backscatter case, where $v\Amb\Tx$ and $v\Amb\Rx$ are unknown modulated signals, this delicate aspect is discussed in \Cref{apdx:amb}. Monostatic backscatter systems face the challenge of canceling the strong self-interference $(R\Rx + jX\Rx) i\Rx$, cf. \cite{Finkenzeller2015}.

For either case, the definitions \Cref{eq:TxSignalDef,eq:NormalizedZ,eq:NoiseDef,eq:ObservationDef} yield a complex-valued, discrete-time signal and noise model:
\begin{align}
y[n] &= Z\RxTx \cdot i[n] + w[n]
\label{eq:SignalModel}
\, , \\
w[n] &\iid \calCN(0,\sigma^2)
\, .
\label{eq:NoiseModel}
\end{align}
The observation $y[n] \in \bbC$ is considered without quantization.

\section{Constraint on the Transmit Current}
\label{sec:constraint}
%The transmit current $i$ takes on a large value $i = \f{|v\Ind\Tx|}{R\Tx}$ when the tag is resonant ($Z\L = 0$) and vanishes to $i = 0$ when the tag is open-circuited ($Z\L = \infty$).
%In contrary to an active wireless node with a transmit amplifier, ...
Backscatter tags are passive and thus limited in their capability to establish a desired transmit current $i[n]$.
Formally, this is due to $\Re(z[n]) \geq 0$ in \Cref{eq:TxSignalDef}.
In the following we determine the set of realizable transmit currents, denoted $i[n] \in \Disk$, as prerequisite for the preceding channel capacity analysis.

In accordance with typical conventions in communication theory, we henceforth discard time indexation $[n]$ for brevity.
From \Cref{eq:TxSignalDef} we observe that the transmit current $i$ is a non-linear map of the normalized load impedance $z$:
\begin{align}
i &= \ChannelFunc(z) = \f{2\cdot\DiskRadius}{1 + z}
\, , \label{eq:ChannelMap} \\[1mm]
%\Re(z) &\geq 0
%\, , \\
z &= \ChannelFunc^{-1}(i) = \f{2\cdot\DiskRadius}{i} - 1
\, . \label{eq:InverseChannelMap} 
\end{align}
The map $g$ is illustrated in \Cref{fig:LoadTransformation}.
The current quantity $\DiskRadius \in \bbR$ will have the meaning of a radius. It is defined as
\begin{align}
\DiskRadius := \f{|v\Ind\Tx|}{2 R\Tx}
\, . \label{eq:DiskRadius}
\end{align}

The impedance $z$ of any passive load must lie in the right half-plane
%$z \in \HalfPlane$,
$\HalfPlane := \{ z \in \bbC \, | \, \Re(z) \geq 0 \}$. To characterize the set $\Disk = \ChannelFunc(\HalfPlane)$, we rewrite \Cref{eq:ChannelMap} as
$i = \ChannelFunc(z) = \DiskRadius (1 - \f{z-1}{z+1})$ or rather $i = \DiskRadius (1 - \Gamma)$. The reflection coefficient $\Gamma = \f{z-1}{z+1}$ is a bijective map from $z \in \HalfPlane$ to the unit disk $|\Gamma| \leq 1$; it is the M\"obius transformation that also underlies the well-known Smith chart \cite[Eq.~(2.53)]{Pozar2004}. This yields a constraint on the transmit current
\begin{align}
|i - \DiskRadius| \leq \DiskRadius
\label{eq:Constraint}
\end{align}
because $|i- \DiskRadius| = |-\DiskRadius\Gamma| = \DiskRadius |\Gamma| \leq \DiskRadius$.
%
The set of realizable transmit currents $i \in \Disk$ is thus given by a disk $\Disk \subset \bbC$ with radius $\DiskRadius$ and center $\DiskRadius$:
\begin{align}
\Disk
= \left\{ \ChannelFunc(z) \, \big| \, \Re(z) \geq 0 \right\}
= \left\{ i \in \bbC \ \big| \ |i - \DiskRadius| \leq \DiskRadius \right\} .
\label{eq:Disk}
\end{align}
%Its center $\DiskRadius \in \bbR$ lies on the real axis of the complex plane.
%Its boundary circle is
%$\partial\Disk = \left\{ s \in \bbC \ \big| \ |s - \DiskRadius| = \DiskRadius \right\}$.
An analogous observation is found in the  literature, regarding transformed RFID transponder impedance. \cite[Sec.~4.1]{Finkenzeller2015}

\begin{figure}[!ht]
\centering
%\resizebox{.96\columnwidth}{!}{\input{LoadTransformationIllustration}}%
\ \ \ \ \includegraphics[width=.88\columnwidth]{LoadTransformationIllustration-crop}
\put(-230,56){\scriptsize{\rotatebox{90}{$\Im(i) \, / \, \DiskRadius$}}}
\put(-184,-8){\scriptsize{$\Re(i) \, / \, \DiskRadius$}}
\put(-106,62){\scriptsize{\rotatebox{90}{$\Im(z)$}}}
\put(-54,-8){\scriptsize{$\Re(z)$}}
\put(-175,125){\footnotesize{$i = \ChannelFunc(z) = \f{2}{1 + z} \DiskRadius$}}
\caption{Map between normalized load impedance $z$ and transmit current $i$.}
\label{fig:LoadTransformation}
\end{figure}


\section{Channel Capacity}
\label{sec:capacity}
We recall the signal model $y = Z\RxTx \cdot i + w$ from \Cref{eq:SignalModel} and the constraint $|i - \DiskRadius| \leq \DiskRadius$ from \Cref{eq:Constraint}. Thereby $Z\RxTx$ is non-random and $w \sim \calCN(0,\sigma^2)$ is additive white Gaussian noise (AWGN). In this simple abstraction, which is visualized in \Cref{fig:BlockDiagram}, the impedance $z$ and the map $\ChannelFunc$ do not occur anymore.

Henceforth, the transmit current $i$ is considered as random variable; the probability density function (PDF) is denoted $f_i$. Its support must lie in the disk, i.e. $\supp(f_{i}) \subseteq \Disk$.

We are interested in the achievable information rates for a given signal-to-noise ratio (SNR). We define the SNR as
\begin{align}
\SNR := \f{|Z\RxTx|^2 \, \DiskRadius^2}{\sigma^2}
%= \f{|Z\RxTx|^2 \, |v\Ind\Tx|^2}{4 R\Tx^2 \,\sigma^2}
\, . \label{eq:SNR}
\end{align}
From an engineering perspective, reliable communication is possible over the channel at any achievable information rate, stated in bit per channel use (bpcu). The largest achievable rate defines the channel capacity $\RateMax$. Suitable error-correcting codes with a very large block length allow for information rates arbitrarily close to $\RateMax$ and with an arbitrarily small block error rate.
Formally, the mutual information $\Rate$ specifies an achievable rate, and the channel capacity $\RateMax$ is the supremum of $\Rate$ over all eligible transmit PDFs $f_i$. \cite{Tse2005}
%More formal detail is given in \Cref{apdx:ar}.

\begin{figure}[t]
\centering
\begin{tikzpicture}[auto,>=latex']
    %\node (in) {$i = \ChannelFunc(z)$};
    \node (in) {$i = \DiskRadius (1 - \Gamma)\ $};
    \node [circle, draw, fill=white,right of=in,node distance=25mm] (mult) {\large$\times$};
    \node (constr) [above of=in,node distance=10mm] {$\begin{array}{ll}\text{The current $i$ is subject to}\\[.6mm]\left|i - \DiskRadius\right| \leq \DiskRadius\text{ at all times.}\ \ \ \ \ \ \ \ \ \ \\[.6mm]\text{All such $i$ can be realized.}\end{array}$};
    \node [circle, draw, fill=white] (add) [right of=mult,node distance=20mm] {\large$+$};
    \node (zrt) [above of=mult,node distance=13mm] {$Z\RxTx$};
    \node (w) [above of=add,node distance=13mm] {$w \sim \mathcal{CN}(0,\sigma^2)$};
    \node (out) [right of=add, node distance=14mm]{$\ y$};
    \path[-{Stealth[length=2.7mm,width=1.8mm]},thick] (in) edge node {} (mult);
    \path[-{Stealth[length=2.7mm,width=1.8mm]},thick] (mult) edge node {} (add);
    \path[-{Stealth[length=2.7mm,width=1.8mm]},thick] (zrt) edge node {} (mult);
    \path[-{Stealth[length=2.7mm,width=1.8mm]},thick] (w) edge node {} (add);
    \draw[-{Stealth[length=2.7mm,width=1.8mm]},thick] (add) edge node {} (out) ;
\end{tikzpicture}
\caption{Communication-theoretic description of load modulation in AWGN.}
\label{fig:BlockDiagram}
\end{figure}

%The transmit current $i \in \bbC$ is now considered as a random variable that can have any distribution that is restricted to the disk $\Disk$ as defined in \eqref{eq:Disk}. In formal terms, the probability density function (PDF) $f_{i}$ must fulfill $\int_\Disk f_{i}(i)\,di = 1$ and its support $\supp(f_{i}) \subseteq \Disk$.

%According to the random-coding argument from information theory \cite[Sec.~X.Y]{Cover2006}, an appropriate code book can be obtained by iid sampling from $f_{i}$.
%Low-complexity designs for the error-correcting encoder and decoder are out of scope.

%
%\footnotetext{For RFID, also the more specific properties
%$\DiskRadius = \f{|Z\RxTx|}{2 R\Tx} |i\Rx|$
%and
%$\SNR \propto |Z\RxTx|^4$ apply.
%For inductive RFID, these become
%$\DiskRadius = \f{\omega M}{2 R\Tx} |i\Rx|$
%and
%$\SNR \propto \omega^4 M^4$.%
%The expression hints to the severe path loss of RFID.
%}

%\begin{align}
%\RateMax = \sup_{\{f_s | \supp(f_s) \subseteq \Disk \}} \Rate
%\label{eq:ChannelCapacityDef}
%\end{align}
%

A crucial insight is that the disk constraint $|i - \DiskRadius| \leq \DiskRadius$ is equivalent to a peak-power-type constraint $|i - \DiskRadius|^2 \leq \DiskRadius^2$ on the signal $i - \DiskRadius$. The $\DiskRadius$-offset does not affect mutual information. Hence, the capacity problem at hand is equivalent to that of the complex-valued, peak-power constrained AWGN channel. The latter has been solved in \cite{ShamaiTIT1995}.

\subsection{Capacity-Achieving Transmit Current Statistics}
\label{sec:CurrentStats}

We translate the results in \cite{ShamaiTIT1995} to our formalism. This readily allows for a characterization of the capacity-achieving distribution on the transmit current $i$. It fulfills:
\begin{enumerate}
\item
The PDF support is given by a finite union 
$\supp(f_{i}) = \calC_1 \cup \ldots \cup \calC_K$
of concentric circles $\calC_k$, $k \in \{ 1 , \ldots , K \}$, with radii $\Radius_k$. All circles have their center at $\DiskRadius$.
More formally, the circles are given by
\begin{align}
\calC_k = \left\{\, \DiskRadius + \Radius_k\, e^{j\phi} \, \big| \, \phi \in (-\pi,\pi] \,\right\} .
\label{eq:CircleDefinition}
\end{align}
We assume that the indexing asserts
$0 \leq \Radius_K < \ldots < \DiskRadius$.
\item
The circles are chosen with non-uniform probabilities, denoted as $\CircProb_k$.
\item
The angle $\phi$ has uniform distribution
$\phi \sim \calU(-\pi,\pi)$
for any circle $k$ and for any $\SNR$.
%This means $i|k \sim \calU(\calC_k)$.
\item
The set of circles always contains the largest possible circle $\calC_1 = \partial\Disk$ (the disk boundary with radius $\DiskRadius$).
\item
The number of circles $K \in \bbN$ increases with $\SNR$.
\item
At low SNR, $K=1$ achieves capacity. This corresponds to a uniform-PSK modulation $i \sim \calU(\calC_1)$.
\end{enumerate}
%
\Cref{fig:CapAchievingDistr_i} shows a high-SNR example of the capacity-achieving distribution of $i$.

\begin{figure}[!ht]
\centering
\subfloat[transmit current]{%
%\ \ \,\includegraphics[height=42.6mm,trim=6.2mm 6.2mm 0 0,clip=true]{CapacityAchievingDistr_sCircles-crop}
%\put(-117,60){\rotatebox{90}{\scriptsize{$\Im(s)\, / \,\DiskRadius$}}}%
%\put(-71,-7){\scriptsize{$\Re(s)\, / \,\DiskRadius$}}%
\resizebox{!}{46mm}{% This file was created by matlab2tikz.
%
%The latest updates can be retrieved from
%  http://www.mathworks.com/matlabcentral/fileexchange/22022-matlab2tikz-matlab2tikz
%where you can also make suggestions and rate matlab2tikz.
%
\definecolor{mycolor1}{rgb}{0.33373,0.33373,0.33373}%
\definecolor{mycolor2}{rgb}{0.46085,0.46085,0.46085}%
\definecolor{mycolor3}{rgb}{0.47869,0.47869,0.47869}%
\definecolor{mycolor4}{rgb}{0.48299,0.48299,0.48299}%
\definecolor{mycolor5}{rgb}{0.48095,0.48095,0.48095}%
%
\begin{tikzpicture}

\begin{axis}[%
width=30mm,
height=38.2mm,
at={(0,0)},
scale only axis,
point meta min=0,
point meta max=1,
xmin=-0.1,
xmax=2.1,
xlabel style={},
xlabel={$\Re(i)\, / \,\DiskRadius$},
ymin=-1.4,
ymax=1.4,
ylabel style={at={(-0.1,0.5)}},
ylabel={$\Im(i)\, / \,\DiskRadius$},
xtick={0,1,2},
ytick={-1,0,1},
axis background/.style={fill=white},
xmajorgrids,
ymajorgrids,
legend style={},
colormap={mymap}{[1pt] rgb(0pt)=(1,1,1); rgb(255pt)=(0,0,0)},
colormap access=piecewise const,
colorbar,
colorbar sampled, % makes it properly printable
colorbar style={at={(1.05,1.0)}, width=3mm, ylabel={probability density}, ylabel style={at={(1.08,0.5)}}, ytick={0,.97}, yticklabels={0, high}, samples=80}
]

\newcommand\myLineWidth{1.5pt}

\addplot [color=black, line width=\myLineWidth, forget plot]
  table[row sep=crcr]{%
2	0\\
1.99860641032153	0.0527753471304624\\
1.99442952547052	0.105403599902848\\
1.98748098717419	0.157738073937438\\
1.97778016225504	0.209632903668547\\
1.96535408865204	0.260943448898017\\
1.95023740006101	0.311526697933392\\
1.93247222940436	0.361241666187153\\
1.91210809139899	0.409949789126054\\
1.88920174454998	0.457515308475323\\
1.86381703295442	0.50380565060133\\
1.83602470835643	0.548691796018074\\
1.80590223294943	0.592048638987638\\
1.77353356347505	0.633755336212316\\
1.73900891722066	0.673695643646557\\
1.7024245205676	0.711758240489971\\
1.66388234079101	0.747837039458364\\
1.62348980185873	0.78183148246803\\
1.58135948502146	0.81364682090916\\
1.53760881502858	0.843194379727219\\
1.49235973284434	0.87039180457622\\
1.44573835577654	0.895163291355062\\
1.39787462596492	0.917439797487156\\
1.34890194820917	0.937159234354465\\
1.29895681814577	0.954266640349623\\
1.24817844181022	0.968714334063789\\
1.19670834764483	0.980462047183287\\
1.14468999203354	0.98947703672462\\
1.0922683594633	0.995734176295034\\
1.03958955842627	0.999216026124288\\
0.986800414189241	0.999912881672411\\
0.934048059565305	0.997822800678005\\
0.881479524828297	0.992951608571673\\
0.829241327913062	0.985312882239497\\
0.777479066043685	0.974927912181824\\
0.726337009927917	0.961825643172819\\
0.675957701648834	0.946042593586195\\
0.626481557374488	0.927622753611948\\
0.578046475992862	0.906617462647813\\
0.530787454762934	0.883085266207151\\
0.484836213053104	0.857091752742105\\
0.440320825215692	0.828709370836819\\
0.397365363620744	0.79801722728024\\
0.356089552844084	0.765100866581308\\
0.316608435973452	0.73005203254107\\
0.279032053962775	0.692968412546247\\
0.243465138928304	0.653953365296968\\
0.210006822241412	0.613115632727537\\
0.178750358231686	0.570569036923155\\
0.149782864270386	0.526432162877356\\
0.123185077958701	0.480828027974359\\
0.099031132097581	0.433883739117558\\
0.0773883480663171	0.38573013845978\\
0.0583170481857884	0.336501438722739\\
0.041870387589332	0.286334849122113\\
0.0280942060698524	0.235370192940843\\
0.0170269003160982	0.18374951781657\\
0.00869931689420322	0.131616699829394\\
0.00313466627277403	0.0791170424934244\\
0.000348458131151141	0.0263968717698365\\
0.000348458131151141	-0.0263968717698358\\
0.00313466627277403	-0.0791170424934246\\
0.00869931689420322	-0.131616699829393\\
0.0170269003160982	-0.183749517816571\\
0.0280942060698524	-0.235370192940842\\
0.0418703875893318	-0.286334849122112\\
0.0583170481857885	-0.336501438722739\\
0.077388348066317	-0.385730138459779\\
0.099031132097581	-0.433883739117558\\
0.123185077958701	-0.480828027974359\\
0.149782864270386	-0.526432162877356\\
0.178750358231687	-0.570569036923155\\
0.210006822241412	-0.613115632727537\\
0.243465138928304	-0.653953365296968\\
0.279032053962775	-0.692968412546247\\
0.316608435973451	-0.73005203254107\\
0.356089552844084	-0.765100866581308\\
0.397365363620743	-0.798017227280239\\
0.440320825215692	-0.828709370836819\\
0.484836213053104	-0.857091752742105\\
0.530787454762934	-0.883085266207151\\
0.578046475992862	-0.906617462647812\\
0.626481557374488	-0.927622753611948\\
0.675957701648835	-0.946042593586195\\
0.726337009927917	-0.961825643172819\\
0.777479066043685	-0.974927912181824\\
0.829241327913062	-0.985312882239497\\
0.881479524828297	-0.992951608571673\\
0.934048059565304	-0.997822800678005\\
0.986800414189241	-0.999912881672411\\
1.03958955842627	-0.999216026124288\\
1.0922683594633	-0.995734176295035\\
1.14468999203354	-0.98947703672462\\
1.19670834764483	-0.980462047183287\\
1.24817844181022	-0.968714334063789\\
1.29895681814577	-0.954266640349623\\
1.34890194820917	-0.937159234354465\\
1.39787462596492	-0.917439797487157\\
1.44573835577654	-0.895163291355063\\
1.49235973284434	-0.870391804576219\\
1.53760881502858	-0.843194379727218\\
1.58135948502146	-0.81364682090916\\
1.62348980185873	-0.78183148246803\\
1.66388234079101	-0.747837039458364\\
1.7024245205676	-0.711758240489971\\
1.73900891722066	-0.673695643646557\\
1.77353356347505	-0.633755336212316\\
1.80590223294943	-0.592048638987639\\
1.83602470835643	-0.548691796018074\\
1.86381703295442	-0.50380565060133\\
1.88920174454998	-0.457515308475323\\
1.91210809139899	-0.409949789126054\\
1.93247222940436	-0.361241666187153\\
1.95023740006101	-0.311526697933391\\
1.96535408865204	-0.260943448898017\\
1.97778016225504	-0.209632903668546\\
1.98748098717419	-0.157738073937437\\
1.99442952547052	-0.105403599902848\\
1.99860641032153	-0.0527753471304627\\
2	-2.44929359829471e-16\\
};
\addplot [color=mycolor1, line width=\myLineWidth, forget plot]
  table[row sep=crcr]{%
1.82610804158895	0\\
1.82495678594889	0.0435981386621231\\
1.82150622778467	0.0870747614921664\\
1.81576598442079	0.130308691344469\\
1.80775205494503	0.173179427502227\\
1.79748677561622	0.215567481534606\\
1.78499875760898	0.257354710332425\\
1.77032280726931	0.298424645394197\\
1.75349982910305	0.338662817444726\\
1.73457671176766	0.377957075481512\\
1.71360619738515	0.416197899359709\\
1.6906467345403	0.453278705044412\\
1.66576231537402	0.489096141679479\\
1.63902229722569	0.523550379644901\\
1.61050120932193	0.556545388799862\\
1.58027854505015	0.587989206135964\\
1.54843854039635	0.617794192094625\\
1.5150699391642	0.645877274834247\\
1.48026574563024	0.672160181766338\\
1.44412296532421	0.696569657715259\\
1.4067423346573	0.71903766909353\\
1.36822804015163	0.739501593523646\\
1.32868742805382	0.757904394377874\\
1.28823070514164	0.774194779749564\\
1.24697063155807	0.78832734541289\\
1.20502220652844	0.800262701372576\\
1.16250234783707	0.809967581650874\\
1.11952956595635	0.817414937005809\\
1.07622363373685	0.822584010322273\\
1.0327052525789	0.825460394465824\\
0.989095716016089	0.826036072437955\\
0.945516571648503	0.824309439720904\\
0.902089282367713	0.820285308749739\\
0.858934887817931	0.813974895499231\\
0.816173667036806	0.805395788222926\\
0.773924803216177	0.794571898431526\\
0.732306051517137	0.781533394247219\\
0.691433410865286	0.766316616319712\\
0.651420800640909	0.748963976538324\\
0.612379743165242	0.729523839822442\\
0.574419052867755	0.708050389319817\\
0.537644533000818	0.684603475388412\\
0.502158680747066	0.65924844878272\\
0.468060401541364	0.63205597850949\\
0.435444733403621	0.603101854860533\\
0.404402582050783	0.572466778171581\\
0.375020467526296	0.540236133895979\\
0.347380283053225	0.506499754620113\\
0.321559066783155	0.471351669683879\\
0.297628787077045	0.434889843104046\\
0.275656141916498	0.397215900530973\\
0.255702373004523	0.358434845999695\\
0.237823095073923	0.318654769264842\\
0.222068140879064	0.277986544535105\\
0.208481422303047	0.236543521446935\\
0.197100807967416	0.194441209138772\\
0.187958017685516	0.15179695430636\\
0.181078534053686	0.108729614136461\\
0.17648153142669	0.0653592250305522\\
0.174179822475355	0.0218066680418542\\
0.174179822475355	-0.0218066680418536\\
0.17648153142669	-0.0653592250305524\\
0.181078534053686	-0.10872961413646\\
0.187958017685516	-0.15179695430636\\
0.197100807967416	-0.194441209138772\\
0.208481422303047	-0.236543521446935\\
0.222068140879064	-0.277986544535105\\
0.237823095073923	-0.318654769264841\\
0.255702373004523	-0.358434845999695\\
0.275656141916498	-0.397215900530973\\
0.297628787077045	-0.434889843104045\\
0.321559066783155	-0.471351669683879\\
0.347380283053225	-0.506499754620113\\
0.375020467526296	-0.540236133895979\\
0.404402582050783	-0.572466778171581\\
0.435444733403621	-0.603101854860533\\
0.468060401541364	-0.63205597850949\\
0.502158680747066	-0.65924844878272\\
0.537644533000818	-0.684603475388412\\
0.574419052867755	-0.708050389319817\\
0.612379743165243	-0.729523839822443\\
0.651420800640909	-0.748963976538324\\
0.691433410865285	-0.766316616319711\\
0.732306051517138	-0.781533394247219\\
0.773924803216176	-0.794571898431526\\
0.816173667036806	-0.805395788222926\\
0.858934887817931	-0.813974895499231\\
0.902089282367713	-0.820285308749739\\
0.945516571648502	-0.824309439720904\\
0.989095716016089	-0.826036072437955\\
1.0327052525789	-0.825460394465824\\
1.07622363373685	-0.822584010322273\\
1.11952956595635	-0.817414937005809\\
1.16250234783707	-0.809967581650874\\
1.20502220652844	-0.800262701372576\\
1.24697063155807	-0.78832734541289\\
1.28823070514164	-0.774194779749564\\
1.32868742805381	-0.757904394377874\\
1.36822804015163	-0.739501593523646\\
1.4067423346573	-0.719037669093529\\
1.44412296532421	-0.696569657715258\\
1.48026574563024	-0.672160181766338\\
1.5150699391642	-0.645877274834247\\
1.54843854039635	-0.617794192094625\\
1.58027854505015	-0.587989206135964\\
1.61050120932193	-0.556545388799862\\
1.63902229722569	-0.5235503796449\\
1.66576231537402	-0.489096141679479\\
1.6906467345403	-0.453278705044413\\
1.71360619738515	-0.416197899359709\\
1.73457671176766	-0.377957075481512\\
1.75349982910305	-0.338662817444726\\
1.77032280726931	-0.298424645394197\\
1.78499875760898	-0.257354710332425\\
1.79748677561622	-0.215567481534606\\
1.80775205494503	-0.173179427502227\\
1.81576598442079	-0.130308691344469\\
1.82150622778467	-0.0870747614921668\\
1.82495678594889	-0.0435981386621234\\
1.82610804158895	-2.02338113776358e-16\\
};
\addplot [color=white!42!black, line width=\myLineWidth, forget plot]
  table[row sep=crcr]{%
1.68695907143333	0\\
1.68600173236185	0.0362545034593141\\
1.68313238342311	0.0724079591149905\\
1.67835902200725	0.108359600801744\\
1.67169495232865	0.144009224846017\\
1.66315874834477	0.179257469351592\\
1.65277420198714	0.214006091139014\\
1.64057025684898	0.248158239566955\\
1.62658092751427	0.281618726472323\\
1.61084520475295	0.314294291476741\\
1.59340694684666	0.346093861919954\\
1.57431475734786	0.376928806695662\\
1.55362184961299	0.406713183282314\\
1.53138589848734	0.43536397728033\\
1.50766887955485	0.462801333788118\\
1.48253689640112	0.488948779972011\\
1.45605999637078	0.513733438209768\\
1.42831197533303	0.537086229213581\\
1.3993701719993	0.558942064566437\\
1.3693152523664	0.579240028135212\\
1.33823098488591	0.59792354585486\\
1.30620400698647	0.614940543410476\\
1.27332358359975	0.630243591377759\\
1.23968135836305	0.643790037417313\\
1.20537109819208	0.65554212515438\\
1.17048843193572	0.665467099412616\\
1.13513058384128	0.673537297508652\\
1.09939610257306	0.679730226352947\\
1.06338458653959	0.684028625142068\\
1.02719640629497	0.686420513467642\\
0.990932424788137	0.686899224707904\\
0.954693716239756	0.685463424608767\\
0.91858128443021	0.682117115002628\\
0.882695781183969	0.676869622654543\\
0.847137225834893	0.66973557326686\\
0.812004726454414	0.660734850714765\\
0.777396203619561	0.649892541626351\\
0.743408117490755	0.637238865461692\\
0.710135198960036	0.622809090285782\\
0.677670185619076	0.606643434470119\\
0.646103563282883	0.588786954596881\\
0.6155233137896	0.56928941987816\\
0.586014669779344	0.548205173440234\\
0.557659877135552	0.525592980859531\\
0.530537965750952	0.501515866372428\\
0.504724529257073	0.476040937215398\\
0.480291514331245	0.449239196585106\\
0.457307020168295	0.421185345739767\\
0.435835108675885	0.39195757579334\\
0.415935625922479	0.361637349782868\\
0.397664035335622	0.330309175616385\\
0.381071263115416	0.298060370534219\\
0.366203556294067	0.26498081774018\\
0.353102353837113	0.231162715880952\\
0.341804171145592	0.196700322071929\\
0.332340498281073	0.161689689185725\\
0.324737712197205	0.126228398135593\\
0.319017003222417	0.0904152858999197\\
0.315194315998669	0.0543501700458341\\
0.31328030504088	0.0181335705197516\\
0.31328030504088	-0.0181335705197511\\
0.315194315998669	-0.0543501700458342\\
0.319017003222417	-0.0904152858999192\\
0.324737712197205	-0.126228398135593\\
0.332340498281073	-0.161689689185725\\
0.341804171145592	-0.196700322071929\\
0.353102353837113	-0.231162715880952\\
0.366203556294067	-0.26498081774018\\
0.381071263115416	-0.298060370534219\\
0.397664035335622	-0.330309175616385\\
0.415935625922479	-0.361637349782867\\
0.435835108675885	-0.39195757579334\\
0.457307020168295	-0.421185345739767\\
0.480291514331245	-0.449239196585106\\
0.504724529257073	-0.476040937215398\\
0.530537965750951	-0.501515866372428\\
0.557659877135552	-0.525592980859531\\
0.586014669779344	-0.548205173440233\\
0.6155233137896	-0.56928941987816\\
0.646103563282883	-0.588786954596881\\
0.677670185619076	-0.606643434470119\\
0.710135198960035	-0.622809090285782\\
0.743408117490755	-0.637238865461692\\
0.777396203619562	-0.649892541626351\\
0.812004726454413	-0.660734850714765\\
0.847137225834893	-0.66973557326686\\
0.882695781183969	-0.676869622654543\\
0.91858128443021	-0.682117115002627\\
0.954693716239755	-0.685463424608767\\
0.990932424788137	-0.686899224707904\\
1.02719640629497	-0.686420513467642\\
1.06338458653959	-0.684028625142068\\
1.09939610257306	-0.679730226352947\\
1.13513058384128	-0.673537297508652\\
1.17048843193572	-0.665467099412616\\
1.20537109819208	-0.65554212515438\\
1.23968135836305	-0.643790037417313\\
1.27332358359974	-0.630243591377759\\
1.30620400698647	-0.614940543410477\\
1.33823098488591	-0.59792354585486\\
1.3693152523664	-0.579240028135212\\
1.3993701719993	-0.558942064566437\\
1.42831197533303	-0.537086229213581\\
1.45605999637078	-0.513733438209768\\
1.48253689640112	-0.488948779972011\\
1.50766887955485	-0.462801333788118\\
1.53138589848734	-0.43536397728033\\
1.55362184961299	-0.406713183282315\\
1.57431475734786	-0.376928806695662\\
1.59340694684666	-0.346093861919954\\
1.61084520475295	-0.314294291476741\\
1.62658092751427	-0.281618726472323\\
1.64057025684898	-0.248158239566955\\
1.65277420198714	-0.214006091139014\\
1.66315874834477	-0.179257469351593\\
1.67169495232865	-0.144009224846017\\
1.67835902200725	-0.108359600801744\\
1.68313238342311	-0.0724079591149908\\
1.68600173236185	-0.0362545034593143\\
1.68695907143333	-1.68256445595213e-16\\
};
\addplot [color=mycolor2, line width=\myLineWidth, forget plot]
  table[row sep=crcr]{%
1.5613112685487	0\\
1.56052903095845	0.029623397045898\\
1.55818449842414	0.0591642283710675\\
1.55428420557847	0.0885401583802523\\
1.54883902323713	0.11766931108774\\
1.54186412809995	0.14647049832042\\
1.53337896045067	0.174863446003781\\
1.5234071699734	0.202769017900158\\
1.5119765498367	0.230109436175619\\
1.49911895922907	0.256808498180735\\
1.48487023456162	0.282791788841037\\
1.46927008958561	0.307986888065172\\
1.45236200470308	0.332323572592685\\
1.43419310577918	0.355734011718846\\
1.41481403279393	0.378152956350985\\
1.39427879869951	0.399517920869419\\
1.37264463887648	0.419769357286082\\
1.34997185160851	0.438850821215444\\
1.32632363002022	0.456709129195141\\
1.30176588594666	0.473294506917823\\
1.27636706622516	0.488560727961073\\
1.25019796192174	0.502465242628743\\
1.22333151102371	0.514969296544582\\
1.1958425951484	0.526038038667637\\
1.16780783083469	0.535640618428357\\
1.13930535599894	0.543750271714658\\
1.11041461215064	0.55034439546831\\
1.08121612297465	0.555404610683709\\
1.05179126989725	0.558916813633465\\
1.02222206526153	0.560871215178019\\
0.992590923744246	0.561262368049731\\
0.962980432651353	0.560089182035392\\
0.933473121732376	0.557354927014842\\
0.904151233155189	0.553067223847232\\
0.875096492282339	0.547238023130319\\
0.846389879887808	0.539883571892008\\
0.81811140644907	0.531024368306973\\
0.790339889143538	0.520685104564564\\
0.763152732170958	0.508894598047251\\
0.736625711014016	0.495685711011406\\
0.710832761238483	0.481095258994303\\
0.685845772421528	0.465163908202613\\
0.661734387782573	0.447936062168391\\
0.63856581007515	0.429459737988467\\
0.616404614280775	0.409786432492188\\
0.595312567626892	0.388970978710515\\
0.575348457430529	0.367071393046535\\
0.556567927247505	0.344148713573335\\
0.539023321783859	0.320266829909948\\
0.522763541001765	0.295492305149527\\
0.507833903826566	0.269894190336059\\
0.494276021834803	0.243543832006732\\
0.482127683275289	0.216514673336326\\
0.471422747746477	0.188882049437925\\
0.46219105182369	0.160722977390435\\
0.454458325899234	0.132115941578178\\
0.448246122467177	0.103140674940832\\
0.443571756052688	0.0738779367434311\\
0.440448254953344	0.0444092874858057\\
0.438884324926929	0.0148168615788444\\
0.438884324926929	-0.014816861578844\\
0.440448254953344	-0.0444092874858059\\
0.443571756052688	-0.0738779367434307\\
0.448246122467177	-0.103140674940832\\
0.454458325899234	-0.132115941578177\\
0.46219105182369	-0.160722977390435\\
0.471422747746477	-0.188882049437925\\
0.482127683275289	-0.216514673336326\\
0.494276021834803	-0.243543832006732\\
0.507833903826566	-0.269894190336059\\
0.522763541001765	-0.295492305149526\\
0.539023321783859	-0.320266829909949\\
0.556567927247505	-0.344148713573335\\
0.575348457430529	-0.367071393046535\\
0.595312567626892	-0.388970978710515\\
0.616404614280775	-0.409786432492188\\
0.63856581007515	-0.429459737988467\\
0.661734387782573	-0.447936062168391\\
0.685845772421528	-0.465163908202613\\
0.710832761238484	-0.481095258994303\\
0.736625711014016	-0.495685711011406\\
0.763152732170957	-0.508894598047251\\
0.790339889143538	-0.520685104564564\\
0.81811140644907	-0.531024368306973\\
0.846389879887808	-0.539883571892008\\
0.875096492282339	-0.547238023130319\\
0.904151233155189	-0.553067223847232\\
0.933473121732376	-0.557354927014842\\
0.962980432651352	-0.560089182035392\\
0.992590923744246	-0.561262368049731\\
1.02222206526153	-0.560871215178019\\
1.05179126989725	-0.558916813633465\\
1.08121612297465	-0.555404610683709\\
1.11041461215064	-0.55034439546831\\
1.13930535599894	-0.543750271714658\\
1.16780783083469	-0.535640618428357\\
1.1958425951484	-0.526038038667637\\
1.22333151102371	-0.514969296544582\\
1.25019796192174	-0.502465242628743\\
1.27636706622516	-0.488560727961073\\
1.30176588594666	-0.473294506917822\\
1.32632363002022	-0.456709129195141\\
1.34997185160851	-0.438850821215444\\
1.37264463887648	-0.419769357286082\\
1.39427879869951	-0.399517920869419\\
1.41481403279393	-0.378152956350985\\
1.43419310577918	-0.355734011718845\\
1.45236200470308	-0.332323572592685\\
1.46927008958561	-0.307986888065172\\
1.48487023456162	-0.282791788841038\\
1.49911895922907	-0.256808498180735\\
1.5119765498367	-0.230109436175619\\
1.5234071699734	-0.202769017900158\\
1.53337896045067	-0.174863446003781\\
1.54186412809995	-0.14647049832042\\
1.54883902323713	-0.11766931108774\\
1.55428420557847	-0.0885401583802521\\
1.55818449842414	-0.0591642283710678\\
1.56052903095845	-0.0296233970458982\\
1.5613112685487	-1.37481609670702e-16\\
};
\addplot [color=mycolor3, line width=\myLineWidth, forget plot]
  table[row sep=crcr]{%
1.44205256165551	0\\
1.44143652176824	0.0233294773912794\\
1.43959011912011	0.0465939313447658\\
1.43651849996646	0.0697285196546497\\
1.43223022546078	0.0926687620739627\\
1.42673724779325	0.115350720032591\\
1.42005487687784	0.137711174845536\\
1.41220173768082	0.159687803914734\\
1.40319971830964	0.181219354433306\\
1.39307390900686	0.202245814108125\\
1.38185253221916	0.222708578424836\\
1.36956686393626	0.242550613989149\\
1.35625114651919	0.261716617489141\\
1.34194249326066	0.2801531698355\\
1.32668078494365	0.297808885050115\\
1.31050855868655	0.314634553488007\\
1.29347088938452	0.330583278993439\\
1.27561526407774	0.345610609607914\\
1.25699144959646	0.359674661465752\\
1.23765135385196	0.372736235531942\\
1.21764888115986	0.384758926856876\\
1.19703978199913	0.395709226043478\\
1.17588149762552	0.405556612643905\\
1.15423299997246	0.414273640225503\\
1.13215462728572	0.421836012868943\\
1.10970791594988	0.428222652885304\\
1.08695542897542	0.433415759563372\\
1.06396058162434	0.437400858783416\\
1.0407874646605	0.440166843359154\\
1.01750066571714	0.441706003995475\\
0.994165089279562	0.442014050775627\\
0.970845775784692	0.441090125117983\\
0.947607720341721	0.438936802169063\\
0.924515691579077	0.435560083626139\\
0.901634051122636	0.430969381009428\\
0.879026574208331	0.425177489430498\\
0.85675627192913	0.418200551929995\\
0.834885215611818	0.410058014484095\\
0.813474363813077	0.40077257180508\\
0.792583392417055	0.390370104087104\\
0.772270528307974	0.378879604873454\\
0.752592387081358	0.366333100246338\\
0.733603815246216	0.352765559564453\\
0.715357737357986	0.338214797997114\\
0.697905008508302	0.322721371126588\\
0.681294272582737	0.306328461912417\\
0.665571826681565	0.289081760332763\\
0.650781492081443	0.271029336038244\\
0.636964492097651	0.252221504373195\\
0.624159337187318	0.232710686137783\\
0.612401717613872	0.21255126148183\\
0.601724403971875	0.191799418337585\\
0.592157155849498	0.170512995813878\\
0.58372663888321	0.14875132298815\\
0.576456350435868	0.126575053545673\\
0.570366554105351	0.10404599672685\\
0.565474223246279	0.0812269450537788\\
0.561792993662231	0.0581814993162272\\
0.55933312560031	0.0349738913048257\\
0.558101475154	0.0116688047855481\\
0.558101475154	-0.0116688047855478\\
0.55933312560031	-0.0349738913048258\\
0.561792993662231	-0.0581814993162269\\
0.565474223246279	-0.0812269450537789\\
0.570366554105351	-0.10404599672685\\
0.576456350435868	-0.126575053545672\\
0.58372663888321	-0.14875132298815\\
0.592157155849498	-0.170512995813878\\
0.601724403971875	-0.191799418337586\\
0.612401717613872	-0.21255126148183\\
0.624159337187318	-0.232710686137783\\
0.636964492097651	-0.252221504373195\\
0.650781492081443	-0.271029336038243\\
0.665571826681565	-0.289081760332763\\
0.681294272582737	-0.306328461912417\\
0.697905008508302	-0.322721371126588\\
0.715357737357986	-0.338214797997114\\
0.733603815246216	-0.352765559564453\\
0.752592387081357	-0.366333100246338\\
0.772270528307974	-0.378879604873454\\
0.792583392417055	-0.390370104087105\\
0.813474363813077	-0.40077257180508\\
0.834885215611818	-0.410058014484095\\
0.85675627192913	-0.418200551929995\\
0.879026574208331	-0.425177489430498\\
0.901634051122636	-0.430969381009428\\
0.924515691579077	-0.435560083626139\\
0.947607720341721	-0.438936802169063\\
0.970845775784691	-0.441090125117983\\
0.994165089279562	-0.442014050775627\\
1.01750066571714	-0.441706003995475\\
1.0407874646605	-0.440166843359154\\
1.06396058162434	-0.437400858783416\\
1.08695542897542	-0.433415759563372\\
1.10970791594988	-0.428222652885304\\
1.13215462728572	-0.421836012868943\\
1.15423299997246	-0.414273640225503\\
1.17588149762552	-0.405556612643905\\
1.19703978199913	-0.395709226043479\\
1.21764888115986	-0.384758926856875\\
1.23765135385196	-0.372736235531941\\
1.25699144959646	-0.359674661465752\\
1.27561526407774	-0.345610609607914\\
1.29347088938452	-0.330583278993439\\
1.31050855868655	-0.314634553488007\\
1.32668078494365	-0.297808885050115\\
1.34194249326066	-0.2801531698355\\
1.35625114651919	-0.261716617489141\\
1.36956686393626	-0.24255061398915\\
1.38185253221916	-0.222708578424836\\
1.39307390900686	-0.202245814108125\\
1.40319971830964	-0.181219354433306\\
1.41220173768082	-0.159687803914734\\
1.42005487687784	-0.137711174845536\\
1.42673724779325	-0.115350720032591\\
1.43223022546078	-0.0926687620739626\\
1.43651849996646	-0.0697285196546496\\
1.43959011912011	-0.046593931344766\\
1.44143652176824	-0.0233294773912795\\
1.44205256165551	-1.0827165093726e-16\\
};
\addplot [color=mycolor4, line width=\myLineWidth, forget plot]
  table[row sep=crcr]{%
1.3251390704107	0\\
1.3246859599581	0.0171593273066003\\
1.32332789150043	0.0342708284903528\\
1.32106865021805	0.0512868107283919\\
1.31791453302162	0.068159847426286\\
1.31387433100149	0.0848429104044619\\
1.30895930492531	0.101289500974176\\
1.30318315385232	0.117453779537702\\
1.29656197695154	0.133290693351506\\
1.28911422863055	0.148756102096329\\
1.28086066709972	0.163806900904172\\
1.27182429651538	0.178401140499291\\
1.26203030286308	0.192498144118358\\
1.25150598375975	0.20605862088389\\
1.24028067237034	0.219044775314977\\
1.22838565565103	0.231420412670061\\
1.21585408714686	0.243151039828179\\
1.2027208945869	0.254203961427471\\
1.18902268253432	0.264548370993022\\
1.17479763036299	0.274155436800031\\
1.16008538584467	0.282998382233\\
1.14492695464358	0.291052560416963\\
1.12936458602624	0.29829552291275\\
1.11344165510521	0.30470708228481\\
1.09720254194486	0.310269368367214\\
1.08069250786615	0.314966878071016\\
1.06395756929526	0.318786518594141\\
1.04704436950752	0.321717643913372\\
1.03000004862422	0.323752084456727\\
1.01287211222469	0.32488416987352\\
0.995708298939684	0.325110744838647\\
0.978556447395282	0.324431177847043\\
0.961464362878039	0.322847362973798\\
0.944479684093088	0.320363714595033\\
0.927649750386524	0.316987155084238\\
0.911021469802152	0.312727095518379\\
0.894641188340357	0.307595409447538\\
0.878554560783491	0.301606399801198\\
0.862806423447802	0.294776759023413\\
0.847440669216584	0.287125522547974\\
0.832500125202833	0.278674015743241\\
0.818026433382405	0.269445794474515\\
0.804059934530361	0.259466579449618\\
0.790639555783991	0.248764184530664\\
0.777802702145897	0.237368439211842\\
0.765585152229548	0.225311105479262\\
0.754020958537864	0.212625789284602\\
0.743142352552782	0.199347846879297\\
0.732979654900334	0.18551428627032\\
0.72356119084163	0.171163664072235\\
0.714913211325266	0.156335978042991\\
0.707059819821231	0.141072555602999\\
0.700022905140206	0.125415938648201\\
0.693822080425528	0.109409764978173\\
0.688474628487836	0.0930986466697504\\
0.683995453634783	0.0765280457351715\\
0.680397040130057	0.0597441474112931\\
0.677689417397494	0.0427937314330526\\
0.675880132067282	0.0257240416499555\\
0.674974226942144	0.00858265434899496\\
0.674974226942144	-0.00858265434899474\\
0.675880132067282	-0.0257240416499555\\
0.677689417397494	-0.0427937314330524\\
0.680397040130057	-0.0597441474112932\\
0.683995453634783	-0.0765280457351714\\
0.688474628487836	-0.0930986466697502\\
0.693822080425528	-0.109409764978173\\
0.700022905140206	-0.125415938648201\\
0.707059819821231	-0.141072555602999\\
0.714913211325266	-0.156335978042991\\
0.723561190841629	-0.171163664072235\\
0.732979654900334	-0.18551428627032\\
0.743142352552782	-0.199347846879296\\
0.754020958537864	-0.212625789284602\\
0.765585152229548	-0.225311105479262\\
0.777802702145897	-0.237368439211842\\
0.79063955578399	-0.248764184530664\\
0.804059934530361	-0.259466579449617\\
0.818026433382405	-0.269445794474515\\
0.832500125202833	-0.278674015743241\\
0.847440669216584	-0.287125522547974\\
0.862806423447802	-0.294776759023413\\
0.87855456078349	-0.301606399801198\\
0.894641188340357	-0.307595409447538\\
0.911021469802152	-0.312727095518379\\
0.927649750386524	-0.316987155084238\\
0.944479684093088	-0.320363714595033\\
0.961464362878038	-0.322847362973798\\
0.978556447395281	-0.324431177847043\\
0.995708298939684	-0.325110744838647\\
1.01287211222469	-0.32488416987352\\
1.03000004862422	-0.323752084456727\\
1.04704436950752	-0.321717643913372\\
1.06395756929526	-0.318786518594141\\
1.08069250786615	-0.314966878071016\\
1.09720254194486	-0.310269368367214\\
1.11344165510521	-0.30470708228481\\
1.12936458602624	-0.29829552291275\\
1.14492695464358	-0.291052560416963\\
1.16008538584467	-0.282998382232999\\
1.17479763036299	-0.27415543680003\\
1.18902268253432	-0.264548370993022\\
1.2027208945869	-0.254203961427471\\
1.21585408714686	-0.243151039828179\\
1.22838565565103	-0.231420412670061\\
1.24028067237034	-0.219044775314976\\
1.25150598375975	-0.20605862088389\\
1.26203030286308	-0.192498144118358\\
1.27182429651538	-0.178401140499291\\
1.28086066709972	-0.163806900904172\\
1.28911422863055	-0.148756102096329\\
1.29656197695154	-0.133290693351506\\
1.30318315385232	-0.117453779537702\\
1.30895930492531	-0.101289500974176\\
1.31387433100149	-0.0848429104044621\\
1.31791453302162	-0.0681598474262859\\
1.32106865021805	-0.0512868107283918\\
1.32332789150043	-0.0342708284903529\\
1.3246859599581	-0.0171593273066004\\
1.3251390704107	-7.96361043712407e-17\\
};
\addplot [color=mycolor5, line width=\myLineWidth, forget plot]
  table[row sep=crcr]{%
1.2083553877801	0\\
1.20806502586224	0.0109960279165969\\
1.20719474939939	0.0219614079311763\\
1.20574698400815	0.0328655775629209\\
1.20372576487034	0.0436781449353283\\
1.2011367254862	0.054368973483823\\
1.19798708197287	0.0649082659517659\\
1.19428561295172	0.0752666474407537\\
1.1900426350808	0.0854152472837291\\
1.18526997430045	0.095325779512708\\
1.17998093287227	0.104970621696846\\
1.17419025230335	0.114322891931105\\
1.16791407225903	0.12335652376095\\
1.16116988557877	0.132046338834225\\
1.15397648952046	0.140368117077742\\
1.14635393336911	0.14829866420297\\
1.13832346255587	0.155815876352669\\
1.12990745944321	0.162898801708317\\
1.12112938094129	0.169527698886574\\
1.11201369312928	0.175684091962065\\
1.10258580306409	0.181350821963099\\
1.09287198796628	0.186512094696795\\
1.08289932198078	0.191153524770333\\
1.07269560071635	0.195262175685626\\
1.06228926377427	0.198826595895659\\
1.05170931548203	0.201836850722002\\
1.04098524405312	0.204284550044544\\
1.03014693939805	0.206162871686262\\
1.01922460981581	0.20746658042785\\
1.00824869779795	0.208192042599216\\
0.997249795179863	0.208337236207172\\
0.986258557875879	0.207901756571091\\
0.975305620435718	0.206886817450825\\
0.964421510660511	0.205295247663738\\
0.953636564516342	0.203131483200287\\
0.94298084158247	0.200401554860117\\
0.932484041269888	0.197113071443143\\
0.922175420043743	0.193275198542461\\
0.912083709880313	0.188898632998195\\
0.902237038185843	0.183995573083484\\
0.892662849400415	0.178579684505707\\
0.883387828505369	0.172666062317708\\
0.874437826647481	0.166271188845174\\
0.865837789087172	0.159412887747439\\
0.8576116856716	0.152110274339745\\
0.849782444026392	0.144383702315433\\
0.842371885652243	0.136254707016551\\
0.835400665104476	0.127745945410987\\
0.828888212425095	0.118881132943442\\
0.82285267898777	0.109684977436228\\
0.817310886906707	0.100183110224138\\
0.812278282150394	0.0904020147153186\\
0.807768891490919	0.080368952577259\\
0.803795283408841	0.0701118877536378\\
0.800368533062578	0.0596594085237942\\
0.797498191419958	0.0490406478220662\\
0.795192258637954	0.0382852020390779\\
0.793457161764814	0.0274230485312904\\
0.792297736826725	0.0164844620687321\\
0.791717215348941	0.00549993045378586\\
0.791717215348941	-0.00549993045378572\\
0.792297736826725	-0.0164844620687321\\
0.793457161764814	-0.0274230485312902\\
0.795192258637954	-0.0382852020390779\\
0.797498191419958	-0.0490406478220662\\
0.800368533062578	-0.0596594085237941\\
0.803795283408841	-0.0701118877536379\\
0.807768891490919	-0.080368952577259\\
0.812278282150394	-0.0904020147153186\\
0.817310886906707	-0.100183110224138\\
0.82285267898777	-0.109684977436228\\
0.828888212425095	-0.118881132943442\\
0.835400665104476	-0.127745945410987\\
0.842371885652243	-0.136254707016551\\
0.849782444026392	-0.144383702315434\\
0.857611685671599	-0.152110274339745\\
0.865837789087172	-0.159412887747439\\
0.87443782664748	-0.166271188845174\\
0.883387828505369	-0.172666062317708\\
0.892662849400415	-0.178579684505707\\
0.902237038185843	-0.183995573083484\\
0.912083709880313	-0.188898632998195\\
0.922175420043742	-0.193275198542461\\
0.932484041269888	-0.197113071443143\\
0.94298084158247	-0.200401554860117\\
0.953636564516342	-0.203131483200287\\
0.964421510660511	-0.205295247663738\\
0.975305620435718	-0.206886817450825\\
0.986258557875879	-0.207901756571091\\
0.997249795179863	-0.208337236207172\\
1.00824869779795	-0.208192042599216\\
1.01922460981581	-0.20746658042785\\
1.03014693939805	-0.206162871686262\\
1.04098524405312	-0.204284550044544\\
1.05170931548203	-0.201836850722002\\
1.06228926377427	-0.198826595895659\\
1.07269560071635	-0.195262175685626\\
1.08289932198078	-0.191153524770333\\
1.09287198796628	-0.186512094696795\\
1.10258580306409	-0.181350821963099\\
1.11201369312928	-0.175684091962065\\
1.12112938094129	-0.169527698886574\\
1.12990745944321	-0.162898801708317\\
1.13832346255587	-0.155815876352669\\
1.14635393336911	-0.14829866420297\\
1.15397648952046	-0.140368117077742\\
1.16116988557877	-0.132046338834225\\
1.16791407225903	-0.12335652376095\\
1.17419025230335	-0.114322891931105\\
1.17998093287227	-0.104970621696846\\
1.18526997430045	-0.095325779512708\\
1.1900426350808	-0.0854152472837291\\
1.19428561295172	-0.0752666474407537\\
1.19798708197287	-0.0649082659517658\\
1.2011367254862	-0.0543689734838231\\
1.20372576487034	-0.0436781449353283\\
1.20574698400815	-0.0328655775629208\\
1.20719474939939	-0.0219614079311764\\
1.20806502586224	-0.0109960279165969\\
1.2083553877801	-5.1032351746001e-17\\
};
\addplot [color=mycolor4!86!mycolor5, line width=\myLineWidth, forget plot]
  table[row sep=crcr]{%
1.09118734048466	0\\
1.09106026274816	0.00481244354798184\\
1.09067938372708	0.00961147395265034\\
1.09004576499959	0.0143837154555283\\
1.0891611725747	0.0191158669636127\\
1.08802807197018	0.0237947391219061\\
1.08664962134063	0.0284072910745155\\
1.08502966267519	0.0329406668118555\\
1.08317271108922	0.0373822310026539\\
1.08108394223984	0.0417196042108857\\
1.07876917790047	0.0459406973994814\\
1.07623486973449	0.0500337456246423\\
1.07348808131331	0.0539873408268481\\
1.07053646842892	0.0577904636271656\\
1.0673882577558	0.0614325140402339\\
1.06405222392177	0.0649033410183248\\
1.06053766505147	0.0681932707441335\\
1.05685437685081	0.0712931335934424\\
1.05301262530464	0.0741942896925086\\
1.04902311806357	0.0768886529989415\\
1.04489697459982	0.0793687138389534\\
1.04064569521527	0.0816275598381671\\
1.03628112898807	0.0836588951876431\\
1.03181544074711	0.0854570581914283\\
1.02726107716647	0.0870170370467181\\
1.0226307320743	0.0883344838126499\\
1.01793731107286	0.0894057265287937\\
1.01319389556829	0.0902277794495649\\
1.00841370631034	0.0907983513660324\\
1.00361006654385	0.091115851991929\\
0.998796364874418	0.0911793963960643\\
0.993986017951957	0.090988807468787\\
0.989192433076114	0.0905446164156207\\
0.984428970827699	0.0898480612766994\\
0.97970890783036	0.0889010834761272\\
0.975045399746246	0.0877063224108814\\
0.970451444608819	0.0862671080943396\\
0.965939846595006	0.0845874518749349\\
0.961523180337657	0.0826720352558088\\
0.957213755877792	0.0805261968466222\\
0.953023584354304	0.0781559174838923\\
0.948964344526767	0.0755678035613292\\
0.945047350224633	0.0727690686166312\\
0.941283518813561	0.0697675132260617\\
0.937683340766764	0.0665715032628441\\
0.934256850426174	0.0631899465799722\\
0.931013598034937	0.0596322681824269\\
0.927962623119165	0.0559083839579965\\
0.925112429293164	0.0520286730399189\\
0.922470960558327	0.0480039488783759\\
0.920045579161785	0.0438454291014677\\
0.917843045076499	0.0395647042496722\\
0.915869497160004	0.0351737054709288\\
0.914130436044313	0.0306846712663899\\
0.912630708804669	0.0261101133795232\\
0.911374495449873	0.0214627819236378\\
0.910365297271858	0.0167556298450326\\
0.909605927086951	0.0120017768208108\\
0.909098501396057	0.00721447269198757\\
0.908844434485585	0.00240706053380611\\
0.908844434485585	-0.00240706053380605\\
0.909098501396057	-0.00721447269198759\\
0.909605927086951	-0.0120017768208108\\
0.910365297271858	-0.0167556298450326\\
0.911374495449873	-0.0214627819236378\\
0.912630708804669	-0.0261101133795231\\
0.914130436044313	-0.03068467126639\\
0.915869497160004	-0.0351737054709287\\
0.917843045076499	-0.0395647042496722\\
0.920045579161785	-0.0438454291014678\\
0.922470960558327	-0.0480039488783759\\
0.925112429293164	-0.052028673039919\\
0.927962623119165	-0.0559083839579965\\
0.931013598034937	-0.0596322681824269\\
0.934256850426174	-0.0631899465799722\\
0.937683340766764	-0.066571503262844\\
0.941283518813561	-0.0697675132260617\\
0.945047350224633	-0.0727690686166312\\
0.948964344526767	-0.0755678035613292\\
0.953023584354304	-0.0781559174838923\\
0.957213755877792	-0.0805261968466222\\
0.961523180337657	-0.0826720352558088\\
0.965939846595006	-0.0845874518749349\\
0.970451444608819	-0.0862671080943396\\
0.975045399746246	-0.0877063224108814\\
0.97970890783036	-0.0889010834761272\\
0.984428970827699	-0.0898480612766994\\
0.989192433076114	-0.0905446164156206\\
0.993986017951957	-0.090988807468787\\
0.998796364874418	-0.0911793963960643\\
1.00361006654385	-0.091115851991929\\
1.00841370631034	-0.0907983513660324\\
1.01319389556829	-0.0902277794495649\\
1.01793731107286	-0.0894057265287937\\
1.0226307320743	-0.0883344838126499\\
1.02726107716647	-0.0870170370467181\\
1.03181544074711	-0.0854570581914283\\
1.03628112898807	-0.0836588951876431\\
1.04064569521527	-0.0816275598381672\\
1.04489697459982	-0.0793687138389534\\
1.04902311806357	-0.0768886529989415\\
1.05301262530464	-0.0741942896925086\\
1.05685437685081	-0.0712931335934424\\
1.06053766505147	-0.0681932707441335\\
1.06405222392177	-0.0649033410183248\\
1.0673882577558	-0.0614325140402339\\
1.07053646842892	-0.0577904636271656\\
1.07348808131331	-0.0539873408268481\\
1.07623486973449	-0.0500337456246423\\
1.07876917790047	-0.0459406973994815\\
1.08108394223984	-0.0417196042108857\\
1.08317271108922	-0.0373822310026539\\
1.08502966267519	-0.0329406668118555\\
1.08664962134063	-0.0284072910745155\\
1.08802807197018	-0.0237947391219062\\
1.0891611725747	-0.0191158669636126\\
1.09004576499959	-0.0143837154555283\\
1.09067938372708	-0.00961147395265039\\
1.09106026274816	-0.00481244354798187\\
1.09118734048466	-2.23344569294609e-17\\
};
\end{axis}

\end{tikzpicture}%}
\label{fig:CapAchievingDistr_i}}
\subfloat[load impedance]{
%\includegraphics[height=46.8mm]{CapacityAchievingDistr_zL-crop}
\resizebox{!}{46mm}{% This file was created by matlab2tikz.
%
%The latest updates can be retrieved from
%  http://www.mathworks.com/matlabcentral/fileexchange/22022-matlab2tikz-matlab2tikz
%where you can also make suggestions and rate matlab2tikz.
%
\definecolor{mycolor1}{rgb}{0.89974,0.89974,0.89974}%
\definecolor{mycolor2}{rgb}{0.89334,0.89334,0.89334}%
\definecolor{mycolor3}{rgb}{0.88634,0.88634,0.88634}%
\definecolor{mycolor4}{rgb}{0.87866,0.87866,0.87866}%
\definecolor{mycolor5}{rgb}{0.87021,0.87021,0.87021}%
\definecolor{mycolor6}{rgb}{0.85061,0.85061,0.85061}%
\definecolor{mycolor7}{rgb}{0.82656,0.82656,0.82656}%
\definecolor{mycolor8}{rgb}{0.81247,0.81247,0.81247}%
\definecolor{mycolor9}{rgb}{0.79674,0.79674,0.79674}%
\definecolor{mycolor10}{rgb}{0.77914,0.77914,0.77914}%
\definecolor{mycolor11}{rgb}{0.75941,0.75941,0.75941}%
\definecolor{mycolor12}{rgb}{0.73724,0.73724,0.73724}%
\definecolor{mycolor13}{rgb}{0.71228,0.71228,0.71228}%
\definecolor{mycolor14}{rgb}{0.65246,0.65246,0.65246}%
\definecolor{mycolor15}{rgb}{0.48143,0.48143,0.48143}%
\definecolor{mycolor16}{rgb}{0.42618,0.42618,0.42618}%
\definecolor{mycolor17}{rgb}{0.36615,0.36615,0.36615}%
\definecolor{mycolor18}{rgb}{0.23632,0.23632,0.23632}%
\definecolor{mycolor19}{rgb}{0.17102,0.17102,0.17102}%
\definecolor{mycolor20}{rgb}{0.11015,0.11015,0.11015}%
\definecolor{mycolor21}{rgb}{0.05828,0.05828,0.05828}%
\definecolor{mycolor22}{rgb}{0.02021,0.02021,0.02021}%
\definecolor{mycolor23}{rgb}{0.44363,0.44363,0.44363}%
\definecolor{mycolor24}{rgb}{0.45297,0.45297,0.45297}%
\definecolor{mycolor25}{rgb}{0.62564,0.62564,0.62564}%
\definecolor{mycolor26}{rgb}{0.79897,0.79897,0.79897}%
\definecolor{mycolor27}{rgb}{0.90104,0.90104,0.90104}%
\definecolor{mycolor28}{rgb}{0.92208,0.92208,0.92208}%
\definecolor{mycolor29}{rgb}{0.94082,0.94082,0.94082}%
\definecolor{mycolor30}{rgb}{0.95704,0.95704,0.95704}%
\definecolor{mycolor31}{rgb}{0.58892,0.58892,0.58892}%
\definecolor{mycolor32}{rgb}{0.66280,0.66280,0.66280}%
\definecolor{mycolor33}{rgb}{0.88298,0.88298,0.88298}%
\definecolor{mycolor34}{rgb}{0.94647,0.94647,0.94647}%
\definecolor{mycolor35}{rgb}{0.95296,0.95296,0.95296}%
\definecolor{mycolor36}{rgb}{0.96619,0.96619,0.96619}%
%
\begin{tikzpicture}

\begin{axis}[%
width=19.1mm,
height=38.2mm,
at={(0,0)},
scale only axis,
point meta min=0,
point meta max=1,
xmin=-0.3,
xmax=2.3,
xtick={-10,  -9,  -8,  -7,  -6,  -5,  -4,  -3,  -2,  -1,   0,   1,   2,   3,   4,   5,   6,   7,   8,   9,  10},
xlabel style={font=\color{white!15!black}},
xlabel={$\text{normalized load resistance }\Re(z)$},
ymin=-2.4,
ymax=2.4,
ytick={-10,  -9,  -8,  -7,  -6,  -5,  -4,  -3,  -2,  -1,   0,   1,   2,   3,   4,   5,   6,   7,   8,   9,  10},
ylabel style={font=\color{white!15!black}},
ylabel={$\text{normalized load reactance }\Im(z)$},
axis background/.style={fill=white},
xmajorgrids,
ymajorgrids,
legend style={},
colormap={mymap}{[1pt] rgb(0pt)=(1,1,1); rgb(255pt)=(0,0,0)},
colormap access=piecewise const,
colorbar,
colorbar sampled, % makes it properly printable
colorbar style={at={(1.05,1.0)}, width=3mm, ylabel={probability density}, ylabel style={at={(1.08,0.5)}}, ytick={0,.97}, yticklabels={0, high}, samples=80},
]

\newcommand\myLineWidth{1.5pt}
% ----------------------------------------------------------------------------
% HIGH RES
% ----------------------------------------------------------------------------    
	\addplot [color=mycolor1, line width=\myLineWidth, forget plot]
	  table[row sep=crcr]{%
	0	-3\\
	0	-2.89830508474576\\
	};
	\addplot [color=mycolor2, line width=\myLineWidth, forget plot]
	  table[row sep=crcr]{%
	0	-2.89830508474576\\
	0	-2.79661016949153\\
	};
	\addplot [color=mycolor3, line width=\myLineWidth, forget plot]
	  table[row sep=crcr]{%
	0	-2.79661016949153\\
	0	-2.69491525423729\\
	};
	\addplot [color=mycolor4, line width=\myLineWidth, forget plot]
	  table[row sep=crcr]{%
	0	-2.69491525423729\\
	0	-2.59322033898305\\
	};
	\addplot [color=mycolor5, line width=\myLineWidth, forget plot]
	  table[row sep=crcr]{%
	0	-2.59322033898305\\
	0	-2.49152542372881\\
	};
	\addplot [color=darkgray!4!mycolor3, line width=\myLineWidth, forget plot]
	  table[row sep=crcr]{%
	0	-2.49152542372881\\
	0	-2.38983050847458\\
	};
	\addplot [color=mycolor6, line width=\myLineWidth, forget plot]
	  table[row sep=crcr]{%
	0	-2.38983050847458\\
	0	-2.28813559322034\\
	};
	\addplot [color=darkgray!5!mycolor5, line width=\myLineWidth, forget plot]
	  table[row sep=crcr]{%
	0	-2.28813559322034\\
	0	-2.1864406779661\\
	};
	\addplot [color=mycolor7, line width=\myLineWidth, forget plot]
	  table[row sep=crcr]{%
	0	-2.1864406779661\\
	0	-2.08474576271186\\
	};
	\addplot [color=mycolor8, line width=\myLineWidth, forget plot]
	  table[row sep=crcr]{%
	0	-2.08474576271186\\
	0	-1.98305084745763\\
	};
	\addplot [color=mycolor9, line width=\myLineWidth, forget plot]
	  table[row sep=crcr]{%
	0	-1.98305084745763\\
	0	-1.88135593220339\\
	};
	\addplot [color=mycolor10, line width=\myLineWidth, forget plot]
	  table[row sep=crcr]{%
	0	-1.88135593220339\\
	0	-1.77966101694915\\
	};
	\addplot [color=mycolor11, line width=\myLineWidth, forget plot]
	  table[row sep=crcr]{%
	0	-1.77966101694915\\
	0	-1.67796610169492\\
	};
	\addplot [color=mycolor12, line width=\myLineWidth, forget plot]
	  table[row sep=crcr]{%
	0	-1.67796610169492\\
	0	-1.57627118644068\\
	};
	\addplot [color=mycolor13, line width=\myLineWidth, forget plot]
	  table[row sep=crcr]{%
	0	-1.57627118644068\\
	0	-1.47457627118644\\
	};
	\addplot [color=darkgray!30!mycolor5, line width=\myLineWidth, forget plot]
	  table[row sep=crcr]{%
	0	-1.47457627118644\\
	0	-1.3728813559322\\
	};
	\addplot [color=mycolor14, line width=\myLineWidth, forget plot]
	  table[row sep=crcr]{%
	0	-1.3728813559322\\
	0	-1.27118644067797\\
	};
	\addplot [color=gray!55!mycolor11, line width=\myLineWidth, forget plot]
	  table[row sep=crcr]{%
	0	-1.27118644067797\\
	0	-1.16949152542373\\
	};
	\addplot [color=black!26!mycolor10, line width=\myLineWidth, forget plot]
	  table[row sep=crcr]{%
	0	-1.16949152542373\\
	0	-1.06779661016949\\
	};
	\addplot [color=gray!91!mycolor6, line width=\myLineWidth, forget plot]
	  table[row sep=crcr]{%
	0	-1.06779661016949\\
	0	-0.966101694915254\\
	};
	\addplot [color=mycolor15, line width=\myLineWidth, forget plot]
	  table[row sep=crcr]{%
	0	-0.966101694915254\\
	0	-0.864406779661017\\
	};
	\addplot [color=mycolor16, line width=\myLineWidth, forget plot]
	  table[row sep=crcr]{%
	0	-0.864406779661017\\
	0	-0.76271186440678\\
	};
	\addplot [color=mycolor17, line width=\myLineWidth, forget plot]
	  table[row sep=crcr]{%
	0	-0.76271186440678\\
	0	-0.661016949152542\\
	};
	\addplot [color=black!59!mycolor12, line width=\myLineWidth, forget plot]
	  table[row sep=crcr]{%
	0	-0.661016949152542\\
	0	-0.559322033898305\\
	};
	\addplot [color=mycolor18, line width=\myLineWidth, forget plot]
	  table[row sep=crcr]{%
	0	-0.559322033898305\\
	0	-0.457627118644068\\
	};
	\addplot [color=mycolor19, line width=\myLineWidth, forget plot]
	  table[row sep=crcr]{%
	0	-0.457627118644068\\
	0	-0.355932203389831\\
	};
	\addplot [color=mycolor20, line width=\myLineWidth, forget plot]
	  table[row sep=crcr]{%
	0	-0.355932203389831\\
	0	-0.254237288135593\\
	};
	\addplot [color=mycolor21, line width=\myLineWidth, forget plot]
	  table[row sep=crcr]{%
	0	-0.254237288135593\\
	0	-0.152542372881356\\
	};
	\addplot [color=mycolor22, line width=\myLineWidth, forget plot]
	  table[row sep=crcr]{%
	0	-0.152542372881356\\
	0	-0.0508474576271186\\
	};
	\addplot [color=black, line width=\myLineWidth, forget plot]
	  table[row sep=crcr]{%
	0	-0.0508474576271186\\
	0	0.0508474576271186\\
	};
	\addplot [color=black, line width=\myLineWidth, forget plot]
	  table[row sep=crcr]{%
	0	0.0508474576271186\\
	0	0.152542372881356\\
	};
	\addplot [color=mycolor22, line width=\myLineWidth, forget plot]
	  table[row sep=crcr]{%
	0	0.152542372881356\\
	0	0.254237288135593\\
	};
	\addplot [color=mycolor21, line width=\myLineWidth, forget plot]
	  table[row sep=crcr]{%
	0	0.254237288135593\\
	0	0.355932203389831\\
	};
	\addplot [color=mycolor20, line width=\myLineWidth, forget plot]
	  table[row sep=crcr]{%
	0	0.355932203389831\\
	0	0.457627118644068\\
	};
	\addplot [color=mycolor19, line width=\myLineWidth, forget plot]
	  table[row sep=crcr]{%
	0	0.457627118644068\\
	0	0.559322033898305\\
	};
	\addplot [color=mycolor18, line width=\myLineWidth, forget plot]
	  table[row sep=crcr]{%
	0	0.559322033898305\\
	0	0.661016949152542\\
	};
	\addplot [color=black!59!mycolor12, line width=\myLineWidth, forget plot]
	  table[row sep=crcr]{%
	0	0.661016949152542\\
	0	0.76271186440678\\
	};
	\addplot [color=mycolor17, line width=\myLineWidth, forget plot]
	  table[row sep=crcr]{%
	0	0.76271186440678\\
	0	0.864406779661017\\
	};
	\addplot [color=mycolor16, line width=\myLineWidth, forget plot]
	  table[row sep=crcr]{%
	0	0.864406779661017\\
	0	0.966101694915254\\
	};
	\addplot [color=mycolor15, line width=\myLineWidth, forget plot]
	  table[row sep=crcr]{%
	0	0.966101694915254\\
	0	1.06779661016949\\
	};
	\addplot [color=gray!91!mycolor6, line width=\myLineWidth, forget plot]
	  table[row sep=crcr]{%
	0	1.06779661016949\\
	0	1.16949152542373\\
	};
	\addplot [color=black!26!mycolor10, line width=\myLineWidth, forget plot]
	  table[row sep=crcr]{%
	0	1.16949152542373\\
	0	1.27118644067797\\
	};
	\addplot [color=gray!55!mycolor11, line width=\myLineWidth, forget plot]
	  table[row sep=crcr]{%
	0	1.27118644067797\\
	0	1.3728813559322\\
	};
	\addplot [color=mycolor14, line width=\myLineWidth, forget plot]
	  table[row sep=crcr]{%
	0	1.3728813559322\\
	0	1.47457627118644\\
	};
	\addplot [color=darkgray!30!mycolor5, line width=\myLineWidth, forget plot]
	  table[row sep=crcr]{%
	0	1.47457627118644\\
	0	1.57627118644068\\
	};
	\addplot [color=mycolor13, line width=\myLineWidth, forget plot]
	  table[row sep=crcr]{%
	0	1.57627118644068\\
	0	1.67796610169492\\
	};
	\addplot [color=mycolor12, line width=\myLineWidth, forget plot]
	  table[row sep=crcr]{%
	0	1.67796610169492\\
	0	1.77966101694915\\
	};
	\addplot [color=mycolor11, line width=\myLineWidth, forget plot]
	  table[row sep=crcr]{%
	0	1.77966101694915\\
	0	1.88135593220339\\
	};
	\addplot [color=mycolor10, line width=\myLineWidth, forget plot]
	  table[row sep=crcr]{%
	0	1.88135593220339\\
	0	1.98305084745763\\
	};
	\addplot [color=mycolor9, line width=\myLineWidth, forget plot]
	  table[row sep=crcr]{%
	0	1.98305084745763\\
	0	2.08474576271186\\
	};
	\addplot [color=mycolor8, line width=\myLineWidth, forget plot]
	  table[row sep=crcr]{%
	0	2.08474576271186\\
	0	2.1864406779661\\
	};
	\addplot [color=mycolor7, line width=\myLineWidth, forget plot]
	  table[row sep=crcr]{%
	0	2.1864406779661\\
	0	2.28813559322034\\
	};
	\addplot [color=darkgray!5!mycolor5, line width=\myLineWidth, forget plot]
	  table[row sep=crcr]{%
	0	2.28813559322034\\
	0	2.38983050847458\\
	};
	\addplot [color=mycolor6, line width=\myLineWidth, forget plot]
	  table[row sep=crcr]{%
	0	2.38983050847458\\
	0	2.49152542372881\\
	};
	\addplot [color=darkgray!4!mycolor3, line width=\myLineWidth, forget plot]
	  table[row sep=crcr]{%
	0	2.49152542372881\\
	0	2.59322033898305\\
	};
	\addplot [color=mycolor5, line width=\myLineWidth, forget plot]
	  table[row sep=crcr]{%
	0	2.59322033898305\\
	0	2.69491525423729\\
	};
	\addplot [color=mycolor4, line width=\myLineWidth, forget plot]
	  table[row sep=crcr]{%
	0	2.69491525423729\\
	0	2.79661016949153\\
	};
	\addplot [color=mycolor3, line width=\myLineWidth, forget plot]
	  table[row sep=crcr]{%
	0	2.79661016949153\\
	0	2.89830508474576\\
	};
	\addplot [color=mycolor23, line width=\myLineWidth, forget plot]
	  table[row sep=crcr]{%
	0.0952254491249849	0\\
	0.0954934911614811	-0.0528130559266341\\
	};
	\addplot [color=lightgray!21!mycolor17, line width=\myLineWidth, forget plot]
	  table[row sep=crcr]{%
	0.0954934911614811	-0.0528130559266341\\
	0.0963036506746893	-0.105918766001882\\
	};
	\addplot [color=mycolor24, line width=\myLineWidth, forget plot]
	  table[row sep=crcr]{%
	0.0963036506746893	-0.105918766001882\\
	0.0976743168790093	-0.159616210581166\\
	};
	\addplot [color=mycolor1!18!mycolor17, line width=\myLineWidth, forget plot]
	  table[row sep=crcr]{%
	0.0976743168790093	-0.159616210581166\\
	0.0996371256508837	-0.214217619442769\\
	};
	\addplot [color=mycolor9!13!mycolor16, line width=\myLineWidth, forget plot]
	  table[row sep=crcr]{%
	0.0996371256508837	-0.214217619442769\\
	0.102238526775186	-0.270055712343974\\
	};
	\addplot [color=mycolor1!10!mycolor23, line width=\myLineWidth, forget plot]
	  table[row sep=crcr]{%
	0.102238526775186	-0.270055712343974\\
	0.105542162452084	-0.327492026655606\\
	};
	\addplot [color=mycolor11!20!mycolor23, line width=\myLineWidth, forget plot]
	  table[row sep=crcr]{%
	0.105542162452084	-0.327492026655606\\
	0.109632270229866	-0.386926683292376\\
	};
	\addplot [color=gray!91!mycolor9, line width=\myLineWidth, forget plot]
	  table[row sep=crcr]{%
	0.109632270229866	-0.386926683292376\\
	0.114618431632945	-0.44881016592718\\
	};
	\addplot [color=gray!77!mycolor13, line width=\myLineWidth, forget plot]
	  table[row sep=crcr]{%
	0.114618431632945	-0.44881016592718\\
	0.120642143524215	-0.513657870788634\\
	};
	\addplot [color=white!52!mycolor20, line width=\myLineWidth, forget plot]
	  table[row sep=crcr]{%
	0.120642143524215	-0.513657870788634\\
	0.127885920851911	-0.582068450456598\\
	};
	\addplot [color=gray!62!mycolor11, line width=\myLineWidth, forget plot]
	  table[row sep=crcr]{%
	0.127885920851911	-0.582068450456598\\
	0.136585992730136	-0.654747364163485\\
	};
	\addplot [color=mycolor25, line width=\myLineWidth, forget plot]
	  table[row sep=crcr]{%
	0.136585992730136	-0.654747364163485\\
	0.147050205386707	-0.732537620281085\\
	};
	\addplot [color=mycolor7!14!mycolor25, line width=\myLineWidth, forget plot]
	  table[row sep=crcr]{%
	0.147050205386707	-0.732537620281085\\
	0.159683626631562	-0.816460550170288\\
	};
	\addplot [color=mycolor3!53!mycolor24, line width=\myLineWidth, forget plot]
	  table[row sep=crcr]{%
	0.159683626631562	-0.816460550170288\\
	0.175025787974332	-0.907770739696575\\
	};
	\addplot [color=mycolor7!84!mycolor20, line width=\myLineWidth, forget plot]
	  table[row sep=crcr]{%
	0.175025787974332	-0.907770739696575\\
	0.193805919701929	-1.00803121220865\\
	};
	\addplot [color=lightgray!32!mycolor12, line width=\myLineWidth, forget plot]
	  table[row sep=crcr]{%
	0.193805919701929	-1.00803121220865\\
	0.217026708801988	-1.11921800472433\\
	};
	\addplot [color=gray!30!mycolor3, line width=\myLineWidth, forget plot]
	  table[row sep=crcr]{%
	0.217026708801988	-1.11921800472433\\
	0.2460945372704	-1.24386805644068\\
	};
	\addplot [color=mycolor26, line width=\myLineWidth, forget plot]
	  table[row sep=crcr]{%
	0.2460945372704	-1.24386805644068\\
	0.2830278297777	-1.38529187244494\\
	};
	\addplot [color=mycolor1!29!mycolor9, line width=\myLineWidth, forget plot]
	  table[row sep=crcr]{%
	0.2830278297777	-1.38529187244494\\
	0.330801270158868	-1.54788430816706\\
	};
	\addplot [color=white!85!mycolor22, line width=\myLineWidth, forget plot]
	  table[row sep=crcr]{%
	0.330801270158868	-1.54788430816706\\
	0.393935731824674	-1.73758506032354\\
	};
	\addplot [color=mycolor1!75!mycolor8, line width=\myLineWidth, forget plot]
	  table[row sep=crcr]{%
	0.393935731824674	-1.73758506032354\\
	0.479552552686613	-1.96256603885877\\
	};
	\addplot [color=mycolor27, line width=\myLineWidth, forget plot]
	  table[row sep=crcr]{%
	0.479552552686613	-1.96256603885877\\
	0.59934999644412	-2.23424789629958\\
	};
	\addplot [color=mycolor28, line width=\myLineWidth, forget plot]
	  table[row sep=crcr]{%
	0.59934999644412	-2.23424789629958\\
	0.773515641924226	-2.56872045410097\\
	};
	\addplot [color=mycolor29, line width=\myLineWidth, forget plot]
	  table[row sep=crcr]{%
	0.773515641924226	-2.56872045410097\\
	1.03895325559561	-2.98830534636558\\
	};
	\addplot [color=mycolor30, line width=\myLineWidth, forget plot]
	  table[row sep=crcr]{%
	1.03895325559561	-2.98830534636558\\
	1.46769144477761	-3.521143301134\\
	};
	\addplot [color=mycolor30, line width=\myLineWidth, forget plot]
	  table[row sep=crcr]{%
	1.4676914447776	3.52114330113399\\
	1.03895325559562	2.98830534636558\\
	};
	\addplot [color=mycolor29, line width=\myLineWidth, forget plot]
	  table[row sep=crcr]{%
	1.03895325559562	2.98830534636558\\
	0.773515641924226	2.56872045410097\\
	};
	\addplot [color=mycolor28, line width=\myLineWidth, forget plot]
	  table[row sep=crcr]{%
	0.773515641924226	2.56872045410097\\
	0.599349996444121	2.23424789629958\\
	};
	\addplot [color=mycolor27, line width=\myLineWidth, forget plot]
	  table[row sep=crcr]{%
	0.599349996444121	2.23424789629958\\
	0.479552552686613	1.96256603885877\\
	};
	\addplot [color=mycolor1!75!mycolor8, line width=\myLineWidth, forget plot]
	  table[row sep=crcr]{%
	0.479552552686613	1.96256603885877\\
	0.393935731824674	1.73758506032354\\
	};
	\addplot [color=white!85!mycolor22, line width=\myLineWidth, forget plot]
	  table[row sep=crcr]{%
	0.393935731824674	1.73758506032354\\
	0.330801270158867	1.54788430816706\\
	};
	\addplot [color=lightgray!63!mycolor30, line width=\myLineWidth, forget plot]
	  table[row sep=crcr]{%
	0.330801270158867	1.54788430816706\\
	0.2830278297777	1.38529187244494\\
	};
	\addplot [color=mycolor26, line width=\myLineWidth, forget plot]
	  table[row sep=crcr]{%
	0.2830278297777	1.38529187244494\\
	0.2460945372704	1.24386805644068\\
	};
	\addplot [color=gray!30!mycolor3, line width=\myLineWidth, forget plot]
	  table[row sep=crcr]{%
	0.2460945372704	1.24386805644068\\
	0.217026708801989	1.11921800472433\\
	};
	\addplot [color=lightgray!32!mycolor12, line width=\myLineWidth, forget plot]
	  table[row sep=crcr]{%
	0.217026708801989	1.11921800472433\\
	0.193805919701929	1.00803121220865\\
	};
	\addplot [color=mycolor7!84!mycolor20, line width=\myLineWidth, forget plot]
	  table[row sep=crcr]{%
	0.193805919701929	1.00803121220865\\
	0.175025787974332	0.907770739696576\\
	};
	\addplot [color=mycolor3!53!mycolor24, line width=\myLineWidth, forget plot]
	  table[row sep=crcr]{%
	0.175025787974332	0.907770739696576\\
	0.159683626631562	0.816460550170288\\
	};
	\addplot [color=mycolor7!14!mycolor25, line width=\myLineWidth, forget plot]
	  table[row sep=crcr]{%
	0.159683626631562	0.816460550170288\\
	0.147050205386707	0.732537620281085\\
	};
	\addplot [color=mycolor25, line width=\myLineWidth, forget plot]
	  table[row sep=crcr]{%
	0.147050205386707	0.732537620281085\\
	0.136585992730136	0.654747364163485\\
	};
	\addplot [color=gray!62!mycolor11, line width=\myLineWidth, forget plot]
	  table[row sep=crcr]{%
	0.136585992730136	0.654747364163485\\
	0.127885920851911	0.582068450456598\\
	};
	\addplot [color=white!52!mycolor20, line width=\myLineWidth, forget plot]
	  table[row sep=crcr]{%
	0.127885920851911	0.582068450456598\\
	0.120642143524215	0.513657870788634\\
	};
	\addplot [color=gray!77!mycolor13, line width=\myLineWidth, forget plot]
	  table[row sep=crcr]{%
	0.120642143524215	0.513657870788634\\
	0.114618431632946	0.448810165927181\\
	};
	\addplot [color=gray!91!mycolor9, line width=\myLineWidth, forget plot]
	  table[row sep=crcr]{%
	0.114618431632946	0.448810165927181\\
	0.109632270229866	0.386926683292376\\
	};
	\addplot [color=black!19!mycolor25, line width=\myLineWidth, forget plot]
	  table[row sep=crcr]{%
	0.109632270229866	0.386926683292376\\
	0.105542162452084	0.327492026655606\\
	};
	\addplot [color=mycolor1!10!mycolor23, line width=\myLineWidth, forget plot]
	  table[row sep=crcr]{%
	0.105542162452084	0.327492026655606\\
	0.102238526775186	0.270055712343975\\
	};
	\addplot [color=mycolor9!13!mycolor16, line width=\myLineWidth, forget plot]
	  table[row sep=crcr]{%
	0.102238526775186	0.270055712343975\\
	0.0996371256508839	0.214217619442769\\
	};
	\addplot [color=mycolor1!18!mycolor17, line width=\myLineWidth, forget plot]
	  table[row sep=crcr]{%
	0.0996371256508839	0.214217619442769\\
	0.0976743168790093	0.159616210581166\\
	};
	\addplot [color=mycolor24, line width=\myLineWidth, forget plot]
	  table[row sep=crcr]{%
	0.0976743168790093	0.159616210581166\\
	0.0963036506746893	0.105918766001883\\
	};
	\addplot [color=lightgray!21!mycolor17, line width=\myLineWidth, forget plot]
	  table[row sep=crcr]{%
	0.0963036506746893	0.105918766001883\\
	0.0954934911614811	0.0528130559266342\\
	};
	\addplot [color=mycolor23, line width=\myLineWidth, forget plot]
	  table[row sep=crcr]{%
	0.0954934911614811	0.0528130559266342\\
	0.0952254491249849	1.21354184138518e-16\\
	};
	\addplot [color=mycolor12!79!mycolor22, line width=\myLineWidth, forget plot]
	  table[row sep=crcr]{%
	0.185565218426251	0\\
	0.186074142533279	-0.0514574030797424\\
	};
	\addplot [color=mycolor31, line width=\myLineWidth, forget plot]
	  table[row sep=crcr]{%
	0.186074142533279	-0.0514574030797424\\
	0.187611924596253	-0.103177478933988\\
	};
	\addplot [color=gray!56!mycolor13, line width=\myLineWidth, forget plot]
	  table[row sep=crcr]{%
	0.187611924596253	-0.103177478933988\\
	0.190212092815495	-0.155427925004053\\
	};
	\addplot [color=mycolor6!37!mycolor24, line width=\myLineWidth, forget plot]
	  table[row sep=crcr]{%
	0.190212092815495	-0.155427925004053\\
	0.193932229029325	-0.208486673074676\\
	};
	\addplot [color=mycolor4!58!mycolor18, line width=\myLineWidth, forget plot]
	  table[row sep=crcr]{%
	0.193932229029325	-0.208486673074676\\
	0.198856659277692	-0.262647478654852\\
	};
	\addplot [color=white!40!mycolor17, line width=\myLineWidth, forget plot]
	  table[row sep=crcr]{%
	0.198856659277692	-0.262647478654852\\
	0.20510051379819	-0.318226104540972\\
	};
	\addplot [color=white!42!mycolor17, line width=\myLineWidth, forget plot]
	  table[row sep=crcr]{%
	0.20510051379819	-0.318226104540972\\
	0.212815491126285	-0.375567343664288\\
	};
	\addplot [color=gray!62!mycolor3, line width=\myLineWidth, forget plot]
	  table[row sep=crcr]{%
	0.212815491126285	-0.375567343664288\\
	0.222197823487995	-0.435053168619648\\
	};
	\addplot [color=mycolor32, line width=\myLineWidth, forget plot]
	  table[row sep=crcr]{%
	0.222197823487995	-0.435053168619648\\
	0.233499168563577	-0.497112349607058\\
	};
	\addplot [color=mycolor5!75!mycolor20, line width=\myLineWidth, forget plot]
	  table[row sep=crcr]{%
	0.233499168563577	-0.497112349607058\\
	0.247041481489265	-0.562231947587381\\
	};
	\addplot [color=gray!33!mycolor9, line width=\myLineWidth, forget plot]
	  table[row sep=crcr]{%
	0.247041481489265	-0.562231947587381\\
	0.263237405821908	-0.63097115967828\\
	};
	\addplot [color=mycolor11!13!mycolor13, line width=\myLineWidth, forget plot]
	  table[row sep=crcr]{%
	0.263237405821908	-0.63097115967828\\
	0.282618450502431	-0.703978053819974\\
	};
	\addplot [color=lightgray!12!mycolor12, line width=\myLineWidth, forget plot]
	  table[row sep=crcr]{%
	0.282618450502431	-0.703978053819974\\
	0.305874332299529	-0.782009742672315\\
	};
	\addplot [color=mycolor5!30!mycolor13, line width=\myLineWidth, forget plot]
	  table[row sep=crcr]{%
	0.305874332299529	-0.782009742672315\\
	0.333908589506982	-0.865956429447817\\
	};
	\addplot [color=black!17!mycolor29, line width=\myLineWidth, forget plot]
	  table[row sep=crcr]{%
	0.333908589506982	-0.865956429447817\\
	0.367918292398612	-0.956869328436752\\
	};
	\addplot [color=mycolor3!6!mycolor9, line width=\myLineWidth, forget plot]
	  table[row sep=crcr]{%
	0.367918292398612	-0.956869328436752\\
	0.409510021759379	-1.05599132313562\\
	};
	\addplot [color=mycolor3!73!mycolor14, line width=\myLineWidth, forget plot]
	  table[row sep=crcr]{%
	0.409510021759379	-1.05599132313562\\
	0.460871315621933	-1.16478650881315\\
	};
	\addplot [color=mycolor3!81!mycolor32, line width=\myLineWidth, forget plot]
	  table[row sep=crcr]{%
	0.460871315621933	-1.16478650881315\\
	0.525028244839401	-1.28495856700544\\
	};
	\addplot [color=lightgray!45!mycolor30, line width=\myLineWidth, forget plot]
	  table[row sep=crcr]{%
	0.525028244839401	-1.28495856700544\\
	0.606238462399726	-1.41843398164271\\
	};
	\addplot [color=mycolor33, line width=\myLineWidth, forget plot]
	  table[row sep=crcr]{%
	0.606238462399726	-1.41843398164271\\
	0.710599005771438	-1.56725479001297\\
	};
	\addplot [color=mycolor1!3!mycolor27, line width=\myLineWidth, forget plot]
	  table[row sep=crcr]{%
	0.710599005771438	-1.56725479001297\\
	0.84699337277146	-1.73325473811431\\
	};
	\addplot [color=lightgray!19!mycolor30, line width=\myLineWidth, forget plot]
	  table[row sep=crcr]{%
	0.84699337277146	-1.73325473811431\\
	1.02855937324732	-1.91723137927382\\
	};
	\addplot [color=white!41!mycolor3, line width=\myLineWidth, forget plot]
	  table[row sep=crcr]{%
	1.02855937324732	-1.91723137927382\\
	1.27488357916021	-2.11695966067174\\
	};
	\addplot [color=mycolor34, line width=\myLineWidth, forget plot]
	  table[row sep=crcr]{%
	1.27488357916021	-2.11695966067174\\
	1.61490787206471	-2.32258184357023\\
	};
	\addplot [color=white!72!mycolor6, line width=\myLineWidth, forget plot]
	  table[row sep=crcr]{%
	1.61490787206471	-2.32258184357023\\
	2.08925307304556	-2.50630718179271\\
	};
	\addplot [color=white!68!mycolor1, line width=\myLineWidth, forget plot]
	  table[row sep=crcr]{%
	2.08925307304556	-2.50630718179271\\
	2.74569093534299	-2.60135585359904\\
	};
	\addplot [color=white!88!mycolor9, line width=\myLineWidth, forget plot]
	  table[row sep=crcr]{%
	2.74569093534299	-2.60135585359904\\
	3.60713630691002	-2.46912356305801\\
	};
	\addplot [color=white!88!mycolor9, line width=\myLineWidth, forget plot]
	  table[row sep=crcr]{%
	3.60713630691003	2.46912356305801\\
	2.74569093534299	2.60135585359904\\
	};
	\addplot [color=white!68!mycolor1, line width=\myLineWidth, forget plot]
	  table[row sep=crcr]{%
	2.74569093534299	2.60135585359904\\
	2.08925307304556	2.50630718179271\\
	};
	\addplot [color=white!72!mycolor6, line width=\myLineWidth, forget plot]
	  table[row sep=crcr]{%
	2.08925307304556	2.50630718179271\\
	1.61490787206471	2.32258184357023\\
	};
	\addplot [color=mycolor34, line width=\myLineWidth, forget plot]
	  table[row sep=crcr]{%
	1.61490787206471	2.32258184357023\\
	1.27488357916021	2.11695966067174\\
	};
	\addplot [color=white!41!mycolor3, line width=\myLineWidth, forget plot]
	  table[row sep=crcr]{%
	1.27488357916021	2.11695966067174\\
	1.02855937324732	1.91723137927382\\
	};
	\addplot [color=lightgray!19!mycolor30, line width=\myLineWidth, forget plot]
	  table[row sep=crcr]{%
	1.02855937324732	1.91723137927382\\
	0.84699337277146	1.73325473811431\\
	};
	\addplot [color=mycolor1!3!mycolor27, line width=\myLineWidth, forget plot]
	  table[row sep=crcr]{%
	0.84699337277146	1.73325473811431\\
	0.710599005771438	1.56725479001297\\
	};
	\addplot [color=mycolor33, line width=\myLineWidth, forget plot]
	  table[row sep=crcr]{%
	0.710599005771438	1.56725479001297\\
	0.606238462399726	1.41843398164271\\
	};
	\addplot [color=lightgray!45!mycolor30, line width=\myLineWidth, forget plot]
	  table[row sep=crcr]{%
	0.606238462399726	1.41843398164271\\
	0.525028244839401	1.28495856700544\\
	};
	\addplot [color=mycolor3!81!mycolor32, line width=\myLineWidth, forget plot]
	  table[row sep=crcr]{%
	0.525028244839401	1.28495856700544\\
	0.460871315621934	1.16478650881315\\
	};
	\addplot [color=mycolor3!73!mycolor14, line width=\myLineWidth, forget plot]
	  table[row sep=crcr]{%
	0.460871315621934	1.16478650881315\\
	0.409510021759379	1.05599132313562\\
	};
	\addplot [color=mycolor3!6!mycolor9, line width=\myLineWidth, forget plot]
	  table[row sep=crcr]{%
	0.409510021759379	1.05599132313562\\
	0.367918292398612	0.956869328436752\\
	};
	\addplot [color=black!17!mycolor29, line width=\myLineWidth, forget plot]
	  table[row sep=crcr]{%
	0.367918292398612	0.956869328436752\\
	0.333908589506982	0.865956429447817\\
	};
	\addplot [color=mycolor5!30!mycolor13, line width=\myLineWidth, forget plot]
	  table[row sep=crcr]{%
	0.333908589506982	0.865956429447817\\
	0.305874332299529	0.782009742672316\\
	};
	\addplot [color=lightgray!12!mycolor12, line width=\myLineWidth, forget plot]
	  table[row sep=crcr]{%
	0.305874332299529	0.782009742672316\\
	0.282618450502431	0.703978053819974\\
	};
	\addplot [color=darkgray!26!mycolor33, line width=\myLineWidth, forget plot]
	  table[row sep=crcr]{%
	0.282618450502431	0.703978053819974\\
	0.263237405821908	0.63097115967828\\
	};
	\addplot [color=gray!33!mycolor9, line width=\myLineWidth, forget plot]
	  table[row sep=crcr]{%
	0.263237405821908	0.63097115967828\\
	0.247041481489265	0.562231947587382\\
	};
	\addplot [color=mycolor5!75!mycolor20, line width=\myLineWidth, forget plot]
	  table[row sep=crcr]{%
	0.247041481489265	0.562231947587382\\
	0.233499168563577	0.497112349607058\\
	};
	\addplot [color=mycolor32, line width=\myLineWidth, forget plot]
	  table[row sep=crcr]{%
	0.233499168563577	0.497112349607058\\
	0.222197823487995	0.435053168619648\\
	};
	\addplot [color=gray!62!mycolor3, line width=\myLineWidth, forget plot]
	  table[row sep=crcr]{%
	0.222197823487995	0.435053168619648\\
	0.212815491126285	0.375567343664288\\
	};
	\addplot [color=white!42!mycolor17, line width=\myLineWidth, forget plot]
	  table[row sep=crcr]{%
	0.212815491126285	0.375567343664288\\
	0.20510051379819	0.318226104540972\\
	};
	\addplot [color=white!40!mycolor17, line width=\myLineWidth, forget plot]
	  table[row sep=crcr]{%
	0.20510051379819	0.318226104540972\\
	0.198856659277692	0.262647478654852\\
	};
	\addplot [color=mycolor4!58!mycolor18, line width=\myLineWidth, forget plot]
	  table[row sep=crcr]{%
	0.198856659277692	0.262647478654852\\
	0.193932229029325	0.208486673074676\\
	};
	\addplot [color=mycolor6!37!mycolor24, line width=\myLineWidth, forget plot]
	  table[row sep=crcr]{%
	0.193932229029325	0.208486673074676\\
	0.190212092815495	0.155427925004053\\
	};
	\addplot [color=gray!56!mycolor13, line width=\myLineWidth, forget plot]
	  table[row sep=crcr]{%
	0.190212092815495	0.155427925004053\\
	0.187611924596253	0.103177478933988\\
	};
	\addplot [color=mycolor31, line width=\myLineWidth, forget plot]
	  table[row sep=crcr]{%
	0.187611924596253	0.103177478933988\\
	0.186074142533279	0.0514574030797424\\
	};
	\addplot [color=mycolor12!79!mycolor22, line width=\myLineWidth, forget plot]
	  table[row sep=crcr]{%
	0.186074142533279	0.0514574030797424\\
	0.185565218426251	1.18247675981982e-16\\
	};
	\addplot [color=mycolor26!39!mycolor31, line width=\myLineWidth, forget plot]
	  table[row sep=crcr]{%
	0.280974550231148	0\\
	0.281709523472291	-0.0490790289582721\\
	};
	\addplot [color=mycolor3!18!mycolor25, line width=\myLineWidth, forget plot]
	  table[row sep=crcr]{%
	0.281709523472291	-0.0490790289582721\\
	0.28392921451762	-0.0983710140907305\\
	};
	\addplot [color=mycolor8!63!mycolor23, line width=\myLineWidth, forget plot]
	  table[row sep=crcr]{%
	0.28392921451762	-0.0983710140907305\\
	0.287678538956823	-0.148091790410907\\
	};
	\addplot [color=mycolor6!65!mycolor17, line width=\myLineWidth, forget plot]
	  table[row sep=crcr]{%
	0.287678538956823	-0.148091790410907\\
	0.29303440116557	-0.198462981753638\\
	};
	\addplot [color=lightgray!91!mycolor21, line width=\myLineWidth, forget plot]
	  table[row sep=crcr]{%
	0.29303440116557	-0.198462981753638\\
	0.30010890433203	-0.249714966526287\\
	};
	\addplot [color=gray!40!mycolor7, line width=\myLineWidth, forget plot]
	  table[row sep=crcr]{%
	0.30010890433203	-0.249714966526287\\
	0.309054143020302	-0.302089908947028\\
	};
	\addplot [color=mycolor1!59!mycolor16, line width=\myLineWidth, forget plot]
	  table[row sep=crcr]{%
	0.309054143020302	-0.302089908947028\\
	0.320068910238794	-0.355844837944433\\
	};
	\addplot [color=mycolor11!9!mycolor13, line width=\myLineWidth, forget plot]
	  table[row sep=crcr]{%
	0.320068910238794	-0.355844837944433\\
	0.333407797857067	-0.411254707992318\\
	};
	\addplot [color=white!34!mycolor31, line width=\myLineWidth, forget plot]
	  table[row sep=crcr]{%
	0.333407797857067	-0.411254707992318\\
	0.349393363014183	-0.468615294623349\\
	};
	\addplot [color=mycolor5!67!mycolor15, line width=\myLineWidth, forget plot]
	  table[row sep=crcr]{%
	0.349393363014183	-0.468615294623349\\
	0.368432293956471	-0.528245639736354\\
	};
	\addplot [color=lightgray!94!mycolor6, line width=\myLineWidth, forget plot]
	  table[row sep=crcr]{%
	0.368432293956471	-0.528245639736354\\
	0.391036863366213	-0.590489530089817\\
	};
	\addplot [color=mycolor1!53!mycolor25, line width=\myLineWidth, forget plot]
	  table[row sep=crcr]{%
	0.391036863366213	-0.590489530089817\\
	0.417853440697106	-0.655715102031369\\
	};
	\addplot [color=mycolor1!6!mycolor10, line width=\myLineWidth, forget plot]
	  table[row sep=crcr]{%
	0.417853440697106	-0.655715102031369\\
	0.449700490265287	-0.724311006323745\\
	};
	\addplot [color=white!79!mycolor21, line width=\myLineWidth, forget plot]
	  table[row sep=crcr]{%
	0.449700490265287	-0.724311006323745\\
	0.487619354306354	-0.796676449522478\\
	};
	\addplot [color=mycolor1!42!mycolor11, line width=\myLineWidth, forget plot]
	  table[row sep=crcr]{%
	0.487619354306354	-0.796676449522478\\
	0.532942238256908	-0.873200526080474\\
	};
	\addplot [color=lightgray!57!mycolor34, line width=\myLineWidth, forget plot]
	  table[row sep=crcr]{%
	0.532942238256908	-0.873200526080474\\
	0.587383135187814	-0.954223004413353\\
	};
	\addplot [color=mycolor2!68!mycolor11, line width=\myLineWidth, forget plot]
	  table[row sep=crcr]{%
	0.587383135187814	-0.954223004413353\\
	0.653158689814251	-1.03996316339631\\
	};
	\addplot [color=mycolor2!85!mycolor13, line width=\myLineWidth, forget plot]
	  table[row sep=crcr]{%
	0.653158689814251	-1.03996316339631\\
	0.733146393344937	-1.13039376435309\\
	};
	\addplot [color=mycolor2!72!mycolor6, line width=\myLineWidth, forget plot]
	  table[row sep=crcr]{%
	0.733146393344937	-1.13039376435309\\
	0.831084831076911	-1.22502116039393\\
	};
	\addplot [color=mycolor1!77!mycolor33, line width=\myLineWidth, forget plot]
	  table[row sep=crcr]{%
	0.831084831076911	-1.22502116039393\\
	0.951809567375785	-1.32250605894978\\
	};
	\addplot [color=white!78!mycolor31, line width=\myLineWidth, forget plot]
	  table[row sep=crcr]{%
	0.951809567375785	-1.32250605894978\\
	1.10148601787855	-1.42001838300368\\
	};
	\addplot [color=mycolor26!22!mycolor30, line width=\myLineWidth, forget plot]
	  table[row sep=crcr]{%
	1.10148601787855	-1.42001838300368\\
	1.2877196203317	-1.51216460791291\\
	};
	\addplot [color=white!49!mycolor5, line width=\myLineWidth, forget plot]
	  table[row sep=crcr]{%
	1.2877196203317	-1.51216460791291\\
	1.51923890895449	-1.58928111861141\\
	};
	\addplot [color=mycolor3!4!mycolor34, line width=\myLineWidth, forget plot]
	  table[row sep=crcr]{%
	1.51923890895449	-1.58928111861141\\
	1.80447244858542	-1.63495620828158\\
	};
	\addplot [color=mycolor35, line width=\myLineWidth, forget plot]
	  table[row sep=crcr]{%
	1.80447244858542	-1.63495620828158\\
	2.14772825293788	-1.6231369833715\\
	};
	\addplot [color=white!33!mycolor29, line width=\myLineWidth, forget plot]
	  table[row sep=crcr]{%
	2.14772825293788	-1.6231369833715\\
	2.54118728688104	-1.5167618056467\\
	};
	\addplot [color=mycolor36, line width=\myLineWidth, forget plot]
	  table[row sep=crcr]{%
	2.54118728688104	-1.5167618056467\\
	2.9522806290906	-1.27312340228675\\
	};
	\addplot [color=white!62!mycolor28, line width=\myLineWidth, forget plot]
	  table[row sep=crcr]{%
	2.9522806290906	-1.27312340228675\\
	3.3129900162707	-0.863730855028229\\
	};
	\addplot [color=white!62!mycolor28, line width=\myLineWidth, forget plot]
	  table[row sep=crcr]{%
	3.3129900162707	0.863730855028227\\
	2.95228062909061	1.27312340228675\\
	};
	\addplot [color=mycolor36, line width=\myLineWidth, forget plot]
	  table[row sep=crcr]{%
	2.95228062909061	1.27312340228675\\
	2.54118728688104	1.51676180564671\\
	};
	\addplot [color=white!33!mycolor29, line width=\myLineWidth, forget plot]
	  table[row sep=crcr]{%
	2.54118728688104	1.51676180564671\\
	2.14772825293788	1.6231369833715\\
	};
	\addplot [color=mycolor35, line width=\myLineWidth, forget plot]
	  table[row sep=crcr]{%
	2.14772825293788	1.6231369833715\\
	1.80447244858542	1.63495620828158\\
	};
	\addplot [color=mycolor3!4!mycolor34, line width=\myLineWidth, forget plot]
	  table[row sep=crcr]{%
	1.80447244858542	1.63495620828158\\
	1.51923890895449	1.58928111861141\\
	};
	\addplot [color=white!49!mycolor5, line width=\myLineWidth, forget plot]
	  table[row sep=crcr]{%
	1.51923890895449	1.58928111861141\\
	1.2877196203317	1.51216460791291\\
	};
	\addplot [color=mycolor6!30!mycolor35, line width=\myLineWidth, forget plot]
	  table[row sep=crcr]{%
	1.2877196203317	1.51216460791291\\
	1.10148601787855	1.42001838300368\\
	};
	\addplot [color=white!78!mycolor31, line width=\myLineWidth, forget plot]
	  table[row sep=crcr]{%
	1.10148601787855	1.42001838300368\\
	0.951809567375786	1.32250605894978\\
	};
	\addplot [color=mycolor1!77!mycolor33, line width=\myLineWidth, forget plot]
	  table[row sep=crcr]{%
	0.951809567375786	1.32250605894978\\
	0.831084831076911	1.22502116039393\\
	};
	\addplot [color=mycolor2!72!mycolor6, line width=\myLineWidth, forget plot]
	  table[row sep=crcr]{%
	0.831084831076911	1.22502116039393\\
	0.733146393344938	1.13039376435309\\
	};
	\addplot [color=mycolor2!85!mycolor13, line width=\myLineWidth, forget plot]
	  table[row sep=crcr]{%
	0.733146393344938	1.13039376435309\\
	0.653158689814251	1.03996316339631\\
	};
	\addplot [color=mycolor2!68!mycolor11, line width=\myLineWidth, forget plot]
	  table[row sep=crcr]{%
	0.653158689814251	1.03996316339631\\
	0.587383135187815	0.954223004413354\\
	};
	\addplot [color=lightgray!57!mycolor34, line width=\myLineWidth, forget plot]
	  table[row sep=crcr]{%
	0.587383135187815	0.954223004413354\\
	0.532942238256908	0.873200526080474\\
	};
	\addplot [color=mycolor1!42!mycolor11, line width=\myLineWidth, forget plot]
	  table[row sep=crcr]{%
	0.532942238256908	0.873200526080474\\
	0.487619354306354	0.796676449522478\\
	};
	\addplot [color=white!79!mycolor21, line width=\myLineWidth, forget plot]
	  table[row sep=crcr]{%
	0.487619354306354	0.796676449522478\\
	0.449700490265287	0.724311006323745\\
	};
	\addplot [color=mycolor1!6!mycolor10, line width=\myLineWidth, forget plot]
	  table[row sep=crcr]{%
	0.449700490265287	0.724311006323745\\
	0.417853440697106	0.655715102031369\\
	};
	\addplot [color=mycolor1!53!mycolor25, line width=\myLineWidth, forget plot]
	  table[row sep=crcr]{%
	0.417853440697106	0.655715102031369\\
	0.391036863366213	0.590489530089817\\
	};
	\addplot [color=lightgray!94!mycolor6, line width=\myLineWidth, forget plot]
	  table[row sep=crcr]{%
	0.391036863366213	0.590489530089817\\
	0.368432293956471	0.528245639736355\\
	};
	\addplot [color=mycolor5!67!mycolor15, line width=\myLineWidth, forget plot]
	  table[row sep=crcr]{%
	0.368432293956471	0.528245639736355\\
	0.349393363014183	0.468615294623348\\
	};
	\addplot [color=white!34!mycolor31, line width=\myLineWidth, forget plot]
	  table[row sep=crcr]{%
	0.349393363014183	0.468615294623348\\
	0.333407797857067	0.411254707992318\\
	};
	\addplot [color=mycolor11!9!mycolor13, line width=\myLineWidth, forget plot]
	  table[row sep=crcr]{%
	0.333407797857067	0.411254707992318\\
	0.320068910238794	0.355844837944433\\
	};
	\addplot [color=mycolor1!59!mycolor16, line width=\myLineWidth, forget plot]
	  table[row sep=crcr]{%
	0.320068910238794	0.355844837944433\\
	0.309054143020302	0.302089908947028\\
	};
	\addplot [color=gray!40!mycolor7, line width=\myLineWidth, forget plot]
	  table[row sep=crcr]{%
	0.309054143020302	0.302089908947028\\
	0.30010890433203	0.249714966526288\\
	};
	\addplot [color=lightgray!91!mycolor21, line width=\myLineWidth, forget plot]
	  table[row sep=crcr]{%
	0.30010890433203	0.249714966526288\\
	0.29303440116557	0.198462981753638\\
	};
	\addplot [color=mycolor6!65!mycolor17, line width=\myLineWidth, forget plot]
	  table[row sep=crcr]{%
	0.29303440116557	0.198462981753638\\
	0.287678538956823	0.148091790410908\\
	};
	\addplot [color=mycolor8!63!mycolor23, line width=\myLineWidth, forget plot]
	  table[row sep=crcr]{%
	0.287678538956823	0.148091790410908\\
	0.28392921451762	0.098371014090731\\
	};
	\addplot [color=mycolor3!18!mycolor25, line width=\myLineWidth, forget plot]
	  table[row sep=crcr]{%
	0.28392921451762	0.098371014090731\\
	0.281709523472291	0.0490790289582722\\
	};
	\addplot [color=mycolor26!39!mycolor31, line width=\myLineWidth, forget plot]
	  table[row sep=crcr]{%
	0.281709523472291	0.0490790289582722\\
	0.280974550231148	1.1279649782883e-16\\
	};
	\addplot [color=gray!23!mycolor9, line width=\myLineWidth, forget plot]
	  table[row sep=crcr]{%
	0.386911998342113	0\\
	0.387846147957595	-0.0452997348859542\\
	};
	\addplot [color=gray!40!mycolor33, line width=\myLineWidth, forget plot]
	  table[row sep=crcr]{%
	0.387846147957595	-0.0452997348859542\\
	0.390665093957083	-0.0907409708450391\\
	};
	\addplot [color=white!23!mycolor14, line width=\myLineWidth, forget plot]
	  table[row sep=crcr]{%
	0.390665093957083	-0.0907409708450391\\
	0.395418875588168	-0.136465434251875\\
	};
	\addplot [color=mycolor1!31!mycolor32, line width=\myLineWidth, forget plot]
	  table[row sep=crcr]{%
	0.395418875588168	-0.136465434251875\\
	0.402192743056325	-0.182615181651937\\
	};
	\addplot [color=lightgray!32!mycolor12, line width=\myLineWidth, forget plot]
	  table[row sep=crcr]{%
	0.402192743056325	-0.182615181651937\\
	0.411110025014124	-0.229332446299459\\
	};
	\addplot [color=lightgray!81!mycolor12, line width=\myLineWidth, forget plot]
	  table[row sep=crcr]{%
	0.411110025014124	-0.229332446299459\\
	0.422336327346695	-0.276759050990018\\
	};
	\addplot [color=mycolor5!27!mycolor13, line width=\myLineWidth, forget plot]
	  table[row sep=crcr]{%
	0.422336327346695	-0.276759050990018\\
	0.436085255732584	-0.325035148857997\\
	};
	\addplot [color=white!69!mycolor18, line width=\myLineWidth, forget plot]
	  table[row sep=crcr]{%
	0.436085255732584	-0.325035148857997\\
	0.452625919677105	-0.374296956120695\\
	};
	\addplot [color=white!9!lightgray, line width=\myLineWidth, forget plot]
	  table[row sep=crcr]{%
	0.452625919677105	-0.374296956120695\\
	0.4722925451475	-0.424672993902014\\
	};
	\addplot [color=black!8!mycolor6, line width=\myLineWidth, forget plot]
	  table[row sep=crcr]{%
	0.4722925451475	-0.424672993902014\\
	0.495496591524024	-0.476278138543495\\
	};
	\addplot [color=mycolor12!74!mycolor35, line width=\myLineWidth, forget plot]
	  table[row sep=crcr]{%
	0.495496591524024	-0.476278138543495\\
	0.522741823708233	-0.529204459224159\\
	};
	\addplot [color=mycolor5!8!mycolor26, line width=\myLineWidth, forget plot]
	  table[row sep=crcr]{%
	0.522741823708233	-0.529204459224159\\
	0.554642804915747	-0.583507352115587\\
	};
	\addplot [color=mycolor3!22!mycolor9, line width=\myLineWidth, forget plot]
	  table[row sep=crcr]{%
	0.554642804915747	-0.583507352115587\\
	0.591947196122308	-0.6391847969824\\
	};
	\addplot [color=mycolor1!62!mycolor13, line width=\myLineWidth, forget plot]
	  table[row sep=crcr]{%
	0.591947196122308	-0.6391847969824\\
	0.635561971381746	-0.696146578574799\\
	};
	\addplot [color=mycolor2!35!mycolor8, line width=\myLineWidth, forget plot]
	  table[row sep=crcr]{%
	0.635561971381746	-0.696146578574799\\
	0.686582993011244	-0.754168925737055\\
	};
	\addplot [color=mycolor1!5!mycolor6, line width=\myLineWidth, forget plot]
	  table[row sep=crcr]{%
	0.686582993011244	-0.754168925737055\\
	0.746325990712442	-0.812828122081641\\
	};
	\addplot [color=mycolor2!82!mycolor12, line width=\myLineWidth, forget plot]
	  table[row sep=crcr]{%
	0.746325990712442	-0.812828122081641\\
	0.816354245602278	-0.871404198952546\\
	};
	\addplot [color=mycolor6!77!mycolor36, line width=\myLineWidth, forget plot]
	  table[row sep=crcr]{%
	0.816354245602278	-0.871404198952546\\
	0.898493167154036	-0.928743045641484\\
	};
	\addplot [color=mycolor1!18!mycolor3, line width=\myLineWidth, forget plot]
	  table[row sep=crcr]{%
	0.898493167154036	-0.928743045641484\\
	0.994812845240502	-0.983063031566396\\
	};
	\addplot [color=mycolor1!96!mycolor27, line width=\myLineWidth, forget plot]
	  table[row sep=crcr]{%
	0.994812845240502	-0.983063031566396\\
	1.1075442908846	-1.03169292573274\\
	};
	\addplot [color=white!21!mycolor3, line width=\myLineWidth, forget plot]
	  table[row sep=crcr]{%
	1.1075442908846	-1.03169292573274\\
	1.23887109962453	-1.07073708617698\\
	};
	\addplot [color=mycolor1!51!mycolor29, line width=\myLineWidth, forget plot]
	  table[row sep=crcr]{%
	1.23887109962453	-1.07073708617698\\
	1.39050572836216	-1.09469285648547\\
	};
	\addplot [color=mycolor3!47!mycolor36, line width=\myLineWidth, forget plot]
	  table[row sep=crcr]{%
	1.39050572836216	-1.09469285648547\\
	1.56292796822373	-1.09611307311087\\
	};
	\addplot [color=mycolor1!31!mycolor35, line width=\myLineWidth, forget plot]
	  table[row sep=crcr]{%
	1.56292796822373	-1.09611307311087\\
	1.75416668865825	-1.06553685118594\\
	};
	\addplot [color=mycolor25!1!mycolor34, line width=\myLineWidth, forget plot]
	  table[row sep=crcr]{%
	1.75416668865825	-1.06553685118594\\
	1.95812495472841	-0.992105804703278\\
	};
	\addplot [color=lightgray!2!mycolor35, line width=\myLineWidth, forget plot]
	  table[row sep=crcr]{%
	1.95812495472841	-0.992105804703278\\
	2.16280891734546	-0.865447259173016\\
	};
	\addplot [color=white!91!mycolor15, line width=\myLineWidth, forget plot]
	  table[row sep=crcr]{%
	2.16280891734546	-0.865447259173016\\
	2.3494803433664	-0.679244961552904\\
	};
	\addplot [color=mycolor4!11!mycolor36, line width=\myLineWidth, forget plot]
	  table[row sep=crcr]{%
	2.3494803433664	-0.679244961552904\\
	2.49437592693222	-0.435975374021999\\
	};
	\addplot [color=mycolor8!5!mycolor36, line width=\myLineWidth, forget plot]
	  table[row sep=crcr]{%
	2.49437592693222	-0.435975374021999\\
	2.57420551985345	-0.15054466289145\\
	};
	\addplot [color=mycolor28!16!mycolor36, line width=\myLineWidth, forget plot]
	  table[row sep=crcr]{%
	2.57420551985345	-0.15054466289145\\
	2.57420551985345	0.150544662891449\\
	};
	\addplot [color=mycolor8!5!mycolor36, line width=\myLineWidth, forget plot]
	  table[row sep=crcr]{%
	2.57420551985345	0.150544662891449\\
	2.49437592693222	0.435975374021998\\
	};
	\addplot [color=mycolor4!11!mycolor36, line width=\myLineWidth, forget plot]
	  table[row sep=crcr]{%
	2.49437592693222	0.435975374021998\\
	2.3494803433664	0.679244961552903\\
	};
	\addplot [color=white!91!mycolor15, line width=\myLineWidth, forget plot]
	  table[row sep=crcr]{%
	2.3494803433664	0.679244961552903\\
	2.16280891734546	0.865447259173017\\
	};
	\addplot [color=lightgray!2!mycolor35, line width=\myLineWidth, forget plot]
	  table[row sep=crcr]{%
	2.16280891734546	0.865447259173017\\
	1.95812495472841	0.992105804703279\\
	};
	\addplot [color=mycolor25!1!mycolor34, line width=\myLineWidth, forget plot]
	  table[row sep=crcr]{%
	1.95812495472841	0.992105804703279\\
	1.75416668865825	1.06553685118594\\
	};
	\addplot [color=mycolor1!31!mycolor35, line width=\myLineWidth, forget plot]
	  table[row sep=crcr]{%
	1.75416668865825	1.06553685118594\\
	1.56292796822373	1.09611307311087\\
	};
	\addplot [color=mycolor3!47!mycolor36, line width=\myLineWidth, forget plot]
	  table[row sep=crcr]{%
	1.56292796822373	1.09611307311087\\
	1.39050572836216	1.09469285648547\\
	};
	\addplot [color=mycolor1!51!mycolor29, line width=\myLineWidth, forget plot]
	  table[row sep=crcr]{%
	1.39050572836216	1.09469285648547\\
	1.23887109962453	1.07073708617698\\
	};
	\addplot [color=white!21!mycolor3, line width=\myLineWidth, forget plot]
	  table[row sep=crcr]{%
	1.23887109962453	1.07073708617698\\
	1.1075442908846	1.03169292573274\\
	};
	\addplot [color=mycolor1!96!mycolor27, line width=\myLineWidth, forget plot]
	  table[row sep=crcr]{%
	1.1075442908846	1.03169292573274\\
	0.994812845240502	0.983063031566396\\
	};
	\addplot [color=mycolor1!18!mycolor3, line width=\myLineWidth, forget plot]
	  table[row sep=crcr]{%
	0.994812845240502	0.983063031566396\\
	0.898493167154036	0.928743045641484\\
	};
	\addplot [color=mycolor6!77!mycolor36, line width=\myLineWidth, forget plot]
	  table[row sep=crcr]{%
	0.898493167154036	0.928743045641484\\
	0.816354245602278	0.871404198952546\\
	};
	\addplot [color=mycolor2!82!mycolor12, line width=\myLineWidth, forget plot]
	  table[row sep=crcr]{%
	0.816354245602278	0.871404198952546\\
	0.746325990712442	0.812828122081642\\
	};
	\addplot [color=mycolor1!5!mycolor6, line width=\myLineWidth, forget plot]
	  table[row sep=crcr]{%
	0.746325990712442	0.812828122081642\\
	0.686582993011244	0.754168925737055\\
	};
	\addplot [color=mycolor2!35!mycolor8, line width=\myLineWidth, forget plot]
	  table[row sep=crcr]{%
	0.686582993011244	0.754168925737055\\
	0.635561971381746	0.696146578574799\\
	};
	\addplot [color=mycolor1!62!mycolor13, line width=\myLineWidth, forget plot]
	  table[row sep=crcr]{%
	0.635561971381746	0.696146578574799\\
	0.591947196122308	0.6391847969824\\
	};
	\addplot [color=mycolor3!22!mycolor9, line width=\myLineWidth, forget plot]
	  table[row sep=crcr]{%
	0.591947196122308	0.6391847969824\\
	0.554642804915747	0.583507352115586\\
	};
	\addplot [color=mycolor5!8!mycolor26, line width=\myLineWidth, forget plot]
	  table[row sep=crcr]{%
	0.554642804915747	0.583507352115586\\
	0.522741823708233	0.52920445922416\\
	};
	\addplot [color=mycolor12!74!mycolor35, line width=\myLineWidth, forget plot]
	  table[row sep=crcr]{%
	0.522741823708233	0.52920445922416\\
	0.495496591524024	0.476278138543496\\
	};
	\addplot [color=black!8!mycolor6, line width=\myLineWidth, forget plot]
	  table[row sep=crcr]{%
	0.495496591524024	0.476278138543496\\
	0.4722925451475	0.424672993902014\\
	};
	\addplot [color=white!9!lightgray, line width=\myLineWidth, forget plot]
	  table[row sep=crcr]{%
	0.4722925451475	0.424672993902014\\
	0.452625919677105	0.374296956120696\\
	};
	\addplot [color=white!69!mycolor18, line width=\myLineWidth, forget plot]
	  table[row sep=crcr]{%
	0.452625919677105	0.374296956120696\\
	0.436085255732584	0.325035148857997\\
	};
	\addplot [color=mycolor5!27!mycolor13, line width=\myLineWidth, forget plot]
	  table[row sep=crcr]{%
	0.436085255732584	0.325035148857997\\
	0.422336327346695	0.276759050990018\\
	};
	\addplot [color=lightgray!81!mycolor12, line width=\myLineWidth, forget plot]
	  table[row sep=crcr]{%
	0.422336327346695	0.276759050990018\\
	0.411110025014124	0.22933244629946\\
	};
	\addplot [color=lightgray!32!mycolor12, line width=\myLineWidth, forget plot]
	  table[row sep=crcr]{%
	0.411110025014124	0.22933244629946\\
	0.402192743056325	0.182615181651937\\
	};
	\addplot [color=mycolor1!31!mycolor32, line width=\myLineWidth, forget plot]
	  table[row sep=crcr]{%
	0.402192743056325	0.182615181651937\\
	0.395418875588168	0.136465434251875\\
	};
	\addplot [color=white!23!mycolor14, line width=\myLineWidth, forget plot]
	  table[row sep=crcr]{%
	0.395418875588168	0.136465434251875\\
	0.390665093957083	0.0907409708450395\\
	};
	\addplot [color=gray!40!mycolor33, line width=\myLineWidth, forget plot]
	  table[row sep=crcr]{%
	0.390665093957083	0.0907409708450395\\
	0.387846147957595	0.0452997348859543\\
	};
	\addplot [color=gray!23!mycolor9, line width=\myLineWidth, forget plot]
	  table[row sep=crcr]{%
	0.387846147957595	0.0452997348859543\\
	0.386911998342113	1.04131607791609e-16\\
	};
	\addplot [color=mycolor4!25!mycolor12, line width=\myLineWidth, forget plot]
	  table[row sep=crcr]{%
	0.509275550512708	0\\
	0.510346222332991	-0.0394452728072118\\
	};
	\addplot [color=mycolor1!85!mycolor21, line width=\myLineWidth, forget plot]
	  table[row sep=crcr]{%
	0.510346222332991	-0.0394452728072118\\
	0.513573109109768	-0.0789396089109722\\
	};
	\addplot [color=mycolor1!60!mycolor31, line width=\myLineWidth, forget plot]
	  table[row sep=crcr]{%
	0.513573109109768	-0.0789396089109722\\
	0.519001132188971	-0.118529772733752\\
	};
	\addplot [color=mycolor2!57!mycolor25, line width=\myLineWidth, forget plot]
	  table[row sep=crcr]{%
	0.519001132188971	-0.118529772733752\\
	0.526706185969026	-0.158257736617554\\
	};
	\addplot [color=white!62!mycolor16, line width=\myLineWidth, forget plot]
	  table[row sep=crcr]{%
	0.526706185969026	-0.158257736617554\\
	0.536796677478764	-0.198157784166679\\
	};
	\addplot [color=mycolor2!41!mycolor13, line width=\myLineWidth, forget plot]
	  table[row sep=crcr]{%
	0.536796677478764	-0.198157784166679\\
	0.549415678546767	-0.238252970536351\\
	};
	\addplot [color=lightgray!33!mycolor8, line width=\myLineWidth, forget plot]
	  table[row sep=crcr]{%
	0.549415678546767	-0.238252970536351\\
	0.564743672185756	-0.278550652111729\\
	};
	\addplot [color=mycolor7!4!mycolor9, line width=\myLineWidth, forget plot]
	  table[row sep=crcr]{%
	0.564743672185756	-0.278550652111729\\
	0.583001844380122	-0.319036730141994\\
	};
	\addplot [color=mycolor4!83!mycolor23, line width=\myLineWidth, forget plot]
	  table[row sep=crcr]{%
	0.583001844380122	-0.319036730141994\\
	0.604455816291265	-0.359668162335171\\
	};
	\addplot [color=gray!11!mycolor6, line width=\myLineWidth, forget plot]
	  table[row sep=crcr]{%
	0.604455816291265	-0.359668162335171\\
	0.629419615383274	-0.400363181018107\\
	};
	\addplot [color=mycolor3!38!mycolor10, line width=\myLineWidth, forget plot]
	  table[row sep=crcr]{%
	0.629419615383274	-0.400363181018107\\
	0.658259525370461	-0.440988516445932\\
	};
	\addplot [color=lightgray!35!mycolor5, line width=\myLineWidth, forget plot]
	  table[row sep=crcr]{%
	0.658259525370461	-0.440988516445932\\
	0.691397202415363	-0.481342765294869\\
	};
	\addplot [color=mycolor2!30!mycolor8, line width=\myLineWidth, forget plot]
	  table[row sep=crcr]{%
	0.691397202415363	-0.481342765294869\\
	0.729311053602801	-0.521134885667644\\
	};
	\addplot [color=mycolor23!16!mycolor28, line width=\myLineWidth, forget plot]
	  table[row sep=crcr]{%
	0.729311053602801	-0.521134885667644\\
	0.772534282459661	-0.559956683089415\\
	};
	\addplot [color=mycolor7!80!mycolor36, line width=\myLineWidth, forget plot]
	  table[row sep=crcr]{%
	0.772534282459661	-0.559956683089415\\
	0.821647139124733	-0.597248161692779\\
	};
	\addplot [color=mycolor2!30!mycolor6, line width=\myLineWidth, forget plot]
	  table[row sep=crcr]{%
	0.821647139124733	-0.597248161692779\\
	0.877259688590739	-0.632254908643945\\
	};
	\addplot [color=mycolor1!90!mycolor25, line width=\myLineWidth, forget plot]
	  table[row sep=crcr]{%
	0.877259688590739	-0.632254908643945\\
	0.93997977473131	-0.663977531505164\\
	};
	\addplot [color=mycolor3!31!mycolor4, line width=\myLineWidth, forget plot]
	  table[row sep=crcr]{%
	0.93997977473131	-0.663977531505164\\
	1.01035885810697	-0.691115019387652\\
	};
	\addplot [color=white!26!mycolor6, line width=\myLineWidth, forget plot]
	  table[row sep=crcr]{%
	1.01035885810697	-0.691115019387652\\
	1.08880634543824	-0.712007396518159\\
	};
	\addplot [color=white!59!lightgray, line width=\myLineWidth, forget plot]
	  table[row sep=crcr]{%
	1.08880634543824	-0.712007396518159\\
	1.17546174173039	-0.72458898898381\\
	};
	\addplot [color=mycolor1!76!mycolor28, line width=\myLineWidth, forget plot]
	  table[row sep=crcr]{%
	1.17546174173039	-0.72458898898381\\
	1.27001521122515	-0.726372724126699\\
	};
	\addplot [color=lightgray!25!mycolor36, line width=\myLineWidth, forget plot]
	  table[row sep=crcr]{%
	1.27001521122515	-0.726372724126699\\
	1.37147404992014	-0.71449793517612\\
	};
	\addplot [color=mycolor9!24!mycolor30, line width=\myLineWidth, forget plot]
	  table[row sep=crcr]{%
	1.37147404992014	-0.71449793517612\\
	1.47788957685676	-0.685886512131378\\
	};
	\addplot [color=white!29!mycolor2, line width=\myLineWidth, forget plot]
	  table[row sep=crcr]{%
	1.47788957685676	-0.685886512131378\\
	1.58609019853358	-0.637557527500053\\
	};
	\addplot [color=mycolor6!18!mycolor34, line width=\myLineWidth, forget plot]
	  table[row sep=crcr]{%
	1.58609019853358	-0.637557527500053\\
	1.6915110842044	-0.567134827833767\\
	};
	\addplot [color=mycolor25!6!mycolor35, line width=\myLineWidth, forget plot]
	  table[row sep=crcr]{%
	1.6915110842044	-0.567134827833767\\
	1.78825494780381	-0.473528118006015\\
	};
	\addplot [color=mycolor5!6!mycolor29, line width=\myLineWidth, forget plot]
	  table[row sep=crcr]{%
	1.78825494780381	-0.473528118006015\\
	1.8695273345942	-0.357668184998173\\
	};
	\addplot [color=mycolor1!41!mycolor36, line width=\myLineWidth, forget plot]
	  table[row sep=crcr]{%
	1.8695273345942	-0.357668184998173\\
	1.92851625260293	-0.223057709044411\\
	};
	\addplot [color=white!73!mycolor10, line width=\myLineWidth, forget plot]
	  table[row sep=crcr]{%
	1.92851625260293	-0.223057709044411\\
	1.95960800356529	-0.0758377138480268\\
	};
	\addplot [color=white!41!mycolor1, line width=\myLineWidth, forget plot]
	  table[row sep=crcr]{%
	1.95960800356529	-0.0758377138480268\\
	1.95960800356529	0.0758377138480265\\
	};
	\addplot [color=white!73!mycolor10, line width=\myLineWidth, forget plot]
	  table[row sep=crcr]{%
	1.95960800356529	0.0758377138480265\\
	1.92851625260293	0.223057709044411\\
	};
	\addplot [color=mycolor1!41!mycolor36, line width=\myLineWidth, forget plot]
	  table[row sep=crcr]{%
	1.92851625260293	0.223057709044411\\
	1.8695273345942	0.357668184998173\\
	};
	\addplot [color=mycolor5!6!mycolor29, line width=\myLineWidth, forget plot]
	  table[row sep=crcr]{%
	1.8695273345942	0.357668184998173\\
	1.78825494780381	0.473528118006015\\
	};
	\addplot [color=mycolor25!6!mycolor35, line width=\myLineWidth, forget plot]
	  table[row sep=crcr]{%
	1.78825494780381	0.473528118006015\\
	1.6915110842044	0.567134827833767\\
	};
	\addplot [color=mycolor6!18!mycolor34, line width=\myLineWidth, forget plot]
	  table[row sep=crcr]{%
	1.6915110842044	0.567134827833767\\
	1.58609019853358	0.637557527500053\\
	};
	\addplot [color=white!29!mycolor2, line width=\myLineWidth, forget plot]
	  table[row sep=crcr]{%
	1.58609019853358	0.637557527500053\\
	1.47788957685676	0.685886512131378\\
	};
	\addplot [color=mycolor9!24!mycolor30, line width=\myLineWidth, forget plot]
	  table[row sep=crcr]{%
	1.47788957685676	0.685886512131378\\
	1.37147404992014	0.71449793517612\\
	};
	\addplot [color=lightgray!25!mycolor36, line width=\myLineWidth, forget plot]
	  table[row sep=crcr]{%
	1.37147404992014	0.71449793517612\\
	1.27001521122515	0.726372724126699\\
	};
	\addplot [color=mycolor1!76!mycolor28, line width=\myLineWidth, forget plot]
	  table[row sep=crcr]{%
	1.27001521122515	0.726372724126699\\
	1.17546174173039	0.72458898898381\\
	};
	\addplot [color=white!59!lightgray, line width=\myLineWidth, forget plot]
	  table[row sep=crcr]{%
	1.17546174173039	0.72458898898381\\
	1.08880634543824	0.71200739651816\\
	};
	\addplot [color=white!26!mycolor6, line width=\myLineWidth, forget plot]
	  table[row sep=crcr]{%
	1.08880634543824	0.71200739651816\\
	1.01035885810697	0.691115019387652\\
	};
	\addplot [color=mycolor3!31!mycolor4, line width=\myLineWidth, forget plot]
	  table[row sep=crcr]{%
	1.01035885810697	0.691115019387652\\
	0.93997977473131	0.663977531505164\\
	};
	\addplot [color=mycolor1!90!mycolor25, line width=\myLineWidth, forget plot]
	  table[row sep=crcr]{%
	0.93997977473131	0.663977531505164\\
	0.877259688590739	0.632254908643945\\
	};
	\addplot [color=mycolor2!30!mycolor6, line width=\myLineWidth, forget plot]
	  table[row sep=crcr]{%
	0.877259688590739	0.632254908643945\\
	0.821647139124733	0.597248161692779\\
	};
	\addplot [color=mycolor7!80!mycolor36, line width=\myLineWidth, forget plot]
	  table[row sep=crcr]{%
	0.821647139124733	0.597248161692779\\
	0.772534282459661	0.559956683089415\\
	};
	\addplot [color=mycolor23!16!mycolor28, line width=\myLineWidth, forget plot]
	  table[row sep=crcr]{%
	0.772534282459661	0.559956683089415\\
	0.729311053602801	0.521134885667644\\
	};
	\addplot [color=mycolor2!30!mycolor8, line width=\myLineWidth, forget plot]
	  table[row sep=crcr]{%
	0.729311053602801	0.521134885667644\\
	0.691397202415363	0.481342765294869\\
	};
	\addplot [color=lightgray!35!mycolor5, line width=\myLineWidth, forget plot]
	  table[row sep=crcr]{%
	0.691397202415363	0.481342765294869\\
	0.658259525370461	0.440988516445932\\
	};
	\addplot [color=mycolor3!38!mycolor10, line width=\myLineWidth, forget plot]
	  table[row sep=crcr]{%
	0.658259525370461	0.440988516445932\\
	0.629419615383274	0.400363181018107\\
	};
	\addplot [color=gray!11!mycolor6, line width=\myLineWidth, forget plot]
	  table[row sep=crcr]{%
	0.629419615383274	0.400363181018107\\
	0.604455816291266	0.359668162335171\\
	};
	\addplot [color=mycolor4!83!mycolor23, line width=\myLineWidth, forget plot]
	  table[row sep=crcr]{%
	0.604455816291266	0.359668162335171\\
	0.583001844380122	0.319036730141995\\
	};
	\addplot [color=mycolor7!4!mycolor9, line width=\myLineWidth, forget plot]
	  table[row sep=crcr]{%
	0.583001844380122	0.319036730141995\\
	0.564743672185756	0.278550652111729\\
	};
	\addplot [color=lightgray!33!mycolor8, line width=\myLineWidth, forget plot]
	  table[row sep=crcr]{%
	0.564743672185756	0.278550652111729\\
	0.549415678546767	0.238252970536351\\
	};
	\addplot [color=mycolor2!41!mycolor13, line width=\myLineWidth, forget plot]
	  table[row sep=crcr]{%
	0.549415678546767	0.238252970536351\\
	0.536796677478764	0.19815778416668\\
	};
	\addplot [color=white!62!mycolor16, line width=\myLineWidth, forget plot]
	  table[row sep=crcr]{%
	0.536796677478764	0.19815778416668\\
	0.526706185969026	0.158257736617554\\
	};
	\addplot [color=mycolor2!57!mycolor25, line width=\myLineWidth, forget plot]
	  table[row sep=crcr]{%
	0.526706185969026	0.158257736617554\\
	0.519001132188971	0.118529772733752\\
	};
	\addplot [color=mycolor1!60!mycolor31, line width=\myLineWidth, forget plot]
	  table[row sep=crcr]{%
	0.519001132188971	0.118529772733752\\
	0.513573109109768	0.0789396089109725\\
	};
	\addplot [color=mycolor1!85!mycolor21, line width=\myLineWidth, forget plot]
	  table[row sep=crcr]{%
	0.513573109109768	0.0789396089109725\\
	0.510346222332991	0.0394452728072119\\
	};
	\addplot [color=mycolor4!25!mycolor12, line width=\myLineWidth, forget plot]
	  table[row sep=crcr]{%
	0.510346222332991	0.0394452728072119\\
	0.509275550512708	9.07020462602019e-17\\
	};
	\addplot [color=white!34!mycolor13, line width=\myLineWidth, forget plot]
	  table[row sep=crcr]{%
	0.65514220421051	0\\
	0.656203163978542	-0.030384686280947\\
	};
	\addplot [color=mycolor1!64!mycolor14, line width=\myLineWidth, forget plot]
	  table[row sep=crcr]{%
	0.656203163978542	-0.030384686280947\\
	0.659394603602455	-0.0607189789973726\\
	};
	\addplot [color=lightgray!59!mycolor27, line width=\myLineWidth, forget plot]
	  table[row sep=crcr]{%
	0.659394603602455	-0.0607189789973726\\
	0.664742168943276	-0.0909491727502356\\
	};
	\addplot [color=mycolor3!19!mycolor9, line width=\myLineWidth, forget plot]
	  table[row sep=crcr]{%
	0.664742168943276	-0.0909491727502356\\
	0.672288471627916	-0.12101482363641\\
	};
	\addplot [color=darkgray!12!mycolor2, line width=\myLineWidth, forget plot]
	  table[row sep=crcr]{%
	0.672288471627916	-0.12101482363641\\
	0.682092839877382	-0.150845092530781\\
	};
	\addplot [color=white!10!mycolor26, line width=\myLineWidth, forget plot]
	  table[row sep=crcr]{%
	0.682092839877382	-0.150845092530781\\
	0.694230861107065	-0.180354743105115\\
	};
	\addplot [color=white!29!lightgray, line width=\myLineWidth, forget plot]
	  table[row sep=crcr]{%
	0.694230861107065	-0.180354743105115\\
	0.708793596810358	-0.209439681387999\\
	};
	\addplot [color=darkgray!18!mycolor35, line width=\myLineWidth, forget plot]
	  table[row sep=crcr]{%
	0.708793596810358	-0.209439681387999\\
	0.725886300302403	-0.23797193051973\\
	};
	\addplot [color=mycolor7!97!mycolor36, line width=\myLineWidth, forget plot]
	  table[row sep=crcr]{%
	0.725886300302403	-0.23797193051973\\
	0.745626406059903	-0.265793950441691\\
	};
	\addplot [color=mycolor4!80!mycolor32, line width=\myLineWidth, forget plot]
	  table[row sep=crcr]{%
	0.745626406059903	-0.265793950441691\\
	0.768140483522995	-0.292712244169846\\
	};
	\addplot [color=mycolor3!38!mycolor8, line width=\myLineWidth, forget plot]
	  table[row sep=crcr]{%
	0.768140483522995	-0.292712244169846\\
	0.793559757521714	-0.318490249521492\\
	};
	\addplot [color=lightgray!36!mycolor1, line width=\myLineWidth, forget plot]
	  table[row sep=crcr]{%
	0.793559757521714	-0.318490249521492\\
	0.822013693914676	-0.342840610807599\\
	};
	\addplot [color=mycolor1!52!mycolor26, line width=\myLineWidth, forget plot]
	  table[row sep=crcr]{%
	0.822013693914676	-0.342840610807599\\
	0.853621039699304	-0.36541707630159\\
	};
	\addplot [color=mycolor2!15!mycolor6, line width=\myLineWidth, forget plot]
	  table[row sep=crcr]{%
	0.853621039699304	-0.36541707630159\\
	0.888477607704473	-0.385806495298581\\
	};
	\addplot [color=mycolor4!76!mycolor8, line width=\myLineWidth, forget plot]
	  table[row sep=crcr]{%
	0.888477607704473	-0.385806495298581\\
	0.926640036907027	-0.403521716080245\\
	};
	\addplot [color=white!12!mycolor6, line width=\myLineWidth, forget plot]
	  table[row sep=crcr]{%
	0.926640036907027	-0.403521716080245\\
	0.968104790713856	-0.417996632176003\\
	};
	\addplot [color=mycolor3!25!mycolor5, line width=\myLineWidth, forget plot]
	  table[row sep=crcr]{%
	0.968104790713856	-0.417996632176003\\
	1.01278185433185	-0.428585193183672\\
	};
	\addplot [color=white!1!mycolor4, line width=\myLineWidth, forget plot]
	  table[row sep=crcr]{%
	1.01278185433185	-0.428585193183672\\
	1.06046306561466	-0.434566858015506\\
	};
	\addplot [color=mycolor1!51!mycolor5, line width=\myLineWidth, forget plot]
	  table[row sep=crcr]{%
	1.06046306561466	-0.434566858015506\\
	1.11078589141681	-0.435161629068957\\
	};
	\addplot [color=lightgray!2!mycolor2, line width=\myLineWidth, forget plot]
	  table[row sep=crcr]{%
	1.11078589141681	-0.435161629068957\\
	1.16319486815226	-0.429558271378094\\
	};
	\addplot [color=mycolor1!94!mycolor7, line width=\myLineWidth, forget plot]
	  table[row sep=crcr]{%
	1.16319486815226	-0.429558271378094\\
	1.21690491714179	-0.416959264558441\\
	};
	\addplot [color=mycolor1!89!mycolor27, line width=\myLineWidth, forget plot]
	  table[row sep=crcr]{%
	1.21690491714179	-0.416959264558441\\
	1.27087320183308	-0.396644998579434\\
	};
	\addplot [color=white!10!mycolor2, line width=\myLineWidth, forget plot]
	  table[row sep=crcr]{%
	1.27087320183308	-0.396644998579434\\
	1.32378867443969	-0.368057197179591\\
	};
	\addplot [color=mycolor1!83!mycolor34, line width=\myLineWidth, forget plot]
	  table[row sep=crcr]{%
	1.32378867443969	-0.368057197179591\\
	1.37409008124764	-0.330897189184223\\
	};
	\addplot [color=mycolor2!39!mycolor28, line width=\myLineWidth, forget plot]
	  table[row sep=crcr]{%
	1.37409008124764	-0.330897189184223\\
	1.42002265554279	-0.285228632979699\\
	};
	\addplot [color=mycolor2!62!mycolor34, line width=\myLineWidth, forget plot]
	  table[row sep=crcr]{%
	1.42002265554279	-0.285228632979699\\
	1.45973963334833	-0.231567770178997\\
	};
	\addplot [color=white!55!mycolor8, line width=\myLineWidth, forget plot]
	  table[row sep=crcr]{%
	1.45973963334833	-0.231567770178997\\
	1.49144634255544	-0.17093952856218\\
	};
	\addplot [color=mycolor6!35!mycolor35, line width=\myLineWidth, forget plot]
	  table[row sep=crcr]{%
	1.49144634255544	-0.17093952856218\\
	1.51357283856098	-0.104877816430176\\
	};
	\addplot [color=mycolor1!18!mycolor28, line width=\myLineWidth, forget plot]
	  table[row sep=crcr]{%
	1.51357283856098	-0.104877816430176\\
	1.52494905141608	-0.0353555615182006\\
	};
	\addplot [color=mycolor1!72!mycolor36, line width=\myLineWidth, forget plot]
	  table[row sep=crcr]{%
	1.52494905141608	-0.0353555615182006\\
	1.52494905141608	0.0353555615182004\\
	};
	\addplot [color=mycolor1!18!mycolor28, line width=\myLineWidth, forget plot]
	  table[row sep=crcr]{%
	1.52494905141608	0.0353555615182004\\
	1.51357283856098	0.104877816430176\\
	};
	\addplot [color=mycolor6!35!mycolor35, line width=\myLineWidth, forget plot]
	  table[row sep=crcr]{%
	1.51357283856098	0.104877816430176\\
	1.49144634255544	0.17093952856218\\
	};
	\addplot [color=white!55!mycolor8, line width=\myLineWidth, forget plot]
	  table[row sep=crcr]{%
	1.49144634255544	0.17093952856218\\
	1.45973963334833	0.231567770178997\\
	};
	\addplot [color=mycolor2!62!mycolor34, line width=\myLineWidth, forget plot]
	  table[row sep=crcr]{%
	1.45973963334833	0.231567770178997\\
	1.42002265554279	0.285228632979699\\
	};
	\addplot [color=mycolor2!39!mycolor28, line width=\myLineWidth, forget plot]
	  table[row sep=crcr]{%
	1.42002265554279	0.285228632979699\\
	1.37409008124764	0.330897189184223\\
	};
	\addplot [color=mycolor1!83!mycolor34, line width=\myLineWidth, forget plot]
	  table[row sep=crcr]{%
	1.37409008124764	0.330897189184223\\
	1.32378867443969	0.368057197179591\\
	};
	\addplot [color=white!10!mycolor2, line width=\myLineWidth, forget plot]
	  table[row sep=crcr]{%
	1.32378867443969	0.368057197179591\\
	1.27087320183308	0.396644998579434\\
	};
	\addplot [color=mycolor1!89!mycolor27, line width=\myLineWidth, forget plot]
	  table[row sep=crcr]{%
	1.27087320183308	0.396644998579434\\
	1.21690491714179	0.416959264558441\\
	};
	\addplot [color=mycolor1!94!mycolor7, line width=\myLineWidth, forget plot]
	  table[row sep=crcr]{%
	1.21690491714179	0.416959264558441\\
	1.16319486815226	0.429558271378094\\
	};
	\addplot [color=lightgray!2!mycolor2, line width=\myLineWidth, forget plot]
	  table[row sep=crcr]{%
	1.16319486815226	0.429558271378094\\
	1.11078589141681	0.435161629068957\\
	};
	\addplot [color=mycolor1!51!mycolor5, line width=\myLineWidth, forget plot]
	  table[row sep=crcr]{%
	1.11078589141681	0.435161629068957\\
	1.06046306561466	0.434566858015506\\
	};
	\addplot [color=white!1!mycolor4, line width=\myLineWidth, forget plot]
	  table[row sep=crcr]{%
	1.06046306561466	0.434566858015506\\
	1.01278185433185	0.428585193183672\\
	};
	\addplot [color=mycolor3!25!mycolor5, line width=\myLineWidth, forget plot]
	  table[row sep=crcr]{%
	1.01278185433185	0.428585193183672\\
	0.968104790713856	0.417996632176003\\
	};
	\addplot [color=white!12!mycolor6, line width=\myLineWidth, forget plot]
	  table[row sep=crcr]{%
	0.968104790713856	0.417996632176003\\
	0.926640036907027	0.403521716080245\\
	};
	\addplot [color=mycolor4!76!mycolor8, line width=\myLineWidth, forget plot]
	  table[row sep=crcr]{%
	0.926640036907027	0.403521716080245\\
	0.888477607704473	0.385806495298581\\
	};
	\addplot [color=mycolor2!15!mycolor6, line width=\myLineWidth, forget plot]
	  table[row sep=crcr]{%
	0.888477607704473	0.385806495298581\\
	0.853621039699304	0.36541707630159\\
	};
	\addplot [color=mycolor1!52!mycolor26, line width=\myLineWidth, forget plot]
	  table[row sep=crcr]{%
	0.853621039699304	0.36541707630159\\
	0.822013693914676	0.342840610807599\\
	};
	\addplot [color=lightgray!36!mycolor1, line width=\myLineWidth, forget plot]
	  table[row sep=crcr]{%
	0.822013693914676	0.342840610807599\\
	0.793559757521714	0.318490249521492\\
	};
	\addplot [color=mycolor3!38!mycolor8, line width=\myLineWidth, forget plot]
	  table[row sep=crcr]{%
	0.793559757521714	0.318490249521492\\
	0.768140483522996	0.292712244169846\\
	};
	\addplot [color=mycolor4!80!mycolor32, line width=\myLineWidth, forget plot]
	  table[row sep=crcr]{%
	0.768140483522996	0.292712244169846\\
	0.745626406059903	0.265793950441691\\
	};
	\addplot [color=mycolor7!97!mycolor36, line width=\myLineWidth, forget plot]
	  table[row sep=crcr]{%
	0.745626406059903	0.265793950441691\\
	0.725886300302404	0.237971930519731\\
	};
	\addplot [color=darkgray!18!mycolor35, line width=\myLineWidth, forget plot]
	  table[row sep=crcr]{%
	0.725886300302404	0.237971930519731\\
	0.708793596810358	0.209439681387999\\
	};
	\addplot [color=white!29!lightgray, line width=\myLineWidth, forget plot]
	  table[row sep=crcr]{%
	0.708793596810358	0.209439681387999\\
	0.694230861107065	0.180354743105115\\
	};
	\addplot [color=white!10!mycolor26, line width=\myLineWidth, forget plot]
	  table[row sep=crcr]{%
	0.694230861107065	0.180354743105115\\
	0.682092839877382	0.150845092530781\\
	};
	\addplot [color=darkgray!12!mycolor2, line width=\myLineWidth, forget plot]
	  table[row sep=crcr]{%
	0.682092839877382	0.150845092530781\\
	0.672288471627916	0.12101482363641\\
	};
	\addplot [color=mycolor3!19!mycolor9, line width=\myLineWidth, forget plot]
	  table[row sep=crcr]{%
	0.672288471627916	0.12101482363641\\
	0.664742168943276	0.0909491727502356\\
	};
	\addplot [color=lightgray!59!mycolor27, line width=\myLineWidth, forget plot]
	  table[row sep=crcr]{%
	0.664742168943276	0.0909491727502356\\
	0.659394603602455	0.0607189789973729\\
	};
	\addplot [color=mycolor1!64!mycolor14, line width=\myLineWidth, forget plot]
	  table[row sep=crcr]{%
	0.659394603602455	0.0607189789973729\\
	0.656203163978542	0.030384686280947\\
	};
	\addplot [color=white!34!mycolor13, line width=\myLineWidth, forget plot]
	  table[row sep=crcr]{%
	0.656203163978542	0.030384686280947\\
	0.65514220421051	6.990145449684e-17\\
	};
	\addplot [color=mycolor2!41!mycolor8, line width=\myLineWidth, forget plot]
	  table[row sep=crcr]{%
	0.832865838703439	0\\
	0.833589163240916	-0.0162948199999828\\
	};
	\addplot [color=mycolor4!60!mycolor9, line width=\myLineWidth, forget plot]
	  table[row sep=crcr]{%
	0.833589163240916	-0.0162948199999828\\
	0.835758449327874	-0.0324893421827109\\
	};
	\addplot [color=mycolor1!62!mycolor11, line width=\myLineWidth, forget plot]
	  table[row sep=crcr]{%
	0.835758449327874	-0.0324893421827109\\
	0.839371495777806	-0.0484820316821503\\
	};
	\addplot [color=mycolor1!49!mycolor9, line width=\myLineWidth, forget plot]
	  table[row sep=crcr]{%
	0.839371495777806	-0.0484820316821503\\
	0.844424169682899	-0.0641689058786721\\
	};
	\addplot [color=gray!21!mycolor29, line width=\myLineWidth, forget plot]
	  table[row sep=crcr]{%
	0.844424169682899	-0.0641689058786721\\
	0.85090970251317	-0.0794423790740377\\
	};
	\addplot [color=lightgray!27!mycolor3, line width=\myLineWidth, forget plot]
	  table[row sep=crcr]{%
	0.85090970251317	-0.0794423790740377\\
	0.858817690750914	-0.0941901969599248\\
	};
	\addplot [color=darkgray!13!mycolor29, line width=\myLineWidth, forget plot]
	  table[row sep=crcr]{%
	0.858817690750914	-0.0941901969599248\\
	0.868132789839666	-0.108294503320797\\
	};
	\addplot [color=gray!21!mycolor34, line width=\myLineWidth, forget plot]
	  table[row sep=crcr]{%
	0.868132789839666	-0.108294503320797\\
	0.8788330911078	-0.121631091968333\\
	};
	\addplot [color=mycolor2!42!mycolor7, line width=\myLineWidth, forget plot]
	  table[row sep=crcr]{%
	0.8788330911078	-0.121631091968333\\
	0.890888175689055	-0.134068909706153\\
	};
	\addplot [color=lightgray!38!mycolor28, line width=\myLineWidth, forget plot]
	  table[row sep=crcr]{%
	0.890888175689055	-0.134068909706153\\
	0.904256848479788	-0.145469890608332\\
	};
	\addplot [color=mycolor5!74!mycolor7, line width=\myLineWidth, forget plot]
	  table[row sep=crcr]{%
	0.904256848479788	-0.145469890608332\\
	0.918884570152455	-0.155689217079517\\
	};
	\addplot [color=mycolor2!76!mycolor11, line width=\myLineWidth, forget plot]
	  table[row sep=crcr]{%
	0.918884570152455	-0.155689217079517\\
	0.934700627524702	-0.164576117474395\\
	};
	\addplot [color=mycolor5!91!mycolor9, line width=\myLineWidth, forget plot]
	  table[row sep=crcr]{%
	0.934700627524702	-0.164576117474395\\
	0.95161511333487	-0.171975321153661\\
	};
	\addplot [color=darkgray!2!mycolor4, line width=\myLineWidth, forget plot]
	  table[row sep=crcr]{%
	0.95161511333487	-0.171975321153661\\
	0.969515826383648	-0.177729296504214\\
	};
	\addplot [color=mycolor3!86!mycolor11, line width=\myLineWidth, forget plot]
	  table[row sep=crcr]{%
	0.969515826383648	-0.177729296504214\\
	0.988265251808509	-0.181681391470501\\
	};
	\addplot [color=gray!2!mycolor4, line width=\myLineWidth, forget plot]
	  table[row sep=crcr]{%
	0.988265251808509	-0.181681391470501\\
	1.00769783716026	-0.183679974547418\\
	};
	\addplot [color=mycolor3!21!mycolor5, line width=\myLineWidth, forget plot]
	  table[row sep=crcr]{%
	1.00769783716026	-0.183679974547418\\
	1.0276178389264	-0.183583631631553\\
	};
	\addplot [color=mycolor3!36!mycolor5, line width=\myLineWidth, forget plot]
	  table[row sep=crcr]{%
	1.0276178389264	-0.183583631631553\\
	1.04779806931467	-0.181267405777326\\
	};
	\addplot [color=lightgray!15!mycolor27, line width=\myLineWidth, forget plot]
	  table[row sep=crcr]{%
	1.04779806931467	-0.181267405777326\\
	1.06797991436985	-0.176629969773394\\
	};
	\addplot [color=mycolor2!81!mycolor7, line width=\myLineWidth, forget plot]
	  table[row sep=crcr]{%
	1.06797991436985	-0.176629969773394\\
	1.08787500874436	-0.169601496226238\\
	};
	\addplot [color=mycolor2!28!mycolor4, line width=\myLineWidth, forget plot]
	  table[row sep=crcr]{%
	1.08787500874436	-0.169601496226238\\
	1.10716892475356	-0.160151842912602\\
	};
	\addplot [color=lightgray!10!mycolor1, line width=\myLineWidth, forget plot]
	  table[row sep=crcr]{%
	1.10716892475356	-0.160151842912602\\
	1.12552714942847	-0.148298516447165\\
	};
	\addplot [color=mycolor1!82!mycolor7, line width=\myLineWidth, forget plot]
	  table[row sep=crcr]{%
	1.12552714942847	-0.148298516447165\\
	1.14260347325484	-0.134113736977249\\
	};
	\addplot [color=mycolor1!31!mycolor33, line width=\myLineWidth, forget plot]
	  table[row sep=crcr]{%
	1.14260347325484	-0.134113736977249\\
	1.15805069740542	-0.117729829891246\\
	};
	\addplot [color=white!50!mycolor10, line width=\myLineWidth, forget plot]
	  table[row sep=crcr]{%
	1.15805069740542	-0.117729829891246\\
	1.17153329545319	-0.0993421497435265\\
	};
	\addplot [color=white!37!mycolor7, line width=\myLineWidth, forget plot]
	  table[row sep=crcr]{%
	1.17153329545319	-0.0993421497435265\\
	1.18274137012559	-0.0792088253167252\\
	};
	\addplot [color=mycolor2!96!mycolor6, line width=\myLineWidth, forget plot]
	  table[row sep=crcr]{%
	1.18274137012559	-0.0792088253167252\\
	1.19140497032454	-0.0576468188852122\\
	};
	\addplot [color=mycolor2!85!mycolor3, line width=\myLineWidth, forget plot]
	  table[row sep=crcr]{%
	1.19140497032454	-0.0576468188852122\\
	1.19730763230537	-0.0350241119799558\\
	};
	\addplot [color=lightgray!34!mycolor36, line width=\myLineWidth, forget plot]
	  table[row sep=crcr]{%
	1.19730763230537	-0.0350241119799558\\
	1.20029793346771	-0.0117482325367299\\
	};
	\addplot [color=lightgray!17!mycolor28, line width=\myLineWidth, forget plot]
	  table[row sep=crcr]{%
	1.20029793346771	-0.0117482325367299\\
	1.20029793346771	0.0117482325367298\\
	};
	\addplot [color=lightgray!34!mycolor36, line width=\myLineWidth, forget plot]
	  table[row sep=crcr]{%
	1.20029793346771	0.0117482325367298\\
	1.19730763230537	0.0350241119799557\\
	};
	\addplot [color=mycolor2!85!mycolor3, line width=\myLineWidth, forget plot]
	  table[row sep=crcr]{%
	1.19730763230537	0.0350241119799557\\
	1.19140497032454	0.0576468188852121\\
	};
	\addplot [color=mycolor2!96!mycolor6, line width=\myLineWidth, forget plot]
	  table[row sep=crcr]{%
	1.19140497032454	0.0576468188852121\\
	1.18274137012559	0.0792088253167253\\
	};
	\addplot [color=white!37!mycolor7, line width=\myLineWidth, forget plot]
	  table[row sep=crcr]{%
	1.18274137012559	0.0792088253167253\\
	1.17153329545319	0.0993421497435266\\
	};
	\addplot [color=white!50!mycolor10, line width=\myLineWidth, forget plot]
	  table[row sep=crcr]{%
	1.17153329545319	0.0993421497435266\\
	1.15805069740542	0.117729829891245\\
	};
	\addplot [color=mycolor1!31!mycolor33, line width=\myLineWidth, forget plot]
	  table[row sep=crcr]{%
	1.15805069740542	0.117729829891245\\
	1.14260347325484	0.134113736977249\\
	};
	\addplot [color=mycolor1!82!mycolor7, line width=\myLineWidth, forget plot]
	  table[row sep=crcr]{%
	1.14260347325484	0.134113736977249\\
	1.12552714942847	0.148298516447165\\
	};
	\addplot [color=lightgray!10!mycolor1, line width=\myLineWidth, forget plot]
	  table[row sep=crcr]{%
	1.12552714942847	0.148298516447165\\
	1.10716892475356	0.160151842912602\\
	};
	\addplot [color=mycolor2!28!mycolor4, line width=\myLineWidth, forget plot]
	  table[row sep=crcr]{%
	1.10716892475356	0.160151842912602\\
	1.08787500874436	0.169601496226238\\
	};
	\addplot [color=mycolor2!81!mycolor7, line width=\myLineWidth, forget plot]
	  table[row sep=crcr]{%
	1.08787500874436	0.169601496226238\\
	1.06797991436985	0.176629969773394\\
	};
	\addplot [color=lightgray!15!mycolor27, line width=\myLineWidth, forget plot]
	  table[row sep=crcr]{%
	1.06797991436985	0.176629969773394\\
	1.04779806931467	0.181267405777326\\
	};
	\addplot [color=mycolor3!36!mycolor5, line width=\myLineWidth, forget plot]
	  table[row sep=crcr]{%
	1.04779806931467	0.181267405777326\\
	1.0276178389264	0.183583631631553\\
	};
	\addplot [color=mycolor3!21!mycolor5, line width=\myLineWidth, forget plot]
	  table[row sep=crcr]{%
	1.0276178389264	0.183583631631553\\
	1.00769783716026	0.183679974547418\\
	};
	\addplot [color=gray!2!mycolor4, line width=\myLineWidth, forget plot]
	  table[row sep=crcr]{%
	1.00769783716026	0.183679974547418\\
	0.988265251808509	0.181681391470501\\
	};
	\addplot [color=mycolor3!86!mycolor11, line width=\myLineWidth, forget plot]
	  table[row sep=crcr]{%
	0.988265251808509	0.181681391470501\\
	0.969515826383648	0.177729296504214\\
	};
	\addplot [color=darkgray!2!mycolor4, line width=\myLineWidth, forget plot]
	  table[row sep=crcr]{%
	0.969515826383648	0.177729296504214\\
	0.95161511333487	0.171975321153661\\
	};
	\addplot [color=mycolor5!91!mycolor9, line width=\myLineWidth, forget plot]
	  table[row sep=crcr]{%
	0.95161511333487	0.171975321153661\\
	0.934700627524702	0.164576117474395\\
	};
	\addplot [color=mycolor2!76!mycolor11, line width=\myLineWidth, forget plot]
	  table[row sep=crcr]{%
	0.934700627524702	0.164576117474395\\
	0.918884570152455	0.155689217079517\\
	};
	\addplot [color=mycolor5!74!mycolor7, line width=\myLineWidth, forget plot]
	  table[row sep=crcr]{%
	0.918884570152455	0.155689217079517\\
	0.904256848479788	0.145469890608332\\
	};
	\addplot [color=lightgray!38!mycolor28, line width=\myLineWidth, forget plot]
	  table[row sep=crcr]{%
	0.904256848479788	0.145469890608332\\
	0.890888175689055	0.134068909706153\\
	};
	\addplot [color=mycolor2!42!mycolor7, line width=\myLineWidth, forget plot]
	  table[row sep=crcr]{%
	0.890888175689055	0.134068909706153\\
	0.8788330911078	0.121631091968333\\
	};
	\addplot [color=gray!21!mycolor34, line width=\myLineWidth, forget plot]
	  table[row sep=crcr]{%
	0.8788330911078	0.121631091968333\\
	0.868132789839666	0.108294503320797\\
	};
	\addplot [color=darkgray!13!mycolor29, line width=\myLineWidth, forget plot]
	  table[row sep=crcr]{%
	0.868132789839666	0.108294503320797\\
	0.858817690750914	0.0941901969599248\\
	};
	\addplot [color=lightgray!27!mycolor3, line width=\myLineWidth, forget plot]
	  table[row sep=crcr]{%
	0.858817690750914	0.0941901969599248\\
	0.85090970251317	0.0794423790740379\\
	};
	\addplot [color=gray!21!mycolor29, line width=\myLineWidth, forget plot]
	  table[row sep=crcr]{%
	0.85090970251317	0.0794423790740379\\
	0.844424169682899	0.0641689058786721\\
	};
	\addplot [color=mycolor1!49!mycolor9, line width=\myLineWidth, forget plot]
	  table[row sep=crcr]{%
	0.844424169682899	0.0641689058786721\\
	0.839371495777806	0.0484820316821503\\
	};
	\addplot [color=mycolor1!62!mycolor11, line width=\myLineWidth, forget plot]
	  table[row sep=crcr]{%
	0.839371495777806	0.0484820316821503\\
	0.835758449327874	0.032489342182711\\
	};
	\addplot [color=mycolor4!60!mycolor9, line width=\myLineWidth, forget plot]
	  table[row sep=crcr]{%
	0.835758449327874	0.032489342182711\\
	0.833589163240916	0.0162948199999829\\
	};
	\addplot [color=mycolor2!41!mycolor8, line width=\myLineWidth, forget plot]
	  table[row sep=crcr]
\label{fig:CapAchievingDistr_z}}
\caption{Capacity-achieving distribution of the relevant complex-valued quantities, plotted for $\SNR = 21\dB$.}
\label{fig:CapAchievingDistr}
\end{figure}

\subsection{Channel Capacity Statement}

For the moment we consider that $K$ circles are given in terms of their radii $\Radius_k$ and probabilities $\CircProb_k$ (their $\SNR$-dependent evolution is covered in the next subsection).
The resulting achievable information rate in bpcu is given by the mutual information expression
\begin{align}
\Rate =&
-\int_0^\infty\! a \cdot \gamma(a) \, \log_2\!\big( \gamma(a) \big)\,da - \log_2(e)
\label{eq:RateFromCircles} \, , \\
\gamma(a) :=&
\sum_{k = 1}^K \,
\CircProb_k \,\exp\!\left( -\f{a^2 + a_k^2}{2} \right)
I_0\!\left( a a_k \right) \, , \\
a_k :=& \f{\Radius_k}{\DiskRadius} \sqrt{2\, \SNR}
%= \sqrt{2} |Z\RxTx| \cdot \Radius_k / \sigma
\, .
\end{align}
Thereby $I_0$ is the modified Bessel function of the first kind and $e$ is the Euler number.
The integral is evaluated numerically. 
The expression \Cref{eq:RateFromCircles} was obtained by adapting the statements \cite[Eq.~(4),(13),(11),(46)]{ShamaiTIT1995}, which relate to the peak-power constrained quadrature AWGN channel, to our system model. More formal detail is given in \Cref{apdx:ar}.

As stated in \cite{ShamaiTIT1995}, the channel capacity $\RateMax$ is obtained by maximizing $\Rate$ with respect to the free circle parameters:
\begin{align}
\RateMax = &\max_{\Radius_2\, , \ldots , \,\Radius_K, \,\CircProb_1\, , \ldots, \,\CircProb_K} \Rate
\label{eq:Capacity} \\
&\mathrm{subject\ to}\
%\Radius_2 , \ldots , \Radius_K \in [0,\Radius_1) \, , \nonumber\\
0 \leq \Radius_K < \ldots < \Radius_2 < \Radius_1 \, , \nonumber\\
&\hphantom{\mathrm{subject\ to}}\
\CircProb_1 , \ldots , \CircProb_K \in [0,1] \, , \nonumber\\
&\hphantom{\mathrm{subject\ to}}\
\CircProb_1 + \ldots + \CircProb_K = 1 \, . \nonumber
\end{align}
In all following evaluations, this problem is tackled with an interior-point algorithm for constrained nonlinear optimization \cite{InteriorPointAlgorithm}, with sensible choices for the initial values.

%\ToDo{From \Cref{fig:CapAchievingDistr_i,fig:Rates_Capacity} we learned that achieving the largest possible data rate in the high-SNR regime requires a transmit current distribution with good coverage of both the disk boundary as well as the disk interior. Hence, adding load resistance to reach the disk interior is crucial.}

%\ToDo{At low SNR $C = I$ and $\gamma(s) = \exp( -s^2 - \SNR ) \, I_0( 2 s \sqrt{\SNR})$.}

\subsection{Optimal Number of Circles}
\label{sec:K}

We have yet to address the optimal number of circles $K \in \bbN$ for a given $\SNR$. The intervals where $K = 1,2,3$ are optimal are stated in \cite[Tab.~1]{ShamaiTIT1995} (please note that their $\SNR$ thresholds are $3\dB$ larger because they assumed an AWGN variance of $2$).
%
A very important fact is the optimality of $K = 1$ for $\SNR < 3.011$ or rather $\SNR < 4.8\dB$ (this threshold was originally determined by solving \cite[Eq.~(45)]{ShamaiTIT1995} numerically).
Beyond that, the optimal $K$ increases with $\SNR$ \cite{ShamaiTIT1995}.

\Cref{fig:Radii_PMF_evolution} shows how new circles emerge with increasing $\SNR$. For this numerical evaluation we iterated through a fine grid of increasing $\SNR$ values. For each $\SNR$ value, all $\Radius_k , \CircProb_k$ were optimized according to \Cref{eq:Capacity}, whereby their preceding values were used as initial values. We added a new smallest circle whenever this addition caused an appreciable rate increase. The associated numerical thresholds have a noticeable effect in the high-SNR regime, because there, parameter fine tuning of the innermost circles only causes rate changes near the floating point accuracy. In detail, we required that a new $K$-th circle must have probability $\CircProb_K \geq 0.003 \cdot \CircProb_{K-1}$ and must yield a rate increase larger than $100$ times the floating-point relative accuracy of Matlab ($2^{-52}$). %The optimized values then served as initial values for the next largest $\SNR$.
%In This evaluation  thresholds have a noticeable effect on the computed high-$\SNR$-thresholds for adding new circles. This is because, at high SNR, 

%We found that the rule of thumb $K \approx \lceil \sqrt{\SNR / 3} \,\rceil$ allows for a decent estimate. The circumferential probability densities $\f{\CircProb_k}{2\pi \Radius_k}$ are very similar for the inner circles but appreciably larger for the outmost circles (cf. \Cref{fig:CapAchievingDistr_i}).

\begin{figure}[t]
\centering
\subfloat[circle radii]{%
\resizebox{!}{55mm}{% This file was created by matlab2tikz.
%
%The latest updates can be retrieved from
%  http://www.mathworks.com/matlabcentral/fileexchange/22022-matlab2tikz-matlab2tikz
%where you can also make suggestions and rate matlab2tikz.
%
\begin{tikzpicture}

\begin{axis}[%
width=35mm,
height=50mm,
at={(0,0)},
scale only axis,
unbounded coords=jump,
xmin=3,
xmax=21,
xtick={3,6,9,12,15,18,21},
xlabel style={font=\color{white!15!black}},
xlabel={SNR [dB]},
ymin=0,
ymax=1,
ytick={0,.1,.2,.3,.4,.5,.6,.7,.8,.9,1},
yticklabels={0,.1,.2,.3,.4,.5,.6,.7,.8,.9,1},
ylabel={normalized radius $\Radius_k / \DiskRadius$},
axis background/.style={fill=white},
xmajorgrids,
ymajorgrids,
legend style={legend cell align=left, align=left, draw=white!15!black}
]
\addplot [color=red, line width=2.0pt, forget plot]
  table[row sep=crcr]{%
3	1\\
3.1	1\\
3.2	1\\
3.3	1\\
3.4	1\\
3.5	1\\
3.6	1\\
3.7	1\\
3.8	1\\
3.9	1\\
4	1\\
4.1	1\\
4.2	1\\
4.3	1\\
4.4	1\\
4.5	1\\
4.6	1\\
4.7	1\\
4.8	1\\
4.9	1\\
5	1\\
5.1	1\\
5.2	1\\
5.3	1\\
5.4	1\\
5.5	1\\
5.6	1\\
5.7	1\\
5.8	1\\
5.9	1\\
6	1\\
6.1	1\\
6.2	1\\
6.3	1\\
6.4	1\\
6.5	1\\
6.6	1\\
6.7	1\\
6.8	1\\
6.9	1\\
7	1\\
7.1	1\\
7.2	1\\
7.3	1\\
7.4	1\\
7.5	1\\
7.6	1\\
7.7	1\\
7.8	1\\
7.9	1\\
8	1\\
8.1	1\\
8.2	1\\
8.3	1\\
8.4	1\\
8.5	1\\
8.6	1\\
8.7	1\\
8.8	1\\
8.9	1\\
9	1\\
9.1	1\\
9.2	1\\
9.3	1\\
9.4	1\\
9.5	1\\
9.6	1\\
9.7	1\\
9.8	1\\
9.9	1\\
10	1\\
10.1	1\\
10.2	1\\
10.3	1\\
10.4	1\\
10.5	1\\
10.6	1\\
10.7	1\\
10.8	1\\
10.9	1\\
11	1\\
11.1	1\\
11.2	1\\
11.3	1\\
11.4	1\\
11.5	1\\
11.6	1\\
11.7	1\\
11.8	1\\
11.9	1\\
12	1\\
12.1	1\\
12.2	1\\
12.3	1\\
12.4	1\\
12.5	1\\
12.6	1\\
12.7	1\\
12.8	1\\
12.9	1\\
13	1\\
13.1	1\\
13.2	1\\
13.3	1\\
13.4	1\\
13.5	1\\
13.6	1\\
13.7	1\\
13.8	1\\
13.9	1\\
14	1\\
14.1	1\\
14.2	1\\
14.3	1\\
14.4	1\\
14.5	1\\
14.6	1\\
14.7	1\\
14.8	1\\
14.9	1\\
15	1\\
15.1	1\\
15.2	1\\
15.3	1\\
15.4	1\\
15.5	1\\
15.6	1\\
15.7	1\\
15.8	1\\
15.9	1\\
16	1\\
16.1	1\\
16.2	1\\
16.3	1\\
16.4	1\\
16.5	1\\
16.6	1\\
16.7	1\\
16.8	1\\
16.9	1\\
17	1\\
17.1	1\\
17.2	1\\
17.3	1\\
17.4	1\\
17.5	1\\
17.6	1\\
17.7	1\\
17.8	1\\
17.9	1\\
18	1\\
18.1	1\\
18.2	1\\
18.3	1\\
18.4	1\\
18.5	1\\
18.6	1\\
18.7	1\\
18.8	1\\
18.9	1\\
19	1\\
19.1	1\\
19.2	1\\
19.3	1\\
19.4	1\\
19.5	1\\
19.6	1\\
19.7	1\\
19.8	1\\
19.9	1\\
20	1\\
20.1	1\\
20.2	1\\
20.3	1\\
20.4	1\\
20.5	1\\
20.6	1\\
20.7	1\\
20.8	1\\
20.9	1\\
21	1\\
};
\addplot [color=red, line width=2.0pt, forget plot]
  table[row sep=crcr]{%
4.8	0\\
4.9	0.00108742990185637\\
5	0.000715168568966009\\
5.1	2.36227519116172e-05\\
5.2	0.000582740548106665\\
5.3	0.000117688654719785\\
5.4	4.354768841701e-05\\
5.5	4.07227550051317e-05\\
5.6	0.000215128478579955\\
5.7	0.000205327289402159\\
5.8	3.53446692264463e-05\\
5.9	1.70413348356601e-05\\
6	0.000186000331085904\\
6.1	0.000182007028711493\\
6.2	3.58439580568565e-05\\
6.3	0.00017729627136276\\
6.4	3.50100025048173e-05\\
6.5	3.51808085250818e-05\\
6.6	7.92683833954985e-05\\
6.7	1.60097160302767e-05\\
6.8	7.25429026856612e-06\\
6.9	0.000186572109911583\\
7	3.86710671470354e-05\\
7.1	0.000405164970284484\\
7.2	0.00017289048178051\\
7.3	0.000188699957254849\\
7.4	9.60216193916144e-05\\
7.5	0.000117226839503867\\
7.6	3.531288872765e-05\\
7.7	0.0549690854986438\\
7.8	0.114545838832552\\
7.9	0.150630870832004\\
8	0.178203319844855\\
8.1	0.200871965836622\\
8.2	0.220215747958138\\
8.3	0.237103894598139\\
8.4	0.252079096778632\\
8.5	0.265508708527985\\
8.6	0.277656964569554\\
8.7	0.288722112724108\\
8.8	0.298858260050282\\
8.9	0.308187718806489\\
9	0.316810524028532\\
9.1	0.324809428397061\\
9.2	0.332254017539524\\
9.3	0.339203628237407\\
9.4	0.345709702593299\\
9.5	0.351816420311785\\
9.6	0.357562834306152\\
9.7	0.362983405560455\\
9.8	0.368108770072515\\
9.9	0.372966294042916\\
10	0.377580600489562\\
10.1	0.381973780801602\\
10.2	0.38616601064915\\
10.3	0.390175668362402\\
10.4	0.394019425639315\\
10.5	0.397712682899409\\
10.6	0.410637958059668\\
10.7	0.420044198704291\\
10.8	0.428723911088846\\
10.9	0.436751279611317\\
11	0.444191708897355\\
11.1	0.451101142294171\\
11.2	0.457529081854976\\
11.3	0.463518842045414\\
11.4	0.469109030696563\\
11.5	0.474334165116755\\
11.6	0.479225181130584\\
11.7	0.483810000974801\\
11.8	0.48811357190989\\
11.9	0.492159206683057\\
12	0.497197964536963\\
12.1	0.504049373583994\\
12.2	0.510773887737557\\
12.3	0.517345176237807\\
12.4	0.52374111407988\\
12.5	0.529946477899216\\
12.6	0.535950706612465\\
12.7	0.54174663213039\\
12.8	0.547331583755147\\
12.9	0.552705709472591\\
13	0.557870968346021\\
13.1	0.562831745831625\\
13.2	0.567593591214104\\
13.3	0.572163079158298\\
13.4	0.576547512791333\\
13.5	0.580754819570623\\
13.6	0.584793191282241\\
13.7	0.591030588806071\\
13.8	0.596143632863425\\
13.9	0.601109555229906\\
14	0.605919570486281\\
14.1	0.610567736138176\\
14.2	0.615050869017847\\
14.3	0.619367301165629\\
14.4	0.623517677147674\\
14.5	0.627503227036324\\
14.6	0.631725093296361\\
14.7	0.63617676679067\\
14.8	0.640576299538078\\
14.9	0.644910538748201\\
15	0.649171387584829\\
15.1	0.653350424177298\\
15.2	0.657440435191642\\
15.3	0.661435420975396\\
15.4	0.665330660878231\\
15.5	0.669122943161098\\
15.6	0.672809366609471\\
15.7	0.67638899500079\\
15.8	0.679861177724221\\
15.9	0.684292289222839\\
16	0.688019932713109\\
16.1	0.691672894516253\\
16.2	0.695245359085815\\
16.3	0.698732374276714\\
16.4	0.702142921935423\\
16.5	0.705644406854649\\
16.6	0.709110374201047\\
16.7	0.712534929871395\\
16.8	0.715913099444947\\
16.9	0.719240572742295\\
17	0.722512588118436\\
17.1	0.725725453115251\\
17.2	0.728876220986828\\
17.3	0.73196123411213\\
17.4	0.73497818296098\\
17.5	0.737924498054913\\
17.6	0.741339317209257\\
17.7	0.744346320482694\\
17.8	0.747301127358671\\
17.9	0.75023240446243\\
18	0.753150450553281\\
18.1	0.756034190028871\\
18.2	0.758880755244656\\
18.3	0.761690133644809\\
18.4	0.764459185865054\\
18.5	0.767186051832777\\
18.6	0.769868066338849\\
18.7	0.772503304649581\\
18.8	0.775089524262916\\
18.9	0.777624908885243\\
19	0.780107716739216\\
19.1	0.782974128760185\\
19.2	0.785499301329301\\
19.3	0.78799565019596\\
19.4	0.790460871934054\\
19.5	0.792894884401883\\
19.6	0.795295858131837\\
19.7	0.797663250039088\\
19.8	0.799994909534099\\
19.9	0.802289743928268\\
20	0.804546312352308\\
20.1	0.806762699628905\\
20.2	0.808938040899097\\
20.3	0.811375308226665\\
20.4	0.813564872980925\\
20.5	0.815727223241016\\
20.6	0.817862171901808\\
20.7	0.819968594585984\\
20.8	0.822045534602101\\
20.9	0.824092547817722\\
21	0.826108041588946\\
};
\addplot [color=red, line width=2.0pt, forget plot]
  table[row sep=crcr]{%
10.5	0\\
10.6	0.000923073163269465\\
10.7	0.000618883144281569\\
10.8	0.0011574234067402\\
10.9	0.000460331138676463\\
11	0.000415420051600965\\
11.1	0.000174133344664228\\
11.2	0.00037075034240293\\
11.3	0.000160924834450618\\
11.4	0.000359438799885663\\
11.5	0.000359863360033692\\
11.6	0.000168896565955589\\
11.7	0.000414740764109663\\
11.8	0.000220174588586026\\
11.9	0.00075059575433045\\
12	0.0538996793827246\\
12.1	0.0992667643096241\\
12.2	0.126749236775295\\
12.3	0.147069302409689\\
12.4	0.163166574924379\\
12.5	0.176406416248957\\
12.6	0.187563631711046\\
12.7	0.197129296526612\\
12.8	0.205445301800387\\
12.9	0.212757586875457\\
13	0.219249737813364\\
13.1	0.225064315400113\\
13.2	0.230313482827112\\
13.3	0.2350870088107\\
13.4	0.239457628431525\\
13.5	0.243486046800225\\
13.6	0.247222170203961\\
13.7	0.267912435562323\\
13.8	0.278913467793153\\
13.9	0.288527843691166\\
14	0.296994504205396\\
14.1	0.30449988346072\\
14.2	0.311194079298154\\
14.3	0.317196639318917\\
14.4	0.322607017741237\\
14.5	0.327505211656219\\
14.6	0.335100640846293\\
14.7	0.344759463164376\\
14.8	0.354057629858766\\
14.9	0.362888951586412\\
15	0.371219954872121\\
15.1	0.379034552512755\\
15.2	0.386339433812921\\
15.3	0.393154458815079\\
15.4	0.399507854918631\\
15.5	0.405433262313741\\
15.6	0.410962413504222\\
15.7	0.416129818505247\\
15.8	0.420967457634276\\
15.9	0.43296688716648\\
16	0.439847021526236\\
16.1	0.446346307625195\\
16.2	0.452463297389828\\
16.3	0.458204825713507\\
16.4	0.463696162578998\\
16.5	0.470290399532957\\
16.6	0.476821120698258\\
16.7	0.483227716926286\\
16.8	0.489468440524674\\
16.9	0.495511285924633\\
17	0.501330953465789\\
17.1	0.506915006185896\\
17.2	0.512258713978786\\
17.3	0.517357241397997\\
17.4	0.522214661660522\\
17.5	0.526835323179665\\
17.6	0.535023152938954\\
17.7	0.540361616034112\\
17.8	0.545504879499927\\
17.9	0.550716165763387\\
18	0.556052032839144\\
18.1	0.5613152722222\\
18.2	0.566479778252446\\
18.3	0.571543065830871\\
18.4	0.576480853278836\\
18.5	0.581282365987822\\
18.6	0.585933942064311\\
18.7	0.59042947223228\\
18.8	0.594761400474814\\
18.9	0.598928624098183\\
19	0.602930138698486\\
19.1	0.609729379767579\\
19.2	0.61429037850039\\
19.3	0.618802051149629\\
19.4	0.623241996015554\\
19.5	0.62760907523295\\
19.6	0.631886246401354\\
19.7	0.636071311683925\\
19.8	0.640148970414413\\
19.9	0.644115169593847\\
20	0.647963521708323\\
20.1	0.651687448451124\\
20.2	0.65528650308097\\
20.3	0.660855830733939\\
20.4	0.664793906404949\\
20.5	0.668671225789928\\
20.6	0.672485138899848\\
20.7	0.676227101094162\\
20.8	0.679889497556996\\
20.9	0.683470246312589\\
21	0.686959071433329\\
};
\addplot [color=red, line width=2.0pt, forget plot]
  table[row sep=crcr]{%
13.6	0\\
13.7	0.00141558287062724\\
13.8	0.00120996023252675\\
13.9	0.00102498266178987\\
14	0.00083270438973438\\
14.1	0.000784001024922552\\
14.2	0.000750998125052325\\
14.3	0.000816669434928737\\
14.4	0.000948208405869488\\
14.5	0.000755359634345097\\
14.6	0.0583270010844054\\
14.7	0.0923512091252504\\
14.8	0.113082477532827\\
14.9	0.127826237921294\\
15	0.139063789184022\\
15.1	0.147979382767321\\
15.2	0.155262814507289\\
15.3	0.161351423337007\\
15.4	0.166540628065096\\
15.5	0.171039840704441\\
15.6	0.174996873911037\\
15.7	0.17852404419213\\
15.8	0.181706893911079\\
15.9	0.213139306244181\\
16	0.222258693849792\\
16.1	0.229908839061069\\
16.2	0.236419881317052\\
16.3	0.242026639815727\\
16.4	0.247474265513175\\
16.5	0.258990723419318\\
16.6	0.269977544340603\\
16.7	0.280119921526679\\
16.8	0.289336418063854\\
16.9	0.297640230540068\\
17	0.30509674034236\\
17.1	0.311803962908302\\
17.2	0.317861368677348\\
17.3	0.323347073831987\\
17.4	0.32834234735422\\
17.5	0.332911035399956\\
17.6	0.351301064655371\\
17.7	0.358268232053218\\
17.8	0.364627486960953\\
17.9	0.371766010191944\\
18	0.379736535571505\\
18.1	0.387476248320953\\
18.2	0.394864462457467\\
18.3	0.401907780329692\\
18.4	0.408534747872063\\
18.5	0.41474943680187\\
18.6	0.420546505003426\\
18.7	0.425951587131456\\
18.8	0.430980736230476\\
18.9	0.435668298297628\\
19	0.440041462931535\\
19.1	0.454718501755763\\
19.2	0.461202697507944\\
19.3	0.467609189656664\\
19.4	0.47382500458015\\
19.5	0.479854319071817\\
19.6	0.48562893648628\\
19.7	0.491160457673095\\
19.8	0.4964069737763\\
19.9	0.50138092963219\\
20	0.50608144933983\\
20.1	0.510512307432464\\
20.2	0.514692809529127\\
20.3	0.526336888574238\\
20.4	0.531834827491359\\
20.5	0.537189824690648\\
20.6	0.54239632867978\\
20.7	0.547426031564634\\
20.8	0.552256296599264\\
20.9	0.556892344554142\\
21	0.561311268548704\\
};
\addplot [color=red, line width=2.0pt, forget plot]
  table[row sep=crcr]{%
15.8	0\\
15.9	0.00182165931880792\\
16	0.0016117464566166\\
16.1	0.00168580488464605\\
16.2	0.00175932224739402\\
16.3	0.00106375676332797\\
16.4	0.0203561362958292\\
16.5	0.0703197376412677\\
16.6	0.0918304845713368\\
16.7	0.10547177141731\\
16.8	0.115094426808308\\
16.9	0.12230257958952\\
17	0.127942154736589\\
17.1	0.132519601509149\\
17.2	0.136349642161871\\
17.3	0.139625115158053\\
17.4	0.142488199693001\\
17.5	0.145032311153903\\
17.6	0.183285851659775\\
17.7	0.189741363978182\\
17.8	0.195111875857795\\
17.9	0.204739276680638\\
18	0.217347335767141\\
18.1	0.228478016173287\\
18.2	0.238004614558236\\
18.3	0.246280917216901\\
18.4	0.25339886718928\\
18.5	0.259599114222038\\
18.6	0.265024912308197\\
18.7	0.269836306012076\\
18.8	0.274130406182358\\
18.9	0.27801038473184\\
19	0.281546771398302\\
19.1	0.310822286773463\\
19.2	0.319516527791491\\
19.3	0.327945076895953\\
19.4	0.335739596793689\\
19.5	0.342991142006973\\
19.6	0.34957433715872\\
19.7	0.355615053141806\\
19.8	0.361083866399667\\
19.9	0.366077424590035\\
20	0.370641685509901\\
20.1	0.374821811530183\\
20.2	0.378679862088016\\
20.3	0.401777824987481\\
20.4	0.408672848644797\\
20.5	0.415189155579872\\
20.6	0.421349086403916\\
20.7	0.427113923912603\\
20.8	0.432459732984393\\
20.9	0.437444782581067\\
21	0.442052561655505\\
};
\addplot [color=red, line width=2.0pt, forget plot]
  table[row sep=crcr]{%
17.5	0\\
17.6	0.00394364207299766\\
17.7	0.00326997529041591\\
17.8	0.00582103735169237\\
17.9	0.0501719093991456\\
18	0.0757536807997611\\
18.1	0.0891765329803347\\
18.2	0.0976341209159847\\
18.3	0.103708933477076\\
18.4	0.108269794194932\\
18.5	0.111902307481826\\
18.6	0.114886420187694\\
18.7	0.117429176144777\\
18.8	0.119638382890491\\
18.9	0.121606345017569\\
19	0.123391092528801\\
19.1	0.175287502001684\\
19.2	0.187623860137304\\
19.3	0.198295834813794\\
19.4	0.206902639553589\\
19.5	0.214154984567676\\
19.6	0.220142378046592\\
19.7	0.22531264952175\\
19.8	0.229739346634419\\
19.9	0.233639372686721\\
20	0.237112576896494\\
20.1	0.240234173707743\\
20.2	0.243088274817109\\
20.3	0.284766591685173\\
20.4	0.292715699107343\\
20.5	0.299729266480319\\
20.6	0.306023117853983\\
20.7	0.31163473190385\\
20.8	0.316605611967998\\
20.9	0.321101490994617\\
21	0.325139070410695\\
};
\addplot [color=red, line width=2.0pt, forget plot]
  table[row sep=crcr]{%
19	0\\
19.1	0.0543986317309281\\
19.2	0.0717068254884314\\
19.3	0.0814882470258835\\
19.4	0.087546023255399\\
19.5	0.0919598091696579\\
19.6	0.0952420128881692\\
19.7	0.0979319354755496\\
19.8	0.100141350213267\\
19.9	0.102051482953941\\
20	0.103737202759292\\
20.1	0.105247539912693\\
20.2	0.106635459616399\\
20.3	0.174938674826805\\
20.4	0.182815781617275\\
20.5	0.188974602054531\\
20.6	0.194126050824037\\
20.7	0.198481965412902\\
20.8	0.202182978099286\\
20.9	0.205461226311456\\
21	0.2083553877801\\
};
\addplot [color=red, line width=2.0pt, forget plot]
  table[row sep=crcr]%
\put(-60,143){\scriptsize{$k \!=\! 1$}}%
\put(-77,78){\scriptsize{$k \!=\! 2$}}%
\put(-64,53){\scriptsize{$k \!=\! 3$}}%
\put(-44,41){\scriptsize{$...$}}%
\label{fig:Radii_evolution}%
}
\
\subfloat[circle probabilities]{%
\resizebox{!}{55mm}{% This file was created by matlab2tikz.
%
%The latest updates can be retrieved from
%  http://www.mathworks.com/matlabcentral/fileexchange/22022-matlab2tikz-matlab2tikz
%where you can also make suggestions and rate matlab2tikz.
%
\begin{tikzpicture}

\begin{axis}[%
width=35mm,
height=50mm,
at={(0,0)},
scale only axis,
unbounded coords=jump,
xmin=3,
xmax=21,
xtick={3,6,9,12,15,18,21},
xlabel={SNR [dB]},
ymin=0,
ymax=1,
ytick={0,.1,.2,.3,.4,.5,.6,.7,.8,.9,1},
yticklabels={0,.1,.2,.3,.4,.5,.6,.7,.8,.9,1},
ylabel={circle probability $\CircProb_k$},
axis background/.style={fill=white},
xmajorgrids,
ymajorgrids,
legend style={legend cell align=left, align=left, draw=white!15!black}
]
\addplot [color=red, line width=1.0pt, forget plot]
  table[row sep=crcr]{%
3	1\\
4.8	1\\
4.9	0.994085857509207\\
5	0.989132055634811\\
5.1	0.984349673716114\\
5.2	0.979733823547703\\
5.3	0.975280773834439\\
5.4	0.970985728233969\\
5.5	0.966844599269895\\
5.6	0.962853263592147\\
5.7	0.959007748738371\\
5.8	0.955304182701924\\
5.9	0.951738727643758\\
6	0.948307729635642\\
6.1	0.945007687487728\\
6.2	0.941835122288339\\
6.3	0.938786659618329\\
6.4	0.935859161704314\\
6.5	0.933049433612488\\
6.6	0.930354471303208\\
6.7	0.927771379502665\\
6.8	0.925297342155032\\
6.9	0.922929629944373\\
7	0.920665662617436\\
7.1	0.91850281781414\\
7.2	0.91643883939068\\
7.3	0.914471233046079\\
7.4	0.912597772524836\\
7.5	0.910816262559762\\
7.6	0.909124609263932\\
7.7	0.906752491209398\\
7.8	0.902491200110124\\
7.9	0.898189164574736\\
8	0.89385384865852\\
8.1	0.889492082872415\\
8.2	0.885110890425168\\
8.3	0.880716832198228\\
8.4	0.876316154368053\\
8.5	0.87191493091916\\
8.6	0.867518860391971\\
8.7	0.863133328576262\\
8.8	0.858763263862367\\
8.9	0.854413458125616\\
9	0.850088230750172\\
9.1	0.845791546616539\\
9.2	0.841527109090658\\
9.3	0.837298249684641\\
9.4	0.833107906357325\\
9.5	0.828958871002651\\
9.6	0.824853533830862\\
9.7	0.820794017897418\\
9.8	0.81678216669955\\
9.9	0.812819659538965\\
10	0.808907740661317\\
10.1	0.80504771520665\\
10.2	0.801240414495478\\
10.3	0.797486562240611\\
10.4	0.793786778958904\\
10.5	0.790141402873227\\
10.6	0.785083547777089\\
10.7	0.780431725120599\\
10.8	0.775799890349058\\
10.9	0.771194162702737\\
11	0.766620129434578\\
11.1	0.762083186911573\\
11.2	0.757588255092395\\
11.3	0.753140015631012\\
11.4	0.748742716464239\\
11.5	0.744400308172314\\
11.6	0.740116320768605\\
11.7	0.735893984869181\\
11.8	0.731736291194374\\
11.9	0.727645721171727\\
12	0.723415028398099\\
12.1	0.718863279504446\\
12.2	0.714300497513231\\
12.3	0.709731065446174\\
12.4	0.705159617052783\\
12.5	0.700590522416287\\
12.6	0.696028118625306\\
12.7	0.69147672924573\\
12.8	0.686940483767266\\
12.9	0.682423284022774\\
13	0.677928971526873\\
13.1	0.673461109205631\\
13.2	0.669023101291096\\
13.3	0.664618149958045\\
13.4	0.66024918449984\\
13.5	0.655919041247411\\
13.6	0.651630145792523\\
13.7	0.646923575522504\\
13.8	0.642408574502066\\
13.9	0.637907210270159\\
14	0.633423598295479\\
14.1	0.628961710257904\\
14.2	0.624525393651876\\
14.3	0.620118556185568\\
14.4	0.615744811910472\\
14.5	0.611407895111781\\
14.6	0.607034623722858\\
14.7	0.602616699040741\\
14.8	0.59820518485\\
14.9	0.593802940910421\\
15	0.589412132351276\\
15.1	0.58503524310378\\
15.2	0.580674769000019\\
15.3	0.576333368398723\\
15.4	0.572013552973276\\
15.5	0.567717812826103\\
15.6	0.563448817139363\\
15.7	0.559208876608805\\
15.8	0.555000404440569\\
15.9	0.550589819508059\\
16	0.546329285973088\\
16.1	0.542087716592573\\
16.2	0.537867496667975\\
16.3	0.533671094391717\\
16.4	0.529498393359138\\
16.5	0.525316049824043\\
16.6	0.521147470848312\\
16.7	0.516994088764049\\
16.8	0.512857266155298\\
16.9	0.508738273575719\\
17	0.504638821877251\\
17.1	0.500560375762574\\
17.2	0.496504398134553\\
17.3	0.492472675346512\\
17.4	0.488466835155232\\
17.5	0.48448869956222\\
17.6	0.480410047975778\\
17.7	0.476434299918982\\
17.8	0.472480617326786\\
17.9	0.468543770569587\\
18	0.464621291767987\\
18.1	0.460717574467965\\
18.2	0.456833543131586\\
18.3	0.452969469594352\\
18.4	0.4491263574374\\
18.5	0.445305008911254\\
18.6	0.441506514372574\\
18.7	0.437731998033998\\
18.8	0.433982462823441\\
18.9	0.43025908141316\\
19	0.426563134248308\\
19.1	0.422779909168014\\
19.2	0.419097079227011\\
19.3	0.415434637823051\\
19.4	0.411793559401216\\
19.5	0.408173895567415\\
19.6	0.404576218323947\\
19.7	0.401001008045177\\
19.8	0.39744910213737\\
19.9	0.393921028391603\\
20	0.390417582782651\\
20.1	0.386939755851191\\
20.2	0.383488089670497\\
20.3	0.379980175733036\\
20.4	0.376552834749719\\
20.5	0.373147540869353\\
20.6	0.369764492731447\\
20.7	0.366404208948985\\
20.8	0.363066959725669\\
20.9	0.359753214155292\\
21	0.3564634561648\\
};
\addplot [color=red, line width=1.0pt, forget plot]
  table[row sep=crcr]{%
4.8	0\\
4.9	0.00591414249079317\\
5	0.0108679443651893\\
5.1	0.015650326283886\\
5.2	0.0202661764522967\\
5.3	0.0247192261655612\\
5.4	0.0290142717660311\\
5.5	0.0331554007301052\\
5.6	0.0371467364078531\\
5.7	0.0409922512616291\\
5.8	0.0446958172980758\\
5.9	0.0482612723562416\\
6	0.0516922703643575\\
6.1	0.0549923125122725\\
6.2	0.0581648777116607\\
6.3	0.0612133403816713\\
6.4	0.0641408382956861\\
6.5	0.0669505663875125\\
6.6	0.0696455286967924\\
6.7	0.072228620497335\\
6.8	0.0747026578449677\\
6.9	0.0770703700556275\\
7	0.0793343373825637\\
7.1	0.0814971821858598\\
7.2	0.0835611606093196\\
7.3	0.0855287669539214\\
7.4	0.0874022274751639\\
7.5	0.0891837374402381\\
7.6	0.090875390736068\\
7.7	0.0932475087906018\\
7.8	0.0975087998898757\\
7.9	0.101810835425264\\
8	0.10614615134148\\
8.1	0.110507917127585\\
8.2	0.114889109574832\\
8.3	0.119283167801772\\
8.4	0.123683845631947\\
8.5	0.128085069080841\\
8.6	0.132481139608029\\
8.7	0.136866671423738\\
8.8	0.141236736137633\\
8.9	0.145586541874384\\
9	0.149911769249828\\
9.1	0.154208453383461\\
9.2	0.158472890909342\\
9.3	0.162701750315359\\
9.4	0.166892093642675\\
9.5	0.171041128997349\\
9.6	0.175146466169138\\
9.7	0.179205982102582\\
9.8	0.18321783330045\\
9.9	0.187180340461035\\
10	0.191092259338683\\
10.1	0.194952284793351\\
10.2	0.198759585504522\\
10.3	0.202513437759389\\
10.4	0.206213221041096\\
10.5	0.209858597126773\\
10.6	0.208690416140647\\
10.7	0.209544461383912\\
10.8	0.210856551619155\\
10.9	0.212534718564383\\
11	0.214505574451963\\
11.1	0.216711647852734\\
11.2	0.219106273719628\\
11.3	0.221651540920499\\
11.4	0.224315989078333\\
11.5	0.227073550563973\\
11.6	0.229902371388309\\
11.7	0.232783874248066\\
11.8	0.235702540464575\\
11.9	0.238644772350827\\
12	0.240521847319539\\
12.1	0.240667267720652\\
12.2	0.24081723076268\\
12.3	0.241012386705513\\
12.4	0.241285521083002\\
12.5	0.241658744323528\\
12.6	0.242146324451877\\
12.7	0.242756722844301\\
12.8	0.243491624752972\\
12.9	0.244348992552272\\
13	0.245324015525645\\
13.1	0.246409311493789\\
13.2	0.247596188581523\\
13.3	0.248875168439049\\
13.4	0.250236523182877\\
13.5	0.251670088662818\\
13.6	0.253166263881957\\
13.7	0.251632717121978\\
13.8	0.251494213508184\\
13.9	0.25145566516308\\
14	0.251529233788657\\
14.1	0.251721970145919\\
14.2	0.252036116562207\\
14.3	0.252471083646932\\
14.4	0.253023039842213\\
14.5	0.253686948842663\\
14.6	0.253817804930899\\
14.7	0.253451618782633\\
14.8	0.253051797868226\\
14.9	0.252644158956415\\
15	0.252245684632386\\
15.1	0.251873583835286\\
15.2	0.251542056497814\\
15.3	0.251262609829986\\
15.4	0.251043996209329\\
15.5	0.250891903723002\\
15.6	0.250810782314701\\
15.7	0.250801985942372\\
15.8	0.250865735797684\\
15.9	0.249007730599352\\
16	0.248376472979845\\
16.1	0.247781600085316\\
16.2	0.247235470398556\\
16.3	0.246748251041932\\
16.4	0.24629974382994\\
16.5	0.24553500587747\\
16.6	0.244733005457586\\
16.7	0.24390913472761\\
16.8	0.24307631394293\\
16.9	0.242246376773067\\
17	0.241431242553555\\
17.1	0.240640509802662\\
17.2	0.23988201717176\\
17.3	0.239164344642736\\
17.4	0.238493261856631\\
17.5	0.23787435395742\\
17.6	0.23607998506316\\
17.7	0.235126833912192\\
17.8	0.234196263535567\\
17.9	0.23321623938345\\
18	0.232164442071271\\
18.1	0.231096770404051\\
18.2	0.230022084208464\\
18.3	0.228942493776107\\
18.4	0.22786759203117\\
18.5	0.22680368398499\\
18.6	0.225758696453643\\
18.7	0.224738489172555\\
18.8	0.223750012332662\\
18.9	0.222798197026333\\
19	0.221888033612889\\
19.1	0.219900982224781\\
19.2	0.21870671600168\\
19.3	0.217500104931213\\
19.4	0.21628938738559\\
19.5	0.215075913001711\\
19.6	0.21386677037751\\
19.7	0.212664589446997\\
19.8	0.211476681555106\\
19.9	0.21030733011409\\
20	0.209161524125983\\
20.1	0.20804516579774\\
20.2	0.206961554756945\\
20.3	0.205056813894727\\
20.4	0.203776265185259\\
20.5	0.202495498706706\\
20.6	0.201216632555265\\
20.7	0.199943967686046\\
20.8	0.198681833866327\\
20.9	0.197432463263473\\
21	0.196202016246899\\
};
\addplot [color=red, line width=1.0pt, forget plot]
  table[row sep=crcr]{%
10.5	0\\
10.6	0.00622603608226431\\
10.7	0.0100238134954888\\
10.8	0.0133435580317879\\
10.9	0.0162711187328798\\
11	0.0188742961134586\\
11.1	0.0212051652356924\\
11.2	0.0233054711879769\\
11.3	0.0252084434484888\\
11.4	0.026941294457428\\
11.5	0.0285261412637135\\
11.6	0.0299813078430856\\
11.7	0.0313221408827529\\
11.8	0.0325611683410505\\
11.9	0.0337095064774462\\
12	0.0360631242823618\\
12.1	0.0404694527749023\\
12.2	0.0448822717240895\\
12.3	0.0492565478483135\\
12.4	0.0535548618642153\\
12.5	0.0577507332601845\\
12.6	0.0618255569228171\\
12.7	0.0657665479099684\\
12.8	0.0695678914797616\\
12.9	0.0732277234249543\\
13	0.076747012947482\\
13.1	0.0801295793005806\\
13.2	0.0833807101273814\\
13.3	0.0865066816029057\\
13.4	0.0895142923172835\\
13.5	0.0924108700897719\\
13.6	0.0952035903255206\\
13.7	0.0952439575642316\\
13.8	0.0974220975502719\\
13.9	0.0999712043377028\\
14	0.102737365321882\\
14.1	0.105619734735191\\
14.2	0.108550744605521\\
14.3	0.111484187847514\\
14.4	0.114389046124542\\
14.5	0.117244146142978\\
14.6	0.11879839774349\\
14.7	0.119442673914126\\
14.8	0.120223785050796\\
14.9	0.121216279036315\\
15	0.122437244704599\\
15.1	0.123879599611189\\
15.2	0.125518384862558\\
15.3	0.127321112684625\\
15.4	0.129253411374006\\
15.5	0.131282399880822\\
15.6	0.133379108041175\\
15.7	0.13551859344427\\
15.8	0.137679415118713\\
15.9	0.135521411218406\\
16	0.136449018702236\\
16.1	0.137567283141683\\
16.2	0.138858625544419\\
16.3	0.140300572184245\\
16.4	0.141774828079013\\
16.5	0.142196912773651\\
16.6	0.142576141867706\\
16.7	0.142975790639794\\
16.8	0.143437246433009\\
16.9	0.143990594447486\\
17	0.144654002977522\\
17.1	0.145431772648536\\
17.2	0.146320378537264\\
17.3	0.14731427231403\\
17.4	0.148401225450933\\
17.5	0.149568901765668\\
17.6	0.147347036837911\\
17.7	0.14774028114771\\
17.8	0.148237891369699\\
17.9	0.148526618362574\\
18	0.148560497201473\\
18.1	0.14857943977813\\
18.2	0.148617363500859\\
18.3	0.148680443897244\\
18.4	0.148798036172801\\
18.5	0.148981263118509\\
18.6	0.149243036363789\\
18.7	0.149586097455773\\
18.8	0.15001406611392\\
18.9	0.150522184628426\\
19	0.151105808997229\\
19.1	0.148369666539907\\
19.2	0.148150157354291\\
19.3	0.147902425878173\\
19.4	0.147660865211137\\
19.5	0.147428343145285\\
19.6	0.147231082756149\\
19.7	0.147071310744736\\
19.8	0.146970334267956\\
19.9	0.146930548313084\\
20	0.146958616468542\\
20.1	0.147058573503706\\
20.2	0.147227709351306\\
20.3	0.144602254102466\\
20.4	0.144164273289576\\
20.5	0.14373489252238\\
20.6	0.143318944676414\\
20.7	0.142930489791075\\
20.8	0.142581799013672\\
20.9	0.14227585750159\\
21	0.142026694907654\\
};
\addplot [color=red, line width=1.0pt, forget plot]
  table[row sep=crcr]{%
13.6	0\\
13.7	0.00619974979128671\\
13.8	0.00867511443947805\\
13.9	0.0106659202290578\\
14	0.0123098025939824\\
14.1	0.0136965848609857\\
14.2	0.0148877451803965\\
14.3	0.0159261723199854\\
14.4	0.0168431021227732\\
14.5	0.0176610099025792\\
14.6	0.0203491736027536\\
14.7	0.0244890082625005\\
14.8	0.0285192322309776\\
14.9	0.0323366210968487\\
15	0.0359049383117385\\
15.1	0.0392115734497452\\
15.2	0.0422647896396086\\
15.3	0.0450829090866661\\
15.4	0.0476890394433889\\
15.5	0.0501078835700741\\
15.6	0.0523612925047608\\
15.7	0.0544705440045532\\
15.8	0.0564544446430336\\
15.9	0.0581606773633244\\
16	0.0608542352139238\\
16.1	0.0635715886532671\\
16.2	0.0662289988019236\\
16.3	0.0687851440679686\\
16.4	0.0710884197887321\\
16.5	0.0718188747547243\\
16.6	0.0727403985243879\\
16.7	0.0739856994985615\\
16.8	0.0755441156625838\\
16.9	0.0773519880161659\\
17	0.0793296862977932\\
17.1	0.0814056835054632\\
17.2	0.0835243579863754\\
17.3	0.085645018948476\\
17.4	0.0877390597234794\\
17.5	0.0897874340683352\\
17.6	0.0877900062859327\\
17.7	0.089355435055888\\
17.8	0.0910580010105438\\
17.9	0.0921695366654812\\
18	0.0927597737718901\\
18.1	0.0934236821075096\\
18.2	0.0942344495475713\\
18.3	0.0951854394613619\\
18.4	0.0962964928799215\\
18.5	0.0975421993060033\\
18.6	0.0989020172990121\\
18.7	0.100343077537413\\
18.8	0.101840686185834\\
18.9	0.103367677143444\\
19	0.104903801001886\\
19.1	0.101336175275471\\
19.2	0.101732780762266\\
19.3	0.102111382107451\\
19.4	0.102563122013005\\
19.5	0.103082329401351\\
19.6	0.103714915316171\\
19.7	0.104442448617261\\
19.8	0.105280048304195\\
19.9	0.106205449225913\\
20	0.107206620142617\\
20.1	0.108267891531661\\
20.2	0.109369273444608\\
20.3	0.105295289099218\\
20.4	0.105455857136396\\
20.5	0.105672441094574\\
20.6	0.105947446439764\\
20.7	0.106299828381131\\
20.8	0.106744399747788\\
20.9	0.107267443258109\\
21	0.107876364559057\\
};
\addplot [color=red, line width=1.0pt, forget plot]
  table[row sep=crcr]{%
15.8	0\\
15.9	0.00672036131085871\\
16	0.0079909871309078\\
16.1	0.00899181152716034\\
16.2	0.00980940858712726\\
16.3	0.0104949383141375\\
16.4	0.0113386149431769\\
16.5	0.0151331567701118\\
16.6	0.0188029833020088\\
16.7	0.0221352863699852\\
16.8	0.0250850578061795\\
16.9	0.027672767187562\\
17	0.0299462462938787\\
17.1	0.0319616582807643\\
17.2	0.0337688481700477\\
17.3	0.035403688748245\\
17.4	0.0368996178137249\\
17.5	0.0382806106463571\\
17.6	0.0422091544087684\\
17.7	0.0445452489731583\\
17.8	0.046701879603102\\
17.9	0.0479378806754042\\
18	0.0487928355789137\\
18.1	0.0500990829091574\\
18.2	0.0517777407982094\\
18.3	0.0536654828751182\\
18.4	0.0556441510063992\\
18.5	0.0576271928472693\\
18.6	0.0595629941525393\\
18.7	0.0614250604001642\\
18.8	0.0631999545793128\\
18.9	0.0648862849126405\\
19	0.0664852617387241\\
19.1	0.0649901893626543\\
19.2	0.0657530520255664\\
19.3	0.0666168617097408\\
19.4	0.0676993304817311\\
19.5	0.0689411735432457\\
19.6	0.0703365809897503\\
19.7	0.0718124369391259\\
19.8	0.073339176049933\\
19.9	0.0748737420358908\\
20	0.0763909693362787\\
20.1	0.0778735911970273\\
20.2	0.0793095663414384\\
20.3	0.0751567299620149\\
20.4	0.0758376517693522\\
20.5	0.0766614793661144\\
20.6	0.0775959714166009\\
20.7	0.078633144969019\\
20.8	0.0797605175825132\\
20.9	0.0809364048260797\\
21	0.0821454585774737\\
};
\addplot [color=red, line width=1.0pt, forget plot]
  table[row sep=crcr]{%
17.5	0\\
17.6	0.00616376942844972\\
17.7	0.00679790099207047\\
17.8	0.00732534715430257\\
17.9	0.00960595434350261\\
18	0.0131011596084653\\
18.1	0.0160834503331861\\
18.2	0.0185148188133102\\
18.3	0.0205566703958169\\
18.4	0.0222673704723093\\
18.5	0.0237406518319748\\
18.6	0.025026741358443\\
18.7	0.0261752774000973\\
18.8	0.0272128179648298\\
18.9	0.028166574875996\\
19	0.0290539604009646\\
19.1	0.0344486275663355\\
19.2	0.0355497880771422\\
19.3	0.037089982653807\\
19.4	0.0388722906239355\\
19.5	0.0407188535277312\\
19.6	0.0425128886020971\\
19.7	0.0442211997311357\\
19.8	0.0458130816694544\\
19.9	0.0472972912937967\\
20	0.0486787746698649\\
20.1	0.049965875718693\\
20.2	0.0511727538411291\\
20.3	0.050154917419502\\
20.4	0.0513733659965498\\
20.5	0.0527648433279035\\
20.6	0.0542264049189918\\
20.7	0.0557052438703645\\
20.8	0.057162629112978\\
20.9	0.0585734506430372\\
21	0.0599212127523752\\
};
\addplot [color=red, line width=1.0pt, forget plot]
  table[row sep=crcr]{%
19	0\\
19.1	0.00817444986283799\\
19.2	0.011010426552044\\
19.3	0.0133446048965633\\
19.4	0.0151214448833854\\
19.5	0.0165794918132617\\
19.6	0.0177615436343759\\
19.7	0.0187870064755665\\
19.8	0.0196715760159861\\
19.9	0.020464610625622\\
20	0.0211859124740639\\
20.1	0.0218491463999816\\
20.2	0.0224710525940772\\
20.3	0.0287388534772191\\
20.4	0.0304522453122732\\
20.5	0.0320940429805066\\
20.6	0.0336266226770359\\
20.7	0.0350315785480674\\
20.8	0.0363038325450486\\
20.9	0.0374769424456324\\
21	0.0385502435236807\\
};
\addplot [color=red, line width=1.0pt, forget plot]
  table[row sep=crcr]%
\put(-48,118){\footnotesize{$\CircProb_1$}}%
\put(-64,53){\scriptsize{$\CircProb_2$}}%
\put(-48,39){\scriptsize{$\CircProb_3$}}%
\put(-37,32){\scriptsize{$...$}}%
\label{fig:PMF_evolution}%
}
\caption{Evolution of the radii and probabilities of the circles that describe the capacity-achieving transmit distribution.}
\label{fig:Radii_PMF_evolution}
\end{figure}

\begin{figure}[t]
\centering
%\includegraphics[width=.92\columnwidth]{Rates_CapacityResults-crop}
\resizebox{.96\columnwidth}{!}{\begin{tikzpicture}

\begin{axis}[%
width=90mm,
height=45mm,
at={(0,0)},
scale only axis,
xmin=-10,
xmax=24,
xtick={ -9, -6, -3, 0, 3, 6, 9, 12, 15, 18, 21, 24},
%xticklabels={-6, -3, 0, 3, 6, 9, 12, 15, 18, 21, 24, 27},
xlabel style={font=\color{white!15!black}},
xlabel={$\SNR$ $[\mathrm{dB}]$},
ymode=log,
ymin=0.1,
ymax=10,
yminorticks=true,
ylabel style={font=\color{white!15!black}},
ylabel={channel capacity [bpcu]},
axis background/.style={fill=white},
xmajorgrids,
ymajorgrids,
yminorgrids,
grid style={opacity=0.13},
minor grid style={opacity=0.3},
label style={font=\normalsize},
legend style={at={(1,0.085)}, anchor=south east, legend cell align=left, align=left, draw=white!15!black}
]
\addplot [color=blue, dashed, line width=0.5pt]
  table[row sep=crcr]{%
-10	0.137503523749935\\
-9.5	0.153418636061023\\
-9	0.171069138514398\\
-8.5	0.190619606352057\\
-8	0.212244743260395\\
-7.5	0.236128803353125\\
-7	0.262464707140878\\
-6.5	0.291452814285687\\
-6	0.323299322693842\\
-5.5	0.358214274110159\\
-5	0.396409161163114\\
-4.5	0.438094149757986\\
-4	0.48347495334568\\
-3.5	0.532749420867551\\
-3	0.586103926445348\\
-2.5	0.643709673944082\\
-2	0.705719050773513\\
-1.5	0.772262179969205\\
-1	0.843443825203652\\
-0.5	0.91934079807483\\
0	1\\
0.5	1.0854372028192\\
1	1.17563663469239\\
1.5	1.27055139420231\\
2	1.37010466975099\\
2.5	1.47419169766592\\
3	1.58268235491156\\
3.5	1.69542425407813\\
4	1.81224619130063\\
4.5	1.9329617924573\\
5	2.0573732086068\\
5.5	2.18527472629821\\
6	2.31645617962626\\
6.5	2.45070607596247\\
7	2.58781437356203\\
7.5	2.72757487451865\\
8	2.86978721917029\\
8.5	3.01425848700632\\
9	3.16080442391302\\
9.5	3.30925032620402\\
10	3.4594316186373\\
10.5	3.61119416705619\\
11	3.76439436704286\\
11.5	3.91889904869198\\
12	4.07458523490543\\
12.5	4.23133978700232\\
13	4.38905896736305\\
13.5	4.54764794461157\\
14	4.70702026272884\\
14.5	4.86709729164214\\
15	5.02780767335052\\
15.5	5.18908677457204\\
16	5.3508761542486\\
16.5	5.51312305200527\\
17	5.67577990180488\\
17.5	5.83880387352813\\
18	6.00215644400198\\
18.5	6.16580299805362\\
19	6.32971245944191\\
19.5	6.49385695097752\\
20	6.65821148275179\\
20.5	6.82275366712554\\
21	6.98746345895592\\
21.5	7.1523229194428\\
22	7.31731600193655\\
22.5	7.48242835805457\\
23	7.64764716249041\\
23.5	7.81296095495953\\
24	7.97835949780125\\
};
\addlegendentry{$\text{upper bound log}_\text{2}\text{(1 + SNR)}$}

\addplot [color=red, line width=1.3pt]
  table[row sep=crcr]{%
-10	0.137490906666417\\
-9.5	0.153399403990599\\
-9	0.171039935249776\\
-8.5	0.190575444753509\\
-8	0.212178257195243\\
-7.5	0.236029182897772\\
-7	0.262316197838123\\
-6.5	0.291232624383457\\
-6	0.322974738129779\\
-5.5	0.357738730230072\\
-5	0.395716964847275\\
-4.5	0.43709348985187\\
-4	0.48203878775833\\
-3.5	0.530703795313945\\
-3	0.583213275981814\\
-2.5	0.639658700930992\\
-2	0.700090880847482\\
-1.5	0.764512690550033\\
-1	0.832872335579065\\
-0.5	0.905057715083341\\
0	0.980892523998895\\
0.5	1.06013479000486\\
1	1.14247853280152\\
1.5	1.22755913836571\\
2	1.31496283468871\\
2.5	1.404240323024\\
3	1.49492416330459\\
3.5	1.58654896613884\\
4	1.67867287507681\\
4.5	1.77089833749213\\
5	1.86336708293004\\
5.5	1.95951129500193\\
6	2.05944392499972\\
6.5	2.16227922545915\\
7	2.26702644878706\\
7.5	2.37263814538457\\
8	2.47879831992885\\
8.5	2.58745529907544\\
9	2.69871645260338\\
9.5	2.81240690699709\\
10	2.92808134803176\\
10.5	3.0452964353854\\
11	3.16430965375948\\
11.5	3.2854181526281\\
12	3.40818887394725\\
12.5	3.53259192319178\\
13	3.65883219897967\\
13.5	3.78674553931811\\
14	3.91623260483693\\
14.5	4.04733161815581\\
15	4.17992003073123\\
15.5	4.31401399928841\\
16	4.44948247348643\\
16.5	4.5860705935919\\
17	4.72339744708351\\
17.5	4.86097776801742\\
18	4.99826548372716\\
18.5	5.13471323627153\\
19	5.26983776127836\\
19.5	5.40327367267533\\
20	5.53479284393184\\
20.5	5.66428043683202\\
21	5.79169176380879\\
21.5	5.91702209959332\\
22	6.04029376930273\\
22.5	6.16154963982862\\
23	6.28084773802992\\
23.5	6.39825660758257\\
24	6.51385156163464\\
};
%\addlegendentry{channel capacity for arbitrary passive load}
\addlegendentry{channel capacity, general passive load}

\addplot [color=black, dashdotted, line width=0.5pt]
  table[row sep=crcr]{%
-10	0.137490906666417\\
-9.5	0.153399403990599\\
-9	0.171039935249776\\
-8.5	0.190575444753509\\
-8	0.212178257195243\\
-7.5	0.236029182897772\\
-7	0.262316197838123\\
-6.5	0.291232624383457\\
-6	0.322974738129779\\
-5.5	0.357738730230072\\
-5	0.395716964847275\\
-4.5	0.43709348985187\\
-4	0.48203878775833\\
-3.5	0.530703795313945\\
-3	0.583213275981814\\
-2.5	0.639658700930992\\
-2	0.700090880847482\\
-1.5	0.764512690550033\\
-1	0.832872335579065\\
-0.5	0.905057715083341\\
0	0.980892523998895\\
0.5	1.06013479000486\\
1	1.14247853280152\\
1.5	1.22755913836571\\
2	1.31496283468871\\
2.5	1.404240323024\\
3	1.49492416330459\\
3.5	1.58654896613884\\
4	1.67867287507681\\
4.5	1.77089833749213\\
5	1.86288989247773\\
5.5	1.95438678032001\\
6	2.04520868673105\\
6.5	2.13525386861277\\
7	2.22449012886201\\
7.5	2.3129403489088\\
8	2.40066521289591\\
8.5	2.48774607413584\\
9	2.57427050396812\\
9.5	2.66032206006254\\
10	2.74597457231891\\
10.5	2.83129019850657\\
11	2.91631995217247\\
11.5	3.00110540225082\\
12	3.08568060483823\\
12.5	3.17007378252874\\
13	3.25430861502594\\
13.5	3.3384051836574\\
14	3.42238066188835\\
14.5	3.50624983219313\\
15	3.59002548434695\\
15.5	3.67371873004793\\
16	3.75733925624968\\
16.5	3.84089553217357\\
17	3.92439498047657\\
17.5	4.00784412016336\\
18	4.09124868688507\\
18.5	4.17461373491025\\
19	4.25794372408082\\
19.5	4.34124259434809\\
20	4.42451382995459\\
20.5	4.50776051491685\\
21	4.59098538115687\\
21.5	4.67419085038305\\
22	4.7573790706311\\
22.5	4.84055194822161\\
23	4.92371117576761\\
23.5	5.00685825676705\\
24	5.08999452723369\\
};
\addlegendentry{channel capacity, purely reactive load\\(equivalent: uniform-PSK modulation)}
\end{axis}

\end{tikzpicture}%}%
\caption{Channel capacity in bit per channel use (bpcu) plotted versus $\SNR$.} % ... could be attained by error-correcting codes ... sampled from ... with infinite block length
\label{fig:Rates_Capacity}
\end{figure}

\Cref{fig:Rates_Capacity} plots $\Rate$ versus $\SNR$ for different assumptions:
\begin{itemize}
\item \textbf{The capacity-achieving $K$ and $\Radius_k$, $\CircProb_k$ are used:} At high SNR, the additional circles inside the disk ensure that the transmit signal space is utilized thoroughly. %(cf. sphere packing).
This is achieved by purposefully raising the load resistance $\Re(z) \geq 0$.
\item \textbf{Only $K = 1$ is used:} This uniform-PSK $i \sim \calU(\calC_1)$ at the disk boundary is associated with a purely reactive load ($z = jx$), which is an interesting feature from a circuit perspective. The resulting information rate also constitutes the channel capacity of reactive load modulation. At high SNR, it lacks behind general passive load modulation, because the disk interior is not utilized.
\end{itemize}
For $\SNR < 4.8\dB$ the two cases coincide precisely, because then $K=1$ achieves capacity.
%the optimal number of circles $K$ is used and 2.) $K = 1$ is enforced, e.g. for technical reasons. 
%The second case uses just the largest circle with radius $\Radius_1$ (and $\CircProb_1 = 1$), hence no optimization must be conducted. In communications-engineering terms, this corresponds to $\infty$-PSK modulation. This case is attractive from a circuit perspective because it can be realized with a purely reactive load, i.e. $\Re(z) = 0$.

\subsection{Capacity-Achieving Load Impedance Statistics}
\label{sec:loadstats}

Herein we characterize the capacity-achieving distribution of the load impedance $z$, which follows from $z = \ChannelFunc^{-1}(i)$ together with the statistics of $i$ described in \Cref{sec:CurrentStats}.

The circle $\calC_1$ maps to the imaginary axis $j\bbR = \ChannelFunc^{-1}(\calC_1)$, i.e. to the set of purely reactive loads.
In detail, $i = \DiskRadius + \DiskRadius\, e^{j\phi}$ maps to $z = jx$ with
$x = -\tan(\phi / 2)$ and $\phi \sim \calU(-\pi,\pi)$. We find that $x$ has standard Cauchy distribution; the PDF is
% with location parameter $0$ and scale parameter $1$.
%In other words, the PDF is the simple Lorentzian function
\begin{align}
f_x(x) &= \f{1}{\pi(1 + x^2)} \, , &
x &\in \bbR \, . &
%z &= jx \, ,
\label{eq:Lorentz}
\end{align}
%which is shown in \Cref{fig:Lorentz}.
This is a complete description of the capacity-achieving load statistics in the low-SNR case (where $K = 1$ is optimal) or when a purely reactive load is enforced for technical reasons. The specific probability density evolution from \Cref{eq:Lorentz} can also seen along the imaginary axis in \Cref{fig:CapAchievingDistr_z}.

%\begin{figure}[!ht]
%\centering
%\resizebox{.7\columnwidth}{!}{% This file was created by matlab2tikz.
%
%The latest updates can be retrieved from
%  http://www.mathworks.com/matlabcentral/fileexchange/22022-matlab2tikz-matlab2tikz
%where you can also make suggestions and rate matlab2tikz.
%
\begin{tikzpicture}

\begin{axis}[%
width=70mm,
height=30mm,
at={(0,0)},
scale only axis,
xmin=-3,
xmax=3,
%xtick={1, 2, 3, 4, 5, 6, 7, 8},
xlabel style={font=\color{white!15!black}},
xlabel={normalized load reactance $x = \Im(z)$}, %  = \f{\Im(Z\L)}{R\Tx}
ymin=0,
ymax=0.35,
ylabel style={font=\color{white!15!black}},
ylabel={PDF $f_x(x)$},
axis background/.style={fill=white},
legend style={legend cell align=left, align=left, draw=white!15!black}
]
\addplot [color=black, line width=1.0pt, forget plot]
  table[row sep=crcr]{%
-3	0.0318309886183791\\
-2.96	0.032608372211911\\
-2.92	0.0334134495910093\\
-2.88	0.0342474916276242\\
-2.84	0.0351118388395463\\
-2.8	0.0360079056768994\\
-2.76	0.036937185084454\\
-2.72	0.0379012533558524\\
-2.68	0.03890177529622\\
-2.64	0.0399405097098713\\
-2.6	0.0410193152298699\\
-2.56	0.0421401565060091\\
-2.52	0.0433051107672767\\
-2.48	0.0445163747739694\\
-2.44	0.0457762721732327\\
-2.4	0.0470872612697915\\
-2.36	0.0484519432208644\\
-2.32	0.0498730706605338\\
-2.28	0.0513535567539673\\
-2.24	0.0528964846755834\\
-2.2	0.0545051174972244\\
-2.16	0.0561829084622618\\
-2.12	0.0579335116088728\\
-2.08	0.0597607926899577\\
-2.04	0.0616688403176904\\
-2	0.0636619772367581\\
-1.96	0.0657447716010804\\
-1.92	0.0679220480931612\\
-1.88	0.0701988986820286\\
-1.84	0.0725806927635422\\
-1.8	0.0750730863641016\\
-1.76	0.0776820300136155\\
-1.72	0.0804137748039083\\
-1.68	0.0832748760422223\\
-1.64	0.0862721937835513\\
-1.6	0.0894128893774693\\
-1.56	0.0927044169920173\\
-1.52	0.0961545088762055\\
-1.48	0.0997711528911079\\
-1.44	0.103562560575153\\
-1.4	0.10753712371074\\
-1.36	0.111703357026878\\
-1.32	0.116069824308558\\
-1.28	0.120645044793735\\
-1.24	0.125437376333461\\
-1.2	0.130454871386799\\
-1.16	0.135705101544931\\
-1.12	0.141194945965131\\
-1.08	0.146930338895767\\
-1.04	0.152915971456471\\
-1	0.159154943091895\\
-0.96	0.165648358755095\\
-0.92	0.172394869033682\\
-0.88	0.179390152267691\\
-0.84	0.186626340398564\\
-0.8	0.194091394014507\\
-0.76	0.201768436982626\\
-0.72	0.209635067297017\\
-0.68	0.217662668342308\\
-0.64	0.225815753535606\\
-0.6	0.234051386899846\\
-0.56	0.242318731869512\\
-0.52	0.250558789502354\\
-0.48	0.258704393842483\\
-0.44	0.266680534671406\\
-0.4	0.274405074296371\\
-0.36	0.28178991340633\\
-0.32	0.288742639861929\\
-0.28	0.295168663004257\\
-0.24	0.30097379555956\\
-0.2	0.306067198253645\\
-0.16	0.31036455361134\\
-0.12	0.313791291584967\\
-0.08	0.316285657972765\\
-0.04	0.317801403937491\\
0	0.318309886183791\\
0.04	0.317801403937491\\
0.08	0.316285657972765\\
0.12	0.313791291584967\\
0.16	0.31036455361134\\
0.2	0.306067198253645\\
0.24	0.30097379555956\\
0.28	0.295168663004257\\
0.32	0.288742639861929\\
0.36	0.28178991340633\\
0.4	0.274405074296371\\
0.44	0.266680534671406\\
0.48	0.258704393842483\\
0.52	0.250558789502354\\
0.56	0.242318731869512\\
0.6	0.234051386899846\\
0.64	0.225815753535606\\
0.68	0.217662668342308\\
0.72	0.209635067297017\\
0.76	0.201768436982626\\
0.8	0.194091394014507\\
0.84	0.186626340398564\\
0.88	0.179390152267691\\
0.92	0.172394869033682\\
0.96	0.165648358755095\\
1	0.159154943091895\\
1.04	0.152915971456471\\
1.08	0.146930338895767\\
1.12	0.141194945965131\\
1.16	0.135705101544931\\
1.2	0.130454871386799\\
1.24	0.125437376333461\\
1.28	0.120645044793735\\
1.32	0.116069824308558\\
1.36	0.111703357026878\\
1.4	0.10753712371074\\
1.44	0.103562560575153\\
1.48	0.0997711528911079\\
1.52	0.0961545088762055\\
1.56	0.0927044169920173\\
1.6	0.0894128893774693\\
1.64	0.0862721937835513\\
1.68	0.0832748760422223\\
1.72	0.0804137748039083\\
1.76	0.0776820300136155\\
1.8	0.0750730863641016\\
1.84	0.0725806927635422\\
1.88	0.0701988986820286\\
1.92	0.0679220480931612\\
1.96	0.0657447716010804\\
2	0.0636619772367581\\
2.04	0.0616688403176904\\
2.08	0.0597607926899577\\
2.12	0.0579335116088728\\
2.16	0.0561829084622618\\
2.2	0.0545051174972244\\
2.24	0.0528964846755834\\
2.28	0.0513535567539673\\
2.32	0.0498730706605338\\
2.36	0.0484519432208644\\
2.4	0.0470872612697915\\
2.44	0.0457762721732327\\
2.48	0.0445163747739694\\
2.52	0.0433051107672767\\
2.56	0.0421401565060091\\
2.6	0.0410193152298699\\
2.64	0.0399405097098713\\
2.68	0.03890177529622\\
2.72	0.0379012533558524\\
2.76	0.036937185084454\\
2.8	0.0360079056768994\\
2.84	0.0351118388395463\\
2.88	0.0342474916276242\\
2.92	0.0334134495910093\\
2.96	0.032608372211911\\
3	0.0318309886183791\\
};
\end{axis}
\end{tikzpicture}%}
%\caption{Capacity-achieving reactance statistics (standard Cauchy distribution) under a purely-reactive-load constraint $\Re(z) = 0$. At low SNR, this scheme achieves capacity even under the looser passive-load constraint $\Re(z) \geq 0$.}
%\label{fig:Lorentz}
%\end{figure}

The inner circles $\calC_k$ with $k \in \{2,\ldots,K\}$ map to circles $z \in \ChannelFunc^{-1}(\calC_k)$ with centers
$\f{\Radius_1^2 + \Radius_k^2}{\Radius_1^2 - \Radius_k^2} \in \bbR$
and radii
$\f{2 \Radius_1 \Radius_k}{\Radius_1^2 - \Radius_k^2}$.
In detail, a current phasor $i = \DiskRadius + \Radius_k\, e^{j\phi} \in \calC_k$ maps to
\begin{align}
z &= g^{-1}\left( i \right)
= \f{\Radius_1^2 + \Radius_k^2}{\Radius_1^2 - \Radius_k^2} + \f{2 \Radius_1 \Radius_k}{\Radius_1^2 - \Radius_k^2} \, e^{j(\theta + \pi)}
, \label{eq:z_from_s} \\
\theta &= 2 \arctan\left( \f{\sin\phi}{\Radius_k  / \Radius_1 + \cos\phi} \right) - \phi
\, . \label{eq:phi_from_phi}
\end{align}
%The relation between $\phi, \theta \in (-\pi,\pi]$ behaves as follows.
If $\Radius_k / \Radius_1 \ll 1$ then the approximate linearity $\theta \approx \phi$ holds.
If $\Radius_k / \Radius_1 \approx 1$ then $\theta$ is pushed towards zero.
Both properties can be observed in the high-SNR example in \Cref{fig:CapAchievingDistr_z}.
The distribution of $\theta|k$ is determined by \Cref{eq:phi_from_phi} and
$\phi \sim \calU(-\pi,\pi)$.%
%\footnote{To state the PDF of $\theta|k$, we first note that $\phi \mapsto \theta$ in \Cref{eq:phi_from_phi} is a monotonously increasing bijective map from and to $(-\pi,\pi]$.
%A change of variables yields
%$
%f_{\theta|k}(\theta|k)
%= f_{\phi}(\phi) \cdot |\fp{\phi}{\theta}|
%= \f{1}{2\pi} / \fp{\theta}{\phi}
%= \f{\Radius_k^2 + 2 \Radius_k \DiskRadius \cos(\phi) + \DiskRadius^2}{2\pi(\DiskRadius^2 - \Radius_k^2)}
%$
%after some rearrangements, an implicit formulation in terms of $\phi$. An explicit formulation is prohibited by the unavailability of the inverse map $\theta \mapsto \phi$ in closed form.}

%A change of variables and some rearrangements yield an implicit formulation of the PDF $f_\theta$ in terms of $\phi(\theta)$,
%\begin{align}
%f_{\theta}(\theta)
%= \f{\Radius_k^2 + 2 \DiskRadius \Radius_k \cos(\phi(\theta)) + \DiskRadius^2}{2\pi(\DiskRadius^2 - \Radius_k^2)} \, .
%\end{align}

% ... , i.e. $z$ is pushed towards $\f{\Radius_1 - \Radius_k}{\Radius_1 + \Radius_k}$.
% ... for most values of $\phi$ (unless $\phi$ is close to $\pm \pi$).
%Two interesting special points are
%$\phi =   0$ $\Longrightarrow$ $z = \f{\Radius_1 - \Radius_k}{\Radius_1 + \Radius_k}$ and
%$\phi = \pi$ $\Longrightarrow$ $z = \f{\Radius_1 + \Radius_k}{\Radius_1 - \Radius_k}$.

%\subsection{Sampling the Distribution}

To draw samples $Z\L$ from the capacity-achieving distribution, the following simple procedure suffices. Choose a circle $k$ according to the probabilities $\CircProb_k$ and draw a sample of the angle $\phi \sim \calU(-\pi,\pi)$. Compute $i = \DiskRadius + \Radius_k\, e^{j\phi}$, \mbox{$z = \ChannelFunc^{-1}(i)$}, and finally $Z\L = R\Tx \cdot z$. This way, a capacity-achieving codebook of load impedances $Z\L$ can be obtained.

\subsection{Upper Bound on the Capacity}

The effective constraint $|i - \DiskRadius|^2 \leq \DiskRadius^2$ of the peak-power type is obviously stricter than a constraint $\EV{|i - \DiskRadius|^2} \leq \DiskRadius^2$ of the average-power type. This inflicts the upper bound \cite{ShamaiTIT1995}
\begin{align}
\RateMax < \log_2(1 + \SNR).
\label{eq:UpperBound}
\end{align}
\Cref{fig:Rates_Capacity} shows that the bound is practically tight at low $\SNR$. Formally however, equality is ruled out by the following argument. By \cite[Appendix~B.4]{Tse2005}, equality would require a Gaussian $i \sim \mathcal{CN}(\DiskRadius, \DiskRadius^2)$ whose PDF support $\bbC \nsubseteq \Disk$ however violates the disk constraint \Cref{eq:Constraint}.

%To attain this bound, a Gaussian $i \sim \mathcal{CN}(\DiskRadius, \DiskRadius^2)$ would be necessary \cite[Appendix~B.4]{Tse2005}. However, because such $i$ could have any realization in $\bbC$, it would violate the disk condition \Cref{eq:Constraint} frequently. Therefore, equality is ruled out in \Cref{eq:UpperBound}. Still, \Cref{fig:Rates_Capacity} shows that the bound is practically tight at low $\SNR$.


\section{Near-Capacity Rates with Finite Alphabets}
\label{sec:practical}
Most every practical digital modulation uses a finite symbol alphabet instead of a continuous transmit distribution.
In our formalism this means that $i$ is chosen from an alphabet $i \in \SymbolAlphabet$, $\SymbolAlphabet = \{ s_1 , \ldots , s_M \} \subset \Disk$,
associated with $z \in \SymbolAlphabetZL$ from a load impedance alphabet $\SymbolAlphabetZL = \ChannelFunc^{-1}(\SymbolAlphabet) = \{ z_1 , \ldots , z_M \} \subset \HalfPlane$.
This caps the achievable rate at $\log_2(M)$ bpcu.
The Euclidean symbol distance $|s_m - s_n| \leq 2\DiskRadius$ is capped by the disk diameter.
This maximum is attained by a binary alphabet 
$\SymbolAlphabet = \{ 0, 2\DiskRadius \}$,
$\SymbolAlphabetZL = \{ \infty , 0\}$
but also by
$\SymbolAlphabet = \{ \DiskRadius(1+j) , \DiskRadius(1-j) \}$,
$\SymbolAlphabetZL = \{ -j , +j \}$
or any rotation of such $\SymbolAlphabet$ about $\DiskRadius$.

$M$-ary phase shift keying ($M$-PSK) is a more capable alphabet. It uses $M = 2^\ell$ symbols %($\ell \in \bbN$)
at $s_m = \DiskRadius(1+\exp(j2\pi\f{m - 1/2}{M}))$.
As mentioned earlier, this modulation is realized with a purely reactive load circuit.
\Cref{fig:Constellation_rate} shows the achievable rate of various PSK schemes. %with soft decoding (i.e. based on the unquantized observation $y$). 
The underlying numerical rate computation is described in \Cref{apdx:ar}.
At low SNR, $M$-PSK with $M \geq 4$ yields data rates very close to channel capacity while uniform-PSK ($M = \infty$) even achieves capacity.

\begin{figure}[!ht]
\centering
\vspace{-5mm}
\subfloat[symbol alphabet, TX current]{%
\resizebox{!}{60mm}{% This file was created by matlab2tikz.
%
%The latest updates can be retrieved from
%  http://www.mathworks.com/matlabcentral/fileexchange/22022-matlab2tikz-matlab2tikz
%where you can also make suggestions and rate matlab2tikz.
%
\definecolor{magenta}{rgb}{1.00000,0.00000,1.00000}%
%
\begin{tikzpicture}[>=latex']

\begin{axis}[%
width=1.6185in,
height=2.0601in,
at={(0,0)},
scale only axis,
xmin=-0.1,
xmax=2.1,
xlabel style={font=\color{white!15!black},at={(0.48,-0.065)},anchor=north},
xlabel={$\Re(i) \, / \, \DiskRadius$},
ymin=-1.4,
ymax=1.4,
ylabel style={font=\color{white!15!black},at={(-0.065,0.585)}},
ylabel={$\Im(i) \, / \, \DiskRadius$},
xtick={0,1,2},
ytick={-1,0,1},
axis background/.style={fill=white},
xmajorgrids,
ymajorgrids,
legend style={legend cell align=left, align=left, draw=white!15!black}
]
\addplot [color=white!85!black, forget plot]
  table[row sep=crcr]{%
2	0\\
1.9993908270191	0.034899496702501\\
1.99756405025982	0.0697564737441253\\
1.99452189536827	0.104528463267653\\
1.99026806874157	0.139173100960065\\
1.98480775301221	0.17364817766693\\
1.97814760073381	0.207911690817759\\
1.970295726276	0.241921895599668\\
1.96126169593832	0.275637355816999\\
1.95105651629515	0.309016994374947\\
1.93969262078591	0.342020143325669\\
1.92718385456679	0.374606593415912\\
1.9135454576426	0.4067366430758\\
1.89879404629917	0.438371146789077\\
1.88294759285893	0.469471562785891\\
1.86602540378444	0.5\\
1.84804809615643	0.529919264233205\\
1.82903757255504	0.559192903470747\\
1.80901699437495	0.587785252292473\\
1.78801075360672	0.615661475325658\\
1.76604444311898	0.642787609686539\\
1.74314482547739	0.669130606358858\\
1.71933980033865	0.694658370458997\\
1.694658370459	0.719339800338651\\
1.66913060635886	0.743144825477394\\
1.64278760968654	0.766044443118978\\
1.61566147532566	0.788010753606722\\
1.58778525229247	0.809016994374947\\
1.55919290347075	0.829037572555042\\
1.5299192642332	0.848048096156426\\
1.5	0.866025403784439\\
1.46947156278589	0.882947592858927\\
1.43837114678908	0.898794046299167\\
1.4067366430758	0.913545457642601\\
1.37460659341591	0.927183854566787\\
1.34202014332567	0.939692620785908\\
1.30901699437495	0.951056516295154\\
1.275637355817	0.961261695938319\\
1.24192189559967	0.970295726275996\\
1.20791169081776	0.978147600733806\\
1.17364817766693	0.984807753012208\\
1.13917310096007	0.99026806874157\\
1.10452846326765	0.994521895368273\\
1.06975647374413	0.997564050259824\\
1.0348994967025	0.999390827019096\\
1	1\\
0.965100503297499	0.999390827019096\\
0.930243526255875	0.997564050259824\\
0.895471536732347	0.994521895368273\\
0.860826899039935	0.99026806874157\\
0.82635182233307	0.984807753012208\\
0.792088309182241	0.978147600733806\\
0.758078104400332	0.970295726275996\\
0.724362644183001	0.961261695938319\\
0.690983005625053	0.951056516295154\\
0.657979856674331	0.939692620785908\\
0.625393406584088	0.927183854566787\\
0.5932633569242	0.913545457642601\\
0.561628853210922	0.898794046299167\\
0.530528437214109	0.882947592858927\\
0.5	0.866025403784439\\
0.470080735766795	0.848048096156426\\
0.440807096529253	0.829037572555042\\
0.412214747707527	0.809016994374947\\
0.384338524674342	0.788010753606722\\
0.357212390313461	0.766044443118978\\
0.330869393641142	0.743144825477394\\
0.305341629541003	0.719339800338651\\
0.280660199661349	0.694658370458997\\
0.256855174522606	0.669130606358858\\
0.233955556881022	0.642787609686539\\
0.211989246393278	0.615661475325658\\
0.190983005625053	0.587785252292473\\
0.170962427444958	0.559192903470747\\
0.151951903843574	0.529919264233205\\
0.133974596215561	0.5\\
0.117052407141073	0.469471562785891\\
0.101205953700833	0.438371146789077\\
0.0864545423573992	0.4067366430758\\
0.0728161454332127	0.374606593415912\\
0.0603073792140917	0.342020143325669\\
0.0489434837048465	0.309016994374948\\
0.0387383040616813	0.275637355817\\
0.0297042737240035	0.241921895599668\\
0.0218523992661943	0.207911690817759\\
0.015192246987792	0.17364817766693\\
0.00973193125842975	0.139173100960066\\
0.00547810463172671	0.104528463267654\\
0.0024359497401758	0.0697564737441255\\
0.000609172980904238	0.0348994967025007\\
0	1.22464679914735e-16\\
0.000609172980904238	-0.0348994967025009\\
0.00243594974017569	-0.0697564737441248\\
0.0054781046317266	-0.104528463267653\\
0.00973193125842975	-0.139173100960066\\
0.015192246987792	-0.17364817766693\\
0.0218523992661943	-0.207911690817759\\
0.0297042737240034	-0.241921895599668\\
0.0387383040616811	-0.275637355816999\\
0.0489434837048465	-0.309016994374948\\
0.0603073792140916	-0.342020143325669\\
0.0728161454332126	-0.374606593415912\\
0.0864545423573989	-0.4067366430758\\
0.101205953700833	-0.438371146789077\\
0.117052407141073	-0.469471562785891\\
0.133974596215561	-0.5\\
0.151951903843574	-0.529919264233205\\
0.170962427444958	-0.559192903470747\\
0.190983005625053	-0.587785252292473\\
0.211989246393278	-0.615661475325658\\
0.233955556881022	-0.642787609686539\\
0.256855174522606	-0.669130606358858\\
0.280660199661349	-0.694658370458997\\
0.305341629541003	-0.719339800338651\\
0.330869393641142	-0.743144825477394\\
0.357212390313461	-0.766044443118978\\
0.384338524674342	-0.788010753606722\\
0.412214747707527	-0.809016994374947\\
0.440807096529253	-0.829037572555041\\
0.470080735766795	-0.848048096156426\\
0.5	-0.866025403784438\\
0.530528437214109	-0.882947592858927\\
0.561628853210922	-0.898794046299167\\
0.5932633569242	-0.913545457642601\\
0.625393406584088	-0.927183854566787\\
0.657979856674331	-0.939692620785908\\
0.690983005625052	-0.951056516295154\\
0.724362644183001	-0.961261695938319\\
0.758078104400332	-0.970295726275996\\
0.79208830918224	-0.978147600733806\\
0.82635182233307	-0.984807753012208\\
0.860826899039935	-0.99026806874157\\
0.895471536732347	-0.994521895368273\\
0.930243526255874	-0.997564050259824\\
0.965100503297498	-0.999390827019096\\
1	-1\\
1.0348994967025	-0.999390827019096\\
1.06975647374413	-0.997564050259824\\
1.10452846326765	-0.994521895368273\\
1.13917310096007	-0.99026806874157\\
1.17364817766693	-0.984807753012208\\
1.20791169081776	-0.978147600733806\\
1.24192189559967	-0.970295726275997\\
1.275637355817	-0.961261695938319\\
1.30901699437495	-0.951056516295154\\
1.34202014332567	-0.939692620785909\\
1.37460659341591	-0.927183854566787\\
1.4067366430758	-0.913545457642601\\
1.43837114678908	-0.898794046299167\\
1.46947156278589	-0.882947592858927\\
1.5	-0.866025403784439\\
1.5299192642332	-0.848048096156426\\
1.55919290347075	-0.829037572555042\\
1.58778525229247	-0.809016994374948\\
1.61566147532566	-0.788010753606722\\
1.64278760968654	-0.766044443118978\\
1.66913060635886	-0.743144825477395\\
1.694658370459	-0.719339800338652\\
1.71933980033865	-0.694658370458998\\
1.74314482547739	-0.669130606358858\\
1.76604444311898	-0.64278760968654\\
1.78801075360672	-0.615661475325659\\
1.80901699437495	-0.587785252292473\\
1.82903757255504	-0.559192903470747\\
1.84804809615643	-0.529919264233206\\
1.86602540378444	-0.5\\
1.88294759285893	-0.469471562785891\\
1.89879404629917	-0.438371146789077\\
1.9135454576426	-0.4067366430758\\
1.92718385456679	-0.374606593415912\\
1.93969262078591	-0.342020143325669\\
1.95105651629515	-0.309016994374948\\
1.96126169593832	-0.275637355817\\
1.970295726276	-0.241921895599668\\
1.97814760073381	-0.20791169081776\\
1.98480775301221	-0.173648177666931\\
1.99026806874157	-0.139173100960066\\
1.99452189536827	-0.104528463267653\\
1.99756405025982	-0.0697564737441248\\
1.9993908270191	-0.0348994967025008\\
2	-2.44929359829471e-16\\
};
\addplot [color=white!85!black, forget plot]
  table[row sep=crcr]{%
1.83	0\\
1.82949438642585	0.0289665822630758\\
1.82797816171565	0.057897873207624\\
1.82545317315567	0.0867586245121524\\
1.8219224970555	0.115513673796854\\
1.81739043500013	0.144127987463552\\
1.81186250860906	0.17256670337874\\
1.80534545280908	0.200795173347724\\
1.7978472076288	0.228779005328109\\
1.78937690852498	0.256484105331206\\
1.7799448752523	0.283876718960305\\
1.76956259929043	0.310923472535207\\
1.75824272984336	0.337591413752914\\
1.74599905842831	0.363848051834934\\
1.73284650207291	0.389661397112289\\
1.71880108514108	0.415\\
1.70387991980983	0.43983298931356\\
1.68810118522068	0.46413010988072\\
1.67148410533121	0.487861759402753\\
1.65404892549358	0.510999024520296\\
1.63581688778875	0.533513716039828\\
1.61681020514624	0.555378403277852\\
1.59705203428108	0.576566447480968\\
1.57656644748097	0.59705203428108\\
1.55537840327785	0.616810205146237\\
1.53351371603983	0.635816887788752\\
1.5109990245203	0.654048925493579\\
1.48786175940275	0.671484105331206\\
1.46413010988072	0.688101185220685\\
1.43983298931356	0.703879919809834\\
1.415	0.718801085141084\\
1.38966139711229	0.732846502072909\\
1.36384805183493	0.745999058428309\\
1.33759141375291	0.758242729843359\\
1.31092347253521	0.769562599290434\\
1.28387671896031	0.779944875252304\\
1.25648410533121	0.789376908524977\\
1.22877900532811	0.797847207628805\\
1.20079517334772	0.805345452809077\\
1.17256670337874	0.811862508609059\\
1.14412798746355	0.817390435000133\\
1.11551367379685	0.821922497055503\\
1.08675862451215	0.825453173155667\\
1.05789787320762	0.827978161715654\\
1.02896658226308	0.829494386425849\\
1	0.83\\
0.971033417736924	0.829494386425849\\
0.942102126792376	0.827978161715654\\
0.913241375487848	0.825453173155667\\
0.884486326203146	0.821922497055503\\
0.855872012536448	0.817390435000133\\
0.82743329662126	0.811862508609059\\
0.799204826652276	0.805345452809077\\
0.771220994671891	0.797847207628805\\
0.743515894668794	0.789376908524977\\
0.716123281039695	0.779944875252304\\
0.689076527464793	0.769562599290434\\
0.662408586247086	0.758242729843359\\
0.636151948165066	0.745999058428308\\
0.610338602887711	0.732846502072909\\
0.585	0.718801085141084\\
0.56016701068644	0.703879919809834\\
0.53586989011928	0.688101185220685\\
0.512138240597247	0.671484105331206\\
0.489000975479704	0.654048925493579\\
0.466486283960172	0.635816887788752\\
0.444621596722148	0.616810205146237\\
0.423433552519032	0.597052034281081\\
0.402947965718919	0.576566447480968\\
0.383189794853763	0.555378403277852\\
0.364183112211248	0.533513716039828\\
0.345951074506421	0.510999024520296\\
0.328515894668794	0.487861759402753\\
0.311898814779315	0.46413010988072\\
0.296120080190166	0.43983298931356\\
0.281198914858916	0.415\\
0.267153497927091	0.38966139711229\\
0.254000941571691	0.363848051834934\\
0.241757270156641	0.337591413752914\\
0.230437400709567	0.310923472535207\\
0.220055124747696	0.283876718960305\\
0.210623091475023	0.256484105331206\\
0.202152792371196	0.22877900532811\\
0.194654547190923	0.200795173347724\\
0.188137491390941	0.17256670337874\\
0.182609564999867	0.144127987463552\\
0.178077502944497	0.115513673796855\\
0.174546826844333	0.0867586245121526\\
0.172021838284346	0.0578978732076242\\
0.170505613574151	0.0289665822630756\\
0.17	1.0164568432923e-16\\
0.170505613574151	-0.0289665822630757\\
0.172021838284346	-0.0578978732076236\\
0.174546826844333	-0.086758624512152\\
0.178077502944497	-0.115513673796854\\
0.182609564999867	-0.144127987463552\\
0.188137491390941	-0.17256670337874\\
0.194654547190923	-0.200795173347724\\
0.202152792371195	-0.228779005328109\\
0.210623091475023	-0.256484105331207\\
0.220055124747696	-0.283876718960305\\
0.230437400709566	-0.310923472535207\\
0.241757270156641	-0.337591413752914\\
0.254000941571691	-0.363848051834934\\
0.267153497927091	-0.389661397112289\\
0.281198914858916	-0.415\\
0.296120080190166	-0.43983298931356\\
0.311898814779315	-0.46413010988072\\
0.328515894668794	-0.487861759402753\\
0.345951074506421	-0.510999024520296\\
0.364183112211248	-0.533513716039828\\
0.383189794853763	-0.555378403277852\\
0.40294796571892	-0.576566447480968\\
0.423433552519032	-0.59705203428108\\
0.444621596722148	-0.616810205146237\\
0.466486283960172	-0.635816887788752\\
0.489000975479704	-0.654048925493579\\
0.512138240597247	-0.671484105331206\\
0.53586989011928	-0.688101185220684\\
0.56016701068644	-0.703879919809834\\
0.585	-0.718801085141084\\
0.610338602887711	-0.732846502072909\\
0.636151948165065	-0.745999058428308\\
0.662408586247086	-0.758242729843359\\
0.689076527464793	-0.769562599290433\\
0.716123281039694	-0.779944875252304\\
0.743515894668794	-0.789376908524977\\
0.771220994671891	-0.797847207628805\\
0.799204826652276	-0.805345452809077\\
0.827433296621259	-0.811862508609059\\
0.855872012536448	-0.817390435000133\\
0.884486326203146	-0.821922497055503\\
0.913241375487848	-0.825453173155667\\
0.942102126792376	-0.827978161715654\\
0.971033417736924	-0.829494386425849\\
1	-0.83\\
1.02896658226308	-0.829494386425849\\
1.05789787320762	-0.827978161715654\\
1.08675862451215	-0.825453173155667\\
1.11551367379685	-0.821922497055503\\
1.14412798746355	-0.817390435000133\\
1.17256670337874	-0.811862508609059\\
1.20079517334772	-0.805345452809077\\
1.22877900532811	-0.797847207628805\\
1.25648410533121	-0.789376908524977\\
1.2838767189603	-0.779944875252304\\
1.31092347253521	-0.769562599290434\\
1.33759141375291	-0.758242729843359\\
1.36384805183493	-0.745999058428309\\
1.38966139711229	-0.732846502072909\\
1.415	-0.718801085141084\\
1.43983298931356	-0.703879919809834\\
1.46413010988072	-0.688101185220685\\
1.48786175940275	-0.671484105331206\\
1.5109990245203	-0.654048925493579\\
1.53351371603983	-0.635816887788752\\
1.55537840327785	-0.616810205146238\\
1.57656644748097	-0.597052034281081\\
1.59705203428108	-0.576566447480968\\
1.61681020514624	-0.555378403277852\\
1.63581688778875	-0.533513716039828\\
1.65404892549358	-0.510999024520297\\
1.67148410533121	-0.487861759402753\\
1.68810118522068	-0.46413010988072\\
1.70387991980983	-0.439832989313561\\
1.71880108514108	-0.415\\
1.73284650207291	-0.389661397112289\\
1.74599905842831	-0.363848051834934\\
1.75824272984336	-0.337591413752914\\
1.76956259929043	-0.310923472535207\\
1.7799448752523	-0.283876718960305\\
1.78937690852498	-0.256484105331207\\
1.7978472076288	-0.22877900532811\\
1.80534545280908	-0.200795173347724\\
1.81186250860906	-0.172566703378741\\
1.81739043500013	-0.144127987463553\\
1.8219224970555	-0.115513673796855\\
1.82545317315567	-0.0867586245121523\\
1.82797816171565	-0.0578978732076235\\
1.82949438642585	-0.0289665822630757\\
1.83	-2.03291368658461e-16\\
};
\addplot [color=white!85!black, forget plot]
  table[row sep=crcr]{%
1.7	0\\
1.69957357891337	0.0244296476917507\\
1.69829483518188	0.0488295316208877\\
1.69616532675779	0.0731699242873574\\
1.6931876481191	0.0974211706720458\\
1.68936542710855	0.121553724366851\\
1.68470332051366	0.145538183572432\\
1.6792070083932	0.169345326919767\\
1.67288318715682	0.192946149071899\\
1.66573956140661	0.216311896062463\\
1.65778483455014	0.239414100327968\\
1.64902869819675	0.262224615391138\\
1.63948182034982	0.28471565015306\\
1.62915583240942	0.306859802752354\\
1.61806331500125	0.328630093950124\\
1.60621778264911	0.35\\
1.5936336673095	0.370943484963243\\
1.58032630078853	0.391435032429523\\
1.56631189606246	0.411449676604731\\
1.55160752752471	0.430963032727961\\
1.53623111018328	0.449951326780577\\
1.52020137783418	0.468391424451201\\
1.50353786023706	0.486260859321298\\
1.4862608593213	0.503537860237056\\
1.4683914244512	0.520201377834176\\
1.44995132678058	0.536231110183285\\
1.43096303272796	0.551607527524705\\
1.41144967660473	0.566311896062463\\
1.39143503242952	0.580326300788529\\
1.37094348496324	0.593633667309498\\
1.35	0.606217782649107\\
1.32863009395012	0.618063315001249\\
1.30685980275235	0.629155832409417\\
1.28471565015306	0.639481820349821\\
1.26222461539114	0.649028698196751\\
1.23941410032797	0.657784834550136\\
1.21631189606246	0.665739561406607\\
1.1929461490719	0.672883187156823\\
1.16934532691977	0.679207008393197\\
1.14553818357243	0.684703320513664\\
1.12155372436685	0.689365427108546\\
1.09742117067205	0.693187648119099\\
1.07316992428736	0.696165326757791\\
1.04882953162089	0.698294835181877\\
1.02442964769175	0.699573578913367\\
1	0.7\\
0.97557035230825	0.699573578913367\\
0.951170468379112	0.698294835181877\\
0.926830075712643	0.696165326757791\\
0.902578829327954	0.693187648119099\\
0.878446275633149	0.689365427108546\\
0.854461816427569	0.684703320513664\\
0.830654673080233	0.679207008393197\\
0.807053850928101	0.672883187156823\\
0.783688103937537	0.665739561406608\\
0.760585899672032	0.657784834550136\\
0.737775384608861	0.649028698196751\\
0.71528434984694	0.639481820349821\\
0.693140197247646	0.629155832409417\\
0.671369906049877	0.618063315001249\\
0.65	0.606217782649107\\
0.629056515036757	0.593633667309498\\
0.608564967570477	0.580326300788529\\
0.588550323395269	0.566311896062463\\
0.569036967272039	0.551607527524705\\
0.550048673219423	0.536231110183285\\
0.531608575548799	0.520201377834176\\
0.513739140678702	0.503537860237056\\
0.496462139762944	0.486260859321298\\
0.479798622165824	0.468391424451201\\
0.463768889816715	0.449951326780578\\
0.448392472475295	0.430963032727961\\
0.433688103937537	0.411449676604731\\
0.419673699211471	0.391435032429523\\
0.406366332690502	0.370943484963243\\
0.393782217350893	0.35\\
0.381936684998751	0.328630093950124\\
0.370844167590583	0.306859802752354\\
0.36051817965018	0.28471565015306\\
0.350971301803249	0.262224615391139\\
0.342215165449864	0.239414100327968\\
0.334260438593393	0.216311896062463\\
0.327116812843177	0.1929461490719\\
0.320792991606803	0.169345326919767\\
0.315296679486336	0.145538183572432\\
0.310634572891454	0.121553724366851\\
0.306812351880901	0.097421170672046\\
0.303834673242209	0.0731699242873576\\
0.301705164818123	0.0488295316208879\\
0.300426421086633	0.0244296476917505\\
0.3	8.57252759403147e-17\\
0.300426421086633	-0.0244296476917506\\
0.301705164818123	-0.0488295316208874\\
0.303834673242209	-0.0731699242873571\\
0.306812351880901	-0.0974211706720459\\
0.310634572891454	-0.121553724366851\\
0.315296679486336	-0.145538183572431\\
0.320792991606802	-0.169345326919767\\
0.327116812843177	-0.192946149071899\\
0.334260438593393	-0.216311896062463\\
0.342215165449864	-0.239414100327968\\
0.350971301803249	-0.262224615391138\\
0.360518179650179	-0.28471565015306\\
0.370844167590583	-0.306859802752354\\
0.381936684998751	-0.328630093950124\\
0.393782217350893	-0.35\\
0.406366332690502	-0.370943484963243\\
0.419673699211471	-0.391435032429523\\
0.433688103937537	-0.411449676604731\\
0.448392472475294	-0.43096303272796\\
0.463768889816715	-0.449951326780577\\
0.479798622165824	-0.468391424451201\\
0.496462139762944	-0.486260859321298\\
0.513739140678702	-0.503537860237056\\
0.531608575548799	-0.520201377834176\\
0.550048673219422	-0.536231110183285\\
0.569036967272039	-0.551607527524705\\
0.588550323395269	-0.566311896062463\\
0.608564967570477	-0.580326300788529\\
0.629056515036757	-0.593633667309498\\
0.65	-0.606217782649107\\
0.671369906049877	-0.618063315001249\\
0.693140197247646	-0.629155832409417\\
0.71528434984694	-0.639481820349821\\
0.737775384608861	-0.649028698196751\\
0.760585899672031	-0.657784834550136\\
0.783688103937537	-0.665739561406607\\
0.807053850928101	-0.672883187156823\\
0.830654673080233	-0.679207008393197\\
0.854461816427568	-0.684703320513664\\
0.878446275633149	-0.689365427108546\\
0.902578829327955	-0.693187648119099\\
0.926830075712643	-0.696165326757791\\
0.951170468379112	-0.698294835181877\\
0.975570352308249	-0.699573578913367\\
1	-0.7\\
1.02442964769175	-0.699573578913367\\
1.04882953162089	-0.698294835181877\\
1.07316992428736	-0.696165326757791\\
1.09742117067205	-0.693187648119099\\
1.12155372436685	-0.689365427108546\\
1.14553818357243	-0.684703320513664\\
1.16934532691977	-0.679207008393198\\
1.1929461490719	-0.672883187156823\\
1.21631189606246	-0.665739561406608\\
1.23941410032797	-0.657784834550136\\
1.26222461539114	-0.649028698196751\\
1.28471565015306	-0.63948182034982\\
1.30685980275235	-0.629155832409417\\
1.32863009395012	-0.618063315001249\\
1.35	-0.606217782649107\\
1.37094348496324	-0.593633667309498\\
1.39143503242952	-0.580326300788529\\
1.41144967660473	-0.566311896062463\\
1.43096303272796	-0.551607527524705\\
1.44995132678058	-0.536231110183285\\
1.4683914244512	-0.520201377834176\\
1.4862608593213	-0.503537860237056\\
1.50353786023706	-0.486260859321298\\
1.52020137783418	-0.468391424451201\\
1.53623111018328	-0.449951326780578\\
1.5516075275247	-0.430963032727961\\
1.56631189606246	-0.411449676604731\\
1.58032630078853	-0.391435032429523\\
1.5936336673095	-0.370943484963244\\
1.60621778264911	-0.35\\
1.61806331500125	-0.328630093950124\\
1.62915583240942	-0.306859802752354\\
1.63948182034982	-0.28471565015306\\
1.64902869819675	-0.262224615391139\\
1.65778483455014	-0.239414100327968\\
1.66573956140661	-0.216311896062463\\
1.67288318715682	-0.1929461490719\\
1.6792070083932	-0.169345326919768\\
1.68470332051366	-0.145538183572432\\
1.68936542710855	-0.121553724366852\\
1.6931876481191	-0.0974211706720461\\
1.69616532675779	-0.0731699242873574\\
1.69829483518188	-0.0488295316208873\\
1.69957357891337	-0.0244296476917506\\
1.7	-1.71450551880629e-16\\
};
\addplot [color=white!85!black, forget plot]
  table[row sep=crcr]{%
1.58	0\\
1.57964667967108	0.0202417080874506\\
1.5785871491507	0.0404587547715927\\
1.5768226993136	0.060626508695239\\
1.57435547987011	0.0807203985568379\\
1.57118849674708	0.10071594304682\\
1.56732560842561	0.1205887806743\\
1.56277152124008	0.140314699447807\\
1.55753178364423	0.159869666373859\\
1.55161277945119	0.179229856737469\\
1.54502172005583	0.198371683128888\\
1.53776663564874	0.217271824181229\\
1.52985636543271	0.235907252983964\\
1.52130054685352	0.254255265137665\\
1.51210960385818	0.272293506415817\\
1.50229473419497	0.29\\
1.49186789577073	0.307353173255259\\
1.48084179208192	0.324331884013033\\
1.46922985673747	0.340915446329634\\
1.4570462370919	0.357083655688882\\
1.44430577700901	0.372816813618193\\
1.43102399877689	0.388095751688138\\
1.41721708419642	0.402901854866218\\
1.40290185486622	0.417217084196418\\
1.38809575168814	0.431023998776889\\
1.37281681361819	0.444305777009007\\
1.35708365568888	0.457046237091899\\
1.34091544632963	0.469229856737469\\
1.32433188401303	0.480841792081924\\
1.30735317325526	0.491867895770727\\
1.29	0.502294734194974\\
1.27229350641582	0.512109603858178\\
1.25425526513766	0.521300546853517\\
1.23590725298396	0.529856365432708\\
1.21727182418123	0.537766635648737\\
1.19837168312889	0.545021720055827\\
1.17922985673747	0.551612779451189\\
1.15986966637386	0.557531783644225\\
1.14031469944781	0.562771521240078\\
1.1205887806743	0.567325608425607\\
1.10071594304682	0.571188496747081\\
1.08072039855684	0.574355479870111\\
1.06062650869524	0.576822699313598\\
1.04045875477159	0.578587149150698\\
1.02024170808745	0.579646679671076\\
1	0.58\\
0.97975829191255	0.579646679671076\\
0.959541245228407	0.578587149150698\\
0.939373491304761	0.576822699313599\\
0.919279601443162	0.574355479870111\\
0.89928405695318	0.571188496747081\\
0.8794112193257	0.567325608425607\\
0.859685300552193	0.562771521240078\\
0.840130333626141	0.557531783644225\\
0.820770143262531	0.551612779451189\\
0.801628316871112	0.545021720055827\\
0.782728175818771	0.537766635648737\\
0.764092747016036	0.529856365432709\\
0.745744734862335	0.521300546853517\\
0.727706493584183	0.512109603858178\\
0.71	0.502294734194974\\
0.692646826744741	0.491867895770727\\
0.675668115986967	0.480841792081924\\
0.659084553670366	0.469229856737469\\
0.642916344311118	0.457046237091899\\
0.627183186381807	0.444305777009007\\
0.611904248311862	0.431023998776889\\
0.597098145133782	0.417217084196418\\
0.582782915803582	0.402901854866218\\
0.568976001223111	0.388095751688138\\
0.555694222990993	0.372816813618193\\
0.542953762908101	0.357083655688882\\
0.530770143262531	0.340915446329634\\
0.519158207918076	0.324331884013033\\
0.508132104229273	0.307353173255259\\
0.497705265805026	0.29\\
0.487890396141823	0.272293506415817\\
0.478699453146483	0.254255265137665\\
0.470143634567292	0.235907252983964\\
0.462233364351263	0.217271824181229\\
0.454978279944173	0.198371683128888\\
0.448387220548811	0.17922985673747\\
0.442468216355775	0.15986966637386\\
0.437228478759922	0.140314699447807\\
0.432674391574393	0.1205887806743\\
0.428811503252919	0.10071594304682\\
0.425644520129889	0.0807203985568381\\
0.423177300686402	0.0606265086952392\\
0.421412850849302	0.0404587547715928\\
0.420353320328924	0.0202417080874504\\
0.42	7.10295143505465e-17\\
0.420353320328924	-0.0202417080874505\\
0.421412850849302	-0.0404587547715924\\
0.423177300686401	-0.0606265086952388\\
0.425644520129889	-0.080720398556838\\
0.428811503252919	-0.10071594304682\\
0.432674391574393	-0.1205887806743\\
0.437228478759922	-0.140314699447807\\
0.442468216355775	-0.159869666373859\\
0.448387220548811	-0.17922985673747\\
0.454978279944173	-0.198371683128888\\
0.462233364351263	-0.217271824181229\\
0.470143634567291	-0.235907252983964\\
0.478699453146483	-0.254255265137665\\
0.487890396141822	-0.272293506415817\\
0.497705265805026	-0.29\\
0.508132104229273	-0.307353173255259\\
0.519158207918076	-0.324331884013033\\
0.53077014326253	-0.340915446329634\\
0.542953762908101	-0.357083655688882\\
0.555694222990993	-0.372816813618193\\
0.568976001223111	-0.388095751688138\\
0.582782915803582	-0.402901854866218\\
0.597098145133782	-0.417217084196417\\
0.611904248311862	-0.431023998776889\\
0.627183186381807	-0.444305777009007\\
0.642916344311118	-0.457046237091899\\
0.659084553670366	-0.469229856737469\\
0.675668115986967	-0.480841792081924\\
0.692646826744741	-0.491867895770727\\
0.71	-0.502294734194974\\
0.727706493584183	-0.512109603858178\\
0.745744734862335	-0.521300546853517\\
0.764092747016036	-0.529856365432709\\
0.782728175818771	-0.537766635648737\\
0.801628316871112	-0.545021720055827\\
0.82077014326253	-0.551612779451189\\
0.840130333626141	-0.557531783644225\\
0.859685300552193	-0.562771521240078\\
0.879411219325699	-0.567325608425607\\
0.89928405695318	-0.571188496747081\\
0.919279601443162	-0.574355479870111\\
0.939373491304761	-0.576822699313599\\
0.959541245228407	-0.578587149150698\\
0.979758291912549	-0.579646679671075\\
1	-0.58\\
1.02024170808745	-0.579646679671076\\
1.04045875477159	-0.578587149150698\\
1.06062650869524	-0.576822699313599\\
1.08072039855684	-0.574355479870111\\
1.10071594304682	-0.571188496747081\\
1.1205887806743	-0.567325608425607\\
1.14031469944781	-0.562771521240078\\
1.15986966637386	-0.557531783644225\\
1.17922985673747	-0.551612779451189\\
1.19837168312889	-0.545021720055827\\
1.21727182418123	-0.537766635648737\\
1.23590725298396	-0.529856365432708\\
1.25425526513766	-0.521300546853517\\
1.27229350641582	-0.512109603858178\\
1.29	-0.502294734194974\\
1.30735317325526	-0.491867895770727\\
1.32433188401303	-0.480841792081924\\
1.34091544632963	-0.46922985673747\\
1.35708365568888	-0.457046237091899\\
1.37281681361819	-0.444305777009007\\
1.38809575168814	-0.431023998776889\\
1.40290185486622	-0.417217084196418\\
1.41721708419642	-0.402901854866219\\
1.43102399877689	-0.388095751688138\\
1.44430577700901	-0.372816813618193\\
1.4570462370919	-0.357083655688882\\
1.46922985673747	-0.340915446329634\\
1.48084179208192	-0.324331884013033\\
1.49186789577073	-0.307353173255259\\
1.50229473419497	-0.29\\
1.51210960385818	-0.272293506415817\\
1.52130054685352	-0.254255265137665\\
1.52985636543271	-0.235907252983964\\
1.53776663564874	-0.217271824181229\\
1.54502172005583	-0.198371683128888\\
1.55161277945119	-0.17922985673747\\
1.55753178364422	-0.15986966637386\\
1.56277152124008	-0.140314699447807\\
1.56732560842561	-0.120588780674301\\
1.57118849674708	-0.10071594304682\\
1.57435547987011	-0.0807203985568382\\
1.5768226993136	-0.060626508695239\\
1.5785871491507	-0.0404587547715924\\
1.57964667967108	-0.0202417080874505\\
1.58	-1.42059028701093e-16\\
};
\addplot [color=white!85!black, forget plot]
  table[row sep=crcr]{%
1.46	0\\
1.45971978042878	0.0160537684831504\\
1.45887946311952	0.0320879779222976\\
1.45748007186941	0.0480830931031206\\
1.45552331162112	0.0640196264416301\\
1.45301156638562	0.079878161726788\\
1.44994789633755	0.0956393777761693\\
1.44633603408696	0.111284071975847\\
1.44218038013163	0.12679318367582\\
1.43748599749577	0.142147817412476\\
1.43225860556152	0.157329265929808\\
1.42650457310072	0.17231903297132\\
1.4202309105156	0.187098855814868\\
1.41344526129762	0.201650727522976\\
1.40615589271511	0.21595691888151\\
1.39837168574084	0.23\\
1.39010212423196	0.243762861547274\\
1.38135728337532	0.257228735596544\\
1.37214781741248	0.270381216054538\\
1.36248494665909	0.283204278649803\\
1.35238044383473	0.295682300455808\\
1.3418466197196	0.307800078925075\\
1.33089630815578	0.319542850411139\\
1.31954285041114	0.33089630815578\\
1.30780007892507	0.341846619719601\\
1.29568230045581	0.35238044383473\\
1.2832042786498	0.362484946659092\\
1.27038121605454	0.372147817412476\\
1.25722873559654	0.381357283375319\\
1.24376286154727	0.390102124231956\\
1.23	0.398371685740842\\
1.21595691888151	0.406155892715106\\
1.20165072752298	0.413445261297617\\
1.18709885581487	0.420230910515596\\
1.17231903297132	0.426504573100722\\
1.15732926592981	0.432258605561518\\
1.14214781741248	0.437485997495771\\
1.12679318367582	0.442180380131627\\
1.11128407197585	0.446336034086958\\
1.09563937777617	0.449947896337551\\
1.07987816172679	0.453011566385616\\
1.06401962644163	0.455523311621122\\
1.04808309310312	0.457480071869406\\
1.0320879779223	0.458879463119519\\
1.01605376848315	0.459719780428784\\
1	0.46\\
0.98394623151685	0.459719780428784\\
0.967912022077702	0.458879463119519\\
0.951916906896879	0.457480071869406\\
0.93598037355837	0.455523311621122\\
0.920121838273212	0.453011566385616\\
0.904360622223831	0.449947896337551\\
0.888715928024153	0.446336034086958\\
0.87320681632418	0.442180380131627\\
0.857852182587524	0.437485997495771\\
0.842670734070192	0.432258605561518\\
0.82768096702868	0.426504573100722\\
0.812901144185132	0.420230910515596\\
0.798349272477024	0.413445261297617\\
0.78404308111849	0.406155892715106\\
0.77	0.398371685740842\\
0.756237138452726	0.390102124231956\\
0.742771264403457	0.381357283375319\\
0.729618783945462	0.372147817412476\\
0.716795721350197	0.362484946659092\\
0.704317699544192	0.35238044383473\\
0.692199921074925	0.341846619719601\\
0.680457149588861	0.33089630815578\\
0.66910369184422	0.319542850411139\\
0.658153380280399	0.307800078925075\\
0.64761955616527	0.295682300455808\\
0.637515053340908	0.283204278649803\\
0.627852182587524	0.270381216054538\\
0.618642716624681	0.257228735596544\\
0.609897875768044	0.243762861547274\\
0.601628314259158	0.23\\
0.593844107284894	0.21595691888151\\
0.586554738702383	0.201650727522976\\
0.579769089484404	0.187098855814868\\
0.573495426899278	0.17231903297132\\
0.567741394438482	0.157329265929808\\
0.562514002504229	0.142147817412476\\
0.557819619868373	0.12679318367582\\
0.553663965913042	0.111284071975847\\
0.550052103662449	0.0956393777761693\\
0.546988433614384	0.0798781617267879\\
0.544476688378878	0.0640196264416302\\
0.542519928130594	0.0480830931031207\\
0.541120536880481	0.0320879779222977\\
0.540280219571216	0.0160537684831503\\
0.54	5.63337527607782e-17\\
0.540280219571216	-0.0160537684831504\\
0.541120536880481	-0.0320879779222974\\
0.542519928130594	-0.0480830931031204\\
0.544476688378878	-0.0640196264416301\\
0.546988433614384	-0.079878161726788\\
0.550052103662449	-0.0956393777761692\\
0.553663965913042	-0.111284071975847\\
0.557819619868373	-0.12679318367582\\
0.562514002504229	-0.142147817412476\\
0.567741394438482	-0.157329265929808\\
0.573495426899278	-0.17231903297132\\
0.579769089484403	-0.187098855814868\\
0.586554738702383	-0.201650727522975\\
0.593844107284894	-0.21595691888151\\
0.601628314259158	-0.23\\
0.609897875768044	-0.243762861547274\\
0.618642716624681	-0.257228735596543\\
0.627852182587524	-0.270381216054538\\
0.637515053340908	-0.283204278649803\\
0.64761955616527	-0.295682300455808\\
0.658153380280399	-0.307800078925075\\
0.66910369184422	-0.319542850411139\\
0.680457149588861	-0.330896308155779\\
0.692199921074925	-0.341846619719601\\
0.704317699544192	-0.35238044383473\\
0.716795721350197	-0.362484946659092\\
0.729618783945462	-0.372147817412476\\
0.742771264403456	-0.381357283375319\\
0.756237138452726	-0.390102124231956\\
0.77	-0.398371685740842\\
0.78404308111849	-0.406155892715106\\
0.798349272477024	-0.413445261297617\\
0.812901144185132	-0.420230910515596\\
0.82768096702868	-0.426504573100722\\
0.842670734070192	-0.432258605561518\\
0.857852182587524	-0.437485997495771\\
0.873206816324181	-0.442180380131627\\
0.888715928024153	-0.446336034086958\\
0.904360622223831	-0.449947896337551\\
0.920121838273212	-0.453011566385616\\
0.93598037355837	-0.455523311621122\\
0.951916906896879	-0.457480071869406\\
0.967912022077702	-0.458879463119519\\
0.983946231516849	-0.459719780428784\\
1	-0.46\\
1.01605376848315	-0.459719780428784\\
1.0320879779223	-0.458879463119519\\
1.04808309310312	-0.457480071869406\\
1.06401962644163	-0.455523311621122\\
1.07987816172679	-0.453011566385616\\
1.09563937777617	-0.449947896337551\\
1.11128407197585	-0.446336034086958\\
1.12679318367582	-0.442180380131627\\
1.14214781741248	-0.437485997495771\\
1.15732926592981	-0.432258605561518\\
1.17231903297132	-0.426504573100722\\
1.18709885581487	-0.420230910515596\\
1.20165072752298	-0.413445261297617\\
1.21595691888151	-0.406155892715106\\
1.23	-0.398371685740842\\
1.24376286154727	-0.390102124231956\\
1.25722873559654	-0.381357283375319\\
1.27038121605454	-0.372147817412476\\
1.2832042786498	-0.362484946659092\\
1.29568230045581	-0.35238044383473\\
1.30780007892507	-0.341846619719602\\
1.31954285041114	-0.33089630815578\\
1.33089630815578	-0.319542850411139\\
1.3418466197196	-0.307800078925075\\
1.35238044383473	-0.295682300455808\\
1.36248494665909	-0.283204278649803\\
1.37214781741248	-0.270381216054538\\
1.38135728337532	-0.257228735596544\\
1.39010212423196	-0.243762861547275\\
1.39837168574084	-0.23\\
1.40615589271511	-0.21595691888151\\
1.41344526129762	-0.201650727522975\\
1.4202309105156	-0.187098855814868\\
1.42650457310072	-0.17231903297132\\
1.43225860556152	-0.157329265929808\\
1.43748599749577	-0.142147817412476\\
1.44218038013163	-0.12679318367582\\
1.44633603408696	-0.111284071975847\\
1.44994789633755	-0.0956393777761695\\
1.45301156638562	-0.0798781617267884\\
1.45552331162112	-0.0640196264416303\\
1.45748007186941	-0.0480830931031206\\
1.45887946311952	-0.0320879779222974\\
1.45971978042878	-0.0160537684831504\\
1.46	-1.12667505521556e-16\\
};
\addplot [color=white!85!black, forget plot]
  table[row sep=crcr]{%
1.35	0\\
1.34978678945668	0.0122148238458753\\
1.34914741759094	0.0244147658104439\\
1.3480826633789	0.0365849621436787\\
1.34659382405955	0.0487105853360229\\
1.34468271355427	0.0607768621834256\\
1.34235166025683	0.0727690917862158\\
1.3396035041966	0.0846726634598837\\
1.33644159357841	0.0964730745359497\\
1.3328697807033	0.108155948031232\\
1.32889241727507	0.119707050163984\\
1.32451434909838	0.131112307695569\\
1.31974091017491	0.14235782507653\\
1.31457791620471	0.153429901376177\\
1.30903165750062	0.164315046975062\\
1.30310889132455	0.175\\
1.29681683365475	0.185471742481622\\
1.29016315039426	0.195717516214761\\
1.28315594803123	0.205724838302366\\
1.27580376376235	0.21548151636398\\
1.26811555509164	0.224975663390289\\
1.26010068891709	0.2341957122256\\
1.25176893011853	0.243130429660649\\
1.24313042966065	0.251768930118528\\
1.2341957122256	0.260100688917088\\
1.22497566339029	0.268115555091642\\
1.21548151636398	0.275803763762353\\
1.20572483830237	0.283155948031232\\
1.19571751621476	0.290163150394265\\
1.18547174248162	0.296816833654749\\
1.175	0.303108891324553\\
1.16431504697506	0.309031657500624\\
1.15342990137618	0.314577916204708\\
1.14235782507653	0.31974091017491\\
1.13111230769557	0.324514349098376\\
1.11970705016398	0.328892417275068\\
1.10815594803123	0.332869780703304\\
1.09647307453595	0.336441593578412\\
1.08467266345988	0.339603504196599\\
1.07276909178622	0.342351660256832\\
1.06077686218343	0.344682713554273\\
1.04871058533602	0.34659382405955\\
1.03658496214368	0.348082663378896\\
1.02441476581044	0.349147417590938\\
1.01221482384588	0.349786789456684\\
1	0.35\\
0.987785176154125	0.349786789456684\\
0.975585234189556	0.349147417590938\\
0.963415037856321	0.348082663378896\\
0.951289414663977	0.34659382405955\\
0.939223137816574	0.344682713554273\\
0.927230908213784	0.342351660256832\\
0.915327336540116	0.339603504196599\\
0.90352692546405	0.336441593578412\\
0.891844051968768	0.332869780703304\\
0.880292949836016	0.328892417275068\\
0.868887692304431	0.324514349098376\\
0.85764217492347	0.31974091017491\\
0.846570098623823	0.314577916204708\\
0.835684953024938	0.309031657500624\\
0.825	0.303108891324554\\
0.814528257518378	0.296816833654749\\
0.804282483785239	0.290163150394265\\
0.794275161697634	0.283155948031232\\
0.78451848363602	0.275803763762353\\
0.775024336609711	0.268115555091642\\
0.7658042877744	0.260100688917088\\
0.756869570339351	0.251768930118528\\
0.748231069881472	0.243130429660649\\
0.739899311082912	0.2341957122256\\
0.731884444908358	0.224975663390289\\
0.724196236237647	0.21548151636398\\
0.716844051968768	0.205724838302366\\
0.709836849605735	0.195717516214761\\
0.703183166345251	0.185471742481622\\
0.696891108675446	0.175\\
0.690968342499376	0.164315046975062\\
0.685422083795292	0.153429901376177\\
0.68025908982509	0.14235782507653\\
0.675485650901624	0.131112307695569\\
0.671107582724932	0.119707050163984\\
0.667130219296696	0.108155948031232\\
0.663558406421588	0.0964730745359499\\
0.660396495803401	0.0846726634598837\\
0.657648339743168	0.0727690917862158\\
0.655317286445727	0.0607768621834256\\
0.65340617594045	0.048710585336023\\
0.651917336621104	0.0365849621436788\\
0.650852582409062	0.0244147658104439\\
0.650213210543316	0.0122148238458752\\
0.65	4.28626379701574e-17\\
0.650213210543316	-0.0122148238458753\\
0.650852582409062	-0.0244147658104437\\
0.651917336621104	-0.0365849621436786\\
0.65340617594045	-0.0487105853360229\\
0.655317286445727	-0.0607768621834257\\
0.657648339743168	-0.0727690917862157\\
0.660396495803401	-0.0846726634598836\\
0.663558406421588	-0.0964730745359496\\
0.667130219296696	-0.108155948031232\\
0.671107582724932	-0.119707050163984\\
0.675485650901624	-0.131112307695569\\
0.68025908982509	-0.14235782507653\\
0.685422083795292	-0.153429901376177\\
0.690968342499376	-0.164315046975062\\
0.696891108675447	-0.175\\
0.703183166345251	-0.185471742481622\\
0.709836849605735	-0.195717516214761\\
0.716844051968768	-0.205724838302366\\
0.724196236237647	-0.21548151636398\\
0.731884444908358	-0.224975663390289\\
0.739899311082912	-0.2341957122256\\
0.748231069881472	-0.243130429660649\\
0.756869570339351	-0.251768930118528\\
0.7658042877744	-0.260100688917088\\
0.775024336609711	-0.268115555091642\\
0.78451848363602	-0.275803763762353\\
0.794275161697634	-0.283155948031232\\
0.804282483785238	-0.290163150394264\\
0.814528257518378	-0.296816833654749\\
0.825	-0.303108891324553\\
0.835684953024938	-0.309031657500624\\
0.846570098623823	-0.314577916204708\\
0.85764217492347	-0.31974091017491\\
0.868887692304431	-0.324514349098376\\
0.880292949836016	-0.328892417275068\\
0.891844051968768	-0.332869780703304\\
0.90352692546405	-0.336441593578412\\
0.915327336540116	-0.339603504196599\\
0.927230908213784	-0.342351660256832\\
0.939223137816574	-0.344682713554273\\
0.951289414663977	-0.34659382405955\\
0.963415037856321	-0.348082663378896\\
0.975585234189556	-0.349147417590938\\
0.987785176154124	-0.349786789456683\\
1	-0.35\\
1.01221482384588	-0.349786789456684\\
1.02441476581044	-0.349147417590938\\
1.03658496214368	-0.348082663378896\\
1.04871058533602	-0.34659382405955\\
1.06077686218343	-0.344682713554273\\
1.07276909178622	-0.342351660256832\\
1.08467266345988	-0.339603504196599\\
1.09647307453595	-0.336441593578412\\
1.10815594803123	-0.332869780703304\\
1.11970705016398	-0.328892417275068\\
1.13111230769557	-0.324514349098376\\
1.14235782507653	-0.31974091017491\\
1.15342990137618	-0.314577916204708\\
1.16431504697506	-0.309031657500624\\
1.175	-0.303108891324553\\
1.18547174248162	-0.296816833654749\\
1.19571751621476	-0.290163150394265\\
1.20572483830237	-0.283155948031232\\
1.21548151636398	-0.275803763762353\\
1.22497566339029	-0.268115555091642\\
1.2341957122256	-0.260100688917088\\
1.24313042966065	-0.251768930118528\\
1.25176893011853	-0.243130429660649\\
1.26010068891709	-0.2341957122256\\
1.26811555509164	-0.224975663390289\\
1.27580376376235	-0.215481516363981\\
1.28315594803123	-0.205724838302366\\
1.29016315039426	-0.195717516214762\\
1.29681683365475	-0.185471742481622\\
1.30310889132455	-0.175\\
1.30903165750062	-0.164315046975062\\
1.31457791620471	-0.153429901376177\\
1.31974091017491	-0.14235782507653\\
1.32451434909838	-0.131112307695569\\
1.32889241727507	-0.119707050163984\\
1.3328697807033	-0.108155948031232\\
1.33644159357841	-0.0964730745359499\\
1.3396035041966	-0.0846726634598838\\
1.34235166025683	-0.0727690917862159\\
1.34468271355427	-0.0607768621834259\\
1.34659382405955	-0.0487105853360231\\
1.3480826633789	-0.0365849621436787\\
1.34914741759094	-0.0244147658104437\\
1.34978678945668	-0.0122148238458753\\
1.35	-8.57252759403147e-17\\
};
\addplot [color=white!85!black, forget plot]
  table[row sep=crcr]{%
1.25	0\\
1.24984770675477	0.00872487417562524\\
1.24939101256496	0.0174391184360313\\
1.24863047384207	0.0261321158169134\\
1.24756701718539	0.0347932752400164\\
1.24620193825305	0.0434120444167326\\
1.24453690018345	0.0519779227044398\\
1.242573931569	0.0604804738999169\\
1.24031542398458	0.0689093389542498\\
1.23776412907379	0.0772542485937368\\
1.23492315519648	0.0855050358314172\\
1.2317959636417	0.093651648353978\\
1.22838636441065	0.10168416076895\\
1.22469851157479	0.109592786697269\\
1.22073689821473	0.117367890696473\\
1.21650635094611	0.125\\
1.21201202403911	0.132479816058301\\
1.20725939313876	0.139798225867687\\
1.20225424859374	0.146946313073118\\
1.19700268840168	0.153915368831415\\
1.19151111077974	0.160696902421635\\
1.18578620636935	0.167282651589715\\
1.17983495008466	0.173664592614749\\
1.17366459261475	0.179834950084663\\
1.16728265158971	0.185786206369349\\
1.16069690242163	0.191511110779745\\
1.15391536883141	0.197002688401681\\
1.14694631307312	0.202254248593737\\
1.13979822586769	0.20725939313876\\
1.1324798160583	0.212012024039107\\
1.125	0.21650635094611\\
1.11736789069647	0.220736898214732\\
1.10959278669727	0.224698511574792\\
1.10168416076895	0.22838636441065\\
1.09365164835398	0.231795963641697\\
1.08550503583142	0.234923155196477\\
1.07725424859374	0.237764129073788\\
1.06890933895425	0.24031542398458\\
1.06048047389992	0.242573931568999\\
1.05197792270444	0.244536900183451\\
1.04341204441673	0.246201938253052\\
1.03479327524002	0.247567017185393\\
1.02613211581691	0.248630473842068\\
1.01743911843603	0.249391012564956\\
1.00872487417563	0.249847706754774\\
1	0.25\\
0.991275125824375	0.249847706754774\\
0.982560881563969	0.249391012564956\\
0.973867884183087	0.248630473842068\\
0.965206724759984	0.247567017185393\\
0.956587955583267	0.246201938253052\\
0.94802207729556	0.244536900183451\\
0.939519526100083	0.242573931568999\\
0.93109066104575	0.24031542398458\\
0.922745751406263	0.237764129073788\\
0.914494964168583	0.234923155196477\\
0.906348351646022	0.231795963641697\\
0.89831583923105	0.22838636441065\\
0.890407213302731	0.224698511574792\\
0.882632109303527	0.220736898214732\\
0.875	0.21650635094611\\
0.867520183941699	0.212012024039107\\
0.860201774132313	0.20725939313876\\
0.853053686926882	0.202254248593737\\
0.846084631168585	0.197002688401681\\
0.839303097578365	0.191511110779745\\
0.832717348410285	0.185786206369349\\
0.826335407385251	0.179834950084663\\
0.820165049915337	0.173664592614749\\
0.814213793630651	0.167282651589715\\
0.808488889220256	0.160696902421635\\
0.80299731159832	0.153915368831415\\
0.797745751406263	0.146946313073118\\
0.79274060686124	0.139798225867687\\
0.787987975960894	0.132479816058301\\
0.78349364905389	0.125\\
0.779263101785268	0.117367890696473\\
0.775301488425208	0.109592786697269\\
0.77161363558935	0.10168416076895\\
0.768204036358303	0.0936516483539781\\
0.765076844803523	0.0855050358314172\\
0.762235870926212	0.0772542485937369\\
0.75968457601542	0.0689093389542499\\
0.757426068431001	0.0604804738999169\\
0.755463099816549	0.0519779227044398\\
0.753798061746948	0.0434120444167326\\
0.752432982814607	0.0347932752400164\\
0.751369526157932	0.0261321158169134\\
0.750608987435044	0.0174391184360314\\
0.750152293245226	0.00872487417562517\\
0.75	3.06161699786838e-17\\
0.750152293245226	-0.00872487417562522\\
0.750608987435044	-0.0174391184360312\\
0.751369526157932	-0.0261321158169133\\
0.752432982814607	-0.0347932752400164\\
0.753798061746948	-0.0434120444167326\\
0.755463099816549	-0.0519779227044398\\
0.757426068431001	-0.0604804738999169\\
0.75968457601542	-0.0689093389542497\\
0.762235870926212	-0.0772542485937369\\
0.765076844803523	-0.0855050358314172\\
0.768204036358303	-0.093651648353978\\
0.77161363558935	-0.10168416076895\\
0.775301488425208	-0.109592786697269\\
0.779263101785268	-0.117367890696473\\
0.78349364905389	-0.125\\
0.787987975960893	-0.132479816058301\\
0.792740606861239	-0.139798225867687\\
0.797745751406263	-0.146946313073118\\
0.802997311598319	-0.153915368831414\\
0.808488889220256	-0.160696902421635\\
0.814213793630651	-0.167282651589715\\
0.820165049915337	-0.173664592614749\\
0.826335407385251	-0.179834950084663\\
0.832717348410285	-0.185786206369349\\
0.839303097578365	-0.191511110779744\\
0.846084631168585	-0.197002688401681\\
0.853053686926882	-0.202254248593737\\
0.860201774132313	-0.20725939313876\\
0.867520183941699	-0.212012024039106\\
0.875	-0.21650635094611\\
0.882632109303527	-0.220736898214732\\
0.890407213302731	-0.224698511574792\\
0.89831583923105	-0.22838636441065\\
0.906348351646022	-0.231795963641697\\
0.914494964168583	-0.234923155196477\\
0.922745751406263	-0.237764129073788\\
0.93109066104575	-0.24031542398458\\
0.939519526100083	-0.242573931568999\\
0.94802207729556	-0.244536900183451\\
0.956587955583267	-0.246201938253052\\
0.965206724759984	-0.247567017185393\\
0.973867884183087	-0.248630473842068\\
0.982560881563969	-0.249391012564956\\
0.991275125824375	-0.249847706754774\\
1	-0.25\\
1.00872487417563	-0.249847706754774\\
1.01743911843603	-0.249391012564956\\
1.02613211581691	-0.248630473842068\\
1.03479327524002	-0.247567017185393\\
1.04341204441673	-0.246201938253052\\
1.05197792270444	-0.244536900183451\\
1.06048047389992	-0.242573931568999\\
1.06890933895425	-0.24031542398458\\
1.07725424859374	-0.237764129073788\\
1.08550503583142	-0.234923155196477\\
1.09365164835398	-0.231795963641697\\
1.10168416076895	-0.22838636441065\\
1.10959278669727	-0.224698511574792\\
1.11736789069647	-0.220736898214732\\
1.125	-0.21650635094611\\
1.1324798160583	-0.212012024039107\\
1.13979822586769	-0.207259393138761\\
1.14694631307312	-0.202254248593737\\
1.15391536883141	-0.19700268840168\\
1.16069690242163	-0.191511110779745\\
1.16728265158971	-0.185786206369349\\
1.17366459261475	-0.179834950084663\\
1.17983495008466	-0.173664592614749\\
1.18578620636935	-0.167282651589715\\
1.19151111077974	-0.160696902421635\\
1.19700268840168	-0.153915368831415\\
1.20225424859374	-0.146946313073118\\
1.20725939313876	-0.139798225867687\\
1.21201202403911	-0.132479816058301\\
1.21650635094611	-0.125\\
1.22073689821473	-0.117367890696473\\
1.22469851157479	-0.109592786697269\\
1.22838636441065	-0.10168416076895\\
1.2317959636417	-0.0936516483539781\\
1.23492315519648	-0.0855050358314172\\
1.23776412907379	-0.0772542485937369\\
1.24031542398458	-0.0689093389542499\\
1.242573931569	-0.060480473899917\\
1.24453690018345	-0.05197792270444\\
1.24620193825305	-0.0434120444167328\\
1.24756701718539	-0.0347932752400165\\
1.24863047384207	-0.0261321158169134\\
1.24939101256496	-0.0174391184360312\\
1.24984770675477	-0.00872487417562521\\
1.25	-6.12323399573677e-17\\
};
\addplot [color=white!85!black, forget plot]
  table[row sep=crcr]{%
1.17	0\\
1.16989644059325	0.00593291443942517\\
1.16958588854417	0.0118586005365013\\
1.16906872221261	0.0177698387555011\\
1.16834557168607	0.0236594271632111\\
1.16741731801208	0.0295201902033782\\
1.16628509212475	0.0353449874390191\\
1.16495027346692	0.0411267222519435\\
1.16341448830951	0.0468583504888899\\
1.16167960777018	0.0525328890437411\\
1.1597477455336	0.0581434243653637\\
1.15762125527635	0.063683120880705\\
1.15530272779924	0.069145229322886\\
1.15279498787086	0.0745230949541432\\
1.15010109078602	0.0798101656736014\\
1.14722431864335	0.085\\
1.14416817634659	0.0900862749196448\\
1.14093638733436	0.095062793590027\\
1.13753288904374	0.0999234928897204\\
1.13396182811314	0.104662450805362\\
1.13022755533023	0.109273893646712\\
1.12633462033116	0.113752203081006\\
1.12228776605757	0.11809192297803\\
1.11809192297803	0.122287766057571\\
1.11375220308101	0.126334620331157\\
1.10927389364671	0.130227555330226\\
1.10466245080536	0.133961828113143\\
1.09992349288972	0.137532889043741\\
1.09506279359003	0.140936387334357\\
1.09008627491964	0.144168176346592\\
1.085	0.147224318643355\\
1.0798101656736	0.150101090786018\\
1.07452309495414	0.152794987870858\\
1.06914522932289	0.155302727799242\\
1.0636831208807	0.157621255276354\\
1.05814342436536	0.159747745533604\\
1.05253288904374	0.161679607770176\\
1.04685835048889	0.163414488309514\\
1.04112672225194	0.164950273466919\\
1.03534498743902	0.166285092124747\\
1.02952019020338	0.167417318012075\\
1.02365942716321	0.168345571686067\\
1.0177698387555	0.169068722212606\\
1.0118586005365	0.16958588854417\\
1.00593291443943	0.169896440593246\\
1	0.17\\
0.994067085560575	0.169896440593246\\
0.988141399463499	0.16958588854417\\
0.982230161244499	0.169068722212606\\
0.976340572836789	0.168345571686067\\
0.970479809796622	0.167417318012075\\
0.964655012560981	0.166285092124747\\
0.958873277748056	0.164950273466919\\
0.95314164951111	0.163414488309514\\
0.947467110956259	0.161679607770176\\
0.941856575634636	0.159747745533604\\
0.936316879119295	0.157621255276354\\
0.930854770677114	0.155302727799242\\
0.925476905045857	0.152794987870858\\
0.920189834326399	0.150101090786018\\
0.915	0.147224318643355\\
0.909913725080355	0.144168176346592\\
0.904937206409973	0.140936387334357\\
0.90007650711028	0.137532889043741\\
0.895337549194638	0.133961828113143\\
0.890726106353288	0.130227555330226\\
0.886247796918994	0.126334620331157\\
0.88190807702197	0.122287766057571\\
0.877712233942429	0.11809192297803\\
0.873665379668843	0.113752203081006\\
0.869772444669774	0.109273893646712\\
0.866038171886857	0.104662450805362\\
0.862467110956259	0.0999234928897205\\
0.859063612665643	0.095062793590027\\
0.855831823653408	0.0900862749196448\\
0.852775681356645	0.085\\
0.849898909213982	0.0798101656736015\\
0.847205012129142	0.0745230949541432\\
0.844697272200758	0.0691452293228861\\
0.842378744723646	0.0636831208807051\\
0.840252254466396	0.0581434243653637\\
0.838320392229824	0.0525328890437411\\
0.836585511690486	0.0468583504888899\\
0.835049726533081	0.0411267222519435\\
0.833714907875253	0.0353449874390191\\
0.832582681987925	0.0295201902033781\\
0.831654428313933	0.0236594271632112\\
0.830931277787394	0.0177698387555011\\
0.83041411145583	0.0118586005365013\\
0.830103559406754	0.00593291443942512\\
0.83	2.0818995585505e-17\\
0.830103559406754	-0.00593291443942515\\
0.83041411145583	-0.0118586005365012\\
0.830931277787394	-0.017769838755501\\
0.831654428313933	-0.0236594271632111\\
0.832582681987925	-0.0295201902033782\\
0.833714907875253	-0.035344987439019\\
0.835049726533081	-0.0411267222519435\\
0.836585511690486	-0.0468583504888898\\
0.838320392229824	-0.0525328890437411\\
0.840252254466396	-0.0581434243653637\\
0.842378744723646	-0.063683120880705\\
0.844697272200758	-0.069145229322886\\
0.847205012129142	-0.0745230949541431\\
0.849898909213982	-0.0798101656736015\\
0.852775681356645	-0.085\\
0.855831823653408	-0.0900862749196448\\
0.859063612665643	-0.0950627935900269\\
0.862467110956259	-0.0999234928897204\\
0.866038171886857	-0.104662450805362\\
0.869772444669774	-0.109273893646712\\
0.873665379668843	-0.113752203081006\\
0.877712233942429	-0.11809192297803\\
0.88190807702197	-0.122287766057571\\
0.886247796918994	-0.126334620331157\\
0.890726106353288	-0.130227555330226\\
0.895337549194638	-0.133961828113143\\
0.90007650711028	-0.137532889043741\\
0.904937206409973	-0.140936387334357\\
0.909913725080355	-0.144168176346592\\
0.915	-0.147224318643355\\
0.920189834326399	-0.150101090786018\\
0.925476905045857	-0.152794987870858\\
0.930854770677114	-0.155302727799242\\
0.936316879119295	-0.157621255276354\\
0.941856575634636	-0.159747745533604\\
0.947467110956259	-0.161679607770176\\
0.95314164951111	-0.163414488309514\\
0.958873277748056	-0.164950273466919\\
0.964655012560981	-0.166285092124747\\
0.970479809796622	-0.167417318012075\\
0.976340572836789	-0.168345571686067\\
0.982230161244499	-0.169068722212606\\
0.988141399463499	-0.16958588854417\\
0.994067085560575	-0.169896440593246\\
1	-0.17\\
1.00593291443943	-0.169896440593246\\
1.0118586005365	-0.16958588854417\\
1.0177698387555	-0.169068722212606\\
1.02365942716321	-0.168345571686067\\
1.02952019020338	-0.167417318012075\\
1.03534498743902	-0.166285092124747\\
1.04112672225194	-0.164950273466919\\
1.04685835048889	-0.163414488309514\\
1.05253288904374	-0.161679607770176\\
1.05814342436536	-0.159747745533604\\
1.0636831208807	-0.157621255276354\\
1.06914522932289	-0.155302727799242\\
1.07452309495414	-0.152794987870858\\
1.0798101656736	-0.150101090786018\\
1.085	-0.147224318643355\\
1.09008627491964	-0.144168176346592\\
1.09506279359003	-0.140936387334357\\
1.09992349288972	-0.137532889043741\\
1.10466245080536	-0.133961828113143\\
1.10927389364671	-0.130227555330226\\
1.11375220308101	-0.126334620331157\\
1.11809192297803	-0.122287766057571\\
1.12228776605757	-0.11809192297803\\
1.12633462033116	-0.113752203081006\\
1.13022755533023	-0.109273893646712\\
1.13396182811314	-0.104662450805362\\
1.13753288904374	-0.0999234928897205\\
1.14093638733436	-0.0950627935900271\\
1.14416817634659	-0.090086274919645\\
1.14722431864335	-0.0850000000000001\\
1.15010109078602	-0.0798101656736014\\
1.15279498787086	-0.0745230949541431\\
1.15530272779924	-0.069145229322886\\
1.15762125527635	-0.0636831208807051\\
1.1597477455336	-0.0581434243653637\\
1.16167960777018	-0.0525328890437411\\
1.16341448830951	-0.04685835048889\\
1.16495027346692	-0.0411267222519435\\
1.16628509212475	-0.0353449874390192\\
1.16741731801208	-0.0295201902033783\\
1.16834557168607	-0.0236594271632112\\
1.16906872221261	-0.0177698387555011\\
1.16958588854417	-0.0118586005365012\\
1.16989644059325	-0.00593291443942514\\
1.17	-4.163799117101e-17\\
};
\addplot [color=white!85!black, forget plot]
  table[row sep=crcr]{%
1.09	0\\
1.08994517443172	0.00314095470322509\\
1.08978076452338	0.00627808263697128\\
1.08950697058314	0.00940756169408881\\
1.08912412618674	0.0125255790864059\\
1.0886326977711	0.0156283359900237\\
1.08803328406604	0.0187120521735983\\
1.08732661536484	0.0217729706039701\\
1.08651355263445	0.0248073620235299\\
1.08559508646656	0.0278115294937453\\
1.08457233587073	0.0307818128993102\\
1.08344654691101	0.0337145934074321\\
1.08221909118783	0.036606297876822\\
1.08089146416692	0.039453403211017\\
1.0794652833573	0.0422524406507302\\
1.0779422863406	0.045\\
1.07632432865408	0.0476927337809884\\
1.07461338152995	0.0503273613123672\\
1.07281152949375	0.0529006727063226\\
1.0709209678246	0.0554095327793092\\
1.06894399988071	0.0578508848717885\\
1.06688303429297	0.0602217545722972\\
1.06474058203048	0.0625192533413098\\
1.06251925334131	0.0647405820304786\\
1.0602217545723	0.0668830342929655\\
1.05785088487179	0.068943999880708\\
1.05540953277931	0.070920967824605\\
1.05290067270632	0.0728115294937453\\
1.05032736131237	0.0746133815299538\\
1.04769273378099	0.0763243286540783\\
1.045	0.0779422863405995\\
1.04225244065073	0.0794652833573034\\
1.03945340321102	0.080891464166925\\
1.03660629787682	0.0822190911878341\\
1.03371459340743	0.0834465469110109\\
1.03078181289931	0.0845723358707317\\
1.02781152949375	0.0855950864665638\\
1.02480736202353	0.0865135526344487\\
1.02177297060397	0.0873266153648397\\
1.0187120521736	0.0880332840660425\\
1.01562833599002	0.0886326977710987\\
1.01252557908641	0.0891241261867413\\
1.00940756169409	0.0895069705831446\\
1.00627808263697	0.0897807645233842\\
1.00314095470323	0.0899451744317186\\
1	0.09\\
0.996859045296775	0.0899451744317186\\
0.993721917363029	0.0897807645233842\\
0.990592438305911	0.0895069705831446\\
0.987474420913594	0.0891241261867413\\
0.984371664009976	0.0886326977710987\\
0.981287947826402	0.0880332840660425\\
0.97822702939603	0.0873266153648397\\
0.97519263797647	0.0865135526344487\\
0.972188470506255	0.0855950864665638\\
0.96921818710069	0.0845723358707317\\
0.966285406592568	0.0834465469110109\\
0.963393702123178	0.0822190911878341\\
0.960546596788983	0.080891464166925\\
0.95774755934927	0.0794652833573034\\
0.955	0.0779422863405995\\
0.952307266219012	0.0763243286540783\\
0.949672638687633	0.0746133815299538\\
0.947099327293677	0.0728115294937453\\
0.944590467220691	0.070920967824605\\
0.942149115128211	0.068943999880708\\
0.939778245427703	0.0668830342929655\\
0.93748074665869	0.0647405820304786\\
0.935259417969521	0.0625192533413097\\
0.933116965707035	0.0602217545722972\\
0.931056000119292	0.0578508848717885\\
0.929079032175395	0.0554095327793093\\
0.927188470506255	0.0529006727063226\\
0.925386618470046	0.0503273613123672\\
0.923675671345922	0.0476927337809884\\
0.9220577136594	0.045\\
0.920534716642697	0.0422524406507302\\
0.919108535833075	0.039453403211017\\
0.917780908812166	0.036606297876822\\
0.916553453088989	0.0337145934074321\\
0.915427664129268	0.0307818128993102\\
0.914404913533436	0.0278115294937453\\
0.913486447365551	0.02480736202353\\
0.91267338463516	0.0217729706039701\\
0.911966715933958	0.0187120521735983\\
0.911367302228901	0.0156283359900237\\
0.910875873813259	0.0125255790864059\\
0.910493029416855	0.00940756169408884\\
0.910219235476616	0.0062780826369713\\
0.910054825568281	0.00314095470322506\\
0.91	1.10218211923262e-17\\
0.910054825568281	-0.00314095470322508\\
0.910219235476616	-0.00627808263697123\\
0.910493029416855	-0.00940756169408877\\
0.910875873813259	-0.0125255790864059\\
0.911367302228901	-0.0156283359900237\\
0.911966715933958	-0.0187120521735983\\
0.91267338463516	-0.0217729706039701\\
0.913486447365551	-0.0248073620235299\\
0.914404913533436	-0.0278115294937453\\
0.915427664129268	-0.0307818128993102\\
0.916553453088989	-0.0337145934074321\\
0.917780908812166	-0.036606297876822\\
0.919108535833075	-0.0394534032110169\\
0.920534716642697	-0.0422524406507302\\
0.922057713659401	-0.045\\
0.923675671345922	-0.0476927337809884\\
0.925386618470046	-0.0503273613123672\\
0.927188470506255	-0.0529006727063226\\
0.929079032175395	-0.0554095327793092\\
0.931056000119292	-0.0578508848717885\\
0.933116965707035	-0.0602217545722972\\
0.935259417969521	-0.0625192533413098\\
0.93748074665869	-0.0647405820304786\\
0.939778245427703	-0.0668830342929655\\
0.942149115128211	-0.068943999880708\\
0.944590467220691	-0.070920967824605\\
0.947099327293677	-0.0728115294937453\\
0.949672638687633	-0.0746133815299537\\
0.952307266219012	-0.0763243286540783\\
0.955	-0.0779422863405995\\
0.95774755934927	-0.0794652833573034\\
0.960546596788983	-0.080891464166925\\
0.963393702123178	-0.0822190911878341\\
0.966285406592568	-0.0834465469110109\\
0.96921818710069	-0.0845723358707317\\
0.972188470506255	-0.0855950864665638\\
0.97519263797647	-0.0865135526344487\\
0.97822702939603	-0.0873266153648397\\
0.981287947826402	-0.0880332840660425\\
0.984371664009976	-0.0886326977710987\\
0.987474420913594	-0.0891241261867413\\
0.990592438305911	-0.0895069705831446\\
0.993721917363029	-0.0897807645233842\\
0.996859045296775	-0.0899451744317186\\
1	-0.09\\
1.00314095470323	-0.0899451744317186\\
1.00627808263697	-0.0897807645233842\\
1.00940756169409	-0.0895069705831446\\
1.01252557908641	-0.0891241261867413\\
1.01562833599002	-0.0886326977710987\\
1.0187120521736	-0.0880332840660425\\
1.02177297060397	-0.0873266153648397\\
1.02480736202353	-0.0865135526344487\\
1.02781152949375	-0.0855950864665638\\
1.03078181289931	-0.0845723358707318\\
1.03371459340743	-0.0834465469110109\\
1.03660629787682	-0.0822190911878341\\
1.03945340321102	-0.080891464166925\\
1.04225244065073	-0.0794652833573034\\
1.045	-0.0779422863405995\\
1.04769273378099	-0.0763243286540783\\
1.05032736131237	-0.0746133815299538\\
1.05290067270632	-0.0728115294937453\\
1.05540953277931	-0.070920967824605\\
1.05785088487179	-0.068943999880708\\
1.0602217545723	-0.0668830342929655\\
1.06251925334131	-0.0647405820304787\\
1.06474058203048	-0.0625192533413098\\
1.06688303429297	-0.0602217545722972\\
1.06894399988071	-0.0578508848717886\\
1.0709209678246	-0.0554095327793093\\
1.07281152949375	-0.0529006727063226\\
1.07461338152995	-0.0503273613123673\\
1.07632432865408	-0.0476927337809885\\
1.0779422863406	-0.045\\
1.0794652833573	-0.0422524406507302\\
1.08089146416692	-0.0394534032110169\\
1.08221909118783	-0.036606297876822\\
1.08344654691101	-0.0337145934074321\\
1.08457233587073	-0.0307818128993102\\
1.08559508646656	-0.0278115294937453\\
1.08651355263445	-0.02480736202353\\
1.08732661536484	-0.0217729706039701\\
1.08803328406604	-0.0187120521735984\\
1.0886326977711	-0.0156283359900238\\
1.08912412618674	-0.0125255790864059\\
1.08950697058314	-0.00940756169408881\\
1.08978076452338	-0.00627808263697123\\
1.08994517443172	-0.00314095470322507\\
1.09	-2.20436423846524e-17\\
};
\addplot [color=green, draw=none, mark size=0.8pt, mark=*, mark options={solid, green!75!black}, forget plot]
  table[row sep=crcr]{%
2	0\\
1.9978589232386	0.0654031292301431\\
1.99144486137381	0.130526192220052\\
1.98078528040323	0.195090322016128\\
1.96592582628907	0.258819045102521\\
1.94693012949511	0.321439465303162\\
1.92387953251129	0.38268343236509\\
1.89687274153269	0.442288690219001\\
1.86602540378444	0.5\\
1.83146961230255	0.555570233019602\\
1.79335334029124	0.608761429008721\\
1.75183980747898	0.659345815100069\\
1.70710678118655	0.707106781186547\\
1.65934581510007	0.751839807478977\\
1.60876142900872	0.793353340291235\\
1.5555702330196	0.831469612302545\\
1.5	0.866025403784439\\
1.442288690219	0.896872741532688\\
1.38268343236509	0.923879532511287\\
1.32143946530316	0.946930129495106\\
1.25881904510252	0.965925826289068\\
1.19509032201613	0.98078528040323\\
1.13052619222005	0.99144486137381\\
1.06540312923014	0.997858923238603\\
1	1\\
0.934596870769857	0.997858923238603\\
0.869473807779948	0.99144486137381\\
0.804909677983872	0.98078528040323\\
0.741180954897479	0.965925826289068\\
0.678560534696838	0.946930129495106\\
0.61731656763491	0.923879532511287\\
0.557711309780999	0.896872741532688\\
0.5	0.866025403784439\\
0.444429766980398	0.831469612302545\\
0.391238570991279	0.793353340291235\\
0.340654184899931	0.751839807478977\\
0.292893218813453	0.707106781186548\\
0.248160192521023	0.659345815100069\\
0.206646659708765	0.608761429008721\\
0.168530387697455	0.555570233019603\\
0.133974596215561	0.5\\
0.103127258467312	0.442288690219002\\
0.0761204674887133	0.38268343236509\\
0.0530698705048944	0.321439465303162\\
0.0340741737109318	0.258819045102521\\
0.0192147195967696	0.195090322016128\\
0.00855513862618962	0.130526192220052\\
0.00214107676139652	0.0654031292301431\\
0	1.22464679914735e-16\\
0.00214107676139652	-0.0654031292301429\\
0.00855513862618962	-0.130526192220052\\
0.0192147195967695	-0.195090322016128\\
0.0340741737109317	-0.258819045102521\\
0.0530698705048943	-0.321439465303162\\
0.0761204674887132	-0.38268343236509\\
0.103127258467312	-0.442288690219001\\
0.133974596215561	-0.5\\
0.168530387697455	-0.555570233019602\\
0.206646659708765	-0.608761429008721\\
0.248160192521023	-0.659345815100069\\
0.292893218813452	-0.707106781186547\\
0.340654184899931	-0.751839807478977\\
0.391238570991279	-0.793353340291235\\
0.444429766980398	-0.831469612302545\\
0.5	-0.866025403784438\\
0.557711309780999	-0.896872741532688\\
0.61731656763491	-0.923879532511287\\
0.678560534696838	-0.946930129495106\\
0.741180954897479	-0.965925826289068\\
0.804909677983871	-0.98078528040323\\
0.869473807779948	-0.99144486137381\\
0.934596870769857	-0.997858923238603\\
1	-1\\
1.06540312923014	-0.997858923238604\\
1.13052619222005	-0.99144486137381\\
1.19509032201613	-0.98078528040323\\
1.25881904510252	-0.965925826289068\\
1.32143946530316	-0.946930129495106\\
1.38268343236509	-0.923879532511287\\
1.442288690219	-0.896872741532688\\
1.5	-0.866025403784439\\
1.5555702330196	-0.831469612302545\\
1.60876142900872	-0.793353340291236\\
1.65934581510007	-0.751839807478977\\
1.70710678118655	-0.707106781186548\\
1.75183980747898	-0.659345815100069\\
1.79335334029123	-0.608761429008721\\
1.83146961230254	-0.555570233019603\\
1.86602540378444	-0.5\\
1.89687274153269	-0.442288690219001\\
1.92387953251129	-0.38268343236509\\
1.94693012949511	-0.321439465303162\\
1.96592582628907	-0.258819045102522\\
1.98078528040323	-0.195090322016129\\
1.99144486137381	-0.130526192220052\\
1.9978589232386	-0.0654031292301428\\
1.82955560760538	-0.027156838742074\\
1.82600332313792	0.0813542264735355\\
1.80831789280019	0.18847329831853\\
1.77680191920858	0.292367539774624\\
1.73199464940913	0.391259291565578\\
1.67466274821104	0.483456488400276\\
1.6057871803392	0.567381610678983\\
1.52654642585583	0.641598676291072\\
1.43829631603981	0.704837810667061\\
1.34254683473765	0.756016974684576\\
1.2409362821212	0.794260478657733\\
1.13520324291751	0.818913965630449\\
1.02715683874207	0.829555607605383\\
0.918645773526464	0.826003323137923\\
0.81152670168147	0.808317892800187\\
0.707632460225376	0.77680191920858\\
0.608740708434422	0.731994649409135\\
0.516543511599724	0.67466274821104\\
0.432618389321017	0.605787180339203\\
0.358401323708928	0.526546425855826\\
0.295162189332939	0.438296316039805\\
0.243983025315424	0.342546834737648\\
0.205739521342267	0.240936282121204\\
0.181086034369551	0.135203242917509\\
0.170444392394617	0.0271568387420741\\
0.173996676862077	-0.0813542264735357\\
0.191682107199813	-0.18847329831853\\
0.22319808079142	-0.292367539774624\\
0.268005350590865	-0.391259291565578\\
0.32533725178896	-0.483456488400276\\
0.394212819660796	-0.567381610678983\\
0.473453574144175	-0.641598676291072\\
0.561703683960195	-0.704837810667061\\
0.657453165262353	-0.756016974684576\\
0.759063717878797	-0.794260478657734\\
0.864796757082492	-0.81891396563045\\
0.972843161257926	-0.829555607605383\\
1.08135422647354	-0.826003323137923\\
1.18847329831853	-0.808317892800187\\
1.29236753977462	-0.77680191920858\\
1.39125929156558	-0.731994649409134\\
1.48345648840028	-0.674662748211041\\
1.56738161067898	-0.605787180339203\\
1.64159867629107	-0.526546425855826\\
1.70483781066706	-0.438296316039805\\
1.75601697468458	-0.342546834737647\\
1.79426047865773	-0.240936282121204\\
1.81891396563045	-0.135203242917508\\
1.69662930867054	-0.068611998230692\\
1.69662930867054	0.0686119982306928\\
1.66985823501255	0.203199274078124\\
1.61734488504385	0.329977715778199\\
1.54110731735392	0.444075298914552\\
1.44407529891455	0.541107317353916\\
1.3299777157782	0.617344885043849\\
1.20319927407812	0.669858235012546\\
1.06861199823069	0.696629308670538\\
0.931388001769307	0.696629308670538\\
0.796800725921876	0.669858235012546\\
0.670022284221801	0.617344885043848\\
0.555924701085448	0.541107317353916\\
0.458892682646084	0.444075298914551\\
0.382655114956151	0.329977715778198\\
0.330141764987454	0.203199274078123\\
0.303370691329462	0.068611998230692\\
0.303370691329462	-0.068611998230693\\
0.330141764987454	-0.203199274078124\\
0.382655114956152	-0.329977715778199\\
0.458892682646084	-0.444075298914552\\
0.555924701085448	-0.541107317353916\\
0.670022284221802	-0.617344885043848\\
0.796800725921876	-0.669858235012546\\
0.931388001769308	-0.696629308670538\\
1.06861199823069	-0.696629308670538\\
1.20319927407812	-0.669858235012546\\
1.3299777157782	-0.617344885043849\\
1.44407529891455	-0.541107317353916\\
1.54110731735392	-0.444075298914552\\
1.61734488504385	-0.329977715778199\\
1.66985823501255	-0.203199274078124\\
1.56885546263387	-0.113152386769354\\
1.58	-6.43929354282591e-17\\
1.56885546263387	0.113152386769354\\
1.53585012885655	0.221956390771752\\
1.48225237513548	0.322230735151369\\
1.4101219330882	0.410121933088197\\
1.32223073515137	0.482252375135476\\
1.22195639077175	0.535850128856546\\
1.11315238676935	0.568855462633874\\
1	0.58\\
0.886847613230646	0.568855462633874\\
0.778043609228248	0.535850128856546\\
0.677769264848631	0.482252375135476\\
0.589878066911802	0.410121933088198\\
0.517747624864524	0.322230735151369\\
0.464149871143454	0.221956390771752\\
0.431144537366126	0.113152386769355\\
0.42	7.10295143505465e-17\\
0.431144537366126	-0.113152386769354\\
0.464149871143454	-0.221956390771752\\
0.517747624864524	-0.322230735151369\\
0.589878066911802	-0.410121933088197\\
0.67776926484863	-0.482252375135476\\
0.778043609228248	-0.535850128856546\\
0.886847613230645	-0.568855462633874\\
1	-0.58\\
1.11315238676935	-0.568855462633874\\
1.22195639077175	-0.535850128856546\\
1.32223073515137	-0.482252375135476\\
1.4101219330882	-0.410121933088198\\
1.48225237513548	-0.32223073515137\\
1.53585012885655	-0.221956390771752\\
1.44019255443682	-0.133530951537053\\
1.45778497426921	0.0450878845515976\\
1.40568378160024	0.216842498939959\\
1.29182091071528	0.355584808546859\\
1.13353095153705	0.440192554436816\\
0.954912115448402	0.457784974269211\\
0.783157501060041	0.405683781600243\\
0.644415191453141	0.291820910715277\\
0.559807445563184	0.133530951537053\\
0.542215025730789	-0.0450878845515975\\
0.594316218399756	-0.216842498939959\\
0.708179089284723	-0.355584808546859\\
0.866469048462947	-0.440192554436816\\
1.0450878845516	-0.457784974269211\\
1.21684249893996	-0.405683781600244\\
1.35558480854686	-0.291820910715277\\
1.30867244252192	-0.1649888578891\\
1.34831465433527	-0.0343059991153467\\
1.33492911750627	0.101599637039061\\
1.27055365867696	0.222037649457276\\
1.1649888578891	0.308672442521924\\
1.03430599911535	0.348314654335269\\
0.898400362960939	0.334929117506273\\
0.777962350542725	0.270553658676958\\
0.691327557478076	0.1649888578891\\
0.651685345664731	0.0343059991153467\\
0.665070882493727	-0.101599637039061\\
0.729446341323042	-0.222037649457275\\
0.8350111421109	-0.308672442521924\\
0.965694000884653	-0.348314654335269\\
1.10159963703906	-0.334929117506273\\
1.22203764945728	-0.270553658676958\\
1.19325261334068	-0.158598321040912\\
1.24879618166805	0.0245042850823895\\
1.15859832104091	0.193252613340684\\
0.97549571491761	0.248796181668049\\
0.806747386659316	0.158598321040912\\
0.751203818331951	-0.0245042850823895\\
0.841401678959088	-0.193252613340684\\
1.02450428508239	-0.248796181668049\\
1.08013744526042	-0.149926614939221\\
1.14992661493922	0.0801374452604191\\
0.919862554739581	0.149926614939221\\
0.850073385060779	-0.0801374452604191\\
0.973874379047098	-0.0861246302158987\\
1.0861246302159	-0.0261256209529019\\
1.0261256209529	0.0861246302158987\\
0.913875369784101	0.026125620952902\\
};

% For inductive RFID: would need C smaller than half of C_res
\addplot [color=magenta, draw=none, mark size=2.2pt, mark=o, mark options={solid, magenta}, forget plot]
  table[row sep=crcr]{%
%0.0192147195967696	0.195090322016128\\
0.00855513862618962	0.130526192220052\\
0.00214107676139652	0.0654031292301431\\
};

% For inductive RFID: would need C larger than 1.5 * C_res
\addplot [color=magenta, draw=none, mark size=2.2pt, mark=o, mark options={solid, magenta}, forget plot]
  table[row sep=crcr]{%
0.00214107676139652	-0.0654031292301429\\
0.00855513862618962	-0.130526192220052\\
0.0192147195967695	-0.195090322016128\\
0.0340741737109317	-0.258819045102521\\
0.0530698705048943	-0.321439465303162\\
0.0761204674887132	-0.38268343236509\\
0.191682107199813	-0.18847329831853\\
};

\node[anchor=west,color=magenta] (text) at (axis cs:-0.1,-1.22){\begin{tabular}{l}
unrealizable when load has\\ constrained value range
\end{tabular}};
\node (source) at (axis cs:0.12,-1.12){};
\node (destination) at (axis cs:0.05,-.45){};
\draw[-{Latex[length=2mm, width=1mm]},color=magenta](source)--(destination);

\end{axis}
\end{tikzpicture}%\vspace{-1mm}}
\label{fig:Constellation_i}}\!\!\!\!
\subfloat[load impedance alphabet]{
\resizebox{!}{60mm}{% This file was created by matlab2tikz.
%
%The latest updates can be retrieved from
%  http://www.mathworks.com/matlabcentral/fileexchange/22022-matlab2tikz-matlab2tikz
%where you can also make suggestions and rate matlab2tikz.
%
\begin{tikzpicture}

\begin{axis}[%
width=1.1158in,
height=2.0601in,
at={(0,0)},
scale only axis,
unbounded coords=jump,
xmin=-0.3,
xmax=2.3,
ymin=-2.4,
ymax=2.4,
%xmin=-0.1667,
%xmax=2,
%ymin=-2,
%ymax=2,
xlabel style={font=\color{white!15!black},at={(0.51,-0.065)},anchor=north},
xlabel={Re(z)},
xtick={0, 1, 2},
ytick={-2, -1,  0,  1, 2},
ylabel style={font=\color{white!15!black},at={(-0.09,0.6)}},
ylabel={Im(z)},
axis background/.style={fill=white},
xmajorgrids,
ymajorgrids,
legend style={legend cell align=left, align=left, draw=white!15!black}
]
\addplot [color=white!85!black, forget plot]
  table[row sep=crcr]{%
0	0\\
0	-0.0174550649282176\\
0	-0.0349207694917477\\
0	-0.0524077792830412\\
0	-0.0699268119435104\\
0	-0.087488663525924\\
0	-0.105104235265676\\
0	-0.122784560902905\\
0	-0.140540834702391\\
0	-0.158384440324536\\
0	-0.176326980708465\\
0	-0.194380309137718\\
2.22044604925031e-16	-0.212556561670022\\
0	-0.230868191125563\\
0	-0.249328002843181\\
0	-0.267949192431123\\
2.22044604925031e-16	-0.286745385758808\\
0	-0.30573068145866\\
0	-0.324919696232906\\
0	-0.344327613289665\\
2.22044604925031e-16	-0.363970234266202\\
0	-0.383864035035416\\
0	-0.404026225835157\\
0	-0.424474816209605\\
0	-0.445228685308536\\
0	-0.466307658154999\\
0	-0.487732588565861\\
0	-0.509525449494429\\
0	-0.531709431661479\\
0	-0.554309051452769\\
0	-0.577350269189626\\
0	-0.60086061902756\\
0	-0.624869351909328\\
0	-0.649407593197511\\
0	-0.674508516842427\\
0	-0.70020753820971\\
0	-0.726542528005361\\
0	-0.753554050102794\\
0	-0.781285626506717\\
0	-0.809784033195007\\
0	-0.83909963117728\\
0	-0.869286737816226\\
0	-0.90040404429784\\
0	-0.932515086137662\\
0	-0.965688774807074\\
0	-1\\
-1.11022302462516e-16	-1.03553031379057\\
2.22044604925031e-16	-1.07236871002468\\
0	-1.11061251482919\\
-1.11022302462516e-16	-1.15036840722101\\
0	-1.19175359259421\\
0	-1.23489715653505\\
-1.11022302462516e-16	-1.27994163219308\\
0	-1.32704482162041\\
-1.11022302462516e-16	-1.37638192047117\\
0	-1.42814800674211\\
0	-1.48256096851274\\
0	-1.53986496381458\\
0	-1.60033452904105\\
0	-1.66427948235052\\
0	-1.73205080756888\\
0	-1.80404775527142\\
0	-1.88072646534633\\
0	-1.96261050550515\\
-1.11022302462516e-16	-2.0503038415793\\
0	-2.14450692050956\\
0	-2.24603677390422\\
-1.11022302462516e-16	-2.35585236582375\\
0	-2.4750868534163\\
0	-2.6050890646938\\
2.22044604925031e-16	-2.74747741945462\\
-1.11022302462516e-16	-2.90421087767582\\
0	-3.07768353717525\\
2.22044604925031e-16	-3.27085261848414\\
2.22044604925031e-16	-3.48741444384091\\
0	-3.73205080756888\\
2.22044604925031e-16	-4.01078093353584\\
2.22044604925031e-16	-4.33147587428416\\
4.44089209850063e-16	-4.70463010947845\\
2.22044604925031e-16	-5.14455401597031\\
2.22044604925031e-16	-5.67128181961771\\
4.44089209850063e-16	-6.31375151467504\\
1.33226762955019e-15	-7.1153697223842\\
-1.11022302462516e-16	-8.14434642797459\\
-1.99840144432528e-15	-9.51436445422259\\
3.33066907387547e-15	-11.4300523027613\\
2.22044604925031e-15	-14.3006662567119\\
3.5527136788005e-15	-19.0811366877282\\
1.4210854715202e-14	-28.6362532829155\\
-3.71924713249427e-14	-57.2899616307599\\
nan	0\\
-4.87387907810444e-14	57.2899616307595\\
-1.15463194561016e-14	28.6362532829158\\
-3.66373598126302e-15	19.0811366877283\\
5.32907051820075e-15	14.3006662567119\\
1.33226762955019e-15	11.4300523027613\\
4.44089209850063e-16	9.5143644542226\\
-1.88737914186277e-15	8.1443464279746\\
6.66133814775094e-16	7.11536972238421\\
-9.99200722162641e-16	6.31375151467504\\
-1.11022302462516e-16	5.67128181961771\\
-3.33066907387547e-16	5.14455401597031\\
-2.22044604925031e-16	4.70463010947846\\
2.22044604925031e-16	4.33147587428416\\
2.22044604925031e-16	4.01078093353584\\
0	3.73205080756888\\
2.22044604925031e-16	3.48741444384091\\
-2.22044604925031e-16	3.27085261848414\\
2.22044604925031e-16	3.07768353717525\\
2.22044604925031e-16	2.90421087767583\\
0	2.74747741945462\\
0	2.6050890646938\\
0	2.4750868534163\\
2.22044604925031e-16	2.35585236582375\\
0	2.24603677390422\\
0	2.14450692050956\\
0	2.0503038415793\\
0	1.96261050550515\\
0	1.88072646534633\\
0	1.80404775527142\\
-2.22044604925031e-16	1.73205080756888\\
-2.22044604925031e-16	1.66427948235052\\
0	1.60033452904105\\
0	1.53986496381458\\
0	1.48256096851274\\
0	1.42814800674212\\
0	1.37638192047117\\
0	1.32704482162041\\
-1.11022302462516e-16	1.27994163219308\\
-1.11022302462516e-16	1.23489715653505\\
0	1.19175359259421\\
2.22044604925031e-16	1.15036840722101\\
0	1.11061251482919\\
0	1.07236871002468\\
2.22044604925031e-16	1.03553031379057\\
0	1\\
0	0.965688774807074\\
0	0.932515086137662\\
0	0.90040404429784\\
0	0.869286737816227\\
0	0.83909963117728\\
0	0.809784033195008\\
0	0.781285626506718\\
0	0.753554050102794\\
0	0.726542528005361\\
0	0.70020753820971\\
0	0.674508516842427\\
0	0.64940759319751\\
0	0.624869351909328\\
0	0.600860619027561\\
0	0.577350269189626\\
0	0.554309051452769\\
0	0.531709431661479\\
0	0.509525449494429\\
0	0.487732588565861\\
0	0.466307658154999\\
0	0.445228685308536\\
0	0.424474816209605\\
0	0.404026225835157\\
0	0.383864035035416\\
0	0.363970234266203\\
0	0.344327613289666\\
0	0.324919696232906\\
0	0.305730681458661\\
0	0.286745385758808\\
0	0.267949192431123\\
0	0.249328002843181\\
0	0.230868191125563\\
0	0.212556561670022\\
0	0.194380309137719\\
0	0.176326980708465\\
0	0.158384440324536\\
0	0.140540834702392\\
0	0.122784560902905\\
0	0.105104235265677\\
0	0.0874886635259245\\
2.22044604925031e-16	0.0699268119435106\\
0	0.0524077792830412\\
0	0.0349207694917475\\
0	0.0174550649282175\\
0	1.22464679914735e-16\\
};
\addplot [color=white!85!black, forget plot]
  table[row sep=crcr]{%
0.0928961748633879	0\\
0.0929242340793204	-0.0173043874682864\\
0.0930084792643229	-0.0346190494354207\\
0.0931491133740723	-0.0519542844799809\\
0.093346475817897	-0.0693204394581452\\
0.0936010441941062	-0.0867279339388205\\
0.0939134367406476	-0.104187284997116\\
0.0942844155226323	-0.121709132490233\\
0.0947148903849062	-0.139304264943875\\
0.0952059237048746	-0.156983646182423\\
0.0957587359882799	-0.174758442842423\\
0.0963747123587413	-0.1926400529165\\
0.0970554100006449	-0.21064013548368\\
0.0978025666246207	-0.228770641792508\\
0.0986181100354775	-0.247043847875266\\
0.0995041688942513	-0.265472388885338\\
0.100463084779183	-0.284069295365409\\
0.101497425665167	-0.302848031672029\\
0.102610000957802	-0.32182253680226\\
0.103803878236859	-0.341007267891088\\
0.105082401885219	-0.360417246674213\\
0.106449213803392	-0.380068109240183\\
0.107908276437208	-0.399976159429091\\
0.109463898377632	-0.420158426272597\\
0.111120762827617	-0.440632725912625\\
0.112883959272193	-0.461417728484205\\
0.114759018735544	-0.482533030502606\\
0.116751953063696	-0.503999233356777\\
0.118869298734931	-0.525838028581534\\
0.121118165773682	-0.54807229066084\\
0.123506292429235	-0.570726178205633\\
0.126042106380326	-0.593825244453381\\
0.128734793343232	-0.617396558155023\\
0.131594374097383	-0.641468836050322\\
0.134631791102734	-0.666072588287666\\
0.137859006071691	-0.69124027832203\\
0.141289110080938	-0.717006499028841\\
0.144936448071905	-0.743408167006151\\
0.148816759901222	-0.770484737307809\\
0.152947340474577	-0.798278441162159\\
0.157347221941592	-0.826834549591131\\
0.162037381461288	-0.856201666261631\\
0.167040978686694	-0.886432053384504\\
0.172383627887389	-0.917581995037453\\
0.178093710560155	-0.949712202940524\\
0.184202735508319	-0.982888270471905\\
0.190745754747132	-1.01718118159632\\
0.197761845275717	-1.05266788241005\\
0.205294668821916	-1.08943192421082\\
0.213393124211853	-1.12756418840683\\
0.222112110165757	-1.16716370522138\\
0.231513420235663	-1.2083385800655\\
0.2416667964863	-1.25120704368485\\
0.252651174647097	-1.29589864478511\\
0.264556161182823	-1.34255560685138\\
0.27748379250524	-1.39133437434931\\
0.291550638989107	-1.44240737746862\\
0.306890332374032	-1.49596504906346\\
0.323656615612751	-1.55221813244105\\
0.342027040722491	-1.61140032406189\\
0.362207474676912	-1.67377130082916\\
0.384437618517587	-1.73962018703981\\
0.408997804325234	-1.80926952046835\\
0.436217413506038	-1.88307977909351\\
0.46648536498124	-1.96145452731893\\
0.500263263003183	-2.04484622923645\\
0.53810198498856	-2.13376274992215\\
0.580662748934347	-2.2287745129058\\
0.628744054333606	-2.33052218439028\\
0.683316377524968	-2.43972458169992\\
0.745567174603139	-2.55718620302048\\
0.816959675270921	-2.68380326027553\\
0.899310238575398	-2.82056621819545\\
0.994890816534536	-2.96855534553194\\
1.10656549238336	-3.12892322974095\\
1.23797328591861	-3.30285383256975\\
1.39377351865458	-3.49148014488623\\
1.57997473174728	-3.69572952809174\\
1.8043721801811	-3.91604342811807\\
2.07711842748108	-4.15187961645313\\
2.41143631003606	-4.40083977933\\
2.82442833394503	-4.65715830475088\\
3.33778282556612	-4.90912654988277\\
3.97779826577577	-5.13482926595128\\
4.77330390774541	-5.29548903622297\\
5.74842943708677	-5.32632314910675\\
6.90489311590568	-5.12767323042671\\
8.18820960335285	-4.56700612281641\\
9.44339042885261	-3.51496124525747\\
10.4007768874728	-1.93683676831165\\
10.7647058823529	-7.03430341378756e-15\\
10.4007768874728	1.93683676831166\\
9.44339042885263	3.51496124525745\\
8.18820960335288	4.56700612281639\\
6.90489311590569	5.12767323042671\\
5.74842943708676	5.32632314910675\\
4.77330390774541	5.29548903622297\\
3.97779826577578	5.13482926595129\\
3.33778282556613	4.90912654988277\\
2.82442833394502	4.65715830475088\\
2.41143631003606	4.40083977933\\
2.07711842748108	4.15187961645313\\
1.80437218018111	3.91604342811807\\
1.57997473174728	3.69572952809174\\
1.39377351865458	3.49148014488624\\
1.23797328591861	3.30285383256974\\
1.10656549238336	3.12892322974095\\
0.994890816534536	2.96855534553194\\
0.899310238575398	2.82056621819546\\
0.816959675270923	2.68380326027553\\
0.745567174603139	2.55718620302048\\
0.683316377524968	2.43972458169992\\
0.628744054333606	2.33052218439028\\
0.580662748934348	2.2287745129058\\
0.53810198498856	2.13376274992215\\
0.500263263003183	2.04484622923645\\
0.46648536498124	1.96145452731893\\
0.436217413506038	1.88307977909351\\
0.408997804325235	1.80926952046835\\
0.384437618517587	1.73962018703981\\
0.362207474676913	1.67377130082916\\
0.342027040722491	1.61140032406189\\
0.323656615612752	1.55221813244105\\
0.306890332374032	1.49596504906346\\
0.291550638989108	1.44240737746862\\
0.27748379250524	1.39133437434931\\
0.264556161182823	1.34255560685138\\
0.252651174647097	1.29589864478511\\
0.2416667964863	1.25120704368485\\
0.231513420235663	1.2083385800655\\
0.222112110165757	1.16716370522138\\
0.213393124211853	1.12756418840683\\
0.205294668821916	1.08943192421082\\
0.197761845275717	1.05266788241005\\
0.190745754747132	1.01718118159632\\
0.184202735508319	0.982888270471905\\
0.178093710560155	0.949712202940524\\
0.172383627887389	0.917581995037453\\
0.167040978686694	0.886432053384505\\
0.162037381461288	0.856201666261631\\
0.157347221941592	0.826834549591131\\
0.152947340474577	0.79827844116216\\
0.148816759901223	0.77048473730781\\
0.144936448071905	0.743408167006151\\
0.141289110080938	0.717006499028841\\
0.137859006071691	0.69124027832203\\
0.134631791102734	0.666072588287666\\
0.131594374097383	0.641468836050322\\
0.128734793343232	0.617396558155023\\
0.126042106380326	0.593825244453382\\
0.123506292429235	0.570726178205633\\
0.121118165773682	0.54807229066084\\
0.118869298734931	0.525838028581534\\
0.116751953063696	0.503999233356777\\
0.114759018735545	0.482533030502606\\
0.112883959272193	0.461417728484205\\
0.111120762827617	0.440632725912625\\
0.109463898377632	0.420158426272598\\
0.107908276437208	0.399976159429091\\
0.106449213803392	0.380068109240183\\
0.105082401885219	0.360417246674213\\
0.103803878236859	0.341007267891088\\
0.102610000957802	0.32182253680226\\
0.101497425665167	0.30284803167203\\
0.100463084779183	0.28406929536541\\
0.0995041688942513	0.265472388885338\\
0.0986181100354775	0.247043847875266\\
0.0978025666246207	0.228770641792508\\
0.0970554100006449	0.21064013548368\\
0.0963747123587415	0.1926400529165\\
0.0957587359882799	0.174758442842423\\
0.0952059237048746	0.156983646182423\\
0.0947148903849064	0.139304264943875\\
0.0942844155226321	0.121709132490233\\
0.0939134367406476	0.104187284997116\\
0.0936010441941062	0.086727933938821\\
0.093346475817897	0.0693204394581455\\
0.0931491133740723	0.0519542844799809\\
0.0930084792643229	0.0346190494354204\\
0.0929242340793204	0.0173043874682863\\
0.0928961748633879	1.2140784655168e-16\\
};
\addplot [color=white!85!black, forget plot]
  table[row sep=crcr]{%
0.176470588235294	0\\
0.176522680272067	-0.0169113211319614\\
0.176679077471975	-0.0338319866673218\\
0.176940143691687	-0.0507713604600582\\
0.17730648732347	-0.0677388453443164\\
0.177778964258808	-0.0847439028225076\\
0.178358682071455	-0.101796072992362\\
0.17904700545442	-0.11890499479484\\
0.179845562955953	-0.136080426666741\\
0.180756255070742	-0.153332267684324\\
0.181781263754378	-0.170670579287198\\
0.182923063441847	-0.188105607675322\\
0.184184433664552	-0.205647806975985\\
0.185568473375314	-0.223307863282371\\
0.187078617107208	-0.241096719670592\\
0.188718653110133	-0.259025602308026\\
0.190492743629048	-0.277106047772426\\
0.192405447510008	-0.295349931708615\\
0.194461745345025	-0.313769498957655\\
0.196667067394564	-0.332377395302257\\
0.199027324557801	-0.351186700981847\\
0.201548942696	-0.370210966141226\\
0.204238900654229	-0.389464248388089\\
0.207104772371764	-0.408961152646904\\
0.210154773522759	-0.42871687350968\\
0.213397813187101	-0.448747240298056\\
0.21684355111781	-0.469068765065773\\
0.220502461247311	-0.4896986937859\\
0.224385902161824	-0.510655060983012\\
0.228506195372787	-0.531956748086621\\
0.232876712328767	-0.553623545798271\\
0.237511971243092	-0.57567622078034\\
0.242427744964446	-0.598136586989191\\
0.247641181293201	-0.621027581988062\\
0.253170937349436	-0.644373348584925\\
0.259037329834106	-0.668199322146024\\
0.265262503298337	-0.692532323935113\\
0.271870618854103	-0.717400660819007\\
0.278888066130376	-0.742834231658732\\
0.286343701711988	-0.768864640668127\\
0.294269117804919	-0.79552531796252\\
0.302698945465215	-0.822851647432014\\
0.311671197425291	-0.850881101947331\\
0.321227656370161	-0.87965338572887\\
0.331414315480287	-0.909210583465365\\
0.342281879194631	-0.939597315436242\\
0.35388633349005	-0.970860897443779\\
0.366289596560616	-1.0030515037613\\
0.379560262659437	-1.03622233050424\\
0.393774454091766	-1.07042975576962\\
0.409016798987261	-1.10573349148401\\
0.42538155560874	-1.14219672004146\\
0.442973907665055	-1.17988620635853\\
0.461911459491316	-1.21887237273246\\
0.482325965159465	-1.25922931960894\\
0.504365331717945	-1.30103476971338\\
0.52819594397216	-1.34436990552571\\
0.554005366649542	-1.38931906019066\\
0.582005489571341	-1.43596920885532\\
0.612436192661469	-1.48440919005134\\
0.645569620253164	-1.53472856366863\\
0.68171516802733	-1.58701598139833\\
0.721225300550089	-1.64135690471893\\
0.764502331775035	-1.69783045118312\\
0.8120063131902	-1.75650507746391\\
0.864264181287679	-1.81743271146486\\
0.921880312257394	-1.8806408181745\\
0.985548608266636	-1.94612171516153\\
1.05606618184676	-2.01381823170575\\
1.13434858960967	-2.08360451651563\\
1.22144635737572	-2.15526042781376\\
1.3185621794694	-2.22843747335106\\
1.42706757965902	-2.30261370252471\\
1.54851686226186	-2.37703430665543\\
1.68465467496922	-2.45063402389295\\
1.83741119624329	-2.5219369360202\\
2.00887552721665	-2.58892922840514\\
2.20123296065471	-2.64890161615092\\
2.41664517759302	-2.69826158010845\\
2.65704429562471	-2.73232320979376\\
2.92380343125746	-2.74509712953071\\
3.21724184883905	-2.72912817417207\\
3.53592997176006	-2.67546694681967\\
3.87579275196896	-2.57391133590372\\
4.22908860276105	-2.41370146436359\\
4.58348085995945	-2.18485948663627\\
4.9216063495083	-1.88026922414147\\
5.22169972829057	-1.49831911282485\\
5.45978161072782	-1.04548462276147\\
5.61347325851817	-0.537784996206515\\
5.66666666666667	-1.90500613200699e-15\\
5.61347325851817	0.537784996206518\\
5.45978161072782	1.04548462276146\\
5.22169972829058	1.49831911282484\\
4.92160634950831	1.88026922414146\\
4.58348085995945	2.18485948663627\\
4.22908860276106	2.41370146436359\\
3.87579275196897	2.57391133590372\\
3.53592997176007	2.67546694681966\\
3.21724184883905	2.72912817417208\\
2.92380343125746	2.74509712953071\\
2.65704429562471	2.73232320979376\\
2.41664517759302	2.69826158010845\\
2.20123296065472	2.64890161615093\\
2.00887552721665	2.58892922840514\\
1.83741119624329	2.5219369360202\\
1.68465467496922	2.45063402389295\\
1.54851686226186	2.37703430665543\\
1.42706757965902	2.30261370252471\\
1.3185621794694	2.22843747335107\\
1.22144635737572	2.15526042781376\\
1.13434858960967	2.08360451651563\\
1.05606618184675	2.01381823170575\\
0.985548608266636	1.94612171516153\\
0.921880312257395	1.8806408181745\\
0.864264181287679	1.81743271146486\\
0.8120063131902	1.75650507746391\\
0.764502331775035	1.69783045118312\\
0.72122530055009	1.64135690471894\\
0.68171516802733	1.58701598139833\\
0.645569620253165	1.53472856366863\\
0.612436192661469	1.48440919005134\\
0.582005489571341	1.43596920885532\\
0.554005366649542	1.38931906019066\\
0.52819594397216	1.34436990552571\\
0.504365331717945	1.30103476971338\\
0.482325965159465	1.25922931960894\\
0.461911459491315	1.21887237273246\\
0.442973907665055	1.17988620635853\\
0.42538155560874	1.14219672004146\\
0.409016798987261	1.10573349148401\\
0.393774454091766	1.07042975576962\\
0.379560262659437	1.03622233050424\\
0.366289596560616	1.0030515037613\\
0.35388633349005	0.97086089744378\\
0.342281879194631	0.939597315436242\\
0.331414315480287	0.909210583465365\\
0.321227656370161	0.87965338572887\\
0.311671197425291	0.850881101947332\\
0.302698945465215	0.822851647432015\\
0.294269117804919	0.795525317962521\\
0.286343701711988	0.768864640668127\\
0.278888066130376	0.742834231658732\\
0.271870618854103	0.717400660819007\\
0.265262503298337	0.692532323935113\\
0.259037329834106	0.668199322146024\\
0.253170937349436	0.644373348584925\\
0.247641181293201	0.621027581988062\\
0.242427744964446	0.598136586989191\\
0.237511971243092	0.575676220780341\\
0.232876712328767	0.553623545798271\\
0.228506195372787	0.531956748086621\\
0.224385902161824	0.510655060983012\\
0.220502461247311	0.4896986937859\\
0.21684355111781	0.469068765065772\\
0.213397813187101	0.448747240298056\\
0.210154773522759	0.42871687350968\\
0.207104772371764	0.408961152646905\\
0.204238900654229	0.389464248388089\\
0.201548942696	0.370210966141226\\
0.199027324557801	0.351186700981848\\
0.196667067394564	0.332377395302257\\
0.194461745345025	0.313769498957655\\
0.192405447510008	0.295349931708615\\
0.190492743629048	0.277106047772426\\
0.188718653110133	0.259025602308026\\
0.187078617107208	0.241096719670592\\
0.185568473375314	0.22330786328237\\
0.184184433664552	0.205647806975985\\
0.182923063441847	0.188105607675323\\
0.181781263754378	0.170670579287198\\
0.180756255070742	0.153332267684324\\
0.179845562955953	0.136080426666742\\
0.17904700545442	0.11890499479484\\
0.178358682071455	0.101796072992362\\
0.177778964258808	0.0847439028225081\\
0.17730648732347	0.0677388453443166\\
0.176940143691687	0.0507713604600582\\
0.176679077471975	0.0338319866673216\\
0.176522680272067	0.0169113211319613\\
0.176470588235294	1.18650900955453e-16\\
};
\addplot [color=white!85!black, forget plot]
  table[row sep=crcr]{%
0.265822784810126	0\\
0.265898050942813	-0.0162213102115928\\
0.266124013410003	-0.0324504104803718\\
0.266501165145765	-0.0486951030922275\\
0.267030330134381	-0.0649632148137771\\
0.267712667069969	-0.0812626091908714\\
0.268549674516907	-0.0976011989164377\\
0.269543197608459	-0.11398695829003\\
0.270695436332347	-0.130427935790774\\
0.272008955463872	-0.146932266784504\\
0.273486696219659	-0.163508186384728\\
0.275131989718365	-0.180164042485595\\
0.276948572348816	-0.196908308983242\\
0.278940603161286	-0.213749599199617\\
0.281112683414095	-0.230696679520143\\
0.283469878425554	-0.247758483253196\\
0.286017741900838	-0.264944124715269\\
0.288762342924724	-0.282262913540667\\
0.291710295834605	-0.299724369208572\\
0.294868793214104	-0.317338235772939\\
0.298245642276169	-0.335114496771889\\
0.301849304936205	-0.353063390282575\\
0.305688941910884	-0.371195424074704\\
0.309774461217269	-0.389521390800437\\
0.314116571490255	-0.408052383139892\\
0.318726840584625	-0.42679980879921\\
0.323617759981813	-0.445775405231455\\
0.328802815581534	-0.464991253918625\\
0.334296565525388	-0.484459794014648\\
0.34011472577434	-0.504193835103263\\
0.34627426424546	-0.524206568769541\\
0.352793504406507	-0.544511578617706\\
0.359692239330959	-0.565122848288713\\
0.366991857332091	-0.586054766936049\\
0.374715480424017	-0.607322131504291\\
0.382888117001453	-0.62894014501829\\
0.391536830289885	-0.650924409926799\\
0.400690924295007	-0.673290915347343\\
0.410382149176274	-0.696056016822496\\
0.420644928185159	-0.719236406913174\\
0.431516608545333	-0.742849074612439\\
0.443037738909736	-0.766911251151508\\
0.455252376308314	-0.791440339273936\\
0.468208425798595	-0.816453822456737\\
0.481958016346465	-0.841969149837512\\
0.49655791679138	-0.868003591739\\
0.51206999608052	-0.894574059636089\\
0.528561732277561	-0.921696883148368\\
0.546106775144948	-0.949387535119208\\
0.564785567337015	-0.977660294007125\\
0.584686029386328	-1.00652783060336\\
0.605904313663554	-1.03600070342639\\
0.628545632266937	-1.06608674394114\\
0.652725163247935	-1.09679030890624\\
0.67856903856368	-1.12811137254862\\
0.70621541547276	-1.16004442577126\\
0.735815630499326	-1.19257714307226\\
0.767535431233344	-1.22568877014505\\
0.801556275647149	-1.25934817609622\\
0.838076680673205	-1.29351150374422\\
0.877313590692755	-1.32811933948962\\
0.919503721269278	-1.36309331083205\\
0.964904812554868	-1.39833200499505\\
1.01379669855786	-1.4337060868753\\
1.06648206070523	-1.46905247972074\\
1.12328668420254	-1.50416745939867\\
1.18455897045367	-1.53879850582272\\
1.25066837475506	-1.57263475772146\\
1.32200233206178	-1.60529593648329\\
1.39896110183541	-1.63631965159517\\
1.48194980454385	-1.66514708882503\\
1.57136673971757	-1.69110723280973\\
1.66758687790764	-1.71340001439218\\
1.77093922752017	-1.7310791324117\\
1.8816766320022	-1.74303581641394\\
1.99993652033888	-1.74798550602253\\
2.12569131862878	-1.74446034571155\\
2.25868778196976	-1.73081151557542\\
2.39837561762634	-1.7052266533392\\
2.54382765981905	-1.66576876447122\\
2.69365670758171	-1.61044368551659\\
2.84593798907048	-1.53730276692232\\
2.99815078733475	-1.44458519020414\\
3.14715723780444	-1.33089940306667\\
3.28923916983732	-1.19543502361914\\
3.42021295283882	-1.03818557253197\\
3.53563532042957	-0.860150368344511\\
3.63109872482125	-0.663474497931426\\
3.70259395673031	-0.451483848445247\\
3.74689507708525	-0.228582147030333\\
3.76190476190476	-8.05323291956309e-16\\
3.74689507708525	0.228582147030334\\
3.70259395673031	0.451483848445243\\
3.63109872482125	0.663474497931422\\
3.53563532042957	0.86015036834451\\
3.42021295283882	1.03818557253198\\
3.28923916983732	1.19543502361914\\
3.14715723780444	1.33089940306667\\
2.99815078733476	1.44458519020413\\
2.84593798907048	1.53730276692232\\
2.69365670758171	1.61044368551658\\
2.54382765981905	1.66576876447122\\
2.39837561762634	1.7052266533392\\
2.25868778196976	1.73081151557542\\
2.12569131862879	1.74446034571155\\
1.99993652033888	1.74798550602253\\
1.88167663200221	1.74303581641394\\
1.77093922752017	1.7310791324117\\
1.66758687790764	1.71340001439218\\
1.57136673971757	1.69110723280973\\
1.48194980454386	1.66514708882503\\
1.39896110183541	1.63631965159517\\
1.32200233206177	1.60529593648329\\
1.25066837475507	1.57263475772146\\
1.18455897045368	1.53879850582272\\
1.12328668420254	1.50416745939867\\
1.06648206070523	1.46905247972074\\
1.01379669855786	1.4337060868753\\
0.964904812554869	1.39833200499505\\
0.919503721269278	1.36309331083205\\
0.877313590692756	1.32811933948962\\
0.838076680673205	1.29351150374422\\
0.80155627564715	1.25934817609622\\
0.767535431233344	1.22568877014505\\
0.735815630499326	1.19257714307226\\
0.70621541547276	1.16004442577126\\
0.67856903856368	1.12811137254862\\
0.652725163247935	1.09679030890624\\
0.628545632266937	1.06608674394114\\
0.605904313663554	1.03600070342639\\
0.584686029386328	1.00652783060336\\
0.564785567337015	0.977660294007125\\
0.546106775144948	0.949387535119208\\
0.528561732277561	0.921696883148368\\
0.512069996080521	0.89457405963609\\
0.49655791679138	0.868003591739001\\
0.481958016346465	0.841969149837512\\
0.468208425798595	0.816453822456737\\
0.455252376308315	0.791440339273937\\
0.443037738909736	0.766911251151508\\
0.431516608545333	0.742849074612439\\
0.420644928185159	0.719236406913174\\
0.410382149176274	0.696056016822496\\
0.400690924295007	0.673290915347343\\
0.391536830289885	0.650924409926799\\
0.382888117001454	0.628940145018291\\
0.374715480424017	0.607322131504291\\
0.366991857332091	0.586054766936049\\
0.359692239330959	0.565122848288713\\
0.352793504406508	0.544511578617706\\
0.34627426424546	0.524206568769541\\
0.340114725774341	0.504193835103263\\
0.334296565525388	0.484459794014648\\
0.328802815581534	0.464991253918625\\
0.323617759981813	0.445775405231455\\
0.318726840584625	0.42679980879921\\
0.314116571490256	0.408052383139893\\
0.309774461217269	0.389521390800438\\
0.305688941910884	0.371195424074705\\
0.301849304936205	0.353063390282575\\
0.298245642276169	0.335114496771889\\
0.294868793214104	0.31733823577294\\
0.291710295834606	0.299724369208572\\
0.288762342924724	0.282262913540667\\
0.286017741900838	0.264944124715269\\
0.283469878425554	0.247758483253197\\
0.281112683414095	0.230696679520143\\
0.278940603161286	0.213749599199617\\
0.276948572348816	0.196908308983242\\
0.275131989718365	0.180164042485595\\
0.273486696219659	0.163508186384727\\
0.272008955463872	0.146932266784504\\
0.270695436332347	0.130427935790774\\
0.269543197608459	0.11398695829003\\
0.268549674516907	0.097601198916438\\
0.267712667069969	0.0812626091908718\\
0.267030330134381	0.0649632148137773\\
0.266501165145765	0.0486951030922275\\
0.266124013410003	0.0324504104803715\\
0.265898050942813	0.0162213102115927\\
0.265822784810126	1.13811110960658e-16\\
};
\addplot [color=white!85!black, forget plot]
  table[row sep=crcr]{%
0.36986301369863	0\\
0.36996028346031	-0.0150666076576168\\
0.370252281231189	-0.0301386276657198\\
0.370739573165269	-0.045221474920134\\
0.371423105177891	-0.0603205693690484\\
0.372304206446904	-0.0754413384427429\\
0.373384594340659	-0.0905892193656722\\
0.374666380799405	-0.105769661308501\\
0.376152080204487	-0.120988127334879\\
0.377844618777794	-0.136250096094146\\
0.379747345562214	-0.151561063206738\\
0.381864045042391	-0.166926542283666\\
0.384198951473995	-0.18235206551511\\
0.386756764998974	-0.197843183755646\\
0.389542669633901	-0.213405466024899\\
0.392562353228645	-0.229044498332289\\
0.395822029503159	-0.24476588172283\\
0.399328462281248	-0.260575229427492\\
0.403088992051746	-0.276478162986172\\
0.407111564999665	-0.292480307193613\\
0.411404764662461	-0.308587283698318\\
0.41597784637974	-0.324804703061305\\
0.420840774718247	-0.341138155055049\\
0.426004264068019	-0.357593196952631\\
0.431479822619771	-0.374175339522531\\
0.437279799948024	-0.390890030404989\\
0.443417438438768	-0.407742634500731\\
0.449906928814442	-0.424738410951407\\
0.456763470022259	-0.4418824862323\\
0.464003333763978	-0.45917982281085\\
0.471643933955492	-0.476635182748076\\
0.479703901412286	-0.494253085532792\\
0.488203164060909	-0.512037759339133\\
0.497163032975849	-0.529993084784719\\
0.506606294534026	-0.548122530137934\\
0.51655730896351	-0.566429076776197\\
0.527042115536528	-0.584915133530642\\
0.538088544616377	-0.603582438363631\\
0.549726336709481	-0.622431945611521\\
0.561987268592996	-0.641463696783117\\
0.574905286479098	-0.660676672631322\\
0.588516646032246	-0.680068623908586\\
0.602860058866567	-0.699635877872834\\
0.617976844906173	-0.719373117226887\\
0.633911089678764	-0.739273127748586\\
0.650709805216243	-0.759326510399472\\
0.66842309273618	-0.779521353186721\\
0.687104304651664	-0.799842857497885\\
0.70681020267669	-0.820272913035417\\
0.727601107827008	-0.840789614856813\\
0.749541036923469	-0.861366715390662\\
0.772697818741185	-0.881973003671259\\
0.797143181161087	-0.902571603447238\\
0.822952798511357	-0.923119181319477\\
0.850206285668862	-0.943565055715414\\
0.878987122353232	-0.963850197399643\\
0.909382487314417	-0.983906112474146\\
0.941482977714772	-1.00365359959415\\
0.975382183874672	-1.02300137462482\\
1.01117608364116	-1.04184455845622\\
1.04896221394359	-1.06006302751688\\
1.08883856966921	-1.07751963209955\\
1.13090217196943	-1.09405829544777\\
1.17524723977324	-1.1095020172543\\
1.22196289011556	-1.12365081948992\\
1.2711302856283	-1.13627969107424\\
1.32281914229396	-1.14713661160216\\
1.37708350888763	-1.15594076385876\\
1.4339567335581	-1.16238108068848\\
1.49344554550216	-1.16611531399256\\
1.55552320414814	-1.16676986152945\\
1.62012170881876	-1.16394063894192\\
1.68712312307348	-1.15719533650517\\
1.75635015441019	-1.14607744668627\\
1.82755624549549	-1.13011248107705\\
1.90041557927088	-1.10881680170548\\
1.97451357498608	-1.08170945698556\\
2.04933864722677	-1.04832731901123\\
2.12427619866686	-1.00824364841534\\
2.19860599245205	-0.961089950543595\\
2.27150416346192	-0.90658062555547\\
2.34205113142678	-0.844539463725866\\
2.40924652139591	-0.774926526402034\\
2.47203183329182	-0.697863441051832\\
2.52932100842074	-0.61365471192682\\
2.58003823113275	-0.522802412703495\\
2.62316134493291	-0.426011693045989\\
2.6577682781299	-0.324184981136075\\
2.68308303501967	-0.218403625548967\\
2.6985173020487	-0.109896935603413\\
2.7037037037037	-3.86376905080784e-16\\
2.6985173020487	0.109896935603413\\
2.68308303501967	0.218403625548965\\
2.6577682781299	0.324184981136073\\
2.62316134493291	0.426011693045989\\
2.58003823113275	0.522802412703496\\
2.52932100842074	0.613654711926819\\
2.47203183329182	0.697863441051831\\
2.40924652139591	0.774926526402033\\
2.34205113142678	0.844539463725866\\
2.27150416346192	0.906580625555469\\
2.19860599245205	0.961089950543594\\
2.12427619866687	1.00824364841534\\
2.04933864722677	1.04832731901123\\
1.97451357498608	1.08170945698556\\
1.90041557927088	1.10881680170548\\
1.82755624549549	1.13011248107705\\
1.75635015441019	1.14607744668627\\
1.68712312307349	1.15719533650517\\
1.62012170881876	1.16394063894192\\
1.55552320414814	1.16676986152946\\
1.49344554550216	1.16611531399256\\
1.4339567335581	1.16238108068848\\
1.37708350888763	1.15594076385876\\
1.32281914229396	1.14713661160216\\
1.2711302856283	1.13627969107424\\
1.22196289011556	1.12365081948992\\
1.17524723977324	1.1095020172543\\
1.13090217196943	1.09405829544777\\
1.08883856966921	1.07751963209955\\
1.04896221394359	1.06006302751688\\
1.01117608364116	1.04184455845622\\
0.975382183874673	1.02300137462482\\
0.941482977714772	1.00365359959415\\
0.909382487314417	0.983906112474146\\
0.878987122353232	0.963850197399644\\
0.850206285668862	0.943565055715414\\
0.822952798511357	0.923119181319477\\
0.797143181161087	0.902571603447238\\
0.772697818741186	0.88197300367126\\
0.749541036923469	0.861366715390662\\
0.727601107827007	0.840789614856813\\
0.70681020267669	0.820272913035417\\
0.687104304651665	0.799842857497885\\
0.668423092736181	0.779521353186722\\
0.650709805216243	0.759326510399472\\
0.633911089678764	0.739273127748586\\
0.617976844906173	0.719373117226888\\
0.602860058866567	0.699635877872834\\
0.588516646032246	0.680068623908586\\
0.574905286479098	0.660676672631322\\
0.561987268592996	0.641463696783117\\
0.549726336709481	0.622431945611521\\
0.538088544616377	0.603582438363631\\
0.527042115536528	0.584915133530642\\
0.51655730896351	0.566429076776197\\
0.506606294534026	0.548122530137934\\
0.497163032975849	0.529993084784719\\
0.488203164060909	0.512037759339133\\
0.479703901412287	0.494253085532792\\
0.471643933955492	0.476635182748076\\
0.464003333763978	0.45917982281085\\
0.456763470022259	0.441882486232301\\
0.449906928814442	0.424738410951407\\
0.443417438438768	0.40774263450073\\
0.437279799948024	0.390890030404989\\
0.431479822619771	0.374175339522532\\
0.426004264068019	0.357593196952632\\
0.420840774718247	0.34113815505505\\
0.41597784637974	0.324804703061305\\
0.411404764662462	0.308587283698318\\
0.407111564999665	0.292480307193614\\
0.403088992051746	0.276478162986172\\
0.399328462281247	0.260575229427492\\
0.395822029503159	0.244765881722831\\
0.392562353228645	0.229044498332289\\
0.389542669633901	0.213405466024899\\
0.386756764998974	0.197843183755646\\
0.384198951473995	0.18235206551511\\
0.381864045042391	0.166926542283666\\
0.379747345562214	0.151561063206738\\
0.377844618777794	0.136250096094146\\
0.376152080204487	0.120988127334879\\
0.374666380799405	0.105769661308501\\
0.373384594340658	0.0905892193656724\\
0.372304206446904	0.0754413384427433\\
0.371423105177891	0.0603205693690486\\
0.370739573165269	0.045221474920134\\
0.370252281231189	0.0301386276657196\\
0.36996028346031	0.0150666076576167\\
0.36986301369863	1.05711677164155e-16\\
};
\addplot [color=white!85!black, forget plot]
  table[row sep=crcr]{%
0.481481481481481	0\\
0.481594162911531	-0.0134076076698943\\
0.481932386364852	-0.0268177010816324\\
0.482496689791934	-0.040232759271768\\
0.483287971314349	-0.0536552477916375\\
0.484307491533934	-0.0670876117774552\\
0.485556876771667	-0.0805322687936767\\
0.487038123242121	-0.093991601370743\\
0.488753602170918	-0.107467949155439\\
0.490706065863888	-0.120963600588455\\
0.492898654737874	-0.134480784019298\\
0.495334905323979	-0.148021658163403\\
0.498018759254725	-0.161588301800148\\
0.500954573246841	-0.175182702603322\\
0.504147130091255	-0.188806744987549\\
0.507601650661167	-0.20246219684491\\
0.511323806947811	-0.216150695035776\\
0.515319736131428	-0.229873729486243\\
0.519596055692128	-0.243632625731761\\
0.524159879561291	-0.257428525732258\\
0.529018835309021	-0.27126236676833\\
0.534181082356516	-0.285134858210658\\
0.53965533119387	-0.299046455935784\\
0.54545086357347	-0.312997334140444\\
0.55157755363645	-0.326987354283858\\
0.55804588991418	-0.341016030862564\\
0.564866998128088	-0.355082493695427\\
0.572052664688631	-0.369185446367411\\
0.579615360767372	-0.383323120449387\\
0.587568266784194	-0.39749322507782\\
0.595925297113752	-0.411692891442517\\
0.604701124770527	-0.425918611693055\\
0.613911205779188	-0.440166171735013\\
0.6235718028751	-0.454430577346312\\
0.633700008107486	-0.46870597300202\\
0.644313763833332	-0.482985552753828\\
0.655431881491931	-0.49726146246877\\
0.667074057436035	-0.511524692691958\\
0.6792608849639	-0.525764961361648\\
0.692013861544633	-0.539970585573895\\
0.705355390054875	-0.554128341570933\\
0.719308772645196	-0.568223312115509\\
0.733898195627031	-0.58223872041669\\
0.749148703512763	-0.596155749796195\\
0.765086160050017	-0.609953348334217\\
0.78173719376392	-0.623608017817372\\
0.799129125156153	-0.63709358643806\\
0.817289872305732	-0.650380964874699\\
0.83624783117367	-0.663437885629073\\
0.856031726433461	-0.676228625825348\\
0.876670428135553	-0.688713714102675\\
0.898192728973515	-0.700849622779937\\
0.920627076363435	-0.71258844715964\\
0.94400125299236	-0.723877574693485\\
0.968341998959287	-0.734659347782055\\
0.993674568154574	-0.744870725253828\\
1.02002221114227	-0.75444294909338\\
1.04740557657865	-0.763301224792098\\
1.07584202318755	-0.771364425800108\\
1.10534483460426	-0.778544834980075\\
1.13592233009709	-0.784747938704346\\
1.16757686540324	-0.789872292279264\\
1.20030371981325	-0.79380947868027\\
1.23408986836534	-0.796444186057494\\
1.26891264173706	-0.797654432995354\\
1.30473828131947	-0.797311973892388\\
1.3415204031896	-0.795282919808421\\
1.37919839239053	-0.79142861236093\\
1.41769575815424	-0.785606789306258\\
1.45691849143631	-0.777673079787063\\
1.49675347821088	-0.76748286425576\\
1.53706703505006	-0.754893528126894\\
1.57770364699952	-0.7397671285881\\
1.61848500079628	-0.721973480083634\\
1.6592094178805	-0.701393645301909\\
1.69965179994492	-0.677923794849826\\
1.73956420319068	-0.65147937040094\\
1.77867715409625	-0.621999453750533\\
1.8167018074465	-0.589451209391875\\
1.85333302495416	-0.553834233231111\\
1.88825341895297	-0.515184607965624\\
1.92113836019224	-0.473578440218251\\
1.9516618928262	-0.429134639898793\\
1.9795034359089	-0.382016702555892\\
2.004355083345	-0.332433274147238\\
2.02592924906021	-0.280637316839894\\
2.04396634794108	-0.226923754336965\\
2.05824216296261	-0.171625553537346\\
2.06857453132667	-0.115108291153936\\
2.07482899211031	-0.0577633519064213\\
2.07692307692308	-2.02900061397195e-16\\
2.07482899211031	0.0577633519064216\\
2.06857453132667	0.115108291153935\\
2.05824216296261	0.171625553537345\\
2.04396634794108	0.226923754336964\\
2.02592924906021	0.280637316839894\\
2.004355083345	0.332433274147237\\
1.9795034359089	0.382016702555892\\
1.9516618928262	0.429134639898792\\
1.92113836019224	0.473578440218252\\
1.88825341895297	0.515184607965623\\
1.85333302495416	0.553834233231111\\
1.8167018074465	0.589451209391874\\
1.77867715409625	0.621999453750533\\
1.73956420319068	0.65147937040094\\
1.69965179994492	0.677923794849826\\
1.6592094178805	0.701393645301909\\
1.61848500079628	0.721973480083633\\
1.57770364699952	0.7397671285881\\
1.53706703505006	0.754893528126893\\
1.49675347821088	0.76748286425576\\
1.45691849143631	0.777673079787063\\
1.41769575815424	0.785606789306258\\
1.37919839239053	0.79142861236093\\
1.3415204031896	0.79528291980842\\
1.30473828131947	0.797311973892389\\
1.26891264173706	0.797654432995354\\
1.23408986836534	0.796444186057495\\
1.20030371981326	0.79380947868027\\
1.16757686540324	0.789872292279264\\
1.13592233009709	0.784747938704346\\
1.10534483460426	0.778544834980075\\
1.07584202318755	0.771364425800108\\
1.04740557657865	0.763301224792098\\
1.02002221114227	0.75444294909338\\
0.993674568154575	0.744870725253829\\
0.968341998959287	0.734659347782054\\
0.94400125299236	0.723877574693485\\
0.920627076363435	0.71258844715964\\
0.898192728973515	0.700849622779938\\
0.876670428135553	0.688713714102675\\
0.856031726433461	0.676228625825348\\
0.83624783117367	0.663437885629073\\
0.817289872305732	0.650380964874699\\
0.799129125156153	0.63709358643806\\
0.78173719376392	0.623608017817372\\
0.765086160050017	0.609953348334217\\
0.749148703512763	0.596155749796196\\
0.733898195627031	0.58223872041669\\
0.719308772645196	0.568223312115509\\
0.705355390054875	0.554128341570933\\
0.692013861544633	0.539970585573895\\
0.6792608849639	0.525764961361649\\
0.667074057436035	0.511524692691958\\
0.655431881491931	0.49726146246877\\
0.644313763833333	0.482985552753828\\
0.633700008107486	0.46870597300202\\
0.6235718028751	0.454430577346311\\
0.613911205779188	0.440166171735013\\
0.604701124770527	0.425918611693055\\
0.595925297113752	0.411692891442517\\
0.587568266784194	0.39749322507782\\
0.579615360767372	0.383323120449388\\
0.572052664688631	0.369185446367411\\
0.564866998128088	0.355082493695427\\
0.55804588991418	0.341016030862564\\
0.55157755363645	0.326987354283858\\
0.545450863573471	0.312997334140444\\
0.53965533119387	0.299046455935785\\
0.534181082356516	0.285134858210658\\
0.529018835309021	0.27126236676833\\
0.524159879561291	0.257428525732258\\
0.519596055692128	0.243632625731761\\
0.515319736131428	0.229873729486244\\
0.511323806947811	0.216150695035776\\
0.507601650661167	0.20246219684491\\
0.504147130091255	0.188806744987549\\
0.500954573246841	0.175182702603322\\
0.498018759254725	0.161588301800148\\
0.495334905323979	0.148021658163403\\
0.492898654737874	0.134480784019298\\
0.490706065863888	0.120963600588455\\
0.488753602170918	0.107467949155439\\
0.487038123242121	0.0939916013707431\\
0.485556876771667	0.0805322687936769\\
0.484307491533934	0.0670876117774556\\
0.483287971314349	0.0536552477916377\\
0.482496689791934	0.0402327592717679\\
0.481932386364852	0.0268177010816322\\
0.481594162911531	0.0134076076698943\\
0.481481481481481	9.40743768892342e-17\\
};
\addplot [color=white!85!black, forget plot]
  table[row sep=crcr]{%
0.6	0\\
0.600116984016654	-0.0111700163758955\\
0.600468067210338	-0.0223394853144232\\
0.601053643118811	-0.0335078472835381\\
0.601874367992163	-0.0446745184969861\\
0.602931161318202	-0.0558388786248063\\
0.604225206547254	-0.0670002583082397\\
0.605757952005395	-0.0781579264126495\\
0.607531111981721	-0.0893110769510415\\
0.609546667971588	-0.100458815609517\\
0.611806870053799	-0.111600145804495\\
0.614314238375437	-0.122733954199847\\
0.617071564713297	-0.133858995610143\\
0.620081914075735	-0.144973877214158\\
0.623348626302991	-0.156077042000482\\
0.626875317617707	-0.167166751364722\\
0.630665882070327	-0.178241066775303\\
0.634724492816201	-0.189297830422341\\
0.639055603152511	-0.200334644761556\\
0.643663947233454	-0.211348850862765\\
0.64855454037129	-0.222337505470194\\
0.653732678818934	-0.233297356679847\\
0.659203938916426	-0.244224818137513\\
0.664974175468916	-0.255115941659824\\
0.671049519207474	-0.265966388180335\\
0.677436373166057	-0.276771396922951\\
0.684141407788134	-0.287525752706502\\
0.691171554554681	-0.298223751287087\\
0.698533997901405	-0.308859162649253\\
0.70623616516698	-0.319425192163559\\
0.714285714285714	-0.329914439536929\\
0.722690518907312	-0.340318855493963\\
0.731458650593246	-0.350629696142524\\
0.740598357703615	-0.360837474996111\\
0.750118040550398	-0.370931912649389\\
0.760026222352771	-0.380901884132634\\
0.770331515487836	-0.390735364006521\\
0.781042582486137	-0.40041936930169\\
0.792168091176056	-0.409939900458848\\
0.803716663335316	-0.419281880485983\\
0.815696816162218	-0.428429092620761\\
0.828116895834931	-0.437364116869718\\
0.84098500238564	-0.446068265892724\\
0.854308905079278	-0.454521520812757\\
0.868095947456141	-0.46270246765873\\
0.882352941176471	-0.470588235294118\\
0.897086047796199	-0.478154435847646\\
0.912300647610246	-0.485375108845193\\
0.92800119472724	-0.492222670444884\\
0.944191057592424	-0.498667869400059\\
0.960872344259385	-0.504679751616884\\
0.978045711832548	-0.510225635433063\\
0.995710159668123	-0.51527110001897\\
1.01386280613899	-0.519779989588239\\
1.0324986490469	-0.523714436395733\\
1.05161031011158	-0.527034905788926\\
1.07118776438897	-0.529700266853837\\
1.09121805597738	-0.531667892445722\\
1.11168500196737	-0.532893792601784\\
1.13256888728243	-0.533332785478897\\
1.15384615384615	-0.532938710021193\\
1.17548908839316	-0.531664684514385\\
1.19746551421059	-0.52946341499708\\
1.21973849313911	-0.526287557142844\\
1.24226604525619	-0.52209013466793\\
1.26500089477964	-0.516825016526123\\
1.28789025182445	-0.51044745409393\\
1.31087564066893	-0.50291467820138\\
1.33389278607428	-0.49418655420873\\
1.35687156987907	-0.48422629136228\\
1.37973607047413	-0.473001200393793\\
1.40240469776259	-0.460483490788347\\
1.42479043573168	-0.446651096389571\\
1.44680120371749	-0.431488515120935\\
1.46834034575329	-0.414987645688275\\
1.4893072549964	-0.397148601332374\\
1.50959813709941	-0.377980478188122\\
1.52910691253866	-0.357502053766761\\
1.54772625339403	-0.335742389709746\\
1.56534874500449	-0.312741312466137\\
1.58186815747591	-0.288549746102519\\
1.59718080642408	-0.263229873213079\\
1.61118697687949	-0.236855102951385\\
1.62379237928426	-0.209509829576929\\
1.63490960231648	-0.181288970534299\\
1.64445952421923	-0.152297279801438\\
1.65237264269157	-0.122648439799981\\
1.65859028345429	-0.0924639432107407\\
1.66306564947845	-0.0618717841667204\\
1.66576467659242	-0.031004986046869\\
1.66666666666667	-1.08857493257543e-16\\
1.66576467659242	0.0310049860468692\\
1.66306564947845	0.0618717841667197\\
1.65859028345429	0.0924639432107401\\
1.65237264269157	0.122648439799981\\
1.64445952421923	0.152297279801438\\
1.63490960231648	0.181288970534299\\
1.62379237928426	0.209509829576929\\
1.61118697687949	0.236855102951384\\
1.59718080642408	0.26322987321308\\
1.58186815747591	0.288549746102519\\
1.56534874500449	0.312741312466137\\
1.54772625339403	0.335742389709746\\
1.52910691253866	0.357502053766761\\
1.50959813709941	0.377980478188122\\
1.4893072549964	0.397148601332374\\
1.46834034575329	0.414987645688275\\
1.44680120371749	0.431488515120935\\
1.42479043573168	0.446651096389571\\
1.40240469776259	0.460483490788347\\
1.37973607047413	0.473001200393793\\
1.35687156987907	0.48422629136228\\
1.33389278607428	0.49418655420873\\
1.31087564066893	0.50291467820138\\
1.28789025182445	0.51044745409393\\
1.26500089477964	0.516825016526123\\
1.24226604525619	0.52209013466793\\
1.21973849313911	0.526287557142844\\
1.19746551421059	0.52946341499708\\
1.17548908839316	0.531664684514385\\
1.15384615384615	0.532938710021193\\
1.13256888728243	0.533332785478897\\
1.11168500196737	0.532893792601784\\
1.09121805597738	0.531667892445722\\
1.07118776438897	0.529700266853837\\
1.05161031011158	0.527034905788926\\
1.0324986490469	0.523714436395733\\
1.01386280613899	0.519779989588239\\
0.995710159668123	0.51527110001897\\
0.978045711832548	0.510225635433063\\
0.960872344259385	0.504679751616884\\
0.944191057592424	0.498667869400059\\
0.92800119472724	0.492222670444884\\
0.912300647610246	0.485375108845193\\
0.897086047796199	0.478154435847646\\
0.882352941176471	0.470588235294118\\
0.868095947456141	0.46270246765873\\
0.854308905079278	0.454521520812757\\
0.84098500238564	0.446068265892725\\
0.828116895834931	0.437364116869718\\
0.815696816162218	0.428429092620761\\
0.803716663335316	0.419281880485984\\
0.792168091176056	0.409939900458848\\
0.781042582486137	0.40041936930169\\
0.770331515487836	0.390735364006521\\
0.760026222352771	0.380901884132634\\
0.750118040550398	0.370931912649389\\
0.740598357703615	0.360837474996111\\
0.731458650593246	0.350629696142524\\
0.722690518907312	0.340318855493963\\
0.714285714285714	0.329914439536929\\
0.70623616516698	0.319425192163559\\
0.698533997901404	0.308859162649253\\
0.691171554554681	0.298223751287087\\
0.684141407788134	0.287525752706502\\
0.677436373166057	0.276771396922951\\
0.671049519207474	0.265966388180335\\
0.664974175468917	0.255115941659824\\
0.659203938916426	0.244224818137513\\
0.653732678818934	0.233297356679847\\
0.64855454037129	0.222337505470194\\
0.643663947233454	0.211348850862765\\
0.639055603152511	0.200334644761556\\
0.634724492816201	0.189297830422341\\
0.630665882070327	0.178241066775303\\
0.626875317617707	0.167166751364722\\
0.623348626302991	0.156077042000482\\
0.620081914075735	0.144973877214158\\
0.617071564713297	0.133858995610143\\
0.614314238375437	0.122733954199847\\
0.611806870053799	0.111600145804495\\
0.609546667971588	0.100458815609517\\
0.607531111981721	0.0893110769510417\\
0.605757952005395	0.0781579264126495\\
0.604225206547254	0.0670002583082399\\
0.602931161318202	0.0558388786248066\\
0.601874367992163	0.0446745184969862\\
0.601053643118811	0.0335078472835381\\
0.600468067210338	0.0223394853144231\\
0.600116984016654	0.0111700163758954\\
0.6	7.83773951454306e-17\\
};
\addplot [color=white!85!black, forget plot]
  table[row sep=crcr]{%
0.70940170940171	0\\
0.709509060321971	-0.00866946050640965\\
0.709831177181936	-0.0173362256792402\\
0.710368252135413	-0.0259975889145782\\
0.711120604957943	-0.0346508210426549\\
0.712088682319573	-0.0432931589820754\\
0.713273056754684	-0.0519217943189889\\
0.71467442531673	-0.0605338617867748\\
0.716293607902199	-0.0691264276223514\\
0.718131545224453	-0.0776964777758964\\
0.720189296414369	-0.0862409059516287\\
0.722468036220866	-0.0947565014583442\\
0.724969051780471	-0.103239936849667\\
0.727693738920937	-0.111687755335476\\
0.730643597959792	-0.120096357947744\\
0.733820228954229	-0.128461990446112\\
0.737225326354314	-0.136780729950942\\
0.740860673006759	-0.145048471294429\\
0.744728133451712	-0.15326091308359\\
0.748829646450057	-0.161413543472713\\
0.753167216673589	-0.169501625647162\\
0.75774290548526	-0.177520183025334\\
0.762558820731388	-0.185463984191182\\
0.767617105462376	-0.193327527576094\\
0.772919925493144	-0.2011050259161\\
0.778469455709234	-0.208790390518516\\
0.78426786501932	-0.216377215381278\\
0.790317299849972	-0.223858761218444\\
0.796619866073874	-0.231227939456763\\
0.803177609258501	-0.238477296280947\\
0.809992493118692	-0.245598996819342\\
0.817066376053699	-0.25258480957726\\
0.824400985647408	-0.259426091242343\\
0.831997891009763	-0.26611377200501\\
0.839858472838111	-0.27263834155746\\
0.847983891079716	-0.278989835956746\\
0.85637505008117	-0.285157825561298\\
0.865032561117422	-0.291131404275712\\
0.873956702202891	-0.296899180365799\\
0.883147375100169	-0.302449269134507\\
0.892604059458596	-0.307769287779321\\
0.902325764035965	-0.312846352782827\\
0.912310974982387	-0.317667080219976\\
0.922557601196421	-0.322217589397741\\
0.93306291680051	-0.326483510274861\\
0.943823500826125	-0.330449995140441\\
0.954835174249344	-0.334101735059614\\
0.966092934575303	-0.337422981621249\\
0.977590888235578	-0.340397574545724\\
0.989322181136247	-0.343008975728737\\
1.00127892777645	-0.345240310308599\\
1.01345213944761	-0.347074415347683\\
1.02583165212181	-0.348493896712006\\
1.03840605474385	-0.349481194714174\\
1.05116261875386	-0.3500186590522\\
1.06408722978557	-0.350088633527635\\
1.07716432260654	-0.349673550959008\\
1.09037682049013	-0.348756038618438\\
1.10370608033029	-0.347319034408526\\
1.11713184492743	-0.3453459138614\\
1.13063220398184	-0.342820627880672\\
1.14418356542624	-0.339727850959206\\
1.15776063880535	-0.336053139390676\\
1.17133643246323	-0.331783098751637\\
1.18488226632131	-0.326905559664807\\
1.19836780201602	-0.321409760566374\\
1.21176109210842	-0.315286535894524\\
1.2250286499727	-0.30852850779875\\
1.23813554181069	-0.301130279147088\\
1.25104550202131	-0.293088625289893\\
1.26372107287307	-0.284402681734611\\
1.27612376908364	-0.275074124607702\\
1.28821426750336	-0.26510734054023\\
1.29995262163216	-0.254509582426181\\
1.3112985001783	-0.243291107380868\\
1.3222114483013	-0.231465293184245\\
1.33265116958217	-0.219048729542728\\
1.34257782614836	-0.206061280653653\\
1.35195235376474	-0.192526115816452\\
1.36073678810838	-0.178469705208379\\
1.36889459789436	-0.163921778430287\\
1.37639102003683	-0.148915244025119\\
1.38319339163645	-0.133486068868741\\
1.38927147330412	-0.117673117114874\\
1.39459775818065	-0.101517949223312\\
1.39914776100798	-0.0850645824891079\\
1.40290028176	-0.0683592153919249\\
1.4058376386542	-0.0514499189687975\\
1.40794586583762	-0.0343862992482481\\
1.40921487166179	-0.0172191355377095\\
1.40963855421687	-6.04412703890405e-17\\
1.40921487166179	0.0172191355377096\\
1.40794586583762	0.0343862992482477\\
1.4058376386542	0.0514499189687972\\
1.40290028176	0.0683592153919248\\
1.39914776100798	0.0850645824891079\\
1.39459775818065	0.101517949223312\\
1.38927147330412	0.117673117114874\\
1.38319339163645	0.133486068868741\\
1.37639102003683	0.148915244025119\\
1.36889459789436	0.163921778430287\\
1.36073678810838	0.178469705208379\\
1.35195235376474	0.192526115816452\\
1.34257782614836	0.206061280653652\\
1.33265116958217	0.219048729542728\\
1.3222114483013	0.231465293184245\\
1.3112985001783	0.243291107380868\\
1.29995262163216	0.254509582426181\\
1.28821426750336	0.26510734054023\\
1.27612376908364	0.275074124607701\\
1.26372107287307	0.284402681734611\\
1.25104550202131	0.293088625289893\\
1.23813554181069	0.301130279147089\\
1.2250286499727	0.30852850779875\\
1.21176109210842	0.315286535894524\\
1.19836780201602	0.321409760566374\\
1.18488226632131	0.326905559664807\\
1.17133643246323	0.331783098751637\\
1.15776063880535	0.336053139390676\\
1.14418356542624	0.339727850959206\\
1.13063220398184	0.342820627880672\\
1.11713184492743	0.3453459138614\\
1.10370608033029	0.347319034408526\\
1.09037682049013	0.348756038618438\\
1.07716432260654	0.349673550959008\\
1.06408722978557	0.350088633527635\\
1.05116261875386	0.3500186590522\\
1.03840605474385	0.349481194714174\\
1.02583165212181	0.348493896712006\\
1.01345213944761	0.347074415347683\\
1.00127892777645	0.345240310308599\\
0.989322181136246	0.343008975728737\\
0.977590888235578	0.340397574545724\\
0.966092934575303	0.337422981621249\\
0.954835174249344	0.334101735059614\\
0.943823500826125	0.330449995140441\\
0.93306291680051	0.326483510274861\\
0.922557601196421	0.322217589397741\\
0.912310974982387	0.317667080219976\\
0.902325764035965	0.312846352782827\\
0.892604059458596	0.307769287779321\\
0.883147375100169	0.302449269134507\\
0.873956702202891	0.296899180365799\\
0.865032561117422	0.291131404275712\\
0.85637505008117	0.285157825561298\\
0.847983891079716	0.278989835956746\\
0.839858472838111	0.27263834155746\\
0.831997891009763	0.26611377200501\\
0.824400985647408	0.259426091242343\\
0.817066376053699	0.25258480957726\\
0.809992493118692	0.245598996819342\\
0.803177609258501	0.238477296280947\\
0.796619866073874	0.231227939456763\\
0.790317299849972	0.223858761218444\\
0.78426786501932	0.216377215381278\\
0.778469455709234	0.208790390518516\\
0.772919925493144	0.2011050259161\\
0.767617105462376	0.193327527576094\\
0.762558820731388	0.185463984191182\\
0.75774290548526	0.177520183025334\\
0.753167216673589	0.169501625647162\\
0.748829646450057	0.161413543472713\\
0.744728133451712	0.15326091308359\\
0.740860673006759	0.145048471294429\\
0.737225326354315	0.136780729950942\\
0.733820228954229	0.128461990446112\\
0.730643597959792	0.120096357947744\\
0.727693738920936	0.111687755335476\\
0.72496905178047	0.103239936849667\\
0.722468036220866	0.0947565014583443\\
0.720189296414369	0.0862409059516287\\
0.718131545224453	0.0776964777758965\\
0.716293607902199	0.0691264276223515\\
0.71467442531673	0.0605338617867748\\
0.713273056754684	0.0519217943189891\\
0.712088682319573	0.0432931589820757\\
0.711120604957943	0.034650821042655\\
0.710368252135413	0.0259975889145782\\
0.709831177181936	0.01733622567924\\
0.709509060321971	0.00866946050640961\\
0.70940170940171	6.08342335758785e-17\\
};
\addplot [color=white!85!black, forget plot]
  table[row sep=crcr]{%
0.834862385321101	0\\
0.834939442857395	-0.00528784548835783\\
0.835170606895141	-0.0105721747878963\\
0.835555851551621	-0.0158494671283976\\
0.83609513312431	-0.0211161925873729\\
0.836788389242876	-0.0263688075403537\\
0.83763553768	-0.0316037501432166\\
0.838636474819066	-0.0368174358577531\\
0.839791073776249	-0.0420062530321531\\
0.841099182174095	-0.0471665585486437\\
0.842560619563231	-0.0522946735512093\\
0.844175174488511	-0.05738687926712\\
0.845942601195572	-0.0624394129369064\\
0.847862615973606	-0.067448463868443\\
0.849934893129981	-0.0724101696319371\\
0.852159060592351	-0.0773206124138639\\
0.854534695133978	-0.0821758155492347\\
0.8570613172182	-0.0869717402530371\\
0.859738385458373	-0.0917042825732343\\
0.862565290690103	-0.096369270589349\\
0.865541349653316	-0.10096246188238\\
0.86866579828259	-0.105479541303605\\
0.871937784605274	-0.109916119071676\\
0.875356361248274	-0.114267729229348\\
0.87892047755592	-0.118529828493119\\
0.882628971323231	-0.122697795531038\\
0.886480560150967	-0.126766930705927\\
0.890473832431318	-0.13073245632318\\
0.894607237975856	-0.134589517424238\\
0.898879078300404	-0.138333183168616\\
0.903287496585011	-0.141958448849102\\
0.907830467330967	-0.145460238586245\\
0.912505785741043	-0.148833408749639\\
0.917311056853707	-0.152072752154579\\
0.922243684467059	-0.155173003083504\\
0.927300859893631	-0.15812884318206\\
0.932479550592938	-0.160934908279675\\
0.937776488734848	-0.163585796184113\\
0.943188159753353	-0.166076075498441\\
0.948710790957155	-0.168400295507311\\
0.954340340270641	-0.170552997177085\\
0.960072485186227	-0.172528725311333\\
0.96590261201661	-0.174322041899291\\
0.971825805543217	-0.175927540690037\\
0.977836839164842	-0.177339863019341\\
0.983930165658169	-0.178553714909235\\
0.990099908669374	-0.179563885452355\\
0.996339855063177	-0.180365266483899\\
1.00264344826248	-0.180952873533648\\
1.00900378271781	-0.181321868038798\\
1.0154135996512	-0.181467580785459\\
1.02186528422334	-0.181385536532434\\
1.02835086427627	-0.181071479755529\\
1.03486201080537	-0.180521401434019\\
1.04139004031487	-0.179731566783284\\
1.04792591920928	-0.178698543818959\\
1.05446027036961	-0.177419232618601\\
1.06098338205723	-0.175890895126807\\
1.06748521927994	-0.174111185329378\\
1.07395543774388	-0.172078179601642\\
1.08038340050103	-0.169790407015792\\
1.08675819738561	-0.167246879372464\\
1.09306866731294	-0.164447120703011\\
1.09930342349188	-0.16139119597168\\
1.10545088157649	-0.158079738691334\\
1.11149929075384	-0.154513977153221\\
1.11743676773449	-0.150695758960795\\
1.12325133357786	-0.146627573550442\\
1.12893095324992	-0.142312572378422\\
1.13446357777261	-0.137754586453942\\
1.13983718878624	-0.13295814090343\\
1.1450398453068	-0.127928466260999\\
1.15005973242085	-0.122671506195214\\
1.15488521162196	-0.117193921402577\\
1.15950487245574	-0.111503089423915\\
1.16390758510517	-0.10560710017085\\
1.16808255351631	-0.0995147469857874\\
1.17201936863597	-0.093235513099942\\
1.17570806130956	-0.086779553399493\\
1.17913915436892	-0.080157671459485\\
1.18230371342767	-0.073381291857829\\
1.18519339589586	-0.066462427836979\\
1.18780049772708	-0.0594136444376345\\
1.1901179974198	-0.0522480172861723\\
1.19213959681065	-0.0449790872743894\\
1.19385975822144	-0.0376208114254526\\
1.19527373755247	-0.0301875102926051\\
1.1963776129529	-0.0226938122860795\\
1.19716830874349	-0.0151545953677887\\
1.19764361431771	-0.00758492659175051\\
1.1978021978022	-2.66195415827223e-17\\
1.19764361431771	0.00758492659175055\\
1.19716830874349	0.0151545953677886\\
1.1963776129529	0.0226938122860793\\
1.19527373755247	0.0301875102926051\\
1.19385975822144	0.0376208114254526\\
1.19213959681065	0.0449790872743893\\
1.1901179974198	0.0522480172861723\\
1.18780049772708	0.0594136444376344\\
1.18519339589586	0.066462427836979\\
1.18230371342767	0.0733812918578289\\
1.17913915436892	0.0801576714594849\\
1.17570806130956	0.0867795533994929\\
1.17201936863597	0.093235513099942\\
1.16808255351631	0.0995147469857874\\
1.16390758510517	0.10560710017085\\
1.15950487245574	0.111503089423915\\
1.15488521162196	0.117193921402577\\
1.15005973242085	0.122671506195214\\
1.1450398453068	0.127928466260999\\
1.13983718878624	0.13295814090343\\
1.13446357777261	0.137754586453942\\
1.12893095324992	0.142312572378422\\
1.12325133357786	0.146627573550442\\
1.11743676773449	0.150695758960795\\
1.11149929075384	0.154513977153221\\
1.10545088157649	0.158079738691334\\
1.09930342349188	0.161391195971679\\
1.09306866731294	0.164447120703011\\
1.08675819738561	0.167246879372464\\
1.08038340050103	0.169790407015792\\
1.07395543774388	0.172078179601642\\
1.06748521927994	0.174111185329378\\
1.06098338205723	0.175890895126807\\
1.05446027036961	0.177419232618601\\
1.04792591920928	0.178698543818959\\
1.04139004031487	0.179731566783284\\
1.03486201080537	0.180521401434019\\
1.02835086427627	0.181071479755529\\
1.02186528422334	0.181385536532434\\
1.0154135996512	0.181467580785459\\
1.00900378271781	0.181321868038798\\
1.00264344826248	0.180952873533648\\
0.996339855063177	0.180365266483899\\
0.990099908669374	0.179563885452355\\
0.983930165658169	0.178553714909235\\
0.977836839164842	0.177339863019341\\
0.971825805543217	0.175927540690037\\
0.96590261201661	0.174322041899291\\
0.960072485186227	0.172528725311333\\
0.954340340270641	0.170552997177085\\
0.948710790957155	0.168400295507311\\
0.943188159753353	0.166076075498441\\
0.937776488734848	0.163585796184113\\
0.932479550592938	0.160934908279676\\
0.927300859893632	0.15812884318206\\
0.922243684467059	0.155173003083504\\
0.917311056853707	0.152072752154579\\
0.912505785741043	0.148833408749639\\
0.907830467330967	0.145460238586246\\
0.903287496585011	0.141958448849102\\
0.898879078300404	0.138333183168616\\
0.894607237975856	0.134589517424238\\
0.890473832431318	0.13073245632318\\
0.886480560150967	0.126766930705927\\
0.882628971323231	0.122697795531038\\
0.87892047755592	0.118529828493119\\
0.875356361248274	0.114267729229348\\
0.871937784605274	0.109916119071676\\
0.86866579828259	0.105479541303605\\
0.865541349653316	0.10096246188238\\
0.862565290690103	0.0963692705893491\\
0.859738385458373	0.0917042825732344\\
0.8570613172182	0.0869717402530371\\
0.854534695133978	0.0821758155492348\\
0.852159060592351	0.077320612413864\\
0.849934893129981	0.0724101696319371\\
0.847862615973606	0.0674484638684429\\
0.845942601195572	0.0624394129369064\\
0.844175174488511	0.0573868792671201\\
0.842560619563231	0.0522946735512093\\
0.841099182174095	0.0471665585486437\\
0.839791073776249	0.0420062530321532\\
0.838636474819066	0.0368174358577531\\
0.83763553768	0.0316037501432167\\
0.836788389242876	0.0263688075403539\\
0.83609513312431	0.021116192587373\\
0.835555851551621	0.0158494671283976\\
0.835170606895141	0.0105721747878962\\
0.834939442857395	0.00528784548835781\\
0.834862385321101	3.71073855477693e-17\\
};
\addplot [color=green, draw=none, mark size=0.65pt, mark=*, mark options={solid, green!75!black}, forget plot]
  table[row sep=crcr]{%
0	0\\
0	-0.0327366104129726\\
0	-0.0655434628152382\\
0	-0.0984914033571642\\
0	-0.131652497587396\\
0	-0.165100668192196\\
0	-0.198912367379658\\
0	-0.233167297170198\\
0	-0.267949192431123\\
0	-0.303346683607342\\
0	-0.339454258863376\\
0	-0.376373348913058\\
0	-0.414213562373095\\
0	-0.453094105301755\\
0	-0.493145426031304\\
0	-0.534511135950791\\
0	-0.577350269189626\\
0	-0.621839960068261\\
0	-0.668178637919299\\
0	-0.716589866095654\\
0	-0.76732698797896\\
0	-0.82067879082866\\
0	-0.876976462992757\\
0	-0.936602207992061\\
0	-1\\
2.22044604925031e-16	-1.06768913362254\\
0	-1.14028145816755\\
2.22044604925031e-16	-1.21850352558798\\
0	-1.30322537284121\\
0	-1.39549838382241\\
0	-1.49660576266549\\
2.22044604925031e-16	-1.60813081213087\\
0	-1.73205080756888\\
0	-1.87086841178939\\
0	-2.02779940198922\\
0	-2.20704703128727\\
0	-2.41421356237309\\
2.22044604925031e-16	-2.65693626524815\\
-3.33066907387547e-16	-2.94590500454579\\
2.22044604925031e-16	-3.29655820893832\\
0	-3.73205080756888\\
-3.33066907387547e-16	-4.28876610114864\\
-4.44089209850063e-16	-5.02733949212585\\
4.44089209850063e-16	-6.05691067728377\\
4.44089209850063e-16	-7.59575411272514\\
1.99840144432528e-15	-10.1531703876089\\
-2.77555756156289e-15	-15.2570516882655\\
1.13242748511766e-14	-30.546839986944\\
nan	0\\
1.88737914186277e-14	30.5468399869441\\
6.66133814775094e-16	15.2570516882655\\
-1.11022302462516e-15	10.1531703876089\\
-8.88178419700125e-16	7.59575411272515\\
-2.22044604925031e-16	6.05691067728377\\
-6.66133814775094e-16	5.02733949212585\\
-3.33066907387547e-16	4.28876610114865\\
0	3.73205080756888\\
-2.22044604925031e-16	3.29655820893832\\
0	2.94590500454579\\
2.22044604925031e-16	2.65693626524815\\
2.22044604925031e-16	2.4142135623731\\
-1.11022302462516e-16	2.20704703128727\\
2.22044604925031e-16	2.02779940198923\\
0	1.87086841178939\\
-2.22044604925031e-16	1.73205080756888\\
0	1.60813081213087\\
0	1.49660576266549\\
0	1.39549838382241\\
0	1.30322537284121\\
2.22044604925031e-16	1.21850352558798\\
0	1.14028145816755\\
2.22044604925031e-16	1.06768913362254\\
0	1\\
0	0.936602207992062\\
0	0.876976462992757\\
0	0.82067879082866\\
0	0.767326987978961\\
0	0.716589866095654\\
0	0.668178637919299\\
0	0.621839960068261\\
0	0.577350269189626\\
2.22044604925031e-16	0.534511135950792\\
0	0.493145426031305\\
0	0.453094105301755\\
0	0.414213562373095\\
0	0.376373348913058\\
0	0.339454258863376\\
0	0.303346683607343\\
0	0.267949192431123\\
0	0.233167297170198\\
0	0.198912367379658\\
0	0.165100668192197\\
0	0.131652497587396\\
0	0.0984914033571645\\
0	0.0655434628152383\\
0	0.0327366104129725\\
0.0929208356849593	0.0162226689198019\\
0.0931184354842083	-0.0487018854981908\\
0.0941148485988925	-0.11403494655212\\
0.0959443737009928	-0.18033442940647\\
0.0986713996322683	-0.248190947720382\\
0.102395296322265	-0.318249247114548\\
0.107258317541975	-0.391233667410372\\
0.113457626195415	-0.467979831450372\\
0.121263259971953	-0.549475607026564\\
0.131044999824604	-0.63691574615839\\
0.143313032937918	-0.731776780065329\\
0.158780671757648	-0.835922273060251\\
0.17846349189331	-0.951754358424516\\
0.20384073060864	-1.08243729266216\\
0.237127324188919	-1.2322356735052\\
0.281751388980009	-1.40703966376955\\
0.343232985815967	-1.61520224441022\\
0.430894171229528	-1.86890546930139\\
0.561413737529534	-2.18641751887952\\
0.766817771537931	-2.59572585590503\\
1.11415773037591	-3.13938430543898\\
1.75895825620782	-3.8735170883937\\
3.09925251008531	-4.80052958620869\\
6.09139217738053	-5.29460608334421\\
10.4435277303573	-1.82329282206166\\
8.43241312042418	4.41023752309484\\
4.30503538417501	5.21622769677559\\
2.29940004476416	4.32188964349314\\
1.3832156423212	3.47924122338168\\
0.916146520784399	2.84742513531698\\
0.651756315546252	2.37733557110146\\
0.489299311650569	2.0182178763429\\
0.382983107486093	1.73539681776256\\
0.309920340746788	1.50630047190357\\
0.257740620290654	1.31606035642299\\
0.219317181850587	1.15462489951886\\
0.190323383791181	1.01500373052013\\
0.168016083262552	0.892200855772973\\
0.150592014192357	0.782553645673427\\
0.136829345570124	0.683312749874164\\
0.125879121458192	0.59236800629842\\
0.117139272500574	0.508064953446989\\
0.110176028659106	0.429078918305184\\
0.1046735564465	0.354326499701206\\
0.100400970645361	0.282901803668805\\
0.0971903840944022	0.214029305293948\\
0.0949221976827845	0.147027974287779\\
0.0935153118704501	0.0812830178548826\\
0.176883196301801	0.0475933707909737\\
0.176883196301801	-0.0475933707909743\\
0.180230070893102	-0.143618115970695\\
0.18717727265647	-0.242213054418726\\
0.198272723091761	-0.34528634813167\\
0.21445235657723	-0.455065644579335\\
0.2372142196598	-0.574286216339787\\
0.268930811245264	-0.706453013965666\\
0.313417206917464	-0.856218086981818\\
0.377002547846749	-1.02992558538632\\
0.47065274703373	-1.2363553663993\\
0.614424838603895	-1.48749816218042\\
0.847388064456672	-1.79814855810148\\
1.25065792793863	-2.17798546348706\\
1.9975697807925	-2.58491573871947\\
3.39358544535887	-2.70421215301572\\
5.27178737767351	-1.41846222050792\\
5.2717873776735	1.41846222050794\\
3.39358544535887	2.70421215301572\\
1.99756978079249	2.58491573871946\\
1.25065792793863	2.17798546348706\\
0.84738806445667	1.79814855810147\\
0.614424838603894	1.48749816218042\\
0.47065274703373	1.2363553663993\\
0.377002547846748	1.02992558538632\\
0.313417206917464	0.856218086981818\\
0.268930811245264	0.706453013965665\\
0.2372142196598	0.574286216339787\\
0.21445235657723	0.455065644579335\\
0.198272723091761	0.345286348131669\\
0.18717727265647	0.242213054418726\\
0.180230070893102	0.143618115970694\\
0.268217561800785	0.0914691298710539\\
0.265822784810126	5.15886359784162e-17\\
0.268217561800785	-0.0914691298710538\\
0.275569921922687	-0.184341486664295\\
0.288408288053582	-0.280090460166527\\
0.307700316407143	-0.380333479737032\\
0.335005758832081	-0.486911762751156\\
0.372743490289762	-0.60197301778845\\
0.424648347683288	-0.728039578897188\\
0.49655791679138	-0.868003591739\\
0.597786553965638	-1.02487687375658\\
0.743540060044397	-1.20080179923084\\
0.959044729436211	-1.39391681323621\\
1.28565749104666	-1.58913904616449\\
1.78437342365486	-1.73291127211836\\
2.506991485019	-1.67704274174301\\
3.33989174238315	-1.13898951839033\\
3.76190476190476	-8.05323291956309e-16\\
3.33989174238315	1.13898951839033\\
2.506991485019	1.67704274174301\\
1.78437342365486	1.73291127211836\\
1.28565749104666	1.58913904616449\\
0.959044729436212	1.39391681323621\\
0.743540060044398	1.20080179923085\\
0.597786553965638	1.02487687375658\\
0.49655791679138	0.868003591739001\\
0.424648347683288	0.728039578897188\\
0.372743490289762	0.60197301778845\\
0.335005758832081	0.486911762751156\\
0.307700316407143	0.380333479737032\\
0.288408288053582	0.280090460166527\\
0.275569921922687	0.184341486664296\\
0.376866927329368	0.127659562174368\\
0.370633291684903	-0.0423923669874873\\
0.389724489083103	-0.214380598962148\\
0.439160892192093	-0.396141404798054\\
0.533184765470006	-0.595393109840275\\
0.703034568255686	-0.816434961286897\\
1.01347833356002	-1.04300284862644\\
1.57544391174116	-1.16627974912518\\
2.38032776038201	-0.806310073094552\\
2.66324312720097	0.304616942293437\\
1.96985533282733	1.08358283338529\\
1.25549762214429	1.13251111517305\\
0.834693701149734	0.932080041809796\\
0.605634256461184	0.703323852132511\\
0.479187497008675	0.493147122988418\\
0.410033522992564	0.303542252926138\\
0.504355306351281	0.189659272855168\\
0.482373625567771	0.0377169439817539\\
0.489578468666928	-0.113369788532648\\
0.527468225732335	-0.266935168102581\\
0.604140077653349	-0.425028816853889\\
0.736706540865561	-0.58485626012107\\
0.954529867383756	-0.72866061792871\\
1.2934375747169	-0.797593773469877\\
1.73709020065286	-0.653220577221794\\
2.06048459747409	-0.16110993225785\\
1.93861916393049	0.448918526295855\\
1.50930692145324	0.763813016863986\\
1.10722438658194	0.778962178520994\\
0.832631170035924	0.661008861948338\\
0.66191482273399	0.505286717817929\\
0.560139050199504	0.345407793502706\\
0.646995595831728	0.21890648580219\\
0.600925959277983	-0.0314138901750529\\
0.679497196271643	-0.28013783239928\\
0.925020155476227	-0.490968496316738\\
1.38684504230346	-0.469229429873583\\
1.65956333239973	0.0867550144200603\\
1.25787706447302	0.518587797270867\\
0.843448287457665	0.447672988488367\\
0.816616637082435	0.25215233879275\\
0.730835476568391	-0.120620508697089\\
1.11797366827478	-0.345204423181178\\
1.3320133080599	0.219841712616208\\
1.03771647093802	0.180205559690462\\
0.840344492304881	0.0442676110007382\\
0.935444747739651	-0.162445474314915\\
1.18669512754444	-0.0625126466153475\\
};

\end{axis}
\end{tikzpicture}%\vspace{-1mm}}
\label{fig:Constellation_z}}\\[2mm]
\subfloat[resulting information rate]{
\includegraphics[width=.85\columnwidth]{Rates_Alphabets_linearCR-crop.pdf}
\label{fig:Constellation_rate}}
\caption{Proposal for a rich 8-bit symbol alphabet that allows for near-capacity information rates at high $\SNR$.}
\label{fig:Constellation}
\end{figure}

The high-SNR gap between PSK and channel capacity confirms that purposefully adding load resistance is crucial for achieving very high data rates. We are interested in a rich symbol alphabet that remedies this gap. Inspired by \Cref{fig:CapAchievingDistr}, the capacity-achieving distribution at $\SNR = 21\dB$, we propose the symbol alphabet in \Cref{fig:Constellation_i}. It uses $M = 2^8 = 256$ and a heuristic construction that ensures large pairwise symbol distances. The symbol probabilities are set such that the outer circle is chosen with $\CircProb_1 = 0.36$ (the high-SNR value from \Cref{fig:PMF_evolution}). The associated rate graph (green dashed) in \Cref{fig:Constellation_rate} indeed demonstrates high-SNR rates very close to channel capacity. The low-SNR gap could be closed by adapting the symbol probabilities $q_m$ to the SNR (like in adaptive modulation), which is omitted for brevity.

The constellation plot \Cref{fig:Constellation_i} highlights certain symbols that are unrealizable when the load is subject to certain value-range constraints. This particular evaluation assumes an inductive RFID tag whose coil antenna ($Z\Tx = R\Tx + j\omega L\Tx$) is loaded with an impedance $R+\f{1}{j\omega C}$ with adaptive $R,C \in \bbR_+$, from the value range
$(1-\Delta)C_\text{res} \leq C \leq (1+\Delta)C_\text{res}$
about the resonance value
$C_\text{res} = 1/(\omega^2 L\Tx)$.
It can be shown that this is equivalently described by our \Cref{sec:model} model with
$z = r + jx$
and
$x = x\Tx (1 - \f{C_\text{res}}{C})$
subject to
$\f{-\Delta}{1-\Delta} x\Tx \leq x \leq \f{\Delta}{1+\Delta} x\Tx$. Thereby $x\Tx = X\Tx / R\Tx = \omega L\Tx / R\Tx$ is the coil Q-factor.
In \Cref{fig:Constellation_i} we assume $\Delta = 0.5$ and $x\Tx = 15$, which yields 9 out of 256 unrealizable symbols. The resultant loss of achievable rate turns out to be negligible at the considered SNR range (the graph is not shown in \Cref{fig:Constellation_rate} because visually it coincides with the dashed green graph). We conclude that mild value-range constraints do not prohibit near-capacity data rates.


\section{Summary}
\label{sec:summary}
For the first time this paper stated the channel capacity of load modulation with a freely adaptable passive impedance. The obtained insights on the capacity-achieving transmit distribution and how to approximate it with finite symbol alphabets have important implications for practical high-data-rate backscatter communication systems. This applies even to the ambient backscatter case, under certain identified conditions.

\begin{appendices}
\crefalias{section}{appendix}
	
	% \section[appendix] % ERROR?
    \section{Conditions on Modulated Ambient Signals}
    \label{apdx:amb}
    The ambient backscatter case requires special care because the voltages $v\Tx\Amb$ and $v\Rx\Amb$ may exhibit fast time-variations from modulation.
In that regard, we require the following conditions:
%(i) All relevant propagation channels (tag $\rightarrow$ receiver, ambient source $\rightarrow$ tag, ambient source $\rightarrow$ receiver) either do not suffer from fading at all or they are subject to slow fading.
\mbox{(i) The} relevant propagation channels are either subject to block fading or no fading at all.
(ii) There is only a single ambient source and it uses digital modulation.
(iii) The channel from ambient source to receiver is much stronger than the backscatter channel, i.e. $\EV{|Z\RxTx i\Tx|^2} \ll \EV{|v\Rx\Amb|^2}$.
(iv) The modulated signal $v\Rx\Amb$ can be decoded correctly.
(v) There is no interference from other backscatter tags.

We identify the following different cases for which the channel capacity result \Cref{eq:Capacity} applies to ABC in some fashion:

\textbf{1.) The ambient source has much faster symbol rate than the load modulation:} Let $L \gg 1$ denote the ratio of symbol rates and assume $L \in \bbN$. We consider the fast symbol rate with time index $\ell$. Let $s[\ell] := \f{-Z\RxTx}{R\Tx(1 + z[\ell])} v\Tx\Amb[\ell] + v\N[\ell]$, which is $v$ after compensation of the decoded $v\Rx\Amb[\ell]$. The modulated $v\Tx\Amb[\ell]$ is i.i.d. random and $v\N[\ell] \iid \calN(0,\sigma^2 L)$ while $z[\ell]$ is constant over length-$L$ blocks. For a specific block %we denote the constant $z[\ell]$ as $z$ and
we collect the various signals in the vectors ${\bf s}, {\bf v}\Tx\Amb, {\bf v}\N \in \bbC^L$ to write
${\bf s}
%= -Z\RxTx {\bf i}\Tx + {\bf v}\N
= \f{-Z\RxTx}{R\Tx(1 + z)} {\bf v}\Tx\Amb + {\bf v}\N$.
We consider maximum-ratio combining
$\tilde{y} = - {\bf u}\H {\bf s} / \sqrt{L}$ at the receiver, whereby
${\bf u} := {\bf v}\Tx\Amb / \|{\bf v}\Tx\Amb\|$.
This results in the relation $\tilde{y} = \f{Z\RxTx}{R\Tx(1 + z)} \|{\bf v}\Tx\Amb\| / \sqrt{L} + \omega$ with $\omega \sim \calN(0,\sigma^2)$. This relation is equivalent to the signal model \Cref{eq:SignalModel} with the exception that $|v\Tx\Amb|$ is replaced by $\|{\bf v}\Tx\Amb\| / \sqrt{L}$. The latter approaches the RMS value of $v\Tx\Amb$ for large $L$. Therefore the system behaves as if $v\Tx\Amb$ was constant.

\textbf{2.) The ambient source has much slower symbol rate than the load modulation:} The effect on the backscatter system is the same as if $v\Tx\Amb$ was unmodulated but subject to block fading. By coding across many such blocks, the information rate $\EVVar{v\Tx\Amb}{\RateMax}$ can be achieved \cite[Sec.~5.4.5]{Tse2005}. Thereby $\RateMax$ is the complicated expression from \Cref{eq:Capacity}.

\textbf{3.) The ambient source is PSK modulated:} PSK has a constant envelope, so $|v\Tx\Amb|$ and $|v\Rx\Amb|$ are constant for the duration of a fading block. The phase shifts in $v$ due to $v\Tx\Amb$ can be compensated with the knowledge from the decoded $v\Rx\Amb$. Then the system behaves as if $v\Tx\Amb$ was constant.

    \section{Mutual Information Details}
    \label{apdx:ar}
    The signal model $y = Z\RxTx \cdot i + w$ from \Cref{eq:SignalModel} exhibits AWGN $w \iid \mathcal{CN}(0,\sigma^2)$ and non-random $Z\RxTx$. The achievable rate is given by the mutual information $\Rate$ between output $y$ and input $i$. For a complex-valued AWGN channel \cite[(B.47)]{Tse2005}
\begin{align}
& \Rate = h(y) - \log_2(\pi e \sigma^2)
\, , \label{eq:MutualInfo} \\
& h(y) = -\int_\bbC f_y(y) \, \log_2\big( f_y(y) \big) \,dy
\label{eq:DifferentialEntropy}
\end{align}
where $h(y)$ is the differential entropy of $y$.
% and $\log_2(\pi e \sigma^2)$ that of the AWGN $w$.
The PDF $f_y = f_v * f_w$ is the convolution of the PDF $f_v$ of the receive-side signal $v := Z\RxTx \cdot i$ and the PDF
$f_w(w) = \f{1}{\pi \sigma^2} \exp(-\f{|w|^2}{\sigma^2})$.
%of the complex-valued AWGN.

With the transmit statistics from \Cref{sec:CurrentStats}, $f_i$ and $f_v$ are supported on concentric circles. Then $f_y$ is characterized by noise-convoluted concentric circles. Conditioned on circle $k$, $|y|$ has Rice distribution and $\arg(y)$ uniform distribution. These facts can be used in \Cref{eq:DifferentialEntropy} to derive the result \Cref{eq:RateFromCircles} for $\Rate$ after lengthy calculation, as detailed in \cite{ShamaiTIT1995}.

When $i$ is instead chosen from a finite symbol alphabet $\{ s_1 , \ldots , s_{M} \} \subseteq \Disk$ with probabilities $\SymbProb_m$, then
%$f_{i}(i) = \sum_{m=1}^M \SymbProb_m \delta(i - i_m)$ and
$f_v(v) = \sum_{m=1}^M \SymbProb_m \delta(v - Z\RxTx s_m)$.
The convolution with $f_w$ yields
$f_y(y) = \sum_{m=1}^M \f{\SymbProb_m}{\pi \sigma^2} \exp(-\f{1}{\sigma^2} |y - Z\RxTx s_m|^2 )$.
This expression allows to calculate the differential entropy $h(y)$ with numerical integration to then evaluate the mutual information \Cref{eq:MutualInfo}.

%A technically attractive choice is modulation according to a finite symbol alphabet. An example is shown in \Cref{fig:Constellation_i}. Formally, this restricts $f_{i}$ to a finite number of mass points, i.e. $f_{i}(i) = \sum_{m=1}^M \SymbProb_m \delta(i - i_m)$. This is implemented by drawing samples $i = i_m$ with probabilities $\SymbProb_m$ from a finite RX-side symbol alphabet
%$\SymbolAlphabet = \{ i_1 , \ldots , i_{M} \} \subseteq \Disk$. The corresponding TX-side load alphabet is
%$\SymbolAlphabetZL = \{ z_1 , \ldots , z_{M} \}$
%with $z_m = \ChannelFunc^{-1}(i_m)$.
%A finite symbol alphabet imposes a limit $\Rate \leq \log_2(M)$ on the achievable rate.

%To calculate the achievable rate that results from a given symbol alphabet $\SymbolAlphabet$ and soft decoding, we first note that the PDF of the AWGN-corrupted observation $y$ is now the Gaussian mixture
%$f_y(y)
%= \sum_{m=1}^M \f{\SymbProb_m}{\pi \sigma^2}\, \exp( -\f{|y - Z\RxTx i_m|^2}{\sigma^2} )$
%with
%$y_m = |Z\RxTx| i_m$.
%This expression allows to calculate $h(y)$ in \Cref{eq:DiffEntropy} with numerical integration to then evaluate the mutual information \Cref{eq:MutualInfo}.

%When the channel is furthermore discretized by quantization of $y$ at the RX (cf. symbol decision thresholds, hard decoding), the achievable rate can only decrease. For brevity, these aspects and the resulting rates are not discussed in detail.

%The PDF $f_{i}$ of the TX current $i$ must fulfill $\supp(f_{i}) \subseteq \Disk$.
%The noise PDF is given by
%$f_w(w) = \f{1}{\pi \sigma^2} \exp(-\f{|w|^2}{\sigma^2})$.
%With $s = |Z\RxTx| i\Tx$, the PDF of $y = s + w$ is given by the convolution
%$f_y(y) = (f_s * f_w)(y)$.

%In general, the achievable rate $\Rate$ of a channel from a TX-side random variable $i$ and RX-side observed random variable $y$ is given by their mutual information in bit,
%\begin{align}
%\Rate = I(y;s) &= h(y) - h(y | i) 
%\label{eq:MutualInfoDef} \\
%&= h(y) - h(y | s) \\
%&= h(y) - h(w) \\
%&= h(y) - \log_2(\pi e \sigma^2)
%\label{eq:MutualInfo} \, .
%\end{align}
%In \Cref{eq:MutualInfo} we used that the noise differential entropy is explicitly given by $h(w) = \log_2(\pi e \sigma^2)$ in the complex AWGN case \cite[(B.47)]{Tse2005}. 
%The differential entropy $h(y)$ is defined as
%\begin{align}
%h(y) = -\int_\bbC f_y(y) \, \log_2\big( f_y(y) \big) \,dy \, .
%\label{eq:DiffEntropy}
%\end{align}
%Now $\Rate$ can be calculated via \Cref{eq:MutualInfo} by first evaluating $h(y)$, e.g. via numerical integration over $\bbC$.




\end{appendices}

\bibliographystyle{IEEEtran}
\bibliography{ref}

\pdfinfo{
	   /Author(Gregor Dumphart, Johannes Sager, Armin Wittneben)
	   /Title(\OurPaperTitle{})
	   /Subject(\OurKeywords{})
	   /Keywords(\OurKeywords{})
}

\end{document}
