The ambient backscatter case requires special care because the voltages $v\Tx\Amb$ and $v\Rx\Amb$ may exhibit fast time-variations from modulation.
In that regard, we require the following conditions:
%(i) All relevant propagation channels (tag $\rightarrow$ receiver, ambient source $\rightarrow$ tag, ambient source $\rightarrow$ receiver) either do not suffer from fading at all or they are subject to slow fading.
\mbox{(i) The} relevant propagation channels are either subject to block fading or no fading at all.
(ii) There is only a single ambient source and it uses digital modulation.
(iii) The channel from ambient source to receiver is much stronger than the backscatter channel, i.e. $\EV{|Z\RxTx i\Tx|^2} \ll \EV{|v\Rx\Amb|^2}$.
(iv) The modulated signal $v\Rx\Amb$ can be decoded correctly.
(v) There is no interference from other backscatter tags.

We identify the following different cases for which the channel capacity result \Cref{eq:Capacity} applies to ABC in some fashion:

\textbf{1.) The ambient source has much faster symbol rate than the load modulation:} Let $L \gg 1$ denote the ratio of symbol rates and assume $L \in \bbN$. We consider the fast symbol rate with time index $\ell$. Let $s[\ell] := \f{-Z\RxTx}{R\Tx(1 + z[\ell])} v\Tx\Amb[\ell] + v\N[\ell]$, which is $v$ after compensation of the decoded $v\Rx\Amb[\ell]$. The modulated $v\Tx\Amb[\ell]$ is i.i.d. random and $v\N[\ell] \iid \calN(0,\sigma^2 L)$ while $z[\ell]$ is constant over length-$L$ blocks. For a specific block %we denote the constant $z[\ell]$ as $z$ and
we collect the various signals in the vectors ${\bf s}, {\bf v}\Tx\Amb, {\bf v}\N \in \bbC^L$ to write
${\bf s}
%= -Z\RxTx {\bf i}\Tx + {\bf v}\N
= \f{-Z\RxTx}{R\Tx(1 + z)} {\bf v}\Tx\Amb + {\bf v}\N$.
We consider maximum-ratio combining
$\tilde{y} = - {\bf u}\H {\bf s} / \sqrt{L}$ at the receiver, whereby
${\bf u} := {\bf v}\Tx\Amb / \|{\bf v}\Tx\Amb\|$.
This results in the relation $\tilde{y} = \f{Z\RxTx}{R\Tx(1 + z)} \|{\bf v}\Tx\Amb\| / \sqrt{L} + \omega$ with $\omega \sim \calN(0,\sigma^2)$. This relation is equivalent to the signal model \Cref{eq:SignalModel} with the exception that $|v\Tx\Amb|$ is replaced by $\|{\bf v}\Tx\Amb\| / \sqrt{L}$. The latter approaches the RMS value of $v\Tx\Amb$ for large $L$. Therefore the system behaves as if $v\Tx\Amb$ was constant.

\textbf{2.) The ambient source has much slower symbol rate than the load modulation:} The effect on the backscatter system is the same as if $v\Tx\Amb$ was unmodulated but subject to block fading. By coding across many such blocks, the information rate $\EVVar{v\Tx\Amb}{\RateMax}$ can be achieved \cite[Sec.~5.4.5]{Tse2005}. Thereby $\RateMax$ is the complicated expression from \Cref{eq:Capacity}.

\textbf{3.) The ambient source is PSK modulated:} PSK has a constant envelope, so $|v\Tx\Amb|$ and $|v\Rx\Amb|$ are constant for the duration of a fading block. The phase shifts in $v$ due to $v\Tx\Amb$ can be compensated with the knowledge from the decoded $v\Rx\Amb$. Then the system behaves as if $v\Tx\Amb$ was constant.