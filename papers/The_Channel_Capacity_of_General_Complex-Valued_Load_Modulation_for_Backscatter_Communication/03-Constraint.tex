%The transmit current $i$ takes on a large value $i = \f{|v\Ind\Tx|}{R\Tx}$ when the tag is resonant ($Z\L = 0$) and vanishes to $i = 0$ when the tag is open-circuited ($Z\L = \infty$).
%In contrary to an active wireless node with a transmit amplifier, ...
Backscatter tags are passive and thus limited in their capability to establish a desired transmit current $i[n]$.
Formally, this is due to $\Re(z[n]) \geq 0$ in \Cref{eq:TxSignalDef}.
In the following we determine the set of realizable transmit currents, denoted $i[n] \in \Disk$, as prerequisite for the preceding channel capacity analysis.

In accordance with typical conventions in communication theory, we henceforth discard time indexation $[n]$ for brevity.
From \Cref{eq:TxSignalDef} we observe that the transmit current $i$ is a non-linear map of the normalized load impedance $z$:
\begin{align}
i &= \ChannelFunc(z) = \f{2\cdot\DiskRadius}{1 + z}
\, , \label{eq:ChannelMap} \\[1mm]
%\Re(z) &\geq 0
%\, , \\
z &= \ChannelFunc^{-1}(i) = \f{2\cdot\DiskRadius}{i} - 1
\, . \label{eq:InverseChannelMap} 
\end{align}
The map $g$ is illustrated in \Cref{fig:LoadTransformation}.
The current quantity $\DiskRadius \in \bbR$ will have the meaning of a radius. It is defined as
\begin{align}
\DiskRadius := \f{|v\Ind\Tx|}{2 R\Tx}
\, . \label{eq:DiskRadius}
\end{align}

The impedance $z$ of any passive load must lie in the right half-plane
%$z \in \HalfPlane$,
$\HalfPlane := \{ z \in \bbC \, | \, \Re(z) \geq 0 \}$. To characterize the set $\Disk = \ChannelFunc(\HalfPlane)$, we rewrite \Cref{eq:ChannelMap} as
$i = \ChannelFunc(z) = \DiskRadius (1 - \f{z-1}{z+1})$ or rather $i = \DiskRadius (1 - \Gamma)$. The reflection coefficient $\Gamma = \f{z-1}{z+1}$ is a bijective map from $z \in \HalfPlane$ to the unit disk $|\Gamma| \leq 1$; it is the M\"obius transformation that also underlies the well-known Smith chart \cite[Eq.~(2.53)]{Pozar2004}. This yields a constraint on the transmit current
\begin{align}
|i - \DiskRadius| \leq \DiskRadius
\label{eq:Constraint}
\end{align}
because $|i- \DiskRadius| = |-\DiskRadius\Gamma| = \DiskRadius |\Gamma| \leq \DiskRadius$.
%
The set of realizable transmit currents $i \in \Disk$ is thus given by a disk $\Disk \subset \bbC$ with radius $\DiskRadius$ and center $\DiskRadius$:
\begin{align}
\Disk
= \left\{ \ChannelFunc(z) \, \big| \, \Re(z) \geq 0 \right\}
= \left\{ i \in \bbC \ \big| \ |i - \DiskRadius| \leq \DiskRadius \right\} .
\label{eq:Disk}
\end{align}
%Its center $\DiskRadius \in \bbR$ lies on the real axis of the complex plane.
%Its boundary circle is
%$\partial\Disk = \left\{ s \in \bbC \ \big| \ |s - \DiskRadius| = \DiskRadius \right\}$.
An analogous observation is found in the  literature, regarding transformed RFID transponder impedance. \cite[Sec.~4.1]{Finkenzeller2015}

\begin{figure}[!ht]
\centering
%\resizebox{.96\columnwidth}{!}{\input{LoadTransformationIllustration}}%
\ \ \ \ \includegraphics[width=.88\columnwidth]{LoadTransformationIllustration-crop}
\put(-230,56){\scriptsize{\rotatebox{90}{$\Im(i) \, / \, \DiskRadius$}}}
\put(-184,-8){\scriptsize{$\Re(i) \, / \, \DiskRadius$}}
\put(-106,62){\scriptsize{\rotatebox{90}{$\Im(z)$}}}
\put(-54,-8){\scriptsize{$\Re(z)$}}
\put(-175,125){\footnotesize{$i = \ChannelFunc(z) = \f{2}{1 + z} \DiskRadius$}}
\caption{Map between normalized load impedance $z$ and transmit current $i$.}
\label{fig:LoadTransformation}
\end{figure}
