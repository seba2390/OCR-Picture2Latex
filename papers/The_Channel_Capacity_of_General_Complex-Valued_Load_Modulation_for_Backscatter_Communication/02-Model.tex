Before studying the information theory of load-modulated BC, we first have to establish an adequate system model. Our approach is based on the circuit models in \Cref{fig:SystemModelCircuit}, which are inspired by \cite{Finkenzeller2015}. They describe the different classes of tag-to-receiver BC links as listed in \cite[Fig.~2]{HuynhCST2018}.
%We develop a system model based on circuit descriptions of the communication systems of interest. We aim for a system model that allows to study the achievable information rate of all cases of interest with the same notation.
In each case, the left-hand circuit is a tag that modulates information via an adaptive passive load. We employ a symbol time index $n \in \bbZ$ and denote the load impedance $Z\L[n] \in \bbC$. It must fulfill $\Re(Z\L[n]) \geq 0$ at all times because the load is passive \cite[Sec.~4.1]{Pozar2004}.
The tag current phasor $i\Tx[n] \in \bbC$ depends on $Z\L[n]$. The tag antenna impedance is $R\Tx + jX\Tx$, however its reactance $X\Tx$ is canceled by the serial $-X\Tx$ element (resonance). The right-hand circuit is an information receiver that measures a voltage phasor $v[n] \in \bbC$. The tag and receiver circuits are coupled via the mutual impedance $Z\RxTx \in \bbC$, which encapsulates all aspects of the propagation channel.

\begin{figure}[!ht]
\centering
\subfloat[ambient or bistatic backscatter]{\centering
\resizebox{.9\columnwidth}{!}{\begin{tikzpicture}

\node (0,0) {\includegraphics[width=80mm]{SystemAmbientBS_noTermBox.pdf}};

%\put (-100, 47) {$Z\Term[n]$}

\put (-60, 37) {$v\Ind\Tx\! = v\Amb\Tx$}
\put (3, 38) {$v\Amb\Rx \! + v\N[n]$}
%\put (23, 38) {$v\N[n]$}
%\put (53, 38) {$v\Amb\Rx\! - \! Z\RxTx \iTx[n]$}
\put (62, 38) {$-Z\RxTx\,\iTx[n]$}

\put (-94, 35) {$-X\Tx$}

\put (115, -5) {$v[n]$}
\put (-131, -3) {$Z\L[n]$}

\put (-20, -15) {$X\Tx$}
\put (-3, -15) {$X\Rx$}

\put (-21, 10) {$R\Tx$}
\put (-4, 10) {$R\Rx$}

\put (-52, -25) {$\iTx[n]$}

\put(32,-6){$\begin{array}{l}
\mathrm{\scriptstyle not\hphantom{.}shown}\!:\\[-0.7mm]
\mathrm{\scriptstyle source\hphantom{.}of}\\[.3mm]
v\Tx\Amb, v\Rx\Amb\end{array}$}

%$\substack{\mathrm{not\hphantom{-}shown:}\\\mathrm{source\hphantom{-}of}\\v\Tx\Amb, v\Rx\Amb}$

\end{tikzpicture}}
\label{fig:SystemModel_AmbientBackscatter}%
}\\[5mm]
\subfloat[monostatic backscatter (e.g., RFID)]{\centering
\resizebox{.9\columnwidth}{!}{\begin{tikzpicture}

\node (0,0) {\includegraphics[width=80mm]{SystemPassiveRFID_noTermBox.pdf}};

%\put (-110, 47) {$Z\Term[n]$}

\put (-60, 37) {$v\Ind\Tx \! = Z\RxTx\,i\Rx$}
\put (23, 38) {$v\N[n]$}
\put (62, 38) {$-Z\RxTx\,\iTx[n]$}

\put (-94, 35) {$-X\Tx$}

%\put (124, -8) {$\begin{array}{c} + \\[2mm] v[n] \\[2mm] - \end{array}$}
\put (115, -5) {$v[n]$}
\put (66, -5) {$i\Rx$}
\put (-131, -3) {$Z\L[n]$}

\put (-20, -15) {$X\Tx$}
\put (-3, -15) {$X\Rx$}

\put (-20, 10) {$R\Tx$}
\put (-3, 10) {$R\Rx$}

\put (-52, -25) {$\iTx[n]$}

\end{tikzpicture}}
\label{fig:SystemModel_RfidBackscatter}%
}%\\
%\subfloat[inductive RFID link]{\centering
%\resizebox{.9\columnwidth}{!}{\begin{tikzpicture}

\node (0,0) {\includegraphics[width=80mm]{figures/SystemInductiveRFID_noTermBox.pdf}};

%\put (-103, 47) {$Z\Term[n]$}

.\put (-104, 33) {$C\Res$}

\put (42, 38) {$v\N[n]$}

%\put (116, -8) {$\begin{array}{c} + \\[2mm] v[n] \\[2mm] - \end{array}$}
\put (115, -5) {$v[n]$}
\put (65, -5) {$i\Rx$}
\put (-131, -3) {$Z\L[n]$}

\put (-4, -6) {$M$}

\put (-18, -25) {$L\Tx$}
\put (7, -25) {$L\Rx$}

\put (-18, 10) {$R\Tx$}
\put (7, 10) {$R\Rx$}

\put (-52, -25) {$\iTx[n]$}

\end{tikzpicture}
}
%\label{fig:SystemModel_RfidInductive}%
%}
\caption{Circuit descriptions of different classes of BC links. In each case, a load-modulating passive tag (left) transmits to an information receiver (right). %The voltmeter is the information sink.
In (b) the information receiver is also the power source (cf. current $i\Rx$).}
\label{fig:SystemModelCircuit}
\end{figure}

The circuit \Cref{fig:SystemModel_AmbientBackscatter} describes both ambient and bistatic backscatter links. These paradigms differ only in the assumptions regarding the voltages
$v\Amb\Tx , v\Amb\Rx \in \bbC$
that are induced by an extrinsic electromagnetic field, generated by some source. In ambient backscatter they are random modulated signals from an ambient source, but in the bistatic case they are unmodulated signals from a dedicated source \cite{HuynhCST2018}. % We consider them constant for the duration of a code word.
In either case, $v\Amb\Tx$ is the crucial cause for any electrical activity at the tag while $v\Amb\Rx$ is receive-side interference.

The monostatic case in \Cref{fig:SystemModel_RfidBackscatter} does not assume any extrinsic source. Instead, the information receiver is the system's power source (e.g., an RFID reader) and the crucial tag-side induced voltage $v\Ind\Tx = Z\RxTx\, i\Rx$ is due to the source current $i\Rx$. A prominent example of monostatic BC is inductive RFID, where $X\Tx, X\Rx, Z\RxTx$ are determined by inductances and where $-X\Tx$ is realized by a resonance capacitor.

%Here the induced tag voltage $v\Ind\Tx$ (whose value is left unspecified) is a consequence of the ambient electromagnetic field. The very similar circuit in \Cref{fig:SystemModel_RfidBackscatter} describes a passive RFID system; here $v\Ind\Tx$ is modeled as dependent source. Specifically, $v\Ind\Tx = Z\RxTx i\Rx$ is induced by the electromagnetic field that was generated by the current $i\Rx$ through the driven RFID reader antenna. A special case thereof is the inductive RFID system in \ToDo{fig:SystemModel\_RfidInductive} where the antenna reactances $X\Tx = \omega L\Tx$, $X\Rx = \omega L\Rx$ are determined by inductances $L\Tx, L\Rx$ and the mutual impedance $Z\RxTx = j\omega M$ is determined by the mutual inductance $M$ with $M^2 \leq L\Tx L\Rx$. In particular, the induced tag voltage is $v\Ind\Tx = j\omega M \cdot i\Rx$ and the resonance capacitance $C\Res$ is chosen such that $\f{1}{j\omega C\Res} = -jX\Tx$ at the resonance frequency $\omega = 2\pi f$.

We assume that $Z\L[n]$ is piecewise constant over time and that it changes instantaneously at the symbol switching instants. We neglect any signal transients which result for $i\Tx$ and $v$. This is a meaningful assumption if the symbol duration is significantly larger than the time constants of the circuits. Our previous work \cite[Appendix~E]{Dumphart2020} showed that transients do not deteriorate the receive processing of load-modulated signals and, when anticipated, can even improve the SNR.
%We consider the steady state (assume that transients have decayed to an extent where they have no appreciable impact).

The noise voltage sequence $v\N[n]$ is white Gaussian noise $v\N[n] \iid \calCN(0,\sigma^2)$ with variance $\sigma^2$, a well-established model for thermal noise \cite{Tse2005}. The samples are statistically independent and identically distributed (iid) for different $n$.
% Could also model sky noise, interference, ...)}.
%Recall that this is a shorthand notation for $\Re(w[n]),\Im(w[n]) \sim \calN(0,\f{\sigma^2}{2})$.

%We note that inductive RFID (\Cref{fig:SystemModel_RfidInductive}) is just a special case of monostatic backscatter (\Cref{fig:SystemModel_RfidBackscatter}) with ...

A basic circuit analysis yields the tag current expression
\begin{align}
\iTx[n] &= \f{v\Ind\Tx}{R\Tx + Z\L[n]} %\, , &
%\Re(Z\L[n]) &\geq 0\ \forall n 
\label{eq:Current} \, .
\end{align}
The receive voltage in the ambient backscatter case is given by
$v[n] = -Z\RxTx\,\iTx[n] + v\Amb\Rx + v\N[n]$.
In the monostatic backscatter case,
$v[n] = -Z\RxTx\,\iTx[n] + (R\Rx + jX\Rx) i\Rx + v\N[n]$.
%
To unify these different cases within the same system model, we consider a phase rotation
$e^{j\alpha} = \f{(v\Ind\Tx)^*}{|v\Ind\Tx|}$, a specific receive signal compensation,
and other transformations:
\begin{align}
i[n] &:= e^{j\alpha} \, \iTx[n]
= \f{|v\Ind\Tx|}{R\Tx} \f{1}{1 + z[n]}
\label{eq:TxSignalDef} \, , \\
z[n] &:= Z\L[n] \, / \, R\Tx
\label{eq:NormalizedZ} \, , \\
w[n] &:= -e^{j\alpha} \, v\N[n]
\label{eq:NoiseDef} \, , \\
y[n] &:= -e^{j\alpha} \!\left( v[n] - v\big|_{\iTx = 0, v\N = 0}  \right)
\label{eq:ObservationDef} \, .
\end{align}
The noise $w[n]$ maintains the statistics of $v\N[n]$. The unitless $z[n]$ is the normalized load impedance. The transformation from $v$ to $y$ in \Cref{eq:ObservationDef} could be practically realized via interference cancellation, calibration, and channel estimation. For the ambient backscatter case, where $v\Amb\Tx$ and $v\Amb\Rx$ are unknown modulated signals, this delicate aspect is discussed in \Cref{apdx:amb}. Monostatic backscatter systems face the challenge of canceling the strong self-interference $(R\Rx + jX\Rx) i\Rx$, cf. \cite{Finkenzeller2015}.

For either case, the definitions \Cref{eq:TxSignalDef,eq:NormalizedZ,eq:NoiseDef,eq:ObservationDef} yield a complex-valued, discrete-time signal and noise model:
\begin{align}
y[n] &= Z\RxTx \cdot i[n] + w[n]
\label{eq:SignalModel}
\, , \\
w[n] &\iid \calCN(0,\sigma^2)
\, .
\label{eq:NoiseModel}
\end{align}
The observation $y[n] \in \bbC$ is considered without quantization.