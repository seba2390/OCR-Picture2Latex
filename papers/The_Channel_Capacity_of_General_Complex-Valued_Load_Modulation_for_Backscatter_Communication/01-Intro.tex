Backscatter communication (BC) via load modulation allows simple passive tags to communicate with essentially zero transmit power and no transmit amplifier. This is achieved by modulating the termination load of the tag antenna in order to affect the reflection of an incident field (possibly an ambient field). This technique found widespread use in radio-frequency identification (RFID) and smart cards \cite{Finkenzeller2015} and is a promising approach to ultra-low-energy communication in the Internet of Things (IoT) \cite{HuynhCST2018}. The high data rate requirements of many IoT applications have recently prompted interest in backscatter modulation beyond binary \cite[Tab.~III]{HuynhCST2018}, e.g. 16-QAM \cite{KimionisNAT2021} or QPSK \cite{WangMercierISSCC2020}, together with error-correcting codes \cite{HuynhCST2018}. %An overview of modulation in ambient backscatter communication is given in .

From the perspective of communication theory, it is natural to ask for the channel capacity of a BC link, i.e. the maximum achievable information rate.
The existing research literature contains only a few related investigations. For example \cite{ZhaoACCESS2018} addresses the calculation of the channel capacity of binary load modulation in ambient backscatter communication (ABC) for various cases of the ambient signal modulation.
The work in \cite{KimSPAWC2017} concerns the maximization of ABC network capacity in terms of redundancy and reflection coefficient (for BPSK, QPSK, and 16-QAM alphabets) in a WiFi setting with OFDM. The focus of \cite{FuschiniAPL2008} is on the effect of the propagation environment on the Euclidean symbol distances and the resulting bit error rate with PSK and ASK for RFID load modulation.
%and diversity to combat fading in \cite{LiTVT2019}.
The literature lacks a complete description of the channel capacity and the capacity-achieving transmit scheme of BC load modulation, which would provide a crucial guideline for the design of practical systems with near-optimal rates \cite{Tse2005}.

This paper describes for the first time the channel capacity of load modulation in the general case of a  freely adaptable passive load. In this case, the load impedance can take on any complex value with non-negative real part for the duration of every symbol period. %which is the criterion for passive loads .
This is a generalization of specific modulation schemes such as QPSK, where the load takes values from a finite alphabet. The results and insights promise useful implications for practical BC systems.

This paper contains the following specific contributions:
\begin{itemize}
\item We develop a signal and noise model that abstracts all major classes of load-modulated single-tag BC links.
%, namely ambient, bistatic, and monostatic links.
\item Based thereon, we study the physical constraints on the tag-side transmit signal, arising from the passive nature of the tag. We find that the transmit current phasor $i \in \bbC$ must lie in a certain disk in the right half-plane. %because it follows from the load impedance via a M\"obius transform.
\item We discover that this disk constraint allows to solve the channel capacity problem at hand with existing theory on peak-power-constrained quadrature AWGN channels.
\item The capacity result is stated and discussed. We identify special cases in which the result even applies to ABC.
\item The capacity-achieving distribution of the transmit current and of the load impedance are characterized in detail. For the low-SNR case we show that a purely reactive load with Cauchy-distributed reactance achieves capacity.
\item We construct a finite symbol alphabet that approximates the capacity-achieving distribution. It yields near-capacity data rates, even if several symbols are unrealizable due to implementation constraints on the load.
\end{itemize}

This paper does not address the tag power consumption or aspects of the energy harvesting circuit. Specific channel models and multi-user interference are also out of scope.

% We assume that (i) the reception is subject to additive white Gaussian noise (AWGN), (ii) the load impedance is piecewise constant over time and just changes at the switching instants, (iii) ideal channel codes with a very large block length are used, and (iv) the signal transients can be ignored (the investigation in \cite[Appendix~E]{Dumphart2020} shows that this is a sensible assumption).

%%%%% it does not require the transmitting tag to provide any physical transmit power. To the best of our knowledge, the state of the art in load modulation considers only schemes with one bit \cite{Finkenzeller2015} or two bit \cite{WangMercierISSCC2020} per symbol (i.e. per load switching period) and the employed error-correcting codes are rudimentary.
%For the sake of improving the data rate of future load modulation systems, this work investigates the maximum achievable information rate when an arbitrary passive load impedance can be realized at the tag. We consider \ToDo{a magneto-inductive RFID load modulation system where the receiving reader observes the voltage change over the reader coil} due to the presence of the loaded tag. The observation is assumed to be subject to complex-valued additive white Gaussian noise (AWGN).

%%%%%We find that the received signal point (without noise) lies within a disc on the complex plane, with real-valued center $\DiskRadius$ and radius $\DiskRadius$. Any signal point within this disk can be realized with a passive tag load. Hence, the channel capacity of this system (which, by assumption, is subject to AWGN) equals the channel capacity of the complex-valued AWGN channel subject to a peak-power constraint, which has been investigated and analyzed in \cite{ShamaiTIT1995}. Hence, we shall make use of the knowledge contained in \cite{ShamaiTIT1995}, as it is the key to maximizing load modulation data rates.

%\subsubsection*{Related Work}

%... A printed millimetre-wave modulator and antenna array for backscatter communications at gigabit data rates \cite{KimionisNAT2021}.

%... passive long-range communication from ambient LoRa \cite{PengACM2018}.

%A circuit implementation for a backscatter IoT tag using 4-PSK modulation, which is capable of communicating with commodity WiFi transceivers, was presented in \cite{WangMercierISSCC2020}.

%... efficiency of RFID load modulation under practical conditions in \cite{FuschiniAPL2008}.

%... channel modeling and code design for near-field passive RFID in \cite{BarberoTC2014}.

%Netscatter \cite{Hessar2019}.

%Optimum modulation and coding \cite{KimSPAWC2017}.

\subsubsection*{Paper Structure}
\Cref{sec:model} describes the employed system model and \Cref{sec:constraint} the special transmit-side constraints. \Cref{sec:capacity} states the channel capacity, the associated distributions, and a familiar upper bound. \Cref{sec:practical} addresses practical modulation aspects and \Cref{sec:summary} concludes the paper.

\subsubsection*{Notation}
For a random variable $x$, the probability density function (PDF) is denoted as $f_x(x)$. For simplicity, we do not use distinct random variable notation.
% $\bbZ$, $\bbR$, $\bbR_+$, $\bbC$