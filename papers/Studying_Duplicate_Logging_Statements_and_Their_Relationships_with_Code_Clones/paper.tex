\documentclass[10pt,journal,compsoc]{IEEEtran}


%!TEX ROOT = ../centralized_vs_distributed.tex

\section{Problem Setup}\label{sec:setup}

We consider {an undirected network} with $ N $ agents 
{in which the state of the $ i $th agent at time $ t $ is given by $ \xbar{i}{t}\in\Real{} $ with the control input $ \u{i}{t}\in\Real{} $.}
For {notational} convenience,
we {introduce} the aggregate {state of the} system  $ \xbar{}{t} $ and {the}
aggregate control input $ \u{}{t} $ {by stacking states and control inputs of each subsystem} $ \xbar{i}{t} $ and $ \u{i}{t} $, 
respectively.

\iffalse
\begin{rem}[Network topology]
	While in the first part of the paper we focus on circular formations for the sake of analysis and ease of presentation,
	the control design can be readily extended to generic undirected topologies.
	We discuss theoretical guarantees in~\autoref{sec:generic-topology} and observe with computational experiments in~\autoref{sec:numerical-results}
	that the \tradeoff holds regardless of the specific topology. % at hand.
\end{rem}
\fi

\myParagraph{Problem Statement}
The agents aim to reach consensus towards a common state trajectory. 
The $i$th component of the vector $ \x{}{t} \doteq \Omega\xbar{}{t} $ represents 
the mismatch between the state of agent $ i $ and the average network state at time $ t $\revision{\cite{bamjovmitpat12}},
where
\begin{equation}\label{eq:error-matrix}
	\Omega \doteq I_{N}-\consMatrix
\end{equation}
and $ \mathds{1}_N \in\Real{N} $ is the vector of all ones,
such that $ \Omega\mathds{1}_N=0 $.
%\red{The target consensus vector is defined as $ \x{m}{t} \doteq \xbar{}{t} - \x{}{t} $.}

\done{
	\tcb{\myParagraph{Ring Topology}
	We focus on ring topology to obtain analytical insights about 
	optimal control design and fundamental performance trade-offs in the presence of communication delays. 
	While some of our notation is tailored to such topology (\eg see equations~\eqref{eq:meas} and~\eqref{eq:feedback-matrix}), 
	in~\autoref{sec:generic-topology} we discuss extension of the optimal control design to generic undirected networks 
	and complement these developments with computational experiments in~\autoref{sec:numerical-results}.}
}

%\myParagraph{\titlecap{Communication model}}
%The agents communicate through a {shared wireless channel}.
%Data are exchanged through a {shared wireless channel} in a symmetric fashion.
%Agent $ i $ communicates with
%\red{$ n $ pairs of agents,
%both agents in each such pair being at equal distance from $ i $}
%\tcb{its} $ 2n $ closest neighbors \tcb{in ring topology.}
%Also, we make the following assumption
%to address channel constraints.

\begin{ass}[Communication model]\label{ass:hypothesis}
	Data are exchanged through a shared wireless channel in a symmetric fashion.
%	\tcb{its} $ 2n $ closest neighbors \tcb{in ring topology.}
	\revisiontwo{Agent $ i $ receives state measurements from
	all agents within $ n $ communication hops.}
	All measurements are received with delay $ \taun \doteq f(n) $
	where $ f(\cdot) $ is a positive increasing sequence.
	\revisiontwo{In particular,
		in ring topology,
		agent $ i $ receives state measurements from the
		$ 2n $ closest agents,
		that is,
		from the $ n $ pairs of agents at distance $ \ell = 1,\dots,n $,
		with $ 1\le n<\nicefrac{N}{2} $.}\footnote{\revision{
%	where both agents in each such pair are at equal distance $ \ell $ from $ i $. % in the ring topology.
%%	located $ \ell $ positions ahead and behind in the formation,.
%	In what follows,
%	without loss of generality,
%	we assume that such $ n $ agent pairs coincide with the
%	$ 2n $ closest agents in ring topology,
%	and that each pair is at distance $ \ell = 1,\dots,n<\nicefrac{N}{2} $.
	\revisiontwo{For example,
	$ n = 1 $ corresponds to nearest-neighbor interaction in ring topology
	and $ n = \floor{\nicefrac{(N-1)}{2}} $ to all-to-all communication topology.}}}
%	Also, each agent measures its own state instantaneously.
\end{ass}

\revision{\begin{rem}[Architecture parametrization]\label{rem:architecture-param}
	Parameter $ n $ will play a crucial role throughout our discussion. 
	In particular,
	we will use it to (i) evaluate the optimal performance %that can be attained 
	for a given
	budget of links
	\revisiontwo{(see~\cref{prob:variance-minimization})};
	and to (ii) compare optimal performance of different control architectures.
	In the first part of the paper, 
	we examine circular formations and
	$ n $ represents how many neighbor pairs communicate with each agent.
	For \linebreak general undirected networks,
	$ n $ determines the number of communication hops for each agent.
	In general,
	$ n $ characterizes sparsity of a controller architecture:
	sparse controllers correspond to small $n$ while highly connected ones to 
	large $ n $.
\end{rem}}

%\begin{rem}
%	The time $ \delayn $ embeds both the communication delay,
%	due to channel constraints,
%	and the computation delay,
%%	Even though the rate $ f(n) = n $ may seem natural,
%%	other rates are possible, 
%	arising if the agents preprocess the acquired measurements.
%	In practice, $ f(n) $ is to be estimated or learned from data.
%\end{rem}

\myParagraph{\titlecap{Feedback control}}
Agent $ i $ uses the received information to compute the
state mismatches $ \meas{i}{\ell^\pm}{t} $ {relative to its} neighbors,
\begin{equation}\label{eq:meas}
	\meas{i}{\ell^\pm}{t} = 
	\begin{cases}
		\xbar{i}{t} - \xbar{i\pm\ell}{t}, & 0<i\pm\ell\le N\\
		\xbar{i}{t} - \xbar{i\pm\ell\mp N}{t}, & \mbox{otherwise},
	\end{cases}
\end{equation}
%Such mismatches are exploited to compute the 
{and} the proportional control input is {given by}
\begin{equation}\label{eq:prop-control}
	\u{P,i}{t} = -\sum_{\ell=1}^{n}k_\ell\left(\meas{i}{\ell^+}{t-\taun}+\meas{i}{\ell^-}{t-\taun}\right),
\end{equation}
where measurements are delayed according to~\cref{ass:hypothesis}.

For networks with double integrator agents,
the control input $u_i(t)$ may also include a derivative term,
%Depending on the agent dynamics, the control input $ \u{i}{t} $
%may be purely proportional or include a derivative term, such as
\begin{equation}\label{eq:control-input-PD}
	\u{i}{t} = \gvel\u{P,i}{t} - \gvel\dfrac{d\xbar{i}{t}}{dt} = \gvel\u{P,i}{t} -\gvel\dfrac{d\x{i}{t}}{dt}.
\end{equation}
The derivative term in~\eqref{eq:control-input-PD} is delay free
because it only requires measurements coming from the agent itself,
which we assume {to be} available instantaneously. 
%The latter will be defined in due time.
The proportional input can be compactly written as $ \u{P}{t} = -K\xbar{}{t-\taun}=-K\x{}{t-\taun} $.
\revision{With ring topology, the feedback gain matrix is}
\begin{equation}\label{eq:feedback-matrix}
%	\begin{array}{c}
%		K \doteq K_f + K_f^\top \\
		K = \mathrm{circ}
		\left(\sum_{\ell=1}^nk_\ell, -k_1, \dots, -k_n, 0,  \dots, 0, -k_n, \dots, -k_1\right),
%	\end{array}
\end{equation}
where $ \mathrm{circ}\left(a_1,\dots,a_n\right) $ denotes the circulant matrix in $ \Real{n\times n} $
with elements $ a_1,\dots,a_n $ in the first row.

\revisiontwo{For agents with additive stochastic disturbances
	(see Sections \ref{sec:cont-time} and~\ref{sec:disc-time}),
	we consider the following problem for each $ n $.}

\begin{prob}\label{prob:variance-minimization}
	Design the feedback gains in order to minimize the steady-state variance of the consensus error,
%	\marginpar{\tiny Added both problems to highlight the optimization variables in the two cases.}
	\blue{\begin{subequations}\label{eq:problem}
		\begin{equation}\label{eq:variance-minimization-P}
		\mbox{P control:} \qquad \argmin_{K} \; \var(K),
		\end{equation}
		\begin{equation}\label{eq:variance-minimization-PD}
		\mbox{PD control:} \qquad \argmin_{\gvel,K} \; \var(\gvel,K),
		\end{equation}
	\end{subequations}}
	where
	\begin{equation}\label{}
	\var \doteq \lim_{t\rightarrow+\infty} \mathbb{E}\left[\lVert\x{}{t}\rVert^2\right]
	\end{equation}
	and w.l.o.g. we assume $ \mathbb{E}\left[\x{}{\cdot}\right] \equiv \mathbb{E}\left[\x{}{0}\right] = 0 $.
	%	and $ \varx{x} \stackrel{!}{=} $ if the system is mean-square unstable.
\end{prob}


\begin{document}


\title{Studying Duplicate Logging Statements and Their Relationships with Code Clones}
%\title{Studying Duplicate Logging Code Smells and Their Relationships with Code Clones}

\author{Zhenhao~Li,~\IEEEmembership{Student Member,~IEEE,}
        Tse-Hsun~(Peter)~Chen,~\IEEEmembership{Member,~IEEE,}
        Jinqiu~Yang,~\IEEEmembership{Member,~IEEE,}
        and~Weiyi~Shang,~\IEEEmembership{Member,~IEEE}
\IEEEcompsocitemizethanks{\IEEEcompsocthanksitem Z. Li, T. Chen J. Yang and W. Shang are with the Department of Computer Science and Software Engineering, Concordia University, Montreal, Quebec, Canada.\protect\\
% note need leading \protect in front of \\ to get a newline within \thanks as
% \\ is fragile and will error, could use \hfil\break instead.
E-mail: {l\_zhenha,peterc,jinqiuy,shang}@encs.concordia.ca 
}% <-this % stops an unwanted space
%\thanks{Manuscript received xxx, 2018; revised xxx, 2018.}
}

% in the abstract or keywords.
\IEEEtitleabstractindextext{%
\begin{abstract}
  In this paper, we explore the connection between secret key agreement and secure omniscience within the setting of the multiterminal source model with a wiretapper who has side information. While the secret key agreement problem considers the generation of a maximum-rate secret key through public discussion, the secure omniscience problem is concerned with communication protocols for omniscience that minimize the rate of information leakage to the wiretapper. The starting point of our work is a lower bound on the minimum leakage rate for omniscience, $\rl$, in terms of the wiretap secret key capacity, $\wskc$. Our interest is in identifying broad classes of sources for which this lower bound is met with equality, in which case we say that there is a duality between secure omniscience and secret key agreement. We show that this duality holds in the case of certain finite linear source (FLS) models, such as two-terminal FLS models and pairwise independent network models on trees with a linear wiretapper. Duality also holds for any FLS model in which $\wskc$ is achieved by a perfect linear secret key agreement scheme. We conjecture that the duality in fact holds unconditionally for any FLS model. On the negative side, we give an example of a (non-FLS) source model for which duality does not hold if we limit ourselves to communication-for-omniscience protocols with at most two (interactive) communications.  We also address the secure function computation problem and explore the connection between the minimum leakage rate for computing a function and the wiretap secret key capacity.
  
%   Finally, we demonstrate the usefulness of our lower bound on $\rl$ by using it to derive equivalent conditions for the positivity of $\wskc$ in the multiterminal model. This extends a recent result of Gohari, G\"{u}nl\"{u} and Kramer (2020) obtained for the two-user setting.
  
   
%   In this paper, we study the problem of secret key generation through an omniscience achieving communication that minimizes the 
%   leakage rate to a wiretapper who has side information in the setting of multiterminal source model.  We explore this problem by deriving a lower bound on the wiretap secret key capacity $\wskc$ in terms of the minimum leakage rate for omniscience, $\rl$. 
%   %The former quantity is defined to be the maximum secret key rate achievable, and the latter one is defined as the minimum possible leakage rate about the source through an omniscience scheme to a wiretapper. 
%   The main focus of our work is the characterization of the sources for which the lower bound holds with equality \textemdash it is referred to as a duality between secure omniscience and wiretap secret key agreement. For general source models, we show that duality need not hold if we limit to the communication protocols with at most two (interactive) communications. In the case when there is no restriction on the number of communications, whether the duality holds or not is still unknown. However, we resolve this question affirmatively for two-user finite linear sources (FLS) and pairwise independent networks (PIN) defined on trees, a subclass of FLS. Moreover, for these sources, we give a single-letter expression for $\wskc$. Furthermore, in the direction of proving the conjecture that duality holds for all FLS, we show that if $\wskc$ is achieved by a \emph{perfect} secret key agreement scheme for FLS then the duality must hold. All these results mount up the evidence in favor of the conjecture on FLS. Moreover, we demonstrate the usefulness of our lower bound on $\wskc$ in terms of $\rl$ by deriving some equivalent conditions on the positivity of secret key capacity for multiterminal source model. Our result indeed extends the work of Gohari, G\"{u}nl\"{u} and Kramer in two-user case.
\end{abstract}

% Note that keywords are not normally used for peerreview papers.
\begin{IEEEkeywords}
log, code smell, duplicate log, code clone, static analysis, empirical study.
\end{IEEEkeywords}
}

\maketitle

\IEEEdisplaynontitleabstractindextext

\IEEEpeerreviewmaketitle


% ------------Main sections------------
\IEEEraisesectionheading{\section{Introduction}\label{sec:intro}}
\input texfiles/intro
\input texfiles/prestudy
\input texfiles/pattern
\input texfiles/detection
\input texfiles/results
\input texfiles/rq4
\input texfiles/rq5
\input texfiles/threats
\input texfiles/related
\input texfiles/conclusion

% ------------reference------------
\bibliographystyle{IEEEtran}
\footnotesize
\bibliography{paper}



% ------------bio info------------
\begin{IEEEbiography}[{\includegraphics[width=1in,height=1.25in,clip,keepaspectratio]{bio/zhenhao.jpg}}]{Zhenhao Li}
Zhenhao Li is a Ph.D. student at the Department of Computer Science and Software Engineering at Concordia University, Montreal, Canada. He obtained his M.ASc degree from Concordia University and B.Eng. from Harbin Institute of Technology. His work has been published at renowned venues such as ICSE and ASE.
His research interests include software log analysis, improving logging practices, program analysis, and mining software repositories. More information at: https://ginolzh.github.io/.
\end{IEEEbiography}


\begin{IEEEbiography}[{\includegraphics[width=1in,height=1.25in,clip,keepaspectratio]{bio/peter.png}}]{Tse-Hsun (Peter) Chen}
Tse-Hsun (Peter) Chen is an Assistant Professor in the Department of Computer Science and Software Engineering at Concordia University, Montreal, Canada. He leads the Software PErformance, Analysis, and Reliability (SPEAR) Lab, which focuses on conducting research on performance engineering, program analysis, log analysis, production debugging, and mining software repositories. His work has been published in flagship conferences and journals such as ICSE, FSE, TSE, EMSE, and MSR. He serves regularly as a program committee member of international conferences in the field of software engineering, such as ASE, ICSME, SANER, and ICPC, and he is a regular reviewer for software engineering journals such as JSS, EMSE, and TSE. Dr. Chen obtained his BSc from the University of British Columbia, and MSc and PhD from Queen's University. Besides his academic career, Dr. Chen also worked as a software performance engineer at BlackBerry for over four years. Early tools developed by Dr. Chen were integrated into industrial practice for ensuring the quality of large-scale enterprise systems. More information at: https://petertsehsun.github.io/.
\end{IEEEbiography}

\begin{IEEEbiography}[{\includegraphics[width=1in,height=1.25in,clip,keepaspectratio]{bio/jinqiu.png}}]{Jinqiu Yang}
        Jinqiu Yang is an Assistant Professor in the Department of Computer Science and Software Engineering at Concordia University, Montreal, Canada. Her research interests include automated program repair, software testing, software text analytics, and mining software repositories. Her work has been published flagship conferences and journals such as ICSE, FSE, EMSE. She serves regularly as a program committee member of international conferences in Software Engineering, such as ASE, ICSE, ICSME and SANER. She is a regular reviewer for Software Engineering journals such as EMSE and JSS. Dr. Yang obtained her BEng from Nanjing University, and MSc and PhD from University of Waterloo. More information at: https://jinqiuyang.github.io/.
\end{IEEEbiography}

\begin{IEEEbiography}[{\includegraphics[width=1in,height=1.25in,clip,keepaspectratio]{bio/ian.png}}]{Weiyi Shang}
Weiyi Shang is an Assistant Professor and Concordia University Research Chair in Ultra-large-scale Systems at the Department of Computer Science and Software Engineering at Concordia University, Montreal.
He has received his Ph.D. and M.Sc. degrees from Queens University (Canada) and he obtained B.Eng.
from Harbin Institute of Technology. His research interests include big data software engineering, software
engineering for ultra-largescale systems, software log mining, empirical software engineering, and software
performance engineering. His work has been published at premier venues such as ICSE, FSE, ASE, ICSME,
MSR and WCRE, as well as in major journals such as TSE, EMSE, JSS, JSEP and SCP. His work has won
premium awards, such as SIGSOFT Distinguished paper award at ICSE 2013 and best paper award at WCRE
2011. His industrial experience includes helping improve the quality and performance of ultra-large-scale
systems in BlackBerry. Early tools and techniques developed by him are already integrated into products
used by millions of users worldwide. Contact him at shang@encs.concordia.ca; \url{https://users.encs.concordia.ca/~shang}.
\end{IEEEbiography}


\clearpage

\appendices


\section{Precision of NiCad on Detecting Duplicate Logging Statements that Reside in Cloned Code}
\label{sec:appendix2}
We rely on NiCad for automated clone detection. To examine the false positives of NiCad, we then manually verify a randomly sampled set of duplicate logging statements (281 sets in total, with 95\% confidence level and 5\% confidence interval) that are classified as clones by NiCad. For each set of the sampled duplicate logging statements, we manually go through the logging statements and their surrounding code to verify whether they are clones or not. Overall, we find that 272 out of the 281 sampled sets (96.8\%) are clones, which is similar to the performance of NiCad that is reported in prior studies. For the 9 false positives, 3 of them are duplicate logging statements located in different branches of a nested method (i.e., developers define a method within a method). In such cases, NiCad would analyze the code block twice. For example, in ElasticSearch~\footnote{\url{https://github.com/elastic/elasticsearch/blob/70b8d7bc64f165735502de9d8c5fa673fa21e02b/server/src/main/java/org/elasticsearch/cluster/InternalClusterInfoService.java}}, two duplicate logging statements with the same static text message {\em ``Failed to execute NodeStatsAction for ClusterInfoUpdateJob''} are located in different branches of the same nested method {\em onFailure(Exception e)}. However, since the method {\em onFailure(Exception e)} is defined in the method {\em refresh()}, NiCad would analyze the same code block twice and detect them as clones. For the remaining 6 out of 9 false positives, we could not identify the reasons that they are classified as clones, since the code snippets look neither structurally nor semantically similar.



\end{document}
