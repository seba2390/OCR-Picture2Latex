%\section{RQ4: What are the Relationships between Duplicate Logging Statements and Code Clone?}

\section{RQ5: What are the Relationships Between Duplicate Logging Statements and Code Clones?}
\label{sec:rq5}


\phead{Motivation. }
In Section~\ref{sec:clone}, we investigate the relationship between code clones and {\em problematic instances of duplicate logging code smells}. As discussed in Section~\ref{sec:manual}, duplicate logging code smells are duplicate logging statements with specific patterns that may be indications of logging problems. In this section, we further investigate the relationship between {\em duplicate logging statements} and code clones. We also study the potential impact of duplicate logging statements on detecting code clones. 


\phead{Approach. }
Similar to Section~\ref{sec:clone}, we use both an automated and a manual approach to study the relationship between code clones and duplicate logging statements. We first leverage NiCad to automatically detect clones. Although we found that NiCad has a great precision (i.e., 96.8\%, as shown in Appendix~\ref{sec:appendix2}), there may still exist false negatives (i.e., the duplicate logging statements are code clones, but are missed by the tool). Therefore, we manually investigate a statistical sample of duplicate logging statements, which reside in code snippets that are classified by NiCad as {\em Non-clones} to study the false negative rate. 


\noindent{\bf Results of automated code clone analysis on duplicate logging statements.} {\em We find that a considerable number of duplicate logging statements (43.7\% on average) reside in cloned code snippets.}
Table~\ref{table:RQ4} presents the results of our code clone analysis. {\sf DupSet} refers to the total sets of duplicate logging statements (a set contains two or more logging statements with the same text message). {\sf CloneSet} refers to the subset of duplicate logging statement sets ({\sf DupSet}) that are from cloned code snippets. The percentage number is the proportion of {\sf CloneSet} out of {\sf DupSet}. Finally, {\sf Avg. Sim.} refers to the average code clone similarity score among the cloned code snippets. As shown in Table~\ref{table:RQ4}, 11.5\% to 51.1\% sets of duplicate logging statements are from the cloned code snippets in the studied systems. Overall, 1,042 out of 2,382 (43.7\%) sets of duplicate logging statements are related to code clones (with an average 80\% similarity score).

Our finding shows that a considerable number of duplicate logging statements are related to code clones, and developers may not change the log messages when they copy a piece of code to another location.
However, due to the importance of logging for understanding system runtime behaviour~\cite{petericseseip2017, Yuan:2012:CLP:2337223.2337236, Yuan:2011:ISD:1950365.1950369}, developers should avoid directly copying logging statements. Developers should consider modifying the log messages (e.g., to include the class name, modify the message to reflect code changes, or record new important dynamic variables) to assist debugging and workload understanding.




\noindent{\bf Results of manual code clone analysis on duplicate logging statements.} {\em We find that more than 50\% of the sampled duplicate logging statements reside in cloned code snippets that are difficult to detect using automated code clone detection tools. In particular, 24.5\% of the manually studied duplicate logging statements are related to code clones, and 26.2\% are related to micro-clones. }
In total, we randomly sample 298 sets of duplicate logging statements to achieve a confidence of 95\% and a confidence interval of 5\%. For each set of the sampled duplicate logging statements, we manually classify them into three types: {\em Clone} (i.e., but not detected by code clone detection tools), {\em Micro-clone} (i.e., code blocks with less than 10 lines of code), and {\em Non-clone}.


Table~\ref{table:manualrq4} presents the results of our manual study. Overall, 73 out of the 298 (24.5\%) manually-studied sets of duplicate logging statements are labeled as {\em Clones}. 78 out of 298 (26.2\%) sets are labeled as {\em Micro-clones}. The remaining 147 out of 298 (49.3\%) sets are labeled as {\em Non-clones}. For 42 out of the 73 cases of {\em Clones}, and 32 out of 78 cases of {\em Micro-clones}, we find that developers often only copy and paste part of the code into another large code block (Category 1 discussed in Section~\ref{sec:clone}). Hence, only small parts of large code blocks are similar, which reduces the similarity score. For 53 out of the 73 cases that are manually identified as {\em Clones}, they reside in code with very similar semantics but have minor differences (Category 2). Note that some cases belong to multiple categories. For 46 out of 78 cases they are classified as {\em Micro-clones}, which reside in very short methods with only a few lines of code (Category 3).




In summary, we find that more than half of the duplicate logging statements reside in cloned code snippets. Our manual study also highlights that many duplicate logging statements reside in cloned code that may be difficult to detect by clone detection tools.



\begin{table}
    \caption{Automated code clone analysis results on duplicate logging statements.}


    \vspace{-0.3cm}
    \centering
    \resizebox{\columnwidth}{!} {
    \tabcolsep=18pt
    \scalebox{0.8}{

    \begin{tabular}{l|c|c|c}

        \toprule
        & \multicolumn{1}{c|}{\textbf{DupSet}} & \multicolumn{1}{c|}{\textbf{CloneSet}}  & \multicolumn{1}{c}{\textbf{Avg. Sim.}} \\

        \midrule
        \textbf{Cassandra}     & 46  & 14 (30.4\%) & 79.7  \\
        \textbf{CloudStack}     & 865  & 442 (51.1\%) & 80.3  \\
        \textbf{Elasticsearch}  & 40  & 17 (42.5\%) & 72.2  \\
        \textbf{Flink}         & 203  & 92 (45.3\%) & 78.8 \\
        \textbf{Hadoop}         & 217  & 25 (11.5\%) & 76  \\
        

        \textbf{Camel}         & 886  & 421 (47.5\%) & 80.7  \\
        \textbf{Kafka}         & 104  & 23 (22.1\%) & 75.4  \\
        \textbf{Wicket}         & 21  & 8 (38.1\%) & 83.1 \\
\midrule
        \textbf{Overall}         & 2,382   & 1,042 (43.7\%) & 80.0  \\
        \bottomrule
    \end{tabular}
    }
    }
    \noindent {\sf DupSet}: Total sets of duplicate logging statements, {\sf CloneSet}: Sets of duplicate logging statements that are from cloned code snippets,  {\sf Avg. Sim.}: Average similarity of the cloned code snippets.
    \vspace{-0.3cm}
\label{table:RQ4}
\end{table}


\begin{table}
  \caption{Manual study results on the recall of clone detection tool on duplicate logging statements. Both the {\sf Clones} and {\sf Micro-clones} are labeled manually and they are not detected by the clone detection tool. }
    \vspace{-0.3cm}
    \centering
    %\resizebox{\textwidth}{!} {
    \tabcolsep=8pt
    \scalebox{0.9}{
    \begin{tabular}{l|c|c|c|c}

        \toprule
     &{\textbf{Clones}} &{\textbf{Micro-clones}} &{\textbf{Non-clones}} &{\textbf{Total}}  \\
\midrule

        \textbf{Cassandra}    & 1 & 3 & 3 & 7  \\
        \textbf{CloudStack}    & 22 & 26 &46  & 94 \\
        \textbf{Elasticsearch} & 1 & 1 & 3 & 5 \\
        \textbf{Flink}       & 5 & 4 & 16 & 25\\
        \textbf{Hadoop}      & 12 & 6 & 25  & 43 \\


       \textbf{Camel}       & 28 & 30 & 45 & 103 \\
       \textbf{Kafka}       & 3 & 7 & 8 & 18 \\
       \textbf{Wicket}       & 1 & 1 & 1 & 3 \\

        \midrule

        \textbf{Total}       & 73 &78 & 147 & 298 \\



        \bottomrule
    \end{tabular}}
    %}
    \vspace{-0.3cm}

    \label{table:manualrq4}
\end{table}





\noindent{\bf Discussion: The Potential Impact of Duplicate Logging Statements on Detecting Code Clones.}
\label{sec:appendix3}
In this RQ, we find that a noticeable number of duplicate logging statements reside in cloned code snippets. %In order to seek whether duplicate logging statements helps detectthe potential of improving code clone detection using duplicate logging statements, 
We further investigate the impact of duplicate logging statements on the detection of code clones, namely, whether considering duplicate logging statements helps detect code clones. Specifically, for each set of 
{\sf CloneSet} presented in Table~\ref{table:RQ4}, we first remove the duplicate logging statements from the related code snippets. We then re-examine how many code snippets related to prior {\sf DupSet} are still identified as cloned code snippets and how many are not, by using NiCad.


Table~\ref{table:appendixc} shows the results of our experiments on investigating the impact of duplicate logging statements on detecting code clones. {\sf CloneSet} refers to the sets of cloned code snippets with duplicate logging statements. {\sf CloneSet-NDL} refers to the sets of cloned code snippets after removing the related duplicate logging statements. {\sf CloneSet-Reduced} represents the number of sets reduced by comparing {\sf CloneSet-DL} with {\sf CloneSet-NDL}. {\sf Per. Reduced} shows the percentage of {\sf CloneSet-Reduced} given {\sf CloneSet-DL}. On average, 28.4\% {\sf CloneSet} are not detected by NiCad as cloned code snippets after removing duplicate logging statements. Specifically for each studied system, the reduction ranges from 25.0\% in Wicket to around a 47.1\% in Elasticsearch. 


We then manually investigate the code snippets that are not detected as cloned code snippets after removing duplicate logging statements (i.e., {\sf CloneSet-Reduced}). We find two potential reasons that the clone detection tool could not detect them as cloned code snippets. 1) {\em Reduced total lines of similar code after removing duplicate logging statements:} The logging statements usually span across one to three, and sometimes even more, lines of code. However, these lines of code in the duplicate logging statements are the main part of the clones. After removing the duplicate logging statements, the total number of similar lines of code snippets is too small for a clone detection tool to consider as clones. 2) {\em Reduced similarity after removing duplicate logging statements:} Duplicate logging statements have exactly the same log message and are represented as Method Invocation nodes in the Abstract Syntax Tree. Removing duplicate logging statements will decrease the similarity of code snippets, both syntactically and semantically. Hence, the similarity might become smaller than the threshold of the clone detection tool and the code snippets are not detected as clones.

In summary, we find that a large portion of the cloned code snippets with duplicate logging statements (from 25.0\% to 47.1\%) are not detected as cloned code snippets after removing the duplicate logging statements. The results show that duplicate logging statements have a non-negligible impact on the detection of code clones. Future code clone studies may consider the effect of logging code in order to further improve the code clone detection techniques.

\rqboxc{More than half of the duplicate logging statements reside in cloned code snippets, and a large portion of them reside in short code blocks which are difficult to detect using existing code clone detection tools. We also find that duplicate logging statements have a non-negligible impact on helping the detection of code clones. Future works may leverage duplicate logging statements to further improve code clone detection tools.
}



\begin{table}
  \caption{The results of investigating the impact of duplicate logging statements on detecting code clones. }
    \vspace{-0.2cm}
    \centering
    \resizebox{\columnwidth}{!} {
    \tabcolsep=5pt
    \renewcommand\arraystretch{1.1}
    \scalebox{1.0}{
    \begin{tabular}{l|c|c|c|c}

        \toprule
     &{\textbf{CloneSet}} &{\textbf{CloneSet-NDL}} &{\textbf{CloneSet-Reduced}} &{\textbf{Per. Reduced }}  \\
\midrule

        \textbf{Cassandra}    & 14 & 10 & 4 & 28.6\%  \\
        \textbf{CloudStack}    & 442 & 329 & 113  & 25.6\% \\
        \textbf{Elasticsearch} & 17 & 9 & 8 & 47.1\% \\
        \textbf{Flink}       & 92 & 64 & 28 & 30.4\%\\
        \textbf{Hadoop}      & 25 & 16 & 9  & 36.0\% \\

       \textbf{Camel}       & 421 & 299 & 122 & 29.0\% \\
       \textbf{Kafka}       & 23 & 13 & 10 & 43.5\% \\
       \textbf{Wicket}       & 8 & 6 & 2 & 25.0\% \\

        \midrule

        \textbf{Total}       & 1042 &746 & 296 & 28.4\% \\



        \bottomrule
    \end{tabular}}
    }
    \vspace{-0.5cm}

    \label{table:appendixc}
\end{table}