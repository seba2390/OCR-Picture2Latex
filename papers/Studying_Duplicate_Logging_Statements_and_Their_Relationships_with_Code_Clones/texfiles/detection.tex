\section{DLFinder: Automatically Detecting Problematic Duplicate Logging Code Smells}
\label{sec:detection}



Section~\ref{sec:manual} uncovers five patterns of duplicate logging code smells, and provides guidance in identifying {\em problematic} logging code smells. 
To help developers detect such problematic code smells and improve logging practices, we propose an automated approach, specifically a static analysis tool, called \tool. \toolS uses abstract syntax tree (AST) analysis, data flow analysis, and text analysis. Note that we exclude the detection result of IL (i.e., inconsistent log level) in this study, since based on the feedback from developers, none of the IL instances are problematic. Below, we discuss how \toolS detects each of the four patterns of duplicate logging code smell (i.e., IC, IE, LM, and DP).



\phead{Detecting inadequate information in catch blocks (IC).}
%\noindent{\bf {\em Detection Approach.}}
\toolS first locates the {\em try-catch} blocks that contain duplicate logging statements. Specifically, \toolS finds the {\em catch} blocks of the same {\em try} block that catch different types of exceptions, and these {\em catch} blocks contain the same set of duplicate logging statements. Then, \toolS uses data flow analysis to analyze whether the handled exceptions in the {\em catch} blocks are logged (e.g., record the exception message). \toolS detects an instance of IC if none of the logging statements in the {\em catch} blocks record either the stack trace or the exception message.



\phead{Detecting inconsistent error-diagnostic information (IE).}
%{\em Detection Approach.}
\toolS first identifies all the {\em catch} blocks that contain duplicate logging statements. Then, for each {\em catch} block, \toolS uses data flow analysis to determine how the exception is logged by analyzing the usage of the exception variable in the logging statement. Namely, the logging statement records 1) the entire stack trace, 2) only the exception message, or 3) nothing at all.
%If there is another log line with the same message in other \texttt{catch} blocks (i.e., duplicate log), \toolS would compare how the exception variable is logged.
Then, \toolS compares how the exception variable is used/recorded in each of the duplicate logging statements.
\toolS detects an instance of IE if a set of duplicate logging statements that appear in {\em catch} blocks has an inconsistent way of recording the exception variables (e.g., the log in one {\em catch} block records the entire stack trace, and the log in another {\em catch} block records only the exception message, while the two catch blocks handle the same type of exception). Note that for each instance of IE, the multiple {\em catch} blocks with duplicate logging statements in the same set may belong to different {\em try} blocks. In addition, \toolS decides if an instance of IE can be excluded if it belongs to one of the three justifiable cases (IE.1--IE.3) by checking the exception types, if the duplicate logging statements are in the same {\em catch} block, and if developers pass the exception variable to another method. %If so, the instance is marked as potentially justifiable, and thus, excluded by \tool. %Finally, \toolS identifies the inconsistent in calls to exception variables in duplicate logs.

\vspace{-0.1cm}
\phead{Detecting log message mismatch (LM)}. LM is about having an incorrect method or class name in the log message (e.g., due to copy-and-paste). Hence, \toolS analyzes the text in both the log message and the class-method name (i.e., concatenation of class name and method name) to detect LM by applying commonly used text analysis approaches~\cite{Chen:2016:SUT:2992358.2992444}. \toolS detects instances of LM using four steps: 1) For each logging statement, \toolS splits class-method name into a set of words (i.e., {\em name set}) and splits log message into a set of words (i.e., {\em log set}) by leveraging naming conventions (e.g., camel cases) and converting the words to lower cases. 2) \toolS applies stemming on all the words using Porter Stemmer~\cite{stemmer}. 3) \toolS removes stop words in the log message. We find that there is a considerable number of words that are generic across the log messages in a system (e.g., on, with, and process). Hence, we obtain the stop words by finding the top 50 most frequent words (our studied systems has an average of 3,352 unique words in the static text messages) across all log messages in each system~\cite{Yang:2014:SIS:2683115.2683138}. 4) For every logging statement, between the name set (i.e., from the class-method name) and its associated log set, \toolS counts the number of common words shared by both sets. Afterward, \toolS detects an instance of LM if the number of common words is inconsistent among the duplicate logging statements in one set. 

For the LM example shown in Table~\ref{tab:patterns}, the common words shared by the first pair (i.e., method {\em doScaleUp()} and its log) are ``scale, up'', while the common word shared by the second pair is ``scale''. Hence, \toolS detects an LM instance due to this inconsistency.
%Note that we ignore the cases where the method only has one word, since we find that such method names are usually generic names such as {\tt run()} and {\tt execute()}.
% \toolS detects a potential instance of LM if the number of common words is inconsistent among the same set of duplicate logs.
The rationale is that the number of common words between the class-method name and the associated logging statement is subject to change if developers make copy-and-paste errors on logging statements (e.g., copy the logging statement in {\em doScaleUp()} to method {\em doScaleDown()}), but forget to update the log message to match with the new method name ``doScaleDown''.
However, the number of common words will remain unchanged (i.e., no inconsistency) if the logging statement (after being pasted at a new location) is updated respectively.



%\phead{Detecting inconsistent log level (IL)}.
%\toolS detects an instance of IL if duplicate logging statements in one set (i.e., have the same static text message) have inconsistent log level. Furthermore, \toolS checks whether an instance of IL belongs to one of the three justifiable cases (IL.1--IL.3) and is justifiable by checking the exception types, if the logging statements are in different branches of the same method, and if developers pass the exception variable to another method in the {\em catch} block. 

\vspace{-0.1cm}
\phead{Detecting duplicate logs in polymorphism (DP)}.
%{\em Detection Approach.}
\toolS generates an object inheritance graph when statically analyzing the Java code. For each overridden method, \toolS checks if there exist any duplicate logging statements in the corresponding method of the sibling and the parent class. If there exist such duplicate logging statements, \toolS detects an instance of DP. Note that, based on the feedback that we received from developers (Section~\ref{sec:manual}), we do not expect developers to fix instances of DP. DP can be viewed more as technical debts~\cite{Kruchten:2012:TDM:2412381.2412847} and our goal is to propose an approach to detect DP to raise the awareness from the research community and developers regarding this issue.

\vspace{-0.1cm}


