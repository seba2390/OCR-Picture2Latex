\section{Conclusion}
\label{sec:conclusion}
Duplicate logging statements may affect developers' understanding of the system execution. In this paper, we study over 4K duplicate logging statements in five large-scale open source systems (Hadoop, CloudStack, Elasticsearch, Cassandra and Flink). We uncover five patterns of duplicate logging code smells. Further, we assess the impact of each uncovered code smell and find not all are problematic and need fixes. In particular, we find six justifiable cases where the uncovered patterns of duplicate logging code smells may not be problematic. We received confirmation from developers on both the problematic and justifiable cases.
%We reported both the problematic and justifiable cases to developers. All of the reported problematic instances are now fixed by developers, and developers agreed and acknowledged the uncovered justifiable cases.
Combining our manual analysis and developers' feedback, we developed a static analysis tool, \tool, which automatically detects problematic duplicate logging code smells.
We applied \toolS on the five manually studied systems and three additional systems. In total, we reported 91 problematic duplicate logging code smell instances in the eight studied systems to developers and all of them are fixed. \toolS successfully detects 81 out of the 91 instances.
We further investigate the relationship between duplicate logging statements and code clones, in order to provide a more comprehensive understanding of duplicate logging statements and duplicate logging code smells. We find that most of the problematic instances of duplicate logging code smells and almost half of the duplicate logging statements reside in cloned code snippets. Among them, a large portion reside in very short code blocks which might be difficult to detect using existing code clone detection tools.

Our study highlights the importance of the context of the logging code, i.e., the nature of logging code is highly associated with both the structure and the functionality of the surrounding code. Future studies should consider the code context when providing guidance to logging practices, more advanced logging libraries are needed to help developers improve logging practice and to avoid logging code smells. Our findings also provide an initial evidence on the prevalence of duplicate logging statements that reside in cloned code snippets, and the potential impact of code clones on logging practices. Future studies may also consider integrating different information in the software artifacts (e.g., duplicate logging statements) to further improve clone detection results.%, \zhenhao{Keep the following sentence or not?} and may highlight the potential of using duplicate logging statements to further improve clone detection tools.



%In this paper, we study duplicate logs in four large-scale open source systems (Hadoop, CloudStack, ElasticSearch, and Cassandra). We find five patterns of duplicate logging code smells based on our manual study. In addition, we find that not all code smell instances are problematic and need fixes. In particular, we find six justifiable cases where the uncovered patterns of duplicate logging code smells may not be problematic. We reported both the problematic code smell instances with fix suggestions and the uncovered justifiable cases to developers. All of the reported instances and fix suggestions are accepted. Moreover, developers agreed and acknowledged the justifiable cases that we uncovered. %the uncovered rationales of having duplicate logs are agreed and acknowledge by developers.
%By incorporating developers' feedbacks, we developed a tool (DLFinder) to automatically detect problematic duplicate logs.



%the special need of developers when intentionally design logging code in a suboptimal manner.

%Prior studies~\cite{log_pattern_ICSE2017, mehran_emse_2018, Yuan:2012:CLP:2337223.2337236} assess the impact of logging code smells by only looking at the logging code itself (i.e., log message and log level). However, we find that the impact of logging code smells is highly associated with both {\em the semantic and syntactic context} (e.g., the structure and the functionality of the surrounding code). Future studies should consider the code context information when refactoring or detecting logging code smells.



%have uncovered five duplicate logging code smells by In order to identify potential instances of duplicate logging code smells,\todo{the following sentences should be changed due to the change of structure} we implement a static analysis tool, namely DLFinder, and further manually classify all instances of the detected results into sub-categories which have different level of potential impacts based on the context of the logging code. Then we report all the detected instances that belongs to the  potentially problematic sub-categories, and a few detected instances that belong to each potentially non-problematic sub-category to developers. The feedbacks from developers and the results of our manual study are integrated for improving DLFinder.


%\ian{moved here from intro}
