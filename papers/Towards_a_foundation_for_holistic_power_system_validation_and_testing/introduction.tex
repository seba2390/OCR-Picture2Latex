Efforts to reduce green-house gas emissions have already had a strong effect on the power system. Integration of renewable energy sources (RES), flexible loads and storage systems into the power system has increased steadily over the past years. 
This has introduced challenges to power system operators due to the fluctuating nature of RES, increased complexity in the system, and heterogeneous components \cite{SET}.
The increased implementation of advanced automation and information and communication technologies (ICT) %as well as methods for operating power systems under the changed conditions 
are transforming the power system into a cyber-physical system which integrates infrastructures of different domains -- a smart grid \cite{SmartGridsPath, SmartGridsRoadmap}.
As such, it is necessary to implement integrated solutions for operating the system that fulfil high-reliability, real-time or regulatory requirements, just to name a few.
Before deployment such solutions have to be validated and tested.
Until now, only certain aspects with a main focus on components are tested \cite{bruendlinger2015}.
However, in order to support the different stages of the overall development process for smart grid solutions, tests in a holistic manner are needed, i.e. integrating different domains on system level \cite{CIGRE}.

%% \cm{I think the last half of the sentence basically says the same thing twice}
The project ERIGrid\footnote{https://www.erigrid.eu/}
addresses the challenging aims of a holistic system testing approach for smart grids %with a holistic, cyber-physical systems based approach. This is being done 
by creating a platform and methodology for integrating 18 European research centres and institutions.
%
A holistic testing methodology allows tests and experiments representative of integrated smart grids to be conducted through testing and experimentation across distributed research infrastructures (RI) which may not necessarily be functionally interconnected.

The paper is outlined as follows.
In Section~\ref{sec:concept}, the concept of holistic testing and its advantages are briefly summarised. % and the general approach of the ERIGrid holistic testing concept is presented.
In Section~\ref{sec:approach} the general approach of the ERIGrid holistic testing concept is presented. %discusses required refinements of the methodology towards a holistic test procedure, including test case definition as well as mapping onto sub-tests.
%the mapping process that separates the holistic test to conduct and combine \hl{individual test results} to draw conclusion on the holistic test.
In Section~\ref{sec:results} the current status and preliminary results on state of the art and first concepts for test case description are presented, which are explained on an example in Section~\ref{sec:example}. 
The paper finishes with the conclusions in Section~\ref{sec:conclusion}.