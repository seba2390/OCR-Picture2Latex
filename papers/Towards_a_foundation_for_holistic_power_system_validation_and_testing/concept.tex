A holistic smart grid research and development approach not only addresses the whole development cycle (design, analysis, simulation, experimentation, testing and deployment), but must also take into account all relevant components, facets, influences that future power systems will comprise of, all of which may affect the controller, algorithm(s), or use case in question.
Testing highly integrated systems without taking into account possible disturbances by users, markets, ICT availability, etc, is invalid. Formal analysis of these vastly complex, integrated systems is not (yet -- if at all) possible. Hence, rigorous testing strategies are required that allow for the validation of integrated systems of different domains represented at different RI. Due to the importance of the system at hand and the immaturity of controllers, applications, and hardware, real-world embedded field tests are, in many cases, out of the question.
Although a functional integration of the aforementioned RI running in parallel and yielding integrated holistic energy systems is theoretically possible it remains practically infeasible.
In order to be capable of conducting tests and experiments representative of integrated smart grid systems, testing and experimentation must be possible across distributed and not necessarily functionally interconnected RI.

The outcomes of experiments at different RI are dependent on each other and must be analysed in an integrated way. 
ERIGrid proposes an approach for realizing a %sustainable and cost efficient 
holistic procedure for smart grid system validation to support comparability between experiments of different setup and design, 
thus facilitating subsequent re-utilization of experimental results from different stakeholders through consecutive, %continuative
sequential, and parallel experiments. 

In the following section a cyber-physical energy system based procedure for holistic testing is proposed.