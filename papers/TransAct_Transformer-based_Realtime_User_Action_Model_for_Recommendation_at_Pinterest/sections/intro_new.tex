\section{Introduction}

% Why’s rec sys important

The proliferation of online content in recent years has created an overwhelming amount of information for users to navigate. 
To address this issue, recommender systems are employed  in various industries to help users find relevant items from a vast selection, including products, images, videos, and music. 
% , making it easier for them to find what they want. 
By providing personalized recommendations, businesses and organizations can better serve their users and keep them engaged with the platform. 
Therefore, recommender systems are vital for businesses as they drive growth by boosting engagement, sales, and revenue.

\begin{figure}[!ht]
  \centering
  \includegraphics[width=0.85\linewidth]{figures/Homefeed_web_square.png}
  % \caption{Illustration of generating Pinterest Homefeed page using a 3-stage RS}
  \caption{Pinterest Homefeed Page}
  \label{fig:hf}
  \Description{Pinterest Homefeed page}
\end{figure}

% Background of PINS Homefeed ranking. Previous PINS work (PinSage, Pinnerformer)
As one of the largest content sharing and social media platforms, Pinterest hosts billions of pins with rich contextual and visual information, and brings inspiration to over 400 million users. 
Upon visiting Pinterest, users are immediately presented with the Homefeed page as shown in Figure~\ref{fig:hf}, which serves as the primary source of inspiration and accounts for the majority of overall user engagement on the platform.
The Homefeed page is powered by a 3-stage recommender system that retrieves, ranks, and blends content based on user interests and activities. 
At the retrieval stage, we filter billions of pins created on Pinterest to thousands, based on a variety of factors such as user interests, followed boards, etc. Then we use a pointwise ranking model to rank candidate pins by predicting their personalized relevance to users. 
Finally, the ranked result is adjusted using a blending layer to meet business requirements.


% Existing work for realtime ranking
Realtime recommendation is crucial because it provides a quick and up-to-date recommendation to users, improving their overall experience and satisfaction. 
The integration of realtime data, such as recent user actions, results in more accurate recommendations and increases the probability of users discovering relevant items~\cite{alibaba_seq_tfmr, pi2020search}.

Longer user action sequences result in improved user representation and hence better recommendation performance. 
However, using long sequences in ranking poses challenges to infrastructure, as they require significant computational resources and can result in increased latency.
To address this challenge, some approaches have utilized hashing and nearest neighbor search in long user sequences~\cite{pi2020search}.
% Research in this area typically focuses on reducing the complexity of the system\cite{chatzopoulos2016readme}, handling high-velocity data streams\cite{6542425}, and dealing with dynamic changes in user behavior~\cite{grbovic2018real, 10.1145/2699670}. 
Other work encodes users' past actions over an extended time frame to a user embedding~\cite{pinnerformer} to represent long-term user interests. User embedding features are often generated as \textit{batch} features (e.g. generated daily), which are cost-effective to serve across multiple applications with low latency. 
The limitation of existing sequential recommendation is that they either only use realtime user actions, or only use a batch user representation learned from long-term user action history.


We introduce a novel realtime-batch hybrid ranking approach that combines both \textit{realtime} user action signals and \textit{batch} user representations. 
To capture the realtime actions of users, we present TransAct - a new transformer-based module designed to encode recent user action sequences and comprehend users' immediate preferences.
For user actions that occur over an extended period of time, we transform them into a batch user representation~\cite{pinnerformer}.

By combining the expressive power of TransAct with batch user embeddings, the hybrid ranking model offers users realtime feedback on their recent actions, while also accounting for their long-term interests. The realtime component and batch component complement each other for recommendation accuracy. This leads to an overall improvement in the user experience on the Homefeed page.
% Limitation of previous sequential recommendation/ novelty
% Sequential recommendation systems use a user's action history as input and apply recommendation algorithms to suggest appropriate items. Recent work~\cite{donkers2017sequential, hidasi2015session, tan2016improved, zhou2019deep, tang2018personalized, tuan20173d} has been using deep learning techniques, such as recurrent neural networks (RNNs)~\cite{rnn} and convolutional neural networks (CNNs)~\cite{cnn}, to process users' action history. Some studies~\cite{DIN, zhang2019next, alibaba_seq_tfmr, li2020time, SASRec, sun2019bert4rec} have also adopted the attention mechanism~\cite{tfmr} to model user action sequence features. 
% In this work, we also employ the self-attention mechanism for realtime user sequence modeling for its superior capability in encoding sequential inputs. 
% % One limitation of the existing sequential recommendation is that they
% To the best of our knowledge, this is the first method that builds a realtime-batch hybrid ranking model which uses users' action sequences in both realtime and batch settings. 
% We provide an in-depth analysis of the challenges posed by utilizing real-time user sequence features, such as the potential decrease in recommendation diversity and engagement decay.

The major contributions of this paper are summarized as follows:
\begin{itemize}
 \item We describe Pinnability, the architecture of Pinterest's Homefeed production ranking system. The Homefeed personalized recommendation product accounts for the majority of the overall user engagement on Pinterest. 

 \item  We propose TransAct, a transformer-based realtime user action sequential model that effectively captures users' short-term interests from their recent actions. We demonstrate that combining TransAct with daily-generated user representations~\cite{pinnerformer} to a hybrid model leads to the best performance in Pinnability. This design choice is justified through a comprehensive ablation study. Our code implementation is publicly available\footnote{Our code is available on Github: \url{https://github.com/pinterest/transformer_user_action}}.

\item We describe the serving optimization implemented in Pinnability to make feasible the computational complexity increase of 65 times when introducing TransAct to the Pinnability model. Specifically, optimizations are done to enable GPU serving of our prior CPU-based model.
\item We describe online A/B experiments on a real-world recommendation system using TransAct. We demonstrate some practical issues in the online environment, such as recommendation diversity drop and engagement decay, and propose solutions to address these issues.  
% \item Our model has been deployed as the Homefeed ranking model of Pinterest, one of the largest content sharing and social media platforms. As a result, it boosts the Homefeed repin\footnote{A "repin" on Pinterest refers to the action of saving an existing pin to another board by a user.} volume by 11\%.

\end{itemize}

% Structure of the paper
% \td{check}
The remainder of this paper is organized as follows: Related
work is reviewed in Section~\ref{sec:related_work}. Section~\ref{sec:method} describes the design of TransAct and the details of bringing it to production. Experiment results are reported in Section~\ref{sec:exp}. We discuss some findings beyond experiments in Section~\ref{sec:discussion}. Finally, we conclude our work in Section~\ref{sec:conclusion}.
% Learnings









% \begin{itemize}
% \item Hybrid setup
% \item Ablation study: Using both positive and negative actions, Early fusion with candidate pin
% \item Diversity: time window mask
% \item Frequent retraining
% \item is good for cold-start users (non-core)
% \end{itemize}


% 1. Hybrid setup
% w. No extensive ablation study on the model architecture Model: early fusion, random time window.
% Did not discuss practical challenge: latency, GPU serving
% Ali’s Search based user seq does have practical challenge
% id feature only (but we don’t have img_sig feature)	
% 3. Diversity


% Most existing recommender systems use a wide and deep architecture, where the inputs are user features, item features and context features. The model usually learns the relevance score between the user and item over a given context. The traditional way to get user or item features are mostly through feature engineering. And then the model use the hand-picked features to predict the relevance between user and items. Raw features are usually very limited in the amount of information that they can represent. More advanced techniques include learning pre-trained item or user embedding, where it can use a reasonable amount of space to represent richer information. Some successful pre-trained embeddings are Pinnerformer\cite{pinnerformer}, itemsage\cite{itemsage}, etc. However, one of the major shortcomings of pre-trained embedding is that they are usually expensive to infer in realtime due to their high computational cost and infra complexity. 