%%
%% This is file `sample-sigconf.tex',
%% generated with the docstrip utility.
%%
%% The original source files were:
%%
%% samples.dtx  (with options: `sigconf')
%% 
%% IMPORTANT NOTICE:
%% 
%% For the copyright see the source file.
%% 
%% Any modified versions of this file must be renamed
%% with new filenames distinct from sample-sigconf.tex.
%% 
%% For distribution of the original source see the terms
%% for copying and modification in the file samples.dtx.
%% 
%% This generated file may be distributed as long as the
%% original source files, as listed above, are part of the
%% same distribution. (The sources need not necessarily be
%% in the same archive or directory.)
%%
%%
%% Commands for TeXCount
%TC:macro \cite [option:text,text]
%TC:macro \citep [option:text,text]
%TC:macro \citet [option:text,text]
%TC:envir table 0 1
%TC:envir table* 0 1
%TC:envir tabular [ignore] word
%TC:envir displaymath 0 word
%TC:envir math 0 word
%TC:envir comment 0 0
%%
%%
%% The first command in your LaTeX source must be the \documentclass
%% command.https://www.overleaf.com/project/6348af571c7b08d1b4af9acb
%%
%% For submission and review of your manuscript please change the
%% command to \documentclass[manuscript, screen, review]{acmart}.
%%
%% When submitting camera ready or to TAPS, please change the command
%% to \documentclass[sigconf]{acmart} or whichever template is required
%% for your publication.
%%
%%
% \documentclass[sigconf]{acmart}
\documentclass[sigconf]{acmart}
\usepackage{amsmath,amssymb,amsfonts}
\usepackage{mathtools}
%\usepackage{amsthm}
% \usepackage[pdfpagelabels=true,linktocpage]{hyperref}
\usepackage[usenames,dvipsnames]{color}
\usepackage{xcolor}
\usepackage{xspace}
\usepackage{microtype}
\usepackage{todonotes}
\usepackage{colonequals}
%\usepackage{wasysym}
 \usepackage{booktabs}



\usepackage{multirow}
\usepackage{multicol}

\usepackage{paralist}

\usepackage{subfigure}
\usepackage{url}
\usepackage{appendix}
\usepackage{graphicx}
%\usepackage{float}
%\usepackage[pdfpagelabels=true]{hyperref}
\usepackage{algorithm}
\usepackage{listings}
%\usepackage{newalg}
\usepackage{nicefrac}
\usepackage{tikz} 
\usepackage{pgfplots}
%\usepgfplotslibrary{external}
%\pgfplotsset{compat=1.8}
%\tikzexternalize[prefix=TikzPictures/]
%\usepackage{marvosym}
%\usepackage{wrapfig}




\lstset{
	basicstyle=\ttfamily,
    keywordstyle=\bfseries,
    showstringspaces=false,
    numbers=left,
    numberstyle=\tiny,
    morekeywords={}
}

\usetikzlibrary{arrows,decorations.pathmorphing,positioning,fit,trees,shapes,shadows,automata,calc} 
%\usetikzlibrary{patterns,arrows,arrows.meta,calc,shapes,shadows,decorations.pathmorphing,decorations.pathreplacing,automata,shapes.multipart,positioning,shapes.geometric,fit,circuits,trees,shapes.gates.logic.US,fit,automata,snakes,shapes.geometric}

\pagestyle{plain}

\tikzset{outline/.style args={#1}{%
  draw=#1,thick,fill=#1!50}}

\tikzset{
  dot hidden/.style={},
  line hidden/.style={},
  dice hidden/.style={},
  dot color/.style={dot hidden/.append style={color=#1}},
  dot color/.default=black,
  line color/.style={line hidden/.append style={color=#1}},
  line color/.default=black,
  dice color/.style={dice hidden/.append style={color=#1,fill}},
  dice color/.default=white
}\def\dotsize{0.1}
\newcommand{\drawdie}[2][]{%
\begin{tikzpicture}[x=1em,y=1em,#1]
  \draw 	[thick, rounded corners=0.5,line hidden,dice hidden] (0,0) rectangle (1,1);
  \ifodd#2
    \fill[dot hidden] (0.5,0.5) circle (\dotsize);
  \fi
  \ifnum#2>1
  \fill[dot hidden] (0.25,0.25) circle (\dotsize);
  \fill[dot hidden] (0.75,0.75) circle (\dotsize);
  \ifnum#2>3
    \fill[dot hidden] (0.25,0.75) circle (\dotsize);
    \fill[dot hidden] (0.75,0.25) circle (\dotsize);
    \ifnum#2>5
      \fill[dot hidden] (0.75,0.5) circle (\dotsize);
      \fill[dot hidden] (0.25,0.5) circle (\dotsize);
    \fi
  \fi
\fi
\end{tikzpicture}
}



%%
%% \BibTeX command to typeset BibTeX logo in the docs
\AtBeginDocument{%
  \providecommand\BibTeX{{%
    Bib\TeX}}}

%% Rights management information.  This information is sent to you
%% when you complete the rights form.  These commands have SAMPLE
%% values in them; it is your responsibility as an author to replace
%% the commands and values with those provided to you when you
%% complete the rights form.
\copyrightyear{2023}
\acmYear{2023}
\setcopyright{acmlicensed}\acmConference[KDD '23]{Proceedings of the 29th ACM SIGKDD Conference on Knowledge Discovery and Data Mining}{August 6--10, 2023}{Long Beach, CA, USA}
\acmBooktitle{Proceedings of the 29th ACM SIGKDD Conference on Knowledge Discovery and Data Mining (KDD '23), August 6--10, 2023, Long Beach, CA, USA}
\acmPrice{15.00}
\acmDOI{10.1145/3580305.3599918}
\acmISBN{979-8-4007-0103-0/23/08}
%%
%% Submission ID.
%% Use this when submitting an article to a sponsored event. You'll
%% receive a unique submission ID from the organizers
%% of the event, and this ID should be used as the parameter to this command.
%%\acmSubmissionID{123-A56-BU3}

%%
%% For managing citations, it is recommended to use bibliography
%% files in BibTeX format.
%%
%% You can then either use BibTeX with the ACM-Reference-Format style,
%% or BibLaTeX with the acmnumeric or acmauthoryear sytles, that include
%% support for advanced citation of software artefact from the
%% biblatex-software package, also separately available on CTAN.
%%
%% Look at the sample-*-biblatex.tex files for templates showcasing
%% the biblatex styles.
%%

%%
%% The majority of ACM publications use numbered citations and
%% references.  The command \citestyle{authoryear} switches to the
%% "author year" style.
%%
%% If you are preparing content for an event
%% sponsored by ACM SIGGRAPH, you must use the "author year" style of
%% citations and references.
%% Uncommenting
%% the next command will enable that style.
%%\citestyle{acmauthoryear}


%%
%% end of the preamble, start of the body of the document source.
\begin{document}

%%
%% The "title" command has an optional parameter,
%% allowing the author to define a "short title" to be used in page headers.
\title{TransAct: Transformer-based Realtime User Action Model for Recommendation at Pinterest}

%%
%% The "author" command and its associated commands are used to define
%% the authors and their affiliations.
%% Of note is the shared affiliation of the first two authors, and the
%% "authornote" and "authornotemark" commands
%% used to denote shared contribution to the research.
% \author{Xue Xia, Chantat Eksombatchai, Nikil Pancha, Dhruvil Deven Badani, Po-Wei Wang,\\ 
% Neng Gu, Saurabh Vishwas Joshi, Nazanin Farahpour, Zhiyuan Zhang, Andrew Zhai }

% \affiliation{%
%   \institution{Pinterest}
%   \streetaddress{P.O. Box 1212}
%   \city{San Francisco}
%   \state{CA}
%   \country{USA}
% %   \postcode{43017-6221}
% }
% \email{{xxia, pong, npancha, dbadani, poweiwang, ngu, sjoshi, nfarahpour, zhiyuan,  andrew}@pinterest.com}

\author{Xue Xia}
\email{xxia@pinterest.com}
\affiliation{%
  \institution{Pinterest}
  \streetaddress{P.O. Box 1212}
  \city{San Francisco}
  \state{CA}
  \country{USA}
%   \postcode{43017-6221}
}

\author{Pong Eksombatchai}
% \authornote{Both authors contributed equally to this research.}
\email{pong@pinterest.com}
\affiliation{%
  \institution{Pinterest}
  \streetaddress{P.O. Box 1212}
  \city{San Francisco}
  \state{CA}
  \country{USA}
%   \postcode{43017-6221}
}


\author{Nikil Pancha}
\email{npancha@pinterest.com}
\affiliation{%
  \institution{Pinterest}
  \streetaddress{P.O. Box 1212}
  \city{San Francisco}
  \state{CA}
  \country{USA}
%   \postcode{43017-6221}
}

\author{Dhruvil Deven Badani}
\email{dbadani@pinterest.com}
\affiliation{%
  \institution{Pinterest}
  \streetaddress{P.O. Box 1212}
  \city{San Francisco}
  \state{CA}
  \country{USA}
%   \postcode{43017-6221}
}

\author{Po-Wei Wang}
\email{poweiwang@pinterest.com}
\affiliation{%
  \institution{Pinterest}
  \streetaddress{P.O. Box 1212}
  \city{San Francisco}
  \state{CA}
  \country{USA}
%   \postcode{43017-6221}
}

\author{Neng Gu}
\email{ngu@pinterest.com}
\affiliation{%
  \institution{Pinterest}
  \streetaddress{P.O. Box 1212}
  \city{San Francisco}
  \state{CA}
  \country{USA}
%   \postcode{43017-6221}
}

\author{Saurabh Vishwas Joshi}
\email{sjoshi@pinterest.com}
\affiliation{%
  \institution{Pinterest}
  \streetaddress{P.O. Box 1212}
  \city{San Francisco}
  \state{CA}
  \country{USA}
%   \postcode{43017-6221}
}

\author{Nazanin Farahpour}
\email{nfarahpour@pinterest.com}
\affiliation{%
  \institution{Pinterest}
  \streetaddress{P.O. Box 1212}
  \city{San Francisco}
  \state{CA}
  \country{USA}
%   \postcode{43017-6221}
}


\author{Zhiyuan Zhang}
\email{zhiyuan@pinterest.com}
\affiliation{%
  \institution{Pinterest}
  \streetaddress{P.O. Box 1212}
  \city{San Francisco}
  \state{CA}
  \country{USA}
%   \postcode{43017-6221}
}

\author{Andrew Zhai}
\authornote{work done at Pinterest}
\email{andrew@aideate.ai}
\affiliation{%
  \institution{Pinterest}
  \streetaddress{P.O. Box 1212}
  \city{San Francisco}
  \state{CA}
  \country{USA}
%   \postcode{43017-6221}
}


%%
%% By default, the full list of authors will be used in the page
%% headers. Often, this list is too long, and will overlap
%% other information printed in the page headers. This command allows
%% the author to define a more concise list
%% of authors' names for this purpose.
\renewcommand{\shortauthors}{Xia et~al.}

%%
%% The abstract is a short summary of the work to be presented in the
%% article.
\input com_notations.tex

\newcommand{\apriori}{\emph{a priori }}
\newcommand{\aposteriori}{\emph{a posteriori }}

% observations
\newcommand{\ALLobs}{\boldsymbol{\calY}}
\newcommand{\allobs}{\boldsymbol{\mathtt{z}}}
\newcommand{\MATobs}[1]{\bfZ_{#1}}
\newcommand{\Vobs}[1]{\bfz_{#1}}

% unknown image
\newcommand{\MATima}{\bfX}
\newcommand{\Vima}{\bfx}
\newcommand{\ima}[1]{x_{#1}}

% transformations
\newcommand{\ftrans}[2]{\calF_{#1}\left(#2\right)}
\newcommand{\MATtrans}[1]{\bfF_{#1}}
\newcommand{\kernel}[1]{\boldsymbol{\kappa}_{#1}}

% noise vector
\newcommand{\MATnoise}[1]{\bfE_{#1}}
\newcommand{\Vnoise}[1]{\bfe_{#1}}
\newcommand{\noisevar}[1]{{s^2_{#1}}}
\newcommand{\Vnoisevar}{\bss^2}

% dimensions
\newcommand{\noobs}{p}
\newcommand{\nbobs}{P}

\newcommand{\nbrowobs}[1]{n_{\mathrm{x},#1}}
\newcommand{\nbcolobs}[1]{n_{\mathrm{y},#1}}
\newcommand{\nbbandobs}[1]{n_{\lambda,#1}}
\newcommand{\nbpixobs}[1]{N_{#1}}

\newcommand{\nbrowima}{m_{\mathrm{x}}}
\newcommand{\nbcolima}{m_{\mathrm{y}}}
\newcommand{\nbbandima}{m_{\lambda}}
\newcommand{\nbpixima}{M}


\newcommand{\imam}[1]{\boldsymbol{\mu}_{#1}}
%\newcommand{\imamean}{\boldsymbol{\mu}_{\Vima}}
\newcommand{\meansub}{\imam{\bsu}}
\newcommand{\Covsub}{\boldsymbol{\Sigma}}
\newcommand{\imamall}{\imam{\bfu}^{\star}}
\newcommand{\imacovall}{\imacovmat{\bfu}^{\star}}
%\newcommand{\imam_whole}{\imam{\bfu}^{\star}}

\newcommand{\imacovmat}[1]{\boldsymbol{\Sigma}_{#1}}
\newcommand{\covmat}[1]{\boldsymbol{\Sigma}_{#1}} 
\newcommand{\CovEstHS}{\hat{\boldsymbol{\Sigma}}_{\bsz_{1}}}

%Parameters for GMM prior
\newcommand{\imacoefgm}{\alpha}
\newcommand{\imahid}{\boldsymbol{\zeta}}
\newcommand{\nei}{\boldsymbol{\nu}}

%% unknown (hyper)parameter vector
\newcommand{\hypervect}{\boldsymbol{\Phi}}
\newcommand{\paramvect}{\boldsymbol{\theta}}
%
%\newcommand{\Valpha}{\boldsymbol{\alpha}}
%\newcommand{\Vomega}{\boldsymbol{\omega}}
%
% samples
\newcommand{\sample}[2]{\tilde{#1}^{#2}}
\newcommand{\samplebis}[2]{{#1}^{\left(#2\right)}}
\newcommand{\samplenoisevar}[1]{{\widetilde{\sigma}}^{2(#1)}}
\newcommand{\sampleparamvect}[1]{\widetilde{\paramvect}^{(#1)}}


\newcommand{\MAP}[1]{\hat{#1}_{\mathrm{MAP}}}
\newcommand{\MMSE}[1]{\hat{#1}_{\mathrm{MMSE}}}
\newcommand{\argmax}{\mathrm{arg}\max}
\newcommand{\argmin}{\mathrm{arg}\min}

% norm
\newcommand{\norm}[1]{\left\|#1\right\|}

% R set
\newcommand{\R}{\mathds{R}}
%\newcommand{\R}{\mathbb{R}}
%\newcommand{\R}{\mbox{\rm I\kern-.25em R}}  % This is for use with cmr11
%\newcommand{\R}{\mbox{\rm I\kern-.2em R}}


\newcommand{\dirac}[1]{\delta\left({#1}\right)}

% inverse
\newcommand{\inv}{^{-1}}

% hermitian operator
\newcommand{\herm}{^{*}}

% transpose
\newcommand{\transp}{^T}

% exponential trace
\newcommand{\etr}{\mathrm{etr}}


\newcommand{\Ndistr}[1]{\mathcal{N}\left(#1\right)}

% vecteur un
\newcommand{\Vun}{{\boldsymbol{1}}}
% vecteur nul
\newcommand{\Vzero}[1]{\boldsymbol{0}_{#1}}
\newcommand{\Vzeros}[1]{\boldsymbol{0}_{#1}}
% identit�
\newcommand{\Id}[1]{\textbf{I}_{#1}}
% fonction indicatrice
%\newcommand{\Indicfun}[2]{\mathbb{I}_{#1}(#2)}
\newcommand{\Indicfun}[2]{\textbf{1}_{#1}\left(#2\right)}
\newcommand{\Diag}[2]{\left[#1\right]_{#2}}




% d�finition de l'environnement algo
\newenvironment{algogo}[1]{
\smallskip
\noindent \hrule\vspace{0.2\baselineskip} \hrule
\smallskip
\begin{small}
\refstepcounter{algo} \center{\bf \textsc{Algorithm \thealgo:}}
\\{\center{\bf #1}}
\smallskip
\flushleft
 } {
\end{small}
\bigskip
\hrule\vspace{0.2\baselineskip} \hrule
%\bigskip
\smallskip }

\newcounter{algo}
\renewcommand{\thealgo}{\arabic{algo}}
\begin{abstract}
\label{sec:abstract}

%% 1. what is the problem 
Scientific applications that run on leadership computing facilities often face the challenge 
of being unable to fit leading science cases onto accelerator devices due to memory constraints 
(memory-bound applications).
%
% 2. what is your solution 
In this work, the authors studied one such US Department of Energy mission-critical condensed matter 
physics application, Dynamical Cluster Approximation (DCA++), and this paper discusses how device memory-bound challenges were successfully reduced  by proposing an effective 
``all-to-all'' communication method---a ring communication algorithm. 
%
This implementation takes advantage of acceleration on GPUs and remote direct memory access (RDMA) for fast data exchange between GPUs. 
%
\\Additionally, the ring algorithm was optimized with sub-ring communicators
and multi-threaded support to further reduce communication overhead and 
expose more concurrency, respectively.
%
% 3. What's the cherry-picked evaluation result you want to mention
The computation and communication were also analyzed 
by using the Autonomic Performance Environment for Exascale 
(APEX) profiling tool,  and this paper further discusses the 
performance trade-off for the ring algorithm implementation. 
%
The memory analysis on the ring algorithm shows that the allocation size for the authors' most 
memory-intensive data structure per GPU is now reduced to $1/p$ of the original size, where $p$ is the number of GPUs in the ring communicator.
%
The communication analysis suggests that 
the distributed Quantum Monte Carlo execution time grows linearly as sub-ring size increases, and the cost of messages passing through the network interface connector could be a limiting factor.


%
% \todoRed{Ronnie: Next sentence needs rewrite, too much information about Green's function that no one knows in the abstract; recommend generalizing.} \emph {However, DCA++ is currently facing memory-bound challenge as 
% a larger device array $G_t$ is limited by device memory size, where
% $G_t$ is a two-particle Green's function that allows condensed matter
% scientists to explore larger and more complex (higher fidelity)
% physics cases.}

\end{abstract}

\keywords{DCA++, Quantum Monte Carlo, GPU Remote Direct Memory Access, memory-bound issue, exascale machines}



%%
%% The code below is generated by the tool at http://dl.acm.org/ccs.cfm.
%% Please copy and paste the code instead of the example below.
%%

\begin{CCSXML}
<ccs2012>
   <concept>
       <concept_id>10002951</concept_id>
       <concept_desc>Information systems</concept_desc>
       <concept_significance>500</concept_significance>
       </concept>
   <concept>
       <concept_id>10002951.10003260.10003261</concept_id>
       <concept_desc>Information systems~Web searching and information discovery</concept_desc>
       <concept_significance>500</concept_significance>
       </concept>
   <concept>
       <concept_id>10002951.10003260.10003261.10003267</concept_id>
       <concept_desc>Information systems~Content ranking</concept_desc>
       <concept_significance>500</concept_significance>
       </concept>
   <concept>
       <concept_id>10002951.10003260.10003261.10003271</concept_id>
       <concept_desc>Information systems~Personalization</concept_desc>
       <concept_significance>500</concept_significance>
       </concept>
 </ccs2012>
\end{CCSXML}

\ccsdesc[500]{Information systems}
\ccsdesc[500]{Information systems~Web searching and information discovery}
\ccsdesc[500]{Information systems~Content ranking}
\ccsdesc[500]{Information systems~Personalization}
%%
%% Keywords. The author(s) should pick words that accurately describe
%% the work being presented. Separate the keywords with commas.
\keywords{Personalization, Recommender Systems, Sequential Recommendation, User Interest Modeling}
%% A "teaser" image appears between the author and affiliation
%% information and the body of the document, and typically spans the
%% page.
% \begin{teaserfigure}
%   \includegraphics[width=\textwidth]{sampleteaser}
%   \caption{Seattle Mariners at Spring Training, 2010.}
%   \Description{Enjoying the baseball game from the third-base
%   seats. Ichiro Suzuki preparing to bat.}
%   \label{fig:teaser}
% \end{teaserfigure}

% \received{20 February 2007}
% \received[revised]{12 March 2009}
% \received[accepted]{5 June 2009}
% \begin{teaserfigure}
%   \includegraphics[width=\textwidth]{figures/hf_screenshot_web.png}
%   \caption{Pinterest Homefeed page}
%   \Description{Pinterest Homefeed page}
% \end{teaserfigure}

%%
%% This command processes the author and affiliation and title
%% information and builds the first part of the formatted document.
\maketitle

% Reinforcement learning has achieved great success in areas such as Game-playing \citep{silver2018general,vinyals2019grandmaster}, robotics \cite{kober2013reinforcement}, large language models \citep{ouyang2022training}, etc.
However, due to safety concerns or physical limitations, in some real-world reinforcement learning problems, we must consider additional constraints that may influence the optimal policy and the learning process \citep{garcia2015comprehensive}.
% For example, a robotic arm must not take actions that may cause harm to itself or the environments.
A standard framework to handle such cases is the constrained Markov Decision Process (CMDP) \citep{altman1999constrained}.
Within the CMDP framework, the agent has to maximize
the expected cumulative reward while
obeying a finite number of constraints, which are usually in the form of expected cumulative cost criteria.

However, we are sometimes concerned with the problem with a continuum of constraints.
For example,
the constraints we meet might be time-evolving or subject to uncertain parameters, which
cannot be formulated as an ordinary CMDP
(see Examples \ref{Example_Time_Evolving} and  \ref{Example_Uncertain}).
In this paper we would study a generalized CMDP  
to address the above problem.  Because the constraints are not only infinite-number but also lie
in a continuous set,
the generalization is not trivial. Fortunately, we find that we can borrow the idea behind semi-infinite programming (SIP) \citep{remez1934determination, hettich1993semi} to deal with the semi-infinite constraints.
Accordingly, we propose \emph{semi-infinitely constrained Markov decision processes} (SICMDPs)
as a novel complement to the ordinary CMDP framework.
%More specifically,  an SICMDP model %, we consider 
%contains a continuum of constraints whereas an ordinary CMDP contains a finite number of constraints. 

%This generalization is natural but not trivial. However, we can brows the idea  
%The idea is quite natural and can be backtracked
%to the practice of extending linear programming to linear semi-infinite programming (LSIP) %\cite{remez1934determination, GobernaLSIO1998}.
%In addition, 
%As a complementary approach to the ordinary CMDP framework, 
%SICMDP can be used to model these problems  which cannot be described by a finite number of constraints
%that are not covered by .
%For example,
%the restrictions we consider can be time-evolving or subject to uncertain parameters
%, thus
%cannot be described by a finite number of constraints but a continuum of constraints 
%(see Examples \ref{Example_Time_Evolving} and  \ref{Example_Uncertain}).

We also present two reinforcement learning algorithms to solve SICMDPs called SI-CRL and SI-CPO, respectively.
SI-CRL is a model-based reinforcement learning algorithm designed for tabular cases, and SI-CPO is a policy optimization algorithm for non-tabular cases.
% and analyze its performance both theoretically and empirically.
The main challenge is that we need to deal with a continuum of constraints, thus reinforcement learning algorithms for ordinary CMDPs do not work anymore.
In SI-CRL, we tackle this difficulty by first transforming the reinforcement learning problem to an equivalent LSIP problem, which can then be solved using methods in the LSIP literature like the dual exchange methods \citep{Hu1990,reemtsen1998numerical}.
In SI-CPO, we resort to the idea of cooperative stochastic approximation developed in \cite{lan2020algorithms, wei2020comirror}.
As far as we know, we are the first to introduce tools from semi-infinitely programming (SIP) into the reinforcement learning community for solving constrained reinforcement learning problems.

% To the best of our knowledge, we are the first to apply tools from semi-infinitely programming (SIP) to solve reinforcement learning problems.
Furthermore, we give theoretical analysis for both SI-CRL and SI-CPO.
We decompose the error of SI-CRL into two parts: the statistical error from approximating the true SICMDP with an offline dataset and the optimization error due to the fact that the solution of the LSIP problem obtained by the dual exchange method is inexact.
On the optimization side, we show that the iteration complexity of SI-CRL is $O\left(\left\{\mathrm{diam}(Y)L\sqrt{|\gS|^2|\gA|m}/\left[(1-\gamma)\epsilon\right]\right\}^m\right)$.
On the statistical side, we show that the sample complexity of SI-CRL is $\widetilde O\left(\frac{|S|^2|A|^2}{\epsilon^2(1-\gamma)^3}\right)$ if the offline dataset is generated by a generative model, and $\widetilde O\left(\frac{|S||A|}{\nu_{\min} \epsilon^2(1-\gamma)^3}\right)$ if the dataset is generated by a probability measure $\nu$ as considered in \cite{chen2019information}.
Here $\widetilde O$ means that all logarithm terms are discarded.
For SI-CPO, things become a little more complicated because other than the statistical error and the optimization error, we also need to consider the function approximation error, which comes from imperfect policy parametrizations.
It is shown if the function approximation error can be controlled to $O(\epsilon)$ order, the iteration complexity of SI-CPO is $\widetilde{O}\left(\frac{1}{\epsilon^2(1-\gamma)^6}\right)$ and the sample complexity of SI-CPO is $\widetilde{O}(\frac{1}{\epsilon^4(1-\gamma)^{10}})$.
Here our iteration complexity bound is equivalent to a typical $\widetilde O(1/\sqrt{T})$ global convergence rate.

We perform a set of numerical experiments to illustrate the SICMDP model and validate our proposed algorithms.
Specifically, we examine two numerical examples, namely the discharge of sewage and ship route planning.
Through the discharge of sewage example, we show the advantage of the SICMDP framework over the CMDP baseline obtained by naive discretization in modeling realistic sequential decision-making problems.
Moreover, we demonstrate the effectiveness of the SI-CRL and SI-CPO algorithms in such tabular environments. 
In the ship route planning example, we illustrate the benefits of the SICMDP framework and the ability of the SI-CPO algorithm to address complex continuous control tasks involving continuous state spaces with modern deep reinforcement learning techniques.

% In summary, our contributions are listed as follows.
% First, we present the SICMDP model, which can be viewed as a generalization of the ordinary CMDP model.
% Second, we propose an algorithm to perform reinforcement learning for SICMDPs, which is called SI-CRL, and we believe that we are the first to apply tools from SIP
% to solve reinforcement learning problems.
% Third, we give a theoretical analysis of SI-CRL and identify both its sample complexity and iteration complexity.
% In addition, we perform numerical experiments to illustrate the SICMDP model and validate the SI-CRL algorithm.
% \{This paragraph can be removed!!! \}





\section{Introduction}

% Why’s rec sys important

The proliferation of online content in recent years has created an overwhelming amount of information for users to navigate. 
To address this issue, recommender systems are employed  in various industries to help users find relevant items from a vast selection, including products, images, videos, and music. 
% , making it easier for them to find what they want. 
By providing personalized recommendations, businesses and organizations can better serve their users and keep them engaged with the platform. 
Therefore, recommender systems are vital for businesses as they drive growth by boosting engagement, sales, and revenue.

\begin{figure}[!ht]
  \centering
  \includegraphics[width=0.85\linewidth]{figures/Homefeed_web_square.png}
  % \caption{Illustration of generating Pinterest Homefeed page using a 3-stage RS}
  \caption{Pinterest Homefeed Page}
  \label{fig:hf}
  \Description{Pinterest Homefeed page}
\end{figure}

% Background of PINS Homefeed ranking. Previous PINS work (PinSage, Pinnerformer)
As one of the largest content sharing and social media platforms, Pinterest hosts billions of pins with rich contextual and visual information, and brings inspiration to over 400 million users. 
Upon visiting Pinterest, users are immediately presented with the Homefeed page as shown in Figure~\ref{fig:hf}, which serves as the primary source of inspiration and accounts for the majority of overall user engagement on the platform.
The Homefeed page is powered by a 3-stage recommender system that retrieves, ranks, and blends content based on user interests and activities. 
At the retrieval stage, we filter billions of pins created on Pinterest to thousands, based on a variety of factors such as user interests, followed boards, etc. Then we use a pointwise ranking model to rank candidate pins by predicting their personalized relevance to users. 
Finally, the ranked result is adjusted using a blending layer to meet business requirements.


% Existing work for realtime ranking
Realtime recommendation is crucial because it provides a quick and up-to-date recommendation to users, improving their overall experience and satisfaction. 
The integration of realtime data, such as recent user actions, results in more accurate recommendations and increases the probability of users discovering relevant items~\cite{alibaba_seq_tfmr, pi2020search}.

Longer user action sequences result in improved user representation and hence better recommendation performance. 
However, using long sequences in ranking poses challenges to infrastructure, as they require significant computational resources and can result in increased latency.
To address this challenge, some approaches have utilized hashing and nearest neighbor search in long user sequences~\cite{pi2020search}.
% Research in this area typically focuses on reducing the complexity of the system\cite{chatzopoulos2016readme}, handling high-velocity data streams\cite{6542425}, and dealing with dynamic changes in user behavior~\cite{grbovic2018real, 10.1145/2699670}. 
Other work encodes users' past actions over an extended time frame to a user embedding~\cite{pinnerformer} to represent long-term user interests. User embedding features are often generated as \textit{batch} features (e.g. generated daily), which are cost-effective to serve across multiple applications with low latency. 
The limitation of existing sequential recommendation is that they either only use realtime user actions, or only use a batch user representation learned from long-term user action history.


We introduce a novel realtime-batch hybrid ranking approach that combines both \textit{realtime} user action signals and \textit{batch} user representations. 
To capture the realtime actions of users, we present TransAct - a new transformer-based module designed to encode recent user action sequences and comprehend users' immediate preferences.
For user actions that occur over an extended period of time, we transform them into a batch user representation~\cite{pinnerformer}.

By combining the expressive power of TransAct with batch user embeddings, the hybrid ranking model offers users realtime feedback on their recent actions, while also accounting for their long-term interests. The realtime component and batch component complement each other for recommendation accuracy. This leads to an overall improvement in the user experience on the Homefeed page.
% Limitation of previous sequential recommendation/ novelty
% Sequential recommendation systems use a user's action history as input and apply recommendation algorithms to suggest appropriate items. Recent work~\cite{donkers2017sequential, hidasi2015session, tan2016improved, zhou2019deep, tang2018personalized, tuan20173d} has been using deep learning techniques, such as recurrent neural networks (RNNs)~\cite{rnn} and convolutional neural networks (CNNs)~\cite{cnn}, to process users' action history. Some studies~\cite{DIN, zhang2019next, alibaba_seq_tfmr, li2020time, SASRec, sun2019bert4rec} have also adopted the attention mechanism~\cite{tfmr} to model user action sequence features. 
% In this work, we also employ the self-attention mechanism for realtime user sequence modeling for its superior capability in encoding sequential inputs. 
% % One limitation of the existing sequential recommendation is that they
% To the best of our knowledge, this is the first method that builds a realtime-batch hybrid ranking model which uses users' action sequences in both realtime and batch settings. 
% We provide an in-depth analysis of the challenges posed by utilizing real-time user sequence features, such as the potential decrease in recommendation diversity and engagement decay.

The major contributions of this paper are summarized as follows:
\begin{itemize}
 \item We describe Pinnability, the architecture of Pinterest's Homefeed production ranking system. The Homefeed personalized recommendation product accounts for the majority of the overall user engagement on Pinterest. 

 \item  We propose TransAct, a transformer-based realtime user action sequential model that effectively captures users' short-term interests from their recent actions. We demonstrate that combining TransAct with daily-generated user representations~\cite{pinnerformer} to a hybrid model leads to the best performance in Pinnability. This design choice is justified through a comprehensive ablation study. Our code implementation is publicly available\footnote{Our code is available on Github: \url{https://github.com/pinterest/transformer_user_action}}.

\item We describe the serving optimization implemented in Pinnability to make feasible the computational complexity increase of 65 times when introducing TransAct to the Pinnability model. Specifically, optimizations are done to enable GPU serving of our prior CPU-based model.
\item We describe online A/B experiments on a real-world recommendation system using TransAct. We demonstrate some practical issues in the online environment, such as recommendation diversity drop and engagement decay, and propose solutions to address these issues.  
% \item Our model has been deployed as the Homefeed ranking model of Pinterest, one of the largest content sharing and social media platforms. As a result, it boosts the Homefeed repin\footnote{A "repin" on Pinterest refers to the action of saving an existing pin to another board by a user.} volume by 11\%.

\end{itemize}

% Structure of the paper
% \td{check}
The remainder of this paper is organized as follows: Related
work is reviewed in Section~\ref{sec:related_work}. Section~\ref{sec:method} describes the design of TransAct and the details of bringing it to production. Experiment results are reported in Section~\ref{sec:exp}. We discuss some findings beyond experiments in Section~\ref{sec:discussion}. Finally, we conclude our work in Section~\ref{sec:conclusion}.
% Learnings









% \begin{itemize}
% \item Hybrid setup
% \item Ablation study: Using both positive and negative actions, Early fusion with candidate pin
% \item Diversity: time window mask
% \item Frequent retraining
% \item is good for cold-start users (non-core)
% \end{itemize}


% 1. Hybrid setup
% w. No extensive ablation study on the model architecture Model: early fusion, random time window.
% Did not discuss practical challenge: latency, GPU serving
% Ali’s Search based user seq does have practical challenge
% id feature only (but we don’t have img_sig feature)	
% 3. Diversity


% Most existing recommender systems use a wide and deep architecture, where the inputs are user features, item features and context features. The model usually learns the relevance score between the user and item over a given context. The traditional way to get user or item features are mostly through feature engineering. And then the model use the hand-picked features to predict the relevance between user and items. Raw features are usually very limited in the amount of information that they can represent. More advanced techniques include learning pre-trained item or user embedding, where it can use a reasonable amount of space to represent richer information. Some successful pre-trained embeddings are Pinnerformer\cite{pinnerformer}, itemsage\cite{itemsage}, etc. However, one of the major shortcomings of pre-trained embedding is that they are usually expensive to infer in realtime due to their high computational cost and infra complexity. 
\subsection{Multitask Learning}

MTL has been succesfully used in different domains, including CV \cite{UberNet,MaskRCNN}. Some challenges appear when applying it \cite{Caruana}: \textit{learning speed} differences between tasks and  deciding \textit{what to share} according to the \textit{relatedness} between tasks in the multitask architecture \cite{Stitch, AdaptiveFeatureSharing}.

\subsection{Semantic Segmentation}

Semantic segmentation aims at partitioning parts of images belonging to the same semantic class, typically via pixel-wise classification. Fully convolutional networks (FCN) \cite{FCN} have improved both accuracy and speed for dense prediction problems by using only convolutional layers. Upsampling layers allow a segmentation output size equal to the input and skip connections add finer details. Other approaches add post-processing steps \cite{DeeplabCRF}, learnable \textit{deconvolution} layers \cite{ Deconv} or global context \cite{ParseNet}.

\subsection{Object Detection}

Object detection aims at finding in an image all instances of objects and classifying them in a number of classes. Faster R-CNN \cite{FasterRCNN} was the first to give close to real-time performance. YOLO \cite{YOLO} avoids the generation of region proposals for increased speed. SSD \cite{SSD} avoids fully-connected layers for speed and takes features at different levels for improved accuracy. 

%\cite{SpeedAccuracy} reviews the speed vs. accuracy trade-off for different object detectors.
The proposed segmentation-by-detection framework, as depicted in Figure \ref{fig:framework}, consists of a detection module and a segmentation module.
In detection stage, 2D slices (layered box) from the input volume are fed to the RPN. Based on the region proposals obtained from RPN, an attention model (block in orange) is formed. The input volume as well as the attention model are further processed in segmentation stage to get the refined anatomical segmentation. 
\vspace{1em} 

\begin{figure}[t]
\centering
\includegraphics[width=0.95\linewidth]{fig/framework.pdf}
\caption{Schematic representation of the segmentation-by-detection framework. The left part is the detection module while the segmentation module is followed on the right. The blue block denotes the input volume which is 3D ultrasound scan of femoral head. The output segmentation is in red.}
\label{fig:framework}
\end{figure}
% dana could you improve the figure. we can try to think together of better ways 

\noindent\textbf{Detection Module:} 
% dana : here you have to make the clarification that you have ground truth on the boxes (in implementation part)
The detection module follows an RPN architecture, a fully convolutional network which takes image slice as input and outputs object region candidates. 
We use the VGG-16 model as the backbone \cite{simonyan2014very} to learn convolutional features and an $3 \times 3$ spatial window to generate region proposals. At each sliding-window location, 9 anchors are predicted associated with different scales and aspect ratios. The last layer consists of a box-regression (reg) layer and a box-classification (cls) layer in parallel. The reg layer outputs 4 regression offsets, $ t = (t_x,t_y,t_w,t_h)$, denoting a scale-invariant translation as well as log-space height and width shift, where $x,y,w$ and $h$ specify two coordinates of the box center, width and height. The cls layer outputs two scores by softmax, related to probabilities of object and background for each proposal. We assign a positive label (of being object) to candidate which has an Intersection-over-Union (IoU) ratio higher than 0.7 with ground truth box. Note that an image slice may contain multiple object regions or none. 

The loss function of RPN follows the multi-task loss \cite{ren2015faster} which is defined as $L = L_{reg} + L_{cls}$. The regression loss, $L_{reg} = -\log p_{obj}$ is log loss and the classification loss,
\begin{equation} \label{eq:loss}
L_{cls} = \sum_{i \in \{x,y,w,h\}} smooth_{L_1} (t_i - t_i^*)
\end{equation}
is smooth $L_1$ loss where $t_i^*$ denotes the ground truth box for the target object. 
\vspace{1em}

\noindent\textbf{Segmentation Module:}
3D U-Net \cite{cciccek20163d} is utilized in the segmentation module as its outstanding performance in medical image segmentation. The u-shaped architecture consists of two paths: a contracting path, where each layer contains two $3\times3\times3$ convolutions followed by a rectified linear unit (ReLU) and then a max pooling, provides high resolution features. While, the symmetric expanding path for semantically richer features replaces max pooling with a upconvolution $2\times2\times2$ with stride of 2 in each dimension, and then two $3\times3\times3$ convolutions each followed by a ReLU. Skip connections between layers of equal resolution in the contracting path and the expanding path enables context information as well as precise localization.

Different from 3D U-Net, to incorporate the attention model detected by the RPN, our architecture takes as input both the volumetric image data and the candidate RoIs proposed by the RPN, concatenated as 3D volume. 
% dana not sure what you like to say below
% densely annotated
The attention model makes the network to focus on the potential RoIs and can reduce the interference of the surrounding noise.
The anatomical segmentation is then generated from a $1\times1\times1$ convolution which reduces the number of feature maps to the number of labels.  The energy function is computed by a pixel-wise softmax combined with the cross entropy loss.
% dana equation ??

\subsection{System and implementation Details}
The segmentation-by-detection approach adopts a cascade structure with two stages: detection and segmentation. The two networks are trained separately in an end-to-end manner. All the new layers are randomly initialized from zero-mean Gaussian distribution with standard deviations 0.01. Biases are initialized to 0. We use Caffe \cite{jia2014caffe} for the implementation and an NVIDIA Titan X GPU for training.

In the detection stage, we initialize the VGG-16 model by the pre-trained model for ImageNet classification \cite{russakovsky2015imagenet} and further fine-tune the model for our detection task. The input fed to the network are image slices with a fixed size of $184\times96$ and the corresponding ground truth boxes are generated from the annotation in the format of tight bounding boxes surrounding the segmentation contour (as illustrated in Figure \ref{fig:hip} (b), the boundary of white area). To optimize the energy function, stochastic gradient descent (SGD) is used. The global learning rate is set to 0.001, while a momentum of 0.9 and a weight decay of 0.0005 are used. The batch size is set to 256 and each mini-batch only contains the positive anchors for training. The region proposals are obtained from the reg path for each image slice. The attention model is then formed by concatenating all the detected regions, as binary masks, into a volume.

In the segmentation stage, we use the Adam optimizer \cite{kingma2014adam} to learn the network parameters. A global learning rate is set to 0.001 while the two momentum coefficients are set to 0.9 and 0.999 respectively. A batch size of 1 is used due to the memory constraints of the GPU. The network takes the volume data as well as the attention model as input. We train the network for a maximum of 30K iterations and reserve the learned weights with the best performance from every 1K iterations. 
\vspace{1em}

\noindent\textbf{Inference:}
At test time, the 2D slices from an input volume are first fed to the detection module. The attention model is obtained based on the output. Then the volume data as well as the attention model are fed to the segmentation module to get the pixel-wise prediction.



\newcommand{\twomoons}{{\tt Twomoons}}
\newcommand{\gauss}{{\tt Gauss}}
\newcommand{\sculpture}{{\tt Sculpture}}
\newcommand{\baseline}{{\tt Baseline}}
\newcommand{\MM}{{\tt MsgPassing}}
\newcommand{\blackboard}{{\tt Blackboard}}
\newcommand{\ncut}{\text{ncut}}
\newcommand{\chensays}[2][]{\textcolor{blue} {\textsc{Jiecao #1:} \emph{#2}}}

\section{Experiments}
In this section we present experimental results for  graph clustering in the message passing and blackboard models. We will compare the following three algorithms. (1) \baseline: each site sends all the data to the coordinator directly; (2) \MM: our algorithm in the message passing model (Section~\ref{sec:gcmessage}); (3) 
\blackboard: our algorithm in  the blackboard model (Section~\ref{sec:bb}).


%Since both of our algorithms are crucially based on the use of spectral scarification, our main focus in the experiments is to investigate to what extend the quality of the spectral clustering algorithms will be affected by using spectral sparsification, the saving of communication costs by using spectral sparsificaion, ...
%
%
%The goal of this experiment is not to demonstrate the effectiveness of the spectral clustering algorithm. We mainly want to investigate the following, 
%\begin{itemize}
%\item to what extend the quality of clustered results will be affected by using spectral sparsification.
%\item saving of communication costs by using spectral sparsifier.
%\item the affect of constants in algorithms of the message passing/blackboard model.
%\end{itemize}
%
%
%\subsection{The Setup}
%\paragraph{Reference Algorithms}
%We compare different algorithms in our experiment.

%Note that we can also run \MM~ in the blackboard model.

Besides giving the visualized results of these algorithms on various datasets, we also measure the qualities of the results via the {\em normalized cut}, defined as 
\[
\ncut(A_1, \ldots, A_{k}) = \frac{1}{2}\sum_{i\in[k]}\frac{w(A_i, V\backslash A_i)}{\vol(A_i)},
\]
 which is a standard objective function to be minimized for spectral clustering algorithms. 
%We will compare the communication costs of these algorithms in different settings.

%We also compare the total communication costs of different algorithms/models. As the unit does not matter in our case, we normalize all communication costs by the cost of \baseline.  Whenever possible, we will visualize the clustered results.

We implemented the algorithms using multiple languages, including Matlab, Python and C++. Our experiments were conducted on an IBM NeXtScale nx360 M4 server, which is equipped with 2 Intel Xeon E5-2652 v2 8-core processors, 32GB RAM and 250GB local storage.


\subsection{Datasets.}
We test the algorithms in the following real and synthetic datasets, which is visualized in \figref{visualization}.


\begin{figure}[h]
     \centering
     \subfigure[\twomoons]{\includegraphics[width=0.23\textwidth]{twomoons-14000-original.png}\label{fig:twomoons}}
     ~~
     \subfigure[\gauss]{\includegraphics[width=0.23\textwidth]{gauss-10000-original.png}\label{fig:gauss}}
     ~~
     \subfigure[\sculpture]{\includegraphics[width=0.13\textwidth,height=0.16\textwidth]{sculpture-11680-original.jpg}\label{fig:sculpture}}
     \caption{Visualization of the datasets for our experiments.}
     \label{fig:visualization}
\end{figure}



\vspace{-1mm}
\begin{itemize}
\item \twomoons : this dataset contains $n=14,000$ coordinates in $\mathbb{R}^2$. We consider each point to be a vertex. For any two vertices $u, v$, we add an edge with weight $w(u,v) = \exp\{-\|u-v\|_2^2/\sigma^2\}$ with $\sigma = 0.1$ when one vertex is among the $7000$-nearest points of the other.  This construction results in a graph with about $110,000,000$ edges.

\item  \gauss : this dataset contains $n = 10,000$ points in $\mathbb{R}^2$. There are $4$ clusters in this dataset, each generated using a Gaussian distribution. We construct a complete graph as the similarity graph.  For any two vertices $u, v$, we define the weight $w(u,v) = \exp\{-\|u-v\|_2^2/\sigma^2\}$ with $\sigma = 1$. The resulting graph has about $100,000,000$ edges.

\item \sculpture : a photo of \textit{The Greek Slave}~\footnote{Available in e.g., \url{http://artgallery.yale.edu/collections/objects/14794}}. We use an $80\times 150$ version of this photo where each pixel is viewed as a vertex. To construct a similarity graph, we map each pixel to a point in $\mathbb{R}^5$, i.e., $(x, y, r, g, b)$, where the latter three coordinates are the RGB values. For any two vertices $u, v$, we  put an edge between $u, v$ with weight $w(u,v) = \exp\{-\|u-v\|_2^2/\sigma^2\}$ with $\sigma = 0.5$ if one of $u, v$ is among the $5000$-nearest points of the other. This results in a graph with about $70,000,000$ edges.
\end{itemize}
\vspace{-1mm}
In the distributed model edges are randomly partitioned across $s$ sites. 

%\vspace{-1.5mm}



\subsection{Results on clustering quality}
%{\em Quality.} \
\begin{figure*}[ht]
     \centering
     \subfigure[\baseline]{\includegraphics[width=0.2\textwidth]{twomoons-14000-original-clustered.png}\label{fig:twomoons-clustered-original}}
     \subfigure[\MM]{\includegraphics[width=0.2\textwidth]{twomoons-14000-sparsify-clustered-15.png}\label{fig:twomoons-clustered-sparsify}}
     \subfigure[\blackboard]{\includegraphics[width=0.2\textwidth]{twomoons-14000-chain-clustered.png}\label{fig:twomoons-clustered-chain}}
     \caption*{\twomoons, $k = 2$;}

\subfigure[\baseline]{\includegraphics[width=0.2\textwidth]{gauss-10000-original-clustered.png}\label{fig:gauss-clustered-original}}
     \subfigure[\MM]{\includegraphics[width=0.2\textwidth]{gauss-10000-sparsify-clustered-15.png}\label{fig:gauss-clustered-sparsify}}
     \subfigure[\blackboard]{\includegraphics[width=0.2\textwidth]{gauss-10000-chain-clustered.png}\label{fig:gauss-clustered-chain}}
     \caption*{\gauss, $k = 4$}


     \subfigure[\baseline]{\includegraphics[width=0.2\textwidth,height=0.2\textwidth]{sculpture-11680-original-clustered.png}\label{fig:sculpture-clustered-original}}  
     \subfigure[\MM]{\includegraphics[width=0.2\textwidth,height=0.2\textwidth]{sculpture-11680-sparsify-clustered-15.png}\label{fig:sculpture-clustered-sparsify}}
     \subfigure[\blackboard]{\includegraphics[width=0.2\textwidth,height=0.2\textwidth]{sculpture-11680-chain-clustered.png}\label{fig:sculpture-clustered-chain}}
     \caption*{\sculpture, $k = 3$. }


     
     \caption{Visualization of the results on \twomoons, \gauss\ and \sculpture. In the message passing model each site samples $5 n$ edges; in the blackboard model all sites jointly sample $10n$ edges (in \twomoons~ and \gauss) or $20n$ edges (in \sculpture) and the chain has length $18$. $s = 15$.}
     \label{fig:quality-1}
\end{figure*}

We visualize the clustered results for 
the \twomoons, \gauss\ and \sculpture\ in Figure~\ref{fig:quality-1}.
% and visualize the clustered results for \gauss\ and \sculpture in Figure~\ref{fig:quality-2}.
It can be seen that \baseline, \MM\ and \blackboard\ give results of very similar qualities.  For simplicity, here we only present the visualization for $s=15$. Similar results were observed when we varied the values of $s$.  
%\he{To Qin: Do you plan to have two titles (Results \& Quality)?}


% \begin{figure*}[h]
%      \centering
% \subfigure[\baseline]{\includegraphics[width=0.3\textwidth]{gauss-10000-original-clustered.png}\label{fig:gauss-clustered-original}}
%      \subfigure[\MM]{\includegraphics[width=0.3\textwidth]{gauss-10000-sparsify-clustered-15.png}\label{fig:gauss-clustered-sparsify}}
%      \subfigure[\blackboard]{\includegraphics[width=0.3\textwidth]{gauss-10000-chain-clustered.png}\label{fig:gauss-clustered-chain}}
%      \caption*{\gauss, $k = 4$}


%      \subfigure[\baseline]{\includegraphics[width=0.2\textwidth]{sculpture-11680-original-clustered.png}\label{fig:sculpture-clustered-original}}  
%      \subfigure[\MM]{\includegraphics[width=0.2\textwidth]{sculpture-11680-sparsify-clustered-15.png}\label{fig:sculpture-clustered-sparsify}}
%      \subfigure[\blackboard]{\includegraphics[width=0.2\textwidth]{sculpture-11680-chain-clustered.png}\label{fig:sculpture-clustered-chain}}
%      \caption*{\sculpture, $k = 3$. }

%      \caption{Visualization of results on \gauss\ and \sculpture; in the message passing model each site samples $5 n$ edges; in the blackboard model all sites jointly sample $10n$ (in \gauss) or $20n$ (in \sculpture) edges and the chain has length $18$.}
%      \label{fig:quality-2}
% \end{figure*}


We also compare the normalized cut (ncut) values of the clustering results of different algorithms.  The results are presented in Figure \ref{fig:quality}. In all datasets, the ncut values of different algorithms are very close. The ncut value of \MM\ slightly decreases when we increase the value of $s$, while the ncut value of \blackboard\ is independent of $s$.
%We comment that in general, it is difficult to compare \MM\ and \blackboard\ directly because they are affected by different parameters.


\begin{figure*}[!ht]
  \centering
  \subfigure[\twomoons]{\includegraphics[width=0.33\textwidth]{twomoons-14000-ncut.png}\label{fig:twomoons-quality}}\hspace*{-1.1em}
  \subfigure[\gauss]{\includegraphics[width=0.31\textwidth]{gauss-10000-ncut.png}\label{fig:gauss-quality}}\hspace*{-1.1em}
  \subfigure[\sculpture]{\includegraphics[width=0.31\textwidth]{sculpture-11680-ncut.png}\label{fig:sculpture-quality}}\hspace*{-1.1em}
  \subfigure{\includegraphics[width=0.14\textwidth]{legend.png}}
     \caption{Comparisons on normalized cuts. In the message passing model, each site samples $5n$ edges; in each round of the algorithm in the blackboard model, all sites jointly sample $10n$ edges (in \twomoons~and \gauss) or $20n$ edges (in \sculpture) edges and the chain has length $18$.}
     \label{fig:quality}
\end{figure*}

%\textcolor{red}{To Jiecao: Can you put the color lines indicating baseline, message passing, and blackboard within one row in Pic 2? Withthis we can save some space.}

%\vspace{-1.5mm}

\subsection{Results on communication costs} 
\begin{figure*}[!ht]
     \centering
     \subfigure[\twomoons]{\includegraphics[width=0.3\textwidth]{twomoons-14000-communication.png}\label{fig:twomoons-communication}}
     \subfigure[\gauss]{\includegraphics[width=0.3\textwidth]{gauss-10000-communication.png}\label{fig:gauss-communication}}
     \subfigure[\sculpture]{\includegraphics[width=0.3\textwidth]{sculpture-11680-communication.png}\label{fig:sculpture-communication}}


     \subfigure[\twomoons]{\includegraphics[width=0.32\textwidth]{twomoons-14000-communication-2.png}\label{fig:twomoons-communication-2}}
     \subfigure[\gauss]{\includegraphics[width=0.32\textwidth]{gauss-10000-communication-2.png}\label{fig:gauss-communication-2}}
     \subfigure[\sculpture]{\includegraphics[width=0.32\textwidth]{sculpture-11680-communication-2.png}\label{fig:sculpture-communication-2}}
     \caption{Comparisons on communication costs. In the message passing model, each site samples $5n$ edges; in each round of the algorithm in the blackboard model, all sites jointly sample $10n$ (in \twomoons~and \gauss) or $20n$ (in \sculpture) edges and the chain has length $18$. }
     \label{fig:communication}
\end{figure*}

We compare the communication costs of different algorithms in Figure \ref{fig:communication}. We observe that while achieving similar clustering qualities as \baseline, both \MM\ and \blackboard\ are significantly more communication-efficient (by one or two orders of magnitudes in our experiments). We also notice that the value of $s$ does not affect the communication cost of \blackboard, while the communication cost of \MM\ grows almost linearly with $s$; when $s$ is large, \MM\ uses significantly more communication than \blackboard. These confirm our theory.  %In Figure~\ref{fig:mm-const} and Figure~\ref{fig:blackboard-const}   in Appendix~\ref{sec:parameters} we present how the performance of \MM\ and \blackboard\ are affected by their parameters.

%
%
%\vspace{-1.5mm}
%\paragraph{Summary.}  From our experimental results we conclude that \MM\ and \blackboard\ achieve similar clustering quality as the native algorithm \baseline, while significantly reduce the communication cost.  When the number of sites is large, \blackboard\ is more communication efficient than \MM, as predicted by our theory.



\subsection{Parameters in \MM\ and \blackboard}
\label{sec:parameters}

Figure \ref{fig:mm-const} shows in \MM how the value of ncut is affected by the number of sites and the number of edges sampled in each site. 
Here, each site samples $cn$ edges. 
When $c=3$ and $s=1$, the ncut value diverges in all datasets. This is because with such a small $c$, the algorithm does not generate a valid sparsifier. In general, increasing $c$ or $s$ will slightly decrease the ncut value. But once they are above some thresholds, the ncut values of \MM\ and \baseline\ become very close.

Figure \ref{fig:blackboard-const} shows in \blackboard  how the ncut value is affected by the number of iterations and the number of edges sampled. When the number of iterations is set to be $5$, ncut values diverge in all datasets. This is because we cannot expect to generate a valid sparsifier by using such few iterations. It can be seen from \ref{fig:bb-gauss-constant} that for a fixed $c$, performing more iterations will help to reduce ncut values. From the same figure, one can also conclude that for fixed iterations, increasing $c$ also helps to reduce the ncut values.



\begin{figure*}[h!t]
     \centering
     \subfigure[\twomoons]{\includegraphics[width=0.3\textwidth]{twomoons-c.png}\label{fig:mm-twomoons-constant}}
     \subfigure[\gauss~dataset]{\includegraphics[width=0.3\textwidth]{gauss-c.png}\label{fig:mm-gauss-constant}}
     \subfigure[\sculpture]{\includegraphics[width=0.3\textwidth]{sculpture-c.png}\label{fig:mm-sculpture-constant}}
     \caption{The pictures above show the $\ncut$ values with respect to the values of $c$ and $s$ for the \MM\ algorithm. Here  
 each site samples $c n$ edges.}
     \label{fig:mm-const}
\end{figure*}


\begin{figure*}[h!t]
     \centering
     \subfigure[\twomoons]{\includegraphics[width=0.3\textwidth]{twomoons-iter.png}\label{fig:bb-twomoons-constant}}
     \subfigure[\gauss]{\includegraphics[width=0.3\textwidth]{gauss-iter.png}\label{fig:bb-gauss-constant}}
     \subfigure[\sculpture]{\includegraphics[width=0.3\textwidth]{sculpture-iter.png}\label{fig:bb-sculpture-constant}}
     \caption{The pictures above show how the $\ncut$ values are affected by the number of iterations and the value of $c$ for the \blackboard\ algorithm. Here 
all sites jointly sample $c n$ edges. }
     \label{fig:blackboard-const}
\end{figure*}






In this paper, 2D and 3D CNN models were used to generate pelvic sCTs from T1-weighted MR images. Our sCT generation methods were fully automated, requiring no deformable registration or manual segmentation of bone tissues. As shown in Figure~\ref{fig3}, the 2D and 3D CNN models generated high quality sCTs. MAE curves shown in Figure~\ref{fig4} indicated that both models could precisely estimate soft-tissue HU values but had difficulty in reproducing air and high-density bone tissues. 

The MAEs within the body contour across all patients were 40.5 $\pm$ 5.4 HU and 37.6 $\pm$ 5.1 HU for the 2D and 3D models, respectively. The time required for generating a pelvic sCT using our CNN models was about 5.5 s. Our MAE results are comparable to previous studies. Kim $et \ al.$\cite{RN41} presented a voxel-based weighted summation method that produced an MAE of 74.3 $\pm$ 3.9 HU. However, manual contouring of bone tissues required for this method can be tedious and time-consuming. An MAE of 40.5 $\pm$ 8.2 HU was achieved by Dowling $et \ al.$\cite{RN11} using an average MRI-CT atlas from 38 patients. Andreasen $et \ al.$\cite{RN42} reported an MAE of 54 $\pm$ 8 HU using an atlas-based method with pattern recognition, and its prediction time was about 20.8 min. Another random forest model proposed by Andreasen $et \ al.$\cite{RN43} generated sCTs with an MAE of 58 $pm$ 9 HU. A hybrid method suggested by Siversson $et \ al.$ \cite{RN45} obtained an MAE of 36.5 $\pm$ 4.1 HU when ignoring errors introduced by gas cavities. This hybrid method was implemented in the cloud-based commercial software MriPlanner (Spectronic Medical AB, Helsingborg, Sweden), which required 50 to 80 min to generate a sCT.\cite{RN45} The patch-based 3D context-aware generative adversarial network presented by Nie $et \ al.$\cite{RN26} achieved an MAE of 39.0 $\pm$ 4.6 HU. 

Our CNN models reproduced low-density bone as shown in Figure ~\ref{fig4}. The bone-region DSCs were 0.81 $\pm$ 0.04 and 0.82 $\pm$ 0.04 from the 2D and 3D models, respectively. These results are comparable to reported DSC results of 0.79 $\pm$ 0.12\cite{RN10} and 0.91$\pm$0.03{\cite{RN11}}, where the authors compared bone contours manually drawn on the sCT and CT.

It was feasible to train the proposed 3D model with 16 image volumes from scratch. Results of the Wilcoxon signed-rank tests shown in Table~\ref{tab1} demonstrated a statistically significant improvement in overall MAE, bone DSC, and bone precision of the 3D model compared to the 2D model. However, as shown in Figure~\ref{fig4}, the 2D model seemed to perform better in estimating the high-density bone HU values. It should be noted that smaller overall MAEs do not guarantee improved sCT dose calculation and patient positioning performance. While the models performed well, we will continue to acquire more patient data to potentially improve model accuracy and further test model differences.

As this was a retrospective study, the MR image voxel sizes were not matched, resulting in different voxel intensities between images. This may have affected the sCT generation accuracy although we applied intensity normalization. A potential study could examine how voxel size variations affects sCT estimation. 

The proposed 3D model can be implemented on a 12 GB GPU to process volumetric images with dimensions of 256 $\times$ 256 $\times$ 30. More GPU memory would be required to process higher resolution 3D images. Considering the limited access to multi-GPU systems, a 3D architecture with fewer convolutional layers could be considered to deal with higher resolutions. However, the performance could be affected by the reduced parameters and smaller receptive fields of the less complex model. Another approach would be to extract 30-slice sub-volumes from CT and MR images for training the 3D model. The sCT could then be generated by averaging 30-slice sCT sub-volumes produced by the model. 

A number of techniques could be investigated for improving model performance.  Nie $et \ al.$\cite{RN26} showed that introducing an additional adversarial discriminator improved overall sCT quality. The same approach could be adapted in our proposed 2D and 3D CNN models.  Non-rigid deformation\cite{RN44} could also be applied to both CT and MR images in the process of the on-the-fly data augmentation to produce more training pairs. Multiple MR images acquired with different sequences could be fed into models to provide more information for distinguishing different tissues. Multi-GPU systems with more memory would enable the exploration of larger batch sizes for training CNN models, which could reduce variances in gradient estimation and accelerate the training. 



\begin{comment}
\begin{figure}
\includegraphics[width=\linewidth]{figs/beyond_tss_lesion.pdf}
\caption[]{End-to-End runtime lesion study of the entire MNIST dataset and the FMA featurized music dataset. Each of DROP's contributions provides a runtime improvement.}
\label{fig:beyond_lesion}
\end{figure}
\end{comment}



\section{Conclusion}
\label{sec:conclusion}

Advanced data analytics techniques must scale to rising data volumes. 
DR techniques offer a powerful toolkit when processing these datasets, with PCA frequently outperforming popular techniques in exchange for high computational cost. 
In response, we propose DROP, a new dimensionality reduction optimizer. 
DROP combines progressive sampling, progress estimation, and online aggregation to identify high quality low dimensional bases via PCA without processing the entire dataset by balancing the runtime of downstream tasks and achieved dimensionality. 
Thus, DROP provides a first step in bridging the gap between quality and efficiency in end-to-end DR for downstream \red{analytics}. 

%We revisit canonical operators for time series dimensionality reduction and the measurement study of~\cite{keogh-study}, and show that PCA is more effective than popular alternatives in the data mining literature often by a margin of over $2\times$ on average on gold-standard time series benchmark data sets with respect to output data dimension. More surprisingly, we empirically demonstrate that a small number of samples are sufficient to accurately characterize directions of maximum variance and obtain a high-quality low-dimensional transformation.










%%
%% The acknowledgments section is defined using the "acks" environment
%% (and NOT an unnumbered section). This ensures the proper
%% identification of the section in the article metadata, and the
%% consistent spelling of the heading.
% \begin{acks}
% % To Robert, for the bagels and explaining CMYK and color spaces.
% \end{acks}

%%
%% The next two lines define the bibliography style to be used, and
%% the bibliography file.
\bibliographystyle{ACM-Reference-Format}
\balance
\bibliography{main}


%%
%% If your work has an appendix, this is the place to put it.
\appendix
\onecolumn


% \tableofcontents{}

% \newpage

\section*{Supplementary Material}
\addcontentsline{toc}{section}{Supplementary Material}


Throughout this discussion, 
we will make frequently use 
of the following standard results
concerning the exponential concentration 
of random variables:

\begin{lemma}[Hoeffding's inequality for independent RVs~\citep{hoeffding1994probability}] Let $Z_1, Z_2, \ldots, Z_n$ be independent bounded random variables with $Z_i \in [a,b]$ for all $i$, then 
    \begin{align*}
        \prob\left( \frac{1}{n} \sum_{i=1}^n (Z_i - \Expo{Z_i}) \ge t \right) \le \exp{\left( -\frac{2nt^2}{(b-a)^2} \right) }
    \end{align*} 
    and 
    \begin{align*}
        \prob\left( \frac{1}{n} \sum_{i=1}^n (Z_i - \Expo{Z_i}) \le -t \right) \le \exp{\left( -\frac{2nt^2}{(b-a)^2} \right) }
    \end{align*} 
    for all $t \ge 0$. 
\end{lemma}

\begin{lemma}[Hoeffding's inequality for sampling with replacement~\citep{hoeffding1994probability}] \label{lem:hoeffding_sampling} Let $\calZ = (Z_1, Z_2, \ldots, Z_N)$ be a finite population of $N$ points with $Z_i \in [a.b]$ for all $i$. Let $X_1, X_2, \ldots X_n$ be a random sample drawn without replacement from $\calZ$. Then for all $t \ge 0$, we have 
    \begin{align*}
        \prob\left( \frac{1}{n} \sum_{i=1}^n (X_i - \mu ) \ge t \right) \le \exp{\left( -\frac{2nt^2}{(b-a)^2} \right) }
    \end{align*} 
    and 
    \begin{align*}
        \prob\left( \frac{1}{n} \sum_{i=1}^n (X_i - \mu ) \le -t \right) \le \exp{\left( -\frac{2nt^2}{(b-a)^2} \right) } \,,
    \end{align*} 
    where $\mu = \frac{1}{N} \sum_{i=1}^{N} Z_i$. 
\end{lemma}

We now discuss one condition that generalizes the exponential concentration to dependent random variables.
\begin{condition}[Bounded difference inequality] \label{cond:BDC} Let $\calZ$ be some set and $\phi: \calZ^n \to \Real$. We say that $\phi$ satisfies the bounded difference assumption if 
there exists $c_1, c_2, \ldots c_n \ge 0$ s.t. for all $i$, we have 
\begin{align*}
    \sup_{Z_1,Z_2, \ldots,Z_n, Z_i^\prime \in \calZ^{n+1} } \abs{\phi (Z_1, \ldots, Z_i, \ldots, Z_n ) - \phi (Z_1, \ldots, Z_i^\prime, \ldots, Z_n ) } \le c_i \,.
\end{align*} 
\end{condition}

\begin{lemma}[McDiarmid’s inequality~\citep{mcdiarmid1989}] \label{lem:McDiarmid} Let $Z_1, Z_2, \ldots, Z_n$ be independent random variables on set $\calZ$ and $\phi : \calZ^n \to \Real$ satisfy bounded difference inequality (\codref{cond:BDC}). Then for all $t>0$, we have 
    \begin{align*}
        \prob\left( \phi(Z_1, Z_2, \ldots, Z_n) - \Expo{\phi(Z_1, Z_2, \ldots, Z_n)} \ge t \right) \le \exp{\left( -\frac{2t^2}{\sum_{i=1}^n c_i^2} \right) } 
    \end{align*} 
    and 
    \begin{align*}
        \prob\left( \phi(Z_1, Z_2, \ldots, Z_n) - \Expo{\phi(Z_1, Z_2, \ldots, Z_n)} \le -t \right) \le \exp{\left( -\frac{2t^2}{\sum_{i=1}^n c_i^2} \right) } \,.
    \end{align*} 
\end{lemma}


\section{Proofs from \secref{sec:ERM_training}}\label{app:proof_erm}

\textbf{Additional notation {} {}} Let $m_1$ be the number of mislabeled points ($\wt S_M$) and $m_2$ be the number of correctly labeled points ($\wt S_C$). Note $m_1 + m_2 = m$. 


\subsection{Proof of \thmref{thm:error_ERM}}


\begin{proof}[Proof of \lemref{lem:fit_mislabeled}] 
    The main idea of our proof is to regard 
    the clean portion of the data 
    ($S \cup \wt S_C$) as fixed.   
    Then, there exists an (unknown) classifier $f^*$ 
    that minimizes the expected risk
    calculated on the (fixed) clean data
    and (random draws of) the mislabeled data $\wt S_M$. 
    % 
    % 
    Formally, 
    \begin{align}
    f^* \defeq \argmin_{f \in \calF} \error_{\widecheck {\calD}} (f) \,, \label{eq:modified_ERM}
    \end{align}
    where $$\widecheck \calD = \frac{n}{m+n} \calS + \frac{m_2}{m+n} \wt \calS_C  + \frac{m_1}{m+n}\calDm \,.$$ 
    Note here that $\widecheck \calD$ is a combination 
    of the \emph{empirical distribution} 
    over correctly labeled data $S \cup \wt S_C$
    and the (population) distribution 
    over mislabeled data $\calDm$.
    Recall that 
    \begin{align}
    \wh f \defeq \argmin_{f \in \calF} \error_{\calS \cup \wt S} (f) \,. \label{eq:orig_ERM}
    \end{align}
    % 
    % 
    Since, $\widehat f$ minimizes 0-1 error 
    on $S \cup \wt S$, using ERM optimality on \eqref{eq:orig_ERM},  
    we have 
    \begin{align}
        \error_{\calS \cup \wt \calS}(\widehat f) \le \error_{
            \calS \cup \wt \calS}(f^*) \,.    \label{eq:step1}
    \end{align}
    Moreover, since $f^*$ is independent of $\wt S_M$, using Hoeffding's bound,
    % \footnote{For a fully rigorous argument,
    % refer to the complete proof in App.~\ref{app:proof_erm}.} 
    we have with probability at least $1-\delta$ that
    \begin{align}
      \error_{\wt \calS_M}(f^*) \le \error_{ \calDm}(f^*) +  \sqrt{\frac{\log(1/\delta)}{2 m_1}} \,. \label{eq:step2} 
    \end{align}
    %$ 
    %for some constant $c_1\le 1/2$. 
    Finally, since $f^*$ is the optimal classifier on $\widecheck \calD$, 
    we have 
    \begin{align}
        \error_{\widecheck \calD}(f^*) \le \error_{\widecheck \calD}(\widehat f) \,. \label{eq:step3}
    \end{align}
    Now to relate \eqref{eq:step1} and \eqref{eq:step3}, we multiply \eqref{eq:step2} by $\frac{m_1}{m+n}$ and add $\frac{n}{m+n} \error_{\calS} (f)  + \frac{m_2}{m+n} \error_{\wt \calS_C} (f)$ both the sides. Hence, 
    we can rewrite \eqref{eq:step2} as follows: 
    \begin{align}
        \error_{\calS \cup \wt\calS}(f^*) \le \error_{ \widecheck \calD}(f^*) +  \frac{m_1}{m+n}\sqrt{\frac{\log(1/\delta)}{2 m_1}} \,. \label{eq:step4} 
    \end{align}
    Now we combine equations \eqref{eq:step1}, \eqref{eq:step4}, and \eqref{eq:step3}, to get 
    \begin{align}
        \error_{\calS \cup \wt \calS}(\wh f) \le \error_{\widecheck \calD}(\wh f) +  \frac{m_1}{m+n}\sqrt{\frac{\log(1/\delta)}{2 m_1}} \,, 
    \end{align}
    which implies 
    \begin{align}
        \error_{ \wt \calS_M}(\wh f) \le \error_{\calDm}(\wh f) + \sqrt{\frac{\log(1/\delta)}{2 m_1}} \,. \label{eq:lemma1_final}
    \end{align}
    Since $\wt S$ is obtained by randomly labeling an unlabeled dataset, we assume $2m_1 \approx m$ \footnote{Formally, with probability at least $1-\delta$, we have  $(m - 2m_1)\le \sqrt{m\log(1/\delta)/2}$.}. Moreover, using $\error_{\calDm} = 1 - \error_{\calD}$ we obtain the desired result.   
    % Combining the above steps and using the fact 
    % that $\error_\calD = 1- \error_{\calDm} $, 
    % we obtain the desired result.
\end{proof}

\begin{proof}[Proof of \lemref{lem:mislabeled_error}]
    Recall $\error_{\wt S} (f) = \frac{m_1}{m} \error_{\wt S_M}(f) + \frac{m_2}{m} \error_{\wt S_C}(f)$. Hence, we have 
    \begin{align}
        2\error_{\wt S}(f) - \error_{\wt S_M}(f) - \error_{\wt S_C}(f) &= \left(\frac{2m_1}{m} \error_{\wt S_M}(f) - \error_{\wt S_M}(f)\right) + \left(\frac{2m_2}{m} \error_{\wt S_C}(f) - \error_{\wt S_C}(f)\right) \\ &= \left(\frac{2m_1}{m} - 1\right) \error_{\wt S_M}(f) + \left(\frac{2m_2}{m} - 1 \right)\error_{\wt S_C} (f) \,.
    \end{align} 
    Since the dataset is labeled uniformly at random, with probability at least $1-\delta$, we have  $\left(\frac{2m_1}{m} - 1\right) \le \sqrt{\frac{\log(1/\delta)}{2m}}$. Similarly, we have with probability at least $1-\delta$, $\left(\frac{2m_2}{m} - 1\right) \le \sqrt{\frac{\log(1/\delta)}{2m}}$. Using union bound, with probability at least $1-\delta$, we have
    % \begin{align}
    %     2\error_{\wt S} - \error_{\wt S_M}(f) - \error_{\wt S_C}(f) \le \sqrt{\frac{\log(2/\delta)}{2m}} \left(\error_{\wt S_M}(f) + \error_{\wt S_C}(f) \right) \le 2\sqrt{\frac{\log(2/\delta)}{2m}} \,. \label{eq:lemma2_final}
    % \end{align}
    \begin{align}
        2\error_{\wt S} - \error_{\wt S_M}(f) - \error_{\wt S_C}(f) \le \sqrt{\frac{\log(2/\delta)}{2m}} \left(\error_{\wt S_M}(f) + \error_{\wt S_C}(f) \right) \,. \label{eq:lemma2_prefinal}
    \end{align}
    With re-arranging $\error_{\wt S_M}(f) + \error_{\wt S_C}(f)$ and using the inequality $ 1- a\le \frac{1}{1+a} $, we have  
    \begin{align}
        2\error_{\wt S} - \error_{\wt S_M}(f) - \error_{\wt S_C}(f) \le 2\error_{\wt \calS} \sqrt{\frac{\log(2/\delta)}{2m}}  \,. \label{eq:lemma2_final}
    \end{align}

    % We obtain the desired result by using 
\end{proof}

\begin{proof}[Proof of \lemref{lem:clear_error}]
% Recall 0-1 error on each point  $(x,y) \in S \cup \wt S$ is given by $\I{ f(x)\ne y}$.
In the set of correctly labeled points $S \cup \wt S_C$, we have $S$ as a random subset of $S \cup \wt S_C$. Hence, using Hoeffding's inequality for sampling without replacement (\lemref{lem:hoeffding_sampling}), we have with probability at least $1-\delta$
\begin{align}
    \error_{\wt \calS_C} (\wh f)- \error_{\calS \cup \wt \calS_C}( \wh f) \le  \sqrt{\frac{\log(1/\delta)}{2m_2}} \,.
\end{align}
Re-writing $\error_{\calS \cup \wt \calS_C}( \wh f)$ as $\frac{m_2}{m_2 + n} \error_{\wt \calS_C }(\wh f) + \frac{n}{m_2 + n} \error_{\calS }(\wh f)$, we have with probability at least $1-\delta$
\begin{align}
   \left(\frac{n}{n+m_2}\right) \left(\error_{\wt \calS_C} (\wh f)- \error_{\calS}( \wh f) \right) \le  \sqrt{\frac{\log(1/\delta)}{2m_2}} \,.
\end{align}
As before, assuming $2m_2 \approx m$, we have with probability at least $1-\delta$ 
\begin{align}
    \error_{\wt \calS_C} (\wh f)- \error_{\calS}( \wh f) \le \left(1+\frac{m_2}{n}\right)  \sqrt{\frac{\log(1/\delta)}{m}} \le \left(1 + \frac{m}{2n}\right) \sqrt{\frac{\log(1/\delta)}{m}} \,. \label{eq:lemma3_final}
\end{align} 
\end{proof}

\begin{proof}[Proof of \thmref{thm:error_ERM}] 
    Having established these core intermediate results, we can now combine above three lemmas to prove the main result. 
    In particular, we bound the population error on clean data ($\error_\calD(\wh f)$) as follows:  
    \begin{enumerate}[(i)]
        \item First, use \eqref{eq:lemma1_final}, to obtain an upper bound on the population error on clean data, i.e., with probability at least $1-\delta/4$, we have
        \begin{align}
            \error_{ \calD} (\wh f) \le 1 - \error_{ \wt \calS_M}(\wh f) + \sqrt{\frac{\log(4/\delta)}{m}} \,. 
        \end{align}
        \item  Second, use \eqref{eq:lemma2_final}, to relate the error on the mislabeled fraction with error on clean portion of randomly labeled data and error on whole randomly labeled dataset, i.e., with probability at least $1-\delta/2$, we have 
        \begin{align}
            - \error_{\wt S_M}(f) \le \error_{\wt S_C}(f) - 2\error_{\wt S}  + 2\error_{\wt S} \sqrt{\frac{\log(4/\delta)}{2m}}  \,. 
        \end{align} 
        \item Finally, use \eqref{eq:lemma3_final} to relate the error on the clean portion of randomly labeled data and error on clean training data, i.e., with probability $1-\delta/4$, we have 
        \begin{align}
            \error_{\wt \calS_C} (\wh f)\le - \error_{\calS}( \wh f) + \left(1 + \frac{m}{2n} \right) \sqrt{\frac{\log(4/\delta)}{m}} \,. 
        \end{align} 
    \end{enumerate}

    Using union bound on the above three steps, we have with probability at least $1-\delta$: 
    \begin{align}
        \error_\calD (\wh f) \le \error_{\calS}(\wh f)   + 1 - 2\error_{\wt \calS}(\wh f)   + \left(\sqrt{2} \error_{\wt S} + 2 + \frac{m}{2n}\right)  \sqrt{\frac{\log(4/\delta)}{m}} \,.
    \end{align}
    % Note that $(1/\sqrt{2} + 2.5)$ is a loose constant. In experiments, we use the ratio $\frac{m}{n}$
    %  the exact error $\error_{\wt \calS}(\wh f)$ 
    % to evaluate R.H.S.    
\end{proof}

\subsection{Proof of \propref{prop:rademacher}}

\begin{proof}[Proof of \propref{prop:rademacher}]
    For a classifier $ f: \calX \to \{-1, 1\}$, we have $1 - 2\,\indict{ f(x) \ne y} = y \cdot f(x)$. Hence, by definition of $\error$, we have 
    \begin{align}
        1 -2\error_{\wt \calS}(f) = \frac{1}{m}\sum_{i=1}^m y_i \cdot f(x_i) \le \sup_{f \in \calF} \, \frac{1}{m} \sum_{i=1}^m y_i \cdot f(x_i)  \,. \label{eq:error_rademacher}
    \end{align}
    Note that for fixed inputs $(x_1, x_2, \ldots, x_m)$ in $\wt S$, $(y_1, y_2, \ldots y_m)$ are random labels. Define $\phi_1 (y_1, y_2, \ldots, y_m) \defeq \sup_{f \in \calF} \, \frac{1}{m} \sum_{i=1}^m y_i \cdot f(x_i)$. We have the following bounded difference condition on $\phi_1$. For all i, 
    \begin{align}
        \sup_{y_1, \ldots y_m, y_i^\prime \in \{-1, 1\}^{m+1} } \abs{ \phi_1 (y_1,\ldots, y_i, \ldots, y_m) - \phi_1 (y_1,\ldots, y_i^\prime, \ldots, y_m)  } \le 1/m \,. \label{cond1_rademacher}
    \end{align} 
    
    Similarly, we define $\phi_2 (x_1, x_2, \ldots, x_m) \defeq \Expt{ y_i \sim_U \{-1, 1\}  }{ \sup_{f \in \calF} \, \frac{1}{m}  \sum_{i=1}^m y_i \cdot f(x_i)}$. We have the following bounded difference condition on $\phi_2$. 
    For all i,
    \begin{align}
        \sup_{x_1, \ldots x_m, x_i^\prime \in \calX^{m+1} } \abs{ \phi_2 (x_1,\ldots, x_i, \ldots, x_m) - \phi_1 (x_1,\ldots, x_i^\prime, \ldots, x_m)  } \le 1/m \,. \label{cond2_rademacher}
    \end{align}
    Using McDiarmid’s inequality (\lemref{lem:McDiarmid}) twice 
    with Condition \eqref{cond1_rademacher} and \eqref{cond2_rademacher}, 
    with probability at least $1-\delta$, we have
    \begin{align}
        \sup_{f \in \calF} \, \frac{1}{m} \sum_{i=1}^m y_i \cdot f(x_i)  - \Expt{x,y}{\sup_{f \in \calF} \, \frac{1}{m} \sum_{i=1}^m y_i \cdot f(x_i) } \le \sqrt{\frac{2\log(2/\delta)}{m}} \,. \label{eq:final_rademacher}
    \end{align} 
    Combining \eqref{eq:error_rademacher} and \eqref{eq:final_rademacher}, we obtain the desired result. 
\end{proof}


\subsection{Proof of \thmref{thm:error_regularized_ERM}}

Proof of \thmref{thm:error_regularized_ERM} follows similar to the proof of \thmref{thm:error_ERM}. Note that the same results in \lemref{lem:fit_mislabeled}, \lemref{lem:mislabeled_error}, and \lemref{lem:clear_error} hold in the regularized ERM case. However, the arguments in the proof of \lemref{lem:fit_mislabeled} change slightly. Hence, we state the lemma for regularized ERM and prove it here for completeness. 

\begin{lemma} \label{lem:lemma1_reg}
    Assume the same setup as \thmref{thm:error_regularized_ERM}. 
    Then for any $\delta >0$, with probability at least  $1-\delta$ 
    over the random draws of mislabeled data $\wt S_M$, we have 
    \begin{align}
        \error_\calD(\widehat f)  \le 1 -\error_{\wt \calS_M}(\widehat f) + \sqrt{\frac{\log(1/\delta)}{m}}\,. 
    \end{align} 
\end{lemma}
\begin{proof}
    The main idea of the proof remains the same, i.e. regard 
    the clean portion of the data 
    ($S \cup \wt S_C$) as fixed.   
    Then, there exists a classifier $f^*$ 
    that is optimal over draws 
    of the mislabeled data $\wt S_M$. 

    
    Formally, 
    \begin{align}
    f^* \defeq \argmin_{f \in \calF} \error_{\widecheck {\calD}} (f)  + \lambda R(f) \,, \label{eq:modified_ERM_reg}
    \end{align}
    where $$\widecheck \calD = \frac{n}{m+n} \calS + \frac{m_1}{m+n} \wt \calS_C  + \frac{m_2}{m+n}\calDm \,.$$ That is, $\widecheck \calD$ a combination of 
    the \emph{empirical distribution} 
    over correctly labeled data $S \cup \wt S_C$
    % in $S\cup \wt S$ 
    and the (population) distribution 
    over mislabeled data $\calDm$.
    Recall that 
    \begin{align}
    \wh f \defeq \argmin_{f \in \calF} \error_{\calS \cup \wt S} (f) + \lambda R(f) \,. \label{eq:orig_ERM_reg}
    \end{align}
    % 
    % 
    Since, $\widehat f$ minimizes 0-1 error 
    on $S \cup \wt S$, using ERM optimality on \eqref{eq:orig_ERM},  
    we have 
    \begin{align}
        \error_{\calS \cup \wt \calS}(\widehat f) + \lambda R(\wh f) \le \error_{
            \calS \cup \wt \calS}(f^*) + \lambda R(f^*) \,.    \label{eq:step1_reg}
    \end{align}
    Moreover, since $f^*$ is independent of $\wt S_M$, using Hoeffding's bound,
    % \footnote{For a fully rigorous argument,
    % refer to the complete proof in App.~\ref{app:proof_erm}.} 
    we have with probability at least $1-\delta$ that
    \begin{align}
      \error_{\wt \calS_M}(f^*) \le \error_{ \calDm}(f^*) +  \sqrt{\frac{\log(1/\delta)}{2 m_1}} \,. \label{eq:step2_reg} 
    \end{align}
    %$ 
    %for some constant $c_1\le 1/2$. 
    Finally, since $f^*$ is the optimal classifier on $\widecheck \calD$, 
    we have 
    \begin{align}
        \error_{\widecheck \calD}(f^*) + \lambda R(f^*) \le \error_{\widecheck \calD}(\widehat f) + \lambda R(\wh f) \,. \label{eq:step3_reg}
    \end{align}
     Now to relate \eqref{eq:step1_reg} and \eqref{eq:step3_reg}, we can re-write the \eqref{eq:step2_reg} as follows: 
    \begin{align}
        \error_{\calS \cup \wt\calS}(f^*) \le \error_{ \widecheck \calD}(f^*) +  \frac{m_1}{m+n}\sqrt{\frac{\log(1/\delta)}{2 m_1}} \,. \label{eq:step4_reg} 
    \end{align}
    After adding $\lambda R(f^*)$ on both sides in \eqref{eq:step4_reg}, we combine equations \eqref{eq:step1_reg}, \eqref{eq:step4_reg}, and \eqref{eq:step3_reg}, to get 
    \begin{align}
        \error_{\calS \cup \wt \calS}(\wh f) \le \error_{\widecheck \calD}(\wh f) +  \frac{m_1}{m+n}\sqrt{\frac{\log(1/\delta)}{2 m_1}} \,, 
    \end{align}
    which implies 
    \begin{align}
        \error_{ \wt \calS_M}(\wh f) \le \error_{\calDm}(\wh f) + \sqrt{\frac{\log(1/\delta)}{2 m_1}} \,. \label{eq:lemma_reg_final}
    \end{align}
    Similar as before, since $\wt S$ is obtained by randomly labeling an unlabeled dataset, we assume 
    $2m_1 \approx m$. Moreover, using $\error_{\calDm} = 1 - \error_{\calD}$ we obtain the desired result. 
\end{proof}
% \begin{proof}[Proof of ]
    
% \end{proof}

\subsection{Proof of \thmref{thm:multiclass_ERM}}

To prove our results in the multiclass case,
we first state and prove lemmas
parallel to those
% We first state and prove lemmas 
% parallel 
% to the three lemmas 
used in the proof of balanced binary case. 
We then combine these results 
% in the three lemmas 
to obtain the result in \thmref{thm:multiclass_ERM}. 

Before stating the result, 
we define mislabeled distribution $\calDm$ for any $\calD$.
While $\calDm$ and $\calD$ share 
the same marginal distribution over inputs $\calX$,
the conditional distribution over labels $y$ 
given an input $x\sim \calD_\calX$ is changed as follows:
For any $x$, the Probability Mass Function (PMF) over $y$ is defined as:  
$p_{\calDm} (\cdot \vert x) \defeq \frac{1 - p_{\calD}(\cdot \vert x)}{k - 1}$, where $ p_{\calD}(\cdot \vert x)$ is the PMF over $y$ for the distribution $\calD$. 

\begin{lemma} \label{lem:fit_mislabeled_multi}
    Assume the same setup as \thmref{thm:multiclass_ERM}. 
    Then for any $\delta >0$, with probability at least  $1-\delta$ 
    over the random draws of mislabeled data $\wt S_M$, we have 
    \begin{align}
        \error_\calD(\widehat f)  \le (k-1)\left(1 -\error_{\wt \calS_M}(\widehat f)\right) + (k-1)\sqrt{\frac{\log(1/\delta)}{m}}\,. \label{eq:lemma1_multi}
    \end{align}   
\end{lemma} 

\begin{proof}
   
    The main idea of the proof remains the same.
    We begin by regarding the clean portion of the data 
    ($S \cup \wt S_C$) as fixed. 
    Then, there exists a classifier $f^*$ 
    that is optimal over draws 
    of the mislabeled data $\wt S_M$. 
    
    However, in the multiclass case,
    we cannot as easily relate the population error on mislabeled data 
    to the population accuracy on clean data.   
    While for binary classification, 
    % we could upper bound $\error_{\wt \calS_M}$ 
    % with $1-\error_\calD$ 
    we could lower bound the population accuracy $1-\error_\calD$
    with the empirical error on mislabeled data $\error_{\wt \calS_M}$ 
    (in the proof of \lemref{lem:fit_mislabeled}), 
    for multiclass classification, 
    error on the mislabeled data 
    and accuracy on the clean data 
    in the population 
    are not so directly related.  
    To establish \eqref{eq:lemma1_multi},
    we break the error on the 
    (unknown) mislabeled data 
    into two parts: one term corresponds 
    to predicting the true label on mislabeled data, 
    and the other corresponds to predicting 
    neither the true label 
    nor the assigned (mis-)label.  
    Finally, we relate these errors to their
    population counterparts to establish \eqref{eq:lemma1_multi}. 
    
    Formally, 
    \begin{align}
    f^* \defeq \argmin_{f \in \calF} \error_{\widecheck {\calD}} (f)  + \lambda R(f) \,, \label{eq:modified_ERM_reg2}
    \end{align}
    where $$\widecheck \calD = \frac{n}{m+n} \calS + \frac{m_1}{m+n} \wt \calS_C  + \frac{m_2}{m+n}\calDm \,.$$ 
    That is, $\widecheck \calD$ is a combination 
    of the \emph{empirical distribution} 
    over correctly labeled data $S \cup \wt S_C$
    % in $S\cup \wt S$ 
    and the (population) distribution 
    over mislabeled data $\calDm$.
    Recall that 
    \begin{align}
    \wh f \defeq \argmin_{f \in \calF} \error_{\calS \cup \wt S} (f) + \lambda R(f) \,. \label{eq:orig_ERM_reg2}
    \end{align}
    % 
    % 
    Following the exact steps from the proof of \lemref{lem:lemma1_reg}, 
    with probability at least $1-\delta$, we have  
    \begin{align}
        \error_{ \wt \calS_M}(\wh f) \le \error_{\calDm}(\wh f) + \sqrt{\frac{\log(1/\delta)}{2 m_1}} \,. \label{eq:lemma1_final_multi_prev}
    \end{align}
    Similar to before, since $\wt S$ is obtained 
    by randomly labeling an unlabeled dataset, 
    we assume 
    $\frac{k}{k-1} m_1 \approx m$. 
    
    Now we will relate $\error_{\calDm} (\wh f)$ with $\error_{\calD}(\wh f)$. 
    Let $y^T$ denote the (unknown) true label 
    for a mislabeled point $(x, y)$ 
    (i.e., label before replacing it with a mislabel). 
    \begin{align*}    
         \Expt{(x, y) \in \sim \calDm}{\indict{ \wh f(x) \ne y }}  &= \underbrace{\Expt{(x, y) \in \sim \calDm}{\indict{ \wh f(x) \ne y \land \wh f(x) \ne y^T}}}_{\RN{1}} \\ &\qquad \qquad + \underbrace{\Expt{(x, y) \in \sim \calDm}{\indict{ \wh f(x) \ne y \land \wh f(x) = y^T}}}_{\RN{2}} \,. \numberthis \label{eq:excess_term}
    \end{align*}
    Clearly, term 2 is one minus the accuracy 
    on the clean unseen data, i.e.,
    \begin{align}
        \RN{2} = 1 - \Expt{{x,y} \sim \calD}{ \indict{ \wh f(x) \ne y}} = 1- \error_{\calD}(\wh f) \,. \label{eq:term1}    
    \end{align}
    Next, we relate term 1 with the error on the unseen clean data. 
    We show that term 1 is equal to the error on the unseen clean data 
    scaled by $\frac{k-2}{k-1}$,
    where $k$ is the number of labels.
    Using the definition of mislabeled distribution $\calDm$,  
    we have 
    \begin{align}
        \RN{1} = \frac{1}{k-1} \left( \Expt{(x, y) \in \sim \calD}{ \sum_{i \in \calY \land i\ne y}  \indict{ \wh f(x) \ne i \land \wh f(x) \ne y}} \right) = \frac{k-2}{k-1} \error_{\calD}(\wh f) \,.\label{eq:term2}
    \end{align}    

    Combining the result in \eqref{eq:term1}, \eqref{eq:term2} and \eqref{eq:excess_term}, we have 
    \begin{align}
        \error_{\calDm}(\wh f) = 1- \frac{1}{k-1} \error_{\calD}(\wh f) \,.\label{eq:combine_terms}
    \end{align}
    Finally, combining the result in \eqref{eq:combine_terms} 
    with equation \eqref{eq:lemma1_final_multi_prev}, 
    we have with probability $1-\delta$, 
    \begin{align}
      \error_{\calD}(\wh f) \le  (k-1) \left( 1- \error_{ \wt \calS_M}(\wh f) \right)  + (k-1) \sqrt{\frac{k \log(1/\delta)}{ 2(k-1)m}} \,. \label{eq:lemma1_final_multi}
    \end{align}
\end{proof}

\begin{lemma} \label{lem:mislabeled_error_multi}
    Assume the same setup as \thmref{thm:multiclass_ERM}. 
    Then for any $\delta >0$, 
    with probability at least $1-\delta$ 
    over the random draws of $\wt S$, we have  
    % \begin{align}
        $$\abs{k\error_{\wt \calS}(\widehat f) - \error_{\wt \calS_C}(\widehat f) -  (k-1)\error_{\wt \calS_M}(\widehat f) } \le  2k\sqrt{\frac{\log(4/\delta)}{2m}}\,. $$ % \label{eq:lemma2}
    % \end{align}   
    %  for some constant $c_3 \le 1.0\,$.
\end{lemma} 


\begin{proof}
    Recall $\error_{\wt S} (f) = \frac{m_1}{m} \error_{\wt S_M}(f) + \frac{m_2}{m} \error_{\wt S_C}(f)$. Hence, we have 
    \begin{align*}
        k\error_{\wt S}(f) - (k-1)\error_{\wt S_M}(f) - \error_{\wt S_C}(f) &= (k-1)\left(\frac{k m_1}{(k-1) m} \error_{\wt S_M}(f) - \error_{\wt S_M}(f)\right) \\ & \qquad \qquad + \left(\frac{km_2}{m} \error_{\wt S_C}(f) - \error_{\wt S_C}(f)\right) \\ &= k \left[ \left(\frac{m_1}{m} - \frac{k-1}{k}\right) \error_{\wt S_M}(f) + \left(\frac{m_2}{m} - \frac{1}{k} \right) \error_{\wt S_C} (f) \right] \,.
    \end{align*} 
    Since the dataset is randomly labeled, 
    we have with probability at least $1-\delta$, 
    $\left(\frac{m_1}{m} - \frac{k-1}{k}\right) \le \sqrt{\frac{\log(1/\delta)}{2m}}$. 
    Similarly, we have with probability at least $1-\delta$, 
    $\left(\frac{m_2}{m} - \frac{1}{k}\right) \le \sqrt{\frac{\log(1/\delta)}{2m}}$. 
    Using union bound, we have with probability at least $1-\delta$
    % \begin{align}
    %     2\error_{\wt S} - \error_{\wt S_M}(f) - \error_{\wt S_C}(f) \le \sqrt{\frac{\log(2/\delta)}{2m}} \left(\error_{\wt S_M}(f) + \error_{\wt S_C}(f) \right) \le 2\sqrt{\frac{\log(2/\delta)}{2m}} \,. \label{eq:lemma2_final}
    % \end{align}
    \begin{align}
        k\error_{\wt S}(f) - (k-1)\error_{\wt S_M}(f) - \error_{\wt S_C}(f)  \le k \sqrt{\frac{\log(2/\delta)}{2m}} \left(\error_{\wt S_M}(f) + \error_{\wt S_C}(f) \right) \,. \label{eq:lemma2_final_multi}
    \end{align}

    % We obtain the desired result by using 
\end{proof}

\begin{lemma} \label{lem:clear_error_multi}
    Assume the same setup as \thmref{thm:multiclass_ERM}. 
    Then for any $\delta >0$, with probability at least $1-\delta$ 
    over the random draws of $\wt S_C$ and $S$, we have 
    % \begin{align}
        $$\abs{\error_{\wt \calS_C}(\widehat f) - \error_{\calS}(\widehat f) } \le 1.5 \sqrt{\frac{k\log(2/\delta)}{2m}}\,.$$ %\label{eq:lemma3}
    % \end{align}   
    % for some constant $c_2 \le 1.2\,$.
\end{lemma} 
\begin{proof}
    % Recall 0-1 error on each point  $(x,y) \in S \cup \wt S$ is given by $\I{ f(x)\ne y}$.
    In the set of correctly labeled points $S \cup \wt S_C$,
    we have $S$ as a random subset of $S \cup \wt S_C$. 
    Hence, using Hoeffding's inequality 
    for sampling without replacement 
    (\lemref{lem:hoeffding_sampling}), 
    we have with probability at least $1-\delta$
    \begin{align}
        \error_{\wt \calS_c} (\wh f)- \error_{\calS \cup \wt \calS_C}( \wh f) \le  \sqrt{\frac{\log(1/\delta)}{2m_2}} \,.
    \end{align}
    Re-writing $\error_{\calS \cup \wt \calS_C}( \wh f)$ 
    as $\frac{m_2}{m_2 + n} \error_{\wt \calS_C }(\wh f) + \frac{n}{m_2 + n} \error_{\calS }(\wh f)$, 
    we have with probability at least $1-\delta$
    \begin{align}
       \left(\frac{n}{n+m_2}\right) \left(\error_{\wt \calS_c} (\wh f)- \error_{\calS}( \wh f) \right) \le  \sqrt{\frac{\log(1/\delta)}{2m_2}} \,.
    \end{align}
    As before, assuming $km_2 \approx m$, 
    we have with probability at least $1-\delta$ 
    \begin{align}
        \error_{\wt \calS_c} (\wh f)- \error_{\calS}( \wh f) \le \left(1+\frac{m_2}{n}\right)  \sqrt{\frac{k\log(1/\delta)}{2m}} \le \left( 1 + \frac{1}{k}\right) \sqrt{\frac{k\log(1/\delta)}{2m}} \,. \label{eq:lemma3_final_multi}
    \end{align} 
\end{proof}

\begin{proof}[Proof of \thmref{thm:multiclass_ERM}] 
    Having established these core intermediate results, 
    we can now combine above three lemmas. 
    In particular, we bound the population error 
    on clean data ($\error_\calD(\wh f)$) as follows:  
    \begin{enumerate}[(i)]
        \item First, use \eqref{eq:lemma1_final_multi}, 
        to obtain an upper bound on the population error on clean data, 
        i.e., with probability at least $1-\delta/4$, we have
        \begin{align}
            \error_{ \calD} (\wh f) \le (k-1)\left(1 - \error_{ \wt \calS_M}(\wh f) \right) + (k-1) \sqrt{\frac{k\log(4/\delta)}{2(k-1)m}} \,. 
        \end{align}
        \item  Second, use \eqref{eq:lemma2_final_multi}
        to relate the error on the mislabeled fraction 
        with error on clean portion of randomly labeled data 
        and error on whole randomly labeled dataset, 
        i.e., with probability at least $1-\delta/2$, we have 
        \begin{align}
            - (k-1)\error_{\wt S_M}(f) \le \error_{\wt S_C}(f) - k\error_{\wt S}  + k\sqrt{\frac{\log(4/\delta)}{2m}}  \,. 
        \end{align} 
        \item Finally, use \eqref{eq:lemma3_final_multi} 
        to relate the error on the clean portion of randomly labeled data 
        and error on clean training data, 
        i.e., with probability $1-\delta/4$, we have 
        \begin{align}
            \error_{\wt \calS_C} (\wh f)\le - \error_{\calS}( \wh f) + \left(1 + \frac{m}{kn} \right) \sqrt{\frac{k\log(4/\delta)}{2m}} \,. 
        \end{align} 
    \end{enumerate}

    Using union bound on the above three steps, 
    we have with probability at least $1-\delta$: 
    \begin{align}
        \error_\calD (\wh f) \le \error_{\calS}(\wh f) + (k-1) - k\error_{\wt \calS}(\wh f)   + (\sqrt{k(k-1)} + k + \sqrt{k} + \frac{m}{n\sqrt{k}})  \sqrt{\frac{\log(4/\delta)}{2m}} \,.\label{eq:multiclass_ERM_final}
    \end{align}
    Simplifying the term in RHS of \eqref{eq:multiclass_ERM_final}, 
    we get the desired result. 
    % Note that since $\frac{m}{n\sqrt{k}}$ 
    % is much smaller than the sum of the other terms
    % the other terms in summation, 
    % we ignore $\frac{m}{n\sqrt{k}}$  
    % Z: ??? --- great
    % that 
    % them
    in the final bound. 
    % we ignore that in the final bound. 
    % Note that $(1/\sqrt{2} + 2.5)$ is a loose constant. In experiments, we use the ratio $\frac{m}{n}$
    %  the exact error $\error_{\wt \calS}(\wh f)$ 
    % to evaluate R.H.S.    
\end{proof}

\newpage
\section{Proofs from \secref{sec:linear_models}}\label{app:proof_gd}
We suppose that the parameters of the linear function 
are obtained via gradient descent on 
the following $L_2$ regularized problem: 
\begin{align}
    % n in denominator is avoided deliberately
    \calL_S(w; \lambda) \defeq \sum_{i=1}^n{(w^Tx_i - y_i)^2} + \lambda \norm{w}{2}^2 \,, \label{eq:l2_MSE_app}   
\end{align}
where $\lambda\ge0$ is a regularization parameter. 
We assume access to a clean dataset 
$S = \{(x_i, y_i)\}_{i=1}^n \sim \calD^n$ 
and randomly labeled dataset 
$\wt S = \{(x_i, y_i)\}_{i=n+1}^{n+m} \sim \wt \calD^m$. 
Let $\bX = [x_1, x_2, \cdots, x_{m+n}]$ 
and $\by = [y_1, y_2, \cdots, y_{m+n}]$. 
Fix a positive learning rate $\eta$ such that 
$\eta \le 1/\left(\norm{\bX^T\bX}{\text{op}} + \lambda^2\right)$ 
and an initialization $w_0 = 0$. 
% \todos{Assumption made for simplicty}. 
Consider the following gradient descent iterates 
to minimize objective \eqref{eq:l2_MSE_app} on $S \cup \wt S$:
\begin{align}
w_t = w_{t-1} - \eta \grad_w \calL_{S \cup \wt S} (w_{t-1}; \lambda) \quad \forall t=1,2,\ldots \label{eq:GD_iterates_app}
\end{align} 
Then we have $\{ w_t\}$ converge to the limiting solution 
$\wh w = \left( \bX^T\bX+\lambda \boldsymbol{I}\right)^{-1}\bX^T\by$. Define $\widehat f (x) \defeq f(x ; \wh w) $.  

% \subsection{\textcolor{red}{Errata}}

% We wish to correct the following error in the body:
% \codref{cond:error_stability} is not enough 
% to guarantee the result in \thmref{thm:linear}. 
% We now present a slightly stronger condition 
% called \emph{hypothesis stability} 
% under which we obtain a result 
% similar to \thmref{thm:linear}. 

% This error doesn't change the main arguments of the proof,
% where we show that the empirical train error 
% is less than or equal to the leave-one-out error.
% We need a stronger condition to relate leave-one-out error 
% with the population error of the original classifier. 
% Specifically, while \codref{cond:error_stability} 
% relates the average population error of leave-one-out classifiers 
% with the population error of the original classifier, 
% we need the new condition to show the concentration 
% of the empirical leave-one-out error 
% and average population error of leave-one-out classifiers. 
% main takeaway 

% Note that the new condition, 
% while being stronger than the previous one, 
% still doesn't imply generalization \citep{bousquet2002stability,elisseeff2003leave,abou2019exponential}. 
% Overall, the main results in \secref{sec:ERM_training} 
% and takeaways of the paper remain unaffected by the error.  

% We now present the new condition 
% and a corrected statement of \thmref{thm:linear}. 
% Recall, for a given training set $S \sim \calD^n $, 
% we use $S_{(i)}$ to denote the training set $S$ 
% with the $i^{\text{th}}$ point removed.

% \begin{condition}[Hypothesis Stability] 
%     \label{cond:hypothesis_stability}
%     We have $\beta$ hypothesis stability 
%     if our training algorithm $\calA$ satisfies the following: 
%     \begin{align*}
%     % ${\sum_{i=1}^n \frac{\error_{\calD}( f(\calA, S_{(i)}))}{n} - \error_\calD(f(\calA, S))} \le \beta\,$.
%     \forall i \in \{1,2,\ldots, n\}, \quad  \Expt{\calS, (x,y) \in \calD}{ \abs{\error\left( f(x) ,y  \right) - \error\left( f_{(i)}(x), y \right) }} \le \frac{\beta}{n} \,,
%     \end{align*}
%     where $f_{(i)} \defeq f(\calA, S_{(i)})$ and $ f \defeq f(\calA, S)$.
% \end{condition}

% \begin{theorem}[Correct statement of \thmref{thm:linear}] \label{thm:new_linear}
%     Assume that this gradient descent algorithm satisfies \codref{cond:hypothesis_stability}
%     with $\beta=\calO(1)$.  
%     Then for any $\delta >0$, with probability at least $1-\delta$ 
%     over the random draws of datasets $\wt S$ and $S$, we have:
%     \begin{align}
%         \error_\calD(\widehat f) \le \error_\calS(\widehat f) + 1 - 2 \error_{\wt\calS}(\widehat f) + \left(\frac{1}{\sqrt{2}} + 1.5 \right) \sqrt{\frac{\log(4/\delta)}{m}} + \sqrt{\frac{4}{\delta}\left(\frac{1}{m} +\frac{3\beta}{m+n} \right)}  \,. \label{eq:gd_error}
%     \end{align} 
%     % for some constant $c\le 3.2$.
% \end{theorem}

\subsection{Proof of \thmref{thm:linear}}
We use a standard result from linear algebra, 
namely the Shermann-Morrison formula 
\citep{sherman1950adjustment} for matrix inversion:  

\begin{lemma}[\citet{sherman1950adjustment}] \label{lem:sherman}
    Suppose $\bA \in \Real^{n \times n}$ 
    is an invertible square matrix 
    and $u,v \in \Real^n$ are column vectors. 
    Then $\bA + uv^T$ is invertible iff $1 + v^T \bA u \ne 0$ 
    and in particular
    \begin{align}
        (\bA + u v^T)^{-1} = \bA^{-1}  - \frac{\bA^{-1} uv^T \bA^{-1} }{ 1 + v^T \bA^{-1} u} \,.
    \end{align}   
\end{lemma}
\newcommand\byy[1]{\by_{\left(#1\right)}}
\newcommand\bXX[1]{\bX_{\left(#1\right)}}
\newcommand\ff[1]{\wh f_{\left(#1\right)}}

For a given training set $S \cup \wt S_C$, 
define leave-one-out error 
on mislabeled points in the training data 
as $$\error_{\text{LOO}(\wt S_M) } = \frac{\sum_{(x_i, y_i) \in \wt S_M} \error( f_{(i)}( x_i), y_i)}{ \abs{\wt S_M }} \,, $$
where $f_{(i)} \defeq f(\calA, (S \cup \wt S)_{(i)})$. 
To relate empirical leave-one-out error and population error 
with hypothesis stability condition, 
we use the following lemma:   

\begin{lemma}[\citet{bousquet2002stability}] \label{lem:stability_error}
    For the leave-one-out error, we have
    \begin{align}
        \Expo{ \left( \error_{\calDm}(\wh f) -\error_{\text{LOO}(\wt S_M) } \right)^2 } \le \frac{1}{2m_1}+  \frac{3\beta}{n + m}\,.
    \end{align}   
    % where $ f \defeq f(\calA, S \cup \wt S) $.
\end{lemma}

Proof of the above lemma is similar 
to the proof of Lemma 9 in \citet{bousquet2002stability} 
and can be found in \appref{app:proof_lem_error}. 
% 
% Before presenting the result, we introduce some notation. 
Before presenting the proof of \thmref{thm:linear}, 
we introduce some more notation. 
Let $\bX_{(i)}$ denote the matrix of covariates 
with the $i^{\text{th}}$ point removed. 
Similarly, let $\by_{(i)}$ be the array of responses 
with the $i^{\text{th}}$ point removed. 
Define the corresponding regularized GD solution 
as $\wh w_{(i)} = \left( \bXX{i}^T\bXX{i}+\lambda \boldsymbol{I}\right)^{-1}\bXX{i}^T\byy{i}$. 
Define $\ff{i}(x) \defeq f(x ; \wh w_{(i)}) $.

\begin{proof}[Proof of \thmref{thm:linear}]
    Because squared loss minimization does not imply 0-1 error minimization, 
    we cannot use arguments from \lemref{lem:fit_mislabeled}. 
    This is the main technical difficulty. 
    To compare the 0-1 error at a train point with an unseen point, 
    we use the closed-form expression for $\widehat{w}$ 
    and Shermann-Morrison formula 
    to upper bound training error 
    with leave-one-out cross validation error. 
    
    The proof is divided into three parts: 
    In part one, we show that 0-1 error 
    on mislabeled points in the training set 
    is lower than the error obtained 
    by leave-one-out error at those points. 
    In part two, we relate this leave-one-out error 
    with the population error on mislabeled distribution
    using \codref{cond:hypothesis_stability}.
    While the empirical leave-one-out error is an unbiased estimator 
    of the average population error of leave-one-out classifiers, 
    we need hypothesis stability 
    to control the variance 
    of empirical leave-one-out error. 
    Finally, in part three, we show 
    that the error on the mislabeled training points 
    can be estimated with just the randomly labeled 
    and clean training data (as in proof of \thmref{thm:error_ERM}).  

    \textbf{Part 1 {} {}} First we relate training error with leave-one-out error.        
    For any training point $(x_i, y_i)$ in $\wt S \cup S$, we have 
    \begin{align}
        \error(\wh f(x_i), y_i ) &= \indict{ y_i \cdot x_i^T \wh w < 0 } = \indict{ y_i \cdot x_i^T \left( \bX^T\bX+\lambda \boldsymbol{I}\right)^{-1}\bX^T\by < 0 } \\
        &= \indict{ y_i \cdot x_i^T \underbrace{\left( \bXX{i}^T\bXX{i} + x_i ^T x_i +\lambda \boldsymbol{I}\right)^{-1}}_{\RN{1}} (\bXX{i}^T\byy{i} + y_i \cdot x_i) < 0 } \,.
    \end{align}
    Letting $\bA = \left(\bXX{i}^T\bXX{i} +\lambda \boldsymbol{I}\right)$ 
    and using \lemref{lem:sherman} on term 1, we have 
    \begin{align}
        \error(\wh f(x_i), y_i ) &= \indict{ y_i \cdot x_i^T \left[\bA^{-1} -  \frac{\bA^{-1} x_i x_i^T \bA^{-1}}{ 1 + x_i ^T \bA^{-1} x_i } \right] (\bXX{i}^T\byy{i} + y_i \cdot x_i) < 0 } \\
        &= \indict{ y_i \cdot\left[ \frac{ x_i^T \bA^{-1} ( 1 + x_i ^T \bA^{-1} x_i ) -  x_i^T \bA^{-1} x_i x_i^T \bA^{-1}}{ 1 + x_i ^T \bA ^{-1}x_i } \right] (\bXX{i}^T\byy{i} + y_i \cdot x_i) < 0 } \\
        &= \indict{ y_i \cdot\left[ \frac{ x_i^T \bA^{-1}}{ 1 + x_i ^T \bA ^{-1}x_i } \right] (\bXX{i}^T\byy{i} + y_i \cdot x_i) < 0 } \,.
    \end{align}

    Since $1 + x_i^T \bA^{-1} x_i > 0$, we have 
    \begin{align}
        \error(\wh f(x_i), y_i ) &= \indict{ y_i \cdot x_i^T \bA^{-1} (\bXX{i}^T\byy{i} + y_i \cdot x_i) < 0 } \\
        &= \indict{ x_i^T \bA^{-1} x_i +  y_i \cdot x_i^T \bA^{-1} (\bXX{i}^T\byy{i}) < 0 } \\
        &\le \indict{ y_i \cdot x_i^T \bA^{-1} (\bXX{i}^T\byy{i}) < 0 } = \error(\ff{i}(x_i), y_i ) \,.\label{eq:LOO_error}
    \end{align}

    Using \eqref{eq:LOO_error}, we have 
    \begin{align}
        \error_{\wt \calS_M } (\wh f) \le \error_{\text{LOO} (\wt S_M)} \defeq \frac{\sum_{(x_i, y_i) \in \wt S_M} \error(\ff{i}(x_i), y_i ) }{\abs{\wt \calS_M}}\label{eq:LOO_error_final} \,.
    \end{align}
    \textbf{Part 2 {}{}} We now relate RHS in \eqref{eq:LOO_error_final} 
    with the population error on mislabeled distribution. 
    To do this, we leverage \codref{cond:hypothesis_stability} 
    and \lemref{lem:stability_error}. 
    In particular, we have 

    \begin{align}
        \Expt{\calS \cup \wt \calS_M }{ \left(\error_{\calDm}(\wh f) - \error_{\text{LOO} (\wt S_M)}\right)^2 } \le \frac{1}{2m_1} + \frac{3\beta}{m+n} \,.
    \end{align}

    Using Chebyshev's inequality, with probability at least $1-\delta$, we have 
    \begin{align}
        \error_{\text{LOO} (\wt S_M)} \le  \error_{\calDm}(\wh f)   + \sqrt{\frac{1}{\delta}\left(\frac{1}{2m_1} +\frac{3\beta}{m+n} \right)} \,. \label{eq:final_mislabeled_linear}
    \end{align}
    

    \textbf{Part 3 {}{}} Combining \eqref{eq:final_mislabeled_linear} and \eqref{eq:LOO_error_final}, we have 

    \begin{align}
        \error_{\wt \calS_M } (\wh f) \le \error_{\calDm}(\wh f)   + \sqrt{\frac{1}{\delta}\left(\frac{1}{2m_1} +\frac{3\beta}{m+n} \right)} \,. \label{eq:linear_parallel_lem1}
    \end{align}

    Compare \eqref{eq:linear_parallel_lem1} with \eqref{eq:lemma1_final} 
    in the proof of \lemref{lem:fit_mislabeled}. 
    We obtain a similar relationship 
    between $\error_{\wt \calS_M }$ and $\error_{\calDm}$ 
    but with a polynomial concentration 
    instead of exponential concentration. 
    In addition, since we just use concentration arguments 
    to relate mislabeled error to the errors
    on the clean and unlabeled portions 
    of the randomly labeled data, 
    we can directly use the results 
    in \lemref{lem:mislabeled_error} and \lemref{lem:clear_error}. 
    Therefore, combining results in \lemref{lem:mislabeled_error}, \lemref{lem:clear_error}, and \eqref{eq:linear_parallel_lem1} with union bound, 
    we have with probability at least $1-\delta$
    \begin{align}
        \error_\calD(\widehat f) \le \error_\calS(\widehat f) + 1 - 2 \error_{\wt\calS}(\widehat f) + \left(\sqrt{2}\error_{\wt\calS}(\widehat f) + 1 + \frac{m}{2n} \right) \sqrt{\frac{\log(4/\delta)}{m}} + \sqrt{\frac{4}{\delta}\left(\frac{1}{m} +\frac{3\beta}{m+n} \right)}  \,.
    \end{align}
    

       
\end{proof}

\subsection{Extension to multiclass classification} \label{app:multiclass_linear}
For multiclass problems with squared loss minimization, as standard practice, we consider one-hot encoding for the underlying label, i.e., a class label $c \in [k]$ is treated as $(0, \cdot, 0,1,0, \cdot, 0) \in \Real^k$ (with $c$-th coordinate being 1).  As before, we suppose that the parameters of the linear function 
are obtained via gradient descent on the following $L_2$ regularized problem: 
\begin{align}
    % n in denominator is avoided deliberately
    \calL_S(w; \lambda) \defeq \sum_{i=1}^n\norm{w^Tx_i - y_i}{2}^2 + \lambda \sum_{j=1}^k \norm{w_j}{2}^2 \,, \label{eq:l2_multiclass_MSE_app}   
\end{align}
where $\lambda\ge0$ is a regularization parameter. 
We assume access to a clean dataset 
$S = \{(x_i, y_i)\}_{i=1}^n \sim \calD^n$ 
and randomly labeled dataset 
$\wt S = \{(x_i, y_i)\}_{i=n+1}^{n+m} \sim \wt \calD^m$. 
Let $\bX = [x_1, x_2, \cdots, x_{m+n}]$ 
and $\by = [e_{y_1}, e_{y_2}, \cdots, e_{y_{m+n}}]$. 
Fix a positive learning rate $\eta$ such that 
$\eta \le 1/\left(\norm{\bX^T\bX}{\text{op}} + \lambda^2\right)$ 
and an initialization $w_0 = 0$. 
% \todos{Assumption made for simplicty}. 
Consider the following gradient descent iterates 
to minimize objective \eqref{eq:l2_MSE_app} on $S \cup \wt S$:
\begin{align}
{w_j}^t = {w_j}^{t-1} - \eta \grad_{w_j} \calL_{S \cup \wt S} (w^{t-1}; \lambda) \quad \forall t=1,2,\ldots \text{ and } j=1,2,\ldots,k  \,. \label{eq:GD_multi_iterates_app}
\end{align} 
Then we have $\{ {w_j}^t\}$ for all $j =1,2,\cdots, k$ converge to the limiting solution 
$\wh w_j = \left( \bX^T\bX+\lambda \boldsymbol{I}\right)^{-1}\bX^T\by_j$. Define $\widehat f (x) \defeq f(x ; \wh w) $.  

\begin{theorem}\label{thm:multi_linear}
    Assume that this gradient descent algorithm satisfies \codref{cond:hypothesis_stability}
    with $\beta=\calO(1)$.  
    Then for a multiclass classification problem wth $k$ classes, for any $\delta >0$, with probability at least $1-\delta$, we have:
    \begin{align*}
        \error_\calD(\widehat f) \le \error_\calS(\widehat f) &+ (k-1)\left(1 - \frac{k}{k-1} \error_{\wt\calS}(\widehat f) \right) \\ &+ \left(k + \sqrt{k} + \frac{m}{n\sqrt{k}} \right) \sqrt{\frac{\log(4/\delta)}{2m}} + \sqrt{k(k-1)} \sqrt{\frac{4}{\delta}\left(\frac{1}{m} +\frac{3\beta}{m+n} \right)}  \,. \numberthis \label{eq:gd_multi_error}
    \end{align*} 
    % for some constant $c\le 3.2$.
\end{theorem}
\begin{proof}
    The proof of this theorem is divided into two parts. In the first part, we relate the error on the mislabeled samples with the population error on the mislabeled data. Similar to the proof of \thmref{thm:linear}, we use Shermann-Morrison formula to upper bound training error with leave-one-out error on each $\wh w^j$. Second part of the proof follows entirely from the proof of \thmref{thm:multiclass_ERM}. In essence, the first part derives an equivalent of \eqref{eq:lemma1_final_multi_prev} for GD training with squared loss and then the second part follows from the proof  of \thmref{thm:multiclass_ERM}. 
    
    \textbf{Part-1:} Consider a training point $(x_i,y_i)$ in $\wt S \cup S $. For simplicity, we use $c_i$ to denote the class of $i$-th point and use $y_i$ as the corresponding one-hot embedding. Recall error in multiclass point is given by $\error(\wh f(x_i), y_i ) = \indict{ c_i \not \in \argmax x_i^T \wh w }$. Thus, there exists a $j \ne c_i \in [k]$, such that we have
     \begin{align}
        \error(\wh f(x_i), y_i ) &= \indict{ c_i \not \in \argmax x_i^T \wh w } = \indict{ x_i^T \wh w_{c_i} < x_i^T \wh w_{j}  } \\ &= \indict{ x_i^T \left( \bX^T\bX+\lambda \boldsymbol{I}\right)^{-1}\bX^T\by_{c_i} < x_i^T \left( \bX^T\bX+\lambda \boldsymbol{I}\right)^{-1}\bX^T\by_{j} } \\
        &= \indict{ x_i^T \underbrace{\left( \bXX{i}^T\bXX{i} + x_i ^T x_i +\lambda \boldsymbol{I}\right)^{-1}}_{\RN{1}} \left(\bXX{i}^T{\by_{c_i}}_{(i)} + x_i - \bXX{i}^T{\by_{j}}_{(i)}\right) < 0 } \,.
    \end{align}
    Letting $\bA = \left(\bXX{i}^T\bXX{i} +\lambda \boldsymbol{I}\right)$ 
    and using \lemref{lem:sherman} on term 1, we have 
    \begin{align}
        \error(\wh f(x_i), y_i ) &= \indict{ x_i^T \left[\bA^{-1} -  \frac{\bA^{-1} x_i x_i^T \bA^{-1}}{ 1 + x_i ^T \bA^{-1} x_i } \right]  \left(\bXX{i}^T{\by_{c_i}}_{(i)} + x_i - \bXX{i}^T{\by_{j}}_{(i)}\right) < 0 } \\
        &= \indict{ \left[ \frac{ x_i^T \bA^{-1} ( 1 + x_i ^T \bA^{-1} x_i ) -  x_i^T \bA^{-1} x_i x_i^T \bA^{-1}}{ 1 + x_i ^T \bA ^{-1}x_i } \right]  \left(\bXX{i}^T{\by_{c_i}}_{(i)} + x_i - \bXX{i}^T{\by_{j}}_{(i)}\right) < 0 } \\
        &= \indict{ \left[ \frac{ x_i^T \bA^{-1}}{ 1 + x_i ^T \bA ^{-1}x_i } \right]  \left(\bXX{i}^T{\by_{c_i}}_{(i)} + x_i - \bXX{i}^T{\by_{j}}_{(i)}\right) < 0} \,.
    \end{align}
    Since $1 + x_i^T \bA^{-1} x_i > 0$, we have 
    \begin{align}
        \error(\wh f(x_i), y_i ) &= \indict{ x_i^T \bA^{-1}  \left(\bXX{i}^T{\by_{c_i}}_{(i)} + x_i - \bXX{i}^T{\by_{j}}_{(i)}\right) < 0 } \\
        &= \indict{ x_i^T \bA^{-1} x_i +  x_i^T \bA^{-1}  \bXX{i}^T{\by_{c_i}}_{(i)}  - x_i^T\bA^{-1}  \bXX{i}^T{\by_{j}}_{(i)} < 0 } \\
        &\le \indict{  x_i^T \bA^{-1}  \bXX{i}^T{\by_{c_i}}_{(i)}  - x_i^T\bA^{-1}  \bXX{i}^T{\by_{j}}_{(i)} < 0  } = \error(\ff{i}(x_i), y_i ) \,.\label{eq:LOO_error_multi}
    \end{align}
    Using \eqref{eq:LOO_error_multi}, we have 
    \begin{align}
        \error_{\wt \calS_M } (\wh f) \le \error_{\text{LOO} (\wt S_M)} \defeq \frac{\sum_{(x_i, y_i) \in \wt S_M} \error(\ff{i}(x_i), y_i ) }{\abs{\wt \calS_M}}\label{eq:LOO_error_multi_final} \,.
    \end{align}
    
    We now relate RHS in \eqref{eq:LOO_error_final} 
    with the population error on mislabeled distribution. 
    Similar as before, to do this, we leverage \codref{cond:hypothesis_stability} 
    and \lemref{lem:stability_error}. Using  \eqref{eq:final_mislabeled_linear} and \eqref{eq:LOO_error_multi_final}, we have 
    \begin{align}
        \error_{\wt \calS_M } (\wh f) \le \error_{\calDm}(\wh f)   + \sqrt{\frac{1}{\delta}\left(\frac{1}{2m_1} +\frac{3\beta}{m+n} \right)} \,. \label{eq:linear_multi_parallel_lem1}
    \end{align}
    
    We have now derived a parallel to \eqref{eq:lemma1_final_multi_prev}. Using the same arguments in the proof of \lemref{lem:fit_mislabeled_multi}, we have 
    \begin{align}
      \error_{\calD}(\wh f) \le  (k-1) \left( 1- \error_{ \wt \calS_M}(\wh f) \right)  + (k-1)\sqrt{\frac{k}{\delta(k-1)}\left(\frac{1}{2m_1} +\frac{3\beta}{m+n} \right)}  \,. \label{eq:lemma1_linear_final_multi}
    \end{align}
    
    \textbf{Part-2:} We now combine the results in \lemref{lem:mislabeled_error_multi} and \lemref{lem:clear_error_multi} to obtain the final inequality in terms of quantities that can be computed from just the randomly labeled and clean data. Similar to the binary case, we obtained a polynomial concentration instead of exponential concentration. Combining \eqref{eq:lemma1_linear_final_multi} with \lemref{lem:mislabeled_error_multi} and \lemref{lem:clear_error_multi}, we have with probability at least $1-\delta$
    \begin{align*}
        \error_\calD(\widehat f) \le \error_\calS(\widehat f) &+ (k-1)\left(1 - \frac{k}{k-1} \error_{\wt\calS}(\widehat f) \right) \\ &+ \left(k + \sqrt{k} + \frac{m}{n\sqrt{k}} \right) \sqrt{\frac{\log(4/\delta)}{2m}} + \sqrt{k(k-1)} \sqrt{\frac{4}{\delta}\left(\frac{1}{m} +\frac{3\beta}{m+n} \right)}  \,. \numberthis \label{eq:gd_multi_error_proof}
    \end{align*} 
\end{proof}

\subsection{Discussion on \codref{cond:hypothesis_stability}} \label{app:discuss_cond1}
The quantity in LHS of \codref{cond:hypothesis_stability} 
measures how much the function learned by the algorithm 
(in terms of error on unseen point) will change 
when one point in the training set is removed. 
% Discussion on exponential concentration and stronger condition. 
% Notice that hypothesis stability implies error stability, i.e., \codref{cond:error_stability} \citep{bousquet2002stability}.  
% In summary, while error stability allowed us 
% to relate the average population error 
% of the leave-one-out classifiers 
% with the population error of the original classifier, 
We need hypothesis stability condition 
to control the variance of the empirical leave-one-out error to show concentration of average leave-one-error with the population error. 

Additionally, we note that while the dominating term in the RHS of \thmref{thm:linear} matches with the dominating term in ERM bound in \thmref{thm:error_ERM}, there is a polynomial concentration term 
(dependence on $1/\delta$ instead of $\log(\sqrt{1/\delta})$) 
in \thmref{thm:linear}. 
Since with hypothesis stability, 
we just bound the variance, 
the polynomial concentration is due 
to the use of Chebyshev's inequality 
instead of an exponential tail inequality
(as in \lemref{lem:fit_mislabeled}).
Recent works have highlighted that 
a slightly stronger condition than hypothesis stability 
can be used to obtain an exponential concentration 
for leave-one-out error \citep{abou2019exponential},
but we leave this for future work for now. 
% We leave 
% However, the constants 

% we also want to highlight  

\subsection{Formal statement and proof of \propref{prop:early_stop}} \label{app:formal_early_stop}

Before formally presenting the result, 
we will introduce some notation.  
By $\calL_{S}(w)$, we denote 
the objective in \eqref{eq:l2_MSE_app} with $\lambda=0$. 
Assume Singular Value Decomposition (SVD) of $\bX$
as $\sqrt{n} \bU \bS^{1/2} \bV^T$. 
Hence $\bX^T \bX = \bV \bS \bV^T$.
Consider the GD iterates defined in \eqref{eq:GD_iterates_app}. 
% 
We now derive closed form expression 
for the $t^\text{th}$ iterate of gradient descent:  
% 
\begin{align}
    w_t = w_{t-1} + \eta \cdot \bX^T (\by - \bX w_{t-1}) = (\bI - \eta \bV \bS \bV^T )w_{k-1} + \eta \bX^T \by \,.
\end{align}
Rotating by $\bV^T$, we get 
\begin{align}
    \wt w_t = (\bI - \eta\bS )\wt w_{k-1} + \eta \wt \by \label{eq:GD_recur},
\end{align}
where $\wt w_t = \bV^T w_t $ and $\wt \by = \bV^T \bX^T \by$. 
Assuming the initial point $w_0 = 0$ 
and applying the recursion in \eqref{eq:GD_recur}, we get
\begin{align}
    \wt w_t = \bS ^{-1} ( \bI - (\bI - \eta \bS)^k ) \wt \by \,, 
\end{align} 
Projecting solution back to the original space, we have 
\begin{align}
     w_t = \bV \bS ^{-1} ( \bI - (\bI - \eta \bS)^k ) \bV^T \bX^T \by \,. 
\end{align} 
% We will work with this GD solution at any iterate $t$ in the next proposition. 
Define $f_t(x) \defeq f(x;w_t)$ 
as the solution at the $t^{\text{th}}$ iterate. 
Let $\wt w_{\lambda} = \argmin_{w} \calL_\calS (w;\lambda) = (\bX^T \bX + \lambda \bI)^{-1} \bX^T \by = \bV (\bS + \lambda \bI )^{-1} \bV^T \bX^T \by $. 
% ) \,,$ for all $t=1,2,\ldots\,.$ 
and define $\wt f_\lambda(x) \defeq f(x;\wt w_\lambda)$ as the regularized solution. 
Assume $\kappa$ be the condition number 
of the population covariance matrix 
and let $s_\text{min}$ be the minimum positive 
singular value of the empirical covariance matrix. 
Our proof idea is inspired from recent work 
on relating gradient flow solution 
and regularized solution 
for regression problems \citep{ali2018continuous}. 
We will use the following lemma in the proof: 
\begin{lemma} \label{lem:ineq_soln}
    For all $x \in [0,1]$ and for all $ k \in \mathbb{N}$, 
    we have (a) $ \frac{kx}{1+kx} \le 1- (1-x)^k$ 
    and (b) $ 1- (1-x)^k \le 2 \cdot \frac{kx}{kx+1} $.
    %  where $g(c)$ is a constant dependent on $c$. For $c = 1$, $g(c) = 2.0$.   
\end{lemma}
\begin{proof}
    % [Proof of \lemref{lem:ineq_soln}]
    % Part (a) is easy. 
    Using $ (1-x)^k \le \frac{1}{1+kx}$, we have part (a). 
    For part (b), we numerically maximize 
    $\frac{ (1+kx ) (1 - (1-x)^k) }{kx}$ 
    for all $k\ge 1$ and for all $x \in [0, 1]$.  
\end{proof}

% 
% Next, 

\begin{prop}[Formal statement of \propref{prop:early_stop}] \label{prop:formal_early_stop}
Let $\lambda = \frac{1}{t\eta}$. 
For a training point $x$, we have 
\begin{align*}
    \Expt{x \sim \calS}{(f_t(x) - \wt f_\lambda(x))^2} &\le c(t,\eta) \cdot \Expt{x \sim \calS}{f_t(x)^2} \,, %\label{eq:early_stop}
\end{align*}
where $c(t, \eta) \defeq \min( 0.25, \frac{1}{s_\text{min}^2 t^2 \eta^2})$. 
Similarly for a test point, we have 
\begin{align*}
    \Expt{x \sim \calD_\calX}{(f_t(x) - \wt f_\lambda(x))^2} &\le \kappa \cdot c(t,\eta) \cdot \Expt{x \sim \calD_\calX}{f_t(x)^2} \,. %\label{eq:early_stop}
\end{align*}
\end{prop} 

\begin{proof}
    %%%%%%%%%%%%% 
    We want to analyze the expected squared difference output 
    of regularized linear regression 
    with regularization constant $\lambda = \frac{1}{\eta t}$ 
    and the gradient descent solution at the $t^\text{th}$ iterate. 
    We separately expand the algebraic expression 
    for squared difference at a training point and a test point. 
    % We start by considering the difference  
    Then the main step is to show that 
    $\left[ \bS ^{-1} ( \bI - (\bI - \eta \bS)^k )  - (\bS + \lambda \bI )^{-1}\right] \preceq c(\eta, t) \cdot \bS ^{-1} ( \bI - (\bI - \eta \bS)^k ) $.

    %%%%%%%%%%%%%
    
   \textbf{Part 1 {} {}} 
    First, we will analyze the squared difference 
    of the output at a training point 
    (for simplicity, we refer to $S \cup \wt S$ as $S$), i.e., 
    \begin{align}
        \Expt{ x \sim \calS }{\left(f_t(x) - \wt f_\lambda (x)\right)^2} &= \norm{\bX w_t - \bX \wt w_\lambda}{2}^2\\ &=   \norm{\bX \bV \bS ^{-1} ( \bI - (\bI - \eta \bS)^t ) \bV^T \bX^T \by - \bX \bV (\bS + \lambda \bI )^{-1} \bV^T \bX^T \by }{2}^2 \\
        &= \norm{\bX \bV \left(\bS ^{-1} ( \bI - (\bI - \eta \bS)^t ) - (\bS + \lambda \bI )^{-1} \right) \bV^T \bX^T \by  }{2} \\
        &=  \by^T \bV \bX \left( \underbrace{\bS ^{-1} ( \bI - (\bI - \eta \bS)^t ) - (\bS + \lambda \bI )^{-1}}_{\RN{1}} \right)^2 \bS \bV^T \bX^T \by \label{eq:train_GD_rel} \,.
        %  (\bX \bV \bS ^{-1} ( \bI - (\bI - \eta \bS)^k ) \bV^T \bX^T \by)^T \bX \bV \bS ^{-1} ( \bI - (\bI - \eta \bS)^k ) \bV^T \bX^T \by
    \end{align}
    We now separately consider term 1. 
    Substituting $\lambda = \frac{1}{t \eta}$, 
    we get
    \begin{align}
        \bS ^{-1} ( \bI - (\bI - \eta \bS)^t ) - (\bS + \lambda \bI )^{-1} &= \bS^{-1} \left( ( \bI - (\bI - \eta \bS)^t ) - (\bI + \bS^{-1} \lambda )^{-1}\right) \\
        &= \underbrace{\bS^{-1} \left( ( \bI - (\bI - \eta \bS)^t ) - (\bI + ( \bS t \eta)^{-1}  )^{-1}\right)}_{\bA} \,.
    \end{align}

    We now separately bound the diagonal entries in matrix $\bA$. 
    With $s_i$, we denote $i^{\text{th}}$ diagonal entry of $\bS$.
    Note that since $ \eta\le 1/\norm{S}{\text{op}}$, 
    for all $i$, $\eta s_i  \le 1$.  
    Consider $i^{\text{th}}$ diagonal term (which is non-zero) 
    of the diagonal matrix $\bA$, we have 
    \begin{align}
        \bA_{ii} = \frac{1}{s_i} \left(  1 - (1 - s_i \eta)^t - \frac{t \eta s_i}{1 + t \eta s_i } \right) &=  \frac{1 - (1 - s_i \eta)^t}{s_i} \left( \underbrace{ 1 - \frac{t \eta s_i}{(1 + t \eta s_i)(1 - (1 - s_i \eta)^t)}}_{\RN{2}} \right) \\ 
         &\le \frac{1}{2}\left[ \frac{1 - (1 - s_i \eta)^t}{ s_i} \right] \tag*{(Using \lemref{lem:ineq_soln} (b))} \,.
    \end{align} 
    Additionally, we can also show the following upper bound on term 2: 
    \begin{align}
         1 - \frac{t \eta s_i}{(1 + t \eta s_i)(1 - (1 - s_i \eta)^t)} &= \frac{(1 + t \eta s_i)(1 - (1 - s_i \eta)^t) - t \eta s_i }{(1 + t \eta s_i)(1 - (1 - s_i \eta)^t)} \\
         & \le  \frac{ 1 -  (1 - s_i \eta)^t - t \eta s_i (1 - s_i \eta)^t}{(1 + t \eta s_i)(1 - (1 - s_i \eta)^t)} \\
         & \le \frac{1}{t\eta s_i} \,. \tag{Using \lemref{lem:ineq_soln} (a)}
        %  &\le \frac{1}{2}\left[ \frac{1 - (1 - s_i \eta)^t}{ s_i} \right] \tag*{(Using \lemref{lem:ineq_soln})} \,.
    \end{align} 

    Combining both the upper bounds 
    on each diagonal entry $\bA_{ii}$, we have 
    \begin{align}
    \bA \preceq c_1(\eta, t) \cdot \bS^{-1} ( \bI - (\bI - \eta \bS)^t ) \,, \label{eq:upperbound_diagonal}
    \end{align}
    where $c_1(\eta, t ) = \min(0.5, \frac{1}{t s_i \eta })$. Plugging this into \eqref{eq:train_GD_rel}, we have 
    \begin{align}
        \Expt{ x \sim \calS }{\left(f_t(x) - \wt f_\lambda (x)\right)^2} &\le c(\eta, t) \cdot \by^T \bV \bX  \left( \bS^{-1} ( \bI - (\bI - \eta \bS)^t ) \right)^2 \bS \bV^T \bX^T \by \\
        &=   c(\eta, t) \cdot \by^T \bV \bX  \left( \bS^{-1} ( \bI - (\bI - \eta \bS)^t ) \right) \bS \left( \bS^{-1} ( \bI - (\bI - \eta \bS)^t ) \right) \bV^T \bX^T \by \\
        & =  c(\eta, t) \cdot \norm{\bX w_t}{2}^2 \\
        &= c(\eta, t) \cdot  \Expt{ x \sim \calS }{\left(f_t(x) \right)^2} \,,
    \end{align}
    where $c(\eta, t ) = \min(0.25, \frac{1}{t^2 s^2_i \eta^2 })$.

    \textbf{Part 2 {} {}} With $\bSigma$, 
    we denote the underlying true covariance matrix. 
    We now consider the squared difference of output at an unseen point: 
    \begin{align}
        \Expt{ x \sim \calD_{\calX} }{\left(f_t(x) - \wt f_\lambda (x)\right)^2} &= \Expt{x \sim \calD_{\calX}}{\norm{x^T w_t - x^T \wt w_\lambda}{2}} \\
        &=   \norm{x^T \bV \bS ^{-1} ( \bI - (\bI - \eta \bS)^t ) \bV^T \bX^T \by - x^T \bV (\bS + \lambda \bI )^{-1} \bV^T \bX^T \by }{2} \\
        &= \norm{x^T \bV \left(\bS ^{-1} ( \bI - (\bI - \eta \bS)^t ) - (\bS + \lambda \bI )^{-1} \right) \bV^T \bX^T \by  }{2} \\
        &= \by^T \bV \bX \left( \bS ^{-1} ( \bI - (\bI - \eta \bS)^t ) - (\bS + \lambda \bI )^{-1} \right) \bV^T \bSigma \bV \\ &\qquad \qquad \qquad \qquad \qquad \left( (\bI - (\bI - \eta \bS)^t ) - (\bS + \lambda \bI )^{-1} \right) \bV^T \bX^T \by \\
        &\le \sigma_{\text{max}} \cdot \by^T \bV \bX \left( \underbrace{\bS ^{-1} ( \bI - (\bI - \eta \bS)^t ) - (\bS + \lambda \bI )^{-1}}_{\RN{1}} \right)^2 \bV^T \bX^T \by \,, \label{eq:test_GD_rel}
        %  (\bX \bV \bS ^{-1} ( \bI - (\bI - \eta \bS)^k ) \bV^T \bX^T \by)^T \bX \bV \bS ^{-1} ( \bI - (\bI - \eta \bS)^k ) \bV^T \bX^T \by
    \end{align}
    where $\sigma_{\text{max}}$ is the maximum eigenvalue 
    of the underlying covariance matrix $\bSigma$. 
    Using the upper bound on term 1 in \eqref{eq:upperbound_diagonal}, 
    we have 
    \begin{align}
        \Expt{ x \sim \calD_{\calX} }{\left(f_t(x) - \wt f_\lambda (x)\right)^2} &\le \sigma_{\text{max}} \cdot c(\eta, t) \cdot \by^T \bV \bX  \left( \bS^{-1} ( \bI - (\bI - \eta \bS)^t ) \right)^2 \bV^T \bX^T \by \\
        &=   \kappa \cdot c(\eta, t) \cdot \sigma_{\text{min}}\cdot \norm{\bV \left( \bS^{-1} ( \bI - (\bI - \eta \bS)^t ) \right) \bV^T \bX^T \by}{2}^2 \\
        &\le \kappa \cdot c(\eta, t) \cdot \left[ \bV \left( \bS^{-1} ( \bI - (\bI - \eta \bS)^t ) \right) \bV^T \bX^T \right]^T \bSigma \\
        &\qquad \qquad \qquad \qquad \qquad \left[ \bV \left( \bS^{-1} ( \bI - (\bI - \eta \bS)^t ) \right) \bV^T \bX^T \right] \by \\
        & = \kappa \cdot c(\eta, t) \cdot \Expt{x \sim \calD_{\calX}}{\norm{x^T w_t}{2}} \,.
    \end{align}
% 
% 
    % Since $ \eta\le 1/\norm{S}{\text{op}}$, invoking \lemref{lem:ineq_soln} to upper bound term 1 with
\end{proof}

\subsection{Extension to deep learning} \label{appsubsec:ext_DL}
Under \asmpref{appsubsec:justifying_assumption1}, we present the formal result parallel to \thmref{thm:multiclass_ERM}. 
\begin{theorem} \label{thm:multiclass_ERM_algoA}
    Consider a multiclass classification problem 
    with $k$ classes. Under \asmpref{asmp:deep_models}, 
    for any $\delta >0$, with probability at least $1-\delta$,
    we have
    \vspace{-10pt}
    \begin{align*}
        \error_\calD(\widehat f)  \le \error_\calS(\widehat f) + (k-1) \left(1 - \tfrac{k}{k-1} \error_{\wt\calS}(\widehat f)\right) + c\sqrt{\frac{\log(\frac{4}{\delta})}{2m}} \,,\numberthis \label{eq:multiclass_ERM_deep}
    % \vspace{-20pt}
    \end{align*}
    for some constant $c \le ((c+1) k+\sqrt{k} + \frac{m}{n\sqrt{k}})$.
\end{theorem}

The proof follows exactly as in step (i) to (iii) in \thmref{thm:multiclass_ERM}.  

\subsection{Justifying~\asmpref{asmp:deep_models}} \label{appsubsec:justifying_assumption1}

Motivated by the analysis on linear models, we now discuss alternate (and weaker) conditions that imply \asmpref{asmp:deep_models}. 
We need hypothesis stability (\codref{cond:hypothesis_stability}) and the following assumption relating training error and leave-one-error: 

\begin{assumption} \label{asmp:loo_error}
Let $\wh f$ be a model obtained by training with algorithm $\calA$ on a mixture of clean $S$ and randomly labeled data $\wt S$. Then we assume we have 
\begin{align*}
    \error_{\wt \calS_M} (\wh f) \le  \error_{\text{LOO} (\wt S_M)} \,, 
\end{align*}
for all $(x_i, y_i) \in  \wt S_M$ where $\wh f_{(i)} \defeq f(\calA, S \cup {{}\wt S_M}_{(i)})$ and  $\error_{\text{LOO} (\wt S_M)} \defeq  \frac{\sum_{(x_i, y_i) \in \wt S_M} \error(\ff{i}(x_i), y_i ) }{\abs{\wt \calS_M}}$.  
\end{assumption}

% we assume this to extend our result (parallel to \thmref{thm:multi_linear}) for deep models. 
Intuitively, this assumption states that the error on a (mislabeled) datum $(x,y)$ included in the training set is less than the error on that datum $(x,y)$ obtained by a model trained on the training set $S - \{(x,y)\}$. We proved this for linear models trained with GD in the proof of \thmref{thm:multi_linear}. 
% 
\codref{cond:hypothesis_stability} with $\beta = \calO(1)$ and \asmpref{asmp:loo_error} together with \lemref{lem:stability_error} implies \asmpref{asmp:deep_models} with a polynomial residual term (instead of logarithmic in $1/\delta$): 
\begin{align}
     \error_{\calS_M} (\wh f) \le  \error_{\calDm}(\wh f)   + \sqrt{\frac{1}{\delta}\left(\frac{1}{m} +\frac{3\beta}{m+n} \right)} \,.
\end{align}
% Note that this  

\newpage 
\section{Additional experiments and details}\label{app:exp}
\newcommand\tab[1][1cm]{\hspace*{#1}}

\subsection{Datasets} \label{sec:app_dataset}

\textbf{Toy Dataset {} {}} Assume fixed constants $\mu$ and $\sigma$. For a given label $y$, we simulate features $x$ in our toy classification setup as follows: 
\begin{align*}
    x \defeq \texttt{concat} \left[ x_1, x_2\right] \quad \text{where} \quad  x_1 \sim  \calN( y \cdot \mu, \sigma^2 I_{d \times d}) \ \  \text{and} \ \  x_1 \sim  \calN( 0, \sigma^2 I_{d \times d}) \,.
\end{align*}  
% where $y$ is the true label and $x$ is the corresponding feature vector. 
In experiements throughout the paper, we fix dimention $d=100$, $\mu = 1.0 $, and $\sigma = \sqrt{d}$. Intuitively, $x_1$ carries the information about the underlying label and $x_2$ is additional noise independent of the underlying label. 

\textbf{CV datasets {} {}} We use MNIST~\citep{lecun1998mnist} and CIFAR10~\cite{krizhevsky2009learning}. 
% For binary tasks, 
We produce a binary variant from the multiclass classification problem by mapping classes $\{0,1,2,3,4\}$ to label $1$ and $\{ 5,6,7,8,9\}$ to label $-1$. For CIFAR dataset, we also use the standard data augementation of random crop and horizontal flip. PyTorch code is as follows: 

\texttt{(transforms.RandomCrop(32, padding=4),\\
\tab transforms.RandomHorizontalFlip())}

\textbf{NLP dataset {} {}} We use IMDb Sentiment analysis~\citep{maas2011learning} corpus.  

\subsection{Architecture Details} 

All experiments were run on NVIDIA GeForce RTX 2080 Ti GPUs. We used PyTorch~\citep{NEURIPS2019a9015} and Keras with Tensorflow~\citep{abadi2016tensorflow} backend for experiments. 
% , ELMo embeddings~\citep{Peters:2018}, and Hugging Face Transformers~\citep{wolf-etal-2020-transformers}. 

\textbf{Linear model {} {}} For the toy dataset, we simulate a linear model with scalar output and the same number of parameters as the number of dimensions.   

\textbf{Wide nets {} {}} To simulate the NTK regime, we experiment with $2-$layered wide nets. The PyTorch code for 2-layer wide MLP is as follows: 


\texttt{ nn.Sequential( \\
\tab     nn.Flatten(),\\
\tab    nn.Linear(input\_dims, 200000, bias=True),\\
\tab    nn.ReLU(),\\
\tab    nn.Linear(200000, 1, bias=True)\\
\tab     )}


We experiment both (i) with the second layer fixed at random initialization; (ii)  and updating both layers' weights.     

\textbf{Deep nets for CV tasks {} {}} We consider a 4-layered MLP. The PyTorch code for 4-layer MLP is as follows: 

\texttt{ nn.Sequential(nn.Flatten(), \\
\tab        nn.Linear(input\_dim, 5000, bias=True),\\
\tab        nn.ReLU(),\\
\tab        nn.Linear(5000, 5000, bias=True),\\
\tab        nn.ReLU(),\\
\tab        nn.Linear(5000, 5000, bias=True),\\
\tab        nn.ReLU(),\\
% \tab        nn.Linear(5000, 5000, bias=True),\\
% \tab        nn.ReLU(),\\
\tab        nn.Linear(1024, num\_label, bias=True)\\
\tab        )}

For MNIST, we use $1000$ nodes instead of $5000$ nodes in the hidden layer. 
% 
We also experiment with convolutional nets. In particular, we use ResNet18 \citep{he2016deep}. Implementation adapted from:  \url{https://github.com/kuangliu/pytorch-cifar.git}. 

\textbf{Deep nets for NLP {} {}} We use a simple LSTM model with embeddings intialized with ELMo embeddings~\citep{Peters:2018}. Code adapted from: \url{https://github.com/kamujun/elmo_experiments/blob/master/elmo_experiment/notebooks/elmo_text_classification_on_imdb.ipynb} 

We also evaluate our bounds with a BERT model. In particular, we fine-tune an off-the-shelf uncased BERT model~\citep{devlin2018bert}. Code adapted from Hugging Face Transformers~\citep{wolf-etal-2020-transformers}: \url{https://huggingface.co/transformers/v3.1.0/custom_datasets.html}. 


\subsection{Additonal experiments}

\textbf{Results with SGD on underparameterized linear models {} {}} 

\begin{figure*}[h]
    \centering 
    % \vspace{-15pt}
    % \includegraphics[width=0.9\linewidth]{example-image-a}
    \includegraphics[width=0.3\linewidth]{figures/lowdim-Gaussian-SGD.pdf}
    % \includegraphics[width=0.9\linewidth]{figures/{CIFAR10_rn=0.1_lr=0.2_wd=0.005}.png}
    \vspace{-5pt}
    \caption{ 
    % Predicted lower bound 
    % on different
    We plot the accuracy and corresponding bound 
    (RHS in \eqref{eq:erm}) at $\delta = 0.1$
    for toy binary classification task. 
    Results aggregated over $3$ seeds. 
    % i.e., $1-\error$ where $\error$ is the term in the RHS of \eqref{eq:erm}
    Accuracy vs fraction of unlabeled data (w.r.t clean data) 
    in the toy setup with a linear model trained with SGD. Results parallel to \figref{fig:error_binary}(a) with SGD.  }
    \label{fig:error_binary_linear}
    \vspace{-5pt}
\end{figure*}

\textbf{Results with wide nets on binary MNIST {} {}}

\begin{figure*}[h]
    \centering 
    % \vspace{-15pt}
    % \includegraphics[width=0.9\linewidth]{example-image-a}
    \subfigure[GD with MSE loss]{\includegraphics[width=0.3\linewidth]{figures/MNIST-GD_MSE.pdf}} \hfil
    \subfigure[SGD with CE loss]{\includegraphics[width=0.3\linewidth]{figures/MNIST-SGD_CE.pdf}}
    \subfigure[SGD with MSE loss]{\includegraphics[width=0.3\linewidth]{figures/MNIST-SGD_MSE-first-layer.pdf}}
    % \includegraphics[width=0.9\linewidth]{figures/{CIFAR10_rn=0.1_lr=0.2_wd=0.005}.png}
    \vspace{-5pt}
    \caption{ 
    % Predicted lower bound 
    % on different
    We plot the accuracy and corresponding bound 
    (RHS in \eqref{eq:erm}) at $\delta = 0.1$ 
    for binary MNIST classification. 
    Results aggregated over $3$ seeds. 
    % i.e., $1-\error$ where $\error$ is the term in the RHS of \eqref{eq:erm}
    Accuracy vs fraction of unlabeled data 
    for a 2-layer wide network on binary MNIST with both the layers training in (a,b) and only first layer training in (c). 
    Results parallel to \figref{fig:error_binary}(b) .  }
    \label{fig:error_binary_MNIST}
    \vspace{-5pt}
\end{figure*}

% \begin{figure*}[h]
%     \centering 
%     % \vspace{-15pt}
%     % \includegraphics[width=0.9\linewidth]{example-image-a}
%     \subfigure[GD with MSE loss]{\includegraphics[width=0.3\linewidth]{figures/MNIST.pdf}} \hfil
    
%     \subfigure[SGD with CE loss]{\includegraphics[width=0.3\linewidth]{figures/MNIST.pdf}}
%     % \includegraphics[width=0.9\linewidth]{figures/{CIFAR10_rn=0.1_lr=0.2_wd=0.005}.png}
%     \vspace{-5pt}
%     \caption{ 
%     % Predicted lower bound 
%     % on different
%     We plot the accuracy and corresponding bound 
%     (RHS in \eqref{eq:erm}) at $\delta = 0.1$
%     for binary MNIST classification. 
%     Results aggregated over $3$ seeds. 
%     % i.e., $1-\error$ where $\error$ is the term in the RHS of \eqref{eq:erm}
%     Accuracy vs fraction of unlabeled data 
%     for a 2-layer wide network on binary MNIST with just the first layer training. 
%     Results parallel to \figref{fig:error_binary}(b) with only the first layer training.  }
%     \label{fig:error_binary_MNIST}
%     \vspace{-5pt}
% \end{figure*}

\textbf{Results on CIFAR 10 and MNIST {} {}} 
% 
We plot epoch wise error curve for results in \tabref{table:multiclass}(\figref{fig:error_epoch_CIFAR10} and \figref{fig:error_epoch_MNIST}). We observe the same trend as in \figref{fig:error_CIFAR10}. Additionally, we plot an \emph{oracle bound} obtained by tracking the error on mislabeled data which nevertheless were predicted as true label. To obtain an exact emprical value of the oracle bound, we need underlying true labels for the randomly labeled data. 
% Note that our bound in \thmref{thm:multiclass_ERM}, lower bounds the accuracy as predicted by the oracle bound. 
While with just access to extra unlabeled data we cannot calculate oracle bound, we note that the oracle bound is very tight and never violated in practice underscoring an importamt aspect of generalization in multiclass problems. This highlight that even a stronger conjecture may hold in multiclass classification, i.e., error on mislabeled data (where nevertheless true label was predicted) lower bounds the population error on the distribution of mislabeled data and hence, the error on (a specific) mislabeled portion predicts the population accuracy on clean data. 
% 
On the other hand, the dominating term of in \thmref{thm:multiclass_ERM} is loose when compared with the oracle bound. The main reason, we believe is the pessimistic upper bound in \eqref{eq:lemma1_final_multi_prev} in the proof of \lemref{lem:fit_mislabeled_multi}. We leave an investigation on this gap for future. 
% of fit 

% However, oracle bound highlights two . One,  



\begin{figure}[h]
    \centering 
    % \vspace{-15pt}
    % \includegraphics[width=0.9\linewidth]{example-image-a}
    \subfigure[MLP]{\includegraphics[width=0.3\linewidth]{figures/CIFAR10-FNN.pdf}} \hfil
    \subfigure[ResNet]{\includegraphics[width=0.3\linewidth]{figures/CIFAR10-Resnet.pdf}}
    % \includegraphics[width=0.9\linewidth]{figures/{CIFAR10_rn=0.1_lr=0.2_wd=0.005}.png}
    % \vspace{-10pt}
    \caption{ Per epoch curves for CIFAR10 corresponding results in \tabref{table:multiclass}. As before, we just plot the dominating term in the RHS of \eqref{eq:multiclass_ERM} as predicted bound. Additionally, we also plot the predicted lower bound by the error on mislabeled data which nevertheless were predicted as true label. We refer to this as ``Oracle bound''. See text for more details. 
    % 
    % except for the stopping point. 
    % The bound predicted by RATT (RHS in \eqref{eq:multiclass_ERM}) is vacuous. 
    }\label{fig:error_epoch_CIFAR10}
    % \vspace{-15pt}
\end{figure}


\begin{figure}[h]
    \centering 
    % \vspace{-15pt}
    % \includegraphics[width=0.9\linewidth]{example-image-a}
    \subfigure[MLP]{\includegraphics[width=0.3\linewidth]{figures/MNIST-FNN.pdf}} \hfil
    \subfigure[ResNet]{\includegraphics[width=0.3\linewidth]{figures/MNIST-Resnet.pdf}}
    % \includegraphics[width=0.9\linewidth]{figures/{CIFAR10_rn=0.1_lr=0.2_wd=0.005}.png}
    % \vspace{-10pt}
    \caption{ Per epoch curves for MNIST corresponding results in \tabref{table:multiclass}. As before, we just plot the dominating term in the RHS of \eqref{eq:multiclass_ERM} as predicted bound. Additionally, we also plot the predicted lower bound by the error on mislabeled data which nevertheless were predicted as true label. We refer to this as ``Oracle bound''. See text for more details. 
    % 
    % except for the stopping point. 
    % The bound predicted by RATT (RHS in \eqref{eq:multiclass_ERM}) is vacuous. 
    }\label{fig:error_epoch_MNIST}
    % \vspace{-15pt}
\end{figure}

\textbf{Results on CIFAR 100 {} {}} 
% 
On CIFAR100, our bound in \eqref{eq:multiclass_ERM} yields vacous bounds. However, the oracle bound as explained above yields tight guarantees in the initial phase of the learning (i.e., when learning rate is less than $0.1$) (\figref{fig:error_CIFAR100}).  

\begin{figure}[h]
    \centering 
    % \vspace{-15pt}
    % \includegraphics[width=0.9\linewidth]{example-image-a}
    \includegraphics[width=0.3\linewidth]{figures/CIFAR100-Resnet.pdf}
    % \includegraphics[width=0.9\linewidth]{figures/{CIFAR10_rn=0.1_lr=0.2_wd=0.005}.png}
    % \vspace{-10pt}
    \caption{ Predicted lower bound by the error on mislabeled data which nevertheless were predicted as true label with ResNet18 on CIFAR100. We refer to this as ``Oracle bound''. See text for more details. 
    % 
    % except for the stopping point. 
    The bound predicted by RATT (RHS in \eqref{eq:multiclass_ERM}) is vacuous. 
    }\label{fig:error_CIFAR100}
    % \vspace{-15pt}
\end{figure}


% \paragraph{Experiments on CIFAR100} 


% \subsection{Model Selection using RATT}


\subsection{Hyperparameter Details}


\textbf{\figref{fig:error_CIFAR10} {} {}} We use clean training dataset of size $40,000$. We fix the amount of unlabeled data at $20\%$ of the clean size, i.e. we include additional $8,000$ points with randomly assigned labels. We use test set of $10,000$ points. For both MLP and ResNet, we use SGD with an initial learning rate of $0.1$ and momentum $0.9$. We fix the weight decay parameter at $5\times 10^{-4}$. After $100$ epochs, we decay the learning rate to $0.01$. We use SGD batch size of $100$. 

\textbf{\figref{fig:error_binary} (a) {} {}} We obtain a toy dataset according to the process described in \secref{sec:app_dataset}. We fix $d=100$ and create a dataset of $50,000$ points with balanced classes. Moreover, we sample additional covariates with the same procedure to create randomly labeled dataset. For both SGD and GD training, we use a fixed learning rate $0.1$.    

\textbf{\figref{fig:error_binary} (b) {} {}} Similar to binary CIFAR, we use clean training dataset of size $40,000$ and fix the amount of unlabeled data at $20\%$ of the clean dataset size. To train wide nets, we use a fixed learning of $0.001$ with GD and SGD. We decide the weight decay parameter and the early stopping point that maximizes our generalization bound (i.e. without peeking at unseen data ).  We use SGD batch size of $100$. 

\textbf{\figref{fig:error_binary} (c) {} {}} With IMDb dataset, we use a clean dataset of size $20,000$ and as before, fix the amount of unlabeled data at $20\%$ of the clean data. To train ELMo model, we use Adam optimizer with a fixed learning rate $0.01$ and weight decay $10^{-6}$ to minimize cross entropy loss. We train with batch size $32$ for 3 epochs. To fine-tune BERT model, we use Adam optimizer with learning rate $5\times 10^{-5}$ to minimize cross entropy loss. We train with a batch size of $16$ for 1 epoch.    

\textbf{\tabref{table:multiclass} {} {}} For multiclass datasets, we train both MLP and ResNet with the same hyperparameters as described before. We sample a clean training dataset of size $40,000$ and fix the amount of unlabeled data at $20\%$ of the clean size. We use SGD with an initial learning rate of $0.1$ and momentum $0.9$. We fix the weight decay parameter at $5\times 10^{-4}$. After $30$ epochs for ResNet and after $50$ epochs for MLP, we decay the learning rate to $0.01$.  We use SGD with batch size $100$. 
For \figref{fig:error_CIFAR100}, we use the same hyperparameters as 
CIFAR10 training, except we now decay learning rate after $100$ epochs. 


In all experiments, to identify the best possible accuracy on just the clean data, we use the exact same set of hyperparamters except the stopping point. We choose a stopping point that maximizes test performance. 

\subsection{Summary of experiments }

\begin{center}
    \begin{table}[H] 
        \centering
        \begin{tabular}{|c|c|c|c|} 
        \hline
        Classification type & Model category & Model & Dataset  \\ [0.5ex] 
        \hline
        \hline
        \multirow{10}{*}{Binary} & Low dimensional & Linear model & Toy Gaussain dataset  \\
                        \cline{2-4}
                         & Overparameterized 
                        %  & Linear model & Toy Gaussain dataset \\
                        %  \cline{3-4}
                        %  & & 2-layer wide net& Toy Gaussain dataset \\
                        %  \cline{3-4}
                         & \multirow{2}{*}{2-layer wide net} & \multirow{2}{*}{Binary MNIST} \\
                         & linear nets & &  
                         \\
                         \cline{2-4}                 
                         & \multirow{6}{*}{Deep nets} & \multirow{2}{*}{MLP} & Binary MNIST \\
                         \cline{4-4}
                         & &  & Binary CIFAR \\
                         \cline{3-4}
                         &  & \multirow{2}{*}{ResNet} & Binary MNIST \\
                         \cline{4-4}
                         & &  & Binary CIFAR \\
                         \cline{3-4}
                         &  & ELMo-LSTM model & IMDb Sentiment Analysis \\
                         \cline{3-4}
                         & & BERT pre-trained model & IMDb Sentiment Analysis \\
        \hline
        \multirow{5}{*}{Multiclass} & \multirow{5}{*}{Deep nets} & \multirow{2}{*}{MLP} & MNIST \\
                        \cline{4-4} 
                        & & & CIFAR10 \\                   
                        \cline{3-4}
                         &   & \multirow{3}{*}{ResNet} & MNIST \\
                         \cline{4-4}
                         &   & & CIFAR10 \\
                         \cline{4-4}
                         &   & & CIFAR100 \\
        \hline
        \end{tabular}
        % \caption{Summary of experiments performed} \label{table:experiments}
    \end{table}    
    % \footnotetext[6]{We use both MSE loss and cross-entropy loss.}
    % \footnotetext[6]{We try 2 variants: one with a fixed first layer and the other with both layers trainable.}
\end{center}

\newpage
\section{Proof of \lemref{lem:stability_error}} \label{app:proof_lem_error}

\begin{proof}[Proof of \lemref{lem:stability_error}]
    Recall, we have a training set $S \cup \wt S_C$. We defined leave-one-out error on mislabeled points as $$\error_{\text{LOO}(\wt S_M) } = \frac{\sum_{(x_i, y_i) \in \wt S_M} \error( f_{(i)}( x_i), y_i)}{ \abs{\wt S_M }} \,, $$
    where $f_{(i)} \defeq f(\calA, (S \cup \wt S)_{(i)})$. Define $S^\prime \defeq S \cup \wt S$. Assume $(x,y)$ and $(x^\prime,y^\prime)$ as i.i.d. samples from ${\calDm}$. 
    Using Lemma 25 in \citet{bousquet2002stability}, we have
    \begin{align*}
        \Expo{ \left( \error_{\calDm}(\wh f) -\error_{\text{LOO}(\wt S_M) } \right)^2 } \le & \Expt{ S^\prime, (x,y), (x^\prime,y^\prime) }{ \error(\wh f(x), y ) \error(\wh f(x^\prime), y^\prime )} - 2 \Expt{ S^\prime, (x,y) }{ \error(\wh f(x), y ) \error(f_{(i)}(x_i), y_i )} \\
        & + \frac{m_1-1}{m_1}\Expt{ S^\prime }{  \error(f_{(i)}(x_i), y_i )  \error(f_{(j)}(x_j), y_j )} + \frac{1}{m_1} \Expt{ S^\prime }{  \error(f_{(i)}(x_i), y_i ) } \,. \numberthis \label{eq:main_reln}
    \end{align*}
    We can rewrite the equation above as : 
    \begin{align*}
        \Expo{ \left( \error_{\calDm}(\wh f) -\error_{\text{LOO}(\wt S_M) } \right)^2 } \le &  \, \underbrace{\Expt{ S^\prime, (x,y), (x^\prime,y^\prime) }{ \error(\wh f(x), y ) \error(\wh f(x^\prime), y^\prime ) - \error(\wh f(x), y ) \error(f_{(i)}(x_i), y_i )}}_{\RN{1}} \\
        & + \underbrace{\Expt{ S^\prime }{  \error(f_{(i)}(x_i), y_i )  \error(f_{(j)}(x_j), y_j ) -  \error(\wh f(x), y ) \error(f_{(i)}(x_i), y_i )}}_{\RN{2}} \\ &+ \underbrace{\frac{1}{m_1} \Expt{ S^\prime }{  \error(f_{(i)}(x_i), y_i ) - \error(f_{(i)}(x_i), y_i )  \error(f_{(j)}(x_j), y_j ) }}_{\RN{3}} \,. \numberthis \label{eq:main_reln2}
    \end{align*}
    
    We will now bound term $\RN{3}$.  Using Cauchy-Schwarz's inequality, we have
    
    \begin{align}
        \Expt{ S^\prime }{  \error(f_{(i)}(x_i), y_i ) - \error(f_{(i)}(x_i), y_i )  \error(f_{(j)}(x_j), y_j ) }^2 &\le  \Expt{ S^\prime }{  \error(f_{(i)}(x_i), y_i ) }^2 \Expt{S^\prime}{1 -   \error(f_{(j)}(x_j), y_j ) }^2 \\
        &\le \frac{1}{4} \,.\label{eq:term1_lem12}
    \end{align}
    
    Note that since $(x_i,y_i)$, $(x_j ,y_j )$, $(x,y)$, and $(x^\prime, y^\prime)$ are all from same distribution $\calDm$, we directly incorporate the bounds on term $\RN{1}$ and $\RN{2}$ from the proof of Lemma 9 in \citet{bousquet2002stability}. Combining that with \eqref{eq:term1_lem12} and our definition of hypothesis stability in \codref{cond:hypothesis_stability}, we have the required claim. 
    
    
    % We now re-write term $\RN{1}$ as
    % \begin{align*}
    %         &\Expt{S^\prime, (x,y), (x^\prime,y^\prime) }{ \error(\wh f(x), y ) \error(\wh f(x^\prime), y^\prime ) - \error(\wh f(x), y ) \error(f_{(i)}(x_i), y_i )} \\ & \qquad = \Expt{ S^\prime, (x,y), (x^\prime,y^\prime) }{ \error(\wh f(x), y ) \error(\wh f  (x^\prime), y^\prime ) - \error(\wh f ^\prime(x), y ) \error(f_{(i)}(x^\prime), y^\prime )} \tag{Exchanging $(x_i, y_i)$ with $(x^\prime, y^\prime)$ in the second term} \\
    %         & \qquad = \Expt{ S^\prime, (x,y), (x^\prime,y^\prime) }{  \left(\error(\wh f(x), y )-  \error(f_{(i)}(x), y ) \right) \error(\wh f  (x^\prime), y^\prime )  } \\
    %         & \qquad  + \Expt{ S^\prime, (x,y), (x^\prime,y^\prime) }{  \left(\error(f_{(i)}(x), y ) -\error(\wh f ^\prime(x), y ) \right) \error(\wh f  (x^\prime), y^\prime )}  \\
    %         & \qquad +\Expt{ S^\prime, (x,y), (x^\prime,y^\prime) }{  \left( \error(\wh f  (x^\prime), y^\prime ) -  \error(f_{(i)}(x^\prime), y^\prime ) \right) \error(\wh f ^\prime(x), y ) }  \,, \numberthis \label{eq:term1_final}
    % \end{align*}
    % where $\wh f^\prime$ is the classifier obtained by training on $ S^\prime_{(i)} \cup \{ (x^\prime, y^\prime) \} $. Similarly we can re-write term $\RN{2}$ as 
    % \begin{align*}
    %     & \Expt{ S^\prime }{  \error(f_{(i)}(x_i), y_i )  \error(f_{(j)}(x_j), y_j ) -  \error(\wh f(x), y ) \error(f_{(i)}(x_i), y_i )} \\
    %     &\quad  = \Expt{ S^\prime, (x,y), (x^\prime,y^\prime)}{  \error(f^{\prime\prime}_{(i)}(x), y )  \error(f_{(j)}^{\prime}(x^\prime), y^\prime ) -  \error(\wh f(x), y ) \error(f_{(i)}(x_i), y_i )} \tag{Exchanging $(x_i, y_i)$ with $(x, y)$ and $(x_j, y_j)$ with $(x^\prime, y^\prime)$ in the first term}\\
    %     &\quad = \Expt{ S^\prime, (x,y), (x^\prime,y^\prime)}{  \error(f^{\prime\prime}_{(j)}(x), y )  \error(f_{(i)}^{\prime}(x^\prime), y^\prime ) -  \error(\wh f^\prime (x), y ) \error(f^\prime_{(j)}(x^\prime), y^\prime )} \tag{Exchanging $(x_i, y_i)$ and $(x_j, y_j)$ and then replacing $(x_j, y_j)$ with $(x^\prime, y^\prime)$ in the second term} \\
    %     & \quad = \Expt{ S^\prime, (x,y), (x^\prime,y^\prime) }{  \left( \error(f_{(i)}^{\prime}(x^\prime), y^\prime )   -  \error(\wh f^{\prime\prime}  (x^\prime), y^\prime ) \right)  \error(f^{\prime\prime}_{(j)}(x), y )   } \\
    %     & \quad  + \Expt{ S^\prime, (x,y), (x^\prime,y^\prime) }{  \left( \error(f^{\prime\prime}_{(j)}(x), y )  -\error(\wh f ^\prime(x), y ) \right) \error(\wh f^{\prime\prime}  (x^\prime), y^\prime )  }  \\
    %     & \quad+ \Expt{ S^\prime, (x,y), (x^\prime,y^\prime) }{  \left( \error(\wh f^{\prime\prime}  (x^\prime), y^\prime )  -  \error(f^\prime_{(j)}(x^\prime), y^\prime ) \right)  \error(\wh f^\prime (x), y ) }   \\
    %     & \quad = \Expt{ S^\prime, (x,y), (x^\prime,y^\prime) }{  \left( \error(f_{(i)}^{\prime}(x^\prime), y^\prime )   -  \error(\wh f (x^\prime), y^\prime ) \right)  \error(f_{(i)}(x_j), y_j )   } \\
    %     & \quad  + \Expt{ S^\prime, (x,y), (x^\prime,y^\prime) }{  \left( \error(f^{\prime\prime}_{(j)}(x), y )  -\error(\wh f (x), y ) \right) \error(\wh f^{\prime\prime}  (x_j), y_j )  }  \\
    %     & \quad+ \Expt{ S^\prime, (x,y), (x^\prime,y^\prime) }{  \left( \error(\wh f^{\prime\prime}  (x^\prime), y^\prime )  -  \error(f^\prime_{(j)}(x^\prime), y^\prime ) \right)  \error(\wh f^\prime (x^\prime), y^\prime ) }  \,, \numberthis \label{eq:term2_final}
    % \end{align*}
    % where $f^{\prime\prime}_{(j)}$ is trained on $S^\prime_{(j,i)} \cup {(x,y)}$, $f^{\prime}_{(i)}$ is trained on $S^\prime_{(j,i)} \cup {(x^\prime,y^\prime)}$, and $\wh f^{\prime\prime} $ is trained on $S^\prime_{(j)} \cup {(x,y)}$. Note in the last line we replaced $(x,y)$ by $(x_j, y_j)$ in the first term, replaced $(x^\prime,y^\prime)$ by $(x_j, y_j)$ in the second term and exchanged $(x_i,y_i)$ with $(x_j,y_j)$ and also $(x,y)$ and $(x^\prime, y^\prime)$
    
    
\end{proof}


% 
% 16th Century Version Control 
% 

% \onecolumn

% \section*{Supplementary Material}
% We will be using the following standard results
% on exponential concentration of random variables 
% all throughout the discussion:

% \begin{lemma}[Hoeffding's inequality for independent RVs~\citep{hoeffding1994probability}] Let $Z_1, Z_2, \ldots, Z_n$ be independent bounded random variables with $Z_i \in [a,b]$ for all $i$, then 
%     \begin{align*}
%         \prob\left( \frac{1}{n} \sum_{i=1}^n (Z_i - \Expo{Z_i}) \ge t \right) \le \exp{\left( -\frac{2nt^2}{(b-a)^2} \right) }
%     \end{align*} 
%     and 
%     \begin{align*}
%         \prob\left( \frac{1}{n} \sum_{i=1}^n (Z_i - \Expo{Z_i}) \le -t \right) \le \exp{\left( -\frac{2nt^2}{(b-a)^2} \right) }
%     \end{align*} 
%     for all $t \ge 0$. 
% \end{lemma}

% \begin{lemma}[Hoeffding's inequality for sampling with replacement~\citep{hoeffding1994probability}] \label{lem:hoeffding_sampling} Let $\calZ = (Z_1, Z_2, \ldots, Z_N)$ be a finite population of $N$ points with $Z_i \in [a.b]$ for all $i$. Let $X_1, X_2, \ldots X_n$ be a random sample drawn without replacement from $\calZ$. Then for all $t \ge 0$, we have 
%     \begin{align*}
%         \prob\left( \frac{1}{n} \sum_{i=1}^n (X_i - \mu ) \ge t \right) \le \exp{\left( -\frac{2nt^2}{(b-a)^2} \right) }
%     \end{align*} 
%     and 
%     \begin{align*}
%         \prob\left( \frac{1}{n} \sum_{i=1}^n (X_i - \mu ) \le -t \right) \le \exp{\left( -\frac{2nt^2}{(b-a)^2} \right) } \,,
%     \end{align*} 
%     where $\mu = \frac{1}{N} \sum_{i=1}^{N} Z_i$. 
% \end{lemma}

% We now discuss one condition that generalizes the exponential concentration to dependent random variables.
% \begin{condition}[Bounded difference inequality] \label{cond:BDC} Let $\calZ$ be some set and $\phi: \calZ^n \to \Real$. We say that $\phi$ satisfies the bounded difference assumption if 
% there exists $c_1, c_2, \ldots c_n \ge 0$ s.t. for all $i$, we have 
% \begin{align*}
%     \sup_{Z_1,Z_2, \ldots,Z_n, Z_i^\prime in \calZ^{n+1} } \abs{\phi (Z_1, \ldots, Z_i, \ldots, Z_n ) - \phi (Z_1, \ldots, Z_i^\prime, \ldots, Z_n ) } \le c_i \,.
% \end{align*} 
% \end{condition}

% \begin{lemma}[McDiarmid’s inequality~\citep{mcdiarmid1989}] \label{lem:McDiarmid} Let $Z_1, Z_2, \ldots, Z_n$ be independent random variables on set $\calZ$ and $\phi : \calZ^n \to \Real$ satisfy bounded difference assumption (\codref{cond:BDC}). Then for all $t>0$, we have 
%     \begin{align*}
%         \prob\left( \phi(Z_1, Z_2, \ldots, Z_n) - \Expo{\phi(Z_1, Z_2, \ldots, Z_n)} \ge t \right) \le \exp{\left( -\frac{2t^2}{\sum_{i=1}^n c_i^2} \right) } 
%     \end{align*} 
%     and 
%     \begin{align*}
%         \prob\left( \phi(Z_1, Z_2, \ldots, Z_n) - \Expo{\phi(Z_1, Z_2, \ldots, Z_n)} \le -t \right) \le \exp{\left( -\frac{2t^2}{\sum_{i=1}^n c_i^2} \right) } \,
%     \end{align*} 
% \end{lemma}


% \section{Proofs from \secref{sec:ERM_training}}\label{app:proof_erm}

% \textbf{Additional notation {} {}} Let $m_1$ be the number of mislabeled points ($\wt S_M$) and $m_2$ be the number of correctly labeled points ($\wt S_C$). Note $m_1 + m_2 = m$. 


% \subsection{Proof of \thmref{thm:error_ERM}}


% \begin{proof}[Proof of \lemref{lem:fit_mislabeled}] 
%     The main idea of our proof is to regard 
%     the clean portion of the data 
%     ($S \cup \wt S_C$) as fixed.   
%     Then, there exists a classifier $f^*$ 
%     that is optimal over draws 
%     of the mislabeled data $\wt S_M$. 
% % 
%     % 
%     Formally, 
%     \begin{align}
%     f^* \defeq \argmin_{f \in \calF} \error_{\widecheck {\calD}} (f) \,, \label{eq:modified_ERM}
%     \end{align}
%     where $$\widecheck \calD = \frac{n}{m+n} \calS + \frac{m_1}{m+n} \wt \calS_C  + \frac{m_2}{m+n}\calDm \,.$$ That is, $\widecheck \calD$ a combination of 
%     the \emph{empirical distribution} 
%     over correctly labeled data $S \cup \wt S_C$
%     % in $S\cup \wt S$ 
%     and the (population) distribution 
%     over mislabeled data $\calDm$.
%     Recall that 
%     \begin{align}
%     \wh f \defeq \argmin_{f \in \calF} \error_{\calS \cup \wt S} (f) \,. \label{eq:orig_ERM}
%     \end{align}
%     % 
%     % 
%     Since, $\widehat f$ minimizes 0-1 error 
%     on $S \cup \wt S$, using ERM optimality on \eqref{eq:orig_ERM},  
%     we have 
%     \begin{align}
%         \error_{\calS \cup \wt \calS}(\widehat f) \le \error_{
%             \calS \cup \wt \calS}(f^*) \,.    \label{eq:step1}
%     \end{align}
%     Moreover, since $f^*$ is independent of $\wt S_M$, using Hoeffding's bound,
%     % \footnote{For a fully rigorous argument,
%     % refer to the complete proof in App.~\ref{app:proof_erm}.} 
%     we have with probability at least $1-\delta$ that
%     \begin{align}
%       \error_{\wt \calS_M}(f^*) \le \error_{ \calDm}(f^*) +  \sqrt{\frac{\log(1/\delta)}{2 m_1}} \,. \label{eq:step2} 
%     \end{align}
%     %$ 
%     %for some constant $c_1\le 1/2$. 
%     Finally, since $f^*$ is the optimal classifier on $\widecheck \calD$, 
%     we have 
%     \begin{align}
%         \error_{\widecheck \calD}(f^*) \le \error_{\widecheck \calD}(\widehat f) \label{eq:step3}
%     \end{align}
%      Now to relate \eqref{eq:step1} and \eqref{eq:step3}, we can re-write the \eqref{eq:step2} as follows: 
%     \begin{align}
%         \error_{\calS \cup \wt\calS}(f^*) \le \error_{ \widecheck \calD}(f^*) +  \frac{m_1}{m+n}\sqrt{\frac{\log(1/\delta)}{2 m_1}} \,. \label{eq:step4} 
%     \end{align}
%     Now we combine equations \eqref{eq:step1}, \eqref{eq:step4}, and \eqref{eq:step3}, to get 
%     \begin{align}
%         \error_{\calS \cup \wt \calS}(\wh f) \le \error_{\widecheck \calD}(\wh f) +  \frac{m_1}{m+n}\sqrt{\frac{\log(1/\delta)}{2 m_1}} \,, 
%     \end{align}
%     which implies 
%     \begin{align}
%         \error_{ \wt \calS_M}(\wh f) \le \error_{\calDm}(\wh f) + \sqrt{\frac{\log(1/\delta)}{2 m_1}} \,. \label{eq:lemma1_final}
%     \end{align}
%     Since $\wt S$ is obtained by randomly labeling an unlabeled dataset, we assume $2m_1 \approx m$ \footnote{Formally, with probability at least $1-\delta$, we have  $(m - 2m_1)\le \sqrt{m\log(1/\delta)/2}$ }. Moreover, using $\error_{\calDm} = 1 - \error_{\calD}$ we obtain the desired result.   
%     % Combining the above steps and using the fact 
%     % that $\error_\calD = 1- \error_{\calDm} $, 
%     % we obtain the desired result.
% \end{proof}

% \begin{proof}[Proof of \lemref{lem:mislabeled_error}]
%     Recall $\error_{\wt S} (f) = \frac{m_1}{m} \error_{\wt S_M}(f) + \frac{m_2}{m} \error_{\wt S_C}(f)$. Hence, we have 
%     \begin{align}
%         2\error_{\wt S}(f) - \error_{\wt S_M}(f) - \error_{\wt S_C}(f) &= \left(\frac{2m_1}{m} \error_{\wt S_M}(f) - \error_{\wt S_M}(f)\right) + \left(\frac{2m_2}{m} \error_{\wt S_C}(f) - \error_{\wt S_C}(f)\right) \\ &= \left(\frac{2m_1}{m} - 1\right) \error_{\wt S_M}(f) + \left(\frac{2m_2}{m} - 1 \right)\error_{\wt S_C} (f) \,.
%     \end{align} 
%     Since the dataset is randomly labeled, with probability at least $1-\delta$, we have  $\left(\frac{2m_1}{m} - 1\right) \le \sqrt{\frac{\log(1/\delta)}{2m}}$. Similarly, we have with probability at least $1-\delta$, $\left(\frac{2m_2}{m} - 1\right) \le \sqrt{\frac{\log(1/\delta)}{2m}}$. Using union bound, we have with probability at least $1-\delta$
%     % \begin{align}
%     %     2\error_{\wt S} - \error_{\wt S_M}(f) - \error_{\wt S_C}(f) \le \sqrt{\frac{\log(2/\delta)}{2m}} \left(\error_{\wt S_M}(f) + \error_{\wt S_C}(f) \right) \le 2\sqrt{\frac{\log(2/\delta)}{2m}} \,. \label{eq:lemma2_final}
%     % \end{align}
%     \begin{align}
%         2\error_{\wt S} - \error_{\wt S_M}(f) - \error_{\wt S_C}(f) \le \sqrt{\frac{\log(2/\delta)}{2m}} \left(\error_{\wt S_M}(f) + \error_{\wt S_C}(f) \right) \,. \label{eq:lemma2_prefinal}
%     \end{align}
%     With re-arranging $\error_{\wt S_M}(f) + \error_{\wt S_C}(f)$ and using the inequality $ 1- a\le \frac{1}{1+a} $, we have  
%     \begin{align}
%         2\error_{\wt S} - \error_{\wt S_M}(f) - \error_{\wt S_C}(f) \le 2\error_{\wt \calS} \sqrt{\frac{\log(2/\delta)}{2m}}  \,. \label{eq:lemma2_final}
%     \end{align}

%     % We obtain the desired result by using 
% \end{proof}

% \begin{proof}[Proof of \lemref{lem:clear_error}]
% % Recall 0-1 error on each point  $(x,y) \in S \cup \wt S$ is given by $\I{ f(x)\ne y}$.
% In the set of correctly labeled points $S \cup \wt S_C$, we have $S$ as a random subset of $S \cup \wt S_C$. Hence, using Hoeffding's inequality for sampling without replacement (\lemref{lem:hoeffding_sampling}), we have with probability at least $1-\delta$
% \begin{align}
%     \error_{\wt \calS_c} (\wh f)- \error_{\calS \cup \wt \calS_C}( \wh f) \le  \sqrt{\frac{\log(1/\delta)}{2m_2}} \,.
% \end{align}
% Re-writing $\error_{\calS \cup \wt \calS_C}( \wh f)$ as $\frac{m_2}{m_2 + n} \error_{\wt \calS_C }(\wh f) + \frac{n}{m_2 + n} \error_{\calS }(\wh f)$, we have with probability at least $1-\delta$
% \begin{align}
%   \left(\frac{n}{n+m_2}\right) \left(\error_{\wt \calS_c} (\wh f)- \error_{\calS}( \wh f) \right) \le  \sqrt{\frac{\log(1/\delta)}{2m_2}} \,.
% \end{align}
% As before, assuming $2m_2 \approx m$, we have with probability at least $1-\delta$ 
% \begin{align}
%     \error_{\wt \calS_c} (\wh f)- \error_{\calS}( \wh f) \le \left(1+\frac{m_2}{n}\right)  \sqrt{\frac{\log(1/\delta)}{m}} \le 1.5 \sqrt{\frac{\log(1/\delta)}{m}} \,. \label{eq:lemma3_final}
% \end{align} 
% \end{proof}

% \begin{proof}[Proof of \thmref{thm:error_ERM}] 
%     Having established these core intermediate results, we can now combine above three lemmas to prove the main result. 
%     In particular, we bound the population error on clean data ($\error_\calD(\wh f)$) as follows:  
%     \begin{enumerate}[(i)]
%         \item First, use \eqref{eq:lemma1_final}, to obtain an upper bound on the population error on clean data, i.e., with probability at least $1-\delta/4$, we have
%         \begin{align}
%             \error_{ \calD} (\wh f) \le 1 - \error_{ \wt \calS_M}(\wh f) + \sqrt{\frac{\log(4/\delta)}{m}} \,. 
%         \end{align}
%         \item  Second, use \eqref{eq:lemma2_final}, to relate the error on the mislabeled fraction with error on clean portion of randomly labeled data and error on whole randomly labeled dataset, i.e., with probability at least $1-\delta/2$, we have 
%         \begin{align}
%             - \error_{\wt S_M}(f) \le \error_{\wt S_C}(f) - 2\error_{\wt S}  + \sqrt{\frac{\log(4/\delta)}{2m}}  \,. 
%         \end{align} 
%         \item Finally, use \eqref{eq:lemma3_final} to relate the error on the clean portion of randomly labeled data and error on clean training data, i.e., with probability $1-\delta/4$, we have 
%         \begin{align}
%             \error_{\wt \calS_C} (\wh f)\le - \error_{\calS}( \wh f) + \left(1 + \frac{m}{2n} \right) \sqrt{\frac{\log(4/\delta)}{m}} \,. 
%         \end{align} 
%     \end{enumerate}

%     Using union bound on the above three steps, we have with probability at least $1-\delta$: 
%     \begin{align}
%         \error_\calD (\wh f) \le \error_{\calS}(\wh f)   + 1 - 2\error_{\wt \calS}(\wh f)   + (1/\sqrt{2} + 2.5)  \sqrt{\frac{\log(4/\delta)}{m}} \,.
%     \end{align}
%     Note that $(1/\sqrt{2} + 2.5)$ is a loose constant. In experiments, we use the ratio $\frac{m}{n}$
%     %  the exact error $\error_{\wt \calS}(\wh f)$ 
%     to evaluate R.H.S.    
% \end{proof}

% \subsection{Proof of \propref{prop:rademacher}}

% \begin{proof}[Proof of \propref{prop:rademacher}]
%     For a classifier $ f: \calX \to \{-1, 1\}$, we have $1 - 2\,\indict{ f(x) \ne y} = y \cdot f(x)$. Hence, by definition of $\error$, we have 
%     \begin{align}
%         1 -2\error_{\wt \calS}(f) = \frac{1}{m}\sum_{i=1}^m y_i \cdot f(x_i) \le \sup_{f \in \calF} \, \frac{1}{m} \sum_{i=1}^m y_i \cdot f(x_i)  \,. \label{eq:error_rademacher}
%     \end{align}
%     Note that for fixed inputs $(x_1, x_2, \ldots, x_m)$ in $\wt S$, $(y_1, y_2, \ldots y_m)$ are random labels. Define $\phi_1 (y_1, y_2, \ldots, y_m) \defeq \sup_{f \in \calF} \, \frac{1}{m} \sum_{i=1}^m y_i \cdot f(x_i)$. We have the following bounded difference condition on $\phi_1$. For all i, 
%     \begin{align}
%         \sup_{y_1, \ldots y_m, y_i^\prime \in \{-1, 1\}^{m+1} } \abs{ \phi_1 (y_1,\ldots, y_i, \ldots, y_m) - \phi_1 (y_1,\ldots, y_i^\prime, \ldots, y_m)  } \le 1/m \,. \label{cond1_rademacher}
%     \end{align} 
    
%     Similarly define $\phi_2 (x_1, x_2, \ldots, x_m) \defeq \Expt{ y_i \sim_U \{-1, 1\}  }{ \sup_{f \in \calF} \, \frac{1}{m}  \sum_{i=1}^m y_i \cdot f(x_i)}$. We have the following bounded difference condition on $\phi_2$. For all i,
%     \begin{align}
%         \sup_{x_1, \ldots x_m, x_i^\prime \in \calX^{m+1} } \abs{ \phi_2 (x_1,\ldots, x_i, \ldots, x_m) - \phi_1 (x_1,\ldots, x_i^\prime, \ldots, x_m)  } \le 1/m \,. \label{cond2_rademacher}
%     \end{align}
%     Using McDiarmid’s inequality (\lemref{lem:McDiarmid}) twice with Condition \eqref{cond1_rademacher} and \eqref{cond2_rademacher}, with probability at least $1-\delta$, we have
%     \begin{align}
%         \sup_{f \in \calF} \, \frac{1}{m} \sum_{i=1}^m y_i \cdot f(x_i)  - \Expt{x,y}{\sup_{f \in \calF} \, \frac{1}{m} \sum_{i=1}^m y_i \cdot f(x_i) } \le \sqrt{\frac{2\log(2/\delta)}{m}} \label{eq:final_rademacher}
%     \end{align} 
%     Combining \eqref{eq:error_rademacher} and \eqref{eq:final_rademacher}, we obtain the desired result. 
% \end{proof}


% \subsection{Proof of \thmref{thm:error_regularized_ERM}}

% Proof of \thmref{thm:error_regularized_ERM} follows similar to the proof of \thmref{thm:error_ERM}. Note that the same results in \lemref{lem:fit_mislabeled}, \lemref{lem:mislabeled_error}, and \lemref{lem:clear_error} hold in the regularized ERM case. However, the arguments in the proof of \lemref{lem:fit_mislabeled} changes slightly. Hence, we state and prove a lemma parallel to \lemref{lem:fit_mislabeled} for completeness. 

% \begin{lemma} \label{lem:lemma1_reg}
%     Assume the same setup as \thmref{thm:error_regularized_ERM}. 
%     Then for any $\delta >0$, with probability at least  $1-\delta$ 
%     over the random draws of mislabeled data $\wt S_M$, we have 
%     \begin{align}
%         \error_\calD(\widehat f)  \le 1 -\error_{\wt \calS_M}(\widehat f) + \sqrt{\frac{\log(1/\delta)}{m}}\,. 
%     \end{align} 
% \end{lemma}
% \begin{proof}
%     The main idea of the proof remains the same, i.e. regard 
%     the clean portion of the data 
%     ($S \cup \wt S_C$) as fixed.   
%     Then, there exists a classifier $f^*$ 
%     that is optimal over draws 
%     of the mislabeled data $\wt S_M$. 

    
%     Formally, 
%     \begin{align}
%     f^* \defeq \argmin_{f \in \calF} \error_{\widecheck {\calD}} (f)  + \lambda R(f) \,, \label{eq:modified_ERM_reg}
%     \end{align}
%     where $$\widecheck \calD = \frac{n}{m+n} \calS + \frac{m_1}{m+n} \wt \calS_C  + \frac{m_2}{m+n}\calDm \,.$$ That is, $\widecheck \calD$ a combination of 
%     the \emph{empirical distribution} 
%     over correctly labeled data $S \cup \wt S_C$
%     % in $S\cup \wt S$ 
%     and the (population) distribution 
%     over mislabeled data $\calDm$.
%     Recall that 
%     \begin{align}
%     \wh f \defeq \argmin_{f \in \calF} \error_{\calS \cup \wt S} (f) + \lambda R(f) \,. \label{eq:orig_ERM_reg}
%     \end{align}
%     % 
%     % 
%     Since, $\widehat f$ minimizes 0-1 error 
%     on $S \cup \wt S$, using ERM optimality on \eqref{eq:orig_ERM},  
%     we have 
%     \begin{align}
%         \error_{\calS \cup \wt \calS}(\widehat f) + \lambda R(\wh f) \le \error_{
%             \calS \cup \wt \calS}(f^*) + \lambda R(f^*) \,.    \label{eq:step1_reg}
%     \end{align}
%     Moreover, since $f^*$ is independent of $\wt S_M$, using Hoeffding's bound,
%     % \footnote{For a fully rigorous argument,
%     % refer to the complete proof in App.~\ref{app:proof_erm}.} 
%     we have with probability at least $1-\delta$ that
%     \begin{align}
%       \error_{\wt \calS_M}(f^*) \le \error_{ \calDm}(f^*) +  \sqrt{\frac{\log(1/\delta)}{2 m_1}} \,. \label{eq:step2_reg} 
%     \end{align}
%     %$ 
%     %for some constant $c_1\le 1/2$. 
%     Finally, since $f^*$ is the optimal classifier on $\widecheck \calD$, 
%     we have 
%     \begin{align}
%         \error_{\widecheck \calD}(f^*) + \lambda R(f^*) \le \error_{\widecheck \calD}(\widehat f) + \lambda R(\wh f) \label{eq:step3_reg}
%     \end{align}
%      Now to relate \eqref{eq:step1_reg} and \eqref{eq:step3_reg}, we can re-write the \eqref{eq:step2_reg} as follows: 
%     \begin{align}
%         \error_{\calS \cup \wt\calS}(f^*) \le \error_{ \widecheck \calD}(f^*) +  \frac{m_1}{m+n}\sqrt{\frac{\log(1/\delta)}{2 m_1}} \,. \label{eq:step4_reg} 
%     \end{align}
%     After adding $\lambda R(f^*)$ on both sides in \eqref{eq:step4_reg}, we combine equations \eqref{eq:step1_reg}, \eqref{eq:step4_reg}, and \eqref{eq:step3_reg}, to get 
%     \begin{align}
%         \error_{\calS \cup \wt \calS}(\wh f) \le \error_{\widecheck \calD}(\wh f) +  \frac{m_1}{m+n}\sqrt{\frac{\log(1/\delta)}{2 m_1}} \,, 
%     \end{align}
%     which implies 
%     \begin{align}
%         \error_{ \wt \calS_M}(\wh f) \le \error_{\calDm}(\wh f) + \sqrt{\frac{\log(1/\delta)}{2 m_1}} \,. \label{eq:lemma_reg_final}
%     \end{align}
%     Similar as before, since $\wt S$ is obtained by randomly labeling an unlabeled dataset, we assume 
%     $2m_1 \approx m$. Moreover, using $\error_{\calDm} = 1 - \error_{\calD}$ we obtain the desired result. 
% \end{proof}
% % \begin{proof}[Proof of ]
    
% % \end{proof}

% \subsection{Proof of \thmref{thm:multiclass_ERM}}

% We first state and prove lemmas parallel to three lemmas used in the proof of balanced binary case. Then we combine the results in the three lemmas to obtain the result in \thmref{thm:multiclass_ERM}. 

% Before stating the result, we define mislabeled distribution $\calDm$ for any $\calD$. While $\calDm$ and $\calD$ share 
% the same marginal distribution over $\calX$, 
% the distribution over labels $y$ 
% given an input $x\sim \calD_\calX$ is changed.
% In particular, for any $x$, the pdf over $y$ is changed to:  
% $p_{\calDm} (\cdot \vert x) \defeq \frac{1 - p_{\calD}(\cdot \vert x)}{k - 1}$.

% \begin{lemma} \label{lem:fit_mislabeled_multi}
%     Assume the same setup as \thmref{thm:multiclass_ERM}. 
%     Then for any $\delta >0$, with probability at least  $1-\delta$ 
%     over the random draws of mislabeled data $\wt S_M$, we have 
%     \begin{align}
%         \error_\calD(\widehat f)  \le (k-1)\left(1 -\error_{\wt \calS_M}(\widehat f)\right) + (k-1)\sqrt{\frac{\log(1/\delta)}{m}}\,. \label{eq:lemma1_multi}
%     \end{align}   
% \end{lemma} 

% \begin{proof}
%     The main idea of the proof remains the same, i.e. regard 
%     the clean portion of the data 
%     ($S \cup \wt S_C$) as fixed. 
%     Then, there exists a classifier $f^*$ 
%     that is optimal over draws 
%     of the mislabeled data $\wt S_M$. 
    
%     However, we need to be careful while relating population error on mislabeled data with population accuracy on clean data.   
%     While for binary classification,  we could upper bound $\error_{\wt \calS_M}$ 
%     with $1-\error_\calD$  (in the proof of \lemref{lem:fit_mislabeled}), 
%     for multiclass classification, 
%     error on the mislabeled data 
%     and accuracy on the clean data 
%     in the population 
%     are not so directly related.  
%     To establish \eqref{eq:lemma1_multi},
%     we break the error on the 
%     (unknown) mislabeled data 
%     into two parts: one term corresponds 
%     to predicting the true label on mislabeled data, 
%     and the other corresponds to predicting 
%     neither the true label 
%     nor the assigned (mis-)label.  
%     Finally, we relate these errors to their
%     population counterparts to establish \eqref{eq:lemma1_multi}. 
    
%     Formally, 
%     \begin{align}
%     f^* \defeq \argmin_{f \in \calF} \error_{\widecheck {\calD}} (f)  + \lambda R(f) \,, \label{eq:modified_ERM_reg2}
%     \end{align}
%     where $$\widecheck \calD = \frac{n}{m+n} \calS + \frac{m_1}{m+n} \wt \calS_C  + \frac{m_2}{m+n}\calDm \,.$$ That is, $\widecheck \calD$ a combination of 
%     the \emph{empirical distribution} 
%     over correctly labeled data $S \cup \wt S_C$
%     % in $S\cup \wt S$ 
%     and the (population) distribution 
%     over mislabeled data $\calDm$.
%     Recall that 
%     \begin{align}
%     \wh f \defeq \argmin_{f \in \calF} \error_{\calS \cup \wt S} (f) + \lambda R(f) \,. \label{eq:orig_ERM_reg2}
%     \end{align}
%     % 
%     % 
%     Following the exact steps from the proof of \lemref{lem:lemma1_reg}, with probability at least $1-\delta$, we have  
%     \begin{align}
%         \error_{ \wt \calS_M}(\wh f) \le \error_{\calDm}(\wh f) + \sqrt{\frac{\log(1/\delta)}{2 m_1}} \,. \label{eq:lemma1_final_multi_prev}
%     \end{align}
%     Similar to before, since $\wt S$ is obtained by randomly labeling an unlabeled dataset, we assume 
%     $\frac{k}{k-1} m_1 \approx m$. 
    
%     Now we will relate $\error_\calDm (\wh f)$ with $\error_{\calD}(\wh f)$. Let $y^T$ denote the (unknown) true label for a mislabeled point $(x, y)$ (i.e., label before replacing it with a mislabel). 
%     \begin{align}    
%          \Expt{(x, y) \in \sim \calDm}{\indict{ \wh f(x) \ne y }}  &= \underbrace{\Expt{(x, y) \in \sim \calDm}{\indict{ \wh f(x) \ne y \land \wh f(x) \ne y^T}}}_{\RN{1}} + \underbrace{\Expt{(x, y) \in \sim \calDm}{\indict{ \wh f(x) \ne y \land \wh f(x) = y^T}}}_{\RN{2}} \,. \label{eq:excess_term}
%     \end{align}
%     Clearly, term 2 is one minus the accuracy on the clean unseen data, i.e. 
%     \begin{align}
%         \RN{2} = 1 - \Expt{{x,y} \sim \calD}{ \indict{ \wh f(x) \ne y}} = 1- \error_{\calD}(\wh f) \,. \label{eq:term1}    
%     \end{align}
%     Next, we  relate term 1 with the error on the unseen clean data. We show that term 1 is equal to the error on the unseen clean data scaled by $\frac{k-2}{k-1}$ where $k$ is the number of labels. Using the definition of mislabeled distribution $\calDm$,  we have 
%     \begin{align}
%         \RN{1} = \frac{1}{k-1} \left( \Expt{(x, y) \in \sim \calD}{ \sum_{i \in \calY \land i\ne y}  \indict{ \wh f(x) \ne i \land \wh f(x) \ne y}} \right) = \frac{k-2}{k-1} \error_{\calD}(\wh f) \,.\label{eq:term2}
%     \end{align}    

%     Combining the result in \eqref{eq:term1}, \eqref{eq:term2} and \eqref{eq:excess_term}, we have 
%     \begin{align}
%         \error_{\calDm}(\wh f) = 1- \frac{1}{k-1} \error_{\calD}(\wh f) \,.\label{eq:combine_terms}
%     \end{align}
%     Finally, combining the result in \eqref{eq:combine_terms} with equation \eqref{eq:lemma1_final_multi_prev}, we have with probability $1-\delta$, 
%     \begin{align}
%       \error_{\calD}(\wh f) \le  (k-1) \left( 1- \error_{ \wt \calS_M}(\wh f) \right)  + (k-1) \sqrt{\frac{k \log(1/\delta)}{ 2(k-1)m}} \,. \label{eq:lemma1_final_multi}
%     \end{align}
% \end{proof}

% \begin{lemma} \label{lem:mislabeled_error_multi}
%     Assume the same setup as \thmref{thm:multiclass_ERM}.  Then for any $\delta >0$, with probability at least $1-\delta$ over the random draws of $\wt S$, we have  
%     % \begin{align}
%         $$\abs{k\error_{\wt \calS}(\widehat f) - \error_{\wt \calS_C}(\widehat f) -  (k-1)\error_{\wt \calS_M}(\widehat f) } \le  2k\sqrt{\frac{\log(4/\delta)}{2m}}\,. $$ % \label{eq:lemma2}
%     % \end{align}   
%     %  for some constant $c_3 \le 1.0\,$.
% \end{lemma} 


% \begin{proof}
%     Recall $\error_{\wt S} (f) = \frac{m_1}{m} \error_{\wt S_M}(f) + \frac{m_2}{m} \error_{\wt S_C}(f)$. Hence, we have 
%     \begin{align}
%         k\error_{\wt S}(f) - (k-1)\error_{\wt S_M}(f) - \error_{\wt S_C}(f) &= (k-1)\left(\frac{k m_1}{(k-1) m} \error_{\wt S_M}(f) - \error_{\wt S_M}(f)\right) + \left(\frac{km_2}{m} \error_{\wt S_C}(f) - \error_{\wt S_C}(f)\right) \\ &= k \left[ \left(\frac{m_1}{m} - \frac{k-1}{k}\right) \error_{\wt S_M}(f) + \left(\frac{m_2}{m} - \frac{1}{k} \right) \error_{\wt S_C} (f) \right] \,.
%     \end{align} 
%     Since the dataset is randomly labeled, we have with probability at least $1-\delta$, $\left(\frac{m_1}{m} - \frac{k-1}{k}\right) \le \sqrt{\frac{\log(1/\delta)}{2m}}$. Similarly, we have with probability at least $1-\delta$, $\left(\frac{m_2}{m} - \frac{1}{k}\right) \le \sqrt{\frac{\log(1/\delta)}{2m}}$. Using union bound, we have with probability at least $1-\delta$
%     % \begin{align}
%     %     2\error_{\wt S} - \error_{\wt S_M}(f) - \error_{\wt S_C}(f) \le \sqrt{\frac{\log(2/\delta)}{2m}} \left(\error_{\wt S_M}(f) + \error_{\wt S_C}(f) \right) \le 2\sqrt{\frac{\log(2/\delta)}{2m}} \,. \label{eq:lemma2_final}
%     % \end{align}
%     \begin{align}
%         k\error_{\wt S}(f) - (k-1)\error_{\wt S_M}(f) - \error_{\wt S_C}(f)  \le k \sqrt{\frac{\log(2/\delta)}{2m}} \left(\error_{\wt S_M}(f) + \error_{\wt S_C}(f) \right) \,. \label{eq:lemma2_final_multi}
%     \end{align}

%     % We obtain the desired result by using 
% \end{proof}

% \begin{lemma} \label{lem:clear_error_multi}
%     Assume the same setup as \thmref{thm:multiclass_ERM}. 
%     Then for any $\delta >0$, with probability at least $1-\delta$ 
%     over the random draws of $\wt S_C$ and $S$, we have 
%     % \begin{align}
%         $$\abs{\error_{\wt \calS_C}(\widehat f) - \error_{\calS}(\widehat f) } \le 1.5 \sqrt{\frac{k\log(2/\delta)}{2m}}\,.$$ %\label{eq:lemma3}
%     % \end{align}   
%     % for some constant $c_2 \le 1.2\,$.
% \end{lemma} 
% \begin{proof}
%     % Recall 0-1 error on each point  $(x,y) \in S \cup \wt S$ is given by $\I{ f(x)\ne y}$.
%     In the set of correctly labeled points $S \cup \wt S_C$, we have $S$ as a random subset of $S \cup \wt S_C$. Hence, using Hoeffding's inequality for sampling without replacement (\lemref{lem:hoeffding_sampling}), we have with probability at least $1-\delta$
%     \begin{align}
%         \error_{\wt \calS_c} (\wh f)- \error_{\calS \cup \wt \calS_C}( \wh f) \le  \sqrt{\frac{\log(1/\delta)}{2m_2}} \,.
%     \end{align}
%     Re-writing $\error_{\calS \cup \wt \calS_C}( \wh f)$ as $\frac{m_2}{m_2 + n} \error_{\wt \calS_C }(\wh f) + \frac{n}{m_2 + n} \error_{\calS }(\wh f)$, we have with probability at least $1-\delta$
%     \begin{align}
%       \left(\frac{n}{n+m_2}\right) \left(\error_{\wt \calS_c} (\wh f)- \error_{\calS}( \wh f) \right) \le  \sqrt{\frac{\log(1/\delta)}{2m_2}} \,.
%     \end{align}
%     As before, assuming $km_2 \approx m$, we have with probability at least $1-\delta$ 
%     \begin{align}
%         \error_{\wt \calS_c} (\wh f)- \error_{\calS}( \wh f) \le \left(1+\frac{m_2}{n}\right)  \sqrt{\frac{k\log(1/\delta)}{2m}} \le \left( 1 + \frac{1}{k}\right) \sqrt{\frac{k\log(1/\delta)}{2m}} \,. \label{eq:lemma3_final_multi}
%     \end{align} 
% \end{proof}

% \begin{proof}[Proof of \thmref{thm:multiclass_ERM}] 
%     Having established these core intermediate results, we can now combine above three lemmas. 
%     In particular, we bound the population error on clean data ($\error_\calD(\wh f)$) as follows:  
%     \begin{enumerate}[(i)]
%         \item First, use \eqref{eq:lemma1_final_multi}, to obtain an upper bound on the population error on clean data, i.e., with probability at least $1-\delta/4$, we have
%         \begin{align}
%             \error_{ \calD} (\wh f) \le (k-1)\left(1 - \error_{ \wt \calS_M}(\wh f) \right) + (k-1) \sqrt{\frac{k\log(4/\delta)}{2(k-1)m}} \,. 
%         \end{align}
%         \item  Second, use \eqref{eq:lemma2_final_multi}, to relate the error on the mislabeled fraction with error on clean portion of randomly labeled data and error on whole randomly labeled dataset, i.e., with probability at least $1-\delta/2$, we have 
%         \begin{align}
%             - (k-1)\error_{\wt S_M}(f) \le \error_{\wt S_C}(f) - k\error_{\wt S}  + k\sqrt{\frac{\log(4/\delta)}{2m}}  \,. 
%         \end{align} 
%         \item Finally, use \eqref{eq:lemma3_final_multi} to relate the error on the clean portion of randomly labeled data and error on clean training data, i.e., with probability $1-\delta/4$, we have 
%         \begin{align}
%             \error_{\wt \calS_C} (\wh f)\le - \error_{\calS}( \wh f) + \left(1 + \frac{m}{kn} \right) \sqrt{\frac{k\log(4/\delta)}{2m}} \,. 
%         \end{align} 
%     \end{enumerate}

%     Using union bound on the above three steps, we have with probability at least $1-\delta$: 
%     \begin{align}
%         \error_\calD (\wh f) \le \error_{\calS}(\wh f) + (k-1) - k\error_{\wt \calS}(\wh f)   + (\sqrt{k(k-1)} + k + \sqrt{k} + \frac{m}{n\sqrt{k}})  \sqrt{\frac{\log(4/\delta)}{2m}} \,.
%     \end{align}
%     % Note that $\frac{m}{n\sqrt{k}}$ is much smaller than the other terms in addition. Hence, we ignore this in the final bound. 
%     % Note that $(1/\sqrt{2} + 2.5)$ is a loose constant. In experiments, we use the ratio $\frac{m}{n}$
%     %  the exact error $\error_{\wt \calS}(\wh f)$ 
%     % to evaluate R.H.S.    
% \end{proof}

% \newpage
% \section{Proofs from \secref{sec:linear_models}}\label{app:proof_gd}

% We suppose that the parameters of the linear function 
% are obtained via gradient descent on 
% the following $L_2$ regularized problem: 
% \begin{align}
%     % n in denominator is avoided deliberately
%     \calL_S(w; \lambda) \defeq \sum_{i=1}^n{(w^Tx_i - y_i)^2} + \lambda \norm{w}{2}^2 \,, \label{eq:l2_MSE_app}   
% \end{align}
% where $\lambda\ge0$ is a regularization parameter. 
% We assume access to a clean dataset 
% $S = \{(x_i, y_i)\}_{i=1}^n \sim \calD^n$ 
% and randomly labeled dataset 
% $\wt S = \{(x_i, y_i)\}_{i=n+1}^{n+m} \sim \wt \calD^m$. 
% Let $\bX = [x_1, x_2, \cdots, x_{m+n}]$ 
% and $\by = [y_1, y_2, \cdots, y_{m+n}]$. 
% Fix a positive learning rate $\eta$ such that 
% $\eta \le 1/\left(\norm{\bX^T\bX}{\text{op}} + \lambda^2\right)$ 
% and an initialization $w_0 = 0$. 
% % \todos{Assumption made for simplicty}. 
% Consider the following gradient descent iterates 
% to minimize objective \eqref{eq:l2_MSE_app} on $S \cup \wt S$:
% \begin{align}
% w_t = w_{t-1} - \eta \grad_w \calL_{S \cup \wt S} (w_{t-1}; \lambda) \quad \forall t=1,2,\ldots \label{eq:GD_iterates_app}
% \end{align} 
% Then we have $\{ w_t\}$ converge to the limiting solution 
% $\wh w = \left( \bX^T\bX+\lambda \boldsymbol{I}\right)^{-1}\bX^T\by$. Define $\widehat f (x) \defeq f(x ; \wh w) $.  

% \subsection{\textcolor{red}{Errata}}

% We wish to correct the following error in the body: \codref{cond:error_stability} is not enough to guarantee the result in \thmref{thm:linear}. We now present a slightly stronger condition called \emph{hypothesis stability} under which we obtain a result similar to \thmref{thm:linear}. 

% This error doesn't change the main arguments of the proof where we show that the empirical train error is less than or equal to the leave-one-out error. We need a stronger condition to relate leave-one-out error with the population error of the original classifier. Specifically, while \codref{cond:error_stability} relates the average population error of leave-one-out classifiers with the population error of the original classifier, we need the new condition to show the concentration of the empirical leave-one-out error and  average population error of leave-one-out classifiers. 
% % main takeaway 

% Note that the new condition, while being stronger than the previous one, still doesn't imply generalization~\cite{bousquet2002stability,elisseeff2003leave,abou2019exponential}. Overall, the main results in \secref{sec:ERM_training} and takeaways of the paper remain unaffected by the error.  

% We now present the new condition and a corrected statement of \thmref{thm:linear}. Recall, for a given training set $S \sim \calD^n $, 
% we use $S_{(i)}$ to denote the training set $S$ 
% with the $i^{\text{th}}$ point removed.

% \begin{condition}[Hypothesis Stability] 
%     \label{cond:hypothesis_stability}
%     We have $\beta$ hypothesis stability 
%     if our training algorithm $\calA$ satisfies the following: 
%     \begin{align*}
%     % ${\sum_{i=1}^n \frac{\error_{\calD}( f(\calA, S_{(i)}))}{n} - \error_\calD(f(\calA, S))} \le \beta\,$.
%     \forall i \in \{1,2,\ldots, n\}, \quad  \Expt{\calS, (x,y) \in \calD}{ \abs{\error\left( f(x) ,y  \right) - \error\left( f_{(i)}(x), y \right) }} \le \frac{\beta}{n} \,,
%     \end{align*}
%     where $f_{(i)} \defeq f(\calA, S_{(i)})$ and $ f \defeq f(\calA, S)$.
% \end{condition}

% \begin{theorem}[Correct statement of \thmref{thm:linear}] \label{thm:new_linear}
%     Assume that this gradient descent algorithm satisfies \codref{cond:hypothesis_stability}
%     with $\beta=\calO(1)$.  
%     Then for any $\delta >0$, with probability at least $1-\delta$ 
%     over the random draws of datasets $\wt S$ and $S$, we have:
%     \begin{align}
%         \error_\calD(\widehat f) \le \error_\calS(\widehat f) + 1 - 2 \error_{\wt\calS}(\widehat f) + \left(\frac{1}{\sqrt{2}} + 1.5 \right) \sqrt{\frac{\log(4/\delta)}{m}} + \sqrt{\frac{4}{\delta}\left(\frac{1}{m} +\frac{3\beta}{m+n} \right)}  \,. \label{eq:gd_error}
%     \end{align} 
%     % for some constant $c\le 3.2$.
% \end{theorem}

% \subsection{Proof of \thmref{thm:new_linear}}
% We use a standard result from linear algebra, namely Shermann-Morrison formula~\citep{sherman1950adjustment} for matrix inversion:  

% \begin{lemma}[\citet{sherman1950adjustment}] \label{lem:sherman}
%     Suppose $\bA \in \Real^{n \times n}$ is an invertible square matrix and $u,v \in \Real^n$ are column vectors. Then $\bA + uv^T$ is invertible iff $1 + v^T \bA u \ne 0$ and in particular
%     \begin{align}
%         (\bA + u v^T)^{-1} = \bA^{-1}  - \frac{\bA^{-1} uv^T \bA^{-1} }{ 1 + v^T \bA^{-1} u} \,.
%     \end{align}   
% \end{lemma}
% \newcommand\byy[1]{\by_{\left(#1\right)}}
% \newcommand\bXX[1]{\bX_{\left(#1\right)}}
% \newcommand\ff[1]{\wh f_{\left(#1\right)}}

% For a given training set $S \cup \wt S_C$, define leave-one-out error on mislabeled points in the training data as $$\error_{\text{LOO}(\wt S_M) } = \frac{\sum_{(x_i, y_i) \in \wt S_M} \error( f_{(i)}( x_i), y_i)}{ \abs{\wt S_M }} \,, $$
% where $f_{(i)} \defeq f(\calA, (S \cup \wt S)_{(i)})$. To relate empirical leave-one-out error and population error with hypothesis stability condition, we use the following lemma:   

% \begin{lemma}[\citet{bousquet2002stability}] \label{lem:stability_error}
%     For the leave-one-out error, we have
%     \begin{align}
%         \Expo{ \left( \error_{\calDm}(\wh f) -\error_{\text{LOO}(\wt S_M) } \right)^2 } \le \frac{1}{2m_1}+  \frac{3\beta}{n + m}\,.
%     \end{align}   
%     % where $ f \defeq f(\calA, S \cup \wt S) $.
% \end{lemma}

% Proof of the above lemma is similar to the proof of  Lemma 9 in \citet{bousquet2002stability} and can be found in \appref{app:proof_lem_error}. 
% % 
% % Before presenting the result, we introduce some notation. 
% Before presenting the proof of \thmref{thm:new_linear}, we introduce some more notation. Let $\bX_{(i)}$ denote the matrix of covariates with $i^{\text{th}}$ point removed. Similarly let $\by_{(i)}$ be the array of responses with $i^{\text{th}}$ point removed. Define the corresponding regularized GD solution as $\wh w_{(i)} = \left( \bXX{i}^T\bXX{i}+\lambda \boldsymbol{I}\right)^{-1}\bXX{i}^T\byy{i}$. Define $\ff{i}(x) \defeq f(x ; \wh w_{(i)}) $.

% \begin{proof}[Proof of \thmref{thm:new_linear}]
%     Because squared loss minimization does not imply 0-1 error minimization, we cannot use arguments from \lemref{lem:fit_mislabeled}. This is the main technical difficulty. To compare the 0-1 error at a train point with an unseen point, 
%     we use the closed-form expression for $\widehat{w}$ and Shermann-Morrison formula to upper bound training error with leave-one-out cross validation error. 
    
%     The proof is divided into three parts: In part one, we show that 0-1 error on mislabeled points in the training set is lower than the error obtained by leave-one-out error at those points. In part two, we relate this leave-one-out error with the population error on mislabeled distribution using \codref{cond:hypothesis_stability}. While the empirical leave-one-out error is unbiased estimator of the average population error of leave-one-out classifiers, we need hypothesis stability to control the variance of empirical leave-one-out error. Finally in part three, we show that the error on the mislabeled training points can be estimated with just the randomly labeled and  clean training data (as in proof of \thmref{thm:error_ERM}).  

%     \textbf{Part 1 {} {}} First we relate training error with leave-one-out error.        
%     For any 
%     training point $(x_i, y_i)$ in $\wt S \cup S$, we have 
%     \begin{align}
%         \error(\wh f(x_i), y_i ) &= \indict{ y_i \cdot x_i^T \wh w < 0 } = \indict{ y_i \cdot x_i^T \left( \bX^T\bX+\lambda \boldsymbol{I}\right)^{-1}\bX^T\by < 0 } \\
%         &= \indict{ y_i \cdot x_i^T \underbrace{\left( \bXX{i}^T\bXX{i} + x_i ^T x_i +\lambda \boldsymbol{I}\right)^{-1}}_{\RN{1}} (\bXX{i}^T\byy{i} + y \cdot x_i) < 0 }
%     \end{align}
%     Letting $\bA = \left(\bXX{i}^T\bXX{i} +\lambda \boldsymbol{I}\right)$ and using \lemref{lem:sherman} on term 1, we have 
%     \begin{align}
%         \error(\wh f(x_i), y_i ) &= \indict{ y_i \cdot x_i^T \left[\bA^{-1} -  \frac{\bA^{-1} x_i x_i^T \bA^{-1}}{ 1 + x_i ^T \bA^{-1} x_i } \right] (\bXX{i}^T\byy{i} + y \cdot x_i) < 0 } \\
%         &= \indict{ y_i \cdot\left[ \frac{ x_i^T \bA^{-1} ( 1 + x_i ^T \bA^{-1} x_i ) -  x_i^T \bA^{-1} x_i x_i^T \bA^{-1}}{ 1 + x_i ^T \bA ^{-1}x_i } \right] (\bXX{i}^T\byy{i} + y \cdot x_i) < 0 } \\
%         &= \indict{ y_i \cdot\left[ \frac{ x_i^T \bA^{-1}}{ 1 + x_i ^T \bA ^{-1}x_i } \right] (\bXX{i}^T\byy{i} + y \cdot x_i) < 0 } \,.
%     \end{align}

%     Since $1 + x_i^T \bA^{-1} x_i > 0$, we have 
%     \begin{align}
%         \error(\wh f(x_i), y_i ) &= \indict{ y_i \cdot x_i^T \bA^{-1} (\bXX{i}^T\byy{i} + y \cdot x_i) < 0 } \\
%         &= \indict{ x_i^T \bA^{-1} x_i +  y_i \cdot x_i^T \bA^{-1} (\bXX{i}^T\byy{i}) < 0 } \\
%         &\le \indict{ y_i \cdot x_i^T \bA^{-1} (\bXX{i}^T\byy{i}) < 0 } = \error(\ff{i}(x_i), y_i ) \,.\label{eq:LOO_error}
%     \end{align}

%     Using \eqref{eq:LOO_error}, we have 
%     \begin{align}
%         \error_{\wt \calS_M } (\wh f) \le \error_{\text{LOO} (S_M)} \defeq \frac{\sum_{(x_i, y_i) \in \wt S_M} \error(\ff{i}(x_i), y_i ) }{\abs{\wt \calS_M}}\label{eq:LOO_error_final}
%     \end{align}
%     \textbf{Part 2 {}{}} We now relate RHS in \eqref{eq:LOO_error_final} with the population error on mislabeled distribution. To do this, we leverage \codref{cond:hypothesis_stability} and \lemref{lem:stability_error}. In particular, we have 

%     \begin{align}
%         \Expt{\calS \cup \wt \calS_M }{ \left(\error_{\calDm}(\wh f) - \error_{\text{LOO} (S_M)}\right)^2 } \le \frac{1}{2m_1} + \frac{3\beta}{m+n} \,.
%     \end{align}

%     Using Chebyshev's inequality, with probability at least $1-\delta$, we have 
%     \begin{align}
%         \error_{\text{LOO} (S_M)} \le  \error_{\calDm}(\wh f)   + \sqrt{\frac{1}{\delta}\left(\frac{1}{2m_1} +\frac{3\beta}{m+n} \right)} \,. \label{eq:final_mislabeled_linear}
%     \end{align}
    

%     \textbf{Part 3 {}{}} Combining \eqref{eq:final_mislabeled_linear} and \eqref{eq:LOO_error_final}, we have 

%     \begin{align}
%         \error_{\wt \calS_M } (\wh f) \le \error_{\calDm}(\wh f)   + \sqrt{\frac{1}{\delta}\left(\frac{1}{2m_1} +\frac{3\beta}{m+n} \right)} \,. \label{eq:linear_parallel_lem1}
%     \end{align}

%     Compare \eqref{eq:linear_parallel_lem1}, with \eqref{eq:lemma1_final} in the proof of \lemref{lem:fit_mislabeled}. We obtain a similar relationship between $\error_{\wt \calS_M }$ and $\error_{\calDm}$ but with a polynomial concentration instead of exponential concentration. 
%     In addition, since we just use concentration arguments to relate mislabeled error with the error on clean portion and unlabeled portion, we can directly use the results in \lemref{lem:mislabeled_error} and \lemref{lem:clear_error}. Therefore, combining results in \lemref{lem:mislabeled_error}, \lemref{lem:clear_error}, and \eqref{eq:linear_parallel_lem1} with union bound, we have with probability at least $1-\delta$

%     \begin{align}
%         \error_\calD(\widehat f) \le \error_\calS(\widehat f) + 1 - 2 \error_{\wt\calS}(\widehat f) + \left(\frac{1}{\sqrt{2}} + 1.5 \right) \sqrt{\frac{\log(4/\delta)}{m}} + \sqrt{\frac{4}{\delta}\left(\frac{1}{m} +\frac{3\beta}{m+n} \right)}  \,.
%     \end{align}
    

       
% \end{proof}

% \subsection{Discussion on \codref{cond:hypothesis_stability}}

% The quantity in LHS of \codref{cond:hypothesis_stability} measures how much the function learned by the algorithm (in terms of error on unseen point) will change when one point in the training set is removed. 
% % Discussion on exponential concentration and stronger condition. 
% Notice that hypothesis stability implies error stability, i.e., \codref{cond:error_stability} ~\cite{bousquet2002stability}.  In summary, while error stability allowed us to relate the average population error of the leave-one-out classifiers with the population error of the original classifier, we need hypothesis stability condition to control the variance of the empirical leave-one-out error. 

% Additionally, we note that while the dominating term in the RHS of \thmref{thm:new_linear} matches with the dominating term in ERM bound in \thmref{thm:error_ERM}, there is a polynomial concentration term (dependence on $1/\delta$ instead of $\log(\sqrt{1/\delta})$) in  \thmref{thm:new_linear}. 
% Since with hypothesis stability, we just bound the variance,  the polynomial concentration is due to the use of Chebyshev's inequality instead of an exponential tail inequality (as in \lemref{lem:fit_mislabeled}).
% Recent works have highlighted that slightly stronger condition than hypothesis stability can be used to obtained an exponential concentration for leave-one-out error~\citep{abou2019exponential}, but we leave this for future work for now. 
% % We leave 
% % However, the constants 

% % we also want to highlight  

% \subsection{Formal statement and proof of  of \propref{prop:early_stop}}

% Before formally presenting the result, we will introduce some notation.  By $\calL_{S}(w)$, we denote 
% the objective in \eqref{eq:l2_MSE_app} with $\lambda=0$. 
% Assume Singular Value Decomposition (SVD) of $\bX$  as $\sqrt{n} \bU \bS^{1/2} \bV^T$. Hence $\bX^T \bX = \bV \bS \bV^T$.
% Consider the GD iterates defined in \eqref{eq:GD_iterates_app}. 
% % 
% We now derive closed form expression for the $t^\text{th}$ iterate of gradient descent:  
% % 
% \begin{align}
%     w_t = w_{t-1} + \eta \cdot \bX^T (\by - \bX w_{t-1}) = (\bI - \eta \bV \bS \bV^T )w_{k-1} + \eta \bX^T \by \,.
% \end{align}
% Rotating by $\bV^T$, we get 
% \begin{align}
%     \wt w_t = (\bI - \eta\bS )\wt w_{k-1} + \eta \wt \by \,, \label{eq:GD_recur}
% \end{align}
% where $\wt w_t = \bV^T w_t $ and $\wt \by = \bV^T \bX^T \by$. Assuming the initial point $w_0 = 0$ and applying the recursion in \eqref{eq:GD_recur}, we get
% \begin{align}
%     \wt w_t = \bS ^{-1} ( \bI - (\bI - \eta \bS)^k ) \wt \by \,, 
% \end{align} 
% Projecting solution back to the original space, we have 
% \begin{align}
%      w_t = \bV \bS ^{-1} ( \bI - (\bI - \eta \bS)^k ) \bV^T \bX^T \by \,, 
% \end{align} 
% % We will work with this GD solution at any iterate $t$ in the next proposition. 
% Define $f_t(x) \defeq f(x;w_t)$ as the solution at the $t^{\text{th}}$ iterate. 
% Let $\wt w_{\lambda} = \argmin_{w} \calL_\calS (w;\lambda) = (\bX^T \bX + \lambda \bI)^{-1} \bX^T \by = \bV (\bS + \lambda \bI )^{-1} \bV^T \bX^T \by $. 
% % ) \,,$ for all $t=1,2,\ldots\,.$ 
% and define $\wt f_\lambda(x) \defeq f(x;\wt w_\lambda)$ as the regularized solution. 
% Assume $\kappa$ be the condition number of the population covariance matrix 
% and 
% let $s_\text{min}$ be the minimum positive singular value of the empirical covariance matrix. Our proof idea is inspired from recent work on relating gradient flow solution and regularized solution for regression problems \citep{ali2018continuous}. We will use the following lemma in the proof: 
% \begin{lemma} \label{lem:ineq_soln}
%     For all $x \in [0,1]$ and for all $ k \in \mathbb{N}$, we have (a) $ \frac{kx}{1+kx} \le 1- (1-x)^k$ and (b) $ 1- (1-x)^k \le 2 \cdot \frac{kx}{kx+1} $.
%     %  where $g(c)$ is a constant dependent on $c$. For $c = 1$, $g(c) = 2.0$.   
% \end{lemma}
% \begin{proof}
%     % [Proof of \lemref{lem:ineq_soln}]
%     % Part (a) is easy. 
%     Using $ (1-x)^k \le \frac{1}{1+kx}$, we have part (a). For part (b), we numerically maximize $\frac{ (1+kx ) (1 - (1-x)^k) }{kx}$ for all $k\ge 1$ and for all $x \in [0, 1]$.  
% \end{proof}

% % 
% % Next, 

% \begin{prop}[Formal statement of \propref{prop:early_stop}] \label{prop:formal_early_stop}
% Let $\lambda = \frac{1}{t\eta}$. For a training point $x$, we have 
% \begin{align*}
%     \Expt{x \sim \calS}{(f_t(x) - \wt f_\lambda(x))^2} &\le c(t,\eta) \cdot \Expt{x \sim \calS}{f_t(x)^2} \,, %\label{eq:early_stop}
% \end{align*}
% where $c(t, \eta) \defeq \min( 0.25, \frac{1}{s_\text{min}^2 t^2 \eta^2})$. Similarly for a test point, we have 
% \begin{align*}
%     \Expt{x \sim \calD_\calX}{(f_t(x) - \wt f_\lambda(x))^2} &\le \kappa \cdot c(t,\eta) \cdot \Expt{x \sim \calD_\calX}{f_t(x)^2} \,. %\label{eq:early_stop}
% \end{align*}
% \end{prop} 

% \begin{proof}
%     %%%%%%%%%%%%% 
%     We want to analyze the expected squared difference output of regularized linear regression with regularization constant $\lambda = \frac{1}{\eta t}$ and gradient descent solution at $t^\text{th}$ iterate. We separately expand the algebraic expression for squared difference at a training point and a test point. 
%     % We start by considering the difference  
%     Then the main step is to show that  $\left[ \bS ^{-1} ( \bI - (\bI - \eta \bS)^k )  - (\bS + \lambda \bI )^{-1}\right] \preceq c(\eta, t) \cdot \bS ^{-1} ( \bI - (\bI - \eta \bS)^k ) $.

%     %%%%%%%%%%%%%
    
%   \textbf{Part 1 {} {}} 
%     First, we will analyze the squared difference of output at a training point (for simplicity, we refer to $S \cup \wt S$ as $S$), i.e. 
%     \begin{align}
%         \Expt{ x \sim \calS }{\left(f_t(x) - \wt f_\lambda (x)\right)^2} &= \norm{\bX w_t - \bX \wt w_\lambda}{2}^2 =   \norm{\bX \bV \bS ^{-1} ( \bI - (\bI - \eta \bS)^t ) \bV^T \bX^T \by - \bX \bV (\bS + \lambda \bI )^{-1} \bV^T \bX^T \by }{2}^2 \\
%         &= \norm{\bX \bV \left(\bS ^{-1} ( \bI - (\bI - \eta \bS)^t ) - (\bS + \lambda \bI )^{-1} \right) \bV^T \bX^T \by  }{2} \\
%         &=  \by^T \bV \bX \left( \underbrace{\bS ^{-1} ( \bI - (\bI - \eta \bS)^t ) - (\bS + \lambda \bI )^{-1}}_{\RN{1}} \right)^2 \bS \bV^T \bX^T \by \label{eq:train_GD_rel}
%         %  (\bX \bV \bS ^{-1} ( \bI - (\bI - \eta \bS)^k ) \bV^T \bX^T \by)^T \bX \bV \bS ^{-1} ( \bI - (\bI - \eta \bS)^k ) \bV^T \bX^T \by
%     \end{align}
%     We now separately consider term 1. Substituting $\lambda = \frac{1}{t \eta}$, we get
%     \begin{align}
%         \bS ^{-1} ( \bI - (\bI - \eta \bS)^t ) - (\bS + \lambda \bI )^{-1} &= \bS^{-1} \left( ( \bI - (\bI - \eta \bS)^t ) - (\bI + \bS^{-1} \lambda )^{-1}\right) \\
%         &= \underbrace{\bS^{-1} \left( ( \bI - (\bI - \eta \bS)^t ) - (\bI + ( \bS t \eta)^{-1}  )^{-1}\right)}_{\bA}
%     \end{align}

%     We now separately bound the diagonal entries in matrix $\bA$. 
%     With $s_i$, we denote $i^{\text{th}}$ diagonal entry of $\bS$. Note that since $ \eta\le 1/\norm{S}{\text{op}}$, for all $i$, $\eta s_i  \le 1$.  Consider $i^{\text{th}}$ diagonal term (which is non-zero) of the diagonal matrix $\bA$, we have 
%     \begin{align}
%         \bA_{ii} = \frac{1}{s_i} \left(  1 - (1 - s_i \eta)^t - \frac{t \eta s_i}{1 + t \eta s_i } \right) &=  \frac{1 - (1 - s_i \eta)^t}{s_i} \left( \underbrace{ 1 - \frac{t \eta s_i}{(1 + t \eta s_i)(1 - (1 - s_i \eta)^t)}}_{\RN{2}} \right) \\ 
%          &\le \frac{1}{2}\left[ \frac{1 - (1 - s_i \eta)^t}{ s_i} \right] \tag*{(Using \lemref{lem:ineq_soln} (b))} \,.
%     \end{align} 
%     Additionally, we can also show the following upper bound on term 2: 
%     \begin{align}
%          1 - \frac{t \eta s_i}{(1 + t \eta s_i)(1 - (1 - s_i \eta)^t)} &= \frac{(1 + t \eta s_i)(1 - (1 - s_i \eta)^t) - t \eta s_i }{(1 + t \eta s_i)(1 - (1 - s_i \eta)^t)} \\
%          & \le  \frac{ 1 -  (1 - s_i \eta)^t - t \eta s_i (1 - s_i \eta)^t}{(1 + t \eta s_i)(1 - (1 - s_i \eta)^t)} \\
%          & \le \frac{1}{t\eta s_i} \,. \tag{Using \lemref{lem:ineq_soln} (a)}
%         %  &\le \frac{1}{2}\left[ \frac{1 - (1 - s_i \eta)^t}{ s_i} \right] \tag*{(Using \lemref{lem:ineq_soln})} \,.
%     \end{align} 

%     Combining both the upper bounds on each diagonal entry $\bA_{ii}$, we have 
%     \begin{align}
%     \bA \preceq c_1(\eta, t) \cdot \bS^{-1} ( \bI - (\bI - \eta \bS)^t ) \,, \label{eq:upperbound_diagonal}
%     \end{align}
%     where $c_1(\eta, t ) = \min(0.5, \frac{1}{t s_i \eta })$. Plugging this into \eqref{eq:train_GD_rel}, we have 
%     \begin{align}
%         \Expt{ x \sim \calS }{\left(f_t(x) - \wt f_\lambda (x)\right)^2} &\le c(\eta, t) \cdot \by^T \bV \bX  \left( \bS^{-1} ( \bI - (\bI - \eta \bS)^t ) \right)^2 \bS \bV^T \bX^T \by \\
%         &=   c(\eta, t) \cdot \by^T \bV \bX  \left( \bS^{-1} ( \bI - (\bI - \eta \bS)^t ) \right) \bS \left( \bS^{-1} ( \bI - (\bI - \eta \bS)^t ) \right) \bV^T \bX^T \by \\
%         & =  c(\eta, t) \cdot \norm{\bX w_t}{2}^2 \\
%         &= c(\eta, t) \cdot  \Expt{ x \sim \calS }{\left(f_t(x) \right)^2} \,,
%     \end{align}
%     where $c(\eta, t ) = \min(0.25, \frac{1}{t^2 s^2_i \eta^2 })$.

%     \textbf{Part 2 {} {}} With $\bSigma$, we denote the underlying true covariance matrix. We now consider the squared difference of output at an unseen point: 
%     \begin{align}
%         \Expt{ x \sim \calD_{\calX} }{\left(f_t(x) - \wt f_\lambda (x)\right)^2} &= \Expt{x \sim \calD_{\calX}}{\norm{x^T w_t - x^T \wt w_\lambda}{2}} \\
%         &=   \norm{x^T \bV \bS ^{-1} ( \bI - (\bI - \eta \bS)^t ) \bV^T \bX^T \by - x^T \bV (\bS + \lambda \bI )^{-1} \bV^T \bX^T \by }{2} \\
%         &= \norm{x^T \bV \left(\bS ^{-1} ( \bI - (\bI - \eta \bS)^t ) - (\bS + \lambda \bI )^{-1} \right) \bV^T \bX^T \by  }{2} \\
%         &= \by^T \bV \bX \left( \bS ^{-1} ( \bI - (\bI - \eta \bS)^t ) - (\bS + \lambda \bI )^{-1} \right) \bV^T \bSigma \bV \\ &\qquad \qquad \qquad \qquad \qquad \left( (\bI - (\bI - \eta \bS)^t ) - (\bS + \lambda \bI )^{-1} \right) \bV^T \bX^T \by \\
%         &\le \sigma_{\text{max}} \cdot \by^T \bV \bX \left( \underbrace{\bS ^{-1} ( \bI - (\bI - \eta \bS)^t ) - (\bS + \lambda \bI )^{-1}}_{\RN{1}} \right)^2 \bV^T \bX^T \by \,, \label{eq:test_GD_rel}
%         %  (\bX \bV \bS ^{-1} ( \bI - (\bI - \eta \bS)^k ) \bV^T \bX^T \by)^T \bX \bV \bS ^{-1} ( \bI - (\bI - \eta \bS)^k ) \bV^T \bX^T \by
%     \end{align}
%     where $\sigma_{\text{max}}$ is the maximum eigenvalue of the underlying covariance matrix $\bSigma$. Using the upper bound on term 1 in \eqref{eq:upperbound_diagonal}, we have 
%     \begin{align}
%         \Expt{ x \sim \calD_{\calX} }{\left(f_t(x) - \wt f_\lambda (x)\right)^2} &\le \sigma_{\text{max}} \cdot c(\eta, t) \cdot \by^T \bV \bX  \left( \bS^{-1} ( \bI - (\bI - \eta \bS)^t ) \right)^2 \bV^T \bX^T \by \\
%         &=   \kappa \cdot c(\eta, t) \cdot \sigma_{\text{min}}\cdot \norm{\bV \left( \bS^{-1} ( \bI - (\bI - \eta \bS)^t ) \right) \bV^T \bX^T \by}{2}^2 \\
%         &\le \kappa \cdot c(\eta, t) \cdot \left[ \bV \left( \bS^{-1} ( \bI - (\bI - \eta \bS)^t ) \right) \bV^T \bX^T \right]^T \bSigma \\
%         &\qquad \qquad \qquad \qquad \qquad \left[ \bV \left( \bS^{-1} ( \bI - (\bI - \eta \bS)^t ) \right) \bV^T \bX^T \right] \by \\
%         & = \kappa \cdot c(\eta, t) \cdot \Expt{x \sim \calD_{\calX}}{\norm{x^T w_t}{2}} \,.
%     \end{align}
% % 
% % 
%     % Since $ \eta\le 1/\norm{S}{\text{op}}$, invoking \lemref{lem:ineq_soln} to upper bound term 1 with
% \end{proof}


% \newpage
% \section{Additional experiments and details}\label{app:exp}
% \newcommand\tab[1][1cm]{\hspace*{#1}}

% \subsection{Datasets} \label{sec:app_dataset}

% \textbf{Toy Dataset {} {}} Assume fixed constants $\mu$ and $\sigma$. For a given label $y$, we simulate features $x$ in our toy classification setup as follows: 
% \begin{align*}
%     x \defeq \texttt{concat} \left[ x_1, x_2\right] \quad \text{where} \quad  x_1 \sim  \calN( y \cdot \mu, \sigma^2 I_{d \times d}) \ \  \text{and} \ \  x_1 \sim  \calN( 0, \sigma^2 I_{d \times d}) \,.
% \end{align*}  
% % where $y$ is the true label and $x$ is the corresponding feature vector. 
% In experiements throughout the paper, we fix dimention $d=100$, $\mu = 1.0 $, and $\sigma = \sqrt{d}$. Intuitively, $x_1$ carries the information about the underlying label and $x_2$ is additional noise independent of the underlying label. 

% \textbf{CV datasets {} {}} We use MNIST~\citep{lecun1998mnist} and CIFAR10~\cite{krizhevsky2009learning}. 
% % For binary tasks, 
% We produce a binary variant from the multiclass classification problem by mapping classes $\{0,1,2,3,4\}$ to label $1$ and $\{ 5,6,7,8,9\}$ to label $-1$. For CIFAR dataset, we also use the standard data augementation of random crop and horizontal flip. PyTorch code is as follows: 

% \texttt{(transforms.RandomCrop(32, padding=4),\\
% \tab transforms.RandomHorizontalFlip())}

% \textbf{NLP dataset {} {}} We use IMDb Sentiment analysis~\citep{maas2011learning} corpus.  

% \subsection{Architecture Details} 

% All experiments were run on NVIDIA GeForce RTX 2080 Ti GPUs. We used PyTorch~\citep{NEURIPS2019a9015} and Keras with Tensorflow~\citep{abadi2016tensorflow} backend for experiments. 
% % , ELMo embeddings~\citep{Peters:2018}, and Hugging Face Transformers~\citep{wolf-etal-2020-transformers}. 

% \textbf{Linear model {} {}} For the toy dataset, we simulate a linear model with scalar output and the same number of parameters as the number of dimensions.   

% \textbf{Wide nets {} {}} To simulate the NTK regime, we experiment with $2-$layered wide nets. The PyTorch code for 2-layer wide MLP is as follows: 


% \texttt{ nn.Sequential( \\
% \tab     nn.Flatten(),\\
% \tab    nn.Linear(input\_dims, 200000, bias=True),\\
% \tab    nn.ReLU(),\\
% \tab    nn.Linear(200000, 1, bias=True)\\
% \tab     )}


% We experiment both (i) with the first layer fixed at random initialization; (ii)  and updating both layers' weights.     

% \textbf{Deep nets for CV tasks {} {}} We consider a 4-layered MLP. The PyTorch code for 4-layer MLP is as follows: 

% \texttt{ nn.Sequential(nn.Flatten(), \\
% \tab        nn.Linear(input\_dim, 5000, bias=True),\\
% \tab        nn.ReLU(),\\
% \tab        nn.Linear(5000, 5000, bias=True),\\
% \tab        nn.ReLU(),\\
% \tab        nn.Linear(5000, 5000, bias=True),\\
% \tab        nn.ReLU(),\\
% % \tab        nn.Linear(5000, 5000, bias=True),\\
% % \tab        nn.ReLU(),\\
% \tab        nn.Linear(1024, num\_label, bias=True)\\
% \tab        )}

% For MNIST, we use $1000$ nodes instead of $5000$ nodes in the hidden layer. 
% % 
% We also experiment with convolutional nets. In particular, we use ResNet18 \citep{he2016deep}. Implementation adapted from:  \url{https://github.com/kuangliu/pytorch-cifar.git}. 

% \textbf{Deep nets for NLP {} {}} We use a simple LSTM model with embeddings intialized with ELMo embeddings~\citep{Peters:2018}. Code adapted from: \url{https://github.com/kamujun/elmo_experiments/blob/master/elmo_experiment/notebooks/elmo_text_classification_on_imdb.ipynb} 

% We also evaluate our bounds with a BERT model. In particular, we fine-tune an off-the-shelf uncased BERT model~\citep{devlin2018bert}. Code adapted from Hugging Face Transformers~\citep{wolf-etal-2020-transformers}: \url{https://huggingface.co/transformers/v3.1.0/custom_datasets.html}. 


% \subsection{Additonal experiments}

% 1. SGD with linear models on cross entropy and MSE loss. 

% 2. CE loss and SGD. GD with MSE loss 

% 3. Binary MNIST with MLP. multiclass MNIST  

% \textbf{Results on CIFAR 10 {} {}} 
% % 
% We plot epoch wise error curve for results in \tabref{table:multiclass}. We observe the same trend as in \figref{fig:error_CIFAR10}. Additionally, we plot an \emph{oracle bound} obtained by tracking the error on mislabeled data which nevertheless were predicted as true label. To obtain an exact emprical value of the oracle bound, we need underlying true labels for the randomly labeled data. 
% % Note that our bound in \thmref{thm:multiclass_ERM}, lower bounds the accuracy as predicted by the oracle bound. 
% While with just access to extra unlabeled data we cannot calculate oracle bound, we note that the oracle bound is very tight and never violated in practice underscoring an importamt aspect of generalization in multiclass problems. This highlight that even a stronger conjecture may hold in multiclass classification, i.e., error on mislabeled data (where nevertheless true label was predicted) lower bounds the population error on the distribution of mislabeled data and hence, the error on (a specific) mislabeled portion predicts the population accuracy on clean data. 
% % 
% On the other hand, the dominating term of in \thmref{thm:multiclass_ERM} is loose when compared with the oracle bound. The main reason, we believe is the pessimistic upper bound in \eqref{eq:lemma1_final_multi_prev} in the proof of \lemref{lem:fit_mislabeled_multi}. We leave an investigation on this gap for future. 
% % of fit 

% % However, oracle bound highlights two . One,  



% \begin{figure}[h]
%     \centering 
%     % \vspace{-15pt}
%     % \includegraphics[width=0.9\linewidth]{example-image-a}
%     \includegraphics[width=0.4\linewidth]{figures/CIFAR10-FNN.pdf} \hfil
%     \includegraphics[width=0.4\linewidth]{figures/CIFAR10-Resnet.pdf}
%     % \includegraphics[width=0.9\linewidth]{figures/{CIFAR10_rn=0.1_lr=0.2_wd=0.005}.png}
%     % \vspace{-10pt}
%     \caption{ Per epoch curves for CIFAR10 corresponding results in \tabref{table:multiclass}. As before, we just plot the dominating term in the RHS of \eqref{eq:multiclass_ERM} as predicted bound. Additionally, we also plot the predicted lower bound by the error on mislabeled data which nevertheless were predicted as true label. We refer to this as ``Oracle bound''. See text for more details. 
%     % 
%     % except for the stopping point. 
%     % The bound predicted by RATT (RHS in \eqref{eq:multiclass_ERM}) is vacuous. 
%     }\label{fig:error_epoch_CIFAR10}
%     % \vspace{-15pt}
% \end{figure}


% \textbf{Results on CIFAR 100 {} {}} 
% % 
% On CIFAR100, our bound in \eqref{eq:multiclass_ERM} yields vacous bounds. However, the oracle bound as explained above yields tight guarantees in the initial phase of the learning (i.e., when learning rate is less than $0.1$). 

% \begin{figure}[h]
%     \centering 
%     % \vspace{-15pt}
%     % \includegraphics[width=0.9\linewidth]{example-image-a}
%     \includegraphics[width=0.4\linewidth]{figures/CIFAR100-Resnet.pdf}
%     % \includegraphics[width=0.9\linewidth]{figures/{CIFAR10_rn=0.1_lr=0.2_wd=0.005}.png}
%     % \vspace{-10pt}
%     \caption{ Predicted lower bound by the error on mislabeled data which nevertheless were predicted as true label with ResNet18 on CIFAR100. We refer to this as ``Oracle bound''. See text for more details. 
%     % 
%     % except for the stopping point. 
%     The bound predicted by RATT (RHS in \eqref{eq:multiclass_ERM}) is vacuous. 
%     }\label{fig:error_CIFAR100}
%     % \vspace{-15pt}
% \end{figure}


% % \paragraph{Experiments on CIFAR100} 



% \subsection{Hyperparameter Details}


% \textbf{\figref{fig:error_CIFAR10} {} {}} We use clean training dataset of size $40,000$. We fix the amount of unlabeled data at $20\%$ of the clean size, i.e. we include additional $8,000$ points with randomly assigned labels. We use test set of $10,000$ points. For both MLP and ResNet, we use SGD with an initial learning rate of $0.1$ and momentum $0.9$. We fix the weight decay parameter at $5\times 10^{-4}$. After $100$ epochs, we decay the learning rate to $0.01$. We use SGD batch size of $100$. 

% \textbf{\figref{fig:error_binary} (a) {} {}} We obtain a toy dataset according to the process described in \secref{sec:app_dataset}. We fix $d=100$ and create a dataset of $50,000$ points with balanced classes. Moreover, we sample additional covariates with the same procedure to create randomly labeled dataset. For both SGD and GD training, we use a fixed learning rate $0.1$.    

% \textbf{\figref{fig:error_binary} (b) {} {}} Similar to binary CIFAR, we use clean training dataset of size $40,000$ and fix the amount of unlabeled data at $20\%$ of the clean dataset size. To train wide nets, we use a fixed learning of $0.001$ with GD and SGD. We decide the weight decay parameter and the early stopping point that maximizes our generalization bound (i.e. without peeking at unseen data ).  We use SGD batch size of $100$. 

% \textbf{\figref{fig:error_binary} (c) {} {}} With IMDb dataset, we use a clean dataset of size $20,000$ and as before, fix the amount of unlabeled data at $20\%$ of the clean data. To train ELMo model, we use Adam optimizer with a fixed learning rate $0.01$ and weight decay $10^{-6}$ to minimize cross entropy loss. We train with batch size $32$ for 3 epochs. To fine-tune BERT model, we use Adam optimizer with learning rate $5\times 10^{-5}$ to minimize cross entropy loss. We train with a batch size of $16$ for 1 epoch.    

% \textbf{\tabref{table:multiclass} {} {}} For multiclass datasets, we train both MLP and ResNet with the same hyperparameters as described before. We sample a clean training dataset of size $40,000$ and fix the amount of unlabeled data at $20\%$ of the clean size. We use SGD with an initial learning rate of $0.1$ and momentum $0.9$. We fix the weight decay parameter at $5\times 10^{-4}$. After $30$ epochs for ResNet and after $50$ epochs for MLP, we decay the learning rate to $0.01$.  We use SGD with batch size $100$. 
% For \figref{fig:error_CIFAR100}, we use the same hyperparameters as 
% CIFAR10 training, except we now decay learning rate after $100$ epochs. 


% In all experiments, to identify the best possible accuracy on just the clean data, we use the exact same set of hyperparamters except the stopping point. We choose a stopping point that maximizes test performance. 

% \subsection{Summary of experiments }

% \begin{center}
%     \begin{table}[H] 
%         \centering
%         \begin{tabular}{|c|c|c|c|} 
%         \hline
%         Classification type & Model category & Model & Dataset  \\ [0.5ex] 
%         \hline
%         \hline
%         \multirow{9}{*}{Binary} & Low dimensional & Linear model & Toy Gaussain dataset  \\
%                         \cline{2-4}
%                          & \multirow{1}{*}{Overparameterized linear nets} 
%                         %  & Linear model & Toy Gaussain dataset \\
%                         %  \cline{3-4}
%                         %  & & 2-layer wide net& Toy Gaussain dataset \\
%                         %  \cline{3-4}
%                          & 2-layer wide net & Binary MNIST \\
%                          \cline{2-4}                 
%                          & \multirow{6}{*}{Deep nets} & \multirow{2}{*}{MLP} & Binary MNIST \\
%                          \cline{4-4}
%                          & &  & Binary CIFAR \\
%                          \cline{3-4}
%                          &  & \multirow{2}{*}{ResNet} & Binary MNIST \\
%                          \cline{4-4}
%                          & &  & Binary CIFAR \\
%                          \cline{3-4}
%                          &  & ELMo-LSTM model & IMDb Sentiment Analysis \\
%                          \cline{3-4}
%                          & & BERT pre-trained model & IMDb Sentiment Analysis \\
%         \hline
%         \multirow{5}{*}{Multiclass} & \multirow{5}{*}{Deep nets} & \multirow{2}{*}{MLP} & MNIST \\
%                         \cline{4-4} 
%                         & & & CIFAR10 \\                   
%                         \cline{3-4}
%                          &   & \multirow{3}{*}{ResNet} & MNIST \\
%                          \cline{4-4}
%                          &   & & CIFAR10 \\
%                          \cline{4-4}
%                          &   & & CIFAR100 \\
%         \hline
%         \end{tabular}
%         % \caption{Summary of experiments performed} \label{table:experiments}
%     \end{table}    
%     % \footnotetext[6]{We use both MSE loss and cross-entropy loss.}
%     % \footnotetext[6]{We try 2 variants: one with a fixed first layer and the other with both layers trainable.}
% \end{center}

% \newpage
% \section{Proof of \lemref{lem:stability_error}} \label{app:proof_lem_error}

% \begin{proof}[Proof of \lemref{lem:stability_error}]
%     Recall, we have a training set $S \cup \wt S_C$. We defined leave-one-out error on mislabeled points as $$\error_{\text{LOO}(\wt S_M) } = \frac{\sum_{(x_i, y_i) \in \wt S_M} \error( f_{(i)}( x_i), y_i)}{ \abs{\wt S_M }} \,, $$
%     where $f_{(i)} \defeq f(\calA, (S \cup \wt S)_{(i)})$. Define $S^\prime \defeq S \cup \wt S$. Assume $(x,y)$ and $(x^\prime,y^\prime)$ as i.i.d. samples from ${\calDm}$. 
%     Using Lemma 25 in \citet{bousquet2002stability}, we have
%     \begin{align*}
%         \Expo{ \left( \error_{\calDm}(\wh f) -\error_{\text{LOO}(\wt S_M) } \right)^2 } \le & \Expt{ S^\prime, (x,y), (x^\prime,y^\prime) }{ \error(\wh f(x), y ) \error(\wh f(x^\prime), y^\prime )} - 2 \Expt{ S^\prime, (x,y) }{ \error(\wh f(x), y ) \error(f_{(i)}(x_i), y_i )} \\
%         & + \frac{m_1-1}{m_1}\Expt{ S^\prime }{  \error(f_{(i)}(x_i), y_i )  \error(f_{(j)}(x_j), y_j )} + \frac{1}{m_1} \Expt{ S^\prime }{  \error(f_{(i)}(x_i), y_i ) } \,. \numberthis \label{eq:main_reln}
%     \end{align*}
%     We can rewrite the equation above as : 
%     \begin{align*}
%         \Expo{ \left( \error_{\calDm}(\wh f) -\error_{\text{LOO}(\wt S_M) } \right)^2 } \le &  \, \underbrace{\Expt{ S^\prime, (x,y), (x^\prime,y^\prime) }{ \error(\wh f(x), y ) \error(\wh f(x^\prime), y^\prime ) - \error(\wh f(x), y ) \error(f_{(i)}(x_i), y_i )}}_{\RN{1}} \\
%         & + \underbrace{\Expt{ S^\prime }{  \error(f_{(i)}(x_i), y_i )  \error(f_{(j)}(x_j), y_j ) -  \error(\wh f(x), y ) \error(f_{(i)}(x_i), y_i )}}_{\RN{2}} \\ &+ \underbrace{\frac{1}{m_1} \Expt{ S^\prime }{  \error(f_{(i)}(x_i), y_i ) - \error(f_{(i)}(x_i), y_i )  \error(f_{(j)}(x_j), y_j ) }}_{\RN{3}} \,. \numberthis \label{eq:main_reln2}
%     \end{align*}
    
%     We will now bound term $\RN{3}$.  Using Schwarz's inequality, we have
    
%     \begin{align}
%         \Expt{ S^\prime }{  \error(f_{(i)}(x_i), y_i ) - \error(f_{(i)}(x_i), y_i )  \error(f_{(j)}(x_j), y_j ) }^2 &\le  \Expt{ S^\prime }{  \error(f_{(i)}(x_i), y_i ) }^2 \Expt{S^\prime}{1 -   \error(f_{(j)}(x_j), y_j ) }^2 \\
%         &\le \frac{1}{4} \label{eq:term1_lem12}
%     \end{align}
    
%     Note that since $(x_i,y_i)$, $(x_j ,y_j )$, $(x,y)$, and $(x^\prime, y^\prime)$ are all from same distribution $\calDm$, we directly incorporate the bounds on term $\RN{1}$ and $\RN{2}$ from proof of Lemma 9 in \citet{bousquet2002stability}. Combining that with \eqref{eq:term1_lem12} and our definition of hypothesis stability in \codref{cond:hypothesis_stability}, we have the required claim. 
    
    
%     % We now re-write term $\RN{1}$ as
%     % \begin{align*}
%     %         &\Expt{S^\prime, (x,y), (x^\prime,y^\prime) }{ \error(\wh f(x), y ) \error(\wh f(x^\prime), y^\prime ) - \error(\wh f(x), y ) \error(f_{(i)}(x_i), y_i )} \\ & \qquad = \Expt{ S^\prime, (x,y), (x^\prime,y^\prime) }{ \error(\wh f(x), y ) \error(\wh f  (x^\prime), y^\prime ) - \error(\wh f ^\prime(x), y ) \error(f_{(i)}(x^\prime), y^\prime )} \tag{Exchanging $(x_i, y_i)$ with $(x^\prime, y^\prime)$ in the second term} \\
%     %         & \qquad = \Expt{ S^\prime, (x,y), (x^\prime,y^\prime) }{  \left(\error(\wh f(x), y )-  \error(f_{(i)}(x), y ) \right) \error(\wh f  (x^\prime), y^\prime )  } \\
%     %         & \qquad  + \Expt{ S^\prime, (x,y), (x^\prime,y^\prime) }{  \left(\error(f_{(i)}(x), y ) -\error(\wh f ^\prime(x), y ) \right) \error(\wh f  (x^\prime), y^\prime )}  \\
%     %         & \qquad +\Expt{ S^\prime, (x,y), (x^\prime,y^\prime) }{  \left( \error(\wh f  (x^\prime), y^\prime ) -  \error(f_{(i)}(x^\prime), y^\prime ) \right) \error(\wh f ^\prime(x), y ) }  \,, \numberthis \label{eq:term1_final}
%     % \end{align*}
%     % where $\wh f^\prime$ is the classifier obtained by training on $ S^\prime_{(i)} \cup \{ (x^\prime, y^\prime) \} $. Similarly we can re-write term $\RN{2}$ as 
%     % \begin{align*}
%     %     & \Expt{ S^\prime }{  \error(f_{(i)}(x_i), y_i )  \error(f_{(j)}(x_j), y_j ) -  \error(\wh f(x), y ) \error(f_{(i)}(x_i), y_i )} \\
%     %     &\quad  = \Expt{ S^\prime, (x,y), (x^\prime,y^\prime)}{  \error(f^{\prime\prime}_{(i)}(x), y )  \error(f_{(j)}^{\prime}(x^\prime), y^\prime ) -  \error(\wh f(x), y ) \error(f_{(i)}(x_i), y_i )} \tag{Exchanging $(x_i, y_i)$ with $(x, y)$ and $(x_j, y_j)$ with $(x^\prime, y^\prime)$ in the first term}\\
%     %     &\quad = \Expt{ S^\prime, (x,y), (x^\prime,y^\prime)}{  \error(f^{\prime\prime}_{(j)}(x), y )  \error(f_{(i)}^{\prime}(x^\prime), y^\prime ) -  \error(\wh f^\prime (x), y ) \error(f^\prime_{(j)}(x^\prime), y^\prime )} \tag{Exchanging $(x_i, y_i)$ and $(x_j, y_j)$ and then replacing $(x_j, y_j)$ with $(x^\prime, y^\prime)$ in the second term} \\
%     %     & \quad = \Expt{ S^\prime, (x,y), (x^\prime,y^\prime) }{  \left( \error(f_{(i)}^{\prime}(x^\prime), y^\prime )   -  \error(\wh f^{\prime\prime}  (x^\prime), y^\prime ) \right)  \error(f^{\prime\prime}_{(j)}(x), y )   } \\
%     %     & \quad  + \Expt{ S^\prime, (x,y), (x^\prime,y^\prime) }{  \left( \error(f^{\prime\prime}_{(j)}(x), y )  -\error(\wh f ^\prime(x), y ) \right) \error(\wh f^{\prime\prime}  (x^\prime), y^\prime )  }  \\
%     %     & \quad+ \Expt{ S^\prime, (x,y), (x^\prime,y^\prime) }{  \left( \error(\wh f^{\prime\prime}  (x^\prime), y^\prime )  -  \error(f^\prime_{(j)}(x^\prime), y^\prime ) \right)  \error(\wh f^\prime (x), y ) }   \\
%     %     & \quad = \Expt{ S^\prime, (x,y), (x^\prime,y^\prime) }{  \left( \error(f_{(i)}^{\prime}(x^\prime), y^\prime )   -  \error(\wh f (x^\prime), y^\prime ) \right)  \error(f_{(i)}(x_j), y_j )   } \\
%     %     & \quad  + \Expt{ S^\prime, (x,y), (x^\prime,y^\prime) }{  \left( \error(f^{\prime\prime}_{(j)}(x), y )  -\error(\wh f (x), y ) \right) \error(\wh f^{\prime\prime}  (x_j), y_j )  }  \\
%     %     & \quad+ \Expt{ S^\prime, (x,y), (x^\prime,y^\prime) }{  \left( \error(\wh f^{\prime\prime}  (x^\prime), y^\prime )  -  \error(f^\prime_{(j)}(x^\prime), y^\prime ) \right)  \error(\wh f^\prime (x^\prime), y^\prime ) }  \,, \numberthis \label{eq:term2_final}
%     % \end{align*}
%     % where $f^{\prime\prime}_{(j)}$ is trained on $S^\prime_{(j,i)} \cup {(x,y)}$, $f^{\prime}_{(i)}$ is trained on $S^\prime_{(j,i)} \cup {(x^\prime,y^\prime)}$, and $\wh f^{\prime\prime} $ is trained on $S^\prime_{(j)} \cup {(x,y)}$. Note in the last line we replaced $(x,y)$ by $(x_j, y_j)$ in the first term, replaced $(x^\prime,y^\prime)$ by $(x_j, y_j)$ in the second term and exchanged $(x_i,y_i)$ with $(x_j,y_j)$ and also $(x,y)$ and $(x^\prime, y^\prime)$
    
    
% \end{proof}

\end{document}
\endinput
%%
%% End of file `sample-sigconf.tex'.
