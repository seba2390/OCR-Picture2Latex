\section{Numerical examples} 
\label{sec:num}
For our numerical examples, we neglect the body forces and moments, i.e. $\overline\bbf =\bzero$ and $\overline\bM = \bzero$, and assume isotropic material behavior which can be described by the set of material parameters  $\lambda_\textrm{micro},\mu_\textrm{micro},\lambda_e,\mu_e,\mu_c ,\mu$ and $\Lc$, where $\lambda_*$ and $\mu_*$ denote the Lam\'e coefficients.  Furthermore, we assume $\IL$ as a fourth-order identity tensor $\II$. Throughout all examples, we consider the Cosserat modulus $\mu_c = 0$, cf.  \cite{NefGhiLazMad:2015:trl,Nef:2006:tcc}, leading to the symmetry of the force stress tensor. The simulations presented in this paper are performed within AceGen and AceFEM programs, which are developed and maintained by  Jo\v{z}e Korelc (University of Ljubljana). The interested reader is referred to \cite{KorWri:2016:aofem}.
\ssect{\hspace{-5mm}. Discontinuous solution: an interface between two different materials} 
\label{bvp:1}
In the first boundary value problem (bvp), we consider a rectangular domain $\B$ with length $l=2$ and height $h=1$ which consists of a side-by-side arrangement of two different materials, see Figure \ref{Figure:Geo1}. The bottom edge is fixed in both directions, $\overline\bu=(0,0)^T$, and we apply a displacement to the upper edge, $\overline\bu=(0.01,0.01)^T$, while the left and right edges are subjected to displacement given by $\overline\bu=(0.01y^2,0.01y^2)^T$. For the micro-distortion field, the consistent coupling  boundary condition $\Bdis \cdot \Btau = \nabla \overline\bu \cdot \Btau$ is enforced on the entire boundary $\partial \B$. The material parameters are given in Table \ref{tab:example1:MP}. Due to the different material parameters of the domains, a discontinuous solution in the micro-distortion is expected which allows us to compare and evaluate the behavior of the implemented finite elements.  \\
\begin{figure}[ht]
\center
	\unitlength=1mm
	\begin{picture}(100,60)
	\put(0,0){\def\svgwidth{9.6cm}{\small\input{figures/example1_geo.eps_tex}}}
	\end{picture}
	\caption{2D rectangular bvp with two different materials. Inspection line, see Figures \ref{fig:example1:convergence1} and \ref{fig:example1:convergence2}, can be seen in red color. }
	\label{Figure:Geo1}
\end{figure} 
\begin{table}[ht]
\center
\begin{tabular}{|c|c|}
\hline
Material 1 & Material 2 \\
\hline
\begin{minipage}[t]{7cm}
	$ \lambda_\textrm{micro} = 555.55 , \quad \mu_\textrm{micro} = 833.33$ \\ $\lambda_e = 486.11, \quad \mu_e = 729.17$\\
	$\mu_c = 0, \quad \mu=833.33$ \\ 
	$\IL = \II, \quad \Lc=1$ 
\end{minipage} & 
\begin{minipage}[t]{7cm}
	$ \lambda_\textrm{micro} = 1111.11 , \quad \mu_\textrm{micro} = 1667.67$ \\ $\lambda_e = 972.22, \quad \mu_e = 1458.33$\\
	$\mu_c = 0, \quad \mu=1666.67$ \\
	 $\IL = \II, \quad \Lc=1$ 
\end{minipage} \\
\hline
\end{tabular}
\caption{Material parameters of the first boundary problem, see Figure \ref{Figure:Geo1}.}
\label{tab:example1:MP}
\end{table}
In Figure \ref{fig:example1:solution}, the displacement and micro-distortion fields obtained for a discretization with 1600 N2NT2 finite elements are shown.  Note that the elements solution is plotted in this work without the usual averaging or smoothing on the element edges in order to investigate the possible discontinuities. The solution of the displacement field is continuous through the interface  between the two domains. The tangential components of the micro-distortion $\Bdis \cdot \be_2 = (\dis_{12},\dis_{22})^T$ are continuous on the interface while the normal components $\Bdis \cdot \be_1 = (\dis_{11},\dis_{21})^T$ exhibit discontinuities. 

\begin{figure}[ht]
     \centering
     \begin{subfigure}[b]{0.45\textwidth}
         \centering
         \includegraphics[width=\textwidth]{figures/interface_Q2NQ2_u1.png}
			\caption{$u_1$}
     \end{subfigure}
     \centering
     \begin{subfigure}[b]{0.45\textwidth}
         \centering
         \includegraphics[width=\textwidth]{figures/interface_Q2NQ2_u2.png}
			\caption{$u_2$}
     \end{subfigure}
     \begin{subfigure}[b]{0.45\textwidth}
         \centering
         \includegraphics[width=\textwidth]{figures/interface_Q2NQ2_chi11.png}
			\caption{$\dis_{11}$}
     \end{subfigure}
     \begin{subfigure}[b]{0.45\textwidth}
         \centering
         \includegraphics[width=\textwidth]{figures/interface_Q2NQ2_chi12.png}
			\caption{$\dis_{12}$}
     \end{subfigure}
     \begin{subfigure}[b]{0.45\textwidth}
         \centering
         \includegraphics[width=\textwidth]{figures/interface_Q2NQ2_chi21.png}
			\caption{$\dis_{21}$}
     \end{subfigure}
     \begin{subfigure}[b]{0.45\textwidth}
         \centering
         \includegraphics[width=\textwidth]{figures/interface_Q2NQ2_chi22.png}
			\caption{$\dis_{22}$}
     \end{subfigure}
        \caption{The displacement and the micro-distortion fields of the first boundary problem.}
\label{fig:example1:solution}
\end{figure}

The component  $\dis_{21}$ along the inspection line $y=0.5$ is plotted in Figure \ref{fig:example1:convergence1}   using $H^1(\B) \times H^1(\B)$ elements  resulting in a continuous solution. This causes a transition zone at the interface which needs to be resolved by increasing the mesh density tremendously in order to approximate the discontinuous solution at the interface. Enhancing the approximation space of the micro-distortion field to the second-order Lagrange space does not improve the convergence behavior to the discontinuous  solution at the interface. However, the discontinuous solution of $\dis_{21}$ can be captured by the $H^1(\B) \times H(\curl, \B)$ elements, see Figure \ref{fig:example1:convergence2}. The second-order N\'ed\'elec formulations T2NT2 and Q2NQ2 shows an instant convergence with a coarse mesh while first-order N\'ed\'elec formulations T2NT1 and Q2NQ1 reach a good accuracy with intermediate/coarse mesh showing the expected linear behavior for the component within one element. 

\begin{figure}[ht]
	\unitlength=1mm
	\center
	\unitlength=1mm
	\center		  
	  	  \begin{subfigure}[b]{0.45\textwidth}
         \includegraphics[width=\textwidth]{figures/chiyx_convergence_T2T1.pdf}
         \put(-38,-2){$x$}
            \put(-75,25){$\dis_{21}$}
	\caption{T2T1}
		  \end{subfigure}
	  	  \begin{subfigure}[b]{0.45\textwidth}
         \includegraphics[width=\textwidth]{figures/chiyx_convergence_T2T2.pdf}
         \put(-38,-2){$x$}
            \put(-75,25){$\dis_{21}$}
	\caption{T2T2}
		  \end{subfigure}
		  	\caption{Illustration of $\dis_{21}$ along the inspection line $y=0.5$ using the nodal elements with different mesh densities.}
\label{fig:example1:convergence1}
\end{figure} 

\begin{figure}[ht]
	\unitlength=1mm
	\center		  
	  	  \begin{subfigure}[b]{0.45\textwidth}
         \includegraphics[width=\textwidth]{figures/chiyx_convergence_T2NT1.pdf}
         \put(-38,-2){$x$}
            \put(-75,25){$\dis_{21}$}
	\caption{T2NT1}
		  \end{subfigure}
	  	  \begin{subfigure}[b]{0.45\textwidth}
         \includegraphics[width=\textwidth]{figures/chiyx_convergence_T2NT2.pdf}
         \put(-38,-2){$x$}
            \put(-75,25){$\dis_{21}$}
	\caption{T2NT2}
		  \end{subfigure}
		  	  	  \begin{subfigure}[b]{0.45\textwidth}
         \includegraphics[width=\textwidth]{figures/chiyx_convergence_Q2NQ1.pdf}
         \put(-38,-2){$x$}
            \put(-75,25){$\dis_{21}$}
	\caption{Q2NQ1}
		  \end{subfigure}
		  	  	  \begin{subfigure}[b]{0.45\textwidth}
         \includegraphics[width=\textwidth]{figures/chiyx_convergence_Q2NQ2.pdf}
         \put(-38,-2){$x$}
            \put(-75,25){$\dis_{21}$}
	\caption{Q2NQ2}
		  \end{subfigure}		  
		  	\caption{Illustration of $\dis_{21}$ along the inspection line $y=0.5$ using $H^1(\B) \times H(\curl, \B)$ elements with different mesh densities.}
\label{fig:example1:convergence2}
\end{figure} 


\FloatBarrier

\ssect{\hspace{-5mm}. Characteristic length analysis: pure shear problem} 

We introduce a second boundary value problem, see Figure  \ref{Figure:Geo2}, consisting of a circular domain $\B$ with a radius $r_\textrm{o}= 25$ and a circular hole at its center with a radius $r_\textrm{i} = 2$. We fix the displacement field $ \overline{\bu} = \bzero$ on the inner boundary $\partial \B_\textrm{i}$ and we rotate the outer boundary $\partial \B_\textrm{o}$ counter clockwise with $ \bar{\bu} = (-\frac{\Delta}{r_\textrm{o}} y,  \frac{\Delta}{r_\textrm{o}} x)^T$ where $\Delta = 0.01$. For the micro-distortion field, we apply the consistent coupling  boundary condition  ($\Bdis \cdot \Btau = \nabla  \overline{\bu} \cdot \Btau$)  on all boundaries  $\partial \B = \partial \B_\textrm{i} \cup  \partial \B_\textrm{o} $.   Two different cases are discussed in the following. For case A, a single material is assumed whereas for case B two materials are considered. The second material is located as a ring with an outer radius $r_\textrm{m} = 10$ and an inner radius $r_\textrm{i} = 2$. The material parameters are shown in Table \ref{tab:example2:MP}. For the analysis of the influence of the characteristic length $\Lc$, the characteristic length will be varied.  
\begin{figure}[ht]
\center
\begin{subfigure}[b]{0.49\textwidth}
	\unitlength=1mm
	\begin{picture}(100,70)
	\put(10,0){\def\svgwidth{6cm}{\small\input{figures/example2_geo.eps_tex}}}
	\end{picture}
\label{fig:example2:BVPA}
\caption{case A}
\end{subfigure}
\begin{subfigure}[b]{0.49\textwidth}
	\unitlength=1mm
	\begin{picture}(100,70)
	\put(10,0){\def\svgwidth{6cm}{\small\input{figures/example2_1_geo.eps_tex}}}
	\end{picture}
\label{fig:example2:BVPB}
\caption{case B}
\end{subfigure}
	\caption{ The geometry of the second boundary value problem. }
	\label{Figure:Geo2}
\end{figure} 


\begin{table}[ht]
\center
\begin{tabular}{|c|c|}
\hline
Material 1 & Material 2 \\
\hline
\begin{minipage}[t]{7cm}
	$ \lambda_\textrm{micro} = 555.55 , \quad \mu_\textrm{micro} = 833.33$ \\ $\lambda_e = 486.11, \quad \mu_e = 729.17$\\
	$\mu_c = 0, \quad \mu=833.33$ \\ 
	$\IL = \II, \quad \Lc \in \{0.001, 5, 1000\}$ 
\end{minipage} & 
\begin{minipage}[t]{7cm}
	$ \lambda_\textrm{micro} = 2777.78 , \quad \mu_\textrm{micro} = 4166.67$ \\ $\lambda_e = 2430.555, \quad \mu_e = 3645.85$\\
	$\mu_c = 0, \quad \mu=4166.67$ \\
	 $\IL = \II, \quad \Lc \in \{0.001, 5, 1000\}$ 
\end{minipage} \\
\hline
\end{tabular}
\caption{Material parameters of the second boundary problem, see Figure \ref{Figure:Geo2}.}
\label{tab:example2:MP}
\end{table}

The problem results in a rotationally-symmetric solution where only the shear components ($u_{r,\theta}, u_{\theta,r}, \dis_{r \theta}, \dis_{\theta r} \neq 0$)  are non-vanishing. The convergence behavior of the different elements is investigated for case B and $\Lc = 5$ using three different mesh densities
(410, 3044 and 30620 triangular elements and 448, 3040 and 30256 quadrilateral elements).  Since the micro-distortion field is in $H(\curl,\B)$, the tangential shear component $\dis_{r \theta}$ has to be continuous while the radial shear component $\dis_{\theta r}$ exhibits a jump, see Figure \ref{fig:example2:solution}, where the Q2NQ1 element is used. Similar to Section \ref{bvp:1}, the $H^1(\B) \times H^1(\B)$ elements are unable to capture this discontinuity in $\dis_{\theta r}$, which is shown in Figure \ref{fig:example2:convergence1}. 
 Actually, $H^1(\B) \times H^1(\B)$ elements approximate the discontinuous  $H(\curl,\B)$ solution only when a very fine discretization is used. The discontinuous solution of the micro-distortion field is demonstrated in Figure  \ref{fig:example2:convergence2}  using the  $H^1(\B) \times H(\curl,\B)$ elements. The higher-order N\'ed\'elec formulation in T2NT2 and Q2NQ2 elements exhibit very satisfactory results already with the coarse mesh. 


\begin{figure}[ht]
     \centering
     \begin{subfigure}[b]{0.4\textwidth}
         \centering
         \includegraphics[width=\textwidth]{figures/rings_Q2NQ2_chi_theta_r.png}
			\caption{$\dis_{\theta r}$}
     \end{subfigure}
     \hspace{2 cm}
     \centering
     \begin{subfigure}[b]{0.4\textwidth}
         \centering
         \includegraphics[width=\textwidth]{figures/rings_Q2NQ2_chi_r_theta.png}
			\caption{$\dis_{r \theta}$}
     \end{subfigure}
        \caption{The non-vanishing micro-distortion components of the second boundary problem using 30256 Q2NQ2 elements for $\Lc=5$.}
\label{fig:example2:solution}
\end{figure}

\begin{figure}[ht]
\unitlength=1mm
	\center
	  	  \begin{subfigure}[b]{0.45\textwidth}
      \includegraphics[width=\textwidth]{figures/example2_convergence_T2T1.pdf}
             \put(-35.903717, -1.2734547){\color[rgb]{0,0,0}\makebox(0,0)[lb]{\smash{$r$}}}%
     \put(-37.903717,12.2734547){\color[rgb]{0,0,0}\makebox(0,0)[lb]{\smash{$\dis_{r \theta}$}}}%
 \put(-37.903717,42.2734547){\color[rgb]{0,0,0}\makebox(0,0)[lb]{\smash{$\dis_{\theta r}$}}}%
	\caption{T2T1}
		  \end{subfigure} 
	  \begin{subfigure}[b]{0.45\textwidth}
      \includegraphics[width=\textwidth]{figures/example2_convergence_T2T2.pdf}
             \put(-35.903717, -1.2734547){\color[rgb]{0,0,0}\makebox(0,0)[lb]{\smash{$r$}}}%
     \put(-37.903717,12.2734547){\color[rgb]{0,0,0}\makebox(0,0)[lb]{\smash{$\dis_{r \theta}$}}}%
 \put(-37.903717,42.2734547){\color[rgb]{0,0,0}\makebox(0,0)[lb]{\smash{$\dis_{\theta r}$}}}%
	\caption{T2T2}
		  \end{subfigure}
	\caption{The non-vanishing components of $\bP$ along the radius for the  $H^1(\B) \times H^1(\B)$ elements using three mesh densities and $\Lc=5$.} 
		  	 \label{fig:example2:convergence1}
\end{figure}   

  \begin{figure}[ht]
	\unitlength=1mm
	\center
	  	  \begin{subfigure}[b]{0.45\textwidth}
      \includegraphics[width=\textwidth]{figures/example2_convergence_T2NT1.pdf}
             \put(-39.903717, -1.2734547){\color[rgb]{0,0,0}\makebox(0,0)[lb]{\smash{$r$}}}%
     \put(-37.903717,12.2734547){\color[rgb]{0,0,0}\makebox(0,0)[lb]{\smash{$\dis_{r \theta}$}}}%
 \put(-37.903717,42.2734547){\color[rgb]{0,0,0}\makebox(0,0)[lb]{\smash{$\dis_{\theta r}$}}}%
	\caption{T2NT1}
		  \end{subfigure} 
	  \begin{subfigure}[b]{0.45\textwidth}
      \includegraphics[width=\textwidth]{figures/example2_convergence_T2NT2.pdf}
             \put(-39.903717, -1.2734547){\color[rgb]{0,0,0}\makebox(0,0)[lb]{\smash{$r$}}}%
     \put(-37.903717,12.2734547){\color[rgb]{0,0,0}\makebox(0,0)[lb]{\smash{$\dis_{r \theta}$}}}%
 \put(-37.903717,42.2734547){\color[rgb]{0,0,0}\makebox(0,0)[lb]{\smash{$\dis_{\theta r}$}}}%
	\caption{T2NT2}
		  \end{subfigure}
	  	  \begin{subfigure}[b]{0.45\textwidth}
      \includegraphics[width=\textwidth]{figures/example2_convergence_Q2NQ1.pdf}
             \put(-39.903717, -1.2734547){\color[rgb]{0,0,0}\makebox(0,0)[lb]{\smash{$r$}}}%
     \put(-37.903717,12.2734547){\color[rgb]{0,0,0}\makebox(0,0)[lb]{\smash{$\dis_{r \theta}$}}}%
 \put(-37.903717,42.2734547){\color[rgb]{0,0,0}\makebox(0,0)[lb]{\smash{$\dis_{\theta r}$}}}%
	\caption{Q2NQ1}
		  \end{subfigure} 
	  \begin{subfigure}[b]{0.45\textwidth}
      \includegraphics[width=\textwidth]{figures/example2_convergence_Q2NQ2.pdf}
             \put(-39.903717, -1.2734547){\color[rgb]{0,0,0}\makebox(0,0)[lb]{\smash{$r$}}}%
     \put(-37.903717,12.2734547){\color[rgb]{0,0,0}\makebox(0,0)[lb]{\smash{$\dis_{r \theta}$}}}%
 \put(-37.903717,42.2734547){\color[rgb]{0,0,0}\makebox(0,0)[lb]{\smash{$\dis_{\theta r}$}}}%
	\caption{Q2NQ2}
		  \end{subfigure}
	\caption{The non-vanishing components of $\bP$ along the radius for $H^1(\B) \times H(\curl,\B)$ elements using three mesh densities and $\Lc=5$.} 
	\label{fig:example2:convergence2}
\end{figure}   
In the following, we will analyze the influence of a variation of the length-scale parameter $\Lc$ on the response of the relaxed micromorphic model for case A. The relation of the relaxed micromorphic model to the classical Cauchy theory has been discussed in detail in \cite{NefEidMad:2019:ios,BarMadDagAbrGhiNef:2017:taf} for the limiting case $\Lc \rightarrow 0$ and $\Lc \rightarrow \infty$. The case $\Lc \rightarrow 0$ relates to a macroscopic view on the material with microstructure,  with the relaxed micromorphic model being equivalent to a linear elasticity model with stiffness $\Cmacro$ defined as the Reuss lower-bound of $\Ce$ and $\Cmicro$, i.e. $\Cmacro := (\Ce^{-1} + \Cmicro^{-1})^{-1}$.  The case  $\Lc \rightarrow \infty$ resembles an infinite zoom into the material, where an equivalence to linear elasticity with $\Cmicro$ can be derived, cf. \cite{NefEidMad:2019:ios}.  In the latter case, it can be shown that it holds that $\Bdis = \nabla \bu$.  In our study, we approximate the limiting cases by $\Lc=10^{-3}$ and $\Lc=10^{3}$, respectively. Figure \ref{fig:example2:energy} shows the elastic energy $W$ along the radius and Figure \ref{fig:example2:chi} illustrates the non-vanishing components of $\Bdis$ together with the respective displacement gradient components using 30624 Q2NQ2 elements.  In Figure \ref{fig:example2:potential}, we plot the total potential of the relaxed micromorphic model varying  the characteristic length parameter $\Lc$. The figures clearly show the above described behavior. The bounded behavior of the relaxed micromorphic model for small sizes, i.e. $\Lc \rightarrow 0$, is an important advantage which most other generalized continua miss. For the linear elasticity model, a standard T2 nodal element is implemented.

	\begin{figure}[ht]
\center
	\unitlength=1mm
  	\begin{picture}(110,60)
	\put(0,0){\def\svgwidth{11 cm}{\input{figures/example2_lc_study_energy.eps_tex}}}
	      	   \put(52.903717, -0.2734547){\color[rgb]{0,0,0}\makebox(0,0)[lb]{\smash{$r$}}}%
   \put(0.203717, 30.2734547){\color[rgb]{0,0,0}\makebox(0,0)[lb]{\rotatebox[origin=c]{90}{ \smash{$W$}}}}%
	\end{picture}
\caption{Elastic energy $W$ along the radius.}
\label{fig:example2:energy}
	\end{figure}
		  	  
		  	  
\begin{figure}[ht]
\center
	\unitlength=1mm
  	\begin{picture}(110,70)
	\put(-2,0){\def\svgwidth{11.1 cm}{\input{figures/example2_lc_study_chi.eps_tex}}}
	      	   \put(52.903717, -0.2734547){\color[rgb]{0,0,0}\makebox(0,0)[lb]{\smash{$r$}}}%
   \put(0.203717, 25.2734547){\color[rgb]{0,0,0}\makebox(0,0)[lb]{\rotatebox[origin=c]{90}{ \smash{$\Bdis, \nabla \bu$}}}}%
	\end{picture}
	\caption{The non-vanishing components of $\Bdis$ and $\nabla \bu$ along the radius. $\nabla \bu$ is not influenced  by the value of the characteristic length $\Lc$.}
\label{fig:example2:chi}
	\end{figure}
		  
		  
\begin{figure}[ht]
\center
	\unitlength=1mm
         \includegraphics[width=0.7 \textwidth]{figures/example2_total_potential_lc_study.pdf} 
 \put(-55.903717, -1.2734547){\color[rgb]{0,0,0}\makebox(0,0)[lb]{\smash{$1/\Lc$}}}%
  \put(-50.903717, 12.2734547){\color[rgb]{0,0,0}\makebox(0,0)[lb]{\smash{linear elasticity with $\Cmacro$}}}%
   \put(-50.903717, 55.2734547){\color[rgb]{0,0,0}\makebox(0,0)[lb]{\smash{linear elasticity with $\Cmicro$}}}%
   \put(-115.203717, 30.2734547){\color[rgb]{0,0,0}\makebox(0,0)[lb]{\rotatebox[origin=c]{90}{ \smash{$\Pi$}}}}%      
		  	\caption{Total potential varying the characteristic length.}
\label{fig:example2:potential}
\end{figure} 
		  
Next, we investigate the behavior of the different stresses $\Bsigma$, $\Bsigma_\textrm{micro}$ and $\bbm$ under a variation of $\Lc$. The force stress tensor $\Bsigma$ shown in Figure  \ref{fig:example2:cauchystress} vanishes for large value of the characteristic length, $\Lc=1000$, while it is bounded from above by the classical linear elasticity stress with elasticity tensor $\Cmacro$ for $\Lc=0.001$. The only non-vanishing component of the moment stress $m_{r z}$ is shown in Figure \ref{fig:example2:momentstress} ($m_{\theta z} = 0$), which behaves opposite to the force stress when varying $\Lc$. It is nearly zero for $\Lc=0.001$  and it rises for growing $\Lc$. The micro-stress shown in  Figure \ref{fig:example2:microstress} is confined between the linear elasticity stress with elasticity tensor $\Cmicro$ from above and the one with $\Cmacro$  from below for large and small values of the characteristic length, respectively. 
Summarizing the previous findings shortly, increasing the characteristic length diminishes force stress and raises the micro- and  moment stresses, while both force and moment stresses vanish for large and small values of the characteristic length, respectively, the micro-stress is always present. 


	
	
\begin{figure}[ht]
\center
	\unitlength=1mm
  	\begin{picture}(120,70)
	\put(0,1){\def\svgwidth{11 cm}{\input{figures/example2_lc_study_sigma.eps_tex}}}
	      	   \put(52.903717, -0.2734547){\color[rgb]{0,0,0}\makebox(0,0)[lb]{\smash{$r$}}}%
	\end{picture}
\caption{Force stress shear component $\sigma_{r \theta} = \sigma_{\theta r}$ plotted along the radius.}
\label{fig:example2:cauchystress}
	\end{figure}
		  	  
		 
	\begin{figure}[ht]
\center
	\unitlength=1mm
  	\begin{picture}(120,70)
	\put(0,1){\def\svgwidth{11 cm}{\input{figures/example2_lc_study_moment_stress.eps_tex}}}
	      	   \put(52.903717, -0.2734547){\color[rgb]{0,0,0}\makebox(0,0)[lb]{\smash{$r$}}}%
	\end{picture}
\caption{Non-zero component of the moment stress $m_{r z}$ plotted along the radius.}
\label{fig:example2:momentstress}
	\end{figure}
	
	\begin{figure}[ht]
\center
	\unitlength=1mm
  	\begin{picture}(120,70)
	\put(0,1){\def\svgwidth{11 cm}{\input{figures/example2_lc_study_sigma_micro.eps_tex}}}
	      	   \put(52.903717, -0.2734547){\color[rgb]{0,0,0}\makebox(0,0)[lb]{\smash{$r$}}}%
	\end{picture}
\caption{Micro-stress shear component $(\sigma_\textrm{micro})_{r \theta} = (\sigma_\textrm{micro})_{\theta r}$ plotted along the radius.}
\label{fig:example2:microstress}
	\end{figure}
		  

		  
		  
\FloatBarrier
