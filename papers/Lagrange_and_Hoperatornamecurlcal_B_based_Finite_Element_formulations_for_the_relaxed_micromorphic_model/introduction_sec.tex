%=========================================================================
\sect{\hspace{-5mm}. Introduction}
\label{sec:intor}
%=========================================================================
 Metamaterials are receiving tremendous attention in both academia and industry due to their unconventional mechanical properties. These are not solely governed by the bulk mechanical properties but also by the geometry of the unit cells which can be designed to attain the desired functionality, see \cite{FisHilEbe:2020:mmo,LeeSinTho:2012:mnm,SurGaoDuLiXioFanLu:2019:AEM,YaZhoLiaJiaWu:2018:mma,Zad:2016:mmm}. Moreover, the recent advances of  the additive manufacturing (AM, or 3D printing) techniques are empowering the fabrication process of three-dimensional architected  metamaterials, c.f. \cite{JiaLi:2018:3pa,MonKuaArnQi:2020:rai,PloPan:2019:rod,LeiHonZhaHamCheLuQi:2019:3po}. To simplify the design process of novel metamaterials, suitable computational tools are needed to capture their unprecedented effective mechanical properties. The classical Cauchy-Boltzmann theory and the first-order homogenization procedures often fail to describe the mechanical macroscopic behavior of mechanical metamaterials since they exhibit the size-effect phenomenon, i.e. small specimens are stiffer than big specimens, and therefore other generalized theories are needed such as the classical Mindlin-Eringen micromorphic theory \cite{Min:1964:msi,SuhEri:1964:nto,EriSub:1964:nto,Eri:1968:mom,LeiMah:2015:coh,LeiMah:2015:tfh}, the Cosserat theory \cite{CosCos:1909:tof,Nef:2006:tcc,NefJeoMueRam:2010:lce}, gradient elasticity  \cite{MinEsh:1968:ofsg,AltAif:1997:osa,FisKlaMerSteMue:2011:iao} or others. 

The relaxed micromorphic model, which we adopt in this work, has been introduced recently in \cite{NefGhiMadPlaRos:2014:aup,GhiNefMadPlaRos:2015:trl,NefGhiLazMad:2015:trl}. It keeps the full kinematics of the  micromorphic theory but employs the matrix Curl operator of a non-symmetric second-order micro-distortion field for the curvature measurement.  The relaxed micromorphic model reduces the complexity of the classical micromorphic theory by decreasing the number of material parameters and has shown many advantages such as the separation of the material parameters into scale-dependent and scale-independent ones, see for example \cite{DagBarGhiEidNefMad:2020:edo}.  Furthermore, it has already been used to obtain the main mechanical characteristics (stiffness, anisotropy, dispersion) of the targeted metamaterials for many well-posed dynamical and statical problems, e.g. \cite{MadNefGhiPlaRos:2015:bgi,MadNefGhiPlaRos:2015:wpi,MadNefGhiRos:2016:rat,MadNefBar:2016:cbg,NefMadBarDagAbrGhi:2017:rwp,BarMadDagAbrGhiNef:2017:taf,MadNefBarGhi:2017:aro,MadColMinBilOuiNef:2018:mpc,BarTalDagAivNefMad:2019:rmm,AivTalAgoDaoNefMad:2020:fan}. Recently, the scale-independent short range elastic parameters in the relaxed micromorphic model were determined for artificial periodic microstructures in \cite{NefEidMad:2019:ios} which are used to capture the band-gaps as  a dynamical property of  mechanical metamaterial in \cite{DagBarGhiEidNefMad:2020:edo}.  Analytical solutions of the relaxed micromorphic compared to the solutions of other generalized continua for some essential boundary value problems, i.e. pure shear, bending, torsion and uniaxial tension, are discussed in \cite{RizHueKhaGhiMadNef:2021:aso1,RizKhaGhiMadNef:2021:aso2,RizHueMadNef:2021:aso3,RizHueMadNef:2021:aso4}, emphasizing the validity of the relaxed micromorphic model for small sizes  (bounded stiffness), where most of the other generalized continua exhibit unphysical stiffness properties. 

As a result of employing the matrix Curl operator of the micro-distortion field for the curvature measurement, the relaxed micromorphic model seeks the solution of the micro-distortion in $H(\curl,\B)$, while the displacement solution is still in $H^1(\B)$.  The appropriate finite elements of such case must be conforming in $H(\curl,\B)$  (tangentially conforming). The first formulation of edge elements was presented in \cite{RavTho:1977:amf}. In fact, the name ``edge'' elements was used because the degree of freedoms (dofs) are associated only with edges for the first-order approximation. $H(\curl,\B)$-conforming finite elements of  first kind were introduced in \cite{Ned:1980:mfe} and second kind in \cite{Ned:1986:anf}, which are comparable with $H(\div,\B)$-conforming  elements of first kind in \cite{RavTho:1977:amf} and second kind in \cite{BreDouMar:1984:tfo}. An extension to elements with curved edges, based on covariant projections, was developed by \cite{CroSilHur:1988:cpe}. A general implementation of  N\'ed\'elec elements of  first kind is presented in \cite{OlmBadMAr:2019:oag} and a detailed review about $H(\div,\B)$- and $H(\curl,\B)$-conforming finite elements is available in \cite{KirLogRogTer:2012:cau} and \cite{RogKirAnd:2009:eao}.  Furthermore, hierarchical $H(\curl)$-conforming finite elements are used to solve Maxwell boundary and eigenvalue problems in \cite{schzag:2005:hon}. A  $H^1(\B) \times H(\curl,\B)$ finite element formulation for a simplified anti-plane shear case of the relaxed micromorphic model utilizing a scalar displacement field and a vectorial micro-distortion field is available in \cite{SkyNeuMueSchNef:2021:CM}.

 In our work, we demonstrate the main technologies related to the finite element realization of the theoretically-sound relaxed micromorphic model. The proper finite element approximation of the micro-distortion field is the N\'ed\'elec space which utilizes tangential-conforming vectorial shape functions. We provide a comprehensive description of the construction of  $H^1(\B) \times H(\curl, \B)$ elements with N\'ed\'elec  formulation of  first kind on triangular and quadrilateral meshes.  Six finite elements are built, which are different in the approximation space of the micro-distortion: two triangular elements with first- and second-order N\'ed\'elec formulation, two quadrilateral elements with first- and second-order N\'ed\'elec formulation, and two nodal triangular elements with standard  first- and second-order Lagrangian formulation. The paper is organized as follows. In Section \ref{sec:model}, we introduce the relaxed micromorphic model and derive the variational problem with the resulting strong forms and  the associated boundary conditions which are modulated in a physical point of view by the so-called consistent coupling condition. We cover in Section \ref{sec:fem} the main components of the implementation of standard nodal and nodal-edge elements. Two numerical example are introduced in Section \ref{sec:fem}. The first numerical example is designed to check the convergence behavior of the different finite elements when the solution is discontinuous in the micro-distortion field. We investigate the influence of the characteristic length in a second example which covers the size-effect property. We conclude the paper in Section \ref{sec:con}. 
