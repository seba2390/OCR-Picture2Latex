\sect{Appendix}
\label{app:Nedelec_shape_functions}
\numberwithin{equation}{section}
\setcounter{equation}{0}
\ssect{Construction of triangular N\'ed\'elec shape functions}
The parameter elements are defined in the natural coordinates $\Bxi = (\xi,\eta)^T$. Triangular elements are defined on the domain $\B_e^\triangle = \{0 \le \xi \le 1, 0 \le \eta \le 1 - \xi\}$.  The finite elements with edge numbering are shown in Figure  \ref{Figure:nedelec_elements}.

\sssect{\hspace{-5mm}. First-order triangular element NT1 \\} 
\label{sec:app:nt1}
The  N\'ed\'elec space of this element reads 
 \begin{equation}
  \left[ \mathcal{ND}^\triangle \right]^{2}_1 = \bigg\{ \left[ \begin{array}{c}
1 \\
0  
  \end{array} \right] \,, \left[ \begin{array}{c}
0 \\
1
  \end{array} \right]  \,, \left[ \begin{array}{c}
- \eta \\
\xi
  \end{array} \right]  \bigg\}\,,
 \end{equation}
 and the general form of the vectorial shape function is  
 \begin{equation}
 \label{eq:app:gf1}
 \bv^1  = \left[\begin{array}{c}
a_1 - a_3 \, \eta \\ 
a_2 + a_3 \, \xi
\end{array}\right],
\end{equation}
where $a_i, i=1,2,3$ are coefficients yet to be defined based on the dofs. Starting from definition in Equation (\ref{eq:t:edge_dof}), we set $r_j =1$ for all edges and the tangential vectors of the edges, see Figure \ref{Figure:nedelec_elements} (right), are 
\begin{equation}
\bt_1  = \frac{1}{\sqrt{2}}\left[ \begin{array}{c}
1 \\
-1 \\
\end{array} \right], \quad \bt_2  = \left[ \begin{array}{c}
0 \\
1 \\
\end{array} \right], \quad \bt_3  =
\left[ \begin{array}{c}
1 \\
0 \\
\end{array} \right].
\end{equation}
We calculate the dofs  following Equation (\ref{eq:t:edge_dof}) using $\xi + \eta = 1$ on the first edge, $\xi=0$ on the second edge and $\eta=0$ on the third edge and obtain
\begin{equation}
 m^{e_1}_{1} = a_1- a_2 - a_3, \qquad m^{e_2}_{1} = a_2, \qquad  m^{e_3}_{1} = a_1\,.
\end{equation}
In order to obtain the three vectorial shape functions $ \bv^1_1,  \bv^1_2$ and $ \bv^1_3$ form the general function in  (\ref{eq:app:gf1}), we have to compute the three associated combinations for $a_1, a_2$ and $a_3$. We derive the explicit expressions for the three vectorial shape functions by enforcing for the function $\bv_j^k$ associated to the edge $e_j$ 
\begin{equation}
 m^{e_i}_{1} (\bv_j^k) = \delta_{ij}\,.
\end{equation}
 The evaluation of the dofs for each edge leads with
\begin{equation}
\begin{aligned}
\textrm{edge 1:} \quad &m^{e_1}_{1} = 1, \qquad m^{e_2}_{1} = 0, \qquad  m^{e_3}_{1} = 0 \quad \Rightarrow \quad a_1=0, \quad a_2=0, \quad a_3=-1 \\ 
\textrm{edge 2:}  \quad  &m^{e_1}_{1} = 0, \qquad m^{e_2}_{1} = 1, \qquad  m^{e_3}_{1} = 0 \quad \Rightarrow \quad a_1=0, \quad a_2=1, \quad a_3=-1 \\ 
\textrm{edge 3:} \quad   &m^{e_1}_{1} = 0, \qquad m^{e_2}_{1} = 0, \qquad  m^{e_3}_{1} = 1 \quad \Rightarrow \quad a_1=1, \quad a_2=0, \quad a_3=\,\,\,\,1 \\ 
 \end{aligned}
\end{equation}


to the shape vectors 
\begin{equation}
\bv^1_1 =  \left( \begin{array}{c}
\eta \\ 
- \xi
\end{array} \right) \,, \quad
 \bv^1_2 =  \left( \begin{array}{c}
\eta \\
 1 - \xi
\end{array} \right)\,, \quad
\bv^1_3 =  \left( \begin{array}{c}
1 - \eta \\ 
\xi
\end{array} \right) \,.
\end{equation}
A visualization of them is depicted in Figure \ref{Fig:shape_function_NT1}.

\begin{figure}[ht]
     \begin{subfigure}[b]{0.32\textwidth}
         \centering
         \includegraphics[width=\textwidth]{figures/NT1_V1.pdf}
        \caption{$\bv^1_1$}
     \end{subfigure}
     \begin{subfigure}[b]{0.32\textwidth}
         \centering
         \includegraphics[width=\textwidth]{figures/NT1_V2.pdf}
  \caption{$\bv^1_2$}
     \end{subfigure}
          \begin{subfigure}[b]{0.32\textwidth}
         \centering
         \includegraphics[width=\textwidth]{figures/NT1_V3.pdf}
  \caption{$\bv^1_3$}
     \end{subfigure}
        \caption{Tangential-conforming vectorial shape functions of NT1 element. Blue circles indicate the position where the dofs are defined. }
        \label{Fig:shape_function_NT1}
\end{figure}


\sssect{\hspace{-5mm}. Second-order triangular element NT2 \\} 
The  N\'ed\'elec space of this element reads 
 \begin{equation}
  \left[ \mathcal{ND}^\triangle \right]^{2}_2 = \bigg\{ \left[ \begin{array}{c}
1 \\
0  
  \end{array} \right] \,, \left[ \begin{array}{c}
 \xi \\
0
  \end{array} \right]  \,,  \left[ \begin{array}{c}
 \eta \\
0  \end{array} \right] \,, \left[ \begin{array}{c}
0 \\
1  
  \end{array} \right] \,, \left[ \begin{array}{c}
0 \\
\xi
  \end{array} \right]  \,,  \left[ \begin{array}{c}
0 \\
\eta  \end{array} \right]   \,,   \left[ \begin{array}{c}
- \eta^2 \\
\xi \eta
  \end{array} \right]   \,,   \left[ \begin{array}{c}
- \xi \eta \\
\xi^2
  \end{array} \right] \bigg\}\,,
 \end{equation}
and the general form of the shape functions reads
\begin{equation}
\bv^2 =  \left( \begin{array}{c}
a_1 + a_2 \, \xi + a_3 \, \eta - a_7 \, \eta^2 - a_8 \, \xi\eta \\
a_4 + a_5 \, \xi + a_6 \, \eta + a_7 \, \xi\eta + a_8 \, \xi^2
\end{array} \right) ,
\end{equation}
where $a_i, i=1,...,8$ are coefficients yet to be defined based on the dofs. 
The explicit functions $r_j$ and $\bq_i$ in Equations (\ref{eq:t:edge_dof}) and (\ref{eq:t:inner_dof_t}) are assumed as  
\begin{equation}
\begin{aligned}
&\textrm{edge 1 :} \qquad  &r_1 =\xi\,,&   \qquad   &r_2 = \eta\,,  \\
&\textrm{edge 2 :} \qquad  &r_1 =  \eta\,,& \qquad &r_2 = 1- \eta\,, \\
&\textrm{edge 3 :} \qquad  &r_1 = 1-\xi\,,& \qquad &r_2 = \xi\,, \\  
&\textrm{inner  \, :} \qquad   &\bq_1 =    \left[ \begin{array}{c}
1 \\
0 
  \end{array} \right],& \qquad &\bq_2 = \left[ \begin{array}{c}
0 \\
1  
  \end{array} \right], 
  \end{aligned}
\end{equation}
and the tangential vectors of the edges are same as in NT1 element. \\ 
The inner and outer dofs are calculated according to Equations
(\ref{eq:t:edge_dof}) and (\ref{eq:t:inner_dof_t}) using  $\xi + \eta = 1$ on the first edge, $\xi=0$ on the second edge and $\eta=0$ on the third edge
 \begin{equation}
 \begin{aligned}
 &m^{e_1}_{1} = \frac{1}{6}(3 a_1 + 2 a_2 + a_3 - 3 a_4 - 2 a_5 - a_6 - a_7 - 2 a_8) \,, \\ 
 &m^{e_1}_{2} =\frac{1}{6}  (3 a_1 + a_2 + 2 a_3 - 3 a_4 - a_5 - 2 a_6 - 2 a_7 - a_8) \,,\\ 
  &m^{e_2}_{1} = \frac{1}{6} (3a_4 + 2 a_6) \,, \qquad
  m^{e_2}_{2} = \frac{1}{6} ( 3a_4 + a_6) \,, \\
& m^{e_3}_{1} = \frac{1}{6} (3a_1 + a_2) \,,  \qquad \,\,\, m^{e_3}_{2} = \frac{1}{6} (3a_1 + 2a_2) \,, \\ 
  &m^\textrm{inner}_2 = \frac{1}{24} (12 a_1 + 4 a_2 + 4 a_3 - 2 a_7 - a_8)  \,, \\
 &m^\textrm{inner}_1 =  \frac{1}{24} (12 a_4 + 4 a_5 + 4 a_6 + a_7 + 2 a_8) \,, 
 \end{aligned}
\end{equation}
and the resulting shape functions shown in Figure \ref{Fig:shape_function_NT2} are obtained by an analogous procedure as before
\begin{equation}
\begin{aligned}
&\textrm{edge 1 :} &\bv_1^2&= 2 \left( \begin{array}{c} 
-\eta + 4 \eta \xi \\
 2 \xi - 4 \xi^2 
 \end{array} \right),& \quad 
  &\bv_2^2=2 \left( \begin{array}{c} 
-2 \eta + 4 \eta^2 \\
\xi - 4 \eta \xi
 \end{array} \right),& \\
&\textrm{edge 2 :}  &\bv_3^2 &= 2\left( \begin{array}{c} 
-2 \eta + 4 \eta^2 \\
 -1 + 3 \eta + \xi - 4 \eta \xi
 \end{array} \right),& \quad 
 &\bv_4^2 = 2\left( \begin{array}{c} 
3 \eta - 4 \eta^2 - 4 \eta \xi \\
 2 - 3 \eta - 6 \xi + 4 \eta \xi + 4 \xi^2
 \end{array} \right),& \\ 
&\textrm{edge 3 :}  &\bv_5^2 &= 2\left( \begin{array}{c} 
2 - 6 \eta + 4 \eta^2 - 3 \xi + 4 \eta \xi \\
 3 \xi - 4 \eta \xi - 4 \xi^2
 \end{array} \right),& \quad
 &\bv_6^2 = 2\left( \begin{array}{c} 
-1 + \eta + 3 \xi - 4 \eta \xi \\
 -2 \xi + 4 \xi^2
 \end{array} \right),& \\ 
&\textrm{inner \, :}  &\bv_7^2 &= 2\left( \begin{array}{c} 
8 \eta - 8 \eta^2 - 4 \eta \xi \\
-4 \xi + 8 \eta \xi +  4 \xi^2
 \end{array} \right),& \quad
  &\bv_8^2 = 2\left( \begin{array}{c} 
-4 \eta + 4 \eta^2 + 8 \eta \xi \\
8 \xi - 4 \eta \xi - 8 \xi^2
 \end{array} \right).&
 \end{aligned}
\end{equation}

\begin{figure}[ht]
     \begin{subfigure}[b]{0.32\textwidth}
         \centering
         \includegraphics[width=\textwidth]{figures/NT2_V1.pdf}
        \caption{$\bv	^2_1$}
     \end{subfigure}
     \begin{subfigure}[b]{0.32\textwidth}
         \centering
         \includegraphics[width=\textwidth]{figures/NT2_V2.pdf}
  \caption{$\bv^2_2$}
     \end{subfigure}
          \begin{subfigure}[b]{0.32\textwidth}
         \centering
         \includegraphics[width=\textwidth]{figures/NT2_V3.pdf}
  \caption{$\bv^2_3$}
     \end{subfigure}
          \begin{subfigure}[b]{0.32\textwidth}
         \centering
         \includegraphics[width=\textwidth]{figures/NT2_V4.pdf}
        \caption{$\bv^2_4$}
     \end{subfigure}
     \begin{subfigure}[b]{0.32\textwidth}
         \centering
         \includegraphics[width=\textwidth]{figures/NT2_V5.pdf}
  \caption{$\bv^2_5$}
     \end{subfigure}
          \begin{subfigure}[b]{0.32\textwidth}
         \centering
         \includegraphics[width=\textwidth]{figures/NT2_V6.pdf}
  \caption{$\bv^2_6$}
     \end{subfigure}
          \begin{subfigure}[b]{0.32\textwidth}
         \centering
         \includegraphics[width=\textwidth]{figures/NT2_V7.pdf}
        \caption{$\bv^2_7$}
     \end{subfigure}
  \begin{subfigure}[b]{0.32\textwidth}
         \centering
         \includegraphics[width=\textwidth]{figures/NT2_V8.pdf}
        \caption{$\bv^2_8$}
     \end{subfigure}
       \caption{Tangential-conforming vectorial shape functions of NT2 element. Blue circles indicate the position where the dofs are defined. }
        \label{Fig:shape_function_NT2}
\end{figure}

\ssect{Construction of quadrilateral N\'ed\'elec shape functions}
 Quadrilateral elements have the domain $\B_e^\square = \{-1 \le \xi \le 1, -1 \le \eta \le 1\}$. 

\sssect{\hspace{-5mm}. First-order quadrilateral element NQ1 \\} 
The  N\'ed\'elec space of this element reads 
 \begin{equation}
  \left[ \mathcal{ND}^\square \right]^{2}_1 = \bigg\{ \left[ \begin{array}{c}
1 \\
0  
  \end{array} \right] \,, \left[ \begin{array}{c}
\eta \\
0
  \end{array} \right]  \,, \left[ \begin{array}{c}
0 \\
1
  \end{array} \right]  \,, \left[ \begin{array}{c}
0 \\
\xi
  \end{array} \right]\bigg\} \,,
 \end{equation}
and the general form of the shape vectors reads 
\begin{equation}
\bv^1 =  \left( \begin{array}{c}
a_1 + a_2 \, \eta  \\ a_3 + a_4 \, \xi
\end{array} \right),
\end{equation}
where $a_i, i=1,..,4$ are coefficients yet to be defined based on the dofs. Starting from definition in Equation (\ref{eq:t:edge_dof}), we set $r_j = 1$ for all edges. The tangential vectors for the first and third edges are $\bt_1 = \bt_3 = (1,0)^T$ and for the second and fourth edges are $\bt_2 = \bt_4 = (0,1)^T$, see Figure \ref{Figure:nedelec_elements} (left).  We calculate the edge dofs taking into consideration $\eta = -1$ on the first edge, $\xi = 1$ on the second edge, $\eta = 1$ on the third edge and $\xi = -1$ on the fourth edge
\begin{equation}
m^{e_1}_{1} = 2(a_1 - a_2)\,, \quad
m^{e_2}_{1} = 2(a_3 + a_4)\,, \quad
m^{e_3}_{1} = 2(a_1 + a_2)\,, \quad
m^{e_4}_{1} = 2(a_3 - a_4) \,.
\end{equation}
We solve the system of equations obtained by an  analogous procedure to Section \ref{sec:app:nt1} leading to the following shape functions demonstrated in Figure \ref{Fig:shape_function_NQ1} where $\bv^1_i$ is associated with the edge $e_i$ for $i=1,..,4$
\begin{equation}
\begin{aligned}
\bv_{1}^1 =&  \left( \begin{array}{c}
(-\eta + 1)/4 \\
0 
\end{array} \right), \quad 
\bv_{2}^1 =  \left( \begin{array}{c}
0 \\
 (\xi + 1)/4 
\end{array} \right) , \\
\bv_{3}^1 =&  \left( \begin{array}{c}
(\eta + 1)/4 \\
0
\end{array} \right) , \quad 
\bv_{4}^1 =  \left( \begin{array}{c}
0 \\
 (-\xi + 1)/4
\end{array}\right).
\end{aligned}
\end{equation}

\begin{figure}[ht]
    \centering
     \begin{subfigure}[b]{0.33\textwidth}
         \centering
         \includegraphics[width=\textwidth]{figures/NQ1_V1.pdf}
        \caption{$\bv_1^1$}
     \end{subfigure}
     \begin{subfigure}[b]{0.33\textwidth}
         \centering
         \includegraphics[width=\textwidth]{figures/NQ1_V2.pdf}
  \caption{$\bv_2^1$}
     \end{subfigure}
          \begin{subfigure}[b]{0.33\textwidth}
         \centering
         \includegraphics[width=\textwidth]{figures/NQ1_V3.pdf}
  \caption{$\bv_3^1$}
     \end{subfigure}
          \begin{subfigure}[b]{0.33\textwidth}
         \centering
         \includegraphics[width=\textwidth]{figures/NQ1_V4.pdf}
        \caption{$\bv_4^1$}
     \end{subfigure}
       \caption{Tangential-conforming vectorial shape functions of NQ1 element. Blue circles indicate the position where the dofs are defined. }
        \label{Fig:shape_function_NQ1}
\end{figure}

\sssect{\hspace{-5mm}. Second-order quadrilateral element NQ2 \\} 
The  N\'ed\'elec space of this element reads 
 \begin{equation}
 \begin{aligned}
  \left[ \mathcal{ND}^\square \right]^{2}_2 = \bigg\{ & \left[ \begin{array}{c}
1 \\
0  
  \end{array} \right] \,, \left[ \begin{array}{c}
\xi \\
0
  \end{array} \right]  \,, \left[ \begin{array}{c}
\eta \\
0
  \end{array} \right]  \,, \left[ \begin{array}{c}
\xi \eta \\
0
  \end{array} \right]  \,,  \left[ \begin{array}{c}
\eta^2 \\
0
  \end{array} \right]  \,,    \left[ \begin{array}{c}
\xi \eta^2 \\
0
  \end{array} \right]  \,,  \\ & \left[ \begin{array}{c} 
0 \\
1 
  \end{array} \right] \,, \left[ \begin{array}{c}
0 \\
\xi
  \end{array} \right]  \,, \left[ \begin{array}{c}
0 \\
\eta
  \end{array} \right]  \,, \left[ \begin{array}{c}
0 \\ 
\xi \eta \\
  \end{array} \right]  \,,  \left[ \begin{array}{c}
0 \\
\xi^2
  \end{array} \right]  \,,  \left[ \begin{array}{c}
0 \\
\eta \xi^2
  \end{array} \right] \bigg\}  \,,
   \end{aligned}
 \end{equation}
and the vectorial shape functions have the following general form 
\begin{equation}
\bv^2 =  \left( \begin{array}{c}
a_1 + a_2 \,\xi + a_3\, \eta + a_4\, \xi \eta + a_5\, \eta^2 +  a_6 \, \xi \eta^2 \\ a_7 + a_8 \, \xi + a_9 \, \eta +  a_{10} \, \xi \eta + a_{11} \, \xi^2 + a_{12}  \,\eta \xi^2
\end{array} \right) \,,
\end{equation}
where $a_i, i=1,..,12$ are coefficients yet to be defined based on the dofs. Starting from Equations  (\ref{eq:t:edge_dof}) and (\ref{eq:t:inner_dof_q}), the explicit functions $r_j$ and $\bq_i$ are set as 
\begin{equation}
 \begin{aligned}
\textrm{edge 1:} \quad  r_1 =& \frac{1}{2}  (1- \xi)\,, \quad &r_2 =&\frac{1}{2}  (1+ \xi)\,, \\
\textrm{edge 2:} \quad r_1 =& \frac{1}{2}  (1- \eta)\,,  \quad &r_2 =&\frac{1}{2}  (1+ \eta)\,, \\
\textrm{edge 3:} \quad r_1 =& \frac{1}{2}  (1+ \xi)\,, \quad &r_2 =&\frac{1}{2}  (1- \xi)\,, \\
\textrm{edge 4:} \quad r_1 =& \frac{1}{2}  (1+ \eta)\,,\quad &r_2 =&\frac{1}{2}  (1- \eta)\,, \\
\textrm{inner :} \quad \bq_1 =& \left[ \begin{array}{c}
\frac{1}{2}  (1+ \xi) \\
0 
  \end{array} \right] , \quad &\bq_2 =&\left[ \begin{array}{c}
\frac{1}{2}  (1- \xi) \\
0 
  \end{array} \right] , \\ \bq_3 =& 
  \left[ \begin{array}{c}
0 \\
 \frac{1}{2}  (1+ \eta) 
  \end{array} \right] , \quad &\bq_4 =&\left[ \begin{array}{c}
0 \\
 \frac{1}{2}  (1- \eta) 
  \end{array} \right]
 \,,
  \end{aligned}
\end{equation}
The edge and inner dofs are calculated according to Equations  (\ref{eq:t:edge_dof}) and (\ref{eq:t:inner_dof_q})  considering that the tangential vector and the coordinates coloration are same as in NQ1 element
\begin{equation}
 \begin{aligned}
  m^{e_1}_{1} =& \frac{1}{3} \, (3a_1 - a_2 - 3a_3 + a_4 + 3a_5 - a_6) \,,  &m^{e_1}_{2} &= \frac{1}{3}  \, (3a_1 + a_2 - 3a_3 - a_4 + 3a_5 + a_6)\,, 
  \\  m^{e_2}_{1} =& \frac{1}{3}  \, (-a_{10} +3a_{11} - a_{12} + 3a_7 + 3a_8 - a_9) \,,   &m^{e_2}_{2} &= \frac{1}{3}  \, (a_{10} +3a_{11} + a_{12} + 3a_7 + 3a_8 + a_9) \,, 
  \\  m^{e_3}_{1} =& \frac{1}{3}  \,(3a_1 + a_2 + 3a_3 + a_4 + 3a_5 + a_6) \,,  &m^{e_3}_{2} &= \frac{1}{3} \,(3a_1 - a_2 + 3a_3 - a_4 + 3a_5 - a_6) \,, 
  \\  m^{e_4}_{1} =& \frac{1}{3} \,( -a_{10} + 3a_{11} + a_{12} + 3a_7 - 3a_8 + a_9) \,,  &m^{e_4}_{2} &= \frac{1}{3} \,( a_{10} + 3a_{11} - a_{12} + 3a_7 - 3a_8 - a_9) \,, 
  \\  m^\textrm{inner}_1 =& \frac{2}{9} (9 a_1 + 3 a_2 + 3 a_5 + a_6) \,, &m^\textrm{inner}_2 &= \frac{2}{9} (9 a_1 - 3 a_2 + 3 a_5 - a_6) \,, 
  \\ m^\textrm{inner}_3 =& \frac{2}{9} (3 a_{11} + a_{12} + 9 a_7 + 3 a_9) \,, &m^\textrm{inner}_4 &= \frac{2}{9} (3 a_{11} - a_{12} + 9 a_7 - 3 a_9) \,.
 \end{aligned}
\end{equation} 


The  basis functions demonstrated in Figure  \ref{Fig:shape_function_NQ2}  are obtained by an analogous procedure as before

\begin{equation}
\begin{aligned}
\textrm{edge 1:} \quad \bv^2_1 &= \left( \begin{array}{c} 
-1/8 - \eta/4 + 3 \eta^2 /8 + 3 \xi/8 +  3 \eta \xi/4 - 9 \eta^2 \xi/8 \\
0
 \end{array} \right), \\
  \bv^2_2 &= \left( \begin{array}{c} 
-1/8 - \eta/4 + 3 \eta^2 /8 - 3 \xi/8 -  3 \eta \xi/4 + 9 \eta^2 \xi/8 \\
0
 \end{array} \right), \\
\textrm{edge 2:} \quad   \bv^2_3& = \left( \begin{array}{c} 
0 \\
- 1/8 + 3 \eta/8 + \xi/4 - 3 \eta \xi/4 +  3 \xi^2/8 - 9 \eta \xi^2/8
 \end{array} \right), \\
 \bv^2_4 &= \left( \begin{array}{c} 
0 \\
- 1/8 - 3 \eta/8 + \xi/4 + 3 \eta \xi/4 +  3 \xi^2/8 + 9 \eta \xi^2/8
 \end{array} \right), \\
\textrm{edge 3:} \quad   \bv^2_5 &= \left( \begin{array}{c} 
-1/8 + \eta/4 + 3 \eta^2/8 - 3 \xi/8 +  3 \eta \xi/4 + 9 \eta^2 \xi/8 \\
0
 \end{array} \right), \\
 \bv^2_6 &= \left( \begin{array}{c} 
-1/8+ \eta/4 + 3 \eta^2/8 +  3 \xi/8 - 3 \eta \xi/4 - 9 \eta^2 \xi/8 \\
0
 \end{array} \right), \\
\textrm{edge 4:} \quad   \bv^2_7 &= \left( \begin{array}{c} 
0 \\
-1/8 - 3 \eta/8 - \xi/4 - 3 \eta \xi/4 +  3 \xi^2/8 + 9 \eta \xi^2/8
 \end{array} \right), \\
  \bv^2_8 &= \left( \begin{array}{c} 
0 \\
-1/8 + 3 \eta/8 - \xi/4 + 3 \eta \xi/4 +  3 \xi^2/8 - 9 \eta \xi^2/8
 \end{array} \right)\,.
  \\
\textrm{inner:} \quad   \bv^2_9 &= \left( \begin{array}{c} 
3/8 - 3 \eta^2/8 + 9 \xi/8 - 9 \eta^2 \xi/8 \\
0
 \end{array} \right)\,, 
 \\ 
  \bv^2_{10} &= \left( \begin{array}{c} 
3/8 - 3 \eta^2/8 - 9 \xi/8 + 9 \eta^2 \xi/8 \\
0
 \end{array} \right)\,, 
 \\
   \bv^2_{11} &= \left( \begin{array}{c} 
0 \\
3/8 +9 \eta/8 - 3 \xi^2/8 - 9 \eta \xi^2/8
 \end{array} \right)\,, 
 \\
   \bv^2_{12} &= \left( \begin{array}{c} 
0 \\
3/8 -9 \eta/8 - 3 \xi^2/8 + 9 \eta \xi^2/8
 \end{array} \right)\,. 
 \end{aligned}
\end{equation}

\begin{figure}[ht]
     \begin{subfigure}[b]{0.32\textwidth}
         \centering
         \includegraphics[width=\textwidth]{figures/NQ2_V1.pdf}
        \caption{$\bv^2_1$}
     \end{subfigure}
     \begin{subfigure}[b]{0.32\textwidth}
         \centering
         \includegraphics[width=\textwidth]{figures/NQ2_V2.pdf}
  \caption{$\bv^2_2$}
     \end{subfigure}
          \begin{subfigure}[b]{0.32\textwidth}
         \centering
         \includegraphics[width=\textwidth]{figures/NQ2_V3.pdf}
  \caption{$\bv^2_3$}
     \end{subfigure}
          \begin{subfigure}[b]{0.32\textwidth}
         \centering
         \includegraphics[width=\textwidth]{figures/NQ2_V4.pdf}
        \caption{$\bv^2_4$}
     \end{subfigure}
       \begin{subfigure}[b]{0.32\textwidth}
         \centering
         \includegraphics[width=\textwidth]{figures/NQ2_V5.pdf}
        \caption{$\bv^2_5$}
     \end{subfigure}
     \begin{subfigure}[b]{0.32\textwidth}
         \centering
         \includegraphics[width=\textwidth]{figures/NQ2_V6.pdf}
  \caption{$\bv^2_6$}
     \end{subfigure}
          \begin{subfigure}[b]{0.32\textwidth}
         \centering
         \includegraphics[width=\textwidth]{figures/NQ2_V7.pdf}
  \caption{$\bv^2_7$}
     \end{subfigure}
          \begin{subfigure}[b]{0.32\textwidth}
         \centering
         \includegraphics[width=\textwidth]{figures/NQ2_V8.pdf}
        \caption{$\bv^2_8$}
     \end{subfigure}
        \begin{subfigure}[b]{0.32\textwidth}
         \centering
         \includegraphics[width=\textwidth]{figures/NQ2_V9.pdf}
        \caption{$\bv^2_9$}
     \end{subfigure}
     \begin{subfigure}[b]{0.32\textwidth}
         \centering
         \includegraphics[width=\textwidth]{figures/NQ2_V10.pdf}
  \caption{$\bv^2_{10}$}
     \end{subfigure}
          \begin{subfigure}[b]{0.32\textwidth}
         \centering
         \includegraphics[width=\textwidth]{figures/NQ2_V11.pdf}
  \caption{$\bv^2_{11}$}
     \end{subfigure}
          \begin{subfigure}[b]{0.32\textwidth}
         \centering
         \includegraphics[width=\textwidth]{figures/NQ2_V12.pdf}
        \caption{$\bv^2_{12}$}
     \end{subfigure}
       \caption{Tangential-conforming vectorial shape functions of NQ2 element.  Blue circles indicate the position where the dofs are defined. }
        \label{Fig:shape_function_NQ2}
\end{figure}
