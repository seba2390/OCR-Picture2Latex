\sect{\hspace{-5mm}. The relaxed micromorphic model}
\label{sec:model} 
%=========================================================================

The relaxed micromorphic model is a continuum model which describes the kinematics of a material point using a displacement vector $\bu\colon\B\subseteq\R^3\to\R^3$ and a non-symmetric micro-distortion field  $\Bdis\colon\B\subseteq\R^3\to\R^{3\times3}$. Both are defined  for the static case by the minimization of potential
\begin{equation}
\label{eq:pot}
\Pi (\bu,\Bdis) = \int_\B W\left(\nabla \bu,\Bdis,\Curl \Bdis \right) \, - \overline{\bbf}\cdot{\bu} \, - \overline{\bM} : \Bdis \,\textrm{d}V\ - \int_{\partial \B_t} \overline{\bt} \cdot \bu  \,\textrm{d}A  \longrightarrow\ \min\,,
\end{equation}
with $(\bu,\Bdis)\in H^1(\B)\times H(\curl,\B)$. The vector $\overline\bbf$ and the tensor $\overline{\bM}$ describe, respectively, the given body force and the body moment, while $\overline\bt$ is the traction vector acting on the boundary $\partial \B_t \subset \partial \B$. The elastic energy density $W$ reads  
\begin{equation}
\begin{aligned}
W\left(\nabla \bu,\Bdis,\Curl \Bdis \right) =& \, \dfrac{1}{2} ( \symb{ \nabla \bu - \Bdis} : \Ce : \symb{ \nabla \bu - \Bdis} +   \sym \Bdis : \Cmicro: \sym \Bdis  \,, \\
&+ \skewb{ \nabla \bu - \Bdis} : \Cc : \skewb{ \nabla \bu - \Bdis} +   \mu \, \Lc^2 \, \textrm{Curl} \Bdis : \IL :\textrm{Curl} \Bdis ) \,. \\
\end{aligned}
\end{equation}
Here, $\Cmicro,\Ce$ are fourth-order positive definite standard elasticity tensors, $\Cc$ is a fourth-order positive semi-definite rotational coupling tensor, $\IL$ is a fourth-order tensor, $\Lc$ is a non-negative parameter describing the characteristic length scale and $\mu$ is a typical shear modulus.  The characteristic length parameter plays a significant role in the relaxed micromorphic model. This parameter is related to the size of the microstructure  and determines its influence on the macroscopic mechanical behavior. A relation of the relaxed micromorphic model to the classical Cauchy model was shown in \cite{NefEidMad:2019:ios} for limiting values of $\Lc$, which we can also observe in our numerical examples. 


The variation of the potential with respect to the displacement field, i.e. $\delta_{\bu} \Pi = 0$, with 
\begin{equation}
\delta_{\bu} \Pi =  \int_{\B} \{ \underbrace{\Ce : \symb{ \nabla \bu - \Bdis}+  \Cc : \skewb{ \nabla \bu - \Bdis}}_{ \textstyle =:\Bsigma} \} :  {\nabla \delta \bu} - \overline\bbf \cdot \delta \bu  \, \textrm{d}V - \int_{\partial \B_t} \overline\bt \cdot \delta \bu  \, \textrm{d}A\,,
\end{equation} 
leads after integration by parts and applying the divergence theorem to the weak form 
 \begin{equation}
\delta_{\bu} \Pi =  \int_{\B}   \{ \div \Bsigma + \overline\bbf \} \cdot \delta \bu \, \textrm{d}V  = 0 \, ,
\end{equation}  
where  $\Bsigma$ denotes the non-symmetric force stress tensor.
The associated strong form with the related boundary conditions reads 
 \begin{equation}
   \label{eq:sf1}
 \div\Bsigma+\overline\bbf = \bzero \,  \quad \textrm{with} \quad \bu = \overline{\bu} \quad \textrm{on} \quad \partial \B_u\, \quad  \textrm{and} \quad \overline\bt = \Bsigma \cdot \bn \quad \textrm{on} \quad  \partial \B_t \,, 
\end{equation}  
satisfying  $\partial \B_u \cap \partial \B_t = \emptyset $ and  $\partial \B_u \cup \partial \B_t = \partial \B $ and $\bn$ is the outward normal on the boundary. In a similar way, the variation of the potential with respect to the micro-distortion field, i.e. $\delta_{\Bdis} \Pi = 0$, with 
\begin{equation}
\label{eq:de1}
 \delta_{\Bdis} \Pi =   \int_{\B} \{ \Bsigma  - \underbrace{\Cmicro : \sym \Bdis}_{\textstyle =:\Bsigma_\textrm{micro} } + \overline\bM\} :  \delta {\Bdis}  -  \underbrace{\mu \, \Lc^2  (\IL : \Curl \Bdis)}_{\textstyle =:\bbm} : \Curl \delta {\Bdis} \, \textrm{d} V\,, 
\end{equation}
yields after integration by parts and applying Stokes' theorem  
\begin{equation}
\begin{aligned}
 \delta_{\Bdis} \Pi = &  \int_{\B}  \{ \Bsigma - \Bsigma_\textrm{micro}  - \Curl \bbm + \overline\bM\} :  \delta {\Bdis} \, \textrm{d} V +  \int_{ \partial \B}  \{ \sum_{i=1}^3  \left( \bbm^i  \times \delta {\Bdis}^i  \right) \cdot \bn  \}  \;\; \textrm{d}A  
 = 0\,, 
\end{aligned} 
\end{equation}
where $\Bsigma_\textrm{micro}  $ and $\bbm$ are  the micro- and moment stresses, respectively, and $\bbm^i$ and $\delta \Bdis^i$ denote the row vectors of the associated second-order tensors. Using the identity of the scalar triple vector product
\begin{equation}
(\ba \times \bb) \cdot \bc = (\bc \times \ba) \cdot \bb  = (\bb \times \bc) \cdot \ba \,,
\end{equation}
allows for the reformulation
\begin{equation}
 \int_{ \partial \B}  \{ \sum_{i=1}^3  \left( \bbm^i  \times \delta {\Bdis}^i  \right) \cdot \bn  \}  \;\; \textrm{d}A =
 \int_{ \partial \B_P}  \{ \sum_{i=1}^3  \left(   \delta {\Bdis}^i  \times \bn  \right) \cdot \bbm^i  \}  \;\; \textrm{d}A   -  \int_{ \partial \B_m}  \{ \sum_{i=1}^3  \left(  \bbm^i  \times \bn  \right) \cdot \delta {\Bdis}^i   \}  \;\; \textrm{d}A  \,.
\end{equation}
The associated strong form reads  
 \begin{equation}
   \label{eq:sf2}
 \Curl \bbm = \Bsigma - \Bsigma_\textrm{micro} + \overline\bM \,,
 \end{equation}
 with the related boundary conditions  
 \begin{equation}
\sum_{i=1}^3  {\Bdis}^i  \times \bn  = \overline{\bt}_p \quad \textrm{on} \quad \partial \B_\dis \quad \textrm{and by definition} \quad \sum_{i=1}^3    \bbm^i  \times \bn = \bzero \quad \textrm{on} \quad  \partial \B_m\,, 
 \end{equation}
 where $\partial \B_\dis \cap \partial \B_m = \emptyset $ and $\partial  \B_\dis \cup \partial \B_m= \partial \B $.  
A dependency between the displacement field and the micro-distortion field on the boundary was proposed by \cite{NefEidMad:2019:ios} and subsequently considered in \cite{SkyNeuMueSchNef:2021:CM,RizHueMadNef:2021:aso3,RizHueMadNef:2021:aso4,DagRizKhaLewMadNef:2021:tcc}. This so-called consistent coupling condition is defined by
 \begin{equation}
  \Bdis \cdot \Btau = \nabla \bu \cdot \Btau \, \Leftrightarrow \,  \Bdis \times \bn = \nabla \bu \times \bn  \quad \textrm{on}\quad  \partial \B_\dis = \partial \B_u  \,,
 \end{equation}
where $\Btau$ is a tangential vector on the Dirichlet boundary.  This condition relates the projection of the displacement gradient on the tangential plane of the boundary to the respective parts of the micro-distortion.
 

The first strong form in Equation (\ref{eq:sf1}) represents a generalized balance of linear momentum (force balance) while the second strong form in Equation (\ref{eq:sf2}) outlines a generalized balance of angular momentum (moment balance). The generalized moment balance invokes the Cosserat theory with the $\Curl \Curl$ operator rising from the matrix $\Curl$ operator of the second-order moment stress $\bbm$.  In comparison to the classical micromorphic model, see \cite{Eri:1968:mom,NefGhiMadPlaRos:2014:aup}, the relaxed micromorphic model uses the same kinematical measures but employs a curvature measure form the Cosserat theory, see \cite{NefJeoMueRam:2010:lce}. The micro-distortion field has the following general form for the three-dimensional case
 \begin{equation}
\Bdis = \left[ \begin{array}{c}
(\Bdis^{1})^T \\
(\Bdis^{2})^T \\
(\Bdis^{3})^T \\
\end{array}\right] = \left[\begin{array}{c c c}
 \dis_{11}  &  \dis_{12}  & \dis_{13}  \\   
 \dis_{21}  &  \dis_{22}  & \dis_{23}  \\ 
 \dis_{31}  &  \dis_{32}  & \dis_{33}  \\ 
\end{array}\right] \quad \textrm{with} \quad \Bdis^{i} = \left[ \begin{array}{c}
\dis_{i1} \\
\dis_{i2} \\ 
\dis_{i3}
\end{array} \right],
 \end{equation}
where $\Bdis^{i}$  are the row vectors of $\Bdis$. 
We let the Curl operator act on the row vectors of the micro-distortion field $\Bdis$, i.e.,
 \begin{equation}
\Curl \Bdis = \left[ \begin{array}{c}
(\curl \Bdis^{1})^T \\
(\curl \Bdis^{2})^T \\
(\curl \Bdis^{3})^T \\
\end{array}\right] = \left[\begin{array}{c|c|c}
 \dis_{13,2} - \dis_{12,3}	& \dis_{11,3} - \dis_{13,1}  &  \dis_{12,1} - \dis_{11,2} \\
 \dis_{23,2} - \dis_{22,3}	& \dis_{21,3} - \dis_{23,1}  &  \dis_{22,1} - \dis_{21,2} \\
 \dis_{33,2} - \dis_{32,3}	& \dis_{31,3} - \dis_{33,1}  &  \dis_{32,1} - \dis_{31,2} 
\end{array}\right].
 \end{equation}
For the two-dimensional case, the micro-distortion field and its Curl operator are reduced to 
 \begin{equation}
\label{eq:t:2dCurlBdis}
\Bdis = \left[ \begin{array}{c}
(\Bdis^{1})^T \\
(\Bdis^{2})^T \\
\bzero^T 
\end{array}\right] = \left[\begin{array}{c c c }
 \dis_{11}  &  \dis_{12}   & 0\\   
 \dis_{21}  &  \dis_{22}   & 0\\
 0 & 0 & 0 
\end{array}\right] \quad \textrm{and} \quad 
\Curl \Bdis = 
\left[\begin{array}{c|c|c}
0	&   0  &  \dis_{12,1} - \dis_{11,2} \\
0   &   0  &  \dis_{22,1} - \dis_{21,2} \\
0 & 0 & 0
\end{array}\right].
 \end{equation}
\sect{\hspace{-5mm}. Approximation spaces}
\label{sec:fem} 
\ssect{\hspace{-5mm}. Nodal elements $(\bu,\Bdis) \in H^1(\B) \times H^1(\B)$}
\label{sec:standardelements}
We introduce the formulation of a  standard nodal element utilizing Lagrange-type shape functions for both displacement and micro-distortion field, see for example \cite{Wri:2008:nfe}. Let us assume that there are $n^u$ nodes in each element for the discretization of the displacement field $\bu$ and $n^\dis$ nodes for micro-distortion field $\Bdis$ in two dimensions. Geometry and displacement field are approximated employing the related Lagrangian shape functions $N^u_I$ defined in the parameter space with the natural coordinates $\Bxi = \{ \xi, \eta\}$ by
\begin{equation}
\label{eq:t:approx_chi_u}
 \bX_h = \sum_{I=1}^{n^u} N^u_I(\Bxi)  \bX_I\,, \qquad \bu_h = \sum_{I=1}^{n^u} N^u_I(\Bxi)  \bd^u_I\,,
\end{equation}
where $\bX_I$  are the coordinates of the displacement node  $I$ and  $\bd^u_I$ are its displacement degrees of freedom. The deformation gradient is obtained in the physical space by 
\begin{equation}
\nabla \bu_h = \sum_{I=1}^{n^u} \bd^u_I \otimes \nabla N^u_I(\Bxi)\, \quad \textrm{with} \quad 
\nabla {N^u_I(\Bxi)} = \bJ^{-T} \cdot \nabla_\Bxi N^u_{I}\,,
\end{equation}
where $\bJ = \frac{\partial \bX}{\partial \Bxi  }$ is the Jacobian, $\nabla$  and $\nabla_\Bxi $ denote the gradient operators with respect to $\bX$ and $\Bxi$, respectively.  
The micro-distortion field $\Bdis$ for the 2D case is approximated using the relevant scalar shape functions $N^\dis_I$  
\begin{equation}
\Bdis_h^1 =  \left[ \begin{array}{c}
\dis_{11} \\
\dis_{12} \\
\end{array} \right] =   \sum_{I=1}^{n^\dis}   N^\dis_I(\Bxi) \bd^{\dis^1}_I\,, \quad
\Bdis_h^2 =  \left[ \begin{array}{c}
\dis_{21} \\
\dis_{22} \\
\end{array} \right] =   \sum_{I=1}^{n^\dis}   N^\dis_I(\Bxi) \bd^{\dis^2}_I\,,
\end{equation}
where  $\bd^{\dis^1}_I$ and $\bd^{\dis^2}_I$ are the micro-distortion row vectors degrees of freedom of node $I$. In order to calculate the Curl of  $\Bdis$, the gradient of the row vectors in the physical space can be calculated by
\begin{equation}
\nabla \Bdis_h^{i} = \bJ^{-T} \cdot \nabla_\Bxi \Bdis_h^{i} \quad \textrm{for} \quad i=1,2 
\end{equation}
and the rotation of the vector $\Bdis_h^{i}$ reads 
\begin{equation}
\curl^{2D}{\Bdis_h^{i}} = (\dis_h)_{i2,1} - (\dis_h)_{i1,2} \quad \textrm{for} \quad i=1,2 \,.
\end{equation}

\ssect{\hspace{-5mm}. Nodal-edge elements $(\bu,\Bdis) \in H^1(\B) \times H(\curl, \B)$ \\} 
\label{sec:mixelements}
The here presented formulation uses different spaces to describe the micro-distortion field. The geometry and the displacement field are approximated in the standard Lagrange space as in Equation (\ref{eq:t:approx_chi_u}).  For the micro-distortion field, its solution is in $ H(\curl,\B)$ and the suitable finite element space is known as N\'ed\'elec space, see \cite{Ned:1980:mfe,Ned:1986:anf}. In this work, we choose the N\'ed\'elec space of  first kind. For more details the reader is referred to  \cite{KirLogRogTer:2012:cau,RogKirAnd:2009:eao,BofBreFor:2014:mfe,Mon:1993:ano}. N\'ed\'elec formulations use vectorial shape functions which satisfy the tangential continuity at element interfaces. The lowest-order two-dimensional N\'ed\'elec elements are depicted in Figure  \ref{Figure:nedelec_elements}. 

\begin{figure}[ht]
\center
	\unitlength=1mm
	\begin{picture}(120,50)
	\put(0,5){\def\svgwidth{11cm}{\small\input{figures/nedelec_elements.eps_tex}}}
	\end{picture}
	\caption{Lowest-order ($k=1$) N\'ed\'elec elements: triangle $\left[ \mathcal{ND}^\triangle \right]^{2}_1$ (right) and quadrilateral $\left[ \mathcal{ND}^\square \right]^{2}_1$ (left). Definition of the individual edges $e_i$. The red arrows indicate the orientation of the tangential flux. }
	\label{Figure:nedelec_elements}
\end{figure} 

 Triangular  N\'ed\'elec elements  of order $k$ are based on the space 
\begin{equation}
\label{eq:t:shape_functions_form_t}
\left[ \mathcal{ND}^\triangle \right]^{2}_k = (\IP_{k-1})^2 \oplus \bS_k \quad \textrm{with} 
 \quad \bS_k = \{ \bp   \in (\tilde{\IP}_k)^2 \,|\, \bp \cdot \Bxi = 0 \}\,,
\end{equation}
 where  $\IP_{k-1}$ is the linear space of polynomials of degree $k-1$ or less and $\tilde{\IP}_{k}$ is the linear space of homogeneous polynomials of degree $k$. Equivalently, the space can be characterized by 
 \begin{equation}
\left[ \mathcal{ND}^\triangle \right]^{2}_k = (\IP_{k-1})^2 \oplus \tilde{\IP}_{k-1} \left[ \begin{array}{c}
- \eta  \\
\xi
\end{array} \right]. 
\end{equation}
  The dimension of this linear space is $
k(k+2) $.  Quadrilateral N\'ed\'elec elements of order $k$ are based on the linear space 
 \begin{equation}
 \label{eq:t:shape_functions_form_q}
\left[ \mathcal{ND}^\square \right]^{2}_k = \left[ \begin{array}{c}
 Q_{k-1,k}  \\
 Q_{k,k-1}
\end{array} \right] \quad \textrm{where} \quad Q_{m,n} = \textrm{span}\{ \xi^i  \eta^j  \,|\, i \leq m,  j \leq n \}, 
 \end{equation}   
with $
  \dim \left( \left[ \mathcal{ND}^\square \right]^{2}_k \right) = 2k(k+1) 
$.
The vectorial shape functions $\bv^k$ in the parametric space are obtained by constructing a linear system of equations based on a set of inner and outer dofs. For the 2D case, the outer dofs of an edge $e_i$ are defined by the integral 
   \begin{equation}
   \label{eq:t:edge_dof}
 m^{e_i}_{j} (\bv^k ) = \int_{e_i} (\bv^k \cdot \bt_i)  \, r_{j} \, \textrm{d} s \,, \quad \forall \, r_j \in \IP_{k-1}(e_i)\,
   \end{equation}  
   where $r_{j}$ is a polynomial $\IP_{k-1}$  along the edge $e_i$ and $\bt_i$ is the normalized tangential vector of the edge $e_i$. The inner dofs are introduced for triangular elements by 
\begin{equation}   
   \label{eq:t:inner_dof_t}
 m^\textrm{inner}_{i}  (\bv^k )   =  \int_{\B_e} \bv^k \cdot \bq_{i} \, \textrm{d} a \,, \quad \forall \, \bq_i \in (\IP_{k-2} (\B_e))^2\, ,
   \end{equation}
   while they are given for quadrilateral elements by
   \begin{equation}   
     \label{eq:t:inner_dof_q}
 m^\textrm{inner}_{i}  (\bv^k )    =  \int_{\B_e} \bv^k   \cdot \bq_{i} \, \textrm{d} a \,, \quad \forall \, \bq_i  \, 
   \in
    \left[ \begin{array}{c}
 Q_{k-1,k-2}  (\B_e) \\
 Q_{k-2,k-1}  (\B_e)
\end{array} \right].
   \end{equation}
 The scalar-valued and vectorial functions $r_j$ and $\bq_i$ are linearly independent polynomials which are chosen as Lagrange polynomials in our work. For lowest-order element ($k=1$), only outer dofs occur. For higher-order elements ($k \ge 2$), the number of outer dofs are increased and additional inner dofs are introduced. E.g. for the $\left[ \mathcal{ND}^\triangle \right]^{2}_2$ with a dimension 8, we have 6 outer dofs and 2 inner ones. The derivations of the $H(\curl,\B)$-conforming vectorial shape functions is shown in Appendix \ref{app:Nedelec_shape_functions}.  Mapping the vectorial shape functions $ \bv^k_I$ from the parametric space to $\hat\Bpsi^k_I $ in the physical space must conserve the tangential continuity property. This is guaranteed by using the covariant Piola transformation, see for example \cite{RogKirAnd:2009:eao}, which reads
\begin{equation}
\hat{\Bpsi}^k_I =  \, \bJ^{-T} \cdot \bv^k_I\, \quad \textrm{and} \quad \curl{\hat{\Bpsi}^k_I} = \frac{1 }{\det{\bJ}} \bJ \cdot \curl_{\Bxi}\bv^k_I\,. 
\end{equation}
For our implementation of the $H(\curl,\B)$-conforming elements, we modify the mapping to enforce the required orientation of the degrees of freedom at the inter-element boundaries and to attach a direct physical interpretation to the Neumann-type boundary conditions. Hence, two additional parameters, $\alpha$ and $\beta$, appear for the vectorial shape functions associated with edge dofs 
\begin{equation}
\label{eq:t:mpiola}
{{\Bpsi}}^k_I  = \alpha_I \beta_I {\hat\Bpsi}^k_I   \quad \textrm{and} \quad \curl{{\Bpsi}^k_I} = \alpha_I \beta_I  \curl{{\hat\Bpsi}^k_I} \,, 
\end{equation}

where $ \alpha_I = \pm 1$ is the orientation consistency function  which ensures that on an edge, belonging to two neighboring finite elements, a positive tangential flux direction is defined. Therefore, a positive tangential direction is defined based on a positive $x$-coordinate. A tangential component pointing in negative $x$-direction is multiplied by a value $\alpha_I = -1$ to obtain the overall positive tangential flux on each edge. If the tangential direction has no projection on $x$-axis, then the same procedure is employed on $y$-direction. Figure \ref{fig:orientation_parameter} illustrates an example of calculating the orientation parameter values of two neighboring elements. 

\begin{figure}[ht]
	\unitlength=1mm
	\center
	\unitlength=1mm
	\center		  
		  	  \begin{subfigure}[b]{0.31\textwidth}
	\begin{picture}(50,45)
	\put(0,0){\def\svgwidth{5cm}{\small\input{figures/orientation_parameter_1.eps_tex}}}
	\end{picture}
	\caption{local orientations of dofs}
		  \end{subfigure}
		  	  \begin{subfigure}[b]{0.35\textwidth}
\begin{tabular}{ c|c c c}
element & $\alpha_1$ & $\alpha_2$ & $\alpha_3$ \\ \hline
1 & -1 & +1 & -1 \\ 
2 & +1 & -1 & +1 
\end{tabular}
	\caption{orientation parameter values}
		  \end{subfigure}		  
 \begin{subfigure}[b]{0.31\textwidth}
	\begin{picture}(50,50)
	\put(0,0){\def\svgwidth{5cm}{\small\input{figures/orientation_parameter_2.eps_tex}}}
	\end{picture}
	\caption{global orientations of dofs}
		  \end{subfigure} 
		  	\caption{Example of assembling of two neighboring elements with satisfying the orientation consistency via the orientation parameter $\alpha_I$.  }
\label{fig:orientation_parameter}
\end{figure} 		 
 The normalization parameter $\beta_I$  enforces that the sum of the vectorial shape functions ${\Bpsi}^k_I $ at a common edge  scalar multiplied with the associated tangential vector has to be equal one in the physical space. Furthermore,  the sum of the shape functions belonging to one edge scalar multiplied with the tangential vector of the other edges must vanish. These conditions are reflected by 
\begin{equation}
\label{eq:t:cond}
\Btau_I \cdot \sum_J {{\Bpsi}}_J^k  \bigg\rvert_{E_I}  \equiv 1  \quad \, \textrm{if}  \quad  I=J  \quad \textrm{and} \, \quad
\Btau_I \cdot \sum_J {{\Bpsi}}_J^k  \bigg\rvert_{E_I}  \equiv 0 \quad \textrm{if} \quad  I \ne J  \,.
\end{equation}
Here, $\sum_J {{\Bpsi}}_J^k  \bigg\rvert_{E_I} $ is the sum of shape vectors related to outer dofs of an edge $E_J$ evaluated on the edge $E_I$ and $\Btau_I$ is the normalized tangential vector of an edge $E_I$ where $E$ denotes the edges in the physical space.  
Based on Equations (\ref{eq:t:mpiola})$_1$ and (\ref{eq:t:cond})$_1$, we compute straightforward the parameters $\beta_I$. In detail we get for the first- and second-order elements 
\begin{equation}
\beta_I = L_I   \quad \textrm{and} \quad   \beta_I = \frac{L_I}{2}\,,
\end{equation}
respectively, where $L_I$ denotes the length of the edge $E_I$ in the physical space. For the 2D case, the rotation of the vectorial shape functions only has one active component out of the plane which reads 
\begin{equation}
 \curl^{2D}{\Bpsi^k_I} = \frac{ \alpha_I  \beta_I }{\det{\bJ}} \curl^{2D}_{\Bxi}\bv^k_I\,.
\end{equation}
The micro-distortion field $\Bdis$ is  approximated by the vectorial dofs $\bd^\dis_I$ representing its tangential components at the location $I=1,...,n^\dis$. The micro-distortion field and its Curl are interpolated as 
\begin{equation}
\label{eq:t:dis_restoring}
\Bdis_h =   \sum_{I=1}^{n^\dis}   \bd^\dis_I \otimes \Bpsi^k_I \, , \quad 
\Curl \Bdis_h  =  \sum_{I=1}^{n^\dis}   \bd^\dis_I  \otimes \curl{{\Bpsi}^k_I} \,.
\end{equation}
The non-vanishing components of the Curl operator of the micro-distortion field for the 2D case are obtained by 
\begin{equation}
\left[\begin{array}{c}
\curl^{2D}{\Bdis_h^{1}} \\
  \curl^{2D}{\Bdis_h^{2}} \\
\end{array}\right]\,
 =  \sum_{I=1}^{n^\dis}   \bd^\dis_I  \curl^{2D}{{\Bpsi}^k_I} =  \left[\begin{array}{c}
\sum_{I=1}^{n^\dis}   (d^\dis_I)_1  \curl^{2D}{{\Bpsi}^k_I} \\
 \sum_{I=1}^{n^\dis}   (d^\dis_I)_2  \curl^{2D}{{\Bpsi}^k_I}  \\
\end{array}\right]. 
\end{equation}


\ssect{\hspace{-5mm}. Implemented finite elements \\} 
In this work, we present four nodal-edge elements based on the formulation in Section \ref{sec:mixelements} and two standard nodal elements based on Section \ref{sec:standardelements}.  All  implemented finite elements employ scalar quadratic shape functions of Lagrange-type for the displacement field approximation with the notation  T2 for triangles and Q2 for quadrilaterals.  The micro-distortion field is approximated using different formulations introduced in Sections \ref{sec:standardelements} and \ref{sec:mixelements}. For the standard nodal elements, Lagrange-type ansatz functions are used resulting in the element types T2T1 (linear ansatz for $\Bdis$) and  T2T2 (quadratic ansatz for $\Bdis$). Different nodal-edge elements  are built utilizing first- and second-order N\'ed\'elec formulations with tangential-conforming shape functions denoted as NT1 and NT2 for triangular elements and QT1 and QT2 for quadrilateral elements. The micro-distortion dofs in the standard nodal elements T2T1 and T2T2 are tensorial with $2 \times 2$ entries while the nodal-edge elements use vectorial dofs for the micro-distortion field which represent the tangential components. The full micro-distortion tensor is restored based on Equation (\ref{eq:t:dis_restoring}).   The used finite elements are depicted in the parameter space in Figure \ref{Fig:Finite_elements}. 
\begin{figure}[ht]
	\unitlength=1mm
	\center
		  	  \begin{subfigure}[b]{0.32\textwidth}
	\begin{picture}(40,40)
	\put(0,0){\def\svgwidth{5cm}{\small\input{figures/T2T1.eps_tex}}}
	\end{picture}
	\caption{T2T1}
		  \end{subfigure}
\begin{subfigure}[b]{0.32\textwidth}
	\begin{picture}(50,45)
	\put(0,0){\def\svgwidth{5cm}{\small\input{figures/T2NT1.eps_tex}}}
	\end{picture}
	\caption{T2NT1}
		  \end{subfigure}
  \begin{subfigure}[b]{0.32\textwidth}
	\begin{picture}(50,50)
	\put(0,0){\def\svgwidth{5cm}{\small\input{figures/Q2NQ1.eps_tex}}}
	\end{picture}
	\caption{Q2NQ1}
		  \end{subfigure}
  \begin{subfigure}[b]{0.32\textwidth}
	\begin{picture}(40,40)
	\put(0,0){\def\svgwidth{5cm}{\small\input{figures/T2T2.eps_tex}}}
	\end{picture}
	\caption{T2T2}
		  \end{subfigure}	
		  	  \begin{subfigure}[b]{0.32\textwidth}
	\begin{picture}(50,45)
	\put(0,0){\def\svgwidth{5cm}{\small\input{figures/T2NT2.eps_tex}}}
	\end{picture}
	\caption{T2NT2}
		  \end{subfigure}
 \begin{subfigure}[b]{0.32\textwidth}
	\begin{picture}(50,50)
	\put(0,0){\def\svgwidth{5cm}{\small\input{figures/Q2NQ2.eps_tex}}}
	\end{picture}
	\caption{Q2NQ2}
		  \end{subfigure} 
		  	\caption{The implemented finite elements in the parameter space. Black dots represent the displacement nodes while red squares stand for micro-distortion field nodes associated with tensorial dofs. Red arrows and crosses indicate the edge and inner vectorial dofs, respectively, of the micro-distortion field used in N\'ed\'elec formulation.}
\label{Fig:Finite_elements}
\end{figure} 
