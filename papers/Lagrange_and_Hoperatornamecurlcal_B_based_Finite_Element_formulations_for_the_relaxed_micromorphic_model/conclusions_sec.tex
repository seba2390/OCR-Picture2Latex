%=========================================================================
\sect{\hspace{-5mm}. Conclusions}
\label{sec:con}
The relaxed  micromorphic model is a generalized continuum model which can suitably reproduce the macroscopic effective  properties of mechanical metamaterials.  First,  we derived the variational problem with the relevant weak and strong forms and the associated boundary conditions.  We put together the main components of  standard nodal and nodal-edge finite element formulations of the relaxed micromorphic model.  The standard nodal elements $H^1(\B) \times H^1(\B)$ are incapable to achieve satisfactory results for a discontinuous solution unlike $H^1(\B) \times H(\curl, \B)$ elements which capture the jumps of the normal components of the micro-distortion field and therefore converges   efficiently. We reveal the role of the characteristic length which governs the scale-dependency property of the relaxed micromorphic model. For $\Lc  \rightarrow 0$, the model is equivalent to the standard Cauchy linear elasticity model with $\Cmacro$ defined as the Reuss lower-limit of elasticity tensors $\Ce$ and $\Cmicro$ while the model is corresponding to  Cauchy linear elasticity model with $\Cmicro$ with $\Bdis = \nabla \bu$ for  $\Lc  \rightarrow \infty$.  Furthermore, we have shown the dependency of the different stress measurements  on the characteristic length.  The force stress is at maximum for $\Lc  \rightarrow 0$  and it vanishes for  $\Lc  \rightarrow \infty$ but the moment stress behaves in the opposite way. The micro-stress varies between Cauchy linear elasticity stresses with  $\Cmicro$  and $\Cmacro$ for   $\Lc  \rightarrow \infty$ and  $\Lc  \rightarrow 0$, respectively. 
 
\vspace{2 cm}

{\bf Acknowledgment} \\
Funded by the Deutsche Forschungsgemeinschaft (DFG, German research Foundation) -  Project number 440935806 (SCHR 570/39-1, SCHE 2134/1-1, NE 902/10-1) within the DFG priority program 2256. The authors gratefully acknowledge Jo\v{z}e Korelc for the development and ongoing support when using AceGen and AceFEM.




%=========================================================================
 
