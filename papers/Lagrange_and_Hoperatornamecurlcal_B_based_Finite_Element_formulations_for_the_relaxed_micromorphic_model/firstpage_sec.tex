\thispagestyle{empty}
\ce{\bf \large
Lagrange and $H(\curl,\B)$ based Finite Element formulations for} 
\ce{\bf \large the relaxed micromorphic model}



\vspace{4mm}
\ce{ J\"org Schr\"oder$^{1,*}$, Mohammad Sarhil$^{1}$, Lisa Scheunemann$^2$ and  Patrizio Neff$^3$}

\vspace{4mm}
\ce{$^1$Institute of Mechanics, University of Duisburg-Essen}
\ce{Universit\"atsstr. 15, 45141 Essen, Germany}
\ce{\small e-mail: 
j.schroeder@uni-due.de
}    

\vspace{4mm}
\ce{$^2$Chair of Applied Mechanics, TU Kaiserslautern,}
\ce{67653 Kaiserslautern, Germany}

\vspace{4mm}
\ce{$^3$Faculty of Mathematics, University of Duisburg-Essen,}
\ce{ 45141 Essen, Germany}

    
\vspace{4mm}
\begin{center}
{\bf \large Abstract}
\bigskip

{\footnotesize
\begin{minipage}{14.5cm}
\noindent
Modeling the unusual mechanical properties of  metamaterials is a challenging topic for the mechanics community and enriched continuum theories are promising computational tools for such materials. The so-called relaxed micromorphic model has shown many advantages in this field. In this contribution, we present the significant aspects related to the relaxed micromorphic model realization with the finite element method. The  variational problem is derived and different FEM-formulations for the two-dimensional case are presented. These are a nodal standard formulation $H^1(\B) \times H^1(\B)$ and a nodal-edge formulation $H^1(\B) \times H(\curl, \B)$, where the latter employs the N\'ed\'elec space. However, the implementation of higher-order N\'ed\'elec elements is not trivial and requires some technicalities which are demonstrated. We discuss the convergence behavior of Lagrange-type and tangential-conforming finite element discretizations. Moreover, we analyze the characteristic length effect on the different components of the model and reveal how the size-effect property is captured via this characteristic length. 
\end{minipage}
}
\end{center}

{\bf Keywords:} 
relaxed micromorphic model,
N\'ed\'elec elements, 
mechanical metamaterials, 
consistent boundary condition,
size-effect



