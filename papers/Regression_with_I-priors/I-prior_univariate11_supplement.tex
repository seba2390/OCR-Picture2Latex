%% 
%% Copyright 2007-2019 Elsevier Ltd
%% 
%% This file is part of the 'Elsarticle Bundle'.
%% ---------------------------------------------
%% 
%% It may be distributed under the conditions of the LaTeX Project Public
%% License, either version 1.2 of this license or (at your option) any
%% later version.  The latest version of this license is in
%%    http://www.latex-project.org/lppl.txt
%% and version 1.2 or later is part of all distributions of LaTeX
%% version 1999/12/01 or later.
%% 
%% The list of all files belonging to the 'Elsarticle Bundle' is
%% given in the file `manifest.txt'.
%% 
%% Template article for Elsevier's document class `elsarticle'
%% with harvard style bibliographic references

\documentclass[preprint,12pt,authoryear]{elsarticle}

\usepackage[english]{babel}
%\usepackage{apacite}
\usepackage{latexsym,graphics,epsfig,subfigure,color,lineno}
\usepackage{amsfonts,amsmath,amssymb,amsthm}
\usepackage{mathtools,xparse}
\usepackage{upgreek}
\usepackage[normalem]{ulem}
\usepackage{pifont}% http://ctan.org/pkg/pifont
%\RequirePackage{lineno}
%\usepackage{hyperref}
%\usepackage[left=3.25cm,right=3.25cm]{geometry}
\usepackage[margin=3.5cm]{geometry}

%\usepackage{mathptmx}
%\usepackage{moresize}


\newcommand{\marginnote}[1]{\marginpar{\framebox{\framebox{#1}}}}
% Comment macro: Usage \comment{Author}{Comment}
\newcommand{\comment}[2]{[\marginnote{#1}\textit{#1}:\ \textit{#2}]}

% \newcommand{\bra}[1]{#1^H}
% \newcommand{\ket}[1]{#1}
% \newcommand{\innerp}[2]{#1^H#2}
% \newcommand{\outerp}[2]{#1#2^H}
% \newcommand{\matrixel}[3]{#1^H#2#3}
% \newcommand{\proj}[1]{\outerp{#1}{#1}}
% \newcommand{\abs}[1]{\vert#1\vert}
% 
% 
% %norms
% \newcommand{\norm}[2]{\left\Vert#1\right\Vert_{#2}}
% 
% %a matrix or vector
% \newcommand{\mat}[2]{\lefto[\begin{array}{#1}#2\end{array}\right]}
% 
% %mathbf etc.
% \newcommand{\mbf}{\mathbf}
% \newcommand{\mbb}{\mathbb}
% \newcommand{\mcl}{\mathcal}
 \newcommand{\trm}{\textrm}
% 
% 
% %Theorems, etc.
% %\theoremstyle{definition}
% \theoremstyle{plain}
%
% 
% %Expectation
% \newcommand{\expect}[1]{\langle#1\rangle}
% 
% %Trace
% \DeclareMathOperator{\tr}{tr}
% \DeclareMathOperator{\Tr}{Tr}
% 
% %Miscellaneous stuff
% \DeclareMathOperator{\supp}{supp}
% \DeclareMathOperator{\Span}{span}
% 
% %identity
% \DeclareMathOperator{\id}{\mathbf{1}}
% 
% %BS addition
% \DeclareMathOperator{\bst}{\stackrel{\land}{+}}
% \DeclareMathOperator{\bs}{\widehat{+}}
% 
% %ladder operators
% \newcommand{\up}{a^\dagger}
% \newcommand{\down}{a}
% 
% %vec operator
% %\DeclareMathOperator{\vect}{vec}
% 
% %equations
% \def\ba#1\ea{\begin{align*}#1\end{align*}}
% %with numeration
% \def\ban#1\ean{\begin{align}#1\end{align}}
% % the same, centered
% \def\bac#1\eac{\vspace{\abovedisplayskip}{\par\centering$\begin{aligned}#1\end{aligned}$\par}\addvspace{\belowdisplayskip}}

%%%%% parentheses denoting arguments, we want an opening object. Use \lefto in
%%%%% these cases.
% \newcommand{\lefto}{\mathopen{}\left}
% 
% %real and imaginary part
% \renewcommand{\Re}{\mathfrak{Re}}
% \renewcommand{\Im}{\mathfrak{Im}}
% 
% % in case you need to break equations over several lines
 \newcommand{\dummyrel}[1]{\mathrel{\hphantom{#1}}\strut\mskip-\medmuskip}
%%%%%%%%%%%%%%%%%%%%%
%:- Theorems
 \newtheorem{thm}{Theorem}
 \newtheorem{cor}[thm]{Corollary}   % Turned off theorem numbering
 \newtheorem{prop}{Proposition}
 \newtheorem{defi}{Definition}
 \newtheorem{lem}[thm]{Lemma}
 \newtheorem{exmpl}{Example}
% \newtheorem{st}{Statement}
% \newtheorem{conj}{Conjecture}

%%%%%
%% robust recovery from sparse noise
\safemath{\dictab}{[\,\dicta\,\,\dictb\,]}


\safemath{\ysig}{\bmy}
\safemath{\ysighat}{\hat{\ysig}}
\safemath{\ysigdim}{M}
%
\safemath{\xsig}{\bmx}
\safemath{\xsigdim}{N}
\safemath{\nx}{n_x}
%
\safemath{\zsig}{\bmz}
\safemath{\zsigdim}{\ysigdim}
%
\safemath{\rsig}{\bmr}
%
\safemath{\Adict}{\bA}
\safemath{\Adicttilde}{\widetilde{\Adict}}
\safemath{\Adictdim}{\outputdim\times\xsigdim}
\safemath{\avec}{\bma}
\safemath{\avectilde}{\tilde{\avec}}
%
\safemath{\Bdict}{\bB}
\safemath{\Bdicttilde}{\widetilde{\Bdict}}
%
\safemath{\Cdict}{\bC}
\safemath{\cvec}{\bmc}
%
\safemath{\Ddict}{\bD}
\safemath{\Ddictdim}{\ysigdim\times\xsigdim}
\safemath{\dvec}{\bmd}
\safemath{\Ddicttilde}{\widetilde{\bD}}
%
\safemath{\Bonb}{\bB}
\safemath{\bvec}{\bmb}
\safemath{\Bonbdim}{\ysigdim\times\ysigdim}
%
\safemath{\noise}{\bmn}
\safemath{\noisedim}{\ysigim}
%
\safemath{\err}{\bme}
\safemath{\errdim}{\ysigdim}
\safemath{\errset}{\setE}
\safemath{\nerr}{n_e}
%
\safemath{\delop}{\bP_\errset}
\safemath{\delopc}{\bP_{{\errset}^c}}

%




%%
% Complex i and j 
\safemath{\cplxi}{\imath}
\safemath{\cplxj}{\jmath}
% Comb signal
%\safemath{\comb}{\matI\matI\matI}
\newcommand{\comb}[1]{\vecdelta_{#1}}
%:- Definition dictionary
\safemath{\dict}{\matD}
\safemath{\inputdim}{N}		% number of columns of dictionary D
\safemath{\outputdim}{M}		%number of rows of dictionary D
\safemath{\sparsity}{S}	%sparsity
\safemath{\inputdimA}{{N_a}}	%total number of elements in dictionary A
\safemath{\inputdimB}{{N_b}}	%total number of elements in dictionary B
\safemath{\elemA}{{n_a}}	%number of elements chosen from dictionary A
\safemath{\elemB}{{n_b}}	%number of elements chosen from dictionary B
\safemath{\resA}{\matR_a}	%restriction map to elements of dictionary A
\safemath{\resB}{\matR_b}	%restriction map to elements of dictionary B
\safemath{\subD}{\matS} %subdictionary
\safemath{\subA}{\matS_a} %subdictionary part of A
\safemath{\subB}{\matS_b} %subdictionary part of B
\safemath{\dicta}{\matA} 	% first subdictionary
\safemath{\dictb}{\matB} 	% second subdictionary
\safemath{\hollowS}{H}
\safemath{\hollowA}{H_a}
\safemath{\hollowB}{H_b}
\safemath{\cross}{Z}
\safemath{\coh}{\mu_d}			% coherence dictionary
\safemath{\coha}{\mu_a}			% coherence first subdictionary
\safemath{\cohb}{\mu_b}			% coherence second subdictionary
\safemath{\mubs}{\nu}	%block sub-coherence
\safemath{\cohm}{\mu_m} %mutual coherence
\safemath{\dictset}{\setD}	% set of dictionaries
\safemath{\dictsetp}{\dictset(\coh,\coha,\cohb)}	% set of dictionaries parametrized
\safemath{\dictsetgen}{\dictset_\text{gen}}
\safemath{\dictsetgenp}{\dictsetgen(\coh)}
\safemath{\dictsetonb}{\dictset_\text{onb}}
\safemath{\dictsetonbp}{\dictsetonb(\coh)}

\safemath{\leftside}{U}
\safemath{\rightsideA}{R_a}
\safemath{\rightsideB}{R_b}

\safemath{\indexS}{\setI_S} %set of indices participating in sub-dictionary S


\safemath{\na}{n_a}			% cardinality of set of linearly independent columns of first dictionary
\safemath{\nb}{n_b}			% cardinality of set of linearly independent columns of second dictionary
\safemath{\coeffa}{p_i}	%coefficients for columns of A
\safemath{\coeffb}{q_j}	%coefficients for columns of B
\safemath{\seta}{\setP}		% set of linearly independent columns of A
\safemath{\setb}{\setQ}     % set of linearly independent columns of B
\safemath{\setw}{\setW}	%set of n largest elements of w
\safemath{\setz}{\setZ}	%set of L-n largest elements of z
\safemath{\cola}{\veca}		% generic element of the dictionary A
\safemath{\colb}{\vecb}		% generic element of the dictionary B
\safemath{\cold}{\vecd}		% generic element of the dictionary D
\safemath{\inputvec}{\vecx} 	%coefficient vector (input)
\safemath{\error}{\vece}	%error vector
\safemath{\noiseout}{\vecz} 	%noisy output vector
\safemath{\inputvecel}{x}
\safemath{\inputveca}{\vecx_a}
\safemath{\inputvecb}{\vecx_b}
\safemath{\outputvec}{\vecy}	%output of Dictionary
\safemath{\lambdamin}{\lambda_{\mathrm{min}}}
%:- Math operators
\DeclareMathOperator{\spark}{spark}
\newcommand{\pos}[1]{\lefto[#1\right]^+}
\newcommand{\normtwo}[1]{\vecnorm{#1}_2}
\newcommand{\normone}[1]{\vecnorm{#1}_1}
\newcommand{\normzero}[1]{\vecnorm{#1}_0}
\newcommand{\norminf}[1]{\vecnorm{#1}_\infty}
\newcommand{\norminftilde}[1]{\vecnorm{#1}_{\widetilde\infty}}
\newcommand{\normfro}[1]{\vecnorm{#1}_F}
%\newcommand{\spectralnorm}[1]{\vecnorm{#1}_{2,2}}
\newcommand{\spectralnorm}[1]{\vecnorm{#1}}
\safemath{\elltwo}{\ell_2}
\safemath{\ellone}{\ell_1}
\safemath{\ellzero}{\ell_0}
\safemath{\ellinf}{\ell_\infty}
\safemath{\ellinftilde}{\ell_{\widetilde\infty}}
\safemath{\licard}{Z(\coh,\coha,\cohb)}
\safemath{\xsol}{\hat{x}}
\safemath{\xbord}{x_b}		%Solution at the border
\safemath{\xstat}{x_s}		%Solution stationary in l0 prob
\safemath{\xstatLone}{\tilde{x}_s}
\safemath{\order}{\mathcal{O}} %order notation (big O)
\safemath{\scales}{\Theta} %scales as
%
\safemath{\ones}{\mathbf{1}} %all ones matrix
\safemath{\zeroes}{\mathbf{0}} %all zeroes matrix
%
\safemath{\thlone}{\kappa(\coh,\cohb)} %treshold l1 problem
\safemath{\constoneA}{\delta} %constant in l1 theorem to save space
\safemath{\constoneB}{\epsilon} %constant in l1 theorem to save space
\safemath{\nlarge}{L}				   %num large elements
\safemath{\sumlarge}{S_\nlarge}
\DeclareMathOperator{\kernel}{kern}	   % kernel of a matrix
\safemath{\maxlarger}{P_\nlarge}	   % maximum in Gribonval and Nielsen
\safemath{\Pzero}{\textrm{P0}}	
\safemath{\Pone}{\textrm{P1}}
\safemath{\vecfir}{\vecw}			 % \vecv element of the kernel of the dictionary, \vecv=[\vecfir \vecsec]
\safemath{\vecsec}{\vecz}
\safemath{\elvecfir}{w}              % element of vecfir
\safemath{\elvecsec}{z}				 % element of vecsec
\safemath{\nlargefir}{n}
\safemath{\normout}{\gamma}
\safemath{\auxfun}{h}
\safemath{\supp}{\textrm{supp}}%support

\safemath{\indexa}{\ell}
\safemath{\indexb}{r}
\safemath{\indexc}{i}
\safemath{\indexd}{j}


\safemath{\project}{P}%projector


\def\lambdaexp{\frac{1}{2\lambda}} %lambda in the exponent, should it be \lambda or 1/(2\lambda)???
\def\lambdavar{\lambda}           %lambda in the variance, half the reciprocal of above

\DeclarePairedDelimiter{\abs}{\lvert}{\rvert}
\DeclarePairedDelimiter{\norm}{\lVert}{\rVert}



\newcommand{\cmark}{\ding{51}}%
\newcommand{\xmark}{\ding{55}}%

%\def\argmax{\mathop{{$\arg\max$}}\nolimits}
\def\id{\mathop{\rm id}\nolimits}
\def\cor{\mathop{\rm cor}\nolimits}
\def\sq{\mathop{\rm sq}\nolimits}
\def\sign{\mathop{\rm sign}\nolimits}
\def\half{{\scriptstyle\frac{1}{2}}}
\def\fourth{{\scriptstyle\frac{1}{4}}}
\def\threefourth{{\scriptstyle\frac{3}{4}}}
\def\Pr{\mathop{\rm Pr}\nolimits}

\def\logit{\mathop{\rm logit}\nolimits}
\def\pc{\mathop{\rm p\hspace{-.5pt}c}\nolimits}
\def\id{\mathop{\rm id}\nolimits}
%\def\KL{\mbox{Karhunen-Lo{\`eve}\,\! }}
\def\KL{\text{KL}}
\def\iprior{\mbox{I-prior\,\!}}
\def\GP{\mbox{\rm GP}}
\DeclareMathOperator{\MVN}{MVN}
\def\ve{\varepsilon}


\newcommand{\TV}{\mathop{\rm TV}\nolimits}
\newcommand{\tr}{\mbox{\rm tr}}
\newcommand{\erf}{\mbox{\rm erf}}
\newcommand{\erfi}{\mbox{\rm erfi}}
\newcommand{\inpr}[2]{\langle #1 , #2 \rangle}
\newcommand{\xf}{X_{\mbox{\scriptsize future}}}
\newcommand{\spa}{\mbox{\rm span}}
\newcommand{\fbm}{{FBM}}
\newcommand{\LL}{L} %for L^2 and L^1
\newcommand*\diff{\mathop{}\!\mathrm{d}}

\newcommand{\inn}{\langle\cdot ,\cdot\rangle}




%% Use the option review to obtain double line spacing
%% \documentclass[authoryear,preprint,review,12pt]{elsarticle}

%% Use the options 1p,twocolumn; 3p; 3p,twocolumn; 5p; or 5p,twocolumn
%% for a journal layout:
%% \documentclass[final,1p,times,authoryear]{elsarticle}
%% \documentclass[final,1p,times,twocolumn,authoryear]{elsarticle}
%% \documentclass[final,3p,times,authoryear]{elsarticle}
%% \documentclass[final,3p,times,twocolumn,authoryear]{elsarticle}
%% \documentclass[final,5p,times,authoryear]{elsarticle}
%% \documentclass[final,5p,times,twocolumn,authoryear]{elsarticle}

%% For including figures, graphicx.sty has been loaded in
%% elsarticle.cls. If you prefer to use the old commands
%% please give \usepackage{epsfig}

%% The amssymb package provides various useful mathematical symbols
\usepackage{amssymb}
%% The amsthm package provides extended theorem environments
%% \usepackage{amsthm}

%% The lineno packages adds line numbers. Start line numbering with
%% \begin{linenumbers}, end it with \end{linenumbers}. Or switch it on
%% for the whole article with \linenumbers.
%% \usepackage{lineno}

\journal{Econometrics and Statistics}

\begin{document}
	
	\begin{frontmatter}
		
		%% Title, authors and addresses
		
		%% use the tnoteref command within \title for footnotes;
		%% use the tnotetext command for theassociated footnote;
		%% use the fnref command within \author or \address for footnotes;
		%% use the fntext command for theassociated footnote;
		%% use the corref command within \author for corresponding author footnotes;
		%% use the cortext command for theassociated footnote;
		%% use the ead command for the email address,
		%% and the form \ead[url] for the home page:
		%% \title{Title\tnoteref{label1}}
		%% \tnotetext[label1]{}
		%% \author{Name\corref{cor1}\fnref{label2}}
		%% \ead{email address}
		%% \ead[url]{home page}
		%% \fntext[label2]{}
		%% \cortext[cor1]{}
		%% \address{Address\fnref{label3}}
		%% \fntext[label3]{}
		
		\title{Supplementary material for the article ``Regression with I-priors''}
		
		%% use optional labels to link authors explicitly to addresses:
		%% \author[label1,label2]{}
		%% \address[label1]{}
		%% \address[label2]{}
		
		%\author{Wicher P.\ Bergsma}
		%\address{London School of Economics, Houghton Street, London, WC2A 2AE, United Kingdom}
		
		\begin{abstract}
Simulations complementing the ones in Section~7 in the main article are presented. 
		
		%%Graphical abstract
%		\begin{graphicalabstract}
			%\includegraphics{grabs}
%		\end{graphicalabstract}
		\end{abstract}

		
	\end{frontmatter}
	
	%% \linenumbers
	
	%% main text


\section{Further simulations}\label{app-sim}

In the main article, we considered median absolute errors (MAEs) based on $L_2$ loss, summarized in Figure 5 there. Here we report to additional figures, visualizing the MAE based on two other norms, namely
\begin{itemize}
	\item $\text{MAE}(\cF_n):=\text{median}(\norm{\hat f-f}_{\cF_n})$
	\item $\text{MAE}(\cF):=\text{median}(\norm{\hat f-f}_{\cF})$
\end{itemize}
The simulation results are displayed in Figures~\ref{fig-sim1} and \ref{fig-sim2} using log-log plots of the MAE as a function of the error standard deviation.
As before, it is seen that the I-prior method always outperforms regularization, though the advantage of the former is small for the roughest functions in the RKHS (see the subfigures (a)).
Note that with respect to MAE($\cF_n$) (Figure~\ref{fig-sim1}) and not too small errors, regularization performs worse that a global constant fit (the horizontal `Baseline'). 
For rougher true regression functions in the RKHS, the I-prior estimator outperforms the SE estimator (the posterior mean under a square exponential process prior), which breaks down numerically for small errors. For analytic truths, the SE estimator outperforms the I-prior, as was to be expected. 



\begin{figure}[tbp]
	\centering
	\subfigure[True regression function has regularity 1]
	{\hspace*{0mm}\includegraphics[width=70mm]{sim_3priors_bound3.pdf}\hspace*{5mm}
		% \hspace{2mm}
		% \subfigure[Random walk prior]
		{\hspace*{0mm}\includegraphics[width=70mm]{sim_3priors_bound6.pdf}%\hspace*{5mm}
	}}
	\\
	\subfigure[True regression function has regularity 3/2]
	{\hspace*{0mm}\includegraphics[width=70mm]{sim_3priors_sm3.pdf}\hspace*{5mm}
		% \hspace{2mm}
		% \subfigure[Summed random walk prior]
		{\hspace*{0mm}\includegraphics[width=70mm]{sim_3priors_sm6.pdf}}}
	\\
	\subfigure[True regression function is a squared exponential Gaussian process path]
	{\hspace*{0mm}\includegraphics[width=70mm]{sim_3priors_exp3.pdf}\hspace*{5mm}
		% \hspace{2mm}
		% \subfigure[Exponential prior]
		{\hspace*{0mm}\includegraphics[width=70mm]{sim_3priors_exp6.pdf}}}
	\\
	\caption{Panels on left: simulated MAE($\cF_n$) for Tikhonov regularizer (`Reg'), I-prior estimator (`I-prior'), and SE estimator (`SE'). The baseline level is the MAE if the zero function is fitted.  
		Panels on right: ratio of MAE($\cF_n$) for regularizer and SE estimator compared to I-prior.
		Model~(1) in the main article is assumed with $\cF$ the FBM-1/2 RKHS and i.i.d.\  normal errors. }
	\label{fig-sim1}
\end{figure}


\begin{figure}[tbp]
	\centering
	\subfigure[True regression function has regularity 1]
	{\hspace*{0mm}\includegraphics[width=70mm]{sim_3priors_bound1.pdf}\hspace*{5mm}
		% \hspace{2mm}
		% \subfigure[Random walk prior]
		{\hspace*{0mm}\includegraphics[width=70mm]{sim_3priors_bound4.pdf}%\hspace*{5mm}
	}}
	\\
	\subfigure[True regression function has regularity 3/2]
	{\hspace*{0mm}\includegraphics[width=70mm]{sim_3priors_sm1.pdf}\hspace*{5mm}
		% \hspace{2mm}
		% \subfigure[Summed random walk prior]
		{\hspace*{0mm}\includegraphics[width=70mm]{sim_3priors_sm4.pdf}}}
	\\
	\subfigure[True regression function is a squared exponential Gaussian process path]
	{\hspace*{0mm}\includegraphics[width=70mm]{sim_3priors_exp1.pdf}\hspace*{5mm}
		% \hspace{2mm}
		% \subfigure[Exponential prior]
		{\hspace*{0mm}\includegraphics[width=70mm]{sim_3priors_exp4.pdf}}}
	\\
	\caption{Panels on left: simulated MAE($\cF$) for Tikhonov regularizer (`Reg'), I-prior estimator (`I-prior'), and SE estimator (`SE'). The baseline level is the MAE if the zero function is fitted.  
		Panels on right: ratio of MAE($\cF$) for regularizer and SE estimator compared to I-prior.
		Model~~(1) in the main article is assumed with $\cF$ the FBM-1/2 RKHS and i.i.d.\  normal errors. }
	\label{fig-sim2}
\end{figure}


%\bibliographystyle{elsarticle-harv}
%\bibliography{stats}


%\footnotesize
%\bibliographystyle{apacite}
%\bibliography{stats}

\end{document}
