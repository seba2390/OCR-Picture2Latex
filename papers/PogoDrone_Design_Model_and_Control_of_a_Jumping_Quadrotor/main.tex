%\documentclass[conference]{IEEEtran}
\documentclass[letterpaper,10pt,conference]{ieeeconf}
\IEEEoverridecommandlockouts
%\IEEEoverridecommandlockouts     
\overrideIEEEmargins


%\renewcommand{\baselinestretch}{0.999}
%\documentclass[a4paper]{article}
%% Language and font encodings
% \usepackage[english]{babel}
%\usepackage[normalem]{ulem}
%\usepackage[utf8]{inputenc}
\usepackage[T1]{fontenc}
\usepackage{float}
\usepackage{amsfonts}
\usepackage{amsthm}
\usepackage{pifont}
%% Sets page size and margins
%\usepackage[a4paper]{geometry}
\usepackage{xcolor}
\usepackage{soul}
% \theoremstyle{plain}
\newtheorem{theorem}{Theorem}
\newtheorem{definition}{Definition}
\newtheorem{proposition}{Proposition}
\newtheorem{remark}{Remark}
\newtheorem{problem}{Problem}
%% Useful packages
\usepackage{glossaries}
\usepackage{listings}
\usepackage{amsmath}
\usepackage{amssymb}
%\usepackage{algorithmicx}
%\usepackage{algpseudocode}
\usepackage{subfig}
% \algdef{SE}[DOWHILE]{Do}{doWhile}{\algorithmicdo}[1]{\algorithmicwhile\ #1}%
% \let\oldReturn\Return
% \renewcommand{\Return}{\State\oldReturn}
\usepackage{graphicx}
\usepackage{stix}
%\usepackage[colorinlistoftodos]{todonotes}
%\usepackage[colorlinks=true, allcolors=black]{hyperref}
%\setlength{\parindent}{4em}
%\setlength{\parskip}{1em}
\usepackage{tikz}
%\usepackage{tablefootnote}
%\usepackage[export]{adjustbox}
% \usepackage{algorithm}
\usepackage{etoolbox}
\usetikzlibrary{shapes,arrows}
\usepackage{verbatim}
\usepackage[hidelinks]{hyperref}


\renewcommand{\baselinestretch}{1.04}
% \makeatletter
% \setlength{\@fptop}{0pt}
% \makeatother

% \patchcmd{\thebibliography}{\section*{\refname}}{}{}{}
% \usepackage[sorting=none]{biblatex}
% \usepackage{csquotes}
% \addbibresource{ref.bib}
% \def\BibTeX{{\rm B\kern-.05em{\sc i\kern-.025em b}\kern-.08em
%     T\kern-.1667em\lower.7ex\hbox{E}\kern-.125emX}}

\newcommand{\fixme}[1]{{\color{red}FIXME: #1}}
\newcommand{\david}[1]{{\color{blue}#1}}
\newcommand{\remove}[1]{{\color{olive}Remove: #1}}
\newcommand{\andrew}[1]{{\color{green}#1}}
\newcommand{\jiawei}[1]{{\color{purple}#1}}
\newcommand{\brian}[1]{{\color{brown}#1}}
\newcommand{\deleted}[1]{}


\begin{document}
\title{\bf 
PogoDrone: Design, Model, and Control  of a Jumping Quadrotor
}
% \large{
% Design Parameters}
% }
%https://www.overleaf.com/project/60df761a27bb63831aee4b19
\author{Brian Zhu, Jiawei Xu, Andrew Charway, and David Saldaña
%\thanks{The authors acknowledge Andrew Charway at Lehigh University for his support and discussions about this project.}
\thanks{
B. Zhu, J. Xu , and D. Salda\~{n}a are with the Autonomous and Intelligent Robotics Laboratory (AIRLab), Lehigh University, PA, USA. Email:~$\{$\texttt{blz222, jix519, anc619, saldana\}@lehigh.edu}}
}
\maketitle
%\tableofcontents
\begin{abstract}
We present a design, model, and control for a novel jumping-flying robot that is called PogoDrone. The robot is composed of a quadrotor with a passive mechanism for jumping.
%
The robot can continuously jump in place or fly like a normal quadrotor. Jumping in place allows the robot to quickly move and operate very close to the ground.
For instance, in agricultural applications, the jumping mechanism allows the robot to take samples of soil.
We propose a hybrid controller that switches from attitude to position control to allow the robot to fall horizontally and recover to the original position.
%We evaluate our robot design with several simulations and experiments with actual robots.
We compare the jumping mode with the hovering mode to analyze the energy consumption.
In simulations, we evaluate the effect of different factors
on energy consumption. In real experiments, we show that our robot can repeatedly impact the ground, jump, and fly in a physical environment.
%


%Traditional quadrotors stay in a hovering state when no task is given, which drains energy constantly. This issue provided the main motivation behind the creation of PogoDrone, a hybrid multi-rotor system that switches between bouncing and hovering when idling. As a result, the battery is more efficiently used, and the maximum flight time is increased. Thus, this hybrid system will be able to perform more tasks per full battery charge. We verify our hypothesis using simulation and create a prototype of PogoDrone to experimentally test its energy consumption over a period of idling time with and without the bouncing behavior. Based on the results of the simulation, the PogoDrone consumes less power with the bouncing behavior, proving the effectiveness of the hybrid design in theory. 
 %The main motivation behind the creation of this hybrid system is increase maximum flight time and battery efficiency \andrew{more for motivation}.
 %provides an easier way to change values like the spring constant and also verify our hypothesis
\end{abstract}

\section{Introduction}
\noindent
%\brian{talk about ground effect}
In robotics, jumping mechanisms have been introduced based on bio-inspired locomotion principles~\cite{article,Truong_2019}. 
A jumping robot has a strong ability to overcome high obstacles~\cite{article}, but it is unable to stay in the air. 
Although many researchers have investigated miniature jumping robots over the last decade~\cite{6204349, woodward2011design, kovac2008miniature}, only a few have shown the integration of other locomotion modes. The  jumpglider~\cite{6181502} is one of the hybrid robots that successfully achieved the integration of two locomotion modes. Their robot is equipped with foldable gliding wings that can improve travel distance and reduce the impact on landing. Though jumping robots are able to move in a complex changeable environment with high obstacles, the height of their jumps is limited by their structural design.
For example, the flea-inspired jumping robot, designed by Noh et al.~\cite{6204349}, can jump over a height of no more than 30 times its body height.

Quadrotors have become popular across various industries and as a research topic in recent years. Some of the common applications include search and rescue operations~\cite{naidoo2011development}, exploration, aerial surveillance~\cite{LAGRING2012644}, and transportation~\cite{loianno2017cooperative}. One challenge that quadrotors must face is the ground effect~\cite{powers2013influence}, which inhibits the vehicle from operating close to the ground due to the turbulent flow produced by its rotors.
Additionally, a major challenge that researchers have to overcome is the short flight time~\cite{BOUKOBERINE2019113823}, which is typically around 5-30 minutes. To deal with this challenge, researchers have come up with strategies and techniques including using power sources with a higher energy density such as fuel cells~\cite{10.1115/IMECE2012-88871}, applying laser-beam in-flight recharging~\cite{8403572,6094731}, connecting with a power source through long-range tethering~\cite{MUTTIN2011332}, and swapping batteries in flight~\cite{9197580}.

Although adding a gliding capability to a jumping robot \cite{6181502} brings together the benefits of fast travel over a long distance in-air and energy-efficient obstacle clearance, the vehicle is not able to actuate in the air.
The integration of multi-rotors in a jumping system gives it more maneuverability in flight. 
In a recent publication~\cite{haldane2017repetitive}, the authors propose an active jumping mechanism with two rotors and a tail. The rotors are used to control roll and yaw angles, but they do not help the robot to stay in the air.

%When there is no need to keep the altitude or fast travel, the jumping mechanism can help preserve the energy for hovering by bouncing. This can be a considerable advantage for applications in search and rescue missions in cluttered environments, such as collapsed buildings~\cite{Truong_2019}. Due to the enhanced maneuverability of this jumping systems from the rotors, they can also be used in many industrial sectors like in the military for reconnaissance. 
%
%Picture of PogoDrone. Currently a place holder for the picture we will use
\begin{figure}[t!]
    \includegraphics[width=8.6cm]{figures/title.png}
    \caption{A PogoDrone during a bounce. It descends, compresses, rebounds, and flies back to its original location. The PogoDrone at the center shows the case when the spring is fully compressed.}
    \vspace{-1em}
    \label{fig:titlePic}
\end{figure}

Inspired by the combination of jumping and aerial robots, we propose a jumping-flying robot, called the \textit{PogoDrone}. 
We design, model, and control a novel robot that combines a quadrotor with a spring-based mechanism.
Fig.~\ref{fig:titlePic} shows the PogoDrone during a jump.
%The robot integrates of two locomotion modes: multi-rotor flight and jumping.
%The robot is a hybrid system composed of a quadrotor with a passive mechanism for jumping. 
One of the main principles of creating this hybrid system is the conservation of energy. Typically, when a normal quadrotor drops from a certain height and collides with the ground, the kinematic energy is dissipated through the collision. In contrast, when a PogoDrone makes contact with the ground, the kinetic energy is stored through the compression of a spring as potential energy. Upon decompression, the spring releases the energy back to the PogoDrone. Note that not all the energy is conserved in such a process because of the imperfection of the spring. Thus, we still need to compensate for the loss of energy by actuating the rotors.
%and then converted  back to kinetic energy.
% \brian{bold claim? maybe say an ideal pogodrone since we were unable to verify through real life experiments -> \jiawei{Not bold imo. We are saying the pogo may lose some energy but we can use the propellers to compensate the loss so that we recover the maximum height. }}
A PogoDrone requires less power to function due to the passive jumping capability that preserves energy, which increases energy efficiency and the maximum flight time.

In comparison to a single-mode jumping robot, such as the Salto-1P \cite{haldane2017repetitive}, we offer a significantly simpler design that can jump and fly continuously. The simplicity gives our robot superior maneuverability which actively flying and addressing the Salto-1P's inability to perform in scenarios that require continuous hovering. Compared to a traditional quadrotor, our PogoDrone allows operations close to the ground by deactivating its rotors when descending, which mitigates the ground effect problem~\cite{kushleyev2013towards}.
%\andrew{explicitly mention the relationship between battery vs actuation }

%\andrew{make contrast to normal drone} When the drone makes contact with the ground, energy is not lost to heat and sound, but is converted to potential and kinetic energy, which is \andrew{change wording} fed back into our system. 

% The next section describes the design of the PogoDrone. Section \ref{sec:Model} discusses the model of the robot, which includes the definition of coordinate frames and dynamics of the vehicle. Section \ref{sec:Control} presents the control of the robot. Section \ref{sec:simulation} outlines the results from simulating the behaviour of the PogoDrone. Section \ref{sec:Experiments} gives a detailed description of the results acquired from simulations and experiments. Section \ref{sec:conclusion} concludes the paper and describes plans for the future. 

%the total energy for actuating the propeller

%\david{1. Why should we do this? Importance of aerial vehicles... importance of jumping robots... }

%\david{2. What are the related works? *Salto, * spring-damper mechanisms, etc..
%What are the advantages and disadvantages of the related works?
%}

%\david{3. What is the main contribution of this work?}

\section{Design}
\label{sec:Design}
\noindent 
Our robot design adopts one of the simplest and most common methods to store potential energy, a spring in a pogo-stick.
%The combination of a spring's conservative properties with a micro quadrotor serves to be our extension of current aerial vehicle flight time. 
A PogoDrone is composed of a quadrotor fitted with a miniature and lightweight pogo-stick. The main components are described as follows.
%An advantage of our design that causes the ground effect to be negligible is as a result of the distance between the motors and the ground created by the addition of the pogo-stick.
\paragraph{Flying Vehicle} 
The quadrotor platform used for the PogoDrone is the Crazyflie 2.1. Open-source software and hardware, and its popularity in aerial robotics research make it an ideal choice for this project. The vehicle weighs 27g with the battery, and can carry a maximum payload of 15g. The dimensions are $92\times92\times29$~mm, and the 1-cell LiPo battery allows up to five minutes of hover time in the air. 
%An advantage of our design is that the distance between the motors and the ground, set by the pogo-stick, allows the ground effect to be negligible in our system. This alleviates an issue that many flying vehicles face.

%An advantage of the DC motors on crazyflies is that the thrust force they generate is so small, which causes negligible ground \brian{Ground effect would be negligible due to the distance between motors and pogo, not the small thrust force. This property is unique to our design.} \david{the ground effect is not negligible} effect even if the drone is close to the ground.


\begin{figure}[t]
\centering
    \includegraphics[width=0.8\linewidth]{figures/Pogo_schematics.png}
    \caption{Components of the PogoDrone}
    \vspace{-1em}
    \label{fig:pogodroneCAD}
\end{figure}
%
\paragraph{Pogo-stick Mechanism}
We show the parts of the miniature pogo-stick in Fig.~\ref{fig:pogodroneCAD}. 
The dimensions of the pogo-stick are $35.25\times11\times55$mm. It is comprised of four 3-D printed parts along with a spring.
The spring's length is $17.88$mm. 
%\david{How are we going to obtain that constant?}. 
The spring constant was obtained experimentally to be $394.58$N/m using weights and calipers. Using built-in mounting holes and a lightweight skeletonized frame, the primary T-shaped housing of the pogo-stick extends down the center of mass by approximately the same distance as the spring length, keeping the spring from generating unwanted torque when compressed. Our spring length, however, is constrained by the instability of the Crazyflie quadrotor. The longer the spring length, the longer the shaft and the lower the center of gravity. This increases the inertia tensor of the system which causes small changes in angular acceleration to generate large magnitudes of torque, thus making the system extremely difficult to fly. 
%\david{What is this text for?}\brian{The work in \cite{DING2019200} presents stability bounds such that a UAV's stability will not be compromised due to increased payload and shifted center of mass, however, this work's analysis on a quadrotor's ability to offset payloads through variable motor input does not aid in the PogoDrone's ability to stabilize attitude through dynamic movement.}
A plunger, held in by a threaded cap, inserts into the main shaft, compressing the spring on impact. Lastly, the semi-sphere foot at the end of the pogo-stick allows rotational motion from a pivot point. Given tight weight constraints, lightweight design of the pogo-stick is integral. Our current design weighs 4g, which is under the payload capacity of the Crazyflie (15g).



%\begin{figure}[t]
%\begin{center}
%\includegraphics[clip, trim=0.1cm 0.4cm 0.cm 0.cm, width=0.2\textwidth]{figures/PogoQuad-Assembly.png}&
%\end{center}
%\caption{TODO: change photo. The PogoDrone robot. It is a miniature pogo-stick that rigidly attaches to the center of the quadrotor. The plunger compresses the spring on impact, storing energy to be utilized in the quadrotor's ascent. This harnesses potential energy inherent in any aerial vehicle, thus saving battery and increasing flight time. } 
%\label{fig:Exp2}
%\end{figure}

\section{Model}
\label{sec:Model}
%Picture of PogoDrone
\begin{figure}[t!]
    \center
    \includegraphics[width=.7\linewidth]{figures/coordinate_frame.png}
    \caption{The PogoDrone with its coordinate frame}
    \vspace{-1em}
    \label{fig:coordinatePic}
\end{figure}
\noindent
Our PogoDrone is composed of a quadrotor with a bottom side attached jumping device, modeled as a spring-damper. 
The robot has a mass $m$ and an inertia tensor $\boldsymbol I$. Since the mass of the jumping mechanism is small in comparison with the quadrotor, we assume that the center of mass of the PogoDrone is at the same location as the center of mass of the quadrotor.
The coordinate frame for the quadrotor is denoted by $\{B\}$ and its origin is at the center of mass of the quadrotor. 
The $x$-axis is aligned with the front of the quadrotor and the $z$-axis is pointing upwards. 
The spring-damper mechanism has one of its ends attached to the origin of $\{B\}$ and extends along the negative $z$-axis of $\{B\}$ (see Fig.~\ref{fig:coordinatePic}). The pogo has a natural length $l_0$ and a minimum length $l_{min}$.
The world coordinate frame is denoted by $\{W\}$ with the $z$-axis pointing upwards. We denote the rotation matrix from body frame to world frame as~${}^W\!\!\boldsymbol{R}\!_B\in\mathsf{SO(3)}$.

\subsection{Sensors and Actuators}
\noindent
We use internal and external sensors to measure the location of the robot $\boldsymbol{r}\in \mathbb{R}^3$ in $\{W\}$, velocity $\boldsymbol{v}\in\mathbb{R}^3$, attitude ${}^W\!\!\boldsymbol{R}\!_B$, and angular velocity $\boldsymbol{\omega}\in\mathbb{R}^3$.
%
The robot has four motor, as active actuators, that generate a thrust $f_q\in \mathbb{R}$ along the $z$-axis of the body frame and a torque $\boldsymbol\tau \in \mathbb{R}^3$.
Therefore, our control input is the tuple $(f_q, \boldsymbol\tau)$ which, through motor power distribution, can map to individual motor forces~\cite{5717652}.
In addition, the pogo mechanism, as a passive actuator, generates a force,
\begin{equation}
f_{s} = - k\Delta L - b\frac{d}{dt}\Delta L,
\label{eq:spring}
\end{equation}
where $k>0$ is the spring constant, $b>0$ is the damping factor, and $\Delta L=l - l_0$ is the amount of deformation of the spring. Since the spring is aligned with $z$-axis of the quadrotor, it does not generate any torque. Although the robot design has a spring and no damper, we included the damping term to take into account the energy lost due to friction and spring imperfections.
The total force of the robot is along $z$-axis in~$\{B\}$,
\begin{equation}
f = f_q + f_s.
\end{equation}


\subsection{Dynamics}
\noindent
The dynamics of the robot depends on whether or not the PogoDrone makes a contact with the floor. The robot can be either in \textit{a)} a flying state or \textit{b)} a contact state.
We describe the dynamics of the vehicle with Newton-Euler's equations.

\textit{a) Flying State}
\begin{eqnarray}
    m\,\boldsymbol{\Ddot{r}}+m\,g\boldsymbol{  e}_3 &=& {f}\: {}^W\!\!\boldsymbol{R}\!_B\,\boldsymbol e_3\label{eq:newton},\\
    \boldsymbol{I}\Dot{\boldsymbol\omega}+\boldsymbol\omega\times \boldsymbol{I}\boldsymbol\omega &=& \boldsymbol\tau\label{eq:euler},
\label{eq:dynamics}
\end{eqnarray}
where $\boldsymbol{e}_3 = [0, 0, 1]^\top$ is a standard unit vector in $\mathbb{R}^3$ along $z$-axis. In this state, the robot does not have contact with the ground, and therefore, there is no spring compression, i.e,~$f_s = 0$.

\textit{b) Contact State}
During contact, the connection between the pogo and the floor can be modeled as a spherical joint. Therefore, the robot rotates around the contact point instead of the center of mass. The Newton equation \eqref{eq:newton} holds the same but we need to consider that force provided by the floor is transferred through the spring to the PogoDrone as $f_s\neq 0$. 
The Euler equation has some differences from \eqref{eq:euler} since the rotation point changed,
\begin{equation}
    \boldsymbol{I'\Dot\omega}+\boldsymbol\omega\times \boldsymbol{I'}\boldsymbol\omega = \boldsymbol\tau + {}^W\!\!\boldsymbol{R}_B(l_0+\Delta L)\boldsymbol{e}_3\times(-mg\boldsymbol{e}_3),
    \label{eq:neweuler}
\end{equation}
where $\boldsymbol{I'}$ is the moment of inertia about the contact point using the parallel axis theorem. In the experiments, the inertia is not adjusted online because the contact state is ephemeral, i.e., according to Section \ref{sec:Experiments}, the contact state lasts less than $100$ ms for each bounce. Therefore, we rely on the attitude controller to compensate for any shifts in the inertia matrix.
Note that there is an additional term due to the moment generated by the gravity force. 
We assume no slipping during the contact between the pogo tip and the ground.


\section{Control}
\label{sec:Control}
\begin{figure}[t]
    \includegraphics[width=\linewidth]{figures/pogo.phases.pdf}
    \caption{Phases of a jump: (a) descend, (b) compression, (c) rebound, and (d) ascend.}
    \label{fig:ControlPhases}
\end{figure}

\noindent
The control objective focuses on allowing the PogoDrone to jump repetitively from a desired position $\boldsymbol{r}^d$ with a desired yaw orientation $\psi^d$. Our robot passes through four phases during each jump (illustrated in Fig.~\ref{fig:ControlPhases}):\\
%
\textit{a) Descend: } The robot falls maintaining its attitude to approach the ground vertically, so the spring will be perpendicular to the ground. During this phase, the propellers are used to maintain attitude only, so the robot can fall vertically to convert as much gravitational potential energy into kinetic energy.\\
%
\textit{b) Compression: } The robot contacts with the ground and the spring compresses until it reaches its maximum compression length. During this phase, the spring converts the kinetic energy accumulated during the descend phase into elastic potential energy.\\
%
\textit{c) Rebound: } The spring expands, pushing the robot upwards. The spring converts the stored energy into kinetic energy. Note that we apply a realistic spring model in~\eqref{eq:spring}, meaning that there exists a loss of energy during the compression and rebound. The amount of energy that we can store with this strategy will depend on the length, stiffness of the spring, loss of energy due to friction, and spring imperfections.\\
%
\textit{d) Ascend: } The robot ascends to its original point $\boldsymbol{r}^d$. Since there is a loss in energy, the robot uses its propellers to reach the goal location.


\begin{figure}[t]
    \centering
    {\includegraphics[width=1.0\linewidth]{figures/pogo.control.pdf}
    \caption{Control Diagram for the PogoDrone. The BHC block reads the current position and angular velocity of the PogoDrone and switches the control mode between the full geometric control and attitude-only control as described in Section \ref{sec:Control}. 
    }
    \vspace{-1em}
    \label{fig:Control}}
\end{figure}
\begin{figure}[b]
\centering
    \includegraphics[width=0.6\linewidth]{figures/pogo.bhc.pdf}
    \caption{Flowchart of the Bounce-Hover Controller (BHC).}
    \label{fig:flowchart}
\end{figure}
We rely on the geometric  controller on $\mathsf{SE(3)}$~\cite{5717652, 5980409} for position and attitude control. We illustrate the control architecture on Fig.~\ref{fig:Control}.
Our module Bounce-Hover Controller (BHC) switches between the position and attitude controller depending on the phase of the jump. Our PogoDrone acts as a mass-damper-spring system when passively jumping, and as a quadrotor when actively hovering. 
%
During (a) descend and (b) compression, the robot does not need to compensate gravity or maintaining the position, so we use an attitude controller with $\boldsymbol{R}=\text{Rot}_z(\psi^d)$ to guarantee that the robot will stay horizontal, roll and pitch equal to zero, while holding its desired orientation in yaw.
Since we are not compensating for gravity, we do not need to generate a thrust force.
However, the geometric controller does not accept $f_q=0$ as an input, so we define an infinitesimally small constant  $\epsilon>0$ such that $f_q=\epsilon$.
%
During (c) rebound and (b) ascending, the PogoDrone uses the stored energy to push upwards 
and the quadrotor's force to fly back to the original position. For this purpose, the PogoDrone uses position control with $\boldsymbol{r}^d$ and $\psi^d$ as an input.

Our switching policy for the PogoDrone is defined by the algorithm in the BHC module.
We summarize our algorithm in the flowchart in Fig.~\ref{fig:flowchart}. 
The algorithm starts using the position controller to move the robot from any position to the desired position $\boldsymbol{r}^d$ with the desired orientation $\psi^d$.
%
Due to the error in the position controller, we use a sphere with center at center~$\boldsymbol{r}^d$, and radius $r>0$.
% \brian{one reviewer requests a different symbol for radius? I think it may be okay...}
If the robot is within the sphere $\|\boldsymbol{r}^d-\boldsymbol{r}\|\leq r$ and the magnitude of its  angular velocity  $\|\boldsymbol{\omega}\|$ is less than a threshold, then it will start descending, leaving the sphere.
During the descend and compression, the robot uses the attitude controller until the rebound starts. At that moment, the robot switches from attitude to position control and comes back to the original position $\boldsymbol{r}^d$. This iterative process continues repeating periodically.






%A hybrid controller applies for this objective, switching modes according to the PogoDrone's current operation mode, \textit{a)} jumping or \textit{b)} flying. Each mode by itself is commonly modelled and controllable, i.e., the 


%Only by carefully controlling the transitions between the two modes can we reveal the advantages of this hybrid system. 


% Energy efficiency during hovering, for example, is one of the advantages when the task and environment do not constrain the PogoDrone in its vertical motion. We introduce the control strategy with which the PogoDrone can conserve energy for hovering by utilizing the continual conversion between gravitational potential energy, elastic potential energy, and kinetic energy through the jumping behavior. The strategy is divided into three stages back-to-back: Falling, Contact and Ascending. All three phases utilize the geometric controller on $\mathsf{SE(3)}$~\cite{5717652, 5980409} for controlling the quadrotor\brian{'s attitude, while a linear PID controller is utilized for position control}. A higher-level controller, shown in Fig. \ref{fig:Control} as the ``BHC'' block, takes in the desired hover position, monitors the state of the PogoDrone, and selects between two types of geometric controller based on the current operation mode.

% In the Falling phase, illustrated as (a) in Fig. \ref{fig:ControlPhases}, after the PogoDrone stably hovers, it starts falling vertically downwards. The BHC block triggers the \emph{Balance Controller}, which takes in the desired yaw angle and outputs the torque $\boldsymbol\tau$ that is needed to track the yaw angle $\psi_d$ while driving the pitch and roll angles to $0$. The Balance Controller does not compensate the position error or the gravity as we intend to convert as much gravitational potential energy into kinetic energy before the impact. Thus, the feedback only compensates for the attitude. Once the pogo touches the floor, the strategy enters the Impact phase. 

% The Impact phase, illustrated as (b) in Fig. \ref{fig:ControlPhases}, consists of two main stages: compression, and rebound. During the compression stage, the PogoDrone converts the kinetic energy accumulated in the Falling phase into the elastic potential energy of the compressed spring. Then in the rebound stage, the spring converts that energy into the kinetic energy. Note that we apply a realistic spring model in \ref{sec:Model}, meaning that there exists loss of energy during the Impact phase. How much energy we can conserve with this strategy depends partly on this portion of loss. During this phase, only Balance Controller is applied. The necessary condition for the PogoDrone to transition from the Impact phase to the Ascending phase is when the PogoDrone crosses a height $h_{b}$ after the compression, which makes BHC to trigger the Rebound Controller. The rebound controller is a full-state geometric controller for the quadrotor that drives the PogoDrone to the desired hover position $\boldsymbol{r}_d$. Ideally, we would like $h_b = l_0$, which makes the propellers to actuate right after the spring reattains its natural length. 

% In the Ascending phase, illustrated as (c) in Fig. \ref{fig:ControlPhases}, the propeller forces drive the PogoDrone to the desired hover position $\boldsymbol{r}_d$, resulting in a transition back to the Falling phase. Two conditions must be met for the PogoDrone to complete the transition: \emph{Proximity} and \emph{Stability}.

%\jiawei{Enrich the reasoning}



\section{Simulation and Experiments}
\noindent

We evaluate our PogoDrone model and design in simulations and actual robots\footnote{
        The simulator and ros packages can be found at: \url{https://github.com/swarmslab/PogoDrone}
    }.
We analyze the energy consumption  in a time interval $[t_0, t_f]$.
Since the current is proportional to the force generated by the spinning propellers, we use the rotors' force as a performance metric,
% \brian{When choosing an energy consumption model to evaluate performance, we considered (15) defined in \cite{8594419}. Our platform, however, is unable to provide measures of voltage and current across each motor. This motivates the utilization of the proportionality of energy consumption to the force generated by the motors. Our solution metric for evaluation counts the force generated by the motors in a time interval $[t_0, t_f]$, }
%To allow the units of this result to be more understandable, we convert the PWM motor inputs to be expressed as a force in grams.
%This allows us to evaluate energy consumption of the system over time.
\begin{eqnarray}
e(t_f)= \int_{0}^{t_f} \sum_{i=1}^4 f_i(t) \: dt.
\label{eq:energy}
\end{eqnarray}
where $f_i$ is the force generated by the rotor $i$ at time $t$.
The final time $t_f$ defines the duration of the experiment with multiple jumps.


    

\subsection{Simulations}{
\label{sec:simulation}
\noindent
%With simulations, we verify the feasibility of the PogoDrone design and show that the PogoDrone is able to conserve energy through bouncing in comparison to hovering.
We compare the PogoDrone in jumping mode versus hovering mode. We run the simulations in batch to evaluate the factors that affect the energy consumption of the PogoDrone: \textit{a)} Noise level, \textit{b)} Spring constant, \textit{c)} Damping factor, \textit{d)} Hover height.

We developed a 2D simulation environment in Python. We implement the dynamics of the PogoDrone and its interaction with the ground surface as in Section \ref{sec:Model} using the Euler integration method with a time step of 1 millisecond. In the planar case, the PogoDrone has two propellers, making it an under-actuated system in 2-D, similar to a quadrotor in a 3-D environment. We first find a setting with the four factors where the PogoDrone
saves
energy with bouncing. Fig.~\ref{fig:simgood} shows the 2-norm of rotor forces, the position of the PogoDrone during the bounces, the spring lengths, and its forces. The numerical values of the four factors used in the reference is shown in the caption. Note that the values stated in Fig. \ref{fig:simgood} are close to, but not precisely the values used in the real robot experiments. The spring constant and hover height are comparable. It is challenging, however, to choose a spring with a specific damping factor since it is difficult to measure. Through different simulations and real experiments, we approximated a damping factor into the simulation that closely reflected the behavior of our real spring. For each numerical value selection of a factor, we run the simulation $20$ times for both normal hovering and bouncing mode while keeping all other factors invariant, and compare \eqref{eq:energy} from the two modes with $t_f = 18 s$.
\begin{figure}[t!]
\centering
    \includegraphics[width=\linewidth]{figures/simgood.pdf}
    \caption{The performance of the reference simulation, which saves 12\% of the energy from hovering with two bounces. The numerical values of the four factors are: noise level = 0.2, spring constant = 400 N/m, damping factor = 1.0 Ns/m, hover height = 0.8 m.} 
    \label{fig:simgood}
\end{figure}

\noindent
\paragraph{Noise level}{
In the experiments with real robots, we notice two unmodelled variants in Section \ref{sec:Model} that keep the PogoDrone from achieving stability quickly. First, the uneven surface of the ground leads to the force received by the pogo having a perpendicular component to the pogo stick when touching the ground, causing large overshoots in $xy$-plane. Second, the force and torque generated by the propellers deviate from the desired input value $f_q$ and~$\boldsymbol{\tau}$ because of the imperfect motors, adding up to the loss of energy in achieving stability. As a result, the ascend phase lasts longer which keeps the motors actuating and increases the energy consumption. To emulate the errors from the input, we add Gaussian noise with standard deviation $\sigma$ to the propeller forces $f_i(\boldsymbol{u})$. For the imperfect ground, we add a Gaussian noise with standard deviation $5\sigma$ to the direction of the spring force such that the spring force aligns with $\text{Rot}_z(G){}^W\!\!\boldsymbol{R}\!_B\boldsymbol{e}_3$, where $G \sim \mathcal{N}(0, (5\sigma)^2)$, instead of ${}^W\!\!\boldsymbol{R}\!_B\boldsymbol{e}_3$.
%
We show in Fig. \ref{fig:simnl} that as the noise level increases, the energy consumption of both modes increases, and for the bouncing mode, it increases faster until catching up with the hovering mode. When the noise level is sufficiently high, the PogoDrone never achieves stability, making the propellers always actuated.
\begin{figure}[t!]
    \includegraphics[width=8.6cm]{figures/simnl.pdf}
    \caption{Energy consumption v.s. noise level. A Gaussian noise is applied on the thrust forces of the propellers and the force direction of the pogo, emulating imperfect motors and ground surface.}
    \label{fig:simnl}
\end{figure}
}

\noindent
\paragraph{Spring constant}{
The spring constant  determines the stiffness of the spring.
%A low constant make the spring behave
%determines the maximum energy the pogo can store. If the spring in the pogo has a small spring constant, the pogo tends to get fully compressed while descending, in which case the motion of the PogoDrone will change into a hard impact with the ground which absorbs energy and surges propeller power to reattain stability. If the spring constant is large, the pogo tends to function as a rigid stick, converting the kinetic energy into rotations centered at the contact point, which requires the PogoDrone to spend more energy recovering the attitude in ascend phase.
%
In Fig. \ref{fig:simsc}, we show the pattern of how the energy consumption changes as the spring constant gets higher. When the spring constant is low, the energy consumption decreases as the spring constant goes up because the spring can store an increasing amount of energy that would otherwise be lost during the hard impact. When the spring constant goes higher, the controller over-compensates the position in $z$, as shown in Fig. \ref{fig:simgood}. 
%
From 200 to 300 N/m is a critical interval (same as 700 to 900 {N/m}) because the number of jumps in the time interval can change depending on the noise. 
Note that between 300 and 700 N/m the energy consumption remains stable because the bouncing behavior couples with the controller tuning. After 900, the energy consumption slowly increases due to the damping term that prevents the spring from rebounding faster. When the spring constant is sufficiently high, then energy spent on recovering stability surpasses the energy conserved by the spring, which supports our hypothesis. The energy consumption of hovering is not affected because the pogo does not make contact with the ground.
\begin{figure}[t!]
    \includegraphics[width=8.6cm]{figures/simsc.pdf}
    \caption{Energy consumption v.s. spring constant. The higher spring constant, the more energy can be stored in the pogo and the stiffer the pogo is.}
    \label{fig:simsc}
\end{figure}
}

\noindent
\paragraph{Damping factor}{
In Section \ref{sec:Model}, we model the pogo as an imperfect spring, which induces energy loss from mechanical motion. The loss is in form of a damping term, which prevents the pogo from being compressed or rebounding quickly and therefore limits the maximum velocity of the PogoDrone after the rebound. 
%
Fig. \ref{fig:simdf} shows that as the damping factor increases, the energy consumption of the bouncing mode first decreases, then increases and stabilizes at a value lower than that of hovering. We notice that with a perfect spring whose damping factor is ignored or very small, the energy consumption becomes even higher than that of hovering. In theory, when a perfect spring is inserted in the pogo, to recover the same height before falling, the propellers do not need to actuate, since the elastic potential energy stored in the spring is equal to the gravitational energy before falling. However, as described in Section \ref{sec:Control}, the BHC activates the propellers to drive the drone to the desired height $z^d$ upon entering rebound phase despite the state of the pogo. Thus, the redundant energy, provided by the propellers, creates an overshoot in the height, requiring more energy to compensate. 

\begin{figure}[t!]
    \includegraphics[width=8.6cm]{figures/simdf.pdf}
    \caption{Energy consumption v.s. damping factor.}
    \label{fig:simdf}
\end{figure}
}

\noindent
\paragraph{Desired height}{
The height of the desired position~$\boldsymbol{r}^d$ determines the maximum amount of energy that needs to be stored in the pogo during a bounce. 
%
As shown in Fig. \ref{fig:simdh}, when the desired height is below 1.5 m, the energy consumption of the PogoDrone is high because of the overshoots created by the controller. As the desired height goes higher, the energy consumption goes down to the low point at 1.5 m when the spring reaches maximum compression when storing all gravitational potential energy. The sudden change between 1.5 m and 2 m is caused by the energy loss due to hard impacts. After 2 m, the trend acts similar to Fig. \ref{fig:simsc}.
%
{Our simulations verify that energy efficiency is strongly dependent on the spring constant, noise level, and damping coefficient. Fine tuning of the system achieves an optimal amount of energy saved but the jumps should perform under very specific conditions.
%Solutions to this problem may exist within an active, variable damper mechanism.
}

\begin{figure}[t!]
    \includegraphics[width=8.6cm]{figures/simdh.pdf}
    \caption{Energy consumption v.s. desired height. The initial desired height determines the initial potential energy.}
    \label{fig:simdh}
\end{figure}
}

}




\subsection{Experiments with an actual robot}
%\david{This text should be aligned with the previous sections}


We studied the physical system's efficiency and dynamic performance through five alterations of the desired height.
%
In our testbed, we used our robot design, described in Sec.~II, and the Crazyflie-ROS package to control the robot's actions. This package communicates with our Optitrack motion capture system, operating at 60 Hz. Our computer runs off-board position and attitude controls through ROS nodes, and these commands are then received by the PogoDrone via a 2.4~GHz radio.
\label{sec:Experiments}
\begin{figure}[t!]
    \includegraphics[width=8.6cm]{figures/motorpowerplot3.pdf}
    \caption{Motor power of the system while hovering and bouncing at each tested height. The top graph shows 0.3 m and each graph below represents the height incremented by 0.1 m.}
    \label{fig:motorpower}
\end{figure}
%
Fig. \ref{fig:motorpower} shows the force of the motors over time for each height. We find that the optimal height for this bouncing system is at 0.3 m. At this height, energy is not lost to over-compression and the time of recovery in between the bouncing cycle is minimal. As we increase the height, we find that our spring is not able to store the entirety of the potential energy which leads to over-compression. Similar to the simulations, the hard impact with the ground requires the motors to compensate more due to the energy loss. Additionally, the system may generate a torque along the $x$-axis or $y$-axis. This leads to an imperfect bounce which further extends recovery time. At our inflection point of 0.4 m, we find that the culmination of these negative forces reward us with only a slight advantage in efficiency while bouncing.
% \begin{figure}[t!]
%     \includegraphics[width=8.6cm]{figures/3dmxyz.pdf}
%     \caption{Bouncing at 0.3 m}
%     \label{fig:Bouncing at 0.3 meters}
% \end{figure}
% %
% \begin{figure}[t!]
%     \includegraphics[width=8.6cm]{figures/5dmxyz.pdf}
%     \caption{Bouncing at 0.5 m}
%     \label{fig:Bouncing at 0.5 meters}
% \end{figure}
At 0.5 m, we find that bouncing performs worse than simply hovering. This is primarily due to over-compression, imperfect bounces, and significant recovery time. 
% Figures \ref{fig:Bouncing at 0.3 meters} and \ref{fig:Bouncing at 0.5 meters} show the differences in the PogoDrone's position and attitude over time. The amount of deviation along the $xy$-plane is significantly higher at 0.5 m versus 0.3 m. This reduces the frequency of the bouncing cycle which thus lowers the total amount of converted potential energy.
%
At 0.6 and 0.7 m, the efficiency performance is approximately equal for hovering and bouncing. This is caused by an extremely high time of recovery in-between bounces. Over the same time cycle, the PogoDrone spends most of the time recovering which leads to comparable energy performance. Similar to the previous experiments, these high recovery times are caused by over-compression and imperfect bounces along the z-axis. In addition, as the PogoDrone falls from higher heights, our attitude controller must compensate for more air resistance that wishes to create a torque on our system. 
%
We assumed a perfectly smooth and level surface to bounce on in our experiments. Our surface additionally is rigid which decreases the amount of energy loss per bounce due to damping or absorption.





\section{Conclusion and Future work}
\label{sec:conclusion}
This paper introduces PogoDrone, a  robot that
is able to jump and fly.
We designed a prototype that works in actual environments. 
Using the modeled dynamics, we simulated and analyzed the behavior of the robot after changing different factors.
{We verified the concept of a dual locomotion quadrotor. Our robot is able to rapidly impact with the ground plane and return to a hover state, making it uniquely equipped to handle operations such as taking soil samples or planting seeds. Other multi-rotor vehicles would have to operate close to the ground for a prolonged period of time, making them susceptible to the ground effect.
%The robot also obtains the ability to become more energy efficient, though our actual experiments showed that its performance was heavily impacted by the experimental setup.
}
We found that the robot can quickly become energy inefficient
when the spring constant or the actuators' noise is high. 
In conclusion, the robot has the potential to be considerably more efficient than a hovering robot (>20\%), but it requires finding an optimal hardware setup and control gains for the actual robot.

%We found that bouncing does indeed improve the battery efficiency, however, the range of success is narrow. One of the primary challenges encountered was the limitations of the spring length and therefore the total amount of potential energy converted per bounce. As mentioned in the design section, the spring length was constrained by the Crazyflie drone's ability to fly with a lowered center of mass. Though the Crazyflie platform is easy to use and well known, it requires extensive amounts of re-tuning when adding the pogo-stick.


Our robot has the potential to efficiently operate in low heights, so our future work is focused on examining the system's performance on imperfect surfaces which should be explored for real-world applications.
%\brian{Future work should also develop a better spring mechanism for the quadrotor. The energy efficiency is most dependent on the recovery time which was impacted by friction, weight, and damping of the imperfect spring. (mention carbon fiber?)}

%Future work also includes implementation on a more robust robot whose ability to fly is not severely impaired by the addition of the pogo-stick.  


\bibliographystyle{IEEEtran}
\bibliography{ref}

\end{document}
