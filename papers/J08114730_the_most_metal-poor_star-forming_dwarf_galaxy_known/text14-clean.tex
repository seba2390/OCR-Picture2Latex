% mn2esample.tex
%
% v2.1 released 22nd May 2002 (G. Hutton)
%
% The mnsample.tex file has been amended to highlight
% the proper use of LaTeX2e code with the class file
% and using natbib cross-referencing. These changes
% do not reflect the original paper by A. V. Raveendran.
%
% Previous versions of this sample document were
% compatible with the LaTeX 2.09 style file mn.sty
% v1.2 released 5th September 1994 (M. Reed)
% v1.1 released 18th July 1994
% v1.0 released 28th January 1994

\documentclass[useAMS,usenatbib]{mn2e}
\usepackage{graphicx}
\usepackage{color}
\usepackage{ulem}
% If your system does not have the AMS fonts version 2.0 installed, then
% remove the useAMS option.
%
% useAMS allows you to obtain upright Greek characters.
% e.g. \umu, \upi etc.  See the section on "Upright Greek characters" in
% this guide for further information.
%
% If you are using AMS 2.0 fonts, bold math letters/symbols are available
% at a larger range of sizes for NFSS release 1 and 2 (using \boldmath or
% preferably \bmath).
%
% The usenatbib command allows the use of Patrick Daly's natbib.sty for
% cross-referencing.
%
% If you wish to typeset the paper in Times font (if you do not have the
% PostScript Type 1 Computer Modern fonts you will need to do this to get
% smoother fonts in a PDF file) then uncomment the next line
%\usepackage{times}

%%%%% AUTHORS - PLACE YOUR OWN MACROS HERE %%%%%
\def\apj{ApJ}
\def\apjs{ApJS}
\def\apjl{ApJL}
\def\aap{A\&A}
\def\aaps{A\&AS}
\def\aj{AJ}
\def\mnras{MNRAS}
\def\apss{Astr.Space Sci.}
\def\pasj{PASJ}
\def\pasp{PASP}
\def\nat{Nature}
\def\na{New A}
\def\araa{Ann.Rev.Astron.Astrophys.}
\def\pra{Phys.Rev.A}
\def\prb{Phys.Rev.B}
\def\prc{Phys.Rev.C}
\def\prd{Phys.Rev.D}
\def\rmxaa{Revista Mexicana de Astronomia y Astrofisica}
\def\sci{Science}
\def\bb{\color{blue}}

%%%%%%%%%%%%%%%%%%%%%%%%%%%%%%%%%%%%%%%%%%%%%%%%

\title[J0811$+$4730: the most-metal poor galaxy]
{J0811$+$4730: the most metal-poor star-forming dwarf galaxy known}
\author[Y. I. Izotov et al.]{Y. I.\ Izotov$^{1}$,
T. X.\ Thuan$^{2}$, N. G.\ Guseva$^{1}$ and S. E.\ Liss$^{2}$\\
                $^{1}$Main Astronomical Observatory,
                     Ukrainian National Academy of Sciences,
                     Zabolotnoho 27, Kyiv 03143,  Ukraine,\\
                     izotov@mao.kiev.ua, guseva@mao.kiev.ua\\
                $^{2}$Astronomy Department, University of Virginia, 
                     P.O. Box 400325, Charlottesville, VA 22904-4325,\\
                     txt@virginia.edu, sel7pa@virginia.edu\\
}
\begin{document}

%\date{Accepted 1988 December 15. Received 1988 December 14; in original form 1988 October 11}

\pagerange{\pageref{firstpage}--\pageref{lastpage}} \pubyear{2012}

\maketitle

\label{firstpage}

\begin{abstract}
We report the discovery of the most metal-poor dwarf star-forming galaxy (SFG)
known to date, J0811$+$4730. 
This galaxy, at a redshift $z$=0.04444, has a Sloan Digital Sky
Survey (SDSS) $g$-band absolute magnitude $M_g$ = $-$15.41 mag. It
was selected by inspecting the spectroscopic data base in the Data 
Release 13 (DR13) of the SDSS. 
LBT/MODS spectroscopic observations reveal its oxygen abundance to be 
12 + log O/H = 6.98 $\pm$ 0.02, the lowest ever observed for a SFG. 
J0811$+$4730 strongly deviates from the main-sequence defined by 
SFGs in the emission-line diagnostic diagrams and the 
metallicity -- luminosity diagram. These differences are caused mainly
by the extremely low oxygen abundance in J0811$+$4730, which is $\sim$ 10 times
lower than that in main-sequence SFGs with similar luminosities. 
By fitting the spectral energy distributions of the 
SDSS and LBT spectra, we derive a stellar mass of $M_\star$ = 
10$^{6.24} - 10^{6.29}$ M$_\odot$
(statistical uncertainties only), and we find that a considerable fraction of
the galaxy stellar mass was formed during the most recent burst of star 
formation.
\end{abstract}

\begin{keywords}
galaxies: dwarf -- galaxies: starburst -- galaxies: ISM -- galaxies: abundances.
\end{keywords}



\section{Introduction}\label{sec:INT}

Extremely metal-deficient 
star-forming galaxies (SFGs) with oxygen abundances
12+log~O/H$\la$7.3 constitute a rare but important class of galaxies
in the nearby Universe. They are thought to be the best local analogs of the
numerous population of the dwarf galaxies at high 
redshifts that played an important role in the reionization of the Universe 
at redshifts $z$ $\sim$ 5--10 \citep[e.g. ][]{O09,K17}. 
Their proximity permits studies of their stellar,
gas and dust content with a sensitivity and spectral resolution that are not 
possible for high-$z$ galaxies. 

Many efforts have been made in the past to discover the most metal-poor nearby
galaxies. The galaxy I~Zw~18, first spectroscopically observed by \citet{SS72},
with oxygen abundances 12+log~O/H~$\sim$~7.17--7.26 in its two brightest
regions \citep[e.g. ][]{SK93,IT98} stood as the lowest-metallicity SFG for a 
long period. It was replaced by SBS~0335$-$052W with an  
oxygen abundance 12+log~O/H=7.12$\pm$0.03, 
derived from the emission of the entire galaxy \citep{I05},
and oxygen abundances in its two brightest knots of star formation 
of 7.22 and 7.01 \citep{I09}. 

The large data base of the Sloan Digital Sky Survey (SDSS) offered a possibility
for a systematic search of the most metal-deficient SFGs. 
In particular, \citet{I12} and \citet{G17}
found two dozens of galaxies with 12+log~O/H$<$7.35 in the 
%\sout{Data Release 12 (DR12) of the} 
SDSS. However, no SFG with a metallicity below that
of SBS 0335$-$052W was found.

Local galaxies with very low-metallicities have also been discovered in other 
contexts. 
\citet*{P05} 
studying the properties of dwarf galaxies in the Cancer-Lynx void
found that the weighted mean oxygen abundance in five regions of 
the irregular dwarf galaxy DDO~68 is 12+log~O/H=7.21$\pm$0.03.

Two extremely metal-poor dwarf galaxies were discovered in the
course of the  Arecibo Legacy Fast ALFA survey \citep[ALFALFA, ][]{G05,H11}.
\citet{S13} found that the oxygen abundance in the nearby dwarf galaxy 
AGC~208583 = Leo~P is 7.17$\pm$0.04, still more metal-rich than SBS~0335$-$052W.
Finally, \citet{H16} showed that the nearby dwarf galaxy AGC~198691
has an oxygen abundance of 12~+~logO/H~=~7.02$\pm$0.03, lower than the 
average value in SBS~0335$-$052W.

%%%%%%%%%%%%%%%%%%%%%%%%%%%%%%%%%%%%%%%%%%%%%%%%%%%%%%%%%%%
%  Table 1
%%%%%%%%%%%%%%%%%%%%%%%%%%%%%%%%%%%%%%%%%%%%%%%%%%%%%%%%%%%
\begin{table}
%\centering
\caption{Observed characteristics of J0811$+$4730 \label{tab1}}
\begin{tabular}{lr} \hline
Parameter                 &  Value       \\ \hline
R.A.(J2000)               &  08:11:52.12 \\
Dec.(J2000)               & +47:30:26.24 \\
  $z$                     &  0.04444$\pm$0.00003     \\
  $g$, mag                &   21.37$\pm$0.05      \\
  $M_g$, mag$^\dag$        & $-$15.41$\pm$0.06     \\
log $M_\star$/M$_\odot$$^\ddag$&   6.24$\pm$0.33       \\
log $M_\star$/M$_\odot$$^{\ddag\ddag}$&   6.29$\pm$0.06       \\
$L$(H$\beta$), erg s$^{-1}$$^*$&(2.1$\pm$0.1)$\times$10$^{40}$\\
SFR, M$_\odot$yr$^{-1}$$^{\dag\dag}$  &     0.48$\pm$0.02 \\
\hline
  \end{tabular}

\noindent$^\dag$Corrected for Milky Way extinction.

\noindent$^\ddag$Derived from the extinction- and aperture-corrected SDSS 
spectrum.

\noindent$^{\ddag\ddag}$Derived from the extinction- and aperture-corrected LBT 
spectrum.

\noindent$^*$Corrected for extinction and the SDSS spectroscopic aperture.

\noindent$^{\dag\dag}$Derived from the \citet{K98} relation using the extinction- 
and aperture-corrected H$\beta$ luminosity.
  \end{table}
%%%%%%%%%%%%%%%%%%%%%%%%%%%%%%%%%%%%%%%%%%%%%%%%%%%%%%%%%%%%%%%%%%%%



In this paper, we present Large Binocular Telescope (LBT)\footnote{The LBT 
is an international collaboration among institutions in the United States, 
Italy and Germany. LBT Corporation partners are: The University of Arizona on 
behalf of the Arizona university system; Istituto Nazionale di Astrofisica, 
Italy; LBT Beteiligungsgesellschaft, Germany, representing the Max-Planck 
Society, 
the Astrophysical Institute Potsdam, and Heidelberg University; The Ohio State 
University, and The Research Corporation, on behalf of The University of Notre 
Dame, University of Minnesota and University of Virginia.} spectroscopic
observations with high signal-to-noise ratio of the compact dwarf SFG 
J0811$+$4730. This galaxy stood out by its emission-line ratios, 
during the inspection of the  
spectra in the SDSS Data Release 13 (DR13) data base \citep{A16}, as 
potentially having a very low metallicity. 
Its coordinates, redshift and other characteristics obtained from the SDSS
photometric and spectroscopic data bases are shown in Table \ref{tab1}.

The LBT observations and data reduction are described in 
Sect.~\ref{sec:observations}. We derive element abundances in 
Sect.~\ref{sec:abundances}. Integrated characteristics of J0811$+$4730 are
discussed in Sect.~\ref{sec:integr}. Emission-line diagnostic diagrams and
the metallicity-luminosity relation for a sample of the most-metal poor
SFGs, including J0811$+$4730, are considered in Sect.~\ref{sec:diagrams}.
Finally, in Sect.~\ref{sec:conclusions} we summarize our main results.

%%%%%%%%%%%%%%%%%%%%%%%%%%%%%%%%%%%%%%%%%%%%%%%%
%    Fig.1 LBT spectra
%%%%%%%%%%%%%%%%%%%%%%%%%%%%%%%%%%%%%%%%%%%%%%%%
\begin{figure*}%[t]
\centering{
\includegraphics[angle=-90,width=0.90\linewidth]{J0811+4730b_2.ps}
\includegraphics[angle=-90,width=0.90\linewidth]{J0811+4730r_2.ps}}
\caption{The rest-frame LBT spectrum of J0811$+$4730 uncorrected for extinction.
Insets show expanded parts of spectral regions in the rest 
wavelength ranges 4320\AA\ -- 4490\AA, 6520\AA\ -- 6770\AA\ and 
8500\AA\ -- 9280\AA\ for a better
view of weak emission lines. Some interesting emission lines are 
labelled. Wavelengths are in \AA\ and fluxes are in 
units of 10$^{-16}$ erg s$^{-1}$ cm$^{-2}$ \AA$^{-1}$.}
\label{fig1}
\end{figure*}
%%%%%%%%%%%%%%%%%%%%%%%%%%%%%%%%%%%%%%%%%%%%%%%%%

\section{LBT Observations and data reduction}\label{sec:observations}

We have obtained LBT long-slit spectrophotometric observations of J0811$+$4730 
on 1 February, 2017 in the twin binocular mode using both the MODS1 and 
MODS2 spectrographs\footnote{This paper used data obtained with the MODS 
spectrographs built with
funding from NSF grant AST-9987045 and the NSF Telescope System
Instrumentation Program (TSIP), with additional funds from the Ohio
Board of Regents and the Ohio State University Office of Research.}, 
equipped with two 8022 pix $\times$ 3088 pix CCDs. The G400L grating in 
the blue beam, with a dispersion of 0.5\AA/pix, and the G670L grating in the 
red beam, with a dispersion of 0.8\AA/pix, were used. Spectra were obtained in 
the wavelength range 3200 -- 10000\AA\ with a 1.2 arcsec wide slit, 
resulting in a resolving power $R$ $\sim$ 2000. The seeing during 
the observations was in the range 0.5 -- 0.6 arcsec.

Six 890~s subexposures were obtained with each spectrograph, resulting
in the total exposure of 2$\times$5340~s counting both spectrographs.
The airmass varied from 1.16 for the first subexposure to 1.05 for
the sixth subexposure. 
The slit was oriented at the fixed position angle of 80${\degr}$. This is to be 
compared with the parallactic angles of 74${\degr}$ during the first subexposure
and of 53${\degr}$ during the sixth subexposure. According to \citet{F82}, such
differences between the position angle and the parallactic angle would result in
an offset perpendicular to the slit of the [O~{\sc ii}] $\lambda$3727 
wavelength region relative to the H$\beta$ wavelength region of less than 
0.1 -- 0.2 arcsec. Thus, 
the effect of atmospheric refraction is small for all subexposures. 

The spectrum of the spectrophotometric standard star GD~71, obtained during 
the same night with a 5 arcsec wide slit, was used for flux calibration.
It was also used to correct the red part of the 
J0811$+$4730 spectrum for telluric absorption lines. 
Additionally, calibration frames of biases, flats and comparison lamps 
were obtained during the daytime, after the observations.

The MODS Basic CCD Reduction package {\sc modsCCDRed}\footnote{http://www.astronomy.ohio-state.edu/MODS/Manuals/ MODSCCDRed.pdf} was used for bias subtraction 
and flat field correction, while wavelength and flux calibration was 
done with {\sc iraf}\footnote{{\sc iraf} is distributed by the 
National Optical Astronomy Observatories, which are operated by the Association
of Universities for Research in Astronomy, Inc., under cooperative agreement 
with the National Science Foundation.}. 
Finally, all MODS1 and MODS2 subexposures were 
combined. The one-dimensional spectrum 
of J0811$+$4730 shown in Fig. \ref{fig1} was extracted 
in a 1.2 arcsec aperture along the spatial axis. Strong emission lines are 
present in this spectrum, suggesting active star formation.
In particular, a strong [O~{\sc iii}] $\lambda$4363 emission line is 
detected, allowing a reliable abundance determination. Because of the
importance of this line, we have checked whether its flux can be affected by 
such events as cosmic ray hits. We find that the flux ratio of 
[O~{\sc iii}]$\lambda$4363 to the nearest H$\gamma$ $\lambda$4340 emission line 
is nearly constant in all MODS1 and MODS2 subexposures, varying by not more than
1 -- 2 per cent. This implies that the [O~{\sc iii}]$\lambda$4363 emission line 
is not affected by cosmic ray hits.

Emission-line fluxes were measured using the  {\sc iraf splot} routine. 
The errors of the line fluxes were calculated from the
photon statistics in the non-flux-calibrated spectra, and by adding a 
relative error of 1 per cent in the 
absolute flux distribution of the spectrophotometric standard star.
They were then propagated in the calculations of the elemental 
abundance errors. 

\begin{deluxetable}{lrrrrrrrrrr}
  \tabletypesize{\scriptsize}
  %\rotate
  \tablecaption{New Mid-Infrared Photometry\label{tab:phot}}
  %\tablewidth{0pt}
  \tablecolumns{11}
  \tablenum{2}
  \tablehead{
    \colhead{Name} & \colhead{8.8~\micron\tablenotemark{a}} & \colhead{9.8~\micron\tablenotemark{a}} & \colhead{11.1~\micron\tablenotemark{b}} & \colhead{19.7~\micron\tablenotemark{b}} & \colhead{25.3~\micron\tablenotemark{b}} & \colhead{31.5~\micron\tablenotemark{b}} & \colhead{34.8~\micron\tablenotemark{b}} & \colhead{37.1~\micron\tablenotemark{b}} & \colhead{70~\micron\tablenotemark{c}} & \colhead{160~\micron\tablenotemark{c}} \\
    \colhead{} & \colhead{(Jy)} & \colhead{(Jy)} & \colhead{(Jy)} & \colhead{(Jy)} & \colhead{(Jy)} & \colhead{(Jy)} & \colhead{(Jy)} & \colhead{(Jy)} & \colhead{(Jy)} & \colhead{(Jy)}
  }
  \startdata
  \objectname{VX Sgr} & \nodata & \nodata & $3740\pm98$ & $2250\pm91$ & $1400\pm85$ & $1090\pm69$ & $835\pm75$ & $697\pm10$ & $153\pm32$ & $23\pm11$ \\
  \objectname{S Per} & $316\pm52$ & $340\pm48$ & \nodata & $342\pm28$ & $187\pm24$ & $146\pm15$ & $109\pm9.4$ & $97\pm9.6$ & $21\pm4.1$ & $3\pm1.2$ \\
  \objectname{RS Per} & \nodata & \nodata & \nodata & $82\pm4.8$ & \nodata & $21\pm2.0$ & \nodata & $15\pm4.1$ & \nodata & \nodata \\
  \objectname{T Per} & $8.8\pm1.4$ & $11.4\pm3.1$ & \nodata & $9.7\pm1.2$ & \nodata & $7.2\pm2.9$ & \nodata & \nodata & \nodata & \nodata \\
  \objectname{NML Cyg}\tablenotemark{d} & $3735\pm63$ & $3780\pm160$ & \nodata & $4868\pm41$ & $3930\pm328$ & $3626\pm110$ & $2849\pm123$ & $2809\pm287$ & $652\pm96$ & $116\pm30$ \\
  \enddata
  \tablenotetext{a}{MMT/MIRAC}
  \tablenotetext{b}{SOFIA/FORCAST}
  \tablenotetext{c}{HERSCHEL/PACS}
  \tablenotetext{d}{NML~Cyg MIRAC photometry originally presented in \cite{schuster2009}.}
\end{deluxetable}


%%%%%%%%%%%%%%%%%%%%%%%%%%%%%%%%%%%%%%%%%%%%%%%%%%%%%%%%%%%
%  Table 3
%%%%%%%%%%%%%%%%%%%%%%%%%%%%%%%%%%%%%%%%%%%%%%%%%%%%%%%%%%%
\begin{table}
\caption{Electron temperatures and element abundances \label{tab3}}
\begin{tabular}{lccccc} \hline
Property                             &Value          \\ \hline
$T_{\rm e}$(O {\sc iii}), K          &21700$\pm$500       \\
$T_{\rm e}$(O {\sc ii}), K           &15600$\pm$500       \\
$T_{\rm e}$(S {\sc iii}), K          &19700$\pm$500       \\
$N_{\rm e}$(S {\sc ii}), cm$^{-3}$    &380$\pm$140         \\ \\
O$^+$/H$^+$$\times$10$^6$            &1.405$\pm$0.115 \\
O$^{2+}$/H$^+$$\times$10$^6$          &7.884$\pm$0.398 \\
O$^{3+}$/H$^+$$\times$10$^6$          &0.244$\pm$0.021 \\
O/H$\times$10$^6$                   &9.532$\pm$0.415 \\
12+log(O/H)                         &6.979$\pm$0.019     \\ \\
N$^+$/H$^+$$\times$10$^7$            &0.404$\pm$0.053 \\
ICF(N)                              &6.384 \\
N/H$\times$10$^7$                   &2.785$\pm$0.358 \\
log(N/O)                            &$-$1.535$\pm$0.044~~~\\ \\
Ne$^{2+}$/H$^+$$\times$10$^6$        &1.409$\pm$0.072 \\
ICF(Ne)                             &1.073 \\
Ne/H$\times$10$^6$                  &1.512$\pm$0.087 \\
log(Ne/O)                           &$-$0.800$\pm$0.031~~~\\ \\
S$^+$/H$^+$$\times$10$^7$            &0.213$\pm$0.021 \\
S$^{2+}$/H$^+$$\times$10$^7$         &0.996$\pm$0.146 \\
ICF(S)                              &1.621 \\
S/H$\times$10$^7$                   &1.961$\pm$0.237 \\
log(S/O)                            &$-$1.687$\pm$0.056~~~\\ \\
Ar$^{2+}$/H$^+$$\times$10$^8$        &2.174$\pm$0.142 \\
Ar$^{3+}$/H$^+$$\times$10$^8$        &2.482$\pm$0.494 \\
ICF(Ar)                             &1.223 \\
Ar/H$\times$10$^8$                  &2.659$\pm$0.628 \\
log(Ar/O)                           &$-$2.555$\pm$0.104~~~\\ \\
Fe$^{2+}$/H$^+$$\times$10$^7$        &0.898$\pm$0.185 \\
ICF(Fe)                             &9.263 \\
Fe/H$\times$10$^7$                  &8.315$\pm$1.716 \\
log(Fe/O)                           &$-$1.060$\pm$0.092~~~\\
\hline
  \end{tabular}
  \end{table}
%%%%%%%%%%%%%%%%%%%%%%%%%%%%%%%%%%%%%%%%%%%%%%%%%%%%%%%%%%%%%%%%%%%%


The observed fluxes were corrected for extinction, 
derived from the observed decrement of the hydrogen Balmer emission lines
H$\alpha$, H$\beta$, H$\gamma$, H$\delta$, H9, H10, H11, and H12.
Two lines, H7 and H8, were however excluded because they are blended with
other lines.
The equivalent widths of the underlying stellar Balmer absorption lines, 
assumed to be the same for each line, were derived simultaneously with the 
extinction in an iterative manner, following \citet{ITL94}. They are shown in 
Table \ref{tab2}. All hydrogen lines were corrected for underlying
absorption, in addition to correction for extinction. It is seen from 
Table \ref{tab2} that the corrected fluxes of the H9 -- H16 emission lines
are somewhat larger than the case B values. This is probably due to an 
overcorrection of underlying absorption for these higher order lines. However, 
this overcorrection makes
little difference on the determination of extinction which depends mainly on the
observed decrement of the H$\alpha$, H$\beta$, H$\gamma$ and H$\delta$ emission
lines.

We note that excluding H$\alpha$, the only Balmer 
hydrogen line observed in the red part of the spectrum, in the extinction 
determination would result in a somewhat higher extinction coefficient,  
$C$(H$\beta$) = 0.185 compared to the value of 0.165 in Table \ref{tab2}.
This disagreement is small, within the errors. It probably comes from a slight 
mismatch between the blue and red parts of the spectrum, not exceeding a few
per cent of the continuum flux in the overlapping wavelength range. 
 %For clarity, 
We have adopted the value of 0.165. It is somewhat higher than the $C$(H$\beta$)
expected for extremely low-metallicity SFGs. However, part of this 
extinction is due to the Milky Way with relatively high
$A(V)_{\rm MW}$ = 0.180 mag according to 
the NASA/IPAC Extragalactic Database (NED) and corresponding to 
$C$(H$\beta$)$_{\rm MW}$=0.085. Therefore, the 
internal extinction coefficient in J0811$+$4730 is only
$C$(H$\beta$)$_{\rm int}$=0.080. We have adopted the reddening law by 
\citet*{C89} with a total-to-selective extinction ratio $R(V)$ = 3.1.

The observed emission-line fluxes $F(\lambda)$/$F$(H$\beta$) multiplied by 100, 
the emission-line fluxes corrected for underlying absorption and extinction $I(\lambda)$/$I$(H$\beta$) multiplied by 100, the extinction 
coefficient $C$(H$\beta$), the rest-frame equivalent width EW(H$\beta$),
the equivalent width of the Balmer absorption lines, and the observed
H$\beta$ flux $F$(H$\beta$) are listed in Table \ref{tab2}. We note that 
the EW(H$\beta$) in J0811$+$4730 is high, $\sim$~280\AA, implying that 
its optical emission is dominated by radiation from a very young burst, with
age $\sim$ 3 Myr.

It is seen in Fig. \ref{fig1} that the ratio of H$\alpha$ and H$\beta$
peak intensities (slightly larger than 2) is lower than the ratio of their 
total fluxes (about a factor of 3). This is in large
part due to the somewhat different spectral resolution of the blue and red 
spectra of J0811$+$4730, as they were obtained with different gratings.
Indeed, the full widths at half maximum (FWHMs) of H$\alpha$ and H$\beta$ are 
4.4\AA\ and 2.9\AA, respectively. These widths are larger than the widths of 
the Ar lines of 3.6\AA\ and 2.2\AA\ in the blue and red comparison spectra,
respectively, indicating that both profiles are 
partially resolved. By deconvolving the instrumental profiles from the  
observed profiles, we obtain an intrinsic 
velocity dispersion $\sigma$ of $\sim$45$\pm$5 km s$^{-1}$ for both the 
H$\alpha$ and H$\beta$ emission lines. This value is separately derived from each of the MODS1 and MODS2 spectra. The uncertainty is derived from the Gaussian fitting of both lines, followed by
the deconvolution of their profiles.The velocity dispersion value is fully consistent with the one expected from the $L$(H$\beta$) --
$\sigma$ relation for supergiant H~{\sc ii} regions \citep[e.g. ][]{C12}, 
and supports the assumption that the emission lines originate from a single 
H~{\sc ii} region. 
%It can be understood if there are two H~{\sc ii} regions in J0811$+$4730 with 
%similar H$\beta$ luminosities but with a difference in radial velocities of 
%$\sim$ 65 km s$^{-1}$. This is similar
%to the case of I~Zw~18, in which the difference in radial velocities of
%the NW and SE components is $\sim$ 50 km s$^{-1}$ \citep[e.g. ][]{IT98}. 
%Observations of J0811$+$4730 with a higher spectral resolution are needed to 
%verify this possibility.

\section{Element abundances}\label{sec:abundances}

The procedures described by \citet{I06} are used to determine element 
abundances from the LBT spectrum. We note that SDSS spectra of
J0811$+$4730 are available in Data Releases 13 and 14, but they cannot be used for  
abundance determination because the strong lines are clipped in those spectra. 
The temperature $T_{\rm e}$(O~{\sc iii}) is calculated from the 
[O~{\sc iii}] $\lambda$4363/($\lambda$4959 + $\lambda$5007) emission-line flux 
ratio. It is used to derive the abundances of O$^{2+}$, Ne$^{2+}$ and Ar$^{3+}$.
To obtain the abundances of O$^{+}$, N$^{+}$, S$^{+}$ and Fe$^{2+}$, the electron
temperature $T_{\rm e}$(O~{\sc ii}) needs to be derived. In principle, this can
be done by using the [O~{\sc ii}]$\lambda$3727/($\lambda$7320+$\lambda$7330) flux
ratio. However, the latter two lines are extremely weak in the spectrum 
of J0811$+$4730,
making such a determination of $T_{\rm e}$(O~{\sc ii}) not possible. Similarly,
the electron temperature $T_{\rm e}$(S~{\sc iii}) is needed to derive the 
S$^{2+}$ and Ar$^{2+}$ abundances. It can be obtained from the 
[S~{\sc iii}]~$\lambda$6312/($\lambda$9069+$\lambda$9531) flux ratio. However,
the [S~{\sc iii}]~$\lambda$6312 line is very weak (Table \ref{tab2}) and
the [S~{\sc iii}]~$\lambda$9069 line, at the J0811$+$4730 redshift, falls at the wavelength of $\sim$9475\AA, in a region with numerous night sky emission lines
\citep[e.g., fig. 9 in ][]{L11} and telluric absorption lines
\citep[e.g., fig. 1 in ][]{R16}. Furthermore, the strongest 
[S~{\sc iii}]~$\lambda$9531 line is at the very edge of the LBT spectrum, where
the sensitivity is low. 
All these factors make the determination of 
$T_{\rm e}$(S~{\sc iii}) somewhat uncertain. In fact, using the 
[S~{\sc iii}]$\lambda$6312 and [S~{\sc iii}]$\lambda$9069 fluxes from 
Table \ref{tab2}, equations from \citet{A84} and adopting 
[S~{\sc iii}]$\lambda$9531/$\lambda$9069 = 2.4 we derive the high and very 
uncertain value $T_{\rm e}$(S~{\sc iii}) = 27000$\pm$6000K.

Therefore we have used the expressions of \citet{I06}, obtained from 
photoionized H~{\sc ii} region models, to derive the electron 
temperatures $T_{\rm e}$(O~{\sc ii}) and $T_{\rm e}$(S~{\sc iii}). We have also
adopted the errors for these temperatures to be equal to the error for
the electron temperature $T_{\rm e}$(O~{\sc iii}).
The errors quoted  for $T_{\rm e}$(O~{\sc ii}) and $T_{\rm e}$(S~{\sc iii}) should 
be considered as lower limits as they do not take into account the uncertainty 
introduced by our reliance on 
grids of photoionization models to determine these temperatures. These models 
show inevitably a 
dispersion due to varying parameters such as ionization parameter, electron 
number density, chemical composition, etc., that will increase the uncertainty 
in these two temperatures. 
 The electron number density $N_{\rm e}$(S~{\sc ii}) is derived from the 
[S~{\sc ii}] $\lambda$6717/$\lambda$6731 emission line ratio. 
However, it is not possible to obtain the electron number density
from the [O~{\sc ii}]$\lambda$3726/$\lambda$3729 flux ratio because of the
insufficient spectral resolution. 

The total oxygen abundance is derived as follows:
\begin{equation}
\frac{\rm O}{\rm H} = \frac{{\rm O}^++{\rm O}^{2+}+{\rm O}^{3+}}{{\rm H}^+}, 
\label{OH} 
\end{equation}
where the abundances of ions O$^+$, O$^{2+}$, O$^{3+}$ are obtained using 
the relations of \citet{I06}. For ions of other heavy elements, we also use
the relations of \citet{I06} to derive the ionic abundances, the ionization
correction factors and the total heavy element abundances.

The electron temperatures, electron number densities, ionic abundances,
ionization correction factors and total heavy element abundances are presented 
in Table~\ref{tab3}. The electron temperature $T_{\rm e}$(O~{\sc iii})
= 21700 $\pm$ 500 K is derived from the fluxes of [O~{\sc iii}] lines, 
which were obtained with a high signal-to-noise ratio. The relatively high value of 
$T_{\rm e}$(O~{\sc iii}) is a consequence of the very low metallicity of J0811$+$4730.
We note that a temperature of $\ga$20000~K has been determined 
in some extremely metal-poor galaxies \citep[e.g. ][]{I09}. {\sc cloudy} 
models also predict the temperature range 20000 -- 23000 K
for objects with 12+logO/H $\sim$ 7.0, depending on the input parameters.

We derive an oxygen abundance of
12+logO/H = 6.98$\pm$0.02, making J0811$+$4730 the lowest-metallicity
SFG known, and the first galaxy with an oxygen abundance below 7.0.
We have also derived the oxygen abundance from the MODS1 and MODS2 spectra 
separately. We obtain respectively 12+logO/H = 6.98$\pm$0.03 and  6.97$\pm$0.03,
fully consistent with the value obtained from combining all data.
If we had adopted the higher extinction coefficient $C$(H$\beta$) = 0.185,  
derived from the hydrogen Balmer decrement by excluding the H$\alpha$ emission line, 
we would have obtained an electron temperature higher by 150 K  and a  slightly lower
oxygen abundance, 12+logO/H = 6.975$\pm$0.019. 
The N/O, Ne/O, S/O, Ar/O and Fe/O abundance ratios 
for this galaxy, shown in Table \ref{tab3}, are similar to those 
derived for low-metallicity SFGs \citep[e.g., ][]{I06}.
We note that the error of the electron temperature 
$T_{\rm e}$(O~{\sc ii}) has little impact on the error of 12+logO/H.
Adopting an error equal to 1500 K instead of 500 K in Table~\ref{tab3} does 
increase the error of O$^+$/H$^+$ by a factor of 3 but does not change much the 
error of 12+logO/H, increasing it by only 0.005 dex. This is because the O$^+$
abundance is several times lower than the O$^{2+}$ abundance. The largest impact
of such an increased error in  $T_{\rm e}$(O~{\sc ii}) is an increase in the errors of log N/O and 
log Fe/O by $\sim$ 0.03 dex. Similarly, increasing the $T_{\rm e}$(S~{\sc iii}) error from
500 K to 1500 K would result in increasing the errors in log S/O and log Ar/O 
by $\sim$ 0.03 dex.

%%%%%%%%%%%%%%%%%%%%%%%%%%%%%%%%%%%%%%%%%%%%%%%%%%%%%%%%%%%
%  Table 4
%%%%%%%%%%%%%%%%%%%%%%%%%%%%%%%%%%%%%%%%%%%%%%%%%%%%%%%%%%%
\begin{table*}
\caption{Parameters of the lowest-metallicity SFGs \label{tab4}}
\begin{tabular}{lcrcrrccr} \hline
Name                             &12+logO/H&$M_B$&$M_g$&O$_3$&O$_{32}$&R$_{23}$&N$_{2}$&Ref.          \\ \hline
J0811$+$4730        &6.98$\pm$0.02& ...~~ &$-$15.4&1.61& 9.65&2.34&0.0023&1\\
SBS0335$-$052W\#2   &7.01$\pm$0.07&$-$14.7& ...   &1.48& 4.91&2.27& ...  &2\\
AGC198691           &7.02$\pm$0.03& ...~~ &$-$11.5&1.28& 2.74&2.17&0.0058&3\\
SBS0335$-$052E\#7   &7.12$\pm$0.04&$-$17.3& ...   &1.94& 7.78&2.84&0.0031&2\\
IZw18NW             &7.16$\pm$0.01& ...~~ &$-$15.3&1.95& 6.83&2.89&0.0034&4\\
LeoP                &7.17$\pm$0.04&$-$8.8 & ...   &1.45& 3.12&2.40&0.0091&5\\
J1220$+$4915        &7.18$\pm$0.03& ...~~ &$-$12.8&2.97&13.48&4.19&0.0051&6\\
IZw18SE             &7.19$\pm$0.02& ...~~ &$-$15.3&1.60& 2.68&2.73&0.0072&4\\
DDO68               &7.20$\pm$0.05&$-$15.8& ...   &1.89& 3.26&2.47&0.0057&7\\
SBS0335$-$052W\#1   &7.22$\pm$0.07&$-$14.7& ...   &1.30& 1.73&2.49&0.0122&2\\
J2104$-$0035        &7.24$\pm$0.02& ...~~ &$-$12.2&2.79&13.95&3.92&0.0025&6\\
SBS0335$-$052E\#1+2 &7.28$\pm$0.01&$-$17.3& ...   &3.03&13.90&4.27&0.0036&2\\
\hline
  \end{tabular}

{\bf Notes.} O$_3$=[O {\sc iii}]$\lambda$5007/H$\beta$, O$_{32}$=[O {\sc iii}]$\lambda$5007/[O {\sc ii}]$\lambda$3727, R$_{23}$=([O {\sc ii}]$\lambda$3727+[O {\sc iii}]$\lambda$4959+[O {\sc iii}]$\lambda$5007)/H$\beta$, N$_2$=[N~{\sc ii}]$\lambda$6584/H$\alpha$.~~~~\\
%Absolute magnitudes $M_B$ are for SBS0335$-$052W, SBS0335$-$052E, A198691, 
%IZw18, LeoP and absolute SDSS magnitudes $M_g$ are for~~~~\\ 
%the rest of galaxies. 
%$M_B$ for A198691 is derived assuming distance 16 Mpc.~~~~~~~~~~~~~~~~~~~~~~~~~~~~~~~~~~~~~~~~~~~~~~~~~~~~~~~~~~~~~~~~~~~~~~~~~~~~~~~~~~~~~~~~~~~~~~~~~~~~~~~~~~~~~~~~~~~~~~~

{\bf References.} 
1 - this paper, 2 - \citet{I09}, 3 - \citet{H16}, 4 - \citet{IT98}, 
5 - \citet{S13}, \\ 
6 - Izotov et al., in preparation, 7 - \citet{B12}.~~~~~~~~~~~~~~~~~~~~~~~~~~~~~~~~~~~~~~~~~~~~~~~~~~~~~~~~~~~~~~~~~~~~~~~~~~~~~~~~~~~~~~~~~~~~~~~~~~~~~~~~~~~~~~~
  \end{table*}
%%%%%%%%%%%%%%%%%%%%%%%%%%%%%%%%%%%%%%%%%%%%%%%%%%%%%%%%%%%%%%%%%%%%


\section{Integrated characteristics of J0811$+$4730}
\label{sec:integr}

To derive the integrated characteristics of J0811$+$4730, we adopt 
the luminosity distance $D$ = 205 Mpc. It was obtained from the galaxy
redshift with the NED cosmological calculator 
\citep{W06}, adopting the cosmological parameters 
$H_0$ = 67.1 km s$^{-1}$ Mpc$^{-1}$, $\Omega_m$ = 0.318, $\Omega_\Lambda$ = 
0.682 \citep{P14} and assuming a flat geometry.
The absolute SDSS $g$ magnitude, corrected for the Milky Way
extinction $M_g$ = $-$15.41 mag (Table \ref{tab1}), characterizes J0811$+$4730 
as a dwarf SFG.

\subsection{H$\beta$ luminosity and star-formation rate}

To derive the total H$\beta$ luminosity $L$(H$\beta$), we need to correct the
observed H$\beta$ flux in the SDSS spectrum for extinction and spectroscopic 
aperture. Since the strong emission lines in the SDSS spectrum of J0811$+$4730
are clipped we adopt the extinction coefficient $C$(H$\beta$) = 0.165 obtained 
from the observed hydrogen Balmer decrement in the LBT spectrum. The aperture 
correction is determined as 2.512$^{g_{\rm ap}-g}$, where, $g$ = 21.38$\pm$0.05 mag 
and $g_{\rm ap}$ = 22.28$\pm$0.08 mag are respectively the SDSS total
modelled magnitude and the magnitude inside the spectroscopic aperture 
of 2 arcsec in diameter.
Both quantities, $g$ and $g_{\rm ap}$, are extracted 
from the SDSS data base. Using these data we obtain a correction factor of
2.3, which is applied to the H$\beta$ luminosity and some other 
quantities, such as the stellar mass $M_\star$, obtained from 
both SDSS and LBT spectra.
Here we have assumed that the spatial brightness distributions of 
the continuum and the hydrogen emission lines are the same. We have checked
this assumption by examining the LBT long-slit spectrum. 
We find from this spectrum that the brightness distributions of H$\beta$, 
H$\alpha$ and the adjacent continua are indeed nearly identical, with FWHMs of 
$\sim$ 0.6 arcsec, indicating that the galaxy 
is mostly unresolved. This angular size corresponds to an upper limit
of the galaxy linear size at FWHM of $\sim$ 600 pc.
 
The star-formation rate (SFR) in J0811$+$4730, 
derived from the aperture- and extinction-corrected H$\beta$ luminosity 
$L$(H$\beta$) using the \citet{K98} calibration, is 0.48~M$_\odot$~yr$^{-1}$.

\subsection{Stellar mass}


%Several techniques have been proposed for the determination of the
%stellar mass of a galaxy. For the nearest galaxies, colour-magnitude
%diagrams (CMD) of the resolved stellar populations can be used 
%\citep[e.g. ][]{A13}. 

%In other cases, the stellar mass is derived from an  
%assumed mass-to-luminosity ratio \citep[e.g. ][]{B01}.
%However, this technique is subject to large uncertainties in SFGs, as its 
%luminosity is determined by the contribution of not only stellar emission, but also
%of nebular emission, which can be high and is often unknown. Furthermore, the 
%stellar mass-to-light ratio depends strongly on the galaxy$'$s star formation 
%history and metallicity. 

%\citet{K03} proposed
%a technique to constrain stellar masses of galaxies, based on two 
%stellar absorption line indices, the 4000\AA\ break strength and the Balmer 
%absorption line index H$\delta$. Together, these indices allow to constrain the
%mean stellar ages of galaxies and the fractional stellar mass formed in bursts 
%over the past few Gyr. However, no absorption features are present in the
%spectrum of J0811$+$4730, precluding the use of this method.

Fitting the spectral energy distribution (SED) of a galaxy is one of the most commonly 
used methods to determine its stellar mass \citep[e.g. ][]{CF05,As07}. The 
advantage of this method is that it can be applied to  
galaxies at any redshift and can separate stellar from nebular emission.
Here we derive the stellar mass of J0811$+$4730 from fitting the 
SEDs of both its SDSS and LBT spectra. 
Given the data at hand, it is the only method we can use.
%This is apparently the only method that can be applied to 
%derive the stellar mass of J0811$+$4730.} 
The equivalent width of the
H$\beta$ emission line in both spectra of J0811$+$4730 is high,
EW(H$\beta$) $\sim$ 280\AA, indicating that star formation occurs in a burst
and that the contribution of the nebular continuum is also high and should be 
taken into account in the SED fitting, in addition to the stellar emission. 
In particular, the fraction of the nebular continuum near H$\beta$ is 
$\sim$ 30 per cent, and it is considerably higher near H$\alpha$.
The details of the SED fitting are described e.g. in \citet*{I11} and 
\citet{I15}. The salient features are as follows.

We carried out a series of Monte Carlo simulations to reproduce  the SED of
J0811$+$4730. To calculate the contribution of the  stellar  emission to the 
SEDs, we adopted the grid of the Padua stellar evolution models by \citet{G00}
with heavy element mass fraction $Z$ = 0.0004, or 1/50 solar. To reproduce the 
SED of the stellar component with any star-formation history, we used the 
{\sc Starburst99} models \citep{L99,L14} 
to calculate a grid of instantaneous
burst SEDs for a stellar mass of 1 M$_\odot$ in a wide range of ages
from 0.5 Myr to 15 Gyr, and \citet*{L97} stellar atmosphere models. 
We adopted a stellar initial mass function with a 
Salpeter slope \citep{S55}, an upper mass limit of 100 M$_\odot$, and a lower 
mass limit of 0.1 M$_\odot$. Then the SED with any star-formation history can 
be obtained by integrating the instantaneous burst SEDs over time with a 
specified time-varying SFR.

We approximated the star-formation history in J0811$+$4730
by a recent short burst forming young stars at age $t_y$ $<$ 10 Myr and a 
prior continuous star formation 
with a constant SFR responsible for the older stars with ages ranging from 
$t_2$ to $t_1$, where $t_2 > t_1$ and varying between 10 Myr and 
15 Gyr. Zero age is now. The contribution of each stellar population to the 
SED was parameterized by the ratio of the masses of the old to young stellar 
populations, $b=M_y/M_o$, which we varied between 0.01 and 100.

The total modelled  monochromatic (nebular and stellar)
continuum flux near the H$\beta$ emission line for a mass of 1 M$_\odot$
was scaled to fit the monochromatic extinction- and aperture-corrected 
luminosity of the galaxy at the same wavelength. The
scaling factor is equal to the total stellar mass $M_\star$ in solar units.
In our fitting model, $M_\star=M_y+M_o$, where $M_y$ and $M_o$ were derived 
using $M_\star$ and $b$.

%%%%%%%%%%%%%%%%%%%%%%%%%%%%%%%%%%%%%%%%%%%%%%%%
%    Fig.2 SDSS spectrum and SED
%%%%%%%%%%%%%%%%%%%%%%%%%%%%%%%%%%%%%%%%%%%%%%%%
\begin{figure}%[t]
\includegraphics[angle=-90,width=0.99\linewidth]{mainsp_1_4.ps}
\includegraphics[angle=-90,width=0.99\linewidth]{mainsp_2_4.ps}
\caption{ {\bf a)} The rest-frame extinction-corrected SDSS spectrum of 
J0811$+$4730
overlaid by the modelled nebular SED (thin solid line), stellar SED (dashed
line), and total (stellar and nebular) SED (thick solid line).
%The error bar at 5500\AA\ 
%indicates the upper and lower limits of the fitted continuum, 
%corresponding to the upper and lower limits of the stellar mass. 
Short horizontal lines indicate wavelength ranges used for SED fitting.
The inset shows the expanded part of the spectrum that falls into the SDSS $i$ 
photometric band. The dotted line in the inset is the transmission curve of the
SDSS $i$ filter. Wavelengths are in \AA\ and fluxes are in units of  
10$^{-16}$ erg s$^{-1}$ cm$^{-2}$ \AA$^{-1}$.
{\bf b)} Same as in {\bf a)}, but the rest-frame extinction-corrected LBT 
spectrum of J0811$+$4730 and SED fits to it are shown. The meaning of the lines is the
same as in {\bf a)}.}
\label{fig2}
\end{figure}
%%%%%%%%%%%%%%%%%%%%%%%%%%%%%%%%%%%%%%%%%%%%%%%%%

The SED of the nebular continuum was taken from \citet{A84}. It
included hydrogen and helium free-bound, free-free, and two-photon emission. 
In our models, this was always calculated with
the electron temperature $T_{\rm e}$(H$^+$) of the H$^+$ zone, which is not 
necessarily equal to $T_{\rm e}$(O~{\sc iii}). We thus vary it in the range
(0.95 -- 1.05)$\times$$T_{\rm e}$(O~{\sc iii}). The fraction of the nebular
continuum near H$\beta$ is defined as the ratio of the observed EW(H$\beta$)
to the pure nebular value, which is $\sim$ 900 -- 1000\AA, depending of the
electron temperature. The observed intensities
of the emission lines from the LBT spectrum were corrected for reddening and 
scaled using the flux of the H$\beta$ 
emission line and were added to the calculated nebular continuum. 
However, these lines were not used in the SED fitting, because we fit only the 
continuum. 
%We do not show in Fig. \ref{fig2} the emission-line intensities from 
%the SDSS spectrum because most of the strongest lines are clipped. This does
%not allow to derive the extinction coefficient from the SDSS spectrum.
We also varied the extinction coefficient 
$C$(H$\beta$)$_{\rm SED}$ in the range of 
(0.9 -- 1.1)$\times$$C$(H$\beta$), where $C$(H$\beta$) = 0.165, the extinction 
coefficient derived from the observed hydrogen Balmer decrement in the LBT
spectrum.
  
For J0811$+$4730, we calculated 5$\times$10$^5$ Monte Carlo models 
for each of the SDSS and LBT spectra by 
varying $t_y, t_1, t_2, b$, and $C$(H$\beta$)$_{\rm SED}$. The best model should 
satisfy three conditions. First, the modelled EW(H$\beta$) should match 
the observed value, within the adopted range between 0.95
and 1.05 times its nominal value. Second, the modelled EW(H$\alpha$) 
should also match the observed value, in the same range between 0.95
and 1.05 times its nominal value. As compared to our previous work, this condition is introduced here 
for the first time as it would better constrain the models.
Third, among the models satisfying 
the first and the second conditions, the best 
modelled SEDs of the SDSS and LBT spectra were found 
from $\chi^2$ minimization of the deviation between the 
modelled and the observed continuum at ten wavelength ranges, which are 
selected to cover the entire spectrum and to be free 
of emission lines and residuals of night sky lines.

From the fit of the SDSS spectrum, we find that the best model for the stellar 
population consists of young
stars with an age $t_y$ = 3.28 Myr and a stellar mass 
$M_y$ = 10$^{6.13}$ M$_\odot$, and of older stars formed continuously with 
a constant SFR over the time period from 770 Myr to 930 Myr and a stellar 
mass $M_o$ = 10$^{5.59}$ M$_\odot$, or 3.5 times lower than the stellar mass of 
the young stellar population. Correspondingly, the total stellar mass of the 
galaxy is $M_\star$ = $M_y$ + $M_o$ = 10$^{6.24\pm0.33}$~M$_\odot$. 
This implies that 78\% of the total stellar mass has been formed during the 
most recent burst of star formation.
Similarly, from the fit of the LBT spectrum, we derive 
$M_y$ = 10$^{6.18}$ M$_\odot$ for the young population with an age 
$t_y$ = 3.28 Myr, and a mass $M_o$ = 10$^{5.63}$ M$_\odot$ of older stars formed
continuously with a constant SFR over the time period from 100 Myr to 290 Myr.
This corresponds to a total stellar mass $M_\star$ = 10$^{6.29\pm0.06}$~M$_\odot$, in
agreement, within the errors, with the value obtained from fitting the SDSS 
spectrum. 
All masses obtained from the SDSS spectrum have been corrected for aperture effects 
using the SDSS $g$-band total magnitude and the 
$g$ magnitude inside the 2 arcsec SDSS aperture. This correction would be somewhat different 
for the LBT spectrum obtained with a 1.2 arcsec wide slit. We do not have enough data to determine the 
LBT aperture correction accurately, so for the sake of simplicity, we have 
%The aperture correction for the 
%LBT spectrum is somewhat uncertain because only the LBT spectrum of J0811$+$4730 with
%narrow slit was obtained. Therefore, for clarity we 
adopted the same aperture correction for the LBT data as for the SDSS data.
The good agreement between the SDSS and LBT stellar masses suggests that this 
procedure is not too far off. 

We note that the determination
of the age and the mass of the old stellar population is somewhat uncertain
because its contribution to the continuum in the optical range is low, not
exceeding 6 per cent near H$\beta$ and 7 per cent near H$\alpha$ in the best
model. However, even if we fix the formation period of the old stellar 
population to be between 1 Gyr and 10 Gyr ago, i.e. we assume the least luminous
old stellar population with the highest mass-to-luminosity ratio, we still find 
from SED fitting a small total stellar mass $M_\star$ $\sim$ 10$^{6.34}$ M$_\odot$.
Moreover, setting additionally $C$(H$\beta$) = 0 in the SED fitting does not 
change the conclusion about the low galaxy stellar mass.

The rest-frame extinction-corrected SDSS and LBT spectra overlaid by the 
SEDs of the best models are shown in Fig.~\ref{fig2}a and \ref{fig2}b,
respectively. The nebular, stellar and total (nebular plus stellar) SEDs are 
shown respectively by thin solid, dashed and thick solid lines. 
%The spectrum is noisy, especially in the red part where the night-sky line residuals are large.
%However, as noted above, 
We use only regions clean of galaxy
emission lines and night-sky residuals to fit the SEDs. They are shown in 
Fig.~\ref{fig2} by short horizontal lines.
%The error bar at 5500\AA\ indicates the upper and lower limits of the fitted 
%continuum, corresponding to the upper and lower limits of the stellar mass. 
%It is consistent with the 1$\sigma$ dispersion of the continuum flux in the 
%SDSS spectrum.

We find 
that the contribution of the nebular continuum near the H$\beta$ emission line 
is nearly 32 per cent of the total continuum emission in that region. 
Furthermore, the SED of the nebular continuum is shallower than 
the stellar continuum SED (compare the thin solid and dashed lines in 
Fig.~\ref{fig2}).
This makes the fraction of nebular continuum
increase to $\sim$50 per cent of the total near the H$\alpha$ emission line.
Therefore, neglecting the nebular continuum, 
i.e. assuming that the continuum shown by the thick solid line in 
Fig.~\ref{fig2} is purely stellar in origin, would result in an older
stellar population and a larger stellar mass. Indeed, with this assumption, we
find a stellar mass which is 0.56 dex higher than the one derived above.
This overestimate of the stellar mass is consistent with the finding of \citet{I11} 
who compared the stellar mass determinations for similar compact SFGs, dubbed "Green Peas",
with the method described in this paper and the method used by \citet{Ca09},
which does not take into account nebular continuum emission. \citet{I11} found that 
the neglect of the nebular continuum contribution would result in an overestimate of the stellar masses of 
compact SFGs with EW(H$\beta$) $\geq$ 100\AA\ by 0.4 dex on average.

In principle, the stellar mass of the old stellar population in galaxies 
can be estimated from the SDSS $i$ magnitude, if all light in this
band is stellar and assuming a value for the  
mass-to-light ratio which for the oldest stars is $\sim$ 1, if the mass and 
luminosity are expressed in solar units. However, this technique does not  
work for J0811$+$4730 because the contribution of the cool and
old stars to its $i$-band luminosity is small. To demonstrate this, we
show in insets of Fig.~\ref{fig2} the parts of the spectra that fall into the SDSS 
$i$-band, with the dotted lines representing the $i$-band transmission curve, with a full passband width FWHM $\sim$1200\AA. 
It is seen that the 
H$\alpha$ emission line, with a EW(H$\alpha$) of $\sim$ 1700\AA, is redshifted to a
wavelength where the sensitivity of the transmission curve is still $\sim$1/3 of 
its maximum value. Therefore, the H$\alpha$ line contributes, within the  
spectroscopic aperture, 1/3$\times$EW(H$\alpha$)/(1/3$\times$EW(H$\alpha$) + 
FWHM) $\sim$ 1/3 of the light in the  $i$-band. 
Furthermore, to account for the high value of EW(H$\alpha$), about half of the
remaining light must be due to the nebular continuum (compare the dashed and 
thin solid lines in the insets of Fig.~\ref{fig2}). This nebular continuum 
comes from hydrogen recombination 
%nature of the hydrogen emission line 
and free-free emission and cannot be 
neglected. We estimate that only $\sim$ 1/3 of the J0811$+$4730 light in the 
$i$-band is of stellar origin. Furthermore, a major fraction of this stellar emission comes from 
the Rayleigh-Jeans tail of the radiation distribution of hot and luminous young stars.
%The emission of young stars in the $i$-band also can not be neglected otherwise 
%it would not be possible to fit simultaneously the blue part of the spectrum.
We thus conclude that only a small fraction of the $i$-band light comes from 
the old stellar population and that photometry 
%Our consideration shows that the photometric data in the $i$-band 
cannot be used for simple mass estimates of the old stellar population in 
J0811$+$4730. 
Only SED fitting of spectra, which includes both the stellar and 
nebular emission in a wide range of wavelengths, can give physically 
justified estimates.

The low stellar mass $M_\star$, in addition to its faint absolute
magnitude, characterizes J0811$+$4730 as a dwarf SFG.

%%%%%%%%%%%%%%%%%%%%%%%%%%%%%%%%%%%%%%%%%%%%%%%%
%    Fig.3 M/L vs. M
%%%%%%%%%%%%%%%%%%%%%%%%%%%%%%%%%%%%%%%%%%%%%%%%
\begin{figure}%[t]
\centering{
\includegraphics[angle=-90,width=0.98\linewidth]{ML_MlowEW_1.ps}}
\caption{Relation between the stellar mass-to-luminosity ratio
and the stellar mass. All quantities are corrected for spectroscopic
aperture and extinction, and are expressed in solar units.
Selected lowest-metallicity SFGs with 12+logO/H$<$7.3 are represented
by labelled filled circles. The galaxy J0811$+$4730 is encircled. 
The error bar indicates the errors of $M_\star$ and $M_\star$/$L_g$ for this SFG.
By an open triangle is shown the SFG J0159$+$0751, characterized by very high
O$_{32}$ = [O~{\sc iii}]$\lambda$5007/[O~{\sc ii}]$\lambda$3727 = 39 and
EW(H$\beta$) = 347\AA\ \citep*{I17}. Arrows indicate the upward shifts of the 
latter SFG and J0811$+$4730 if the contribution of the H$\beta$ and 
[O~{\sc iii}]$\lambda$4959, $\lambda$5007 emission in the SDSS $g$-band 
is excluded. For comparison, 
are also shown compact SFGs from the SDSS DR12 with redshifts $z$$>$0.01
\citep[grey dots, ][]{I16c}. Dotted lines represent {\sc Starburst99} models 
with continuous star formation and a SFR varying from 
0.001 to 10  M$_\odot$ yr$^{-1}$, during the periods from the present to 10 Gyr, 
1 Gyr, 100 Myr, 10 Myr, and 1 Myr in 
the past (horizontal solid lines). These models include the contribution of
both the stellar and nebular continua, but not of nebular emission lines.}
\label{fig3}
\end{figure}
%%%%%%%%%%%%%%%%%%%%%%%%%%%%%%%%%%%%%%%%%%%%%%%%%

\subsection{Comparison with I Zw 18}

We test the robustness of our method of stellar mass determination by applying it to 
another extremely metal-poor SFG, I Zw 18. This SFG has been selected because
its stellar mass can be determined by at least another independent method.
\citet{A13} have used the CMD of its resolved stellar populations 
to derive a stellar mass of $>$10$^{7.24}$ M$_\odot$ of its main body, adopting
a distance of 18.2 Mpc \citep{A07}. 

This is to be compared with the stellar mass obtained from the SED fitting
described above. We use the SDSS spectra of the two brightest NW and SE 
components constituting the main body of I Zw 18. We obtain aperture-corrected
stellar masses 10$^{7.34}$ M$_\odot$ and 10$^{6.30}$ M$_\odot$ for the NW and SE
components, respectively. The total stellar mass of the I Zw 18 main body derived by SED fitting is therefore 10$^{7.38}$ M$_\odot$, consistent with the lower limit derived by \citet{A13} from the CMD.

\subsection{Mass-to-luminosity ratio}

In Fig. \ref{fig3} we show the stellar mass-to-luminosity 
ratio vs. stellar mass diagram for some of the most metal-poor SFGs known (filled circles). The galaxy
J0811+4730 is encircled. For three SFGs (J0811$+$4730, AGC~198691 and I~Zw~18),
we use SDSS $g$-band luminosities
calculated from the modelled total apparent magnitudes and adopting the 
absolute $g$-band magnitude of 5.45 mag for the Sun \citep{B03}.  
We note that the total magnitude used for I Zw 18 includes both the SE 
and NW components. The photometric
data for SBS~0335$-$052E are not present in the SDSS data base.
We have therefore adopted for it the $B$ magnitude given by \citet*{P04}, taking the absolute
$B$ magnitude of the Sun to be 5.48 \citep{BM98}. 
We have used the stellar masses of AGC~198691 and SBS~0335$-$052E derived by
\citet{H16} and \citet{I14}, respectively. The stellar masses of I~Zw~18 
and of J0811$+$4730 are derived in this paper by SED fitting
of their SDSS spectra. 
%The stellar mass derived for I~Zw~18 is similar to that
%estimated by \citet{A13} from the study of its resolved stellar populations.
%The distances of AGC~198691
%and I~Zw~18 have been taken from \citet{H16} and \citet{A13}, respectively.
%The distances of the two remaining galaxies have been derived from their redshifts.

The mass-to-luminosity ratio of J0811$+$4730 is extremely
low, $\sim$ 1/100, as derived from both the SDSS and LBT SEDs,
or approximately one order of magnitude lower than the 
mass-to-luminosity 
ratio of the second most metal-poor galaxy known, SFG AGC~198691 \citep{H16}, 
%\sout{.
%It is a factor of 10 lower than the mass-to-luminosity ratio}
and of I~Zw~18 (Fig. \ref{fig3}), but only a factor of 3 lower than that of 
another extremely metal-deficient SFG, SBS~0335$-$052E. 
%However, we note that the mass-to-luminosity ratio of J0811$+$4730 is 
%somewhat uncertain because the noisy SDSS spectrum introduces uncertainties in 
%the stellar mass determination. Higher signal-to-noise ratio optical spectra 
%are needed to confirm its very low value.
For comparison, we show also in Fig. \ref{fig3} by grey dots the compact SFGs in
the SDSS DR12 \citep{I16c}. The stellar masses for these galaxies are derived 
from the SEDs, using the same technique as described above. It is evident that, 
compared to  J0811$+$4730, the majority of the compact SFGs in  
the SDSS DR12 are characterised by much higher
$M_\star$/$L_g$ ratios. However, there is a 
small, but non-negligible number of SFGs with mass-to-luminosity 
ratios similar to that of J0811$+$4730. They are characterized by high 
equivalent widths EW(H$\beta$) ($>$ 200\AA),  as derived from their SDSS 
spectra.

To study and compare the evolutionary status of compact SFGs, we
present in Fig. \ref{fig3} {\sc Starburst99} continuous star
formation models with a constant SFR, occurring from a specified time $t$ 
in the past to the present. For the sake of definiteness, models have been 
calculated for a metallicity of 1/20 solar. They are shown by dotted lines, 
labelled by their SFRs which vary from 0.001 to 10 M$_\odot$ yr$^{-1}$. 
We indicate by horizontal
solid lines the $M_\star$/$L_g$ ratios corresponding to $t$ = 1 Myr, 10 Myr, 
100 Myr, 1 Gyr and 10 Gyr. The behavior of models for other metallicities is 
similar to that represented in Fig. \ref{fig3}.
It is seen in the Figure that for $t$ $\leq$ 10 Myr,
corresponding to a burst model, the $M_\star$/$L_g$ ratio is 
very low, $\sim$ 0.01, and nearly constant, with a value 
consistent with the one for an instantaneous burst \citep{L99}. For larger
$t$, the $M_\star$/$L_g$ ratio increases with stellar mass, 
following the relation $\sim$ $M_\star^{2/3}$, with a corresponding increase 
of the mass fraction of old stars, approximately following the relation
[$M_\star$($t$) -- $M_\star$($\leq$10 Myr)]/$M_\star$($t$). It is interesting to
note that the distribution of compact SFGs from the SDSS DR12 (grey dots in 
Fig. \ref{fig3}) follows these relations, with an increasing mass fraction of 
stars with age $>$ 10 Myr in more massive compact SFGs, in agreement with the
conclusion of \citet{I11}.

%\sout{Our data do not exclude a small fraction of old
%stars in J0811$+$4730, which contribute little to the luminosity of the 
%galaxy in the optical range. However, the very low $M_\star$/$L_g$ ratio of 
%J0811$+$4730 is also consistent with the hypothesis that this compact SFG 
%could be intrinsically a young galaxy, forming stars now for the first time.}
According to our SED fitting, a considerable fraction of the stellar mass 
in J0811$+$4730 was created during the last burst of star formation.
As for the other very metal-poor SFGs shown in Fig. \ref{fig3},
the mass fraction of older stars is higher. According to the locations
of I~Zw~18, SBS~0335$-$052E and AGC~198691 in this diagram, older stars with 
age $>$ 10 Myr dominate the stellar mass in those galaxies. 

The models in Fig. \ref{fig3} include both the stellar and nebular
continua. However, we note the important contribution of the 
H$\beta$ and [O~{\sc iii}]$\lambda$4959, $\lambda$5007 emission lines to  
$L_g$ in SFGs with a very high EW(H$\beta$). Therefore, to compare the $M_\star$/$L_g$ of those SFGs with model predictions, the contribution of the emission
lines should be subtracted from the $L_g$'s derived from 
their $g$-band magnitudes. The magnitude of the effect depends on the EW(H$\beta$) and the
metallicity. To demonstrate this, we show in Fig. \ref{fig3} by an open
triangle the location of the compact SFG J0159$+$0751 with a $M_\star$/$L_g$
ratio far below the model predictions. This SFG, with an 
oxygen abundance 12+logO/H = 7.55, is characterised by
a very high [O~{\sc iii}]$\lambda$5007/[O~{\sc ii}]$\lambda$3727 flux
ratio of 39 and an EW(H$\beta$) = 347\AA, indicative of  a very young starburst 
\citep*{I17}. However, subtracting the nebular line emission would put J0159$+$0751
in the region allowed by models (shown by an upward arrow in Fig. \ref{fig3}). 
Excluding nebular line emission results in a more modest upward
shift for J0811$+$4730, also shown by an arrow. The shift is smaller because of
a much lower metallicity and thus a much fainter 
[O~{\sc iii}]$\lambda$4959, $\lambda$5007 emission. For the other most 
metal-poor
SFGs shown in Fig. \ref{fig3}, the contribution of nebular emission lines is
even lower because of their considerably lower EW(H$\beta$).



\section{Emission-line diagnostic and metallicity-luminosity diagrams}
\label{sec:diagrams}

\citet{I12} and \citet{G17} have demonstrated that the lowest-metallicity
SFGs occupy a region in the diagram by \citet*{BPT81} (hereafter BPT), that is 
quite different from the location of the 
main-sequence defined by other nearby SFGs. 
This fact is clearly shown in Fig.~\ref{fig4}a.
The data for the lowest-metallicity SFGs in this Figure are 
presented in Table \ref{tab4}. The references in the Table are for the
oxygen abundances.
The most metal-deficient SFG known, 
J0811$+$4730, is shown as an encircled filled circle in this diagram.
It is the most outlying
object among the sequence defined by the lowest-metallicity SFGs 
(filled circles) which is itself much shifted to the left from the ``normal'' 
SFG main-sequence (grey dots).  


%%%%%%%%%%%%%%%%%%%%%%%%%%%%%%%%%%%%%%%%%%%%%%%%
%    Fig.4 diagrams
%%%%%%%%%%%%%%%%%%%%%%%%%%%%%%%%%%%%%%%%%%%%%%%%
\begin{figure}%[t]
\centering{
\includegraphics[angle=-90,width=0.98\linewidth]{diagnDR12_2.ps}
\includegraphics[angle=-90,width=0.98\linewidth]{oiii_oii_c2.ps}
}
\caption{ {\bf (a)} The [O~{\sc iii}]$\lambda$5007/H$\beta$ -- 
[N~{\sc ii}]$\lambda$6584/H$\alpha$ diagnostic diagram 
\citep*[BPT;][]{BPT81}. The lowest-metallicity SFGs with 12+logO/H$<$7.3
are shown by filled circles. The galaxy J0811$+$4730 is encircled.
For comparison, are shown compact SFGs from the SDSS DR12
\citep[grey dots, ][]{I16c}. The thin solid line by \citet{K03} 
separates SFGs from active galactic nuclei (AGN). 
Dashed and thick solid lines represent relations obtained for
{\sc cloudy} photoionized H~{\sc ii} region models with 12+logO/H = 7.3 and
8.0, respectively, and with different starburst 
ages ranging from 0 Myr to 6 Myr. Models with 12+logO/H = 8.0 and 
starburst ages of 0 - 4 Myr, corresponding to the highest H$\beta$ luminosity 
and equivalent width are represent by a thicker solid line.
The model dependences with a zero
starburst age but with a varying filling factor in the range 10$^{-1}$ -- 
10$^{-3}$, for two oxygen abundances 12+logO/H = 8.0 and 8.3, are shown by dotted
line and dash-dotted line, respectively. The directions of increasing age 
and filling factor are indicated respectively by downward and upward arrows.
{\bf (b)} The O$_{32}$ - R$_{23}$ diagram for SFGs, where 
O$_{32}$=[O~{\sc iii}]$\lambda$5007/[O~{\sc ii}]$\lambda$3727 and 
R$_{23}$=([O~{\sc ii}]$\lambda$3727 +
[O~{\sc iii}]$\lambda$4959 + [O~{\sc iii}]$\lambda$5007)/H$\beta$.
The meaning of lines and symbols is the same as in (a). 
}
\label{fig4}
\end{figure}
%%%%%%%%%%%%%%%%%%%%%%%%%%%%%%%%%%%%%%%%%%%%%%%%%

%%%%%%%%%%%%%%%%%%%%%%%%%%%%%%%%%%%%%%%%%%%%%%%%
%    Fig.5 luminosity-metallicity relation
%%%%%%%%%%%%%%%%%%%%%%%%%%%%%%%%%%%%%%%%%%%%%%%%
\begin{figure}%[t]
\centering{
\includegraphics[angle=-90,width=0.98\linewidth]{Mg_o.ps}
}
\caption{The oxygen abundance -- absolute magnitude diagram. Low-luminosity
galaxies (crosses), with the fit to these data by \citet{B12} (dashed line).
The solid line is the fit to the SDSS compact SFGs from DR12 with oxygen 
abundances derived by the direct $T_{\rm e}$ method \citep[grey dots, ][]{I16c}.
%Its extrapolation to low luminosities is shown by the dotted line.
Other symbols are the same as in Fig. \ref{fig4}.
Vertical solid lines indicate the ranges of oxygen abundance in different 
star-forming regions in I Zw 18, SBS 0335$-$052W and SBS 0335$-$052E. 
Horizontal line connects the positions of AGC~198691, calculated 
with two distances, 8 and 16 Mpc.}
\label{fig5}
\end{figure}
%%%%%%%%%%%%%%%%%%%%%%%%%%%%%%%%%%%%%%%%%%%%%%%%%

The shift of the sequence  
to the left of the BPT diagram is caused by lower metallicities.
To demonstrate this, we plot the [O~{\sc iii}]$\lambda$5007/H$\beta$ -- 
[N~{\sc ii}]$\lambda$6584/H$\alpha$ relations as derived from photoionization
H~{\sc ii} region models for two oxygen abundances, 12+logO/H = 7.3 (dashed 
line) and 8.0 (solid line), and for various ages of the 
starburst, ranging from 0 to 6 Myr. The direction of age increase is
indicated by the downward arrow. The models are calculated with the 
{\sc cloudy} code c13.04 \citep{F98,F13}, adopting a production rate
of ionizing radiation $Q$ = 10$^{53}$ s$^{-1}$ and a filling factor 
$f$ = 10$^{-1}$. Since the location of the theoretical relations depends on the
adopted input N/O abundance ratio, we have adopted logN/O = $-$1.5 for models 
with 12+logO/H = 7.3, and logN/O = $-$1.2 for models with 12+logO/H = 8.0, 
values that are typical of low-metallicity SFGs \citep[e.g. ][]{I06}. 
The part of the modelled dependence with 12+logO/H = 8.0 and starburst
ages 0 -- 4 Myr, corresponding to the stages with highest H$\beta$ luminosities
and equivalent widths, is shown by a thicker solid line. \citet{I16c} selected
the SDSS DR12 compact SFGs (grey dots) to have high EW(H$\beta$) and thus they 
correspond to this part of the modelled sequence.
 
We note however that the location of the modelled sequences in 
Fig. \ref{fig4} depends on the adopted input parameters and therefore they
are shown only to illustrate dependences on metallicities. The location of 
main-sequence SFGs is closest to models with oxygen abundance 
12+logO/H = 8.0. 
For comparison, we also show the modelled dependences with a fixed 
starburst age of 0 Myr, but varying the filling factor 
in the range 10$^{-1}$ -- 10$^{-3}$ for oxygen abundances 12~+~logO/H = 8.0 
(dotted line) and 8.3 (dash-dotted line). The effect of an increasing filling factor is shown by an upward arrow. These models reproduce much better
the location of the main-sequence SFGs (grey dots). This is to be expected 
since \citet{I16c}
selected from the SDSS DR12 only compact SFGs with high EW(H$\beta$), as noted
above, i.e. with young bursts.
On the other hand, the location of the lowest-metallicity 
SFGs agrees well with models calculated for an oxygen abundance 
12+logO/H = 7.3. 

Similarly, the lowest-metallicity SFGs are located in the O$_{32}$ -- R$_{23}$ 
diagram in a region which is very different from that of the main-sequence of 
the SDSS SFGs (Fig. \ref{fig4}b), with J0811$+$4730 being one of the 
galaxies  most shifted to the left (encircled filled circle). 
These shifts to the left are again primarily due to lower metallicities.
This is seen by comparing the 
predictions of models with varying starburst age for 
12+logO/H = 7.3 (dashed line) and
8.0 (solid line). It is also seen that models for zero-age starbursts 
with varying filling factors for 12+logO/H = 8.0 (dotted line) and
8.3 (dash-dotted line) reproduce much better the location of the main-sequence
galaxies (grey dots) as compared to the modelled sequence of bursts with a 
varying age (solid line).

We emphasize that the modelled emission-line
ratios in Fig. \ref{fig4} have been calculated taking into account only stellar
ionizing radiation. These ratios may be changed if additional sources of 
ionization and heating such as shocks and X-ray emission are present.
Inclusion of these additional sources would enhance 
the [O~{\sc ii}]$\lambda$3727 emission line relative  
to the [O~{\sc iii}]$\lambda$5007 emission line \citep[e.g. ][]{S15}, reducing 
O$_{32}$ and shifting the modelled sequences downward.

Finally, in Fig. \ref{fig5} we present the oxygen abundance -- absolute 
magnitude relation for SFGs. 
For some galaxies, absolute 
$B$-band magnitudes have been adopted, while for others absolute SDSS $g$-band
magnitudes have been used. This difference in the adopted magnitudes 
will change little the distributions of SFGs in 
Fig. \ref{fig5} because the differences in $g$ and $B$ magnitudes 
for SFGs are small, $\la$ 0.1 mag. We also note that $M_B$ for AGC~198691 
in Table \ref{tab4} is derived assuming a distance of 16 Mpc.

Oxygen abundances for
all galaxies shown in this Figure have been derived by the direct $T_{\rm e}$ 
method.
In particular, only compact SFGs from the SDSS DR12 (grey dots) 
with an [O~{\sc iii}]$\lambda$4363 emission line detected in their spectra
with an accuracy better than 50 per cent, allowing a reasonably good determination
of 12+logO/H by the direct method, are included. Following \citet{I15}, we stress the 
importance of using the same method of metallicity determination for all 
data, to exclude biases introduced by different methods and to produce an 
homogeneous set of data. In particular,
\citet{I15} showed that, concerning the often cited oxygen abundance -- absolute 
magnitude and oxygen abundance -- stellar mass relations derived for SFGs by 
\citet{T04}, the metallicities
derived by strong-line methods can be by as much as 0.5 dex 
higher compared to the metallicities derived by the direct method for the
same SFGs. They are inconsistent with those derived by other strong-line 
methods 
such as those proposed e.g. by \citet{PP04}, \citet{PT05} and \citet{I15}. 
Those authors  
calibrated their relations using oxygen abundances derived by the direct 
method, and the abundances obtained from their calibrations are thus 
consistent with the $T_{\rm e}$ method.

The lowest-metallicity SFGs present a wide range of absolute magnitudes. The 
two galaxies with the lowest luminosities are Leo P
\citep{S13} and AGC~198691 \citep{H16}. The first galaxy follows well  
the relation found by \citet{B12} for relatively quiescent 
SFGs, while the second one deviates somewhat from it to a higher luminosity.
%can be fitted by extrapolating to lower 
%luminosities the fit to SDSS compact SFGs by \citet{I16c}. 

On the other hand, some other lowest-metallicity SFGs, including 
J0811$+$4730, strongly deviate from the \citet{B12} relation.
\citet{I12} and \citet{G17} have attributed these deviations 
to the enhanced brightnesses of the galaxies undergoing  
active star formation. In particular, we estimate that the luminosity 
increase for J0811$+$4730
with its high EW(H$\beta$) would be as much as 2 magnitudes, due to the 
contribution of the nebular continuum and emission lines to the $g$-band 
luminosity. However, such an increase is insufficient to
explain the whole deviation.

Additionally, these galaxies can also be  
chemically unevolved objects with a short star formation history,  
having too low metallicities for their high luminosities. Indeed, 
star formation in a system such as SBS~0335$-$052E is very different from that
in dwarfs like Leo~P and AGC~198691. It is currently undergoing a powerful 
burst, giving birth to bright 
low-metallicity super-star clusters containing thousands of 
massive O-stars \citep*{TIL97}, followed by a delayed element enrichment. 
A similar scenario can be at work in J0811$+$4730 and at high redshifts, in 
zero-metallicity
dwarf primeval galaxies, undergoing their first burst of star formation.

Finally, \citet{EC10} have proposed to explain these low metallicities by 
the infall of gas from galactic halos. The effective mixing
of the metal-poor gas in the halos with the more metal-rich 
gas in the central part of the galaxies will dilute the latter, decreasing its metallicity. Clearly more 
observational and modelling work is needed to understand the true nature of J0811$+$4730.

\section{Conclusions}\label{sec:conclusions}

In this paper we present Large Binocular Telescope (LBT) 
spectrophotometric observations of the compact star-forming galaxy (SFG)
J0811$+$4730 selected from the Data Release 13 (DR13) of the Sloan Digital Sky
Survey (SDSS). We find that the oxygen abundance of this galaxy is 12+logO/H 
= 6.98$\pm$0.02, the lowest ever found for a nearby SFG, and the first one 
below 12+logO/H = 7.0. J0811$+$4730 strongly deviates from the SDSS
main-sequence SFGs in the [O~{\sc iii}]$\lambda$5007/H$\beta$ -- 
[N~{\sc ii}]$\lambda$6584/H$\alpha$ and 
[O~{\sc iii}]$\lambda$5007/[O~{\sc ii}]$\lambda$3727 -- 
([O~{\sc ii}]$\lambda$3727 + [O~{\sc iii}]$\lambda$4959 + 
[O~{\sc iii}]$\lambda$5007)/H$\beta$ diagrams because of its extremely low 
metallicity. In the same way as other galaxies with very low metallicities, it 
is also strongly offset in the oxygen abundance -- absolute magnitude 
diagram from the relations defined by nearby galaxies \citep{B12} and %SDSS 
compact SFGs \citep{I16c}. This offset can probably be explained by a 
combination of its chemically unevolved nature, 
an enhanced brightness of its star-forming regions, 
and gas infall resulting in the dilution of the more metal-rich gas in 
the inner region by the outer more metal-poor gas in the halo.

The properties of the local lowest-metallicity SFGs are likely close to those 
of the high-redshift low-luminosity SFGs recently found at $z$ $>$ 3
\citep{K17}, and thought to have played an important role in the 
reionization of the Universe at $z$ $>$ 5 \citep{O09}.




\section*{Acknowledgements}

We thank D. M. Terndrup and C. Wiens for help with the LBT observations.
We are grateful to anonymous referees for useful comments on the manuscript.
Funding for the Sloan Digital Sky Survey IV has been provided by
the Alfred P. Sloan Foundation, the U.S. Department of Energy Office of
Science, and the Participating Institutions. SDSS-IV acknowledges
support and resources from the Center for High-Performance Computing at
the University of Utah. The SDSS web site is www.sdss.org.
SDSS-IV is managed by the Astrophysical Research Consortium for the 
Participating Institutions of the SDSS Collaboration. 
%Based on observations 
%made with the NASA Galaxy Evolution Explorer. GALEX is operated for NASA by 
%the California Institute of Technology under NASA contract NAS5-98034. 
This research has made use of the NASA/IPAC Extragalactic Database (NED), which 
is operated by the Jet Propulsion Laboratory, California Institute of 
Technology, under contract with the National Aeronautics and Space 
Administration.


\bibliographystyle{./IEEEtran}
\bibliography{./IEEEabrv,./IEEEexample}

@article{kousha2,
	title={Robust Privacy-Utility Tradeoffs Under Differential Privacy and Hamming Distortion},
	author={Kousha Kalantari and Lalitha Sankar and Anand D. Sarwate},
	journal={IEEE Transactions on Information Forensics and Security},
	volume={13},
	number={11},
	pages={2816--2830},
	year={2018},
	publisher={IEEE}
}



@article{DBLP:ITjournal,
  author    = {Naanin Takbiri and
               Amir Houmansadr and
               Dennis Goeckel and
               Hossein Pishro-Nik},
  title     = {Matching Anonymized and Obfuscated Time Series to Users' Profiles
},
  journal   = {CoRR},
  volume    = {abs/1710.00197},
  year      = {2017},
  url       = {https://arxiv.org/abs/1710.00197},
  archivePrefix = {arXiv},
  eprint    = {1710.00197},
  timestamp = {Sat, 30 Sep 2017 13:03:19 GMT},
}


@inproceedings{KeConferance,
   title={Bayesian Time Series Matching and Privacy},
    author={Ke Le and 	Hossein Pishro-Nik and Dennis Goeckel},
	booktitle={51th Asilomar Conference on Signals, Systems and Computers},
   year={2017},
  	address={Pacific Grove, CA}
   }	

@article{DBLP:journals/corr/LiaoSCT17,
  author    = {Jiachun Liao and
               Lalitha Sankar and
               Fl{\'{a}}vio du Pin Calmon and
               Vincent Yan Fu Tan},
  title     = {Hypothesis Testing under Maximal Leakage Privacy Constraints},
  journal   = {CoRR},
  volume    = {abs/1701.07099},
  year      = {2017},
  url       = {http://arxiv.org/abs/1701.07099},
  timestamp = {Wed, 07 Jun 2017 14:40:47 +0200},
  biburl    = {http://dblp.uni-trier.de/rec/bib/journals/corr/LiaoSCT17},
  bibsource = {dblp computer science bibliography, http://dblp.org}
}


@inproceedings{sankar,
    title={On information-theoretic privacy with general distortion cost functions},
    author={Kalantari, Kousha and Sankar, Lalitha and  Kosut, Oliver},
    booktitle={2017 IEEE International Symposium on Information Theory (ISIT)},
    year={2017},
      	address={Aachen, Germany},

    organization={IEEE}
    
  }	


@article{hyposankar,
  author    = {Jiachun Liao and
               Lalitha Sankar and
               Vincent Y. F. Tan and
               Fl{\'{a}}vio du Pin Calmon},
  title     = {Hypothesis Testing in the High Privacy Limit},
  journal   = {CoRR},
  volume    = {abs/1607.00533},
  year      = {2016},
  url       = {http://arxiv.org/abs/1607.00533},
  timestamp = {Wed, 07 Jun 2017 14:41:19 +0200},
  biburl    = {http://dblp.uni-trier.de/rec/bib/journals/corr/LiaoSTC16},
  bibsource = {dblp computer science bibliography, http://dblp.org}
}

@article{battery15,
  author    = {Simon Li and
               Ashish Khisti and
               Aditya Mahajan},
  title     = {Privacy-Optimal Strategies for Smart Metering Systems with a Rechargeable
               Battery},
  journal   = {CoRR},
  volume    = {abs/1510.07170},
  year      = {2015},
  url       = {http://arxiv.org/abs/1510.07170},
  timestamp = {Wed, 07 Jun 2017 14:40:53 +0200},
  biburl    = {http://dblp.uni-trier.de/rec/bib/journals/corr/LiKM15},
  bibsource = {dblp computer science bibliography, http://dblp.org}
}



@inproceedings{geo2013,
   title={Geo-indistinguishability: differential privacy for location-based systems},
    author={Miguel E. Andres and Nicolas E. Bordenabe and Konstantinos Chatzikokolakis and Catuscia Palamidessi},
	booktitle={Proceedings of the 2013 ACM SIGSAC conference on Computer and communications security},

   year={2013},
   pages={901--914},
  	address={New York, NY}
   }	

@article{Yeb17,
  author    = {Min Ye and
               Alexander Barg},
  title     = {Optimal Schemes for Discrete Distribution Estimation under Locally
               Differential Privacy},
  journal   = {CoRR},
  volume    = {abs/1702.00610},
  year      = {2017},
  url       = {http://arxiv.org/abs/1702.00610},
  timestamp = {Wed, 07 Jun 2017 14:41:08 +0200},
  biburl    = {http://dblp.uni-trier.de/rec/bib/journals/corr/YeB17},
  bibsource = {dblp computer science bibliography, http://dblp.org}
}

@inproceedings{info2012,
    title={Information-Theoretic Foundations of Differential Privacy},
    author={Mir J. Darakhshan},
    booktitle={International Symposium on Foundations and Practice of Security},
    year={2012},
    organization={Springer}
  }	

@inproceedings{diff2017,
    title={Dynamic Differential Location Privacy with Personalized Error Bounds},
    author={Lei Yu and Ling Liu and Calton Pu},
    booktitle={The Network and Distributed System Security Symposium},
    year={2017}
  }	

@inproceedings{ciss2017,
    title={Fundamental Limits of Location Privacy using Anonymization},
    author={N. Takbiri and A. Houmansadr and D.L. Goeckel and H. Pishro-Nik},
    booktitle={Annual Conference on Information Science and Systems (CISS)},
    year={2017},
    organization={IEEE}
  }	

@inproceedings{sit2017,
    title={Limits of Location Privacy under Anonymization and Obfuscation},
    author={Nazanin Takbiri and Amir Houmansadr and Dennis L. Goeckel and Hossein Pishro-Nik},
    booktitle={International Symposium on Information Theory (ISIT)},
    year={2017},
    organization={IEEE}
  }	



@article{tifs2016,
	Author = {Z. Montazeri and A. Houmansadr and H. Pishro-Nik},
	Journal = {IEEE Transaction on Information Forensics and Security, to appear},
	Publisher = {IEEE},
	Title = {{Achieving Perfect Location Privacy in Wireless Devices Using Anonymization}},
	Year = {2017}}
	
	@article{matching,
  title={Asymptotically Optimal Matching of Multiple Sequences to Source Distributions and Training Sequences},
  author={Jayakrishnan Unnikrishnan},
  journal={IEEE Transactions on Information Theory},
  volume={61},
  number={1},
  pages={452-468},
  year={2015},
  publisher={IEEE}
}

@article{Naini2016,
	Author = {F. Naini and J. Unnikrishnan and P. Thiran and M. Vetterli},
	Journal = {IEEE Transactions on Information Forensics and Security},
	Publisher = {IEEE},
	Title = {Where You Are Is Who You Are: User Identification by Matching Statistics},
	 volume={11},
    number={2},
     pages={358--372},
    Year = {2016}
}



@inproceedings{montazeri2016defining,
    title={Defining perfect location privacy using anonymization},
    author={Montazeri, Zarrin and Houmansadr, Amir and Pishro-Nik, Hossein},
    booktitle={2016 Annual Conference on Information Science and Systems (CISS)},
    pages={204--209},
    year={2016},
    organization={IEEE}
  }	
	
	
@inproceedings{Mont1610Achieving,
  title={Achieving Perfect Location Privacy in Markov Models Using Anonymization},
  author={Montazeri, Zarrin and Houmansadr, Amir and H.Pishro-Nik},
  booktitle="2016 International Symposium on Information Theory and its Applications
  (ISITA2016)",
  address="Monterey, USA",
  days=30,
  month="oct",
  year=2016,
}

@misc{uber-stats,
title = {{By The Numbers 24 Amazing Uber Statistics}},
author = {Craig Smith},
note = {\url{http://expandedramblings.com/index.php/uber-statistics/}},
year=2015,
month= "September"
}


@misc{GMaps-users,
title = {{55\% of U.S. iOS users with Google Maps use it weekly}},
author = {Mike Dano},
note = {\url{http://www.fiercemobileit.com/story/55-us-ios-users-google-maps-use-it-weekly/2013-08-27}},
year=2013
}

@misc{Google-stats,
title = {{Statistics and facts about Google}},
note = {\url{http://www.statista.com/topics/1001/google/}},
}

@misc{yelp-stats,
title = {{By The Numbers: 45 Amazing Yelp Statistics}},
author = {Craig Smith},
note = {\url{http://expandedramblings.com/index.php/yelp-statistics/}},
year=2015,
month= "May"
}


@misc{Uber-hacker,
title = {{Is Uber's rider database a sitting duck for hackers?}},
author = {Craig Timberg},
note = {\url{https://www.washingtonpost.com/news/the-switch/wp/2014/12/01/is-ubers-rider-database-a-sitting-duck-for-hackers/}},
year=2014,
month= "December"
}

@misc{Uber-godview,
title = {{``God View'': Uber Investigates Its Top New York Executive For Privacy Violations}},
note = {\url{https://www.washingtonpost.com/news/the-switch/wp/2014/12/01/is-ubers-rider-database-a-sitting-duck-for-hackers/}},
year=2014,
month= "November"
}


@misc{Uber-breach-statement,
title = {{Uber Statement}},
note = {\url{http://newsroom.uber.com/2015/02/uber-statement/}},
year=2015,
month= "February"
}



@inproceedings{zhou2007privacy,
  title={Privacy-preserving detection of sybil attacks in vehicular ad hoc networks},
  author={Zhou, Tong and Choudhury, Romit Roy and Ning, Peng and Chakrabarty, Krishnendu},
  booktitle={Mobile and Ubiquitous Systems: Networking \& Services, 2007. MobiQuitous 2007. Fourth Annual International Conference on},
  pages={1--8},
  year={2007},
  organization={IEEE}
}

@article{chang2012footprint,
  title={Footprint: Detecting sybil attacks in urban vehicular networks},
  author={Chang, Shan and Qi, Yong and Zhu, Hongzi and Zhao, Jizhong and Shen, Xuemin Sherman},
  journal={Parallel and Distributed Systems, IEEE Transactions on},
  volume={23},
  number={6},
  pages={1103--1114},
  year={2012},
  publisher={IEEE}
}

@inproceedings{shokri2012protecting,
	Author = {Shokri, Reza and Theodorakopoulos, George and Troncoso, Carmela and Hubaux, Jean-Pierre and Le Boudec, Jean-Yves},
	Booktitle = {Proceedings of the 2012 ACM conference on Computer and communications security},
	Organization = {ACM},
	Pages = {617--627},
	Title = {Protecting location privacy: optimal strategy against localization attacks},
	Year = {2012}}


@inproceedings{gruteser2003anonymous,
	Author = {Gruteser, Marco and Grunwald, Dirk},
	Booktitle = {Proceedings of the 1st international conference on Mobile systems, applications and services},
	Organization = {ACM},
	Pages = {31--42},
	Title = {Anonymous usage of location-based services through spatial and temporal cloaking},
	Year = {2003}}



@inproceedings{hoh2007preserving,
	Author = {Hoh, Baik and Gruteser, Marco and Xiong, Hui and Alrabady, Ansaf},
	Booktitle = {Proceedings of the 14th ACM conference on Computer and communications security},
	Organization = {ACM},
	Pages = {161--171},
	Title = {Preserving privacy in gps traces via uncertainty-aware path cloaking},
	Year = {2007}}


@inproceedings{shokri2011quantifying,
	Author = {Shokri, Reza and Theodorakopoulos, George and Le Boudec, Jean-Yves and Hubaux, Jean-Pierre},
	Booktitle = {Security and Privacy (SP), 2011 IEEE Symposium on},
	Organization = {IEEE},
	Pages = {247--262},
	Title = {Quantifying location privacy},
	Year = {2011}}


@inproceedings{bordenabe2014optimal,
	Author = {Bordenabe, Nicol{\'a}s E and Chatzikokolakis, Konstantinos and Palamidessi, Catuscia},
	Booktitle = {Proceedings of the 2014 ACM SIGSAC Conference on Computer and Communications Security},
	Organization = {ACM},
	Pages = {251--262},
	Title = {Optimal geo-indistinguishable mechanisms for location privacy},
	Year = {2014}}




@inproceedings{freudiger2009non,
	Author = {Freudiger, Julien and Manshaei, Mohammad Hossein and Hubaux, Jean-Pierre and Parkes, David C},
	Booktitle = {Proceedings of the 16th ACM conference on Computer and communications security},
	Organization = {ACM},
	Pages = {324--337},
	Title = {On non-cooperative location privacy: a game-theoretic analysis},
	Year = {2009}}

@inproceedings{ma2009location,
	Author = {Ma, Zhendong and Kargl, Frank and Weber, Michael},
	Booktitle = {Sarnoff Symposium, 2009. SARNOFF'09. IEEE},
	Organization = {IEEE},
	Pages = {1--6},
	Title = {A location privacy metric for v2x communication systems},
	Year = {2009}}

@article{1corser2016evaluating,
  title={Evaluating Location Privacy in Vehicular Communications and Applications},
  author={Corser, George P and Fu, Huirong and Banihani, Abdelnasser},
  journal={IEEE Transactions on Intelligent Transportation Systems},
  volume={17},
  number={9},
  pages={2658-2667},
  year={2016},
  publisher={IEEE}
}
@article{2zhang2016designing,
  title={On Designing Satisfaction-Ratio-Aware Truthful Incentive Mechanisms for k-Anonymity Location Privacy},
  author={Zhang, Yuan and Tong, Wei and Zhong, Sheng},
  journal={IEEE Transactions on Information Forensics and Security},
  volume={11},
  number={11},
  pages={2528--2541},
  year={2016},
  publisher={IEEE}
}

@article{11dewri2014exploiting,
  title={Exploiting service similarity for privacy in location-based search queries},
  author={Dewri, Rinku and Thurimella, Ramakrisha},
  journal={IEEE Transactions on Parallel and Distributed Systems},
  volume={25},
  number={2},
  pages={374--383},
  year={2014},
  publisher={IEEE}
}

@inproceedings{gedik2005location,
	Author = {Gedik, Bu{\u{g}}ra and Liu, Ling},
	Booktitle = {Distributed Computing Systems, 2005. ICDCS 2005. Proceedings. 25th IEEE International Conference on},
	Organization = {IEEE},
	Pages = {620--629},
	Title = {Location privacy in mobile systems: A personalized anonymization model},
	Year = {2005}}

@inproceedings{zhong2009distributed,
	Author = {Zhong, Ge and Hengartner, Urs},
	Booktitle = {Pervasive Computing and Communications, 2009. PerCom 2009. IEEE International Conference on},
	Organization = {IEEE},
	Pages = {1--10},
	Title = {A distributed k-anonymity protocol for location privacy},
	Year = {2009}}

@inproceedings{mokbel2006new,
	Author = {Mokbel, Mohamed F and Chow, Chi-Yin and Aref, Walid G},
	Booktitle = {Proceedings of the 32nd international conference on Very large data bases},
	Organization = {VLDB Endowment},
	Pages = {763--774},
	Title = {The new Casper: query processing for location services without compromising privacy},
	Year = {2006}}

@article{kalnis2007preventing,
	Author = {Kalnis, Panos and Ghinita, Gabriel and Mouratidis, Kyriakos and Papadias, Dimitris},
	Journal = {Knowledge and Data Engineering, IEEE Transactions on},
	Number = {12},
	Pages = {1719--1733},
	Publisher = {IEEE},
	Title = {Preventing location-based identity inference in anonymous spatial queries},
	Volume = {19},
	Year = {2007}}


@article{sweeney2002k,
	Author = {Sweeney, Latanya},
	Journal = {International Journal of Uncertainty, Fuzziness and Knowledge-Based Systems},
	Number = {05},
	Pages = {557--570},
	Publisher = {World Scientific},
	Title = {k-anonymity: A model for protecting privacy},
	Volume = {10},
	Year = {2002}}

@article{sweeney2002achieving,
	Author = {Sweeney, Latanya},
	Journal = {International Journal of Uncertainty, Fuzziness and Knowledge-Based Systems},
	Number = {05},
	Pages = {571-588},
	Publisher = {World Scientific},
	Title = {Achieving k-anonymity privacy protection using generalization and suppression},
	Volume = {10},
	Year = {2002}}

@inproceedings{liu2013game,
	Author = {Liu, Xinxin and Liu, Kaikai and Guo, Linke and Li, Xiaolin and Fang, Yuguang},
	Booktitle = {INFOCOM, 2013 Proceedings IEEE},
	Organization = {IEEE},
	Pages = {2985--2993},
	Title = {A game-theoretic approach for achieving k-anonymity in location based services},
	Year = {2013}}

@inproceedings{hoh2005protecting,
	Author = {Hoh, Baik and Gruteser, Marco},
	Booktitle = {Security and Privacy for Emerging Areas in Communications Networks, 2005. SecureComm 2005. First International Conference on},
	Organization = {IEEE},
	Pages = {194--205},
	Title = {Protecting location privacy through path confusion},
	Year = {2005}}

@article{beresford2003location,
	Author = {Beresford, Alastair R and Stajano, Frank},
	Journal = {IEEE Pervasive computing},
	Number = {1},
	Pages = {46--55},
	Publisher = {IEEE},
	Title = {Location privacy in pervasive computing},
	Year = {2003}}


@inproceedings{palanisamy2011mobimix,
	Author = {Palanisamy, Balaji and Liu, Ling},
	Booktitle = {Data Engineering (ICDE), 2011 IEEE 27th International Conference on},
	Organization = {IEEE},
	Pages = {494--505},
	Title = {Mobimix: Protecting location privacy with mix-zones over road networks},
	Year = {2011}}
@inproceedings{freudiger2009optimal,
	Author = {Freudiger, Julien and Shokri, Reza and Hubaux, Jean-Pierre},
	Booktitle = {Privacy enhancing technologies},
	Organization = {Springer},
	Pages = {216--234},
	Title = {On the optimal placement of mix zones},
	Year = {2009}}

@article{manshaei2013game,
	Author = {Manshaei, Mohammad Hossein and Zhu, Quanyan and Alpcan, Tansu and Bac{\c{s}}ar, Tamer and Hubaux, Jean-Pierre},
	Journal = {ACM Computing Surveys (CSUR)},
	Number = {3},
	Pages = {25},
	Publisher = {ACM},
	Title = {Game theory meets network security and privacy},
	Volume = {45},
	Year = {2013}}

@article{19freudiger2013non,
  title={Non-cooperative location privacy},
  author={Freudiger, Julien and Manshaei, Mohammad Hossein and Hubaux, Jean-Pierre and Parkes, David C},
  journal={IEEE Transactions on Dependable and Secure Computing},
  volume={10},
  number={2},
  pages={84--98},
  year={2013},
  publisher={IEEE}
}


@article{paulet2014privacy,
	Author = {Paulet, Russell and Kaosar, Md Golam and Yi, Xun and Bertino, Elisa},
	Journal = {Knowledge and Data Engineering, IEEE Transactions on},
	Number = {5},
	Pages = {1200--1210},
	Publisher = {IEEE},
	Title = {Privacy-preserving and content-protecting location based queries},
	Volume = {26},
	Year = {2014}}

@article{khoshgozaran2011location,
	Author = {Khoshgozaran, Ali and Shahabi, Cyrus and Shirani-Mehr, Houtan},
	Journal = {Knowledge and Information Systems},
	Number = {3},
	Pages = {435--465},
	Publisher = {Springer},
	Title = {Location privacy: going beyond K-anonymity, cloaking and anonymizers},
	Volume = {26},
	Year = {2011}}

@article{18shokri2014hiding,
  title={Hiding in the mobile crowd: Locationprivacy through collaboration},
  author={Shokri, Reza and Theodorakopoulos, George and Papadimitratos, Panos and Kazemi, Ehsan and Hubaux, Jean-Pierre},
  journal={IEEE transactions on dependable and secure computing},
  volume={11},
  number={3},
  pages={266--279},
  year={2014},
  publisher={IEEE}
}

@article{8zurbaran2015near,
  title={Near-Rand: Noise-based Location Obfuscation Based on Random Neighboring Points},
  author={Zurbaran, Mayra Alejandra and Avila, Karen and Wightman, Pedro and Fernandez, Michael},
  journal={IEEE Latin America Transactions},
  volume={13},
  number={11},
  pages={3661--3667},
  year={2015},
  publisher={IEEE}
}

@inproceedings{hoh2007preserving,
	Author = {Hoh, Baik and Gruteser, Marco and Xiong, Hui and Alrabady, Ansaf},
	Booktitle = {Proceedings of the 14th ACM conference on Computer and communications security},
	Organization = {ACM},
	Pages = {161--171},
	Title = {Preserving privacy in gps traces via uncertainty-aware path cloaking},
	Year = {2007}}

@inproceedings{ho2011differential,
	Author = {Ho, Shen-Shyang and Ruan, Shuhua},
	Booktitle = {Proceedings of the 4th ACM SIGSPATIAL International Workshop on Security and Privacy in GIS and LBS},
	Organization = {ACM},
	Pages = {17--24},
	Title = {Differential privacy for location pattern mining},
	Year = {2011}}


@article{12hwang2014novel,
  title={A novel time-obfuscated algorithm for trajectory privacy protection},
  author={Hwang, Ren-Hung and Hsueh, Yu-Ling and Chung, Hao-Wei},
  journal={IEEE Transactions on Services Computing},
  volume={7},
  number={2},
  pages={126--139},
  year={2014},
  publisher={IEEE}
  }

  @article{16haghnegahdar2014privacy,
  title={Privacy Risks in Publishing Mobile Device Trajectories},
  author={Haghnegahdar, Alireza and Khabbazian, Majid and Bhargava, Vijay K},
  journal={IEEE Wireless Communications Letters},
  volume={3},
  number={3},
  pages={241--244},
  year={2014},
  publisher={IEEE}
}

@article{20gao2013trpf,
  title={TrPF: A trajectory privacy-preserving framework for participatory sensing},
  author={Gao, Sheng and Ma, Jianfeng and Shi, Weisong and Zhan, Guoxing and Sun, Cong},
  journal={IEEE Transactions on Information Forensics and Security},
  volume={8},
  number={6},
  pages={874--887},
  year={2013},
  publisher={IEEE}
}

@article{21ma2013privacy,
  title={Privacy vulnerability of published anonymous mobility traces},
  author={Ma, Chris YT and Yau, David KY and Yip, Nung Kwan and Rao, Nageswara SV},
  journal={IEEE/ACM Transactions on Networking},
  volume={21},
  number={3},
  pages={720--733},
  year={2013},
  publisher={IEEE}
}

@article{6li2016privacy,
  title={Privacy Leakage of Location Sharing in Mobile Social Networks: Attacks and Defense},
  author={Li, Huaxin and Zhu, Haojin and Du, Suguo and Liang, Xiaohui and Shen, Xuemin},
  journal={IEEE Transactions on Dependable and Secure Computing},
  year={2016},
  volume={PP},
  number={99},
  publisher={IEEE}
}


@article{14zhang2014privacy,
  title={Privacy quantification model based on the Bayes conditional risk in Location-Based Services},
  author={Zhang, Xuejun and Gui, Xiaolin and Tian, Feng and Yu, Si and An, Jian},
  journal={Tsinghua Science and Technology},
  volume={19},
  number={5},
  pages={452--462},
  year={2014},
  publisher={TUP}
}

@article{4olteanu2016quantifying,
  title={Quantifying Interdependent Privacy Risks with Location Data},
  author={Olteanu, Alexandra-Mihaela and Huguenin, K{\'e}vin and Shokri, Reza and Humbert, Mathias and Hubaux, Jean-Pierre},
  journal={IEEE Transactions on Mobile Computing},
  year={2016},
  volume={PP},
  number={99},
  pages={1-1},
  publisher={IEEE}
}















@misc{Leberknight2010,
	Author = {Leberknight, C. and Chiang, M. and Poor, H. and Wong, F.},
	Howpublished = {\url{http://www.princeton.edu/~chiangm/anticensorship.pdf}},
	Title = {{A Taxonomy of Internet Censorship and Anti-censorship}},
	Year = {2010}}

@techreport{ultrasurf-analysis,
	Author = {Appelbaum, Jacob},
	Institution = {The Tor Project},
	Title = {{Technical analysis of the Ultrasurf proxying software}},
	Url = {http://scholar.google.com/scholar?hl=en\&btnG=Search\&q=intitle:Technical+analysis+of+the+Ultrasurf+proxying+software\#0},
	Year = {2012},
	Bdsk-Url-1 = {http://scholar.google.com/scholar?hl=en%5C&btnG=Search%5C&q=intitle:Technical+analysis+of+the+Ultrasurf+proxying+software%5C#0}}

@misc{gifc:07,
	Howpublished = {\url{http://www.internetfreedom.org/archive/Defeat\_Internet\_Censorship\_White\_Paper.pdf}},
	Key = {defeatcensorship},
	Publisher = {Global Internet Freedom Consortium (GIFC)},
	Title = {{Defeat Internet Censorship: Overview of Advanced Technologies and Products}},
	Type = {White Paper},
	Year = {2007}}

@article{pan2011survey,
	Author = {Pan, J. and Paul, S. and Jain, R.},
	Journal = {Communications Magazine, IEEE},
	Number = {7},
	Pages = {26--36},
	Publisher = {IEEE},
	Title = {{A Survey of the Research on Future Internet Architectures}},
	Volume = {49},
	Year = {2011}}

@misc{nsf-fia,
	Howpublished = {\url{http://www.nets-fia.net/}},
	Key = {FIA},
	Title = {{NSF Future Internet Architecture Project}}}

@misc{NDN,
	Howpublished = {\url{http://www.named- data.net}},
	Key = {NDN},
	Title = {{Named Data Networking Project}}}

@inproceedings{MobilityFirst,
	Author = {Seskar, I. and Nagaraja, K. and Nelson, S. and Raychaudhuri, D.},
	Booktitle = {Asian Internet Engineering Conference},
	Title = {{Mobilityfirst Future internet Architecture Project}},
	Year = {2011}}

@incollection{NEBULA,
	Author = {Anderson, T. and Birman, K. and Broberg, R. and Caesar, M. and Comer, D. and Cotton, C. and Freedman, M.~J. and Haeberlen, A. and Ives, Z.~G. and Krishnamurthy, A. and others},
	Booktitle = {The Future Internet},
	Pages = {16--26},
	Publisher = {Springer},
	Title = {{The NEBULA Future Internet Architecture}},
	Year = {2013}}

@inproceedings{XIA,
	Author = {Anand, A. and Dogar, F. and Han, D. and Li, B. and Lim, H. and Machado, M. and Wu, W. and Akella, A. and Andersen, D.~G. and Byers, J.~W. and others},
	Booktitle = {ACM Workshop on Hot Topics in Networks},
	Pages = {2},
	Title = {{XIA: An Architecture for an Evolvable and Trustworthy Internet}},
	Year = {2011}}

@inproceedings{ChoiceNet,
	Author = {Rouskas, G.~N. and Baldine, I. and Calvert, K.~L. and Dutta, R. and Griffioen, J. and Nagurney, A. and Wolf, T.},
	Booktitle = {ONDM},
	Title = {{ChoiceNet: Network Innovation Through Choice}},
	Year = {2013}}

@misc{nsf-find,
	Howpublished = {http://www.nets-find.net/},
	Title = {{NSF NeTS FIND Initiative}}}

@article{traid,
	Author = {Cheriton, D.~R. and Gritter, M.},
	Title = {{TRIAD: A New Next-Generation Internet Architecture}},
	Year = {2000}}

@inproceedings{dona,
	Author = {Koponen, T. and Chawla, M. and Chun, B-G. and Ermolinskiy, A. and Kim, K.~H. and Shenker, S. and Stoica, I.},
	Booktitle = {ACM SIGCOMM Computer Communication Review},
	Number = {4},
	Organization = {ACM},
	Pages = {181--192},
	Title = {{A Data-Oriented (and Beyond) Network Architecture}},
	Volume = {37},
	Year = {2007}}

@misc{ultrasurf,
	Howpublished = {\url{http://www.ultrareach.com}},
	Key = {ultrasurf},
	Title = {{Ultrasurf}}}

@misc{tor-bridge,
	Author = {Dingledine, R. and Mathewson, N.},
	Howpublished = {\url{https://svn.torproject.org/svn/projects/design-paper/blocking.html}},
	Title = {{Design of a Blocking-Resistant Anonymity System}}}

@inproceedings{McLachlanH09,
	Author = {J. McLachlan and N. Hopper},
	Booktitle = {WPES},
	Title = {{On the Risks of Serving Whenever You Surf: Vulnerabilities in Tor's Blocking Resistance Design}},
	Year = {2009}}

@inproceedings{mahdian2010,
	Author = {Mahdian, M.},
	Booktitle = {{Fun with Algorithms}},
	Title = {{Fighting Censorship with Algorithms}},
	Year = {2010}}

@inproceedings{McCoy2011,
	Author = {McCoy, D. and Morales, J.~A. and Levchenko, K.},
	Booktitle = {FC},
	Title = {{Proximax: A Measurement Based System for Proxies Dissemination}},
	Year = {2011}}

@inproceedings{Sovran2008,
	Author = {Sovran, Y. and Libonati, A. and Li, J.},
	Booktitle = {IPTPS},
	Title = {{Pass it on: Social Networks Stymie Censors}},
	Year = {2008}}

@inproceedings{rbridge,
	Author = {Wang, Q. and Lin, Zi and Borisov, N. and Hopper, N.},
	Booktitle = {{NDSS}},
	Title = {{rBridge: User Reputation based Tor Bridge Distribution with Privacy Preservation}},
	Year = {2013}}

@inproceedings{telex,
	Author = {Wustrow, E. and Wolchok, S. and Goldberg, I. and Halderman, J.},
	Booktitle = {{USENIX Security}},
	Title = {{Telex: Anticensorship in the Network Infrastructure}},
	Year = {2011}}

@inproceedings{cirripede,
	Author = {Houmansadr, A. and Nguyen, G. and Caesar, M. and Borisov, N.},
	Booktitle = {CCS},
	Title = {{Cirripede: Circumvention Infrastructure Using Router Redirection with Plausible Deniability}},
	Year = {2011}}

@inproceedings{decoyrouting,
	Author = {Karlin, J. and Ellard, D. and Jackson, A. and Jones, C. and Lauer, G. and Mankins, D. and Strayer, W.},
	Booktitle = {{FOCI}},
	Title = {{Decoy Routing: Toward Unblockable Internet Communication}},
	Year = {2011}}

@inproceedings{routing-around-decoys,
	Author = {M.~Schuchard and J.~Geddes and C.~Thompson and N.~Hopper},
	Booktitle = {{CCS}},
	Title = {{Routing Around Decoys}},
	Year = {2012}}

@inproceedings{parrot,
	Author = {A. Houmansadr and C. Brubaker and V. Shmatikov},
	Booktitle = {IEEE S\&P},
	Title = {{The Parrot is Dead: Observing Unobservable Network Communications}},
	Year = {2013}}

@misc{knock,
	Author = {T. Wilde},
	Howpublished = {\url{https://blog.torproject.org/blog/knock-knock-knockin-bridges-doors}},
	Title = {{Knock Knock Knockin' on Bridges' Doors}},
	Year = {2012}}

@inproceedings{china-tor,
	Author = {Winter, P. and Lindskog, S.},
	Booktitle = {{FOCI}},
	Title = {{How the Great Firewall of China Is Blocking Tor}},
	Year = {2012}}

@misc{discover-bridge,
	Howpublished = {\url{https://blog.torproject.org/blog/research-problems-ten-ways-discover-tor-bridges}},
	Key = {tenways},
	Title = {{Ten Ways to Discover Tor Bridges}}}

@inproceedings{freewave,
	Author = {A.~Houmansadr and T.~Riedl and N.~Borisov and A.~Singer},
	Booktitle = {{NDSS}},
	Title = {{I Want My Voice to Be Heard: IP over Voice-over-IP for Unobservable Censorship Circumvention}},
	Year = 2013}

@inproceedings{censorspoofer,
	Author = {Q. Wang and X. Gong and G. Nguyen and A. Houmansadr and N. Borisov},
	Booktitle = {CCS},
	Title = {{CensorSpoofer: Asymmetric Communication Using IP Spoofing for Censorship-Resistant Web Browsing}},
	Year = {2012}}

@inproceedings{skypemorph,
	Author = {H. Moghaddam and B. Li and M. Derakhshani and I. Goldberg},
	Booktitle = {CCS},
	Title = {{SkypeMorph: Protocol Obfuscation for Tor Bridges}},
	Year = {2012}}

@inproceedings{stegotorus,
	Author = {Weinberg, Z. and Wang, J. and Yegneswaran, V. and Briesemeister, L. and Cheung, S. and Wang, F. and Boneh, D.},
	Booktitle = {CCS},
	Title = {{StegoTorus: A Camouflage Proxy for the Tor Anonymity System}},
	Year = {2012}}

@techreport{dust,
	Author = {{B.~Wiley}},
	Howpublished = {\url{http://blanu.net/ Dust.pdf}},
	Institution = {School of Information, University of Texas at Austin},
	Title = {{Dust: A Blocking-Resistant Internet Transport Protocol}},
	Year = {2011}}

@inproceedings{FTE,
	Author = {K.~Dyer and S.~Coull and T.~Ristenpart and T.~Shrimpton},
	Booktitle = {CCS},
	Title = {{Protocol Misidentification Made Easy with Format-Transforming Encryption}},
	Year = {2013}}

@inproceedings{fp,
	Author = {Fifield, D. and Hardison, N. and Ellithrope, J. and Stark, E. and Dingledine, R. and Boneh, D. and Porras, P.},
	Booktitle = {PETS},
	Title = {{Evading Censorship with Browser-Based Proxies}},
	Year = {2012}}

@misc{obfsproxy,
	Howpublished = {\url{https://www.torproject.org/projects/obfsproxy.html.en}},
	Key = {obfsproxy},
	Publisher = {The Tor Project},
	Title = {{A Simple Obfuscating Proxy}}}

@inproceedings{Tor-instead-of-IP,
	Author = {Liu, V. and Han, S. and Krishnamurthy, A. and Anderson, T.},
	Booktitle = {HotNets},
	Title = {{Tor instead of IP}},
	Year = {2011}}

@misc{roger-slides,
	Howpublished = {\url{https://svn.torproject.org/svn/projects/presentations/slides-28c3.pdf}},
	Key = {torblocking},
	Title = {{How Governments Have Tried to Block Tor}}}

@inproceedings{infranet,
	Author = {Feamster, N. and Balazinska, M. and Harfst, G. and Balakrishnan, H. and Karger, D.},
	Booktitle = {USENIX Security},
	Title = {{Infranet: Circumventing Web Censorship and Surveillance}},
	Year = {2002}}

@inproceedings{collage,
	Author = {S.~Burnett and N.~Feamster and S.~Vempala},
	Booktitle = {USENIX Security},
	Title = {{Chipping Away at Censorship Firewalls with User-Generated Content}},
	Year = {2010}}

@article{anonymizer,
	Author = {Boyan, J.},
	Journal = {Computer-Mediated Communication Magazine},
	Month = sep,
	Number = {9},
	Title = {{The Anonymizer: Protecting User Privacy on the Web}},
	Volume = {4},
	Year = {1997}}

@article{schulze2009internet,
	Author = {Schulze, H. and Mochalski, K.},
	Journal = {IPOQUE Report},
	Pages = {351--362},
	Title = {Internet Study 2008/2009},
	Volume = {37},
	Year = {2009}}

@inproceedings{cya-ccs13,
	Author = {J.~Geddes and M.~Schuchard and N.~Hopper},
	Booktitle = {{CCS}},
	Title = {{Cover Your ACKs: Pitfalls of Covert Channel Censorship Circumvention}},
	Year = {2013}}

@inproceedings{andana,
	Author = {DiBenedetto, S. and Gasti, P. and Tsudik, G. and Uzun, E.},
	Booktitle = {{NDSS}},
	Title = {{ANDaNA: Anonymous Named Data Networking Application}},
	Year = {2012}}

@inproceedings{darkly,
	Author = {Jana, S. and Narayanan, A. and Shmatikov, V.},
	Booktitle = {IEEE S\&P},
	Title = {{A Scanner Darkly: Protecting User Privacy From Perceptual Applications}},
	Year = {2013}}

@inproceedings{NS08,
	Author = {A.~Narayanan and V.~Shmatikov},
	Booktitle = {IEEE S\&P},
	Title = {Robust de-anonymization of large sparse datasets},
	Year = {2008}}

@inproceedings{NS09,
	Author = {A.~Narayanan and V.~Shmatikov},
	Booktitle = {IEEE S\&P},
	Title = {De-anonymizing social networks},
	Year = {2009}}

@inproceedings{memento,
	Author = {Jana, S. and Shmatikov, V.},
	Booktitle = {IEEE S\&P},
	Title = {{Memento: Learning secrets from process footprints}},
	Year = {2012}}

@misc{plugtor,
	Howpublished = {\url{https://www.torproject.org/docs/pluggable-transports.html.en}},
	Key = {PluggableTransports},
	Publisher = {The Tor Project},
	Title = {{Tor: Pluggable transports}}}

@misc{psiphon,
	Author = {J.~Jia and P.~Smith},
	Howpublished = {\url{http://www.cdf.toronto.edu/~csc494h/reports/2004-fall/psiphon_ae.html}},
	Title = {{Psiphon: Analysis and Estimation}},
	Year = 2004}

@misc{china-github,
	Howpublished = {\url{http://mobile.informationweek.com/80269/show/72e30386728f45f56b343ddfd0fdb119/}},
	Key = {github},
	Title = {{China's GitHub Censorship Dilemma}}}

@inproceedings{txbox,
	Author = {Jana, S. and Porter, D. and Shmatikov, V.},
	Booktitle = {IEEE S\&P},
	Title = {{TxBox: Building Secure, Efficient Sandboxes with System Transactions}},
	Year = {2011}}

@inproceedings{airavat,
	Author = {I. Roy and S. Setty and A. Kilzer and V. Shmatikov and E. Witchel},
	Booktitle = {NSDI},
	Title = {{Airavat: Security and Privacy for MapReduce}},
	Year = {2010}}

@inproceedings{osdi12,
	Author = {A. Dunn and M. Lee and S. Jana and S. Kim and M. Silberstein and Y. Xu and V. Shmatikov and E. Witchel},
	Booktitle = {OSDI},
	Title = {{Eternal Sunshine of the Spotless Machine: Protecting Privacy with Ephemeral Channels}},
	Year = {2012}}

@inproceedings{ymal,
	Author = {J. Calandrino and A. Kilzer and A. Narayanan and E. Felten and V. Shmatikov},
	Booktitle = {IEEE S\&P},
	Title = {{``You Might Also Like:'' Privacy Risks of Collaborative Filtering}},
	Year = {2011}}

@inproceedings{srivastava11,
	Author = {V. Srivastava and M. Bond and K. McKinley and V. Shmatikov},
	Booktitle = {PLDI},
	Title = {{A Security Policy Oracle: Detecting Security Holes Using Multiple API Implementations}},
	Year = {2011}}

@inproceedings{chen-oakland10,
	Author = {Chen, S. and Wang, R. and Wang, X. and Zhang, K.},
	Booktitle = {IEEE S\&P},
	Title = {{Side-Channel Leaks in Web Applications: A Reality Today, a Challenge Tomorrow}},
	Year = {2010}}

@book{kerck,
	Author = {Kerckhoffs, A.},
	Publisher = {University Microfilms},
	Title = {{La cryptographie militaire}},
	Year = {1978}}

@inproceedings{foci11,
	Author = {J. Karlin and D. Ellard and A.~Jackson and C.~ Jones and G. Lauer and D. Mankins and W.~T.~Strayer},
	Booktitle = {FOCI},
	Title = {{Decoy Routing: Toward Unblockable Internet Communication}},
	Year = 2011}

@inproceedings{sun02,
	Author = {Sun, Q. and Simon, D.~R. and Wang, Y. and Russell, W. and Padmanabhan, V. and Qiu, L.},
	Booktitle = {IEEE S\&P},
	Title = {{Statistical Identification of Encrypted Web Browsing Traffic}},
	Year = {2002}}

@inproceedings{danezis,
	Author = {Murdoch, S.~J. and Danezis, G.},
	Booktitle = {IEEE S\&P},
	Title = {{Low-Cost Traffic Analysis of Tor}},
	Year = {2005}}

@inproceedings{pakicensorship,
	Author = {Z.~Nabi},
	Booktitle = {FOCI},
	Title = {The Anatomy of {Web} Censorship in {Pakistan}},
	Year = {2013}}

@inproceedings{irancensorship,
	Author = {S.~Aryan and H.~Aryan and A.~Halderman},
	Booktitle = {FOCI},
	Title = {Internet Censorship in {Iran}: {A} First Look},
	Year = {2013}}

@inproceedings{ford10efficient,
	Author = {Amittai Aviram and Shu-Chun Weng and Sen Hu and Bryan Ford},
	Booktitle = {\bibconf[9th]{OSDI}{USENIX Symposium on Operating Systems Design and Implementation}},
	Location = {Vancouver, BC, Canada},
	Month = oct,
	Title = {Efficient System-Enforced Deterministic Parallelism},
	Year = 2010}

@inproceedings{ford10determinating,
	Author = {Amittai Aviram and Sen Hu and Bryan Ford and Ramakrishna Gummadi},
	Booktitle = {\bibconf{CCSW}{ACM Cloud Computing Security Workshop}},
	Location = {Chicago, IL},
	Month = oct,
	Title = {Determinating Timing Channels in Compute Clouds},
	Year = 2010}

@inproceedings{ford12plugging,
	Author = {Bryan Ford},
	Booktitle = {\bibconf[4th]{HotCloud}{USENIX Workshop on Hot Topics in Cloud Computing}},
	Location = {Boston, MA},
	Month = jun,
	Title = {Plugging Side-Channel Leaks with Timing Information Flow Control},
	Year = 2012}

@inproceedings{ford12icebergs,
	Author = {Bryan Ford},
	Booktitle = {\bibconf[4th]{HotCloud}{USENIX Workshop on Hot Topics in Cloud Computing}},
	Location = {Boston, MA},
	Month = jun,
	Title = {Icebergs in the Clouds: the {\em Other} Risks of Cloud Computing},
	Year = 2012}

@misc{mullenize,
	Author = {Washington Post},
	Howpublished = {\url{http://apps.washingtonpost.com/g/page/world/gchq-report-on-mullenize-program-to-stain-anonymous-electronic-traffic/502/}},
	Month = {oct},
	Title = {{GCHQ} report on {`MULLENIZE'} program to `stain' anonymous electronic traffic},
	Year = {2013}}

@inproceedings{shue13street,
	Author = {Craig A. Shue and Nathanael Paul and Curtis R. Taylor},
	Booktitle = {\bibbrev[7th]{WOOT}{USENIX Workshop on Offensive Technologies}},
	Month = aug,
	Title = {From an {IP} Address to a Street Address: Using Wireless Signals to Locate a Target},
	Year = 2013}

@inproceedings{knockel11three,
	Author = {Jeffrey Knockel and Jedidiah R. Crandall and Jared Saia},
	Booktitle = {\bibbrev{FOCI}{USENIX Workshop on Free and Open Communications on the Internet}},
	Location = {San Francisco, CA},
	Month = aug,
	Year = 2011}

@misc{rfc4960,
	Author = {R. {Stewart, ed.}},
	Month = sep,
	Note = {RFC 4960},
	Title = {Stream Control Transmission Protocol},
	Year = 2007}

@inproceedings{ford07structured,
	Author = {Bryan Ford},
	Booktitle = {\bibbrev{SIGCOMM}{ACM SIGCOMM}},
	Location = {Kyoto, Japan},
	Month = aug,
	Title = {Structured Streams: a New Transport Abstraction},
	Year = {2007}}

@misc{spdy,
	Author = {Google, Inc.},
	Note = {\url{http://www.chromium.org/spdy/spdy-whitepaper}},
	Title = {{SPDY}: An Experimental Protocol For a Faster {Web}}}

@misc{quic,
	Author = {Jim Roskind},
	Month = jun,
	Note = {\url{http://blog.chromium.org/2013/06/experimenting-with-quic.html}},
	Title = {Experimenting with {QUIC}},
	Year = 2013}

@misc{podjarny12not,
	Author = {G.~Podjarny},
	Month = jun,
	Note = {\url{http://www.guypo.com/technical/not-as-spdy-as-you-thought/}},
	Title = {{Not as SPDY as You Thought}},
	Year = 2012}

@inproceedings{cor,
	Author = {Jones, N.~A. and Arye, M. and Cesareo, J. and Freedman, M.~J.},
	Booktitle = {FOCI},
	Title = {{Hiding Amongst the Clouds: A Proposal for Cloud-based Onion Routing}},
	Year = {2011}}

@misc{torcloud,
	Howpublished = {\url{https://cloud.torproject.org/}},
	Key = {tor cloud},
	Title = {{The Tor Cloud Project}}}

@inproceedings{scramblesuit,
	Author = {Philipp Winter and Tobias Pulls and Juergen Fuss},
	Booktitle = {WPES},
	Title = {{ScrambleSuit: A Polymorphic Network Protocol to Circumvent Censorship}},
	Year = 2013}

@article{savage2000practical,
	Author = {Savage, S. and Wetherall, D. and Karlin, A. and Anderson, T.},
	Journal = {ACM SIGCOMM Computer Communication Review},
	Number = {4},
	Pages = {295--306},
	Publisher = {ACM},
	Title = {Practical network support for IP traceback},
	Volume = {30},
	Year = {2000}}

@inproceedings{ooni,
	Author = {Filast, A. and Appelbaum, J.},
	Booktitle = {{FOCI}},
	Title = {{OONI : Open Observatory of Network Interference}},
	Year = {2012}}

@misc{caida-rank,
	Howpublished = {\url{http://as-rank.caida.org/}},
	Key = {caida rank},
	Title = {{AS Rank: AS Ranking}}}

@inproceedings{usersrouted-ccs13,
	Author = {A.~Johnson and C.~Wacek and R.~Jansen and M.~Sherr and P.~Syverson},
	Booktitle = {CCS},
	Title = {{Users Get Routed: Traffic Correlation on Tor by Realistic Adversaries}},
	Year = {2013}}

@inproceedings{edman2009awareness,
	Author = {Edman, M. and Syverson, P.},
	Booktitle = {{CCS}},
	Title = {{AS-awareness in Tor path selection}},
	Year = {2009}}

@inproceedings{DecoyCosts,
	Author = {A.~Houmansadr and E.~L.~Wong and V.~Shmatikov},
	Booktitle = {NDSS},
	Title = {{No Direction Home: The True Cost of Routing Around Decoys}},
	Year = {2014}}

@article{cordon,
	Author = {Elahi, T. and Goldberg, I.},
	Journal = {University of Waterloo CACR},
	Title = {{CORDON--A Taxonomy of Internet Censorship Resistance Strategies}},
	Volume = {33},
	Year = {2012}}

@inproceedings{privex,
	Author = {T.~Elahi and G.~Danezis and I.~Goldberg	},
	Booktitle = {{CCS}},
	Title = {{AS-awareness in Tor path selection}},
	Year = {2014}}

@inproceedings{changeGuards,
	Author = {T.~Elahi and K.~Bauer and M.~AlSabah and R.~Dingledine and I.~Goldberg},
	Booktitle = {{WPES}},
	Title = {{ Changing of the Guards: Framework for Understanding and Improving Entry Guard Selection in Tor}},
	Year = {2012}}

@article{RAINBOW:Journal,
	Author = {A.~Houmansadr and N.~Kiyavash and N.~Borisov},
	Journal = {IEEE/ACM Transactions on Networking},
	Title = {{Non-Blind Watermarking of Network Flows}},
	Year = 2014}

@inproceedings{info-tod,
	Author = {A.~Houmansadr and S.~Gorantla and T.~Coleman and N.~Kiyavash and and N.~Borisov},
	Booktitle = {{CCS (poster session)}},
	Title = {{On the Channel Capacity of Network Flow Watermarking}},
	Year = {2009}}

@inproceedings{johnson2014game,
	Author = {Johnson, B. and Laszka, A. and Grossklags, J. and Vasek, M. and Moore, T.},
	Booktitle = {Workshop on Bitcoin Research},
	Title = {{Game-theoretic Analysis of DDoS Attacks Against Bitcoin Mining Pools}},
	Year = {2014}}

@incollection{laszka2013mitigation,
	Author = {Laszka, A. and Johnson, B. and Grossklags, J.},
	Booktitle = {Decision and Game Theory for Security},
	Pages = {175--191},
	Publisher = {Springer},
	Title = {{Mitigation of Targeted and Non-targeted Covert Attacks as a Timing Game}},
	Year = {2013}}

@inproceedings{schottle2013game,
	Author = {Schottle, P. and Laszka, A. and Johnson, B. and Grossklags, J. and Bohme, R.},
	Booktitle = {EUSIPCO},
	Title = {{A Game-theoretic Analysis of Content-adaptive Steganography with Independent Embedding}},
	Year = {2013}}

@inproceedings{CloudTransport,
	Author = {C.~Brubaker and A.~Houmansadr and V.~Shmatikov},
	Booktitle = {PETS},
	Title = {{CloudTransport: Using Cloud Storage for Censorship-Resistant Networking}},
	Year = {2014}}

@inproceedings{sweet,
	Author = {W.~Zhou and A.~Houmansadr and M.~Caesar and N.~Borisov},
	Booktitle = {HotPETs},
	Title = {{SWEET: Serving the Web by Exploiting Email Tunnels}},
	Year = {2013}}

@inproceedings{ahsan2002practical,
	Author = {Ahsan, K. and Kundur, D.},
	Booktitle = {Workshop on Multimedia Security},
	Title = {{Practical data hiding in TCP/IP}},
	Year = {2002}}

@incollection{danezis2011covert,
	Author = {Danezis, G.},
	Booktitle = {Security Protocols XVI},
	Pages = {198--214},
	Publisher = {Springer},
	Title = {{Covert Communications Despite Traffic Data Retention}},
	Year = {2011}}

@inproceedings{liu2009hide,
	Author = {Liu, Y. and Ghosal, D. and Armknecht, F. and Sadeghi, A.-R. and Schulz, S. and Katzenbeisser, S.},
	Booktitle = {ESORICS},
	Title = {{Hide and Seek in Time---Robust Covert Timing Channels}},
	Year = {2009}}

@misc{image-watermark-fing,
	Author = {Jonathan Bailey},
	Howpublished = {\url{https://www.plagiarismtoday.com/2009/12/02/image-detection-watermarking-vs-fingerprinting/}},
	Title = {{Image Detection: Watermarking vs. Fingerprinting}},
	Year = {2009}}

@inproceedings{Servetto98,
	Author = {S. D. Servetto and C. I. Podilchuk and K. Ramchandran},
	Booktitle = {Int. Conf. Image Processing},
	Title = {Capacity issues in digital image watermarking},
	Year = {1998}}

@inproceedings{Chen01,
	Author = {B. Chen and G.W.Wornell},
	Booktitle = {IEEE Trans. Inform. Theory},
	Pages = {1423--1443},
	Title = {Quantization index modulation: A class of provably good methods for digital watermarking and information embedding},
	Year = {2001}}

@inproceedings{Karakos00,
	Author = {D. Karakos and A. Papamarcou},
	Booktitle = {IEEE Int. Symp. Information Theory},
	Pages = {47},
	Title = {Relationship between quantization and distribution rates of digitally watermarked data},
	Year = {2000}}

@inproceedings{Sullivan98,
	Author = {J. A. OSullivan and P. Moulin and J. M. Ettinger},
	Booktitle = {IEEE Int. Symp. Information Theory},
	Pages = {297},
	Title = {Information theoretic analysis of steganography},
	Year = {1998}}

@inproceedings{Merhav00,
	Author = {N. Merhav},
	Booktitle = {IEEE Trans. Inform. Theory},
	Pages = {420--430},
	Title = {On random coding error exponents of watermarking systems},
	Year = {2000}}

@inproceedings{Somekh01,
	Author = {A. Somekh-Baruch and N. Merhav},
	Booktitle = {IEEE Int. Symp. Information Theory},
	Pages = {7},
	Title = {On the error exponent and capacity games of private watermarking systems},
	Year = {2001}}

@inproceedings{Steinberg01,
	Author = {Y. Steinberg and N. Merhav},
	Booktitle = {IEEE Trans. Inform. Theory},
	Pages = {1410--1422},
	Title = {Identification in the presence of side information with application to watermarking},
	Year = {2001}}

@article{Moulin03,
	Author = {P. Moulin and J.A. O'Sullivan},
	Journal = {IEEE Trans. Info. Theory},
	Number = {3},
	Title = {Information-theoretic analysis of information hiding},
	Volume = 49,
	Year = 2003}

@article{Gelfand80,
	Author = {S.I.~Gelfand and M.S.~Pinsker},
	Journal = {Problems of Control and Information Theory},
	Number = {1},
	Pages = {19-31},
	Title = {{Coding for channel with random parameters}},
	Url = {citeseer.ist.psu.edu/anantharam96bits.html},
	Volume = {9},
	Year = {1980},
	Bdsk-Url-1 = {citeseer.ist.psu.edu/anantharam96bits.html}}

@book{Wolfowitz78,
	Author = {J. Wolfowitz},
	Edition = {3rd},
	Location = {New York},
	Publisher = {Springer-Verlag},
	Title = {Coding Theorems of Information Theory},
	Year = 1978}

@article{caire99,
	Author = {G. Caire and S. Shamai},
	Journal = {IEEE Transactions on Information Theory},
	Number = {6},
	Pages = {2007--2019},
	Title = {On the Capacity of Some Channels with Channel State Information},
	Volume = {45},
	Year = {1999}}

@inproceedings{wright2007language,
	Author = {Wright, Charles V and Ballard, Lucas and Monrose, Fabian and Masson, Gerald M},
	Booktitle = {USENIX Security},
	Title = {{Language identification of encrypted VoIP traffic: Alejandra y Roberto or Alice and Bob?}},
	Year = {2007}}

@inproceedings{backes2010speaker,
	Author = {Backes, Michael and Doychev, Goran and D{\"u}rmuth, Markus and K{\"o}pf, Boris},
	Booktitle = {{European Symposium on Research in Computer Security (ESORICS)}},
	Pages = {508--523},
	Publisher = {Springer},
	Title = {{Speaker Recognition in Encrypted Voice Streams}},
	Year = {2010}}

@phdthesis{lu2009traffic,
	Author = {Lu, Yuanchao},
	School = {Cleveland State University},
	Title = {{On Traffic Analysis Attacks to Encrypted VoIP Calls}},
	Year = {2009}}

@inproceedings{wright2008spot,
	Author = {Wright, Charles V and Ballard, Lucas and Coull, Scott E and Monrose, Fabian and Masson, Gerald M},
	Booktitle = {IEEE Symposium on Security and Privacy},
	Pages = {35--49},
	Title = {Spot me if you can: Uncovering spoken phrases in encrypted VoIP conversations},
	Year = {2008}}

@inproceedings{white2011phonotactic,
	Author = {White, Andrew M and Matthews, Austin R and Snow, Kevin Z and Monrose, Fabian},
	Booktitle = {IEEE Symposium on Security and Privacy},
	Pages = {3--18},
	Title = {Phonotactic reconstruction of encrypted VoIP conversations: Hookt on fon-iks},
	Year = {2011}}

@inproceedings{fancy,
	Author = {Houmansadr, Amir and Borisov, Nikita},
	Booktitle = {Privacy Enhancing Technologies},
	Organization = {Springer},
	Pages = {205--224},
	Title = {The Need for Flow Fingerprints to Link Correlated Network Flows},
	Year = {2013}}

@article{botmosaic,
	Author = {Amir Houmansadr and Nikita Borisov},
	Doi = {10.1016/j.jss.2012.11.005},
	Issn = {0164-1212},
	Journal = {Journal of Systems and Software},
	Keywords = {Network security},
	Number = {3},
	Pages = {707 - 715},
	Title = {BotMosaic: Collaborative network watermark for the detection of IRC-based botnets},
	Url = {http://www.sciencedirect.com/science/article/pii/S0164121212003068},
	Volume = {86},
	Year = {2013},
	Bdsk-Url-1 = {http://www.sciencedirect.com/science/article/pii/S0164121212003068},
	Bdsk-Url-2 = {http://dx.doi.org/10.1016/j.jss.2012.11.005}}

@inproceedings{ramsbrock2008first,
	Author = {Ramsbrock, Daniel and Wang, Xinyuan and Jiang, Xuxian},
	Booktitle = {Recent Advances in Intrusion Detection},
	Organization = {Springer},
	Pages = {59--77},
	Title = {A first step towards live botmaster traceback},
	Year = {2008}}

@inproceedings{potdar2005survey,
	Author = {Potdar, Vidyasagar M and Han, Song and Chang, Elizabeth},
	Booktitle = {Industrial Informatics, 2005. INDIN'05. 2005 3rd IEEE International Conference on},
	Organization = {IEEE},
	Pages = {709--716},
	Title = {A survey of digital image watermarking techniques},
	Year = {2005}}

@book{cole2003hiding,
	Author = {Cole, Eric and Krutz, Ronald D},
	Publisher = {John Wiley \& Sons, Inc.},
	Title = {Hiding in plain sight: Steganography and the art of covert communication},
	Year = {2003}}

@incollection{akaike1998information,
	Author = {Akaike, Hirotogu},
	Booktitle = {Selected Papers of Hirotugu Akaike},
	Pages = {199--213},
	Publisher = {Springer},
	Title = {Information theory and an extension of the maximum likelihood principle},
	Year = {1998}}

@misc{central-command-hack,
	Author = {Everett Rosenfeld},
	Howpublished = {\url{http://www.cnbc.com/id/102330338}},
	Title = {{FBI investigating Central Command Twitter hack}},
	Year = {2015}}

@misc{sony-psp-ddos,
	Howpublished = {\url{http://n4g.com/news/1644853/sony-and-microsoft-cant-do-much-ddos-attacks-explained}},
	Key = {sony},
	Month = {December},
	Title = {{Sony and Microsoft cant do much -- DDoS attacks explained}},
	Year = {2014}}

@misc{sony-hack,
	Author = {David Bloom},
	Howpublished = {\url{http://goo.gl/MwR4A7}},
	Title = {{Online Game Networks Hacked, Sony Unit President Threatened}},
	Year = {2014}}

@misc{home-depot,
	Author = {Dune Lawrence},
	Howpublished = {\url{http://www.businessweek.com/articles/2014-09-02/home-depots-credit-card-breach-looks-just-like-the-target-hack}},
	Title = {{Home Depot's Suspected Breach Looks Just Like the Target Hack}},
	Year = {2014}}

@misc{target,
	Author = {Julio Ojeda-Zapata},
	Howpublished = {\url{http://www.mercurynews.com/business/ci_24765398/how-did-hackers-pull-off-target-data-heist}},
	Title = {{Target hack: How did they do it?}},
	Year = {2014}}


@article{probabilitycourse,
	Author = {H. Pishro-Nik},
	note = {\url{http://www.probabilitycourse.com}},
	Title = {Introduction to probability, statistics, and random processes},
    Year = {2014}}

@article{our-isit-location,
	Author = {Z. Montazeri and A. Houmansadr and H. Pishro-Nik},
	Journal = {To be submitted to IEEE ISIT},
	Title = {Location Privacy for Multi-State Networks},
	Year = {2016}}






@article{kafsi2013entropy,
	Author = {Kafsi, Mohamed and Grossglauser, Matthias and Thiran, Patrick},
	Journal = {Information Theory, IEEE Transactions on},
	Number = {9},
	Pages = {5577--5583},
	Publisher = {IEEE},
	Title = {The entropy of conditional Markov trajectories},
	Volume = {59},
	Year = {2013}}

@inproceedings{gruteser2003anonymous,
	Author = {Gruteser, Marco and Grunwald, Dirk},
	Booktitle = {Proceedings of the 1st international conference on Mobile systems, applications and services},
	Organization = {ACM},
	Pages = {31--42},
	Title = {Anonymous usage of location-based services through spatial and temporal cloaking},
	Year = {2003}}

@inproceedings{husted2010mobile,
	Author = {Husted, Nathaniel and Myers, Steven},
	Booktitle = {Proceedings of the 17th ACM conference on Computer and communications security},
	Organization = {ACM},
	Pages = {85--96},
	Title = {Mobile location tracking in metro areas: malnets and others},
	Year = {2010}}

@inproceedings{li2009tradeoff,
	Author = {Li, Tiancheng and Li, Ninghui},
	Booktitle = {Proceedings of the 15th ACM SIGKDD international conference on Knowledge discovery and data mining},
	Organization = {ACM},
	Pages = {517--526},
	Title = {On the tradeoff between privacy and utility in data publishing},
	Year = {2009}}





@incollection{humbert2010tracking,
	Author = {Humbert, Mathias and Manshaei, Mohammad Hossein and Freudiger, Julien and Hubaux, Jean-Pierre},
	Booktitle = {Decision and Game Theory for Security},
	Pages = {38--57},
	Publisher = {Springer},
	Title = {Tracking games in mobile networks},
	Year = {2010}}


@article{palamidessi2006probabilistic,
	Author = {Palamidessi, Catuscia},
	Journal = {Electronic Notes in Theoretical Computer Science},
	Pages = {33--42},
	Publisher = {Elsevier},
	Title = {Probabilistic and nondeterministic aspects of anonymity},
	Volume = {155},
	Year = {2006}}



@inproceedings{freudiger2007mix,
	Author = {Freudiger, Julien and Raya, Maxim and F{\'e}legyh{\'a}zi, M{\'a}rk and Papadimitratos, Panos and Hubaux, Jean-Pierre},
	Booktitle = {CM Workshop on Wireless Networking for Intelligent Transportation Systems (WiN-ITS)},
	Title = {Mix-zones for location privacy in vehicular networks},
	Year = {2007}}



@inproceedings{niu2014achieving,
	Author = {Niu, Ben and Li, Qinghua and Zhu, Xiaoyan and Cao, Guohong and Li, Hui},
	Booktitle = {INFOCOM, 2014 Proceedings IEEE},
	Organization = {IEEE},
	Pages = {754--762},
	Title = {Achieving k-anonymity in privacy-aware location-based services},
	Year = {2014}}



@inproceedings{kido2005protection,
	Author = {Kido, Hidetoshi and Yanagisawa, Yutaka and Satoh, Tetsuji},
	Booktitle = {Data Engineering Workshops, 2005. 21st International Conference on},
	Organization = {IEEE},
	Pages = {1248--1248},
	Title = {Protection of location privacy using dummies for location-based services},
	Year = {2005}}

@inproceedings{gedik2005location,
	Author = {Gedik, Bu{\u{g}}ra and Liu, Ling},
	Booktitle = {Distributed Computing Systems, 2005. ICDCS 2005. Proceedings. 25th IEEE International Conference on},
	Organization = {IEEE},
	Pages = {620--629},
	Title = {Location privacy in mobile systems: A personalized anonymization model},
	Year = {2005}}


@incollection{duckham2005formal,
	Author = {Duckham, Matt and Kulik, Lars},
	Booktitle = {Pervasive computing},
	Pages = {152--170},
	Publisher = {Springer},
	Title = {A formal model of obfuscation and negotiation for location privacy},
	Year = {2005}}

@inproceedings{kido2005anonymous,
	Author = {Kido, Hidetoshi and Yanagisawa, Yutaka and Satoh, Tetsuji},
	Booktitle = {Pervasive Services, 2005. ICPS'05. Proceedings. International Conference on},
	Organization = {IEEE},
	Pages = {88--97},
	Title = {An anonymous communication technique using dummies for location-based services},
	Year = {2005}}

@incollection{duckham2006spatiotemporal,
	Author = {Duckham, Matt and Kulik, Lars and Birtley, Athol},
	Booktitle = {Geographic Information Science},
	Pages = {47--64},
	Publisher = {Springer},
	Title = {A spatiotemporal model of strategies and counter strategies for location privacy protection},
	Year = {2006}}

@inproceedings{shankar2009privately,
	Author = {Shankar, Pravin and Ganapathy, Vinod and Iftode, Liviu},
	Booktitle = {Proceedings of the 11th international conference on Ubiquitous computing},
	Organization = {ACM},
	Pages = {31--40},
	Title = {Privately querying location-based services with SybilQuery},
	Year = {2009}}

@inproceedings{chow2009faking,
	Author = {Chow, Richard and Golle, Philippe},
	Booktitle = {Proceedings of the 8th ACM workshop on Privacy in the electronic society},
	Organization = {ACM},
	Pages = {105--108},
	Title = {Faking contextual data for fun, profit, and privacy},
	Year = {2009}}

@incollection{xue2009location,
	Author = {Xue, Mingqiang and Kalnis, Panos and Pung, Hung Keng},
	Booktitle = {Location and Context Awareness},
	Pages = {70--87},
	Publisher = {Springer},
	Title = {Location diversity: Enhanced privacy protection in location based services},
	Year = {2009}}

@article{wernke2014classification,
	Author = {Wernke, Marius and Skvortsov, Pavel and D{\"u}rr, Frank and Rothermel, Kurt},
	Journal = {Personal and Ubiquitous Computing},
	Number = {1},
	Pages = {163--175},
	Publisher = {Springer-Verlag},
	Title = {A classification of location privacy attacks and approaches},
	Volume = {18},
	Year = {2014}}

@misc{cai2015cloaking,
	Author = {Cai, Y. and Xu, G.},
	Month = jan # {~1},
	Note = {US Patent App. 14/472,462},
	Publisher = {Google Patents},
	Title = {Cloaking with footprints to provide location privacy protection in location-based services},
	Url = {https://www.google.com/patents/US20150007341},
	Year = {2015},
	Bdsk-Url-1 = {https://www.google.com/patents/US20150007341}}

@article{gedik2008protecting,
	Author = {Gedik, Bu{\u{g}}ra and Liu, Ling},
	Journal = {Mobile Computing, IEEE Transactions on},
	Number = {1},
	Pages = {1--18},
	Publisher = {IEEE},
	Title = {Protecting location privacy with personalized k-anonymity: Architecture and algorithms},
	Volume = {7},
	Year = {2008}}

@article{kalnis2006preserving,
	Author = {Kalnis, Panos and Ghinita, Gabriel and Mouratidis, Kyriakos and Papadias, Dimitris},
	Publisher = {TRB6/06},
	Title = {Preserving anonymity in location based services},
	Year = {2006}}



@article{terrovitis2011privacy,
	Author = {Terrovitis, Manolis},
	Journal = {ACM SIGKDD Explorations Newsletter},
	Number = {1},
	Pages = {6--18},
	Publisher = {ACM},
	Title = {Privacy preservation in the dissemination of location data},
	Volume = {13},
	Year = {2011}}

@article{shin2012privacy,
	Author = {Shin, Kang G and Ju, Xiaoen and Chen, Zhigang and Hu, Xin},
	Journal = {Wireless Communications, IEEE},
	Number = {1},
	Pages = {30--39},
	Publisher = {IEEE},
	Title = {Privacy protection for users of location-based services},
	Volume = {19},
	Year = {2012}}



@incollection{chatzikokolakis2015geo,
	Author = {Chatzikokolakis, Konstantinos and Palamidessi, Catuscia and Stronati, Marco},
	Booktitle = {Distributed Computing and Internet Technology},
	Pages = {49--72},
	Publisher = {Springer},
	Title = {Geo-indistinguishability: A Principled Approach to Location Privacy},
	Year = {2015}}

@inproceedings{ngo2015location,
	Author = {Ngo, Hoa and Kim, Jong},
	Booktitle = {Computer Security Foundations Symposium (CSF), 2015 IEEE 28th},
	Organization = {IEEE},
	Pages = {63--74},
	Title = {Location Privacy via Differential Private Perturbation of Cloaking Area},
	Year = {2015}}


@inproceedings{um2010advanced,
	Author = {Um, Jung-Ho and Kim, Hee-Dae and Chang, Jae-Woo},
	Booktitle = {Social Computing (SocialCom), 2010 IEEE Second International Conference on},
	Organization = {IEEE},
	Pages = {1093--1098},
	Title = {An advanced cloaking algorithm using Hilbert curves for anonymous location based service},
	Year = {2010}}

@inproceedings{bamba2008supporting,
	Author = {Bamba, Bhuvan and Liu, Ling and Pesti, Peter and Wang, Ting},
	Booktitle = {Proceedings of the 17th international conference on World Wide Web},
	Organization = {ACM},
	Pages = {237--246},
	Title = {Supporting anonymous location queries in mobile environments with privacygrid},
	Year = {2008}}

@inproceedings{zhangwei2010distributed,
	Author = {Zhangwei, Huang and Mingjun, Xin},
	Booktitle = {Networks Security Wireless Communications and Trusted Computing (NSWCTC), 2010 Second International Conference on},
	Organization = {IEEE},
	Pages = {468--471},
	Title = {A distributed spatial cloaking protocol for location privacy},
	Volume = {2},
	Year = {2010}}

@article{chow2011spatial,
	Author = {Chow, Chi-Yin and Mokbel, Mohamed F and Liu, Xuan},
	Journal = {GeoInformatica},
	Number = {2},
	Pages = {351--380},
	Publisher = {Springer},
	Title = {Spatial cloaking for anonymous location-based services in mobile peer-to-peer environments},
	Volume = {15},
	Year = {2011}}

@inproceedings{lu2008pad,
	Author = {Lu, Hua and Jensen, Christian S and Yiu, Man Lung},
	Booktitle = {Proceedings of the Seventh ACM International Workshop on Data Engineering for Wireless and Mobile Access},
	Organization = {ACM},
	Pages = {16--23},
	Title = {Pad: privacy-area aware, dummy-based location privacy in mobile services},
	Year = {2008}}

@incollection{khoshgozaran2007blind,
	Author = {Khoshgozaran, Ali and Shahabi, Cyrus},
	Booktitle = {Advances in Spatial and Temporal Databases},
	Pages = {239--257},
	Publisher = {Springer},
	Title = {Blind evaluation of nearest neighbor queries using space transformation to preserve location privacy},
	Year = {2007}}

@inproceedings{ghinita2008private,
	Author = {Ghinita, Gabriel and Kalnis, Panos and Khoshgozaran, Ali and Shahabi, Cyrus and Tan, Kian-Lee},
	Booktitle = {Proceedings of the 2008 ACM SIGMOD international conference on Management of data},
	Organization = {ACM},
	Pages = {121--132},
	Title = {Private queries in location based services: anonymizers are not necessary},
	Year = {2008}}



@article{nguyen2013differential,
	Author = {Nguyen, Hiep H and Kim, Jong and Kim, Yoonho},
	Journal = {Journal of Computing Science and Engineering},
	Number = {3},
	Pages = {177--186},
	Title = {Differential privacy in practice},
	Volume = {7},
	Year = {2013}}

@inproceedings{lee2012differential,
	Author = {Lee, Jaewoo and Clifton, Chris},
	Booktitle = {Proceedings of the 18th ACM SIGKDD international conference on Knowledge discovery and data mining},
	Organization = {ACM},
	Pages = {1041--1049},
	Title = {Differential identifiability},
	Year = {2012}}

@inproceedings{andres2013geo,
	Author = {Andr{\'e}s, Miguel E and Bordenabe, Nicol{\'a}s E and Chatzikokolakis, Konstantinos and Palamidessi, Catuscia},
	Booktitle = {Proceedings of the 2013 ACM SIGSAC conference on Computer \& communications security},
	Organization = {ACM},
	Pages = {901--914},
	Title = {Geo-indistinguishability: Differential privacy for location-based systems},
	Year = {2013}}

@inproceedings{machanavajjhala2008privacy,
	Author = {Machanavajjhala, Ashwin and Kifer, Daniel and Abowd, John and Gehrke, Johannes and Vilhuber, Lars},
	Booktitle = {Data Engineering, 2008. ICDE 2008. IEEE 24th International Conference on},
	Organization = {IEEE},
	Pages = {277--286},
	Title = {Privacy: Theory meets practice on the map},
	Year = {2008}}

@article{dewri2013local,
	Author = {Dewri, Rinku},
	Journal = {Mobile Computing, IEEE Transactions on},
	Number = {12},
	Pages = {2360--2372},
	Publisher = {IEEE},
	Title = {Local differential perturbations: Location privacy under approximate knowledge attackers},
	Volume = {12},
	Year = {2013}}

@inproceedings{chatzikokolakis2013broadening,
	Author = {Chatzikokolakis, Konstantinos and Andr{\'e}s, Miguel E and Bordenabe, Nicol{\'a}s Emilio and Palamidessi, Catuscia},
	Booktitle = {Privacy Enhancing Technologies},
	Organization = {Springer},
	Pages = {82--102},
	Title = {Broadening the Scope of Differential Privacy Using Metrics.},
	Year = {2013}}



@inproceedings{cheng2006preserving,
	Author = {Cheng, Reynold and Zhang, Yu and Bertino, Elisa and Prabhakar, Sunil},
	Booktitle = {Privacy Enhancing Technologies},
	Organization = {Springer},
	Pages = {393--412},
	Title = {Preserving user location privacy in mobile data management infrastructures},
	Year = {2006}}




@article{krumm2009survey,
	Author = {Krumm, John},
	Journal = {Personal and Ubiquitous Computing},
	Number = {6},
	Pages = {391--399},
	Publisher = {Springer},
	Title = {A survey of computational location privacy},
	Volume = {13},
	Year = {2009}}

@article{Rakhshan2015letter,
	Author = {Rakhshan, Ali and Pishro-Nik, Hossein},
	Journal = {IEEE Wireless Communications Letter},
	Publisher = {IEEE},
	Title = {A Stochastic Geometry Model for Customized Vehicular Communication},
	Year = {2015, submitted}}

@article{Rakhshan2015Journal,
	Author = {Rakhshan, Ali and Pishro-Nik, Hossein},
	Journal = {IEEE Transactions on Wireless Communications},
	Publisher = {IEEE},
	Title = {Improving Safety on Highways by Customizing Vehicular Ad Hoc Networks},
	Year = {2015, submitted}}

@inproceedings{Rakhshan2015Cogsima,
	Author = {Rakhshan, Ali and Pishro-Nik, Hossein},
	Booktitle = {IEEE International Multi-Disciplinary Conference on Cognitive Methods in Situation Awareness and Decision Support},
	Organization = {IEEE},
	Title = {A New Approach to Customization of Accident Warning Systems to Individual Drivers},
	Year = {2015}}

@inproceedings{Rakhshan2015CISS,
	Author = {Rakhshan, Ali and Pishro-Nik, Hossein and Nekoui, Mohammad},
	Booktitle = {Conference on Information Sciences and Systems},
	Organization = {IEEE},
	Pages = {1--6},
	Title = {Driver-based adaptation of Vehicular Ad Hoc Networks for design of active safety systems},
	Year = {2015}}

@inproceedings{Rakhshan2014IV,
	Author = {Rakhshan, Ali and Pishro-Nik, Hossein and Ray, Evan},
	Booktitle = {Intelligent Vehicles Symposium},
	Organization = {IEEE},
	Pages = {1181--1186},
	Title = {Real-time estimation of the distribution of brake response times for an individual driver using Vehicular Ad Hoc Network.},
	Year = {2014}}

@inproceedings{Rakhshan2013Globecom,
	Author = {Rakhshan, Ali and Pishro-Nik, Hossein and Fisher, Donald and Nekoui, Mohammad},
	Booktitle = {IEEE Global Communications Conference},
	Organization = {IEEE},
	Pages = {1333--1337},
	Title = {Tuning collision warning algorithms to individual drivers for design of active safety systems.},
	Year = {2013}}

@article{Nekoui2012Journal,
	Author = {Nekoui, Mohammad and Pishro-Nik, Hossein},
	Journal = {IEEE Transactions on Wireless Communications},
	Number = {8},
	Pages = {2895--2905},
	Publisher = {IEEE},
	Title = {Throughput Scaling laws for Vehicular Ad Hoc Networks},
	Volume = {11},
	Year = {2012}}

@article{Nekoui2013Journal,
	Author = {Nekoui, Mohammad and Pishro-Nik, Hossein},
	Journal = {Journal on Selected Areas in Communications, Special Issue on Emerging Technologies in Communications},
	Number = {9},
	Pages = {491--503},
	Publisher = {IEEE},
	Title = {Analytic Design of Active Safety Systems for Vehicular Ad hoc Networks},
	Volume = {31},
	Year = {2013}}

@article{Nekoui2011Journal,
	Author = {Nekoui, Mohammad and Pishro-Nik, Hossein and Ni, Daiheng},
	Journal = {International Journal of Vehicular Technology},
	Pages = {1--11},
	Publisher = {Hindawi Publishing Corporation},
	Title = {Analytic Design of Active Safety Systems for Vehicular Ad hoc Networks},
	Volume = {2011},
	Year = {2011}}

@inproceedings{Nekoui2011MOBICOM,
	Author = {Nekoui, Mohammad and Pishro-Nik, Hossein},
	Booktitle = {MOBICOM workshop on VehiculAr InterNETworking},
	Organization = {ACM},
	Title = {Analytic Design of Active Vehicular Safety Systems in Sparse Traffic},
	Year = {2011}}

@inproceedings{Nekoui2011VTC,
	Author = {Nekoui, Mohammad and Pishro-Nik, Hossein},
	Booktitle = {VTC-Fall},
	Organization = {IEEE},
	Title = {Analytical Design of Inter-vehicular Communications for Collision Avoidance},
	Year = {2011}}

@inproceedings{Bovee2011VTC,
	Author = {Bovee, Ben Louis and Nekoui, Mohammad and Pishro-Nik, Hossein},
	Booktitle = {VTC-Fall},
	Organization = {IEEE},
	Title = {Evaluation of the Universal Geocast Scheme For VANETs},
	Year = {2011}}

@inproceedings{Nekoui2010MOBICOM,
	Author = {Nekoui, Mohammad and Pishro-Nik, Hossein},
	Booktitle = {MOBICOM},
	Organization = {ACM},
	Title = {Fundamental Tradeoffs in Vehicular Ad Hoc Networks},
	Year = {2010}}

@inproceedings{Nekoui2010IVCS,
	Author = {Nekoui, Mohammad and Pishro-Nik, Hossein},
	Booktitle = {IVCS},
	Organization = {IEEE},
	Title = {A Universal Geocast Scheme for Vehicular Ad Hoc Networks},
	Year = {2010}}

@inproceedings{Nekoui2009ITW,
	Author = {Nekoui, Mohammad and Pishro-Nik, Hossein},
	Booktitle = {IEEE Communications Society Conference on Sensor, Mesh and Ad Hoc Communications and Networks Workshops},
	Organization = {IEEE},
	Pages = {1--3},
	Title = {A Geometrical Analysis of Obstructed Wireless Networks},
	Year = {2009}}

@article{Eslami2013Journal,
	Author = {Eslami, Ali and Nekoui, Mohammad and Pishro-Nik, Hossein and Fekri, Faramarz},
	Journal = {ACM Transactions on Sensor Networks},
	Number = {4},
	Pages = {51},
	Publisher = {ACM},
	Title = {Results on finite wireless sensor networks: Connectivity and coverage},
	Volume = {9},
	Year = {2013}}

@article{shokri2014optimal,
	  title={Optimal user-centric data obfuscation},
 	 author={Shokri, Reza},
 	 journal={arXiv preprint arXiv:1402.3426},
 	 year={2014}
	}
@article{chatzikokolakis2015location,
  title={Location privacy via geo-indistinguishability},
  author={Chatzikokolakis, Konstantinos and Palamidessi, Catuscia and Stronati, Marco},
  journal={ACM SIGLOG News},
  volume={2},
  number={3},
  pages={46--69},
  year={2015},
  publisher={ACM}

}
@inproceedings{shokri2011quantifying2,
  title={Quantifying location privacy: the case of sporadic location exposure},
  author={Shokri, Reza and Theodorakopoulos, George and Danezis, George and Hubaux, Jean-Pierre and Le Boudec, Jean-Yves},
  booktitle={Privacy Enhancing Technologies},
  pages={57--76},
  year={2011},
  organization={Springer}
}

@inproceedings{calmon2015fundamental,
  title={Fundamental limits of perfect privacy},
  author={Calmon, Flavio P and Makhdoumi, Ali and M{\'e}dard, Muriel},
  booktitle={Information Theory (ISIT), 2015 IEEE International Symposium on},
  pages={1796--1800},
  year={2015},
  organization={IEEE}
}

@inproceedings{salamatian2013hide,
  title={How to hide the elephant-or the donkey-in the room: Practical privacy against statistical inference for large data.},
  author={Salamatian, Salman and Zhang, Amy and du Pin Calmon, Flavio and Bhamidipati, Sandilya and Fawaz, Nadia and Kveton, Branislav and Oliveira, Pedro and Taft, Nina},
  booktitle={GlobalSIP},
  pages={269--272},
  year={2013}
}

@article{sankar2013utility,
  title={Utility-privacy tradeoffs in databases: An information-theoretic approach},
  author={Sankar, Lalitha and Rajagopalan, S Raj and Poor, H Vincent},
  journal={Information Forensics and Security, IEEE Transactions on},
  volume={8},
  number={6},
  pages={838--852},
  year={2013},
  publisher={IEEE}
}
@inproceedings{ghinita2007prive,
  title={PRIVE: anonymous location-based queries in distributed mobile systems},
  author={Ghinita, Gabriel and Kalnis, Panos and Skiadopoulos, Spiros},
  booktitle={Proceedings of the 16th international conference on World Wide Web},
  pages={371--380},
  year={2007},
  organization={ACM}
}

@article{beresford2004mix,
  title={Mix zones: User privacy in location-aware services},
  author={Beresford, Alastair R and Stajano, Frank},
  year={2004},
  publisher={IEEE}
}


@article{csiszar1996almost,
  title={Almost independence and secrecy capacity},
  author={Csisz{\'a}r, Imre},
  journal={Problemy Peredachi Informatsii},
  volume={32},
  number={1},
  pages={48--57},
  year={1996},
  publisher={Russian Academy of Sciences, Branch of Informatics, Computer Equipment and Automatization}
}

@article{yamamoto1983source,
  title={A source coding problem for sources with additional outputs to keep secret from the receiver or wiretappers (corresp.)},
  author={Yamamoto, Hirosuke},
  journal={IEEE Transactions on Information Theory},
  volume={29},
  number={6},
  pages={918--923},
  year={1983},
  publisher={IEEE}
}



  @inproceedings{golle2009anonymity,
  title={On the anonymity of home/work location pairs},
  author={Golle, Philippe and Partridge, Kurt},
  booktitle={International Conference on Pervasive Computing},
  pages={390--397},
  year={2009},
  organization={Springer}
}

@inproceedings{zang2011anonymization,
  title={Anonymization of location data does not work: A large-scale measurement study},
  author={Zang, Hui and Bolot, Jean},
  booktitle={Proceedings of the 17th annual international conference on Mobile computing and networking},
  pages={145--156},
  year={2011},
  organization={ACM}
}
@article{wang2015privacy,
  title={Privacy-preserving collaborative spectrum sensing with multiple service providers},
  author={Wang, Wei and Zhang, Qian},
  journal={IEEE Transactions on Wireless Communications},
  volume={14},
  number={2},
  pages={1011--1019},
  year={2015},
  publisher={IEEE}
}

@article{wang2015toward,
  title={Toward long-term quality of protection in mobile networks: a context-aware perspective},
  author={Wang, Wei and Zhang, Qian},
  journal={IEEE Wireless Communications},
  volume={22},
  number={4},
  pages={34--40},
  year={2015},
  publisher={IEEE}
}

@inproceedings{niu2015enhancing,
  title={Enhancing privacy through caching in location-based services},
  author={Niu, Ben and Li, Qinghua and Zhu, Xiaoyan and Cao, Guohong and Li, Hui},
  booktitle={2015 IEEE Conference on Computer Communications (INFOCOM)},
  pages={1017--1025},
  year={2015},
  organization={IEEE}
}

%% This BibTeX bibliography file was created using BibDesk.
%% http://bibdesk.sourceforge.net/

%% Created for Zarrin Montazeri at 2015-11-09 18:45:31 -0500


%% Saved with string encoding Unicode (UTF-8)



%%%%%%%%%%%%%%IOT%%%%%%%%%%%%%%%%%%%%%%%%%%%%%%%%%%%%%%%%%%%%%%%%%%%


@article{osma2015,
	title={Impact of Time-to-Collision Information on Driving Behavior in Connected Vehicle Environments Using A Driving Simulator Test Bed},
	journal{Journal of Traffic and Logistics Engineering},
	author={Osama A. Osman, Julius Codjoe, and Sherif Ishak},
	volume={3},
	number={1},
	pages={18--24},
	year={2015}
}


@article{charisma2010,
	title={Dynamic Latent Plan Models},
	author={Charisma F. Choudhurya, Moshe Ben-Akivab and Maya Abou-Zeid},
	journal={Journal of Choice Modelling},
	volume={3},
	number={2},
	pages={50--70},
	year={2010},
	publisher={Elsvier}
}


@misc{noble2014,
	author = {A. M. Noble, Shane B. McLaughlin, Zachary R. Doerzaph and Thomas A. Dingus},
	title = {Crowd-sourced Connected-vehicle Warning Algorithm using Naturalistic Driving Data},
	howpublished = {Downloaded from \url{http://hdl.handle.net/10919/53978}},
	
	month = August,
	year = 2014
}


@phdthesis{charisma2007,
	title    = {Modeling Driving Decisions with Latent Plans},
	school   = {Massachusetts Institute of Technology },
	author   = {Charisma Farheen Choudhury},
	year     = {2007}, %other attributes omitted
}


@article{chrysler2015,
	title={Cost of Warning of Unseen Threats:Unintended Consequences of Connected Vehicle Alerts},
	author={S. T. Chrysler, J. M. Cooper and D. C. Marshall},
	journal={Transportation Research Record: Journal of the Transportation Research Board},
	volume={2518},
	pages={79--85},
	year={2015},
}





@article{FTC2015,
	title={Internet of Things: Privacy and Security in a Connected World},
	author={FTC Staff Report},
	year={2015}
}



%% Saved with string encoding Unicode (UTF-8)
@inproceedings{1zhou2014security,
	title={Security/privacy of wearable fitness tracking {I}o{T} devices},
	author={Zhou, Wei and Piramuthu, Selwyn},
	booktitle={Information Systems and Technologies (CISTI), 2014 9th Iberian Conference on},
	pages={1--5},
	year={2014},
	organization={IEEE}
}


@inproceedings{3ukil2014iot,
	title={{I}o{T}-privacy: To be private or not to be private},
	author={Arijit Ukil and Soma Bandyopadhyay and Arpan Pal},
	booktitle={Computer Communications Workshops (INFOCOM WKSHPS), IEEE Conference on},
	pages={123--124},
	year={2014},
	organization={IEEE}
}

@inproceedings{4Hosseinzadeh2014,
	title={Security in the Internet of Things through obfuscation and diversification},
	author={Hosseinzadeh, Shohreh and Rauti, Sampsa and Hyrynsalmi, Sami and Leppanen, Ville},
	booktitle={Computing, Communication and Security (ICCCS), IEEE Conference on},
	pages={123--124},
	year={2015},
	organization={IEEE}
}
@article{4arias2015privacy,
	title={Privacy and security in internet of things and wearable devices},
	author={Arias, Orlando and Wurm, Jacob and Hoang, Khoa and Jin, Yier},
	journal={IEEE Transactions on Multi-Scale Computing Systems},
	volume={1},
	number={2},
	pages={99--109},
	year={2015},
	publisher={IEEE}
}
@inproceedings{5ullah2016novel,
	title={A novel model for preserving Location Privacy in Internet of Things},
	author={Ullah, Ikram and Shah, Munam Ali},
	booktitle={Automation and Computing (ICAC), 2016 22nd International Conference on},
	pages={542--547},
	year={2016},
	organization={IEEE}
}
@inproceedings{6sathishkumar2016enhanced,
	title={Enhanced location privacy algorithm for wireless sensor network in Internet of Things},
	author={Sathishkumar, J and Patel, Dhiren R},
	booktitle={Internet of Things and Applications (IOTA), International Conference on},
	pages={208--212},
	year={2016},
	organization={IEEE}
}
@inproceedings{7zhou2012preserving,
	title={Preserving sensor location privacy in internet of things},
	author={Zhou, Liming and Wen, Qiaoyan and Zhang, Hua},
	booktitle={Computational and Information Sciences (ICCIS), 2012 Fourth International Conference on},
	pages={856--859},
	year={2012},
	organization={IEEE}
}

@inproceedings{8ukil2015privacy,
	title={Privacy for {I}o{T}: Involuntary privacy enablement for smart energy systems},
	author={Ukil, Arijit and Bandyopadhyay, Soma and Pal, Arpan},
	booktitle={Communications (ICC), 2015 IEEE International Conference on},
	pages={536--541},
	year={2015},
	organization={IEEE}
}

@inproceedings{9dalipi2016security,
	title={Security and Privacy Considerations for {I}o{T} Application on Smart Grids: Survey and Research Challenges},
	author={Dalipi, Fisnik and Yayilgan, Sule Yildirim},
	booktitle={Future Internet of Things and Cloud Workshops (FiCloudW), IEEE International Conference on},
	pages={63--68},
	year={2016},
	organization={IEEE}
}
@inproceedings{10harris2016security,
	title={Security and Privacy in Public {I}o{T} Spaces},
	author={Harris, Albert F and Sundaram, Hari and Kravets, Robin},
	booktitle={Computer Communication and Networks (ICCCN), 2016 25th International Conference on},
	pages={1--8},
	year={2016},
	organization={IEEE}
}

@inproceedings{11al2015security,
	title={Security and privacy framework for ubiquitous healthcare {I}o{T} devices},
	author={Al Alkeem, Ebrahim and Yeun, Chan Yeob and Zemerly, M Jamal},
	booktitle={Internet Technology and Secured Transactions (ICITST), 2015 10th International Conference for},
	pages={70--75},
	year={2015},
	organization={IEEE}
}
@inproceedings{12sivaraman2015network,
	title={Network-level security and privacy control for smart-home {I}o{T} devices},
	author={Sivaraman, Vijay and Gharakheili, Hassan Habibi and Vishwanath, Arun and Boreli, Roksana and Mehani, Olivier},
	booktitle={Wireless and Mobile Computing, Networking and Communications (WiMob), 2015 IEEE 11th International Conference on},
	pages={163--167},
	year={2015},
	organization={IEEE}
}

@inproceedings{13srinivasan2016privacy,
	title={Privacy conscious architecture for improving emergency response in smart cities},
	author={Srinivasan, Ramya and Mohan, Apurva and Srinivasan, Priyanka},
	booktitle={Smart City Security and Privacy Workshop (SCSP-W), 2016},
	pages={1--5},
	year={2016},
	organization={IEEE}
}
@inproceedings{14sadeghi2015security,
	title={Security and privacy challenges in industrial internet of things},
	author={Sadeghi, Ahmad-Reza and Wachsmann, Christian and Waidner, Michael},
	booktitle={Design Automation Conference (DAC), 2015 52nd ACM/EDAC/IEEE},
	pages={1--6},
	year={2015},
	organization={IEEE}
}
@inproceedings{15otgonbayar2016toward,
	title={Toward Anonymizing {I}o{T} Data Streams via Partitioning},
	author={Otgonbayar, Ankhbayar and Pervez, Zeeshan and Dahal, Keshav},
	booktitle={Mobile Ad Hoc and Sensor Systems (MASS), 2016 IEEE 13th International Conference on},
	pages={331--336},
	year={2016},
	organization={IEEE}
}
@inproceedings{16rutledge2016privacy,
	title={Privacy Impacts of {I}o{T} Devices: A SmartTV Case Study},
	author={Rutledge, Richard L and Massey, Aaron K and Ant{\'o}n, Annie I},
	booktitle={Requirements Engineering Conference Workshops (REW), IEEE International},
	pages={261--270},
	year={2016},
	organization={IEEE}
}

@inproceedings{17andrea2015internet,
	title={Internet of Things: Security vulnerabilities and challenges},
	author={Andrea, Ioannis and Chrysostomou, Chrysostomos and Hadjichristofi, George},
	booktitle={Computers and Communication (ISCC), 2015 IEEE Symposium on},
	pages={180--187},
	year={2015},
	organization={IEEE}
}






























%%%%%%%%%%%%%%%%%%%%%%%%%%%%%%%%%%%%%%%%%%%%%%%%%%%%%%%%%%%


@misc{epfl-mobility-20090224,
	author = {Michal Piorkowski and Natasa Sarafijanovic-Djukic and Matthias Grossglauser},
	title = {{CRAWDAD} dataset epfl/mobility (v. 2009-02-24)},
	howpublished = {Downloaded from \url{http://crawdad.org/epfl/mobility/20090224}},
	doi = {10.15783/C7J010},
	month = feb,
	year = 2009
}

@misc{roma-taxi-20140717,
	author = {Lorenzo Bracciale and Marco Bonola and Pierpaolo Loreti and Giuseppe Bianchi and Raul Amici and Antonello Rabuffi},
	title = {{CRAWDAD} dataset roma/taxi (v. 2014-07-17)},
	howpublished = {Downloaded from \url{http://crawdad.org/roma/taxi/20140717}},
	doi = {10.15783/C7QC7M},
	month = jul,
	year = 2014
}

@misc{rice-ad_hoc_city-20030911,
	author = {Jorjeta G. Jetcheva and Yih-Chun Hu and Santashil PalChaudhuri and Amit Kumar Saha and David B. Johnson},
	title = {{CRAWDAD} dataset rice/ad\_hoc\_city (v. 2003-09-11)},
	howpublished = {Downloaded from \url{http://crawdad.org/rice/ad_hoc_city/20030911}},
	doi = {10.15783/C73K5B},
	month = sep,
	year = 2003
}

@misc{china:2012,
	author = {Microsoft Research Asia},
	title = {GeoLife GPS Trajectories},
	year = {2012},
	howpublished= {\url{https://www.microsoft.com/en-us/download/details.aspx?id=52367}},
}


@misc{china:2011,
	ALTauthor = {Microsoft Research Asia)},
	ALTeditor = {},
	title = {GeoLife GPS Trajectories,
	year = {2012},
	url = {https://www.microsoft.com/en-us/download/details.aspx?id=52367},
	}
	
	
	@misc{longversion,
	author = {N. Takbiri and A. Houmansadr and D.L. Goeckel and H. Pishro-Nik},
	title = {{Limits of Location Privacy under Anonymization and Obfuscation}},
	howpublished = "\url{https://dl.dropboxusercontent.com/u/49263048/ISIT_2017-2.pdf}",
	year = 2017,
	month= "January",
	note = "[Online; accessed 22-Jan-2017]"
	}
	
	
	
	@article{matching,
	title={Asymptotically Optimal Matching of Multiple Sequences to Source Distributions and Training Sequences},
	author={Jayakrishnan Unnikrishnan},
	journal={ IEEE Transactions on Information Theory},
	volume={61},
	number={1},
	pages={452-468},
	year={2015},
	publisher={IEEE}
	}
	
	
	@article{Naini2016,
	Author = {F. Naini and J. Unnikrishnan and P. Thiran and M. Vetterli},
	Journal = {IEEE Transactions on Information Forensics and Security},
	Publisher = {IEEE},
	Title = {Where You Are Is Who You Are: User Identification by Matching Statistics},
	volume={11},
	number={2},
	pages={358--372},
	Year = {2016}
	}
	
	
	
	@inproceedings{holowczak2015cachebrowser,
	title={{CacheBrowser: Bypassing Chinese Censorship without Proxies Using Cached Content}},
	author={Holowczak, John and Houmansadr, Amir},
	booktitle={Proceedings of the 22nd ACM SIGSAC Conference on Computer and Communications Security},
	pages={70--83},
	year={2015},
	organization={ACM}
	}
	@misc{cb-website,
	Howpublished = {\url{https://cachebrowser.net/#/}},
	Title = {{CacheBrowser}},
	key={cachebrowser}
	}
	
	@inproceedings{GameOfDecoys,
	title={{GAME OF DECOYS: Optimal Decoy Routing Through Game Theory}},
	author={Milad Nasr and Amir Houmansadr},
	booktitle={The $23^{rd}$ ACM Conference on Computer and Communications Security (CCS)},
	year={2016}
	}
	
	@inproceedings{CDNReaper,
	title={{Practical Censorship Evasion Leveraging Content Delivery Networks}},
	author={Hadi Zolfaghari and Amir Houmansadr},
	booktitle={The $23^{rd}$ ACM Conference on Computer and Communications Security (CCS)},
	year={2016}
	}
	
	@misc{Leberknight2010,
	Author = {Leberknight, C. and Chiang, M. and Poor, H. and Wong, F.},
	Howpublished = {\url{http://www.princeton.edu/~chiangm/anticensorship.pdf}},
	Title = {{A Taxonomy of Internet Censorship and Anti-censorship}},
	Year = {2010}}
	
	@techreport{ultrasurf-analysis,
	Author = {Appelbaum, Jacob},
	Institution = {The Tor Project},
	Title = {{Technical analysis of the Ultrasurf proxying software}},
	Url = {http://scholar.google.com/scholar?hl=en\&btnG=Search\&q=intitle:Technical+analysis+of+the+Ultrasurf+proxying+software\#0},
	Year = {2012},
	Bdsk-Url-1 = {http://scholar.google.com/scholar?hl=en%5C&btnG=Search%5C&q=intitle:Technical+analysis+of+the+Ultrasurf+proxying+software%5C#0}}
	
	@misc{gifc:07,
	Howpublished = {\url{http://www.internetfreedom.org/archive/Defeat\_Internet\_Censorship\_White\_Paper.pdf}},
	Key = {defeatcensorship},
	Publisher = {Global Internet Freedom Consortium (GIFC)},
	Title = {{Defeat Internet Censorship: Overview of Advanced Technologies and Products}},
	Type = {White Paper},
	Year = {2007}}
	
	@article{pan2011survey,
	Author = {Pan, J. and Paul, S. and Jain, R.},
	Journal = {Communications Magazine, IEEE},
	Number = {7},
	Pages = {26--36},
	Publisher = {IEEE},
	Title = {{A Survey of the Research on Future Internet Architectures}},
	Volume = {49},
	Year = {2011}}
	
	@misc{nsf-fia,
	Howpublished = {\url{http://www.nets-fia.net/}},
	Key = {FIA},
	Title = {{NSF Future Internet Architecture Project}}}
	
	@misc{NDN,
	Howpublished = {\url{http://www.named- data.net}},
	Key = {NDN},
	Title = {{Named Data Networking Project}}}
	
	@inproceedings{MobilityFirst,
	Author = {Seskar, I. and Nagaraja, K. and Nelson, S. and Raychaudhuri, D.},
	Booktitle = {Asian Internet Engineering Conference},
	Title = {{Mobilityfirst Future internet Architecture Project}},
	Year = {2011}}
	
	@incollection{NEBULA,
	Author = {Anderson, T. and Birman, K. and Broberg, R. and Caesar, M. and Comer, D. and Cotton, C. and Freedman, M.~J. and Haeberlen, A. and Ives, Z.~G. and Krishnamurthy, A. and others},
	Booktitle = {The Future Internet},
	Pages = {16--26},
	Publisher = {Springer},
	Title = {{The NEBULA Future Internet Architecture}},
	Year = {2013}}
	
	@inproceedings{XIA,
	Author = {Anand, A. and Dogar, F. and Han, D. and Li, B. and Lim, H. and Machado, M. and Wu, W. and Akella, A. and Andersen, D.~G. and Byers, J.~W. and others},
	Booktitle = {ACM Workshop on Hot Topics in Networks},
	Pages = {2},
	Title = {{XIA: An Architecture for an Evolvable and Trustworthy Internet}},
	Year = {2011}}
	
	@inproceedings{ChoiceNet,
	Author = {Rouskas, G.~N. and Baldine, I. and Calvert, K.~L. and Dutta, R. and Griffioen, J. and Nagurney, A. and Wolf, T.},
	Booktitle = {ONDM},
	Title = {{ChoiceNet: Network Innovation Through Choice}},
	Year = {2013}}
	
	@misc{nsf-find,
	Howpublished = {http://www.nets-find.net/},
	Title = {{NSF NeTS FIND Initiative}}}
	
	@article{traid,
	Author = {Cheriton, D.~R. and Gritter, M.},
	Title = {{TRIAD: A New Next-Generation Internet Architecture}},
	Year = {2000}}
	
	@inproceedings{dona,
	Author = {Koponen, T. and Chawla, M. and Chun, B-G. and Ermolinskiy, A. and Kim, K.~H. and Shenker, S. and Stoica, I.},
	Booktitle = {ACM SIGCOMM Computer Communication Review},
	Number = {4},
	Organization = {ACM},
	Pages = {181--192},
	Title = {{A Data-Oriented (and Beyond) Network Architecture}},
	Volume = {37},
	Year = {2007}}
	
	@misc{ultrasurf,
	Howpublished = {\url{http://www.ultrareach.com}},
	Key = {ultrasurf},
	Title = {{Ultrasurf}}}
	
	@misc{tor-bridge,
	Author = {Dingledine, R. and Mathewson, N.},
	Howpublished = {\url{https://svn.torproject.org/svn/projects/design-paper/blocking.html}},
	Title = {{Design of a Blocking-Resistant Anonymity System}}}
	
	@inproceedings{McLachlanH09,
	Author = {J. McLachlan and N. Hopper},
	Booktitle = {WPES},
	Title = {{On the Risks of Serving Whenever You Surf: Vulnerabilities in Tor's Blocking Resistance Design}},
	Year = {2009}}
	
	@inproceedings{mahdian2010,
	Author = {Mahdian, M.},
	Booktitle = {{Fun with Algorithms}},
	Title = {{Fighting Censorship with Algorithms}},
	Year = {2010}}
	
	@inproceedings{McCoy2011,
	Author = {McCoy, D. and Morales, J.~A. and Levchenko, K.},
	Booktitle = {FC},
	Title = {{Proximax: A Measurement Based System for Proxies Dissemination}},
	Year = {2011}}
	
	@inproceedings{Sovran2008,
	Author = {Sovran, Y. and Libonati, A. and Li, J.},
	Booktitle = {IPTPS},
	Title = {{Pass it on: Social Networks Stymie Censors}},
	Year = {2008}}
	
	@inproceedings{rbridge,
	Author = {Wang, Q. and Lin, Zi and Borisov, N. and Hopper, N.},
	Booktitle = {{NDSS}},
	Title = {{rBridge: User Reputation based Tor Bridge Distribution with Privacy Preservation}},
	Year = {2013}}
	
	@inproceedings{telex,
	Author = {Wustrow, E. and Wolchok, S. and Goldberg, I. and Halderman, J.},
	Booktitle = {{USENIX Security}},
	Title = {{Telex: Anticensorship in the Network Infrastructure}},
	Year = {2011}}
	
	@inproceedings{cirripede,
	Author = {Houmansadr, A. and Nguyen, G. and Caesar, M. and Borisov, N.},
	Booktitle = {CCS},
	Title = {{Cirripede: Circumvention Infrastructure Using Router Redirection with Plausible Deniability}},
	Year = {2011}}
	
	@inproceedings{decoyrouting,
	Author = {Karlin, J. and Ellard, D. and Jackson, A. and Jones, C. and Lauer, G. and Mankins, D. and Strayer, W.},
	Booktitle = {{FOCI}},
	Title = {{Decoy Routing: Toward Unblockable Internet Communication}},
	Year = {2011}}
	
	@inproceedings{routing-around-decoys,
	Author = {M.~Schuchard and J.~Geddes and C.~Thompson and N.~Hopper},
	Booktitle = {{CCS}},
	Title = {{Routing Around Decoys}},
	Year = {2012}}
	
	@inproceedings{parrot,
	Author = {A. Houmansadr and C. Brubaker and V. Shmatikov},
	Booktitle = {IEEE S\&P},
	Title = {{The Parrot is Dead: Observing Unobservable Network Communications}},
	Year = {2013}}
	
	@misc{knock,
	Author = {T. Wilde},
	Howpublished = {\url{https://blog.torproject.org/blog/knock-knock-knockin-bridges-doors}},
	Title = {{Knock Knock Knockin' on Bridges' Doors}},
	Year = {2012}}
	
	@inproceedings{china-tor,
	Author = {Winter, P. and Lindskog, S.},
	Booktitle = {{FOCI}},
	Title = {{How the Great Firewall of China Is Blocking Tor}},
	Year = {2012}}
	
	@misc{discover-bridge,
	Howpublished = {\url{https://blog.torproject.org/blog/research-problems-ten-ways-discover-tor-bridges}},
	Key = {tenways},
	Title = {{Ten Ways to Discover Tor Bridges}}}
	
	@inproceedings{freewave,
	Author = {A.~Houmansadr and T.~Riedl and N.~Borisov and A.~Singer},
	Booktitle = {{NDSS}},
	Title = {{I Want My Voice to Be Heard: IP over Voice-over-IP for Unobservable Censorship Circumvention}},
	Year = 2013}
	
	@inproceedings{censorspoofer,
	Author = {Q. Wang and X. Gong and G. Nguyen and A. Houmansadr and N. Borisov},
	Booktitle = {CCS},
	Title = {{CensorSpoofer: Asymmetric Communication Using IP Spoofing for Censorship-Resistant Web Browsing}},
	Year = {2012}}
	
	@inproceedings{skypemorph,
	Author = {H. Moghaddam and B. Li and M. Derakhshani and I. Goldberg},
	Booktitle = {CCS},
	Title = {{SkypeMorph: Protocol Obfuscation for Tor Bridges}},
	Year = {2012}}
	
	@inproceedings{stegotorus,
	Author = {Weinberg, Z. and Wang, J. and Yegneswaran, V. and Briesemeister, L. and Cheung, S. and Wang, F. and Boneh, D.},
	Booktitle = {CCS},
	Title = {{StegoTorus: A Camouflage Proxy for the Tor Anonymity System}},
	Year = {2012}}
	
	@techreport{dust,
	Author = {{B.~Wiley}},
	Howpublished = {\url{http://blanu.net/ Dust.pdf}},
	Institution = {School of Information, University of Texas at Austin},
	Title = {{Dust: A Blocking-Resistant Internet Transport Protocol}},
	Year = {2011}}
	
	@inproceedings{FTE,
	Author = {K.~Dyer and S.~Coull and T.~Ristenpart and T.~Shrimpton},
	Booktitle = {CCS},
	Title = {{Protocol Misidentification Made Easy with Format-Transforming Encryption}},
	Year = {2013}}
	
	@inproceedings{fp,
	Author = {Fifield, D. and Hardison, N. and Ellithrope, J. and Stark, E. and Dingledine, R. and Boneh, D. and Porras, P.},
	Booktitle = {PETS},
	Title = {{Evading Censorship with Browser-Based Proxies}},
	Year = {2012}}
	
	@misc{obfsproxy,
	Howpublished = {\url{https://www.torproject.org/projects/obfsproxy.html.en}},
	Key = {obfsproxy},
	Publisher = {The Tor Project},
	Title = {{A Simple Obfuscating Proxy}}}
	
	@inproceedings{Tor-instead-of-IP,
	Author = {Liu, V. and Han, S. and Krishnamurthy, A. and Anderson, T.},
	Booktitle = {HotNets},
	Title = {{Tor instead of IP}},
	Year = {2011}}
	
	@misc{roger-slides,
	Howpublished = {\url{https://svn.torproject.org/svn/projects/presentations/slides-28c3.pdf}},
	Key = {torblocking},
	Title = {{How Governments Have Tried to Block Tor}}}
	
	@inproceedings{infranet,
	Author = {Feamster, N. and Balazinska, M. and Harfst, G. and Balakrishnan, H. and Karger, D.},
	Booktitle = {USENIX Security},
	Title = {{Infranet: Circumventing Web Censorship and Surveillance}},
	Year = {2002}}
	
	@inproceedings{collage,
	Author = {S.~Burnett and N.~Feamster and S.~Vempala},
	Booktitle = {USENIX Security},
	Title = {{Chipping Away at Censorship Firewalls with User-Generated Content}},
	Year = {2010}}
	
	@article{anonymizer,
	Author = {Boyan, J.},
	Journal = {Computer-Mediated Communication Magazine},
	Month = sep,
	Number = {9},
	Title = {{The Anonymizer: Protecting User Privacy on the Web}},
	Volume = {4},
	Year = {1997}}
	
	@article{schulze2009internet,
	Author = {Schulze, H. and Mochalski, K.},
	Journal = {IPOQUE Report},
	Pages = {351--362},
	Title = {Internet Study 2008/2009},
	Volume = {37},
	Year = {2009}}
	
	@inproceedings{cya-ccs13,
	Author = {J.~Geddes and M.~Schuchard and N.~Hopper},
	Booktitle = {{CCS}},
	Title = {{Cover Your ACKs: Pitfalls of Covert Channel Censorship Circumvention}},
	Year = {2013}}
	
	@inproceedings{andana,
	Author = {DiBenedetto, S. and Gasti, P. and Tsudik, G. and Uzun, E.},
	Booktitle = {{NDSS}},
	Title = {{ANDaNA: Anonymous Named Data Networking Application}},
	Year = {2012}}
	
	@inproceedings{darkly,
	Author = {Jana, S. and Narayanan, A. and Shmatikov, V.},
	Booktitle = {IEEE S\&P},
	Title = {{A Scanner Darkly: Protecting User Privacy From Perceptual Applications}},
	Year = {2013}}
	
	@inproceedings{NS08,
	Author = {A.~Narayanan and V.~Shmatikov},
	Booktitle = {IEEE S\&P},
	Title = {Robust de-anonymization of large sparse datasets},
	Year = {2008}}
	
	@inproceedings{NS09,
	Author = {A.~Narayanan and V.~Shmatikov},
	Booktitle = {IEEE S\&P},
	Title = {De-anonymizing social networks},
	Year = {2009}}
	
	@inproceedings{memento,
	Author = {Jana, S. and Shmatikov, V.},
	Booktitle = {IEEE S\&P},
	Title = {{Memento: Learning secrets from process footprints}},
	Year = {2012}}
	
	@misc{plugtor,
	Howpublished = {\url{https://www.torproject.org/docs/pluggable-transports.html.en}},
	Key = {PluggableTransports},
	Publisher = {The Tor Project},
	Title = {{Tor: Pluggable transports}}}
	
	@misc{psiphon,
	Author = {J.~Jia and P.~Smith},
	Howpublished = {\url{http://www.cdf.toronto.edu/~csc494h/reports/2004-fall/psiphon_ae.html}},
	Title = {{Psiphon: Analysis and Estimation}},
	Year = 2004}
	
	@misc{china-github,
	Howpublished = {\url{http://mobile.informationweek.com/80269/show/72e30386728f45f56b343ddfd0fdb119/}},
	Key = {github},
	Title = {{China's GitHub Censorship Dilemma}}}
	
	@inproceedings{txbox,
	Author = {Jana, S. and Porter, D. and Shmatikov, V.},
	Booktitle = {IEEE S\&P},
	Title = {{TxBox: Building Secure, Efficient Sandboxes with System Transactions}},
	Year = {2011}}
	
	@inproceedings{airavat,
	Author = {I. Roy and S. Setty and A. Kilzer and V. Shmatikov and E. Witchel},
	Booktitle = {NSDI},
	Title = {{Airavat: Security and Privacy for MapReduce}},
	Year = {2010}}
	
	@inproceedings{osdi12,
	Author = {A. Dunn and M. Lee and S. Jana and S. Kim and M. Silberstein and Y. Xu and V. Shmatikov and E. Witchel},
	Booktitle = {OSDI},
	Title = {{Eternal Sunshine of the Spotless Machine: Protecting Privacy with Ephemeral Channels}},
	Year = {2012}}
	
	@inproceedings{ymal,
	Author = {J. Calandrino and A. Kilzer and A. Narayanan and E. Felten and V. Shmatikov},
	Booktitle = {IEEE S\&P},
	Title = {{``You Might Also Like:'' Privacy Risks of Collaborative Filtering}},
	Year = {2011}}
	
	@inproceedings{srivastava11,
	Author = {V. Srivastava and M. Bond and K. McKinley and V. Shmatikov},
	Booktitle = {PLDI},
	Title = {{A Security Policy Oracle: Detecting Security Holes Using Multiple API Implementations}},
	Year = {2011}}
	
	@inproceedings{chen-oakland10,
	Author = {Chen, S. and Wang, R. and Wang, X. and Zhang, K.},
	Booktitle = {IEEE S\&P},
	Title = {{Side-Channel Leaks in Web Applications: A Reality Today, a Challenge Tomorrow}},
	Year = {2010}}
	
	@book{kerck,
	Author = {Kerckhoffs, A.},
	Publisher = {University Microfilms},
	Title = {{La cryptographie militaire}},
	Year = {1978}}
	
	@inproceedings{foci11,
	Author = {J. Karlin and D. Ellard and A.~Jackson and C.~ Jones and G. Lauer and D. Mankins and W.~T.~Strayer},
	Booktitle = {FOCI},
	Title = {{Decoy Routing: Toward Unblockable Internet Communication}},
	Year = 2011}
	
	@inproceedings{sun02,
	Author = {Sun, Q. and Simon, D.~R. and Wang, Y. and Russell, W. and Padmanabhan, V. and Qiu, L.},
	Booktitle = {IEEE S\&P},
	Title = {{Statistical Identification of Encrypted Web Browsing Traffic}},
	Year = {2002}}
	
	@inproceedings{danezis,
	Author = {Murdoch, S.~J. and Danezis, G.},
	Booktitle = {IEEE S\&P},
	Title = {{Low-Cost Traffic Analysis of Tor}},
	Year = {2005}}
	
	@inproceedings{pakicensorship,
	Author = {Z.~Nabi},
	Booktitle = {FOCI},
	Title = {The Anatomy of {Web} Censorship in {Pakistan}},
	Year = {2013}}
	
	@inproceedings{irancensorship,
	Author = {S.~Aryan and H.~Aryan and A.~Halderman},
	Booktitle = {FOCI},
	Title = {Internet Censorship in {Iran}: {A} First Look},
	Year = {2013}}
	
	@inproceedings{ford10efficient,
	Author = {Amittai Aviram and Shu-Chun Weng and Sen Hu and Bryan Ford},
	Booktitle = {\bibconf[9th]{OSDI}{USENIX Symposium on Operating Systems Design and Implementation}},
	Location = {Vancouver, BC, Canada},
	Month = oct,
	Title = {Efficient System-Enforced Deterministic Parallelism},
	Year = 2010}
	
	@inproceedings{ford10determinating,
	Author = {Amittai Aviram and Sen Hu and Bryan Ford and Ramakrishna Gummadi},
	Booktitle = {\bibconf{CCSW}{ACM Cloud Computing Security Workshop}},
	Location = {Chicago, IL},
	Month = oct,
	Title = {Determinating Timing Channels in Compute Clouds},
	Year = 2010}
	
	@inproceedings{ford12plugging,
	Author = {Bryan Ford},
	Booktitle = {\bibconf[4th]{HotCloud}{USENIX Workshop on Hot Topics in Cloud Computing}},
	Location = {Boston, MA},
	Month = jun,
	Title = {Plugging Side-Channel Leaks with Timing Information Flow Control},
	Year = 2012}
	
	@inproceedings{ford12icebergs,
	Author = {Bryan Ford},
	Booktitle = {\bibconf[4th]{HotCloud}{USENIX Workshop on Hot Topics in Cloud Computing}},
	Location = {Boston, MA},
	Month = jun,
	Title = {Icebergs in the Clouds: the {\em Other} Risks of Cloud Computing},
	Year = 2012}
	
	@misc{mullenize,
	Author = {Washington Post},
	Howpublished = {\url{http://apps.washingtonpost.com/g/page/world/gchq-report-on-mullenize-program-to-stain-anonymous-electronic-traffic/502/}},
	Month = {oct},
	Title = {{GCHQ} report on {`MULLENIZE'} program to `stain' anonymous electronic traffic},
	Year = {2013}}
	
	@inproceedings{shue13street,
	Author = {Craig A. Shue and Nathanael Paul and Curtis R. Taylor},
	Booktitle = {\bibbrev[7th]{WOOT}{USENIX Workshop on Offensive Technologies}},
	Month = aug,
	Title = {From an {IP} Address to a Street Address: Using Wireless Signals to Locate a Target},
	Year = 2013}
	
	@inproceedings{knockel11three,
	Author = {Jeffrey Knockel and Jedidiah R. Crandall and Jared Saia},
	Booktitle = {\bibbrev{FOCI}{USENIX Workshop on Free and Open Communications on the Internet}},
	Location = {San Francisco, CA},
	Month = aug,
	Year = 2011}
	
	@misc{rfc4960,
	Author = {R. {Stewart, ed.}},
	Month = sep,
	Note = {RFC 4960},
	Title = {Stream Control Transmission Protocol},
	Year = 2007}
	
	@inproceedings{ford07structured,
	Author = {Bryan Ford},
	Booktitle = {\bibbrev{SIGCOMM}{ACM SIGCOMM}},
	Location = {Kyoto, Japan},
	Month = aug,
	Title = {Structured Streams: a New Transport Abstraction},
	Year = {2007}}
	
	@misc{spdy,
	Author = {Google, Inc.},
	Note = {\url{http://www.chromium.org/spdy/spdy-whitepaper}},
	Title = {{SPDY}: An Experimental Protocol For a Faster {Web}}}
	
	@misc{quic,
	Author = {Jim Roskind},
	Month = jun,
	Note = {\url{http://blog.chromium.org/2013/06/experimenting-with-quic.html}},
	Title = {Experimenting with {QUIC}},
	Year = 2013}
	
	@misc{podjarny12not,
	Author = {G.~Podjarny},
	Month = jun,
	Note = {\url{http://www.guypo.com/technical/not-as-spdy-as-you-thought/}},
	Title = {{Not as SPDY as You Thought}},
	Year = 2012}
	
	@inproceedings{cor,
	Author = {Jones, N.~A. and Arye, M. and Cesareo, J. and Freedman, M.~J.},
	Booktitle = {FOCI},
	Title = {{Hiding Amongst the Clouds: A Proposal for Cloud-based Onion Routing}},
	Year = {2011}}
	
	@misc{torcloud,
	Howpublished = {\url{https://cloud.torproject.org/}},
	Key = {tor cloud},
	Title = {{The Tor Cloud Project}}}
	
	@inproceedings{scramblesuit,
	Author = {Philipp Winter and Tobias Pulls and Juergen Fuss},
	Booktitle = {WPES},
	Title = {{ScrambleSuit: A Polymorphic Network Protocol to Circumvent Censorship}},
	Year = 2013}
	
	@article{savage2000practical,
	Author = {Savage, S. and Wetherall, D. and Karlin, A. and Anderson, T.},
	Journal = {ACM SIGCOMM Computer Communication Review},
	Number = {4},
	Pages = {295--306},
	Publisher = {ACM},
	Title = {Practical network support for IP traceback},
	Volume = {30},
	Year = {2000}}
	
	@inproceedings{ooni,
	Author = {Filast, A. and Appelbaum, J.},
	Booktitle = {{FOCI}},
	Title = {{OONI : Open Observatory of Network Interference}},
	Year = {2012}}
	
	@misc{caida-rank,
	Howpublished = {\url{http://as-rank.caida.org/}},
	Key = {caida rank},
	Title = {{AS Rank: AS Ranking}}}
	
	@inproceedings{usersrouted-ccs13,
	Author = {A.~Johnson and C.~Wacek and R.~Jansen and M.~Sherr and P.~Syverson},
	Booktitle = {CCS},
	Title = {{Users Get Routed: Traffic Correlation on Tor by Realistic Adversaries}},
	Year = {2013}}
	
	@inproceedings{edman2009awareness,
	Author = {Edman, M. and Syverson, P.},
	Booktitle = {{CCS}},
	Title = {{AS-awareness in Tor path selection}},
	Year = {2009}}
	
	@inproceedings{DecoyCosts,
	Author = {A.~Houmansadr and E.~L.~Wong and V.~Shmatikov},
	Booktitle = {NDSS},
	Title = {{No Direction Home: The True Cost of Routing Around Decoys}},
	Year = {2014}}
	
	@article{cordon,
	Author = {Elahi, T. and Goldberg, I.},
	Journal = {University of Waterloo CACR},
	Title = {{CORDON--A Taxonomy of Internet Censorship Resistance Strategies}},
	Volume = {33},
	Year = {2012}}
	
	@inproceedings{privex,
	Author = {T.~Elahi and G.~Danezis and I.~Goldberg	},
	Booktitle = {{CCS}},
	Title = {{AS-awareness in Tor path selection}},
	Year = {2014}}
	
	@inproceedings{changeGuards,
	Author = {T.~Elahi and K.~Bauer and M.~AlSabah and R.~Dingledine and I.~Goldberg},
	Booktitle = {{WPES}},
	Title = {{ Changing of the Guards: Framework for Understanding and Improving Entry Guard Selection in Tor}},
	Year = {2012}}
	
	@article{RAINBOW:Journal,
	Author = {A.~Houmansadr and N.~Kiyavash and N.~Borisov},
	Journal = {IEEE/ACM Transactions on Networking},
	Title = {{Non-Blind Watermarking of Network Flows}},
	Year = 2014}
	
	@inproceedings{info-tod,
	Author = {A.~Houmansadr and S.~Gorantla and T.~Coleman and N.~Kiyavash and and N.~Borisov},
	Booktitle = {{CCS (poster session)}},
	Title = {{On the Channel Capacity of Network Flow Watermarking}},
	Year = {2009}}
	
	@inproceedings{johnson2014game,
	Author = {Johnson, B. and Laszka, A. and Grossklags, J. and Vasek, M. and Moore, T.},
	Booktitle = {Workshop on Bitcoin Research},
	Title = {{Game-theoretic Analysis of DDoS Attacks Against Bitcoin Mining Pools}},
	Year = {2014}}
	
	@incollection{laszka2013mitigation,
	Author = {Laszka, A. and Johnson, B. and Grossklags, J.},
	Booktitle = {Decision and Game Theory for Security},
	Pages = {175--191},
	Publisher = {Springer},
	Title = {{Mitigation of Targeted and Non-targeted Covert Attacks as a Timing Game}},
	Year = {2013}}
	
	@inproceedings{schottle2013game,
	Author = {Schottle, P. and Laszka, A. and Johnson, B. and Grossklags, J. and Bohme, R.},
	Booktitle = {EUSIPCO},
	Title = {{A Game-theoretic Analysis of Content-adaptive Steganography with Independent Embedding}},
	Year = {2013}}
	
	@inproceedings{CloudTransport,
	Author = {C.~Brubaker and A.~Houmansadr and V.~Shmatikov},
	Booktitle = {PETS},
	Title = {{CloudTransport: Using Cloud Storage for Censorship-Resistant Networking}},
	Year = {2014}}
	
	@inproceedings{sweet,
	Author = {W.~Zhou and A.~Houmansadr and M.~Caesar and N.~Borisov},
	Booktitle = {HotPETs},
	Title = {{SWEET: Serving the Web by Exploiting Email Tunnels}},
	Year = {2013}}
	
	@inproceedings{ahsan2002practical,
	Author = {Ahsan, K. and Kundur, D.},
	Booktitle = {Workshop on Multimedia Security},
	Title = {{Practical data hiding in TCP/IP}},
	Year = {2002}}
	
	@incollection{danezis2011covert,
	Author = {Danezis, G.},
	Booktitle = {Security Protocols XVI},
	Pages = {198--214},
	Publisher = {Springer},
	Title = {{Covert Communications Despite Traffic Data Retention}},
	Year = {2011}}
	
	@inproceedings{liu2009hide,
	Author = {Liu, Y. and Ghosal, D. and Armknecht, F. and Sadeghi, A.-R. and Schulz, S. and Katzenbeisser, S.},
	Booktitle = {ESORICS},
	Title = {{Hide and Seek in Time---Robust Covert Timing Channels}},
	Year = {2009}}
	
	@misc{image-watermark-fing,
	Author = {Jonathan Bailey},
	Howpublished = {\url{https://www.plagiarismtoday.com/2009/12/02/image-detection-watermarking-vs-fingerprinting/}},
	Title = {{Image Detection: Watermarking vs. Fingerprinting}},
	Year = {2009}}
	
	@inproceedings{Servetto98,
	Author = {S. D. Servetto and C. I. Podilchuk and K. Ramchandran},
	Booktitle = {Int. Conf. Image Processing},
	Title = {Capacity issues in digital image watermarking},
	Year = {1998}}
	
	@inproceedings{Chen01,
	Author = {B. Chen and G.W.Wornell},
	Booktitle = {IEEE Trans. Inform. Theory},
	Pages = {1423--1443},
	Title = {Quantization index modulation: A class of provably good methods for digital watermarking and information embedding},
	Year = {2001}}
	
	@inproceedings{Karakos00,
	Author = {D. Karakos and A. Papamarcou},
	Booktitle = {IEEE Int. Symp. Information Theory},
	Pages = {47},
	Title = {Relationship between quantization and distribution rates of digitally watermarked data},
	Year = {2000}}
	
	@inproceedings{Sullivan98,
	Author = {J. A. OSullivan and P. Moulin and J. M. Ettinger},
	Booktitle = {IEEE Int. Symp. Information Theory},
	Pages = {297},
	Title = {Information theoretic analysis of steganography},
	Year = {1998}}
	
	@inproceedings{Merhav00,
	Author = {N. Merhav},
	Booktitle = {IEEE Trans. Inform. Theory},
	Pages = {420--430},
	Title = {On random coding error exponents of watermarking systems},
	Year = {2000}}
	
	@inproceedings{Somekh01,
	Author = {A. Somekh-Baruch and N. Merhav},
	Booktitle = {IEEE Int. Symp. Information Theory},
	Pages = {7},
	Title = {On the error exponent and capacity games of private watermarking systems},
	Year = {2001}}
	
	@inproceedings{Steinberg01,
	Author = {Y. Steinberg and N. Merhav},
	Booktitle = {IEEE Trans. Inform. Theory},
	Pages = {1410--1422},
	Title = {Identification in the presence of side information with application to watermarking},
	Year = {2001}}
	
	@article{Moulin03,
	Author = {P. Moulin and J.A. O'Sullivan},
	Journal = {IEEE Trans. Info. Theory},
	Number = {3},
	Title = {Information-theoretic analysis of information hiding},
	Volume = 49,
	Year = 2003}
	
	@article{Gelfand80,
	Author = {S.I.~Gelfand and M.S.~Pinsker},
	Journal = {Problems of Control and Information Theory},
	Number = {1},
	Pages = {19-31},
	Title = {{Coding for channel with random parameters}},
	Url = {citeseer.ist.psu.edu/anantharam96bits.html},
	Volume = {9},
	Year = {1980},
	Bdsk-Url-1 = {citeseer.ist.psu.edu/anantharam96bits.html}}
	
	@book{Wolfowitz78,
	Author = {J. Wolfowitz},
	Edition = {3rd},
	Location = {New York},
	Publisher = {Springer-Verlag},
	Title = {Coding Theorems of Information Theory},
	Year = 1978}
	
	@article{caire99,
	Author = {G. Caire and S. Shamai},
	Journal = {IEEE Transactions on Information Theory},
	Number = {6},
	Pages = {2007--2019},
	Title = {On the Capacity of Some Channels with Channel State Information},
	Volume = {45},
	Year = {1999}}
	
	@inproceedings{wright2007language,
	Author = {Wright, Charles V and Ballard, Lucas and Monrose, Fabian and Masson, Gerald M},
	Booktitle = {USENIX Security},
	Title = {{Language identification of encrypted VoIP traffic: Alejandra y Roberto or Alice and Bob?}},
	Year = {2007}}
	
	@inproceedings{backes2010speaker,
	Author = {Backes, Michael and Doychev, Goran and D{\"u}rmuth, Markus and K{\"o}pf, Boris},
	Booktitle = {{European Symposium on Research in Computer Security (ESORICS)}},
	Pages = {508--523},
	Publisher = {Springer},
	Title = {{Speaker Recognition in Encrypted Voice Streams}},
	Year = {2010}}
	
	@phdthesis{lu2009traffic,
	Author = {Lu, Yuanchao},
	School = {Cleveland State University},
	Title = {{On Traffic Analysis Attacks to Encrypted VoIP Calls}},
	Year = {2009}}
	
	@inproceedings{wright2008spot,
	Author = {Wright, Charles V and Ballard, Lucas and Coull, Scott E and Monrose, Fabian and Masson, Gerald M},
	Booktitle = {IEEE Symposium on Security and Privacy},
	Pages = {35--49},
	Title = {Spot me if you can: Uncovering spoken phrases in encrypted VoIP conversations},
	Year = {2008}}
	
	@inproceedings{white2011phonotactic,
	Author = {White, Andrew M and Matthews, Austin R and Snow, Kevin Z and Monrose, Fabian},
	Booktitle = {IEEE Symposium on Security and Privacy},
	Pages = {3--18},
	Title = {Phonotactic reconstruction of encrypted VoIP conversations: Hookt on fon-iks},
	Year = {2011}}
	
	@inproceedings{fancy,
	Author = {Houmansadr, Amir and Borisov, Nikita},
	Booktitle = {Privacy Enhancing Technologies},
	Organization = {Springer},
	Pages = {205--224},
	Title = {The Need for Flow Fingerprints to Link Correlated Network Flows},
	Year = {2013}}
	
	@article{botmosaic,
	Author = {Amir Houmansadr and Nikita Borisov},
	Doi = {10.1016/j.jss.2012.11.005},
	Issn = {0164-1212},
	Journal = {Journal of Systems and Software},
	Keywords = {Network security},
	Number = {3},
	Pages = {707 - 715},
	Title = {BotMosaic: Collaborative network watermark for the detection of IRC-based botnets},
	Url = {http://www.sciencedirect.com/science/article/pii/S0164121212003068},
	Volume = {86},
	Year = {2013},
	Bdsk-Url-1 = {http://www.sciencedirect.com/science/article/pii/S0164121212003068},
	Bdsk-Url-2 = {http://dx.doi.org/10.1016/j.jss.2012.11.005}}
	
	@inproceedings{ramsbrock2008first,
	Author = {Ramsbrock, Daniel and Wang, Xinyuan and Jiang, Xuxian},
	Booktitle = {Recent Advances in Intrusion Detection},
	Organization = {Springer},
	Pages = {59--77},
	Title = {A first step towards live botmaster traceback},
	Year = {2008}}
	
	@inproceedings{potdar2005survey,
	Author = {Potdar, Vidyasagar M and Han, Song and Chang, Elizabeth},
	Booktitle = {Industrial Informatics, 2005. INDIN'05. 2005 3rd IEEE International Conference on},
	Organization = {IEEE},
	Pages = {709--716},
	Title = {A survey of digital image watermarking techniques},
	Year = {2005}}
	
	@book{cole2003hiding,
	Author = {Cole, Eric and Krutz, Ronald D},
	Publisher = {John Wiley \& Sons, Inc.},
	Title = {Hiding in plain sight: Steganography and the art of covert communication},
	Year = {2003}}
	
	@incollection{akaike1998information,
	Author = {Akaike, Hirotogu},
	Booktitle = {Selected Papers of Hirotugu Akaike},
	Pages = {199--213},
	Publisher = {Springer},
	Title = {Information theory and an extension of the maximum likelihood principle},
	Year = {1998}}
	
	@misc{central-command-hack,
	Author = {Everett Rosenfeld},
	Howpublished = {\url{http://www.cnbc.com/id/102330338}},
	Title = {{FBI investigating Central Command Twitter hack}},
	Year = {2015}}
	
	@misc{sony-psp-ddos,
	Howpublished = {\url{http://n4g.com/news/1644853/sony-and-microsoft-cant-do-much-ddos-attacks-explained}},
	Key = {sony},
	Month = {December},
	Title = {{Sony and Microsoft cant do much -- DDoS attacks explained}},
	Year = {2014}}
	
	@misc{sony-hack,
	Author = {David Bloom},
	Howpublished = {\url{http://goo.gl/MwR4A7}},
	Title = {{Online Game Networks Hacked, Sony Unit President Threatened}},
	Year = {2014}}
	
	@misc{home-depot,
	Author = {Dune Lawrence},
	Howpublished = {\url{http://www.businessweek.com/articles/2014-09-02/home-depots-credit-card-breach-looks-just-like-the-target-hack}},
	Title = {{Home Depot's Suspected Breach Looks Just Like the Target Hack}},
	Year = {2014}}
	
	@misc{target,
	Author = {Julio Ojeda-Zapata},
	Howpublished = {\url{http://www.mercurynews.com/business/ci_24765398/how-did-hackers-pull-off-target-data-heist}},
	Title = {{Target hack: How did they do it?}},
	Year = {2014}}
	
	
	@article{probabilitycourse,
	Author = {H. Pishro-Nik},
	note = {\url{http://www.probabilitycourse.com}},
	Title = {Introduction to probability, statistics, and random processes},
	Year = {2014}}
	
	
	
	@inproceedings{shokri2011quantifying,
	Author = {Shokri, Reza and Theodorakopoulos, George and Le Boudec, Jean-Yves and Hubaux, Jean-Pierre},
	Booktitle = {Security and Privacy (SP), 2011 IEEE Symposium on},
	Organization = {IEEE},
	Pages = {247--262},
	Title = {Quantifying location privacy},
	Year = {2011}}
	
	@inproceedings{hoh2007preserving,
	Author = {Hoh, Baik and Gruteser, Marco and Xiong, Hui and Alrabady, Ansaf},
	Booktitle = {Proceedings of the 14th ACM conference on Computer and communications security},
	Organization = {ACM},
	Pages = {161--171},
	Title = {Preserving privacy in gps traces via uncertainty-aware path cloaking},
	Year = {2007}}
	
	
	
	@article{kafsi2013entropy,
	Author = {Kafsi, Mohamed and Grossglauser, Matthias and Thiran, Patrick},
	Journal = {Information Theory, IEEE Transactions on},
	Number = {9},
	Pages = {5577--5583},
	Publisher = {IEEE},
	Title = {The entropy of conditional Markov trajectories},
	Volume = {59},
	Year = {2013}}
	
	@inproceedings{gruteser2003anonymous,
	Author = {Gruteser, Marco and Grunwald, Dirk},
	Booktitle = {Proceedings of the 1st international conference on Mobile systems, applications and services},
	Organization = {ACM},
	Pages = {31--42},
	Title = {Anonymous usage of location-based services through spatial and temporal cloaking},
	Year = {2003}}
	
	@inproceedings{husted2010mobile,
	Author = {Husted, Nathaniel and Myers, Steven},
	Booktitle = {Proceedings of the 17th ACM conference on Computer and communications security},
	Organization = {ACM},
	Pages = {85--96},
	Title = {Mobile location tracking in metro areas: malnets and others},
	Year = {2010}}
	
	@inproceedings{li2009tradeoff,
	Author = {Li, Tiancheng and Li, Ninghui},
	Booktitle = {Proceedings of the 15th ACM SIGKDD international conference on Knowledge discovery and data mining},
	Organization = {ACM},
	Pages = {517--526},
	Title = {On the tradeoff between privacy and utility in data publishing},
	Year = {2009}}
	
	@inproceedings{ma2009location,
	Author = {Ma, Zhendong and Kargl, Frank and Weber, Michael},
	Booktitle = {Sarnoff Symposium, 2009. SARNOFF'09. IEEE},
	Organization = {IEEE},
	Pages = {1--6},
	Title = {A location privacy metric for v2x communication systems},
	Year = {2009}}
	
	@inproceedings{shokri2012protecting,
	Author = {Shokri, Reza and Theodorakopoulos, George and Troncoso, Carmela and Hubaux, Jean-Pierre and Le Boudec, Jean-Yves},
	Booktitle = {Proceedings of the 2012 ACM conference on Computer and communications security},
	Organization = {ACM},
	Pages = {617--627},
	Title = {Protecting location privacy: optimal strategy against localization attacks},
	Year = {2012}}
	
	@inproceedings{freudiger2009non,
	Author = {Freudiger, Julien and Manshaei, Mohammad Hossein and Hubaux, Jean-Pierre and Parkes, David C},
	Booktitle = {Proceedings of the 16th ACM conference on Computer and communications security},
	Organization = {ACM},
	Pages = {324--337},
	Title = {On non-cooperative location privacy: a game-theoretic analysis},
	Year = {2009}}
	
	@incollection{humbert2010tracking,
	Author = {Humbert, Mathias and Manshaei, Mohammad Hossein and Freudiger, Julien and Hubaux, Jean-Pierre},
	Booktitle = {Decision and Game Theory for Security},
	Pages = {38--57},
	Publisher = {Springer},
	Title = {Tracking games in mobile networks},
	Year = {2010}}
	
	@article{manshaei2013game,
	Author = {Manshaei, Mohammad Hossein and Zhu, Quanyan and Alpcan, Tansu and Bac{\c{s}}ar, Tamer and Hubaux, Jean-Pierre},
	Journal = {ACM Computing Surveys (CSUR)},
	Number = {3},
	Pages = {25},
	Publisher = {ACM},
	Title = {Game theory meets network security and privacy},
	Volume = {45},
	Year = {2013}}
	
	@article{palamidessi2006probabilistic,
	Author = {Palamidessi, Catuscia},
	Journal = {Electronic Notes in Theoretical Computer Science},
	Pages = {33--42},
	Publisher = {Elsevier},
	Title = {Probabilistic and nondeterministic aspects of anonymity},
	Volume = {155},
	Year = {2006}}
	
	@inproceedings{mokbel2006new,
	Author = {Mokbel, Mohamed F and Chow, Chi-Yin and Aref, Walid G},
	Booktitle = {Proceedings of the 32nd international conference on Very large data bases},
	Organization = {VLDB Endowment},
	Pages = {763--774},
	Title = {The new Casper: query processing for location services without compromising privacy},
	Year = {2006}}
	
	@article{kalnis2007preventing,
	Author = {Kalnis, Panos and Ghinita, Gabriel and Mouratidis, Kyriakos and Papadias, Dimitris},
	Journal = {Knowledge and Data Engineering, IEEE Transactions on},
	Number = {12},
	Pages = {1719--1733},
	Publisher = {IEEE},
	Title = {Preventing location-based identity inference in anonymous spatial queries},
	Volume = {19},
	Year = {2007}}
	
	@article{freudiger2007mix,
	title={Mix-zones for location privacy in vehicular networks},
	author={Freudiger, Julien and Raya, Maxim and F{\'e}legyh{\'a}zi, M{\'a}rk and Papadimitratos, Panos and Hubaux, Jean-Pierre},
	year={2007}
	}
	@article{sweeney2002k,
	Author = {Sweeney, Latanya},
	Journal = {International Journal of Uncertainty, Fuzziness and Knowledge-Based Systems},
	Number = {05},
	Pages = {557--570},
	Publisher = {World Scientific},
	Title = {k-anonymity: A model for protecting privacy},
	Volume = {10},
	Year = {2002}}
	
	@article{sweeney2002achieving,
	Author = {Sweeney, Latanya},
	Journal = {International Journal of Uncertainty, Fuzziness and Knowledge-Based Systems},
	Number = {05},
	Pages = {571--588},
	Publisher = {World Scientific},
	Title = {Achieving k-anonymity privacy protection using generalization and suppression},
	Volume = {10},
	Year = {2002}}
	
	@inproceedings{niu2014achieving,
	Author = {Niu, Ben and Li, Qinghua and Zhu, Xiaoyan and Cao, Guohong and Li, Hui},
	Booktitle = {INFOCOM, 2014 Proceedings IEEE},
	Organization = {IEEE},
	Pages = {754--762},
	Title = {Achieving k-anonymity in privacy-aware location-based services},
	Year = {2014}}
	
	@inproceedings{liu2013game,
	Author = {Liu, Xinxin and Liu, Kaikai and Guo, Linke and Li, Xiaolin and Fang, Yuguang},
	Booktitle = {INFOCOM, 2013 Proceedings IEEE},
	Organization = {IEEE},
	Pages = {2985--2993},
	Title = {A game-theoretic approach for achieving k-anonymity in location based services},
	Year = {2013}}
	
	@inproceedings{kido2005protection,
	Author = {Kido, Hidetoshi and Yanagisawa, Yutaka and Satoh, Tetsuji},
	Booktitle = {Data Engineering Workshops, 2005. 21st International Conference on},
	Organization = {IEEE},
	Pages = {1248--1248},
	Title = {Protection of location privacy using dummies for location-based services},
	Year = {2005}}
	
	@inproceedings{gedik2005location,
	Author = {Gedik, Bu{\u{g}}ra and Liu, Ling},
	Booktitle = {Distributed Computing Systems, 2005. ICDCS 2005. Proceedings. 25th IEEE International Conference on},
	Organization = {IEEE},
	Pages = {620--629},
	Title = {Location privacy in mobile systems: A personalized anonymization model},
	Year = {2005}}
	
	@inproceedings{bordenabe2014optimal,
	Author = {Bordenabe, Nicol{\'a}s E and Chatzikokolakis, Konstantinos and Palamidessi, Catuscia},
	Booktitle = {Proceedings of the 2014 ACM SIGSAC Conference on Computer and Communications Security},
	Organization = {ACM},
	Pages = {251--262},
	Title = {Optimal geo-indistinguishable mechanisms for location privacy},
	Year = {2014}}
	
	@incollection{duckham2005formal,
	Author = {Duckham, Matt and Kulik, Lars},
	Booktitle = {Pervasive computing},
	Pages = {152--170},
	Publisher = {Springer},
	Title = {A formal model of obfuscation and negotiation for location privacy},
	Year = {2005}}
	
	@inproceedings{kido2005anonymous,
	Author = {Kido, Hidetoshi and Yanagisawa, Yutaka and Satoh, Tetsuji},
	Booktitle = {Pervasive Services, 2005. ICPS'05. Proceedings. International Conference on},
	Organization = {IEEE},
	Pages = {88--97},
	Title = {An anonymous communication technique using dummies for location-based services},
	Year = {2005}}
	
	@incollection{duckham2006spatiotemporal,
	Author = {Duckham, Matt and Kulik, Lars and Birtley, Athol},
	Booktitle = {Geographic Information Science},
	Pages = {47--64},
	Publisher = {Springer},
	Title = {A spatiotemporal model of strategies and counter strategies for location privacy protection},
	Year = {2006}}
	
	@inproceedings{shankar2009privately,
	Author = {Shankar, Pravin and Ganapathy, Vinod and Iftode, Liviu},
	Booktitle = {Proceedings of the 11th international conference on Ubiquitous computing},
	Organization = {ACM},
	Pages = {31--40},
	Title = {Privately querying location-based services with SybilQuery},
	Year = {2009}}
	
	@inproceedings{chow2009faking,
	Author = {Chow, Richard and Golle, Philippe},
	Booktitle = {Proceedings of the 8th ACM workshop on Privacy in the electronic society},
	Organization = {ACM},
	Pages = {105--108},
	Title = {Faking contextual data for fun, profit, and privacy},
	Year = {2009}}
	
	@incollection{xue2009location,
	Author = {Xue, Mingqiang and Kalnis, Panos and Pung, Hung Keng},
	Booktitle = {Location and Context Awareness},
	Pages = {70--87},
	Publisher = {Springer},
	Title = {Location diversity: Enhanced privacy protection in location based services},
	Year = {2009}}
	
	@article{wernke2014classification,
	Author = {Wernke, Marius and Skvortsov, Pavel and D{\"u}rr, Frank and Rothermel, Kurt},
	Journal = {Personal and Ubiquitous Computing},
	Number = {1},
	Pages = {163--175},
	Publisher = {Springer-Verlag},
	Title = {A classification of location privacy attacks and approaches},
	Volume = {18},
	Year = {2014}}
	
	@misc{cai2015cloaking,
	Author = {Cai, Y. and Xu, G.},
	Month = jan # {~1},
	Note = {US Patent App. 14/472,462},
	Publisher = {Google Patents},
	Title = {Cloaking with footprints to provide location privacy protection in location-based services},
	Url = {https://www.google.com/patents/US20150007341},
	Year = {2015},
	Bdsk-Url-1 = {https://www.google.com/patents/US20150007341}}
	
	@article{gedik2008protecting,
	Author = {Gedik, Bu{\u{g}}ra and Liu, Ling},
	Journal = {Mobile Computing, IEEE Transactions on},
	Number = {1},
	Pages = {1--18},
	Publisher = {IEEE},
	Title = {Protecting location privacy with personalized k-anonymity: Architecture and algorithms},
	Volume = {7},
	Year = {2008}}
	
	@article{kalnis2006preserving,
	Author = {Kalnis, Panos and Ghinita, Gabriel and Mouratidis, Kyriakos and Papadias, Dimitris},
	Publisher = {TRB6/06},
	Title = {Preserving anonymity in location based services},
	Year = {2006}}
	
	@inproceedings{hoh2005protecting,
	Author = {Hoh, Baik and Gruteser, Marco},
	Booktitle = {Security and Privacy for Emerging Areas in Communications Networks, 2005. SecureComm 2005. First International Conference on},
	Organization = {IEEE},
	Pages = {194--205},
	Title = {Protecting location privacy through path confusion},
	Year = {2005}}
	
	@article{terrovitis2011privacy,
	Author = {Terrovitis, Manolis},
	Journal = {ACM SIGKDD Explorations Newsletter},
	Number = {1},
	Pages = {6--18},
	Publisher = {ACM},
	Title = {Privacy preservation in the dissemination of location data},
	Volume = {13},
	Year = {2011}}
	
	@article{shin2012privacy,
	Author = {Shin, Kang G and Ju, Xiaoen and Chen, Zhigang and Hu, Xin},
	Journal = {Wireless Communications, IEEE},
	Number = {1},
	Pages = {30--39},
	Publisher = {IEEE},
	Title = {Privacy protection for users of location-based services},
	Volume = {19},
	Year = {2012}}
	
	@article{khoshgozaran2011location,
	Author = {Khoshgozaran, Ali and Shahabi, Cyrus and Shirani-Mehr, Houtan},
	Journal = {Knowledge and Information Systems},
	Number = {3},
	Pages = {435--465},
	Publisher = {Springer},
	Title = {Location privacy: going beyond K-anonymity, cloaking and anonymizers},
	Volume = {26},
	Year = {2011}}
	
	@incollection{chatzikokolakis2015geo,
	Author = {Chatzikokolakis, Konstantinos and Palamidessi, Catuscia and Stronati, Marco},
	Booktitle = {Distributed Computing and Internet Technology},
	Pages = {49--72},
	Publisher = {Springer},
	Title = {Geo-indistinguishability: A Principled Approach to Location Privacy},
	Year = {2015}}
	
	@inproceedings{ngo2015location,
	Author = {Ngo, Hoa and Kim, Jong},
	Booktitle = {Computer Security Foundations Symposium (CSF), 2015 IEEE 28th},
	Organization = {IEEE},
	Pages = {63--74},
	Title = {Location Privacy via Differential Private Perturbation of Cloaking Area},
	Year = {2015}}
	
	@inproceedings{palanisamy2011mobimix,
	Author = {Palanisamy, Balaji and Liu, Ling},
	Booktitle = {Data Engineering (ICDE), 2011 IEEE 27th International Conference on},
	Organization = {IEEE},
	Pages = {494--505},
	Title = {Mobimix: Protecting location privacy with mix-zones over road networks},
	Year = {2011}}
	
	@inproceedings{um2010advanced,
	Author = {Um, Jung-Ho and Kim, Hee-Dae and Chang, Jae-Woo},
	Booktitle = {Social Computing (SocialCom), 2010 IEEE Second International Conference on},
	Organization = {IEEE},
	Pages = {1093--1098},
	Title = {An advanced cloaking algorithm using Hilbert curves for anonymous location based service},
	Year = {2010}}
	
	@inproceedings{bamba2008supporting,
	Author = {Bamba, Bhuvan and Liu, Ling and Pesti, Peter and Wang, Ting},
	Booktitle = {Proceedings of the 17th international conference on World Wide Web},
	Organization = {ACM},
	Pages = {237--246},
	Title = {Supporting anonymous location queries in mobile environments with privacygrid},
	Year = {2008}}
	
	@inproceedings{zhangwei2010distributed,
	Author = {Zhangwei, Huang and Mingjun, Xin},
	Booktitle = {Networks Security Wireless Communications and Trusted Computing (NSWCTC), 2010 Second International Conference on},
	Organization = {IEEE},
	Pages = {468--471},
	Title = {A distributed spatial cloaking protocol for location privacy},
	Volume = {2},
	Year = {2010}}
	
	@article{chow2011spatial,
	Author = {Chow, Chi-Yin and Mokbel, Mohamed F and Liu, Xuan},
	Journal = {GeoInformatica},
	Number = {2},
	Pages = {351--380},
	Publisher = {Springer},
	Title = {Spatial cloaking for anonymous location-based services in mobile peer-to-peer environments},
	Volume = {15},
	Year = {2011}}
	
	@inproceedings{lu2008pad,
	Author = {Lu, Hua and Jensen, Christian S and Yiu, Man Lung},
	Booktitle = {Proceedings of the Seventh ACM International Workshop on Data Engineering for Wireless and Mobile Access},
	Organization = {ACM},
	Pages = {16--23},
	Title = {Pad: privacy-area aware, dummy-based location privacy in mobile services},
	Year = {2008}}
	
	@incollection{khoshgozaran2007blind,
	Author = {Khoshgozaran, Ali and Shahabi, Cyrus},
	Booktitle = {Advances in Spatial and Temporal Databases},
	Pages = {239--257},
	Publisher = {Springer},
	Title = {Blind evaluation of nearest neighbor queries using space transformation to preserve location privacy},
	Year = {2007}}
	
	@inproceedings{ghinita2008private,
	Author = {Ghinita, Gabriel and Kalnis, Panos and Khoshgozaran, Ali and Shahabi, Cyrus and Tan, Kian-Lee},
	Booktitle = {Proceedings of the 2008 ACM SIGMOD international conference on Management of data},
	Organization = {ACM},
	Pages = {121--132},
	Title = {Private queries in location based services: anonymizers are not necessary},
	Year = {2008}}
	
	@article{paulet2014privacy,
	Author = {Paulet, Russell and Kaosar, Md Golam and Yi, Xun and Bertino, Elisa},
	Journal = {Knowledge and Data Engineering, IEEE Transactions on},
	Number = {5},
	Pages = {1200--1210},
	Publisher = {IEEE},
	Title = {Privacy-preserving and content-protecting location based queries},
	Volume = {26},
	Year = {2014}}
	
	@article{nguyen2013differential,
	Author = {Nguyen, Hiep H and Kim, Jong and Kim, Yoonho},
	Journal = {Journal of Computing Science and Engineering},
	Number = {3},
	Pages = {177--186},
	Title = {Differential privacy in practice},
	Volume = {7},
	Year = {2013}}
	
	@inproceedings{lee2012differential,
	Author = {Lee, Jaewoo and Clifton, Chris},
	Booktitle = {Proceedings of the 18th ACM SIGKDD international conference on Knowledge discovery and data mining},
	Organization = {ACM},
	Pages = {1041--1049},
	Title = {Differential identifiability},
	Year = {2012}}
	
	@inproceedings{andres2013geo,
	Author = {Andr{\'e}s, Miguel E and Bordenabe, Nicol{\'a}s E and Chatzikokolakis, Konstantinos and Palamidessi, Catuscia},
	Booktitle = {Proceedings of the 2013 ACM SIGSAC conference on Computer \& communications security},
	Organization = {ACM},
	Pages = {901--914},
	Title = {Geo-indistinguishability: Differential privacy for location-based systems},
	Year = {2013}}
	
	@inproceedings{machanavajjhala2008privacy,
	Author = {Machanavajjhala, Ashwin and Kifer, Daniel and Abowd, John and Gehrke, Johannes and Vilhuber, Lars},
	Booktitle = {Data Engineering, 2008. ICDE 2008. IEEE 24th International Conference on},
	Organization = {IEEE},
	Pages = {277--286},
	Title = {Privacy: Theory meets practice on the map},
	Year = {2008}}
	
	@article{dewri2013local,
	Author = {Dewri, Rinku},
	Journal = {Mobile Computing, IEEE Transactions on},
	Number = {12},
	Pages = {2360--2372},
	Publisher = {IEEE},
	Title = {Local differential perturbations: Location privacy under approximate knowledge attackers},
	Volume = {12},
	Year = {2013}}
	
	@inproceedings{chatzikokolakis2013broadening,
	Author = {Chatzikokolakis, Konstantinos and Andr{\'e}s, Miguel E and Bordenabe, Nicol{\'a}s Emilio and Palamidessi, Catuscia},
	Booktitle = {Privacy Enhancing Technologies},
	Organization = {Springer},
	Pages = {82--102},
	Title = {Broadening the Scope of Differential Privacy Using Metrics.},
	Year = {2013}}
	
	@inproceedings{zhong2009distributed,
	Author = {Zhong, Ge and Hengartner, Urs},
	Booktitle = {Pervasive Computing and Communications, 2009. PerCom 2009. IEEE International Conference on},
	Organization = {IEEE},
	Pages = {1--10},
	Title = {A distributed k-anonymity protocol for location privacy},
	Year = {2009}}
	
	@inproceedings{ho2011differential,
	Author = {Ho, Shen-Shyang and Ruan, Shuhua},
	Booktitle = {Proceedings of the 4th ACM SIGSPATIAL International Workshop on Security and Privacy in GIS and LBS},
	Organization = {ACM},
	Pages = {17--24},
	Title = {Differential privacy for location pattern mining},
	Year = {2011}}
	
	@inproceedings{cheng2006preserving,
	Author = {Cheng, Reynold and Zhang, Yu and Bertino, Elisa and Prabhakar, Sunil},
	Booktitle = {Privacy Enhancing Technologies},
	Organization = {Springer},
	Pages = {393--412},
	Title = {Preserving user location privacy in mobile data management infrastructures},
	Year = {2006}}
	
	@article{beresford2003location,
	Author = {Beresford, Alastair R and Stajano, Frank},
	Journal = {IEEE Pervasive computing},
	Number = {1},
	Pages = {46--55},
	Publisher = {IEEE},
	Title = {Location privacy in pervasive computing},
	Year = {2003}}
	
	@inproceedings{freudiger2009optimal,
	Author = {Freudiger, Julien and Shokri, Reza and Hubaux, Jean-Pierre},
	Booktitle = {Privacy enhancing technologies},
	Organization = {Springer},
	Pages = {216--234},
	Title = {On the optimal placement of mix zones},
	Year = {2009}}
	
	@article{krumm2009survey,
	Author = {Krumm, John},
	Journal = {Personal and Ubiquitous Computing},
	Number = {6},
	Pages = {391--399},
	Publisher = {Springer},
	Title = {A survey of computational location privacy},
	Volume = {13},
	Year = {2009}}
	
	@article{Rakhshan2016letter,
	Author = {Rakhshan, Ali and Pishro-Nik, Hossein},
	Journal = {IEEE Wireless Communications Letter},
	Publisher = {IEEE},
	Title = {Interference Models for Vehicular Ad Hoc Networks},
	Year = {2016, submitted}}
	
	@article{Rakhshan2015Journal,
	Author = {Rakhshan, Ali and Pishro-Nik, Hossein},
	Journal = {IEEE Transactions on Wireless Communications},
	Publisher = {IEEE},
	Title = {Improving Safety on Highways by Customizing Vehicular Ad Hoc Networks},
	Year = {to appear, 2017}}
	
	@inproceedings{Rakhshan2015Cogsima,
	Author = {Rakhshan, Ali and Pishro-Nik, Hossein},
	Booktitle = {IEEE International Multi-Disciplinary Conference on Cognitive Methods in Situation Awareness and Decision Support},
	Organization = {IEEE},
	Title = {A New Approach to Customization of Accident Warning Systems to Individual Drivers},
	Year = {2015}}
	
	@inproceedings{Rakhshan2015CISS,
	Author = {Rakhshan, Ali and Pishro-Nik, Hossein and Nekoui, Mohammad},
	Booktitle = {Conference on Information Sciences and Systems},
	Organization = {IEEE},
	Pages = {1--6},
	Title = {Driver-based adaptation of Vehicular Ad Hoc Networks for design of active safety systems},
	Year = {2015}}
	
	@inproceedings{Rakhshan2014IV,
	Author = {Rakhshan, Ali and Pishro-Nik, Hossein and Ray, Evan},
	Booktitle = {Intelligent Vehicles Symposium},
	Organization = {IEEE},
	Pages = {1181--1186},
	Title = {Real-time estimation of the distribution of brake response times for an individual driver using Vehicular Ad Hoc Network.},
	Year = {2014}}
	
	@inproceedings{Rakhshan2013Globecom,
	Author = {Rakhshan, Ali and Pishro-Nik, Hossein and Fisher, Donald and Nekoui, Mohammad},
	Booktitle = {IEEE Global Communications Conference},
	Organization = {IEEE},
	Pages = {1333--1337},
	Title = {Tuning collision warning algorithms to individual drivers for design of active safety systems.},
	Year = {2013}}
	
	@article{Nekoui2012Journal,
	Author = {Nekoui, Mohammad and Pishro-Nik, Hossein},
	Journal = {IEEE Transactions on Wireless Communications},
	Number = {8},
	Pages = {2895--2905},
	Publisher = {IEEE},
	Title = {Throughput Scaling laws for Vehicular Ad Hoc Networks},
	Volume = {11},
	Year = {2012}}
	
	
	
	
	
	
	
	
	
	@article{Nekoui2011Journal,
	Author = {Nekoui, Mohammad and Pishro-Nik, Hossein and Ni, Daiheng},
	Journal = {International Journal of Vehicular Technology},
	Pages = {1--11},
	Publisher = {Hindawi Publishing Corporation},
	Title = {Analytic Design of Active Safety Systems for Vehicular Ad hoc Networks},
	Volume = {2011},
	Year = {2011}}
	
	
	
	
	
	
	@article{shokri2014optimal,
	title={Optimal user-centric data obfuscation},
	author={Shokri, Reza},
	journal={arXiv preprint arXiv:1402.3426},
	year={2014}
	}
	@article{chatzikokolakis2015location,
	title={Location privacy via geo-indistinguishability},
	author={Chatzikokolakis, Konstantinos and Palamidessi, Catuscia and Stronati, Marco},
	journal={ACM SIGLOG News},
	volume={2},
	number={3},
	pages={46--69},
	year={2015},
	publisher={ACM}
	
	}
	@inproceedings{shokri2011quantifying2,
	title={Quantifying location privacy: the case of sporadic location exposure},
	author={Shokri, Reza and Theodorakopoulos, George and Danezis, George and Hubaux, Jean-Pierre and Le Boudec, Jean-Yves},
	booktitle={Privacy Enhancing Technologies},
	pages={57--76},
	year={2011},
	organization={Springer}
	}
	
	
	@inproceedings{Mont1603:Defining,
	AUTHOR="Zarrin Montazeri and Amir Houmansadr and Hossein Pishro-Nik",
	TITLE="Defining Perfect Location Privacy Using Anonymization",
	BOOKTITLE="2016 Annual Conference on Information Science and Systems (CISS) (CISS
	2016)",
	ADDRESS="Princeton, USA",
	DAYS=16,
	MONTH=mar,
	YEAR=2016,
	KEYWORDS="Information Theoretic Privacy; location-based services; Location Privacy;
	Information Theory",
	ABSTRACT="The popularity of mobile devices and location-based services (LBS) have
	created great concerns regarding the location privacy of users of such
	devices and services. Anonymization is a common technique that is often
	being used to protect the location privacy of LBS users. In this paper, we
	provide a general information theoretic definition for location privacy. In
	particular, we define perfect location privacy. We show that under certain
	conditions, perfect privacy is achieved if the pseudonyms of users is
	changed after o(N^(2/r?1)) observations by the adversary, where N is the
	number of users and r is the number of sub-regions or locations.
	"
	}
	@article{our-isita-location,
	Author = {Zarrin Montazeri and Amir Houmansadr and Hossein Pishro-Nik},
	Journal = {IEEE International Symposium on Information Theory and Its Applications (ISITA)},
	Title = {Achieving Perfect Location Privacy in Markov Models Using Anonymization},
	Year = {2016}
	}
	@article{our-TIFS,
	Author = {Zarrin Montazeri and Hossein Pishro-Nik and Amir Houmansadr},
	Journal = {IEEE Transactions on Information Forensics and Security, under revison},
	Title = {Perfect Location Privacy Using Anonymization in Mobile Networks},
	Year = {2016},
	note={Available on arxiv.org}
	}
	
	
	
	@techreport{sampigethaya2005caravan,
	title={CARAVAN: Providing location privacy for VANET},
	author={Sampigethaya, Krishna and Huang, Leping and Li, Mingyan and Poovendran, Radha and Matsuura, Kanta and Sezaki, Kaoru},
	year={2005},
	institution={DTIC Document}
	}
	@incollection{buttyan2007effectiveness,
	title={On the effectiveness of changing pseudonyms to provide location privacy in VANETs},
	author={Butty{\'a}n, Levente and Holczer, Tam{\'a}s and Vajda, Istv{\'a}n},
	booktitle={Security and Privacy in Ad-hoc and Sensor Networks},
	pages={129--141},
	year={2007},
	publisher={Springer}
	}
	@article{sampigethaya2007amoeba,
	title={AMOEBA: Robust location privacy scheme for VANET},
	author={Sampigethaya, Krishna and Li, Mingyan and Huang, Leping and Poovendran, Radha},
	journal={Selected Areas in communications, IEEE Journal on},
	volume={25},
	number={8},
	pages={1569--1589},
	year={2007},
	publisher={IEEE}
	}
	
	@article{lu2012pseudonym,
	title={Pseudonym changing at social spots: An effective strategy for location privacy in vanets},
	author={Lu, Rongxing and Li, Xiaodong and Luan, Tom H and Liang, Xiaohui and Shen, Xuemin},
	journal={Vehicular Technology, IEEE Transactions on},
	volume={61},
	number={1},
	pages={86--96},
	year={2012},
	publisher={IEEE}
	}
	@inproceedings{lu2010sacrificing,
	title={Sacrificing the plum tree for the peach tree: A socialspot tactic for protecting receiver-location privacy in VANET},
	author={Lu, Rongxing and Lin, Xiaodong and Liang, Xiaohui and Shen, Xuemin},
	booktitle={Global Telecommunications Conference (GLOBECOM 2010), 2010 IEEE},
	pages={1--5},
	year={2010},
	organization={IEEE}
	}
	@inproceedings{lin2011stap,
	title={STAP: A social-tier-assisted packet forwarding protocol for achieving receiver-location privacy preservation in VANETs},
	author={Lin, Xiaodong and Lu, Rongxing and Liang, Xiaohui and Shen, Xuemin Sherman},
	booktitle={INFOCOM, 2011 Proceedings IEEE},
	pages={2147--2155},
	year={2011},
	organization={IEEE}
	}
	@inproceedings{gerlach2007privacy,
	title={Privacy in VANETs using changing pseudonyms-ideal and real},
	author={Gerlach, Matthias and Guttler, Felix},
	booktitle={Vehicular Technology Conference, 2007. VTC2007-Spring. IEEE 65th},
	pages={2521--2525},
	year={2007},
	organization={IEEE}
	}
	@inproceedings{el2002security,
	title={Security issues in a future vehicular network},
	author={El Zarki, Magda and Mehrotra, Sharad and Tsudik, Gene and Venkatasubramanian, Nalini},
	booktitle={European Wireless},
	volume={2},
	year={2002}
	}
	
	@article{hubaux2004security,
	title={The security and privacy of smart vehicles},
	author={Hubaux, Jean-Pierre and Capkun, Srdjan and Luo, Jun},
	journal={IEEE Security \& Privacy Magazine},
	volume={2},
	number={LCA-ARTICLE-2004-007},
	pages={49--55},
	year={2004}
	}
	
	
	
	@inproceedings{duri2002framework,
	title={Framework for security and privacy in automotive telematics},
	author={Duri, Sastry and Gruteser, Marco and Liu, Xuan and Moskowitz, Paul and Perez, Ronald and Singh, Moninder and Tang, Jung-Mu},
	booktitle={Proceedings of the 2nd international workshop on Mobile commerce},
	pages={25--32},
	year={2002},
	organization={ACM}
	}
	@misc{NS-3,
	Howpublished = {\url{https://www.nsnam.org/}}},
}
@misc{testbed,
	Howpublished = {\url{http://www.its.dot.gov/testbed/PDF/SE-MI-Resource-Guide-9-3-1.pdf}}},
@misc{NGSIM,
	Howpublished = {\url{http://ops.fhwa.dot.gov/trafficanalysistools/ngsim.htm}},
}

@misc{National-a2013,
	Author = {National Highway Traffic Safety Administration},
	Howpublished = {\url{http://ops.fhwa.dot.gov/trafficanalysistools/ngsim.htm}},
	Title = {2013 Motor Vehicle Crashes: Overview. Traffic Safety Factors},
	Year = {2013}
}

@inproceedings{karnadi2007rapid,
	title={Rapid generation of realistic mobility models for VANET},
	author={Karnadi, Feliz Kristianto and Mo, Zhi Hai and Lan, Kun-chan},
	booktitle={Wireless Communications and Networking Conference, 2007. WCNC 2007. IEEE},
	pages={2506--2511},
	year={2007},
	organization={IEEE}
}
@inproceedings{saha2004modeling,
	title={Modeling mobility for vehicular ad-hoc networks},
	author={Saha, Amit Kumar and Johnson, David B},
	booktitle={Proceedings of the 1st ACM international workshop on Vehicular ad hoc networks},
	pages={91--92},
	year={2004},
	organization={ACM}
}
@inproceedings{lee2006modeling,
	title={Modeling steady-state and transient behaviors of user mobility: formulation, analysis, and application},
	author={Lee, Jong-Kwon and Hou, Jennifer C},
	booktitle={Proceedings of the 7th ACM international symposium on Mobile ad hoc networking and computing},
	pages={85--96},
	year={2006},
	organization={ACM}
}
@inproceedings{yoon2006building,
	title={Building realistic mobility models from coarse-grained traces},
	author={Yoon, Jungkeun and Noble, Brian D and Liu, Mingyan and Kim, Minkyong},
	booktitle={Proceedings of the 4th international conference on Mobile systems, applications and services},
	pages={177--190},
	year={2006},
	organization={ACM}
}

@inproceedings{choffnes2005integrated,
	title={An integrated mobility and traffic model for vehicular wireless networks},
	author={Choffnes, David R and Bustamante, Fabi{\'a}n E},
	booktitle={Proceedings of the 2nd ACM international workshop on Vehicular ad hoc networks},
	pages={69--78},
	year={2005},
	organization={ACM}
}

@inproceedings{Qian2008Globecom,
	title={CA Secure VANET MAC Protocol for DSRC Applications},
	author={Yi, Q. and Lu, K. and Moyeri, N.{\'a}n E},
	booktitle={Proceedings of IEEE GLOBECOM 2008},
	pages={1--5},
	year={2008},
	organization={IEEE}
}





@inproceedings{naumov2006evaluation,
	title={An evaluation of inter-vehicle ad hoc networks based on realistic vehicular traces},
	author={Naumov, Valery and Baumann, Rainer and Gross, Thomas},
	booktitle={Proceedings of the 7th ACM international symposium on Mobile ad hoc networking and computing},
	pages={108--119},
	year={2006},
	organization={ACM}
}
@article{sommer2008progressing,
	title={Progressing toward realistic mobility models in VANET simulations},
	author={Sommer, Christoph and Dressler, Falko},
	journal={Communications Magazine, IEEE},
	volume={46},
	number={11},
	pages={132--137},
	year={2008},
	publisher={IEEE}
}




@inproceedings{mahajan2006urban,
	title={Urban mobility models for vanets},
	author={Mahajan, Atulya and Potnis, Niranjan and Gopalan, Kartik and Wang, Andy},
	booktitle={2nd IEEE International Workshop on Next Generation Wireless Networks},
	volume={33},
	year={2006}
}

@inproceedings{Rakhshan2016packet,
	title={Packet success probability derivation in a vehicular ad hoc network for a highway scenario},
	author={Rakhshan, Ali and Pishro-Nik, Hossein},
	booktitle={2016 Annual Conference on Information Science and Systems (CISS)},
	pages={210--215},
	year={2016},
	organization={IEEE}
}

@inproceedings{Rakhshan2016CISS,
	Author = {Rakhshan, Ali and Pishro-Nik, Hossein},
	Booktitle = {Conference on Information Sciences and Systems},
	Organization = {IEEE},
	Pages = {210--215},
	Title = {Packet Success Probability Derivation in a Vehicular Ad Hoc Network for a Highway Scenario},
	Year = {2016}}

@article{Nekoui2013Journal,
	Author = {Nekoui, Mohammad and Pishro-Nik, Hossein},
	Journal = {Journal on Selected Areas in Communications, Special Issue on Emerging Technologies in Communications},
	Number = {9},
	Pages = {491--503},
	Publisher = {IEEE},
	Title = {Analytic Design of Active Safety Systems for Vehicular Ad hoc Networks},
	Volume = {31},
	Year = {2013}}


@inproceedings{Nekoui2011MOBICOM,
	Author = {Nekoui, Mohammad and Pishro-Nik, Hossein},
	Booktitle = {MOBICOM workshop on VehiculAr InterNETworking},
	Organization = {ACM},
	Title = {Analytic Design of Active Vehicular Safety Systems in Sparse Traffic},
	Year = {2011}}

@inproceedings{Nekoui2011VTC,
	Author = {Nekoui, Mohammad and Pishro-Nik, Hossein},
	Booktitle = {VTC-Fall},
	Organization = {IEEE},
	Title = {Analytical Design of Inter-vehicular Communications for Collision Avoidance},
	Year = {2011}}

@inproceedings{Bovee2011VTC,
	Author = {Bovee, Ben Louis and Nekoui, Mohammad and Pishro-Nik, Hossein},
	Booktitle = {VTC-Fall},
	Organization = {IEEE},
	Title = {Evaluation of the Universal Geocast Scheme For VANETs},
	Year = {2011}}

@inproceedings{Nekoui2010MOBICOM,
	Author = {Nekoui, Mohammad and Pishro-Nik, Hossein},
	Booktitle = {MOBICOM},
	Organization = {ACM},
	Title = {Fundamental Tradeoffs in Vehicular Ad Hoc Networks},
	Year = {2010}}

@inproceedings{Nekoui2010IVCS,
	Author = {Nekoui, Mohammad and Pishro-Nik, Hossein},
	Booktitle = {IVCS},
	Organization = {IEEE},
	Title = {A Universal Geocast Scheme for Vehicular Ad Hoc Networks},
	Year = {2010}}

@inproceedings{Nekoui2009ITW,
	Author = {Nekoui, Mohammad and Pishro-Nik, Hossein},
	Booktitle = {IEEE Communications Society Conference on Sensor, Mesh and Ad Hoc Communications and Networks Workshops},
	Organization = {IEEE},
	Pages = {1--3},
	Title = {A Geometrical Analysis of Obstructed Wireless Networks},
	Year = {2009}}

@article{Eslami2013Journal,
	Author = {Eslami, Ali and Nekoui, Mohammad and Pishro-Nik, Hossein and Fekri, Faramarz},
	Journal = {ACM Transactions on Sensor Networks},
	Number = {4},
	Pages = {51},
	Publisher = {ACM},
	Title = {Results on finite wireless sensor networks: Connectivity and coverage},
	Volume = {9},
	Year = {2013}}


@article{Jiafu2014Journal,
	Author = {Jiafu, W. and Zhang, D. and Zhao, S. and Yang, L. and Lloret, J.},
	Journal = {Communications Magazine},
	Number = {8},
	Pages = {106-113},
	Publisher = {IEEE},
	Title = {Context-aware vehicular cyber-physical systems with cloud support: architecture, challenges, and solutions},
	Volume = {52},
	Year = {2014}}

@inproceedings{Haas2010ACM,
	Author = {Haas, J.J. and Hu, Y.},
	Booktitle = {international workshop on VehiculAr InterNETworking},
	Organization = {ACM},
	Title = {Communication requirements for crash avoidance.},
	Year = {2010}}

@inproceedings{Yi2008GLOBECOM,
	Author = {Yi, Q. and Lu, K. and Moayeri, N.},
	Booktitle = {GLOBECOM},
	Organization = {IEEE},
	Title = {CA Secure VANET MAC Protocol for DSRC Applications.},
	Year = {2008}}

@inproceedings{Mughal2010ITSim,
	Author = {Mughal, B.M. and Wagan, A. and Hasbullah, H.},
	Booktitle = {International Symposium on Information Technology (ITSim)},
	Organization = {IEEE},
	Title = {Efficient congestion control in VANET for safety messaging.},
	Year = {2010}}

@article{Chang2011Journal,
	Author = {Chang, Y. and Lee, C. and Copeland, J.},
	Journal = {Selected Areas in Communications},
	Pages = {236 –249},
	Publisher = {IEEE},
	Title = {Goodput enhancement of VANETs in noisy CSMA/CA channels},
	Volume = {29},
	Year = {2011}}

@article{Garcia-Costa2011Journal,
	Author = {Garcia-Costa, C. and Egea-Lopez, E. and Tomas-Gabarron, J. B. and Garcia-Haro, J. and Haas, Z. J.},
	Journal = {Transactions on Intelligent Transportation Systems},
	Pages = {1 –16},
	Publisher = {IEEE},
	Title = {A stochastic model for chain collisions of vehicles equipped with vehicular communications},
	Volume = {99},
	Year = {2011}}

@article{Carbaugh2011Journal,
	Author = {Carbaugh, J. and Godbole,  D. N. and Sengupta, R. and Garcia-Haro, J. and Haas, Z. J.},
	Publisher = {Transportation Research Part C (Emerging Technologies)},
	Title = {Safety and capacity analysis of automated and manual highway systems},
	Year = {1997}}

@article{Goh2004Journal,
	Author = {Goh, P. and Wong, Y.},
	Publisher = {Appl Health Econ Health Policy},
	Title = {Driver perception response time during the signal change interval},
	Year = {2004}}

@article{Chang1985Journal,
	Author = {Chang, M.S. and Santiago, A.J.},
	Pages = {20-30},
	Publisher = {Transportation Research Record},
	Title = {Timing traffic signal changes based on driver behavior},
	Volume = {1027},
	Year = {1985}}

@article{Green2000Journal,
	Author = {Green, M.},
	Pages = {195-216},
	Publisher = {Transportation Human Factors},
	Title = {How long does it take to stop? Methodological analysis of driver perception-brake times.},
	Year = {2000}}

@article{Koppa2005,
	Author = {Koppa, R.J.},
	Pages = {195-216},
	Publisher = {http://www.fhwa.dot.gov/publications/},
	Title = {Human Factors},
	Year = {2005}}

@inproceedings{Maxwell2010ETC,
	Author = {Maxwell, A. and Wood, K.},
	Booktitle = {Europian Transport Conference},
	Organization = {http://www.etcproceedings.org/paper/review-of-traffic-signals-on-high-speed-roads},
	Title = {Review of Traffic Signals on High Speed Road},
	Year = {2010}}

@article{Wortman1983,
	Author = {Wortman, R.H. and Matthias, J.S.},
	Publisher = {Arizona Department of Transportation},
	Title = {An Evaluation of Driver Behavior at Signalized Intersections},
	Year = {1983}}
@inproceedings{Zhang2007IASTED,
	Author = {Zhang, X. and Bham, G.H.},
	Booktitle = {18th IASTED International Conference: modeling and simulation},
	Title = {Estimation of driver reaction time from detailed vehicle trajectory data.},
	Year = {2007}}


@inproceedings{bai2003important,
	title={IMPORTANT: A framework to systematically analyze the Impact of Mobility on Performance of RouTing protocols for Adhoc NeTworks},
	author={Bai, Fan and Sadagopan, Narayanan and Helmy, Ahmed},
	booktitle={INFOCOM 2003. Twenty-second annual joint conference of the IEEE computer and communications. IEEE societies},
	volume={2},
	pages={825--835},
	year={2003},
	organization={IEEE}
}


@inproceedings{abedi2008enhancing,
	title={Enhancing AODV routing protocol using mobility parameters in VANET},
	author={Abedi, Omid and Fathy, Mahmood and Taghiloo, Jamshid},
	booktitle={Computer Systems and Applications, 2008. AICCSA 2008. IEEE/ACS International Conference on},
	pages={229--235},
	year={2008},
	organization={IEEE}
}


@article{AlSultan2013Journal,
	Author = {Al-Sultan, Saif and Al-Bayatti, Ali H. and Zedan, Hussien},
	Journal = {IEEE Transactions on Vehicular Technology},
	Number = {9},
	Pages = {4264-4275},
	Publisher = {IEEE},
	Title = {Context Aware Driver Behaviour Detection System in Intelligent Transportation Systems},
	Volume = {62},
	Year = {2013}}






@article{Leow2008ITS,
	Author = {Leow, Woei Ling and Ni, Daiheng and Pishro-Nik, Hossein},
	Journal = {IEEE Transactions on Intelligent Transportation Systems},
	Number = {2},
	Pages = {369--374},
	Publisher = {IEEE},
	Title = {A Sampling Theorem Approach to Traffic Sensor Optimization},
	Volume = {9},
	Year = {2008}}



@article{REU2007,
	Author = {D. Ni and H. Pishro-Nik and R. Prasad and M. R. Kanjee and H. Zhu and T. Nguyen},
	Journal = {in 14th World Congress on Intelligent Transport Systems},
	Title = {Development of a prototype intersection collision avoidance system under VII},
	Year = {2007}}




@inproceedings{salamatian2013hide,
	title={How to hide the elephant-or the donkey-in the room: Practical privacy against statistical inference for large data.},
	author={Salamatian, Salman and Zhang, Amy and du Pin Calmon, Flavio and Bhamidipati, Sandilya and Fawaz, Nadia and Kveton, Branislav and Oliveira, Pedro and Taft, Nina},
	booktitle={GlobalSIP},
	pages={269--272},
	year={2013}
}

@article{sankar2013utility,
	title={Utility-privacy tradeoffs in databases: An information-theoretic approach},
	author={Sankar, Lalitha and Rajagopalan, S Raj and Poor, H Vincent},
	journal={Information Forensics and Security, IEEE Transactions on},
	volume={8},
	number={6},
	pages={838--852},
	year={2013},
	publisher={IEEE}
}
@inproceedings{ghinita2007prive,
	title={PRIVE: anonymous location-based queries in distributed mobile systems},
	author={Ghinita, Gabriel and Kalnis, Panos and Skiadopoulos, Spiros},
	booktitle={Proceedings of the 16th international conference on World Wide Web},
	pages={371--380},
	year={2007},
	organization={ACM}
}

@article{beresford2004mix,
	title={Mix zones: User privacy in location-aware services},
	author={Beresford, Alastair R and Stajano, Frank},
	year={2004},
	publisher={IEEE}
}

%@inproceedings{Mont1610Achieving,
	%  title={Achieving Perfect Location Privacy in Markov Models Using Anonymization},
	%  author={Montazeri, Zarrin and Houmansadr, Amir and H.Pishro-Nik},
	%  booktitle="2016 International Symposium on Information Theory and its Applications
	%  (ISITA2016)",
	%  address="Monterey, USA",
	%  days=30,
	%  month=oct,
	%  year=2016,
	%}

@article{csiszar1996almost,
	title={Almost independence and secrecy capacity},
	author={Csisz{\'a}r, Imre},
	journal={Problemy Peredachi Informatsii},
	volume={32},
	number={1},
	pages={48--57},
	year={1996},
	publisher={Russian Academy of Sciences, Branch of Informatics, Computer Equipment and Automatization}
}

@article{yamamoto1983source,
	title={A source coding problem for sources with additional outputs to keep secret from the receiver or wiretappers (corresp.)},
	author={Yamamoto, Hirosuke},
	journal={IEEE Transactions on Information Theory},
	volume={29},
	number={6},
	pages={918--923},
	year={1983},
	publisher={IEEE}
}


@inproceedings{calmon2015fundamental,
	title={Fundamental limits of perfect privacy},
	author={Calmon, Flavio P and Makhdoumi, Ali and M{\'e}dard, Muriel},
	booktitle={Information Theory (ISIT), 2015 IEEE International Symposium on},
	pages={1796--1800},
	year={2015},
	organization={IEEE}
}



@inproceedings{Lehman1999Large-Sample-Theory,
	title={Elements of Large Sample Theory},
	author={E. L. Lehman},
	organization={Springer},
	year={1999}
}


@inproceedings{Ferguson1999Large-Sample-Theory,
	title={A Course in Large Sample Theory},
	author={Thomas S. Ferguson},
	organization={CRC Press},
	year={1996}
}



@inproceedings{Dembo1999Large-Deviations,
	title={Large Deviation Techniques and Applications, Second Edition},
	author={A. Dembo and O. Zeitouni},
	organization={Springer},
	year={1998}
}


%%%%%%%%%%%%%%%%%%%%%%%%%%%%%%%%%%%%%%%%%%%%%%%%
Hossein's Coding Journals
%%%%%%%%%%%%%%%%%%%%%%

@ARTICLE{myoptics,
	AUTHOR =       "H. Pishro-Nik and N. Rahnavard and J. Ha and F. Fekri and A. Adibi ",
	TITLE =        "Low-density parity-check codes for volume holographic memory systems",
	JOURNAL =      " Appl. Opt.",
	YEAR =         "2003",
	volume =       "42",
	pages =        "861-870  "
}






@ARTICLE{myit,
	AUTHOR =       "H. Pishro-Nik and F. Fekri  ",
	TITLE =        "On Decoding of Low-Density Parity-Check Codes on the Binary Erasure Channel",
	JOURNAL =      "IEEE Trans. Inform. Theory",
	YEAR =         "2004",
	volume =       "50",
	pages =        "439--454"
}




@ARTICLE{myitpuncture,
	AUTHOR =       "H. Pishro-Nik and F. Fekri  ",
	TITLE =        "Results on Punctured Low-Density Parity-Check Codes and Improved Iterative Decoding Techniques",
	JOURNAL =      "IEEE Trans. on Inform. Theory",
	YEAR =         "2007",
	volume =       "53",
	number=        "2",
	pages =        "599--614",
	month= "February"
}




@ARTICLE{myitlinmimdist,
	AUTHOR =       "H. Pishro-Nik and F. Fekri",
	TITLE =        "Performance of Low-Density Parity-Check Codes With Linear Minimum Distance",
	JOURNAL =         "IEEE Trans. Inform. Theory ",
	YEAR =         "2006",
	volume =       "52",
	number="1",
	pages =        "292 --300"
}






@ARTICLE{myitnonuni,
	AUTHOR =       "H. Pishro-Nik and N. Rahnavard and F. Fekri  ",
	TITLE =        "Non-uniform Error Correction Using Low-Density Parity-Check Codes",
	JOURNAL =      "IEEE Trans. Inform. Theory",
	YEAR =         "2005",
	volume =       "51",
	number=  "7",
	pages =        "2702--2714"
}





@article{eslamitcomhybrid10,
	author = {A. Eslami and S. Vangala and H. Pishro-Nik},
	title = {Hybrid channel codes for highly efficient FSO/RF communication systems},
	journal = {IEEE Transactions on Communications},
	volume = {58},
	number = {10},
	year = {2010},
	pages = {2926--2938},
}


@article{eslamitcompolar13,
	author = {A. Eslami and H. Pishro-Nik},
	title = {On Finite-Length Performance of Polar Codes: Stopping Sets, Error Floor, and Concatenated Design},
	journal = {IEEE Transactions on Communications},
	volume = {61},
	number = {13},
	year = {2013},
	pages = {919--929},
}



@article{saeeditcom11,
	author = {H. Saeedi and H. Pishro-Nik and  A. H. Banihashemi},
	title = {Successive maximization for the systematic design of universally capacity approaching rate-compatible
	sequences of LDPC code ensembles over binary-input output-symmetric memoryless channels},
	journal = {IEEE Transactions on Communications},
	year = {2011},
	volume={59},
	number = {7}
}


@article{rahnavard07,
	author = {Rahnavard, N. and Pishro-Nik, H. and Fekri, F.},
	title = {Unequal Error Protection Using Partially Regular LDPC Codes},
	journal = {IEEE Transactions on Communications},
	year = {2007},
	volume = {55},
	number = {3},
	pages = {387 -- 391}
}


@article{hosseinira04,
	author = {H. Pishro-Nik and F. Fekri},
	title = {Irregular repeat-accumulate codes for volume holographic memory systems},
	journal = {Journal of Applied Optics},
	year = {2004},
	volume = {43},
	number = {27},
	pages = {5222--5227},
}


@article{azadeh2015Ephemeralkey,
	author = {A. Sheikholeslami and D. Goeckel and H. Pishro-Nik},
	title = {Jamming Based on an Ephemeral Key to Obtain Everlasting Security in Wireless Environments},
	journal = {IEEE Transactions on Wireless Communications},
	year = {2015},
	volume = {14},
	number = {11},
	pages = {6072--6081},
}


@article{azadeh2014Everlasting,
	author = {A. Sheikholeslami and D. Goeckel and H. Pishro-Nik},
	title = {Everlasting secrecy in disadvantaged wireless environments against sophisticated eavesdroppers},
	journal = {48th Asilomar Conference on Signals, Systems and Computers},
	year = {2014},
	pages = {1994--1998},
}


@article{azadeh2013ISIT,
	author = {A. Sheikholeslami and D. Goeckel and H. Pishro-Nik},
	title = {Artificial intersymbol interference (ISI) to exploit receiver imperfections for secrecy},
	journal = {IEEE International Symposium on Information Theory (ISIT)},
	year = {2013},
}


@article{azadeh2013Jsac,
	author = {A. Sheikholeslami and D. Goeckel and H. Pishro-Nik},
	title = {Jamming Based on an Ephemeral Key to Obtain Everlasting Security in Wireless Environments},
	journal = {IEEE Journal on Selected Areas in Communications},
	year = {2013},
	volume = {31},
	number = {9},
	pages = {1828--1839},
}


@article{azadeh2012Allerton,
	author = {A. Sheikholeslami and D. Goeckel and H. Pishro-Nik},
	title = {Exploiting the non-commutativity of nonlinear operators for information-theoretic security in disadvantaged wireless environments},
	journal = {50th Annual Allerton Conference on Communication, Control, and Computing},
	year = {2012},
	pages = {233--240},
}


@article{azadeh2012Infocom,
	author = {A. Sheikholeslami and D. Goeckel and H. Pishro-Nik},
	title = {Jamming Based on an Ephemeral Key to Obtain Everlasting Security in Wireless Environments},
	journal = {IEEE INFOCOM},
	year = {2012},
	pages = {1179--1187},
}

@article{1corser2016evaluating,
	title={Evaluating Location Privacy in Vehicular Communications and Applications},
	author={Corser, George P and Fu, Huirong and Banihani, Abdelnasser},
	journal={IEEE Transactions on Intelligent Transportation Systems},
	volume={17},
	number={9},
	pages={2658-2667},
	year={2016},
	publisher={IEEE}
}
@article{2zhang2016designing,
	title={On Designing Satisfaction-Ratio-Aware Truthful Incentive Mechanisms for k-Anonymity Location Privacy},
	author={Zhang, Yuan and Tong, Wei and Zhong, Sheng},
	journal={IEEE Transactions on Information Forensics and Security},
	volume={11},
	number={11},
	pages={2528--2541},
	year={2016},
	publisher={IEEE}
}
@article{3li2016privacy,
	title={Privacy-preserving Location Proof for Securing Large-scale Database-driven Cognitive Radio Networks},
	author={Li, Yi and Zhou, Lu and Zhu, Haojin and Sun, Limin},
	journal={IEEE Internet of Things Journal},
	volume={3},
	number={4},
	pages={563-571},
	year={2016},
	publisher={IEEE}
}
@article{4olteanu2016quantifying,
	title={Quantifying Interdependent Privacy Risks with Location Data},
	author={Olteanu, Alexandra-Mihaela and Huguenin, K{\'e}vin and Shokri, Reza and Humbert, Mathias and Hubaux, Jean-Pierre},
	journal={IEEE Transactions on Mobile Computing},
	year={2016},
	volume={PP},
	number={99},
	pages={1-1},
	publisher={IEEE}
}
@article{5yi2016practical,
	title={Practical Approximate k Nearest Neighbor Queries with Location and Query Privacy},
	author={Yi, Xun and Paulet, Russell and Bertino, Elisa and Varadharajan, Vijay},
	journal={IEEE Transactions on Knowledge and Data Engineering},
	volume={28},
	number={6},
	pages={1546--1559},
	year={2016},
	publisher={IEEE}
}
@article{6li2016privacy,
	title={Privacy Leakage of Location Sharing in Mobile Social Networks: Attacks and Defense},
	author={Li, Huaxin and Zhu, Haojin and Du, Suguo and Liang, Xiaohui and Shen, Xuemin},
	journal={IEEE Transactions on Dependable and Secure Computing},
	year={2016},
	volume={PP},
	number={99},
	publisher={IEEE}
}

@article{7murakami2016localization,
	title={Localization Attacks Using Matrix and Tensor Factorization},
	author={Murakami, Takao and Watanabe, Hajime},
	journal={IEEE Transactions on Information Forensics and Security},
	volume={11},
	number={8},
	pages={1647--1660},
	year={2016},
	publisher={IEEE}
}
@article{8zurbaran2015near,
	title={Near-Rand: Noise-based Location Obfuscation Based on Random Neighboring Points},
	author={Zurbaran, Mayra Alejandra and Avila, Karen and Wightman, Pedro and Fernandez, Michael},
	journal={IEEE Latin America Transactions},
	volume={13},
	number={11},
	pages={3661--3667},
	year={2015},
	publisher={IEEE}
}

@article{9tan2014anti,
	title={An anti-tracking source-location privacy protection protocol in wsns based on path extension},
	author={Tan, Wei and Xu, Ke and Wang, Dan},
	journal={IEEE Internet of Things Journal},
	volume={1},
	number={5},
	pages={461--471},
	year={2014},
	publisher={IEEE}
}

@article{10peng2014enhanced,
	title={Enhanced Location Privacy Preserving Scheme in Location-Based Services},
	author={Peng, Tao and Liu, Qin and Wang, Guojun},
	journal={IEEE Systems Journal},
	year={2014},
	volume={PP},
	number={99},
	pages={1-12},
	publisher={IEEE}
}
@article{11dewri2014exploiting,
	title={Exploiting service similarity for privacy in location-based search queries},
	author={Dewri, Rinku and Thurimella, Ramakrisha},
	journal={IEEE Transactions on Parallel and Distributed Systems},
	volume={25},
	number={2},
	pages={374--383},
	year={2014},
	publisher={IEEE}
}

@article{12hwang2014novel,
	title={A novel time-obfuscated algorithm for trajectory privacy protection},
	author={Hwang, Ren-Hung and Hsueh, Yu-Ling and Chung, Hao-Wei},
	journal={IEEE Transactions on Services Computing},
	volume={7},
	number={2},
	pages={126--139},
	year={2014},
	publisher={IEEE}
}
@article{13puttaswamy2014preserving,
	title={Preserving location privacy in geosocial applications},
	author={Puttaswamy, Krishna PN and Wang, Shiyuan and Steinbauer, Troy and Agrawal, Divyakant and El Abbadi, Amr and Kruegel, Christopher and Zhao, Ben Y},
	journal={IEEE Transactions on Mobile Computing},
	volume={13},
	number={1},
	pages={159--173},
	year={2014},
	publisher={IEEE}
}

@article{14zhang2014privacy,
	title={Privacy quantification model based on the Bayes conditional risk in Location-Based Services},
	author={Zhang, Xuejun and Gui, Xiaolin and Tian, Feng and Yu, Si and An, Jian},
	journal={Tsinghua Science and Technology},
	volume={19},
	number={5},
	pages={452--462},
	year={2014},
	publisher={TUP}
}

@article{15bilogrevic2014privacy,
	title={Privacy-preserving optimal meeting location determination on mobile devices},
	author={Bilogrevic, Igor and Jadliwala, Murtuza and Joneja, Vishal and Kalkan, K{\"u}bra and Hubaux, Jean-Pierre and Aad, Imad},
	journal={IEEE transactions on information forensics and security},
	volume={9},
	number={7},
	pages={1141--1156},
	year={2014},
	publisher={IEEE}
}
@article{16haghnegahdar2014privacy,
	title={Privacy Risks in Publishing Mobile Device Trajectories},
	author={Haghnegahdar, Alireza and Khabbazian, Majid and Bhargava, Vijay K},
	journal={IEEE Wireless Communications Letters},
	volume={3},
	number={3},
	pages={241--244},
	year={2014},
	publisher={IEEE}
}
@article{17malandrino2014verification,
	title={Verification and inference of positions in vehicular networks through anonymous beaconing},
	author={Malandrino, Francesco and Borgiattino, Carlo and Casetti, Claudio and Chiasserini, Carla-Fabiana and Fiore, Marco and Sadao, Roberto},
	journal={IEEE Transactions on Mobile Computing},
	volume={13},
	number={10},
	pages={2415--2428},
	year={2014},
	publisher={IEEE}
}
@article{18shokri2014hiding,
	title={Hiding in the mobile crowd: Locationprivacy through collaboration},
	author={Shokri, Reza and Theodorakopoulos, George and Papadimitratos, Panos and Kazemi, Ehsan and Hubaux, Jean-Pierre},
	journal={IEEE transactions on dependable and secure computing},
	volume={11},
	number={3},
	pages={266--279},
	year={2014},
	publisher={IEEE}
}
@article{19freudiger2013non,
	title={Non-cooperative location privacy},
	author={Freudiger, Julien and Manshaei, Mohammad Hossein and Hubaux, Jean-Pierre and Parkes, David C},
	journal={IEEE Transactions on Dependable and Secure Computing},
	volume={10},
	number={2},
	pages={84--98},
	year={2013},
	publisher={IEEE}
}
@article{20gao2013trpf,
	title={TrPF: A trajectory privacy-preserving framework for participatory sensing},
	author={Gao, Sheng and Ma, Jianfeng and Shi, Weisong and Zhan, Guoxing and Sun, Cong},
	journal={IEEE Transactions on Information Forensics and Security},
	volume={8},
	number={6},
	pages={874--887},
	year={2013},
	publisher={IEEE}
}
@article{21ma2013privacy,
	title={Privacy vulnerability of published anonymous mobility traces},
	author={Ma, Chris YT and Yau, David KY and Yip, Nung Kwan and Rao, Nageswara SV},
	journal={IEEE/ACM Transactions on Networking},
	volume={21},
	number={3},
	pages={720--733},
	year={2013},
	publisher={IEEE}
}
@article{22niu2013pseudo,
	title={Pseudo-Location Updating System for privacy-preserving location-based services},
	author={Niu, Ben and Zhu, Xiaoyan and Chi, Haotian and Li, Hui},
	journal={China Communications},
	volume={10},
	number={9},
	pages={1--12},
	year={2013},
	publisher={IEEE}
}
@article{23dewri2013local,
	title={Local differential perturbations: Location privacy under approximate knowledge attackers},
	author={Dewri, Rinku},
	journal={IEEE Transactions on Mobile Computing},
	volume={12},
	number={12},
	pages={2360--2372},
	year={2013},
	publisher={IEEE}
}
@inproceedings{24kanoria2012tractable,
	title={Tractable bayesian social learning on trees},
	author={Kanoria, Yashodhan and Tamuz, Omer},
	booktitle={Information Theory Proceedings (ISIT), 2012 IEEE International Symposium on},
	pages={2721--2725},
	year={2012},
	organization={IEEE}
}
@inproceedings{25farias2005universal,
	title={A universal scheme for learning},
	author={Farias, Vivek F and Moallemi, Ciamac C and Van Roy, Benjamin and Weissman, Tsachy},
	booktitle={Proceedings. International Symposium on Information Theory, 2005. ISIT 2005.},
	pages={1158--1162},
	year={2005},
	organization={IEEE}
}
@inproceedings{26misra2013unsupervised,
	title={Unsupervised learning and universal communication},
	author={Misra, Vinith and Weissman, Tsachy},
	booktitle={Information Theory Proceedings (ISIT), 2013 IEEE International Symposium on},
	pages={261--265},
	year={2013},
	organization={IEEE}
}
@inproceedings{27ryabko2013time,
	title={Time-series information and learning},
	author={Ryabko, Daniil},
	booktitle={Information Theory Proceedings (ISIT), 2013 IEEE International Symposium on},
	pages={1392--1395},
	year={2013},
	organization={IEEE}
}
@inproceedings{28krzakala2013phase,
	title={Phase diagram and approximate message passing for blind calibration and dictionary learning},
	author={Krzakala, Florent and M{\'e}zard, Marc and Zdeborov{\'a}, Lenka},
	booktitle={Information Theory Proceedings (ISIT), 2013 IEEE International Symposium on},
	pages={659--663},
	year={2013},
	organization={IEEE}
}
@inproceedings{29sakata2013sample,
	title={Sample complexity of Bayesian optimal dictionary learning},
	author={Sakata, Ayaka and Kabashima, Yoshiyuki},
	booktitle={Information Theory Proceedings (ISIT), 2013 IEEE International Symposium on},
	pages={669--673},
	year={2013},
	organization={IEEE}
}
@inproceedings{30predd2004consistency,
	title={Consistency in a model for distributed learning with specialists},
	author={Predd, Joel B and Kulkarni, Sanjeev R and Poor, H Vincent},
	booktitle={IEEE International Symposium on Information Theory},
	year={2004},
	organization={IEEE}
}
@inproceedings{31nokleby2016rate,
	title={Rate-Distortion Bounds on Bayes Risk in Supervised Learning},
	author={Nokleby, Matthew and Beirami, Ahmad and Calderbank, Robert},
	booktitle={2016 IEEE International Symposium on Information Theory (ISIT)},
	pages={2099-2103},
	year={2016},
	organization={IEEE}
}

@inproceedings{32le2016imperfect,
	title={Are imperfect reviews helpful in social learning?},
	author={Le, Tho Ngoc and Subramanian, Vijay G and Berry, Randall A},
	booktitle={Information Theory (ISIT), 2016 IEEE International Symposium on},
	pages={2089--2093},
	year={2016},
	organization={IEEE}
}
@inproceedings{33gadde2016active,
	title={Active Learning for Community Detection in Stochastic Block Models},
	author={Gadde, Akshay and Gad, Eyal En and Avestimehr, Salman and Ortega, Antonio},
	booktitle={2016 IEEE International Symposium on Information Theory (ISIT)},
	pages={1889-1893},
	year={2016}
}
@inproceedings{34shakeri2016minimax,
	title={Minimax Lower Bounds for Kronecker-Structured Dictionary Learning},
	author={Shakeri, Zahra and Bajwa, Waheed U and Sarwate, Anand D},
	booktitle={2016 IEEE International Symposium on Information Theory (ISIT)},
	pages={1148-1152},
	year={2016}
}
@article{35lee2015speeding,
	title={Speeding up distributed machine learning using codes},
	author={Lee, Kangwook and Lam, Maximilian and Pedarsani, Ramtin and Papailiopoulos, Dimitris and Ramchandran, Kannan},
	booktitle={2016 IEEE International Symposium on Information Theory (ISIT)},
	pages={1143-1147},
	year={2016}
}
@article{36oneto2016statistical,
	title={Statistical Learning Theory and ELM for Big Social Data Analysis},
	author={Oneto, Luca and Bisio, Federica and Cambria, Erik and Anguita, Davide},
	journal={ieee CompUTATionAl inTelliGenCe mAGAzine},
	volume={11},
	number={3},
	pages={45--55},
	year={2016},
	publisher={IEEE}
}
@article{37lin2015probabilistic,
	title={Probabilistic approach to modeling and parameter learning of indirect drive robots from incomplete data},
	author={Lin, Chung-Yen and Tomizuka, Masayoshi},
	journal={IEEE/ASME Transactions on Mechatronics},
	volume={20},
	number={3},
	pages={1036--1045},
	year={2015},
	publisher={IEEE}
}
@article{38wang2016towards,
	title={Towards Bayesian Deep Learning: A Framework and Some Existing Methods},
	author={Wang, Hao and Yeung, Dit-Yan},
	journal={IEEE Transactions on Knowledge and Data Engineering},
	volume={PP},
	number={99},
	year={2016},
	publisher={IEEE}
}


%%%%%Informationtheoreticsecurity%%%%%%%%%%%%%%%%%%%%%%%




@inproceedings{Bloch2011PhysicalSecBook,
	title={Physical-Layer Security},
	author={M. Bloch and J. Barros},
	organization={Cambridge University Press},
	year={2011}
}



@inproceedings{Liang2009InfoSecBook,
	title={Information Theoretic Security},
	author={Y. Liang and H. V. Poor and S. Shamai (Shitz)},
	organization={Now Publishers Inc.},
	year={2009}
}


@inproceedings{Zhou2013PhysicalSecBook,
	title={Physical Layer Security in Wireless Communications},
	author={ X. Zhou and L. Song and Y. Zhang},
	organization={CRC Press},
	year={2013}
}

@article{Ni2012IEA,
	Author = {D. Ni and H. Liu and W. Ding and  Y. Xie and H. Wang and H. Pishro-Nik and Q. Yu},
	Journal = {IEA/AIE},
	Title = {Cyber-Physical Integration to Connect Vehicles for Transformed Transportation Safety and Efficiency},
	Year = {2012}}



@inproceedings{Ni2012Inproceedings,
	Author = {D. Ni, H. Liu, Y. Xie, W. Ding, H. Wang, H. Pishro-Nik, Q. Yu and M. Ferreira},
	Booktitle = {Spring Simulation Multiconference},
	Date-Added = {2016-09-04 14:18:42 +0000},
	Date-Modified = {2016-09-06 16:22:14 +0000},
	Title = {Virtual Lab of Connected Vehicle Technology},
	Year = {2012}}

@inproceedings{Ni2012Inproceedings,
	Author = {D. Ni, H. Liu, W. Ding, Y. Xie, H. Wang, H. Pishro-Nik and Q. Yu,},
	Booktitle = {IEA/AIE},
	Date-Added = {2016-09-04 09:11:02 +0000},
	Date-Modified = {2016-09-06 14:46:53 +0000},
	Title = {Cyber-Physical Integration to Connect Vehicles for Transformed Transportation Safety and Efficiency},
	Year = {2012}}


@article{Nekoui_IJIPT_2009,
	Author = {M. Nekoui and D. Ni and H. Pishro-Nik and R. Prasad and M. Kanjee and H. Zhu and T. Nguyen},
	Journal = {International Journal of Internet Protocol Technology (IJIPT)},
	Number = {3},
	Pages = {},
	Publisher = {},
	Title = {Development of a VII-Enabled Prototype Intersection Collision Warning System},
	Volume = {4},
	Year = {2009}}


@inproceedings{Pishro_Ganz_Ni,
	Author = {H. Pishro-Nik, A. Ganz, and Daiheng Ni},
	Booktitle = {Forty-Fifth Annual Allerton Conference on Communication, Control, and Computing. Allerton House, Monticello, IL},
	Date-Added = {},
	Date-Modified = {},
	Number = {},
	Pages = {},
	Title = {The capacity of vehicular ad hoc networks},
	Volume = {},
	Year = {September 26-28, 2007}}

@inproceedings{Leow_Pishro_Ni_1,
	Author = {W. L. Leow, H. Pishro-Nik and Daiheng Ni},
	Booktitle = {IEEE Global Telecommunications Conference, Washington, D.C.},
	Date-Added = {},
	Date-Modified = {},
	Number = {},
	Pages = {},
	Title = {Delay and Energy Tradeoff in Multi-state Wireless Sensor Networks},
	Volume = {},
	Year = {November 26-30, 2007}}


@misc{UMass-Trans,
	title = {{UMass Transportation Center}},
	note = {\url{http://www.umasstransportationcenter.org/}},
}


@inproceedings{Haenggi2013book,
	title={Stochastic geometry for wireless networks},
	author={M. Haenggi},
	organization={Cambridge Uinversity Press},
	year={2013}
}


%%%%%%%%%%%%%%%%personalization%%%%%%%%%%%%%%%%%%%%%%%%%%%%%%%%%%

@article{osma2015,
	title={Impact of Time-to-Collision Information on Driving Behavior in Connected Vehicle Environments Using A Driving Simulator Test Bed},
	journal{Journal of Traffic and Logistics Engineering},
	author={Osama A. Osman, Julius Codjoe, and Sherif Ishak},
	volume={3},
	number={1},
	pages={18--24},
	year={2015}
}


@article{charisma2010,
	title={Dynamic Latent Plan Models},
	author={Charisma F. Choudhurya, Moshe Ben-Akivab and Maya Abou-Zeid},
	journal={Journal of Choice Modelling},
	volume={3},
	number={2},
	pages={50--70},
	year={2010},
	publisher={Elsvier}
}


@misc{noble2014,
	author = {A. M. Noble, Shane B. McLaughlin, Zachary R. Doerzaph and Thomas A. Dingus},
	title = {Crowd-sourced Connected-vehicle Warning Algorithm using Naturalistic Driving Data},
	howpublished = {Downloaded from \url{http://hdl.handle.net/10919/53978}},
	
	month = August,
	year = 2014
}


@phdthesis{charisma2007,
	title    = {Modeling Driving Decisions with Latent Plans},
	school   = {Massachusetts Institute of Technology },
	author   = {Charisma Farheen Choudhury},
	year     = {2007}, %other attributes omitted
}


@article{chrysler2015,
	title={Cost of Warning of Unseen Threats:Unintended Consequences of Connected Vehicle Alerts},
	author={S. T. Chrysler, J. M. Cooper and D. C. Marshall},
	journal={Transportation Research Record: Journal of the Transportation Research Board},
	volume={2518},
	pages={79--85},
	year={2015},
}

@misc{nsf_cps,
	title = {Cyber-Physical Systems (CPS) PROGRAM SOLICITATION NSF 17-529},
	howpublished = {Downloaded from \url{https://www.nsf.gov/publications/pub_summ.jsp?WT.z_pims_id=503286&ods_key=nsf17529}},
}



%%%%%%%%%%%%%%IOT%%%%%%%%%%%%%%%%%%%%%%%%%%%%%%%%%%%%%%%%%%%%%%%%%%%




@article{FTC2015,
	title={Internet of Things: Privacy and Security in a Connected World},
	author={FTC Staff Report},
	year={2015}
}



%% Saved with string encoding Unicode (UTF-8)
@inproceedings{1zhou2014security,
	title={Security/privacy of wearable fitness tracking {I}o{T} devices},
	author={Zhou, Wei and Piramuthu, Selwyn},
	booktitle={Information Systems and Technologies (CISTI), 2014 9th Iberian Conference on},
	pages={1--5},
	year={2014},
	organization={IEEE}
}

@article{2nia2016comprehensive,
	title={A comprehensive study of security of internet-of-things},
	author={Mohsenia, Arsalan and Jha, Niraj K},
	journal={IEEE Transactions on Emerging Topics in Computing},
	volum={5},
	number={4},
	pages={586--602},
	year={2017},
	publisher={IEEE}
}

@inproceedings{3ukil2014iot,
	title={{I}o{T}-privacy: To be private or not to be private},
	author={Ukil, Arijit and Bandyopadhyay, Soma and Pal, Arpan},
	booktitle={Computer Communications Workshops (INFOCOM WKSHPS), IEEE Conference on},
	pages={123--124},
	year={2014},
	organization={IEEE}
}

@article{4arias2015privacy,
	title={Privacy and security in internet of things and wearable devices},
	author={Arias, Orlando and Wurm, Jacob and Hoang, Khoa and Jin, Yier},
	journal={IEEE Transactions on Multi-Scale Computing Systems},
	volume={1},
	number={2},
	pages={99--109},
	year={2015},
	publisher={IEEE}
}
@inproceedings{5ullah2016novel,
	title={A novel model for preserving Location Privacy in Internet of Things},
	author={Ullah, Ikram and Shah, Munam Ali},
	booktitle={Automation and Computing (ICAC), 2016 22nd International Conference on},
	pages={542--547},
	year={2016},
	organization={IEEE}
}
@inproceedings{6sathishkumar2016enhanced,
	title={Enhanced location privacy algorithm for wireless sensor network in Internet of Things},
	author={Sathishkumar, J and Patel, Dhiren R},
	booktitle={Internet of Things and Applications (IOTA), International Conference on},
	pages={208--212},
	year={2016},
	organization={IEEE}
}
@inproceedings{7zhou2012preserving,
	title={Preserving sensor location privacy in internet of things},
	author={Zhou, Liming and Wen, Qiaoyan and Zhang, Hua},
	booktitle={Computational and Information Sciences (ICCIS), 2012 Fourth International Conference on},
	pages={856--859},
	year={2012},
	organization={IEEE}
}

@inproceedings{8ukil2015privacy,
	title={Privacy for {I}o{T}: Involuntary privacy enablement for smart energy systems},
	author={Ukil, Arijit and Bandyopadhyay, Soma and Pal, Arpan},
	booktitle={Communications (ICC), 2015 IEEE International Conference on},
	pages={536--541},
	year={2015},
	organization={IEEE}
}

@inproceedings{9dalipi2016security,
	title={Security and Privacy Considerations for {I}o{T} Application on Smart Grids: Survey and Research Challenges},
	author={Dalipi, Fisnik and Yayilgan, Sule Yildirim},
	booktitle={Future Internet of Things and Cloud Workshops (FiCloudW), IEEE International Conference on},
	pages={63--68},
	year={2016},
	organization={IEEE}
}
@inproceedings{10harris2016security,
	title={Security and Privacy in Public {I}o{T} Spaces},
	author={Harris, Albert F and Sundaram, Hari and Kravets, Robin},
	booktitle={Computer Communication and Networks (ICCCN), 2016 25th International Conference on},
	pages={1--8},
	year={2016},
	organization={IEEE}
}

@inproceedings{11al2015security,
	title={Security and privacy framework for ubiquitous healthcare {I}o{T} devices},
	author={Al Alkeem, Ebrahim and Yeun, Chan Yeob and Zemerly, M Jamal},
	booktitle={Internet Technology and Secured Transactions (ICITST), 2015 10th International Conference for},
	pages={70--75},
	year={2015},
	organization={IEEE}
}
@inproceedings{12sivaraman2015network,
	title={Network-level security and privacy control for smart-home {I}o{T} devices},
	author={Sivaraman, Vijay and Gharakheili, Hassan Habibi and Vishwanath, Arun and Boreli, Roksana and Mehani, Olivier},
	booktitle={Wireless and Mobile Computing, Networking and Communications (WiMob), 2015 IEEE 11th International Conference on},
	pages={163--167},
	year={2015},
	organization={IEEE}
}

@inproceedings{13srinivasan2016privacy,
	title={Privacy conscious architecture for improving emergency response in smart cities},
	author={Srinivasan, Ramya and Mohan, Apurva and Srinivasan, Priyanka},
	booktitle={Smart City Security and Privacy Workshop (SCSP-W), 2016},
	pages={1--5},
	year={2016},
	organization={IEEE}
}
@inproceedings{14sadeghi2015security,
	title={Security and privacy challenges in industrial internet of things},
	author={Sadeghi, Ahmad-Reza and Wachsmann, Christian and Waidner, Michael},
	booktitle={Design Automation Conference (DAC), 2015 52nd ACM/EDAC/IEEE},
	pages={1--6},
	year={2015},
	organization={IEEE}
}
@inproceedings{15otgonbayar2016toward,
	title={Toward Anonymizing {I}o{T} Data Streams via Partitioning},
	author={Otgonbayar, Ankhbayar and Pervez, Zeeshan and Dahal, Keshav},
	booktitle={Mobile Ad Hoc and Sensor Systems (MASS), 2016 IEEE 13th International Conference on},
	pages={331--336},
	year={2016},
	organization={IEEE}
}
@inproceedings{16rutledge2016privacy,
	title={Privacy Impacts of {I}o{T} Devices: A SmartTV Case Study},
	author={Rutledge, Richard L and Massey, Aaron K and Ant{\'o}n, Annie I},
	booktitle={Requirements Engineering Conference Workshops (REW), IEEE International},
	pages={261--270},
	year={2016},
	organization={IEEE}
}

@inproceedings{17andrea2015internet,
	title={Internet of Things: Security vulnerabilities and challenges},
	author={Andrea, Ioannis and Chrysostomou, Chrysostomos and Hadjichristofi, George},
	booktitle={Computers and Communication (ISCC), 2015 IEEE Symposium on},
	pages={180--187},
	year={2015},
	organization={IEEE}
}






























%%%%%%%%%%%%%%%%%%%%%%%%%%%%%%%%%%%%%%%%%%%%%%%%%%%%%%%%%%%


@misc{epfl-mobility-20090224,
	author = {Michal Piorkowski and Natasa Sarafijanovic-Djukic and Matthias Grossglauser},
	title = {{CRAWDAD} dataset epfl/mobility (v. 2009-02-24)},
	howpublished = {Downloaded from \url{http://crawdad.org/epfl/mobility/20090224}},
	doi = {10.15783/C7J010},
	month = feb,
	year = 2009
}

@misc{roma-taxi-20140717,
	author = {Lorenzo Bracciale and Marco Bonola and Pierpaolo Loreti and Giuseppe Bianchi and Raul Amici and Antonello Rabuffi},
	title = {{CRAWDAD} dataset roma/taxi (v. 2014-07-17)},
	howpublished = {Downloaded from \url{http://crawdad.org/roma/taxi/20140717}},
	doi = {10.15783/C7QC7M},
	month = jul,
	year = 2014
}

@misc{rice-ad_hoc_city-20030911,
	author = {Jorjeta G. Jetcheva and Yih-Chun Hu and Santashil PalChaudhuri and Amit Kumar Saha and David B. Johnson},
	title = {{CRAWDAD} dataset rice/ad\_hoc\_city (v. 2003-09-11)},
	howpublished = {Downloaded from \url{http://crawdad.org/rice/ad_hoc_city/20030911}},
	doi = {10.15783/C73K5B},
	month = sep,
	year = 2003
}

@misc{china:2012,
	author = {Microsoft Research Asia},
	title = {GeoLife GPS Trajectories},
	year = {2012},
	howpublished= {\url{https://www.microsoft.com/en-us/download/details.aspx?id=52367}},
}


@misc{china:2011,
	ALTauthor = {Microsoft Research Asia)},
	ALTeditor = {},
	title = {GeoLife GPS Trajectories,
	year = {2012},
	url = {https://www.microsoft.com/en-us/download/details.aspx?id=52367},
	}
	
	
	@misc{longversion,
	author = {N. Takbiri and A. Houmansadr and D.L. Goeckel and H. Pishro-Nik},
	title = {{Limits of Location Privacy under Anonymization and Obfuscation}},
	howpublished = "\url{http://www.ecs.umass.edu/ece/pishro/Papers/ISIT_2017-2.pdf}",
	year = 2017,
	month= "January",
	note = "Summarized version submitted to IEEE ISIT 2017"
	}
	
	@misc{isit_ke,
	author = {K. Li and D. Goeckel and H. Pishro-Nik},
	title = {{Bayesian Time Series Matching and Privacy}},
	note = "submitted to IEEE ISIT 2017"
	}
	
	@article{matching,
	title={Asymptotically Optimal Matching of Multiple Sequences to Source Distributions and Training Sequences},
	author={Jayakrishnan Unnikrishnan},
	journal={ IEEE Transactions on Information Theory},
	volume={61},
	number={1},
	pages={452-468},
	year={2015},
	publisher={IEEE}
	}
	
	
	@article{Naini2016,
	Author = {F. Naini and J. Unnikrishnan and P. Thiran and M. Vetterli},
	Journal = {IEEE Transactions on Information Forensics and Security},
	Publisher = {IEEE},
	Title = {Where You Are Is Who You Are: User Identification by Matching Statistics},
	volume={11},
	number={2},
	pages={358--372},
	Year = {2016}
	}
	
	
	
	@inproceedings{holowczak2015cachebrowser,
	title={{CacheBrowser: Bypassing Chinese Censorship without Proxies Using Cached Content}},
	author={Holowczak, John and Houmansadr, Amir},
	booktitle={Proceedings of the 22nd ACM SIGSAC Conference on Computer and Communications Security},
	pages={70--83},
	year={2015},
	organization={ACM}
	}
	@misc{cb-website,
	Howpublished = {\url{https://cachebrowser.net/#/}},
	Title = {{CacheBrowser}},
	key={cachebrowser}
	}
	
	@inproceedings{GameOfDecoys,
	title={{GAME OF DECOYS: Optimal Decoy Routing Through Game Theory}},
	author={Milad Nasr and Amir Houmansadr},
	booktitle={The $23^{rd}$ ACM Conference on Computer and Communications Security (CCS)},
	year={2016}
	}
	
	@inproceedings{CDNReaper,
	title={{Practical Censorship Evasion Leveraging Content Delivery Networks}},
	author={Hadi Zolfaghari and Amir Houmansadr},
	booktitle={The $23^{rd}$ ACM Conference on Computer and Communications Security (CCS)},
	year={2016}
	}
	
	@misc{Leberknight2010,
	Author = {Leberknight, C. and Chiang, M. and Poor, H. and Wong, F.},
	Howpublished = {\url{http://www.princeton.edu/~chiangm/anticensorship.pdf}},
	Title = {{A Taxonomy of Internet Censorship and Anti-censorship}},
	Year = {2010}}
	
	@techreport{ultrasurf-analysis,
	Author = {Appelbaum, Jacob},
	Institution = {The Tor Project},
	Title = {{Technical analysis of the Ultrasurf proxying software}},
	Url = {http://scholar.google.com/scholar?hl=en\&btnG=Search\&q=intitle:Technical+analysis+of+the+Ultrasurf+proxying+software\#0},
	Year = {2012},
	Bdsk-Url-1 = {http://scholar.google.com/scholar?hl=en%5C&btnG=Search%5C&q=intitle:Technical+analysis+of+the+Ultrasurf+proxying+software%5C#0}}
	
	@misc{gifc:07,
	Howpublished = {\url{http://www.internetfreedom.org/archive/Defeat\_Internet\_Censorship\_White\_Paper.pdf}},
	Key = {defeatcensorship},
	Publisher = {Global Internet Freedom Consortium (GIFC)},
	Title = {{Defeat Internet Censorship: Overview of Advanced Technologies and Products}},
	Type = {White Paper},
	Year = {2007}}
	
	@article{pan2011survey,
	Author = {Pan, J. and Paul, S. and Jain, R.},
	Journal = {Communications Magazine, IEEE},
	Number = {7},
	Pages = {26--36},
	Publisher = {IEEE},
	Title = {{A Survey of the Research on Future Internet Architectures}},
	Volume = {49},
	Year = {2011}}
	
	@misc{nsf-fia,
	Howpublished = {\url{http://www.nets-fia.net/}},
	Key = {FIA},
	Title = {{NSF Future Internet Architecture Project}}}
	
	@misc{NDN,
	Howpublished = {\url{http://www.named- data.net}},
	Key = {NDN},
	Title = {{Named Data Networking Project}}}
	
	@inproceedings{MobilityFirst,
	Author = {Seskar, I. and Nagaraja, K. and Nelson, S. and Raychaudhuri, D.},
	Booktitle = {Asian Internet Engineering Conference},
	Title = {{Mobilityfirst Future internet Architecture Project}},
	Year = {2011}}
	
	@incollection{NEBULA,
	Author = {Anderson, T. and Birman, K. and Broberg, R. and Caesar, M. and Comer, D. and Cotton, C. and Freedman, M.~J. and Haeberlen, A. and Ives, Z.~G. and Krishnamurthy, A. and others},
	Booktitle = {The Future Internet},
	Pages = {16--26},
	Publisher = {Springer},
	Title = {{The NEBULA Future Internet Architecture}},
	Year = {2013}}
	
	@inproceedings{XIA,
	Author = {Anand, A. and Dogar, F. and Han, D. and Li, B. and Lim, H. and Machado, M. and Wu, W. and Akella, A. and Andersen, D.~G. and Byers, J.~W. and others},
	Booktitle = {ACM Workshop on Hot Topics in Networks},
	Pages = {2},
	Title = {{XIA: An Architecture for an Evolvable and Trustworthy Internet}},
	Year = {2011}}
	
	@inproceedings{ChoiceNet,
	Author = {Rouskas, G.~N. and Baldine, I. and Calvert, K.~L. and Dutta, R. and Griffioen, J. and Nagurney, A. and Wolf, T.},
	Booktitle = {ONDM},
	Title = {{ChoiceNet: Network Innovation Through Choice}},
	Year = {2013}}
	
	@misc{nsf-find,
	Howpublished = {http://www.nets-find.net/},
	Title = {{NSF NeTS FIND Initiative}}}
	
	@article{traid,
	Author = {Cheriton, D.~R. and Gritter, M.},
	Title = {{TRIAD: A New Next-Generation Internet Architecture}},
	Year = {2000}}
	
	@inproceedings{dona,
	Author = {Koponen, T. and Chawla, M. and Chun, B-G. and Ermolinskiy, A. and Kim, K.~H. and Shenker, S. and Stoica, I.},
	Booktitle = {ACM SIGCOMM Computer Communication Review},
	Number = {4},
	Organization = {ACM},
	Pages = {181--192},
	Title = {{A Data-Oriented (and Beyond) Network Architecture}},
	Volume = {37},
	Year = {2007}}
	
	@misc{ultrasurf,
	Howpublished = {\url{http://www.ultrareach.com}},
	Key = {ultrasurf},
	Title = {{Ultrasurf}}}
	
	@misc{tor-bridge,
	Author = {Dingledine, R. and Mathewson, N.},
	Howpublished = {\url{https://svn.torproject.org/svn/projects/design-paper/blocking.html}},
	Title = {{Design of a Blocking-Resistant Anonymity System}}}
	
	@inproceedings{McLachlanH09,
	Author = {J. McLachlan and N. Hopper},
	Booktitle = {WPES},
	Title = {{On the Risks of Serving Whenever You Surf: Vulnerabilities in Tor's Blocking Resistance Design}},
	Year = {2009}}
	
	@inproceedings{mahdian2010,
	Author = {Mahdian, M.},
	Booktitle = {{Fun with Algorithms}},
	Title = {{Fighting Censorship with Algorithms}},
	Year = {2010}}
	
	@inproceedings{McCoy2011,
	Author = {McCoy, D. and Morales, J.~A. and Levchenko, K.},
	Booktitle = {FC},
	Title = {{Proximax: A Measurement Based System for Proxies Dissemination}},
	Year = {2011}}
	
	@inproceedings{Sovran2008,
	Author = {Sovran, Y. and Libonati, A. and Li, J.},
	Booktitle = {IPTPS},
	Title = {{Pass it on: Social Networks Stymie Censors}},
	Year = {2008}}
	
	@inproceedings{rbridge,
	Author = {Wang, Q. and Lin, Zi and Borisov, N. and Hopper, N.},
	Booktitle = {{NDSS}},
	Title = {{rBridge: User Reputation based Tor Bridge Distribution with Privacy Preservation}},
	Year = {2013}}
	
	@inproceedings{telex,
	Author = {Wustrow, E. and Wolchok, S. and Goldberg, I. and Halderman, J.},
	Booktitle = {{USENIX Security}},
	Title = {{Telex: Anticensorship in the Network Infrastructure}},
	Year = {2011}}
	
	@inproceedings{cirripede,
	Author = {Houmansadr, A. and Nguyen, G. and Caesar, M. and Borisov, N.},
	Booktitle = {CCS},
	Title = {{Cirripede: Circumvention Infrastructure Using Router Redirection with Plausible Deniability}},
	Year = {2011}}
	
	@inproceedings{decoyrouting,
	Author = {Karlin, J. and Ellard, D. and Jackson, A. and Jones, C. and Lauer, G. and Mankins, D. and Strayer, W.},
	Booktitle = {{FOCI}},
	Title = {{Decoy Routing: Toward Unblockable Internet Communication}},
	Year = {2011}}
	
	@inproceedings{routing-around-decoys,
	Author = {M.~Schuchard and J.~Geddes and C.~Thompson and N.~Hopper},
	Booktitle = {{CCS}},
	Title = {{Routing Around Decoys}},
	Year = {2012}}
	
	@inproceedings{parrot,
	Author = {A. Houmansadr and C. Brubaker and V. Shmatikov},
	Booktitle = {IEEE S\&P},
	Title = {{The Parrot is Dead: Observing Unobservable Network Communications}},
	Year = {2013}}
	
	@misc{knock,
	Author = {T. Wilde},
	Howpublished = {\url{https://blog.torproject.org/blog/knock-knock-knockin-bridges-doors}},
	Title = {{Knock Knock Knockin' on Bridges' Doors}},
	Year = {2012}}
	
	@inproceedings{china-tor,
	Author = {Winter, P. and Lindskog, S.},
	Booktitle = {{FOCI}},
	Title = {{How the Great Firewall of China Is Blocking Tor}},
	Year = {2012}}
	
	@misc{discover-bridge,
	Howpublished = {\url{https://blog.torproject.org/blog/research-problems-ten-ways-discover-tor-bridges}},
	Key = {tenways},
	Title = {{Ten Ways to Discover Tor Bridges}}}
	
	@inproceedings{freewave,
	Author = {A.~Houmansadr and T.~Riedl and N.~Borisov and A.~Singer},
	Booktitle = {{NDSS}},
	Title = {{I Want My Voice to Be Heard: IP over Voice-over-IP for Unobservable Censorship Circumvention}},
	Year = 2013}
	
	@inproceedings{censorspoofer,
	Author = {Q. Wang and X. Gong and G. Nguyen and A. Houmansadr and N. Borisov},
	Booktitle = {CCS},
	Title = {{CensorSpoofer: Asymmetric Communication Using IP Spoofing for Censorship-Resistant Web Browsing}},
	Year = {2012}}
	
	@inproceedings{skypemorph,
	Author = {H. Moghaddam and B. Li and M. Derakhshani and I. Goldberg},
	Booktitle = {CCS},
	Title = {{SkypeMorph: Protocol Obfuscation for Tor Bridges}},
	Year = {2012}}
	
	@inproceedings{stegotorus,
	Author = {Weinberg, Z. and Wang, J. and Yegneswaran, V. and Briesemeister, L. and Cheung, S. and Wang, F. and Boneh, D.},
	Booktitle = {CCS},
	Title = {{StegoTorus: A Camouflage Proxy for the Tor Anonymity System}},
	Year = {2012}}
	
	@techreport{dust,
	Author = {{B.~Wiley}},
	Howpublished = {\url{http://blanu.net/ Dust.pdf}},
	Institution = {School of Information, University of Texas at Austin},
	Title = {{Dust: A Blocking-Resistant Internet Transport Protocol}},
	Year = {2011}}
	
	@inproceedings{FTE,
	Author = {K.~Dyer and S.~Coull and T.~Ristenpart and T.~Shrimpton},
	Booktitle = {CCS},
	Title = {{Protocol Misidentification Made Easy with Format-Transforming Encryption}},
	Year = {2013}}
	
	@inproceedings{fp,
	Author = {Fifield, D. and Hardison, N. and Ellithrope, J. and Stark, E. and Dingledine, R. and Boneh, D. and Porras, P.},
	Booktitle = {PETS},
	Title = {{Evading Censorship with Browser-Based Proxies}},
	Year = {2012}}
	
	@misc{obfsproxy,
	Howpublished = {\url{https://www.torproject.org/projects/obfsproxy.html.en}},
	Key = {obfsproxy},
	Publisher = {The Tor Project},
	Title = {{A Simple Obfuscating Proxy}}}
	
	@inproceedings{Tor-instead-of-IP,
	Author = {Liu, V. and Han, S. and Krishnamurthy, A. and Anderson, T.},
	Booktitle = {HotNets},
	Title = {{Tor instead of IP}},
	Year = {2011}}
	
	@misc{roger-slides,
	Howpublished = {\url{https://svn.torproject.org/svn/projects/presentations/slides-28c3.pdf}},
	Key = {torblocking},
	Title = {{How Governments Have Tried to Block Tor}}}
	
	@inproceedings{infranet,
	Author = {Feamster, N. and Balazinska, M. and Harfst, G. and Balakrishnan, H. and Karger, D.},
	Booktitle = {USENIX Security},
	Title = {{Infranet: Circumventing Web Censorship and Surveillance}},
	Year = {2002}}
	
	@inproceedings{collage,
	Author = {S.~Burnett and N.~Feamster and S.~Vempala},
	Booktitle = {USENIX Security},
	Title = {{Chipping Away at Censorship Firewalls with User-Generated Content}},
	Year = {2010}}
	
	@article{anonymizer,
	Author = {Boyan, J.},
	Journal = {Computer-Mediated Communication Magazine},
	Month = sep,
	Number = {9},
	Title = {{The Anonymizer: Protecting User Privacy on the Web}},
	Volume = {4},
	Year = {1997}}
	
	@article{schulze2009internet,
	Author = {Schulze, H. and Mochalski, K.},
	Journal = {IPOQUE Report},
	Pages = {351--362},
	Title = {Internet Study 2008/2009},
	Volume = {37},
	Year = {2009}}
	
	@inproceedings{cya-ccs13,
	Author = {J.~Geddes and M.~Schuchard and N.~Hopper},
	Booktitle = {{CCS}},
	Title = {{Cover Your ACKs: Pitfalls of Covert Channel Censorship Circumvention}},
	Year = {2013}}
	
	@inproceedings{andana,
	Author = {DiBenedetto, S. and Gasti, P. and Tsudik, G. and Uzun, E.},
	Booktitle = {{NDSS}},
	Title = {{ANDaNA: Anonymous Named Data Networking Application}},
	Year = {2012}}
	
	@inproceedings{darkly,
	Author = {Jana, S. and Narayanan, A. and Shmatikov, V.},
	Booktitle = {IEEE S\&P},
	Title = {{A Scanner Darkly: Protecting User Privacy From Perceptual Applications}},
	Year = {2013}}
	
	@inproceedings{NS08,
	Author = {A.~Narayanan and V.~Shmatikov},
	Booktitle = {IEEE S\&P},
	Title = {Robust de-anonymization of large sparse datasets},
	Year = {2008}}
	
	@inproceedings{NS09,
	Author = {Arvind Narayanan and Vitaly Shmatikov},
	Booktitle = {IEEE S\&P},
	Title = {De-anonymizing Social Networks},
	Year = {2009}}
	
	@inproceedings{memento,
	Author = {Jana, S. and Shmatikov, V.},
	Booktitle = {IEEE S\&P},
	Title = {{Memento: Learning secrets from process footprints}},
	Year = {2012}}
	
	@misc{plugtor,
	Howpublished = {\url{https://www.torproject.org/docs/pluggable-transports.html.en}},
	Key = {PluggableTransports},
	Publisher = {The Tor Project},
	Title = {{Tor: Pluggable transports}}}
	
	@misc{psiphon,
	Author = {J.~Jia and P.~Smith},
	Howpublished = {\url{http://www.cdf.toronto.edu/~csc494h/reports/2004-fall/psiphon_ae.html}},
	Title = {{Psiphon: Analysis and Estimation}},
	Year = 2004}
	
	@misc{china-github,
	Howpublished = {\url{http://mobile.informationweek.com/80269/show/72e30386728f45f56b343ddfd0fdb119/}},
	Key = {github},
	Title = {{China's GitHub Censorship Dilemma}}}
	
	@inproceedings{txbox,
	Author = {Jana, S. and Porter, D. and Shmatikov, V.},
	Booktitle = {IEEE S\&P},
	Title = {{TxBox: Building Secure, Efficient Sandboxes with System Transactions}},
	Year = {2011}}
	
	@inproceedings{airavat,
	Author = {I. Roy and S. Setty and A. Kilzer and V. Shmatikov and E. Witchel},
	Booktitle = {NSDI},
	Title = {{Airavat: Security and Privacy for MapReduce}},
	Year = {2010}}
	
	@inproceedings{osdi12,
	Author = {A. Dunn and M. Lee and S. Jana and S. Kim and M. Silberstein and Y. Xu and V. Shmatikov and E. Witchel},
	Booktitle = {OSDI},
	Title = {{Eternal Sunshine of the Spotless Machine: Protecting Privacy with Ephemeral Channels}},
	Year = {2012}}
	
	@inproceedings{ymal,
	Author = {J. Calandrino and A. Kilzer and A. Narayanan and E. Felten and V. Shmatikov},
	Booktitle = {IEEE S\&P},
	Title = {{``You Might Also Like:'' Privacy Risks of Collaborative Filtering}},
	Year = {2011}}
	
	@inproceedings{srivastava11,
	Author = {V. Srivastava and M. Bond and K. McKinley and V. Shmatikov},
	Booktitle = {PLDI},
	Title = {{A Security Policy Oracle: Detecting Security Holes Using Multiple API Implementations}},
	Year = {2011}}
	
	@inproceedings{chen-oakland10,
	Author = {Chen, S. and Wang, R. and Wang, X. and Zhang, K.},
	Booktitle = {IEEE S\&P},
	Title = {{Side-Channel Leaks in Web Applications: A Reality Today, a Challenge Tomorrow}},
	Year = {2010}}
	
	@book{kerck,
	Author = {Kerckhoffs, A.},
	Publisher = {University Microfilms},
	Title = {{La cryptographie militaire}},
	Year = {1978}}
	
	@inproceedings{foci11,
	Author = {J. Karlin and D. Ellard and A.~Jackson and C.~ Jones and G. Lauer and D. Mankins and W.~T.~Strayer},
	Booktitle = {FOCI},
	Title = {{Decoy Routing: Toward Unblockable Internet Communication}},
	Year = 2011}
	
	@inproceedings{sun02,
	Author = {Sun, Q. and Simon, D.~R. and Wang, Y. and Russell, W. and Padmanabhan, V. and Qiu, L.},
	Booktitle = {IEEE S\&P},
	Title = {{Statistical Identification of Encrypted Web Browsing Traffic}},
	Year = {2002}}
	
	@inproceedings{danezis,
	Author = {Murdoch, S.~J. and Danezis, G.},
	Booktitle = {IEEE S\&P},
	Title = {{Low-Cost Traffic Analysis of Tor}},
	Year = {2005}}
	
	@inproceedings{pakicensorship,
	Author = {Z.~Nabi},
	Booktitle = {FOCI},
	Title = {The Anatomy of {Web} Censorship in {Pakistan}},
	Year = {2013}}
	
	@inproceedings{irancensorship,
	Author = {S.~Aryan and H.~Aryan and A.~Halderman},
	Booktitle = {FOCI},
	Title = {Internet Censorship in {Iran}: {A} First Look},
	Year = {2013}}
	
	@inproceedings{ford10efficient,
	Author = {Amittai Aviram and Shu-Chun Weng and Sen Hu and Bryan Ford},
	Booktitle = {\bibconf[9th]{OSDI}{USENIX Symposium on Operating Systems Design and Implementation}},
	Location = {Vancouver, BC, Canada},
	Month = oct,
	Title = {Efficient System-Enforced Deterministic Parallelism},
	Year = 2010}
	
	@inproceedings{ford10determinating,
	Author = {Amittai Aviram and Sen Hu and Bryan Ford and Ramakrishna Gummadi},
	Booktitle = {\bibconf{CCSW}{ACM Cloud Computing Security Workshop}},
	Location = {Chicago, IL},
	Month = oct,
	Title = {Determinating Timing Channels in Compute Clouds},
	Year = 2010}
	
	@inproceedings{ford12plugging,
	Author = {Bryan Ford},
	Booktitle = {\bibconf[4th]{HotCloud}{USENIX Workshop on Hot Topics in Cloud Computing}},
	Location = {Boston, MA},
	Month = jun,
	Title = {Plugging Side-Channel Leaks with Timing Information Flow Control},
	Year = 2012}
	
	@inproceedings{ford12icebergs,
	Author = {Bryan Ford},
	Booktitle = {\bibconf[4th]{HotCloud}{USENIX Workshop on Hot Topics in Cloud Computing}},
	Location = {Boston, MA},
	Month = jun,
	Title = {Icebergs in the Clouds: the {\em Other} Risks of Cloud Computing},
	Year = 2012}
	
	@misc{mullenize,
	Author = {Washington Post},
	Howpublished = {\url{http://apps.washingtonpost.com/g/page/world/gchq-report-on-mullenize-program-to-stain-anonymous-electronic-traffic/502/}},
	Month = {oct},
	Title = {{GCHQ} report on {`MULLENIZE'} program to `stain' anonymous electronic traffic},
	Year = {2013}}
	
	@inproceedings{shue13street,
	Author = {Craig A. Shue and Nathanael Paul and Curtis R. Taylor},
	Booktitle = {\bibbrev[7th]{WOOT}{USENIX Workshop on Offensive Technologies}},
	Month = aug,
	Title = {From an {IP} Address to a Street Address: Using Wireless Signals to Locate a Target},
	Year = 2013}
	
	@inproceedings{knockel11three,
	Author = {Jeffrey Knockel and Jedidiah R. Crandall and Jared Saia},
	Booktitle = {\bibbrev{FOCI}{USENIX Workshop on Free and Open Communications on the Internet}},
	Location = {San Francisco, CA},
	Month = aug,
	Year = 2011}
	
	@misc{rfc4960,
	Author = {R. {Stewart, ed.}},
	Month = sep,
	Note = {RFC 4960},
	Title = {Stream Control Transmission Protocol},
	Year = 2007}
	
	@inproceedings{ford07structured,
	Author = {Bryan Ford},
	Booktitle = {\bibbrev{SIGCOMM}{ACM SIGCOMM}},
	Location = {Kyoto, Japan},
	Month = aug,
	Title = {Structured Streams: a New Transport Abstraction},
	Year = {2007}}
	
	@misc{spdy,
	Author = {Google, Inc.},
	Note = {\url{http://www.chromium.org/spdy/spdy-whitepaper}},
	Title = {{SPDY}: An Experimental Protocol For a Faster {Web}}}
	
	@misc{quic,
	Author = {Jim Roskind},
	Month = jun,
	Note = {\url{http://blog.chromium.org/2013/06/experimenting-with-quic.html}},
	Title = {Experimenting with {QUIC}},
	Year = 2013}
	
	@misc{podjarny12not,
	Author = {G.~Podjarny},
	Month = jun,
	Note = {\url{http://www.guypo.com/technical/not-as-spdy-as-you-thought/}},
	Title = {{Not as SPDY as You Thought}},
	Year = 2012}
	
	@inproceedings{cor,
	Author = {Jones, N.~A. and Arye, M. and Cesareo, J. and Freedman, M.~J.},
	Booktitle = {FOCI},
	Title = {{Hiding Amongst the Clouds: A Proposal for Cloud-based Onion Routing}},
	Year = {2011}}
	
	@misc{torcloud,
	Howpublished = {\url{https://cloud.torproject.org/}},
	Key = {tor cloud},
	Title = {{The Tor Cloud Project}}}
	
	@inproceedings{scramblesuit,
	Author = {Philipp Winter and Tobias Pulls and Juergen Fuss},
	Booktitle = {WPES},
	Title = {{ScrambleSuit: A Polymorphic Network Protocol to Circumvent Censorship}},
	Year = 2013}
	
	@article{savage2000practical,
	Author = {Savage, S. and Wetherall, D. and Karlin, A. and Anderson, T.},
	Journal = {ACM SIGCOMM Computer Communication Review},
	Number = {4},
	Pages = {295--306},
	Publisher = {ACM},
	Title = {Practical network support for IP traceback},
	Volume = {30},
	Year = {2000}}
	
	@inproceedings{ooni,
	Author = {Filast, A. and Appelbaum, J.},
	Booktitle = {{FOCI}},
	Title = {{OONI : Open Observatory of Network Interference}},
	Year = {2012}}
	
	@misc{caida-rank,
	Howpublished = {\url{http://as-rank.caida.org/}},
	Key = {caida rank},
	Title = {{AS Rank: AS Ranking}}}
	
	@inproceedings{usersrouted-ccs13,
	Author = {A.~Johnson and C.~Wacek and R.~Jansen and M.~Sherr and P.~Syverson},
	Booktitle = {CCS},
	Title = {{Users Get Routed: Traffic Correlation on Tor by Realistic Adversaries}},
	Year = {2013}}
	
	@inproceedings{edman2009awareness,
	Author = {Edman, M. and Syverson, P.},
	Booktitle = {{CCS}},
	Title = {{AS-awareness in Tor path selection}},
	Year = {2009}}
	
	@inproceedings{DecoyCosts,
	Author = {A.~Houmansadr and E.~L.~Wong and V.~Shmatikov},
	Booktitle = {NDSS},
	Title = {{No Direction Home: The True Cost of Routing Around Decoys}},
	Year = {2014}}
	
	@article{cordon,
	Author = {Elahi, T. and Goldberg, I.},
	Journal = {University of Waterloo CACR},
	Title = {{CORDON--A Taxonomy of Internet Censorship Resistance Strategies}},
	Volume = {33},
	Year = {2012}}
	
	@inproceedings{privex,
	Author = {T.~Elahi and G.~Danezis and I.~Goldberg	},
	Booktitle = {{CCS}},
	Title = {{AS-awareness in Tor path selection}},
	Year = {2014}}
	
	@inproceedings{changeGuards,
	Author = {T.~Elahi and K.~Bauer and M.~AlSabah and R.~Dingledine and I.~Goldberg},
	Booktitle = {{WPES}},
	Title = {{ Changing of the Guards: Framework for Understanding and Improving Entry Guard Selection in Tor}},
	Year = {2012}}
	
	@article{RAINBOW:Journal,
	Author = {A.~Houmansadr and N.~Kiyavash and N.~Borisov},
	Journal = {IEEE/ACM Transactions on Networking},
	Title = {{Non-Blind Watermarking of Network Flows}},
	Year = 2014}
	
	@inproceedings{info-tod,
	Author = {A.~Houmansadr and S.~Gorantla and T.~Coleman and N.~Kiyavash and and N.~Borisov},
	Booktitle = {{CCS (poster session)}},
	Title = {{On the Channel Capacity of Network Flow Watermarking}},
	Year = {2009}}
	
	@inproceedings{johnson2014game,
	Author = {Johnson, B. and Laszka, A. and Grossklags, J. and Vasek, M. and Moore, T.},
	Booktitle = {Workshop on Bitcoin Research},
	Title = {{Game-theoretic Analysis of DDoS Attacks Against Bitcoin Mining Pools}},
	Year = {2014}}
	
	@incollection{laszka2013mitigation,
	Author = {Laszka, A. and Johnson, B. and Grossklags, J.},
	Booktitle = {Decision and Game Theory for Security},
	Pages = {175--191},
	Publisher = {Springer},
	Title = {{Mitigation of Targeted and Non-targeted Covert Attacks as a Timing Game}},
	Year = {2013}}
	
	@inproceedings{schottle2013game,
	Author = {Schottle, P. and Laszka, A. and Johnson, B. and Grossklags, J. and Bohme, R.},
	Booktitle = {EUSIPCO},
	Title = {{A Game-theoretic Analysis of Content-adaptive Steganography with Independent Embedding}},
	Year = {2013}}
	
	@inproceedings{CloudTransport,
	Author = {C.~Brubaker and A.~Houmansadr and V.~Shmatikov},
	Booktitle = {PETS},
	Title = {{CloudTransport: Using Cloud Storage for Censorship-Resistant Networking}},
	Year = {2014}}
	
	@inproceedings{sweet,
	Author = {W.~Zhou and A.~Houmansadr and M.~Caesar and N.~Borisov},
	Booktitle = {HotPETs},
	Title = {{SWEET: Serving the Web by Exploiting Email Tunnels}},
	Year = {2013}}
	
	@inproceedings{ahsan2002practical,
	Author = {Ahsan, K. and Kundur, D.},
	Booktitle = {Workshop on Multimedia Security},
	Title = {{Practical data hiding in TCP/IP}},
	Year = {2002}}
	
	@incollection{danezis2011covert,
	Author = {Danezis, G.},
	Booktitle = {Security Protocols XVI},
	Pages = {198--214},
	Publisher = {Springer},
	Title = {{Covert Communications Despite Traffic Data Retention}},
	Year = {2011}}
	
	@inproceedings{liu2009hide,
	Author = {Liu, Y. and Ghosal, D. and Armknecht, F. and Sadeghi, A.-R. and Schulz, S. and Katzenbeisser, S.},
	Booktitle = {ESORICS},
	Title = {{Hide and Seek in Time---Robust Covert Timing Channels}},
	Year = {2009}}
	
	@misc{image-watermark-fing,
	Author = {Jonathan Bailey},
	Howpublished = {\url{https://www.plagiarismtoday.com/2009/12/02/image-detection-watermarking-vs-fingerprinting/}},
	Title = {{Image Detection: Watermarking vs. Fingerprinting}},
	Year = {2009}}
	
	@inproceedings{Servetto98,
	Author = {S. D. Servetto and C. I. Podilchuk and K. Ramchandran},
	Booktitle = {Int. Conf. Image Processing},
	Title = {Capacity issues in digital image watermarking},
	Year = {1998}}
	
	@inproceedings{Chen01,
	Author = {B. Chen and G.W.Wornell},
	Booktitle = {IEEE Trans. Inform. Theory},
	Pages = {1423--1443},
	Title = {Quantization index modulation: A class of provably good methods for digital watermarking and information embedding},
	Year = {2001}}
	
	@inproceedings{Karakos00,
	Author = {D. Karakos and A. Papamarcou},
	Booktitle = {IEEE Int. Symp. Information Theory},
	Pages = {47},
	Title = {Relationship between quantization and distribution rates of digitally watermarked data},
	Year = {2000}}
	
	@inproceedings{Sullivan98,
	Author = {J. A. OSullivan and P. Moulin and J. M. Ettinger},
	Booktitle = {IEEE Int. Symp. Information Theory},
	Pages = {297},
	Title = {Information theoretic analysis of steganography},
	Year = {1998}}
	
	@inproceedings{Merhav00,
	Author = {N. Merhav},
	Booktitle = {IEEE Trans. Inform. Theory},
	Pages = {420--430},
	Title = {On random coding error exponents of watermarking systems},
	Year = {2000}}
	
	@inproceedings{Somekh01,
	Author = {A. Somekh-Baruch and N. Merhav},
	Booktitle = {IEEE Int. Symp. Information Theory},
	Pages = {7},
	Title = {On the error exponent and capacity games of private watermarking systems},
	Year = {2001}}
	
	@inproceedings{Steinberg01,
	Author = {Y. Steinberg and N. Merhav},
	Booktitle = {IEEE Trans. Inform. Theory},
	Pages = {1410--1422},
	Title = {Identification in the presence of side information with application to watermarking},
	Year = {2001}}
	
	@article{Moulin03,
	Author = {P. Moulin and J.A. O'Sullivan},
	Journal = {IEEE Trans. Info. Theory},
	Number = {3},
	Title = {Information-theoretic analysis of information hiding},
	Volume = 49,
	Year = 2003}
	
	@article{Gelfand80,
	Author = {S.I.~Gelfand and M.S.~Pinsker},
	Journal = {Problems of Control and Information Theory},
	Number = {1},
	Pages = {19-31},
	Title = {{Coding for channel with random parameters}},
	Url = {citeseer.ist.psu.edu/anantharam96bits.html},
	Volume = {9},
	Year = {1980},
	Bdsk-Url-1 = {citeseer.ist.psu.edu/anantharam96bits.html}}
	
	@book{Wolfowitz78,
	Author = {J. Wolfowitz},
	Edition = {3rd},
	Location = {New York},
	Publisher = {Springer-Verlag},
	Title = {Coding Theorems of Information Theory},
	Year = 1978}
	
	@article{caire99,
	Author = {G. Caire and S. Shamai},
	Journal = {IEEE Transactions on Information Theory},
	Number = {6},
	Pages = {2007--2019},
	Title = {On the Capacity of Some Channels with Channel State Information},
	Volume = {45},
	Year = {1999}}
	
	@inproceedings{wright2007language,
	Author = {Wright, Charles V and Ballard, Lucas and Monrose, Fabian and Masson, Gerald M},
	Booktitle = {USENIX Security},
	Title = {{Language identification of encrypted VoIP traffic: Alejandra y Roberto or Alice and Bob?}},
	Year = {2007}}
	
	@inproceedings{backes2010speaker,
	Author = {Backes, Michael and Doychev, Goran and D{\"u}rmuth, Markus and K{\"o}pf, Boris},
	Booktitle = {{European Symposium on Research in Computer Security (ESORICS)}},
	Pages = {508--523},
	Publisher = {Springer},
	Title = {{Speaker Recognition in Encrypted Voice Streams}},
	Year = {2010}}
	
	@phdthesis{lu2009traffic,
	Author = {Lu, Yuanchao},
	School = {Cleveland State University},
	Title = {{On Traffic Analysis Attacks to Encrypted VoIP Calls}},
	Year = {2009}}
	
	@inproceedings{wright2008spot,
	Author = {Wright, Charles V and Ballard, Lucas and Coull, Scott E and Monrose, Fabian and Masson, Gerald M},
	Booktitle = {IEEE Symposium on Security and Privacy},
	Pages = {35--49},
	Title = {Spot me if you can: Uncovering spoken phrases in encrypted VoIP conversations},
	Year = {2008}}
	
	@inproceedings{white2011phonotactic,
	Author = {White, Andrew M and Matthews, Austin R and Snow, Kevin Z and Monrose, Fabian},
	Booktitle = {IEEE Symposium on Security and Privacy},
	Pages = {3--18},
	Title = {Phonotactic reconstruction of encrypted VoIP conversations: Hookt on fon-iks},
	Year = {2011}}
	
	@inproceedings{fancy,
	Author = {Houmansadr, Amir and Borisov, Nikita},
	Booktitle = {Privacy Enhancing Technologies},
	Organization = {Springer},
	Pages = {205--224},
	Title = {The Need for Flow Fingerprints to Link Correlated Network Flows},
	Year = {2013}}
	
	@article{botmosaic,
	Author = {Amir Houmansadr and Nikita Borisov},
	Doi = {10.1016/j.jss.2012.11.005},
	Issn = {0164-1212},
	Journal = {Journal of Systems and Software},
	Keywords = {Network security},
	Number = {3},
	Pages = {707 - 715},
	Title = {BotMosaic: Collaborative network watermark for the detection of IRC-based botnets},
	Url = {http://www.sciencedirect.com/science/article/pii/S0164121212003068},
	Volume = {86},
	Year = {2013},
	Bdsk-Url-1 = {http://www.sciencedirect.com/science/article/pii/S0164121212003068},
	Bdsk-Url-2 = {http://dx.doi.org/10.1016/j.jss.2012.11.005}}
	
	@inproceedings{ramsbrock2008first,
	Author = {Ramsbrock, Daniel and Wang, Xinyuan and Jiang, Xuxian},
	Booktitle = {Recent Advances in Intrusion Detection},
	Organization = {Springer},
	Pages = {59--77},
	Title = {A first step towards live botmaster traceback},
	Year = {2008}}
	
	@inproceedings{potdar2005survey,
	Author = {Potdar, Vidyasagar M and Han, Song and Chang, Elizabeth},
	Booktitle = {Industrial Informatics, 2005. INDIN'05. 2005 3rd IEEE International Conference on},
	Organization = {IEEE},
	Pages = {709--716},
	Title = {A survey of digital image watermarking techniques},
	Year = {2005}}
	
	@book{cole2003hiding,
	Author = {Cole, Eric and Krutz, Ronald D},
	Publisher = {John Wiley \& Sons, Inc.},
	Title = {Hiding in plain sight: Steganography and the art of covert communication},
	Year = {2003}}
	
	@incollection{akaike1998information,
	Author = {Akaike, Hirotogu},
	Booktitle = {Selected Papers of Hirotugu Akaike},
	Pages = {199--213},
	Publisher = {Springer},
	Title = {Information theory and an extension of the maximum likelihood principle},
	Year = {1998}}
	
	@misc{central-command-hack,
	Author = {Everett Rosenfeld},
	Howpublished = {\url{http://www.cnbc.com/id/102330338}},
	Title = {{FBI investigating Central Command Twitter hack}},
	Year = {2015}}
	
	@misc{sony-psp-ddos,
	Howpublished = {\url{http://n4g.com/news/1644853/sony-and-microsoft-cant-do-much-ddos-attacks-explained}},
	Key = {sony},
	Month = {December},
	Title = {{Sony and Microsoft cant do much -- DDoS attacks explained}},
	Year = {2014}}
	
	@misc{sony-hack,
	Author = {David Bloom},
	Howpublished = {\url{http://goo.gl/MwR4A7}},
	Title = {{Online Game Networks Hacked, Sony Unit President Threatened}},
	Year = {2014}}
	
	@misc{home-depot,
	Author = {Dune Lawrence},
	Howpublished = {\url{http://www.businessweek.com/articles/2014-09-02/home-depots-credit-card-breach-looks-just-like-the-target-hack}},
	Title = {{Home Depot's Suspected Breach Looks Just Like the Target Hack}},
	Year = {2014}}
	
	@misc{target,
	Author = {Julio Ojeda-Zapata},
	Howpublished = {\url{http://www.mercurynews.com/business/ci_24765398/how-did-hackers-pull-off-target-data-heist}},
	Title = {{Target hack: How did they do it?}},
	Year = {2014}}
	
	
	@article{probabilitycourse,
	Author = {H. Pishro-Nik},
	note = {\url{http://www.probabilitycourse.com}},
	Title = {Introduction to probability, statistics, and random processes},
	Year = {2014}}
	
	
	
	@inproceedings{shokri2011quantifying,
	Author = {Shokri, Reza and Theodorakopoulos, George and Le Boudec, Jean-Yves and Hubaux, Jean-Pierre},
	Booktitle = {Security and Privacy (SP), 2011 IEEE Symposium on},
	Organization = {IEEE},
	Pages = {247--262},
	Title = {Quantifying location privacy},
	Year = {2011}}
	
	@inproceedings{hoh2007preserving,
	Author = {Hoh, Baik and Gruteser, Marco and Xiong, Hui and Alrabady, Ansaf},
	Booktitle = {Proceedings of the 14th ACM conference on Computer and communications security},
	Organization = {ACM},
	Pages = {161--171},
	Title = {Preserving privacy in gps traces via uncertainty-aware path cloaking},
	Year = {2007}}
	
	
	
	@article{kafsi2013entropy,
	Author = {Kafsi, Mohamed and Grossglauser, Matthias and Thiran, Patrick},
	Journal = {Information Theory, IEEE Transactions on},
	Number = {9},
	Pages = {5577--5583},
	Publisher = {IEEE},
	Title = {The entropy of conditional Markov trajectories},
	Volume = {59},
	Year = {2013}}
	
	@inproceedings{gruteser2003anonymous,
	Author = {Gruteser, Marco and Grunwald, Dirk},
	Booktitle = {Proceedings of the 1st international conference on Mobile systems, applications and services},
	Organization = {ACM},
	Pages = {31--42},
	Title = {Anonymous usage of location-based services through spatial and temporal cloaking},
	Year = {2003}}
	
	@inproceedings{husted2010mobile,
	Author = {Husted, Nathaniel and Myers, Steven},
	Booktitle = {Proceedings of the 17th ACM conference on Computer and communications security},
	Organization = {ACM},
	Pages = {85--96},
	Title = {Mobile location tracking in metro areas: malnets and others},
	Year = {2010}}
	
	@inproceedings{li2009tradeoff,
	Author = {Li, Tiancheng and Li, Ninghui},
	Booktitle = {Proceedings of the 15th ACM SIGKDD international conference on Knowledge discovery and data mining},
	Organization = {ACM},
	Pages = {517--526},
	Title = {On the tradeoff between privacy and utility in data publishing},
	Year = {2009}}
	
	@inproceedings{ma2009location,
	Author = {Ma, Zhendong and Kargl, Frank and Weber, Michael},
	Booktitle = {Sarnoff Symposium, 2009. SARNOFF'09. IEEE},
	Organization = {IEEE},
	Pages = {1--6},
	Title = {A location privacy metric for v2x communication systems},
	Year = {2009}}
	
	@inproceedings{shokri2012protecting,
	Author = {Shokri, Reza and Theodorakopoulos, George and Troncoso, Carmela and Hubaux, Jean-Pierre and Le Boudec, Jean-Yves},
	Booktitle = {Proceedings of the 2012 ACM conference on Computer and communications security},
	Organization = {ACM},
	Pages = {617--627},
	Title = {Protecting location privacy: optimal strategy against localization attacks},
	Year = {2012}}
	
	@inproceedings{freudiger2009non,
	Author = {Freudiger, Julien and Manshaei, Mohammad Hossein and Hubaux, Jean-Pierre and Parkes, David C},
	Booktitle = {Proceedings of the 16th ACM conference on Computer and communications security},
	Organization = {ACM},
	Pages = {324--337},
	Title = {On non-cooperative location privacy: a game-theoretic analysis},
	Year = {2009}}
	
	@incollection{humbert2010tracking,
	Author = {Humbert, Mathias and Manshaei, Mohammad Hossein and Freudiger, Julien and Hubaux, Jean-Pierre},
	Booktitle = {Decision and Game Theory for Security},
	Pages = {38--57},
	Publisher = {Springer},
	Title = {Tracking games in mobile networks},
	Year = {2010}}
	
	@article{manshaei2013game,
	Author = {Manshaei, Mohammad Hossein and Zhu, Quanyan and Alpcan, Tansu and Bac{\c{s}}ar, Tamer and Hubaux, Jean-Pierre},
	Journal = {ACM Computing Surveys (CSUR)},
	Number = {3},
	Pages = {25},
	Publisher = {ACM},
	Title = {Game theory meets network security and privacy},
	Volume = {45},
	Year = {2013}}
	
	@article{palamidessi2006probabilistic,
	Author = {Palamidessi, Catuscia},
	Journal = {Electronic Notes in Theoretical Computer Science},
	Pages = {33--42},
	Publisher = {Elsevier},
	Title = {Probabilistic and nondeterministic aspects of anonymity},
	Volume = {155},
	Year = {2006}}
	
	@inproceedings{mokbel2006new,
	Author = {Mokbel, Mohamed F and Chow, Chi-Yin and Aref, Walid G},
	Booktitle = {Proceedings of the 32nd international conference on Very large data bases},
	Organization = {VLDB Endowment},
	Pages = {763--774},
	Title = {The new Casper: query processing for location services without compromising privacy},
	Year = {2006}}
	
	@article{kalnis2007preventing,
	Author = {Kalnis, Panos and Ghinita, Gabriel and Mouratidis, Kyriakos and Papadias, Dimitris},
	Journal = {Knowledge and Data Engineering, IEEE Transactions on},
	Number = {12},
	Pages = {1719--1733},
	Publisher = {IEEE},
	Title = {Preventing location-based identity inference in anonymous spatial queries},
	Volume = {19},
	Year = {2007}}
	
	@article{freudiger2007mix,
	title={Mix-zones for location privacy in vehicular networks},
	author={Freudiger, Julien and Raya, Maxim and F{\'e}legyh{\'a}zi, M{\'a}rk and Papadimitratos, Panos and Hubaux, Jean-Pierre},
	year={2007}
	}
	@article{sweeney2002k,
	Author = {Sweeney, Latanya},
	Journal = {International Journal of Uncertainty, Fuzziness and Knowledge-Based Systems},
	Number = {05},
	Pages = {557--570},
	Publisher = {World Scientific},
	Title = {k-anonymity: A model for protecting privacy},
	Volume = {10},
	Year = {2002}}
	
	@article{sweeney2002achieving,
	Author = {Sweeney, Latanya},
	Journal = {International Journal of Uncertainty, Fuzziness and Knowledge-Based Systems},
	Number = {05},
	Pages = {571--588},
	Publisher = {World Scientific},
	Title = {Achieving k-anonymity privacy protection using generalization and suppression},
	Volume = {10},
	Year = {2002}}
	
	@inproceedings{niu2014achieving,
	Author = {Niu, Ben and Li, Qinghua and Zhu, Xiaoyan and Cao, Guohong and Li, Hui},
	Booktitle = {INFOCOM, 2014 Proceedings IEEE},
	Organization = {IEEE},
	Pages = {754--762},
	Title = {Achieving k-anonymity in privacy-aware location-based services},
	Year = {2014}}
	
	@inproceedings{liu2013game,
	Author = {Liu, Xinxin and Liu, Kaikai and Guo, Linke and Li, Xiaolin and Fang, Yuguang},
	Booktitle = {INFOCOM, 2013 Proceedings IEEE},
	Organization = {IEEE},
	Pages = {2985--2993},
	Title = {A game-theoretic approach for achieving k-anonymity in location based services},
	Year = {2013}}
	
	@inproceedings{kido2005protection,
	Author = {Kido, Hidetoshi and Yanagisawa, Yutaka and Satoh, Tetsuji},
	Booktitle = {Data Engineering Workshops, 2005. 21st International Conference on},
	Organization = {IEEE},
	Pages = {1248--1248},
	Title = {Protection of location privacy using dummies for location-based services},
	Year = {2005}}
	
	@inproceedings{gedik2005location,
	Author = {Gedik, Bu{\u{g}}ra and Liu, Ling},
	Booktitle = {Distributed Computing Systems, 2005. ICDCS 2005. Proceedings. 25th IEEE International Conference on},
	Organization = {IEEE},
	Pages = {620--629},
	Title = {Location privacy in mobile systems: A personalized anonymization model},
	Year = {2005}}
	
	@inproceedings{bordenabe2014optimal,
	Author = {Bordenabe, Nicol{\'a}s E and Chatzikokolakis, Konstantinos and Palamidessi, Catuscia},
	Booktitle = {Proceedings of the 2014 ACM SIGSAC Conference on Computer and Communications Security},
	Organization = {ACM},
	Pages = {251--262},
	Title = {Optimal geo-indistinguishable mechanisms for location privacy},
	Year = {2014}}
	
	@incollection{duckham2005formal,
	Author = {Duckham, Matt and Kulik, Lars},
	Booktitle = {Pervasive computing},
	Pages = {152--170},
	Publisher = {Springer},
	Title = {A formal model of obfuscation and negotiation for location privacy},
	Year = {2005}}
	
	@inproceedings{kido2005anonymous,
	Author = {Kido, Hidetoshi and Yanagisawa, Yutaka and Satoh, Tetsuji},
	Booktitle = {Pervasive Services, 2005. ICPS'05. Proceedings. International Conference on},
	Organization = {IEEE},
	Pages = {88--97},
	Title = {An anonymous communication technique using dummies for location-based services},
	Year = {2005}}
	
	@incollection{duckham2006spatiotemporal,
	Author = {Duckham, Matt and Kulik, Lars and Birtley, Athol},
	Booktitle = {Geographic Information Science},
	Pages = {47--64},
	Publisher = {Springer},
	Title = {A spatiotemporal model of strategies and counter strategies for location privacy protection},
	Year = {2006}}
	
	@inproceedings{shankar2009privately,
	Author = {Shankar, Pravin and Ganapathy, Vinod and Iftode, Liviu},
	Booktitle = {Proceedings of the 11th international conference on Ubiquitous computing},
	Organization = {ACM},
	Pages = {31--40},
	Title = {Privately querying location-based services with SybilQuery},
	Year = {2009}}
	
	@inproceedings{chow2009faking,
	Author = {Chow, Richard and Golle, Philippe},
	Booktitle = {Proceedings of the 8th ACM workshop on Privacy in the electronic society},
	Organization = {ACM},
	Pages = {105--108},
	Title = {Faking contextual data for fun, profit, and privacy},
	Year = {2009}}
	
	@incollection{xue2009location,
	Author = {Xue, Mingqiang and Kalnis, Panos and Pung, Hung Keng},
	Booktitle = {Location and Context Awareness},
	Pages = {70--87},
	Publisher = {Springer},
	Title = {Location diversity: Enhanced privacy protection in location based services},
	Year = {2009}}
	
	@article{wernke2014classification,
	Author = {Wernke, Marius and Skvortsov, Pavel and D{\"u}rr, Frank and Rothermel, Kurt},
	Journal = {Personal and Ubiquitous Computing},
	Number = {1},
	Pages = {163--175},
	Publisher = {Springer-Verlag},
	Title = {A classification of location privacy attacks and approaches},
	Volume = {18},
	Year = {2014}}
	
	@misc{cai2015cloaking,
	Author = {Cai, Y. and Xu, G.},
	Month = jan # {~1},
	Note = {US Patent App. 14/472,462},
	Publisher = {Google Patents},
	Title = {Cloaking with footprints to provide location privacy protection in location-based services},
	Url = {https://www.google.com/patents/US20150007341},
	Year = {2015},
	Bdsk-Url-1 = {https://www.google.com/patents/US20150007341}}
	
	@article{gedik2008protecting,
	Author = {Gedik, Bu{\u{g}}ra and Liu, Ling},
	Journal = {Mobile Computing, IEEE Transactions on},
	Number = {1},
	Pages = {1--18},
	Publisher = {IEEE},
	Title = {Protecting location privacy with personalized k-anonymity: Architecture and algorithms},
	Volume = {7},
	Year = {2008}}
	
	@article{kalnis2006preserving,
	Author = {Kalnis, Panos and Ghinita, Gabriel and Mouratidis, Kyriakos and Papadias, Dimitris},
	Publisher = {TRB6/06},
	Title = {Preserving anonymity in location based services},
	Year = {2006}}
	
	@inproceedings{hoh2005protecting,
	Author = {Hoh, Baik and Gruteser, Marco},
	Booktitle = {Security and Privacy for Emerging Areas in Communications Networks, 2005. SecureComm 2005. First International Conference on},
	Organization = {IEEE},
	Pages = {194--205},
	Title = {Protecting location privacy through path confusion},
	Year = {2005}}
	
	@article{terrovitis2011privacy,
	Author = {Terrovitis, Manolis},
	Journal = {ACM SIGKDD Explorations Newsletter},
	Number = {1},
	Pages = {6--18},
	Publisher = {ACM},
	Title = {Privacy preservation in the dissemination of location data},
	Volume = {13},
	Year = {2011}}
	
	@article{shin2012privacy,
	Author = {Shin, Kang G and Ju, Xiaoen and Chen, Zhigang and Hu, Xin},
	Journal = {Wireless Communications, IEEE},
	Number = {1},
	Pages = {30--39},
	Publisher = {IEEE},
	Title = {Privacy protection for users of location-based services},
	Volume = {19},
	Year = {2012}}
	
	@article{khoshgozaran2011location,
	Author = {Khoshgozaran, Ali and Shahabi, Cyrus and Shirani-Mehr, Houtan},
	Journal = {Knowledge and Information Systems},
	Number = {3},
	Pages = {435--465},
	Publisher = {Springer},
	Title = {Location privacy: going beyond K-anonymity, cloaking and anonymizers},
	Volume = {26},
	Year = {2011}}
	
	@incollection{chatzikokolakis2015geo,
	Author = {Chatzikokolakis, Konstantinos and Palamidessi, Catuscia and Stronati, Marco},
	Booktitle = {Distributed Computing and Internet Technology},
	Pages = {49--72},
	Publisher = {Springer},
	Title = {Geo-indistinguishability: A Principled Approach to Location Privacy},
	Year = {2015}}
	
	@inproceedings{ngo2015location,
	Author = {Ngo, Hoa and Kim, Jong},
	Booktitle = {Computer Security Foundations Symposium (CSF), 2015 IEEE 28th},
	Organization = {IEEE},
	Pages = {63--74},
	Title = {Location Privacy via Differential Private Perturbation of Cloaking Area},
	Year = {2015}}
	
	@inproceedings{palanisamy2011mobimix,
	Author = {Palanisamy, Balaji and Liu, Ling},
	Booktitle = {Data Engineering (ICDE), 2011 IEEE 27th International Conference on},
	Organization = {IEEE},
	Pages = {494--505},
	Title = {Mobimix: Protecting location privacy with mix-zones over road networks},
	Year = {2011}}
	
	@inproceedings{um2010advanced,
	Author = {Um, Jung-Ho and Kim, Hee-Dae and Chang, Jae-Woo},
	Booktitle = {Social Computing (SocialCom), 2010 IEEE Second International Conference on},
	Organization = {IEEE},
	Pages = {1093--1098},
	Title = {An advanced cloaking algorithm using Hilbert curves for anonymous location based service},
	Year = {2010}}
	
	@inproceedings{bamba2008supporting,
	Author = {Bamba, Bhuvan and Liu, Ling and Pesti, Peter and Wang, Ting},
	Booktitle = {Proceedings of the 17th international conference on World Wide Web},
	Organization = {ACM},
	Pages = {237--246},
	Title = {Supporting anonymous location queries in mobile environments with privacygrid},
	Year = {2008}}
	
	@inproceedings{zhangwei2010distributed,
	Author = {Zhangwei, Huang and Mingjun, Xin},
	Booktitle = {Networks Security Wireless Communications and Trusted Computing (NSWCTC), 2010 Second International Conference on},
	Organization = {IEEE},
	Pages = {468--471},
	Title = {A distributed spatial cloaking protocol for location privacy},
	Volume = {2},
	Year = {2010}}
	
	@article{chow2011spatial,
	Author = {Chow, Chi-Yin and Mokbel, Mohamed F and Liu, Xuan},
	Journal = {GeoInformatica},
	Number = {2},
	Pages = {351--380},
	Publisher = {Springer},
	Title = {Spatial cloaking for anonymous location-based services in mobile peer-to-peer environments},
	Volume = {15},
	Year = {2011}}
	
	@inproceedings{lu2008pad,
	Author = {Lu, Hua and Jensen, Christian S and Yiu, Man Lung},
	Booktitle = {Proceedings of the Seventh ACM International Workshop on Data Engineering for Wireless and Mobile Access},
	Organization = {ACM},
	Pages = {16--23},
	Title = {Pad: privacy-area aware, dummy-based location privacy in mobile services},
	Year = {2008}}
	
	@incollection{khoshgozaran2007blind,
	Author = {Khoshgozaran, Ali and Shahabi, Cyrus},
	Booktitle = {Advances in Spatial and Temporal Databases},
	Pages = {239--257},
	Publisher = {Springer},
	Title = {Blind evaluation of nearest neighbor queries using space transformation to preserve location privacy},
	Year = {2007}}
	
	@inproceedings{ghinita2008private,
	Author = {Ghinita, Gabriel and Kalnis, Panos and Khoshgozaran, Ali and Shahabi, Cyrus and Tan, Kian-Lee},
	Booktitle = {Proceedings of the 2008 ACM SIGMOD international conference on Management of data},
	Organization = {ACM},
	Pages = {121--132},
	Title = {Private queries in location based services: anonymizers are not necessary},
	Year = {2008}}
	
	@article{paulet2014privacy,
	Author = {Paulet, Russell and Kaosar, Md Golam and Yi, Xun and Bertino, Elisa},
	Journal = {Knowledge and Data Engineering, IEEE Transactions on},
	Number = {5},
	Pages = {1200--1210},
	Publisher = {IEEE},
	Title = {Privacy-preserving and content-protecting location based queries},
	Volume = {26},
	Year = {2014}}
	
	@article{nguyen2013differential,
	Author = {Nguyen, Hiep H and Kim, Jong and Kim, Yoonho},
	Journal = {Journal of Computing Science and Engineering},
	Number = {3},
	Pages = {177--186},
	Title = {Differential privacy in practice},
	Volume = {7},
	Year = {2013}}
	
	@inproceedings{lee2012differential,
	Author = {Lee, Jaewoo and Clifton, Chris},
	Booktitle = {Proceedings of the 18th ACM SIGKDD international conference on Knowledge discovery and data mining},
	Organization = {ACM},
	Pages = {1041--1049},
	Title = {Differential identifiability},
	Year = {2012}}
	
	@inproceedings{andres2013geo,
	Author = {Andr{\'e}s, Miguel E and Bordenabe, Nicol{\'a}s E and Chatzikokolakis, Konstantinos and Palamidessi, Catuscia},
	Booktitle = {Proceedings of the 2013 ACM SIGSAC conference on Computer \& communications security},
	Organization = {ACM},
	Pages = {901--914},
	Title = {Geo-indistinguishability: Differential privacy for location-based systems},
	Year = {2013}}
	
	@inproceedings{machanavajjhala2008privacy,
	Author = {Machanavajjhala, Ashwin and Kifer, Daniel and Abowd, John and Gehrke, Johannes and Vilhuber, Lars},
	Booktitle = {Data Engineering, 2008. ICDE 2008. IEEE 24th International Conference on},
	Organization = {IEEE},
	Pages = {277--286},
	Title = {Privacy: Theory meets practice on the map},
	Year = {2008}}
	
	@article{dewri2013local,
	Author = {Dewri, Rinku},
	Journal = {Mobile Computing, IEEE Transactions on},
	Number = {12},
	Pages = {2360--2372},
	Publisher = {IEEE},
	Title = {Local differential perturbations: Location privacy under approximate knowledge attackers},
	Volume = {12},
	Year = {2013}}
	
	@inproceedings{chatzikokolakis2013broadening,
	Author = {Chatzikokolakis, Konstantinos and Andr{\'e}s, Miguel E and Bordenabe, Nicol{\'a}s Emilio and Palamidessi, Catuscia},
	Booktitle = {Privacy Enhancing Technologies},
	Organization = {Springer},
	Pages = {82--102},
	Title = {Broadening the Scope of Differential Privacy Using Metrics.},
	Year = {2013}}
	
	@inproceedings{zhong2009distributed,
	Author = {Zhong, Ge and Hengartner, Urs},
	Booktitle = {Pervasive Computing and Communications, 2009. PerCom 2009. IEEE International Conference on},
	Organization = {IEEE},
	Pages = {1--10},
	Title = {A distributed k-anonymity protocol for location privacy},
	Year = {2009}}
	
	@inproceedings{ho2011differential,
	Author = {Ho, Shen-Shyang and Ruan, Shuhua},
	Booktitle = {Proceedings of the 4th ACM SIGSPATIAL International Workshop on Security and Privacy in GIS and LBS},
	Organization = {ACM},
	Pages = {17--24},
	Title = {Differential privacy for location pattern mining},
	Year = {2011}}
	
	@inproceedings{cheng2006preserving,
	Author = {Cheng, Reynold and Zhang, Yu and Bertino, Elisa and Prabhakar, Sunil},
	Booktitle = {Privacy Enhancing Technologies},
	Organization = {Springer},
	Pages = {393--412},
	Title = {Preserving user location privacy in mobile data management infrastructures},
	Year = {2006}}
	
	@article{beresford2003location,
	Author = {Beresford, Alastair R and Stajano, Frank},
	Journal = {IEEE Pervasive computing},
	Number = {1},
	Pages = {46--55},
	Publisher = {IEEE},
	Title = {Location privacy in pervasive computing},
	Year = {2003}}
	
	@inproceedings{freudiger2009optimal,
	Author = {Freudiger, Julien and Shokri, Reza and Hubaux, Jean-Pierre},
	Booktitle = {Privacy enhancing technologies},
	Organization = {Springer},
	Pages = {216--234},
	Title = {On the optimal placement of mix zones},
	Year = {2009}}
	
	@article{krumm2009survey,
	Author = {Krumm, John},
	Journal = {Personal and Ubiquitous Computing},
	Number = {6},
	Pages = {391--399},
	Publisher = {Springer},
	Title = {A survey of computational location privacy},
	Volume = {13},
	Year = {2009}}
	
	@article{Rakhshan2016letter,
	Author = {Rakhshan, Ali and Pishro-Nik, Hossein},
	Journal = {IEEE Wireless Communications Letter},
	Publisher = {IEEE},
	Title = {Interference Models for Vehicular Ad Hoc Networks},
	Year = {2016, submitted}}
	
	@article{Rakhshan2015Journal,
	Author = {Rakhshan, Ali and Pishro-Nik, Hossein},
	Journal = {IEEE Transactions on Wireless Communications},
	Publisher = {IEEE},
	Title = {Improving Safety on Highways by Customizing Vehicular Ad Hoc Networks},
	Year = {to appear, 2017}}
	
	@inproceedings{Rakhshan2015Cogsima,
	Author = {Rakhshan, Ali and Pishro-Nik, Hossein},
	Booktitle = {IEEE International Multi-Disciplinary Conference on Cognitive Methods in Situation Awareness and Decision Support},
	Organization = {IEEE},
	Title = {A New Approach to Customization of Accident Warning Systems to Individual Drivers},
	Year = {2015}}
	
	@inproceedings{Rakhshan2015CISS,
	Author = {Rakhshan, Ali and Pishro-Nik, Hossein and Nekoui, Mohammad},
	Booktitle = {Conference on Information Sciences and Systems},
	Organization = {IEEE},
	Pages = {1--6},
	Title = {Driver-based adaptation of Vehicular Ad Hoc Networks for design of active safety systems},
	Year = {2015}}
	
	@inproceedings{Rakhshan2014IV,
	Author = {Rakhshan, Ali and Pishro-Nik, Hossein and Ray, Evan},
	Booktitle = {Intelligent Vehicles Symposium},
	Organization = {IEEE},
	Pages = {1181--1186},
	Title = {Real-time estimation of the distribution of brake response times for an individual driver using Vehicular Ad Hoc Network.},
	Year = {2014}}
	
	@inproceedings{Rakhshan2013Globecom,
	Author = {Rakhshan, Ali and Pishro-Nik, Hossein and Fisher, Donald and Nekoui, Mohammad},
	Booktitle = {IEEE Global Communications Conference},
	Organization = {IEEE},
	Pages = {1333--1337},
	Title = {Tuning collision warning algorithms to individual drivers for design of active safety systems.},
	Year = {2013}}
	
	@article{Nekoui2012Journal,
	Author = {Nekoui, Mohammad and Pishro-Nik, Hossein},
	Journal = {IEEE Transactions on Wireless Communications},
	Number = {8},
	Pages = {2895--2905},
	Publisher = {IEEE},
	Title = {Throughput Scaling laws for Vehicular Ad Hoc Networks},
	Volume = {11},
	Year = {2012}}
	
	
	
	
	
	
	
	
	
	@article{Nekoui2011Journal,
	Author = {Nekoui, Mohammad and Pishro-Nik, Hossein and Ni, Daiheng},
	Journal = {International Journal of Vehicular Technology},
	Pages = {1--11},
	Publisher = {Hindawi Publishing Corporation},
	Title = {Analytic Design of Active Safety Systems for Vehicular Ad hoc Networks},
	Volume = {2011},
	Year = {2011}}
	
	
	
	
	
	
	@article{shokri2014optimal,
	title={Optimal user-centric data obfuscation},
	author={Shokri, Reza},
	journal={arXiv preprint arXiv:1402.3426},
	year={2014}
	}
	@article{chatzikokolakis2015location,
	title={Location privacy via geo-indistinguishability},
	author={Chatzikokolakis, Konstantinos and Palamidessi, Catuscia and Stronati, Marco},
	journal={ACM SIGLOG News},
	volume={2},
	number={3},
	pages={46--69},
	year={2015},
	publisher={ACM}
	
	}
	@inproceedings{shokri2011quantifying2,
	title={Quantifying location privacy: the case of sporadic location exposure},
	author={Shokri, Reza and Theodorakopoulos, George and Danezis, George and Hubaux, Jean-Pierre and Le Boudec, Jean-Yves},
	booktitle={Privacy Enhancing Technologies},
	pages={57--76},
	year={2011},
	organization={Springer}
	}
	
	
	@inproceedings{Mont1603:Defining,
	AUTHOR="Zarrin Montazeri and Amir Houmansadr and Hossein Pishro-Nik",
	TITLE="Defining Perfect Location Privacy Using Anonymization",
	BOOKTITLE="2016 Annual Conference on Information Science and Systems (CISS) (CISS
	2016)",
	ADDRESS="Princeton, USA",
	DAYS=16,
	MONTH=mar,
	YEAR=2016,
	KEYWORDS="Information Theoretic Privacy; location-based services; Location Privacy;
	Information Theory",
	ABSTRACT="The popularity of mobile devices and location-based services (LBS) have
	created great concerns regarding the location privacy of users of such
	devices and services. Anonymization is a common technique that is often
	being used to protect the location privacy of LBS users. In this paper, we
	provide a general information theoretic definition for location privacy. In
	particular, we define perfect location privacy. We show that under certain
	conditions, perfect privacy is achieved if the pseudonyms of users is
	changed after o(N^(2/r?1)) observations by the adversary, where N is the
	number of users and r is the number of sub-regions or locations.
	"
	}
	@article{our-isita-location,
	Author = {Zarrin Montazeri and Amir Houmansadr and Hossein Pishro-Nik},
	Journal = {IEEE International Symposium on Information Theory and Its Applications (ISITA)},
	Title = {Achieving Perfect Location Privacy in Markov Models Using Anonymization},
	Year = {2016}
	}
	@article{our-TIFS,
	Author = {Zarrin Montazeri and Hossein Pishro-Nik and Amir Houmansadr},
	Journal = {IEEE Transactions on Information Forensics and Security, accepted with mandatory minor revisions},
	Title = {Perfect Location Privacy Using Anonymization in Mobile Networks},
	Year = {2017},
	note={Available on arxiv.org}
	}
	
	
	
	@techreport{sampigethaya2005caravan,
	title={CARAVAN: Providing location privacy for VANET},
	author={Sampigethaya, Krishna and Huang, Leping and Li, Mingyan and Poovendran, Radha and Matsuura, Kanta and Sezaki, Kaoru},
	year={2005},
	institution={DTIC Document}
	}
	@incollection{buttyan2007effectiveness,
	title={On the effectiveness of changing pseudonyms to provide location privacy in VANETs},
	author={Butty{\'a}n, Levente and Holczer, Tam{\'a}s and Vajda, Istv{\'a}n},
	booktitle={Security and Privacy in Ad-hoc and Sensor Networks},
	pages={129--141},
	year={2007},
	publisher={Springer}
	}
	@article{sampigethaya2007amoeba,
	title={AMOEBA: Robust location privacy scheme for VANET},
	author={Sampigethaya, Krishna and Li, Mingyan and Huang, Leping and Poovendran, Radha},
	journal={Selected Areas in communications, IEEE Journal on},
	volume={25},
	number={8},
	pages={1569--1589},
	year={2007},
	publisher={IEEE}
	}
	
	@article{lu2012pseudonym,
	title={Pseudonym changing at social spots: An effective strategy for location privacy in vanets},
	author={Lu, Rongxing and Li, Xiaodong and Luan, Tom H and Liang, Xiaohui and Shen, Xuemin},
	journal={Vehicular Technology, IEEE Transactions on},
	volume={61},
	number={1},
	pages={86--96},
	year={2012},
	publisher={IEEE}
	}
	@inproceedings{lu2010sacrificing,
	title={Sacrificing the plum tree for the peach tree: A socialspot tactic for protecting receiver-location privacy in VANET},
	author={Lu, Rongxing and Lin, Xiaodong and Liang, Xiaohui and Shen, Xuemin},
	booktitle={Global Telecommunications Conference (GLOBECOM 2010), 2010 IEEE},
	pages={1--5},
	year={2010},
	organization={IEEE}
	}
	@inproceedings{lin2011stap,
	title={STAP: A social-tier-assisted packet forwarding protocol for achieving receiver-location privacy preservation in VANETs},
	author={Lin, Xiaodong and Lu, Rongxing and Liang, Xiaohui and Shen, Xuemin Sherman},
	booktitle={INFOCOM, 2011 Proceedings IEEE},
	pages={2147--2155},
	year={2011},
	organization={IEEE}
	}
	@inproceedings{gerlach2007privacy,
	title={Privacy in VANETs using changing pseudonyms-ideal and real},
	author={Gerlach, Matthias and Guttler, Felix},
	booktitle={Vehicular Technology Conference, 2007. VTC2007-Spring. IEEE 65th},
	pages={2521--2525},
	year={2007},
	organization={IEEE}
	}
	@inproceedings{el2002security,
	title={Security issues in a future vehicular network},
	author={El Zarki, Magda and Mehrotra, Sharad and Tsudik, Gene and Venkatasubramanian, Nalini},
	booktitle={European Wireless},
	volume={2},
	year={2002}
	}
	
	@article{hubaux2004security,
	title={The security and privacy of smart vehicles},
	author={Hubaux, Jean-Pierre and Capkun, Srdjan and Luo, Jun},
	journal={IEEE Security \& Privacy Magazine},
	volume={2},
	number={LCA-ARTICLE-2004-007},
	pages={49--55},
	year={2004}
	}
	
	
	
	@inproceedings{duri2002framework,
	title={Framework for security and privacy in automotive telematics},
	author={Duri, Sastry and Gruteser, Marco and Liu, Xuan and Moskowitz, Paul and Perez, Ronald and Singh, Moninder and Tang, Jung-Mu},
	booktitle={Proceedings of the 2nd international workshop on Mobile commerce},
	pages={25--32},
	year={2002},
	organization={ACM}
	}
	@misc{NS-3,
	Howpublished = {\url{https://www.nsnam.org/}}},
}
@misc{testbed,
	Howpublished = {\url{http://www.its.dot.gov/testbed/PDF/SE-MI-Resource-Guide-9-3-1.pdf}}},
@misc{NGSIM,
	Howpublished = {\url{http://ops.fhwa.dot.gov/trafficanalysistools/ngsim.htm}},
}

@misc{National-a2013,
	Author = {National Highway Traffic Safety Administration},
	Howpublished = {\url{http://ops.fhwa.dot.gov/trafficanalysistools/ngsim.htm}},
	Title = {2013 Motor Vehicle Crashes: Overview. Traffic Safety Factors},
	Year = {2013}
}

@inproceedings{karnadi2007rapid,
	title={Rapid generation of realistic mobility models for VANET},
	author={Karnadi, Feliz Kristianto and Mo, Zhi Hai and Lan, Kun-chan},
	booktitle={Wireless Communications and Networking Conference, 2007. WCNC 2007. IEEE},
	pages={2506--2511},
	year={2007},
	organization={IEEE}
}
@inproceedings{saha2004modeling,
	title={Modeling mobility for vehicular ad-hoc networks},
	author={Saha, Amit Kumar and Johnson, David B},
	booktitle={Proceedings of the 1st ACM international workshop on Vehicular ad hoc networks},
	pages={91--92},
	year={2004},
	organization={ACM}
}
@inproceedings{lee2006modeling,
	title={Modeling steady-state and transient behaviors of user mobility: formulation, analysis, and application},
	author={Lee, Jong-Kwon and Hou, Jennifer C},
	booktitle={Proceedings of the 7th ACM international symposium on Mobile ad hoc networking and computing},
	pages={85--96},
	year={2006},
	organization={ACM}
}
@inproceedings{yoon2006building,
	title={Building realistic mobility models from coarse-grained traces},
	author={Yoon, Jungkeun and Noble, Brian D and Liu, Mingyan and Kim, Minkyong},
	booktitle={Proceedings of the 4th international conference on Mobile systems, applications and services},
	pages={177--190},
	year={2006},
	organization={ACM}
}

@inproceedings{choffnes2005integrated,
	title={An integrated mobility and traffic model for vehicular wireless networks},
	author={Choffnes, David R and Bustamante, Fabi{\'a}n E},
	booktitle={Proceedings of the 2nd ACM international workshop on Vehicular ad hoc networks},
	pages={69--78},
	year={2005},
	organization={ACM}
}

@inproceedings{Qian2008Globecom,
	title={CA Secure VANET MAC Protocol for DSRC Applications},
	author={Yi, Q. and Lu, K. and Moyeri, N.{\'a}n E},
	booktitle={Proceedings of IEEE GLOBECOM 2008},
	pages={1--5},
	year={2008},
	organization={IEEE}
}





@inproceedings{naumov2006evaluation,
	title={An evaluation of inter-vehicle ad hoc networks based on realistic vehicular traces},
	author={Naumov, Valery and Baumann, Rainer and Gross, Thomas},
	booktitle={Proceedings of the 7th ACM international symposium on Mobile ad hoc networking and computing},
	pages={108--119},
	year={2006},
	organization={ACM}
}
@article{sommer2008progressing,
	title={Progressing toward realistic mobility models in VANET simulations},
	author={Sommer, Christoph and Dressler, Falko},
	journal={Communications Magazine, IEEE},
	volume={46},
	number={11},
	pages={132--137},
	year={2008},
	publisher={IEEE}
}




@inproceedings{mahajan2006urban,
	title={Urban mobility models for vanets},
	author={Mahajan, Atulya and Potnis, Niranjan and Gopalan, Kartik and Wang, Andy},
	booktitle={2nd IEEE International Workshop on Next Generation Wireless Networks},
	volume={33},
	year={2006}
}

@inproceedings{Rakhshan2016packet,
	title={Packet success probability derivation in a vehicular ad hoc network for a highway scenario},
	author={Rakhshan, Ali and Pishro-Nik, Hossein},
	booktitle={2016 Annual Conference on Information Science and Systems (CISS)},
	pages={210--215},
	year={2016},
	organization={IEEE}
}

@inproceedings{Rakhshan2016CISS,
	Author = {Rakhshan, Ali and Pishro-Nik, Hossein},
	Booktitle = {Conference on Information Sciences and Systems},
	Organization = {IEEE},
	Pages = {210--215},
	Title = {Packet Success Probability Derivation in a Vehicular Ad Hoc Network for a Highway Scenario},
	Year = {2016}}

@article{Nekoui2013Journal,
	Author = {Nekoui, Mohammad and Pishro-Nik, Hossein},
	Journal = {Journal on Selected Areas in Communications, Special Issue on Emerging Technologies in Communications},
	Number = {9},
	Pages = {491--503},
	Publisher = {IEEE},
	Title = {Analytic Design of Active Safety Systems for Vehicular Ad hoc Networks},
	Volume = {31},
	Year = {2013}}


@inproceedings{Nekoui2011MOBICOM,
	Author = {Nekoui, Mohammad and Pishro-Nik, Hossein},
	Booktitle = {MOBICOM workshop on VehiculAr InterNETworking},
	Organization = {ACM},
	Title = {Analytic Design of Active Vehicular Safety Systems in Sparse Traffic},
	Year = {2011}}

@inproceedings{Nekoui2011VTC,
	Author = {Nekoui, Mohammad and Pishro-Nik, Hossein},
	Booktitle = {VTC-Fall},
	Organization = {IEEE},
	Title = {Analytical Design of Inter-vehicular Communications for Collision Avoidance},
	Year = {2011}}

@inproceedings{Bovee2011VTC,
	Author = {Bovee, Ben Louis and Nekoui, Mohammad and Pishro-Nik, Hossein},
	Booktitle = {VTC-Fall},
	Organization = {IEEE},
	Title = {Evaluation of the Universal Geocast Scheme For VANETs},
	Year = {2011}}

@inproceedings{Nekoui2010MOBICOM,
	Author = {Nekoui, Mohammad and Pishro-Nik, Hossein},
	Booktitle = {MOBICOM},
	Organization = {ACM},
	Title = {Fundamental Tradeoffs in Vehicular Ad Hoc Networks},
	Year = {2010}}

@inproceedings{Nekoui2010IVCS,
	Author = {Nekoui, Mohammad and Pishro-Nik, Hossein},
	Booktitle = {IVCS},
	Organization = {IEEE},
	Title = {A Universal Geocast Scheme for Vehicular Ad Hoc Networks},
	Year = {2010}}

@inproceedings{Nekoui2009ITW,
	Author = {Nekoui, Mohammad and Pishro-Nik, Hossein},
	Booktitle = {IEEE Communications Society Conference on Sensor, Mesh and Ad Hoc Communications and Networks Workshops},
	Organization = {IEEE},
	Pages = {1--3},
	Title = {A Geometrical Analysis of Obstructed Wireless Networks},
	Year = {2009}}

@article{Eslami2013Journal,
	Author = {Eslami, Ali and Nekoui, Mohammad and Pishro-Nik, Hossein and Fekri, Faramarz},
	Journal = {ACM Transactions on Sensor Networks},
	Number = {4},
	Pages = {51},
	Publisher = {ACM},
	Title = {Results on finite wireless sensor networks: Connectivity and coverage},
	Volume = {9},
	Year = {2013}}


@article{Jiafu2014Journal,
	Author = {Jiafu, W. and Zhang, D. and Zhao, S. and Yang, L. and Lloret, J.},
	Journal = {Communications Magazine},
	Number = {8},
	Pages = {106-113},
	Publisher = {IEEE},
	Title = {Context-aware vehicular cyber-physical systems with cloud support: architecture, challenges, and solutions},
	Volume = {52},
	Year = {2014}}

@inproceedings{Haas2010ACM,
	Author = {Haas, J.J. and Hu, Y.},
	Booktitle = {international workshop on VehiculAr InterNETworking},
	Organization = {ACM},
	Title = {Communication requirements for crash avoidance.},
	Year = {2010}}

@inproceedings{Yi2008GLOBECOM,
	Author = {Yi, Q. and Lu, K. and Moayeri, N.},
	Booktitle = {GLOBECOM},
	Organization = {IEEE},
	Title = {CA Secure VANET MAC Protocol for DSRC Applications.},
	Year = {2008}}

@inproceedings{Mughal2010ITSim,
	Author = {Mughal, B.M. and Wagan, A. and Hasbullah, H.},
	Booktitle = {International Symposium on Information Technology (ITSim)},
	Organization = {IEEE},
	Title = {Efficient congestion control in VANET for safety messaging.},
	Year = {2010}}

@article{Chang2011Journal,
	Author = {Chang, Y. and Lee, C. and Copeland, J.},
	Journal = {Selected Areas in Communications},
	Pages = {236 –249},
	Publisher = {IEEE},
	Title = {Goodput enhancement of VANETs in noisy CSMA/CA channels},
	Volume = {29},
	Year = {2011}}

@article{Garcia-Costa2011Journal,
	Author = {Garcia-Costa, C. and Egea-Lopez, E. and Tomas-Gabarron, J. B. and Garcia-Haro, J. and Haas, Z. J.},
	Journal = {Transactions on Intelligent Transportation Systems},
	Pages = {1 –16},
	Publisher = {IEEE},
	Title = {A stochastic model for chain collisions of vehicles equipped with vehicular communications},
	Volume = {99},
	Year = {2011}}

@article{Carbaugh2011Journal,
	Author = {Carbaugh, J. and Godbole,  D. N. and Sengupta, R. and Garcia-Haro, J. and Haas, Z. J.},
	Publisher = {Transportation Research Part C (Emerging Technologies)},
	Title = {Safety and capacity analysis of automated and manual highway systems},
	Year = {1997}}

@article{Goh2004Journal,
	Author = {Goh, P. and Wong, Y.},
	Publisher = {Appl Health Econ Health Policy},
	Title = {Driver perception response time during the signal change interval},
	Year = {2004}}

@article{Chang1985Journal,
	Author = {Chang, M.S. and Santiago, A.J.},
	Pages = {20-30},
	Publisher = {Transportation Research Record},
	Title = {Timing traffic signal changes based on driver behavior},
	Volume = {1027},
	Year = {1985}}

@article{Green2000Journal,
	Author = {Green, M.},
	Pages = {195-216},
	Publisher = {Transportation Human Factors},
	Title = {How long does it take to stop? Methodological analysis of driver perception-brake times.},
	Year = {2000}}

@article{Koppa2005,
	Author = {Koppa, R.J.},
	Pages = {195-216},
	Publisher = {http://www.fhwa.dot.gov/publications/},
	Title = {Human Factors},
	Year = {2005}}

@inproceedings{Maxwell2010ETC,
	Author = {Maxwell, A. and Wood, K.},
	Booktitle = {Europian Transport Conference},
	Organization = {http://www.etcproceedings.org/paper/review-of-traffic-signals-on-high-speed-roads},
	Title = {Review of Traffic Signals on High Speed Road},
	Year = {2010}}

@article{Wortman1983,
	Author = {Wortman, R.H. and Matthias, J.S.},
	Publisher = {Arizona Department of Transportation},
	Title = {An Evaluation of Driver Behavior at Signalized Intersections},
	Year = {1983}}
@inproceedings{Zhang2007IASTED,
	Author = {Zhang, X. and Bham, G.H.},
	Booktitle = {18th IASTED International Conference: modeling and simulation},
	Title = {Estimation of driver reaction time from detailed vehicle trajectory data.},
	Year = {2007}}


@inproceedings{bai2003important,
	title={IMPORTANT: A framework to systematically analyze the Impact of Mobility on Performance of RouTing protocols for Adhoc NeTworks},
	author={Bai, Fan and Sadagopan, Narayanan and Helmy, Ahmed},
	booktitle={INFOCOM 2003. Twenty-second annual joint conference of the IEEE computer and communications. IEEE societies},
	volume={2},
	pages={825--835},
	year={2003},
	organization={IEEE}
}


@inproceedings{abedi2008enhancing,
	title={Enhancing AODV routing protocol using mobility parameters in VANET},
	author={Abedi, Omid and Fathy, Mahmood and Taghiloo, Jamshid},
	booktitle={Computer Systems and Applications, 2008. AICCSA 2008. IEEE/ACS International Conference on},
	pages={229--235},
	year={2008},
	organization={IEEE}
}


@article{AlSultan2013Journal,
	Author = {Al-Sultan, Saif and Al-Bayatti, Ali H. and Zedan, Hussien},
	Journal = {IEEE Transactions on Vehicular Technology},
	Number = {9},
	Pages = {4264-4275},
	Publisher = {IEEE},
	Title = {Context Aware Driver Behaviour Detection System in Intelligent Transportation Systems},
	Volume = {62},
	Year = {2013}}






@article{Leow2008ITS,
	Author = {Leow, Woei Ling and Ni, Daiheng and Pishro-Nik, Hossein},
	Journal = {IEEE Transactions on Intelligent Transportation Systems},
	Number = {2},
	Pages = {369--374},
	Publisher = {IEEE},
	Title = {A Sampling Theorem Approach to Traffic Sensor Optimization},
	Volume = {9},
	Year = {2008}}



@article{REU2007,
	Author = {D. Ni and H. Pishro-Nik and R. Prasad and M. R. Kanjee and H. Zhu and T. Nguyen},
	Journal = {in 14th World Congress on Intelligent Transport Systems},
	Title = {Development of a prototype intersection collision avoidance system under VII},
	Year = {2007}}




@inproceedings{salamatian2013hide,
	title={How to hide the elephant-or the donkey-in the room: Practical privacy against statistical inference for large data.},
	author={Salamatian, Salman and Zhang, Amy and du Pin Calmon, Flavio and Bhamidipati, Sandilya and Fawaz, Nadia and Kveton, Branislav and Oliveira, Pedro and Taft, Nina},
	booktitle={GlobalSIP},
	pages={269--272},
	year={2013}
}

@article{sankar2013utility,
	title={Utility-privacy tradeoffs in databases: An information-theoretic approach},
	author={Sankar, Lalitha and Rajagopalan, S Raj and Poor, H Vincent},
	journal={Information Forensics and Security, IEEE Transactions on},
	volume={8},
	number={6},
	pages={838--852},
	year={2013},
	publisher={IEEE}
}
@inproceedings{ghinita2007prive,
	title={PRIVE: anonymous location-based queries in distributed mobile systems},
	author={Ghinita, Gabriel and Kalnis, Panos and Skiadopoulos, Spiros},
	booktitle={Proceedings of the 16th international conference on World Wide Web},
	pages={371--380},
	year={2007},
	organization={ACM}
}

@article{beresford2004mix,
	title={Mix zones: User privacy in location-aware services},
	author={Beresford, Alastair R and Stajano, Frank},
	year={2004},
	publisher={IEEE}
}

%@inproceedings{Mont1610Achieving,
	%  title={Achieving Perfect Location Privacy in Markov Models Using Anonymization},
	%  author={Montazeri, Zarrin and Houmansadr, Amir and H.Pishro-Nik},
	%  booktitle="2016 International Symposium on Information Theory and its Applications
	%  (ISITA2016)",
	%  address="Monterey, USA",
	%  days=30,
	%  month=oct,
	%  year=2016,
	%}

@article{csiszar1996almost,
	title={Almost independence and secrecy capacity},
	author={Csisz{\'a}r, Imre},
	journal={Problemy Peredachi Informatsii},
	volume={32},
	number={1},
	pages={48--57},
	year={1996},
	publisher={Russian Academy of Sciences, Branch of Informatics, Computer Equipment and Automatization}
}

@article{yamamoto1983source,
	title={A source coding problem for sources with additional outputs to keep secret from the receiver or wiretappers (corresp.)},
	author={Yamamoto, Hirosuke},
	journal={IEEE Transactions on Information Theory},
	volume={29},
	number={6},
	pages={918--923},
	year={1983},
	publisher={IEEE}
}


@inproceedings{calmon2015fundamental,
	title={Fundamental limits of perfect privacy},
	author={Calmon, Flavio P and Makhdoumi, Ali and M{\'e}dard, Muriel},
	booktitle={Information Theory (ISIT), 2015 IEEE International Symposium on},
	pages={1796--1800},
	year={2015},
	organization={IEEE}
}



@inproceedings{Lehman1999Large-Sample-Theory,
	title={Elements of Large Sample Theory},
	author={E. L. Lehman},
	organization={Springer},
	year={1999}
}


@inproceedings{Ferguson1999Large-Sample-Theory,
	title={A Course in Large Sample Theory},
	author={Thomas S. Ferguson},
	organization={CRC Press},
	year={1996}
}



@inproceedings{Dembo1999Large-Deviations,
	title={Large Deviation Techniques and Applications, Second Edition},
	author={A. Dembo and O. Zeitouni},
	organization={Springer},
	year={1998}
}


%%%%%%%%%%%%%%%%%%%%%%%%%%%%%%%%%%%%%%%%%%%%%%%%
Hossein's Coding Journals
%%%%%%%%%%%%%%%%%%%%%%

@ARTICLE{myoptics,
	AUTHOR =       "H. Pishro-Nik and N. Rahnavard and J. Ha and F. Fekri and A. Adibi ",
	TITLE =        "Low-density parity-check codes for volume holographic memory systems",
	JOURNAL =      " Appl. Opt.",
	YEAR =         "2003",
	volume =       "42",
	pages =        "861-870  "
}






@ARTICLE{myit,
	AUTHOR =       "H. Pishro-Nik and F. Fekri  ",
	TITLE =        "On Decoding of Low-Density Parity-Check Codes on the Binary Erasure Channel",
	JOURNAL =      "IEEE Trans. Inform. Theory",
	YEAR =         "2004",
	volume =       "50",
	pages =        "439--454"
}




@ARTICLE{myitpuncture,
	AUTHOR =       "H. Pishro-Nik and F. Fekri  ",
	TITLE =        "Results on Punctured Low-Density Parity-Check Codes and Improved Iterative Decoding Techniques",
	JOURNAL =      "IEEE Trans. on Inform. Theory",
	YEAR =         "2007",
	volume =       "53",
	number=        "2",
	pages =        "599--614",
	month= "February"
}




@ARTICLE{myitlinmimdist,
	AUTHOR =       "H. Pishro-Nik and F. Fekri",
	TITLE =        "Performance of Low-Density Parity-Check Codes With Linear Minimum Distance",
	JOURNAL =         "IEEE Trans. Inform. Theory ",
	YEAR =         "2006",
	volume =       "52",
	number="1",
	pages =        "292 --300"
}






@ARTICLE{myitnonuni,
	AUTHOR =       "H. Pishro-Nik and N. Rahnavard and F. Fekri  ",
	TITLE =        "Non-uniform Error Correction Using Low-Density Parity-Check Codes",
	JOURNAL =      "IEEE Trans. Inform. Theory",
	YEAR =         "2005",
	volume =       "51",
	number=  "7",
	pages =        "2702--2714"
}





@article{eslamitcomhybrid10,
	author = {A. Eslami and S. Vangala and H. Pishro-Nik},
	title = {Hybrid channel codes for highly efficient FSO/RF communication systems},
	journal = {IEEE Transactions on Communications},
	volume = {58},
	number = {10},
	year = {2010},
	pages = {2926--2938},
}


@article{eslamitcompolar13,
	author = {A. Eslami and H. Pishro-Nik},
	title = {On Finite-Length Performance of Polar Codes: Stopping Sets, Error Floor, and Concatenated Design},
	journal = {IEEE Transactions on Communications},
	volume = {61},
	number = {13},
	year = {2013},
	pages = {919--929},
}



@article{saeeditcom11,
	author = {H. Saeedi and H. Pishro-Nik and  A. H. Banihashemi},
	title = {Successive maximization for the systematic design of universally capacity approaching rate-compatible
	sequences of LDPC code ensembles over binary-input output-symmetric memoryless channels},
	journal = {IEEE Transactions on Communications},
	year = {2011},
	volume={59},
	number = {7}
}


@article{rahnavard07,
	author = {Rahnavard, N. and Pishro-Nik, H. and Fekri, F.},
	title = {Unequal Error Protection Using Partially Regular LDPC Codes},
	journal = {IEEE Transactions on Communications},
	year = {2007},
	volume = {55},
	number = {3},
	pages = {387 -- 391}
}


@article{hosseinira04,
	author = {H. Pishro-Nik and F. Fekri},
	title = {Irregular repeat-accumulate codes for volume holographic memory systems},
	journal = {Journal of Applied Optics},
	year = {2004},
	volume = {43},
	number = {27},
	pages = {5222--5227},
}


@article{azadeh2015Ephemeralkey,
	author = {A. Sheikholeslami and D. Goeckel and H. Pishro-Nik},
	title = {Jamming Based on an Ephemeral Key to Obtain Everlasting Security in Wireless Environments},
	journal = {IEEE Transactions on Wireless Communications},
	year = {2015},
	volume = {14},
	number = {11},
	pages = {6072--6081},
}


@article{azadeh2014Everlasting,
	author = {A. Sheikholeslami and D. Goeckel and H. Pishro-Nik},
	title = {Everlasting secrecy in disadvantaged wireless environments against sophisticated eavesdroppers},
	journal = {48th Asilomar Conference on Signals, Systems and Computers},
	year = {2014},
	pages = {1994--1998},
}


@article{azadeh2013ISIT,
	author = {A. Sheikholeslami and D. Goeckel and H. Pishro-Nik},
	title = {Artificial intersymbol interference (ISI) to exploit receiver imperfections for secrecy},
	journal = {IEEE International Symposium on Information Theory (ISIT)},
	year = {2013},
}


@article{azadeh2013Jsac,
	author = {A. Sheikholeslami and D. Goeckel and H. Pishro-Nik},
	title = {Jamming Based on an Ephemeral Key to Obtain Everlasting Security in Wireless Environments},
	journal = {IEEE Journal on Selected Areas in Communications},
	year = {2013},
	volume = {31},
	number = {9},
	pages = {1828--1839},
}


@article{azadeh2012Allerton,
	author = {A. Sheikholeslami and D. Goeckel and H. Pishro-Nik},
	title = {Exploiting the non-commutativity of nonlinear operators for information-theoretic security in disadvantaged wireless environments},
	journal = {50th Annual Allerton Conference on Communication, Control, and Computing},
	year = {2012},
	pages = {233--240},
}


@article{azadeh2012Infocom,
	author = {A. Sheikholeslami and D. Goeckel and H. Pishro-Nik},
	title = {Jamming Based on an Ephemeral Key to Obtain Everlasting Security in Wireless Environments},
	journal = {IEEE INFOCOM},
	year = {2012},
	pages = {1179--1187},
}

@article{1corser2016evaluating,
	title={Evaluating Location Privacy in Vehicular Communications and Applications},
	author={Corser, George P and Fu, Huirong and Banihani, Abdelnasser},
	journal={IEEE Transactions on Intelligent Transportation Systems},
	volume={17},
	number={9},
	pages={2658-2667},
	year={2016},
	publisher={IEEE}
}
@article{2zhang2016designing,
	title={On Designing Satisfaction-Ratio-Aware Truthful Incentive Mechanisms for k-Anonymity Location Privacy},
	author={Zhang, Yuan and Tong, Wei and Zhong, Sheng},
	journal={IEEE Transactions on Information Forensics and Security},
	volume={11},
	number={11},
	pages={2528--2541},
	year={2016},
	publisher={IEEE}
}
@article{3li2016privacy,
	title={Privacy-preserving Location Proof for Securing Large-scale Database-driven Cognitive Radio Networks},
	author={Li, Yi and Zhou, Lu and Zhu, Haojin and Sun, Limin},
	journal={IEEE Internet of Things Journal},
	volume={3},
	number={4},
	pages={563-571},
	year={2016},
	publisher={IEEE}
}
@article{4olteanu2016quantifying,
	title={Quantifying Interdependent Privacy Risks with Location Data},
	author={Olteanu, Alexandra-Mihaela and Huguenin, K{\'e}vin and Shokri, Reza and Humbert, Mathias and Hubaux, Jean-Pierre},
	journal={IEEE Transactions on Mobile Computing},
	year={2016},
	volume={PP},
	number={99},
	pages={1-1},
	publisher={IEEE}
}
@article{5yi2016practical,
	title={Practical Approximate k Nearest Neighbor Queries with Location and Query Privacy},
	author={Yi, Xun and Paulet, Russell and Bertino, Elisa and Varadharajan, Vijay},
	journal={IEEE Transactions on Knowledge and Data Engineering},
	volume={28},
	number={6},
	pages={1546--1559},
	year={2016},
	publisher={IEEE}
}
@article{6li2016privacy,
	title={Privacy Leakage of Location Sharing in Mobile Social Networks: Attacks and Defense},
	author={Li, Huaxin and Zhu, Haojin and Du, Suguo and Liang, Xiaohui and Shen, Xuemin},
	journal={IEEE Transactions on Dependable and Secure Computing},
	year={2016},
	volume={PP},
	number={99},
	publisher={IEEE}
}

@article{7murakami2016localization,
	title={Localization Attacks Using Matrix and Tensor Factorization},
	author={Murakami, Takao and Watanabe, Hajime},
	journal={IEEE Transactions on Information Forensics and Security},
	volume={11},
	number={8},
	pages={1647--1660},
	year={2016},
	publisher={IEEE}
}
@article{8zurbaran2015near,
	title={Near-Rand: Noise-based Location Obfuscation Based on Random Neighboring Points},
	author={Zurbaran, Mayra Alejandra and Avila, Karen and Wightman, Pedro and Fernandez, Michael},
	journal={IEEE Latin America Transactions},
	volume={13},
	number={11},
	pages={3661--3667},
	year={2015},
	publisher={IEEE}
}

@article{9tan2014anti,
	title={An anti-tracking source-location privacy protection protocol in wsns based on path extension},
	author={Tan, Wei and Xu, Ke and Wang, Dan},
	journal={IEEE Internet of Things Journal},
	volume={1},
	number={5},
	pages={461--471},
	year={2014},
	publisher={IEEE}
}

@article{10peng2014enhanced,
	title={Enhanced Location Privacy Preserving Scheme in Location-Based Services},
	author={Peng, Tao and Liu, Qin and Wang, Guojun},
	journal={IEEE Systems Journal},
	year={2014},
	volume={PP},
	number={99},
	pages={1-12},
	publisher={IEEE}
}
@article{11dewri2014exploiting,
	title={Exploiting service similarity for privacy in location-based search queries},
	author={Dewri, Rinku and Thurimella, Ramakrisha},
	journal={IEEE Transactions on Parallel and Distributed Systems},
	volume={25},
	number={2},
	pages={374--383},
	year={2014},
	publisher={IEEE}
}

@article{12hwang2014novel,
	title={A novel time-obfuscated algorithm for trajectory privacy protection},
	author={Hwang, Ren-Hung and Hsueh, Yu-Ling and Chung, Hao-Wei},
	journal={IEEE Transactions on Services Computing},
	volume={7},
	number={2},
	pages={126--139},
	year={2014},
	publisher={IEEE}
}
@article{13puttaswamy2014preserving,
	title={Preserving location privacy in geosocial applications},
	author={Puttaswamy, Krishna PN and Wang, Shiyuan and Steinbauer, Troy and Agrawal, Divyakant and El Abbadi, Amr and Kruegel, Christopher and Zhao, Ben Y},
	journal={IEEE Transactions on Mobile Computing},
	volume={13},
	number={1},
	pages={159--173},
	year={2014},
	publisher={IEEE}
}

@article{14zhang2014privacy,
	title={Privacy quantification model based on the Bayes conditional risk in Location-Based Services},
	author={Zhang, Xuejun and Gui, Xiaolin and Tian, Feng and Yu, Si and An, Jian},
	journal={Tsinghua Science and Technology},
	volume={19},
	number={5},
	pages={452--462},
	year={2014},
	publisher={TUP}
}

@article{15bilogrevic2014privacy,
	title={Privacy-preserving optimal meeting location determination on mobile devices},
	author={Bilogrevic, Igor and Jadliwala, Murtuza and Joneja, Vishal and Kalkan, K{\"u}bra and Hubaux, Jean-Pierre and Aad, Imad},
	journal={IEEE transactions on information forensics and security},
	volume={9},
	number={7},
	pages={1141--1156},
	year={2014},
	publisher={IEEE}
}
@article{16haghnegahdar2014privacy,
	title={Privacy Risks in Publishing Mobile Device Trajectories},
	author={Haghnegahdar, Alireza and Khabbazian, Majid and Bhargava, Vijay K},
	journal={IEEE Wireless Communications Letters},
	volume={3},
	number={3},
	pages={241--244},
	year={2014},
	publisher={IEEE}
}
@article{17malandrino2014verification,
	title={Verification and inference of positions in vehicular networks through anonymous beaconing},
	author={Malandrino, Francesco and Borgiattino, Carlo and Casetti, Claudio and Chiasserini, Carla-Fabiana and Fiore, Marco and Sadao, Roberto},
	journal={IEEE Transactions on Mobile Computing},
	volume={13},
	number={10},
	pages={2415--2428},
	year={2014},
	publisher={IEEE}
}
@article{18shokri2014hiding,
	title={Hiding in the mobile crowd: Locationprivacy through collaboration},
	author={Shokri, Reza and Theodorakopoulos, George and Papadimitratos, Panos and Kazemi, Ehsan and Hubaux, Jean-Pierre},
	journal={IEEE transactions on dependable and secure computing},
	volume={11},
	number={3},
	pages={266--279},
	year={2014},
	publisher={IEEE}
}
@article{19freudiger2013non,
	title={Non-cooperative location privacy},
	author={Freudiger, Julien and Manshaei, Mohammad Hossein and Hubaux, Jean-Pierre and Parkes, David C},
	journal={IEEE Transactions on Dependable and Secure Computing},
	volume={10},
	number={2},
	pages={84--98},
	year={2013},
	publisher={IEEE}
}
@article{20gao2013trpf,
	title={TrPF: A trajectory privacy-preserving framework for participatory sensing},
	author={Gao, Sheng and Ma, Jianfeng and Shi, Weisong and Zhan, Guoxing and Sun, Cong},
	journal={IEEE Transactions on Information Forensics and Security},
	volume={8},
	number={6},
	pages={874--887},
	year={2013},
	publisher={IEEE}
}
@article{21ma2013privacy,
	title={Privacy vulnerability of published anonymous mobility traces},
	author={Ma, Chris YT and Yau, David KY and Yip, Nung Kwan and Rao, Nageswara SV},
	journal={IEEE/ACM Transactions on Networking},
	volume={21},
	number={3},
	pages={720--733},
	year={2013},
	publisher={IEEE}
}
@article{22niu2013pseudo,
	title={Pseudo-Location Updating System for privacy-preserving location-based services},
	author={Niu, Ben and Zhu, Xiaoyan and Chi, Haotian and Li, Hui},
	journal={China Communications},
	volume={10},
	number={9},
	pages={1--12},
	year={2013},
	publisher={IEEE}
}
@article{23dewri2013local,
	title={Local differential perturbations: Location privacy under approximate knowledge attackers},
	author={Dewri, Rinku},
	journal={IEEE Transactions on Mobile Computing},
	volume={12},
	number={12},
	pages={2360--2372},
	year={2013},
	publisher={IEEE}
}
@inproceedings{24kanoria2012tractable,
	title={Tractable bayesian social learning on trees},
	author={Kanoria, Yashodhan and Tamuz, Omer},
	booktitle={Information Theory Proceedings (ISIT), 2012 IEEE International Symposium on},
	pages={2721--2725},
	year={2012},
	organization={IEEE}
}
@inproceedings{25farias2005universal,
	title={A universal scheme for learning},
	author={Farias, Vivek F and Moallemi, Ciamac C and Van Roy, Benjamin and Weissman, Tsachy},
	booktitle={Proceedings. International Symposium on Information Theory, 2005. ISIT 2005.},
	pages={1158--1162},
	year={2005},
	organization={IEEE}
}
@inproceedings{26misra2013unsupervised,
	title={Unsupervised learning and universal communication},
	author={Misra, Vinith and Weissman, Tsachy},
	booktitle={Information Theory Proceedings (ISIT), 2013 IEEE International Symposium on},
	pages={261--265},
	year={2013},
	organization={IEEE}
}
@inproceedings{27ryabko2013time,
	title={Time-series information and learning},
	author={Ryabko, Daniil},
	booktitle={Information Theory Proceedings (ISIT), 2013 IEEE International Symposium on},
	pages={1392--1395},
	year={2013},
	organization={IEEE}
}
@inproceedings{28krzakala2013phase,
	title={Phase diagram and approximate message passing for blind calibration and dictionary learning},
	author={Krzakala, Florent and M{\'e}zard, Marc and Zdeborov{\'a}, Lenka},
	booktitle={Information Theory Proceedings (ISIT), 2013 IEEE International Symposium on},
	pages={659--663},
	year={2013},
	organization={IEEE}
}
@inproceedings{29sakata2013sample,
	title={Sample complexity of Bayesian optimal dictionary learning},
	author={Sakata, Ayaka and Kabashima, Yoshiyuki},
	booktitle={Information Theory Proceedings (ISIT), 2013 IEEE International Symposium on},
	pages={669--673},
	year={2013},
	organization={IEEE}
}
@inproceedings{30predd2004consistency,
	title={Consistency in a model for distributed learning with specialists},
	author={Predd, Joel B and Kulkarni, Sanjeev R and Poor, H Vincent},
	booktitle={IEEE International Symposium on Information Theory},
	year={2004},
	organization={IEEE}
}
@inproceedings{31nokleby2016rate,
	title={Rate-Distortion Bounds on Bayes Risk in Supervised Learning},
	author={Nokleby, Matthew and Beirami, Ahmad and Calderbank, Robert},
	booktitle={2016 IEEE International Symposium on Information Theory (ISIT)},
	pages={2099-2103},
	year={2016},
	organization={IEEE}
}

@inproceedings{32le2016imperfect,
	title={Are imperfect reviews helpful in social learning?},
	author={Le, Tho Ngoc and Subramanian, Vijay G and Berry, Randall A},
	booktitle={Information Theory (ISIT), 2016 IEEE International Symposium on},
	pages={2089--2093},
	year={2016},
	organization={IEEE}
}
@inproceedings{33gadde2016active,
	title={Active Learning for Community Detection in Stochastic Block Models},
	author={Gadde, Akshay and Gad, Eyal En and Avestimehr, Salman and Ortega, Antonio},
	booktitle={2016 IEEE International Symposium on Information Theory (ISIT)},
	pages={1889-1893},
	year={2016}
}
@inproceedings{34shakeri2016minimax,
	title={Minimax Lower Bounds for Kronecker-Structured Dictionary Learning},
	author={Shakeri, Zahra and Bajwa, Waheed U and Sarwate, Anand D},
	booktitle={2016 IEEE International Symposium on Information Theory (ISIT)},
	pages={1148-1152},
	year={2016}
}
@article{35lee2015speeding,
	title={Speeding up distributed machine learning using codes},
	author={Lee, Kangwook and Lam, Maximilian and Pedarsani, Ramtin and Papailiopoulos, Dimitris and Ramchandran, Kannan},
	booktitle={2016 IEEE International Symposium on Information Theory (ISIT)},
	pages={1143-1147},
	year={2016}
}
@article{36oneto2016statistical,
	title={Statistical Learning Theory and ELM for Big Social Data Analysis},
	author={Oneto, Luca and Bisio, Federica and Cambria, Erik and Anguita, Davide},
	journal={ieee CompUTATionAl inTelliGenCe mAGAzine},
	volume={11},
	number={3},
	pages={45--55},
	year={2016},
	publisher={IEEE}
}
@article{37lin2015probabilistic,
	title={Probabilistic approach to modeling and parameter learning of indirect drive robots from incomplete data},
	author={Lin, Chung-Yen and Tomizuka, Masayoshi},
	journal={IEEE/ASME Transactions on Mechatronics},
	volume={20},
	number={3},
	pages={1036--1045},
	year={2015},
	publisher={IEEE}
}
@article{38wang2016towards,
	title={Towards Bayesian Deep Learning: A Framework and Some Existing Methods},
	author={Wang, Hao and Yeung, Dit-Yan},
	journal={IEEE Transactions on Knowledge and Data Engineering},
	volume={PP},
	number={99},
	year={2016},
	publisher={IEEE}
}


%%%%%Informationtheoreticsecurity%%%%%%%%%%%%%%%%%%%%%%%




@inproceedings{Bloch2011PhysicalSecBook,
	title={Physical-Layer Security},
	author={M. Bloch and J. Barros},
	organization={Cambridge University Press},
	year={2011}
}



@inproceedings{Liang2009InfoSecBook,
	title={Information Theoretic Security},
	author={Y. Liang and H. V. Poor and S. Shamai (Shitz)},
	organization={Now Publishers Inc.},
	year={2009}
}


@inproceedings{Zhou2013PhysicalSecBook,
	title={Physical Layer Security in Wireless Communications},
	author={ X. Zhou and L. Song and Y. Zhang},
	organization={CRC Press},
	year={2013}
}

@article{Ni2012IEA,
	Author = {D. Ni and H. Liu and W. Ding and  Y. Xie and H. Wang and H. Pishro-Nik and Q. Yu},
	Journal = {IEA/AIE},
	Title = {Cyber-Physical Integration to Connect Vehicles for Transformed Transportation Safety and Efficiency},
	Year = {2012}}



@inproceedings{Ni2012Inproceedings,
	Author = {D. Ni, H. Liu, Y. Xie, W. Ding, H. Wang, H. Pishro-Nik, Q. Yu and M. Ferreira},
	Booktitle = {Spring Simulation Multiconference},
	Date-Added = {2016-09-04 14:18:42 +0000},
	Date-Modified = {2016-09-06 16:22:14 +0000},
	Title = {Virtual Lab of Connected Vehicle Technology},
	Year = {2012}}

@inproceedings{Ni2012Inproceedings,
	Author = {D. Ni, H. Liu, W. Ding, Y. Xie, H. Wang, H. Pishro-Nik and Q. Yu,},
	Booktitle = {IEA/AIE},
	Date-Added = {2016-09-04 09:11:02 +0000},
	Date-Modified = {2016-09-06 14:46:53 +0000},
	Title = {Cyber-Physical Integration to Connect Vehicles for Transformed Transportation Safety and Efficiency},
	Year = {2012}}


@article{Nekoui_IJIPT_2009,
	Author = {M. Nekoui and D. Ni and H. Pishro-Nik and R. Prasad and M. Kanjee and H. Zhu and T. Nguyen},
	Journal = {International Journal of Internet Protocol Technology (IJIPT)},
	Number = {3},
	Pages = {},
	Publisher = {},
	Title = {Development of a VII-Enabled Prototype Intersection Collision Warning System},
	Volume = {4},
	Year = {2009}}


@inproceedings{Pishro_Ganz_Ni,
	Author = {H. Pishro-Nik, A. Ganz, and Daiheng Ni},
	Booktitle = {Forty-Fifth Annual Allerton Conference on Communication, Control, and Computing. Allerton House, Monticello, IL},
	Date-Added = {},
	Date-Modified = {},
	Number = {},
	Pages = {},
	Title = {The capacity of vehicular ad hoc networks},
	Volume = {},
	Year = {September 26-28, 2007}}

@inproceedings{Leow_Pishro_Ni_1,
	Author = {W. L. Leow, H. Pishro-Nik and Daiheng Ni},
	Booktitle = {IEEE Global Telecommunications Conference, Washington, D.C.},
	Date-Added = {},
	Date-Modified = {},
	Number = {},
	Pages = {},
	Title = {Delay and Energy Tradeoff in Multi-state Wireless Sensor Networks},
	Volume = {},
	Year = {November 26-30, 2007}}


@misc{UMass-Trans,
	title = {{UMass Transportation Center}},
	note = {\url{http://www.umasstransportationcenter.org/}},
}


@inproceedings{Haenggi2013book,
	title={Stochastic geometry for wireless networks},
	author={M. Haenggi},
	organization={Cambridge Uinversity Press},
	year={2013}
}


%% This BibTeX bibliography file was created using BibDesk.
%% http://bibdesk.sourceforge.net/

%% Created for Zarrin Montazeri at 2015-11-09 18:45:31 -0500


%% Saved with string encoding Unicode (UTF-8)


%%%%%%%%%%%%%%%%personalization%%%%%%%%%%%%%%%%%%%%%%%%%%%%%%%%%%

@article{osma2015,
	title={Impact of Time-to-Collision Information on Driving Behavior in Connected Vehicle Environments Using A Driving Simulator Test Bed},
	journal{Journal of Traffic and Logistics Engineering},
	author={Osama A. Osman, Julius Codjoe, and Sherif Ishak},
	volume={3},
	number={1},
	 pages={18--24},
	year={2015}
}


@article{charisma2010,
	title={Dynamic Latent Plan Models},
	author={Charisma F. Choudhurya, Moshe Ben-Akivab and Maya Abou-Zeid},
	journal={Journal of Choice Modelling},
	volume={3},
	number={2},
	pages={50--70},
	year={2010},
	publisher={Elsvier}
}


@misc{noble2014,
	author = {A. M. Noble, Shane B. McLaughlin, Zachary R. Doerzaph and Thomas A. Dingus},
	title = {Crowd-sourced Connected-vehicle Warning Algorithm using Naturalistic Driving Data},
	howpublished = {Downloaded from \url{http://hdl.handle.net/10919/53978}},

	month = August,
	year = 2014
}


@phdthesis{charisma2007,
	title    = {Modeling Driving Decisions with Latent Plans},
	school   = {Massachusetts Institute of Technology },
	author   = {Charisma Farheen Choudhury},
	year     = {2007}, %other attributes omitted
}


@article{chrysler2015,
	title={Cost of Warning of Unseen Threats:Unintended Consequences of Connected Vehicle Alerts},
	author={S. T. Chrysler, J. M. Cooper and D. C. Marshall},
	journal={Transportation Research Record: Journal of the Transportation Research Board},
	volume={2518},
	pages={79--85},
	year={2015},
}

@misc{nsf_cps,
        title = {Cyber-Physical Systems (CPS) PROGRAM SOLICITATION NSF 17-529},
        howpublished = {Downloaded from \url{https://www.nsf.gov/publications/pub_summ.jsp?WT.z_pims_id=503286&ods_key=nsf17529}},
}



%%%%%%%%%%%%%%IOT%%%%%%%%%%%%%%%%%%%%%%%%%%%%%%%%%%%%%%%%%%%%%%%%%%%




@article{FTC2015,
  title={Internet of Things: Privacy and Security in a Connected World},
  author={FTC Staff Report},
  year={2015}
}



@article{0Quest2016,
  title={The Quest for Privacy in the Internet of Things},
  author={ Pawani Porambag and Mika Ylianttila and Corinna Schmitt and  Pardeep Kumar and  Andrei Gurtov and  Athanasios V. Vasilakos},
  journal={IEEE Cloud Computing},
  volum={3},
  number={2},
  year={2016},
  publisher={IEEE}
}

%% Saved with string encoding Unicode (UTF-8)
@inproceedings{1zhou2014security,
  title={Security/privacy of wearable fitness tracking {I}o{T} devices},
  author={Zhou, Wei and Piramuthu, Selwyn},
  booktitle={Information Systems and Technologies (CISTI), 2014 9th Iberian Conference on},
  pages={1--5},
  year={2014},
  organization={IEEE}
}


@inproceedings{3ukil2014iot,
  title={{I}o{T}-privacy: To be private or not to be private},
  author={Ukil, Arijit and Bandyopadhyay, Soma and Pal, Arpan},
  booktitle={Computer Communications Workshops (INFOCOM WKSHPS), IEEE Conference on},
  pages={123--124},
  year={2014},
  organization={IEEE}
}


@article{4arias2015privacy,
  title={Privacy and security in internet of things and wearable devices},
  author={Arias, Orlando and Wurm, Jacob and Hoang, Khoa and Jin, Yier},
  journal={IEEE Transactions on Multi-Scale Computing Systems},
  volume={1},
  number={2},
  pages={99--109},
  year={2015},
  publisher={IEEE}
}
@inproceedings{5ullah2016novel,
  title={A novel model for preserving Location Privacy in Internet of Things},
  author={Ullah, Ikram and Shah, Munam Ali},
  booktitle={Automation and Computing (ICAC), 2016 22nd International Conference on},
  pages={542--547},
  year={2016},
  organization={IEEE}
}
@inproceedings{6sathishkumar2016enhanced,
  title={Enhanced location privacy algorithm for wireless sensor network in Internet of Things},
  author={Sathishkumar, J and Patel, Dhiren R},
  booktitle={Internet of Things and Applications (IOTA), International Conference on},
  pages={208--212},
  year={2016},
  organization={IEEE}
}
@inproceedings{7zhou2012preserving,
  title={Preserving sensor location privacy in internet of things},
  author={Zhou, Liming and Wen, Qiaoyan and Zhang, Hua},
  booktitle={Computational and Information Sciences (ICCIS), 2012 Fourth International Conference on},
  pages={856--859},
  year={2012},
  organization={IEEE}
}

@inproceedings{8ukil2015privacy,
  title={Privacy for {I}o{T}: Involuntary privacy enablement for smart energy systems},
  author={Ukil, Arijit and Bandyopadhyay, Soma and Pal, Arpan},
  booktitle={Communications (ICC), 2015 IEEE International Conference on},
  pages={536--541},
  year={2015},
  organization={IEEE}
}

@inproceedings{9dalipi2016security,
  title={Security and Privacy Considerations for {I}o{T} Application on Smart Grids: Survey and Research Challenges},
  author={Dalipi, Fisnik and Yayilgan, Sule Yildirim},
  booktitle={Future Internet of Things and Cloud Workshops (FiCloudW), IEEE International Conference on},
  pages={63--68},
  year={2016},
  organization={IEEE}
}
@inproceedings{10harris2016security,
  title={Security and Privacy in Public {I}o{T} Spaces},
  author={Harris, Albert F and Sundaram, Hari and Kravets, Robin},
  booktitle={Computer Communication and Networks (ICCCN), 2016 25th International Conference on},
  pages={1--8},
  year={2016},
  organization={IEEE}
}

@inproceedings{11al2015security,
  title={Security and privacy framework for ubiquitous healthcare {I}o{T} devices},
  author={Al Alkeem, Ebrahim and Yeun, Chan Yeob and Zemerly, M Jamal},
  booktitle={Internet Technology and Secured Transactions (ICITST), 2015 10th International Conference for},
  pages={70--75},
  year={2015},
  organization={IEEE}
}
@inproceedings{12sivaraman2015network,
  title={Network-level security and privacy control for smart-home {I}o{T} devices},
  author={Sivaraman, Vijay and Gharakheili, Hassan Habibi and Vishwanath, Arun and Boreli, Roksana and Mehani, Olivier},
  booktitle={Wireless and Mobile Computing, Networking and Communications (WiMob), 2015 IEEE 11th International Conference on},
  pages={163--167},
  year={2015},
  organization={IEEE}
}

@inproceedings{13srinivasan2016privacy,
  title={Privacy conscious architecture for improving emergency response in smart cities},
  author={Srinivasan, Ramya and Mohan, Apurva and Srinivasan, Priyanka},
  booktitle={Smart City Security and Privacy Workshop (SCSP-W), 2016},
  pages={1--5},
  year={2016},
  organization={IEEE}
}
@inproceedings{14sadeghi2015security,
  title={Security and privacy challenges in industrial internet of things},
  author={Sadeghi, Ahmad-Reza and Wachsmann, Christian and Waidner, Michael},
  booktitle={Design Automation Conference (DAC), 2015 52nd ACM/EDAC/IEEE},
  pages={1--6},
  year={2015},
  organization={IEEE}
}
@inproceedings{15otgonbayar2016toward,
  title={Toward Anonymizing {I}o{T} Data Streams via Partitioning},
  author={Otgonbayar, Ankhbayar and Pervez, Zeeshan and Dahal, Keshav},
  booktitle={Mobile Ad Hoc and Sensor Systems (MASS), 2016 IEEE 13th International Conference on},
  pages={331--336},
  year={2016},
  organization={IEEE}
}
@inproceedings{16rutledge2016privacy,
  title={Privacy Impacts of {I}o{T} Devices: A SmartTV Case Study},
  author={Rutledge, Richard L and Massey, Aaron K and Ant{\'o}n, Annie I},
  booktitle={Requirements Engineering Conference Workshops (REW), IEEE International},
  pages={261--270},
  year={2016},
  organization={IEEE}
}

@inproceedings{17andrea2015internet,
  title={Internet of Things: Security vulnerabilities and challenges},
  author={Andrea, Ioannis and Chrysostomou, Chrysostomos and Hadjichristofi, George},
  booktitle={Computers and Communication (ISCC), 2015 IEEE Symposium on},
  pages={180--187},
  year={2015},
  organization={IEEE}
}






























%%%%%%%%%%%%%%%%%%%%%%%%%%%%%%%%%%%%%%%%%%%%%%%%%%%%%%%%%%%


@misc{epfl-mobility-20090224,
    author = {Michal Piorkowski and Natasa Sarafijanovic-Djukic and Matthias Grossglauser},
    title = {{CRAWDAD} dataset epfl/mobility (v. 2009-02-24)},
    howpublished = {Downloaded from \url{http://crawdad.org/epfl/mobility/20090224}},
    doi = {10.15783/C7J010},
    month = feb,
    year = 2009
}

@misc{roma-taxi-20140717,
    author = {Lorenzo Bracciale and Marco Bonola and Pierpaolo Loreti and Giuseppe Bianchi and Raul Amici and Antonello Rabuffi},
    title = {{CRAWDAD} dataset roma/taxi (v. 2014-07-17)},
    howpublished = {Downloaded from \url{http://crawdad.org/roma/taxi/20140717}},
    doi = {10.15783/C7QC7M},
    month = jul,
    year = 2014
}

@misc{rice-ad_hoc_city-20030911,
    author = {Jorjeta G. Jetcheva and Yih-Chun Hu and Santashil PalChaudhuri and Amit Kumar Saha and David B. Johnson},
    title = {{CRAWDAD} dataset rice/ad\_hoc\_city (v. 2003-09-11)},
    howpublished = {Downloaded from \url{http://crawdad.org/rice/ad_hoc_city/20030911}},
    doi = {10.15783/C73K5B},
    month = sep,
    year = 2003
}

@misc{china:2012,
author = {Microsoft Research Asia},
title = {GeoLife GPS Trajectories},
year = {2012},
howpublished= {\url{https://www.microsoft.com/en-us/download/details.aspx?id=52367}},
}


@misc{china:2011,
ALTauthor = {Microsoft Research Asia)},
ALTeditor = {},
title = {GeoLife GPS Trajectories,
year = {2012},
url = {https://www.microsoft.com/en-us/download/details.aspx?id=52367},
}


@misc{longversion,
  author = {N. Takbiri and A. Houmansadr and D.L. Goeckel and H. Pishro-Nik},
  title = {{Limits of Location Privacy under Anonymization and Obfuscation}},
  howpublished = "\url{http://www.ecs.umass.edu/ece/pishro/Papers/ISIT_2017-2.pdf}",
  year = 2017,
month= "January",
  note = "Summarized version submitted to IEEE ISIT 2017"
}

@misc{isit_ke,
  author = {K. Li and D. Goeckel and H. Pishro-Nik},
  title = {{Bayesian Time Series Matching and Privacy}},
  note = "submitted to IEEE ISIT 2017"
}

@article{matching,
  title={Asymptotically Optimal Matching of Multiple Sequences to Source Distributions and Training Sequences},
  author={Jayakrishnan Unnikrishnan},
  journal={ IEEE Transactions on Information Theory},
  volume={61},
  number={1},
  pages={452-468},
  year={2015},
  publisher={IEEE}
}


@article{Naini2016,
	Author = {F. Naini and J. Unnikrishnan and P. Thiran and M. Vetterli},
	Journal = {IEEE Transactions on Information Forensics and Security},
	Publisher = {IEEE},
	Title = {Where You Are Is Who You Are: User Identification by Matching Statistics},
	 volume={11},
    number={2},
     pages={358--372},
    Year = {2016}
}



@inproceedings{holowczak2015cachebrowser,
  title={{CacheBrowser: Bypassing Chinese Censorship without Proxies Using Cached Content}},
  author={Holowczak, John and Houmansadr, Amir},
  booktitle={Proceedings of the 22nd ACM SIGSAC Conference on Computer and Communications Security},
  pages={70--83},
  year={2015},
  organization={ACM}
}
@misc{cb-website,
	Howpublished = {\url{https://cachebrowser.net/#/}},
	Title = {{CacheBrowser}},
	key={cachebrowser}
}

@inproceedings{GameOfDecoys,
 title={{GAME OF DECOYS: Optimal Decoy Routing Through Game Theory}},
 author={Milad Nasr and Amir Houmansadr},
 booktitle={The $23^{rd}$ ACM Conference on Computer and Communications Security (CCS)},
 year={2016}
}

@inproceedings{CDNReaper,
 title={{Practical Censorship Evasion Leveraging Content Delivery Networks}},
 author={Hadi Zolfaghari and Amir Houmansadr},
 booktitle={The $23^{rd}$ ACM Conference on Computer and Communications Security (CCS)},
 year={2016}
}

@misc{Leberknight2010,
	Author = {Leberknight, C. and Chiang, M. and Poor, H. and Wong, F.},
	Howpublished = {\url{http://www.princeton.edu/~chiangm/anticensorship.pdf}},
	Title = {{A Taxonomy of Internet Censorship and Anti-censorship}},
	Year = {2010}}

@techreport{ultrasurf-analysis,
	Author = {Appelbaum, Jacob},
	Institution = {The Tor Project},
	Title = {{Technical analysis of the Ultrasurf proxying software}},
	Url = {http://scholar.google.com/scholar?hl=en\&btnG=Search\&q=intitle:Technical+analysis+of+the+Ultrasurf+proxying+software\#0},
	Year = {2012},
	Bdsk-Url-1 = {http://scholar.google.com/scholar?hl=en%5C&btnG=Search%5C&q=intitle:Technical+analysis+of+the+Ultrasurf+proxying+software%5C#0}}

@misc{gifc:07,
	Howpublished = {\url{http://www.internetfreedom.org/archive/Defeat\_Internet\_Censorship\_White\_Paper.pdf}},
	Key = {defeatcensorship},
	Publisher = {Global Internet Freedom Consortium (GIFC)},
	Title = {{Defeat Internet Censorship: Overview of Advanced Technologies and Products}},
	Type = {White Paper},
	Year = {2007}}

@article{pan2011survey,
	Author = {Pan, J. and Paul, S. and Jain, R.},
	Journal = {Communications Magazine, IEEE},
	Number = {7},
	Pages = {26--36},
	Publisher = {IEEE},
	Title = {{A Survey of the Research on Future Internet Architectures}},
	Volume = {49},
	Year = {2011}}

@misc{nsf-fia,
	Howpublished = {\url{http://www.nets-fia.net/}},
	Key = {FIA},
	Title = {{NSF Future Internet Architecture Project}}}

@misc{NDN,
	Howpublished = {\url{http://www.named- data.net}},
	Key = {NDN},
	Title = {{Named Data Networking Project}}}

@inproceedings{MobilityFirst,
	Author = {Seskar, I. and Nagaraja, K. and Nelson, S. and Raychaudhuri, D.},
	Booktitle = {Asian Internet Engineering Conference},
	Title = {{Mobilityfirst Future internet Architecture Project}},
	Year = {2011}}

@incollection{NEBULA,
	Author = {Anderson, T. and Birman, K. and Broberg, R. and Caesar, M. and Comer, D. and Cotton, C. and Freedman, M.~J. and Haeberlen, A. and Ives, Z.~G. and Krishnamurthy, A. and others},
	Booktitle = {The Future Internet},
	Pages = {16--26},
	Publisher = {Springer},
	Title = {{The NEBULA Future Internet Architecture}},
	Year = {2013}}

@inproceedings{XIA,
	Author = {Anand, A. and Dogar, F. and Han, D. and Li, B. and Lim, H. and Machado, M. and Wu, W. and Akella, A. and Andersen, D.~G. and Byers, J.~W. and others},
	Booktitle = {ACM Workshop on Hot Topics in Networks},
	Pages = {2},
	Title = {{XIA: An Architecture for an Evolvable and Trustworthy Internet}},
	Year = {2011}}

@inproceedings{ChoiceNet,
	Author = {Rouskas, G.~N. and Baldine, I. and Calvert, K.~L. and Dutta, R. and Griffioen, J. and Nagurney, A. and Wolf, T.},
	Booktitle = {ONDM},
	Title = {{ChoiceNet: Network Innovation Through Choice}},
	Year = {2013}}

@misc{nsf-find,
	Howpublished = {http://www.nets-find.net/},
	Title = {{NSF NeTS FIND Initiative}}}

@article{traid,
	Author = {Cheriton, D.~R. and Gritter, M.},
	Title = {{TRIAD: A New Next-Generation Internet Architecture}},
	Year = {2000}}

@inproceedings{dona,
	Author = {Koponen, T. and Chawla, M. and Chun, B-G. and Ermolinskiy, A. and Kim, K.~H. and Shenker, S. and Stoica, I.},
	Booktitle = {ACM SIGCOMM Computer Communication Review},
	Number = {4},
	Organization = {ACM},
	Pages = {181--192},
	Title = {{A Data-Oriented (and Beyond) Network Architecture}},
	Volume = {37},
	Year = {2007}}

@misc{ultrasurf,
	Howpublished = {\url{http://www.ultrareach.com}},
	Key = {ultrasurf},
	Title = {{Ultrasurf}}}

@misc{tor-bridge,
	Author = {Dingledine, R. and Mathewson, N.},
	Howpublished = {\url{https://svn.torproject.org/svn/projects/design-paper/blocking.html}},
	Title = {{Design of a Blocking-Resistant Anonymity System}}}

@inproceedings{McLachlanH09,
	Author = {J. McLachlan and N. Hopper},
	Booktitle = {WPES},
	Title = {{On the Risks of Serving Whenever You Surf: Vulnerabilities in Tor's Blocking Resistance Design}},
	Year = {2009}}

@inproceedings{mahdian2010,
	Author = {Mahdian, M.},
	Booktitle = {{Fun with Algorithms}},
	Title = {{Fighting Censorship with Algorithms}},
	Year = {2010}}

@inproceedings{McCoy2011,
	Author = {McCoy, D. and Morales, J.~A. and Levchenko, K.},
	Booktitle = {FC},
	Title = {{Proximax: A Measurement Based System for Proxies Dissemination}},
	Year = {2011}}

@inproceedings{Sovran2008,
	Author = {Sovran, Y. and Libonati, A. and Li, J.},
	Booktitle = {IPTPS},
	Title = {{Pass it on: Social Networks Stymie Censors}},
	Year = {2008}}

@inproceedings{rbridge,
	Author = {Wang, Q. and Lin, Zi and Borisov, N. and Hopper, N.},
	Booktitle = {{NDSS}},
	Title = {{rBridge: User Reputation based Tor Bridge Distribution with Privacy Preservation}},
	Year = {2013}}

@inproceedings{telex,
	Author = {Wustrow, E. and Wolchok, S. and Goldberg, I. and Halderman, J.},
	Booktitle = {{USENIX Security}},
	Title = {{Telex: Anticensorship in the Network Infrastructure}},
	Year = {2011}}

@inproceedings{cirripede,
	Author = {Houmansadr, A. and Nguyen, G. and Caesar, M. and Borisov, N.},
	Booktitle = {CCS},
	Title = {{Cirripede: Circumvention Infrastructure Using Router Redirection with Plausible Deniability}},
	Year = {2011}}

@inproceedings{decoyrouting,
	Author = {Karlin, J. and Ellard, D. and Jackson, A. and Jones, C. and Lauer, G. and Mankins, D. and Strayer, W.},
	Booktitle = {{FOCI}},
	Title = {{Decoy Routing: Toward Unblockable Internet Communication}},
	Year = {2011}}

@inproceedings{routing-around-decoys,
	Author = {M.~Schuchard and J.~Geddes and C.~Thompson and N.~Hopper},
	Booktitle = {{CCS}},
	Title = {{Routing Around Decoys}},
	Year = {2012}}

@inproceedings{parrot,
	Author = {A. Houmansadr and C. Brubaker and V. Shmatikov},
	Booktitle = {IEEE S\&P},
	Title = {{The Parrot is Dead: Observing Unobservable Network Communications}},
	Year = {2013}}

@misc{knock,
	Author = {T. Wilde},
	Howpublished = {\url{https://blog.torproject.org/blog/knock-knock-knockin-bridges-doors}},
	Title = {{Knock Knock Knockin' on Bridges' Doors}},
	Year = {2012}}

@inproceedings{china-tor,
	Author = {Winter, P. and Lindskog, S.},
	Booktitle = {{FOCI}},
	Title = {{How the Great Firewall of China Is Blocking Tor}},
	Year = {2012}}

@misc{discover-bridge,
	Howpublished = {\url{https://blog.torproject.org/blog/research-problems-ten-ways-discover-tor-bridges}},
	Key = {tenways},
	Title = {{Ten Ways to Discover Tor Bridges}}}

@inproceedings{freewave,
	Author = {A.~Houmansadr and T.~Riedl and N.~Borisov and A.~Singer},
	Booktitle = {{NDSS}},
	Title = {{I Want My Voice to Be Heard: IP over Voice-over-IP for Unobservable Censorship Circumvention}},
	Year = 2013}

@inproceedings{censorspoofer,
	Author = {Q. Wang and X. Gong and G. Nguyen and A. Houmansadr and N. Borisov},
	Booktitle = {CCS},
	Title = {{CensorSpoofer: Asymmetric Communication Using IP Spoofing for Censorship-Resistant Web Browsing}},
	Year = {2012}}

@inproceedings{skypemorph,
	Author = {H. Moghaddam and B. Li and M. Derakhshani and I. Goldberg},
	Booktitle = {CCS},
	Title = {{SkypeMorph: Protocol Obfuscation for Tor Bridges}},
	Year = {2012}}

@inproceedings{stegotorus,
	Author = {Weinberg, Z. and Wang, J. and Yegneswaran, V. and Briesemeister, L. and Cheung, S. and Wang, F. and Boneh, D.},
	Booktitle = {CCS},
	Title = {{StegoTorus: A Camouflage Proxy for the Tor Anonymity System}},
	Year = {2012}}

@techreport{dust,
	Author = {{B.~Wiley}},
	Howpublished = {\url{http://blanu.net/ Dust.pdf}},
	Institution = {School of Information, University of Texas at Austin},
	Title = {{Dust: A Blocking-Resistant Internet Transport Protocol}},
	Year = {2011}}

@inproceedings{FTE,
	Author = {K.~Dyer and S.~Coull and T.~Ristenpart and T.~Shrimpton},
	Booktitle = {CCS},
	Title = {{Protocol Misidentification Made Easy with Format-Transforming Encryption}},
	Year = {2013}}

@inproceedings{fp,
	Author = {Fifield, D. and Hardison, N. and Ellithrope, J. and Stark, E. and Dingledine, R. and Boneh, D. and Porras, P.},
	Booktitle = {PETS},
	Title = {{Evading Censorship with Browser-Based Proxies}},
	Year = {2012}}

@misc{obfsproxy,
	Howpublished = {\url{https://www.torproject.org/projects/obfsproxy.html.en}},
	Key = {obfsproxy},
	Publisher = {The Tor Project},
	Title = {{A Simple Obfuscating Proxy}}}

@inproceedings{Tor-instead-of-IP,
	Author = {Liu, V. and Han, S. and Krishnamurthy, A. and Anderson, T.},
	Booktitle = {HotNets},
	Title = {{Tor instead of IP}},
	Year = {2011}}

@misc{roger-slides,
	Howpublished = {\url{https://svn.torproject.org/svn/projects/presentations/slides-28c3.pdf}},
	Key = {torblocking},
	Title = {{How Governments Have Tried to Block Tor}}}

@inproceedings{infranet,
	Author = {Feamster, N. and Balazinska, M. and Harfst, G. and Balakrishnan, H. and Karger, D.},
	Booktitle = {USENIX Security},
	Title = {{Infranet: Circumventing Web Censorship and Surveillance}},
	Year = {2002}}

@inproceedings{collage,
	Author = {S.~Burnett and N.~Feamster and S.~Vempala},
	Booktitle = {USENIX Security},
	Title = {{Chipping Away at Censorship Firewalls with User-Generated Content}},
	Year = {2010}}

@article{anonymizer,
	Author = {Boyan, J.},
	Journal = {Computer-Mediated Communication Magazine},
	Month = sep,
	Number = {9},
	Title = {{The Anonymizer: Protecting User Privacy on the Web}},
	Volume = {4},
	Year = {1997}}

@article{schulze2009internet,
	Author = {Schulze, H. and Mochalski, K.},
	Journal = {IPOQUE Report},
	Pages = {351--362},
	Title = {Internet Study 2008/2009},
	Volume = {37},
	Year = {2009}}

@inproceedings{cya-ccs13,
	Author = {J.~Geddes and M.~Schuchard and N.~Hopper},
	Booktitle = {{CCS}},
	Title = {{Cover Your ACKs: Pitfalls of Covert Channel Censorship Circumvention}},
	Year = {2013}}

@inproceedings{andana,
	Author = {DiBenedetto, S. and Gasti, P. and Tsudik, G. and Uzun, E.},
	Booktitle = {{NDSS}},
	Title = {{ANDaNA: Anonymous Named Data Networking Application}},
	Year = {2012}}

@inproceedings{darkly,
	Author = {Jana, S. and Narayanan, A. and Shmatikov, V.},
	Booktitle = {IEEE S\&P},
	Title = {{A Scanner Darkly: Protecting User Privacy From Perceptual Applications}},
	Year = {2013}}

@inproceedings{NS08,
	Author = {A.~Narayanan and V.~Shmatikov},
	Booktitle = {IEEE S\&P},
	Title = {Robust de-anonymization of large sparse datasets},
	Year = {2008}}

@inproceedings{NS09,
	Author = {A.~Narayanan and V.~Shmatikov},
	Booktitle = {IEEE S\&P},
	Title = {De-anonymizing social networks},
	Year = {2009}}

@inproceedings{memento,
	Author = {Jana, S. and Shmatikov, V.},
	Booktitle = {IEEE S\&P},
	Title = {{Memento: Learning secrets from process footprints}},
	Year = {2012}}

@misc{plugtor,
	Howpublished = {\url{https://www.torproject.org/docs/pluggable-transports.html.en}},
	Key = {PluggableTransports},
	Publisher = {The Tor Project},
	Title = {{Tor: Pluggable transports}}}

@misc{psiphon,
	Author = {J.~Jia and P.~Smith},
	Howpublished = {\url{http://www.cdf.toronto.edu/~csc494h/reports/2004-fall/psiphon_ae.html}},
	Title = {{Psiphon: Analysis and Estimation}},
	Year = 2004}

@misc{china-github,
	Howpublished = {\url{http://mobile.informationweek.com/80269/show/72e30386728f45f56b343ddfd0fdb119/}},
	Key = {github},
	Title = {{China's GitHub Censorship Dilemma}}}

@inproceedings{txbox,
	Author = {Jana, S. and Porter, D. and Shmatikov, V.},
	Booktitle = {IEEE S\&P},
	Title = {{TxBox: Building Secure, Efficient Sandboxes with System Transactions}},
	Year = {2011}}

@inproceedings{airavat,
	Author = {I. Roy and S. Setty and A. Kilzer and V. Shmatikov and E. Witchel},
	Booktitle = {NSDI},
	Title = {{Airavat: Security and Privacy for MapReduce}},
	Year = {2010}}

@inproceedings{osdi12,
	Author = {A. Dunn and M. Lee and S. Jana and S. Kim and M. Silberstein and Y. Xu and V. Shmatikov and E. Witchel},
	Booktitle = {OSDI},
	Title = {{Eternal Sunshine of the Spotless Machine: Protecting Privacy with Ephemeral Channels}},
	Year = {2012}}

@inproceedings{ymal,
	Author = {J. Calandrino and A. Kilzer and A. Narayanan and E. Felten and V. Shmatikov},
	Booktitle = {IEEE S\&P},
	Title = {{``You Might Also Like:'' Privacy Risks of Collaborative Filtering}},
	Year = {2011}}

@inproceedings{srivastava11,
	Author = {V. Srivastava and M. Bond and K. McKinley and V. Shmatikov},
	Booktitle = {PLDI},
	Title = {{A Security Policy Oracle: Detecting Security Holes Using Multiple API Implementations}},
	Year = {2011}}

@inproceedings{chen-oakland10,
	Author = {Chen, S. and Wang, R. and Wang, X. and Zhang, K.},
	Booktitle = {IEEE S\&P},
	Title = {{Side-Channel Leaks in Web Applications: A Reality Today, a Challenge Tomorrow}},
	Year = {2010}}

@book{kerck,
	Author = {Kerckhoffs, A.},
	Publisher = {University Microfilms},
	Title = {{La cryptographie militaire}},
	Year = {1978}}

@inproceedings{foci11,
	Author = {J. Karlin and D. Ellard and A.~Jackson and C.~ Jones and G. Lauer and D. Mankins and W.~T.~Strayer},
	Booktitle = {FOCI},
	Title = {{Decoy Routing: Toward Unblockable Internet Communication}},
	Year = 2011}

@inproceedings{sun02,
	Author = {Sun, Q. and Simon, D.~R. and Wang, Y. and Russell, W. and Padmanabhan, V. and Qiu, L.},
	Booktitle = {IEEE S\&P},
	Title = {{Statistical Identification of Encrypted Web Browsing Traffic}},
	Year = {2002}}

@inproceedings{danezis,
	Author = {Murdoch, S.~J. and Danezis, G.},
	Booktitle = {IEEE S\&P},
	Title = {{Low-Cost Traffic Analysis of Tor}},
	Year = {2005}}

@inproceedings{pakicensorship,
	Author = {Z.~Nabi},
	Booktitle = {FOCI},
	Title = {The Anatomy of {Web} Censorship in {Pakistan}},
	Year = {2013}}

@inproceedings{irancensorship,
	Author = {S.~Aryan and H.~Aryan and A.~Halderman},
	Booktitle = {FOCI},
	Title = {Internet Censorship in {Iran}: {A} First Look},
	Year = {2013}}

@inproceedings{ford10efficient,
	Author = {Amittai Aviram and Shu-Chun Weng and Sen Hu and Bryan Ford},
	Booktitle = {\bibconf[9th]{OSDI}{USENIX Symposium on Operating Systems Design and Implementation}},
	Location = {Vancouver, BC, Canada},
	Month = oct,
	Title = {Efficient System-Enforced Deterministic Parallelism},
	Year = 2010}

@inproceedings{ford10determinating,
	Author = {Amittai Aviram and Sen Hu and Bryan Ford and Ramakrishna Gummadi},
	Booktitle = {\bibconf{CCSW}{ACM Cloud Computing Security Workshop}},
	Location = {Chicago, IL},
	Month = oct,
	Title = {Determinating Timing Channels in Compute Clouds},
	Year = 2010}

@inproceedings{ford12plugging,
	Author = {Bryan Ford},
	Booktitle = {\bibconf[4th]{HotCloud}{USENIX Workshop on Hot Topics in Cloud Computing}},
	Location = {Boston, MA},
	Month = jun,
	Title = {Plugging Side-Channel Leaks with Timing Information Flow Control},
	Year = 2012}

@inproceedings{ford12icebergs,
	Author = {Bryan Ford},
	Booktitle = {\bibconf[4th]{HotCloud}{USENIX Workshop on Hot Topics in Cloud Computing}},
	Location = {Boston, MA},
	Month = jun,
	Title = {Icebergs in the Clouds: the {\em Other} Risks of Cloud Computing},
	Year = 2012}

@misc{mullenize,
	Author = {Washington Post},
	Howpublished = {\url{http://apps.washingtonpost.com/g/page/world/gchq-report-on-mullenize-program-to-stain-anonymous-electronic-traffic/502/}},
	Month = {oct},
	Title = {{GCHQ} report on {`MULLENIZE'} program to `stain' anonymous electronic traffic},
	Year = {2013}}

@inproceedings{shue13street,
	Author = {Craig A. Shue and Nathanael Paul and Curtis R. Taylor},
	Booktitle = {\bibbrev[7th]{WOOT}{USENIX Workshop on Offensive Technologies}},
	Month = aug,
	Title = {From an {IP} Address to a Street Address: Using Wireless Signals to Locate a Target},
	Year = 2013}

@inproceedings{knockel11three,
	Author = {Jeffrey Knockel and Jedidiah R. Crandall and Jared Saia},
	Booktitle = {\bibbrev{FOCI}{USENIX Workshop on Free and Open Communications on the Internet}},
	Location = {San Francisco, CA},
	Month = aug,
	Year = 2011}

@misc{rfc4960,
	Author = {R. {Stewart, ed.}},
	Month = sep,
	Note = {RFC 4960},
	Title = {Stream Control Transmission Protocol},
	Year = 2007}

@inproceedings{ford07structured,
	Author = {Bryan Ford},
	Booktitle = {\bibbrev{SIGCOMM}{ACM SIGCOMM}},
	Location = {Kyoto, Japan},
	Month = aug,
	Title = {Structured Streams: a New Transport Abstraction},
	Year = {2007}}

@misc{spdy,
	Author = {Google, Inc.},
	Note = {\url{http://www.chromium.org/spdy/spdy-whitepaper}},
	Title = {{SPDY}: An Experimental Protocol For a Faster {Web}}}

@misc{quic,
	Author = {Jim Roskind},
	Month = jun,
	Note = {\url{http://blog.chromium.org/2013/06/experimenting-with-quic.html}},
	Title = {Experimenting with {QUIC}},
	Year = 2013}

@misc{podjarny12not,
	Author = {G.~Podjarny},
	Month = jun,
	Note = {\url{http://www.guypo.com/technical/not-as-spdy-as-you-thought/}},
	Title = {{Not as SPDY as You Thought}},
	Year = 2012}

@inproceedings{cor,
	Author = {Jones, N.~A. and Arye, M. and Cesareo, J. and Freedman, M.~J.},
	Booktitle = {FOCI},
	Title = {{Hiding Amongst the Clouds: A Proposal for Cloud-based Onion Routing}},
	Year = {2011}}

@misc{torcloud,
	Howpublished = {\url{https://cloud.torproject.org/}},
	Key = {tor cloud},
	Title = {{The Tor Cloud Project}}}

@inproceedings{scramblesuit,
	Author = {Philipp Winter and Tobias Pulls and Juergen Fuss},
	Booktitle = {WPES},
	Title = {{ScrambleSuit: A Polymorphic Network Protocol to Circumvent Censorship}},
	Year = 2013}

@article{savage2000practical,
	Author = {Savage, S. and Wetherall, D. and Karlin, A. and Anderson, T.},
	Journal = {ACM SIGCOMM Computer Communication Review},
	Number = {4},
	Pages = {295--306},
	Publisher = {ACM},
	Title = {Practical network support for IP traceback},
	Volume = {30},
	Year = {2000}}

@inproceedings{ooni,
	Author = {Filast, A. and Appelbaum, J.},
	Booktitle = {{FOCI}},
	Title = {{OONI : Open Observatory of Network Interference}},
	Year = {2012}}

@misc{caida-rank,
	Howpublished = {\url{http://as-rank.caida.org/}},
	Key = {caida rank},
	Title = {{AS Rank: AS Ranking}}}

@inproceedings{usersrouted-ccs13,
	Author = {A.~Johnson and C.~Wacek and R.~Jansen and M.~Sherr and P.~Syverson},
	Booktitle = {CCS},
	Title = {{Users Get Routed: Traffic Correlation on Tor by Realistic Adversaries}},
	Year = {2013}}

@inproceedings{edman2009awareness,
	Author = {Edman, M. and Syverson, P.},
	Booktitle = {{CCS}},
	Title = {{AS-awareness in Tor path selection}},
	Year = {2009}}

@inproceedings{DecoyCosts,
	Author = {A.~Houmansadr and E.~L.~Wong and V.~Shmatikov},
	Booktitle = {NDSS},
	Title = {{No Direction Home: The True Cost of Routing Around Decoys}},
	Year = {2014}}

@article{cordon,
	Author = {Elahi, T. and Goldberg, I.},
	Journal = {University of Waterloo CACR},
	Title = {{CORDON--A Taxonomy of Internet Censorship Resistance Strategies}},
	Volume = {33},
	Year = {2012}}

@inproceedings{privex,
	Author = {T.~Elahi and G.~Danezis and I.~Goldberg	},
	Booktitle = {{CCS}},
	Title = {{AS-awareness in Tor path selection}},
	Year = {2014}}

@inproceedings{changeGuards,
	Author = {T.~Elahi and K.~Bauer and M.~AlSabah and R.~Dingledine and I.~Goldberg},
	Booktitle = {{WPES}},
	Title = {{ Changing of the Guards: Framework for Understanding and Improving Entry Guard Selection in Tor}},
	Year = {2012}}

@article{RAINBOW:Journal,
	Author = {A.~Houmansadr and N.~Kiyavash and N.~Borisov},
	Journal = {IEEE/ACM Transactions on Networking},
	Title = {{Non-Blind Watermarking of Network Flows}},
	Year = 2014}

@inproceedings{info-tod,
	Author = {A.~Houmansadr and S.~Gorantla and T.~Coleman and N.~Kiyavash and and N.~Borisov},
	Booktitle = {{CCS (poster session)}},
	Title = {{On the Channel Capacity of Network Flow Watermarking}},
	Year = {2009}}

@inproceedings{johnson2014game,
	Author = {Johnson, B. and Laszka, A. and Grossklags, J. and Vasek, M. and Moore, T.},
	Booktitle = {Workshop on Bitcoin Research},
	Title = {{Game-theoretic Analysis of DDoS Attacks Against Bitcoin Mining Pools}},
	Year = {2014}}

@incollection{laszka2013mitigation,
	Author = {Laszka, A. and Johnson, B. and Grossklags, J.},
	Booktitle = {Decision and Game Theory for Security},
	Pages = {175--191},
	Publisher = {Springer},
	Title = {{Mitigation of Targeted and Non-targeted Covert Attacks as a Timing Game}},
	Year = {2013}}

@inproceedings{schottle2013game,
	Author = {Schottle, P. and Laszka, A. and Johnson, B. and Grossklags, J. and Bohme, R.},
	Booktitle = {EUSIPCO},
	Title = {{A Game-theoretic Analysis of Content-adaptive Steganography with Independent Embedding}},
	Year = {2013}}

@inproceedings{CloudTransport,
	Author = {C.~Brubaker and A.~Houmansadr and V.~Shmatikov},
	Booktitle = {PETS},
	Title = {{CloudTransport: Using Cloud Storage for Censorship-Resistant Networking}},
	Year = {2014}}

@inproceedings{sweet,
	Author = {W.~Zhou and A.~Houmansadr and M.~Caesar and N.~Borisov},
	Booktitle = {HotPETs},
	Title = {{SWEET: Serving the Web by Exploiting Email Tunnels}},
	Year = {2013}}

@inproceedings{ahsan2002practical,
	Author = {Ahsan, K. and Kundur, D.},
	Booktitle = {Workshop on Multimedia Security},
	Title = {{Practical data hiding in TCP/IP}},
	Year = {2002}}

@incollection{danezis2011covert,
	Author = {Danezis, G.},
	Booktitle = {Security Protocols XVI},
	Pages = {198--214},
	Publisher = {Springer},
	Title = {{Covert Communications Despite Traffic Data Retention}},
	Year = {2011}}

@inproceedings{liu2009hide,
	Author = {Liu, Y. and Ghosal, D. and Armknecht, F. and Sadeghi, A.-R. and Schulz, S. and Katzenbeisser, S.},
	Booktitle = {ESORICS},
	Title = {{Hide and Seek in Time---Robust Covert Timing Channels}},
	Year = {2009}}

@misc{image-watermark-fing,
	Author = {Jonathan Bailey},
	Howpublished = {\url{https://www.plagiarismtoday.com/2009/12/02/image-detection-watermarking-vs-fingerprinting/}},
	Title = {{Image Detection: Watermarking vs. Fingerprinting}},
	Year = {2009}}

@inproceedings{Servetto98,
	Author = {S. D. Servetto and C. I. Podilchuk and K. Ramchandran},
	Booktitle = {Int. Conf. Image Processing},
	Title = {Capacity issues in digital image watermarking},
	Year = {1998}}

@inproceedings{Chen01,
	Author = {B. Chen and G.W.Wornell},
	Booktitle = {IEEE Trans. Inform. Theory},
	Pages = {1423--1443},
	Title = {Quantization index modulation: A class of provably good methods for digital watermarking and information embedding},
	Year = {2001}}

@inproceedings{Karakos00,
	Author = {D. Karakos and A. Papamarcou},
	Booktitle = {IEEE Int. Symp. Information Theory},
	Pages = {47},
	Title = {Relationship between quantization and distribution rates of digitally watermarked data},
	Year = {2000}}

@inproceedings{Sullivan98,
	Author = {J. A. OSullivan and P. Moulin and J. M. Ettinger},
	Booktitle = {IEEE Int. Symp. Information Theory},
	Pages = {297},
	Title = {Information theoretic analysis of steganography},
	Year = {1998}}

@inproceedings{Merhav00,
	Author = {N. Merhav},
	Booktitle = {IEEE Trans. Inform. Theory},
	Pages = {420--430},
	Title = {On random coding error exponents of watermarking systems},
	Year = {2000}}

@inproceedings{Somekh01,
	Author = {A. Somekh-Baruch and N. Merhav},
	Booktitle = {IEEE Int. Symp. Information Theory},
	Pages = {7},
	Title = {On the error exponent and capacity games of private watermarking systems},
	Year = {2001}}

@inproceedings{Steinberg01,
	Author = {Y. Steinberg and N. Merhav},
	Booktitle = {IEEE Trans. Inform. Theory},
	Pages = {1410--1422},
	Title = {Identification in the presence of side information with application to watermarking},
	Year = {2001}}

@article{Moulin03,
	Author = {P. Moulin and J.A. O'Sullivan},
	Journal = {IEEE Trans. Info. Theory},
	Number = {3},
	Title = {Information-theoretic analysis of information hiding},
	Volume = 49,
	Year = 2003}

@article{Gelfand80,
	Author = {S.I.~Gelfand and M.S.~Pinsker},
	Journal = {Problems of Control and Information Theory},
	Number = {1},
	Pages = {19-31},
	Title = {{Coding for channel with random parameters}},
	Url = {citeseer.ist.psu.edu/anantharam96bits.html},
	Volume = {9},
	Year = {1980},
	Bdsk-Url-1 = {citeseer.ist.psu.edu/anantharam96bits.html}}

@book{Wolfowitz78,
	Author = {J. Wolfowitz},
	Edition = {3rd},
	Location = {New York},
	Publisher = {Springer-Verlag},
	Title = {Coding Theorems of Information Theory},
	Year = 1978}

@article{caire99,
	Author = {G. Caire and S. Shamai},
	Journal = {IEEE Transactions on Information Theory},
	Number = {6},
	Pages = {2007--2019},
	Title = {On the Capacity of Some Channels with Channel State Information},
	Volume = {45},
	Year = {1999}}

@inproceedings{wright2007language,
	Author = {Wright, Charles V and Ballard, Lucas and Monrose, Fabian and Masson, Gerald M},
	Booktitle = {USENIX Security},
	Title = {{Language identification of encrypted VoIP traffic: Alejandra y Roberto or Alice and Bob?}},
	Year = {2007}}

@inproceedings{backes2010speaker,
	Author = {Backes, Michael and Doychev, Goran and D{\"u}rmuth, Markus and K{\"o}pf, Boris},
	Booktitle = {{European Symposium on Research in Computer Security (ESORICS)}},
	Pages = {508--523},
	Publisher = {Springer},
	Title = {{Speaker Recognition in Encrypted Voice Streams}},
	Year = {2010}}

@phdthesis{lu2009traffic,
	Author = {Lu, Yuanchao},
	School = {Cleveland State University},
	Title = {{On Traffic Analysis Attacks to Encrypted VoIP Calls}},
	Year = {2009}}

@inproceedings{wright2008spot,
	Author = {Wright, Charles V and Ballard, Lucas and Coull, Scott E and Monrose, Fabian and Masson, Gerald M},
	Booktitle = {IEEE Symposium on Security and Privacy},
	Pages = {35--49},
	Title = {Spot me if you can: Uncovering spoken phrases in encrypted VoIP conversations},
	Year = {2008}}

@inproceedings{white2011phonotactic,
	Author = {White, Andrew M and Matthews, Austin R and Snow, Kevin Z and Monrose, Fabian},
	Booktitle = {IEEE Symposium on Security and Privacy},
	Pages = {3--18},
	Title = {Phonotactic reconstruction of encrypted VoIP conversations: Hookt on fon-iks},
	Year = {2011}}

@inproceedings{fancy,
	Author = {Houmansadr, Amir and Borisov, Nikita},
	Booktitle = {Privacy Enhancing Technologies},
	Organization = {Springer},
	Pages = {205--224},
	Title = {The Need for Flow Fingerprints to Link Correlated Network Flows},
	Year = {2013}}

@article{botmosaic,
	Author = {Amir Houmansadr and Nikita Borisov},
	Doi = {10.1016/j.jss.2012.11.005},
	Issn = {0164-1212},
	Journal = {Journal of Systems and Software},
	Keywords = {Network security},
	Number = {3},
	Pages = {707 - 715},
	Title = {BotMosaic: Collaborative network watermark for the detection of IRC-based botnets},
	Url = {http://www.sciencedirect.com/science/article/pii/S0164121212003068},
	Volume = {86},
	Year = {2013},
	Bdsk-Url-1 = {http://www.sciencedirect.com/science/article/pii/S0164121212003068},
	Bdsk-Url-2 = {http://dx.doi.org/10.1016/j.jss.2012.11.005}}

@inproceedings{ramsbrock2008first,
	Author = {Ramsbrock, Daniel and Wang, Xinyuan and Jiang, Xuxian},
	Booktitle = {Recent Advances in Intrusion Detection},
	Organization = {Springer},
	Pages = {59--77},
	Title = {A first step towards live botmaster traceback},
	Year = {2008}}

@inproceedings{potdar2005survey,
	Author = {Potdar, Vidyasagar M and Han, Song and Chang, Elizabeth},
	Booktitle = {Industrial Informatics, 2005. INDIN'05. 2005 3rd IEEE International Conference on},
	Organization = {IEEE},
	Pages = {709--716},
	Title = {A survey of digital image watermarking techniques},
	Year = {2005}}

@book{cole2003hiding,
	Author = {Cole, Eric and Krutz, Ronald D},
	Publisher = {John Wiley \& Sons, Inc.},
	Title = {Hiding in plain sight: Steganography and the art of covert communication},
	Year = {2003}}

@incollection{akaike1998information,
	Author = {Akaike, Hirotogu},
	Booktitle = {Selected Papers of Hirotugu Akaike},
	Pages = {199--213},
	Publisher = {Springer},
	Title = {Information theory and an extension of the maximum likelihood principle},
	Year = {1998}}

@misc{central-command-hack,
	Author = {Everett Rosenfeld},
	Howpublished = {\url{http://www.cnbc.com/id/102330338}},
	Title = {{FBI investigating Central Command Twitter hack}},
	Year = {2015}}

@misc{sony-psp-ddos,
	Howpublished = {\url{http://n4g.com/news/1644853/sony-and-microsoft-cant-do-much-ddos-attacks-explained}},
	Key = {sony},
	Month = {December},
	Title = {{Sony and Microsoft cant do much -- DDoS attacks explained}},
	Year = {2014}}

@misc{sony-hack,
	Author = {David Bloom},
	Howpublished = {\url{http://goo.gl/MwR4A7}},
	Title = {{Online Game Networks Hacked, Sony Unit President Threatened}},
	Year = {2014}}

@misc{home-depot,
	Author = {Dune Lawrence},
	Howpublished = {\url{http://www.businessweek.com/articles/2014-09-02/home-depots-credit-card-breach-looks-just-like-the-target-hack}},
	Title = {{Home Depot's Suspected Breach Looks Just Like the Target Hack}},
	Year = {2014}}

@misc{target,
	Author = {Julio Ojeda-Zapata},
	Howpublished = {\url{http://www.mercurynews.com/business/ci_24765398/how-did-hackers-pull-off-target-data-heist}},
	Title = {{Target hack: How did they do it?}},
	Year = {2014}}


@article{probabilitycourse,
	Author = {H. Pishro-Nik},
	note = {\url{http://www.probabilitycourse.com}},
	Title = {Introduction to probability, statistics, and random processes},
    Year = {2014}}



@inproceedings{shokri2011quantifying,
	Author = {Shokri, Reza and Theodorakopoulos, George and Le Boudec, Jean-Yves and Hubaux, Jean-Pierre},
	Booktitle = {Security and Privacy (SP), 2011 IEEE Symposium on},
	Organization = {IEEE},
	Pages = {247--262},
	Title = {Quantifying location privacy},
	Year = {2011}}

@inproceedings{hoh2007preserving,
	Author = {Hoh, Baik and Gruteser, Marco and Xiong, Hui and Alrabady, Ansaf},
	Booktitle = {Proceedings of the 14th ACM conference on Computer and communications security},
	Organization = {ACM},
	Pages = {161--171},
	Title = {Preserving privacy in gps traces via uncertainty-aware path cloaking},
	Year = {2007}}



@article{kafsi2013entropy,
	Author = {Kafsi, Mohamed and Grossglauser, Matthias and Thiran, Patrick},
	Journal = {Information Theory, IEEE Transactions on},
	Number = {9},
	Pages = {5577--5583},
	Publisher = {IEEE},
	Title = {The entropy of conditional Markov trajectories},
	Volume = {59},
	Year = {2013}}

@inproceedings{gruteser2003anonymous,
	Author = {Gruteser, Marco and Grunwald, Dirk},
	Booktitle = {Proceedings of the 1st international conference on Mobile systems, applications and services},
	Organization = {ACM},
	Pages = {31--42},
	Title = {Anonymous usage of location-based services through spatial and temporal cloaking},
	Year = {2003}}

@inproceedings{husted2010mobile,
	Author = {Husted, Nathaniel and Myers, Steven},
	Booktitle = {Proceedings of the 17th ACM conference on Computer and communications security},
	Organization = {ACM},
	Pages = {85--96},
	Title = {Mobile location tracking in metro areas: malnets and others},
	Year = {2010}}

@inproceedings{li2009tradeoff,
	Author = {Li, Tiancheng and Li, Ninghui},
	Booktitle = {Proceedings of the 15th ACM SIGKDD international conference on Knowledge discovery and data mining},
	Organization = {ACM},
	Pages = {517--526},
	Title = {On the tradeoff between privacy and utility in data publishing},
	Year = {2009}}

@inproceedings{ma2009location,
	Author = {Ma, Zhendong and Kargl, Frank and Weber, Michael},
	Booktitle = {Sarnoff Symposium, 2009. SARNOFF'09. IEEE},
	Organization = {IEEE},
	Pages = {1--6},
	Title = {A location privacy metric for v2x communication systems},
	Year = {2009}}

@inproceedings{shokri2012protecting,
	Author = {Shokri, Reza and Theodorakopoulos, George and Troncoso, Carmela and Hubaux, Jean-Pierre and Le Boudec, Jean-Yves},
	Booktitle = {Proceedings of the 2012 ACM conference on Computer and communications security},
	Organization = {ACM},
	Pages = {617--627},
	Title = {Protecting location privacy: optimal strategy against localization attacks},
	Year = {2012}}

@inproceedings{freudiger2009non,
	Author = {Freudiger, Julien and Manshaei, Mohammad Hossein and Hubaux, Jean-Pierre and Parkes, David C},
	Booktitle = {Proceedings of the 16th ACM conference on Computer and communications security},
	Organization = {ACM},
	Pages = {324--337},
	Title = {On non-cooperative location privacy: a game-theoretic analysis},
	Year = {2009}}

@incollection{humbert2010tracking,
	Author = {Humbert, Mathias and Manshaei, Mohammad Hossein and Freudiger, Julien and Hubaux, Jean-Pierre},
	Booktitle = {Decision and Game Theory for Security},
	Pages = {38--57},
	Publisher = {Springer},
	Title = {Tracking games in mobile networks},
	Year = {2010}}

@article{manshaei2013game,
	Author = {Manshaei, Mohammad Hossein and Zhu, Quanyan and Alpcan, Tansu and Bac{\c{s}}ar, Tamer and Hubaux, Jean-Pierre},
	Journal = {ACM Computing Surveys (CSUR)},
	Number = {3},
	Pages = {25},
	Publisher = {ACM},
	Title = {Game theory meets network security and privacy},
	Volume = {45},
	Year = {2013}}

@article{palamidessi2006probabilistic,
	Author = {Palamidessi, Catuscia},
	Journal = {Electronic Notes in Theoretical Computer Science},
	Pages = {33--42},
	Publisher = {Elsevier},
	Title = {Probabilistic and nondeterministic aspects of anonymity},
	Volume = {155},
	Year = {2006}}

@inproceedings{mokbel2006new,
	Author = {Mokbel, Mohamed F and Chow, Chi-Yin and Aref, Walid G},
	Booktitle = {Proceedings of the 32nd international conference on Very large data bases},
	Organization = {VLDB Endowment},
	Pages = {763--774},
	Title = {The new Casper: query processing for location services without compromising privacy},
	Year = {2006}}

@article{kalnis2007preventing,
	Author = {Kalnis, Panos and Ghinita, Gabriel and Mouratidis, Kyriakos and Papadias, Dimitris},
	Journal = {Knowledge and Data Engineering, IEEE Transactions on},
	Number = {12},
	Pages = {1719--1733},
	Publisher = {IEEE},
	Title = {Preventing location-based identity inference in anonymous spatial queries},
	Volume = {19},
	Year = {2007}}
	
@article{freudiger2007mix,
  title={Mix-zones for location privacy in vehicular networks},
  author={Freudiger, Julien and Raya, Maxim and F{\'e}legyh{\'a}zi, M{\'a}rk and Papadimitratos, Panos and Hubaux, Jean-Pierre},
  year={2007}
}
@article{sweeney2002k,
	Author = {Sweeney, Latanya},
	Journal = {International Journal of Uncertainty, Fuzziness and Knowledge-Based Systems},
	Number = {05},
	Pages = {557--570},
	Publisher = {World Scientific},
	Title = {k-anonymity: A model for protecting privacy},
	Volume = {10},
	Year = {2002}}

@article{sweeney2002achieving,
	Author = {Sweeney, Latanya},
	Journal = {International Journal of Uncertainty, Fuzziness and Knowledge-Based Systems},
	Number = {05},
	Pages = {571--588},
	Publisher = {World Scientific},
	Title = {Achieving k-anonymity privacy protection using generalization and suppression},
	Volume = {10},
	Year = {2002}}

@inproceedings{niu2014achieving,
	Author = {Niu, Ben and Li, Qinghua and Zhu, Xiaoyan and Cao, Guohong and Li, Hui},
	Booktitle = {INFOCOM, 2014 Proceedings IEEE},
	Organization = {IEEE},
	Pages = {754--762},
	Title = {Achieving k-anonymity in privacy-aware location-based services},
	Year = {2014}}

@inproceedings{liu2013game,
	Author = {Liu, Xinxin and Liu, Kaikai and Guo, Linke and Li, Xiaolin and Fang, Yuguang},
	Booktitle = {INFOCOM, 2013 Proceedings IEEE},
	Organization = {IEEE},
	Pages = {2985--2993},
	Title = {A game-theoretic approach for achieving k-anonymity in location based services},
	Year = {2013}}

@inproceedings{kido2005protection,
	Author = {Kido, Hidetoshi and Yanagisawa, Yutaka and Satoh, Tetsuji},
	Booktitle = {Data Engineering Workshops, 2005. 21st International Conference on},
	Organization = {IEEE},
	Pages = {1248--1248},
	Title = {Protection of location privacy using dummies for location-based services},
	Year = {2005}}

@inproceedings{gedik2005location,
	Author = {Gedik, Bu{\u{g}}ra and Liu, Ling},
	Booktitle = {Distributed Computing Systems, 2005. ICDCS 2005. Proceedings. 25th IEEE International Conference on},
	Organization = {IEEE},
	Pages = {620--629},
	Title = {Location privacy in mobile systems: A personalized anonymization model},
	Year = {2005}}

@inproceedings{bordenabe2014optimal,
	Author = {Bordenabe, Nicol{\'a}s E and Chatzikokolakis, Konstantinos and Palamidessi, Catuscia},
	Booktitle = {Proceedings of the 2014 ACM SIGSAC Conference on Computer and Communications Security},
	Organization = {ACM},
	Pages = {251--262},
	Title = {Optimal geo-indistinguishable mechanisms for location privacy},
	Year = {2014}}

@incollection{duckham2005formal,
	Author = {Duckham, Matt and Kulik, Lars},
	Booktitle = {Pervasive computing},
	Pages = {152--170},
	Publisher = {Springer},
	Title = {A formal model of obfuscation and negotiation for location privacy},
	Year = {2005}}

@inproceedings{kido2005anonymous,
	Author = {Kido, Hidetoshi and Yanagisawa, Yutaka and Satoh, Tetsuji},
	Booktitle = {Pervasive Services, 2005. ICPS'05. Proceedings. International Conference on},
	Organization = {IEEE},
	Pages = {88--97},
	Title = {An anonymous communication technique using dummies for location-based services},
	Year = {2005}}

@incollection{duckham2006spatiotemporal,
	Author = {Duckham, Matt and Kulik, Lars and Birtley, Athol},
	Booktitle = {Geographic Information Science},
	Pages = {47--64},
	Publisher = {Springer},
	Title = {A spatiotemporal model of strategies and counter strategies for location privacy protection},
	Year = {2006}}

@inproceedings{shankar2009privately,
	Author = {Shankar, Pravin and Ganapathy, Vinod and Iftode, Liviu},
	Booktitle = {Proceedings of the 11th international conference on Ubiquitous computing},
	Organization = {ACM},
	Pages = {31--40},
	Title = {Privately querying location-based services with SybilQuery},
	Year = {2009}}

@inproceedings{chow2009faking,
	Author = {Chow, Richard and Golle, Philippe},
	Booktitle = {Proceedings of the 8th ACM workshop on Privacy in the electronic society},
	Organization = {ACM},
	Pages = {105--108},
	Title = {Faking contextual data for fun, profit, and privacy},
	Year = {2009}}

@incollection{xue2009location,
	Author = {Xue, Mingqiang and Kalnis, Panos and Pung, Hung Keng},
	Booktitle = {Location and Context Awareness},
	Pages = {70--87},
	Publisher = {Springer},
	Title = {Location diversity: Enhanced privacy protection in location based services},
	Year = {2009}}

@article{wernke2014classification,
	Author = {Wernke, Marius and Skvortsov, Pavel and D{\"u}rr, Frank and Rothermel, Kurt},
	Journal = {Personal and Ubiquitous Computing},
	Number = {1},
	Pages = {163--175},
	Publisher = {Springer-Verlag},
	Title = {A classification of location privacy attacks and approaches},
	Volume = {18},
	Year = {2014}}

@misc{cai2015cloaking,
	Author = {Cai, Y. and Xu, G.},
	Month = jan # {~1},
	Note = {US Patent App. 14/472,462},
	Publisher = {Google Patents},
	Title = {Cloaking with footprints to provide location privacy protection in location-based services},
	Url = {https://www.google.com/patents/US20150007341},
	Year = {2015},
	Bdsk-Url-1 = {https://www.google.com/patents/US20150007341}}

@article{gedik2008protecting,
	Author = {Gedik, Bu{\u{g}}ra and Liu, Ling},
	Journal = {Mobile Computing, IEEE Transactions on},
	Number = {1},
	Pages = {1--18},
	Publisher = {IEEE},
	Title = {Protecting location privacy with personalized k-anonymity: Architecture and algorithms},
	Volume = {7},
	Year = {2008}}

@article{kalnis2006preserving,
	Author = {Kalnis, Panos and Ghinita, Gabriel and Mouratidis, Kyriakos and Papadias, Dimitris},
	Publisher = {TRB6/06},
	Title = {Preserving anonymity in location based services},
	Year = {2006}}

@inproceedings{hoh2005protecting,
	Author = {Hoh, Baik and Gruteser, Marco},
	Booktitle = {Security and Privacy for Emerging Areas in Communications Networks, 2005. SecureComm 2005. First International Conference on},
	Organization = {IEEE},
	Pages = {194--205},
	Title = {Protecting location privacy through path confusion},
	Year = {2005}}

@article{terrovitis2011privacy,
	Author = {Terrovitis, Manolis},
	Journal = {ACM SIGKDD Explorations Newsletter},
	Number = {1},
	Pages = {6--18},
	Publisher = {ACM},
	Title = {Privacy preservation in the dissemination of location data},
	Volume = {13},
	Year = {2011}}

@article{shin2012privacy,
	Author = {Shin, Kang G and Ju, Xiaoen and Chen, Zhigang and Hu, Xin},
	Journal = {Wireless Communications, IEEE},
	Number = {1},
	Pages = {30--39},
	Publisher = {IEEE},
	Title = {Privacy protection for users of location-based services},
	Volume = {19},
	Year = {2012}}

@article{khoshgozaran2011location,
	Author = {Khoshgozaran, Ali and Shahabi, Cyrus and Shirani-Mehr, Houtan},
	Journal = {Knowledge and Information Systems},
	Number = {3},
	Pages = {435--465},
	Publisher = {Springer},
	Title = {Location privacy: going beyond K-anonymity, cloaking and anonymizers},
	Volume = {26},
	Year = {2011}}

@incollection{chatzikokolakis2015geo,
	Author = {Chatzikokolakis, Konstantinos and Palamidessi, Catuscia and Stronati, Marco},
	Booktitle = {Distributed Computing and Internet Technology},
	Pages = {49--72},
	Publisher = {Springer},
	Title = {Geo-indistinguishability: A Principled Approach to Location Privacy},
	Year = {2015}}

@inproceedings{ngo2015location,
	Author = {Ngo, Hoa and Kim, Jong},
	Booktitle = {Computer Security Foundations Symposium (CSF), 2015 IEEE 28th},
	Organization = {IEEE},
	Pages = {63--74},
	Title = {Location Privacy via Differential Private Perturbation of Cloaking Area},
	Year = {2015}}

@inproceedings{palanisamy2011mobimix,
	Author = {Palanisamy, Balaji and Liu, Ling},
	Booktitle = {Data Engineering (ICDE), 2011 IEEE 27th International Conference on},
	Organization = {IEEE},
	Pages = {494--505},
	Title = {Mobimix: Protecting location privacy with mix-zones over road networks},
	Year = {2011}}

@inproceedings{um2010advanced,
	Author = {Um, Jung-Ho and Kim, Hee-Dae and Chang, Jae-Woo},
	Booktitle = {Social Computing (SocialCom), 2010 IEEE Second International Conference on},
	Organization = {IEEE},
	Pages = {1093--1098},
	Title = {An advanced cloaking algorithm using Hilbert curves for anonymous location based service},
	Year = {2010}}

@inproceedings{bamba2008supporting,
	Author = {Bamba, Bhuvan and Liu, Ling and Pesti, Peter and Wang, Ting},
	Booktitle = {Proceedings of the 17th international conference on World Wide Web},
	Organization = {ACM},
	Pages = {237--246},
	Title = {Supporting anonymous location queries in mobile environments with privacygrid},
	Year = {2008}}

@inproceedings{zhangwei2010distributed,
	Author = {Zhangwei, Huang and Mingjun, Xin},
	Booktitle = {Networks Security Wireless Communications and Trusted Computing (NSWCTC), 2010 Second International Conference on},
	Organization = {IEEE},
	Pages = {468--471},
	Title = {A distributed spatial cloaking protocol for location privacy},
	Volume = {2},
	Year = {2010}}

@article{chow2011spatial,
	Author = {Chow, Chi-Yin and Mokbel, Mohamed F and Liu, Xuan},
	Journal = {GeoInformatica},
	Number = {2},
	Pages = {351--380},
	Publisher = {Springer},
	Title = {Spatial cloaking for anonymous location-based services in mobile peer-to-peer environments},
	Volume = {15},
	Year = {2011}}

@inproceedings{lu2008pad,
	Author = {Lu, Hua and Jensen, Christian S and Yiu, Man Lung},
	Booktitle = {Proceedings of the Seventh ACM International Workshop on Data Engineering for Wireless and Mobile Access},
	Organization = {ACM},
	Pages = {16--23},
	Title = {Pad: privacy-area aware, dummy-based location privacy in mobile services},
	Year = {2008}}

@incollection{khoshgozaran2007blind,
	Author = {Khoshgozaran, Ali and Shahabi, Cyrus},
	Booktitle = {Advances in Spatial and Temporal Databases},
	Pages = {239--257},
	Publisher = {Springer},
	Title = {Blind evaluation of nearest neighbor queries using space transformation to preserve location privacy},
	Year = {2007}}

@inproceedings{ghinita2008private,
	Author = {Ghinita, Gabriel and Kalnis, Panos and Khoshgozaran, Ali and Shahabi, Cyrus and Tan, Kian-Lee},
	Booktitle = {Proceedings of the 2008 ACM SIGMOD international conference on Management of data},
	Organization = {ACM},
	Pages = {121--132},
	Title = {Private queries in location based services: anonymizers are not necessary},
	Year = {2008}}

@article{paulet2014privacy,
	Author = {Paulet, Russell and Kaosar, Md Golam and Yi, Xun and Bertino, Elisa},
	Journal = {Knowledge and Data Engineering, IEEE Transactions on},
	Number = {5},
	Pages = {1200--1210},
	Publisher = {IEEE},
	Title = {Privacy-preserving and content-protecting location based queries},
	Volume = {26},
	Year = {2014}}

@article{nguyen2013differential,
	Author = {Nguyen, Hiep H and Kim, Jong and Kim, Yoonho},
	Journal = {Journal of Computing Science and Engineering},
	Number = {3},
	Pages = {177--186},
	Title = {Differential privacy in practice},
	Volume = {7},
	Year = {2013}}

@inproceedings{lee2012differential,
	Author = {Lee, Jaewoo and Clifton, Chris},
	Booktitle = {Proceedings of the 18th ACM SIGKDD international conference on Knowledge discovery and data mining},
	Organization = {ACM},
	Pages = {1041--1049},
	Title = {Differential identifiability},
	Year = {2012}}

@inproceedings{andres2013geo,
	Author = {Andr{\'e}s, Miguel E and Bordenabe, Nicol{\'a}s E and Chatzikokolakis, Konstantinos and Palamidessi, Catuscia},
	Booktitle = {Proceedings of the 2013 ACM SIGSAC conference on Computer \& communications security},
	Organization = {ACM},
	Pages = {901--914},
	Title = {Geo-indistinguishability: Differential privacy for location-based systems},
	Year = {2013}}

@inproceedings{machanavajjhala2008privacy,
	Author = {Machanavajjhala, Ashwin and Kifer, Daniel and Abowd, John and Gehrke, Johannes and Vilhuber, Lars},
	Booktitle = {Data Engineering, 2008. ICDE 2008. IEEE 24th International Conference on},
	Organization = {IEEE},
	Pages = {277--286},
	Title = {Privacy: Theory meets practice on the map},
	Year = {2008}}

@article{dewri2013local,
	Author = {Dewri, Rinku},
	Journal = {Mobile Computing, IEEE Transactions on},
	Number = {12},
	Pages = {2360--2372},
	Publisher = {IEEE},
	Title = {Local differential perturbations: Location privacy under approximate knowledge attackers},
	Volume = {12},
	Year = {2013}}

@inproceedings{chatzikokolakis2013broadening,
	Author = {Chatzikokolakis, Konstantinos and Andr{\'e}s, Miguel E and Bordenabe, Nicol{\'a}s Emilio and Palamidessi, Catuscia},
	Booktitle = {Privacy Enhancing Technologies},
	Organization = {Springer},
	Pages = {82--102},
	Title = {Broadening the Scope of Differential Privacy Using Metrics.},
	Year = {2013}}

@inproceedings{zhong2009distributed,
	Author = {Zhong, Ge and Hengartner, Urs},
	Booktitle = {Pervasive Computing and Communications, 2009. PerCom 2009. IEEE International Conference on},
	Organization = {IEEE},
	Pages = {1--10},
	Title = {A distributed k-anonymity protocol for location privacy},
	Year = {2009}}

@inproceedings{ho2011differential,
	Author = {Ho, Shen-Shyang and Ruan, Shuhua},
	Booktitle = {Proceedings of the 4th ACM SIGSPATIAL International Workshop on Security and Privacy in GIS and LBS},
	Organization = {ACM},
	Pages = {17--24},
	Title = {Differential privacy for location pattern mining},
	Year = {2011}}

@inproceedings{cheng2006preserving,
	Author = {Cheng, Reynold and Zhang, Yu and Bertino, Elisa and Prabhakar, Sunil},
	Booktitle = {Privacy Enhancing Technologies},
	Organization = {Springer},
	Pages = {393--412},
	Title = {Preserving user location privacy in mobile data management infrastructures},
	Year = {2006}}

@article{beresford2003location,
	Author = {Beresford, Alastair R and Stajano, Frank},
	Journal = {IEEE Pervasive computing},
	Number = {1},
	Pages = {46--55},
	Publisher = {IEEE},
	Title = {Location privacy in pervasive computing},
	Year = {2003}}

@inproceedings{freudiger2009optimal,
	Author = {Freudiger, Julien and Shokri, Reza and Hubaux, Jean-Pierre},
	Booktitle = {Privacy enhancing technologies},
	Organization = {Springer},
	Pages = {216--234},
	Title = {On the optimal placement of mix zones},
	Year = {2009}}

@article{krumm2009survey,
	Author = {Krumm, John},
	Journal = {Personal and Ubiquitous Computing},
	Number = {6},
	Pages = {391--399},
	Publisher = {Springer},
	Title = {A survey of computational location privacy},
	Volume = {13},
	Year = {2009}}

@article{Rakhshan2016letter,
	Author = {Rakhshan, Ali and Pishro-Nik, Hossein},
	Journal = {IEEE Wireless Communications Letter},
	Publisher = {IEEE},
	Title = {Interference Models for Vehicular Ad Hoc Networks},
	Year = {2016, submitted}}

@article{Rakhshan2015Journal,
	Author = {Rakhshan, Ali and Pishro-Nik, Hossein},
	Journal = {IEEE Transactions on Wireless Communications},
	Publisher = {IEEE},
	Title = {Improving Safety on Highways by Customizing Vehicular Ad Hoc Networks},
	Year = {to appear, 2017}}

@inproceedings{Rakhshan2015Cogsima,
	Author = {Rakhshan, Ali and Pishro-Nik, Hossein},
	Booktitle = {IEEE International Multi-Disciplinary Conference on Cognitive Methods in Situation Awareness and Decision Support},
	Organization = {IEEE},
	Title = {A New Approach to Customization of Accident Warning Systems to Individual Drivers},
	Year = {2015}}

@inproceedings{Rakhshan2015CISS,
	Author = {Rakhshan, Ali and Pishro-Nik, Hossein and Nekoui, Mohammad},
	Booktitle = {Conference on Information Sciences and Systems},
	Organization = {IEEE},
	Pages = {1--6},
	Title = {Driver-based adaptation of Vehicular Ad Hoc Networks for design of active safety systems},
	Year = {2015}}

@inproceedings{Rakhshan2014IV,
	Author = {Rakhshan, Ali and Pishro-Nik, Hossein and Ray, Evan},
	Booktitle = {Intelligent Vehicles Symposium},
	Organization = {IEEE},
	Pages = {1181--1186},
	Title = {Real-time estimation of the distribution of brake response times for an individual driver using Vehicular Ad Hoc Network.},
	Year = {2014}}

@inproceedings{Rakhshan2013Globecom,
	Author = {Rakhshan, Ali and Pishro-Nik, Hossein and Fisher, Donald and Nekoui, Mohammad},
	Booktitle = {IEEE Global Communications Conference},
	Organization = {IEEE},
	Pages = {1333--1337},
	Title = {Tuning collision warning algorithms to individual drivers for design of active safety systems.},
	Year = {2013}}

@article{Nekoui2012Journal,
	Author = {Nekoui, Mohammad and Pishro-Nik, Hossein},
	Journal = {IEEE Transactions on Wireless Communications},
	Number = {8},
	Pages = {2895--2905},
	Publisher = {IEEE},
	Title = {Throughput Scaling laws for Vehicular Ad Hoc Networks},
	Volume = {11},
	Year = {2012}}









@article{Nekoui2011Journal,
	Author = {Nekoui, Mohammad and Pishro-Nik, Hossein and Ni, Daiheng},
	Journal = {International Journal of Vehicular Technology},
	Pages = {1--11},
	Publisher = {Hindawi Publishing Corporation},
	Title = {Analytic Design of Active Safety Systems for Vehicular Ad hoc Networks},
	Volume = {2011},
	Year = {2011}}





	
@article{shokri2014optimal,
	  title={Optimal user-centric data obfuscation},
 	 author={Shokri, Reza},
 	 journal={arXiv preprint arXiv:1402.3426},
 	 year={2014}
	}
@article{chatzikokolakis2015location,
  title={Location privacy via geo-indistinguishability},
  author={Chatzikokolakis, Konstantinos and Palamidessi, Catuscia and Stronati, Marco},
  journal={ACM SIGLOG News},
  volume={2},
  number={3},
  pages={46--69},
  year={2015},
  publisher={ACM}

}
@inproceedings{shokri2011quantifying2,
  title={Quantifying location privacy: the case of sporadic location exposure},
  author={Shokri, Reza and Theodorakopoulos, George and Danezis, George and Hubaux, Jean-Pierre and Le Boudec, Jean-Yves},
  booktitle={Privacy Enhancing Technologies},
  pages={57--76},
  year={2011},
  organization={Springer}
}


@inproceedings{Mont1603:Defining,
AUTHOR="Zarrin Montazeri and Amir Houmansadr and Hossein Pishro-Nik",
TITLE="Defining Perfect Location Privacy Using Anonymization",
BOOKTITLE="2016 Annual Conference on Information Science and Systems (CISS) (CISS
2016)",
ADDRESS="Princeton, USA",
DAYS=16,
MONTH=mar,
YEAR=2016,
KEYWORDS="Information Theoretic Privacy; location-based services; Location Privacy;
Information Theory",
ABSTRACT="The popularity of mobile devices and location-based services (LBS) have
created great concerns regarding the location privacy of users of such
devices and services. Anonymization is a common technique that is often
being used to protect the location privacy of LBS users. In this paper, we
provide a general information theoretic definition for location privacy. In
particular, we define perfect location privacy. We show that under certain
conditions, perfect privacy is achieved if the pseudonyms of users is
changed after o(N^(2/r?1)) observations by the adversary, where N is the
number of users and r is the number of sub-regions or locations.
"
}
@article{our-isita-location,
	Author = {Zarrin Montazeri and Amir Houmansadr and Hossein Pishro-Nik},
	Journal = {IEEE International Symposium on Information Theory and Its Applications (ISITA)},
	Title = {Achieving Perfect Location Privacy in Markov Models Using Anonymization},
	Year = {2016}
	}
@article{our-TIFS,
	Author = {Zarrin Montazeri and Hossein Pishro-Nik and Amir Houmansadr},
	Journal = {IEEE Transactions on Information Forensics and Security, accepted with mandatory minor revisions},
	Title = {Perfect Location Privacy Using Anonymization in Mobile Networks},
	Year = {2017},
    note={Available on arxiv.org}
	}



@techreport{sampigethaya2005caravan,
  title={CARAVAN: Providing location privacy for VANET},
  author={Sampigethaya, Krishna and Huang, Leping and Li, Mingyan and Poovendran, Radha and Matsuura, Kanta and Sezaki, Kaoru},
  year={2005},
  institution={DTIC Document}
}
@incollection{buttyan2007effectiveness,
  title={On the effectiveness of changing pseudonyms to provide location privacy in VANETs},
  author={Butty{\'a}n, Levente and Holczer, Tam{\'a}s and Vajda, Istv{\'a}n},
  booktitle={Security and Privacy in Ad-hoc and Sensor Networks},
  pages={129--141},
  year={2007},
  publisher={Springer}
}
@article{sampigethaya2007amoeba,
  title={AMOEBA: Robust location privacy scheme for VANET},
  author={Sampigethaya, Krishna and Li, Mingyan and Huang, Leping and Poovendran, Radha},
  journal={Selected Areas in communications, IEEE Journal on},
  volume={25},
  number={8},
  pages={1569--1589},
  year={2007},
  publisher={IEEE}
}

@article{lu2012pseudonym,
  title={Pseudonym changing at social spots: An effective strategy for location privacy in vanets},
  author={Lu, Rongxing and Li, Xiaodong and Luan, Tom H and Liang, Xiaohui and Shen, Xuemin},
  journal={Vehicular Technology, IEEE Transactions on},
  volume={61},
  number={1},
  pages={86--96},
  year={2012},
  publisher={IEEE}
}
@inproceedings{lu2010sacrificing,
  title={Sacrificing the plum tree for the peach tree: A socialspot tactic for protecting receiver-location privacy in VANET},
  author={Lu, Rongxing and Lin, Xiaodong and Liang, Xiaohui and Shen, Xuemin},
  booktitle={Global Telecommunications Conference (GLOBECOM 2010), 2010 IEEE},
  pages={1--5},
  year={2010},
  organization={IEEE}
}
@inproceedings{lin2011stap,
  title={STAP: A social-tier-assisted packet forwarding protocol for achieving receiver-location privacy preservation in VANETs},
  author={Lin, Xiaodong and Lu, Rongxing and Liang, Xiaohui and Shen, Xuemin Sherman},
  booktitle={INFOCOM, 2011 Proceedings IEEE},
  pages={2147--2155},
  year={2011},
  organization={IEEE}
}
@inproceedings{gerlach2007privacy,
  title={Privacy in VANETs using changing pseudonyms-ideal and real},
  author={Gerlach, Matthias and Guttler, Felix},
  booktitle={Vehicular Technology Conference, 2007. VTC2007-Spring. IEEE 65th},
  pages={2521--2525},
  year={2007},
  organization={IEEE}
}
@inproceedings{el2002security,
  title={Security issues in a future vehicular network},
  author={El Zarki, Magda and Mehrotra, Sharad and Tsudik, Gene and Venkatasubramanian, Nalini},
  booktitle={European Wireless},
  volume={2},
  year={2002}
}

@article{hubaux2004security,
  title={The security and privacy of smart vehicles},
  author={Hubaux, Jean-Pierre and Capkun, Srdjan and Luo, Jun},
  journal={IEEE Security \& Privacy Magazine},
  volume={2},
  number={LCA-ARTICLE-2004-007},
  pages={49--55},
  year={2004}
}



@inproceedings{duri2002framework,
  title={Framework for security and privacy in automotive telematics},
  author={Duri, Sastry and Gruteser, Marco and Liu, Xuan and Moskowitz, Paul and Perez, Ronald and Singh, Moninder and Tang, Jung-Mu},
  booktitle={Proceedings of the 2nd international workshop on Mobile commerce},
  pages={25--32},
  year={2002},
  organization={ACM}
}
@misc{NS-3,
	Howpublished = {\url{https://www.nsnam.org/}}},
}
@misc{testbed,
	Howpublished = {\url{http://www.its.dot.gov/testbed/PDF/SE-MI-Resource-Guide-9-3-1.pdf}}},
@misc{NGSIM,
	Howpublished = {\url{http://ops.fhwa.dot.gov/trafficanalysistools/ngsim.htm}},
	}

@misc{National-a2013,
	Author = {National Highway Traffic Safety Administration},
	Howpublished = {\url{http://ops.fhwa.dot.gov/trafficanalysistools/ngsim.htm}},
	Title = {2013 Motor Vehicle Crashes: Overview. Traffic Safety Factors},
	Year = {2013}
	}

	@inproceedings{karnadi2007rapid,
	  title={Rapid generation of realistic mobility models for VANET},
	  author={Karnadi, Feliz Kristianto and Mo, Zhi Hai and Lan, Kun-chan},
	  booktitle={Wireless Communications and Networking Conference, 2007. WCNC 2007. IEEE},
	  pages={2506--2511},
	  year={2007},
	  organization={IEEE}
	}
	@inproceedings{saha2004modeling,
  title={Modeling mobility for vehicular ad-hoc networks},
  author={Saha, Amit Kumar and Johnson, David B},
  booktitle={Proceedings of the 1st ACM international workshop on Vehicular ad hoc networks},
  pages={91--92},
  year={2004},
  organization={ACM}
}
@inproceedings{lee2006modeling,
  title={Modeling steady-state and transient behaviors of user mobility: formulation, analysis, and application},
  author={Lee, Jong-Kwon and Hou, Jennifer C},
  booktitle={Proceedings of the 7th ACM international symposium on Mobile ad hoc networking and computing},
  pages={85--96},
  year={2006},
  organization={ACM}
}
@inproceedings{yoon2006building,
  title={Building realistic mobility models from coarse-grained traces},
  author={Yoon, Jungkeun and Noble, Brian D and Liu, Mingyan and Kim, Minkyong},
  booktitle={Proceedings of the 4th international conference on Mobile systems, applications and services},
  pages={177--190},
  year={2006},
  organization={ACM}
}

@inproceedings{choffnes2005integrated,
	title={An integrated mobility and traffic model for vehicular wireless networks},
	author={Choffnes, David R and Bustamante, Fabi{\'a}n E},
	booktitle={Proceedings of the 2nd ACM international workshop on Vehicular ad hoc networks},
	pages={69--78},
	year={2005},
	organization={ACM}
}

@inproceedings{Qian2008Globecom,
	title={CA Secure VANET MAC Protocol for DSRC Applications},
	author={Yi, Q. and Lu, K. and Moyeri, N.{\'a}n E},
	booktitle={Proceedings of IEEE GLOBECOM 2008},
	pages={1--5},
	year={2008},
	organization={IEEE}
}





	@inproceedings{naumov2006evaluation,
  title={An evaluation of inter-vehicle ad hoc networks based on realistic vehicular traces},
  author={Naumov, Valery and Baumann, Rainer and Gross, Thomas},
  booktitle={Proceedings of the 7th ACM international symposium on Mobile ad hoc networking and computing},
  pages={108--119},
  year={2006},
  organization={ACM}
}
	@article{sommer2008progressing,
  title={Progressing toward realistic mobility models in VANET simulations},
  author={Sommer, Christoph and Dressler, Falko},
  journal={Communications Magazine, IEEE},
  volume={46},
  number={11},
  pages={132--137},
  year={2008},
  publisher={IEEE}
}




	@inproceedings{mahajan2006urban,
  title={Urban mobility models for vanets},
  author={Mahajan, Atulya and Potnis, Niranjan and Gopalan, Kartik and Wang, Andy},
  booktitle={2nd IEEE International Workshop on Next Generation Wireless Networks},
  volume={33},
  year={2006}
}

@inproceedings{Rakhshan2016packet,
  title={Packet success probability derivation in a vehicular ad hoc network for a highway scenario},
  author={Rakhshan, Ali and Pishro-Nik, Hossein},
  booktitle={2016 Annual Conference on Information Science and Systems (CISS)},
  pages={210--215},
  year={2016},
  organization={IEEE}
}

@inproceedings{Rakhshan2016CISS,
	Author = {Rakhshan, Ali and Pishro-Nik, Hossein},
	Booktitle = {Conference on Information Sciences and Systems},
	Organization = {IEEE},
	Pages = {210--215},
	Title = {Packet Success Probability Derivation in a Vehicular Ad Hoc Network for a Highway Scenario},
	Year = {2016}}

@article{Nekoui2013Journal,
	Author = {Nekoui, Mohammad and Pishro-Nik, Hossein},
	Journal = {Journal on Selected Areas in Communications, Special Issue on Emerging Technologies in Communications},
	Number = {9},
	Pages = {491--503},
	Publisher = {IEEE},
	Title = {Analytic Design of Active Safety Systems for Vehicular Ad hoc Networks},
	Volume = {31},
	Year = {2013}}


@inproceedings{Nekoui2011MOBICOM,
	Author = {Nekoui, Mohammad and Pishro-Nik, Hossein},
	Booktitle = {MOBICOM workshop on VehiculAr InterNETworking},
	Organization = {ACM},
	Title = {Analytic Design of Active Vehicular Safety Systems in Sparse Traffic},
	Year = {2011}}

@inproceedings{Nekoui2011VTC,
	Author = {Nekoui, Mohammad and Pishro-Nik, Hossein},
	Booktitle = {VTC-Fall},
	Organization = {IEEE},
	Title = {Analytical Design of Inter-vehicular Communications for Collision Avoidance},
	Year = {2011}}

@inproceedings{Bovee2011VTC,
	Author = {Bovee, Ben Louis and Nekoui, Mohammad and Pishro-Nik, Hossein},
	Booktitle = {VTC-Fall},
	Organization = {IEEE},
	Title = {Evaluation of the Universal Geocast Scheme For VANETs},
	Year = {2011}}

@inproceedings{Nekoui2010MOBICOM,
	Author = {Nekoui, Mohammad and Pishro-Nik, Hossein},
	Booktitle = {MOBICOM},
	Organization = {ACM},
	Title = {Fundamental Tradeoffs in Vehicular Ad Hoc Networks},
	Year = {2010}}

@inproceedings{Nekoui2010IVCS,
	Author = {Nekoui, Mohammad and Pishro-Nik, Hossein},
	Booktitle = {IVCS},
	Organization = {IEEE},
	Title = {A Universal Geocast Scheme for Vehicular Ad Hoc Networks},
	Year = {2010}}

@inproceedings{Nekoui2009ITW,
	Author = {Nekoui, Mohammad and Pishro-Nik, Hossein},
	Booktitle = {IEEE Communications Society Conference on Sensor, Mesh and Ad Hoc Communications and Networks Workshops},
	Organization = {IEEE},
	Pages = {1--3},
	Title = {A Geometrical Analysis of Obstructed Wireless Networks},
	Year = {2009}}

@article{Eslami2013Journal,
	Author = {Eslami, Ali and Nekoui, Mohammad and Pishro-Nik, Hossein and Fekri, Faramarz},
	Journal = {ACM Transactions on Sensor Networks},
	Number = {4},
	Pages = {51},
	Publisher = {ACM},
	Title = {Results on finite wireless sensor networks: Connectivity and coverage},
	Volume = {9},
	Year = {2013}}


@article{Jiafu2014Journal,
	Author = {Jiafu, W. and Zhang, D. and Zhao, S. and Yang, L. and Lloret, J.},
	Journal = {Communications Magazine},
	Number = {8},
	Pages = {106-113},
	Publisher = {IEEE},
	Title = {Context-aware vehicular cyber-physical systems with cloud support: architecture, challenges, and solutions},
	Volume = {52},
	Year = {2014}}

@inproceedings{Haas2010ACM,
	Author = {Haas, J.J. and Hu, Y.},
	Booktitle = {international workshop on VehiculAr InterNETworking},
	Organization = {ACM},
	Title = {Communication requirements for crash avoidance.},
	Year = {2010}}

@inproceedings{Yi2008GLOBECOM,
	Author = {Yi, Q. and Lu, K. and Moayeri, N.},
	Booktitle = {GLOBECOM},
	Organization = {IEEE},
	Title = {CA Secure VANET MAC Protocol for DSRC Applications.},
	Year = {2008}}

@inproceedings{Mughal2010ITSim,
	Author = {Mughal, B.M. and Wagan, A. and Hasbullah, H.},
	Booktitle = {International Symposium on Information Technology (ITSim)},
	Organization = {IEEE},
	Title = {Efficient congestion control in VANET for safety messaging.},
	Year = {2010}}

@article{Chang2011Journal,
	Author = {Chang, Y. and Lee, C. and Copeland, J.},
	Journal = {Selected Areas in Communications},
	Pages = {236 –249},
	Publisher = {IEEE},
	Title = {Goodput enhancement of VANETs in noisy CSMA/CA channels},
	Volume = {29},
	Year = {2011}}

@article{Garcia-Costa2011Journal,
	Author = {Garcia-Costa, C. and Egea-Lopez, E. and Tomas-Gabarron, J. B. and Garcia-Haro, J. and Haas, Z. J.},
	Journal = {Transactions on Intelligent Transportation Systems},
	Pages = {1 –16},
	Publisher = {IEEE},
	Title = {A stochastic model for chain collisions of vehicles equipped with vehicular communications},
	Volume = {99},
	Year = {2011}}

@article{Carbaugh2011Journal,
	Author = {Carbaugh, J. and Godbole,  D. N. and Sengupta, R. and Garcia-Haro, J. and Haas, Z. J.},
	Publisher = {Transportation Research Part C (Emerging Technologies)},
	Title = {Safety and capacity analysis of automated and manual highway systems},
	Year = {1997}}

@article{Goh2004Journal,
	Author = {Goh, P. and Wong, Y.},
	Publisher = {Appl Health Econ Health Policy},
	Title = {Driver perception response time during the signal change interval},
	Year = {2004}}

@article{Chang1985Journal,
	Author = {Chang, M.S. and Santiago, A.J.},
	Pages = {20-30},
	Publisher = {Transportation Research Record},
	Title = {Timing traffic signal changes based on driver behavior},
	Volume = {1027},
	Year = {1985}}

@article{Green2000Journal,
	Author = {Green, M.},
	Pages = {195-216},
	Publisher = {Transportation Human Factors},
	Title = {How long does it take to stop? Methodological analysis of driver perception-brake times.},
	Year = {2000}}

@article{Koppa2005,
	Author = {Koppa, R.J.},
	Pages = {195-216},
	Publisher = {http://www.fhwa.dot.gov/publications/},
	Title = {Human Factors},
	Year = {2005}}

@inproceedings{Maxwell2010ETC,
	Author = {Maxwell, A. and Wood, K.},
	Booktitle = {Europian Transport Conference},
	Organization = {http://www.etcproceedings.org/paper/review-of-traffic-signals-on-high-speed-roads},
	Title = {Review of Traffic Signals on High Speed Road},
	Year = {2010}}

@article{Wortman1983,
	Author = {Wortman, R.H. and Matthias, J.S.},
	Publisher = {Arizona Department of Transportation},
	Title = {An Evaluation of Driver Behavior at Signalized Intersections},
	Year = {1983}}
@inproceedings{Zhang2007IASTED,
	Author = {Zhang, X. and Bham, G.H.},
	Booktitle = {18th IASTED International Conference: modeling and simulation},
	Title = {Estimation of driver reaction time from detailed vehicle trajectory data.},
	Year = {2007}}


@inproceedings{bai2003important,
  title={IMPORTANT: A framework to systematically analyze the Impact of Mobility on Performance of RouTing protocols for Adhoc NeTworks},
  author={Bai, Fan and Sadagopan, Narayanan and Helmy, Ahmed},
  booktitle={INFOCOM 2003. Twenty-second annual joint conference of the IEEE computer and communications. IEEE societies},
  volume={2},
  pages={825--835},
  year={2003},
  organization={IEEE}
}


@inproceedings{abedi2008enhancing,
	  title={Enhancing AODV routing protocol using mobility parameters in VANET},
	  author={Abedi, Omid and Fathy, Mahmood and Taghiloo, Jamshid},
	  booktitle={Computer Systems and Applications, 2008. AICCSA 2008. IEEE/ACS International Conference on},
	  pages={229--235},
	  year={2008},
	  organization={IEEE}
	}


@article{AlSultan2013Journal,
	Author = {Al-Sultan, Saif and Al-Bayatti, Ali H. and Zedan, Hussien},
	Journal = {IEEE Transactions on Vehicular Technology},
	Number = {9},
	Pages = {4264-4275},
	Publisher = {IEEE},
	Title = {Context Aware Driver Behaviour Detection System in Intelligent Transportation Systems},
	Volume = {62},
	Year = {2013}}






@article{Leow2008ITS,
	Author = {Leow, Woei Ling and Ni, Daiheng and Pishro-Nik, Hossein},
	Journal = {IEEE Transactions on Intelligent Transportation Systems},
	Number = {2},
	Pages = {369--374},
	Publisher = {IEEE},
	Title = {A Sampling Theorem Approach to Traffic Sensor Optimization},
	Volume = {9},
	Year = {2008}}



@article{REU2007,
	Author = {D. Ni and H. Pishro-Nik and R. Prasad and M. R. Kanjee and H. Zhu and T. Nguyen},
	Journal = {in 14th World Congress on Intelligent Transport Systems},
	Title = {Development of a prototype intersection collision avoidance system under VII},
	Year = {2007}}




@inproceedings{salamatian2013hide,
  title={How to hide the elephant-or the donkey-in the room: Practical privacy against statistical inference for large data.},
  author={Salamatian, Salman and Zhang, Amy and du Pin Calmon, Flavio and Bhamidipati, Sandilya and Fawaz, Nadia and Kveton, Branislav and Oliveira, Pedro and Taft, Nina},
  booktitle={GlobalSIP},
  pages={269--272},
  year={2013}
}

@article{sankar2013utility,
  title={Utility-privacy tradeoffs in databases: An information-theoretic approach},
  author={Sankar, Lalitha and Rajagopalan, S Raj and Poor, H Vincent},
  journal={Information Forensics and Security, IEEE Transactions on},
  volume={8},
  number={6},
  pages={838--852},
  year={2013},
  publisher={IEEE}
}
@inproceedings{ghinita2007prive,
  title={PRIVE: anonymous location-based queries in distributed mobile systems},
  author={Ghinita, Gabriel and Kalnis, Panos and Skiadopoulos, Spiros},
  booktitle={Proceedings of the 16th international conference on World Wide Web},
  pages={371--380},
  year={2007},
  organization={ACM}
}

@article{beresford2004mix,
  title={Mix zones: User privacy in location-aware services},
  author={Beresford, Alastair R and Stajano, Frank},
  year={2004},
  publisher={IEEE}
}

%@inproceedings{Mont1610Achieving,
%  title={Achieving Perfect Location Privacy in Markov Models Using Anonymization},
%  author={Montazeri, Zarrin and Houmansadr, Amir and H.Pishro-Nik},
%  booktitle="2016 International Symposium on Information Theory and its Applications
%  (ISITA2016)",
%  address="Monterey, USA",
%  days=30,
%  month=oct,
%  year=2016,
%}

@article{csiszar1996almost,
  title={Almost independence and secrecy capacity},
  author={Csisz{\'a}r, Imre},
  journal={Problemy Peredachi Informatsii},
  volume={32},
  number={1},
  pages={48--57},
  year={1996},
  publisher={Russian Academy of Sciences, Branch of Informatics, Computer Equipment and Automatization}
}

@article{yamamoto1983source,
  title={A source coding problem for sources with additional outputs to keep secret from the receiver or wiretappers (corresp.)},
  author={Yamamoto, Hirosuke},
  journal={IEEE Transactions on Information Theory},
  volume={29},
  number={6},
  pages={918--923},
  year={1983},
  publisher={IEEE}
}


@inproceedings{calmon2015fundamental,
  title={Fundamental limits of perfect privacy},
  author={Calmon, Flavio P and Makhdoumi, Ali and M{\'e}dard, Muriel},
  booktitle={Information Theory (ISIT), 2015 IEEE International Symposium on},
  pages={1796--1800},
  year={2015},
  organization={IEEE}
}



@inproceedings{Lehman1999Large-Sample-Theory,
	title={Elements of Large Sample Theory},
	author={E. L. Lehman},
	organization={Springer},
	year={1999}
}


@inproceedings{Ferguson1999Large-Sample-Theory,
	title={A Course in Large Sample Theory},
	author={Thomas S. Ferguson},
	organization={CRC Press},
	year={1996}
}



@inproceedings{Dembo1999Large-Deviations,
	title={Large Deviation Techniques and Applications, Second Edition},
	author={A. Dembo and O. Zeitouni},
	organization={Springer},
	year={1998}
}


%%%%%%%%%%%%%%%%%%%%%%%%%%%%%%%%%%%%%%%%%%%%%%%%
Hossein's Coding Journals
%%%%%%%%%%%%%%%%%%%%%%

@ARTICLE{myoptics,
  AUTHOR =       "H. Pishro-Nik and N. Rahnavard and J. Ha and F. Fekri and A. Adibi ",
  TITLE =        "Low-density parity-check codes for volume holographic memory systems",
  JOURNAL =      " Appl. Opt.",
  YEAR =         "2003",
  volume =       "42",
  pages =        "861-870  "
 }






@ARTICLE{myit,
  AUTHOR =       "H. Pishro-Nik and F. Fekri  ",
  TITLE =        "On Decoding of Low-Density Parity-Check Codes on the Binary Erasure Channel",
  JOURNAL =      "IEEE Trans. Inform. Theory",
  YEAR =         "2004",
  volume =       "50",
  pages =        "439--454"
  }




@ARTICLE{myitpuncture,
  AUTHOR =       "H. Pishro-Nik and F. Fekri  ",
  TITLE =        "Results on Punctured Low-Density Parity-Check Codes and Improved Iterative Decoding Techniques",
  JOURNAL =      "IEEE Trans. on Inform. Theory",
  YEAR =         "2007",
  volume =       "53",
  number=        "2",
  pages =        "599--614",
  month= "February"
  }




@ARTICLE{myitlinmimdist,
  AUTHOR =       "H. Pishro-Nik and F. Fekri",
  TITLE =        "Performance of Low-Density Parity-Check Codes With Linear Minimum Distance",
  JOURNAL =         "IEEE Trans. Inform. Theory ",
  YEAR =         "2006",
  volume =       "52",
  number="1",
  pages =        "292 --300"
  }






@ARTICLE{myitnonuni,
  AUTHOR =       "H. Pishro-Nik and N. Rahnavard and F. Fekri  ",
  TITLE =        "Non-uniform Error Correction Using Low-Density Parity-Check Codes",
  JOURNAL =      "IEEE Trans. Inform. Theory",
  YEAR =         "2005",
  volume =       "51",
  number=  "7",
  pages =        "2702--2714"
 }





@article{eslamitcomhybrid10,
 author = {A. Eslami and S. Vangala and H. Pishro-Nik},
 title = {Hybrid channel codes for highly efficient FSO/RF communication systems},
 journal = {IEEE Transactions on Communications},
 volume = {58},
 number = {10},
 year = {2010},
 pages = {2926--2938},
 }


@article{eslamitcompolar13,
 author = {A. Eslami and H. Pishro-Nik},
 title = {On Finite-Length Performance of Polar Codes: Stopping Sets, Error Floor, and Concatenated Design},
 journal = {IEEE Transactions on Communications},
 volume = {61},
 number = {13},
 year = {2013},
 pages = {919--929},
 }



 @article{saeeditcom11,
 author = {H. Saeedi and H. Pishro-Nik and  A. H. Banihashemi},
 title = {Successive maximization for the systematic design of universally capacity approaching rate-compatible
 sequences of LDPC code ensembles over binary-input output-symmetric memoryless channels},
 journal = {IEEE Transactions on Communications},
 year = {2011},
 volume={59},
 number = {7}
 }


@article{rahnavard07,
 author = {Rahnavard, N. and Pishro-Nik, H. and Fekri, F.},
 title = {Unequal Error Protection Using Partially Regular LDPC Codes},
 journal = {IEEE Transactions on Communications},
 year = {2007},
 volume = {55},
 number = {3},
 pages = {387 -- 391}
 }


 @article{hosseinira04,
 author = {H. Pishro-Nik and F. Fekri},
 title = {Irregular repeat-accumulate codes for volume holographic memory systems},
 journal = {Journal of Applied Optics},
 year = {2004},
 volume = {43},
 number = {27},
 pages = {5222--5227},
 }


@article{azadeh2015Ephemeralkey,
 author = {A. Sheikholeslami and D. Goeckel and H. Pishro-Nik},
 title = {Jamming Based on an Ephemeral Key to Obtain Everlasting Security in Wireless Environments},
 journal = {IEEE Transactions on Wireless Communications},
 year = {2015},
 volume = {14},
 number = {11},
 pages = {6072--6081},
}


@article{azadeh2014Everlasting,
 author = {A. Sheikholeslami and D. Goeckel and H. Pishro-Nik},
 title = {Everlasting secrecy in disadvantaged wireless environments against sophisticated eavesdroppers},
 journal = {48th Asilomar Conference on Signals, Systems and Computers},
 year = {2014},
 pages = {1994--1998},
}


@article{azadeh2013ISIT,
 author = {A. Sheikholeslami and D. Goeckel and H. Pishro-Nik},
 title = {Artificial intersymbol interference (ISI) to exploit receiver imperfections for secrecy},
 journal = {IEEE International Symposium on Information Theory (ISIT)},
 year = {2013},
}


@article{azadeh2013Jsac,
 author = {A. Sheikholeslami and D. Goeckel and H. Pishro-Nik},
 title = {Jamming Based on an Ephemeral Key to Obtain Everlasting Security in Wireless Environments},
 journal = {IEEE Journal on Selected Areas in Communications},
 year = {2013},
 volume = {31},
 number = {9},
 pages = {1828--1839},
}


@article{azadeh2012Allerton,
 author = {A. Sheikholeslami and D. Goeckel and H. Pishro-Nik},
 title = {Exploiting the non-commutativity of nonlinear operators for information-theoretic security in disadvantaged wireless environments},
 journal = {50th Annual Allerton Conference on Communication, Control, and Computing},
 year = {2012},
 pages = {233--240},
}


@article{azadeh2012Infocom,
 author = {A. Sheikholeslami and D. Goeckel and H. Pishro-Nik},
 title = {Jamming Based on an Ephemeral Key to Obtain Everlasting Security in Wireless Environments},
 journal = {IEEE INFOCOM},
 year = {2012},
 pages = {1179--1187},
}

@article{1corser2016evaluating,
  title={Evaluating Location Privacy in Vehicular Communications and Applications},
  author={Corser, George P and Fu, Huirong and Banihani, Abdelnasser},
  journal={IEEE Transactions on Intelligent Transportation Systems},
  volume={17},
  number={9},
  pages={2658-2667},
  year={2016},
  publisher={IEEE}
}
@article{2zhang2016designing,
  title={On Designing Satisfaction-Ratio-Aware Truthful Incentive Mechanisms for k-Anonymity Location Privacy},
  author={Zhang, Yuan and Tong, Wei and Zhong, Sheng},
  journal={IEEE Transactions on Information Forensics and Security},
  volume={11},
  number={11},
  pages={2528--2541},
  year={2016},
  publisher={IEEE}
}
@article{3li2016privacy,
  title={Privacy-preserving Location Proof for Securing Large-scale Database-driven Cognitive Radio Networks},
  author={Li, Yi and Zhou, Lu and Zhu, Haojin and Sun, Limin},
  journal={IEEE Internet of Things Journal},
  volume={3},
  number={4},
  pages={563-571},
  year={2016},
  publisher={IEEE}
}
@article{4olteanu2016quantifying,
  title={Quantifying Interdependent Privacy Risks with Location Data},
  author={Olteanu, Alexandra-Mihaela and Huguenin, K{\'e}vin and Shokri, Reza and Humbert, Mathias and Hubaux, Jean-Pierre},
  journal={IEEE Transactions on Mobile Computing},
  year={2016},
  volume={PP},
  number={99},
  pages={1-1},
  publisher={IEEE}
}
@article{5yi2016practical,
  title={Practical Approximate k Nearest Neighbor Queries with Location and Query Privacy},
  author={Yi, Xun and Paulet, Russell and Bertino, Elisa and Varadharajan, Vijay},
  journal={IEEE Transactions on Knowledge and Data Engineering},
  volume={28},
  number={6},
  pages={1546--1559},
  year={2016},
  publisher={IEEE}
}
@article{6li2016privacy,
  title={Privacy Leakage of Location Sharing in Mobile Social Networks: Attacks and Defense},
  author={Li, Huaxin and Zhu, Haojin and Du, Suguo and Liang, Xiaohui and Shen, Xuemin},
  journal={IEEE Transactions on Dependable and Secure Computing},
  year={2016},
  volume={PP},
  number={99},
  publisher={IEEE}
}

@article{7murakami2016localization,
  title={Localization Attacks Using Matrix and Tensor Factorization},
  author={Murakami, Takao and Watanabe, Hajime},
  journal={IEEE Transactions on Information Forensics and Security},
  volume={11},
  number={8},
  pages={1647--1660},
  year={2016},
  publisher={IEEE}
}
@article{8zurbaran2015near,
  title={Near-Rand: Noise-based Location Obfuscation Based on Random Neighboring Points},
  author={Zurbaran, Mayra Alejandra and Avila, Karen and Wightman, Pedro and Fernandez, Michael},
  journal={IEEE Latin America Transactions},
  volume={13},
  number={11},
  pages={3661--3667},
  year={2015},
  publisher={IEEE}
}

@article{9tan2014anti,
  title={An anti-tracking source-location privacy protection protocol in wsns based on path extension},
  author={Tan, Wei and Xu, Ke and Wang, Dan},
  journal={IEEE Internet of Things Journal},
  volume={1},
  number={5},
  pages={461--471},
  year={2014},
  publisher={IEEE}
}

@article{10peng2014enhanced,
  title={Enhanced Location Privacy Preserving Scheme in Location-Based Services},
  author={Peng, Tao and Liu, Qin and Wang, Guojun},
  journal={IEEE Systems Journal},
  year={2014},
  volume={PP},
  number={99},
  pages={1-12},
  publisher={IEEE}
}
@article{11dewri2014exploiting,
  title={Exploiting service similarity for privacy in location-based search queries},
  author={Dewri, Rinku and Thurimella, Ramakrisha},
  journal={IEEE Transactions on Parallel and Distributed Systems},
  volume={25},
  number={2},
  pages={374--383},
  year={2014},
  publisher={IEEE}
}

@article{12hwang2014novel,
  title={A novel time-obfuscated algorithm for trajectory privacy protection},
  author={Hwang, Ren-Hung and Hsueh, Yu-Ling and Chung, Hao-Wei},
  journal={IEEE Transactions on Services Computing},
  volume={7},
  number={2},
  pages={126--139},
  year={2014},
  publisher={IEEE}
}
@article{13puttaswamy2014preserving,
  title={Preserving location privacy in geosocial applications},
  author={Puttaswamy, Krishna PN and Wang, Shiyuan and Steinbauer, Troy and Agrawal, Divyakant and El Abbadi, Amr and Kruegel, Christopher and Zhao, Ben Y},
  journal={IEEE Transactions on Mobile Computing},
  volume={13},
  number={1},
  pages={159--173},
  year={2014},
  publisher={IEEE}
}

@article{14zhang2014privacy,
  title={Privacy quantification model based on the Bayes conditional risk in Location-Based Services},
  author={Zhang, Xuejun and Gui, Xiaolin and Tian, Feng and Yu, Si and An, Jian},
  journal={Tsinghua Science and Technology},
  volume={19},
  number={5},
  pages={452--462},
  year={2014},
  publisher={TUP}
}

@article{15bilogrevic2014privacy,
  title={Privacy-preserving optimal meeting location determination on mobile devices},
  author={Bilogrevic, Igor and Jadliwala, Murtuza and Joneja, Vishal and Kalkan, K{\"u}bra and Hubaux, Jean-Pierre and Aad, Imad},
  journal={IEEE transactions on information forensics and security},
  volume={9},
  number={7},
  pages={1141--1156},
  year={2014},
  publisher={IEEE}
}
@article{16haghnegahdar2014privacy,
  title={Privacy Risks in Publishing Mobile Device Trajectories},
  author={Haghnegahdar, Alireza and Khabbazian, Majid and Bhargava, Vijay K},
  journal={IEEE Wireless Communications Letters},
  volume={3},
  number={3},
  pages={241--244},
  year={2014},
  publisher={IEEE}
}
@article{17malandrino2014verification,
  title={Verification and inference of positions in vehicular networks through anonymous beaconing},
  author={Malandrino, Francesco and Borgiattino, Carlo and Casetti, Claudio and Chiasserini, Carla-Fabiana and Fiore, Marco and Sadao, Roberto},
  journal={IEEE Transactions on Mobile Computing},
  volume={13},
  number={10},
  pages={2415--2428},
  year={2014},
  publisher={IEEE}
}
@article{18shokri2014hiding,
  title={Hiding in the mobile crowd: Locationprivacy through collaboration},
  author={Shokri, Reza and Theodorakopoulos, George and Papadimitratos, Panos and Kazemi, Ehsan and Hubaux, Jean-Pierre},
  journal={IEEE transactions on dependable and secure computing},
  volume={11},
  number={3},
  pages={266--279},
  year={2014},
  publisher={IEEE}
}
@article{19freudiger2013non,
  title={Non-cooperative location privacy},
  author={Freudiger, Julien and Manshaei, Mohammad Hossein and Hubaux, Jean-Pierre and Parkes, David C},
  journal={IEEE Transactions on Dependable and Secure Computing},
  volume={10},
  number={2},
  pages={84--98},
  year={2013},
  publisher={IEEE}
}
@article{20gao2013trpf,
  title={TrPF: A trajectory privacy-preserving framework for participatory sensing},
  author={Gao, Sheng and Ma, Jianfeng and Shi, Weisong and Zhan, Guoxing and Sun, Cong},
  journal={IEEE Transactions on Information Forensics and Security},
  volume={8},
  number={6},
  pages={874--887},
  year={2013},
  publisher={IEEE}
}
@article{21ma2013privacy,
  title={Privacy vulnerability of published anonymous mobility traces},
  author={Ma, Chris YT and Yau, David KY and Yip, Nung Kwan and Rao, Nageswara SV},
  journal={IEEE/ACM Transactions on Networking},
  volume={21},
  number={3},
  pages={720--733},
  year={2013},
  publisher={IEEE}
}
@article{22niu2013pseudo,
  title={Pseudo-Location Updating System for privacy-preserving location-based services},
  author={Niu, Ben and Zhu, Xiaoyan and Chi, Haotian and Li, Hui},
  journal={China Communications},
  volume={10},
  number={9},
  pages={1--12},
  year={2013},
  publisher={IEEE}
}
@article{23dewri2013local,
  title={Local differential perturbations: Location privacy under approximate knowledge attackers},
  author={Dewri, Rinku},
  journal={IEEE Transactions on Mobile Computing},
  volume={12},
  number={12},
  pages={2360--2372},
  year={2013},
  publisher={IEEE}
}
@inproceedings{24kanoria2012tractable,
  title={Tractable bayesian social learning on trees},
  author={Kanoria, Yashodhan and Tamuz, Omer},
  booktitle={Information Theory Proceedings (ISIT), 2012 IEEE International Symposium on},
  pages={2721--2725},
  year={2012},
  organization={IEEE}
}
@inproceedings{25farias2005universal,
  title={A universal scheme for learning},
  author={Farias, Vivek F and Moallemi, Ciamac C and Van Roy, Benjamin and Weissman, Tsachy},
  booktitle={Proceedings. International Symposium on Information Theory, 2005. ISIT 2005.},
  pages={1158--1162},
  year={2005},
  organization={IEEE}
}
@inproceedings{26misra2013unsupervised,
  title={Unsupervised learning and universal communication},
  author={Misra, Vinith and Weissman, Tsachy},
  booktitle={Information Theory Proceedings (ISIT), 2013 IEEE International Symposium on},
  pages={261--265},
  year={2013},
  organization={IEEE}
}
@inproceedings{27ryabko2013time,
  title={Time-series information and learning},
  author={Ryabko, Daniil},
  booktitle={Information Theory Proceedings (ISIT), 2013 IEEE International Symposium on},
  pages={1392--1395},
  year={2013},
  organization={IEEE}
}
@inproceedings{28krzakala2013phase,
  title={Phase diagram and approximate message passing for blind calibration and dictionary learning},
  author={Krzakala, Florent and M{\'e}zard, Marc and Zdeborov{\'a}, Lenka},
  booktitle={Information Theory Proceedings (ISIT), 2013 IEEE International Symposium on},
  pages={659--663},
  year={2013},
  organization={IEEE}
}
@inproceedings{29sakata2013sample,
  title={Sample complexity of Bayesian optimal dictionary learning},
  author={Sakata, Ayaka and Kabashima, Yoshiyuki},
  booktitle={Information Theory Proceedings (ISIT), 2013 IEEE International Symposium on},
  pages={669--673},
  year={2013},
  organization={IEEE}
}
@inproceedings{30predd2004consistency,
  title={Consistency in a model for distributed learning with specialists},
  author={Predd, Joel B and Kulkarni, Sanjeev R and Poor, H Vincent},
  booktitle={IEEE International Symposium on Information Theory},
  year={2004},
organization={IEEE}
}
@inproceedings{31nokleby2016rate,
  title={Rate-Distortion Bounds on Bayes Risk in Supervised Learning},
  author={Nokleby, Matthew and Beirami, Ahmad and Calderbank, Robert},
  booktitle={2016 IEEE International Symposium on Information Theory (ISIT)},
pages={2099-2103},
  year={2016},
organization={IEEE}
}

@inproceedings{32le2016imperfect,
  title={Are imperfect reviews helpful in social learning?},
  author={Le, Tho Ngoc and Subramanian, Vijay G and Berry, Randall A},
  booktitle={Information Theory (ISIT), 2016 IEEE International Symposium on},
  pages={2089--2093},
  year={2016},
  organization={IEEE}
}
@inproceedings{33gadde2016active,
  title={Active Learning for Community Detection in Stochastic Block Models},
  author={Gadde, Akshay and Gad, Eyal En and Avestimehr, Salman and Ortega, Antonio},
  booktitle={2016 IEEE International Symposium on Information Theory (ISIT)},
  pages={1889-1893},
  year={2016}
}
@inproceedings{34shakeri2016minimax,
  title={Minimax Lower Bounds for Kronecker-Structured Dictionary Learning},
  author={Shakeri, Zahra and Bajwa, Waheed U and Sarwate, Anand D},
  booktitle={2016 IEEE International Symposium on Information Theory (ISIT)},
  pages={1148-1152},
  year={2016}
}
@article{35lee2015speeding,
  title={Speeding up distributed machine learning using codes},
  author={Lee, Kangwook and Lam, Maximilian and Pedarsani, Ramtin and Papailiopoulos, Dimitris and Ramchandran, Kannan},
  booktitle={2016 IEEE International Symposium on Information Theory (ISIT)},
  pages={1143-1147},
  year={2016}
}
@article{36oneto2016statistical,
  title={Statistical Learning Theory and ELM for Big Social Data Analysis},
  author={Oneto, Luca and Bisio, Federica and Cambria, Erik and Anguita, Davide},
  journal={ieee CompUTATionAl inTelliGenCe mAGAzine},
  volume={11},
  number={3},
  pages={45--55},
  year={2016},
  publisher={IEEE}
}
@article{37lin2015probabilistic,
  title={Probabilistic approach to modeling and parameter learning of indirect drive robots from incomplete data},
  author={Lin, Chung-Yen and Tomizuka, Masayoshi},
  journal={IEEE/ASME Transactions on Mechatronics},
  volume={20},
  number={3},
  pages={1036--1045},
  year={2015},
  publisher={IEEE}
}
@article{38wang2016towards,
  title={Towards Bayesian Deep Learning: A Framework and Some Existing Methods},
  author={Wang, Hao and Yeung, Dit-Yan},
  journal={IEEE Transactions on Knowledge and Data Engineering},
  volume={PP},
  number={99},
  year={2016},
  publisher={IEEE}
}


%%%%%Informationtheoreticsecurity%%%%%%%%%%%%%%%%%%%%%%%




@inproceedings{Bloch2011PhysicalSecBook,
	title={Physical-Layer Security},
	author={M. Bloch and J. Barros},
	organization={Cambridge University Press},
	year={2011}
}



@inproceedings{Liang2009InfoSecBook,
	title={Information Theoretic Security},
	author={Y. Liang and H. V. Poor and S. Shamai (Shitz)},
	organization={Now Publishers Inc.},
	year={2009}
}


@inproceedings{Zhou2013PhysicalSecBook,
	title={Physical Layer Security in Wireless Communications},
	author={ X. Zhou and L. Song and Y. Zhang},
	organization={CRC Press},
	year={2013}
}

@article{Ni2012IEA,
	Author = {D. Ni and H. Liu and W. Ding and  Y. Xie and H. Wang and H. Pishro-Nik and Q. Yu},
	Journal = {IEA/AIE},
	Title = {Cyber-Physical Integration to Connect Vehicles for Transformed Transportation Safety and Efficiency},
	Year = {2012}}



@inproceedings{Ni2012Inproceedings,
	Author = {D. Ni, H. Liu, Y. Xie, W. Ding, H. Wang, H. Pishro-Nik, Q. Yu and M. Ferreira},
	Booktitle = {Spring Simulation Multiconference},
	Date-Added = {2016-09-04 14:18:42 +0000},
	Date-Modified = {2016-09-06 16:22:14 +0000},
	Title = {Virtual Lab of Connected Vehicle Technology},
	Year = {2012}}

@inproceedings{Ni2012Inproceedings,
	Author = {D. Ni, H. Liu, W. Ding, Y. Xie, H. Wang, H. Pishro-Nik and Q. Yu,},
	Booktitle = {IEA/AIE},
	Date-Added = {2016-09-04 09:11:02 +0000},
	Date-Modified = {2016-09-06 14:46:53 +0000},
	Title = {Cyber-Physical Integration to Connect Vehicles for Transformed Transportation Safety and Efficiency},
	Year = {2012}}


@article{Nekoui_IJIPT_2009,
	Author = {M. Nekoui and D. Ni and H. Pishro-Nik and R. Prasad and M. Kanjee and H. Zhu and T. Nguyen},
	Journal = {International Journal of Internet Protocol Technology (IJIPT)},
	Number = {3},
	Pages = {},
	Publisher = {},
	Title = {Development of a VII-Enabled Prototype Intersection Collision Warning System},
	Volume = {4},
	Year = {2009}}


@inproceedings{Pishro_Ganz_Ni,
	Author = {H. Pishro-Nik, A. Ganz, and Daiheng Ni},
	Booktitle = {Forty-Fifth Annual Allerton Conference on Communication, Control, and Computing. Allerton House, Monticello, IL},
	Date-Added = {},
	Date-Modified = {},
	Number = {},
	Pages = {},
	Title = {The capacity of vehicular ad hoc networks},
	Volume = {},
	Year = {September 26-28, 2007}}

@inproceedings{Leow_Pishro_Ni_1,
	Author = {W. L. Leow, H. Pishro-Nik and Daiheng Ni},
	Booktitle = {IEEE Global Telecommunications Conference, Washington, D.C.},
	Date-Added = {},
	Date-Modified = {},
	Number = {},
	Pages = {},
	Title = {Delay and Energy Tradeoff in Multi-state Wireless Sensor Networks},
	Volume = {},
	Year = {November 26-30, 2007}}


@misc{UMass-Trans,
title = {{UMass Transportation Center}},
note = {\url{http://www.umasstransportationcenter.org/}},
}


@inproceedings{Haenggi2013book,
	title={Stochastic geometry for wireless networks},
	author={M. Haenggi},
	organization={Cambridge Uinversity Press},
	year={2013}
}


%\renewcommand{\baselinestretch}{1.0}


\bsp

\label{lastpage}

\end{document}






