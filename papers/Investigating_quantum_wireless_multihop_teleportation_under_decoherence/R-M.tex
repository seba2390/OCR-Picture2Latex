\documentclass[aps,pra,twocolumn,showpacs,superscriptaddress,groupedaddress]{revtex4}
\usepackage{graphicx} 
\usepackage{dcolumn}   
\usepackage{bm}        
\usepackage{amssymb}   
\hyphenation{ALPGEN}
\hyphenation{EVTGEN}
\hyphenation{PYTHIA}
\usepackage{amsmath}

\begin{document}
\preprint{APS/123-QED}
\title{Investigating quantum wireless multihop teleportation under decoherence\footnote{https://arxiv.org/abs/1708.00087}}
\author{Marlon David Gonz\'alez Ram\'irez}
\affiliation{CIDETEC, Instituto Polit\'{e}cnico Nacional, UPALM, CDMX 07700, M{e}xico}
\author{Christian Nwachioma}
\affiliation{Department of Physics, COMSATS Institute of Information Technology, Park Road, Islamabad 45550, Pakistan.}
\affiliation{CIDETEC, Instituto Polit\'{e}cnico Nacional, UPALM, CDMX 07700, M{e}xico}
\author{Adenike Grace Adepoju}
\affiliation{CIDETEC, Instituto Polit\'{e}cnico Nacional, UPALM, CDMX 07700, M{e}xico}
\author{Babatunde James Falaye}
\email{babatunde.falaye@gmail.com}
\affiliation{Applied Theoretical Physics Division, Department of Physics, Federal University Lafia,  P. M. B. 146, Lafia, Nigeria}
\date{\today}


\begin{abstract}
This research work scrutinizes quantum routing protocol with multihop teleportation for wireless mesh backbone networks, in amplitude and phase damping channels. After analyzing the quantum multihop protocol, we select a four-qubit cluster state as the quantum channel for the protocol. The quantum channel linking the intermediate nodes has been established via entanglement swapping based on four-qubit cluster state. Also, we established the classical and the quantum route in a distributed manner. We show that from the source node to the destination node, quantum information can be teleported hop-by-hop through an amplitude damping channel. We show that the quantum teleportation could be successful if the sender node performs Bell state measurements (BSM), and the receiver introduces auxiliary particles, applies positive operative value measure and then utilizes corresponding unitary transformation to recover the transmitted state. We scrutinize the success probability of transferring the quantum state through a noisy channel. We found that optimum probability would be attained if decoherence rate of amplitude damping channel ($\xi_a$) is zero or the number of hops ($N$) is above $75$. Our numerical results evince susceptibility of success probability to $\xi_a$ and $N$. It has been shown that as the decoherence increases, the fidelity exponentially decays until it vanishes. This decay is as a consequence of information loss from the system to the surrounding. However, the fidelity can be enhanced by considering fewer hops.\vspace{5mm}\\
{\bf Keywords}:  Multihop teleportation; Amplitude damping channel; Phase damping channel; Four-qubit cluster state; Fidelity.
\pacs{03.67.Hk, 03.65.Ud, 42.50.-p, 03.67.-a, 03.65.Yz.}
\end{abstract}
\maketitle

\section{Introduction}
In the recent years, there has been incessant avidity in studying multi-user quantum communication because it offers the opportunity to construct quantum networks. With quantum networks, quantum information between physically separate quantum systems can be transmitted. In fact, it forms a salient component of quantum computing and quantum cryptography systems. It has been discerned that transmission via quantum teleportation and, directly from one node to another are two methods to transmit an unknown quantum state between two nodes \cite{MA1}. In this paper, the later, i.e.,  the node-to-node transmission will be considered. Since quantum systems unavoidably interact with the environment, node-to-node transmission easily debase the quantum states.  Consequently, it would be reasonable to take the influence of noise into consideration while investigating quantum wireless multihop teleportation.

Recently, a scheme for faithful quantum communication in quantum wireless multihop networks, by performing quantum teleportation between two distant nodes which do not initially share entanglement with each other, was proposed by Wang et al. \cite{MA2}. It has been found in Ref. \cite{RE1} that wireless quantum networks can be established between nodes of different hops sharing two qubit states. Xiong et al. \cite{MA3} proposed a quantum communication network model where a mesh backbone network structure was introduced. The entanglement source deployment problem has been scrutinized by Zou {\it et al.} \cite{RE2} in a quantum multihop network, which has a  notable impact on quantum connectivity.

Some other outstanding reports can be found in Refs. (\cite{MA4,RE3,RE4,RE5} and references therein). Although several relevant results have been obtained along this direction, most contributions have been based on 2- and 3-qubit Greenberger-Horne-Zeilinger (GHZ) state as the quantum channel in a closed quantum system. However, quantum systems cannot but interact with the environment. These unavoidable interactions make quantum systems to lose some of their properties. In this paper, we examine a quantum routing protocol with multihop teleportation for wireless mesh backbone networks, based on four-qubit cluster state, in an amplitude damping channel, which can be induced experimentally.

The cluster state \cite{MA6}, which is a type of highly entangled state of multiple qubits, is generated in lattices of qubits with Ising type interactions. On the basis of single qubit operation, the cluster state serves as the initial resource for a universal computation scheme \cite{MA7}. Cluster state has been realized experimentally in photonic experiments \cite{MA7} and in optical lattices of cold atoms \cite{MA8}. In this paper, we select four-qubit cluster state as the entangled resource. When the state of one particle in the entangled pair changes, the state of other particle situated at a distant node changes non-contemporaneously. Thus, entanglement swapping can be applied. Using a classical communication channel, the results of the local measurement can be transmitted from node-to-node.

The rest of this paper is organized as follows. In Section II, we discuss wireless network and routing protocol. Section III deals with the process of establishing the quantum channel. In section IV, we present quantum wireless multihop teleportation in noisy channels. Some relevant results and discussion are presented in Section V. Section VI gives the conclusion.
\begin{figure*}[!ht]
\centering \includegraphics[width=\linewidth]{MFIG-1.eps}
\caption{\protect\small The quantum mesh backbone network. The dotted lines represent quantum channels while the solid lines denote classical channels. Node {\bf S} is not directly entangled with the node {\bf D}. However, quantum channels between them can be established via entanglement swapping.}
\label{Mfig1}
\end{figure*}

\section{Wireless mesh network and routing protocol}
A wireless mesh network (WMN) can be described as a mesh network established through the connection of wireless access points which have been installed at the location of each network users. It consists of mesh routers, which are stationary, and the mesh client, which are removable. In WMN, there exist quantum wireless channel and classical wireless channel for communications. The classical channel serves the purpose of classical information transmission while the quantum wireless channel exists between neighbor nodes. Quantum information can be transmitted from node-to-node only when quantum route and classical route co-exist. Classical information is transmitted along the classical route while quantum information is via the quantum route. 

In wireless quantum communication, there exists mesh backbone network which consists of route nodes and edge route nodes. We delineate the quantum mesh network in Fig. \ref{Mfig1}. Node {\bf S} wishes to send information to node {\bf D}. To achieve this, it scrutinizes its routing table to find if there is any available route to {\bf D}. If there are available routes, it forwards the packet to next hop node. However, in the absence of none, source node {\bf S} requests for a quantum route discovery from the neighboring node {\bf E} and thus, the quantum route finding process commences. Once a routing path that permits co-existence of quantum and classical route, from the source node to the destination is found and selected, the edge route node {\bf I} sends a route reply to node {\bf S}. At this moment, the process of establishing the quantum channel commences.

\section{Process of establishing the quantum channel}
In this section, we establish the quantum channel linking the nodes. As it can be seen in Figs. \ref{Mfig2} (a) and (b), {\bf S} denotes the source node while the destination node is denoted by {\bf D}. The node  {\bf S} is not directly entangled with {\bf D} but entanglement swapping can be used to set-up quantum channels between the two nodes. Thus, with this swapping, quantum information can be transmitted hop-by-hop from node  {\bf S} to node {\bf D}. In the source node {\bf S}, there exists an arbitrary two-qubit entangled state, $\left|\chi\right\rangle_{S_1S_2}=a_0\left|00\right\rangle+d_0\left|11\right\rangle$, whose density matrix can be written as:
\begin{equation}
\rho_{\rm in_S}=\left[\begin{matrix}a_0^*a_0&0&0&d_0^*a_0\\
                                        0&0&0&0\\
                                        0&0&0&0\\
                                  a_0^*d_0&0&0&d_0^*d_0
																	\end{matrix}\right].
\label{SUS1}
\end{equation}
Let $N_n$ denotes number of nodes such that between {\bf S} and {\bf D}, we have $N_n-1$ nodes. Also, let us represent the number of hops by $N$. The entangle state of the neighboring nodes is 4-qubit cluster state of the form $\left|\mathcal{CS}\right\rangle=\tau_0\left|0000\right\rangle+\tau_1\left|0011\right\rangle+\tau_2\left|1100\right\rangle-\tau_3\left|1111\right\rangle$ where $\tau_0^2+\tau_1^2+\tau_2^2+\tau_3^3=1$. Now, the edge node {\bf I} performs Bell state measurement on particle pairs $(I_3,I_4)$ and $(I_1,I_2)$ to obtain
\begin{widetext}
\begin{eqnarray}
&&\left|\Pi\right\rangle=\nonumber\\
&&\left|\chi\right\rangle_{S_1S_2}\otimes\frac{1}{8}\sum_{\varsigma,\kappa\in[+,-]}\bigg[\left|\Phi^\varsigma\right\rangle_{I_3I_4}\left|\Phi^\kappa\right\rangle_{I_1I_2}\left(\left|0000\right\rangle+\left|0011\right\rangle\pm^{(\varrho)}\mp^{(\zeta)}\left|1100\right\rangle\pm^{(\varrho)}\pm^{(\zeta)}\left|1111\right\rangle\right)_{R_4,D_1,D_2,D_3}\nonumber\\
&&\ \ \ \ \ \ \ \ \ \ \ \ \ \ \ \ \ \ \ \ \ \ \ \ \ \ +\left|\Psi^\varsigma\right\rangle_{I_3I_4}\left|\Phi^\kappa\right\rangle_{I_1I_2}\left(\left|0100\right\rangle-\left|0111\right\rangle\pm^{(\varrho)}\mp^{(\zeta)}\left|1000\right\rangle\pm^{(\varrho)}\mp^{(\zeta)}\left|1011\right\rangle\right)_{R_4,D_1,D_2,D_3}\nonumber\\
&&\ \ \ \ \ \ \ \ \ \ \ \ \ \ \ \ \ \ \ \ \ \ \ \ \ \ +\left|\Phi^\varsigma\right\rangle_{I_3I_4}\left|\Psi^\kappa\right\rangle_{I_1I_2}\left(\pm^{(\varrho)}\left|0100\right\rangle\mp^{(\varrho)}\left|0111\right\rangle\pm^{(\zeta)}\left|1000\right\rangle\pm^{(\zeta)}\left|1011\right\rangle\right)_{R_4,D_1,D_2,D_3}\nonumber\\
&&\ \ \ \ \ \ \ \ \ \ \ \ \ \ \ \ \ \ \ \ \ \ \ \ \ \  +\left|\Psi^\varsigma\right\rangle_{I_3I_4}\left|\Psi^\kappa\right\rangle_{I_1I_2}\left(\pm^{(\varrho)}\left|0000\right\rangle\pm^{(\varrho)}\left|0011\right\rangle\pm^{(\zeta)}\left|1100\right\rangle\mp^{(\zeta)}\left|1111\right\rangle\right)_{R_4,D_1,D_2,D_3}\bigg]\nonumber\\
&&\ \ \ \ \ \ \ \ \ \ \ \ \ \ \ \ \ \ \ \ \ \ \ \ \ \  \otimes_{i=3}^N\left|\mathcal{CS}\right\rangle^{n_i},
\label{SUS2}
\end{eqnarray}
\end{widetext}
where $n_i=n$, and $i=1,...,N$. We have denoted the four Bell states as $\left|\Phi^{\pm}\right\rangle=2^{-1/2}(\left|00\right\rangle\pm\left|11\right\rangle)$ and $\left|\Psi^{\pm}\right\rangle=2^{-1/2}(\left|01\right\rangle\pm\left|10\right\rangle)$. The $\pm^{(\varrho)}, \mp^{(\varrho)}$ and $\pm^{(\zeta)},\mp^{(\zeta)}$  represent the results corresponding to BSM on qubit pairs ($I_3,I_4$) and ($I_1,I_2$) respectively. Now, with the application of proper Pauli operator on qubit $R_4$, the entangled state of $R_4,D_1,D_2,D_3$ can be realized. For instance, if the entangled state of $R_4,D_1,D_2,D_3$ is $\left|0100\right\rangle-\left|0111\right\rangle-\left|1000\right\rangle-\left|1011\right\rangle$, applying the Pauli $z$ matrix and then the Pauli $x$, entangled state $\left|\mathcal{CS}\right\rangle$ can be realized. Now, edge route node {\bf I} sends the result of the measurements along with route reply to node {\bf S} through edge node {\bf E}. Once node {\bf S} receives the information, quantum channel between nodes {\bf S} and {\bf D} would be established. Thus the quantum state can then be transferred from node {\bf A} to node {\bf J}. A similar study was reported recently in Ref. \cite{MA3}. However, in the present study, we shall consider the viability of a cluster state as the entanglement resource and the influence of noisy channels on the multihop teleportation.
 
\section{Quantum wireless multihop teleportation in noisy channels}
In this section, we investigate the influence of noisy channel on quantum wireless multihop teleportation. Two noisy channels: amplitude and phase damping channels would be considered.

\subsection{Quantum wireless multihop teleportation in amplitude damping channel}
The general behavior of model for amplitude damping channel is characterized by the following set of Kraus operators \cite{MA10}
\begin{equation}
\mathcal{K}_0^{\rm Am}=\left[\begin{matrix}1&0\\0&\sqrt{\bar{\xi_{a}}}\end{matrix}\right], \mathcal{K}_1^{\rm Am}=\left[\begin{matrix}0&\sqrt{\xi_{a}}\\0&0\end{matrix}\right],
\label{SUS3}
\end{equation}
where $\xi_{a}(0\leq\xi_{a}\leq 1)$ represents the decoherence rate which characterize the probability error of amplitude-damping when a particle passes through a noisy environment and $\bar{\xi}_{a}=1-\xi_{a}$. 

To achieve the aim of this section, first, let us consider a one-hop teleportation through the amplitude damping channel. From Figs. \ref{Mfig2} (a) and (b), it can be seen that {\bf S} is a neighborhood node of {\bf E} and consequently, we can infer that particles $S_3$, $E_1$, $E_3$ and $E_2$ are entangled such that the quantum entanglement is $\left|\mathcal{CS}\right\rangle_{S_3E_1E_3E_2}$. In order to transmit $\left|\chi\right\rangle_{S_1S_2}$ through a noisy channel, we perform Bell state measurements on particles pairs $(S_1,S_3)$ and $(S_2,E_3)$, and then take the noise model into consideration, thus we obtain.
\begin{eqnarray}
\left|\Omega\right\rangle&=&\frac{1}{2}\sum_{\mathcal{A},\mathcal{B}}\ _{S_2E_3}\left\langle\mathcal{A}^{\pm}\right| _{S_1S_3}\left\langle\mathcal{B}^{\pm}\right.\left|\Pi''\right\rangle,\ \ \ \ \ \mathcal{A},\mathcal{B}\in\left[\Phi,\Psi\right],\nonumber\\
&&\mbox{with}\ \ \ \left|\Pi''\right\rangle=\left|\chi\right\rangle\otimes\left|\mathcal{CS'}\right\rangle,
\label{SUS5}
\end{eqnarray}
where we have used the following formulas for mathematical simplicity:
\begin{widetext}
\begin{eqnarray}
_{S_2E_3}\left\langle\Psi^{\pm}\right|\ _{S_1S_3}\left\langle\Psi^{\pm}\right.\left|\Pi''\right\rangle&=&\mp^{(\varsigma)}\mp^{(\sigma)}\bar{\xi}_a^2d_0\tau_0\left|00\right\rangle_{E_1E_3}+a_0\left(\xi_a^2\tau_0-\tau_3\right)\left|11\right\rangle_{E_1E_2}\nonumber\\
_{S_2E_3}\left\langle\Phi^{\pm}\right|\ _{S_1S_3}\left\langle\Psi^{\pm}\right.\left|\Pi''\right\rangle&=&\pm^{(\varsigma)}\pm^{(\sigma)}\bar{\xi}_ad_0\tau_1\left|01\right\rangle_{E_1E_3}+\bar{\xi}_aa_0\tau_2\left|10\right\rangle_{E_1E_2}\nonumber\\
_{S_2E_3}\left\langle\Psi^{\pm}\right|\ _{S_1S_3}\left\langle\Phi^{\pm}\right.\left|\Pi''\right\rangle&=&\pm^{(\varsigma)}\pm^{(\sigma)}\bar{\xi}_ad_0\tau_2\left|10\right\rangle_{E_1E_3}+\bar{\xi}_aa_0\tau_1\left|01\right\rangle_{E_1E_2}\nonumber\\
_{S_2E_3}\left\langle\Phi^{\pm}\right|\ _{S_1S_3}\left\langle\Phi^{\pm}\right.\left|\Pi''\right\rangle&=&\pm^{(\varsigma)}\mp^{(\sigma)}d_0\left(\xi_a^2\tau_0-\tau_3\right)\left|11\right\rangle_{E_1E_3}+\bar{\xi}_a^2a_0\tau_0\left|00\right\rangle_{E_1E_2}.
\label{SUS6}
\end{eqnarray}
\end{widetext}

\begin{figure*}[!t]
\centering \includegraphics[width=\linewidth]{MFIG-2.eps}
\caption{\protect\small The process of establishing quantum channel. (a) Before route-finding process. (b) After route-finding process. The red lines represent Bell state measurement.}
\label{Mfig2}
\end{figure*}
The $\pm^{(\varsigma)}$ and $\pm^{(\sigma)},\mp^{(\sigma)}$ represent the results corresponding to BSM on qubit pairs ($S_1,S_3$) and ($S_2,E_3$) respectively. Now, node {\bf S} transmits this result to node {\bf E}. By using an appropriate unitary transformation, the state $\left|\chi\right\rangle$ can be retrieved. Thus, the quantum communication is completed successfully. The output density that reaches node {\bf E} is
\begin{widetext}
\begin{eqnarray}
\rho_{\rm out_E}=\left[\begin{matrix}\left|a_0\right|^2\left[\tau_0^2\left(\bar{\xi}_a^4+\xi_a^4\right)+\left(\tau_1^2+\tau_2^2\right)\bar{\xi}_a^2-2\tau_0\tau_3\xi_a^2+\tau_3^2\right]&0&0&d_0^*a_0\\
                                        0&0&0&0\\
                                        0&0&0&0\\
                                  a_0^*d_0&0&0&\left|d_0\right|^2\left[\tau_0^2\left(\bar{\xi}_a^4+\xi_a^4\right)+\left(\tau_1^2+\tau_2^2\right)\bar{\xi}_a^2-2\tau_0\tau_3\xi_a^2+\tau_3^2\right]
																	\label{SUS7}
\end{matrix}\right].\nonumber\\
\end{eqnarray}
\end{widetext}

Accordingly to Eq. \ref{SUS6}, the states of $E_1, E_2$ are not normalized, it then implies that each outcome of the measurement has different probability. In order to avoid redundancy, we shall not discuss all the outcomes but one. For other cases, node {\bf E} can apply similar approach to reconstruct the original state. Now, suppose the result of Bell state measurement is $\left|\Psi^{+}\right\rangle_{S_1,S_3}\left|\Phi^{-}\right\rangle_{S_2,E_3}$, consequently, without loss of generality, the state of qubit pair $(E_1,E_3)$ collapses to $\mathcal{G}=2^{-1}\bar{\xi}_a(a_0\tau_2\left|10\right\rangle_{E_1E_2}-d_0\tau_1\left|01\right\rangle_{E_1,E_3})$. With this result, the state $\left|\chi\right\rangle$ can be recovered at Node {\bf E}. In order for this to be achieved, it is required to apply positive-operator valued measure (POVM) \cite{MA9}.

In utilizing POVM, first an ancilla (i.e., auxiliary quantum system) is prepared in a known state, say $\rho_{anc}$. Combining this ancilla with the original quantum state gives an uncorrelated state. Now, the combined Hilbert space will be subjected to a maximal test which is represented by orthogonal resolution of the identity. The results of this test is related to orthogonal projector which satisfies the relations: $\mathcal{K}_\mu \mathcal{K}_\nu=\delta_{\mu\nu} \mathcal{K}_\nu$ and $\sum_\mu \mathcal{K}_\mu =1$. The probability that preparation $k$ will be followed by outcome $\mu$ is represented by $\mathcal{K}_{\mu k}=Tr(A_\mu\rho_k)$, where $A_\mu$ denotes an operator acting on the Hilbert space. The set of $A_\mu$ is called POVM \cite{MA9}. Several studies and applications of POVM have been noted down in many literature \cite{MA91,MA92,MA93,MA94}. The current study will also utilize it.

To do that, we need to set up a close indistinguishability such that the coefficient of $\left|00\right\rangle_{E_1,E_2}$ is $a_0$ and that of $\left|11\right\rangle_{E_1,E_2}$ should be $d_0$. To accomplish this, node {\bf E} performs a local unitary operation $\mathcal{UT}=\sigma_x\otimes I_{2\times 2}$ on $\mathcal{G}$ to obtain $\mathcal{G}_0=2^{-1}\bar{\xi}_a(a_0\tau_2\left|00\right\rangle_{E_1E_2}-d_0\tau_1\left|11\right\rangle_{E_1E_3})$.  Now, the node introduces auxiliary qubits, say $\mathcal{DE}$ with state $\left|00\right\rangle_{\mathcal{DE}}$. Entangling these qubits with $\mathcal{G}_0$ gives $\mathcal{G}_1=2^{-1}\bar{\xi}_a(a_0\tau_2\left|0000\right\rangle_{E_1E_2\mathcal{DE}}-d_0\tau_1\left|1100\right\rangle_{E_1E_2\mathcal{DE}})$. The node then performs a C-NOT operation on qubit pairs $(E_1,\mathcal{D})$ and $(E_2,\mathcal{E})$ to obtain 
\begin{eqnarray}
\mathcal{G}_2&=&1/4\left(a_0\left|00\right\rangle_{E_1E_2}+d_0\left|11\right\rangle_{E_1E_2}\right)\nonumber\\
&&\ \otimes\left(\bar{\xi}_a\tau_2\left|00\right\rangle_{\mathcal{DE}}-\bar{\xi}_a\tau_1\left|11\right\rangle_{\mathcal{DE}}\right)\nonumber\\
&&\ +\left(a_0\left|00\right\rangle_{E_1E_2}-d_0\left|11\right\rangle_{E_1E_2}\right)\nonumber\\
&&\ \otimes\left(\bar{\xi}_a\tau_2\left|00\right\rangle_{\mathcal{DE}}+\bar{\xi}_a\tau_1\left|11\right\rangle_{\mathcal{DE}}\right).
\end{eqnarray}
With the condition that $\tau_2\left|00\right\rangle_{\mathcal{DE}}\mp\tau_1\left|11\right\rangle_{\mathcal{DE}}$ can be conclusively discerned using a suitable measurement, state $\left(a_0\left|00\right\rangle_{E_1E_2}\pm d_0\left|11\right\rangle_{E_1E_2}\right)$ can be obtained at node {\bf E}. The optimal POVM required can be written in the following subspace
\begin{eqnarray}
&&\mathcal{P}_1=\frac{1}{{\varrho}}\left|\Lambda_1\right\rangle\left\langle\Lambda_1\right|,\ \ \ \mathcal{P}_2=\frac{1}{{\varrho}}\left|\Lambda_2\right\rangle\left\langle\Lambda_2\right|,\nonumber\\
&&\mathcal{P}_3=I-\frac{1}{\varrho}\sum_{i=1}^2\left|\Lambda_i\right\rangle\left\langle\Lambda_i\right|\label{EQ4},
\end{eqnarray}
where
\begin{eqnarray}
&&\left|\Lambda_1\right\rangle=\frac{1}{\sqrt{\gamma_a}}\left(\frac{1}{\bar{\xi}_a\tau_2}\left|00\right\rangle-\frac{1}{\bar{\xi}_a\tau_1}\left|11\right\rangle\right)_{\mathcal{DE}},\nonumber\\
&&\left|\Lambda_2\right\rangle=\frac{1}{\sqrt{\gamma_a}}\left(\frac{1}{\bar{\xi}_a\tau_2}\left.|00\right\rangle+\frac{1}{\bar{\xi}_a\tau_1}\left|11\right\rangle\right)_{\mathcal{DE}},\nonumber\\
&&\mbox{with}\ \ \gamma_a=\frac{1}{\left(\bar{\xi}_a\tau_1\right)^2}+\frac{1}{\left(\bar{\xi}_a\tau_2\right)^2}.\label{EQ5}
\end{eqnarray}

\begin{figure*}[!t]
\centering \includegraphics[width=\linewidth]{MFIG-3.eps}
\caption{\protect\footnotesize (a) Variation of the probability of success as a function of number of hops ($N$) for various decoherence rate of amplitude damping channel $\xi_a$.  (b) Variation of the probability of success as a function of $\xi_a$ for various $N$. (c) Plot of fidelity as a function of $\xi_a$ for various $N$. (d) Plot of fidelity as a function of $N$ for various $\xi_a$.   We take $\varrho=1$ in our numerical computations.} 
\label{Mfig3}
\end{figure*}
\begin{figure*}[!t]
\centering \includegraphics[width=\linewidth]{MFIG-4.eps}
\caption{\protect\footnotesize (a) Variation of the probability of success as a function of number of hops ($N$) for various decoherence rate of phase damping channel $\xi_p$.  (b) Variation of the probability of success as a function of $\xi_p$ for various $N$. (c) Plot of fidelity as a function of $\xi_p$ for various $N$. (d) Plot of fidelity as a function of $N$ for various $\xi_p$.   We take $\varrho=1$ in our numerical computations.} 
\label{Mfig4}
\end{figure*}
$I$ denotes an identity operator and $\varrho$ is a parameter which defines the range of positivity of operator $\mathcal{P}_3$. Now, if the node's POVM result yields $\mathcal{P}_1$ whose probability is $\left\langle\mathcal{G}_1\right|\mathcal{P}_1\left|\mathcal{G}_1\right\rangle=1/(4\varrho\gamma_a)$, then one can infer the state of qubits $E_1E_2$ to be $a_0\left|00\right\rangle-d_0\left|11\right\rangle$. Afterward, the node  performs unitary operation $I_{2\times 2}\otimes\sigma_z$ on the particles in order to retrieve the original state. However, suppose the result is $\mathcal{P}_2$, with a probability calculated by $\left\langle\mathcal{G}_1\right.|\mathcal{P}_2|\left.\mathcal{G}_1\right\rangle=1/(4\varrho\gamma_a)$, then the node finds that the state of qubits $E_1E_2$ is $a_0\left|00\right\rangle+d_0\left|11\right\rangle$ which is the original state of the particle.  Suppose the result is $\mathcal{P}_3$, the teleportation fails because of the node's ineptitude to infer anything about the identity of the particles state. Thus the success probability becomes
\begin{equation}
\text{P}_{\text{suc}}^{\rm\bf S-E}=\frac{1}{2\varrho\gamma_a},
\end{equation}
and using Eqs. (\ref{SUS1}) and Eqs. (\ref{SUS7}), we obtain the analytical expression for the fidelity as
\begin{eqnarray}
F_{\rm\bf S-E}&=&\left[\tau_0^2\left(\bar{\xi}_a^4+\xi_a^4\right)+\left(\tau_1^2+\tau_2^2\right)\bar{\xi}_a^2-2\tau_0\tau_3\xi_a^2+\tau_3^2\right]\nonumber\\
&&\times\left(\left|a_0\right|^2+\left|d_0\right|^2\right)^2.
\end{eqnarray} 
Now, let us  consider two-hop teleportation, i.e., {\bf S-E-T}. There exit no direct quantum channel between node {\bf S} and {\bf T}. However, nodes {\bf S} and {\bf E} are entangled while nodes  {\bf E} and  {\bf T} are also entangled. In that case, node  {\bf S} transmits $\rho_{\rm in_S}$ through a noisy channel to node {\bf E} and then node  {\bf E} transmit $\rho_{\rm out_E}$ to node {\bf T} through a noisy channel. Consequently, the quantum channel between nodes {\bf S} and {\bf T} will be established. Following the same procedure, it is easy to find that the success probability is
\begin{equation}
\text{P}_{\text{suc}}^{\rm\bf E-T}=\frac{1}{\varrho\gamma_a}\left(1-\frac{1}{4\varrho\gamma_a}\right),
\end{equation}
 and the analytical expression for the fidelity 
\begin{eqnarray}
F_{\rm\bf E-T}&=&\left[\tau_0^2\left(\bar{\xi}_a^4+\xi_a^4\right)+\left(\tau_1^2+\tau_2^2\right)\bar{\xi}_a^2-2\tau_0\tau_3\xi_a^2+\tau_3^2\right]^3\nonumber\\
&&\times\left(\left|a_0\right|^2+\left|d_0\right|^2\right)^2.
\end{eqnarray} 
Now, let's assume that the information is transmitted to node $\mathcal{N}^{i}$, where $i=2,4,6,...2k$ denotes the number of Bell state measurement needed to be performed. In that sense, the source node is equivalent to $\mathcal{N}^{0}$, node {\bf E} is equivalent to  $\mathcal{N}^{2}$ and the destination is equivalent to  $\mathcal{N}^{2k}$.  Thus, then the total probability of success can be written as
\begin{equation}
\text{P}_{\text{suc}}^{a}=1-\left(1-\frac{1}{2\varrho\gamma_a}\right)^{N},\label{EQ6}
\end{equation}
and the fidelity of the complete multihop teleportation becomes
\begin{eqnarray}
F_{\rm tot.}^a&=&\left[\tau_0^2\left(\bar{\xi}_a^4+\xi_a^4\right)+\left(\tau_1^2+\tau_2^2\right)\bar{\xi}_a^2-2\tau_0\tau_3\xi_a^2+\tau_3^2\right]^{2N}\nonumber\\
&&\times\left(\left|a_0\right|^2+\left|d_0\right|^2\right)^2.
\end{eqnarray} 
We would like to remind the readers that the above equation should not be confused with the fidelity of the N-th hop teleportation. 

\subsection{Quantum wireless multihop teleportation in phase damping channel}
The general behavior of the model for phase damping channel is characterized by the following set of Kraus operators \cite{MA10}
\begin{equation}
\mathcal{K}_0^{\rm Ph}=\sqrt{\bar{\xi_{p}}}\left[\begin{matrix}1&0\\0&1\end{matrix}\right], \mathcal{K}_1^{\rm Ph}=\sqrt{\xi_{p}}\left[\begin{matrix}1&0\\0&0\end{matrix}\right], \mathcal{K}_2^{\rm Ph}=\sqrt{\xi_{p}}\left[\begin{matrix}0&0\\0&1\end{matrix}\right]
\label{}
\end{equation}
where $\xi_p(0\leq\xi_p\leq1)$ represents the decoherence rate of phase damping noise and $\bar{\xi_{p}}=1-\xi_{p}$. Following the same procedure of section IV A, we can calculate the total probability of success of multihop teleportation from source to the destination node as
\begin{equation}
\text{P}_{\text{suc}}^{p}=1-\left(1-\frac{1}{2\varrho\gamma_p}\right)^{N}, \ \mbox{where}\ \gamma_p=\frac{1}{\left(\bar{\xi}_p^2\tau_1\right)^2}+\frac{1}{\left(\bar{\xi}_p^2\tau_2\right)^2}.\label{}
\end{equation}
and the fidelity of the complete multihop teleportation becomes
\begin{eqnarray}
F_{\rm tot.}^p&=&\big[\tau_0^2\left(\bar{\xi}_p^2+\xi_p^2\right)^2+\left(\tau_1^2+\tau_2^2+\tau_3^2\right)\bar{\xi}_p^4+2\tau_3^2\xi^2_p\bar{\xi}_p^2\nonumber\\
&&{\xi}_p^4\tau_3^2\big]^{2N}\left(\left|a_0\right|^2+\left|d_0\right|^2\right)^2.
\end{eqnarray} 


\section{Results and Discussion}
Figure \ref{Mfig3}(a) gives the variation of success chance as a function of the number of hops for various decoherence rate of amplitude damping channel ($\xi_a$). We first consider a case where there is no interaction between the quantum system and the environment (i.e. $\xi_a=0$). We observe that the success probability proliferates until it reaches $\approx1$ (at $N\approx75$) where no significant variation can be discerned. However, as the $\xi_a$ increases, the probability of having successful teleportation decreases until it finally varnished. This insinuates that the success chance of multihop teleportation in a very noisy channel is very slim.  Figure  \ref{Mfig3} (b) explains this further. In addition, this figure reveals that the success probability of multihop teleportation in a noisy environment can be improved within some range of $\xi_a$ by increasing the number of nodes.  Figure \ref{Mfig3}(b), indeed shows the susceptibility of P$_{\text{suc}}$ to $N$ and $\xi_a$. Furthermore, Fig. \ref{Mfig3}(a) shows that optimum probability would be attained if $\xi_a=0$ and $N>75$. Figure \ref{Mfig3}(b) corroborates this fact

In Figures \ref{Mfig3}(c) and (d), we show the variation of fidelity as a function of decoherence rate and the number of hops. As it can be found in Figures \ref{Mfig3}(c), irrespective of the number of hops, the fidelity dwindles as the decoherence increases. This figure reveals that information loss can be minimized via considering few nodes. In other words, the more the nodes, the less the fidelity of output quantum system at the destination node. As anticipated, in \ref{Mfig3}(d), it is found that when there is no interaction with the environment, the fidelity is 1 for any number of hops. However, as the decoherence increases, the fidelity decays until it vanishes. This decay is as a consequence of information loss from the system to the surrounding. Figures \ref{Mfig3}(a-d) show that success probability and fidelity are highly sensitive to variations in $N$.

Figure \ref{Mfig4}(a) is the same as \ref{Mfig3} (a) but for phase damping channel. This figure reveals that if the noisy channel is phase damping channel, then about 80 hops would be needed in order to obtain success probability of 1. This shows that phase damping channel is noisier than the amplitude damping channel. Figure \ref{Mfig4} (b) explains this further. Figure \ref{Mfig4}(c) reveals a situation whereby more noise leads to more efficiency, for $N = 2$. In Figure \ref{Mfig4}(d), we found that the fidelity decays more rapidly than its counterpart in phase damping channel. The effect of phase damping channel is more dominant on the fidelity and success probability of the state being teleported as compared to amplitude damping channel. From Figures \ref{Mfig3} and \ref{Mfig4}, we found that the multihop teleportation with fewer hops are more prone to noise in comparison to more hops.

Figures \ref{Mfig3}(a), \ref{Mfig3}(b), \ref{Mfig4}(a) and \ref{Mfig4}(b) shows that success probability increases as the number hop increase. This is because increase in hop’s number leads to a corresponding boost of the network connectivity. Thus, the probability of success is enhanced greatly.

\section{Conclusion}
To sum it up, in this paper, we have examined a quantum routing protocol with multihop teleportation for wireless mesh backbone networks in amplitude and phase damping channels. The quantum channel that linked the intermediate nodes has been realized through entanglement swapping based on four-qubit cluster state. After quantum entanglement swapping, the quantum link was established between the source node and the destination node and quantum states are transferred via quantum teleportation from one hop to the other. We have shown that the quantum teleportation would be successful if the sender node performs a Bell state measurement, and the receiver node introduces auxiliary particles, applies positive operative value measure and then utilizes corresponding unitary transformation to recover the transmitted state. We have numerically scrutinized the success probability of transferring the quantum state through an amplitude and phase damping channel.  We found that, as the decoherence increases, the probability of having successful teleportation decreases until it finally becomes zero which insinuates that the success chance of multihop teleportation in a very noisy channel is very slim. It has also been shown that the enhancement of the fidelity becomes smaller as the number of hops increase. This corroborates result of Ref. \cite{MA12} i.e., the best transmission efficiency can be achieved via using of quantum routing protocol based on the minimal hops principle. This study represents the furtherance of recent studies \cite{MA2,MA3,MA93,MA10,MA11,MA12}.

\section*{Acknowledgments}
We thank the referees for the positive enlightening comments and suggestions, which have greatly
helped us in making improvements to this paper. BJF acknowledges the supports from CONACYT during his stay at ESFM-IPN, where he started this work.

\begin{thebibliography}{99} 
\bibitem{MA1} C. Pan, F.G. Deng and G. L. Long, ``High-dimension multiparty quantum secret sharing scheme with Einstein-Podolsky-Rosen pairs", Chin. Phys. {\bf15} (2006) 2228.
\bibitem{MA2} K. Wang, X. T. Yu, S. L. Lu, Y. X. Gong, ``Quantum wireless multihop communication based on arbitrary Bell pairs and teleportation", Phys. Rev. A {\bf89} (2014) 022329.
\bibitem{MA3} P. Y. Xiong, X. T. Yu, Z. C. Zhang, H. T. Zhan and J. Y. Hua, ``Routing protocol for wireless quantum multi-hop mesh backbone network based on partially entangled GHZ state", Front. Phys. {\bf12} (2017) 120302.
\bibitem{RE1} N. Metwally, ``Entanglement routers via a wireless quantum network based on arbitrary two qubit systems", Phys. Scri. {\bf89} (2014) 125103.
\bibitem{RE2} Z. Z. Zou, X. T. Yu and Z. C. Zhang, ``Quantum connectivity optimization algorithms for entanglement source deployment in a quantum multi-hop network", Front. Phys. {\bf13} (2018) 130202.
\bibitem{MA4} S. T. Cheng, C. Y. Wang and M. H. Tao, ``Quantum communication for wireless wide-area networks", IEEE Journal on Selected Areas in Communications {\bf23} (2005) 1424.\\
M. Jiang, H. Li, Z. K. Zhang and J. Zeng, ``Faithful teleportation via multi-particle quantum states in a network with many agents", Quantum Inf. Process. {\bf11} (2012) 23.
\bibitem{RE3} N. Metwally, ``Entangled network and quantum communication", Phys. Lett. A {\bf375} (2011) 4268.
\bibitem{RE4} Z. Z. Li, G. Xu, X. B. Chen, X. Sun and Y. X. Yang, ``Multi-user quantum wireless network communication based on multi-qubit GHZ state", IEEE Commun. Lett. {\bf20} (2016) 2470.
\bibitem{RE5} Z. Z. Li, G. Xu, X. B. Chen, X. Sun and Y. X. Yang, ``Multi-user quantum wireless network communication based on multi-qubit GHZ state", IEEE Commun. Lett. {\bf20} (2016) 2470.\\
K. Wang, X. Yu and S. Lu, ``Quantum state propagation in quantum wireless multi-hop network based on EPR pairs", Antennas \& Propagation (ISAP), 2013 Proceedings of the International Symposium {\bf2} (2013) 1260 IEEE.

\bibitem{MA6} H. J. Briegel and R. Raussendorf, ``Persistent entanglement in arrays of interacting particles", Phys. Rev. Lett. {\bf86} (2001) 910. 
\bibitem{MA7} N. Kiesel, C. Schmid, U. Weber, G. Toth, O. G\"uhne, R. Ursin and H. Weinfurter, ``Experimental analysis of a four-qubit photon cluster state", Phys. Rev. Lett. {\bf95} (2005) 210502.
\bibitem{MA8} O. Mandel, M. Greiner, A. Widera, T. Rom, T. W. H\"ansch and I. Bloch ``Controlled collisions for multi-particle entanglement of optically trapped atoms", Science {\bf425} (2003) 937.
\bibitem{MA10} A. G. Adepoju, B. J. Falaye, G. H. Sun, O. Camacho-Nieto and S. H. Dong ``Joint remote state preparation (JRSP) of two-qubit equatorial state in quantum noisy channels", Phys. Lett. A {\bf381} (2017) 581.\\
B. J. Falaye, G. H. Sun, O. Camacho-Nieto and S. H. Dong, ``JRSP of three-particle state via three tripartite GHZ class in quantum noisy channels", Int. J. Quantum Inf. {\bf14} (2016) 1650034.
\bibitem{MA9} E. B. Davies and J. T. Lewis, ``An operational approach to quantum probability", Commun. Math. Phys. {\bf17} (1970) 239.\\
A. Peres, {\it Quantum Theory: Concepts and Methods} Vol. 57, Springer, 2006.
\bibitem{MA91} J. Wu, ``Symmetric and probabilistic quantum state sharing via positive operator-valued measure", Int. J. Theor. Phys. {\bf49} (2010) 324.
\bibitem{MA92} G. M. D'Ariano, P. Lo Presti and M. F. Sacchi, ``Bell measurements and observables", Phys. Lett. A {\bf272} (2000) 32. 
\bibitem{MA93} M. D. Gonz\'alez Ramirez, B. J. Falaye, G. H. Sun, M. Cruz-Irisson and S. H. Dong, ``Quantum teleportation and information splitting via four-qubit cluster state and a Bell state", Front. Phys. {\bf12} (2017) 120306.
\bibitem{MA94} F. L. Yan and H. W. Ding, ``Probabilistic teleportation of an unknown two-particle state with a four-particle pure entangled state and positive operator valued measure", Chin. Phys. Lett. {\bf23} (2006) 17.
\bibitem{MA11} A. G. Adepoju, B. J. Falaye, G. H. Sun, O. Camacho-Nieto and S. H. Dong, ``Teleportation with two-dimensional electron gas formed at the interface of a GaAs heterostructure", Laser Phys. {\bf27} (2017) 035201.\\
B. J. Falaye, A. G. Adepoju, A. S. Aliyu, M. M. Melchor, M. S. Liman, O. J. Oluwadare, M. D. Gonz\'alez-Ramirez and K. J. Oyewumi, ``Investigating quantum metrology in noisy channels", Scientific Reports {\bf7} (2017) 16622.
\bibitem{MA12} L. H. Shi, X. T. Tao, X. F. Cai, Y. X. Gong and Z. C. Zhang, ``Quantum information transmission in the quantum wireless multihop network based on Werner state", Chinese Physics B, {\bf24} (2014) 050308.
\end{thebibliography}
\end{document}
