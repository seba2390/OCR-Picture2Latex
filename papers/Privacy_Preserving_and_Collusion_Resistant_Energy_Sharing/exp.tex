We have evaluated the performance of our revised secure transformation protocol and secure reconstruction protocol using two different key length (512-bit and 1024-bit) and varying number of parties (from 20 to 500). The computational costs of two protocols are plotted in Figure \ref{fig:runtime}. To significantly improve the security/privacy (provable), the protocols take longer time compare to \cite{HongIJER15}, and such computational costs are still tolerable with an polynomial increasing trend as the number of parties increase. 


\begin{figure}[!tbh]
	\centering
	\includegraphics[angle=0, width=1\linewidth]{runtime}\vspace{-0.4in}
	\caption{Computational Costs} \label{fig:runtime}
\end{figure}

In addition, we present the communication overheads of the two protocols per party in Table \ref{tab:comm}. As the number of parties increase, the average bandwidth consumption (size of the transmitted messages) of the extended secure communication protocol and secure reconstruction protocol also grow polynomially.  Therefore, the two protocols can be implemented in most of the current networking environment. 

\begin{table}[!h]
	\small\caption{Communication Overheads} \centering
	\begin{tabular}{|c|c|c|}
		\hline
		Number of Parties& ExtSecTransform & SecReconstruction\\
		\hline
		20&0.00904 MB&0.0004 MB\\
		40&0.0761 MB&0.0019 MB\\
		60&0.261 MB&0.0045 MB\\
		80&0.624 MB&0.0078 MB\\
		100&1.23 MB&0.014 MB\\
		200&9.96 MB&0.051 MB\\
		300&33.5 MB&0.112 MB\\
		400&79.6 MB&0.119 MB\\
		500&155.6 MB&0.312 MB\\
		\hline
	\end{tabular}
	\label{tab:comm}
\end{table}
