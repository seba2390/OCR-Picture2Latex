
Energy has been increasingly generated or collected by different entities on the power grid (e.g., universities, hospitals and householdes) via solar panels, wind turbines or local generators in the past decade. With local energy, such electricity consumers can be considered as ``microgrids'' which can simulataneously generate and consume energy \cite{SaadHPB12,ArboleyaGCFMSBP15}. More recently, the research on cooperation among entities on the power grid (e.g., microgrids) has attracted great interests in both industry and academia \cite{SaadHPB12}. For instance, microgrids can share their local energy to improve the efficiency and resilience of power supply \cite{HongIJER15}. 

Specifically, microgrids can transmit their excessive energy to the microgrids close to them. In the cooperation, all the participating microgrids jointly seek an energy transmission assignment that minimizes the global energy loss during transmission. However, to this end, all the microgrids should disclose their local information (e.g., local supply, local demand, and power quality for transmission) to each other or a third party. Then, the data recipient (which is a microgrid or a third party) formulates an optimization problem by denoting the amount of energy transmitted from $M_i$ to $M_j$ as $x_{ij}$ and determining the objective function as well as the constraints. 


Disclosing such local information to each other or a third party would compromise the corresponding microgrid's local information. To tackle the privacy concerns, the proposed approach in \cite{HongIJER15} efficiently transforms the shares of the optimization problem to a privacy-complaint format and enables any party to solve the problem. However, the algorithms in \cite{HongIJER15} pursue high efficiency but cannot quantify the privacy leakage in the protocol. In this paper, we extend the transformation and optimal solution reconstruction to two secure communication protocols in which privacy leakage can be quantified and bounded. In the meanwhile, we give formal security/privacy analysis for the protocols and identify that our proposed secure communication protocols can prevent additional information leakage against the potential collusion among microgrids while executing the protocols. Finally, we present some experimental results to demonstrate the effectiveness and efficiency of our approach.