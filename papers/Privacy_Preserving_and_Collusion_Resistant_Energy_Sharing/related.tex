In smart grid infrastructure, privacy concerns were recently raised in the fine-grained smart meter readings, which is frequently reported to the utility \cite{SankarRMP13,HongTIFS17,SGSBook}.  To prevent information leakage in smart metering, three different categories of privacy preserving schemes were proposed in the past few years. The first category of techniques built cryptographic protocols to directly aggregate or analyze such meter readings without sharing the raw data. For instance, Rottondi et al. \cite{RottondiVC13} proposed a privacy preserving infrastructure based on cryptographic primitives to enable utilities and data consumers to collect and aggregate metering data. The second category of techniques obfuscate the meter readings to prevent adversaries from learning the status of the appliances at different times. For instance, Hong et al. \cite{HongTIFS17} defined a privacy notion to quantitatively bound the information leakage in smart meter readings, and proposed streaming algorithms for converting the readings with guaranteed output utility. Finally, the third category of techniques utilize renewable energy sources like batteries to hide the actual load of different households, which can be found in \cite{McLaughlinMA11}, \cite{YangLQQMM12}, etc.

Furthermore, energy sharing problem among microgrids \cite{SaadHPB12,ZhuHSSIMMS13} has been recently studied -- locally generated energy can be shared among homes due to the mismatch between generation harvesting and consumption time in microgrids. Zhu et al. \cite{ZhuHSSIMMS13} developed an energy sharing approach to determine which homes should share energy, and when to minimize system-wide efficiency loss. Zhu et al.  \cite{ZhuXPTG11} also proposed a secure energy routing approach to renewable energy sharing against security attacks such as spoofed routing signaling and fabricated routing messages. Also, some game theoretical models \cite{SaadHPB12,MaityR10,DuanD09} were proposed to mitigate the risks of self-interested behaviors in the energy sharing/exchange. So far, Hong et al. \cite{HongIJER15} is the only work that resolves the privacy issues in energy sharing/exchange. The proposed scheme can provide some ad-hoc privacy guarantee based on matrix multiplication. Instead, we extend the approach in \cite{HongIJER15} to ensure provable security.
