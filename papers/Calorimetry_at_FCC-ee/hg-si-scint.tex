The idea to apply the particle-flow approach in a future $e^+e^-$ collider detector, for the precision study of heavy particles predominantly decaying into jets, has driven the development of highly granular calorimeters from the beginning~\cite{Brient:2002gh,Morgunov:2001cd}. 
The PFlow method optimises the jet energy resolution by individually reconstructing each particle and using the best measurement for each, technique which poses high demands on the imaging capabilities of calorimeters.
Charged particles are best measured with tracking detectors and photon energies can be measured with a relative precision of about $15\%/\sqrt{E(\rm{GeV})}$, or better, in electromagnetic calorimeters. 
In a typical jet, about 60\% of the energy is carried by charged particles, 30\% by photons and only 10\% by long-lived neutral hadrons ($K^0_L$ and neutrons), for which detection hadronic calorimetry is imperative. 
Assuming a hadronic energy resolution of $55\%/\sqrt{E(\rm{GeV})}$, then, if ideally each particle is resolved, a jet energy resolution of $19\%/\sqrt{E(\rm{GeV})}$ could be obtained where the dominant part ($17\%/\sqrt{E(\rm{GeV})}$) is still due to the calorimeter resolution for neutral hadrons.

In practice, mis-assignments give rise to additional measurement uncertainties, called {\it confusion}\index{confusion} term. 
For example, a neutral particle shower could be misinterpreted as part of a nearby charged hadron shower, and the neutral energy would be lost, or a detached fragment of a charged particle shower could be misidentified as a separate neutral hadron, and the fragment energy would be double counted. 

Particle-flow calorimeters, with their emphasis on imaging, must still feature a good hadron energy resolution.  The neutral-hadron energy uncertainty is the dominant contribution to the jet resolution for low energy jets, where particles are well separated. At higher energies, the confusion effects take over, and a good calorimetric resolution improves the energy-momentum match used in the assignment of energy depositions. For the typical particle-flow driven $e^+e^-$ detectors, the transition is at jet energies around 100~GeV.

The principle has been experimentally tested~\cite{Sefkow:2015hna} by the CALICE collaboration with test beam prototypes using different absorber materials and readout techniques. 
The most commonly proposed technologies are silicon diodes for the electromagnetic section and scintillating tiles, individually read out by silicon photomultipliers (SiPMs), for the hadronic part. 
Scintillator ECAL and gaseous HCAL technologies are being explored, too.

In all cases, the high channel density requires the integration of the front-end electronics into the active layers, such that the digitised and zero-suppressed data can be extracted from the volume via a small number of readout lines.
The initial focus on applications at linear colliders, with their low duty cycle, has helped to keep the associated requirements for data transfer, power and cooling, manageable. 
These need to be revised for an implementation at FCC-ee, with possible implications on the overall calorimeter architectures and integration concepts. 



\subsection{State of the Art}
A first systematic optimisation of the 3D detector segmentation parameters, using detailed simulations and the PANDORA reconstruction algorithm, has been done in~\cite{Thomson:2009rp}, for jet energies up to 250~GeV, and later confirmed~\cite{Tran:2017tgr}, for the HCAL with software compensation taken into account.  The proposed HCAL cell sizes are of the order of a radiation length, the characteristic scale of shower sub-structure, about 3~cm in a steel-plastic structure. In concepts with 1- or 2-bit ((semi-) digital) readout, 1~cm cells are needed. 
ECAL cell sizes of typically 0.5~cm, well below 1~$X_0$, were shown to provide superior separation power for nearby electromagnetic showers in their early evolution stage, before they attain their full width. 

The first generation prototypes built by the CALICE collaboration did not yet have fully integrated front-end electronics, but were successfully used to validate the  particle separation power~\cite{Adloff:2011ha} and the single-particle energy resolutions expected from simulations. 
For the ECAL, 
a stochastic term of 
$16.5\%/\sqrt{E({\rm GeV})}$ was measured in the range 6 - 45~GeV with a constant term of 1.1\%~\cite{Anduze:2008hq}.
%\begin{equation*}
%\label{eq:datares}
% \sigma (E_{\mathrm{meas}})/E_{\mathrm{meas}}   = 
% (16.53\pm0.14(\mathrm{stat})\pm0.4%(\mathrm{syst}) )\%/\sqrt{E(\mathrm{GeV})}  
%\oplus   \left(1.07\pm0.07(\mathrm{stat})\pm0.1(\mathrm{syst})\right)\%
%\end{equation*}
For the HCAL, $44.2\%/\sqrt{E({\rm GeV})}$ and 1.8\% were measured in the range 10 - 80~GeV~\cite{Adloff:2012gv}, using a cell-energy based weighting procedure.

For the second generation, ultra-compact-integration solutions have been developed, which minimise the active gap widths, thus the effective Moli\`ere radius characterising the  transverse electromagnetic shower extension in the ECAL. These compact solutions - assuming a fixed hadronic interaction depth of the HCAL - also lead to a smaller calorimeter outer radius, which drives the cost of solenoid coil and return yoke in the barrel section of the detector.
The CALICE solution, adopted by ILD~\cite{ILD:2020qve} and depicted in Fig.~\ref{fig:hg-ecal-ahcal}, foresees carbon-fibre based alveolar structures with pairs of active elements attached back-to-back to an I-beam shaped tungsten plate of 1.9~mm thickness.
The silicon sensors with square pads are connected to the readout PCBs with conductive glue. 
SiD has developed a more aggressive integration concept~\cite{Barkeloo:2019zow} with ASICs directly bonded to the silicon sensors and connected to hexagonal pads via traces on the silicon.
Prototypes with up to $\sim 10$ layers have been tested for both concepts.
\begin{figure}
    \centering
%    \sidecaption
%\includegraphics[width=0.5\textwidth]{./slab_test.jpg}  
\resizebox{0.54\textwidth}{!}{\includegraphics{./slab_test.jpg}} 
\resizebox{0.45\textwidth}{!}{\includegraphics{./CALICE_AHCAL.jpg}}
    \caption{Left: Active layer stack-up of the silicon-tungsten ECAL in the ILD design. The Figure shows the supporting structure including a tungsten absorber element, sensors, embedded front-end electronics as well as an external interface card \cite{Brient:2018ydi}.
%    Reproduced with permission from J.-C.\ Brient,  dd.mm.2021.
    Right: Highly granular scintillator-tile / steel hadronic calorimeter technological prototype of the CALICE collaboration, showing %one active layer with the scintillator tiles mounted on circuit boards housing the very front-end electronics and the photon sensors (left), and 
    the absorber structure with the readout interfaces for the active elements.
    }
    \label{fig:hg-ecal-ahcal}
\end{figure}

The CALICE SiPM-on-tile HCAL is a self-supporting stainless steel structure with 19~mm thick absorber plates and minimal un-instrumented zones, interleaved with active elements inserted as cassettes, which add another mm to the absorber thickness. The cassettes contain readout units, PCBs that hold the SiPMs, readout ASICs and LEDs for calibration, and onto which injection-moulded tiles wrapped in ESR foil are glued. 
%see Fig.~\ref{fig:hg-ecal-hcal}.
The width of the active gap is 6.5~mm, including the 3~mm thick scintillator. 
A prototype~\cite{Sefkow:2018rhp} (Fig.~\ref{fig:hg-ecal-ahcal}) with 38~layers and 21888~SiPMs has been tested in 2018 at the CERN SPS. 

The highly granular silicon-tungsten and scintillator-steel technologies are currently being applied in the endcap calorimeter upgrade of CMS~\cite{collaboration:2017gbu} for the high-luminosity phase of the LHC. 
This brings additional challenges in terms of radiation tolerance, cooling and data rates.
The channel counts and instrumented areas are at an intermediate level with respect to a full $e^+e^-$ collider detector - e.g.\ 600~m$^2$ of silicon sensors, and 240000 SiPMs - and will represent an important step in scaling production, quality control and calibration techniques, upon which an FCC-ee detector could build. 
However, the integration solutions cannot be transferred. 
The high readout bandwidth and the spatial constraints of an existing detector require to integrate into the active gaps, not only the front-end electronics, but also the first level data concentrators and the power converters, which leads to a less compact structure than that conceived for a barrel calorimeter at FCC-ee. 

\subsection{Conceptual Implementation: CLD}
While the CALICE prototypes have mainly followed the ILD detector concept~\cite{ILD:2020qve} at the ILC, the technologies have also been  adopted for the CLIC detector~\cite{AlipourTehrani:2017gov}. 
Based on the CLICdet design, the CLIC-like detector model, CLD~\cite{Bacchetta:2019fmz}, has been adapted to match the experimental conditions and physics requirements at FCC-ee.
The CLD ECAL has 40 identical silicon-tungsten layers and a total depth of 23~$X_0$. 
The silicon area of about 4000~m$^2$ is segmented into 160M cells.
The HCAL has 44 scintillator-steel layers corresponding to 5.5~$\lambda_I$.
The tiles cover an area of about 8000~m$^2$ and are read out by 9M SiPMs. 

The energy resolution for single photons has a stochastic term of $15\%/\sqrt{E({\rm GeV})}$ in the range 5 - 100~GeV. The particle-flow jet energy resolution is 4.5\% at 50 GeV and below 4\% for energies of 100~GeV and higher. 
This results in a W-Z separation power of $2.5\,\sigma$ for boson energies of 125~GeV.
A simulated di-jet event is shown in Fig.~\ref{fig:CLDjets}. 
\begin{figure}
    \centering
%    \sidecaption
%\includegraphics[width=0.5\textwidth]{figures/slab_test.jpg}  
\resizebox{0.5\textwidth}{!}{\includegraphics{./FCCee3.png}} 
    \caption{Event display in the CLD detector for a 
    $\mathrm{Z}/\gamma^*\rightarrow q\bar{q}$
    %\PZgstarToqq 
    event, with $m_{\PZ}$ = 365 GeV.
    From~\cite{Bacchetta:2019fmz}.
    }
    \label{fig:CLDjets}
\end{figure}

\subsection{Challenges and Future Developments}
In view of the challenges imposed by the tremendous statistical power of the FCC-ee on the control of systematic effects, in particular when running at the Z pole, it is mandatory to continue driving the refinement of shower simulation models and their validation using highly granular prototype data. 
The CALICE collaboration plans to continue their test beam program. 
With a fully commissioned HCAL readout, an intrinsic time resolution below 1~ns becomes possible. This will enable unprecedented studies of the shower evolution in space and time and to  explore the use of timing information for the reconstruction of topology and energy. 
A tungsten absorber structure is available to be used instead of steel and to validate simulations of hadron interactions in the preferred ECAL absorber material with fully contained showers. 
Once an ECAL prototype is fully instrumented, the combined performance of the ECAL-HCAL system will also be studied. 

The active detector elements currently at hand, like silicon sensors and SiPMs, meet the requirements; nevertheless developments in industry need to be followed. 
For example, SiPMs with smaller pixels and larger dynamic range become possible, but their reduced gain will also require more sensitive front-end electronics. 
In view of the channel counts which are more than an order of magnitude higher than in the upgraded CMS endcap calorimeter, more work on the scalability of production techniques will be needed, and different concepts will have to be followed, e.g.\ the so-called mega-tile arrays for the scintillator.   

The main challenge will be the development of an electronics system for continuous and dead-time free readout, and to address the implications for system integration. 
New front-end ASICs for energy and time measurements are needed in a common architecture for ECAL and HCAL. 
Data concentrators need to sustain much larger throughputs than the existing systems and still be highly compact, to minimise  dead zones and their impact on systematics. 
Both front-end and concentrator electronics will need active cooling, which introduces additional requirements for space for services. 
Realistic solutions have to be designed for active layers, interfaces and cooling, and should be prototyped. 

The development of integration solutions has to go hand in hand with detailed simulation studies, for example to re-optimise the absorber structure with the inclusion of copper cooling plates à la CMS. 
The CLD concept is still undetermined in some basic questions regarding the overall architecture, the segmentation of the detector into modules and the detailed design of the barrel-endcap transition.  
Signals from the embedded front-end in the barrel can be routed either along axial paths (parallel to the beam line, like in the ILD scintillator HCAL) towards interfaces at the end of the barrel, or in a tangential direction (like in the ILD ECAL) to interfaces in the gap between ECAL and HCAL.
Studies of different detector configurations in simulations with realistic assumptions on interfaces and services, validated by prototypes, will have to provide input to such key decisions. 

Finally, new ideas may be followed, for example the addition of spectral or ps timing information for the application of dual-readout methods, thus combining several of the approaches presented here. 





