To optimally benefit from the large datasets that will be available at FCC-ee and fully exploit its physics program, the intrinsic measurement resolutions and event-to-event information have to be substantially increased. The 20-year-long $R\&D$ program on Dual-Readout Calorimetry (DR, DRC) of the DREAM/RD52 collaboration~\cite{DRC_Wigmans,RD52_emshow,RD52_emperf,RD52_hadperf,RD52_MC,RD52_PID,RD52_SiPM,IDEA_SiPM} shows that, with a detector fully calibrated at the EM scale, the independent readout of scintillation (S) and \v{C}erenkov (C) light allows the cancellation of the effects of the fluctuations in the EM fraction of hadronic showers. The DR fibre-sampling approach brings the stochastic term down close to or even below $30\%/\sqrt{E}$ through a high sampling frequency and the integration of the shower over its longitudinal development. The latter, in addition, leads to a less-noisy information.
With an ideal detector (with an energy resolution of $\sim 30\%/\sqrt{E}$), the expected separation reachable for the $W/Z/H \rightarrow jj$ peaks is shown in Fig.~\ref{fig:WZH-jj}.
Stand-alone results show as well excellent particle-ID performance and competitive EM energy resolution.

\begin{figure}
    \centering
    %\resizebox{0.5\textwidth}{!}{\includegraphics{figures/WZH-jj.pdf}} 
    \resizebox{0.7\textwidth}{!}
        {\includegraphics{./WZH-jj.pdf}} 
    \caption{Reconstructed invariant mass distributions for  
    $W/Z/H \rightarrow jj$ events in a dual-readout fibre calorimeter.
    }
    \label{fig:WZH-jj}
\end{figure}

The advancements in solid-state light sensors such as SiPMs have opened the way for highly granular fibre-sampling detectors with the capability to resolve the shower angular position at the mrad level or even better.
In the present design, 1-mm diameter fibres are placed, at a distance (apex to apex) of 1.5-2 mm, in a brass absorber matrix (copper, iron and lead being the alternative materials under consideration). This means that the lateral segmentation could be pushed down to the mm level, largely enhancing the resolving power for close-by showers, with a significant impact, for example, on channels like $\tau \rightarrow \rho \nu$.

The high PDE of SiPMs should permit to obtain light yields of O(100) p.e./GeV for both the S and C signals, which can guarantee an EM resolution close to $10\%/\sqrt{E}$. In the above geometry, the lower limit, as set by the sampling fluctuations, is around $(8-9)\%/\sqrt{E}$.
Readout ASICs providing time information with $\sim$~100 ps resolution may allow the reconstruction of the shower position with $\sim$~5 cm of longitudinal resolution.

On the other hand, the large number and density of channels call for an innovative readout architecture for efficient information extraction. Both charge-integrator and waveform-sampling ASICs are available on the market and candidates for the first tests have been identified (the Weeroc Citiroc 1A charge integrator and Nalu Scientific system-on-chip digitisers). At the time of writing, a first implementation of a scalable readout system is almost ready for testing in a time scale of few months.
Looking further ahead, digital SiPMs (dSiPMs) should allow significant simplification of the readout architecture but the technology does not yet appear mature enough. A specific $R\&D$ program has been submitted for approval.

The mechanical assembly and integration of a system with $O(10^8)$ sensitive elements require the development of a robust and engineered procedure.
A scalable mechanical solution, that should work for both non-projective and projective modules, has been defined. Based on the gluing of capillary tubes, it is being exploited for building a small ($\sim 10 \times 10 \times 100\ cm^3$) EM prototype to be tested with beam (yet within few months). The preliminary results are very positive and this approach is presently the basis of a project for the construction of a hadronic prototype (of size $\sim 60 \times 60 \times 200\ cm^3$) in 3-4 years, depending on funding approval. Alternative approaches, in particular with 3D printing, are being investigated within a South Korean $R\&D$ project.

The performance, in the reconstruction of the properties of both hadronic and EM showers, is good enough to open the possibility to exploit a single integrated dual-readout fibre-sampling solution, for the calorimetric system of an FCC-ee experiment. This is the baseline choice in the IDEA~\cite{IDEA_tb1} detector concept. 

The huge amount of information made available by the fibre SiPM readout should be likely take advantage of deep-learning algorithms, in order to be maximally exploited. The preliminary performance in the identification of $\tau$-decay final states, using calorimetric information only, looks very promising. With reduced (full) fibre information the average classification accuracy was estimated to be $\sim 90\%$ ($> 99\%$).