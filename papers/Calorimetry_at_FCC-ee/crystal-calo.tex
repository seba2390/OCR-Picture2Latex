Crystal calorimeters have a long history of pushing the frontier on high-resolution electromagnetic
(EM) calorimetry for photons and electrons.
More recently, with the advent of the SiPM photodetection technology, segmented crystal calorimeters incorporate and achieve new performance benchmarks for precision timing, particle identification and $e/h$ response compensation through dual readout.  These extended capabilities are central to the FCC-ee physics program where a high level of precision is required to comprehensively measure and identify all particles forming the event from a wide range of processes, from rare Higgs decays, heavy flavour and $\tau$-lepton decay chains, to low systematic electroweak measurements.  Segmented crystal calorimeters are only one part of a complete measurement system but studies, in concert with a low-mass tracking system and a dual-readout fibre hadron calorimeter, highlight which parameters are most important for the combined detector performance~\cite{Lucchini_2020}. 


Inorganic crystals have intrinsically low response to neutrons and low-mass nuclear fragments due to inefficient momentum transfer to the heavy materials that form the crystal lattice.  However, the bias towards low response is well measured by the high EM response in the form of both scintillation and \v{C}erenkov light, providing an accurate dual-readout compensation.
Considering first the neutral components of the $ZH$ events, those not captured by the tracking system, adding dual readout to the rear compartment of a segmented crystal provides an excellent neutral hadron resolution of 30\%/$\sqrt{E} \oplus 2$\% when combined with the dual-readout fibre calorimetry, while achieving 3\%/$\sqrt{E}$ for low energy photons~\cite{Lucchini_2020}.  This combination balances the need to bring down the leading contribution to the overall jet energy resolution, originating primarily from $K_L$'s with an average energy of approximately 5.5~GeV, while at the same time maximising the EM resolution.
The EM resolution for low energy photons is a sub-dominant component of the total jet energy resolution but the ability to resolve and correctly pair photons into $\pi^0$'s, the underlying hadron momenta, resolves an important particle assignment ambiguity when forming jets in 4- and 6-jet $ZH$ events.  Similarly, the effects of dead material from the finite inner radius of the solenoid can be mitigated by placing the solenoid between the crystal calorimeter and the dual-readout fiber calorimeter, where low energy EM particles are directly measured with crystals with a minimum of dead material losses.


The inner tracking system measures the charged hadron momenta with a higher resolution than can be achieved with calorimetry but, to fully benefit from this increased resolution, particle-flow algorithms rely on particle identification and low-ambiguity cluster assignment from the calorimeter.  A segmented crystal calorimeter with precision timing layers can tag arrival times for MIPs to better than 20~ps.  Precision timing fills the well-known gap in $dE/dx$ particle identification from the minimum region of the Bethe-Bloch energy-loss function.  Charged hadrons are further characterised by the penetration depth before showering begins, determined from the delayed energy profiles in longitudinally segmented crystal readout, and the relative amount of \v{C}erenkov (C) to scintillation (S) light produced in the shower.  These ratios have been shown to be powerful tools to resolve particle identification within a segmented crystal calorimeter, acting as a linchpin between tracks and dual-readout fibre calorimeter clusters.  With crystal measurements alone, when using a front/rear ratio, transverse profile and C/S, a rejection of 99.4\% on charged pions, for a 99\% efficiency to select 10~GeV electrons, has been estimated~\cite{Lucchini_2020}.  An example layout of a segmented crystal calorimeter integrated into an FCC-ee experiment is shown in Fig.~\ref{fig:calo-crystals}.

\begin{figure}[ht]
\centering
\resizebox{0.5\textwidth}{!}{\includegraphics{./quarterview3-flat.png}}
\resizebox{0.27\textwidth}{!}{\includegraphics{./SCEPCal_solid_10GeV_pi-_quarter_side_MC.png}}
\caption{A segmented crystal calorimeter integrated into an FCC-ee experiment.  ({\it left}) The precision timing layers (green) are followed by projective crystals longitudinally segmented into front (blue) and rear (grey) compartments.  The rear crystals are instrumented with dual readout and are surrounded by a solenoid (red) in the barrel region and hermetically by a dual-readout fibre calorimeter (yellow).  ({\it right}) A 10~GeV pion shower through the crystal calorimeter option of the IDEA detector~\cite{ALY2020162088}.}\label{fig:calo-crystals}
\end{figure}

The material budget of the inner tracker has a profound impact on the overall performance of crystal calorimeters.  Studies on the material thickness in radiation lengths show that the electrons from $Z\rightarrow e^+e^-$ decays radiate a substantial fraction of their momenta into bremsstrahlung photons.  The $Z\rightarrow e^+e^-$ mass recoil is a crucial quantity for $HZ$ associated Higgs production studies.  Compared to $Z\rightarrow \mu^+\mu^-$ decays measured with a tracking momentum resolution of 0.3\%, the resolution on the $Z\rightarrow e^+e^-$ mass recoil is approximately 3 times worse for an EM resolution of 15\%/$\sqrt{E}$ for a tracker thickness of 0.4~$X_0$.
With segmented crystal calorimetry, the recoil mass resolution of $Z\rightarrow e^+e^-$ is within 25\% of the muons for tracker thicknesses up to 0.4~$X_0$~\cite{Lucchini_2020}. For a substantially thicker tracker, the bremsstrahlung recovery does not show significant improvement on the electron momentum.


Crystal R\&D continues to be a major component of new advances in crystal calorimetry.  Bright, dense crystals, such as LYSO, and with ultra-fast rise time, with doped-BaF$_2$, are being produced in large-scale for high-precision timing detectors~\cite{Kratochwil_2020,OMELKOV2018260}.  Cutting and growth methods are providing new possibilities for segmentation at low cost.  Fast photon production processes in crystals, like \v{C}erenkov photons and intra-band luminescence, are promising avenues of exploration for timing application.  New techniques, such as 3D printing, laser growth and nano-engineered materials, are creating new types of crystal scintillator structures~\cite{TURTOS2019116613}.  The customization of SiPM parameters to match crystal parameters is an area of rapid development with many new directions to increase photodetection efficiencies, wavelength coverage, timing uniformity, dynamic range, fast recovery time and low noise performance.


Beyond these important benchmarks, the full power of segmented crystal geometries is still unknown.  A so-called ``5D Crystal'' geometry leverages multiple stacks of segmented crystal timing layers to form the entire crystal calorimeter, following the single-layer geometry of the CMS barrel timing layer~\cite{CERN-LHCC-2017-027}.  The individual positions of shower energy deposits are resolved using pattern recognition and timing.  The resulting spatial shower information is comparable to a 16-layer silicon-tungsten sampling calorimeter, without the loss in sampling fraction from the interleaved absorber material.  A key component of future crystal calorimeter developments is an optimisation procedure for particle-flow algorithms that attempts to use coarse or even non-uniform longitudinal segmentation while still retaining more information per cell due to the high, near unity, sampling fraction of homogeneous crystals.  With the unprecedented statistical power, diversity and breadth of the FCC-ee physics program, the challenges in crystal calorimeter design are even tougher, with the need to evaluate detector parameters and their potential impact on the ultimate limits on never-before-achieved measurement precisions and new physics search sensitivities.