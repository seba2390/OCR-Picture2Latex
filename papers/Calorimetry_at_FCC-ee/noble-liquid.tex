%As outlined in Sec.~\ref{sec:intro}, the FCC-ee has an ambitious physics program at various center-of-mass energies~\cite{Mangano:2651294} which can be translated into stringent detector requirements for the FCC-ee experiments. The role of calorimetry will be to complement the tracking system in an optimal particle-flow event reconstruction. Furthermore, several B-physics measurements with $\pi^0$'s in the final states (e.g. $\mathrm{B_s}\rightarrow\mathrm{D_sK}$ in modes with $\pi^0$'s) will profit from excellent photon resolution of $\sigma_E/E\approx 5\,\%/\sqrt{E}$ and measurements with $\tau$ decays will benefit from an energy measurement of photons down to well below $1\,\mathrm{GeV}$. On top of that the requirements of particle identification will call for high transverse granularity. 

Noble-liquid calorimetry was successfully used in many high-energy experiments (e.g. E706 at FNAL, R806 at ISR, D0~\cite{D0:1993bqa}, H1~\cite{ANDRIEU1993460}, NA48~\cite{Unal:2000eu}, ATLAS~\cite{CERN-LHCC-96-041}, SLD~\cite{Benvenuti:1989ht}) due to its excellent energy resolution, linearity, stability, uniformity and radiation hardness. While radiation hardness is not a concern for lepton colliders, all other properties are clearly essential for high precision measurements, e.g. at the Z-pole, but also for the planned Higgs measurement program. 
As an example, at the Z pole, typically $10^{11}$ $\mathrm{Z}\rightarrow\mu^+\mu^-$ or Z$\rightarrow\tau^+\tau^-$ decays and $2\times 10^{12}$ hadronic Z decays will enable measurements with a statistical uncertainty up to 300 times smaller than at LEP, from a few per mil to $10^{-5}$. Such unprecedented statistical precision will have to be complemented by an extremely well controlled systematic error which requires an excellent understanding of the detector and the event reconstruction.  
A highly uniform, linear and stable measurement in the calorimeters will be a prerequisit to achieve this ambitious goal.

\subsection{Layout Optimisation for FCC-ee}
Recently, highly granular noble-liquid sampling calorimetry was proposed for a possible FCC-hh experiment~(\cite{Benedikt:2651300},~\cite{aleksa2019calorimeters} and~\cite{Aleksa:2020qdy}). It has been shown that - on top of its intrinsic excellent electromagnetic energy resolution - noble-liquid calorimetry can be optimised in terms of granularity to allow for 4D imaging, machine learning or - in combination with the tracker measurements - particle-flow reconstruction.  

Studies have started to adapt noble-liquid sampling calorimetry for an electromagnetic calorimeter of an FCC-ee experiment. Such an electromagnetic calorimeter could then be complemented by a CLD-style hadron calorimeter made of steel absorber plates, interleaved with scintillating tiles read out by SiPMs. The solenoid coil could either be located outside the hadron calorimeter, at a radius $r$ of $\approx 3.9\,\mathrm{m}$, or - provided the coil can be made thin enough - inside the noble-liquid calorimeter, at a radius $r$ of $\approx 2.1\,\mathrm{m}$, and possibly housed inside the same cryostat as the electromagnetic calorimeter. R\&D on thin carbon-fibre cryostats and thin solenoid coils as well as R\&D on high density signal feedthroughs has started in the framework of the CERN EP R\&D program~\cite{Aleksa:2649646}. 



\begin{figure}
\centering
\resizebox{0.5\textwidth}{!}{\includegraphics{./inclinedCalo_cdr}}
\caption{A noble-liquid sampling calorimeter for an FCC-ee experiment.}\label{fig:calo-fccee}
\end{figure}

Figure~\ref{fig:calo-fccee} shows a possible noble-liquid calorimeter adapted to the central region of an FCC-ee experiment, a cylindrical stack of absorbers, readout electrodes and active gaps with an inner radius of 2.1\,m, compatible with an IDEA-style tracking system (silicon vertex detector and drift chamber, see~\cite{Benedikt:2651299}). Such a configuration using liquid argon (LAr) as active material, with 1536 lead/steel absorbers of 2\,mm total thickness ($100\,\mu\mathrm{m}$ steel sheets glued onto each side of the lead absorbers), 1.2\,mm thick readout electrodes and a total depth of 40\,cm, will lead to an effective total thickness of $\sim 22$~radiation lengths, $X_0$, and a Moli\`ere radius of $R_M\approx 4\,\mathrm{cm}$. Tungsten absorbers or liquid krypton (LKr) as active material are interesting options due to the resulting smaller radiation length and smaller Moli\`ere radius which will lead to smaller showers and hence better separation of close-by particles with potentially positive impact on particle identification and particle-flow reconstruction. Studies have started to identify the best solution for an FCC-ee calorimeter. The stack of absorbers, active gaps and readout electrodes will be housed in a cryostat to reach cryogenic working temperatures. 
The electrodes, as well as the absorber plates, are arranged radially but azimuthally inclined by $\sim 50^{\circ}$ with respect to the radial direction, as shown in Fig.\,\ref{fig:calo-fccee}. This ensures that services on the inner and outer radius can pick-up the signals without creating any gaps in acceptance and allows for a high sampling frequency. The inclination of the plates has been chosen to ensure a uniform response in $\varphi$ for the full energy range despite the bending of particle tracks in the 2\,T magnetic field. 
Together with spacers defining the exact width of the active gaps, high mechanical precision and hence minimal impact on the energy resolution and uniformity can be achieved with this relatively simple structure. 
The granularity of each longitudinal compartment can be optimised according to the needs of particle-flow reconstruction and particle ID. Currently, a granularity of $\Delta\theta\times\Delta\varphi = 2.5\,\mathrm{mrad}\times 8.2\,\mathrm{mrad}$ ($5.4\,\mathrm{mm}\times 17.7\,\mathrm{mm}$) is foreseen in the first calorimeter compartment to optimise the $\pi^0$ rejection. 

In order to achieve high lateral granularity (in $\theta$), the signals will need to be brought to the edges of the electrodes via strip lines, the readout electrodes will therefore need to consist of several layers. A solution has been worked out using seven-layer printed circuit boards (PCBs), with the following layer attributions (from outside in): 
\begin{itemize}
\item The two outermost layers will be HV layers connected to HV power supplies outside the cryostat via high resistances. Together with the grounded steel surfaces of the absorbers, they will provide the electric drift field of $E_\mathrm{drift}\approx 1\,\mathrm{kV/mm}$ in the active gaps. 
\item Two layers of signal pads, $\sim 100\,\mu\mathrm{m}$ below the HV layers, will pick up the signals of the charges drifting in the drift gaps. The necessary granularity in $\theta$ will define the sizes of these pads. 
\item Two layers of ground shields will shield the signal pads from the central signal traces to avoid cross talk. 
\item The signal traces in the central layer (connected with vias to the signal pads) will bring the signals to the outer edges of the electrodes. These signal traces will form transmission lines (strip lines) together with the ground shields with an impedance matched to the input impedance of the preamplifiers. 
\end{itemize}
Such multi-layer readout electrodes will lead to increased cell capacitances potentially leading to higher series noise. For the electrodes proposed here, 2 to 10\,MeV per readout cell (at the EM scale) were estimated depending on many parameters, such as the cell size, the exact electrode design, and the electronics time constants. These low noise values obtained assuming preamplification outside the cryostats can be further improved by using cold preamplifiers inside the noble-liquid bath. Both options are studied at the moment. It is foreseen that one calorimeter cell would extend in $\phi$ over one-to-four electrodes, depending on the longitudinal compartment, which will be achieved by summing the signals on the edges of the electrodes. The granularity per longitudinal compartment will be defined by physics requirements such as $\pi^0$ identification and particle-flow reconstruction. It should be noted that the expected MIP energy deposit per double gap is $\sim 1.4\,\mathrm{MeV}$ which - correcting for a sampling fraction of $\sim 1/6$ - corresponds to a cell deposited energy of $\sim 8\,\mathrm{MeV}$. Similar to Si-based calorimeters, it will therefore be possible to track single particles in the calorimeter even before the shower starts. 

In the design described above, the active gaps between two absorbers are radially increasing from $2\times1.2\,\mathrm{mm}$, at the inner radius, to $2\times2.4\,\mathrm{mm}$, at the outer radius, leading to a sampling fraction changing with depth. Due to the longitudinally segmented readout, the shower profile will be measured for each particle, and an energy calibration, based on simulations and complemented by an $E/p$ cross calibration with the tracker, will correct for the radially dependent sampling fraction. 
%It has been shown~\cite{aleksa2019calorimeters}, that with a minimum of eight longitudinal compartments, the effect of the changing sampling fraction on the energy resolution is negligible. For FCC-ee at up to 12 longitudinal compartments are envisaged to ease particle flow reconstruction and particle identification. 
The first longitudinal compartment is realised without any absorber and will serve as a pre-sampler to be able to correct for energy lost upstream. This is especially important for low-energetic particles which loose a large fraction of their energy in the dead material in front of the calorimeter. It has been shown in~\cite{aleksa2019calorimeters} that such a correction is important for a linear energy response and improves the resolution for particles below 20\,GeV by more than 30\,\%.

The design described above will be optimised in the coming months and years to adapt it to the performance requirements by FCC-ee. Work on the readout PCBs has also started with the goal to minimise noise while achieving the necessary granularity for particle-flow reconstruction and particle ID (e.g. $\pi^0$ rejection). It is further planned to produce such readout electrodes to validate the concept in a small test calorimeter. This work as well as R\&D on high-density signal feedthroughs, thin carbon-fibre cryostats and thin solenoid coils is part of the CERN EP R\&D program~\cite{Aleksa:2649646}. 

\subsection{Performance}
The performance of such a calorimeter has been evaluated for an FCC-hh experiment~\cite{aleksa2019calorimeters,Aleksa:2020qdy}. The below quoted photon, electron and pion stand-alone performance is therefore meant to demonstrate that such a calorimeter - if further optimised - has big potential to achieve all above listed performance requirements. It should be noted that eventually particle-flow event reconstruction will be used. 
%It consists in a combination of the tracker measurement of all charged particles and the calorimetry measurement, relying on high calorimeter granularity and excellent position resolution.
At this moment, particle-flow event reconstruction is being implemented into the FCC software, but is unfortunately not yet available for performance studies.

Single-particle simulations of electrons and photons have resulted in a stochastic term of 8.2\,\%~\cite{aleksa2019calorimeters,Aleksa:2020qdy} for the standalone electromagnetic energy resolution. This value can be further improved by increasing the sampling fraction or the sampling frequency. It has been shown that the noise contribution, which dominates energy resolution at low particle energies, can be kept below 50\,MeV by optimising the cluster size. For charged particles, the reconstruction will rely on a particle-flow combination of the tracker and the calorimetry measurement, relying on an excellent position resolution. An ECAL position resolution of $<500\,\mu\mathrm{m}$ was simulated for energy deposits $>30$\,GeV. Fine lateral segmentation is also essential for the required $\pi^0$ rejection\footnote{The $\pi^0$ rejection factor $R_{\pi^0}$ is defined as the fraction of the total number of $\pi^0$ divided by the number of non-rejected $\pi^0$.}, which was shown to be $R_{\pi^0}>5$ for transverse momenta up to $30$\,GeV in a deep-neural-network based analysis (assuming 90\,\% signal efficiency)~\cite{aleksa2019calorimeters}.

The single-$\pi^-$ resolution has been obtained from simulation using a simple hadron calibration, the so-called benchmark method~\cite{aleksa2019calorimeters}. It consists in adding the simulated energy deposits in a window of defined size, taking into account the different hadronic response of ECAL and HCAL, and correcting the obtained energy for the energy lost in the dead material in between the calorimeters. In the FCC-hh simulation, a stochastic term of 48\,\% (44\,\%) for single $\pi^-$ was achieved in this way for $B=4$\,T ($B=0$\,T), respectively. Exploiting the full 4D imaging information, it was demonstrated that a deep-neural-network analysis can achieve a single-$\pi^-$ resolution stochastic term of 37\,\% ($B=4$\,T). It should be noted that this remarkable result relies on the calorimeter measurement alone, the planned particle-flow reconstruction combined with the tracker will further substantially improve the hadronic resolution.
