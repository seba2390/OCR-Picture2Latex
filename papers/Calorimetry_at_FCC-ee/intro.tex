The Future Circular Collider (FCC) is an ambitious project of an accelerator complex in the CERN area for the era after LHC~\cite{Benedikt:2653673}.  An electron-positron collider, FCC-ee~\cite{Benedikt:2651299}, is considered as a possible first step to precisely measure the Higgs properties, improve by orders of magnitude the measurement of key electroweak parameters and complement the study of heavy flavours of Belle2~\cite{Kou:2018nap} and LHCb\cite{Alves:2008zz,Bediaga:2018lhg}. 

This vast physics program relies in many ways on the calorimeters, whose performance is enhanced by the inclusion of the tracking information with particle-flow methods.  In particular, calorimeters must provide precise hadronic jet measurements in two- or more-jet final states.  Z, W or Higgs decays into two jets have the largest branching fractions for each of the bosons, and ZZ and WW final states represent the major backgrounds to most Higgs-boson decay modes.  A $\sim$3-4\% two-jet invariant-mass resolution is needed to adequately classify all relevant final states. This is a hard requirement on the hadronic calorimetry that cannot be achieved with conventional methods.  New detector concepts that can meet these goals are high-granularity tracking calorimeters, optimised for particle flow (PFlow), and dual-readout calorimeters.  Recently, it has been shown that calorimeters based on cryogenic liquids with high readout granularity can also be optimised for particle-flow and 4D imaging techniques and hence become viable choices.  

%New reconstruction and analysis tools based on machine learning as well as a large increase in read-out granularity have been shown to improve hadronic calorimeter resolution in many contexts, thus making also more conventional technologies, like those based on cryogenic liquids, viable choices.

The requirements on electromagnetic (EM) calorimeters are mainly driven by the need of a good $\pi^0$ reconstruction, that is relevant for the identification of specific $\tau$-lepton or heavy-flavoured-hadron final states. Physics sensitivity to some processes accessible via radiative return also requires a very good EM resolution.  Best performances are given by technologies based on cryogenic noble liquids or crystals, the latter providing extreme EM resolution.

In the following we present the current status and prospects for all the calorimeter technologies relevant for FCC-ee and discuss their key R\&D issues. 


