% version prd_rc_v2

\documentclass[aps,prl,twocolumn,a4paper]{revtex4}


\usepackage{amsmath,amssymb} 
\usepackage{bm} %
 
\usepackage{color} %
\usepackage{graphicx}



\usepackage[colorlinks]{hyperref} %
\hypersetup{
citecolor=blue,
linkcolor=blue,
urlcolor=blue
}


\usepackage{epstopdf}
\epstopdfsetup{suffix={}}


\usepackage{multirow}


\usepackage{comment}

%\newcommand{\note}[1]{\textcolor{red}{#1}}{\ignorespacesafterend}
\newcommand{\note}[1]{#1}{\ignorespacesafterend}
\newenvironment{notered}{\color{red}}{\ignorespacesafterend}
\newenvironment{noteblue}{\color{blue}} {\ignorespacesafterend}

\newcommand{\mnote}[1]{\marginpar{\textcolor{red}{#1}}}

\newcommand{\fnote}[1]{\footnote{\textcolor{red}{#1}}\marginpar{\textcolor{red}{\tiny footnote}}}

\newcommand{\chk}{\textcolor{red}{\bf{?} }\marginpar{\textcolor{red}{***}}}
\newcommand{\mk}[1]{{\bf [ #1 ]\\}}




\newcommand{\Sec}[1]{Sec.~\ref{#1}}  %
\renewcommand{\figurename}{FIG.}
\newcommand{\Fig}[1]{Fig.~\ref{#1}}
\newcommand{\fig}[1]{Fig.~\ref{#1}}
\renewcommand{\tablename}{TAB.} %
\newcommand{\Tab}[1]{Tab.~\ref{#1}}
\newcommand{\tab}[1]{Tab.~\ref{#1}}
\newcommand{\Eq}[1]{Eq.~(\ref{#1})}
\newcommand{\Eqs}[1]{Eqs.~(\ref{#1})}
\newcommand{\eq}[1]{Eq.~(\ref{#1})}
\newcommand{\eqs}[1]{Eqs.~(\ref{#1})}

\newcommand{\tx}[1]{\text{#1}}
\newcommand{\vrel}{v_{\text{rel}}}	
\newcommand{\sigmav}{\langle \sigma v \rangle}
\newcommand{\rsun}{r_{\odot}}
\newcommand{\VEV}[1]{\left\langle #1\right\rangle}
\newcommand{\ekin}{E_{\text{kin}}}
\newcommand{\pbar}{\bar p}

\listfiles




\begin{document}
\title{Origins of  sharp cosmic-ray electron  structures and the DAMPE excess}
\author{Xian-Jun Huang$^{a,b}$}
\author{Yue-Liang Wu$^{a}$}
\author{Wei-Hong Zhang$^{a}$}
\author{Yu-Feng Zhou$^{a}$}
\affiliation{$^{a}$
	CAS Key Laboratory of Theoretical Physics, 
	Institute of Theoretical Physics, Chinese Academy of Sciences,
	%ZhongGuanCun East Rd.55, 
	Beijing, 100190, China,
	}
\affiliation{
	University of Chinese Academy of Sciences, 
	%Yuquan Rd.19A, 
	Beijing 100049, China,
}
\affiliation{$^{b}$
 Sichuan University of Science and Engineering, Zigong 643000, China.}
\begin{abstract}
Nearby sources  may contribute to cosmic-ray electron  (CRE) structures 
at high energies.
Recently, the first  DAMPE results on the CRE flux hinted at 
a  narrow  excess 
at   energy $\sim 1.4$~TeV.  
We show that in general a  spectral structure with a narrow width 
appears  in two scenarios:
I) ``{\it Spectrum broadening}'' for the continuous sources 
with a $\delta$-function-like  injection spectrum.
In this scenario, %
a finite width can develop after propagation through the Galaxy,
which
can reveal %
the distance of the source.
Well-motivated   sources include mini-spikes and subhalos formed 
by dark matter (DM) particles $\chi_{s}$ which annihilate directly into $e^{+}e^{-}$ pairs.
II) ``{\it Phase-space shrinking}'' for burst-like sources with a power-law-like injection spectrum.
The  spectrum after propagation can shrink at a cooling-related cutoff energy 
and form a sharp spectral peak.
The peak can be  more prominent due to the energy-dependent diffusion.
In this scenario, the width of the excess constrains both 
the power index and the distance of the source.
Possible such sources are pulsar wind nebulae (PWNe) and supernova remnants (SNRs).
We analysis the DAMPE excess and find that 
the continuous DM sources should be fairly close within $\sim 0.3$~kpc, 
and the annihilation cross sections are close to 
the thermal value.
For the burst-like source, 
the narrow width of the excess suggests that 
the injection spectrum must be  hard 
with power index significantly less than two, 
the distance is within $\sim(3-4)$~kpc, and
the age of the source is  $\sim 0.16$~Myr.
In both scenarios,  
large  anisotropies  in the CRE flux are predicted. 
We identify possible candidates of mini-spike and PWN sources
in the current Fermi-LAT 3FGL and ATNF catalog, respectively.
The diffuse $\gamma$-rays from these sources
can be well below the Galactic diffuse $\gamma$-ray backgrounds
and less constrained by the Ferm-LAT data,
if they are located at the low Galactic latitude regions.	
\end{abstract}
\pacs{xx.xx}
\preprint{ [\today ]}
\maketitle %



\paragraph{\bf Introduction.}
Cosmic-ray (CR)  electrons and positrons (CREs)
with energies above TeV  plays an important role in understanding 
the nearby origins of CRs within a few kpc
\cite{Shen:1970apj}.
%
Structures in the energy spectrum of CREs are expected, 
if the nearby sources are dominated by one or a few discrete sources.
%
The current space experiments have begun to directly probe  this energy region.
For instance, 
the AMS-02~\cite{Aguilar:2014fea}, 
Fermi-LAT~\cite{Abdollahi:2017nat} and 
CALET~\cite{Adriani:2017efm} experiments have measured 
the  flux of CRE up to 1, 2 and 3~TeV, respectively,  
without observing any significant structures so far.
%\cite{
%	Aguilar:2014fea,%
%	Abdollahi:2017nat,%
%	Adriani:2017efm%
%}.
Recently, 
the DAMPE experiment has reported  the first high energy resolution 
measurement of  the CRE flux up to  4.6~TeV
\cite{Chang:2017xx}.
The measured energy spectrum of CRE 
steepened above $\sim0.9$~TeV,
consistent with the  results from 
the ground-based atmospheric Cherenkov telescopes
\cite{
	Aharonian:2008aa,%
	Aharonian:2009ah,%
	HESS:2017ICRC,%
	BorlaTridon:2011dk,%
	Staszak:2015kza%
}.
Of interest, the DAMPE data  also hinted at an excess over  
the expected background  in a narrow energy interval  $\sim(1.3-1.5)$~TeV.
Making use of the  DAMPE data in the energy range 55~GeV--4.6~TeV, and 
assuming  a broken power-law background flux,
we find that the local and global significance of the possible narrow excess is
$\sim 3.7~\sigma$ and $\sim2.5~\sigma$, respectively
(details of the data analysis are shown in the supplementary material).
%we find that the  background-only fit  gives
%$\chi^{2}=25.9$,
%while the scenario of background plus a Gaussian signal leads to 
%$\chi^{2}=12.2$,
%indicating a local significance of  $\sim 3.7~\sigma$.
%



%An  excess with a narrow width, as hinted by DAMPE,
%may shed light on the nature of the nearby sources.
\note{
In light of the possible DAMPE ``excess'',
it is of general interest to address the question of 
what kind of sources are responsible for a sharp
spectral feature in CRE flux.}
In this work, 
we explore the origins of a sharp spectral structure and 
emphasize  that  the space-time location of  source can be inferred from 
the spectral feature of the CRE flux.
We show that in general a sharp spectral structure can be produced   in two  complementary scenarios:
I)
for  continuous sources with a line-shape injection spectrum, 
a  finite width can develop after propagation in the Galaxy
(dubbed ``spectrum broadening'').
Well-motivated  sources are nearby DM substructures 
such as mini-spikes and DM subhalos of DM particles $\chi_{s}$
with  $e^{+}e^{-}$ the dominant annihilation final states.
In this scenario, the spectral shape or the width of the excess can be used to estimate  
the distance to the source.
II)
for burst-like sources with a power-law injection spectrum,
the spectrum after propagation can shrink at a cooling-related cutoff energy 
and form a sharp spectral peak  (dubbed ``phase-space shrinking'').
Energy-dependent diffusion also contributes  to the spectral rising. 
%if the distance of the sources exceeds  the diffusion length. 
Typical  sources of this type are pulsar wind nebulae (PWNe) and 
supernova remnants (SNRs).
In this scenario, the power index and the distance of the source are strongly 
constrained by the width of the excess.

%We analyses these generic scenarios for the possible DAMPE excess and 
%show that:
In view of the DAMPE excess, we find: 
\romannumeral 1)
for the continuous sources, 
the favoured distance should be less than $\sim 0.3$~kpc.
The source can be  ``mini-spikes'' or DM subhalos  with 
the favoured DM annihilation cross section around the typical thermal value.
\romannumeral 2)
For the burst-like source, 
the injection spectrum must be hard with power index significantly below two, and
the distance within $\sim(3-4)$~kpc. 
The age of the source is determined to be $\sim0.16$~Myr.
\romannumeral 3) 
For both sources,
large anisotropies in the arrival direction of the CRE are predicted,
which are close to the current Ferm-LAT upper limits.
\romannumeral 4) 
We identify possible candidates for  mini-spikes (PWNe)
from the catalogue of Fermi-LAT  3GFL 
%unassociated point-sources 
(ATNF pulsar catalogue).
%\romannumeral 5) 
%The sources are likely to be located in the low-latitude regions
%where the background of Galactic diffuse $\gamma$-rays is relatively large,
%which evades  the stringent constraints from the Fermi-LAT data.
\begin{figure}[thb]
	\centering
	\includegraphics[width=0.48\columnwidth]{flux_pointDM_rDep.eps}
	\includegraphics[width=0.48\columnwidth]{flux_SNR_rDep.eps}
	\caption{		
		Left) 
		Effect of  ``spectrum broadening'' on CRE flux from 
		a continuous point-like source described in 
		\eq{eq:continuous-point-solution} with a growing distance $r=0.1-1.0$~kpc.		 
		$E_{0}$ is fixed at 1.4~TeV.
		Right)
		Effect of ``phase-space shrinking'' for the burst-like sources in 
		\eq{eq:burst--solution} with  a decreasing $\alpha$ from 2.0 to 0.7 and fixed $r=1$~kpc,
		and ``energy-dependent diffusion''  for  a growing  distance $r=1-3$~kpc with fixed $\alpha=2$.
		The age of the source is  fixed at $t=0.15$~Myr.
		For both sources, the spectra are normalized to a total flux
		$\Phi=10^{-9}\mbox{m}^{-2}\mbox{s}^{-1}\mbox{sr}^{-1}$.
	}\label{fig:spectral-feature}
\end{figure}
\begin{figure*}[htb]
	\centering
	\includegraphics[width=0.95\textwidth]{flux_log.eps}
	\caption{
		Best-fit CRE flux from fitting to the DAMPE data~\cite{Chang:2017xx}  
		for three type of sources.
		Left) Continuous point-like sources (mini-spikes) with distance $r$=0.1, 0.2, 0.3~kpc, 				respectively.
		Center) Continuous extended sources (DM subhalos) for the same distances with subhalo
		mass fixed at $10^{7}M_{\odot}$.
		Right) Burst-like sources (PWNe/SNRs) with $(r (\tx{kpc}),\alpha)$ values 
		(1, 0.5), 
		(2, 0.7), 
		(3, 1.3),			
		respectively.	
		The solid (dashed) curves are  the sum of the signal and background (signal only).
		The data of  DAMPE~\cite{Chang:2017xx},
		AMS-02~\cite{Aguilar:2014fea} and
		Ferm-LAT~\cite{Abdollahi:2017nat} 
		are also shown.
	}\label{fig:flux}  
\end{figure*}

\paragraph{\bf Continuous  sources.}
The propagation of  CR electrons  is described by  
the following diffusion equation
\cite{
	Ginzburg%
}
\begin{align}\label{eq:propagation}
\frac{\partial f}{\partial t}  =
\frac{D(E)}{r^{2}}\frac{\partial}{\partial r}r^{2}\frac{\partial}{\partial r} f
+\frac{\partial}{\partial E}\left(B(E)f\right)
+Q(r,t,E),
\end{align}
where $f(r,t,E)$ is the number density function per unit energy
with $r$ the distance to the source,
$Q(r,t,E)$ is the source term.
In the  equation we have neglected the effects of 
convection and re-acceleration
as they are only important at low energies. %
The energy-dependent spatial diffusion coefficient $D(E)$  is parametrized as
$D(E)=D_{0}(E/\mbox{GeV})^{\delta}$,
where $\delta=0.31$ is a power law index and  
$D_{0}=5.49\times 10^{28}\mbox{ cm}^{2}\mbox{s}^{-1}$ %
\cite{
	Trotta:2010mx%
}. 
The energy-losses due to ICS processes and synchrotron radiations  are prametrized as 
$B(E)=b_{0} E^{2}$ with
$b_{0}=1.4\times10^{-16}~\mbox{GeV}^{-1}\mbox{s}^{-1}$
\cite{
	Linden:2013mqa%
}.





The sources of CRE can be roughly divided into 
continuous  and burst-like sources,
according to the time scale of  electron injection from the sources relative to that of  the propagation time.
A possible continuous source  is the annihilation of DM particles in the Galaxy.
Very sharp $\delta$-function-like CRE spectrum can be produced from
DM annihilation directly into  $e^{+}e^{-}$ pairs 
%%
\note{in models with  enhanced DM-electron coupling,
or  through light mediators with mass very close
to twice the electron mass.
}
%%
However, for any continuous source, the finally observed spectrum is
a superposition of electrons injected at different time.
The electrons injected earlier suffer from more energy losses.
Thus the superposition inevitably results in the broadening of  the spectrum.
For a continuous point-like source
with a $\delta$-function injection spectrum
$Q(r,E) \approx Q_{0}\delta(E-E_{0})\delta^{(3)}(\mathbf{r})$
with $E_{0}$ the central energy and  $Q_{0}$ the normalization constant,
the analytic solution to  \eq{eq:propagation} is  given by
\cite{
	Atoian:1995ux%
}
\begin{align}\label{eq:continuous-point-solution}
f(r,E)=\frac{Q_{0} E^{-2}}{\pi^{3/2}b_{0} 
	r^{3}_{d}(E)}\exp\left(-\frac{r^{2}}{r^{2}_{d}(E)}\right)  ,
\end{align}
where 
$r^{2}_{d}(E)=4 D_{0}
[(E/\mbox{GeV})^{\delta-1}-(E_{0}/\mbox{GeV})^{\delta-1}]/(1-\delta) (b_{0}\text{GeV})$ 
is the diffusion length.
As the solution shows,  after the propagation, 
the  spectrum is broadened.
For $E\ll E_{0}$, it is  
an approximate power law  $f\propto E^{-(1+3\delta)/2}$.
The spectrum rises rapidly when $E$ is approaching $E_{0}$ and 
eventually  cut off exponentially at $E_{0}$ as $r_{d}(E)\approx 0$.
In the left panel of \fig{fig:spectral-feature}, we show how the 
spectral shape of CRE changes with growing distance $r$. 
In the region near  $E\approx E_{0}$  
the spectral shape is very sensitive to the distance.
Increasing the  distance will  result in a broader excess. 
Therefore,  a precision measurement on the spectral shape can be used to 
determine the distance. 
Note that the diffusion length $r_{d}$ 
can only set the scale of the maximal distance.

In the left panel of \fig{fig:flux},
we show the best-fit fluxes obtained from fitting to the DAMPE data 
for three fixed values of  $r=0.1-0.3$~kpc. 
Other parameters such as $E_{0}$, $Q_{0}$ and the background parameters
are allowed to vary in the fits.
In all the  three cases, the best-fit values of $E_{0}$ are quite similar 
$E_{0}\approx1.4-1.5$~TeV. 
With increasing value of $r$, the best-fit spectrum becomes broader and  
the fit  quality becomes lower.
From $r=0.1$ to 0.3~kpc, the $\chi^{2}$-value increases from  $14.2$ to $19.2$.
The fit including  $r$ as a free parameter shows  that 
the DAMPE data place an upper limit of 
$r \lesssim 0.3$~kpc at $95\%$~C.L.. 
For the three cases, the best-fit values of the normalization constants are $Q_{0}$=$(0.47-2.1)\times 10^{33}~\mbox{s}^{-1}$, respectively.
The detailed list of best-fit parameters and allowed regions are shown 
in Tab.~S-2 and Fig.~S-3
of the supplementary material. 

\paragraph{Mini-spikes.}
One of the possible continuous point sources is the ``mini-spike", 
i.e., the large DM density enhancements around the
intermediate mass black holes (IMBHs) with mass $\sim 10^{2}-10^{6} M_{\odot}$
\cite{
	Miller:2003sc,%
	Zhao:2005zr,%
	Bertone:2005xz,%
	Bertone:2009kj%
}.
The IMBH can form 
out of popIII stars
\cite{
	Heger:2002by%
} 
or collapsing of primordial gas in early-forming halos 
\cite{
	Koushiappas:2003zn%
}.
In this letter we consider the latter case of IMBH formation.
For the Milky Way-sized Galaxy, 
the total number of this type of IMBHs is around $\mathcal{O}(100)$
with $\sim 30\%$ of them located in the inner region $ \lesssim10$~kpc
\cite{
	Bertone:2005xz%
}.
Starting from an initial NFW DM profile~\cite{Navarro:1996gj}, 
the spiked DM profile of the mini-spike after the adiabatic growth
of the IMBH follows a power law
$\rho_{sp}(r)=\rho(r_{sp})(r/r_{sp})^{-\gamma_{sp}}$
where 
$\rho(r)$ is the initial DM profile,
$r_{sp}\sim \text{pc}$ is the typical radius of the mini-spike and 
$\gamma_{sp}\approx 7/3$ is the power index%
~\cite{
	Gondolo:1999ef%
}.
Due to the DM annihilation, the spiked DM profile is cut off at a very small distance  $r_{\text{cut}}\sim 10^{-3}$~pc
\cite{
	Bertone:2005xz%
}.
Assuming Majorana DM particles $\chi_{s}$ which annihilate dominantly 
into $e^{+}e^{-}$ pairs  with velocity-averaged  annihilation cross section 
$\langle \sigma v \rangle$, 
the source term of ``mini-spikes" can be estimated as 
\begin{align}\label{eq:Q0}
Q_{0}\approx
&3.1\times 10^{33}~\mbox{s}^{-1}
\left( \frac{\langle\sigma v\rangle}{3\cdot 10^{-26}~\mbox{cm}^{3}\mbox{s}^{-1}} \right) 
\left( \frac{1.4~\mbox{TeV}}{m_{\chi}} \right)^{2}
\nonumber\\
&\left(\frac{\rho(r_{sp})}{10^{2}~\mbox{GeV}\mbox{cm}^{-3}} \right)^{2}
\left(\frac{r_{sp}}{\mbox{pc}} \right)^{14/3}
\left( \frac{r_{\text{cut}}}{10^{-3}~\mbox{pc}}\right)^{-5/3}     .
\end{align}
For the three cases of 
$r=0.1-0.3$~kpc, using the best-fit values of $Q_{0}$, the corresponding cross section are
%$\langle \sigma v \rangle=(0.48,\ 1.36,\ \text{and}\ 2.48)\times 10^{-26}\text{cm}^{3}\text{s}^{-1}$, respectively, which are close to the typical thermal value.
$\langle \sigma v \rangle=(0.48-2.48)\times 10^{-26}\text{cm}^{3}\text{s}^{-1}$, 
which are close to the typical thermal value.



\paragraph{DM subhalos.}
It is straightforward to extend the analysis to 
the  spatially-extended sources.
N-body simulations of cold and collisionless  DM predict that 
the Galaxy  should contain large number of small subhalos
%
~\cite{
	Springel:2008cc,%
	Diemand:2008in,%
	Garrison-Kimmel:2013eoa%
}.
%
\note{
Based on a joint analysis to the Via Lactea II and  ELVIS simulations
\cite{Hooper:2016cld}, 
we estimate that the possibility of finding a nearby subhalo  
within $\lesssim 1 (0.3)$~kpc and total mass 
$M_{h} \gtrsim10^{6} M_{\odot}$ is around 
$\sim 1.2\% (0.03\%)$. 
%
An alternative possibility is  the ultra compact mini-halos (UCMHs) formed  in the early epochs of the Universe~\cite{
%% proposal of UCMH
Ricotti:2009bs,%% 0908.0735  Astrophys. J. 707 979-987 (2009)
%% Ricotti,Gould,
%% A New Probe of Dark Matter and High-Energy Universe Using Microlensing
%%
%% prediction for gamma-rays
Scott:2009tu,%% 0908.4082  Phys. Rev. Lett. 103 211301 (2009)
%% Scott,Sivertsson,
%% Gamma-Rays from Ultracompact Primordial Dark Matter Minihalos
%%
%%
Josan:2010vn,%% 1006.4970  Phys. Rev. D82 083527 (2010)
%% Josan,Green,
%% Gamma-rays from ultracompact minihalos: potential constraints on the primordial curvature perturbation
%%
%%
Bringmann:2011ut%% 1110.2484  Phys. Rev. D85 125027 (2012)
%% Bringmann,Scott,Akrami,
%% Improved constraints on the primordial power spectrum at small scales from ultracompact minihalos
%%
}.
Finding a nearby UCMH within $\sim0.1$~kpc requires
that UCMHs contribute to a fraction of  above $\sim1\%$~ of the total DM density~\cite{Scott:2009tu}.
%
In this work, we shall focus on the case of  DM subhalo. %with 
%mass $M_{h}$ and at a distance $r$.
%We consider the possibility that the source is  a nearby DM %subhalo 
%(see ~\cite{X} for UCMH explation of DAMPE data).
}
As subhalos may experience a significant degree of mass loss due to tidal stripping,
especially for those located at  the inner volume of the Galaxy,
we  adopt a tidally truncated  density profile
$\rho(r)=\rho_{0}(r/\text{kpc})^{-\gamma}\exp(-r/R_{b})$
\cite{
	Kazantzidis:2003hb,%
	Penarrubia:2007zx,%
	Hooper:2016cld%
}.
The parameters  $\rho_{0}$, $\gamma$ and $R_{b}$  depend on 
the distance  $d$ from the center of the subhalo to the Galactic center
and 
the total mass $M_{h}$ of the subhalo, which  
can be extrapolated from the N-body simulation data.
%Making use of the results of a joint analysis to the Via Lactea II %and  ELVIS simulations
From the analysis in Ref.
\cite{
	Hooper:2016cld%
},
we obtain $\rho_{0}=5.3~\text{GeV}\cdot\text{cm}^{-3}$, $\gamma=0.78$ and $R_{b}=0.096$~kpc, 
for a typical $M_{h}=10^{7} M_{\odot}$.

The CRE flux from the subhalo is calculated by integrating the solution of 
\eq{eq:continuous-point-solution} over the subhalo density distribution.
For $r=0.1$, 0.2, and 0.3~kpc,
we find  the best-fit annihilation cross sections 
$(1.04, \ 2.55,\  4.62)\times 10^{-26}~\mbox{cm}^{3}\text{s}^{-1}$, respectively,
which are well below  the current limits on DM subhalos
\cite{
	Fermi-LAT:2016uux,%
	Schoonenberg:2016aml,%
	Hooper:2016cld%
}.
Similar to the case of point source, 
when $r$ increases from 0.1 to 0.3~kpc, 
the best-fit spectrum becomes broader and the fit qualities become lower.
The corresponding $\chi^{2}$ value increases from 13.4 to  19.2.
From a fit with $r$ as a free parameter, 
we find that  the source should be within $r \lesssim 0.3$~kpc at $95\%$~C.L., 
very close to the case of mini-spikes.
We also find that modifying the subhalo mass $M_{h}$ 
does not change the conclusion.
%
The detailed list of best-fit parameters and allowed regions
are shown in Tab.~S-4 and Fig.~S-4
of the supplementary material. 
%
Note that 
for  an  annihilation cross section of 
$\mathcal{O}(10^{-26})~\tx{cm}^{3}\tx{s}^{-1}$ %
the contribution from the whole Galactic halo DM 
to the CRE flux  is typically two orders of magnitude  smaller than 
that from the nearby sources,
which can be safely neglected.
%
\note{
The halo DM also contribute to extra CR positrons which 
could be constrained by the experiments.
The constraints, however, turn out to be rather weak for 
TeV scale DM, 
as the current AMS-02  experiment only measured the 
positrions up to 
$\sim 350$~GeV~\cite{Accardo:2014lma}
%
(see Fig.~S-6 in the supplementary material).
%for the Galactic halo DM contributions
%to CRE and CR positrons).
}






\paragraph{\bf Burst-like  sources.}


For burst-like sources such as PWNe/SNRs,  
the injection spectrum is expected to be
a broad power-law with an exponential cutoff,
$Q(r,t,E)=N_{0} (E/\mbox{GeV})^{-\alpha} 
\exp(-E/E_{c}) \delta^{(3)}(\mathbf{r})\delta(t)$, 
where $\alpha$ is the power-law index and
$E_{c}$ is the cutoff energy.
The normalization constant $N_{0}$ can be  related to 
the total released energy  $E_{\tx{tot}}$.
%as $E_{\text{tot}}=\int E Q(r,t,E) dE d^{3}r dt$.
The solution to the diffusion equation~\eq{eq:propagation} 
for this type of source is given by~\cite{Atoian:1995ux}
\begin{align}\label{eq:burst--solution}
f(r,E)=\frac{N_{0}
	(E/\text{GeV})^{-\alpha}}{\pi^{3/2}r_{d}^{3}} 
\xi(E)^{\alpha-2}
\exp\left(-\frac{r^{2}}{r_{d}^{2}}-\frac{E}{\xi(E) E_{c}}\right) ,
\end{align}
where $\xi(E)=1-E/E_{\text{max}}$ 
with $E_{\text{max}}=(b_{0}t)^{-1}$  the maximal possible energy of 
an electron from a source of age~$t$.
The diffusion length for this type of source is
$r_{d}(E) =2\sqrt{ \lambda(E) D(E)t}$,
where $\lambda(E)=[1-\xi(E)^{1-\delta}]/(1-\delta)(1-\xi(E))$.
For $E\ll E_{\text{max}}$, $r_{d}(E) \approx 2\sqrt{D(E)t}$.
While the value of $\alpha$ is commonly considered to be $\sim 2$,
the cutoff $E_{c}$ is poorly constrained.
We shall focus on the large cutoff limit, 
i.e. $E_{c}\gg E_{\text{max}}$.
In this case,
$E_{\text{max}}$ will play the role of spectral cutoff 
instead of $E_{c}$, namely,
a cooling cutoff will appear,
as can be seen in \eq{eq:burst--solution}.




In this work, we emphasize that for the burst-like source 
a sharp spectral rise near the cutoff energy $E_{\text{max}}$
can appear for some choices of $\alpha$ and $r$
for two reasons:
{\bf i)}  Cooling related ``phase-space shrinking''.
An initial  electron with energy $E_{s}$ at time $t=0$
is related to its energy $E$ observed at later time $t$ as $E_{s}=E/\xi(E)$.
Thus an initial energy interval  $\Delta E_{s}$ will shrink to 
$\Delta E=\xi(E)^{2} \Delta E_{s}$ at time $t$.
Since the number of electrons  is unchanged during cooling,
the energy spectrum at time $t$ is  
$\Delta N/\Delta E \approx E^{-\alpha}\xi(E)^{\alpha-2}$.
For a relatively hard spectrum with  $\alpha <2$, 
the shrinking of phase space  can enhance the number density.
Since $\xi(E)$ vanishes when $E$ is approaching $E_{\text{max}}$,
the shrinking of phase space leads 
to a rapid rise of the energy spectrum.
This effect of ``phase-space shrinking'' is illustrated  in
the right panel of \fig{fig:spectral-feature}.
%where the $\alpha$ dependence of the electron spectrum is shown 
%for a fixed distance $r$.
The DAMPE data suggest that the cutoff  should be 
in the range $E_{\text{max}}\approx 1.3-1.5$~TeV,
which in turn sets the  age of the source
$t  %
\approx  (0.15-0.17)~\text{Myr}$,
and the diffusion length at $E_{\text{max}}$, 
$r_{d}(E_{\text{max}}) \approx (0.79-0.84)~\text{kpc}$.
{\bf ii)}``Energy-dependent diffusion'' which is related to 
the fact that for $\delta>0$,
higher energy electrons have larger diffusion coefficients.
The energy dependence in the exponential factor
of \eq{eq:burst--solution} can be written as
$\exp(-r^2/r_d^2)\approx \exp[- \kappa^{2}(E/E_{\text{max}})^{-\delta}]$
where $\kappa=r^{2}/r_{d}^{2}(E_{\text{max}})$.
For relatively  large distance  $\kappa >1$, 
the energy-dependent factor also contributes to the rising of the spectrum near $E_{\text{max}}$,
which  is illustrated in  \fig{fig:spectral-feature}.
Of course, in order to  compensate the exponential suppression of the flux at 
large $\kappa$, 
the  normalization constant $N_{0}$ 
or $E_{\text{tot}}$ has to be large enough.
The reasonable value of $E_{\text{tot}}$ should be smaller than 
the typical kinetic energy carried by SNR 
%of  $\sim10^{51}$~erg
or the total energy of supernova explosion of  
$\sim(10^{51}-10^{53})$~erg,
which sets the scale of the distance of the sources.






In general, unconventional values of  $\alpha$ 
which is  significantly smaller than $\sim 2$ is required to 
reproduce the DAMPE excess, 
especially for small $r$.
In the right panel of  \fig{fig:flux}  we show the best-fit spectra for  
three typical combinations of  $r$ (in kpc) and $\alpha$ with 
$(r,\alpha)$=(1, 0.5), (2, 0.7) and (3, 1.3), respectively.
A scan in the $(r,\alpha)$ parameter space shows that in the region 
$r<r_{d}$, the allowed $\alpha$ has to be very small $\alpha \lesssim0.65$.
In the region $r>r_{d}$ the value of $\alpha$ can reach at most $1.4$ at 3~kpc,
as the effect of  energy-dependent diffusion is  significant.
By imposing the condition of $E_{\text{tot}}<10^{51}(10^{53})$~erg,
the distance $r$ is restricted in the range $r\lesssim$3(4)~kpc.
Together with the required  age of the source,
we find 7 candidate pulsars with $r \lesssim 4$~kpc
in the ATNF catalog of pulsars~\cite{Manchester:2004bp}:
B0740-28, J0922-4949,   J1055-6022,  J1151-6108,
J1509-5850,  J1616-5017 and J1739-3023.
Distances of the sources lie in the range (2.0--3.6)~kpc.
In this region, both the effects of ``phase-space shrinking '' and 
``energy-dependent diffusion'' are relavant.
The detailed list of best-fit parameters, allowed regions
and the list of  the candidate pulsars are shown in
Tab.~S-6, Fig.~S-7 and Tab.~S-7
of the supplementary material. 
%
\note{
For both continuous and burst-like sources, 
varying the propagation parameters
$D_{0}$ and $\delta$ within uncertainties 
($\sim 20\%$ for $D_{0}$ and $\sim10\%$ for $\delta$
as determined in \cite{Trotta:2010mx}) mainly results in 
the changes in the over all normalization factors 
$Q_{0}$ and $N_{0}$ up to $\sim30\%$.
}







\begin{figure}[!thbp]
	\begin{center}
		\includegraphics[width=0.65\columnwidth]{anisotropy_1.eps}
		\caption{
			Predictions for electron anisotropies corresponding to 
			a selection of the cases considered in \fig{fig:flux}:
			i) the continuous point-like source (mini-spike) with $r=0.2$~kpc (dashed blue);
			ii) the continuous extended source (subhalo) with $r=0.2$~kpc and $M_{h}=10^{7}M_{\odot}$ (dashed green);
			iii) the burst-like source (PWN/SNR) with $r=2$~kpc and $\alpha$=0.7 (dashed magenta).
			The solid curves with the same color correspond to the anisotropies convoluted with 
			an energy resolution of $15\%$.
			%The solid horizontal lines with the same color correspond to that averaged over the 
			%energy interval 0.55--2~TeV.  
			The current upper limits from Fermi-LAT in the  energy interval 0.55--2~TeV
			(using the method of shuffling technique)
			are also shown%
			~\cite{Abdollahi:2017kyf}.
		}
		\label{fig:anisotropy-combined}
	\end{center}
\end{figure}




\paragraph{\bf Anisotropies.}
Nearby sources can generate non-negligible anisotropy in the CRE flux.
%CRE flux produced by nearby sources can have non-negligible anisotropy.
For an illustration,
we show in \fig{fig:anisotropy-combined}  the predicted 
dipole anisotropies from the sources as a function of the CRE energy,
corresponding to one of the  parameter sets
considered in each type of the sources shown in \fig{fig:flux}.
Assuming  a perfect energy resolution of the detector,
%and neglecting  the anisotropies of the background flux.
%For all the cases considered, 
large anisotropies of $\mathcal{O}(10^{-1})$ with 
sharp structures  are predicted.
The anisotropies in continuous sources are in general 
larger than that in the burst-like sources,
which is  related to the relatively small diffusion length.
%
The Fermi-LAT  has reported  upper limits on the  
dipole anisotropy of $\lesssim3\times 10^{-2}$  at $95\%$ C.L.
over the energy interval  0.55--2~TeV%
~\cite{Abdollahi:2017kyf}.
Note that the energy resolution of Fermi-LAT is $\sim 10\%~(17\%)$ 
at 1 (2)~TeV.
In  \fig{fig:anisotropy-combined}, 
we also show  the predicted anisotropies convoluted 
with an energy resolution of $15\%$.
%the anisotropies averaged over the interval 0.55--2~TeV,
%together with the Fermi-LAT limits% 
%derived from  log-likelihood ratio using the methods of  shuffling technique 
%~\cite{Abdollahi:2017kyf}.
After the convolution, 
the predicted anisotropies are  smaller and can reach $\mathcal{O}(10^{-2})$,
which is comparable with  the current Fermi-LAT limits.
%
Note that a quantitative comparison with the data requires 
a reliable estimation of the anisotropies contributed by the backgrounds 
which can easily reach $\mathcal{O}(10^{-3}-10^{-2})$ alone,  but 
%has large uncertainties as it strongly 
depends strongly on the assumed spatial distribution of 
the astrophysical sources%
~\cite{Manconi:2016byt}.
%The total anisotropy of also  depends on the relative directions of the nearby sources.
%

%which is highly testable in the near future by Fermi-LAT
%with higher statistics or using  long-tail events to improve the energy resolution to $\sim4\%$ at 1~TeV \cite{Abdollahi:2017nat}.






\begin{figure}[thb]
	\begin{center}
		\includegraphics[width=0.49\columnwidth]{GAMMA_deg.eps}
		\includegraphics[width=0.49\columnwidth]{GAMMA_sptr_f_30.eps}
		\caption{
			Predictions for the Galactic diffuse $\gamma$-ray fluxes in the three cases
			considered in \fig{fig:anisotropy-combined} assuming the direction of the sources
			is coincide with the GC. %($b=0^{\circ},\ell=0^{\circ}$).
			Left) 
			$d\Phi_{\gamma}/d\Omega$ as a function of Galactic latitude $b$
			along the direction of  $\ell=0^{\circ}$.
			The inset shows the fluxes  in the  inner region,
			which indicates that the $\gamma$-rays from the mini-spike
			increase sharply towards low $b$ due to prompt photons from DM annihilation, 
			making it appears as a point-like source.
			Right)
			Energy spectra of $\gamma$-ray fluxes averaged over a circular region with radius $30^{\circ}$
			centered at the GC.
			The Galactic diffuse $\gamma$-ray background is 
			calculated using  GALPROP-v54 \cite{galprop}
			with a reference propagation model 
			adopted by Fermi-LAT
			\cite{
				Ackermann:2012pya%
			}.
			The ISRF in the solar neighbourhood is interpolated from Ref.
			\cite{
				Porter:2008ve%
			}.
		}
		\label{fig:GAMMA}
	\end{center}
\end{figure}



\paragraph{\bf Gamma-ray signals.}
%The sources of CRE are inevitably sources of $\gamma$-rays. 
DM annihilation can produce prompt photons through final state radiation (FSR)  of charged leptons. 
The FSR photon spectrum has a distinct feature of an approximate power law with index $\sim1$ for $E\ll m_{\chi}$, plus a sharp cutoff at $m_{\chi}$.
In morphology, 
mini-spikes appear as point-like sources due to 
the high concentration of DM density.
For the cases of mini-spike considered in \fig{fig:flux}, 
the total fluxes ($E>1$~GeV) %($1<E<100$~GeV with $68\%$ containment) 
are in the range  
$\sim (0.71-1.51)\times 10^{-10}\text{cm}^{-2}\text{s}^{-1}$.
In the Fermi-LAT 3FGL catalogue of unassociated point sources~\cite{Acero:2015hja},
we find 6 candidate  sources with low power indices
which can reach $\sim1$ within $2\sigma$ error:
J0603.3+2042, J1250.2-0233, J2209.8-0450, J1705.5-4128c, J2142.6-2029
and J2300.0+4053.
Most of them have total fluxes of 
$\mathcal{O}(10^{-10})~\text{cm}^{-2}\text{s}^{-1}$  %
which are consistent with that favoured by the mini-spikes.
%%%
%\note{
The favoured 
DM annihilation cross sections 
for both mini-spikes and DM subhalos
are well below the  current upper limits derived from  
the $\gamma$ rays towards 
dwarf Galaxies~\cite{Ackermann:2015zua} and 
the Galactic center (GC)~\cite{Abdallah:2016ygi}
under the assumption of   smooth DM  profiles.
%without taking the substructures into account.
%
Taking into account the distribution of mini-spikes,
very stringent constraints on the annihilation cross sections were
obtained by HESS from the GC $\gamma$-ray data 
\cite{%
	%Bringmann:2009ip,%
	Aharonian:2008wt%
}, 
which are highly model dependent. 
%strongly depends on the modeling
%of the mini-spike distribution in the GC region. 
%
The $\gamma$-rays constraints   from 
other sky regions are much weaker
~\cite{Bringmann:2009ip}.
%and dwarf galaxies
%\cite{
%	Wanders:2014xia,%
%	Gonzalez-Morales:2014eaa%
%}.
%Note, however, that these constraints strongly depend on 
%the models of the merging history of blackholes 
%which have large uncertainties. 
%are largely unknown.
%
%}

All the CRE sources can produce diffuse $\gamma$-rays through 
electron inverse Compton scattering (ICS) off 
interstellar radiation fields (ISRF),
the corresponding energy spectrum is softer compared with 
that of the FSR photons~\cite{Meade:2009iu}.
%
For the typical parameters considered in the three scenarios in \fig{fig:flux}, 
the predicted differential flux $E^{3}d\Phi/dE$ can reach 
$\mathcal{O}(10^{-8})~\tx{GeV}\tx{cm}^{-2}\tx{s}^{-1}\tx{sr}^{-1}$ 
in the energy range $\sim(0.1-1)$~TeV,
which may be subject to the constraints from the Fermi-LAT data 
on the Galactic diffuse $\gamma$-rays~\cite{Ackermann:2012rg}.
%
However, the Fermi-LAT constraints vary with sky regions.
%the constraints can be weakened,
In the  low Galactic latitudes regions the backgrounds can easily 
reach $\sim10^{-6}~\tx{GeV}\tx{cm}^{-2}\tx{s}^{-1}\tx{sr}^{-1}$
%if the sources are located at regions with low Galactic latitudes
%where the backgrounds can easily reach 
%$\sim10^{-6}~\tx{GeV}\tx{cm}^{-2}\tx{s}^{-1}\tx{sr}^{-1}$
and dominate the total diffuse $\gamma$-ray  emission. 
In this case the  Fermi-LAT data cannot place stringent constraints 
on these CRE  sources.
% Fig.4
This possibility is illustrated in \fig{fig:GAMMA}, where
the direction of the source is assumed to  coincide with the GC.
For the three cases discussed  in  \fig{fig:anisotropy-combined},
we show the spatial extension as well as the energy spectrum of the associated $\gamma$-rays,
together with the corresponding Galactic diffuse $\gamma$-ray backgrounds.
%In the left panel of \fig{fig:GAMMA}, 
%we show
%the  $\gamma$-ray fluxes  $d\Phi_{\gamma}/d\Omega$ 
%(with $E>$1 GeV) 
%as a function of  the Galactic latitude $b$  
%along the direction of Galactic altitude $\ell=0^{\circ}$
%(the direction along which the background  decreases in the fastest way)
%for the three cases considered in  \fig{fig:anisotropy-combined}.
%The  energy spectra averaged over a circular region with  a $30^{\circ}$ radius 
%centered at the GC are shown in the right panel of \fig{fig:GAMMA}.
In the calculations, 
the contributions from the Galactic halo DM 
for mini-spikes and DM subhalos are included
assuming an Einasto DM profile.
%
The Galactic diffuse $\gamma$-ray background is 
calculated using a reference propagation model 
adopted by Fermi-LAT~\cite{Ackermann:2012pya}
which agrees with the data well.
%
It can be seen that for  the typical parameters  considered, 
the predicted $\gamma$-ray fluxes can be 
a few orders of magnitude below the background,
%
which suggests that the Ferm-LAT constraints should be 
rather weak, 
as the uncertainties in the background model are still significant
(see the supplementary material for detailed calculations on 
the diffuse $\gamma$-rays).
%for each sources in different sky regions). 
%

%In summary, 
%%the current experiments have entered the multi-TeV region where
%%the CRE  spectrum is unlikely to be smooth.
%we have proposed generic mechanisms of the origins of the CRE structures  and 
%analysed the nature of sources responsible for 
%the possible DAMPE excess.
%The predictions of these scenarios are highly testable in the near future
%with more accurate data.



\begin{acknowledgments}
This work is supported in part by the National Key R$\&$D Program of China 
No. 2017YFA0402204, the NSFC under Grants No. 11335012, 11690022, 11475237 and U1738209,
and the CAS Key Research Programs, No. XDB23030100 and QYZDY-SSW-SYS007.
{\it Note added.} After submitting the first version of the manuscript,
a number of analyses on the DAMPE data appeared%
~\cite{
Yuan:2017ysv,Fan:2017sor,Fang:2017tvj,Duan:2017pkq,Gu:2017gle,
Athron:2017drj,Cao:2017ydw,Liu:2017rgs,Zu:2017dzm,Tang:2017lfb,
Chao:2017yjg,Gu:2017bdw,Duan:2017qwj,Cholis:2017ccs,Jin:2017qcv,
Gao:2017pym,Niu:2017hqe,Chao:2017emq,Chen:2017tva,Li:2017tmd,
Zhu:2017tvk,Gu:2017lir,Nomura:2017ohi,Ghorbani:2017cey,Cao:2017sju,
Yang:2017cjm,Ding:2017jdr,Liu:2017obm,Ge:2017tkd,Zhao:2017nrt,
Sui:2017qra,Okada:2017pgr,Cao:2017rjr,Dutta:2017sod,Fowlie:2017fya,Han:2017ars,
Niu:2017lts%
}.
We found that most of the proposed models fall into the scenarios 
discussed in our work.
Ref.~\cite{Yuan:2017ysv} discussed the contributions from PWNe and DM substructures.
Their conclusions are also consistent with ours.
\end{acknowledgments}




\bibliographystyle{apsrev4-1} %
\bibliography{dampe,misc,dampe_papers}

\clearpage
\onecolumngrid
\appendix 
\section*{Supplementary Materials}
\section{Background: Standard ADMM Training of DNNs} \label{sec:admm_nn}

Alternating Direction Method of Multipliers (ADMM) \cite{gabay1975dual,boyd2011distributed} is a class of optimization methods belonging to  \textit{operator splitting techniques} which borrows benefits from both dual decomposition and augmented Lagrangian methods for constrained optimization. %To show the potentials of standard ADMM, we first revisit a general formulation of ADMM in DNN training, similar to those used in prior work. Then, we propose our stochastic block-ADMM in the next subsection.

To formulate training an $L$-layer DNN in a general supervised setting, we would have the following non-convex constrained optimization problem \cite{zeng2018global}:
% \vspace{-0.1in}
\begin{align} \label{eq:obj}
	\minimize_{ \mathcal{W}, \mathcal{A}, \mathcal{Z}} \quad &\mathcal{J}\left(\mY, \mZ_{L} \right) + \sum_{\ell = 1}^{L} \lambda_{\ell}  {\bf r}_{\ell} (\mW_{\ell}) \\
	 {\rm subject~to} \quad & \mA_{\ell} - {\bm \phi}_{\ell } \left( \mZ_{\ell} \right) = {\bf 0}, \quad \ell = 1,\dots, L-1   \nonumber \\
	 {\rm subject~to} \quad & \mZ_{\ell} - \mW_{\ell} \mA_{\ell-1} = {\bf 0}, \quad \ell = 1, \dots , L \nonumber 
\end{align}
where $\mathcal{J}$ is the main objective (\textit{e.g.}, cross-entropy, mean-squared-error loss functions) that needs to be minimized. The subscript $\ell$ denotes the $\ell$-th layer in the network. The optimization variables are $\mathcal{W} = \{ \mW_\ell\}_{\ell=1}^{L}$, $\mathcal{A} = \{ \mA_{\ell}\}_{\ell=1}^{L-1}$, and $\mathcal{Z} = \{ \mZ_{\ell}\}_{\ell=1}^{L}$ where $\mW_\ell$, $\mZ_{\ell}$, $\mA_\ell$, and ${\bm \phi}_\ell (.)$ are the weight matrix, output matrix, activation matrix, and the activation function (\textit{e.g.}, ReLU) at the $\ell$-th layer, respectively. Note that $\mA_{0} = \mX$ where $\mX = \{ \vx_1,\dots, \vx_N \} \in  \R^{M \times N}$ is the input data matrix containing $N$ samples with input dimensionality $M$; $\mY = \{\vy_1,\dots, \vy_N \} \in \R^{C \times N}$ is the target matrix pair comprised of $N$ one-hot vector label of dimension $C$, representing number of prediction classes. Also, ${\bf r(.)}$ is the regularization term with (\textit{e.g.}, Frobenius norm $\|.\|_F^2$) corresponding penalty weight $\lambda_{\ell}$. Note that the regularization term can be simply ignored by setting $\lambda_\ell$ to zero. In this formulation, the intercept in each layer is ignored for simplicity as it can be simply be added by slightly modifying the $\mW_\ell$ and the input to each layer. The formulation in Eq. (\ref{eq:obj}) breaks the the conventional multi-layer backpropagation optimization of DNNs into simpler sub-problems that can be solved efficiently (e.g. reducing to least-squares problem). This also facilitates training in a distributed manner --- as the layers of the DNN are decoupled and the variables can be updated in parallel across layers ($\mW_\ell$) and data points (\ $\mW_\ell, \mZ_\ell, \mA_\ell$).



To enforce the constraints in problem (\ref{eq:obj}) and solve the optimization using ADMM, we would have the following augmented Lagrangian problem:

\begin{eqnarray} \label{eq:augmented}
	\minimize_{ \mathcal{W}, \mathcal{A}, \mathcal{Z}} \quad &\mathcal{J}\left(\mY, \mZ_{L} \right) + \sum_{\ell = 1}^{L} \lambda_{\ell}  {\bf r}_{\ell} (\mW_{\ell}) \\
	& + \sum_{\ell=1}^{L} \frac{\beta_{\ell}}{2} \| \mZ_{\ell} - \mW_{\ell} \mA_{\ell-1} + \mU_{\ell}\|_{F}^{2} \nonumber\\
	& + \sum_{\ell=1}^{L-1} \frac{\gamma_{\ell}}{2} \| \mA_{\ell} - {\bm \phi}_{\ell}(\mZ_{\ell}) + \mV_{\ell}\|_{F}^{2}\nonumber
\end{eqnarray}
where $\beta_{\ell}, \gamma_\ell >0$ are the step sizes, $\mU_{\ell}$ and $\mV_{\ell}$ are the \textit{(scaled) dual variables} \cite{boyd2011distributed} for the equality constraint at the layer $\ell$. 
Algorithm \ref{alg:admm} shows a standard ADMM scheme for optimizing Eq. (\ref{eq:augmented}). Note, the parameters are updated in a closed-form as analytical solution can be simply derived. For simplicity of the equations, we denote $\gP_\ell (.) = \frac{\beta_{\ell}}{2} \| \mZ_{\ell} - \mW_{\ell} \mA_{\ell-1} + \mU_{\ell}\|_{F}^{2} $ and $\gQ_\ell (.) = \frac{\gamma_{\ell}}{2} \| \mA_{\ell} - {\bm \phi}_{\ell}(\mZ_{\ell}) + \mV_{\ell}\|_{F}^{2}$. This algorithm is similar to \cite{taylor2016training,wang2019admm} with the difference that all the equality constraints in problem (\ref{eq:obj}) are enforced using multipliers, while previous work only enforced the constraints on the last layer $L$ while other constraints were only loosely enforced using quadratic penalty. 

\begin{algorithm}[htb]
  \caption{Standard ADMM for DNN Training}
  \label{alg:admm}
\begin{algorithmic}
  {\STATE \scalebox{1}{\bfseries Input:} data $\mX$, labels $\mY$}
  \STATE  \scalebox{1}{{\bfseries Params:} $\beta_\ell >0, \gamma_\ell >0,\lambda_\ell > 0$ }
  \STATE  \scalebox{0.8}{{\bfseries Initialize:} $\{\mW_\ell^0\}_{\ell=1}^{L}, \{ \mU_\ell^0\}_{\ell=1}^{L}, \{ \mV_\ell^0\}_{\ell=1}^{L-1}, \{\mZ^0_\ell\}_{\ell=1}^{L}, \{\mA^0_\ell\}_{\ell=1}^{L-1}\; k \leftarrow 0$ }
  \REPEAT
  \FOR{$\ell=1$ {\bfseries to} $L$}
  \STATE \scalebox{1}{$\mW_\ell^{k+1} \leftarrow \argmin\; \{ \gP_\ell (.) +  \lambda_{\ell}  {\bf r}_{\ell} (\mW_{\ell}^{k})\}$}
  \ENDFOR
  \FOR{$\ell=1$ {\bfseries to} $L-1$}
  \STATE \scalebox{1}{ $\mZ_\ell^{k+1} \leftarrow \argmin\; \{ \gP_\ell (.) +  \gQ_\ell (.) \}$ }
  \STATE \scalebox{1}{$\mA_\ell^{k+1} \leftarrow \argmin\; \{ \gP_{\ell+1} (.) +  \gQ_\ell (.) \} $}
  \ENDFOR
    \STATE \scalebox{1}{ $\mZ_{L}^{k+1} \leftarrow \argmin\; \{ \mathcal{J}\left(\mY, \mZ_{L}^{k} \right) + \gP_L (.) \}$ }
  \FOR{$\ell=1$ {\bfseries to} $L-1$}
  \STATE \scalebox{1}{$\mU_\ell^{k+1} \leftarrow \mU_\ell^{k} + \mZ_{\ell}^{k+1} - \mW_{\ell}^{k+1} \mA_{\ell-1}^{k+1}$}
  \STATE \scalebox{1}{$\mV_\ell^{k+1} \leftarrow \mV_\ell^{k} + \mA_{\ell}^{k+1} - {\bm \phi}_{\ell}(\mZ_{\ell}^{k+1})$}
  \ENDFOR
  \STATE \scalebox{1}{$\mU_L^{k+1} \leftarrow \mU_L^{k} + \mZ_{L}^{k+1} - \mW_{L}^{k+1} \mA_{L-1}^{k+1}$}
  \UNTIL{some stopping criterion is reached.}
\end{algorithmic}
\end{algorithm}


While the standard ADMM Algorithm \ref{alg:admm} has potentials in training (simple) DNNs \cite{taylor2016training}, there exists hurdles that confines extending ADMM to more complex problems --- the global convergence proof of the ADMM \cite{deng2016global} assumes that $\mathcal{J}$ is deterministic and the global solution is calculated at each iteration of the cyclic parameter updates.
% and during each iteration of the cyclic parameter updates, all the data samples are visited.
This makes standard ADMM computationally expensive thus impractical for training of many large-scale optimization problems. Specifically, for  deep learning, this would impose a severe restriction on training set size when limited computational resources are available. In addition, since the variable updates in standard ADMM are analytically driven, the extent of its applications is limit to trivial tasks \cite{taylor2016training}, making it incompetent to perform on par with the recent complex architectures introduced in deep learning (e.g. \cite{he2016deep}).


\section{Proof for Proposition 1}\label{sec:proof}

We follow the steps in the proof for similar problems in \cite{fu2018anchor} and \cite{shi2017penalty} with deterministic primal updates. Proper modifications are made to cover the stochastic primal update in our proof.


Note that we have
              \[     \nabla{\cal L}_{\rho_k}(\X^k)= \nabla f(\X^k) + \nabla h(\X^k)^T\bm \mu^k,          \]
              where 
              \[      \bm \mu^k = (1/\rho_k)h(\bm X^k)+\bm \lambda^k.   
              \]
              Our first step is to show that $\{\bm \mu^k\}$ is a convergent sequence. To see this, we define 
              \[ \bm \bar{\bm \mu}^k = \frac{\bm \mu^k}{\|{\bm \mu}^k\|}. \]
              Since $\bm \bar{\bm \mu}^k$ is bounded, it converges to a limit point $\bm \bar{\bm \mu}$. Also let $\x^\star$ be a limit point of $\x^k$.
              Because we have assumed that 
              $$\varepsilon_k\rightarrow 0,\quad \sigma_k^2\rightarrow 0,$$ 
              it means that the mean and variance of the stochastic gradient of our primal update goes to zero.
              Since our stochastic gradient is unbiased, we have
              \[       {\cal G}(\X^k) \rightarrow \nabla {\cal L}_{\rho_k}(\X^\star). \]  
              This also means that  we must have ${\cal G}(\x^k)\rightarrow \bm 0$ and $$\nabla L_{\rho_k}(\bm x^k)\rightarrow \bm 0.$$
     Hence, the following holds when $k\rightarrow \infty$:
              \begin{equation}\label{eq:approxkkt}
                 \nabla L_{\rho_k}(\bm X^\star)=\nabla f(\X^\star)+\nabla h(\X^\star)^T\bm {\bm \mu}^\infty = 0,
              \end{equation}           
               
               
              Suppose $\bm \mu^k$ is unbounded. By dividing \eqref{eq:approxkkt} by the above $\|\bm \mu^k\|$ and considering $k\rightarrow \infty$, we must have 
              \begin{equation}\label{eq:key}
                \nabla h(\X^\star)^T\bm \bar{\bm \mu}= 0,\quad \forall \X.    
              \end{equation}               
              The term $\nabla f(\bm X^\star)/\|\bm \mu\|$ is zero since we assumed $\bar{\bm \mu}$ is unbounded.
              Since $h(\bm X)=\bm 0$ satisfies the Robinson's condition, then, for any $\bm w$, there exists $\beta>0$ and $\bm x$ such that
              \[      \bm w = \beta \nabla h(\X^\star)(\X-\X^\star).        \]
              This together with \eqref{eq:key} says that $\bar{\bm \mu}=\bm 0$. This contradicts to the fact $\|\bar{\bm \mu}\|=1$. Hence, $\{ \bm \mu^k \}$ must be a bounded sequence and thus admits a limit point. Denote $\bm \mu^\star$ as this limit point, and take limit of both sides of \eqref{eq:approxkkt}. We have:
              \begin{equation}
              \nabla f(\X^\star)+\nabla h(\X^\star)^T\bm \mu^\star= \bm 0,\quad \forall \X.
              \end{equation}
               
              In addition, since $$\rho_k(\bm \mu^k-\bm \lambda^k) = h(\mathbf{\X^k})$$ with $\rho_k \rightarrow 0$ or $\bm \mu_k-\bm \lambda_k \rightarrow 0$ (per our updating rule and $\eta_k\rightarrow 0$), the constraints will be enforced in the limit.      $\mbox{     } \square$   \\
              

% \subsection*{{\uppercase\expandafter{\romannumeral D}. Supervised training on Fashion-Mnist}}\label{fmnist}


% To compare our method with dlADMM \citet{wang2019admm}, we evaluated the performance of our method on the Fashion-MNIST dataset \citep{xiao2017/online} with 60,000 training samples and 10,000 testing samples. We followed the settings in \citet{wang2019admm} by having 2 hidden layers with 1000 neurons each, and Cross-Entropy loss at the final layer. Also, the batch size is set to 128, $\beta_t = 1$, and the updates for $\mZ_t$ and $\Theta_t$ (eq. 6a) are performed 3 times at each epoch. Figure \ref{fig:fmnist_acc} shows the test set accuracy results over 200 epochs of training. It can be noticed that Stochastic Block ADMM is converging at lower epochs and reaching a higher test accuracy while performing efficient mini-batch updates. Further, in section C., it will be demonstrated that Stochastic Block ADMM converges drastically faster than dlADMM in terms of wall clock time.

   
% \begin{figure}[ht]
% \begin{center}
% \centerline{
% \includesvg[width=\columnwidth]{img/fmnist_acc.svg}
% }
% \caption{Test accuracy comparison of Stochastic Block ADMM and dlADMM \citep{wang2019admm} on Fashion-MNIST dataset using a network with 3 fully-connected layers: $784-1000-1000-10$. Final test accuracy: "Stochastic Block ADMM": $\bf 90.39\%$, "Wang \etal":$84.67 \%$ (averaged over 5 runs).}
% \vskip -0.25in
% \label{fig:fmnist_acc}
% \end{center}
% \end{figure}


\begin{figure}[ht]
\begin{center}
\centerline{
\includegraphics[width=\columnwidth]{imgs/fmnist_acc.pdf}
}
\caption{Test accuracy comparison of Stochastic Block ADMM and dlADMM on Fashion-MNIST dataset using a network with 3 fully-connected layers: $784-1000-1000-10$. Final test accuracy: "Stochastic Block ADMM": $\bf 90.39\%$, "Wang \textit{et al.}":$84.67 \%$ (averaged over 5 runs).}
% \vskip -0.25in
\label{fig:fmnist_acc}
\end{center}
\end{figure}




%----------------------------
\section{Supervised training of DNNs}\label{sec:sup_train}

\textbf{Fashion-MNIST.}
To compare our method with dlADMM \cite{wang2019admm}, we evaluated the performance of our method on the Fashion-MNIST dataset \cite{xiao2017/online} with 60,000 training samples and 10,000 testing samples. We followed the settings in \cite{wang2019admm} by having 2 hidden layers with 1000 neurons each, and Cross-Entropy loss at the final layer. Also, the batch size is set to 128, $\beta_t = 1$, and the updates for $\mZ_t$ and $\Theta_t$ (eq. 6a) are performed 3 times at each epoch. Figure \ref{fig:fmnist_acc} shows the test set accuracy results over 200 epochs of training. It can be noticed that Stochastic Block ADMM is converging at lower epochs and reaching a higher test accuracy while performing efficient mini-batch updates. Further, in section C., it will be demonstrated that Stochastic Block ADMM converges drastically faster than dlADMM in terms of wall clock time.



\textbf{CIFAR-10.}
The previous works on training deep netowrks using ADMM have been limited to trivial networks and datasets (e.g. MNIST) \cite{taylor2016training,wang2019admm}. However, our proposed method does not have many of the existing restrictions and assumptions in the network architecture, as in previous works do, and can easily be extended to train non-trivial applications. It is critical to validate stochastic block-ADMM in settings where deep and modern architectures such as deep residual networks, convolutional layers, cross-entropy loss function, etc., are used. To that end, we validate the ability of our method is a supervised setting (image classification) on the CIFAR-10 dataset \cite{cifar} using ResNet-18 \cite{he2016deep}. To best of our knowledge, this is the first attempt of using ADMM for training complex networks such as ResNets. 


For this purpose, we used 50,000 samples for training and the remaining 10,000 for evaluation. 
To have a fair comparison, we followed the configuration suggested in \cite{gotmare2018decoupling} by converting Resnet-18 network into two blocks $(T=2)$, with the splitting point located at the end of {\sc conv3\_x} layer. We used the Adam optimizer to update both the blocks and the decoupling variables with the learning rates of $\eta_t = 5e^{-3}$ and $\zeta_t = 0.5$. We noted since the auxiliary variables $\mZ_t$ are not "shared parameters" across data samples, they usually require a higher learning rate in Algorithm \ref{alg:blockadmm}. Also, we found the ADMM step size $\beta_t = 1$ to be sufficient for enforcing the block's coupling. 


Figure. \ref{fig:cifar} shows the results from our method compared with two baselines: \cite{gotmare2018decoupling}, and conventional end-to-end neural network training using back-propagation and SGD. Our algorithm consistently outperformed ~\cite{gotmare2018decoupling} however cannot match the conventional SGD results. There are several factors that we hypothesize that might have contributed to the performance difference: 1) in a ResNet the residual structure already partially solved the vanishing gradient problem, hence SGD/Adam performs significantly better than a fully-connected version; 
% 2) The common data augmentation in CIFAR will end up sending a different training example to the optimization algorithm at each iteration, which does not seem to affect SGD but seem to affect ADMM convergence somewhat; 
2) we noticed decreasing the learning rate for $\Theta_t$ updates does not impact the performance as it does for an end-to-end back-propagation using SGD. Still, we obtained the best performance of ADMM-type methods on both MNIST and CIFAR datasets, showing the promise of our approach.
% As illustrated, ADMM gets to a good performance fast and then slowly progress to higher accuracy..


%---------------------------- fig cifar  ------------------------------

\begin{figure}[htb]
% \vskip 0.15in
\begin{center}
\centerline{
\includegraphics[width=\columnwidth]{imgs/cifar.pdf}
}
% \vskip -0.05in
\caption{Test set accuracy on CIFAR-10 dataset. Final accuracy "Block ADMM": $89.66\%$, "Gotmare \etal":$87.12 \%$, "SGD": $\bf 92.70\%$. (Best viewed in color.)}
\label{fig:cifar}
\end{center}
% \vskip -0.2in
\end{figure}

 
 
 
% \subsection*{{\uppercase\expandafter{\romannumeral C}. Wall Clock Time Comparison}} \label{time_cmp}

% In this section, we setup a experiment to further analyse the efficiency of Stochastic Block ADMM and compare its training wall clock time against other baselines: \citet{gotmare2018decoupling,zeng2018global} (BCD), and \citet{wang2019admm} (ADMM). 
% For this purpose, we follow the similar settings as in section 4.1 for a supervised Deep Neural Network (DNN) training over MNIST dataset. Figure \ref{fig:time} shows the test set accuracy v.s. the training wall clock time from different methods. All the experiments are run on a machine with a single NVIDIA GeForce RTX 2080 Ti GPU. The methods are implemented in PyTorch framework -- except for dlADMM \citep{wang2019admm} that is implemented\footnote{code taken from \url{https://github.com/xianggebenben/dlADMM}} in "cupy", a NumPy-compatible matrix library accelerated by CUDA. \citet{gotmare2018decoupling} and Stochastic Block ADMM are trained with a mini-batch size of 128 and \citet{zeng2018global,wang2019admm} are trained in a batch setting. Note that in Figure \ref{fig:time}, the time recorded merely shows the \emph{training time} and excludes the time taken for initialization, data loading, etc. It can be observed that \citet{gotmare2018decoupling} and dlADMM are showing much slower convergence behaviors than Stochastic Block ADMM. We speculate that enforcing all the constraints by dual variables along with the efficient and cheap mini-batch updates in our method highly contributes to the convergence speed as well as its performance superiority over the other methods, including \citet{zeng2018global}.


% \begin{figure}[ht]
% \begin{center}
% \centerline{
% \includesvg[width=\columnwidth]{img/time_comparison.svg}
% }
% \caption{Test set accuracy v.s. training wall clock time comparison of different alternating optimization methods for training DNNs on MNIST dataset. Our method (blue) shows superior performance while presenting comparable convergence speed against \citet{zeng2018global} (green).}
% \vskip - 0.15in
% \label{fig:time}
% \end{center}
% \end{figure}


\begin{table*}[htb]
\caption{Prediction accuracy (\%) of individual attributes in LFWA dataset. DeepFacto with other weakly-supervised and supervised baselines.}
\label{table:attr_lfw}
\vskip 0.15in
\begin{center}
\begin{small}
\begin{sc}
\begin{tabular}{lcccccc}
\toprule
{Attributes} & \multicolumn{3}{c}{\small DeepFacto} & \small \cite{liu2015deep} & \small \cite{liu2018exploring} & \small \cite{zhang2014panda}\\
 {} & \multicolumn{3}{c}{\tiny (Weakly-Supervised)} & {\tiny (Weakly-Supervised)} & {\tiny (Supervised)} & {\tiny (Supervised)} \\
 {} & $r= $256 & 32 & 4 \\
\midrule
‘5 o Clock Shadow’ & 83.3 & 80.0 & 68.7 & 78.8 & \bf84 & \bf84\\
‘Arched Eyebrows’ & \bf86.6 & 83.9 & 79.2 & 78.1 & 82 & 79\\
‘Attractive’ & \bf84.3 & 79.8 & 73.3 & 79.2 & 83 & 81\\
‘Bags Under Eyes’ & \bf83.9 & 72.5 & 64.5 & 83.1 & 83 & 80 \\
‘Bald’ & \bf94.3 & 93.3 & 89.3 & 84.8 & 88 & 84\\
‘Bangs’ & \bf93.2 & 88.4 & 84.4 & 86.5 & 88 & 84\\
‘Big Lips’ & \bf83.2 & 77.0 & 71.9 & 75.2 & 75 & 73\\
‘Big Nose' & 80.1 & 68.7 & 61.4 & \bf81.3 & 81 & 79\\
‘Black Hair’ & \bf92.7 & 91.4 & 87.4 & 87.4 & 90 & 87\\
‘Blond Hair’ & \bf97.9 & 97.3 & 93.2 & 94.2 & 97 & 94\\
‘Blurry’ & \bf90.4 & 90.5 & 86.5 & 78.4 & 74 & 74\\
‘Brown Hair’ & \bf78.4 & 74.4 & 70.2 & 72.9 & 77 & 74\\
‘Bushy Eyebrows’ & \bf84.0 & 78.6 & 63.4 & 83.0 & 82 & 79\\
‘Chubby’ & \bf80.5 & 75.2 & 71.1 & 74.6 & 73 & 69\\
‘Double Chin’ & \bf86.0 & 77.9 & 72.3 & 80.2 & 78 & 75\\
‘Eyeglasses’ & 94.3 & 89.6 & 84.8 & 89.5 & \bf95 & 89\\
‘Goatee’ & \bf89.1 & 85.4 & 80.0 & 78.6 & 78 & 75\\
‘Gray Hair’ & \bf91.9 & 90 & 85.6 & 86.9 & 84 & 81\\
‘Heavy Makeup’ & \bf96.3 & 91.5 & 87.4 & 94.5 & 95 & 93\\
‘High Cheekbones’ & \bf90.4 & 79.0 & 72.1 & 88.8 & 88 & 86\\
‘Male’ & 81.3 & 76.6 & 70.5 & \bf94.3 & 94 & 92\\
‘Mouth Slightly Open’ & \bf85.4 & 78.0 & 73.3 & 81.7 & 82 & 78 \\
‘Mustache’ & \bf96.6 & 93.2 & 91.3 & 83.3 & 92 & 87\\
‘Narrow Eyes’ & \bf78.3 & 69.3 & 58.4 & 77.5 & 81 & 73\\
‘No Beard’ & \bf79.5 & 73.0 & 65.5 & 77.7 & 79 & 75\\
‘Oval Face’ & \bf80.6 & 73.2 & 66.1 & 78.7 & 74 & 72\\
‘Pale Skin’ & 75.1 & 66.7 & 60.6 & \bf89.8 & 84 & 84\\
‘Pointy Nose'& \bf81.6 & 73.7 & 62.2 & 79.8 & 80 & 76\\
‘Receding Hairline’ & 84.0 & 80.9 & 73.8 & \bf88.0 & 85 & 84 \\
‘Rosy Cheeks’ & \bf87.3 & 87.4 & 83.4 & 79.9 & 78 & 73\\
‘Sideburns’ & \bf85.4 & 81.5 & 75.8 & 80.5 & 77 & 76\\
‘Smiling’ & \bf92.6 & 78.7 & 69.8 & 92.2 & 91 & 89\\
‘Straight Hair’ & \bf82.8 & 77.0 & 72.1 &  73.6 & 76 & 73\\
‘Wavy Hair’ & 80.4 & 77.0 & 68.3 & \bf81.7 & 76 & 75\\
‘Wearing Earrings’ & \bf95.4 & 91.6 & 87.1 & 89.7 & 94 & 92\\
‘Wearing Hat’ & \bf93.0 & 90.2 & 87.0 & 80.5 & 88 & 82\\
‘Wearing Lipstick’ & \bf95.8 & 92.8 & 89.0 & 91.4 & 95 & 93\\
‘Wearing Necklace’ & \bf93.0 & 89.8 & 85.1 & 84.0 & 88 & 86\\
‘Wearing Necktie’ & \bf79.8 & 75.2 & 70.6 & 78.7 & 79 & 79\\
‘Young’ & \bf91.0 & 88.4 & 84.4 & 79.2 & 86 & 82\\
\midrule
Average & \bf87.0 & 81.4 & 74.8 &  83.1 & 84 & 81\\
\bottomrule
\end{tabular}
\end{sc}
\end{small}
\end{center}
\vskip -0.25in
\end{table*}




\section{Weakly Supervised Attribute Prediction}\label{sec:weakly_sup}


\subsection*{Factorizing the activations}\label{sec:factor_layer} 

With the assumption that the observations are formed by a linear combination of few basis vectors, one can approximate a given matrix $\mX \in \R^{m \times n}$ into a \textit{basis} matrix $\mM \in \R^{m \times r}$ and an \textit{score} matrix $\mS \in \R^{r \times n}$ such that $\mX \approx \mM \mS$ where $r$ is the (reduced) \textit{rank} of the factorized matrices -- commonly $r \ll \min(m, n)$.
Methods such as NMF would restrict the entries of $\mM$ and $\mS$ to be non-negative $(\forall i,j \;  \mM_{ij} \ge 0,\; \mS_{ij} \ge 0)$ which forces the decomposition to be only \textit{additive}. This has been shown to result in a parts-based representation that is intuitively more close to human perception. It is also worth mentioning that obviously, the matrix $\mX$ needs to be positive $({\forall i,j} \;  \mX_{ij} \ge 0)$. For non-negative factorization on the activations of the DNNS, due to the common use of activation functions such as \textit{ReLU}, this would not impose any constraints in most of the problems.

Activations of the CNN networks are generally tensors of the shape $\tZ_{\ell} \in \R^{(N, C, H, W)}$ which namely represent the batch size of the input, the number of the channels, the height of each channel, and the corresponding width. To adapt such tensors for the NMF problem, we reshape the tensor into the matrix $\mZ_{\ell} \in \R^{ C \times (N * H * W)}$ by stacking it over its channels while flattening the other dimensions. This way, the channels would be embedded into a pre-defined small dimension $r$ while keeping each sample and pixels information. For the weakly-supervised problem of attribute classification using DeepFacto, we attached the DeepFacto module to the last convolutional layer of the Inception-Resnet-V1 architecture followed by a \emph{ReLU}. This layer has 1792 channels and, for a given input of the size $160 \times 160$ pixels (the original input size from the LFWA dataset), the height and the width are both equal to 3. 

\begin{figure}[htb]
\vskip -0.05in
\begin{center}
\centerline{
\includegraphics[width=\columnwidth]{imgs/heatmap.jpg}
}
\caption{Heat map visualizations from three different dimensions of the score matrix $\mS$ (rows) trained by DeepFacto-32 over different samples (columns) in LFWA dataset. These dimensions can capture interpretable representations over different faces identities: \emph{eyes} (top), \emph{forehead} (middle), and \emph{nose} (bottom).}
\label{fig:heatmap}
\end{center}
\vskip -0.15in
\end{figure}

% Table \ref{table:attr_lfw} shows the prediction accuracy of each attribute in LFWA dataset and compares DeepFacto with different ranks ($r=4,32,256$) against other supervised and weakly-supervised baselines. It can be noted that our method can generate highly informative representation of the LFWA attributes without accessing their labels. This supports our conjecture that DeepFacto, by non-negatively factorizing the activations of the DNNs in and end-to-end training, can lead to an interpretable decomposition of the DNN activations.



\subsection*{Heat maps}\label{sec:heatmap}
To qualitatively investigate the interpretability of the factorized representations learned from DeepFacto, similar to \cite{collins2018deep}, one can visualize the score matrix $\mS$. Each dimension of the score matrix $\mS$ can be reshaped back to the original activation size and be up-sampled to the size of the input using bi-linear interpolation. In Figure \ref{fig:heatmap}, the score matrix learned form the DeepFacto with $r=32$ (average attribute prediction of 81.4\%) is used where three different heat maps (out of 32) are depicted over different samples from LFWA dataset. We have found $r=4$ to be very low to represent interpretable heat maps for the attributes and $r=256$ to contain redundant heat maps. It can be seen, that the heat maps can show local and persistent attention over different face identities: \emph{eyes}, \emph{forehead}, \emph{nose}, etc.





\end{document}


