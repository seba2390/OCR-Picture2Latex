% version prd_rc_v2

\documentclass[aps,prl,twocolumn,a4paper]{revtex4}


\usepackage{amsmath,amssymb} 
\usepackage{bm} %
 
\usepackage{color} %
\usepackage{graphicx}



\usepackage[colorlinks]{hyperref} %
\hypersetup{
citecolor=blue,
linkcolor=blue,
urlcolor=blue
}


\usepackage{epstopdf}
\epstopdfsetup{suffix={}}


\usepackage{multirow}


\usepackage{comment}

%\newcommand{\note}[1]{\textcolor{red}{#1}}{\ignorespacesafterend}
\newcommand{\note}[1]{#1}{\ignorespacesafterend}
\newenvironment{notered}{\color{red}}{\ignorespacesafterend}
\newenvironment{noteblue}{\color{blue}} {\ignorespacesafterend}

\newcommand{\mnote}[1]{\marginpar{\textcolor{red}{#1}}}

\newcommand{\fnote}[1]{\footnote{\textcolor{red}{#1}}\marginpar{\textcolor{red}{\tiny footnote}}}

\newcommand{\chk}{\textcolor{red}{\bf{?} }\marginpar{\textcolor{red}{***}}}
\newcommand{\mk}[1]{{\bf [ #1 ]\\}}




\newcommand{\Sec}[1]{Sec.~\ref{#1}}  %
\renewcommand{\figurename}{FIG.}
\newcommand{\Fig}[1]{Fig.~\ref{#1}}
\newcommand{\fig}[1]{Fig.~\ref{#1}}
\renewcommand{\tablename}{TAB.} %
\newcommand{\Tab}[1]{Tab.~\ref{#1}}
\newcommand{\tab}[1]{Tab.~\ref{#1}}
\newcommand{\Eq}[1]{Eq.~(\ref{#1})}
\newcommand{\Eqs}[1]{Eqs.~(\ref{#1})}
\newcommand{\eq}[1]{Eq.~(\ref{#1})}
\newcommand{\eqs}[1]{Eqs.~(\ref{#1})}

\newcommand{\tx}[1]{\text{#1}}
\newcommand{\vrel}{v_{\text{rel}}}	
\newcommand{\sigmav}{\langle \sigma v \rangle}
\newcommand{\rsun}{r_{\odot}}
\newcommand{\VEV}[1]{\left\langle #1\right\rangle}
\newcommand{\ekin}{E_{\text{kin}}}
\newcommand{\pbar}{\bar p}

\listfiles




\begin{document}
\title{Origins of  sharp cosmic-ray electron  structures and the DAMPE excess}
\author{Xian-Jun Huang$^{a,b}$}
\author{Yue-Liang Wu$^{a}$}
\author{Wei-Hong Zhang$^{a}$}
\author{Yu-Feng Zhou$^{a}$}
\affiliation{$^{a}$
	CAS Key Laboratory of Theoretical Physics, 
	Institute of Theoretical Physics, Chinese Academy of Sciences,
	%ZhongGuanCun East Rd.55, 
	Beijing, 100190, China,
	}
\affiliation{
	University of Chinese Academy of Sciences, 
	%Yuquan Rd.19A, 
	Beijing 100049, China,
}
\affiliation{$^{b}$
 Sichuan University of Science and Engineering, Zigong 643000, China.}
\begin{abstract}
Nearby sources  may contribute to cosmic-ray electron  (CRE) structures 
at high energies.
Recently, the first  DAMPE results on the CRE flux hinted at 
a  narrow  excess 
at   energy $\sim 1.4$~TeV.  
We show that in general a  spectral structure with a narrow width 
appears  in two scenarios:
I) ``{\it Spectrum broadening}'' for the continuous sources 
with a $\delta$-function-like  injection spectrum.
In this scenario, %
a finite width can develop after propagation through the Galaxy,
which
can reveal %
the distance of the source.
Well-motivated   sources include mini-spikes and subhalos formed 
by dark matter (DM) particles $\chi_{s}$ which annihilate directly into $e^{+}e^{-}$ pairs.
II) ``{\it Phase-space shrinking}'' for burst-like sources with a power-law-like injection spectrum.
The  spectrum after propagation can shrink at a cooling-related cutoff energy 
and form a sharp spectral peak.
The peak can be  more prominent due to the energy-dependent diffusion.
In this scenario, the width of the excess constrains both 
the power index and the distance of the source.
Possible such sources are pulsar wind nebulae (PWNe) and supernova remnants (SNRs).
We analysis the DAMPE excess and find that 
the continuous DM sources should be fairly close within $\sim 0.3$~kpc, 
and the annihilation cross sections are close to 
the thermal value.
For the burst-like source, 
the narrow width of the excess suggests that 
the injection spectrum must be  hard 
with power index significantly less than two, 
the distance is within $\sim(3-4)$~kpc, and
the age of the source is  $\sim 0.16$~Myr.
In both scenarios,  
large  anisotropies  in the CRE flux are predicted. 
We identify possible candidates of mini-spike and PWN sources
in the current Fermi-LAT 3FGL and ATNF catalog, respectively.
The diffuse $\gamma$-rays from these sources
can be well below the Galactic diffuse $\gamma$-ray backgrounds
and less constrained by the Ferm-LAT data,
if they are located at the low Galactic latitude regions.	
\end{abstract}
\pacs{xx.xx}
\preprint{ [\today ]}
\maketitle %



\paragraph{\bf Introduction.}
Cosmic-ray (CR)  electrons and positrons (CREs)
with energies above TeV  plays an important role in understanding 
the nearby origins of CRs within a few kpc
\cite{Shen:1970apj}.
%
Structures in the energy spectrum of CREs are expected, 
if the nearby sources are dominated by one or a few discrete sources.
%
The current space experiments have begun to directly probe  this energy region.
For instance, 
the AMS-02~\cite{Aguilar:2014fea}, 
Fermi-LAT~\cite{Abdollahi:2017nat} and 
CALET~\cite{Adriani:2017efm} experiments have measured 
the  flux of CRE up to 1, 2 and 3~TeV, respectively,  
without observing any significant structures so far.
%\cite{
%	Aguilar:2014fea,%
%	Abdollahi:2017nat,%
%	Adriani:2017efm%
%}.
Recently, 
the DAMPE experiment has reported  the first high energy resolution 
measurement of  the CRE flux up to  4.6~TeV
\cite{Chang:2017xx}.
The measured energy spectrum of CRE 
steepened above $\sim0.9$~TeV,
consistent with the  results from 
the ground-based atmospheric Cherenkov telescopes
\cite{
	Aharonian:2008aa,%
	Aharonian:2009ah,%
	HESS:2017ICRC,%
	BorlaTridon:2011dk,%
	Staszak:2015kza%
}.
Of interest, the DAMPE data  also hinted at an excess over  
the expected background  in a narrow energy interval  $\sim(1.3-1.5)$~TeV.
Making use of the  DAMPE data in the energy range 55~GeV--4.6~TeV, and 
assuming  a broken power-law background flux,
we find that the local and global significance of the possible narrow excess is
$\sim 3.7~\sigma$ and $\sim2.5~\sigma$, respectively
(details of the data analysis are shown in the supplementary material).
%we find that the  background-only fit  gives
%$\chi^{2}=25.9$,
%while the scenario of background plus a Gaussian signal leads to 
%$\chi^{2}=12.2$,
%indicating a local significance of  $\sim 3.7~\sigma$.
%



%An  excess with a narrow width, as hinted by DAMPE,
%may shed light on the nature of the nearby sources.
\note{
In light of the possible DAMPE ``excess'',
it is of general interest to address the question of 
what kind of sources are responsible for a sharp
spectral feature in CRE flux.}
In this work, 
we explore the origins of a sharp spectral structure and 
emphasize  that  the space-time location of  source can be inferred from 
the spectral feature of the CRE flux.
We show that in general a sharp spectral structure can be produced   in two  complementary scenarios:
I)
for  continuous sources with a line-shape injection spectrum, 
a  finite width can develop after propagation in the Galaxy
(dubbed ``spectrum broadening'').
Well-motivated  sources are nearby DM substructures 
such as mini-spikes and DM subhalos of DM particles $\chi_{s}$
with  $e^{+}e^{-}$ the dominant annihilation final states.
In this scenario, the spectral shape or the width of the excess can be used to estimate  
the distance to the source.
II)
for burst-like sources with a power-law injection spectrum,
the spectrum after propagation can shrink at a cooling-related cutoff energy 
and form a sharp spectral peak  (dubbed ``phase-space shrinking'').
Energy-dependent diffusion also contributes  to the spectral rising. 
%if the distance of the sources exceeds  the diffusion length. 
Typical  sources of this type are pulsar wind nebulae (PWNe) and 
supernova remnants (SNRs).
In this scenario, the power index and the distance of the source are strongly 
constrained by the width of the excess.

%We analyses these generic scenarios for the possible DAMPE excess and 
%show that:
In view of the DAMPE excess, we find: 
\romannumeral 1)
for the continuous sources, 
the favoured distance should be less than $\sim 0.3$~kpc.
The source can be  ``mini-spikes'' or DM subhalos  with 
the favoured DM annihilation cross section around the typical thermal value.
\romannumeral 2)
For the burst-like source, 
the injection spectrum must be hard with power index significantly below two, and
the distance within $\sim(3-4)$~kpc. 
The age of the source is determined to be $\sim0.16$~Myr.
\romannumeral 3) 
For both sources,
large anisotropies in the arrival direction of the CRE are predicted,
which are close to the current Ferm-LAT upper limits.
\romannumeral 4) 
We identify possible candidates for  mini-spikes (PWNe)
from the catalogue of Fermi-LAT  3GFL 
%unassociated point-sources 
(ATNF pulsar catalogue).
%\romannumeral 5) 
%The sources are likely to be located in the low-latitude regions
%where the background of Galactic diffuse $\gamma$-rays is relatively large,
%which evades  the stringent constraints from the Fermi-LAT data.
\begin{figure}[thb]
	\centering
	\includegraphics[width=0.48\columnwidth]{flux_pointDM_rDep.eps}
	\includegraphics[width=0.48\columnwidth]{flux_SNR_rDep.eps}
	\caption{		
		Left) 
		Effect of  ``spectrum broadening'' on CRE flux from 
		a continuous point-like source described in 
		\eq{eq:continuous-point-solution} with a growing distance $r=0.1-1.0$~kpc.		 
		$E_{0}$ is fixed at 1.4~TeV.
		Right)
		Effect of ``phase-space shrinking'' for the burst-like sources in 
		\eq{eq:burst--solution} with  a decreasing $\alpha$ from 2.0 to 0.7 and fixed $r=1$~kpc,
		and ``energy-dependent diffusion''  for  a growing  distance $r=1-3$~kpc with fixed $\alpha=2$.
		The age of the source is  fixed at $t=0.15$~Myr.
		For both sources, the spectra are normalized to a total flux
		$\Phi=10^{-9}\mbox{m}^{-2}\mbox{s}^{-1}\mbox{sr}^{-1}$.
	}\label{fig:spectral-feature}
\end{figure}
\begin{figure*}[htb]
	\centering
	\includegraphics[width=0.95\textwidth]{flux_log.eps}
	\caption{
		Best-fit CRE flux from fitting to the DAMPE data~\cite{Chang:2017xx}  
		for three type of sources.
		Left) Continuous point-like sources (mini-spikes) with distance $r$=0.1, 0.2, 0.3~kpc, 				respectively.
		Center) Continuous extended sources (DM subhalos) for the same distances with subhalo
		mass fixed at $10^{7}M_{\odot}$.
		Right) Burst-like sources (PWNe/SNRs) with $(r (\tx{kpc}),\alpha)$ values 
		(1, 0.5), 
		(2, 0.7), 
		(3, 1.3),			
		respectively.	
		The solid (dashed) curves are  the sum of the signal and background (signal only).
		The data of  DAMPE~\cite{Chang:2017xx},
		AMS-02~\cite{Aguilar:2014fea} and
		Ferm-LAT~\cite{Abdollahi:2017nat} 
		are also shown.
	}\label{fig:flux}  
\end{figure*}

\paragraph{\bf Continuous  sources.}
The propagation of  CR electrons  is described by  
the following diffusion equation
\cite{
	Ginzburg%
}
\begin{align}\label{eq:propagation}
\frac{\partial f}{\partial t}  =
\frac{D(E)}{r^{2}}\frac{\partial}{\partial r}r^{2}\frac{\partial}{\partial r} f
+\frac{\partial}{\partial E}\left(B(E)f\right)
+Q(r,t,E),
\end{align}
where $f(r,t,E)$ is the number density function per unit energy
with $r$ the distance to the source,
$Q(r,t,E)$ is the source term.
In the  equation we have neglected the effects of 
convection and re-acceleration
as they are only important at low energies. %
The energy-dependent spatial diffusion coefficient $D(E)$  is parametrized as
$D(E)=D_{0}(E/\mbox{GeV})^{\delta}$,
where $\delta=0.31$ is a power law index and  
$D_{0}=5.49\times 10^{28}\mbox{ cm}^{2}\mbox{s}^{-1}$ %
\cite{
	Trotta:2010mx%
}. 
The energy-losses due to ICS processes and synchrotron radiations  are prametrized as 
$B(E)=b_{0} E^{2}$ with
$b_{0}=1.4\times10^{-16}~\mbox{GeV}^{-1}\mbox{s}^{-1}$
\cite{
	Linden:2013mqa%
}.





The sources of CRE can be roughly divided into 
continuous  and burst-like sources,
according to the time scale of  electron injection from the sources relative to that of  the propagation time.
A possible continuous source  is the annihilation of DM particles in the Galaxy.
Very sharp $\delta$-function-like CRE spectrum can be produced from
DM annihilation directly into  $e^{+}e^{-}$ pairs 
%%
\note{in models with  enhanced DM-electron coupling,
or  through light mediators with mass very close
to twice the electron mass.
}
%%
However, for any continuous source, the finally observed spectrum is
a superposition of electrons injected at different time.
The electrons injected earlier suffer from more energy losses.
Thus the superposition inevitably results in the broadening of  the spectrum.
For a continuous point-like source
with a $\delta$-function injection spectrum
$Q(r,E) \approx Q_{0}\delta(E-E_{0})\delta^{(3)}(\mathbf{r})$
with $E_{0}$ the central energy and  $Q_{0}$ the normalization constant,
the analytic solution to  \eq{eq:propagation} is  given by
\cite{
	Atoian:1995ux%
}
\begin{align}\label{eq:continuous-point-solution}
f(r,E)=\frac{Q_{0} E^{-2}}{\pi^{3/2}b_{0} 
	r^{3}_{d}(E)}\exp\left(-\frac{r^{2}}{r^{2}_{d}(E)}\right)  ,
\end{align}
where 
$r^{2}_{d}(E)=4 D_{0}
[(E/\mbox{GeV})^{\delta-1}-(E_{0}/\mbox{GeV})^{\delta-1}]/(1-\delta) (b_{0}\text{GeV})$ 
is the diffusion length.
As the solution shows,  after the propagation, 
the  spectrum is broadened.
For $E\ll E_{0}$, it is  
an approximate power law  $f\propto E^{-(1+3\delta)/2}$.
The spectrum rises rapidly when $E$ is approaching $E_{0}$ and 
eventually  cut off exponentially at $E_{0}$ as $r_{d}(E)\approx 0$.
In the left panel of \fig{fig:spectral-feature}, we show how the 
spectral shape of CRE changes with growing distance $r$. 
In the region near  $E\approx E_{0}$  
the spectral shape is very sensitive to the distance.
Increasing the  distance will  result in a broader excess. 
Therefore,  a precision measurement on the spectral shape can be used to 
determine the distance. 
Note that the diffusion length $r_{d}$ 
can only set the scale of the maximal distance.

In the left panel of \fig{fig:flux},
we show the best-fit fluxes obtained from fitting to the DAMPE data 
for three fixed values of  $r=0.1-0.3$~kpc. 
Other parameters such as $E_{0}$, $Q_{0}$ and the background parameters
are allowed to vary in the fits.
In all the  three cases, the best-fit values of $E_{0}$ are quite similar 
$E_{0}\approx1.4-1.5$~TeV. 
With increasing value of $r$, the best-fit spectrum becomes broader and  
the fit  quality becomes lower.
From $r=0.1$ to 0.3~kpc, the $\chi^{2}$-value increases from  $14.2$ to $19.2$.
The fit including  $r$ as a free parameter shows  that 
the DAMPE data place an upper limit of 
$r \lesssim 0.3$~kpc at $95\%$~C.L.. 
For the three cases, the best-fit values of the normalization constants are $Q_{0}$=$(0.47-2.1)\times 10^{33}~\mbox{s}^{-1}$, respectively.
The detailed list of best-fit parameters and allowed regions are shown 
in Tab.~S-2 and Fig.~S-3
of the supplementary material. 

\paragraph{Mini-spikes.}
One of the possible continuous point sources is the ``mini-spike", 
i.e., the large DM density enhancements around the
intermediate mass black holes (IMBHs) with mass $\sim 10^{2}-10^{6} M_{\odot}$
\cite{
	Miller:2003sc,%
	Zhao:2005zr,%
	Bertone:2005xz,%
	Bertone:2009kj%
}.
The IMBH can form 
out of popIII stars
\cite{
	Heger:2002by%
} 
or collapsing of primordial gas in early-forming halos 
\cite{
	Koushiappas:2003zn%
}.
In this letter we consider the latter case of IMBH formation.
For the Milky Way-sized Galaxy, 
the total number of this type of IMBHs is around $\mathcal{O}(100)$
with $\sim 30\%$ of them located in the inner region $ \lesssim10$~kpc
\cite{
	Bertone:2005xz%
}.
Starting from an initial NFW DM profile~\cite{Navarro:1996gj}, 
the spiked DM profile of the mini-spike after the adiabatic growth
of the IMBH follows a power law
$\rho_{sp}(r)=\rho(r_{sp})(r/r_{sp})^{-\gamma_{sp}}$
where 
$\rho(r)$ is the initial DM profile,
$r_{sp}\sim \text{pc}$ is the typical radius of the mini-spike and 
$\gamma_{sp}\approx 7/3$ is the power index%
~\cite{
	Gondolo:1999ef%
}.
Due to the DM annihilation, the spiked DM profile is cut off at a very small distance  $r_{\text{cut}}\sim 10^{-3}$~pc
\cite{
	Bertone:2005xz%
}.
Assuming Majorana DM particles $\chi_{s}$ which annihilate dominantly 
into $e^{+}e^{-}$ pairs  with velocity-averaged  annihilation cross section 
$\langle \sigma v \rangle$, 
the source term of ``mini-spikes" can be estimated as 
\begin{align}\label{eq:Q0}
Q_{0}\approx
&3.1\times 10^{33}~\mbox{s}^{-1}
\left( \frac{\langle\sigma v\rangle}{3\cdot 10^{-26}~\mbox{cm}^{3}\mbox{s}^{-1}} \right) 
\left( \frac{1.4~\mbox{TeV}}{m_{\chi}} \right)^{2}
\nonumber\\
&\left(\frac{\rho(r_{sp})}{10^{2}~\mbox{GeV}\mbox{cm}^{-3}} \right)^{2}
\left(\frac{r_{sp}}{\mbox{pc}} \right)^{14/3}
\left( \frac{r_{\text{cut}}}{10^{-3}~\mbox{pc}}\right)^{-5/3}     .
\end{align}
For the three cases of 
$r=0.1-0.3$~kpc, using the best-fit values of $Q_{0}$, the corresponding cross section are
%$\langle \sigma v \rangle=(0.48,\ 1.36,\ \text{and}\ 2.48)\times 10^{-26}\text{cm}^{3}\text{s}^{-1}$, respectively, which are close to the typical thermal value.
$\langle \sigma v \rangle=(0.48-2.48)\times 10^{-26}\text{cm}^{3}\text{s}^{-1}$, 
which are close to the typical thermal value.



\paragraph{DM subhalos.}
It is straightforward to extend the analysis to 
the  spatially-extended sources.
N-body simulations of cold and collisionless  DM predict that 
the Galaxy  should contain large number of small subhalos
%
~\cite{
	Springel:2008cc,%
	Diemand:2008in,%
	Garrison-Kimmel:2013eoa%
}.
%
\note{
Based on a joint analysis to the Via Lactea II and  ELVIS simulations
\cite{Hooper:2016cld}, 
we estimate that the possibility of finding a nearby subhalo  
within $\lesssim 1 (0.3)$~kpc and total mass 
$M_{h} \gtrsim10^{6} M_{\odot}$ is around 
$\sim 1.2\% (0.03\%)$. 
%
An alternative possibility is  the ultra compact mini-halos (UCMHs) formed  in the early epochs of the Universe~\cite{
%% proposal of UCMH
Ricotti:2009bs,%% 0908.0735  Astrophys. J. 707 979-987 (2009)
%% Ricotti,Gould,
%% A New Probe of Dark Matter and High-Energy Universe Using Microlensing
%%
%% prediction for gamma-rays
Scott:2009tu,%% 0908.4082  Phys. Rev. Lett. 103 211301 (2009)
%% Scott,Sivertsson,
%% Gamma-Rays from Ultracompact Primordial Dark Matter Minihalos
%%
%%
Josan:2010vn,%% 1006.4970  Phys. Rev. D82 083527 (2010)
%% Josan,Green,
%% Gamma-rays from ultracompact minihalos: potential constraints on the primordial curvature perturbation
%%
%%
Bringmann:2011ut%% 1110.2484  Phys. Rev. D85 125027 (2012)
%% Bringmann,Scott,Akrami,
%% Improved constraints on the primordial power spectrum at small scales from ultracompact minihalos
%%
}.
Finding a nearby UCMH within $\sim0.1$~kpc requires
that UCMHs contribute to a fraction of  above $\sim1\%$~ of the total DM density~\cite{Scott:2009tu}.
%
In this work, we shall focus on the case of  DM subhalo. %with 
%mass $M_{h}$ and at a distance $r$.
%We consider the possibility that the source is  a nearby DM %subhalo 
%(see ~\cite{X} for UCMH explation of DAMPE data).
}
As subhalos may experience a significant degree of mass loss due to tidal stripping,
especially for those located at  the inner volume of the Galaxy,
we  adopt a tidally truncated  density profile
$\rho(r)=\rho_{0}(r/\text{kpc})^{-\gamma}\exp(-r/R_{b})$
\cite{
	Kazantzidis:2003hb,%
	Penarrubia:2007zx,%
	Hooper:2016cld%
}.
The parameters  $\rho_{0}$, $\gamma$ and $R_{b}$  depend on 
the distance  $d$ from the center of the subhalo to the Galactic center
and 
the total mass $M_{h}$ of the subhalo, which  
can be extrapolated from the N-body simulation data.
%Making use of the results of a joint analysis to the Via Lactea II %and  ELVIS simulations
From the analysis in Ref.
\cite{
	Hooper:2016cld%
},
we obtain $\rho_{0}=5.3~\text{GeV}\cdot\text{cm}^{-3}$, $\gamma=0.78$ and $R_{b}=0.096$~kpc, 
for a typical $M_{h}=10^{7} M_{\odot}$.

The CRE flux from the subhalo is calculated by integrating the solution of 
\eq{eq:continuous-point-solution} over the subhalo density distribution.
For $r=0.1$, 0.2, and 0.3~kpc,
we find  the best-fit annihilation cross sections 
$(1.04, \ 2.55,\  4.62)\times 10^{-26}~\mbox{cm}^{3}\text{s}^{-1}$, respectively,
which are well below  the current limits on DM subhalos
\cite{
	Fermi-LAT:2016uux,%
	Schoonenberg:2016aml,%
	Hooper:2016cld%
}.
Similar to the case of point source, 
when $r$ increases from 0.1 to 0.3~kpc, 
the best-fit spectrum becomes broader and the fit qualities become lower.
The corresponding $\chi^{2}$ value increases from 13.4 to  19.2.
From a fit with $r$ as a free parameter, 
we find that  the source should be within $r \lesssim 0.3$~kpc at $95\%$~C.L., 
very close to the case of mini-spikes.
We also find that modifying the subhalo mass $M_{h}$ 
does not change the conclusion.
%
The detailed list of best-fit parameters and allowed regions
are shown in Tab.~S-4 and Fig.~S-4
of the supplementary material. 
%
Note that 
for  an  annihilation cross section of 
$\mathcal{O}(10^{-26})~\tx{cm}^{3}\tx{s}^{-1}$ %
the contribution from the whole Galactic halo DM 
to the CRE flux  is typically two orders of magnitude  smaller than 
that from the nearby sources,
which can be safely neglected.
%
\note{
The halo DM also contribute to extra CR positrons which 
could be constrained by the experiments.
The constraints, however, turn out to be rather weak for 
TeV scale DM, 
as the current AMS-02  experiment only measured the 
positrions up to 
$\sim 350$~GeV~\cite{Accardo:2014lma}
%
(see Fig.~S-6 in the supplementary material).
%for the Galactic halo DM contributions
%to CRE and CR positrons).
}






\paragraph{\bf Burst-like  sources.}


For burst-like sources such as PWNe/SNRs,  
the injection spectrum is expected to be
a broad power-law with an exponential cutoff,
$Q(r,t,E)=N_{0} (E/\mbox{GeV})^{-\alpha} 
\exp(-E/E_{c}) \delta^{(3)}(\mathbf{r})\delta(t)$, 
where $\alpha$ is the power-law index and
$E_{c}$ is the cutoff energy.
The normalization constant $N_{0}$ can be  related to 
the total released energy  $E_{\tx{tot}}$.
%as $E_{\text{tot}}=\int E Q(r,t,E) dE d^{3}r dt$.
The solution to the diffusion equation~\eq{eq:propagation} 
for this type of source is given by~\cite{Atoian:1995ux}
\begin{align}\label{eq:burst--solution}
f(r,E)=\frac{N_{0}
	(E/\text{GeV})^{-\alpha}}{\pi^{3/2}r_{d}^{3}} 
\xi(E)^{\alpha-2}
\exp\left(-\frac{r^{2}}{r_{d}^{2}}-\frac{E}{\xi(E) E_{c}}\right) ,
\end{align}
where $\xi(E)=1-E/E_{\text{max}}$ 
with $E_{\text{max}}=(b_{0}t)^{-1}$  the maximal possible energy of 
an electron from a source of age~$t$.
The diffusion length for this type of source is
$r_{d}(E) =2\sqrt{ \lambda(E) D(E)t}$,
where $\lambda(E)=[1-\xi(E)^{1-\delta}]/(1-\delta)(1-\xi(E))$.
For $E\ll E_{\text{max}}$, $r_{d}(E) \approx 2\sqrt{D(E)t}$.
While the value of $\alpha$ is commonly considered to be $\sim 2$,
the cutoff $E_{c}$ is poorly constrained.
We shall focus on the large cutoff limit, 
i.e. $E_{c}\gg E_{\text{max}}$.
In this case,
$E_{\text{max}}$ will play the role of spectral cutoff 
instead of $E_{c}$, namely,
a cooling cutoff will appear,
as can be seen in \eq{eq:burst--solution}.




In this work, we emphasize that for the burst-like source 
a sharp spectral rise near the cutoff energy $E_{\text{max}}$
can appear for some choices of $\alpha$ and $r$
for two reasons:
{\bf i)}  Cooling related ``phase-space shrinking''.
An initial  electron with energy $E_{s}$ at time $t=0$
is related to its energy $E$ observed at later time $t$ as $E_{s}=E/\xi(E)$.
Thus an initial energy interval  $\Delta E_{s}$ will shrink to 
$\Delta E=\xi(E)^{2} \Delta E_{s}$ at time $t$.
Since the number of electrons  is unchanged during cooling,
the energy spectrum at time $t$ is  
$\Delta N/\Delta E \approx E^{-\alpha}\xi(E)^{\alpha-2}$.
For a relatively hard spectrum with  $\alpha <2$, 
the shrinking of phase space  can enhance the number density.
Since $\xi(E)$ vanishes when $E$ is approaching $E_{\text{max}}$,
the shrinking of phase space leads 
to a rapid rise of the energy spectrum.
This effect of ``phase-space shrinking'' is illustrated  in
the right panel of \fig{fig:spectral-feature}.
%where the $\alpha$ dependence of the electron spectrum is shown 
%for a fixed distance $r$.
The DAMPE data suggest that the cutoff  should be 
in the range $E_{\text{max}}\approx 1.3-1.5$~TeV,
which in turn sets the  age of the source
$t  %
\approx  (0.15-0.17)~\text{Myr}$,
and the diffusion length at $E_{\text{max}}$, 
$r_{d}(E_{\text{max}}) \approx (0.79-0.84)~\text{kpc}$.
{\bf ii)}``Energy-dependent diffusion'' which is related to 
the fact that for $\delta>0$,
higher energy electrons have larger diffusion coefficients.
The energy dependence in the exponential factor
of \eq{eq:burst--solution} can be written as
$\exp(-r^2/r_d^2)\approx \exp[- \kappa^{2}(E/E_{\text{max}})^{-\delta}]$
where $\kappa=r^{2}/r_{d}^{2}(E_{\text{max}})$.
For relatively  large distance  $\kappa >1$, 
the energy-dependent factor also contributes to the rising of the spectrum near $E_{\text{max}}$,
which  is illustrated in  \fig{fig:spectral-feature}.
Of course, in order to  compensate the exponential suppression of the flux at 
large $\kappa$, 
the  normalization constant $N_{0}$ 
or $E_{\text{tot}}$ has to be large enough.
The reasonable value of $E_{\text{tot}}$ should be smaller than 
the typical kinetic energy carried by SNR 
%of  $\sim10^{51}$~erg
or the total energy of supernova explosion of  
$\sim(10^{51}-10^{53})$~erg,
which sets the scale of the distance of the sources.






In general, unconventional values of  $\alpha$ 
which is  significantly smaller than $\sim 2$ is required to 
reproduce the DAMPE excess, 
especially for small $r$.
In the right panel of  \fig{fig:flux}  we show the best-fit spectra for  
three typical combinations of  $r$ (in kpc) and $\alpha$ with 
$(r,\alpha)$=(1, 0.5), (2, 0.7) and (3, 1.3), respectively.
A scan in the $(r,\alpha)$ parameter space shows that in the region 
$r<r_{d}$, the allowed $\alpha$ has to be very small $\alpha \lesssim0.65$.
In the region $r>r_{d}$ the value of $\alpha$ can reach at most $1.4$ at 3~kpc,
as the effect of  energy-dependent diffusion is  significant.
By imposing the condition of $E_{\text{tot}}<10^{51}(10^{53})$~erg,
the distance $r$ is restricted in the range $r\lesssim$3(4)~kpc.
Together with the required  age of the source,
we find 7 candidate pulsars with $r \lesssim 4$~kpc
in the ATNF catalog of pulsars~\cite{Manchester:2004bp}:
B0740-28, J0922-4949,   J1055-6022,  J1151-6108,
J1509-5850,  J1616-5017 and J1739-3023.
Distances of the sources lie in the range (2.0--3.6)~kpc.
In this region, both the effects of ``phase-space shrinking '' and 
``energy-dependent diffusion'' are relavant.
The detailed list of best-fit parameters, allowed regions
and the list of  the candidate pulsars are shown in
Tab.~S-6, Fig.~S-7 and Tab.~S-7
of the supplementary material. 
%
\note{
For both continuous and burst-like sources, 
varying the propagation parameters
$D_{0}$ and $\delta$ within uncertainties 
($\sim 20\%$ for $D_{0}$ and $\sim10\%$ for $\delta$
as determined in \cite{Trotta:2010mx}) mainly results in 
the changes in the over all normalization factors 
$Q_{0}$ and $N_{0}$ up to $\sim30\%$.
}







\begin{figure}[!thbp]
	\begin{center}
		\includegraphics[width=0.65\columnwidth]{anisotropy_1.eps}
		\caption{
			Predictions for electron anisotropies corresponding to 
			a selection of the cases considered in \fig{fig:flux}:
			i) the continuous point-like source (mini-spike) with $r=0.2$~kpc (dashed blue);
			ii) the continuous extended source (subhalo) with $r=0.2$~kpc and $M_{h}=10^{7}M_{\odot}$ (dashed green);
			iii) the burst-like source (PWN/SNR) with $r=2$~kpc and $\alpha$=0.7 (dashed magenta).
			The solid curves with the same color correspond to the anisotropies convoluted with 
			an energy resolution of $15\%$.
			%The solid horizontal lines with the same color correspond to that averaged over the 
			%energy interval 0.55--2~TeV.  
			The current upper limits from Fermi-LAT in the  energy interval 0.55--2~TeV
			(using the method of shuffling technique)
			are also shown%
			~\cite{Abdollahi:2017kyf}.
		}
		\label{fig:anisotropy-combined}
	\end{center}
\end{figure}




\paragraph{\bf Anisotropies.}
Nearby sources can generate non-negligible anisotropy in the CRE flux.
%CRE flux produced by nearby sources can have non-negligible anisotropy.
For an illustration,
we show in \fig{fig:anisotropy-combined}  the predicted 
dipole anisotropies from the sources as a function of the CRE energy,
corresponding to one of the  parameter sets
considered in each type of the sources shown in \fig{fig:flux}.
Assuming  a perfect energy resolution of the detector,
%and neglecting  the anisotropies of the background flux.
%For all the cases considered, 
large anisotropies of $\mathcal{O}(10^{-1})$ with 
sharp structures  are predicted.
The anisotropies in continuous sources are in general 
larger than that in the burst-like sources,
which is  related to the relatively small diffusion length.
%
The Fermi-LAT  has reported  upper limits on the  
dipole anisotropy of $\lesssim3\times 10^{-2}$  at $95\%$ C.L.
over the energy interval  0.55--2~TeV%
~\cite{Abdollahi:2017kyf}.
Note that the energy resolution of Fermi-LAT is $\sim 10\%~(17\%)$ 
at 1 (2)~TeV.
In  \fig{fig:anisotropy-combined}, 
we also show  the predicted anisotropies convoluted 
with an energy resolution of $15\%$.
%the anisotropies averaged over the interval 0.55--2~TeV,
%together with the Fermi-LAT limits% 
%derived from  log-likelihood ratio using the methods of  shuffling technique 
%~\cite{Abdollahi:2017kyf}.
After the convolution, 
the predicted anisotropies are  smaller and can reach $\mathcal{O}(10^{-2})$,
which is comparable with  the current Fermi-LAT limits.
%
Note that a quantitative comparison with the data requires 
a reliable estimation of the anisotropies contributed by the backgrounds 
which can easily reach $\mathcal{O}(10^{-3}-10^{-2})$ alone,  but 
%has large uncertainties as it strongly 
depends strongly on the assumed spatial distribution of 
the astrophysical sources%
~\cite{Manconi:2016byt}.
%The total anisotropy of also  depends on the relative directions of the nearby sources.
%

%which is highly testable in the near future by Fermi-LAT
%with higher statistics or using  long-tail events to improve the energy resolution to $\sim4\%$ at 1~TeV \cite{Abdollahi:2017nat}.






\begin{figure}[thb]
	\begin{center}
		\includegraphics[width=0.49\columnwidth]{GAMMA_deg.eps}
		\includegraphics[width=0.49\columnwidth]{GAMMA_sptr_f_30.eps}
		\caption{
			Predictions for the Galactic diffuse $\gamma$-ray fluxes in the three cases
			considered in \fig{fig:anisotropy-combined} assuming the direction of the sources
			is coincide with the GC. %($b=0^{\circ},\ell=0^{\circ}$).
			Left) 
			$d\Phi_{\gamma}/d\Omega$ as a function of Galactic latitude $b$
			along the direction of  $\ell=0^{\circ}$.
			The inset shows the fluxes  in the  inner region,
			which indicates that the $\gamma$-rays from the mini-spike
			increase sharply towards low $b$ due to prompt photons from DM annihilation, 
			making it appears as a point-like source.
			Right)
			Energy spectra of $\gamma$-ray fluxes averaged over a circular region with radius $30^{\circ}$
			centered at the GC.
			The Galactic diffuse $\gamma$-ray background is 
			calculated using  GALPROP-v54 \cite{galprop}
			with a reference propagation model 
			adopted by Fermi-LAT
			\cite{
				Ackermann:2012pya%
			}.
			The ISRF in the solar neighbourhood is interpolated from Ref.
			\cite{
				Porter:2008ve%
			}.
		}
		\label{fig:GAMMA}
	\end{center}
\end{figure}



\paragraph{\bf Gamma-ray signals.}
%The sources of CRE are inevitably sources of $\gamma$-rays. 
DM annihilation can produce prompt photons through final state radiation (FSR)  of charged leptons. 
The FSR photon spectrum has a distinct feature of an approximate power law with index $\sim1$ for $E\ll m_{\chi}$, plus a sharp cutoff at $m_{\chi}$.
In morphology, 
mini-spikes appear as point-like sources due to 
the high concentration of DM density.
For the cases of mini-spike considered in \fig{fig:flux}, 
the total fluxes ($E>1$~GeV) %($1<E<100$~GeV with $68\%$ containment) 
are in the range  
$\sim (0.71-1.51)\times 10^{-10}\text{cm}^{-2}\text{s}^{-1}$.
In the Fermi-LAT 3FGL catalogue of unassociated point sources~\cite{Acero:2015hja},
we find 6 candidate  sources with low power indices
which can reach $\sim1$ within $2\sigma$ error:
J0603.3+2042, J1250.2-0233, J2209.8-0450, J1705.5-4128c, J2142.6-2029
and J2300.0+4053.
Most of them have total fluxes of 
$\mathcal{O}(10^{-10})~\text{cm}^{-2}\text{s}^{-1}$  %
which are consistent with that favoured by the mini-spikes.
%%%
%\note{
The favoured 
DM annihilation cross sections 
for both mini-spikes and DM subhalos
are well below the  current upper limits derived from  
the $\gamma$ rays towards 
dwarf Galaxies~\cite{Ackermann:2015zua} and 
the Galactic center (GC)~\cite{Abdallah:2016ygi}
under the assumption of   smooth DM  profiles.
%without taking the substructures into account.
%
Taking into account the distribution of mini-spikes,
very stringent constraints on the annihilation cross sections were
obtained by HESS from the GC $\gamma$-ray data 
\cite{%
	%Bringmann:2009ip,%
	Aharonian:2008wt%
}, 
which are highly model dependent. 
%strongly depends on the modeling
%of the mini-spike distribution in the GC region. 
%
The $\gamma$-rays constraints   from 
other sky regions are much weaker
~\cite{Bringmann:2009ip}.
%and dwarf galaxies
%\cite{
%	Wanders:2014xia,%
%	Gonzalez-Morales:2014eaa%
%}.
%Note, however, that these constraints strongly depend on 
%the models of the merging history of blackholes 
%which have large uncertainties. 
%are largely unknown.
%
%}

All the CRE sources can produce diffuse $\gamma$-rays through 
electron inverse Compton scattering (ICS) off 
interstellar radiation fields (ISRF),
the corresponding energy spectrum is softer compared with 
that of the FSR photons~\cite{Meade:2009iu}.
%
For the typical parameters considered in the three scenarios in \fig{fig:flux}, 
the predicted differential flux $E^{3}d\Phi/dE$ can reach 
$\mathcal{O}(10^{-8})~\tx{GeV}\tx{cm}^{-2}\tx{s}^{-1}\tx{sr}^{-1}$ 
in the energy range $\sim(0.1-1)$~TeV,
which may be subject to the constraints from the Fermi-LAT data 
on the Galactic diffuse $\gamma$-rays~\cite{Ackermann:2012rg}.
%
However, the Fermi-LAT constraints vary with sky regions.
%the constraints can be weakened,
In the  low Galactic latitudes regions the backgrounds can easily 
reach $\sim10^{-6}~\tx{GeV}\tx{cm}^{-2}\tx{s}^{-1}\tx{sr}^{-1}$
%if the sources are located at regions with low Galactic latitudes
%where the backgrounds can easily reach 
%$\sim10^{-6}~\tx{GeV}\tx{cm}^{-2}\tx{s}^{-1}\tx{sr}^{-1}$
and dominate the total diffuse $\gamma$-ray  emission. 
In this case the  Fermi-LAT data cannot place stringent constraints 
on these CRE  sources.
% Fig.4
This possibility is illustrated in \fig{fig:GAMMA}, where
the direction of the source is assumed to  coincide with the GC.
For the three cases discussed  in  \fig{fig:anisotropy-combined},
we show the spatial extension as well as the energy spectrum of the associated $\gamma$-rays,
together with the corresponding Galactic diffuse $\gamma$-ray backgrounds.
%In the left panel of \fig{fig:GAMMA}, 
%we show
%the  $\gamma$-ray fluxes  $d\Phi_{\gamma}/d\Omega$ 
%(with $E>$1 GeV) 
%as a function of  the Galactic latitude $b$  
%along the direction of Galactic altitude $\ell=0^{\circ}$
%(the direction along which the background  decreases in the fastest way)
%for the three cases considered in  \fig{fig:anisotropy-combined}.
%The  energy spectra averaged over a circular region with  a $30^{\circ}$ radius 
%centered at the GC are shown in the right panel of \fig{fig:GAMMA}.
In the calculations, 
the contributions from the Galactic halo DM 
for mini-spikes and DM subhalos are included
assuming an Einasto DM profile.
%
The Galactic diffuse $\gamma$-ray background is 
calculated using a reference propagation model 
adopted by Fermi-LAT~\cite{Ackermann:2012pya}
which agrees with the data well.
%
It can be seen that for  the typical parameters  considered, 
the predicted $\gamma$-ray fluxes can be 
a few orders of magnitude below the background,
%
which suggests that the Ferm-LAT constraints should be 
rather weak, 
as the uncertainties in the background model are still significant
(see the supplementary material for detailed calculations on 
the diffuse $\gamma$-rays).
%for each sources in different sky regions). 
%

%In summary, 
%%the current experiments have entered the multi-TeV region where
%%the CRE  spectrum is unlikely to be smooth.
%we have proposed generic mechanisms of the origins of the CRE structures  and 
%analysed the nature of sources responsible for 
%the possible DAMPE excess.
%The predictions of these scenarios are highly testable in the near future
%with more accurate data.



\begin{acknowledgments}
This work is supported in part by the National Key R$\&$D Program of China 
No. 2017YFA0402204, the NSFC under Grants No. 11335012, 11690022, 11475237 and U1738209,
and the CAS Key Research Programs, No. XDB23030100 and QYZDY-SSW-SYS007.
{\it Note added.} After submitting the first version of the manuscript,
a number of analyses on the DAMPE data appeared%
~\cite{
Yuan:2017ysv,Fan:2017sor,Fang:2017tvj,Duan:2017pkq,Gu:2017gle,
Athron:2017drj,Cao:2017ydw,Liu:2017rgs,Zu:2017dzm,Tang:2017lfb,
Chao:2017yjg,Gu:2017bdw,Duan:2017qwj,Cholis:2017ccs,Jin:2017qcv,
Gao:2017pym,Niu:2017hqe,Chao:2017emq,Chen:2017tva,Li:2017tmd,
Zhu:2017tvk,Gu:2017lir,Nomura:2017ohi,Ghorbani:2017cey,Cao:2017sju,
Yang:2017cjm,Ding:2017jdr,Liu:2017obm,Ge:2017tkd,Zhao:2017nrt,
Sui:2017qra,Okada:2017pgr,Cao:2017rjr,Dutta:2017sod,Fowlie:2017fya,Han:2017ars,
Niu:2017lts%
}.
We found that most of the proposed models fall into the scenarios 
discussed in our work.
Ref.~\cite{Yuan:2017ysv} discussed the contributions from PWNe and DM substructures.
Their conclusions are also consistent with ours.
\end{acknowledgments}




\bibliographystyle{apsrev4-1} %
\bibliography{dampe,misc,dampe_papers}

\clearpage
\onecolumngrid
\setcounter{section}{0} 

\section*{Supplementary Material} In the supplementary material, we first discuss more implementation details in Section~\ref{sec:implem} and present additional experiments in Section~\ref{sec:add_exp}. Then we present some examples that were selected as `hard to verify' in the user study in Section~\ref{sec:user_study_ex}. We present other qualitative examples in Section~\ref{sec:qual_analysis2}. Finally, we discuss societal aspects and potential risks in Section~\ref{sec:risks}.


\section{Implementation Details} \label{sec:implem}
We elaborate on some implementation details of our framework. 
\begin{itemize}

\item \textbf{Sentence representation.}
We preprocessed the crawled captions to remove some artefacts (e.g., HTML tags). When using BERT+LSTM, we used the pre-trained `bert-base-uncased' model, whose dimension is 768. We set a maximum length of 150 tokens for the captions. Items (i.e., query captions, evidence captions, and entities) are padded to the maximum sequence length in this item's batch. When using the sentence transformer model, we used the `paraphrase-mpnet-base-v2' model\footnote{https://huggingface.co/sentence-transformers/paraphrase-mpnet-base-v2}. For both, we used the Hugging Face library\footnote{https://huggingface.co/}. We used the PyTorch framework\footnote{https://pytorch.org/} for all our experiments.

\item \textbf{Memory.}
The items in each memory (images, entities, and captions) are padded to the maximum number of evidence items in this memory's batch.

\item \textbf{CLIP.}
We used the pre-trained ViT-B/32 CLIP model\footnote{https://github.com/openai/CLIP}, where the text length is truncated at 77 tokens.

\item \textbf{Training details.}
When fine-tuning CLIP, we follow the implementation details in~\cite{luo2021newsclippings}, we used a learning rate of 5e-5 for
the linear classifier and 5e-7 for other layers of the CLIP model itself, in addition to using the Adam optimizer~\cite{kingma2014adam}. We used a batch size of 64 and trained the model for 100 epochs. For training \model{}, we used a batch size of 32, the Adam optimizer, and a cyclical learning rate~\cite{smith2017cyclical} with a maximum value of 6e-5. We trained the model for 30 epochs. We used a dropout~\cite{srivastava2014dropout} value of 0.05 to the input representations, 0.25 to domain embeddings, and 0.25 to the memory representations. Experiments were done on one NVIDIA A100 GPU. With precomputing the representations, the training takes roughly 5 hours. When training using BERT without precomputing, training takes roughly 30 hours.

\end{itemize}
\section{Additional Experiments} \label{sec:add_exp}

\paragraph{Evidence-only classification.} We examine whether claims (and consequently, the evidence) are having different characteristics (and thus, unwanted biases or naive give-aways) between pristine and falsified classes. The NewsCLIPpings dataset avoided linguistic biases in creating falsified examples by using real news \textbf{\textcolor{myblue}{captions}} mismatched with real news \textbf{\textcolor{myOrange}{images}}, instead of introducing manipulations in the captions. Also, to avoid text bias, each \textbf{\textcolor{myblue}{caption}} (and consequently, its \textbf{\textcolor{myOrange}{visual evidence}} in our dataset) appears twice (within the same split), once as pristine and once as falsified. Therefore, we hypothesize that the evidence websites for both classes are similar. To confirm, we ran an \textit{evidence-only} model, which achieved 53.4\% (\textit{basically chance level}), showing that \textit{reasoning against the query} is the distinguishing factor.

\begin{table}[!b]
\begin{center}
\vspace{-1mm}
\resizebox{0.65\linewidth}{!}{
\begin{tabular}{cccc}
\toprule
Conc. & Avg-pool & Max-pool & Multiply \\ \midrule
\textbf{83.9} &  82.46 & 82.48 & 77.1 \\ \bottomrule
\end{tabular}}
\end{center}
\vspace{-6mm}
\caption{Accuracy (\%) vs. aggregation strategies.}
\vspace{-3mm}
\label{rebuttal_tab:ablation1}
\end{table}

\paragraph{Additional ablation studies.} We include further experiments related to the fusion of the different components in our model (visual reasoning, textual reasoning, and CLIP). We tried a late fusion by having a separate classifier on top of each branch and aggregating the decision, however, this performed worse than the current intermediate fusion we employ. We also tried other strategies (\autoref{rebuttal_tab:ablation1}) to combine visual and textual memories before concatenating with CLIP, where we found that concatenation had the highest performance. 

Finally, we found that changing the dimension of the penultimate layer had a relatively small effect; e.g., increasing the dimension to 2048 increased the accuracy by 0.3 percentage points.

\section{User Study: `Hard to Verify' Examples} \label{sec:user_study_ex}
In Figure~\ref{tbl:qual_study_appendix}, we show some examples that were selected as `hard to verify' in the user study. This is possibly due to: 1) the \textbf{\textcolor{myblue}{captions}} could contain specific context information (e.g., locations such as \textit{`Denver'} or \textit{`Massachusetts'}) that is hard to verify with the \textbf{\textcolor{myOrange}{image}} alone, 2) the lack of \textbf{\textcolor{myblue}{textual}} evidence returned by the search \vcenteredinclude{figs/icon3.pdf}; the \textbf{\textcolor{myOrange}{images}} were not found by the inverse image search, so there are no \textbf{\textcolor{myblue}{captions/titles}} found. Moreover, the \textbf{\textcolor{myblue}{entities}} are generic descriptions of the \textbf{\textcolor{myOrange}{image}}, or not at all related (the first example). The performance of the model on these examples is possibly dependent on how similar the \textbf{\textcolor{myOrange}{visual}} evidence is to the query \textbf{\textcolor{myOrange}{image}} \vcenteredinclude{figs/icon4.pdf}. 

Another possible reason is having falsified pairs that are highly similar in context to the original ones (and, therefore, to the evidence as well). For instance, the last example shows a `hard to verify' falsified example (that was also misclassified by our model); the \textbf{\textcolor{myOrange}{image}} shows the same people mentioned in the \textbf{\textcolor{myblue}{caption}}, and thus, they also appeared in the \textbf{\textcolor{myOrange}{visual}} evidence. Additionally, the \textbf{\textcolor{myblue}{caption}} mentions the band name \textit{`One Direction'} that is also mentioned in the \textbf{\textcolor{myblue}{textual}} evidence, without strong contradictions. Meanwhile, the actual \textbf{\textcolor{myOrange}{image}} of this \textbf{\textcolor{myblue}{caption}} showed the band performing on a stage, however, this was not clearly emphasized by the \textbf{\textcolor{myblue}{caption}}; that is possibly why the \textbf{\textcolor{myOrange}{visual}} evidence is generic.

\section{Qualitative Examples} \label{sec:qual_analysis2}
In Figure~\ref{tbl:qual_appendix}, we show more qualitative examples. \model{} predicted many examples correctly despite not having a one-to-one matching with the evidence in the case of pristine examples and having close similarity to the evidence in the case of falsified examples. 

For instance, in the first three examples (pristine), we observed that the model highly attended to supporting evidence such as persons' and countries' names, topics, and events. Additionally, in the third example, we observed that the model prioritized the \textbf{\textcolor{myOrange}{image}} that is from the same scene and the evidence \textbf{\textcolor{myblue}{caption}} that contains a subset from the query \textbf{\textcolor{myblue}{caption}} (\textit{`soon to be a Trump International Hotel'}). 

The fourth and fifth examples (falsified) suggest that the model does not simply rely on having any similarity or overlap between the query and evidence in order to identify pristine examples. Despite having the same persons in the evidence, they were correctly predicted as falsified, possibly as they have contradicting location information and different scene details (e.g., lighting, stage setup, or colours), indicating a different context or event. The last falsified example also indicates that both \textbf{\textcolor{myblue}{textual}} and \textbf{\textcolor{myOrange}{visual}} evidence is helpful, as the evidence \textbf{\textcolor{myOrange}{images}} are clearly different from the falsified one (showing a different building and place). 

%As for the last example, it highlights one of the `hard to detect' examples in the dataset, even with the presence of evidence, as the falsified image is also showing a similar object \textit{`iPhone'} without a strongly different context. 
%\clearpage

\section{Limitations and Societal Aspects}  \label{sec:risks}
%Automating fact-checking can be beneficial to fight the spread of misinformation. 
Nowadays, with the spread and reliance on social media to digest and get updated with news, misinformation (e.g., on Twitter) can reach hundreds of millions of users~\cite{vo2020facts}. This crucially motivates the need to fact-check and verify the credibility of online content, especially during critical times such as a pandemic or political instabilities. On the other hand, manual fact-checking is usually time-consuming, needing from less than one hour to many days to verify a claim~\cite{thorne2018automated}. Therefore, automating fact-checking can be extremely beneficial to alleviate the burden upon fact-checkers and journalists. 

However, completely or overly relying on automated tools might give an unwanted sense of security and could have many dangerous consequences. These include the dangers of flagging many true examples as falsified due to the real-life class imbalance, and missing out challenging falsified examples that require more fine-grained and complex reasoning. In addition, a currently active and much-needed research direction in the textual domain shows that fact-verification models might be partially relying on dataset biases without in-depth understanding and reasoning~\cite{schuster2019towards}. They might also be brittle to complex claims that require multi-hop reasoning~\cite{hidey2020deseption}. Additionally, as facts are continuously evolving, we face the danger of relying on old retrieved evidence~\cite{schuster2021get} or even possibly outdated world knowledge that is implicitly stored in pre-trained language models during training~\cite{schuster2019towards}.

In addition to their inherent limitations in reasoning and interpretation, several works have shown that textual verifications models are also vulnerable to adversarial attacks~\cite{thorne2019evaluating}, such as inserting trigger words~\cite{atanasova2020generating}, introducing lexical variations~\cite{hidey2020deseption}, or paraphrasing\cite{thorne2019evaluating}. As we have a multi-modal task, our model might also be vulnerable to image-based adversarial attacks~\cite{goodfellow2014explaining}. %Beyond manipulating claims via adversarial attacks, adversaries can also poison the evidence and introduce items that lead to the required entailment. 
Another potential misuse scenario is using the fact-checking model as an adversarial filter in order to curate hard examples that might be misclassified by fact-checking models in general. 

As a conclusion, we believe that automating fact-checking is strongly beneficial and that there have been many encouraging advancements to improve and harden it in the textual domain and the multi-modal domain, as we propose. However, due to their limitations and vulnerabilities to active attacks and manipulation, they should be used to assist humans and speed up the process, while still keeping them in the loop to avoid such dangers and consequences. In this regard, in our framework, we show that the model can filter and select the most important evidence, which would enable quicker inspection of the evidence items. 


\section{SIMULATION RESULTS}
\label{sec:examples}
This section presents simulation results of the proposed method implemented on the unicycle model example.
Each semidefinite program was prepared using a custom software toolbox and the modeling tool YALMIP \cite{lofberg2004yalmip}.
The programs are run with commercial solver MOSEK on a machine with $1$ TB availabe memory. 

\subsection{FRS Computation}
We computed the FRS for a 3$^\text{rd}$ order Taylor-expanded Dubins car as the low-fidelity model $f_s$.
Trajectories produced by this model were tracked by the unicycle model from Equation \eqref{eq:big_dyn} as the high-fidelity model $f$.
The vehicle's representation as an initial distribution $X_0 \subset X_s$, was a rectangle of length $0.2$ [m] in $x$ and width $0.1$ [m] in $y$, at $0^\circ$ initial heading, and centered at $x=-0.75$ and $y=0$.
This is the same vehicle representation shown in all previous figures.

% The error function $g$, illustrated in Figure \ref{fig:error_dynamics}, was given by:
% \begin{equation}
% \label{eq:g_definition}
% g(t,x_s) = \begin{bmatrix}
% v_\text{err}\cdot(1 - \frac{1}{2}\theta^2)  \\
% v_\text{err}\cdot(\theta - \frac{1}{6}\theta^3) \\
% \dot{\theta}_\text{err}
% \end{bmatrix}
% \end{equation}
% where $v_\text{err} = (t-1)^2$ and $\dot{\theta}_\text{err} = (t-1)^4$.
We chose $\tau_\text{stop} = \tau_\text{plan} = 0.5$ [s], so $T = 1$ [s].
The stopping time can be seen in Figure \ref{fig:error_dynamics}. 
The FRS computation took 79 hours and used a maximum of 150 GB of memory 
%on a server with 1 TB of available memory and 18 processors each running at 1.2 GHz.

\subsection{Set Intersection and Trajectory Planning}

We used the precomputed FRS for safe trajectory planning in $1000$ simulated trials in MATLAB on the aforementioned machine.
For each trial, the vehicle began at the same initial location and heading, surrounded by $1-10$ randomized obstacles and a randomly-located goal to reach.
%If the planning time took more than $\tau_\text{plan}$, the simulation paused until the computation was complete. 
%In practice, if $\tau_\text{plan}$ was exceeded the vehicle could begin braking to ensure safety.
The vehicle's initial speed, and the desired speed to maintain for the duration of the trial, were randomly chosen between $0.25$ and $0.75$ [m/s].
% The trials ran in 12.7 hours.
% Prior to running these trials, several example trials were run on a laptop with a 2.3 GHz processor and 16 GB of RAM.
% The trials run on the server were individually no faster than running on the laptop, because the set intersection optimization is a single-core process that uses very little memory. 
% Therefore, the server did not provide any significant decrease in the implemented planning time.


Obstacles were represented as line segments between $0.1$ and $0.2$[m] in length, with random location and orientation.
The obstacles were always placed between the vehicle and the goal.
We checked for crashes conservatively for each trial, by inspecting if any obstacle was within a circle circumscribing the rectangular vehicle at any point of the vehicle's trajectory. 
Using this method, \emph{no crashes were detected in any trial}.
Out of all the trials, $82\%$ reached the goal, and $15\%$ performed an emergency braking maneuver (by setting $v_\text{des} = 0$). 
The remaining 3\% hit a simulation iteration limit.
Examples of the vehicle's path from a randomly-generated trial and from two constructed emergency braking cases are shown in Figure \ref{fig:example_trial}.


\begin{figure}
\centering
\includegraphics[width=1\columnwidth]{running_examples.pdf}
\caption{The top subplot shows an example result out of the $1000$ trials.
This trial used eight randomly-generated obstacles.
The vehicle begins on the left at $x = -0.75$ and reaches a randomly-generated goal near $(2.5, 0.5)$, plotted as a blue circle.
Every $\tau_\text{plan} = 0.5$[s], the vehicle replans its trajectory, shown by an asterisk plotted on the global trajectory in blue.
The bounding box of the vehicle at each planning step is shown as a grey rectangle. In the bottom-left subplot, an obstacle was constructed between the vehicle and the goal, forcing an emergency braking maneuver. In the bottom-right subplot, an obstacle was constructed with a hole that would allow the vehicle to pass, but the set intersection result is overly conservative, resulting in a braking maneuver.}
\label{fig:example_trial}
\end{figure}

Currently, our implementation cannot consistently achieve $\tau_\text{plan} = 0.5$ [s].
Consequently, instead of replanning and driving simultaneously, we pause time every 0.5 [s] of the simulation to guarantee that the vehicle can finish replanning.
In a physical implementation, if $\tau_\text{plan}$ is exceeded, then the vehicle must emergency brake; recall that a safe braking trajectory is always available.
As shown in Figure \ref{fig:planning_time_vs_Nobs}, $\tau_\text{plan}$ scales linearly with the number of obstacles.
%Methods for reducing the set intersection to meet $\tau_\text{plan}$ will be presented in future work.

\begin{figure}
\centering
\includegraphics[scale=0.45,trim={1cm 6cm 1cm 7cm},clip]{planning_time_vs_Nobs.pdf}
\caption{The mean set intersection time (top) and trajectory optimization time (bottom) versus the number of obstacles. Over the $1000$ trials, each number of obstacles from $1$ to $10$ was used for $100$ trials. Notice that set intersection takes up to $3$[s], and scales with the number of obstacles. On the other hand, the trajectory optimization takes around $80$ [ms] and has low correlation with number of obstacles.}
\label{fig:planning_time_vs_Nobs}
\end{figure}

% \begin{figure}
% \centering
% \includegraphics[scale=0.5,trim={1cm 8cm 1cm 8cm},clip]{example_trial_bluecar.pdf}
% \caption{An example result out of the 1000 trials.
% This trial used eight randomly-generated obstacles.
% The vehicle begins on the left at $x = -0.75$ and reaches a randomly-generated goal near $(2.5, 0.5)$, plotted as a blue circle.
% Every $\tau_\text{plan} = 0.5$ [s], the vehicle replans its trajectory, shown by an asterisk plotted on the global trajectory in blue.
% The bounding box of the vehicle at each planning step is shown as a grey rectangle.}
% \label{fig:example_trial}
% \end{figure}

% \begin{figure}
% \centering
% \includegraphics[scale=0.4,trim={1cm 7cm 1cm 7cm},clip]{example_emergency_brake.pdf}
% \caption{An example of a forced emergency braking situation. The vehicle cannot find a path to the desired location (plotted as a blue circle), so it brakes.}
% \label{fig:example_emergency_brake}
% \end{figure}

% \begin{figure}
% \centering
% \includegraphics[scale=0.4,trim={1cm 7cm 1cm 7cm},clip]{example_overly_conservative.pdf}
% \caption{An example of an unnecessary emergency braking situation. The vehicle cannot find a path to the desired location despite an obviously-safe path existing, because the FRS is overly conservative.}
% \label{fig:example_overly_conservative}
% \end{figure}


\begin{table*}[!t]
\centering
\resizebox{\linewidth}{!}{%
\begin{tabular}{c|c c}
\toprule
\textbf{\textcolor{myOrange}{\large{Image}}}-\textbf{\textcolor{myblue}{\large{caption}}} \large{pair} & \large{\textbf{\textcolor{myblue}{Textual evidence}}} \largericon{figs/icon3.pdf} & \large{\textbf{\textcolor{myOrange}{Visual evidence}}} \largericon{figs/icon4.pdf} \\ \midrule
\makecell{\fcolorbox{ao(english)}{lightgreen}{
\begin{varwidth}{\textwidth} \begin{center}\fcolorbox{myOrange}{white}{\includegraphics[width=5cm,keepaspectratio]{figs/qual2/260/235.jpg}}\end{center} 
\fcolorbox{myblue}{white}{\begin{varwidth}{\textwidth}\normalsize{Hungary has erected a fence on\\its border with Serbia}\end{varwidth} }\end{varwidth}}} & 

\makecell{\fcolorbox{myblue}{white}{\begin{varwidth}{\textwidth} \normalsize{\hlc[light_yellow]{`Hungary'}, \hlc[light_yellow]{`European migrant crisis'},\\\hlc[light_yellow]{`Refugee'}, `Human migration',\\\hlc[light_yellow]{`Immigration'}, `Border', \\`Fence',`Hungarians', `Asylum seeker'\\`Hungary–Serbia border',\\`Hungarian border barrier',\\`International law',`Refugee law',\\`hungary fences refugees'} \end{varwidth}}
\fcolorbox{myblue}{white}{\begin{varwidth}{\textwidth} \normalsize{\hlc[light_yellow]{1- Hungary police recruit}\\\hlc[light_yellow]{border-hunters.}\\2-Migrants and refugees walk\\near razor-wire along a 3-meter-high\\fence secured by Hungarian police\\at the official border crossing\\between Serbia and Hungary.} \end{varwidth}}}
& 
\makecell{ \fcolorbox{myOrange}{light_yellow}{\includegraphics[width=5cm,keepaspectratio]{figs/qual2/260/8.jpg}} \fcolorbox{myOrange}{white}{\includegraphics[width=5cm,keepaspectratio]{figs/qual2/260/5.jpg}}
\fcolorbox{myOrange}{white}{\includegraphics[width=5cm,keepaspectratio]{figs/qual2/260/9.jpg}}}
\\ & \multicolumn{2}{c}{\hspace{-4cm}\large{\textbf{Prediction: \textcolor{ao(english)}{Pristine}}}} \\


\makecell{\fcolorbox{ao(english)}{lightgreen}{
\begin{varwidth}{\textwidth} \begin{center}\fcolorbox{myOrange}{white}{\includegraphics[width=5cm,keepaspectratio]{figs/qual2/5884/417.jpg}}\end{center} 
\fcolorbox{myblue}{white}{\begin{varwidth}{\textwidth}\normalsize{Last year Shinzo Abe said Africa\\would help drive global growth\\in the future
}\end{varwidth} }\end{varwidth}}} & 

\makecell{\fcolorbox{myblue}{white}{\begin{varwidth}{\textwidth} \normalsize{\hlc[light_yellow]{`Shinzo Abe'}, `Akie Abe',\\`Prime Minister of Japan',\\`Japan', `Prime minister',\\\hlc[light_yellow]{`Trinidad and Tobago'}, \\`Dominica Vibes News',\\\hlc[light_yellow]{`United National Congress'},\\`Week', `Businessperson', 'official'\\\hlc[light_yellow]{`Dominica Housing Recovery Project'}} \end{varwidth}}
\fcolorbox{myblue}{white}{\begin{varwidth}{\textwidth} \normalsize{\hlc[light_yellow]{1-Japanese Prime Minister}\\\hlc[light_yellow]{Shinzo Abe and his wife,}\\\hlc[light_yellow]{Akie Abe.}\\2-Japanese Prime Minister\\Shinzo Abe, center, and\\his wife Akie wave as they\\depart for Africa, at Haneda\\Airport in Tokyo Thursday.} \end{varwidth}}}
& 
\makecell{ \fcolorbox{myOrange}{light_yellow}{\includegraphics[width=5cm,keepaspectratio]{figs/qual2/5884/1.jpg}} \fcolorbox{myOrange}{white}{\includegraphics[width=5cm,keepaspectratio]{figs/qual2/5884/8.jpg}}
\fcolorbox{myOrange}{white}{\includegraphics[width=6cm,keepaspectratio]{figs/qual2/5884/5.jpg}}}  
\\ & \multicolumn{2}{c}{\hspace{-4cm}\large{\textbf{Prediction: \textcolor{ao(english)}{Pristine}}}} \\

\makecell{\fcolorbox{ao(english)}{lightgreen}{
\begin{varwidth}{\textwidth} \begin{center}\fcolorbox{myOrange}{white}{\includegraphics[width=5cm,keepaspectratio]{figs/qual2/146/019.jpg}}\end{center} 
\fcolorbox{myblue}{white}{\begin{varwidth}{\textwidth}\normalsize{The GOP candidate at the soontobe\\Trump International Hotel a couple\\of blocks from the White House\\on Pennsylvania Avenue}\end{varwidth} }\end{varwidth}}} & 

\makecell{\fcolorbox{myblue}{white}{\begin{varwidth}{\textwidth} \normalsize{`Judge', \hlc[light_yellow]{`Legal case'}, `official'\\\hlc[light_yellow]{`Superior Court of the District}\\\hlc[light_yellow]{of Columbia'},\\`President-Elect', \hlc[light_yellow]{`Deposition'},\\`Court',`Plea', `Chef',\\\hlc[light_yellow]{`A Washington Law Firm'}} \end{varwidth}}
\fcolorbox{myblue}{white}{\begin{varwidth}{\textwidth} \normalsize{\hlc[light_yellow]{1-Republican presidential candidate}\\\hlc[light_yellow]{Donald Trump speaks during}\\\hlc[light_yellow]{a campaign press conference at}\\\hlc[light_yellow]{the at the Old Post Office Pavilion,}\\\hlc[light_yellow]{soon to be a Trump International Hotel}\\2-Judge rejects Trump plea to avoid\\deposition in José Andrés case} \end{varwidth}}}
& 
\makecell{ \fcolorbox{myOrange}{light_yellow}{\includegraphics[width=5cm,keepaspectratio]{figs/qual2/146/0.jpg}} \fcolorbox{myOrange}{white}{\includegraphics[width=5cm,keepaspectratio]{figs/qual2/146/6.jpg}}
\fcolorbox{myOrange}{white}{\includegraphics[width=5cm,keepaspectratio]{figs/qual2/146/2.jpg}}}  
\\ & \multicolumn{2}{c}{\hspace{-4cm}\large{\textbf{Prediction: \textcolor{ao(english)}{Pristine}}}} \\

\makecell{\fcolorbox{darkred}{lightred}{\begin{varwidth}{\textwidth}   \begin{center} \fcolorbox{myOrange}{white}{\includegraphics[width=5cm,keepaspectratio]{figs/qual2/3283/099.jpg}}\end{center}
\fcolorbox{myblue}{white}{\begin{varwidth}{\textwidth}\normalsize{Hillary Clinton speaks at a campaign\\event at Truckee Meadows Community\\College in Reno Nev Aug 25}\end{varwidth}}\end{varwidth}}} & 

\makecell{\fcolorbox{myblue}{white}{\begin{varwidth}{\textwidth} \normalsize{\hlc[light_yellow]{`United States'},`Commentator',\\\hlc[light_yellow]{`President of the United States'},\\\hlc[light_yellow]{`Clinton Foundation'},\\`Dinesh DSouza', `performance'} \end{varwidth}}   
\fcolorbox{myblue}{white}{\begin{varwidth}{\textwidth} \normalsize{\hlc[light_yellow]{1-Democratic presidential candidate}\\\hlc[light_yellow]{Hillary Clinton speaks at the}\\\hlc[light_yellow]{Iowa Democratic Wing Ding on}\\\hlc[light_yellow]{Friday in Clear Lake, Iowa.}\\2-At Wing Ding dinner, Clinton\\proves she still dominates Iowa} \end{varwidth} }}
& 
\makecell{ \fcolorbox{myOrange}{light_yellow}{\includegraphics[width=5cm,keepaspectratio]{figs/qual2/3283/8.jpg}} \fcolorbox{myOrange}{white}{\includegraphics[width=5cm,keepaspectratio]{figs/qual2/3283/4.jpg}}
\fcolorbox{myOrange}{white}{\includegraphics[width=5cm,keepaspectratio]{figs/qual2/3283/9.jpg}}} \\
&\multicolumn{2}{c}{\large{\hspace{-4cm}\textbf{Prediction: \textcolor{darkred}{Falsified}}}}\\

\makecell{\fcolorbox{darkred}{lightred}{\begin{varwidth}{\textwidth}   \begin{center} \fcolorbox{myOrange}{white}{\includegraphics[width=5cm,keepaspectratio]{figs/qual2/3023/620.jpg}}\end{center}
\fcolorbox{myblue}{white}{\begin{varwidth}{\textwidth}\normalsize{Taylor Swift performs during a concert at\\the Lanxess Arena in Cologne Germany}\end{varwidth}}\end{varwidth}}} & 

\makecell{\fcolorbox{myblue}{white}{\begin{varwidth}{\textwidth} \normalsize{\hlc[light_yellow]{`Taylor Swift'}, `The 1989 World Tour',\\\hlc[light_yellow]{`Grammy Award for Album of the Year'},\\\hlc[light_yellow]{`Grammy Awards'}, `Album', \\`Welcome to New York', `Speak Now',\\`Pop music', `reputation',\\`Apple Music', `Kendrick Lamar',\\`Adele', \hlc[light_yellow]{'taylor swift 1989'}} \end{varwidth}}   
\fcolorbox{myblue}{white}{\begin{varwidth}{\textwidth} \normalsize{\hlc[light_yellow]{1-Taylor Swift performs during}\\\hlc[light_yellow]{her '1989' World Tour Nov. 28,}\\\hlc[light_yellow]{2015, in Sydney, Australia.}\\2-Taylor Swift earned nominations\\in the major categories.\\3-Taylor Swift performs\\during her `1989' tour.} \end{varwidth} }}
& 
\makecell{ \fcolorbox{myOrange}{light_yellow}{\includegraphics[width=5cm,keepaspectratio]{figs/qual2/3023/3.jpg}} \fcolorbox{myOrange}{white}{\includegraphics[width=5cm,keepaspectratio]{figs/qual2/3023/6.jpg}}
\fcolorbox{myOrange}{white}{\includegraphics[width=5cm,keepaspectratio]{figs/qual2/3023/5.jpg}}} \\
&\multicolumn{2}{c}{\hspace{-4cm}\large{\textbf{Prediction: \textcolor{darkred}{Falsified}}}}\\ 

\makecell{\fcolorbox{darkred}{lightred}{\begin{varwidth}{\textwidth}   \begin{center} \fcolorbox{myOrange}{white}{\includegraphics[width=4cm,keepaspectratio]{figs/qual2/415/507.jpg}}\end{center}
\fcolorbox{myblue}{white}{\begin{varwidth}{\textwidth}\normalsize{The Blue House the executive office\\and residence of Korea s president}\end{varwidth}}\end{varwidth}}} & 

\makecell{\fcolorbox{myblue}{white}{\begin{varwidth}{\textwidth} \normalsize{`Parliament Hill',`Parliament of Canada'\\, \hlc[light_yellow]{`2014 shootings at Parliament Hill, Ottawa'},\\`Prime Minister of Canada', \\`Royal Canadian Mounted Police', \hlc[light_yellow]{`Ottawa'},\\`Terrorism', `Prime minister',\\`Michael Zehaf-Bibeau', `Stephen Harper',\\ \hlc[light_yellow]{`Kevin Vickers'}, \hlc[light_yellow]{`Ontario'},\\`Canada', `Ottawa'} \end{varwidth}}   
\fcolorbox{myblue}{white}{\begin{varwidth}{\textwidth} \normalsize{\hlc[light_yellow]{1-Police tape surrounds the Canadian}\\\hlc[light_yellow]{War Memorial in Ottawa after a soldier}\\\hlc[light_yellow]{guarding the monument was shot on}\\\hlc[light_yellow]{Wednesday.}\\2-Shooting Near Canada's Parliament.\\3-Shooting at War Memorial in Canada\\Photos} \end{varwidth} }}
& 
\makecell{ \fcolorbox{myOrange}{light_yellow}{\includegraphics[width=5cm,keepaspectratio]{figs/qual2/415/0.jpg}} \fcolorbox{myOrange}{white}{\includegraphics[width=5cm,keepaspectratio]{figs/qual2/415/7.jpg}}
\fcolorbox{myOrange}{white}{\includegraphics[width=5cm,keepaspectratio]{figs/qual2/415/5.jpg}}}\\ 
&\multicolumn{2}{c}{\hspace{-4cm}\large{\textbf{Prediction: \textcolor{darkred}{Falsified}}}}\\ \bottomrule 

\end{tabular}}
\captionof{figure}{Other qualitative examples. The ground truth is indicated by the pairs' background colour; examples with \hlc[lightgreen]{green background} are pristine, \hlc[lightred]{red background} are falsified. The model's prediction is indicated below each example's set; \textbf{\textcolor{ao(english)}{green}} for predicting pristine and \textbf{\textcolor{darkred}{red}} for predicting falsified. \hlc[light_yellow]{Highlighted items} are the ones with the highest attention.}
\label{tbl:qual_appendix}
\end{table*}



\end{document}


