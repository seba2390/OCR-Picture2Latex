
\section{Supplementary Material}

\renewcommand{\theequation}{S-\arabic{equation}}
\setcounter{equation}{0}

\renewcommand{\thefigure}{S-\arabic{figure}}
\setcounter{figure}{0}

\renewcommand{\thetable}{S-\arabic{table}}
\setcounter{table}{0}


\vskip 12pt
\centerline{\bf \large Origins of  sharp cosmic-ray electron structures and the DAMPE excess}
\centerline{\it Xian-Jun Huang, Yue-Liang Wu, Wei-Hong Zhang, and Yu-Feng Zhou}
\vskip 12pt

In this supplementary material, we give more details and discussions of the results obtained 
in the letter.

\subsection{A. Properties of the DAMPE excess}
We adopt a two-component description of the 
CRE flux, namely, the main bulk of the CRE flux
arises  from  distant astrophysical sources in the Galaxy,
which  leads to  a smooth power law-like background.
In addition, there is a dominant nearby source of high energy CRE which 
contributes to an excess over the smooth background.
To estimate the significance of the DAMPE excess, 
we consider a  reference  background model in which  the flux $\Phi_{b}$ 
is assumed to be a broken power law 
$\Phi_{b}(E)=N_{b} (E/E_{\tx{brk}})^{-\gamma}$, 
where $\gamma=\gamma_{1(2)}$ for 
the electron energy below (above) a break energy $E_{\text{brk}}$, 
and $N_{b}$ is a normalization factor. 
The flux of  the  excess is parametriezed as  Gaussian 
$\Phi_{s}(E)=N_{s} \exp[(E-\mu)^{2}/2\sigma^{2}]$,
where
$\mu$ and $\sigma$ are the central value and half-width, respectively,
and $N_{s}$ is a normalization factor.
We fit to the DAMPE data in the energy interval $55$~GeV--4.6~TeV.
In total 32 data points are included.
The model parameters are determined through 
minimizing the $\chi^{2}$-function defined as follows
\begin{align}
\chi^{2}=\sum_{i}
\frac{(\Phi_{b,i}+\Phi_{s,i}-\Phi_{\text{exp},i})^{2}}{\sigma_{\text{exp},i}^{2}},
\end{align}
where $\Phi_{b(s),i}$ is the theoretical value of the  
background (excess)  {\it averaged} over the width of the $i$-th energy bin,
$\Phi_{\text{exp},i}$ and $\sigma_{i}$ are the measured mean value 
and error of the flux in the energy bin.
The effect of finite energy resolution is neglected as the energy resolution
of DAMPE is quite high, better than 1.5$\%$ at 800~GeV.
The minimization of $\chi^{2}$ is performed using the MINUIT package.
The fit results in the scenarios of background-only and background plus the 
excess are  summarized in \tab{tab:gaussfree}.
It can be seen that in the background-only scenario
while the power index $\gamma_{1}$ is well constrained 
with uncertainty within $\sim 0.3\%$,
the uncertainty in $\gamma_{2}$ is much  larger around $\sim 5\%$,
which is due to the limited statistics at high energies.
Compared with the background-only scenario, the inclusion of 
a Gaussian excess leads to a reduction of $\Delta\chi^{2}=13.7$, 
roughly corresponding to  $\sim 3.7\sigma$ local significance.
Since  the excess is observed in a single energy bin, 
the lower limit of the width $\sigma$ cannot be determined,
namely, the possibility of an electron-line is not ruled out for the moment.
We find that at $68\% ~(95\%)$~C.L., 
the upper limit on the width is $\sigma \leq 60.4~(89.6)$~GeV.
In \fig{fig:flux_SNR_multi}, we show the best-fit fluxes for fixed
$\sigma=20,\ 60$ and 90~GeV, the corresponding $\chi^{2}$ values are
$12.2$, 13.2 and 16.2, respectively.


\begin{table}[htbp]
	\centering
		\begin{tabular}{c|ccccccc|c}
		\hline\hline
		   & $E_{\text{brk}}$ (GeV) & $\gamma_1$ & $\gamma_2$ & $N_b $ & $\mu$ (GeV) &$\sigma$ (GeV) & $N_s$& $\chi^2/\text{d.o.f.}$ \\
		 \hline
		 background-only & $884^{+93}_{-93}$ & $3.10\pm 0.01$ & $3.95^{+0.19}_{-0.16}$& $1.91^{+0.81}_{-0.52}$  & ...&...&...&25.85/28 \\
		background+excess & $863^{+74}_{-90}$& $3.10 \pm0.01$ & $4.03^{+0.19}_{-0.18}$ & $2.06^{+0.86}_{-0.48}$ & $1418.2^{+95.4}_{-99.9}$ & $0^{+60.36}_{-0.0}$ & $0.643$ &12.18/25 \\
		\hline\hline
		\end{tabular}
	\caption{
	Best-fit parameters in the scenarios of background-only and background plus a 
	Gaussian excess. The normalization constants $N_b, N_s$ are in units of 
	$10^{-7}$ GeV$^{-1}$m$^{-2}$s$^{-1}$sr$^{-1}$.}
\label{tab:gaussfree}
\end{table}


\begin{figure}[h]
	\centering
	\includegraphics[width=0.4\columnwidth]{flux_fit.eps}
	\caption{
	Best-fit energy spectra in the scenarios of background-only and
	the  background plus a Gaussian signal with width
	$\sigma=20$, 60 and 90~GeV, respectively.
	The data of  AMS-02 [2], %~\cite{Aguilar:2014fea},
	Ferm-LAT [3], %~\cite{Abdollahi:2017nat} 
	and 
	DAMPE [4] %~\cite{Chang:2017xx} 
	are also shown.
	}
	\label{fig:flux_SNR_multi}
\end{figure} 

\subsection{B. Continuous sources}

\subsubsection{Point-like sources}
The fit results for the scenarios of continuous point-like sources for 
three representative choices of distances $r=0.1$, 0.2 and 0.3~kpc,
together with the best-fit background parameters  are 
summarized in \tab{tab:pointDM},
where the break energy of the background is fixed at 
the best-fit value in the background-only scenario, i.e., $E_{\text{brk}}=884$~GeV.
The propagation parameters $D_{0}$, $\delta$ and $b_{0}$ are given in the letter.
\begin{table}[htbp]
	\centering
		\begin{tabular}{c|ccccc|c}
		\hline\hline
		  $r$ [kpc] & $Q_0\ (10^{32} \text{ s}^{-1})$ & $E_0$ (GeV)& $\gamma_1$ & $\gamma_2$ & $N_b$  & $\chi^2/\text{d.o.f.}$ \\
\hline
0.1 & $4.67^{+1.39}_{-1.48}$& $1446^{+32}_{-77}$& $3.10\pm0.01$ & $4.08^{+0.14}_{-0.13}$& $1.90\pm0.03$  & 14.18/27 \\
0.2 & $11.84^{+3.82}_{-4.00}$ & $1525^{+34}_{-19}$ & $3.10\pm0.01$ & $4.09^{+0.15}_{-0.13}$ & $1.90\pm0.03$ & 15.81/27 \\
0.3 &  $20.78^{+8.28}_{-8.62}$ & $1554^{+34}_{-33}$ & $3.10\pm0.01$ & $4.09^{+0.16}_{-0.14}$ & $1.89^{+0.03}_{-0.04}$ & 19.15/27 \\
\hline\hline
\end{tabular}
	\caption{
	Best-fit parameters  for continuous point-like sources with distance $r$=0.1, 0.2 and 0.3 kpc.
	The normalization constant $N_b$ is in units of  
	$10^{-7}$GeV$^{-1}$m$^{-2}$s$^{-1}$sr$^{-1}$.
	The best-fit values of $Q_{0}$ correspond to the DM annihilation cross section 
	$\sigmav=4.82\times10^{-27}$, $1.36\times10^{-26}$, 
	and $2.48\times10^{-26} \text{cm}^3\text{s}^{-1}$ for
	$r=0.1$, 0.2 and 0.3~kpc, respectively.
	}
\label{tab:pointDM}
\end{table}
In the case where the distance $r$ is allowed to vary freely, 
the value of $\chi^{2}$ decrease with decreasing $r$ and gradually reaches 
a minimal value  $\chi^{2}_{\tx{min}}$=13.0 in the limit  $r\to 0$. 
This result is again related to the fact that the excess is observed in a single
energy bin and the $\chi^{2}$ value is based on the bin-average.
We thus define the  allowed regions in $(r, E_{0})$ and $(r,Q_{0})$ planes  
at $68\%$ ($95\%$)~C.L. for two parameters,
corresponding  to $\Delta \chi^{2}=2.3$ (5.99),
which are shown  in \fig{fig:pointDM-contour}.
It can be seen that the allowed distance is constrained to be
$r \lesssim 0.3$~kpc at $95\%$~C.L.
The value of $E_{0}$ is constrained in the $1.35-1.55$~TeV range which is compatible with the 
width of the excess.
\begin{figure}[htb]
	\centering
	\includegraphics[width=0.4\textwidth]{pointDM_r_E0_log.eps}
	\includegraphics[width=0.4\columnwidth]{pointDM_r_N0_log.eps}
	\caption{		
	Left) 
	Allowed regions at $68\%$ and $95\%$ C.Ls. in $(r, E_{0})$ plane for 
	continuous point-like sources.
	Right)
	The same as Left, but for the allowed regions in $(r, Q_{0})$ plane.	
	}\label{fig:pointDM-contour}
\end{figure}



\subsubsection{Mini-spikes}
We consider the scenario of IMBHs originated  from massive objects formed directly during
the collapse of primordial gas (pGas) in early-forming halos.
The initial DM profile of the halo prior to  the adiabatic growth of the blackhole  is 
assumed to be NFW [20] %~\cite{Navarro:1996gj},
\begin{align}
\rho(r)=\frac{\rho_{s}}{\left( \frac{r}{r_{s}} \right)^{\gamma} \left( 1+\frac{r}{r_{s}} \right)^{3-\gamma} }  ,
\end{align}
where $\rho_{s}$ and $r_{s}$ are reference energy density and distance, respectively.
$\gamma$ is the inner slope of the profile, and in the standard NFW profile $\gamma=1$.
After the adiabatic growth of the blackhole, the spiked profile is a power law%
[21] %~\cite{Gondolo:1999ef}
\begin{align}
\rho_{sp}(r)=\rho(r_{sp})\left( \frac{r}{r_{sp}}\right)^{-\gamma_{sp}} ,
\end{align}
where
$r_{sp}\sim \tx{pc}$ and
$\gamma_{sp}=(9-2\gamma)/(4-\gamma)$.
For $\gamma=1$, $\gamma_{sp}=7/3$.
The spiked DM profile is apparently divergent at small radii.
However, the DM annihilation sets an upper limit $\rho_{\text{lim}}$ on the DM density
of order 
$\rho_{\text{lim}}=\rho_{sp}(r_{\text{lim}})\approx m_{\chi}/\langle\sigma v\rangle t$ with $t$ the age of the mini-spike.
Thus $\rho_{sp}(r)$ should have an inner cut-off at radius $r_{\text{cut}}$,
which is defined as
$r_{\text{cut}}=\text{max}\{4 R_{\text{Schw}}, r_{\text{lim}}\}$,
where $R_{\text{Schw}}$ is the Schwarzschild radius of the IMBH.
For typical value of $r_{\text{lim}}\approx 10^{-3}$~pc, 
which is larger than the typical Schwartzchild radius  
$R_{\text{Schw}}\approx 2.95~\text{km}~ (M_{bh}/M_{\odot})$.
Thus we take $r_{\tx{cut}}=r_{\tx{lim}}$.
The source term from the annihilation of Majorana DM particles (with mass $m_{\chi}$) 
directly into $e^{+}e^{-}$ can be written as
\begin{align}
Q_{e}(E)=\frac{N_{e} \langle\sigma v\rangle}{2 m_{\chi}^{2}}  
\delta(E-E_{0})
\int_{r_{\text{cut}}}^{r_{sp}} \rho_{sp}^{2}(r') 4\pi r'^{2} dr' ,
\end{align}
where $N_{e}=2$ is the number of CRE per DM annihilation, and  
$E_{0}=m_{\chi}$.
After integrating over $r'$, the source term can be written as 
\begin{align}
Q_{e}(E)=\frac{N_{e} \langle\sigma v\rangle L}{2 m_{\chi}^{2}} 
\delta(E-E_{0})  ,
\end{align}
where  $L$ is the annihilation luminosity
\begin{align}
L
\equiv\int_{r_{\text{cut}}}^{r_{sp}} \rho_{sp}^{2}(r') 4\pi r'^{2} dr' 
=\frac{4\pi r_{sp}^{3}}{2\gamma_{sp}-3}\rho^{2}(r_{sp}) 
\left( \frac{r_{\text{cut}}}{r_{sp}}\right)^{3-2\gamma_{sp}} .
\end{align}
In the case of  $\gamma_{sp}=7/3$, it can be written as
\begin{align}
Q_{e}(E)=\frac{6 \pi N_{e} \langle\sigma v\rangle}{5 m_{\chi}^{2}} 
\rho^{2}(r_{sp}) r_{sp}^{14/3} r_{\text{cut}}^{-5/3} 
\delta(E-E_{0})  ,
\end{align}
which leads to the expression  of Eq.~(3) %\eq{eq:Q0} 
in the letter.








\subsubsection{DM subhalos}
The DM subhalos located at  the inner volume of the Galaxy
may experience a significant degree of mass loss due to tidal stripping,
as they encounter other subhalos more frequently.
Therefore we  adopt a tidally truncated  density profile for DM subhalos
[25-27].
%\cite{
%	Kazantzidis:2003hb,%
%	Penarrubia:2007zx,%
%	Hooper:2016cld%
%}.
\begin{align}
\rho(r)	
=
\frac{\rho_{0}}{(r/\tx{kpc})^{\gamma}}e^{-r/R_{b}}  .
\end{align}
The parameters  $\rho_{0}$, $\gamma$ and $R_{b}$ can be parametrized as 
$\gamma(d)=g_{1}(d/\text{kpc})^{g_{2}}$ and $R_{b}=b_{1}(d/\text{kpc})^{b_{2}}$,
where $d\approx 8.5$~kpc is  
the distance from the center of the subhalo to the GC.
The values of the parameters $g_{1,2}$ and $b_{1,2}$   
can be extrapolated from the N-body simulations
for different ranges of the total mass $M_{h}$. 
For fixed $r$ and $R_{b}$,
the normalization factor $\rho_{0}$ can be  determined by the subhalo total mass $M_{h}$.
Based on  a joint analysis to the Via Lactea II and  ELVIS simulations
in Ref. [27],
%\cite{
%Hooper:2016cld%
%},
we obtain the values of $\gamma$, $R_{b}$ and $\rho_{0}$ for selected values of 
$M_{h}$, which are listed in \tab{tab:subhalo-profile}.


\begin{table}[htb]
	\centering
		\begin{tabular}{c|ccc}
		\hline\hline
		$M_{h}~(M_{\odot})$ & $\gamma$ & $R_{b}~\text{(kpc)}$ & $\rho_{0}$\\
		\hline
		$1\times 10^{6}$ 	& 0.81 	&	0.054	& 1.8\\
		$1\times 10^{7}$ 	& 0.78 	&	0.096 	& 5.3 \\
		$3\times 10^{7}$ 	& 0.65	&	0.19         & 4.5 \\
		$1\times 10^{8}$ 	& 0.75	&	0.22         & 9.4 \\
		\hline\hline
		\end{tabular}
	\caption{Prameters of DM subhalo density profiles extrapolated to the distance $d\approx 8.5$~kpc 
	for difference choices of the subhalo masses
	from [27]. %~\cite{Hooper:2016cld}.
	The reference density $\rho_{0}$ is in units of $\tx{GeV}\cdot\tx{cm}^{-3}$.
	}
\label{tab:subhalo-profile}
\end{table}
For a typical subhalo masses $M_{h}=10^{7} M_{\odot}$,  
the best-fit parameters 
for three choices of ($r$,$m_{\chi}$) are shown in \tab{tab:extendDM},
which correspond to the three cases plotted in the middle panel of Fig.~2. %\fig{fig:flux} in the letter.
Similar to the case of continuous point-like sources,
when $r$ is allowed to vary freely, 
the value of $\chi^{2}$ decreases with decreasing distance and
approaches a minimal value $\chi^{2}_{\text{min}}=13.2$  in the limit of  $r\to 0$. 
The  allowed regions at $68\%$ ($95\%$) C.L. in $(r, m_{\chi})$ and $(r, \sigmav)$ planes 
are shown  in \fig{fig:extendDM-contour}.
The allowed range of distance is $r \lesssim 0.3$~kpc at $95\%$~C.L., 
which is almost the same as that in the case of mini-spikes.
The fit results for other subhalo masses 
$M_{h}=10^{6} M_{\odot}$, $3\times 10^{7} M_{\odot}$ and $10^{8} M_{\odot}$
are shown separately in \tab{tab:extendDM2}.
The results show that the DAMPE excess can be well fitted for a large range of $M_{h}$.
Increasing the value of $M_{h}$ leads to a decrease of the DM annihilation cross section.


\begin{table}[htbp]
	\centering
		\begin{tabular}{cc|cccc|c}
		\hline\hline
		  $r$ [kpc] & $m_\chi$[GeV] & $\langle \sigma v \rangle$  & $\gamma_1$ & $\gamma_2$ & $N_b$  & $\chi^2/\tx{d.o.f.}$ \\
				 \hline
				 0.1& 1510 & $ 1.04\pm0.30$  & $3.10\pm0.01$ & $4.08^{+0.14}_{-0.13}$ & $1.90\pm0.03$ & 13.37/28\\
		0.2 &1520 & $2.55^{+0.82}_{-0.83}$ & $3.10\pm0.01$& $4.10^{+0.15}_{-0.13}$&  $1.90\pm0.03$ & 15.90/28\\
		0.3 &1550 & $4.62^{+1.83}_{-1.87}$ & $3.10\pm0.01$& $4.10^{+0.16}_{-0.14}$& $1.89\pm0.03$& 19.20/28\\
		  \hline\hline
		\end{tabular}
	\caption{
	Best-fit parameters for three choices of $(r, m_{\chi})$ values for a source of  DM subhalo 
	with mass $M_{h}=10^{7} M_{\odot}$. The corresponding subhalo density parameters are listed
	in \tab{tab:subhalo-profile}.  
	$\langle \sigma v\rangle$ is in units of $10^{-26}\text{cm}^{3}\text{s}^{-1}$ and 
	$N_{b}$ is in units of $10^{-7}$~GeV$^{-1}$m$^{-2}$s$^{-1}$sr$^{-1}$.
	}
\label{tab:extendDM}
\end{table}



\begin{table}[htbp]
	\centering
		\begin{tabular}{ccc|cccc|c}
		\hline\hline
		 $M [M_\odot]$ & $r$ [kpc] & $m_\chi$[GeV] & $\langle \sigma v \rangle$  & $\gamma_1$ & $\gamma_2$ & $N_b$  & $\chi^2/\tx{d.o.f.}$ \\
		  \hline
				$10^6$ & 0.1& 1510 & $ 15.15\pm4.32$  & $3.10\pm0.01$ & $4.08^{+0.14}_{-0.13}$ & $1.90\pm0.03$ & 13.26/28\\
		-- & 0.2 &1520 & $39.88^{+12.79}_{-12.83}$ & $3.10\pm0.01$& $4.10^{+0.15}_{-0.13}$&  $1.90\pm0.03$ & 15.81/28\\
		-- & 0.3 &1550 & $72.9^{+28.73}_{-29.29}$ & $3.10\pm0.01$& $4.10^{+0.16}_{-0.13}$& $1.89\pm0.03$& 19.15/28\\ 
		\hline
		$3\times10^7$ & 0.1& 1510 & $ 1.49\pm0.44$  & $3.10\pm0.01$ & $4.08^{+0.14}_{-0.13}$ & $1.90\pm0.03$ & 13.88/28\\
		-- & 0.2 &1520 & $2.95^{+0.93}_{-0.94}$ & $3.10\pm0.01$& $4.11^{+0.15}_{-0.14}$&  $1.90\pm0.03$ & 16.43/28\\
		-- & 0.3 &1550 & $4.93^{+2.03}_{-2.05}$ & $3.10\pm0.01$& $4.11^{+0.16}_{-0.14}$& $1.89^{+0.03}_{-0.04}$& 19.60/28\\
		\hline
		$10^8$ & 0.1& 1510 & $ 0.17\pm0.05$  & $3.10\pm0.01$ & $4.08^{+0.14}_{-0.13}$ & $1.90\pm0.03$ & 13.89/28\\
		-- & 0.2 &1520 & $ 0.33\pm0.11$ & $3.10\pm0.01$& $4.11^{+0.15}_{-0.14}$&  $1.90\pm0.03$ & 16.46/28\\
		-- & 0.3 &1550 & $0.55^{+0.22}_{-0.23}$ & $3.10\pm0.01$& $4.11^{+0.16}_{-0.14}$& $1.89^{+0.03}_{-0.04}$& 19.64/28\\
		\hline\hline
		\end{tabular}
	\caption{
	The same as \tab{tab:extendDM}, but with different DM subhalo masses
	$M_{h}=10^{6} M_{\odot},~3\times 10^{7}M_{\odot}$ and $10^{8} M_{\odot}$.
	The corresponding DM subhalo density parameters are listed
	in \tab{tab:subhalo-profile}.
	 }
\label{tab:extendDM2}
\end{table}


\begin{figure}[htb]
	\centering
	\includegraphics[width=0.4\textwidth]{extendDM_r_mchi_log.eps}
	\includegraphics[width=0.4\columnwidth]{extendDM_r_sv_log.eps}
	\caption{		
	Left) 
	Allowed regions at $68\%$ and $95\%$ C.Ls. in $(r, m_{\chi})$ plane for 
	the source of DM subhalos.
	Right)
	Allowed region in $(r, \langle \sigma v\rangle)$ plane.	
	}\label{fig:extendDM-contour}
\end{figure}

\subsubsection{Contributions from the Galactic halo DM}
For DM sources, the contributions to CRE also come from the whole Galactic DM halo.
If the favoured  DM annihilation cross sections are around the thermal value
$\sigmav_{F}=3\times 10^{-26}~\tx{cm}^{3}\tx{s}^{-1}$,
the contributions to the CRE flux  from the whole Galactic DM above TeV
are found  to be  negligibly small. 
In \fig{fig:DM-Galactic-halo}, we compare the contributions
to CRE from DM substructures and DM Galactic halo for typical cases.
For calculating the contributions from Galactic DM halo,
we make use of the GALPROP v54 package [32] %~\cite{galprop}
with a reference propagation parameter set
$^SL^Z6^R20^T100000^C5$ [33] %
%~\cite{
%Ackermann:2012pya%
%}
and the Einasto DM profile with a local DM density of 
$0.4~\text{GeV}\text{cm}^{-3}$.
The contributions from remote DM subhalos in the Galaxy can be safely  neglected, 
as the N-body simulations show that they contribute only to a tiny  fraction of 
the total Galactic DM mass [22-24]. %
%~\cite{
%		Springel:2008cc,%
%		Diemand:2008in,%
%		Garrison-Kimmel:2013eoa%
%}.
\begin{figure}[!htb]
	\centering
	\includegraphics[width=0.8\columnwidth]{flux_log_Gal.eps}
	\caption{
	Left)
	Comparison of the CRE energy spectrum from a mini-spike and that from the whole Galactic halo DM
	with the same DM annihilation cross section. 
	The parameters of the mini-spike correspond to the case of $r=0.2$~kpc in \tab{tab:pointDM}.
	Right)
	Comparison of  electron energy spectrum from a nearby DM subhalos 
	and that from the whole Galactic halo DM with the same annihilation cross section. 
	The parameters of the DM subhalo correspond to the case of $r=0.2$~kpc in \tab{tab:extendDM}.
	}
	\label{fig:DM-Galactic-halo}
\end{figure}



\subsection{C. Burst-like sources}
For the burst-like sources, 
the best-fit parameters for three representative choices  of 
$(r, \alpha)$ shown in the right panel of Fig.~2 in the letter  %\fig{fig:flux} in the letter 
are summarized  in  \tab{tab:SNR},
where the total released energy $E_{\text{tot}}$ is derived from the value of $N_{0}$,
using
$E_{\text{tot}}=\int_{E_{\tx{min}}} E Q(r,t,E) dE d^{3}r dt$ with a lower cutoff $E_{\tx{min}}=0.1$~GeV.
When $r$ and $\alpha$ are allowed to vary freely in the fit, 
the $\chi^{2}$ gradually approaches a minimal value  of  12.5 for $\alpha\to 0$.
The allowed regions in ($r$, $\alpha$) at $68\%$ and $95\%$~C.L.
are shown in  \fig{fig:r-alpha-dependenc}.
In the region $r \leq r_{d}(E_{\text{max}})$, 
the favoured $\alpha$ is less than 0.65 at $95\%$~C.L. which is highly 
insensitive to the distance, as the spectral shape is dominated by the effect
of ``phase-space shrinking''. 
In the region $r > r_{d}(E_{\text{max}})$, the effect of energy-dependent diffusion
becomes important and the allowed value of $\alpha$ can be larger which can reach 
1.3 (1.6) at $r\approx 3 (4)$~kpc.
By imposing the condition of $E_{\text{tot}}<10^{51}(10^{53})$~erg,
the distance $r$ is restricted in range $r\lesssim 3$(4)~kpc.
Together with the required  $t=(1.5-1.7)\times 10^{5}$~yr,
we find the 7 candidate pulsars in the ATNF catalogue.
The relevant parameters of these candidates are summarized in \tab{tab:ATNF}.



\begin{table}[htbp]
	\centering
		\begin{tabular}{cc|ccccc|c}
		\hline\hline
$r$ [kpc] & $\alpha $& $t$ (Myr) & $N_0\ (\text{GeV}^{-1})$ & $\gamma_1$ & $\gamma_2$ & $N_b$  & $\chi^2/\text{d.o.f.}$ \\
		  \hline
		  1.0 & 0.5 & $0.147^{+0.002}_{-0.003}$ & $ 3.72^{+1.18}_{-1.21}\times10^{44}$  & $3.10\pm0.01$ & $4.09^{+0.15}_{-0.13}$ & $1.90\pm0.03$ & 15.56/27\\
		  2.0 & 0.7 & $0.147^{+0.002}_{-0.003}$  & $2.33^{+0.74}_{-0.79}\times10^{46}$  & $3.10\pm0.01$ & $4.09^{+0.15}_{-0.13}$ & $1.90\pm0.03$ & 15.62/27\\
		  3.0 & 1.3 & $0.146^{+0.002}_{-0.003}$  & $3.10^{+1.13}_{-1.26}\times10^{50}$  &  $3.10\pm0.01$ & $4.10^{+0.16}_{-0.14}$ & $1.90\pm0.03$ & 17.51/27\\
		\hline\hline
		\end{tabular}
	\caption{ 
	Best-fit parameters for three representative choices of $(r, \alpha)$ in the case of 
	burst-like sources. The cutoff energy is fixed at $E_c=5\times10^{4}$ GeV.
	$N_b$ is in units of $10^{-7}$GeV$^{-1}$m$^{-2}$s$^{-1}$sr$^{-1}$. 
	The total released energy for the cases of $r=1$, 2 and 3~kpc is 	
	 $ E_{tot}=2.53^{+0.80}_{-0.82}\times10^{48}$,
	 $2.18^{+0.70}_{-0.74}\times10^{49}$ and
	 $9.56^{+3.49}_{-3.88}\times10^{50}$~erg, respectively.
	}
\label{tab:SNR}
\end{table}

\begin{figure}[htb]
	\centering
	\includegraphics[width=0.4\columnwidth]{SNR_r_alpha_log.eps}	
	\caption{
	 Allowed regions at $68\%$ and $95\%$ C.L. in $(r, \alpha)$ plane for the burst-like point source.
	 The region excluded at large distances    $r\gtrsim 3$~kpc is due to the requirement that 
	 the total energy $E_{\text{tot}} <10^{51}$~erg.
	}\label{fig:r-alpha-dependenc}
\end{figure}


\begin{table}[htbp]
	\centering
		\begin{tabular}{l|cc|cccc} 
		\hline\hline
		  Source Name & $\ell$(deg) &  $b$(deg) & $r$ (kpc) & $t$ (yr) & $\dot{E}$ (erg/s)  \\
		\hline
		B0740-28 & 243.77 & -2.44 & 2.00  & $1.57\times10^{5}$ & $1.4\times10^{35}$  \\
		J0922-4949 & 272.24 & 0.16 &  2.70  & $1.54\times10^{5}$ & $4.5\times10^{33}$  \\
		J1055-6022 & 289.11 & -0.65 & 3.60 & $1.62\times10^{5}$ & $4.3\times10^{33}$  \\
		J1151-6108 & 295.81 & 0.91  & 2.22  & $1.57\times10^{5}$ & $3.9\times10^{35}$  \\
		J1509-5850 & 319.97 & -0.62 & 3.35 & $1.54\times10^{5}$ & $5.1\times10^{35}$  \\
		J1616-5017 & 332.83 & 0.29 & 3.48  & $1.67\times10^{5}$ & $1.6\times10^{34}$ \\
		J1739-3023 & 358.09 & 0.34 & 3.07 & $1.59\times10^{5}$ & $3.0\times10^{35}$ \\
				\hline\hline
	 \end{tabular}%
	 \caption{
	 	Parameters 
		of candidate pulsars  in ATNF pulsar catalog [30] %~\cite{Manchester:2004bp} 
		satisfying the conditions of 
		 age $t=0.15-0.17$~Myr  as required to reproduce 
		 the DAMPE excess, and distance $r<4$~kpc from the requirement of $E_{\tx{tot}}<10^{53}$~erg.
		The quantities  $(\ell,b)$, $r$, $t$ and $\dot{E}$ stand for the 
		 direction, distance, age and spin-down energy loss rate, respectively.
	 }
	\label{tab:ATNF}%
\end{table}%


\newpage
\subsection{D. Electron anisotropies}
For one or a few nearby sources,
the anisotropy in the arrival directions of CREs 
is dominated by the dipole term which can be approximated as 
\begin{align}
\Delta \approx \frac{3 D(E)}{c}
\left|\frac{\bigtriangledown \Phi_{\text{tot}}}{\Phi_{\text{tot}}}\right|  ,
\end{align}
where  
$\Phi_{\text{tot}}$ is the total flux of the nearby sources and the background,
and $c$ is the speed of light.
In the approximation that the background is nearly isotropic, %
the anisotropy  for the continuous point-like sources  is given by 
\begin{align}
\Delta_{\text{conti}} \approx\frac{3r}{2c}\frac{(1-\delta)b_{0}E }{[1-(E/E_{0})^{1-\delta}]}
\frac{f(r,E)}{f_{\text{tot}}(r,E)}  ,
\end{align}
where $f_{\text{tot}}=f(r,E)+f_{bg}(r,E)$ is the total distribution function.
For the burst-like source it can be written as
\begin{align}
\Delta_{\text{burst}} \approx\frac{3 r}{2ct}\frac{(1-\delta) E/E_{\text{max}}}{1-\xi(E)^{1-\delta}}
\frac{f(r,E)}{f_{\text{tot}}(r,E)}  .
\end{align}
The predicted anisotropies according to the best-fit parameters in each case of 
the three type of sources listed in \tab{tab:pointDM}, \tab{tab:extendDM} and \tab{tab:SNR} are shown 
in \fig{fig:anisotropy_DAMPE}.
The anisotropy of the background CRE depends on the assumed spatial distribution of 
primary CRE sources, the typical values estimated from GALPROP is of $\mathcal{O}(10^{-3})$
[31]
%~\cite{
%Abdollahi:2017kyf%
%}
which is significantly smaller than that generated from nearby sources.

\begin{figure}[htb]
\begin{center}
\includegraphics[width=1\textwidth]{anisotropy.eps}
\caption{
Predictions for electron anisotropies in the arrival directions  (dotted curves).
Left) for continuous point-like sources with three parameter sets 
listed in \tab{tab:pointDM}
Middle) for continuous extended sources with parameter sets 
listed in \tab{tab:extendDM}.
Right) for burst-like sources with  parameter sets listed in \tab{tab:SNR}.
For all the type of sources, 
the dashed curves correspond to the anisotropies convoluted with an energy resolution of $15\%$,
and the horizontal lines indicate the averaged value in the energy bin  0.55-2~TeV, 
the corresponding  current upper limits from Fermi-LAT are also shown.
}
\label{fig:anisotropy_DAMPE}
\end{center}
\end{figure}


\subsection{E. Associated gamma-ray signals}
\subsubsection{Prompt gamma-rays}
Prompt photons are generated from DM annihilation through FSR of final 
state charged leptons, which is of particular importance for electron final states.
The flux of FSR photon from the annihilation of Majorana DM particles (with mass $m_{\chi}$)
per solid angle $d\Omega$ is given by
\begin{align}
\frac{d\Phi_{\gamma}}{d\Omega dE_{\gamma}}
=
\frac{\langle \sigma v \rangle}{8\pi m_{\chi}^{2}} 
\frac{dN_{\gamma}}{dE_{\gamma}}
\int_{\text{l.o.s}} \rho^{2}(r') ds  .
\end{align}
where $\rho(r')$ is the DM density profile,
and the integration is performed along the light-of-sight $s$ which is related to $r'$ as
$r'=\sqrt{s^{2}+r^{2}-2 s r \cos\theta}$ with $r$ the distance from the detector to the center of the DM halo
and $\theta$ is the angle away from the center.
The spectrum  for $e^{+}e^{-}$ final state is given by
\begin{align}
\frac{dN_{\gamma}}{d x}
=
\frac{\alpha_{\text{em}}}{\pi}
\frac{1+(1-x)^{2}}{x}
\left[ 
	-1+
	\ln\left( \frac{4(1-x)}{\epsilon^{2}}\right)
\right]  ,
\end{align} 
where  $x=E_{\gamma}/m_{\chi}$, $\epsilon=m_{e}/m_{\chi}$ and
$\alpha_{\text{em}}$ is the fine structure constant.
An important feature of the FSR photon spectrum is that 
it can be approximated as a power law at low energies
\begin{align}
\frac{dN_{\gamma}}{d E}\propto E^{-1}, 
\ \ \ \ \ (~\tx{for} E_{\gamma}\ll m_{\chi}~)
\end{align}
and has a sharp cutoff at $E_{\gamma}=m_{\chi}$.
In numerical calculations we use the package PYTHIA 8.1 which 
adopts the same formula.
Since mini-spikes are point-like, 
the flux of prompt $\gamma$-rays from DM annihilation in mini-spikes 
can be approximated as
\begin{align}
\frac{d\Phi_{\gamma}}{dE_{\gamma}}=
\frac{\langle\sigma v\rangle}{2 m_{\chi}^{2}} \frac{L}{4\pi r^{2}}
\frac{dN_{\gamma}}{dE_{\gamma}} ,
\end{align}
where $L$ is the annihilation luminosity given in the letter,
and $r$ is the distance from the detector to the center of the mini-spike.
The prompt $\gamma$-rays from DM subhalos  and Galactic DM are calculated by
direct integration over the DM density profile.
Unlike the case of the CRE, for prompt $\gamma$-rays, the contribution from the 
whole Galactic halo  DM is significant. 









\subsubsection{ICS  gamma-rays}


The distribution of final state photon from the  ICS process 
$e(E_{e})+\gamma(E_{\gamma})\rightarrow e'(E'_{e})+\gamma'(E'_{\gamma})$, 
where $E_{e}(E'_{e})$, $E_{\gamma}(E'_{\gamma})$ are the energies of initial (final) 
state electron and photon, respectively,  is given by 
\begin{equation}
\frac{dN_{\gamma'}}{dE_{\gamma'}dt}=2\pi \alpha^{2}_{\text{em}}\frac{u_{\gamma}}{E_e^2 E_{\gamma}^2}f_{\tx{IC}}(q,\varepsilon), 
\end{equation}
where $u_{\gamma}$ is the energy density of the initial photons.
The function $f_{\tx{IC}}(q,\epsilon)$ is 
\begin{equation}
f_{\tx{IC}}(q,\varepsilon)=2q\ln(q)+(1+2q)(1-q)+\frac{(\varepsilon q)^2}{2(1+\varepsilon q)}(1-q)  ,
\end{equation}
where 
$\varepsilon=E_{\gamma}'/E_e$, 
$\Gamma=4E_{\gamma}E_e/m_e^2$  and  
$q=\varepsilon/\Gamma(1-\varepsilon)$.
The final energy  satisfies the relation
$E_{\gamma}/E_e\leq \varepsilon \leq \Gamma/(1+\Gamma)$.
The photon flux $\Phi_{\gamma}=dN_{\gamma}/dA dt$ obtained for 
a given line of sight is [42] %
%\cite{
%Meade:2009iu%
%}
\begin{equation}
\frac{d^2\Phi_{\gamma'}}{dE_{\gamma'}d\Omega}
=
\frac{1}{2}\alpha_{\text{em}}^2  \int_{\text{l.o.s}}\,ds \int\int \frac{dE_e}{E_e^2}\frac{dE_{\gamma}}{E_{\gamma}^2}
f(r,E_{e})%
u_{\gamma}(E_{\gamma})
f_{\tx{IC}}  ,
\end{equation}
where
$f(r,E_{e})$ is the density of initial state electrons.
The energy density $u_{\gamma}(E_{\gamma})$ is assumed to be black body like
\begin{equation}
u_{\gamma}(E_{\gamma})=\mathcal{N}\frac{E_{\gamma}^3}{\pi^2\left(e^{E_{\gamma}/T}-1\right)} ,
\end{equation}
where $\mathcal{N}$ is a position-dependent normalization constant.
The three major components of the ISRF are
cosmic microwave background with temperature
$T_{\text{CMB}}=2.35\times 10^{-4}$~eV,
the infrared radiation produced by the absorption and  
re-emission of star light by 
the interstellar dust
with temperature
$T_{\text{dust}}=3.5\times 10^{-3}$~eV
and the start light with temperature
$T_{\text{star}}=0.3$~eV.
For the nomalization factors, we use the values
$\mathcal{N}_{\text{CMB}}=1$, 
$\mathcal{N}_{\text{dust}}=2.5\times 10^{-5}$ and 
$\mathcal{N}_{\text{star}}=6.0\times 10^{-13}$, respectively,
which are the values at the solar neighbourhood $d\approx 8$~kpc interpolated from [34].
%\cite{
%Porter:2008ve%
%}.






\subsubsection{Gamma-ray signals of  mini-spikes, DM subhalos and burst-like sources}
In the top-left panel of \fig{fig:GAMMA_DM_energy}, 
we show the spatial extension of the $\gamma$-ray flux 
(with energy above 1~GeV) of the mini-spike
with the parameters corresponding to 
the best-fit values  in the case of $r=0.2$~kpc  in \tab{tab:pointDM}
and the direction of the location  coincides with the GC.
The corresponding DM annihilation cross section is 
$\langle \sigma v\rangle=1.36\times 10^{-26}~\text{cm}^{3}\text{s}^{-1}$ 
according to  Eq.~(3) in the letter. %\eq{eq:Q0} in the letter.
In the figure, the contributions of prompt and ICS photons are shown explicitly.
For mini-spikes, the spatial extension is $\ll 1^{\circ}$ within $68\%$ containment. 
Thus they can be treated as point-like sources.
For the three cases considered in \tab{tab:pointDM}, 
the total fluxes integrated in the energy range $1-100$~GeV
are in the range $(0.47-1.01)\times 10^{-10}~\text{cm}^{-2}\text{s}^{-1}$,
which are within the current sensitivity of Fermi-LAT in point-source searches.
For the point-source searches, the energy spectrum is dominated by the FSR photons.
Thus the expected spectrum should have a power index close to $\sim 1$. 
In the Fermi-LAT 3FGL catalog of unassociated point sources
(sources that have not been associated with emission observed at other wavelengths),
we find 6 candidate 
sources with low power indexes (index $\alpha$ can reach 1.0 withint $2\sigma$  error ), 
which are listed in the \tab{tab:3FGL}.
It is also possible that the mini-spike is located in the direction of the low Galactic
latitudes where the Fermi-LAT sensitivity  is much lower. 
\begin{table}[htbp]
	\centering
		\begin{tabular}{l|cc|cccc|c} 
		\hline\hline
        Source Name & $\ell$(deg) & $b$(deg) & $\Phi\ (10^{-10}\text{cm}^{-2}\text{s}^{-1})$ & $\alpha$ & Significance ($\sigma$) \\
     \hline
     J0603.3+2042 & 189.124 & -0.690422 & $6.180$ & $1.50\pm0.50$ & 4.37156 \\
     J1250.2-0233 & 302.344 & 60.3066 & $0.926\pm0.521$ & $1.10\pm0.30$ & 5.12018 \\
     J2209.8-0450 & 55.6854 & -45.5583 & $1.784\pm0.974$ & $1.27\pm0.32$ & 6.46061 \\
     J1705.5-4128c & 345.052 & -0.281031 & $48.379$ & $2.77 \pm1.06$ & 5.12339 \\
     J2142.6-2029 & 31.1422 & -46.5567 & $1.245\pm0.683$ & $1.52\pm0.33$ & 4.05409 \\
     J2300.0+4053 & 101.243 & -17.2428 & $1.719\pm0.746$ & $1.51\pm0.26$ & 5.2434 \\
		\hline\hline
	 \end{tabular}%
	 \caption{
	 Parameters of selected point sources in the Fermi-LAT 3FGL  ``unassociated'' point-source catalog [35], %~\cite{Acero:2015hja}, 
	 which have power-law type of spectrum with low power indices (the power
	 index $\alpha$ can reach one with $2\sigma$ error) and are possibly due to  FSR of
	 DM particles in mini-spikes. The total flux is integrated from 
	 1 to 100 GeV. %
	}
	\label{tab:3FGL}%
\end{table}%



In the top-right panel of \fig{fig:GAMMA_DM_energy}, 
we show the spatial extension of the DM subhalo
with the parameters corresponding to 
the best-fit values  in the case of $r=0.2$~kpc of \tab{tab:extendDM},
and the direction of the location  coincides with the GC.
It can be seen that the $\gamma$-rays are significantly extended to 
a few tens of degrees, as they are dominated by the ICS photons.
In \fig{fig:GAMMA_DM_energy}, for a comparison, 
we also show the prompt and ICS photons from 
the Galactic halo DM with the same DM annihilation cross section,
which is calculated by using  GALPROP with the same parameter set
as that use in obtaining the \fig{fig:DM-Galactic-halo}.
Unlike the case of electrons, the contribution to  diffuse $\gamma$-ray
from the Galactic halo DM is very significant.


The $\gamma$-ray energy spectra of mini-spikes and DM subhalos 
are shown in the left and right panels of \fig{fig:GAMMA_DM_energy}, 
respectively. 
The spectra are averaged over two regions of interest (ROI)
which represent the low and high Galactic latitude limits,
one is a circular ROI with an angular radius $30^{\circ}$ centered
at the GC ($\ell=0^{\circ}$ and $b=0^{\circ}$), 
the other one has the same shape but centered at 
high latitude   ($\ell=0^{\circ}$ and $b=90^{\circ}$).
The components of prompt and ICS photons are explicitly shown.
The contributions from the Galactic halo DM are included in the same way.
At low latitudes  close to the Galactic disk
the contributions of ICS photons are significant,
which is expected as the ISRF has large intensity.
After summing up all the components of the contributions, 
the total spectrum scales with energy approximately as $E^{-2}$
with a sharp cutoff at the DM particle mass $m_{\chi}$.
On the other hand, at high latitudes, the FSR photons are dominant at
high energies, and the spectrum scales approximately as $E^{-1}$.


The spatial extension and energy spectra of $\gamma$-rays  
in all the cases of  the three type of sources 
listed in Tabs.~\ref{tab:pointDM}, \ref{tab:extendDM} and \ref{tab:SNR} are
summarized  in  \fig{fig:GAMMA_cases_30},
together with the calculated Galactic diffuse $\gamma$-ray backgrounds 
in the same ROIs.
The figure shows that at high latitude $b=90^{\circ}$, 
the predicted flux of burst-like source can exceed the diffuse background.
Since the background is in an overall agreement with the Fermi-LAT data,
this scenario is disfavoured. 
We thus conclude that the sources are more likely to be located 
at relatively low Galactic latitudes where the background is large.


\begin{figure}[!htbp]
\begin{center}
\includegraphics[width=0.8\textwidth]{GAMMA_detail_deg.eps}
\includegraphics[width=0.8\textwidth]{GAMMA_detail_f_30.eps}
\includegraphics[width=0.8\textwidth]{GAMMA_detail_t_30.eps}
\caption{
Left panels)
Top: 
spatial extension of the $\gamma$-ray flux (above 1~GeV) of 
the  mini-spike with parameters corresponding to the case of $r=0.2$~kpc in \tab{tab:pointDM}.
The contributions from prompt and ICS photons are shown explicitly, 
together with that from the Galactic halo DM.
Middle:
averaged energy spectrum of $\gamma$-rays over  
the ROI of a circular region with angular radius $30^{\circ}$ centered
at GC $(\ell=0^{\circ}, b=0^{\circ})$ 
for the same parameters.
Bottom:
the same as middle panels but for the ROI centered at  ($\ell=0^{\circ}$ and $b=90^{\circ}$).
Right panels)
The same as left, but for the DM subhalos with 
parameters corresponding to the case of $r=0.2$~kpc in \tab{tab:extendDM}.
}
\label{fig:GAMMA_DM_energy}
\end{center}
\end{figure} 










\begin{figure}[!htbp]
\begin{center}
\includegraphics[width=0.9\textwidth]{GAMMA_cases_deg.eps}
\\
\includegraphics[width=0.9\textwidth]{GAMMA_cases_f_30.eps}
\\
\includegraphics[width=0.9\textwidth]{GAMMA_cases_t_30.eps}
\caption{
Predicted $\gamma$-ray spatial extension and energy spectra 
for all the three parameter sets in the three type of sources considered in 
Fig.~2 %\fig{fig:flux} 
in the letter with all the components summed up together.
Left panels) for mini-spikes with parameters in \tab{tab:pointDM}
Center panels) for DM subhalos with parameters in \tab{tab:extendDM}
Right panels) for burst-like sources with parameters in \tab{tab:SNR}.
The ROI are the same as that in \fig{fig:GAMMA_DM_energy}.
The corresponding Galactic diffuse $\gamma$-ray backgrounds are also shown.
}
\label{fig:GAMMA_cases_30}
\end{center}
\end{figure}


