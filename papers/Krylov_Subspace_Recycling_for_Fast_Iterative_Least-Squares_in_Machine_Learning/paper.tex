\documentclass{article}

% if you need to pass options to natbib, use, e.g.:
\PassOptionsToPackage{numbers, compress}{natbib}
% before loading nips_2017
%
% to avoid loading the natbib package, add option nonatbib:
% \usepackage[nonatbib]{nips_2017}

%\usepackage{nips_2017}
% to compile a camera-ready version, add the [final] option, e.g.:
\usepackage[final]{nips_2017}


% OLD PREAMBLE:

% \usepackage{jsen}
% \usepackage{cite}
% \usepackage{amsmath,amssymb,amsfonts, bbm, mathtools}
% \usepackage{algorithm,algorithmic}
% \usepackage{graphicx}
% \usepackage{textcomp}
% \usepackage{wrapfig}
% \usepackage{xfrac}
% \usepackage{stackengine}
% \usepackage{subfigure}
% \def\delequal{\mathrel{\ensurestackMath{\stackon[1pt]{=}{\scriptstyle\Delta}}}}



% \usepackage{color, soul}
% \newcommand{\hlt}[1]{\hl{#1}}
% \newcommand{\red}[1]{\textcolor{red}{#1}}

% \def\BibTeX{{\rm B\kern-.05em{\sc i\kern-.025em b}\kern-.08em
%     T\kern-.1667em\lower.7ex\hbox{E}\kern-.125emX}}
% \markboth{\journalname, VOL. XX, NO. XX, XXXX 2017}
% {Author \MakeLowercase{\textit{et al.}}: Preparation of Papers for IEEE TRANSACTIONS and JOURNALS (February 2017)}
% \definecolor{abstractbg}{rgb}{0.89804,0.94510,0.83137}
% \setlength{\fboxrule}{0pt}
% \setlength{\fboxsep}{0pt}

% NEW PREAMBLE:


\usepackage{amsmath,amsfonts,amssymb,bbm, amsthm, xfrac}
\usepackage{algorithmic}
\usepackage{algorithm}
\usepackage{array, multirow}
% \usepackage[caption=false,font=normalsize,labelfont=sf,textfont=sf]{subfig}
\usepackage{caption, subcaption}
\usepackage{textcomp}
\usepackage{stfloats}
\usepackage{url}
\usepackage{verbatim}
\usepackage{graphicx}
\usepackage{cite}
\usepackage{caption}
\usepackage{subcaption}
\hyphenation{}

\theoremstyle{plain}
\newtheorem{theorem}{Theorem}

\usepackage{color, soul}
\newcommand{\hlt}[1]{\hl{#1}}
\newcommand{\red}[1]{\textcolor{red}{#1}}


\usetikzlibrary{external}
% \tikzexternalize%[mode=list and make]
\tikzset{external/force remake=false}
\tikzsetexternalprefix{fig/external/}
% pdflatex --shell-escape paper.tex


% \usepackage{caption}
% \usepackage{subcaption}
\graphicspath{{fig/}}

\newlength{\figheight}
\newlength{\figwidth}

\setlength{\figwidth}{.9\textwidth}
%\setlength{\figheight}{0.61803398875\figwidth}
%\setlength{\figheight}{0.2\textheight}



\usepackage[utf8]{inputenc} % allow utf-8 input
\usepackage[T1]{fontenc}    % use 8-bit T1 fonts
\usepackage[colorlinks   = true, %Colours links instead of ugly boxes
  urlcolor     = blue, %Colour for external hyperlinks
  linkcolor    = blue, %Colour of internal links
  citecolor   = red %Colour of citations
  ]{hyperref}       % hyperlinks
\usepackage{url}            % simple URL typesetting
\usepackage{booktabs}       % professional-quality tables
\usepackage{amsfonts,amsmath,amssymb,mathtools}       % blackboard math symbols
\usepackage{nicefrac}       % compact symbols for 1/2, etc.
\usepackage{microtype}      % microtypography

%\pgfplotsset{compat=newest}

\newcommand{\CITE}{{\color{red}\textbf{[CITE]}}}

\newcommand{\spa}{\operatorname{span}}

\title{Krylov Subspace Recycling for\\{Fast} Iterative Least-Squares in Machine Learning}

% The \author macro works with any number of authors. There are two
% commands used to separate the names and addresses of multiple
% authors: \And and \AND.
%
% Using \And between authors leaves it to LaTeX to determine where to
% break the lines. Using \AND forces a line break at that point. So,
% if LaTeX puts 3 of 4 authors names on the first line, and the last
% on the second line, try using \AND instead of \And before the third
% author name.

\author{
  Filip de Roos and Philipp Hennig\\
  Max Planck Institute for Intelligent Systems
  Spemannstr. 34, T\"ubingen, Germany\\
  \texttt{[fderoos|ph]@tue.mpg.de}
}

\begin{document}
% \nipsfinalcopy is no longer used
%\usepgfplotslibrary{plotmarks}
\maketitle

\begin{abstract}
  Solving symmetric positive definite linear problems is a fundamental computational task in machine learning. The exact solution, famously, is cubicly expensive in the size of the matrix. To alleviate this problem, several linear-time approximations, such as spectral and inducing-point methods, have been suggested and are now in wide use. These are low-rank approximations that choose the low-rank space a priori and do not refine it over time. While this allows linear cost in the data-set size, it also causes a finite, uncorrected approximation error. Authors from numerical linear algebra have explored ways to iteratively refine such low-rank approximations, at a cost of a small number of matrix-vector multiplications. This idea is particularly interesting in the many situations in machine learning where one has to solve a sequence of related symmetric positive definite linear problems. From the machine learning perspective, such \emph{deflation} methods can be interpreted as transfer learning of a low-rank approximation across a time-series of numerical tasks. We study the use of such methods for our field. Our empirical results show that, on regression and classification problems of intermediate size, this approach can interpolate between low computational cost and numerical precision.
\end{abstract}

\section{Introduction}
\label{sec:introduction}

Many of the most prominent machine learning problems can be seen as a sequence of linear systems, finding $\vec{x}^{(i)}$ such that 
\begin{equation}\label{eq:task}
A^{(i)} \vec{x}^{(i)} = \vec{b}^{(i)}\qq \text{for}\qq A^{(i)}\in\Re^{n\times n},\q \vec{x}^{(i)},\vec{b}^{(i)}\in\Re^{n}, \qq\text{and}\qq i\in\mathbb{N}.	
\end{equation}
A prominent example is nonparametric logistic regression, further explained below. But there are many more: Model adaptation in Gaussian process models \citep[][\textsection5.2]{RW-GPML} requires the solution of the problem $k_{\theta,XX}^{-1}\vec{y}$ for a sequence of parameter estimates $\theta$, where $k_{XX}$ is a kernel Gram matrix over the data set $X=[\vec{x}_1,\dots,\vec{x}_n]\Trans$, and $\vec{y}$ is the vector of target data. Further afield, although this view is not currently the popular standard, deep learning tasks have in the past been addressed by methods like Hessian-free optimization \cite{martens10}, which consist of such sequences. Importantly, in these machine learning examples, the matrix $A$ is usually symmetric positive definite (or, in the generally non-convex case of deep learning, is at least approximated by an spd matrix in the optimizer to ensure a descent step). This means that Eq.~\eqref{eq:task} is indeed an optimization problem, because $\vec{x}^{(i)}$ then equals the minimum of the quadratic function 
\begin{equation}\label{eq:optimization-form}
	\vec{x}^{(i)} = \argmin_{\tilde{\vec{x}}} \frac{1}{2} \tilde{\vec{x}}\Trans A^{(i)} \tilde{\vec{x}} - \tilde{\vec{x}}\Trans \vec{b}^{(i)} \q\text{with gradient}\q \vec{r}^{(i)}(\tilde{\vec{x}}) \ce \nabla f(\tilde{\vec{x}}) = A^{(i)} \tilde{\vec{x}} - \vec{b}^{(i)}.
\end{equation}

When facing linear problems of small to moderate size (i.e.~$n\lesssim 10^4$) in machine learning , the typical approach is to rely on standard algorithms (in particular, Cholesky decompositions) provided by toolboxes like BLAS libraries, or iterative linear solvers like the method of conjugate gradients~\cite{hestenes52} (CG). Exact methods like the Cholesky decomposition have cubic cost, $\mathcal{O}(n^3)$, iterative solvers like CG have quadratic cost, $\mathcal{O}(n^2m)$ for a small number of $m$ iterative steps. These algorithms are self-contained, generic ``black boxes''---they are designed to work on \emph{any} spd matrix and approach every new problem in the same way. A more critical way to phrase this property is that these algorithms are \emph{non-adaptive}. If the sequence of tasks ($A^{(i)},\vec{b}^{(i)}$) are related to each other, for example because they are created by an outer optimization loop, it seems natural to want to propagate information from one linear solver to another. One can think of this notion as a form of \emph{computational transfer learning}, in the ``probabilistic numerics'' sense of treating a computation as an inference procedure~\cite{hennig2015probabilistic}. As it turns out, the computational linear algebra community has already addressed this issue to some extent. In that community, the idea of re-using information from previous problems in subsequent ones is known as \emph{subspace-recycling}~\cite{parks06}. But these numerical algorithms have not yet found their way into the machine learning community. Below, we explore the utility of such a resulting method for application in machine learning, by empirically evaluating it on the test problem of Bayesian logistic regression (aka. Gaussian process classification, GPC).

\subsection{Relation to Linear-Cost Methods}
\label{sub:relation_to_linear_cost_methods}

For linear problems of \emph{large} dimensionality (data-sets of size $n\gtrsim 10^4$), the current standard approach is to introduce an ``a-priori'' low-rank approximation: Sampling a small set of approximate eigenvectors of the kernel gram matrix~\cite{NIPS2007_3182,NIPS2008_3495}, and introducing various conditional independence assumptions over sets of inducing points~\cite{NIPS2000_1866,quinonero05,snelson07a}. These methods achieve \emph{linear} cost $\mathcal{O}(nm^2)$ because they project the problem onto a projective space of dimensionality $m$, and this space is not adapted over time (when it is adapted \cite{titsias09a}, additional computational overhead is created outside the solver). The downside of this approach is that it yields an approximation of finite quality---these methods fundamentally can not converge, in general, to the exact solution. The algorithms we consider below can be seen in some sense as the ``missing link'' between these linear-cost-but-finite-error methods and the cubic-cost, exact solvers for smaller problems: They \emph{adapt} the projective sub-space over time at the cost of a small number of quadratic cost steps, while also attempting to re-use as much information from previous runs as possible. In fact, the ``guessed'' projective space of the aforementioned methods could be used as the first initialization of the methods discussed below.

\section{Method}
\label{sec:method}

\begin{figure}
\centering
\def\svgwidth{\linewidth}
\input{fig/def_CG.pdf_tex}
\caption{\label{fig:def_cg} The def-CG algorithm applied to a sequence of linear systems with the implicit preconditioning visualized. The first solution is obtained through normal CG and approximate eigenvectors $W$, corresponding to the largest eigenvalues are calculated from a low-rank approximation. If prior knowledge about the eigenvectors of the first system is available, def-CG can be used for the first system as well.}
\end{figure}


\subsection{Krylov Subspace Recycling}
The \emph{Krylov sequence} of subspaces is a core concept of iterative methods for linear systems $A\vec{x}=\vec{b}$~\cite{saad11}. The \emph{Krylov subspace} $\mathcal{K}_j(A,\vec{r}_0)$ is the span of the truncated power iteration of $A$ operating on $\vec{r}_0$, where $\vec{r}_{0}$ is the gradient (\emph{residual}) from Eq.~\eqref{eq:optimization-form} at the starting point $\vec{x}_0$ of the optimization:
\begin{equation}
\mathcal{K}_j(A,\vec{r}_0)=\spa\{\vec{r}_0,A\vec{r}_0,...,A^{j-1}\vec{r}_0\}.
\end{equation}
Methods that \emph{transfer} information from one such iteration to the next in order to faster converge to a solution in subsequent systems are referred to as \emph{Krylov subspace recycling methods}~\cite{parks06}. The ``recycling'' of information is traditionally done by \textit{deflation}~\cite{saad00,frank01}, \textit{augmentation}~\cite{gaul13,morgan95}, or combinations thereof~\cite{ebadi16,chapman97}. These two approaches differ in their implementation, but have the same goal: restricting the solution to a simpler search space, to speed up convergence. Both store a set of $k$ linearly independent vectors $W^{(i)}\in\mathbb{R}^{n\times k}$. For problem $i$ in the above sequence of tasks, a subspace-recycling Krylov method computes solutions that satisfy
\begin{align}
\vec{x}_j ^{(i)}&\in \vec{x}_0+\mathcal{K}_j(A,\vec{r}_0) \cup \spa\{W^{(i)}\}, \\
\vec{r}_j ^{(i)}&= \vec{b}-A\vec{x}_j \perp \mathcal{K}_j(A,\vec{r}_0) \cup \spa\{W^{(i)}\}. \label{eq:res}
\end{align}
An augmented iterative solver keeps the vectors in $W$ and orthogonalizes the updated residuals $\vec{r}_j ^{(i)}$ against $W$. This method is easily included in methods that contain an explicit orthogonalization step, an example of which is the General Minimum Residual method (GMRES)~\cite{morgan95}.\\
For $A$ spd, one usually chooses CG as the iterative solver and deflation is easier incorporated~\cite{saad00}. A deflated method ``deflates'' a part of the spectrum of $A$ by projecting the solution onto the orthogonal complement of $W$. This can be viewed as a form of preconditioning\footnote{Not all authors agree: \citet{gaul13} argue that the projector $P_W$ should not be considered a preconditioner since its application removes a part of the spectrum of A while leaving the remainder untouched.} with a singular projector $P_W$, i.e.~solving $P^{(i)}_W A^{(i)
} \vec{x}^{(i)}=P^{(i)}_W\vec{b}^{(i)}$. This property is in contrast to normal preconditioning where the whole spectrum of $A$ is modified by a non-singular matrix.  % \rightarrow \hat{A}^{(i)}\hat{\vec{x}}^{(i)}=\hat{\vec{b}}^{(i)}$ 
 In order to keep the additional computational overhead incurred by subspace recycling low, the dimension of $W$ should be low, and contain vectors that optimally speed up the convergence. For iterative linear solvers, the rate of convergence is directly proportional to the condition number 
\begin{equation}
\kappa (A)=\frac{\lambda_{n}(A)}{\lambda_{1}(A)},
\end{equation} 
where $\lambda_j$ refers to the eigenvalues of $A$ sorted in ascending order \cite{nocedal06}. Typically, the smallest eigenvalues are the limiting factors for convergence; hence $W^{(i)}$ should ideally contain the eigenvectors related to the $k$ \emph{smallest} eigenvalues---this yields an effective condition number of $\kappa_{\text{eff}} = \nicefrac{\lambda_{n}}{\lambda_{k+1}}$, which can drastically improve the convergence rate~\cite{saad11,ebadi16}. Of course the same improvement in the condition number can also be achieved by changing the \emph{largest} eigenvalue.

\subsection{Deflated Conjugate Gradient}
\label{sec:def_cg}

In the special case when $A$ is symmetric and positive definite (SPD), the conjugate gradient method (CG) \cite{hestenes52} is a popular choice to iteratively solve the system. As noted above, in machine learning, where linear tasks almost invariably arise in the form of least-squares problems, this is actually the typical setting. \citet{saad00} derived a deflated version of CG based on the Lanczos iteration (partly represented in Algorithm \ref{alg:defcg}): it implicitly forms a tri-diagonal low-rank approximation to a Hermitian matrix $A$, which translates to symmetric when $A$ is real-valued. By storing quantities that are readily available from the CG iterations, the low-rank approximation can be obtained without costly matrix-vector computations. 
The eigenvalues of the approximation are known as Ritz values, which tend to approximate the extreme ends of the spectrum of $A$~\cite{saad11}. The corresponding Ritz vectors are used to find good approximations of the eigenvectors that can be used for a deflated subspace $W$ to improve the condition number \cite[see][for more details]{saad00}. 
The algorithm will be referred to as def-CG($k,\ell$) where $\ell$ is the number of CG iterations from which information is stored in order to generate $k$ approximate eigenvectors. 
\\
For deflated CG, the orthogonality constraint of Eq.~\eqref{eq:res} is replaced by a constraint of \emph{conjugacy}, i.e.~$\vec{r}_i\Trans A\vec{r}_j=0$ for $i\neq j$.
The associated preconditioner $P_W=I-AW(W^T A W)^{-1}W^T$ projects the residual onto the $A$-conjugate complement of $W$. Figure \ref{fig:def_cg} illustrates the effect of applying $P_W$ to $A$. For this figure, $W$ was chosen by the def-CG algorithm according to the harmonic projection method~\cite{morgan95}, so the approximate eigenvectors correspond to the largest eigenvalues. The algorithm (Algorithm \ref{alg:defcg}) differs from the standard method of conjugate gradients only in line 11 and the initialization in line 3. How the eigenvectors are approximated is outlined in Section \ref{sub:approximate_eigenvectors}.
The additional inputs to the solver are a set of $k$ linearly independent vectors in $W$ and, optionally, $W$ multiplied with $A$ if it can be obtained cheaply. \\
To estimate the computational cost of def-CG we assume $k\ll n$ so terms not containing $n$ in the computational complexity can be ignored. Each iteration in Algorithm \ref{alg:defcg} has a computational overhead of $\mathcal{O}(kn^2)$ of solving the linear system in line 11. Computing the matrix $W^TAW$ has complexity $\mathcal{O}(n^2k^2)$ but it only has to be computed once and if the procedure used by \citet{saad00} and further outlined in section \ref{sub:approximate_eigenvectors}, $W$ and $AW$ are obtained in $\mathcal{O}(n^2(\ell+1)k)$. By choosing $\ell$ and $k$ to be small the computational overhead can be kept modest. This shows the importance of choosing vectors in $W$ that significantly reduce the number of required iterations to make up for the computational overhead. Another factor to take into account is the additional storage requirements of the deflated CG. The main contributing factors are the matrices $W$ and $AW$, which each are of size $n\times k$ and $\ell$ search directions of size $n$. 

\begin{algorithm}
\begin{algorithmic}[1]
\Procedure{Deflated-CG($k,\ell$)}{$A$, $b$, $x_{-1}$, $W$, $(AW)$, tol}
\Ensure{$W\in\Re^{n\times k}$} \Comment{$k$ included in Alg. definition for interpretability, can obviously be inferred internally.}
\LState \raisebox{0pt}[0pt][0pt]{Choose $x_0$ such that $W^Tr_0=0$ where $r_0=b-Ax_0$}
\LState \raisebox{0pt}[0pt][0pt]{$x_0=x_{-1}+W(W^TAW)^{-1}W^Tr_{-1}$}
\LState \raisebox{0pt}[0pt][0pt]{Solve $W^TAW\mu_0=W^TAr_0$ for $\mu$ and set $p_0=r_0-W\mu_0$} \Comment{deflation for initial iteration}
\While{\raisebox{0pt}[0pt][0pt]{$|r_j|>\text{tol}$ (For j=1...)}}
\LState \raisebox{0pt}[0pt][0pt]{$\phantom{\alpha_{j-1}}\mathllap{d_{j-1}}=p_{j-1}^TAp_{j-1}$} 
\LState \raisebox{0pt}[0pt][0pt]{$\phantom{\alpha_{j-1}}\mathllap{\alpha_{j-1}}=r_{j-1}^Tr_{j-1}/{d_{j-1}}$}
\LState \raisebox{0pt}[0pt][0pt]{$\phantom{\alpha_{j-1}}\mathllap{x_j}=x_{j-1}+\alpha_{j-1}p_{j-1}$}
\LState \raisebox{0pt}[0pt][0pt]{$\phantom{\alpha_{j-1}}\mathllap{r_j}=r_{j-1}-\alpha_{j-1}Ap_{j-1}$}
\LState \raisebox{0pt}[0pt][0pt]{$\phantom{\alpha_{j-1}}\mathllap{\beta_{j-1}}=r_{j}^Tr_{j}/r_{j-1}^Tr_{j-1}$} \Comment{from line 6 to here: standard conjugate gradient}
\LState \raisebox{0pt}[0pt][0pt]{$\phantom{\alpha_{j-1}}\mathllap{\mu_j} = $ Solve $W^TAW\mu_j=W^TAr_j$} \Comment{deflation for following iteration}
\LState \raisebox{0pt}[0pt][0pt]{$\phantom{\alpha_{j-1}}\mathllap{p_j}=\beta_{j-1}p_{j-1}+r_j-W\mu_j$}
\If{$j<\ell$}
\LState \raisebox{0pt}[0pt][0pt]{Store $d_j$, $\alpha_j$, $\beta_j$, $\mu_j$, $p_j$}
\EndIf
\EndWhile
\EndProcedure
\end{algorithmic}
\caption{Deflated Conjugate Gradient method.}
\label{alg:defcg}
\end{algorithm}

\subsection{Approximate Eigenvectors}
\label{sub:approximate_eigenvectors}
One way of obtaining approximate eigenvectors is with the Lanczos algorithm and extract Ritz value/vector pairs. The Lanczos algorithm from which \citet{saad00} derived def-CG, generates a sequence of vectors $\{\vec{p}_j\}$  such that
\begin{equation*}
\vec{p}_{j+1} \perp_A \spa\{W,\vec{p}_0,...,\vec{p}_j\}.
\end{equation*} 
The matrix $A$ can then be transformed into a symmetric and partly tridiagonal matrix $T_{l+k}=Z^TAZ$, with $Z=[W,P_l]\in \Re^{n\times (\ell +k)}$ and $P_\ell=[\vec{p}_0,...,\vec{p}_{\ell-1}]$. The eigendecomposition of $T_{l+k}$ produces pairs $(\theta_j,\vec{u}_j)_{j=1,...,\ell+k}$, of which $\theta_j$ are Ritz values that approximate the eigenvalues of $A$. The corresponding approximate eigenvector is obtained as the Ritz vector $\vec{v}_j=Z \vec{u}_j$. For an orthogonal projection technique, such as deflation, the residual of $(A-\theta I)\vec{v}$ should be orthogonal to $Z$ \cite{chapman97}, leading to the generalized eigenvalue problem
\begin{equation*}
Z\Trans (A-\theta I) Z \vec{u}=0 \q \Rightarrow \q Z\Trans A Z\vec{u}=\theta Z\Trans Z \vec{u}.
\end{equation*}
This can be transformed to a normal eigenvalue problem by multiplying both sides with $(Z\Trans Z)^{-1}$, which exist because the columns in $Z$ are linearly independent, but the symmetric properties would be lost. Computing these matrices generates an additional, non-negligible computational cost, because CG does not generate orthogonal, but $A$-conjugate directions.\footnote{Despite its name, the gradients produced by CG are actually orthogonal, not conjugate. The name arose out of the historical context, because it is a \emph{conjugate} directions method that uses \emph{gradients}.} By instead considering the base $AZ$ for orthogonality, \citet{morgan95} rephrased the problem as a Galerkin method for approximating the eigenvalues of the inverse of $A$  
\begin{equation*}
(AZ)^T(AZ\vec{u} - \theta Z\vec{u})=0.
\end{equation*}
This method is referred to as \textit{harmonic projection}. % or a Galerkin procedure for approximating harmonic values. 
By introducing 
\begin{equation*}
F=(AZ)^TZ, \qquad \qquad G=(AZ)^T(AZ),
\end{equation*}
the problem is conveniently formulated as 
\begin{equation}
G\vec{u}=\theta F\vec{u}.
\label{eq:eig}
\end{equation}
By using properties of the search directions and residuals generated from Algorithm \ref{alg:defcg}, \citet{saad00} explicitly formed $F$ and $G$ from sparse matrices containing the stored quantitities from the first $\ell$ iterations of Algorithm \ref{alg:defcg}. This effectively reduces the computational overhead of finding approximate eigenvectors and makes the method competitive. Once Eq.~\eqref{eq:eig} is solved, one chooses $k$ of the $\ell+k$ Ritz values $\theta$ with corresponding vectors $\vec{u}$ and stores them in $U$. The $k$ approximate eigenvectors for the next system are obtained as $W=ZU$. Preferably the largest or smallest $\theta$ are chosen but this is not required.   

\paragraph{Remark: Connection to First-Order Methods}
\label{sub:connection_to_first_order_methods}

In very high-dimensional machine learning models, second-order optimization methods based on the quadratic model of Eq.~\eqref{eq:optimization-form} are not as popular as first-order methods, like (stochastic) gradient descent and its many flavors. Although we do not further consider this area below, it may be helpful to note that the schemes described here have a weak connection to it, at least in the case of noise-free gradients. That is because the first step of CG is simply gradient descent. The recycling schemes described above can thus be compared conceptually to methods like momentum-based gradient descent \cite{polyak1964some} and other acceleration techniques.

\section{Experiments}
\label{sec:experiments}

We test the utility of def-CG in a classical machine learning benchmark: The infinite MNIST~\cite{loosli06} suite is a tool to automatically create arbitrary size datasets containing images of ``hand-written'' digits, by applying transformations to the classic MNIST set\footnote{\texttt{http://yann.lecun.com/exdb/mnist/}} of actual hand-written digits. We used it to generate a training set $X$ of $36\,551$ images for the digits three and five, each of size $28\times 28$ gray-scale pixels (This means the training set is three times larger than the set of threes and fives in the original MNIST set). By the standards of kernel methods, this is thus a comparably big data set. We consider binary probabilistic classification on this dataset, and follow a setup made popular by \citet{kuss06}. This setting is a good example of the role of linear solvers in machine learning. It involves two nested loops of repeated linear optimization problems. %

This involves computing a (Gaussian) Laplace approximation to the posterior arising from a Gaussian process prior on a latent function $f$ (in our case, $p(f)=\GP(0,k)$, with the Gaussian/RBF kernel $k(\vec{x}_i,\vec{x}_j)=\theta^2 \exp \left(-\nicefrac{(\vec{x}_i-\vec{x}_j)^2}{2\lambda^2}\right)$), and the logistic link function $p(y_i \g f_i) = \sigma(y_i f_i) = 1/(1+e^{-y_if_i})$ as the likelihood (see also \cite[][\textsection 3.7.3, which also outlines the explicit algorithm]{RW-GPML}). 

The outer loop will find the optimal hyperparameters for the kernel and the inner (which is the focus of this study) will find the $\vec{f}$ that maximize
\begin{equation}
\Psi(\vec{f})=\log p(\vec{y}|\vec{f}) + \log p(\vec{f}|X)=\log p(\vec{y}|\vec{f})-\frac{1}{2}\vec{f}^TK^{-1}\vec{f} - \frac{1}{2}\log |K| -\frac{n}{2} \log 2\pi,
\label{eq:obj}
\end{equation}
for a given kernel matrix $K$ with Newton's method.

Newton's method converges to an extremum by evaluating the Jacobian and Hessian of a function at the current location $\vec{x}_n$ and finds a new location by computing $\vec{x}_{n+1}=\vec{x}_n-\operatorname{Hess}_{(\Psi)}^{-1}\operatorname{Jac}_{(\Psi)}$. These iterations involve the solution of a linear system that changes in each iteration.
%Each iteration of Newton's method requires the solution of a linear system $A^{(i)}\vec{x}^(i)=\vec{b}^{(i)}$ where both the target $\vec{b}^{(i)}$ and the matrix $A^{(i)}$ change with each new iteration. 
%$\vec{x}_{n+1}=\vec{x}_n-H_{(f)}^{-1}J_{f}$
%$x_{n+1}=x_n-\nice\frac{\nabla f(x_n)}{\nabla^2 f(x_n)}$
For Laplace approximation the system to be solved can be made numerically stable by restructuring the computations \cite{kuss06}. In each iteration the target and matrix get a new value 
\begin{align}
\vec{b}^{(i)} &=H^{[i] \frac{1}{2}}K(H^{(i)}\vec{f}_{X}^{(i)} + \nabla \log p(y|\vec{f}_X^{(i)})),\\
A^{(i)} &=I+H^{(i)\frac{1}{2}}K H^{(i) \frac{1}{2}},
\label{eq:linsys}
\end{align}
with $H=-\nabla \nabla \log p(\vec{y}|\vec{f}_X^{(i)})$. This restructuring assures that the eigenvalues $\lambda_i$ of $A$ are contained in $[1,n \max_{ij}(K_{ij})/4]$ and is therefore well-conditioned for most kernels~\cite{RW-GPML}. 

Note how this task fits the setting sub-space recycling methods are designed for: It is difficult to analytically track how an update to $\vec{f}_X^{(i)}$ affects the elements of $(A^{(i)})^{-1}$ and $\vec{b}^{(i)}$, due the non-linear dependence in $\vec{f}_X$. But as the Newton optimizer converges, the iterates change less and less: $|\vec{f}_X^{(i)}-\vec{f}_X^{(i-1)}|<|\vec{f}_X^{(i-1)}-\vec{f}_X^{(i-2)}|$. Thus, $A^{(i)}$ and $\vec{b}^{(i)}$ will change less and less between iterations, and subspace recycling should become increasingly advantageous---up to a point, because of course we have to limit the dimensionality of the deflated space for computational reasons.

%The Newton optimization process for the Laplace approximation involves repeatedly computing a gradient and Hessian matrix of the posterior log marginal likelihoood  $\log p(\vec{f}_X \g X,\vec{y})$. Here $\vec{f}_X$ is the vector of latent function values at the training locations $X\in \mathbb{R}^{n\times d}$ with labels $\vec{y}\in\{-1;1\}^n$. It yields an approximate distribution  
%\begin{equation}
%q(\vec{f}|X,\vec{y})=\mathcal{N}(\vec
%{f},\hat{\vec{f}},D^{-1})
%\label{eq:q}
%\end{equation}
%where the mean $\hat{\vec{f}}=\text{argmax}_{\vec{f}} p(\vec{f}|X,\vec{y})$ is the MAP estimate, and the covariance matrix $D=-\nabla\nabla \log p(\vec{f}|X,\vec{y})|_{\vec{f}=\hat{\vec{f}}}$ is the Hessian of the negative log posterior evaluated at this maximum. The posterior distribution $p(\vec{f}|X,\vec{y})$ can be rewritten using Bayes' rule as $p(\vec{y}|\vec{f}) p(\vec{f}|X)/p(\vec{y}|X)$, of which only the nominator depend on $\vec{f}$. To find the mode $\hat{\vec{f}}$ that maximize the posterior it is enough to consider the logarithm of the nominator \textit{i.e.} the objective
 %To maximize the posterior it is enough to consider the 
%In the following experiments, the logistic regression function  was used for the likelihood $p(y_i|f_i)$. Rasmussen and Williams \cite{RW-GPML} presents an algorithm using Newton's method to iteratively find the optimal $\hat{\vec{f}}$ to maximize Eq.~\eqref{eq:obj}. For each iteration of the method, a system involving the matrix $A=I+H^{\frac{1}{2}}KH^{\frac{1}{2}}$ needs to be solved, where $H=-\nabla \nabla \log p(y|f)$ is the diagonal Hessian of the negative log likelihood. The diagonal matrix $H$ changes between the iterations of Newton's method and consequently $A$ as well. The eigenvalues of $A$ are contained in $\lambda_i \in \left[1,n \max_{ij}(K_{ij})/4\right]$, which makes $A$ well-conditioned for most covariance functions. Finding a solution to the system involving $A$ can be done via a Cholesky decomposition. 

Solving the linear system in Eq.~\eqref{eq:linsys} with a Cholesky decomposition is $\mathcal{O}(n^3)$ expensive, where $n$ is the dimension of $A$ i.e.~the size of the training set $X$. Iterative methods, such as CG due to the kernel matrix $K$ being SPD, have been used to speed up the mode-finding of $\hat{\vec{f}}$~\cite{RW-GPML,davies14}. Table \ref{tab:acc} compares the cumulative computational cost and accuracy of the exact Cholesky decomposition, standard CG and deflated CG for each iteration in Newtons method. A reduction of the relative error $\epsilon=|\vec{b}-A\vec{x}_i|/|\vec{b}|$ was used as stopping criterion and was chosen to be $10^{-5}$. 
%Iterative methods such as CG can be used to speed up the solution of the system \cite{davies14,RW-GPML}. When the problem is well-conditioned and the solution does not have to be exact, iterative methods generally require lower computational cost to find an approximate solution satisfying the error tolerance.

%\begin{figure}[t]
%  \centering \scriptsize
%  

\section{An example}
\label{sec:6}
\begin{figure}[H]

\includegraphics[scale=.6]{schematic}
\centering
\caption{ A sketch of the example constructed in this section. The manifold decomposes into three pieces, left and right caps and a middle region. The loop drawn on the boundary of the left cap only bounds surfaces of high topological complexity that are at least partly contained in the right cap. This ensures the isoperimetric ratio of that loop is very small.}
\label{fig:example_sketch}
\end{figure}

The aim of this section is to show that the first positive eigenvalue of the 1-form Laplacian can vanish exponentially fast in relation to volume. This contrasts the behaviour of the first positive eigenvalue of the Laplacian on functions.

Our construction is similar to that in \cite{BD}. Essentially, we choose a hyperbolic 3-manifold with totally geodesic boundary and glue it to itself using a particular psuedoAnosov with several useful properties. By \cite{BMNS}, this family has geometry that up to bounded error can be understood in terms of a simple model family. Using this model family, we show that one can find curves with uniformly bounded length whose stable commutator length grows exponentially in the volume. We then use the spectral gap upper bound in Theorem A to conclude the first positive eigenvalue vanishes exponentially fast.

Throughout this section, we need to compare geodesic lengths in different submanifolds of a given manifold $M$. Let $|\cdot|_{X}$ denote the geodesic length of a homotopy class of curves relative endpoints in a manifold $X$ and $\text{length}(\cdot)$ be the length in $M$ of the curve. Similarly, when we compute stable commutator length for the fundamental group of a manifold $X$, which may or may not be a submanifold of $M$, we denote it $\scl_X$.

We will need that for certain curves, $\scl$ is comparable to length. We begin with a simple but essential technical lemma.
\begin{figure}[H]
\labellist
\small\hair 2pt
 \pinlabel {$a_0$} [ ] at 830 1100
 \pinlabel {$a_1$} [ ] at 1300 1100
 \pinlabel {$b_0$} [ ] at 830 1250
 \pinlabel {$b_1$} [ ] at 1300 1250
 \pinlabel {$t_1$} [ ] at 380 1290
 \pinlabel {$t_0$} [ ] at 380 1120
\endlabellist
\centering
\includegraphics[width = 15cm]{lemdiagram}
\caption{ Illustrating Lemma \ref{lem:6.1} , the rectangular base of the figure is part of the totally geodesic surface $S$ and the box is the corresponding part of the tubular neighborhood $N_\e(S)$ foliated by surfaces $S_t$ parallel to $S$. Drawn in the box is the surface $\Sigma$, which is transverse the foliation except at isolated points. The multicurve $a_0\cup a_1$ is part of a single level set $S_{t_0} \cap \Sigma$, but only $a_0$ is part of the curve $c_{t_0}$ described in the lemma, whereas the multicurve $b_0\cup b_1$ forms the multicurve $c_{t_1}$ in the lemma.}
\label{fig:box}
\end{figure}


\begin{lem} \label{lem:6.1} Let $M$ be a compact hyperbolic 3-manifold with totally geodesic boundary $ \d M = S$. Let $\e$ be smaller than the injectivity radius of $M$ and such that $N_\e(S)$ is an embedded tubular neighborhood. Let $\{S_t\}$ be the leaves of the foliation of $N_\e(S)$ by surfaces equidistant from $S$.  Let $\Sigma$ be a smooth incompressible proper not necessarily immersed surface in $M$ that is transverse to the foliation $\{S_t\}$ except at isolated points. Let $c = \d \Sigma$. By transversality, for generic $t$ the multicurve $c_t$ given by the part of $S_t\cap \Sigma$ that cobounds a subsurface of $\Sigma$ with $c = c_0$ is a smooth multicurve.  Let $T$ be the set (of full measure) of all $t\in[0,\e)$ such that $c_t$ is a smooth multicurve. Since $S$ is totally geodesic, each multicurve $c_t$ is homotopic to a possibly degenerate geodesic multicurve $\gamma_t$ in $S$. Let $\Sigma_\e$ be the part of $\Sigma$ contained in $N_\e(S)$. Then for $C = 1/\e$, we have that $$\inf\limits_{t\in T} |\gamma_t|_{S} \leq C \emph{Area}(\Sigma_\e).$$
\end{lem}

\begin{proof} The coarea formula implies the inequality $\inf\limits_{t\in T} |c_t|_{S_t} \leq C \text{Area}(\Sigma_\e)$.  Since $S$ is totally geodesic, for all $t\in T$, we have $|\gamma_t|_S\leq |c_t|_{S_t}.$  \end{proof}
Note that in the previous lemma, when $\inf\limits_{t\in T} |\gamma_t|_S$ is zero, because $\Sigma$ is incompressible and any loop with length less than $\inj(M)$ bounds a disk, every component of $\Sigma$ can either be homotoped to be disjoint from $N_\e(S)$ or be contained in $S$.

The next proposition requires a notion of geometric complexity for homology classes. For any compact Riemannian manifold $M$ one can define the stable norm on the first homology of $M$ (see \cite{Gromovmetric} Section 4C). The mass of a Lipschitz 1-chain $\alpha = \sum_i t_i\alpha_i$ in $M$ is defined to be $\text{mass}(\alpha) = \sum_i |t_i|\text{length}(\alpha_i).$ The mass of a class $a\in H_1(M)$ is then the infimal value of the mass of a chain $\alpha$ representing $a$.
For a class $a\in H_1(M)$, the stable norm of $a$ is then given by $$||a||_{s,M} = \inf\limits_{m>0}\frac{\text{mass}(m a)}{m}.$$

Stable commutator length can also be generalized to geodesic multicurves (see Section 2.6 of \cite{Calegari}), which can naturally be viewed as Lipschitz chains. Suppose $\gamma_i\in \pi_1 M$ and $ \sum_i n [\gamma_i] = 0$ in $H_1(M)$. Let $\gamma$ be the geodesic multicurve, which is not necessarily simple, consisting of the geodesic loops determined by $\gamma_i$. Say a surface $f:S\to M$ is admissible of degree $n(S)$ if it has no closed components and $\d S$ is a union of circles $S^1_i$ with $f|_{S^1_i}$ a degree $n(S)$ cover of $\gamma_i$.
Then we define stable commutator length of $\gamma$ to be $$\scl(\gamma) = \inf\limits_{S \text{ admissible}} \frac{\chi_-(S)}{2n(S)}.$$ When $\gamma$ is a single loop, this definition agrees with the usual definition of stable commutator length.

 \begin{prop} \label{prop:6.2} Let $M$ be a compact oriented hyperbolic 3-manifold with totally geodesic boundary $\d M = S$. Let $\gamma$ be a geodesic multicurve in $S$ that is rationally nullhomologous in $M$. Then there is a constant $D> 0$ depending only on $M$ such that $$||[\gamma]||_{s,S}\leq D\scl_M(\gamma).$$ \end{prop}

 \begin{proof} If $\gamma$ is nullhomologous in $S$, then the left hand side is zero and the inequality holds. Assume now that $[\gamma]\neq 0\in H_1(S)$. Fix $\delta>0$. Let $\Sigma_m$ be an incompressible admissible surface for $\gamma$ of degree $m = n(S)$ such that $\chi_-(\Sigma_m)/2m -\scl(\gamma) < \delta$. We can triangulate $\Sigma_m$ so that there is a single vertex on each boundary component. This triangulation has $4g +3b - 4$ faces, where $g$ is the genus of $\Sigma_m$ and $b$ the number of boundary components. We can then straighten this triangulation to obtain a piecewise totally geodesic triangulated surface. Replace $\Sigma_m$ with this surface. Since every face of this triangulation of $\Sigma_m$ is geodesic, every face has area at most $\pi$. Since there are $4g+3b - 4$ faces and $\chi_-(\Sigma_m) = 2g-2 + b$, we can estimate  $$\text{Area}(\Sigma_m) \leq 3\pi\chi_-(\Sigma_m).$$

We can perturb $\Sigma_m$ to obtain a smooth surface $\Sigma’_m$ that it is transverse the foliation of $N_\e(S)$ except at isolated points and in doing so increase the area by less than $\delta$. Let $\gamma_t$ be the family of multicurves in Lemma \ref{lem:6.1}  applied to $\Sigma’_m$. Since each curve $\gamma_t$ cobounds a surface in $S$ with $\d \Sigma_m$, they are homologous, thus $||[\gamma_t]||_{s,S} = m||[\gamma]||_{s,S}$. Since $||[\gamma_t]||_{s,S}\leq |\gamma_t|_S$,
Lemma \ref{lem:6.1}  implies that $$m||[\gamma]||_{s,S}\leq C\text{Area}(\Sigma_m’) \leq C \text{Area}(\Sigma_m) + C\delta \leq 3C\pi\chi_-(\Sigma_m) + C\delta.$$
From this we get  $$||[\gamma]||_{s,S}\leq 6C\pi\chi_-(\Sigma_m)/2m + C\delta/m\leq 6C\pi\scl_M(\gamma) + 6C\pi\delta +C\delta/m.$$
Since the stable commutator length of a nontrivial rational commutator is bounded away from zero by a constant only depending on $M$, by Theorem 3.9 in \cite{Calegari}, we can replace $C$ with a larger constant $D$ such that $$||[\gamma]||_{s,S}\leq D\scl_M(\gamma),$$ as desired. \end{proof}

 We now introduce the family of manifolds that we use in our construction. The family $\{W_n\}$ of manifolds we study are easily understood using the model manifold theory of \cite{BMNS}. In particular, there is a $K$-biLipschitz map between $W_n$ and a model manifold $M_n$, where $K$ is independent of $n$. The base of the construction is Thurston’s tripus manifold $W$ (see \cite{thurstonbook}, Section 3.3.12), a hyperbolic manifold with totally geodesic boundary, and a psuedoAnosov homeomorphism $f$ of the boundary surface $\d W$. The model manifold $ M_n$ is a degree $n$ cyclic cover of the mapping torus $M_{f}$ cut open along a fiber with two oppositely oriented copies of $W$, denoted $W^+$ and $W^-$ glued as described in \cite{BMNS} Section 2.15 to the two boundary components of the cut open mapping torus. This decomposes $W_n$ into three pieces, a product region $S\times [0, n]$ and the caps $W^+$ and $W^-$ in a metrically controlled way. It will be convenient to set $M^+ = W^+\subset M_n$ and $M^- = W^-\subset M_n$ when talking about the caps of the model manifold $M_n$ for fixed $n$, and to let $W^+$ and $W^-$ denote the images of these spaces under the natural inclusion into $W_n$.

 \vspace{1cm}
\begin{figure}[H]
\labellist
\small\hair 2pt
 \pinlabel {$M^+$} [ ] at 950 1100
 \pinlabel {$M^-$} [ ] at 2000 1100
 \pinlabel {$S\times[0,n]$} [ ] at 1500 1600
\endlabellist
\centering
\includegraphics[scale=.15]{modelscheme}
\caption{ A schematic picture of the model manifold $M_n$ with caps $M^+$ and $M^-$ two oppositely oriented copies of the tripus manifold.}
\label{fig:caps_schematic}
\end{figure}

Given a multicurve $c$ in $M^{\pm}$, we say $c$ \textbf{bounds on both sides} if there are incompressible surfaces $S^+$ in $M^+$ and $S^-$ in $M^-$ both with boundary homotopic to $c$.

We encode the construction and its essential properties in the following proposition.

\begin{prop}  \label{prop:6.3}There is a family $\{W_n\}$ of closed hyperbolic 3-manifolds with injectivity radius uniformly bounded below and volume growing linearly in $n$ constructed from the tripus and a pseudoAnosov $f$ as described above. Each manifold $W_n$ is $K$-biLipschitz equivalent to the model manifold $M_n$ for some constant $K$ independent of $n$. Any homologically nontrivial loop in $H_1(\d W^{\pm})$ that bounds a surface in $M^{\pm}$ cannot bound on both sides. The pseudoAnosov $f$ is such that for any nonzero class $a\in H_1(\d W^+)$, the stable norm of $f_*^n(a)$ grows exponentially.
\end{prop}

\begin{proof}
Let $W$ be Thurston’s tripus manifold, a compact hyperbolic 3-manifold with totally geodesic boundary a genus 2 surface for which the inclusion map $H_1(\d W;\Z)\to H_1(W;\Z)$ is onto.
The homology of the boundary $\d W$ decomposes as the direct sum of rank 2 submodules $U$ and $V$, where $V\subset H_1(\d W)$ is the image of the boundary map $\d : H_2(W,\d W;\Z)\to H_1(\d W;\Z)$ (which is also the kernel of the inclusion $H_1(\d W)\to H_1(W)$) and $U$ is a compliment of $V$ (note that the inclusion map $H_1(\d W)\to H_1(W)$ restricted to $U$ is an isomorphism).
Let $S$ be a genus 2 surface, which we will use to mark the boundaries of $W^{+}$ and $W^-$. Assume $H_1(S;\Z)$ is generated by $e_1,~e_2,~e_3,~e_4$. Choose a marking $S\to \d W^+$ so in $W^+$ one has $U = \langle e_1,e_2\rangle $ and $V = \langle e_3, e_4\rangle$.
Similarly, choose a marking $S\to \d W^-$ so that in $W^-$ one has $V = \langle e_1,e_2\rangle $ and $U = \langle e_3, e_4\rangle$. We then define $$W_n = W^+\cup_{f^n}W^-$$ where $f:S\to S$ is a pseudo-Anosov that acts on $H_1(S)$ by the symplectic matrix\[ F = \begin{pmatrix}
 2 &  1 & 0 & 0 \\
 1 & 1 & 0 & 0 \\
 0 & 0 & 1 & -1\\
 0 & 0 & -1 & 2
\end{pmatrix} \] For the existence of such a pseudoAnosov mapping class, see the proof Lemma 7.1 in \cite{BD}. This matrix preserves the subspace decomposition above, and so ensures that every curve in $\d W^{\pm}$ that is not nullhomologous in $\d W^{\pm}$ but bounds a surface in $M^{\pm}$ cannot bound on both sides.

The mapping class $f$ acts as an Anosov matrix on $U$ and $V$. This ensures the standard Euclidean $\ell^2$-norm $||F^n(a)||_E$ of an element $a\in H_1(S)$ grows exponentially in $n$ (indeed, for our choice of $F$, it grows like $(\frac{3+\sqrt{5}}{2})^n$). Since norms on finite dimensional real vector spaces are comparable, there is a constant comparing the stable norm induced by the metric inherited from $W$ to the standard Euclidean $\ell^2$-norm on $H_1(S)$.

Lemma 7.3 in \cite{BD} explains how Theorem 8.1 in \cite{BMNS} implies that for large $n$ the manifolds $W_n$ admit a $K$-biLipschitz diffeomorphism $\mu$ from the model manifold $ M_n$ as described above. After increasing $K$, we can drop the large $n$ condition. This then also implies the linear volume growth and injectivity radius bounds.
\end{proof}

\begin{remark}
Using the model manifold, one can easily estimate the Cheeger constant of $W_n$, which will decay like $1/n$.
\end{remark}


\begin{mainthm}  \label{thm:C} The family $W_n$ of closed hyperbolic 3-manifolds from Proposition \ref{prop:6.3}  has 1-form Laplacian spectral gap that vanishes exponentially fast in relation to volume:
$$\sqrt{\lambda(W_n)}\leq B\vol(W_n)e^{-r\vol(W_n)},$$
where $r$ and $B$ are positive positive constants and $\lambda(W_n)$ is the first positive eigenvalue of the 1-form Laplacian on $W_n$.
\end{mainthm}


\begin{proof}
We continue using notation introduced in the previous propositions. Take $\gamma$ in $\d M^+\subset M_n$ to be an embedded geodesic loop representing the class $e_1 \in U \subset H_1(\d W^+)$. Recall from the proof of Proposition \ref{prop:6.3} that $\gamma\subset \d M^+$ does not bound a surface in $M^+$ but that $f^n(\gamma)\subset \d M^-$ bounds a surface in $M^-$. Let $\alpha_n = f^n(\gamma)\subset \d M^- \subset M_n$. Note that $\alpha_n$ and $\gamma$ are isotopic in $M_n$.

Fix $\delta>0$. Consider some positive integer $m$ and  incompressible surface $\Sigma_m$ bounding $\alpha_n^m$ in $M_n$ with $\chi_-(\Sigma_m)/2m - \scl_{M_n}(\gamma) < \delta$ and which minimizes $\chi_-$ among surfaces with boundary $\alpha_n^m$. We can then replace $\Sigma_m$ with a homotopic surface that pushes the boundary of $\Sigma_m$ into the interior of $M^-$ and which intersects $\d M^-$ transversely and essentially in both $\d M^-$ and $\Sigma_m$.
We can then attach an annulus to $\Sigma_m$ cobounding $\alpha_n^m$ and the boundary of the modified surface $\Sigma_m$. This new $\Sigma_m$ bounds $\alpha_n^m$ with a collar neighborhood of the boundary contained entirely in $M^-$ and intersects $\d M^-$ transversely in a union of loops essential in both $\Sigma_m$ and $\d M^-$.

We focus on the portion of $\Sigma_m$ that lies in $M^-$. Define $\Sigma^-_m =\Sigma_m\cap M^-$. If $\Sigma_m$ is contained in $M^-$, then Proposition \ref{prop:6.2}  applied to $\alpha_n^m$ in $M^-$ implies that
$$||[\alpha_n^m]||_{s,\d M^-} = m||[\alpha_n]||_{s,\d M^-} \leq m\scl_{M^-}(\alpha_n)\leq D\chi_-(\Sigma_m^-) = D\chi_-(\Sigma_m) ,$$
where $||\cdot||_{s, \d M^-}$ is the stable norm of $H_1(\d M^-)$. Since $\chi_-(\Sigma_m)/m- \scl_{M_n}(\gamma)\leq \delta$, we conclude $$||[\alpha_n]||_{s,\d M^-}\leq D\scl_{M_n}(\gamma) + D\delta.$$

Our goal now is to get this same estimate for the other possible ways $\Sigma_m$ sits in $M_n$.
\vspace{1cm}
\begin{figure}[H]
\labellist
\small\hair 2pt
 \pinlabel {$M^+$} [ ] at 900 1560
 \pinlabel {$S\times[0,n]$} [ ] at 1480 1650
 \pinlabel {$M^-$} [ ] at 2000 1560
 \pinlabel {$\alpha$} [ ] at 1750 1050
\endlabellist
\centering
\includegraphics[scale=.2]{schematic5}
\caption{ A schematic picture of the simplest case of a surface bounding $\alpha$ in $M^-$.}
\label{fig:alpha_bounds}
\end{figure}

Consider the case that $\Sigma_m$ does not lie entirely in $M^-$. There are two possibilities. The first involves the surface $\Sigma_m$ passing into the product region but not intersecting $M^+$. In this case the surface can be homotoped to lie in $M^-$, so that Proposition \ref{prop:6.2} applies, giving the desired estimate as in the previous case.
\vspace{2cm}
\begin{figure}[H]
\labellist
\small\hair 2pt
 \pinlabel {$M^+$} [ ] at 900 1560
 \pinlabel {$S\times[0,n]$} [ ] at 1480 1650
 \pinlabel {$M^-$} [ ] at 2000 1560
 \pinlabel {$\alpha$} [ ] at 1750 850
\endlabellist
\centering

\includegraphics[scale=.2]{schematic3}
\caption{ A schematic picture of a surface bounding $\alpha$ that passes back into $M^-$ but does not pass into $M^+$.}
\label{fig:pass_back_schematic}
\end{figure}

\vspace{2cm}
\begin{figure}[H]
\labellist
\small\hair 2pt
 \pinlabel {$M^+$} [ ] at 900 1560
 \pinlabel {$S\times[0,n]$} [ ] at 1480 1650
 \pinlabel {$M^-$} [ ] at 2000 1560
 \pinlabel {$\alpha$} [ ] at 1750 850
 \pinlabel {$c_1$} [ ] at 1750 1285
 \pinlabel {$c_0$} [ ] at 1750 1070

\endlabellist
\centering

\includegraphics[scale=.2]{schematic6}
\caption{ A schematic picture of a surface bounding $\alpha$ that passes back into $M^+$. Notice the multicurve $c^- = c_0\cup c_1$ bounds surfaces in $M^+$ and $M^-$, so is homologically trivial. }
\label{fig:bounds_both_sides}
\end{figure}



The second possibility concerns the surface $\Sigma_m$ crossing through the product region into $M^+$ with an essential intersection with $\d \Sigma^+$. In this case, we will see that the surface $\Sigma_m^-$ has boundary homologous to $\alpha_n^m$, which will allow us to apply Proposition \ref{prop:6.2} to obtain the desired estimate. By construction, a sufficiently small collar $C$ of the boundary $\d \Sigma_m$ in $\Sigma_m$ maps into $M^-$, so in particular, a subsurface of $\Sigma_m^-$ has some boundary component that maps to $\alpha_n^m$. That boundary component can be closed by attaching a surface $S^-$ that bounds $\alpha_n^m$ in $M^-$ to $\Sigma_m$.
From this, we see that the multicurve $c^- = \d \Sigma^-_m -\alpha_n $ bounds surfaces in $M^+$ and $M^-$.
Thus by Proposition \ref{prop:6.3}, $c^-$ must be homologically trivial in $\d M^{-}$. Let $x = \d\Sigma_m^- = c^- + \alpha_n^m$. Since $c^-$ is nullhomologous, $||[x]||_{s,\d M^-} = ||[\alpha_n^m]||_{s,\d M^-} = m||[\alpha_n]||_{s,\d M^-}.$
By Proposition \ref{prop:6.2} , $||[x]||_{s,\d M^-}\leq D\scl_{M^-}(x).$ Since $x$ is essential in $\Sigma_m$, we get that $\chi_-(\Sigma_m^-) \leq \chi_-(\Sigma_m)$, then using that $\chi_-(\Sigma_m)/2m - \delta \leq \scl_{M_n}(\alpha_n)$,
we obtain $\scl_{M^-}(x) \leq \chi_-(\Sigma_m^-)/2 \leq m\scl_{M_n}(\alpha_n) + \delta m$.
Putting this all together and dividing by $m$, we get that $$||[\alpha_n]||_{s,\d M^-}\leq D\scl_{M_n}(\alpha_n) + D\delta.$$

We therefore have in each case that there is a constant $D$ independent of $n$ such that $$||[\alpha_n]||_{s,\d M^-}\leq D\scl_{M_n}(\alpha_n) + D\delta.$$ By Proposition \ref{prop:6.3} , $||[\alpha_n]||_{s,\d M^-} = ||[f^n(\gamma)]||_{s,\d M^+}$ grows exponentially in $n$.
Thus for some constants $B>0$ and $r >0$, we have $$Be^{rn}\leq D\scl_{M_n}(\gamma) + D\delta,$$ where we use that $\gamma$ and $\alpha_n$ are homotopic in $M_n$. Using the injectivity radius lower bound and Theorem 3.9 of \cite{Calegari}, we can increase $D$ and drop the additive constant in this inequality. By Proposition \ref{prop:6.3} , the volume growth of the $W_n$ is proportional to $n$ so there is a constant $C$ such that $\vol(W_n)\leq Cn$.
Additionally, using the $K$-biLipschitz comparison of Proposition \ref{prop:6.3} , the length of $\gamma$ in $W_n$ is bounded from above by $2K|\gamma|_{W}$, where $W$ is the tripus. As a result, Theorem A implies that the spectral gap for the 1-form Laplacian of the manifolds $W_n$ vanishes exponentially fast in $n$.
In particular, we have \[\sqrt{\lambda(W_n)} \leq A\vol(W_n)\frac{|\gamma|_{W_n}}{\scl_{W_n}(\gamma)} \leq 2KACB^{-1}D|\gamma|_Wne^{-rn},\] so the result holds after redefining $B$ to be $2KACB^{-1}D|\gamma|_W.$

\end{proof}

%  \caption{Caption here}
%  \label{fig:figure1}
%\end{figure}

\setlength{\tabcolsep}{5.4pt} % <- default: 6pt
\begin{table}
	\caption{Iterative solvers operating on the MNIST classification task. The table shows the progress over Newton iterations. Within each Newton iteration the system in Eq.~\eqref{eq:linsys} needs to solved. Both iterative solvers were set to run until they achieve a relative error of $\epsilon=10^{-5}$. The column labeled $t$ shows cumulative runtimes for each method with the time to extract $W$ included for def-CG.}
	\label{tab:acc}
	\centering
	\begin{tabular}{c|cc|ccc|ccc}
	\toprule
	 & \multicolumn{2}{|c|}{Cholesky} &\multicolumn{3}{|c|}{CG} &\multicolumn{3}{|c}{def-CG($k=8,\,\ell = 12$)}  \\
	\midrule
	%\cline{1-6}
	   It. & $\log p(\vec{y}|\vec{f})$ & $t\,[\mathrm{s}]$  & $\log p(\vec{y}|\vec{f})$ & rel. error $\delta$ & $t\,[\mathrm{s}]$ & $\log p(\vec{y}|\vec{f})$ & rel. error $\delta$ & $t\,[\mathrm{s}]$\\
	\midrule
	1 & -4926.523 & 426& -4968.760 &$8.573\cdot10^{-3}$ &231 &  -4968.760 & $8.573\cdot10^{-3}$ & 245   \\
	2 & -1915.537 &896 & -1931.348 & $8.254\cdot10^{-3}$&492 &  -1938.585 & $1.203\cdot10^{-3}$ & 436   \\
	3 & -919.124  &1366 & -924.891 & $6.274\cdot10^{-3}$&715 &  -926.668 & $8.208\cdot10^{-3}$ &  617  \\
	4 & -549.182  &1875 & -551.432 & $4.097\cdot10^{-3}$&920 &  -551.796 & $4.760\cdot10^{-3}$ &  790  \\
	5 & -407.058  &2362 & -408.010 & $2.339\cdot10^{-3}$&1088 &  -408.133 & $2.641\cdot10^{-3}$ &  947  \\
	6 & -353.632  &2856 & -354.040 & $1.154\cdot10^{-3}$&1246 &  -354.085 & $1.281\cdot10^{-3}$ &  1101  \\
	7 & -335.575  & 3342& -335.711 & $4.053\cdot10^{-4}$& 1444&  -335.744 & $5.036\cdot10^{-4}$ &  1258  \\
	8 & -331.326  &3815& -331.346 & $6.036\cdot10^{-5}$&1647 &  -331.355 & $8.753\cdot10^{-5}$ &  1418  \\
	9 & -330.997  &4317& -330.998 & $2.309\cdot10^{-6}$&1821 &  -330.996  & $4.018\cdot10^{-6}$ &  1571  \\
	\bottomrule
	\end{tabular}
\end{table}
Table \ref{tab:acc} shows that iterative methods on their own already save computations. Figure~\ref{fig:results} also shows a plot of these results. The increasingly steep downward slope of the CG error in later optimization problems suggests that the systems are getting easier to solve, possibly because the matrix $A^{(i)}$ becomes better conditioned. But the figure also shows the additional advantage of sub-space recycling in def-CG. Eight approximate eigenvectors were used, yielding a saving of at least 12 CG iterations per system ($\sim 25\%$). Enough to outweigh the the computational overhead of finding $W$ and $AW$. Initially, the reduction of iterations for def-CG stagnate (become parallel to CG) over the course of a few Newton iterations, which suggests the recycled subspace fails to reduce the effective condition number further. Either the recycled vectors do not find good approximations to the extreme eigenvalues and the algorithm can be improved, or the difference between successive $A^{(i)}$ is too significant so the propagated information is less useful. Additional experiments (not shown) suggest that both effects play a role, but the former, i.e.~numerical stability, dominates. % By choosing a larger $l$ a better convergence to the eigenvectors are expected. A much larger value of $l$ was not possible to use, likely due to numerical precision issues leading to non-orthogonal residuals.
% The approximate eigenvalues would then become complex-valued instead of real-valued and positive as expected from a SPD matrix. 
Methods that try to alleviate this problem by estimating the convergence of the approximate eigenvectors exist \cite{gosselet13}, but cause additional computational overhead.
% increases so the problem sequences are very high-dimensional
%({\color{red} Should w argue why it was not implemented?}).

\setlength{\figwidth}{.5\textwidth}
\setlength{\figheight}{0.61803398875\figwidth}

\begin{figure}[t]
    \centering \scriptsize
        \hfill% This file was created by matplotlib2tikz v0.6.3.
\begin{tikzpicture}

\definecolor{color1}{RGB}{000,125,122} % MPG green
\definecolor{color0}{HTML}{FF9933} %EI orange
\definecolor{color2}{RGB}{130,0,0} % dred

\begin{axis}[
xlabel={Newton iteration},
ylabel={CPU time [s]},
xmin=0.65, xmax=8.35,
ymin=0, ymax=550,
width=\figwidth,
height=\figheight,
tick align=outside,
x grid style={lightgray!92.026143790849673!black},
y grid style={lightgray!92.026143790849673!black},
legend style={draw=white!80.0!black,at={(axis cs:5,300)},anchor=west},
legend entries={{CG},{def-CG(8,12)},{Cholesky}},
legend cell align={left},
mystyle
]
\addplot [semithick, color0, solid, mark=square*, mark size=1, mark options={solid}]
table {%
1 230.4
2 177.648
3 156.018
4 126.346
5 112.917
6 113.662
7 109.538
8 113.281
};
\addplot [semithick, color1, solid, mark=diamond*, mark size=1, mark options={solid}]
table {%
1 243.445
2 130.813
3 127.016
4 105.099
5 94.049
6 99.265
7 86.129
8 99.452
};
\addplot [semithick, color2, solid, mark=*, mark size=1, mark options={solid}]
table {%
1 460.978
2 494.108
3 510.731
4 533.842
5 494.803
6 512.84
7 509.529
8 525.17
};
\end{axis}

\end{tikzpicture}\hfill%
        % This file was created by matplotlib2tikz v0.6.3.
\begin{tikzpicture}

\definecolor{color1}{RGB}{000,125,122} % MPG green
\definecolor{color0}{HTML}{FF9933} %EI orange

\begin{axis}[
xlabel={Newton Iteration},
ylabel={CG Iterations},
xmin=0.65, xmax=8.35,
ymin=0, ymax=105,
width=\figwidth,
height=\figheight,
tick align=outside,
x grid style={lightgray!92.026143790849673!black},
y grid style={lightgray!92.026143790849673!black},
legend style={draw=white!80.0!black},
legend entries={{CG},{def-CG(8,12)}},
legend cell align={left},
mystyle
]
\addplot [semithick, color0, mark=square*, mark size=1, mark options={solid}]
table {%
1 100
2 83
3 66
4 54
5 50
6 50
7 49
8 48
};
\addplot [semithick, color1, solid, mark=diamond*, mark size=1,mark options={solid}]
table {%
1 100
2 56
3 50
4 43
5 39
6 39
7 37
8 36
};
\end{axis}

\end{tikzpicture}\hfill\null
    \caption{\textbf{Left:} Computational cost (CPU time) per iteration of Newton's method. \textbf{Right:} number of iterations required for CG and def-CG(8,12) to solve a single system (Eq.~\eqref{eq:linsys}) to a relative error of $\epsilon=10^{-5}$. The stopping criterion for the Newton iteration was $\Delta \Psi(\vec{f}) < 1$, thus only requiring the first two terms in Eq.~\eqref{eq:obj} to be computed.}\label{fig:results}
\end{figure}
\setlength{\figwidth}{.9\textwidth}

To better understand the effect of the deflation on individual solutions of Eq.~\eqref{eq:linsys}, Figure~\ref{fig:comparison} compares the convergence of def-CG and standard CG. Each solver was set to run until a relative error of $10^{-8}$ was achieved and the results are shown in Fig.~\ref{fig:comparison}. The figure indicates that the computational savings do not stem from the initial projection onto the A-orthogonal complement of $W$ as one could suspect, but rather from a steeper convergence. This fits with the idea that deflation lowers the effective condition number of deflated system relative to the original problem. 

%\setlength{\figwidth}{.49\textwidth}
%\setlength{\figheight}{0.61803398875\figwidth}
%\begin{figure}
%    \centering \scriptsize
%        % This file was created by matplotlib2tikz v0.6.3.
\begin{tikzpicture}

%\definecolor{color1}{rgb}{1,0.498039215686275,0.0549019607843137}
%\definecolor{color0}{rgb}{0.12156862745098,0.466666666666667,0.705882352941177}
%\definecolor{color2}{rgb}{0.172549019607843,0.627450980392157,0.172549019607843}

\definecolor{color1}{RGB}{000,125,122} % MPG green
\definecolor{color0}{HTML}{FF9933} %EI orange

\begin{axis}[
title={i=1},
xlabel={CPU time},
ylabel={relative residual},
xmin=-17.73529565, xmax=426.27712265,
ymin=3.63890102429955e-09, ymax=2.52265906415052,
ymode=log,
width=\figwidth,
height=\figheight,
tick align=outside,
x grid style={lightgray!92.026143790849673!black},
y grid style={lightgray!92.026143790849673!black},
legend style={draw=white!80.0!black},
legend entries={{CG},{def-CG(8,12)},{1-rec-def-CG(8,12)}},
legend cell align={left},
mystyle
]
\addplot [semithick, color0]
table {%
2.55172099999993 1
4.6798839999999 0.804575147171834
6.63857199999995 0.194574803468408
8.73465899999997 0.0524702710386406
10.772392 0.0403274484736291
12.967594 0.0187561729091768
15.1668269999999 0.0165370912766679
17.239797 0.0115256957522894
19.2344509999999 0.00795507095065917
21.244998 0.00603031544117425
23.3661109999999 0.00481715637993339
25.402647 0.00394061134811415
27.5293459999999 0.014360747699105
29.65121 0.00306217856628012
31.821684 0.0024477155993977
33.9993039999999 0.00194251536143408
36.1847339999999 0.00180798618746539
38.233615 0.00141692253262101
40.419504 0.00118349559957808
42.602415 0.00104840707038964
44.7859199999999 0.000955152306238132
46.9601389999999 0.00208689287483618
49.128689 0.000825049650553405
51.340777 0.0006589620211942
53.3834879999999 0.00130359203122416
55.1580839999999 0.000595655378265468
57.322322 0.000474008293798442
59.325972 0.000480906427342067
61.3363989999999 0.00079653823535531
63.5053519999999 0.000382003569928178
65.662552 0.000329864833929373
67.735006 0.00176894964924077
69.7473709999999 0.000299190516312295
71.8709749999999 0.000267501004799817
73.882079 0.000418143275401713
75.9259529999999 0.000237857500753838
78.136578 0.000211955878682798
80.005821 0.000198898988204841
81.9656299999999 0.000242371597449379
84.0052149999999 0.000162484581424941
85.9428089999999 0.000726734086528887
88.1172029999999 0.000144360744603273
90.2970039999999 0.000124849982940771
92.4869309999999 0.000219541672526653
94.642804 0.000137929860874211
96.5726069999999 0.000184011099998317
98.792881 0.000104456445270027
100.986266 9.42339492780038e-05
103.170069 8.82555076188387e-05
105.392319 0.000332543818754805
107.580902 0.000175431225171645
109.780786 7.94487008276935e-05
111.954313 7.97250665375181e-05
114.130104 9.58632008036806e-05
116.349503 5.74586868542394e-05
118.409058 4.86931230699351e-05
120.620264 4.60342963871433e-05
122.656177 8.24021236889478e-05
124.506986 9.02404516530702e-05
126.273468 5.29976876838894e-05
128.197849 5.54407600850154e-05
130.383179 4.16616369077177e-05
132.251563 7.52268505124214e-05
134.096514 3.58205820419052e-05
136.054887 3.11910350573727e-05
138.102043 3.92315623508584e-05
140.110204 2.71039335298286e-05
142.117256 4.03358175663988e-05
144.12832 0.000122917082338616
146.140419 2.6448684122239e-05
148.143043 3.00782917692962e-05
150.153576 2.32100036450274e-05
152.152991 2.85900385818874e-05
154.156398 1.63625098611934e-05
156.171224 2.14825775764505e-05
158.258816 1.61089455600537e-05
160.436578 1.38496335938258e-05
162.610246 5.6249888493482e-05
164.660878 2.45333425648609e-05
166.412796 1.14114814467705e-05
168.20635 1.91310404911606e-05
170.294302 1.97121785311141e-05
172.339765 1.08959638592439e-05
174.255002 1.03169836644582e-05
176.176372 1.56883638775297e-05
178.155198 8.08185173814642e-06
179.964433 1.05643320110876e-05
181.71069 2.00047109248023e-05
183.44984 8.23406831499031e-06
185.284349 1.02046215485314e-05
187.427631 7.03527235466151e-06
189.424238 5.60683242777821e-06
191.425945 1.34005130877811e-05
193.428771 4.80888563319628e-06
195.431366 7.01408688683218e-06
197.433887 5.44625053754784e-06
199.471128 1.23915630042102e-05
201.477503 6.4769049303526e-06
203.466323 3.60953730554169e-06
205.470285 9.47559185518589e-06
207.595497 3.31689516089873e-06
209.594901 2.82217423055585e-06
211.588104 4.1417020471358e-06
213.582655 3.14917285931108e-06
215.39974 3.05318699727901e-06
217.340313 2.01324648249847e-05
219.363236 2.151782729349e-06
221.374343 5.21320469533925e-06
223.432549 2.5806239523675e-06
225.455544 2.15754963687444e-06
227.452772 2.97621882910623e-06
229.287399 1.77755155551252e-06
231.113618 3.11894827027881e-06
232.850977 2.95501259972015e-06
234.598445 1.54884044494216e-06
236.336389 3.37620993152913e-06
238.087306 1.35514797889014e-06
239.850247 1.58509547404461e-06
241.66254 2.08854768613547e-06
243.611127 1.16499874944654e-06
245.795739 1.39916825150902e-06
247.993538 8.26365012415274e-07
250.190021 1.02097000461651e-06
252.215282 1.51525768317309e-06
254.350604 1.9659208662347e-06
256.367534 1.3861786556825e-06
258.371452 8.38692448202519e-07
260.418073 8.8949215171609e-07
262.371477 9.81951455512862e-07
264.446401 5.77030515331467e-07
266.479455 5.28491049242911e-07
268.322478 1.26110528024692e-06
270.396794 7.55525911643681e-07
272.219872 2.18447625977866e-06
273.966772 9.80138225609487e-07
275.785513 4.49492299999525e-07
277.98027 4.77287994622972e-07
280.153621 5.49762903514828e-07
282.326806 8.46292150464813e-07
284.521696 4.99040717902428e-07
286.69854 3.38197255623425e-07
288.890096 2.60618927489777e-07
291.072076 1.83432171859272e-06
293.250045 6.35886136086282e-07
295.420767 2.54129859594308e-07
297.603343 2.97014176321847e-07
299.787959 2.50440978936578e-07
301.966363 2.90400547931784e-07
304.14194 2.45358484997045e-07
306.325933 2.18022511074055e-07
308.515764 1.80045624553343e-07
310.696749 5.34009250040473e-07
312.854181 2.825629076779e-07
314.933834 3.67858098175075e-07
317.013854 2.29997691876114e-07
319.01248 1.24443476672167e-07
320.975838 4.12849268964643e-07
323.024747 1.72633806500882e-07
324.957534 1.21941397474908e-07
326.856592 2.12532327967004e-07
328.911307 1.29704462042672e-07
330.974673 2.12578762708253e-07
333.007583 1.54720503626113e-07
335.00725 1.50181684008079e-07
337.052919 1.12577556770473e-07
338.867534 9.04763356497316e-08
340.624722 7.65609381332944e-08
342.665994 1.02923977888375e-07
344.854375 8.41964226950324e-08
346.853629 5.41145975360906e-08
349.033821 5.54194375664303e-07
351.205944 5.30467905822276e-08
353.383223 8.21704845192905e-08
355.563444 7.84203278452345e-08
357.73158 6.43895891719471e-08
359.910956 5.22976235310876e-08
361.90703 7.74909720254807e-08
364.071077 3.88364292108061e-08
366.242292 5.27207474930154e-08
368.312914 4.37758133172815e-08
370.483668 6.23803676764154e-08
372.656131 4.13944400024831e-08
374.489496 6.66071523362415e-08
376.23547 2.26168806919645e-08
377.983623 4.07823313946301e-08
379.730353 2.98526544570711e-08
381.487078 2.5637902643913e-08
383.240531 2.19290658593133e-08
384.980807 3.1925208186662e-08
386.723816 3.88627167423959e-08
388.659453 2.7149101341402e-08
390.404622 1.49908134793956e-08
392.143525 4.2107343284912e-08
393.88077 2.83900849040112e-08
395.628248 1.63216863919426e-08
397.39042 2.08105443213209e-08
399.130464 2.40842599056665e-08
400.871148 1.55609848413076e-08
402.612123 2.61816966050055e-08
404.353878 1.11829488289751e-08
406.09474 9.17970665249584e-09
};
\addplot [semithick, color1]
table {%
2.68475900000067 1
9.52088300000014 0.0897552067633381
11.4530260000001 0.0241981817251772
13.467447 0.0140607115939351
15.4177870000003 0.00602391122871624
17.290352 0.00408331427437668
19.075433 0.000808146812399748
20.8700200000003 0.00155537747561955
22.6533050000007 0.00127766321052858
24.7947820000009 0.00071706472859905
26.9841350000006 0.000696868707444039
28.7817230000001 0.000854313528022512
30.5635780000002 0.000543446783628864
32.3507530000006 0.00047469131055112
34.1325500000003 0.000469185696986753
35.9079170000005 0.000378272672897542
37.686498 0.000420219355936952
39.4713550000006 0.000308726644628059
41.3765580000008 0.000281369058136524
43.5794700000006 0.000261792957284976
45.4511750000001 0.000227597072448966
47.2246570000007 0.000199679755822227
49.0070930000002 0.000198074733048661
50.8150869999999 0.000155846072429997
52.630658 0.000145689959146884
54.432624 0.000141820296638258
56.2277180000001 0.000122331211446387
58.1778330000006 0.000115931264263827
60.2112630000001 0.000102471437404643
61.9935810000006 0.000118107572153779
63.9802160000008 0.000103013276388464
65.9745900000007 8.25140638173735e-05
68.0249350000004 7.98367549348567e-05
70.3149960000001 7.8296655715013e-05
72.4467630000008 5.94798291262974e-05
74.4841050000005 5.62678607314492e-05
76.4981630000002 7.83912860344326e-05
78.6957110000003 5.22249198803824e-05
80.6600180000005 4.72995040938741e-05
82.7744440000006 5.50386125643815e-05
85.0455100000008 4.33693623754731e-05
87.241505 3.90597992496879e-05
89.2033370000008 4.31168002199793e-05
90.9842530000005 3.59395668742404e-05
92.7594120000003 3.0593568008885e-05
94.5388360000006 2.51722790449838e-05
96.3199100000002 2.29344280939236e-05
98.0925280000001 2.15044623623177e-05
99.8725340000001 1.80100639273911e-05
101.652959 2.68935393340049e-05
103.437574 1.86206589665474e-05
105.220135 2.89566768755171e-05
107.000765000001 1.57228544860959e-05
108.800694 1.95600711052586e-05
110.582067 1.38702850663627e-05
112.360087 1.22392287571959e-05
114.134575 1.11907818363793e-05
115.914047 9.82061222498215e-06
117.68742 9.10229177514851e-06
119.462627000001 1.03777704124574e-05
121.239361000001 9.49194524278866e-06
123.023943 1.20248894899714e-05
124.799133 7.4277346701762e-06
126.57812 7.12088822973319e-06
128.35904 7.89190413789684e-06
130.140007 5.27414886426533e-06
131.922677 4.84791943807432e-06
133.703139 7.56903042899967e-06
135.479884 4.18236168870855e-06
137.258448 4.98849413070217e-06
139.03992 4.44395294841323e-06
140.815202000001 3.98192252880781e-06
142.608928000001 4.79470834798883e-06
144.400836000001 4.4846828174337e-06
146.173163 3.17189709787398e-06
147.948825 2.40120195830553e-06
149.744449000001 2.59609605026357e-06
151.523603000001 2.72248316304636e-06
153.397573 2.06232523436312e-06
155.175636000001 1.59103005980662e-06
156.951009 2.39021719224409e-06
158.727910000001 1.47240642569083e-06
160.507628 1.71639819407172e-06
162.292639 1.7055619774947e-06
164.069826000001 1.87006245930523e-06
165.845960000001 1.30289939557844e-06
167.668505000001 1.62403253886185e-06
169.465626 9.80402248552936e-07
171.247638000001 8.6523124730344e-07
173.02656 1.20855767548888e-06
174.801539 8.66719947685454e-07
176.607584 8.94372216832799e-07
178.403669 8.1732192126493e-07
180.206085000001 9.13640837367273e-07
182.008294 8.99495447912707e-07
184.114714 5.45223059923352e-07
186.127456 4.2815130376724e-07
187.948847000001 8.58640054267441e-07
189.733455000001 4.49427404009971e-07
191.521224 4.79750309884542e-07
193.300203000001 3.61600588056346e-07
195.089735 4.39717121578615e-07
196.870677000001 4.10208723874874e-07
198.649757 2.61074594441271e-07
200.435782 2.66800289876789e-07
202.212961 2.6670250682126e-07
203.996567 2.6736831379393e-07
205.774208000001 2.53623677126009e-07
207.555461000001 2.20509852960166e-07
209.333554000001 2.54016287100818e-07
211.111846000001 2.09284054464626e-07
212.886751 1.70250522984431e-07
214.664913000001 1.81707234575347e-07
216.447672 1.40241425886295e-07
218.265399000001 2.49650235202672e-07
220.046081 1.04975469491967e-07
221.830144 9.91257054173385e-08
223.671245 1.41893010533713e-07
225.683433 1.09369448336331e-07
227.715528000001 1.06521310610131e-07
230.080804 1.08027941365595e-07
232.271871000001 6.92103857854144e-08
234.430996 5.72674845590428e-08
236.469399000001 6.79709603550055e-08
238.761022000001 6.05855440771429e-08
240.795230000001 5.24939141833735e-08
242.930708000001 4.72284134367944e-08
245.030174 4.6772427232461e-08
247.090474000001 4.95331791908153e-08
249.029405 4.94771659940246e-08
251.157857 4.26021426236251e-08
253.198861000001 3.68297433523107e-08
255.221703 5.04028096371161e-08
257.252903000001 3.07372696040459e-08
259.314251000001 3.85694056198679e-08
261.451906 2.39313202871365e-08
263.66409 3.38912251494762e-08
265.738933000001 3.04436501485057e-08
267.670029000001 2.05422662705129e-08
269.447968 1.47871925503288e-08
271.226094000001 2.01171191249903e-08
273.000998 1.37990722253255e-08
274.779443 1.91533956650386e-08
276.556570000001 1.14939754282137e-08
278.334206 1.32231279634868e-08
280.110077 9.41393594701483e-09
};
\addplot [semithick, color1, dashed]
table {%
2.44708700000001 1
9.48120500000005 0.0897552067633381
11.3040120000001 0.0241981817251772
13.5066219999999 0.0140607115939351
15.6653110000002 0.00602391122871624
17.8146900000002 0.00408331427437668
19.86321 0.000808146812399748
22.0784210000002 0.00155537747561955
23.9976109999998 0.00127766321052858
26.1752999999999 0.00071706472859905
28.1629889999999 0.000696868707444039
30.3669970000001 0.000854313528022512
32.376604 0.000543446783628864
34.4362860000001 0.00047469131055112
36.4520510000002 0.000469185696986753
38.5672140000001 0.000378272672897542
40.6276560000001 0.000420219355936952
42.6471799999999 0.000308726644628059
44.758491 0.000281369058136524
46.9160729999999 0.000261792957284976
49.12392 0.000227597072448966
51.4000470000001 0.000199679755822227
53.7351469999999 0.000198074733048661
55.8879689999999 0.000155846072429997
57.7686269999999 0.000145689959146884
59.5381980000002 0.000141820296638258
61.739689 0.000122331211446387
63.8825219999999 0.000115931264263827
65.8923359999999 0.000102471437404643
68.1246449999999 0.000118107572153779
70.2451890000002 0.000103013276388464
72.274782 8.25140638173735e-05
74.3030800000001 7.98367549348567e-05
76.2846509999999 7.8296655715013e-05
78.07323 5.94798291262974e-05
80.1907409999999 5.62678607314492e-05
82.4043449999999 7.83912860344326e-05
84.6665889999999 5.22249198803824e-05
86.8682020000001 4.72995040938741e-05
88.8647940000001 5.50386125643815e-05
90.6383190000001 4.33693623754731e-05
92.4073039999998 3.90597992496879e-05
94.1774399999999 4.31168002199793e-05
96.1815759999999 3.59395668742404e-05
98.2315659999999 3.0593568008885e-05
100.399218 2.51722790449838e-05
102.473791 2.29344280939236e-05
104.506341 2.15044623623177e-05
106.369703 1.80100639273911e-05
108.579908 2.68935393340049e-05
110.515952 1.86206589665474e-05
112.446886 2.89566768755171e-05
114.541262 1.57228544860959e-05
116.537728 1.95600711052586e-05
118.755082 1.38702850663627e-05
120.729522 1.22392287571959e-05
122.513586 1.11907818363793e-05
124.285841 9.82061222498215e-06
126.084847 9.10229177514851e-06
127.862825 1.03777704124574e-05
129.64208 9.49194524278866e-06
131.415474 1.20248894899714e-05
133.186965 7.4277346701762e-06
134.950079 7.12088822973319e-06
137.044306 7.89190413789684e-06
139.142389 5.27414886426533e-06
140.921876 4.84791943807432e-06
142.69226 7.56903042899967e-06
144.46234 4.18236168870855e-06
146.233097 4.98849413070217e-06
148.011531 4.44395294841323e-06
149.774527 3.98192252880781e-06
151.550027 4.79470834798883e-06
153.323998 4.4846828174337e-06
155.093904 3.17189709787398e-06
156.863949 2.40120195830553e-06
158.635755 2.59609605026357e-06
160.407118 2.72248316304636e-06
162.171163 2.06232523436312e-06
163.94967 1.59103005980662e-06
165.72121 2.39021719224409e-06
167.49833 1.47240642569083e-06
169.361115 1.71639819407172e-06
171.333939 1.7055619774947e-06
173.161076 1.87006245930523e-06
174.938434 1.30289939557844e-06
176.711252 1.62403253886185e-06
178.481047 9.80402248552936e-07
180.26202 8.6523124730344e-07
182.363264 1.20855767548888e-06
184.16578 8.66719947685454e-07
185.938707 8.94372216832799e-07
187.71102 8.1732192126493e-07
189.484382 9.13640837367273e-07
191.256512 8.99495447912707e-07
193.03421 5.45223059923352e-07
194.81381 4.2815130376724e-07
196.587835 8.58640054267441e-07
198.787098 4.49427404009971e-07
200.719082 4.79750309884542e-07
202.494713 3.61600588056346e-07
204.265645 4.39717121578615e-07
206.256821 4.10208723874874e-07
208.482606 2.61074594441271e-07
210.490528 2.66800289876789e-07
212.520834 2.6670250682126e-07
214.62833 2.6736831379393e-07
216.649956 2.53623677126009e-07
218.861241 2.20509852960166e-07
220.810875 2.54016287100818e-07
222.602523 2.09284054464626e-07
224.373407 1.70250522984431e-07
226.147017 1.81707234575347e-07
227.925457 1.40241425886295e-07
229.696143 2.49650235202672e-07
231.468376 1.04975469491967e-07
233.241464 9.91257054173385e-08
235.015393 1.41893010533713e-07
236.794733 1.09369448336331e-07
238.568188 1.06521310610131e-07
240.352644 1.08027941365595e-07
242.140796 6.92103857854144e-08
243.928967 5.72674845590428e-08
246.129513 6.79709603550055e-08
247.933135 6.05855440771429e-08
249.922673 5.24939141833735e-08
252.124572 4.72284134367944e-08
254.191778 4.6772427232461e-08
256.232869 4.95331791908153e-08
258.318023 4.94771659940246e-08
260.354981 4.26021426236251e-08
262.179744 3.68297433523107e-08
264.119137 5.04028096371161e-08
266.079804 3.07372696040459e-08
267.909606 3.85694056198679e-08
269.680738 2.39313202871365e-08
271.450744 3.38912251494762e-08
273.22483 3.04436501485057e-08
274.993581 2.05422662705129e-08
276.762957 1.47871925503288e-08
278.535242 2.01171191249903e-08
280.306937 1.37990722253255e-08
282.080647 1.91533956650386e-08
283.851832 1.14939754282137e-08
285.623389 1.32231279634868e-08
287.434128 9.41393594701483e-09
};
\end{axis}

\end{tikzpicture}\hfill%
%        % This file was created by matplotlib2tikz v0.6.3.
\begin{tikzpicture}

\definecolor{color1}{RGB}{000,125,122} % MPG green
\definecolor{color0}{HTML}{FF9933} %EI orange

\begin{axis}[
title={i=2},
xlabel={CPU time},
ylabel={relative residual},
xmin=-10.7675264499996, xmax=274.57648745,
ymin=3.39783558964469e-09, ymax=2.5309063651565,
ymode=log,
width=\figwidth,
height=\figheight,
tick align=outside,
x grid style={lightgray!92.026143790849673!black},
y grid style={lightgray!92.026143790849673!black},
mystyle
]
\addplot [semithick, color0]
table {%
2.23323300000015 1
4.03679699999998 0.807403182734015
5.81138899999996 0.209157114227959
7.54248500000017 0.0574498774819136
9.27355499999999 0.0446522480661267
11.004422 0.021515678867764
12.738863 0.01898376625362
14.465295 0.0135876740716136
16.199541 0.0097633141335379
17.930842 0.00719162288226795
19.6591640000001 0.0056447829620256
21.3897960000002 0.00546213918872563
23.119381 0.00782191004677242
24.848608 0.00354062408711678
26.5757180000001 0.00293077112260834
28.3027070000001 0.00226161916542685
30.054421 0.00211147008791219
31.788939 0.00158923643794072
33.5194040000001 0.00132321276618765
35.2489370000001 0.00122176486723707
36.980088 0.00126752461784391
38.713573 0.00227100326614295
40.4439030000001 0.000865550973060411
42.1751040000001 0.000757840274601954
43.9041260000001 0.00132290242930215
45.6441710000001 0.000650559728886852
47.377246 0.000497834137512665
49.106417 0.000484482082311342
50.834787 0.000397790354591021
52.564126 0.00074561922245561
54.298403 0.00168431595146445
56.0264070000001 0.000324518751709368
57.754455 0.000279859328742144
59.4802550000002 0.000238440666883214
61.212475 0.000220165377066704
62.941421 0.000338852078456284
64.671969 0.000177489429138718
66.40236 0.000142270254458325
68.134266 0.000159501309910695
69.8656170000002 0.000212061945448712
71.595196 0.000209648088037511
73.328164 0.000111691730066137
75.059514 9.16810210602272e-05
76.7898130000001 0.000180035915238713
78.5198210000001 8.83870313585687e-05
80.2494960000001 0.000130595844544588
81.9810230000001 7.02311933404634e-05
83.713649 5.68297645914652e-05
85.440372 0.000258934230200693
87.167496 5.04355111432053e-05
88.896962 4.73762192540598e-05
90.631707 8.16791906138276e-05
92.3591390000001 4.05947264633152e-05
94.0856760000001 5.24617180539135e-05
95.815521 3.24212464087842e-05
97.5460250000001 2.4368516382776e-05
99.278431 3.50133555204801e-05
101.006898 2.9822027252813e-05
102.74002 2.07094735804864e-05
104.470806 3.7457645772989e-05
106.199272 1.91396065178415e-05
107.926806 2.13446692831919e-05
109.654522 2.75260994599881e-05
111.383452 1.49341099551232e-05
113.114154 1.0860205766136e-05
114.84245 1.46389166817763e-05
116.57436 1.78671602748245e-05
118.303292 1.11031852053794e-05
120.032417 9.23627940321391e-06
121.758818 1.15838426987398e-05
123.484742 1.16625303453547e-05
125.212328 7.09566927830926e-06
126.94636 5.49690866714071e-06
128.678882 8.41254450610259e-06
130.405744 1.15476033958252e-05
132.136671 2.10957994179589e-05
133.867962 4.65859258687873e-06
135.59438 4.53046830906909e-06
137.322796 5.00417946939788e-06
139.050205 3.70827709781267e-06
140.780522 4.18350189483589e-06
142.509502 4.8532798718905e-06
144.237895 2.91052216683018e-06
145.964434 2.11123759840904e-06
147.695266 1.69706825613142e-05
149.422784 1.93635846353211e-06
151.150683 2.49724846501887e-06
152.879739 2.1482938601225e-06
154.608752 2.07555137796272e-06
156.338053 1.33382577306514e-06
158.065459 1.11569496150927e-06
159.795979 1.78186305929105e-06
161.528355 1.02950308906753e-06
163.265383 2.06075102679317e-06
164.992944 1.14267862893736e-06
166.720162 9.38344168164203e-07
168.450142 1.37073445321779e-06
170.178136 6.25581512992715e-07
171.907618 9.48062299177348e-07
173.635459 6.90850374150217e-07
175.363107 9.43664068923229e-07
177.094369 4.03508236254046e-07
178.823905 3.75035282333256e-06
180.555305 3.39710293232837e-07
182.280634 5.04279806626755e-07
184.010077 4.11847057317572e-07
185.739545 3.99350062824171e-07
187.468845 2.19606602154228e-07
189.201372 5.1033114827104e-07
190.935149 2.37306173851025e-07
192.669345 3.76463096543722e-07
194.397959 9.57076822669299e-07
196.124922 2.69570308643675e-07
197.85444 1.65084644131239e-07
199.583983 2.52828564942088e-07
201.31456 1.18417407378208e-07
203.044323 1.89798195459004e-07
204.775942 1.82703903127406e-07
206.503371 1.19046156343376e-07
208.233522 2.68669391511545e-07
209.976158 1.57975850567369e-07
211.706951 1.45138287498865e-07
213.432657 9.28719796512181e-08
215.162087 7.58191210310504e-08
216.891019 9.79943289968511e-08
218.621067 5.74798952982427e-08
220.352809 7.00413077416874e-08
222.083123 4.70681097360707e-08
223.809592 7.32093012470285e-08
225.538216 3.83147668879135e-07
227.266051 3.72633515029962e-08
228.994781 5.65965518493804e-08
230.725712 3.70864168336453e-08
232.456286 3.33515292710637e-08
234.189403 4.27205437214068e-08
235.918917 4.71793225355922e-08
237.646863 3.856842018558e-08
239.374565 2.7975251130289e-08
241.104544 4.96557065601286e-08
242.833648 2.22926349586089e-08
244.560255 2.18027178469088e-08
246.289335 4.37559402946719e-08
248.02228 1.47637472246613e-08
249.75097 2.74915914415221e-08
251.480437 1.62098218694057e-08
253.210731 1.16568639792656e-08
254.940315 8.89342010875944e-09
};
\addplot [semithick, color1]
table {%
2.20521499999995 1
8.78531399999974 0.0911427645143914
10.8442779999996 0.0265460284643973
12.8905749999994 0.0173918933167727
14.782107 0.0106697408679668
16.6681980000003 0.00718434276370028
18.8659129999996 0.0059451486867994
21.0773410000002 0.00433426407875372
23.0886739999996 0.00320165343098197
25.0245599999998 0.00236278380739917
26.9264670000002 0.00220587518437407
28.7296390000001 0.00168009747355593
30.8347510000003 0.00129557758420921
32.9836429999996 0.00110850750341525
35.040086 0.000899516553039836
37.0740470000001 0.000798780552315318
39.127512 0.000665987531215606
41.172458 0.000532838217230583
43.2097830000002 0.0005156920881273
45.0632139999998 0.0004500637145942
46.9814399999996 0.000371819768007053
49.0356620000002 0.000339162631367577
51.0772740000002 0.000284859629643476
53.017433 0.000248735298034049
55.2340589999994 0.000236389684194195
57.3281019999995 0.000196114453155886
59.5153099999998 0.000184813127254151
61.5473929999998 0.000163807909102225
63.5107399999997 0.000142435892639105
65.4733079999996 0.000122934130068368
67.5906679999998 0.000116381051480572
69.6443650000001 0.000124709158517965
71.7482460000001 0.000113119160576514
73.7644899999996 0.000102659673570746
75.9841859999997 7.52584160591281e-05
77.8967809999995 7.02924360700946e-05
79.937175 9.75968738366879e-05
81.8139629999996 6.49425179471096e-05
83.5892779999995 5.05939243124861e-05
85.3635159999994 4.30297196663709e-05
87.1372329999995 3.87138524133327e-05
88.9091710000002 3.85377869538616e-05
90.6824569999999 4.7410047074645e-05
92.4569089999995 2.75073774483697e-05
94.227022 2.24484045605451e-05
95.9975509999995 2.10531286132097e-05
97.7730679999995 2.93697223491933e-05
99.6981180000002 1.7362107023584e-05
101.838545 1.48300289899869e-05
103.628412 1.44804775138738e-05
105.417717 1.63985687657753e-05
107.207809 1.05779312175704e-05
108.999237 8.69882987843914e-06
110.798530999999 8.24829580366975e-06
112.592328 8.20662625990197e-06
114.363264 8.12357072589293e-06
116.137008 7.38463308911791e-06
117.914831999999 6.6437176821422e-06
119.713724 4.81033387294915e-06
121.483782 4.04421314046696e-06
123.251212 3.38553932464208e-06
125.032056999999 5.60370681452225e-06
126.806901 3.03374256979834e-06
128.576583 3.63335992710881e-06
130.34662 2.37671192775622e-06
132.118011 2.32211118284288e-06
133.888812 2.54800342613899e-06
135.660213 1.55688579704828e-06
137.438915 1.24851917305657e-06
139.227248 1.089411429342e-06
140.99926 1.19780745028929e-06
142.773167 1.16125314997807e-06
144.540917 7.52912778914789e-07
146.31206 6.68472894720509e-07
148.081539 8.36895725441414e-07
149.882767999999 4.93597812638444e-07
151.688638 7.58270738226459e-07
153.475676 4.82776327203579e-07
155.248071 5.51510175313411e-07
157.020111 3.65682351411149e-07
158.792107 5.77271729765553e-07
160.621425 2.71195426786616e-07
162.400117 2.5768860423996e-07
164.169874 3.01334364369782e-07
165.953799 1.75072512846909e-07
167.731548 2.19792227628896e-07
169.503239 1.69753392356333e-07
171.279009 2.27710265932426e-07
173.058907 1.4584269290771e-07
174.833063 1.39758028486441e-07
176.605025 1.10000419769023e-07
178.37478 1.02996819657457e-07
180.144737 7.4177566513742e-08
181.915619 9.11045871190199e-08
183.689339 5.98608589855025e-08
185.465533 8.31116232384855e-08
187.231576 4.82981438922929e-08
189.005932 3.9313119457933e-08
190.780133 3.06257332725343e-08
192.548211 5.15425351329012e-08
194.326098 3.15086626360504e-08
196.097565999999 3.10649939652032e-08
197.878272 2.06028889207575e-08
199.649259 3.13353552281446e-08
201.421344 1.89835659628716e-08
203.196952 1.75218332149065e-08
204.967234 1.63770858570623e-08
206.743445999999 1.68780845709502e-08
208.530483 1.15569242916659e-08
210.306342 1.01080054544374e-08
212.07871 1.13059852481353e-08
213.851658 9.63252484795002e-09
};
\addplot [semithick, color1, dashed]
table {%
2.20265600000039 1
8.39207400000032 0.960108891749278
10.2803740000004 0.821902534928197
12.0624830000002 0.196835447989596
13.841692 0.039613540247409
15.6208280000001 0.0310581421582737
17.7843150000003 0.0239279060028304
19.8387010000001 0.0173388139424129
21.6743020000004 0.0105134891346929
23.4697120000001 0.00895441183510875
25.2458820000002 0.0071193630131582
27.0242410000001 0.00364221088915829
28.7952280000004 0.0066296836407533
30.5741980000003 0.00243713215745911
32.3535490000004 0.00210760798623358
34.1284640000003 0.00153912159528546
35.9083410000003 0.00125331772586834
37.6794440000003 0.000907016993662114
39.449325 0.000742226178742439
41.2271380000002 0.000582553409413649
43.0063490000002 0.000419886026584176
44.7793810000003 0.00153282829563233
46.5548050000002 0.000403459873943984
48.331173 0.000330101827585343
50.1011780000003 0.000237917923952033
51.8790360000003 0.000263672978670691
53.6576420000001 0.000531303660228161
55.4330880000002 0.000244555471023451
57.212665 0.000197126876941876
58.9884110000003 0.000206806158936622
60.7638880000004 0.000357332207983173
62.627387 0.000331897623760676
64.8470240000001 0.000146051101746345
66.9545550000003 0.000122415796442465
69.146534 0.000196121450595083
71.37129 0.000117566273297109
73.2489310000001 0.000102390016934501
75.0252650000002 0.00015563965384959
76.7969640000001 9.65997736052565e-05
78.576665 0.000131350419069771
80.344505 9.03582983917678e-05
82.1183120000001 0.000117731322645057
83.8966680000003 8.13841360185274e-05
85.6671800000004 5.73132218272188e-05
87.4364300000002 7.7357388503192e-05
89.2100730000002 4.62628595824608e-05
90.982751 4.06640927057248e-05
92.7592790000003 0.000102327257960275
94.5330700000004 4.15278018269417e-05
96.3063320000001 0.000124252495365146
98.0801290000004 3.12327030224113e-05
99.8545320000003 2.73434972305346e-05
101.636629 3.82888268961566e-05
103.412253 2.41613023480974e-05
105.183811 5.12768699316251e-05
106.949635 1.98095942856697e-05
108.722651 1.64411766378594e-05
110.496498 0.00013418923766431
112.280453 1.74986866700687e-05
114.052994 2.2059157100119e-05
115.824 1.3186234161389e-05
117.597758 1.11276703428786e-05
119.364425 2.2066272925433e-05
121.135587 9.73338148095923e-06
122.90788 1.52790170636669e-05
124.684546 8.34966065266429e-06
126.478663 1.33473384860554e-05
128.251301 1.02804222320502e-05
130.021207 1.29953576758581e-05
131.796606 8.76002647933445e-06
133.573244 5.56399164533381e-06
135.353022 4.80535741883482e-06
137.343483 4.14920595453193e-06
139.263959 6.5565486933784e-06
141.366361 3.4743263190249e-06
143.221216 1.61426090560626e-05
145.01545 7.62553475203518e-06
146.795606 5.09525369594166e-06
148.57644 3.13173925455262e-06
150.357493 4.75472379797493e-06
152.147407 2.23982481630715e-06
154.281949 1.81793586468465e-06
156.262348 1.70386248275905e-06
158.473515 1.38531982899481e-06
160.32273 7.79910084906167e-06
162.097597 1.5615030248491e-06
163.873214 2.58449899987888e-06
165.652032 1.46724553201278e-06
167.427172 1.15418504257392e-06
169.640448 1.16982411458712e-06
171.679194 1.50497170657855e-06
173.584829 7.84513806323951e-07
175.383618 7.59859576128397e-07
177.157574 2.0422034789519e-06
178.930358 1.06693827126178e-06
180.714068 1.56663456403479e-06
182.499154 5.54192505147945e-07
184.28023 6.79448730416919e-07
186.071104 3.73861418701483e-07
187.847015 5.02507239114664e-07
189.612863 6.62370933100169e-07
191.391086 3.70782293227745e-07
193.205276 2.66373701078625e-06
194.979136 2.612339775198e-07
196.754277 4.3944297856997e-07
198.537085 2.353965731216e-07
200.330738 5.05207327733749e-07
202.13843 1.83415484009845e-07
203.936012 2.01441376352582e-07
205.718989 1.67444257962236e-07
207.546884 3.03698008294722e-07
209.65827 1.64986057330594e-07
211.719164 2.18924973512261e-07
213.662561 1.10959617749594e-07
215.485359 1.49280169683899e-07
217.262085 8.14285903056834e-08
219.034133 1.11310836447607e-07
220.809988 1.34266096179018e-07
222.586911 6.80394931912028e-08
224.363932 4.80950218108747e-07
226.135798 5.43959539064862e-08
227.910131 1.07519766924359e-07
229.676946 5.96471338205045e-08
231.446443 6.295481739129e-08
233.213998 3.90389489301695e-08
234.998337 5.79955768688655e-08
236.770416 5.97064885229579e-08
238.543101 3.99085585959791e-08
240.315344 6.25429129588215e-08
242.084867 3.84553495606586e-08
243.852166 7.54226202105868e-08
245.627294 2.91737790013686e-08
247.398213 2.88328995319557e-08
249.171347 1.81785505910004e-08
250.946717 2.15324810435769e-08
252.721498 2.16051286324768e-08
254.516855 1.62282720235693e-08
256.294088 1.17592692308876e-07
258.063903 1.13892026344605e-08
259.833072 1.31364398011166e-08
261.606305 8.59960372158702e-09
};
\end{axis}

\end{tikzpicture}\hfill\null%
%        % This file was created by matplotlib2tikz v0.6.3.
\begin{tikzpicture}

\definecolor{color1}{RGB}{000,125,122} % MPG green
\definecolor{color0}{HTML}{FF9933} %EI orange
\begin{axis}[
title={i=3},
xlabel={CPU time},
ylabel={relative residual},
xmin=-8.03392694999997, xmax=209.30930795,
ymin=2.63838746133272e-09, ymax=2.56157860317183,
ymode=log,
width=\figwidth,
height=\figheight,
tick align=outside,
x grid style={lightgray!92.026143790849673!black},
y grid style={lightgray!92.026143790849673!black},
mystyle
]
\addplot [semithick, color0]
table {%
1.84531100000004 1
3.63934300000005 0.818272315902229
5.404943 0.242259321513288
7.13650500000017 0.0710913023207334
8.86690300000009 0.0556398448146228
10.600197 0.0293573872881461
12.3314990000001 0.0244467802092831
14.0703100000001 0.018876000113289
15.8064060000002 0.0146852573463301
17.5382510000002 0.0100442269234059
19.2728090000001 0.00769706451104584
21.0059250000002 0.0147300271265303
22.7372370000001 0.0066261787876248
24.47236 0.00462561180337677
26.2061310000001 0.00371654280960053
27.9366520000001 0.0031330192027257
29.6678860000002 0.00263286899937473
31.399856 0.00196412558348577
33.1314190000001 0.00159251555946919
34.862161 0.00136424434615748
36.5930270000001 0.00231762050031698
38.324474 0.00122468103688415
40.0567550000001 0.000947809494064252
41.789892 0.000706626621682135
43.528192 0.000790674521935505
45.2727120000002 0.000949542829447962
47.0064400000001 0.000556709642374778
48.7374170000001 0.000484319921080416
50.4693100000002 0.000377723181005276
52.2014300000001 0.000623244987678761
53.9320110000001 0.0010554229636783
55.6622560000001 0.000272202217708809
57.395722 0.000234619553298099
59.127892 0.000194464164953167
60.8581920000001 0.000159516955075347
62.5892690000001 0.000261338034655031
64.3197010000001 0.000132677495893211
66.0507610000002 9.28776262420741e-05
67.782101 8.72960374667698e-05
69.5138820000002 0.000355980076217417
71.2478620000002 0.000110314641985049
72.985437 6.70357473157585e-05
74.717842 5.26030935087663e-05
76.4495350000002 4.6612590571314e-05
78.178351 6.41647155277276e-05
79.9082100000001 6.41451134254005e-05
81.642204 3.66401395363573e-05
83.3753320000001 2.82704255270533e-05
85.1096730000002 0.000171431785847554
86.8424460000001 2.38691572997502e-05
88.5760950000001 1.7702520771327e-05
90.3056250000002 3.12936155691308e-05
92.0360910000002 1.90744630601135e-05
93.767421 1.27879934823477e-05
95.5005270000001 9.24395305244678e-06
97.233477 8.47735722519072e-06
98.9652390000001 2.39824869043666e-05
100.697648 7.73857605881764e-06
102.428457 8.93455016341e-06
104.168666 5.81913773060934e-06
105.900333 1.09251957080432e-05
107.630179 4.43513888357525e-06
109.364098 5.93268483333126e-06
111.099936 3.48712072513073e-06
112.833869 6.53278212473133e-06
114.565531 2.81284629139959e-06
116.297029 4.57188042380182e-06
118.030184 2.86303222681479e-06
119.76401 2.24290652882191e-06
121.497428 2.52445594546981e-06
123.231095 1.78064188233118e-06
124.962681 1.47538838909502e-06
126.693404 1.4612330842354e-06
128.425635 1.18829122215443e-06
130.157435 2.3301508608057e-06
131.89169 2.66631313960355e-06
133.62914 1.37337200020953e-06
135.359972 6.82829221450436e-07
137.091397 7.08653441443475e-07
138.820801 8.16341906620547e-07
140.553896 6.18384548198649e-07
142.284509 4.87808863166941e-07
144.020098 3.27400830837085e-07
145.753025 1.40613222722748e-06
147.4866 1.04775348040622e-06
149.218022 2.74444745464533e-07
150.953681 2.40215681854498e-07
152.688918 2.80679847491535e-07
154.421072 3.9119459747821e-07
156.153987 1.6877731031517e-07
157.887563 1.30363645981031e-07
159.617999 1.06284502519644e-07
161.350546 2.93965754844562e-07
163.082691 1.29074461612528e-07
164.819725 7.77591938405139e-08
166.551817 2.13230220749015e-07
168.283072 5.63950486028745e-08
170.01434 6.69144576779616e-08
171.746919 6.2682215350718e-08
173.478684 5.26029015069904e-08
175.206333 7.73653963252643e-08
176.936161 4.1348679137388e-08
178.664919 7.34040638254236e-08
180.393548 5.45694037316012e-08
182.125493 2.75133692473097e-08
183.853792 2.21402557574557e-08
185.581746 2.38147117772554e-08
187.309638 1.64391008919924e-08
189.040387 2.91792117845172e-08
190.772977 6.40116666148779e-08
192.500785 1.19096080549985e-08
194.238735 2.7067697549069e-08
195.96873 1.00314415199265e-08
197.698945 1.33535134615926e-08
199.43007 6.75843686782672e-09
};
\addplot [semithick, color1]
table {%
2.12792699999954 1
8.84107799999947 0.10156732384356
10.693037 0.034333629422772
12.7866260000001 0.021487107744974
14.6559509999997 0.0138854234494908
16.4537069999997 0.00955073799095624
18.2259359999998 0.00777911114272731
20.0098239999998 0.00605273401915171
21.7910179999999 0.00488180646261698
23.5910599999997 0.00356105219105137
25.3673819999995 0.0029941091248344
27.1502199999995 0.00245762483640191
28.9224690000001 0.00191275880848553
31.0694450000001 0.00157445887118763
33.0201879999995 0.00127088380669054
34.8001399999994 0.00122616446681266
36.574713 0.000919890690026271
38.6502860000001 0.000760802437338683
40.747875 0.000727005094669489
42.7902729999996 0.000667927679653431
44.8283469999997 0.000519185496661831
46.8199089999998 0.000458152528791348
48.6495100000002 0.000410201370307888
50.4270449999995 0.000333359942648819
52.2014170000002 0.000314954159889192
53.9859120000001 0.00022964971893214
55.8236429999997 0.000220898959037529
57.6145239999996 0.000186867176685361
59.6345119999996 0.000151058030942693
61.4142629999997 0.000125294983620029
63.1867949999996 0.000117519028426575
64.9619039999998 9.30152481163594e-05
66.7528059999995 0.000118722038892285
68.6239519999999 7.8521292930063e-05
70.7215919999999 5.37699796259381e-05
72.6138019999999 4.55365510355356e-05
74.7962499999994 3.80995269133743e-05
76.7386099999994 4.65841952017412e-05
78.5184099999997 3.23320843077165e-05
80.2916559999994 2.53200748840635e-05
82.0679279999995 2.0877762682147e-05
83.862298 2.14041361689364e-05
85.6684289999994 1.99249040396012e-05
87.4404720000002 1.20244797711863e-05
89.2141799999999 1.07392430590736e-05
91.0135719999998 1.32194145497855e-05
93.2168499999998 8.30165360466446e-06
95.1015779999998 6.44623980666414e-06
97.0426799999996 5.51847291357396e-06
99.2632640000002 4.10323806522608e-06
101.487516 4.3511679118104e-06
103.439813999999 3.96337805581642e-06
105.253275999999 2.48620031448565e-06
107.079567 2.2368982930796e-06
109.276207 2.68876788633142e-06
111.371889999999 2.1576943229498e-06
113.451453 2.12493141145042e-06
115.440033 1.23668312345997e-06
117.425059 1.09024037994297e-06
119.276572 8.14937324216612e-07
121.104171 6.39699828103158e-07
123.027063 6.07664316514598e-07
124.945569 6.45717500065571e-07
126.718752 4.83414718006493e-07
128.494209 4.64397473398051e-07
130.291104 3.2387585852231e-07
132.091147 3.59119990097014e-07
133.871693999999 2.37948425856868e-07
135.645395 1.72862562392539e-07
137.592868 1.43998487909675e-07
139.46722 1.26411093038564e-07
141.281457 1.62786090936517e-07
143.094139 8.02907907287171e-08
145.287791 6.30940909105587e-08
147.259579 7.01825829657808e-08
149.130424999999 5.47233223150631e-08
151.102147 6.53818319404438e-08
152.884086 3.89237122835261e-08
154.872299 5.74004655882073e-08
156.701502 3.09746006127651e-08
158.499414 2.8447839205526e-08
160.295668 1.74408932904179e-08
162.08943 1.88443915062669e-08
164.171996 1.43130021924602e-08
166.330868 1.04925205360079e-08
168.33521 1.17765795673953e-08
170.323141 7.80492179169907e-09
};
\addplot [semithick, color1, dashed]
table {%
2.63458900000023 1
9.32964399999992 0.140234190716474
11.2262180000002 0.0512498403028979
13.424422 0.0280708642469202
15.4749980000001 0.0184998598113033
17.6751800000002 0.0112683629692619
19.5583350000002 0.00749016067852361
21.342517 0.0037426305078876
23.1208850000003 0.00186446069675283
25.2303380000003 0.00157309495298194
27.452612 0.00166395518235858
29.6571159999999 0.00188486189324893
31.6097690000001 0.001426863485096
33.818162 0.00102649528438541
35.6684519999999 0.000887470478880295
37.4422060000002 0.000664289722469876
39.226991 0.000646147116547214
41.244616 0.000471877485768521
43.0285650000001 0.000482072435264529
44.9018420000002 0.000419740174804969
47.1196660000001 0.000375381007429798
49.3374269999999 0.000282756350654851
51.5616500000001 0.000292261590897341
53.766282 0.000373988783973723
55.9920179999999 0.000195543776436742
57.9747700000003 0.000174472952814121
60.1900169999999 0.000171992092481241
62.0799500000003 0.000124090426324692
63.8798590000001 0.0001104995598354
65.8381119999999 8.39053655074477e-05
68.039534 7.39172572510232e-05
69.935019 6.63833843995661e-05
71.7188100000003 5.75653480938942e-05
73.4921770000001 7.08269337734246e-05
75.2676959999999 4.90821714019366e-05
77.0408729999999 3.99156423344997e-05
78.815705 4.50778736578782e-05
80.5881279999999 3.14672373496005e-05
82.3622500000001 2.48285416313379e-05
84.130662 1.99257608329434e-05
85.9048109999999 3.34378595366055e-05
87.6807720000002 1.78904078155084e-05
89.4508180000003 1.8754241138076e-05
91.223516 1.37056267358183e-05
92.9987030000002 9.51284721684955e-06
94.7774060000002 7.97084155188818e-06
96.554838 1.02522189685803e-05
98.327147 6.30224459223497e-06
100.0976 4.90510633707394e-06
101.873347 3.92506329866833e-06
103.645312 3.29763997091087e-06
105.41498 2.65952474594784e-06
107.187989 2.59854696912444e-06
108.99257 2.42405400746596e-06
110.768358 1.57118550648773e-06
112.548377 2.28992205641001e-06
114.322016 1.42445508218886e-06
116.53573 1.6225912074662e-06
118.607432 1.17604155574557e-06
120.404637 8.4965657653227e-07
122.178002 7.88236185143414e-07
123.948345 7.77264510352082e-07
125.722288 5.24567966821284e-07
127.492972 8.15556912740251e-07
129.269886 4.10190884864017e-07
131.04052 3.998227686942e-07
132.818266 3.37382821853851e-07
134.595143 2.20182152875785e-07
136.372336 2.56019857896086e-07
138.144954 2.13744799145757e-07
139.922992 1.89054099772182e-07
141.700803 1.50996783898704e-07
143.493815 1.08610822932826e-07
145.267455 1.93046119097089e-07
147.039267 8.54909546731249e-08
148.812509 6.20635568224061e-08
150.58501 9.64422755594133e-08
152.358414 5.06031971994696e-08
154.128492 3.63405561603757e-08
155.901449 3.43523310803895e-08
157.676471 4.27157805676002e-08
159.450628 3.56478319962279e-08
161.218548 2.20518735424185e-08
162.995434 1.62958493375458e-08
164.78301 2.00414632844504e-08
166.559205 1.1533238982297e-08
168.335262 1.23431829081864e-08
170.120501 9.08620970284913e-09
};
\end{axis}

\end{tikzpicture}\hfill
%        \hfill% This file was created by matplotlib2tikz v0.6.3.
\begin{tikzpicture}

\definecolor{color1}{RGB}{000,125,122} % MPG green
\definecolor{color0}{HTML}{FF9933} %EI orange

\begin{axis}[
title={i=4},
xlabel={CPU time},
ylabel={relative residual},
xmin=-6.97981194999952, xmax=190.597060949999,
ymin=2.92480831875826e-09, ymax=2.75174713785757,
ymode=log,
width=\figwidth,
height=\figheight,
tick align=outside,
x grid style={lightgray!92.026143790849673!black},
y grid style={lightgray!92.026143790849673!black},
mystyle
]
\addplot [semithick, color0]
table {%
2.08149000000003 1
4.13531999999987 0.882677761201002
6.37457599999993 0.309670955099524
8.55053799999996 0.10576572804898
10.7342659999999 0.0841112047270652
12.926074 0.0502922147108168
14.9913689999998 0.0403269882840885
16.7303339999999 0.0315104234797059
18.4621359999999 0.027613822219445
20.1962099999998 0.0171940460380396
21.9304089999998 0.0125649054617688
23.663491 0.0342342385468219
25.3977889999999 0.0107315233431743
27.1389119999999 0.00761682477566453
28.8706129999998 0.00566665432208058
30.6026899999999 0.00489004668308005
32.3434729999999 0.00352785854055746
34.1080949999998 0.00282697958975048
35.8645879999999 0.00235946084946117
37.991327 0.00182507379066131
39.8995459999999 0.00208056048933845
42.072234 0.00211087048621502
44.238456 0.00124616187543843
46.4105319999999 0.000836458243197229
48.579371 0.000724884148310258
50.7505159999998 0.000694596454165217
52.765868 0.000615514181703636
54.9364949999999 0.000395194927668283
56.8066920000001 0.0004208377585385
58.5394619999997 0.000395598198124697
60.2739269999997 0.000918697799807139
62.0081179999997 0.000278425181621581
63.742479 0.000209302300691924
65.474416 0.000162037682858397
67.2064409999998 0.000172972758771172
68.9518189999999 0.000172795670286419
70.6841879999997 0.000106841712176908
72.4161119999999 7.17351870660708e-05
74.1502949999999 6.996355438425e-05
75.8832349999998 0.000169351233257836
77.617315 9.26325676648504e-05
79.3500119999999 4.12270510874369e-05
81.0825909999999 3.57924453080998e-05
82.8130229999997 2.7372889238935e-05
84.5437849999998 2.12147511537555e-05
86.2774019999997 1.87129615904653e-05
88.0127189999998 2.49156838541279e-05
90.1796469999999 1.26337037957049e-05
92.3498650000001 6.9573822985203e-05
94.5194069999998 1.26408242352784e-05
96.6914569999999 1.17676306081066e-05
98.8687799999998 7.76051210680125e-06
101.028274 7.03555586866665e-06
103.202754 7.05708037656616e-06
104.958726 4.01992387792646e-06
106.692716 3.25292429229784e-06
108.426434 2.89266193791756e-06
110.159869 7.84981550740253e-06
111.890914 4.47804441175053e-06
114.068127 1.97169134247244e-06
116.22765 2.07539211294119e-06
118.402825 1.50822500545013e-06
120.565492 1.045288234975e-06
122.734473 1.43537780006783e-06
124.895469 8.02656255925569e-07
127.069463 1.14550723615524e-06
129.248879 1.07220168230004e-06
131.132263 8.18439245667697e-07
132.864498 1.2441629989285e-06
134.596798 6.8774926389615e-07
136.330389 4.19806217553207e-07
138.064758 3.36708429312322e-07
139.797282 3.7560073000031e-07
141.531106 2.2196092391392e-07
143.264429 4.34596119915634e-07
144.997394 1.71274263888174e-07
146.7353 1.13821555298461e-06
148.467886 1.555080215204e-07
150.200178 1.39607463487244e-07
151.930695 8.99884722858898e-08
153.663096 6.19499067838696e-08
155.394396 9.31021016056566e-08
157.128018 6.01216944570034e-08
158.860684 6.53646779220535e-08
160.592455 3.69775889570879e-08
162.324834 9.62830557047104e-08
164.056958 5.18306051100901e-08
165.788531 2.90717947577393e-08
167.520769 4.54180198390428e-08
169.252462 3.15081233426165e-08
170.983379 1.83680779528913e-08
172.717866 1.25065458903252e-08
174.45095 1.26269733196054e-08
176.182566 1.10461286522775e-08
177.922029 2.85007190283326e-08
179.654748 1.75190956363716e-08
181.389076 8.70914992295231e-09
};
\addplot [semithick, color1]
table {%
2.00095500000043 1
8.27846800000043 0.13520718401907
10.0964920000006 0.0543183634497806
12.0688209999998 0.0329497808419938
14.120809 0.0234263668176792
16.2583140000006 0.0169864028726687
18.3156559999998 0.0139925486597589
20.348739 0.0108878574031763
22.4142120000006 0.00835649750440487
24.4431869999999 0.00599055969610091
26.3371930000003 0.00516434095737581
28.1130830000002 0.00398818462354353
29.894346 0.00334485994762383
31.6721440000001 0.00268269421467095
33.4554830000006 0.002213496586031
35.2275150000005 0.00217778682619692
36.9964380000001 0.00147603175232371
38.7695949999998 0.00140803123645132
40.5420389999999 0.00111901873216995
42.3134460000001 0.00100821084068454
44.087329 0.000789506322812134
45.8561239999999 0.000628416467606209
47.6456470000003 0.000496562012986069
49.4130100000002 0.000416613206417103
51.1850320000003 0.000326130903870491
53.1335230000004 0.000240681733847384
54.9570240000003 0.00019939158291641
56.727766 0.000153429129976108
58.4984830000003 0.000125459126038543
60.296593 0.000107667269272456
62.069375 8.08197106722744e-05
63.845053 6.05792561752281e-05
65.6163450000004 4.62158857679012e-05
67.3865660000001 5.00116746900274e-05
69.1619900000005 4.08836028273995e-05
70.9318540000004 2.86250313377046e-05
72.7030810000006 2.29849390035916e-05
74.4813530000001 2.38844045026314e-05
76.2534240000005 1.61717841866269e-05
78.0241020000003 1.28511216688041e-05
79.7973160000001 9.2082271165195e-06
81.5727210000005 7.46076994452548e-06
83.3471660000005 8.48595947701931e-06
85.1199710000001 5.3869116265434e-06
86.8950750000004 4.25051634464303e-06
88.6822410000004 3.16872208231232e-06
90.4937230000005 2.56724956173524e-06
92.2824540000001 2.92011637242875e-06
94.0861169999998 1.85091885574955e-06
95.9356550000002 1.34180340620055e-06
97.7183789999999 1.04021183408033e-06
99.6632369999998 8.88284719387982e-07
101.749606 9.00360980531013e-07
103.965722 5.72383741803389e-07
106.17561 4.33376277687863e-07
108.147132 3.65196509836731e-07
110.213169000001 3.49028192103689e-07
112.10048 4.06189863793683e-07
113.879466 2.16513536739336e-07
115.679675 1.62231314001954e-07
117.453654 1.34268313240392e-07
119.228269 1.28666785902168e-07
121.008362 8.43227511086363e-08
122.784615 6.65527680240578e-08
124.559468 7.61233551018288e-08
126.333494 6.61199555504632e-08
128.110029 4.29516124953038e-08
129.885623 2.72205754419616e-08
131.869524000001 2.05953997160241e-08
134.086115 1.63769786953586e-08
136.291331 1.55773465574139e-08
138.510107 1.19390572108686e-08
140.748014 9.864500766699e-09
};
\addplot [semithick, color1, dashed]
table {%
2.73948399999972 1
8.8782199999996 0.781636126362367
10.6809999999996 1.07577551309337
12.4629019999993 0.290142031287773
14.2355069999994 0.0768798731477282
16.0287099999996 0.0554512140434473
17.8031349999992 0.0495217974820493
19.577687 0.0386636541644567
21.3536029999996 0.0235964911962285
23.123576 0.0233990128549268
24.9000699999997 0.0132779374925089
26.6675259999993 0.00828333520971171
28.4433559999998 0.0056764147723767
30.2242059999999 0.00622155338327997
32.0380499999992 0.00279341650325736
34.2505179999998 0.00187593155556046
36.1099749999994 0.00143536717519645
37.8778919999995 0.00115192263420771
39.6555719999997 0.000976965072842846
41.4301889999997 0.000940015783294228
43.2103879999995 0.000724671952709949
44.9837539999999 0.00201505180241223
46.7582059999995 0.000622584638671632
48.526245 0.000551351639350393
50.3156689999996 0.00100525024246351
52.1176809999997 0.000407024435848557
54.1261249999998 0.000315677689501285
55.9668499999998 0.000639743836937692
58.1835369999999 0.000232559384982242
60.2172909999999 0.000218196731009302
62.4575459999996 0.000224125732569412
64.4003749999993 0.000324373199918111
66.2576049999998 0.000266196683051406
68.0506329999998 0.000120275722569669
70.0357909999993 0.000158345317017682
72.2702519999993 0.000115878309078541
74.228932 6.80683679590548e-05
76.1399189999993 6.69249408870075e-05
78.0356379999994 7.9237462019774e-05
79.9274079999996 4.67168440220282e-05
81.7397919999994 0.000163531735507004
83.5667469999999 3.29894672511147e-05
85.6485489999995 2.80357236677354e-05
87.5158409999995 5.35024763522814e-05
89.4998989999995 2.25529519491743e-05
91.4894809999996 1.6886577196872e-05
93.6173489999992 2.120768260081e-05
95.5088519999999 1.26463918358365e-05
97.3956269999999 1.27343879544528e-05
99.2125259999993 1.19420533226033e-05
101.092248 2.77641983393908e-05
103.242824999999 6.88151367223046e-06
105.442673 5.02878700239033e-06
107.494978999999 8.43649310165136e-06
109.475133999999 4.13372530353898e-06
111.530599 2.62088183625268e-06
113.56513 2.05614321201066e-06
115.563568 1.87795336988656e-06
117.619083 3.9117201080848e-06
119.660035 3.08069179317807e-06
121.693939 1.28061358656914e-06
123.660425 1.56908576267244e-06
125.855629 1.09387433143757e-06
127.904211999999 9.31865550308919e-07
130.085195 7.1868537620077e-07
132.016098 5.8261400788461e-07
134.020971999999 7.62954061158586e-07
136.103058 2.49800507929533e-06
137.876528999999 3.95502198729703e-07
139.647868 8.72191401267838e-07
141.420174999999 3.2771420695774e-07
143.344617 3.02690560829111e-07
145.285553 1.83554572651635e-07
147.174919 1.41489088324079e-07
148.946330999999 2.44938473582928e-07
151.005972999999 1.14656572805542e-07
152.928142999999 6.60624791770696e-07
154.711468 1.01164787774868e-07
156.492418999999 8.1485019498347e-08
158.286569 5.2452343684646e-08
160.062843 1.29258292352261e-07
161.841012999999 6.88469089978307e-08
163.609759 3.85995846655309e-08
165.391212 2.58359820844745e-08
167.442013 3.07956225976224e-08
169.64327 6.09801634707231e-08
171.857362999999 2.40278809304563e-08
173.765057 1.35245308504898e-08
175.617424 3.17004700866569e-08
177.69119 1.17687185675387e-08
179.603343 1.18364153835755e-08
181.616293999999 7.48142416514224e-09
};
\end{axis}

\end{tikzpicture}\hfill%
%        % This file was created by matplotlib2tikz v0.6.3.
\begin{tikzpicture}

\definecolor{color1}{RGB}{000,125,122} % MPG green
\definecolor{color0}{HTML}{FF9933} %EI orange
\begin{axis}[
title={i=7},
xlabel={CPU time},
ylabel={relative residual},
xmin=-5.94763169999981, xmax=164.7538817,
ymin=3.49157349330361e-09, ymax=2.52762869328516,
ymode=log,
width=\figwidth,
height=\figheight,
tick align=outside,
x grid style={lightgray!92.026143790849673!black},
y grid style={lightgray!92.026143790849673!black},
mystyle
]
\addplot [semithick, color0]
table {%
1.81152800000018 1
3.601044 0.978578487844548
5.34007599999995 0.452630563012444
7.06861299999991 0.274506903191449
8.79665199999999 0.230636837425874
10.5220439999998 0.205634860470716
12.2499250000001 0.194905875311664
13.9814630000001 0.125777342856689
15.709265 0.110853460026038
17.4365469999998 0.0751821745110982
19.16552 0.0523902427312033
20.8955000000001 0.0534836584554684
22.624104 0.0468410307120514
24.348653 0.0250704821030212
26.0748429999999 0.0179723169546249
27.801688 0.0126781550712591
29.543365 0.0103906532419307
31.2806139999998 0.00647661407922633
33.4583429999998 0.00515565497167974
35.6333759999998 0.00419943050401319
37.7937550000001 0.00528616058830989
39.9565280000002 0.00356924231719373
42.126174 0.0021050195767773
44.292833 0.00168785945796043
46.4676469999999 0.00117694991622965
48.6465480000002 0.00114137418621747
50.8197799999998 0.0012595414942975
52.9900250000001 0.000701007980341736
55.1623770000001 0.000510373561962083
57.32672 0.000908121008188929
59.5017290000001 0.00178312939009155
61.672059 0.000332659735507866
63.84159 0.000242016394424124
66.0148939999999 0.000199956950225066
68.1872239999998 0.000169630048693226
70.362924 0.0002105572371367
72.538168 0.000122838759098158
74.461886 7.79197040421013e-05
76.1983540000001 8.79692660001444e-05
77.9308740000001 0.000132502315850497
79.66165 7.25339475016501e-05
81.3920320000002 5.43068058497639e-05
83.1231200000002 3.14615270801098e-05
84.85268 2.70563742576924e-05
86.583556 1.66898627284937e-05
88.3129140000001 1.64722788963308e-05
90.045517 1.64316168148198e-05
92.2188040000001 8.50499946109101e-06
94.3840070000001 2.57065762104438e-05
96.5574360000001 8.72205463337251e-06
98.7298930000002 8.16807893516338e-06
100.901546 8.4258445422712e-06
103.071184 4.29879752529914e-06
105.231852 2.93090947467014e-06
107.207817 2.34502604490608e-06
109.372063 1.80445393824927e-06
111.553897 1.99123899973758e-06
113.732366 2.22919136150953e-06
115.903606 1.28194182769298e-06
118.067829 7.80871393310539e-07
120.172228 1.30118071933961e-06
122.324747 7.84109815477666e-07
124.475945 5.55260509393333e-07
126.63238 3.55224195546022e-07
128.564083 2.8117081372453e-07
130.408933 3.99012994904481e-07
132.543132 3.85357106086966e-07
134.421477 2.23205152649638e-07
136.152729 3.30299610360842e-07
137.881473 1.22597770804339e-07
139.606745 9.99988249960541e-08
141.338345 1.04661614490551e-07
143.067555 1.42071353790736e-07
144.829597 6.11181144876138e-08
146.557374 4.17217084338014e-08
148.283493 4.52974888977384e-08
150.012826 3.53898800085139e-08
151.741802 3.82022082772067e-08
153.535893 1.9922256886248e-08
155.264523 1.39429825929878e-08
156.994722 9.38682032132173e-09
};
\addplot [semithick, color1]
table {%
2.07214199999999 1
8.55915099999947 0.415138940640922
10.5761699999994 0.201786927513696
12.6736279999996 0.153465543069231
14.7755549999993 0.138416196785426
16.9144299999998 0.0948650869994917
19.0344049999994 0.0588001068933884
20.8983199999993 0.0409193790337813
22.6889009999995 0.0270938656936288
24.4658599999993 0.0184627650968213
26.245328 0.0141737862276452
28.0214129999995 0.00970778588435321
29.7994279999994 0.00804460483706191
31.5776689999993 0.00654858093199894
33.3579119999995 0.00535640605312761
35.1558919999998 0.00389457406828932
37.0974619999997 0.00305031768163603
39.2801679999993 0.00272901347237773
41.4867629999999 0.00192725340983415
43.4694659999996 0.00133868271419171
45.5193929999996 0.00122108529963149
47.5560109999997 0.000883020505791603
49.4359969999996 0.000624586217139113
51.2172399999999 0.000481627268158425
52.9960719999999 0.000357234466519351
54.7761819999996 0.000256912520271691
56.5549299999993 0.000204857796809637
58.3444989999998 0.000138094002682542
60.1232249999994 0.000100140907137282
61.9062969999995 7.58271608846032e-05
63.6877849999992 5.99594594544956e-05
65.465056 4.18097269430846e-05
67.2438359999996 5.41397883831579e-05
69.0186359999998 3.04663280621217e-05
70.8001759999997 2.13390731639165e-05
72.5787779999991 1.39255969181318e-05
74.3574189999999 9.92869882951228e-06
76.1368409999995 7.23231268488847e-06
77.9204649999992 5.18832537139365e-06
79.7532939999992 5.91038234848873e-06
81.5954509999992 3.92958359441134e-06
83.3936009999998 2.5189604902865e-06
85.1765789999999 1.70733758399763e-06
86.9602269999996 1.31287972177135e-06
89.0891789999996 1.1205546024035e-06
91.2213769999998 1.038567921471e-06
93.0362669999995 6.56269946016138e-07
95.1927259999993 4.55284074944354e-07
97.3980369999999 3.03739644204172e-07
99.4129229999999 2.23766353002694e-07
101.313542 2.32691913907532e-07
103.104622 1.58045659834813e-07
104.884507999999 1.03401052527595e-07
106.661032999999 1.08555934737461e-07
108.440603 6.72858130184609e-08
110.219061 8.41847317733126e-08
112.004588 4.19631870484859e-08
113.968237 4.32045163999037e-08
116.044177 3.03694836805597e-08
118.27164 2.01252993604734e-08
120.297704 1.77806790591312e-08
122.284682 1.48806038779074e-08
124.111335 8.8254013463881e-09
};
\addplot [semithick, color1, dashed]
table {%
2.16303499999958 1
9.35316799999964 0.522833721108446
11.1926560000002 0.467842972487433
12.9671349999999 0.241062744580488
14.7229230000003 0.155743391992435
16.5449989999997 0.149979868244292
18.7228829999995 0.100458619152368
20.9009779999997 0.0579872585575005
22.7720579999996 0.0326069256045739
24.5478940000003 0.0196859818839339
26.3104050000002 0.00543356004372351
28.0728719999997 0.0130887173872133
29.8359460000001 0.0043838248024854
31.5986309999998 0.00472590180018089
33.3747329999997 0.00332415212077732
35.1619479999999 0.00203214699469315
36.9434329999995 0.00168191183243539
38.7025999999996 0.00158743099623005
40.4734589999998 0.000996650799027943
42.2324600000002 0.000729991487837843
43.9958139999999 0.000730371847008524
45.7651449999994 0.00203753646651434
47.5242829999997 0.000588400967431125
49.2884109999995 0.000381508945300172
51.0481289999998 0.000307885553512091
52.8086069999999 0.000243057472683168
54.5667349999994 0.000188877945371645
56.566221 0.000139686397881767
58.7473789999995 9.87775012389346e-05
60.6393619999999 8.69473050430732e-05
62.3966249999994 0.000157416708790087
64.1543469999997 5.59385199529513e-05
65.9142529999999 7.74229168078034e-05
67.6714519999996 3.48403134190048e-05
69.4277940000002 2.61354152868308e-05
71.1904180000001 1.6089862152166e-05
72.9469959999997 1.48949967048997e-05
74.7148049999996 1.06867419061266e-05
76.4731609999999 1.58265478363833e-05
78.2335519999997 1.82334370900943e-05
79.9958820000002 7.44392073849188e-06
81.7532329999995 5.27636213061906e-06
83.5110519999998 3.64029867967868e-06
85.2722400000002 4.93423573490855e-06
87.0327219999999 2.33051272912497e-06
88.7893830000003 1.76779422404196e-06
90.5470909999995 1.3259177467133e-06
92.3077599999997 2.10102847736115e-06
94.065055 1.39142261357594e-06
95.8210419999996 1.02512283856587e-06
97.5782689999996 6.11574481533737e-07
99.3400599999995 4.0911314221344e-07
101.098618 4.52218815307971e-07
102.858956 2.85012080762739e-07
104.615028 2.17934406790088e-07
106.368304 1.3477897251985e-07
108.13564 5.86547006642038e-07
109.891832 1.70836263394498e-07
111.659524 1.05034398731061e-07
113.42004 6.48394869787313e-08
115.177361 6.50865670994825e-08
116.939531 4.5342392887266e-08
118.698748 2.89173092741641e-08
120.457378 2.0581673664194e-08
122.218691 1.49609183300853e-08
123.978588 5.24624602108916e-08
125.736064 1.5987063494173e-08
127.495242999999 8.8772544067145e-09
};
\end{axis}

\end{tikzpicture}\hfill \null
%    \caption{{\color{red} Align pictures, i+=1} The computational resources required to solve successive systems to a relative %residual of $\epsilon=10^{-8}$. The yellow and green curves show the convergence of standard CG and def-CG(8,12). The dashed green is the convergence of def-CG(8,12) with only one step recurrence. Only approximate eigenvectors from the preceeding iteration are saved in $W$. The benefit of maintaining a set of approximate eigenvectors that are refined in each iteration of systems is thus made clear. In iteration $i=1$ are the same vectors in $W$ used for the 1-step recurrence and the normal def-CG and the curves are therefore almost indiscernible. Def-CG shows a steeper convergence compared to standard CG and the computaional gain should be more obvious for higher precision.}\label{fig:suc_cpu}
%\end{figure}
%\setlength{\figwidth}{.9\textwidth}



\setlength{\figwidth}{.99\textwidth}
\setlength{\figheight}{0.41803398875\figwidth}
%\setlength{\figheight}{0.61803398875\figwidth}

\begin{figure}[b]
    \centering \scriptsize
        % This file was created by matplotlib2tikz v0.6.3.
\begin{tikzpicture}

\definecolor{color1}{RGB}{000,125,122} % MPG green
\definecolor{color0}{HTML}{FF9933} %EI orange

\begin{axis}[
xmin=-100.1212309, xmax=2151.4841009,
ymin=2.53430998098285e-09, ymax=4.2233316248538,
ymode=log,
width=\figwidth,
height=\figheight,
tick align=outside,
xlabel={CPU time [$\mathrm{s}$]},
ylabel={relative residual},
x grid style={lightgray!92.026143790849673!black},
y grid style={lightgray!92.026143790849673!black},
extra y ticks=1e-8,
extra y tick style={grid=major, grid style={dashed,black}},
legend style={draw=white!80.0!black},
legend entries={{CG},{def-CG(8,12)}},
legend cell align={left},
mystyle
]
\addplot [semithick, color0,]
table {%
2.338201 1
4.44607500000001 0.803355123226035
6.645453 0.189723157739223
8.847655 0.0509960204551763
10.740944 0.0389494404201885
12.681271 0.0179898688439675
14.469114 0.0157165818767063
16.244612 0.0109148932485257
18.092409 0.00744639345220638
19.873594 0.00565091648891621
21.661486 0.0045672164315901
23.440174 0.00655416754453231
25.218376 0.00441835200120656
27.00421 0.00286988796812528
28.777418 0.00229073118286192
30.554253 0.00184696207454626
32.335071 0.00170948531149521
34.115036 0.00135532406667619
35.917596 0.0011314129954786
37.717639 0.000995099708333701
39.509563 0.00124467117208623
41.308675 0.00127919261125498
43.100188 0.000745617290911589
44.877516 0.000645207200387889
46.65213 0.00125335739014907
48.495569 0.000566671088344948
50.35954 0.000465779260059283
52.157285 0.000472832773089773
53.98484 0.000706055999276449
56.200514 0.00039557438505856
58.268514 0.00218584404908133
60.050247 0.000329904898579162
61.831192 0.000302504584667049
63.608298 0.000282349949251237
65.390862 0.00027375733823991
67.162513 0.000345627901558582
68.938952 0.000234773985971648
70.717576 0.000218933309531953
72.495541 0.000264446005703375
74.268853 0.00102511903397826
76.052934 0.000177183810119042
77.905004 0.000164585352865341
79.68121 0.000156758732823845
81.463985 0.000271408914110484
83.244103 0.000137873326241013
85.020203 0.000209610075810377
86.794582 0.000131164238986208
88.571451 0.000120571009634111
90.35171 0.000472924440607135
92.134302 0.000104111867724803
93.91936 0.000215831811934717
95.689264 0.000108344843979051
97.467556 0.000145406337107936
99.249585 9.83319255693291e-05
101.029253 7.9487119502651e-05
102.804547 6.9054667817296e-05
104.582844 9.89496263218335e-05
106.362049 8.76192149021089e-05
108.198216 6.96416350425278e-05
109.979402 0.000115284771851654
111.758071 8.63971853518058e-05
113.538694 9.66753862630729e-05
115.325372 7.83450597254574e-05
117.11079 5.31825919489748e-05
118.889627 6.20404156580138e-05
120.667321 5.49629834279613e-05
122.442378 0.00030823276302295
124.257998 6.91611168839292e-05
126.088834 4.64409689899364e-05
127.964068 4.1347240542324e-05
129.926714 5.84786482145998e-05
131.717248 3.6371742320493e-05
133.514839 7.81016437735521e-05
135.295211 3.19694953292453e-05
137.10018 4.8206838178297e-05
139.311255 0.000131282086413705
141.456877 2.95912943760967e-05
143.520072 2.62136224738801e-05
145.565082 4.96911563905546e-05
147.893901 4.47914934213392e-05
150.103769 4.24372369989435e-05
152.038478 2.89649614903263e-05
154.135428 2.06097244640404e-05
155.917417 2.04425661632688e-05
157.697338 7.33055032296042e-05
159.47799 5.57994015255639e-05
161.270785 1.89621655930058e-05
163.050111 2.32149615651989e-05
164.832774 1.917168942288e-05
166.639812 2.38945127824719e-05
168.57736 1.85833834033099e-05
170.782275 1.47446788164736e-05
172.70998 3.07085402215687e-05
174.677387 1.83196245982353e-05
176.860804 6.27528159346378e-05
178.868486 1.60151633576816e-05
180.994855 2.76801388864942e-05
183.112804 1.20206223901352e-05
185.320823 3.31946439301624e-05
187.547378 1.10926071890035e-05
189.775237 9.4849723170377e-06
191.980904 8.66702062223968e-06
194.22611 1.33812332422693e-05
196.410421 4.20534792627039e-05
198.380116 9.87248514735162e-06
200.404731 1.07072202690904e-05
202.424446 7.04485740535497e-06
204.383753 1.81801055858743e-05
206.178944 1.2318902778476e-05
208.009403 6.83609948340216e-06
209.887263 9.7815196887844e-06
212.098816 1.26733265782489e-05
214.3046 1.74476735041725e-05
216.339067 6.27326363064533e-06
218.475676 1.08830719167175e-05
220.561385 6.01750198590015e-06
222.741816 1.94412969696978e-05
224.83513 4.58107886856281e-06
226.64807 7.66490811369606e-06
228.488537 4.91966509463524e-06
230.270216 5.65438808258883e-06
232.227768 1.11606189759778e-05
234.325583 3.9513132800811e-06
236.453178 7.98742878527026e-06
238.653277 4.60243296866998e-06
240.857861 5.30994424478289e-06
242.966102 4.2631298318275e-06
244.999716 4.17060567204133e-06
247.040269 4.9521921937178e-06
249.075768 2.81434243472579e-06
251.138078 2.53381900440129e-05
253.353627 3.62877818960533e-06
255.519492 5.79065437344612e-06
257.673829 3.81501327973428e-06
259.498022 4.41125423694519e-06
261.281506 2.98596763496664e-06
263.231347 2.88701774135704e-06
265.30782 6.40870524569588e-06
267.258152 2.3020460420503e-06
269.184366 2.9617821781976e-06
271.585356 1.12170725426271e-05
273.824399 1.92744750815247e-06
276.09324 1.85123981537782e-06
278.388852 3.1652393314562e-06
280.514078 2.34480432564453e-06
282.609874 1.50771212984554e-06
284.606495 1.46728969469757e-06
286.459528 2.5387782910075e-06
288.331303 9.20990293955467e-06
290.361776 1.69090060045012e-06
292.472552 3.28713504324724e-06
294.519925 1.54867660786071e-06
296.552238 2.20460788169525e-06
298.59819 1.28858013741564e-06
300.758958 2.28949956912794e-06
302.977901 1.26440924020322e-06
305.047106 1.53250668648788e-06
307.248305 2.02469489887452e-06
309.277452 5.82783252743368e-06
311.283774 2.30716484539411e-06
313.498701 8.95382050802481e-07
315.445712 1.19027406565537e-06
317.580155 9.50379943558741e-07
319.555588 1.09011919340949e-06
321.419928 6.9483841758977e-07
323.192191 1.16614183972475e-06
324.971577 7.90381684433361e-07
327.041162 3.08178994890944e-06
329.060966 1.03760383795321e-06
330.877475 1.5216072227517e-06
332.664154 1.00821898282087e-06
334.540456 5.68559994423168e-07
336.558855 5.05917187612717e-07
338.481377 5.96003687564043e-07
340.664056 5.86768563194272e-07
342.655193 7.07241889247326e-07
344.515758 2.54940696385307e-06
346.717585 1.16871035189459e-06
348.858481 7.96416102535291e-07
350.798598 4.86781827240055e-07
352.815058 7.32731090048667e-07
354.859185 7.86865457870226e-07
356.910023 3.84759482407189e-07
358.967741 7.09138343674972e-07
360.976073 3.19060013913001e-07
363.195237 6.22596622166789e-07
365.405211 4.87610919073146e-07
367.348338 5.99377354796012e-07
369.322147 4.07436862774794e-07
371.497061 3.4125699307762e-07
373.543511 3.72597211615982e-07
375.441781 3.53086391548906e-07
377.464704 2.71439627694361e-07
379.472276 3.00772598839259e-07
381.659171 6.60802967590278e-07
383.675025 2.22023788220869e-07
385.810649 2.60646008558975e-07
388.00976 2.8467347221045e-07
390.000912 2.06328828309457e-07
391.889738 4.04562759325244e-07
393.691625 1.72173943351701e-07
395.773745 2.9980832344414e-07
397.983738 2.08984014081915e-07
400.075862 4.81362687324553e-07
401.968513 3.19293565350821e-07
403.818335 2.61291157832008e-07
405.91258 2.91008075524331e-07
407.882786 1.26840547569393e-07
409.939741 1.57541858549776e-07
411.772569 2.13413363664401e-07
413.686072 1.30821150905034e-07
415.705307 1.4440711653756e-07
417.70636 2.71571428450793e-07
419.835866 1.7083290185777e-07
421.769995 1.69734922414137e-07
423.824447 1.1361833431033e-07
425.704792 1.17728606387229e-07
427.490536 1.33116961703577e-07
429.417078 1.05139352450153e-07
431.325926 1.39259352406075e-07
433.464613 1.54548931788732e-07
435.51068 1.46477016087427e-07
437.398136 2.09946784222416e-07
439.317214 1.505516081264e-07
441.25917 8.29329684320074e-08
443.23825 9.47698040005921e-08
445.345181 9.34863948626903e-08
447.219825 6.59789709325083e-08
449.021049 1.74877392556874e-07
450.809176 7.35633748656762e-08
452.590264 7.06333602257365e-08
454.370597 4.1141633194515e-07
456.151862 7.38636058797e-08
458.170466 5.538996909896e-08
460.388864 6.04532009544436e-08
462.336406 8.66246006460668e-08
464.304225 4.11137870206209e-08
466.087653 6.12876010497824e-08
467.892891 6.44424248653451e-08
469.710288 8.93149242336997e-08
471.506856 9.8085964971925e-08
473.686097 5.12127589946733e-08
475.851004 9.35977469492779e-08
478.066035 7.49654333598934e-08
480.270371 5.23990680640004e-08
482.234778 5.17801527848953e-08
484.164215 4.25592987973186e-08
486.123476 5.10230952115914e-08
488.253673 3.02851599610119e-08
490.238791 1.97618347654298e-07
492.06488 2.5607939622142e-08
493.843298 5.08623643285511e-08
495.622514 3.22037227176915e-08
497.412571 5.51916478545168e-08
499.244174 3.01661628436876e-08
501.047335 3.0893641001417e-08
502.824304 2.96390250672025e-08
504.603488 3.52550575355969e-08
506.386801 3.75916213525004e-08
508.173135 2.66444763533281e-08
509.951188 1.63352975273386e-08
511.737682 2.35445950198274e-08
513.515971 4.20728014346737e-08
515.291317 2.67069250777757e-08
517.068324 1.45463061165461e-08
518.851024 1.26473636318996e-08
520.627858 4.71535035106676e-08
522.403976 2.40417356677701e-08
524.202576 1.86162202249697e-08
525.994692 2.24707199894313e-08
527.789523 1.25236620686235e-08
529.750549 1.61462618135815e-08
531.684237 2.01524212212717e-08
533.46938 1.1020618434418e-08
535.243347 3.79774087073663e-08
537.020355 1.71280102719761e-08
538.796092 3.22786474090885e-08
540.578704 1.07235377431837e-08
542.363263 1.94078266109902e-08
544.142314 1.27330531704612e-08
545.924869 8.46042523209604e-09
};
\addplot [semithick, color1]
table {%
2.22446599999967 1
4.19218799999999 0.803355123226035
5.98680299999978 0.189723157739223
7.76439500000015 0.0509960204551763
9.54846399999951 0.0389494404201885
11.3233769999997 0.0179898688439675
13.0934209999996 0.0157165818767063
14.8648519999997 0.0109148932485257
16.7672469999998 0.00744639345220638
18.9040519999999 0.00565091648891621
20.7229090000001 0.0045672164315901
22.5221819999997 0.00655416754453231
24.2965999999997 0.00441835200120656
26.0706829999999 0.00286988796812528
27.8376129999997 0.00229073118286192
29.6076379999995 0.00184696207454626
31.3784839999998 0.00170948531149521
33.1586429999998 0.00135532406667619
34.931979 0.0011314129954786
36.7055209999999 0.000995099708333701
38.4824440000002 0.00124467117208623
40.3009350000002 0.00127919261125498
42.4953749999995 0.000745617290911589
44.7041339999996 0.000645207200387889
46.7680979999996 0.00125335739014907
48.8930559999999 0.000566671088344948
51.0705399999997 0.000465779260059283
53.0762919999997 0.000472832773089773
55.2792819999995 0.000706055999276449
57.20543 0.00039557438505856
59.2867630000001 0.00218584404908133
61.4789639999999 0.000329904898579162
63.5860720000001 0.000302504584667049
65.3556900000003 0.000282349949251237
67.1415850000003 0.00027375733823991
68.928664 0.000345627901558582
70.7004710000001 0.000234773985971648
72.4768880000001 0.000218933309531953
74.2511139999997 0.000264446005703375
76.0227669999995 0.00102511903397826
77.7931280000003 0.000177183810119042
79.5560679999999 0.000164585352865341
81.3223429999998 0.000156758732823845
83.095577 0.000271408914110484
84.861868 0.000137873326241013
86.6282590000001 0.000209610075810377
88.5654709999999 0.000131164238986208
90.4364690000002 0.000120571009634111
92.2144859999999 0.000472924440607135
93.9810669999997 0.000104111867724803
95.7490749999997 0.000215831811934717
97.5169310000001 0.000108344843979051
99.2862029999997 0.000145406337107936
101.071564 9.83319255693291e-05
102.842027 7.9487119502651e-05
104.612332 6.9054667817296e-05
106.384138 9.89496263218335e-05
108.153292 8.76192149021089e-05
109.924455 6.96416350425278e-05
111.694002 0.000115284771851654
113.476625 8.63971853518058e-05
115.250521 9.66753862630729e-05
117.018706 7.83450597254574e-05
118.788136 5.31825919489748e-05
120.560193 6.20404156580138e-05
122.336273 5.49629834279613e-05
124.111134 0.00030823276302295
125.880635 6.91611168839292e-05
127.653796 4.64409689899364e-05
129.440205 4.1347240542324e-05
131.214586999999 5.84786482145998e-05
132.986638 3.6371742320493e-05
135.136946 7.81016437735521e-05
137.163728 3.19694953292453e-05
139.142878 4.8206838178297e-05
141.336893 0.000131282086413705
143.541996 2.95912943760967e-05
145.735251 2.62136224738801e-05
147.542289 4.96911563905546e-05
149.310673 4.47914934213392e-05
151.083446 4.24372369989435e-05
152.860701 2.89649614903263e-05
154.62869 2.06097244640404e-05
156.403672 2.04425661632688e-05
158.166777 7.33055032296042e-05
159.940034 5.57994015255639e-05
161.712177 1.89621655930058e-05
163.492911 2.32149615651989e-05
165.269185 1.917168942288e-05
167.042007 2.38945127824719e-05
168.810693 1.85833834033099e-05
170.577724 1.47446788164736e-05
172.345684 3.07085402215687e-05
174.119295 1.83196245982353e-05
175.894581 6.27528159346378e-05
177.666609 1.60151633576816e-05
179.439276 2.76801388864942e-05
181.208162 1.20206223901352e-05
182.989824 3.31946439301624e-05
184.761413 1.10926071890035e-05
186.547602 9.4849723170377e-06
188.333492 8.66702062223968e-06
190.113738 1.33812332422693e-05
191.900723 4.20534792627039e-05
193.688798 9.87248514735162e-06
195.474046 1.07072202690904e-05
197.282048999999 7.04485740535497e-06
199.065946 1.81801055858743e-05
200.878821 1.2318902778476e-05
202.658414 6.83609948340216e-06
204.439488 9.7815196887844e-06
206.221798 1.26733265782489e-05
208.000516999999 1.74476735041725e-05
209.776125 6.27326363064533e-06
211.550275 1.08830719167175e-05
213.318036 6.01750198590015e-06
215.097348 1.94412969696978e-05
216.871671 4.58107886856281e-06
218.645511 7.66490811369606e-06
220.415583 4.91966509463524e-06
222.187818 5.65438808258883e-06
223.968736 1.11606189759778e-05
225.740724 3.9513132800811e-06
227.512031 7.98742878527026e-06
229.294886 4.60243296866998e-06
231.064859999999 5.30994424478289e-06
232.843573 4.2631298318275e-06
234.611613999999 4.17060567204133e-06
236.37876 4.9521921937178e-06
238.153952 2.81434243472579e-06
239.936802 2.53381900440129e-05
241.710253 3.62877818960533e-06
243.496681 5.79065437344612e-06
245.27204 3.81501327973428e-06
247.041607 4.41125423694519e-06
248.811899 2.98596763496664e-06
250.586556 2.88701774135704e-06
252.358173 6.40870524569588e-06
254.133049 2.3020460420503e-06
255.907339 2.9617821781976e-06
257.679241 1.12170725426271e-05
259.449399 1.92744750815247e-06
261.218518 1.85123981537782e-06
262.98727 3.1652393314562e-06
264.758334 2.34480432564453e-06
266.526025 1.50771212984554e-06
268.292112 1.46728969469757e-06
270.056889 2.5387782910075e-06
271.820917 9.20990293955467e-06
273.591663 1.69090060045012e-06
275.380991 3.28713504324724e-06
277.148391 1.54867660786071e-06
278.918891 2.20460788169525e-06
280.688185 1.28858013741564e-06
282.452136 2.28949956912794e-06
284.220461 1.26440924020322e-06
285.992494 1.53250668648788e-06
287.759636 2.02469489887452e-06
289.532626 5.82783252743368e-06
291.299367 2.30716484539411e-06
293.163358 8.95382050802481e-07
294.933479 1.19027406565537e-06
296.699176 9.50379943558741e-07
298.474821 1.09011919340949e-06
300.243727 6.9483841758977e-07
302.013184 1.16614183972475e-06
303.78463 7.90381684433361e-07
305.54939 3.08178994890944e-06
307.319703 1.03760383795321e-06
309.080318 1.5216072227517e-06
310.84569 1.00821898282087e-06
312.615683 5.68559994423168e-07
314.38778 5.05917187612717e-07
316.160316 5.96003687564043e-07
317.932616 5.86768563194272e-07
319.705391 7.07241889247326e-07
321.477002 2.54940696385307e-06
323.249755 1.16871035189459e-06
325.022163 7.96416102535291e-07
326.822678 4.86781827240055e-07
328.845966 7.32731090048667e-07
330.864743 7.86865457870226e-07
332.903497 3.84759482407189e-07
334.955528 7.09138343674972e-07
337.161867 3.19060013913001e-07
339.215036 6.22596622166789e-07
341.152724 4.87610919073146e-07
343.005508 5.99377354796012e-07
345.100635 4.07436862774794e-07
347.013422 3.4125699307762e-07
348.781882 3.72597211615982e-07
350.562654 3.53086391548906e-07
352.368244 2.71439627694361e-07
354.146013 3.00772598839259e-07
355.915945 6.60802967590278e-07
357.684158 2.22023788220869e-07
359.447429 2.60646008558975e-07
361.216865 2.8467347221045e-07
363.019247 2.06328828309457e-07
364.848731 4.04562759325244e-07
366.621995 1.72173943351701e-07
368.391514 2.9980832344414e-07
370.161582 2.08984014081915e-07
371.932469 4.81362687324553e-07
373.711446 3.19293565350821e-07
375.480089 2.61291157832008e-07
377.249441 2.91008075524331e-07
379.015735 1.26840547569393e-07
380.789417 1.57541858549776e-07
382.561863 2.13413363664401e-07
384.333979 1.30821150905034e-07
386.10125 1.4440711653756e-07
387.940037 2.71571428450793e-07
390.010831 1.7083290185777e-07
391.79038 1.69734922414137e-07
393.559244999999 1.1361833431033e-07
395.32804 1.17728606387229e-07
397.099475 1.33116961703577e-07
398.877590999999 1.05139352450153e-07
400.65009 1.39259352406075e-07
402.422142 1.54548931788732e-07
404.19011 1.46477016087427e-07
405.957033 2.09946784222416e-07
407.720349 1.505516081264e-07
409.492905999999 8.29329684320074e-08
411.267612 9.47698040005921e-08
413.043241 9.34863948626903e-08
414.814588 6.59789709325083e-08
416.593885 1.74877392556874e-07
418.366715 7.35633748656762e-08
420.13807 7.06333602257365e-08
421.907922 4.1141633194515e-07
423.680072 7.38636058797e-08
425.443371 5.538996909896e-08
427.215354 6.04532009544436e-08
428.989978 8.66246006460668e-08
431.121993 4.11137870206209e-08
433.074396 6.12876010497824e-08
435.046712 6.44424248653451e-08
436.896206 8.93149242336997e-08
438.671736 9.8085964971925e-08
440.437752 5.12127589946733e-08
442.210488 9.35977469492779e-08
443.989649 7.49654333598934e-08
445.762587 5.23990680640004e-08
447.584348 5.17801527848953e-08
449.604417 4.25592987973186e-08
451.375092 5.10230952115914e-08
453.140997 3.02851599610119e-08
454.915563 1.97618347654298e-07
456.683235 2.5607939622142e-08
458.449288 5.08623643285511e-08
460.222291999999 3.22037227176915e-08
461.989305999999 5.51916478545168e-08
463.757513 3.01661628436876e-08
465.524748 3.0893641001417e-08
467.294398 2.96390250672025e-08
469.064138 3.52550575355969e-08
470.838889 3.75916213525004e-08
472.616002 2.66444763533281e-08
474.390451 1.63352975273386e-08
476.162384 2.35445950198274e-08
477.929805 4.20728014346737e-08
479.699537 2.67069250777757e-08
481.470825 1.45463061165461e-08
483.24925 1.26473636318996e-08
485.02701 4.71535035106676e-08
486.796453 2.40417356677701e-08
488.565196 1.86162202249697e-08
490.335886 2.24707199894313e-08
492.109134 1.25236620686235e-08
493.881522 1.61462618135815e-08
495.653009 2.01524212212717e-08
497.422271 1.1020618434418e-08
499.190762 3.79774087073663e-08
500.965683 1.71280102719761e-08
502.73518 3.22786474090885e-08
504.513565 1.07235377431837e-08
506.28157 1.94078266109902e-08
508.055238 1.27330531704612e-08
509.823391 8.46042523209604e-09
};
\addplot [semithick, color0]
table {%
548.47659 1
550.604753 0.804575147171834
552.563441 0.194574803468408
554.659528 0.0524702710386406
556.697261 0.0403274484736291
558.892463 0.0187561729091768
561.091696 0.0165370912766679
563.164666 0.0115256957522894
565.15932 0.00795507095065917
567.169867 0.00603031544117425
569.29098 0.00481715637993339
571.327516 0.00394061134811415
573.454215 0.014360747699105
575.576079 0.00306217856628012
577.746553 0.0024477155993977
579.924173 0.00194251536143408
582.109603 0.00180798618746539
584.158484 0.00141692253262101
586.344373 0.00118349559957808
588.527284 0.00104840707038964
590.710789 0.000955152306238132
592.885008 0.00208689287483618
595.053558 0.000825049650553405
597.265646 0.0006589620211942
599.308357 0.00130359203122416
601.082953 0.000595655378265468
603.247191 0.000474008293798442
605.250841 0.000480906427342067
607.261268 0.00079653823535531
609.430221 0.000382003569928178
611.587421 0.000329864833929373
613.659875 0.00176894964924077
615.67224 0.000299190516312295
617.795844 0.000267501004799817
619.806948 0.000418143275401713
621.850822 0.000237857500753838
624.061447 0.000211955878682798
625.93069 0.000198898988204841
627.890499 0.000242371597449379
629.930084 0.000162484581424941
631.867678 0.000726734086528887
634.042072 0.000144360744603273
636.221873 0.000124849982940771
638.4118 0.000219541672526653
640.567673 0.000137929860874211
642.497476 0.000184011099998317
644.71775 0.000104456445270027
646.911135 9.42339492780038e-05
649.094938 8.82555076188387e-05
651.317188 0.000332543818754805
653.505771 0.000175431225171645
655.705655 7.94487008276935e-05
657.879182 7.97250665375181e-05
660.054973 9.58632008036806e-05
662.274372 5.74586868542394e-05
664.333927 4.86931230699351e-05
666.545133 4.60342963871433e-05
668.581046 8.24021236889478e-05
670.431855 9.02404516530702e-05
672.198337 5.29976876838894e-05
674.122718 5.54407600850154e-05
676.308048 4.16616369077177e-05
678.176432 7.52268505124214e-05
680.021383 3.58205820419052e-05
681.979756 3.11910350573727e-05
684.026912 3.92315623508584e-05
686.035073 2.71039335298286e-05
688.042125 4.03358175663988e-05
690.053189 0.000122917082338616
692.065288 2.6448684122239e-05
694.067912 3.00782917692962e-05
696.078445 2.32100036450274e-05
698.07786 2.85900385818874e-05
700.081267 1.63625098611934e-05
702.096093 2.14825775764505e-05
704.183685 1.61089455600537e-05
706.361447 1.38496335938258e-05
708.535115 5.6249888493482e-05
710.585747 2.45333425648609e-05
712.337665 1.14114814467705e-05
714.131219 1.91310404911606e-05
716.219171 1.97121785311141e-05
718.264634 1.08959638592439e-05
720.179871 1.03169836644582e-05
722.101241 1.56883638775297e-05
724.080067 8.08185173814642e-06
725.889302 1.05643320110876e-05
727.635559 2.00047109248023e-05
729.374709 8.23406831499031e-06
731.209218 1.02046215485314e-05
733.3525 7.03527235466151e-06
735.349107 5.60683242777821e-06
737.350814 1.34005130877811e-05
739.35364 4.80888563319628e-06
741.356235 7.01408688683218e-06
743.358756 5.44625053754784e-06
745.395997 1.23915630042102e-05
747.402372 6.4769049303526e-06
749.391192 3.60953730554169e-06
751.395154 9.47559185518589e-06
753.520366 3.31689516089873e-06
755.51977 2.82217423055585e-06
757.512973 4.1417020471358e-06
759.507524 3.14917285931108e-06
761.324609 3.05318699727901e-06
763.265182 2.01324648249847e-05
765.288105 2.151782729349e-06
767.299212 5.21320469533925e-06
769.357418 2.5806239523675e-06
771.380413 2.15754963687444e-06
773.377641 2.97621882910623e-06
775.212268 1.77755155551252e-06
777.038487 3.11894827027881e-06
778.775846 2.95501259972015e-06
780.523314 1.54884044494216e-06
782.261258 3.37620993152913e-06
784.012175 1.35514797889014e-06
785.775116 1.58509547404461e-06
787.587409 2.08854768613547e-06
789.535996 1.16499874944654e-06
791.720608 1.39916825150902e-06
793.918407 8.26365012415274e-07
796.11489 1.02097000461651e-06
798.140151 1.51525768317309e-06
800.275473 1.9659208662347e-06
802.292403 1.3861786556825e-06
804.296321 8.38692448202519e-07
806.342942 8.8949215171609e-07
808.296346 9.81951455512862e-07
810.37127 5.77030515331467e-07
812.404324 5.28491049242911e-07
814.247347 1.26110528024692e-06
816.321663 7.55525911643681e-07
818.144741 2.18447625977866e-06
819.891641 9.80138225609487e-07
821.710382 4.49492299999525e-07
823.905139 4.77287994622972e-07
826.07849 5.49762903514828e-07
828.251675 8.46292150464813e-07
830.446565 4.99040717902428e-07
832.623409 3.38197255623425e-07
834.814965 2.60618927489777e-07
836.996945 1.83432171859272e-06
839.174914 6.35886136086282e-07
841.345636 2.54129859594308e-07
843.528212 2.97014176321847e-07
845.712828 2.50440978936578e-07
847.891232 2.90400547931784e-07
850.066809 2.45358484997045e-07
852.250802 2.18022511074055e-07
854.440633 1.80045624553343e-07
856.621618 5.34009250040473e-07
858.77905 2.825629076779e-07
860.858703 3.67858098175075e-07
862.938723 2.29997691876114e-07
864.937349 1.24443476672167e-07
866.900707 4.12849268964643e-07
868.949616 1.72633806500882e-07
870.882403 1.21941397474908e-07
872.781461 2.12532327967004e-07
874.836176 1.29704462042672e-07
876.899542 2.12578762708253e-07
878.932452 1.54720503626113e-07
880.932119 1.50181684008079e-07
882.977788 1.12577556770473e-07
884.792403 9.04763356497316e-08
886.549591 7.65609381332944e-08
888.590863 1.02923977888375e-07
890.779244 8.41964226950324e-08
892.778498 5.41145975360906e-08
894.95869 5.54194375664303e-07
897.130813 5.30467905822276e-08
899.308092 8.21704845192905e-08
901.488313 7.84203278452345e-08
903.656449 6.43895891719471e-08
905.835825 5.22976235310876e-08
907.831899 7.74909720254807e-08
909.995946 3.88364292108061e-08
912.167161 5.27207474930154e-08
914.237783 4.37758133172815e-08
916.408537 6.23803676764154e-08
918.581 4.13944400024831e-08
920.414365 6.66071523362415e-08
922.160339 2.26168806919645e-08
923.908492 4.07823313946301e-08
925.655222 2.98526544570711e-08
927.411947 2.5637902643913e-08
929.1654 2.19290658593133e-08
930.905676 3.1925208186662e-08
932.648685 3.88627167423959e-08
934.584322 2.7149101341402e-08
936.329491 1.49908134793956e-08
938.068394 4.2107343284912e-08
939.805639 2.83900849040112e-08
941.553117 1.63216863919426e-08
943.315289 2.08105443213209e-08
945.055333 2.40842599056665e-08
946.796017 1.55609848413076e-08
948.536992 2.61816966050055e-08
950.278747 1.11829488289751e-08
952.019609 9.17970665249584e-09
};
\addplot [semithick, color1]
table {%
513.593691 1
520.429815 0.0897552067633381
522.361958 0.0241981817251772
524.376379 0.0140607115939351
526.326719 0.00602391122871624
528.199284 0.00408331427437668
529.984365 0.000808146812399748
531.778952 0.00155537747561955
533.562237000001 0.00127766321052858
535.703714000001 0.00071706472859905
537.893067 0.000696868707444039
539.690655 0.000854313528022512
541.47251 0.000543446783628864
543.259685 0.00047469131055112
545.041482 0.000469185696986753
546.816849 0.000378272672897542
548.59543 0.000420219355936952
550.380287 0.000308726644628059
552.285490000001 0.000281369058136524
554.488402 0.000261792957284976
556.360107 0.000227597072448966
558.133589 0.000199679755822227
559.916025 0.000198074733048661
561.724019 0.000155846072429997
563.53959 0.000145689959146884
565.341556 0.000141820296638258
567.13665 0.000122331211446387
569.086765 0.000115931264263827
571.120195 0.000102471437404643
572.902513 0.000118107572153779
574.889148000001 0.000103013276388464
576.883522000001 8.25140638173735e-05
578.933867 7.98367549348567e-05
581.223928 7.8296655715013e-05
583.355695000001 5.94798291262974e-05
585.393037 5.62678607314492e-05
587.407095 7.83912860344326e-05
589.604643 5.22249198803824e-05
591.56895 4.72995040938741e-05
593.683376 5.50386125643815e-05
595.954442000001 4.33693623754731e-05
598.150437 3.90597992496879e-05
600.112269000001 4.31168002199793e-05
601.893185 3.59395668742404e-05
603.668344 3.0593568008885e-05
605.447768 2.51722790449838e-05
607.228842 2.29344280939236e-05
609.00146 2.15044623623177e-05
610.781466 1.80100639273911e-05
612.561891 2.68935393340049e-05
614.346506 1.86206589665474e-05
616.129067 2.89566768755171e-05
617.909697000001 1.57228544860959e-05
619.709626 1.95600711052586e-05
621.490999 1.38702850663627e-05
623.269019 1.22392287571959e-05
625.043507 1.11907818363793e-05
626.822979 9.82061222498215e-06
628.596352 9.10229177514851e-06
630.371559000001 1.03777704124574e-05
632.148293000001 9.49194524278866e-06
633.932875 1.20248894899714e-05
635.708065 7.4277346701762e-06
637.487052 7.12088822973319e-06
639.267972 7.89190413789684e-06
641.048939 5.27414886426533e-06
642.831609 4.84791943807432e-06
644.612071 7.56903042899967e-06
646.388816 4.18236168870855e-06
648.16738 4.98849413070217e-06
649.948852 4.44395294841323e-06
651.724134000001 3.98192252880781e-06
653.51786 4.79470834798883e-06
655.309768000001 4.4846828174337e-06
657.082095 3.17189709787398e-06
658.857757 2.40120195830553e-06
660.653381000001 2.59609605026357e-06
662.432535 2.72248316304636e-06
664.306505 2.06232523436312e-06
666.084568000001 1.59103005980662e-06
667.859941 2.39021719224409e-06
669.636842 1.47240642569083e-06
671.41656 1.71639819407172e-06
673.201571 1.7055619774947e-06
674.978758000001 1.87006245930523e-06
676.754892 1.30289939557844e-06
678.577437 1.62403253886185e-06
680.374558 9.80402248552936e-07
682.156570000001 8.6523124730344e-07
683.935492 1.20855767548888e-06
685.710471 8.66719947685454e-07
687.516516 8.94372216832799e-07
689.312601 8.1732192126493e-07
691.115017000001 9.13640837367273e-07
692.917226 8.99495447912707e-07
695.023646 5.45223059923352e-07
697.036388 4.2815130376724e-07
698.857779 8.58640054267441e-07
700.642387 4.49427404009971e-07
702.430156 4.79750309884542e-07
704.209135000001 3.61600588056346e-07
705.998667 4.39717121578615e-07
707.779609000001 4.10208723874874e-07
709.558689 2.61074594441271e-07
711.344714 2.66800289876789e-07
713.121893 2.6670250682126e-07
714.905499 2.6736831379393e-07
716.683140000001 2.53623677126009e-07
718.464393000001 2.20509852960166e-07
720.242486000001 2.54016287100818e-07
722.020778000001 2.09284054464626e-07
723.795683 1.70250522984431e-07
725.573845 1.81707234575347e-07
727.356604 1.40241425886295e-07
729.174331000001 2.49650235202672e-07
730.955013 1.04975469491967e-07
732.739076 9.91257054173385e-08
734.580177 1.41893010533713e-07
736.592365 1.09369448336331e-07
738.62446 1.06521310610131e-07
740.989736 1.08027941365595e-07
743.180803000001 6.92103857854144e-08
745.339928 5.72674845590428e-08
747.378331 6.79709603550055e-08
749.669954 6.05855440771429e-08
751.704162 5.24939141833735e-08
753.839640000001 4.72284134367944e-08
755.939106 4.6772427232461e-08
757.999406 4.95331791908153e-08
759.938337 4.94771659940246e-08
762.066789 4.26021426236251e-08
764.107793000001 3.68297433523107e-08
766.130635 5.04028096371161e-08
768.161835 3.07372696040459e-08
770.223183000001 3.85694056198679e-08
772.360838 2.39313202871365e-08
774.573022 3.38912251494762e-08
776.647865 3.04436501485057e-08
778.578961000001 2.05422662705129e-08
780.3569 1.47871925503288e-08
782.135026 2.01171191249903e-08
783.90993 1.37990722253255e-08
785.688375 1.91533956650386e-08
787.465502 1.14939754282137e-08
789.243138 1.32231279634868e-08
791.019009 9.41393594701483e-09
};
\addplot [semithick, color0]
table {%
954.252842 1
956.056406 0.807403182734015
957.830998 0.209157114227959
959.562094 0.0574498774819136
961.293164 0.0446522480661267
963.024031 0.021515678867764
964.758472 0.01898376625362
966.484904 0.0135876740716136
968.21915 0.0097633141335379
969.950451 0.00719162288226795
971.678773 0.0056447829620256
973.409405 0.00546213918872563
975.13899 0.00782191004677242
976.868217 0.00354062408711678
978.595327 0.00293077112260834
980.322316 0.00226161916542685
982.07403 0.00211147008791219
983.808548 0.00158923643794072
985.539013 0.00132321276618765
987.268546 0.00122176486723707
988.999697 0.00126752461784391
990.733182 0.00227100326614295
992.463512 0.000865550973060411
994.194713 0.000757840274601954
995.923735 0.00132290242930215
997.66378 0.000650559728886852
999.396855 0.000497834137512665
1001.126026 0.000484482082311342
1002.854396 0.000397790354591021
1004.583735 0.00074561922245561
1006.318012 0.00168431595146445
1008.046016 0.000324518751709368
1009.774064 0.000279859328742144
1011.499864 0.000238440666883214
1013.232084 0.000220165377066704
1014.96103 0.000338852078456284
1016.691578 0.000177489429138718
1018.421969 0.000142270254458325
1020.153875 0.000159501309910695
1021.885226 0.000212061945448712
1023.614805 0.000209648088037511
1025.347773 0.000111691730066137
1027.079123 9.16810210602272e-05
1028.809422 0.000180035915238713
1030.53943 8.83870313585687e-05
1032.269105 0.000130595844544588
1034.000632 7.02311933404634e-05
1035.733258 5.68297645914652e-05
1037.459981 0.000258934230200693
1039.187105 5.04355111432053e-05
1040.916571 4.73762192540598e-05
1042.651316 8.16791906138276e-05
1044.378748 4.05947264633152e-05
1046.105285 5.24617180539135e-05
1047.83513 3.24212464087842e-05
1049.565634 2.4368516382776e-05
1051.29804 3.50133555204801e-05
1053.026507 2.9822027252813e-05
1054.759629 2.07094735804864e-05
1056.490415 3.7457645772989e-05
1058.218881 1.91396065178415e-05
1059.946415 2.13446692831919e-05
1061.674131 2.75260994599881e-05
1063.403061 1.49341099551232e-05
1065.133763 1.0860205766136e-05
1066.862059 1.46389166817763e-05
1068.593969 1.78671602748245e-05
1070.322901 1.11031852053794e-05
1072.052026 9.23627940321391e-06
1073.778427 1.15838426987398e-05
1075.504351 1.16625303453547e-05
1077.231937 7.09566927830926e-06
1078.965969 5.49690866714071e-06
1080.698491 8.41254450610259e-06
1082.425353 1.15476033958252e-05
1084.15628 2.10957994179589e-05
1085.887571 4.65859258687873e-06
1087.613989 4.53046830906909e-06
1089.342405 5.00417946939788e-06
1091.069814 3.70827709781267e-06
1092.800131 4.18350189483589e-06
1094.529111 4.8532798718905e-06
1096.257504 2.91052216683018e-06
1097.984043 2.11123759840904e-06
1099.714875 1.69706825613142e-05
1101.442393 1.93635846353211e-06
1103.170292 2.49724846501887e-06
1104.899348 2.1482938601225e-06
1106.628361 2.07555137796272e-06
1108.357662 1.33382577306514e-06
1110.085068 1.11569496150927e-06
1111.815588 1.78186305929105e-06
1113.547964 1.02950308906753e-06
1115.284992 2.06075102679317e-06
1117.012553 1.14267862893736e-06
1118.739771 9.38344168164203e-07
1120.469751 1.37073445321779e-06
1122.197745 6.25581512992715e-07
1123.927227 9.48062299177348e-07
1125.655068 6.90850374150217e-07
1127.382716 9.43664068923229e-07
1129.113978 4.03508236254046e-07
1130.843514 3.75035282333256e-06
1132.574914 3.39710293232837e-07
1134.300243 5.04279806626755e-07
1136.029686 4.11847057317572e-07
1137.759154 3.99350062824171e-07
1139.488454 2.19606602154228e-07
1141.220981 5.1033114827104e-07
1142.954758 2.37306173851025e-07
1144.688954 3.76463096543722e-07
1146.417568 9.57076822669299e-07
1148.144531 2.69570308643675e-07
1149.874049 1.65084644131239e-07
1151.603592 2.52828564942088e-07
1153.334169 1.18417407378208e-07
1155.063932 1.89798195459004e-07
1156.795551 1.82703903127406e-07
1158.52298 1.19046156343376e-07
1160.253131 2.68669391511545e-07
1161.995767 1.57975850567369e-07
1163.72656 1.45138287498865e-07
1165.452266 9.28719796512181e-08
1167.181696 7.58191210310504e-08
1168.910628 9.79943289968511e-08
1170.640676 5.74798952982427e-08
1172.372418 7.00413077416874e-08
1174.102732 4.70681097360707e-08
1175.829201 7.32093012470285e-08
1177.557825 3.83147668879135e-07
1179.28566 3.72633515029962e-08
1181.01439 5.65965518493804e-08
1182.745321 3.70864168336453e-08
1184.475895 3.33515292710637e-08
1186.209012 4.27205437214068e-08
1187.938526 4.71793225355922e-08
1189.666472 3.856842018558e-08
1191.394174 2.7975251130289e-08
1193.124153 4.96557065601286e-08
1194.853257 2.22926349586089e-08
1196.579864 2.18027178469088e-08
1198.308944 4.37559402946719e-08
1200.041889 1.47637472246613e-08
1201.770579 2.74915914415221e-08
1203.500046 1.62098218694057e-08
1205.23034 1.16568639792656e-08
1206.959924 8.89342010875944e-09
};
\addplot [semithick, color1]
table {%
793.557571 1
800.13767 0.0911427645143914
802.196634 0.0265460284643973
804.242931 0.0173918933167727
806.134463 0.0106697408679668
808.020554 0.00718434276370028
810.218269 0.0059451486867994
812.429697 0.00433426407875372
814.44103 0.00320165343098197
816.376916 0.00236278380739917
818.278823 0.00220587518437407
820.081995 0.00168009747355593
822.187107 0.00129557758420921
824.335999 0.00110850750341525
826.392442 0.000899516553039836
828.426403 0.000798780552315318
830.479868 0.000665987531215606
832.524814 0.000532838217230583
834.562139 0.0005156920881273
836.41557 0.0004500637145942
838.333796 0.000371819768007053
840.388018 0.000339162631367577
842.42963 0.000284859629643476
844.369789 0.000248735298034049
846.586415 0.000236389684194195
848.680458 0.000196114453155886
850.867666 0.000184813127254151
852.899749 0.000163807909102225
854.863096 0.000142435892639105
856.825664 0.000122934130068368
858.943024 0.000116381051480572
860.996721 0.000124709158517965
863.100602 0.000113119160576514
865.116846 0.000102659673570746
867.336542 7.52584160591281e-05
869.249137 7.02924360700946e-05
871.289531 9.75968738366879e-05
873.166319 6.49425179471096e-05
874.941634 5.05939243124861e-05
876.715872 4.30297196663709e-05
878.489589 3.87138524133327e-05
880.261527 3.85377869538616e-05
882.034813 4.7410047074645e-05
883.809265 2.75073774483697e-05
885.579378 2.24484045605451e-05
887.349907 2.10531286132097e-05
889.125424 2.93697223491933e-05
891.050474 1.7362107023584e-05
893.190901 1.48300289899869e-05
894.980768 1.44804775138738e-05
896.770073 1.63985687657753e-05
898.560165 1.05779312175704e-05
900.351593 8.69882987843914e-06
902.150887 8.24829580366975e-06
903.944684 8.20662625990197e-06
905.71562 8.12357072589293e-06
907.489364 7.38463308911791e-06
909.267188 6.6437176821422e-06
911.06608 4.81033387294915e-06
912.836138 4.04421314046696e-06
914.603568 3.38553932464208e-06
916.384413 5.60370681452225e-06
918.159257 3.03374256979834e-06
919.928939 3.63335992710881e-06
921.698976 2.37671192775622e-06
923.470367 2.32211118284288e-06
925.241168 2.54800342613899e-06
927.012569 1.55688579704828e-06
928.791271 1.24851917305657e-06
930.579604 1.089411429342e-06
932.351616 1.19780745028929e-06
934.125523 1.16125314997807e-06
935.893273 7.52912778914789e-07
937.664416 6.68472894720509e-07
939.433895 8.36895725441414e-07
941.235124 4.93597812638444e-07
943.040994 7.58270738226459e-07
944.828032 4.82776327203579e-07
946.600427 5.51510175313411e-07
948.372467 3.65682351411149e-07
950.144463 5.77271729765553e-07
951.973781 2.71195426786616e-07
953.752473 2.5768860423996e-07
955.52223 3.01334364369782e-07
957.306155 1.75072512846909e-07
959.083904 2.19792227628896e-07
960.855595 1.69753392356333e-07
962.631365 2.27710265932426e-07
964.411263 1.4584269290771e-07
966.185419 1.39758028486441e-07
967.957381 1.10000419769023e-07
969.727136 1.02996819657457e-07
971.497093 7.4177566513742e-08
973.267975 9.11045871190199e-08
975.041695 5.98608589855025e-08
976.817889 8.31116232384855e-08
978.583932 4.82981438922929e-08
980.358288 3.9313119457933e-08
982.132489 3.06257332725343e-08
983.900567 5.15425351329012e-08
985.678454 3.15086626360504e-08
987.449922 3.10649939652032e-08
989.230628 2.06028889207575e-08
991.001615 3.13353552281446e-08
992.7737 1.89835659628716e-08
994.549308 1.75218332149065e-08
996.31959 1.63770858570623e-08
998.095802 1.68780845709502e-08
999.882839 1.15569242916659e-08
1001.658698 1.01080054544374e-08
1003.431066 1.13059852481353e-08
1005.204014 9.63252484795002e-09
};
\addplot [semithick, color0]
table {%
1208.805235 1
1210.599267 0.818272315902229
1212.364867 0.242259321513288
1214.096429 0.0710913023207334
1215.826827 0.0556398448146228
1217.560121 0.0293573872881461
1219.291423 0.0244467802092831
1221.030234 0.018876000113289
1222.76633 0.0146852573463301
1224.498175 0.0100442269234059
1226.232733 0.00769706451104584
1227.965849 0.0147300271265303
1229.697161 0.0066261787876248
1231.432284 0.00462561180337677
1233.166055 0.00371654280960053
1234.896576 0.0031330192027257
1236.62781 0.00263286899937473
1238.35978 0.00196412558348577
1240.091343 0.00159251555946919
1241.822085 0.00136424434615748
1243.552951 0.00231762050031698
1245.284398 0.00122468103688415
1247.016679 0.000947809494064252
1248.749816 0.000706626621682135
1250.488116 0.000790674521935505
1252.232636 0.000949542829447962
1253.966364 0.000556709642374778
1255.697341 0.000484319921080416
1257.429234 0.000377723181005276
1259.161354 0.000623244987678761
1260.891935 0.0010554229636783
1262.62218 0.000272202217708809
1264.355646 0.000234619553298099
1266.087816 0.000194464164953167
1267.818116 0.000159516955075347
1269.549193 0.000261338034655031
1271.279625 0.000132677495893211
1273.010685 9.28776262420741e-05
1274.742025 8.72960374667698e-05
1276.473806 0.000355980076217417
1278.207786 0.000110314641985049
1279.945361 6.70357473157585e-05
1281.677766 5.26030935087663e-05
1283.409459 4.6612590571314e-05
1285.138275 6.41647155277276e-05
1286.868134 6.41451134254005e-05
1288.602128 3.66401395363573e-05
1290.335256 2.82704255270533e-05
1292.069597 0.000171431785847554
1293.80237 2.38691572997502e-05
1295.536019 1.7702520771327e-05
1297.265549 3.12936155691308e-05
1298.996015 1.90744630601135e-05
1300.727345 1.27879934823477e-05
1302.460451 9.24395305244678e-06
1304.193401 8.47735722519072e-06
1305.925163 2.39824869043666e-05
1307.657572 7.73857605881764e-06
1309.388381 8.93455016341e-06
1311.12859 5.81913773060934e-06
1312.860257 1.09251957080432e-05
1314.590103 4.43513888357525e-06
1316.324022 5.93268483333126e-06
1318.05986 3.48712072513073e-06
1319.793793 6.53278212473133e-06
1321.525455 2.81284629139959e-06
1323.256953 4.57188042380182e-06
1324.990108 2.86303222681479e-06
1326.723934 2.24290652882191e-06
1328.457352 2.52445594546981e-06
1330.191019 1.78064188233118e-06
1331.922605 1.47538838909502e-06
1333.653328 1.4612330842354e-06
1335.385559 1.18829122215443e-06
1337.117359 2.3301508608057e-06
1338.851614 2.66631313960355e-06
1340.589064 1.37337200020953e-06
1342.319896 6.82829221450436e-07
1344.051321 7.08653441443475e-07
1345.780725 8.16341906620547e-07
1347.51382 6.18384548198649e-07
1349.244433 4.87808863166941e-07
1350.980022 3.27400830837085e-07
1352.712949 1.40613222722748e-06
1354.446524 1.04775348040622e-06
1356.177946 2.74444745464533e-07
1357.913605 2.40215681854498e-07
1359.648842 2.80679847491535e-07
1361.380996 3.9119459747821e-07
1363.113911 1.6877731031517e-07
1364.847487 1.30363645981031e-07
1366.577923 1.06284502519644e-07
1368.31047 2.93965754844562e-07
1370.042615 1.29074461612528e-07
1371.779649 7.77591938405139e-08
1373.511741 2.13230220749015e-07
1375.242996 5.63950486028745e-08
1376.974264 6.69144576779616e-08
1378.706843 6.2682215350718e-08
1380.438608 5.26029015069904e-08
1382.166257 7.73653963252643e-08
1383.896085 4.1348679137388e-08
1385.624843 7.34040638254236e-08
1387.353472 5.45694037316012e-08
1389.085417 2.75133692473097e-08
1390.813716 2.21402557574557e-08
1392.54167 2.38147117772554e-08
1394.269562 1.64391008919924e-08
1396.000311 2.91792117845172e-08
1397.732901 6.40116666148779e-08
1399.460709 1.19096080549985e-08
1401.198659 2.7067697549069e-08
1402.928654 1.00314415199265e-08
1404.658869 1.33535134615926e-08
1406.389994 6.75843686782672e-09
};
\addplot [semithick, color1]
table {%
1008.026384 1
1014.739535 0.10156732384356
1016.591494 0.034333629422772
1018.685083 0.021487107744974
1020.554408 0.0138854234494908
1022.352164 0.00955073799095624
1024.124393 0.00777911114272731
1025.908281 0.00605273401915171
1027.689475 0.00488180646261698
1029.489517 0.00356105219105137
1031.265839 0.0029941091248344
1033.048677 0.00245762483640191
1034.820926 0.00191275880848553
1036.967902 0.00157445887118763
1038.918645 0.00127088380669054
1040.698597 0.00122616446681266
1042.47317 0.000919890690026271
1044.548743 0.000760802437338683
1046.646332 0.000727005094669489
1048.68873 0.000667927679653431
1050.726804 0.000519185496661831
1052.718366 0.000458152528791348
1054.547967 0.000410201370307888
1056.325502 0.000333359942648819
1058.099874 0.000314954159889192
1059.884369 0.00022964971893214
1061.7221 0.000220898959037529
1063.512981 0.000186867176685361
1065.532969 0.000151058030942693
1067.31272 0.000125294983620029
1069.085252 0.000117519028426575
1070.860361 9.30152481163594e-05
1072.651263 0.000118722038892285
1074.522409 7.8521292930063e-05
1076.620049 5.37699796259381e-05
1078.512259 4.55365510355356e-05
1080.694707 3.80995269133743e-05
1082.637067 4.65841952017412e-05
1084.416867 3.23320843077165e-05
1086.190113 2.53200748840635e-05
1087.966385 2.0877762682147e-05
1089.760755 2.14041361689364e-05
1091.566886 1.99249040396012e-05
1093.338929 1.20244797711863e-05
1095.112637 1.07392430590736e-05
1096.912029 1.32194145497855e-05
1099.115307 8.30165360466446e-06
1101.000035 6.44623980666414e-06
1102.941137 5.51847291357396e-06
1105.161721 4.10323806522608e-06
1107.385973 4.3511679118104e-06
1109.338271 3.96337805581642e-06
1111.151733 2.48620031448565e-06
1112.978024 2.2368982930796e-06
1115.174664 2.68876788633142e-06
1117.270347 2.1576943229498e-06
1119.34991 2.12493141145042e-06
1121.33849 1.23668312345997e-06
1123.323516 1.09024037994297e-06
1125.175029 8.14937324216612e-07
1127.002628 6.39699828103158e-07
1128.92552 6.07664316514598e-07
1130.844026 6.45717500065571e-07
1132.617209 4.83414718006493e-07
1134.392666 4.64397473398051e-07
1136.189561 3.2387585852231e-07
1137.989604 3.59119990097014e-07
1139.770151 2.37948425856868e-07
1141.543852 1.72862562392539e-07
1143.491325 1.43998487909675e-07
1145.365677 1.26411093038564e-07
1147.179914 1.62786090936517e-07
1148.992596 8.02907907287171e-08
1151.186248 6.30940909105587e-08
1153.158036 7.01825829657808e-08
1155.028882 5.47233223150631e-08
1157.000604 6.53818319404438e-08
1158.782543 3.89237122835261e-08
1160.770756 5.74004655882073e-08
1162.599959 3.09746006127651e-08
1164.397871 2.8447839205526e-08
1166.194125 1.74408932904179e-08
1167.987887 1.88443915062669e-08
1170.070453 1.43130021924602e-08
1172.229325 1.04925205360079e-08
1174.233667 1.17765795673953e-08
1176.221598 7.80492179169907e-09
};
\addplot [semithick, color0]
table {%
1408.471484 1
1410.525314 0.882677761201002
1412.76457 0.309670955099524
1414.940532 0.10576572804898
1417.12426 0.0841112047270652
1419.316068 0.0502922147108168
1421.381363 0.0403269882840885
1423.120328 0.0315104234797059
1424.85213 0.027613822219445
1426.586204 0.0171940460380396
1428.320403 0.0125649054617688
1430.053485 0.0342342385468219
1431.787783 0.0107315233431743
1433.528906 0.00761682477566453
1435.260607 0.00566665432208058
1436.992684 0.00489004668308005
1438.733467 0.00352785854055746
1440.498089 0.00282697958975048
1442.254582 0.00235946084946117
1444.381321 0.00182507379066131
1446.28954 0.00208056048933845
1448.462228 0.00211087048621502
1450.62845 0.00124616187543843
1452.800526 0.000836458243197229
1454.969365 0.000724884148310258
1457.14051 0.000694596454165217
1459.155862 0.000615514181703636
1461.326489 0.000395194927668283
1463.196686 0.0004208377585385
1464.929456 0.000395598198124697
1466.663921 0.000918697799807139
1468.398112 0.000278425181621581
1470.132473 0.000209302300691924
1471.86441 0.000162037682858397
1473.596435 0.000172972758771172
1475.341813 0.000172795670286419
1477.074182 0.000106841712176908
1478.806106 7.17351870660708e-05
1480.540289 6.996355438425e-05
1482.273229 0.000169351233257836
1484.007309 9.26325676648504e-05
1485.740006 4.12270510874369e-05
1487.472585 3.57924453080998e-05
1489.203017 2.7372889238935e-05
1490.933779 2.12147511537555e-05
1492.667396 1.87129615904653e-05
1494.402713 2.49156838541279e-05
1496.569641 1.26337037957049e-05
1498.739859 6.9573822985203e-05
1500.909401 1.26408242352784e-05
1503.081451 1.17676306081066e-05
1505.258774 7.76051210680125e-06
1507.418268 7.03555586866665e-06
1509.592748 7.05708037656616e-06
1511.34872 4.01992387792646e-06
1513.08271 3.25292429229784e-06
1514.816428 2.89266193791756e-06
1516.549863 7.84981550740253e-06
1518.280908 4.47804441175053e-06
1520.458121 1.97169134247244e-06
1522.617644 2.07539211294119e-06
1524.792819 1.50822500545013e-06
1526.955486 1.045288234975e-06
1529.124467 1.43537780006783e-06
1531.285463 8.02656255925569e-07
1533.459457 1.14550723615524e-06
1535.638873 1.07220168230004e-06
1537.522257 8.18439245667697e-07
1539.254492 1.2441629989285e-06
1540.986792 6.8774926389615e-07
1542.720383 4.19806217553207e-07
1544.454752 3.36708429312322e-07
1546.187276 3.7560073000031e-07
1547.9211 2.2196092391392e-07
1549.654423 4.34596119915634e-07
1551.387388 1.71274263888174e-07
1553.125294 1.13821555298461e-06
1554.85788 1.555080215204e-07
1556.590172 1.39607463487244e-07
1558.320689 8.99884722858898e-08
1560.05309 6.19499067838696e-08
1561.78439 9.31021016056566e-08
1563.518012 6.01216944570034e-08
1565.250678 6.53646779220535e-08
1566.982449 3.69775889570879e-08
1568.714828 9.62830557047104e-08
1570.446952 5.18306051100901e-08
1572.178525 2.90717947577393e-08
1573.910763 4.54180198390428e-08
1575.642456 3.15081233426165e-08
1577.373373 1.83680779528913e-08
1579.10786 1.25065458903252e-08
1580.840944 1.26269733196054e-08
1582.57256 1.10461286522775e-08
1584.312023 2.85007190283326e-08
1586.044742 1.75190956363716e-08
1587.77907 8.70914992295231e-09
};
\addplot [semithick, color1]
table {%
1178.841828 1
1185.119341 0.13520718401907
1186.937365 0.0543183634497806
1188.909694 0.0329497808419938
1190.961682 0.0234263668176792
1193.099187 0.0169864028726687
1195.156529 0.0139925486597589
1197.189612 0.0108878574031763
1199.255085 0.00835649750440487
1201.28406 0.00599055969610091
1203.178066 0.00516434095737581
1204.953956 0.00398818462354353
1206.735219 0.00334485994762383
1208.513017 0.00268269421467095
1210.296356 0.002213496586031
1212.068388 0.00217778682619692
1213.837311 0.00147603175232371
1215.610468 0.00140803123645132
1217.382912 0.00111901873216995
1219.154319 0.00100821084068454
1220.928202 0.000789506322812134
1222.696997 0.000628416467606209
1224.48652 0.000496562012986069
1226.253883 0.000416613206417103
1228.025905 0.000326130903870491
1229.974396 0.000240681733847384
1231.797897 0.00019939158291641
1233.568639 0.000153429129976108
1235.339356 0.000125459126038543
1237.137466 0.000107667269272456
1238.910248 8.08197106722744e-05
1240.685926 6.05792561752281e-05
1242.457218 4.62158857679012e-05
1244.227439 5.00116746900274e-05
1246.002863 4.08836028273995e-05
1247.772727 2.86250313377046e-05
1249.543954 2.29849390035916e-05
1251.322226 2.38844045026314e-05
1253.094297 1.61717841866269e-05
1254.864975 1.28511216688041e-05
1256.638189 9.2082271165195e-06
1258.413594 7.46076994452548e-06
1260.188039 8.48595947701931e-06
1261.960844 5.3869116265434e-06
1263.735948 4.25051634464303e-06
1265.523114 3.16872208231232e-06
1267.334596 2.56724956173524e-06
1269.123327 2.92011637242875e-06
1270.92699 1.85091885574955e-06
1272.776528 1.34180340620055e-06
1274.559252 1.04021183408033e-06
1276.50411 8.88284719387982e-07
1278.590479 9.00360980531013e-07
1280.806595 5.72383741803389e-07
1283.016483 4.33376277687863e-07
1284.988005 3.65196509836731e-07
1287.054042 3.49028192103689e-07
1288.941353 4.06189863793683e-07
1290.720339 2.16513536739336e-07
1292.520548 1.62231314001954e-07
1294.294527 1.34268313240392e-07
1296.069142 1.28666785902168e-07
1297.849235 8.43227511086363e-08
1299.625488 6.65527680240578e-08
1301.400341 7.61233551018288e-08
1303.174367 6.61199555504632e-08
1304.950902 4.29516124953038e-08
1306.726496 2.72205754419616e-08
1308.710397 2.05953997160241e-08
1310.926988 1.63769786953586e-08
1313.132204 1.55773465574139e-08
1315.35098 1.19390572108686e-08
1317.588887 9.864500766699e-09
};
\addplot [semithick, color0]
table {%
1590.072993 1
1591.996014 1.25892340553418
1593.743862 0.442958859796317
1595.473873 0.185694102027004
1597.210363 0.151451458936597
1598.945029 0.110107164027802
1600.678151 0.091285960116365
1602.410667 0.0668819947984154
1604.145524 0.0628093350737482
1605.878837 0.0382698521275427
1607.614893 0.0284050184408119
1609.349519 0.033009727833984
1611.082547 0.0296224079852868
1612.816953 0.0153204175148354
1614.551214 0.0117143666948396
1616.287004 0.00918396629108533
1618.043492 0.00728446409947007
1619.780719 0.00461891472592516
1621.527364 0.00359158743000327
1623.262503 0.00274889589344499
1624.997354 0.00234963525990905
1626.731281 0.00498093613847313
1628.464912 0.00168576115129762
1630.199232 0.00123569166507798
1631.93316 0.0010187562771903
1633.682345 0.000807146533896264
1635.414763 0.000961809132624422
1637.148322 0.000717089619685102
1638.883645 0.00054770179952566
1640.619241 0.000612130759018172
1642.354858 0.000682541181725462
1644.090482 0.000471364143750875
1645.824617 0.000228299866220261
1647.5583 0.000187122410793583
1649.291422 0.000170318550035927
1651.028993 0.000232088725388518
1652.764687 0.000115406604247722
1654.499762 8.18384851495256e-05
1656.235887 6.84440300056229e-05
1657.970962 9.30080873146337e-05
1659.703651 0.000186988534823705
1661.43703 4.29866688706606e-05
1663.172153 3.0907492083339e-05
1664.9097 2.58953729816432e-05
1666.6441 1.8045045124242e-05
1668.377166 1.38973259073298e-05
1670.116167 1.51074310372519e-05
1671.850153 1.36125838215815e-05
1673.586075 1.30768739075335e-05
1675.321337 2.06918168410443e-05
1677.055453 8.38913499861498e-06
1678.786231 1.28357867859607e-05
1680.522656 4.86294924314107e-06
1682.257798 4.08087258479645e-06
1683.99204 2.96812707794602e-06
1685.724492 2.70870310039537e-06
1687.457157 2.34157594631056e-06
1689.191914 1.60298452760894e-06
1690.925322 8.62164357617996e-06
1692.656514 2.69109454005255e-06
1694.3901 1.10412322931446e-06
1696.12125 8.01960121460129e-07
1697.851742 6.31862840211085e-07
1699.58141 7.31915168134056e-07
1701.316647 3.9440803740715e-07
1703.050034 3.6731612072615e-07
1704.78348 4.94461537583423e-07
1706.518148 9.78266498181808e-07
1708.252351 2.33359068066861e-07
1709.98883 4.94624990325069e-07
1711.72497 1.59385161322244e-07
1713.459568 2.6601747708775e-07
1715.192126 1.87988661888046e-07
1716.924748 1.12245659277174e-07
1718.657787 7.98322771954845e-08
1720.391543 9.81216548199836e-08
1722.125686 4.0279891026414e-07
1723.85745 5.51054433793084e-08
1725.590938 3.74640692593947e-08
1727.322162 3.75950902710955e-08
1729.053481 2.94659355287259e-08
1730.787947 1.80174587536791e-08
1732.522982 3.76465844157703e-08
1734.256435 1.76188211405712e-08
1735.99344 1.87362006990659e-08
1737.752661 1.67487427214135e-08
1739.505631 1.70493400741162e-08
1741.252861 7.23071475363706e-09
};
\addplot [semithick, color1]
table {%
1320.172017 1
1326.824743 0.248820955880855
1328.898003 0.110938529457055
1330.981807 0.0680584079472129
1332.964065 0.0530319842512949
1335.035651 0.0432812590168041
1337.024247 0.0290319309620613
1339.100885 0.0215095709670067
1341.321199 0.0165463544529106
1343.264962 0.0116356076744143
1345.137517 0.0103846620281675
1346.908663 0.00733379388851304
1348.689059 0.0061812640986991
1350.461473 0.00503817009499729
1352.23187 0.00447537959462417
1354.000812 0.00403918739648224
1355.777635 0.00261750702591131
1357.549266 0.00244847962430531
1359.319457 0.001807193859749
1361.086574 0.00156688199664357
1362.851893 0.00117020423617186
1364.626302 0.00080464978310612
1366.397237 0.000588750070398256
1368.167652 0.000529442505587108
1369.960278 0.000390064757151127
1371.729834 0.000278719310274855
1373.494386 0.000221103058918435
1375.268717 0.000167335997845522
1377.042393 0.000128817456067335
1378.817544 9.85794604413643e-05
1380.584571 7.32331435181233e-05
1382.35231 5.86309366721891e-05
1384.130212 6.27102454367108e-05
1385.899304 4.92368810137248e-05
1387.671045 3.03859760972182e-05
1389.440414 2.55207676238092e-05
1391.208498 1.66665476925435e-05
1392.977207 1.35957861821071e-05
1394.744511 9.36607641539168e-06
1396.508564 1.21060531480456e-05
1398.278595 6.98556633115141e-06
1400.052354 4.91233816940033e-06
1401.820311 3.88509637596899e-06
1403.588055 2.90601404228149e-06
1405.357952 3.32252067601299e-06
1407.123394 1.97786965152602e-06
1408.891356 1.38612881252431e-06
1410.680053 1.59783277047001e-06
1412.454561 9.83046042466924e-07
1414.218987 6.59162320027348e-07
1415.991977 6.22255018953725e-07
1417.760024 5.70648123453716e-07
1419.529344 3.72820586568715e-07
1421.290176 2.58925618781072e-07
1423.062691 1.77268371316047e-07
1424.837578 1.99827673818756e-07
1426.608988 1.53273920435383e-07
1428.377263 1.23145007188581e-07
1430.159132 9.50118086640836e-08
1431.958279 6.35138877843641e-08
1433.728565 6.97286083710095e-08
1435.499225 3.94692042075118e-08
1437.265621 3.08024622860708e-08
1439.038878 3.07869168197978e-08
1440.808669 1.76577180060294e-08
1442.574609 1.19313316220521e-08
1444.348112 8.19325548405378e-09
};
\addplot [semithick, color0]
table {%
1743.081169 1
1744.837177 1.6087281600166
1746.569698 0.612555880206803
1748.30407 0.299152102915402
1750.035878 0.22578444367571
1751.772563 0.20018263128823
1753.504789 0.197681977754026
1755.234285 0.142765506974438
1756.965293 0.124288691845791
1758.697984 0.0792975588056127
1760.429985 0.0569366242232655
1762.161353 0.0403555901346313
1763.893457 0.169941208598842
1765.624901 0.0277019738404317
1767.356214 0.0207037950721268
1769.089007 0.0153863239446574
1770.827023 0.01232196007754
1772.57467 0.00759756077679174
1774.304391 0.00605331258245564
1776.036406 0.00473727048218437
1777.771687 0.0036466881102338
1779.503999 0.0144757330779211
1781.237009 0.00271976864732875
1782.970098 0.00201117440806909
1784.701284 0.0014362629351664
1786.439858 0.00117220300212632
1788.173764 0.00118742340991963
1789.905262 0.00112767108830797
1791.636974 0.000843674848908517
1793.370425 0.000753957886684349
1795.105767 0.00333452281850322
1796.838609 0.000424798653575849
1798.569465 0.000383475486763922
1800.303753 0.000245431838664922
1802.038162 0.000205782629008761
1803.772779 0.000293485858723227
1805.505167 0.000154889851826305
1807.237562 0.00011113088340707
1808.972213 0.000130098810389568
1810.704808 0.000134255897430585
1812.439238 0.000114491063082427
1814.170661 5.92295296251871e-05
1815.89989 3.69088063694307e-05
1817.633735 3.38979856246879e-05
1819.364751 2.11799976931397e-05
1821.095563 1.60173824157012e-05
1822.827076 2.54866034409014e-05
1824.558366 1.77367131290851e-05
1826.290908 1.41656668462055e-05
1828.020872 2.07643033991534e-05
1829.752398 8.69657910757747e-06
1831.488751 1.62855602744456e-05
1833.22151 5.24287492846242e-06
1834.951268 3.99701014946331e-06
1836.680444 3.3162210231057e-06
1838.413004 4.58477845992848e-06
1840.143487 4.06028282602606e-06
1841.874651 2.22043435992531e-06
1843.605665 1.51957930207912e-06
1845.336458 1.39316571883172e-06
1847.069223 1.55621092681295e-06
1848.800348 1.04847372456203e-06
1850.531879 8.63736940517262e-07
1852.263831 5.03621528054854e-07
1853.993922 3.7412409988474e-06
1855.724394 3.35328587008411e-07
1857.456996 6.02706331113166e-07
1859.188915 2.58143501846866e-07
1860.931455 2.25158311744436e-07
1862.66886 2.17672344134753e-07
1864.40331 2.89419910988686e-07
1866.137952 1.27161749539068e-07
1867.869257 1.76259821333223e-07
1869.598891 3.07405572612669e-07
1871.329954 9.04060318348974e-08
1873.089656 8.40568981406434e-08
1874.820289 6.53056491989719e-08
1876.552431 3.56952539713205e-08
1878.2839 2.75971298851568e-08
1880.014758 2.02004631003805e-08
1881.744379 1.56038499490164e-08
1883.47592 6.70587097891658e-08
1885.207601 1.82194004176855e-08
1886.941591 1.04182971392006e-08
1888.673493 1.25039199229798e-08
1890.406906 1.23938650793195e-08
1892.143682 6.65322566974713e-09
};
\addplot [semithick, color1]
table {%
1446.742298 1
1453.942018 0.462463990035682
1456.024538 0.220454760080254
1458.156198 0.143243260718361
1460.102175 0.11374811977842
1462.305647 0.0885366158767486
1464.209153 0.0576879343029282
1466.182507 0.0386524078028073
1468.159069 0.0290530859483476
1470.380217 0.0207234434443234
1472.29642 0.017687825704345
1474.072631 0.0125626346138141
1475.843736 0.0107730687095972
1477.622151 0.00860285057191416
1479.408135 0.00686192462637046
1481.187231 0.00500734795190893
1482.96076 0.00372117130603528
1484.747816 0.00293474986743842
1486.515929 0.00288232682526603
1488.290221 0.00179098897016448
1490.064122 0.00146874469030106
1491.836347 0.00132455250346479
1493.606052 0.000859943607649922
1495.382367 0.000664281431034654
1497.151724 0.000544940222915885
1498.93018 0.000351859080644449
1500.7151 0.000295217641772103
1502.487155 0.000218394908454212
1504.262271 0.000150733893476565
1506.036649 0.000113443121893792
1507.815119 8.77812300387742e-05
1509.587957 8.0817571941136e-05
1511.358618 6.84928731478125e-05
1513.127968 4.60042641426994e-05
1514.903094 3.40307701289556e-05
1516.671278 2.2977819028737e-05
1518.45845 1.69775235891245e-05
1520.336821 1.32872536731276e-05
1522.53478 8.7447279774397e-06
1524.597956 6.8673291979135e-06
1526.568557 6.90311790188025e-06
1528.651328 4.69654276875943e-06
1530.859438 3.25579616969358e-06
1532.928032 2.31678357237145e-06
1535.021159 2.50225635255009e-06
1536.822597 1.66410519234052e-06
1538.599561 1.17045100333878e-06
1540.37169 8.49354302047422e-07
1542.149035 6.12857934367282e-07
1543.923435 4.68834861809905e-07
1545.697113 4.67865506071294e-07
1547.472964 3.17112820986606e-07
1549.24114 2.28135198557699e-07
1551.017161 2.72115425801557e-07
1552.791613 1.44295595848111e-07
1554.559443 1.5720567023961e-07
1556.32982 8.95871393903555e-08
1558.098082 6.83756160352466e-08
1559.870402 7.75648895535855e-08
1561.688677 3.8153441630022e-08
1563.45981 3.46663322001941e-08
1565.226822 2.98051227293421e-08
1566.99629 1.67068694587224e-08
1568.765712 1.73865701083947e-08
1570.537633 1.08279317373893e-08
1572.307504 7.69155557707964e-09
};
\addplot [semithick, color0]
table {%
1893.95521 1
1895.744726 0.978578487844548
1897.483758 0.452630563012444
1899.212295 0.274506903191449
1900.940334 0.230636837425874
1902.665726 0.205634860470716
1904.393607 0.194905875311664
1906.125145 0.125777342856689
1907.852947 0.110853460026038
1909.580229 0.0751821745110982
1911.309202 0.0523902427312033
1913.039182 0.0534836584554684
1914.767786 0.0468410307120514
1916.492335 0.0250704821030212
1918.218525 0.0179723169546249
1919.94537 0.0126781550712591
1921.687047 0.0103906532419307
1923.424296 0.00647661407922633
1925.602025 0.00515565497167974
1927.777058 0.00419943050401319
1929.937437 0.00528616058830989
1932.10021 0.00356924231719373
1934.269856 0.0021050195767773
1936.436515 0.00168785945796043
1938.611329 0.00117694991622965
1940.79023 0.00114137418621747
1942.963462 0.0012595414942975
1945.133707 0.000701007980341736
1947.306059 0.000510373561962083
1949.470402 0.000908121008188929
1951.645411 0.00178312939009155
1953.815741 0.000332659735507866
1955.985272 0.000242016394424124
1958.158576 0.000199956950225066
1960.330906 0.000169630048693226
1962.506606 0.0002105572371367
1964.68185 0.000122838759098158
1966.605568 7.79197040421013e-05
1968.342036 8.79692660001444e-05
1970.074556 0.000132502315850497
1971.805332 7.25339475016501e-05
1973.535714 5.43068058497639e-05
1975.266802 3.14615270801098e-05
1976.996362 2.70563742576924e-05
1978.727238 1.66898627284937e-05
1980.456596 1.64722788963308e-05
1982.189199 1.64316168148198e-05
1984.362486 8.50499946109101e-06
1986.527689 2.57065762104438e-05
1988.701118 8.72205463337251e-06
1990.873575 8.16807893516338e-06
1993.045228 8.4258445422712e-06
1995.214866 4.29879752529914e-06
1997.375534 2.93090947467014e-06
1999.351499 2.34502604490608e-06
2001.515745 1.80445393824927e-06
2003.697579 1.99123899973758e-06
2005.876048 2.22919136150953e-06
2008.047288 1.28194182769298e-06
2010.211511 7.80871393310539e-07
2012.31591 1.30118071933961e-06
2014.468429 7.84109815477666e-07
2016.619627 5.55260509393333e-07
2018.776062 3.55224195546022e-07
2020.707765 2.8117081372453e-07
2022.552615 3.99012994904481e-07
2024.686814 3.85357106086966e-07
2026.565159 2.23205152649638e-07
2028.296411 3.30299610360842e-07
2030.025155 1.22597770804339e-07
2031.750427 9.99988249960541e-08
2033.482027 1.04661614490551e-07
2035.211237 1.42071353790736e-07
2036.973279 6.11181144876138e-08
2038.701056 4.17217084338014e-08
2040.427175 4.52974888977384e-08
2042.156508 3.53898800085139e-08
2043.885484 3.82022082772067e-08
2045.679575 1.9922256886248e-08
2047.408205 1.39429825929878e-08
2049.138404 9.38682032132173e-09
};
\addplot [semithick, color1]
table {%
1575.095645 1
1581.582654 0.415138940640922
1583.599673 0.201786927513696
1585.697131 0.153465543069231
1587.799058 0.138416196785426
1589.937933 0.0948650869994917
1592.057908 0.0588001068933884
1593.921823 0.0409193790337813
1595.712404 0.0270938656936288
1597.489363 0.0184627650968213
1599.268831 0.0141737862276452
1601.044916 0.00970778588435321
1602.822931 0.00804460483706191
1604.601172 0.00654858093199894
1606.381415 0.00535640605312761
1608.179395 0.00389457406828932
1610.120965 0.00305031768163603
1612.303671 0.00272901347237773
1614.510266 0.00192725340983415
1616.492969 0.00133868271419171
1618.542896 0.00122108529963149
1620.579514 0.000883020505791603
1622.4595 0.000624586217139113
1624.240743 0.000481627268158425
1626.019575 0.000357234466519351
1627.799685 0.000256912520271691
1629.578433 0.000204857796809637
1631.368002 0.000138094002682542
1633.146728 0.000100140907137282
1634.9298 7.58271608846032e-05
1636.711288 5.99594594544956e-05
1638.488559 4.18097269430846e-05
1640.267339 5.41397883831579e-05
1642.042139 3.04663280621217e-05
1643.823679 2.13390731639165e-05
1645.602281 1.39255969181318e-05
1647.380922 9.92869882951228e-06
1649.160344 7.23231268488847e-06
1650.943968 5.18832537139365e-06
1652.776797 5.91038234848873e-06
1654.618954 3.92958359441134e-06
1656.417104 2.5189604902865e-06
1658.200082 1.70733758399763e-06
1659.98373 1.31287972177135e-06
1662.112682 1.1205546024035e-06
1664.24488 1.038567921471e-06
1666.05977 6.56269946016138e-07
1668.216229 4.55284074944354e-07
1670.42154 3.03739644204172e-07
1672.436426 2.23766353002694e-07
1674.337045 2.32691913907532e-07
1676.128125 1.58045659834813e-07
1677.908011 1.03401052527595e-07
1679.684536 1.08555934737461e-07
1681.464106 6.72858130184609e-08
1683.242564 8.41847317733126e-08
1685.028091 4.19631870484859e-08
1686.99174 4.32045163999037e-08
1689.06768 3.03694836805597e-08
1691.295143 2.01252993604734e-08
1693.321207 1.77806790591312e-08
1695.308185 1.48806038779074e-08
1697.134838 8.8254013463881e-09
};
%\addplot [semithick, black, dashed]
%table {%
%0 1e-08
%1697.134838 1e-08
%};
\end{axis}

\end{tikzpicture}
    \caption{\label{fig:comparison}Relative residual over multiple solutions of Newton's method i.e. the systems described in Eq.~\eqref{eq:linsys} were solved. Each solver had a relative error of $\epsilon=10^{-8}$ as stopping criterion. Similarly to the results in Fig.~\ref{fig:results}, the time required to find a solution that satisfies the error bound becomes faster for each Newton iteration. The deflated method achieves a faster convergence rate as seen by the slope of the relative residual. This confirms that re-using information from a previous system indeed lowers the effective condition number.}

\end{figure}
\setlength{\figwidth}{.9\textwidth}

\subsection{Comparison to Linear-Cost Approximations}%\label{sub:ind_point}

We now investigate the utility of sub-space recycling relative to the linear cost approximation methods of finite error discussed above in Section~\ref{sub:relation_to_linear_cost_methods}, in particular to inducing point methods. These methods assume that the training set elements $\vec{f}_n$ at $X\in \Re^{n\times d}$ are approximately independent of each other when conditioned on a smaller set of values $\vec{f}_m$ at representer points $X_m\in \Re^{m\times d}$. In GPC, this effectively reduces the task to optimizing only the latent variables $\vec{f}_m\in \Re^{m}$ with $m<n$, to maximize the objective in Eq.~\eqref{eq:obj}. The latent variables for the remaining points in the data set are then \emph{induced} by the conditional distribution $p(\vec{f}_{n-m}\g\vec{f}_m)$, with mean $\Exp [\vec{f}_{n-m}\g\vec{f}_m]=K_{(n-m)m} K^{-1}_{mm}\vec{f}_m$. A measure of the performance over the training set is then obtained by evaluating the objective with the inferred latent variables.  

%A common approach to handle high-dimensional datasets in Gaussian process regression is to use various low-rank approximations by considering a subset of the data \cite{quinonero05}. These methods are referred to as inducing point methods that give rise to inducing variables. As explained in section \ref{sub:ind_point}. Such methods are there to make approximate inference tractable when ... solved. 
We compared the accuracy of iterative methods to inducing point methods with a randomly selected subset of the training data. Figure \ref{fig:convlin} shows the convergence of the Newton optimizer for subsets $X_m$ of $\log p(\vec{y}\g \vec{f})$ of varying sizes. (each time, error was evaluated on the entire training set $X$). The results confirm the expected picture: The approximate methods can be significantly faster than the iterative solves, but they also incur a significant approximation error. If an accurate solution is required, the iterative solvers can be competitive. For this experiment, the iterative methods have a computational cost comparable to that of the approximate methods running on a (comparably large) subset of between $25\%$ and $50\%$ of the data set. But they also achieve an improvement of about 6 orders of magnitude in precision.

\setlength{\figwidth}{.99\textwidth}
\setlength{\figheight}{0.41803398875\figwidth}
\begin{figure}
    \centering \scriptsize
        % This file was created by matplotlib2tikz v0.6.3.
\begin{tikzpicture}


\definecolor{color0}{HTML}{FF9933} %EI orange
\definecolor{color1}{RGB}{000,125,122} % MPG green
\definecolor{color2}{RGB}{130,0,0}  %PN red
\definecolor{color3}{rgb}{0,0,0}
\definecolor{color4}{rgb}{0.35557281,0.35557281,0.35557281}
\definecolor{color5}{rgb}{0.50838015,0.50838015,0.50838015}
\definecolor{color6}{rgb}{0.65,0.65,0.65}
\definecolor{color7}{rgb}{0.81,0.81,0.81}
\definecolor{color8}{rgb}{0.87,0.87,0.87}


%\definecolor{color1}{rgb}{1,0.498039215686275,0.0549019607843137}
%\definecolor{color0}{rgb}{0.12156862745098,0.466666666666667,0.705882352941177}

%\definecolor{color2}{rgb}{0.172549019607843,0.627450980392157,0.172549019607843}



%\definecolor{color8}{rgb}{0.68377223,0.68377223,0.68377223}


%\def\mSize{1}
\newcommand{\mSize}{1}
\tikzset{mark size=0.7}

\begin{axis}[
xlabel={CPU time},
ylabel={rel. $\log\, p(\vec{y} | \vec{f})$},
xmin=-242.75775, xmax=5407.70275,
ymin=6.61209485199062e-07, ymax=32393.1774573419,
ymode=log,
width=\figwidth,
height=\figheight,
tick align=outside,
x grid style={lightgray!92.026143790849673!black},
y grid style={lightgray!92.026143790849673!black},
legend style={draw=white!80.0!black},
legend entries={{CG},{def-CG(8,12)},{100\%},{95\%},{75\%},{50\%},{25\%},{10\%},{5\%}},
legend cell align={left},
mystyle
]
\addplot [semithick, color0, mark=square*, mark size=0.9, mark options={solid}]
table {%
234.234 14.0115848254505
472.974 4.9002182217254
662.053 1.81563625976223
842.217 0.675879540476442
1008.405 0.237160064955664
1179.182 0.0715239263432982
1353.027 0.0149403326334234
1514.169 0.00117208477469438
1687.844 8.35118355262598e-06
};
\addplot [semithick, color1, mark=diamond*, mark size=1.1, mark options={solid}]
table {%
228.019 14.0115848245442
412.137 4.83608200728108
591.732 1.7946146588317
766.293 0.665719763138416
913.287 0.232615263070439
1068.877 0.0695473617426246
1214.731 0.0141987582894002
1359.045 0.00105346002205464
1500.274 2.02419976136714e-06
};
\addplot [semithick, color2, mark=*, mark options={solid}]
table {%
425.113 13.8839801205456
891.749 4.78720826598589
1401.94 1.77685161407272
1918.046 0.659185184066225
2418.924 0.229801054396592
2912.795 0.0683907611897461
3404.559 0.0138370670251816
3879.709 0.00100001510596842
4382.189 6.04238734726975e-06
4864.5 0
};
\addplot [semithick, color3, mark=*, mark options={solid}]
table {%
389.218 2464.93800208462
789.425 917.61264067433
1185.997 374.927944530884
1589.675 152.579691536126
2003.268 51.206969893805
2408.886 11.1672653665463
2810.647 0.816643755947975
3221.507 0.0794694783909122
3634.548 0.0761975256423813
4057.781 0.0762277375791175
};
\addplot [semithick, color4, mark=*, mark options={solid}]
table {%
174.27 10581.305560507
368.783 3939.59855284823
592.393 1595.68232752761
767.575 625.125143884349
943.809 195.497089079895
1115.593 38.1031616791794
1298.459 2.35554917748002
1478.111 0.509282617562199
1671.194 0.501107267481382
1848.437 0.501128415837097
};
\addplot [semithick, color5, mark=*, mark options={solid}]
table {%
54.124 10257.2859076421
109.123 3856.45010045469
164.144 1588.19623257149
219.005 628.721297904802
273.996 192.882309400444
338.003 32.1224308524298
395.248 2.14376350095923
450.256 1.28404356561277
505.258 1.28316741944742
560.291 1.28316741944742
};
\addplot [semithick, color6, mark=*, mark options={solid}]
table {%
11.118 6991.35015936797
22.863 2656.30398948625
34.562 1092.69346062629
46.299 417.827746642699
56.835 117.635631958187
68.639 11.3683952929803
80.395 2.81854710796236
90.92 2.74849469025212
101.537 2.74839499086089
};
\addplot [semithick, color7, mark=*, mark options={solid}]
table {%
1.695 3129.07345428179
3.986 1157.92056073355
6.263 423.819701203946
8.538 118.144322421789
10.813 19.094992371486
13.085 5.72625266242693
15.357 5.54286922763184
17.634 5.54517137721114
};
\addplot [semithick, color8, mark=*, mark options={solid}]
table {%
0.445 3454.21567395278
1.025 1227.06122751099
1.53 441.22038701491
2.046 112.616519887007
2.547 14.1225094034653
3.074 8.22942944757474
3.579 8.17629571443677
4.09 8.17619601504554
};%
\end{axis}
\end{tikzpicture}
   \caption{\label{fig:convlin}Comparison of accuracy between the iterative methods CG, def-CG and differently sized subsets of data measured as the relative error of $\log p(\vec{y}\g \vec{f})$ to the ``exact'' (up to machine precision) value achieved by a direct (Cholesky) solver on the full data set. Each dot represents the approximated $\log p(\vec{y}\g \vec{f})$ for the full training set after each iteration of Newton's method. The CPU time refers to the cumulative time spent solving the linear systems in Eq.~\eqref{eq:linsys}, which is the computationally most expensive part of each Newton iteration.}
\end{figure}
\setlength{\figwidth}{.9\textwidth}


% \subsection{outlook}

% \begin{itemize}
% \item approximate $\log |K|$
% \item improve $W$
% \end{itemize}
% \cite{gosselet13} looked for converged Ritz vectors to dynamically change $W$

\section{Conclusion} 
\label{sec:conclusion}

We have investigated the use of Krylov sub-space recycling methods for a realistic example application in machine learning. ML problems often involve outer loops of (hyper-) parameter optimization, which produce a sequence of interrelated linear, symmetric positive definite (aka.~least-squares) optimization problems. Subspace-recycling methods allow for iterative linear solvers to share information across this sequence, so that later instances become progressively cheaper to solve. Our experiments suggests that doing so can lead to a useful reduction in computational cost. While non-trivial, sub-space recycling methods can be implemented and used with moderate coding overhead. 

We also compared empirically to the popular option of fixing a low-rank sub-space a priori, in the form of spectral or inducing point methods. The overarching intuition here is that these methods can achieve much lower computational cost when they use a low-dimensional basis. But in exchange they also incur a significant computational error. In applications where computational \emph{precision} is at least as important as computational \emph{cost}, sub-space recycling iterative solvers provide a reliable answer of high quality. While their run-time scales quadratically with data-set size, they are certainly scalable, at acceptable run-times, to data-sets containing $\sim 10^5$ to $\sim 10^6$ data points.

% \newpage
\small
\bibliographystyle{abbrvnat}
\bibliography{bibfile}


\end{document}
