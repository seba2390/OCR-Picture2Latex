\usepackage{microtype,marvosym}


\usepackage{tikz,pgfplots}
\pgfplotsset{compat=newest}
\pgfplotsset{plot coordinates/math parser=false}
%\usepackage{pgfplots}
%\usepgfplotslibrary{plotmarks}

% tikz stuff
\usetikzlibrary{arrows,shapes,plotmarks,pgfplots.colormaps}
\tikzset{>=stealth'} 
\tikzstyle{graphnode} = 
   [circle,draw=black,minimum size=22pt,text centered,text
     width=22pt,inner sep=0pt] 
\tikzstyle{var}   =[graphnode,fill=white]
\tikzstyle{obs}   =[graphnode,fill=black,text=white]
\tikzstyle{fac}   =[rectangle,draw=black,fill=black!25,minimum size=5pt]
\tikzstyle{facprior} =[rectangle,draw=black,fill=black,text=white,minimum size=5pt]
\tikzstyle{edge}  =[draw=white,double=black,thick,-]
\tikzstyle{prior} =[rectangle, draw=black, fill=black, minimum size=
5pt, inner sep=0pt]
\tikzstyle{dirprior} = [circle, draw=black, fill=black, minimum
size=5pt, inner sep=0pt]

% pgfplots stuff
\pgfplotsset{compat=newest}
\pgfplotsset{
  every axis legend/.append style =
    {
      cells = { anchor = east },
      draw  = none
    },
}  
\pgfkeys{/pgfplots/mystyle/.style={
  semithick,
  tick style={major tick length=4pt,semithick,gray},
  xtick align = inside,
  ytick align = inside
  }}

\newcommand{\g}{\,|\,} 
\newcommand{\de}{\partial}
\renewcommand{\d}{\:d} 
\newcommand{\eps}{\epsilon}
\newcommand{\dd}{\mathrm{d}}
\newcommand{\ddd}{\mathsf{d}}
\newcommand{\DD}{\mathsf{D}}
\newcommand{\Exp}{\mathbb{E}}
\newcommand{\Ent}{\mathbb{H}}
\newcommand{\Cov}{\operatorname{cov}} 
\newcommand{\cov}{\operatorname{cov}} 
\newcommand{\erf}{\operatorname{erf}} 
\renewcommand{\H}{\mathcal{H}}
\newcommand{\K}{\mathcal{K}} 
\newcommand{\KL}{\text{KL}} 
\renewcommand{\Re}{\mathbb{R}}
\newcommand{\Co}{\mathbb{C}}
\newcommand{\one}{\mathbf{1}}
\newcommand{\const}{\text{const.}}
\newcommand{\diag}{\operatorname{diag}}
\newcommand{\Dcal}{\mathcal{D}} 
\newcommand{\N}{\mathcal{N}} 
\newcommand{\W}{\mathcal{W}} 
\newcommand{\cS}{\mathcal{S}} 
\newcommand{\sT}{\mathsf{T}} 
\newcommand{\B}{\mathcal{B}} 
\newcommand{\C}{\mathcal{C}}
\renewcommand{\L}{\mathcal{L}}
\newcommand{\Trans}{^{\intercal}} 
\newcommand{\argmin}{\operatorname*{arg\:min}}
\newcommand{\argmax}{\operatorname*{arg\:max}}
\renewcommand{\det}{\operatorname{det}}
\newcommand{\var}{\operatorname{var}}
\renewcommand{\=}{\operatorname*{=}}
\newcommand{\myexp}[1]{\exp{\left[ #1 \right] }}
\newcommand{\expmap}{\mathrm{Exp}}
\newcommand{\logmap}{\mathrm{Log}}

\newcommand{\q}{\quad}
\newcommand{\qq}{\qquad}
\newcommand{\qqq}{\quad\qquad}
\newcommand{\qqqq}{\qquad\qquad}

\renewcommand{\vec}{\boldsymbol} 
\newcommand{\mat}{\boldsymbol} 
\newcommand{\vect}[1]{\overrightarrow{#1}}
\newcommand{\vest}[1]{\overrightharpoon{#1}}
\newcommand{\fun}[1]{\mathsf{#1}}
\newcommand{\logm}{\operatorname{Log}}
\newcommand{\expm}{\operatorname{Exp}}
\renewcommand{\O}{\mathcal{O}} 
\newcommand{\G}{\mathcal{G}} 
\newcommand{\GP}{\mathcal{GP}}
\newcommand{\Id}{\vec{I}}
\newcommand{\zero}{\vec{0}}
\newcommand{\tr}{\operatorname{tr}}
\newcommand{\rk}{\operatorname{rk}}
\newcommand{\II}{\mathbb{I}}
\renewcommand{\L}{\mathcal{L}}

\newcommand{\bm}{\boldsymbol{m}}
\newcommand{\bmu}{\boldsymbol{\mu}}
\newcommand{\bSigma}{\boldsymbol{\Sigma}}
\newcommand{\bPhi}{\boldsymbol{\Phi}}
\newcommand{\bphi}{\boldsymbol{\phi}}
\newcommand{\balpha}{\boldsymbol{\alpha}}
\newcommand{\bbeta}{\boldsymbol{\beta}}

\newcommand{\w}{\vec{w}}
\newcommand{\f}{\vec{f}}
\newcommand{\y}{\vec{y}}
\newcommand{\x}{\vec{x}}
\newcommand{\V}{\mathbb{V}}
\newcommand{\X}{\mathbb{X}}

\newcommand{\fS}{\mathsf{S}}
\newcommand{\fW}{\mathsf{W}}
\newcommand{\fB}{\mathsf{B}}
\newcommand{\fK}{\mathsf{K}}
\newcommand{\fH}{\mathsf{H}}
\newcommand{\fk}{\mathsf{k}}
\newcommand{\fh}{\mathsf{h}}
\newcommand{\fL}{\mathsf{L}}

\newcommand{\sA}{\boldsymbol{\mathsf{A}}}
\newcommand{\sB}{\boldsymbol{\mathsf{B}}}
\newcommand{\sC}{\boldsymbol{\mathsf{C}}}

\usepackage{colonequals}
\newcommand{\ce}{\colonequals}
\newcommand{\ec}{\equalscolon}

% to avoid warnings, copy only two symbols from stmaryrd
\DeclareSymbolFont{stmry}{U}{stmry}{m}{n}
\DeclareMathSymbol\leftarrowtriangle\mathrel{stmry}{"5E}
\DeclareMathSymbol\rightarrowtriangle\mathrel{stmry}{"5F}
\DeclareMathSymbol\leftrightarrowtriangle\mathrel{stmry}{"5D}
\DeclareMathSymbol\obar\mathrel{stmry}{"3A}
\DeclareMathSymbol\otimes\mathrel{stmry}{"0F}
\DeclareMathSymbol\ominus\mathrel{stmry}{"17}
\DeclareMathSymbol\sslash\mathrel{stmry}{"0C}
\renewcommand{\gets}{\operatorname*{\leftarrowtriangle}}
\renewcommand{\to}{\operatorname*{\rightarrowtriangle}}
\renewcommand{\leftrightarrow}{\operatorname*{\leftrightarrowtriangle}}

\usepackage{mathtools}
\usepackage{algorithm}
\usepackage{algpseudocode}
\algrenewcommand{\algorithmiccomment}[1]{\hfill {\footnotesize $\sslash$ #1}}
\algrenewcommand{\alglinenumber}[1]{\tt\scriptsize #1}

%%% counting algorithms for each chapter
\makeatletter
\@addtoreset{algorithm}{chapter} % algorithm counter resets every chapter
\makeatother
%\renewcommand{\thealgorithm}{\thechapter.\arabic{algorithm}}% Algorithm # is <chapter>.<algorithm

% %%%%%% adding vertical bars to algorithmicx. Based on a response at 
% % http://tex.stackexchange.com/questions/52473/is-it-possible-to-have-connecting-loop-lines-like-algorithm2e-in-algorithmic/52778#52778
\makeatletter
% This is the vertical rule that is inserted
% \def\therule{\makebox[\algorithmicindent][l]{\hspace*{.5em}\vrule height .75\baselineskip depth .25\baselineskip}}%
\def\therule{\makebox[\algorithmicindent][l]{\hspace*{.5em}\vrule height .75\baselineskip depth .25\baselineskip}}%

\newtoks\therules% Contains rules
\therules={}% Start with empty token list
\def\appendto#1#2{\expandafter#1\expandafter{\the#1#2}}% Append to token list
\def\gobblefirst#1{% Remove (first) from token list
  #1\expandafter\expandafter\expandafter{\expandafter\@gobble\the#1}}%
\def\LState{\State\unskip\the\therules}% New line-state
\def\pushindent{\appendto\therules\therule}%
\def\popindent{\gobblefirst\therules}%
\def\printindent{\unskip\the\therules}%
\def\printandpush{\printindent\pushindent}%
\def\popandprint{\popindent\printindent}%

%      ***      DECLARED LOOPS      ***
% (from algpseudocode.sty)
\algdef{SE}[WHILE]{While}{EndWhile}[1]
  {\printandpush\algorithmicwhile\ #1\ \algorithmicdo}
  {\popandprint\algorithmicend\ \algorithmicwhile}%
\algdef{SE}[FOR]{For}{EndFor}[1]
  {\printandpush\algorithmicfor\ #1\ \algorithmicdo}
  {\popandprint\algorithmicend\ \algorithmicfor}%
\algdef{S}[FOR]{ForAll}[1]
  {\printindent\algorithmicforall\ #1\ \algorithmicdo}%
\algdef{SE}[LOOP]{Loop}{EndLoop}
  {\printandpush\algorithmicloop}
  {\popandprint\algorithmicend\ \algorithmicloop}%
\algdef{SE}[REPEAT]{Repeat}{Until}
  {\printandpush\algorithmicrepeat}[1]
  {\popandprint\algorithmicuntil\ #1}%
\algdef{SE}[IF]{If}{EndIf}[1]
  {\printandpush\algorithmicif\ #1\ \algorithmicthen}
  {\popandprint\algorithmicend\ \algorithmicif}%
\algdef{C}[IF]{IF}{ElsIf}[1]
  {\popandprint\pushindent\algorithmicelse\ \algorithmicif\ #1\ \algorithmicthen}%
\algdef{Ce}[ELSE]{IF}{Else}{EndIf}
  {\popandprint\pushindent\algorithmicelse}%
\algdef{SE}[PROCEDURE]{Procedure}{EndProcedure}[2]
   {\printandpush\algorithmicprocedure\ \textproc{#1}\ifthenelse{\equal{#2}{}}{}{(#2)}}%
   {\popandprint\algorithmicend\ \algorithmicprocedure}%
\algdef{SE}[FUNCTION]{Function}{EndFunction}[2]
   {\printandpush\algorithmicfunction\ \textproc{#1}\ifthenelse{\equal{#2}{}}{}{(#2)}}%
   {\popandprint\algorithmicend\ \algorithmicfunction}%
\makeatother
