\section{Analysis}
\label{sec:analysis}

In this section, we analyze the coding scheme proposed in Section~\ref{subsec:instantly_decodable}, and the chaining algorithm of Section~\ref{subsubsec:chaining_algorithm}.
%In particular, we show that the solution of a linear programming problem characterizes the number of instantly decodable, distortion-innovative symbols that can be sent by Algorithm~\ref{alg:three_users}.  
For the analysis of the coding scheme in Section~\ref{subsec:instantly_decodable}, we show in Section~\ref{subsec:analysis_instantly_decodable} that the solution of a linear program characterizes the number of instantly decodable, distortion-innovative symbols that can be sent by the algorithm in Section~\ref{subsec:instantly_decodable}.
We furthermore give sufficient conditions for which all users can be point-to-point optimal.  In some cases, we find that users can be point-to-point optimal regardless of their distortion constraints.  Following this, in Section~\ref{subsec:chaining_algorithm_analysis} we analyze the chaining algorithm and give a sufficient condition for point-to-point optimal performance.

\subsection{Analysis of Coding Scheme of Section~\ref{subsec:instantly_decodable}}
\label{subsec:analysis_instantly_decodable}

Our analysis begins by first considering the systematic phase of our coding, which transmits the $N$ source symbols so that each symbol is recovered by at least one user.  Let $T_0$ be a random variable representing the number of transmissions required for this after being normalized by $N$.  In other words, $T_0$ will be said to be the \emph{latency} required.  Then it is clear that $T_0$ has expected value $\bar{T_0}$ given by
%
\begin{equation}
\label{eq:T0}
	\bar{T}_{0} = \frac{1}{1 - \epsilon_1\epsilon_2\epsilon_3}.
\end{equation}

During the systematic transmissions, we direct erased symbols to their appropriate queues as outlined in the algorithm of Section~\ref{subsec:instantly_decodable}.  Then, when the $NT_0$ transmissions are completed, we start sending linear combinations of the form $q_i \oplus q_{j,k}$, where $q_i \in Q_i, q_{j,k} \in \Q{j,k}$.
and $i,j,k\in \mathcal{U}$  are unique.

Let $T_i$ be a random variable representing the normalized number of transmissions we can send of the form $q_i \oplus q_{j,k}$.  We are able to do this so long as $Q_i$ and \Q{j,k} are non-empty.  Thus, we must bound the cardinality of these queues.  For \Q{j,k}, we have that a symbol is added during the systematic transmissions whenever user~$i$ receives a symbol that is erased at users~$j$ and~$k$.  Symbols are no longer added to this queue after the systematic transmissions, and so we have that $\Q{j,k}^{+}$, the  expected maximum normalized cardinality of \Q{j,k}, is given by
%
\begin{equation}
\label{eq:Qjk}
	\Q{j,k}^{+} = \bar{T}_0(1 - \epsilon_i)\epsilon_j\epsilon_k.
\end{equation}
%
Now, a symbol is removed from \Q{j,k} whenever user~$j$ or user~$k$ receives a linear combination involving one of its elements.  Thus,  \bT{i}, the expected value of $T_i$, is bounded as
%
\begin{equation}
\label{eq:Ti_Qjk}
	\bar{T}_i \leq \frac{\Q{j,k}^{+}}{1 - \epsilon_j\epsilon_k}.
\end{equation}

Inequality~\eqref{eq:Ti_Qjk} bounds \bT{i} relative to the size of \Q{j,k}.  We must also bound \bT{i} in terms of the cardinality of $Q_i$, since transmissions must stop if $Q_i$ is empty.  This bounding is not as straightforward, however, as symbols are added and removed from $Q_i$ as the algorithm progresses.  For example, $Q_1$ may be empty at a certain time, but later replenished when we send linear combinations of the form $q_2 \oplus q_{1,3}$.  In particular, when this linear combination is received by user~3 and not user~1, we add $q_{1,3}$ to $Q_1$ as user~1 is now the only user in need of it. If $N\bar{T}_2$ linear combinations of this form are sent, we can expect that $N\bar{T}_2\epsilon_1(1 - \epsilon_3)$ symbols are added to $Q_1$.  In general, after $N\bT{j}$ and $N\bT{k}$ linear combinations are sent from $Q_j$, \Q{i,k} and $Q_k$, \Q{i,j} respectively, we can expect that $N\bar{T}_j\epsilon_i(1 - \epsilon_k) + N\bar{T}_k\epsilon_i(1 - \epsilon_j)$ symbols are added to $Q_i$.  Given that $Q_i$ initially has $N\bar{T}_0\epsilon_i(1 - \epsilon_j)(1 - \epsilon_k)$ symbols after $N\bar{T}_0$ uncoded transmissions, we have then that the expected maximum normalized cardinality of $Q_i$ is given by \Qp{i}{\bT{j}}{\bT{k}}  where
%
\begin{equation}
\label{eq:Qi_plus}
	\Qp{i}{\bT{j}}{\bT{k}} = \bar{T}_0\epsilon_i(1 - \epsilon_j)(1 - \epsilon_k) + \bT{j}\epsilon_i(1 - \epsilon_k) + \bT{k}\epsilon_i(1 - \epsilon_j).
\end{equation}
%
Since a symbol is removed from $Q_i$ each time user~$i$ successfully receives a channel symbol, we can also write that
%
\begin{equation}
\label{eq:Ti_Qi}
	\bT{i} \leq \frac{\Qp{i}{\bT{j}}{\bT{k}} }{1 - \epsilon_i}.
\end{equation}

Notice that by the definition of our stopping condition, \bT{i} must actually meet \eqref{eq:Ti_Qjk} or \eqref{eq:Ti_Qi} with equality, since otherwise $Q_i$ and \Q{j,k} would still be non-empty, and we would not have reached the stopping condition.  So we in fact must have that for all $i \in \mathcal{U}$ with $j,k \in \mathcal{U}\setminus \{i\} \ s.t. \ j \neq k$,
%
\begin{equation}
\label{eq:Ti_min}
	\bT{i} = \min\left(\frac{\Qp{i}{\bT{j}}{\bT{k}} }{1 - \epsilon_i}, \frac{\Q{j,k}^{+}}{1 - \epsilon_j\epsilon_k} \right),
\end{equation}
where $\bT{i} > 0$.
%\color{red}
The following theorem proposes that the solution to \eqref{eq:Ti_min} is unique and can be characterized by solving the linear program 
%We show the existence of a solution to \eqref{eq:Ti_min} by considering the linear program

\begin{equation}
\label{eq:LP}
\begin{aligned}
	& \underset{\bT{1}, \bT{2}, \bT{3}}{\text{max}}
	&& \!\! \bT{1} + \bT{2} + \bT{3}\\
	& \text{subject to}
%\end{aligned}
%\end{equation}
%\[
%\begin{aligned}
	&& 	\!\! \bT{i} \leq \frac{\Qp{i}{\bT{j}}{\bT{k}} }{1 - \epsilon_i},\\
	&&& \!\!\bar{T}_i \leq \frac{\Q{j,k}^{+}}{1 - \epsilon_j\epsilon_k} \quad \forall i \in \mathcal{U}, j,k \in \mathcal{U}\setminus \{i\}, j \neq k.
\end{aligned}
%\]
\end{equation}
%
\begin{theorem}
\label{thm:LP_instantly_decodable}
	Let $\bT{i}$ be the expected value of the normalized number of analog transmissions that can be sent of the form $q_i \oplus q_{j,k}$, where $i \in \mathcal{U}, j,k \in \mathcal{U}\setminus\{i\}, j\neq k$.  Then $\bT{i}$ is uniquely given by the solution of~\eqref{eq:LP}.
\end{theorem}
\begin{proof}
We proceed toward this end by establishing several lemmas in the appendix in the accompanying supplemental material of this paper and which is also available in the extended version of this paper~\cite{TMK_feedback3_arxiv}.
%that collectively show that the only $(\bT{1}, \bT{2}, \bT{3})$ satisfying~\eqref{eq:Ti_min} is the unique optimal solution of~\eqref{eq:LP}.  
Lemma~\ifarxiv\ref{lem:LPToMin}\else\lemOPSatisfiesCond{}\fi~\cite{TMK_feedback3_arxiv} first establishes that any optimal solution to~\eqref{eq:LP} also satisfies~\eqref{eq:Ti_min}.  As $\bT{1} = \bT{2} = \bT{3} = 0$ is clearly a feasible solution of~\eqref{eq:LP}, we have that the feasible set of~\eqref{eq:LP} is non-empty.  Therefore, such a solution of~\eqref{eq:LP}, and by extension~\eqref{eq:Ti_min}, indeed exists.  
%
Conversely, Lemma~\ifarxiv\ref{lem:MinToLP}\else\lemCondSatisfiesOP{}\fi~\cite{TMK_feedback3_arxiv} establishes that any $(\bT{1}, \bT{2}, \bT{3})$ that satisfies~\eqref{eq:Ti_min} is also an optimal solution to~\eqref{eq:LP}.  
%Furthermore, we know that 
%Such a solution to~\eqref{eq:LP} exists since %the feasible set of~\eqref{eq:LP} 
%we use the fact that a solution to~\eqref{eq:LP} must exist since 
%
%We first notice that 
%the feasible set of this linear program is non-empty as $\bT{1} = \bT{2} = \bT{3} = 0$ is clearly feasible.  
%As $\bT{1} = \bT{2} = \bT{3} = 0$ is clearly a feasible solution of~\eqref{eq:LP}, we have that the feasible set of~\eqref{eq:LP} is non-empty.  Therefore, such a solution of~\eqref{eq:LP}, and by extension~\eqref{eq:Ti_min}, indeed exists.  
Finally, Lemma~\ifarxiv\ref{lem:unique}\else\lemOPunique{}\fi~\cite{TMK_feedback3_arxiv} then shows that the optimal solution for~\eqref{eq:LP} is unique.  We conclude that any $(\bT{1}, \bT{2}, \bT{3})$ satisfying~\eqref{eq:Ti_min} is itself unique and characterized by~\eqref{eq:LP}.
%Combining Lemmas~\ref{lem:LPToMin}, \ref{lem:MinToLP} and~\ref{lem:unique}, we therefore conclude that the solution to~\eqref{eq:LP} uniquely characterizes the number of 
%instantly-decodable, distortion-innovative transmissions that can be sent, which is suggested by~\eqref{eq:Ti_min}.
%Combining Lemmas~\ref{lem:LPToMin}, \ref{lem:MinToLP} and~\ref{lem:unique}, we therefore conclude that the only solution to~\eqref{eq:Ti_min} is the unique optimal solution of~\eqref{eq:LP}.  
\end{proof}
%\color{black}
Now, let $\bar{T}^{*} = (\bTs{1}, \bTs{2}, \bTs{3})$ be the optimal solution of~\eqref{eq:LP}, %for some scheduler, 
and let %let $t^{*}$ be given as
\begin{equation}
\label{eq:t_star}
	t^{*} = \bT{0} + \bTs{1} + \bTs{2} + \bTs{3},
\end{equation}  
%
where $\bar{T}_{0}$ is given by~\eqref{eq:T0}.
It is important to determine $t^{*}$ since it provides a lower bound for the number of instantly decodable, distortion-innovative transmissions possible.  From the discussion at the beginning of Section~\ref{sec:three_users}, we know that for any latency $w \leq t^{*}$, we can meet the point-to-point outer bound at $w$.  Say that user~$i$'s distortion, $d_i$, necessitates a minimum latency of $w_i(d_i)$ where

\begin{equation}
\label{eq:wioptimal}
	w_i(d_i) = \frac{1 - d_i}{1 - \epsilon_i}.
\end{equation}
Let
%
%\[
%w^{-}\dvec = \min_{i \in \mathcal{U}}  w_i (d_i)\quad
%w^{+}\dvec = \max_{i \in \mathcal{U}} w_i(d_i)
%\]
\begin{eqnarray}
\label{eq:w_minus}
%	w^{-} &=& \min_i  \frac{1 - d_i}{1 - \epsilon_i} \\
	w^{-}\dvec &=& \min_{i \in \mathcal{U}}  w_i (d_i)\\
\label{eq:w_plus}
	w^{+}\dvec &=& \max_{i \in \mathcal{U}} w_i(d_i).
%	w^{+} &=& \max_i \frac{1 - d_i}{1 - \epsilon_i}.
%		w^{-} &=& \min_i  \frac{(1 - d_i)}{(1 - \epsilon_i)}\\
%		w^{+} &=& \max_i \frac{(1 - d_i)}{(1 - \epsilon_i)}\\
\end{eqnarray}
%
It is clear then that $w^{+}$ is an outer bound for our problem, and if $w^{+} \leq t^{*}$, we are optimal for all users.  Notice however, that if $w^{-} \leq t^{*}$, we can also be optimal for all users.  This is because if $w^{-} \leq t^{*}$, one user can be fully satisfied at latency $w^{-}$, and so from thereon, we are left with only two users.  From~\cite{TMKS_TIT20} and the discussion in Section~\ref{subsec:instantly_decodable}, we know that we can remain optimal for the remaining two users, which leads us to the following theorem.

\begin{theorem}
\label{thm:w_minus}
	Given $\dvec \in \mathcal{D}^{3}$, let $t^{*}, w^{-}\dvec$ and $w^{+}\dvec$ be given by~\eqref{eq:t_star}, \eqref{eq:w_minus} and~\eqref{eq:w_plus} respectively.  Then if $w^{-}\dvec \leq t^{*}$, the latency $w^{+}\dvec$ is $(d_1, d_2, d_3)$-achievable.
%	\begin{eqnarray}
%		w^{-} &=& \min_i  \frac{1 - d_i}{1 - \epsilon_i} \\
%		w^{+} &=& \max_i \frac{1 - d_i}{1 - \epsilon_i}
%%		w^{-} &=& \min_i  \frac{(1 - d_i)}{(1 - \epsilon_i)}\\
%%		w^{+} &=& \max_i \frac{(1 - d_i)}{(1 - \epsilon_i)}\\
%	\end{eqnarray}
\end{theorem}
%
%Theorem~\ref{thm:w_minus} states that having $w^{-} \leq t^{*}$ is sufficient for point-to-point optimality.  
If $w^{-} > t^{*}$,  it may still be possible to achieve optimality.  In particular, this may happen if after $Nt^{*}$ transmissions have completed, we are left with non-empty queues $Q_i, i\in \mathcal{U}$. %$Q_1$, $Q_2$ and $Q_3$.  
%For $i \in \mathcal{U}$, 
We can calculate the expected normalized cardinality of $Q_i$ after $t^{*}$ transmissions, denoted as $| Q_i(t^{*}) |$, by noting that in $NT_i^{*}$ transmissions, a symbol is removed from $Q_i$ with probability $(1 - \epsilon_i)$, and so
%
\begin{equation}
\label{eq:Q_t_star}
	| Q_i(t^{*}) | = \Qp{i}{\bT{j}^{*}}{\bT{k}^{*}} - \bar{T}_i^{*}(1 - \epsilon_i),
%	| Q_i(t^{*}) | = \frac{\Qp{i}{\bT{j}^{*}}{\bT{k}^{*}} - (t^{*} - \bar{T}_0)(1 - \epsilon_i)}{N},
\end{equation}
%
where \Qp{i}{\bT{i}}{\bT{j}} is given by~\eqref{eq:Qi_plus}.
If $| Q_i(t^{*}) |$ is non-zero for all $i \in \mathcal{U}$, we can continue sending linear combinations of the form $q_1 \oplus q_2 \oplus q_3$ until a user's distortion constraint is met or one of the queues $Q_i$ is exhausted.  If the latter were to happen, we have that user~$i$ has actually reconstructed every source symbol since all queues $Q_U, U \subset \mathcal{U} \ s.t.\ i \in U$ are empty.  Therefore, we are again left with the situation in~\cite{TMKS_TIT20} involving only two users. We conclude that we can send instantly decodable, distortion-innovative symbols until \emph{all} users achieve lossless reconstructions.

\begin{theorem}
\label{thm:all}
	Let $Q^{-} = \min_{i \in \mathcal{U}} |Q_i(t^{*})|$, where $|Q_i(t^{*})|$ is given by~\eqref{eq:Q_t_star}.  If $Q^{-} > 0$, then for \textbf{any} $\dvec \in \mathcal{D}^{3}$, the latency $w^{+}\dvec$ is $(d_1, d_2, d_3)$-achievable.
\end{theorem}

\subsection{Analysis of Chaining Algorithm of Section~\ref{subsubsec:chaining_algorithm}}
\label{subsec:chaining_algorithm_analysis}

In this section, we give a sufficient condition for all users to simultaneously achieve point-to-point optimal performance when the chaining algorithm is invoked.  Our analysis is based on a restricted version of the chaining algorithm previously described.  Specifically, we assume that we do not opportunistically send linear combinations of the forms specified in~\Crefrange{eq:linear_comb_available_a}{eq:linear_comb_available_c} when the opportunities present themselves (see the description of State~4 in Section~\ref{subsubsec:chaining_algorithm}). Thus, we can argue a fortiori that the actual unrestricted chaining algorithm would achieve only better performance.  Before beginning the analysis, we mention that we will rely on many of the results on Markov rewards processes with impulse rewards and absorbing states derived in~\cite{TMK_markov}.  We will cite the specific theorems and corollaries we use from~\cite{TMK_markov} when applicable. %Section~\ref{sec:markov}.  We encourage the reader to become familiarized with this section before reading further.  

We begin by deriving the transition matrix for the Markov rewards process.  Consider Table~\ref{tab:state1_transitions_all}.  The table shows all outbound transitions given the channel noise realization, however to construct a transition matrix from this information, we must combine all outbound transitions to the same state.  For example, rows 1 and 5 both show transitions from State~1 to State~5, and so to get $\trans{1}{5}$, the total probability of transitioning from State~1 to~5, we must add the corresponding probabilities under the $\Prob(\zall)$ columns.  This is given by 

\begin{align}
	\trans{1}{5} &= \Prob(\zall = (0, 0, 0)) + \Prob(\zall = (1, 0, 0)) \\
	&= (1 -\epsilon_i)(1 -\epsilon_j)(1 - \epsilon_k) + \epsilon_i(1 -\epsilon_j)(1 - \epsilon_k) \\
	&= (1 -\epsilon_j)(1 - \epsilon_k).
\end{align}
We continue in this manner to find $\trans{i}{j}$ for all $i, j \in \myState \triangleq \{1, 2, \ldots, 6\}$ to populate the transition matrix $\transM$ where the $(i, j)$th entry of $\transM$ is given by $\trans{i}{j}$.

Our analysis also requires the derivation of several rewards matrices.  Consider deriving the rewards matrix $\orhoe$, whose $(i,j)$th element,  $\rholcarg{i}{j}$, gives the expected number of equations (rewards) received by user~$i$ for transitioning from State~$i$ to State~$j$, where $i, j \in \myState$.  Continuing with our previous example, say we would like to derive $\rholcarg{1}{5}$.  Of the two possible paths for an outbound transition from State~1 to~5, only one, when $\zall = (0, 0, 0)$, is associated with a reward in the column of $\rhoc$ in Table~\ref{tab:state1_transitions_all}.  We therefore calculate $\rholcarg{1}{5}$ by weighting the reward with the conditional probability of the transition resulting from the channel noise $\zall = (0, 0, 0)$ given that the transition from State~1 to~5 occurred.  Therefore,

\begin{align}
	\rholcarg{1}{5} &= \Prob(\zall = (0, 0, 0) | \textrm{transition from State~1 to~5}) \cdot 1 \\
	&= \frac{(1 -\epsilon_i)(1 -\epsilon_j)(1 - \epsilon_k)}{(1 -\epsilon_i)(1 -\epsilon_j)(1 - \epsilon_k) + \epsilon_i(1 -\epsilon_j)(1 - \epsilon_k)} \\
	&= (1 -\epsilon_i).
\end{align}
By performing this calculation for all $i, j \in \myState$, we are able to populate the entire rewards matrix $\orhoe$.  Similarly, we can calculate the $|\myState| \times |\myState|$ rewards matrices, $\orhoj$, $\orhok$, $\orhoqi$, $\orhoqj$, $\orhoqk$, and $\orhoqstar$ corresponding to each rewards column in Table~\ref{tab:state1_transitions_all}.

Given the transition matrix, for each reward matrix, we can use the results in \cite[Corollary~\markovcor{}]{TMK_markov} to calculate the expected accumulated reward before absorption \emph{each time} the Markov rewards process is reset and allowed to run until absorption.  To find the \emph{total} expected reward, we must first find a lower bound for $\Mstar$, the number of times the Markov process is \emph{reset} before user~$j$ or~$k$ has met their distortion constraint.  That is, $\Mstar$ is the number of times the Markov rewards process has reached an absorbing state after having been restarted in the initial state.  Therefore, $\Mstar$ also represents the number of chains built by user~$i$ before user~$j$ or~$k$ has satisfied their distortion constraint.

Say that for $r \in \{i, j, k\}$, user~$r$ requires $N(1 - \hat{d}_r)$ symbols at the beginning of the chaining algorithm, where $\hat{d}_r \in (0, \epsilon_r)$.  Furthermore, suppose that of the two users for whom we are targeting point-to-point optimal performance, user~$u$ is \emph{not} the bottleneck user, i.e., 

\begin{align}
\label{eq:u_bottleneck}
	u = \argmin_{r \in \{j, k\}} w_r(\hat{d}_r),
\end{align}
where $w_r(\hat{d}_r)$ is given by~\eqref{eq:wioptimal}.  For $l = 1, 2, \ldots $, let $\oRhojl$ be the expected accumulated reward for user~$u$ during the $l$th time the Markov rewards process has been reset.  Then we can define $\Mstar$ as

\begin{align}
	\Mstar = \min\left\{ m : \sum_{l = 1}^{m} \oRhojl \geq N(1 - \hat{d}_{u}) \right\},
\end{align}
where the $\oRhojl$ are i.i.d.\ and $\mathbb{E}\oRhojl$ can be calculated as in \cite[Corollary~\markovcor{}]{TMK_markov}.

We see that $\Mstar$ is a stopping rule~\cite{Gallager96}.  In order to calculate $\mathbb{E}\Mstar$, we could use the discrete version of the renewal equation~\cite[Chapter~2]{MitovOmey14}.  However, to find a lower bound for $\mathbb{E}\Mstar$, we may simply use Wald's equation\cite{Gallager96}.  Let $\sigma_m = \sum_{l = 1}^{m} \oRhojl$.  Then, by Wald's equation, 

\begin{align}
	\mathbb{E}\Mstar &= \frac{\mathbb{E}\sigma_{\Mstar}}{\mathbb{E}\oRhojl} \\
	\label{eq:EM_lb}
	&\geq \frac{N(1 - \hat{d}_u)}{\mathbb{E}\oRhojl},
\end{align}
where again, $\mathbb{E}\oRhojl$ is calculated as in \cite[Corollary~\markovcor{}]{TMK_markov}.  
Now, let 

\begin{align}
\label{eq:Mstar}
	M^{-} = \left\lfloor{\frac{N(1 - \hat{d}_u)}{\mathbb{E}\oRhojl}}\right\rfloor
\end{align}
be the result of applying the floor function to the right-hand-side of~\eqref{eq:EM_lb}.  We have that $M^{-}$ gives a lower bound for the expected number of times the Markov chain is reset.  If user~$i$, the user building the chains, can meet their distortion constraint within the $M^{-}$ times the Markov rewards process is being run, then all users will be point-to-point optimal.  This is because user~$i$ is able to decode all his required symbols despite the fact that we are targeting optimal performance for the other two users.

Let $\oRhoel$ be the expected number of symbols in the chain that can be decoded in the $l$th run of the Markov rewards process given that the decoding absorbing state, State~6, was reached after having started in the initial state, State~1.  
%From the discussion leading to the derivation of Corollary~\ref{cor:barRinfi}, 
We mention that we can easily calculate $\oRhoel$ as in~\cite[Theorem~\markovthm{}]{TMK_markov}.  %we can similarly argue that 
%we have that $\mathbb{E}\oRhoel$ can be derived by substituting the transition matrix and rewards matrix $\orhoe$ into Theorem~\ref{thm:hatR_inf} and simply taking the $(1, 6)$th element from the resultant matrix.  This is because the rewards matrix $\orhoe$ is used to calculate the expected number of symbols in the chain built by user~$i$, however, user $i$ is able to decode these symbols only if State~6, the decoding absorbing state, is reached.
%
%By Theorem~\ref{thm:barRinf_absorb} of Section~\ref{subsec:unscaled_rewards}, we can calculate $\mathbb{E}\oRhoel$ from the transition matrix, and the rewards matrix $\orhoe$.  
%
Now, let $\overline{\sigma}$ be the expected number of symbols that can be decoded in $M^{-}$ iterations of the Markov rewards process.  By the linearity of the expectation operator we have

\begin{align}
	\overline{\sigma} &= \sum_{l = 1}^{M^{-}} \mathbb{E}\oRhoel \\
	\label{eq:normalized_chain_received}
	&= M^{-} \times \mathbb{E}\oRhoel.
\end{align}
If the right-hand-side of~\eqref{eq:normalized_chain_received} is greater than $N(1 - \hat{d}_i)$, the fraction of symbols user~$i$ requires, then we are point-to-point optimal for all users.  Combining~\eqref{eq:normalized_chain_received} and~\eqref{eq:Mstar}, we see that this happens when 

\begin{align}
\label{eq:chain_optimality}
%	\frac{1}{N}\left\lfloor{\frac{N(1 - \hat{d}_u)}{\mathbb{E}\oRhojl}}\right\rfloor \mathbb{E}\oRhoel \geq 1 - \hat{d}_i.
	\left\lfloor{\frac{N(1 - \hat{d}_u)}{\mathbb{E}\oRhojl}}\right\rfloor \geq \frac{N(1 - \hat{d}_i)}{\mathbb{E}\oRhoel}.
\end{align}

\begin{theorem}
\label{thm:chaining_sufficient}
	Let $u$ be the user satisfying~\eqref{eq:u_bottleneck}.  Then $w^{+}\dvec$ is $\dvec$-achievable if~\eqref{eq:chain_optimality} is satisfied, where $\mathbb{E}\oRhojl$ is calculated from the transition matrix and rewards matrix $\overline{\rho}_{u}$ via \cite[Corollary~\markovcor{}]{TMK_markov}, and $\mathbb{E}\oRhoel$ is calculated from the transition matrix and rewards matrix $\orhoe$ via \cite[Theorem~\markovthm{}]{TMK_markov}.
\end{theorem}

Finally, we again remark that the analysis we have just described is only a sufficient condition for a restricted version of the chaining algorithm in which we do not opportunistically send linear combinations of the form in~\Crefrange{eq:linear_comb_available_a}{eq:linear_comb_available_c}.  Therefore, we expect the unrestricted algorithm to perform better.

\subsection{Operational Significance of Theorem~\ref{thm:chaining_sufficient}}
\label{sec:operational_meaning}

Consider a hypothetical situation in which we have queues $\Q{1, 2}, \Q{1, 3}$ and $\Q{2, 3}$.   We consider quadratic distortions in which for $i \in \{2, 3\}$,  $d_i = \epsilon_i^2$, and we fix $\epsilon_1 = 0.1$, $\epsilon_3 = 0.6$ and vary $\epsilon_2 \in (0.2, 0.6)$.  

We illustrate the chaining algorithm when user~1 is the user who builds chains and we send linear combinations of the symbols in $\Q{1, 2}$ and $\Q{1, 3}$ as if users~2 and~3 were the only users in the network.  In this case, for $i \in \{2, 3\}$, user~$i$'s point-to-point optimal latency is given by $\wi = (1 - \epsilon_i^2)/(1 - \epsilon_i)$, and since $\wi$ is an increasing function of $\epsilon_i \in [0, 1)$, we have that between users~2 and~3, user~2 is \emph{not} the bottleneck user.  That is, in~\eqref{eq:chain_optimality}, user~2 takes the place of user~$u$, and since user~1 is building the chains, user~1 takes the place of user~$i$.

We rearrange~\eqref{eq:chain_optimality} of Theorem~\ref{thm:chaining_sufficient} to find a lower bound for $\hat{d}_i$, and since $\hat{d}_u = \epsilon_u^2$, we  plot this lower bound as a function of $\epsilon_u$ in Figure~\ref{fig:operational_meaning}.  From this figure, we can read which values of $\hat{d}_i$ would yield optimal performance for a given value of $\epsilon_u$ by considering all values of $\hat{d}_u$ above the lower bound.  Recall that user~$i$ is able to decode symbols in the chain only if there have been two consecutive transmissions for which user~$i$ is the only user to have received the transmission.  We see that the probability of this event increases as $\epsilon_u$ increases in Figure~\ref{fig:operational_meaning}.  Therefore, user~$i$ is able to achieve lower distortions as $\epsilon_u$ increases.  

Finally, we again mention that Theorem~\ref{thm:chaining_sufficient} merely gives a conservative \emph{sufficient} condition for optimality and ignores other network coding opportunities in its analysis.  Therefore, it is possible that point-to-point optimal performance can still be met if user~$i$ has a distortion constraint below the lower bound of Figure~\ref{fig:operational_meaning}.

\begin{figure}
	\centering
	\setlength\figurewidth{2.65in} 
	\setlength\figureheight{2.07in} 
%	\includegraphics[scale=0.5]{fig/ttotal_N_10e7_eps1_3_eps2_4}
	\input{fig/simulations/operational.tikz}
	\caption{The lower bound from Theorem~\ref{thm:chaining_sufficient} that delineates a conservative boundary of distortion values for which the minmax optimal latency can be achieved.  }
	\label{fig:operational_meaning}
\end{figure}
