%\newcounter{magicrownumbers}
%\newcommand\rownumber{\stepcounter{magicrownumbers}\arabic{magicrownumbers}}

\setcounter{mymagicrownumbers}{0} 

\begin{table}
	\begin{center}
%		\begin{tabular}{ | r}
%		\begin{tabular}{@{\makebox[3em][r]{\myrownumber\space}} | c l *{8}{c}}
		\begin{tabular}{c c l *{8}{c}}
			\multicolumn{10}{c}{{\bf State~1 Outgoing Transitions}} \\		
			& $\zall$              & $\Prob(\zall)$ & Next State & $\rhoj$ & $\rhok$ & $\rhoc$ & $\rhoqi$ & $\rhoqj$ & $\rhoqk$ & $\rhosub{\Qstar}$\\
%			\gdef\myrownumber{\stepcounter{mymagicrownumbers}\arabic{mymagicrownumbers}} \\			
			\hline
			\myrownumber & $(0, 0, 0)$ & $(1 -\epsilon_i)(1 -\epsilon_j)(1 - \epsilon_k)$ & 5 &1 & 1 & 1 & 0 & 0 & 0 & 1 \\
			\myrownumber & $(0, 0, 1)$ & $(1 -\epsilon_i)(1 -\epsilon_j)\epsilon_k$ & 2 &1 & 0 & 1 & 0 & 0 & 0 & 0 \\
			\myrownumber & $(0, 1, 0)$ & $(1 -\epsilon_i)\epsilon_j(1 - \epsilon_k)$ & 3 & 0 & 1 & 1 & 0 & 0 & 0 & 0 \\
			\myrownumber & $(0, 1, 1)$ & $(1 -\epsilon_i)\epsilon_j\epsilon_k$ & 4 & 0 & 0 & 1 & 0 & 0 & 0 & 0 \\
			\myrownumber & $(1, 0, 0)$ & $\epsilon_i(1 -\epsilon_j)(1 - \epsilon_k)$ & 5 & 1 & 1 & 0 & 2 & 0 & 0 & 0 \\
			\myrownumber & $(1, 0, 1)$ & $\epsilon_i(1 -\epsilon_j)\epsilon_k$ & 1 &1 & 0 & 0 & 1 & 0 & 0 & 0 \\											\myrownumber & $(1, 1, 0)$ & $\epsilon_i\epsilon_j(1 - \epsilon_k)$ & 1 &0 & 1 & 0 & 1 & 0 & 0 & 0 \\
			\myrownumber & $(1, 1, 1)$ & $\epsilon_i\epsilon_j\epsilon_k$ & 1 &0 & 0 & 0 & 0 & 0 & 0 & 0 \\
		\end{tabular}
	\end{center}
	\caption{A detailed table of all outgoing transitions from State~1 based on the channel noise realization.  The table includes the probability of each transition and the reward accumulated for each transition.}	
	\label{tab:state1_transitions_all}
\end{table}

\setcounter{mymagicrownumbers}{0} 
\begin{LaTeXdescription}
	\item[State~1] We enumerate all outgoing transitions from State~1, the initial state, based on all possible channel conditions in Table~\ref{tab:state1_transitions_all}.  
	\begin{enumerate}
		\item $\zall = (0, 0, 0)$.  In this case, all users have received the transmission.  We therefore see that $\rhoj = \rhok = 1$, since users~$j$ and $k$ can each decode a symbol.  Since this is the initial state however, user~$i$ has not received any previous equations, and therefore has only one equation in two unknown variables.  Since we must replace both symbols in the next transmission, we cannot continue building the current chain.  The next state is therefore State~5, the absorbing state in which we are not able to decode any symbols in the chain.  However, we do increase the number of equations received by user~$i$ by setting $\rhoc = 1$.  Furthermore, we arbitrarily place one of the symbols from the linear combination into $\Qstar$, the queue containing symbols that can decode an entire chain of symbols.  We set $\rhoqstar = 1$ to reflect this.  
		\item $\zall = (0, 0, 1)$. In this case, only users $i$ and~$j$ have received the transmission.  We set $\rhoj =  \rhoc =1$ to indicate that user~$j$ can decode one symbol, and user~$i$ has received one more equation.  Since the symbol for user~$j$ will be replaced in the next transmission, and the chain-building process has just begun, we therefore have that the next state is State~2.  
		\item $\zall = (0, 1, 0)$. Similar to the previous case, however, we now have that only users $i$ and~$k$ have received the transmission.  We set $\rhok =  \rhoc =1$ to indicate that user~$k$ can decode one symbol, and user~$i$ has received one more equation.  Since the symbol for user~$k$ will be replaced in the next transmission, and the chain-building process has just begun, we therefore have that the next state is State~3.  
		\item $\zall = (0, 1, 1)$.  In this case, only user~$i$ has received the transmission so we set only $\rhoc =1$.  For the subsequent timeslot, we must send a different linear combination of the same two symbols previously sent and so the next state is State~4.
		\item $\zall = (1, 0, 0)$.  In this case, both users~$j$ and~$k$ can decode a symbol so we set $\rhoj = \rhok =1$.  If $q_{i, j} \oplus q_{i, k}$ was sent in the previous timeslot, we now have that both $q_{i, j}$ and $q_{i, k}$ are required by only user~$i$.  We therefore place both these symbols in $Q_{i}$, and we set $\rhoqi = 2$ to reflect this.  Since both $q_{i, j}$ and $q_{i, k}$ will not be sent in the next transmission, we  have that the next state is the non-decoding absorbing state, State~5.
		\item $\zall = (1, 0, 1)$.  In this case, only user~$j$ received the transmission so we set $\rhoj = 1$.  Notice now that $q_{i, j}$ is no longer needed by user~$j$.  Although we must replace this symbol in the next transmission, the chain-building process has not yet begun and user~$i$ has still not received any equations involving $q_{i, j}$.  We therefore place this symbol in $Q_{i}$, set $\rhoqi = 1$ to reflect this, and return to State~1 in the next timeslot.
		\item $\zall = (1, 1, 0)$	.  Similar to the previous case, only user~$k$ has received the transmission and so we set $\rhok = \rhoqi = 1$, and return to State~1 in the next timeslot.
		\item $\zall = (1, 1, 1)$	.  In this case, no users have received the transmission.  No rewards are assigned and we return to State~1 to retransmit the same linear combination.
	\end{enumerate}		
\end{LaTeXdescription}

