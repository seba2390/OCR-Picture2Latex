\subsubsection{Scaled Rewards}
\label{subsec:scaled_rewards}

We first derive Lemma~\ref{lem:barRij_property} to express $\barRij{i}{j}$ in terms of the elements in both the reward matrix and the transition matrix.  We then use this lemma to derive a recurrence relation for the scaled transient accumulated reward, $\hatR$, in~Lemma~\ref{lem:recurrence}.  We use the recurrence relation to derive an actual expression for the transient scaled reward in Lemma~\ref{lem:hatRn_matrix}.  Finally, we use properties of absorbing Markov chains to derive a single letter expression for the long-term scaled reward, $\hatRinf$, in Theorem~\ref{thm:hatR_inf}.

\begin{myLemma}
	\label{lem:barRij_property}
	For $n = 2, 3, \ldots $
	\begin{equation}
		\barRij{i}{j} = \chainsum \left( \rew{i}{k_1} + \rew{k_1}{k_2} + \ldots + \rew{k_{n-1}}{j}\right) \times \frac{\trans{i}{k_1}\trans{k_1}{k_2} \ldots \trans{k_{n-1}}{j}}{\Prob(\Si{n} = j| \Si{0} = i)} 
	\end{equation}
\end{myLemma}


%%\begin{fleqn}
%%	\begin{align}
%%	\label{eq:sum_of_equations}
%%	\mathbb{E}(\Rn | \Si{0} = i, \Si{n} = j) &=  \phantom{\sum_{k_1} \sum_{k_2} \ldots \sum_{k_{n-1}} \mathbb{E}(\Rn | \Si{0} = i, \Si{1} = k_1, \Si{2} = k_2, \ldots, \Si{n-1} = k_{n-1},\Si{n} = j)} \mathllap{\mathbb{E}_{\Si{1}, \Si{2}, \ldots, \Si{n-1}} \mathbb{E}(\Rn | \Si{0} = i, \Si{1}, \Si{2}, \ldots, \Si{n-1}, \Si{n} = j)} \hidewidth\\
%%%	&\stackrel{(\alph{cnt})}{=}& \sum_{i = 0}^{M-1} H(\SHatK{2}(Y_i^{n}), \mathcal{S}_i | N_0^n) -  H( \mathcal{S}_i | \SHatK{2}(Y_i^{n}), N_0^n)\\
%%%		\addtocounter{cnt}{1}
%%%	&\stackrel{(\alph{cnt})}{=}& \sum_{i = 0}^{M-1} H(\mathcal{S}_i | N_0^n) \\
%%%		\addtocounter{cnt}{1}
%%%	&\stackrel{(\alph{cnt})}{=}& H(\mathcal{S}_0, \mathcal{S}_1, \ldots, \mathcal{S}_{M-1}) | N_0^n) + \sum_{i = 0}^{M-1} I(\mathcal{S}_i; \mathcal{S}_0^{i-1} | N_0^n) \\
%%	&\mathllap{\stackrel{(\alph{cnt})}{=} \sum_{k_1} \sum_{k_2} \ldots \sum_{k_{n-1}} &&\mathbb{E}(\Rn | \Si{0} = i, \Si{1} = k_1, \Si{2} = k_2, \ldots, \Si{n-1} = k_{n-1},\Si{n} = j)} \nonumber \\
%%		&&&\quad  \times \Prob(\Si{1} = k_1, \Si{2} = k_2, \ldots, \Si{n-1} = k_{n-1}| \Si{0} = i, \Si{n} = j)\\
%%	&\stackrel{(\alph{cnt})}{=} \sum_{k_1, k_2, \ldots, k_{n-1}} &&\mathbb{E}(\Rn | \Si{0} = i, \Si{1} = k_1, \Si{2} = k_2, \ldots, \Si{n-1} = k_{n-1},\Si{n} = j) \nonumber \\
%%	&&&\quad \times \Prob(\Si{1} = k_1, \Si{2} = k_2, \ldots, \Si{n} = j | \Si{0} = i)/\Prob(\Si{n}| \Si{0})\\
%%	\end{align}
%%\end{fleqn},

%%% Prob only indented after equal sign
%%%\begin{fleqn}
%%%	\begin{align}
%%%	\label{eq:sum_of_equations}
%%%	\mathbb{E}(\Rn | \Si{0} = i, \Si{n} = j) ={}& \mathbb{E}_{\Si{1}, \Si{2}, \ldots, \Si{n-1}} \mathbb{E}(\Rn | \Si{0} = i, \Si{1}, \Si{2}, \ldots, \Si{n-1}, \Si{n} = j)\\
%%%%	&\stackrel{(\alph{cnt})}{=}& \sum_{i = 0}^{M-1} H(\SHatK{2}(Y_i^{n}), \mathcal{S}_i | N_0^n) -  H( \mathcal{S}_i | \SHatK{2}(Y_i^{n}), N_0^n)\\
%%%%		\addtocounter{cnt}{1}
%%%%	&\stackrel{(\alph{cnt})}{=}& \sum_{i = 0}^{M-1} H(\mathcal{S}_i | N_0^n) \\
%%%%		\addtocounter{cnt}{1}
%%%%	&\stackrel{(\alph{cnt})}{=}& H(\mathcal{S}_0, \mathcal{S}_1, \ldots, \mathcal{S}_{M-1}) | N_0^n) + \sum_{i = 0}^{M-1} I(\mathcal{S}_i; \mathcal{S}_0^{i-1} | N_0^n) \\
%%%	\begin{split}
%%%		\stackrel{(\alph{cnt})}{=}{}& \sum_{k_1} \sum_{k_2} \ldots \sum_{k_{n-1}} \mathbb{E}(\Rn | \Si{0} = i, \Si{1} = k_1, \Si{2} = k_2, \ldots, \Si{n-1} = k_{n-1},\Si{n} = j) \\
%%%		& \times \Prob(\Si{1} = k_1, \Si{2} = k_2, \ldots, \Si{n-1} = k_{n-1}| \Si{0} = i, \Si{n} = j)
%%%	\end{split} \\
%%%	\begin{split}
%%%		\stackrel{(\alph{cnt})}{=}{}& \sum_{k_1, k_2, \ldots, k_{n-1}} \mathbb{E}(\Rn | \Si{0} = i, \Si{1} = k_1, \Si{2} = k_2, \ldots, \Si{n-1} = k_{n-1},\Si{n} = j) \\
%%%		& \times \Prob(\Si{1} = k_1, \Si{2} = k_2, \ldots, \Si{n} = j | \Si{0} = i)/\Prob(\Si{n} = j| \Si{0} = i)
%%%	\end{split} \\
%%%	\end{align}
%%%\end{fleqn},

%\begin{dmath}
%		\stackrel{(\alph{cnt})}{=} \sum_{k_1} \sum_{k_2} \ldots \sum_{k_{n-1}} \mathbb{E}(\Rn | \Si{0} = i, \Si{1} = k_1, \Si{2} = k_2, \ldots, \Si{n-1} = k_{n-1},\Si{n} = j)  \times \Prob(\Si{1} = k_1, \Si{2} = k_2, \ldots, \Si{n-1} = k_{n-1}| \Si{0} = i, \Si{n} = j)
%\end{dmath}

% Quad
\begin{proof}
	We calculate 
	\setcounter{cnt}{1}
	\begin{fleqn}
		\begin{alignat}{2}
	%	\begin{align}
%		&\mathbb{E}(\Rn | \Si{0} = i, \Si{n} = j) \nonumber \\
			&\barRij{i}{j} \nonumber \\
			&\stackrel{(\alph{cnt})}{=} \mathbb{E}_{\Si{1}, \Si{2}, \ldots, \Si{n-1}} \mathbb{E}(\Rn | \Si{0} = i, \Si{1}, \Si{2}, \ldots, \Si{n-1}, \Si{n} = j) \\
				\addtocounter{cnt}{1}
			&= \chainsum  \Big\{ \mathbb{E}(\Rn | \Si{0} = i, \Si{1} = k_1, \Si{2} = k_2, \ldots, \Si{n-1} = k_{n-1},\Si{n} = j) \phantom{\Big\}} \nonumber \\
				& \hphantom{=\chainsum} \qquad \times \vphantom{\Big\{} \Prob(\Si{1} = k_1, \Si{2} = k_2, \ldots, \Si{n-1} = k_{n-1}| \Si{0} = i, \Si{n} = j) \Big\} \\
			&\stackrel{(\alph{cnt})}{=} \chainsum  \left\{ \mathbb{E}(\Rn | \Si{0} = i, \Si{1} = k_1, \Si{2} = k_2, \ldots, \Si{n-1} = k_{n-1},\Si{n} = j) \vphantom{\frac{\Prob\Si{1}}{\Prob\Si{2}}} \right. \nonumber \\
				&\hphantom{=\chainsum} \qquad \times \left. \frac{\Prob(\Si{1} = k_1, \Si{2} = k_2, \ldots, \Si{n} = j | \Si{0} = i)}{\Prob(\Si{n} = j| \Si{0} = i)} \right\} \\
		\addtocounter{cnt}{1}
			&\stackrel{(\alph{cnt})}{=} \chainsum \left( \rew{i}{k_1} + \rew{k_1}{k_2} + \ldots + \rew{k_{n-1}}{j}\right) \times \frac{\Prob(\Si{1} = k_1, \Si{2} = k_2, \ldots, \Si{n} = j | \Si{0} = i)}{\Prob(\Si{n} = j| \Si{0} = i)} \\
		\addtocounter{cnt}{1}
			&\stackrel{(\alph{cnt})}{=} \chainsum \left( \rew{i}{k_1} + \rew{k_1}{k_2} + \ldots + \rew{k_{n-1}}{j}\right) \times \frac{\trans{i}{k_1}\trans{k_1}{k_2} \ldots \trans{k_{n-1}}{j}}{\Prob(\Si{n} = j| \Si{0} = i)} 
		\end{alignat}
	\end{fleqn}
	where 
	\begin{enumerate}[(a)]
		\item follows from Definition~\ref{def:barRij} and the law of total expectation
		\item follows from Bayes' Theorem
		\item follows from the fact that we have conditioned on each state from time $t = 0, 1, 2, \ldots, n$, and so the additive rewards for each transition is known (see~\eqref{eq:Rn})
		\item follows from the Markov property
	\end{enumerate}
\end{proof}

\begin{myLemma}
	\begin{equation}
		\label{eq:hatR2}
		\hatRsub{2} = \HtransRew \transM + \transM \HtransRew
	\end{equation}
\end{myLemma}

\begin{proof}
	\setcounter{cnt}{1}
	We use Definition~\ref{def:hatRij} and substitute $n = 2$ into Lemma~\ref{lem:barRij_property} to get 
	\begin{align}
		\hatRsubij{2}{i}{j} &= \sum_{k_1 = 1}^{\sizeS} (\rew{i}{k_1} + \rew{k_1}{j}) \trans{i}{k_1}\trans{k_1}{j} \\
		&= \sum_{k_1 = 1}^{\sizeS}(\rew{i}{k_1}\trans{i}{k_1}) \trans{k_1}{j} +  \trans{i}{k_1}(\rew{k_1}{j} \trans{k_1}{j} ) \\
		&\stackrel{(\alph{cnt})}{=} \sum_{k_1 = 1}^{\sizeS}\HtransRewij{i}{k_1} \trans{k_1}{j} +  \trans{i}{k_1}\HtransRewij{k_1}{j} \\	
		\addtocounter{cnt}{1}		
		&\stackrel{(\alph{cnt})}{=} [\HtransRew \transM]_{i, j} +  [\transM \HtransRew]_{i, j} 	
	\end{align}
	where 
\begin{enumerate}[(a)]
	\item follows from Definition~\ref{def:HtransRew}
	\item follows from the definition of matrix multiplication
\end{enumerate}
\end{proof}

% --------------------------------------------------------------
% Recurrence Relation
% --------------------------------------------------------------

We now use Lemma~\ref{lem:barRij_property} to derive a recurrence relation for $\hatRij{i}{j}$.

\begin{myLemma}
\label{lem:recurrence}
	For $n = 2, 3,  \ldots $
	\begin{equation}
	\label{eq:lem_recurrence}
		\hatR = \hatRsub{n-1} \transM + \transMsup{n-1}\HtransRew
	\end{equation}
\end{myLemma}

\begin{proof}
We first prove the lemma for $n = 3, 4, \ldots $
\setcounter{cnt}{1}
\begin{fleqn}
	\begin{alignat}{2}
		&\hatRij{i}{j} \nonumber \\
			&\stackrel{(\alph{cnt})}{=} \sum_{k_{n-1}=1}^{\sizeS} \left( \left\{\chainsumOne \left( \rew{i}{k_1} + \rew{k_1}{k_2} + \ldots + \rew{k_{n-2}}{k_{n-1}}\right) \times \trans{i}{k_1}\trans{k_1}{k_2} \ldots \trans{k_{n-2}}{k_{n-1}}\right\} \right.  \\
			&\hphantom{=\sum_{k_{n-1} = 1}^{\sizeS} } \qquad \times \trans{k_{n-1}}{j}  + \left.  \chainsumOne  \rew{k_{n-1}}{j} \times \trans{i}{k_1}\trans{k_1}{k_2} \ldots \trans{k_{n-1}}{j}\right) \\
		\addtocounter{cnt}{1}
			&\stackrel{(\alph{cnt})}{=} \sum_{k_{n-1}=1}^{\sizeS} \left(  \Big\{ \hatRsubij{n-1}{i}{k_{n-1}} \Big\} \trans{k_{n-1}}{j} +\chainsumOne  \rew{k_{n-1}}{j} \times \trans{i}{k_1}\trans{k_1}{k_2} \ldots \trans{k_{n-1}}{j} \right) \\
		\addtocounter{cnt}{1}
%			&\hphantom{=\sum_{k_{n-1} = 1}^{\sizeS} }  \qquad + \left. \chainsumOne  \rew{k_{n-1}}{j} \times \trans{i}{k_1}\trans{k_1}{k_2} \ldots \trans{k_{n-1}}{j} \right) \\
			&= \sum_{k_{n-1}=1}^{\sizeS} \left(   \hatRsubij{n-1}{i}{k_{n-1}}  \trans{k_{n-1}}{j} + \rew{k_{n-1}}{j} \trans{k_{n-1}}{j}\chainsumOne   \trans{i}{k_1}\trans{k_1}{k_2} \ldots \trans{k_{n-2}}{k_{n-1}} \right) \\
			&\stackrel{(\alph{cnt})}{=} \sum_{k_{n-1}=1}^{\sizeS}   \hatRsubij{n-1}{i}{k_{n-1}}  \trans{k_{n-1}}{j} + \rew{k_{n-1}}{j} \trans{k_{n-1}}{j}\transMsupij{n-1}{i}{k_{n-1}} \\
		\addtocounter{cnt}{1}
			&\stackrel{(\alph{cnt})}{=} \sum_{k_{n-1}=1}^{\sizeS}   \hatRsubij{n-1}{i}{k_{n-1}}  \trans{k_{n-1}}{j} + \HtransRewij{k_{n-1}}{j} \transMsupij{n-1}{i}{k_{n-1}} \\
		\addtocounter{cnt}{1}
			&\stackrel{(\alph{cnt})}{=}  [\hatRsub{n-1}  \transM]_{i,j} +  [\transMsup{n-1} \HtransRew ]_{i, j}
	\end{alignat}
\end{fleqn}
where 
\begin{enumerate}[(a)]
	\item follows from Definition~\ref{def:hatRij} and rearranging Lemma~\ref{lem:barRij_property}
	\item follows from Definition~\ref{def:hatRij} and the application of Lemma~\ref{lem:barRij_property} for $\barRsubij{n-1}{i}{k_{n-1}}$
	\item follows from Lemma~\ref{lem:transMsup} in Appendix~\ref{sec:matrix_properties}, where we have used the fact that $n \geq 3$
	\item follows from Definition~\ref{def:HtransRew}
	\item follows from the definition of matrix multiplication
\end{enumerate}

We mention that although we derived the lemma assuming $n \in \{3, 4, \ldots\}$, the lemma also holds if $n=2$.  We can see this by using  Corollary~\ref{cor:hatR1} to compare the right-hand-sides of \eqref{eq:hatR2} and \eqref{eq:lem_recurrence} when $n=2$.

\end{proof}

% --------------------------------------------------------------
% R for general n
% --------------------------------------------------------------
\begin{myLemma}
	\label{lem:hatRn_matrix}
	For $n = 1, 2, \ldots $
	\begin{equation}
	\label{eq:hatRn_matrix}
		\hatR = 
		\begin{bmatrix}
			\RewSubD & \RewSubC\\
			0 & 0 
		\end{bmatrix},
	\end{equation}	
	where
	\begin{align}
		\label{eq:RewSubDn}
			\RewSubD &= \sum_{i = 0}^{n-1}\transQ^{i}\HSubTransTrans\transQ^{n - i - 1}, \\
		\label{eq:RewSubCn}
			\RewSubC &= \fundMat(I - \transQ^{n})\HSubTransAbs + \fundMat(I - \transQ^{n-1})\HSubTransTrans\fundMat\transR - \sum_{i = 0}^{n-2}\transQ^{i}\HSubTransTrans\fundMat\transQ^{n - i - 1}\transR.
	\end{align}
\end{myLemma}

\begin{proof}
	We proceed by induction.
	\begin{LaTeXdescription}
		\item[Base case] We use Corollary~\ref{cor:hatR1} and Remark~\ref{rem:HtransRew} to verify \eqref{eq:hatRn_matrix} for the base case after substituting $n = 1$ into~\eqref{eq:RewSubDn}  and~\eqref{eq:RewSubCn} to get that 
		\setcounter{cnt}{1}
		\begin{align}
			\RewSubDi{1} &= \HSubTransTrans, \\		
			\RewSubCi{1} &= \fundMat(I - \transQ)\HSubTransAbs \\
			&\stackrel{(\alph{cnt})}{=} \HSubTransAbs,
		\end{align}
		where 
		\begin{enumerate}[(a)]
			\item follows from the Definition of the fundamental matrix in Lemma~\ref{lem:fund_mat}.
		\end{enumerate}
		
		\item[Inductive Hypothesis] We assume that for $n - 1 \geq 1$, 
			\begin{align}
				\RewSubDi{n-1} &= \sum_{i = 0}^{n-2}\transQ^{i}\HSubTransTrans\transQ^{n - i - 2}, \\			
				\RewSubCi{n-1} &= \fundMat(I - \transQ^{n-1})\HSubTransAbs + \fundMat(I - \transQ^{n-2})\HSubTransTrans\fundMat\transR - \sum_{i = 0}^{n-3}\transQ^{i}\HSubTransTrans\fundMat\transQ^{n - i - 2}\transR.
			\end{align}
		
		\item[Induction Step] We begin with the recurrence relation in Lemma~\ref{lem:recurrence} from which we get 
			\setcounter{cnt}{1}
			\begin{align}
				\hatR &= \hatRsub{n-1} \transM + \transMsup{n-1}\HtransRew \\
				&\stackrel{(\alph{cnt})}{=} 	
					\begin{bmatrix}
						\RewSubDi{n-1} & \RewSubCi{n-1}\\
						0 & 0
					\end{bmatrix}
					P + \transMsup{n-1}\HtransRew \\
				\addtocounter{cnt}{1}
				&\stackrel{(\alph{cnt})}{=} 	
					\begin{bmatrix}
						\RewSubDi{n-1} & \RewSubCi{n-1}\\
						0 & 0
					\end{bmatrix}
					\begin{bmatrix}
						\transQ & \transR \\ 
						0 & I
					\end{bmatrix} + 
					\begin{bmatrix}
						\transQ^{n-1} & \sum_{i = 0}^{n-2} \transQ^{i}\transR \\ 
						0 & I
					\end{bmatrix} H\\
				\addtocounter{cnt}{1}
				&\stackrel{(\alph{cnt})}{=}
					\label{eq:before_block}
					\begin{bmatrix}
						\RewSubDi{n-1} & \RewSubCi{n-1}\\
						0 & 0
					\end{bmatrix}
					\begin{bmatrix}
						\transQ & \transR \\ 
						0 & I
					\end{bmatrix} + 
					\begin{bmatrix}
						\transQ^{n-1} & \sum_{i = 0}^{n-2} \transQ^{i}\transR \\ 
						0 & I
					\end{bmatrix}
					\begin{bmatrix}
						\HSubTransTrans & \HSubTransAbs\\ 
						0 & 0
					\end{bmatrix} 
				\addtocounter{cnt}{1}					
			\end{align}
			where 
			\begin{enumerate}[(a)]
				\item follows from the inductive hypothesis
				\item follows from Lemma~\ref{lem:transM_n}
				\item follows from Remark~\ref{rem:HtransRew}.
			\end{enumerate}

			We first consider $\RewSubDi{n}$.  We perform the block matrix multiplication in~\eqref{eq:before_block} to get that 
			\begin{align}
				\setcounter{cnt}{1}
				\RewSubDi{n} &=	\RewSubDi{n-1} \transQ + \transQ^{n-1}\HSubTransTrans \\
				&\stackrel{(\alph{cnt})}{=} \left( \sum_{i = 0}^{n-2}\transQ^{i}\HSubTransTrans\transQ^{n - i - 2} \right) \transQ +  \transQ^{n-1}\HSubTransTrans\\
				\addtocounter{cnt}{1}	
				&=  \sum_{i = 0}^{n-2}\transQ^{i}\HSubTransTrans\transQ^{n - i - 1}+  \transQ^{n-1}\HSubTransTrans\\
				\label{eq:RewSubDi_final}
				&=  \sum_{i = 0}^{n-1}\transQ^{i}\HSubTransTrans\transQ^{n - i - 1}
			\end{align}
			where 
			\begin{enumerate}[(a)]
				\item follows from the inductive hypothesis.
			\end{enumerate}
			We conclude that~\eqref{eq:RewSubDn} holds after comparing with~\eqref{eq:RewSubDi_final} in the induction step. Next, for $\RewSubCi{n}$, we again perform the block matrix multiplication in~\eqref{eq:before_block} to get that 
			\begin{align}
				\setcounter{cnt}{1}
				\RewSubCi{n} &= \RewSubDi{n-1} \transR +  \RewSubCi{n-1}  + \transQ^{n-1}\HSubTransAbs \\
				&\stackrel{(\alph{cnt})}{=} \left( \sum_{i = 0}^{n-2}\transQ^{i}\HSubTransTrans\transQ^{n - i - 2} \right) \transR + \left( \fundMat(I - \transQ^{n-1})\HSubTransAbs + \fundMat(I - \transQ^{n-2})\HSubTransTrans\fundMat\transR \vphantom{\sum_{i=0}^{n}} \right. \\ \nonumber
				&\qquad \left. - \sum_{i = 0}^{n-3}\transQ^{i}\HSubTransTrans\fundMat\transQ^{n - i - 2}\transR \right) + \transQ^{n-1}\HSubTransAbs \\
				\addtocounter{cnt}{1}				
				&\stackrel{(\alph{cnt})}{=}  \left( \sum_{i = 0}^{n-2}\transQ^{i}\HSubTransTrans \fundMat (I - \transQ)\transQ^{n - i - 2}  \transR  - \sum_{i = 0}^{n-3}\transQ^{i}\HSubTransTrans\fundMat\transQ^{n - i - 2}\transR \right) \\ \nonumber
				&\qquad +  \fundMat(I - \transQ^{n-1})\HSubTransAbs + \fundMat(I - \transQ^{n-2})\HSubTransTrans\fundMat\transR  + \transQ^{n-1}\HSubTransAbs \\
				\addtocounter{cnt}{1}				
				&=  \left( \sum_{i = 0}^{n-2} \left( \transQ^{i}\HSubTransTrans \fundMat \transQ^{n - i - 2}  \transR -  \transQ^{i}\HSubTransTrans \fundMat\transQ^{n - i - 1} \transR \right) - \sum_{i = 0}^{n-3}\transQ^{i}\HSubTransTrans\fundMat\transQ^{n - i - 2}\transR \right) \\ \nonumber 
				&\qquad +  \fundMat(I - \transQ^{n-1})\HSubTransAbs + \fundMat(I - \transQ^{n-2})\HSubTransTrans\fundMat\transR + \transQ^{n-1}\HSubTransAbs \\
				&=  \left(   \transQ^{n-2} \HSubTransTrans  \fundMat  \transR -  \sum_{i = 0}^{n-2} \transQ^{i}\HSubTransTrans \fundMat\transQ^{n - i - 1} \transR \right) \\ \nonumber
				&\qquad +  \fundMat(I - \transQ^{n-1})\HSubTransAbs + \fundMat(I - \transQ^{n-2})\HSubTransTrans\fundMat\transR + \transQ^{n-1}\HSubTransAbs \\
				&=   \fundMat(I - \transQ^{n-1} + \fundMat^{-1}\transQ^{n-1})\HSubTransAbs + \fundMat(I - \transQ^{n-2} + \fundMat^{-1}\transQ^{n-2})\HSubTransTrans\fundMat\transR  \\ \nonumber
				&\qquad -  \sum_{i = 0}^{n-2} \transQ^{i}\HSubTransTrans \fundMat\transQ^{n - i - 1} \transR   \\
				\label{eq:RewSubCn_final}
				&\stackrel{(\alph{cnt})}{=}\fundMat(I - \transQ^{n})\HSubTransAbs + \fundMat(I - \transQ^{n-1})\HSubTransTrans\fundMat\transR  -  \sum_{i = 0}^{n-2} \transQ^{i}\HSubTransTrans \fundMat\transQ^{n - i - 1} \transR
			\end{align}
			where 
			\begin{enumerate}[(a)]
				\item follows from the inductive hypothesis
				\item and (c) follow from the definition of the fundamental matrix, i.e., $\fundMat^{-1} = (I - \transQ)$ in Lemma~\ref{lem:fund_mat}.
%				\item follows from the definition of the fundamental matrix, i.e., $\fundMat = (I - \transQ)^{-1}$ in Lemma~\ref{lem:fund_mat}.
			\end{enumerate}
			We conclude the proof after comparing~\eqref{eq:RewSubCn} with~\eqref{eq:RewSubCn_final} in the induction step.
	\end{LaTeXdescription}
\end{proof}

% --------------------------------------------------------------
% Limit R_inf
% --------------------------------------------------------------
\begin{theorem}
	\label{thm:hatR_inf}
	Let $\RewSubDi{\infty} = \lim_{n \to \infty} \RewSubD$, and $\RewSubCi{\infty} = \lim_{n \to \infty} \RewSubC$.  Then $\RewSubDi{\infty} = 0$ and %and define 
%	Let $\hatRinf = \lim_{n\to \infty} \hatR$.  Then 
%	Let $\RewSubCi{\infty} = \lim_{n \to \infty} \RewSubC$.  Then %and define 
	\begin{equation}
	\label{eq:hatRnInf_matrix}
		\hatRinf = 
		\begin{bmatrix}
			0 & \RewSubCi{\infty} \\
			0 & 0 
		\end{bmatrix}
	\end{equation}	
	where
	\begin{align}
		\label{eq:RewSubCInf}
		\RewSubCi{\infty} &= \fundMat(\HSubTransAbs + \HSubTransTrans\fundMat\transR).
	\end{align}
\end{theorem}

\begin{proof}	
	We begin by writing
	\setcounter{cnt}{1}
	\begin{align}
		\RewSubCi{\infty} &\stackrel{(\alph{cnt})}{=} \lim_{n \to \infty} \left( \fundMat(I - \transQ^{n})\HSubTransAbs + \fundMat(I - \transQ^{n-1})\HSubTransTrans\fundMat\transR - \sum_{i = 0}^{n-2}\transQ^{i}\HSubTransTrans\fundMat\transQ^{n - i - 1}\transR\right) \\
		\addtocounter{cnt}{1}
		\label{eq:RewSubCInf_summation}
		&\stackrel{(\alph{cnt})}{=}  \fundMat\HSubTransAbs + \fundMat\HSubTransTrans\fundMat\transR - \lim_{n \to \infty}  \sum_{i = 0}^{n-2}\transQ^{i}\HSubTransTrans\fundMat\transQ^{n - i - 1}\transR 
		\addtocounter{cnt}{1}
%		\RewSubDi{\infty} &= \lim_{n \to \infty} \sum_{i = 0}^{n-1}\transQ^{i}\HSubTransTrans\transQ^{n - i - 1} \\
%		&= \lim_{n \to \infty} \left( \transQ^{n-1} \HSubTransTrans + \sum_{i = 0}^{n-2}\transQ^{i}\HSubTransTrans\transQ^{n - i - 1} \right)\\
	\end{align}
	where 
	\begin{enumerate}[(a)]
		\item follows from Lemma~\ref{lem:hatRn_matrix}
		\item follows from Lemma~\ref{lem:transQn_zero}
	\end{enumerate}
	In order to prove~\eqref{eq:RewSubCInf}, we must now show the last term in~\eqref{eq:RewSubCInf_summation} converges to the zero matrix.  We use Corollary~\ref{cor:transQ_norm} and Lemma~\ref{lem:summation_matrix_zero} in Appendix~\ref{sec:matrix_properties} to conclude that this is indeed the case.  
	
	For $\RewSubDi{\infty}$, we have that 
	\begin{align}
		\RewSubDi{\infty} &= \lim_{n \to \infty} \sum_{i = 0}^{n-1}\transQ^{i}\HSubTransTrans\transQ^{n - i - 1} \\
		\label{eq:RewSubDi_inf_proof}
		&= \lim_{n \to \infty} \left( \transQ^{n-1} \HSubTransTrans + \sum_{i = 0}^{n-2}\transQ^{i}\HSubTransTrans\transQ^{n - i - 1} \right)
		\addtocounter{cnt}{1}
	\end{align}
	We conclude from Lemma~\ref{lem:transQn_zero} that the first term in~\eqref{eq:RewSubDi_inf_proof} converges to the zero matrix.  Finally we again use Corollary~\ref{cor:transQ_norm} and Lemma~\ref{lem:summation_matrix_zero} in Appendix~\ref{sec:matrix_properties} to conclude that the second term in~\eqref{eq:RewSubDi_inf_proof} also converges to the zero matrix so that $\RewSubDi{\infty} = 0$.
\end{proof}
