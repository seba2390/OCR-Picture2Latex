\section{Simulations}
\label{sec:simulations}

We demonstrate the performance of the algorithms in Section~\ref{sec:three_users} with simulations.  We consider distortions for which $d_i = \epsilon_i^{2}$ for all $i \in \mathcal{U}$.  In Fig.~\ref{fig:t_star}, we choose a blocklength of $N=10^7$, fix $\epsilon_1 = 0.3$, $\epsilon_2 = 0.4$ and vary $\epsilon_3$ on the $x$-axis while plotting the total number of channel symbols sent per source symbol on the $y$-axis for different coding schemes.  
%instantly decodable, distortion-innovative transmissions sent on the $y$-axis.  
%
%\newlength\figureheight 
%\newlength\figurewidth 
\begin{figure}
	\centering
	\setlength\figurewidth{2.65in} 
\setlength\figureheight{2.07in} 
%	\includegraphics[scale=0.5]{fig/tstar_N_10e7_eps1_3_eps2_4-fixed}
	\input{fig/simulations/analog.tikz}
	\caption{The normalized number of channel symbols sent per source symbol.  We fix $\epsilon_1=0.3$, $\epsilon_2=0.4$ and vary $\epsilon_3$ on the x-axis.  We show the number of uncoded transmissions sent via~\eqref{eq:LP}, alongside what is obtained from simulations  for a blocklength of $N=10^7$.  Finally, we also plot the latency required to achieve the equivalent distortion values \emph{without} feedback based on a segmentation-based coding scheme.}
%	\caption{The total normalized number of instantly decodable, distortion-innovative packets that can be sent.}
	\label{fig:t_star}
\end{figure}
%
The first coding scheme plotted is based on simulations and plots the total number of instantly decodable, distortion-innovative transmissions sent.  Alongside this curve, we plot the number of possible innovative transmissions suggested by the solution of~\eqref{eq:LP}.  We observe a close resemblance in this plot and the empirical simulation curve. Finally, we know that although each user may not have their final distortion constraint met after $Nt^{*}$ transmissions, the provisional distortion they \emph{do} achieve after $Nt^{*}$ transmissions is optimal.  Say instead, that we were given these provisional distortion values from the onset and asked what latency would be required to achieve these distortions if feedback were not available.  The final plot shows this required latency for the segmentation-based coding scheme of~\cite{LTKS_ISIT14}, which does not incorporate feedback. The gap between these curves is indicative of the benefit that feedback provides.
%In order to achieve the same distortion values with We also plot the latency required by the segmentation-based coding scheme of~\cite{} that does not incorporate feedback, if it were used to achieve the optimal distortions at $t^{*}$.  The gap between these curves is indicative of the benefit that feedback allows.

In practice, the values of $\epsilon_i$ are much lower than what we have chosen.  The values for $\epsilon_3$, for example, were deliberately chosen to be high ($\epsilon_3 \geq 0.85$) as we have found that for values even as high as $\epsilon_3 = 0.8$, we achieve point-to-point optimality for all users (see Theorems~\ref{thm:w_minus} and~\ref{thm:all}).  If we increase $\epsilon_3$ even higher however, we observe a situation where many symbols destined to user~3 are erased, and so when the stopping condition of the algorithm in Section~\ref{subsec:instantly_decodable} is reached, we are left with queues $Q_3$, \Q{1,3} and \Q{2,3}.
%
\begin{figure}
	\centering
	\setlength\figurewidth{2.65in} 
	\setlength\figureheight{2.07in} 
%	\includegraphics[scale=0.5]{fig/ttotal_N_10e7_eps1_3_eps2_4}
	\input{fig/simulations/latency.tikz}
	\caption{The latency when we are forced to invoke the algorithms of Section~\ref{subsec:non_instant_coding}.  We fix $d_i = \epsilon_{i}^{2}$, $\epsilon_1 = 0.3$, $\epsilon_2 = 0.4$ and vary $\epsilon_3$ on the x-axis.  }
	\label{fig:t_total}
\end{figure}
%
When this occurs, we have not yet satisfied all users, and so we resort to the coding schemes proposed in Section~\ref{subsec:non_instant_coding}.  Fig.~\ref{fig:t_total} shows a plot where each point required the invocation of the queue preprocessing scheme for channel coding in Section~\ref{subsec:channel_coding}.  It plots the \emph{overall} latency required to achieve distortions $d_i = \epsilon_i^{2}$ for the values of $\epsilon_i$ given earlier.  Alternatively, if the chaining algorithm of Section~\ref{subsubsec:chaining_algorithm} is used, we find that the latency coincides with the outer bound.  Again, the segmentation-based scheme is provided as a benchmark along with the outer bound $w^{+}\dvec$.

