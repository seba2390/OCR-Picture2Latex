\subsection{Instantly Decodable, Distortion-Innovative Transmissions}
\label{subsec:instantly_decodable}

%The first part of the coding scheme is shown in Algorithm~\ref{alg:three_users} of Appendix~\ref{app:pseudocode}. 
The first part of the code involves
%There is again 
an initial phase that transmits each source symbol until at least one user receives it.  No further processing of source symbol $S(t)$ is done if it was received by all users.  Let $\mathcal{E}(T) \subset \mathcal{U} \triangleq \{1, 2, 3\}$ represent the set of users whose channel output was an erasure at time $T$.  We will at times drop the time index $T$ when it is obvious from the context.  Then after the transmission at time $T$, we place $S(t)$ in queue $Q_{\mathcal{E}(T)}$ so that, in general, $Q_{U}$ maintains a record of which symbols were not received by all users $i \in U$.
%Note that by definition $Q_{U} = Q_{\{1, 2, 3\}}$ are always empty since we have a

%If at any time during or after the transmission of uncoded source symbols we should find that some user has already met their distortion requirement, 

%%\algdef{SE}[EVENT]{Event}{EndEvent}[1]{\textbf{upon event}\ #1\ \algorithmicdo}{\algorithmicend\ \textbf{event}}%
%%\algtext*{EndEvent}
%%
%%\begin{algorithm}
%%\caption{Coding for Three Users with Feedback}\label{alg:three_users}
%%\begin{algorithmic}[1]
%%\State \textbf{Initialize}: $z_i \gets 0, \forall i \in \mathcal{U}$  % \Comment{Packets received by each user}
%%\ForAll {$t \in [N]$}
%%    \While {$\nexists \ i \in \mathcal{U} \ s.t.\ N_i = 0$}
%%%    \State Send $S(t)$ until at least one user receives it
%%    	\State Transmit $S(t)$
%%    \EndWhile
%%%    \If {$\exists \ i \in \mathcal{U}, T \in \mathbb{N} \ s.t.\ N_i(T) = 0$}
%%%    \If {$\exists \ i \in \mathcal{U} \ s.t.\ N_i = 0$}
%%        \State $Q_{\mathcal{E}} \gets Q_{\mathcal{E}} \cup \{S(t)\}$
%%        \State $z_j\gets z_j + 1 \qquad \forall \ j \ s.t.\  N_j = 0$
%%%    \EndIf
%%\EndFor
%%
%%%\While{$\exists \ \textrm{distinct} i, j, k \in \mathcal{U}, i\neq j \neq k \ s.t.\ Q_i \neq \varnothing \ \textbf{and} \ \Q{j,k} \neq \varnothing$}
%%\While{$\exists \ \textrm{distinct} \ i, j, k \in \mathcal{U}, \ s.t.\ Q_i \neq \varnothing \ \textbf{and} \ \Q{j,k} \neq \varnothing$}
%%    \While {$\nexists \ l \in \mathcal{U} \ s.t.\ N_l = 0$}
%%%	\State Let $q_i \in Q_i$, $q_{j,k} \in \Q{j,k}$ 
%%%       \State Transmit $q_i \oplus q_{j,k}$ until at least one user receives it  
%%       \State Transmit $q_i \oplus q_{j,k}$ where $q_i \in Q_i$, $q_{j,k} \in \Q{j,k}$ 
%%    \EndWhile
%%    \State $z_l\gets z_l + 1 \qquad \forall \ l \ s.t.\  N_l = 0$
%%    \If {$N_i = 0$}
%%    	\State $Q_i \gets Q_i \setminus \{q_i\}$
%%    \EndIf
%%    \If {$N_j = 0 \ \textbf{and} \ N_k = 1$}
%%    	\State $Q_k \gets Q_k \cup \{q_{j, k}\}$
%%       \State $\Q{j,k} \gets \Q{j, k} \setminus \{q_{j,k}\}$
%%    \ElsIf {$N_j = 1 \ \textbf{and} \ N_k = 0$}
%%    	\State $Q_j \gets Q_j \cup \{q_{j, k}\}$
%%	\State $\Q{j,k} \gets \Q{j, k} \setminus \{q_j,k\}$
%%    \ElsIf {$N_j = 0 \ \textbf{and} \ N_k = 0$}
%%    	\State $\Q{j,k} \gets \Q{j, k} \setminus \{q_j,k\}$
%%    \EndIf    
%%\EndWhile
%%
%%\While{$Q_i \neq \varnothing \  \forall \ i \in \mathcal{U}$}
%%	\State Transmit $q_1 \oplus q_{2} \oplus q_{3}$, where $q_i \in Q_i \ \forall i \in \mathcal{U}$
%%\EndWhile
%%
%%\Event{$ \exists \ i \ s.t.\ z_i \geq 1- d_i  $} %\Comment{One user finished}
%%%	\State Let $j, k \in \mathcal{U} \setminus \{i\}, \ s.t.\ j \neq k$
%%	\State $Q_j \gets Q_j \cup \Q{i,j} \qquad \forall \ j \in \mathcal{U} \setminus \{i\}$
%%	\State \textbf{run} two-user algorithm of Section~\ref{sec:two_users}
%%\EndEvent
%%\end{algorithmic}
%%\end{algorithm}

The algorithm contains a subroutine that acts as an \emph{event handler}.  In the event that any user~$i$'s distortion constraint is met during or after the uncoded transmissions, the algorithm's control flow is passed to this subroutine.  Within this subroutine, we simply use the two-user algorithm outlined in~\cite{TMKS_TIT20} to serve the remaining users.  In~\cite{TMKS_TIT20}, it was shown that for the two-receiver version of the problem we consider, the optimal latency can be achieved regardless of either user's distortion constraint.  The coding scheme maintained queues of symbols $Q_{\{1\}}$, $Q_{\{2\}}$, and $Q_{\{1, 2\}}$ where the symbols in $Q_U$ are instantly-decodable, distortion-innovative for all users $i \in U$.  To incorporate~\cite{TMKS_TIT20} into our current coding scheme in the event that user~$i$ has met their distortion constraint, we first discard $Q_i$ since users $j$ and $k$ have already received all symbols from this queue for $j, k \in \mathcal{U} \setminus\{i\}, j\neq k$.  We then merge queues $Q_j$ and $Q_{\{i, j\}}$ and queues $Q_k$ and $Q_{\{i, k\}}$, since user~$i$ is no longer relevant, and symbols in $Q_{\{i, j\}}$ can be regarded as symbols that only user~$j$ needs at this point.  Finally, since $Q_{\{j,k\}}$ contains source symbols that neither user has received so far, we can  treat them as source symbols that have yet to be sent uncoded in the algorithm of~\cite{TMKS_TIT20}.

If there are no users with distortion constraints met however, the algorithm continues by recognizing network coding opportunities and sending linear combinations in a manner that is similar to that in~\cite{TMKS_TIT20}.  First, consider $Q_1$ and $Q_{\{2, 3\}}$.  If they are both non-empty, let $q_1 \in Q_1$ and $q_{2,3} \in \Q{2,3}$.  Notice that if we transmit $q_1 \oplus q_{2,3}$, since user~1 has access to $q_{2,3}$, and user~2 and user~3 have access to $q_1$, every user is able to receive an instantly decodable, distortion-innovative symbol.  A symbol is removed from a queue if any user for whom the queue is intended receives the linear combination.  However, if we have a type of situation where only user~2 receives the linear combination but not user~3, then $q_{2, 3}$ is transferred from \Q{2,3} to $Q_3$, since only user~3 is in need of this symbol now.  We continue this procedure with queues $Q_2$ and \Q{1,3} and queues $Q_3$ and \Q{1,2} to the extent that it is possible, i.e., for as long as the relevant queues required remain non-empty. 
%
Upon the emptying of these queues, we determine if we can send linear combinations of the form $q_1 \oplus q_2 \oplus q_3$,  which are similarly instantly decodable and distortion-innovative.  If this is possible, we again continue to do so until no longer possible.

We mention that in general, for $n$ users, we can similarly maintain queues that manage which symbols are erased by which users.  In this case, we can send an instantly decodable, distortion-innovative symbol if there exists non-empty queues $Q_{\Gamma_1}, Q_{\Gamma_2}, \ldots, Q_{\Gamma_m}$ such that $\cup_{i=1}^{m}\Gamma_i = [n]$ and $\sum_{i=1}^{m}|\Gamma_i| = n$, where $[n]$ denotes the set $\{1, 2, \ldots, n\}$.  These two conditions ensure that the index~$j$ for user~$j$ appears in exactly one $\Gamma_l, l \in [m]$.  Thus, user~$j$ is in possession of all symbols in these queues except for $Q_{\Gamma_l}$.
We then send the linear combination $\sum_{i = 1}^{m} q_{\Gamma_i}$, where $q_{\Gamma_i} \in Q_{\Gamma_i}$. We note however, that this requires an amount of queues that is exponential in the number of users.

%\algdef{SE}[EVENT]{Event}{EndEvent}[1]{\textbf{upon event}\ #1\ \algorithmicdo}{\algorithmicend\ \textbf{event}}%
%\algtext*{EndEvent}
%
%
%\begin{algorithm}
%\caption{Coding for Three Users with Feedback}\label{alg:three_users}
%\begin{algorithmic}[1]
%\State \textbf{Initialize}: $z_i \gets 0, \forall i \in \mathcal{U}$  % \Comment{Packets received by each user}
%\ForAll {$t \in [N]$}
%    \While {$\nexists \ i \in \mathcal{U} \ s.t.\ N_i = 0$}
%%    \State Send $S(t)$ until at least one user receives it
%    	\State Transmit $S(t)$
%    \EndWhile
%%    \If {$\exists \ i \in \mathcal{U}, T \in \mathbb{N} \ s.t.\ N_i(T) = 0$}
%%    \If {$\exists \ i \in \mathcal{U} \ s.t.\ N_i = 0$}
%        \State $Q_{\mathcal{E}} \gets Q_{\mathcal{E}} \cup \{S(t)\}$
%        \State $z_j\gets z_j + 1 \qquad \forall \ j \ s.t.\  N_j = 0$
%%    \EndIf
%\EndFor
%
%%\While{$\exists \ \textrm{distinct} i, j, k \in \mathcal{U}, i\neq j \neq k \ s.t.\ Q_i \neq \varnothing \ \textbf{and} \ \Q{j,k} \neq \varnothing$}
%\While{$\exists \ \textrm{distinct} \ i, j, k \in \mathcal{U}, \ s.t.\ Q_i \neq \varnothing \ \textbf{and} \ \Q{j,k} \neq \varnothing$}
%    \While {$\nexists \ l \in \mathcal{U} \ s.t.\ N_l = 0$}
%%	\State Let $q_i \in Q_i$, $q_{j,k} \in \Q{j,k}$ 
%%       \State Transmit $q_i \oplus q_{j,k}$ until at least one user receives it  
%       \State Transmit $q_i \oplus q_{j,k}$ where $q_i \in Q_i$, $q_{j,k} \in \Q{j,k}$ 
%    \EndWhile
%    \State $z_l\gets z_l + 1 \qquad \forall \ l \ s.t.\  N_l = 0$
%    \If {$N_i = 0$}
%    	\State $Q_i \gets Q_i \setminus \{q_i\}$
%    \EndIf
%    \If {$N_j = 0 \ \textbf{and} \ N_k = 1$}
%    	\State $Q_k \gets Q_k \cup \{q_{j, k}\}$
%       \State $\Q{j,k} \gets \Q{j, k} \setminus \{q_{j,k}\}$
%    \ElsIf {$N_j = 1 \ \textbf{and} \ N_k = 0$}
%    	\State $Q_j \gets Q_j \cup \{q_{j, k}\}$
%	\State $\Q{j,k} \gets \Q{j, k} \setminus \{q_j,k\}$
%    \ElsIf {$N_j = 0 \ \textbf{and} \ N_k = 0$}
%    	\State $\Q{j,k} \gets \Q{j, k} \setminus \{q_j,k\}$
%    \EndIf    
%\EndWhile
%
%\While{$Q_i \neq \varnothing \  \forall \ i \in \mathcal{U}$}
%	\State Transmit $q_1 \oplus q_{2} \oplus q_{3}$, where $q_i \in Q_i \ \forall i \in \mathcal{U}$
%\EndWhile
%
%\Event{$ \exists \ i \ s.t.\ z_i \geq 1- d_i  $} %\Comment{One user finished}
%%	\State Let $j, k \in \mathcal{U} \setminus \{i\}, \ s.t.\ j \neq k$
%	\State $Q_j \gets Q_j \cup \Q{i,j} \qquad \forall \ j \in \mathcal{U} \setminus \{i\}$
%	\State \textbf{run} two-user algorithm of Section~\ref{sec:two_users}
%\EndEvent
%\end{algorithmic}
%\end{algorithm}



%\begin{algorithm}
%\caption{Coding for Three Users with Feedback}\label{lag:three_users}
%\begin{algorithmic}[1]
%\For {$t = 1, 2, \ldots N$ \textbf{AND} $R_i < (1 - d_i)N, i \in \{1, 2\}$}
%    \State Send $S(t)$ until at least one user receives it
%    \If {$\exists T, i s.t. N_i = 1, N_j = 0$ for some $i, j \in \{1, 2\}, i \neq j$}
%        \State $Q_i\gets Q_i \cup \{S(t)\}$
%    \EndIf
%    \If {$N_i = 0$}
%        \State $R_i\gets R_i + 1$
%    \EndIf
%\EndFor
%
%\While{$R_i < (1 - d_i) N, i \in \{1, 2\}$}
%    \If {$R_i < (1 - d_i)N$ \textbf{AND} $Q_i \neq \emptyset$}
%        \State Let $q_i \in Q_i$
%    \Else
%    	\State $q_i = 0$
%    \EndIf	
%    \State Send $q_1 \oplus q_2$ until at least one user receives it
%    \If {$N_i = 0$}
%        \State $R_i\gets R_i + 1$
%        \State $Q_i \gets Q_i \setminus \{q_i\}$
%    \EndIf    
%\EndWhile
%
%
%\end{algorithmic}
%\end{algorithm}

