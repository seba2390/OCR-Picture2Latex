\begin{abstract}
Pre-trained machine learning (ML) models have shown great performance for a wide range of applications, in particular in natural language processing (NLP) and computer vision (CV).
Here, we study how pre-training could be used for scientific machine learning (SciML) applications, specifically in the context of transfer learning.
We study the transfer behavior of these models as
(i) the pre-trained model size is scaled,
(ii) the downstream training dataset size is scaled,
(iii) the physics parameters are systematically pushed out of distribution, and 
(iv) how a single model pre-trained on a mixture of different physics problems can be adapted to various downstream applications. 
We find that---when fine-tuned appropriately---transfer learning can help reach desired accuracy levels with orders of magnitude fewer downstream examples (across different tasks that can even be out-of-distribution) than training from scratch, with consistent behaviour across a wide range of downstream examples. 
We also find that fine-tuning these models yields more performance gains as model size increases, compared to training from scratch on new downstream tasks. 
These results hold for a broad range of PDE learning tasks.
All in all, our results demonstrate the potential of the ``pre-train and fine-tune'' paradigm for SciML problems, demonstrating a path towards building SciML foundation models.
We open-source our code at \cite{github}.
\end{abstract}
