\documentclass[12pt,aps,superscriptaddress,prl]{revtex4-1}  % for review and submission
%\documentclass[aps,preprint,showpacs,superscriptaddress,groupedaddress]{revtex4}  % for double-spaced preprint
\usepackage{graphicx}  % needed for figures
\usepackage{dcolumn}   % needed for some tables
\usepackage{bm}        % for math
\usepackage{amssymb}   % for math
\usepackage{amsmath}   % for math
\usepackage{amsthm}    % for math
\usepackage{xcolor}    % pretty self explanatory
\usepackage{tikz}      % for drawing neural network
\usepackage{ifthen}    % for labelling edges easily
\usepackage{multirow}  % for tables 
\usepackage{tabu}      % for tables    
\usepackage{subfigure}
\usepackage{bbold}
\usepackage{xspace}
\usepackage[unicode=true,
linktocpage,
linkbordercolor={0.5 0.5 1},
citebordercolor={0.5 1 0.5},
linkcolor=blue]{hyperref}
% avoids incorrect hyphenation, added Nov/08 by SSR
\hyphenation{ALPGEN}
\hyphenation{EVTGEN}
\hyphenation{PYTHIA}

\newcommand{\sro}{Sr$_2$RuO$_4$\xspace}
\newcommand{\Tc}{$T_\mathrm{c}$\xspace}
\newcommand{\Tcs}{$T_\mathrm{c}$s\xspace}
\newcommand{\highTc}{high-$T_\mathrm{c}$\xspace}
\newcommand{\brad}[1]{{\color{red} #1}}

\begin{document}

\title{Strong Increase in Ultrasound Attenuation Below T$_\mathrm{c}$ in \sro: Possible Evidence for Domains}

\date{\today}
\author{Sayak Ghosh}
\affiliation{Laboratory of Atomic and Solid State Physics, Cornell University, Ithaca, NY 14853, USA}
\author{Thomas G. Kiely}
\affiliation{Laboratory of Atomic and Solid State Physics, Cornell University, Ithaca, NY 14853, USA}
\author{Arkady Shekhter}
\affiliation{National High Magnetic Field Laboratory, Florida State University, Tallahassee, FL 32310, USA}
\author{F. Jerzembeck}
\affiliation{Max Planck Institute for Chemical Physics of Solids, Dresden, Germany}
\author{N. Kikugawa}
\affiliation{National Institute for Materials Science, Tsukuba, Ibaraki 305-0003, Japan}
\author{Dmitry A. Sokolov}
\affiliation{Max Planck Institute for Chemical Physics of Solids, Dresden, Germany}
\author{A. P. Mackenzie}
\affiliation{Max Planck Institute for Chemical Physics of Solids, Dresden, Germany}
\affiliation{SUPA, School of Physics and Astronomy, University of St Andrews, North Haugh, St Andrews KY16 9SS, UK}
\author{B.~J.~Ramshaw}
\email{bradramshaw@cornell.edu}
\affiliation{Laboratory of Atomic and Solid State Physics, Cornell University, Ithaca, NY 14853, USA}	
\maketitle

\newpage


\textbf{Recent experiments suggest that the superconducting order parameter of \sro has two components. A two-component order parameter has multiple degrees of freedom in the superconducting state that can result in low-energy collective modes or the formation of domain walls---a possibility that would explain a number of experimental observations including the smallness of the time reversal symmetry breaking signal at \Tc and telegraph noise in critical current experiments. We perform ultrasound attenuation measurements across the superconducting transition of \sro using resonant ultrasound spectroscopy (RUS). We find that the attenuation for compressional sound increases by a factor of seven immediately below \Tc, in sharp contrast with what is found in both conventional ($s$-wave) and high-\Tc ($d$-wave) superconductors. We find our observations to be most consistent with the presence of domain walls between different configurations of the superconducting state. The fact that we observe an increase in sound attenuation for compressional strains, and not for shear strains, suggests an inhomogeneous superconducting state formed of two distinct, accidentally-degenerate superconducting order parameters that are not related to each other by symmetry. Whatever the mechanism, a factor of seven increase in sound attenuation is a singular characteristic with which any potential theory of the superconductivity in \sro must be reconciled.}

\section*{Introduction}

One firm, if perhaps counter-intuitive, prediction of Bardeen, Cooper, and Schrieffer (BCS) theory is the contrasting behavior of the nuclear spin relaxation rate, $1/T_1$, and the ultrasonic attenuation, $\alpha$ \cite{BCS1957}. Upon cooling from the normal state to the superconducting (SC) state, one might expect both $1/T_1$ and $\alpha$ to decrease as both processes involve the scattering of normal quasiparticles. In the SC state, however, Cooper pairing produces correlations between quasiparticles of opposite spin and momentum. These correlations produce ``coherence factors'' that add constructively for nuclear relaxation and produce a peak---the Hebel-Slichter peak---in $1/T_1$ immediately below \Tc \cite{SlichterPRL1957}. In contrast, the coherence factors add destructively for sound attenuation and there is an immediate drop in $\alpha$ below \Tc \cite{MorsePRL1959}. These experiments provided some of the strongest early evidence for the validity of BCS theory \cite{BCS1957}, and the drop in sound attenuation below \Tc was subsequently confirmed in many elemental superconductors \cite{LevyPRL1963,LeibowitzPRL1964,ClaibornePRL1965,FossheimPRL1967}.

It came as a surprise, then, when peaks in the sound attenuation were discovered below \Tc in two heavy-fermion superconductors: UPt$_3$ and UBe$_{13}$ \cite{BatloggPRL1985,GoldingPRL1985,MullerSolStComm1986}. Specifically, peaks were observed in the longitudinal sound attenuation---when the sound propagation vector $\mathbf{q}$ is parallel to the sound polarization $\mathbf{u}$: ($\mathbf{q}\parallel\mathbf{u}$). Transverse sound attenuation ($\mathbf{q}\perp\mathbf{u}$), on the other hand, showed no peak below \Tc but instead decreased with power law dependencies on $T$ that were ultimately understood in terms of the presence of nodes in the SC gap \cite{MorenoPRB1996}. Various theoretical proposals were put forward to understand the peaks in the longitudinal sound attenuation, including collective modes, domain-wall friction, and coherence-factors \cite{MiyakePRL1986,MonienSol1987,JoyntPRL1986,CoffeyPRB1986}, but the particular mechanisms for UPt$_3$ and UBe$_{13}$ were never pinned down (see \citet{SigristRMP1991} for a review). What is clear, however, is that a peak in sound attenuation below \Tc is not a prediction of BCS theory and surely indicates unconventional superconductivity.

\begin{figure*}
	\centering
	\includegraphics[width=0.99\linewidth]{Fig1.pdf}
	\caption{\label{fig:spectra} Measuring ultrasonic attenuation with resonant ultrasound spectroscopy. (a) The \sro unit cell under a deformation corresponding to the longitudinal strain $\epsilon_{xx}$, associated with the elastic constant $c_{11}$. This mode is a superposition of pure compression $\epsilon_{xx}+\epsilon_{yy}$ and pure shear $\epsilon_{xx}-\epsilon_{yy}$, associated with the elastic constants $(c_{11}+c_{12})/2$ and $(c_{11}-c_{12})/2$, respectively. (b) Resonant ultrasound spectrum of \sro between 2.2-2.8 MHz. $X(\omega)$ and $Y(\omega)$ are the real and the imaginary parts of the response. The boxed resonance is shown in detail in (c). (c) Zoom-in on the resonance near 2.34 MHz. The center of the resonance and the linewidth are indicated. Inset shows the same resonance plotted in complex plane and fit to a circle---$z_c$ denotes the center of the circle.} 
	
	%The extrema of the resonance ($\omega\ll\omega_0$, $\omega\gg\omega_0$) approach the cyan point asymptotically in $\omega$. The resonance frequency $\omega_0$ and the linewidth $\Gamma$ are obtained precisely by fitting the function $\tan\theta= (\omega-\omega_0)/(\Gamma/2)$, with $\theta$ as indicated on the plot \cite{ShekhterNat2013}.}
\end{figure*}

The superconductivity of \sro has many unconventional aspects, including time reversal symmetry (TRS) breaking \cite{luke1998time,xia2006high,KidwingiraScience2006}, the presence of nodal quasiparticles \cite{LupienPRL2001,HassingerPRX2017,SharmaPNAS2020}, and two components in its SC order parameter \cite{GhoshNatPhys2020,BenhabibNatPhys2020}. These observations have led to various recent theoretical proposals for the SC state in \sro \cite{AlinePRB2019,romer2019knight,RoisingPRR2019,ScaffidiArxiv2020,SuhPRR2020,KivelsonNPJ2020,WillaPRB2021}, requiring further experimental inputs to differentiate between them. Not only should the coherence factors differ for \sro compared to the $s-$wave BCS case, but there is the possibility of low-energy collective modes \cite{chung2012charge} and domain-wall motion \cite{Yuan2021}, all of which could be observable in the ultrasonic attenuation.

%{\color{red}should we cite recent \sro theory papers here? probably shows it's an active field right now. } {\color{blue} Yes, add these} 

Prior ultrasonic attenuation measurements on \sro reported a power-law temperature dependence of the transverse sound attenuation, interpreted as evidence for nodes in the gap \cite{LupienPRL2001}, but found no other unconventional behaviour. This may be in part due to the specific ultrasound technique employed: pulse-echo ultrasound. While pulse-echo can measure a pure shear-strain response in the transverse configuration, the in-plane longitudinal configuration is a combination of both compression strain and shear strain in a tetragonal crystal like \sro \cite{Brugger1965}. Specifically, the L100 mode measures the elastic constant $c_{11}$, which is a mixture of pure compression, $(c_{11}+c_{12})/2$, and pure shear, $(c_{11}-c_{12})/2$ (see \autoref{fig:spectra}(a)). Shear and compression strains couple to physical processes in fundamentally different ways and thus effects that couple exclusively to compressional sound may have been missed in previous measurements.

%Shear attenuation is known to be large in \sro and grows quadratically with frequency \cite{pippard1955cxxii}. This leads to shear attenuation overwhelming the the compressional attenuation in a longitudinal pulse-echo measurement. We measure the ultrasound attenuation in \sro at relatively low frequencies ($\sim2$ MHz), which reduces the contribution of the shear attenuation. Our experimental technique---resonant ultrasound spectroscopy---allows us to completely separate the compression and shear responses, and we find novel features in the compressional attenuation not reported previously. Past theoretical works on sound attenuation in \sro primarily assumed a $(p_x+ip_y)\hat{z}$ order parameter \cite{Tewordt1999,TewordtPRB2000}, which has recently been ruled out\cite{PustogowNat2019}. We therefore believe our results will motivate work on understanding sound attenuation in \sro in light of the recently proposed OP candidates.

%However, the experiments in Ref.\cite{LupienPRL2001} were at relatively high ultrasound frequencies, which generally suppress the intrinsic thermodynamic response \cite{Nyhus2002}. For example, the magnitude of discontinuity in the elastic constant $c_{66}$, which establishes the two-component nature of the OP, reported in Refs.\cite{GhoshNatPhys2020} and \cite{BenhabibNatPhys2020} differ by a factor of 50, possibly due to the experiments operating at 2 MHz and 169 MHz, respectively.

\section*{Experiment}

We have measured the ultrasonic attenuation of \sro across \Tc using an ultrasound technique distinct from pulse-echo ultrasound---resonant ultrasound spectroscopy (RUS). RUS allows us to obtain the attenuation in all the independent symmetry channels in a single experiment (i.e. for all 5 irreducible representations of strain in \sro). See \citet{GhoshNatPhys2020} for details of our custom built low-temperature RUS apparatus. The high-quality \sro crystal used in this experiment was grown by the floating zone method---more details about the sample growth can be found in \citet{BobowskiCondMat2019}. A single crystal was precision-cut along the [110], [1$\bar{1}$0] and [001] directions and polished to the dimensions 1.50~mm $\times$ 1.60~mm $\times$ 1.44~mm,  with 1.44~mm along the tetragonal $c$ axis. The sample quality was characterized by heat capacity and AC susceptibility measurements, as reported in \citet{GhoshNatPhys2020}. The SC \Tc measured by these techniques---approximately 1.43 K---agrees well with the \Tc seen in our RUS experiment, indicating that the sample underwent uniform cooling during the experiment.
%\begin{figure}
%	\centering
%	\includegraphics[width=0.99\linewidth]{Fig2.png}
%	\caption{\label{fig:data} Normalized resonance linewidth $\Gamma/\omega_0$---proportional to the sound attenuation coefficient---of predominantly (a) compressional and (b) shear modes of a single-crystal \sro specimen across \Tc. The dashed line shows \Tc determined independently by susceptibility measurements. The linewidths of the compressional modes shows a peak immediately below \Tc before slowly decreasing at lower temperatures. The linewidths of the shear modes start decreasing immediately below \Tc. Note that each mode is a superposition of different amounts of compression and shear elastic strain: the decomposition into pure symmetry channels is shown in \autoref{fig:visc}.}
%\end{figure}

RUS measures the mechanical resonances of a three-dimensional solid. The frequencies of these resonances depend on the elastic moduli, density, and geometry of the sample, while the widths of these resonances are determined by the ultrasonic attenuation \cite{RamshawPNAS,GhoshSciAdv2020}. Because each resonance mode is a superposition of multiple kinds of strain, the attenuation in all strain channels can be extracted by measuring a sufficient number of resonances---typically 2 or 3 times the number of unique strains (of which there are 5 for \sro). 

A typical RUS spectrum from our \sro sample is shown in \autoref{fig:spectra}(b) (see methods for details of the measurement.) Each resonance can be modeled as the response $Z(\omega)$ of a damped harmonic oscillator driven at frequency $\omega$ (see \autoref{fig:spectra}(c)),
\begin{equation}
	Z(\omega)=X(\omega)+iY(\omega)= Ae^{i\phi}/((\omega-\omega_0)+i\Gamma)
\end{equation}
where $X$ and $Y$ are the real and imaginary parts of the response, and $A$, $\Gamma$, and $\phi$ are the amplitude, linewidth, and phase, respectively. The real and imaginary parts of the response form a circle in the complex plane. The response is measured at a set of frequencies that space the data points evenly around this circle: this is the most efficient way to precisely determine the resonant frequency $\omega_0$ and the linewidth $\Gamma$ in a finite time (see \citet{ShekhterNat2013} for details of the fitting procedure). We plot the temperature dependence of the linewidth of all our experimentally measured resonances through \Tc in the SI. For comparison, the attenuation $\alpha$ measured in conventional pulse-echo ultrasound is related to the resonance linewidth via $\alpha = \Gamma/v$, where $v$ is the sound velocity.

%We plot the data and fit for a particular resonance in \autoref{fig:spectra}(d), from which we can extract $\omega_0$ and $\Gamma$ with high precision.

\section*{Results}

When the sound wavelength, $\lambda = \frac{2\pi}{q}$, is much longer than the electronic mean free path $l$, i.e. when $ql\ll1$, the electron-phonon system is said to be in the `hydrodynamic' limit \cite{Khan1987} (this is different than the hydrodynamic limit of electron transport). Given that the best \sro has a mean free path that is at most of order a couple of microns, and that our experimental wavelengths are of the order of 1 mm, we are well within the hydrodynamic limit. In this regime, we can express the linewidth $\Gamma$ of a resonance $\omega_0$ as,
\begin{equation}
	\frac{\Gamma}{\omega_0^2}=\frac{1}{2}\sum_{j}\alpha_j\frac{\eta_j}{c_j},
	\label{eqn:visc}
\end{equation} 
where $\eta_j$ and $c_j$ are the independent components of the viscosity and elastic moduli tensors, respectively (see SI for details). The $\alpha$ coefficients define the composition of a resonance, such that $\alpha_j=\partial(\ln \omega_0^2)/\partial(\ln c_j)$ and $\sum_{j}\alpha_j=1 $\cite{RamshawPNAS}.



%In this regime, we calculate a viscosity $\eta_i$ for each resonance as 
%\begin{equation}
%	\eta=\frac{\rho v_i^2}{\omega_i^2}\Gamma,
%	\label{eqn:visc}
%\end{equation} 
%where $\omega_i$ is the angular frequency of the $i^{\rm th}$ resonance, $\rho$ is the density of \sro ($\rho=5920$ kg/m$^3$), and $v_i$ is the velocity of the sound for the $i^{\rm th}$ resonance (see SI for details.)

%There are two independent compressional strains, transforming as the $A_{1g}$ irrep, which give rise to the viscosities $(\eta_{11}+\eta_{12})/2$ and $\eta_{33}$, and a third component $\eta_{13}$ which arises due to coupling between the two. For shear strains, there are three viscosities $(\eta_{11}-\eta_{12})/2$, $\eta_{66}$ and $\eta_{44}$, corresponding to strains transforming as the $B_{1g}$, $B_{2g}$ and $E_{g}$ irreps respectively. Physically, compressional strains cause a change in volume of the unit cell without breaking the tetragonal symmetry, whereas shear strains preserve the volume of the unit cell while changing its shape, thereby breaking tetragonal symmetry.

We measured the linewidths of 17 resonances and resolved them into the independent components of the viscosity tensor. The tetragonal symmetry of \sro dictates that there are only six independent components, arising from the five irreducible representations (irreps) of strain in $D_{4h}$ plus one component arising from coupling between the two distinct compression strains \cite{GhoshNatPhys2020}. The six symmetry-resolved components of viscosity in \sro are plotted in \autoref{fig:visc}. 

The shear viscosity $(\eta_{11}-\eta_{12})/2$ decreases below \Tc in a manner similar to what is observed in conventional superconductors \cite{MorsePRL1959,LevyPRL1963}. We find that $(\eta_{11}-\eta_{12})/2$ is much larger than the other two shear viscosities, which is consistent with previous pulse-echo ultrasound experiments \cite{LupienPRL2001,LupienThesis}. On converting attenuation to viscosity, we find relatively good agreement (within a factor of 2) between the resonant ultrasound and pulse-echo measurements of $(\eta_{11}-\eta_{12})/2$. This is particularly non-trivial given that sound attenuation scales as frequency squared in the hydrodynamic regime and the pulse-echo ultrasound measurements were performed at frequencies roughly two orders of magnitude higher than those used in the RUS measurements. The much larger magnitude of $(\eta_{11}-\eta_{12})/2$, in comparison to $\eta_{66}$, may be due to the fact that the $\epsilon_{xx}-\epsilon_{yy}$ strain is associated with pushing the $\gamma$ Fermi surface pocket toward the van Hove singularity \cite{barber2019role}. The small values of $\eta_{44}$ and $\eta_{66}$ are comparable to the experimental noise and any changes at \Tc are too small to resolve at these low frequencies.

In contrast with the rather conventional shear viscosities, the three compressional viscosities each exhibit a strong increase below \Tc. For in-plane compression---the strain that should couple strongest to the rather two-dimensional superconductivity of \sro---the increase is by more than a factor of seven. After peaking just below \Tc, the attenuation slowly decreases as the temperature is lowered. The large increase below \Tc was not observed in previous longitudinal sound attenuation measurements made by pulse-echo ultrasound \cite{LupienPRL2001,LupienThesis}. Longitudinal sound is a mixture of pure shear and pure compression, as shown in \autoref{fig:spectra}(a). At the frequencies where pulse-echo ultrasound is measured---of order 100 MHz---the shear viscosity $(\eta_{11}-\eta_{12})/2$ is at least one order of magnitude larger than the compression viscosity \cite{LupienThesis} and thus completely dominates the sound attenuation. The relative time-scales between the dynamics of the attenuation mechanism and the measurement frequency may also play a role---we will return to this idea later on in the discussion. 

\begin{figure}
	\centering
	\includegraphics[width=0.65\linewidth]{Fig2.pdf}
	\caption{\label{fig:visc} Symmetry-resolved sound attenuation in \sro. (a) Compressional and (b) shear viscosities through \Tc. The irreducible strain corresponding to each viscosity is shown---$\eta_{13}$ arises due to coupling between the two $A_{1g}$ strains. An increase in the compressional viscosities is seen immediately below \Tc, while no such feature is seen in the shear viscosities.}
\end{figure}

\section*{Analysis}

%\sro is one of the most well-studied unconventional superconductors, with a Fermi liquid normal state whose properties are known in great detail \cite{BergemannPRL2000,tamai2019high}. Exceptionally clean single crystals of this material are available, and \Tc is lower than all other energy scales (e.g. lower than the Fermi energy and inter-layer coupling). This has led to the expectation that the intrinsic properties of the superconducting state in \sro should be accessible, and BCS-like weak-coupling theories of unconventional superconductivity should work \cite{KivelsonNPJ2020}. In spite of this, the order parameter symmetry in this material remains a mystery \cite{MackenzieNPJ2017}. New NMR measurements have ruled out most triplet order parameters (OPs), including the once widely-believed $(p_x+ip_y)\hat{z}$ state \cite{PustogowNat2019,ishida2019reduction,ChronisterArxiv2020}, and ultrasound measurements have shown that the OP must have two components \cite{GhoshNatPhys2020,BenhabibNatPhys2020}. Concurrent muon spin resonance ($\mu$SR) measurements have found that time-reversal symmetry breaking (TRSB) onsets at a temperature lower than \Tc under uniaxial strain \cite{GrinenkoArxiv2020}, lending further support to a two-component OP. These results have led to new theoretical proposals \cite{AlinePRB2019,romer2019knight,RoisingPRR2019,ScaffidiArxiv2020,SuhPRR2020,KivelsonNPJ2020,WillaArxiv2020}, and more experimental inputs are needed to differentiate between them.

\begin{figure*}
	\centering
	\includegraphics[width=0.99\linewidth]{Fig3.pdf}
	\caption{\label{fig:fits} Comparison of different mechanisms for sound attenuation in the superconducting state. (a) Attenuation in the superconducting state $\alpha_S$ (normalized by the normal state attenuation $\alpha_N$) for an isotropic $s$-wave gap and a $d_{x^2-y^2}$ gap, calculated within the BCS framework. (b) $\alpha_S/\alpha_N$ for a time reversal symmetry breaking gap below \Tc. A peak is seen at high enough frequencies ($\sim$THz) but not at our experimental frequencies ($\sim$MHz). (c) Attenuation peak at different frequencies due to pair-breaking effects in a $d_{x^2-y^2}$ gap. The inset shows the plot at our experimental frequency in detail---a tiny peak is seen about 0.01 nK below \Tc. (d) Normalized attenuation in  the $A_{1g}$ channels of \sro through \Tc, fit to the attenuation expected from domain wall motion below \Tc. The fit works well only close to \Tc, probably because it does not include other temperature-dependent effects (see text for details).}
\end{figure*}

%Revisit BCS with nodal gaps, and TRSB gaps.
%BCS SCs have a pair breaking mode, large energy, not at our frequencies
%Domains, how it works
%What it means for OP

We now analyze possible mechanisms that could give rise to such an increase in sound attenuation below \Tc. First, we calculate sound attenuation within a BCS-like framework, accounting for the differences in coherence factors that occur for various unconventional SC order parameters. We find that a peak can indeed arise under certain circumstances but not under our experimental conditions. Second, we consider phonon-induced Cooper pair breaking in the SC state that does lead to a sound attenuation peak just below \Tc, but which is inaccessibly narrow in our experiment. Finally, we show that a simple model of sound attenuation due to the formation of SC domains best matches the experimental data. 

%The BCS calculation of sound attenuation accounts for the fact that quasiparticle (QP) scattering is different in the SC state differently than in the normal state. Assuming that an ultrasound phonon does not flip the QP's spin, BCS showed that QP scattering in the SC state is proportional to the coherence factor $F_-=(1-\Delta_0^2/E_k E_{k'})$, where $\Delta_0$ is the uniform $s$-wave gap and $E_k$ is the dispersion of the Bogoliubov quasiparticles. The negative sign cancels out the increase in the SC density of states at energies close to $\Delta_0$ and leads to an exponential decay in attenuation below \Tc \cite{Tinkham}. The equivalent coherence factor for NMR relaxation, which does flip the QP's spin, is given by $F_+=(1+\Delta_0^2/E_k E_{k'})$. This gives rise to the Hebel-Slichter peak right below \Tc. Within the BCS framework, we find that attenuation for a $d_{x^2-y^2}$ gap decays slowly compared to the isotropic $s$-wave gap (\autoref{fig:fits}(a)), which can be attributed to the presence of nodes in the $d_{x^2-y^2}$ gap. For a TRS-breaking gap (\autoref{fig:fits}(b)), we find that a peak can show up if sufficiently large-angle scattering is allowed. However, at typical sound speeds, such scattering would require frequencies of order of THz, while our experiment operates in the MHz range. Hence we rule this out as the mechanism leading to our observed peak. 

First we examine the possibility of increased sound attenuation due to coherent scattering in the SC state. Sound attenuation and nuclear spin relaxation in an $s$-wave superconductor are proportional to the coherence factors $F_{\pm} = (1\pm \Delta_0^2/E_k E_{k'})$, where $\Delta_0$ is the uniform $s$-wave gap and $E_k$ is the Bogoliubov quasiparticle dispersion \cite{BCS1957}. Scattering off of a nucleus flips the spin of the quasiparticle and the resultant coherence factor is $F_+$, where the $+$ sign produces the Hebel-Slichter peak below \Tc. Scattering off of a phonon, on the other hand, does not flip quasiparticle spin and the resultant coherence factor is $F_-$, producing a sharp drop in sound attenuation below \Tc. In general, the coherence factors depend on the structure of the superconducting gap, motivating the idea that an unconventional superconducting OP might produce a peak in the sound attenuation. Calculating within the BCS framework, we find that attenuation for a $d_{x^2-y^2}$ gap decays slowly compared to the isotropic $s$-wave gap, but does not exhibit a peak (\autoref{fig:fits}(a), see SI for details of the calculation). This slow decrease can be attributed to the presence of nodes in the $d_{x^2-y^2}$ gap \cite{Maki1994,Vekhter1999}. For a TRS breaking gap such as $p_x+ip_y$ or $d_{xz}+id_{yz}$, a Hebel-Slichter-like peak appears below \Tc if sufficiently large-angle scattering is allowed (\autoref{fig:fits}(b)). Scattering at these large wavevectors---essentially scattering across the Fermi surface---would require ultrasound with nanometer wavelengths. This regime is only accessible at THz frequencies, whereas our experiment operates in the MHz range. Hence we rule out coherent scattering as the mechanism of increased compressional sound attenuation below \Tc. 

%For an unconventional ($\mathbf{k}$-dependent) gap $\Delta_{k}$, the coherence factors become (see SI for details)
%\begin{equation}
%	F_\pm=1\pm\frac{\Delta_k\Delta_{k'}}{E_k E_{k'}}
%\end{equation}  
%For a $d_{x^2-y^2}$ gap, the presence of nodes leads to a slower decrease than the exponential decay in BCS case (see \autoref{fig:fits}(a)). We further find that scattering within a TRS-breaking gap can give a peak if sufficiently large-angle scattering is allowed. Our probe is small- $k$ transfer, no change really.

%We now mention how a large coherence factor may arise in an anisotropic gap SC, which follows directly from the original BCS calculation of sound attenuation in a superconductor. Assuming that ultrasound couples to electron concentration, BCS showed that quasiparticle scattering in the SC state is proportional to the coherence factor $F_-=(1-\Delta_s^2/E_k E_{k'})$, where $\Delta_s$ is the uniform s-wave gap and $E_k$ is the dispersion of the Bogoliubov quasiparticles. This negative sign cancels out the increase in superconducting density of states at energies close to $\Delta$, and leads to an exponential decay in attenuation below \Tc \cite{Tinkham}. The equivalent coherence factor for NMR relaxation is given by $F_+=(1+\Delta_s^2/E_k E_{k'})$, and gives rise to the Hebel-Slichter peak right below \Tc. \citet{CoffeyPRB1986} argued that in contrast to a BCS superconductor, the enhancement in DOS is not canceled out by the negative sign in $F_-$ in an unconventional SC. For an unconventional ($\mathbf{k}$-dependent) gap $\Delta_{k}$, the coherence factors become (see SI)
%\begin{equation}
%	F_\pm=1\pm\frac{\Delta_k\Delta_{k'}}{E_k E_{k'}}
%\end{equation} 
%Scattering between regions of the gap with opposite signs clearly reverses the sign in the coherence factors. It is thus plausible that if such scattering is strong enough, it could lead to a peak in the ultrasound attenuation, similar to the NMR peak for BCS superconductors.
%For example, a BCS superconductor with \Tc= 1.5 K would have 2$\Delta\sim$ 0.4 meV, corresponding to a frequency of 0.3 THz.

Next we consider how phonon-induced Cooper pair breaking may give rise to a sound attenuation peak, similar to what has been observed in superfluid $^3$He-$B$ below \Tc \cite{AdenwallaPRL1989}. Pair-breaking in BCS superconductors requires a minimum energy of 2$\Delta_0$, where $\Delta_0$ is the gap magnitude. This energy scale is generally much higher than typical ultrasound energies. For example, the maximum gap magnitude in \sro is 2$\Delta\sim$ 0.65 meV \cite{FirmoPRB2013}, which would require a frequency of approximately 1 THz to break the Cooper pairs. However, the pair-breaking energy is lowered for a gap with nodes, such as $d_{x^2-y^2}$. In particular, since the gap goes to zero at \Tc, it may be small enough near \Tc such that pair-breaking is possible at a few MHz. Our calculations for a $d_{x^2-y^2}$ gap, however, show that $\sim$10 GHz frequencies are required to produce an experimentally discernible peak (\autoref{fig:fits}(c)). At our experimental frequencies, the peak is only visible within 0.01 nK of \Tc. For a fully gapped superconductor, like the TRS breaking state $p_x + i p_y$, the peak will be even smaller. This clearly rules out pair-breaking as the origin of the increased sound attenuation. 

%at \Tc, one could imagine that near \Tc, the gap is small enough such that this mode is excited at a few MHz. Assuming a mean-field temperature evolution of the gap, $\Delta(T)=\Delta_0\sqrt{1-T/T_c}$, we find that $T$ needs to be within a nanokelvin of \Tc for the $2\Delta$ mode to be excited at 2 MHz. Our observed peak is roughly 30 mK below \Tc and quite broad in temperature, thereby ruling out this effect as the origin of the peak.

%In contrast to BCS superconductors, unconventional SCs can host low energy modes of the order parameter, since they have multiple degrees of freedom. Collective modes of such OPs can exist at energies much lower than 2$\Delta$\cite{SigristRMP1991,TewordtPRB2000}, and hence maybe excited at ultrasound frequencies. Additionally, order parameter dynamics near \Tc usually relax over long timescales (``critical slowing down"), giving rise to enhanced ultrasound absorption. \citet{MiyakePRL1986} showed that a peak in longitudinal attenuation can be obtained from the relaxation of order parameter amplitude, based on the Landau-Khalatnikov (LK) mechanism. In essence, the LK mechanism assumes that fluctuations in the OP are damped by a restoring force proportional to the free energy change caused by the fluctuations. Ref.\cite{MiyakePRL1986} also explained why the coupling of shear sound to such modes is much weaker, leading to the absence of a peak in shear attenuation. From our experimental frequencies, and using a similar OP relaxation model (see SI for details), we estimate the relaxation timescale to be of order 10 ns in \sro. The same model applied to the $c_{66}$ elastic constant jump measured at different ultrasound frequencies \cite{GhoshNatPhys2020,BenhabibNatPhys2020} gives a relaxation time of $\tau\sim6.6$ ns, comparable to the 10 ns we estimate from the position of attenuation peak. 

Finally, we consider the formation of SC domains. When different configurations of a SC order parameter are degenerate, such as $p_x + i p_y$ and $p_x - i p_y$, domains of each configuration will form separated by domain walls. These domain walls can oscillate about their equilibrium positions when sound propagates through the sample \cite{JoyntPRL1986}. \citet{SigristRMP1991} derive an expression for the sound attenuation coefficient, $\alpha$, from domain wall motion of the form
\begin{equation}
\alpha\left(\omega,T\right) \propto \frac{\omega^2}{\omega^2+\omega_{DW}^2}\rho_s^2,
\label{eq:atten1}
\end{equation}
where $\rho_s$ is the superfluid density (proportional to the square of the superconducting gap), $\omega$ is the angular frequency of the sound wave, and $\omega_{DW}$ is the lowest vibrational frequency of the domain wall. While the full functional form of $\rho_s$ and $\omega_{DW}$ are unknown, near \Tc they can be written within a Ginzburg-Landau (GL) formalism as $\rho_s\propto|T-T_c|$ and $\omega_{DW}\propto|T-T_c|^{3/2}$ which gives the explicit temperature dependence of the above equation as 
\begin{equation}
\alpha\left(\omega,T\right) \propto \frac{\omega^2}{\omega^2+\omega_1^2\left|T/T_{\rm c}-1\right|^3}\left|T/T_{\rm c}-1\right|^2,
\label{eq:atten}
\end{equation}
where $\omega_1$ is the domain wall frequency as $T\rightarrow 0$. We fit all three attenuation channels to \autoref{eq:atten} and extract $\omega_1 = 500\pm25$ MHz (\autoref{fig:fits}(d)). As the temperature approaches \Tc from below the domain wall frequency decreases to zero, producing a peak in the attenuation when the experimental frequency is approximately equal to the domain wall frequency. This may partially explain why no peak was observed in previous pulse-echo ultrasound measurements at $\sim$100 MHz \cite{LupienPRL2001} (as mentioned above, there is also the issue that the attenuation for longitudinal sound along the [100] direction is almost entirely dominated by $(\eta_{11}-\eta_{12})/2$).) The fit of \autoref{eq:atten} deviates substantially from the data for $T/T_{\rm c}\lesssim 0.95$: this is to be expected, as the GL theory is only valid near \Tc \cite{SigristRMP1991}. Nevertheless, \autoref{eq:atten} captures the correct shape of the rapid increase in attenuation below \Tc in all three compression channels, using the same value of $\omega_1$ for all three fits. The extracted frequency scale of $\omega_1 \approx 500$ MHz is also reasonable: studies of sound attenuation in nickel at MHz frequencies show similar magnitudes of increase in the magnetically ordered state when domains are present \cite{Leonard1962}. We note that the results of Josephson interferometry measurements have previously been interpreted as evidence for SC domains in \sro \cite{KidwingiraScience2006}.

%Whether a particular domain wall is sensitive to a particular strain depends on both the order parameter symmetry and the symmetry of the strain.  Further, if strain induced by the sound wave couples differently to the OP in adjacent domains, domain wall motion absorbs energy from the sound wave and may attenuate it strongly. Strains preserve TRS and thus there is no strain that couples to domain walls between $\eta_x + i \eta_y$ and $\eta_x - i \eta_y$ states. States that break rotational symmetry, on the other hand, can couple to strains that break the same symmetry, e.g. the $\eta_x$ and $\eta_y$ nematic states will couple to the $\epsilon_{xx} - \epsilon_{yy}$ strain, leading to attenuation in the $(\eta_{11}-\eta_{12})/2$ channel. We find an increase in sound attenuation below \Tc only in the symmetric, $A_{1g}$ channels, which do not break any symmetry. Thus we can rule out symmetry-related domains, such as the ``nematic'' $d_{xz}$ state proposed in \citet{BenhabibNatPhys2020}, as the source of the excess attenuation. 
%
%Next we consider domains of two different superconducting states that are accidentally degenerate (or near-degenerate). An example is the proposed  state \cite{GhoshNatPhys2020,KivelsonNPJ2020,Yuan2021}. Because $d$ and $g$ are not related by symmetry in the tetragonal lattice, 
%In superconductors with multi-component order parameters (OPs), such as $\left\{p_x,p_y\right\}$ or $\left\{d_{xz},d_{yz}\right\}$ (which we generically refer to as $\left\{\eta_x,\eta_y\right\}$), domains occur when different configurations the OP have the same condensation energy. For example, domains of $\eta_x + i \eta_y$ and $\eta_x - i \eta_y$ can form if the TRS breaking state is favored, or domains of $\eta_x$ and $\eta_y$ can form if the nematic state is favored. 
%
%It is natural to ask why the shear attenuation does not show a similar increase below \Tc. In particular, the shear strain $\epsilon_{xy}$ might be expected to couple to the superconducting OP just as well as the compressional strains because its associated elastic modulus, $c_{66}$, shows a thermodynamic discontinuity at \Tc that is comparable in size to the discontinuities in the compressional moduli \cite{GhoshNatPhys2020,BenhabibNatPhys2020}. The key difference between compression and shear, however, is that compressional strains change the volume of the domains, whereas shear strains are volume-preserving and only change the shape of the domains. This implies that under shear strains, the condensation energy of the different domains remains the same and hence they do not absorb energy out of shear sound. In other words, since steady-state (zero frequency) compression changes superconducting condensation energy (and hence \Tc) linearly \cite{ForsythePRL2002}, finite frequency modulation of regions of different \Tc present in the sample requires energy and therefore strongly attenuate compressional waves. In contrast, shear strains only couple to \Tc at quadratic order \cite{HicksSc2014}, and thus at low frequencies, domains of different \Tc do not absorb energy out of shear waves.
%In contrast, there is no energy difference between domains of $d_{xz}$ and $d_{yz}$, since their degeneracy is enforced by symmetry. Another possibility is the formation of TRS breaking $d_{xz}\pm id_{yz}$ or $d\pm ig$ domains, which are also degenerate under $A_{1g}$ strain and should not lead to enhanced sound attenuation. 
%Taken at face value, the fact that we observe peaks only in the $A_{1g}$ channel favors an accidental degeneracy, such as $\{d_{x^2-y^2},g_{xy(x^2-y^2)}\}$ or $\{s,d_{xy}\}$ (proposed in \cite{RomerArXiv2021}), over a symmetry-enforced degeneracy, such as $\{d_{xz},d_{yz}\}$. This is because $A_{1g}$ strain is expected to couple differently to the $d_{x^2-y^2}$ and $g_{xy(x^2-y^2)}$ components \cite{KivelsonNPJ2020} and thus moving the wall separating domains of $d_{x^2-y^2}$ and $g_{xy(x^2-y^2)}$ requires energy. 

\section*{Discussion}

The factor of seven increase we find in the in-plane compressional viscosity is without precedent in a superconductor. For comparison, longitudinal attenuation increases by 50\% below \Tc in UPt$_3$ \cite{MullerSolStComm1986}, and by a bit more than a factor of two in UBe$_{13}$ \cite{GoldingPRL1985}. There is also a qualitative difference between the increase in \sro and the increase seen in the heavy fermion superconductors: the attenuation peaks sharply below \Tc in both UPt$_3$ and UBe$_{13}$, with a peak width of approximately 10\% of \Tc. The compressional attenuation in \sro, by contrast, decreases by only about 10\% over the same relative temperature range. This suggests that something highly unconventional occurs in the SC state of \sro, leading to a large increase in sound attenuation that is not confined to temperatures near \Tc. The mechanism we find most consistent with the data is domain wall motion.

Assuming that we have established the likely origin of the increase in sound attenuation, we consider its implications for the superconductivity of \sro. The formation of domains requires a two-component order parameter (OP), either symmetry-enforced or accidental, reaffirming the conclusions of recent ultrasound studies of the elastic moduli and the sound velocity \cite{GhoshNatPhys2020,BenhabibNatPhys2020}. We can learn more about which particular OPs are consistent with our experiment by considering which symmetry channels show the increase in attenuation. Domains attenuate ultrasound when the application of strain raises or lowers the condensation energy of one domain in comparison to a neighboring domain. A simple example is the ``nematic'' superconducting state proposed by \citet{BenhabibNatPhys2020}, which is a $d-$wave OP of the $E_g$ representation, transforming as $\left\{d_{xz},d_{yz}\right\}$. Under $(\epsilon_{xx}-\epsilon_{yy})$ strain, domains of the $d_{xz}$ configuration will be favored over the $d_{yz}$ configuration (depending on the sign of the strain). This will cause some domains to grow and others to shrink, attenuating sound through the mechanism proposed by \citet{SigristRMP1991}. We find no increase in $\left(\eta_{11}-\eta_{12}\right)/2$ below \Tc, suggesting that a $\left\{d_{xz},d_{yz}\right\}$ OP cannot explain the increase in compressional sound attenuation.

More generally, the lack of increase in attenuation in any of the shear channels implies that that the SC state of \sro does not break rotational symmetry. Domains that are related to each other by time reversal symmetry can also be ruled out: there is no strain that can lift the degeneracy between, for example, a $p_x + i p_y$ domain and a $p_x - i p_y$ domain. The observed increase in sound attenuation  under compressional strain is therefore quite unusual: as \citet{SigristRMP1991} point out, compressional strains can never lift the degeneracy between domains that are related by \textit{any} symmetry, since compressional strains do not break the point group symmetry of the lattice. Instead, attenuation in the compressional channel requires domains that couple differently to compressional strain, which in turn requires domains that are accidentally degenerate. Examples that are consistent with both NMR \cite{PustogowNat2019} and ultrasound \cite{GhoshNatPhys2020,BenhabibNatPhys2020} include $\{d_{x^2-y^2},g_{xy(x^2-y^2)}\}$ \cite{KivelsonNPJ2020,WillaPRB2021,ClepkensArxiv2021} and $\{s,d_{xy}\}$ \cite{RomerArXiv2021}. Then, for example, domains of $d_{x^2-y^2}$ will couple differently to compressional strain than domains of $g_{xy(x^2-y^2)}$, leading to the growth of one domain type and an increase in compressional sound attenuation below \Tc. Shear strain, meanwhile, does not change the condensation energy of any single-component order parameter (e.g. $s$, $d_{xy}$, $d_{x^2-y^2}$, or $g_{xy(x^2-y^2)}$) to first order in strain, which means that the lack of increase in shear attenuation below \Tc is also consistent with an accidentally-degenerate OP. This is also consistent with the lack of a cusp in \Tc under applied shear strain \cite{HicksSc2014,watson2018micron}.

Recent theoretical work \cite{Yuan2021} has shown that domain walls between $d_{x^2-y^2}$ and $g_{xy(x^2-y^2)}$ OPs may provide a route to explain the observation of half-quantum vortices in \sro \textit{without} a spin-triplet order parameter \cite{Jang2011}---a result that is otherwise inconsistent with the singlet pairing suggested by NMR \cite{PustogowNat2019}. \citet{WillaPRB2021}, followed by \citet{Yuan2021},  have shown that domains between such states stabilize a TRS-breaking $d_{x^2-y^2} \pm i g_{xy(x^2-y^2)}$ state near the domain wall. This would naturally explain why probes of TRS breaking, such as the Kerr effect and $\mu$SR \cite{xia2006high,GrinenkoNatPhys2021}, see such a small effect at \Tc in \sro. 

One significant challenge for the two-component order parameter scenario is that, whether accidentally degenerate or not, a two component order parameter should generically produce two superconducting \Tcs. The lack of a heat capacity signature from an expected second transition under uniaxial strain \cite{LiPNAS2021} can only be explained if the second, TRS-breaking transition is particularly weak---a result that might be consistent with the TRS-breaking state appearing only along domain walls. Finally, it is worth noting that there are other mechanisms of ultrasonic attenuation that we have not explored here, including collective modes and gapless excitations such as edge currents that might appear along domain walls even if the domains are related by symmetry. Future ultrasound experiments under applied static strain and magnetic fields are warranted as certain types of domain walls can couple to these fields, thereby affecting the sound attenuation through \Tc.


%We find that our observed increase in compressional sound attenuation below \Tc in \sro is most consistent with domain wall motion between domains of accidentally-degenerate OPs.


%In the presence of inhomogeneous strains (which are always present in a sample), such domains would be separated by domain walls where the order parameter is $d\pm ig$ and thus TRS breaking \cite{ClepkensArxiv2021}. This would naturally explain why probes of TRS breaking, such as Kerr effect and $\mu$SR, see a small signal in \sro at \Tc. 
%
%The lack of a heat capacity signature from an expected second transition under uniaxial strain \cite{LiPNAS2021} still remains a challenge for the two-component scenario.  
%
%However, the presence of disorder may change the energy of competing OP combinations \cite{FlorensPRB2005}, an effect which is not captured in our simple analysis. Further, the mechanism of sound attenuation considered by \citet{SigristRMP1991} only accounts for differences in condensation energy: active degrees of freedom along a domain wall, such as gapless edge currents, could potentially also produce sound attenuation. 



%Irrespective of these details, our data provides strong evidence for the presence of superconducting domains in \sro.

%In summary, we performed low frequency ultrasonic attenuation measurements and observe, for the first time, a sound attenuation peak in \sro below \Tc. We find that the peak is only present in compressional ($A_{1g}$) channels, and not in the shear channels. We consider various mechanisms for the origin of this peak, and identify domain formation in the SC state as the only one that can account for our observations. Our data appears to favor an accidentally-degenerate SC state, such as $d_{x^2-y^2}+ig_{xy(x^2-y^2)}$, but further theoretical efforts on the dynamics of domain walls are warranted.

%We mention a few possible mechanisms that may give rise to the peak, although our data does not allow us to conclude which is the dominant mechanism in \sro. Whatever maybe the underlying reason for this peak, our results add another piece to the puzzle that is \sro.

%We now mention how a large coherence factor may arise in an anisotropic gap SC, which follows directly from the original BCS calculation of sound attenuation in a superconductor. Assuming that ultrasound couples to electron concentration, BCS showed that quasiparticle scattering in the SC state is proportional to the coherence factor $F_-=(1-\Delta_s^2/E_k E_{k'})$, where $\Delta_s$ is the uniform s-wave gap and $E_k$ is the dispersion of the Bogoliubov quasiparticles. This negative sign cancels out the increase in superconducting density of states at energies close to $\Delta$, and leads to an exponential decay in attenuation below \Tc \cite{Tinkham}. The equivalent coherence factor for NMR relaxation is given by $F_+=(1+\Delta_s^2/E_k E_{k'})$, and gives rise to the Hebel-Slichter peak right below \Tc. \citet{CoffeyPRB1986} argued that in contrast to a BCS superconductor, the enhancement in DOS is not canceled out by the negative sign in $F_-$ in an unconventional SC. For an unconventional ($\mathbf{k}$-dependent) gap $\Delta_{k}$, the coherence factors become (see SI)
%\begin{equation}
%	F_\pm=1\pm\frac{\Delta_k\Delta_{k'}}{E_k E_{k'}}
%\end{equation} 
%Scattering between regions of the gap with opposite signs clearly reverses the sign in the coherence factors. It is thus plausible that if such scattering is strong enough, it could lead to a peak in the ultrasound attenuation, similar to the NMR peak for BCS superconductors.
 
%{\color{red} Any other mechanisms, BCS-like with anisotropic/TRSB gap? Bogoliubov FS?}



%\bibliography{library}

%apsrev4-2.bst 2019-01-14 (MD) hand-edited version of apsrev4-1.bst
%Control: key (0)
%Control: author (8) initials jnrlst
%Control: editor formatted (1) identically to author
%Control: production of article title (0) allowed
%Control: page (0) single
%Control: year (1) truncated
%Control: production of eprint (0) enabled
\begin{thebibliography}{54}%
\makeatletter
\providecommand \@ifxundefined [1]{%
 \@ifx{#1\undefined}
}%
\providecommand \@ifnum [1]{%
 \ifnum #1\expandafter \@firstoftwo
 \else \expandafter \@secondoftwo
 \fi
}%
\providecommand \@ifx [1]{%
 \ifx #1\expandafter \@firstoftwo
 \else \expandafter \@secondoftwo
 \fi
}%
\providecommand \natexlab [1]{#1}%
\providecommand \enquote  [1]{``#1''}%
\providecommand \bibnamefont  [1]{#1}%
\providecommand \bibfnamefont [1]{#1}%
\providecommand \citenamefont [1]{#1}%
\providecommand \href@noop [0]{\@secondoftwo}%
\providecommand \href [0]{\begingroup \@sanitize@url \@href}%
\providecommand \@href[1]{\@@startlink{#1}\@@href}%
\providecommand \@@href[1]{\endgroup#1\@@endlink}%
\providecommand \@sanitize@url [0]{\catcode `\\12\catcode `\$12\catcode
  `\&12\catcode `\#12\catcode `\^12\catcode `\_12\catcode `\%12\relax}%
\providecommand \@@startlink[1]{}%
\providecommand \@@endlink[0]{}%
\providecommand \url  [0]{\begingroup\@sanitize@url \@url }%
\providecommand \@url [1]{\endgroup\@href {#1}{\urlprefix }}%
\providecommand \urlprefix  [0]{URL }%
\providecommand \Eprint [0]{\href }%
\providecommand \doibase [0]{https://doi.org/}%
\providecommand \selectlanguage [0]{\@gobble}%
\providecommand \bibinfo  [0]{\@secondoftwo}%
\providecommand \bibfield  [0]{\@secondoftwo}%
\providecommand \translation [1]{[#1]}%
\providecommand \BibitemOpen [0]{}%
\providecommand \bibitemStop [0]{}%
\providecommand \bibitemNoStop [0]{.\EOS\space}%
\providecommand \EOS [0]{\spacefactor3000\relax}%
\providecommand \BibitemShut  [1]{\csname bibitem#1\endcsname}%
\let\auto@bib@innerbib\@empty
%</preamble>
\bibitem [{\citenamefont {Bardeen}\ \emph {et~al.}(1957)\citenamefont
  {Bardeen}, \citenamefont {Cooper},\ and\ \citenamefont
  {Schrieffer}}]{BCS1957}%
  \BibitemOpen
  \bibfield  {author} {\bibinfo {author} {\bibfnamefont {J.}~\bibnamefont
  {Bardeen}}, \bibinfo {author} {\bibfnamefont {L.~N.}\ \bibnamefont
  {Cooper}},\ and\ \bibinfo {author} {\bibfnamefont {J.~R.}\ \bibnamefont
  {Schrieffer}},\ }\bibfield  {title} {\bibinfo {title} {Theory of
  superconductivity},\ }\href@noop {} {\bibfield  {journal} {\bibinfo
  {journal} {Phys. Rev.}\ }\textbf {\bibinfo {volume} {108}},\ \bibinfo {pages}
  {1175} (\bibinfo {year} {1957})}\BibitemShut {NoStop}%
\bibitem [{\citenamefont {Hebel}\ and\ \citenamefont
  {Slichter}(1957)}]{SlichterPRL1957}%
  \BibitemOpen
  \bibfield  {author} {\bibinfo {author} {\bibfnamefont {L.~C.}\ \bibnamefont
  {Hebel}}\ and\ \bibinfo {author} {\bibfnamefont {C.~P.}\ \bibnamefont
  {Slichter}},\ }\bibfield  {title} {\bibinfo {title} {Nuclear relaxation in
  superconducting aluminum},\ }\href {https://doi.org/10.1103/PhysRev.107.901}
  {\bibfield  {journal} {\bibinfo  {journal} {Phys. Rev.}\ }\textbf {\bibinfo
  {volume} {107}},\ \bibinfo {pages} {901} (\bibinfo {year}
  {1957})}\BibitemShut {NoStop}%
\bibitem [{\citenamefont {Morse}\ \emph {et~al.}(1959)\citenamefont {Morse},
  \citenamefont {Olsen},\ and\ \citenamefont {Gavenda}}]{MorsePRL1959}%
  \BibitemOpen
  \bibfield  {author} {\bibinfo {author} {\bibfnamefont {R.~W.}\ \bibnamefont
  {Morse}}, \bibinfo {author} {\bibfnamefont {T.}~\bibnamefont {Olsen}},\ and\
  \bibinfo {author} {\bibfnamefont {J.~D.}\ \bibnamefont {Gavenda}},\
  }\bibfield  {title} {\bibinfo {title} {Evidence for anisotropy of the
  superconducting energy gap from ultrasonic attenuation},\ }\href
  {https://doi.org/10.1103/PhysRevLett.3.15} {\bibfield  {journal} {\bibinfo
  {journal} {Phys. Rev. Lett.}\ }\textbf {\bibinfo {volume} {3}},\ \bibinfo
  {pages} {15} (\bibinfo {year} {1959})}\BibitemShut {NoStop}%
\bibitem [{\citenamefont {Levy}(1963)}]{LevyPRL1963}%
  \BibitemOpen
  \bibfield  {author} {\bibinfo {author} {\bibfnamefont {M.}~\bibnamefont
  {Levy}},\ }\bibfield  {title} {\bibinfo {title} {Ultrasonic attenuation in
  superconductors for $\mathrm{ql}<1$},\ }\href
  {https://doi.org/10.1103/PhysRev.131.1497} {\bibfield  {journal} {\bibinfo
  {journal} {Phys. Rev.}\ }\textbf {\bibinfo {volume} {131}},\ \bibinfo {pages}
  {1497} (\bibinfo {year} {1963})}\BibitemShut {NoStop}%
\bibitem [{\citenamefont {Leibowitz}(1964)}]{LeibowitzPRL1964}%
  \BibitemOpen
  \bibfield  {author} {\bibinfo {author} {\bibfnamefont {J.~R.}\ \bibnamefont
  {Leibowitz}},\ }\bibfield  {title} {\bibinfo {title} {Ultrasonic shear wave
  attenuation in superconducting tin},\ }\href
  {https://doi.org/10.1103/PhysRev.133.A84} {\bibfield  {journal} {\bibinfo
  {journal} {Phys. Rev.}\ }\textbf {\bibinfo {volume} {133}},\ \bibinfo {pages}
  {A84} (\bibinfo {year} {1964})}\BibitemShut {NoStop}%
\bibitem [{\citenamefont {Claiborne}\ and\ \citenamefont
  {Einspruch}(1965)}]{ClaibornePRL1965}%
  \BibitemOpen
  \bibfield  {author} {\bibinfo {author} {\bibfnamefont {L.~T.}\ \bibnamefont
  {Claiborne}}\ and\ \bibinfo {author} {\bibfnamefont {N.~G.}\ \bibnamefont
  {Einspruch}},\ }\bibfield  {title} {\bibinfo {title} {Energy-gap anisotropy
  in {In}-doped {Sn}},\ }\href {https://doi.org/10.1103/PhysRevLett.15.862}
  {\bibfield  {journal} {\bibinfo  {journal} {Phys. Rev. Lett.}\ }\textbf
  {\bibinfo {volume} {15}},\ \bibinfo {pages} {862} (\bibinfo {year}
  {1965})}\BibitemShut {NoStop}%
\bibitem [{\citenamefont {Fossheim}(1967)}]{FossheimPRL1967}%
  \BibitemOpen
  \bibfield  {author} {\bibinfo {author} {\bibfnamefont {K.}~\bibnamefont
  {Fossheim}},\ }\bibfield  {title} {\bibinfo {title} {Electromagnetic
  shear-wave interaction in a superconductor},\ }\href
  {https://doi.org/10.1103/PhysRevLett.19.344.2} {\bibfield  {journal}
  {\bibinfo  {journal} {Phys. Rev. Lett.}\ }\textbf {\bibinfo {volume} {19}},\
  \bibinfo {pages} {344} (\bibinfo {year} {1967})}\BibitemShut {NoStop}%
\bibitem [{\citenamefont {Batlogg}\ \emph {et~al.}(1985)\citenamefont
  {Batlogg}, \citenamefont {Bishop}, \citenamefont {Golding}, \citenamefont
  {Varma}, \citenamefont {Fisk}, \citenamefont {Smith},\ and\ \citenamefont
  {Ott}}]{BatloggPRL1985}%
  \BibitemOpen
  \bibfield  {author} {\bibinfo {author} {\bibfnamefont {B.}~\bibnamefont
  {Batlogg}}, \bibinfo {author} {\bibfnamefont {D.}~\bibnamefont {Bishop}},
  \bibinfo {author} {\bibfnamefont {B.}~\bibnamefont {Golding}}, \bibinfo
  {author} {\bibfnamefont {C.~M.}\ \bibnamefont {Varma}}, \bibinfo {author}
  {\bibfnamefont {Z.}~\bibnamefont {Fisk}}, \bibinfo {author} {\bibfnamefont
  {J.~L.}\ \bibnamefont {Smith}},\ and\ \bibinfo {author} {\bibfnamefont
  {H.~R.}\ \bibnamefont {Ott}},\ }\bibfield  {title} {\bibinfo {title}
  {$\ensuremath{\lambda}$-shaped ultrasound-attenuation peak in superconducting
  {(U,Th)${\mathrm{Be}}_{13}$}},\ }\href
  {https://doi.org/10.1103/PhysRevLett.55.1319} {\bibfield  {journal} {\bibinfo
   {journal} {Phys. Rev. Lett.}\ }\textbf {\bibinfo {volume} {55}},\ \bibinfo
  {pages} {1319} (\bibinfo {year} {1985})}\BibitemShut {NoStop}%
\bibitem [{\citenamefont {Golding}\ \emph {et~al.}(1985)\citenamefont
  {Golding}, \citenamefont {Bishop}, \citenamefont {Batlogg}, \citenamefont
  {Haemmerle}, \citenamefont {Fisk}, \citenamefont {Smith},\ and\ \citenamefont
  {Ott}}]{GoldingPRL1985}%
  \BibitemOpen
  \bibfield  {author} {\bibinfo {author} {\bibfnamefont {B.}~\bibnamefont
  {Golding}}, \bibinfo {author} {\bibfnamefont {D.~J.}\ \bibnamefont {Bishop}},
  \bibinfo {author} {\bibfnamefont {B.}~\bibnamefont {Batlogg}}, \bibinfo
  {author} {\bibfnamefont {W.~H.}\ \bibnamefont {Haemmerle}}, \bibinfo {author}
  {\bibfnamefont {Z.}~\bibnamefont {Fisk}}, \bibinfo {author} {\bibfnamefont
  {J.~L.}\ \bibnamefont {Smith}},\ and\ \bibinfo {author} {\bibfnamefont
  {H.~R.}\ \bibnamefont {Ott}},\ }\bibfield  {title} {\bibinfo {title}
  {Observation of a collective mode in superconducting
  {U${\mathrm{Be}}_{13}$}},\ }\href@noop {} {\bibfield  {journal} {\bibinfo
  {journal} {Phys. Rev. Lett.}\ }\textbf {\bibinfo {volume} {55}},\ \bibinfo
  {pages} {2479} (\bibinfo {year} {1985})}\BibitemShut {NoStop}%
\bibitem [{\citenamefont {Müller}\ \emph {et~al.}(1986)\citenamefont
  {Müller}, \citenamefont {Maurer}, \citenamefont {Scheidt}, \citenamefont
  {Roth}, \citenamefont {Lüders}, \citenamefont {Bucher},\ and\ \citenamefont
  {Bömmel}}]{MullerSolStComm1986}%
  \BibitemOpen
  \bibfield  {author} {\bibinfo {author} {\bibfnamefont {V.}~\bibnamefont
  {Müller}}, \bibinfo {author} {\bibfnamefont {D.}~\bibnamefont {Maurer}},
  \bibinfo {author} {\bibfnamefont {E.}~\bibnamefont {Scheidt}}, \bibinfo
  {author} {\bibfnamefont {C.}~\bibnamefont {Roth}}, \bibinfo {author}
  {\bibfnamefont {K.}~\bibnamefont {Lüders}}, \bibinfo {author} {\bibfnamefont
  {E.}~\bibnamefont {Bucher}},\ and\ \bibinfo {author} {\bibfnamefont
  {H.}~\bibnamefont {Bömmel}},\ }\bibfield  {title} {\bibinfo {title}
  {Observation of a lambda-shaped ultrasonic attenuation peak in
  superconducting {U${\mathrm{Pt}}_{3}$}},\ }\href@noop {} {\bibfield
  {journal} {\bibinfo  {journal} {Solid State Communications}\ }\textbf
  {\bibinfo {volume} {57}},\ \bibinfo {pages} {319 } (\bibinfo {year}
  {1986})}\BibitemShut {NoStop}%
\bibitem [{\citenamefont {Moreno}\ and\ \citenamefont
  {Coleman}(1996)}]{MorenoPRB1996}%
  \BibitemOpen
  \bibfield  {author} {\bibinfo {author} {\bibfnamefont {J.}~\bibnamefont
  {Moreno}}\ and\ \bibinfo {author} {\bibfnamefont {P.}~\bibnamefont
  {Coleman}},\ }\bibfield  {title} {\bibinfo {title} {Ultrasound attenuation in
  gap-anisotropic systems},\ }\href@noop {} {\bibfield  {journal} {\bibinfo
  {journal} {Phys. Rev. B}\ }\textbf {\bibinfo {volume} {53}},\ \bibinfo
  {pages} {R2995} (\bibinfo {year} {1996})}\BibitemShut {NoStop}%
\bibitem [{\citenamefont {Miyake}\ and\ \citenamefont
  {Varma}(1986)}]{MiyakePRL1986}%
  \BibitemOpen
  \bibfield  {author} {\bibinfo {author} {\bibfnamefont {K.}~\bibnamefont
  {Miyake}}\ and\ \bibinfo {author} {\bibfnamefont {C.~M.}\ \bibnamefont
  {Varma}},\ }\bibfield  {title} {\bibinfo {title} {Landau-{K}halatnikov
  damping of ultrasound in heavy-fermion superconductors},\ }\href@noop {}
  {\bibfield  {journal} {\bibinfo  {journal} {Phys. Rev. Lett.}\ }\textbf
  {\bibinfo {volume} {57}},\ \bibinfo {pages} {1627} (\bibinfo {year}
  {1986})}\BibitemShut {NoStop}%
\bibitem [{\citenamefont {Monien}\ \emph {et~al.}(1987)\citenamefont {Monien},
  \citenamefont {Tewordt},\ and\ \citenamefont {Scharnberg}}]{MonienSol1987}%
  \BibitemOpen
  \bibfield  {author} {\bibinfo {author} {\bibfnamefont {H.}~\bibnamefont
  {Monien}}, \bibinfo {author} {\bibfnamefont {L.}~\bibnamefont {Tewordt}},\
  and\ \bibinfo {author} {\bibfnamefont {K.}~\bibnamefont {Scharnberg}},\
  }\bibfield  {title} {\bibinfo {title} {Ultrasound attenuation due to order
  parameter collective modes in impure anisotropic p-wave superconductors},\
  }\href {https://doi.org/https://doi.org/10.1016/0038-1098(87)90654-5}
  {\bibfield  {journal} {\bibinfo  {journal} {Solid State Communications}\
  }\textbf {\bibinfo {volume} {63}},\ \bibinfo {pages} {1027 } (\bibinfo {year}
  {1987})}\BibitemShut {NoStop}%
\bibitem [{\citenamefont {Joynt}\ \emph {et~al.}(1986)\citenamefont {Joynt},
  \citenamefont {Rice},\ and\ \citenamefont {Ueda}}]{JoyntPRL1986}%
  \BibitemOpen
  \bibfield  {author} {\bibinfo {author} {\bibfnamefont {R.}~\bibnamefont
  {Joynt}}, \bibinfo {author} {\bibfnamefont {T.~M.}\ \bibnamefont {Rice}},\
  and\ \bibinfo {author} {\bibfnamefont {K.}~\bibnamefont {Ueda}},\ }\bibfield
  {title} {\bibinfo {title} {Acoustic attenuation due to domain walls in
  anisotropic superconductors, with application to
  {${\mathrm{U}}_{1\ensuremath{-}x}{\mathrm{Th}}_{x}{\mathrm{Be}}_{13}$}},\
  }\href@noop {} {\bibfield  {journal} {\bibinfo  {journal} {Phys. Rev. Lett.}\
  }\textbf {\bibinfo {volume} {56}},\ \bibinfo {pages} {1412} (\bibinfo {year}
  {1986})}\BibitemShut {NoStop}%
\bibitem [{\citenamefont {Coffey}(1987)}]{CoffeyPRB1986}%
  \BibitemOpen
  \bibfield  {author} {\bibinfo {author} {\bibfnamefont {L.}~\bibnamefont
  {Coffey}},\ }\bibfield  {title} {\bibinfo {title} {Theory of ultrasonic
  attenuation in impure anisotropic p-wave superconductors},\ }\href
  {https://doi.org/10.1103/PhysRevB.35.8440} {\bibfield  {journal} {\bibinfo
  {journal} {Phys. Rev. B}\ }\textbf {\bibinfo {volume} {35}},\ \bibinfo
  {pages} {8440} (\bibinfo {year} {1987})}\BibitemShut {NoStop}%
\bibitem [{\citenamefont {Sigrist}\ and\ \citenamefont
  {Ueda}(1991)}]{SigristRMP1991}%
  \BibitemOpen
  \bibfield  {author} {\bibinfo {author} {\bibfnamefont {M.}~\bibnamefont
  {Sigrist}}\ and\ \bibinfo {author} {\bibfnamefont {K.}~\bibnamefont {Ueda}},\
  }\bibfield  {title} {\bibinfo {title} {Phenomenological theory of
  unconventional superconductivity},\ }\href@noop {} {\bibfield  {journal}
  {\bibinfo  {journal} {Rev. Mod. Phys.}\ }\textbf {\bibinfo {volume} {63}},\
  \bibinfo {pages} {239} (\bibinfo {year} {1991})}\BibitemShut {NoStop}%
\bibitem [{\citenamefont {Luke}\ \emph {et~al.}(1998)\citenamefont {Luke},
  \citenamefont {Fudamoto}, \citenamefont {Kojima}, \citenamefont {Larkin},
  \citenamefont {Merrin}, \citenamefont {Nachumi}, \citenamefont {Uemura},
  \citenamefont {Maeno}, \citenamefont {Mao}, \citenamefont {Mori},
  \citenamefont {Nakamura},\ and\ \citenamefont {Sigrist}}]{luke1998time}%
  \BibitemOpen
  \bibfield  {author} {\bibinfo {author} {\bibfnamefont {G.~M.}\ \bibnamefont
  {Luke}}, \bibinfo {author} {\bibfnamefont {Y.}~\bibnamefont {Fudamoto}},
  \bibinfo {author} {\bibfnamefont {K.~M.}\ \bibnamefont {Kojima}}, \bibinfo
  {author} {\bibfnamefont {M.~I.}\ \bibnamefont {Larkin}}, \bibinfo {author}
  {\bibfnamefont {J.}~\bibnamefont {Merrin}}, \bibinfo {author} {\bibfnamefont
  {B.}~\bibnamefont {Nachumi}}, \bibinfo {author} {\bibfnamefont {Y.~J.}\
  \bibnamefont {Uemura}}, \bibinfo {author} {\bibfnamefont {Y.}~\bibnamefont
  {Maeno}}, \bibinfo {author} {\bibfnamefont {Z.~Q.}\ \bibnamefont {Mao}},
  \bibinfo {author} {\bibfnamefont {Y.}~\bibnamefont {Mori}}, \bibinfo {author}
  {\bibfnamefont {H.}~\bibnamefont {Nakamura}},\ and\ \bibinfo {author}
  {\bibfnamefont {M.}~\bibnamefont {Sigrist}},\ }\bibfield  {title} {\bibinfo
  {title} {Time-reversal symmetry-breaking superconductivity in
  {${\mathrm{Sr}}_{2}{\mathrm{RuO}}_{4}$}},\ }\href
  {https://doi.org/10.1038/29038} {\bibfield  {journal} {\bibinfo  {journal}
  {Nature}\ }\textbf {\bibinfo {volume} {394}},\ \bibinfo {pages} {558}
  (\bibinfo {year} {1998})}\BibitemShut {NoStop}%
\bibitem [{\citenamefont {Xia}\ \emph {et~al.}(2006)\citenamefont {Xia},
  \citenamefont {Maeno}, \citenamefont {Beyersdorf}, \citenamefont {Fejer},\
  and\ \citenamefont {Kapitulnik}}]{xia2006high}%
  \BibitemOpen
  \bibfield  {author} {\bibinfo {author} {\bibfnamefont {J.}~\bibnamefont
  {Xia}}, \bibinfo {author} {\bibfnamefont {Y.}~\bibnamefont {Maeno}}, \bibinfo
  {author} {\bibfnamefont {P.~T.}\ \bibnamefont {Beyersdorf}}, \bibinfo
  {author} {\bibfnamefont {M.~M.}\ \bibnamefont {Fejer}},\ and\ \bibinfo
  {author} {\bibfnamefont {A.}~\bibnamefont {Kapitulnik}},\ }\bibfield  {title}
  {\bibinfo {title} {High resolution polar kerr effect measurements of
  {${\mathrm{Sr}}_{2}{\mathrm{RuO}}_{4}$}: Evidence for broken time-reversal
  symmetry in the superconducting state},\ }\href
  {https://doi.org/10.1103/PhysRevLett.97.167002} {\bibfield  {journal}
  {\bibinfo  {journal} {Phys. Rev. Lett.}\ }\textbf {\bibinfo {volume} {97}},\
  \bibinfo {pages} {167002} (\bibinfo {year} {2006})}\BibitemShut {NoStop}%
\bibitem [{\citenamefont {Kidwingira}\ \emph {et~al.}(2006)\citenamefont
  {Kidwingira}, \citenamefont {Strand}, \citenamefont {Van~Harlingen},\ and\
  \citenamefont {Maeno}}]{KidwingiraScience2006}%
  \BibitemOpen
  \bibfield  {author} {\bibinfo {author} {\bibfnamefont {F.}~\bibnamefont
  {Kidwingira}}, \bibinfo {author} {\bibfnamefont {J.~D.}\ \bibnamefont
  {Strand}}, \bibinfo {author} {\bibfnamefont {D.~J.}\ \bibnamefont
  {Van~Harlingen}},\ and\ \bibinfo {author} {\bibfnamefont {Y.}~\bibnamefont
  {Maeno}},\ }\bibfield  {title} {\bibinfo {title} {Dynamical superconducting
  order parameter domains in {${\mathrm{Sr}}_{2}{\mathrm{RuO}}_{4}$}},\
  }\href@noop {} {\bibfield  {journal} {\bibinfo  {journal} {Science}\ }\textbf
  {\bibinfo {volume} {314}},\ \bibinfo {pages} {1267} (\bibinfo {year}
  {2006})}\BibitemShut {NoStop}%
\bibitem [{\citenamefont {Lupien}\ \emph {et~al.}(2001)\citenamefont {Lupien},
  \citenamefont {MacFarlane}, \citenamefont {Proust}, \citenamefont
  {Taillefer}, \citenamefont {Mao},\ and\ \citenamefont
  {Maeno}}]{LupienPRL2001}%
  \BibitemOpen
  \bibfield  {author} {\bibinfo {author} {\bibfnamefont {C.}~\bibnamefont
  {Lupien}}, \bibinfo {author} {\bibfnamefont {W.~A.}\ \bibnamefont
  {MacFarlane}}, \bibinfo {author} {\bibfnamefont {C.}~\bibnamefont {Proust}},
  \bibinfo {author} {\bibfnamefont {L.}~\bibnamefont {Taillefer}}, \bibinfo
  {author} {\bibfnamefont {Z.~Q.}\ \bibnamefont {Mao}},\ and\ \bibinfo {author}
  {\bibfnamefont {Y.}~\bibnamefont {Maeno}},\ }\bibfield  {title} {\bibinfo
  {title} {Ultrasound attenuation in {${\mathrm{Sr}}_{2}{\mathrm{RuO}}_{4}$}:
  An angle-resolved study of the superconducting gap function},\ }\href@noop {}
  {\bibfield  {journal} {\bibinfo  {journal} {Phys. Rev. Lett.}\ }\textbf
  {\bibinfo {volume} {86}},\ \bibinfo {pages} {5986} (\bibinfo {year}
  {2001})}\BibitemShut {NoStop}%
\bibitem [{\citenamefont {Hassinger}\ \emph {et~al.}(2017)\citenamefont
  {Hassinger}, \citenamefont {Bourgeois-Hope}, \citenamefont {Taniguchi},
  \citenamefont {Ren\'e~de Cotret}, \citenamefont {Grissonnanche},
  \citenamefont {Anwar}, \citenamefont {Maeno}, \citenamefont
  {Doiron-Leyraud},\ and\ \citenamefont {Taillefer}}]{HassingerPRX2017}%
  \BibitemOpen
  \bibfield  {author} {\bibinfo {author} {\bibfnamefont {E.}~\bibnamefont
  {Hassinger}}, \bibinfo {author} {\bibfnamefont {P.}~\bibnamefont
  {Bourgeois-Hope}}, \bibinfo {author} {\bibfnamefont {H.}~\bibnamefont
  {Taniguchi}}, \bibinfo {author} {\bibfnamefont {S.}~\bibnamefont {Ren\'e~de
  Cotret}}, \bibinfo {author} {\bibfnamefont {G.}~\bibnamefont
  {Grissonnanche}}, \bibinfo {author} {\bibfnamefont {M.~S.}\ \bibnamefont
  {Anwar}}, \bibinfo {author} {\bibfnamefont {Y.}~\bibnamefont {Maeno}},
  \bibinfo {author} {\bibfnamefont {N.}~\bibnamefont {Doiron-Leyraud}},\ and\
  \bibinfo {author} {\bibfnamefont {L.}~\bibnamefont {Taillefer}},\ }\bibfield
  {title} {\bibinfo {title} {Vertical line nodes in the superconducting gap
  structure of {${\mathrm{Sr}}_{2}{\mathrm{RuO}}_{4}$}},\ }\href@noop {}
  {\bibfield  {journal} {\bibinfo  {journal} {Phys. Rev. X}\ }\textbf {\bibinfo
  {volume} {7}},\ \bibinfo {pages} {011032} (\bibinfo {year}
  {2017})}\BibitemShut {NoStop}%
\bibitem [{\citenamefont {Sharma}\ \emph {et~al.}(2020)\citenamefont {Sharma},
  \citenamefont {Edkins}, \citenamefont {Wang}, \citenamefont {Kostin},
  \citenamefont {Sow}, \citenamefont {Maeno}, \citenamefont {Mackenzie},
  \citenamefont {Davis},\ and\ \citenamefont {Madhavan}}]{SharmaPNAS2020}%
  \BibitemOpen
  \bibfield  {author} {\bibinfo {author} {\bibfnamefont {R.}~\bibnamefont
  {Sharma}}, \bibinfo {author} {\bibfnamefont {S.~D.}\ \bibnamefont {Edkins}},
  \bibinfo {author} {\bibfnamefont {Z.}~\bibnamefont {Wang}}, \bibinfo {author}
  {\bibfnamefont {A.}~\bibnamefont {Kostin}}, \bibinfo {author} {\bibfnamefont
  {C.}~\bibnamefont {Sow}}, \bibinfo {author} {\bibfnamefont {Y.}~\bibnamefont
  {Maeno}}, \bibinfo {author} {\bibfnamefont {A.~P.}\ \bibnamefont
  {Mackenzie}}, \bibinfo {author} {\bibfnamefont {J.~C.~S.}\ \bibnamefont
  {Davis}},\ and\ \bibinfo {author} {\bibfnamefont {V.}~\bibnamefont
  {Madhavan}},\ }\bibfield  {title} {\bibinfo {title} {Momentum-resolved
  superconducting energy gaps of {${\mathrm{Sr}}_{2}{\mathrm{RuO}}_{4}$} from
  quasiparticle interference imaging},\ }\href
  {https://doi.org/10.1073/pnas.1916463117} {\bibfield  {journal} {\bibinfo
  {journal} {Proceedings of the National Academy of Sciences}\ }\textbf
  {\bibinfo {volume} {117}},\ \bibinfo {pages} {5222} (\bibinfo {year}
  {2020})}\BibitemShut {NoStop}%
\bibitem [{\citenamefont {Ghosh}\ \emph {et~al.}(2021)\citenamefont {Ghosh},
  \citenamefont {Shekhter}, \citenamefont {Jerzembeck}, \citenamefont
  {Kikugawa}, \citenamefont {Sokolov}, \citenamefont {Brando}, \citenamefont
  {Mackenzie}, \citenamefont {Hicks},\ and\ \citenamefont
  {Ramshaw}}]{GhoshNatPhys2020}%
  \BibitemOpen
  \bibfield  {author} {\bibinfo {author} {\bibfnamefont {S.}~\bibnamefont
  {Ghosh}}, \bibinfo {author} {\bibfnamefont {A.}~\bibnamefont {Shekhter}},
  \bibinfo {author} {\bibfnamefont {F.}~\bibnamefont {Jerzembeck}}, \bibinfo
  {author} {\bibfnamefont {N.}~\bibnamefont {Kikugawa}}, \bibinfo {author}
  {\bibfnamefont {D.~A.}\ \bibnamefont {Sokolov}}, \bibinfo {author}
  {\bibfnamefont {M.}~\bibnamefont {Brando}}, \bibinfo {author} {\bibfnamefont
  {A.~P.}\ \bibnamefont {Mackenzie}}, \bibinfo {author} {\bibfnamefont {C.~W.}\
  \bibnamefont {Hicks}},\ and\ \bibinfo {author} {\bibfnamefont {B.~J.}\
  \bibnamefont {Ramshaw}},\ }\bibfield  {title} {\bibinfo {title}
  {Thermodynamic evidence for a two-component superconducting order parameter
  in {${\mathrm{Sr}}_{2}{\mathrm{RuO}}_{4}$}},\ }\href
  {https://doi.org/10.1038/s41567-020-1032-4} {\bibfield  {journal} {\bibinfo
  {journal} {Nature Physics}\ }\textbf {\bibinfo {volume} {17}},\ \bibinfo
  {pages} {199} (\bibinfo {year} {2021})}\BibitemShut {NoStop}%
\bibitem [{\citenamefont {Benhabib}\ \emph {et~al.}(2021)\citenamefont
  {Benhabib}, \citenamefont {Lupien}, \citenamefont {Paul}, \citenamefont
  {Berges}, \citenamefont {Dion}, \citenamefont {Nardone}, \citenamefont
  {Zitouni}, \citenamefont {Mao}, \citenamefont {Maeno}, \citenamefont
  {Georges}, \citenamefont {Taillefer},\ and\ \citenamefont
  {Proust}}]{BenhabibNatPhys2020}%
  \BibitemOpen
  \bibfield  {author} {\bibinfo {author} {\bibfnamefont {S.}~\bibnamefont
  {Benhabib}}, \bibinfo {author} {\bibfnamefont {C.}~\bibnamefont {Lupien}},
  \bibinfo {author} {\bibfnamefont {I.}~\bibnamefont {Paul}}, \bibinfo {author}
  {\bibfnamefont {L.}~\bibnamefont {Berges}}, \bibinfo {author} {\bibfnamefont
  {M.}~\bibnamefont {Dion}}, \bibinfo {author} {\bibfnamefont {M.}~\bibnamefont
  {Nardone}}, \bibinfo {author} {\bibfnamefont {A.}~\bibnamefont {Zitouni}},
  \bibinfo {author} {\bibfnamefont {Z.~Q.}\ \bibnamefont {Mao}}, \bibinfo
  {author} {\bibfnamefont {Y.}~\bibnamefont {Maeno}}, \bibinfo {author}
  {\bibfnamefont {A.}~\bibnamefont {Georges}}, \bibinfo {author} {\bibfnamefont
  {L.}~\bibnamefont {Taillefer}},\ and\ \bibinfo {author} {\bibfnamefont
  {C.}~\bibnamefont {Proust}},\ }\bibfield  {title} {\bibinfo {title}
  {Ultrasound evidence for a two-component superconducting order parameter in
  {${\mathrm{Sr}}_{2}{\mathrm{RuO}}_{4}$}},\ }\href
  {https://doi.org/10.1038/s41567-020-1033-3} {\bibfield  {journal} {\bibinfo
  {journal} {Nature Physics}\ }\textbf {\bibinfo {volume} {17}},\ \bibinfo
  {pages} {194} (\bibinfo {year} {2021})}\BibitemShut {NoStop}%
\bibitem [{\citenamefont {Ramires}\ and\ \citenamefont
  {Sigrist}(2019)}]{AlinePRB2019}%
  \BibitemOpen
  \bibfield  {author} {\bibinfo {author} {\bibfnamefont {A.}~\bibnamefont
  {Ramires}}\ and\ \bibinfo {author} {\bibfnamefont {M.}~\bibnamefont
  {Sigrist}},\ }\bibfield  {title} {\bibinfo {title} {Superconducting order
  parameter of {${\mathrm{Sr}}_{2}{\mathrm{RuO}}_{4}$}: A microscopic
  perspective},\ }\href@noop {} {\bibfield  {journal} {\bibinfo  {journal}
  {Phys. Rev. B}\ }\textbf {\bibinfo {volume} {100}},\ \bibinfo {pages}
  {104501} (\bibinfo {year} {2019})}\BibitemShut {NoStop}%
\bibitem [{\citenamefont {R\o{}mer}\ \emph {et~al.}(2019)\citenamefont
  {R\o{}mer}, \citenamefont {Scherer}, \citenamefont {Eremin}, \citenamefont
  {Hirschfeld},\ and\ \citenamefont {Andersen}}]{romer2019knight}%
  \BibitemOpen
  \bibfield  {author} {\bibinfo {author} {\bibfnamefont {A.~T.}\ \bibnamefont
  {R\o{}mer}}, \bibinfo {author} {\bibfnamefont {D.~D.}\ \bibnamefont
  {Scherer}}, \bibinfo {author} {\bibfnamefont {I.~M.}\ \bibnamefont {Eremin}},
  \bibinfo {author} {\bibfnamefont {P.~J.}\ \bibnamefont {Hirschfeld}},\ and\
  \bibinfo {author} {\bibfnamefont {B.~M.}\ \bibnamefont {Andersen}},\
  }\bibfield  {title} {\bibinfo {title} {Knight shift and leading
  superconducting instability from spin fluctuations in
  {${\mathrm{Sr}}_{2}{\mathrm{RuO}}_{4}$}},\ }\href@noop {} {\bibfield
  {journal} {\bibinfo  {journal} {Phys. Rev. Lett.}\ }\textbf {\bibinfo
  {volume} {123}},\ \bibinfo {pages} {247001} (\bibinfo {year}
  {2019})}\BibitemShut {NoStop}%
\bibitem [{\citenamefont {R\o{}ising}\ \emph {et~al.}(2019)\citenamefont
  {R\o{}ising}, \citenamefont {Scaffidi}, \citenamefont {Flicker},
  \citenamefont {Lange},\ and\ \citenamefont {Simon}}]{RoisingPRR2019}%
  \BibitemOpen
  \bibfield  {author} {\bibinfo {author} {\bibfnamefont {H.~S.}\ \bibnamefont
  {R\o{}ising}}, \bibinfo {author} {\bibfnamefont {T.}~\bibnamefont
  {Scaffidi}}, \bibinfo {author} {\bibfnamefont {F.}~\bibnamefont {Flicker}},
  \bibinfo {author} {\bibfnamefont {G.~F.}\ \bibnamefont {Lange}},\ and\
  \bibinfo {author} {\bibfnamefont {S.~H.}\ \bibnamefont {Simon}},\ }\bibfield
  {title} {\bibinfo {title} {Superconducting order of
  {${\mathrm{Sr}}_{2}{\mathrm{RuO}}_{4}$} from a three-dimensional microscopic
  model},\ }\href {https://doi.org/10.1103/PhysRevResearch.1.033108} {\bibfield
   {journal} {\bibinfo  {journal} {Phys. Rev. Research}\ }\textbf {\bibinfo
  {volume} {1}},\ \bibinfo {pages} {033108} (\bibinfo {year}
  {2019})}\BibitemShut {NoStop}%
\bibitem [{\citenamefont {{Scaffidi}}(2020)}]{ScaffidiArxiv2020}%
  \BibitemOpen
  \bibfield  {author} {\bibinfo {author} {\bibfnamefont {T.}~\bibnamefont
  {{Scaffidi}}},\ }\href@noop {} {\bibinfo {title} {{Degeneracy between even-
  and odd-parity superconductivity in the quasi-1D Hubbard model and
  implications for {${\mathrm{Sr}}_{2}{\mathrm{RuO}}_{4}$}}}} (\bibinfo {year}
  {2020}),\ \Eprint {https://arxiv.org/abs/2007.13769} {arXiv:2007.13769}
  \BibitemShut {NoStop}%
\bibitem [{\citenamefont {Suh}\ \emph {et~al.}(2020)\citenamefont {Suh},
  \citenamefont {Menke}, \citenamefont {Brydon}, \citenamefont {Timm},
  \citenamefont {Ramires},\ and\ \citenamefont {Agterberg}}]{SuhPRR2020}%
  \BibitemOpen
  \bibfield  {author} {\bibinfo {author} {\bibfnamefont {H.~G.}\ \bibnamefont
  {Suh}}, \bibinfo {author} {\bibfnamefont {H.}~\bibnamefont {Menke}}, \bibinfo
  {author} {\bibfnamefont {P.~M.~R.}\ \bibnamefont {Brydon}}, \bibinfo {author}
  {\bibfnamefont {C.}~\bibnamefont {Timm}}, \bibinfo {author} {\bibfnamefont
  {A.}~\bibnamefont {Ramires}},\ and\ \bibinfo {author} {\bibfnamefont {D.~F.}\
  \bibnamefont {Agterberg}},\ }\bibfield  {title} {\bibinfo {title}
  {Stabilizing even-parity chiral superconductivity in
  {${\mathrm{Sr}}_{2}{\mathrm{RuO}}_{4}$}},\ }\href
  {https://doi.org/10.1103/PhysRevResearch.2.032023} {\bibfield  {journal}
  {\bibinfo  {journal} {Phys. Rev. Research}\ }\textbf {\bibinfo {volume}
  {2}},\ \bibinfo {pages} {032023} (\bibinfo {year} {2020})}\BibitemShut
  {NoStop}%
\bibitem [{\citenamefont {Kivelson}\ \emph {et~al.}(2020)\citenamefont
  {Kivelson}, \citenamefont {Yuan}, \citenamefont {Ramshaw},\ and\
  \citenamefont {Thomale}}]{KivelsonNPJ2020}%
  \BibitemOpen
  \bibfield  {author} {\bibinfo {author} {\bibfnamefont {S.~A.}\ \bibnamefont
  {Kivelson}}, \bibinfo {author} {\bibfnamefont {A.~C.}\ \bibnamefont {Yuan}},
  \bibinfo {author} {\bibfnamefont {B.}~\bibnamefont {Ramshaw}},\ and\ \bibinfo
  {author} {\bibfnamefont {R.}~\bibnamefont {Thomale}},\ }\bibfield  {title}
  {\bibinfo {title} {A proposal for reconciling diverse experiments on the
  superconducting state in {${\mathrm{Sr}}_{2}{\mathrm{RuO}}_{4}$}},\ }\href
  {https://doi.org/10.1038/s41535-020-0245-1} {\bibfield  {journal} {\bibinfo
  {journal} {npj Quantum Materials}\ }\textbf {\bibinfo {volume} {5}},\
  \bibinfo {pages} {43} (\bibinfo {year} {2020})}\BibitemShut {NoStop}%
\bibitem [{\citenamefont {Willa}\ \emph {et~al.}(2021)\citenamefont {Willa},
  \citenamefont {Hecker}, \citenamefont {Fernandes},\ and\ \citenamefont
  {Schmalian}}]{WillaPRB2021}%
  \BibitemOpen
  \bibfield  {author} {\bibinfo {author} {\bibfnamefont {R.}~\bibnamefont
  {Willa}}, \bibinfo {author} {\bibfnamefont {M.}~\bibnamefont {Hecker}},
  \bibinfo {author} {\bibfnamefont {R.~M.}\ \bibnamefont {Fernandes}},\ and\
  \bibinfo {author} {\bibfnamefont {J.}~\bibnamefont {Schmalian}},\ }\bibfield
  {title} {\bibinfo {title} {Inhomogeneous time-reversal symmetry breaking in
  {${\mathrm{Sr}}_{2}{\mathrm{RuO}}_{4}$}},\ }\href
  {https://doi.org/10.1103/PhysRevB.104.024511} {\bibfield  {journal} {\bibinfo
   {journal} {Phys. Rev. B}\ }\textbf {\bibinfo {volume} {104}},\ \bibinfo
  {pages} {024511} (\bibinfo {year} {2021})}\BibitemShut {NoStop}%
\bibitem [{\citenamefont {Chung}\ \emph {et~al.}(2012)\citenamefont {Chung},
  \citenamefont {Raghu}, \citenamefont {Kapitulnik},\ and\ \citenamefont
  {Kivelson}}]{chung2012charge}%
  \BibitemOpen
  \bibfield  {author} {\bibinfo {author} {\bibfnamefont {S.~B.}\ \bibnamefont
  {Chung}}, \bibinfo {author} {\bibfnamefont {S.}~\bibnamefont {Raghu}},
  \bibinfo {author} {\bibfnamefont {A.}~\bibnamefont {Kapitulnik}},\ and\
  \bibinfo {author} {\bibfnamefont {S.~A.}\ \bibnamefont {Kivelson}},\
  }\bibfield  {title} {\bibinfo {title} {Charge and spin collective modes in a
  quasi-one-dimensional model of {${\mathrm{Sr}}_{2}{\mathrm{RuO}}_{4}$}},\
  }\href {https://doi.org/10.1103/PhysRevB.86.064525} {\bibfield  {journal}
  {\bibinfo  {journal} {Phys. Rev. B}\ }\textbf {\bibinfo {volume} {86}},\
  \bibinfo {pages} {064525} (\bibinfo {year} {2012})}\BibitemShut {NoStop}%
\bibitem [{\citenamefont {Yuan}\ \emph {et~al.}(2021)\citenamefont {Yuan},
  \citenamefont {Berg},\ and\ \citenamefont {Kivelson}}]{Yuan2021}%
  \BibitemOpen
  \bibfield  {author} {\bibinfo {author} {\bibfnamefont {A.~C.}\ \bibnamefont
  {Yuan}}, \bibinfo {author} {\bibfnamefont {E.}~\bibnamefont {Berg}},\ and\
  \bibinfo {author} {\bibfnamefont {S.~A.}\ \bibnamefont {Kivelson}},\
  }\href@noop {} {\bibinfo {title} {Strain-induced time reversal breaking and
  half quantum vortices near a putative superconducting tetra-critical point in
  {${\mathrm{Sr}}_{2}{\mathrm{RuO}}_{4}$}}} (\bibinfo {year} {2021}),\ \Eprint
  {https://arxiv.org/abs/2106.00935} {arXiv:2106.00935} \BibitemShut {NoStop}%
\bibitem [{\citenamefont {Brugger}(1965)}]{Brugger1965}%
  \BibitemOpen
  \bibfield  {author} {\bibinfo {author} {\bibfnamefont {K.}~\bibnamefont
  {Brugger}},\ }\bibfield  {title} {\bibinfo {title} {{Pure modes for elastic
  waves in crystals}},\ }\href {https://doi.org/10.1063/1.1714215} {\bibfield
  {journal} {\bibinfo  {journal} {Journal of Applied Physics}\ }\textbf
  {\bibinfo {volume} {36}},\ \bibinfo {pages} {759} (\bibinfo {year}
  {1965})}\BibitemShut {NoStop}%
\bibitem [{\citenamefont {Bobowski}\ \emph {et~al.}(2019)\citenamefont
  {Bobowski}, \citenamefont {Kikugawa}, \citenamefont {Miyoshi}, \citenamefont
  {Suwa}, \citenamefont {Xu}, \citenamefont {Yonezawa}, \citenamefont
  {Sokolov}, \citenamefont {Mackenzie},\ and\ \citenamefont
  {Maeno}}]{BobowskiCondMat2019}%
  \BibitemOpen
  \bibfield  {author} {\bibinfo {author} {\bibfnamefont {J.~S.}\ \bibnamefont
  {Bobowski}}, \bibinfo {author} {\bibfnamefont {N.}~\bibnamefont {Kikugawa}},
  \bibinfo {author} {\bibfnamefont {T.}~\bibnamefont {Miyoshi}}, \bibinfo
  {author} {\bibfnamefont {H.}~\bibnamefont {Suwa}}, \bibinfo {author}
  {\bibfnamefont {H.-S.}\ \bibnamefont {Xu}}, \bibinfo {author} {\bibfnamefont
  {S.}~\bibnamefont {Yonezawa}}, \bibinfo {author} {\bibfnamefont {D.~A.}\
  \bibnamefont {Sokolov}}, \bibinfo {author} {\bibfnamefont {A.~P.}\
  \bibnamefont {Mackenzie}},\ and\ \bibinfo {author} {\bibfnamefont
  {Y.}~\bibnamefont {Maeno}},\ }\bibfield  {title} {\bibinfo {title} {Improved
  single-crystal growth of {${\mathrm{Sr}}_{2}{\mathrm{RuO}}_{4}$}},\
  }\href@noop {} {\bibfield  {journal} {\bibinfo  {journal} {Condensed Matter}\
  }\textbf {\bibinfo {volume} {4}} (\bibinfo {year} {2019})}\BibitemShut
  {NoStop}%
\bibitem [{\citenamefont {Ramshaw}\ \emph {et~al.}(2015)\citenamefont
  {Ramshaw}, \citenamefont {Shekhter}, \citenamefont {McDonald}, \citenamefont
  {Betts}, \citenamefont {Mitchell}, \citenamefont {Tobash}, \citenamefont
  {Mielke}, \citenamefont {Bauer},\ and\ \citenamefont
  {Migliori}}]{RamshawPNAS}%
  \BibitemOpen
  \bibfield  {author} {\bibinfo {author} {\bibfnamefont {B.~J.}\ \bibnamefont
  {Ramshaw}}, \bibinfo {author} {\bibfnamefont {A.}~\bibnamefont {Shekhter}},
  \bibinfo {author} {\bibfnamefont {R.~D.}\ \bibnamefont {McDonald}}, \bibinfo
  {author} {\bibfnamefont {J.~B.}\ \bibnamefont {Betts}}, \bibinfo {author}
  {\bibfnamefont {J.~N.}\ \bibnamefont {Mitchell}}, \bibinfo {author}
  {\bibfnamefont {P.~H.}\ \bibnamefont {Tobash}}, \bibinfo {author}
  {\bibfnamefont {C.~H.}\ \bibnamefont {Mielke}}, \bibinfo {author}
  {\bibfnamefont {E.~D.}\ \bibnamefont {Bauer}},\ and\ \bibinfo {author}
  {\bibfnamefont {A.}~\bibnamefont {Migliori}},\ }\bibfield  {title} {\bibinfo
  {title} {Avoided valence transition in a plutonium superconductor},\
  }\href@noop {} {\bibfield  {journal} {\bibinfo  {journal} {Proceedings of the
  National Academy of Sciences}\ }\textbf {\bibinfo {volume} {112}},\ \bibinfo
  {pages} {3285} (\bibinfo {year} {2015})}\BibitemShut {NoStop}%
\bibitem [{\citenamefont {Ghosh}\ \emph {et~al.}(2020)\citenamefont {Ghosh},
  \citenamefont {Matty}, \citenamefont {Baumbach}, \citenamefont {Bauer},
  \citenamefont {Modic}, \citenamefont {Shekhter}, \citenamefont {Mydosh},
  \citenamefont {Kim},\ and\ \citenamefont {Ramshaw}}]{GhoshSciAdv2020}%
  \BibitemOpen
  \bibfield  {author} {\bibinfo {author} {\bibfnamefont {S.}~\bibnamefont
  {Ghosh}}, \bibinfo {author} {\bibfnamefont {M.}~\bibnamefont {Matty}},
  \bibinfo {author} {\bibfnamefont {R.}~\bibnamefont {Baumbach}}, \bibinfo
  {author} {\bibfnamefont {E.~D.}\ \bibnamefont {Bauer}}, \bibinfo {author}
  {\bibfnamefont {K.~A.}\ \bibnamefont {Modic}}, \bibinfo {author}
  {\bibfnamefont {A.}~\bibnamefont {Shekhter}}, \bibinfo {author}
  {\bibfnamefont {J.~A.}\ \bibnamefont {Mydosh}}, \bibinfo {author}
  {\bibfnamefont {E.-A.}\ \bibnamefont {Kim}},\ and\ \bibinfo {author}
  {\bibfnamefont {B.~J.}\ \bibnamefont {Ramshaw}},\ }\bibfield  {title}
  {\bibinfo {title} {{One-component order parameter in
  ${\mathrm{U}\mathrm{R}\mathrm{u}_{2}\mathrm{S}\mathrm{i}_{2}}$ uncovered by
  resonant ultrasound spectroscopy and machine learning}},\ }\href@noop {}
  {\bibfield  {journal} {\bibinfo  {journal} {Science Advances}\ }\textbf
  {\bibinfo {volume} {6}} (\bibinfo {year} {2020})}\BibitemShut {NoStop}%
\bibitem [{\citenamefont {Shekhter}\ \emph {et~al.}(2013)\citenamefont
  {Shekhter}, \citenamefont {Ramshaw}, \citenamefont {Liang}, \citenamefont
  {Hardy}, \citenamefont {Bonn}, \citenamefont {Balakirev}, \citenamefont
  {McDonald}, \citenamefont {Betts}, \citenamefont {Riggs},\ and\ \citenamefont
  {Migliori}}]{ShekhterNat2013}%
  \BibitemOpen
  \bibfield  {author} {\bibinfo {author} {\bibfnamefont {A.}~\bibnamefont
  {Shekhter}}, \bibinfo {author} {\bibfnamefont {B.~J.}\ \bibnamefont
  {Ramshaw}}, \bibinfo {author} {\bibfnamefont {R.}~\bibnamefont {Liang}},
  \bibinfo {author} {\bibfnamefont {W.~N.}\ \bibnamefont {Hardy}}, \bibinfo
  {author} {\bibfnamefont {D.~A.}\ \bibnamefont {Bonn}}, \bibinfo {author}
  {\bibfnamefont {F.~F.}\ \bibnamefont {Balakirev}}, \bibinfo {author}
  {\bibfnamefont {R.~D.}\ \bibnamefont {McDonald}}, \bibinfo {author}
  {\bibfnamefont {J.~B.}\ \bibnamefont {Betts}}, \bibinfo {author}
  {\bibfnamefont {S.~C.}\ \bibnamefont {Riggs}},\ and\ \bibinfo {author}
  {\bibfnamefont {A.}~\bibnamefont {Migliori}},\ }\bibfield  {title} {\bibinfo
  {title} {Bounding the pseudogap with a line of phase transitions in
  {YBa$_2\mathrm{Cu_3O_{6+\delta}}$}},\ }\href@noop {} {\bibfield  {journal}
  {\bibinfo  {journal} {Nature}\ }\textbf {\bibinfo {volume} {498}},\ \bibinfo
  {pages} {75 EP } (\bibinfo {year} {2013})}\BibitemShut {NoStop}%
\bibitem [{\citenamefont {Khan}\ and\ \citenamefont {Allen}(1987)}]{Khan1987}%
  \BibitemOpen
  \bibfield  {author} {\bibinfo {author} {\bibfnamefont {F.~S.}\ \bibnamefont
  {Khan}}\ and\ \bibinfo {author} {\bibfnamefont {P.~B.}\ \bibnamefont
  {Allen}},\ }\bibfield  {title} {\bibinfo {title} {Sound attenuation by
  electrons in metals},\ }\href {https://doi.org/10.1103/PhysRevB.35.1002}
  {\bibfield  {journal} {\bibinfo  {journal} {Phys. Rev. B}\ }\textbf {\bibinfo
  {volume} {35}},\ \bibinfo {pages} {1002} (\bibinfo {year}
  {1987})}\BibitemShut {NoStop}%
\bibitem [{\citenamefont {Lupien}(2002)}]{LupienThesis}%
  \BibitemOpen
  \bibfield  {author} {\bibinfo {author} {\bibfnamefont {C.}~\bibnamefont
  {Lupien}},\ }\emph {\bibinfo {title} {Ultrasound attenuation in the
  unconventional superconductor {${\mathrm{Sr}}_{2}{\mathrm{RuO}}_{4}$}}},\
  \href@noop {} {Ph.D. thesis} (\bibinfo {year} {2002})\BibitemShut {NoStop}%
\bibitem [{\citenamefont {Barber}\ \emph {et~al.}(2019)\citenamefont {Barber},
  \citenamefont {Lechermann}, \citenamefont {Streltsov}, \citenamefont
  {Skornyakov}, \citenamefont {Ghosh}, \citenamefont {Ramshaw}, \citenamefont
  {Kikugawa}, \citenamefont {Sokolov}, \citenamefont {Mackenzie}, \citenamefont
  {Hicks},\ and\ \citenamefont {Mazin}}]{barber2019role}%
  \BibitemOpen
  \bibfield  {author} {\bibinfo {author} {\bibfnamefont {M.~E.}\ \bibnamefont
  {Barber}}, \bibinfo {author} {\bibfnamefont {F.}~\bibnamefont {Lechermann}},
  \bibinfo {author} {\bibfnamefont {S.~V.}\ \bibnamefont {Streltsov}}, \bibinfo
  {author} {\bibfnamefont {S.~L.}\ \bibnamefont {Skornyakov}}, \bibinfo
  {author} {\bibfnamefont {S.}~\bibnamefont {Ghosh}}, \bibinfo {author}
  {\bibfnamefont {B.~J.}\ \bibnamefont {Ramshaw}}, \bibinfo {author}
  {\bibfnamefont {N.}~\bibnamefont {Kikugawa}}, \bibinfo {author}
  {\bibfnamefont {D.~A.}\ \bibnamefont {Sokolov}}, \bibinfo {author}
  {\bibfnamefont {A.~P.}\ \bibnamefont {Mackenzie}}, \bibinfo {author}
  {\bibfnamefont {C.~W.}\ \bibnamefont {Hicks}},\ and\ \bibinfo {author}
  {\bibfnamefont {I.~I.}\ \bibnamefont {Mazin}},\ }\bibfield  {title} {\bibinfo
  {title} {Role of correlations in determining the van hove strain in
  {${\mathrm{Sr}}_{2}{\mathrm{RuO}}_{4}$}},\ }\href
  {https://doi.org/10.1103/PhysRevB.100.245139} {\bibfield  {journal} {\bibinfo
   {journal} {Phys. Rev. B}\ }\textbf {\bibinfo {volume} {100}},\ \bibinfo
  {pages} {245139} (\bibinfo {year} {2019})}\BibitemShut {NoStop}%
\bibitem [{\citenamefont {Won}\ and\ \citenamefont {Maki}(1994)}]{Maki1994}%
  \BibitemOpen
  \bibfield  {author} {\bibinfo {author} {\bibfnamefont {H.}~\bibnamefont
  {Won}}\ and\ \bibinfo {author} {\bibfnamefont {K.}~\bibnamefont {Maki}},\
  }\bibfield  {title} {\bibinfo {title} {d-wave superconductor as a model of
  high-{${\mathit{T}}_{\mathit{c}}$} superconductors},\ }\href
  {https://doi.org/10.1103/PhysRevB.49.1397} {\bibfield  {journal} {\bibinfo
  {journal} {Phys. Rev. B}\ }\textbf {\bibinfo {volume} {49}},\ \bibinfo
  {pages} {1397} (\bibinfo {year} {1994})}\BibitemShut {NoStop}%
\bibitem [{\citenamefont {Vekhter}\ \emph {et~al.}(1999)\citenamefont
  {Vekhter}, \citenamefont {Nicol},\ and\ \citenamefont
  {Carbotte}}]{Vekhter1999}%
  \BibitemOpen
  \bibfield  {author} {\bibinfo {author} {\bibfnamefont {I.}~\bibnamefont
  {Vekhter}}, \bibinfo {author} {\bibfnamefont {E.~J.}\ \bibnamefont {Nicol}},\
  and\ \bibinfo {author} {\bibfnamefont {J.~P.}\ \bibnamefont {Carbotte}},\
  }\bibfield  {title} {\bibinfo {title} {Ultrasonic attenuation in clean d-wave
  superconductors},\ }\href {https://doi.org/10.1103/PhysRevB.59.7123}
  {\bibfield  {journal} {\bibinfo  {journal} {Phys. Rev. B}\ }\textbf {\bibinfo
  {volume} {59}},\ \bibinfo {pages} {7123} (\bibinfo {year}
  {1999})}\BibitemShut {NoStop}%
\bibitem [{\citenamefont {Adenwalla}\ \emph {et~al.}(1989)\citenamefont
  {Adenwalla}, \citenamefont {Zhao}, \citenamefont {Ketterson},\ and\
  \citenamefont {Sarma}}]{AdenwallaPRL1989}%
  \BibitemOpen
  \bibfield  {author} {\bibinfo {author} {\bibfnamefont {S.}~\bibnamefont
  {Adenwalla}}, \bibinfo {author} {\bibfnamefont {Z.}~\bibnamefont {Zhao}},
  \bibinfo {author} {\bibfnamefont {J.~B.}\ \bibnamefont {Ketterson}},\ and\
  \bibinfo {author} {\bibfnamefont {B.~K.}\ \bibnamefont {Sarma}},\ }\bibfield
  {title} {\bibinfo {title} {Measurements of the pair-breaking edge in
  superfluid $^{3}\mathrm{He-}\mathit{B}$},\ }\href
  {https://doi.org/10.1103/PhysRevLett.63.1811} {\bibfield  {journal} {\bibinfo
   {journal} {Phys. Rev. Lett.}\ }\textbf {\bibinfo {volume} {63}},\ \bibinfo
  {pages} {1811} (\bibinfo {year} {1989})}\BibitemShut {NoStop}%
\bibitem [{\citenamefont {Firmo}\ \emph {et~al.}(2013)\citenamefont {Firmo},
  \citenamefont {Lederer}, \citenamefont {Lupien}, \citenamefont {Mackenzie},
  \citenamefont {Davis},\ and\ \citenamefont {Kivelson}}]{FirmoPRB2013}%
  \BibitemOpen
  \bibfield  {author} {\bibinfo {author} {\bibfnamefont {I.~A.}\ \bibnamefont
  {Firmo}}, \bibinfo {author} {\bibfnamefont {S.}~\bibnamefont {Lederer}},
  \bibinfo {author} {\bibfnamefont {C.}~\bibnamefont {Lupien}}, \bibinfo
  {author} {\bibfnamefont {A.~P.}\ \bibnamefont {Mackenzie}}, \bibinfo {author}
  {\bibfnamefont {J.~C.}\ \bibnamefont {Davis}},\ and\ \bibinfo {author}
  {\bibfnamefont {S.~A.}\ \bibnamefont {Kivelson}},\ }\bibfield  {title}
  {\bibinfo {title} {Evidence from tunneling spectroscopy for a
  quasi-one-dimensional origin of superconductivity in
  {${\mathrm{Sr}}_{2}{\mathrm{RuO}}_{4}$}},\ }\href
  {https://doi.org/10.1103/PhysRevB.88.134521} {\bibfield  {journal} {\bibinfo
  {journal} {Phys. Rev. B}\ }\textbf {\bibinfo {volume} {88}},\ \bibinfo
  {pages} {134521} (\bibinfo {year} {2013})}\BibitemShut {NoStop}%
\bibitem [{\citenamefont {Leonard}\ \emph {et~al.}(1962)\citenamefont
  {Leonard}, \citenamefont {Barone}, \citenamefont {Truell}, \citenamefont
  {Elbaum},\ and\ \citenamefont {Noltingk}}]{Leonard1962}%
  \BibitemOpen
  \bibfield  {author} {\bibinfo {author} {\bibfnamefont {R.}~\bibnamefont
  {Leonard}}, \bibinfo {author} {\bibfnamefont {A.}~\bibnamefont {Barone}},
  \bibinfo {author} {\bibfnamefont {R.}~\bibnamefont {Truell}}, \bibinfo
  {author} {\bibfnamefont {C.}~\bibnamefont {Elbaum}},\ and\ \bibinfo {author}
  {\bibfnamefont {B.~E.}\ \bibnamefont {Noltingk}},\ }\href@noop {} {\emph
  {\bibinfo {title} {Acoustics II}}}\ (\bibinfo  {publisher}
  {Springer-Verlag},\ \bibinfo {address} {Berlin Heidelberg},\ \bibinfo {year}
  {1962})\BibitemShut {NoStop}%
\bibitem [{\citenamefont {Pustogow}\ \emph {et~al.}(2019)\citenamefont
  {Pustogow}, \citenamefont {Luo}, \citenamefont {Chronister}, \citenamefont
  {Su}, \citenamefont {Sokolov}, \citenamefont {Jerzembeck}, \citenamefont
  {Mackenzie}, \citenamefont {Hicks}, \citenamefont {Kikugawa}, \citenamefont
  {Raghu}, \citenamefont {Bauer},\ and\ \citenamefont
  {Brown}}]{PustogowNat2019}%
  \BibitemOpen
  \bibfield  {author} {\bibinfo {author} {\bibfnamefont {A.}~\bibnamefont
  {Pustogow}}, \bibinfo {author} {\bibfnamefont {Y.}~\bibnamefont {Luo}},
  \bibinfo {author} {\bibfnamefont {A.}~\bibnamefont {Chronister}}, \bibinfo
  {author} {\bibfnamefont {Y.-S.}\ \bibnamefont {Su}}, \bibinfo {author}
  {\bibfnamefont {D.~A.}\ \bibnamefont {Sokolov}}, \bibinfo {author}
  {\bibfnamefont {F.}~\bibnamefont {Jerzembeck}}, \bibinfo {author}
  {\bibfnamefont {A.~P.}\ \bibnamefont {Mackenzie}}, \bibinfo {author}
  {\bibfnamefont {C.~W.}\ \bibnamefont {Hicks}}, \bibinfo {author}
  {\bibfnamefont {N.}~\bibnamefont {Kikugawa}}, \bibinfo {author}
  {\bibfnamefont {S.}~\bibnamefont {Raghu}}, \bibinfo {author} {\bibfnamefont
  {E.~D.}\ \bibnamefont {Bauer}},\ and\ \bibinfo {author} {\bibfnamefont
  {S.~E.}\ \bibnamefont {Brown}},\ }\bibfield  {title} {\bibinfo {title}
  {Constraints on the superconducting order parameter in
  {${\mathrm{Sr}}_{2}{\mathrm{RuO}}_{4}$} from oxygen-17 nuclear magnetic
  resonance},\ }\href@noop {} {\bibfield  {journal} {\bibinfo  {journal}
  {Nature}\ }\textbf {\bibinfo {volume} {574}},\ \bibinfo {pages} {72}
  (\bibinfo {year} {2019})}\BibitemShut {NoStop}%
\bibitem [{\citenamefont {{Clepkens}}\ \emph {et~al.}(2021)\citenamefont
  {{Clepkens}}, \citenamefont {{Lindquist}}, \citenamefont {{Liu}},\ and\
  \citenamefont {{Kee}}}]{ClepkensArxiv2021}%
  \BibitemOpen
  \bibfield  {author} {\bibinfo {author} {\bibfnamefont {J.}~\bibnamefont
  {{Clepkens}}}, \bibinfo {author} {\bibfnamefont {A.~W.}\ \bibnamefont
  {{Lindquist}}}, \bibinfo {author} {\bibfnamefont {X.}~\bibnamefont {{Liu}}},\
  and\ \bibinfo {author} {\bibfnamefont {H.-Y.}\ \bibnamefont {{Kee}}},\
  }\href@noop {} {\bibinfo {title} {{Higher angular momentum pairings in
  inter-orbital shadowed-triplet superconductors: Application to
  Sr$_{2}$RuO$_{4}$}}} (\bibinfo {year} {2021}),\ \Eprint
  {https://arxiv.org/abs/2107.00047} {arXiv:2107.00047} \BibitemShut {NoStop}%
\bibitem [{\citenamefont {R\o{}mer}\ \emph {et~al.}(2021)\citenamefont
  {R\o{}mer}, \citenamefont {Hirschfeld},\ and\ \citenamefont
  {Andersen}}]{RomerArXiv2021}%
  \BibitemOpen
  \bibfield  {author} {\bibinfo {author} {\bibfnamefont {A.~T.}\ \bibnamefont
  {R\o{}mer}}, \bibinfo {author} {\bibfnamefont {P.~J.}\ \bibnamefont
  {Hirschfeld}},\ and\ \bibinfo {author} {\bibfnamefont {B.~M.}\ \bibnamefont
  {Andersen}},\ }\bibfield  {title} {\bibinfo {title} {Superconducting state of
  ${\mathrm{sr}}_{2}{\mathrm{ruo}}_{4}$ in the presence of longer-range coulomb
  interactions},\ }\href {https://doi.org/10.1103/PhysRevB.104.064507}
  {\bibfield  {journal} {\bibinfo  {journal} {Phys. Rev. B}\ }\textbf {\bibinfo
  {volume} {104}},\ \bibinfo {pages} {064507} (\bibinfo {year}
  {2021})}\BibitemShut {NoStop}%
\bibitem [{\citenamefont {Hicks}\ \emph {et~al.}(2014)\citenamefont {Hicks},
  \citenamefont {Brodsky}, \citenamefont {Yelland}, \citenamefont {Gibbs},
  \citenamefont {Bruin}, \citenamefont {Barber}, \citenamefont {Edkins},
  \citenamefont {Nishimura}, \citenamefont {Yonezawa}, \citenamefont {Maeno},\
  and\ \citenamefont {Mackenzie}}]{HicksSc2014}%
  \BibitemOpen
  \bibfield  {author} {\bibinfo {author} {\bibfnamefont {C.~W.}\ \bibnamefont
  {Hicks}}, \bibinfo {author} {\bibfnamefont {D.~O.}\ \bibnamefont {Brodsky}},
  \bibinfo {author} {\bibfnamefont {E.~A.}\ \bibnamefont {Yelland}}, \bibinfo
  {author} {\bibfnamefont {A.~S.}\ \bibnamefont {Gibbs}}, \bibinfo {author}
  {\bibfnamefont {J.~A.~N.}\ \bibnamefont {Bruin}}, \bibinfo {author}
  {\bibfnamefont {M.~E.}\ \bibnamefont {Barber}}, \bibinfo {author}
  {\bibfnamefont {S.~D.}\ \bibnamefont {Edkins}}, \bibinfo {author}
  {\bibfnamefont {K.}~\bibnamefont {Nishimura}}, \bibinfo {author}
  {\bibfnamefont {S.}~\bibnamefont {Yonezawa}}, \bibinfo {author}
  {\bibfnamefont {Y.}~\bibnamefont {Maeno}},\ and\ \bibinfo {author}
  {\bibfnamefont {A.~P.}\ \bibnamefont {Mackenzie}},\ }\bibfield  {title}
  {\bibinfo {title} {Strong increase of {${\mathit{T}}_{\mathit{c}}$} of
  {${\mathrm{Sr}}_{2}{\mathrm{RuO}}_{4}$} under both tensile and compressive
  strain},\ }\href {https://doi.org/10.1126/science.1248292} {\bibfield
  {journal} {\bibinfo  {journal} {Science}\ }\textbf {\bibinfo {volume}
  {344}},\ \bibinfo {pages} {283} (\bibinfo {year} {2014})}\BibitemShut
  {NoStop}%
\bibitem [{\citenamefont {Watson}\ \emph {et~al.}(2018)\citenamefont {Watson},
  \citenamefont {Gibbs}, \citenamefont {Mackenzie}, \citenamefont {Hicks},\
  and\ \citenamefont {Moler}}]{watson2018micron}%
  \BibitemOpen
  \bibfield  {author} {\bibinfo {author} {\bibfnamefont {C.~A.}\ \bibnamefont
  {Watson}}, \bibinfo {author} {\bibfnamefont {A.~S.}\ \bibnamefont {Gibbs}},
  \bibinfo {author} {\bibfnamefont {A.~P.}\ \bibnamefont {Mackenzie}}, \bibinfo
  {author} {\bibfnamefont {C.~W.}\ \bibnamefont {Hicks}},\ and\ \bibinfo
  {author} {\bibfnamefont {K.~A.}\ \bibnamefont {Moler}},\ }\bibfield  {title}
  {\bibinfo {title} {Micron-scale measurements of low anisotropic strain
  response of local ${T}_{c}$ in {${\mathrm{Sr}}_{2}{\mathrm{RuO}}_{4}$}},\
  }\href {https://doi.org/10.1103/PhysRevB.98.094521} {\bibfield  {journal}
  {\bibinfo  {journal} {Phys. Rev. B}\ }\textbf {\bibinfo {volume} {98}},\
  \bibinfo {pages} {094521} (\bibinfo {year} {2018})}\BibitemShut {NoStop}%
\bibitem [{\citenamefont {Jang}\ \emph {et~al.}(2011)\citenamefont {Jang},
  \citenamefont {Ferguson}, \citenamefont {Vakaryuk}, \citenamefont {Budakian},
  \citenamefont {Chung}, \citenamefont {Goldbart},\ and\ \citenamefont
  {Maeno}}]{Jang2011}%
  \BibitemOpen
  \bibfield  {author} {\bibinfo {author} {\bibfnamefont {J.}~\bibnamefont
  {Jang}}, \bibinfo {author} {\bibfnamefont {D.~G.}\ \bibnamefont {Ferguson}},
  \bibinfo {author} {\bibfnamefont {V.}~\bibnamefont {Vakaryuk}}, \bibinfo
  {author} {\bibfnamefont {R.}~\bibnamefont {Budakian}}, \bibinfo {author}
  {\bibfnamefont {S.~B.}\ \bibnamefont {Chung}}, \bibinfo {author}
  {\bibfnamefont {P.~M.}\ \bibnamefont {Goldbart}},\ and\ \bibinfo {author}
  {\bibfnamefont {Y.}~\bibnamefont {Maeno}},\ }\bibfield  {title} {\bibinfo
  {title} {Observation of half-height magnetization steps in
  {${\mathrm{Sr}}_{2}{\mathrm{RuO}}_{4}$}},\ }\href
  {https://doi.org/10.1126/science.1193839} {\bibfield  {journal} {\bibinfo
  {journal} {Science}\ }\textbf {\bibinfo {volume} {331}},\ \bibinfo {pages}
  {186} (\bibinfo {year} {2011})}\BibitemShut {NoStop}%
\bibitem [{\citenamefont {Grinenko}\ \emph {et~al.}(2021)\citenamefont
  {Grinenko}, \citenamefont {Ghosh}, \citenamefont {Sarkar}, \citenamefont
  {Orain}, \citenamefont {Nikitin}, \citenamefont {Elender}, \citenamefont
  {Das}, \citenamefont {Guguchia}, \citenamefont {Br{\"u}ckner}, \citenamefont
  {Barber}, \citenamefont {Park}, \citenamefont {Kikugawa}, \citenamefont
  {Sokolov}, \citenamefont {Bobowski}, \citenamefont {Miyoshi}, \citenamefont
  {Maeno}, \citenamefont {Mackenzie}, \citenamefont {Luetkens}, \citenamefont
  {Hicks},\ and\ \citenamefont {Klauss}}]{GrinenkoNatPhys2021}%
  \BibitemOpen
  \bibfield  {author} {\bibinfo {author} {\bibfnamefont {V.}~\bibnamefont
  {Grinenko}}, \bibinfo {author} {\bibfnamefont {S.}~\bibnamefont {Ghosh}},
  \bibinfo {author} {\bibfnamefont {R.}~\bibnamefont {Sarkar}}, \bibinfo
  {author} {\bibfnamefont {J.-C.}\ \bibnamefont {Orain}}, \bibinfo {author}
  {\bibfnamefont {A.}~\bibnamefont {Nikitin}}, \bibinfo {author} {\bibfnamefont
  {M.}~\bibnamefont {Elender}}, \bibinfo {author} {\bibfnamefont
  {D.}~\bibnamefont {Das}}, \bibinfo {author} {\bibfnamefont {Z.}~\bibnamefont
  {Guguchia}}, \bibinfo {author} {\bibfnamefont {F.}~\bibnamefont
  {Br{\"u}ckner}}, \bibinfo {author} {\bibfnamefont {M.~E.}\ \bibnamefont
  {Barber}}, \bibinfo {author} {\bibfnamefont {J.}~\bibnamefont {Park}},
  \bibinfo {author} {\bibfnamefont {N.}~\bibnamefont {Kikugawa}}, \bibinfo
  {author} {\bibfnamefont {D.~A.}\ \bibnamefont {Sokolov}}, \bibinfo {author}
  {\bibfnamefont {J.~S.}\ \bibnamefont {Bobowski}}, \bibinfo {author}
  {\bibfnamefont {T.}~\bibnamefont {Miyoshi}}, \bibinfo {author} {\bibfnamefont
  {Y.}~\bibnamefont {Maeno}}, \bibinfo {author} {\bibfnamefont {A.~P.}\
  \bibnamefont {Mackenzie}}, \bibinfo {author} {\bibfnamefont {H.}~\bibnamefont
  {Luetkens}}, \bibinfo {author} {\bibfnamefont {C.~W.}\ \bibnamefont
  {Hicks}},\ and\ \bibinfo {author} {\bibfnamefont {H.-H.}\ \bibnamefont
  {Klauss}},\ }\bibfield  {title} {\bibinfo {title} {Split superconducting and
  time-reversal symmetry-breaking transitions in
  {${\mathrm{Sr}}_{2}{\mathrm{RuO}}_{4}$} under stress},\ }\href
  {https://doi.org/10.1038/s41567-021-01182-7} {\bibfield  {journal} {\bibinfo
  {journal} {Nature Physics}\ }\textbf {\bibinfo {volume} {17}},\ \bibinfo
  {pages} {748} (\bibinfo {year} {2021})}\BibitemShut {NoStop}%
\bibitem [{\citenamefont {Li}\ \emph {et~al.}(2021)\citenamefont {Li},
  \citenamefont {Kikugawa}, \citenamefont {Sokolov}, \citenamefont
  {Jerzembeck}, \citenamefont {Gibbs}, \citenamefont {Maeno}, \citenamefont
  {Hicks}, \citenamefont {Schmalian}, \citenamefont {Nicklas},\ and\
  \citenamefont {Mackenzie}}]{LiPNAS2021}%
  \BibitemOpen
  \bibfield  {author} {\bibinfo {author} {\bibfnamefont {Y.-S.}\ \bibnamefont
  {Li}}, \bibinfo {author} {\bibfnamefont {N.}~\bibnamefont {Kikugawa}},
  \bibinfo {author} {\bibfnamefont {D.~A.}\ \bibnamefont {Sokolov}}, \bibinfo
  {author} {\bibfnamefont {F.}~\bibnamefont {Jerzembeck}}, \bibinfo {author}
  {\bibfnamefont {A.~S.}\ \bibnamefont {Gibbs}}, \bibinfo {author}
  {\bibfnamefont {Y.}~\bibnamefont {Maeno}}, \bibinfo {author} {\bibfnamefont
  {C.~W.}\ \bibnamefont {Hicks}}, \bibinfo {author} {\bibfnamefont
  {J.}~\bibnamefont {Schmalian}}, \bibinfo {author} {\bibfnamefont
  {M.}~\bibnamefont {Nicklas}},\ and\ \bibinfo {author} {\bibfnamefont {A.~P.}\
  \bibnamefont {Mackenzie}},\ }\bibfield  {title} {\bibinfo {title}
  {High-sensitivity heat-capacity measurements on
  {${\mathrm{Sr}}_{2}{\mathrm{RuO}}_{4}$} under uniaxial pressure},\ }\href
  {https://www.pnas.org/content/118/10/e2020492118} {\bibfield  {journal}
  {\bibinfo  {journal} {Proceedings of the National Academy of Sciences}\
  }\textbf {\bibinfo {volume} {118}} (\bibinfo {year} {2021})}\BibitemShut
  {NoStop}%
  \bibitem [{\citenamefont {Tinkham}(1996)}]{Tinkham}%
  \BibitemOpen
  \bibfield  {author} {\bibinfo {author} {\bibfnamefont {M.}~\bibnamefont
  		{Tinkham}},\ }\href@noop {} {\emph {\bibinfo {title} {Introduction to
  			Superconductivity}}}\ (\bibinfo  {publisher} {McGraw-Hill},\ \bibinfo
  {address} {New York},\ \bibinfo {year} {1996})\BibitemShut {NoStop}%
  \bibitem [{\citenamefont {Schrieffer}(1964)}]{Schrieffer}%
  \BibitemOpen
  \bibfield  {author} {\bibinfo {author} {\bibfnamefont {J.}~\bibnamefont
  		{Schrieffer}},\ }\href@noop {} {\emph {\bibinfo {title} {Theory of
  			Superconductivity}}}\ (\bibinfo  {publisher} {Perseus Books},\ \bibinfo
  {year} {1964})\BibitemShut {NoStop}%
  \bibitem [{\citenamefont {Sigrist}(2002)}]{SigristPTP2002}%
  \BibitemOpen
  \bibfield  {author} {\bibinfo {author} {\bibfnamefont {M.}~\bibnamefont
  		{Sigrist}},\ }\bibfield  {title} {\bibinfo {title} {{Ehrenfest Relations for
  			Ultrasound Absorption in $\mathrm{Sr}_{2}\mathrm{RuO}_{4}$}},\ }\href@noop {}
  {\bibfield  {journal} {\bibinfo  {journal} {Progress of Theoretical Physics}\
  	}\textbf {\bibinfo {volume} {107}},\ \bibinfo {pages} {917} (\bibinfo {year}
  	{2002})}\BibitemShut {NoStop}%
\end{thebibliography}%



\section*{Acknowledgments}
B. J. R. and S. G. acknowledge support for building the experiment, collecting and analyzing the data, and writing the manuscript from the Office of Basic Energy Sciences of the United States Department of Energy under award no. DE-SC0020143.  B.J.R. and S.G. acknowledge support from the Cornell Center for Materials Research with funding from the Materials Research Science and Engineering Centers program of the National Science Foundation (cooperative agreement no. DMR-1719875). T. G. K. acknowledge support from the National Science Foundation under grant no. PHY-2110250.  N. K. acknowledges support from Japan Society for the Promotion of Science (JSPS) KAKENHI (Nos. JP17H06136, JP18K04715, and 21H01033)) and Japan Science and Technology Agency Mirai Program (JPMJMI18A3) in Japan.

\newpage

% This version of CVPR template is provided by Ming-Ming Cheng.
% Please leave an issue if you found a bug:
% https://github.com/MCG-NKU/CVPR_Template.

\documentclass[review]{cvpr}
%\documentclass[final]{cvpr}

\usepackage{times}
\usepackage{epsfig}
\usepackage{graphicx}
\usepackage{amsmath}
\usepackage{amssymb}
\usepackage{xcolor}
\usepackage{multirow}
\usepackage{multicol}
\usepackage{booktabs}
\usepackage{subcaption}
\usepackage{float}
\newcommand{\iie}{\emph{i.e.}}

% Include other packages here, before hyperref.

% If you comment hyperref and then uncomment it, you should delete
% egpaper.aux before re-running latex.  (Or just hit 'q' on the first latex
% run, let it finish, and you should be clear).
\usepackage[pagebackref=true,breaklinks=true,colorlinks,bookmarks=false]{hyperref}


\def\cvprPaperID{1430} % *** Enter the CVPR Paper ID here
\def\confYear{CVPR 2022}
%\setcounter{page}{4321} % For final version only


\begin{document}

%%%%%%%%% TITLE
\title{Supplementary for Decoupled One Stage Action Detection Network}

\author{First Author\\
Institution1\\
Institution1 address\\
{\tt\small firstauthor@i1.org}
% For a paper whose authors are all at the same institution,
% omit the following lines up until the closing ``}''.
% Additional authors and addresses can be added with ``\and'',
% just like the second author.
% To save space, use either the email address or home page, not both
\and
Second Author\\
Institution2\\
First line of institution2 address\\
{\tt\small secondauthor@i2.org}
}

\maketitle

%%%%%%%%% BODY TEXT

\section{Additional experiments}

\subsection{Detection performance (action agnostic)}
As a one-stage method, we use the person bounding boxes generated by our network instead of an off-the-shelf detector. We now investigate in detail at the detection performance. We report the performance in Table~\ref{detection} with a standard 0.5 IOU threshold. The off-the-shelf detector is Faster R-CNN framework with ResNeXt-101-FPN backbone from maskrcnn-benchmark, which is widely applied in two-stage methods [\textcolor{green}{48}, \textcolor{green}{38}, \textcolor{green}{26}]. The model is first pre-trained on ImageNet, then fine-tuned on MSCOCO dataset, and finally fine-tuned on AVA dataset for higher person detection precision. We can see that our person detection result is still lower than the specialized detector, which is the crucial reason that the performance of one-stage methods cannot surpass the state-of-the-art two-stage methods.

\subsection{Backbone modification}
We are the first to use transformer-based backbone, Swin-B, in this task. A tough nut is how to maintain a larger spatial resolution of the feature map due to the overlarge spatial downsampling rate in the original version. The downsampling of Swin-B is mainly from the patch merging layer followed by a swin transformer block. We propose two schemes to modify the Swin-B: (i) removing the last patch merging layer and its following swin transformer block; (ii) just removing the last patch merging layer and modifying the dimension of the weights of the last swin transformer block. Note that the last swin transformer block in scheme (ii) cannot be initialized from a pre-trained model and can only be trained from scratch. Their results are presented in Table \ref{backbone}. Scheme (i) is slightly higher than scheme (ii) and contains fewer parameters. Thus, we adopt scheme (i) in our method.

\section{Qualitative results}
We randomly visualize some cases of our model in Figure \ref{qualitative}. Our model is able to exploit the person-context relationships to recognize interaction categories such as ``watch sb" and ``listen to sb", which are inherently hard if only focus on the actor, as shown in the first row of Figure \ref{qualitative}. There are two failure detection cases in the second row, which shows that our detection is hard to handle crowded and dark scenes. 


\begin{table} [t]
\begin{center}
\small
%\resizebox{!}{1.1cm}{
\begin{tabular}{c|c}
\toprule
Method & mAP (IOU@0.5)  \\
\midrule
Off-the-shelf detector & 93.5 \\
Ours & 89.9 \\
\bottomrule
\end{tabular}
\end{center}
% \vspace{-5mm}
\caption{We perform classification-agnostic evaluation to evaluate the performance of our person detection and compare it with the off-the-shelf detector.}
% \vspace{-1mm}
\label{detection}
\end{table}

\begin{table} [t]
\begin{center}
\small
%\resizebox{!}{1.1cm}{
\begin{tabular}{c|c}
\toprule
Scheme & mAP (IOU@0.5)  \\
\midrule
 i & 28.8 \\
 ii & 28.5 \\
\bottomrule
\end{tabular}
\end{center}
\vspace{-5mm}
\caption{Comparison of backbone modification schemes.}
\vspace{-3mm}
\label{backbone}
\end{table}

\begin{figure}[t]
%\includegraphics[width=1.0\linewidth]{LaTeX/supplement.png}
\includegraphics[width=1.0\linewidth]{latex/qualitative.png}
\centering
\caption{Per category results for the proposed network and the baseline model on the validation set of AVA dataset.}
\label{qualitative}
%\vspace{-4mm}
\end{figure}

\begin{figure*}[!t]
%\includegraphics[width=1.0\linewidth]{LaTeX/supplement.png}
\includegraphics[width=0.95\linewidth]{latex/figure.jpeg}
\centering
\caption{Per category results for the proposed network and the baseline model on the validation set of AVA dataset.}
\label{visualization}
\end{figure*}

\section{Per category analysis}
The per category results for our method and the baseline (the full model without TransPC) are shown in Figure \ref{visualization}. Our method improves the baseline performance in about 50 out of 60 classes. We can see that the categories getting the largest performance boost are from interaction categories, e.g., ``hand clap", ``work on a computer", ``smoke", and ``listen to sb", which require more attention on the supporting actors and context.





% {\small
% \bibliographystyle{ieee_fullname}
% %\bibliography{egbib}
% }

\end{document}


\end{document}


