\section*{Supplemental Material}
	
	\subsection*{Viscosity from RUS Measurements}
	
	RUS measures the mechanical resonances of a solid, which are determined by the elastic constants and the density of the material along with its dimensions. Each resonance $\omega_0$ is a superposition of the various irreducible strains, and therefore is a function of the independent elastic moduli $c_j$. In a tetragonal system like \sro, there are six such moduli ($j=1,2,...,6$) such that
	\begin{equation}
	\omega_0^2=\mathfrak{F}(c_j,\rho,l_k),
	\label{eqn:freqdef}
	\end{equation}
	where $\rho$ is the density and $l_k$ are the dimensions of the sample. Sound attenuation in the solid leads to these frequencies having a finite linewidth $\Gamma$, giving them a characteristic Lorentzian shape (as discussed in the main). Within linear response, sound attenuation is related to the strain rate through the viscosity tensor \cite{MorenoPRB1996}, which has the same symmetries as the elastic moduli tensor. To relate the experimentally measured linewidths to the irreducible viscosities, we replace $\omega_0\rightarrow\omega_0+i\Gamma$ and $c_j\rightarrow c_j+i\omega_0\eta_j$ in \autoref{eqn:freqdef},
	\begin{equation}
	\begin{aligned}
	&(\omega_0+i\Gamma)^2=\mathfrak{F}(c_j+i\omega_0\eta_j,\rho,l_k)\\
	\implies &\omega_0^2 + 2i\omega_0\Gamma \approx \mathfrak{F}(c_j,\rho,l_k)+\sum_{j}\frac{\partial \mathfrak{F}}{\partial c_j}\cdot i\omega_0\eta_j\\
	\implies&\Gamma =\frac{1}{2}\sum_{j}\frac{\partial \mathfrak{F}}{\partial c_j}\cdot \eta_j =\frac{1}{2}\omega_0^2\sum_{j}\alpha_j \frac{\eta_j}{c_j},
	\end{aligned}
	\label{eqn:gammadef}
	\end{equation}
	where $\alpha_j=\partial(\ln \omega_0^2)/\partial(\ln c_j)$ and $\sum_{j}\alpha_j=1$ \cite{RamshawPNAS}. The fit procedure outlined in \citet{RamshawPNAS} gives us the $\alpha_j$ coefficients for all our experimentally measured resonances. Knowing these coefficients, we can calculate the six independent viscosities as a function of temperature by measuring the temperature evolution of sufficiently many resonance linewidths (typically 2-3 times the number of independent viscosities). Note that \autoref{eqn:gammadef} is true in the weak attenuation limit ($\Gamma\ll\omega_0$), which is easily satisfied in our experiments ($\Gamma/\omega_0\sim10^{-4}$ for all our measured resonances, see \autoref{fig:gammavT}).
	
	
	%Velocity of a resonance can be calculated from its composition in terms of the independent elastic moduli, which we know from our fitting procedure (see \cite{RamshawPNAS} for details). For example, if a resonance is composed of 40$\%$ $c_{33}$ and 60$\%$ $c_{66}$, its velocity would be $v=(c_{33}^{0.4}c_{66}^{0.6}/\rho)^{\frac{1}{2}}=4373$ m/s, where $\rho$ is the density of \sro. The attenuation arises due to energy dissipation in the crystal when a sound wave propagates through it. Within linear response, the sound attenuation coefficient is defined as the ratio of the time average energy dissipation to twice the energy flux in the wave \cite{RodriguezPRL1985,MorenoPRB1996} and is given by
	
	%Here $\omega_0$ is the resonance frequency and $\eta$ the viscosity of the mode. Using this, we convert the linewidths of the resonances to viscosities, which allows us to extract the six independent viscosity components in \sro. Formally, viscosity is the imaginary part of the stress-stress correlation function of the electron fluid and hence is related to the energy dissipation, while elastic moduli are the real part of the same correlation function and hence is related to the sound velocity.
	
	\begin{figure*}
		\centering
		\includegraphics[width=.99\linewidth]{figSI.pdf}
		\caption{\textbf{RUS attenuation data.} (a) Temperature evolution of normalized resonance linewidth of 18 resonances of \sro through \Tc, with panels (a) and (b) each showing 9 resonances. These 18 resonances were used to calculate the six independent components of the viscosity tensor through \Tc.}
		\label{fig:gammavT}
	\end{figure*}
	
	\subsection*{Quasiparticle Scattering in the Superconducting State}
	
	The conventional method for describing ultrasound attenuation assumes that sound waves attenuate by scattering quasiparticles. In particular, we assume that the attenuation rate is proportional to the scattering rate induced by the sound wave. The effect of the superconducting transition on scattering was introduced by BCS~\cite{BCS1957} and is described pedagogically by Tinkham~\cite{Tinkham} and Schrieffer~\cite{Schrieffer}. We assume that the induced scattering can be described by an interaction Hamiltonian
	\begin{equation}
	\mathcal{H}_{\rm int}=\sum_{k,k^\prime,\sigma}M_{k,k^\prime}c^\dagger_{k,\sigma}c_{k^\prime,\sigma}
	\end{equation}
	where $M_{k,k^\prime}$ is symmetric under time reversal (TR). This is crucial: as an $s$-wave superconductor consists of time-reversed fermionic pairs, the symmetry of the interaction under TR has drastic effects on the scattering properties. In particular, an interaction Hamiltonian that is even under TR will result in destructive interference between Bogoliubov quasiparticles, and vice versa.
	
	The scattering rate can be computed by Fermi's golden rule. Following Tinkham~\cite{Tinkham}, we take $M_{k,k^\prime}=Me^{i\theta_{k,k^\prime}}$ to have a constant modulus. This is a drastic approximation that allows for arbitrarily wide-angle scattering. While it has little effect on the $s$-wave calculation, we will have to remedy this approximation for the $p$-wave gap function. We perform the calculations in 2D for each of the Fermi surfaces of ${\rm Sr}_2{\rm RuO}_4$. The band structure adds a Jacobian factor to the integrand, $J_i(\phi)$, where $\phi$ is the azimuthal angle in momentum space and $i\in\{\alpha,\beta,\gamma\}$ for the $\alpha$, $\beta$ and $\gamma$ bands, respectively. The general integral is of the form
	\begin{multline}
	\Gamma_s(\omega,T)=|M|^2\int_0^\infty dE~\int\frac{d\phi_1d\phi_2}{(2\pi)^2}J_i(\phi_1)J_i(\phi_2)n_s(E,\phi_1)n_s(E+\hbar\omega,\phi_2)\times\\\big(f(E)-f(E+\hbar\omega)\big)\big(F_{\gamma^\dagger\gamma}(E,\phi_1,\phi_2)+F_{\gamma\gamma}(E,\phi_1,\phi_2)\big)
	\label{eq:gammaS}
	\end{multline}
	where $n_s(E,\phi)$ is the angle-resolved density of states, obeying $\int \frac{d\phi}{2\pi}n_s(E,\phi)=N_s(E)$, $f(E)$ is the Fermi factor, and $F(E,\phi_1,\phi_2)$ is the coherence factor. The coherence factors arise due to interference between quasiparticles. They are derived by rewriting the interaction Hamiltonian in terms of Bogoliubov quasiparticles:
	\begin{equation}
	\mathcal{H}_{\rm int}=M\sum_{k,k^\prime}F_{\gamma^\dagger\gamma}(k,k^\prime)\gamma^\dagger_{k,\sigma}\gamma_{k^\prime,\sigma}+M\sum_{k,k^\prime}F_{\gamma\gamma}(k,k^\prime)\gamma_{k,\sigma}\gamma_{k^\prime,\sigma}+h.c.
	\end{equation}
	Linearizing the functions $F(k,k^\prime)$ about the Fermi surface and evaluating the energy delta function gives us the functions $F(E,\phi,\phi^\prime)$.
	
	The coherence factors depend on the structure of the gap function in spin and momentum space. We will not treat the general case here.  
	For the case of a $p_x+ip_y$ gap function, the coherence factors are both of the form
	\begin{equation}
	F_\pm(E,\phi,\phi^\prime)=1\pm\frac{\Delta_0^2}{E(E+\hbar\omega)}\cos(\phi-\phi^\prime).
	\end{equation}
	In Figure~3 we plot the results for the $(p_x+ip_y)\hat{z}$ gap function, for which $F_{\gamma^\dagger\gamma}=F_{\gamma\gamma}=F_+$.
	As we are considering a 2D scattering problem, the gap function has a constant modulus in momentum space, so the angle-resolved density of states is equal to the density of states:
	\begin{equation}
	n_s(E,\phi)=N_s(E)=\frac{1}{\sqrt{E^2-\Delta_0^2}}.
	\end{equation}
	Collecting terms, we find that the scattering rate is given by
	\begin{multline}
	\Gamma_{x+iy}(\omega,T)=2|M|^2\int_0^\infty dE~\int\frac{d\phi_1d\phi_2}{(2\pi)^2}J_i(\phi_1)J_i(\phi_2)\frac{1}{\sqrt{(E^2-\Delta_0^2)((E+\hbar\omega)^2-\Delta_0^2)}}\times\\\big(f(E)-f(E+\hbar\omega)\big)\bigg(1-\frac{\Delta_0^2}{E(E+\hbar\omega)}\cos(\phi_1-\phi_2)\bigg).
	\end{multline}
	This is essentially equivalent to the s-wave scattering problem except for the factor of $\cos(\phi_1-\phi_2)$. Importantly, one can immediately see that the integral over $J_i(\phi_1)J_i(\phi_2)\cos(\phi_1-\phi_2)$ will vanish due to the $\mathbb{C}_4$ symmetry of the band structure. This means that the coherence factors are effectively equal to 1, as if the Bogoliubov quasiparticles do not interfere with one another. It is this lack of interference that leads to the peak shown in panel (b) of Fig.~3. 
	
	Note that the vanishing (average) interference between quasiparticles depends on integrating the relative angle, $\phi_1-\phi_2$, over the full period of the cosine function. This involves wide-angle scattering events, where the particles being scattered sit on opposite sides of the Fermi surface. Such scattering events would require that the sound wave have a momentum ${\bf q}={\bf k_1}-{\bf k_2}$ that can be as large as $2k_F$. Experimentally, however, the sound waves have a frequency $q/2k_F\sim 10^{-7}$. We therefore propose a phenomenological method for confining the relative angle $|\phi_1-\phi_2|\lesssim\eta$ by including the Gaussian factor $G(\phi_1-\phi_2)$ in the integrand, where
	\begin{equation}
	G(x)=\frac{1}{\eta\sqrt{2\pi}}e^{-(x/\eta)^2/2}.
	\end{equation}
	As the relative angle on a circular Fermi surface is given by $\phi_1-\phi_2=\arcsin(q/2k_F)$, we convert the parameter $\eta$ to an equivalent sound frequency using $\omega_\eta=2v_sk_F\sin(\eta)$. For large $\omega_\eta$, we recover the non-interfering results. For $\omega_\eta$ on the order of the experimental frequency, however, $\cos(\phi_1-\phi_2)\approx 1$ and we recover the standard s-wave calculation with a pronounced dip across $T_c$. This corresponds to strong destructive interference between quasiparticles. These curves are compared in panel (b) of Fig.~3.
	
	The coherence factors for a $d_{x^2-y^2}$ gap function are of the form
	\begin{equation}
	F_\pm(E,\phi,\phi^\prime)=1\pm\frac{\Delta_0^2}{E(E+\hbar\omega)}|\cos(2\phi)||\cos(2\phi^\prime)|
	\end{equation}
	and the angle-resolved density of states is
	$n_s(E,\phi)=|E|/\sqrt{E^2-\Delta_0^2\cos^2(2\phi)}$. Inserting these directly into \autoref{eq:gammaS}, one obtains
	\begin{multline}
	\Gamma_{x^2-y^2}(\omega,T)=2|M|^2\int_0^\infty dE~\int\frac{d\phi_1d\phi_2}{(2\pi)^2}J_i(\phi_1)J_i(\phi_2)\big(f(E)-f(E+\hbar\omega)\big)\times\\\frac{1}{\sqrt{(E^2-\Delta_0^2\cos^2(2\phi_1))((E+\hbar\omega)^2-\Delta_0^2\cos^2(2\phi_2))}}\bigg(1-\frac{\Delta_0^2}{E(E+\hbar\omega)}|\cos(2\phi_1)||\cos(2\phi_2)|\bigg).
	\label{eq:dWaveEq}
	\end{multline}
	We do not include the phenomenological Gaussian factor for the results in panel (a) of Fig.~3 because there is no erroneous peak, so we do not expect any qualitative change in the results. In principle, however, the structure of the $d$-wave gap in momentum space means that a quantitative calculation of the scattering rate should remove wide-angle scattering events. All theory curves in panels (a), (b) and (c) of Fig.~3 were computed for the $\beta$ band. \autoref{fig:dWvFig} shows the normalized scattering rate for the $\alpha$, $\beta$, and $\gamma$ bands.
	\begin{figure}
		\centering
		\includegraphics[width=.65\linewidth]{dWaveGap_bands.pdf}
		\caption{{\bf Scattering rate for ${\rm Sr}_2{\rm RuO}_4$ bands.} Comparison of the scattering rate with a $d_{x^2-y^2}$ gap function for each of the three ${\rm Sr}_2{\rm RuO}_4$ Fermi surfaces. The electron-like $\gamma$ and $\beta$ bands show nearly identical behavior while the hole-like $\alpha$ band differs significantly at low temperatures. We find that these discrepancies do not change the qualitative behavior across $T_c$. Note that the $\beta$ curve is reproduced in panel (a) of Fig.~3 in the main text.}
		\label{fig:dWvFig}
	\end{figure}
	The $\alpha$ band is hole-like, and its scattering rate differs qualitatively from that of the electron-like bands. We note that this effect appears most pronounced at low temperatures and that it does not change the behavior across $T_c$. The $s$-wave curve, also plotted in panel (a) of Fig.~3, has coherence factors $1\pm\frac{\Delta_0^2}{E(E+\hbar\omega)}$ and a density of states $n_s(E)=(E^2-\Delta_0^2)^{-1/2}$. The $s$-wave singlet problem is covered extensively in Refs.~\cite{BCS1957,Tinkham,Schrieffer} so we do not review it here.
	
	\subsection*{Cooper Pair-breaking below \Tc}
	
	In \autoref{eq:gammaS}, we neglected a set of quadratic Bogoliubov terms that are referred to as pair-breaking terms. These terms are neglected because the domain of the integral formally vanishes when the driving frequency $\omega<2\Delta(T)$ where $\Delta(T)$ is the temperature-dependent s-wave gap. For gap functions with nodes, however, the frequency domain does not vanish explicitly. Crucially, the pair breaking coherence factors have the opposite sign of the dominant coherence factors discussed above. This means that an interaction that leads to destructive interference between quasiparticles (and thus a sharp drop in the scattering rate across $T_c$) will exhibit constructive interference in its pair-breaking terms. Thus, pair breaking terms will induce a peak where there is otherwise none.
	
	Pair breaking terms can be identified immediately by the energy delta function. For example, in going from the general statement of Fermi's golden rule to \autoref{eq:gammaS}, we evaluated the energy delta function $\delta(E-E^\prime-\hbar\omega)$ assuming that $E,E^\prime>0$. This is the natural choice in the $\omega\to 0$ limit. For $\omega>0$, however, there are also solutions to the delta function where $E<0$ ($E^\prime>0$). These contributions are what we will explicitly include now. Note that even with nodes in the gap function, the domain of this integral formally vanishes as $\omega\to 0$. 
	
	For the d-wave gap, normal contributions are given by \autoref{eq:dWaveEq}. Pair breaking contributions differ in the domain of the $E$ integral, the sign of the coherence factor, and in the product of Fermi factors:
	\begin{multline}
	\Gamma^{PB}_{x^2-y^2}(\omega,T)=2|M|^2\int_0^\omega dE~\int\frac{d\phi_1d\phi_2}{(2\pi)^2}J_i(\phi_1)J_i(\phi_2)\big(1-f(E)-f(E+\hbar\omega)\big)\times\\\frac{1}{\sqrt{(E^2-\Delta_0^2\cos^2(2\phi_1))((E+\hbar\omega)^2-\Delta_0^2\cos^2(2\phi_2))}}\bigg(1+\frac{\Delta_0^2}{E(E+\hbar\omega)}|\cos(2\phi_1)||\cos(2\phi_2)|\bigg).
	\end{multline}
	We sum these two terms, normalized by the scattering rate in the normal state, for two values of $\omega$ in panel (c) of Fig.~3. An interpretation of these results is covered in the main text.
	
	\begin{figure*}
		\centering
		\includegraphics[width=.7\linewidth]{fig_SI_compare.pdf}
		\caption{\textbf{Sound attenuation from order parameter modes.} Normalized $(\eta_{11}+\eta_{12})/2$ in \sro fit to two different models of increased sound attenuation below \Tc. The green curve is a fit to \autoref{eqn:att1}, which models sound attenuation due to OP modes. The red curve is a fit to \autoref{eqn:att2}, which models the sound attenuation arising from domain wall motion. Near \Tc, the red curve clearly fits the experimental data better than the green one.}
		\label{fig:compare}
	\end{figure*} 
	
	\subsection*{Attenuation Due to Order Parameter Relaxation}
	
	The formation of the SC order parameter below \Tc can lead to relaxational dynamics as the OP interacts with the strain. Within a Landau theory, the relaxation timescale diverges as $\left|T/T_{\rm c}-1\right|^{-1}$ close to \Tc. Unlike the resonant sound absorption arising from domains, OP relaxation can cause non-resonant absorption of ultrasound and lead to a broad peak in sound attenuation below \Tc \cite{SigristPTP2002}. We fit our measured $(\eta_{11}+\eta_{12})/2$ to the attenuation expression derived by \citet{SigristPTP2002},
	\begin{equation}
	\alpha(\omega,T)\propto \frac{\omega^2\tau}{1+\omega^2\tau^2} \sim \frac{\omega^2\tau_0/\left|T/T_{\rm c}-1\right|}{1+\omega^2\tau_0^2/\left|T/T_{\rm c}-1\right|^2}
	\label{eqn:att1}
	\end{equation}
	as shown in \autoref{fig:compare}. However we find that this expression does not capture the sharp increase in attenuation below \Tc, which the expression for attenuation from domain wall motion given in \citet{SigristRMP1991} does, 
	\begin{equation}
	\alpha\left(\omega,T\right) \propto \frac{\omega^2}{\omega^2+\omega_1^2\left|T/T_{\rm c}-1\right|^3}\left|T/T_{\rm c}-1\right|^2.
	\label{eqn:att2}
	\end{equation}
	In fact, the sharp peak-like behavior of attenuation right below \Tc, which is already present in the raw data (for example, 2495 kHz and 2573 kHz in \autoref{fig:gammavT}), points strongly to a resonant absorption mechanism compared to a non-resonant one.
	