% !TEX root = ./main.tex
\section{The cosmic ray intensity}
\label{app:intensity}

In this appendix, we introduce the various definitions used to describe CR density. 

The number of particles in a volume element $d^3 \vb{r}$ about $\vb r$ and in the momentum interval $d^3 \vb p$ about $\vb p$ is given by
%
\begin{equation}
d\rho = \phi(\vb r, \vb p, t) \, d^3 \vb r \, d^3 \vb p
\end{equation}
%
with $\phi$ the distribution function.

Expanding $d^3 \vb p$ in spherical coordinates gives:
%
\begin{equation}
d\rho  = \phi(\vb r, p, t) \, d^3 \vb r \, p^2 dp \, d\Omega
\end{equation}

Typically we are not able to measure $\phi$ but only averages over momentum space, thereby we conveniently introduce the \emph{phase-space distribution function} as:
%
\begin{equation}
f(\vb r, p, t) = \frac{1}{4\pi} \int_\Omega \phi(\vb r, p, t) d\Omega
\end{equation}

Correspondingly, the number of particles in $d^3 \vb r$ and in $dp$ (independent of direction of $\vb p$) is:
%
\begin{equation}
dn = \int_\Omega \! d\Omega \, F(\vb r, p, t) \, d^3 \vb r \, p^2 dp = 4 \pi p^2 f(\vb r, p, t) \, d^3 \vb r \, dp
\end{equation}
 
Consequently, when the spectral index of the CR phase distribution in momentum is represented by $\gamma$, the corresponding slope, when expressed in terms of energy, becomes $\gamma - 2$:
%
\begin{equation}
f \propto p^{-\gamma} \, \rightarrow \, n \propto E^{2-\gamma}
\end{equation}

Experimental measurements typically present their results in terms of the CR differential intensity, denoted as $I$. This quantity represents the number of particles collected per unit energy, unit area, unit time, and unit steradians.

Furthermore, it is convenient to describe the spectra of the different CR components as a function of the kinetic energy per nucleon, denoted as $T$. This choice is advantageous because the kinetic energy per nucleon is approximately conserved when a nucleus fragments due to interactions with the interstellar gas.

In terms of the phase-space distribution, the intensity of a specific nuclear species $\alpha$, can be expressed as 
%
\begin{equation}
I_\alpha (T) dT = c p^2 f_\alpha(p) \beta (p) dp 
\end{equation}
%
which can be further simplified as 
%
\begin{equation}
I_\alpha(T) = A p^2 f_\alpha(p)
\label{eq:f2I}
\end{equation}
%
where $A$ is the nuclear mass.
