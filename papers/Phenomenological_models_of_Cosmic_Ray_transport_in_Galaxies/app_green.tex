% !TEX root = ./main.tex
\section{Green's functions of the transport equations}
\label{sec:appgreen}

\subsection{Green's function of the pure diffusive equation}

To construct the Green's function, we start by taking the Fourier transform with respect to the spatial variable $\vb r$, using the property that $\mathcal F[\delta^{(3)}(\vb r - \vb r_\star)] = {\rm e}^{-i \vb k \cdot \vb r_\star}$ :
%
\begin{equation}
\frac{\partial}{\partial t} \tilde{\mathcal G}(\vb k, t \leftarrow \vb r_\star, t_\star) 
+ D k^2 \tilde{\mathcal G}(\vb k, t \leftarrow \vb r_\star, t_\star) 
= {\rm e}^{-i \vb k \cdot \vb r_\star} \delta(t - t_\star)
\end{equation}
%
subject to the initial condition $\tilde{\mathcal G}(\vb k, 0 \leftarrow \vb r_\star, t_\star) = 0$.

After multiplying both sides by ${\rm e}^{D k^2 t}$, we obtain
%
\begin{equation}
\frac{\partial}{\partial t} \left[ {\rm e}^{D k^2 t} \tilde{\mathcal G}(\vb k, t \leftarrow \vb r_\star, t_\star) \right]
= {\rm e}^{-i \vb k \cdot \vb r_\star + D k^2 t} \, \delta(t - t_\star)
\end{equation}

This equation can be easily solved, leading to the Fourier transform of the Green's function:
%
\begin{equation}
\tilde{\mathcal G}(\vb k, t) = 
{\rm e}^{-i \vb k \cdot \vb r_\star - D k^2 t} 
\int_0^t {\rm e}^{D k^2 t^\prime} \delta(t^\prime-t_\star) dt^\prime
= \begin{cases}
0 & t < t_\star \\
{\rm e}^{- i \vb k \cdot \vb r_\star - D k^2 (t-t_\star)} & t > t_\star
\end{cases}
\end{equation}

The upper limit $t$ of this integral expresses \emph{causality}: the solution at time $t$ depends only on causes lying in its past, i.e., $t_\star < t$.

Finally, by taking the inverse Fourier transform with respect to $\vb k$, we obtain the Green's function we are looking for:
%
\begin{equation}
\mathcal G (\vb r, t \leftarrow \vb r_\star, t_\star) 
= \frac{\Theta(t-t_\star)}{(2\pi)^3} 
\int {\rm e}^{i \vb k \cdot (\vb r - \vb r_\star)} {\rm e}^{-Dk^2 (t-t_\star)} d^3 \vb k
\end{equation}

Recognizing the integral as the (inverse) Fourier transform of a Gaussian, we find:
%
\begin{equation} 
\mathcal G_{\rm free} (\vb r, t \leftarrow \vb r_\star, t_\star) 
= \frac{\Theta(\tau)}{(4\pi D \tau)^{3/2}} 
{\rm e}^{-\frac{d^2}{4 D \tau}}
\end{equation}
%
where $\tau = t-t_\star$ and $\vb d = \vb r - \vb r_\star$.

Thus, an initial Gaussian distribution retains its Gaussian form, with its squared width spreading linearly with time. This linear growth of variance is characteristic of \emph{diffusing} probabilistic processes.

One notable property of this solution is that $\mathcal G > 0$ everywhere for any finite $t > 0$, regardless of how small $t$ is. However, this violates Special Relativity, and it is often resolved by replacing the theta argument with $\Theta(c \Delta t - d)$.

It's worth mentioning that the halo size parameter $H$ does not appear in the Green's function, as we derived the free-space Green's function without imposing any boundary condition in the $z$ direction.

To enforce the desired boundary conditions, we introduce a set of image charges with coordinates~\cite{Cowsik1979apj,Baltz1998prd,Mertsch2011jcap}:
%
\begin{equation}
\vb r_{\star,n}^\prime = 
\left(\begin{array}{c}
x_\star\\
y_\star\\
2 H n + (-1)^n z_\star 
\end{array}\right)
\end{equation}
%
as a consequence, the Green's function associated with these image charges becomes
%
\begin{equation}
\mathcal G_{\rm H} (\vb r, t \leftarrow \vb r_\star, t_\star) = \sum_{n=-\infty}^{\infty} (-1)^n \mathcal G_{\rm free} (\vb r, t \leftarrow \vb r_{\star,n}^\prime, t_\star)  
\end{equation}

It is easy to check that $\mathcal G_{\rm H}(x, y, {z = \pm {\textrm H}}, t\leftarrow \vb r_\star, t_\star) = 0$, satisfying the desired boundary conditions.

\subsection{Green's function of the diffusion-losses equation}

The equation of interest for leptons, assuming steady-state, is given by:
%
\begin{equation}
%\cancelto{0}{\frac{\partial I_e}{\partial t}} 
- D(E) \nabla^2 n_e(E)
- \frac{\partial}{\partial E} \left[ b(E) n_e \right] 
= Q(\vb r, E, t)
\end{equation}
%
where $b(E) = dE/dt$ represents the energy losses.

It is convenient to introduce a new variable:
%
\begin{equation}
\tilde t = 4 \int_E^\infty \frac{D(E^\prime)}{b(E^\prime)} dE^\prime
\end{equation}
%
which allows us to rewrite the derivative as:
%
\begin{equation}
\frac{d}{d\tilde t} = - \frac{b(E)}{4 D(E)} \frac{d}{dE}
\end{equation}

Using this new variable, we obtain the equation:
%
\begin{equation}
- D(E) \nabla^2 n_e(E)
+ 4 \frac{D(E)}{b(E)}\frac{\partial}{\partial \tilde t} \left[ b(E) n_e(E) \right] 
= Q(\vb r, E, t)
\end{equation} 

By rearranging the terms, we arrive at the diffusion equation for the new variable $\tilde{n} = b(E) n(E)$:
%
\begin{equation}
\frac{\partial}{\partial \tilde t} \left[ \tilde n_e(E) \right] 
- \frac{1}{4} \nabla^2 \left[ \tilde n_e(E) \right]
= \tilde Q(\vb r, E, t)
\end{equation} 

Next, we consider the Green's function in the new variables, introducing $\lambda^2 = \tilde{t} - \tilde{t}_\star$, which is given by:
%
\begin{equation}
\tilde{\mathcal G} (\vb r, \tilde t \leftarrow \vb r_\star, \tilde t_\star) 
= \frac{\Theta(\lambda^2)}{(\pi \lambda^2)^{3/2}} 
{\rm e}^{-\frac{d^2}{\lambda^2}}
\end{equation}

The Green's function for the time-dependent solution can be expressed in terms of the steady-state solution as:
%
\begin{equation}
{\mathcal G} (\vb r, t, E \leftarrow \vb r_\star, t_\star, E_\star) 
= \delta(\Delta t - \tau) \frac{\tilde{\mathcal G} (\vb r, \tilde t \leftarrow \vb r_\star, \tilde t_\star)}{|b(E)|} 
= \frac{1}{|b(E)|}\frac{\delta(\Delta t - \tau)}{(\pi \lambda^2)^{3/2}} {\rm e}^{-\frac{d^2}{\lambda^2}}
\end{equation}
%
where $\lambda^2(E, E_\star) = 4 \int_{E}^{E_\star} dE^\prime \frac{D(E^\prime)}{b(E^\prime)}$ represents the propagation scale (also known as the Syrovatskii variable), and $\tau(E, E_\star) = \int_E^{E_\star} \frac{dE^\prime}{b(E^\prime)}$ is the loss time, which corresponds to the average time during which the energy of a particle decreases from $E_\star$ to $E$ due to losses.

Therefore, the particles we observe with energy $E$ have been injected with energy $\tilde{E}$ at a time $\Delta t$ that satisfies $\tau(E, \tilde{E}) = \Delta t$.
 
This condition sets a maximum energy $E_{\rm max}$ as $\tau(E_{\rm max}, \infty) = t_{\rm age}$, which in the Thomson limit, is given by:
%
\begin{equation}
E_{\rm max} = \frac{E_0^2}{b_0 t_{\rm age}} \simeq 400~{\rm GeV} \left( \frac{t_{\rm age}}{\rm Myr} \right)^{-1}
\end{equation}

This result provides the maximum energy of observed particles based on the age of the source.
