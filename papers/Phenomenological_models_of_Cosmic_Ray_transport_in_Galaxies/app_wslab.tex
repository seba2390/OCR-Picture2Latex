% !TEX root = ./main.tex
\section{The weighted slab technique}
\label{app:wsbla}

At present, the diffusion-losses model with the potential inclusion of advection is considered as the most suitable description for CR transport in the Galaxy at energies below approximately $10^{15}$ eV. Therefore, in this appendix we present the generalization of the transport model described in section~\ref{sec:nuclei} by including advection and ionization energy losses. 
%
This model is referred to as the \emph{weighted slab model}~\cite{Ptuskin1996apj,Jones2001apj}.

The distribution function $f_\alpha(z, p)$, which characterizes a stable species $\alpha$ subject to spallation and ionization losses, follows the transport equation:
%
\begin{multline}
-\frac{\partial}{\partial z} \left[ D_{\alpha}(p) \frac{\partial f_\alpha}{\partial z} \right]
+ u_z(z) \frac{\partial f_\alpha}{\partial z}
- \frac{du_z}{dz} \frac{p}{3} \frac{\partial f_\alpha}{\partial z}
+ \frac{f_\alpha}{\tau_{\rm sp,\alpha}}
+ \frac{1}{p^2} \frac{\partial}{\partial p}(p^2 \dot{p}_\alpha f_\alpha ) \\
= Q_\alpha(p,z) + \sum_{\alpha'>\alpha} \frac{f_{\alpha'}}{\tau_{\rm sp, \alpha'}}
\label{eq:fulltransport}
\end{multline}

Here, $D_\alpha(p, z)$ represents the diffusion coefficient of the particle, $u(z)$ is the advection velocity along the $z$-direction, and the term proportional to $du_z/dz$ accounts for adiabatic energy losses. The timescale for spallation is denoted as $\tau_{\rm sp}$, and $\dot{p} = dp/dt < 0$ captures ionization losses. The right-hand side of the equation includes the primary source term $Q_\alpha(p,z)$ and the contribution from the fragmentation of heavier nuclei.

Due to the coupling induced by nuclear fragmentation, to model CR propagation, it is required to solve a set of approximately $\sim$~100 coupled transport equations for all isotopes involved in the nuclear chain, accounting for all nuclear fragmentation yields.

In the one-dimensional slab model, a thin disk of matter represents the CR sources, $Q_\alpha(p,z) = 2 h \delta(z) Q_{\alpha, 0}(p)$, while an extended halo corresponds to the region where CRs diffuse with a diffusion coefficient that depends on rigidity.
%
In this model, we also make the assumption that $D$ and $u$ remain constant in the halo, and the advection velocity changes sign above and below the plane. Specifically, $u(z) = u_0 \left[2\Theta(z)-1\right]$, and consequently its derivative reads $\frac{du}{dz} = 2 u_0 \delta(z)$.

The majority of matter concentrates within the disk, characterized by a density $n_{\rm d}$, while the halo's density, $n_{\rm H}$, is considered negligible. Therefore, spallation and ionization losses only occur within the thin disk, resulting in $\dot{p}_\alpha(p,z) = 2 h \delta(z) \dot{p}_{0,\alpha}(p)$. It is important to note that this approximation is valid when $n_{\rm H} H \lesssim n_{\rm d} h$. If this condition is not satisfied, spallation in the halo can no longer be neglected.

To incorporate the presence of helium as a target in the ISM into the spallation term, we write the term for total inelastic fragmentation as:
%
\begin{equation}
\frac{f}{\tau_{\rm sp}} = v [n_{\rm H} \sigma_{\rm H} + n_{\rm He} \sigma_{\rm He}] f = v n_{\rm H} \sigma_{\rm H} [1 + f_{\rm He} \Sigma_{\rm He}] f
\end{equation}

Here, $f_{\rm He} = n_{\rm He} / n_{\rm H} \simeq 0.08$ represents the number density fraction of helium in the ISM, and $\Sigma_{\rm He} = \sigma_{\rm He} / \sigma_{\rm H}$ is the ratio between the inelastic cross-sections on helium and on hydrogen (with $\sigma_{\rm H}$ and $\sigma_{\rm He}$ being the respective cross-sections).

By expressing the interstellar gas density as:
%
\begin{equation}
\rho_{\rm ISM} = m_p n_{\rm H} + m_{\rm He} n_{\rm He} = m_p n_{\rm H} [ 1 + 4 f_{\rm He}]
\end{equation}
%
we obtain:
%
\begin{equation}
\frac{f}{\tau_{\rm sp}} = v \rho \frac{\sigma_{\rm H}}{m_p} \frac{[1 + f_{\rm He} \Sigma_{\rm He}]}{[1 + 4 f_{\rm He}]} f = \rho \frac{v \sigma_\alpha}{m} f = \mu \delta(z) \frac{v \sigma_\alpha}{m} f
\end{equation}
%
where $\mu = 2.3$ g cm$^{-2}$ represents the ISM surface density, and we define:
% 
\begin{equation}
\sigma_\alpha \equiv \sigma_{\rm H} \frac{1 + f_{\rm He} \Sigma_{\rm He}}{1 + f_{\rm He}} \, , \,\,\, m \equiv m_p \frac{1 + 4 f_{\rm He}}{1 + f_{\rm He}} 
\end{equation}

Adopting the assumptions described earlier, we can derive the transport equation as follows.
%
The transport equation in \ref{eq:fulltransport} takes the form:
%
\begin{multline}
-\frac{\partial}{\partial z} \left[ D_{\alpha} \frac{\partial f_\alpha}{\partial z} \right]
+ u_0 [2 \Theta(z) - 1] \frac{\partial f_\alpha}{\partial z}
- \frac{2}{3} u_0 \delta(z) p \frac{\partial f_\alpha}{\partial p}
+ \frac{\mu v \sigma_\alpha}{m} \delta(z) f_\alpha
+ \frac{2 h}{p^2} \frac{\partial}{\partial p}(p^2 \dot{p}_0 f_\alpha ) \delta(z) \\
= 2h q_{0,\alpha} \delta(z) 
+ \sum_{\alpha'>\alpha} \frac{\mu v \sigma_{\alpha'\rightarrow\alpha}}{m} \delta(z) f_{\alpha'}
\label{eq:fulltransport2}
\end{multline}

To find a solution for this equation, we focus on the region outside the disc ($z \ne 0$), where the equation can be simplified as:
%
\begin{equation}
-D_{\alpha} \frac{\partial^2 f_\alpha}{\partial z^2} 
+ u_0 \frac{\partial f_\alpha}{\partial z} = 0
\end{equation}

We assume the solution takes the form $f = A {\rm e}^{r_+ z} + B {\rm e}^{r_- z}$, where $r_\pm$ are the roots of the equation $-D_{\alpha} r^2 + u_0 r = 0$, and we find that $r_+ = 0$ and $r_- = u_0 / D_\alpha$.

Using the boundary conditions $f(z=0) = f_0$ and $f(z=H) = 0$, we determine the coefficients $A$ and $B$ as:
%
\begin{equation}
A = f_0 \frac{{\rm e}^\xi}{{\rm e}^\xi - 1} \,\,\, , \,\,\, B = f_0 \frac{1}{1-{\rm e}^\xi}
\end{equation}

Here, $\xi = \frac{u_0 H}{D_\alpha}$, and the spatial distribution of the solution in the halo becomes:
%
\begin{equation}
f(z) = f_0 \frac{1-{\exp}\left[-\xi(1-\frac{|z|}{H})\right]}{1- \exp(-\xi)}
\end{equation}
%
where $z$ is taken with the absolute value as the solution is symmetric above and below the disc.

The derivative of $f_\alpha$ evaluated at the disc location can be expressed as:
%
\begin{equation}
\left.\frac{\partial f}{\partial z}\right|_0 = -\frac{f_0}{H} \frac{\xi}{\exp (\xi) - 1}
\label{eq:fgradient}
\end{equation}
%
which is proportional to the average gradient $\sim f_0/H$.

Next, we integrate equation~\ref{eq:fulltransport2} in an infinitesimal height around the disc, resulting in:
%
\begin{multline}
-2 D_{\alpha} \frac{\partial f_{0,\alpha}}{\partial z} 
- \frac{2}{3} u_0 p \frac{\partial f_{0,\alpha}}{\partial p}
+ \frac{\mu v \sigma_\alpha}{m} f_{0,\alpha}
+ \frac{2 h}{p^2} \frac{\partial}{\partial p}(p^2 \dot{p}_{0,\alpha} f_{0,\alpha} ) \\
= 2h q_{0,\alpha} 
+ \sum_{\alpha'>\alpha} \frac{\mu v \sigma_{\alpha'\rightarrow\alpha}}{m}  f_{0,\alpha'}
\label{eq:fulltransportabovebelow}
\end{multline}

Here the symmetry of the problem above and below the disc, namely $f_\alpha(z, p) = f_\alpha(-z, p)$, leads to the factor of two in front of the diffusion term and lets the advection term disappear.

Using the spatial derivative of $f_\alpha$ (equation~\ref{eq:fgradient}) reduces the equation to a first-order differential equation for $f_{0,\alpha}(p)$:
%
\begin{multline}
\frac{2 u_0}{\mu v} \frac{1}{{\rm e}^\xi - 1} f_{0,\alpha}
- \frac{2 u_0}{3 \mu v}  p \frac{\partial f_{0,\alpha}}{\partial p}
+ \frac{2 h}{\mu v p^2} \frac{\partial}{\partial p}(p^2 \dot{p}_{0,\alpha} f_{0,\alpha} ) 
+ \frac{\sigma_\alpha}{m} f_{0,\alpha} \\
= \frac{2h}{\mu v} q_{0,\alpha} 
+ \sum_{\alpha'>\alpha} \frac{\sigma_{\alpha'\rightarrow\alpha}}{m}  f_{0,\alpha'}
\end{multline}

Multiplying both sides by $A p^2$ we express the equation in terms of the intensity as a function of the kinetic energy per nucleon $E$, which relates to $f_\alpha$ as described in equation~\ref{eq:f2I}: 
%
\begin{multline}
\frac{2 u_0}{\mu v} \frac{1}{{\rm e}^\xi - 1} I_\alpha
- \frac{2 u_0}{3 \mu v} p^3 \frac{\partial}{\partial p} \frac{I_\alpha}{p^2}
+ \frac{2 h}{\mu v} \frac{\partial}{\partial p}(\dot{p}_0 I_\alpha) 
+ \frac{\sigma_\alpha}{m} I_\alpha \\
= \frac{2h A p^2}{\mu v} q_{0,\alpha} 
+ \sum_{\alpha'>\alpha} \frac{\sigma_{\alpha'\rightarrow\alpha}}{m}  I_{\alpha'}
\end{multline}
%
where $A$ is the nuclear mass number.

To simplify the equation, we add and subtract the term $\frac{2u_0}{\mu v} I_\alpha$, resulting in:
%
\begin{multline}
\frac{2 u_0}{\mu v} \frac{1}{1-{\rm e}^{-\xi}} I_\alpha
- \frac{2 u_0}{3 \mu v} I_\alpha 
- \frac{2 u_0}{3 \mu v} p \frac{\partial I_\alpha}{\partial p} 
+ \frac{2 h}{\mu v} \frac{\partial}{\partial p}(\dot{p}_{0,\alpha} I_\alpha) 
+ \frac{\sigma_\alpha}{m} I_\alpha \\
= \frac{2h A p^2}{\mu v} q_{0,\alpha} 
+ \sum_{\alpha'>\alpha} \frac{\sigma_{\alpha'\rightarrow\alpha}}{m}  I_{\alpha'}
\end{multline}

This equation can be rewritten conveniently as:
%
\begin{multline}
\frac{I_\alpha(E)}{\rchi_\alpha}
+ \frac{d}{dE} \left( \left[ \left( \frac{dE}{d\rchi}\right)_{\rm ad} + \left( \frac{dE}{d\rchi} \right)_{\rm ion, \alpha} \right] I_\alpha(E) \right)
+ \frac{I_\alpha(E)}{\rchi_{\rm cr, \alpha}} \\
= \frac{2h A p^2}{\mu v} q_{0,\alpha} 
+ \sum_{\alpha'>\alpha} \frac{\sigma_{\alpha'\rightarrow\alpha}}{m}  I_{\alpha'}
\label{eq:fulltransport3}
\end{multline}

Here, we introduce the effective grammage:
%
\begin{equation}
\rchi_\alpha = \frac{\mu v}{2 u_0} \left[ 1 - \exp \left(-\frac{u_0 H}{D_\alpha} \right) \right]
\end{equation}
%
and the term\footnote{We use $\partial p/\partial E = A / c \beta$}:
%
\begin{equation}
 \left( \frac{dE}{d\rchi}\right)_{\rm ad} + \left( \frac{dE}{d\rchi} \right)_{\rm ion, \alpha} = 
 -\frac{2 u_0}{3 \mu c} \sqrt{E^2 + 2 m_p c^2 T} - \frac{2 h}{\mu A} |\dot p_{0,\alpha}|
\end{equation}

Notice that in the limit of $H^2 / D_\alpha \ll H / u_0$, the grammage reduces to the expression derived in the diffusion scenario in equation~\ref{eq:grammage}.

We observe that equation~\ref{eq:fulltransport3} can be written in the form:
%
\begin{equation}
\Lambda_{1,\alpha} I_\alpha(E) + \Lambda_{2, \alpha}(E) \partial_E I_\alpha(E) = Q_\alpha(E)
\end{equation}
%
which has a formal solution given by:
%
\begin{equation}
I_\alpha(E) = \int_E dE^\prime \frac{Q_\alpha(E^\prime)}{|\Lambda_{2,\alpha}(E^\prime)|} \exp\left[ -\int_E^{E^\prime} dE^{\prime\prime} \frac{\Lambda_{1,\alpha}(E^{\prime\prime})}{|\Lambda_{2,\alpha}(E^{\prime\prime})|} \right]
\end{equation}

In the more general case, this expression has to be solved numerically to derive the local interstellar spectrum for all isotopes in cosmic radiation\footnote{A practical application of this approach is implemented in the CRAMS code available at \url{https://github.com/carmeloevoli/crams}}.

Finally, we highlight that this formalism can be straightforwardly extended to model the transport of unstable nuclei, consider the effects of a finite thick disc $h$, or account for changes in the secondary production due to the grammage accumulated in the halo (see, e.g.,~\cite{Morlino2020prd,Evoli2020prd}).
