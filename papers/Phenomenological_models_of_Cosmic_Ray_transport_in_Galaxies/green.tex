% !TEX root = ./main.tex
\section{The Green function formalism}
\label{sec:green}
In the previous sections, our calculations were based on the assumption that we can approximate the distribution of CR sources in the Galaxy as continuous in both space and time within a volume that represents the overall structure of the Galaxy.

However, it is important to note that Galactic CRs are actually accelerated at discrete sources, and we do not have detailed knowledge of the individual positions and ages of these sources. Instead, we can only infer statistical properties about them.

While the continuous source approximation is valid when the density of sources is sufficiently high in both space and time, allowing us to describe them as a continuum, it fails when the transport distances and times become comparable to or shorter than the typical separations and ages of the sources. This is particularly relevant for low-energy nuclei (affected by ionization and Coulomb energy losses) and high-energy electrons (subject to inverse Compton and synchrotron losses), for which the discrete nature of the sources needs to be taken into account.

This raises the question of how we can deduce the statistical distribution of the predicted CR flux from the statistical properties of these discrete sources, as well as whether the fluctuations in CR density are significant. 

The objective of this section is to introduce the fundamental principles of this theory by developing a method for constructing the statistical quantities of interest, such as the expectation value of the flux at Earth, through the ensemble average over similar galaxies with randomly distributed galactic sources~\cite{Lee1979apj,Ptuskin2006asr,Evoli2021prdb}.

Specifically, the statistical characteristics of the quantities under investigation will be expressed in terms of a Green's function that describes the propagation of particles from a discrete monoenergetic galactic source, and of a formal expression for the probability density of CR sources at a given point in space.

In order to have a model that is relatively simple and amendable to calculations, we consider the diffusion equation for protons with a spatially constant diffusion coefficient, denoted as $D$, and neglect any energy losses. The diffusion equation takes the form:
%
\begin{equation}
\frac{\partial n(\vb r, E, t)}{\partial t} - D(E) \nabla^2 n(\vb r, E, t) = Q(\vb r, E, t)
\end{equation}

To find a suitable Green's function, denoted as $\mathcal{G}$, we seek a solution that satisfies the following equation:
%
\begin{equation}
\frac{\partial}{\partial t} \mathcal G(\vb r, t \leftarrow \vb r_\star, t_\star) 
- D \nabla^2 \mathcal G(\vb r, t \leftarrow \vb r_\star, t_\star) 
= \delta^{(3)}(\vb r - \vb r_\star) \delta(t - t_\star)
\end{equation}
%
where $\vb r_\star$ and $t_\star$ represent the position and time of injection of a CR particle. The boundary conditions for $\mathcal{G}$ are the same as those assumed for $n(\vb r, t)$.

In this formalism, the Green's function $\mathcal{G}(\vb r, t \leftarrow \vb r_\star, t_\star)$ represents the probability for a CR injected at position $\vb r_\star$ and time $t_\star$ to propagate through the Galaxy and reach an observer located at $\vb r$ at time $t$.

The formal solution of the diffusion equation can then be expressed as the convolution of the Green's function with the source term over the Galactic volume and over the past Galactic history as
%
\begin{equation}
n(\vb r, t) = \int_0^{\infty} \! dt_\star \! \int_{V} \!  d^3 \vb r_\star \, \mathcal G(\vb r, t \leftarrow \vb r_\star, t_\star) \, Q(\vb r_\star, t_\star)
\label{eq:greenconv}
\end{equation}

In particular, we can replace the continuous and smooth source term, $Q(\vb r, t)$, with an ensemble of $N$ sources with distances $\{ \vb r_i \}$ and ages $\{ t_i \}$: 
%
\begin{equation}
Q(\vb r_\star, t_\star) = \sum_{i=0}^N Q_0 \delta^{(3)}(\vb r_\star - \vb r_i) \delta(t_\star - t_i)
\label{eq:qsources}
\end{equation}

Notice that, for simplicity, we assume in this expression a common, time-independent, spectrum for all sources, whereas the Green function formalism would easily allow us to take into account an arbitrary temporal evolution of the injection of CRs and the variety of source properties.

Equation~\eqref{eq:qsources} implies that the total flux can be expressed as the sum of the individual fluxes from each source as:
%
\begin{equation}
n(\vb r, t) = Q_0 \sum_{i=0}^N \mathcal G(\vb r, t \leftarrow \vb r_i, t_i)
\end{equation}

However, since we lack knowledge about the actual positions and ages of the sources, we must resort to a Monte Carlo approach. In this approach, a large ensemble of $\mathcal{M}$ galaxies is simulated. In each realization $\alpha$ of the ensemble, the properties of each source are randomly sampled. It is important to note that, by adopting this approach, the CR density $n_\alpha$ behaves as a stochastic variable, allowing us to analyze its statistical properties.

To validate the Monte Carlo approach, we aim to demonstrate that the conventional CR model can be recovered in the \emph{mean field limit}, where we average the flux over the ensemble of all possible realizations.

To achieve this, we consider an ensemble of similar galaxies in which the $N$ sources are distributed according to the probability $\tilde P(\mathbf{r}_1, t_1; \mathbf{r}_2, t_2; \dots; \mathbf{r}_n, t_n)$, which needs to be specified. Here, $\tilde P$ represents the probability of a specific realization, characterized by ${ \mathbf{r}_i, t_i }$, to be obtained.
%
Hence, the normalization condition for this probability distribution is given by:
%
\begin{equation}
1 = \int d\mathbf{r}_1 dt_1 d\mathbf{r}_2, dt_2 \dots d\mathbf{r}_n, dt_n \tilde P(\mathbf{r}_1, t_1; \mathbf{r}_2, t_2; \dots; \mathbf{r}_n, t_n)
\end{equation}

Consequently, the ensemble-averaged density of protons at the Sun's position can be expressed as the convolution of the density, computed given a specific source configuration, and the probability of that configuration to be realized:
%
\begin{multline}
n = \langle n_\alpha(\vb r_\odot, t_\odot) \rangle_\alpha = \\
= \int \! d\vb r_1 dt_1 \dots d\vb r_n dt_n \tilde P(\vb r_1, t_1; \dots; \vb r_n, t_n) n_\alpha(\vb r_\odot, t_\odot; \vb r_1, t_1; \dots; \vb r_n, t_n)
\end{multline}

Assuming that sources are not correlated with each other, we can write this as:
%
\begin{equation}
n = Q_0 N \int_0^{\infty} \! dt_\star \! \int_{V} \! d^3 \vb r_\star \, P(\vb r_\star, t_\star) \, {\mathcal G(\vb r_\odot, t_\odot \leftarrow \vb r_\star, t_\star)} 
\label{eq:greenn}
\end{equation}

Notice that the problem of modeling galactic CRs naturally splits in two independent components: one part describes the CR transport and its geometry, embedded in the Green function $\mathcal G$, while a second part accounts of the injection, $Q_0$, and the distribution, $P$, of sources.

The probability distribution for a constant rate of injection and spatially homogeneous sources is given by:
%
\begin{equation}
P = \frac{1}{N}\frac{\mathcal{R}}{\pi R_d^2}
\end{equation}

We remind that in the case of the pure diffusive problem, the Green's function would simply be (see appendix~\ref{sec:appgreen}):
%
\begin{equation}
\mathcal{G}(\vb r_\odot, t_\odot \leftarrow \vb r, t) = \frac{1}{(4\pi D \tau)^{3/2}} 
\exp \left[ -\frac{r^2}{4 D \tau} \right] 
\sum_{n=-\infty}^{+\infty} (-1)^n \exp \left[ -\frac{(2nH)^2}{4 D \tau} \right]
\label{eq:gdiffdisk}
\end{equation}
%
where $\tau = t_\odot - t$ is the time elapsed between the injection and the observation of a CR.

As we are describing a homogeneous model, in equation~\eqref{eq:gdiffdisk}, we choose the observer position at the center of the disc, i.e., $\vb r_\odot = (0,0,0)$, without loss of generality. Additionally, we consider the disc to be infinitely thin in the $z$ direction, so the sources lie in the plane $z = 0$.

We now have all the necessary components to compute the mean Galactic density. Evaluating the integral expression in equation~\eqref{eq:greenn}, we have:
%
\begin{equation}
n = \frac{Q_0 \mathcal R}{\pi R_d^2}
\int_0^{\infty} \!\! \frac{d\tau}{(4\pi D \tau)^{3/2}} \int_{0}^{R_d} 2 \pi r_\star^2 d r_\star
\exp \left[ -\frac{r_\star^2}{4 D \tau} \right]  
\sum_{n=-\infty}^{+\infty} (-1)^n \exp \left[ -\frac{(2nH)^2}{4 D \tau} \right]
\end{equation}

Performing the integrals, first with respect to $\tau$ and then with respect to $r_\star$, we obtain:
%
\begin{equation}
n = \frac{Q_0\mathcal R}{2\pi D R_d} \sum_{n=-\infty}^{+\infty} (-1)^n 
\left[ \sqrt{1 + \left(\frac{2 n H}{R_d}\right)^2} - \sqrt{\left(\frac{2 n H}{R_d}\right)^2} \, \right]
\end{equation}

If $H \ll R_{\rm d}$, it is easy to show that the sum over $n$ tends to $H/R_d$. Therefore, we have:
%
\begin{equation}
n =\frac{Q_0 \mathcal R}{2\pi R_{\rm d}^2}\frac{H}{D}
\end{equation}

This result demonstrates that we recover the solution of the diffusion equation in steady state in equation~\eqref{eq:protonsimplesolution}.
%
The mean value of the density from randomly distributed point sources is equal to the steady-state flux obtained with a continuous source distribution. This is also true in more general cases such as thick disk, finite radius, with wind and spallation.

This situation is reminiscent of the one encountered in thermodynamics, where the values of observable (macroscopic) quantities are calculated without knowledge of the details of microscopic states.

An important advantage of introducing this approach is the ability to calculate all the higher moments of the CR observables, providing us with a comprehensive understanding of the statistical properties associated with CR densities.

In particular, we are interested in the \emph{spread} of the density around its average value. 

To compute the variance of the CR density, we can utilize a property of uncorrelated random variables with a common mean and variance. Let us consider the sum, $\psi$, of $N$ uncorrelated variables, $\phi_i$, drawn from the same probability density with a mean, $\mu$, and variance, $\sigma^2$.

As a random variable, $\psi$ has a mean given simply by:
%
\begin{equation}
\langle \psi \rangle = N \mu
\label{eq:meanpsi}
\end{equation}

Taking advantage of the uncorrelatedness of $\phi_i$, the variance of $\psi$ can be expressed as $\sigma^2(\psi) = N \sigma^2$, which can be written using the mean value in the equation above:
%
\begin{equation}
\sigma^2(\psi) = N \langle \phi^2 \rangle - \frac{\langle \psi \rangle^2}{N} \simeq N \langle \phi^2 \rangle
\label{eq:sigmapsi2}
\end{equation}

In this case, the last term is neglected, assuming that the number of sources, $N$, is sufficiently large.

It is important to note that equation~\eqref{eq:meanpsi} was implicitly employed in equation~\eqref{eq:greenn}, while equation~\eqref{eq:sigmapsi2} is used here to determine the variance, as follows:
%
\begin{equation}
\sigma^2 \left[ n(\vb r, t) \right] =
Q_0^2 N  
\int_0^{\infty} \! dt_\star \! \int_{V} \! d^3 \vb r_\star
P(\vb r_\star, t_\star) 
\mathcal G^2(\vb r, t \leftarrow \vb r_\star, t_\star)
\end{equation}

To proceed, we make use of the \emph{free} Green's function derived in appendix~\ref{sec:appgreen}. 
%
Extending the generalization to incorporate the halo function would necessitate additional, intricate calculations.

Therefore, the variance becomes:
%
\begin{equation}
\sigma^2 \left[ n(\vb r, t) \right] =
\frac{Q_0^2 \mathcal{R}}{\pi R_d^2} 
\int_0^{\infty} \! \frac{d\tau}{(4\pi D \tau)^{3}} \! \int_0^{R_{\rm d}} \! 2 \pi r_\star^2 dr_\star 
\exp \left[ -\frac{2  r_\star^2}{4 D \tau} \right] 
\end{equation}

As before, we first integrate over $\tau$, which yields:
%
\begin{equation}
\sigma^2 \left[ n(\vb r, t) \right] =
\frac{Q_0^2 \mathcal{R}}{8 \pi^3 R_d^2 D} 
\! \int_0^{R_{\rm d}} \! \frac{dr_\star}{r_\star^2}
%\! \int_0^{\infty} \! \frac{dx}{x^{3}}  \exp \left[ -\frac{2  r_\star^2}{x} \right] 
\end{equation}

We immediately notice that the spatial integral \emph{diverges} unless a lower bound is imposed on the minimum distance from the closest source, denoted as $R_{\rm min}$, such that the variance, after applying this regularization, becomes:
%
\begin{equation}
\sigma^2 \left[ n(\vb r, t) \right] =
\frac{Q_0^2 \mathcal{R}}{8 \pi^3 R_d^2 D R_{\rm min}} 
%\! \int_{R_{\min}}^\infty} \! \frac{dr_\star}{r_\star^2}
%\! \int_0^{\infty} \! \frac{dx}{x^{3}}  \exp \left[ -\frac{2  r_\star^2}{x} \right] 
\end{equation}

Finally, the strength of the fluctuation is given by:
%
\begin{equation}
\delta n (\vb r, t)  = \frac{\sqrt{\sigma^2}}{n} = \frac{1}{\sqrt{2\pi}} \frac{R_{\rm d}}{H} \frac{D^{1/2}}{\mathcal R^{1/2} R_{\rm min}^{1/2}}
\end{equation}

This result highlights the strong dependence of $\delta n$ on the cutoff radius $R_{\rm min}$ that we introduced to avoid the divergence in the integral. Moreover, if $R_{\rm min}$ is naively chosen as a fixed value, then $\delta n$ becomes an increasing function with energy proportional to $D^{1/2}$.

It must be noted that the calculation related to the standard deviation of this quantity diverges when we allow for the possibility of sources that are extremely close ($r_\star \rightarrow 0$) and young ($\tau \rightarrow 0$), which contribute to the total density with $n \rightarrow \infty$.

As the standard deviation is commonly interpreted as the typical spread of random values around the mean, a high standard deviation implies that the actual value of the flux has a disturbingly high probability of being very far from the mean value.

One might argue that the problem we considered is physically irrelevant because there are no sources with zero age and null distance to the Earth. We can impose a lower cutoff on ages and distances, based on observations for instance, to eliminate these very rare events that lead to a very high standard deviation without significantly contributing to the mean value.

However, the level of variance is highly sensitive to this cutoff, and it is not clear what value should be adopted~\cite{Mertsch2011jcap}. This issue has important implications for the theoretical limitations in the extraction of propagation parameters and for comparing theoretical models with current experimental precision~\cite{Genolini2017aa}.

In fact, it has been noted that the divergence arises from a long power law tail in the probability density function for the density from individual sources. Hence, this difficulty is not as severe as it initially seems because meaningful statistical quantities, such as confidence levels or quantiles, can still be computed in this limit. This makes the problem tractable again, and it turns out that the total flux is distributed following a stable distribution~\cite{Bernard2012aa}.

This situation, where some rare events have a very small contribution to the mean but result in a very high standard deviation, is not uncommon in physics and arises in various contexts.

