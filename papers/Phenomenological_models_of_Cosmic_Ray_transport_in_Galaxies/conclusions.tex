% !TEX root = ./main.tex
%%%%%%%%%% SECTION %%%%%%%%%
\section{Conclusions}
\label{sec:conclusions}
These notes started off in \S~\ref{sec:prelude} with an interesting anecdote about the beginning of the CR field and the link to the Varenna International School of Physics. Subsequently, they rested upon the crucial observation from our solar neighborhood (see \S~\ref{sec:intro}) contrasting the isotopic composition of the local CR flux with that of the gaseous surroundings in the solar system. Notably, the relative abundance of elements like Lithium, Beryllium, and Boron - elements that are not expected to be produced except through CRs - indicates that a lot happens between the time CRs are generated at the source and the moment they are detected. This implies a complex propagation history for each CR constituent while diffusing throughout our Galaxy. Recalling CR composition, we have seen that CRs can be classified as either primary, originating from the same ISM particles accelerated to relativistic energies, or secondary, produced during propagation thanks to different mechanisms. The secondary component is insightful in confirming the diffusive propagation of CRs across the galaxy and gives us hints about processes at play, including interaction with the galactic large-scale magnetic field, spallation, and radiative decay. A crucial role is played by grammage, which represents the amount of material a particle goes through and is an indicator of the frequency of interaction with the ISM material, and is thus directly related to the CR propagation time.

Let us now recall the key points we have pointed out throughout these lecture notes. We stress again the assumptions at the basis of every CR transport model: high-energy particles are produced and accelerated in sources in the disk with a spectrum proportional to $E^{-\gamma}$, then propagate in the Galactic halo following a diffusive motion with a diffusion coefficient proportional to $E^{\delta}$, and finally escape freely at the boundaries. Proceeding along these lines we first derived in \S~\ref{sec:protons} the spectrum of CR protons by solving the diffusion equation, resting upon the assumptions of both stationarity and a spatially constant diffusion coefficient. It is important to emphasize that the observed spectrum, proportional to $E^{-\gamma-\delta}$, is always steeper than the injected one. This observation alone does not provide sufficient insights to disentangle between acceleration effects, the $\gamma$ index, and the propagation ones, the $\delta$ index. In \S~\ref{sec:nuclei} we introduced spallation, a key mechanism consisting of fragmentation of ISM nuclei that occurs while CRs propagate and becomes relevant when a critical value for grammage is reached. An important measured quantity, as discussed in \S~\ref{sec:secondaryoverprimary}, is the secondary-over-primary ratio in both regimes of strong and weak spallation, as it helps us to estimate a value for the diffusion coefficient, provided that an assumption about the halo size has to be made. In \S~\ref{sec:unstable} we showed how secondary unstable isotopes, such as Berillium, can serve us as a cosmic clock to estimate the CR residence time and thus allow us to assess both the value of the diffusion coefficient and that of the halo size. After a quick recap of the general form of CR spectrum (see \S~\ref{sec:poorphysicist}) and the limitations of the leaky box approximation to the transport equation (see \S~\ref{sec:leakybox}), finally in \S~\ref{sec:leptons} we investigated the leptonic component, i.e. electrons and positrons. The observed positron fraction poses numerous theoretical challenges up to postulating the existence of a new population of sources of antimatter in the Universe. Last but not least, in \S~\ref{sec:implications}, we delved into some considerations about the micro-physics behind the transport process and we concluded with \S~\ref{sec:green} showing the basics of the Green's function formalism.

The outlined phenomenological model captures the fundamental observations and highlights the key physical processes CRs experience while diffusing within our Galaxy. Notwithstanding, this framework extends beyond our Galaxy's confines, with applications to a wide array of astrophysical environments.