% !TEX root = ./main.tex
%%%%%%%%%% SECTION %%%%%%%%%
\section{Prelude}
\label{sec:prelude}

\begin{figure}[t]
\centering
\includegraphics[width=0.7\textwidth]{figures/FermiRossiVarenna.pdf}
\caption{Bruno Rossi (left) and Enrico Fermi (right) in Villa Monastero in 1954.}
\label{fig:rossifermi}
\end{figure}

In 1948, the Italian physicist Enrico Fermi attended a seminar in Chicago conducted by the Swedish physicist Hannes Alfvén, which focused on the motion of magnetic fields in dilute plasma and its potential connection to the origin of cosmic rays within the solar system (later published in~\cite{Alfven1949pr}).
%
Intrigued by Alfvén's arguments, Fermi approached him after the seminar and requested a more detailed explanation of ``magneto-hydrodynamic'' waves. According to Alfvén's recollections~\cite{Alfven1988amsci}, six years had passed since the publication of his initial paper on the subject~\cite{Alfven1942nat}, but only a few scientists acknowledged the existence of these waves, while most of his contemporaries dismissed the idea as foolish naivety.
%
After listening to Alfvén's explanation, for ``five or ten minutes'', Fermi responded, ``of course, such waves could exist!''. 
%
Fermi held such authority in the scientific community that, as Alfvén described it, ``if he said 'of course' today, every physicist said 'of course' tomorrow.''

Within a span of a few months, Fermi submitted a paper titled ``On the Origin of the Cosmic Radiation'' to the journal Physical Review, which was subsequently published in the April 15, 1949, issue, containing his prompt application of Alfvén's ideas~\cite{Fermi1949pr}.

In this seminal paper, Fermi utilized Alfvén's theory of the velocity of magnetoelastic wave propagation to propose the existence of magnetic fields in the rarefied interstellar medium. He made a surprisingly accurate prediction, given the time, that these fields should have a magnitude on the order of $5~\mu$G.

Furthermore, Fermi proposed that the intensity of these magnetic fields may vary in space, potentially being stronger within denser interstellar clouds. In this setup, charged particles would gain energy through head-on collisions with these clouds or lose energy during overtaking collisions. On average, head-on collisions were more probable, leading to a net energy gain for the particles.
%
As a consequence, ``the theory naturally yields an inverse power law for the spectral distribution of the cosmic rays,'' which remained unexplained at that time.

In hindsight, this mechanism proved to be too slow and inefficient to account for the observed energies of cosmic rays. Moreover, the power index depended on local details of the model, failing to generate a universal power law distribution. Nonetheless, this concept, known today as the \emph{second order Fermi mechanism}, became the foundation of many subsequent discussions on the subject, as it provided the minimum conditions necessary for the emergence of a power-law distribution in energy.

Of particular significance for our notes, Fermi's remarkable theory rested on the assumption that magnetic fields on a galactic scale existed within the interstellar medium, trapping cosmic rays. This proposal marked a pioneering hypothesis regarding the \emph{Galactic origin} of cosmic rays.

Fermi promptly presented his theory during the second edition of the International Cosmic Ray Conference in Como, Italy, in 1949, in a presentation titled ``Una teoria sull'accelerazione dei raggi cosmici''~\cite{Fermi1949icrc}.
%
He continued his work on this subject thereafter, and it is highly likely that he discussed his theory on galactic cosmic rays with Bruno Rossi during his last visit to Italy which took place during the second edition of the International School of Physics organized by the Italian Physical Society in Varenna in 1954 (figure~\ref{fig:rossifermi}).

With courage befitting the setting where Fermi delivered his last public seminar, the following lectures were given about 70 years later on the subsequent progresses of this topic.