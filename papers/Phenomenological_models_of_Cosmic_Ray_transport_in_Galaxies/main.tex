\documentclass[varenna]{cimento}
% OLD PREAMBLE:

% \usepackage{jsen}
% \usepackage{cite}
% \usepackage{amsmath,amssymb,amsfonts, bbm, mathtools}
% \usepackage{algorithm,algorithmic}
% \usepackage{graphicx}
% \usepackage{textcomp}
% \usepackage{wrapfig}
% \usepackage{xfrac}
% \usepackage{stackengine}
% \usepackage{subfigure}
% \def\delequal{\mathrel{\ensurestackMath{\stackon[1pt]{=}{\scriptstyle\Delta}}}}



% \usepackage{color, soul}
% \newcommand{\hlt}[1]{\hl{#1}}
% \newcommand{\red}[1]{\textcolor{red}{#1}}

% \def\BibTeX{{\rm B\kern-.05em{\sc i\kern-.025em b}\kern-.08em
%     T\kern-.1667em\lower.7ex\hbox{E}\kern-.125emX}}
% \markboth{\journalname, VOL. XX, NO. XX, XXXX 2017}
% {Author \MakeLowercase{\textit{et al.}}: Preparation of Papers for IEEE TRANSACTIONS and JOURNALS (February 2017)}
% \definecolor{abstractbg}{rgb}{0.89804,0.94510,0.83137}
% \setlength{\fboxrule}{0pt}
% \setlength{\fboxsep}{0pt}

% NEW PREAMBLE:


\usepackage{amsmath,amsfonts,amssymb,bbm, amsthm, xfrac}
\usepackage{algorithmic}
\usepackage{algorithm}
\usepackage{array, multirow}
% \usepackage[caption=false,font=normalsize,labelfont=sf,textfont=sf]{subfig}
\usepackage{caption, subcaption}
\usepackage{textcomp}
\usepackage{stfloats}
\usepackage{url}
\usepackage{verbatim}
\usepackage{graphicx}
\usepackage{cite}
\usepackage{caption}
\usepackage{subcaption}
\hyphenation{}

\theoremstyle{plain}
\newtheorem{theorem}{Theorem}

\usepackage{color, soul}
\newcommand{\hlt}[1]{\hl{#1}}
\newcommand{\red}[1]{\textcolor{red}{#1}}


\title{Phenomenological models of Cosmic Ray transport in Galaxies}
\author{Carmelo Evoli \atque Ulyana Dupletsa}
\institute{Gran Sasso Science Institute, Viale Francesco Crispi 7, 67100 L'Aquila, Italy \\ INFN-Laboratori Nazionali del Gran Sasso (LNGS), via G. Acitelli 22, 67100 Assergi (AQ), Italy}
\shortauthor{C.~Evoli \atque U.~Dupletsa}

\makeindex

\begin{document}

\maketitle

\begin{abstract}
When examining the abundance of elements in the placid interstellar medium, a deep hollow between helium and carbon becomes apparent.
%
Notably, the fragile light nuclei Lithium, Beryllium, and Boron (collectively known as LiBeB) are not formed, with the exception of Li7, during the process of Big Bang nucleosynthesis, nor do they arise as byproducts of stellar lifecycles.
%
In contrast to the majority of elements, these species owe their existence to the most energetic particles in the Universe. 

Cosmic rays, originating in the most powerful Milky Way's particle accelerators, reach the Earth after traversing tangled and lengthy paths spanning millions of years.
% 
During their journey, these primary particles undergo transformations through collisions with interstellar matter. This process, known as \emph{spallation}, alters their composition and introduces secondary light elements in the cosmic-ray beam.

In light of this, the relatively large abundance of LiBeB in the cosmic radiation provides remarkable insights into the  mechanisms of particle acceleration, as well as the micro-physics of confinement within galactic magnetic fields.

These lecture notes are intended to equip readers with basic knowledge necessary for examining the chemical and isotopic composition, as well as the energy spectra, of cosmic rays, finally fostering a more profound comprehension of the complex high-energy astrophysical processes occurring within our Galaxy.
\end{abstract}

%\tableofcontents

\usepackage[pdftex]{graphicx}
\usepackage{siunitx}
% \usepackage{gitinfo2}
\usepackage{amsfonts}
\usepackage{amsmath}
% \usepackage[textsize=tiny]{todonotes}
% \setuptodonotes{inline}
\usepackage[hidelinks]{hyperref}
\usepackage{csquotes}

\newlength{\onecolfig}
\setlength{\onecolfig}{86mm} % single-column width
\newlength{\twocolfig}
\setlength{\twocolfig}{178mm} % double-column width

\newcommand{\eqtodo}[1]{\textrm{#1}}
\newcommand{\eqnref}[1]{eq.~(\ref{#1})}
\newcommand{\citeref}[1]{ref.~\cite{#1}}
\newcommand{\figref}[1]{fig.~\ref{#1}}
\newcommand{\secref}[1]{section~\ref{#1}}
\newcommand{\ish}{\mbox{$\sim$}\,}
\newcommand{\ltish}{\protect\raisebox{-0.4ex}{$\,\stackrel{<}{\scriptstyle\sim}\,$}}
\newcommand{\gtish}{\protect\raisebox{-0.4ex}{$\,\stackrel{>}{\scriptstyle\sim}\,$}}

\newcommand{\RF}{RF}
\newcommand{\DC}{DC}
\newcommand{\hessian}[1]{\mathbf{H}_{#1}}
\newcommand{\FPGA}{FPGA}
\newcommand{\kvec}{\vec{k}}
\newcommand{\caplus}{\textsuperscript{43}Ca\textsuperscript{+}}
\newcommand{\srplus}{\textsuperscript{88}Sr\textsuperscript{+}}
\newcommand{\charge}{Z e}  % can't use q, Q…
\newcommand{\wRF}{\Omega_{\textrm{rf}}}
\newcommand{\wAM}{\omega_{\textrm{am}}}
\newcommand{\wm}{\omega_{\textrm{0}}}
\newcommand{\dcpot}{\Phi_{\textrm{dc}}}
\newcommand{\rfpot}{\Phi_{\textrm{rf}}}
\newcommand{\amat}{\mathcal{A}}
\newcommand{\qmat}{\mathcal{Q}}
\newcommand{\ee}{\mathrm{e}}
\newcommand{\ii}{\mathrm{i}}
\newcommand{\h}{h}
\newcommand{\hred}{\tilde{h}}
\newcommand{\prindex}{n}
\newcommand{\defeq}{:=}
\newcommand{\re}[1]{\operatorname{re}\left(#1\right)}
\newcommand{\im}[1]{\operatorname{im}\left(#1\right)}
\newcommand{\diff}[2]{\frac{\operatorname{d}\!{#1}}{\operatorname{d}\!{#2}}}
\newcommand{\difftwo}[2]{\frac{\operatorname{d}^2\!{#1}}{\operatorname{d}\!{#2}^2}}
\newcommand{\kron}[1]{\mathrm{\delta}_{#1}}
\renewcommand{\vec}[1]{\boldsymbol{#1}}


Reinforcement learning has achieved great success in areas such as Game-playing \citep{silver2018general,vinyals2019grandmaster}, robotics \cite{kober2013reinforcement}, large language models \citep{ouyang2022training}, etc.
However, due to safety concerns or physical limitations, in some real-world reinforcement learning problems, we must consider additional constraints that may influence the optimal policy and the learning process \citep{garcia2015comprehensive}.
% For example, a robotic arm must not take actions that may cause harm to itself or the environments.
A standard framework to handle such cases is the constrained Markov Decision Process (CMDP) \citep{altman1999constrained}.
Within the CMDP framework, the agent has to maximize
the expected cumulative reward while
obeying a finite number of constraints, which are usually in the form of expected cumulative cost criteria.

However, we are sometimes concerned with the problem with a continuum of constraints.
For example,
the constraints we meet might be time-evolving or subject to uncertain parameters, which
cannot be formulated as an ordinary CMDP
(see Examples \ref{Example_Time_Evolving} and  \ref{Example_Uncertain}).
In this paper we would study a generalized CMDP  
to address the above problem.  Because the constraints are not only infinite-number but also lie
in a continuous set,
the generalization is not trivial. Fortunately, we find that we can borrow the idea behind semi-infinite programming (SIP) \citep{remez1934determination, hettich1993semi} to deal with the semi-infinite constraints.
Accordingly, we propose \emph{semi-infinitely constrained Markov decision processes} (SICMDPs)
as a novel complement to the ordinary CMDP framework.
%More specifically,  an SICMDP model %, we consider 
%contains a continuum of constraints whereas an ordinary CMDP contains a finite number of constraints. 

%This generalization is natural but not trivial. However, we can brows the idea  
%The idea is quite natural and can be backtracked
%to the practice of extending linear programming to linear semi-infinite programming (LSIP) %\cite{remez1934determination, GobernaLSIO1998}.
%In addition, 
%As a complementary approach to the ordinary CMDP framework, 
%SICMDP can be used to model these problems  which cannot be described by a finite number of constraints
%that are not covered by .
%For example,
%the restrictions we consider can be time-evolving or subject to uncertain parameters
%, thus
%cannot be described by a finite number of constraints but a continuum of constraints 
%(see Examples \ref{Example_Time_Evolving} and  \ref{Example_Uncertain}).

We also present two reinforcement learning algorithms to solve SICMDPs called SI-CRL and SI-CPO, respectively.
SI-CRL is a model-based reinforcement learning algorithm designed for tabular cases, and SI-CPO is a policy optimization algorithm for non-tabular cases.
% and analyze its performance both theoretically and empirically.
The main challenge is that we need to deal with a continuum of constraints, thus reinforcement learning algorithms for ordinary CMDPs do not work anymore.
In SI-CRL, we tackle this difficulty by first transforming the reinforcement learning problem to an equivalent LSIP problem, which can then be solved using methods in the LSIP literature like the dual exchange methods \citep{Hu1990,reemtsen1998numerical}.
In SI-CPO, we resort to the idea of cooperative stochastic approximation developed in \cite{lan2020algorithms, wei2020comirror}.
As far as we know, we are the first to introduce tools from semi-infinitely programming (SIP) into the reinforcement learning community for solving constrained reinforcement learning problems.

% To the best of our knowledge, we are the first to apply tools from semi-infinitely programming (SIP) to solve reinforcement learning problems.
Furthermore, we give theoretical analysis for both SI-CRL and SI-CPO.
We decompose the error of SI-CRL into two parts: the statistical error from approximating the true SICMDP with an offline dataset and the optimization error due to the fact that the solution of the LSIP problem obtained by the dual exchange method is inexact.
On the optimization side, we show that the iteration complexity of SI-CRL is $O\left(\left\{\mathrm{diam}(Y)L\sqrt{|\gS|^2|\gA|m}/\left[(1-\gamma)\epsilon\right]\right\}^m\right)$.
On the statistical side, we show that the sample complexity of SI-CRL is $\widetilde O\left(\frac{|S|^2|A|^2}{\epsilon^2(1-\gamma)^3}\right)$ if the offline dataset is generated by a generative model, and $\widetilde O\left(\frac{|S||A|}{\nu_{\min} \epsilon^2(1-\gamma)^3}\right)$ if the dataset is generated by a probability measure $\nu$ as considered in \cite{chen2019information}.
Here $\widetilde O$ means that all logarithm terms are discarded.
For SI-CPO, things become a little more complicated because other than the statistical error and the optimization error, we also need to consider the function approximation error, which comes from imperfect policy parametrizations.
It is shown if the function approximation error can be controlled to $O(\epsilon)$ order, the iteration complexity of SI-CPO is $\widetilde{O}\left(\frac{1}{\epsilon^2(1-\gamma)^6}\right)$ and the sample complexity of SI-CPO is $\widetilde{O}(\frac{1}{\epsilon^4(1-\gamma)^{10}})$.
Here our iteration complexity bound is equivalent to a typical $\widetilde O(1/\sqrt{T})$ global convergence rate.

We perform a set of numerical experiments to illustrate the SICMDP model and validate our proposed algorithms.
Specifically, we examine two numerical examples, namely the discharge of sewage and ship route planning.
Through the discharge of sewage example, we show the advantage of the SICMDP framework over the CMDP baseline obtained by naive discretization in modeling realistic sequential decision-making problems.
Moreover, we demonstrate the effectiveness of the SI-CRL and SI-CPO algorithms in such tabular environments. 
In the ship route planning example, we illustrate the benefits of the SICMDP framework and the ability of the SI-CPO algorithm to address complex continuous control tasks involving continuous state spaces with modern deep reinforcement learning techniques.

% In summary, our contributions are listed as follows.
% First, we present the SICMDP model, which can be viewed as a generalization of the ordinary CMDP model.
% Second, we propose an algorithm to perform reinforcement learning for SICMDPs, which is called SI-CRL, and we believe that we are the first to apply tools from SIP
% to solve reinforcement learning problems.
% Third, we give a theoretical analysis of SI-CRL and identify both its sample complexity and iteration complexity.
% In addition, we perform numerical experiments to illustrate the SICMDP model and validate the SI-CRL algorithm.
% \{This paragraph can be removed!!! \}





\input{protons}
\input{implications}
% !TEX root = ./main.tex
\section{The Green function formalism}
\label{sec:green}
In the previous sections, our calculations were based on the assumption that we can approximate the distribution of CR sources in the Galaxy as continuous in both space and time within a volume that represents the overall structure of the Galaxy.

However, it is important to note that Galactic CRs are actually accelerated at discrete sources, and we do not have detailed knowledge of the individual positions and ages of these sources. Instead, we can only infer statistical properties about them.

While the continuous source approximation is valid when the density of sources is sufficiently high in both space and time, allowing us to describe them as a continuum, it fails when the transport distances and times become comparable to or shorter than the typical separations and ages of the sources. This is particularly relevant for low-energy nuclei (affected by ionization and Coulomb energy losses) and high-energy electrons (subject to inverse Compton and synchrotron losses), for which the discrete nature of the sources needs to be taken into account.

This raises the question of how we can deduce the statistical distribution of the predicted CR flux from the statistical properties of these discrete sources, as well as whether the fluctuations in CR density are significant. 

The objective of this section is to introduce the fundamental principles of this theory by developing a method for constructing the statistical quantities of interest, such as the expectation value of the flux at Earth, through the ensemble average over similar galaxies with randomly distributed galactic sources~\cite{Lee1979apj,Ptuskin2006asr,Evoli2021prdb}.

Specifically, the statistical characteristics of the quantities under investigation will be expressed in terms of a Green's function that describes the propagation of particles from a discrete monoenergetic galactic source, and of a formal expression for the probability density of CR sources at a given point in space.

In order to have a model that is relatively simple and amendable to calculations, we consider the diffusion equation for protons with a spatially constant diffusion coefficient, denoted as $D$, and neglect any energy losses. The diffusion equation takes the form:
%
\begin{equation}
\frac{\partial n(\vb r, E, t)}{\partial t} - D(E) \nabla^2 n(\vb r, E, t) = Q(\vb r, E, t)
\end{equation}

To find a suitable Green's function, denoted as $\mathcal{G}$, we seek a solution that satisfies the following equation:
%
\begin{equation}
\frac{\partial}{\partial t} \mathcal G(\vb r, t \leftarrow \vb r_\star, t_\star) 
- D \nabla^2 \mathcal G(\vb r, t \leftarrow \vb r_\star, t_\star) 
= \delta^{(3)}(\vb r - \vb r_\star) \delta(t - t_\star)
\end{equation}
%
where $\vb r_\star$ and $t_\star$ represent the position and time of injection of a CR particle. The boundary conditions for $\mathcal{G}$ are the same as those assumed for $n(\vb r, t)$.

In this formalism, the Green's function $\mathcal{G}(\vb r, t \leftarrow \vb r_\star, t_\star)$ represents the probability for a CR injected at position $\vb r_\star$ and time $t_\star$ to propagate through the Galaxy and reach an observer located at $\vb r$ at time $t$.

The formal solution of the diffusion equation can then be expressed as the convolution of the Green's function with the source term over the Galactic volume and over the past Galactic history as
%
\begin{equation}
n(\vb r, t) = \int_0^{\infty} \! dt_\star \! \int_{V} \!  d^3 \vb r_\star \, \mathcal G(\vb r, t \leftarrow \vb r_\star, t_\star) \, Q(\vb r_\star, t_\star)
\label{eq:greenconv}
\end{equation}

In particular, we can replace the continuous and smooth source term, $Q(\vb r, t)$, with an ensemble of $N$ sources with distances $\{ \vb r_i \}$ and ages $\{ t_i \}$: 
%
\begin{equation}
Q(\vb r_\star, t_\star) = \sum_{i=0}^N Q_0 \delta^{(3)}(\vb r_\star - \vb r_i) \delta(t_\star - t_i)
\label{eq:qsources}
\end{equation}

Notice that, for simplicity, we assume in this expression a common, time-independent, spectrum for all sources, whereas the Green function formalism would easily allow us to take into account an arbitrary temporal evolution of the injection of CRs and the variety of source properties.

Equation~\eqref{eq:qsources} implies that the total flux can be expressed as the sum of the individual fluxes from each source as:
%
\begin{equation}
n(\vb r, t) = Q_0 \sum_{i=0}^N \mathcal G(\vb r, t \leftarrow \vb r_i, t_i)
\end{equation}

However, since we lack knowledge about the actual positions and ages of the sources, we must resort to a Monte Carlo approach. In this approach, a large ensemble of $\mathcal{M}$ galaxies is simulated. In each realization $\alpha$ of the ensemble, the properties of each source are randomly sampled. It is important to note that, by adopting this approach, the CR density $n_\alpha$ behaves as a stochastic variable, allowing us to analyze its statistical properties.

To validate the Monte Carlo approach, we aim to demonstrate that the conventional CR model can be recovered in the \emph{mean field limit}, where we average the flux over the ensemble of all possible realizations.

To achieve this, we consider an ensemble of similar galaxies in which the $N$ sources are distributed according to the probability $\tilde P(\mathbf{r}_1, t_1; \mathbf{r}_2, t_2; \dots; \mathbf{r}_n, t_n)$, which needs to be specified. Here, $\tilde P$ represents the probability of a specific realization, characterized by ${ \mathbf{r}_i, t_i }$, to be obtained.
%
Hence, the normalization condition for this probability distribution is given by:
%
\begin{equation}
1 = \int d\mathbf{r}_1 dt_1 d\mathbf{r}_2, dt_2 \dots d\mathbf{r}_n, dt_n \tilde P(\mathbf{r}_1, t_1; \mathbf{r}_2, t_2; \dots; \mathbf{r}_n, t_n)
\end{equation}

Consequently, the ensemble-averaged density of protons at the Sun's position can be expressed as the convolution of the density, computed given a specific source configuration, and the probability of that configuration to be realized:
%
\begin{multline}
n = \langle n_\alpha(\vb r_\odot, t_\odot) \rangle_\alpha = \\
= \int \! d\vb r_1 dt_1 \dots d\vb r_n dt_n \tilde P(\vb r_1, t_1; \dots; \vb r_n, t_n) n_\alpha(\vb r_\odot, t_\odot; \vb r_1, t_1; \dots; \vb r_n, t_n)
\end{multline}

Assuming that sources are not correlated with each other, we can write this as:
%
\begin{equation}
n = Q_0 N \int_0^{\infty} \! dt_\star \! \int_{V} \! d^3 \vb r_\star \, P(\vb r_\star, t_\star) \, {\mathcal G(\vb r_\odot, t_\odot \leftarrow \vb r_\star, t_\star)} 
\label{eq:greenn}
\end{equation}

Notice that the problem of modeling galactic CRs naturally splits in two independent components: one part describes the CR transport and its geometry, embedded in the Green function $\mathcal G$, while a second part accounts of the injection, $Q_0$, and the distribution, $P$, of sources.

The probability distribution for a constant rate of injection and spatially homogeneous sources is given by:
%
\begin{equation}
P = \frac{1}{N}\frac{\mathcal{R}}{\pi R_d^2}
\end{equation}

We remind that in the case of the pure diffusive problem, the Green's function would simply be (see appendix~\ref{sec:appgreen}):
%
\begin{equation}
\mathcal{G}(\vb r_\odot, t_\odot \leftarrow \vb r, t) = \frac{1}{(4\pi D \tau)^{3/2}} 
\exp \left[ -\frac{r^2}{4 D \tau} \right] 
\sum_{n=-\infty}^{+\infty} (-1)^n \exp \left[ -\frac{(2nH)^2}{4 D \tau} \right]
\label{eq:gdiffdisk}
\end{equation}
%
where $\tau = t_\odot - t$ is the time elapsed between the injection and the observation of a CR.

As we are describing a homogeneous model, in equation~\eqref{eq:gdiffdisk}, we choose the observer position at the center of the disc, i.e., $\vb r_\odot = (0,0,0)$, without loss of generality. Additionally, we consider the disc to be infinitely thin in the $z$ direction, so the sources lie in the plane $z = 0$.

We now have all the necessary components to compute the mean Galactic density. Evaluating the integral expression in equation~\eqref{eq:greenn}, we have:
%
\begin{equation}
n = \frac{Q_0 \mathcal R}{\pi R_d^2}
\int_0^{\infty} \!\! \frac{d\tau}{(4\pi D \tau)^{3/2}} \int_{0}^{R_d} 2 \pi r_\star^2 d r_\star
\exp \left[ -\frac{r_\star^2}{4 D \tau} \right]  
\sum_{n=-\infty}^{+\infty} (-1)^n \exp \left[ -\frac{(2nH)^2}{4 D \tau} \right]
\end{equation}

Performing the integrals, first with respect to $\tau$ and then with respect to $r_\star$, we obtain:
%
\begin{equation}
n = \frac{Q_0\mathcal R}{2\pi D R_d} \sum_{n=-\infty}^{+\infty} (-1)^n 
\left[ \sqrt{1 + \left(\frac{2 n H}{R_d}\right)^2} - \sqrt{\left(\frac{2 n H}{R_d}\right)^2} \, \right]
\end{equation}

If $H \ll R_{\rm d}$, it is easy to show that the sum over $n$ tends to $H/R_d$. Therefore, we have:
%
\begin{equation}
n =\frac{Q_0 \mathcal R}{2\pi R_{\rm d}^2}\frac{H}{D}
\end{equation}

This result demonstrates that we recover the solution of the diffusion equation in steady state in equation~\eqref{eq:protonsimplesolution}.
%
The mean value of the density from randomly distributed point sources is equal to the steady-state flux obtained with a continuous source distribution. This is also true in more general cases such as thick disk, finite radius, with wind and spallation.

This situation is reminiscent of the one encountered in thermodynamics, where the values of observable (macroscopic) quantities are calculated without knowledge of the details of microscopic states.

An important advantage of introducing this approach is the ability to calculate all the higher moments of the CR observables, providing us with a comprehensive understanding of the statistical properties associated with CR densities.

In particular, we are interested in the \emph{spread} of the density around its average value. 

To compute the variance of the CR density, we can utilize a property of uncorrelated random variables with a common mean and variance. Let us consider the sum, $\psi$, of $N$ uncorrelated variables, $\phi_i$, drawn from the same probability density with a mean, $\mu$, and variance, $\sigma^2$.

As a random variable, $\psi$ has a mean given simply by:
%
\begin{equation}
\langle \psi \rangle = N \mu
\label{eq:meanpsi}
\end{equation}

Taking advantage of the uncorrelatedness of $\phi_i$, the variance of $\psi$ can be expressed as $\sigma^2(\psi) = N \sigma^2$, which can be written using the mean value in the equation above:
%
\begin{equation}
\sigma^2(\psi) = N \langle \phi^2 \rangle - \frac{\langle \psi \rangle^2}{N} \simeq N \langle \phi^2 \rangle
\label{eq:sigmapsi2}
\end{equation}

In this case, the last term is neglected, assuming that the number of sources, $N$, is sufficiently large.

It is important to note that equation~\eqref{eq:meanpsi} was implicitly employed in equation~\eqref{eq:greenn}, while equation~\eqref{eq:sigmapsi2} is used here to determine the variance, as follows:
%
\begin{equation}
\sigma^2 \left[ n(\vb r, t) \right] =
Q_0^2 N  
\int_0^{\infty} \! dt_\star \! \int_{V} \! d^3 \vb r_\star
P(\vb r_\star, t_\star) 
\mathcal G^2(\vb r, t \leftarrow \vb r_\star, t_\star)
\end{equation}

To proceed, we make use of the \emph{free} Green's function derived in appendix~\ref{sec:appgreen}. 
%
Extending the generalization to incorporate the halo function would necessitate additional, intricate calculations.

Therefore, the variance becomes:
%
\begin{equation}
\sigma^2 \left[ n(\vb r, t) \right] =
\frac{Q_0^2 \mathcal{R}}{\pi R_d^2} 
\int_0^{\infty} \! \frac{d\tau}{(4\pi D \tau)^{3}} \! \int_0^{R_{\rm d}} \! 2 \pi r_\star^2 dr_\star 
\exp \left[ -\frac{2  r_\star^2}{4 D \tau} \right] 
\end{equation}

As before, we first integrate over $\tau$, which yields:
%
\begin{equation}
\sigma^2 \left[ n(\vb r, t) \right] =
\frac{Q_0^2 \mathcal{R}}{8 \pi^3 R_d^2 D} 
\! \int_0^{R_{\rm d}} \! \frac{dr_\star}{r_\star^2}
%\! \int_0^{\infty} \! \frac{dx}{x^{3}}  \exp \left[ -\frac{2  r_\star^2}{x} \right] 
\end{equation}

We immediately notice that the spatial integral \emph{diverges} unless a lower bound is imposed on the minimum distance from the closest source, denoted as $R_{\rm min}$, such that the variance, after applying this regularization, becomes:
%
\begin{equation}
\sigma^2 \left[ n(\vb r, t) \right] =
\frac{Q_0^2 \mathcal{R}}{8 \pi^3 R_d^2 D R_{\rm min}} 
%\! \int_{R_{\min}}^\infty} \! \frac{dr_\star}{r_\star^2}
%\! \int_0^{\infty} \! \frac{dx}{x^{3}}  \exp \left[ -\frac{2  r_\star^2}{x} \right] 
\end{equation}

Finally, the strength of the fluctuation is given by:
%
\begin{equation}
\delta n (\vb r, t)  = \frac{\sqrt{\sigma^2}}{n} = \frac{1}{\sqrt{2\pi}} \frac{R_{\rm d}}{H} \frac{D^{1/2}}{\mathcal R^{1/2} R_{\rm min}^{1/2}}
\end{equation}

This result highlights the strong dependence of $\delta n$ on the cutoff radius $R_{\rm min}$ that we introduced to avoid the divergence in the integral. Moreover, if $R_{\rm min}$ is naively chosen as a fixed value, then $\delta n$ becomes an increasing function with energy proportional to $D^{1/2}$.

It must be noted that the calculation related to the standard deviation of this quantity diverges when we allow for the possibility of sources that are extremely close ($r_\star \rightarrow 0$) and young ($\tau \rightarrow 0$), which contribute to the total density with $n \rightarrow \infty$.

As the standard deviation is commonly interpreted as the typical spread of random values around the mean, a high standard deviation implies that the actual value of the flux has a disturbingly high probability of being very far from the mean value.

One might argue that the problem we considered is physically irrelevant because there are no sources with zero age and null distance to the Earth. We can impose a lower cutoff on ages and distances, based on observations for instance, to eliminate these very rare events that lead to a very high standard deviation without significantly contributing to the mean value.

However, the level of variance is highly sensitive to this cutoff, and it is not clear what value should be adopted~\cite{Mertsch2011jcap}. This issue has important implications for the theoretical limitations in the extraction of propagation parameters and for comparing theoretical models with current experimental precision~\cite{Genolini2017aa}.

In fact, it has been noted that the divergence arises from a long power law tail in the probability density function for the density from individual sources. Hence, this difficulty is not as severe as it initially seems because meaningful statistical quantities, such as confidence levels or quantiles, can still be computed in this limit. This makes the problem tractable again, and it turns out that the total flux is distributed following a stable distribution~\cite{Bernard2012aa}.

This situation, where some rare events have a very small contribution to the mean but result in a very high standard deviation, is not uncommon in physics and arises in various contexts.



\section{Discussion and Conclusions}



Our method based on stabilizing forward and backward pass, resulted in improved accuracy over the baseline and it was able to predict optimal dampening, sharpness and tail-fatness before training. 
Our findings are coherent with the line of research that has established that stabilizing gradients and representations at initialization results in better performance \cite{glorot2010understanding, orthogonal_initialization, he2015delving, roberts2022principles, defazio2022scaling, bengio1994learning, hochreiter1997long, hochreiter2001gradient, arjovsky2016unitary, pascanu2013difficulty}. Moreover it gives an initial reply to the question raised by
\cite{surrogate2019, zenke2021remarkable}, which asked  for a theoretical justification of initialization and SG choice for Spiking Neural Networks. With a similar intention, \cite{rossbroich2022fluctuation} proposed an approach that guarantees sparsity of activity at initialization to pick the weights distribution at initialization, resulting in improved accuracy. Our method differs from theirs in that it starts from a principle of stability to derive constraints, instead of a principle of sparsity. It differs also in that we use it to define the SG shape at initialization, not only the weights distribution, and we can show mathematically how weights initialization is intertwined to the SG shape choice. Our results suggest that a tedious hyper-parameter grid-search can be often avoided by making use of sound and established principles of learning.

One of the conditions was designed to hit the most sensitive part of an SG, its center, which resulted in a low sparsity requirement at initialization. This is very uncommon in the Neuromorphic literature, since sparsity brings large energy gains \cite{henderson2020towards,blouw2019benchmarking, 9395703,taulsnn, rossbroich2022fluctuation}.
However, the energy gains of SNNs also come from their binary activity. A matrix-vector multiplication, with a $\mathbb{R}^{m\times n}$ matrix, has an energy cost of $mnE_{MAC}$ for a real vector, and of $mn\rho E_{AC}$ for a binary vector, where $\rho$ is the Bernouilli probability of the binary vector, and in our case the neuron firing rate, and $E_{AC}, E_{MAC}$ are the energies of an accumulate and a multiply-accumulate operation \cite{yin2021accurate, hunger2005floating}. Since MAC are more costly than AC, 31 times on a $45$nm complementary metal–oxide–semiconductor \cite{yin2021accurate, horowitz20141}, we have energy savings with any $\rho$, e.g., when all neurons fire ($\rho=1$) and when they fire half of the time steps ($\rho=1/2$). This gain does not depend on the simulation speed, since it compares a spiking and an analogue computation, at the same computation speed.
Typically requiring more sparsity through a sparsity encouraging loss term, leads to a measurable decrease in performance \cite{zenke2021remarkable, rossbroich2022fluctuation}. However we observed that it is actually possible to achieve higher performance with higher sparsity, by starting with a strong firing rate at initialization, since their synergy acts as a regularization mechanism. This was possible also because the sparsity encouraging loss term was introduced gradually, and because its contribution was kept comparable to the task loss towards the end of training.

We observed that the more complex the task is and the more complex the network to train is, the more drastic is the difference in performance of different SG shapes. It is known that learning is possible with a wide variety of SG shapes \cite{zenke2021remarkable} and the community has not yet settled for one shape or one method to reliably choose which SG to use in each case \cite{surrogate2019}. We showed how to apply a well known stability principle to the forward and backward pass of the simplest Spiking Neural Network, the LIF, as a starting point, but we think that the principles of good Neuromorphic initialization can be further elaborated, in order to tackle more complex tasks and networks.





\appendix
    
% !TEX root = ./main.tex
\section{The cosmic ray intensity}
\label{app:intensity}

In this appendix, we introduce the various definitions used to describe CR density. 

The number of particles in a volume element $d^3 \vb{r}$ about $\vb r$ and in the momentum interval $d^3 \vb p$ about $\vb p$ is given by
%
\begin{equation}
d\rho = \phi(\vb r, \vb p, t) \, d^3 \vb r \, d^3 \vb p
\end{equation}
%
with $\phi$ the distribution function.

Expanding $d^3 \vb p$ in spherical coordinates gives:
%
\begin{equation}
d\rho  = \phi(\vb r, p, t) \, d^3 \vb r \, p^2 dp \, d\Omega
\end{equation}

Typically we are not able to measure $\phi$ but only averages over momentum space, thereby we conveniently introduce the \emph{phase-space distribution function} as:
%
\begin{equation}
f(\vb r, p, t) = \frac{1}{4\pi} \int_\Omega \phi(\vb r, p, t) d\Omega
\end{equation}

Correspondingly, the number of particles in $d^3 \vb r$ and in $dp$ (independent of direction of $\vb p$) is:
%
\begin{equation}
dn = \int_\Omega \! d\Omega \, F(\vb r, p, t) \, d^3 \vb r \, p^2 dp = 4 \pi p^2 f(\vb r, p, t) \, d^3 \vb r \, dp
\end{equation}
 
Consequently, when the spectral index of the CR phase distribution in momentum is represented by $\gamma$, the corresponding slope, when expressed in terms of energy, becomes $\gamma - 2$:
%
\begin{equation}
f \propto p^{-\gamma} \, \rightarrow \, n \propto E^{2-\gamma}
\end{equation}

Experimental measurements typically present their results in terms of the CR differential intensity, denoted as $I$. This quantity represents the number of particles collected per unit energy, unit area, unit time, and unit steradians.

Furthermore, it is convenient to describe the spectra of the different CR components as a function of the kinetic energy per nucleon, denoted as $T$. This choice is advantageous because the kinetic energy per nucleon is approximately conserved when a nucleus fragments due to interactions with the interstellar gas.

In terms of the phase-space distribution, the intensity of a specific nuclear species $\alpha$, can be expressed as 
%
\begin{equation}
I_\alpha (T) dT = c p^2 f_\alpha(p) \beta (p) dp 
\end{equation}
%
which can be further simplified as 
%
\begin{equation}
I_\alpha(T) = A p^2 f_\alpha(p)
\label{eq:f2I}
\end{equation}
%
where $A$ is the nuclear mass.

% !TEX root = ./main.tex
\section{Calculation of diffusion coefficient in quasilinear-theory}
\label{app:diffusioncoefficient}

In this appendix, we explore the interaction between charged particles and an astrophysical plasma to derive the spatial diffusion coefficient using the quasilinear theory (QLT). 
%
QLT allows to directly compute this coefficient and other transport parameters based on the previous knowledge of the turbulent spectra.
%
The quasilinear approximation can be seen as a first-order perturbation theory and here we follow standard derivations as in~\cite{Blandford1987pr,Shalchi2009book,Blasi2013aar}.

Recent developments on this subject are given in the P.~Blasi and A.~Marcowith lecture notes in this volume.

First, we consider the equations of motion for a particle traveling within an ordered magnetic field aligned with the $\hat{\vb z}$ axis, denoted as $\vb B_0 = B_0 \hat{\vb z}$. In the absence of a large-scale electric field, the particle's motion is described by the Lorentz force:
%
\begin{equation}
\frac{d \vb p}{dt} = \frac{q}{c} (\vb v \wedge \vb B_0)
\label{eq:lorentz}
\end{equation}

The Lorentz force acts perpendicular to the particle's motion, preserving the velocity's magnitude. By splitting the motion into its components, we obtain:
%
\begin{equation}
m \gamma \frac{d \vb v}{dt} = \frac{q}{c} (\vb v \wedge \vb B_0) \quad \rightarrow \quad 
\begin{cases}
m \gamma \frac{dv_x}{dt} = \frac{q}{c} v_y B_0 \\
m \gamma \frac{dv_y}{dt} = -\frac{q}{c} v_x B_0\\
\frac{dv_z}{dt} = 0
\end{cases}
\label{eq:lorentzcomponents}
\end{equation}

The last equation shows that $v_z = v_\| = v \cos \theta$ remains constant. Consequently, the pitch angle, defined as the cosine of the angle between the particle velocity and the magnetic field direction ($\mu = \cos \theta$), is a conserved quantity, as $dv_z/dt = v d\mu/dt = 0$.

Combining the first two equations yields two second-order differential equations:
%
\begin{equation}
\frac{d^2 v_{x,y}}{dt^2} = - \Omega^2 v_{x,y}
\end{equation}

Here, we introduce the Larmor frequency $\Omega = \frac{q B_0}{m \gamma c}$.

This equation can be easily solved as simple harmonic motion along the $\hat{\vb x}$ axis, where $v_{x} = v_{0,x} \cos (\Omega t)$. Using this solution, we can solve the system~\eqref{eq:lorentzcomponents} as follows:
%
\begin{equation}
\begin{cases}
v_x = v_{0,\perp} \cos (\phi - \Omega t) \\
v_y = - v_{0,\perp} \sin (\phi - \Omega t) \\
v_z = v_{0,\parallel}
\end{cases}
\rightarrow \, 
\begin{cases}
v_x = v_{0} (1 - \mu^2)^{\frac{1}{2}} \cos (\phi - \Omega t) \\
v_y = - v_{0} (1 - \mu^2)^{\frac{1}{2}} \sin (\phi - \Omega t) \\
v_z = v_{0} \mu
\end{cases}
\end{equation}

Here, $\phi$ is an arbitrary phase, $v_{0,\perp}$ represents the initial velocity of the particle in the $xy$-plane, given by $v_{0,\perp} = v_0 \sin \theta = v_0 (1-\mu^2)^{1/2}$.

The solution above represents a helical motion with a uniform drift along $\hat{\vb z}$, described by the equation of motion $z = v \mu t$.

Now we consider introducing a perturbation to the magnetic field with components $\delta \mathbf{B} \equiv (\delta {\rm B}_x, \delta {\rm B}_y, \delta {\rm B}_z)$, where $|\delta \mathbf{B}| \ll |\mathbf{B}_0|$. In this case, we assume a pure Alfvénic wave propagating along the background magnetic field, which implies $\delta {\rm B}_z = 0$ and the wave oscillates such that $\delta \mathbf{B} \perp \mathbf{k}$.

This allows us to express the system of equations~\eqref{eq:lorentz} as follows:
%
\begin{equation}
m \gamma \frac{d\vb v}{dt} = \frac{q}{c}
\left|
\begin{array}{ccc}
\hat {\vb x}  & \hat {\vb y}  & \hat {\vb z}  \\
v_x & v_y & v_z \\
\delta {\rm B}_x & \delta {\rm B}_y & {\rm B}_0 
\end{array}
\right|
\oset{\delta {\rm B} \ll {\rm B}_0} \simeq
\frac{q}{c}
\left(
\begin{array}{c}
v_y {\rm B}_0 \\
-v_x {\rm B}_0 \\
v_x \delta {\rm B}_y - v_y \delta {\rm B}_x
\end{array}
\right)
\end{equation}

As prescribed by QLT, we neglect the perturbation field in the $x$ and $y$ components. This implies that the circular orbits in the plane perpendicular to the background field are approximately unaffected. However, the perturbation does cause a change in the $z$ component of the velocity, leading to a modification in the pitch angle $\mu$ of the particle. It's important to note that the perturbation does not affect the particle's momentum value; we are describing the motion in the reference frame of the perturbation, where the only force acting on the particle is the Lorentz force. Consequently, while numerous pitch-angle changes can eventually reverse the parallel velocity of the particle, they cannot shift the guiding center of the orbits.

To examine the extent of this change, we focus on the last equation of the system mentioned above, which governs the \emph{perturbed} motion along $z$:
%
\begin{equation}
m \gamma \frac{dv_z}{dt} = \frac{q}{c} \left[v_x(t) \delta {\rm B}_y - v_y(t) \delta {\rm B}_x \right]
\end{equation}

As a consequence, the pitch angle changes with time according to:
%
\begin{equation}
m\gamma v \frac{d\mu}{dt} = \frac{q}{c}v_{0,\perp} \left[\cos(\phi-\Omega t) \delta {\rm B}_y - \sin(\phi - \Omega t) \delta {\rm B}_x\right]
\label{eq:pitchanglemotion}
\end{equation}

To proceed, we make the simplifying assumption that the perturbed field is circularly polarized, meaning the wave components have the same amplitude: $|\delta {\rm B}_{x}| = |\delta {\rm B}_{y}| = |\delta {\rm B}|$. Thus, we can express the perturbation as:
%
\begin{equation}
\begin{cases}
\delta {\rm B}_y = &  \delta {\rm B} \exp \left[ i (kz - \omega t) \right] \\ %= \cos(kz - \omega t + \psi) + i \sin ((kz - \omega t + \psi)\\
\delta {\rm B}_x = & \pm i \delta {\rm B}
\end{cases}
\end{equation}

Taking the real part gives:
%
\begin{equation}
\begin{cases}\delta {\rm B}_y = & \delta {\rm B} \cos (kz - \omega t) \\
\delta {\rm B}_x = & \mp\delta {\rm B} \sin (kz - \omega t) 
\end{cases}
\end{equation}
%
therefore, by substituting in equation~\eqref{eq:pitchanglemotion}, we find
%
\begin{equation}
m\gamma v \frac{d\mu}{dt} = 
\frac{q}{c}v_{0,\perp} \delta{\rm B} \left[\cos(\phi-\Omega t) \cos (kz - \omega t) \pm \sin(\phi - \Omega t) \sin (kz - \omega t)\right]
\end{equation}
%
which simplifies to\footnote{We use the trigonometric relation $\cos \alpha \cos \beta \pm \sin \alpha \sin \beta = \cos (\alpha \mp \beta)$}:
%
\begin{equation}
m\gamma v \frac{d\mu}{dt} = 
\frac{q}{c}v_{0,\perp} \delta {\rm B} \cos(\phi-\Omega t \mp kz \pm \omega t)
\end{equation}

For Alfvén waves, the dispersion relation is given by $\omega = k v_{\rm A}$, where $v_{\rm A}$ represents the Alfvén velocity. By comparing the spatial frequency with the temporal frequency in the argument of the cosine function, we can derive the following relation:
%
\begin{equation}
\frac{kz}{\omega t} \simeq \frac{k v \mu t}{k v_{\rm A} t} \sim \frac{v}{v_{\rm A}} \mu
\label{eq:kmomegat}
\end{equation}

Here, we utilize the fact that for an unperturbed orbit, the position $z$ of the particle is given by $z = v \mu t$.

Considering that $v$ is of the order of the speed of light, while $v_{\rm A}$ in the average ISM is approximately 10 km/s, we find that the ratio in equation~\eqref{eq:kmomegat} is significantly greater than 1 unless $\mu \ll v_{\rm A} / v$.

Consequently, we can neglect the term $\omega t$ in comparison to $kz$. This choice is equivalent to selecting a reference frame in which the waves appear stationary. In this frame, there is no electric field associated with the waves.

We can approximate the pitch angle equation of motion as
%
\begin{equation}
\frac{d\mu}{dt} \simeq \Omega
(1-\mu^2)^{\frac 1 2} \frac{\delta {\rm B}}{{\rm B}_0} \cos\left[\phi + (\Omega \pm k v \mu) t \right]
\end{equation}

The equation above implies a periodic variation in the pitch angle. When we integrate this equation over a sufficiently long time interval, the average of the integrated quantity becomes zero. This result is physically expected since the particle orbits are concentric circles.

However, if we instead consider the square of the pitch-angle variation: 
%
\begin{multline}
\langle \Delta \mu \Delta \mu \rangle = \int_0^{2\pi} \frac{d\phi}{2\pi} \int_0^{\Delta t} dt \frac{d\mu}{dt}(t) \, \int_0^{\Delta t} dt^\prime \frac{d\mu}{dt}(t^\prime) \\ = \Omega^2 (1 - \mu^2) \left( \frac{\delta {\rm B}}{{\rm B}_0} \right)^2 \int_{0}^{\Delta t} dt \int_{0}^{\Delta t} dt^\prime \, \cos[(\Omega \pm k v \mu) t] \cos[(\Omega \pm k v \mu) t^\prime]
\end{multline}

The integrand functions are even, so we can double the interval of the $dt^\prime$ integral as $\int_{-\Delta t}^{\Delta t} dt^\prime$ and add a factor of $\frac{1}{2}$. Additionally, as we are considering sufficiently large times to evaluate the effect of scattering ($\Delta t \gg t, t^\prime$), the same interval can be approximated as $\int_{-\infty}^{\infty} dt^\prime$.

Therefore, we have\cite{Shalchi2009book}:
%
\begin{multline}
\langle \Delta \mu \Delta \mu \rangle = \\ \Omega^2 \frac{(1 - \mu^2)}{2} \left( \frac{\delta {\rm B}}{{\rm B}_0} \right)^2 \int_{0}^{\Delta t} \!\! dt \, {\rm Re} \{ \exp[i(\Omega \pm k v \mu) t] \} \, \int_{-\infty}^{\infty} \!\! dt^\prime \,  {\rm Re}\{\exp[i(\Omega \pm k v \mu) t^\prime]\}
\end{multline}
%
and solve the integral on $t^\prime$\footnote{We use the property $\delta(x-a) = \frac{1}{2\pi}\int_{-\infty}^\infty dy {\rm e}^{iy(x-a)}$}, as to obtain:
%
\begin{equation}
\langle \Delta \mu \Delta \mu \rangle = \Omega^2 \frac{(1 - \mu^2)}{2} \left( \frac{\delta {\rm B}}{{\rm B}_0} \right)^2 \int_{0}^{\Delta t} \!\! dt \, {\rm Re} \{ \exp[i(\Omega \pm k v \mu) t] \} \, 2\pi \delta (\Omega \pm k v \mu)
\end{equation}

The motion of the pitch angle, having null mean and variance different from zero, is typical of a diffusive process with diffusion coefficient $D_{\mu\mu}$, which represents the average rate of change of the square of the pitch angle over the time interval $\Delta t$.

Now the second integral, because of the presence of the~\emph{delta} function, gives just a factor $\int_{0}^{\Delta t} dt = \Delta t$, and we can write:
%
\begin{equation}
D_{\mu\mu} \equiv \left\langle \frac{\Delta \mu \Delta \mu}{\Delta t} \right\rangle = \Omega^2 \left( \frac{\delta {\rm B}}{{\rm B}_0} \right)^2 (1 - \mu^2) \, \pi \delta(\Omega \pm k v_{\parallel})
\label{eq:dmumuvar}
\end{equation}

In general, one must consider a packet of turbulent waves with energy distribution per wave number denoted as $W(k) dk$. This distribution represents the energy density contained within the range of wavenumbers $[k, k + dk]$ and is normalized to the energy density of the background magnetic field, $\frac{B_0^2}{8\pi}$. Specifically:
%
\begin{equation}
\left( \frac{\delta {\rm B}(k)}{{\rm B}_0} \right)^2 = W(k) dk.
\end{equation}

By incorporating this consideration, we can extend equation~\eqref{eq:dmumuvar} to obtain:
%
\begin{equation}
D_{\mu \mu} \equiv \left\langle \frac{\Delta \mu \Delta \mu}{\Delta t} \right\rangle = \Omega^2 (1 - \mu^2) \pi \int dk \, W(k) \delta(\Omega \pm k v_{\parallel}) \, .
\end{equation}

Introducing the resonant wavenumber $k_{\rm res}$, defined as the inverse of the Larmor radius $k_{\rm res} = r_{\rm L}^{-1} = \Omega/ v_\|$, we can express $D_{\mu \mu}$ as follows\footnote{We use the property $\int dx \delta (c x) = \frac{1}{|c|} \int dx \delta (x)$}:
%
\begin{equation} 
D_{\mu \mu} = \Omega (1 - \mu^2) \pi k_{\mathrm{res}} \int dk \, W(k) \delta(k \pm k_{\mathrm{res}}) = \Omega (1 - \mu^2) \pi k_{\mathrm{res}} W(k_{\rm res})
\end{equation}

These equations reveal that a wave-particle interaction is only possible when the inverse Larmor radius of the particle matches (i.e., is resonant) with the wavenumber of the turbulent wave. This type of process is commonly referred to as \emph{gyroresonant} scattering\footnote{It is worth noting that the QLT is consistently inadequate when attempting to describe pitch-angle diffusion at 90 degrees ($\mu = 0$) and reversing direction becomes a consideration. To address these and other limitations, several Nonlinear Theories have been formulated and developed.}.

The typical diffusion time, defined as the timescale to invert the pitch angle by about one radian is
%
\begin{equation}
\tau_{\rm diff} \simeq \frac{1}{D_{\theta\theta}} = \frac{1-\mu^2}{D_{\mu\mu}} = \frac{1}{\pi \Omega k_{\rm res} W(k_{\rm res})}
\end{equation}
%
where $D_{\theta\theta}$ is the diffusion coefficient in angle.

As in a diffusion timescale the particle moves by a distance of about $\Delta z = v \tau_{\rm diff}$, the spatial diffusion coefficient coefficient can be roughly estimated as
%
\begin{equation}
D_{zz} = v (v \tau_{\rm diff}) = \frac{v^2}{\pi \Omega k_{\rm res} W(k_{\rm res})} \simeq \frac{1}{3} r_{\rm L} v \frac{1}{k_{\rm res} W(k_{\rm res})} 
\end{equation}
%
where we have made the approximation $\pi \sim 3$. This informs us that the spatial diffusion coefficient is always much larger than the Bohm diffusion ($D_{\rm B} = \frac{1}{3} r_{\rm L} v$) since $k_{\rm res} W(k_{\rm res}) \ll 1$ at the relevant scales.
% !TEX root = ./main.tex
\section{The weighted slab technique}
\label{app:wsbla}

At present, the diffusion-losses model with the potential inclusion of advection is considered as the most suitable description for CR transport in the Galaxy at energies below approximately $10^{15}$ eV. Therefore, in this appendix we present the generalization of the transport model described in section~\ref{sec:nuclei} by including advection and ionization energy losses. 
%
This model is referred to as the \emph{weighted slab model}~\cite{Ptuskin1996apj,Jones2001apj}.

The distribution function $f_\alpha(z, p)$, which characterizes a stable species $\alpha$ subject to spallation and ionization losses, follows the transport equation:
%
\begin{multline}
-\frac{\partial}{\partial z} \left[ D_{\alpha}(p) \frac{\partial f_\alpha}{\partial z} \right]
+ u_z(z) \frac{\partial f_\alpha}{\partial z}
- \frac{du_z}{dz} \frac{p}{3} \frac{\partial f_\alpha}{\partial z}
+ \frac{f_\alpha}{\tau_{\rm sp,\alpha}}
+ \frac{1}{p^2} \frac{\partial}{\partial p}(p^2 \dot{p}_\alpha f_\alpha ) \\
= Q_\alpha(p,z) + \sum_{\alpha'>\alpha} \frac{f_{\alpha'}}{\tau_{\rm sp, \alpha'}}
\label{eq:fulltransport}
\end{multline}

Here, $D_\alpha(p, z)$ represents the diffusion coefficient of the particle, $u(z)$ is the advection velocity along the $z$-direction, and the term proportional to $du_z/dz$ accounts for adiabatic energy losses. The timescale for spallation is denoted as $\tau_{\rm sp}$, and $\dot{p} = dp/dt < 0$ captures ionization losses. The right-hand side of the equation includes the primary source term $Q_\alpha(p,z)$ and the contribution from the fragmentation of heavier nuclei.

Due to the coupling induced by nuclear fragmentation, to model CR propagation, it is required to solve a set of approximately $\sim$~100 coupled transport equations for all isotopes involved in the nuclear chain, accounting for all nuclear fragmentation yields.

In the one-dimensional slab model, a thin disk of matter represents the CR sources, $Q_\alpha(p,z) = 2 h \delta(z) Q_{\alpha, 0}(p)$, while an extended halo corresponds to the region where CRs diffuse with a diffusion coefficient that depends on rigidity.
%
In this model, we also make the assumption that $D$ and $u$ remain constant in the halo, and the advection velocity changes sign above and below the plane. Specifically, $u(z) = u_0 \left[2\Theta(z)-1\right]$, and consequently its derivative reads $\frac{du}{dz} = 2 u_0 \delta(z)$.

The majority of matter concentrates within the disk, characterized by a density $n_{\rm d}$, while the halo's density, $n_{\rm H}$, is considered negligible. Therefore, spallation and ionization losses only occur within the thin disk, resulting in $\dot{p}_\alpha(p,z) = 2 h \delta(z) \dot{p}_{0,\alpha}(p)$. It is important to note that this approximation is valid when $n_{\rm H} H \lesssim n_{\rm d} h$. If this condition is not satisfied, spallation in the halo can no longer be neglected.

To incorporate the presence of helium as a target in the ISM into the spallation term, we write the term for total inelastic fragmentation as:
%
\begin{equation}
\frac{f}{\tau_{\rm sp}} = v [n_{\rm H} \sigma_{\rm H} + n_{\rm He} \sigma_{\rm He}] f = v n_{\rm H} \sigma_{\rm H} [1 + f_{\rm He} \Sigma_{\rm He}] f
\end{equation}

Here, $f_{\rm He} = n_{\rm He} / n_{\rm H} \simeq 0.08$ represents the number density fraction of helium in the ISM, and $\Sigma_{\rm He} = \sigma_{\rm He} / \sigma_{\rm H}$ is the ratio between the inelastic cross-sections on helium and on hydrogen (with $\sigma_{\rm H}$ and $\sigma_{\rm He}$ being the respective cross-sections).

By expressing the interstellar gas density as:
%
\begin{equation}
\rho_{\rm ISM} = m_p n_{\rm H} + m_{\rm He} n_{\rm He} = m_p n_{\rm H} [ 1 + 4 f_{\rm He}]
\end{equation}
%
we obtain:
%
\begin{equation}
\frac{f}{\tau_{\rm sp}} = v \rho \frac{\sigma_{\rm H}}{m_p} \frac{[1 + f_{\rm He} \Sigma_{\rm He}]}{[1 + 4 f_{\rm He}]} f = \rho \frac{v \sigma_\alpha}{m} f = \mu \delta(z) \frac{v \sigma_\alpha}{m} f
\end{equation}
%
where $\mu = 2.3$ g cm$^{-2}$ represents the ISM surface density, and we define:
% 
\begin{equation}
\sigma_\alpha \equiv \sigma_{\rm H} \frac{1 + f_{\rm He} \Sigma_{\rm He}}{1 + f_{\rm He}} \, , \,\,\, m \equiv m_p \frac{1 + 4 f_{\rm He}}{1 + f_{\rm He}} 
\end{equation}

Adopting the assumptions described earlier, we can derive the transport equation as follows.
%
The transport equation in \ref{eq:fulltransport} takes the form:
%
\begin{multline}
-\frac{\partial}{\partial z} \left[ D_{\alpha} \frac{\partial f_\alpha}{\partial z} \right]
+ u_0 [2 \Theta(z) - 1] \frac{\partial f_\alpha}{\partial z}
- \frac{2}{3} u_0 \delta(z) p \frac{\partial f_\alpha}{\partial p}
+ \frac{\mu v \sigma_\alpha}{m} \delta(z) f_\alpha
+ \frac{2 h}{p^2} \frac{\partial}{\partial p}(p^2 \dot{p}_0 f_\alpha ) \delta(z) \\
= 2h q_{0,\alpha} \delta(z) 
+ \sum_{\alpha'>\alpha} \frac{\mu v \sigma_{\alpha'\rightarrow\alpha}}{m} \delta(z) f_{\alpha'}
\label{eq:fulltransport2}
\end{multline}

To find a solution for this equation, we focus on the region outside the disc ($z \ne 0$), where the equation can be simplified as:
%
\begin{equation}
-D_{\alpha} \frac{\partial^2 f_\alpha}{\partial z^2} 
+ u_0 \frac{\partial f_\alpha}{\partial z} = 0
\end{equation}

We assume the solution takes the form $f = A {\rm e}^{r_+ z} + B {\rm e}^{r_- z}$, where $r_\pm$ are the roots of the equation $-D_{\alpha} r^2 + u_0 r = 0$, and we find that $r_+ = 0$ and $r_- = u_0 / D_\alpha$.

Using the boundary conditions $f(z=0) = f_0$ and $f(z=H) = 0$, we determine the coefficients $A$ and $B$ as:
%
\begin{equation}
A = f_0 \frac{{\rm e}^\xi}{{\rm e}^\xi - 1} \,\,\, , \,\,\, B = f_0 \frac{1}{1-{\rm e}^\xi}
\end{equation}

Here, $\xi = \frac{u_0 H}{D_\alpha}$, and the spatial distribution of the solution in the halo becomes:
%
\begin{equation}
f(z) = f_0 \frac{1-{\exp}\left[-\xi(1-\frac{|z|}{H})\right]}{1- \exp(-\xi)}
\end{equation}
%
where $z$ is taken with the absolute value as the solution is symmetric above and below the disc.

The derivative of $f_\alpha$ evaluated at the disc location can be expressed as:
%
\begin{equation}
\left.\frac{\partial f}{\partial z}\right|_0 = -\frac{f_0}{H} \frac{\xi}{\exp (\xi) - 1}
\label{eq:fgradient}
\end{equation}
%
which is proportional to the average gradient $\sim f_0/H$.

Next, we integrate equation~\ref{eq:fulltransport2} in an infinitesimal height around the disc, resulting in:
%
\begin{multline}
-2 D_{\alpha} \frac{\partial f_{0,\alpha}}{\partial z} 
- \frac{2}{3} u_0 p \frac{\partial f_{0,\alpha}}{\partial p}
+ \frac{\mu v \sigma_\alpha}{m} f_{0,\alpha}
+ \frac{2 h}{p^2} \frac{\partial}{\partial p}(p^2 \dot{p}_{0,\alpha} f_{0,\alpha} ) \\
= 2h q_{0,\alpha} 
+ \sum_{\alpha'>\alpha} \frac{\mu v \sigma_{\alpha'\rightarrow\alpha}}{m}  f_{0,\alpha'}
\label{eq:fulltransportabovebelow}
\end{multline}

Here the symmetry of the problem above and below the disc, namely $f_\alpha(z, p) = f_\alpha(-z, p)$, leads to the factor of two in front of the diffusion term and lets the advection term disappear.

Using the spatial derivative of $f_\alpha$ (equation~\ref{eq:fgradient}) reduces the equation to a first-order differential equation for $f_{0,\alpha}(p)$:
%
\begin{multline}
\frac{2 u_0}{\mu v} \frac{1}{{\rm e}^\xi - 1} f_{0,\alpha}
- \frac{2 u_0}{3 \mu v}  p \frac{\partial f_{0,\alpha}}{\partial p}
+ \frac{2 h}{\mu v p^2} \frac{\partial}{\partial p}(p^2 \dot{p}_{0,\alpha} f_{0,\alpha} ) 
+ \frac{\sigma_\alpha}{m} f_{0,\alpha} \\
= \frac{2h}{\mu v} q_{0,\alpha} 
+ \sum_{\alpha'>\alpha} \frac{\sigma_{\alpha'\rightarrow\alpha}}{m}  f_{0,\alpha'}
\end{multline}

Multiplying both sides by $A p^2$ we express the equation in terms of the intensity as a function of the kinetic energy per nucleon $E$, which relates to $f_\alpha$ as described in equation~\ref{eq:f2I}: 
%
\begin{multline}
\frac{2 u_0}{\mu v} \frac{1}{{\rm e}^\xi - 1} I_\alpha
- \frac{2 u_0}{3 \mu v} p^3 \frac{\partial}{\partial p} \frac{I_\alpha}{p^2}
+ \frac{2 h}{\mu v} \frac{\partial}{\partial p}(\dot{p}_0 I_\alpha) 
+ \frac{\sigma_\alpha}{m} I_\alpha \\
= \frac{2h A p^2}{\mu v} q_{0,\alpha} 
+ \sum_{\alpha'>\alpha} \frac{\sigma_{\alpha'\rightarrow\alpha}}{m}  I_{\alpha'}
\end{multline}
%
where $A$ is the nuclear mass number.

To simplify the equation, we add and subtract the term $\frac{2u_0}{\mu v} I_\alpha$, resulting in:
%
\begin{multline}
\frac{2 u_0}{\mu v} \frac{1}{1-{\rm e}^{-\xi}} I_\alpha
- \frac{2 u_0}{3 \mu v} I_\alpha 
- \frac{2 u_0}{3 \mu v} p \frac{\partial I_\alpha}{\partial p} 
+ \frac{2 h}{\mu v} \frac{\partial}{\partial p}(\dot{p}_{0,\alpha} I_\alpha) 
+ \frac{\sigma_\alpha}{m} I_\alpha \\
= \frac{2h A p^2}{\mu v} q_{0,\alpha} 
+ \sum_{\alpha'>\alpha} \frac{\sigma_{\alpha'\rightarrow\alpha}}{m}  I_{\alpha'}
\end{multline}

This equation can be rewritten conveniently as:
%
\begin{multline}
\frac{I_\alpha(E)}{\rchi_\alpha}
+ \frac{d}{dE} \left( \left[ \left( \frac{dE}{d\rchi}\right)_{\rm ad} + \left( \frac{dE}{d\rchi} \right)_{\rm ion, \alpha} \right] I_\alpha(E) \right)
+ \frac{I_\alpha(E)}{\rchi_{\rm cr, \alpha}} \\
= \frac{2h A p^2}{\mu v} q_{0,\alpha} 
+ \sum_{\alpha'>\alpha} \frac{\sigma_{\alpha'\rightarrow\alpha}}{m}  I_{\alpha'}
\label{eq:fulltransport3}
\end{multline}

Here, we introduce the effective grammage:
%
\begin{equation}
\rchi_\alpha = \frac{\mu v}{2 u_0} \left[ 1 - \exp \left(-\frac{u_0 H}{D_\alpha} \right) \right]
\end{equation}
%
and the term\footnote{We use $\partial p/\partial E = A / c \beta$}:
%
\begin{equation}
 \left( \frac{dE}{d\rchi}\right)_{\rm ad} + \left( \frac{dE}{d\rchi} \right)_{\rm ion, \alpha} = 
 -\frac{2 u_0}{3 \mu c} \sqrt{E^2 + 2 m_p c^2 T} - \frac{2 h}{\mu A} |\dot p_{0,\alpha}|
\end{equation}

Notice that in the limit of $H^2 / D_\alpha \ll H / u_0$, the grammage reduces to the expression derived in the diffusion scenario in equation~\ref{eq:grammage}.

We observe that equation~\ref{eq:fulltransport3} can be written in the form:
%
\begin{equation}
\Lambda_{1,\alpha} I_\alpha(E) + \Lambda_{2, \alpha}(E) \partial_E I_\alpha(E) = Q_\alpha(E)
\end{equation}
%
which has a formal solution given by:
%
\begin{equation}
I_\alpha(E) = \int_E dE^\prime \frac{Q_\alpha(E^\prime)}{|\Lambda_{2,\alpha}(E^\prime)|} \exp\left[ -\int_E^{E^\prime} dE^{\prime\prime} \frac{\Lambda_{1,\alpha}(E^{\prime\prime})}{|\Lambda_{2,\alpha}(E^{\prime\prime})|} \right]
\end{equation}

In the more general case, this expression has to be solved numerically to derive the local interstellar spectrum for all isotopes in cosmic radiation\footnote{A practical application of this approach is implemented in the CRAMS code available at \url{https://github.com/carmeloevoli/crams}}.

Finally, we highlight that this formalism can be straightforwardly extended to model the transport of unstable nuclei, consider the effects of a finite thick disc $h$, or account for changes in the secondary production due to the grammage accumulated in the halo (see, e.g.,~\cite{Morlino2020prd,Evoli2020prd}).

% !TEX root = ./main.tex
\section{Green's functions of the transport equations}
\label{sec:appgreen}

\subsection{Green's function of the pure diffusive equation}

To construct the Green's function, we start by taking the Fourier transform with respect to the spatial variable $\vb r$, using the property that $\mathcal F[\delta^{(3)}(\vb r - \vb r_\star)] = {\rm e}^{-i \vb k \cdot \vb r_\star}$ :
%
\begin{equation}
\frac{\partial}{\partial t} \tilde{\mathcal G}(\vb k, t \leftarrow \vb r_\star, t_\star) 
+ D k^2 \tilde{\mathcal G}(\vb k, t \leftarrow \vb r_\star, t_\star) 
= {\rm e}^{-i \vb k \cdot \vb r_\star} \delta(t - t_\star)
\end{equation}
%
subject to the initial condition $\tilde{\mathcal G}(\vb k, 0 \leftarrow \vb r_\star, t_\star) = 0$.

After multiplying both sides by ${\rm e}^{D k^2 t}$, we obtain
%
\begin{equation}
\frac{\partial}{\partial t} \left[ {\rm e}^{D k^2 t} \tilde{\mathcal G}(\vb k, t \leftarrow \vb r_\star, t_\star) \right]
= {\rm e}^{-i \vb k \cdot \vb r_\star + D k^2 t} \, \delta(t - t_\star)
\end{equation}

This equation can be easily solved, leading to the Fourier transform of the Green's function:
%
\begin{equation}
\tilde{\mathcal G}(\vb k, t) = 
{\rm e}^{-i \vb k \cdot \vb r_\star - D k^2 t} 
\int_0^t {\rm e}^{D k^2 t^\prime} \delta(t^\prime-t_\star) dt^\prime
= \begin{cases}
0 & t < t_\star \\
{\rm e}^{- i \vb k \cdot \vb r_\star - D k^2 (t-t_\star)} & t > t_\star
\end{cases}
\end{equation}

The upper limit $t$ of this integral expresses \emph{causality}: the solution at time $t$ depends only on causes lying in its past, i.e., $t_\star < t$.

Finally, by taking the inverse Fourier transform with respect to $\vb k$, we obtain the Green's function we are looking for:
%
\begin{equation}
\mathcal G (\vb r, t \leftarrow \vb r_\star, t_\star) 
= \frac{\Theta(t-t_\star)}{(2\pi)^3} 
\int {\rm e}^{i \vb k \cdot (\vb r - \vb r_\star)} {\rm e}^{-Dk^2 (t-t_\star)} d^3 \vb k
\end{equation}

Recognizing the integral as the (inverse) Fourier transform of a Gaussian, we find:
%
\begin{equation} 
\mathcal G_{\rm free} (\vb r, t \leftarrow \vb r_\star, t_\star) 
= \frac{\Theta(\tau)}{(4\pi D \tau)^{3/2}} 
{\rm e}^{-\frac{d^2}{4 D \tau}}
\end{equation}
%
where $\tau = t-t_\star$ and $\vb d = \vb r - \vb r_\star$.

Thus, an initial Gaussian distribution retains its Gaussian form, with its squared width spreading linearly with time. This linear growth of variance is characteristic of \emph{diffusing} probabilistic processes.

One notable property of this solution is that $\mathcal G > 0$ everywhere for any finite $t > 0$, regardless of how small $t$ is. However, this violates Special Relativity, and it is often resolved by replacing the theta argument with $\Theta(c \Delta t - d)$.

It's worth mentioning that the halo size parameter $H$ does not appear in the Green's function, as we derived the free-space Green's function without imposing any boundary condition in the $z$ direction.

To enforce the desired boundary conditions, we introduce a set of image charges with coordinates~\cite{Cowsik1979apj,Baltz1998prd,Mertsch2011jcap}:
%
\begin{equation}
\vb r_{\star,n}^\prime = 
\left(\begin{array}{c}
x_\star\\
y_\star\\
2 H n + (-1)^n z_\star 
\end{array}\right)
\end{equation}
%
as a consequence, the Green's function associated with these image charges becomes
%
\begin{equation}
\mathcal G_{\rm H} (\vb r, t \leftarrow \vb r_\star, t_\star) = \sum_{n=-\infty}^{\infty} (-1)^n \mathcal G_{\rm free} (\vb r, t \leftarrow \vb r_{\star,n}^\prime, t_\star)  
\end{equation}

It is easy to check that $\mathcal G_{\rm H}(x, y, {z = \pm {\textrm H}}, t\leftarrow \vb r_\star, t_\star) = 0$, satisfying the desired boundary conditions.

\subsection{Green's function of the diffusion-losses equation}

The equation of interest for leptons, assuming steady-state, is given by:
%
\begin{equation}
%\cancelto{0}{\frac{\partial I_e}{\partial t}} 
- D(E) \nabla^2 n_e(E)
- \frac{\partial}{\partial E} \left[ b(E) n_e \right] 
= Q(\vb r, E, t)
\end{equation}
%
where $b(E) = dE/dt$ represents the energy losses.

It is convenient to introduce a new variable:
%
\begin{equation}
\tilde t = 4 \int_E^\infty \frac{D(E^\prime)}{b(E^\prime)} dE^\prime
\end{equation}
%
which allows us to rewrite the derivative as:
%
\begin{equation}
\frac{d}{d\tilde t} = - \frac{b(E)}{4 D(E)} \frac{d}{dE}
\end{equation}

Using this new variable, we obtain the equation:
%
\begin{equation}
- D(E) \nabla^2 n_e(E)
+ 4 \frac{D(E)}{b(E)}\frac{\partial}{\partial \tilde t} \left[ b(E) n_e(E) \right] 
= Q(\vb r, E, t)
\end{equation} 

By rearranging the terms, we arrive at the diffusion equation for the new variable $\tilde{n} = b(E) n(E)$:
%
\begin{equation}
\frac{\partial}{\partial \tilde t} \left[ \tilde n_e(E) \right] 
- \frac{1}{4} \nabla^2 \left[ \tilde n_e(E) \right]
= \tilde Q(\vb r, E, t)
\end{equation} 

Next, we consider the Green's function in the new variables, introducing $\lambda^2 = \tilde{t} - \tilde{t}_\star$, which is given by:
%
\begin{equation}
\tilde{\mathcal G} (\vb r, \tilde t \leftarrow \vb r_\star, \tilde t_\star) 
= \frac{\Theta(\lambda^2)}{(\pi \lambda^2)^{3/2}} 
{\rm e}^{-\frac{d^2}{\lambda^2}}
\end{equation}

The Green's function for the time-dependent solution can be expressed in terms of the steady-state solution as:
%
\begin{equation}
{\mathcal G} (\vb r, t, E \leftarrow \vb r_\star, t_\star, E_\star) 
= \delta(\Delta t - \tau) \frac{\tilde{\mathcal G} (\vb r, \tilde t \leftarrow \vb r_\star, \tilde t_\star)}{|b(E)|} 
= \frac{1}{|b(E)|}\frac{\delta(\Delta t - \tau)}{(\pi \lambda^2)^{3/2}} {\rm e}^{-\frac{d^2}{\lambda^2}}
\end{equation}
%
where $\lambda^2(E, E_\star) = 4 \int_{E}^{E_\star} dE^\prime \frac{D(E^\prime)}{b(E^\prime)}$ represents the propagation scale (also known as the Syrovatskii variable), and $\tau(E, E_\star) = \int_E^{E_\star} \frac{dE^\prime}{b(E^\prime)}$ is the loss time, which corresponds to the average time during which the energy of a particle decreases from $E_\star$ to $E$ due to losses.

Therefore, the particles we observe with energy $E$ have been injected with energy $\tilde{E}$ at a time $\Delta t$ that satisfies $\tau(E, \tilde{E}) = \Delta t$.
 
This condition sets a maximum energy $E_{\rm max}$ as $\tau(E_{\rm max}, \infty) = t_{\rm age}$, which in the Thomson limit, is given by:
%
\begin{equation}
E_{\rm max} = \frac{E_0^2}{b_0 t_{\rm age}} \simeq 400~{\rm GeV} \left( \frac{t_{\rm age}}{\rm Myr} \right)^{-1}
\end{equation}

This result provides the maximum energy of observed particles based on the age of the source.


\acknowledgments

We are grateful to P.~Blasi for providing the inspiration for these lecture notes, which stem from the course taught effortlessly over the past decade at GSSI.

We extend our sincere appreciation to our experimental colleagues, whose tireless efforts have consistently pushed the boundaries of knowledge and understanding in the subject matter. 

We would also like to acknowledge the usage of the CRDB~\cite{Maurin2023} database of CR measurements. 

Finally, CE would like to extend his heartfelt thanks to the Italian Physical Society for granting us the opportunity to host this School at the enchanting and cozy Villa Monastero in \emph{Varenna}.
%
%The serene setting, nestled against the beautiful backdrop of Como Lake, will forever hold a cherished place in his memories.

\bibliography{2022-varenna.bib}
\bibliographystyle{varenna}

\end{document}
%%
%% End of file `cimsmple.tex'.
