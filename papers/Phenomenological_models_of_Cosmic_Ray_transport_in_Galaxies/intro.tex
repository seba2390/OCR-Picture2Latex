% !TEX root = ./main.tex
%%%%%%%%%% SECTION %%%%%%%%%
\section{Introduction: the grammage pillar}
\label{sec:intro}

In figure~\ref{fig:composition}, we compare the isotope composition in the local cosmic ray (CR) flux against the composition in the ambient gas surrounding the solar system (detected, e.g., from CI-chondrites and solar photospheric measurements~\cite{Looders2009}). While the two compositions appear quite similar overall, suggesting that most CRs are accelerated from the average interstellar medium (ISM), intriguing differences can be observed.
%
Notably, elements like Lithium, Beryllium, and Boron in this plot closely resemble the abundance of Carbon or Oxygen in the CR flux, despite being negligibly present in the ISM on average. This feature is also confirmed for elements below Iron (Sc, Ti, and V), historically known as Sub-Iron elements, as well as for Nitrogen and Fluorine.

The striking separation between elements like Carbon and elements like Boron stands out. If we assume that the environment around the Sun is a typical representation of a Galactic star, reflecting a combination of stellar nucleosynthesis and pollution processes, and that some process within this environment is responsible for accelerating the surrounding elements, we would expect the relative ratio of CR abundances to match the interstellar one.
%
However, as our observations reveal, this is not the case. Thus, it indicates the presence of a second component, formed during propagation (hence termed \emph{secondary}), most likely resulting from the fragmentation of heavier CR nuclei into lighter ones during interactions with the ISM targets.

\begin{figure}
\centering
\includegraphics[width=0.65\textwidth]{figures/composition_CRIS.pdf}    
\caption{Abundances for Galactic cosmic rays (blue bars) and solar system elements (red circles) normalized to carbon atoms. Galactic cosmic ray abundances are a composite from~\cite{Young1981apj,AMS02results}. The solar system abundances are from~\cite{Looders2009}.}
\label{fig:composition}
\end{figure}

This secondary component provides relevant insights into the travel history of parent nuclei as they propagate through the ISM before reaching us.
%
As a matter of fact, it serves as one of the most compelling pieces of evidence supporting the idea of CR \emph{diffusive} propagation in our Galaxy.

To clarify this point, let's consider an extremely simplified scenario with only one primary species $n_p$ and one secondary species $n_s$. Their abundances can be described along the trajectory $s$ parameterized as \emph{grammage}, denoting the amount of material traversed by the particle during propagation\footnote{Analogous to the concept of ``column density'' in astrophysics}. 
%
In fact, the grammage, denoted by $\rchi$, is defined as the integral of the ISM density $\rho$ along $s$: $\rchi = \int \! ds \, \rho(s)$.

After an average interaction length $\lambda$, the CR nucleus (either primary, p, or secondary, s) undergoes an inelastic scattering with ISM targets, defined in terms of the inelastic cross-section $\sigma$ as $\lambda_{p(s)} = m / \sigma_{p(s)}$, where $m \sim 1.4~m_p$ is the mean target mass.

The evolution of the primary component involves only the disappearance of nuclei due to this effect. However, for the secondary component, we must consider production from the break-up of the primary component:
%
\begin{equation}
\begin{aligned}\frac{dn_p}{d\chi} & = - \frac{n_p}{\lambda_p} \\
\frac{dn_s}{d\chi} & = - \frac{n_s}{\lambda_s} + P_{p\rightarrow s} \frac{n_p}{\lambda_p}
\end{aligned}
\end{equation}
%
where $P_{p\rightarrow s}$ represents the spallation probability for producing $s$ from $p$ fragmentation.

By assuming that the system contains only primary nuclei for $\rchi = 0$ (initial condition), we can derive how the species ratio, $n_s / n_p$, depends on $\rchi$:
%
\begin{equation}
\frac{n_s}{n_p} = P_{p\rightarrow s} \frac{\lambda_s}{\lambda_s - \lambda_p} \left[ \exp\left( -\frac{\rchi}{\lambda_s} + \frac{\rchi}{\lambda_p} \right) - 1 \right]
\label{eq:chievolution}
\end{equation}

Laboratory experiments yield scattering lengths of about $\lambda_{\rm C} \sim 9.1$~g/cm$^2$ for primaries and $\lambda_{\rm B} \sim 10.4$~g/cm$^2$ for secondary nuclei. While the fragmentation probability is more uncertain, we assume a value of $P_{\textrm{C} \rightarrow \textrm{B}} \sim 0.25$~\cite{Evoli2019prd} for our purposes.

In figure~\ref{fig:grammage10gev}, we compare the ratio calculated using equation~\eqref{eq:chievolution} with the most recent measurements of the secondary-over-primary ratio by AMS-02~\cite{AMS02libeb} at a given rigidity $R = p/Z$. 
%
The comparison reveals the canonical \emph{$\mathcal O(10)$~grams per centimeters squared of traversed material for $\mathcal O(10)$ GV CRs}, a widely known fact in CR physics.

\begin{figure}
\centering
\includegraphics[width=0.6\textwidth]{figures/grammage_pillar.pdf}    
\caption{The secondary-over-primary ratio as a function of the grammage as given by equation~\eqref{eq:chievolution}. The value measured by AMS-02 at 10 GV is also shown with a dashed blue line~\cite{AMS02results}.}
\label{fig:grammage10gev}
\end{figure}

The gaseous component of the Milky Way is concentrated in a relatively thin disk with a half-thickness of approximately $h \sim 100$~pc. This disk has an average surface density of interstellar material, denoted as $\mu_d$, which is estimated to be around $\mu_d \sim 2.3 \times 10^{-3}$ g/cm$^{-2}$~\cite{Ferriere2001rmp}.

Given that the Galactic plane contains a significant portion of the GeV $\gamma$-ray emission, predominantly produced by the interactions of CR primaries with the interstellar material~\cite{Tibaldo2021universe}, it is reasonable to assume that the majority of the grammage accumulation occurs in this region.

However, if we consider a single crossing of the Galactic disk, the grammage accumulated is only roughly $\mu_d$. This value is clearly inconsistent with the estimation obtained from CR measurements. The discrepancy indicates that additional processes or regions contribute significantly to the total grammage encountered by CRs as they propagate through the Galaxy. 

The prevailing hypothesis is that particles cross the disk many times, and we can eventually estimate the minimum time that this primary component spends in the gas region in order to accumulate the required grammage as: 
%
\begin{equation}
\tau_{h} \gtrsim \frac{\chi_{\rm B/C}}{\mu_{\rm d}} \frac{2 h}{v} \sim 3 \times 10^6 \, \text{years}
\end{equation}

This timescale is clearly a factor $\frac{\chi_{\rm B/C}}{\mu_{\rm d}}$ larger than the typical time spent by a relativistic particle crossing straightly the disk which is $2h/c \sim 700$~years. 

The actual age of CRs in the Galaxy can be inferred by the relative abundances of radioactive nuclei. 
%
The measured abundances of CR clocks indicate a mean residence time for $\sim 10$~GV CR particles of about $\tau_{\rm H} \sim 6 \times 10^7$~yr\footnote{At this point, we kindly plead the reader's trust, as the justification for this value will be provided in \S\ref{sec:unstable}.}, which implies that CRs have to spend most of their time in low-density regions, with average density {$\bar n \lesssim (\mu_{\rm d} / 2h m) (\tau_{h}/\tau_{H}) \sim 0.1$~cm$^{-3}$}, not to overshoot the observed grammage.

To recap, the simple observation that the observed composition of CRs is different from that of solar elements in that rare solar-system nuclei such as B has provided compelling evidence that there is a process efficiently confining CRs within the Galaxy, enabling them to return multiple times from a low-density ``halo'' to the Galactic disk. 

The goal of these lecture notes is to develop a simplistic yet effective galactic model that can account for this fundamental observation and subsequently extract crucial physical quantities. 
%
In pursuit of this goal, we will discuss the foundational assumptions inherent to this class of models, while also outlining the consequential astrophysical implications that may capture also the interest of researchers in plasma and nuclear astrophysics.

These notes are intended for nonspecialists, and as such, the mathematical complexity of the model remains approachable. However, a broad spectrum of background knowledge is necessary, drawing from undergraduate courses in plasma physics, nuclear physics, and astrophysics.

While we leverage the notable advancements made in recent years within the realm of galactic CR physics, it's important to clarify that these notes are not intended as a comprehensive review of this subject.
%
To this end, we would like to refer to excellent recent monographs on this subject~\cite{Blasi2013aar,Zweibel2013pp,Grenier2015araa,Amato2018asr,Kachelriess2019ppnp,Gabici2019ijmpd}.
