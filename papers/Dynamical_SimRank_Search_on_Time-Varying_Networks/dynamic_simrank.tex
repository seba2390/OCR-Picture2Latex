%%%%%%%%%%%%%%%%%%%%%%% file template.tex %%%%%%%%%%%%%%%%%%%%%%%%%
%
% This is a general template file for the LaTeX package SVJour3
% for Springer journals.          Springer Heidelberg 2010/09/16
%
% Copy it to a new file with a new name and use it as the basis
% for your article. Delete % signs as needed.
%
% This template includes a few options for different layouts and
% content for various journals. Please consult a previous issue of
% your journal as needed.
%
%%%%%%%%%%%%%%%%%%%%%%%%%%%%%%%%%%%%%%%%%%%%%%%%%%%%%%%%%%%%%%%%%%%
%
%
%\RequirePackage{fix-cm}
%
%\documentclass{svjour3}                     % onecolumn (standard format)
%\documentclass[smallcondensed]{svjour3}     % onecolumn (ditto)
%\documentclass[smallextended]{svjour3}       % onecolumn (second format)
\documentclass[twocolumn]{svjour3}          % twocolumn
%
%\smartqed  % flush right qed marks, e.g. at end of proof
%
\usepackage[english]{babel}
\usepackage{kbordermatrix}
%\usepackage{widetext}
\usepackage{color,soul}
\usepackage{blkarray, colortbl,etoolbox,mathrsfs,amsmath,epsfig,multirow,amssymb,amsfonts,graphicx,subfig,cite,paralist,wrapfig,setspace,bm}
\usepackage{enumitem}
\usepackage{mathtools}
\usepackage[ruled,vlined,nokwfunc]{algorithm2e}
\usepackage[dvipsnames]{xcolor}
\usepackage[titletoc,title]{appendix}
\RequirePackage{amsmath, amssymb}

\newcommand\hlc[1]{{\color{blue}{#1}}}

\makeatletter
\newcommand{\removelatexerror}{\let\@latex@error\@gobble}
\makeatother

\newcommand{\specialcell}[2][c]{%
  {\renewcommand{\arraystretch}{1.2}%
   \begin{tabular}[#1]{@{}l@{}}#2\end{tabular}}}

\definecolor{LightGray}{rgb}{.8,.8,.8}
\def\Big#1{\makebox(0,0){{\Large#1}}}

\makeatletter
\def\mathcolor#1#{\@mathcolor{#1}}
\def\@mathcolor#1#2#3{%
  \protect\leavevmode
  \begingroup
    \color#1{#2}#3%
  \endgroup
}
\makeatother

\newcommand{\ie}{\textit{i.e.,~}}
\newcommand{\viz}{\textit{viz.,~}}
\renewcommand{\st}{\textit{s.t.~}}
\newcommand{\eg}{\textit{e.g.,~}}
\newcommand{\Resp}{\textit{resp.~}}
\newcommand{\iid}{\textit{i.i.d.~}}
\newcommand{\cdf}{\textit{c.d.f.~}}
\newcommand{\wrt}{\textit{w.r.t.~}}
\newcommand{\etal}{\textit{et~al.~}}

%% algorithms

\newcommand{\IncUSR}{\mbox{\textsf{Inc-uSR}}}
\newcommand{\IncUSRone}{\mbox{\textsf{Inc-uSR}}}
\newcommand{\IncUSRtwo}{\mbox{\textsf{Inc-uSR-C1}}}
\newcommand{\IncUSRthree}{\mbox{\textsf{Inc-uSR-C2}}}
\newcommand{\IncUSRfour}{\mbox{\textsf{Inc-uSR-C3}}}
\newcommand{\IncSRAll}{\mbox{\textsf{Inc-SR-All}}}
\newcommand{\IncSRAllP}{\mbox{\textsf{Inc-SR-All-P}}}
\newcommand{\LTSF}{\mbox{\textsf{L-TSF}}}
\newcommand{\IncBSR}{\mbox{\textsf{Inc-bSR}}}

\newcommand{\PartialSim}{\mbox{\textsf{PartialSim}}}

\newcommand{\Arnoldi}{\mbox{\textsf{Arnodi}}}

\newcommand{\IncSR}{\mbox{\textsf{Inc-SR}}}
\newcommand{\IncSVD}{\mbox{\textsf{Inc-SVD}}}
\newcommand{\Batch}{\mbox{\textsf{Batch}}}

%% datasets

\newcommand{\YOUTU}{\mbox{\textsc{YouTu}}}
\newcommand{\CITH}{\mbox{\textsc{CitH}}}
\newcommand{\DBLP}{\mbox{\textsc{DBLP}}}
\newcommand{\SYN}{\mbox{\textsc{Syn}}}

\newcommand{\WEBB}{\mbox{\textsc{WebB}}}
\newcommand{\WEBG}{\mbox{\textsc{WebG}}}
\newcommand{\CITP}{\mbox{\textsc{CitP}}}
\newcommand{\SOCL}{\mbox{\textsc{SocL}}}
\newcommand{\UK}{\mbox{\textsc{UK05}}}
\newcommand{\IT}{\mbox{\textsc{IT04}}}


% variables

\newcommand{\AFF}{\mbox{\textsf{AFF}}}
\newcommand{\op}{\mbox{\textsf{op}}}

%\newtheorem{theorem}{Theorem}
%\newtheorem{assumption}{Assumption}
%\newtheorem{lemma}{Lemma}
%\newtheorem{definition}{Definition}
%\newtheorem{proposition}{Proposition}
%\newtheorem{corollary}{Corollary}
%\newtheorem{remark}{Remark}
%\newtheorem{example}{Example}
%\AtEndEnvironment{example}{\null\hfill$\Box$}%
%\newtheorem{case}{Case}
%\newtheorem{claim}{Claim}
%\newtheorem{problem}{Problem}
%\newtheorem{property}{Property}
%\newtheorem{observation}{Observation}

\newcommand{\mysmall}{\fontsize{7pt}{0.5\baselineskip}\selectfont}
\newcommand{\desmall}{\fontsize{7pt}{0.5\baselineskip}\selectfont}
\newcommand{\mysmaller}{\fontfamily{ptm}\fontsize{7pt}{0.75\baselineskip}\selectfont}

%
% \usepackage{mathptmx}      % use Times fonts if available on your TeX system
%
% insert here the call for the packages your document requires
%\usepackage{latexsym}
% etc.
%
% please place your own definitions here and don't use \def but
% \newcommand{}{}
%
% Insert the name of "your journal" with
% \journalname{myjournal}
%
\begin{document}

\title{Dynamical SimRank Search on Time-Varying Networks
%Fast Incremental SimRank on Link-Evolving Graphs%\thanks{Grants or other notes
%about the article that should go on the front page should be
%placed here. General acknowledgments should be placed at the end of the article.}
}
%\subtitle{Do you have a subtitle?\\ If so, write it here}

%\titlerunning{Short form of title}        % if too long for running head

\author{}
\institute{}


\author{Weiren Yu     \and
        Xuemin Lin    \and
        Wenjie Zhang  \and
        Julie A. McCann
}

%\authorrunning{Short form of author list} % if too long for running head

\institute{W. Yu \at
              School of Engineering and Applied Science,
              Aston University, \\
%              Imperial College London, \\
%              180 Queens Gate, London, United Kingdom \\
              \email{w.yu3@aston.ac.uk}           %  \\
%             \emph{Present address:} of F. Author  %  if needed
           \and
           X. Lin \and W. Zhang \at
              School of Computer Science and Engineering,\\
              University of New South Wales, \\
              %Kensington, NSW, Australia \\
             \email{\{lxue, zhangw\}@cse.unsw.edu.au}
           \and
           J. A. McCann \at
              Department of Computing,
              Imperial College London, \\
              %180 Queens Gate, London, United Kingdom \\
              \email{j.mccann@imperial.ac.uk}           %  \\
}

\date{Received: date / Accepted: date}
% The correct dates will be entered by the editor


\maketitle

\begin{abstract}
SimRank is an appealing pair-wise similarity measure based on graph structure.
It iteratively follows the intuition that two nodes are assessed as similar if they are pointed to by similar nodes.
Many real graphs are large, and links are constantly subject to minor changes.
%It is rather costly to reassess similarities of all pairs of nodes when the graph is updated.
In this article, we study the efficient dynamical computation of all-pairs SimRanks on time-varying graphs.
Existing methods for the dynamical SimRank computation (\eg L-TSF \cite{Shao2015} and READS \cite{Zhang2017}) mainly focus on top-$k$ search with respect to a given query.
For all-pairs dynamical SimRank search, Li \etal\!\!'s approach \cite{Li2010} was proposed for this problem.
It first factorizes the graph via a singular value decomposition (SVD),
and then incrementally maintains such a factorization in response to link updates at the expense of exactness.
As a result, all pairs of SimRanks are updated approximately,
yielding $O({r}^{4}n^2)$ time and $O({r}^{2}n^2)$ memory in a graph with $n$ nodes,
where $r$ is the target rank of the low-rank SVD.
%but $r$ is not negligibly small in practice.
% is approximate,
%Worse yet, it estimates any node-pair similarity in $O({k'}^{4})$ time for each link update,
 %er than the rank of the adjacency matrix.

Our solution to the dynamical computation of SimRank comprises of five ingredients:
%In this paper, we propose a novel fast incremental paradigm.
%we precompute all SimRank once via a batch algorithm first,
%and then incrementally maintain the SimRank update matrix $\mathbf{\Delta S}$.
(1) We first consider edge update that does not accompany new node insertions.
We show that the SimRank update $\mathbf{\Delta S}$ in response to every link update is expressible as a rank-one Sylvester matrix equation.
This provides an incremental method requiring $O(Kn^2)$ time and $O(n^2)$ memory in the worst case to update $n^2$ pairs of similarities for $K$ iterations.
(2) To speed up the computation further,
we propose a lossless pruning strategy that captures the ``affected areas'' of $\mathbf{\Delta S}$ to eliminate unnecessary retrieval.
%,
%without loss of exactness.
This reduces the time of the incremental SimRank to $O(K(m+|\AFF|))$,
where $m$ is the number of edges in the old graph, and $|\AFF| \ (\le n^2)$ is the size of ``affected areas'' in $\mathbf{\Delta S}$,
and in practice, $|\AFF| \ll n^2$.
(3) We also consider edge updates that accompany node insertions, and categorize them into three cases,
according to which end of the inserted edge is a new node.
For each case, we devise an efficient incremental algorithm that can support new node insertions and accurately update the affected SimRanks.
(4) We next study batch updates for dynamical SimRank computation,
and design an efficient batch incremental method that handles ``similar sink edges'' simultaneously and eliminates redundant edge updates.
(5) To achieve linear memory,
we %formulate the SimRank changes as the sum of the outer products of two vectors,
devise a memory-efficient strategy that dynamically updates all pairs of SimRanks column by column in just $O(Kn+m)$ memory,
without the need to store all $(n^2)$ pairs of old SimRank scores.
%(5) For accelerating convergence,
%we build two sets of orthogonal basis that can map
%the computation of $\mathbf{\Delta S}$ to two low-dimensional subspaces. % as a linear combination of the previous $K$ iterative results, with provable guaranteed accuracy.
%This can speed up the convergence rate of incremental SimRank iterations further,
%with guaranteed accuracy.
%
%which is much smaller than $n^2$ in practice.
%
%
%to skip unnecessary similarity recomputations
%
%update SimRank of any node-pair in constant time
%
%This can skip unnecessary similarity recomputations for link updates.
%(2) We devise a fast incremental algorithm, \IncSR,
%to update SimRank of any node-pair in constant time (independent of $r$) for each link update,
%plus $O(\tilde{m})$ precomputation time,
%where $\tilde{m}$ is the number of links in the updated graph.
%
%(3) We further devise a novel ``multi-hop concatenation'' strategy %for the batch SimRank computation, aiming
%to speed up the precomputation of \IncSR~by an exponential reduction in the number of iterations for attaining a given accuracy.
%As opposed to the differential SimRank [] that may not fully preserve the ordinal ranking,
%this method yields exactly the same SimRank scores.
Experimental studies on various datasets demonstrate that our solution substantially outperforms the existing incremental SimRank methods,
and is faster and more memory-efficient than its competitors on million-scale graphs. % when link updates are small.
%
%
%can incrementally compute SimRank (up to) 20X faster than the previous approach while guaranteeing exactness.
%and its precomputation time is accelerated by 8.5X.
%
\keywords{similarity search \and SimRank computation \and dynamical networks \and optimization}
% \PACS{PACS code1 \and PACS code2 \and more}
% \subclass{MSC code1 \and MSC code2 \and more}
\end{abstract}
\begin{figure*}[h!t] \centering
  \includegraphics[width=\linewidth]{e10a.eps}
  \caption{Incremental SimRank problem can decentralise large-scale SimRank retrieval over $G$} \label{fig:10} %\vspace{-10pt}
\end{figure*}
%
\section{Introduction}
%
Recent rapid advances in web data management reveal that
link analysis is becoming an important tool for similarity assessment.
Due to the growing number of applications in
\eg social networks, recommender systems, citation analysis, and link prediction \cite{Jeh2002},
a surge of graph-based similarity measures have surfaced over the past decade.
For instance, Brin and Page \cite{Berkhin2005} proposed a very successful relevance measure, called Google PageRank, to rank web pages.
Jeh and Widom \cite{Jeh2002} devised SimRank, an appealing pair-wise similarity measure that quantifies the structural equivalence of two nodes based on link structure.
Recently, Sun \etal \cite{Sun2011} invented PathSim to retrieve nodes proximities in a heterogeneous graph.
Among these emerging link based measures,
SimRank has stood out as an attractive one in recent years,
due to its simple and iterative philosophy that ``two nodes are similar if they are pointed to by similar nodes'',
coupled with the base case that ``every node is most similar to itself''.
This recursion %set SimRank apart from other traditional measures, \eg Coupling [] and Co-citation[]
not only allows SimRank to capture the global structure of a graph,
but also equips SimRank with mathematical insights that attract many researchers. % that can inspire research in recent years.
For example, Fogaras and R{\'a}cz \cite{Fogaras2007} interpreted SimRank as the meeting time of the coalescing pair-wise random walks.
Li \etal \cite{Li2010} harnessed an elegant matrix equation to formulate the closed form of SimRank.

Nevertheless, the batch computation of SimRank is costly:
$O(Kd'n^2)$ time for all node-pairs \cite{Yu2013},
where $K$ is the total number of iterations,
and $d' \le d$ ($d$ is the average in-degree of a graph).
Generally, many real graphs are large, with links constantly evolving with minor changes.
This is especially apparent in \eg co-citation networks, web graphs, and social networks.
As a statistical example \cite{Ntoulas2004}, there are 5\%--10\% links updated every week in a web graph.
It is rather expensive to recompute similarities for all pairs of nodes from scratch when a graph is updated.
Fortunately, we observe that when link updates are small, the affected areas for SimRank updates are often small as well.
With this comes the need for incremental algorithms that compute changes to SimRank in response to link updates,
to discard unnecessary recomputations.
In this article, we investigate the following problem for SimRank evaluation:
\begin{description}
  \item[\textbf{Problem}] (\textsc{Incremental SimRank Computation})
  \item[\textbf{Given}] an old digraph $G$, old similarities in $G$, link changes $\Delta G$
\footnote{$\Delta G$ consists of a sequence of edges to be inserted/deleted.} to $G$, and a damping factor $C \in (0,1)$.
  \item[\textbf{Retrieve}] the changes to the old similarities.
\end{description}
\begin{figure*}[t] \centering
  \includegraphics[width=.93\linewidth]{e11.eps}
  \caption{{Incremental SimRank problem can be applied to accelerate the batch computation of SimRank on $G$}} \label{fig:11} %\vspace{-10pt}
\end{figure*}
{{
Our research for the above SimRank problem is motivated by the following real applications:
\begin{example}[Decentralise Large-Scale SimRank Retrieval]  \label{eg:11}
Consider the web graph $G$ in Figure~\ref{fig:10}. There are $n=14$ nodes (web pages) in $G$, and each edge is a hyperlink.
To evaluate the SimRank scores of all $({n} \times n)$ pairs of web pages in $G$,
existing all-pairs SimRank algorithms need iteratively compute the SimRank matrix $\mathbf{S}$ of size $({n} \times n)$ in a centralised way (by using a single machine).
In contrast, our incremental approach can significantly improve the computational efficiency of all pairs of SimRanks by retrieving $\mathbf{S}$ in a decentralised way as follows:

We first employ a graph partitioning algorithm (\textit{e.g.,} METIS\footnote{http://glaros.dtc.umn.edu/gkhome/views/metis}) that can decompose the large graph $G$ into several small blocks such that the number of the edges with endpoints in different blocks is minimised. In this example, we partition $G$ into 3 blocks, $G_1 \cup G_2 \cup G_3$, along with 2 edges $\{(f,c),(f,k)\}$ across the blocks, as depicted in the first row of Figure~\ref{fig:10}.

Let $G_{\textrm{old}}:= G_1 \cup G_2 \cup G_3$ and $\Delta G:=\{(f,c),(f,k)\}$.
Then, $G$ can be viewed as ``$G_{\textrm{old}}$ perturbed by $\Delta G$ edge insertions''. That is,
\[
    G =  \overbrace{( G_1 \cup G_2 \cup G_3 )}^{:= G_{\textrm{old}}} \cup \overbrace{\{(f,c),(f,k)\}}^{:= \Delta G} = G_{\textrm{old}} \cup \Delta G
\]

Based on this decomposition,
we can efficiently compute $\mathbf{S}$ over $G$ by dividing $\mathbf{S}$ into two parts:
\[
\mathbf{S} = \mathbf{S}_{\textrm{old}} + \mathbf{\Delta S}
\]
where $\mathbf{S}_{\textrm{old}}$ is obtained by using a batch SimRank algorithm over $G_{\textrm{old}}$,
and $\mathbf{\Delta S}$ is derived from our proposed incremental method which takes $\mathbf{S}_{\textrm{old}}$ and $\Delta G$ as input.

It is worth mentioning that this way of retrieving $\mathbf{S}$ is far more efficient than directly computing $\mathbf{S}$ over $G$ via a batch algorithm.
There are two reasons:

Firstly, $\mathbf{S}_{\textrm{old}}$ can be efficiently computed in a decentralised way. It is a block diagonal matrix with no need of $n \times n$ space to store $\mathbf{S}_{\textrm{old}}$.
This is because $G_{\textrm{old}}$ is only comprised of several connected components $(G_1, G_2, G_3)$, and there are no edges across distinct components.
Thus, $\mathbf{S}_{\textrm{old}}$ exhibits a block diagonal structure:
\[
\mathbf{S}_{\textrm{old}} := diag(\mathbf{S}_{G_1}, \mathbf{S}_{G_2},\mathbf{S}_{G_3}) :=  \left[
\scalebox{1}{$
\begin{array}{c|c|c}
  {\mathbf{S}}_{G_1} & \mathbf{0} & \mathbf{0}   \\ \hline
  \mathbf{0} & {\mathbf{S}}_{G_2} & \mathbf{0} \\ \hline
  \mathbf{0} & \mathbf{0} & {\mathbf{S}}_{G_3}
\end{array}$} \right]
\]
To obtain $\mathbf{S}_{\textrm{old}}$, instead of applying the batch SimRank algorithm over the entire $G_{\textrm{old}}$,
we can apply the batch SimRank algorithm over each component $G_i \ (i=1,2,3)$ independently to obtain the $i$-th diagonal block of $\mathbf{S}_{\textrm{old}}$, $\mathbf{S}_{G_i}$.
Indeed, each $\mathbf{S}_{G_i}$ is computable in parallel.
Even if $\mathbf{S}_{\textrm{old}}$ is computed using a single machine,
only $O(n_1^2+n_2^2+n_3^2)$ space is required to store its diagonal blocks,
where $n_i$ is the number of nodes in each $G_i$,
rather than $O(n^2)$ space to store the entire $\mathbf{S}_{\textrm{old}}$ (see Figure~\ref{fig:10}).

Secondly, after graph partitioning, there are not many edges across components.
Small size of $\Delta G$ often leads to sparseness of $\mathbf{\Delta S}$ in general.
Hence, $\mathbf{\Delta S}$ is stored in a sparse format.
In addition, our incremental SimRank method will greatly accelerate the computation of $\mathbf{\Delta S}$.

Hence, along with graph partitioning, our incremental SimRank research will significantly enhance the computational efficiency of SimRank on large graphs, using a decentralised fashion. \qed
\end{example}

Our research on the incremental SimRank problem not only can decentralise large-scale SimRank retrieval, but also will enable a substantial speedup on the batch computation of SimRank, as indicated below.
\begin{example}[Accelerate Batch Computation of SimRank]  \label{eg:12}
Consider the citation network $G$ in Figure~\ref{fig:11},
where each node denotes a paper, and an edge a reference from one paper to another.
We wish to assess all pairs of similarities between papers.
Unlike existing batch computation that iteratively retrieves all-pairs SimRanks over the entire $G$,
our incremental method significantly accelerates the batch computation of SimRank as follows:

We first utilise BFS or DFS search to find a spanning tree (or arborescence) of $G$, denoted as $G_\textrm{old}$.
We observe that, due to the tree structure, all-pairs SimRank scores in $G_\textrm{old}$ are relatively easier to compute.
For example, each entry of Li \etal\!\!'s SimRank matrix $\mathbf{S}_{\textrm{old}}$ over $G_\textrm{old}$ can be obtained from a lightweight formula:
\[
\begin{split}
\mathbf{S}_{\textrm{old}}(a,b) = \scalebox{0.85}{$ \left\{
                                   \begin{array}{ll}
                                     0,  & \textrm{if $a$ and $b$ are not on the} \\
                                        &  \textrm{same level of the tree $G_\textrm{old}$;} \\
                                     C^{\lambda_{a,b}}(1-C^{H-\lambda_{a,b}+1}),  &\hbox{otherwise.}
                                   \end{array}
                                 \right. $}
\end{split}
\]
where $C$ is a damping factor, $\lambda_{a,b}$ is the number of edges from the lowest common ancestor of $(a,b)$ to node $a$ (or equivalently, to $b$) in the tree $G_\textrm{old}$, and $H$ is the number of edges from the root to node $a$ (or equivalently, to $b$) in the tree $G_\textrm{old}$.

Given $\mathbf{S}_{\textrm{old}}$, we next apply our incremental SimRank method that can significantly speed up the computation of new SimRank scores $\mathbf{S}$ in $G$.
Specifically, we denote by $\Delta G$ the set of edges in $G$ but not in $G_\textrm{old}$.
In Figure~\ref{fig:11},
\[
\Delta G := G - G_\textrm{old} = \{ (c,b), (c,f), (f,e) \}.
\]
Based on $\mathbf{S}_{\textrm{old}}$ and $\Delta G$, our incremental SimRank method can dynamically retrieve only the changes to $\mathbf{S}_{\textrm{old}}$ in response to $\Delta G$,
whose result $\mathbf{\Delta S}$ is a sparse matrix, as illustrated in Figure~\ref{fig:11}.

It is important to note that it does not require $n \times n$ memory to store $\mathbf{S}_{\textrm{old}}$ because $G_{\textrm{old}}$ is a tree structure.
If there are $H$ levels in the tree $G_{\textrm{old}}$ with $n_l$ nodes on level $l \ (l=1,\cdots,H)$, then we only need the space
\[ \textstyle
O \big( \sum_{l=1}^H ({{n_l}^2}) \big) \ll O \big(({\sum_{l=1}^H {n_l}})^2 \big) = O(n^2)
\]
to store the nonzero diagonal blocks of $\mathbf{S}_{\textrm{old}}$. \qed
\end{example}

These examples show that our incremental SimRank is useful to (i) decentralise large-scale SimRank retrieval, and (ii) accelerate the batch computation of SimRank.
Despite its usefulness,
existing work on incremental SimRank computation is rather limited.
%Indeed, because of the recursive definition of SimRank,
%it is a big challenge to identify ``affected areas'' for efficiently updating SimRank in an incremental manner.
To the best of our knowledge,
there is a relative paucity of work \cite{Jiang2017,Shao2015,Yu2014,Li2010} on incremental SimRank problems.
Shao \etal \cite{Shao2015} proposed a novel two-stage random-walk sampling scheme, named TSF,
which can support top-$k$ SimRank search over dynamic graphs.
In the preprocessing stage, TSF samples $R_g$ one-way graphs that serve as an index for querying process.
At query stage, for each one-way graph, $R_q$ new random walks of node $u$ are sampled.
However, the dynamic SimRank problems studied in \cite{Shao2015} and this work are different:
This work focuses on \emph{all $(n^2)$ pairs} of SimRank retrieval,
which requires $O(K(m+|\AFF|))$ time to compute the \emph{entire matrix} $\mathbf{S}$ in a deterministic style.
In Section~\ref{sec:07}, we have proposed a memory-efficient version of our incremental method that updates all pairs of similarities in a column-by-column fashion within only $O(Kn+m)$ memory.
In comparison,
\cite{Shao2015} focuses on top-$k$ SimRank dynamic search \wrt a given query $u$.
It incrementally retrieves \emph{only $k \ (\le n)$ nodes} with highest SimRank scores in \emph{a single column} $\mathbf{S}_{\star,u}$,
which requires $O(K^2 R_q R_g)$ \emph{average} query time\footnote{Recently, Jiang \etal \cite{Jiang2017} has argued that, to retrieve $\mathbf{S}_{\star,u}$, the querying time of Shao \etal\!\!'s TSF \cite{Shao2015} is $O(K n R_q R_g)$. The $n$ factor is due to the time to traverse the reversed one-way graph; in the worst case, all $n$ nodes are visited.}
 to retrieve $\mathbf{S}_{\star,u}$ along with $O(n \log k)$ time to return top-$k$ results from $\mathbf{S}_{\star,u}$.
Recently, Jiang \etal \cite{Jiang2017} pointed out that the probabilistic error guarantee of Shao \etal\!\!'s method is based on the assumption that no cycle in the given graph has a length shorter than $K$,
and they proposed READS, a new efficient indexing scheme that improves precision and indexing space for dynamic SimRank search.
The querying time of READS is $O(r n)$ to retrieve one column $\mathbf{S}_{\star,u}$, where $r$ is the number of sets of random walks.
Hence, TSF and READS are highly efficient for \emph{top-$k$ single-source} SimRank search.
Moreover, optimization methods in this work are based on a rank-one Sylvester
matrix equation characterising changes to the entire SimRank matrix $\mathbf{S}$ for all-pairs dynamical search,
which is fundamentally different from \cite{Shao2015,Jiang2017}'s methods that maintain one-way graphs (or SA forests) updating.
\begin{figure}[t] \centering
  \includegraphics[width=1\linewidth]{e01.eps}
  \caption{Incrementally update SimRanks when a new edge $(i,j)$ (with $\{i, j\} \subseteq V$) is inserted into $G=(V,E)$} \label{fig:01} %\vspace{-10pt}
\end{figure}
It is important to note that, for large-scale graphs, our incremental methods do not need to memoize all $(n^2)$ pairs of old SimRank scores.
Instead, $\mathbf{S}$ can be dynamically updated column-wisely in $O(Kn+m)$ memory.
For updating each column of $\mathbf{S}$,
our experiments in Section~\ref{sec:09} verify that
our memory-efficient incremental method is scalable on large real graphs while running 4--7x faster than the dynamical TSF \cite{Shao2015} per edge update,
due to the high cost of \cite{Shao2015} merging one-way graph's log buffers for TSF indexing.}}

Among the existing studies \cite{Jiang2017,Shao2015,Li2010} on dynamical SimRank retrieval,
the problem setting of Li \etal\!\!'s \cite{Li2010} on all-pairs dynamic search is exactly the same as ours:
the goal is to retrieve changes $\mathbf{\Delta S}$ to all-pairs SimRank scores $\mathbf{S}$,
given old graph $G$, link changes $\Delta G$ to $G$. %, and old SimRank matrix $\mathbf{S}$.
To address this problem,
% \cite{Shao2015,Li2010}.
%Shao \etal \cite{Shao2015} gave a picturesque exposition of a one-way graph, a novel compact indexing structure, for incremental SimRank retrieval,
%but their random-walk sampling framework delivers probabilistic results.
%Regarding deterministic methods,
% proposed a pioneering strategy that can incrementally retrieve changes to SimRank in response to link updates.
%Precisely, their
the central idea of \cite{Li2010} is to factorize the backward transition matrix $\mathbf{Q}$
%\footnote{In the notation of \cite{Li2010}, the backward transition matrix $\mathbf{Q}$ is denoted as $\tilde{\mathbf{W}}$,
%which is the row-normalized transpose of the adjacency matrix.}
of the original graph into $\mathbf{U} \cdot \mathbf{\Sigma} \cdot {\mathbf{V}}^T$
%\footnote{Throughout this article, we use ${\mathbf{X}}^T$ (instead of $\tilde{\mathbf{X}}$ in \cite{Li2010}) to denote the transpose of matrix $\mathbf{X}$.}
via a singular value decomposition (SVD) first,
and then incrementally estimate the updated matrices of $\mathbf{U}$, $\mathbf{\Sigma}$, ${\mathbf{V}}^T$ for link changes at the expense of exactness.
Consequently, updating all pairs of similarities entails $O({r}^{4}n^2)$ time and $O({r}^{2}n^2)$ memory yet without guaranteed accuracy,
where % $n$ is the number of nodes, and
$r \ (\le n)$ is the target rank of the low-rank SVD approximation\footnote{According to \cite{Li2010}, using our notation,
$r \le \textrm{rank}(\mathbf{\Sigma} + \mathbf{U}^T \cdot \mathbf{\Delta Q} \cdot \mathbf{V})$,
where $\mathbf{\Delta Q}$ is the changes to $\mathbf{Q}$ for link updates.}.
This method is efficient to graphs when $r$ is extremely small, \eg a star graph $(r=1)$.
However, in general, $r$ is not always negligibly small.

(Please refer to Appendix~\ref{app:01} for a discussion in detail, and Appendix~\ref{app:03a} for an example.)
%as illustrated in the following example.
%%        % \footnote{Herein, an ``accurate'' algorithm means that its iterative result will converge to the exact SimRank solution when the number of iterations increases.}
%%        Instead of incrementally retrieving \emph{the changes to the SVD of $\mathbf{Q}$} for evaluating new similarities,
%%        our method can cope with the dynamic nature of a real network,
%%        by maximally reusing only a small fraction of SimRanks in an old graph
%%        %
%%        %precomputing SimRank on the old entire network once via a batch algorithm first,
%%        %
%%        and dynamically retrieving \emph{SimRank changes} $\mathbf{\Delta S}$ \wrt link updates.
%%        Moreover, as graphs are often updated with small changes,
%%        not all pairs of similarities need recomputing.
%%        For example, in the table of Figure~\ref{fig:01},
%%        many pairs of similarities (highlighted in gray) remain unchanged when the edge $(i,j)$ is added.
%%        %However, it is a grand challenge to identify the ``affected areas'' of $\mathbf{\Delta S}$
%%        %because SimRank is defined in a recursive fashion \cite{Jeh2002}.
%%        %To resolve this problem,
%%        To efficiently identify ``affected areas'',
%%        we can express $\mathbf{\Delta S}$ as an aggregate of similarities with respect to the pairs of incoming paths, and detect the changes in these paths.
%%
%%
%%        Besides, it is difficult to achieve high memory efficiency when all pairs of SimRanks are incrementally updated.
%%        This is because conventional approaches evaluate \emph{all} pairs of similarities \emph{at the same time}, and thus at least $O(n^2)$ memory is required to $n^2$ pairs of similarity outputs.
%%        The existing work of Li \etal \cite{Li2010} entails even $O({r}^{2}n^2)$ memory to store the intermediate results of the Kronecker product of two $n\times r$ matrices,
%%        which may become cost-inhibitive for large scale networks.
%%        Fortunately, we notice that our characterization of $\mathbf{\Delta S}$ exhibits an elegant structure,
%%        which allows all-pairs SimRanks being updated column by column.
%%        This enables a significant reduction in memory usage from quadratic to linear in the number of nodes even though all pairs of SimRanks are incrementally updated.
%
%\vspace{1pt} \noindent \textbf{Contributions.}
\subsection{Main Contributions}
%
Motivated by this,
we propose an efficient and accurate scheme for incrementally computing all-pairs SimRanks on link-evolving graphs.
Our main contributions consist of the following five ingredients:
%
\begin{itemize}[itemsep=3pt] %\setlength\itemsep{1em}
\item
We first focus on unit edge update that does not accompany new node insertions.
By characterizing the SimRank update matrix $\mathbf{\Delta S}$ \wrt every link update as a rank-one Sylvester matrix equation,
we devise a fast incremental SimRank algorithm,
which entails $O(Kn^2)$ time in the worst case to update $n^2$ pairs of similarities for $K$ iterations. (Section~\ref{sec:04})
%
%We first consider \emph{unit update} (\ie a single edge insertion or deletion) for SimRank, and divide unit insertion into four cases, according to whether an added edge will incur node insertion.
%For each case, we characterize the SimRank update matrix $\mathbf{\Delta S}$ \wrt every link update as a rank-one Sylvester matrix equation.
%This produces a fast dynamical SimRank algorithm that entails $O(Kn^2)$ time and $O(n^2)$ memory in the worst case to update $n^2$ pairs of similarities for $K$ iterations. %(Section \ref{sec:04a})
%
\item
To speed up the computation further,
we also propose an effective pruning strategy that captures the ``affected areas'' of $\mathbf{\Delta S}$ to discard unnecessary retrieval,
without loss of accuracy.
This reduces the time of incremental SimRank to $O(K(m+|\AFF|))$,
where
%$d$ is the average in-degree of the old graph, and
$|\AFF| \ (\le n^2)$ is the size of ``affected areas'' in $\mathbf{\Delta S}$,
and in practice, $|\AFF| \ll n^2$. (Section~\ref{sec:05}) %(Section \ref{sec:04b})
%
\item
We also consider edge updates that accompany new node insertions, and distinguish them into three categories, according to which end of the inserted edge is a new node.
For each case, we devise an efficient incremental SimRank algorithm that can support new nodes insertion and accurately update affected SimRank scores. (Section~\ref{sec:06})
\item
We next investigate the batch updates of dynamical SimRank computation.
Instead of dealing with each edge update one by one,
we devise an efficient algorithm that can handle a sequence of edge insertions and deletions simultaneously,
by merging ``similar sink edges'' and minimizing unnecessary updates. (Section~\ref{sec:08}) % that can cancel out each other.
\item
To achieve linear memory efficiency,
we also express $\mathbf{\Delta S}$ as the sum of many rank-one tensor products,
and devise a memory-efficient technique that updates all-pairs SimRanks in a column-by-column style in $O(Kn+m)$ memory,
without loss of exactness. (Section~\ref{sec:07})
%\item
%For accelerating convergence,
%we also construct two sets of orthogonal basis that can project the computation of $\mathbf{\Delta S}$ onto two low-dimensional subspaces.
%Due to small size and orthogonality of the subspaces,
%the convergence rate of incremental SimRank can be improved further with guaranteed accuracy.
%
%describe $\mathbf{\Delta S}$ as a linear combination of the previous $K$ iterative results, with provable guaranteed accuracy.
%This can reduce the total number of iterations further by several times to attain a desired accuracy.
%
\item
  We conduct extensive experiments on real and synthetic datasets to demonstrate that our algorithm
(a) is consistently faster than the existing incremental methods from several times to over one order of magnitude;
%while guaranteeing the exactness of SimRank.
(b) is faster than its batch counterparts especially when link updates are small;
%(d) achieves faster convergence rate by several times when the small orthogonal subspaces are leveraged.
(c) for batch updates, runs faster than the repeated unit update algorithms; %(Section \ref{sec:05}) %%are updated up to 37\% in average on real evolving networks.
(d) entails linear memory and scales well on billion-edge graphs for all-pairs SimRank update; % even if all-pairs SimRank scores are computed incrementally;
(e) is faster than {\LTSF} and its memory space is less than {\LTSF};
(f) entails more time on Cases (C0) and (C2) than Cases (C1) and (C3) for four edge types, and Case (C3) runs the fastest. (Section~\ref{sec:09})
\end{itemize}

This article is a substantial extension of our previous work~\cite{Yu2014}.
We have made the following new updates:
(1) In Section~\ref{sec:06}, we study three types of edge updates that accompany new node insertions.
This solidly extends \cite{Yu2014} and Li \etal's incremental method \cite{Li2010} whose edge updates disallow node changes.
%
%For unit update, we provide an in-depth analysis of \emph{four} cases for edge insertion that can be accompanied with new node insertion (in Section~X).
%This solidly extends the prior work of \cite{Yu2014} whose focus is solely on \emph{one} case of edge insertion that disallows node changes.
(2) In Section~\ref{sec:08}, we
%resort to two low-dimensional and orthogonal subspaces,
%aiming to speed up the convergence rate of the incremental SimRank iterations in \cite{Yu2014} further by several times with guaranteed accuracy.
also investigate batch updates for dynamic SimRank computation, and devise an efficient algorithm that can handle ``similar sink edges'' simultaneously and discard unnecessary unit updates further.
(3) In Section~\ref{sec:07},  we propose a memory-efficient strategy that significantly reduces the memory from $O(n^2)$ to $O(Kn+m)$ for incrementally updating all pairs of SimRanks on million-scale graphs, without compromising running time and accuracy.
%
%also propose a novel dynamical updating strategy in a column-by-column fashion,
%which reduces the worst-case memory usage from $O(Kn^2)$ \cite{Yu2014} to $O(Kn)$ space for incrementally updating all pairs of SimRanks, with no compromise in computational time and accuracy.
(4) In Section~\ref{sec:09}, we conduct additional experiments on real and synthetic datasets to verify the high scalability and fast computational time of our memory-efficient methods, as compared with the L-TSF method.
(5) In Section~\ref{sec:02}, we update the related work section by incorporating state-of-the-art SimRank research.
%
%
%Our work is the first to take advantage of the incoming paths for the incremental computation of SimRank.
%We contend that our incremental techniques will yield a promising approach to other ranking models,
%\eg Personalized PageRank (PPR), and Random Work with Restart (RWR).
%
%%        \subsection{Organization}
%%        %
%%        %\vspace{1pt} \noindent \textbf{Organization.}
%%        %
%%        The remainder of this article is structured as follows:
%%        Section~\ref{sec:03} recaps the SimRank background.
%%        %Section~\ref{sec:03b} explains the limitation of Li \etal\!\!'s incremental way~\cite{Li2010}.
%%        Section~\ref{sec:04} presents our dynamical method to deal with edge update that does not accompany nodes insertion.
%%        Section~\ref{sec:05} provides our pruning strategy to reduce the running time further.
%%        Section~\ref{sec:06} extends our method to deal with three different types of edge updates that accompany node insertions.
%%        Section~\ref{sec:08} considers batch updates for dynamical SimRank computation.
%%        Section~\ref{sec:07} reduces the memory space.
%%        %accelerates the convergence of incremental SimRank iterations.
%%        Section~\ref{sec:09} demonstrates the experimental results.
%%        The related work is in Section~\ref{sec:02},
%%        %Section~X considers incremental bulk updates.
%%        followed by conclusions and future work in Section~\ref{sec:11}.
%
\section{SimRank Background} \label{sec:03}
%
In this section,
we give a broad overview of SimRank.
Intuitively,
the central theme behind SimRank is that
``two nodes are considered as similar if their incoming neighbors are themselves similar''.
Based on this idea,
there have emerged two widely-used SimRank models:
(1) Li \etal\!\!'s model (\eg \cite{Rothe2014,Yu2015a,Li2010,He2010,Fujiwara2013}) and
(2) Jeh and Widom's model (\eg \cite{Jeh2002,Lizorkin2008,Fogaras2005,Shao2015,Kusumoto2014}).
%The most recent work of \cite{Yu2015a} has showed the subtle semantic difference between the two models,
%and validated that Li \etal\!\!'s SimRank model can effectively capture far more pairs of self-intersecting paths that are neglected by Jeh and Widom's model,
%and thereby has more meaningful semantics to evaluate pair-wise similarities.

Throughout this article, our focus is on Li \etal\!\!'s SimRank model, also known as Co-SimRank in \cite{Rothe2014},
since the recent work \cite{Rothe2014} by Rothe and Sch\"{u}tze has showed that Co-SimRank is more accurate than Jeh and Widom's SimRank model in real applications such as bilingual lexicon extraction.
\noindent (Please refer to Remark~\ref{rem:01} for detailed explanations.)

%Therefore, in this paper, our focus is on Li \etal\!\!'s SimRank model.
%
%The consistency of two forms was pointed out in \cite{Li2010}.
%Both forms have the same sematic meaning except for the slight difference in formulating the base case
%that ``each node is maximally similar to itself''.
%In this paper, our incremental techniques is based on the matrix form of SimRank.
%We shall further clarify this in Subsection X,
%The reason will be also explained in Subsection X.
%
\subsection{Li \etal\!\!'s SimRank model}
%
Given a directed graph ${G}=({V},{E})$ with a node set ${V}$ and an edge set ${E}$,
%let $\mathcal{I}\left(a \right)$ be the in-neighbor set of node $a$, and
let $\mathbf{Q}$ be its backward transition matrix (that is, the transpose of the column-normalized adjacency matrix),
whose entry $[\mathbf{Q}]_{i,j}=1/\textrm{in-degree}(i)$ if there is an edge from $j$ to $i$, and 0 otherwise.
Then, Li \etal\!\!'s SimRank matrix, denoted by $\mathbf{S}$, is defined as
\begin{equation}  \label{eq:03a}
   {{\mathbf{S}}}= C\cdot (\mathbf{Q} \cdot {{\mathbf{S}}}\cdot {{\mathbf{Q}}^{T}} ) + (1-C) \cdot {{\mathbf{I}}_{n}},
\end{equation}
where $C \in \left( 0,1 \right)$ is a damping factor, which is generally taken to be 0.6--0.8,
and $\mathbf{I}_n$ is an $n \times n$ identity matrix $(n=|V|)$.
The notation ${(\star)}^{T}$ is the matrix transpose.


Recently, Rothe and Sch\"{u}tze \cite{Rothe2014} have introduced Co-SimRank, whose definition is
\begin{equation}  \label{eq:03b}
   {{\mathbf{\tilde{S}}}}= C\cdot (\mathbf{Q} \cdot {{\mathbf{\tilde{S}}}}\cdot {{\mathbf{Q}}^{T}} ) + {{\mathbf{I}}_{n}},
\end{equation}

Comparing Eqs.\eqref{eq:03a} and \eqref{eq:03b},
we can readily verify that Li \etal\!\!'s SimRank scores equal Co-SimRank scores scaled by a constant factor $(1-C)$, \ie
${{\mathbf{S}}} = (1-C) \cdot   {{\mathbf{\tilde{S}}}}$.
Hence, the relative order of all Co-SimRank scores in ${{\mathbf{\tilde{S}}}}$ is exactly the same as that of Li \etal\!\!'s SimRank scores in ${{\mathbf{S}}}$ even though the entries in ${{\mathbf{\tilde{S}}}}$ can be larger than 1.
That is, the ranking of Co-SimRank ${{\mathbf{\tilde{S}}}}(*,*)$ is identical to the ranking of Li \etal\!\!'s SimRank ${{\mathbf{S}}}(*,*)$.

%
\subsection{Jeh and Widom's SimRank model}
%
Jeh and Widom's SimRank model, in matrix notation, can be formulated as
\begin{equation}
   {{\mathbf{S}'}}= \max \{C\cdot (\mathbf{Q} \cdot {{\mathbf{S}'}}\cdot {{\mathbf{Q}}^{T}} ), {{\mathbf{I}}_{n}}\},
\end{equation}
where ${{\mathbf{S}'}}$ is their SimRank similarity matrix;
$\max \{ \mathbf{X}, \mathbf{Y} \}$ is matrix element-wise maximum,
\ie $[\max \{ \mathbf{X}, \mathbf{Y} \}]_{i,j}:=\max\{[\mathbf{X}]_{i,j}, [\mathbf{Y}]_{i,j}\}$.

\begin{remark} \label{rem:01}
The recent work by Kusumoto \etal \cite{Kusumoto2014} has showed that ${{\mathbf{S}}}$ and ${{\mathbf{S}'}}$ do not produce the same results.
%implying that it is ill-advised to use these two models interchangeably.
Recently, Yu and McCann \cite{Yu2015a} have showed the subtle difference of the two SimRank models from a semantic perspective,
and also justified that Li \etal\!\!'s SimRank ${{\mathbf{S}}}$ can capture more pairs of self-intersecting paths that are neglected by Jeh and Widom's SimRank ${{\mathbf{S}'}}$.

The recent work \cite{Rothe2014} by Rothe and Sch\"{u}tze has demonstrated further that, in real applications such as bilingual lexicon extraction,
the ranking of Co-SimRank ${{\mathbf{\tilde{S}}}}$ (\ie the ranking of Li \etal\!\!'s SimRank ${{\mathbf{S}}}$) is more accurate than that of Jeh and Widom's SimRank ${{\mathbf{S}'}}$ (see \cite[Table~4]{Rothe2014}).

%
%justified Li \etal\!\!'s SimRank model to be more semantically meaningful than Jeh and Widom's model in that
%(a) ${{\mathbf{S}}}$ can capture more pairs of self-intersecting paths that are neglected by ${{\mathbf{S}'}}$, and
%(b) the diagonal entries of ${{\mathbf{S}}}$ not only can guarantee that each node is maximally similar to itself, but also distinguish the relative importance of each node, unlike ${{\mathbf{S}'}}$ whose diagonals are always 1s.
%
%The iterative form of SimRank suggests that the similarity of any node with itself is always maximally~1,
%whereas the matrix form of SimRank ensures that the similarity of a node with itself is maximal (but is not necessarily 1).
%
\end{remark}

%$\mathbf{S}$ is a SimRank similarity matrix whose entry $[\mathbf{S}]_{i,j}$ denotes the similarity score $s(i,j)$,
%$\mathbf{Q}$ is the backward transition matrix whose entry $[\mathbf{Q}]_{i,j}=1/|\mathcal{I}(i)|$ if there is an edge from $j$ to $i$, and 0 otherwise,
%and $\mathbf{I}_n$ is an $n \times n$ identity matrix.
%
%
%\vspace{1pt} \noindent \textbf{(1) Iterative Form of SimRank.} \
%Given a digraph ${G}=({V},{E})$ with a vertex set ${V}$ and an edge set ${E}$,
%the SimRank similarity between two nodes $a$ and $b$, denoted by $s(a,b)$, is defined as %\vspace{-10pt}
%(i) $s(a,b)=0$, if $\mathsf{\mathcal{I}}(a)=\varnothing \text{ or } \mathsf{\mathcal{I}}(b)=\varnothing $;
%(ii) otherwise,
%\begin{equation} \label{eq:01a} \scalebox{.95}[.95]{$
%s(a,b)=
%\left\{
%  \begin{array}{ll}
%    1, & {a=b;} \\
%    \frac{C}{\left| \mathsf{\mathcal{I}}(a) \right|\left| \mathsf{\mathcal{I}}(b) \right|}\sum_{j \in \mathsf{\mathcal{I}}\left( b \right) }{\sum_{i \in \mathsf{\mathcal{I}}\left( a \right) }{s(i,j)}},  & {a \neq b.}
%  \end{array}
%\right. $}
%\end{equation}
%%(i) $s(a,a)=1$;
%%(ii) $s(a,b)=0$, if $\mathsf{\mathcal{I}}(a)=\varnothing \text{ or } \mathsf{\mathcal{I}}(b)=\varnothing $;
%%(iii) otherwise,
%%    \begin{equation} \label{eq:01a}
%%    s(a,b)=\tfrac{C}{\left| \mathsf{\mathcal{I}}(a) \right|\left| \mathsf{\mathcal{I}}(b) \right|}\sum\nolimits_{j \in \mathsf{\mathcal{I}}\left( b \right) }{\sum\nolimits_{i \in \mathsf{\mathcal{I}}\left( a \right) }{s(i,j)}},
%%    \end{equation}
%%
%%\begin{enumerate}[(i)]
%%\setlength{\parskip}{-1pt}
%%  \item $s(a,a)=1$;
%%  \item $s(a,b)=0$, if $\mathsf{\mathcal{I}}(a)=\varnothing \text{ or } \mathsf{\mathcal{I}}(b)=\varnothing $;
%%  \item otherwise,
%%    \begin{equation} \label{eq:01}
%%    s(a,b)=\frac{C}{\left| \mathsf{\mathcal{I}}(a) \right|\left| \mathsf{\mathcal{I}}(b) \right|}\sum_{j \in \mathsf{\mathcal{I}}\left( b \right) }{\sum_{i \in \mathsf{\mathcal{I}}\left( a \right) }{s(i,j)}},
%%    \end{equation}
%%\end{enumerate} %\vspace{-10pt}
%where $C \in \left( 0,1 \right)$ is a damping factor,
%$\mathcal{I}\left(a \right)$ is the in-neighbor set of node $a$, and
%$| \mathcal{I} (a) |$ is the cardinality of $\mathcal{I} (a)$.
%
%The exact $s(a,b)$ in Eq.\eqref{eq:01a} can be iteratively reached as
%$s_0(a,b)=\left\{
%  \begin{smallmatrix}
%    1, & {a=b;} \\
%    0, & {a \neq b.}
%  \end{smallmatrix}
%\right.$;
%$s_{k}( a,b )=0$, if $\mathsf{\mathcal{I}}\left( a \right)=\varnothing \text{ or } \mathsf{\mathcal{I}}\left( b \right)=\varnothing $,
%for $k=1,2,\cdots$,
%otherwise,
%\begin{equation*} \label{eq:02a} \scalebox{.95}[.95]{$
%s_k(a,b)=
%\left\{
%  \begin{array}{ll}
%    1, & {a=b;} \\
%    \frac{C}{\left| \mathsf{\mathcal{I}}(a) \right|\left| \mathsf{\mathcal{I}}(b) \right|}\sum_{j \in \mathsf{\mathcal{I}}\left( b \right) }{\sum_{i \in \mathsf{\mathcal{I}}\left( a \right) }{s_{k-1}(i,j)}},  & {a \neq b.}
%  \end{array}
%\right. $}
%\end{equation*}
%%Start with
%%$s_0 (a,a)=1$ and $s_0(a,b)=0$ if $a \ne b$, and
%%for $k=0,1,2,\cdots$, set
%%(i) $s_{k+1}( a,a )=1$;
%%(ii)
%%(iii) otherwise,
%%\begin{equation} \label{eq:02a}
%%s_{k+1}( a,b )=\tfrac{C}{| \mathsf{\mathcal{I}}( a ) | | \mathsf{\mathcal{I}}( b ) |}\sum\nolimits_{j \in \mathsf{\mathcal{I}}( b ) }{\sum\nolimits_{i \in \mathsf{\mathcal{I}}( a ) }{s_{k}( i, j )}}.
%%\end{equation}
%%
%%\begin{enumerate}[(i)]
%%%\setlength{\parskip}{-1pt}
%%\item $s_{k+1}( a,a )=1$;
%%\item $s_{k+1}( a,b )=0$, if $\mathsf{\mathcal{I}}\left( a \right)=\varnothing \text{ or } \mathsf{\mathcal{I}}\left( b \right)=\varnothing $;
%%\item otherwise,
%%\begin{small}
%%\begin{equation} \label{eq:02}
%%s_{k+1}( a,b )=\frac{C}{| \mathsf{\mathcal{I}}( a ) | | \mathsf{\mathcal{I}}( b ) |}\sum_{j \in \mathsf{\mathcal{I}}( b ) }{\sum_{i \in \mathsf{\mathcal{I}}( a ) }{s_{k}( i, j )}}.
%%\end{equation}
%%\end{small}
%%\end{enumerate}
%
%The resulting sequence ${\{s_k(a,b)\}}_{k=0}^{\infty}$ converges to $s(a,b)$. %, the \emph{exact} solution of Eq.\eqref{eg:01}.
%
%\vspace{1pt} \noindent \textbf{(2) Matrix Form of SimRank.} \
%In matrix notations, SimRank can be formulated as follows.
%\begin{equation}  \label{eq:03a}
%   {{\mathbf{S}}}= C\cdot (\mathbf{Q} \cdot {{\mathbf{S}}}\cdot {{\mathbf{Q}}^{T}} ) + (1-C) \cdot {{\mathbf{I}}_{n}},
%\end{equation}
%where $\mathbf{S}$ is the similarity matrix whose entry $[\mathbf{S}]_{i,j}$ denotes the similarity score $s(i,j)$,
%$\mathbf{Q}$ is the backward transition matrix whose entry $[\mathbf{Q}]_{i,j}=1/|\mathcal{I}(i)|$ if there is an edge from $j$ to $i$, and 0 otherwise,
%and $\mathbf{I}_n$ is an $n \times n$ identity matrix.
%
%The notation ${(\star)}^{T}$ denotes the matrix transpose.
%
%By using the \emph{Kronecker product} ($\otimes$) and $vec$ operator \cite{Li2010},
%a closed-form of SimRank can be obtained below.
%\begin{equation}  \label{eq:04}
%vec(\mathbf{S})=(1-C)\cdot {(\mathbf{I}_{n^2}-C \cdot(\mathbf{Q} \otimes \mathbf{Q}))}^{-1} \cdot vec(\mathbf{I}_n).
%\end{equation}

%\vspace{1pt} \noindent \textbf{Remarks.} \
%Both of the two SimRank representations follow the consistent semantic meaning except for a slight difference in formulating the base case
%that ``every node is maximally similar to itself''.
%The iterative form of SimRank suggests that the similarity of any node with itself is always maximally~1,
%whereas the matrix form of SimRank ensures that the similarity of a node with itself is maximal (but is not necessarily 1).
%
%which has two parts:
%\setlength\arraycolsep{1pt}
%\begin{eqnarray*}
%  {[{\mathbf{S}}]}_{a,a} & = & C\cdot {[\mathbf{Q} \cdot {{\mathbf{S}}}\cdot {{\mathbf{Q}}^{T}} ]}_{a,a} + (1-C) \cdot [{\mathbf{I}}_n]_{a,a} \\
%  & = & C\cdot {[\mathbf{Q} \cdot {{\mathbf{S}}}\cdot {{\mathbf{Q}}^{T}} ]}_{(a,a)} + (1-C)
%\end{eqnarray*}
%The consistency of the two forms is discussed in \cite{Li2010}.
%
%In this paper, for ease of presentation,
%our incremental techniques are mainly based on the matrix form \cite{Li2010} of SimRank.
%The similar incremental paradigm can be ported to computing the iterative form \cite{Jeh2002} of SimRank.
%
%
%

Despite the high precision of Li \etal\!\!'s SimRank model,
the existing incremental approach of Li \etal \cite{Li2010} for updating SimRank does not always obtain the correct solution $\mathbf{S}$ to Eq.\eqref{eq:03a}.
(Please refer to Appendix~\ref{app:01} for theoretical explanations).

Table~\ref{tab:01} lists the notations used in this article.
\begin{table}\small
\begin{tabular}{c|p{6cm}}
  \hline
\textbf{Symbol}             & \textbf{Description} \\
  \hline
$n$                 & number of nodes in old graph $G$ \\
$m$                 & number of edges in old graph $G$ \\
${d_i}$             & in-degree of node $i$ in old graph $G$ \\
$d$                 & average in-degree of graph $G$ \\
$C$                 & damping factor ($0<C<1$) \\
$K$                 & iteration number \\
$\mathbf{e}_i$      & $n \times 1$ unit vector with a 1 in the $i$-th entry and 0s elsewhere \\
$\mathbf{Q} / \tilde{\mathbf{Q}}$        & old/new (backward) transition matrix \\
$\mathbf{S} / \tilde{\mathbf{S}}$        & old/new SimRank matrix \\
$\mathbf{I}_n$        & $n \times n$ identity matrix \\
${\mathbf{X}}^T$  &  transpose of matrix $\mathbf{X}$ \\
${[\mathbf{X}]}_{i,\star}$  &  $i$-th row of matrix $\mathbf{X}$ \\
${[\mathbf{X}]}_{\star,j}$  &  $j$-th column of matrix $\mathbf{X}$ \\
${[\mathbf{X}]}_{i,j}$  &  $(i,j)$-th entry of matrix $\mathbf{X}$ \\
  \hline
\end{tabular}
\caption{Symbol and Description} \label{tab:01}
\end{table}
%
%\section{Our Incremental Solution} \label{sec:04}
%
\section{Edge Update without node insertions} \label{sec:04}
%
In this section, we consider edge update that does not accompany new node insertions,
\ie the insertion
%\footnote{Due to many commonalities of ``insertion'' and ``deletion'', we will mainly focus on ``insertion'' here, and briefly summarize ``deletion'' in Subsection~\ref{sec:04f}.}
of new edge $(i,j)$ into $G=(V,E)$ with $i \in V$ and $j \in V$.
%In this case, the inserted edge $(i,j)$ would not incur new node insertion.
In this case, the new SimRank matrix $\mathbf{\tilde{S}}$ and the old one $\mathbf{S}$ are of the same size.
As such, it makes sense to denote the SimRank change $\mathbf{\Delta {S}} $ as $\mathbf{\tilde{S}} -\mathbf{S}$.
%
%\footnote{
%As will be seen in Section~\ref{sec:06},
%the inserted edge $(i,j)$ that accompany node insertions cannot keep the same size of the new $\mathbf{\tilde{S}}$ and old $\mathbf{S}$. Thus,
%$\mathbf{\tilde{S}} -\mathbf{S}$ makes no sense in such cases. % of edge updates.
%}
%in this section. % the case (C1).

%We first theoretically show that Li \etal\!\!'s incremental way \cite{Li2010} may miss some eigen-information for computing SimRank.
%We  propose our incremental method to handle unit update (\ie a single edge insertion or deletion).
%\subsection{A Fly in the Ointment in \cite{Li2010}}
%\subsection{Our Incremental Method for Unit Update}
%
%For each case, we characterize the SimRank update matrix $\mathbf{\Delta S}$ \wrt every link update as a rank-one Sylvester matrix equation.
%This produces a fast dynamical SimRank algorithm that entails $O(Kn^2)$ time and $O(n^2)$ memory in the worst case to update $n^2$ pairs of similarities for $K$ iterations. %(Section
%
%We now propose our incremental techniques for computing SimRank,
%with the focus on handling \emph{unit update} (\ie a single edge insertion or deletion).
%%As will be depicted in Section \ref{sec:05},
%Since \emph{batch update} (\ie a list of link insertions and deletions mixed together)
%can be decomposed into a sequence of unit updates,
%unit update plays a vital role in our incremental method.
%
%The main idea of our solution is based on two methods.
%
%(i) We first show that SimRank update matrix $\mathbf{\Delta S} \in \mathbb{R}^{n \times n}$ can be characterized as \emph{a rank-one Sylvester matrix equation}\footnote{Given the matrices $\mathbf{A},\mathbf{B},\mathbf{C} \in \mathbb{R}^{n \times n}$,
%the Sylvester matrix equation in terms of $\mathbf{X} \in \mathbb{R}^{n \times n}$ takes the form:
%$\mathbf{X}=\mathbf{A}\cdot \mathbf{X} \cdot \mathbf{B} + \mathbf{C}$.
%When $\mathbf{C}$ is a rank-$\alpha \ (\le n)$ matrix,
%we call it \emph{the rank-$\alpha$ Sylvester equation}.}.
%By leveraging the rank-one structure of the matrix,
%we provide a novel efficient paradigm for incrementally computing $\mathbf{\Delta S}$,
%which only involves \emph{matrix-vector} and \emph{vector-vector} multiplications,
%as opposed to \emph{matrix-matrix} multiplications to directly compute the new SimRank matrix $\tilde{\mathbf{S}}$.
%
%(ii) In light of our representation of $\mathbf{\Delta S}$,
%we then identify the ``affected areas'' of $\mathbf{\Delta S}$ in response to link update $\mathbf{\Delta Q}$,
%and devise an effective pruning strategy to skip unnecessary similarity recomputations for link updates.
%%
%\subsection{Notations \& Framework} %\label{sec:04a}
%%
%Throughout the paper, the following notations are used.
%Before detailing our two methods in the subsections below,
%we introduce the following notations.
%
%\begin{itemize}
%  \item
%  .
%  \item
%  $\mathbf{e}_i$ is the $n \times 1$ unit vector with a 1 in the $i$-th entry and 0s elsewhere.
%  \item
%  ${d_i}$ is the in-degree of node $i$ in the old graph $G$.
%  \item
%  For any matrix $\mathbf{X}$,
% we denote by ${[\mathbf{X}]}_{i,\star}$ the $i$-th row of $\mathbf{X}$, %${[\mathbf{X}]}_{i,j}$ denotes the $(i,j)$-entry of $\mathbf{X}$,
% and ${[\mathbf{X}]}_{\star,j}$ the $j$-th column of $\mathbf{X}$.
%\end{itemize}
%%\vspace{3pt} \noindent \textbf{Main Idea.} \
%
%
%The rest of this section is structured as follows:
%
%Subsections~\ref{sec:04b}--\ref{sec:04e} deal with four cases (C1)--(C4) of unit insertion.
%Subsection~\ref{sec:04f} considers unit deletion.
%Finally, Subsection~\ref{sec:04g} presents a complete algorithm for unit update, by integrating all the above techniques.
%%
%\subsection{Inserting an edge $(i,j)$ with $i \in V$ and $j \in V$} \label{sec:04b}
%%
%In this subsection, let us consider the case (C1), \ie the insertion of an edge $(i,j)$ with $i \in V$ and $j \in V$.
%In this case, the inserted edge $(i,j)$ would not incur new node insertion.
%Hence, the new SimRank matrix $\mathbf{\tilde{S}}$ and the old one $\mathbf{S}$ are of the same size.
%As such, it makes sense to denote the change $\mathbf{\Delta {S}} $ as $\mathbf{\tilde{S}} -\mathbf{S}$ \footnote{
%As will be seen in Subsections~\ref{sec:04c}--\ref{sec:04e},
%the inserted edge $(i,j)$ in the cases (C2)--(C4) cannot keep the same size of the new $\mathbf{\tilde{S}}$ and old $\mathbf{S}$. Thus,
%$\mathbf{\tilde{S}} -\mathbf{S}$ makes no sense in these cases.
%}
%for the case (C1).
%
%To incrementally compute $\mathbf{\Delta {S}} $,
Below we first introduce the big picture of our main idea, and then present rigorous justifications and proofs. %mathematical
%%
%\subsection{Characterizing $\mathbf{\Delta S}$ via Rank-One Sylvester Equation} %\label{sec:04a}
%%
%We first give the big picture, followed by rigorous proofs.
%
%\vspace{2pt} \noindent \textbf{Main Idea.} \
%
\subsection{The main idea} %$ to characterize $\mathbf{\Delta {S}} $}
%
For each edge $(i,j)$ insertion,
we can show that $\mathbf{\Delta Q}$ is a \emph{rank-one} matrix,
\ie there exist two column vectors $\mathbf{u},\mathbf{v} \in \mathbb{R}^{n \times 1} $ such that
$\mathbf{\Delta Q} \in \mathbb{R}^{n \times n}$ can be decomposed into the {outer product} of $\mathbf{u}$ and $\mathbf{v}$ as follows:
%
%\footnote{The \emph{outer product} of the vectors $\mathbf{x}, \mathbf{y} \in \mathbb{R}^{n \times 1}$ is an $n\times n$ rank-1 matrix $\mathbf{x} \cdot \mathbf{y}^T$,
%in contrast with the \emph{inner product} $\mathbf{x}^T \cdot \mathbf{y}$, which is a scalar.}
\begin{equation} \label{eq:12}
  \mathbf{\Delta Q} = \mathbf{u} \cdot \mathbf{v}^T.  %\footnote{The explict expression of $\mathbf{u}$ and $\mathbf{v}$ will be given after a few discussions.}
\end{equation}

Based on Eq.\eqref{eq:12},
we then have an opportunity to efficiently compute $\mathbf{\Delta S}$,
by characterizing it as
\begin{equation} \label{eq:13}
  \mathbf{\Delta S} = \mathbf{M} + \mathbf{M}^T, %% \qquad (\exists \mathbf{v},\mathbf{u})
\end{equation}
where the auxiliary matrix $\mathbf{M}\in \mathbb{R}^{n \times n}$ satisfies the following \emph{rank-one} Sylvester equation:
\begin{equation} \label{eq:14}
\mathbf{M} = C \cdot \tilde{\mathbf{Q}} \cdot \mathbf{M} \cdot \tilde{\mathbf{Q}}^T + C \cdot \mathbf{u} \cdot \mathbf{w}^T.
\end{equation}
Here, $\mathbf{u}, \mathbf{w}$ are two obtainable column vectors:
$\mathbf{u}$ can be derived from Eq.\eqref{eq:12},
and $\mathbf{w}$ can be described by the old $\mathbf{Q}$ and $\mathbf{S}$
(we will provide their exact expressions later after some discussions);
%\footnote{The expression of $\mathbf{w}$ will be provided };
and $\tilde{\mathbf{Q}}=\mathbf{Q} + \mathbf{\Delta Q}$.

Thus, computing $\mathbf{\Delta S}$ boils down to solving $\mathbf{M}$ in Eq.\eqref{eq:14}.
The main advantage of solving $\mathbf{M}$ via Eq.\eqref{eq:14},
as compared to directly computing the new scores $\tilde{\mathbf{S}}$ via SimRank formula
\begin{equation} \label{eq:15}
\tilde{\mathbf{S}} = C \cdot \tilde{\mathbf{Q}} \cdot \tilde{\mathbf{S}} \cdot \tilde{\mathbf{Q}}^T + (1-C) \cdot \mathbf{I}_n,
\end{equation}
is the high computational efficiency.
More specifically,
solving $\tilde{\mathbf{S}}$ via Eq.\eqref{eq:15} needs expensive \emph{matrix-matrix} multiplications,
whereas solving $\mathbf{M}$ via Eq.\eqref{eq:14} involves only \emph{matrix-vector} and \emph{vector-vector} multiplications,
which is a substantial improvement achieved by our observation that $(C \cdot \mathbf{u} \mathbf{w}^T) \in \mathbb{R}^{n \times n}$ in Eq.\eqref{eq:14} is a \emph{rank-one} matrix,
as opposed to the (full) \emph{rank-$n$} matrix $(1-C) \cdot \mathbf{I}_n$ in Eq.\eqref{eq:15}.
To further elaborate on this,
we can readily convert the recursive forms of Eqs.\eqref{eq:14} and \eqref{eq:15}, respectively, into the series forms:
%\footnote{One can readily verify that if $\mathbf{X}=\sum_{k=0}^{\infty }{{{\mathbf{A}}^{k}}\cdot \mathbf{C}\cdot {{\mathbf{B}}^{k}}}$ is a convergent matrix series,
%it is the solution of the Sylvester equation $\mathbf{X}=\mathbf{A}\cdot \mathbf{X}\cdot \mathbf{B}+\mathbf{C}$.}
\setlength\arraycolsep{2pt}
\begin{eqnarray}
\mathbf{M} & = & \sum\nolimits_{k=0}^{\infty }{{{C}^{k+1}}\cdot {{\tilde{\mathbf{Q}}}^{k}}\cdot \mathbf{u}\cdot {{\mathbf{w}}^{T}}\cdot {{({{\tilde{\mathbf{Q}}}^{T}})}^{k}}}, \label{eq:16} \\
\tilde{\mathbf{S}} & = & (1-C)\cdot \sum\nolimits_{k=0}^{\infty }{{{C}^{k}}\cdot {{\tilde{\mathbf{Q}}}^{k}}\cdot \mathbf{I}_n \cdot {{({{\tilde{\mathbf{Q}}}^{T}})}^{k}}}. \label{eq:17}
\end{eqnarray}

To compute the sums in Eq.\eqref{eq:16} for $\mathbf{M}$,
a conventional way is to memoize
$\mathbf{M}_0 \leftarrow C \cdot \mathbf{u}\cdot {{\mathbf{w}}^{T}}$ first
(where the intermediate result $\mathbf{M}_0$ is an $n \times n$ matrix),
and then iterate as
\[\mathbf{M}_{k+1} \leftarrow \mathbf{M}_{0} + C \cdot {\tilde{\mathbf{Q}}} \cdot \mathbf{M}_{k}  \cdot {\tilde{\mathbf{Q}}}^T, \quad (k=0,1,2,\cdots)\]
which involves expensive \emph{matrix-matrix} multiplications (\eg ${\tilde{\mathbf{Q}}} \cdot \mathbf{M}_{k}$).
In contrast,
our method takes advantage of the \emph{rank-one} structure of $\mathbf{u}\cdot {{\mathbf{w}}^{T}}$ to compute the sums in Eq.\eqref{eq:16} for $\mathbf{M}$,
by converting the conventional \emph{matrix-matrix} multiplications ${\tilde{\mathbf{Q}}} \cdot (\mathbf{u} {{\mathbf{w}}^{T}}) \cdot {\tilde{\mathbf{Q}}}^T$
into only \emph{matrix-vector} and \emph{vector-vector} multiplications $({\tilde{\mathbf{Q}}} \mathbf{u}) \cdot (\tilde{\mathbf{Q}} {{\mathbf{w}}})^T$.
To be specific, we leverage two vectors ${\bm{\xi}}_{k}, {\bm{\eta}}_{k}$,
and iteratively compute Eq.\eqref{eq:16} as
%the \emph{vectors} ${\bm{\xi}}_{k}, {\bm{\eta}}_{k}$ via the following iteration:
\begin{eqnarray}
\nonumber & &   \textrm{initialize } {{\bm{\xi }}_{0}}\leftarrow C\cdot \mathbf{u},\quad {{\bm{\eta }}_{0}}\leftarrow \mathbf{w},\quad {{\mathbf{M}}_{0}}\leftarrow C \cdot \mathbf{u}\cdot {{\mathbf{w}}^{T}}   \\
\nonumber & &   \textbf{for }  k=0,1,2,\cdots  \\
\nonumber & &   \quad {{\bm{\xi }}_{k+1}}\leftarrow C\cdot \mathbf{\tilde{Q}}\cdot {{\bm{\xi }}_{k}},\quad {{\bm{\eta }}_{k+1}}\leftarrow \mathbf{\tilde{Q}}\cdot {{\bm{\eta }}_{k}} \\
\label{eq:16a} & &   \quad {{\mathbf{M}}_{k+1}}\leftarrow {{\bm{\xi }}_{k+1}}\cdot \bm{\eta }_{k+1}^{T}+{{\mathbf{M}}_{k}}
\end{eqnarray}
%\begin{description} \setlength{\itemsep}{2pt}
%  \item
%  initialize ${{\bm{\xi }}_{0}}\leftarrow C\cdot \mathbf{u},\quad {{\bm{\eta }}_{0}}\leftarrow \mathbf{w},\quad {{\mathbf{M}}_{0}}\leftarrow C \cdot \mathbf{u}\cdot {{\mathbf{w}}^{T}}$
%  \item
% for $k=0,1,2,\cdots $
% \item
% $\quad {{\bm{\xi }}_{k+1}}\leftarrow C\cdot \mathbf{\tilde{Q}}\cdot {{\bm{\xi }}_{k}},\quad {{\bm{\eta }}_{k+1}}\leftarrow \mathbf{\tilde{Q}}\cdot {{\bm{\eta }}_{k}}$
% \item
% $\quad {{\mathbf{M}}_{k+1}}\leftarrow {{\bm{\xi }}_{k+1}}\cdot \bm{\eta }_{k+1}^{T}+{{\mathbf{M}}_{k}}$
%\end{description}
so that \emph{matrix-matrix} multiplications are safely avoided.
%
%which only requires \emph{matrix-vector} multiplications (\eg $\mathbf{\tilde{Q}}\cdot {{\bm{\xi }}_{k}}$)
%and \emph{vector-vector} multiplications (\eg ${{\bm{\xi }}_{k+1}}\cdot \bm{\eta }_{k+1}^{T}$),
%without the need to perform \emph{matrix-matrix} multiplications.
%%        \begin{remark}
%%        It is worth mentioning that our above method is solely suitable for efficiently computing $\mathbf{M}$ in Eq.\eqref{eq:16},
%%        but not applicable to accelerating $\tilde{\mathbf{S}}$ computation in Eq.\eqref{eq:17}.
%%        This is because $\mathbf{I}_n$ is a (full) rank-$n$ matrix that cannot be decomposed into the outer product of two vectors.
%%        Thus, our method is particularly tailored to improve the \emph{incremental} computation of $\mathbf{\Delta S}$ via Eq.\eqref{eq:14},
%%        rather than the \emph{batch} computation of $\tilde{\mathbf{S}}$ via Eq.\eqref{eq:15}.
%%        %%there is an opportunity to use \emph{matrix-vector} multiplications.
%%        \end{remark}
%\vspace{3pt} \noindent \textbf{Finding $\mathbf{u}, \mathbf{v},\mathbf{w}$ for Eqs.\eqref{eq:12} and \eqref{eq:14}.} \
%
\subsection{Describing $\mathbf{u}, \mathbf{v},\mathbf{w}$ in Eqs.\eqref{eq:12} and \eqref{eq:14}}
%
%%        In computing $\mathbf{\Delta S}$, we address two problems:
%%        One is to obtain the vectors $\mathbf{u}, \mathbf{v}$ in Eq.\eqref{eq:12} from the \emph{rank-one} decomposition of $\mathbf{\Delta Q}$.
%%        The other task is the description of the vector $\mathbf{w}$ in Eq.\eqref{eq:14} in terms of the old matrices $\mathbf{Q}$ and $\mathbf{S}$,
%%        in order to guarantee that Eq.\eqref{eq:14} is a \emph{rank-one} Sylvester equation.
%%        %We shall address these problems shortly in the next subsection.

To obtain $ \mathbf{u}$ and $\mathbf{v}$ in Eq.\eqref{eq:12} at a low cost,
we have the following theorem.

\begin{theorem} \label{thm:01}
Given an old digraph $G=(V,E)$,
if there is a new edge $(i,j)$ with $i \in V$ and $j \in V$ to be added to $G$,
then the change to $\mathbf{Q}$ is an $n \times n$ rank-one matrix, \ie
$\mathbf{\Delta Q} = \mathbf{u} \cdot \mathbf{v}^T$,
where
\begin{equation} \label{eq:18} \small
\mathbf{u}=\left\{ \begin{matrix}
   {{\mathbf{e}}_{j}} & \left( {{d}_{j}}=0 \right)  \\
   \tfrac{1}{{{d}_{j}}+1}{{\mathbf{e}}_{j}} & \left( {{d}_{j}}>0 \right)  \\
\end{matrix} \right.
, \quad
\mathbf{v}=\left\{ \begin{matrix}
   {{\mathbf{e}}_{i}} & \left( {{d}_{j}}=0 \right)  \\
   {{\mathbf{e}}_{i}}-{{[\mathbf{Q}]}_{j,\star}^T} & \left( {{d}_{j}}>0 \right)  \\
\end{matrix} \right.
\end{equation}
\qed
%
%If there is an edge $(i,j)$ deleted from $G$,
%then the change in $\mathbf{Q}$ can be decomposed as $\mathbf{\Delta Q} = \mathbf{u} \cdot \mathbf{v}^T$,
%where
%\begin{equation} \label{eq:19} \small
%\mathbf{u}=\left\{ \begin{matrix}
%   {{\mathbf{e}}_{j}} & \left( {{d}_{j}}=1 \right)  \\
%   \tfrac{1}{{{d}_{j}}-1}{{\mathbf{e}}_{j}} & \left( {{d}_{j}}>1 \right)  \\
%\end{matrix} \right.
%, \quad
%\mathbf{v}=\left\{ \begin{matrix}
%   {-{\mathbf{e}}_{i}} & \left( {{d}_{j}}=1 \right)  \\
%   {{[\mathbf{Q}]}_{j,\star}^T}-{{\mathbf{e}}_{i}} & \left( {{d}_{j}}>1 \right)  \\
%\end{matrix} \right.
%\end{equation}
\end{theorem}

(Please refer to Appendix~\ref{app:02a} for the proof of Theorem~\ref{thm:01}, and Appendix~\ref{app:03b} for an example.)

%For each insertion of case (C1), % every link update,
Theorem \ref{thm:01} suggests that the change $\mathbf{\Delta Q}$ is an $n\times n$ \emph{rank-one} matrix,
which can be obtain in only constant time from $d_j$ and ${{[\mathbf{Q}]}_{j,\star}^T}$.
%
%has a very special structure --- the $n\times n$ \emph{rank-one} matrix.
%More importantly, it finds a rank-one decomposition for $\mathbf{\Delta Q}$,
%by expressing the vectors $\mathbf{u}$ and $\mathbf{v}$ in terms of $d_j$ and ${{[\mathbf{Q}]}_{j,\star}^T}$.
In light of this, %the rank-one decomposition of $\mathbf{\Delta Q}$,
%we next show that
%
%Note that such a rank-one decomposition is not unique because, for any scalar $\lambda \neq 0$,
%the vectors $\mathbf{u}'\triangleq\lambda \cdot \mathbf{u}$ and $\mathbf{v}'\triangleq\frac{\mathbf{v}}{\lambda}$ can be another rank-one decomposition of $\mathbf{\Delta Q}$.
%However,
%for any $\mathbf{u}$ and $\mathbf{v}$ that satisfy Eq.\eqref{eq:12},
%there exists a vector $\mathbf{w}$ such that Eq.\eqref{eq:14} is a \emph{rank-one} Sylvester equation.
%
%Capitalizing on Theorem \ref{thm:01},
we next describe $\mathbf{w}$ in Eq.\eqref{eq:14} in terms of the old $\mathbf{Q}$ and $\mathbf{S}$ such that Eq.\eqref{eq:14} is a \emph{rank-one} Sylvester equation.

\begin{theorem} \label{thm:02}
Let $(i,j)_{i \in V, \ j \in V}$ be a new edge to be added to $G$ (\Resp an existing edge to be deleted from $G$).
Let $\mathbf{u}$ and $\mathbf{v}$ be the rank-one decomposition of $\mathbf{\Delta Q} = \mathbf{u} \cdot \mathbf{v}^T$. % via Theorem~\ref{thm:01}.
Then, (i) there exists a vector $\mathbf{w}=\mathbf{y}+\tfrac{\lambda }{2}\mathbf{u}$
with
\begin{equation} \label{eq:20}
  \mathbf{y}=\mathbf{Q}\cdot \mathbf{z},\quad \lambda ={{\mathbf{v}}^{T}}\cdot \mathbf{z},\quad \mathbf{z}=\mathbf{S}\cdot \mathbf{v}
\end{equation}
such that Eq.\eqref{eq:14} is the \emph{rank-one} Sylvester equation.

(ii) Utilizing the solution $\mathbf{M}$ to Eq.\eqref{eq:14},
the SimRank update matrix $\mathbf{\Delta S}$ can be represented by Eq.\eqref{eq:13}. \qed
\end{theorem}

(The proof of Theorem~\ref{thm:02} is in Appendix~\ref{app:02b}.)

Theorem~\ref{thm:02} provides an elegant expression of $\mathbf{w}$ in Eq.\eqref{eq:14}.
To be precise,
given $\mathbf{Q}$ and $\mathbf{S}$ in the old graph $G$, and an edge $(i,j)$ inserted to $G$,
one can find $\mathbf{u}$ and $\mathbf{v}$ via Theorem~\ref{thm:01} first,
and then resort to Theorem~\ref{thm:02} to compute $\mathbf{w}$ from $\mathbf{u},\mathbf{v},\mathbf{Q},\mathbf{S}$.
Due to the existence of the vector $\mathbf{w}$,
it can be guaranteed that the Sylvester equation \eqref{eq:14} is \emph{rank-one}.
Henceforth, our aforementioned method can be employed to iteratively compute $\mathbf{M}$ in Eq.\eqref{eq:16},
requiring no \emph{matrix-matrix} multiplications.
%
%\vspace{3pt} \noindent \textbf{Computing $\mathbf{\Delta S}$.} \
%
\subsection{Characterizing $\mathbf{\Delta S}$}
%
%%        Obtaining $\mathbf{w}$ from Theorem \ref{thm:02} is intended to speed up the computation of $\mathbf{\Delta S}$.
%%        Indeed, when edge $(i,j)_{i \in V, j \in V}$ is added,
%%        the \emph{whole} process of computing $\mathbf{\Delta S}$ in Eq.\eqref{eq:13}, given $\mathbf{Q}$ and $\mathbf{S}$, needs no \emph{matrix-matrix} multiplications.
%%        Precisely,
%%        the computation of $\mathbf{\Delta S}$ consists of two phases:
%%        (i) Given $\mathbf{Q}$ and $\mathbf{S}$, we compute $\mathbf{w}$ from Theorems~\ref{thm:01} and~\ref{thm:02}.
%%        This phase includes only the matrix-vector multiplications (\eg $\mathbf{Q}  \mathbf{z}, \mathbf{S}  \mathbf{v}$),
%%        the inner product of vectors (\eg ${{\mathbf{v}}^{T}}  \mathbf{z}$),
%%        and the vector scaling and additions, \ie SAXPY (\eg $ \mathbf{y}+\tfrac{\lambda }{2}\mathbf{u}$).
%%        (ii) Given $\mathbf{w}$, we compute $\mathbf{M}$ via Eq.\eqref{eq:16}.
%%        In this phase,
%%        our novel iterative model for Eq.\eqref{eq:16} can circumvent the \emph{matrix-matrix} multiplications.
%%        Thus, taking (i) and (ii) together,
%%        it suffices to use only the \emph{matrix-vector} and \emph{vector-vector} operations in the whole process of $\mathbf{\Delta S}$ computation.

Leveraging Theorems \ref{thm:01} and \ref{thm:02},
we next characterize the SimRank change $\mathbf{\Delta S}$. %,based on the following theorem.

\begin{theorem} \label{thm:03}
If there is a new edge $(i,j)$ with $i \in V$ and $j \in V$ to be inserted to $G$,
then the SimRank change $\mathbf{\Delta S}$ can be characterized as
\[
\mathbf{\Delta S}=\mathbf{M}+{{\mathbf{M}}^{T}} \quad \textrm{ with }
\]
\begin{equation}  \label{eq:29c}
\mathbf{M}=\sum\nolimits_{k=0}^{\infty }{{{C}^{k+1}}\cdot {{{\mathbf{\tilde{Q}}}}^{k}}\cdot {{\mathbf{e}}_{j}}\cdot {{\bm{\gamma }}^{T}}\cdot {{({{{\mathbf{\tilde{Q}}}}^{T}})}^{k}}},
\end{equation}
where the auxiliary vector $\bm{\gamma }$ is obtained as follows:

(i) when ${{d}_{j}}=0$,
\begin{equation}\label{eq:29aa}
\bm{\gamma } = \mathbf{Q}\cdot {{[\mathbf{S}]}_{\star,i}}+\tfrac{1}{2}{{[\mathbf{S}]}_{i,i}}\cdot {{\mathbf{e}}_{j}}
\end{equation}

(ii) when ${{d}_{j}}>0$,
\begin{equation}\label{eq:29bb}
\scalebox{0.92}{$
\bm{\gamma } = \tfrac{1}{({{d}_{j}}+1)} \left( \mathbf{Q} {{[\mathbf{S}]}_{\star,i}}-\tfrac{1}{C} {{[\mathbf{S}]}_{\star,j}}+( \tfrac{\lambda }{2\left( {{d}_{j}}+1 \right)}+ \tfrac{1}{C}-1 ) {{\mathbf{e}}_{j}} \right)$}
\end{equation}
%
%(i) For the edge insertion, $\bm{\gamma } = $
%\begin{equation}\label{eq:29}
%\scalebox{0.85}{$
%\left\{ \begin{array}{lc}
%    \mathbf{Q}\cdot {{[\mathbf{S}]}_{\star,i}}+\frac{1}{2}{{[\mathbf{S}]}_{i,i}}\cdot {{\mathbf{e}}_{j}}  & ({{d}_{j}}=0)  \\
%   \tfrac{1}{({{d}_{j}}+1)} \left( \mathbf{Q}\cdot {{[\mathbf{S}]}_{\star,i}}-\frac{1}{C}\cdot {{[\mathbf{S}]}_{\star,j}}+( \frac{\lambda }{2\left( {{d}_{j}}+1 \right)}+ \frac{1}{C}-1 )\cdot {{\mathbf{e}}_{j}} \right) & ({{d}_{j}}>0)  \\
%\end{array} \right.
%$}
%\end{equation}
%
%(ii) For the edge deletion, $\bm{\gamma } = $
%\begin{equation}\label{eq:29a}
%\scalebox{0.85}{$
%\left\{ \begin{array}{lc}
%    -\mathbf{Q}\cdot {{[\mathbf{S}]}_{\star,i}}+\frac{1}{2}{{[\mathbf{S}]}_{i,i}}\cdot {{\mathbf{e}}_{j}}  & ({{d}_{j}}=1)  \\
%   \tfrac{1}{({{d}_{j}}-1)} \left(\frac{1}{C}\cdot {{[\mathbf{S}]}_{\star,j}} - \mathbf{Q}\cdot {{[\mathbf{S}]}_{\star,i}} +( \frac{\lambda }{2\left( {{d}_{j}}-1 \right)}- \frac{1}{C}+1 )\cdot {{\mathbf{e}}_{j}} \right) & ({{d}_{j}}>1)  \\
%\end{array} \right.
%$}
%\end{equation}
and the scalar $\lambda$ can be derived from
\begin{equation} \label{eq:29b}
\lambda = {{[\mathbf{S}]}_{i,i}}+\tfrac{1}{C} \cdot {[\mathbf{S}]}_{j,j}-2\cdot {{[\mathbf{Q}]}_{j,\star}}\cdot {{[\mathbf{S}]}_{\star,i}} - \tfrac{1}{C} +1.
\end{equation}
\qed
\end{theorem}

(The proof of Theorem~\ref{thm:03} is in Appendix~\ref{app:02c}.)

%For unit insertion of the case (C1), % link update,
Theorem \ref{thm:03} provides an efficient method to compute the incremental SimRank matrix $\mathbf{\Delta S}$,
by utilizing the previous information of $\mathbf{Q}$ and $\mathbf{S}$, % in the old graph $G$,
as opposed to \cite{Li2010} that requires to maintain the incremental SVD.

%%        To achieve even higher efficiency for computing $\mathbf{\Delta S}$ by Theorem \ref{thm:03},
%%        two extra methods are worth mentioning:
%%        (i) Note that, by viewing the matrix $\mathbf{Q}$ as a stack of row vectors,
%%        the $j$-th row of the term $(\mathbf{Q}\cdot {{[\mathbf{S}]}_{\star,i}})$ in Eqs.\eqref{eq:29aa} and \eqref{eq:29bb} is the inner product ${{[\mathbf{Q}]}_{j,\star}}\cdot {{[\mathbf{S}]}_{\star,i}}$,
%%        which is the term in Eq.\eqref{eq:29b}.
%%        Thus, the value ${[\mathbf{Q}\cdot {{[\mathbf{S}]}_{\star,i}}]}_{j,\star}$, once computed, can be reused to compute ${{[\mathbf{Q}]}_{j,\star}}\cdot {{[\mathbf{S}]}_{\star,i}}$ in $\lambda$.
%%        (ii) As suggested earlier,
%%        computing the matrix series for $\mathbf{M}$ needs no matrix-matrix multiplications,
%%        but involves the matrix-vector multiplications (\eg ${{\bm{\eta }}_{k+1}}\leftarrow \mathbf{\tilde{Q}}\cdot {{\bm{\eta }}_{k}}$).
%%        Since $\tilde{\mathbf{Q}} = \mathbf{Q} + \mathbf{u} \cdot \mathbf{v}^T$ via Theorem \ref{thm:01},
%%        we notice that $\mathbf{\tilde{Q}}\cdot {{\bm{\eta }}_{k}}$ can be computed more efficiently, with no need to memoize $\tilde{\mathbf{Q}}$ in extra memory space, as follows:
%%        \[\mathbf{\tilde{Q}}\cdot {{\bm{\eta }}_{k}} = \mathbf{Q}\cdot {{\bm{\eta }}_{k}} + (\mathbf{v}^T\cdot {{\bm{\eta }}_{k}}) \cdot \mathbf{u}.\]
%
\subsection{Deleting an edge $(i,j)_{i \in V, \ j \in V}$ from $G=(V,E)$} \label{sec:04f}
%
For an edge deletion, we next propose a Theorem~\ref{thm:03}-like technique that can efficiently update SimRanks.
\begin{theorem} \label{thm:09}
%If there is an edge $(i,j)$ inserted into $G$,
%then the change in $\mathbf{Q}$ is an $n \times n$ rank-one matrix, \ie
%$\mathbf{\Delta Q} = \mathbf{u} \cdot \mathbf{v}^T$,
%where
%\begin{equation} \label{eq:18} \small
%\mathbf{u}=\left\{ \begin{matrix}
%   {{\mathbf{e}}_{j}} & \left( {{d}_{j}}=0 \right)  \\
%   \tfrac{1}{{{d}_{j}}+1}{{\mathbf{e}}_{j}} & \left( {{d}_{j}}>0 \right)  \\
%\end{matrix} \right.
%, \quad
%\mathbf{v}=\left\{ \begin{matrix}
%   {{\mathbf{e}}_{i}} & \left( {{d}_{j}}=0 \right)  \\
%   {{\mathbf{e}}_{i}}-{{[\mathbf{Q}]}_{j,\star}^T} & \left( {{d}_{j}}>0 \right)  \\
%\end{matrix} \right.
%\end{equation}
%
When an edge $(i,j)_{i \in V, \ j \in V}$ is deleted from $G=(V,E)$,
the changes to $\mathbf{Q}$ is a rank-one matrix, which can be described as $\mathbf{\Delta Q} = \mathbf{u} \cdot \mathbf{v}^T$,
where
\begin{equation*} \label{eq:19} %\small
\mathbf{u}=\left\{ \begin{matrix}
   {{\mathbf{e}}_{j}} & \left( {{d}_{j}}=1 \right)  \\
   \tfrac{1}{{{d}_{j}}-1}{{\mathbf{e}}_{j}} & \left( {{d}_{j}}>1 \right)  \\
\end{matrix} \right.
, \quad
\mathbf{v}=\left\{ \begin{matrix}
   {-{\mathbf{e}}_{i}} & \left( {{d}_{j}}=1 \right)  \\
   {{[\mathbf{Q}]}_{j,\star}^T}-{{\mathbf{e}}_{i}} & \left( {{d}_{j}}>1 \right)  \\
\end{matrix} \right.
\end{equation*}
The changes $\mathbf{\Delta S}$ to SimRank can be characterized as
%\end{theorem}
%
%
%\begin{theorem} %\label{thm:03}
%If there is a new edge $(i,j)$ with $i \in V$ and $j \in V$ to be inserted to $G$,
%then the SimRank change $\mathbf{\Delta S}$ can be characterized as
\[
\mathbf{\Delta S}=\mathbf{M}+{{\mathbf{M}}^{T}} \ \textrm{ with }
\mathbf{M}=\sum\nolimits_{k=0}^{\infty }{{{C}^{k+1}} {{{\mathbf{\tilde{Q}}}}^{k}} {{\mathbf{e}}_{j}} {{\bm{\gamma }}^{T}} {{({{{\mathbf{\tilde{Q}}}}^{T}})}^{k}}},
\]
%\begin{equation*} % \label{eq:29c}
%
%\end{equation*}
where the auxiliary vector $\bm{\gamma }:=$ %is obtained as follows:
%
%(i) For the edge insertion, $\bm{\gamma } = $
%\begin{equation}\label{eq:29}
%\scalebox{0.85}{$
%\left\{ \begin{array}{lc}
%    \mathbf{Q}\cdot {{[\mathbf{S}]}_{\star,i}}+\frac{1}{2}{{[\mathbf{S}]}_{i,i}}\cdot {{\mathbf{e}}_{j}}  & ({{d}_{j}}=0)  \\
%   \tfrac{1}{({{d}_{j}}+1)} \left( \mathbf{Q}\cdot {{[\mathbf{S}]}_{\star,i}}-\frac{1}{C}\cdot {{[\mathbf{S}]}_{\star,j}}+( \frac{\lambda }{2\left( {{d}_{j}}+1 \right)}+ \frac{1}{C}-1 )\cdot {{\mathbf{e}}_{j}} \right) & ({{d}_{j}}>0)  \\
%\end{array} \right.
%$}
%\end{equation}
%
%(ii) For the edge deletion, $\bm{\gamma } = $
\begin{equation*} %\label{eq:29a}
\scalebox{0.86}{$
\left\{ \begin{array}{lc}
    -\mathbf{Q}\cdot {{[\mathbf{S}]}_{\star,i}}+\frac{1}{2}{{[\mathbf{S}]}_{i,i}}\cdot {{\mathbf{e}}_{j}}  & ({{d}_{j}}=1)  \\
   \tfrac{1}{({{d}_{j}}-1)} \left(\frac{1}{C}\cdot {{[\mathbf{S}]}_{\star,j}} - \mathbf{Q}\cdot {{[\mathbf{S}]}_{\star,i}} +( \frac{\lambda }{2\left( {{d}_{j}}-1 \right)}- \frac{1}{C}+1 )\cdot {{\mathbf{e}}_{j}} \right) & ({{d}_{j}}>1)  \\
\end{array} \right.
$}
\end{equation*}
and $\lambda:={{[\mathbf{S}]}_{i,i}}+\tfrac{1}{C} \cdot {[\mathbf{S}]}_{j,j}-2\cdot {{[\mathbf{Q}]}_{j,\star}}\cdot {{[\mathbf{S}]}_{\star,i}} - \tfrac{1}{C} +1$. \qed %can be derived from
%\begin{equation*} %\label{eq:29b}
%%\lambda =
%{{[\mathbf{S}]}_{i,i}}+\tfrac{1}{C} \cdot {[\mathbf{S}]}_{j,j}-2\cdot {{[\mathbf{Q}]}_{j,\star}}\cdot {{[\mathbf{S}]}_{\star,i}} - \tfrac{1}{C} +1.
%\end{equation*}
\end{theorem}

(The proof of Theorem~\ref{thm:09} is in Appendix~\ref{app:02d}.)

%(The proof is similar to those of Theorems~\ref{thm:01}--\ref{thm:03}, and is omitted due to space limitations.)
%
%\textfloatsep 1mm plus 1mm \intextsep 1mm plus 1mm
%
\subsection{{\IncUSRone} Algorithm}
%
%\noindent \textbf{Algorithm.} \
%In virtue of Theorem \ref{thm:03},
We present our efficient incremental approach, denoted as {\IncUSRone} (in Appendix~\ref{app:04a}), that supports the edge insertion without accompanying new node insertions. %of the case (C1).
The complexity of \IncUSRone~is bounded by $O(Kn^2)$ time and $O(n^2)$ memory\footnote{In the next sections, we shall substantially reduce its time and memory complexity further.}
in the worst case for updating all $n^2$ pairs of similarities.

(Please refer to Appendix~\ref{app:04a} for a detailed description of {\IncUSRone}, and Appendix~\ref{app:03c} for an example.)

%
\section{Pruning Unnecessary Node-Pairs in $\mathbf{\Delta S}$} \label{sec:05}
%
After the SimRank update matrix $\mathbf{\Delta S}$ has been characterized as a rank-one Sylvester equation,
pruning techniques can further skip node-pairs with unchanged SimRanks in $\mathbf{\Delta S}$ (called ``unaffected areas'').
%to avoid unnecessary recomputation. % for link update.

%%        In practice, we observe that when link updates are small,
%%        affected areas in similarity updates $\mathbf{\Delta S}$ are often small as well.
%%        As demonstrated in Example \ref{eg:04},
%%        many entries in matrix $\mathbf{M}_{K}$ are 0s,
%%        implying that $\mathbf{\Delta S} \ (=\mathbf{M}_{K}+\mathbf{M}_{K}^T)$ is a sparse matrix.
%%        However, it is a grand challenge to identify such ``affected areas'' in $\mathbf{\Delta S}$ in response to link updates.
%%        To address this problem,
%%        we first introduce a nice property of the adjacency matrix:
%%
%%        \begin{lemma} \label{lem:01}
%%        Let $\mathbf{A}$ be an adjacency matrix.
%%        Then ${[\mathbf{A}^k]}_{i,j}$ counts the number of length-$k$ paths from node $i$ to $j$.
%%        \end{lemma}
%%
%%        For example, ${[\mathbf{A}^4]}_{i,j}$ counts the number of paths
%%        $\rho: i \rightarrow \circ \rightarrow \circ \rightarrow \circ \rightarrow j$ in $G$,
%%        with $\circ$ denoting any node.
%%
%%        Lemma \ref{lem:01} can be extended to count the number of ``specific paths'' whose edges are not necessarily in the same direction.
%%        For example,
%%        we can use ${[\mathbf{A} \mathbf{A}^T   \mathbf{A}  \mathbf{A}^T ]}_{i,j}$
%%        to count the paths $\rho: i \rightarrow \circ \leftarrow \circ \rightarrow \circ \leftarrow j$ in $G$,
%%        where $\mathbf{A}$ (\Resp $\mathbf{A}^T$) appears at the positions 1,3 (\Resp 2,4),
%%        corresponding to the positions of $\rightarrow$ (\Resp $\leftarrow$) in $\rho$.
%%
%%        As $\mathbf{Q}$ is the row-normalized matrix of $\mathbf{A}^T$,
%%        we can prove that
%%        ${[\mathbf{Q}^{k} \cdot {(\mathbf{Q}^T)}^{k}]}_{i,j} = 0 \Leftrightarrow {[{(\mathbf{A}^T)}^{k} \cdot {\mathbf{A}}^{k}]}_{i,j} = 0$.
%%        The following corollary is immediate.
%%        \begin{corollary} \label{cor:01}
%%        Given $k=0,1,\cdots$,
%%        the entry ${[\mathbf{Q}^{k} \cdot {(\mathbf{Q}^T)}^{k}]}_{i,j}$ counts the weights of the specific paths
%%        whose left $k$ edges in ``$\leftarrow$'' direction  and right $k$ edges in ``$\rightarrow$'' direction: % as follows:
%%        \begin{equation} \label{eq:37}
%%        \underbrace{i\leftarrow \circ \leftarrow \cdots \leftarrow}_{\textrm{length } k} \bullet \underbrace{\rightarrow \cdots \rightarrow \circ \rightarrow j}_{\textrm{length } k}.
%%        \end{equation}
%%        \end{corollary}
%%
%%        \begin{definition}
%%        We call the paths in Eq.\eqref{eq:37} \emph{the symmetric in-link paths of length $2k$ for node-pair $(i,j)$}.
%%        \end{definition}
%%        By virtue of Eq.\eqref{eq:28},
%%        the recursive form of SimRank Eq.\eqref{eq:03a} naturally leads itself to the following series form:
%%        \begin{equation} \label{eq:38}
%%        {[\mathbf{S}]}_{a,b}  =  (1-C)\cdot \sum\nolimits_{k=0}^{\infty }{{{C}^{k}}\cdot {[ {{{\mathbf{Q}}}^{k}}\cdot  {({{\mathbf{Q}}^{T}})}^{k}}]}_{a,b}.
%%        \end{equation}
%%
%%        Capitalizing on Corollary \ref{cor:01},
%%        Eq.\eqref{eq:38} provides a reinterpretation of SimRank:
%%        ${[\mathbf{S}]}_{a,b}$ is the weighted sum of all in-link paths of length $2k\ (k=0,1,2,\cdots)$ for node-pair $(a,b)$.
%%        The weight $C^k$ in Eq.\eqref{eq:38} is to reduce the contributions of in-link paths with \emph{long} lengths relative to those with \emph{short} ones.
%%        The factor $(1-C)$ aims at normalizing ${[\mathbf{S}]}_{a,b}$ into $[0,1]$ since
%%        \[{\big\|\sum\nolimits_{k=0}^{\infty }{{{C}^{k}}\cdot { {{{\mathbf{Q}}}^{k}}\cdot  {({{\mathbf{Q}}^{T}})}^{k}}}\big\|}_{\max} \le {\sum\nolimits_{k=0}^{\infty }{C^k}} \le \tfrac{1}{1-C}.\]
%%        %
%\vspace{3pt} \noindent \textbf{Affected Areas in $\mathbf{\Delta S}$.} \
%
\subsection{Affected Areas in $\mathbf{\Delta S}$}
%
%%        In light of our interpretation for $\mathbf{S}$ via Eq.\eqref{eq:38},
We next reinterpret the series $\mathbf{M}$ in Theorem \ref{thm:03},
aiming to identify ``affected areas'' in $\mathbf{\Delta S}$.
Due to space limitations,
we mainly focus on the edge insertion case of $d_j>0$.
Other cases have the similar results.

By substituting Eq.\eqref{eq:29bb} back into Eq.\eqref{eq:29c},
we can readily split the series form of $\mathbf{M}$ into three parts:

\vspace{-8pt} \begin{small}
\begin{eqnarray*}
{[\mathbf{M}]}_{a,b}= && \tfrac{1}{{{d}_{j}}+1} \bigg(\underbrace{\sum\nolimits_{k=0}^{\infty }{{{C}^{k+1}} \cdot {{[{{{\mathbf{\tilde{Q}}}}^{k}}]}_{a,j}}  {{[\mathbf{S}]}_{i,\star}}  {{\mathbf{Q}}^{T}}\cdot {[{{({{{\mathbf{\tilde{Q}}}}^{T}})}^{k}}]}_{\star,b}}}_{\text{Part 1}} - \\
&& - \underbrace{\sum\nolimits_{k=0}^{\infty }{{{C}^{k}} {{[{{{\mathbf{\tilde{Q}}}}^{k}}]}_{a,j}}  {{[\mathbf{S}]}_{j,\star}}  {[{{({{{\mathbf{\tilde{Q}}}}^{T}})}^{k}}]}_{\star,b}}}_{\text{Part 2}} + \\
&&  + \mu  \underbrace{\sum\nolimits_{k=0}^{\infty }{{{C}^{k+1}}  {{[{{{\mathbf{\tilde{Q}}}}^{k}}]}_{a,j}}  {{[{{({{{\mathbf{\tilde{Q}}}}^{T}})}^{k}}]}_{j,b}}}}_{\text{Part 3}} \bigg)
\end{eqnarray*}
\end{small}
with the scalar $\mu :=\frac{\lambda }{2\left( {{d}_{j}}+1 \right)}+\frac{1}{C}-1$.

%By Lemma \ref{lem:01} and Corollary \ref{cor:01},
Intuitively, when edge $(i,j)$ is inserted and $d_j>0$,
Part~1 of ${[\mathbf{M}]}_{a,b}$ tallies the weighted sum of the following new paths for node-pair $(a,b)$: % in graph $G\cup\{(i,j)\}$:
\begin{equation} \label{eq:39a}
\scalebox{0.74}{$
\underbrace{\overbrace{a\leftarrow \circ \cdots \circ \leftarrow j}^{{{[{{{\mathbf{\tilde{Q}}}}^{k}}]}_{a,j}}}}_{\text{length }k} \Leftarrow \underbrace{\overbrace{i\leftarrow \circ \cdots \circ \leftarrow \bullet \to \circ \cdots \circ \to \star}^{{{[\mathbf{S}]}_{i,\star}}}}_{\mathclap{\text{all symmetric in-link paths for node-pair }(i,\star)}}\overbrace{\to }^{{{\mathbf{Q}}^{T}}}\underbrace{\overbrace{\blacktriangle \to \cdots \circ \to b}^{{{[{{({{{\mathbf{\tilde{Q}}}}^{T}})}^{k}}]}_{\blacktriangle,b}}}}_{\text{length }k}
$}
\end{equation}

Such paths are the concatenation of four types of sub-paths (as depicted above)
associated with four matrices, respectively, ${{[{{{\mathbf{\tilde{Q}}}}^{k}}]}_{a,j}}, {{[\mathbf{S}]}_{i,\star}}, {{\mathbf{Q}}^{T}},{{[{{({{{\mathbf{\tilde{Q}}}}^{T}})}^{k}}]}_{\blacktriangle,b}} $, plus the inserted edge $j \Leftarrow i$.
When such entire concatenated paths exist in the new graph,
they should be accommodated for assessing the new SimRank ${[\tilde{\mathbf{S}}]}_{a,b}$ in response to the edge insertion $(i,j)$
because our reinterpretation of SimRank indicates that SimRank counts \emph{all} the symmetric in-link paths,
and the entire concatenated paths can prove to be symmetric in-link paths.

Likewise,  Parts 2 and 3 of ${[\mathbf{M}]}_{a,b}$, respectively,
tally the weighted sum of the following paths for pair $(a,b)$:
\begin{equation} \label{eq:39b}
\scalebox{0.9}{$
\underbrace{\overbrace{a\leftarrow \circ \cdots \circ \leftarrow}^{{{[{{{\mathbf{\tilde{Q}}}}^{k}}]}_{a,j}}}}_{\text{length }k} j  \underbrace{\overbrace{\leftarrow \circ \cdots \circ \leftarrow \bullet \to \circ \cdots \circ \to }^{{{[\mathbf{S}]}_{j,\star}}}}_{\mathclap{\text{all symmetric in-link paths for }(j,\star)}} \star \underbrace{\overbrace{\to \cdots \circ \to b}^{{{[{{({{{\mathbf{\tilde{Q}}}}^{T}})}^{k}}]}_{\star,b}}}}_{\text{length }k}
$}
\end{equation}
\begin{equation} \label{eq:39c}
\scalebox{0.95}{$
\underbrace{\overbrace{a\leftarrow \circ \cdots \circ \leftarrow}^{{{[{{{\mathbf{\tilde{Q}}}}^{k}}]}_{a,j}}}}_{\text{length }k} j \underbrace{\overbrace{ \to \circ \cdots \circ \to b}^{{{[{{({{{\mathbf{\tilde{Q}}}}^{T}})}^{k}}]}_{j,b}}}}_{\text{length }k}
$}
\end{equation}

Indeed, when edge $(i,j)$ is inserted,
only these three kinds of paths have extra contributions for $\mathbf{M}$ (therefore for $\mathbf{\Delta S}$).
As incremental updates in SimRank merely tally these paths,
node-pairs without having such paths could be safely pruned.
In other words,
for those pruned node-pairs,
the three kinds of paths will have ``zero contributions'' to the changes in $\mathbf{M}$ in response to edge insertion.
Thus, after pruning, the remaining node-pairs in $G$ constitute the ``affected areas'' of $\mathbf{M}$.

We next identify ``affected areas'' of $\mathbf{M}$,
by pruning redundant node-pairs in $G$,
based on the following. % theorem.
\begin{theorem} \label{thm:04}
For the edge $(i,j)$ insertion,
let $\mathsf{\mathcal{O}}(a)$ and $\mathsf{\tilde{\mathcal{O}}}(a)$ be the out-neighbors of node $a$ in old $G$ and new $G\cup \{(i,j)\}$, respectively.
Let $\mathbf{M}_k$ be the $k$-th iterative matrix in Eq.\eqref{eq:16a}, and let
\begin{eqnarray}
 {{\mathsf{\mathcal{F}}}_{1}}&:=&\{ b \ | \ b\in \mathsf{\mathcal{O}}(y),\ \exists y,\ s.t.\ {{[\mathbf{S}]}_{i,y}}\ne 0\} \label{eq:39} \\
{{\mathsf{\mathcal{F}}}_{2}}&:=&\left\{\begin{array}{lc}
   \varnothing  & \ \ ({{d}_{j}}=0)  \\
   \{  y  \ | \ {{[\mathbf{S}]}_{j,y}}\ne 0\} & \ \  ({{d}_{j}}>0)  \\
\end{array} \right.  \label{eq:40}
\end{eqnarray}
\begin{align}
\label{eq:41} & {{\mathsf{\mathcal{A}}}_{k}}\times {{\mathsf{\mathcal{B}}}_{k}}:= \\
\nonumber & \scalebox{0.82}{$\left\{ \begin{array}{lc}
   \{j\}\times \left( {{\mathsf{\mathcal{F}}}_{1}}\cup {{\mathsf{\mathcal{F}}}_{2}}\cup \{j\} \right) & (k=0)  \\
   \{(a,\left. b) \right|a\in \mathsf{\tilde{\mathcal{O}}}(x),\ b\in \mathsf{\tilde{\mathcal{O}}}(y),\ \exists x,\ \exists y,\ s.t.\ {{[{{\mathbf{M}}_{k-1}}]}_{x,y}}\ne 0\} & (k>0)  \\
\end{array} \right.  $}
\end{align}
%
%\begin{equation}\label{eq:41}
%\begin{split}
%&\scalebox{0.96}{$\quad   {{\mathsf{\mathcal{A}}}_{k}}\times {{\mathsf{\mathcal{B}}}_{k}}:= $} \\
%&\scalebox{0.76}{$
%\left\{ \begin{array}{lc}
%   \{j\}\times \left( {{\mathsf{\mathcal{F}}}_{1}}\cup {{\mathsf{\mathcal{F}}}_{2}}\cup \{j\} \right) & (k=0)  \\
%   \{(a,\left. b) \right|a\in \mathsf{\tilde{\mathcal{O}}}(x),\ b\in \mathsf{\tilde{\mathcal{O}}}(y),\ \exists x,\ \exists y,\ s.t.\ {{[{{\mathbf{M}}_{k-1}}]}_{x,y}}\ne 0\} & (k>0)  \\
%\end{array} \right.$}
%\end{split}
%\end{equation}

Then, for every iteration $k=0,1,\cdots$,
the matrix ${{\mathbf{M}}_{k}}$ has the following sparse property:
\[
{{[{{\mathbf{M}}_{k}}]}_{a,b}}=0 \quad \textrm{for all } (a,b)\notin ({{\mathsf{\mathcal{A}}}_{k}}\times {{\mathsf{\mathcal{B}}}_{k}}) \cup ({{\mathsf{\mathcal{A}}}_{0}}\times {{\mathsf{\mathcal{B}}}_{0}}).
\]

For the edge $(i,j)$ deletion case,
all the above results hold except that, in Eq.\eqref{eq:40},
the conditions $d_j=0$ and $d_j>0$ are, respectively, replaced by $d_j=1$ and $d_j>1$. \qed
\end{theorem}

(Please refer to Appendix~\ref{app:02e} for the proof and intuition of Theorem~\ref{thm:04}, and Appendix~\ref{app:03d} for an example.)

Theorem~\ref{thm:04} provides a pruning strategy to iteratively eliminate node-pairs with a-priori zero values in $\mathbf{M}_k$ (thus in $\mathbf{\Delta S}$).
Hence, by Theorem~\ref{thm:04},
when edge $(i,j)$ is updated,
we just need to consider node-pairs in $({{\mathsf{\mathcal{A}}}_{k}}\times {{\mathsf{\mathcal{B}}}_{k}}) \cup ({{\mathsf{\mathcal{A}}}_{0}}\times {{\mathsf{\mathcal{B}}}_{0}})$ for incrementally updating $\mathbf{\Delta S}$.

%
%\textfloatsep 1mm plus 1mm \intextsep 1mm plus 1mm
%
%\noindent \textbf{Algorithm.}
%
\subsection{{\IncSR} Algorithm with Pruning}
%
Based on Theorem~\ref{thm:04}, we provide a complete incremental algorithm, referred to as \IncSR, by incorporating our pruning strategy into \IncUSR.
The total time of \IncSR~is $O(K(m+|\AFF|))$ for $K$ iterations,
where $|\AFF|:= \textrm{avg}_{k \in[0,K]} ( |{\cal A}_k| \cdot |{\cal B}_k|)$
with ${\cal A}_k, {\cal B}_k$ in Eq.\eqref{eq:41},
being the average size of ``affected areas'' in $\mathbf{M}_k$ for $K$ iterations.

(Please refer to Appendix~\ref{app:04b} for {\IncSR} algorithm description and its complexity analysis.)
%
\section{Edge Update with node insertions} \label{sec:06}
%
%In Section~\ref{sec:04}, the edge update considered for incremental SimRank does not accompany any new node insertions.
In this section, we focus on the edge update that accompanies new node insertions.
%We distinguish them into three categories, according to which end of the inserted edge is a new node.
Specifically, given a new edge $(i,j)$ to be inserted into the old graph $G=(V,E)$, we consider the following cases when
\begin{center}
%   (C0) $i \in V$ and $j \in V$; \qquad (in Section~\ref{sec:04}) \\
   (C1) $i \in V$ and $j \notin V$; \qquad (in Subsection~\ref{sec:04c}) \\
   (C2) $i \notin V$ and $j \in V$; \qquad (in Subsection~\ref{sec:04d})  \\
   (C3) $i \notin V$ and $j \notin V$.  \qquad (in Subsection~\ref{sec:04e})  \\
%
\end{center}

For each case, we devise an efficient incremental algorithm that can support new node insertions and can accurately update only ``affected areas'' of SimRanks.
\begin{remark}
 Let $n=|V|$, without loss of generality, it can be tacitly assumed that \\
a) in case (C1), new node $j \notin V$ is indexed by $(n+1)$; \\
b) in case (C2), new node $i \notin V$ is indexed by $(n+1)$; \\
c) in case (C3), new nodes $i \notin V$ and $j \notin V$ are indexed by $(n+1)$ and $(n+2)$, respectively.
\end{remark}

%with no need to compute new SimRanks from scratch.
%
%
%We are ready to introduce our scheme for dynamically computing SimRank.
%In this section, we first focus on \emph{unit update} (\ie a single edge insertion or deletion).
%For unit insertion, we divide it further into four cases, as to whether the inserted edge $(i,j)$ would incur new node insertion.
%More precisely, given an old graph $G=(V,E)$ and a new edge $(i,j)$ to be inserted into $G$, we consider each of the following cases:
%\begin{center}
%   (C1) $i \in V$ and $j \in V$; \qquad
%   (C2) $i \in V$ and $j \notin V$; \\[1pt]
%   (C3) $i \notin V$ and $j \in V$; \qquad
%   (C4) $i \notin V$ and $j \notin V$.
%\end{center}
%For each case, our central idea is to devise a new model,
%in which the changes to the old SimRank matrix $\mathbf{S}$ are expressible as \emph{a rank-one Sylvester matrix equation}\footnote{Given the matrices $\mathbf{A},\mathbf{B},\mathbf{C} \in \mathbb{R}^{n \times n}$,
%the Sylvester matrix equation in terms of the unknown $\mathbf{X} \in \mathbb{R}^{n \times n}$ takes the form:
%$\mathbf{X}=\mathbf{A}\cdot \mathbf{X} \cdot \mathbf{B} + \mathbf{C}$.
%When $\mathbf{C}$ is a rank-$\alpha \ (\le n)$ matrix,
%we call this equation \emph{the rank-$\alpha$ Sylvester equation}.}.
%By taking advantage of the rank-one structure,
%we also design efficient algorithms for dynamically updating $\mathbf{S}$ that involves only \emph{matrix-vector} and \emph{vector-vector} multiplications,
%with no need of using \emph{matrix-matrix} multiplications to compute the new SimRank $\tilde{\mathbf{S}}$ from scratch.
%
\subsection{Inserting an edge $(i,j)$ with $i \in V$ and $j \notin V$} \label{sec:04c}
%
%We first consider the case (C1), the insertion of an edge $(i,j)$ with $i \in V$ and $j \notin V$.
In this case, the inserted new edge $(i,j)$ accompanies the insertion of a new node $j$.
Thus, the size of the new SimRank matrix $\mathbf{\tilde{S}}$ is different from that of the old $\mathbf{{S}} $.
As a result, we cannot simply evaluate the changes to $\mathbf{{S}} $ by adopting $\mathbf{\tilde{S}} -\mathbf{S}$ as we did in Section~\ref{sec:04}.

To resolve this problem, we introduce the block matrix representation of new matrices for edge insertion.
Firstly, when a new edge $(i,j)_{i \in V, j \notin V}$ is inserted to $G$,
the new transition matrix $\mathbf{\tilde{Q}}$ can be described as
\begin{equation} \label{eq:61}
\renewcommand\arraystretch{1.2}
\mathbf{\tilde{Q}}=\left[ \begin{array}{c|c}
   \mathbf{Q} & \mathbf{0}  \\  \hline
   {{\mathbf{e}}_{i}^{T}} & 0  \\
\end{array} \right] \begin{array}{l}
   \  \} \ n \textrm{ rows}  \\
   \rightarrow \textrm{row }j  \\
\end{array} \in \mathbb{R}^{(n+1)\times (n+1)}
\end{equation}
Intuitively, $\mathbf{\tilde{Q}}$ is formed by bordering the old $\mathbf{{Q}}$ by 0s except $[\mathbf{\tilde{Q}}]_{j,i}=1$.
Utilizing this block structure of $\mathbf{\tilde{Q}}$,
we can obtain the new SimRank matrix, which exhibits a similar block structure, as shown below:
\begin{theorem} \label{thm:08}
Given an old digraph $G=(V,E)$,
if there is a new edge $(i,j)$ with $i \in V$ and $j \notin V$ to be inserted,
then the new SimRank matrix becomes
\begin{equation} \label{eq:62}
\scalebox{0.82}{$
\renewcommand\arraystretch{1.2}
\mathbf{\tilde{S}}=\left[ \begin{array}{c|c}
   \mathbf{S} & \mathbf{y}  \\ \hline
   {{\mathbf{y}}^{T}} & C{{[\mathbf{S}]}_{i,i}}+(1-C)  \\
\end{array} \right] \begin{array}{l}
   \  \} \ n \textrm{ rows}  \\
   \rightarrow \textrm{row }j  \\
\end{array} \ \textrm{ with } \ \mathbf{y}=C\mathbf{Q}{{[\mathbf{S}]}_{\star,i}}
$}
\end{equation}
where $\mathbf{S} \in \mathbb{R}^{n \times n}$ is the old SimRank matrix of $G$. \qed
\end{theorem}
\begin{proof}
We substitute the new $\mathbf{\tilde{Q}}$ in Eq.\eqref{eq:61} back into the SimRank equation $\mathbf{\tilde{S}}=C \cdot \mathbf{\tilde{Q}} \cdot \mathbf{\tilde{S}}  \cdot {{\mathbf{\tilde{Q}}}^{T}}+(1-C) \cdot {{\mathbf{I}}_{n+1}}$:
\renewcommand\arraystretch{1.2}
\begin{eqnarray*}
\mathbf{S}
:= \left[ \begin{array}{c|c}
   {{{\mathbf{\tilde{S}}}}_{\mathbf{11}}} & {{{\mathbf{\tilde{S}}}}_{\mathbf{12}}}  \\ \hline
   {{{\mathbf{\tilde{S}}}}_{\mathbf{21}}} & {{{\mathbf{\tilde{S}}}}_{\mathbf{22}}}  \\
\end{array} \right]
= && C\left[ \begin{array}{c|c}
   \mathbf{Q} & \mathbf{0}  \\ \hline
   {{\mathbf{e}}_{i}^{T}} & 0  \\
\end{array} \right]\left[ \begin{array}{c|c}
   {{{\mathbf{\tilde{S}}}}_{\mathbf{11}}} & {{{\mathbf{\tilde{S}}}}_{\mathbf{12}}}  \\ \hline
   {{{\mathbf{\tilde{S}}}}_{\mathbf{21}}} & {{{\mathbf{\tilde{S}}}}_{\mathbf{22}}}  \\
\end{array} \right]\left[ \begin{array}{c|c}
   {{\mathbf{Q}}^{T}} & {{\mathbf{e}}_{i}}  \\ \hline
   \mathbf{0} & 0  \\
\end{array} \right] \\
&& +(1-C)\left[ \begin{array}{c|c}
   {{\mathbf{I}}_{n}} & \mathbf{0}  \\ \hline
   \mathbf{0} & 1  \\
\end{array} \right]
\end{eqnarray*}
By expanding the right-hand side, we can obtain
\[
\scalebox{0.85}{$
\left[ \begin{array}{c|c}
   {{{\mathbf{\tilde{S}}}}_{\mathbf{11}}} & {{{\mathbf{\tilde{S}}}}_{\mathbf{12}}}  \\ \hline
   {{{\mathbf{\tilde{S}}}}_{\mathbf{21}}} & {{{\mathbf{\tilde{S}}}}_{\mathbf{22}}}  \\
\end{array} \right]=\left[ \begin{array}{c|c}
   C\mathbf{Q}{{{\mathbf{\tilde{S}}}}_{\mathbf{11}}}{{\mathbf{Q}}^{T}}+(1-C){{\mathbf{I}}_{n}} & C\mathbf{Q}{{{\mathbf{\tilde{S}}}}_{\mathbf{11}}}{{\mathbf{e}}_{i}}  \\ \hline
   C{{\mathbf{e}}_{i}}^{T}{{{\mathbf{\tilde{S}}}}_{\mathbf{11}}}{{\mathbf{Q}}^{T}} & C{{\mathbf{e}}_{i}}^{T}{{{\mathbf{\tilde{S}}}}_{\mathbf{11}}}{{\mathbf{e}}_{i}}+(1-C)  \\
\end{array} \right]$}
\]
The above block matrix equation implies that
\[
{{\mathbf{\tilde{S}}}_{\mathbf{11}}}=C\mathbf{Q}{{\mathbf{\tilde{S}}}_{\mathbf{11}}}{{\mathbf{Q}}^{T}}+(1-C){{\mathbf{I}}_{n}}
\]
Due to the uniqueness of $\mathbf{S}$ in Eq.\eqref{eq:03a}, it follows that
\[{{\mathbf{\tilde{S}}}_{\mathbf{11}}}=\mathbf{S}\]
Thus, we have
\[
\begin{split}
{{\mathbf{\tilde{S}}}_{\mathbf{12}}}&={{\mathbf{\tilde{S}}}_{\mathbf{21}}}^{T}=C\mathbf{Q}{{\mathbf{\tilde{S}}}_{\mathbf{11}}}{{\mathbf{e}}_{i}}=C\mathbf{Q}{{[\mathbf{S}]}_{\star,i}} \\
{{\mathbf{\tilde{S}}}_{\mathbf{22}}}&=C{{\mathbf{e}}_{i}}^{T}{{\mathbf{\tilde{S}}}_{\mathbf{11}}}{{\mathbf{e}}_{i}}+(1-C)=C{{[\mathbf{S}]}_{i,i}}+(1-C)
\end{split}
\]
Combining all blocks of ${{\mathbf{\tilde{S}}}}$ together yields Eq.\eqref{eq:62}. \qed
\end{proof}

Theorem~\ref{thm:08} provides an efficient incremental way of computing the new SimRank matrix $\mathbf{\tilde{S}}$ for unit insertion of the case (C1).
Precisely, the new $\mathbf{\tilde{S}}$ is formed by bordering the old $\mathbf{{S}}$ by the auxiliary vector $\mathbf{y}$.
To obtain $\mathbf{y}$ (and thereby $\mathbf{\tilde{S}}$),
we just need use the $i$-th column of $\mathbf{{S}}$ with one matrix-vector multiplication $(\mathbf{Q}{{[\mathbf{S}]}_{\star,i}})$.
Thus, the total cost of computing new $\mathbf{\tilde{S}}$ requires $O(m)$ time,
as illustrated in Algorithm~\ref{alg:03}.
\begin{figure}[t] \centering
  \includegraphics[width=1\linewidth]{e02.eps}
  \caption{Incrementally updating SimRank when an edge $(i,p)$ with $i \in V$ and $p \notin V$ is inserted into $G=(V,E)$} \label{fig:02} %\vspace{-10pt}
\end{figure}
\begin{example}
Consider the citation digraph $G$ in Fig.~\ref{fig:02}.
If the new edge $(i,p)$ with new node $p$ is inserted to $G$, the new $\mathbf{\tilde{S}}$ can be updated from the old $\mathbf{{S}}$ as follows:

According to Theorem~\ref{thm:08},
since $C=0.8$ and  \\[-10pt]
\[\scalebox{0.86}{$
{{[\mathbf{S}]}_{\star,i}} = \kbordermatrix{
 \hspace*{-0.5em} & (a) & \cdots  & (e)  &   (f)   &  (g) & (h) & (i)     & (j)     & (k) & \cdots  & (o) \\
 \hspace*{-0.5em} & 0,  & \cdots, & 0,   & 0.2464, &   0, & 0,  & 0.5904, & 0.3104, & 0,  & \cdots, &  0  \\
}{}^T $} %\in \mathbb{R}^{15 \times 1}
\]
it follows that
%via Eq.\eqref{eq:62}
\begin{equation*}
\renewcommand\arraystretch{1.2}
\mathbf{\tilde{S}}=\left[ \begin{array}{c|c}
   \mathbf{S} & \mathbf{y}  \\ \hline
   {{\mathbf{y}}^{T}} & z  \\
\end{array} \right]  \ \textrm{ with } z=0.8{{[\mathbf{S}]}_{i,i}}+(1-0.8) = 0.6723
\end{equation*} \\[-20pt]
\[\scalebox{0.9}{$
 \quad \mathbf{y} = 0.8 \mathbf{Q} {{[\mathbf{S}]}_{\star,i}}
= \kbordermatrix{
 \hspace*{-0.5em} &  (a)   &  (b)     & (c) & \cdots  & (o)  \\
 \hspace*{-0.5em} & 0.0828, & 0.1114, & 0,  & \cdots, & 0    \\
}{}^T \in \mathbb{R}^{15 \times 1}$} \ \ \qed
\]
\end{example}
\begin{algorithm}[t]
\small
\DontPrintSemicolon
%\SetCommentSty{textsf}
\SetKwInOut{Input}{Input}
\SetKwInOut{Output}{Output}
%\SetKwFunction{Len}{Len}
\Input{a directed graph $G=(V,E)$, \\
       a new edge $(i,j)_{i \in V, \ j \notin V}$ inserted to $G$, \\
       the old similarities $\mathbf{S}$ in $G$, \\
%       the number of iterations $K$, \\
       the damping factor $C$.}
\Output{the new similarities ${\mathbf{\tilde{S}}}$ in $G \cup\{(i,j)\}$.}
\nl \label{ln:a03-01} initialize the transition matrix $\mathbf{Q}$ in $G$ ; \;
\nl \label{ln:a03-02} compute $\mathbf{y} := C \cdot \mathbf{Q}\cdot {{[\mathbf{S}]}_{\star,i}}$ ; \;
\nl \label{ln:a03-03} compute $z:= C \cdot {{[\mathbf{S}]}_{i,i}}+(1-C)$ ; \;
\nl \label{ln:a03-04} \Return $\tilde{\mathbf{S}} := \left[ \renewcommand\arraystretch{1.2} \begin{array}{c|c}
   \mathbf{S} & \mathbf{y}  \\ \hline
   {{\mathbf{y}}^{T}} & z  \\
\end{array} \right]$ ; \;
%\nl \label{ln:a01-19} $\tilde{\mathbf{S}}$ ; \;
\caption{\IncUSRtwo~($G, (i,j), \mathbf{S}, C$)}  \label{alg:03}
\end{algorithm}
%
\subsection{Inserting an edge $(i,j)$ with $i \notin V$ and $j \in V$} \label{sec:04d}
%
We now focus on the case (C2), the insertion of an edge $(i,j)$ with $i \notin V$ and $j \in V$.
Similar to the case (C1), the new edge accompanies the insertion of a new node $i$.
Hence, $\mathbf{\tilde{S}} -\mathbf{S}$ makes no sense. % to evaluate the changes to $\mathbf{{S}} $ since $\mathbf{\tilde{S}}$ and $\mathbf{{S}}$ are of different sizes.

However, in this case, the dynamic computation of SimRank is far more complicated than that of the case (C1),
in that such an edge insertion not only increases the dimension of the old transition matrix $\mathbf{{Q}}$ by one,
but also changes several original elements of $\mathbf{{Q}}$, which may recursively influence SimRank similarities.
Specifically, the following theorem shows, in the case (C2), how $\mathbf{{Q}}$ changes with the insertion of an edge $(i,j)_{i \notin V, j \in V}$.
\begin{theorem} \label{thm:05}
Given an old digraph $G=(V,E)$,
if there is a new edge $(i,j)$ with $i \notin V$ and $j \in V$ to be added to $G$,
then the new transition matrix can be expressed as
\begin{equation} \label{eq:50}
\scalebox{0.8}{$
\mathbf{\tilde{Q}}=\left[ \begin{array}{c|c}
   {\mathbf{\hat{Q}}} & \tfrac{1}{{{d}_{j}}+1}{{\mathbf{e}}_{j}}  \\ \hline
   \mathbf{0} & 0  \\
\end{array} \right] \begin{array}{l}
   \  \} \ n \textrm{ rows}  \\
   \rightarrow \textrm{row }i  \\
\end{array} %\in \mathbb{R}^{(n+1) \times (n+1)}
\ \textrm{  with } \mathbf{\hat{Q}}:=\mathbf{Q}-\tfrac{1}{{{d}_{j}}+1}{{\mathbf{e}}_{j}} {{[\mathbf{Q}]}_{j,\star}}
$}
\end{equation}
where
$\mathbf{Q}$ is the old transition matrix of $G$. \qed
\end{theorem}
\begin{proof}
When edge $(i,j)$ with $i \notin V$ and $j \in V$ is added,
there will be two changes to the old $\mathbf{Q}$:

\noindent (i) All nonzeros in ${{[\mathbf{Q}]}_{j,\star}}$ are updated from $\tfrac{1}{d_j}$ to $\tfrac{1}{d_j+1}$:
\begin{equation} \label{eq:51}
{{[\mathbf{\hat{Q}}]}_{j,\star}}
= \tfrac{{{d}_{j}}}{{{d}_{j}}+1}  {{[\mathbf{Q}]}_{j,\star}}
=  {{[\mathbf{Q}]}_{j,\star}} - \tfrac{1}{{{d}_{j}}+1}  {{[\mathbf{Q}]}_{j,\star}}
\end{equation}
(ii) The size of the old $\mathbf{{Q}}$ is added by 1,
with new entry ${{[\tilde{\mathbf{Q}}]}_{j,i}} = \tfrac{1}{d_j+1}$ in the bordered areas and 0s elsewhere:
\begin{equation} \label{eq:52}
\mathbf{\tilde{Q}}=\left[ \begin{array}{c|c}
   {\mathbf{\hat{Q}}} & \tfrac{1}{{{d}_{j}}+1}{{\mathbf{e}}_{j}}  \\ \hline
   \mathbf{0} & 0  \\
\end{array} \right] %\in \mathbb{R}^{(n+1) \times (n+1)}
%\textrm{ with } \mathbf{\hat{Q}}:=\mathbf{Q}-\tfrac{1}{{{d}_{j}}+1}{{\mathbf{e}}_{j}}\cdot {{[\mathbf{Q}]}_{j,\star}}
\end{equation}
Combining Eqs.\eqref{eq:51} and \eqref{eq:52} yields \eqref{eq:50}. \qed
\end{proof}

Theorem~\ref{thm:05} exhibits a special structure of the new ${\mathbf{\tilde{Q}}}$:
it is formed by bordering ${\mathbf{\hat{Q}}}$ by 0s except $[{\mathbf{\tilde{Q}}}]_{j,i}=\tfrac{1}{d_j+1}$,
where ${\mathbf{\hat{Q}}}$ is a rank-one update of the old ${\mathbf{{Q}}}$.
The block structure of ${\mathbf{\tilde{Q}}}$ inspires us to partition the new SimRank matrix ${\mathbf{\tilde{S}}}$ conformably into the similar block structure:
\[
\renewcommand\arraystretch{1.2}
\mathbf{\tilde{S}}= \left[ \begin{array}{c|c}
   {{{\mathbf{\tilde{S}}}}_{\mathbf{11}}} & {{{\mathbf{\tilde{S}}}}_{\mathbf{12}}}  \\ \hline
   {{{\mathbf{\tilde{S}}}}_{\mathbf{21}}} & {{{\mathbf{\tilde{S}}}}_{\mathbf{22}}}  \\
\end{array} \right]
\ \ \textrm{ where }  \
 \begin{array}{ll}
{{{\mathbf{\tilde{S}}}}_{\mathbf{11}}} \in \mathbb{R}^{n \times n}, &
{{{\mathbf{\tilde{S}}}}_{\mathbf{12}}} \in \mathbb{R}^{n \times 1}, \\[3pt]
{{{\mathbf{\tilde{S}}}}_{\mathbf{21}}} \in \mathbb{R}^{1 \times n}, &
{{{\mathbf{\tilde{S}}}}_{\mathbf{22}}} \in \mathbb{R}. %^{1 \times 1}.
\end{array}
\]
To determine each block of $\mathbf{\tilde{S}}$ with respect to the old $\mathbf{S}$,
we next present the following theorem.
\begin{theorem} \label{thm:06}
If there is a new edge $(i,j)$ with $i \notin V$ and $j \in V$ to be added to the old digraph $G=(V,E)$,
then there exists a vector
\begin{equation} \label{eq:53}
  \mathbf{z}= \alpha {{\mathbf{e}}_{j}}-\mathbf{y} \textrm{ with }
  \mathbf{y}:=\mathbf{Q}  \mathbf{S}  {{[\mathbf{Q}]}_{j,\star}^{T}}   \textrm{ and }
  \alpha : =\tfrac{{{\mathbf{y}}_{j}}+1-C}{2\left( {{d}_{j}}+1 \right)}
\end{equation}
such that the new SimRank matrix $\mathbf{\tilde{S}}$ is expressible as
\begin{equation} \label{eq:54}
\mathbf{\tilde{S}}=\left[ \begin{array}{c|c}
   \mathbf{S}+\mathbf{\Delta }{{{\mathbf{\tilde{S}}}}_{\mathbf{11}}} & \mathbf{0}  \\  \hline
   \mathbf{0} & 1-C  \\
\end{array}  \right] \begin{array}{l}
   \  \} \ n \textrm{ rows}  \\
   \rightarrow \textrm{row }i  \\
\end{array}
\end{equation}
where $\mathbf{S}$ is the old SimRank of $G$, and $\mathbf{\Delta }{{{\mathbf{\tilde{S}}}}_{\mathbf{11}}}$ satisfies the rank-two Sylvester equation:
\begin{equation} \label{eq:55}
\mathbf{\Delta }{{\mathbf{\tilde{S}}}_{\mathbf{11}}}=C\mathbf{\hat{Q}\Delta }{{\mathbf{\tilde{S}}}_{\mathbf{11}}}{{\mathbf{\hat{Q}}}^{T}}+\tfrac{C}{{{d}_{j}}+1}\left( {{\mathbf{e}}_{j}}{{\mathbf{z}}^{T}}+\mathbf{z}{{\mathbf{e}}_{j}}^{T} \right)
\end{equation}
with $\mathbf{\hat{Q}}$ being defined by Theorem~\ref{thm:05}. \qed
\end{theorem}
\begin{proof}
We plug $\mathbf{\tilde{Q}}$ of Eq.\eqref{eq:50} into the SimRank formula:
\[\mathbf{\tilde{S}}=C\cdot \mathbf{\tilde{Q}}\cdot \mathbf{\tilde{S}}\cdot {{\mathbf{\tilde{Q}}}^{T}}+(1-C) \cdot {{\mathbf{I}}_{n+1}}, \]
which produces
\begin{equation*}
\small \def\arraystretch{1.2}
\begin{split}
\mathbf{\tilde{S}}= \left[ \begin{array}{c|c}
   {{{\mathbf{\tilde{S}}}}_{\mathbf{11}}} & {{{\mathbf{\tilde{S}}}}_{\mathbf{12}}}  \\ \hline
   {{{\mathbf{\tilde{S}}}}_{\mathbf{21}}} & {{{\mathbf{\tilde{S}}}}_{\mathbf{22}}}  \\
\end{array} \right]= & C\left[ \begin{array}{c|c}
   {\mathbf{\hat{Q}}} & \tfrac{1}{{{d}_{j}}+1}{{\mathbf{e}}_{j}}  \\ \hline
   \mathbf{0} & 0  \\
\end{array} \right]\left[ \begin{array}{c|c}
   {{{\mathbf{\tilde{S}}}}_{\mathbf{11}}} & {{{\mathbf{\tilde{S}}}}_{\mathbf{12}}}  \\ \hline
   {{{\mathbf{\tilde{S}}}}_{\mathbf{21}}} & {{{\mathbf{\tilde{S}}}}_{\mathbf{22}}}  \\
\end{array} \right]{{\left[ \begin{array}{c|c}
   {\mathbf{\hat{Q}}^{T}} & \mathbf{0}  \\ \hline
   \tfrac{1}{{{d}_{j}}+1}{{\mathbf{e}}_{j}^{T}} & 0  \\
\end{array} \right]}} \\
& +(1-C)\left[ \begin{array}{c|c}
   {{\mathbf{I}}_{n}} & \mathbf{0}  \\ \hline
   \mathbf{0} & 1  \\
\end{array} \right]
\end{split}
\end{equation*}
By using block matrix multiplications, the above equation can be simplified as
\begin{equation} \label{eq:56}
\left[ \begin{array}{c|c}
   {{{\mathbf{\tilde{S}}}}_{\mathbf{11}}} & {{{\mathbf{\tilde{S}}}}_{\mathbf{12}}}  \\ \hline
   {{{\mathbf{\tilde{S}}}}_{\mathbf{21}}} & {{{\mathbf{\tilde{S}}}}_{\mathbf{22}}}  \\
\end{array} \right]=C\left[ \begin{array}{c|c}
   \mathbf{P} & \mathbf{0}  \\ \hline
   \mathbf{0} & 0  \\
\end{array} \right]+(1-C)\left[ \begin{array}{c|c}
   {{\mathbf{I}}_{n}} & \mathbf{0}  \\ \hline
   \mathbf{0} & 1  \\
\end{array} \right]
\end{equation}
\begin{equation} \label{eq:57}
\begin{split}
 \textrm{with } \mathbf{P}=& \mathbf{\hat{Q}}{{\mathbf{\tilde{S}}}_{\mathbf{11}}}{{\mathbf{\hat{Q}}}^{T}} +\tfrac{1}{{{\left( {{d}_{j}}+1 \right)}^{2}}}{{\mathbf{e}}_{j}}{{\mathbf{\tilde{S}}}_{\mathbf{22}}}{{\mathbf{e}}_{j}}^{T} \\
 & +\tfrac{1}{{{d}_{j}}+1}{{\mathbf{e}}_{j}}{{\mathbf{\tilde{S}}}_{\mathbf{21}}}{{\mathbf{\hat{Q}}}^{T}}+\tfrac{1}{{{d}_{j}}+1}\mathbf{\hat{Q}}{{\mathbf{\tilde{S}}}_{\mathbf{12}}}{{\mathbf{e}}_{j}}^{T}
\end{split}
\end{equation}
Block-wise comparison of both sides of Eq.\eqref{eq:56} yields
\[
\left\{
\begin{array}{l}
   {{{\mathbf{\tilde{S}}}}_{\mathbf{12}}}={{{\mathbf{\tilde{S}}}}_{\mathbf{21}}}=\mathbf{0} \\
  {{{\mathbf{\tilde{S}}}}_{\mathbf{22}}}=1-C \\
  {{{\mathbf{\tilde{S}}}}_{\mathbf{11}}}=C\cdot \mathbf{P}+(1-C)\cdot {{\mathbf{I}}_{n}}
\end{array}
\right.
\]
Combing the above equations with Eq.\eqref{eq:57} produces
\begin{equation} \label{eq:58}
{{\mathbf{\tilde{S}}}_{\mathbf{11}}}=C\mathbf{\hat{Q}}{{\mathbf{\tilde{S}}}_{\mathbf{11}}}{{\mathbf{\hat{Q}}}^{T}}+\tfrac{\left( 1-C \right)C}{{{\left( {{d}_{j}}+1 \right)}^{2}}}{{\mathbf{e}}_{j}}{{\mathbf{e}}_{j}}^{T}+(1-C){{\mathbf{I}}_{n}}
\end{equation}
Applying ${{\mathbf{\tilde{S}}}_{\mathbf{11}}}=\mathbf{S}+\mathbf{\Delta }{{\mathbf{\tilde{S}}}_{\mathbf{11}}}$ and $\mathbf{S}=C \mathbf{Q} \mathbf{S} {{\mathbf{Q}}^{T}}+(1-C) {{\mathbf{I}}_{n}}$ to Eq.\eqref{eq:58} and rearranging the terms, we have
\[\mathbf{\Delta }{{\mathbf{\tilde{S}}}_{\mathbf{11}}}=C\mathbf{\hat{Q}\Delta }{{\mathbf{\tilde{S}}}_{\mathbf{11}}}{{\mathbf{\hat{Q}}}^{T}}+\tfrac{C}{{{d}_{j}}+1}\left( 2\alpha {{\mathbf{e}}_{j}}{{\mathbf{e}}_{j}}^{T}-{{\mathbf{e}}_{j}}{{\mathbf{y}}^{T}}-\mathbf{y}{{\mathbf{e}}_{j}}^{T} \right)\]
with ${\alpha}$ and $\mathbf{y}$ being defined by Eq.\eqref{eq:53}. \qed
\end{proof}

Theorem~\ref{thm:06} implies that, in the case (C2), after a new edge $(i,j)$ is inserted, the new SimRank matrix $\mathbf{\tilde{S}}$ takes an elegant diagonal block structure:
the upper-left block of $\mathbf{\tilde{S}}$ is perturbed by $\mathbf{\Delta \tilde{S}_{11}}$ which is the solution to the rank-two Sylvester equation~\eqref{eq:55};
the lower-right block of $\mathbf{\tilde{S}}$ is a constant $(1-C)$.
This structure of $\mathbf{\tilde{S}}$ suggests that the inserted edge $(i,j)_{i \notin V, j \in V}$ only has a recursive impact on the SimRanks with pairs $(x,y) \in V \times V$,
but with no impacts on pairs $(x,y) \in (V \times \{i\}) \cup (\{i\} \times V)$.
Thus, our incremental way of computing the new $\mathbf{\tilde{S}}$ will focus on the efficiency of obtaining $\mathbf{\Delta \tilde{S}_{11}}$ from Eq.\eqref{eq:55}.
Fortunately, we notice that $\mathbf{\Delta \tilde{S}_{11}}$ satisfies the rank-two Sylvester equation,
whose algebraic structure is similar to that of $\mathbf{\Delta {S}}$ in Eqs.\eqref{eq:13} and \eqref{eq:14} (in Section~\ref{sec:04}).
Hence, our previous techniques to compute $\mathbf{\Delta {S}}$ in Eqs.\eqref{eq:13} and \eqref{eq:14} can be analogously applied to compute $\mathbf{\Delta \tilde{S}_{11}}$ in Eq.\eqref{eq:55},
thus eliminating costly matrix-matrix multiplications, as will be illustrated in Algorithm~\ref{alg:04}.

One disadvantage of Theorem~\ref{thm:06} is that, in order to get the auxiliary vector $\mathbf{z}$ for evaluating $\mathbf{\tilde{S}}$,
one has to memorize the \emph{entire} old matrix $\mathbf{S}$ in Eq.\eqref{eq:53}.
In fact, we can utilize the technique of rearranging the terms of the SimRank Eq.\eqref{eq:03a} to characterize $\mathbf{Q}  \mathbf{S}  {{[\mathbf{Q}]}_{j,\star}^{T}}$ in terms of only one vector $[\mathbf{S}]_{\star,j}$ so as to avoid memoizing the entire $\mathbf{S}$,
as shown below.
\begin{theorem} \label{thm:07}
The auxiliary matrix $\mathbf{\Delta }{{{\mathbf{\tilde{S}}}}_{\mathbf{11}}}$ in Theorem~\ref{thm:06} can be represented as
\begin{equation}  \label{eq:59}
\begin{split}
\mathbf{\Delta }{{\mathbf{\tilde{S}}}_{\mathbf{11}}}=\tfrac{C}{{{d}_{j}}+1}\left( \mathbf{M}+{{\mathbf{M}}^{T}} \right) \textrm{ with } \\
\mathbf{M}=\sum\nolimits_{k=0}^{\infty }{{{C}^{k}}{{{\mathbf{\hat{Q}}}}^{k}} {{\mathbf{e}}_{j}}{{\mathbf{z}}^{T}} {{\left( {{{\mathbf{\hat{Q}}}}^{T}} \right)}^{k}}}
\end{split}
\end{equation}
where $\mathbf{\hat{Q}}$ is defined by Theorem~\ref{thm:05} and
\begin{equation}  \label{eq:60}
\scalebox{0.88}{$\mathbf{z}:=\left( \tfrac{1}{2C\left( {{d}_{j}}+1 \right)}\left( {{[\mathbf{S}]}_{j,j}}-{{(1-C)}^{2}} \right)+\tfrac{1-C}{C} \right){{\mathbf{e}}_{j}}-\tfrac{1}{C}{{[\mathbf{S}]}_{\star,j}}$}
\end{equation}
and $\mathbf{S}$ is the old SimRank matrix of $G$. \qed
\end{theorem}
\begin{proof}
  We multiply the SimRank equation by ${{\mathbf{e}}_{j}}$ to get
\[
{{[\mathbf{S}]}_{\star,j}}=C\cdot \mathbf{QS}{{[\mathbf{Q}]}_{j,\star}^{T}}+(1-C)\cdot {{\mathbf{e}}_{j}}.
\]
Combining this with $\mathbf{y}=\mathbf{QS}{{[\mathbf{Q}]}_{j,\star}^{T}}$ in Eq.\eqref{eq:53} produces
\[\mathbf{y}=\tfrac{1}{C}{{[\mathbf{S}]}_{\star,j}}-\tfrac{1-C}{C}{{\mathbf{e}}_{j}} \ \textrm{ and } \
{\mathbf{y}}_j = \tfrac{1}{C}{{[\mathbf{S}]}_{j,j}}-\tfrac{1-C}{C}.
\]
Plugging these results into Eq.\eqref{eq:53},
we can get Eq.\eqref{eq:60}.

Also, the recursive form of $\mathbf{\Delta }{{\mathbf{\tilde{S}}}_{\mathbf{11}}}$ in Eq.\eqref{eq:55} can be converted into the following series:
\begin{eqnarray*}
  \mathbf{\Delta }{{\mathbf{\tilde{S}}}_{\mathbf{11}}}
&=&\tfrac{C}{{{d}_{j}}+1}\sum\nolimits_{k=0}^{\infty }{{{C}^{k}}{{{\mathbf{\hat{Q}}}}^{k}}\left( {{\mathbf{e}}_{j}}{{\mathbf{z}}^{T}}+\mathbf{z}{{\mathbf{e}}_{j}}^{T} \right){{\left( {{{\mathbf{\hat{Q}}}}^{T}} \right)}^{k}}} \\
&=&\mathbf{M} + \mathbf{M}^T
\end{eqnarray*}
with $\mathbf{M}$ being defined by Eq.\eqref{eq:59}. \qed
\end{proof}

For edge insertion of the case (C2),
Theorem \ref{thm:07} gives an efficient method to compute the update matrix $\mathbf{\Delta }{{{\mathbf{\tilde{S}}}}_{\mathbf{11}}}$.
We note that the form of $\mathbf{\Delta }{{{\mathbf{\tilde{S}}}}_{\mathbf{11}}}$ in Eq.\eqref{eq:59} is similar to that of $\mathbf{\Delta }{{{\mathbf{\tilde{S}}}}}$ in Eq.\eqref{eq:29c}.
Thus, similar to Theorem~\ref{thm:03},
the follow method can be applied to compute $\mathbf{M}$ so as to avoid matrix-matrix multiplications.
%%        \begin{description} \setlength{\itemsep}{2pt}
%%          \item
%%          initialize ${{\bm{\xi }}_{0}}\leftarrow \mathbf{e}_j,\quad {{\bm{\eta }}_{0}}\leftarrow \mathbf{z},\quad {{\mathbf{M}}_{0}}\leftarrow  \mathbf{e}_j \cdot {{\mathbf{z}}^{T}}$ \item
%%         for $k=0,1,2,\cdots $
%%         \item
%%         $\quad {{\bm{\xi }}_{k+1}}\leftarrow C\cdot \mathbf{\hat{Q}}\cdot {{\bm{\xi }}_{k}},\quad {{\bm{\eta }}_{k+1}}\leftarrow \mathbf{\hat{Q}}\cdot {{\bm{\eta }}_{k}}$
%%         \item
%%         $\quad {{\mathbf{M}}_{k+1}}\leftarrow {{\bm{\xi }}_{k+1}}\cdot \bm{\eta }_{k+1}^{T}+{{\mathbf{M}}_{k}}$
%%        \end{description}
%%
%%
%%        Note that, in the above formulas, to avoid memorizing the auxiliary $\mathbf{\hat{Q}}$,
%%        we can compute $\mathbf{\hat{Q}}\cdot {{\bm{\xi }}_{k}}$ as follows:
%%        \[
%%        \mathbf{\hat{Q}}\cdot {{\bm{\xi }}_{k}} = \mathbf{Q}\cdot {{\bm{\xi  }}_{k}} - \tfrac{1}{{{d}_{j}}+1}({{[\mathbf{Q}]}_{j,\star}} \cdot \bm{\xi }_{k}) \cdot {{\mathbf{e}}_{j}}.
%%        \]

In Algorithm~\ref{alg:04}, we present the edge insertion of our method for the case (C2) to incrementally update new SimRank scores.
The total complexity of Algorithm~\ref{alg:04} is $O(Kn^2)$ time and $O(n^2)$ memory in the worst case for retrieving all $n^2$ pairs of scores, which is dominated by Line~\ref{ln:a04-08}.
To reduce its computational time further, the similar pruning techniques in Section~\ref{sec:05} can be applied to Algorithm~\ref{alg:04}.
This can speed up the computational time to $O(K(m+|\AFF|))$,
where $|\AFF|$ is the size of ``affected areas'' in $\mathbf{\Delta S}_{11}$.
\begin{figure}[t] \centering
  \includegraphics[width=1\linewidth]{e03.eps}
  \caption{Incrementally update SimRank when a new edge $(p,j)$ with $p \notin V$ and $j \in V$ is inserted into $G=(V,E)$} \label{fig:03} %\vspace{-10pt}
\end{figure}

% $d$ is the average in-degree of the old graph, and
%In the next section, we shall also substantially reduce its memory from $O(n^2)$ to $O(nd)$.
\begin{example}
Consider the citation digraph $G$ in Fig.\ref{fig:03}.
If the new edge $(p,j)$ with new node $p$ is inserted to $G$,
the new $\mathbf{\tilde{S}}$ can be incrementally derived from the old $\mathbf{S}$ as follows:

First, we obtain $\mathbf{\Delta }{{{\mathbf{\tilde{S}}}}_{\mathbf{11}}}$ according to Theorem~\ref{thm:07}.
Note that $C=0.8$, $d_j=2$, and the old SimRank scores \\[-10pt]
\[\scalebox{0.85}{$
{{[\mathbf{S}]}_{\star,j}} = \kbordermatrix{
 \hspace*{-0.5em} & (a) & \cdots  & (e)  &   (f)   &  (g) & (h) & (i)     & (j)     & (k) & \cdots  & (o) \\
 \hspace*{-0.5em} & 0,  & \cdots, & 0,   & 0.2064, &   0, & 0,  & 0.3104, & 0.5104, & 0,  & \cdots, &  0  \\
}{}^T $} %\in \mathbb{R}^{15 \times 1}
\]
It follows from Eq.\eqref{eq:60} that the auxiliary vector
\begin{eqnarray*}
\mathbf{z} &= & \scalebox{0.88}{$\left( \tfrac{1}{2 \times 0.8\left( 2+1 \right)}\left( {0.5104}-{{(1-0.8)}^{2}} \right)+\tfrac{1-0.8}{0.8} \right){{\mathbf{e}}_{j}}-\tfrac{1}{0.8}{{[\mathbf{S}]}_{\star,j}}$} \\
&=& \scalebox{0.88}{$\kbordermatrix{
 \hspace*{-0.5em} & (a) & \cdots  & (e)  &   (f)   &  (g) & (h) & (i)     & (j)     & (k) & \cdots  & (o) \\
 \hspace*{-0.5em} & 0,  & \cdots, & 0,   & -0.258, &   0, & 0,  & -0.388, & -0.29,  & 0,  & \cdots, &  0  \\
}{}^T $}
\end{eqnarray*}
Utilizing $\mathbf{z}$, we can obtain $\mathbf{M}$ from Eq.\eqref{eq:59}.
%\[
%\scalebox{0.55}{$
%\begin{blockarray}{ccccc|cccccc|c}
%       & (a)      & (b)       & (c)& (d)      & (e)       & (f)    &  (g)     & (h)          & (i)     &   (j)  & (k)\cdots(o)  \\[3pt]
% \begin{block}{c(cccc|cccccc|c)}
%   (a) & -0.0258  & -0.0365   &  0 &  0       &           &        &             &              &         &        &               \\
%   (b) & -0.0193  & -0.0274   &  0 &  0       &           &        & \Big{$0$}   &              &         &        &    \Big{$0$}  \\
%   (c) & 0        &  0        &  0 &  0       &           &        &             &              &         &        &               \\
%   (d) & 0        &  0        &  0 &  -0.0218 &           &        &             &              &         &        &               \\\cline{1-12}
%%%   (e) &         &          &    &          &           &        &             &              &                 &               \\
%\vdots &          & \Big{$0$} &    &          &           &        & \Big{$0$}   &              &         &        &    \Big{$0$}  \\
%   (i) &          &           &    &          &           &        &             &              &         &        &               \\\cline{1-12}
%   (j) &          &     0      &    &          &  0         & -0.258 &    0         &     0         & -0.388  & -0.29  &     0          \\\cline{1-12}
%%%   (k) &         &          &    &          &           &        &             &              &         &               \\
%\vdots &          & \Big{$0$}&    &          &           &        & \Big{$0$}   &                &         &        &    \Big{$0$}  \\
%   (o) &          &          &    &          &           &        &             &                &         &        &               \\
% \end{block}
%\end{blockarray}$}\]
Thus, $\mathbf{\Delta }{{\mathbf{\tilde{S}}}_{\mathbf{11}}}$ can be computed from $\mathbf{M}$ as
\[
\mathbf{\Delta }{{\mathbf{\tilde{S}}}_{\mathbf{11}}}=\tfrac{0.8}{{2}+1}\left( \mathbf{M}+{{\mathbf{M}}^{T}} \right) =
\]
\[
\scalebox{0.52}{$
\begin{blockarray}{ccccc|ccccc|c|c}
       & (a)      & (b)       & (c)& (d)      & (e)       & (f)    &  (g)     & (h)          & (i)     &   (j)  & (k)\cdots(o)  \\[3pt]
 \begin{block}{c(cccc|ccccc|c|c)}
   (a) & -0.0137  & -0.0149   &  0 &  0       &           &        &             &              &         &        &               \\
   (b) & -0.0149  & -0.0146   &  0 &  0       &           &        & \Big{$0$}   &              &         &   \Big{$0$}    &    \Big{$0$}  \\
   (c) & 0        &  0        &  0 &  0       &           &        &             &              &         &        &               \\
   (d) & 0        &  0        &  0 &  -0.0116 &           &        &             &              &         &        &               \\\cline{1-12}
   (e) &          &           &    &          &           &        &             &              &         &   0     &               \\
   (f) &          &           &    &          &           &        &             &              &         & -0.0688 &               \\
   (g) &          & \Big{$0$} &    &          &           &        & \Big{$0$}   &              &         &   0      &   \Big{$0$}   \\
   (h) &          &           &    &          &           &        &             &              &         &   0      &               \\
   (i) &          &           &    &          &           &        &             &              &         & -0.1035 &               \\\cline{1-12}
   (j) &          &     0     &    &          &  0        & -0.0688&    0        &     0        & -0.1035 & -0.1547 &     0          \\\cline{1-12}
%%   (k) &         &          &    &          &           &        &             &              &         &               \\
\vdots &          & \Big{$0$}&    &          &           &        & \Big{$0$}   &                &         &  \Big{$0$}     &    \Big{$0$}  \\
   (o) &          &          &    &          &           &        &             &                &         &        &               \\
 \end{block}
\end{blockarray}$}\]

Next, by Theorem~\ref{thm:06}, we obtain the new SimRank
\begin{equation*}
\mathbf{\tilde{S}}=\left[ \begin{array}{c|c}
   \mathbf{S}+\mathbf{\Delta }{{{\mathbf{\tilde{S}}}}_{\mathbf{11}}} & \mathbf{0}  \\  \hline
   \mathbf{0} & 0.2  \\
\end{array}  \right]
\end{equation*}
which is partially illustrated in Fig.\ref{fig:03}. \qed
\end{example}
\begin{algorithm}[t]
\small
\DontPrintSemicolon
%\SetCommentSty{textsf}
\SetKwInOut{Input}{Input}
\SetKwInOut{Output}{Output}
%\SetKwFunction{Len}{Len}
\Input{a directed graph $G=(V,E)$, \\
       a new edge $(i,j)_{i \notin V, \ j \in V}$ inserted to $G$, \\
       the old similarities $\mathbf{S}$ in $G$, \\
       the number of iterations $K$, \\
       the damping factor $C$.}
\Output{the new similarities ${\mathbf{\tilde{S}}}$ in $G \cup\{(i,j)\}$.}
\nl \label{ln:a04-01} initialize the transition matrix $\mathbf{Q}$ in $G$ ; \;
\nl \label{ln:a04-02}  ${{d}_{j}}:=$ in-degree of node $j$ in $G$ ; \;
\nl \label{ln:a04-03}  $\mathbf{z}:=\big( \tfrac{1}{2C ( {{d}_{j}}+1 )}\big( {{[\mathbf{S}]}_{j,j}}-{{(1-C)}^{2}} \big)+\tfrac{1-C}{C} \big){{\mathbf{e}}_{j}}-\tfrac{1}{C}{{[\mathbf{S}]}_{\star,j}}$ ;\;
\nl \label{ln:a04-04}  initialize ${{\bm{\xi }}_{0}} := \mathbf{e}_j,\quad {{\bm{\eta }}_{0}} := {\mathbf{z}},\quad {{\mathbf{M}}_{0}} := \mathbf{e}_j  {\mathbf{z}}^T$ ; \;
\nl \label{ln:a04-05}  \For {$k=0,1,\cdots, K-1$} {
\nl \label{ln:a04-06}   ${{\bm{\xi }}_{k+1}} :=  C \cdot \mathbf{Q} \cdot {{\bm{\xi  }}_{k}} - \tfrac{C}{{{d}_{j}}+1}({{[\mathbf{Q}]}_{j,\star}} \cdot \bm{\xi }_{k}) \cdot  {{\mathbf{e}}_{j}}$ ; \;
\nl \label{ln:a04-07}   ${{\bm{\eta }}_{k+1}} := \mathbf{Q}\cdot {{\bm{\eta  }}_{k}} - \tfrac{1}{{{d}_{j}}+1}({{[\mathbf{Q}]}_{j,\star}} \cdot \bm{\eta  }_{k}) \cdot {{\mathbf{e}}_{j}}$ ;\;
\nl \label{ln:a04-08}   ${{\mathbf{M}}_{k+1}} := {{\bm{\xi }}_{k+1}}\cdot \bm{\eta }_{k+1}^{T}+{{\mathbf{M}}_{k}}$ ; \;
}
\nl \label{ln:a04-09} compute $\mathbf{\Delta }{{\mathbf{\tilde{S}}}_{\mathbf{11}}}:=\tfrac{C}{{{d}_{j}}+1}\left( \mathbf{M}_K+{{\mathbf{M}_K^{T}}} \right)$ ; \;
\nl \label{ln:a04-10} \Return $\mathbf{\tilde{S}}:=\left[ \begin{array}{c|c}
   \mathbf{S}+\mathbf{\Delta }{{{\mathbf{\tilde{S}}}}_{\mathbf{11}}} & \mathbf{0}  \\  \hline
   \mathbf{0} & 1-C  \\
\end{array}  \right]$ ; \;
\caption{\IncUSRthree~($G, (i,j), \mathbf{S}, K, C$)}  \label{alg:04}
\end{algorithm}
%
\subsection{Inserting an edge $(i,j)$ with $i \notin V$ and $j \notin V$} \label{sec:04e}
%
We next focus on the case (C3), the insertion of an edge $(i,j)$ with $i \notin V$ and $j \notin V$.
Without loss of generality,
it can be tacitly assumed that nodes $i$ and $j$ are indexed by $n+1$ and $n+2$, respectively.
In this case, the inserted edge $(i,j)$ accompanies the insertion of two new nodes,
which can form another independent component in the new graph.

In this case, the new transition matrix $\mathbf{\tilde{Q}}$ can be characterized as a block diagonal matrix
\[
\mathbf{\tilde{Q}}=\left[ \begin{array}{c|c}
   \mathbf{Q} & \mathbf{0}  \\ \hline
   \mathbf{0} & \mathbf{N}  \\
\end{array} \right] \begin{array}{l}
     \} \ n \textrm{ rows}  \\
     \} \ 2 \textrm{ rows}  \\
\end{array} % \in \mathbb{R}^{(n+2) \times (n+2)}
\ \textrm{ with } \
\mathbf{N}:=\left[ \begin{array}{cc}
   0 & 0  \\
   1 & 0  \\
\end{array} \right]. %\in \mathbb{R}^{2 \times 2}
\]
With this structure, we can infer that the new SimRank matrix $\mathbf{\tilde{S}}$ takes the block diagonal form as
\[
\renewcommand\arraystretch{1.2}
\mathbf{\tilde{S}}=\left[ \begin{array}{c|c}
   \mathbf{S} & \mathbf{0}  \\ \hline
   \mathbf{0} & \mathbf{\hat{S}}  \\
\end{array} \right] \begin{array}{l}
     \} \ n \textrm{ rows}  \\
     \} \ 2 \textrm{ rows}  \\
\end{array} % \in \mathbb{R}^{(n+2) \times (n+2)}
 \ \textrm{ with } \
\mathbf{\hat{S}} \in \mathbb{R}^{2 \times 2}. %\in \mathbb{R}^{2 \times 2}
\]
This is because, after a new edge $(i,j)_{i \notin V, j \notin V}$ is added,
all node-pairs $(x,y) \in (V \times \{i,j\} \cup \{i,j\} \times V)$ have zero SimRank scores since there are no connections between nodes $x$ and $y$.
Besides, the inserted edge $(i,j)$ is an independent component that has no impact on $s(x,y)$ for $\forall (x,y)\in V \times V$.
Hence, the submatrix $\mathbf{\hat{S}}$ of the new SimRank matrix can be derived by solving the equation:
\[\mathbf{\hat{S}}=C\cdot \mathbf{N}\cdot \mathbf{\hat{S}}\cdot {{\mathbf{N}}^{T}}+(1-C)\cdot {{\mathbf{I}}_{2}}
\quad \Rightarrow \ \mathbf{\hat{S}}=\left[ \begin{matrix}
   1-C & 0  \\
   0 & 1-C^2  \\
\end{matrix} \right]\]
This suggests that, for unit insertion of the case (C3),
the new SimRank matrix becomes
\[
\renewcommand\arraystretch{1.2}
\mathbf{\tilde{S}}=\left[ \begin{array}{c|c}
   \mathbf{S} & \mathbf{0}  \\ \hline
   \mathbf{0} & \mathbf{\hat{S}}  \\
\end{array} \right] \in \mathbb{R}^{(n+2) \times (n+2)} \ \textrm{ with } \
\mathbf{\hat{S}}=\left[ \begin{matrix}
   1-C & 0  \\
   0 & 1-C^2  \\
\end{matrix} \right]. %\in \mathbb{R}^{2 \times 2}
\]

Algorithm~\ref{alg:05} presents our incremental method to obtain the new SimRank matrix $\mathbf{\tilde{S}}$ for edge insertion of the case (C3),
which requires just $O(1)$ time.

\begin{algorithm}[t]
\small
\DontPrintSemicolon
%\SetCommentSty{textsf}
\SetKwInOut{Input}{Input}
\SetKwInOut{Output}{Output}
%\SetKwFunction{Len}{Len}
\Input{a directed graph $G=(V,E)$, \\
       a new edge $(i,j)_{i \notin V, \ j \notin V}$ inserted to $G$, \\
       the old similarities $\mathbf{S}$ in $G$, \\
%       the number of iterations $K$, \\
       the damping factor $C$.}
\Output{the new similarities ${\mathbf{\tilde{S}}}$ in $G \cup\{(i,j)\}$.}
\nl \label{ln:a05-01} compute $\mathbf{\hat{S}}:=\left[ \begin{matrix}
   1-C & 0  \\
   0 & 1-C^2  \\
\end{matrix} \right]$ ; \;
\nl \label{ln:a05-02} \Return $\renewcommand\arraystretch{1.2}
\mathbf{\tilde{S}}:=\left[ \begin{array}{c|c}
   \mathbf{S} & \mathbf{0}  \\  \hline
   \mathbf{0} & \mathbf{\hat{S}}  \\
\end{array}  \right]$ ; \;
\caption{\IncUSRfour~($G, (i,j), \mathbf{S}, C$)}  \label{alg:05}
\end{algorithm}
%\begin{algorithm}[t]
%\small
%\DontPrintSemicolon
%%\SetCommentSty{textsf}
%\SetKwInOut{Input}{Input}
%\SetKwInOut{Output}{Output}
%%\SetKwFunction{Len}{Len}
%\Input{a graph $G$, old similarities $\mathbf{S}$ for $G$, \#-iteration $K$, \\
%     \ the edge $(i,j)$ updated to $G$, and damping factor $C$.}
%\Output{the new similarities ${\mathbf{\tilde{S}}}$ for $G \cup\{(i,j)\}$.}
%\nl \label{ln:a01-01} initialize the transition matrix $\mathbf{Q}$ in $G$ ; \;
%\nl \label{ln:a01-02}  ${{d}_{j}}:=$ in-degree of node $j$ in $G$ ; \;
%\nl \label{ln:a01-03} memoize $\mathbf{w} := \mathbf{Q}\cdot {{[\mathbf{S}]}_{\star,i}}$ ; \;
%\nl \label{ln:a01-04} compute $\lambda := {{[\mathbf{S}]}_{i,i}}+\tfrac{1}{C} \cdot {[\mathbf{S}]}_{j,j}-2\cdot {[\mathbf{w}]}_{j} - \tfrac{1}{C} +1$ ; \;
%\nl \label{ln:a01-05}     \uIf {edge $(i,j)$ is to be inserted} {
%\nl \label{ln:a01-06}         \lIf {${{d}_{j}}=0$} {$\mathbf{u} := \mathbf{e}_j, \ \mathbf{v} := \mathbf{e}_i, \ {\bm \gamma} :=  \mathbf{w} +\frac{1}{2}{{[\mathbf{S}]}_{i,i}}\cdot {{\mathbf{e}}_{j}}$; \;}
%\nl \label{ln:a01-07}         \lElse {$\mathbf{u}:= \tfrac{1}{d_j+1} \mathbf{e}_j, \quad \mathbf{v} := \mathbf{e}_i-{[\mathbf{Q}]}_{j,\star}^T$ ; \;
%\nl \label{ln:a01-08}  $\qquad {\bm\gamma} := \tfrac{1}{({{d}_{j}}+1)} \big( \mathbf{w}-\frac{1}{C}  {{[\mathbf{S}]}_{\star,j}}+( \frac{\lambda }{2\left( {{d}_{j}}+1 \right)}+ \frac{1}{C}-1 )  {{\mathbf{e}}_{j}} \big)$; \;}}
%\nl \label{ln:a01-09}     \ElseIf {edge $(i,j)$ is to be deleted} {
%\nl \label{ln:a01-10}         \lIf {${{d}_{j}}=1$} {$\mathbf{u} := \mathbf{e}_j, \ \mathbf{v} := -\mathbf{e}_i, \ {\bm \gamma} := \frac{1}{2}{{[\mathbf{S}]}_{i,i}}\cdot {{\mathbf{e}}_{j}} - \mathbf{w}$;\;}
%\nl \label{ln:a01-11}         \lElse {$\mathbf{u}:= \tfrac{1}{d_j-1} \mathbf{e}_j, \quad \mathbf{v} := {[\mathbf{Q}]}_{j,\star}^T - \mathbf{e}_i$; \;
%\nl \label{ln:a01-12}  $\qquad {\bm\gamma} := \tfrac{1}{({{d}_{j}}-1)} \big( \frac{1}{C}  {{[\mathbf{S}]}_{\star,j}} - \mathbf{w} +( \frac{\lambda }{2\left( {{d}_{j}}-1 \right)}- \frac{1}{C}+1 )  {{\mathbf{e}}_{j}} \big)$;\;}}
%\nl \label{ln:a01-13}      initialize ${{\bm{\xi }}_{0}} := C \cdot \mathbf{e}_j,\quad {{\bm{\eta }}_{0}} := {\bm\gamma},\quad {{\mathbf{M}}_{0}} := C \cdot \mathbf{e}_j \cdot {\bm\gamma}^T$ ; \;
%\nl \label{ln:a01-14}  \For {$k=0,1,\cdots, K-1$} {
%\nl \label{ln:a01-15}   ${{\bm{\xi }}_{k+1}} := C \cdot \mathbf{Q}\cdot {{\bm{\xi }}_{k}} + C \cdot (\mathbf{v}^T\cdot {{\bm{\xi }}_{k}}) \cdot \mathbf{u}$ ; \;
%\nl \label{ln:a01-16}   ${{\bm{\eta }}_{k+1}} := \mathbf{Q}\cdot {{\bm{\eta }}_{k}} + (\mathbf{v}^T\cdot {{\bm{\eta }}_{k}}) \cdot \mathbf{u}$ ;\;
%\nl \label{ln:a01-17}   ${{\mathbf{M}}_{k+1}} := {{\bm{\xi }}_{k+1}}\cdot \bm{\eta }_{k+1}^{T}+{{\mathbf{M}}_{k}}$ ; \;
%}
%\nl \label{ln:a01-18} $\tilde{\mathbf{S}} := \mathbf{S} + \mathbf{M}_{K} + \mathbf{M}_{K}^T$ ; \;
%\nl \label{ln:a01-19} \Return $\tilde{\mathbf{S}}$ ; \;
%\caption{\IncUSRone~($G, \mathbf{S}, K, (i,j), C$)}  \label{alg:01}
%\end{algorithm}
%
%
%\subsection{A complete algorithm for unit update} \label{sec:04g}
%
%
\section{Batch Updates} \label{sec:08}
%
\begin{figure*}[!t] \centering
  \includegraphics[width=0.8\linewidth]{e07.eps}
  \caption{Batch updates for incremental SimRank when a sequence of edges $\Delta G$ are updated to $G=(V,E)$} \label{fig:07} %\vspace{-10pt}
\end{figure*}
\begin{table*} \renewcommand\arraystretch{1}
  \centering \small
  \begin{tabular}{c|c|l|p{10cm}}
    \hline
&    when  & new transition matrix $\mathbf{\tilde{Q}}$ & new SimRank matrix $\mathbf{\tilde{S}}$ \\ \hline
\parbox[t]{2mm}{\multirow{4}{*}{\rotatebox[origin=c]{90}{without new node insertions}}}
&    \specialcell[c]{(C0) \\[1pt]
                    insert \\
                    $i_1 \in V$ \\
                    $ \cdots$ \\
                    $i_{\delta} \in V$ \\
                    $j \in V$}
&
    \specialcell[c]{
            $\mathbf{\tilde{Q}} = \mathbf{Q}+\mathbf{u} \cdot \mathbf{v}^T$  with  \\[5pt]
            $\quad \mathbf{u}:=\left\{ \begin{matrix}
               {{\mathbf{e}}_{j}} & \left( {{d}_{j}}=0 \right)  \\
               \tfrac{\delta}{{{d}_{j}}+\delta}{{\mathbf{e}}_{j}} & \left( {{d}_{j}}>0 \right)  \\
            \end{matrix} \right. ,$ \\[15pt]
            $\quad \mathbf{v}:=\left\{ \begin{matrix}
               \tfrac{1}{\delta}{{\mathbf{e}}_{I}} & \left( {{d}_{j}}=0 \right)  \\
               \tfrac{1}{\delta}{{\mathbf{e}}_{I}} - {{[\mathbf{Q}]}_{j,\star}^T} & \left( {{d}_{j}}>0 \right)  \\
            \end{matrix} \right.$
    }
&
    \specialcell[c]{
            $\mathbf{\Delta S}=\mathbf{M}+{{\mathbf{M}}^{T}}$ with \\[5pt]
            $\quad \mathbf{M}:=\sum\nolimits_{k=0}^{\infty }{{{C}^{k+1}} {{{\mathbf{\tilde{Q}}}}^{k}} {{\mathbf{e}}_{j}} {{\bm{\gamma }}^{T}} {{({{{\mathbf{\tilde{Q}}}}^{T}})}^{k}}},$ \\[5pt]
            $\quad \bm{\gamma }:= \left\{ \begin{array}{lc}
               \tfrac{1}{\delta} \mathbf{Q}\cdot {{[\mathbf{S}]}_{\star,I}}+\frac{1}{2 \delta^2}{{[\mathbf{S}]}_{I,I}}\cdot {{\mathbf{e}}_{j}}  & ({{d}_{j}}=0)  \\
               \tfrac{\delta}{({{d}_{j}}+\delta)} \left( \tfrac{1}{\delta} \mathbf{Q}\cdot {{[\mathbf{S}]}_{\star,I}} - \frac{1}{C}\cdot {{[\mathbf{S}]}_{\star,j}}  +( \frac{\lambda \delta }{2\left( {{d}_{j}} + \delta \right)} + \frac{1}{C}-1 )\cdot {{\mathbf{e}}_{j}} \right) & ({{d}_{j}}>0)  \\
            \end{array} \right.$  \\[15pt]
            $\quad \lambda:=\tfrac{1}{\delta^2}{{[\mathbf{S}]}_{I,I}}+\tfrac{1}{C} \cdot {[\mathbf{S}]}_{j,j}-\tfrac{2}{\delta}\cdot {{[\mathbf{Q}]}_{j,\star}}\cdot {{[\mathbf{S}]}_{\star,I}} - \tfrac{1}{C} +1$ \\[5pt]
    }
 \\
    \cline{2-4}
&    \specialcell[c]{(C0) \\[1pt]
                    delete \\
                    $i_1 \in V$ \\
                    $ \cdots$ \\
                    $i_{\delta} \in V$ \\
                    $j \in V$}
&
    \specialcell[c]{
            $\mathbf{\tilde{Q}} = \mathbf{Q}+\mathbf{u} \cdot \mathbf{v}^T$  with  \\[5pt]
            $\quad \mathbf{u}:=\left\{ \begin{matrix}
               {{\mathbf{e}}_{j}} & \left( {{d}_{j}}=1 \right)  \\
               \tfrac{\delta}{{{d}_{j}}-\delta}{{\mathbf{e}}_{j}} & \left( {{d}_{j}}>1 \right)  \\
            \end{matrix} \right. ,$ \\[15pt]
            $\quad \mathbf{v}:=\left\{ \begin{matrix}
               {-\tfrac{1}{\delta} {\mathbf{e}}_{I}} & \left( {{d}_{j}}=1 \right)  \\
               {{[\mathbf{Q}]}_{j,\star}^T}- \tfrac{1}{\delta} {{\mathbf{e}}_{I}} & \left( {{d}_{j}}>1 \right)  \\
            \end{matrix} \right.$
    }
&
    \specialcell[c]{
            $\mathbf{\Delta S}=\mathbf{M}+{{\mathbf{M}}^{T}}$ with \\[5pt]
            $\quad \mathbf{M}:=\sum\nolimits_{k=0}^{\infty }{{{C}^{k+1}} {{{\mathbf{\tilde{Q}}}}^{k}} {{\mathbf{e}}_{j}} {{\bm{\gamma }}^{T}} {{({{{\mathbf{\tilde{Q}}}}^{T}})}^{k}}},$ \\[5pt]
            $\quad \bm{\gamma }:= \left\{ \begin{array}{lc}
                - \tfrac{1}{\delta} \mathbf{Q}\cdot {{[\mathbf{S}]}_{\star,I}}+\frac{1}{2 \delta^2}{{[\mathbf{S}]}_{I,I}}\cdot {{\mathbf{e}}_{j}}  & ({{d}_{j}}=1)  \\
               \tfrac{\delta}{({{d}_{j}}-\delta)} \left(\frac{1}{C}\cdot {{[\mathbf{S}]}_{\star,j}} - \tfrac{1}{\delta}\mathbf{Q}\cdot {{[\mathbf{S}]}_{\star,I}} +( \frac{\lambda \delta }{2\left( {{d}_{j}}-\delta \right)}- \frac{1}{C}+1 )\cdot {{\mathbf{e}}_{j}} \right) & ({{d}_{j}}>1)  \\
            \end{array} \right.$  \\[15pt]
            $\quad \lambda:=\tfrac{1}{\delta^2} {{[\mathbf{S}]}_{I,I}}+\tfrac{1}{C} \cdot {[\mathbf{S}]}_{j,j}-\tfrac{2}{\delta}\cdot {{[\mathbf{Q}]}_{j,\star}}\cdot {{[\mathbf{S}]}_{\star,I}} - \tfrac{1}{C} +1$ \\[5pt]
    }
 \\
    \hline
\parbox[t]{2mm}{\multirow{13}{*}{\rotatebox[origin=c]{90}{with new node insertions}}}
&    \specialcell[c]{(C1) \\[1pt]
                     insert \\
                    $i_1 \in V$ \\
                    $ \cdots$ \\
                    $i_{\delta} \in V$ \\
                    $j \notin V$}
&
    \specialcell[c]{
            $\renewcommand\arraystretch{1.2}
                    \mathbf{\tilde{Q}}=\left[ \begin{array}{c|c}
                       \mathbf{Q} & \mathbf{0}  \\  \hline
                       \tfrac{1}{\delta}{{\mathbf{e}}_{I}^{T}} & 0  \\
                    \end{array} \right] \begin{array}{l}
                       \  \} \ n \textrm{ rows}  \\
                       \rightarrow \textrm{row }j  \\
                    \end{array} $
    }
&
    \specialcell[c]{
            $\renewcommand\arraystretch{1.2}
                \mathbf{\tilde{S}}=\left[ \begin{array}{c|c}
                   \mathbf{S} & \mathbf{y}  \\ \hline
                   {{\mathbf{y}}^{T}} & \tfrac{C}{\delta^2}{{[\mathbf{S}]}_{I,I}}+(1-C)  \\
                \end{array} \right] \begin{array}{l}
                   \  \} \ n \textrm{ rows}  \\
                   \rightarrow \textrm{row }j  \\
                \end{array}$ \quad with \\[15pt]
            $\quad \mathbf{y}:=\tfrac{C}{\delta}\mathbf{Q}{{[\mathbf{S}]}_{\star,I}}$
    }
 \\
    \cline{2-4}
&    \specialcell[c]{(C2) \\[1pt]
                    insert \\
                    $i_1 \notin V$ \\
                    $ \cdots$ \\
                    $i_{\delta} \notin V$ \\
                    $j \in V$}
&
    \specialcell[c]{
            $\renewcommand\arraystretch{1.2}
                                \mathbf{\tilde{Q}}=\left[ \begin{array}{c|c}
               {\mathbf{\hat{Q}}} & \tfrac{1}{{{d}_{j}}+\delta}{{\mathbf{e}}_{j} \mathbf{1}_{\delta}^T}  \\ \hline
               \mathbf{0} & 0  \\
            \end{array} \right] \begin{array}{l}
               \  \} \ n \textrm{ rows}  \\
               \  \} \ \delta \textrm{ rows}  \\
            \end{array}$ \\[15pt]
            $\quad \textrm{with } \mathbf{\hat{Q}}:=\mathbf{Q}-\tfrac{\delta}{{{d}_{j}}+\delta}{{\mathbf{e}}_{j}} {{[\mathbf{Q}]}_{j,\star}}$
    }
&
    \specialcell[c]{
            $\renewcommand\arraystretch{1.2}
                \mathbf{\tilde{S}}=\left[ \begin{array}{c|c}
                   \mathbf{S}+\tfrac{C \delta}{{{d}_{j}}+\delta}\left( \mathbf{M}+{{\mathbf{M}}^{T}} \right) & \mathbf{0}  \\  \hline
                   \mathbf{0} & (1-C) \mathbf{I}_{\delta}  \\
                \end{array}  \right] \begin{array}{l}
               \  \} \ n \textrm{ rows}  \\
               \  \} \ \delta \textrm{ rows}  \\
            \end{array}$ \quad with \\[15pt]
            $\quad \mathbf{M}:=\sum\nolimits_{k=0}^{\infty }{{{C}^{k}}{{{\mathbf{\hat{Q}}}}^{k}} {{\mathbf{e}}_{j}}{{\mathbf{z}}^{T}} {{\left( {{{\mathbf{\hat{Q}}}}^{T}} \right)}^{k}}},$ \\[5pt]
            $\quad \mathbf{z}:=\left( \tfrac{1}{2C\left( {{d}_{j}}+ \delta \right)}\left( \delta {{[\mathbf{S}]}_{j,j}}-{{(\delta-C)(1-C)}} \right)+\tfrac{1-C}{C} \right){{\mathbf{e}}_{j}}-\tfrac{1}{C}{{[\mathbf{S}]}_{\star,j}}$ \\[10pt]
    }
 \\
    \cline{2-4}
&    \specialcell[c]{(C3) \\[1pt]
                    insert \\
                    $i_1 \notin V$ \\
                    $ \cdots$ \\
                    $i_{\delta} \notin V$ \\
                    $j \notin V$}
&
    \specialcell[c]{
            $\mathbf{\tilde{Q}}=\left[ \begin{array}{c|c}
               \mathbf{Q} & \mathbf{0}  \\ \hline
               \mathbf{0} & \mathbf{N}  \\
            \end{array} \right] \begin{array}{l}
                 \} \ n \textrm{ rows}  \\
                 \} \ \delta+1 \textrm{ rows}  \\
            \end{array}$   \\[15pt]
            $\quad \textrm{with } \mathbf{N}:=\left[ \begin{array}{c|c}
               \mathbf{0} & \mathbf{0}  \\ \hline
               \tfrac{1}{\delta}\mathbf{1}_{\delta}^T & 0  \\
            \end{array} \right] \begin{array}{l}
                 \} \ \delta \textrm{ rows}  \\
                 \rightarrow \textrm{row }j  \\
            \end{array}$
    }
&
    \specialcell[c]{
            $\renewcommand\arraystretch{1.2}
            \mathbf{\tilde{S}}=\left[ \begin{array}{c|c}
               \mathbf{S} & \mathbf{0}  \\ \hline
               \mathbf{0} & \mathbf{\hat{S}}  \\
            \end{array} \right]\begin{array}{l}
                 \} \ n \textrm{ rows}  \\
                 \} \ \delta+1 \textrm{ rows}  \\
            \end{array}$  \\[15pt]
            $\quad \textrm{with } \mathbf{\hat{S}}:=\left[ \begin{array}{c|c}
               (1-C)\mathbf{I}_{\delta} & \mathbf{0}  \\ \hline
               \mathbf{0} & (1-C)(1+\tfrac{C}{\delta})  \\
            \end{array} \right]. \begin{array}{l}
                 \} \ \delta \textrm{ rows}  \\
                 \rightarrow \textrm{row }j   \\
            \end{array}$
    } \\  \hline
  \end{tabular}
  \caption{{Batch updates for a sequence of edges $\{(i_1,j), \cdots, (i_{\delta},j)\}$ to the old graph $G=(V,E)$, \\
 where $[\mathbf{S}]_{\star,I} := \sum_{i \in I} [\mathbf{S}]_{\star,i}, \quad [\mathbf{S}]_{I,I} := \sum_{i \in I} [\mathbf{S}]_{i,I}, \quad \mathbf{1}_{\delta} := (1,1,\cdots, 1)^T \in \mathbb{R}^{\delta \times 1}$}}  \label{tab:02}
\end{table*}
%
In this section, we consider the batch updates problem for incremental SimRank, \ie
given an old graph $G=(V,E)$ and a sequence of edges $\Delta G$ to be updated to $G$, the retrieval of new SimRank scores in $G\oplus \Delta G$.
Here, the set $\Delta G$ can be mixed with insertions and deletions: %, which is defined as
\[
\scalebox{0.95}{$
\Delta G := \{(i_1, j_1, {\op}_1), (i_2, j_2, {\op_2}), \cdots, (i_{|\Delta G|}, j_{|\Delta G|}, {\op_{|\Delta G|}}) \}
$}
\]
where $(i_q, j_q)$ is the $q$-th edge in $\Delta G$ to be inserted into (if $\op_q =$``$+$'') or deleted from (if $\op_q =$``$-$'') $G$.

The straightforward approach to this problem is to update each edge of $\Delta G$ one by one, by running a unit update algorithm for $|\Delta G|$ times.
However, this would produce many unnecessary intermediate results and redundant updates that may cancel out each other.
\begin{example} \label{eg:09}
  Consider the old citation graph $G$ in Fig.~\ref{fig:07}, and a sequence of edge updates $\Delta G$ to $G$:
\[\scalebox{0.95}{$
\begin{split}
\Delta G =\{& (q,i,+), \ \bm{(b,h,+)}, \ (f,b,-), \ \bm{(l,f,+)}, \  (p,f,+), \\
             & \bm{(l,f,-)}, \ (j,i,+), \ (r,f,+), \ \bm{(b,h,-)}, \ (k,i,+)\}
\end{split}$}
\]
We notice that, in $\Delta G$, the edge insertion $(b,h,+)$ can cancel out the edge deletion $(b,h,-)$.
Similarly, $(l,f,+)$ can cancel out $(l,f,-)$.
Thus, after edge cancellation, the \emph{net} update of $\Delta G$, denoted as $\Delta G_{\textrm{net}}$, is
\[
\begin{split}
\Delta G_{\textrm{net}} =\{& (q,i,+),  \ (f,b,-), \ (p,f,+), \\
             &  (j,i,+), \ (r,f,+),  \ (k,i,+)\}  \qquad \qed
\end{split}
\]
\end{example}

Example~\ref{eg:09} suggests that a portion of redundancy in $\Delta G$ arises from the insertion and deletion of the same edge that may cancel out each other.
After cancellation, it is easy to verify that
\[
|\Delta G_{\textrm{net}}| \le |\Delta G| \ \textrm{ yet } \  G \oplus \Delta G_{\textrm{net}} = G \oplus \Delta G.
\]

To obtain $\Delta G_{\textrm{net}}$ from $\Delta G$,
we can readily use hashing techniques to count occurrences of updates in $\Delta G$.
More specifically, we use each edge of $\Delta G$ as a hash key,
and initialize each key with zero count.
Then, we scan each edge of $\Delta G$ once,
and increment (\Resp decrement) its count by one each time an edge insertion (\Resp deletion) appears in $\Delta G$.
After all edges in $\Delta G$ are scanned,
the edges whose counts are nonzeros make a net update $\Delta G_{\textrm{net}}$.
All edges in $\Delta G_{\textrm{net}}$ with $+1$ (\Resp $-1$) counts make a net insertion update $\Delta G_{\textrm{net}}^{+}$ (\Resp a net deletion update $\Delta G_{\textrm{net}}^{-}$).
Clearly, %we have
$
\Delta G_{\textrm{net}} = \Delta G_{\textrm{net}}^{+} \cup \Delta G_{\textrm{net}}^{-}.
$


Having reduced $\Delta G$ to the net edge updates $\Delta G_{\textrm{net}}$,
we next merge the updates of ``similar sink edges'' in $\Delta G_{\textrm{net}}$ to speedup the batch updates further.

We first introduce the notion of ``similar sink edges''.
\begin{definition}
 Two distinct edges $(a,c)$ and $(b,c)$ are called ``similar sink edges'' \wrt node $c$ if they have a common end node $c$ that both $a$ and $b$ point to. \qed
\end{definition}

``Similar sink edges'' is introduced to partition $\Delta G_{\textrm{net}}$.
To be specific,
we first sort all the edges $\{(i_p,j_p)\}$ of $\Delta G_{\textrm{net}}^{+}$ (\Resp $\Delta G_{\textrm{net}}^{-}$) according to its end node $j_p$.
Then, the ``similar sink edges'' \wrt node $j_p$ form a partition of $\Delta G_{\textrm{net}}^{+}$ (\Resp $\Delta G_{\textrm{net}}^{-}$).
For each block $\{(i_{p_k},j_p)\}$ in $\Delta G_{\textrm{net}}^{+}$, we next split it further into two sub-blocks according to whether its end node $i_{p_k}$ is in the old $V$.
Thus, after partitioning,
each block in $\Delta G_{\textrm{net}}^{+}$ (\Resp $\Delta G_{\textrm{net}}^{-}$), denoted as
$
\{(i_1,j), \ (i_2,j), \ \cdots, \ (i_{\delta},j)\},
$
falls into one of the following cases:
\begin{center}
   (C0) $i_1 \in V, \ i_2 \in V, \ \cdots, i_{\delta} \in V$ and $j \in V$; \\
   (C1) $i_1 \in V, \ i_2 \in V, \ \cdots, i_{\delta} \in V$ and $j \notin V$; \\
   (C2) $i_1 \notin V, \ i_2 \notin V, \ \cdots, i_{\delta} \notin V$ and $j \in V$;  \\
   (C3) $i_1 \notin V, \ i_2 \notin V, \ \cdots, i_{\delta} \notin V$ and $j \notin V$.  \\
\end{center}
\begin{example} \label{eg:10}
Let us recall $\Delta G_{\textrm{net}}$ derived by Example~\ref{eg:09},
in which $\Delta G_{\textrm{net}} = \Delta G_{\textrm{net}}^{+} \cup \Delta G_{\textrm{net}}^{-}$ with
\[ \small
\begin{split}
  \Delta G_{\textrm{net}}^{+} = \{& (q,i,+), \ (p,f,+), \ (j,i,+), \ (r,f,+),  \ (k,i,+)\} \\
  \Delta G_{\textrm{net}}^{-} = \{& (f,b,-)\}.
\end{split}
\]
We first partition $\Delta G_{\textrm{net}}^{+}$ by ``similar sink edges'' into
\[ \scalebox{0.9}{$
\Delta G_{\textrm{net}}^{+} = \{ (q,i,+), \ (j,i,+), \ (k,i,+) \}  \cup  \{ (p,f,+),  \ (r,f,+) \}
$}
\]

In the first block of $\Delta G_{\textrm{net}}^{+}$, since the nodes $q \notin V$, $j \in V$, and $k \in V$,
we will partition this block further into $\{ (q,i,+)\} \cup \{ (j,i,+),  (k,i,+) \}$.
Eventually,
\[ \scalebox{0.85}{$
\Delta G_{\textrm{net}}^{+} = \{ (q,i,+)\}  \cup \{ (j,i,+),  (k,i,+) \} \cup  \{ (p,f,+),   (r,f,+) \}
$} \quad \qed
\]
\end{example}

The main advantage of our partitioning approach is that, after partition, all the edge updates in each block can be processed simultaneously,
instead of one by one.
To elaborate on this, we use case (C0) as an example, \ie
the insertion of $\delta$ edges $\{(i_1,j), \ (i_2,j), \ \cdots, \ (i_{\delta},j)\}$ into $G=(V,E)$ when $i_1 \in V, \cdots, i_{\delta} \in V$, and $j \in V$.
Analogous to Theorem~\ref{thm:01}, one can readily prove that,
after such $\delta$ edges are inserted,
the changes $\mathbf{\Delta Q}$ to the old transition matrix is still a \emph{rank-one} matrix that can be decomposed as
$\mathbf{\tilde{Q}} = \mathbf{Q}+\mathbf{u} \cdot \mathbf{v}^T \  \textrm{ with } $
\[
\begin{split}
& \mathbf{u}:=\left\{ \begin{matrix}
               {{\mathbf{e}}_{j}} & \left( {{d}_{j}}=0 \right)  \\
               \tfrac{\delta}{{{d}_{j}}+\delta}{{\mathbf{e}}_{j}} & \left( {{d}_{j}}>0 \right)  \\
            \end{matrix} \right. ,
            \quad \mathbf{v}:=\left\{ \begin{matrix}
               \tfrac{1}{\delta}{{\mathbf{e}}_{I}} & \left( {{d}_{j}}=0 \right)  \\
               \tfrac{1}{\delta}{{\mathbf{e}}_{I}} - {{[\mathbf{Q}]}_{j,\star}^T} & \left( {{d}_{j}}>0 \right)  \\
            \end{matrix} \right.
\end{split}
\]
where ${\mathbf{e}}_{I}$ is an $n \times 1$ vector with its entry $[{\mathbf{e}}_{I}]_x=1$ if $x \in I\triangleq \{i_1,i_2, \cdots, i_{\delta}\}$, and $[{\mathbf{e}}_{I}]_x=0$ if $x \notin V$.
Since the rank-one structure of $\mathbf{\Delta Q}$ is preserved for updating $\delta$ edges,
Theorem~\ref{thm:02} still holds under the new settings of $\mathbf{u}$ and $\mathbf{v}$ for batch updates.
Therefore, the changes $\mathbf{\Delta S}$ to the SimRank matrix in response to $\delta$ edges insertion can be represented as a similar formulation to Theorem~\ref{thm:03},
as illustrated in the first row of Table~\ref{tab:02}.
Similarly,
we can also extend Theorems~\ref{thm:08}--\ref{thm:07} in Section~\ref{sec:06} to
support batch updates of $\delta$ edges for other cases (C1)--(C3) that accompany new node insertions.
Table~\ref{tab:02} summarizes the new $\mathbf{Q}$ and $\mathbf{S}$ in response to such batch edge updates of all the cases.
When $\delta=1$, these batch update results in Table~\ref{tab:02} can be reduced to the unit update results of Theorems~\ref{thm:01}--\ref{thm:07}.

\begin{algorithm}[t]
\small
\DontPrintSemicolon
%\SetCommentSty{textsf}
\SetKwInOut{Input}{Input}
\SetKwInOut{Output}{Output}
%\SetKwFunction{Len}{Len}
\Input{a directed graph $G=(V,E)$, \\
       a sequence of edge updates $\Delta G=\{(i,j,\op) \}$, \\
       the old similarities $\mathbf{S}$ in $G$, \\
%       the number of iterations $K$, \\
       the damping factor $C$.}
\Output{the new similarities ${\mathbf{\tilde{S}}}$ in $G \oplus \Delta G$.}
\nl \label{ln:a06-01}  obtain the net update $\Delta G_{\textrm{net}}$ from $\Delta G$ via hashing ; \;
\nl \label{ln:a06-02}  split $\Delta G_{\textrm{net}} = \Delta G_{\textrm{net}}^{+} \cup  \Delta G_{\textrm{net}}^{-}$ according to {\op} ; \;
\nl \label{ln:a06-03}  partition $\Delta G_{\textrm{net}}^{+}$ and $\Delta G_{\textrm{net}}^{-}$ by ``similar sink edges'' ; \;
\nl \label{ln:a06-04}  \For {each block of $\Delta G_{\textrm{net}}^{+}$} {
\nl \label{ln:a06-05}  split all edges $\{(i,j)\}$ of each block further into (at most) two sub-blocks based on whether $i \in V$ \ \; }
\nl \label{ln:a06-06}  \For {each block of $\Delta G_{\textrm{net}}^{-}$} {
\nl \label{ln:a06-07}  delete all edges of each block and update ${\mathbf{\tilde{S}}}$ via Table~\ref{tab:02} ;}
\nl \label{ln:a06-08}  remove all singleton nodes in the graph; \;
\nl \label{ln:a06-09}  \For {each sub-block of $\Delta G_{\textrm{net}}^{+}$} {
\nl \label{ln:a06-10}  insert all edges of each sub-block and update ${\mathbf{\tilde{S}}}$  via Table~\ref{tab:02} ;}
\nl \label{ln:a06-11}  \Return ${\mathbf{\tilde{S}}}$ ; \;
\caption{\IncBSR~($G, (i,j), \mathbf{S}, C$)}  \label{alg:06}
\end{algorithm}

Algorithm~\ref{alg:06} presents an efficient batch updates algorithm, \IncBSR, for dynamical SimRank computation.
The actual computational time of {\IncBSR} depends on the input parameter $\Delta G$ since different update types in Table~\ref{tab:02} would result in different computational time.
However, we can readily show that {\IncBSR} is superior to the $|\Delta G|$ executions of the unit update algorithm,
because {\IncBSR} can process the ``similar sink updates'' of each block simultaneously and can cancel out redundant updates.
To clarify this, let us assume that $|\Delta G_{\textrm{net}}|$ can be partitioned into $|B|$ blocks,
with $\delta_t$ denoting the number of edge updates in $t$-th block.
In the worst case, we assume that all edge updates happen to be the most time-consuming case (C0) or (C2). %, \ie the insertion without new node insertions.
Then, the total time for handling $|\Delta G|$ updates is bounded by
\[ \small
\begin{split}
     & O\bigg(\sum\nolimits_{t=1}^{|B|} \big(n\delta_t + \delta_t^2 +  K(nd + \delta_t +|\AFF|) \big)\bigg) \\
 \le & O\bigg( n |\Delta G_{\textrm{net}}| + |\Delta G_{\textrm{net}}| \sum\nolimits_{t=1}^{|B|}\delta_t + K \sum\nolimits_{t=1}^{|B|} (nd+\delta_t+|\AFF|) \bigg) \\
 \le & O\big(  (n +|\Delta G_{\textrm{net}}|) |\Delta G_{\textrm{net}}| +  K(|B|nd+|\Delta G_{\textrm{net}}|+|B||\AFF|) \big) \\
\end{split}
\]

Note that $|B| \le |\Delta G_{\textrm{net}}|$, in general $|B| \ll |\Delta G_{\textrm{net}}|$.
Thus, {\IncBSR} is typically much faster than the $|\Delta G|$ executions of the unit update algorithm that is bounded by $O\big( |\Delta G| K(nd+\Delta G+|\AFF|) \big)$. % when edge updates are small ($|\Delta G| < nd$).
\begin{example}
Recall from Example~\ref{eg:09} that a sequence of edge updates $\Delta G$ to the graph $G=(V,E)$ in Fig.~\ref{fig:07}.
We want to compute new SimRank scores in $G \oplus \Delta G$.

First, we can use hashing method to obtain the net update $\Delta G_{\textrm{net}}$ from $\Delta G$, as shown in Example~\ref{eg:09}.

Next, by Example~\ref{eg:10}, we can partition $\Delta G_{\textrm{net}}$ into
\[ \small
\begin{split}
    \Delta G_{\textrm{net}}^{+} = \{& (q,i,+)\}  \cup \{ (j,i,+),  (k,i,+) \} \cup  \{ (p,f,+),   (r,f,+) \} \\
    \Delta G_{\textrm{net}}^{-} = \{& (f,b,-)\}
\end{split}
\]

Then, for each block, we can apply the formulae in Table~\ref{tab:02} to update all edges simultaneously in a batch fashion.
The results are partially depicted as follows:
 \[\scalebox{.9}{$
\begin{tabular}{c|c|c|c|c|c}
  \hline
Node    & $\textsf{sim}_\textsf{old}$   & \multirow{2}{*}{$(f,b,-)$}         & \multirow{2}{*}{$(q,i,+)$}      & $(j,i,+)$         &       $(p,f,+)$ \\
Pairs   & in $G$                              &                   &                & $(k,i,+)$         &       $(r,f,+)$ \\ \hline
$(a,b)$ & 0.0745   & 0.0809         & 0.0809      & 0.0809         &       0.0809 \\
$(a,i)$ & 0        & 0              & 0           & 0.0340         &       0.0340 \\
$(b,i)$ & 0        & 0              & 0           & 0.0340         &       0.0340 \\
$(f,i)$ & 0.2464   & 0.2464         & 0.1232      & 0.1032         &       0.0516 \\
$(f,j)$ & 0.2064   & 0.2064         & 0.2064      & 0.2064         &       0.1032 \\
$(g,h)$ & 0.128    & 0.128          & 0.128       & 0.128          &       0.128  \\
$(g,k)$ & 0.128    & 0.128          & 0.128       & 0.128          &       0.128  \\
$(h,k)$ & 0.288    & 0.288          & 0.288       & 0.288          &       0.288  \\
$(i,j)$ & 0.3104   & 0.3104         & 0.1552      & 0.1552         &       0.1552 \\
$(l,m)$ & 0.16     & 0.16           & 0.16        & 0.16           &       0.16   \\
$(l,n)$ & 0.16     & 0.16           & 0.16        & 0.16           &       0.16   \\
$(m,n)$ & 0.16     & 0.16           & 0.16        & 0.16           &       0.16   \\
  \hline
\end{tabular}
$}\]
The column `$(q,i,+)$' represents the updated SimRank scores after the edge $(q,i)$ is added to $G \oplus \{(f,b,-)\} $.
The last column is the new SimRanks in $G \oplus \Delta G$. \qed
\end{example}
%
%To efficiently handle the batch updates, % of ``similar sink edges'' in $\Delta G_{\textrm{net}}$ simultaneously,
%we first partition $\Delta G_{\textrm{net}}$ into $\Delta G_{\textrm{net}}^{+}$
%
%%
%\subsection{Deleting an edge $(i,j)_{i \in V, \ j \in V}$ from $G=(V,E)$} %\label{sec:04f}
%%
%For an edge deletion, we next propose a Theorem~\ref{thm:03}-like technique that can efficiently update SimRanks.
%\begin{theorem}[Batch Deletions] %\label{thm:01}
%When an edge $(i,j)_{i \in V, \ j \in V}$ is deleted from $G=(V,E)$,
%the changes to $\mathbf{Q}$ is a rank-one matrix, which can be described as $\mathbf{\Delta Q} = \mathbf{u} \cdot \mathbf{v}^T$,
%where
%\begin{equation*} \label{eq:19} %\small
%\mathbf{u}=\left\{ \begin{matrix}
%   {{\mathbf{e}}_{j}} & \left( {{d}_{j}}=1 \right)  \\
%   \tfrac{1}{{{d}_{j}}-1}{{\mathbf{e}}_{j}} & \left( {{d}_{j}}>1 \right)  \\
%\end{matrix} \right.
%, \quad
%\mathbf{v}=\left\{ \begin{matrix}
%   {-{\mathbf{e}}_{i}} & \left( {{d}_{j}}=1 \right)  \\
%   {{[\mathbf{Q}]}_{j,\star}^T}-{{\mathbf{e}}_{i}} & \left( {{d}_{j}}>1 \right)  \\
%\end{matrix} \right.
%\end{equation*}
%The changes $\mathbf{\Delta S}$ to SimRank can be characterized as
%\[
%\mathbf{\Delta S}=\mathbf{M}+{{\mathbf{M}}^{T}} \ \textrm{ with }
%\mathbf{M}=\sum\nolimits_{k=0}^{\infty }{{{C}^{k+1}} {{{\mathbf{\tilde{Q}}}}^{k}} {{\mathbf{e}}_{j}} {{\bm{\gamma }}^{T}} {{({{{\mathbf{\tilde{Q}}}}^{T}})}^{k}}},
%\]
%where the auxiliary vector $\bm{\gamma }:=$ %is obtained as follows:
%\begin{equation*} %\label{eq:29a}
%\scalebox{0.86}{$
%\left\{ \begin{array}{lc}
%    -\mathbf{Q}\cdot {{[\mathbf{S}]}_{\star,i}}+\frac{1}{2}{{[\mathbf{S}]}_{i,i}}\cdot {{\mathbf{e}}_{j}}  & ({{d}_{j}}=1)  \\
%   \tfrac{1}{({{d}_{j}}-1)} \left(\frac{1}{C}\cdot {{[\mathbf{S}]}_{\star,j}} - \mathbf{Q}\cdot {{[\mathbf{S}]}_{\star,i}} +( \frac{\lambda }{2\left( {{d}_{j}}-1 \right)}- \frac{1}{C}+1 )\cdot {{\mathbf{e}}_{j}} \right) & ({{d}_{j}}>1)  \\
%\end{array} \right.
%$}
%\end{equation*}
%and $\lambda:={{[\mathbf{S}]}_{i,i}}+\tfrac{1}{C} \cdot {[\mathbf{S}]}_{j,j}-2\cdot {{[\mathbf{Q}]}_{j,\star}}\cdot {{[\mathbf{S}]}_{\star,i}} - \tfrac{1}{C} +1$. \qed %can be derived from
%\end{theorem}
%
%%
%\section{Speeding up Convergence Rate} \label{sec:08}
%%
%Having pruned unnecessary computations and achieved high memory efficiency,
%we next devise novel techniques to accelerate the convergence of incremental SimRank.
%%%
%%\subsection{Motivation}
%%%
%
%Recall the following two cases:
%(a) the edge insertion that does not accompany node insertions (in Section~\ref{sec:04}), and
%(b) the edge insertion of the case (C2) that accompanies a node insertion (in Section~\ref{sec:04d}).
%%Recall the four cases of unit insertion in Section~\ref{sec:04}.
%We notice that % (C0) and (C2),
%the characterization of the incremental SimRank matrix involves a process of iteratively computing the matrix series $\mathbf{M}$,
%as illustrated in Eq.\eqref{eq:29c} (in Section~\ref{sec:04}), and Eq.\eqref{eq:59} (in Section~\ref{sec:04d}), respectively, \ie
%\begin{equation*}  %
%\textrm{In Eq.\eqref{eq:29c}:}\quad  \mathbf{M}:=\sum\nolimits_{k=0}^{\infty }{{{C}^{k+1}}\cdot {{{\mathbf{\tilde{Q}}}}^{k}}\cdot {{\mathbf{e}}_{j}}\cdot {{\bm{\gamma }}^{T}}\cdot {{({{{\mathbf{\tilde{Q}}}}^{T}})}^{k}}}
%\end{equation*}
%\begin{equation*}  %
%\textrm{In Eq.\eqref{eq:59}:}\quad  \mathbf{M}:=\sum\nolimits_{k=0}^{\infty }{{{C}^{k}} \cdot {{{\mathbf{\hat{Q}}}}^{k}} \cdot {{\mathbf{e}}_{j}} \cdot {{\mathbf{z}}^{T}} \cdot {{( {{{\mathbf{\hat{Q}}}}^{T}} )}^{k}}}
%\end{equation*}
%In Section~\ref{sec:04}, it has been emphasized that,
%if we can take advantage of the rank-one structure of $\mathbf{M}$,
%for instance, by converting the computation of ${{{\mathbf{\tilde{Q}}}}^{k}}\cdot ({{\mathbf{e}}_{j}}\cdot {{\bm{\gamma }}^{T}}) \cdot {{({{{\mathbf{\tilde{Q}}}}^{T}})^{k}}}$
%into $(\mathbf{\tilde{Q}}\cdots (\mathbf{\tilde{Q}}\cdot (\mathbf{\tilde{Q}}\cdot {{\mathbf{e}}_{j}})))\cdot  (\mathbf{\tilde{Q}}\cdots (\mathbf{\tilde{Q}}\cdot (\mathbf{\tilde{Q}}\cdot \bm{\gamma })))^T $,
%then matrix-matrix multiplications are effectively avoidable. % when $\mathbf{M}$ is iteratively computed.
%However, our previous methods would exhibit slow convergence when the damping factor $C$ is large ($\ge 0.8$).
%This is because the difference between the first $k$-th partial sums of $\mathbf{M}$ and the exact $\mathbf{M}$ are bounded by $C^{k+1}$.
%As a result, a large damping factor $C$ means that more iterations are required to attain a desired accuracy.
%
%Such a slow convergence issue also exists in the \emph{batch} computation of SimRank, \ie
%when the SimRank score $\mathbf{S}$ is iteratively computed from scratch, a large $C$ will incur more iterations,
%as observed by Lizorkin \etal \cite{Lizorkin2008}.
%To resolve this problem,
%they suggested setting a small damping factor $C$ around 0.6.
%The method of modifying $C$ is feasible to batch SimRank computation,
%but may not be suitable to incremental computation.
%This is because, for incremental computation, we must adopt the same setting of $C$ as was previously used by the old $\mathbf{S}$.
%Thus, instead of modifying $C$, it is imperative to devise another efficient way to speed up the convergence of $\mathbf{M}$ for incremental SimRank computation,
%in case a large $C$ was set a priori for the old $\mathbf{S}$ assessment.
%%
%\subsection{The main idea}
%%
%To speed up the convergence rate of iteratively computing $\mathbf{M}$ in Eqs.\eqref{eq:29c} and \eqref{eq:59},
%we next present our ideas.
%Due to similar algebraic structures of Eqs.\eqref{eq:29c} and \eqref{eq:59},
%the following discussion is mainly based on Eq.\eqref{eq:29c}.
%
%Let us first introduce two vector-valued sequences:
%\[
%\begin{split}
%{\mathbf{V}}_{\infty}&:=\{C{{\mathbf{e}}_{j}},{{C}^{2}}\mathbf{\tilde{Q}}{{\mathbf{e}}_{j}},\cdots ,{{C}^{l}}{{\mathbf{\tilde{Q}}}^{l-1}}{{\mathbf{e}}_{j}},\cdots \} \\
%{\mathbf{W}}_{\infty}&:=\{{\bm \gamma} ,\mathbf{\tilde{Q}}{\bm \gamma} ,{{\mathbf{\tilde{Q}}}^{2}}{\bm \gamma} ,\cdots ,{{\mathbf{\tilde{Q}}}^{l-1}}{\bm \gamma},\cdots  \}
%\end{split}
%\]
%Using ${\mathbf{V}}_{\infty}$ and ${\mathbf{W}}_{\infty}$, we can rewrite $\mathbf{M}$ in Eq.\eqref{eq:29c} as
%\[
%\mathbf{M} = {\mathbf{V}}_{\infty} \cdot {\mathbf{W}}_{\infty}^T
%\]
%
%\vspace{5pt} \noindent \textbf{Dimensionality Reduction for ${\mathbf{V}}_{\infty}$ and ${\mathbf{W}}_{\infty}$.} \
%To reduce the dimensionality of ${\mathbf{V}}_{\infty}$ and ${\mathbf{W}}_{\infty}$,
%we choose the first $l$ vectors from $\mathbf{V}_{\infty}$ and $\mathbf{W}_{\infty}$, respectively, to form two $l$-dimensional subspaces. % for $\mathbf{M}$. %, denoted as $\mathbf{V}_{l}$ and $\mathbf{W}_{l}$.
%Then, in either subspace, to prevent vectors taking the similar direction,
%we apply the Gram-Schmidt method to orthonormalize these vectors,
%and get two orthogonal sets:
%\[\mathbf{V}_l:= [\mathbf{v}_1 | \mathbf{v}_2 | \cdots | \mathbf{v}_l]
%\ \textrm{ and }  \
%\mathbf{W}_l:= [\mathbf{w}_1 | \mathbf{w}_2 | \cdots | \mathbf{w}_l]
%%\ \textrm{ with } \ \mathbf{v}_k = {{C}^{k}}{{\mathbf{\tilde{Q}}}^{k-1}}{{\mathbf{e}}_{j}} \ (\forall k=1,\cdots, l)
%\]
%The following process, also known as Arnoldi iteration, describes how to get $\mathbf{V}_l$ and $\mathbf{W}_l$ via the Gram-Schmidt orthonormalization \cite{}:
%\begin{eqnarray}
%\label{eq:69a}  [\mathbf{V}_l, \mathbf{H}_l, h_{l+1,l}, \mathbf{v}_{l+1}] & \leftarrow & {\Arnoldi}~[\mathbf{\tilde{Q}}, C{{\mathbf{e}}_{j}}, l]  \\
%\label{eq:69b}  [\mathbf{W}_l, \mathbf{G}_l, g_{l+1,l}, \mathbf{w}_{l+1}] & \leftarrow & {\Arnoldi}~[\mathbf{\tilde{Q}}, {\bm \gamma}, l]
%\end{eqnarray}
%\begin{procedure}[!h]
%%\small
%\DontPrintSemicolon
%%\SetCommentSty{textsf}
%%\SetKwInOut{Input}{Input}
%%\SetKwInOut{Output}{Output}
%%\SetKwFunction{Len}{Len}
%%\Input{a graph $G=(V,E)$, decay factor $\gamma$, \#-iteration $k$, \\
%%       diagonal matrices $D_0, \cdots, D_{k-1}$.}
%%\Output{the diagonal matrix ${{({{D}_{k}})}_{i,i}}$.}
%%\nl \label{ln:p01-01} initialize $h:=\vec{0}, \ x:={{e}_{i}}$ ; \;
%\nl initialize $\mathbf{v}_1  \leftarrow \mathbf{t}/{\|\mathbf{t}\|}_2$ \;
%\nl \For {$y \leftarrow 1,2,\cdots,l$} {
%\nl     initialize $\mathbf{u} \leftarrow \mathbf{\tilde{Q}} \cdot \mathbf{v}_x$ \;
%\nl     \For {$x \leftarrow 1,2,\cdots,y$} {
%\nl         compute $h_{x,y} \leftarrow \mathbf{v}_x^T \cdot \mathbf{u}$ \;
%\nl         update $\mathbf{u} \leftarrow \mathbf{u}-h_{x,y} \cdot \mathbf{v}_x$ \;}
%\nl         set $h_{y+1,y} \leftarrow {\|\mathbf{u} \|}_2$ \;
%\nl         \lIf {$h_{y+1,y}=0$} {stop}
%\nl         compute $\mathbf{v}_{y+1} \leftarrow \mathbf{v}/h_{y+1,y}$ \;}
%\nl set $\mathbf{V}_l \leftarrow [\mathbf{v}_1 | \mathbf{v}_2 | \cdots | \mathbf{v}_l]$ \;
%\nl set $\mathbf{H}_l \leftarrow (h_{x,y})_{1 \le x \le l, \ 1 \le y \le l}$ \;
%\nl \Return $\mathbf{V}_l, \ \mathbf{H}_l, \ h_{l+1,l}, \mathbf{v}_{l+1}$
%\caption{\Arnoldi($\mathbf{\tilde{Q}}, {\mathbf{t}}, l$)}   \label{proc:01}
%\end{procedure}
%%
%%
%
%Intuitively, Procedure {\Arnoldi} gives a decomposition of $\mathbf{\tilde{Q}}$,
%as visualized in Fig.~\ref{fig:06}.
%It takes as input an $n \times n$ matrix $\mathbf{\tilde{Q}}$, a starting vector $\mathbf{t}$, and a reduced dimension $l$, and
%outputs an $n \times l$ orthonormal matrix $\mathbf{V}_l$, an $l \times l$  almost triangular matrix $\mathbf{H}_l$,
%and a residual unit vector ${{\mathbf{v}}_{l+1}}$ with its magnitude ${{h}_{l+1,l}}$, such that
%\begin{equation} \label{eq:70a}
%\mathbf{\tilde{Q}}{{\mathbf{V}}_{l}}={{\mathbf{V}}_{l}}{{\mathbf{H}}_{l}}+{{h}_{l+1,l}}{{\mathbf{v}}_{l+1}}\mathbf{e}_{l}^{T}
%\end{equation}
%Note that this decomposition of $\mathbf{\tilde{Q}}$ hinges on the choice of the starting vector $\mathbf{t}$.
%The vectors $[\mathbf{v}_1, \mathbf{v}_2, \cdots, \mathbf{v}_l]$ in $\mathbf{V}_l$ are the orthonormalized vectors $[\mathbf{t}, \mathbf{Q}\mathbf{t}, \cdots, \mathbf{Q}^{l-1}\mathbf{t}]$.
%
%Now let us denote ${{\mathbf{V}}_{l+1}}:=[{{\mathbf{V}}_{l}} \ | \ {{\mathbf{v}}_{l+1}}] \in \mathbb{R}^{n \times (l+1)}$. % and ${{\mathbf{W}}_{l+1}}:=[{{\mathbf{W}}_{l}} \ | \ {{\mathbf{w}}_{l+1}}]$,
%Then, the decomposition \eqref{eq:70a} can be simplified as
%\begin{equation} \label{eq:70}
% \renewcommand\arraystretch{1.2}
%\mathbf{\tilde{Q}}{{\mathbf{V}}_{l}}={{\mathbf{V}}_{l+1}}\left[ \begin{matrix}
%   {{\mathbf{H}}_{l}}  \\ \hline
%   {{h}_{l+1,l}}\mathbf{e}_{l}^{T}  \\
%\end{matrix} \right]
%\end{equation}
%
%Let Eq.\eqref{eq:70} be the Arnoldi decomposition of Eq.\eqref{eq:69a}.
%Similarly, we can denote Eq.\eqref{eq:69b} as
%\begin{equation} \label{eq:71}
% \renewcommand\arraystretch{1.2}
%\mathbf{\tilde{Q}}{{\mathbf{W}}_{l}}={{\mathbf{W}}_{l+1}}\left[ \begin{matrix}
%   {{\mathbf{G}}_{l}}  \\ \hline
%   {{g}_{l+1,l}}\mathbf{e}_{l}^{T}  \\
%\end{matrix} \right]
%\end{equation}
%
%\vspace{5pt} \noindent \textbf{Projecting ${\mathbf{M}}$ onto Small Orthogonal Subspace.} \
%Having obtained the two $l$-dimensional orthonormal subspaces ${{\mathbf{V}}_{l}}$ and ${{\mathbf{W}}_{l}}$, respectively, from ${{\mathbf{V}}_{\infty}}$ and  ${{\mathbf{W}}_{\infty}}$,
%we now speed up the convergence to iteratively compute $\mathbf{M}$ by using these small orthogonal subspaces.
%
%Our main idea is to exploit the small dimensions of the matrices ${{\mathbf{H}}_{l}}$ and ${{\mathbf{G}}_{l}}$ in Eqs.\eqref{eq:70} and \eqref{eq:71},
%aiming to construct an $l$-dimensional equation in terms of $\mathbf{\hat{M}}_l$:
%\begin{equation} \label{eq:74}
%{{\mathbf{\hat{M}}}_{l}}=C{{\mathbf{H}}_{l}}{{\mathbf{\hat{M}}}_{l}}{{\mathbf{G}}_{l}^{T}}+C{{\left\| {{\mathbf{e}}_{j}} \right\|}_{2}}{{\left\| {\bm \gamma}  \right\|}_{2}}{{\mathbf{e}}_{1}}{{\mathbf{e}}_{1}^{T}},
%\end{equation}
%where $\mathbf{\hat{M}}_l \in \mathbb{R}^{l \times l}$ is a ``reduced version'' of $\mathbf{M} \in \mathbb{R}^{n \times n}$ projected onto the small subspace.
%Due to small dimensionality and orthogonality of the subspace,
%it is easier to compute $\mathbf{\hat{M}}_l$ and attain its fast rate of convergence.
%After $\mathbf{\hat{M}}_l$ is obtained,
%we utilize ${{\mathbf{V}}_{l}}$ and ${{\mathbf{W}}_{l}}$ to project $\mathbf{\hat{M}}_l \in \mathbb{R}^{l \times l}$ back to $n \times n$ space,
%denoted by $\mathbf{M}_l \in \mathbb{R}^{n \times n}$,
%so as to estimate the exact $\mathbf{M}\in \mathbb{R}^{n \times n}$ as
%\begin{equation} \label{eq:73}
%{{\mathbf{M}}_{l}}:={{\mathbf{V}}_{l}}{{\mathbf{\hat{M}}}_{l}}{{\mathbf{W}}_{l}^{T}}.
%\end{equation}
%%
%\subsection{Accuracy Guarantee}
%%
%One challenging problem for this estimation is to bound the approximation error ${{\mathbf{M}}_{l}} - \mathbf{M} $.
%To address this issue, we have the following theorem.
%\begin{figure}[t] \centering
%  \includegraphics[width=\linewidth]{e06.eps}
%  \caption{Decomposition of $\mathbf{\tilde{Q}}$ via {\Arnoldi} procedure} \label{fig:06} %\vspace{-10pt}
%\end{figure}
%
%%\[{{\cal K}_{l}}(\mathbf{\tilde{Q}},C{{\mathbf{e}}_{j}})=\textrm{span}\{C{{\mathbf{e}}_{j}},{{C}^{2}}\mathbf{\tilde{Q}}{{\mathbf{e}}_{j}},\cdots ,{{C}^{l}}{{\mathbf{\tilde{Q}}}^{l-1}}{{\mathbf{e}}_{j}}\}\]
%%\[{{\cal K}_{l}}(\mathbf{\tilde{Q}}, {\bm \gamma} )=\textrm{span}\{{\bm \gamma} ,\mathbf{\tilde{Q}}{\bm \gamma} ,{{\mathbf{\tilde{Q}}}^{2}}{\bm \gamma} ,\cdots ,{{\mathbf{\tilde{Q}}}^{l-1}}{\bm \gamma} \}\]
%%\[
%%\mathbf{\tilde{Q}}{{\mathbf{V}}_{l}}={{\mathbf{V}}_{l}}{{\mathbf{H}}_{l}}+{{h}_{l+1,l}}{{\mathbf{v}}_{l+1}}\mathbf{e}_{l}^{T}
%%\]
%%\[
%%\mathbf{\tilde{Q}}{{\mathbf{W}}_{l}}={{\mathbf{W}}_{l}}{{\mathbf{H}}_{l}}+{{h}_{l+1,l}}{{\mathbf{w}}_{l+1}}\mathbf{e}_{l}^{T}
%%\]
%%Let ${{\mathbf{V}}_{l+1}}:=[{{\mathbf{V}}_{l}} \ | \ {{\mathbf{v}}_{l+1}}]$ and ${{\mathbf{W}}_{l+1}}:=[{{\mathbf{W}}_{l}} \ | \ {{\mathbf{w}}_{l+1}}]$,
%%then the above equations can be respectively rewritten as
%%\begin{equation} \label{eq:70}
%% \renewcommand\arraystretch{1.2}
%%\mathbf{\tilde{Q}}{{\mathbf{V}}_{l}}={{\mathbf{V}}_{l+1}}\left[ \begin{matrix}
%%   {{\mathbf{H}}_{l}}  \\ \hline
%%   {{h}_{l+1,l}}\mathbf{e}_{l}^{T}  \\
%%\end{matrix} \right]
%%\end{equation}
%%\begin{equation} \label{eq:71}
%% \renewcommand\arraystretch{1.2}
%%\mathbf{\tilde{Q}}{{\mathbf{W}}_{l}}={{\mathbf{W}}_{l+1}}\left[ \begin{matrix}
%%   {{\mathbf{G}}_{l}}  \\ \hline
%%   {{g}_{l+1,l}}\mathbf{e}_{l}^{T}  \\
%%\end{matrix} \right]
%%\end{equation}
%\begin{theorem} \label{thm:09}
%  Let $\mathbf{M}\in {{\mathbb{R}}^{n\times n}}$ be the (exact) solution to Eq.\eqref{eq:29c},
%%that is,
%%\begin{equation} \label{eq:72}
%%\mathbf{M}=C\mathbf{\tilde{Q}M}{{\mathbf{\tilde{Q}}}^{T}}+C{{\mathbf{e}}_{j}}{{\bm \gamma }^{T}}
%%\end{equation}
%and ${{\mathbf{M}}_{l}}$ in Eq.\eqref{eq:73} be the approximation of $\mathbf{M}$,
%%\begin{equation} \label{eq:73}
%%{{\mathbf{M}}_{l}}:={{\mathbf{V}}_{l}}{{\mathbf{\hat{M}}}_{l}}{{\mathbf{W}}_{l}^{T}}
%%\end{equation}
%%where ${{\mathbf{\hat{M}}}_{l}}\in {{\mathbb{R}}^{l\times l}}$ is the solution Eq.\eqref{eq:74}.
%%\begin{equation} \label{eq:74}
%%{{\mathbf{\hat{M}}}_{l}}=C{{\mathbf{H}}_{l}}{{\mathbf{\hat{M}}}_{l}}{{\mathbf{G}}_{l}^{T}}+C{{\left\| {{\mathbf{e}}_{j}} \right\|}_{2}}{{\left\| {\bm \gamma}  \right\|}_{2}}{{\mathbf{e}}_{1}}{{\mathbf{e}}_{1}^{T}}.
%%\end{equation}
%Then, the gap between $\mathbf{M}$ and ${{\mathbf{M}}_{l}}$ can be bounded by \\[5pt]
% ${{\left\| \mathbf{M}-{{\mathbf{M}}_{l}} \right\|}_{\max }} \le \tfrac{C}{1-C} \times$
%%\begin{widetext}
%\[
% \scalebox{0.8}{$\times \sqrt{{{| {{g}_{l+1,l}} |}^{2}}{{\| {{\mathbf{H}}_{l}}{{[{{{\mathbf{\hat{M}}}}_{l}}]}_{\star,l}} \|}_{F}^{2}}+{{| {{h}_{l+1,l}} |}^{2}}{{\| {{\mathbf{G}}_{l}}{{[{{{\mathbf{\hat{M}}}}_{l}}]}_{\star,l}} \|}_{F}^{2}}+ {|{{ {{g}_{l+1,l}} }}{{ {{h}_{l+1,l}} }}{{[{{{\mathbf{\hat{M}}}}_{l}}]}_{l,l}}|}^2}$}
%\]
%%\end{widetext}
%\end{theorem}
%\begin{proof}
%Subtracting ${{\mathbf{M}}_{l}}$ from Eq.\eqref{eq:29c} on both sides yields
%\begin{equation} \label{eq:75}
%\mathbf{M}-{{\mathbf{M}}_{l}}=C\mathbf{\tilde{Q}}\left( \mathbf{M}-{{\mathbf{M}}_{l}} \right){{\mathbf{\tilde{Q}}}^{T}}+{{\bm \epsilon }_{{{\mathbf{M}}_{l}}}}
%\end{equation}
%where
%\begin{equation} \label{eq:76}
%{{\bm \epsilon }_{{{\mathbf{M}}_{l}}}}:=C\mathbf{\tilde{Q}}{{\mathbf{M}}_{l}}{{\mathbf{\tilde{Q}}}^{T}}-{{\mathbf{M}}_{l}}+C{{\mathbf{e}}_{j}}{{\bm \gamma }^{T}}
%\end{equation}
%
%First, we obtain a tight bound for the error ${{\bm \epsilon }_{{{\mathbf{M}}_{l}}}}$.
%It follows from ${{\mathbf{v}}_{1}}=C{{\mathbf{e}}_{j}}/{{\left\| C{{\mathbf{e}}_{j}} \right\|}_{2}}$ and ${{\mathbf{w}}_{1}}= {\bm \gamma} /{{\left\| {\bm \gamma}  \right\|}_{2}}$ that
%\begin{equation} \label{eq:77}
%C{{\mathbf{e}}_{j}}=C{{\left\| {{\mathbf{e}}_{j}} \right\|}_{2}}{{\mathbf{V}}_{l}}{{\mathbf{e}}_{1}}, \qquad {\bm \gamma} ={{\left\| {\bm \gamma}  \right\|}_{2}}{{\mathbf{W}}_{l}}{{\mathbf{e}}_{1}}
%\end{equation}
%Plugging Eq.\eqref{eq:73} into \eqref{eq:76} yields
%\begin{eqnarray*}
%  {{\bm \epsilon }_{{{\mathbf{M}}_{l}}}}
%&=& C\underbrace{\mathbf{\tilde{Q}}{{\mathbf{V}}_{l}}}_{\mathclap{\textrm{by Eq.\eqref{eq:70}}}}{{{\mathbf{\hat{M}}}}_{l}}\underbrace{{{\mathbf{W}}_{l}}^{T}{{{\mathbf{\tilde{Q}}}}^{T}}}_{\textrm{by Eq.\eqref{eq:71}}}-{{\mathbf{V}}_{l}}{{{\mathbf{\hat{M}}}}_{l}}{{\mathbf{W}}_{l}}^{T}+\underbrace{C{{\mathbf{e}}_{j}}{{\bm \gamma }^{T}}}_{\textrm{by Eq.\eqref{eq:77}}} \\
%&=& C{{\mathbf{V}}_{l+1}}\left[ \renewcommand\arraystretch{1.2}  \begin{matrix}
%   {{\mathbf{H}}_{l}}  \\  \hline
%   {{h}_{l+1,l}}\mathbf{e}_{l}^{T}  \\
%\end{matrix} \right]{{{\mathbf{\hat{M}}}}_{l}}\left[ \begin{array}{c|c}
%   {{\mathbf{G}}_{l}}^{T} & {{g}_{l+1,l}}\mathbf{e}_{l}  \\
%\end{array} \right]{{\mathbf{W}}_{l+1}}^{T} \\
%&& - {{\mathbf{V}}_{l+1}}\left[ \begin{array}{c|c}
%   {{{\mathbf{\hat{M}}}}_{l}} & \mathbf{0}  \\ \hline
%   \mathbf{0} & 0  \\
%\end{array} \right]{{\mathbf{W}}_{l+1}}^{T} \\
%&& +{{\mathbf{V}}_{l+1}}\left[ \begin{array}{c|c}
%   C{{\left\| {{\mathbf{e}}_{j}} \right\|}_{2}}{{\left\| \bm \gamma  \right\|}_{2}}{{\mathbf{e}}_{1}}{{\mathbf{e}}_{1}^{T}} & \mathbf{0}  \\ \hline
%   \mathbf{0} & 0  \\
%\end{array} \right]{{\mathbf{W}}_{l+1}}^{T} \\
%&=& C \cdot {{\mathbf{V}}_{l+1}} \cdot \mathbf{\Xi} \cdot {{\mathbf{W}}_{l+1}}^{T}
%\end{eqnarray*}
%where
%\begin{eqnarray*}
%\mathbf{\Xi}
%& := & \scalebox{0.9}{$\Bigg[ \renewcommand\arraystretch{1.2}  \begin{array}{c|c}
%   \smash{\overbrace{{{\mathbf{H}}_{l}}{{\mathbf{G}}_{l}}^{T}-\tfrac{1}{C}{{{\mathbf{\hat{M}}}}_{l}}+{{\left\| {{\mathbf{e}}_{j}} \right\|}_{2}}{{\left\| \bm \gamma  \right\|}_{2}}{{\mathbf{e}}_{1}}{{\mathbf{e}}_{1}^{T}}}^{=\{\textrm{by Eq.}\eqref{eq:74}\}=\mathbf{0}}} & {{g}_{l+1,l}}{{\mathbf{H}}_{l}}{{{\mathbf{\hat{M}}}}_{l}}\mathbf{e}_{l} \\ \hline
%   {{h}_{l+1,l}}\mathbf{e}_{l}^{T}{{{\mathbf{\hat{M}}}}_{l}}{{\mathbf{G}}_{l}}^{T} & {{h}_{l+1,l}}{{g}_{l+1,l}}\mathbf{e}_{l}^{T}{{{\mathbf{\hat{M}}}}_{l}}\mathbf{e}_{l}  \\
%\end{array} \Bigg]$} \\
%&=& \scalebox{0.9}{$\left[ \renewcommand\arraystretch{1.2} \begin{array}{c|c}
%   \mathbf{0} & {{g}_{l+1,l}}{{\mathbf{H}}_{l}}{{[{{{\mathbf{\hat{M}}}}_{l}}]}_{\star,l}}  \\ \hline
%   {{h}_{l+1,l}}{{[{{{\mathbf{\hat{M}}}}_{l}}]}_{l,\star}}{{\mathbf{G}}_{l}^{T}} & {{h}_{l+1,l}}{{g}_{l+1,l}}{{[{{{\mathbf{\hat{M}}}}_{l}}]}_{l,l}}  \\
%\end{array} \right]$} \\
%\end{eqnarray*}
%Taking ${{\| \star\|}_{F}}$ norm on ${{\bm \epsilon }_{{{\mathbf{M}}_{l}}}}$, we can obtain
%\begin{eqnarray}
% && {{\|  {{\bm \epsilon }_{{{\mathbf{M}}_{l}}}} \|}_{F}} = C {{\|   {{\mathbf{V}}_{l+1}}  \mathbf{\Xi}  {{\mathbf{W}}_{l+1}}^{T}  \|}_{F}} = C {{\|   \mathbf{\Xi}  \|}_{F}}  \label{eq:78} \\ \nonumber
%&=& \scalebox{0.8}{$ C \sqrt{ {\| {{g}_{l+1,l}}{{\mathbf{H}}_{l}}{{[{{{\mathbf{\hat{M}}}}_{l}}]}_{\star,l}} \|}^2_F
%+ {\| {{h}_{l+1,l}}{{\mathbf{G}}_{l}^{T}}{{[{{{\mathbf{\hat{M}}}}_{l}}]}_{\star,l}} \|}^2_F
%+ {| {{h}_{l+1,l}}{{g}_{l+1,l}}[{{{\mathbf{\hat{M}}}}_{l}}]_{l,l} |}^2 }$}
%\end{eqnarray}
%
%Next, we take ${{\left\| \star \right\|}_{\max }}$  norm on both sides of Eq.\eqref{eq:75}:
%\begin{eqnarray*}
% {{\| \mathbf{M}-{{\mathbf{M}}_{l}} \|}_{\max }} &\le & C{{\| \mathbf{\tilde{Q}}( \mathbf{M}-{{\mathbf{M}}_{l}} ){{{\mathbf{\tilde{Q}}}}^{T}} \|}_{\max }}+{{\| {{\bm \epsilon }_{{{\mathbf{M}}_{l}}}} \|}_{\max }} \\
% & \le & C{{\| \mathbf{M}-{{\mathbf{M}}_{l}} \|}_{\max }}+{{\| {{\bm \epsilon }_{{{\mathbf{M}}_{l}}}} \|}_{F}}
%\end{eqnarray*}
%which implies that
%${{\left\| \mathbf{M}-{{\mathbf{M}}_{l}} \right\|}_{\max }}\le \tfrac{1}{1-C}{{\left\| {{\bm \epsilon }_{{{\mathbf{M}}_{l}}}} \right\|}_{F}}$ with ${{\| {{\bm \epsilon }_{{{\mathbf{M}}_{l}}}} \|}_{F}}$ being defined by Eq.\eqref{eq:78}. \qed
%\end{proof}
%
%Theorem~\ref{thm:09} provides a posteriori error for the approximation of $\mathbf{M}$ by $\mathbf{M}_l$.
%Note that, according to the property of Arnodi iteration, the residual factors $g_{l+1,l}$ and $h_{l+1,l}$ both approach 0 as $l$ grows.
%Hence, $\mathbf{M}_l \to \mathbf{M}$ as $l$ increases.
%Our experiments in the next section will demonstrate that, at $l$-th iteration,
%our error bound in Theorem~\ref{thm:09} is typically much tighter than the existing bound $C^{l+1}$ by Lizorkin \etal \cite{Lizorkin2008}.
%Moreover, unlike the bound $C^{l+1}$ that solely relies on the damping factor $C$,
%our error bound also takes the structure information of a real graph into consideration.
%%Thus, to guarantee a given accuracy, our method can achieve fast convergence rate.
%
{
\section{Memory Efficiency} \label{sec:07} %\label{sec:05}
%
\begin{table*}[t]
\centering
\scalebox{.85}{$ \renewcommand\arraystretch{1.1}
\begin{tabular}{l|l|l}
\hline
\textbf{Line} & \textbf{Description}                                                                                                                                                                       & \textbf{Required Elements from old $\mathbf{S}$}       \\ \hline
3                & $\mathbf{w} \leftarrow \mathbf{Q}\cdot \mathcolor{red}{{{[\mathbf{S}]}_{\star,i}}}$                                                                                                                                 & $i$-th column of $\mathbf{S}$                    \\
4                & $\lambda \leftarrow \mathcolor{red}{{[\mathbf{S}]}_{i,i}}+\tfrac{1}{C} \cdot \mathcolor{red}{{[\mathbf{S}]}_{j,j}}-2\cdot {[\mathbf{w}]}_{j} - \tfrac{1}{C} +1$                                                            & $(i,i)$- and $(j,j)$-th elements of $\mathbf{S}$ \\
6                & ${\bm \gamma} \leftarrow \mathbf{w} +\frac{1}{2}\mathcolor{red}{{[\mathbf{S}]}_{i,i}}\cdot {{\mathbf{e}}_{j}}$                                                                                                    & $(i,i)$-th element of $\mathbf{S}$               \\
9                & ${\bm\gamma} \leftarrow \tfrac{1}{({{d}_{j}}+1)} \big( \mathbf{w}-\frac{1}{C}\mathcolor{red}{{[\mathbf{S}]}_{\star,j}}+( \frac{\lambda }{2\left( {{d}_{j}}+1 \right)}+ \frac{1}{C}-1 ) {{\mathbf{e}}_{j}} \big)$ & $j$-th column of $\mathbf{S}$ \\
15               & $\mathcolor{red}{\tilde{\mathbf{S}}} \leftarrow \mathcolor{red}{\mathbf{S}} + \mathbf{M}_{K} + \mathbf{M}_{K}^T$ & all elements of old $\mathbf{S}$ and new $\tilde{\mathbf{S}}$ \\ \hline
\end{tabular}$}
\caption{{Lines of \IncUSRone~(in Appendix~\ref{app:04a}) that require to get elements from old $\mathbf{S}$ (highlighted in red color)}} \label{tab:04}
\end{table*}

In previous sections, our main focus was devoted to speeding up the computational time of incremental SimRank.
However, for updating all pairs of SimRank scores,
the memory requirement for Algorithms~\ref{alg:03}--\ref{alg:06} remains at $O(n^2)$ since they need to store all $(n^2)$ pairs of old SimRank $\mathbf{S}$ into memory,
which hinders its scalability on large graphs.
We call Algorithms~\ref{alg:03}--\ref{alg:06} \emph{in-memory algorithms}.
\begin{table}[t]
\centering
\scalebox{.85}{$ \renewcommand\arraystretch{1.1}
\begin{tabular}{l|l|l}
\hline
\textbf{Line} & \textbf{Description}                                                                                                                                                                       & \textbf{Storage of $\mathbf{M}_k$}       \\ \hline
10                & $\mathcolor{red}{{\mathbf{M}}_{0}} \leftarrow C \cdot \mathbf{e}_j \cdot {\bm\gamma}^T$                                                                                                                                 & all elements of $\mathbf{M}_0$                    \\
14                & $\mathcolor{red}{{\mathbf{M}}_{k+1}} \leftarrow {{\bm{\xi }}_{k+1}}\cdot \bm{\eta }_{k+1}^{T}+ \mathcolor{red}{{\mathbf{M}}_{k}}$                                                            & all elements of $\mathbf{M}_k \quad (\forall k)$ \\
15                & ${\tilde{\mathbf{S}}} \leftarrow {\mathbf{S}} + \mathcolor{red}{\mathbf{M}_{K}} + (\mathcolor{red}{\mathbf{M}_{K}})^T$                                                                                                    & all elements of $\mathbf{M}_K$              \\ \hline
\end{tabular}$}
\caption{{Lines of \IncUSRone~(in Appendix~\ref{app:04a}) that require to store $\mathbf{M}_k$ (highlighted in red color)}} \label{tab:05}
\end{table}

In this section, we propose a novel scalable method based on Algorithms~\ref{alg:03}--\ref{alg:06} for dynamical SimRank search,
which updates all pairs of SimRanks column by column using only $O(Kn+m)$ memory,
with no need to store all $(n^2)$ pairs of old SimRank $\mathbf{S}$ into memory, and with no loss of accuracy.

Let us first analyze the $O(n^2)$ memory requirement for Algorithms~\ref{alg:03}--\ref{alg:06} in Sections~\ref{sec:04}--\ref{sec:06}.
We notice that there are two factors dominating the original $O(n^2)$ memory:
(1) the storage of the entire $n \times n$ old SimRank matrix $\mathbf{S}$,
and (2) the computation of $\mathbf{M}_k$ from one outer product.
%Apart from the storage of the old $\mathbf{S}$ and $\mathbf{M}_k$,
%the space required for the remaining steps is dominated by $O(m)$.
For example, in \IncUSRone~(in Appendix~\ref{app:04a}),
Lines 3, 4, 6, 9, 15 need to get elements from old $\mathbf{S}$ (see Table~\ref{tab:04});
Lines 10, 14, 15 require to store $ n \times n$ entries of matrix $\mathbf{M}_k$ (see Table~\ref{tab:05}).
Indeed, the storage of $\mathbf{S}$ and $\mathbf{M}_k$ are the main obstacles to the scalability of our in-memory algorithms on large graphs,
resulting in $O(n^2)$ memory space.
Apart from these lines, the memory required for the remaining steps of {\IncUSRone} is $O(m)$,
dominated by (a) the storage of sparse matrix $\mathbf{Q}$ and (b) sparse matrix-vector products.

To overcome the bottleneck of the $O(n^2)$ memory,
our main idea is to update all pairs of $\mathbf{S}$ in a column-by-column style, with no need to store the entire $\mathbf{S}$ and $\mathbf{M}_k$. %just for once.
Specifically,
we update $\mathbf{S}$ by updating each column $[\mathbf{{S}}]_{\star,x} \ (\forall x=1,2,\cdots)$ of $\mathbf{S}$ individually.
Let us rewrite Line 15 of Table~\ref{tab:04} into the column-wise style:
\begin{equation} \label{eq:70}
{[\tilde{\mathbf{S}}]}_{\star, x} = {[\mathbf{S}]}_{\star,x} + {[\mathbf{M}_{K}]}_{\star,x} + {[(\mathbf{M}_{K})^T]}_{\star,x} \qquad (\forall x)
\end{equation}
Applying the following facts
\[{[{\mathbf{\Delta S}}]}_{\star, x} = {[\tilde{\mathbf{S}}]}_{\star, x} - {[\mathbf{S}]}_{\star,x}$ and ${[(\mathbf{M}_{K})^T]}_{\star,x} =  ({[\mathbf{M}_{K}]}_{x,\star})^T\]
into Eq.\eqref{eq:70} produces
\begin{equation} \label{eq:71}
{[{\mathbf{\Delta S}}]}_{\star, x} = {[\mathbf{M}_{K}]}_{\star,x} + ({[\mathbf{M}_{K}]}_{x,\star})^T \qquad (\forall x)
\end{equation}
This implies that, to compute one column of ${\mathbf{\Delta S}}$, we only need prepare one row and one column of $\mathbf{M}_{K}$.
To compute only the $x$-th row and $x$-th column of $\mathbf{M}_{K}$, there are two challenges:
(1) From Line 10 of Table~\ref{tab:04},
we notice that $\mathbf{M}_{K}$ is derived from the auxiliary vector $\bm \gamma$, and $\bm \gamma$ depends on the $i$-th and $j$-th column of old ${\mathbf{S}}$ according to Lines 3, 4, 6, 9 of Table~\ref{tab:04}.
Since the update edge $(i,j)$ can be arbitrary,
it is hard to determine which columns of old ${\mathbf{S}}$ will be used in future.
Thus, all our in-memory algorithms in Section \ref{sec:06} prepare $n \times n$ elements of ${\mathbf{S}}$ into memory, leading to $O(n^2)$ memory.
(2) According to Lines 10, 14, 15 of Table~\ref{tab:05},
it also requires $O(n^2)$ memory to iteratively compute $\mathbf{M}_{K}$.
It is not easy to use just linear memory for iteratively computing only one row and one column of $\mathbf{M}_{K}$.
In the next two subsections, we will address these two challenges, respectively.
%
\subsection{Avoid storing $n \times n$ elements of old $\mathbf{S}$}
%
\begin{figure*}[t]
\begin{minipage}{.495\textwidth}
\removelatexerror
\begin{algorithm}[H]
\small
\DontPrintSemicolon
\LinesNumbered
%\SetCommentSty{textsf}
\SetKwInOut{Input}{Input}
\SetKwInOut{Output}{Output}
%\SetKwFunction{Len}{Len}
\Input{an old digraph $G=(V,E)$, \\
       a collection of edges $\Delta G$ inserted into $G$, \\
       $x$-th column $[\mathbf{S}]_{\star, x}$ of old SimRank in $G$, \\
       number of iterations $K$, \ \
       damping factor $C$.}
\Output{$x$-th column $[\tilde{\mathbf{S}}]_{\star, x}$ of new SimRank in $G \cup \Delta G$}
\SetKwBlock{Begin}{\textbf{foreach} {edge $(i,j) \in \Delta G$}}{...~(Continue~on~right~side)}
initialize the transition matrix $\mathbf{Q}$ in $G$ ; \;
\lForEach {$v \in V$} {${{d}_{v}} \leftarrow$ in-degree of node $v$ in $G$ ; }
\SetAlgoLined
\Begin
 {  \SetAlgoVlined
    \lIf {$i \in V$} {
%%        initialize $\mathbf{x}_{0} \leftarrow \mathbf{e}_i$ \;      %%% s(*,i)
%%        \For {$t \leftarrow 1, 2, \cdots, K$} {
%%            $\mathbf{x}_{t+1} \leftarrow \mathbf{Q}^T \cdot \mathbf{x}_{t} $ \;
%%        }
%%        initialize $\mathbf{y} \leftarrow \mathbf{x}_{K+1}$ \;
%%        \For {$t \leftarrow 1, 2, \cdots, K$} {
%%            $\mathbf{y} \leftarrow \mathbf{x}_{K+1-t} + C \cdot \mathbf{Q} \cdot \mathbf{y} $ \;
%%        }
%%        $[\mathbf{S}]_{\star,i} \leftarrow (1-C) \cdot \mathbf{y} $ \;
        $[\mathbf{S}]_{\star,i} \leftarrow \PartialSim (\mathbf{Q}, i, K, C) $
        }
    \lIf {$j \in V$} {
        %%% s(*,j)
%%        initialize $\mathbf{x}_{0} \leftarrow \mathbf{e}_j$ \;      %%% s(*,j)
%%        \For {$t \leftarrow 1, 2, \cdots, K$} {
%%            $\mathbf{x}_{t+1} \leftarrow \mathbf{Q}^T \cdot \mathbf{x}_{t} $ \;
%%        }
%%        initialize $\mathbf{y} \leftarrow \mathbf{x}_{K+1}$ \;
%%        \For {$t \leftarrow 1, 2, \cdots, K$} {
%%            $\mathbf{y} \leftarrow \mathbf{x}_{K+1-t} + C \cdot \mathbf{Q} \cdot \mathbf{y} $ \;
%%        }
%%        $[\mathbf{S}]_{\star,j} \leftarrow (1-C) \cdot \mathbf{y} $ \;
        $[\mathbf{S}]_{\star,j} \leftarrow \PartialSim (\mathbf{Q}, j, K, C) $
    }
    \uIf(\tcp*[f]{Case (C0)}) {$i \in V$ and $j \in V$} {
        $\mathbf{w} \leftarrow \mathbf{Q}\cdot {{[\mathbf{S}]}_{\star,i}}$; \;
        $\lambda \leftarrow {{[\mathbf{S}]}_{i,i}}+\tfrac{1}{C} \cdot {[\mathbf{S}]}_{j,j}-2\cdot {[\mathbf{w}]}_{j} - \tfrac{1}{C} +1$ ; \;
        \uIf {${{d}_{j}}=0$} {
            $\mathbf{u} \leftarrow \mathbf{e}_j, \ \mathbf{v} := \mathbf{e}_i, \ {\bm \gamma} :=  \mathbf{w} +\frac{1}{2}{{[\mathbf{S}]}_{i,i}}\cdot {{\mathbf{e}}_{j}}$; \;}
        \Else {
            $\mathbf{u} \leftarrow \tfrac{1}{d_j+1} \mathbf{e}_j, \quad \mathbf{v} := \mathbf{e}_i-{[\mathbf{Q}]}_{j,\star}^T$ ; \;
            ${\bm\gamma} \leftarrow \tfrac{1}{({{d}_{j}}+1)} \big( \mathbf{w}-\frac{1}{C}  {{[\mathbf{S}]}_{\star,j}}+( \frac{\lambda }{2\left( {{d}_{j}}+1 \right)}+ \frac{1-C}{C} )  {{\mathbf{e}}_{j}} \big)$;
        }
        initialize ${{\bm{\xi }}_{0}} \leftarrow C \cdot \mathbf{e}_j,\quad {{\bm{\eta }}_{0}} \leftarrow {\bm\gamma}$; \;
        ${{\mathbf{m}}} \leftarrow C \cdot [{\bm\gamma}]_x \cdot \mathbf{e}_j, \ \ {{\mathbf{n}}} \leftarrow C \cdot [\mathbf{e}_j]_x \cdot {\bm\gamma} $; \;
        \For {$k=0,1,\cdots, K-1$} {
            ${{\bm{\xi }}_{k+1}} \leftarrow C \cdot \mathbf{Q}\cdot {{\bm{\xi }}_{k}} + C \cdot (\mathbf{v}^T\cdot {{\bm{\xi }}_{k}}) \cdot \mathbf{u}$ ; \;
            ${{\bm{\eta }}_{k+1}} \leftarrow  \mathbf{Q}\cdot {{\bm{\eta }}_{k}} + (\mathbf{v}^T\cdot {{\bm{\eta }}_{k}}) \cdot \mathbf{u}$ ;\;
            ${{\mathbf{m}}} \leftarrow [\bm{\eta }_{k+1}]_x \cdot {{\bm{\xi }}_{k+1}} +{{\mathbf{m}}}$ ; \;
            ${{\mathbf{n}}} \leftarrow {[{\bm{\xi }}_{k+1}]}_x \cdot \bm{\eta }_{k+1}+{{\mathbf{n}}}$ ; \;
        }
        $[{\mathbf{S}}]_{\star, x} \leftarrow [\mathbf{S}]_{\star, x} + \mathbf{m} + \mathbf{n}$ ; \;
        $d_j \leftarrow d_j+1, \quad \mathbf{Q} \leftarrow \mathbf{Q}  + \mathbf{u} \cdot \mathbf{v}^T $ ; \;
    }
    \uElseIf(\tcp*[f]{Case (C1)}) {$i \in V$ and $j \notin V$} {
        $\mathbf{y} \leftarrow C \cdot \mathbf{Q}\cdot {{[\mathbf{S}]}_{\star,i}}$ ; \;
        \If {$x=j$} {
            $z \leftarrow C \cdot {{[\mathbf{S}]}_{i,i}}+(1-C)$ ; \;
            $[\mathbf{S}]_{\star,x} \leftarrow  \left[ \renewcommand\arraystretch{1.2} \begin{array}{c}
               \mathbf{y}  \\ \hline
               z  \\
            \end{array} \right]$ ; \;
        }
        \Else {
            $[\mathbf{S}]_{\star,x} \leftarrow \left[ \renewcommand\arraystretch{1.2} \begin{array}{c}
               [\mathbf{S}]_{\star,x}   \\ \hline
               {[{\mathbf{y}}]}_x  \\
            \end{array} \right]$ ; \;
        }
$d_j \leftarrow 0, \quad V \leftarrow V \cup \{j\}, \quad \mathbf{Q} \leftarrow \left[ \begin{array}{c|c}
               \mathbf{Q} & \mathbf{0}  \\ \hline
               {{\mathbf{e}}_{i}^{T}} & 0  \\
            \end{array} \right]$; \;
}}
\caption{\IncSRAllP~($G, \Delta G, [\mathbf{S}]_{\star,x}, K, C$)}  \label{alg:07}
\end{algorithm}
\end{minipage}
\begin{minipage}{.495\textwidth}
\removelatexerror
\setcounter{algocf}{4}
\begin{algorithm}[H]
\small
\SetAlgoVlined
\DontPrintSemicolon
%\LinesNumbered
%
\SetKwBlock{Begin}{...~(Continued)}{end}
\Begin {
    \everypar={\nl}
    \setcounter{AlgoLine}{30}
    \uElseIf(\tcp*[f]{Case (C2)}) {$i \notin V$ and $j \in V$} {
    \everypar={\nl}
        \uIf {$x=i$} {
            $[\mathbf{S}]_{\star,x} \leftarrow  \left[ \renewcommand\arraystretch{1.2} \begin{array}{c}
               \mathbf{0}  \\ \hline
               1-C  \\
            \end{array} \right]$ ; \; }
        \Else {
                $\mathbf{z} \leftarrow \big( \tfrac{1}{2C ( {{d}_{j}}+1 )}\big( {{[\mathbf{S}]}_{j,j}}-{{(1-C)}^{2}} \big)+\tfrac{1-C}{C} \big){{\mathbf{e}}_{j}}-\tfrac{1}{C}{{[\mathbf{S}]}_{\star,j}}$ ;\;
                initialize ${{\bm{\xi }}_{0}} \leftarrow \mathbf{e}_j,\quad {{\bm{\eta }}_{0}} \leftarrow {\mathbf{z}}$ ; \;
                ${{\mathbf{m}}} \leftarrow {[\mathbf{z}]}_x \cdot \mathbf{e}_j, \quad {{\mathbf{n}}} \leftarrow {[\mathbf{e}_j]}_x \cdot \mathbf{z}$ ; \;
                \For {$k \leftarrow 0,1,\cdots, K-1$} {
                    ${{\bm{\xi }}_{k+1}} \leftarrow  C \cdot \mathbf{Q} \cdot {{\bm{\xi  }}_{k}} - \tfrac{C}{{{d}_{j}}+1}({{[\mathbf{Q}]}_{j,\star}} \cdot \bm{\xi }_{k}) \cdot  {{\mathbf{e}}_{j}}$;\;
                    ${{\bm{\eta }}_{k+1}} \leftarrow \mathbf{Q}\cdot {{\bm{\eta  }}_{k}} - \tfrac{1}{{{d}_{j}}+1}({{[\mathbf{Q}]}_{j,\star}} \cdot \bm{\eta  }_{k}) \cdot {{\mathbf{e}}_{j}}$ ;\;
                    ${{\mathbf{m}}} \leftarrow [\bm{\eta }_{k+1}]_x \cdot {{\bm{\xi }}_{k+1}} +{{\mathbf{m}}}$ ; \;
                    ${{\mathbf{n}}} \leftarrow {[{\bm{\xi }}_{k+1}]}_x \cdot \bm{\eta }_{k+1}+{{\mathbf{n}}}$ ; \;
                }
            $[\mathbf{S}]_{\star,x} \leftarrow  \left[ \renewcommand\arraystretch{1.2} \begin{array}{c}
               [\mathbf{S}]_{\star,x}  + \frac{C}{d_j+1} \cdot (\mathbf{m} + \mathbf{n})  \\ \hline
               0  \\
            \end{array} \right]$ ; \;
        }
        $d_i \leftarrow 0, \quad d_j \leftarrow d_j + 1, \quad V \leftarrow V \cup \{i\}$ ; \;
        $\mathbf{Q} \leftarrow \left[ \begin{array}{c|c}
           {\mathbf{Q}-\tfrac{1}{{{d}_{j}}+1}{{\mathbf{e}}_{j}} {{[\mathbf{Q}]}_{j,\star}}} & \tfrac{1}{{{d}_{j}}+1}{{\mathbf{e}}_{j}}  \\ \hline
           \mathbf{0} & 0  \\
        \end{array} \right] $; \;
    }
    \everypar={\nl} \ElseIf(\tcp*[f]{Case (C3)}) {$i \notin V$ and $j \notin V$} {
    \everypar={\nl}
        \uIf {$x=i$} {
            $[\mathbf{S}]_{\star,x} \leftarrow  \left[ \begin{array}{c}
               \mathbf{0} \\  \hline
               1-C  \\
               0 \\
            \end{array}  \right] \begin{array}{c}
               \\
               (i) \\
               (j) \\
            \end{array}$ ; \;
        }
        \uElseIf {$x=j$} {
            $[\mathbf{S}]_{\star,x} \leftarrow  \left[ \begin{array}{c}
               \mathbf{0} \\  \hline
               0  \\
               1-C^2 \\
            \end{array}  \right] \begin{array}{c}
               \\
               (i) \\
               (j) \\
            \end{array}$ ; \;
        }
        \Else {
            $[\mathbf{S}]_{\star,x} \leftarrow  \left[ \begin{array}{c}
               [\mathbf{S}]_{\star,x}  \\  \hline
               0  \\
               0 \\
            \end{array}  \right] \begin{array}{c}
               \\
               (i) \\
               (j) \\
            \end{array}$ ; \;
        }
        $\mathbf{{Q}} \leftarrow \left[ \begin{array}{c|c}
               \mathbf{Q} & \mathbf{0}  \\  \hline
               \mathbf{0} & \left[ \begin{matrix}
               0 & 0  \\
               1 & 0  \\
            \end{matrix} \right]  \\
            \end{array}  \right] \begin{array}{c}
               \\
               (i) \\
               (j) \\
            \end{array} $ ; \;
        $d_i \leftarrow 0, \quad d_j \leftarrow 0, \quad V \leftarrow V \cup \{i, j \}$ ; \;
    }
    \everypar={\nl} $G \leftarrow G \cup \{(i,j)\}$ ; \;
}
\everypar={\nl}  \Return $[\tilde{\mathbf{S}}]_{\star,x} \leftarrow [\mathbf{S}]_{\star,x} $ ; \;
\caption{\textbf{(Continued)}~~\IncSRAllP}
\end{algorithm}
\end{minipage}
\end{figure*}
Our above analysis imply that, to compute each column ${[{\mathbf{\Delta S}}]}_{\star, x}$, we only need prepare two columns information ($i$-th and $j$-th) from old $\mathbf{S}$.
Since the update edge $(i,j)$ can be arbitrary, there are no prior knowledge which $i$-th and $j$-th columns in old $\mathbf{S}$ will be used.
As opposed to Algorithms~\ref{alg:03}--\ref{alg:06} that memoize all $(n^2)$ pairs of old $\mathbf{S}$,
we use the following scalable method to compute only the $i$-th and $j$-th columns of old $\mathbf{S}$ on demand in linear memory.
Specifically, based on our previous work \cite{Yu2015} on partial-pairs SimRank retrieval,
we can readily verify that the following iterations will yield $[\mathbf{S}]_{\star,i}$ and $[\mathbf{S}]_{\star,j}$ in just $O(Kn+m)$ memory.
\[  \renewcommand\arraystretch{1}
\scalebox{0.85}{$ \begin{tabular}{|l|l|}
\hline
\textrm{initialize} $\mathbf{x}_{0} \leftarrow \mathbf{e}_i$                                    & \textrm{initialize} $\mathbf{x}_{0} \leftarrow \mathbf{e}_j$  \\
\textbf{for} $t \leftarrow 1, 2, \cdots, K$                                                     & \textbf{for} $t \leftarrow 1, 2, \cdots, K$ \\
\qquad  $\mathbf{x}_{t+1} \leftarrow \mathbf{Q}^T \cdot \mathbf{x}_{t} $                        & \qquad  $\mathbf{x}_{t+1} \leftarrow \mathbf{Q}^T \cdot \mathbf{x}_{t} $ \\
\textrm{initialize} $\mathbf{y} \leftarrow \mathbf{x}_{K+1}$                                    & \textrm{initialize} $\mathbf{y} \leftarrow \mathbf{x}_{K+1}$  \\
\textbf{for} $t \leftarrow 1, 2, \cdots, K$                                                     & \textbf{for} $t \leftarrow 1, 2, \cdots, K$ \\
\qquad  $\mathbf{y} \leftarrow \mathbf{x}_{K+1-t} + C \cdot \mathbf{Q} \cdot \mathbf{y} $       & \qquad  $\mathbf{y} \leftarrow \mathbf{x}_{K+1-t} + C \cdot \mathbf{Q} \cdot \mathbf{y}  $ \\
$[\mathbf{S}]_{\star,i} \leftarrow (1-C) \cdot \mathbf{y} $                                     & $[\mathbf{S}]_{\star,j} \leftarrow (1-C) \cdot \mathbf{y} $\\ \hline
\end{tabular} $}
\]
Next, $[\mathbf{S}]_{i,i}$ is obtained from the $i$-th element of $[\mathbf{S}]_{\star,i}$, and $[\mathbf{S}]_{j,j}$ from the $j$-th element of $[\mathbf{S}]_{\star,j}$.
Having prepared $[\mathbf{S}]_{\star,i}, [\mathbf{S}]_{\star,j}, [\mathbf{S}]_{i,i}$, and $ [\mathbf{S}]_{j,j}$,
we follow Lines 3, 4, 6, 9 of Table~\ref{tab:04} to derive the vector $\bm \gamma$ in linear memory.
In addition, since Line 15 of Table~\ref{tab:04} can be computed column-wisely via Eq.\eqref{eq:71}.
Throughout all lines in Table~\ref{tab:04}, we do not need store $n^2$ pairs of old $\mathbf{S}$ in memory.
However, $O(n^2)$ memory is still required to store $\mathbf{M}_k$.
In the next subsection, we will show how to avoid $O(n^2)$ memory to compute $\mathbf{M}_k$.

%
\subsection{Compute ${[\mathbf{M}_K]}_{\star, x}$ and ${[\mathbf{M}_K]}_{x, \star}$ in linear memory}
%
Using $\bm \gamma$, we next devise our method based on Table~\ref{tab:05},
aiming to use linear memory to compute each column ${[\mathbf{M}_K]}_{\star, x}$ and each row ${[\mathbf{M}_K]}_{x, \star}$ for Eq.\eqref{eq:71}.
In Table~\ref{tab:05}, our key observation is that $\mathbf{M}_k$ is the summation of the outer product of two vectors.
Due to this structure, instead of using $O(n^2)$ memory to store $\mathbf{M}_k$,
we can use only $O(n)$ memory to compute ${[\mathbf{M}_K]}_{\star, x}$  and ${[\mathbf{M}_K]}_{x,\star}$.
Specifically, we can compute Lines~10 and 14 of Table~\ref{tab:05} in a column-wise style for ${[\mathbf{M}_K]}_{\star, x}$ as follows:
\[  \renewcommand\arraystretch{1}
\scalebox{1}{$ \begin{tabular}{|l|l|}
\hline
${[{\mathbf{M}}_{0}]}_{\star, x } \leftarrow C \cdot {[\bm\gamma]}_x \cdot  \mathbf{e}_j $                                                      \\
\textbf{for~} $ k \leftarrow 0, \cdots,K-1$                                                                                                     \\
\qquad ${[{\mathbf{M}}_{k+1}]}_{\star, x } \leftarrow [\bm{\eta }_{k+1}]_{x} \cdot {{\bm{\xi }}_{k+1}} + {[{\mathbf{M}}_{k}]}_{\star, x }$      \\
\hline
\end{tabular} $}
\]
and in a row-wise style for ${[\mathbf{M}_K]}_{x,\star}$ as follows:
\[  \renewcommand\arraystretch{1}
\scalebox{1}{$ \begin{tabular}{|l|l|}
\hline
${[{\mathbf{M}}_{0}]}_{x, \star} \leftarrow C \cdot {[\mathbf{e}_j]}_x \cdot \bm\gamma  $ \\
\textbf{for~} $ k \leftarrow 0, \cdots,K-1$ \\
\qquad ${[{\mathbf{M}}_{k+1}]}_{x, \star} \leftarrow [{{\bm{\xi }}_{k+1}}]_{x} \cdot \bm{\eta }_{k+1} + {[{\mathbf{M}}_{k}]}_{x, \star}$ \\
\hline
\end{tabular} $}
\]
Fig.~\ref{fig:04} pictorially visualizes the column-wise computation of ${[{\mathbf{M}}_{K}]}_{\star, x }$.
Having computed ${[{\mathbf{M}}_{K}]}_{\star, x }$ and ${[{\mathbf{M}}_{K}]}_{x, \star}$,
we can use Eq.\eqref{eq:71} to derive the column ${[{\mathbf{\Delta S}}]}_{\star, x}$ of ${\mathbf{\Delta S}}$.

The main advantage of our method is that, throughout the entire updating process,
we need not store $n \times n$ pairs of $\mathbf{M}_k$ and $\mathbf{S}$,
and thereby, significantly reduce the memory usage from $O(n^2)$ to $O(Kn+m)$.
In addition to the insertion case (C0), our memory-efficient methods are applicable to other insertion cases in Subsection~\ref{sec:04c}.
The complete algorithm, denoted as \IncSRAllP, is described in Algorithm~\ref{alg:07}.
{\IncSRAllP} is a memory-efficient version of Algorithms~\ref{alg:03}--\ref{alg:06}.
It includes a procedure {\PartialSim} that allows us to compute two columns information of old $\mathbf{S}$ on demand in linear memory,
rather than store $n^2$ pairs of old $\mathbf{S}$ in memory.
In response to each edge update $(i,j)$,
once the two old columns $\mathbf{S}_{\star,i}$ and $\mathbf{S}_{\star,j}$ are computed via {\PartialSim} for updating the $x$-th column $[\mathbf{\Delta S}]_{\star,x}$,
they can be memoized in only $O(n)$ memory and reused later to compute another $y$-th column $[\mathbf{\Delta S}]_{\star,y}$ in response to the edge update $(i,j)$.

%Precisely, we can first partition the new entire SimRank matrix $\mathbf{\tilde{S}}$ into column vectors $[\mathbf{\tilde{S}}]_{\star,1}, [\mathbf{\tilde{S}}]_{\star,2}, \cdots$,
%and then update every column $[\mathbf{\tilde{S}}]_{\star,x}$ separately.
%
%For example, for insertion case (C1) in Subsection~\ref{sec:04c},
%it is not necessary to compute all columns of new $\mathbf{\tilde{S}}$ simultaneously by storing the \emph{entire} old $\mathbf{S}$ in Algorithm~\ref{alg:03}.
%Instead, we can compute $\tilde{\mathbf{S}}$ (Line~\ref{ln:a03-04}) column by column: % as follows:
%\[\tilde{\mathbf{S}} := \left[ \renewcommand\arraystretch{1.2} \begin{array}{c|c}
%\smash{\overbrace{\vphantom{\mathbf{S}}{\mathbf{S}}}^{n {\textrm{ cols}}}}  & \smash{\overbrace{\vphantom{\mathbf{S}}{\mathbf{y}}}^{{\textrm{col }}j}} \\ \hline
%%   \mathbf{S} & \mathbf{y}  \\ \hline
%   {{\mathbf{y}}^{T}} & z  \\
%\end{array} \right]
%\quad \Rightarrow \quad
%\begin{tabular}{|p{3.8cm}|}
%  \hline
%  \textrm{memoize} $\mathbf{y}$ \textrm{and} $z$  \\
%  \textbf{for each} $x \leftarrow 1, \cdots,n$ \\
%  \quad \textrm{set} $[\tilde{\mathbf{S}}]_{\star,x} \leftarrow \left[ \renewcommand\arraystretch{1.2} \begin{array}{c}
%   [\mathbf{S}]_{\star,x}   \\ \hline
%   [{{\mathbf{y}}}]_{x}  \\
%\end{array} \right]$ \\
%  \textrm{set} $[\tilde{\mathbf{S}}]_{\star,n+1} \leftarrow \left[ \renewcommand\arraystretch{1.2} \begin{array}{c}
%   \mathbf{y}  \\ \hline
%   z  \\
%\end{array} \right]$ \\
%  \hline
%\end{tabular}
%\]
%Similarly, one can modify Lines~\ref{ln:a01-18}, \ref{ln:a04-10}, \ref{ln:a05-02}, respectively, in Algorithms~\ref{alg:01}, \ref{alg:04}, \ref{alg:05} into a column-by-column style.
%
%In this way, although Algorithms~\ref{alg:03} and \ref{alg:05} can achieve $O(dn)$ memory,
%Algorithms~\ref{alg:01} and \ref{alg:04} still require $O(n^2)$.
%The reason is that the computation of $\mathbf{M}_k$ in Line~\ref{ln:a01-17} (\Resp \ref{ln:a04-08}) dominates the memory of Algorithm~\ref{alg:01} (\Resp \ref{alg:04}).
%To resolve this problem,
%we also need split the computation of the entire $\mathbf{M}_k$ in a column-by-column fashion.
%Note that $\mathbf{M}_k$ in Line~\ref{ln:a01-17}, Algorithm~\ref{alg:01} and $\mathbf{M}_k$ in Line~\ref{ln:a04-08}, Algorithm~\ref{alg:04} exhibit a similar structure, \ie
%$\mathbf{M}_k$ is the summation of the outer product of two vectors:
%\[ \renewcommand\arraystretch{1}
%\begin{tabular}{l}
%\textbf{for~} $ k \leftarrow 0, \cdots,K-1$ \\
%\quad \textrm{set} ${{\mathbf{M}}_{k+1}}\leftarrow {{\bm{\xi }}_{k+1}}\cdot \bm{\eta }_{k+1}^{T}+{{\mathbf{M}}_{k}}$ \\
%\textrm{update} $\tilde{\mathbf{S}}\leftarrow \mathbf{S} + \mathbf{M}_{K} + \mathbf{M}_{K}^T$ \\
%\end{tabular}
%\]
%Thus, we can split the above computation column-wisely:
%\[  \renewcommand\arraystretch{1}
%\begin{tabular}{l}
%\textbf{for} $x \leftarrow 1, \cdots,n$ \\
%\quad \textrm{initialize} $[\tilde{\mathbf{S}}]_{\star,x} \leftarrow [\mathbf{S}]_{\star,x}$, \quad $\mathbf{y} \leftarrow \mathbf{0}$, \quad $\mathbf{z} \leftarrow \mathbf{0}$ \\
%\quad \textbf{for~} $ k \leftarrow 0, \cdots,K-1$ \\
%\quad \quad \textrm{set} $\mathbf{y} \leftarrow [\bm{\eta }_{k+1}]_{x}\cdot {{\bm{\xi }}_{k+1}}+ \mathbf{y} $ \\
%\quad \quad \textrm{set} $\mathbf{z} \leftarrow [\bm{\xi }_{k+1}]_{x}\cdot {{\bm{\eta }}_{k+1}}+ \mathbf{z}$ \\
%\quad \quad \textrm{update} $[\tilde{\mathbf{S}}]_{\star,x} \leftarrow [\mathbf{\tilde{S}}]_{\star,x} + \mathbf{y} + \mathbf{z}$ \\
%\end{tabular}
%\]
%The main advantage of our revision is that, throughout the entire updating process,
%we need not store the entire matrix $\mathbf{M}_k$ and $\mathbf{S}$,
%and thereby, significantly reduce the memory usage from $O(n^2)$ to $O(dn)$.

\vspace{5pt} \noindent \textbf{Correctness.} \
{\IncSRAllP} correctly returns similarity.
It consists of four update cases:
lines 6--22 for Case (C0),
lines 23--30 for Case (C1),
lines 31--45 for Case (C2), and
lines 46--54 for Case (C3).
The correctness of each case can be verified by Theorems~\ref{thm:03}, \ref{thm:08}, \ref{thm:06}, and \ref{thm:07}, respectively.
For instance, to verify the correctness for Case (C0),
we apply successive substitution to \textsf{{for}-loop} in lines~14--21, which produces the following result:
\[
[\tilde{\mathbf{S}}]_{u,v} = [\mathbf{{S}}]_{u,v}+ \sum_{k=1}^K {[\bm{\xi }_{k}]_{u} \cdot [\bm{\eta }_{k}]_{v}} +   \sum_{k=1}^K {[\bm{\xi }_{k}]_{v} \cdot [\bm{\eta }_{k}]_{u}}
\]
This is consistent with Eq.\eqref{eq:70}, implying that our memory-efficient method does not compromise any accuracy for scalability.
\setcounter{algocf}{0}
\begin{algorithm}[!t]
\small
\SetAlgorithmName{Procedure}{}
\DontPrintSemicolon
\LinesNumbered
%\SetCommentSty{textsf}
\SetKwInOut{Input}{Input}
\SetKwInOut{Output}{Output}
%\SetKwFunction{Len}{Len}
\Input{transition matrix $\mathbf{Q}$ in $G$, \\
       query node $q$, \\
       number of iterations $K$, \\
%       the number of iterations $K$, \\
       damping factor $C$.}
\Output{$q$-th column $[{\mathbf{{S}}}]_{\star,q}$ of SimRank scores in $G$.}

initialize $\mathbf{x}_{0} \leftarrow \mathbf{e}_q$ \;      %%% s(*,q)
\For {$t \leftarrow 1, 2, \cdots, K$} {
    $\mathbf{x}_{t+1} \leftarrow \mathbf{Q}^T \cdot \mathbf{x}_{t} $ \;
}
initialize $\mathbf{y} \leftarrow \mathbf{x}_{K+1}$ \;
\For {$t \leftarrow 1, 2, \cdots, K$} {
    $\mathbf{y} \leftarrow \mathbf{x}_{K+1-t} + C \cdot \mathbf{Q} \cdot \mathbf{y} $ \;
}
\Return $[\mathbf{S}]_{\star,q} \leftarrow (1-C) \cdot \mathbf{y} $ \;
\caption{\PartialSim $(\mathbf{Q}, q, K, C)$}  \label{alg:08}
\end{algorithm}

\begin{figure}[!t] \centering
  \includegraphics[width=.9\linewidth]{e04.eps}
  \caption{Memory usage reduction by partitioning $\mathbf{M}_K$ in a column-by-column style} \label{fig:04} %\vspace{-10pt}
\end{figure}
%
%we can verify that, in our above methods,
%every new score $[\tilde{\mathbf{S}}]_{u,v}$ has the consistent result\footnote{Here, our main focus is devoted to $\tilde{\mathbf{S}}$ in Algorithm~\ref{alg:01}.
%Note that similar techniques can be applied to ${\mathbf{\tilde{S}_{11}}}$ in Algorithm~\ref{alg:04}.}:
%\[
%[\tilde{\mathbf{S}}]_{u,v} = [\mathbf{{S}}]_{u,v}+ \sum_{k=1}^K {[\bm{\xi }_{k}]_{u} \cdot [\bm{\eta }_{k}]_{v}} +   \sum_{k=1}^K {[\bm{\xi }_{k}]_{v} \cdot [\bm{\eta }_{k}]_{u}}
%\]
%Thus, our partitioning approach does not compromise accuracy for high memory efficiency,
%which is achieved by the separable structure of ${\mathbf{M}}_K$ described as the sum of rank-one matrices,
%as pictorially depicted in Fig.~\ref{fig:04}.
%%
        It is worth mentioning that {\IncSRAllP} can be also combined with our batch updating method in Section~\ref{sec:08}.
        This will speed up the dynamical update of SimRank further, with $O(n(\max_{t=1}^{|B|}\delta_t) + m + Kn)$ memory.
        Here $O(n\delta_t)$ memory is needed to store $\delta_t$ columns of $\mathbf{S}$ when $[\mathbf{S}]_{\star,I}$ is required for processing the $t$-th block.
%
%The memory usage of {\IncBSR}, if we incorporate the column-wise technique of Section~\ref{sec:07}, can be bounded by $O(n(\max_{t=1}^{|B|}\delta_t) +nd)$ in the worst case,
%because $O(n\delta_t)$ memory is needed to store $\delta_t$ columns of $\mathbf{S}$ when $[\mathbf{S}]_{\star,I}$ is required for the $t$-th block.
%
%\begin{tabular}{|l|l|l|l|l|}
%\hline
%Phase                                         & \# Line & Time                    & Space                  & Operations                                      \\ \hline
%\multirow{3}{*}{Preprocessing}                & 1       & O(m)                    & O(m)                   & normalize nonzero elements of $W$               \\ \cline{2-5}
%                                              & 2       & O(n\textasciicircum 2r) & O(nr)                  & Gram-Schmidt decomposition                      \\ \cline{2-5}
%                                              & 3       & O(r\textasciicircum 2n) & O(nr)                  & multiplication of Hr\textasciicircum T and V\_r \\ \hline
%\multirow{2}{*}{Computing Sr in rxr Subspace} & 4       & O(r)                    & O(r)                   & initialize Sr(0)                                \\ \cline{2-5}
%                                              & 5-6     & O(Kr\textasciicircum 3) & O(r\textasciicircum 2) & iteratively compute Sr(K)                       \\ \hline
%Computing S in nxn space                      & 7       & O(rn\textasciicircum 2) & O(n\textasciicircum 2) & calculate S from Sr(K)                          \\ \hline
%\end{tabular}
%
}
%
\section{Experimental Evaluation} \label{sec:09} %\label{sec:05}
%
In this section, we present a comprehensive experimental study on real and synthetic datasets,
to demonstrate
(i) the fast computational time of \IncSR~to incrementally update SimRanks on large time-varying networks, % ation in terms of time and space,
(ii) the pruning power of {\IncSR} that can discard unnecessary incremental updates outside ``affected areas'';
(iii) the exactness of \IncSR, as compared with \IncSVD;
(iv) the high efficiency of our complete scheme that integrates {\IncSR} with {\IncUSRtwo, \IncUSRthree, \IncUSRfour}
to support link updates that allow new node insertions;
(v) the fast computation time of our batch update algorithm {\IncBSR} against the unit update method {\IncSR};
(vi) the scalability of our memory-efficient algorithm {\IncSRAllP} in Section~\ref{sec:07} on million-scale large graphs for dynamical updates;
(vii) the performance comparison between {\IncSRAllP} and {\LTSF} in terms of computational time, memory space, and top-$k$ exactness;
(viii) the average updating time and memory usage of {\IncSRAllP} for each case of edge updates.
\begin{table*}[t]
\centering \scalebox{0.8}{$
\begin{tabular}{|l|ll|rr|rl|l|}
\hline
\multicolumn{3}{|c|}{\textbf{Datasets}}                & \textbf{$|V|$} & \textbf{$|E|$} & \multicolumn{2}{c|}{\textbf{\# of Pairs To Be Assessed}} & \multicolumn{1}{c|}{\textbf{Description}}   \\ \hline
\parbox[t]{2mm}{\multirow{2}{*}{\rotatebox[origin=c]{90}{Small}}}  & \DBLP  & (DBLP)            & 13,634         & 93,560         & 185,885,956                 & $(={|V|}^2) $                  & DBLP citation network                      \\
                        & \CITH  & (cit-HepPh)       & 34,546         & 421,578        & 1,193,426,116               & $(={|V|}^2) $                  & High Energy Physics citation network \\ \hline
\parbox[t]{2mm}{\multirow{3}{*}{\rotatebox[origin=c]{90}{Medium}}} & \YOUTU & (YouTube)         & 178,470        & 953,534        & 1,784,700,000               & $(= {10}^4 {|V|})$             & Social network of YouTube videos           \\
                        & \WEBB  & (web-BerkStan)    & 685,230        & 7,600,595      & 6,852,300,000               & $(= {10}^4 {|V|})$             & Web graph of Berkeley and Stanford         \\
                        & \WEBG  & (web-Google)      & 916,428        & 5,105,039      & 9,164,280,000               & $(= {10}^4 {|V|})$             & Web graph from Google                      \\ \hline
\parbox[t]{2mm}{\multirow{4}{*}{\rotatebox[origin=c]{90}{Large}}}  & \CITP  & (cit-Patents)     & 3,774,768      & 16,518,948     & 3,774,768,000               & $(= {10}^3 {|V|})$             & Citation network among US Patents          \\
                        & \SOCL  & (soc-LiveJournal) & 4,847,571      & 68,993,773     & 4,847,571,000               & $(= {10}^3 {|V|})$             & LiveJournal online social network          \\
                        & \UK    & (uk-2005)         & 39,459,925     & 936,364,282    & 39,459,925,000              & $(= {10}^3 {|V|})$             & Web graph from 2005 crawl of .uk domain    \\
                        & \IT    & (it-2004)         & 41,291,594     & 1,150,725,436  & 41,291,594,000              & $(= {10}^3 {|V|})$             & Web graph from 2004 crawl of .it domain \\ \hline
\end{tabular}
%
%\multicolumn{2}{c|}{Data ($|E|$)}    & \IncBSR  & \IncSRAll & (\%) \\ \hline
%\parbox[t]{2mm}{\multirow{3}{*}{\rotatebox[origin=c]{90}{\DBLP}}}    & 75K  & 14.9    & 16.3     & 8.8  \\
%                         & 83K  & 70.5    & 82.0     & 14.0 \\
%                         & 91K  & 315.9   & 363.8    & 13.1 \\ \hline
%\parbox[t]{2mm}{\multirow{3}{*}{\rotatebox[origin=c]{90}{\CITH}}}    & 395K & 50.5    & 54.5     & 7.3  \\
%                         & 407K & 241.9   & 283.5    & 14.6 \\
%                         & 419K & 1869.1  & 2357.4   & 20.7 \\ \hline
%\parbox[t]{2mm}{\multirow{3}{*}{\rotatebox[origin=c]{90}{\YOUTU}}} & 889K & 876.6   & 921.9    & 4.9  \\
%                         & 901K & 2756.8  & 3297.4   & 16.4 \\
%                         & 913K & 10256.1 & 12109.2  & 15.3 \\ \hline
$}
  \caption{Description of Real-World Datasets} \label{tab:03}
\end{table*}
%
\subsection{Experimental Settings}
%
\noindent \textbf{Datasets.} \
We adopt both real and synthetic datasets.
The real datasets include small-scale ({\DBLP} and {\CITH}), medium-scale ({\YOUTU}, {\WEBB} and {\WEBG}), and large-scale graphs ({\CITP}, {\SOCL}, {\UK}, and {\IT}).
Table~\ref{tab:03} summarises the description of these datasets.

(Please refer to Appendix~\ref{app:05} for details.)

%%\emph{Synthetic Data.}
To generate synthetic graphs and updates, we adopt \textsf{GraphGen}\footnote{http://www.cse.ust.hk/graphgen/} generation engine.
The graphs are controlled by (a) the number of nodes $|V|$, and (b) the number of edges $|E|$.
We produce a sequence of graphs that follow the linkage generation model \cite{Garg2009}.
To control graph updates, we use two parameters simulating real evolution:
(a) update type (edge/node insertion or deletion), and (b) the size of updates $|\Delta G|$.

\noindent \textbf{Algorithms.} \
We implement the following algorithms:
(a) \IncSVD,
the SVD-based link-update algorithm \cite{Li2010}; % via SVD,
(b) \IncUSR, our incremental method without pruning;
(c) \Batch, the batch SimRank method via fine-grained memoization \cite{Yu2013};
(d) \IncSR, our incremental method with pruning power but not supporting node insertions;
(e) \IncSRAll, our complete enhanced version of {\IncSR} that allows node insertions by incorporating {\IncUSRtwo, \IncUSRthree, and \IncUSRfour};
(f) \IncBSR, our batch incremental update version of {\IncSR};
(g) \IncSRAllP, our memory-efficient version of {\IncSRAll} that dynamically computes the SimRank matrix column by column without the need to store all pairs of old similarities;
(h) \LTSF, the log-based implementation of the existing competitor, TSF \cite{Shao2015}, which supports dynamic SimRank updates for top-$k$ querying.


\noindent \textbf{Parameters.} \
We set the damping factor $C=0.6$, as used in \cite{Jeh2002}.
By default, the total number of iterations is set to $K=15$ to guarantee accuracy ${C}^{K} \le 0.0005$~\cite{Lizorkin2008}.
On {\CITH} and {\YOUTU}, we set $K=10$;
On large graphs ({\CITP}, {\SOCL}, {\UK}, and {\IT}), we set $K=5$.
The target rank $r$ for {\IncSVD} is a speed-accuracy trade-off, we set $r=5$ in our time evaluation since,
as shown in the experiments of \cite{Li2010},
the highest speedup is achieved when $r=5$.
In our exactness evaluation, we shall tune this value.
For {\LTSF} algorithm, we set the number of one-way graphs $R_g = 100$, and the number of samples at query time $R_q=20$, as suggested in \cite{Shao2015}.

All the algorithms are implemented in Visual C++ and Matlab.
%Each experiment is run 5 times, and the average results are reported here.
For small-scale graphs, we use a machine with an Intel Core 2.80GHz CPU and 8GB RAM.
For medium- and large-scale graphs, we use a processor with Intel Core i7-6700 3.40GHz CPU and 64GB RAM. %, running Windows 8.

%\begin{figure*}[ht]
%\centering
%\begin{minipage}[b]{.70\linewidth}
%   \subfloat[Time Efficiency of Incremental SimRank on Real Data]{
%    \includegraphics[width=.96\linewidth,height=3cm]{exp_01_02_03}%
%    \label{fig:exp_01_02_03}
%    }
%\end{minipage} \
%\begin{minipage}[b]{.27\linewidth}
%   \subfloat[\% of Lossless SVD Rank \wrt $|\Delta E|$]{
%    \includegraphics[width=.96\linewidth,height=3cm]{exp_04}%
%    \label{fig:exp_04}
%    }
%\end{minipage} \\[3pt]
%\begin{minipage}[b]{.44\linewidth}
%   \subfloat[Time Efficiency of Incremental SimRank on Synthetic Data]{
%    \includegraphics[width=.96\linewidth,height=3cm]{exp_05_06}%
%    \label{fig:exp_05_06}
%    }
%\end{minipage} \
%\begin{minipage}[b]{.26\linewidth}
%   \subfloat[Effect of Pruning]{
%    \includegraphics[width=.96\linewidth]{exp_07}%
%    \label{fig:exp_07} \
%    }
%\end{minipage} \
%\begin{minipage}[b]{.26\linewidth}
%   \subfloat[\% of Affected Areas \wrt $|\Delta E|$]{
%    \includegraphics[width=.96\linewidth]{exp_08}%
%    \label{fig:exp_08} \
%    }
%\end{minipage}\\
%%\begin{minipage}[b]{.28\linewidth}
%%   \subfloat[Amortized Space on Real Data]{
%%    \includegraphics[width=.99\linewidth]{exp_09_10}%
%%    \label{fig:exp_09_10}
%%    }
%%\end{minipage} \\[3pt]
% \caption{Performance Evaluations of \IncUSR~and \IncSR~on Real and Synthetic Datasets} % \vspace{-17pt}
%\end{figure*}
%%
%
\subsection{Experimental Results}
%
%
\subsubsection{Time Efficiency of {\IncSR} and {\IncUSR}}
%
\begin{figure*}
\centering
\begin{minipage}[t]{0.71\linewidth}
\centering
  \includegraphics[width=\linewidth]{exp_01_02_03} \\
  \caption{Time Efficiency on Real Data ($\Delta E$ does not accompany new nodes)}\label{fig:exp_01_02_03}
\end{minipage}
\begin{minipage}[t]{0.28\linewidth}
\centering
  \includegraphics[width=.95\linewidth]{exp_04}
  \caption{\% of Lossless SVD Rank}\label{fig:exp_04}
\end{minipage}
\end{figure*}
\begin{figure*}
\centering
\begin{minipage}[t]{.44\linewidth}
\centering
  \includegraphics[width=\linewidth]{exp_05_06} \\
  \caption{Time Efficiency on Synthetic Data}\label{fig:exp_05_06}
\end{minipage}
\begin{minipage}[t]{0.26\linewidth}
\centering
  \includegraphics[width=.95\linewidth]{exp_07}
  \caption{Pruning Power}\label{fig:exp_07}
\end{minipage}
\begin{minipage}[t]{0.26\linewidth}
\centering
  \includegraphics[width=.95\linewidth]{exp_08}
  \caption{\% of Affected Areas}\label{fig:exp_08}
\end{minipage}
\end{figure*}
We first evaluate the computational time of {\IncSR} and {\IncUSR} against {\IncSVD} and {\Batch} on real datasets.

Note that, to favor {\IncSVD} that only works on small graphs (due to memory crash for high-dimension SVD $n>10^5$),
we just use {\IncSVD} on {\DBLP} and {\CITH}. % are used, though \IncSR~works well on a variety of graphs (\eg \YOUTU, \SYN).

Fig.\ref{fig:exp_01_02_03} depicts the results when edges are added to \DBLP, \CITH, \YOUTU, respectively.
For each dataset, we fix $|V|$ and increase $|E|$ by $|\Delta E|$, as shown in the $x$-axis.
Here, the edge updates are the differences between snapshots \wrt the ``year'' (\Resp ``video age'') attribute of \DBLP, \CITH~(\Resp \YOUTU),
reflecting their real-world evolution.
We observe the following.
(1) \IncSR~\emph{always} outperforms \IncSVD~and \IncUSR~when edges are increased.
For example, on \DBLP, when the edge changes are 10.7\%,
the time for \IncSR~(83.7s) is 11.2x faster than \IncSVD~(937.4s), and 4.2x faster than \IncUSR~(348.7s).
This is because \IncSR~employs a rank-one matrix method to update the similarities,
with an effective pruning strategy to skip unnecessary recomputations,
as opposed to \IncSVD~that entails rather expensive costs to incrementally update the SVD.
The results on \CITH~are more pronounced, \eg
\IncSR~is 30x better than \IncSVD~when $|E|$ is increased to 401K.
%On \YOUTU, \IncSVD~fails due to the memory crash for SVD.
(2) \IncSR~is consistently better than \Batch~when the edge changes are fewer than 19.7\% on \DBLP, and 7.2\% on \CITH.
When link updates are 5.3\% on \DBLP~(\Resp 3.9\% on \CITH), \IncSR~improves \Batch~by 10.2x (\Resp 4.9x).
This is because (i) \IncSR~can exploit the sparse structure of $\mathbf{\Delta S}$ for incremental update,
and (ii) small link perturbations may keep $\mathbf{\Delta S}$ sparsity.
Hence, \IncSR~is highly efficient when link updates are small.
(3) The computational time of \IncSR, \IncUSR, \IncSVD, unlike \Batch, is sensitive to the edge updates $|\Delta E|$.
The reason is that \Batch~needs to reassess all similarities from scratch in response to link updates,
whereas \IncSR~and \IncUSR~can reuse the old information in SimRank for incremental updates.
In addition, \IncSVD~is too sensitive to $|\Delta E|$,
as it entails expensive tensor products to compute SimRank from the updated SVD matrices. %In contrast, \IncSR~is less sensitive than \IncSVD~as it \emph{directly} computes SimRank changes \wrt link updates, without the need of computing SVD.

Fig.\ref{fig:exp_04} shows the target rank $r$ required for the Li \etal\!\!'s {lossless} SVD approach \wrt the edge changes $|\Delta E|$ on \DBLP~and \CITH.
The $y$-axis is $\frac{r}{n} \times 100\%$. %, where $n=|V|$, and $r$ is the rank of lossless SVD for $\mathbf{C}$ in Eq.\eqref{eq:03}.
On each dataset, when increasing $|\Delta E|$ from 6K to 18K,
we see that $\frac{r}{n}$ is 95\% on \DBLP~(\Resp 80\% on \CITH),
Thus, $r$ is not negligibly smaller than $n$ in real graphs.
Due to the quartic time \wrt $r$,
\IncSVD~may be slow in practice to get a high accuracy.

On synthetic data, we fix $|V|=79,483$ and vary $|E|$ from 485K to 560K (\Resp 560K to 485K) in 15K increments (\Resp decrements).
The results are shown in Fig.\ref{fig:exp_05_06}. %,
We can see that,
%confirming our observations on real datasets.
%For example,
when 6.4\% edges are increased,
\IncSR~runs 8.4x faster than \IncSVD, 4.7x faster than \Batch, and 2.7x faster than \IncUSR.
When 8.8\% edges are deleted,
{\IncSR} outperforms {\IncSVD} by 10.4x, {\Batch} by 5.5x, and {\IncUSR} by 2.9x.
This justifies our complexity analysis of {\IncSR} and \IncUSR.
%
\subsubsection{Effectiveness of Pruning} %\vspace{-5pt}
%
Fig.\ref{fig:exp_07} shows the pruning power of {\IncSR} as compared with {\IncUSR} on \DBLP, \CITH, and \YOUTU,
in which
%
%As mentioned in Subsection \ref{sec:05},
%\IncSR~skips needless computations for incremental updates.
%To show the effectiveness of our pruning strategy in \IncSR,
%we compare its time with that of \IncUSR, \ie original version of \IncSR~without pruning rules, on \DBLP, \CITH, \YOUTU.
%The results are shown in Fig.\ref{fig:exp_07},
the percentage of the pruned node-pairs of each graph is depicted on the black bar.
The $y$-axis is in a log scale.
It can be discerned that, on every dataset,
\IncSR~constantly outperforms \IncUSR~by nearly 0.5 order of magnitude.
For instance, the running time of \IncSR~(64.9s) improves that of \IncUSR~(314.2s)~by 4.8x on \CITH,
with approximately 82.1\% node-pairs being pruned.
That is, our pruning strategy is powerful to discard unnecessary node-pairs on graphs with different link distributions.
\begin{figure*}
\centering
\begin{minipage}[b]{0.71\linewidth}
\centering
  \includegraphics[width=\linewidth]{exp_11_12_13} \\
  \caption{Time Efficiency on Real Data ($\Delta E$ accompanies new node insertions)}\label{fig:exp_11_12_13}
\end{minipage}
\begin{minipage}[b]{0.28\linewidth}
\centering
\scalebox{0.8}{$
  \begin{tabular}{c|r|rr|r}
\hline
\multicolumn{2}{c|}{Data ($|E|$)}    & \IncBSR  & \IncSRAll & (\%) \\ \hline
\parbox[t]{2mm}{\multirow{3}{*}{\rotatebox[origin=c]{90}{\DBLP}}}    & 75K  & 14.9    & 16.3     & 8.8  \\
                         & 83K  & 70.5    & 82.0     & 14.0 \\
                         & 91K  & 315.9   & 363.8    & 13.1 \\ \hline
\parbox[t]{2mm}{\multirow{3}{*}{\rotatebox[origin=c]{90}{\CITH}}}    & 395K & 50.5    & 54.5     & 7.3  \\
                         & 407K & 241.9   & 283.5    & 14.6 \\
                         & 419K & 1869.1  & 2357.4   & 20.7 \\ \hline
\parbox[t]{2mm}{\multirow{3}{*}{\rotatebox[origin=c]{90}{\YOUTU}}} & 889K & 876.6   & 921.9    & 4.9  \\
                         & 901K & 2756.8  & 3297.4   & 16.4 \\
                         & 913K & 10256.1 & 12109.2  & 15.3 \\ \hline
\end{tabular}$}
  \caption{Time for Batch Updates}\label{fig:exp_14}
\end{minipage}
\end{figure*}
%
%
%it is imperative to evaluate, on real graphs, that how large these ``affected areas'' are when links are evolved.
%The results are visualized in Fig.\ref{fig:exp_08},

Since our pruning strategy hinges mainly on the size of the ``affected areas'' of the SimRank update matrix,
Fig.\ref{fig:exp_08} illustrates the percentage of the ``affected areas'' of SimRank scores \wrt link updates $|\Delta E|$ on \DBLP, \CITH, and \YOUTU.
We find the following.
(1) When $|\Delta E|$ is varied from 6K to 18K on every real dataset,
the ``affected areas'' of SimRank scores are fairly small.
For instance, when $|\Delta E|=12$K,
the percentage of the ``affected areas'' is only 23.9\% on \DBLP, 27.5\% on \CITH, and 24.8\% on \YOUTU, respectively.
This highlights the effectiveness of our pruning method in real applications,
where a larger number of elements of the SimRank update matrix with zero scores can be discarded.
(2) For each dataset, the size of the ``affect areas'' mildly grows when $|\Delta E|$ is increased.
For example, on \YOUTU, the percentage of $|\AFF|$ increases from 19.0\% to 24.8\% when $|\Delta E|$ is changed from 6K to 12K.
This agrees with our time efficiency analysis,
where the speedup of {\IncSR} is more obvious for smaller $|\Delta E|$.
%tends to be far more apparent when the size of $|\Delta E|$ gets smaller.
\begin{figure*}[t]
\centering
\begin{minipage}[b]{0.77\linewidth}
\centering
\scalebox{0.8}{$
\begin{tabular}{c|ccc|c|ccc}
\hline
\multirow{4}{*}{Datasets} & \multicolumn{3}{c|}{\IncSRAll}                                                                                                                                                                         & \IncBSR                                                                                  & \multicolumn{3}{c}{\IncSVD} \\ \cline{2-8}
                          & \begin{tabular}[c]{@{}c@{}}No \\ Optimization\end{tabular} & \begin{tabular}[c]{@{}c@{}}Turn on \\ Pruning\end{tabular} & \begin{tabular}[c]{@{}c@{}}Turn on Column-\\ wise Partitioning\end{tabular} & \begin{tabular}[c]{@{}c@{}}Turn on Pruning\\ \& Column-wise\\ Partitioning\end{tabular} & $r=5$  & $r=15$  & $r=25$  \\ \hline
\DBLP                     & 722.5M                                                     & 163.1M                                                     & 1.3M                                                                        & 15.0M                                                                                   & 1.36G  & 1.97G   & 3.86G   \\
\CITH                     & 1.64G                                                      & 413.9M                                                     & 4.2M                                                                        & 34.8M                                                                                   & 4.83G  & ---     & ---     \\
\YOUTU                    & ---                                                        & ---                                                        & 12.7M                                                                       & 186.2M                                                                                  & ---    & ---     & ---     \\ \hline
\end{tabular}$}
%  \includegraphics[height=2.6cm]{exp_09.eps}
  \caption{Total Memory Efficiency on Real Data \ (``---'' means memory explosion)}\label{fig:exp_09}
\end{minipage}
\begin{minipage}[b]{0.22\linewidth}
\centering
  \includegraphics[height=2.6cm]{exp_15.eps}
  \caption{Exactness}\label{fig:exp_15}
\end{minipage}
\end{figure*}
%
\subsubsection{Time Efficiency of {\IncSRAll} and {\IncBSR}} %\vspace{-7pt}
%
We next compare the computational time of {\IncSRAll} with {\IncSVD} and {\Batch} on \DBLP, \CITH, and \YOUTU.
For each dataset, we increase $|E|$ by $|\Delta E|$ that might accompany new node insertions.
Note that {\IncSR} cannot deal with such incremental updates as $\mathbf{\Delta S}$ does not make any sense in such situations.
To enable {\IncSVD} to handle new node insertions, we view new inserted nodes as singleton nodes in the old graph $G$.
Fig.~\ref{fig:exp_11_12_13} depicts the results.
We can discern that
(1) on every dataset, {\IncSRAll} runs substantially faster than {\IncSVD} and {\Batch} when $|\Delta E|$ is small.
For example, as $|\Delta E|=6K$ on {\CITH}, {\IncSRAll} (186s) is 30.6x faster than {\IncSVD} (5692s) and 15.1x faster than {\Batch} (2809s).
The reason is that {\IncSRAll} can integrate the merits of {\IncSR} with {\IncUSRtwo, \IncUSRthree, \IncUSRfour}
to dynamically update SimRank scores in a rank-one style with no need to do costly matrix-matrix multiplications.
Moreover, the complete framework of {\IncSRAll} allows itself to support link updates that enables new node insertions.
(2) When $|\Delta E|$ grows larger on each dataset,
the time of {\IncSVD} increases significantly faster than {\IncSRAll}.
This larger increase is due to the SVD tensor products used by {\IncSVD}.
In contrast, {\IncSRAll} can effectively reuse the old SimRank scores to compute changes even if such changes may accompany new node insertions.

Fig.~\ref{fig:exp_14} compares the computational time of {\IncBSR} with {\IncSRAll}.
From the results, we can notice that, on each graph,
{\IncBSR} is consistently faster than {\IncSRAll}.
The last column ``(\%)'' denotes the percentage of {\IncBSR} improvement on speedup.
On each dataset,
the speedup of {\IncBSR} is more apparent when $|\Delta E|$ grows larger.
For example, on {\DBLP}, the improvement of {\IncBSR} over {\IncSRAll} is 8.8\% when $|E|=75$K, and 14.0\% when $|E|=83$K.
On {\CITH} (\Resp {\YOUTU}), the highest speedup of {\IncBSR} over {\IncSRAll} is 20.7\% for $|E|=419$K (\Resp 16.4\% for $|E|=901$K).
This is because the large size of $|\Delta E|$ may increase the number of the new inserted edges with one endpoint overlapped.
Hence, more edges can be handled simultaneously by {\IncBSR}, highlighting its high efficiency over {\IncSRAll}.
%
\subsubsection{Total Memory Usage}
%
%\noindent \textbf{Exp-3: Memory Space.}
Fig.~\ref{fig:exp_09} evaluates the total memory usage of {\IncSRAll} and {\IncBSR} against {\IncSVD} on real datasets.
Note that the total memory usage includes the storage of the old SimRanks required for all-pairs dynamical evaluation.
For {\IncSRAll}, we test its three versions:
(a) We first switch off our methods of ``pruning'' and ``column-wise partitioning'',
denoted as ``No Optimization''; (b) next turn on ``pruning'' only; and (c) finally turn on both.
For {\IncSVD}, we also tune the default target rank $r=5$ larger to see how the memory space is affected by $r$.

The results indicate that
(1) on each dataset when the memory of {\IncSVD} and {\IncBSR} does not explode,
the total spaces of {\IncSRAll} and {\IncBSR} are consistently much smaller {\IncSVD} whatever target rank $r$ is.
This is because, unlike {\IncSVD}, {\IncSRAll} and {\IncBSR} need not memorize the results of SVD tensor products.
(2) When the ``pruning'' switch is open, the space of {\IncSRAll} can be reduced by $\sim4$x further on real data,
due to our pruning method that discards many zeros in auxiliary vectors and final SimRanks.
(3) When the ``column-wise partitioning'' switch is open,
the space of {\IncSRAll} can be saved by $\sim100$x further.
The reason is that, as all pairs of SimRanks can be computed in a column-by-column style,
there is no need to memorize the entire old SimRank $\mathbf{S}$ and auxiliary $\mathbf{M}$.
This improvement agrees with our space analysis in Section~\ref{sec:07}.
(4) The space of {\IncBSR} is 8-11x larger than {\IncSRAll}, but is still acceptable.
This is because batch updates require more space to memoize several columns from the old $\mathbf{S}$ to handle a subset of edge updates simultaneously.
(5) For {\IncSVD}, when the target rank $r$ is varied from 5 to 25, its total space increases from 1.36G to 3.86G on \DBLP,
but crashes on \CITH~and \YOUTU.
This implies that $r$ has a huge impact on the space of \IncSVD,
and is not negligible in the big-$O$ analysis of \cite{Li2010}.
%
%Here, the memory space means ``\emph{intermediate} space'',
%where the last step of writing $n^2$ node-pairs of the similarity outputs is not accommodated.
%We also tune the default target rank $r=5$ larger for \IncSVD~to see how memory increases \wrt $r$.
%
%The results are depicted in Fig.\ref{fig:exp_09},
%where, for \IncSVD, we report $r=15, 25$ on only small \DBLP,
%as its memory space will explode on larger networks when $r$ and $|V|$ grow.
%We notice that
%(1) \IncSR~and \IncUSR~consume far smaller space than \IncSVD~by at least 1.5 orders of magnitude on \DBLP~and \CITH~no matter what target rank $r$ might be.
%This is because \IncSR~and \IncUSR~use the rank-one method to convert $\mathbf{\Delta S}$ computations into the sequence of \emph{vector} operations,
%whereas \IncSVD~needs to memoize the decomposed SVD matrices and to perform costly matrix tensor products.
%(2) \IncSR~has 4.1x (\Resp 4.5x) smaller space than \IncUSR~on \DBLP~(\Resp \YOUTU),
%due to our pruning method reducing the memoization of many entries in auxiliary vectors, \eg $\mathbf{w}$.
%(3) When $r$ is varied from 5 to 25,
%the space of \IncSVD~is increased from 637.9M to 3.15G on \DBLP,
%but crashes on \CITH~and \YOUTU.
%This tells that $r$ has a large impact on the performance of \IncSVD,
%which cannot be ignored in the big-$O$ notation of the complexity analysis \cite{Li2010}.
%Thus, to get \IncSVD~feasible on \CITH, we set $r=5$ in the evaluations.
\begin{figure*}[t]
\centering
\begin{minipage}[b]{0.48\linewidth}
\centering
\scalebox{0.8}{$
  \begin{tabular}{l|r|r|rr}
  \hline
\multicolumn{1}{c|}{\multirow{2}{*}{Datasets}} & \multicolumn{1}{c|}{\multirow{2}{*}{\IncSRAllP}} & \multicolumn{3}{c}{\LTSF}                                                                 \\ \cline{3-5}
\multicolumn{1}{c|}{}                          & \multicolumn{1}{c|}{}                         & \multicolumn{1}{c|}{Total} & \multicolumn{1}{c}{Index (Merge)} & \multicolumn{1}{c}{Query} \\ \hline
\WEBB                                         & 0.453                                        & 4.764                     & 4.758                             & 0.006                     \\
\WEBG                                         & 1.440                                        & 6.883                     & 6.876                             & 0.007                     \\
\CITP                                         & 3.820                                        & 20.549                    & 20.536                            & 0.013                     \\
\SOCL                                         & 35.393                                       & 67.372                    & 67.322                            & 0.050                     \\
\UK                                           & 63.125                                       & 460.718                   & 460.360                           & 0.358                     \\
\IT                                           & 69.301                                       & 505.794                   & 505.400                           & 0.393 \\ \hline
\end{tabular}$}
  \caption{Avg Time (secs) for $\mathbf{S}_{\star,u}$ per Edge Update}\label{fig:exp_17}
\end{minipage} \qquad
\begin{minipage}[b]{0.42\linewidth}
\centering
  \includegraphics[height=2.9cm]{exp_16.eps}
  \caption{Avg Time for Each Insertion Case}\label{fig:exp_16}
\end{minipage}
\end{figure*}
%
\subsubsection{Exactness}
%
%\noindent \textbf{Exp-4: Exactness.}
We next evaluate the exactness of {\IncSRAll}, {\IncBSR}, and {\IncSVD} on real datasets.
We leverage the NDCG metrics \cite{Li2010} to assess the top-100 most similar pairs. % on \DBLP, \CITH, \YOUTU.
We adopt the results of the batch algorithm \cite{Fujiwara2013} on each dataset as the $\textrm{NDCG}_{100}$ baselines, due to its exactness.
For {\IncSRAll}, we evaluate its two enhanced versions: ``with column-wise partitioning'' and ``with pruning'';
for {\IncSVD}, we tune its target rank $r$ from 5 to 25.

%For baselines of $\textrm{NDCG}_{100}$,
%we use the results of \cite{Fujiwara2013} on each dataset. % for 35 iterations.\footnote{As the diameters (\ie the longest paths) of \DBLP, \CITH, \YOUTU~are 16,11,7, respectively,
%it suffices to perform $K=35$ iterations to accommodate \emph{all} path-pairs between two nodes for assessing SimRank.
%Thus, the resulting scores of \Batch~for $K=35$ can be viewed as the \emph{exact} baseline solutions.}

Fig.~\ref{fig:exp_15} depicts the results, showing the following.
(1) On each dataset, the $\textrm{NDCG}_{100}$s of {\IncSRAll} and {\IncBSR} are 1, which are better than {\IncSVD} ($<0.62$).
This agrees with our observation that {\IncSVD} may loss eigen-information in real graphs.
In contrast, {\IncSRAll} and {\IncBSR} guarantee the exactness.
(2) The $\textrm{NDCG}_{100}$s for the two versions of {\IncSRAll} are exactly the same,
implying that both our pruning and column-wise partitioning methods are lossless while achieving high speedup.
%
%
%
%For \IncSR~and \IncUSR, we perform $K=5,15$ iterations~on each graph;
%for \IncSVD, due to its non-iterative paradigm, we tune the rank $r$ from 5 to 15.
%The results are depicted in Fig.\ref{fig:exp_10},
%telling us the following.
%(1) In all the cases,
%\IncSR~and \IncUSR~have much better accuracy than \IncSVD.
%For example, the $\textrm{NDCG}_{30}$ of \IncSR~and \IncUSR~are both 0.88 at $K=5$,
%much better than \IncSVD~(0.36) at $r=25$.
%This confirms our observations in Section \ref{sec:03b},
%where we envision that \IncSVD~may miss some eigen-information in many real graphs.
%When $K=10$, the $\textrm{NDCG}_{30}$ of \IncSR~and \IncUSR~are 1s,
%indicating that their top-30 node-pairs are perfectly accurate.
%This justifies the correctness of our algorithms.
%(2) For each dataset and the fixed iteration $K$,
%the $\textrm{NDCG}_{30}$ of \IncSR~and \IncUSR~are exactly the same.
%This indicates that our pruning strategy is lossless, \ie
%it does not sacrifice any exactness for speedup.
%
%
\subsubsection{Scalability on Large Graphs}
%
To evaluate the scalability of our incremental techniques,
we run {\IncSRAllP}, a memory-efficient version of {\IncSR}, on six real graphs ({\WEBB}, {\WEBG}, {\CITP}, {\SOCL}, {\UK}, and {\IT}),
and compare its performance with {\LTSF}.
Both {\IncSRAllP} and {\LTSF} can compute any single column, $\mathbf{S}_{\star, u}$, of $\mathbf{S}$
with no need to memoize all $n^2$ pairs of the old $\mathbf{S}$.
To choose the query node $u$, we randomly pick up 10,000 queries from each medium-sized graph ({\WEBB} and {\WEBG}),
and 1,000 queries from each large-sized graph ({\CITP}, {\SOCL}, {\UK}, and {\IT}).
To ensure every selected $u$ can cover a board range of any possible queries,
for each dataset, we first sort all nodes in $V$ in descending order based on their importance that is measured by PageRank (PR),
and then split all nodes into 10 buckets: nodes with $\textrm{PR} \in [0.9, 1]$ are in the first bucket;
nodes with $\textrm{PR} \in [0.8, 0.9)$ the second, etc.
For every medium- (\Resp large-) sized graph,
we randomly select 1,000 (\Resp 100) queries from each bucket,
such that $u$ contains a wide range of various types of queries.
To generate dynamical updates, we follow the settings in \cite{Shao2015},
randomly choosing 1,000 edges, and considering 80\% of them as insertions and 20\% deletions.

Fig.~\ref{fig:exp_17} compares the average time of {\IncSRAllP} and {\LTSF} required to compute any column $\mathbf{S}_{\star, u}$ \wrt a given query $u$ for each edge update on six real graphs.
It can be discerned that, on each dataset, {\IncSRAllP} is scalable well over large graphs, and runs consistently 4--7x faster than log-based {\LTSF} per edge update.
On one-billion edge graphs (\IT), for every edge update,
the updating time of {\IncSRAllP} (69.301s) is 7.3x faster than that of {\LTSF} (505.794s).
This is because the time of {\LTSF} is dominated by its cost of merging $R_g=100$ one-way graphs' log buffers for updating the index.
For example, on large {\IT}, almost 99.92\% time required by {\LTSF} is due to its merge operations.
In comparison, our memory-efficient method for {\IncSRAllP} takes advantage of the rank-one Sylvester equation which computes the updates to $\mathbf{S}_{\star, u}$ in a column-by-column style on demand,
without the need to merge one-way graphs and memoize all pairs of old $\mathbf{S}$ in advance.

Fig.~\ref{fig:exp_16} shows the time complexities of {\IncSRAllP} for four cases of edge insertions on each real dataset.
For every graph, we randomly select 1,000 edges $\{(i,j)\}$ for insertion updates,
with nodes $i$ and $j$ respectively having the probability ${1}/{2}$ to be picked up from the old vertex set $V$.
Hence, each case of edge insertion occurs at ${1}/{4}$ probability.
For each insertion case, we sum all the time spent in this case, and divide it by the total number of edge insertions counted for this case.
Fig.~\ref{fig:exp_16} reports the average time per edge update for each case, together with the preprocessing time over each dataset (including the cost of loading the graph and preparing its transition matrix $\mathbf{Q}$).
From the results, we see that, on each dataset, the time spent for Cases (C0) and (C2) is moderately higher than that for Case (C1);
Case (C0) is slightly slower than Case (C2); Case (C3) entails the lowest time cost.
These results are consistent with our intuition and mathematical formulation of $\mathbf{\Delta S}$ for each case.
Case (C0) has the most expensive time cost as it needs to iteratively prepare vectors $\bm\xi_k$ and $\bm\eta_k$, and old similarities $\mathbf{S}_{\star,i}$ and $\mathbf{S}_{\star,j}$ via matrix-vector products.
In contrast, Case (C2) only requires to iteratively prepare $\bm\xi_k, \bm\eta_k$ and $\mathbf{S}_{\star,i}$;
Case (C1) just requires to perform one matrix-vector product to prepare one vector $\mathbf{y}$.
For Case (C3), the new inserted edge forms a new component of the graph.
There is no precomputation of any auxiliary vectors, and thus Case (C3) is the fastest.
\begin{figure*}[t]
\centering
\begin{minipage}[b]{0.25\linewidth}
\centering
  \includegraphics[height=2.9cm]{exp_18.eps}
  \caption{Precision on {\YOUTU}}\label{fig:exp_18}
\end{minipage}
\begin{minipage}[b]{0.38\linewidth}
\centering
  \includegraphics[height=2.9cm]{exp_19.eps}
  \caption{Memory of {\IncSRAllP} \& {\LTSF}}\label{fig:exp_19}
\end{minipage}
\begin{minipage}[b]{0.35\linewidth}
\centering
  \includegraphics[height=2.9cm]{exp_20.eps}
  \caption{Memory for Each Insertion Case}\label{fig:exp_20}
\end{minipage}
\end{figure*}
%
\subsubsection{Precision}
%
To compare the precision of {\IncSRAllP} and {\LTSF},
we define the \emph{precision} measure  \cite{Jiang2017} for top-$k$ querying:
\[
\textrm{Precision}= \frac{|\textrm{approximate top-$k$ set} \cap \textrm{exact top-$k$ set}|}{k}
\]
The original batch algorithm in \cite{Jeh2002} (\Resp \cite{Li2010}) serves as the exact solution to obtain SimRank results for {\LTSF} (\Resp {\IncSRAllP}).
We evaluate the precision of both algorithms on several real datasets.
Fig.~\ref{fig:exp_18} reports the results on {\YOUTU};
the tendencies on other datasets are similar.
We see that, when top-$k$ varies from 10 to $10^5$,
the precision of {\LTSF} remains high $(>84\%)$ for small top-$k$ $(<1000)$,
but is lower (68\%--75\%) for large top-$k$ $(>{10}^4)$.
This is because the probabilistic guarantee for the error bound of {\LTSF} is based on the assumption that no cycle in the given graph has a length shorter than $K$ (the total number of steps).
Hence, {\LTSF} is highly efficient for top-$k$ single-source querying, where $k$ is not large.
In contrast, the precision of {\IncSRAllP} is stable at 1,
meaning that it produces the exact SimRank results of \cite{Li2010}, regardless of its top-$k$ size.
Thus, {\IncSRAllP} is better for non top-$k$ query.
%
\subsubsection{Memory of {\IncSRAllP}}
%
Fig.~\ref{fig:exp_19} evaluates the memory usage of {\IncSRAllP} and {\LTSF} over six real datasets.
We observe that both algorithms scale well on large graphs.
On {\WEBB}, {\IT}, and {\UK}, the memory space of {\IncSRAllP} is almost the same as {\LTSF};
On {\WEBG}, {\CITP}, and {\SOCL}, the memory usage of {\IncSRAllP} is 5--8x less than {\LTSF}.
This is because, unlike {\LTSF} that need load a one-way graph to memory,
{\IncSRAllP} only requires to prepare the vector information of $\bm\xi_k, \bm\eta_k$, old $\mathbf{S}_{\star,i}$, and old $\mathbf{S}_{\star,j}$ to assess the changes to each column of $\mathbf{S}$ in response to edge update $(i,j)$.
The memory space of these auxiliary vectors can sometimes be comparable to the size of the one-way graph, and sometimes be much smaller.
However, such memory space is linear to $n$ as we do not need $n^2$ space to store the entire old $\mathbf{S}$.
Note that the old $\mathbf{S}_{\star,j}$ and $\mathbf{S}_{\star,i}$ can be computed on demand with only linear memory by our partial-pairs SimRank approach \cite{Yu2015}.
Moreover, we see that, with the growing scale of the real datasets,
the memory space of {\IncSRAllP} is increasing linearly, highlighting its scalability on large graphs.

Fig.~\ref{fig:exp_20} depicts further the average memory usage of {\IncSRAllP} for each case of edge insertion.
We randomly pick up 1,000 edges $\{(i,j)\}$ for insertion updates on each dataset,
with nodes $i$ and $j$ respectively having the probability ${1}/{2}$ to be chosen from the old vertex set $V$.
The average memory space of {\IncSRAllP} for each case is reported in Fig.~\ref{fig:exp_20}.
We see that, on each dataset, the memory required for Cases (C0), (C1), and (C2) are similar,
whereas the memory space of Case (C3) is much smaller than the other cases.
The reason is that, for Cases (C0), (C1), and (C2),
{\IncSRAllP} needs linear memory to store some auxiliary vectors
 (\eg $\bm\xi_k, \bm\eta_k$, $\mathbf{y}$, old $\mathbf{S}_{\star,i}$, and old $\mathbf{S}_{\star,j}$) for updating SimRank scores,
whereas for Case (C3), no auxiliary vectors are required for precomputation, thus saving much memory space.
%
\section{Related Work} \label{sec:02}
%
%Assessing node-pair similarity based on network structures is an important task in hyperlink analysis.
%SimRank is arguably one of the most appealing graph-based node-to-node similarity measures. % in a graph.
%which was invented by Jeh and Widom \cite{Jeh2002}.
%Due to the broad range of applications
%(\eg link prediction, web page ranking),
%the last decade has witnessed a growing interest in SimRank.
%Particularly,
%there has been significant attention to efficiently computing SimRank
%since the naive method~\cite{Jeh2002} of computing SimRank requires $O(Kd^2n^2)$ time for $K$ iterations,
%where $d$ is the average in-degree of the graph.
%
        Recent results on SimRank can be distinguished into two categories:
        (i) dynamical SimRank \cite{He2010,Li2010,Yu2014,Shao2015,Jiang2017}, and
        (ii) static SimRank \cite{Kusumoto2014,Fujiwara2013,Li2010a,Li2015,Fogaras2007,Lee2012,Lizorkin2008,Yu2013}.

\subsection{Incremental SimRank}

%Incremental computation is useful since real graphs are typically updated with small changes.
%However, only a paucity of work is known about incremental SimRank update \cite{He2010,Li2010,Shao2015},
%far less than their batch counterparts. % (\eg \cite{Yu2014, Antonellis2008,Fogaras2007,Lee2012,Li2010a,Lizorkin2008,Yu2013,Fujiwara2013}).
%Generally,
%there are two types of dynamical algorithms:
%(i) deterministic method \cite{He2010,Li2010}, and (ii) probabilistic method \cite{Shao2015}.
%Regarding deterministic approaches,
%the pioneering work of
%
Li \etal \cite{Li2010} devised an interesting matrix representation of SimRank,
%(i) link updates, and (ii) node updates.
%About link-incremental SimRank algorithms,
%we are merely aware of \cite{Li2010} by Li \etal who gave an excellent matrix representation of SimRank,
and was the first to show a SVD method for incrementally updating all-pairs SimRanks, which requires $O(r^4n^2)$ time
%\footnote{According to the \emph{proof} of Lemma 2 in \cite{Li2010}, the time is actually $O(r^4n^2)$,
%though, the statement of Lemma 2 says ``it is bounded by $O(n^2)$''.
%%which is claimed to be bounded by $O(n^2)$.
%% time only when $r$ is \emph{much} smaller than $n$.
%Observing that $r \ll n$ is not often the case,
%we do not explicitly omit $r^4$ in $O(\star)$ here.}
and $O({r}^{2}n^2)$ memory.
%where $r \ (\le n)$ is the target rank of the low-rank approximation.
However, their incremental techniques are \emph{inherently} inexact, with no guaranteed accuracy.
%It may miss some eigen-information (as we explained in Section~\ref{sec:03b})
% and verified by our experimental study in Fig.\ref{fig:exp_04})
%%        even though $r$ is chosen to be exactly the full rank (instead of low rank) of the target matrix for the lossless SVD.
%%        (ii) In practice, $r$ seems not much smaller than $n$ for attaining a desired accuracy,
%%        but this may lead to prohibitively expensive updating costs for \cite{Li2010} because its time complexity $O(r^4n^2)$ is quartic \wrt $r$.
%%        In comparison,
%%        our work adopts a completely different framework from~\cite{Li2010}.
%%        Instead of incrementally updating SVD,
%%        we first describe the changes to SimRank in response to every link update as a rank-one Sylvester equation,
%%        and then use graph topology to discard ``unaffected areas'' for speeding up the incremental computation of SimRank,
%%        without a compromise in accuracy.
%%        Our methods yield only linear time and memory \wrt $n$ (independent of $r$) to incrementally compute all pairs of SimRanks for every link update.
%%        Moreover, for some types of link updates, \eg the insertion of edge $(i,j)$ with $i$ or $j$ being a new node,
%%        the existing method by Li \etal \cite{Li2010} does not work effectively
%%        since their computational framework tacitly implies an assumption that new and old SimRank matrices must retain the same size.
%%        In contrast, our solution in this work can efficiently deal with such types of link updates,
%%        allowing new node insertions and deletions.


%plus $O(\tilde{m})$ precomputation time,
%where $\tilde{m}$ is the number of links in the updated graph.

%%        Another interesting piece of work is due to He \etal \cite{He2010},
%%        who proposed the parallel computation of SimRank on digraphs,
%%        by leveraging an iterative aggregation method.
%%        Indeed, the parallel computing technique in \cite{He2010} can be regarded as an efficient way to dynamically update new SimRank blocks.
%%        It differs from our work in that \cite{He2010} is based on GPU to improve the parallel efficiency by reordering and splitting the first-order Markov chain,
%%        whereas our methods eliminate unnecessary recomputations in ``unaffected areas'' in terms of graph updates on CPU via a rank-one Sylvestor equation.
%instead of capturing the ``unaffected areas'' of SimRank \wrt link updates,
%whereas our methods utilize pruning rules to eliminate unnecessary recomputations for links updates on CPU via a rank-one Sylvestor equation.

{Recently, Shao \etal \cite{Shao2015} provided an excellent exposition of a two-stage random sampling framework, TSF, for top-$k$ SimRank dynamic search \wrt one query $u$.
In the preprocessing stage, they sampled a collection of one-way graphs to index random walks in a scalable manner.
In the query stage, they retrieved similar nodes by pruning unqualified nodes based on the connectivity of one-way graph.
To retrieve \emph{top-$k$ nodes} with highest SimRank scores in \emph{a single column} $\mathbf{S}_{\star,u}$,
\cite{Shao2015} requires $O(K^2 R_q R_g)$ \emph{average} query time to retrieve $\mathbf{S}_{\star,u}$ along with $O(n \log k)$ time to return top-$k$ results from $\mathbf{S}_{\star,u}$.
The recent work of Jiang \etal \cite{Jiang2017} has argued that, to retrieve $\mathbf{S}_{\star,u}$, the querying time of \cite{Shao2015} is $O(K n R_q R_g)$.
The $n$ factor is due to the time to traverse the reversed one-way graph; in the worst case, all $n$ nodes are visited.
Moreover, Jiang \etal \cite{Jiang2017} observed that the probabilistic error guarantee of Shao \etal\!\!'s method is based on the assumption that no cycle in the given graph has a length shorter than $K$,
and they proposed READS, a new efficient indexing scheme that improves precision and indexing space for dynamic SimRank search.
The query time of READS is $O(r n)$ to retrieve one column $\mathbf{S}_{\star,u}$, where $r$ is the number of sets of random walks.
Hence, TSF and READS are highly efficient for \emph{top-$k$ single-source} SimRank search.
In comparison, our dynamical method focuses on \emph{all $(n^2)$-pairs} SimRank search in $O(K(m+|\AFF|))$ time.
Optimization methods in this work are based on a rank-one Sylvester matrix equation characterising changes to $n^2$ pairs of SimRank scores,
which is fundamentally different from \cite{Shao2015,Jiang2017}'s methods that maintain one-way graphs (or SA forests) updating.
It is important to note that, for large-scale graphs, our incremental methods do not need to memoize all $(n^2)$ pairs of old SimRank scores,
and can dynamically update $\mathbf{S}$ column-wisely in only $O(Kn+m)$ memory.
For updating each column of $\mathbf{S}$,
our experiments in Section~\ref{sec:09} verify that
our memory-efficient incremental method is scalable on large real graphs while running 4--7 times faster than the dynamical TSF \cite{Shao2015} per edge update,
due to the high cost of \cite{Shao2015} merging one-way graph's log buffers for TSF indexing.
}

There has also been a body of work on incremental computation of other graph-based relevance measures. % \cite{Desikan2005,Bahmani2010,Sarma2011,Fujiwara2012}.
Banhmani \etal \cite{Bahmani2010} utilized the Monte Carlo method for incrementally computing Personalized PageRank.
Desikan \etal \cite{Desikan2005} proposed an excellent incremental PageRank algorithm for node updating.
Their central idea revolves around the first-order Markov chain.
%Yu \etal \cite{Yu2012a} provided an incremental algorithm for SimFusion+ update, by representing the changes to an eigenvector as a few original eigenvectors.
Sarma \etal \cite{Sarma2011} presented an excellent exposition of randomly sampling random walks of short length, and merging them together to estimate PageRank on graph streams.
%All these incremental methods are designed only for a specific measure, and may not be applied to SimRank.
%
%\emph{probabilistic} in nature, with the focus on node ranking,
%and hence cannot be directly applied in SimRank node-pair scoring.
%Fujiwara \etal \cite{Fujiwara2012} proposed \textsf{K-dash} for finding top-$k$ highest Random Walk with Restart (RWR) proximity nodes for a given query,
%which involves a strategy to incrementally \emph{estimate} proximity bounds.
%However, their incremental process is \emph{approximate}.
%Later, Yu \etal \cite{Yu2013a} proposed an incremental strategy for RWR link updates.
%
\subsection{Batch SimRank}
%
        In comparison to incremental algorithms,
        the batch SimRank computation has been well-studied on static graphs.
%Recent results on batch SimRank can be categorized into
%(i) deterministic computation \cite{Jeh2002,Fujiwara2013,Li2010,Lizorkin2008,Yu2013}, and
%(ii) probabilistic estimation \cite{Fogaras2007,Lee2012,Li2010a}.
%The deterministic methods may obtain similarities of high accuracy,
%but the time complexity is less desirable than the probabilistic approaches.

For deterministic methods,
Jeh and Widom \cite{Jeh2002} were the first to propose an iterative paradigm for SimRank,
entailing $O(Kd^2n^2)$ time for $K$ iterations,
where $d$ is the average in-degree.
Later,
Lizorkin \etal \cite{Lizorkin2008} utilized the partial sums memoization to speed up SimRank computation to $O(Kdn^2)$.
%Li \etal \cite{Li2010} proposed a novel non-iterative matrix formula for SimRank.
%Apart from the incremental SimRank requiring $O(r^4n^2)$ time,
%they also used a SVD method for computing batch SimRank in $O(\alpha^4n^2)$ time,
%where $\alpha$ is the target rank of matrix $\mathbf{Q}$.
Yu \etal \cite{Yu2013} have also improved SimRank computation to $O(Kd'n^2)$ time (with $d' \le d$)
via a fine-grained memoization to share the common parts among different partial sums.
Fujiwara \etal \cite{Fujiwara2013} exploited the matrix form of \cite{Li2010} to find the top-$k$ similar nodes in $O(n)$ time \wrt a given query node.
All these methods require $O(n^2)$ memory to output all pairs of SimRanks.
Recently, Kusumoto \etal \cite{Kusumoto2014} proposed a linearized method that requires only $O(dn)$ memory and $O(K^2 d n^2)$ time to compute all pairs of SimRanks.
%However, the computational time of \cite{Kusumoto2014} for all pairs of SimRanks is increased to $O(K^2 d n^2)$ time.
The recent work of \cite{Yu2015} proposes an efficient aggregate method for computing partial pairs of SimRank scores.
The main ideas of partial-pairs SimRank search are also incorporated into the incremental model of our work,
achieving linear memory to update $n^2$-pairs similarities.

For parallel SimRank computing,
Li \etal~\cite{Li2015} proposed a highly parallelizable algorithm, called CloudWalker, for large-scale SimRank search on Spark with ten machines.
Their method consists of offline and online phases.
For offline processing, an indexing vector is derived by solving a linear system in parallel.
For online querying, similarities are computed instantly from the index vector.
Throughout, the Monte Carlo method is used to maximally reduce time and space.

The recent work of Zhang \etal \cite{Zhang2017} conducted extensive experiments and discussed in depth many existing SimRank algorithms in a unified environment using different metrics,
encompassing efficiency, effectiveness, robustness, and scalability.
The empirical study for 10 algorithms from 2002 to 2015 shows that, despite many recent research efforts,
the running time and precision of known algorithms have still much space for improvement.
This work makes a further step towards this goal.


%For other probabilistic methods,
Fogaras and R{\'a}cz \cite{Fogaras2007} proposed P-SimRank in linear time to estimate a single-pair SimRank $s(a,b)$ from the probability that two random surfers, starting from $a$ and $b$, will finally meet at a node.
%In \cite{Fogaras2007}, the lower bounds are further analyzed for the accuracy of P-SimRank on large graphs.
Li \etal \cite{Li2010a} harnessed the random walks to compute local SimRank for a single node-pair.
Later, Lee \etal \cite{Lee2012} employed the Monte Carlo method to find top-$k$ SimRank node-pairs.
Tao \etal \cite{Tao2014} proposed an excellent two-stage way for the top-$k$ SimRank-based similarity join.

Recently, Tian and Xiao \cite{Tian2016} proposed SLING, an efficient index structure for static SimRank computation.
SLING requires $O(n/\epsilon)$ space and $O(m/\epsilon+n\log \tfrac{n}{\delta} /\epsilon)$ pre-computation time, and answers any single-pair (\textit{resp.} single-source) query in $O(1/\epsilon)$ (\Resp $O(n/\epsilon)$) time.

%They first identified top-$k$ similar pairs of vectors with the largest dot product, which constitutes a set of candidate nodes,
%and then utilized a tree-based method to efficiently identify answers based on these candidate nodes, with guaranteed accuracy.
%
%However, it needs $O(Kd^2n^2)$ time for $K$ iterations to compute all node-pair similarities.
%To further speed up the iterations of SimRank,
%they devised a pruning heuristic to set the score between far-apart nodes to be zero.
%Later, Lizorkin \etal \cite{Lizorkin2008} observed that such a pruning heuristic may produce huge errors for a certain kind of graphs,
%in which all the similarities are relatively small (\eg scale-free graphs).
%Thus, they provided the accuracy guarantee for SimRank iterations,
%and more importantly,
%they proposed partial sums memorization to substantially speed up the computations of SimRank to $O(Kdn^2)$.
%
\section{Conclusions} \label{sec:11} %\label{sec:08}
%
In this article,
we study the problem of incrementally updating SimRank scores on time-varying graphs.
Our complete scheme, {\IncSRAll}, consists of five ingredients:
(1) For edge updates that do not accompany new node insertions,
we characterize the SimRank update matrix $\mathbf{\Delta S}$ via a rank-one Sylvester equation.
Based on this, a novel efficient algorithm is devised,
which reduces the incremental computation of SimRank from $O(r^4n^2)$ to $O(Kn^2)$ for each link update.
(2) To eliminate unnecessary SimRank updates further,
we also devise an effective pruning strategy that can improve the incremental computation of SimRank to $O(K(m+|\AFF|))$,
where $|\AFF|\ (\ll n^2)$ is the size of the ``affected areas'' in the SimRank update matrix.
(3) For edge updates that accompany new node insertions,
we consider three insertion cases, according to which end of the inserted edge is a new node.
For each case, we devise an efficient incremental SimRank algorithm that can support new node insertions and accurately update the affected similarities.
(4) For batch updates, we also propose efficient batch incremental methods that can handle ``similar sink edges'' simultaneously and eliminate redundant edge updates.
(5) To optimize the memory for all-pairs SimRank updates,
we also devise a column-wise memory-efficient technique that significantly reduces the storage from $O(n^2)$ to $O(Kn+m)$, without the need to memoize $n^2$ pairs of SimRank scores.
Our experimental evaluations on real and synthetic datasets demonstrate that
(a) our incremental scheme is consistently 5--10 times faster than Li \etal\!\!'s SVD based method;
(b) our pruning strategy can speed up the incremental SimRank further by 3--6 times;
(c) the batch update algorithm enables an extra 5--15\% speedup, with just a little compromise in memory;
(d) our memory-efficient incremental method is scalable on billion-edge graphs;
for every edge update, its computational time can be 4--7 times faster than {\LTSF} and its memory space can be 5--8 times less than {\LTSF};
(e) for different cases of edge updates, Cases (C0) and (C2) entail more time than Case (C1), and Case (C3) runs the fastest.
%
\bibliographystyle{abbrv} %IEEEtran}
\bibliography{ref}

%
\begin{appendices}
%
\section{Limitation of Li \etal\!\!'s SVD \cite{Li2010}} \label{app:01}
%
%\section{A Fly in the Ointment in \cite{Li2010}} \label{sec:03b}
%
We rigorously explain the reason why Li \etal\!\!'s incremental method may miss some eigen-information even if a lossless SVD is utilized for SimRank computation.

%In this section, we provide theoretical analysis to show that Li \etal\!\!'s incremental approach \cite{Li2010} is \emph{approximate} in nature,
%which might miss some eigen-information even if the lossless SVD is utilized for computing SimRank.

Let us first revisit the main idea of Li \etal\!\!'s incremental method~\cite{Li2010}.
Briefly, \cite{Li2010} characterizes SimRank matrix $\mathbf{S}$ in Eq.\eqref{eq:03a} in terms of three matrices $\mathbf{U},\mathbf{\Sigma},{\mathbf{V}}$,
%
%
%The existing incremental method \cite{Li2010} computes SimRank
%by expressing similarity matrix $\mathbf{S}$ in terms of matrices $\mathbf{U},\mathbf{\Sigma},{\mathbf{V}}$,
where $\mathbf{U}, \mathbf{\Sigma},{\mathbf{V}}$ are derived by the SVD of $\mathbf{Q}$, \ie
\begin{equation} \label{eq:01}
  \mathbf{Q}=\mathbf{U} \cdot \mathbf{\Sigma} \cdot {\mathbf{V}}^T.
\end{equation}
Then, when links are changed,
\cite{Li2010} incrementally computes the new SimRank matrix $\tilde{\mathbf{S}}$
by updating the old matrices $\mathbf{U}, \mathbf{\Sigma}, {\mathbf{V}}$ respectively as
\begin{equation} \label{eq:02}
\tilde{\mathbf{U}}=\mathbf{U}\cdot \mathbf{U_C}, \quad
\tilde{\mathbf{\Sigma}}=\mathbf{\Sigma_C}, \quad
\tilde{{\mathbf{V}}}= {\mathbf{V}} \cdot {\mathbf{V_C}},
\footnote{In the sequel, we abuse a tilde to denote the updated version of a matrix,
\eg $\tilde{\mathbf{U}}$ is the updated matrix of old $\mathbf{U}$ after link updates.}
\end{equation}
where $\mathbf{U_C},\mathbf{\Sigma_C},{\mathbf{V_C}}$ are derived from the SVD of the auxiliary matrix $\mathbf{C} \triangleq \mathbf{\Sigma} + \mathbf{U}^T \cdot \mathbf{\Delta Q} \cdot \mathbf{V}$, \ie
\begin{equation} \label{eq:03}
\mathbf{C}= \mathbf{U_C} \cdot \mathbf{\Sigma_C} \cdot{\mathbf{V_C}}^T,
\end{equation}
and $\mathbf{\Delta Q}$ is the changes to $\mathbf{Q}$ in response to link updates.

However,
%in the above computing process,
%we notice that using Eq.\eqref{eq:02} to update the old $\mathbf{U}, \mathbf{\Sigma}, {\mathbf{V}}$ may miss some eigen-information.
the main problem is that
the derivation of Eq.\eqref{eq:02} rests on the assumption that
\begin{equation} \label{eq:08}
\mathbf{U} \cdot \mathbf{U}^T = \mathbf{V} \cdot \mathbf{V}^T = \mathbf{I}_n.
\end{equation}
Unfortunately, Eq.\eqref{eq:08} does \emph{not} hold (unless $\mathbf{Q}$ is a full-rank matrix, \ie $\textrm{rank}(\mathbf{Q})=n$)
because in the case of $\textrm{rank}(\mathbf{Q})<n$,
even a ``perfect'' (lossless) SVD of $\mathbf{Q}$ via Eq.\eqref{eq:01} would produce $n \times \alpha$ \emph{rectangular} matrices $\mathbf{U}$ and $\mathbf{V}$ with $\alpha=\textrm{rank}(\mathbf{Q})<n$.
Thus,
\[\textrm{rank}(\mathbf{U} \cdot \mathbf{U}^T)=\alpha<n=\textrm{rank}(\mathbf{I}_n),\]
which implies that $\mathbf{U} \cdot \mathbf{U}^T \neq \mathbf{I}_n$.
Similarly, $\mathbf{V} \cdot \mathbf{V}^T \neq \mathbf{I}_n$ when $\textrm{rank}(\mathbf{Q})<n$.
Hence, Eq.\eqref{eq:08} is not always true,
as visualized in Fig.~\ref{fig:05}.
\begin{example} \label{eg:02}
Consider a graph with the matrix $\mathbf{Q}=\left[\begin{smallmatrix}0 & 1 \\0 & 0 \end{smallmatrix}\right]$,
and its lossless SVD:
\[\mathbf{Q}=\mathbf{U} \cdot \mathbf{\Sigma} \cdot {\mathbf{V}}^T \ \textrm{with} \ \mathbf{U}=\left[\begin{smallmatrix}1 \\0 \end{smallmatrix}\right], \ \mathbf{\Sigma}=[1], \ {\mathbf{V}}=\left[\begin{smallmatrix}0 \\1 \end{smallmatrix}\right].\]
One can readily verify that
\[
\mathbf{U} \cdot \mathbf{U}^T = \left[\begin{smallmatrix}1 \\0 \end{smallmatrix}\right] \cdot [\begin{smallmatrix}1 & 0 \end{smallmatrix}] = \left[\begin{smallmatrix}1 & 0\\0 & 0 \end{smallmatrix}\right] \neq \left[\begin{smallmatrix}1 & 0\\0 & 1 \end{smallmatrix}\right] = \mathbf{I}_n
\quad (n=2),
\]
whereas
\[
\mathbf{U}^T \cdot \mathbf{U} = [\begin{smallmatrix}1 & 0 \end{smallmatrix}] \cdot  \left[\begin{smallmatrix}1 \\0 \end{smallmatrix}\right]  = 1 = \mathbf{I}_\alpha
\footnote{The notation $\mathbf{I}_{\alpha}$ denotes the $\alpha \times \alpha$ identity matrix.}
\quad  (\alpha=\textrm{rank}(\mathbf{Q})=1).
\]
Thus, Eq.\eqref{eq:08} does not hold when $\mathbf{Q}$ is not full-rank.
%as opposed to the following identity that is always true:
%\[\mathbf{U}^T \cdot \mathbf{U} = \mathbf{V}^T \cdot \mathbf{V} = \mathbf{I}_\alpha. \]
%This is because the SVD factorization only guarantees that $\mathbf{U}$ and $\mathbf{V}$ are \emph{column}-orthonormal (instead of \emph{row}-orthonormal) matrices,
%\ie every two column-vectors, $\mathbf{x}_i$ and $\mathbf{x}_j$ of $\mathbf{U}$ (\Resp $\mathbf{V}$) satisfy ${\mathbf{x}_i}^T \cdot \mathbf{x}_j= \left\{
%                                                                                                                   \begin{smallmatrix}
%                                                                                                                     1, & {i=j;} \\
%                                                                                                                     0, & {i\neq j.}
%                                                                                                                   \end{smallmatrix}
%                                                                                                                 \right.
%$
\qed
\end{example}
%
To clarify why Eq.\eqref{eq:08} gets involved in the derivation of Eq.\eqref{eq:02},
let us briefly recall from \cite{Li2010} the four steps of obtaining Eq.\eqref{eq:02},
and the problem lies in the last step.
\begin{figure}[t] \centering
  \includegraphics[width=\linewidth]{e05.eps}
  \caption{$\mathbf{U}\cdot \mathbf{U}^T \neq \mathbf{I}_n$ whenever $\textrm{rank}(\mathbf{Q})=r<n$} \label{fig:05} %\vspace{-10pt}
\end{figure}

\textsc{Step 1}. \
Initially, when links are changed,
the old $\mathbf{Q}$ is updated to new $\tilde{\mathbf{Q}}  = \mathbf{Q} + \mathbf{\Delta Q}$.
By replacing $\mathbf{Q}$ with Eq.\eqref{eq:01}, it follows that
\begin{equation} \label{eq:05}
\tilde{\mathbf{Q}} = \mathbf{U} \cdot \mathbf{\Sigma} \cdot {\mathbf{V}}^T + \mathbf{\Delta Q}.
\end{equation}

\textsc{Step 2}. \
Premultiply by $\mathbf{U}^T$ and postmultiply by $\mathbf{V}$ on both sides of Eq.\eqref{eq:05},
and then apply the property $\mathbf{U}^T \cdot \mathbf{U} = \mathbf{V}^T \cdot \mathbf{V} = \mathbf{I}_{\alpha}$.
%\footnote{As remarked in Example \ref{eg:02},
%since $\mathbf{U} \in \mathbb{R}^{n \times \alpha}$ is a \emph{column}-orthonormal (rather than \emph{row}-orthonormal) matrix,
%it follows that $\mathbf{U}^T \cdot \mathbf{U} = \mathbf{I}_{\alpha}$,  whereas $\mathbf{U} \cdot \mathbf{U}^T \neq \mathbf{I}_n$.}
It follows that
\begin{equation} \label{eq:06}
\mathbf{U}^T \cdot \tilde{\mathbf{Q}} \cdot {\mathbf{V}}=  \mathbf{\Sigma} +\mathbf{U}^T \cdot \mathbf{\Delta Q} \cdot {\mathbf{V}}.
\end{equation}

\textsc{Step 3}. \
Let $\mathbf{C}$ be the right-hand side of Eq.\eqref{eq:06}. Applying Eq.\eqref{eq:03} to Eq.\eqref{eq:06} yields
\begin{equation} \label{eq:07}
\mathbf{U}^T \cdot \tilde{\mathbf{Q}} \cdot {\mathbf{V}}= \mathbf{U_C} \cdot \mathbf{\Sigma_C} \cdot{\mathbf{V_C}}^T.
\end{equation}

\textsc{Step 4}. \
Li \etal \cite{Li2010} attempted to premultiply by $\mathbf{U}$ and postmultiply by $\mathbf{V}^T$ on both sides of Eq.\eqref{eq:07} first,
and then rested on the assumption of Eq.\eqref{eq:08} to obtain
\begin{equation} \label{eq:09}
\underbrace{\mathbf{U} \cdot \mathbf{U}^T}_{\hspace{10pt}\scalebox{1.2}{$? \hspace{-10pt}=\mathbf{I}_n$}} \cdot \tilde{\mathbf{Q}} \cdot \underbrace{\mathbf{V} \cdot \mathbf{V}^T}_{\hspace{10pt}\scalebox{1.2}{$? \hspace{-10pt}=\mathbf{I}_n$}}  =\underbrace{(\mathbf{U} \cdot  \mathbf{U_C})}_{\triangleq \tilde{\mathbf{U}}} \cdot \underbrace{\mathbf{\Sigma_C}}_{\triangleq \tilde{\mathbf{\Sigma}}} \cdot \underbrace{({\mathbf{V_C}} \cdot {\mathbf{V}})^T}_{\triangleq {\tilde{\mathbf{V}}}^T},
\end{equation}
which is the result of Eq.\eqref{eq:02}.

However, the problem lies in \textsc{Step} 4.
As mentioned before, Eq.\eqref{eq:08} does not hold when $\textrm{rank}(\mathbf{Q})<n$,
which means that $\tilde{\mathbf{Q}} \neq \tilde{\mathbf{U}} \cdot \tilde{\mathbf{\Sigma}} \cdot \tilde{\mathbf{V}}^T$ in Eq.\eqref{eq:09}.
Consequently, updating the old ${\mathbf{U}},{\mathbf{\Sigma}},{\mathbf{V}}$ via Eq.\eqref{eq:02}
may produce an error (up to $\|\mathbf{I}_n - \mathbf{U} \cdot \mathbf{U}^T  \|_2=1$, which is not practically small) for incrementally ``approximating'' ${\mathbf{S}}$.

%as illustrated in the following example.
\begin{example} \label{eg:03}
Recall the old $\mathbf{Q}$ and its SVD in Example \ref{eg:02}.
%Consider the tiny digraph $G$ consisting of two nodes $a$ and $b$, and an edge $(a,b)$.
Suppose there is a new edge insertion,
associated with $\mathbf{\Delta Q}=\left[\begin{smallmatrix}0 & 0 \\ 1 & 0 \end{smallmatrix}\right]$.
\cite{Li2010} first computes auxiliary matrix $\mathbf{C}$ as
\setlength\arraycolsep{1pt}
\[
\mathbf{C} \triangleq  \mathbf{\Sigma} + \mathbf{U}^T \cdot \mathbf{\Delta Q} \cdot \mathbf{V} = [1] + \left[\begin{smallmatrix}1 & 0 \end{smallmatrix}\right] \cdot \left[\begin{smallmatrix}0 & 0 \\ 1 & 0 \end{smallmatrix}\right] \cdot \left[\begin{smallmatrix}0 \\1 \end{smallmatrix}\right] =[1].
\]
Then, the matrix $\mathbf{C}$ is decomposed via Eq.\eqref{eq:03} into
\[
\mathbf{C}= \mathbf{U_C} \cdot \mathbf{\Sigma_C} \cdot{\mathbf{V_C}}^T \textrm{ with } \mathbf{U_C}=\mathbf{\Sigma_C}=\mathbf{V_C}=[1].
\]
Finally, \cite{Li2010} updates the new SVD of $\tilde{\mathbf{Q}}$ via Eq.\eqref{eq:02} as
\[
\tilde{\mathbf{U}}=\mathbf{U}\cdot \mathbf{U_C}=\left[\begin{smallmatrix}1 \\0 \end{smallmatrix}\right], \quad
\tilde{\mathbf{\Sigma}}=\mathbf{\Sigma_C}=[1], \quad
\tilde{{\mathbf{V}}}= {\mathbf{V}} \cdot {\mathbf{V_C}}=\left[\begin{smallmatrix}0 \\1 \end{smallmatrix}\right].
\]

However, one can readily verify that
\[
\tilde{\mathbf{U}} \cdot \tilde{\mathbf{\Sigma}} \cdot {\tilde{{\mathbf{V}}}}^T = \left[\begin{smallmatrix}0 & 1 \\0 & 0 \end{smallmatrix}\right] \neq  \left[\begin{smallmatrix}0 & 1 \\ 1 & 0 \end{smallmatrix}\right] = \mathbf{Q} + \mathbf{\Delta} \mathbf{Q} = \tilde{\mathbf{Q}}.
\]
In comparison, a ``true'' SVD of $\tilde{\mathbf{Q}}$ should be
\[
\tilde{\mathbf{Q}} = \hat{\mathbf{U}} \cdot \hat{\mathbf{\Sigma}} \cdot{\hat{\mathbf{V}}}^T  \textrm{ with } \hat{\mathbf{U}} = \left[\begin{smallmatrix}0 & 1 \\1 & 0\end{smallmatrix}\right], \ \hat{\mathbf{\Sigma}}=\hat{\mathbf{V}}= \left[\begin{smallmatrix}1 & 0 \\0 & 1\end{smallmatrix}\right].
\]
%
%This suggests that Li \etal\!\!'s incremental way \cite{Li2010} of updating $\mathbf{U}, \mathbf{\Sigma}, {\mathbf{V}}$ misses the eigenvector $\left[\begin{smallmatrix}0  \\1 \end{smallmatrix}\right]$
%(\eg $\tilde{\mathbf{U}}=\left[\begin{smallmatrix}1 \\0 \end{smallmatrix}\right]$,
%as compared with its ``true'' version $\hat{\mathbf{U}}=\left[\begin{smallmatrix}0 & 1 \\1 & 0\end{smallmatrix}\right]$).
Besides, the approximation error is not small in practice
\[{\|\tilde{\mathbf{Q}} - \tilde{\mathbf{U}} \cdot \tilde{\mathbf{\Sigma}} \cdot {\tilde{{\mathbf{V}}}}^T\|}_{2} = {\| \left[\begin{smallmatrix}0 & 1 \\ 1 & 0 \end{smallmatrix}\right] - \left[\begin{smallmatrix}0 & 1 \\0 & 0 \end{smallmatrix}\right] \|}_{2}=1. \quad \qed \]
\end{example}

%
%On the other hand,
Our analysis suggests that, only when
%Eq.\eqref{eq:08} holds only when
(i) $\mathbf{Q}$ is full-rank, and
(ii) the SVD of $\mathbf{Q}$ is lossless $(n=\textrm{rank}(\mathbf{Q})=\alpha)$,
Li \etal\!\!'s incremental way \cite{Li2010} can produce the \emph{exact} $\mathbf{S}$, % in \eqref{eq:03a}.
%which does not miss any eigen-information.
but the time complexity of \cite{Li2010}, $O(r^4n^2)$, would become $O(n^6)$,
which is prohibitively expensive.
%
In practice,
as evidenced by our statistical experiments in Fig.\ref{fig:exp_04} on Stanford Large Network Datasets (SNAP), %\footnote{http://snap.stanford.edu/data/},
most real graphs are not full-rank,
highlighting our need to devise an efficient method for dynamic SimRank computation.
%which is also in part demonstrated by our evaluations.
%Thus, \cite{Li2010} produces the undesirable solution of $\mathbf{S}$ in most cases.
%the condition $r=n$ is too restrictive in real applications;
%and it is often the case that \cite{Li2010} gives an approximate solution.
%
%These motivate us to devise efficient methods for incrementally computing SimRank, without loss of exactness.
%
\section{Proofs \& Intuitions of Theorems} \label{app:02}
%
%
\subsection{Proof of Theorem~\ref{thm:01}} \label{app:02a}
%
%
%The proof of Theorem~\ref{thm:01} can be obtained by a SVD-like method.
%We omit it here due to space limitations.
%
\begin{proof}
We show this by considering the two cases below:
%
%Due to space limitations,
%we shall only prove the insertion case.
%A similar proof holds for the deletion case.

(i) If ${{d}_{j}}=0$, then ${{[\mathbf{Q}]}_{j,\star}}=\mathbf{0}$, and the inserted edge $(i,j)$ will update ${{[\mathbf{Q}]}_{j,i}}$ from 0 to 1,
\ie $\mathbf{\Delta Q}={{\mathbf{e}}_{j}}\mathbf{e}_{i}^{T}$.

(ii) If ${{d}_{j}}>0$, then all nonzeros in old ${{[\mathbf{Q}]}_{j,\star}}$ are $\tfrac{1}{d_j}$.
The inserted edge $(i,j)$ will update ${{[\mathbf{Q}]}_{j,\star}}$ via 2 steps:
first, all nonzeros in ${{[\mathbf{Q}]}_{j,\star}}$ are changed from $\tfrac{1}{d_j}$ to $\tfrac{1}{d_j+1}$;
then, the entry ${{[\mathbf{Q}]}_{j,i}}$ is changed from 0 to $\tfrac{1}{d_j+1}$.
\[
{{[\mathbf{\tilde{Q}}]}_{j,\star}}
= \tfrac{{{d}_{j}}}{{{d}_{j}}+1}  {{[\mathbf{Q}]}_{j,\star}}  + \tfrac{1}{{{d}_{j}}+1}  \mathbf{e}_{i}^{T}
=  {{[\mathbf{Q}]}_{j,\star}} + \tfrac{1}{{{d}_{j}}+1} ( \mathbf{e}_{i}^{T} - {{[\mathbf{Q}]}_{j,\star}} )
\]
Since only the $j$-th row of $\mathbf{Q}$ is affected, it follows that
\[
\mathbf{\tilde{Q}} - \mathbf{Q}
= \underbrace{\tfrac{1}{{{d}_{j}}+1} \mathbf{e}_{j}}_{:= \mathbf{u}} \underbrace{( \mathbf{e}_{i}^{T} - {{[\mathbf{Q}]}_{j,\star}} )}_{:= \mathbf{v}^T}
= \mathbf{u}\cdot {\mathbf{v}}^T % , \ \textrm{ with } \mathbf{u} := {\tfrac{1}{{{d}_{j}}+1} {{\mathbf{e}}_{j}}}, \quad {\mathbf{v}}^T := {(\mathbf{e}_{i}^{T} -  {[\mathbf{Q}]}_{j,\star})}.
\]

Finally, combining (i) and (ii), Eq.\eqref{eq:18} holds.  \qed
%
%Thus, for the inserted edge $(i,j)$,
%the old $\mathbf{Q}$ can be converted into the new $\tilde{\mathbf{Q}}$ via 2 steps.
%,
%as depicted below:
%\begin{mysmall}
%\begin{eqnarray*}
%   \mathbf{Q}=&& \arraycolsep=4.5pt
%\begin{blockarray}{cccccccc}
%   {} & {} & {} & {\hspace{-20pt}(i\textrm{-th col}) \hspace{-20pt}} & {} & {} & {}  \\
%\begin{block}{[ccccccc]c}
%   \cdots & {} & {\cdots} & {} & {\cdots} & {} & {\cdots}  \\
%   \cdots  & \tfrac{1}{{{d}_{j}}} & \cdots  & 0 & \cdots  & \tfrac{1}{{{d}_{j}}} & \cdots & (j\textrm{-th row})  \\
%   \cdots & {} & {\cdots} & {} & {\cdots} & {} & {\cdots}  \\
%\end{block}
%\end{blockarray} \\[-8pt]
% \xrightarrow{\tfrac{{{d}_{j}}}{{{d}_{j}}+1}\times (j\textrm{-th row})} && \arraycolsep=2.8pt
%\left[
%\begin{array}{ccccccc}
%   {\cdots} & {} & {\cdots} & {} & {\cdots} & {} & {\cdots}  \\
%   \cdots  & \fbox{$\tfrac{1}{{{d}_{j}}+1}$} & \cdots  & 0 & \cdots  & \fbox{$\tfrac{1}{{{d}_{j}}+1}$} & \cdots   \\
%   {\cdots} & {} & {\cdots} & {} & {\cdots} & {} & {\cdots}  \\
%\end{array}
%\right]
%\ (j\textrm{-th row}) \\
%  \xrightarrow{\tfrac{1}{{{d}_{j}}+1} + (j,i)\textrm{-entry}} && \arraycolsep=2.2pt \left[ \begin{array}{*{35}{c}}
%   {\cdots} & {} & {\cdots} & {} & {\cdots} & {} & {\cdots}  \\
%   \cdots  & \tfrac{1}{{{d}_{j}}+1} & \cdots  & \fbox{$\tfrac{1}{{{d}_{j}}+1}$} & \cdots  & \tfrac{1}{{{d}_{j}}+1} & \cdots   \\
%   {\cdots} & {} & {\cdots} & {} & {\cdots} & {} & {\cdots}  \\
%\end{array} \right]=\mathbf{\tilde{Q}} \\
%\end{eqnarray*}
%\end{mysmall}
%(a) We change all nonzero entries of ${{[\mathbf{Q}]}_{j,\star}}$ from $\tfrac{1}{d_j}$ to $\tfrac{1}{d_j+1}$,
%by multiplying $\tfrac{{{d}_{j}}}{{{d}_{j}}+1}$ on the $j$-th row of $\mathbf{Q}$.
%Recall from \emph{the elementary matrix property} that multiplying the $j$-th row of a matrix by $\alpha \ne 0$ can be accomplished
%by using $\mathbf{I}-(1-\alpha ){{\mathbf{e}}_{j}}\mathbf{e}_{j}^{T}$ as a left-hand multiplier on the matrix.
%Hence, after this step, $\mathbf{Q}$ is converted into the matrix $\mathbf{Q}'$, \ie
%\begin{eqnarray*}
%\mathbf{Q}'
%= (\mathbf{I}-(1-\tfrac{{{d}_{j}}}{{{d}_{j}}+1} ){{\mathbf{e}}_{j}}\mathbf{e}_{j}^{T}) \cdot \mathbf{Q}
%= \mathbf{Q}-\tfrac{1}{{{d}_{j}}+1} {{\mathbf{e}}_{j}} \cdot {[\mathbf{Q}]}_{j,\star}.
%\end{eqnarray*}
%
%(b) We next update the $(j,i)$-entry of $\mathbf{Q'}$ from 0 to $\tfrac{1}{d_j+1}$,
%which yields the new $\tilde{\mathbf{Q}}$, \ie
%\begin{eqnarray*}
% \tilde{\mathbf{Q}} &=& \mathbf{Q}' + \tfrac{1}{{{d}_{j}}+1}  {{\mathbf{e}}_{j}}\mathbf{e}_{i}^{T}
% =  \mathbf{Q}-\tfrac{1}{{{d}_{j}}+1} {{\mathbf{e}}_{j}} \cdot ( {[\mathbf{Q}]}_{j,\star} - \mathbf{e}_{i}^{T}).
%\end{eqnarray*}
%
%Since $\mathbf{\Delta Q} = \tilde{\mathbf{Q}} - \mathbf{Q}$,
%it follows that
%\[
%\mathbf{\Delta Q} = \mathbf{u}\cdot {\mathbf{v}}^T , \ \textrm{ with } \mathbf{u} := {\tfrac{1}{{{d}_{j}}+1} {{\mathbf{e}}_{j}}}, \quad {\mathbf{v}}^T := {(\mathbf{e}_{i}^{T} -  {[\mathbf{Q}]}_{j,\star})}.
%\]
%which proves the case $d_j>0$ in Eq.\eqref{eq:18}.
\end{proof}
%
%
\subsection{Proof of Theorem~\ref{thm:02}} \label{app:02b}
%
\begin{proof}
We show this by following the two steps:

%\vspace{3pt} \noindent
(a) We first formulate $\mathbf{\Delta S}$ recursively.
%
%By definition, the old SimRank matrix $\mathbf{S}$ satisfies Eq.\eqref{eq:03a}:
%\begin{equation}\label{eq:21}
%\mathbf{S}=C\cdot \mathbf{Q}\cdot \mathbf{S}\cdot {{\mathbf{Q}}^{T}}+(1-C)\cdot {{\mathbf{I}}_{n}},
%\end{equation}
%and its new version $\mathbf{\tilde{S}}$ satisfies Eq.\eqref{eq:15}, \ie
%\begin{equation}\label{eq:22}
%\mathbf{\tilde{S}}=C\cdot \mathbf{\tilde{Q}}\cdot \mathbf{\tilde{S}}\cdot {{\mathbf{\tilde{Q}}}^{T}}+(1-C)\cdot {{\mathbf{I}}_{n}}.
%\end{equation}
%
To describe $\mathbf{\Delta S}$ in terms of the old $\mathbf{Q}$ and $\mathbf{S}$,
we subtract Eq.\eqref{eq:03a} from Eq.\eqref{eq:15},
and apply $\mathbf{\Delta S}=\mathbf{\tilde{S}}-\mathbf{S}$, yielding
\begin{equation}  \label{eq:23}
\scalebox{0.88}{$
\mathbf{\Delta S} = C\cdot \mathbf{\tilde{Q}}\cdot \mathbf{S}\cdot {{\mathbf{\tilde{Q}}}^{T}} + C\cdot \mathbf{\tilde{Q}}\cdot \mathbf{\Delta S}\cdot {{\mathbf{\tilde{Q}}}^{T}}
-C\cdot \mathbf{Q}\cdot \mathbf{S}\cdot {{\mathbf{Q}}^{T}}. $}
\end{equation}
%By Theorem \ref{thm:01},
If there are two vectors $\mathbf{u}$ and $\mathbf{v}$ such that
\begin{equation}\label{eq:24}
\mathbf{\tilde{Q}}=\mathbf{Q}+\mathbf{\Delta Q}=\mathbf{Q}+\mathbf{u} \cdot {{\mathbf{v}}^{T}},
\end{equation}
then we can plug Eq.\eqref{eq:24} into the term $C\cdot \mathbf{\tilde{Q}}\cdot \mathbf{S}\cdot {{\mathbf{\tilde{Q}}}^{T}}$ of Eq.\eqref{eq:23},
and simplify the result into
\begin{equation}\label{eq:25}
\mathbf{\Delta S}=C\cdot \mathbf{\tilde{Q}}\cdot \mathbf{\Delta S}\cdot {{\mathbf{\tilde{Q}}}^{T}}+C\cdot \mathbf{T}
\end{equation}
%%the auxiliary matrix
\begin{equation}\label{eq:26}
\textrm { with }\mathbf{T}=\mathbf{u}{{(\mathbf{QSv})}^{T}}+(\mathbf{QSv}){{\mathbf{u}}^{T}}+({{\mathbf{v}}^{{T}}}\mathbf{Sv})\mathbf{u}{{\mathbf{u}}^{T}}.
\end{equation}
%\vspace{3pt}
%(b) Based on the recursive form of $\mathbf{\Delta S}$,
%we next show that $\mathbf{T}$ in Eq.\eqref{eq:26} is the sum of two rank-one matrices.
%
We can verify that $\mathbf{T}$ is a symmetric matrix ($\mathbf{T}={{\mathbf{T}}^{T}}$).
Moreover, we note that $\mathbf{T}$ is the sum of two rank-one matrices.
This can be verified by letting
\[\mathbf{z}\triangleq \mathbf{S}\cdot \mathbf{v},\ \mathbf{y}\triangleq \mathbf{Q}\cdot \mathbf{z},\ \lambda \triangleq {{\mathbf{v}}^{T}}\cdot \mathbf{z}.\]
Then, using the auxiliary vectors $\mathbf{z}, \mathbf{y}$ and the scalar $\lambda$,
we can simplify Eq.\eqref{eq:26} into
\begin{eqnarray}
  \mathbf{T} %% &=& \mathbf{u}\cdot {{\mathbf{y}}^{T}}+\mathbf{y}\cdot {{\mathbf{u}}^{T}}+\lambda \cdot \mathbf{u}\cdot {{\mathbf{u}}^{T}} \nonumber \\
 &=&  \mathbf{u}\cdot {{\mathbf{w}}^{T}}+\mathbf{w}\cdot {{\mathbf{u}}^{T}}, \ \ \textrm{ with } \mathbf{w}= \mathbf{y}+\tfrac{\lambda }{2}\mathbf{u}.  \label{eq:27}
\end{eqnarray}


%\vspace{3pt} \noindent
(b) We next convert the recursive form of $\mathbf{\Delta S}$ into the series form.
One can readily verify that %the solution $\mathbf{X}$ to the matrix equation $\mathbf{X}=\mathbf{A}\cdot \mathbf{X}\cdot \mathbf{B}+\mathbf{C}$ has the following closed form:
\begin{equation}\label{eq:28}
\mathbf{X}=\mathbf{A}\cdot \mathbf{X}\cdot \mathbf{B}+\mathbf{C}\quad \Leftrightarrow \quad \mathbf{X}=\sum\nolimits_{k=0}^{\infty }{{{\mathbf{A}}^{k}}\cdot \mathbf{C}\cdot {{\mathbf{B}}^{k}}}
\end{equation}
Thus, based on Eq.\eqref{eq:28},
the recursive definition of $\mathbf{\Delta S}$ in Eq.\eqref{eq:25} naturally leads itself to the series form:
\[\mathbf{\Delta S}=\sum\nolimits_{k=0}^{\infty }{{{C}^{k+1}}\cdot {{{\mathbf{\tilde{Q}}}}^{k}}\cdot \mathbf{T}\cdot {{({{{\mathbf{\tilde{Q}}}}^{T}})}^{k}}}.\]
Combining this with Eq.\eqref{eq:27} yields
\begin{eqnarray*}
\mathbf{\Delta S} &=& \sum\nolimits_{k=0}^{\infty }{{{C}^{k+1}}\cdot {{{\mathbf{\tilde{Q}}}}^{k}}\cdot \left( \mathbf{u}\cdot {{\mathbf{w}}^{T}}+\mathbf{w}\cdot {{\mathbf{u}}^{T}} \right)\cdot {{({{{\mathbf{\tilde{Q}}}}^{T}})}^{k}}} \\
 &=&  \mathbf{M}+{{\mathbf{M}}^{T}}  \ \textrm{ with } \mathbf{M}\textrm{ being defined in Eq.}\eqref{eq:16}.
\end{eqnarray*}
By Eq.\eqref{eq:28},
the series form of $\mathbf{M}$ in Eq.\eqref{eq:16} satisfies the rank-one Sylvester recursive form of Eq.\eqref{eq:14}. \qed
\end{proof}
%
\subsection{Proof of Theorem~\ref{thm:03}} \label{app:02c}
%
\begin{proof}
We divide the proof into the following two cases:
%For space interests,
%we merely show insertion case.
%The proof for the deletion case is similar.

(i) When ${{d}_{j}}=0$,
according to Eq.\eqref{eq:18} in Theorem~\ref{thm:01},
%it follows that
%\begin{equation}\label{eq:30}
$\mathbf{v}={{\mathbf{e}}_{i}}, \ \mathbf{u}={{\mathbf{e}}_{j}}$.
%\end{equation}
Plugging them into Eq.\eqref{eq:20} gets
%\begin{eqnarray*}
\[
\mathbf{z} ={{[\mathbf{S}]}_{\star,i}}, \quad
\mathbf{y} =\mathbf{Q}\cdot {{[\mathbf{S}]}_{\star,i}}, \quad
\lambda = {{[\mathbf{S}]}_{i,i}}.
\]
%\end{eqnarray*}
Thus, applying $\mathbf{w}=\mathbf{y}+\tfrac{\lambda }{2}\mathbf{u}$ in Theorem \ref{thm:02},
we have
%%\begin{equation*}\label{eq:31}
\[
\mathbf{w}=\mathbf{Q}\cdot {{[\mathbf{S}]}_{\star,i}}+\tfrac{1}{2}{{[\mathbf{S}]}_{i,i}}\cdot {{\mathbf{e}}_{j}}.
\]
%%\end{equation*}
Coupling this with Eq.\eqref{eq:16}, $\mathbf{u}={{\mathbf{e}}_{j}}$, and Theorem~\ref{thm:02} completes the proof of the case ${{d}_{j}}=0$ for Eq.\eqref{eq:29aa}.

(ii) When ${{d}_{j}}>0$, Eq.\eqref{eq:18} in Theorem \ref{thm:01} implies that
\begin{equation}\label{eq:32}
\mathbf{v}={{\mathbf{e}}_{i}}-[\mathbf{Q}]_{j,\star}^{T}, \quad \mathbf{u}=\tfrac{1}{{{d}_{j}}+1}\cdot {{\mathbf{e}}_{j}}.
\end{equation}
Substituting these back into Eq.\eqref{eq:20} yields
\begin{eqnarray*}
\mathbf{z}&=&%%\mathbf{S}\cdot \mathbf{v}=\mathbf{S}\cdot \left( {{\mathbf{e}}_{i}}-[\mathbf{Q}]_{j,\star}^{T} \right) \\
{{[\mathbf{S}]}_{\star,i}}-\mathbf{S}\cdot [\mathbf{Q}]_{j,\star}^{T}, \quad
\mathbf{y}  =  \mathbf{Q}\cdot {{[\mathbf{S}]}_{\star,i}}-\mathbf{Q}\cdot \mathbf{S}\cdot [\mathbf{Q}]_{j,\star}^{T}, \\[3pt]
\lambda  %% &=& {{\mathbf{v}}^{T}}\cdot \mathbf{z} \\
%% &=& {{( {{\mathbf{e}}_{i}}-[\mathbf{Q}]_{j,\star}^{T} )}^{T}}\cdot ( {{[\mathbf{S}]}_{\star,i}}-\mathbf{S}\cdot [\mathbf{Q}]_{j,\star}^{T} ) \\
 &=& {{[\mathbf{S}]}_{i,i}}-2\cdot {{[\mathbf{Q}]}_{j,\star}}\cdot {{[\mathbf{S}]}_{\star,i}}+{{[\mathbf{Q}]}_{j,\star}}\cdot \mathbf{S}\cdot [\mathbf{Q}]_{j,\star}^{T}.
\end{eqnarray*}
%
To simplify $\mathbf{Q}\cdot \mathbf{S}\cdot [\mathbf{Q}]_{j,\star}^{T}$ in $\mathbf{y}$,
and ${{[\mathbf{Q}]}_{j,\star}}\cdot \mathbf{S}\cdot [\mathbf{Q}]_{j,\star}^{T}$ in $\lambda$,
we postmultiply both sides of Eq.\eqref{eq:03a} by ${{\mathbf{e}}_{j}}$ to obtain
%%Eq.\eqref{eq:33},
%%we postmultiply ${{\mathbf{e}}_{j}}$ on both sides of Eq.\eqref{eq:03a} produces
%\[
%{{[\mathbf{S}]}_{\star,j}}=C\cdot \mathbf{Q}\cdot \mathbf{S}\cdot {{[{{\mathbf{Q}}^{T}}]}_{\star,j}}+(1-C)\cdot {{\mathbf{e}}_{j}}.
%\]
%Using the fact that $[\mathbf{Q}]_{j,\star}^{T}={{[{{\mathbf{Q}}^{T}}]}_{\star,j}}$,
%and rearranging the terms in the above equation yield
\begin{equation} \label{eq:33}
\mathbf{Q}\cdot \mathbf{S}\cdot [\mathbf{Q}]_{j,\star}^{T}=\tfrac{1}{C}\cdot ( {{[\mathbf{S}]}_{\star,j}}-(1-C)\cdot {{\mathbf{e}}_{j}} ).
\end{equation}
We also premultiply both sides of Eq.\eqref{eq:33} by $\mathbf{e}_j^T$ to get
\begin{equation} \label{eq:34}
{[\mathbf{Q}]}_{j,\star} \cdot \mathbf{S}\cdot [\mathbf{Q}]_{j,\star}^{T}=\tfrac{1}{C}\cdot ( {{[\mathbf{S}]}_{j,j}}-1) + 1.
\end{equation}
Plugging Eqs.\eqref{eq:33} and \eqref{eq:34} into $\mathbf{y}$ and $\lambda$, respectively,
and then putting $\mathbf{y}$ and $\lambda$ into $\mathbf{w}=\mathbf{y}+\tfrac{\lambda }{2}\mathbf{u}$ produce
%we can obtain
%%Plugging these into $\mathbf{w}=\mathbf{y}+\tfrac{\lambda }{2}\mathbf{u}$ produces
%\begin{eqnarray}
%\mathbf{w} &=&  \mathbf{y}+\tfrac{\lambda }{2}\mathbf{u} \nonumber \\
%   &=&  \mathbf{Q}\cdot {{[\mathbf{S}]}_{\star,i}}-\mathbf{Q}\cdot \mathbf{S}\cdot [\mathbf{Q}]_{j,\star}^{T}+\tfrac{\lambda }{2(d_j+1)}\cdot {{\mathbf{e}}_{j}}.  \label{eq:35}
%\end{eqnarray}
%
%Substituting this back into Eq.\eqref{eq:35} gets
\begin{equation*}
%%\scalebox{0.95}{$
\mathbf{w}=  \mathbf{Q}\cdot {{[\mathbf{S}]}_{\star,i}}-\tfrac{1}{C}\cdot {{[\mathbf{S}]}_{\star,j}}+ ( \tfrac{1}{C}+\tfrac{\lambda }{2 ( {{d}_{j}}+1 )}-1 )\cdot {{\mathbf{e}}_{j}},
%%$}
\end{equation*}
where
$\lambda = {{[\mathbf{S}]}_{i,i}}+\tfrac{1}{C} \cdot {[\mathbf{S}]}_{j,j}-2\cdot {{[\mathbf{Q}]}_{j,\star}}\cdot {{[\mathbf{S}]}_{\star,i}} - \tfrac{1}{C} +1$.

Combining this with Eqs.\eqref{eq:16} and \eqref{eq:32} shows the case ${{d}_{j}}>0$ for Eq.\eqref{eq:29bb}.

Finally, taking (i) and (ii)
together with Theorem~\ref{thm:02}
completes the entire proof. \qed
\end{proof}
%
\subsection{Proof of Theorem~\ref{thm:09}} \label{app:02d}
%
\begin{proof}
We prove this by considering two cases:

  (i) If ${{d}_{j}}=1$, then after the edge $(i,j)$ is deleted, ${{[\mathbf{Q}]}_{j,i}}$ will change from 1 to 0,
\ie
\[
\mathbf{\Delta Q}=\mathbf{u} \cdot \mathbf{v}^T \quad \textrm{ with }  \mathbf{u} = {{\mathbf{e}}_{j}} \textrm{ and } \mathbf{v} = - \mathbf{e}_{i}.
\]
According to Eq.\eqref{eq:20} in Theorem~\ref{thm:02}, we have
\[
\mathbf{z} =-{{[\mathbf{S}]}_{\star,i}}, \quad
\mathbf{y} =-\mathbf{Q}\cdot {{[\mathbf{S}]}_{\star,i}}, \quad
\lambda = {{[\mathbf{S}]}_{i,i}}.
\]
Thus, plugging them into $\mathbf{w}=\mathbf{y}+\tfrac{\lambda }{2}\mathbf{u}$ produces
%%\begin{equation*}\label{eq:31}
\[
\mathbf{w}=-\mathbf{Q}\cdot {{[\mathbf{S}]}_{\star,i}}+\tfrac{1}{2}{{[\mathbf{S}]}_{i,i}}\cdot {{\mathbf{e}}_{j}}.
\]
Combining this with Theorem~\ref{thm:02} completes the proof of the case when $d_j= 1$.

(ii) If ${{d}_{j}}>1$, then all nonzeros in old ${{[\mathbf{Q}]}_{j,\star}}$ are $\tfrac{1}{d_j}$.
The deleted edge $(i,j)$ will update ${{[\mathbf{Q}]}_{j,\star}}$ via 2 steps:
first, all nonzeros in ${{[\mathbf{Q}]}_{j,\star}}$ are changed from $\tfrac{1}{d_j}$ to $\tfrac{1}{d_j-1}$;
then, the entry ${{[\mathbf{Q}]}_{j,i}}$ is changed from $\tfrac{1}{d_j}$ to 0.
\[
{{[\mathbf{\tilde{Q}}]}_{j,\star}}
= \tfrac{{{d}_{j}}}{{{d}_{j}}-1} ( {{[\mathbf{Q}]}_{j,\star}}  - \tfrac{1}{{{d}_{j}}}  \mathbf{e}_{i}^{T} )
=  {{[\mathbf{Q}]}_{j,\star}} + \tfrac{1}{{{d}_{j}}-1} ( {{[\mathbf{Q}]}_{j,\star}} - \mathbf{e}_{i}^{T} )
\]
Since only the $j$-th row of $\mathbf{Q}$ is affected, it follows that
\[
\mathbf{\tilde{Q}} - \mathbf{Q}
= \underbrace{\tfrac{1}{{{d}_{j}}-1} \mathbf{e}_{j}}_{:= \mathbf{u}} \underbrace{( {{[\mathbf{Q}]}_{j,\star}} - \mathbf{e}_{i}^{T} )}_{:= \mathbf{v}^T}
= \mathbf{u}\cdot {\mathbf{v}}^T % , \ \textrm{ with } \mathbf{u} := {\tfrac{1}{{{d}_{j}}+1} {{\mathbf{e}}_{j}}}, \quad {\mathbf{v}}^T := {(\mathbf{e}_{i}^{T} -  {[\mathbf{Q}]}_{j,\star})}.
\]
By virtue of Eq.\eqref{eq:20} in Theorem~\ref{thm:02}, we have
\begin{eqnarray*}
\mathbf{z} &=& \mathbf{S}\cdot \mathbf{v}=\mathbf{S}\cdot \left( [\mathbf{Q}]_{j,\star}^{T} - {{\mathbf{e}}_{i}} \right)
= \mathbf{S}\cdot [\mathbf{Q}]_{j,\star}^{T} - {{[\mathbf{S}]}_{\star,i}}, \\
\mathbf{y} &=& \mathbf{Q}\cdot \mathbf{z} = \mathbf{Q}\cdot \mathbf{S}\cdot [\mathbf{Q}]_{j,\star}^{T} - \mathbf{Q}\cdot {{[\mathbf{S}]}_{\star,i}}
= \textrm{\{using Eq.\eqref{eq:33}\}} \\
&=& \tfrac{1}{C}\cdot ( {{[\mathbf{S}]}_{\star,j}}-(1-C)\cdot {{\mathbf{e}}_{j}} ) - \mathbf{Q}\cdot {{[\mathbf{S}]}_{\star,i}}. \\
\lambda &=& {{\mathbf{v}}^{T}}\cdot \mathbf{z} = {{( [\mathbf{Q}]_{j,\star} - {{\mathbf{e}}_{i}^{T}} )} }\cdot ( \mathbf{S}\cdot [\mathbf{Q}]_{j,\star}^{T} - {{[\mathbf{S}]}_{\star,i}} ) \\
 &=& {{[\mathbf{S}]}_{i,i}}-2\cdot {{[\mathbf{Q}]}_{j,\star}}\cdot {{[\mathbf{S}]}_{\star,i}}+{{[\mathbf{Q}]}_{j,\star}}\cdot \mathbf{S}\cdot [\mathbf{Q}]_{j,\star}^{T}. \\
 &=& \textrm{\{using Eq.\eqref{eq:34}\}}  \\
 &=& {{[\mathbf{S}]}_{i,i}}-2\cdot {{[\mathbf{Q}]}_{j,\star}}\cdot {{[\mathbf{S}]}_{\star,i}}+\tfrac{1}{C}\cdot ( {{[\mathbf{S}]}_{j,j}}-1) + 1.
\end{eqnarray*}
Hence, substituting them into $\mathbf{w}=\mathbf{y}+\tfrac{\lambda }{2}\mathbf{u}$ yields
%%\begin{equation*}\label{eq:31}
\[
\mathbf{w}=  -\mathbf{Q}\cdot {{[\mathbf{S}]}_{\star,i}}+\tfrac{1}{C}\cdot {{[\mathbf{S}]}_{\star,j}}+ ( 1-\tfrac{1}{C}+\tfrac{\lambda }{2 ( {{d}_{j}}-1 )} )\cdot {{\mathbf{e}}_{j}}.
\]
Combining this with Theorem~\ref{thm:02} completes the proof of the case when $d_j> 1$.

Finally, coupling (i) and (ii) proves Theorem~\ref{thm:09}.  \qed
\end{proof}
%
\subsection{Proof and Intuition of Theorem~\ref{thm:04}} \label{app:02e}
%
\begin{proof}
We only show the edge insertion case ${{d}_{j}}>0$, due to space limits.
The proofs of other cases are similar.

For $k=0$,
it follows from Eq.\eqref{eq:29c} that ${{[{{\mathbf{M}}_{0}}]}_{a,b}}={{[{{\mathbf{e}}_{j}}]}_{a}} {{[\bm{\gamma }]}_{b}}$.
Thus, $\forall (a,b)\notin {{\mathsf{\mathcal{A}}}_{0}}\times {{\mathsf{\mathcal{B}}}_{0}}$,
there are two cases:
(i) $a\ne j$, or
(ii) $a=j$, $b\in {{\mathsf{\mathcal{F}}}_{1}}^{C}\cap {{\mathsf{\mathcal{F}}}_{2}}^{C}$, and $b\ne j$.

For case (i), ${{[{{\mathbf{e}}_{j}}]}_{a}}=0$ for $a\ne j$.
Thus, ${{[{{\mathbf{M}}_{0}}]}_{a,b}}=0$.
For case (ii),
${{[{{\mathbf{e}}_{j}}]}_{a}}=1$ for $a=j$.
Thus, ${{[{{\mathbf{M}}_{0}}]}_{a,b}}={{[\bm{\gamma }]}_{b}}$,
where ${{[\bm{\gamma }]}_{b}}$ is the linear combinations of the 3 terms:
${{[\mathbf{Q}]}_{b,\star}}\cdot {{[\mathbf{S}]}_{\star,i}}$, ${{[\mathbf{S}]}_{b,j}}$, and ${{[{{\mathbf{e}}_{j}}]}_{b}}$,
according to the case of ${{d}_{j}}>0$ in Eq.\eqref{eq:29bb}.

Next, our goal is to show the 3 terms are all 0s.
%which implies that ${{[{{\mathbf{M}}_{0}}]}_{a,b}}=0$.
(a) For $b\notin {{\mathsf{\mathcal{F}}}_{1}}$,
by definition in Eq.\eqref{eq:39},
$b\in \mathsf{\mathcal{O}}(y)$ for $\forall y$,
we have ${{[\mathbf{S}]}_{i,y}}=0$.
Due to symmetry,
$b\in \mathsf{\mathcal{O}}(y)\Leftrightarrow y\in \mathsf{\mathcal{I}}(b)$,
which implies that ${{[\mathbf{S}]}_{i,y}}=0$ for $\forall y\in \mathsf{\mathcal{I}}(b)$.
\footnote{Herein, we denote by $\mathcal{I}(a)$ the in-neighbor set of node $a$.}
Thus, ${{[\mathbf{Q}]}_{b,\star}}\cdot {{[\mathbf{S}]}_{\star,i}}=\frac{1}{\mathsf{\mathcal{I}}(b)}\sum_{x\in \mathsf{\mathcal{I}}(b)}^{{}}{{{[\mathbf{S}]}_{x,i}}}=0$.
(b) For $b\notin {{\mathsf{\mathcal{F}}}_{2}}$,
it follows from the case ${{d}_{j}}>0$ in Eq.\eqref{eq:40} that ${{[\mathbf{S}]}_{j,b}}=0$.
Hence, by $\mathbf{S}$ symmetry,
${{[\mathbf{S}]}_{b,j}}={{[\mathbf{S}]}_{j,b}}=0$.
(c) ${{[{{\mathbf{e}}_{j}}]}_{b}}=0$ since $b\ne j$.

Taking (a)--(c) together,
it follows that ${{[{{\mathbf{M}}_{0}}]}_{a,b}}=0$,
which completes the proof for the case $k=0$.

For $k>0$, one can readily prove that the $k$-th iterative ${{\mathbf{M}}_{k}}$ in Line~\ref{ln:a01-17} of Algorithm~\ref{alg:01} is the first $k$-th partial sum of $\mathbf{M}$ in Eq.\eqref{eq:29c}.
Thus, ${{\mathbf{M}}_{k+1}}$ can be derived from ${{\mathbf{M}}_{k}}$ as follows:
\[
\mathbf{M}_{k} = C \cdot \tilde{\mathbf{Q}} \cdot \mathbf{M}_{k-1} \cdot \tilde{\mathbf{Q}}^T + C \cdot \mathbf{e}_j \cdot \bm{\gamma}^T.
\]
Thus, the $(a,b)$-entry form of the above equation is
\[ \scalebox{0.92}{$
{[\mathbf{M}_{k}]}_{a,b} = \tfrac{C}{|\tilde{{\cal I}}(a)||\tilde{{\cal I}}(b)|}   \sum\nolimits_{x \in \tilde{{\cal I}}(a)}  \sum\nolimits_{y \in \tilde{{\cal I}}(b)} {[\mathbf{M}_{k-1}]}_{x,y} + C \cdot {[\mathbf{e}_j]}_{a} \cdot {[\bm{\gamma}]}_{b}.
$}\]
To show that ${[\mathbf{M}_{k}]}_{a,b}=0$ for $(a,b)\notin {{\mathsf{\mathcal{A}}}_{0}}\times {{\mathsf{\mathcal{B}}}_{0}} \cup {{\mathsf{\mathcal{A}}}_{k}}\times {{\mathsf{\mathcal{B}}}_{k}}$,
we follow the 2 steps:
(i) For $(a,b)\notin {{\mathsf{\mathcal{A}}}_{0}}\times {{\mathsf{\mathcal{B}}}_{0}}$,
as proved in the case $k=0$, the term $C \cdot {{[{{\mathbf{e}}_{j}}]}_{a}} {{[\bm{\gamma }]}_{b}}$ in the above equation is obviously 0.
(ii) For $(a,b)\notin {{\mathsf{\mathcal{A}}}_{k}}\times {{\mathsf{\mathcal{B}}}_{k}}$,
by virtue of Eq.\eqref{eq:41},
$a \in \tilde{\cal O}(x), b \in \tilde{\cal O}(y)$, for $\forall x,y$,
we have ${[\mathbf{M}_{k-1}]}_{x,y}=0$.
Hence, by symmetry, it follows that
$x \in \tilde{\cal I}(a), y \in \tilde{\cal I}(b)$, ${[\mathbf{M}_{k-1}]}_{x,y}=0$.

Taking (i) and (ii) together,
we can conclude that
${[\mathbf{M}_{k}]}_{a,b} = 0$
for $(a,b)\notin {{\mathsf{\mathcal{A}}}_{0}}\times {{\mathsf{\mathcal{B}}}_{0}} \cup {{\mathsf{\mathcal{A}}}_{k}}\times {{\mathsf{\mathcal{B}}}_{k}}$. \qed
\end{proof}

Intuitively, ${\cal F}_1$ in Eq.\eqref{eq:39} captures the nodes ``$\blacktriangle$'' in \eqref{eq:39a}.
To be specific,
${\cal F}_1$ can be obtained via 2 phases:
(i) For the given node $i$,
we first build an intermediate set ${\cal T}:=\{y  |  {{[\mathbf{S}]}_{i,y}}\ne 0\}$,
which consists of nodes ``$\star$'' in \eqref{eq:39a}.
(ii) For each node $x \in {\cal T}$,
we then find all out-neighbors of $x$ in $G$, which produces ${\cal F}_1$,
\ie ${\cal F}_1 = \bigcup_{x\in {\cal T}}{{\cal O}(x)}$.
Analogously, the set ${\cal F}_2$ in Eq.\eqref{eq:40},
in the case of $d_j>0$, consists of the nodes ``$\star$'' depicted in~\eqref{eq:39b}.
When $d_j=0$, ${\cal F}_2 = \varnothing $ as the term ${{[\mathbf{S}]}_{\star,i}}$ is not in the expression of $\bm{\gamma }$ in Eq.\eqref{eq:29aa}, % for the case when $d_j=0$,
in contrast to the case $d_j>0$.

After obtaining ${\cal F}_1$ and ${\cal F}_2$,
we can readily find ${{\mathsf{\mathcal{A}}}_{0}}\times {{\mathsf{\mathcal{B}}}_{0}}$, according to Eq.\eqref{eq:41}.
For $k>0$, to iteratively derive the node-pair set ${{\mathsf{\mathcal{A}}}_{k}}\times {{\mathsf{\mathcal{B}}}_{k}}$,
we take the following two steps:
(i) we first construct a node-pair set ${\cal T}_1 \times {\cal T}_2 :=\{(x,y) | {[\mathbf{M}_{k-1}]}_{x,y} \neq 0\}$.
(ii) For every node $x \in {\cal T}_1$ (\Resp $y \in {\cal T}_2$),
we then find all out-neighbors of $x$ (\Resp $y$) in $G \cup \{(i,j)\}$, which yields ${\cal A}_k$ (\Resp ${\cal B}_k$),
\ie ${\cal A}_k = \bigcup_{x\in {\cal T}_1}{\tilde{{\cal O}}(x)}$ and $ {\cal B}_k = \bigcup_{y\in {\cal T}_2}{\tilde{{\cal O}}(y)}$.

The node selectivity of Theorem \ref{thm:04} hinges on $\mathbf{\Delta S}$ sparsity.
Since real graphs are constantly updated with \emph{minor} changes,
$\mathbf{\Delta S}$ is often \emph{sparse} in general.
Hence, many node-pairs with zero scores in $\mathbf{\Delta S}$ can be discarded. % in practice.
As demonstrated by our experiments in Fig.\ref{fig:exp_07},
76.3\% paper-pairs on \DBLP~can be pruned,
significantly reducing unnecessary similarity recomputations. % in response to link updates.
%
\section{Examples} \label{app:03}
%
%
\subsection{Li \etal\!\!�s SVD incremental approach} \label{app:03a}
%
\begin{example} \label{eg:01}
Figure~\ref{fig:01} depicts a citation graph $G$, a tiny fraction of \DBLP,
where each node is a paper, and an edge represents a reference from one paper to another.
Suppose $G$ is updated by adding an edge $(i,j)$, denoted by $\Delta G$ (see the dash arrow).
Using the damping factor $C=0.8$,
%\footnote{According to \cite{Jeh2002},
%$C$ is empirically set around 0.6--0.8, indicating the rate of decay as similarity flows across edges.},
we would like to compute SimRank scores in the new graph $G \cup \Delta G$.

The results are compared in the table of Figure~\ref{fig:01},
where Column `$\textsf{sim}_\textsf{Li et al.}$' denotes the approximation of SimRank scores returned by Li \etal\!\!'s Algorithm~3~\cite{Li2010},
and Column `$\textsf{sim}_\textsf{true}$' denotes the ``true'' SimRank scores returned by a batch algorithm \cite{Fujiwara2013} that runs in $G \cup \Delta G$ from scratch.
%
%
%The existing method by Li \etal (see Algorithm~3 in \cite{Li2010}) first decomposes the old matrix $\mathbf{Q}=\mathbf{U} \cdot \mathbf{\Sigma} \cdot {\mathbf{V}}^T$ as a precomputation step.
%Then, when new edge $(i,j)$ is inserted,
%it incrementally updates the old $\mathbf{U},\mathbf{\Sigma},{\mathbf{V}}^T$,
%and utilizes their updated versions to evaluate the new SimRank scores in $G \cup \Delta G$.
%The results are shown in Column `$\textsf{sim}_\textsf{Li et al.}$' of the table.
%For comparison,
%we also run a batch algorithm \cite{Yu2013} to compute the ``true'' SimRank scores in $G \cup \Delta G$ from scratch,
%as illustrated in Column `$\textsf{sim}_\textsf{true}$'.
It can be noticed that for some node-pairs (not highlighted in gray),
the similarities obtained by Li \etal\!\!'s incremental method are different from the ``true'' SimRank scores
even if lossless SVD is used
\footnote{A \emph{rank-$\alpha$ SVD} of the matrix $\mathbf{X} \in \mathbb{R}^{n \times n}$ is a factorization of the form
$\mathbf{X}_{\alpha} = \mathbf{U} \cdot \mathbf{\Sigma} \cdot {\mathbf{V}}^T$,
where $\mathbf{U},\mathbf{V} \in \mathbb{R}^{n \times \alpha}$ are column-orthonormal matrices, and
$\mathbf{\Sigma} \in \mathbb{R}^{\alpha \times \alpha}$ is a diagonal matrix,
$\alpha$ is called \emph{the target rank} of the SVD, as specified by the user.

If $\alpha=\textrm{rank}(\mathbf{X})$,
then $\mathbf{X}_{\alpha}=\mathbf{X}$,
and we call it the \emph{lossless SVD}.

If $\alpha<\textrm{rank}(\mathbf{X})$,
then ${\|\mathbf{X}-\mathbf{X}_{\alpha}\|}_{2}$ gives the least square estimate error,
and we call it the \emph{low-rank SVD}.
}
during the process of updating ${\mathbf{U}},{\mathbf{\Sigma}},{{\mathbf{V}}}^T$.
This suggests that Li~\etal\!\!'s incremental approach~\cite{Li2010} is inherently \emph{approximate}.
In fact, % as will be rigorously explained in Section~\ref{sec:03b},
their incremental strategy would neglect some useful eigen-information whenever $\textrm{rank}(\mathbf{Q})<n$.

We also notice that the target rank $r$ for the SVD of the matrix $\mathbf{C}$
\footnote{As defined in \cite{Li2010}, $r$ is the target rank for the SVD of the auxiliary matrix $\mathbf{C} \triangleq \mathbf{\Sigma} + \mathbf{U}^T \cdot \mathbf{\Delta Q} \cdot \mathbf{V}$,
where $\mathbf{\Delta Q}$ is the changes to $\mathbf{Q}$ for link updates.}
is not always negligibly smaller than~$n$.
For example, in Column `$\textsf{sim}_\textsf{Li et al.}$' of Figure~\ref{fig:01},
$r$ is chosen to be $ \textrm{rank}(\mathbf{C})=9$ to get a \emph{lossless} SVD of $\mathbf{C}$.
Although $r=9$ appears not negligibly smaller than $n=15$,
the accuracy of `$\textsf{sim}_\textsf{Li et al.}$' is still undesirable as compared with `$\textsf{sim}_\textsf{true}$',
not to mention using $r<9$. \qed
\end{example}

Example~\ref{eg:01} implies that Li \etal\!\!'s incremental approach~\cite{Li2010} is approximate and may produce high computational overheads
since $r$ is not always much smaller.
%
%,
%and the $O({r}^{4}n^2)$ time and $O({r}^{2}n^2)$ memory for updating all pairs of SimRank might exacerbate the computational cost,
%as $r$ is not always much smaller than $n$.
%
\subsection{Example of Theorem~\ref{thm:01}} \label{app:03b}
%
\begin{example} \label{eg:04a}
Recall the digraph $G$ in Fig.~\ref{fig:01},
and the edge $(i,j)$ to be inserted into $G$.
%Suppose there is an edge $(i,j)$ inserted into $G$.
Notice that, in the old $G$, $d_j=2>0$ and
\vspace{-15pt} \[{[\mathbf{Q}]}_{j,\star}=\kbordermatrix{
 \hspace*{-0.5em} &   &        &   &         (h)   &     &   &  (k)            &   &        &   \\
 \hspace*{-0.5em} & 0 & \cdots & 0 & \tfrac{1}{2} &   0    & 0    &  \tfrac{1}{2}  & 0 & \cdots & 0 \\
} \in \mathbb{R}^{1 \times 15}.\]
According to Theorem \ref{thm:01},
the change $\mathbf{\Delta Q}$ is a $15 \times 15$ rank-one matrix,
and can be decomposed as
$\mathbf{u}\cdot \mathbf{v}^T$  with

\scalebox{0.85}{$  \mathbf{u}= \tfrac{1}{{{d}_{j}}+1}{{\mathbf{e}}_{j}}=\tfrac{1}{3}{{\mathbf{e}}_{j}}
=\kbordermatrix{
 \hspace*{-0.5em} &   &        &   &     (j)      &   &        &   \\
 \hspace*{-0.5em} & 0 & \cdots & 0 & \tfrac{1}{3} & 0 & \cdots & 0 \\
}{}^T \in \mathbb{R}^{15 \times 1}, $} \\[3pt]
\scalebox{0.85}{$  \mathbf{v}= {{\mathbf{e}}_{i}}-{{[\mathbf{Q}]}_{j,\star}^T}
=\kbordermatrix{
 \hspace*{-0.5em} &   &        &   &         (h)   &  (i)   & (j)  &  (k)            &   &        &   \\
 \hspace*{-0.5em} & 0 & \cdots & 0 & -\tfrac{1}{2} &   1    & 0    &  -\tfrac{1}{2}  & 0 & \cdots & 0 \\
}{}^T \in \mathbb{R}^{15 \times 1}. $}  \qed
\end{example}
%
\subsection{Example of Algorithm~\ref{alg:01}} \label{app:03c}
%
\begin{example} \label{eg:04}
Consider the old digraph $G$ and $\mathbf{S}$ in Fig.~\ref{fig:01}.
When the new edge $(i,j)$ is inserted to $G$,
{\IncUSRone} computes the new $\tilde{\mathbf{S}}$ as follows,
whose results are partially depicted in Column `$\textsf{sim}_\textsf{true}$' of Fig.~\ref{fig:01}.

Given the following information from the old $\mathbf{S}$:
%\footnote{
%Due to space limitations, only the $i$-th and $j$-th columns of $\mathbf{S}$ are displayed here,
%which is sufficient to compute $\tilde{\mathbf{S}}$.}

\scalebox{0.8}{$  {[\mathbf{S}]}_{\star,i}
=\kbordermatrix{
 \hspace*{-0.5em} &   &        &   &  (f)   &  (g)   & (h)  &  (i)    &   (j)  &   &        &   \\
 \hspace*{-0.5em} & 0, & \cdots, & 0, &  0.246, &   0,    & 0,    &  0.590,  &  0.310, & 0, & \cdots, & 0 \\
}{}^T \in \mathbb{R}^{15 \times 1}, $}

\scalebox{0.8}{$  {[\mathbf{S}]}_{\star,j}
=\kbordermatrix{
 \hspace*{-0.5em} &   &        &   &  (f)   &  (g)   & (h)  &  (i)    &   (j)  &   &        &   \\
 \hspace*{-0.5em} & 0, & \cdots, & 0, &  0.246, &   0,    & 0,    &  0.310,  &  0.510, & 0, & \cdots, & 0 \\
}{}^T \in \mathbb{R}^{15 \times 1}, $} \\[1pt]

\IncUSRone~first computes $\mathbf{w}$ and $\lambda$ via lines \ref{ln:a01-03}--\ref{ln:a01-04}:\\[-15pt]
\begin{eqnarray*}
 {\mathbf{w}}
&=& \kbordermatrix{
 \hspace*{-0.5em} &  (a)   &  (b)   &    &         &   \\
 \hspace*{-0.5em} & 0.104, & 0.139, & 0, & \cdots, & 0 \\
}{}^T \in \mathbb{R}^{15 \times 1}, \\
\lambda &=& 0.590+\tfrac{1}{0.8} \times 0.510 -2 \times 0 - \tfrac{1}{0.8} +1 = 0.978.
\end{eqnarray*}
%As an ``edge insertion'' operation,

Since $d_j=2$,
the vectors $\mathbf{u}$ and $\mathbf{v}$ for the rank-one decomposition of $\mathbf{\Delta Q}$ can be computed via line~\ref{ln:a01-07a}.
Their results are depicted in Example \ref{eg:04a}.

Next, $\bm \gamma$ can be obtained from $\mathbf{w}$ and $\lambda$ via line~\ref{ln:a01-08}:
\begin{eqnarray*}
&&{\bm\gamma} = \tfrac{1}{(2+1)} \big( \mathbf{w}-\tfrac{1}{0.8}  {{[\mathbf{S}]}_{\star,j}}+( \tfrac{\lambda }{2 \times ( 2+1 )}+ \tfrac{1}{0.8}-1 )  {{\mathbf{e}}_{j}} \big) \\
&=& \scalebox{0.72}{$ \kbordermatrix{
 \hspace*{-0.5em} &  (a)   &  (b)   &    &    &    &   (f)   &    &     &  (i)    &  (j)    &    &         &   \\
 \hspace*{-0.5em} & 0.035, & 0.046, & 0, & 0, & 0, & -0.086  & 0, &  0, & -0.129, & -0.075, & 0, & \cdots, & 0 \\
}{}^T \in \mathbb{R}^{15 \times 1}$}
\end{eqnarray*}

In light of $\bm \gamma$, $\mathbf{M}_k$ can be computed via lines \ref{ln:a01-13}--\ref{ln:a01-17}.
After $K=10$ iterations, $\mathbf{M}_{K}$ can be derived as follows:
\[
\scalebox{0.6}{$
\begin{blockarray}{ccccc|cccc|c|c}
       & (a)     & (b)      & (c)& (d)      & (e)       & (f)    & \cdots      & (i)          & (j)     & (k)\cdots(o)  \\[3pt]
 \begin{block}{c(cccc|cccc|c|c)}
   (a) & -0.005  & -0.009   &  0 & 0.009    &           &        &             &              & -0.009  &               \\
   (b) & -0.004  & -0.006   &  0 & 0.006    &           &        & \Big{$0$}   &     & -0.007  & \Big{$0$}     \\
   (c) & 0       &  0       &  0 & 0        &           &        &             &              &     0   &               \\
   (d) & -0.002  & -0.002   &  0 &  -0.005  &           &        &             &              &     0   &               \\\cline{1-11}
%%   (e) &         &          &    &          &           &        &             &              &         &               \\
\vdots &         & \Big{$0$}&    &          &           &        & \Big{$0$}   &              &\Big{$0$}& \Big{$0$}     \\
   (i) &         &          &    &          &           &        &             &              &         &               \\\cline{1-11}
   (j) &  0.028  &  0.037   &  0 &  0       &           & -0.068 &             & -0.104       & -0.060  &               \\\cline{1-11}
%%   (k) &         &          &    &          &           &        &             &              &         &               \\
\vdots &         & \Big{$0$}&    &          &           &        & \Big{$0$}   &              &\Big{$0$}& \Big{$0$}     \\
   (o) &         &          &    &          &           &        &             &              &         &               \\
 \end{block}
\end{blockarray}$}\]

\vspace{-12pt} Finally, using $\mathbf{M}_{K}$ and the old $\mathbf{S}$, the new $\tilde{\mathbf{S}}$ is obtained via line \ref{ln:a01-18},
as partly shown in Column `$\textsf{sim}_\textsf{true}$' of Fig.~\ref{fig:01}. \qed
\end{example}
%
%
\subsection{Example of Theorem~\ref{thm:04}} \label{app:03d}
%
\begin{example} \label{eg:05}
Recall Example~\ref{eg:04} and the old graph $G$ in Fig.~\ref{fig:01}.
When edge $(i,j)$ is inserted to $G$,
according to Theorem~\ref{thm:04},
${\cal F}_1 =\{a,b\}, \ {\cal F}_2=\{f,i,j\}, \ {\cal A}_0 \times {\cal B}_0=\{j\} \times \{a,b,f,i,j\}$.
Hence, instead of computing the entire vector $\bm \gamma$ in Eqs.\eqref{eq:29aa} and \eqref{eq:29bb},
we only need to compute part of its entries ${[\bm \gamma]}_{x}$ for $\forall x \in {\cal B}_0$.

For the first iteration, since ${\cal A}_1 \times {\cal B}_1=\{a,b\} \times \{a,b,d,j\}$,
then we only need to compute $18 \ (=3 \times 6)$ entries ${[\mathbf{M}_1]}_{x,y}$ for $\forall (x,y)\in \{a,b,j\}\times \{a,b,d,f,i,j\}$,
skipping the computations of $207 \ (={15}^{2} - 18)$ remaining entries in $\mathbf{M}_1$.
After $K=10$ iterations, many unnecessary node-pairs are pruned,
as in part highlighted in the gray rows of the table in Fig.~\ref{fig:01}. \qed
\end{example}
%
\section{Algorithms \& Analysis} \label{app:04}
%
%
\subsection{{\IncUSRone} Algorithm} \label{app:04a}
%
\setcounter{algocf}{5}
\begin{algorithm}[t]
\small
\DontPrintSemicolon
%\SetCommentSty{textsf}
\SetKwInOut{Input}{Input}
\SetKwInOut{Output}{Output}
%\SetKwFunction{Len}{Len}
\Input{a directed graph $G=(V,E)$, \\
       a new edge $(i,j)_{i \in V, \ j \in V}$ inserted to $G$, \\
       the old similarities $\mathbf{S}$ in $G$, \\
       the number of iterations $K$, \\
       the damping factor $C$.}
\Output{the new similarities ${\mathbf{\tilde{S}}}$ in $G \cup\{(i,j)\}$.}
\nl \label{ln:a01-01} initialize the transition matrix $\mathbf{Q}$ in $G$ ; \;
\nl \label{ln:a01-02}  ${{d}_{j}}:=$ in-degree of node $j$ in $G$ ; \;
\nl \label{ln:a01-03} memoize $\mathbf{w} := \mathbf{Q}\cdot {{[\mathbf{S}]}_{\star,i}}$ ; \;
\nl \label{ln:a01-04} compute $\lambda := {{[\mathbf{S}]}_{i,i}}+\tfrac{1}{C} \cdot {[\mathbf{S}]}_{j,j}-2\cdot {[\mathbf{w}]}_{j} - \tfrac{1}{C} +1$ ; \;
%\nl \label{ln:a01-05}     \uIf {edge $(i,j)$ is to be inserted} {
\nl \label{ln:a01-06}         \uIf {${{d}_{j}}=0$} {
\nl \label{ln:a01-06a}              $\mathbf{u} := \mathbf{e}_j, \ \mathbf{v} := \mathbf{e}_i, \ {\bm \gamma} :=  \mathbf{w} +\frac{1}{2}{{[\mathbf{S}]}_{i,i}}\cdot {{\mathbf{e}}_{j}}$; \;}
\nl \label{ln:a01-07}         \Else {
\nl \label{ln:a01-07a}           $\mathbf{u}:= \tfrac{1}{d_j+1} \mathbf{e}_j, \quad \mathbf{v} := \mathbf{e}_i-{[\mathbf{Q}]}_{j,\star}^T$ ; \;
\nl \label{ln:a01-08}  ${\bm\gamma} := \tfrac{1}{({{d}_{j}}+1)} \big( \mathbf{w}-\frac{1}{C}  {{[\mathbf{S}]}_{\star,j}}+( \frac{\lambda }{2\left( {{d}_{j}}+1 \right)}+ \frac{1}{C}-1 )  {{\mathbf{e}}_{j}} \big)$; }
%}
%\nl \label{ln:a01-09}     \ElseIf {edge $(i,j)$ is to be deleted} {
%\nl \label{ln:a01-10}         \lIf {${{d}_{j}}=1$} {$\mathbf{u} := \mathbf{e}_j, \ \mathbf{v} := -\mathbf{e}_i, \ {\bm \gamma} := \frac{1}{2}{{[\mathbf{S}]}_{i,i}}\cdot {{\mathbf{e}}_{j}} - \mathbf{w}$;\;}
%\nl \label{ln:a01-11}         \lElse {$\mathbf{u}:= \tfrac{1}{d_j-1} \mathbf{e}_j, \quad \mathbf{v} := {[\mathbf{Q}]}_{j,\star}^T - \mathbf{e}_i$; \;
%\nl \label{ln:a01-12}  $\qquad {\bm\gamma} := \tfrac{1}{({{d}_{j}}-1)} \big( \frac{1}{C}  {{[\mathbf{S}]}_{\star,j}} - \mathbf{w} +( \frac{\lambda }{2\left( {{d}_{j}}-1 \right)}- \frac{1}{C}+1 )  {{\mathbf{e}}_{j}} \big)$;\;}}
\nl \label{ln:a01-13}      initialize ${{\bm{\xi }}_{0}} := C \cdot \mathbf{e}_j,\quad {{\bm{\eta }}_{0}} := {\bm\gamma},\quad {{\mathbf{M}}_{0}} := C \cdot \mathbf{e}_j \cdot {\bm\gamma}^T$ ; \;
\nl \label{ln:a01-14}  \For {$k=0,1,\cdots, K-1$} {
\nl \label{ln:a01-15}   ${{\bm{\xi }}_{k+1}} := C \cdot \mathbf{Q}\cdot {{\bm{\xi }}_{k}} + C \cdot (\mathbf{v}^T\cdot {{\bm{\xi }}_{k}}) \cdot \mathbf{u}$ ; \;
\nl \label{ln:a01-16}   ${{\bm{\eta }}_{k+1}} := \mathbf{Q}\cdot {{\bm{\eta }}_{k}} + (\mathbf{v}^T\cdot {{\bm{\eta }}_{k}}) \cdot \mathbf{u}$ ;\;
\nl \label{ln:a01-17}   ${{\mathbf{M}}_{k+1}} := {{\bm{\xi }}_{k+1}}\cdot \bm{\eta }_{k+1}^{T}+{{\mathbf{M}}_{k}}$ ; \;
}
\nl \label{ln:a01-18} \Return $\tilde{\mathbf{S}} := \mathbf{S} + \mathbf{M}_{K} + \mathbf{M}_{K}^T$ ; \;
%\nl \label{ln:a01-19} $\tilde{\mathbf{S}}$ ; \;
\caption{\IncUSRone~($G, (i,j), \mathbf{S}, K, C$)}  \label{alg:01}
\end{algorithm}

Algorithm~\ref{alg:01} illustrates the pseudo code of \IncUSRone.

Given an old graph $G=(V,E)$, a new edge $(i,j)$ with $i \in V$ and $j \in V$ to be inserted to $G$, the old similarities $\mathbf{S}$ in $G$,
and the damping factor $C$,
{\IncUSRone} incrementally computes $\tilde{\mathbf{S}}$ in $G \cup \{(i,j)\}$ as follows:

First, it initializes the transition matrix $\mathbf{Q}$ and in-degree $d_j$ of node $j$ in $G$ (lines \ref{ln:a01-01}--\ref{ln:a01-02}).
Using $\mathbf{Q}$ and $\mathbf{S}$,
it precomputes the auxiliary vector $\mathbf{w}$ and scalar $\lambda$ (lines \ref{ln:a01-03}--\ref{ln:a01-04}).
Once computed, both $\mathbf{w}$ and $\lambda$ are memoized for precomputing
(i) the vectors $\mathbf{u}$ and $\mathbf{v}$ for a rank-one factorization of $\mathbf{\Delta Q}$, and
(ii) the initial vector ${\bm \gamma}$  for subsequent $\mathbf{M}_{k}$ iterations (lines \ref{ln:a01-06}--\ref{ln:a01-08}).
Then, the algorithm maintains two auxiliary vectors ${\bm{\xi }}_{k}$ and ${\bm{\eta }}_{k}$ to iteratively compute matrix $\mathbf{M}_{k}$ (lines \ref{ln:a01-13}--\ref{ln:a01-17}).
The process continues until the number of iterations reaches a given $K$.
Finally, the new $\tilde{\mathbf{S}}$ is obtained by $\mathbf{M}_{K}$\footnote{We can show ${\|\mathbf{M}_{K}-\mathbf{M}\|}_{\max}\le C^{K+1}$ with $\mathbf{M}$ in Eq.\eqref{eq:29c}.} (line \ref{ln:a01-18}).

\noindent \textbf{Correctness.} \
\IncUSRone~can \emph{correctly} compute new SimRanks for edge update that does not accompany new node insertions,
as verified by Theorems \ref{thm:01}--\ref{thm:03}. \

\noindent \textbf{Complexity.} \
The total complexity of \IncUSRone~is bounded by $O(Kn^2)$ time and $O(n^2)$ memory in the worst case for updating \emph{all} similarities of $n^2$ node-pairs.
Precisely,
\IncUSRone~runs in two phases:
preprocessing (lines \ref{ln:a01-01}--\ref{ln:a01-08}),
and incremental iterations (lines \ref{ln:a01-13}--\ref{ln:a01-18}):

(a) For the preprocessing,
it requires $O(m)$ time in total ($m$ is the number of edges in the old $G$),
which is dominated by computing $\mathbf{w}$ (lines \ref{ln:a01-03}), involving the matrix-vector multiplication $\mathbf{Q}\cdot {{[\mathbf{S}]}_{\star,i}}$.
The time for computing vectors $\mathbf{u}, \mathbf{v}, {\bm \gamma}$ is bounded by $O(n)$,
which includes only vector scaling and additions, \ie SAXPY.

(b) For the incremental iterative phase,
computing ${{\bm{\xi }}_{k+1}}$ and ${{\bm{\eta }}_{k+1}}$ needs $O(m+n)$ time for each iteration (lines \ref{ln:a01-15}--\ref{ln:a01-16}).
Computing ${{\mathbf{M}}_{k+1}}$ entails $O(n^2)$ time for performing one outer product of two vectors and one matrix addition (lines \ref{ln:a01-17}).
Thus, the cost of this phase is $O(Kn^2)$ time for $K$ iterations.

Collecting (a) and (b), all $n^2$ node-pair similarities can be incrementally computed in $O(Kn^2)$ total time.
%as opposed to the $O(r^4n^2)$ time of its counterpart \cite{Li2010} via incremental SVD.
%
\subsection{{\IncSR} Algorithm with Pruning} \label{app:04b}
%
\begin{algorithm}[t]
\small
\DontPrintSemicolon
%\SetCommentSty{textsf}
\SetKwInOut{Input}{Input / Output}
%\SetKwInOut{Output}{Output}
%\SetKwFunction{Len}{Len}
\Input{the same as Algorithm~\ref{alg:01}.}
%\Output{the new similarities ${\mathbf{\tilde{S}}}$ for $G \cup\{(i,j)\}$.}
\lnlset{ln:a02-01}{1-2} the same as Algorithm~\ref{alg:01} ;\;
\lnlset{ln:a02-02}{3} find ${{\mathsf{\mathcal{B}}}_{0}}$ via Eq.\eqref{eq:41} ; \;
\lnlset{ln:a02-03}{} memoize ${[\mathbf{w}]}_{b} := {[\mathbf{Q}]}_{b,\star}\cdot {{[\mathbf{S}]}_{\star,i}}$,  for all $b \in {{\mathsf{\mathcal{B}}}_{0}}$ ; \;
\lnlset{ln:a02-04}{4-12} almost the same as Algorithm~\ref{alg:01} except that the computations of the entire vector $\bm \gamma$ in Lines $6,8,10,12$ are replaced by the computations of only parts of entries in $\bm \gamma$, respectively,
\eg  in Line~6 of Algorithm~\ref{alg:01}, ``${\bm \gamma}:=\mathbf{w}+\tfrac{1}{2}{[\mathbf{S}]}_{i,i}\cdot \mathbf{e}_j$''
are replaced by ``${[{\bm \gamma}]}_{b}:={[\mathbf{w}]}_{b}+\tfrac{1}{2}{[\mathbf{S}]}_{i,i}\cdot {[\mathbf{e}_j]}_{b}$, for all $b \in {\cal B}_{0}$'' ; \;
\lnlset{ln:a02-13}{13} ${[{{\bm{\xi }}_{0}}]}_j := C ,\ {[{\bm{\eta }}_{0}]}_{b} := {[{\bm\gamma}]}_{b}, \ {[{\mathbf{M}}_{0}]}_{j,b} := C \cdot {[{\bm\gamma}]}_{b}, \forall b \in {\cal B}_0$;\;
\lnlset{ln:a02-14}{14} \For {$k=1,\cdots, K$} {
\lnlset{ln:a02-15}{15} find ${{\mathsf{\mathcal{A}}}_{k}}\times {{\mathsf{\mathcal{B}}}_{k}}$ via Eq.\eqref{eq:41} ; \;
\lnlset{ln:a02-16}{16}  memoize $\sigma_1 := C \cdot (\mathbf{v}^T\cdot {{\bm{\xi }}_{k-1}}), \ \sigma_2 :=\mathbf{v}^T\cdot {{\bm{\eta }}_{k-1}} $ ; \;
\lnlset{ln:a02-17}{17} ${[{\bm{\xi }}_{k}]}_{a} := C \cdot {[\mathbf{Q}]}_{a, \star}\cdot {{\bm{\xi }}_{k-1}} +  \sigma_1 \cdot {[\mathbf{u}]}_{a}$, for all $a \in {\cal A}_{k}$ ; \;
\lnlset{ln:a02-18}{18} ${[{\bm{\eta }}_{k}]}_{b} := {[\mathbf{Q}]}_{b,\star}\cdot {{\bm{\eta }}_{k-1}} + \sigma_2 \cdot {[\mathbf{u}]}_{b}$, for all $b \in {\cal B}_{k}$ ;\;
\lnlset{ln:a02-19}{19} \scalebox{0.92}{${[{\mathbf{M}}_{k}]}_{a,b} := {[{\bm{\xi }}_{k}]}_{a}\cdot {[\bm{\eta }_{k}]}_{b}+{[{\mathbf{M}}_{k-1}]}_{a,b}, \ \forall (a,b) \in {{\mathsf{\mathcal{A}}}_{k}}\times {{\mathsf{\mathcal{B}}}_{k}}$;}\;
}
\lnlset{ln:a02-20}{20} \scalebox{0.95}{${[\tilde{\mathbf{S}}]}_{a,b} := {[\mathbf{S}]}_{a,b} + {[\mathbf{M}_{K}]}_{a,b} + {[\mathbf{M}_{K}]}_{b,a}, \ \forall (a,b) \in {{\mathsf{\mathcal{A}}}_{K}}\times {{\mathsf{\mathcal{B}}}_{K}}$;}\;
\lnlset{ln:a02-21}{21} \Return $\tilde{\mathbf{S}}$ ; \;
\caption{\IncSR~($G, \mathbf{S}, K, (i,j), C$)}  \label{alg:02}
\end{algorithm}

Algorithm~\ref{alg:02} illustrates the pseudo code of {\IncSR}.

\noindent \textbf{Correctness.} \
\IncSR~can \emph{correctly} prune the node-pairs with a-priori zero scores in $\mathbf{\Delta S}$,
which is verified by Theorem~\ref{thm:04}.
It also \emph{correctly} returns the new similarities,
as evidenced by Theorems \ref{thm:01}--\ref{thm:03}.

\noindent \textbf{Complexity.} \
The total time of \IncSR~is $O(K(m+|\AFF|))$ for $K$ iterations,
where $|\AFF|:= \textrm{avg}_{k \in[0,K]} ( |{\cal A}_k| \cdot |{\cal B}_k|)$
with ${\cal A}_k, {\cal B}_k$ in Eq.\eqref{eq:41},
being the average size of ``affected areas'' in $\mathbf{M}_k$ for $K$ iterations.
More concretely, (a) for the preprocessing,
finding ${{\mathsf{\mathcal{B}}}_{0}}$ (line~\ref{ln:a02-02}) needs $O(dn)$ time.
Utilizing ${{\mathsf{\mathcal{B}}}_{0}}$,
computing ${[\mathbf{w}]}_{b}$ reduces from $O(m)$ to $O(d|{\cal B}_0|)$ time,
with $|{\cal B}_0| \ll n$.
Analogously, $\bm \gamma$ in lines 6,8,10,12 of Algorithm~\ref{alg:01} needs only $O(|{\cal B}_0|)$ time.
(b) For each iteration,
finding ${\cal A}_k \times {{\mathsf{\mathcal{B}}}_{k}}$ (line \ref{ln:a02-15}) entails $O(dn)$ time.
Memoizing $\sigma_1, \sigma_2$ needs $O(n)$ time (line \ref{ln:a02-16}).
Computing ${\bm \xi}$ (\Resp $\bm \eta$) reduces from $O(m)$ to $O(d|{\cal A}_k|)$ (\Resp $O(d|{\cal B}_k|)$) time (lines \ref{ln:a02-17}--\ref{ln:a02-18}).
Computing ${[{\mathbf{M}}_{k}]}_{a,b}$ reduces from $O(n^2)$ to $O(|{\cal A}_k||{\cal B}_k|)$ time (line \ref{ln:a02-19}).
Thus,
the total time complexity can be bounded by $O(K(m+|\AFF|))$ for $K$ iterations.

It is worth mentioning that \IncSR, in the worst case, has the same complexity bound as \IncUSR.
However, in practice, $|\AFF| \ll n^2$, as demonstrated by our experimental study in Fig.\ref{fig:exp_08}.
%since real graphs are constantly updated with \emph{small} changes.
%Hence, $O(K(m+|\AFF|))$ is generally much smaller than $O(Kn^2)$.
%In the next section, we shall further confirm the efficiency of \IncSR~by conducting extensive experiments.

%
\section{Description of Real Datasets} \label{app:05}
%
The description of the real datasets is as follows:

(1) \underline{\DBLP}\footnote{http://dblp.uni-trier.de/\~{}ley/db/} is a co-citation graph,
where each node is a paper with attributes (\eg publication year), and edges are citations.
%%The dataset has 146K nodes and 927K edges.
By virtue of the publication year,
we extract several snapshots. %, each consisting of 93,560 edges and 13,634 nodes.

(2) \underline{\CITH}\footnote{http://snap.stanford.edu/data/} is a reference network (cit-HepPh) from e-Arxiv.
If a paper $u$ references $v$,
there is a link $u \to v$.
%The dataset has 421,578 edges and 34,546 nodes.

(3) \underline{\YOUTU}\footnote{http://netsg.cs.sfu.ca/youtubedata/} is a YouTube network,  %each node is a video.
where a video $u$ (node) is linked to $v$ if $v$ is in the relevant video list of $u$.
We extract snapshots based on the age of videos.

(4) \underline{{\WEBB}} is a Berkeley-Stanford web graph, where nodes are pages from \textsf{berkely.edu} and \textsf{stanford.edu} domains,
and edges are hyperlinks.

(5) \underline{{\WEBG}} is a Google web graph, where nodes are web pages, and edges are hyperlinks.

(6) \underline{{\CITP}} is a patent citation network among US, where a node is a granted patent, and a link a citation.
%and each has 953,534 edges and 178,470 nodes.

(7) \underline{{\SOCL}} is a LiveJournal friendship social network, where a node is a user, and a link denotes friendship.

(8) \underline{{\UK}}\footnote{http://law.di.unimi.it/datasets.php} is a web graph obtained from a 2005 crawl of the \textsf{.uk} domain, where an edge is a link from one web page (node) to another.

(9) \underline{{\IT}} is a web graph crawled from the \textsf{.it} domain, where an edge is a link from a page (node) to another.

\end{appendices}
\end{document}
% end of file template.tex

