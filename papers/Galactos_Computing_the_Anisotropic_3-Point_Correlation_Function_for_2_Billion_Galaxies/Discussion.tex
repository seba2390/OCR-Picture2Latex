\section{Discussion}

\subsection{Accelerating analysis of current and future sky surveys}

The enormous speed of Galactos offers the opportunity to substantially accelerate all analysis steps for an anisotropic 3PCF measurement. In addition to the 3PCF of the data, a full 3PCF analysis demands running the algorithm on hundreds to thousands of catalogs with spatially random clustering as well as similar numbers of mock, simulated-galaxy-data catalogs. 

Astronomical surveys of the sky have many blind spots. For example, they cannot see through the dense center of the Milky Way, or identify galaxies behind the glare of a bright star. Further, the distance to which they can observe galaxies varies over the sky due to atmospheric fluctuations and instrumental effects. The survey geometry therefore ends up very different from the perfect cube of simulated data used in this work. Consequently, in addition to computing the 3PCF of the data, one must also compute the 3PCF of hundreds of full-survey-size random catalogs, which Monte-Carlo sample the complicated survey geometry. These random catalog results enable the removal of spurious signal generated by the survey geometry. This correction step is key for extracting constraints on cosmological parameters from the data. 

Full survey-size mock catalogs, generated from simulations such as Outer Rim, are equally important for a full 3PCF analysis. First, they enable end to end verification of the analysis pipeline for systematic errors -- a simulation with known input parameters can be analyzed to check that the output parameters from fitting a 3PCF model faithfully recover the inputs. Second, the mocks can be used to test whether the theoretical models fully capture the underlying physics we believe affects the 3PCF. These models' goodness of fit to a large sample of mocks (necessary to reduce random scatter) helps determine which models should be fit to the data, and on what physical scales different models apply.

Further, a large number of mocks are often required to estimate the 3PCF's covariance matrix, which describes how independent the measurements of different triangles are from each other. This matrix needs to be inverted to optimally weight the data when fitting a model to it, and the inverse can be highly sensitive to random scatter introduced if one does not use a large number of mocks. Indeed, modern redshift surveys spend most of their compute time calculating clustering statistics on an enormous number of mocks to derive smoothly invertible covariance matrices. While other approaches to the covariance do exist \cite{SE3ptalg}, this is a standard technique that Galactos will accelerate tremendously for the 3PCF. 

An alternative estimation technique that Galactos also enables is the use of jack-knifing to estimate the small-scale covariance matrix for the 3PCF. Partitioning the survey spatially to parallelize over many nodes amounts to jack-knifing: retaining the local 3PCF results on a per node basis would therefore constitute many samples of the 3PCF over small volumes. These can be combined to provide a covariance matrix. 

\subsection{Upcoming datasets and possible science return with Galactos}

Current astronomical datasets already consist of roughly 1 million galaxies, a problem tractable in seconds on Cori with Galactos. In the next decade, these datasets will expand by orders of magnitude with surveys such as DESI, LSST, Euclid, and WFIRST. Galactos' speed will be essential for enabling the full scientific return from these datasets, and given the need for random and mock catalogs as well. 

A full description of the likely science returns of using the anisotropic 3PCF via Galactos with these enormous datasets is beyond the scope of this work, but to give a sense of what can be expected we quote some rough estimates based on our Outer Rim dataset. A survey with similar properties to Outer Rim would likely increase the precision on cosmological constraints relative to BOSS, the present state-of-the art galaxy survey, by a factor of roughly 100.  This would help cosmologists close in on the true nature of dark energy, be an unprecedentedly precise probe of a complete theory of gravity, and offer a stringent test of our best current models of galaxy formation.


\subsection{Prospects outside cosmology}
We stress that that the anisotropic 3PCF framework applies beyond just cosmological datasets. The core algorithm can be applied to any point set and can also be generalized to gridded data, enabling further acceleration. One very different astronomical application of the 3PCF is studying the properties of the Inter Stellar Medium (ISM), birthplace of stars. 
Physical conditions in the ISM such as magnetic fields, turbulence, and shocks move around dust that can be observed, and it has been shown that the 3PCF's Fourier-space analog (bispectrum) is therefore a sensitive probe of these conditions \cite{Burkhart09}. Galactos will be particularly enabling because the large parameter space of different physical conditions means one needs to compute the 3PCF of many simulations.

Any application that has translation-invariant clustering of a point set in a known but possibly irregular volume could consider these methods as a way to expand from a two-point to a three-point analysis, and many physical situations will generate non-Gaussian randomness in which the 3PCF contains new information.  Simple alterations to the algorithm  enabling use with 2-D data (e.g. generalizing \cite{SE3ptalg}) or in more than 3-D are also possible (e.g. using the higher-dimensional spherical harmonics of \cite{Frye2012}).
