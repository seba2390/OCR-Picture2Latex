\section{Overview}
\label{sec:overview}
\subsection{Probing dark energy, gravity, and galaxy formation}
Measuring the anisotropic 3PCF can illuminate two of the most significant challenges in present-day  cosmology: dark energy's fundamental nature, and the complete theory of gravity. Dark energy drives accelerated expansion of the Universe. It constitutes 72\% of the Universe's current energy density, but its fundamental nature remains unknown. Unlike all other known substances, dark energy has negative pressure, and it is not predicted by the Standard Model of particle physics.

An alternative explanation for the accelerated expansion of the Universe is that the most widely accepted theory of gravity, General Relativity (GR), requires modification. Indeed, we already know that GR must be only an approximation to the true theory of gravity because GR cannot be unified with quantum field theory, which governs the interactions of the fundamental particles \cite{Copeland}.

The clustering of galaxies throughout space offers the largest possible laboratory  for pursuing these fundamental questions about the Universe's contents and governing laws. Galaxy clustering is often quantified via correlation functions. These measure the excess of pairs of galaxies as a function of the distance between the galaxies (2-point correlation function, 2PCF) or the excess of triplets as a function of triangle configuration (3PCF) compared to a random distribution \cite{Peebles}.

The 2PCF has been a highly successful tool in the past for exploring both dark energy and gravity.  In particular, 
the Baryon Acoustic Oscillation (BAO) method uses a sharp feature in the 2PCF  as a ``standard ruler'' \cite{Eisenstein1998}. Locating this feature in the 2PCF of galaxy samples from different epochs in the Universe's history enables the mapping of the Universe's expansion over time,  in turn illuminating the nature of dark energy \cite{Weinberg,Alam2016}.

Further, the growth rate of structure can be probed using the { \it{anisotropic}} (direction-dependent) 2PCF. This tracks the excess pairs of galaxies compared to a random distribution as a function of both the separation between the galaxies and the angle between the separation vector and the line of sight to the galaxy pair. 

While the underlying clustering of galaxies is independent of angle to the line of sight, the clustering {\it observed} in a redshift survey is modulated because of Redshift Space Distortions (RSD)~\cite{HamiltonRSD}. RSD occur because galaxies' own (``peculiar'') velocities with respect to the background expansion of the Universe affect our inference of their positions along the line of sight from their redshifts.  

Galaxies' velocities are generated by growth of structure due to gravity, and GR makes a particular prediction for the growth rate. Any measured deviation would offer a vital clue to the modifications required for a complete theory of gravity \cite{Lindergrowth}.

\subsection{Unlocking the anisotropic 3PCF's potential}
In principle, the 3PCF offers a new lever to understand both dark energy and gravity.  
\cite{Sefasutti2006} demonstrated that adding 3PCF information to an analysis of the 2PCF can improve constraints on the cosmological parameters that describe the nature of the universe. 
Measurements of the amount of dark energy growth and the rate of structure (which constrains gravity) can be improved by up to a factor of 2, and biasing, which describes how galaxies trace the underlying matter and is vital for understanding galaxy formation \cite{Fry1994,SERV,GilMarin2}, can be improved by a factor of $~$2 compared to using the 2PCF alone.
Like the 2PCF, the 3PCF has BAO features that can be used as a standard ruler to trace the expansion history \cite{SERSDmodel,SEBaoDetxn}.  And similarly to the anisotropic 2PCF, the anisotropic 3PCF contains valuable information on the growth rate. 

The anisotropic 3PCF depends on two triangle sides, their enclosed angle, and the angle of each side to the line of sight to the galaxy triplet (see Figure \ref{fig:alg_plus_eqn}). Consequently it has much richer structure than the anisotropic 2PCF, and offers many different configurations that all ultimately probe gravity \cite{RampfWong,Scoccimarro} as well as dark energy. 
It has never been measured. Not only would it provide additional information compared to the anisotropic 2PCF, but it would break significant degeneracies between the growth rate and other cosmological parameters that cannot be broken by the anisotropic 2PCF alone.

\begin{figure}
\centering
\includegraphics[width=\columnwidth]{Figures/updated_alg_fig_b.png}
\caption{Representation of a triangle configuration for the 3PCF. Each dot represents a galaxy. The anisotropic 3PCF depends on the vectors $\vec{r}_1$ and $\vec{r}_2$, the relative distances to the primary galaxy, which is at the bottom left vertex. The relevant quantities are the triangle side lengths $r_1$ and $r_2$, the angle between $\vec{r}_1$ and $\vec{r}_2$, and $\vec{r}_1$ and $\vec{r}_2$'s angles to the line of sight (dashed arrow). We expand the angular dependence of the anisotropic 3PCF in the basis of spherical harmonics, with the dependence on triangle side lengths $r_1$ and $r_2$ encoded in the radial coefficient $\zeta^m_{\ell \ell'}(r_1,r_2)$. The panel on the right (taken from~\cite{SE3ptalg}) shows a schematic of the algorithm's output: a coefficient $\zeta_{\ell \ell'}^m$ as a function of the triangle side lengths $r_1$ and $r_2$. The color indicates the number of triangles; red is an excess over a spatially random distribution and blue a deficit. The features are from BAO. 
\label{fig:alg_plus_eqn}}
\end{figure}


Finally, the 3PCF is a highly sensitive measure of how galaxies form. Galaxies are termed a ``biased'' tracer of the underlying dark matter density: they do not follow it with perfect fidelity, but rather form based on local conditions, such as the value of the local gravitational tidal forces or the relative velocity between regular matter and dark matter \cite{McDonald2009, TH2010}. 

 These effects enter at sub-leading order in the 2PCF, meaning one is searching for a small change to a large baseline signal. However, in the 3PCF, these effects are at leading order, making the 3PCF a uniquely powerful probe of galaxy biasing \cite{SERV,SERSDmodel}. 

 
\subsection{The 3PCF is computationally demanding}
Unfortunately, in practice, the 3PCF has rarely been used to constrain models of cosmology: counting all possible triangles formed by galaxy triplets in a modern redshift survey is combinatorially explosive. It naively scales as $\mathcal{O}(N^3)$ compared to $\mathcal{O}(N^2)$ scaling for the 2PCF, where $N$ is the number of galaxies in the survey. The 3PCF and bispectrum (its Fourier-space analog) have had other uses, but given this combinatoric challenge have been restricted to only particular triangle configurations (e.g. isosceles) \cite{McBride, GilMarin1,GilMarin2}.  
Recent work with an algorithm \cite{SE3ptalg} related to the one used in this work has measured many triangle configurations for the spherically-averaged (isotropic) 3PCF \cite{SEBaoDetxn,SERV}, but the full anisotropic dependence of the 3PCF has never been measured. We note that the anisotropic 3PCF contains the isotropic 3PCF as a proper subset. 


Until now no algorithm has existed to measure the full anisotropic 3PCF for all triangles, on large scales, and in million- to billion-galaxy surveys. In this work, we develop and implement an algorithm that measures the anisotropic 3PCF in $\mathcal{O}(N^2)$ time. Galactos, our high-performance parallel computing anisotropic 3PCF code will, for the first time, make the measurement of the 3PCF feasible for modern astronomical surveys. The mathematical framework of the algorithm and other details relevant for cosmology are presented more fully in a companion paper ~\cite{SE3ptAniso}.
Galactos' single node performance has been highly optimized for Intel Xeon Phi, achieving 39\%\ of peak single-node performance with efficient use of vectorization and the full memory hierarchy. 
Galactos presents almost perfect weak- and strong-scaling, and achieves 5.06~PF across 9636 nodes.

The only required input is the 3-D positions of the galaxies, which is already demanded by the 2PCF.  Thus, for zero additional cost in telescope time, our algorithmic and computing advances will yield significant additional insight on the most fundamental questions facing cosmology.










