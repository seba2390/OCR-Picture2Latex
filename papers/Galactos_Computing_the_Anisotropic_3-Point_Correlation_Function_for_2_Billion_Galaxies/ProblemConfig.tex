\section{Problem Configuration}

In this section we describe the dataset configurations that were used to perform the weak- and strong-scaling measurements, as well as the Cori system itself on which these computations were performed.

\subsection{Description of Cori system}
\label{sec:cori}

The Galactos code was run on the Cori system at the National Energy Research Scientific Computing Center (NERSC) at Lawrence Berkeley National Laboratory.
Cori is a Cray XC40 system featuring 2,388 nodes of Intel Xeon Processor E5-2698 v3 (``Haswell'') and 9,688 nodes (recently expanded from 9,304) of Intel Xeon Phi Processor 7250 (``Knights Landing'').
All computations presented here were performed on Xeon Phi nodes.
Each of these nodes contains \num{68} cores (each supporting \num{4} simultaneous hardware threads), \num{16}~\si{\giga\byte} of on-package, multi-channel DRAM (``MCDRAM''), and \num{96}~\si{\giga\byte} of DDR4-2400 DRAM.
Cores are connected in a 2D mesh network with 2 cores per tile, and 1 MB cache-coherent L2 cache per tile. 
Each core has 32 KB instruction and 32 KB data in L1 cache. 
The nodes are connected via the Cray Aries interconnect.
In all Galactos computations presented here, we compiled the code using the Intel C++ v17.0.1 compiler with Cray MPI, and ran the code with 1 MPI process per Xeon Phi compute node, using 272 threads per node (4 threads per physical core).
Because the arithmetic intensity of our application is high, its memory bandwidth needs are modest; our measurements show that running the code with the Xeon Phi's high-bandwidth MCDRAM configured in ``flat'' mode (exposed as a separate NUMA domain from the DRAM) yields nearly identical performance as when the MCDRAM is configured in ``cache mode'' (configured as a transparent, direct-mapped cache to the DRAM).
As a result, all measurements reported in this work were performed with the MCDRAM in ``cache'' mode.




\subsection{Simulation data}
\label{sec:outer-rim}
We ran the Galactos code over one of the largest cosmological simulation datasets available---the Outer Rim simulation \cite{Habib2016}. 
This simulation used over a trillion particles of dark matter, each of roughly $1.85~h^{-1}M_{\text{sun}}$, contained in a box of $3000~\text{Mpc}/h$ on each side. 
This distance corresponds to roughly 9.8 billion light-years, or $1/5$ the distance to the edge of the observable universe. 
The simulation was evolved over time, allowing structures to form in the distribution of the dark matter particles through gravitational attraction, and expansion to occur according to GR and dark energy with cosmological parameters close to the current concordance values \cite{WMAP7}.
We used a snapshot of the simulation at redshift $z=0$, i.e. the present day. The dark matter particles were grouped into gravitationally-bound "halos", which we take to represent galaxies (though the larger halos may actually host more than one galaxy). 
We used these $\sim\!2$ billion galaxies for our 3PCF analysis. 
If we partition these galaxies evenly between compute nodes on Cori (making a conservative estimate of 9000 nodes available) then each compute node is assigned 225,000 galaxies, visualized in Figure~\ref{fig:outer-rim}. 
To make partitions for our weak scaling tests, we selected cubes within Outer Rim  such that each cube enclosed the appropriate number of galaxies scaled down from the full 2 billion, as shown in Table~\ref{tab:weak_scaling_problem_sets}. 






\begin{table}
\centering
\begin{tabular}{|c|c|c|}
\hline
    \# of nodes & \# of galaxies & cubic box length ($\text{Mpc}/h$)\\
\hline
\num{128} & \num{2.880e7} & \num{734.5}\\
\hline
\num{256} & \num{5.760e7} & \num{925.8}\\
\hline
\num{512} & \num{1.152e8} & \num{1166.9}\\
\hline
\num{1024} & \num{2.304e8} & \num{1470.9}\\
\hline
\num{2048} & \num{4.608e8} & \num{1853.3}\\
\hline
\num{4096} & \num{9.216e8} & \num{2334.7}\\
\hline
\num{8192} & \num{1.843e9} & \num{2934.4}\\
\hline
\num{9636} & \num{1.951e9} & \num{3000.0}\\
\hline
\end{tabular}
\caption{Datasets for weak scaling tests and full system run.}

\label{tab:weak_scaling_problem_sets}
\end{table}


\begin{figure}
\centering
\includegraphics[width=\columnwidth]{Figures/225k-viz.pdf}
\caption{Visualization of a box containing 225,000 galaxies in the Outer Rim simulation. 
\label{fig:outer-rim}}
\end{figure}
