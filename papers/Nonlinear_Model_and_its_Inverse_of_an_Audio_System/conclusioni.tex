\chapter{Conclusion and future development}
As seen in the results presented in the previous chapter, exploiting the Volterra series can correct a large amount of distortion produced by loudspeaker. In the last years there has been a renewed interest in the commercial and scientific context about the implementation of these filters; a lot of results presented in other papers consider only the first two orders of the series.\\
In this thesis a systematic set of tests has been made to find the best parameters and training methods, in order to estimate the Volterra kernel with high quality results on different types of signal. All tests have been performed considering the first three orders of the series, therefore we provided a practical contribution to the adaptive nonlinear filters based on this series.\\\\
All scientific papers found about Volterra filters for loudspeaker signal correction are based on signals recorded with a microphone, often subject to non negligible noise. In this project, all filters exploited signals collected with an interferometer from the Zurich University of Applied Sciences. This method is very accurate accurate and the results can be considered more reliable than those collected with a microphone.\\\\
We have written a paper summarizing the results of the loudspeaker model, it must be submitted to the journal (\textit{Nonlinear Volterra model of a loudspeaker behavior based on interferometry measurements. Alessandro Loriga, Elizabeth Dumont}).\\\\
A set of systematic tests has been performed in order to ensure a complete vision of the model requirements and parameters. The evaluation has been made for different types of signals (monochromatic, superimposed ...) and the behavior of the filters are clear in several situations.\\\\
All algorithms have been implemented in a C++ library, composed of two main parts: \textit{Handler} and \textit{Estimator}. \\
The estimation algorithm can be called from other methods allocating a Handler instance. Different implementations of the estimation process can be written with a derived class that inherits from the Estimator class. This structure allows the library to be easily expanded with new features.\\\\
Future developments of this project will be made by \textit{Florence Technologies s.r.l} and \textit{Intranet Standard GmbH}, the aim of which consists of a filter implementation for dedicated loudspeakers; Zurich University of Applied Sciences will carry out further measurement phases. In the following paragraphs we will introduce the main steps of the project.\\\\
A good loudspeaker system is composed of a set of different membranes, as shown in fig. \ref{fig:loudspekerscheme}. In this project we focused on a sub-woofer, a similar study can be made for woofers of mid-ranges.\\\\
\begin{figure}[h]\centering
\includegraphics[scale=.2]{img/loudspeaker-scheme.png} 
\caption{Loudspeaker system with different membranes, each one has a dedicated range of frequencies. 1 - Midrange (250Hz to 2 KHz), 2 - Tweeter (2KHz to 20KHz), 3 - Woofers (20Hz to 250Hz)} \label{fig:loudspekerscheme}
\end{figure}
The changing of membrane characteristics during the loudspeaker life-cycle can not be neglected: elasticity, movement capabilities and system resonance do not remain the same due to loudspeaker usage, therefore the filters can not be considered reliable after a certain period.\\
Our idea is to provide a feedback channel obtained with a good quality microphone. This channel is a standard method in control systems and it is described in the fig. \ref{fig:feedbackchannel}; this is the same method presented in the Active Nose Control. \\
\begin{figure}[h]\centering
\includegraphics[scale=.4]{img/feedback.png} 
\caption{Feedback control}\label{fig:feedbackchannel}
\end{figure}
We think that this periodical calibration will not require a large number of iterations because the initial kernel will be a good estimation of the searched kernel. This implementation step has to pass for the optimization of kernel estimation algorithm.\\\\ 
A code optimization must be performed for the filter application, as each filter has a natural delay due to the computational time required. In an audio context, it should be very small, but in live music context this is a crucial point. It is impossible play an instrument with a system that introduces a delay greater than a certain threshold.\\
In all likelihood the entire computation system will be implemented in a GPU-accelerated embedded system. In these systems the GPU is integrated in the main board, reducing the transfer latency between the CPU memory and the device memory.\\\\
The last step will consider a method to extend the filters from a single loudspeaker to multiple devices. This step creates several communication problems between the various parts that are essential for the synchronization.\\
The communication between the components will probably be implemented with Ethernet or wi-fi, because several low latency synchronized protocols exist for these connections.