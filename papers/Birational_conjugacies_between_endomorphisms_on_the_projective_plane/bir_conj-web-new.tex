\documentclass[reqno,oneside,12pt]{amsart}

%%%%%%%%%%%%%%%%%%%%%%%%%%%%%%%%%%%%%%%%%%%%%%%%%%%%%%%%%%%%%%%%%%
%  Texfile for the article : Geometric Bogomolov Conjecture
%%%%%%%%%%%%%%%%%%%%%%%%%%%%%%%%%%%%%%%%%%%%%%%%%%%%%%%%%%%%%%%%%%
%
\usepackage[T1]{fontenc}
\usepackage{times,mathptm}
\usepackage{amssymb,epsfig,verbatim,xypic}
%
%\usepackage{ulem} %% for \sout, remove in final version
%
%%%%%%%%%%%%%%%%%%%%%%%%%%%%%%%%%%%%%%%%%%%%%%%%%%%%%%%%%%%%%%%%%%
%

\def\vs{\vspace{0.1cm}}


\theoremstyle{plain}

\newtheorem{thm}{Theorem}%[section]

\newtheorem{cor}[thm]{Corollary}
\newtheorem{pro}[thm]{Proposition}
\newtheorem{lem}[thm]{Lemma}
\newtheorem{proposition-principale}[thm]{Proposition principale}
\newtheorem{thm-principal}{Th\'eor\`eme principal}[section]

\theoremstyle{definition}
\newtheorem{defi}[thm]{Definition}
\newtheorem{que}{Question}[section]
\newtheorem{eg}[thm]{Example}
\newtheorem{rem}[thm]{Remark}


\newenvironment{congb}
{{\vs \noindent \bf Geometric Bogomolov Conjecture.--$\,$}\it}{\vs}


\newenvironment{thm-A}
{{\vs \noindent \bf Theorem A.$\,$}\it}{\vs}


\newenvironment{thm-B}
{{\vs \noindent \bf Theorem B.$\,$}\it}{\vs}


\newenvironment{thm-BB}
{{\vs \noindent \bf Theorem B'.$\,$}\it}{\vs}


\interfootnotelinepenalty=10000

%%%%%%%%%%%%%%%%%%%%%%%%%%%%%%%%%%%%%%%%%%%%%
\def\vv{\vspace{0.2cm}}

%%%%%%%%%%%%%%%%%%%%%%%%%%%%%%%%%%
%   Lettres
%%%%%%%%%%%%%%%%%%%%%%%%%%%%%
\def\C{\mathbf{C}}
\def\R{\mathbf{R}}
\def\Q{\mathbf{Q}}
\def\Z{\mathbf{Z}}
\def\N{\mathbf{N}}

\def\bfk{{\mathbf{k}}}
\def\bfF{{\mathbf{F}}}



\def\J{{\textsc{j}}}

\newcommand{\bbA}{{\mathbb{A}}}

\newcommand{\Spec}{{\mathrm{Spec}}}


\newcommand{\sA}{{\mathcal A}}
\newcommand{\sB}{{\mathcal B}}
\newcommand{\sC}{{\mathcal C}}
\newcommand{\sD}{{\mathcal D}}
\newcommand{\sE}{{\mathcal E}}
\newcommand{\sF}{{\mathcal F}}
\newcommand{\sG}{{\mathcal G}}
\newcommand{\sH}{{\mathcal H}}
\newcommand{\sI}{{\mathcal I}}
\newcommand{\sJ}{{\mathcal J}}
\newcommand{\sK}{{\mathcal K}}
\newcommand{\sL}{{\mathcal L}}
\newcommand{\sM}{{\mathcal M}}
\newcommand{\sN}{{\mathcal N}}
\newcommand{\sO}{{\mathcal O}}
\newcommand{\sP}{{\mathcal P}}
\newcommand{\sQ}{{\mathcal Q}}
\newcommand{\sR}{{\mathcal R}}
\newcommand{\sS}{{\mathcal S}}
\newcommand{\sT}{{\mathcal T}}
\newcommand{\sU}{{\mathcal U}}
\newcommand{\sV}{{\mathcal V}}
\newcommand{\sW}{{\mathcal W}}
\newcommand{\sX}{{\mathcal X}}
\newcommand{\sY}{{\mathcal Y}}
\newcommand{\sZ}{{\mathcal Z}}



%%%%%%%%%%%%%%%%%%%%%%%%%%%%%%%%%%%%%%%%%%

\newcommand{\id}{{\rm id}}


\def\P{\mathbb{P}}
\def\Sphere{\mathbb{S}}
\def\disk{\mathbb{D}}
 
\def\M{\mathcal{M}}

\def\ess{\mathrm{ess}}

%\def\da{{\dasharrow}}


%%%%%%%%%%%%%%%%%%%%%%%%%%%%%%%%%%%%%%%%%%


%\def\disc{\mathbb{D}}
%\def\H{\mathbb{H}}
%\def\B{\mathbb{B}}

%\def\dev{{\mathbf{dev}}}
%\def\Lin{{\mathbf{Lin}}}
%\def\Af{{\mathbf{Aff}}\,}

%\def\A{\mathcal{A}}
%\def\Out{{\sf{Out}}}

%\def\Mon{{\sf{Mon}}}


\def\Orb{{\text{Orb}}}
%%%%%%%%%%%%%%%%%%%%%%%%%%%%%%%%%%%%%%%%%%%%%%%%%%%%%%%%%%%%%%%%%
%   Abbrev.
%%%%%%%%%%%%%%%%%%%%%%%%%%%%%%%%%%%%%%%%%%%%%%%%%%%%%%%%%%%%%%%%%
\def\DF{{\sf{DF}}}
\def\Int{{\sf{Int}}}

\def\MCG{{\sf{MCG}}}
\def\Aut{{\sf{Aut}}}
\def\Bir{{\sf{Bir}}}

\def\PGL{{\sf{PGL}}}
\def\PSL{{\sf{PSL}}}
\def\O{{\sf{O}}}
\def\End{{\sf{End}}}
\def\GL{{\sf{GL}}}
\def\SL{{\sf{SL}}}
\def\Aff{{\sf{Aff}}}
\def\SU{{\sf{SU}}}

\def\aa{{\mathfrak{a}}}
\def\hh{{\mathfrak{h}}}
\def\vol{{\sf{vol}}}
\def\dl{{\sf{dl}}}


\def\tr{{\sf{tr}}}
\def\Tr{{\sf{Tr}}}

\def\Pic{{\mathrm{Pic}}}


\def\diam{{\text{diam}}}

\def\Dom{{\text{Dom}}}
\def\Ind{{\text{Ind}}}
\def\Exc{{\text{Exc}}}
\def\Crit{{\text{Crit}}}
\def\Ram{{\text{Ram}}}

\def\dist{{\sf{dist}}}

\def\gro{{\sf{gro}}}
\def\Gro{{\sf{Gro}}}
\def\fieldchar{{\rm{char}\,}}

%
%%%%%%%%%%%%%%%%%%%%%%%%%%%%%%%%%%%%%%%%%%%%%%%%%%%%%%%%%%%%%%%%%%
%


\setlength{\textwidth}{13.2cm}                       %{12.8cm}
\setlength{\textheight}{21.2cm}                     %{20.5cm}{20.8cm}
\setlength{\topmargin}{0.15cm}                     %{0.2cm}
\setlength{\headheight}{0.7cm}                     %{0.8cm}
\setlength{\headsep}{0.7cm}                         %{0.8cm}
\setlength{\oddsidemargin}{1.6cm}                %{1.2cm}  
\setlength{\evensidemargin}{1.6cm}              %{1.2cm}

%%%%%%%%%%%%%%%%%%%%%%%%%%%%%%%%%%%%%%%%%%%%%%%%%%%%%%%%%%%%%%%%%%
%%%%%%%%%%%%%%%%%%%%%%%%%%%%%%%%%%%%%%%%%%%%%%%%%%%%%%%%%%%%%%%%%%
%
\addtocounter{section}{0}             % Start with section 1
%\numberwithin{equation}{section}       % Number formulas within sections
%%%%%%%%%%%%%%%%%%%%%%


\usepackage{color}
\newcommand{\Smodif}{\textcolor{blue}{(SC)}}
\newcommand{\Scomment}[1]{\textcolor{blue}{(SC)}\marginpar{\tiny\textcolor{blue}{#1}}}
\newcommand{\Pcomment}[1]{\textcolor{red}{(PH)}\marginpar{\tiny\textcolor{red}{#1}}}
\newcommand{\Vcommenti}[1]{{\textcolor{red}{#1 ---VG}}}
\newcommand{\Dmodif}{\textcolor{blue}{(DC)}}
\newcommand{\Gcomment}[1]{\textcolor{magenta}{(ZG)}\marginpar{\tiny\textcolor{magenta}{#1}}}
\newcommand{\Jcomment}[1]{\textcolor{cyan}{(JX)}\marginpar{\tiny\textcolor{cyan}{#1}}}
\newcommand{\Jmodif}{\textcolor{cyan}{(JX)}}
%\newcommand{\Scomment}[1]{\textcolor{blue}{(SC)}\marginpar{\tiny\textcolor{blue}{#1}}}




\begin{document}

\setlength{\baselineskip}{0.53cm}        % Previous 0.47
%
%%%%%%%%%%%%%%%%%%%%%%%%%%%%%%%%%%%%%%%%%%%%%%%%%%%%%%%%%%%%%%%%%%
%
\title[]
{Birational conjugacies between endomorphisms on the projective plane}
\date{2019}

\author{Serge Cantat}
\address{Serge Cantat, IRMAR, Campus de Beaulieu,
b\^atiments 22-23
263 avenue du G\'en\'eral Leclerc, CS 74205
35042  RENNES C\'edex}
\email{serge.cantat@univ-rennes1.fr}

\author{Junyi Xie}
\address{Junyi Xie, IRMAR, Campus de Beaulieu,
b\^atiments 22-23
263 avenue du G\'en\'eral Leclerc, CS 74205
35042  RENNES C\'edex}
\email{junyi.xie@univ-rennes1.fr}

\thanks{The last-named author is partially supported by project ``Fatou'' ANR-17-CE40-0002-01, the first-named author by the french academy of sciences (fondation del Duca). }

%\author{}
%\address{D\'epartement de math\'ematiques\\
%         Universit\'e de Rennes\\
%         Rennes\\
%         France}
%\email{serge.cantat@univ-rennes1.fr, junyi.xie@univ-rennes1.fr}
%
%%%%%%%%%%%%%%%%%%%%%%%%%%%%%%%%%%%%%%%%%%%%%%%%%%%%%%%%%%%%%%%%%%
%

\maketitle
 
%\setcounter{tocdepth}{1}
%\tableofcontents


%
%%%%%%%%%%%%%%%%%%%%%%%%%%%%%%%%%%%%%%%%%%%%%%%%%%%%%%%%%%%%%%%%%%
% 
%
%%%%%%%%%%%%%%%%%%%%%%%%%%%%%%%%%%%%%%%%%%%%%%%%%%%%%%%%%%%%%%%%%%
%

%%%%%%%%%%%%%%%%%%%%%%%%%%%%%%%%%%%%%%%%%%%%%%%%%%%%%%%%%%%%%%
%%%%%%%%%%%%%%%%%%%%%%%%%%%%%%%%%%%%%%%%%%%%%%%%%%%%%%%%%%%%%%



%\begin{abstract}
%\end{abstract}


%\tableofcontents

 

{\noindent}{\bf{1. The statement. --}} Let $\bfk$ be an algebraically closed field of characteristic $0$. If $f_1$ and 
$f_2$ are two endomorphisms of a projective surface $X$ over $\bfk$ and  $f_1$ is conjugate to $f_2$
by a birational transformation of $X$, then $f_1$ and $f_2$ have the same topological degree. 
When $X$ is the projective plane $\P^2_\bfk$, $f_1$ (resp. $f_2$) is given by homogeneous formulas of the same degree $d$ without common 
factor, and $d$ is called the degree, or algebraic degree of $f_1$; in that case the topological degree is $d^2$, so, $f_1$ and $f_2$ have the same degree $d$ if they are conjugate. 

\smallskip

\begin{thm-A} 
Let $\bfk$ be an algebraically closed field of characteristic $0$.  Let $f_1$ and $f_2$ be dominant endomorphisms of $\P^2_\bfk$ over $\bfk$. 
Let $h:\P^2_\bfk\dashrightarrow \P^2_\bfk$ be a birational map such that $h\circ f_1=f_2\circ h$.
If the degree $d$ of $f_1$ is $\geq 2$, there exists an isomorphism $h': \P^2_\bfk\to \P^2_\bfk$ such that $h'\circ f_1=f_2\circ h'$. 

Moreover, 
$h$ itself is in $\Aut(\P^2_\bfk)$, except may be if $f_1$ is conjugate by an element of $\Aut(\P^2_\bfk)$  to

\begin{enumerate}
\item  the composition of $g_d: [x:y:z]\mapsto [x^d:y^d:z^d]$ and a permutation of the coordinates,

\item or the endomorphism $ (x,y) \mapsto (x^d, y^d +\sum_{j=2}^d a_j y^{d-j})$ of the open subset 
$\bbA^1_\bfk\setminus \{0\}\times \bbA^1_\bfk\subset \P^2_\bfk$, for some coefficients $a_j\in \bfk$.
\end{enumerate}
\end{thm-A}






\smallskip

Theorem~A is proved in Sections~2 to 6. A counter-example is given in Section~7 when ${\mathrm{char}}(\bfk)\neq 0$.
The case $d=1$ is covered by~\cite{Blanc2006}; in particular, there are automorphisms $f_1,f_2\in \Aut(\P^2_{\bfk})$ which are conjugate by some birational transformation but not by an automorphism.

\begin{eg}
When $f_1=f_2$ is the composition of $g_d$ and a permutation of the coordinates and $h$ is the Cremona involution $[x:y:z]\mapsto [x^{-1}:y^{-1}:z^{-1}]$, we have $h\circ f_1=f_2\circ h.$
\end{eg}
\begin{eg} When
$$
f_1(x,y) =(x^d, y^d +\sum_{j=2}^d a_j y^{d-j}) \;  {\text{ and }} \; f_2(x,y) =(x^d, y^d +\sum_{j=2}^d a_j (B/A)^j x^{j} y^{d-j})
$$
with $a_j\in \bfk$ then $h(x,y)=(Ax, Bxy)$ conjugates $f_1$ to $f_2$ if $A$ and $B$ are
roots of unity of order dividing $d-1$, and $\deg(h)=2$. On the other hand, 
$h'[x:y:z]=[Az/B:y:x]$ is an automorphism of $\P^2$ that conjugates $f_1$ to $f_2$.
\end{eg}

%: every element $f$ of $\Aut(\P^2_\bfk)$ is conjugate to a diagonal transformation 
%$g_{\alpha, \beta}(x,y)=(\alpha x, \beta y)$ for some $(\alpha,\beta)\in 
%(\bfk^*)^2$, or to an almost diagonal map $h_\alpha(x,y)=(\alpha x, y+1)$ for some $\alpha\in \bfk^*$, by an element of 
%$\Bir(\P^2_\bfk)$; almost diagonal maps are not conjugate to diagonal ones, and $h_\alpha$ is not conjugate 
%to $h_{\alpha'}$ if $\alpha\neq \alpha'$; and two diagonal maps $g_{\alpha, \beta}$ and $g_{\alpha', \beta'}$ 
%are conjugate if and only if $(\alpha, \beta)$ and $(\alpha', \beta')$ are in the same 
%orbit for the group $\GL_2(\Z)$ of monomial transformations.

\smallskip

{\noindent}{\bf{Acknowledgement. --}} Theorem~A answers a question of T. Gauthier and G. Vigny in dimension $2$. We thank them 
for sharing their ideas. We also thank D.-Q. Zhang for answering our questions on the theorem of R. V. Gurjar and D. V. Paz for pointing out a mistake in the first version of Section~7.

\medskip


{\noindent}{\bf{2. The exceptional locus. --}}
If $h:\P^2\dashrightarrow \P^2$ is a birational map, we denote by $\Ind(h)$ its {\bf{indeterminacy locus}} (a finite subset of 
$\P^2(\bfk)$), and by $\Exc(h)$ its {\bf{exceptional set}}, i.e. the union of the curves contracted by $h$ (a finite union of irreducible curves). 
Let $U_h=\P^2_\bfk\setminus \Exc(h)$ be the complement of $\Exc(h)$; it is a Zariski dense open subset of $\P^2_\bfk$.
If $C\subset \P^2_\bfk$ is a curve, we denote by $h_\circ(C)$ the {\bf{strict transform}} of $C$, i.e. the Zariski closure of $h(C\setminus \Ind(f))$.

\begin{pro}\label{prouhreg} If $h$ is a birational transformation of the projective plane, then 
(1) $\Ind(h)\subseteq \Exc(h)$, (2) $h|_{U_h}(U_h)=U_{h^{-1}}$, and  (3) $h|_{U_h}: U_h\to U_{h^{-1}}$ is an isomorphism. 
\end{pro}
\begin{proof}
There is a smooth projective surface $X$ and two birational morphisms $\pi_1,\pi_2: X\to \P^2$ such that $h=\pi_2\circ\pi_1^{-1}$;
we choose $X$ minimal, in the sense that there is no $(-1)$-curve $C$ of $X$ which is contracted by both $\pi_1$ and~$\pi_2$ (\cite{Hartshorne1977}).

Pick a point $p\in \Ind(h)$. The divisor $\pi_1^{-1}(p)$ is a tree of rational curves of negative self-intersections, 
with at least one $(-1)$-curve. If $p\notin \Exc(h)$, any curve  contracted by $\pi_2$ that intersects $\pi_1^{-1}(p)$ is in fact contained in $\pi_1^{-1}(p)$.
But $\pi_2$ may be decomposed as a succession of contractions of $(-1)$-curves: since it does not contract any $(-1)$-curve in $\pi_1^{-1}(p)$, 
we deduce that $\pi_2$ is a local isomorphism along $\pi_1^{-1}(p)$. This contradicts the minimality of $\P^2_\bfk$, hence $\Ind(h)\subset \Exc(h)$.
% There exists at least one $(-1)$-curve in $\pi_1^{-1}(p).$
%Since $\pi_2$ can not contract any $(-1)$-curve in $\pi_1^{-1}(p)$, $\pi_2$ can not contract any curve in $\pi_1^{-1}(p)$. So $\pi_2$ is a local isomorphism along $\pi_1^{-1}(p)$.
%Then $\pi_2(\pi_1^{-1}(p))$ contains a $(-1)$-curve, which contradicts the fact that $\P^2$ is minimal.
%This shows that $I(h)\setminus \Exc(h).$
Thus  $h|_{U_h}: U_h\to \P^2$ is regular.  Since $U_h\cap \Exc(h)=\emptyset,$ $h|_{U_h}$ is an open immersion,  $h^{-1}$ is well defined on $h|_{U_h}(U_h)$, and $h^{-1}$ is an open immersion on $h|_{U_h}(U_h)$. It follows that $h|_{U_h}(U_h)\subseteq U_{h^{-1}}.$
The same argument shows that $h^{-1}|_{U_{h^{-1}}}: U_{h^{-1}}\to \P^2$ is well defined and its image is in $U_h.$
Since $h^{-1}|_{U_{h^{-1}}}\circ h|_{U_h}=\id$ and $h|_{U_h}\circ h^{-1}|_{U_{h^{-1}}}=\id$; this concludes the proof. 
\end{proof}

\medskip

Let $f_1$ and $f_2$ be dominant endomorphisms of $\P^2_\bfk$. Let $h:\P^2\dashrightarrow \P^2$ be a birational map such that $f_1=h^{-1}\circ f_2\circ h$.
Let $d$ be the common (algebraic) degree of $f_1$ and $f_2$.
Recall that an algebraic subset $D$ of $\P^2_\bfk$ is {\bf{totally invariant}} under the action of the endomorphism $g$ if $g^{-1}(C)=C$
(then $g(C)=C$, and if $\deg(g)\geq 2$, $g$ ramifies along $C$).

\begin{lem}\label{lemtotinv} The exceptional set of $h$ is totally invariant under the action of $f_1$: $f_1^{-1}(\Exc(h))=\Exc(h)$.
\end{lem}
\begin{proof} Since $h\circ f_1= f_2\circ h$, the strict transform of $f_1^{-1}(\Exc(h))$ by $f_2\circ h$ is a finite set, but every dominant endomorphism of $\P^2_\bfk$ is a finite map, so the strict transform of $f_1^{-1}(\Exc(h))$ by $h$ is already a finite set. This means that $f_1^{-1}(\Exc(h))$ is contained in 
$\Exc(h)$;  this implies $f_1(\Exc(E))\subset E$ and then $f_1^{-1}(\Exc(h))=\Exc(h)=f_1(\Exc(h))$ because $f_1$ is onto.
\end{proof}

\begin{lem}\label{lemconfexc} If $d\geq 2$ then $\Exc(h)$ and $\Exc(h^{-1})$ are two isomorphic configurations of lines, and this
configuration falls in the following list:
\begin{enumerate}
\item[(P0)] the empty set;
\item[(P1)] one line in $\P^2$;
\item[(P2)]  two lines in $\P^2$;
\item[(P3)]  three lines in $\P^2$ in general position.
\end{enumerate}
\end{lem}

\begin{proof} Assume $\Exc(h)$ is not empty; then, by Lemma \ref{lemtotinv},  the curve $\Exc(h)$ is totally invariant under $f_1$.
According to \cite[\S 4]{Fornaess-Sibony} and~\cite[Proposition 2]{Cerveau-LN}, $\Exc(h)$ is one of the three curves listed in (P1) to (P3).


Changing $h$ into $h^{-1}$ and permuting the role of $f_1$ and $f_2$, we see that $\Exc(h^{-1})$ is also a configuration of type (P$i$) for some $i$.
Proposition \ref{prouhreg} shows that $U_h\simeq U_{h^{-1}}$.
Since the four possibilities (P$i$) correspond to pairwise 
non-isomorphic complements, we deduce that $\Exc(h)$ and $\Exc(h^{-1})$ have the same type. 
\end{proof}


\begin{rem}
One can also refer to \cite{Gurjar2003} to prove this lemma. Indeed,  
$f_1$ induces a map from the set of irreducible components of $\Exc(h)$ into itself, 
and since $f_1$ is onto, this map is a permutation; the same applies to $f_2$. Thus,
replacing $f_1$ and $f_2$ by $f_1^m$ and $f_2^m$ for some suitable $m\geq 1$, we may assume that $f_1(C)=C$ for every irreducible component $C$ of $\Exc(h)$.
Since $f_1$ is finite, $\Exc(h)$ has only finitely many irreducible components, and $f_1(\Exc(h))=\Exc(h)$, we obtain $f_1^{-1}(C)=C$ for every component. Since $f_1$ acts by multiplication by $d$ on $\Pic(\P^2_\bfk)$, the ramification index of $f_1$ along $C$ is $d>1$, and
the main theorem of \cite{Gurjar2003} implies that $C$ is a line. 
%So $\Exc(h)$ is a finite union $\cup_{i=1}^sL_i$ of $s$ distinct lines $L_i\subset \P^2_\bfk$.
%When $s\leq 2$, this configuration of lines is of type (P0), (P1) or (P2).  Now assume that $s\geq 3$. 
%We fix homogenous coordinates $[x : y : z]$  such that $L_1=\{z=0\}$ and $L_2=\{x=0\}$, and we identify $\P^2\setminus L_1$ to the plane $\bbA^2_\bfk=\Spec( \bfk[x,y])$. Since $f_1^*L_i=dL_i$,
%$f_1$ induces an endomorphism of $\bbA^2_\bfk$ of the form $f_1(x,y)= (x^d, F(x,y))$, with $\deg_y F(x,y)=d$.
%If $L_1\cap L_2\cap L_3\neq \emptyset$, then the equation of $L_3\cap \bbA^2$ is $x=a$, for some $a\in \bfk^*$ and
%$(f_1^*L_3)\cap \bbA^2=\{x^d=a\}\neq L_3\cap \bbA^2$. This contradiction shows that $L_1$, $L_2$, and $L_3$ are in general 
%position, and we may take $L_3=\{y=0\}$ after a linear change of coordinates. 
%Then $f_1$ takes form $(x,y)\mapsto (x^d,y^d)$. Since $f|_{\P^2\setminus (L_1\cup L_2\cup L_3)}$ is \'etale, we obtain that $s=3$ and $(P3)$ is satisfied.
\end{rem}

\begin{rem}
Totally invariant hypersurfaces of endomorphisms of $\P^3$  are unions of hyperplanes, at most four of them (we refer to~\cite{Horing:2017} for a proof and important additional 
references, notably the work of J.-M. Hwang, N. Nakayama and D.-Q. Zhang). So, an analog of Lemma~5 holds
in dimension $3$ too; but our proof in case (P1), see \S~4 below, does not apply in dimension $3$, at least not directly.
(Note that~\cite{Briend-Cantat-Shishikura:2004} contains an important gap, 
since its main result is based on a wrong lemma from~\cite{Briend-Duval:2001}).
\end{rem}




\medskip

{\noindent}{\bf{3. Normal forms. --}}
Two configurations of the same type (P$i$) are equivalent under the action of $\Aut(\P^2_\bfk)=\PGL_3(\bfk)$. 
If we change $h$ into $A\circ h\circ B$ for some well chosen pair of automorphisms $(A,B)$, or equivalently if we change $f_1$ into $B\circ f_1\circ B^{-1}$ and 
$f_2$ into $A^{-1}\circ f_2\circ A$, we may assume that $\Exc(h)=\Exc(h^{-1})$ and that exactly  one of the following situation occurs (see also~\cite{Fornaess-Sibony}):

\smallskip

{{\bf{(P0).--}} {$\Exc(h)=\Exc(h^{-1})=\emptyset$.--}\indent Then $h$ is an automorphism of $\P^2_\bfk$ and Theorem~A is proved. 

\smallskip

{{\bf{(P1).--}} {$\Exc(h)=\Exc(h^{-1})=\{z=0\}$.--}\indent  Then $h$ induces an automorphism of $\bbA^2_\bfk$ and
$f_1$ and $f_2$ restrict to endomorphisms of $\bbA^2_\bfk=\P^2_\bfk\setminus \{z=0\}$ (that extend to endomorphisms of $\P^2_\bfk$).

\smallskip

{{\bf{(P2).--}} {$\Exc(h)=\Exc(h^{-1})=\{x=0\}\cup \{z=0\}$.--}\indent  
Then, 
$U_h$ and $U_{h^{-1}}$ are both equal to the open set $U:=\{(x,y)\in \bbA^2|\,\, x\neq 0\}$. Moreover,
\begin{equation}\label{eq:h-jonq}
h|_U(x,y) = (Ax, Bx^{m}y+C(x))
\end{equation}
for some regular function $C(x)$ on $\bbA^1_\bfk\setminus\{ 0\}$ and $m\in \Z$, and 
\begin{equation}\label{eqf_ip2}
f_i|_U(x,y)= (x^{\pm d}, F_i(x,y))
\end{equation} 
for some rational functions $F_i\in \bfk(x)[y]$ which are regular on $(\bbA^1_\bfk\setminus\{ 0\})\times \bbA^1$
and have degree $d$ (more precisely, $f_i$ must define an endomorphism of $\P^2$ of degree $d$).
Moreover, the signs of the exponent $\pm d$ in Equation~\eqref{eqf_ip2} are the same for $f_1$ and $f_2$.

\smallskip

{{\bf{(P3).--}} {$\Exc(h)=\Exc(h^{-1})=\{x=0\}\cup\{y=0\}\cup \{z=0\}$.--}\indent  
In this case, 
each $f_i$ is equal to $a_i\circ g_d$ where  $g_d( [x:y:z])= [x^d:y^d:z^d]$
and each $a_i$  is an automorphism of $\P^2_\bfk$ acting by permutation of the coordinates, while 
$h$ is an automorphism of $(\bbA^1\setminus \{0\})\times (\bbA^1\setminus \{0\})$.






\medskip


{\noindent}{\bf{4. Endomorphisms of $\bbA^2_\bfk$. --}}
This section proves Theorem~A in case (P1):  

\begin{pro}\label{propolyendo} Let $f_1$ and $f_2$ be endomorphisms  of $\bbA^2$ that extend to endomorphisms of $\P^2$ of degree $d\geq 2$. If $h$ is an automorphism of $\bbA^2$ that conjugates $f_1$ to $f_2$ then  $h$ is an affine automorphism i.e. $\deg h=1$.
\end{pro}

We follow the notation from \cite{Favre2011} and denote by 
$V_{\infty}$ the valuative tree of $\bbA^2=\Spec ( \bfk[x,y] )$ at infinity.
If $g$ is an endomorphism of $\bbA^2$, we denote by $g_\bullet$ its action on $V_{\infty}$.
 
Set $V_1=\{v\in V_{\infty} \; ; \; \alpha(v)\geq 0, A(v)\leq 0\}$, where $\alpha$ 
and $A$ are respectively the skewness and thinness function, as defined in page 216 of
\cite{Favre2011}; the set $V_1$  is a closed subtree of $V_{\infty}$. For $v\in V_1$, $v(F)\leq  0$ for every $F\in \bfk[x,y]\setminus \{0\}$. 
Then $V_1$ is invariant under each $({f_i})_{\bullet}$, and if we 
set 
\begin{equation}
\sT_i=\{v \in V_1 \; ; \; ({f_i})_{\bullet}v=v\}
\end{equation} 
then $\sT_2=h_{\bullet}\sT_1$. Since each $f_i$ extends to an endomorphism of $\P^2_\bfk$, the valuation  $-\deg$ is an element of $\sT_1\cap \sT_2$. Also, in the terminology of \cite{Favre2011}, $\lambda_2(f_i)=\lambda_1(f_i)^2=d^2$ and $\deg(f_i^n)=\lambda_1^n=d^n$ 
for all $n\geq 1$ and for $i=1$ and $2$, because $f_1$ and $f_2$ extend to regular endomorphisms of $\P^2_\bfk$ of degree $d$. So
by \cite[Proposition 5.3 (a)]{Favre2011}, $\sT_i$ is a single point or a closed segment.  

\smallskip 

A valuation $v\in V_{\infty}$ is {\bf{monomial}}  of weight $(s,t)$ for the pair of polynomial functions $(P,Q)\in \bfk[x,y]^2$ if 
\begin{enumerate}
\item $P$ and $Q$ generate $\bfk[x,y]$ as a $\bfk$-algebra,
\item if $F$ is any non-zero element of $\bfk[x,y]$ and $F=\sum_{i,j\geq 0}a_{ij}P^iQ^j$ is its decomposition as a polynomial function
of  $P$ and $Q$ then
\begin{equation}
v(F)=-\max\{si+tj \; ; \; a_{i,j}\neq 0\}.
\end{equation}
\end{enumerate}
We say that $v$ is monomial for the basis  $(P,Q)$ of $\bfk[x,y]$, if  $v$ is monomial for $(P,Q)$ and some weight $(s,t)$.
In particular, $-\deg$ is monomial for $(x,y)$, of weight $(1,1).$

\begin{lem}\label{lemprovow}If $v\in V_1$ is monomial for $(P,Q)$ of weight $(s,t)$, then   $s,t\geq 0$, and $\min\{s,t\}=\min\{-v(F) \; ; \; F\in \bfk[x,y]\setminus \bfk\}.$
\end{lem}
\begin{proof}
First, assume that $(P,Q)=(x,y)$. For an element $v$ of $V_1$, $v(F)\leq 0$ for every $F$ in $\bfk[x,y]$, hence $s=-v(x)$ and $t=-v(y)$
are non-negative; and the formula for $\min\{s,t\}$ follows from the inequality $-v(F)\geq \min\{s,t\}$. To get the statement for any pair $(P,Q)$, 
change $v$ into $g^{-1}_\bullet v$ where $g$ is the automorphism defined by $g(x,y)=(P(x,y),Q(x,y))$.
\end{proof}

\begin{lem}\label{lemdegmon}If $-\deg$ is monomial for $(P,Q)$, of weight $(s,t)$, then $s=t=1$ and $P$ and $Q$ are of degree one in $\bfk[x,y].$
\end{lem}

\begin{proof}By Lemma \ref{lemprovow}, we may assume that $1=s\leq t$; thus, 
after an affine change of variables, we may assume that $P=x.$ Since $\bfk[x,y]$ is generated by $x$ and $Q$, $Q$ takes form 
$Q=ay+C(x)$ where $a\in \bfk^*$ and $C\in \bfk[x]$. 
If $C$ is a constant, we conclude the proof. Now we assume $\deg (C)\geq 1$.
Then $t=\deg(Q)=\deg(C)$. Since $y=a^{-1}(Q-C(x))$ and $-\deg$ is monomial for $(x,Q)$ of weight $(1,t)$, we get $1=\deg(y)=\max\{t, \deg C\}=t$. It follows that $t=\deg Q=1,$ which concludes the proof.
\end{proof}

\proof[Proof of Proposition \ref{propolyendo}]
By \cite[Proposition 5.3 (b), (d)]{Favre2011}, there exists $P$ and $Q\in \bfk[x,y]$ such that 
for every $v\in \sT_1$, $v$ is monomial for $(P,Q)$. Moreover, $-\deg$ is in $\sT_1\cap \sT_2$. By Lemma~\ref{lemdegmon},  $P=x$ and $Q=y$ after an affine change of coordinates.
Since $\sT_2=h_{\bullet} \sT_1$, for every $v\in \sT_2$, $v$ is monomial for $(h^*x,h^*y).$ Since $-\deg\in \sT_2$,  Lemma \ref{lemdegmon} implies $\deg h^*x=\deg h^*y=1$ and this concludes the proof.
\endproof

\medskip

{\noindent}{\bf{5. Endomorphisms of $(\bbA^1_\bfk\setminus \{0 \})\times \bbA^1_\bfk$. --}}
We now arrive at case (P2), namely $\Exc(h)=\Exc(h^{-1})=\{x=0\}\cup \{z=0\}$, and keep the notations from Section~4.
Our first goal is to prove that, 

\begin{lem}
If $h$ is not an affine automorphism of the affine plane, then after a conjugacy by an affine transformation of the 
plane,

$\bullet$ Either $f_1$ and $f_2$ are equal to $(x^d,y^d)$ and $h(x,y)=(Ax, Bx^my)$ with $A$ and $B$ two 
roots of unity of order dividing $d-1$ and $m\in \Z\setminus\{0\}$.

$\bullet$ Or, up to a permutation of $f_1$ and $f_2$, 
$$
f_1(x,y) =(x^d, y^d +\sum_{j=2}^d a_j y^{d-j}) \;  {\text{ and }} \; f_2(x,y) =(x^d, y^d +\sum_{j=2}^d a_j (B/A)^j x^{j} y^{d-j})
$$
with $a_j\in \bfk$,
 and $h(x,y)=(Ax, Bxy)$ with $A$ and $B$ two 
roots of unity of order dividing $d-1$; then $h'[x:y:z]=[Az/B:y:x]$ is an automorphism of $\P^2$
that conjugates $f_1$ to $f_2$.
%$\bullet$ Or, up to a permutation of $f_1$ and $f_2$, $f_1$ and $f_2$ are conjugate to 
%$$
%f_1(x,y) =(x^d, y^d +\sum_{j=1}^d a_j y^{d-j}) \;  {\text{ and }} \; f_2(x,y) =(x^d, y^d +\sum_{j=1}^d a_j (B/A)^j x^{j} y^{d-j})
%$$
%with $a_j\in \bfk$,
%by some some affine change of variables and $h(x,y)=(Ax, Bxy+C(x))$ with $A$ and $B$ two 
%roots of unity of order dividing $d-1$; but then $h'[x:y:z]=[Az/B:y:x]$ is an automorphism of $\P^2$
%that conjugates $f_1$ to $f_2$.
\end{lem}

\begin{proof} We split the proof in two steps.
%{\bf{Step 0.--}} Since 

\smallskip

{\bf{Step 1.--}} 
We  assume that $f_i|_U(x,y)= (x^{d}, F_i(x,y))$, with $d>0$.
%Changing $f_i$ into $f_i^2$ and $d$ in $d^2$, we  assume that $f_i|_U(x,y)= (x^{d}, F_i(x,y))$.

Since $f_i$ extends to a degree $d$ endomorphism of $\P^2_\bfk$, we can write $F_1(x,y)=a_0y^d+\sum_{j=1}^da_j(x)y^{d-j}$ where $a_0\in \bfk^*$  and the $a_j\in \bfk[x]$
satisfy $\deg(a_j)\leq j$ for all $j$.
Changing the coordinates to $(x,by)$ with $b^d=a_0$, we assume $a_0=1$. We can also conjugate $f_1$ by the automorphism 
\begin{equation}
(x,y)\mapsto \left( x,y+\frac{1}{d} a_1(x) \right)
\end{equation}
and assume $a_1=0$. Altogether, the change of coordinates $(x,y)\mapsto (x, by+\frac{1}{d} a_1(x))$ is affine because 
$\deg(a_1)\leq 1$, and conjugates $f_1$ to an endomorphism $(x^{d}, F_1(x,y))$ normalized 
by $F_1(x,y)=y^d + \sum_{j=2}^d a_j(x) y^{d-j}$ with $\deg(a_j)\leq j$. 
Similarly, we may assume
that $F_2(x,y)=y^d+\sum_{j=2}^db_j(x)y^{d-j}$ for some polynomial functions $b_j$ with $\deg(b_j)\leq j$ for all $j$.

Now, with the notation used in Equation~\eqref{eq:h-jonq}, the two terms of the conjugacy relation $h\circ f_1=f_2\circ h$ are
\begin{align}
h\circ f_1 &=(Ax^d, \; Bx^{dm}(y^d+\sum_{j=2}^da_j(x)y^{d-j})+C(x^d) )\label{eq:conj2}\\
f_2\circ h &=(A^d x^d, \; (Bx^my+C(x))^d+\sum_{j=2}^db_j(Ax) (Bx^my+C(x))^{d-j} ) \label{eq:conj1}.
\end{align}
This gives $A^{d-1}=1$, and comparing the terms of degree $d$ in $y$ we get %$m=0$ and 
$B^{d-1}=1$. 
Then, looking at the term of degree $d-1$ in $y$, 
%and using $B^{d-1}=1$, 
we obtain $C(x)=0.$
%\begin{equation}\label{eq:C_is_zero}
%%dC(x) = Bx^ma_1-b_1(Ax)=0
%C(x)=0.
%\end{equation}
%because $a_1$ and $b_1$ are both $0$.
Thus $h(x,y)=(Ax, Bx^m y)$ for some roots of unity $A$ and $B$, the orders of which divide $d-1$. 
Since $h$ is not an automorphism, we have 
\begin{equation}\label{eq:mneq0}
m\neq 0.
\end{equation}


Permuting the role of $f_1$ and $f_2$ (or changing $h$ in its inverse), we suppose $m\geq 1$.
Coming back to~\eqref{eq:conj2} and~\eqref{eq:conj1}, we obtain the sequence of equalities 
\begin{equation}
b_j(Ax) = a_j(x) (Bx^m)^j 
\end{equation}
for all indices $j$ between $2$ and $d$. On the other hand, $a_j$ and $b_j$ are elements of $\bfk[x]$ of degree at most $j$. Since $m\geq 1$, there are only two possibilities.
\begin{enumerate}
\item[(a)] All $a_j$ and $b_j$ are equal to $0$; then  $f_1(x,y)=f_2(x,y)=(x^d, y^d)$, which concludes the proof.
\item[(b)] Some $a_j$ is different from $0$ and $m=1$. Then all coefficients $a_j$ are constant, and $b_j(x)=a_j \left(\frac{Bx}{A}\right)^j$
for all indices $j=2, \ldots, d$.
\end{enumerate}
%Case (a) will be covered by Section~6 below; indeed, since $h$ conjugates $f_1$ to $f_2$ it maps the critical locus of $f_1$ to 
%the critical locus of $f_2$, hence the line $\{ y=0\}$ to itself, so that $h$ induces an automorphism of $(\bbA^1\setminus\{0\})^2$ (this is 
%all we assume in Section~6). 
%In case (b), $h$ is an affine automorphism of the affine plane.
In case (b), we set $\alpha=B/A$ (a root of unity of order dividing $d-1$),  and use homogeneous coordinates to write
\begin{align}
f_1[x:y:z] &=[x^d : y^d +\sum_{j=2}^d a_j z^{j} y^{d-j}: z^d] \\ 
f_2[x:y:z] &=[x^d : y^d +\sum_{j=2}^d a_j \alpha^j x^{j} y^{d-j}: z^d].
\end{align}
The conjugacy $h[x:y:z]=[Axz : Bxy : z^2]$ is not a linear projective automorphism of $\P^2$, but
the automorphism defined by 
$[x:y:z]\mapsto \left[ z/\alpha:y:x \right]$ conjugates $f_1$ to $f_2$. 

\smallskip

{\bf{Step 2.--}} The only remaining case is when $f_i=(x^{-d}, F_i(x,y))$, for $i=1,2$, with 
\begin{equation}
F_1(x,y)=\sum_{j=0}^d a_j(x)x^{-d} y^{d-j} \; \text{ and } \;  F_2(x,y)=\sum_{j=0}^d b_j(x)x^{-d} y^{d-j}
\end{equation}
for some polynomial functions $a_j, b_j\in \bfk[x]$ that satisfy $\deg(a_j), \deg(b_j)\leq j$ and $a_0b_0\neq 0$. 
Writing the conjugacy equation $h\circ f_1=f_2\circ h$ and looking at the term of degree $d$ in $y$, we get the relation
\begin{equation}
Bx^{-md} a_0 x^{-d} y^d= b_0 (Ax)^{-d} (Bx^my)^d.
\end{equation}
Comparing the degree in $x$ we get $-md-d=md-d$, hence $m=0$. Moreover, $h$ conjugates $f_1^2$
to $f_2^2$; thus, by the first step, $h$ should be an affine automorphism since $m=0$ (see Equation \eqref{eq:mneq0}).
\end{proof}

\medskip

{\noindent}{\bf{6. Endomorphisms of $(\bbA^1_\bfk\setminus \{0 \})^2$. --}} Denote by $[x:y:z]$ the homogeneous coordinates of $\P^2_\bfk$
and by $(x,y)$ the coordinates of the open subset $V:=(\bbA^1_\bfk\setminus \{0 \})^2$ defined by $xy\neq 0$, $z=1$. 
We write $f_i=a_i\circ g_d$ 
as in case (P3) of Section~{\bf{3}}. 
Since $h$ is an automorphism of $(\bbA^1_\bfk\setminus \{0 \})^2$, it is the composition $t_h\circ m_h$ of a 
diagonal map $t_h(x,y)=(ux,vy)$, for some pair  $(u,v)\in (\bfk^*)^2$, and  a monomial map 
$m_h(x,y)=(x^ay^b, x^cy^d)$, for some matrix
\begin{equation}
M_h:=\left(\begin{array}{lr}a & b \\ c & d \end{array}\right) \in \GL_2(\Z).
\end{equation}
Also, note that the group $\mathfrak{S}_3\subset \Bir(\P^2_\bfk)$ of permutations of the coordinates $[x:y:z]$ corresponds to a finite 
subgroup $S_3$ of $\GL_2(\Z)$. 

Since $m_h$ commutes to $g_d$ and $g_d \circ t_h=t_h^d\circ g_d$,  the conjugacy equation is equivalent to 
\begin{equation}\label{eq:conj-eq-P3}
t_h\circ (m_h\circ a_1\circ m_h^{-1})\circ (g_d\circ m_h)= a_2\circ t_h^d\circ (g_d\circ m_h).
\end{equation}
The automorphisms $a_1$ and $a_2$ are monomial maps, induced by elements $A_1$ and $A_2$ of $S_3$, 
and Equation~\eqref{eq:conj-eq-P3} implies that {\sl{$M_h$ conjugates $A_1$ to $A_2$ in $\GL_2(\Z)$}}; indeed, 
the matrices can be recovered by looking at the action on the set of units $wx^m y^n$ in $\bfk(V)$ (or on the fundamental group 
$\pi_1(V(\C))$ if $\bfk=\C$). There are two possibilities : 
\begin{itemize}
\item[(a)] either $A_1=A_2=\mathrm{Id}$, there is no constraint on $m_h$;
\item[(b)] or $A_1$ and $A_2$ are non-trivial permutations, they are conjugate by an element $P\in S_3$,  and 
$M_h=\pm A_2^j \circ P$, for some $j\in \Z$. 
\end{itemize}
In both cases, $u$ and $v$ are roots of unity (there order is determined by $d$ and the $A_i$).
Let $p$ be the monomial transformation associated to $P$; it is a permutation of the coordinates, 
hence an element of $\Aut(\P^2_\bfk)$. Then, $h'(x,y)=t_h\circ p$ is an element of $\Aut(\P^2_\bfk)$ 
that conjugates $f_1$ to $f_2$. 
%So, we get 
%
%
%\smallskip
%
%{\sl{In case}} (P3), {\sl{there is an element of $\Aut(\P^2_\bfk)$ that conjugates $f_1$ to $f_2$, and $h$ is itself in $\Aut(\P^2_\bfk)$ as soon 
%as $a_1\neq {\mathrm{Id}}$.}}

 


\medskip

%{\noindent}{\bf{7. An example in positive characteristic. --}} We provide examples showing that Theorem~A does not hold in positive 
%characteristic. Let $p$ be a prime number, $s$ a positive integer, and $q=p^s$.
%Let $\bfk$ be an extension of the finite field $\bfF_q$.
%
%\medskip
%{\noindent}{\sl{7.1. Preliminary Remark.--}}
%For a dominant endomorphisms $f,g$ of $\P^1_{\bfk}$, we write $f\sim g$ if $f$ and $g$ are conjugate by some automorphisms of $\P^1_{\bfk}$.
%
%\begin{lem}\label{lemconjpoly}Assume $s\geq 2.$
%Let $f=y^q+a_1y^p+a_0y$ and $g=y^q+b_1y^p+b_0y$ be two polynomial endomorphism on $\P^1_{\bfk}$.
%If $f\sim g$, then there exists $u\in \bfk^*$ satisfying $u^{q-1}=1, a_1u^{p-1}=b_1$ and $a_0=b_0.$ 
%\end{lem}
%
%
%
%
%
%
%
%
%
%
%
%
%
%
%\newpage

%
%
%{\noindent}{\sl{7.1. Preliminary Remark.--}} Let $G\in \bfk[x_0, \ldots, x_k]_d$ be a homogeneous polynomial function of degree $d$; 
%write 
%\begin{equation}
%G(x_0, \ldots, x_k)=\sum_{i_0+\cdots+i_k=d} c_{i_0, \ldots, i_k} x_0^{i_0}\cdots x_k^{i_k}
%\end{equation}
%and define the {\bf{mixed part}} of $G$ by 
%\begin{equation}
%m(G)=G(x_0, \ldots, x_k)- (c_{d, 0, \ldots, 0} x_0^d +\cdots + c_{0, \ldots, 0, d}x_k^d ).
%\end{equation}
%Now, let $g:\P^k_\bfk \dasharrow \P^k_\bfk$ be a rational map, defined by 
%homogeneous polynomial functions $G_i$ of degree $d$ without common factor: $g[x_0:\cdots : x_n]=[G_0:\cdots :G_n]$.
%We define the mixed part of $g$ to be the element
%\begin{equation}
%m(g) =(m(G_0):\cdots:m(G_k)),
%\end{equation}
%of $(\bfk[x_0, \ldots, x_k]_d^{k+1})/\bfk^*$.
%
%We shall denote the projection 
%$\pi\colon\bfk[x_0, \ldots, x_k]_d^{k+1}\to (\bfk[x_0, \ldots, x_k]_d^{k+1})/\bfk^*$ 
%by $\pi\colon(F_0,\ldots,F_{k+1})\mapsto (F_0 : \dots : F_{k+1})$. 
%
%An element $(F_0,\dots,F_{k+1})\in \bfk[x_0, \ldots, x_k]_d^{k+1}$ can be viewed as an endomorphism of $\bbA^{k+1}_{\bfk}$. 
%The group $\GL_{k+1}(\bfk)$ acts by right and left composition on such endomorphisms; this determines two actions 
%of $\PGL_{k+1}(\bfk)$ on  the quotient space $(\bfk[x_0, \ldots, x_k]_d^{k+1})/\bfk^*$ such that 
%\begin{align}
%(F_0:\dots: F_k)\circ h & :=((F_0,\dots,F_k)\circ H) \\
% h\circ (F_0:\dots: F_k) & :=(H\circ (F_0,\dots,F_k))
% \end{align}
%for every $H$ in $\GL_{k+1}(\bfk)$ whose image in $\PGL_{k+1}(\bfk)$ is equal to $h$.
%
%
%For every $(F_0:\dots: F_N)\in (\bfk[x_0, \ldots, x_k]_d^{k+1}\setminus \{0\})/\bfk^*$,  the rational transformation  $[F_0:\dots: F_N]$
%of $\P^k_\bfk$ will be denoted $\overline{(F_0:\dots: F_N)}$; and we shall set ${\overline{m}}(g)={\overline{m(g)}}$
%for every rational map $g:\P^k_\bfk \dasharrow \P^k_\bfk$ whose mixed part is not $0$.
%
%
%
%%; if the $m(G_i)$ are not all equal to $0$, one can consider $m(g)$ 
%%as a rational transformation of $\P^k_\bfk$.  
%
%The next result follows from the fact that $(ax+by)^q=a^qx^q+b^qy^q$ on $\bfk$.
%
%\begin{lem}\label{lem:mixed_part} Let $g$ be a rational transformation of $\P^k_\bfk$ of degree $q$. 
%Let $h$ be an element of $\Aut(\P^k_\bfk)$. Then 
%$m(h\circ g)=h\circ m(g)$ and
%$m(g\circ h)=m(g)\circ h$. In particular, we have $m(h\circ g\circ h^{-1})=h\circ m(g)\circ h^{-1}$.
%\end{lem}
%
%
%In particular if $g_1$ and $g_2$ are conjugated by some automorphism, then their mixed parts are 
%conjugated by the same automorphism.
%
%
%\medskip

%{\noindent}{\sl{7.1. The examples.--}}
%If $G$ is an element of $\bfk[x,y]$, then 
%\begin{equation}
%f_1(x,y)= (x^q, y^q+G(x,y))
%\end{equation}
%is a polynomial endomorphism of the affine plane $\bbA^2_\bfk$. If $\deg(G)< q$, then $f_1$ extends to the endomorphism 
%$f_1([x:y:z])=[x^q:y^q+z^q G(x/z,y/z):z^q]$ of the projective plane $\P^2_\bfk$. 
%
%Now fix an integer $m\geq 2$ and consider an element $P$ of $\bfF_q[x]$ of degree $m$. Then
%$h(x,y)= (x,y-P(x))$
%is an automorphism of $\bbA^2_\bfk$ that conjugates $f_1$ to 
%\begin{align}f_2(x,y) &:=h\circ f_1\circ h^{-1}(x,y) \notag \\
%&=(x^q,y^q+P(x)^q+G(x,y+P(x))-P(x^q))\\
%&=(x^q,y^q+G(x,y+P(x))).\notag
%\end{align}
%If $\deg(G)< q/m$ then $\deg G(x,y+P(x))\leq m\deg G<q$ and $f_2$ extends to an endomorphism of $\P^2_\bfk$. 
%So, $f_1$ and $f_2$ are regular endomorphisms of $\P^2_\bfk$ of degree $q\geq 2$ and $h$ is a birational conjugacy 
%between them. 
%
%\begin{pro}\label{pronotaut}
%If $q/2>\deg(G(x,y+P(x))>\deg(G(x,y))\geq 1$,
%the endomorphisms $f_1$ and $f_2$ are not conjugate by an automorphism 
%of $\P^2_\bfk$.
%\end{pro}  
%
%\begin{proof}
%Set $G_1:=G(x,y)-G(0,0)$ and $G_2:=G(x,y+P(x)))-G(0,P(0))$.
%Then we have $1\leq \deg G_1<\deg G_2<q/2$.
%The mixed parts of $f_1$ and $f_2$ are respectively equal to 
%\begin{equation}
%m(f_1)  =(0:G_1(x/z,y/z)z^q:0) \ \,  {\text{ and }} \ \, 
%m(f_2)  =(0:G_2(x/z,y/z)z^q:0).
%\end{equation}
%Then we have $\overline{m}(f_1)=\overline{m}(f_2)=[0:1:0]$.
%Assume that $f_1$ and $f_2$ are conjugate by an automorphism $h=[H_0:H_1:H_2]$
%of $\P^2_\bfk$, where $H_0$, $H_1$, $H_2\in \bfk[x,y,z]_1$.  By Lemma~\ref{lem:mixed_part},  $h\circ m(f_1)=m(f_2)\circ h$.
%This implies $h\circ \overline{m}(f_1)=\overline{m}(f_2)\circ h$, hence $H_0$, $H_2\in \bfk[x,z]_1$ and $H_1(0,1,0)\neq 0$. And this gives also
%\begin{align}
%h\circ m(f_1) & =(0:H_1(0,1,0)G_1(x/z,y/z)z^q:0)=(0:G_1(x/z,y/z)z^q:0);\\
%m(f_2)\circ h &=(0:G_2(H_0/H_2, H_1/H_2)H_2^q,0).
%\end{align}
%Then there exists $c\in \bfk^*,$ such that 
%\begin{equation}
%Q:=G_1(x/z,y/z)z^q=cG_2(H_0/H_2, H_1/H_2)H_2^q\in \bfk[x,y,z]_q.
%\end{equation}
%The order of $z$ in $Q$ is $q-\deg G_1>q/2=\deg(Q)/2$ and the order of $H_2$ in $Q$ is $q-\deg G_2>q/2.$
%So $H_2=c'z$ for some $c'\in \bfk^*$ and $q-\deg G_1=q-\deg G_2,$ which is a contradiction.
%%Write 
%%\begin{equation}
%%G(x,y)=\sum_{i+j\leq \deg(G)} c_{i,j} x^i y^j \; {\text{ and }} P(x)=\sum_{i\leq m} d_i x^i.
%%\end{equation}
%%The mixed parts of $f_1$ and $f_2$ are respectively equal to 
%%\begin{align}
%%m(f_1) & =[0:G(x/z,y/z)z^q-c_{0,0}z^q:0]\\
%%m(f_2) & =[0:G(x/z,y/z+P(x/z))z^q-(\sum_j c_{0,j} d_0^j)z^q:0],
%%\end{align}
%%because $G(x,y)$ and $G(x,y+P(x))$ have degree $< q$. If $h$ is an automorphism of 
%%$\P^2_\bfk$, the formulas defining $h\circ m(f_1)\circ h$ have degree $\leq \deg(G)$; thus, 
%%$h\circ m(f_1)\circ h$ can not be equal to $m(f_2)$, and the conclusion follows from lemma~\ref{lem:mixed_part}.
%\end{proof}
%
%\begin{eg}
%Let $q\geq 5$ be a prime power, $P(x)=x^2$ and $G(x,y)=y$, we get $f_1=(x^q,y^q+y)$ and $f_2=(x^q, y^q+x^2).$ Then $h:=(x,y-x^2)$ conjugates $f_1$ to $f_2$.
%But by Proposition \ref{pronotaut}, they are not conjugate by an automorphism 
%of $\P^2_\bfk$.
%\end{eg}



{\noindent}{\bf{7. An example in positive characteristic. --}}  Assume that $q=p^s$ with $s\geq 2$.
Set $G:=xy^p+(x-1)y$. Then, 
$$f_1(x,y)=(x^q, y^q+G(x,y))$$
defines an endomorphism of $\bbA^2$ that extends to an endomorphism of $\P^2$.

Consider a polynomial $P(x)\in\bfF_q[x]$ such that  $2 \leq \deg(P) \leq \frac{q}{p}-1$.
Observe that $\deg (G) < \deg (G(x,y+P(x))) < q$.
Then $g(x,y)= (x,y-P(x))$
is an automorphism of $\bbA^2_\bfk$ that conjugates $f_1$ to 
\begin{align}f_2(x,y) &:=g\circ f_1\circ g^{-1}(x,y) \notag \\
&=(x^q,y^q+P(x)^q+G(x,y+P(x))-P(x^q))\\
&=(x^q,y^q+G(x,y+P(x))).\notag
\end{align}
As $f_1$, $f_2$ is an endomorphism of $\bbA^2$ that extends to a regular endomorphism of~$\P^2$
(here we use the inequality $\deg (G(x,y+P(x))) < q$).

Let us prove that $f_1$ and $f_2$ are not conjugate by any automorphism of $\P^2$.
We assume that there exists $h\in \PGL_3(\overline{\bfF_q})$ such that $h\circ f_1=f_2\circ h$ and seek a contradiction.
Consider the pencils of lines through the point $[0:1:0]$ in $\P^2$; for $a\in \bfF_q$ we denote by 
$L_a$ the line $\{x=az\}$, and by $L_\infty$ the line $\{ z=0\}$. 
Then 
\begin{align}\{L_a \; ; \; a\in \bf F_q\cup \{\infty\}\} & = \{{\text{lines}} \ L \ {\text{such that}}\  f_1^{-1}L=L\} \\
&  =\{{\text{lines}} \ L \ {\text{such that}}\ f_2^{-1}L=L\};
\end{align}                  
in other words, the lines $L_a$ for $a\in \bfF_q\cup\{\infty\}$ are exactly the lines which are totally invariant under
the action of $f_1$ (resp. of $f_2$).
Since $h$ conjugates $f_1$ to $f_2$, it permutes these lines. In particular, $h$ fixes the point $[0:1:0]$, and if 
we identify $L_a\cap \bbA^2$ to $\bbA^1$ with its coordinate $y$ by the parametrization $y\mapsto (a,y) $ then 
$h$ maps $L_a$ to another line $L_{a'}$ in an affine way: $h(a,y)=(a',\alpha y +\beta)$.

Since $g$ conjugates $f_1$ to $f_2$ and $g$ fixes each of the lines $L_a$, we know that $f_1|_{L_a}$ is conjugated to $f_2|_{L_a}$
for every  $a\in \bfF_q$; for $a=\infty$, both $f_1|_{L_\infty}$ and $f_2|_{L_\infty}$ are conjugate to $y\mapsto y^q$. Moreover
\begin{itemize}
\item $a=\infty$ is the unique parameter  such that $f_1|_{L_a}$  is conjugate to $y\mapsto y^q$ by an affine map $y \mapsto \alpha y + \beta$;
\item $a=0$ is the unique parameter such that $f_1|_{L_a}$ is conjugate to $y\mapsto y^q-y$ by an affine map;
\item $a=1$  is the unique parameter such that $f_1|_{L_a}$ is conjugate to $y\mapsto y^q+y^p$ by an affine map.
\end{itemize}
And the same properties hold for $f_2$.
As a consequence, we obtain $h(L_{\infty})=L_{\infty}$, $h(L_0)=L_0$ and $h(L_1)=L_1$; this means that there are coefficients  $\alpha \in {\overline{\bfF_q}}^*$ and $\beta, \gamma\in {\overline{\bfF_q}}$ such that 
$h(x,y)= (x, \alpha y+\beta x+\gamma)$.
Writing down the relation $h\circ f_1=f_2\circ h$ we obtain the relation
\begin{align}
\alpha y^q+\alpha G(x,y)+\beta x^q+\gamma = &\;  \alpha^q y^q+\beta^qx^q+\gamma^q\\
&+ G(x, \alpha y+\beta x+\gamma +P(x)).
\end{align}
We note that $1<\deg G(x,y)<\deg G(x, \alpha y+\beta x+\gamma +P(x))<q$.
Compare the terms of degree $q$, we get $\alpha y^q+\beta x^q=\alpha^q y^q+\beta^qx^q.$
It follows that 
\begin{align}
\alpha G(x,y)+\gamma =  \gamma^q+ G(x, \alpha y+\beta x+\gamma +P(x)).
\end{align}
Then $\deg G(x,y)=\deg G(x, \alpha y+\beta x+\gamma +P(x))$, which is a contradiction.


%\begin{align}
%\alpha y^q + \alpha xy^p+\alpha (x-1) y +\beta  = &\;  \alpha^q y^q + \alpha^p x y^p + \alpha (x-1) y \\ 
%& + x (\beta^p+\beta+P(x)^p+P(x))-P(x)-\beta.
%\end{align}
%Looking at the highest monome $x^d$, with $d=p\deg(P)$, we obtain a contradiction.


%\subsection{7.1.  An example in positive characteristic} Assume that $q=p^s$ with $s\geq 2$.
%Set $G:=xy^p+(x-1)y$. Then, 
%$$f_1=(x^q, y^q+G(x,y))$$
%is an endomorphism of $\bbA^2$ that extends to an endomorphism of $\P^2$.
%
%Consider a polynomial $P(x)\in\bfF_q[x]$ such that  $2 \leq \deg(P) \leq \frac{q}{p}-1$ and $P(0)=0$.
%Observe that $\deg (G) < \deg (G(x,y+P(x))) < q$.
%Then $g(x,y)= (x,y-P(x))$
%is an automorphism of $\bbA^2_\bfk$ that conjugates $f_1$ to 
%\begin{align}f_2(x,y) &:=g\circ f_1\circ g^{-1}(x,y) \notag \\
%&=(x^q,y^q+P(x)^q+G(x,y+P(x))-P(x^q))\\
%&=(x^q,y^q+G(x,y+P(x))).\notag
%\end{align}
%As $f_1$, $f_2$ is an endomorphism of $\bbA^2$ that extends to a regular endomorphism of~$\P^2$
%(here we use the inequality $\deg (G(x,y+P(x))) < q$).
%
%Let us prove that $f_1$ and $f_2$ are not conjugate by any automorphism of $\P^2$.
%So, assume that there exists $h\in \PGL_3(\overline{\bfF_q})$ such that $h\circ f_1=f_2\circ h$.
%%Then $h$ acts on set of lines in $\P^2$.
%
%Consider the pencils of lines through the point $[0:1:0]$ in $\P^2$; for $a\in \bfF_q$ we denote by 
%$L_a$ the line $\{x=az\}$ and we denote by $L_\infty$ the line $\{ z=0\}$. 
%Then 
%\begin{align}\{L_a \; ; \; a\in \bf F_q\cup {\infty}\} & = \{{\text{lines}} \ L \ {\text{such that}}\  f_1^{-1}L=L\} \\
%&  =\{{\text{lines}} \ L \ {\text{such that}}\ f_2^{-1}L=L\};
%\end{align}                  
%in other words, the lines $L_a$ for $a\in \bfF_q\cup\{\infty\}$ are exactly the lines which are totally invariant under
%the action of $f_1$ (resp. of $f_2$).
%Since $h$ conjugates $f_1$ to $f_2$, it permutes these lines. In particular, $h$ fixes the point $[0:1:0]$, and if 
%we identify $L_a\cap \bbA^2$ with $\bbA^1$ with coordinate $y$ by the parametrization $y\mapsto (a,y) $ then 
%$h$ maps $L_a$ to another line $L_{a'}$ in an affine way: $h(a,y)=(a',\alpha y +\beta)$.
%
%%For  $a\in F_q\cup {\infty}$, we may check that
%%$f_1|_{L_a}$ conjugates to $f_2|_{L_a}$
%%and also $f_1|_{l_a}$ conjugates to $f_2|_{h(l_a)}$.
%
%We note that 
%\begin{itemize}
%\item $a=\infty$ is the unique parameter  such that $f_1|_{L_a}$ is conjugate to $y\mapsto y^q$ by an affine map $y \mapsto \alpha y + \beta$;
%\item $a=0$ is the unique parameter such that $f_1|_{L_a}$ is conjugate to $y\mapsto y^q-y$ by an affine map;
%\item $a=1$  is the unique parameter such that $f_1|_{L_a}$ is conjugate to $y\mapsto y^q+y^p$ by an affine map.
%\end{itemize}
%And the same properties hold for $f_2$: here we use the condition $P(0)=0$ in case $a=0$, and the fact that $y\mapsto y^q+y^p$ is conjugate
%to $y\mapsto y^q+y^p+P(1)^p$ by any translation $y\mapsto y+\beta$ satisfying $\beta^q+\beta^p-\beta=P(1)^p$.
%
%As a consequence, we obtain $h(L_{\infty})=L_{\infty}$, $h(L_0)=L_0$ and $h(L_1)=L_1$; this means that
%$h$ takes form $(x,y)\mapsto (x, \alpha y+\beta)$ for some fixed parameters  $\alpha \in {\overline{\bfF_q}}^*$ and $\beta\in {\overline{\bfF_q}}$.
%Writing down the relation $h\circ f_1=f_2\circ h$ we obtain the relation
%\begin{align}
%\alpha y^q + \alpha xy^p+\alpha (x-1) y +\beta  = &\;  \alpha^q y^q + \alpha^p x y^p + \alpha (x-1) y \\ 
%& + x (\beta^p+\beta+P(x)^p+P(x))-P(x)-\beta.
%\end{align}
%Looking at the highest monome $x^d$, with $d=p\deg(P)$, we obtain a contradiction.






%\newpage
\bibliographystyle{plain}
\bibliography{dd}

\end{document}

 