\documentclass[1 [leqno, 11pt]{amsart}
\usepackage{amssymb,  amsmath, amsmath, latexsym, amssymb, amsfonts, amsbsy, amsthm,mathtools, graphicx, CJKutf8, CJKnumb, CJKulem, extarrows, color, dsfont,verbatim}
\usepackage{amssymb, amsmath,amsmath,latexsym,amssymb,amsfonts,amsbsy, amsthm,mathtools,graphicx,CJKutf8,CJKnumb,CJKulem,color}
\usepackage{graphicx,epsfig}
\usepackage{graphicx}
\usepackage{amssymb}
\usepackage{graphicx}
\usepackage{graphicx,epsfig}
\usepackage{amsmath}
\usepackage{mathrsfs}
\usepackage{amssymb}


\usepackage{amssymb,mathrsfs,amsfonts,amsmath,amsbsy,tikz}\usepackage{fontenc}\usepackage{textcomp}

\usepackage{amssymb,  amsmath, amsmath, latexsym, amssymb, amsfonts, amsbsy, amsthm,mathtools, graphicx, CJKutf8, CJKnumb, CJKulem, extarrows, color, dsfont,verbatim}
%%%%%%%%%%%%%%%%%%%%%%%%%%%%%%%%%%

%\usepackage{pdfsync}
\setlength{\oddsidemargin}{0mm}
\setlength{\evensidemargin}{0mm} \setlength{\topmargin}{0mm}
\setlength{\textheight}{220mm} \setlength{\textwidth}{155mm}

\renewcommand{\theequation}{\thesection.\,\arabic{equation}}
\numberwithin{equation}{section}

\allowdisplaybreaks
%%%%%%%%%%%%%%%%%%%%%%%%%%%%%%%%%%

%%%My setting%%%%%%%%%%%%%%%%%%%%%%%%%%%%%%%%%%%%%%%%%%%%%%%%%%%

%%%ABREVIATIONS%%%%%%
\let\al=\alpha
\let\b=\beta
\let\g=\gamma
\let\d=\delta
\let\e=\varvarepsilon
\let\z=\zeta
\let\la=\lambda
\let\r=\rho
\let\s=\sigma
\let\f=\frac
\let\p=\psi
\let\om=\omega
\let\G= \Gamma
\let\D=\Delta
\let\La=\Lambda
\let\S=\Sigma
\let\ep=\epsilon
\let\Om=\Omega
\let\wt=\widetilde
\let\wh=\widehat
\let\convf=\leftharpoonup
\let\tri=\triangle
\let\na=\nabla
\let\th=\theta
\let\pa=\partial


%%LETTRES RONDES%%
\def\cA{\mathcal{A}}
\def\cB{\mathcal{B}}
\def\cC{{\cal C}}
\def\cD{\mathcal{D}}
\def\cE{{\cal E}}
\def\cF{{\cal F}}
\def\cG{{\mathcal G}}
\def\cH{{\cal H}}
\def\cI{{\cal I}}
\def\cJ{{\cal J}}
\def\cK{{\cal K}}
\def\cL{{\cal L}}
\def\cM{{\mathcal M}}
\def\cN{{\cal N}}
\def\cO{{\cal O}}
\def\cP{{\cal P}}
\def\cQ{{\cal Q}}
\def\cR{{\mathcal R}}
\def\cS{{\mathcal S}}
\def\cT{\mathcal{T}}
\def\cU{{\cal U}}
\def\cV{{\cal V}}
\def\cW{{\cal W}}
\def\cX{{\cal X}}
\def\cY{{\cal Y}}
\def\cZ{{\cal Z}}
\def\sech{{\rm{sech}}}


%%MACROS SANS ARGUMENTS%%%%%%%%%%%%%%%%%
\def\tw{\widetilde{w}}
\def\tB{\widetilde{B}}
\def\tD{\widetilde{D}}
\def\tA{\widetilde{A}}
\def\tT{\widetilde{T}}
\def\tL{\widetilde{L}}
\def\tH{\widetilde{H}}
\def\ta{\widetilde{a}}
\def\tb{\widetilde{b}}
\def\tc{\widetilde{c}}
\def\tg{\widetilde{g}}
\def\tJ{\widetilde{\cJ}}
\def\cp{c''}
\def\h{\frak h}
\def\l{\frak l}
\def\R{\mathbf R}
\def\Z{\mathbf Z}
\def\tphi{\widetilde{\phi}}

\def\ds{\displaystyle}
\def\no{\noindent}

\def\Di{{\bf D}}
\def\dv{\mbox{div}}
\def\dive{\mathop{\rm div}\nolimits}
\def\curl{\mathop{\rm curl}\nolimits}
\def\Ds{\langle D\rangle^s}
\def\eqdef{\buildrel\hbox{\footnotesize def}\over =}
\def\ef{\hphantom{MM}\hfill\llap{$\square$}\goodbreak}
\def\tep{\theta_\epsilon}
\def\gep{\gamma_\epsilon}
\def\bbT{\mathbb{T}}
\def\mF{\mathrm{F}}
\def\mf{\mathrm{f}}

\def\mI{\mathrm{III}}
\def\wml{\mathrm{\widetilde{III}}}



\newcommand{\beq}{\begin{equation}}
\newcommand{\eeq}{\end{equation}}
\newcommand{\ben}{\begin{eqnarray}}
\newcommand{\een}{\end{eqnarray}}
\newcommand{\beno}{\begin{eqnarray*}}
\newcommand{\eeno}{\end{eqnarray*}}





%%theorem%%%%%%%%%%%%%%%%%%%%%%%%%%%%%%%%

\newtheorem{Theorem}{Theorem}[section]

\newtheorem{theorem}{theorem}[section]
\newtheorem{definition}[Theorem]{Definition}
\newtheorem{lemma}[Theorem]{Lemma}
\newtheorem{proposition}[Theorem]{Proposition}
\newtheorem{corol}[Theorem]{Corollary}
\newtheorem{remark}[Theorem]{Remark}
%%

\newtheorem{Definition}[Theorem]{Definition}
\newtheorem{Lemma}[Theorem]{Lemma}
\newtheorem{Proposition}[Theorem]{Proposition}
\newtheorem{Corollary}[Theorem]{Corollary}
\newtheorem{Remark}[Theorem]{Remark}
%%%%%%%%%%%%%%%%%%%%%%%%%%%%%%%%%%%%%%%%%%%%%%

%%%%%%%%%%%%%%end of my setting%%%%%%%%%%%%%%%%%%%%%%%%%%%%%%%%%%%%%%
\begin{document}
\begin{CJK*}{GBK}{song}
\title[Stability and instability of Kelvin-Stuart cat's eyes flows]{\textbf {On the  stability and instability of Kelvin-Stuart cat's eyes flows}}
%\begin{center}
%\textcolor[rgb]{0,0,1}{what are known}\\
%\textcolor[rgb]{1,0,0}{what we may do better}\\
%\textcolor[rgb]{0,1,0}{where I have a doubt}\\
%\end{center}


\author{Shasha Liao}
\address{Department of Mathematics, Georgia Institute of Technology, Atlanta, USA}
\email{sliao7@gatech.edu}

\author{Zhiwu Lin}
\address{Department of Mathematics, Georgia Institute of Technology, Atlanta, USA}
\email{zlin@math.gatech.edu}

\author{Hao Zhu}
\address{Department of Mathematics, Nanjing University,  210093, Nanjing, Jiangsu, P. R. China}
\email{haozhu@nju.edu.cn}





\date{\today}

\maketitle

\begin{abstract}
Kelvin-Stuart vortices are classical mixing layer flows with many applications in fluid mechanics,  plasma physics and  astrophysics. We prove that the whole family of Kelvin-Stuart  vortices is nonlinearly stable
for co-periodic perturbations, and linearly unstable for multi-periodic or modulational perturbations.
This verifies a long-standing conjecture  since the discovery  of the Kelvin-Stuart  cat's eyes flows in the 1960s.
Kelvin-Stuart cat's eyes also appear as magnetic islands which are magnetostatic equilibria for the 2D ideal MHD equations in plasmas.
We prove  nonlinear  stability of  Kelvin-Stuart magnetic islands for co-periodic perturbations, and give the first rigorous proof of the coalescence instability,
which is important for magnetic reconnection.
\end{abstract}
\tableofcontents
\section{Introduction}
Consider the 2D Euler equation  for an incompressible inviscid fluid
 \begin{equation}\label{euler}
 \partial_t \vec{u} + (\vec{u}\cdot \nabla) \vec{u}  = -\nabla p, \quad \nabla \cdot \vec{u} = 0,
 \end{equation}
where $\vec{u} = (u_1, u_2)$ is the velocity field  and $p$ is the pressure. We study the fluid in the unbounded domain
$\Omega=\mathbb{T}_{2\pi}\times \mathbb{R}$, where $\mathbb{T}_{2\pi}$ means that the period is $2\pi$ in the $x$ direction.
%The asymptotic behavior of the velocity field  $\vec{u}$ is
%$\vec{u}(t,x,y) \to(\pm1, 0) $ as $y\to\pm\infty$
%for $t\geq0$ and $x\in\mathbb{T}_{2\pi}$.
% By the incompressible condition $\nabla \cdot \vec{u} = 0$, we can define the stream function $\psi$, whose level curves are the  stream lines of the fluid,
The stream function $\psi$  satisfies   $ \vec{u} = \nabla ^\bot \psi = (\psi_y, -\psi_x)$.
 Taking the curl of \eqref{euler} gives the following evolution equation for the scalar-valued vorticity $\omega =- \Delta \psi$:
\begin{equation} \label{vor}
\partial_t \omega +  \{\omega , \psi\} = 0,
\end{equation}
where  $\{\omega , \psi\}  := \partial_y\psi \partial_x \omega - \partial_x\psi\partial_y \omega$ is the canonical Poisson bracket.

In 1967, Stuart \cite{stuart1967finite} found a family of exact solutions  to the 2D steady  Euler equation \eqref{vor}, known as Kelvin-Stuart cat's eyes flows. The stream functions of Stuart's solutions  are given explicitly by
\begin{equation}\label{catseye}
\psi_\epsilon(x,y) =  \ln \left(\frac{\cosh (y) + \epsilon \cos (x)}{\sqrt{1-\epsilon^2}} \right),\quad x\in\mathbb{T}_{2\pi},\quad y\in\mathbb{R}
\end{equation}
 with  the parameter $\epsilon \in [0, 1)$.
  These exact solutions correspond qualitatively to the co-rotating vortices \cite{Tabeling-Perrin-Fauve1987}, and
    describe the mixing process of two currents flowing in opposite directions with the same speed.
  Such cat's eyes flows have many applications. For example, their streamline patterns  are typical for the wave-current interactions in  the ocean \cite{Martin2018}.
  These flows are used for potentially effective mixing strategies in the industry \cite{Rossi-Doorly-Kustrin2013} and are applied to describe the tropical
storm \cite{Dunkerton-Montgomery-Wang2009}.
\if0
   and are also important equilibria in   slab models arising from plasma physics and  planetary rings.
  Independently, in 1965, Schmid-Burgk \cite{Schmid-Burgk1965} also found this family of solutions   when he worked on self-gravitating isothermal gas layer, where \eqref{catseye}  acts  as the scaled  gravitational potential.
  At about the same time,
 Fadeev  {\it et al.} \cite{Fadeev et al-1965} found that the Kelvin-Stuart cat's eyes are  static equilibria  for the 2D ideal MHD equations, where \eqref{catseye} serves as the  magnetic potential, see \eqref{Kelvin-Stuart cat's eyes-mhd-m-p}.
In a plasmas model which takes both the gravitational and the
magnetic fields into account,
  Fleischer \cite{Fleischer1998} obtained a magnetohydrostatic equilibrium of
a self-gravitating plasma, the gravitational potential of which recovers Schmid-Burgk's solutions in the pure gravitational limit and the magnetic flux function of which recovers Fadeev  {\it et al.}'s solutions
in
case of the MHD limit.
   Kelvin-Stuart cat's eyes are also  applied  to understand the
 spatial
structures  in Saturn's ring system \cite{Shukla-Sen1996}, and   when the electron number density
is completely depleted, the electromagnetic
equilibrium of the dust grains  is governed by
the Liouville's equation (see \eqref{elip}), one of whose solution is given as Kelvin-Stuart vortices.
\fi
 The vorticity and  velocity of the  Kelvin-Stuart cat's eyes flows are given by
 \begin{align} \label{steadyw}
\omega_\epsilon =& -\Delta \psi_\epsilon = \frac{-(1- \epsilon^2 )}{(\cosh y + \epsilon\cos x)^2},\\\label{steadyv}
\vec{u}_\epsilon =& (u_{\ep,1},u_{\ep,2})= (\partial_y\psi_{\epsilon }, -\partial_x\psi_{\epsilon })= \left(\frac{\sinh(y)}{\cosh y + \epsilon\cos x}, \frac{\epsilon\sin(x)}{\cosh y + \epsilon\cos x}\right).
\end{align}
  The stream functions  satisfy the  Liouville's equation
 \begin{equation}\label{elip}
  - \Delta \psi_\epsilon = g(\psi_\epsilon)\quad \text{with}\quad g(\psi_\epsilon) =- e^{-2\psi_\epsilon},
 \end{equation}
 where $\ep\in[0,1)$.
 The streamlines for $\ep = 0.5$ are of the form in Figure \ref{fig:firstFig}. Such kind of streamline patterns with the fashion of cat's eyes   were first described by Kelvin \cite{kelvin1880disturbing} in 1880.
  The  Kelvin-Stuart  cat's eyes flow becomes  the  hyperbolic tangent shear flow when $\ep=0$ and tends to  a single row of
co-rotating point vortices  periodically spaced along the $x$-axis when $\ep\to1$:
\begin{itemize}
\item \textbf{Shear case} ($\epsilon = 0$):  $$\psi_0 = \ln(\cosh (y)), \quad \omega_0 =\frac{-1}{\cosh ^2 (y)}, \quad \vec{u}_0 = (\tanh y, 0).$$
\item \textbf{Singular case} ($\epsilon =1$): A point vortex system with vorticity concentrating at these singular points $$\{\cdots,(-3\pi,0),(-\pi,0),(\pi, 0), (3\pi,0),\cdots\}.$$
\end{itemize}
%In this paper, we are interested in the  stability/instability of Kelvin-Stuart  cat's eyes flows for co-periodic ($2\pi$-), multi-periodic ($2m\pi$-) and modulational perturbations, where  $m\geq2$ is an integer.
%Enter cat's eyes picture

\begin{figure}[ht]
    \centering
%\includegraphics[width=0.48\textwidth]{surface.png}
\includegraphics[width=0.48\textwidth]{contour.png}
\caption{Streamlines  for $\ep=0.5$}
	\label{fig:firstFig}
\end{figure}

Stability/instability of Stuart's exact solutions  is of considerable interest since its discovery.
%In the fluid literature, there appear some methods to study the stability/instability of the Kelvin-Stuart vortices, including  restricting  the  domain to be bounded by taking a truncation in the $y$ direction, perturbation from the two limit cases-shear flow and point vortex, asymptotic analysis and numerical simulations.
Some special cases are known.
 In the singular case  $\epsilon = 1$,
% spectral stability of the point vortex system for co-periodic perturbations and linear instability for double-periodic perturbations
%in the sense that the vortices deviate from their original positions were obtained
 Lamb \cite{lamb1932hydrodynamics}
 described the row of point vortex system  and proved that it is linearly  unstable  for double-periodic perturbations.
 \if0
 The  steady state \eqref{catseye} with $\ep \in [0, 1)$ was found by   Schmid-Burgk \cite{Schmid-Burgk1965},  Stuart \cite{stuart1967finite} and Fadeev  {\it et al.} \cite{Fadeev et al-1965},  independently.
 \fi
In the case that  $0<\epsilon \ll 1$,  Kelly \cite{kelly1967stability} numerically observed  that the Kelvin-Stuart vortex is unstable for double-periodic perturbations.
Indeed, for $\ep=0$,  it can be deduced from  \cite{lin2003instability} that
the hyperbolic tangent  flow
 is   unstable for any multi-periodic perturbations.
 Based on Lamb and Kelly's observations for the two extreme cases, in his original paper \cite{stuart1967finite}, Stuart himself conjectured that ``{\it from a stability analysis, the wavelength doubling phenomenon might be
typical for all or many members of the class.}" That is, instability  for double-periodic perturbations might hold true for the whole family of the  Kelvin-Stuart vortex ($\ep$ runs from $0$ to $1$), if not, what is the exact  range of the  parameter $\ep$ such that double-periodic instability is true.
 The  double-periodic instability plays an important role in explaining the vortex pairing in physical phenomenon of  vortex merging.
 In the fluid literature, there exists some numerical evidence supporting Stuart's conjecture. In particular,
Pierrehumbert and Windnall \cite{pierrehumbert1982two} numerically found that double-periodic instability is true for $0\leq \ep\leq 0.3$ and the most unstable eigenvalue is real. Klaassen and Peltier \cite{Klaassen-Peltier1987} observed a slowly growing mode with $\ep= 0.1$ for double-periodic perturbations. It is  pointed out in \cite{Klaassen-Peltier1989} that  triple-periodic instability is also   physically interesting  in the collective amalgamation of vortices, since the unstable modes contribute to  merging   three vortices into  either one or two.


%They formulated  the linear stability problem  as a
%non-separable eigenvalue problem with respect to two independent variables, and  solved
%numerically using spectral methods.
%They found that there are two main classes of
%instabilities:  subharmonic and transversal.
%Instead of the whole line,  they took a truncation in the $y$ direction and posed suitable boundary conditions.

%For $\epsilon =1$, \eqref{catseye} is reduced to a point vortex system. It is well-known \cite{lamb1932hydrodynamics} that the point vortex system is stable for co-periodic perturbations and unstable for double-periodic perturbations .
\if0
  Kelvin-Stuart's cat's eyes  satisfy the  elliptic equation
 \begin{equation}\label{elip}
  - \Delta \psi_\epsilon = g(\psi_\epsilon)\quad \text{with}\quad g(\psi_\epsilon) =- e^{-2\psi_\epsilon},
 \end{equation}
 where $\ep\in[0,1)$. \eqref{elip} is a special form of the following Liouville's equation
\begin{equation}\label{Liouville's equation}
   \Delta \phi = c_1 e^{c_2\phi},
 \end{equation}
where $c_1$ and $c_2$ are real parameters. Liouville's equation has important applications in fluid dynamics, space plasma physics, high energy physics and differential geometry.  Such equations and their generalizations have attracted considerable   attention   since Liouville's paper \cite{Liouville1853} in 1853, and    stimulated numerous works (see, for example, \cite{Poincare18998,Picard1,Picard3,lichtenstein1913,lichtenstein1915,Walker1915,Bieberrach1916,Brodetsky1924}) in mathematical physics. \eqref{Liouville's equation} can be  derived  from  the Gaussian curvature of a surface in isothermal coordinates \cite{Lutzen2012}. It also appears in the theory of the space charge of electricity round a glowing wire \cite{Richardson1921} and in the  quantum geometry of bosonic strings \cite{Polyakov81}. Liouville's equation was considered as an example by  Hilbert  in the formulation of his 19-th problem \cite{Hilbert1900}.  In plasma physics, Liouville's equation is taken as  a special form of Grad-Shafranov equation, which is
the equilibrium equation in ideal magnetohydrodynamics  for a two dimensional plasma \cite{Schindler2006}.
It also occurs in the magnetohydrostatic model of the
earth's magnetosphere \cite{Schindler2006} and in the study of isothermal gas
spheres in astrophysics \cite{Emden1907}.
Many exact solutions of Liouville's equation, including the Kelvin-Stuart cat's eyes, have been obtained in the literature, see \cite{Crowdy97} and references therein. In particular,
Liouville found a class of solutions, which are generated by analytic functions,  of
\eqref{Liouville's equation} for $c_1c_2<0$ \cite{Liouville1853}, while
\eqref{Liouville's equation} has no solutions valid in the whole plane for $c_1c_2>0$ \cite{Keller1957,Wittich44}.
%More solutions and their applications of the Liouville's equation can be found in \cite{Liouville1853,Schindler2006}.
See more discussions on Liouville's equation and its solutions in \cite{Bogatov-Kichenassamy22,Bateman32}.
\fi

\if0
As the parameter $\ep$ changes from $0$ to $1$, the distribution of $\omega_\ep$ changes. However, the total vorticity within $\Omega$ remains unchanged since $\iint_\Omega \omega_\ep dxdy = -4\pi$ does not depend on $\epsilon$. We find a change of variables $(\theta_\ep,\gamma_\ep)$, and  the domain $\Omega$ is then transformed  to  $\mathbb{T}_{2\pi}\times [-1,1]$ on which the streamlines are shown in Figure \ref{fig:secondFig}, where $\theta_\ep$ and $\gamma_\ep$ are defined in \eqref{transf1}-\eqref{transf2}.  The upper and lower untrapped regions in the original variables $(x,y)$ are transformed to  the four corner regions of the rectangle $\mathbb{T}_{2\pi}\times [-1,1]$, while the trapped region occupies most of the areas located in the middle.


\begin{figure}[ht]
    \centering
	\includegraphics[width=0.8\textwidth]{Figure-streamlines-new.png}
	\caption{Streamlines of the cat's eyes flow in $(\theta_\ep,\gamma_\ep)$ coordinate with $\ep = 0.5$}
	\label{fig:secondFig}
\end{figure}
\fi





%They then   provided a  close correspondence to the
%growth rates predicted by von K$\acute{a}$rm$\acute{a}$n's  analysis \cite{lamb1932hydrodynamics} of the stability of  point vortices  in the limit $\ep\to1^-$, and gave different growth rates from Kelly's prediction in the limit $\ep\to0^+$ for modulational perturbations \cite{Klaassen-Peltier1989}.

For co-periodic perturbations,
Holm, Marsden and Ratiu \cite{holm1986nonlinear} considered a truncated domain bounded by a pair of steady streamlines, and proved nonlinear stability of Kelvin-Stuart vortices for a certain range of  $\ep$-parameter, which depends on the domain's size.
% Here, the  perturbation of vorticity is
% in the  $L^2$ norm.
Even for the truncated domain, their stability result can not be extended to  the whole family of  Kelvin-Stuart vortices. For example,
in the domain bounded exactly by the separatrices (i.e. the trapped region), nonlinear stability holds true only for $\ep\in[0,\ep_0]$ according to their theory, where $\ep_0\approx0.525$. In the truncated domain, they also
proved nonlinear stability of Kelvin-Stuart vortices for double-periodic perturbations, where the allowed   range of  $\ep$-parameter becomes smaller.
They speculated that the  reason for the potential instability is that the domain is not truncated in the $y$ direction.
In the original unbounded domain $\Omega$, even the linear stability/instability of the whole family of Kelvin-Stuart vortices is unknown for co-periodic perturbations.
%Their method is however invalid in a domain with sufficiently large size or the
%original unbounded domain, since  this prevents
%their estimates, including the Poincar\'e-type inequality, from being carried out.
%A possible  method to prove the spectral stability of  Kelvin-Stuart vortices, as mentioned in  \cite{Klaassen-Peltier1989}, is to study the
% perturbation of the linearized operator or the energy quadratic form   from the hyperbolic tangent shear flow ($\ep=0$) and the point vortices ($\ep=1$). This method might be applicable to the Kelvin-Stuart vortices near the shear flow or the point vortices  but is difficult to extend to the whole family. Another difficulty is that the perturbation of the point vortices is singular and needs to be carefully treated.
It is thus widely open to prove/disprove the nonlinear stability of such a family of steady states for co-periodic perturbations  in the original setting.

In the present paper, we prove Stuart's conjecture and solve  the above open problem rigorously.
More precisely,  we prove that the whole family of Kelvin-Stuart  vortices is
 linearly unstable for any multi-periodic  perturbations,
 and
 nonlinearly stable
for co-periodic perturbations in the original  unbounded domain $\Omega$.
Moreover, we prove linear modulational instability for the whole family of Kelvin-Stuart  vortices, which is stronger than multi-periodic instability. The modulational perturbations of the vorticity take the form ${\omega}(x, y) e^{i\alpha x}$, where $\omega$ is $2\pi$-periodic in $x$ and $\alpha \in \mathbb{R}\setminus\mathbb{Z}$.
Modulational instability was well-known in the setting of water waves, first observed by Benjamin and Feir \cite{Benjamin-Feir1967} for the small-amplitude Stokes waves (steady water waves in a moving frame). For  the linear modulational instability of the small-amplitude Stokes waves,   rigorous proofs in  finite and infinite depth  were obtained by Bridges-Mielke \cite{Bridges-Mielke1995}, Nguyen-Strauss \cite{Nguyen-Strauss2023} and Berti-Maspero-Ventura \cite{Berti-Maspero-Ventura2022}.
Chen and Su \cite{Chen-Su2020} proved  nonlinear modulational instability for the small-amplitude Stokes waves with infinite depth.
 Modulational instability has been studied  in various dispersive  wave models and we refer
to the  survey \cite{Bronski-Hur-Johnson2016} for more details. For a class of dispersive models, it was proved in \cite{Jin-Liao-Lin2019} that  linear   modulational instability implies nonlinear instability.

\medskip

\noindent{\bf{Main results for the 2D Euler equation.}} First, we provide a complete answer to Stuart's conjecture.

\begin{Theorem}\label{main result2-multi-periodic perturbations}
Let
$0 \leq \ep < 1$. Then the steady state $\omega_\ep$ in \eqref{steadyw} is linearly unstable for $2m\pi$-periodic perturbations, where $m\geq2$ is an integer.
\end{Theorem}


%That is, we prove linear instability of the whole family of Kelvin-Stuart vortices for any multi-periodic perturbations.
Linear instability for multi-periodic perturbations  implies modulational instability for some but not all rational modulational parameters, and thus far from all modulational  parameters. Our next  result is to cover all modulational parameters, which is stronger  than Theorem \ref{main result2-multi-periodic perturbations}.

\begin{Theorem}\label{main result3-modulation-unstable}
Let $0 \leq \ep < 1$. Then the steady state $\omega_\ep$ in \eqref{steadyw} is linearly modulationally unstable  for all $\alpha\in\mathbb{R}\setminus\mathbb{Z}$.
\end{Theorem}

Based on Theorems \ref{main result2-multi-periodic perturbations}-\ref{main result3-modulation-unstable}, it is expected to prove nonlinear instability for  multi-periodic or localized perturbations. To prove nonlinear instability for  localized perturbations in $\mathbb{R}^2$,
one may construct the unstable initial data in the form
$\omega_\ep(x,y)+2Re(\int_I\omega_u(\alpha;,x,y) e^{i\alpha x}d\alpha)$, where
$I$ is a small interval near the most unstable frequency $\alpha_0$,
 $\omega_u(\alpha;,x,y)$ is an eigenfunction of the eigenvalue $\lambda(\alpha)$ for the linearized operator  $J_{\ep,\alpha} L_{\ep,\alpha}$,
 $\{\lambda(\alpha):\alpha\in I\}$ is a curve of  unstable eigenvalues   bifurcating from the most unstable eigenvalue $\lambda(\alpha_0)$,   and $J_{\ep,\alpha}, L_{\ep,\alpha}$ are defined in \eqref{def-J-ep-al}-\eqref{def-L-ep-al}.





Then we prove  stability of the whole family of Kelvin-Stuart vortices for co-periodic perturbations.  Let us first state
our  result at the linear level.

\begin{Theorem}\label{main result1-co-periodic perturbations}
Let $0 \leq \ep < 1$. Then
the steady state $\omega_\ep$ in \eqref{steadyw} is spectrally stable for co-periodic perturbations.
\end{Theorem}

Based on spectral stability in Theorem \ref{main result1-co-periodic perturbations}, our main result for co-periodic perturbations is  that the whole family of Kelvin-Stuart vortices is nonlinear orbitally stable.

\begin{Theorem}\label{main result4-nonlinear orbital stability}
Let $\ep_0 \in (0, 1)$. For any $\kappa>0$, there exists $\delta=\delta(\ep_0,\kappa)>0$ such that if
$$\inf_{(x_0,y_0)\in\Omega} d(\tilde \omega_0,\omega_{\ep_0}(x+x_0,y+y_0))+\inf_{(x_0,y_0)\in\Omega}\|\tilde \omega_0-\omega_{\ep_0}(x+x_0,y+y_0)\|_{L^2(\Omega)}<\delta,$$
 then for any $t\geq0$, we have
\begin{align}\label{onlinear orbital stability-goal}
\inf_{(x_0,y_0)\in\Omega}d(\tilde \omega(t),\omega_{\ep_0}(x+x_0,y+y_0))<\kappa,
\end{align}
where $\tilde \omega(t)=\curl(\vec{v}(t))$, $\vec{v}(t)$ is a weak solution to the nonlinear 2D Euler equation \eqref{euler} with the initial vorticity
 \begin{align}\label{def-X-non-ep}
\tilde \omega(0)=\tilde \omega_0\in Y_{non}= \left\{ \tilde \omega|\tilde\omega\in L^1(\Omega)\cap L^2(\Omega),
 %\tilde \omega\ln(-\tilde \omega)\in L^1(\Omega),
  y\tilde \omega\in L^1(\Omega),\tilde \omega<0,\iint_{\Omega}\tilde \omega dxdy=-4\pi\right\}.
\end{align} The distance functional $d$ is defined by
\begin{align}\nonumber
d(\tilde\omega,\omega_\ep)&=\iint_{\Omega}(h(\tilde \omega ) - h(\omega_\ep)  - \psi_\ep(\tilde  \omega-\omega_\ep)+(G*(\tilde \omega-\omega_\ep))(\tilde \omega-\omega_\ep))dxdy, \quad\tilde \omega\in Y_{non},
\end{align}
where $h(s)={1\over2}(s-s\ln(-s))$ for $s<0$ and $G(x,y)=-{1\over 4\pi}\ln(\cosh(y)-\cos(x))$.
\end{Theorem}

Since the  velocity of the Kelvin-Stuart cat's eyes flow converges to
$ (\pm1, 0) $ as $y$ goes to $\pm\infty$
 for $x\in\mathbb{T}_{2\pi}$ and $\epsilon\in[0,1)$, physically we  consider perturbed flows with the same asymptotic behavior of  the  velocity, which implies that
 %the total perturbed vorticity is $-4\pi$ (see \eqref{perturbed vorticity condition}).
%Moreover, the  perturbation of vorticity is considered in the weakest topology such that  the energy quadratic form $\langle L_\ep\cdot,\cdot\rangle$ defined in \eqref{J-ep-J-ep-def} is finite. Combining the above two points, the space of the vorticity's perturbations is defined in
we need the constraint  $\iint_{\Omega}\tilde \omega dxdy=-4\pi$ in the space $Y_{non}$.  The sign-constraint  $\tilde \omega<0$ in $Y_{non}$ ensures  that the Casimir functional $\iint_\Omega h(\tilde \omega)dxdy$ is well-defined.


Stuart-type solutions \eqref{catseye}-\eqref{steadyv} have many other applications in plasma physics and astrophysics.
  Independently, in 1965, Schmid-Burgk \cite{Schmid-Burgk1965}  found this family of solutions   when  working on self-gravitating isothermal gas layer, where \eqref{catseye}  acts  as the scaled  gravitational potential.
  At about the same time,
 Fadeev  {\it et al.} \cite{Fadeev et al-1965} also found that the Kelvin-Stuart cat's eyes are  static equilibria  for the 2D ideal MHD equations, where \eqref{catseye} serves as the  magnetic potential, see \eqref{Kelvin-Stuart cat's eyes-mhd-m-p}.
For a plasmas model which takes both the gravitational and the
magnetic fields into account,
  Fleischer \cite{Fleischer1998} obtained a magnetohydrostatic equilibrium of
a self-gravitating plasma, the gravitational potential of which recovers Schmid-Burgk's solutions in the pure gravitational limit and the magnetic flux function of which recovers the solutions found by Fadeev  {\it et al.} 
in
case of the MHD limit.


Next, we study stability/instability of  the  magnetic islands of Kelvin-Stuart type found by  Fadeev  {\it et al.} in \cite{Fadeev et al-1965}.
We
 consider
the planar incompressible magnetohydrodynamics (MHD) in the  unbounded domain $\Omega$.
In the incompressible MHD approximation,  plasma motion in  3D is governed by
\begin{align*}
\partial_t \vec{v}+\vec{v}\cdot\nabla\vec{v}=-\nabla p+\vec{J}\times \vec{B},\quad \partial_t\vec{B}=-\text{curl}(\vec{E}),\quad \text{div} (\vec{B})=0,\quad\text{div}(\vec{v})=0,
\end{align*}
where $\vec{v}$ is the fluid velocity, $p$ is the pressure, $\vec{B}$ is the magnetic field,  $\vec{J}=\text{curl}(\vec{B})$ is the electric current density,  and $\vec{E}=-\vec{v}\times \vec{B}$ is the electric field. We are interested in the incompressible MHD taking place on the planar domain $\Omega$. The velocity field and the magnetic field in the $xy$ plane are still denoted by $\vec{v}$ and $\vec{B}$,
and  the scalar  vorticity $\omega$ and the scalar  electrical current density $J$ are given by $\omega=-\nabla^\bot  \cdot\vec{v}$ and $J=-\nabla^\bot  \cdot \vec{B}$.
%The asymptotic behaviors of the velocity field   $\vec{v}$ and the magnetic field $\vec{B}$ are
%$\vec{v}(t,x,y) \to(0, 0) $ and $\vec{B}(t,x,y) \to(\pm1, 0) $  as $y\to\pm\infty$
%for $t\geq0$ and $x\in\mathbb{T}_{2\pi}$.
Since $\text{div}(\vec{v})=\text{div}(\vec{B})=0$, there exist a scalar  stream function $\psi$ and a scalar magnetic potential $\phi$  such that   $\vec{v}=\nabla^{\bot}\psi$ and $\vec{B}=\nabla^{\bot}\phi$.
Then $\omega=-\Delta\psi$ and $J=-\Delta \phi$.
We determine  $\phi
=G*J-\ln\sqrt{1-\epsilon^2}$.
 The planar ideal MHD equations then take the form
 \begin{align}\label{mhd}
\left\{ \begin{array}{lll} \partial_t \phi=\{\psi,\phi\},\\
 \partial_t \omega=\{\psi,\omega\}+\{J,\phi\}.
 \end{array} \right.
\end{align}
%We check that our choice of the magnetic potential satisfies the first equation in \eqref{mhd}.
As is pointed out above, Kelvin-Stuart cat's eyes are founded to be  a family of Grad-Shafranov static equilibria of \eqref{mhd}  by Fadeev  {\it et al.} \cite{Fadeev et al-1965}. The equilibria are given by the Kelvin-Stuart magnetic island solutions $(\omega=0,\phi_{\ep})$, where the steady  magnetic potential
 \begin{align}\label{Kelvin-Stuart cat's eyes-mhd-m-p}\phi_\ep(x,y)=\ln \left(\frac{\cosh (y) + \epsilon \cos (x)}{\sqrt{1-\epsilon^2}} \right),\quad x\in\mathbb{T}_{2\pi},\quad y\in\mathbb{R}
 \end{align}
satisfies
 \begin{align*}
 %\label{Kelvin-Stuart cat's eyes-mhd-1}
J^\epsilon =& -\Delta \phi_\epsilon = \frac{-(1- \epsilon^2 )}{(\cosh y + \epsilon\cos x)^2} =g(\phi_\epsilon),
\\
%\label{Kelvin-Stuart cat's eyes-mhd-2}
\vec{B}^\epsilon =& (B_{1,{\ep}},B_{2.{\ep}})= (\partial_y\phi_{\epsilon }, -\partial_x\phi_{\epsilon })= \left(\frac{\sinh(y)}{\cosh y + \epsilon\cos x}, \frac{\epsilon\sin(x)}{\cosh y + \epsilon\cos x}\right).
\end{align*}
\if0
Here, $g_\ep(\phi_\epsilon)=-(1-\ep^2) e^{-2\phi_\epsilon}$.
%Kelvin-Stuart cat's eyes are also Grad-Shafranov equilibria of the planar ideal MHD equations, where $\psi_{\ep}$ is the scalar magnetic potentials  of the static equilibria.

It is more convenient to use $\ln \left(\cosh (y) + \epsilon \cos (x) \right)$ instead of $\ln \left(\frac{\cosh (y) + \epsilon \cos (x)}{\sqrt{1-\epsilon^2}} \right)$ as the  magnetic potential $\phi_\ep$ of the steady state in the MHD case, since $\ln \left(\cosh (y) + \epsilon \cos (x) \right)=G*J^{\ep}$ and $G*\tilde J$ satisfies \eqref{mhd}, where $G$ is defined in \eqref{green function} and $\tilde J$ is the perturbed electrical current density.  See \eqref{steady-mhd} and \eqref{constant-G-con-tildephi}.
\fi

For a chain of magnetic islands  in a current slab, neighboring islands have a tendency to merge in the nonlinear evolution. Such coalescence instability has important applications in magnetic reconnection and we refer to surveys in \cite{Pontin-Priest2022,Priest1985,Priest-Forbes2000} for more details. At the linear level, the coalescence instability corresponds to linear double-periodic instability of $(\omega=0,\phi_{\ep})$.
Finn and Kaw \cite{Finn-Kaw1977} numerically found that these magnetic island solutions are coalescence unstable for $\ep$ not close to $0$, and moreover, they predicted a threshold of  coalescence instability at $\ep$. Namely, there exists $\ep_0\in(0,1)$ such that the coalescence instability occurs only for $\ep\in(\ep_0,1)$ and stability arises for $\ep\in[0,\ep_0]$.
\if0
This kind of instability   and describes the process that parallel currents in neighboring islands tend to coalesce into larger units.
\fi
By treating the coalescence process as an initial-value problem, Pritchett and Wu \cite{Pritchett-Wu1979}  numerically   obtained the  growth rates of instability as $\ep\to0$, and thus, denied the Finn-Kaw hypothesis of an instability  threshold. Later, Bondeson  \cite{Bondeson1983} confirmed the coalescence instability of the  Kelvin-Stuart magnetic islands for small $\ep$.
There is, however, no rigorous proof of the  coalescence instability for the whole family of Kelvin-Stuart magnetic islands.

For  co-periodic perturbations,
similar to the 2D Euler case \cite{holm1986nonlinear},
Holm {\it et al.} \cite{holm1985nonlinear} considered a truncated domain bounded by a pair of level curves of the steady magnetic potentials,
and proved nonlinear stability of Kelvin-Stuart
   magnetic islands for a certain range of  $\ep$-parameter.
 In particular, when the domain is the trapped region, they proved
 nonlinear stability of the magnetic islands  for  $\ep\in[0,0.525]$.
  In a model of the hot-ion limit,
 Tassi \cite{Tassi2022} considered the  same domain and
    obtained nonlinear stability of
 the
   magnetic island solution
 for  $\ep\in[0,0.223]$.
 It is still an open problem to prove nonlinear  stability of the whole family of Kelvin-Stuart magnetic islands for co-periodic perturbations.
Holm {\it et al.} \cite{holm1985nonlinear} argued that the coalescence  instability in \cite{Finn-Kaw1977,Pritchett-Wu1979,Bondeson1983} can happen only if  one allows arbitrary disturbances
in the $y$ direction. We will see that it is not the un-truncated domain but the perturbation of double period that causes instability.

 %It is also worthy to mentioning that  Longcope and  Strauss \cite{Longcope-Strauss1993} proved  coalescence instability of an equilibrium in a bounded channel.

\medskip

\noindent{\bf{Main results for the MHD equations.}}
First, we study the stability/instability of the Kelvin-Stuart magnetic islands  $(\omega=0,\phi_{\ep})$ at the linear level.
In particular, we give a rigorous proof of  coalescence instability of the whole family of the magnetic islands.



\begin{Theorem}\label{main result1-mhd-all1} Let $0 \leq \ep < 1$.  Then

$(1)$
 the  magnetic island solution $(\omega=0,\phi_{\ep})$ is linearly unstable for double-periodic perturbations.

$(2)$
 the  magnetic island solution $(\omega=0,\phi_{\ep})$  is spectrally stable for co-periodic perturbations.
\end{Theorem}

Then we prove nonlinear orbital  stability of the whole family of Kelvin-Stuart magnetic islands for co-periodic perturbations.




\begin{Theorem}\label{main result1-mhd-all}
 Assume that

 $({\rm i})$ for the initial data $\tilde \omega(0)=\tilde \omega_0\in\tilde Y$ and $\tilde \phi(0)=\tilde \phi_0\in\tilde Z_{non,\ep}$,  there exists a global weak solution $(\tilde \omega(t),\tilde \phi(t))$ in the distributional sense to the nonlinear MHD equations \eqref{mhd} such that $\tilde \omega(t)\in\tilde Y$ and $\tilde \phi(t)\in \tilde  Z_{non,\ep}$ for $t\geq0$,


 $({\rm ii})$ the distance functional $\hat d((\tilde \omega(t),\tilde \phi(t)),(0,\phi_{\ep}))$
is  continuous on $t$,

$({\rm iii})$  the energy-Casimir functional $\hat H$ satisfies that  $\hat H(\tilde \omega(t),\tilde \phi(t))\leq \hat H(\tilde \omega(0),\tilde \phi(0))$  and $\iint_\Omega e^{-j\tilde\phi(t)}dxdy$ is conserved for $t\geq0$ and $j=2,3$. \\
Let $\ep_0\in(0,1)$. For any $\kappa>0$, there exists $\delta=\delta(\ep_0,\kappa)>0$ such that if
\begin{align}\label{initial data-mhd}
\inf_{(x_0,y_0)\in\Omega} \hat d((\tilde\omega_0,\tilde \phi_0),(0,\phi_{\ep_0}(x+x_0,y+y_0)))+\left|\iint_{\Omega}(e^{-2\tilde \phi_0}-e^{-2\phi_{\ep_0}})dxdy\right|<\delta,
 \end{align}
  then
  for any $t\geq0$, we have
\begin{align}\label{onlinear orbital stability-goal-mhd}
\inf_{(x_0,y_0)\in\Omega} \hat d((\tilde\omega(t),\tilde \phi(t)),(0,\phi_{\ep_0}(x+x_0,y+y_0)))<\kappa,
\end{align}
where the distance  $\hat d$ is defined in \eqref{distance-mhd}, the functional $\hat H$ is defined in \eqref{EC-functional-mhd}, and the spaces $\tilde Y, \tilde Z_{non,\ep}$ are defined in \eqref{tilde Y-space}, \eqref{def-Z-non-ep}, respectively.
\end{Theorem}

%Thanks to the Casimir functional $\iint_\Omega -{1\over2}e^{-2\tilde \phi} dxdy$, there is no sign-constraint for the magnetic potential $\tilde \phi$ in Theorem \ref{main result1-mhd-all}.

\medskip

\noindent{\bf{Main ideas in the proof.}}
\medskip

\noindent{\it{Proof of spectral  stability of Kelvin-Stuart vortices for co-periodic perturbations}}: It is challenging to study linear stability of general non-parallel flows.
Our starting point for the Kelvin-Stuart vortices is that  the linearized vorticity equation around $\omega_\ep$ has
 the following  Hamiltonian structure
 \begin{equation}\label{hami}
 \partial_t \omega = J_\epsilon L_\epsilon\omega, \quad \omega \in X_\ep,
 \end{equation}
 where
 \begin{align}\label{J-ep-J-ep-def}J_\epsilon = -g'(\psi_\epsilon)\vec{u}_\epsilon\cdot\nabla: X_\ep^* \supset D(J_\epsilon) \rightarrow X_\ep, \quad
 L_\epsilon = \frac {1} {g'(\psi_\epsilon)} - (-\Delta)^{-1}: X_\ep \rightarrow X_\ep^*,\end{align}
\begin{align}\label{spaceXep}
X_\ep = \left\{\omega\bigg| \iint_\Omega \frac{|\omega|^2}{g'_\epsilon(\psi_\epsilon)} dxdy < \infty, \iint_\Omega \omega dxdy = 0 \right\},\quad \epsilon\in[0,1),
\end{align}
and $(-\Delta)^{-1}\omega$ is clarified in Lemmas \ref{1-1correspond} and \ref{1-1correspond-ep}.
The constraint $\iint_\Omega\omega dxdy=0$ in $X_\ep$ is   again due to the asymptotic behavior of the velocity.
Unlike  the truncated  domain in \cite{holm1986nonlinear},  we need to  make some fundamental  modifications to  deal with  the lack of compactness in the original unbounded domain $\Omega$. Such modifications include introducing  two weighted  Poincar\'e-type inequalities (see \eqref{Poincare inequality I-ep22},  \eqref{Poincare inequality II-ep22}) in a new Hilbert space $\tilde X_\ep$ (see \eqref{tilde-X-e}) of the stream functions.
Hamiltonian structure of the linearized vorticity operator \eqref{hami} enables us to adopt  the index  formula
 \begin{align}\label{index-formula-stuart}
k_{r,\ep} + 2k_{c,\ep}+2k_{i,\ep}^{\leq0}+k_{0,\ep}^{\leq0} = n^-(L_\ep)
\end{align}
 to study the linear stability/instability of the Kelvin-Stuart vortex,
where $k_{r,\ep}$ is the sum of algebraic multiplicities of positive eigenvalues of $J_\ep L_\ep$, $k_{c,\ep}$ is the sum of algebraic multiplicities of eigenvalues of $J_\ep L_\ep$ in the first quadrant, $k_{i,\ep}^{\leq 0}$ is the total number of non-positive dimensions of $\langle L_\ep\cdot, \cdot \rangle$ restricted to the generalized eigenspaces of pure imaginary eigenvalues of $J_\ep L_\ep$ with positive imaginary parts, and $k_{0,\ep}^{\leq 0}$ is the number of non-positive directions of $\langle L_\ep\cdot, \cdot \rangle$ restricted to the generalized kernel of $J_\ep L_\ep$ modulo $\ker L_\ep$. The index formula
\eqref{index-formula-stuart} is developed   for general Hamiltonian systems   in \cite{lin2022instability}.
By  \eqref{index-formula-stuart}, a sufficient condition for the spectral stability of the Kelvin-Stuart vortex is that
the energy quadratic form is non-negative, that is,
\begin{align*}
\langle L_\epsilon\omega,\omega\rangle\geq0,\quad\omega\in  X_\ep.
\end{align*}
This is equivalent to  the dual energy quadratic form being non-negative, that is,
 \begin{align}\label{dual energy quadratic form non-negative}
 \langle\tilde A_\epsilon\psi,\psi\rangle\geq0,\quad \psi\in \tilde{X}_\ep,
 \end{align}
where
    \begin{align*}
 \tilde{A}_\ep=-\Delta-g'(\psi_\ep)(I - P_\ep): \tilde{X}_\ep \rightarrow \tilde{X}_\ep^*,
\end{align*}
and the $1$-dimensional projection $P_\ep \psi={1\over8\pi}\iint_\Omega g'(\psi_\ep)\psi dxdy$ is added due to the constraint $\iint_\Omega\omega dxdy=0$.
To confirm that $\tilde A_\epsilon\geq0$, it is equivalent to show that the principal eigenvalue of the associated  PDE eigenvalue problem
\begin{align}\label{eigenvalue problem-introduction}
-\Delta \psi = \lambda g'(\psi_\ep)(\psi -  P_\ep\psi), \quad \psi \in \tilde{X}_\ep
\end{align}
is $1$. Moreover, we will prove that
 \begin{align}\label{dual energy quadratic form ker}
 \dim(\ker(\tilde A_\epsilon))=3,
 \end{align}
and the kernels are due to translations in $x, y$ and change of parameter $\ep$. This non-degeneracy  property plays an important role in the proof of nonlinear orbital stability.

Let us first consider the shear case ($\ep=0$). Because of the separability of the variables $(x,y)$,
 it reduces to study a series of  Sturm-Liouville type ODE eigenvalue problems \eqref{mode0}-\eqref{modek} for the Fourier modes. By numerical computations in Subsection \ref{eigenfunction-motivation} and the calculation of the first few eigenvalues with corresponding eigenfunctions in \eqref{eigen value-function}, we find a change of variable $$\gamma = \tanh(y),$$ which surprisingly transforms the ODEs \eqref{mode0}-\eqref{modek}  to the well-known  Legendre-type differential  equations \eqref{eigenvalue problem for 0 mode} and \eqref{eigenvalue problem2 non-zero modes varepsilon=0}, from which we solve all the exact eigenvalues with corresponding eigenfunctions by the (associated) Legendre polynomials.
 In particular,
 the principal eigenvalue of \eqref{eigenvalue problem-introduction} is $1$.
 This confirms spectral stability for $\ep=0$.

For the Kelvin-Stuart vortices ($0<\ep<1$), the associated PDE eigenvalue problem \eqref{eigenvalue problem-introduction} can not be solved by separation of the original variables $(x,y)$. This is a major difficulty in our study.
%The non-decoupled PDE version might be one of the main reasons that the linear stability/instability problem  is unsolved in the fluid literature.
  We introduce a nonlinear  change of variables  $(x,y)\mapsto(\theta_\ep,\gamma_\ep)$
and the associated PDE eigenvalue problems become decoupled in the new variables $(\theta_\ep,\gamma_\ep)$.
The important nonlinear change of  variables $(x,y)\mapsto(\theta_\ep,\gamma_\ep)$ is given by
  \begin{align}\label{transformation-isospectrum-inroduction1}
 \theta_\ep(x,y) & = \left\{ \begin{array}{llll} \arccos \left( \frac{\xi_\ep}{\sqrt{1-\gamma_\ep^2}} \right) & \mbox{ for } & (x,y) \in [0, \pi]\times\mathbb{R}, \\
 2\pi - \arccos \left( \frac{\xi_\ep}{\sqrt{1-\gamma_\ep^2}} \right) & \mbox{ for } & (x,y) \in (\pi, 2\pi]\times\mathbb{R}, \end{array}\right.\\\label{transformation-isospectrum-inroduction2}
 \gamma_\ep(x,y) & = \frac{\sqrt{1 - \epsilon^2}\sinh(y)}{\cosh(y)+\epsilon \cos(x)}\quad\text{for}\quad  (x,y) \in [0, 2\pi]\times\mathbb{R},
 \end{align}
where
$\xi_\ep(x, y)   = (1 - \epsilon^2) \frac{ \partial \psi_\ep}{\partial \ep} = \frac{\epsilon \cosh(y) + \cos(x)}{\cosh(y)+\epsilon \cos(x)}$.
 The new variables are compatible to the shear case, and the  parameter $\ep$ in the whole family of  steady states is fully encoded in the new variables.
 %Thus, we can study the eigenvalue problems in a unified way for the whole family
%and  reduce the eigenvalue problem \eqref{eigenvalue problem-introduction}
 %into the well-studied shear case.
Under the change of variables $(x,y)\mapsto(\theta_\ep,\gamma_\ep)$, we prove that $\tilde A_\ep$ is iso-spectral to $\tilde A_0$ (i.e. they have the same eigenvalues).
In particular, 
 \eqref{dual energy quadratic form non-negative} and \eqref{dual energy quadratic form ker} hold true, which is crucial to  study the nonlinear stability of the Kelvin-Stuart vortices in Section \ref{Sec-Nonlinear orbital stability for co-periodic perturbations}.
  For the motivation of introducing the new variables $(\theta_\ep,\gamma_\ep)$, we refer  to \eqref{eigen-p-cat-eyes}-\eqref{def-xi-ep}.



 \medskip

\noindent{\it{Proof of linear  instability of Kelvin-Stuart vortices for multi-periodic perturbations}}:
As in the co-periodic case, the linearized equation around  $\omega_\ep$ can be written as
the  Hamiltonian system
$
 \partial_t \omega = J_{\epsilon,m} L_{\epsilon,m}\omega,  \omega \in X_{\ep,m},
 $
where we add $m$ in the subscript to indicate the $2m\pi$-periodic perturbations with $m\geq2$.
The difference from the co-periodic case is that $n^-(L_{\epsilon,m})>0$, where $n^-(L_{\epsilon,m})$ is the negative dimension   of the energy quadratic form $\langle L_{\epsilon,m}\cdot,\cdot\rangle$. If we still use a similar index formula
 $
k_{r,\ep,m} + 2k_{c,\ep,m}+2k_{i,\ep,m}^{\leq0}+k_{0,\ep,m}^{\leq0} = n^-(L_{\ep,m})
$
  as  in the co-periodic case, we have to compute
the indices $k_{i,\ep,m}^{\leq0}$ and  $k_{0,\ep,m}^{\leq0}$, which involve the spectral information of $J_{\epsilon,m} L_{\epsilon,m}$ on the pure imaginary axis and are difficult to study.
Here, $k_{r,\ep,m}, k_{c,\ep,m}, k_{i,\ep,m}^{\leq0}, k_{0,\ep,m}^{\leq0}$ are the indices defined similarly as in \eqref{index-formula-stuart}.
One of the key observations is that  the linearized vorticity equation could be formulated as a
separable Hamiltonian system
\begin{align}\label{s Hamiltonian system-introduction}
\partial_t \left( \begin{array}{c} \omega_1 \\ \omega_2 \end{array} \right) = \left( \begin{array}{cc} 0 & B_\ep \\ -B'_\ep & 0 \end{array} \right)\left( \begin{array}{cc} L_{\ep,e} & 0 \\ 0 & L_{\ep,o} \end{array} \right) \left( \begin{array}{c} \omega_1 \\ \omega_2 \end{array} \right),
%= \mathbf{J}_{\ep,m} \mathbf{L}_{\ep,m} \left( \begin{array}{c} \omega_1 \\ \omega_2 \end{array} \right),
\end{align}
 which is
due to the symmetry of the steady state   in the $y$ direction and the fact that $L_{\ep,o}\geq0$. Here,
\begin{align*}B_\ep &= -g'(\psi_\ep) \vec{u}_\ep \cdot \nabla : X_{\ep, o}^* \supset D(B_\ep) \rightarrow X_{\ep, e}, \\
 L_{\ep,o} &= \frac{1}{g'(\psi_\ep)} - (-\Delta)^{-1}: X_{\ep, o} \rightarrow X_{\ep, o}^*, \quad\quad
 L_{\ep,e} = \frac{1}{g'(\psi_\ep)} - (-\Delta)^{-1}: X_{\ep, e} \rightarrow X_{\ep, e}^*,
 \end{align*}
and the spaces are $X_{\ep, e} = \left\{ \omega \in X_{\ep,m} | \omega \text{ is even in }y \right\}$,
$X_{\ep, o} = \left\{ \omega \in X_{\ep,m} | \omega \text{ is odd in }y \right\}.$
This allows us to apply a precise  formula $n^-\left(L_{\ep,e}|_{\overline{\text{R}(B_\ep)}}\right)$ for counting unstable modes. Thus, $\omega_\ep$ is linearly unstable if and only if $$n^-\left(L_{\ep,e}|_{\overline{\text{R}(B_\ep)}}\right)>0.$$
This is equivalent to
\begin{align}\label{multi-periodic instabilibity criterion intruduction}
 n^-\left(\hat{A}_{\ep,e}\right)>0,
\end{align}
where  the  alternative dual quadratic form $\hat{A}_{\ep,e}$ has the form
$$\hat{A}_{\ep,e} = - \Delta - g'(\psi_\ep)(I - \hat{P}_{\ep,e}): \tilde{X}_{\ep, e} \rightarrow \tilde{X}^*_{\ep, e}.$$
Here, the operator $\hat{P}_{\ep,e}$   defined by \eqref{def-hat-P-ep-e} is an  infinite-dimensional projection to $\ker (B_\ep')$ and can be traced back to the constraint space $\overline{\text{R}(B_\ep)}$ for $L_{\ep,e}$.
Due to the  nonlocal  projection $\hat{P}_{\ep,e}$, the spectra of $\hat{A}_{\ep,e}$  are difficult to find explicitly.
To obtain linear instability, it is sufficient to construct a suitable test function $\psi$ such that  $\langle\hat{A}_{\ep,e}\psi,\psi\rangle<0$. For $4k\pi$-periodic case, our construction of the test function \eqref{test-even} is based on an explicit
 eigenfunction of the associated PDE eigenvalue problem $-\Delta \psi = \lambda g'(\psi_\ep)(\psi -  P_{\ep,m}\psi),  \psi \in \tilde{X}_{\ep,m}$, where  the nonlocal projection term vanishes. For  $(4k+2)$-periodic case,
it is impossible to choose a periodic test function such that the nonlocal term of the quadratic form vanishes, which makes the construction of test functions  much more subtle.
Our construction is a delicate combination of different eigenfunctions in different regions, which are given in \eqref{test-odd} for $\ep\in[0,{4\over5}]$ and \eqref{test-odd-2} for $\ep\in\left({4\over5},1\right)$. The choice of the test functions for $\ep$ in the two subintervals is to make the contribution of the projection term as small as possible. It is difficult to estimate the projection accurately. Our approach is to reduce the estimates to the nested property of the trapped regions in the variables $(\theta_\ep,\gamma_\ep)$, see Lemma \ref{D-theta-gamma-ep-nest}. We find that the level curves of $\omega_\ep$ in  alternative variables $(\xi_\ep,\eta_\ep)$ are parts of some ellipses in the closed unit desk $D_1$, where $(\xi_\ep,\eta_\ep)$ are given in \eqref{three-kers3} and \eqref{three-kers1}. We obtain the desired property by proving that the inner boundary elliptic curves are nested.

\if0
For the  eigenvalue problems \eqref{eigenvalue problem-ep-new-m} or \eqref{eigenvalue problem-ep-original-m} induced by investigating the negative directions of the dual quadratic forms, they could be converted to a kind of  hypergeometric differential equations \eqref{eigenvalue problem2 non-zero modes varepsilon=0m} but tedious to solve. Motivated by the explicit first  eigenfunction for $1$ mode in the double-periodic case, we find that the first few eigenfunctions are polynomials multiplying the first one, and this induces a transformation \eqref{transformation-multi-periodic perturbations} to convert the ODEs \eqref{eigenvalue problem2 non-zero modes varepsilon=0m} to the classical Gegenbauer differential equations \eqref{eigenvalue problem2 non-zero modes varepsilon=0-transform}, from which we solve the eigenvalue problems completely, and in particular, verify the spectral assumptions in Lemma \ref{indice-theorem-sep} for  the validness of the index formula \eqref{index-formula-neg}.
\fi








\noindent{\it{Proof of  modulational  instability of Kelvin-Stuart vortices}}:
 The proof  is mostly analytical, and the  only computer assistant part is  the calculation of the integral in \eqref{func-b2-alpha}-\eqref{func-b1-alpha-2}.
  %with explicit  integrand expressions in \eqref{Gamma-rho-nabla-psi1}-\eqref{1-gamma-ep-2-expression}.
In  this case, the linearized vorticity equation  is formulated as a complex Hamiltonian system \eqref{complex Ham-modu}.
 To apply the index formula \eqref{index-formula-neg},
we reformulate the complex Hamiltonian system \eqref{complex Ham-modu} into a real separable Hamiltonian one \eqref{sep-hamiltonian-alpha}. Then we
derive an instability criterion in Lemma \ref{L e-hat A-alpha} based on  the dual quadratic form associated with a different nonlocal projection term from the multi-periodic case. We construct the
 test function \eqref{test-function-modulational-instability} by  the  first  eigenfunction of the associated PDE eigenvalue problem \eqref{elip02-alpha}, and  the value of corresponding  dual quadratic form is checked to be negative for all $\alpha\in(0,{1\over2}]$.

In the above construction of test functions for multi-periodic/modulational instability, we use the eigenfunctions of the first few eigenvalues of the eigenvalue problems  $-\Delta \psi = \lambda g'(\psi_\ep)(\psi -  P_{\ep,m}\psi),  \psi \in \tilde{X}_{\ep,m}$ or \eqref{elip02-alpha}, where $ P_{\ep,m}$ is  a $1$-dimensional projection defined similarly as $P_\ep$. Such eigenvalue problems
 are more involved to  solve  than the eigenvalue problem \eqref{eigenvalue problem-introduction} for the co-periodic case,  no matter in the original variables or in the new variables. To solve the  eigenvalue problems $-\Delta \psi = \lambda g'(\psi_\ep)(\psi -  P_{\ep,m}\psi),  \psi \in \tilde{X}_{\ep,m}$ or \eqref{elip02-alpha},
 we introduce two different transformations \eqref{transformation-modu} and \eqref{modulational-transformation}, by which the  ODEs for the nonzero modes are surprisingly converted  to  Gegenbauer differential equations. This enables us to solve the eigenvalue problems completely by Gegenbauer/ultraspherical polynomials.





\if0
Instead of restricting to a truncated domain, the first goal of this paper is to thoroughly solve the linear stability/instability problem of Kelvin-Stuart vortices in the original  unbounded domain for co-periodic, multi-periodic and modulational perturbations.
 For  the whole family of Kelvin-Stuart vortices,
we prove spectral stability for co-periodic perturbations, linear instability for any multi-periodic perturbations, as well as linear modulational instability for any modulational parameters. The problem of linear stability/instability  is to understand the spectra of  the linearized vorticity operator.
\fi




%if we direct use the more precise formula \eqref{index-formula-neg}, we need to study the negative direction of the energy quadratic form restricted to the closure of the range of the anti-self-dual part of the linearized vorticity operator,  the corresponding elliptic operator of  the dual quadratic form is  with an infinite-dimensional projection,


\if0
In general, it is still difficult to explicitly solve the ODE eigenvalue problems in a Sturm-Liouville type.
 %Our new variables are motivated by computation of the first few eigenvalues with corresponding polynomial-type eigenfunctions for  hyperbolic tangent shear flow, and by eigenfunctions  bifurcation and  De Moivre's formula for  Kelvin-Stuart vortices.
  In the co-periodic case, our choice of the new variables makes  the ODEs surprisingly correspond to the well-known Legendre-type differential equations, from which we solve all the exact eigenvalues with corresponding eigenfunctions by the (associated) Legendre polynomials as well as the spherical harmonics.
In the multi-periodic and modulational  cases, we introduce two different transformations to convert our ODEs for the nonzero modes to  Gegenbauer differential equations, from which we solve the eigenvalue problems by Gegenbauer/ultraspherical polynomials.
\fi




\if0
The idea in the proof is as follows.
First, we formulate the linearized vorticity equation  as a Hamiltonian system. Unlike in the bounded truncated  domain with suitable boundary conditions,  we need to  make some fundamental  modifications to  deal with  the lack of compactness
induced by the original unbounded domain $\Omega$. We introduce two weighted  Poincar\'e-type inequalities  and a new Hilbert space $\tilde X_\ep$ of the stream functions (see \eqref{tilde-X-e}) to ensure the solvability of the Poisson equation and the boundedness of the energy quadratic form $\langle L_\epsilon\cdot,\cdot\rangle$, where $L_\epsilon$ is given in \eqref{J-ep-J-ep-def}. To study the negative and kernel directions of the energy quadratic form, as usual we transform it into a dual form $\langle\tilde A_\epsilon\cdot,\cdot\rangle$
 to avoid handling the nonlocal term, where $\tilde A_\epsilon$ is given in \eqref{A0} for $\ep=0$ and \eqref{tilde-A-ep-A-ep} for $0<\ep<1$.  Thanks to the exponential decay weight, we have a compact embedding lemma, which helps us to reduce the variational problem induced by the dual form to a PDE eigenvalue problem.

Because of the separability of the variables $(x,y)$ in the case of hyperbolic tangent shear flow ($\ep=0$), it reduces to study the Sturm-Liouville type ODE eigenvalue problems \eqref{mode0}-\eqref{modek} for each Fourier mode. By numerical computations in Subsection \ref{eigenfunction-motivation} and the calculation of the first few eigenvalues with corresponding polynomial-type eigenfunctions in \eqref{eigen value-function}, we find a change of variable in \eqref{change of variable for 0 mode}, which transforms our ODEs \eqref{mode0}-\eqref{modek}  to the Legendre-type differential  equations \eqref{eigenvalue problem for 0 mode} and \eqref{eigenvalue problem2 non-zero modes varepsilon=0}. This makes us able to solve the eigenvalue problems completely, and as a consequence, there are no negative directions and exactly three kernel directions to the energy quadratic form. This confirms spectral stability by  the index formula \eqref{index-formula} and is useful to study the  stability in a nonlinear level (see Section \ref{Sec-Nonlinear orbital stability for co-periodic perturbations}).
For the Kelvin-Stuart vortices ($0<\ep<1$), the PDE eigenvalue problem \eqref{eigenvalue problem-ep-original} can not be decoupled by separation of variables $(x,y)$, and we observe that the kernel functions can be viewed as bifurcation from the shear case. After finding a few eigenvalues and eigenfunctions by a similar bifurcation technique, we conjecture that all the eigenvalues should be the  same as the shear case, and all the eigenfunctions could be bifurcated from the corresponding ones in the shear case.  The normalization of kernel functions in \eqref{three-kers1}-\eqref{three-kers3} and the De Moivre's formulae in \eqref{De Moivres formula1}-\eqref{De Moivres formula2} motivate us to find  the explicit forms of all the  eigenfunctions, which induce the change of  variables $(x,y)\mapsto(\theta_\ep,\gamma_\ep)$ in \eqref{transf1}-\eqref{transf2} for Kelvin-Stuart vortices. The new variables are compatible to the shear case, and make us able to   reduce the eigenvalue problem \eqref{eigenvalue problem-ep-original}
 into \eqref{elip02-x-gamma} in the well-studied shear case.
\fi


\if0
\begin{remark}
 For $\ep=0$, the number of unstable eigenvalues is $2(m-1)$ for $2m\pi$-periodic perturbations ($m\geq2$), see Remark $\ref{number of unstable eigenvalues for multi-periodic perturbations} \;(2)$ for details.
This also holds true for $\ep\ll1$. The number may decrease as $\ep$ goes far from $0$, but is at least one by Theorem $\ref{main result2-multi-periodic perturbations}$.
\end{remark}
\fi


\noindent{\it{Proof of  nonlinear  stability of Kelvin-Stuart vortices for co-periodic perturbations}}:
Let us first give a  sketch of the proof for nonlinear  stability in a truncated   domain $\Omega_{trun}$ bounded  by a pair of streamlines in \cite{holm1986nonlinear}. In this work,
Holm, Marsden and Ratiu adopted Arnol$'$d's original  method \cite{Arnold65,Arnold69}. They used the energy-Casimir (EC) functional $\tilde H(\tilde\omega)= \iint_{\Omega_{trun}}\left(h(\tilde\omega)-{1\over2}|\nabla\tilde\psi|^2\right)$ $dxdy$, where $\tilde \omega$ and $\tilde \psi$ are the perturbed vorticity and stream functions, and  $h(s)=\int_0^sg^{-1}(\tilde s)d\tilde s=-\int_0^s{1\over 2}\ln(-\tilde s)d\tilde s={1\over2}(s-s\ln(-s))$ for $s<0$. To highlight the idea, we ignore the boundary effect here. Then $\tilde H'(\omega_\ep)=0$ and
\begin{align*}
\tilde H(\tilde\omega)-\tilde H(\omega_\ep)=\iint_{\Omega_{trun}}\left((h(\tilde\omega)-h(\omega_\ep)-h'(\omega_\ep)\omega)-{1\over2}|\nabla\psi|^2\right)dxdy,
\end{align*}
where $\omega=\tilde \omega-\omega_\ep$ and $\psi=\tilde \psi-\psi_\ep$. Note that  $h''(\omega_\ep)$ has a uniformly positive upper bound $C_{trun}$ and lower bound $c_0$ in $\Omega_{trun}$. By extending $h|_{Ran(\omega_\ep)}$ to the entire axis with the same bounds of the second derivative, for the first term we have
$${1\over2} C_{trun}\|\omega\|_{L^2(\Omega_{trun})}^2\geq\iint_{\Omega_{trun}}\left(h(\tilde\omega)-h(\omega_\ep)-h'(\omega_\ep)\omega\right)dxdy\geq {1\over2} c_0\|\omega\|_{L^2(\Omega_{trun})}^2,$$
where $C_{trun}\to\infty$ if the size of  the truncated domain goes to infinity while $c_0$  depends only on $\ep$.
For the second term, the Poincar\'e type inequality
\begin{align}\label{Poincare type inequality truncated domain}
\iint_{\Omega_{trun}}|\nabla\psi|^2dxdy\leq k_{\min}^{-2}\|\omega\|_{L^2(\Omega_{trun})}^2
\end{align}
 holds,
where $k_{\min}^2$ is the principal eigenvalue of $-\Delta$ on  $\Omega_{trun}$.
Note that $k_{\min}^2$ is a decreasing function of the size of the truncated domain $\Omega_{trun}$. When the  size of $\Omega_{trun}$ is not so large, it follows that $ k_{\min}^{-2}<c_0$, which along with  the upper bound $C_{trun}$ of  $h''(\omega_\ep)$, implies
\begin{align*}
{1\over2}C_{trun}\|\omega^0\|_{L^2(\Omega_{trun})}^2\geq\tilde H(\tilde\omega)-\tilde H(\omega_\ep)\geq {1\over2} (c_0-k_{\min}^{-2})\|\omega\|_{L^2(\Omega_{trun})}^2,
\end{align*}
where $\omega^0$ is the initial perturbation of the vorticity.
This gives nonlinear stability.
When the  size of $\Omega_{trun}$ is  large enough,  $ k_{\min}^{-2}>c_0$ prevents the estimates above from being carried out.
It is much more difficult to study nonlinear stability in the original domain  via this approach, since, on the one hand,
 the above  Poincar\'e type inequality \eqref{Poincare type inequality truncated domain} holds only in the bounded domains, let alone $k_{\min}^{-2}<c_0$,
and on the other hand,
 $h''(\omega_\ep)$ is unbounded from above.

Now, we give the main ideas for our proof of nonlinear stability in the original unbounded domain $\Omega$.
Since the perturbed velocity  tends to $(\pm1,0)$ as $y\to\pm\infty$, the classical kinetic energy $\iint_{\Omega}|\vec{u}|^2dxdy$ is not well-defined.
We use the pseudoenergy $\iint_{\Omega}(G\ast\tilde \omega)\tilde \omega dxdy$ to replace the  kinetic energy and study the
pseudoenergy-Casimir (PEC) functional
$ H(\tilde\omega)= \iint_{\Omega}\left(h(\tilde\omega)-{1\over2}(G\ast\tilde \omega)\tilde \omega\right)dxdy.$ Then
\begin{align}\label{pec-introduction}
 H(\tilde\omega)- H(\omega_\ep)=\iint_{\Omega}\left((h(\tilde\omega)-h(\omega_\ep)-h'(\omega_\ep)\omega)-{1\over2}(G*\omega)\omega\right)dxdy.
\end{align}
 Since $h''(\omega_\ep)$ is unbounded from above,
the enstrophy norm used in the truncated domain is not applicable in the original domain $\Omega$
and it is impossible to extend $h|_{Ran(\omega_\ep)}$ to be a convex function  on the entire axis. Instead,
 we define the distance functionals to be the sum of the first term in \eqref{pec-introduction} and the  pseudoenergy.
 In this way, the upper bound of $H(\tilde\omega)- H(\omega_\ep)$ can be directly controlled by the initial data. For the lower bound,  the  argument for the truncated domain can not be applied to the original unbounded  domain $\Omega$, since  the Poincar\'e type inequality \eqref{Poincare type inequality truncated domain} fails for $\Omega$. We use a different approach, and  summarize the ideas  and methods to overcome the difficulties  as follows:

1. We try to study the precise Taylor expansion of $H$ at $\omega_\ep$ directly. The first order variation $H'(\omega_\ep)=0$ and the second order variation  exactly corresponds to the energy quadratic form at the linear level, that is, $\langle H''(\omega_\ep)\omega,\omega\rangle=\langle L_\ep\omega,\omega\rangle$.
 The remainder terms, however, can not be controlled since  $H$ is not $C^2$ near $\omega_\ep$.
 Therefore,  based on the Legendre transformation  we introduce a dual functional of stream functions
  \begin{align*}
\mathscr{B}_\ep( \psi)=
  & \iint_{\Omega} \left(\frac 1 2 |\nabla \psi|^2 -\frac 1 4 g'(\psi_\ep)(e^{-2\psi} + 2\psi - 1)\right) dxdy,\quad \psi\in \tilde X_\ep,
\end{align*}
  and prove that  it  is $C^2$ on $\tilde X_\ep$, which is enough to control the remainder terms.
   The first order variation $\mathscr{B}_\ep'(0)=0$ and the second order variation $\mathscr{B}_\ep''(0)$ corresponds to the dual energy quadratic form at the linear level, that is,
   $$\langle \mathscr{B}_\ep''(0)\psi,\psi\rangle=\langle A_\ep\psi,\psi\rangle,$$
   where $A_\ep=\tilde A_\ep-g'(\psi_\ep)P_\ep\geq0$.
 %We remark that the constraint $\iint_\Omega\omega dxdy=0$ is used in several places to  remove a constant difference  or the  1-dimensional projection $P_\ep\psi$ difference of $\psi$.

2. Since $\dim(\ker(A_\ep))=3$ and the kernels are induced by the translations of the steady states  in  $x,y$ and  change of parameter $\ep$,
 we prove the nonlinear 3D orbital stability of Kelvin-Stuart vortices as a first step. Here, the 3D orbit consists of the translations (in $x, y$) of the whole
 family of Kelvin-Stuart vortices.
 %To this end, we  carefully choose the translations and change of parameter to ensure the perturbation of the stream function perpendicular to the three  kernel functions.

3. To prove the nonlinear 2D orbital (due to the translations   in  $x,y$) stability of a fixed Kelvin-Stuart vortex, we use an additional vorticity constraint $\iint_{\Omega} (-\omega)^{3\over2} dxdy$ to ensure that  the change of parameter $\ep$ of the steady states  remains small enough for all times.
Thus, the 3D orbital stability implies the 2D orbital stability of any fixed Kelvin-Stuart vortex.

4. Finally, if we carry out  the  analysis of nonlinear  stability to the weak solution directly, the distance functional is not necessarily  continuous on $t$ so that the solution may jump from a neighborhood of one steady state to others. To overcome this difficulty,
    we first  construct the approximate strong solutions by smoothing the initial data and  prove nonlinear orbital stability for the approximate solutions.
Then we prove the nonlinear orbital stability for the weak solution by taking limits, where we  use the convexity of the Casimir functional and a careful study on the convergence of the initial data of approximate solutions.

\if0
In addition,  we use another constraint ($\iint_{\Omega}y\tilde \omega dxdy$ is conserved) to ensure the uniform boundedness of the translations in the $y$ direction for the approximate solutions so that we obtain a finite limiting translation in this direction.
\fi

   \if0
   We expand the dual functional in the actual operation and use the codimension-3 positivity of the elliptic operator $A_\ep$ (see \eqref{tilde-A-ep-A-ep} in the analysis of linear stability) to
obtain a desired  lower bound of the relative pseudoenergy-Casimir functional in terms of the distance. We use a limiting case of Sobolev embedding to prove the $C^{\geq2}$ regularity of the dual functional in Lemma \ref{B-C2}.
\fi



\if0
Since the velocity of the steady states tends to $(\pm1,0)$ as $y\to\pm\infty$,
the existence of the weak solution to the nonlinear 2D Euler equation \eqref{euler} needs to be proved.
For the initial vorticity $\tilde \omega_0\in Y_{non}$, it is proved by using the shear-energy decomposition, constructing an  approximate solution sequence and studying their limit in the Appendix.
\fi





\noindent{\it{Proof of   stability and instability  of Kelvin-Stuart magnetic islands}}:
Compared with the separable Hamiltonian form \eqref{s Hamiltonian system-introduction} in the 2D Euler case,  the linearized  planar ideal MHD equations around the  magnetic island $(0,\phi_{\ep})$ have a  different separable Hamiltonian structure
\begin{align*}
\partial_t \left( \begin{array}{c} \phi \\ \omega \end{array} \right) = \left( \begin{array}{cc} 0 & D_\ep \\ -D_{\ep}' & 0 \end{array} \right)\left( \begin{array}{cc}-\Delta-g'(\phi_{\ep}) & 0 \\ 0 & (-\Delta)^{-1} \end{array} \right) \left( \begin{array}{c} \phi \\ \omega \end{array} \right)
\end{align*}
for  co-periodic perturbations, where
$\phi\in\tilde W_{\ep}=\{ \phi \in\dot{H}^1(\Omega) | \iint_\Omega g'(\phi_\ep)\phi dxdy=0\}$ is the perturbation of magnetic potential,
$\omega\in \tilde Y=\{\omega\in L^1\cap L^3 (\Omega)|\iint_{\Omega}\omega dxdy=0,y\omega\in L^1(\Omega)\}$ is the perturbation of vorticity,
  and $D_\ep=-\{\phi_{\ep},\cdot\}:\tilde Y^*\supset D(D_\ep)\to\tilde W_{\ep}$.
Based on this structure, the criterion for co-periodic spectral stability  is
$$n^-\left(\tilde A_{\ep}|_{\overline{R(D_{\ep})}}\right)=0.$$
Then spectral stability of $(0,\phi_{\ep})$ is recovered by our linear analysis in the 2D Euler case since $\tilde A_\ep|_{\tilde X_\ep}\geq0$.
Similarly, the criterion for multi-periodic linear instability is
\begin{align}\label{multi-periodic linear instability-introduction}
n^-\left(\tilde A_{\ep,m}|_{\overline{R(D_{\ep,m})}}\right)\geq1,
\end{align}
where the subscript $m$ is  used  to indicate the $2m\pi$-periodic perturbations, $m\geq2$. The condition \eqref{multi-periodic linear instability-introduction} is more restrictive than \eqref{multi-periodic instabilibity criterion intruduction} in the 2D Euler case.
  Thanks to the symmetry of
  the test function $\tilde{\psi}_\ep$ (see \eqref{test-even}) for double-periodic perturbations in the 2D Euler case,  $\tilde{\psi}_\ep$ is in $\overline{R(D_{\ep,2})}$, and this gives linear instability of $(\omega=0,\phi_{\ep})$ for double-periodic perturbations.
  That is, the coalescence instability is proved for the whole family of Kelvin-Stuart magnetic islands. This verifies the physical observations in \cite{Finn-Kaw1977,Pritchett-Wu1979,Bondeson1983}.
  \begin{remark}
  It is still open to prove  triple-periodic linear instability of Kelvin-Stuart magnetic islands.
  The test function for triple-periodic perturbations in the 2D Euler case does not work here, since it is not in $\overline{R(D_{\ep,3})}$.
  \end{remark}
\if0
while the test functions \eqref{test-odd} or \eqref{test-odd-2} for $6\pi$-periodic perturbations are not in $\overline{R(D_{\ep,3})}$, and it is interesting to prove linear instability of  $(\omega=0,\phi_{\ep})$ in this case,  see Remark \ref{linear instability for odd periodic perturbations}.
\fi
Nonlinear orbital stability of Kelvin-Stuart magnetic islands for co-periodic perturbations is proved by
 the energy-Casimir method.
Besides similar difficulties arising from $2$D Euler case, there is another difficulty in the MHD nonlinear analysis.
Note that the perturbation of the stream function is allowed to be differed by a constant in   the 2D Euler case due to $\iint_\Omega\omega dxdy=0$.
 In the MHD case, however, the perturbation of the magnetic potential can not be changed by a constant and the perturbation is not necessarily in the space  $\tilde X_\ep$ after translations. Thus,  the $C^2$ regularity of the EC functional can not be proved  in the space $\tilde X_\ep$ directly. Our approach is to add a projection term $P_\ep \phi={1\over8\pi}\iint_\Omega g'(\phi_\ep)\phi dxdy$   into the EC functional, through which a constant difference can be allowed in the perturbation. This enables us to  prove
 the $C^2$ regularity of the main term of  the EC functional  in the space $\tilde X_\ep$  and make use of
 the linear analysis.
 In addition, the remainder term caused by the projection turns out to be a high order term of the distance functional.

\if0
In the modulational case,
our numerical computations in Section \ref{Numerical Results} show an interesting phenomenon that  the number of unstable eigenvalues changes from $2$ to $1$ when  $\ep$
passes through $0.16$ for $\alpha={1\over2}$, and a similar phenomenon occurs for $\alpha={1\over3}$, see Figure \ref{fig:eleventhFig}.
\fi



 Kelvin-Stuart cat's eyes also  appear   in  the study of  planetary rings. They are applied to  understand the
 spatial
structures  in Saturn's ring system \cite{Shukla-Sen1996}, and   when the electron number density
is completely depleted, the electromagnetic
equilibrium of the dust grains  is governed by
the Liouville's equation (see \eqref{elip}), one of whose solutions is given as Kelvin-Stuart vortices.

Recently,   Kelvin-Stuart vortices are generalized  in different settings.
 Crowdy \cite{Crowdy04} and Constantin {\it et al.} \cite{Constantin-Crowdy-Krishnamurthy-Wheeler2021}  generalized
the planar Stuart vortices to the cases of  non-rotating and rotating
spheres, respectively.
Sakajo \cite{Sakajo2019} and Yoon {\it et al.}  \cite{Yoon20} extended the planar Stuart vortices to the settings
of  a torus and  a
hyperbolic sphere, respectively. The geometry of the domain and rotation could affect the stability of equilibria. It is very interesting to study  stability/instability of the generalized Stuart vortices in the above settings by  our methods developed in this paper.
See other discussions on Kelvin-Stuart vortex, its stability and related hybrid vortex
equilibria in \cite{Klaassen-Peltier1991, Dauxois-Fauve-Tuckerman1996, BD01, Majda-Bertozzi02, Constantin-Krishnamurthy2019, Krishnamurthy2019, Krishnamurthy2021}.


Liouville's equation with general form $\Delta \phi = c_1 e^{c_2\phi}$ has important applications in fluid dynamics, space plasma physics, high energy physics and differential geometry, where $c_1$ and $c_2$ are real numbers.  Such equations and their generalizations have attracted considerable   attention   since Liouville's paper \cite{Liouville1853} in 1853, and    stimulated numerous works  in mathematical physics.
For example,
 it  appears in the theory of the space charge of electricity round a glowing wire \cite{Richardson1921}  and also occurs in the magnetohydrostatic model of the
earth's magnetosphere \cite{Schindler2006}.
We refer
to the recent survey \cite{Bogatov-Kichenassamy22} for more discussions and references.
 % Liouville's equation was considered as an example by  Hilbert  in the formulation of his 19-th problem \cite{Hilbert1900}.
Some exact solutions of Liouville's equation, including the Kelvin-Stuart cat's eyes, have been obtained in the literature. See \cite{Crowdy97} and references therein.
\if0
In particular,
Liouville found a class of solutions, which are generated by analytic functions,  of
\eqref{Liouville's equation} for $c_1c_2<0$ \cite{Liouville1853}, while
\eqref{Liouville's equation} has no solutions valid in the whole plane for $c_1c_2>0$ \cite{Keller1957,Wittich44}.
%More solutions and their applications of the Liouville's equation can be found in \cite{Liouville1853,Schindler2006}.
See more discussions on Liouville's equation and its solutions in \cite{Bogatov-Kichenassamy22,Bateman32,Lutzen2012}.
\fi
In particular,  Taylor \cite{Taylor2018} found a 2-parameter family of  cat's eyes solutions of \eqref{elip} with stream functions of the form
\begin{align}\label{Taylor 2 parameter solutions}
\psi_{\gamma,\sigma}=\ln\left({\gamma\over2}e^y+{\sigma^2+1\over2\gamma}e^{-y}+\sigma\cos(x)\right),
\end{align}
where $\gamma$ and $\sigma$ are two independent positive numbers. The special choice $\sigma=\sqrt{\gamma^2-1}$ with $\gamma \geq1$ corresponds to Kelvin-Stuart cat's eyes. For the linear stability/instability problem of the whole family of cat's eyes \eqref{Taylor 2 parameter solutions}, it is interesting to study whether there exists such a similar iso-spectral transformation as \eqref{transformation-isospectrum-inroduction1}-\eqref{transformation-isospectrum-inroduction2}.

The rest of this paper is organized as follows. We
prove that the steady state $\omega_\ep$ with $\ep\in[0,1)$ is  spectrally stable
for co-periodic perturbations in Section 2,   linearly unstable for multi-periodic  perturbations in Section 3, and
linearly modulationally unstable in Section 4. We  show that the Kelvin-Stuart vortices are nonlinearly  orbitally stable for co-periodic perturbations in Section 5. We give some numerical simulations  in Section 6.  We study  stability/instability of  magnetic island solutions $(\omega=0,\phi_{\ep})$ of  the planar ideal MHD equations \eqref{mhd} for co-periodic and double-periodic perturbations in Section 7.
In the Appendix, we prove the existence of weak solutions to the 2D Euler equation in the unbounded domain $\Omega$ with non-vanishing velocity at infinity.

\section{Spectral stability for  co-periodic perturbations}\label{co-periodic-linear}

In this section, we consider  linear stability of the whole family of the steady states $\omega_\ep$ for co-periodic  perturbations. Our results reveal that spectral stability  holds true  for $\omega_\ep$ with  all $\ep \in [0, 1)$.

First, we  formulate the linearized vorticity  equation as a Hamiltonian PDE, and transform the self-adjoint part of the  linearized vorticity operator to an   elliptic operator of stream functions.
\subsection{Hamiltonian formulation of the linearized Euler equation}
Linearizing the vorticity equation \eqref{vor} around the steady state $\omega_\ep$, we have
 \begin{equation*}
 \partial_t \omega + \partial_y \psi_{\epsilon} \partial_x \omega - \partial_x \psi_{\epsilon}\partial_y \omega + \partial_y \psi \partial_x \omega_\epsilon - \partial_x\psi\partial_y \omega_\epsilon = 0,
 \end{equation*}
 which can be rewritten as
 \begin{equation}\label{linearized vorticity equation}
 \partial_t\omega = - \vec{u}_\epsilon\cdot \nabla\omega + g'(\psi_\epsilon)\vec{u}_\epsilon\cdot\nabla\psi,
 \end{equation}
 where we used $\omega_\epsilon=g(\psi_\epsilon)$ by \eqref{elip}. Note that
  \begin{align}\label{def-g-psi-ep-derivative}
    g'(\psi_\epsilon) = 2 e^{-2\psi_\epsilon} = \frac{2(1-\ep^2)}{(\cosh (y) + \ep \cos(x))^2}>0,\quad(x,y)\in\Omega,\; \epsilon\in[0,1).
  \end{align}
   The linearized equation \eqref{linearized vorticity equation} has
 the following  Hamiltonian structure
 \begin{equation*}
 \partial_t \omega = J_\epsilon L_\epsilon\omega, \quad \omega \in X_\ep,
 \end{equation*}
 where
 \begin{align*}J_\epsilon = -g'(\psi_\epsilon)\vec{u}_\epsilon\cdot\nabla: X_\ep^* \supset D(J_\epsilon) \rightarrow X_\ep, \quad
 L_\epsilon = \frac {1} {g'(\psi_\epsilon)} - (-\Delta)^{-1}: X_\ep \rightarrow X_\ep^*,\end{align*}
\begin{align*}
X_\ep = \left\{\omega\bigg| \iint_\Omega \frac{|\omega|^2}{g'_\epsilon(\psi_\epsilon)} dxdy < \infty, \iint_\Omega \omega dxdy = 0 \right\},\quad \epsilon\in[0,1),
\end{align*}
$X_\ep^*$ is the dual space of $X_\ep$ and $(-\Delta)^{-1}\omega$ is defined as the unique weak solution to the Poisson equation
\begin{align}\label{Poisson equation}
-\Delta \psi = \omega
\end{align}
 in $\tilde X_\epsilon$
 (see Lemmas \ref{1-1correspond} and \ref{1-1correspond-ep}).  Here, $\tilde X_\epsilon$ is defined in \eqref{tilde-X0} and \eqref{tilde-X-e} for $\epsilon=0$ and  $\epsilon\in(0,1)$, respectively.
%The well-posedness of  the Poisson equation is essentially due to the the Poincar\'e inequality, the proof of which is left to  Lemmas \ref{poincare1} and \ref{poincare1ep}.


The vorticity space $X_\ep$ equipped with the inner product
$$(\omega_1, \omega_2) = \iint_\Omega \frac{\omega_1\omega_2}{g'_\epsilon(\psi_\epsilon)} dxdy$$ is a Hilbert space since it is a closed subspace of the Hilbert space $L^2_{\frac{1}{g'(\psi_\ep)}}(\Omega).$
We  denote  the dual bracket between $X_\ep$ and $X_\ep^*$ by $\langle \cdot, \cdot \rangle$. Thanks to the Poincar\'e inequality in Lemmas  \ref{poincare1} and \ref{poincare1ep}, we will prove that  $\langle L_\epsilon\cdot,\cdot\rangle$ is  a bounded  symmetric bilinear form on  $X_\ep$, see Lemmas  \ref{Lbounded} and \ref{Lbounded-ep}.


We explain why the condition $\iint_\Omega \omega dxdy = 0$ should be added in the function space $X_\ep$. Indeed,
by \eqref{steadyv}, we have
$$\lim_{y\rightarrow \pm\infty}\vec{u}_\ep(x,y) = (\pm1, 0) $$
 for $x\in\mathbb{T}_{2\pi}$ and $\epsilon\in[0,1)$.
Note that the perturbed flows have the same pattern of the velocity, i.e. the perturbed  velocity  $\vec{v}(x,y)$  satisfies
\begin{align*}\lim_{y\rightarrow \pm\infty}\vec{v}(x,y) = (\pm1, 0) \end{align*}
for $x\in\mathbb{T}_{2\pi}$, where $\vec{v}=(v_1,v_2)$.
\if0
Note that $\omega(x,y)$ and $\psi(x,y)$ can be written in the following Fourier series form:
$$\omega(x,y) = \omega^0(y) + \sum_{k\neq 0}e^{ikx}\omega^k(y) \text{ with } \omega^k(y) = \frac{1}{2\pi}\int_{0}^{2\pi} \omega(x,y) e^{-ikx} dx,$$
$$\psi(x,y) = \psi^0(y) + \sum_{k\neq 0}e^{ikx}\psi^k(y) \text{ with } \psi^k(y) = \frac{1}{2\pi}\int_{0}^{2\pi} \psi(x,y) e^{-ikx} dx.$$
Since $-\Delta \psi(x,y) = \omega(x,y)$, we have $\omega^k(y) = k^2\psi^k(y) -\frac{d^2 \psi^k(y)}{dy^2}$.
\fi
So  the perturbed  vorticity $\tilde \omega$ satisfies
\begin{align}\label{perturbed vorticity condition}
  \iint_\Omega \tilde \omega(x,y) dxdy=-\int_0^{2\pi}v_1(x,y)|_{y=-\infty}^\infty dx=-4\pi= \iint_\Omega \omega_\ep(x,y) dxdy.
\end{align}
For the perturbation of  vorticity $\omega=\tilde \omega-\omega_\ep$, we thus add the condition $\iint_\Omega \omega dxdy = 0$  in  $X_\ep$.


To understand  linear stability of the steady state $\omega_\ep$, it suffices to study the spectrum of the operator $J_\ep L_\ep$ on  $X_\ep$. Based on Hamiltonian structure of the linearized equation \eqref{hami}, we will study the spectral distribution  of $J_\ep L_\ep$  by the index formula \eqref{index-formula-stuart} developed in \cite{lin2022instability}. To verify the assumptions  in the Index Theorem (see {\bf(H1)-(H3)} in Lemma \ref{theorem-index}) and compute the indices  $n^{0}(L_\ep)$ and $n^{-}(L_\ep)$ (i.e. the number of kernel and negative directions of the self-adjoint operator $L_\ep$),   we will define  a dual  elliptic operator $\tilde{A}_\ep$ on a Hilbert space $\tilde{X}_\ep$ of stream functions, and reduce the computation of  the two indices  to the kernel and negative dimensions of $\tilde{A}_\ep$.

We divide the discussions into the  case $\ep = 0$ (hyperbolic tangent shear flow) and the case $0<\ep<1 $ (Kelvin-Stuart's cat's eyes flows) separately.
\subsection{Dual quadratic form and variational problem for the shear case}
The advantage of the shear case $\ep = 0$ is that $g'(\psi_0) = {2}{\sech^2(y)}$ depends only on $y$, and thus, we can separate the variables $(x,y)$ of functions and reduce our discussions into   one dimensional problems.
\subsubsection{Space of Stream Functions, Poisson equation and energy quadratic form}
First, we define explicitly   the space of stream functions such that the Poisson equation \eqref{Poisson equation} is well-posed in this space.
\begin{lemma}\label{Hilbert}
The function space
\begin{align}\label{tilde-X0}
\tilde{X}_0 = \left\{ \psi \bigg| \|\nabla \psi\|_{L^2(\Omega)} < \infty\quad {\rm{ and }}\quad\widehat\psi_0(0)={1\over2\pi} \int_{0}^{2\pi} \psi(x, 0)d x= 0 \right\}
\end{align} equipped with the inner product $$(\psi_1, \psi_2) = \iint_\Omega \nabla \psi_1 \cdot \nabla \psi_2 dxdy, \quad \forall\; \psi_1, \psi_2 \in \tilde{X}_0$$ is a Hilbert space.
\end{lemma}
Note that two functions differing from a constant belong to a same element in the space $\dot{H}^1(\Omega)$. We add the condition $\widehat \psi_0(0)={1\over 2\pi}\int_{0}^{2\pi} \psi(x, 0)d x= 0$ in \eqref{tilde-X0} to remove the disturbing of constants and make $\tilde{X}_0 $ a Hilbert space.
\begin{proof}
First, we prove that $\|\psi\|_{\tilde{X}_0}= \|\nabla \psi\|_{L^2(\Omega)} = 0$ implies $\psi = 0$ in $\tilde{X}_0$. Since $\psi(x,y)=\sum_{k\in\mathbb{Z}}\widehat{\psi}_{k}(y)e^{ik x}$,  we have
\begin{align}\label{tilde-X0-norm}
\|\nabla \psi\|_{L^2(\Omega)}^2
=  2\pi \left( \int_{-\infty}^{+\infty} \sum_{k\neq0} k^2 \left|\widehat{\psi}_k(y)\right|^2 dy + \int_{-\infty}^{+\infty}\left( \left| \widehat{\psi}_0'(y)\right|^2  +  \sum_{k\neq 0}\left| \widehat{\psi}_k'(y)\right|^2\right) dy \right).
\end{align}
Then we infer from $\|\nabla \psi\|_{L^2(\Omega)} = 0$ that
$ \widehat{\psi}_k= 0$ for $k\neq 0$   and $\widehat{\psi}_0' =0$.
By the condition $\widehat{\psi}_0(0)= 0$, we have
$$\widehat{\psi}_0(y) = \widehat{\psi}_0(0) + \int_0^{y}\widehat{\psi}_0'(s) ds = 0$$
for $y \in \mathbb{R}$.
So $\widehat{\psi}_k = 0$ for  $k\in\mathbb{Z}$, and thus, $\psi= 0$.
Now we prove the completeness of the space $\tilde{X}_0$.
Let $\{\psi_m \}_{m=1}^{+\infty}$ be a Cauchy sequence in $\tilde{X}_0$, i.e.
$\|\psi_m - \psi_n\|_{\tilde{X}_0} \to 0$ as $m,n \rightarrow \infty$,
where
\begin{align}\label{psi-m-dec}
\psi_m(x,y) = \widehat{\psi}_{m,0}(y) + \sum_{k\neq0}\widehat{\psi}_{m,k}(y)e^{ikx} =: \widehat{\psi}_{m,0}(y) + {\psi}_{m,\neq 0}(x,y)
\end{align}
for $m\geq1$.
By \eqref{tilde-X0-norm}, we have
\begin{align*}
\|\psi_m\|^2_{\tilde{X}_0}
& = \| \widehat{\psi}_{m,0}'\|^2_{L^2(\Omega)} + \|\nabla\psi_{m,\neq 0}\|^2_{L^2(\Omega)} < \infty.
\end{align*}
Since
\begin{align*}
\|\psi_{m,\neq 0}\|^2_{L^2(\Omega)} =&2\pi \int_{-\infty}^{+\infty} \sum_{k\neq0}  \left|\widehat{\psi}_{m,k}(y)\right|^2 dy \\
\leq&
2\pi \int_{-\infty}^{+\infty} \sum_{k\neq0}\left(  k^2 \left|\widehat{\psi}_{m,k}(y)\right|^2 +\left| \widehat{\psi}_{m,k}'(y)\right|^2\right) dy = \|\nabla\psi_{m,\neq 0}\|^2_{L^2(\Omega)}, \end{align*}
we have $\psi_{m,\neq 0}\in H^1(\Omega)$. Similarly, we have $\|\psi_{m,\neq 0} -  \psi_{n,\neq 0}\|_{H^1(\Omega)}^2\leq 2\|\nabla(\psi_{m,\neq 0} -  \psi_{n,\neq 0})\|_{L^2(\Omega)}^2$ $\leq2\|\psi_m -  \psi_n\|_{\tilde{X}_0}^2$ for $m,n\geq1$. Since $\|\psi_m - \psi_n\|_{\tilde{X}_0} \to 0$ as $m,n \rightarrow \infty$, we obtain that $\{\psi_{m,\neq 0} \}_{m=1}^{+\infty}$ is a Cauchy sequence in the Hilbert space $H^1(\Omega)$. Then
there exists $ \psi_{\neq0} \in H^1(\Omega)$   such that
$ \psi_{m,\neq 0} \to\psi_{\neq0}$ in $H^1(\Omega)$. By the Trace Theorem, $\{\psi_{m,\neq 0}(\cdot,0) \}_{m=1}^{+\infty}$ is a Cauchy sequence in $L^{2}(\mathbb{T}_{2\pi})$ (and thus in $L^{1}(\mathbb{T}_{2\pi})$). Then
$$\widehat{\psi}_{\neq0,0}(0)={1\over2\pi}\int_{0}^{2\pi}\psi_{\neq0}(x,0)dx=\lim_{m\to\infty}{1\over2\pi}\int_{0}^{2\pi}\psi_{\neq0}(x,0)dx=0.$$
Thus, $\widehat{\psi}_{\neq0,0}\in\tilde{X}_0$.
Since $ \|\widehat{\psi}_{m,0}'-\widehat{\psi}_{n,0}'\|_{L^2(\Omega)}\leq \|{\psi}_{m}-{\psi}_{n}\|_{\tilde{X}_0}$,
 $\{\widehat{\psi}_{m,0}' \}_{m=1}^{+\infty}$ is a Cauchy sequence in the Hilbert space $L^2(\Omega)$.
Thus, there exists $ \psi^0_* \in L^2(\Omega)$ such that
$\widehat{\psi}_{m,0}'\to  \psi^0_*$ in $L^2(\Omega)$.
Now we define $$\psi^0(y) = \int_{0}^{y} \psi^0_*(s) ds\quad\text{for}\quad y\in\mathbb{R}.$$
Then $\psi^0(0) = 0$ and
$ \widehat{\psi}_{m,0} \to \psi^{0}$ in $\tilde{X}_0$.
Let $\psi^*(x,y) = \psi^0(y) + \psi_{\neq0}(x,y)$ for $(x,y)\in\Omega$. Then $\psi^*\in \tilde{X}_0$ and
$$\|\psi_m - \psi^*\|_{\tilde{X}_0}\leq  \|\widehat\psi_{m,0} - \psi^0\|_{\tilde{X}_0} +  \| \psi_{m,\neq 0} -  \psi_{\neq0}\|_{\tilde{X}_0} \to0$$
as $m\to\infty$. Thus, $\tilde{X}_0$ is a Hilbert space.
\end{proof}
\subsubsection{Poincar\'e Inequalities}
First, we give a Poincar\'e-type inequality for functions with exponential decay  weight.
\begin{lemma}[Poincar\'e inequality I-$0$]\label{poincare1}
For any $\psi \in \tilde{X}_0$, we have
\begin{align}\label{Poincare inequality I022}
\iint_\Omega g'(\psi_0)|\psi|^2 dxdy  \leq C \|\nabla \psi\|_{L^2(\Omega)}^2.
\end{align}
\end{lemma}
\begin{proof}
\if0
We can write $\psi(x, y) \in \tilde{X}_0$ in the Fourier series form
$$\psi(x,y) = \psi^0(y) + \sum_{k\neq 0}e^{ikx}\psi^k(y), $$
with $$\psi^k(y) = \frac{1}{2\pi}\int_{0}^{2\pi} \psi(x,y) e^{-ikx} dx.$$
So
\begin{align*}
\|\nabla \psi\|_{L^2(\Omega)}^2
& = \iint_\Omega \psi_x^2 + \psi_y^2 dxdy \\
& = \iint_\Omega \sum_{k\neq0} k^2 (\psi^k(y))^2 dxdy + \iint_\Omega (\frac{d}{dy} \psi^0(y))^2  +  \sum_{k\neq 0}(\frac{d}{dy} \psi^k(y))^2 dxdy\\
& =  2\pi \left( \int_{-\infty}^{+\infty} \sum_{k\neq0} k^2 (\psi^k(y))^2 dy + \int_{-\infty}^{+\infty} (\frac{d}{dy} \psi^0(y))^2  +  \sum_{k\neq 0}(\frac{d}{dy} \psi^k(y))^2 dy \right).\\
\end{align*}
\fi
For $\psi \in \tilde{X}_0$, we have
\begin{align*}
\iint_\Omega g'(\psi_0)|\psi|^2 dxdy
& = 2\pi\left( \int_{-\infty}^{+\infty} g'(\psi_0) \left|\widehat{\psi}_0\right|^2 dy  + \int_{-\infty}^{+\infty} g'(\psi_0) \sum_{k\neq 0}\left|\widehat{\psi}_k\right|^2 dy \right)\\
& = 2\pi (I + II).
\end{align*}
Since $0<g'(\psi_0(y)) = 2 \sech^2(y) \leq 2$ for $y\in\mathbb{R}$, we get by \eqref{tilde-X0-norm} that for the part of non-zero modes,
\begin{align*}
II \leq 2\int_{-\infty}^{+\infty} \sum_{k\neq 0}\left|\widehat{\psi}_k\right|^2 dy
\leq C \|\nabla \psi\|_{L^2(\Omega)}^2.
\end{align*}
For the  part of zero mode, by the fact that
$\widehat{\psi}_0(0)  = 0$, we have
\begin{align*}
I& = \int_{-\infty}^{+\infty} g'(\psi_0) \left|\int_0^y \widehat{\psi}_0'(s) ds\right|^2 dy
 \leq\|\widehat{\psi}_0'\|_{L^2(\mathbb{R})}^2 \int_{-\infty}^{+\infty} g'(\psi_0)|y| dy
\leq C\|\nabla \psi\|_{L^2(\Omega)}^2
\end{align*}
since $ g'(\psi_0)$ decays exponentially near $\pm\infty$.
\end{proof}


We define  a $1$-dimensional projection operator $P_0$ on $ \tilde{X}_0$ by
\begin{align}\label{P0-psi-def}
P_0\psi = \frac{\iint_\Omega g'(\psi_0)\psi dxdy}{\iint_\Omega g'(\psi_0) dxdy}=\frac{\iint_\Omega g'(\psi_0)\psi dxdy}{8\pi},\quad \psi \in \tilde{X}_0,
\end{align}
where we used
 $$\iint_\Omega g'(\psi_0) dxdy = \int_{-\infty}^{\infty} \int_0^{2\pi} 2 \sech^2(y) dxdy = 8\pi.$$
The projection $P_0$ will be used later to introduce  a suitable dual elliptic operator acting at the stream functions.
\begin{Corollary}\label{projection}
 The projection operator $P_0 $ is well-defined  on $ \tilde{X}_0$.
\end{Corollary}
\begin{proof}
By  Lemma \ref{poincare1}, we have
\begin{align}\nonumber
|P_0\psi|
& \leq \frac{1}{8\pi} \iint_\Omega g'(\psi_0)|\psi| dxdy
 \leq \frac{1}{8\pi} \left(\iint_\Omega g'(\psi_0)|\psi|^2 dxdy\right)^{1/2} \left( \iint_\Omega g'(\psi_0)dxdy \right)^{1/2} \\\label{p0-psi-estimates-2}
& \leq C \|\nabla \psi\|_{L^2(\Omega)}.
\end{align}
\end{proof}
Next, we give another Poincar\'e-type  inequality, which involves the projection defined above.
\begin{lemma}[Poincar\'e inequality II-$0$]\label{poincare2}
For any $\psi \in \tilde{X}_0$,
we have
\begin{align}\label{Poincare inequality II022}
\iint_\Omega g'(\psi_0)|\psi - P_0\psi|^2 dxdy  \leq C \|\nabla \psi\|_{L^2(\Omega)}^2.
\end{align}
\end{lemma}
\begin{proof}
By Corollary \ref{projection}, we have
\begin{align}\label{projection-estimate-nabla-psi}
\iint_\Omega g'(\psi_0)|P_0\psi|^2 dxdy = 8\pi |P_0\psi|^2 \leq C \|\nabla \psi\|_{L^2(\Omega)}^2.
\end{align}
Then
\begin{align*}
\iint_\Omega g'(\psi_0)|\psi - P_0\psi|^2 dxdy  \leq 2 \iint_\Omega g'(\psi_0)\left(|\psi|^2  + |P_0\psi|^2 \right)dxdy  \leq C  \|\nabla \psi\|_{L^2(\Omega)}^2
\end{align*}
by  Lemma \ref{poincare1} and \eqref{projection-estimate-nabla-psi}.
\end{proof}
Now we consider the existence and uniqueness of the weak solution to the Poisson equation \eqref{Poisson equation} in $\tilde{X}_0$.
\begin{lemma}\label{1-1correspond}
For  $\omega \in X_0$, the Poisson equation
\eqref{Poisson equation}
has a unique weak solution in $\tilde{X}_0$.
\end{lemma}
\begin{proof}
By  Lemma \ref{poincare1},  we have
\begin{align*}
\iint_\Omega \omega \tilde {\psi} dxdy
& \leq \left( \iint_\Omega \frac{|\omega|^2}{g'(\psi_0)} dxdy  \right)^{1/2} \left( \iint_\Omega g'(\psi_0) |\tilde\psi|^2 dxdy  \right)^{1/2}  \leq C \|\omega\|_{X_0} \|\tilde{\psi}\|_{\tilde{X}_0}
\end{align*}
for any $\tilde{\psi}\in\tilde{X}_0$. Note that $\tilde{X}_0$ is a Hilbert space by Lemma \ref{Hilbert}. Thus,
by the Riesz Representation Theorem, there exists a unique $\psi \in \tilde{X}_0$ such that
$$\iint_\Omega \omega \tilde{\psi} dxdy = \langle \omega, \tilde{\psi} \rangle = (\psi, \tilde{\psi}) = \iint_\Omega \nabla \psi \cdot \nabla \tilde{\psi}dxdy.$$
Then $\psi $ is the unique weak solution in $\tilde X_0$ to the Poisson equation \eqref{Poisson equation}.
\end{proof}

For $\omega \in X_0$, we denote $(-\Delta)^{-1}\omega\in\tilde{X}_0$ to be the weak solution of the Poisson equation
\eqref{Poisson equation}.
Then we prove that the bilinear form
\begin{align}\label{L0-quadratic form}
 \langle L_0\omega_1,\omega_2\rangle=\iint_\Omega\left(\frac {\omega_1\omega_2} {g'(\psi_0)} - (-\Delta)^{-1}\omega_1\omega_2 \right) dxdy,\quad\omega_1,\omega_2\in X_0
\end{align}
 is bounded and symmetric  on $ X_0$.
\begin{lemma}\label{Lbounded}
For  $\omega_1,\omega_2 \in X_0$, we have
$\langle L_0 \omega_1, \omega_2 \rangle=\langle \omega_1, L_0 \omega_2 \rangle \leq C\|\omega_1\|_{X_0}\|\omega_2\|_{X_0}.$
\end{lemma}
\begin{proof}
For $\omega \in X_0$, let $\psi=(-\Delta)^{-1}\omega\in\tilde{X}_0$, we infer from  Lemma \ref{poincare1} that
\begin{align*}
\|\psi\|_{\tilde X_0}^2=\iint_\Omega \omega \psi dxdy
&   \leq C\|\omega\|_{X_0} \|\psi\|_{\tilde{X}_0},
\end{align*}
which gives $\|\psi\|_{\tilde X_0}\leq C \|\omega\|_{X_0}$. Let $\psi_i=(-\Delta)^{-1}\omega_i\in\tilde{X}_0$ for $i=1,2$. Then
\begin{align*}
\langle L_0 \omega_1, \omega_2 \rangle
 = &\iint_\Omega \left(\frac{\omega_1\omega_2}{g'(\psi_0)} dxdy -  \nabla\psi_1\cdot\nabla\psi_2\right) dxdy=
 \langle  \omega_1, L_0\omega_2 \rangle
\end{align*}
and
\begin{align*}
\langle L_0 \omega_1, \omega_2 \rangle
 \leq& \|\omega_1\|_{X_0}\|\omega_2\|_{X_0} + \|\psi_1\|_{\tilde{X}_0}\|\psi_2\|_{\tilde{X}_0}
 \leq C\|\omega_1\|_{X_0}\|\omega_2\|_{X_0}.
\end{align*}
\end{proof}

\subsubsection{Compact embedding lemma   and the variational problems}
% Elliptic Operator Definition
Define
 \begin{align}\label{A0}
 \tilde{A}_0=-\Delta-g'(\psi_0)(I - P_0): \tilde{X}_0 \rightarrow \tilde{X}_0^*,
\end{align}
where the negative Laplacian operator should be understood in the weak sense.
Then
\begin{align}\label{A0-quadratic form}
 \langle\tilde{A}_0\psi,\psi\rangle=\iint_\Omega|\nabla\psi|^2-g'(\psi_0)(\psi - P_0\psi)^2dxdy,\quad\psi\in \tilde{X}_0
\end{align}
defines a  bounded symmetric quadratic form on $\tilde{X}_0$ by the Poincar\'e inequality II-0 \eqref{Poincare inequality II022}.
Define another elliptic operator without the projection
\begin{equation}\label{A0 without projection}
A_0 =-\Delta -g'(\psi_0):\tilde{X}_0 \rightarrow \tilde{X}_0^*.
\end{equation}
The corresponding quadratic form
\begin{align*}
\langle A_0 \psi,\psi\rangle=\iint_{\Omega}\left(|\nabla \psi|^2-g'(\psi_0)|\psi|^2\right)dxdy,\quad\psi\in \tilde{X}_0
\end{align*}
is bounded and symmetric  on $\tilde{X}_0$ by the Poincar\'e inequality I-$0$ \eqref{Poincare inequality I022}.
Then
\begin{align}\label{tilde A0-A0}
\langle\tilde  A_0 \psi,\psi\rangle=\langle A_0 \psi,\psi\rangle+{\left(\iint_\Omega g'(\psi_0)\psi dxdy\right)^2\over \iint_\Omega g'(\psi_0)dxdy}=\langle A_0 \psi,\psi\rangle+8\pi(P_0 \psi)^2,\quad \psi\in\tilde X_0,
\end{align}
where we used $ \iint_\Omega g'(\psi_0)dxdy=8\pi$.
In particular,
\begin{equation*}
n^{\leq0}(\tilde A_0)\leq n^{\leq0}(A_0),\quad n^{-}(\tilde A_0)\leq n^{-}(A_0),
\end{equation*}
where $n^{\leq0}(\tilde A_0)$ and $n^-(\tilde A_0)$ are the number of non-positive and negative eigenvalues of $\tilde A_0$, respectively.
The operator $A_0$ and its quadratic form are useful in our study on nonlinear stability of the steady states.

Then we show that the study  on the dimensions of kernel and negative subspaces of the quadratic form $\langle L_0\cdot,\cdot\rangle$
defined in \eqref{L0-quadratic form}
  could be reduced to the corresponding dimensions for $ \langle\tilde{A}_0\cdot,\cdot\rangle$.
\begin{lemma}\label{equal-indices0}
\begin{align*}
\dim\ker(L_0)=\dim\ker(\tilde{A}_0) \quad {\rm{and}} \quad n^-(L_0)=n^-(\tilde{A}_0).
\end{align*}
\end{lemma}
\begin{proof}
First, we prove that $\dim\ker(L_0) =\dim\ker(\tilde{A}_0)$.

For  $\omega \in \ker L_0$, let $\psi = (-\Delta)^{-1} \omega\in \tilde X_0$, we have
\begin{align}\label{L0-dual-omega}\langle L_0 \omega,\tilde \omega\rangle = \iint_\Omega\left(\frac{\omega\tilde \omega}{g'(\psi_0)} - \psi\tilde \omega\right) dxdy = 0,\quad\forall\; \tilde \omega\in X_0.
\end{align}
For any $\tilde \psi\in \tilde X_0$, we define $\omega_{\tilde \psi}=g'(\psi_0)(\tilde \psi - P_0\tilde\psi)$.
Then $\iint_\Omega\omega_{\tilde \psi}dxdy=0$, and thus, $\omega_{\tilde \psi}\in X_0$ by  Lemma \ref{poincare2}. By \eqref{L0-dual-omega}, we have
\begin{align*}\langle L_0 \omega, \omega_{\tilde \psi}\rangle= \iint_\Omega\left(\omega\tilde \psi - g'(\psi_0)\psi(\tilde\psi - P_0\tilde\psi)\right)dxdy =  \iint_\Omega\left(\omega\tilde \psi - g'(\psi_0)(\psi-P_0\psi)\tilde\psi\right) dxdy = 0,\end{align*}
where we used $\iint_\Omega\omega dxdy=0$ and $\iint_\Omega g'(\psi_0)(\tilde\psi - P_0\tilde\psi)dxdy=\iint_\Omega g'(\psi_0)(\psi-P_0\psi)dxdy=0$. This implies that $\psi\in \ker(\tilde{A}_0)$ since
\begin{align*}\langle\tilde{A}_0 \psi,\tilde \psi\rangle= \iint_\Omega\left(\omega\tilde \psi - g'(\psi_0)(\psi-P_0\psi)\tilde\psi \right)dxdy = 0,\quad\forall \; \tilde \psi\in \tilde X_0.\end{align*}
 Thus,  $\dim\ker(L_0) \leq \dim\ker(\tilde{A}_0)$.

For  $\psi \in \ker \tilde{A}_0$, let $\omega = g'(\psi_0)(\psi - P_0\psi)$, we have $\omega\in X_0$ and
\begin{align}\label{tilde-A0-psi}\langle\tilde{A}_0 \psi,\tilde\psi\rangle =\iint_\Omega \left(-\Delta \psi\tilde\psi - g'(\psi_0)(\psi - P_0\psi)\tilde\psi \right)dxdy =0,\quad\forall \; \tilde \psi\in \tilde X_0.\end{align}
For any $\tilde \omega\in X_0$, let $\psi_{\tilde \omega}=(-\Delta)^{-1}\tilde \omega\in \tilde X_0$, we have
\begin{align*}\langle L_0 \omega, \tilde{\omega}\rangle =& \iint_\Omega\left( {\omega\tilde \omega\over g'(\psi_0)}-(-\Delta)^{-1}\omega\tilde\omega \right)dxdy= \iint_\Omega \left((\psi-P_0\psi)\tilde \omega-\omega(-\Delta)^{-1}\tilde\omega \right) dxdy\\
=& \iint_\Omega \left(\psi(-\Delta)\psi_{\tilde \omega}-g'(\psi_0)(\psi - P_0\psi)\psi_{\tilde \omega} \right)dxdy\\
=&\iint_\Omega  \left( -\Delta\psi\psi_{\tilde \omega}-g'(\psi_0)(\psi - P_0\psi)\psi_{\tilde \omega}\right) dxdy=0
\end{align*}
by \eqref{tilde-A0-psi}, which gives  $L_0 \omega = 0$.  This proves  $\dim\ker(L_0) \geq \dim\ker(\tilde{A}_0)$, and thus, $\dim\ker(L_0) = \dim\ker(\tilde{A}_0)$.


For any $\omega \in X_0$, let $\psi = (-\Delta)^{-1} \omega\in \tilde X_0$ and we have
\begin{align}\nonumber
\langle L_0 \omega, \omega\rangle
& = \iint_\Omega \left(\frac{|\omega|^2}{g'(\psi_0)} -   \psi\omega\right) dxdy  = \iint_\Omega |\nabla \psi|^2 dxdy + \iint_\Omega \left(\frac{|\omega|^2}{g'(\psi_0)} -  2 \psi\omega\right) dxdy \\\nonumber
& =\|\nabla \psi\|_{L^2(\Omega)}^2 +\iint_\Omega\left( \frac{|\omega|^2}{g'(\psi_0)} -  2 (\psi-P_0\psi)\omega \right)dxdy\\\nonumber
& \geq\|\nabla \psi\|_{L^2(\Omega)}^2 - \iint_\Omega g'(\psi_0)(\psi - P_0\psi)^2 dxdy \\\label{L0omega-omega}
& = \|\nabla \psi\|_{L^2(\Omega)}^2 - \iint_\Omega g'(\psi_0)(\psi - P_0\psi)\psi dxdy  = \langle\tilde{A}_0 \psi, \psi\rangle.
\end{align}
Thus, $n^{\leq 0} (L_0) \leq n^{\leq 0} (\tilde{A}_0)$.

For any $\psi \in \tilde{X}_0$, let $\tilde{\omega} = g'(\psi_0)(\psi - P_0\psi)$, we have
$\tilde{\omega} \in X_0$, $\psi_{\tilde{\omega}} = (-\Delta)^{-1}\tilde{\omega}\in \tilde X_0$, and
\begin{align*}
\langle\tilde{A}_0 \psi, \psi\rangle
& = \iint_\Omega\left( |\nabla \psi|^2  - g'(\psi_0)(\psi - P_0\psi)^2 \right)dxdy
 = \iint_\Omega \left(|\nabla \psi|^2  - \frac{\tilde{\omega}^2}{g'(\psi_0)}\right)dxdy \\
& = \iint_\Omega \left(\frac{\tilde{\omega}^2}{g'(\psi_0)} + |\nabla \psi|^2  - 2  \tilde{\omega}(\psi - P_0\psi)\right)dxdy\\
&= \iint_\Omega \left(\frac{\tilde{\omega}^2}{g'(\psi_0)} + |\nabla \psi|^2  - 2  \tilde{\omega}\psi\right)dxdy
 = \iint_\Omega \left(\frac{\tilde{\omega}^2}{g'(\psi_0)} +  |\nabla \psi|^2 - 2  \nabla\psi_{\tilde\omega}\cdot\nabla\psi \right)dxdy \\
  &\geq \iint_\Omega \left(\frac{\tilde{\omega}^2}{g'(\psi_0)} -  |\nabla \psi_{\tilde \omega}|^2\right) dxdy= \langle L_0 \tilde{\omega}, \tilde{\omega}\rangle.
\end{align*}
This proves $n^{\leq 0} (L_0) \geq n^{\leq 0} (\tilde{A}_0)$. Then $n^{\leq 0} (L_0) =n^{\leq 0} (\tilde{A}_0)$, which, along with
$\dim\ker(L_0) = \dim\ker(\tilde{A}_0)$, gives $n^{-} (L_0) =n^{-} (\tilde{A}_0)$.
\end{proof}

To compute $n^{-}(\tilde{A}_0)$, we study the variational problem
\begin{align}\label{variational problem1}
\lambda_1= \inf_{\psi\in\tilde X_0}{\iint_\Omega|\nabla\psi|^2dxdy\over\iint_\Omega g'(\psi_0)(\psi - P_0\psi)^2dxdy}.
\end{align}
$\lambda_1$ is finite due to the Poincar\'e inequality II-$0$ \eqref{Poincare inequality II022}.
We
need the following compact embedding result.
\begin{lemma}\label{compact2P}
$(1)$ $\tilde X_0$ is compactly embedded in $L_{g'(\psi_0)}^2(\Omega)$.

 $ (2)$ $\tilde X_0$ is compactly embedded in
 \begin{equation*}
 Z_{0}:=\left\{\psi\bigg|\iint_{\Omega}g'(\psi_0)|\psi-P_0\psi|^2dxdy<\infty\right\}.
  \end{equation*}
\end{lemma}


\begin{proof} First, we prove (1).
By the Poincar\'e inequality I-$0$ \eqref{Poincare inequality I022}, $\tilde X_0$ is embedded in $L_{g'(\psi_0)}^2(\Omega)$. To prove that the embedding is compact, let $\{\psi_n\}_{n\geq1}$ be a bounded sequence in $\tilde{X}_0$. We decompose $\psi_n=\widehat{\psi}_{n,0}+\psi_{n,{\neq0}}$ as in \eqref{psi-m-dec}.  By  \eqref{tilde-X0-norm} we have
 \begin{align}\label{psi-n-0-mode-unidform bound}
 \|\widehat{\psi}_{n,0}'\|_{L^2(\mathbb{R})}<C \quad\text{and} \quad\|\psi_{n,{\neq0}}\|_{H^1(\Omega)}<C,\quad n\geq1.
 \end{align}
For any $\kappa > 0$, there exists $K>0$ such that $g'(\psi_0(y))=2\sech^2(y)<\kappa$ for $y\in (-\infty,-K]\cup[K,\infty)$, and
 \begin{align*}
 \int_{(-\infty, -K)\cup(K,\infty)}g'(\psi_0)|y|dy =2\int_{(-\infty, -K)\cup(K,\infty)} \sech^2(y)|y|dy <\kappa.
 \end{align*}
  Then by \eqref{psi-n-0-mode-unidform bound} and $\widehat{\psi}_{n,0}(0)=0$ for $n\geq1$, we have
\begin{align*}
&\int_{(-\infty, -K)\cup(K,\infty)}g'(\psi_0)(\widehat{\psi}_{n,0}-\widehat{\psi}_{m,0})^2d y\\
 \leq& \|\widehat{\psi}_{n,0}'-\widehat{\psi}_{m,0}'\|_{L^2(\mathbb{R})}^2\int_{(-\infty, -K)\cup(K,\infty)}g'(\psi_0)|y|dy \leq C\kappa
\end{align*}
and
\begin{align*}
\int_0^{2\pi}\int_{(-\infty, -K)\cup(K,\infty)} g'(\psi_0) (\psi_{n,{\neq 0}}-\psi_{m,{\neq 0}})^2 dydx \leq  \kappa \| \psi_{n,{\neq 0}} - \psi_{m,{\neq 0}}\|_{H^1(\Omega)}^2 \leq C \kappa
\end{align*}
for $m,n\geq1$.
Thus,
\begin{align*}
&\int_0 ^ {2\pi} \int_{(-\infty, -K)\cup(K,\infty)} g'(\psi_0) (\psi_n -\psi_m)^2 dydx \\
\leq&2\int_0 ^ {2\pi} \int_{(-\infty, -K)\cup(K,\infty)} g'(\psi_0) \left((\widehat{\psi}_{n,0}-\widehat{\psi}_{m,0})^2+ (\psi_{n,{\neq 0}}-\psi_{m,{\neq 0}})^2 \right)dydx\leq C \kappa.
\end{align*}
Since $\|\widehat{\psi}_{n,0}\|_{L^2(-K,K)}^2\leq 2K^2\|\widehat{\psi}_{n,0}'\|_{L^2(-K,K)}^2\leq C_K$, we infer from \eqref{psi-n-0-mode-unidform bound} that
$\{\sqrt{g'(\psi_0)}\psi_n\}_{n\geq1}$ is a  bounded sequence in $H^1(\mathbb{T}_{2\pi} \times [-K,K])$. Since the embedding $H^1\hookrightarrow L^2(\mathbb{T}_{2\pi} \times [-K,K])$ is compact, then up to a subsequence, there exists $N>0$ such that $\|\psi_n -\psi_m\|_{L^2_{g'(\psi_0)}(\mathbb{T}_{2\pi} \times [-K,K])}=\|\sqrt{g'(\psi_0)}(\psi_n-\psi_m)\|_{L^2(\mathbb{T}_{2\pi} \times [-K,K])}<\kappa$ for $m,n>N$. Thus, up to a subsequence,
\begin{align*}
& \|\psi_n -\psi_m\|_{L^2_{g'(\psi_0)}(\Omega)}^2 = \|\sqrt{g'(\psi_0)}(\psi_n-\psi_m)\|_{L^2(\mathbb{T}_{2\pi} \times [-K,K])}^2  \\ &+\|\sqrt{g'(\psi_0)}(\psi_n-\psi_m)\|_{L^2(\mathbb{T}_{2\pi} \times ((-\infty, -K)\cup(K,\infty)))}^2   \leq \kappa^2+ C \kappa
\end{align*}
for $m,n>N$, which implies that there exists $\psi_* \in L^2_{g'(\psi_0)}(\Omega)$ such that $\psi_n\rightarrow\psi_*$ in $ L^2_{g'(\psi_0)}(\Omega)$.

Then we prove (2).
By the Poincar\'e inequality II-$0$ \eqref{Poincare inequality II022}, $\tilde X_0$ is embedded in $Z_0$. Let $\{\psi_n\}_{n\geq1}$ be a bounded sequence in $\tilde{X}_0$. By (1), we know that there exists $\psi_* \in L^2_{g'(\psi_0)}(\Omega)$ such that, up to a subsequence, $\psi_n \rightarrow \psi_*$ in $L^2_{g'(\psi_0)}(\Omega)$, and  it follows from \eqref{p0-psi-estimates-2} that
\begin{align*}
|P_0( \psi_n - \psi_*)| \leq C \|\psi_n - \psi_*\|_{L^2_{g'(\psi_0)}(\Omega)} \rightarrow 0\quad \text{as}\quad n \rightarrow  \infty.
\end{align*}
Thus, up to a subsequence, we have
\begin{align*}
&\iint_{\Omega} g'(\psi_0) \left((\psi_n - \psi_*) - P_0 (\psi_n -  \psi_*)\right)^2 dxdy \\
\leq & 2 \iint_{\Omega} g'(\psi_0) \left((\psi_n - \psi_*)^2 + \left(P_0 (\psi_n -  \psi_*)\right)^2\right) dxdy \\
\leq & 2\|\psi_n - \psi_*\|_{L^2_{g'(\psi_0)}(\Omega)}^2 + C|P_0( \psi_n -  \psi_*)|^2 \\
\leq & C \|\psi_n - \psi_*\|_{L^2_{g'(\psi_0)}(\Omega)}^2 \rightarrow 0 \quad\text{as}\quad n \rightarrow  \infty.
\end{align*}
\end{proof}


Since  the embedding  $\tilde X_0\hookrightarrow Z_{0}$ is  compact, a standard argument in variational method implies that
 the  infimum in  \eqref{variational problem1} can be attained in $\tilde X_0$, and we can
inductively define $\lambda_n$ as follows for $n\geq1$,
\begin{align}\nonumber
\lambda_n=& \inf_{\psi \in \tilde X_0, (\psi, \psi_{i})_{Z_0} = 0, i = 1, 2, \cdots, n-1}{\iint_\Omega|\nabla\psi|^2dxdy\over\iint_\Omega g'(\psi_0)(\psi - P_0\psi)^2dxdy}\\\label{variational problem2}
=&\min_{\psi \in \tilde X_0, (\psi, \psi_{i})_{Z_0} = 0, i = 1, 2, \cdots, n-1}{\iint_\Omega|\nabla\psi|^2dxdy\over\iint_\Omega g'(\psi_0)(\psi - P_0\psi)^2dxdy},
\end{align}
where the infimum for $\lambda_i$ is attained at $\psi_{i} \in \tilde X_0$ and $\iint_\Omega g'(\psi_0)(\psi_{i} - P_0\psi_{i})^2 dxdy = 1$, $1\leq i \leq n-1$.
To solve the variational problem \eqref{variational problem2}, we compute the 1-order variation of $G(\psi)={\iint_\Omega|\nabla\psi|^2dxdy\over\iint_\Omega g'(\psi_0)(\psi - P_0\psi)^2dxdy}$ at $\psi_{n}$:
\begin{align*}
\frac{d}{d \tau} G(\psi_{n} + \tau \psi)|_{\tau = 0} = \iint_{\Omega} 2\left(-\Delta\psi_n - \lambda_ng'(\psi_0)(\psi_n - P_0\psi_n)\right)\psi dxdy,\quad\forall\; \psi\in \tilde X_0.
\end{align*}
Due to the fact that  $\widehat{\psi}_0(0)=0$ for $\psi\in \tilde X_0$, we
derive the corresponding Euler-Lagrangian equation
\begin{align}\label{elip0}
-\Delta \psi = \lambda g'(\psi_0)(\psi - P_0\psi)+a\delta(y), \quad \psi \in \tilde{X}_0,
\end{align}
where $\delta$ is the Dirac delta function and $a\in\mathbb{R}$ is to be determined. Thanks to the projection $P_0$, integrating \eqref{elip0} on $\Omega$, we have
$$
2\pi a=\iint_\Omega-\Delta \psi -\lambda g'(\psi_0)(\psi - P_0\psi) dxdy=0\;\;\Longrightarrow \;\;a=0,
$$
and thus,  we arrive at the associated eigenvalue problem
\begin{align}\label{elip02}
-\Delta \psi = \lambda g'(\psi_0)(\psi - P_0\psi), \quad \psi \in \tilde{X}_0.
\end{align}
Since $g'(\psi_0)$ depends only on $y$, we can use the Fourier expansion of
$\psi$ to separate the variables.
Since $\psi(x,y)=\sum_{k\in\mathbb{Z}}\widehat{\psi}_{k}(y)e^{ik x} \in \tilde{X}_0$, we infer from \eqref{tilde-X0-norm} that
\begin{align}\label{def-space-Y0Yk}
\widehat{\psi}_0 \in Y_0 = \{\phi | \phi \in \dot{H}^1(\mathbb{R}), \phi(0) = 0\}\quad\text{and}\quad\widehat{\psi}_{k} \in Y_1 =  H^1(\mathbb{R})\quad \text{for}\quad k\neq0.
\end{align}
Plugging the Fourier expansion $\psi(x,y)=\sum_{k\in\mathbb{Z}}\widehat{\psi}_{k}(y)e^{ik x}$ into \eqref{elip02}, we get the eigenvalue problem for the  $0$ mode
\begin{align}\label{mode0}
- \phi'' = 2\lambda \sech^2(y) (I - P_0) \phi, \quad \phi\in Y_0,
\end{align}
with $$P_0 \phi = \frac{1}{2} \int_{\mathbb{R}} \sech^2(y) \phi(y) dy,$$
and the eigenvalue problem for the  $k$ mode
\begin{align}\label{modek}
-  \phi'' + k^2 \phi = 2\lambda \sech^2(y)\phi, \quad \phi\in Y_1, \quad k \neq 0,
\end{align}
since $$P_0 (\phi e^{ikx}) =\frac{1}{4\pi} \iint_\Omega  \sech^2(y) \phi(y)e^{ikx} dxdy = 0.$$

\subsection{Exact solutions to the associated  eigenvalue problems for the shear case}
\subsubsection{A change of variable}\label{Change of variable}

Our motivation for introducing a change of variable is to understand the eigenvalue problem \eqref{mode0} for the $0$ mode.
By taking derivative of $-\Delta\psi_0=g(\psi_0)$ with respect to $y$, we obtain an eigenvalue $\lambda=1$ of \eqref{mode0} with a corresponding eigenfunction $\tanh(y)$, see also (16.3) in \cite{Laugesen2011}.
Thanks to the numerical simulation in Subsection \ref{eigenfunction-motivation}, we derive another eigenvalue $\lambda=3$ with a corresponding eigenfunction $\tanh^2(y)$.
Our observation is that all the eigenfunctions might be polynomials of $\tanh(y)$.
By putting the  polynomials of $\tanh(y)$ into \eqref{mode0}, we obtain five interesting eigenvalues and corresponding eigenfunctions as follows:
\begin{equation}\label{eigen value-function}\begin{aligned}\begin{array}{llll}
&\lambda_1=1=1, &\phi_1(y)=\tanh(y),\\
&\lambda_2=1+2=3, &\phi_2(y)=\tanh^2(y),\\
&\lambda_3=1+2+3=6, &\phi_3(y)=5\tanh^3(y)-3\tanh(y),\\
&\lambda_4=1+2+3+4=10, &\phi_4(y)=7\tanh^4(y)-6\tanh^2(y),\\
&\lambda_5=1+2+3+4+5=15, &\phi_5(y)=9\tanh^5(y)-10\tanh^3(y)+{15\over7}\tanh(y).
\end{array}
\end{aligned}
\end{equation}
This suggests us to  expect  that all the eigenvalues of \eqref{mode0} are $\lambda_n={n(1+n)\over2}$   with corresponding eigenfunctions  to be   polynomials of $\tanh(y)$.
With \eqref{eigen value-function} in mind, we  make a change of variable
\begin{align}\label{change of variable for 0 mode}\gamma = \tanh(y) \in (-1, 1).\end{align}
The novelty of this change of variable is that the eigenvalue problems  \eqref{mode0} for the $0$ mode and \eqref{modek} for the non-zero mode are surprisingly transformed to the well-known Legendre and general Legendre differential equations associated with  projection terms and specific function spaces, which is discussed in the next subsection. For the Kelvin-Stuart vortices $\omega_\epsilon$ with $0<\epsilon<1$, we also introduce a  change of variables, which is more delicate, to transform the corresponding eigenvalue problems to the  Legendre-type boundary value problems in Subsection \ref{change of variables for cat's eyes  flows}. This even makes our stability analysis for the Kelvin-Stuart vortices  closely related to the spherical harmonics.

In the new variables $(x,\gamma)$, we rewrite the spaces of stream functions $\tilde X_0$ and  $Z_{0}$,  Poincar\'e inequality I-II (see \eqref{Poincare inequality I022}, \eqref{Poincare inequality II022}) and the compact embedding   $\tilde X_0\hookrightarrow Z_{0}$, respectively. These statements in the new variables are also useful in establishing the correspondence of stream functions between the hyperbolic tangent shear case ($\ep=0$) and the cat's eyes case ($0<\ep<1$).

First, the space $\tilde X_0$ in \eqref{tilde-X0} is rewritten as the following space in the new variables $(x,\gamma)$.
\if0
\begin{lemma}\label{Hilbert}
The function space
\begin{align}\label{tilde-X0}
\tilde{X}_0 = \left\{ \psi \bigg| \|\nabla \psi\|_{L^2(\Omega)} < \infty\quad {\rm{ and }}\quad \int_{0}^{2\pi} \psi(x, 0)d x= 0 \right\}
\end{align} equipped with the inner product $$(\psi_1, \psi_2) = \iint_\Omega \nabla \psi_1 \cdot \nabla \psi_2 dxdy \quad \forall\; \psi_1, \psi_2 \in \tilde{X}_0$$ is a Hilbert space.
\end{lemma}
\fi
\begin{lemma}\label{Hilbert-new variables-0}
The function space
\begin{align}\label{tilde-Y0-def}
\tilde{Y}_0 = \left\{ \Psi\bigg|\iint_{\tilde \Omega}\left({1\over1-\gamma^2}|\Psi_{x}|^2+(1-\gamma^2)|\Psi_{\gamma}|^2\right)d x d\gamma< \infty \text{ and } \widehat{\Psi}_0(0)=0 \right\}
\end{align}
equipped with the inner product $$(\Psi_1, \Psi_2) = \iint_{\tilde{\Omega}}  \left({1\over1-\gamma^2}(\Psi_1)_{x}(\Psi_2)_{x} +(1-\gamma^2)(\Psi_1)_{\gamma}(\Psi_2)_{\gamma}\right)d x d\gamma, \quad \forall\; \Psi_1, \Psi_2 \in \tilde{Y}_0$$ is a Hilbert space, where $\tilde \Omega=\mathbb{T}_{2\pi}\times [-1,1]$.
\end{lemma}
\begin{proof}
For $\psi_i(x,y)=\Psi_i(x,\gamma)$, $i=1,2$, we have
\begin{align}\label{cor-0-new}
\iint_\Omega\nabla \psi_1\cdot\nabla \psi_2dxdy
=\iint_{\tilde \Omega}\left({1\over1-\gamma^2}(\Psi_1)_{x}(\Psi_2)_{x}+(1-\gamma^2)(\Psi_1)_{\gamma}(\Psi_2)_{\gamma}\right)d x d\gamma.
\end{align}
Moreover, $y=0\Longleftrightarrow\gamma=0$, and thus,
\begin{align}\label{0 mode x gamma}
\widehat{\psi}_0(0)=\widehat{\Psi}_0(0)
\end{align}
for $\psi(x,y)=\Psi(x,\gamma)$.
The conclusion follows from \eqref{cor-0-new}-\eqref{0 mode x gamma} and  the fact that $\tilde X_0$ is a Hilbert space by Lemma \ref{Hilbert}.
\end{proof}
Let  $\psi \in \tilde{X}_0$ and $\Psi \in \tilde{Y}_0$ such that $\psi(x,y) = \Psi(x, \gamma)$. It follows from \eqref{cor-0-new} that
\begin{align}\label{cor-0-new-norm}
\|\psi\|_{\tilde X_0}^2= \|\nabla \psi\|_{L^2(\Omega)}^2=\iint_{\tilde \Omega}\left({1\over1-\gamma^2}|\Psi_{x}|^2+(1-\gamma^2)|\Psi_{\gamma}|^2\right)d x d\gamma=\|\Psi\|_{\tilde Y_0}^2.
\end{align}
Corresponding to $P_0$ in \eqref{P0-psi-def}, we define a $1$-dimensional projection operator $\tilde P_0$ on $ \tilde{Y}_0$ by
 \begin{align}\label{def-tilde-P0-Psi}
 \tilde P_0\Psi = \frac{\iint_{\tilde\Omega} \Psi dxd\gamma}{\iint_{\tilde\Omega}  dxd\gamma}=\frac{\iint_{\tilde \Omega} \Psi dxd\gamma}{4\pi},\quad \Psi \in \tilde{Y}_0.
 \end{align}
Then we prove that $\tilde P_0$ is well-defined on $ \tilde{Y}_0$, and give the Poincar\'e-type inequalities in the new variables $(x,\gamma)$.
\begin{lemma}\label{Poincare ineqalities-new-variable0}
$(1)$ Poincar\'e inequality $\textup{I-}0'$:
\begin{align*}
\|\Psi\|_{L^2(\tilde\Omega)}^2  \leq C \iint_{\tilde \Omega}\left({1\over1-\gamma^2}|\Psi_{x}|^2+(1-\gamma^2)|\Psi_{\gamma}|^2\right)d x d\gamma=C\|\Psi\|_{\tilde Y_0}^2, \quad \Psi \in \tilde{Y}_0.
\end{align*}

$(2)$ The projection operator $\tilde P_0 $ is well-defined  on $ \tilde{Y}_0$, $|\tilde P_0\Psi|\leq C \|\Psi\|_{\tilde Y_0}$, and $P_0\psi=\tilde P_0\Psi$ for  $\psi \in \tilde{X}_0$ and $\Psi \in \tilde{Y}_0$ such that $\psi(x,y) = \Psi(x, \gamma)$.

$(3)$ Poincar\'e inequality $\textup{II-}0'$:
\begin{align*}
\iint_{\tilde \Omega}|\Psi - \tilde P_0\Psi|^2 dxd\gamma  \leq C \iint_{\tilde \Omega}\left({1\over1-\gamma^2}|\Psi_{x}|^2+(1-\gamma^2)|\Psi_{\gamma}|^2\right)d x d\gamma=C\|\Psi\|_{\tilde Y_0}^2, \quad \Psi \in \tilde{Y}_0.
\end{align*}
\end{lemma}
\begin{proof}  Let $\psi(x,y)=\Psi(x,\gamma)$. Then $\psi\in \tilde {X}_0$. First, we prove (1).
By Lemma  \ref{poincare1} and \eqref{cor-0-new-norm}, we have
\begin{align*}
&2\iint_{\tilde \Omega}|\Psi|^2 dxd\gamma=\iint_\Omega g'(\psi_0)|\psi|^2 dxdy  \\
\leq& C \|\nabla \psi\|_{L^2(\Omega)}^2=C \iint_{\tilde \Omega}\left({1\over1-\gamma^2}|\Psi_{x}|^2+(1-\gamma^2)|\Psi_{\gamma}|^2\right)d x d\gamma.
\end{align*}

Next, we prove (2). By \eqref{P0-psi-def} and \eqref{def-tilde-P0-Psi}, we have $P_0\psi=\tilde P_0\Psi$.  Thus, we infer from
\eqref{p0-psi-estimates-2} that
\begin{align*}
|\tilde P_0\Psi|=|P_0\psi|\leq C \| \psi\|_{\tilde X_0}=C \|\Psi\|_{\tilde Y_0}.
\end{align*}

Finally, we prove (3).
By Lemma \ref{poincare2}, $P_0\psi=\tilde P_0\Psi$ and \eqref{cor-0-new-norm} we have
\begin{align*}
&2\iint_{\tilde \Omega} |\Psi -\tilde P_0\Psi|^2 dxd\gamma =\iint_\Omega g'(\psi_0)|\psi - P_0\psi|^2 dxdy  \\
\leq & C \|\nabla \psi\|_{L^2(\Omega)}^2=C \iint_{\tilde \Omega}\left({1\over1-\gamma^2}|\Psi_{x}|^2+(1-\gamma^2)|\Psi_{\gamma}|^2\right)d x d\gamma.
\end{align*}
\end{proof}

Then we give the  compact embedding lemma in the new variables.
\begin{lemma}\label{compact2P-new-variable0}
$(1)$ $\tilde Y_0$ is compactly embedded in $L^2(\tilde \Omega)$.

$(2)$  $\tilde Y_0$ is compactly embedded in
 \begin{equation*}
 \tilde Z_{0}:=\left\{\Psi\bigg|\iint_{\tilde \Omega}|\Psi-\tilde P_0\Psi|^2dxd\gamma<\infty\right\}.
  \end{equation*}
\end{lemma}
\begin{proof} We only prove (2), and the proof of (1) is similar.
By Lemma \ref{Poincare ineqalities-new-variable0} (3), $\tilde Y_0$ is embedded in $\tilde Z_{0}$. Let $\{\Psi_n\}_{n\geq1}$ be a bounded sequence in $\tilde{Y}_0$ and $\psi_n(x,y)=\Psi_n(x,\gamma)$. Then it follows from \eqref{cor-0-new-norm} that  $\{\psi_n\}_{n\geq1}$ is a bounded sequence in $\tilde{X}_0$. By Lemma \ref{compact2P} (2), there exists $\psi_*\in Z_0$ such that up to a subsequence, $\|\psi_n-\psi_*\|_{Z_0}\to0$. Let $\Psi_*(x,\gamma)=\psi_*(x,y)$. Then $\Psi_*\in \tilde Z_0$ and up to a subsequence,  $\|\Psi_n-\Psi_*\|_{\tilde Z_0}=\|\psi_n-\psi_*\|_{Z_0}\to0$.
\end{proof}
\subsubsection{Solutions to the eigenvalue problems}\label{Solutions to the eigenvalue problem ep=0}
 We  study the eigenvalue problems \eqref{mode0} for the $0$ mode and \eqref{modek} for the non-zero modes, separately.\medskip


\noindent{\bf{Eigenvalue problem for the $0$ mode}}\medskip

In this part, we solve the eigenvalue problem \eqref{mode0} for the $0$ mode. We use the change of variable $\gamma=\tanh(y)$ and
denote $\phi(y)=\phi(\tanh^{-1}(\gamma))=\varphi(\gamma)$.
 Then $d\gamma=(1-\gamma^2)dy={1\over2}g'(\psi_0)dy$ and
\begin{align*}
\phi'(y)&=(1-\gamma^2)\varphi'(\gamma),\;\;\phi''(y)=(1-\gamma^2)(-2\gamma\varphi'(\gamma)+(1-\gamma^2)\varphi''(\gamma)),\\
P_0\phi&={1\over4}\int_{\mathbb{R}}g'(\psi_0)\phi(y)dy={1\over2}\int_{-1}^{1}\varphi(\gamma)d\gamma=:\hat P_0\varphi.
\end{align*}

Since
\begin{align}\label{phi-der-varohi-der}
\int_{\mathbb{R}}|\phi'(y)|^2dy=\int_{-1}^1(1-\gamma^2)|\varphi'(\gamma)|^2d\gamma,
\end{align}
 the space $Y_0$ (see \eqref{def-space-Y0Yk}) for $\phi$ in the variable $y$ is transformed to
\begin{equation*}
\hat Y_0=\left\{\varphi\bigg|\int_{-1}^1(1-\gamma^2)|\varphi'(\gamma)|^2d\gamma<\infty\text{ and }\varphi(0)=0 \right\}
\end{equation*}
for $\varphi$ in the new variable $\gamma$.
Thus, the eigenvalue problem \eqref{mode0} is transformed to
\begin{align}\label{eigenvalue problem for 0 mode}
-\left((1-\gamma^2)  \varphi'\right)' = 2 \lambda(\varphi-\hat{P}_0\varphi) \quad \text{on}\quad (-1,1),\quad\varphi\in \hat Y_0.
\end{align}
If we neglect the term $-2\lambda\hat P_0\varphi$ and change the space $\hat Y_0$ to $L^2(-1,1)$ for a while, \eqref{eigenvalue problem for 0 mode} surprisingly becomes the  Legendre equation
\begin{equation}\label{eigenvalue problem3}
-\left((1-\gamma^2)  \varphi'\right)' = 2 \lambda\varphi \quad \text{on}\quad (-1,1),\quad \varphi\in L^2(-1,1).
\end{equation}
If we require that  the solution is regular at $\gamma=\pm1$, then
it is well-known that the eigenvalues to the boundary value problems \eqref{eigenvalue problem3} are  $\lambda_n={n(n+1)\over2}$ for $n\geq0$, and the corresponding eigenfunctions are the Legendre polynomials
$
L_n(\gamma)={1\over2^nn!}{d^n\over d\gamma^n}(\gamma^2-1)^n.
$
Moreover, $\{L_n\}_{n=0}^\infty$ is a complete and  orthogonal basis in $L^2(-1,1)$ \cite{Weidmann80}.

By \eqref{phi-der-varohi-der} and the fact that $d\gamma=(1-\gamma^2)dy={1\over2}g'(\psi_0)dy$, we get the Poincar\'e inequalities
% and compact embedding result
 in the new variable $\gamma$, which are direct consequence of Lemma \ref{Poincare ineqalities-new-variable0} (1) and (3).
% and \ref{compact2P}.
\begin{lemma}\label{Poincare inequalities compact embedding result  new variables}
For any $\varphi \in \hat{Y}_0$, we have
\begin{align*}
 \|\varphi\|^2_{L^2(-1,1)}   \leq C\int_{-1}^1 (1-\gamma)^2| \varphi'|^2d\gamma,\quad \|\varphi-\hat P_0\varphi\|^2_{L^2(-1,1)}   \leq C\int_{-1}^1 (1-\gamma)^2| \varphi'|^2d\gamma.
\end{align*}
\if0
$(2)$ $\hat Y_0$ is compactly embedded in $\hat Z_0 := \{ \varphi | \int_{-1}^1 |\varphi - \hat{P}_0 \varphi|^2 d \gamma < \infty \}.$
\fi
\end{lemma}
Thus, in the new variable $\gamma$, $\hat Y_0$ is embedded in $L^2(-1,1)$. Let us compare the eigenfunctions $\phi_{n}$, $1\leq n\leq 5$, in \eqref{eigen value-function} with the  Legendre polynomials
\begin{equation*}\begin{aligned}\begin{array}{llll}
&L_1(\gamma)=\gamma,
\quad L_2(\gamma)={1\over2}(3\gamma^2-1),\quad
L_3(\gamma)={1\over2}(5\gamma^3-3\gamma),\\
&L_4(\gamma)={1\over8}(35\gamma^4-30\gamma^2+3),\quad
L_5(\gamma)={1\over8}(63\gamma^5-70\gamma^3+15\gamma).
\end{array}
\end{aligned}
\end{equation*}
Then we find that up to a constant factor,
\begin{equation*}
\phi_{n}(y)=L_{n}(\tanh(y))-L_{n}(0)=L_{n}(\gamma)-L_{n}(0), \;\;1\leq n\leq 5.
\end{equation*}
This provides a hint that
 the eigenvalues  for  \eqref{eigenvalue problem for 0 mode} might be $\lambda_n={n(n+1)\over2}$, $n\geq1$, with corresponding eigenfunctions $L_{n}(\gamma)-L_{n}(0)$, which is confirmed in the next lemma.

\begin{lemma}\label{sol to eigenvalue problem}
All the eigenvalues  of the eigenvalue problem \eqref{eigenvalue problem for 0 mode} are $\lambda_n={n(n+1)\over2}$, $n\geq1$. For $n\geq1$, the eigenspace associated to $\lambda_n={n(n+1)\over2}$ is $\text{span}\{L_{n}(\gamma)-L_{n}(0)\}.$  Consequently, all the eigenvalues  of the eigenvalue problem \eqref{mode0} are $\lambda_n={n(n+1)\over2}$, $n\geq1$. For $n\geq1$, the eigenspace associated to $\lambda_n={n(n+1)\over2}$ is $\text{span}\{L_{n}(\tanh(y))-L_{n}(0)\}$.
\end{lemma}
\begin{proof}
Due to  the projection's term, we need to  check that $\varphi(\gamma)=\varphi_n(\gamma)=L_{n}(\gamma)-L_{n}(0)\in \hat Y_0$ and $\lambda=\lambda_n={n(n+1)\over2}$  solve \eqref{eigenvalue problem for 0 mode}.
Thanks to the property of Legendre polynomials that
\begin{equation*}\label{int-0}\int_{-1}^{1} L_n(\gamma) d\gamma=0\end{equation*}
for $n\geq1$ \cite{Byerly59},
we have
 $\hat P_0\varphi_n=\hat P_0 (L_n(\gamma)-L_n(0))=-L_n(0)$,  and thus,
\begin{equation*}\begin{aligned}\begin{array}{llll}
&((1-\gamma^2)\varphi_n')'+2\lambda(\varphi_n-\hat P_0\varphi_n)=(1-\gamma^2)\varphi_n''-2\gamma\varphi_n'+2\lambda(\varphi_n-\hat P_0\varphi_n)\\
=&(1-\gamma^2)(L_n(\gamma)-L_n(0))''-2\gamma(L_n(\gamma)-L_n(0))'+2\lambda((L_n(\gamma)-L_n(0))+L_n(0))\\
=&(1-\gamma^2)L_n''(\gamma)-2\gamma L_n'(\gamma)+2\lambda L_n(\gamma)=0.
\end{array}
\end{aligned}
\end{equation*}
Since  $\varphi_n(0)=0$ and $\int_{-1}^1(1-\gamma^2)|\varphi_n'(\gamma)|^2d\gamma<\infty$, we have $\varphi_n\in \hat Y_0$.
So $\varphi_n$ solves  \eqref{eigenvalue problem for 0 mode}.
%By \eqref{int-0}, we also have
%\begin{equation*}
%\int_{-1}^1\varphi_n(z)\varphi_m(z)dz=\int_{-1}^1(L_n(z)-L_n(0))(L_m(z)-L_n(0))dz
%\end{equation*}

Next, we prove that the eigenspace associated to $\lambda_n={n(n+1)\over2}$ is $\text{span}\{\varphi_n\}$, and there are no more eigenvalues for \eqref{eigenvalue problem for 0 mode}. From the variational problem, we know that it suffices to prove that $\{\varphi_n\}_{n=1}^\infty$ is a complete and orthogonal basis of  $\hat Y_0$ under the inner product
$$(\varphi_1, \varphi_2)_{ \hat Z_0} = \int_{-1}^1 (\varphi_1 - \hat{P}_0\varphi_1)(\varphi_2  - \hat{P}_0 \varphi_2)  d \gamma, \quad\forall \varphi_1, \varphi_2 \in \hat Z_0,$$
where $\hat Z_0 := \{ \varphi | \int_{-1}^1 |\varphi - \hat{P}_0 \varphi|^2 d \gamma < \infty \}$ corresponds to the space $\{ \phi | \int_{\mathbb{R}} g'(\psi_0) |\phi - {P}_0 \phi|^2 d y < \infty \}$ in the original variable $y$.

 To see this, we note that
\begin{align}\nonumber
(\varphi_n,\varphi_m)_{\hat Z_0}=&\int_{-1}^1(\varphi_n-\hat P_0 \varphi_n)(\varphi_m-\hat P_0 \varphi_m)d\gamma=\int_{-1}^1(\varphi_n+L_n(0))(\varphi_m+L_m(0))d\gamma\\\nonumber
=&\int_{-1}^1L_nL_md\gamma=\left\{\begin{array}{llll}0,\;\;\;\;\;\;\;\text{if}\;\;m\neq n,\\
{2\over2n+1},\;\;\text{if}\;\;m= n.\end{array}\right.
\end{align}
This proves the orthogonality of  $\{\varphi_n\}_{n=1}^\infty$.
For any $\varphi\in \hat Y_0$, by Lemma \ref{Poincare inequalities compact embedding result  new variables}  we have  $\varphi\in L^2(-1,1)$  and thus, $\varphi(\gamma)=\sum_{n=0}^{\infty}a_nL_n(\gamma)$, where $a_n={{2n+1\over2}}\int_{-1}^1\varphi L_n d\gamma$.
$\varphi\in \hat Y_0$ implies that $\varphi(0)=\sum_{n=0}^{\infty}a_nL_n(0)=0$. Thus, we have \begin{align*}
\varphi(\gamma)=\sum_{n=0}^{\infty}a_n(L_n(\gamma)-L_n(0))=\sum_{n=1}^{\infty}a_n\varphi_n(\gamma)
\end{align*}
 for $\gamma\in(-1,1)$, with
\begin{align*}
a_n={{2n+1\over2}}\int_{-1}^1(\varphi-\hat P_0 \varphi)(\varphi_n-\hat P_0 \varphi_n)d\gamma=(\varphi,\varphi_n)_{\hat Z_0}.
\end{align*}
 For any $\varepsilon>0$, there exists $N_\varepsilon>0$ such that
 \begin{align*}
 \left\|\varphi-\sum_{n=0}^{N_\varepsilon}a_nL_n\right\|_{L^2(-1,1)}<{\varepsilon\over4}\quad
 \text{and}\quad\left|\sum_{n=0}^{N_\varepsilon}a_nL_n(0)\right|<{\sqrt{2}\varepsilon\over8}. \end{align*}
 Then \begin{align*}
 \left\|\hat P_0\left(\varphi-\sum_{n=1}^{N_\varepsilon}a_n\varphi_n\right)\right\|_{L^2(-1,1)}=\sqrt{2}\left|\hat P_0\left(\varphi-\sum_{n=1}^{N_\varepsilon}a_n\varphi_n\right)\right|\leq  \left\| \varphi-\sum_{n=1}^{N_\varepsilon}a_n\varphi_n\right\|_{L^2(-1,1)},
 \end{align*}
 and
 \begin{align}\nonumber
 &\left\|\varphi-\sum_{n=1}^{N_\varepsilon}a_n\varphi_n\right\|_{\hat Z_0}\leq
 \left\|\varphi-\sum_{n=1}^{N_\varepsilon}a_n\varphi_n\right\|_{L^2(-1,1)}+\left\|\hat P_0\left(\varphi-\sum_{n=1}^{N_\varepsilon}a_n\varphi_n\right)\right\|_{L^2(-1,1)}\\\nonumber
 \leq &
2\left\|\varphi-\sum_{n=1}^{N_\varepsilon}a_n\varphi_n\right\|_{L^2(-1,1)}= 2\left\|\varphi-\sum_{n=0}^{N_\varepsilon}a_n(L_n-L_n(0))\right\|_{L^2(-1,1)}
 \\\nonumber
 \leq&
2\left\|\varphi-\sum_{n=0}^{N_\varepsilon}a_nL_n\right\|_{L^2(-1,1)}+2\left\|\sum_{n=0}^{N_\varepsilon}a_nL_n(0)\right\|_{L^2(-1,1)}<{\varepsilon\over2}+{\varepsilon
 \over2}=\varepsilon.
 \end{align}
This proves the completeness of  $\{\varphi_n\}_{n=1}^\infty$.
\if0
 %Then  $\varphi_\lambda(z)=\sum_{n\geq0}a_nL_n(z)$ for $z\in(-1,1)$, where $a_n={2n+1\over2}\int_{-1}^1\varphi_\lambda(z)L_n(z)dz$.
 Define
 $$J(\varphi)={\int_{-1}^1(1-z^2)|\varphi'(z)|^2 dz\over 2\int_{-1}^1(\varphi-\hat P \varphi)^2dz}$$
 for $\varphi\in Y_0$.
 By the Poincare-type inequality
 \begin{equation*}
 \int_{-1}^1(\varphi-\hat P \varphi)^2dz\leq C\int_{-1}^1(1-z^2)|\varphi'(z)|^2 dz,\;\;\forall \varphi\in Y_0
 \end{equation*}
 and the compactness of the embedding $Y_0\hookrightarrow Y_m$, we can inductively define $\mu_n$  as follows:
  \begin{equation*}\begin{aligned}\begin{array}{llll}
  \mu_1=\min\limits_{\varphi\in Y_0}J(\varphi)=J(\varphi_{\mu_1}),\\
  \mu_2=\min\limits_{\varphi\in Y_0,\;\langle\varphi,\varphi_{\mu_1}\rangle_{Y_m}=0}J(\varphi)=J(\varphi_{\mu_2}),\\
  \mu_3=\min\limits_{\varphi\in Y_0,\;\langle\varphi,\varphi_{\mu_i}\rangle_{Y_m}=0,\;i=1,2}J(\varphi)=J(\varphi_{\mu_3}),\cdots\\
  \mu_n=\min\limits_{\varphi\in Y_0,\;\langle\varphi,\varphi_{\mu_i}\rangle_{Y_m}=0,\;i=1,\cdots,n-1}J(\varphi)=J(\varphi_{\mu_n}),\cdots
  \end{array}
\end{aligned}
   \end{equation*}
for some  $\varphi_{\mu_n}\in Y_0$ with $\int_{-1}^1(\varphi_{\mu_n}(z)-\hat P\varphi_{\mu_n})^2dz=1$, $n\geq1$. Then
\begin{equation*}
{d\over d\tau}J(\varphi_{\mu_n}+\tau \phi)|_{\tau=0}=\int_{-1}^1[-((1-z^2)\varphi'_{\mu_n}(z))'-\mu_n(\varphi_{\mu_n}(z)-\hat P\varphi_{\mu_n})]\phi (z) dz
\end{equation*}
for any $\varphi\in Y_0$.   $\lambda=\mu_n$ and  $\varphi=\varphi_{\mu_n}$ gives all the nontrivial  solutions of  \eqref{eigenvalue problem2}.
Then $\{\lambda_n\}_{n=1}^\infty\subset\{\mu_n\}_{n=1}^\infty$, and
%we can choose $\{{\varphi_n\over\|\varphi_n\|_{Y_m}}\}_{n=1}^\infty\subset \{\varphi_{\mu_n}\}_{n=1}^\infty$.
and $Y_0\subset\text{span}\{\varphi_n\}_{n=1}^\infty\subset\text{span}\{\varphi_{\mu_n}\}_{n=1}^\infty\subset Y_0$.
Then $\text{span}\{\varphi_n\}_{n=1}^\infty=\text{span}\{\varphi_{\mu_n}\}_{n=1}^\infty=Y_0$.
Suppose there exists $\lambda_*\in\{\mu_n\}_{n=1}^\infty\setminus\{\lambda_n\}_{n=1}^\infty$ with $\lambda_*=J(\varphi_*)$, $\int_{-1}^1(\varphi_{*}(z)-\hat P\varphi_{*})^2dz=1$. Then
$\varphi_*=\sum_{n\geq1}b_{n}\varphi_n$ for some  $b_n\in\mathbf{R}$, $n\geq1$. But $\langle\varphi_*,\varphi_n\rangle_{Y_m}=0$, $n\geq1$, gives $\varphi_*\equiv0$, which is a contradiction. Suppose there exists an eigenfunction $\phi_*\in Y_0$ of  $\lambda_{n_0}$ for some $n_0\geq1$ such that $\phi_*\neq c\varphi_{n_0}$ for any $c\neq0$. Then  there exists an eigenfunction $0\neq\phi_{n_0}\in\text{span}\{\phi_*,\varphi_{n_0}\}$ of $\lambda_{n_0}$ such that $\langle\phi_{n_0},\varphi_{n_0}\rangle_{Y_m}=0$, and $J(\phi_{n_0})=\lambda_{n_0}$. Note that  $\phi_{n_0}=\sum_{n\geq1}d_{n}\varphi_n$ for some  $d_n\in\mathbf{R}$, $n\geq1$, and
 $\langle\phi_{n_0},\varphi_{n}\rangle_{Y_m}=0$ for any $1\leq n\leq n_0$.
 Thus,  $\phi_{n_0}=\sum_{n\geq n_0+1}d_{n}\varphi_n\in\overline{\text{span}\{\varphi_n\}_{n=n_0+1}^\infty}$, which implies that
 $J(\phi_{n_0})\geq\min\limits_{\varphi\in \overline{\text{span}\{\varphi_n\}_{n=n_0+1}^\infty}}J(\varphi)$.
 Since $\{\lambda_n\}_{n=1}^\infty\subset\{\mu_n\}_{n=1}^\infty$, there exists $N_0>n_0$ such that $\lambda_{n_0+1}=\mu_{N_0}$ and $\overline{\text{span}\{\varphi_n\}_{n=n_0+1}^\infty}\subset\overline{\text{span}\{\varphi_{\mu_n}\}_{n=N_0}^\infty}$. Then $J(\phi_{n_0})\geq\min\limits_{\varphi\in \overline{\text{span}\{\varphi_{\mu_n}\}_{n=N_0}^\infty}}J(\varphi)=\mu_{N_0}=\lambda_{n_0+1}$.  This is a contradiction.
 \fi
\end{proof}


\noindent{\bf{Eigenvalue problem for the non-zero mode}}\medskip

For the $k$ mode with $k\neq0$, we solve the eigenvalue problem
\eqref{modek}.
It suffices to consider $k\geq1$. We use the change of variable  \eqref{change of variable for 0 mode}
 and denote $\phi(y)=\varphi(\gamma)$.
%Then $z\in(-1,1)$, and
%\begin{equation*}\begin{aligned}\begin{array}{llll}
%u'(y)=(1-z^2)\phi'(z),\;\;u''(y)=(1-z^2)(-2z\phi'(z)+(1-z^2)\phi''(z)),\;\;g'(\psi_0)=2(1-z^2).
%\end{array}
%\end{aligned}
%\end{equation*}
Since $$\|\phi\|_{H^1(\mathbb{R})}^2=\int_{-1}^1\left({1\over1-\gamma^2}|\varphi(\gamma)|^2+(1-\gamma^2)|\varphi'(\gamma)|^2\right)d\gamma,$$
the space $Y_1=H^1(\mathbb{R})$ for $\phi$ in the variable $y$ is transformed to
\begin{align}\label{def-hat-Y1}
\hat Y_1=\left\{\varphi\bigg|\int_{-1}^1\left({1\over1-\gamma^2}|\varphi(\gamma)|^2+(1-\gamma^2)|\varphi'(\gamma)|^2\right)d\gamma<\infty \right\}
\end{align}
for $\varphi$ in the new variable $\gamma$. Then the eigenvalue problem \eqref{modek}  is equivalent to the general Legendre equation
\begin{equation}\label{eigenvalue problem2 non-zero modes varepsilon=0}
-((1-\gamma^2)\varphi')'+{k^2\over1-\gamma^2}\varphi =2\lambda\varphi \quad \text{on}\quad (-1,1),\quad\varphi\in \hat Y_1.
\end{equation}

\if0
This so-called   general Legendre equation can be obtained by first differentiating the classical Legendre equation \eqref{eigenvalue problem3} $k$ times, and then transforming  the unknown function $\varphi(z)$ to $\phi(z)=(1-z^2)^{k\over2}{d^k\over dz^k}\varphi(z)$.
\fi

 The Poincar\'e inequality in Lemma \ref{Poincare ineqalities-new-variable0} (3) reads as follows.
% and compact embedding result
% and \ref{compact2P}.
\begin{lemma}\label{Poincare inequalities compact embedding result new variables k mode}
For any $\varphi \in \hat{Y}_1$, we have
\begin{align*}
 \|\varphi\|^2_{L^2(-1,1)}   \leq C\int_{-1}^1\left({1\over1-\gamma^2}|\varphi(\gamma)|^2+(1-\gamma^2)|\varphi'(\gamma)|^2\right)d\gamma.
\end{align*}
\end{lemma}
Then we give all the eigenvalues of \eqref{eigenvalue problem2 non-zero modes varepsilon=0} with corresponding eigenfunctions.
\begin{lemma}\label{sol to eigenvalue problem non-zero modes varepsilon=0 original} Fix $k\geq1.$ Then
all the eigenvalues  of the eigenvalue problem \eqref{eigenvalue problem2 non-zero modes varepsilon=0} are $\lambda_n={n(n+1)\over2}$, $n\geq k$. For $n\geq k$, the eigenspace associated to $\lambda_n={n(n+1)\over2}$ is $\text{span}\{L_{n,k}(\gamma)\}$.
Consequently, all the eigenvalues  of the eigenvalue problem \eqref{modek} are $\lambda_n={n(n+1)\over2}$, $n\geq k$. For $n\geq k$, the eigenspace associated to $\lambda_n={n(n+1)\over2}$ is $\text{span}\{L_{n,k}(\tanh(y))\}$.
\end{lemma}
\begin{proof}
It is well-known in \cite{Courant-Hilbert53} that for $n\geq k$ and $\lambda_n={n(n+1)\over2}$, the associated Legendre polynomials of $k$-th order  \begin{equation}\nonumber
L_{n,k}(\gamma)=(1-\gamma^2)^{k\over2}{d^k\over d\gamma^k}L_n(\gamma)
\end{equation}
are   solutions of the equation in \eqref{eigenvalue problem2 non-zero modes varepsilon=0}.
$k\geq1$ implies
\begin{equation*}\begin{aligned}
&\int_{-1}^1{1\over 1-\gamma^2}|L_{n,k}(\gamma)|^2d\gamma=\int_{-1}^1(1-\gamma^2)^{k-1}\left|{d^k\over d\gamma^k}L_n(\gamma)\right|^2d\gamma<\infty,\\
&\int_{-1}^1{ (1-\gamma^2)}|L'_{n,k}(\gamma)|^2d\gamma=\int_{-1}^1(1-\gamma^2)^{k-1}\left|-k\gamma{d^k\over d\gamma^k}L_n(\gamma)+(1-\gamma^2){d^{k+1}\over d\gamma^{k+1}}L_n(\gamma)\right|^2d\gamma<\infty,
\end{aligned}\end{equation*}
and thus, $L_{n,k}\in \hat Y_1$. Thus, $\lambda_n={n(n+1)\over2}$ is an eigenvalue of \eqref{eigenvalue problem2 non-zero modes varepsilon=0} with corresponding eigenfunction  $L_{n,k}(\gamma)$, where $n\geq k$. It suffices to show that $\{L_{n,k}\}_{n=k}^\infty$ is a complete and orthogonal basis of $\hat Y_1$ under the inner product of  $L^2(-1,1)$.
In fact,  $\{L_{n,k}\}_{n=k}^\infty$ is a complete and orthogonal basis of $L^2(-1,1)$  \cite{Courant-Hilbert53,Dominguez-Heuer-Sayas11}. The conclusion follows from the embedding $\hat Y_1\hookrightarrow L^2(-1,1)$ by Lemma \ref{Poincare inequalities compact embedding result new variables k mode}.
\end{proof}





In summary, under the new coordinate $(x,\gamma=\tanh(y))\in\mathbb{T}_{2\pi}\times (-1,1)$,  the associated eigenvalue problem \eqref{elip02} is transformed to
\begin{align}\label{elip02-x-gamma}
-{1\over1-\gamma^2}\pa_x^2\Psi-\pa_\gamma\left((1-\gamma^2)\pa_\gamma\Psi\right)=2\lambda(\Psi-\tilde P_0\Psi),\quad \Psi\in\tilde Y_0,
\end{align}
where $\Psi(x, \gamma) = \psi(x, y)$, $\tilde P_0$ is defined in \eqref{def-tilde-P0-Psi} and $\tilde{Y}_0$ is given in \eqref{tilde-Y0-def}.
\if0
\begin{align*}
\tilde P_0\Psi =& \frac{1}{4\pi}\int_{-1}^1 \int_{0}^{2\pi} \Psi(x, \gamma) dx d \gamma,\\
\tilde{Y}_0 =& \left\{ \Psi \bigg| \int_{-1}^1\int_{0}^{2\pi}{1\over1-\gamma^2}|\Psi_{x}|^2+(1-\gamma^2)|\Psi_{\gamma}|^2dx d\gamma< \infty \text{ and }\int_0^{2\pi} \Psi(x, 0)dx= 0 \right\}.
\end{align*}
\fi

Combining the conclusions for  the $0$ mode in Lemma \ref{sol to eigenvalue problem} and for the non-zero modes in Lemma \ref{sol to eigenvalue problem non-zero modes varepsilon=0 original}, we solve the eigenvalue problems \eqref{elip02-x-gamma} and \eqref{elip02}.
\begin{Theorem}\label{associate_ep0}
All the eigenvalues of the eigenvalue problem \eqref{elip02-x-gamma} are $ \lambda_n  = \frac{n(n+1)}{2}$, $n\geq1$. For $n\geq1$, the eigenspace associated to $\lambda_n$ is  spanned by
\begin{align*}
 L_{n}(\gamma) - L_n(0), \quad  L_{n,k}(\gamma)\cos(kx), \quad L_{n,k}(\gamma)\sin(kx), \quad  1 \leq k\leq n.
 \end{align*}
Consequently, all the eigenvalues of the associated eigenvalue problem \eqref{elip02} are $\lambda_n  = \frac{n(n+1)}{2}$, $n\geq1$. For $n\geq1$, the eigenspace associated to $\lambda_n$ is  spanned by
\begin{align}\label{sol-elip02}
 L_{n}(\tanh(y)) - L_n(0), \quad  L_{n,k}(\tanh(y))\cos(kx), \quad L_{n,k}(\tanh(y))\sin(kx), \quad  1 \leq k\leq n.
 \end{align}
\end{Theorem}
In particular, we obtain the kernel of  the operator $\tilde A_0$ and a decomposition of $\tilde X_{0}$ as follows.


\begin{Corollary}\label{kernel of  the operator tilde A0 and a decomposition of tilde X0}
$(1)$ $\ker (\tilde A_0)={\rm{span}}\left\{\tanh(y),{\cos(x)\over \cosh(y)},{\sin(x)\over \cosh(y)}\right\}$.

$(2)$ Let $\tilde X_{0+}=\tilde X_0 \ominus\ker (\tilde A_0)$. Then
\begin{align*}
\langle \tilde A_0 \psi,\psi\rangle \geq {2\over3} \| \psi\|_{\tilde X_0}^2, \quad \quad \psi\in \tilde X_{0+}.
\end{align*}
\end{Corollary}

\begin{proof}
By Theorem \ref{associate_ep0}, we infer that $\lambda_1=1$ is the principal eigenvalue of \eqref{elip02} with multiplicity $3$, and the corresponding eigenfunctions are $\tanh(y),{\cos(x)\over \cosh(y)},{\sin(x)\over \cosh(y)}$. This proves (1).

For $\psi\in\tilde X_0$ and $\phi\in\ker (\tilde A_0)$, we note that $(\psi,\phi)_{Z_0}=\iint_{\Omega}g'(\psi_0)(\psi-P_0\psi)\phi dxdy=\iint_{\Omega}g'(\psi_0)\psi\phi dxdy=\iint_{\Omega}\psi(-\Delta)\phi dxdy=(\psi,\phi)_{\tilde X_0}$, where we used $P_0\phi=0$.
Since $\lambda_2=3$ is the second eigenvalue of \eqref{elip02}, we get by the variational problem \eqref{variational problem2} that
\begin{align*}
{1\over3}\iint_{\Omega}|\nabla\psi|^2dxdy\geq\iint_{\Omega}g'(\psi_0)(\psi-P_0\psi)^2dxdy,\quad \psi\in\tilde X_{0+},
\end{align*}
and thus, by \eqref{A0-quadratic form} we have
\begin{align*}
 \langle\tilde{A}_0\psi,\psi\rangle=\iint_\Omega|\nabla\psi|^2-g'(\psi_0)(\psi - P_0\psi)^2dxdy\geq {2\over3} \| \psi\|_{\tilde X_0}^2.
\end{align*}
This proves (2).
\end{proof}
We also get  the kernel of  the operator $A_0$ defined in \eqref{A0 without projection} and a decomposition of $\tilde X_{0}$ associated to $A_0$, which plays important roles in the study on nonlinear stability.
\begin{Corollary}\label{kernel of  the operator A0 and a decomposition of tilde X0}
$(1)$ $\ker ( A_0)=\ker (\tilde A_0)={\rm{span}}\left\{\tanh(y),{\cos(x)\over \cosh(y)},{\sin(x)\over \cosh(y)}\right\}$.

$(2)$ Let $\tilde X_{0+}$ be defined as above. Then
\begin{align*}
\langle  A_0 \psi,\psi\rangle \geq C_0 \| \psi\|_{\tilde X_0}^2, \quad \quad \psi\in \tilde X_{0+}
\end{align*}
for some $C_0>0$.
\end{Corollary}
\begin{proof} (1)
Since $P_0|_{\ker ( A_0)}=0$, we have by \eqref{tilde A0-A0} that $\ker (\tilde A_0)\subset \ker ( A_0)$.
For $\psi=\widehat\psi_0+\psi_{\neq0}\in \ker ( A_0)\backslash\ker (\tilde A_0)$, we have $\psi=\widehat\psi_0$ since $\tilde A_0\psi_{\neq0}=A_0\psi_{\neq0}=0$.
Then $\langle A_0 \widehat\psi_0,\phi\rangle=2\pi\int_\mathbb{R}\left(\widehat\psi_0'\phi'-g'(\psi_0)\widehat\psi_0\phi \right)dy=0$ for $\phi\in Y_0=\{\phi|\phi\in\dot{H}^1(\mathbb{R}),\phi(0)=0\}$. Thus,
  $-\widehat\psi_0''-g'(\psi_0)\widehat\psi_0=a_0\delta(y)$ for some  $a_0\in \mathbb{R}$. Thus,  $-\widehat\psi_0''-g'(\psi_0)\widehat\psi_0=0$ for $y\neq0$. Then $\widehat\psi_0(y)=c_1\tanh(y)+c_2(y\tanh(y)-1)$ for $y\neq0$. Since $y\tanh(y)-1\notin \dot{H}^1(\mathbb{R})$, we have $\widehat\psi_0(y)=c_1\tanh(y)$. Thus, $\ker (\tilde A_0)=\ker ( A_0)$.


(2) First, we claim that $\langle A_0\phi,\phi\rangle\geq0$ for $\phi\in Y_0$.
In fact, since $(\sech^2(y))'=-2\sech^2(y)$ $\tanh(y)$, we have
\begin{align*}
\langle A_0\phi,\phi\rangle=&2\pi\int_{-\infty}^\infty\left(|\phi'(y)|^2+{(\sech^2(y))'\over\tanh(y)}\phi(y)^2\right)dy\\
=&2\pi\int_{-\infty}^\infty|\phi'(y)|^2dy+2\pi{\sech^2(y)\phi(y)^2\over\tanh(y)}\bigg|_{-\infty}^\infty\\
&-
2\pi\int_{-\infty}^\infty\left({2\phi(y)\phi'(y)\sech^2(y)\over\tanh(y)}
-{\phi(y)^2\sech^4(y)\over\tanh^2(y)}\right)dy\\
=&2\pi\int_{-\infty}^\infty\left(\phi'(y)
-{\phi(y)\sech^2(y)\over\tanh(y)}\right)^2dy\geq0,
\end{align*}
where we used $\phi(y)^2\leq\|\phi'\|_{L^2(\mathbb{R})}^2|y|$, $\phi(y)=\tanh(y)\sum_{k\geq0}P_k(\tanh(y))$, and $P_k(\tanh(y))$ is a $k$-order polynomial of $\tanh(y)$.

Let $\psi=\widehat\psi_0+\psi_{\neq0}\in \tilde X_0$. Then
$\langle A_0 \psi_{\neq0},\psi_{\neq0}\rangle =
\langle \tilde A_0 \psi_{\neq0},\psi_{\neq0}\rangle \geq 0$ by Theorem \ref{associate_ep0}.
Thus, $
\langle A_0\psi,\psi\rangle= \langle A_0\widehat\psi_0,\widehat\psi_0\rangle+\langle A_0\psi_{\neq0},\psi_{\neq0}\rangle\geq0$.
Since  $\tilde X_0$ is compactly embedded in $L^2_{g'(\psi_0)}(\Omega)$ by Lemma \ref{compact2P}, we have
\begin{align*}\inf_{\psi \in \tilde X_0, (\psi, \phi)_{L^2_{g'(\psi_0)}(\Omega)} = 0, \phi\in\ker(A_0)}{\iint_\Omega|\nabla\psi|^2dxdy\over\iint_\Omega g'(\psi_0)\psi^2dxdy}=\mu_0>1,
\end{align*}
which implies that
\begin{align*}
 \langle A_0\psi,\psi\rangle=\iint_\Omega|\nabla\psi|^2-g'(\psi_0)\psi^2dxdy\geq \left(1-{1\over\mu_0}\right) \| \psi\|_{\tilde X_0}^2,\quad \psi\in\tilde X_{0+},
\end{align*}
where we used  $(\psi,\phi)_{L^2_{g'(\psi_0)}(\Omega)}=\iint_{\Omega}g'(\psi_0)\psi\phi dxdy=\iint_{\Omega}\nabla\psi\cdot\nabla\phi dxdy=(\psi,\phi)_{\tilde X_0}$ for $\phi\in\ker (\tilde A_0)$.
\if0
 In fact, by \eqref{p0-psi-estimates-2} and \eqref{tilde A0-A0} we have
\begin{align*}
\langle\tilde  A_0 \psi,\psi\rangle=\langle A_0 \psi,\psi\rangle+8\pi(P_0 \psi)^2\leq\langle A_0 \psi,\psi\rangle+\iint_\Omega g'(\psi_0)|\psi|^2dxdy,\quad \psi\in\tilde X_{0+}.
\end{align*}
Thus, we infer from \eqref{sec-eigenvalue1-tilde-A0}-\eqref{sec-eigenvalue2} that
\begin{align*}
\langle A_0 \psi,\psi\rangle\geq
\langle\tilde  A_0 \psi,\psi\rangle-\iint_\Omega g'(\psi_0)|\psi|^2dxdy\geq {2\over3} \| \psi\|_{\tilde X_0}^2-{1\over3} \| \psi\|_{\tilde X_0}^2={1\over3} \| \psi\|_{\tilde X_0}^2.
\end{align*}
\fi
\end{proof}

\if0
\begin{proof}



We know that $\{ (\lambda_n, \phi_n^0) \}_{n \geq 1}$ in \eqref{eigen00} solves \eqref{eigenvalue problem for 0 mode}. Now we prove that the eigenspace for each eigenvalue $\lambda_n$ is $span \{\phi_n^0\}$, and there are no more other eigenvalues for \eqref{eigenvalue problem for 0 mode}. We first claim that $\{\phi_n^0\}_{n=1}^\infty$ is an orthogonal basis in $Z_0$ under the inner product of $Z_m$ where
$$Z_m = \{ \phi | \int_{-1}^1 |\phi (\gamma) - \tilde{P}_0 \phi |^2 d \gamma < \infty \}$$
with inner product
$$(\phi_1, \phi_2)_{Z_m} = \int_{-1}^1 (\phi_1 (\gamma) - \tilde{P}_0 \phi_1)(\psi_2 (\gamma) - \tilde{P}_0 \phi_2)  d \gamma \text{ for any } \phi_1, \phi_2 \in Z_m.$$
We have the orthogonality of $\{\phi_n^0\}_{n=1}^\infty$ from
\begin{align*}
(\phi_i^0, \phi_j^0)_{Z_m}
& = \int_{-1}^1 (\phi_i^0 (\gamma) - \tilde{P}_0 \phi_i^0)(\phi_j^0 (\gamma) - \tilde{P}_0 \phi_j^0)d \gamma \\
& = \int_{-1}^1 (\phi_i^0 (\gamma) + L_i(0))(\phi_j^0 (\gamma) + L_j(0))d \gamma \\
& = \int_{-1}^1 L_i(\gamma)L_j(\gamma)d \gamma \\
& = \frac{2}{2i+1}\delta_{ij}
\end{align*}
where $\delta_{ij} = 1$ if $i = j$ else $0$. Now we prove that $\{\phi_n^0\}_{n=1}^\infty$ spans $Z_0$. Since \{$L_n\}_{n=0}^\infty$ is an orthogonal and complete basis of $L^2(-1,1)$. For any $\phi \in Z_0$, we have
$$\phi(\gamma) = \sum_{n=0}^\infty a_n L_n(\gamma)$$ for any fixed $\gamma \in (-1, 1)$, where $a_n = \frac{2n+1}{2}\int_{-1}^1 \phi(\gamma) L_n(\gamma) d \gamma$. Since $\phi \in Z_0$, we have $\phi(0) = \sum_{n=0}^\infty a_n L_n(0) = 0$. So, we have
$$\phi(\gamma) = \sum_{n=0}^\infty a_n (L_n(\gamma) - L_n(0)) =  \sum_{n=1}^\infty a_n \phi_n^0(\gamma) \text{ for } \gamma \in (-1, 1)$$
and
$$a_n = \frac{2n+1}{2}(\phi, \phi_n^0)_{Z_m}.$$ Moreover, for any $\ep > 0$, there exists $N_\ep > 0$ such that
$$\|\phi(\gamma) - \sum_{n=0}^{N_\ep} a_n L_n(\gamma) \|_{L^2(-1,1)} < \frac \ep 2 \text{ and } |\sum_{n=0}^{N_\ep} a_n L_n(0)| < \frac{\ep}{2\sqrt{2}}.$$
So
\begin{align*}
&\|\phi - \sum_{n=1}^{N_\ep} a_n \phi^0_n \|_{L^2(-1,1)} \\
= & \|\phi - \sum_{n=0}^{N_\ep} a_n (L_n - L_n(0))\|_{L^2(-1,1)} \\
\leq & \|\phi - \sum_{n=0}^{N_\ep} a_n L_n \|_{L^2(-1,1)}  + \|\sum_{n=0}^{N_\ep} a_n L_n(0)\|_{L^2(-1,1)} \\
< & \frac{\ep}{2} + \frac{\ep}{2} = \ep,
\end{align*}
and
\begin{align*}
& \|\phi - \sum_{n=1}^{N_\ep} a_n \phi^0_n \|_{Z_m}\\
 \leq & \|\phi - \sum_{n=1}^{N_\ep} a_n \phi^0_n \|_{L^2(-1,1)} + \| \tilde{P}_0( \phi - \sum_{n=1}^{N_\ep} a_n \phi^0_n) \|_{L^2(-1,1)} \\
 < & \ep + \sqrt{2} |\tilde{P}_0( \phi - \sum_{n=1}^{N_\ep} a_n \phi^0_n)| \\
 < & \ep + \frac{\sqrt{2} }{2} \|\phi - \sum_{n=1}^{N_\ep} a_n \phi^0_n\|_{L^1(-1,1)} \\
 < & \ep + \ep = 2\ep.
\end{align*}
So we proved that $Z_0 \subseteq span \{\phi_n^0\}_{n=1}^\infty$. Now we are going to prove that $span \{\phi_n^0\}_{n=1}^\infty \subseteq Z_0$. Define
$$J(\phi) = \frac{\int_{-1}^1(1-\gamma^2)|\phi'(\gamma)|^2 d\gamma }{2\int_{-1}^1 (\phi - \tilde{P}\phi)^2 d \gamma}$$
for $\phi \in Z_0$. By the Poincar\'e-type inequality from Lemma \ref{poincare1}
$$\int_{-1}^1 (\phi - \tilde{P}\phi)^2 d \gamma \leq C \int_{-1}^1(1-\gamma^2)|\phi'(\gamma)|^2 d\gamma$$
and the compactness of the embedding $Z_0 \hookrightarrow Z_m$ (Lemma \ref{compact2}), we can inductively define $\mu_n$ as follows:
\begin{align*}
\mu_1 &= \min_{\phi \in Z_0} J(\phi) = J(\phi_{\mu_1}), \\
\mu_2 &= \min_{\phi \in Z_0, (\phi, \phi_{\mu_1})_{Z_m} = 0} J(\phi) = J(\phi_{\mu_2}),\\
\mu_3 &= \min_{\phi \in Z_0, (\phi, \phi_{\mu_i})_{Z_m} = 0, i = 1, 2} J(\phi) = J(\phi_{\mu_3}),\\
\cdots\\
\mu_n &= \min_{\phi \in Z_0, (\phi, \phi_{\mu_i})_{Z_m} = 0, i = 1, 2, \cdots, n-1} J(\phi) = J(\phi_{\mu_n}),\\
\cdots
\end{align*}
for $\phi_{\mu_n} \in Z_0$ such that $\int_{-1}^1 (\phi_{\mu_n} - \tilde{P}\phi_{\mu_n})^2 d \gamma = 1$, $n \geq 1$. Then we have
$$\frac{d}{d \tau} J(\phi_{\mu_n} + \tau \phi)|_{\tau = 0} = \int_{-1}^1 (1-\gamma^2)\phi'_{\mu_n}\phi' - \mu_n (\phi_{\mu_n} - \tilde{P}\phi_{\mu_n})\phi d \gamma$$
for any $\phi \in Z_0$. So $\{\lambda = \mu_n, \phi^0 = \phi_{\mu_n}\}_{n=1}^\infty$ gives all the nontrivial weak solutions of \eqref{eigenvalue problem for 0 mode}. Then we have $\{\lambda_n\}_{n=1}^\infty \subseteq \{\mu_n\}_{n=1}^\infty$, and
$$Z_0 \subseteq span\{\phi^0_n\}_{n=1}^\infty \subseteq span\{\phi_{\mu_n}\}_{n=1}^\infty \subseteq Z_0.$$
So we have $$span\{\phi^0_n\}_{n=1}^\infty = span\{\phi_{\mu_n}\}_{n=1}^\infty = Z_0.$$
Suppose there exists $\lambda_* \in \{\mu_n\}_{n=1}^\infty \backslash \{\lambda_n\}_{n=1}^\infty$ with $\lambda_* = J(\phi_*)$ and $\int_{-1}^1 (\phi_*- \tilde{P}\phi_*)^2 d \gamma = 1$. Then $\phi_* = \sum_{n=1}^\infty b_n \phi^0_n$ with $b_n \in \mathbb{R}$. But $(\phi_*, \phi^0_n)_{Z_m} = 0$ for all $n \geq 1$, which gives that $\phi_* = 0$, contradicting to the the fact that $\int_{-1}^1 (\phi_*- \tilde{P}\phi_*)^2 d \gamma = 1$. So we have $\lambda_* \in \{\mu_n\}_{n=1}^\infty = \{\lambda_n\}_{n=1}^\infty$. Now we prove that the eigenspace of $\lambda_n$ is $span \{\phi^0_n\}$ for all $n\geq 1$. Suppose there exists $n_0 \geq 1$ such that the eigenspace of $\lambda_{n_0}$ is not exactly $span \{\phi^0_{n_0}\}$, i.e. there exists another eigenfunction $\phi_* \in Z_0$ of $\lambda_{n_0}$ such that $\phi_* \neq c \phi^0_{n_0}$ for any $c \neq 0$. Then there exists an eigenfunction $0\neq \phi_{n_0} \in span \{\phi^0_{n_0}, \phi_*\}$ of $\lambda_{n_0}$ such that $(\phi_{n_0}, \phi^0_{n_0})_{Z_m} = 0$ and $J(\phi_{n_0}) = \lambda_{n_0}$. Note that $\phi_{n_0} = \sum_{n=1}^\infty d_n \phi_n^0$ with $d_n \in \mathbb{R}$ and $(\phi_{n_0}, \phi^0_n)_{Z_m} = 0$ for all $1\leq n \leq n_0$. Thus $\phi_{n_0} = \sum_{n=n_0 + 1}^\infty d_n \phi_n^0 \in span \{\phi^0_n\}_{n=n_0 + 1}^\infty$, which implies that
$J(\phi_{n_0}) \geq \min_{\phi \in span \{\phi^0_n\}_{n=n_0 + 1}^\infty} J(\phi) = \lambda_{n_0 + 1}$. This is a contradiction. So $\{ (\lambda_n, \phi_n^0) \}_{n \geq 1}$ in \eqref{eigen00} solves the eigenvalue problem \eqref{eigenvalue problem for 0 mode}. Moreover, $\{\phi_n^0) \}_{n \geq 1}$ is an orthogonal basis of $Z_0$ under the inner product of $Z_m$ and each $\phi_n^0$ spans the eigenspace of $\lambda_n$.

For the case of non-zero modes, it's easy to see that $\{L_{n,k}(\gamma))\}_{n=k}^{\infty}$ is an orthogonal basis of $Z_k$ in \eqref{spacekk} under the inner product of $L^2(-1,1)$. Define
$$J_k(\phi) = \frac{\int_{-1}^1(1-\gamma^2)|\phi'(\gamma)|^2 + \frac{k^2}{1-\gamma^2}|\phi(\gamma)|^2 d\gamma }{2\int_{-1}^1 \phi (\gamma)|^2 d \gamma}$$
for $\phi \in Z_k$. By the Poincar\'e-type inequality
$$\int_{-1}^1 |\phi(\gamma)|^2 d \gamma \leq C \int_{-1}^1(1-\gamma^2)|\phi'(\gamma)|^2 + \frac{k^2}{1-\gamma^2} |\phi(\gamma)|^2 d\gamma, \forall \phi \in Z_k$$
and the compactness of the embedding $Z_k \hookrightarrow L^2(-1,1)$ (Lemma \ref{compact1}), a similar argument as above gives that  the eigenspace for $\lambda_n$ is $span \{L_{n,k}(\gamma)\}$ and that there are no other eigenvalues for \eqref{modekk}.
\end{proof}
\fi

\begin{remark} If we neglect the projection term $-\lambda g'(\psi_0)P_0\psi$ in \eqref{elip02}, the equation becomes
 \begin{align}\label{elip02-without-projection}
-\Delta \psi = \lambda g'(\psi_0)\psi.
\end{align}
By changing the variable $y$ to $\gamma=\tanh(y)$ and denoting $\psi(x,y)=\Psi(x,\gamma)$, we have
\begin{align*}
-{1\over1-\gamma^2}\pa_x^2\Psi-\pa_\gamma\left((1-\gamma^2)\pa_\gamma\Psi\right)=2\lambda\Psi.
\end{align*}
Furthermore, by changing the variable $\gamma$ to $\beta=\cos^{-1}(\gamma)$, $\beta\in(0,\pi)$, and denoting $\Psi(x,\gamma)=\hat \Psi(x,\beta)$, we have
\begin{align}\label{spherical Laplacian}
-\Delta^*\hat\Psi=-{1\over\sin^2(\beta)}\pa_x^2\hat \Psi-{1\over \sin(\beta)}\pa_\beta\left(\sin(\beta)\pa_\beta\hat\Psi\right)=2\lambda\hat \Psi,
\end{align}
where $\Delta^*$ is the spherical Laplacian. It is well-known \cite{Courant-Hilbert53} that
 if $\hat\Psi\in L^2(S^2)$, and the boundary terms $\hat\Psi(\cdot,0)$ and $\hat\Psi(\cdot,\pi)$ are regular, then all the eigenvalues  of \eqref{spherical Laplacian} are $\lambda=\frac{n(n+1)}{2}$ with $n\geq0$.
For $n\geq0$, the eigenspace associated to $\lambda_n$ is  spanned by
$$ L_{n}(\cos(\beta)) , \quad  L_{n,k}(\cos(\beta))\cos(kx), \quad L_{n,k}(\cos(\beta))\sin(kx), \quad  0 \leq k\leq n,$$
which are exactly the spherical harmonic functions of degree $n$ and order $k$. Moreover, the  spherical harmonic functions
 form a complete and orthonormal basis of $L^2(S^2)$.
 Correspondingly, we find a series of solutions to \eqref{elip02-without-projection}
\begin{align*} L_{n}(\tanh(y)) , \quad  L_{n,k}(\tanh(y))\cos(kx), \quad L_{n,k}(\tanh(y))\sin(kx), \quad  0 \leq k\leq n,
\end{align*}
with $\lambda=\lambda_n =\frac{n(n+1)}{2}$, where $n\geq0$ is an integer.
 The difference between  \eqref{elip02-without-projection}  and our case  \eqref{elip02} is that we need to deal with the projection occurring in the equation \eqref{elip02}  as well as  the function spaces.
 The change of variables $\gamma=\tanh(y)$ and $\beta=\cos^{-1}(\gamma)$
% plays an important role in our spectral analysis to \eqref{elip02}, which
 is interesting independently.
\end{remark}


\subsection{Change of variables for  Kelvin-Stuart vortices  and reduction to the shear case}\label{Kelvin-stuart cats eyes flows}
%\subsubsection{Change of Variables}
Unlike the hyperbolic tangent shear flow ($\ep=0$), the Kelvin-stuart vortex  $\omega_\ep$ ($0 < \ep < 1$) depends on both $x$ and $y$ which are non-separable anymore.  In the original variables $(x,y)$, this makes it impossible to decompose the associated eigenvalue problem arising from the variational problem
 into a series of 1-dimensional eigenvalue problems  like what we did from \eqref{elip02} to \eqref{mode0}-\eqref{modek} for the shear case.
Fortunately, we  find a perfect change of variables, through which we can reduce the non-shear case  $0 < \ep < 1$ into the shear case  $\ep = 0$.

\subsubsection{Change of variables}\label{change of variables for cat's eyes  flows}
The main difficulty for  the Kelvin-stuart vortex  $\omega_\ep$ ($0 < \ep < 1$)   is to understand
the associated eigenvalue problem
\begin{align}\label{eigen-p-cat-eyes}
-\Delta \psi = \lambda g'(\psi_\epsilon)(I - P_\epsilon)\psi
\end{align}
in a suitable function space $\tilde X_\ep$ (see \eqref{tilde-X-e}).
\if0
$\dot{H}^1(\Omega)$ with an additional suitable condition to remove the disturbing of constants,
\fi
Here, $g'(\psi_\epsilon)$ is defined in \eqref{def-g-psi-ep-derivative} and $P_\epsilon$ (see  \eqref{P-ep}) is a similar projection as $P_0$.
The change of variable $\gamma=\tanh(y)$ for the shear case does not work here since  $g'(\psi_\epsilon)$ involves the variable $x$ deeply. In the shear case ($\ep=0$), recall that the birth of the transformation  $\gamma=\tanh(y)$ is motivated by explicitly finding some eigenvalues and corresponding eigenfunctions  in \eqref{eigen value-function} for  the eigenvalue problem \eqref{mode0}. So in the non-shear case ($0<\ep<1$), we again pay our attention to getting some explicit solutions  to \eqref{eigen-p-cat-eyes}, from which we may refine an applicable change of variables.
By taking derivative of  $-\Delta \psi_\epsilon = g(\psi_\epsilon)$, we see that    $\lambda=1$ is an eigenvalue   of $
-\Delta \psi = \lambda g'(\psi_\epsilon)\psi, \psi\in\dot{H}^1(\Omega)
$  with   eigenfunctions  $\partial_x \psi_\ep, \partial_y \psi_\ep$ and $\partial_\epsilon \psi_\ep$  for all $0<\epsilon<1$. The eigenfunctions  could be viewed  as bifurcation from  the three eigenfunctions of the eigenvalue $\lambda=1$ for the corresponding equation
$-\Delta\psi=\lambda g'(\psi_0)\psi, \psi\in\dot{H}^1(\Omega)$  (i.e. $\ep=0$) as follows:
\begin{align} \label{bifurcation1}\begin{array}{llll}
\ep=0&&0<\ep<1\\
{\sin(x)\over \cosh(y)}&\longrightarrow&\frac{\sin(x)}{\cosh(y)+\epsilon \cos(x)}=-{1\over \ep}{\partial \psi_\ep\over\partial x},\\
 \tanh(y)&\longrightarrow& \frac{\sinh(y)}{\cosh(y)+\epsilon \cos(x)}=\frac{ \partial \psi_\ep}{\partial y}, \\
 {\cos(x)\over \cosh(y)}&\longrightarrow&  \frac{\epsilon \cosh(y) + \cos(x)}{\cosh(y)+\epsilon \cos(x)}=(1 - \epsilon^2) \frac{ \partial \psi_\ep}{\partial \ep}.
 \end{array}
\end{align}
This gives a hint that $\cosh(y)$ for $\ep=0$ branches to $\cosh(y)+\epsilon \cos(x)$ for $0<\ep<1$, and $\cos(x)$ branches to $\epsilon \cosh(y) + \cos(x)$.
 Motivated by this observation, we find that $\lambda=3$ is also an eigenvalue of $
-\Delta \psi = \lambda g'(\psi_\epsilon)\psi, \psi\in\dot{H}^1(\Omega)$ for all $0<\lambda<1$, since the eigenfunctions can be obtained by the similar bifurcation:
\begin{equation} \label{bifurcation2}
\begin{array}{llll}
\ep=0&&0<\ep<1\\
3\tanh^2-1&\longrightarrow&3\left({\sqrt{1-\ep^2}\sinh(y)\over \cosh(y)+\epsilon \cos(x)}\right)^2-1=3\left(\sqrt{1 - \epsilon^2}\frac{ \partial \psi_\ep}{\partial y}\right)^2-1,\\
 {\sin(x)\sinh(y)\over \cosh^2(y)}&\longrightarrow& \frac{\sin(x)\sinh(y)}{(\cosh(y)+\epsilon \cos(x))^2}=-{1\over \ep}{\partial \psi_\ep\over\partial x}\frac{ \partial \psi_\ep}{\partial y}, \\
 {\sinh(y)\cos(x)\over \cosh^2(y)}&\longrightarrow&  \frac{\sinh(y)(\ep\cosh(y)+\cos(x))}{(\cosh(y)+\epsilon \cos(x))^2}=\frac{ \partial \psi_\ep}{\partial y}\left((1 - \epsilon^2) \frac{ \partial \psi_\ep}{\partial \ep}\right),\\
 {\sin(2x)\over\cosh^2(y)}&\longrightarrow&  \frac{\sin(x)(\ep\cosh(y)+\cos(x))}{(\cosh(y)+\epsilon \cos(x))^2}=-{1\over \ep}{\partial \psi_\ep\over\partial x}\left((1 - \epsilon^2) \frac{ \partial \psi_\ep}{\partial \ep}\right),\\
 {\cos(2x)\over\cosh^2(y)}&\longrightarrow&  \frac{(\ep\cosh(y)+\cos(x))^2-(\sqrt{1-\ep^2}\sin(x))^2}{(\cosh(y)+\epsilon \cos(x))^2}=\left((1 - \epsilon^2) \frac{ \partial \psi_\ep}{\partial \ep}\right)^2-\left(-{\sqrt{1 - \epsilon^2}\over \ep}{\partial \psi_\ep\over\partial x}\right)^2.
 \end{array}
\end{equation}
This gives a hint that $\sin(x)$ for $\ep=0$ branches to $\sqrt{1-\ep^2}\sin(x)$ for $0<\ep<1$, and $\sinh(y)$ branches to $\sqrt{1-\ep^2}\sinh(y)$.
This also motivates us to rescale   $\partial_x \psi_\ep, \partial_y \psi_\ep$ and $\partial_\epsilon \psi_\ep$ to be
\begin{align}\label{three-kers1}
\eta_\ep(x, y) & :=  \frac{-\sqrt{1 - \epsilon^2}}{\ep} \frac{ \partial \psi_\ep}{\partial x}= \frac{\sqrt{1 - \epsilon^2}\sin(x)}{\cosh(y)+\epsilon \cos(x)},\\\label{three-kers2}
\gamma_\ep(x, y)  & :=  \sqrt{1 - \epsilon^2}\frac{ \partial \psi_\ep}{\partial y}= \frac{\sqrt{1 - \epsilon^2}\sinh(y)}{\cosh(y)+\epsilon \cos(x)},\\\label{three-kers3}
\xi_\ep(x, y)  & := (1 - \epsilon^2) \frac{ \partial \psi_\ep}{\partial \ep} = \frac{\epsilon \cosh(y) + \cos(x)}{\cosh(y)+\epsilon \cos(x)},
\end{align}
since  the above eigenfunctions of $\lambda=3$ can be written as polynomials of $\eta_\ep$, $\gamma_\ep$ and $\xi_\ep$, and
\begin{align}\label{eta-gamma-xi}\eta_\ep^2 + \gamma_\ep^2 + \xi_\ep^2 = 1.\end{align}
%As the cat's eyes steady states could be regarded as bifurcation from the hyperbolic tangent shear flow,
Now, we know how to bifurcate  $\cos(x),\sin(x),\sinh(y),\cosh(y)$ from $\ep=0$ to $0<\ep<1$.
However,   $\cos(kx)$ and $\sin(kx)$  appear in the eigenfunctions in \eqref{sol-elip02} for $\ep=0$. It is difficult to study how such functions branch to the case $0<\ep<1$. Our  observation is that using the De Moivre's formulae, we  can expand $\cos(kx)$ and $\sin(kx)$ by $\sin(x)$ and $\cos(x) $ as follows:
 \begin{align}\label{De Moivres formula1}
 &\cos(kx)=\sum_{j=0}^k\begin{pmatrix}k\\j\end{pmatrix}\cos^j(x)\sin^{k-j}(x)\cos\left({(k-j)\pi\over2}\right),\\\label{De Moivres formula2}
 &\sin(kx)=\sum_{j=0}^k\begin{pmatrix}k\\j\end{pmatrix}\cos^j(x)\sin^{k-j}(x)\sin\left({(k-j)\pi\over2}\right).
 \end{align}
 In this way, the bifurcation of $\cos(kx)$ and $\sin(kx)$ reduce to that of $\cos(x)$ and $\sin(x)$. Now, every component in the eigenfunctions of \eqref{sol-elip02}
 is a combination of $\cos(x),\sin(x),\sinh(y),\cosh(y)$.
 \if0
Another point of applying  the De Moivre's formulae is that the eigenfunctions  $L_{n,k}(\tanh(y))\cos(kx)$  and $L_{n,k}(\tanh(y))\sin(kx)$   in \eqref{sol-elip02} for $\ep=0$  can be rewritten as polynomials  of $
\eta_0,
\gamma_0,$ and $
\xi_0$:
\begin{align}\label{ep=0-cos}
 &L_{n,k}(\tanh(y))\cos(kx)
 ={d^k\over d\gamma_0^k}L_n(\gamma_0)
 \sum_{j=0}^k\begin{pmatrix}k\\j\end{pmatrix}\xi_0^j\eta_0^{k-j}\cos\left({(k-j)\pi\over2}\right),\\\label{ep=0-sin}
 &L_{n,k}(\tanh(y))\sin(kx)=
 {d^k\over d\gamma_0^k}L_n(\gamma_0)
 \sum_{j=0}^k\begin{pmatrix}k\\j\end{pmatrix}\xi_0^j\eta_0^{k-j}\sin\left({(k-j)\pi\over2}\right),
\end{align}
where $\gamma_0=\gamma=\tanh(y)$,  $\xi_0=\cos(x)\text{sech}(y)=\cos(x)\sqrt{1-\gamma_0^2}$, and $\eta_0=\sin(x)\text{sech}(y)=\sin(x)\sqrt{1-\gamma_0^2}$.
We note that the remaining eigenfunctions  $L_{n}(\tanh(y))-L_{n}(0)$ have already been functions $L_{n}(\gamma_0)-L_{n}(0)$ of $\gamma_0$.
\fi
%Since $
%\eta_\ep,
%\gamma_\ep,$ and $
%\xi_\ep$, $0<\epsilon<1$, bifurcate from   $
%\eta_0,
%\gamma_0,$ and $
%\xi_0$,
%by \eqref{ep=0-cos}-\eqref{ep=0-sin}
\if0
After finding a few eigenvalues with corresponding  eigenfunctions by  bifurcation at the shear flow and combining the two observations
\fi
Using the above branches and after direct computations, the branches of  the eigenfunctions
are  polynomials of the three functions  $
\eta_\epsilon,
\gamma_\epsilon,$ and $
\xi_\epsilon$:
\begin{align}\label{ep larger than 0-sol1}
&L_{n}(\gamma_\epsilon)-L_{n}(0)\\\label{ep larger than 0-sol2}
 &{d^k\over d\gamma_\ep^k}L_n(\gamma_\ep)
 \sum_{j=0}^k\begin{pmatrix}k\\j\end{pmatrix}\xi_\ep^j\eta_\ep^{k-j}\cos\left({(k-j)\pi\over2}\right),\\\label{ep larger than 0-sol3}
 &{d^k\over d\gamma_\ep^k}L_n(\gamma_\ep)
 \sum_{j=0}^k\begin{pmatrix}k\\j\end{pmatrix}\xi_\ep^j\eta_\ep^{k-j}\sin\left({(k-j)\pi\over2}\right).
\end{align}
Another approach to obtain \eqref{ep larger than 0-sol2}-\eqref{ep larger than 0-sol3} is first applying the De Moivre's formulae to the eigenfunctions  $L_{n,k}(\tanh(y))\cos(kx)$  and $L_{n,k}(\tanh(y))\sin(kx)$   in \eqref{sol-elip02} for $\ep=0$ to get
\begin{align}\label{ep=0-cos}
 &L_{n,k}(\tanh(y))\cos(kx)
 ={d^k\over d\gamma_0^k}L_n(\gamma_0)
 \sum_{j=0}^k\begin{pmatrix}k\\j\end{pmatrix}\xi_0^j\eta_0^{k-j}\cos\left({(k-j)\pi\over2}\right),\\\label{ep=0-sin}
 &L_{n,k}(\tanh(y))\sin(kx)=
 {d^k\over d\gamma_0^k}L_n(\gamma_0)
 \sum_{j=0}^k\begin{pmatrix}k\\j\end{pmatrix}\xi_0^j\eta_0^{k-j}\sin\left({(k-j)\pi\over2}\right),
\end{align}
and then carrying out the branches from $\xi_0$, $\gamma_0$, $\eta_0$ to $\xi_\ep$, $\gamma_\ep$, $\eta_\ep$,
where $\gamma_0=\gamma=\tanh(y)$,  $\xi_0=\cos(x)\text{sech}(y)=\cos(x)\sqrt{1-\gamma_0^2}$, and $\eta_0=\sin(x)\text{sech}(y)=\sin(x)\sqrt{1-\gamma_0^2}$.
%We note that the remaining eigenfunctions  $L_{n}(\tanh(y))-L_{n}(0)$ have already been functions $L_{n}(\gamma_0)-L_{n}(0)$ of $\gamma_0$.
By induction one can prove that   the functions in \eqref{ep larger than 0-sol1}-\eqref{ep larger than 0-sol3} are exactly  eigenfunctions of $
-\Delta \psi = \lambda g'(\psi_\epsilon)\psi
$ with $\lambda=n(n+1)/2$ for all $0<\epsilon<1$.
% we need further to deal with the projection terms and the function spaces in our setting, as well as
A natural question is whether there are  other linearly independent eigenfunctions.
With this problem  and our approach for $\ep=0$ in mind, we proceed to look for  change of variables for $0<\ep<1$.
Since $\gamma_\ep$ is branched from $\tanh(y)$ and recall that the change of variable is $y\mapsto \tanh(y)$ for $\ep=0$,  it is reasonable to define a new variable $\gamma_\ep$  for $0<\ep<1$.
%The   difficulty is how to get suitable bifurcation for  the factors  $\cos(kx)$ and $\sin(kx)$ for $\epsilon=0$, from which we may get
%another new variable $\theta_\ep$ to replace the original variable $x$.
The discovery of the other new variable, which is denoted by $\theta_\epsilon$ and should be branched  from the original variable $x$, is more subtle. Note that the eigenfunctions  \eqref{ep larger than 0-sol2}-\eqref{ep larger than 0-sol3} for  $0<\ep<1$  have the same forms with the eigenfunctions  \eqref{ep=0-cos}-\eqref{ep=0-sin} for $\ep=0$. The left hand sides of  \eqref{ep=0-cos}-\eqref{ep=0-sin}  for $\ep=0$ inspire us that  in the new variables $(\theta_\ep,\gamma_\ep)$,   the eigenfunctions for $0<\ep<1$ might  have the same forms
  $L_{n,k}(\gamma_\epsilon)\cos(k\theta_\epsilon)$  and $L_{n,k}(\gamma_\epsilon)\sin(k\theta_\epsilon)$. Applying the De Moivre's formula to $\cos(k\theta_\epsilon)$  and $\sin(k\theta_\epsilon)$, we have
 \begin{align}\nonumber
&L_{n,k}(\gamma_\epsilon)\cos(k\theta_\epsilon)\\\label{eigen-ep-gamma-theta1}
=&
 {d^k\over d\gamma_\ep^k}L_n(\gamma_\ep) \sum_{j=0}^k\begin{pmatrix}k\\j\end{pmatrix}\left(\sqrt{1-\gamma_\ep^2}\cos(\theta_\ep)\right)^j\left(\sqrt{1-\gamma_\ep^2}\sin(\theta_\ep)\right)^{k-j}\cos\left({(k-j)\pi\over2}\right),\\\nonumber
 &L_{n,k}(\gamma_\epsilon)\sin(k\theta_\epsilon)\\\label{eigen-ep-gamma-theta2}
=&
 {d^k\over d\gamma_\ep^k}L_n(\gamma_\ep) \sum_{j=0}^k\begin{pmatrix}k\\j\end{pmatrix}\left(\sqrt{1-\gamma_\ep^2}\cos(\theta_\ep)\right)^j\left(\sqrt{1-\gamma_\ep^2}\sin(\theta_\ep)\right)^{k-j}\sin\left({(k-j)\pi\over2}\right).
  \end{align}
\if0
A natural question is whether other eigenfunctions for $0<\epsilon<1$ could be bifurcated  from the eigenfunctions in \eqref{sol-elip02}  for $\varepsilon=0$. Since $\gamma_\ep$ is bifurcated from $\tanh(y)$,  it is reasonable to get the bifurcations  $L_{n}(\gamma_\ep)$ and $L_{n,k}(\gamma_\ep)$  for the factors $L_{n}(\tanh(y))$ and $L_{n,k}(\tanh(y))$  in \eqref{sol-elip02}.
Thus, we could get a new variable $\gamma_\ep$  for $0<\epsilon<1$ as what we did for $\epsilon=0$.
The   difficulty is how to get suitable bifurcation for  the factors  $\cos(kx)$ and $\sin(kx)$ for $\epsilon=0$, from which we may get
another new variable $\theta_\ep$ to replace the original variable $x$.
\fi
Comparing the factors in  \eqref{ep larger than 0-sol2}-\eqref{ep larger than 0-sol3} and \eqref{eigen-ep-gamma-theta1}-\eqref{eigen-ep-gamma-theta2}, and in view of  \eqref{eta-gamma-xi}, we can
define the other new variable as  an angle $\theta_\ep \in [0, 2\pi]$ such that
\begin{align}\label{def-eta-ep}
\eta_\ep& = \sqrt{1-\gamma_\ep^2} \sin(\theta_\ep), \\\label{def-xi-ep}
\xi_\ep& = \sqrt{1-\gamma_\ep^2} \cos(\theta_\ep),
\end{align}
where $\ep\in[0,1)$. In summary, we change the original variables $(x,y)$ to the new ones  $(\theta_\ep,\gamma_\ep)$ as follows
 \begin{align} \label{transf1}
 \theta_\ep(x,y) & = \left\{ \begin{array}{rcl} \arccos \left( \frac{\xi_\ep}{\sqrt{1-\gamma_\ep^2}} \right) & \mbox{ for } & (x,y) \in [0, \pi]\times\mathbb{R}, \\
 2\pi - \arccos \left( \frac{\xi_\ep}{\sqrt{1-\gamma_\ep^2}} \right) & \mbox{ for } & (x,y) \in (\pi, 2\pi]\times\mathbb{R}, \end{array}\right.\\
  \label{transf2}
 \gamma_\ep(x,y) & = \frac{\sqrt{1 - \epsilon^2}\sinh(y)}{\cosh(y)+\epsilon \cos(x)}\quad\text{for}\quad  (x,y) \in [0, 2\pi]\times\mathbb{R}.
 \end{align}
Here, $(\theta_\ep, \gamma_\ep) \in \tilde \Omega = \mathbb{T}_{2\pi} \times [-1, 1]$  and $\ep\in[0,1)$.
 The change of variables in $\eqref{transf1}$ and $\eqref{transf2}$ is well-defined and plays an important role in solving the associated eigenvalue problem \eqref{eigen-p-cat-eyes}. First,  $\eqref{transf1}$-$\eqref{transf2}$ reduce to  the change of variable in the shear case $\ep = 0$ as $\gamma_0 = \tanh(y) = \gamma$ and $\theta_0 = x$. Second,
for the new variables $\theta_\ep$ and $\gamma_\ep$,
  the Jacobian of this transformation is
\begin{align}\label{Jacobian of the transformation-ep}
\frac{\partial (\theta_\ep, \gamma_\ep)}{\partial (x, y)} = \frac{\partial \theta_\ep}{\partial x}\frac{\partial \gamma_\ep}{\partial y} - \frac{\partial \theta_\ep}{\partial y}\frac{\partial \gamma_\ep}{\partial x} = \frac 1 2 g'(\psi_\epsilon) > 0,
\end{align}
where $\ep\in[0,1)$. More importantly, the parameter $\ep$ is fully encoded into the new variables. This enables us to reduce the eigenvalue problem in the cat's eyes case ($0<\ep<1$) to the hyperbolic tangent shear case ($\ep=0$),  which has been studied in Subsection \ref{Solutions to the eigenvalue problem ep=0}. More precisely, the associated eigenvalue problem \eqref{eigen-p-cat-eyes} is transformed to \eqref{eigenvalue problem-ep-new}, which is the same one with \eqref{elip02-x-gamma}.  In particular, the eigenfunctions \eqref{ep larger than 0-sol1}-\eqref{ep larger than 0-sol3} form a complete and orthogonal basis after taking the projection terms and specific spaces in consideration.

\if0
\begin{remark}
The transformation above was inspired by the form of eigenfunctions in Lemma \ref{eigen-ep}, which was initially found by bifurcating from the explicit eigenfunctions of the special case $\ep = 0$ in Lemma \ref{associate_ep0} and then proved in a lengthy induction argument.
\end{remark}
\fi
By direct computation, we  obtain many  properties of $\eta_\ep, \gamma_\ep, \xi_\ep$ and $\theta_\ep$. We present some of them below in Propositions \ref{prop0}, \ref{prop1} and \ref{prop3}.
\begin{proposition}\label{prop0}
$(1)$ In terms of $\eta_\ep, \gamma_\ep, \xi_\ep$ and $\ep$, the steady state $ \omega_\ep$ is represented by
\begin{align}\label{omega-xi-eta-gamma-ep}\omega_\ep = -\left(\frac{(\xi_\ep - \ep)^2}{1-\ep^2} + \eta_\ep^2\right).\end{align}
$(2)$ The partial derivatives of $\eta_\ep(x, y), \gamma_\ep(x, y), \xi_\ep(x, y)$ and $\theta_\ep(x, y)$ are represented by
\begin{align*}\frac{\partial \xi_\ep}{\partial x} =& -\frac{\eta_\ep (1-\xi_\ep \ep)}{\sqrt{1-\ep^2}}, \quad \frac{\partial \xi_\ep}{\partial y} = -\frac{\gamma_\ep (\xi_\ep - \ep)}{\sqrt{1-\ep^2}},\quad
\frac{\partial \eta_\ep}{\partial x} = \frac{\xi_\ep - \ep + \eta_\ep^2 \ep}{\sqrt{1-\ep^2}}, \quad \frac{\partial \eta_\ep}{\partial y} = \frac{-\gamma_\ep \eta_\ep}{\sqrt{1-\ep^2}},\\
\frac{\partial \gamma_\ep}{\partial x} =& \frac{\ep \gamma_\ep \eta_\ep}{\sqrt{1-\ep^2}}, \quad \frac{\partial \gamma_\ep}{\partial y} = \frac{1-\xi_\ep \ep - \gamma_\ep^2}{\sqrt{1-\ep^2}},\quad
\frac{\partial \theta_\ep}{\partial x} = \frac{\gamma_{\ep y}}{1-\gamma_\ep^2}, \quad \frac{\partial \theta_\ep}{\partial y} = - \frac{\gamma_{\ep x}}{1-\gamma_\ep^2}.\end{align*}
\end{proposition}

As a consequence, the representation of $\psi_\ep = - \frac 1 2 \ln(-\omega_\ep)$ and $ g'(\psi_\ep) = -2\omega_\ep$ in terms of $\eta_\ep, \gamma_\ep, \xi_\ep$ and $\ep$ can be directly obtained by  \eqref{omega-xi-eta-gamma-ep}.
\begin{proof}
By \eqref{three-kers3}, we have
\begin{align}\label{cosh-cos}
\frac{\cosh(y)}{\cos(x)} = \frac{1-\xi_\ep \ep}{\xi_\ep - \ep}.
\end{align}
Together with \eqref{three-kers1}-\eqref{three-kers2},
we get
\begin{align}\label{tan-tanh}
\tan(x) = \frac{\sqrt{1-\ep^2}\eta_\ep}{\xi_\ep - \ep}, \quad  \tanh(y) = \frac{\sqrt{1-\ep^2}\gamma_\ep}{1- \xi_\ep\ep}.
\end{align}
Then
$$\omega_\ep =  - \frac{(1-\ep^2) \sec^2(x)}{\left( \frac{\cosh(y)}{\cos(x)} + \ep \right)^2} = -\left(\frac{(\xi_\ep - \ep)^2}{1-\ep^2} + \eta_\ep^2\right).$$
Moreover,
\begin{align}\label{tan-theta-ep}
\tan(\theta_\ep) = \frac{\eta_\ep}{\xi_\ep}.
\end{align}
The conclusions in (2) then follow from taking partial derivatives on \eqref{cosh-cos}, \eqref{tan-tanh} and \eqref{tan-theta-ep}.
\end{proof}

\begin{proposition}\label{prop1}
With $(\theta_\ep, \gamma_\ep)$ defined in \eqref{transf1}-\eqref{transf2}, we have
\begin{itemize}
\item $(\theta_\ep)_x^2 + (\theta_\ep)_y^2 =\frac 1 2 \frac{ g'(\psi_\epsilon)}{1-\gamma_\ep^2}.$

\item $-\Delta \theta_\ep = -(\theta_\ep)_{xx} - (\theta_\ep)_{yy} = 0.$

\item $$ -\Delta \eta_\ep = g'(\psi_\epsilon) \eta_\ep, \quad
-\Delta \gamma_\ep = g'(\psi_\epsilon) \gamma_\ep, \quad
-\Delta \xi_\ep = g'(\psi_\epsilon) \xi_\ep.$$

\item \begin{align*}  \nabla \eta_\ep \cdot \nabla \gamma_\ep &= -\frac 1 2 g'(\psi_\epsilon) \eta_\ep\gamma_\ep, \quad  \nabla \eta_\ep \cdot \nabla \eta_\ep = \frac 1 2 g'(\psi_\epsilon)(1 - \eta_\ep^2),  \\
 \nabla \gamma_\ep \cdot \nabla \xi_\ep &= -\frac 1 2 g'(\psi_\epsilon) \gamma_\ep\xi_\ep, \quad \nabla \gamma_\ep \cdot \nabla \gamma_\ep = \frac 1 2 g'(\psi_\epsilon) (1 - \gamma_\ep^2 ), \\
 \nabla \xi_\ep \cdot \nabla \eta_\ep &= -\frac 1 2 g'(\psi_\epsilon) \xi_\ep\eta_\ep, \quad \nabla \xi_\ep \cdot \nabla \xi_\ep = \frac 1 2 g'(\psi_\epsilon)( 1 - \xi_\ep^2).
\end{align*}

\item
\begin{align*}
-\Delta (\eta_\ep\gamma_\ep) &=  3g'(\psi_\epsilon) \eta_\ep\gamma_\ep, \quad -\Delta (3\eta_\ep^2-1) = 3g'(\psi_\epsilon) (3\eta_\ep^2 - 1), \\
-\Delta (\gamma_\ep \xi_\ep) &= 3g'(\psi_\epsilon) \gamma_\ep \xi_\ep, \quad -\Delta (3\gamma_\ep^2-1) = 3g'(\psi_\epsilon) (3\gamma_\ep^2 - 1), \\
-\Delta (\xi_\ep \eta_\ep) & =   3g'(\psi_\epsilon) \xi_\ep\eta_\ep,  \quad -\Delta (3\xi_\ep^2-1) = 3g'(\psi_\epsilon) (3\xi_\ep^2 - 1).
\end{align*}

\end{itemize}
\end{proposition}

\begin{proposition}\label{prop3} Let $\Psi(\theta_\ep, \gamma_\ep) = \psi(x(\theta_\ep, \gamma_\ep),y(\theta_\ep, \gamma_\ep))$. Then
\begin{align}\label{laplacian}-\Delta \psi = \frac 1 2 g'(\psi_\ep) \left( -\frac{\Psi_{\theta_\ep \theta_\ep}}{1-\gamma_\ep^2} - \left( (1-\gamma_\ep^2)\Psi_{\gamma_\ep} \right)_{\gamma_\ep} \right)\end{align}
and
\begin{align}\label{gradient-psi}\| \nabla \psi\|_{L^2(\Omega)}^2=\iint_{\tilde \Omega}\left({1\over1-\gamma_\ep^2}|\Psi_{\theta_\ep}|^2+(1-\gamma_\ep^2)|\Psi_{\gamma_\ep}|^2\right)d \theta_\ep d\gamma_\ep.\end{align}
\end{proposition}
\begin{proof} First, we prove \eqref{laplacian}.
By Proposition \ref{prop1}, we have
$-\Delta \theta_\ep = 0$, $(\theta_\ep)_x(\gamma_\ep)_x + (\theta_\ep)_y(\gamma_\ep)_y=0$, $(\theta_\ep)_x^2 + (\theta_\ep)_y^2 =\frac 1 2 \frac{ g'(\psi_\epsilon)}{1-\gamma_\ep^2}$,
$-\Delta \gamma_\ep = g'(\psi_\epsilon)\gamma_\ep$, and $(\gamma_\ep)_x^2 + (\gamma_\ep)_y^2  = \frac 1 2 g'(\psi_\epsilon)(1-\gamma_\ep^2).$ Thus,
\begin{align*}
\begin{split}
- \Delta \psi & = - \psi_{xx} - \psi_{yy} \\
&= -\Psi_{\theta_\ep \theta_\ep}((\theta_\ep)_x^2 + (\theta_\ep)_y^2) + \Psi_{\theta_\ep}(-\Delta \theta_\ep) - \Psi_{\gamma_\ep \gamma_\ep}\left((\gamma_\ep)_x^2 + (\gamma_\ep)_y^2 \right) + \Psi_{\gamma_\ep}(-\Delta \gamma_\ep) \\
&= -\frac 1 2 g'(\psi_\epsilon) \frac{\Psi_{\theta_\ep \theta_\ep}}{1-\gamma_\ep^2} - \frac 1 2 g'(\psi_\epsilon)(1-\gamma_\ep^2)\Psi_{\gamma_\ep \gamma_\ep} +  g'(\psi_\epsilon)\Psi_{\gamma_\ep}\gamma_\ep\\
&= \frac 1 2 g'(\psi_\ep) \left( -\frac{\Psi_{\theta_\ep \theta_\ep}}{1-\gamma_\ep^2} - \left( (1-\gamma_\ep^2)\Psi_{\gamma_\ep} \right)_{\gamma_\ep} \right)
\end{split}
\end{align*}
and
\begin{align*}
\|\nabla \psi\|^2_{L^2(\Omega)}
& = \iint_\Omega\left( |\psi_x|^2 + |\psi_y|^2\right)dxdy \\
& = \iint_\Omega\left( |\Psi_{\theta_\ep}|^2 \left((\partial_x \theta_\ep)^2 + (\partial_y \theta_\ep)^2\right) + |\Psi_{\gamma_\ep}|^2 \left((\partial_x \gamma_\ep)^2 + (\partial_y \gamma_\ep)^2\right)\right) dxdy \\
& = \iint_\Omega \frac 1 2 g'(\psi_\ep) \left( {1\over1-\gamma_\ep^2}|\Psi_{\theta_\ep}|^2+(1-\gamma_\ep^2)|\Psi_{\gamma_\ep}|^2 \right) dx dy \\
&=\int_{-1}^1\int_{0}^{2\pi} \left({1\over1-\gamma_\ep^2}|\Psi_{\theta_\ep}|^2+(1-\gamma_\ep^2)|\Psi_{\gamma_\ep}|^2\right)d \theta_\ep d\gamma_\ep.
\end{align*}
\end{proof}
Similar to \eqref{gradient-psi}, we have
\begin{align}\label{psi-X-ep-inner product}
( \psi_1,\psi_2)_{\tilde X_\ep}=\iint_{\tilde \Omega}\left({1\over1-\gamma_\ep^2}(\Psi_{1})_{\theta_\ep}(\Psi_{2})_{\theta_\ep}+(1-\gamma_\ep^2)(\Psi_{1})_{\gamma_\ep}(\Psi_{2})_{\gamma_\ep}\right)d \theta_\ep d\gamma_\ep\end{align}
for $\Psi_i(\theta_\ep, \gamma_\ep) = \psi_i(x(\theta_\ep, \gamma_\ep),y(\theta_\ep, \gamma_\ep))$, $i=1,2$.
\if0
The new coordinate $(\theta_\ep, \gep)$ plays an important role in  the spectral  analysis for  the case $0 < \ep < 1$.
\fi
Then we will prove that under the new coordinate $(\theta_\ep, \gep)$, the associated eigenvalue problem \eqref{eigen-p-cat-eyes} can be reduced to the corresponding one \eqref{elip02-x-gamma} in the  case  $\ep = 0$, which is solved in Theorem \ref{associate_ep0}. To this end, we preliminarily clarify the space of stream functions, solvability of the Poisson equation and boundedness of the energy quadratic form in the next subsection.
\subsubsection{Space of stream functions, Poisson equation and energy quadratic form}
Let $0<\epsilon<1$ and $\Psi(\theta_\ep, \gamma_\ep) = \psi(x(\theta_\ep, \gamma_\ep),y(\theta_\ep, \gamma_\ep))$.
Recall that the space $\tilde X_0$ of stream functions $\psi$  for  $\ep = 0$ is $\dot{H}^1(\Omega)$ with an additional condition that $\widehat{\psi}_0(0)=0$. If we use the same space  $\tilde X_0$ for $0<\epsilon<1$, then
$n^-(A_\epsilon)\geq1$ for the elliptic operator $A_\epsilon$ without projection (see Remark \ref{X-ep-X0-A-ep-neg}), which is inapplicable in the proof of  nonlinear stability. Furthermore,
it is inappropriate to establish an isomorphism for the spaces  of stream functions   between $\epsilon=0$ and $0<\epsilon<1$, since the variable $\theta_\epsilon$ involves $x$ and $y$ in a very coupled way so  that in  the new variables, $\widehat{\psi}_0$ is no longer  the 0 mode of $\Psi$ after writing it in the Fourier series with respect to $\theta_\epsilon$.  Instead, our choice is to  replace the condition that $\widehat{\psi}_0(0)=0$  to $\widehat{\Psi}_0(0)=0$ in the definition of the space  of stream functions, where $\widehat{\Psi}_0(0)={1\over 2\pi}\int_{0}^{2\pi}\Psi(\theta_\epsilon,0)d\theta_\epsilon$. In this way, we can not only ensure that $\dim \ker (A_\epsilon)=3$ and $n^-(A_\epsilon)=0$ (see Corollary \ref{kernel of  the operator A-ep and a decomposition of tilde Xep}), but also establish an isomorphism for the spaces  of stream functions  between $\epsilon=0$ and $0<\epsilon<1$.  Noting that $y=0$ if and only if $\gamma_\epsilon=0$, by Proposition \ref{prop0} (2)  we have
\begin{align}\nonumber
\widehat{\Psi}_0(0)=&{1\over2\pi}\int_{0}^{2\pi}\Psi(\theta_\epsilon,0)d\theta_\epsilon={1\over2\pi}\int_{0}^{2\pi}\psi(x(\theta_\epsilon,0),0){\partial{\theta_\epsilon}\over \partial x}|_{y=0}dx\\\nonumber
=&{1\over2\pi}\int_{0}^{2\pi}\psi(x,0){\gamma_{\epsilon y}}|_{y=0}dx
={1\over2\pi\sqrt{1-\ep^2}}\int_{0}^{2\pi}\psi(x,0)(1-\xi_\ep \ep)|_{y=0}dx\\\label{widehat-Psi-to-psi-1+ep-cos}
=&{\sqrt{1-\ep^2}\over2\pi}\int_{0}^{2\pi}\psi(x,0){1\over1+\epsilon\cos(x)} dx.
\end{align}
Thus,  we define the space  of stream functions specifically in  the original variables as follows
\begin{align}\label{tilde-X-e}\tilde{X}_\ep = \left\{ \psi\bigg| \iint_\Omega |\nabla \psi|^2 dxdy <  \infty \text{ and } \int_{0}^{2\pi}\psi(x,0){1\over1+\epsilon\cos(x)} dx = 0  \right\}.\end{align}
In  the new variables, by \eqref{gradient-psi}-\eqref{widehat-Psi-to-psi-1+ep-cos} $\tilde{X}_\ep$ is equivalent  to the following space
\begin{align*}
\tilde{Y}_\ep = \left\{ \Psi \bigg| \iint_{\tilde \Omega}\left({1\over1-\gamma_\ep^2}|\Psi_{\theta_\ep}|^2+(1-\gamma_\ep^2)|\Psi_{\gamma_\ep}|^2\right)d \theta_\ep d\gamma_\ep< \infty \text{ and } \widehat{\Psi}_0(0)=0 \right\},
\end{align*}
where $\tilde \Omega = \mathbb{T}_{2\pi} \times [-1, 1]$.
\if0
Obviously, $\eta_\ep, \gamma_\ep, \xi_\ep \in \tilde{X_\ep}$. We will prove that $\tilde{X}_\ep$ is a Hilbert space and the Poincar\'e inequalities hold in $\tilde{X}_\ep$. For the sake of convenience, we also define the space
The functions in $\tilde{X}_\ep$ and functions in $\tilde{Y}_\ep$ are equivalent under the change of variables in \eqref{transf1} and \eqref{transf2}. The constraint of $\int_{0}^{2\pi}\Psi(\theta_\ep, 0)d\theta_\ep = 0$ is chosen to make sure that the Poincar\'e inequalities in Lemma \ref{poincare1} and Lemma \ref{poincare2} hold, $\tilde{X}_\ep$ is a Hilbert space in Lemma \ref{Hilbert}, and the Poisson equation $-\Delta \psi = \omega$ has a unique weak solution in $\tilde{X}_\ep$ for any $\omega \in X_\ep$. Also, observe that we can apply another change of variables from $(\theta_\ep, \gamma_\ep)$ to $(\tilde{x}, \tilde{y})$ using \eqref{change of variables2} so that $\tilde{X}_\ep$ is changed to $\tilde{X}_0$ in the shear case.
\fi
Noting that $\tilde{Y}_\ep$ is the same space as $\tilde Y_0$ as defined in \eqref{tilde-Y0-def}, we thus get the following result.



\begin{lemma}\label{hilbert-ep}
Let $0<\epsilon<1$. Then

$(1)$ the function space $\tilde{Y}_\ep$ equipped with the inner product
 $$(\Psi_1, \Psi_2) = \iint_{\tilde{\Omega}}  \left({1\over1-\gamma_\ep^2}(\Psi_1)_{\theta_\ep}(\Psi_2)_{\theta_\ep} +(1-\gamma_\ep^2)(\Psi_1)_{\gamma_\ep}(\Psi_2)_{\gamma_\ep}\right)d \theta_\ep d\gamma_\ep, \quad \forall\; \Psi_1, \Psi_2 \in \tilde{Y}_\ep$$
 is  a Hilbert space;

$(2)$
the function space $\tilde{X}_\ep$ equipped with the inner product
$$(\psi_1, \psi_2) = \iint_{\Omega} \nabla \psi_1 \cdot \nabla \psi_2 dxdy, \quad \forall\; \psi_1, \psi_2 \in \tilde{X}_\ep$$
 is a Hilbert space. Moreover,
\begin{align}\label{Psi-psi-norm-eq} \| \psi\|_{\tilde X_\epsilon}^2= \| \nabla \psi\|_{L^2(\Omega)}^2=\iint_{\tilde \Omega}\left({1\over1-\gamma_\ep^2}|\Psi_{\theta_\ep}|^2+(1-\gamma_\ep^2)|\Psi_{\gamma_\ep}|^2\right)d \theta_\ep d\gamma_\ep=\|\Psi\|_{\tilde Y_\epsilon}^2
\end{align}
for   $\psi \in \tilde{X}_\ep$ and $\Psi \in \tilde{Y}_\ep$ such that $\psi(x,y) = \Psi(\theta_\ep, \gamma_\ep)$.
\end{lemma}
\begin{proof}
(1) follows from Lemma \ref{Hilbert-new variables-0}, and (2) is obtained by \eqref{gradient-psi}-\eqref{widehat-Psi-to-psi-1+ep-cos} and (1).
\end{proof}
\if0
\begin{proof}
Firstly, we prove that $\|\nabla \psi\|_{L^2(\Omega)} = 0$ implies $\psi = 0$ in $\tilde{X_\ep}$. Expanding $\Psi(\theta_\ep, \gamma_\ep)$ in Fourier series form and using equaiton \eqref{laplacian}, we have that
\begin{align}\label{Psi_fourier}
\|\nabla \psi\|_{L^2(\Omega)}^2
& = 2\pi \int_{-1}^{1} \sum_{k \neq 0} \frac{k^2 \Psi^k(\gamma_\ep)^2}{1-\gamma_\ep^2} + (1-\gamma_\ep^2) \left( \Psi^0_{\gamma_\ep}(\gamma_\ep)^2 + \sum_{k \neq 0}  \Psi^k_{\gamma_\ep}(\gamma_\ep)^2 \right) d \gamma_\ep = 0
\end{align}
where $$\Psi^k(\gamma_\ep) = \frac{1}{2\pi} \int_0^{2\pi} \Psi(\theta_\ep, \gamma_\ep) e^{-ik\theta_\ep} d \theta_\ep.$$
Since the integrand in \eqref{Psi_fourier} has every term nonnegative, we know that
$$ \Psi^k(\gamma_\ep) = 0 \text{ for } k \neq 0, \text{ and } \Psi^0_{\gamma_\ep}(\gamma_\ep) = 0.$$
Moreover, we have $$\Psi^0(\gamma_\ep) = \Psi^0(0) + \int_0^{\gamma_\ep} \Psi^0_{\hat{\gamma_\ep}}(\hat{\gamma_\ep})d\hat{\gamma_\ep} = 0,$$
for $\gamma_\ep \in [-1, 1]$.
So $\Psi^k(\gamma_\ep) = 0$ for all integers $k$ and thus $$\psi(x,y) = \Psi(\theta_\ep, \gamma_\ep) = 0.$$
To prove the completeness, let's introduce the following change of variables:
\begin{align}\label{change of variables2}
\left\{ \begin{array}{ccc} \tilde{x}  & = & \theta_\ep
\\ \tilde{y}  & = & \tanh(\gamma_\ep) \end{array} \right.
\text{ with }
\left\{ \begin{array}{ccc} d\tilde{x}  & = & d\theta_\ep
\\ d\tilde{y}  & = & (1-\gamma_\ep^2)^{-1}d\gamma_\ep, \end{array} \right.
\end{align}
where $(\tilde{x}, \tilde{y}) \in \tilde{\Omega} = \mathbb{T}_{2\pi} \times \mathbb{R}$.
Suppose $\{\psi_m(x, y) \}_{m=1}^{+\infty}$ is a Cauchy sequence in $\tilde{X}_\ep$, i.e.,
$$\lim_{m,n \rightarrow +\infty}\|\psi_m - \psi_n\|_{\tilde{X}_\ep} = 0,$$
where
\begin{align}\label{decom-phi}
\psi_m(x,y) = \Psi_m(\theta_\ep, \gamma_\ep) = \phi_m(\tilde{x}, \tilde{y}) = \phi_m^0(\tilde{y}) + \sum_{k\neq0}e^{ik\tilde{x}}\phi_m^k(\tilde{y}) := \phi_m^0(\tilde{y}) + \phi_m^{\neq 0}(\tilde{x}, \tilde{y}),
\end{align}
with $$\phi_m^k(\tilde{y}) = \frac{1}{2\pi} \int_0^{2\pi} \phi_m(\tilde{x}, \tilde{y})e^{-ik\tilde{x}} d\tilde{x}\quad \text{ and } \quad \phi_{m,{\neq0}}(\tilde{x}, \tilde{y}) = \sum_{k\neq0}e^{ik\tilde{x}}\phi_m^k(\tilde{y}).$$
Moreover, we have
\begin{align}\label{norm-phi}
\|\psi_m\|^2_{\tilde{X}_\ep}
& =  2\pi \int_{-1}^{1} \sum_{k \neq 0} \frac{k^2 \Psi_m^k(\gamma_\ep)^2}{1-\gamma_\ep^2} + (1-\gamma_\ep^2) \left( \left(\frac{d}{d \gamma_\ep} \Psi^0_m(\gamma_\ep)\right)^2 + \sum_{k \neq 0}  \left(\frac{d}{d \gamma_\ep} \Psi^k_m(\gamma_\ep)\right)^2  \right) d \gamma_\ep \\
& = 2\pi \int_{-\infty}^{+\infty} \sum_{k \neq 0} k^2 \phi_m^k(\tilde{y})^2 + \left(\frac{d}{d \tilde{y}} \phi^0_m(\tilde{y})\right)^2 + \sum_{k \neq 0}  \left(\frac{d}{d \tilde{y}} \phi^k_m(\tilde{y})\right)^2   d \tilde{y} \\
& = \|\frac{d}{d \tilde{y}} \phi_m^0\|^2_{L^2(\tilde{\Omega})} + \|\phi^{\neq 0}_m\|^2_{\dot{H}^1(\tilde{\Omega})} < \infty.
\end{align}
Actually, it is obvious that $\phi^{\neq 0}_m \in H^1(\tilde{\Omega})$ and $\exists \phi_{\neq0} \in H^1(\tilde{\Omega})$ such that
$$\lim_{m\rightarrow +\infty} \|\phi^{\neq 0}_m -  \phi_{\neq0}\|_{H^1(\tilde{\Omega})} = 0.$$
Also, $\exists \phi^0_*(\tilde{y}) \in L^2(\tilde{\Omega})$ such that
$$\lim_{m\rightarrow +\infty}\| \frac{d}{d \tilde{y}} \phi_m^0 -  \phi^0_*\|_{L^2(\tilde{\Omega})} = 0.$$
Now let $$\phi^0(\tilde{y}) = \int_{0}^{\tilde{y}} \phi^0_*(\hat{y}) d\hat{y},$$
so that we have $\phi^0(0) = 0$ and
$$ \lim_{m\rightarrow +\infty}\| \phi^{0}_m -  \phi^{0}\|_{\dot{H}^1(\tilde{\Omega})} = 0. $$
Finally, let $\phi^*(\tilde{x}, \tilde{y}) = \phi^0(\tilde{y}) + \phi_{\neq0}(\tilde{x}, \tilde{y}) = \Psi^*(\theta_\ep, \gamma_\ep) = \psi^*(x,y)$. We have $$\psi^*(x,y) \in \tilde{X}_\ep$$ and
$$\lim_{m \rightarrow 0}\|\psi_m - \psi^*\|^2_{\tilde{X}_\ep} = \lim_{m \rightarrow 0}\left( \|\phi^0_m - \phi^0\|^2_{\dot{H}^1(\tilde{\Omega})} +  \| \phi^{\neq 0}_m -  \phi_{\neq0}\|_{\dot{H}^1(\tilde{\Omega})} \right) = 0.$$
Therefore, $\tilde{X_\ep}$ is a Hilbert space. So do $\tilde{Y}_\ep$.
\end{proof}
\fi

Then we give the Poincar\'e inequality I for $0<\epsilon<1$.

\begin{lemma}[Poincar\'e inequality I-$\ep$]\label{poincare1ep}
$(1)$ For any $\Psi \in \tilde{Y_\ep}$, we have
\begin{align*}
 \|\Psi\|_{L^2(\tilde \Omega)}^2  \leq C \iint_{\tilde \Omega}\left({1\over1-\gamma_\ep^2}|\Psi_{\theta_\ep}|^2+(1-\gamma_\ep^2)|\Psi_{\gamma_\ep}|^2\right)d \theta_\ep d\gamma_\ep.
\end{align*}

$(2)$
For any $\psi \in \tilde{X_\ep}$, we have
\begin{align}\label{Poincare inequality I-ep22}
\iint_\Omega g'(\psi_\epsilon)|\psi|^2 dxdy  \leq C \|\nabla \psi\|_{L^2(\Omega)}^2.
\end{align}
\end{lemma}
\begin{proof}
(1) is the same as Lemma \ref{Poincare ineqalities-new-variable0} (1). To prove (2), let $  \Psi(\theta_\ep, \gamma_\ep)=\psi(x,y)$ for $\psi\in\tilde{X_\ep}$. By \eqref{Jacobian of the transformation-ep} we have
\begin{align}\label{Psi-psi-L2-ep}
2\iint_{\tilde \Omega}|\Psi|^2d \theta_\ep d\gamma_\ep= \iint_\Omega g'(\psi_\epsilon)|\psi|^2 dxdy.
\end{align}
 By \eqref{gradient-psi} and \eqref{Psi-psi-L2-ep}, we know that  (2) is a restatement of  (1) in the original variables $(x,y)$.
\end{proof}
\if0
\begin{proof}
 The proof for $0 < \ep < 1$ is slightly different from the proof of the case $\ep = 0$ in Lemma \ref{poincare1}. We write $\Psi(\theta_\ep, \gamma_\ep)$ in the Fourier series form
$$\Psi(\theta_\ep \gamma_\ep) = \Psi^0(\gamma_\ep) + \sum_{k\neq 0}e^{ik\theta_\ep}\Psi^k(\gamma_\ep), $$
with $$\Psi^k(\gamma_\ep) = \frac{1}{2\pi}\int_{0}^{2\pi} \Psi(\theta_\ep,\gamma_\ep) e^{-ik\theta_\ep} d\theta_\ep.$$
So
\begin{align}\label{gradient_norm}
\begin{split}
\|\nabla \psi\|_{L^2(\Omega)}^2
& = \int_{-1}^1 \int_0 ^{2\pi} -\frac{\Psi \Psi_{\theta_\ep \theta_\ep}}{1-\gamma_\ep^2} + (1-\gamma_\ep^2)\Psi_{\gamma_\ep}^2  d\theta_\ep d\gamma_\ep \\
& = 2\pi \int_{-1}^{1} \sum_{k \neq 0} \frac{k^2 \Psi^k(\gamma_\ep)^2}{1-\gamma_\ep^2} + (1-\gamma_\ep^2) \left( \Psi^0_{\gamma_\ep}(\gamma_\ep)^2 + \sum_{k \neq 0}  \Psi^k_{\gamma_\ep}(\gamma_\ep)^2 \right) d \gamma_\ep.
\end{split}
\end{align}
On the other hand, we have
\begin{align*}
\frac{1}{2}\iint_\Omega g'(\psi_\epsilon)\psi^2 dxdy
& =  \int_{-1}^1 \int_0 ^{2\pi} \Psi^2(\theta_\ep, \gamma_\ep) d\theta_\ep d\gamma_\ep \\
& = 2 \pi \int_{-1}^1  \Psi^0(\gamma_\ep)^2 + \sum_{k \neq 0}  \Psi^k(\gamma_\ep)^2d\gamma_\ep \\
& = 2\pi (I + II).
\end{align*}
For the non-zero modes, it is obvious that
$$II = \int_{-1}^1 \sum_{k \neq 0}  \Psi^k(\gamma_\ep)^2 d\gamma_\ep
\leq \int_{-1}^1 \sum_{k \neq 0} \frac{k^2 \Psi^k(\gamma_\ep)^2}{1-\gamma_\ep^2} d\gamma_\ep
\leq C \|\nabla \psi\|_{L^2(\Omega)}^2.$$
For the zero mode part, we have
\begin{align*}
I & = \int_{-1}^1  \Psi^0(\gamma_\ep)^2 d\gamma_\ep \\
& = \int_{-1}^1 \left( \Psi^0(\gamma_\ep) - \Psi^0(0)\right)^2 d\gamma_\ep \\
& = \int_{-1}^1 \left(\int_0^\gamma \Psi^0_{\hat{\gamma_\ep}}(\hat{\gamma_\ep}) d\hat{\gamma_\ep} \right)^2 d\gamma_\ep \\
& = \int_{-1}^0 \left(\int_{\gamma_\ep}^0 \Psi^0_{\hat{\gamma_\ep}}(\hat{\gamma_\ep}) d\hat{\gamma_\ep} \right)^2 d\gamma_\ep +  \int_{0}^1 \left(\int_0^{\gamma_\ep} \Psi^0_{\hat{\gamma_\ep}}(\hat{\gamma_\ep}) d\hat{\gamma_\ep} \right)^2 d\gamma_\ep \\
& \leq \int_{-1}^0 \int_{\gamma_\ep}^0 (1-\hat{\gamma_\ep}^2) \Psi^0_{\hat{\gamma_\ep}}(\hat{\gamma_\ep})^2 d\hat{\gamma_\ep} \int_{\gamma_\ep}^0 \frac{1}{1-\hat{\gamma_\ep}^2}d\hat{\gamma_\ep} d \gamma_\ep \\
& \qquad \qquad \qquad \qquad +  \int_{0}^1 \int_0^{\gamma_\ep} (1-\hat{\gamma_\ep}^2) \Psi^0_{\hat{\gamma_\ep}}(\hat{\gamma_\ep})^2 d\hat{\gamma_\ep} \int_0^{\gamma_\ep} \frac{1}{1-\hat{\gamma_\ep}^2}d\hat{\gamma_\ep} d \gamma_\ep \\
& \leq \int_{-1}^0 (1-\hat{\gamma_\ep}^2) \Psi^0_{\hat{\gamma_\ep}}(\hat{\gamma_\ep})^2 d\hat{\gamma_\ep} \int_{-1}^0  \int_{\gamma_\ep}^0 \frac{1}{1-\hat{\gamma_\ep}^2}d\hat{\gamma_\ep} d \gamma_\ep \\
& \qquad \qquad \qquad \qquad +   \int_0^{1} (1-\hat{\gamma_\ep}^2) \Psi^0_{\hat{\gamma_\ep}}(\hat{\gamma_\ep})^2 d\hat{\gamma_\ep} \int_{0}^1 \int_0^{\gamma_\ep} \frac{1}{1-\hat{\gamma_\ep}^2}d\hat{\gamma_\ep} d \gamma_\ep  \\
& = \int_{-1}^{1} (1-\hat{\gamma_\ep}^2) \Psi^0_{\hat{\gamma_\ep}}(\hat{\gamma_\ep})^2 d\hat{\gamma_\ep} \int_{0}^1 \int_0^{\gamma_\ep} \frac{1}{1-\hat{\gamma_\ep}^2}d\hat{\gamma_\ep} d \gamma_\ep  \\
&= \ln(2) \int_{-1}^{1} (1-\hat{\gamma_\ep}^2) \Psi^0_{\hat{\gamma_\ep}}(\hat{\gamma_\ep})^2 d\hat{\gamma_\ep} \\
& \leq C \|\nabla \psi\|_{L^2(\Omega)}^2.
\end{align*}
(2) is obtained by \eqref{gradient-psi} and (1).\end{proof}
\fi

For $0<\epsilon<1$, we define the projection
\begin{align}\label{P-ep}P_\ep \psi := \frac{\iint_\Omega g'(\psi_\ep)\psi dxdy}{\iint_\Omega g'(\psi_\ep) dxdy}={\iint_\Omega g'(\psi_\ep)\psi dxdy\over8\pi},\quad\psi \in \tilde{X}_\ep,\end{align}
 and
 %in the new variables $(\theta_\ep, \gamma_\ep),$ we define
\begin{align}\label{P-ep-new-variables}
\tilde P_\ep\Psi :=\frac{\iint_{\tilde \Omega}  \Psi d \theta_\ep d \gamma_\ep}{\iint_{\tilde \Omega}   d \theta_\ep d \gamma_\ep}= \frac{\iint_{\tilde \Omega}  \Psi d \theta_\ep d \gamma_\ep}{4\pi},\quad \Psi \in \tilde{Y}_\ep.\end{align}

\begin{Corollary}\label{P-welldefined}
The projections $P_\ep $
 and
$\tilde P_\ep$ are well-defined. Moreover,
$P_\ep \psi = \tilde P_\ep\Psi$
for  $\psi \in \tilde{X}_\ep$ and $\Psi \in \tilde{Y}_\ep$ such that $\psi(x,y) = \Psi(\theta_\ep, \gamma_\ep)$.
\end{Corollary}
\begin{proof}
The projection $\tilde P_\ep$ is the same one with  $\tilde P_0$ in \eqref{def-tilde-P0-Psi}. Let $\psi \in \tilde{X}_\ep$ and $\Psi \in \tilde{Y}_\ep$ such that $\psi(x,y) = \Psi(\theta_\ep, \gamma_\ep)$. Then $\tilde P_\ep$ is well-defined and $|\tilde P_\epsilon\Psi|\leq C \|\Psi\|_{\tilde Y_\epsilon}$ by Lemma
\ref{Poincare ineqalities-new-variable0} (2).
By \eqref{Jacobian of the transformation-ep},  $P_\ep \psi = \tilde P_\ep\Psi$ follows directly from  the definitions of $P_\ep$ and $\tilde P_\ep$. Then we have by \eqref{Psi-psi-norm-eq} that
\begin{align}\label{projection-controlled by-X-ep}
|P_\epsilon\psi|=|\tilde P_\epsilon\Psi|\leq C \|\Psi\|_{\tilde Y_\epsilon}=C \| \psi\|_{\tilde X_\epsilon}.
\end{align}
\end{proof}

\if0
\begin{proof}
Since $$\iint_\Omega g'(\psi_\ep) dxdy = \int_{-1}^{1} \int_0^{2\pi} 2 d\theta_\ep d \gamma_\ep = 8\pi,$$
we have
$$P_\ep \psi = \frac{1}{8\pi} \iint_\Omega g'(\psi_\ep)\psi dxdy = \frac{1}{4\pi}\int_{-1}^1 \int_{0}^{2\pi} \Psi(\theta_\ep, \gamma_\ep) d \theta_\ep d \gamma_\ep$$
and
\begin{align}\label{Pep}
\begin{split}
P_\ep \psi
& = \frac{1}{8\pi} \iint_\Omega g'(\psi_\ep)\psi dxdy\\
& \leq \frac{1}{8\pi} \left(\iint_\Omega g'(\psi_\ep)\psi^2 dxdy\right)^{1/2} \left( \iint_\Omega g'(\psi_\ep)dxdy \right)^{1/2} \\
& \leq C \|\nabla \psi\|_{L^2(\Omega)} \\
& < + \infty.
\end{split}
\end{align}
\end{proof}
\fi

Next,  we give the Poincar\'e inequality II for $0<\epsilon<1$.
\begin{lemma}[Poincar\'e inequality II-$\ep$]\label{poincare2ep}
$(1)$ For any $\Psi \in \tilde{Y_\ep}$,
we have
\begin{align*}
\iint_{\tilde \Omega} (\Psi -\tilde P_\epsilon\Psi)^2 d \theta_\ep d\gamma_\ep  \leq C \iint_{\tilde \Omega}\left({1\over1-\gamma_\ep^2}|\Psi_{\theta_\ep}|^2+(1-\gamma_\ep^2)|\Psi_{\gamma_\ep}|^2\right)d \theta_\ep d\gamma_\ep.
\end{align*}

$(2)$ For any $\psi \in \tilde{X}_\ep$,
we have
\begin{align}\label{Poincare inequality II-ep22}
\iint_\Omega g'(\psi_\epsilon)(\psi - P_\epsilon\psi)^2 dxdy  \leq C \|\nabla \psi\|_{L^2(\Omega)}^2.
\end{align}


\end{lemma}
\begin{proof}
(1)  follows from Lemma \ref{Poincare ineqalities-new-variable0} (3). By \eqref{Jacobian of the transformation-ep}, \eqref{Psi-psi-norm-eq} and Corollary \ref{P-welldefined},  we infer that   (2) is a restatement of  (1) in the original variables $(x,y)$.
\end{proof}
By Lemma \ref{hilbert-ep} (2) and  the Poincar\'e inequality I-$\ep$ \eqref{Poincare inequality I-ep22}, one can prove the existence and uniqueness of solutions in $\tilde X_\ep$ to the Poisson equation $-\Delta \psi = \omega\in X_\ep$ in the weak sense. The proof is similar to
Lemma  \ref{1-1correspond}, and we omit it.

\begin{lemma}\label{1-1correspond-ep}
For any $\omega \in X_\ep$, the Poisson equation
\begin{align*}
-\Delta \psi = \omega
\end{align*}
has a unique weak solution in $\tilde{X}_\ep$.
\end{lemma}
Recall that  $L_\ep$ and $X_\epsilon$ are defined in \eqref{J-ep-J-ep-def}-\eqref{spaceXep}, and the corresponding quadratic form for $L_\ep$ is
\begin{align*}
 \langle L_\epsilon\omega,\omega\rangle=\iint_\Omega\left(\frac {|\omega|^2} {g'(\psi_\epsilon)} - (-\Delta)^{-1}\omega\omega \right) dxdy,\quad\omega\in X_\epsilon.
\end{align*}
In view of Lemmas \ref{poincare1ep} (2) and \ref{1-1correspond-ep}, one can prove that $\langle L_\epsilon\cdot,\cdot\rangle$ is bounded on $ X_\epsilon$ by a similar way as Lemma \ref{Lbounded}.
\begin{lemma}\label{Lbounded-ep}
For any $\omega_1,\omega_1 \in X_\ep$, we have
$\langle L_\ep \omega_1, \omega_2 \rangle=\langle  \omega_1, L_\ep\omega_2 \rangle < C\|\omega_1\|_{X_\epsilon}\|\omega_2\|_{X_\epsilon}$.
\end{lemma}
\subsubsection{Reduction of the eigenvalue problems from Kelvin-Stuart vortex to hyperbolic tangent shear flow}

Define two elliptic operators
 \begin{align}\label{tilde-A-ep-A-ep}
 \tilde{A}_\ep=-\Delta-g'(\psi_\ep)(I - P_\ep): \tilde{X}_\ep \rightarrow \tilde{X}_\ep^*,\quad\quad
 A_\ep =-\Delta -g'(\psi_\ep):\tilde{X}_\ep \rightarrow \tilde{X}_\ep^*.
\end{align}
Then the corresponding quadratic forms
\begin{align*}
 \langle\tilde{A}_\ep\psi,\psi\rangle=&\iint_\Omega\left(|\nabla\psi|^2-g'(\psi_\ep)(\psi - P_\ep\psi)^2\right)dxdy,\\
 \langle A_\ep \psi,\psi\rangle=&\iint_{\Omega}\left(|\nabla \psi|^2-g'(\psi_\ep)|\psi|^2\right)dxdy,
\end{align*}
are bounded and symmetric on $\tilde{X}_\ep$ by the Poincar\'e inequalities I-$\ep$ \eqref{Poincare inequality I-ep22}, II-$\ep$ \eqref{Poincare inequality II-ep22}.
Then similar to \eqref{tilde A0-A0}, we have
\begin{align*}
\langle\tilde  A_\ep \psi,\psi\rangle=\langle A_\ep \psi,\psi\rangle+8\pi(P_\ep \psi)^2,\quad\psi\in \tilde X_\ep.
\end{align*}
Thus,
\begin{equation*}
n^{\leq0}(\tilde A_\ep)\leq n^{\leq0}(A_\ep),\quad n^{-}(\tilde A_\ep)\leq n^{-}(A_\ep).
\end{equation*}
%The operator $A_\ep$ without projection and its quadratic form $\langle A_\ep\cdot,\cdot\rangle$  are useful in our study on nonlinear stability of the steady states.
By means of Lemmas \ref{poincare2ep} (2) and \ref{1-1correspond-ep}, we have the following result by a similar argument to Lemma \ref{equal-indices0}.
%Then we show that the study  on the dimensions of kernel and negative subspaces of the quadratic form $\langle L_0\cdot,\cdot\rangle$
%defined in \eqref{L0-quadratic form}
 % could be reduced to the corresponding dimensions for $ \langle\tilde{A}_0\cdot,\cdot\rangle$.

\begin{lemma}\label{equal-indices} Let $0<\epsilon<1$. Then
$$\dim \ker (\tilde{A}_\ep) = \dim \ker (L_\ep), \quad n^-(\tilde{A}_\ep) = n^-(L_\ep).$$
\end{lemma}

To compute $n^-(\tilde{A}_\ep)$, we also need the compact embedding results.

\begin{lemma}\label{compact2P-new-variable-ep} Let $0<\ep<1$.
$(1)$ $\tilde Y_\ep$ is compactly embedded in $L^2(\tilde \Omega)$ and
\begin{equation*}
 \tilde Z_{\ep}:=\left\{\Psi\bigg|\iint_{\tilde \Omega}|\Psi-\tilde P_\ep\Psi|^2d\theta_\ep d\gamma_\ep<\infty\right\},
  \end{equation*}
respectively.

$(2)$  $\tilde X_\ep$ is compactly embedded in $L_{g'(\psi_\ep)}^2(\Omega)$ and
 \begin{equation*}
 Z_{\ep}:=\left\{\psi\bigg|\iint_{\Omega}g'(\psi_\ep)|\psi-P_\ep\psi|^2dxdy<\infty\right\},
  \end{equation*}
respectively.
\end{lemma}
\begin{proof}
(1) is equivalent to Lemma \ref{compact2P-new-variable0}. (2) is a consequence of (1), \eqref{Psi-psi-norm-eq} and Corollary \ref{P-welldefined}.
\end{proof}

By the compact embedding  $\tilde X_\ep\hookrightarrow Z_{\ep}$,   we can
inductively define $\lambda_{n}(\ep)$ as follows
\begin{align}\label{variational problem2-ep}
\lambda_n(\ep)=& \inf_{\psi \in \tilde X_\ep, (\psi, \psi_{i})_{Z_\ep} = 0, i = 1, 2, \cdots, n-1}{\iint_\Omega|\nabla\psi|^2dxdy\over\iint_\Omega g'(\psi_\ep)(\psi - P_\ep\psi)^2dxdy},\quad n\geq1,
%\\\label{variational problem2}
%=&\min_{\psi \in \tilde X_0, (\psi, \psi_{i})_{Z_0} = 0, i = 1, 2, \cdots, n-1}{\iint_\Omega|\nabla\psi|^2dxdy\over\iint_\Omega g'(\psi_0)(\psi - P_0\psi)^2dxdy},
\end{align}
where the  infimum for $\lambda_i(\ep)$ is attained at $\psi_{i}\in \tilde X_\ep$ and $\iint_\Omega g'(\psi_\ep)(\psi_{i} - {P_\ep}\psi_{i})^2 dxdy = 1$, $1\leq i \leq n-1$.
By computing the 1-order variation of the functional $G_\ep(\psi)={\iint_\Omega|\nabla\psi|^2dxdy\over\iint_\Omega g'(\psi_\ep)(\psi - P_\ep\psi)^2dxdy}$ at $\psi_{_n}$,  we have
\begin{align*}
&\frac{d}{d \tau} G_\ep(\psi_{n} + \tau \psi)|_{\tau = 0} = 2\iint_{\Omega} \left(-\Delta\psi_n - \lambda_n(\epsilon)g'(\psi_\ep)(\psi_n - P_\ep\psi_n)\right)\psi dxdy\\
=&2\iint_{\tilde\Omega}\left(-{1\over1-\gamma_\ep^2}\pa_{\theta_\ep}^2\Psi_n-\pa_{\gamma_\ep}\left((1-\gamma_\ep^2)\pa_{\gamma_\ep}\Psi_n\right)
-2\lambda_n(\epsilon)(\Psi_n-\tilde P_\ep\Psi_n)\right)\Psi d\theta_\ep d\gamma_\ep
\end{align*}
for $\psi\in \tilde X_\ep$ and $\Psi\in \tilde Y_\ep$ with $\psi(x,y)=\Psi(\theta_\ep,\gamma_\ep)$, where $\Psi_n(\theta_\ep,\gamma_\ep)=\psi_n(x,y)$.
Since   $\widehat{\Psi}_0(0)=0$ for $\Psi\in \tilde Y_\ep$, we
derive the Euler-Lagrangian equation in the new variables
\begin{align}\label{elip-ep}
-{1\over1-\gamma_\ep^2}\pa_{\theta_\ep}^2\Psi-\pa_{\gamma_\ep}\left((1-\gamma_\ep^2)\pa_{\gamma_\ep}\Psi\right)
=2\lambda(\Psi-\tilde P_\ep\Psi)+a\delta(\gamma_\ep), \quad \Psi \in \tilde{Y}_\ep,
\end{align}
where $a\in\mathbb{R}$ is to be determined. By the definition of  $\tilde P_\ep$ in \eqref{P-ep-new-variables}, integrating \eqref{elip-ep} on $\tilde \Omega$, we have
$$
2\pi a=\iint_{\tilde \Omega}\left(-{1\over1-\gamma_\ep^2}\pa_{\theta_\ep}^2\Psi-\pa_{\gamma_\ep}\left((1-\gamma_\ep^2)\pa_{\gamma_\ep}\Psi\right)
-2\lambda(\Psi-\tilde P_\ep\Psi)\right) d\theta_\ep d\gamma_\ep=0\;\;\Longrightarrow \;\;a=0,
$$
and thus,  we get the eigenvalue problem
\begin{align}\label{eigenvalue problem-ep-new}
-{1\over1-\gamma_\ep^2}\pa_{\theta_\ep}^2\Psi-\pa_{\gamma_\ep}\left((1-\gamma_\ep^2)\pa_{\gamma_\ep}\Psi\right)
=2\lambda(\Psi-\tilde P_\ep\Psi), \quad \Psi \in \tilde{Y}_\ep,
\end{align}
which, in the original variables, is exactly
\begin{align}\label{eigenvalue problem-ep-original}
-\Delta \psi = \lambda g'(\psi_\ep)(\psi -  P_\ep\psi), \quad \psi \in \tilde{X}_\ep.
\end{align}
Noting that  the eigenvalue problem \eqref{eigenvalue problem-ep-new} is the same one as
\eqref{elip02-x-gamma}, we have the following conclusions by Theorem \ref{associate_ep0}.

\begin{Theorem}\label{associate_ep-new-variable-original-variable}
All the eigenvalues of the eigenvalue problem \eqref{eigenvalue problem-ep-new} are $\lambda_n  = \frac{n(n+1)}{2}, n\geq1$. For $n\geq1$, the eigenspace associated to $\lambda_n$ is  spanned by
\begin{align*}
 L_{n}(\gamma_\ep) - L_n(0), \quad  L_{n,k}(\gamma_\ep)\cos(k\theta_\ep), \quad L_{n,k}(\gamma_\ep)\sin(k\theta_\ep), \quad  1 \leq k\leq n.
 \end{align*}
Consequently, all the eigenvalues of the associated eigenvalue problem \eqref{eigenvalue problem-ep-original} are $\lambda_n  = \frac{n(n+1)}{2}, n\geq1$. For $n\geq1$, the eigenspace associated to $\lambda_n$ is  spanned by
\begin{align*}
& L_{n}(\gamma_\ep(x,y)) - L_n(0), \quad  L_{n,k}(\gamma_\ep(x,y))\cos(k\theta_\ep(x,y)), \\
 & L_{n,k}(\gamma_\ep(x,y))\sin(k\theta_\ep(x,y)), \quad  1 \leq k\leq n,
 \end{align*}
where $\gamma_\ep(x,y)$ and $\theta_\ep(x,y)$ are defined in \eqref{transf1}-\eqref{transf2},  $L_{n,k}(\gamma_\ep)=(1-\gamma_\ep^2)^{k\over2}{d^k\over d\gamma_\ep^k}L_n(\gamma_\ep)$, and $L_n$ is the Legendre polynomial of degree $n$.
\end{Theorem}
Then we get the kernel of  the operators $\tilde A_\ep$ and $A_\ep$, as well as  decompositions of $\tilde X_{\ep}$ associated to the two operators.

\begin{Corollary}\label{kernel of  the operator tilde A-ep and a decomposition of tilde Xep}
$(1)$ $\ker (\tilde A_\ep)={\rm{span}}\left\{\eta_\ep(x,y), \gamma_\ep(x,y), \xi_\ep(x,y)\right\}$.

$(2)$ Let $\tilde X_{\ep+}=\tilde X_\ep \ominus\ker (\tilde A_\ep)$. Then
\begin{align*}
\langle \tilde A_\ep \psi,\psi\rangle \geq {2\over3} \| \psi\|_{\tilde X_\ep}^2, \quad \quad \psi\in \tilde X_{\ep+}.
\end{align*}
\end{Corollary}

\begin{proof}
By means of  Theorem \ref{associate_ep-new-variable-original-variable} and \eqref{variational problem2-ep}, the proof is similar to Corollary \ref{kernel of  the operator tilde A0 and a decomposition of tilde X0}. Here, we used $\tilde P_\ep\eta_\ep={1\over 4\pi}\iint_{\tilde \Omega}\sqrt{1-\gamma_\ep^2}\sin(\theta_\ep)d\theta_\ep d\gamma_\ep=0,$ $\tilde P_\ep\gamma_\ep={1\over 4\pi}\iint_{\tilde \Omega}\gamma_\ep d\theta_\ep d\gamma_\ep=0$, and $\tilde P_\ep\xi_\ep={1\over 4\pi}\iint_{\tilde \Omega}\sqrt{1-\gamma_\ep^2}\cos(\theta_\ep)d\theta_\ep d\gamma_\ep=0$ by \eqref{P-ep-new-variables}.
\end{proof}
The decomposition of $\tilde X_{\ep}$ associated to $A_\ep$ will be used  in the study on nonlinear stability.
\begin{Corollary}\label{kernel of  the operator A-ep and a decomposition of tilde Xep}
$(1)$ $\ker ( A_\ep)=\ker (\tilde A_\ep)={\rm{span}}\left\{\eta_\ep(x,y), \gamma_\ep(x,y), \xi_\ep(x,y)\right\}$.

$(2)$ Let $\tilde X_{\ep+}$ be defined as above. Then
\begin{align*}
\langle  A_\ep \psi,\psi\rangle \geq C_0 \| \psi\|_{\tilde X_\ep}^2, \quad \quad \psi\in \tilde X_{\ep+}
\end{align*}
for some $C_0>0$.
\end{Corollary}
\begin{proof}
\if0Thanks to \eqref{A-ep-tilde-A-ep} and Corollary \ref{kernel of  the operator tilde A-ep and a decomposition of tilde Xep}, the proof is similar to Corollary \ref{kernel of  the operator A0 and a decomposition of tilde X0} using the new variables $(\theta_\ep,\gamma_\ep)$.
\fi
\if0
Since $P_0\psi=0$ for any $\psi\in\ker ( A_0)$, we have by \eqref{tilde A0-A0} that $\ker (\tilde A_0)\subset \ker ( A_0)$. To show (1)-(2), it suffices to prove \eqref{A0psi}. In fact, by \eqref{p0-psi-estimates-2} and \eqref{tilde A0-A0} we have
\begin{align*}
\langle\tilde  A_0 \psi,\psi\rangle=\langle A_0 \psi,\psi\rangle+8\pi(P_0 \psi)^2\leq\langle A_0 \psi,\psi\rangle+\iint_\Omega g'(\psi_0)|\psi|^2dxdy.
\end{align*}
Thus, we infer from \eqref{sec-eigenvalue1}-\eqref{sec-eigenvalue2} that
\begin{align*}
\langle A_0 \psi,\psi\rangle\geq
\langle\tilde  A_0 \psi,\psi\rangle-\iint_\Omega g'(\psi_0)|\psi|^2dxdy\geq {2\over3} \| \psi\|_{\tilde X_0}^2-{1\over3} \| \psi\|_{\tilde X_0}^2={1\over3} \| \psi\|_{\tilde X_0}^2.
\end{align*}
\fi
Define the quadratic form
 \begin{align*}
 \langle\mathscr{A}_\ep\Psi,\Psi\rangle=\iint_{\tilde\Omega}\left({|\partial_{\theta_\ep}\Psi|^2\over 1-\gamma_\ep^2}+(1-\gamma_\ep^2)|\partial_{\gamma_\ep}\Psi|^2-2|\Psi|^2\right)d\theta_\ep d\gamma_\ep,\quad \Psi\in\tilde{Y}_\ep,
 \end{align*}
 where $\ep\in[0,1)$. Note that $\langle\mathscr{A}_\ep\Psi,\Psi\rangle=\langle{A}_\ep\psi,\psi\rangle$ for   $\psi \in \tilde{X}_\ep$ and $\Psi \in \tilde{Y}_\ep$ such that $\psi(x,y) = \Psi(\theta_\ep, \gamma_\ep)$, where $\ep\in[0,1)$.
 By Corollary \ref{kernel of  the operator A0 and a decomposition of tilde X0}, $\ker(\mathscr{A}_0)=\text{span}\{\gamma_0,\sqrt{1-\gamma_0^2}\cos(x),$ $\sqrt{1-\gamma_0^2}\sin(x)\}$, and $\langle\mathscr{A}_0\Psi,\Psi\rangle\geq C_0\|\Psi\|_{\tilde Y_0}$ for $\Psi\in \tilde Y_{0+}$, where $\tilde Y_{0+}=\tilde Y_0 \ominus\ker (\mathscr{A}_0)$. Thus, we  have $\ker(\mathscr{A}_\ep)=\text{span}\{\gamma_\ep,\sqrt{1-\gamma_\ep^2}\cos(\theta_\ep),$ $\sqrt{1-\gamma_\ep^2}\sin(\theta_\ep)\}$, and $\langle\mathscr{A}_\ep\Psi,\Psi\rangle\geq C_0\|\Psi\|_{\tilde Y_\ep}$ for $\Psi\in \tilde Y_{\ep+}$, where $\tilde Y_{\ep+}=\tilde Y_{\ep} \ominus\ker (\mathscr{A}_{\ep})$ and $\ep\in(0,1)$. This proves (1)-(2).
\end{proof}

\if0
\begin{lemma}\label{eigen-ep}
The eigenvalue problem \eqref{elipg0}
is equivalent to
\begin{align}\label{eigen_proj}
 - \frac{\partial}{\partial \gamma_\ep} \left( (1-\gamma_\ep^2) \frac{\partial \Psi}{ \partial \gamma_\ep}  \right) -\frac{1}{1-\gamma_\ep^2}\frac{\partial ^2 \Psi}{\partial \theta_\ep ^2}= 2 \lambda (\Psi - \tilde{P}_\ep \Psi), \quad  \Psi \in \tilde{Y}_\ep,
\end{align}
where
$$\tilde{P}_\ep \Psi = \frac{1}{4\pi}\int_{-1}^1 \int_{0}^{2\pi} \Psi(\theta_\ep, \gamma_\ep) d \theta_\ep d \gamma_\ep,$$
and $$\tilde{Y}_\ep = \left\{ \Psi | \int_{-1}^1\int_{0}^{2\pi}{1\over1-\gamma_\ep^2}|\Psi_{\theta_\ep}|^2+(1-\gamma_\ep^2)|\Psi_{\gamma_\ep}|^2d \theta_\ep d\gamma_\ep< \infty \text{ and } \Psi^0( 0)= 0 \right\}.$$
All the eigenvalues of this system are $\{\lambda_n = \frac{n(n+1)}{2}\}_{n=1}^\infty$. Each $\lambda_n$ is associated with exactly $2n + 1$ corresponding eigenfunctions
\begin{align}\label{eigenfuncg0}
L_{n}(\gamma_\ep)- L_{n}(0), \quad L_{n,k}(\gamma_\ep) \cos(k\theta_\ep), \quad L_{n,k}(\gamma_\ep)\sin(k\theta_\ep), \quad  1 \leq k\leq n,
\end{align}
where $L_{n,k}(\gamma_\ep) = (1-\gamma_\ep^2)^{k/2}L^{(k)}_n(\gamma_\ep)$ and $L^{(k)}_n(\gamma_\ep)$ is the $k$th derivative of the Legendre Polynomial $L_{n}(\gamma_\ep)$ for $n \geq 1 $.
\end{lemma}
\begin{proof}
Using equation \eqref{laplacian}, we have that the eigenvalue problem
$$\Delta \psi = \lambda g'(\psi_\ep)(I-P_\ep)\psi, \quad \psi \in \tilde{X}_\ep$$ in the $(x,y)$ coordinate is equivalent to equation \eqref{eigen_proj} in the $(\theta_\ep, \gamma_\ep)$ coordinate.
By expanding $\Psi(\theta_\ep, \gamma_\ep)$ in the following Fourier series form
$$\Psi(\theta_\ep, \gamma_\ep) = \Psi^0(\gamma_\ep) + \sum_{k\neq 0}e^{ik\theta_\ep} \Psi^k(\gamma_\ep),$$
and plugging it into \eqref{eigen_proj}, we can obtain equations parallel to equations \eqref{eigenvalue problem for 0 mode} and \eqref{modekk}, which give us the eigenvalues $\lambda_n$ and eigenfunctions in \eqref{eigenfuncg0}.
To prove that $\{\lambda_n\}_{n=1}^\infty$ are all the eigenvalues and that each $\lambda_n$ is associated with exactly $2n+1$ eigenfunctions, the steps are the same as the arguments in Lemma \ref{associate_ep0}.
\end{proof}
\fi
\begin{remark}\label{X-ep-X0-A-ep-neg}
In the definition of $\tilde X_\epsilon$, if we replace the condition $\widehat\Psi_0(0)=0$ by $\widehat \psi_0(0)=0$  as in $\tilde X_0$ for $\ep\in(0,1)$, then $n^-(A_\ep)\geq1$. In fact,
$\partial_\ep\psi_\ep\not\in \tilde X_\ep$ since
\begin{align*}
(\widehat{\partial_\ep\psi_\ep})_{\,0}(0)={1\over 2\pi}\int_0^{2\pi}\partial_\ep\psi_\ep(x,0)dx={1\over 2\pi}\int_0^{2\pi}\left({\epsilon\over 1-\epsilon^2}+{\cos(x)\over 1+\ep\cos(x)}\right)dx={1\over \ep-\ep^3}\neq0
\end{align*}
for $\ep\in(0,1)$. This implies that $\partial_\ep\psi_\ep-c_\ep\in\tilde X_\ep$ for $c_\ep={1\over \ep-\ep^3}$. Then
\begin{align*}
&\langle A_\ep(\partial_\ep\psi_\ep-c_\ep),\partial_\ep\psi_\ep-c_\ep\rangle=\langle(-\Delta -g'(\psi_\ep))(\partial_\ep\psi_\ep-c_\ep),\partial_\ep\psi_\ep-c_\ep\rangle\\
=&\langle g'(\psi_\ep)c_\ep,\partial_\ep\psi_\ep-c_\ep\rangle=-c_\ep^2\iint_{\Omega}g'(\psi_\ep)dxdy<0,
\end{align*}
where we used $-\Delta\partial_\ep\psi_\ep=g'(\psi_\ep)\partial_\ep\psi_\ep$ and $\iint_{\Omega} g(\psi_\ep)dxdy=8\pi\Longrightarrow\iint_{\Omega} g'(\psi_\ep)\partial_\ep \psi_\ep dxdy=0$. Thus, $n^-(A_\ep)\geq1$.
\end{remark}
\subsection{The proof of linear stability of Kelvin-Stuart vortices}
Based on our solutions to  the eigenvalue problems \eqref{elip02} and \eqref{eigenvalue problem-ep-original}, we  prove linear stability of the hyperbolic tangent shear flow and the Kelvin-Stuart vortices for  co-periodic perturbations. The approach is to apply the following index formula for general linear Hamiltonian PDEs developed in \cite{lin2022instability}.

\begin{lemma}
\label{theorem-index}
Consider a linear Hamiltonian system
$$\partial_t \omega = JL\omega, \quad \omega \in X,$$
where $X$ is a real Hilbert space.
%Let $(\cdot, \cdot)$ denote the inner product on $X$ and $ \langle \cdot, \cdot \rangle$ the dual bracket between $X^*$ and $X$.
Assume that

{\rm \textbf{(H1)}} $J:X^{\ast} \supset D(J)  \rightarrow X$ is anti-self-dual.
% in the sense $J'= - J$, where $J'$ is the dual operator of $J$.

{\rm \textbf{(H2)}} $L:X\rightarrow X^{\ast}$ is bounded and self-dual. Moreover, there exists a decomposition of $X$ into the direct sum of three closed subspaces
$$X=X_{-}\oplus\ker L\oplus X_{+}, \quad n^-(L)= \dim X_- < \infty$$ satisfying

$\quad$ {\rm \textbf{(H2.a)}}
 $\left\langle L\omega ,\omega \right\rangle <0$ for all $\omega \in X_- \backslash \{0\}$;

$\quad$ {\rm \textbf{(H2.b)}} there exists $\delta >0$ such that
\[
\left\langle L\omega ,\omega \right\rangle \geq\delta \left\Vert \omega \right\Vert _{X}
^{2},\quad\forall \; \omega \in X_{+}.
\]


{\rm \textbf{(H3)}} $\dim\ker
L<\infty$.\\
Then
\begin{align}\label{index-formula}
k_r + 2k_c+2k_i^{\leq0}+k_0^{\leq0} = n^-(L),
\end{align}
where $k_r$ is the sum of algebraic multiplicities of positive eigenvalues of $JL$, $k_c$ is the sum of algebraic multiplicities of eigenvalues of $JL$ in the first quadrant, $k_i^{\leq 0}$ is the total number of non-positive dimensions of $\langle L\cdot, \cdot \rangle$ restricted to the generalized eigenspaces of pure imaginary eigenvalues of $JL$ with positive imaginary parts, and $k_0^{\leq 0}$ is the number of non-positive directions of $\langle L\cdot, \cdot \rangle$ restricted to the generalized kernel of $JL$ modulo $\ker L$.
\end{lemma}

Now  we are in a position to prove  Theorem \ref{main result1-co-periodic perturbations}.
%Theorem
 %\begin{Theorem}\label{linear}
 %The steady state $\omega_\epsilon$ is spectrally stable for all $\epsilon \in [0,1)$.
 %\end{Theorem}
 \begin{proof}[Proof of Theorem \ref{main result1-co-periodic perturbations}]
We check \textbf{(H1-3)} in Lemma \ref{theorem-index} and then  apply the index formula \eqref{index-formula-stuart}  to prove  spectral stability of $\omega_\ep$, $0\leq \ep<1$. Recall that $J_\epsilon$, $L_\epsilon$ and $X_\ep$ are defined in \eqref{J-ep-J-ep-def}-\eqref{spaceXep}.
First, we define the space $\hat L^2(\Omega)=\{\omega\in L^2(\Omega)|\iint_\Omega \sqrt{g'(\psi_\ep)}\omega dxdy=0\}$ and the isometry
$$S: L^2(\Omega) \rightarrow X_\ep, \quad S\omega = \sqrt{g'(\psi_\ep)}\omega.$$
 Since $g'(\psi_\ep)\cdot$ and $\vec{u}_\ep \cdot \nabla$ are commutative, and $\nabla \cdot \vec{u}_\ep = 0$,
\begin{align}\label{tilde-J-ep}\tilde{J_\ep} := S^{-1} J_\ep (S')^{-1} = -\vec{u}_\ep \cdot \nabla:(\hat L^2(\Omega))^*\supset D(\tilde{J_\ep}) \rightarrow \hat L^2(\Omega)\end{align}
is anti-self-dual, where
$$D(\tilde{J_\ep}) =  \left\{ \omega \in  (\hat L^2(\Omega))^* | (\vec{u}_\ep \cdot \nabla) \omega \in \hat L^2(\Omega) \text{ in the distribution sense} \right\}.$$
Then $J_\ep' = -J_\ep$, and thus, \textbf{(H1)} is satisfied.
%Next, for any $\omega_1, \omega_2 \in X_\ep$, we know that there exist $\psi_1, \psi_2 \in \tilde{X}_\ep$ such that
%\begin{align*}
%\langle L_\ep \omega_1, \omega_2 \rangle & = \iint_\Omega \frac{\omega_1 \omega_2}{g'(\psi_\ep)} - ((-\Delta)^{-1} \omega_1 ) \omega_2 dxdy \\
%& = \iint_\Omega \frac{\omega_1 \omega_2}{g'(\psi_\ep)} - \nabla \psi_1 \nabla \psi_2 dxdy \\
%& \leq \left( \iint_\Omega \frac{\omega_1^2}{g'(\psi_\ep)}dxdy  \right)^{1/2} \left( \iint_\Omega \frac{\omega_2^2}{g'(\psi_\ep)}dxdy  \right)^{1/2} - \iint_\Omega \nabla \psi_1 \nabla \psi_2 dxdy \\
%& \leq \|\omega_1\|_{X_\ep}\|\omega_1\|_{X_\ep} + \|\psi_1\|_{\tilde{X}_\ep}\|\psi_2\|_{\tilde{X}_\ep} < \infty,
%\end{align*}
By Lemmas  \ref{Lbounded} and \ref{Lbounded-ep},
 the  operator $L_\ep:X_\ep\to X_\ep^*$ is self-dual and bounded for $0\leq \ep<1$.

It follows from Corollaries  \ref{kernel of  the operator tilde A0 and a decomposition of tilde X0} and  \ref{kernel of  the operator tilde A-ep and a decomposition of tilde Xep} that
$$n^-(\tilde{A}_\ep) = 0, \quad \dim \ker (\tilde{A}_\ep) = 3 \quad\text{ for all } \epsilon \in [0,1),$$
and $\tilde X_\ep$ can be decomposed as  $\tilde X_\ep=\ker (\tilde A_\ep) \oplus\tilde X_{\ep+}$ such that
\begin{align}\label{tilde-A-ep-psi-psi-Xep+}
\langle \tilde A_\ep \psi,\psi\rangle \geq {2\over3} \| \psi\|_{\tilde X_\ep}^2, \quad \quad \psi\in \tilde X_{\ep+}.
\end{align}
Then Lemmas \ref{equal-indices0} and \ref{equal-indices} tell us
$$n^-(L_\ep) = n^-(\tilde{A}_\ep) = 0, \quad \dim \ker (L_\ep) = \dim \ker (\tilde{A}_\ep) = 3 \quad\text{ for all } \epsilon \in [0,1).$$
Thus,  \textbf{(H2.a)} and \textbf{(H3)} are satisfied.
Since $\ker (\tilde{A}_\ep)={\rm{span}}\left\{\eta_\ep(x,y), \gamma_\ep(x,y), \xi_\ep(x,y)\right\}$  for all  $\epsilon \in [0,1)$, the kernel of $L_\ep$ is given explicitly by
\begin{align}\label{ker-L-ep}
\ker (L_\ep)={\rm{span}}\left\{g'(\psi_\ep)\eta_\ep(x,y), g'(\psi_\ep)\gamma_\ep(x,y), g'(\psi_\ep)\xi_\ep(x,y)\right\}.
\end{align}
Noting that $n^-(L_\ep)=0$, we  decompose $X_\ep$ into
$$X_\ep = \ker L_\ep \oplus X_{\ep+}.$$
  To verify \textbf{(H2.b)}, let us first note that for any $\omega \in X_{\ep+}$, we have $\psi=(-\Delta)^{-1}\omega\in \tilde X_{\ep+}$. In fact, it follows from \eqref{ker-L-ep} that $\tilde \omega:=g'(\psi_\ep)\tilde \psi\in\ker (L_\ep)$ for any $\tilde\psi\in \ker(\tilde A_\ep)$, and thus,   $(\psi,\tilde \psi)_{\tilde X_\ep}=\iint_{\Omega}-\Delta\psi\tilde \psi dxdy=\iint_{\Omega}{\omega\tilde \omega\over g'(\psi_\ep)} dxdy=(\omega,\tilde \omega)_{X_\ep}=0$.
By a similar argument to
\eqref{L0omega-omega}, we infer from \eqref{tilde-A-ep-psi-psi-Xep+} that
$$\langle L_\ep \omega, \omega \rangle \geq \langle \tilde{A}_\ep \psi, \psi \rangle\geq {2\over3} \|\nabla \psi\|_{L^2(\Omega)}^2, \quad \omega\in X_{\ep+}.$$
So, we have
\begin{align}\nonumber
\langle L_\ep \omega, \omega \rangle & = \kappa \iint_\Omega\left( \frac{\omega^2}{g'(\psi_\ep)} - |\nabla \psi|^2 \right)dxdy + (1-\kappa)\langle L_\ep \omega, \omega \rangle \\\nonumber
& \geq \kappa \iint_\Omega\left( \frac{\omega^2}{g'(\psi_\ep)} - |\nabla \psi|^2 \right)dxdy + {2\over3}(1-\kappa)  \|\nabla \psi\|_{L^2(\Omega)}^2\\\label{positive-decom}
& \geq \kappa\iint_\Omega \frac{\omega^2}{g'(\psi_\ep)} dxdy=\kappa\|\omega\|_{X_\ep}^2, \quad\forall \; \omega \in X_{\ep+}
\end{align}
by choosing $\kappa > 0$ such that ${2\over3}(1-\kappa) > \kappa$. This verifies \textbf{(H2.b)}. Now by the index formula \eqref{index-formula-stuart}, we have
  $$k_{r,\ep} + 2k_{c,\ep}+2k_{i,\ep}^{\leq0}+k_{0,\ep}^{\leq0}  = n^-(L_\epsilon) = 0.$$
In particular, $$k_{r,\ep} = 2k_{c,\ep}= 0,$$
which implies that there exist no exponential unstable solutions to the linearized vorticity equation \eqref{hami}. Therefore, the steady solution $\omega_\ep$ is spectrally stable.
\end{proof}

\section{Linear instability for  multi-periodic perturbations}\label{multi-periodic-linear}
In this section, we prove the linear instability of Kelvin-Stuart cat's eyes flows for $2m\pi$-periodic perturbations with $m\geq2$.
\subsection{Parity decomposition in the $y$ direction and separable Hamiltonian structure}

Let $\Omega_m = \mathbb{T}_{2m\pi} \times \mathbb{R}$ for $m\geq2$.
As in  \eqref{hami} for co-periodic perturbations, the linearized equation around the Kelvin-Stuart vortex $\omega_\ep$ can be written as
the  Hamiltonian system
 \begin{equation}\label{hami-m}
 \partial_t \omega = J_{\epsilon,m} L_{\epsilon,m}\omega, \quad \omega \in X_{\ep,m},
 \end{equation}
 where
 \begin{align*}
 %\label{J-ep-J-ep-def-m}
 J_{\ep,m} = -g'(\psi_\epsilon)\vec{u}_\epsilon\cdot\nabla: X_{\ep,m}^* \supset D(J_{\epsilon,m}) \rightarrow X_{\ep,m}, \quad
 L_{\epsilon,m} = \frac {1} {g'(\psi_\ep)} - (-\Delta)^{-1}: X_{\ep,m} \rightarrow X_{\ep,m}^*,\end{align*}
 and
\begin{align*}
X_{\ep,m} = \left\{\omega\bigg| \iint_{\Omega_m} \frac{|\omega|^2}{g'_\epsilon(\psi_\epsilon)} dxdy < \infty, \iint_{\Omega_m} \omega dxdy = 0 \right\},\quad \epsilon\in[0,1).
\end{align*}
To understand the linear stability/instability of the Kelvin-Stuart vortices for  multi-periodic perturbations, we first try to compute the index $n^-(L_{\ep,m})$ as what we did  for  co-periodic perturbations. Unlike the co-periodic case, $n^-(L_{\ep,m})>0$ in the multi-periodic case. Thus, if we use a similar index formula 
 $$
k_{r,\ep,m} + 2k_{c,\ep,m}+2k_{i,\ep,m}^{\leq0}+k_{0,\ep,m}^{\leq0} = n^-(L_{\ep,m})
$$
  as \eqref{index-formula-stuart} in the co-periodic case,
%\eqref{index-formula-stuart}, 
we have to compute the two indices
$k_{i,\ep,m}^{\leq0}$ and $k_{0,\ep,m}^{\leq0}$ for $J_{\epsilon,m} L_{\epsilon,m}$, which involves a tough and tedious  study  on the pure imaginary eigenvalues of $J_{\epsilon,m} L_{\epsilon,m}$. Here, $k_{r,\ep,m}, k_{c,\ep,m}, k_{i,\ep,m}^{\leq0}, k_{0,\ep,m}^{\leq0}$ are the indices defined similarly as in \eqref{index-formula-stuart}.
To avoid such a difficult part, we  observe that $g'(\psi_\epsilon)\vec{u}_\epsilon\cdot\nabla$ is odd in $y$ and  $ g'(\psi_\epsilon)$ is even in $y$, which implies that $L_{\epsilon,m}$ maps  odd (even) functions in $y$ to  odd (even) functions in $y$, while  $J_{\epsilon,m}$ maps odd (even) functions in $y$ to even (odd) functions in $y$. Based on this observation, we find that the linearized equation \eqref{hami-m} has indeed a separable  Hamiltonian structure. To make it clear, we give some preliminaries. Define two space
\begin{align*}
X_{\ep, e} = \left\{ \omega \in X_{\ep,m} | \omega \text{ is even in }y \right\},\quad\text{and}\quad
X_{\ep, o} = \left\{ \omega \in X_{\ep,m} | \omega \text{ is odd in }y \right\}.\end{align*}
Then $X_{\ep,m}, X_{\ep, e}$ and  $X_{\ep, o}$ are  Hilbert spaces with the $\frac{1}{g'(\psi_\ep)}$-weighted $L^2$ inner product on $\Omega_m$, since they are closed subspaces of $L_{\frac{1}{g'(\psi_\ep)}}^2(\Omega_m)$. Without loss of generality, we denote the dual space of  $X_{\ep, o}$ (resp. $X_{\ep, e}$) restricted into the class of odd (resp. even) functions by $X_{\ep, o}^*$ (resp. $X_{\ep, e}^*$).
Based on above properties on $L_{\epsilon,m}$ and  $J_{\epsilon,m}$, we can define
\begin{align*}B_\ep &= -g'(\psi_\ep) \vec{u}_\ep \cdot \nabla : X_{\ep, o}^* \supset D(B_\ep) \rightarrow X_{\ep, e}, \\
 L_{\ep,o} &= \frac{1}{g'(\psi_\ep)} - (-\Delta)^{-1}: X_{\ep, o} \rightarrow X_{\ep, o}^* \quad\text{and}\quad
 L_{\ep,e} = \frac{1}{g'(\psi_\ep)} - (-\Delta)^{-1}: X_{\ep, e} \rightarrow X_{\ep, e}^*.
 \end{align*}
Here, $(-\Delta)^{-1}\omega$ is the unique  weak solution in $\tilde X_{\ep,o}$ or $\tilde X_{\ep,e}$ of $-\Delta\psi=\omega$ for $\omega\in X_{\ep, o} \text{ or }X_{\ep, e}$, see Lemma \ref{1-1correspond-Lbounded-m} (1).
Then the dual operator of $B_\ep$ is
$$B'_\ep = g'(\psi_\ep) \vec{u}_\ep \cdot \nabla : X_{\ep, e}^* \supset D(B'_\ep) \rightarrow X_{\ep, o}.$$
We decompose $\omega \in X_{\ep,m}$ as $\omega = \left( \begin{array}{c} \omega_1 \\ \omega_2 \end{array} \right) $ such that $\omega_1 \in X_{\ep, e}$ and $\omega_2 \in X_{\ep, o}$. Then the linearized equation \eqref{hami-m} can be written as  the following separable Hamiltonian system
\begin{align}\label{sep-hamiltonian}
\partial_t \left( \begin{array}{c} \omega_1 \\ \omega_2 \end{array} \right) = \left( \begin{array}{cc} 0 & B_\ep \\ -B'_\ep & 0 \end{array} \right)\left( \begin{array}{cc} L_{\ep,e} & 0 \\ 0 & L_{\ep,o} \end{array} \right) \left( \begin{array}{c} \omega_1 \\ \omega_2 \end{array} \right),
%= \mathbf{J}_{\ep,m} \mathbf{L}_{\ep,m} \left( \begin{array}{c} \omega_1 \\ \omega_2 \end{array} \right),
\end{align}
or
$$\partial_t \omega = \mathbf{J}_{\ep,m} \mathbf{L}_{\ep,m}  \omega,$$
where $\omega \in \mathbf{X}_{\ep,m} = X_{\ep, e} \times X_{\ep, o}$ and
\begin{align*}\mathbf{J}_{\ep,m} = \left( \begin{array}{cc} 0 & B_\ep \\ -B'_\ep & 0 \end{array} \right): \mathbf{X}_{\ep,m}^* \supset D(\mathbf{J}_{\ep,m}) \rightarrow \mathbf{X}_{\ep,m},\quad\mathbf{L}_{\ep,m} = \left( \begin{array}{cc} L_{\ep,e} & 0 \\ 0 & L_{\ep,o} \end{array} \right): \mathbf{X}_{\ep,m} \rightarrow \mathbf{X}_{\ep,m}^*.\end{align*}
One of the advantage of the separable Hamiltonian system is a  precise counting formula  of unstable modes, see the next lemma \cite{lin2020separable,lin2021linear}.



\begin{lemma}  \label{indice-theorem-sep}
Let $X$ and $Y$ be real Hilbert spaces. Consider a linear Hamiltonian system of the separable form
\begin{align}\label{sep-hamil}
\partial_t \left( \begin{array}{c} u \\ v \end{array} \right) = \left( \begin{array}{cc} 0 & B \\ -B' & 0 \end{array} \right)\left( \begin{array}{cc} L & 0 \\ 0 & A \end{array} \right) \left( \begin{array}{c} u \\ v \end{array} \right) = \mathbf{J}  \mathbf{L} \left( \begin{array}{c} u \\ v \end{array} \right),
\end{align}
where $u \in X$ and $v \in Y$.
Assume that
\begin{itemize}
\item[{\textbf{(G1)}}] The operator $B: Y^* \supset D(B) \rightarrow X$ and its dual operator $B': X^* \supset D(B') \rightarrow Y$ are densely defined and closed.
\item[{\textbf{(G2)}}] The operator $A: Y \rightarrow Y^*$ is bounded and self-dual. Moreover, there exist $\delta > 0$  and a closed subspace $Y_+ \subset Y$ such that
$$ Y = \ker A \oplus Y_+,  \quad \langle Au, u \rangle \geq \delta \|u\|_Y^2,\quad \forall\; u \in Y_+.$$
\item[{\textbf{(G3)}}] The operator $L: X \rightarrow X^*$ is bounded and self-dual, and there exists a decomposition of $X$ into the direct sum of three closed subspaces
$$X = X_- \oplus \ker L \oplus X_+, \quad \dim \ker L < \infty,  \quad n^-(L) = \dim X_- < \infty$$
satisfying
\begin{itemize}
\item[{\textbf{(G3.a)}}]
$\langle Lu, u \rangle < 0$ for all $u\in X_- \backslash \{0\}$;
\item[{\textbf{(G3.b)}}] there exists $\delta > 0$ such that
$$\langle Lu, u \rangle \geq \delta \|u\|_X^2, \quad\forall \; u \in X_+.$$
\end{itemize}
\item[{\textbf{(G4)}}] $\dim \ker L < \infty$ \text{and} $\dim \ker A < \infty$.
\end{itemize}
Then the operator $\mathbf{JL}$ generates a $C^0$ group $e^{t\mathbf{JL}}$ of bounded linear operators on $\mathbf{X} = X \times Y$ and there exists a decomposition
$$\mathbf{X} = E^u \oplus E^c \oplus E^s$$
of closed subspaces $E^{u,s,c}$ with the following properties:

{\rm(i)} $E^c, E^u$ and $ E^s$ are invariant under $e^{t\mathbf{JL}}$.

{\rm(ii)} $E^u (E^s)$ only consists of eigenvectors corresponding to positive (negative) eigenvalues of $\mathbf{JL}$ and
\begin{align}\label{index-formula-neg}
\dim E^u = \dim E^s = n^-\left(L|_{\overline{R(B)}} \right),
\end{align}
where $n^-\left(L|_{\overline{R(B)}} \right)$ denotes the number of negative modes of $\langle L\cdot, \cdot \rangle|_{{\overline{R(B)}} }$. If $n^-\left(L|_{\overline{R(B)}} \right) >0$, then there exists $M>0$ such that
\begin{align}\label{trichotomy}
|e^{t\mathbf{JL}}|_{E^s}| \leq Me^{-\lambda_ut},\quad t \geq 0; \quad |e^{t\mathbf{JL}}|_{E^u}| \leq Me^{\lambda_ut},\quad t \leq 0,
\end{align}
where $\lambda_u = \min \{ \lambda | \lambda \in \sigma(\mathbf{JL}_{E^u}) \} > 0.$

{\rm(iii)} The quadratic form $\langle \mathbf{L}\cdot, \cdot \rangle$ vanishes on $E^{u,s}$, i.e. $\langle \mathbf{L}\mathbf{u}, \mathbf{u} \rangle = 0$ for all $\mathbf{u} \in E^{u,s}$, but is non-degenerate on $E^u \oplus E^s$ and
$$E^c = \{ \mathbf{u} \in \mathbf{X} | \langle \mathbf{L}\mathbf{u}, \mathbf{v} \rangle = 0, \forall \mathbf{v} \in E^s \oplus E^u \}.$$
There exists $M > 0$ such that
\begin{align}\label{trichotomy2}|e^{t\mathbf{JL}}|_{E^c} | \leq M(1+t^2), \quad   t \in \mathbb{R}.\end{align}
\end{lemma}



Lemma \ref{indice-theorem-sep} reveals that under the assumptions $\textbf{(G1-4)}$, the solutions of \eqref{sep-hamil} is spectrally stable if and only if $ L|_{{\overline{R(B)}} }\geq0$. Moreover,  the number of unstable modes is  $n^-\left(L|_{\overline{R(B)}} \right)$. In addition, the exponential trichotomy estimates \eqref{trichotomy}-\eqref{trichotomy2} are useful in the  study of the nonlinear dynamics, including nonlinear instability and invariant manifolds, near an unstable steady state.



To prove  linear instability of Kelvin-Stuart vortices,
 we will apply the index formula \eqref{index-formula-neg} to the Hamiltonian system \eqref{sep-hamiltonian}  after verifying the assumptions $\textbf{(G1-4)}$ in Lemma \ref{indice-theorem-sep}.
To prove linear instability,
 it suffices to show that  $n^-\left(L_{\ep,e}|_{\overline{\text{R}(B_\ep)}}\right)>0$, the proof of which  is reduced to delicate constructions of test functions to an elliptic operator later.

%As a first step, we will verify the assumptions $\textbf{(G1-3)}$, and apply the Theorem above to prove the instability by showing that there are at least one unstable mode.

First, we show that  the Hamiltonian system \eqref{sep-hamiltonian} satisfies {\textbf{(G1)}} in Lemma \ref{indice-theorem-sep}.
Since  $(C_0^\infty(\Omega_m)/\mathbb{R})\cap X_{\ep, o}^*\subset D( B_\ep)$ and $(C_0^\infty(\Omega_m)/\mathbb{R})\cap X_{\ep, e}^*\subset D( B_\ep')$, we know that both $B_\ep$ and $B_\ep'$ are densely defined. To prove that they are closed operators, we first prove that the operator $\hat{J}_{\ep,m} = -g'(\psi_\ep) \vec{u}_\ep \cdot \nabla : \hat{X}_{\ep,m}^* \supset D(\hat{J}_{\ep,m}) \rightarrow \hat{X}_{\ep,m}$ with $\hat{X}_{\ep,m} = L^2_{\frac{1}{g'(\psi_\ep)}}(\Omega_m)$ is closed. To show this, by a similar argument to \eqref{tilde-J-ep}, we know that
 %$$S: L^2(\Omega) \rightarrow \hat{X}_\ep, \quad S\omega = g'(\psi_\ep)^{1/2} \omega$$ defines an isometry. As $f(\psi_\ep)\cdot$ and $u_\ep \cdot \nabla$ are commutative for any function $f$, we have
%$$\tilde{B}_\ep := S^{-1} \hat{B}_\ep (S^*)^{-1} = u_\ep \cdot \nabla : (L^2)^* \rightarrow L^2$$
$\hat{J}_{\ep,m}$
is anti-self-dual, (i.e. $\hat{J}_{\ep,m}'= -\hat{J}_{\ep,m}$), and thus, $\hat{J}_{\ep,m}$ is closed. Since $B_\ep$ and $B'_\ep$ are restrictions of  $\hat{J}_{\ep,m}$  to two closed subspaces of $\hat{X}_{\ep,m}$, we infer that  both $B_\ep$ and $B'_\ep$ are also closed operators, which can be verified directly by  Proposition 1 in Chapter 5 of \cite{Weidmann80}.
\if0
\begin{lemma}\label{closed-operator}
Let $H_1$, $H_2$ be two Hilbert spaces, and $T: H_1 \supset D(T) \rightarrow H_2$ be a closed operator. Suppose $\tilde{H}_1$ is a closed subspace of $H_1$ and the operator $\tilde{T} : \tilde{H}_1 \supset D(\tilde{T}) \rightarrow H_2$ is the operator $T$ restricting to $\tilde{H}_1$, i.e., $\tilde{T}f = Tf$ for all $f \in D(\tilde{T})$. Then $\tilde{T}$ is also closed.
\end{lemma}
\begin{proof}
From the first Proposition in Chapter 5 of \cite{weidmann2012linear}, we know that $T$ is closed if and only if the following holds: if $\{f_n\}$ is a sequence in $D(T)$ that is convergent in $H_1$ and the sequence $\{Tf_n\}$ is convergent in $H_2$, then we have $\lim f_n \in D(T)$ and $T(\lim f_n) = \lim Tf_n$. To prove $\tilde{T}$ is closed, suppose $\{f_n\}$ is a sequence in $D(\tilde{T}) = D(T) \bigcap \tilde{H}_1$, $\{f_n\}$ is convergent in $\tilde{H}_1$,  and $\{\tilde{T}f_n\}$ is convergent in $H_2$. Since $T$ is closed and $\{f_n\}$ is a sequence in $D(T)$ and converges in $H_1$ with $\{Tf_n\}$ converges in $H_2$, we have $\lim f_n \in D(T) \bigcap \tilde{H}_1 = D(\tilde{T})$ and $\tilde{T}(\lim f_n) = T(\lim f_n) = \lim T f_n = \lim \tilde{T} f_n$, from which we have that $\tilde{T}$ is closed.
\end{proof}
\fi

To confirm that system \eqref{sep-hamiltonian} satisfies {\textbf{(G2-4)}} in Lemma \ref{indice-theorem-sep}, we transform the operators $L_{\ep,o} $
 and  $L_{\ep,e}$ of vorticity to elliptic operators of stream functions as what we did for the co-periodic case.  To this end, we
use the new variables $(\theta_\ep,\gamma_\ep)$ for $(x,y)\in[0,2\pi]\times\mathbb{R}$, and add the definitions $\theta_\ep(x,y)$ and $\gamma_\ep(x,y)$ for $(x,y)\in(2\pi,2m\pi]\times\mathbb{R}$ by $2\pi$-periodic extensions in the $\theta_\ep$ direction.
First, we  give the spaces of stream functions.
Let
\begin{align}\label{space-tilde-Xep-m}
\tilde{X}_{\ep,m} = \left\{\psi \bigg| \iint_{\Omega_m} |\nabla \psi|^2 dxdy <  \infty \text{ and } \int_{0}^{2m\pi}\psi(x,0){1\over1+\epsilon\cos(x)} dx = 0 \right\},
\end{align}
where $\ep\in[0,1)$.
By \eqref{gradient-psi}-\eqref{widehat-Psi-to-psi-1+ep-cos}, in the new variables, $\tilde{X}_{\ep,m}$ is equivalent  to the following space
\begin{align}\label{def-Y-ep-m}
\tilde{Y}_{\ep,m} = \left\{ \Psi \bigg| \iint_{\tilde \Omega_{m}}\left({1\over1-\gamma_\ep^2}|\Psi_{\theta_\ep}|^2+(1-\gamma_\ep^2)|\Psi_{\gamma_\ep}|^2\right)d \theta_\ep d\gamma_\ep< \infty \text{ and } \widehat{\Psi}_0(0)=0 \right\},
\end{align}
 where
% $\Psi(\theta_\ep, \gamma_\ep) = \psi(x(\theta_\ep, \gamma_\ep), y(\theta_\ep, \gamma_\ep))$ by the change of variables in \eqref{transf1} and \eqref{transf2} extended to
  $\tilde{\Omega}_{m} = \mathbb{T}_{2m\pi} \times [-1, 1]$.
Then we  define
$$\tilde{X}_{\ep, e} = \left\{ \psi \in \tilde{X}_{\ep,m} | \psi \text{ is even in }y \right\} \quad \text{and} \quad \tilde{X}_{\ep, o} = \left\{ \psi \in \tilde{X}_{\ep,m} | \psi \text{ is odd in }y \right\},$$
$$\tilde{Y}_{\ep, e} = \left\{ \Psi \in \tilde{Y}_{\ep,m} | \Psi \text{ is even in } \gamma_\ep  \right\} \quad \text{and} \quad \tilde{Y}_{\ep, o} = \left\{ \Psi \in \tilde{Y}_{\ep,m} | \Psi \text{ is odd in }\gamma_\ep \right\}.$$
Following the same steps in Lemmas \ref{Hilbert}, \ref{Hilbert-new variables-0} and \ref{hilbert-ep}, we can prove that $\tilde{X}_{\ep,m}$ is a Hilbert space under the inner product
$$(\psi_1, \psi_2)_{\tilde{X}_{\ep,m}} = \iint_{\Omega_m} \nabla \psi_1 \cdot \nabla \psi_2 dxdy, \quad \forall \;\psi_1, \psi_2 \in \tilde{X}_{\ep,m}.$$
Then $\tilde{X}_{\ep, e}$ and  $\tilde{X}_{\ep, o}$ are Hilbert spaces since they are closed subspaces of $\tilde{X}_{\ep,m}$.
 Correspondingly, $\tilde{Y}_{\ep,m}$ is also a Hilbert space under the inner product
 $$(\Psi_1, \Psi_2)_{\tilde{Y}_{\ep,m}} = \iint_{\tilde{\Omega}_{m}}  \left({1\over1-\gamma_\ep^2}(\Psi_1)_{\theta_\ep}(\Psi_2)_{\theta_\ep} +(1-\gamma_\ep^2)(\Psi_1)_{\gamma_\ep}(\Psi_2)_{\gamma_\ep}\right)d \theta_\ep d\gamma_\ep, \; \forall\; \Psi_1, \Psi_2 \in \tilde{Y}_{\ep,m},$$
and so are $\tilde{Y}_{\ep, e}$ and  $\tilde{Y}_{\ep, o}$.
Moreover,
\begin{align*}
(\psi_1, \psi_2)_{\tilde{X}_{\ep,m}}=(\Psi_1, \Psi_2)_{\tilde{Y}_{\ep,m}}
\end{align*}
for   $\psi_i \in \tilde{X}_{\ep,m}$ and $\Psi_i \in \tilde{Y}_{\ep,m}$ such that $\psi_i(x,y) = \Psi_i(\theta_\ep, \gamma_\ep)$, $i=1,2$.
Then we give the Poincar\'e inequality I for $\ep\in[0,1)$:
\begin{align}\label{Poincare inequality I-m}
\iint_{\Omega_m} g'(\psi_\epsilon)|\psi|^2 dxdy  \leq C \|\nabla \psi\|_{L^2(\Omega_m)}^2,\quad \psi \in \tilde{X}_{\ep,m},
\end{align}
and correspondingly, in the new variables,
\begin{align}\label{Poincare inequality I-m-new}
 \|\Psi\|_{L^2(\tilde \Omega_{m})}^2  \leq C \iint_{\tilde \Omega_m}\left({1\over1-\gamma_\ep^2}|\Psi_{\theta_\ep}|^2+(1-\gamma_\ep^2)|\Psi_{\gamma_\ep}|^2\right)d \theta_\ep d\gamma_\ep,\quad \Psi \in \tilde{Y}_{\ep,m}.
\end{align}
The proof of \eqref{Poincare inequality I-m}-\eqref{Poincare inequality I-m-new} is similar to Lemmas \ref{poincare1} and \ref{Poincare ineqalities-new-variable0} (1) for $\ep=0$, and similar to Lemma \ref{poincare1ep} for $\ep\in(0,1)$.
Let the projection be defined by
\begin{align}\label{def-projection-multi periodic}
P_{\ep,m} \psi = \frac{\iint_{\Omega_m}g'(\psi_\ep) \psi dxdy}{\iint_{\Omega_m}g'(\psi_\ep) dxdy} = \frac{1}{8m\pi}\iint_{\Omega_m}g'(\psi_\ep) \psi dxdy,\quad\psi\in \tilde X_{\ep,m},
\end{align}
and in the new variables, the corresponding projection is
$$\tilde{P}_{\ep,m} \Psi = \frac{\iint_{\tilde{\Omega}_{m}} \Psi d\theta_\ep d\gep}{\iint_{\tilde{\Omega}_{m}}1 d\theta_\ep d\gep} = \frac{1}{4m\pi}\iint_{\tilde{\Omega}_{m}}\Psi d\theta_\ep d\gep,\quad\Psi\in\tilde Y_{\ep,m}.$$
By \eqref{Poincare inequality I-m}-\eqref{Poincare inequality I-m-new}, $P_{\ep,m}$ and $\tilde{P}_{\ep,m}$ are well-defined on $\tilde X_{\ep,m}$ and $\tilde Y_{\ep,m}$, respectively. Then we give the
Poincar\'e inequality II for $\ep\in[0,1)$:
\begin{align}\label{Poincare inequality II-m}
\iint_{\Omega_m} g'(\psi_\epsilon)(\psi - P_{\epsilon,m}\psi)^2 dxdy  \leq C \|\nabla \psi\|_{L^2(\Omega_m)}^2, \quad \psi \in \tilde{X}_{\ep,m},
\end{align}
and correspondingly, in the new variables,
\begin{align}\nonumber
&\iint_{\tilde \Omega_{m}} (\Psi -\tilde P_{\epsilon,m}\Psi)^2 d \theta_\ep d\gamma_\ep  \\\label{Poincare inequality II-m-new}
\leq& C \iint_{\tilde \Omega_{m}}\left({1\over1-\gamma_\ep^2}|\Psi_{\theta_\ep}|^2+(1-\gamma_\ep^2)|\Psi_{\gamma_\ep}|^2\right)d \theta_\ep d\gamma_\ep,\;\Psi \in \tilde{Y}_{\ep,m}.
\end{align}
The proof of \eqref{Poincare inequality II-m}-\eqref{Poincare inequality II-m-new} is similar to Lemmas \ref{poincare2} and \ref{Poincare ineqalities-new-variable0} (3) for $\ep=0$, and similar to Lemma \ref{poincare2ep} for $\ep\in(0,1)$.
By the fact that $X_{\ep,o}$ (resp.  $X_{\ep,e}$) is a Hilbert space and the Poincar\'e inequality I \eqref{Poincare inequality I-m}, one can prove the following results by a similar argument to Lemmas \ref{1-1correspond} and \ref{Lbounded}.

\begin{lemma}\label{1-1correspond-Lbounded-m} Let $\ep\in[0,1)$.
$(1)$ For $\omega \in X_{\ep,o}$ (resp.  $X_{\ep,e}$), the Poisson equation
$-\Delta\psi=\omega$
has a unique weak solution in $\tilde{X}_{\ep,o}$ (resp.  $\tilde X_{\ep,e}$).

$(2)$
For  $\omega_1,\omega_2 \in X_{\ep,o}$, we have
$\langle L_{\ep,o} \omega_1, \omega_2 \rangle=\langle \omega_1, L_{\ep,o} \omega_2 \rangle \leq C\|\omega_1\|_{X_{\ep,o}}\|\omega_2\|_{X_{\ep,o}}.$

$(3)$
For  $\omega_1,\omega_2 \in X_{\ep,e}$, we have
$\langle L_{\ep,e} \omega_1, \omega_2 \rangle=\langle \omega_1, L_{\ep,e} \omega_2 \rangle \leq C\|\omega_1\|_{X_{\ep,e}}\|\omega_2\|_{X_{\ep,e}}.$
\end{lemma}
By Lemma \ref{1-1correspond-Lbounded-m} (2)-(3), both $L_{\ep,o}:X_{\ep, o}\rightarrow X_{\ep, o}^*$ and $L_{\ep,e}:X_{\ep, e}\rightarrow X_{\ep, e}^*$ are self-dual and bounded.

\subsection{Exact solutions to the associated eigenvalue problems  for the multi-periodic case}
Next, we consider the decomposition of $X_{\ep, o}$ and $X_{\ep, e}$ associated to $L_{\ep,o}$ and $L_{\ep,e}$, respectively.
Define the elliptic operators
$$\tilde{A}_{\ep,o} = -\Delta - g'(\psi_\ep)(I-P_{\ep,m}) =  -\Delta - g'(\psi_\ep): \tilde{X}_{\ep, o} \rightarrow \tilde{X}_{\ep, o}^*$$
and
$$\tilde{A}_{\ep,e} = -\Delta - g'(\psi_\ep)(I-P_{\ep,m}): \tilde{X}_{\ep, e} \rightarrow \tilde{X}_{\ep, e}^*,$$
where we used $P_{\ep,m}\psi=0$ for $\psi\in\tilde{X}_{\ep, o}$.
The dual space of  $\tilde X_{\ep, o}$ (resp. $\tilde X_{\ep, e}$)  restricted into the class of odd (resp. even) functions is denoted  by $\tilde X_{\ep, o}^*$ (resp. $\tilde X_{\ep, e}^*$).
Based on Lemma \ref{1-1correspond-Lbounded-m} and  \eqref{Poincare inequality II-m}, we  prove
\begin{align}\label{n-ker-o}
n^-(L_{\ep,o}) &= n^-(\tilde A_{\ep,o}), \quad \dim \ker(L_{\ep,o}) = \dim \ker(\tilde A_{\ep,o}),\\\label{n-ker-e}
n^-(L_{\ep,e}) & = n^-(\tilde A_{\ep,e}), \quad \dim \ker(L_{\ep,e}) = \dim \ker(\tilde A_{\ep,e})
\end{align}
by a similar way as  Lemma \ref{equal-indices0}. Similar to Lemmas \ref{compact2P}, \ref{compact2P-new-variable0} and \ref{compact2P-new-variable-ep},
 $\tilde Y_{\ep,m}$ is compactly embedded in $L^2(\tilde \Omega_{m})$ and
$$
 \tilde Z_{\ep,m}:=\left\{\Psi\bigg|\iint_{\tilde \Omega_{m}}|\Psi-\tilde P_{\ep,m}\Psi|^2d\theta_\ep d\gamma_\ep<\infty\right\},
 $$
respectively. Correspondingly,
 $\tilde X_{\ep,m}$ is compactly embedded in $L_{g'(\psi_\ep)}^2(\Omega_m)$ and
$$
 Z_{\ep,m}:=\left\{\psi\bigg|\iint_{\Omega_m}g'(\psi_\ep)|\psi-P_{\ep,m}\psi|^2dxdy<\infty\right\},
$$
respectively.
Thus,  we can
inductively define
\begin{align}\label{variational problem2-ep-m}
\lambda_n(\ep,m)=& \inf_{\psi \in \tilde X_{\ep,m}, (\psi, \psi_{i})_{Z_{\ep,m}} = 0, i = 1, 2, \cdots, n-1}{\|\psi\|_{\tilde{X}_{\ep,m}}^2\over\iint_{\Omega_m} g'(\psi_\ep)(\psi - P_{\ep,m}\psi)^2dxdy},\quad n\geq1,
%\\\label{variational problem2}
%=&\min_{\psi \in \tilde X_0, (\psi, \psi_{i})_{Z_0} = 0, i = 1, 2, \cdots, n-1}{\iint_\Omega|\nabla\psi|^2dxdy\over\iint_\Omega g'(\psi_0)(\psi - P_0\psi)^2dxdy},
\end{align}
where the  infimum for $\lambda_i(\ep,m)$ is attained at $\psi_{i} \in \tilde X_{\ep,m}$ and $\iint_{\Omega_m} g'(\psi_\ep)(\psi_{i} - {P_{\ep,m}}\psi_{i})^2 dxdy = 1$, $1\leq i \leq n-1$.
Then in the new variables,
\begin{align}\label{variational problem2-ep-m-new variables}
\lambda_n(\ep,m)
=&\inf_{\Psi \in \tilde Y_{\ep,m}, (\Psi, \Psi_{i})_{\tilde Z_{\ep,m}} = 0, i = 1, 2, \cdots, n-1}{\|\Psi\|_{\tilde{Y}_{\ep,m}}^2\over\iint_{\tilde \Omega_{m}}2|\Psi-\tilde P_{\ep,m}\Psi|^2d\theta_\ep d\gamma_\ep},\quad
n\geq1.
%\\\label{variational problem2}
%=&\min_{\psi \in \tilde X_0, (\psi, \psi_{i})_{Z_0} = 0, i = 1, 2, \cdots, n-1}{\iint_\Omega|\nabla\psi|^2dxdy\over\iint_\Omega g'(\psi_0)(\psi - P_0\psi)^2dxdy},
\end{align}
By a similar argument to \eqref{variational problem2-ep}-\eqref{eigenvalue problem-ep-original},  we arrive at the eigenvalue problem
\begin{align}\label{eigenvalue problem-ep-new-m}
-\pa_{\gamma_\ep}\left((1-\gamma_\ep^2)\pa_{\gamma_\ep}\Psi\right)-{1\over1-\gamma_\ep^2}\pa_{\theta_\ep}^2\Psi
=2\lambda(\Psi-\tilde P_{\ep,m}\Psi), \quad \Psi \in \tilde{Y}_{\ep,m},
\end{align}
which, in the original variables, is exactly
\begin{align}\label{eigenvalue problem-ep-original-m}
-\Delta \psi = \lambda g'(\psi_\ep)(\psi -  P_{\ep,m}\psi), \quad \psi \in \tilde{X}_{\ep,m}.
\end{align}
In the new variables $(\theta_\ep,\gamma_\ep)$, we use the Fourier expansion $\Psi(\theta_\ep,\gamma_\ep)=\sum_{k\in\mathbb{Z}}\widehat{\Psi}_{k}(\gamma_\ep)e^{i{k\over m} \theta_\ep}$ to separate the variables, and study the eigenvalue problem \eqref{eigenvalue problem-ep-new-m} for the $0$ mode and the non-zero modes, separately.  For the $0$ mode,
the eigenvalue problem  is
\begin{align}\label{eigenvalue problem for 0 mode-m}
-\left((1-\gamma_\ep^2)  \varphi'\right)' = 2 \lambda(\varphi-\hat{P}_{0}^\ep\varphi) \quad \text{on}\quad (-1,1),\quad\varphi\in \hat Y_{0}^\ep,
\end{align}
where $\hat{P}_{0}^\ep\varphi={1\over2}\int_{-1}^{1}\varphi(\gamma_\ep)d\gamma_\ep$  and
\begin{equation*}
\hat Y_0^\ep=\left\{\varphi\bigg|\int_{-1}^1(1-\gamma_\ep^2)|\varphi'(\gamma_\ep)|^2d\gamma_\ep<\infty\text{ and }\varphi(0)=0 \right\}.
\end{equation*}
%which is the same space $\hat Y_0$ in \eqref{def-hat-Y0}.
Since the eigenvalue problem \eqref{eigenvalue problem for 0 mode-m} for the $0$ mode  is the same one to \eqref{eigenvalue problem for 0 mode}, by applying Lemma
\ref{sol to eigenvalue problem},
all the eigenvalues  of the eigenvalue problem \eqref{eigenvalue problem for 0 mode-m} with corresponding eigenfunctions are as follows:
\begin{align}\label{0 mode-eigenvalue-eigenfunction-m}
\lambda_{n,0}={n(n+1)\over2},\quad\varphi_{n,0}(\gamma_\ep)= L_{n}(\gamma_\ep)-L_{n}(0),\quad n\geq1.
\end{align}







\if0
For  $\psi \in \tilde{X}_{\ep,m}$, let $\Psi \in \tilde{Y}_{\ep,m}$ such that $\Psi(\theta_\ep, \gep) = \psi(x(\theta_\ep, \gep), y(\theta_\ep, \gep))$. Let
$$\Psi(\theta_\ep, \gep) =  \sum_{k \in \mathbb{Z}} \Psi^k(\gep) e^{\frac{ik\theta_\ep}{m}}$$
where
$$\Psi^k(\gep) = \frac{1}{2m\pi} \int_0^{2m\pi} \Psi(\theta_\ep, \gep) e^{-\frac{ik\theta_\ep}{m}} d \theta_\ep.$$
\fi

The difference comes from the non-zero modes. For the $k$ mode, the eigenvalue problem \eqref{eigenvalue problem-ep-new-m}
is
\begin{equation}\label{eigenvalue problem2 non-zero modes varepsilon=0m}
-((1-\gamma_\ep^2)\varphi')'+{{k^2\over m^2}\over1-\gamma_\ep^2}\varphi =2\lambda\varphi \quad \text{on}\quad (-1,1),\quad\varphi\in \hat Y_1^\ep,
\end{equation}
where $k\neq0$ and
\begin{align}\label{Y-k-ep-def}
\hat Y_1^\ep=\left\{\varphi\bigg|\int_{-1}^1\left({1\over1-\gamma_\ep^2}|\varphi(\gamma_\ep)|^2+(1-\gamma_\ep^2)|\varphi'(\gamma_\ep)|^2\right)d\gamma_\ep
<\infty \right\},
\end{align}
which is the same space
 $\hat Y_1$ defined in \eqref{def-hat-Y1} if we replace the variable $\gamma_\ep$ by $\gamma$ in \eqref{Y-k-ep-def}.
To the best of our knowledge, the existing approach to solving the eigenvalue problem \eqref{eigenvalue problem2 non-zero modes varepsilon=0m} is via the hypergeometric functions directly, but it seems a tedious task to compute all the eigenvalues and corresponding eigenfunctions in this way. Our method is motivated as follows. For $m=2$ and $k=1$, we observe that $\varphi(\gamma_\ep)=(1-\gamma_\ep^2)^{1\over 4}$ and $\lambda={3\over 8}$ solve \eqref{eigenvalue problem2 non-zero modes varepsilon=0m}. Taking
$
\varphi=(1-\gamma_\ep^2)^{1\over4}\phi,
$
then $\phi$ solves
\begin{equation}\label{tran-m=2-equa}
(1-\gamma_\ep^2)\phi''-3\gamma_\ep\phi'+\left(-{3\over4}+2\lambda\right)\phi =0 \quad \text{on}\quad (-1,1),\quad\phi\in W_{1\over 2},
\end{equation}
where $W_{1\over 2}=\{\phi|(1-\gamma_\ep^2)^{1\over4}\phi\in\hat Y_1^\ep\}$.
Then $\phi=1$ and $\lambda={3\over 8}$ solve \eqref{tran-m=2-equa}. Moreover, $\phi=\gamma_\ep$ and $\lambda={15\over 8}$ also solve \eqref{tran-m=2-equa}. As in the co-periodic case, our perspective is that all the eigenfunctions for \eqref{tran-m=2-equa}  might be polynomials of $\gamma_\ep$. They are indeed polynomials of $\gamma_\ep$ after we find that \eqref{tran-m=2-equa} is exactly the Gegenbauer differential equation
\begin{equation}\label{tran-m=2-equa-g-2}
(1-\gamma_\ep^2)\phi''-(2\beta+1)\gamma_\ep\phi'+n(n+2\beta)\phi =0 \quad \text{on}\quad (-1,1)
\end{equation}
for $\beta=1$ in \eqref{tran-m=2-equa-g-2} and $\lambda={1\over2} \left(n^2+2n+{3\over4}\right)$, $n\geq0$, in \eqref{tran-m=2-equa}. All the solutions of \eqref{tran-m=2-equa-g-2} are given by  Gegenbauer polynomials.
To
 solve the eigenvalue problem \eqref{eigenvalue problem2 non-zero modes varepsilon=0m} for general $k\geq1$ and $m\geq2$,
we introduce the transformation
\begin{align}\label{transformation-multi-periodic perturbations}
\varphi=(1-\gamma_\ep^2)^{k\over2m}\phi.
\end{align}
Then \eqref{eigenvalue problem2 non-zero modes varepsilon=0m} is transformed to
\begin{equation}\label{eigenvalue problem2 non-zero modes varepsilon=0-transform}
(1-\gamma_\ep^2)\phi''-2\left({k\over m}+1\right)\gamma_\ep\phi'+\left(-{k^2\over m^2}-{k\over m}+2\lambda\right)\phi =0 \quad \text{on}\quad (-1,1),\quad\varphi\in W_{k\over m},
\end{equation}
where $W_{k\over m}=\{\phi|(1-\gamma_\ep^2)^{k\over2m}\phi\in\hat Y_1^\ep\}$. It is  well-known \cite{Suetin2001} that the
Gegenbauer polynomials
\begin{align}\label{Gegenbauer polynomials}
C_n^\beta(\gamma_\ep)={(-1)^n\over 2^n n!}{\Gamma(\beta+{1\over2})\Gamma(n+2\beta)\over\Gamma(2\beta)\Gamma(\beta+n+{1\over2})}(1-\gamma_\ep^2)^{-\beta+{1\over2}}{d^n\over d\gamma_\ep^n}\left((1-\gamma_\ep^2)^{n+\beta-{1\over2}}\right)
\end{align}
are solutions of
 the   Gegenbauer
 differential equations
\begin{equation}\label{Gegenbauer differential equation}
(1-\gamma_\ep^2)\phi''-(2\beta+1)\gamma_\ep\phi'+n\left(n+2\beta\right)\phi =0 \quad \text{on}\quad (-1,1),\quad\phi\in L_{\hat g_\beta}^2(-1,1),
\end{equation}
where $n\geq0$ and $\hat g_\beta(\gamma_\ep)=(1-\gamma_\ep^2)^{\beta-{1\over2}}$. Moreover, $\{C_n^\beta\}_{n=0}^\infty$ is a complete and  orthogonal basis of $ L_{\hat g_\beta}^2(-1,1)$ for $\beta>-{1\over2}$. Set
\begin{align*}
\beta\triangleq{k\over m}+{1\over2},\quad \lambda\triangleq{1\over2}\left({k^2\over m^2}+{k\over m}+n^2+{2nk\over m}+n\right)={1\over2}\left(n+{k\over m}\right)\left(n+{k\over m} +1\right),
\end{align*}
and then the two equations
in \eqref{Gegenbauer differential equation} and \eqref{eigenvalue problem2 non-zero modes varepsilon=0-transform} surprisingly coincide.
Furthermore, $(1-\gamma_\ep^2)^{k\over2m}C_n^\beta\in\hat Y_1^\ep$ for $n\geq0$.
In fact,
\begin{align}\nonumber
&\int_{-1}^1\left({1\over1-\gamma_\ep^2}(1-\gamma_\ep^2)^{k\over m}|C_n^\beta(\gamma_\ep)|^2+(1-\gamma_\ep^2)\left|\left((1-\gamma_\ep^2)^{k\over 2m}C_n^\beta(\gamma_\ep)\right)'\right|^2\right)d\gamma_\ep\\\nonumber
=&\int_{-1}^1(1-\gamma_\ep^2)^{{k\over m}-1}|C_n^\beta(\gamma_\ep)|^2d\gamma_\ep\\
&+\int_{-1}^1\left|-{k\over m}\gamma_\ep(1-\gamma_\ep^2)^{{k\over 2m}-{1\over2}} C_n^\beta(\gamma_\ep)+(1-\gamma_\ep^2)^{{k\over 2m}+{1\over2}}(C_n^\beta(\gamma_\ep))'\right|^2d\gamma_\ep
<\infty.\label{1-gammacnykep}
\end{align}
This implies that
\begin{align*}
&\varphi_{n,{k\over m}}(\gamma_\ep)\triangleq(1-\gamma_\ep^2)^{k\over2m}C_n^{{k\over m}+{1\over2}}(\gamma_\ep)\in\hat Y_1^\ep,\quad\lambda=\lambda_{n,{k\over m}}\triangleq{1\over2}\left(n+{k\over m}\right)\left(n+{k\over m} +1\right)
\end{align*}
solves \eqref{eigenvalue problem2 non-zero modes varepsilon=0m} for $n\geq0$.
Since $\{C_n^\beta\}_{n=0}^\infty$ is a complete and  orthogonal basis of $ L_{\hat g_\beta}^2(-1,1)$, and
\begin{align*}
\int_{-1}^1\hat g_\beta(\gamma_\ep) C_{n_1}^\beta (\gamma_\ep)C_{n_2}^\beta (\gamma_\ep) d\gamma_\ep=&\int_{-1}^1(1-\gamma_\ep^2)^{k\over m}C_{n_1}^\beta (\gamma_\ep)C_{n_2}^\beta (\gamma_\ep) d\gamma_\ep \\
=&\int_{-1}^1\varphi_{n_1,{k\over m}}(\gamma_\ep)\varphi_{n_2,{k\over m}}(\gamma_\ep)  d\gamma_\ep
\end{align*}
for $n_1,n_2\geq0$, we know that  $\{\varphi_{n,{k\over m}}\}_{n=0}^\infty$ is a complete and  orthogonal basis of $ L^2(-1,1)$.
Since $\hat Y_1^\ep$ is embedded in $ L^2(-1,1)$ by Lemma
\ref{Poincare inequalities compact embedding result new variables k mode}, we infer that
$\{\varphi_{n,{k\over m}}\}_{n=0}^\infty$ is a complete and  orthogonal basis of $\hat Y_1^\ep$ under the inner product of $ L^2(-1,1)$.
In summary, the eigenvalue problem \eqref{eigenvalue problem2 non-zero modes varepsilon=0m} is solved as follows.

\begin{lemma}\label{sol to eigenvalue problem non-zero modes varepsilon=0} Fix $m\geq2$ and $k\geq1.$ Then
all the eigenvalues  of the eigenvalue problem \eqref{eigenvalue problem2 non-zero modes varepsilon=0m} are $\lambda_{n,{k\over m}}={1\over2}\left(n+{k\over m}\right)\left(n+{k\over m} +1\right)$, $n\geq 0$. For $n\geq0$, the eigenspace associated to $\lambda_{n,{k\over m}}$ is $\text{span}\{\varphi_{n,{k\over m}}(\gamma_\ep)\}=\text{span}\{(1-\gamma_\ep^2)^{k\over2m}C_n^{{k\over m}+{1\over2}}(\gamma_\ep)\}$.
\end{lemma}

Combining \eqref{0 mode-eigenvalue-eigenfunction-m} and Lemma \ref{sol to eigenvalue problem non-zero modes varepsilon=0}, we solve
the eigenvalue problem \eqref{eigenvalue problem-ep-new-m} (and hence, \eqref{eigenvalue problem-ep-original-m}).
\begin{Theorem}\label{sol to eigenvalue problem varepsilon=0-pde} Fix $m\geq2$.

$(1)$
All the eigenvalues  of the eigenvalue problem \eqref{eigenvalue problem-ep-new-m} are
\begin{align}\label{km-eigenvalues}
{1\over2}n\left(n+1\right), &\quad n\geq1,\\\label{nonkm-eigenvalues}
{1\over2}\left(n+{i\over m}\right)\left( n+{i\over m}+1\right), &\quad 1\leq i\leq m-1,\; n\geq0.
\end{align} The corresponding eigenspaces are given as follows.
\if0
$(1)$ For $1\leq i\leq m-1$, the  eigenspace associated to the eigenvalue  ${i\over 2m}\left({i\over m}+1\right)$ is spanned by
 \begin{align*}
 (1-\gamma_\ep^2)^{i\over2m}C_0^{{i\over m}+{1\over2}}(\gamma_\ep)\cos\left({i\over m}\theta_\ep\right),\;\;(1-\gamma_\ep^2)^{i\over2m}C_0^{{i\over m}+{1\over2}}(\gamma_\ep)\sin\left({i\over m}\theta_\ep\right).
 \end{align*}
 \fi
 \begin{itemize}
 \item For $n\geq1$,
the eigenspace associated to the eigenvalue  ${1\over2}n\left( n+1\right)$ is spanned by
 \begin{align}\label{km-eigenfunctions}
L_{n}(\gamma_\ep)-L_{n}(0),\;\; L_{n,j}(\gamma_\ep)\cos(j\theta_\ep),\;\;
 L_{n,j}(\gamma_\ep)\sin(j\theta_\ep), \quad  1 \leq j\leq n.
 \end{align}

\item For $1\leq i\leq m-1$ and $n\geq0$,
the eigenspace associated to the eigenvalue  ${1\over2}\left(n+{i\over m}\right)$ $\left( n+{i\over m}+1\right)$ is spanned by
\begin{align}\nonumber
 &(1-\gamma_\ep^2)^{(n-j)m+i\over2m}C_j^{{(n-j)m+i\over m}+{1\over2}}(\gamma_\ep)\cos\left({(n-j)m+i\over m}\theta_\ep\right),\\\label{nonkm-eigenfunctions}
 &(1-\gamma_\ep^2)^{(n-j)m+i\over2m}C_j^{{(n-j)m+i\over m}+{1\over2}}(\gamma_\ep)\sin\left({(n-j)m+i\over m}\theta_\ep\right),\;\;0\leq j\leq n.
 \end{align}
 \end{itemize}

$(2)$
All the eigenvalues  of the associated eigenvalue problem \eqref{eigenvalue problem-ep-original-m} are
given in \eqref{km-eigenvalues}-\eqref{nonkm-eigenvalues}. The corresponding eigenspaces are given as follows.
\if0
$(1)$ For $1\leq i\leq m-1$, the  eigenspace associated to the eigenvalue  ${i\over 2m}\left({i\over m}+1\right)$ is spanned by
 \begin{align*}
 (1-\gamma_\ep^2)^{i\over2m}C_0^{{i\over m}+{1\over2}}(\gamma_\ep)\cos\left({i\over m}\theta_\ep\right),\;\;(1-\gamma_\ep^2)^{i\over2m}C_0^{{i\over m}+{1\over2}}(\gamma_\ep)\sin\left({i\over m}\theta_\ep\right).
 \end{align*}
 \fi
 \begin{itemize}
 \item For $n\geq1$,
the eigenspace associated to the eigenvalue  ${1\over2}n\left( n+1\right)$ is spanned by
 \begin{align*}
L_{n}(\gamma_\ep(x,y))-L_{n}(0),\;\; L_{n,j}(\gamma_\ep(x,y))\cos(j\theta_\ep(x,y)),\;\;
 L_{n,j}(\gamma_\ep(x,y))\sin(j\theta_\ep(x,y)), \quad  1 \leq j\leq n.
 \end{align*}
\item For $1\leq i\leq m-1$ and $n\geq0$,
the eigenspace associated to the eigenvalue  ${1\over2}\left(n+{i\over m}\right)$ $\left( n+{i\over m}+1\right)$ is spanned by
\begin{align*}
 &(1-\gamma_\ep(x,y)^2)^{(n-j)m+i\over2m}C_j^{{(n-j)m+i\over m}+{1\over2}}(\gamma_\ep(x,y))\cos\left({(n-j)m+i\over m}\theta_\ep(x,y)\right),\\
 &(1-\gamma_\ep(x,y)^2)^{(n-j)m+i\over2m}C_j^{{(n-j)m+i\over m}+{1\over2}}(\gamma_\ep(x,y))\sin\left({(n-j)m+i\over m}\theta_\ep(x,y)\right),\;\;0\leq j\leq n.
 \end{align*}
 \end{itemize}
Here $\theta_\ep(x,y)$ and  $\gamma_\ep(x,y)$ are defined in \eqref{transf1} and \eqref{transf2}.

 In particular,
the multiplicity of $ {1\over2}n\left(n+1\right)$ is $2n+1$ for  $n\geq1$, and the multiplicity of
${1\over2}\left(n+{i\over m}\right)\left( n+{i\over m}+1\right)$ is $2n+2$ for $ 1\leq i\leq m-1$ and  $n\geq0$.
\end{Theorem}
\begin{proof}
By
\eqref{0 mode-eigenvalue-eigenfunction-m} and Lemma \ref{sol to eigenvalue problem non-zero modes varepsilon=0}  the set of all the eigenvalues of \eqref{eigenvalue problem-ep-new-m} is
\begin{align*}
&\left\{{1\over2}n\left(n+1\right)\right\}_{n=1}^\infty\cup\left(\bigcup_{k=1}^\infty\left\{{1\over2}\left(n+{k\over m}\right)\left(n+{k\over m} +1\right)\right\}_{n=0}^\infty\right)\\
=&\left\{{1\over2}n\left(n+1\right)\right\}_{n=1}^\infty\cup
\left(\bigcup_{i=1}^{m-1}\left\{{1\over2}\left(n+{i\over m}\right)\left(n+{i\over m} +1\right)\right\}_{n=0}^\infty\right).
\end{align*}

Let  $n\geq1$. Then ${1\over2}n\left( n+1\right)$  is the eigenvalue of the $0$ mode with
an eigenfunction   $L_{n}(\gamma_\ep)-L_{n}(0)$. It is also  the eigenvalue $\lambda_{n-j,{k\over m}}$ of the $k=jm$ mode with an eigenfunction
$(1-\gamma_\ep^2)^{j\over2}C_{n-j}^{j+{1\over2}}(\gamma_\ep)$ for $1\leq j\leq n$. Then up to a constant factor, the equality $(1-\gamma_\ep^2)^{j\over2}C_{n-j}^{j+{1\over2}}(\gamma_\ep)=L_{n,j}(\gamma_\ep)$ gives \eqref{km-eigenfunctions}.

Let $1\leq i\leq m-1$ and  $n\geq0$. Then  ${1\over2}\left(n+{i\over m}\right)\left( n+{i\over m}+1\right)$ is the eigenvalue $\lambda_{j,{k\over m}}$ of the $k=(n-j)m+i$ mode with an eigenfunction $(1-\gamma_\ep^2)^{(n-j)m+i\over2m}C_j^{{(n-j)m+i\over m}+{1\over2}}(\gamma_\ep)$ for  $0\leq j\leq  n$, which gives \eqref{nonkm-eigenfunctions}.
\end{proof}

As an application,
we prove that  $\tilde A_{\ep,o}$ and $L_{\ep,o}$ are  non-negative, present their explicit kernel,  and obtain  decompositions of
$\tilde X_{\ep,o}$ and $X_{\ep,o}$  associated to the two operators. This verifies  {\textbf{(G2)}} in Lemma \ref{indice-theorem-sep} for \eqref{sep-hamiltonian}.


\begin{Corollary}\label{A-L-dec-o}
 Let $\ep\in[0,1)$. Then

$(1)$ $\ker (\tilde A_{\ep,o})=\textup{span}\{\gamma_\ep(x,y)\}$ and $\ker (L_{\ep,o})=\textup{span}\{g'(\psi_\ep)\gamma_\ep(x,y)\}$. Thus,  $\dim \ker(L_{\ep,o})=\dim\ker (\tilde A_{\ep,o})=1$.

$(2)$ Let $\tilde X_{\ep,o+}=\tilde X_{\ep,o} \ominus\ker (\tilde A_{\ep,o})$ and $ X_{\ep,o+}= X_{\ep,o} \ominus\ker (L_{\ep,o})$. Then
\begin{align*}
\langle \tilde A_{\ep,o} \psi,\psi\rangle \geq \left(1-{2m^2\over (m+1)(2m+1)}\right) \| \psi\|_{\tilde X_{\ep,o}}^2, \quad\forall \psi\in \tilde X_{\ep,o+},
\end{align*}
and there exists $\delta>0$ such that
\begin{align*}
\langle L_{\ep,o} \omega,\omega\rangle \geq \delta \| \omega\|_{ X_{\ep,o}}^2, \quad \forall \omega\in  X_{\ep,o+}.
\end{align*}
\end{Corollary}
\begin{proof}
Note that $\psi(x,y)$ is odd in $y$ if and only if $\Psi(\theta_\ep, \gep)$ is odd in $\gep$ for $\psi \in \tilde{X}_{\ep,m}$ and $\Psi \in \tilde{Y}_{\ep,m}$ such that $\psi(x,y) = \Psi(\theta_\ep, \gamma_\ep)$. Thus, $\psi\in\tilde{X}_{\ep,o}$ if and only if $\Psi\in\tilde{Y}_{\ep,o}$. We consider the eigenvalue problem \eqref{eigenvalue problem-ep-new-m} with $\Psi \in \tilde{Y}_{\ep,o}$ by separating it into the Fourier modes.

For the $0$ mode, the eigenvalue problem \eqref{eigenvalue problem-ep-new-m} is reduced to \eqref{eigenvalue problem for 0 mode-m}. Noting that the eigenfunction $\varphi_{n,0}$ in \eqref{0 mode-eigenvalue-eigenfunction-m} is odd if and only if $n\geq1$ is odd, we obtain that all the eigenvalues and corresponding eigenfunctions  are given in \eqref{0 mode-eigenvalue-eigenfunction-m} with $n\geq1$ to be odd. Thus, the principal eigenvalue  for the $0$ mode is $1$ with an eigenfunction $\gamma_\ep$. This implies that
 there is no contribution to the negative directions of $ \tilde A_{\ep,o} $ from the $0$ mode, and $\gamma_\ep(x,y)\in\ker( \tilde A_{\ep,o} )$.

For the $k$ mode with $k\neq0$, the eigenvalue problem \eqref{eigenvalue problem-ep-new-m} is reduced to \eqref{eigenvalue problem2 non-zero modes varepsilon=0m}. Noting that the eigenfunction $\varphi_{n,{k\over m}}(\gamma_\ep)$ in Lemma \ref{sol to eigenvalue problem non-zero modes varepsilon=0} is odd if and only if $n\geq0$ is odd, we know that
all the eigenvalues and corresponding  eigenfunctions  are given in Lemma  \ref{sol to eigenvalue problem non-zero modes varepsilon=0}  with $n\geq0$ to be odd. Thus, the principal eigenvalue  for the $k$ mode is ${1\over2}\left(1+{k\over m}\right)\left(2+{k\over m} \right)>1$. Then  there is no contribution to the negative and kernel directions of $ \tilde A_{\ep,o} $ from the $k$ mode.
This confirms that $\ker (\tilde A_{\ep,o})=\textup{span}\{\gamma_\ep(x,y)\}$.




Since the second  eigenvalue  for the $0$ mode is $6$ and the  principal eigenvalue  for the $k$ mode is ${1\over2}\left(1+{k\over m}\right)\left(2+{k\over m} \right)>1$ with $k\neq0$, by the variational problem \eqref{variational problem2-ep-m}-\eqref{variational problem2-ep-m-new variables} we have
\begin{align*}
\iint_{\Omega_m}|\nabla\psi|^2dxdy\geq{1\over2}\left(1+{1\over m}\right)\left(2+{1\over m}\right)\iint_{\Omega_m}g'(\psi_\ep)(\psi-P_{\ep,m}\psi)^2dxdy,\;\psi\in\tilde X_{\ep,o+},
\end{align*}
where $\tilde X_{\ep,o+}=\tilde X_{\ep,o} \ominus\ker (\tilde A_{\ep,o})$.
Thus,
\begin{align*}
 \langle\tilde A_{\ep,o}\psi,\psi\rangle=&\iint_{\Omega_m} \left(|\nabla\psi|^2-g'(\psi_\epsilon)(\psi - P_{\epsilon,m}\psi)^2\right) dxdy \\
 \geq& \left(1-{2m^2\over (m+1)(2m+1)}\right) \| \psi\|_{\tilde X_{\ep,o}}^2
\end{align*}
for $\psi\in\tilde X_{\ep,o+}$.

By \eqref{n-ker-o},  $\ker (L_{\ep,o})=\textup{span}\{g'(\psi_\ep)\gamma_\ep(x,y)\}$.
The proof of $\langle L_{\ep,o} \omega,\omega\rangle \geq \delta \| \omega\|_{ X_{\ep,o}}^2$ for   $\omega\in  X_{\ep,o+} $ is similar to  \eqref{positive-decom}.
\end{proof}


Next, we give
 the explicit negative directions and  kernel of  the operators $\tilde A_{\ep,e}$ and $L_{\ep,e}$, as well as   decompositions of
$\tilde X_{\ep,e}$ and $X_{\ep,e}$  associated to $\tilde A_{\ep,e}$ and $L_{\ep,e}$, respectively. This verifies  {\textbf{(G3)}} in Lemma \ref{indice-theorem-sep} for \eqref{sep-hamiltonian}.

\begin{Corollary}\label{A-L-dec-e}
 Let $\ep\in[0,1)$. Then

$(1)$  the negative subspaces of  $\tilde X_{\ep, e}$ and  $ X_{\ep, e}$  associated to $\tilde A_{\ep,e}$ and $L_{\ep,e}$ are
 \begin{align*}
\tilde X_{\ep,e-}&=\textup{span}\left\{(1-\gamma_\ep^2)^{i\over 2m}\cos\left({i\theta_\ep\over m}\right),(1-\gamma_\ep^2)^{i\over 2m}\sin\left({i\theta_\ep\over m}\right), 1\leq i\leq m-1\right\},\\
 X_{\ep,e-}&=\textup{span}\left\{g'(\psi_\ep)(1-\gamma_\ep^2)^{i\over 2m}\cos\left({i\theta_\ep\over m}\right),g'(\psi_\ep)(1-\gamma_\ep^2)^{i\over 2m}\sin\left({i\theta_\ep\over m}\right), 1\leq i\leq m-1\right\},
\end{align*}
respectively, where $\gamma_\ep=\gamma_\ep(x,y)$ and $\theta_\ep=\theta_\ep(x,y)$.
Thus,  $\dim \tilde X_{\ep,e-}=\dim X_{\ep,e-}=2(m-1)$.

$(2)$ $\ker (\tilde A_{\ep,e})=\textup{span}\{(1-\gamma_\ep^2)^{1\over 2}\cos\left({\theta_\ep}\right),(1-\gamma_\ep^2)^{1\over 2}\sin\left({\theta_\ep}\right)\}$ and $\ker (L_{\ep,e})=\textup{span}\{g'(\psi_\ep)(1-\gamma_\ep^2)^{1\over 2}\cos\left({\theta_\ep}\right),g'(\psi_\ep)(1-\gamma_\ep^2)^{1\over 2}\sin\left({\theta_\ep}\right)\}$. Thus,  $\dim\ker (\tilde A_{\ep,e})=\dim \ker(L_{\ep,e})=2$.

$(3)$ Let  $ X_{\ep,e+}= X_{\ep,e} \ominus\left(\ker (L_{\ep,e})\oplus X_{\ep,e-}\right)$ and $\tilde X_{\ep,e+}=\tilde X_{\ep,e} \ominus\left(\ker (\tilde A_{\ep,e})\oplus \tilde  X_{\ep,e-}\right)$. Then
\begin{align*}
\langle \tilde A_{\ep,e} \psi,\psi\rangle \geq \left(1-{2m^2\over (m+1)(2m+1)}\right) \| \psi\|_{\tilde X_{\ep,e}}^2, \quad\forall \psi\in \tilde X_{\ep,e+},
\end{align*}
there exists $\delta>0$ such that
\begin{align*}
\langle L_{\ep,e} \omega,\omega\rangle \geq \delta \| \omega\|_{ X_{\ep,e}}^2, \quad \forall \omega\in  X_{\ep,e+}.
\end{align*}
\end{Corollary}




\begin{proof}
Note that $\psi\in\tilde{X}_{\ep,e}$ if and only if $\Psi\in\tilde{Y}_{\ep,e}$ for $\psi \in \tilde{X}_{\ep,m}$ and $\Psi \in \tilde{Y}_{\ep,m}$ such that $\psi(x,y) = \Psi(\theta_\ep, \gamma_\ep)$.  We also consider the eigenvalue problem \eqref{eigenvalue problem-ep-new-m} with $\Psi \in \tilde{Y}_{\ep,e}$ by separating it into the Fourier modes.

For the $0$ mode, the eigenvalue problem \eqref{eigenvalue problem-ep-new-m} is reduced to \eqref{eigenvalue problem for 0 mode-m}. Since $\varphi_{n,0}$ in \eqref{0 mode-eigenvalue-eigenfunction-m} is even if and only if $n\geq1$ is even, all the eigenvalues and corresponding eigenfunctions   are given in \eqref{0 mode-eigenvalue-eigenfunction-m} with $n\geq1$ to be even. Thus, the principal eigenvalue  for the $0$ mode is $3$. This implies that
 there is no contribution to the negative directions and kernel of $ \tilde A_{\ep,e} $ from the $0$ mode.

For the $k$ mode with $k\neq0$, the eigenvalue problem \eqref{eigenvalue problem-ep-new-m} is reduced to \eqref{eigenvalue problem2 non-zero modes varepsilon=0m}. Since $\varphi_{n,{k\over m}}(\gamma_\ep)$ in Lemma \ref{sol to eigenvalue problem non-zero modes varepsilon=0} is even if and only if $n\geq0$ is even, we know that
all the eigenvalues and corresponding  eigenfunctions are given in Lemma  \ref{sol to eigenvalue problem non-zero modes varepsilon=0}  with $n\geq0$ to be even. Thus, the principal eigenvalue  for the $k$ mode is ${1\over2}{k\over m}\left({k\over m}+1 \right)$ with an eigenfunction $(1-\gamma_\ep^2)^{k\over 2m}$.
 For the $k$ mode  with $1\leq k\leq m-1$, the principal eigenvalue satisfies ${1\over2}{k\over m}\left({k\over m}+1 \right)<1$, which gives $2m-2$
 negative directions of $ \tilde A_{\ep,e} $
\begin{align*} (1-\gamma_\ep^2)^{k\over 2m}\cos\left({k\theta_\ep\over m}\right),(1-\gamma_\ep^2)^{k\over 2m}\sin\left({k\theta_\ep\over m}\right), 1\leq k\leq m-1.
 \end{align*}
 For the $m$ mode, the principal eigenvalue is $1$, which implies that
 \begin{align*}(1-\gamma_\ep^2)^{1\over 2}\cos\left({\theta_\ep}\right),(1-\gamma_\ep^2)^{1\over 2}\sin\left({\theta_\ep}\right) \in \ker (\tilde A_{\ep,e}).\end{align*}
 For the $k$ mode  with $k\geq m+1$, the principal eigenvalue satisfies
 \begin{align}\label{k-m+1-pr}
 {1\over2}{k\over m}\left({k\over m}+1 \right)\geq{1\over2}\left({1\over m}+1\right)\left({1\over m}+2 \right)>1.
 \end{align}
  For the $k$ mode  with $k\geq 1$, the second eigenvalue satisfies
    \begin{align}\label{k-1-se}
 {1\over2}\left({k\over m}+2\right)\left({k\over m} +3\right)>3.
 \end{align}
Then $\tilde X_{\ep,e-}$ and $\ker (\tilde A_{\ep,e})$ have no more linearly independent functions, and thus, are given in (1)-(2).




Note that the principal  eigenvalue  for the $0$ mode is $3$. By \eqref{k-m+1-pr}-\eqref{k-1-se},  the minimal  eigenvalue, which is larger than $1$,  for the nonzero modes is ${1\over2}\left({1\over m}+1\right)\left({1\over m}+2 \right)$. By the variational problem \eqref{variational problem2-ep-m}-\eqref{variational problem2-ep-m-new variables} we also have
\begin{align*}
\iint_{\Omega_m}|\nabla\psi|^2dxdy\geq{1\over2}\left(1+{1\over m}\right)\left(2+{1\over m}\right)\iint_{\Omega_m}g'(\psi_\ep)(\psi-P_{\ep,m}\psi)^2dxdy,\;\psi\in\tilde X_{\ep,e+},
\end{align*}
where $\tilde X_{\ep,e+}=X_{\ep,e} \ominus\left(\ker (L_{\ep,e})\oplus X_{\ep,e-}\right)$.
Thus,
\begin{align*}
 \langle\tilde A_{\ep,e}\psi,\psi\rangle
 \geq& \left(1-{2m^2\over (m+1)(2m+1)}\right) \| \psi\|_{\tilde X_{\ep,e}}^2,\quad \psi\in\tilde X_{\ep,e+}.
\end{align*}

The rest of the proof follows from \eqref{n-ker-e} and a similar argument to \eqref{positive-decom}.
\end{proof}




By Corollaries \ref{A-L-dec-o}-\ref{A-L-dec-e}, the assumptions \textbf{(G2-4)} in Lemma \ref{indice-theorem-sep} are verified for the Hamiltonian system
\eqref{sep-hamiltonian}.

\subsection{A linear instability criterion}
Applying Lemma \ref{indice-theorem-sep} to the Hamiltonian system \eqref{sep-hamiltonian},  the criterion for  linear instability of the cats' eyes flows is  that   $n^-\left( L_{\ep,e} |_{\overline{R(B_\ep)}} \right) \geq 1$.
To study $n^-\left( L_{\ep,e} |_{\overline{R(B_\ep)}} \right)$,  we denote $\hat L_e^2(\Omega_m)=\{\omega\in L^2(\Omega_m)|\omega \text{ is even in }y,\text{ and }\iint_{\Omega_m}\sqrt{g'(\psi_\ep)}\omega dxdy=0\}$. Then we define
$\bar{P}_{\ep,e}$ to be the orthogonal   projection of the space $ (\hat L_e^2(\Omega_m))^*$ on $\ker(\vec{u}_\ep\cdot\nabla)$. It induces  a   projection
$\hat{P}_{\ep,e}$ of $X_{\ep, e}^*$ on $\ker (B_\ep')$ by $\hat{P}_{\ep,e}=(S_{e,m}')^{-1}\bar{P}_{\ep,e} S_{e,m}'$, where
$S_{e,m}: \hat L_e^2(\Omega_m) \rightarrow X_{\ep,e},  S_{e,m}\omega = \sqrt{g'(\psi_\ep)}\omega$
defines an isometry.
Similar to \cite{lin2004some}, it takes the form
\begin{align}\label{def-hat-P-ep-e}
(\hat{P}_{\ep,e} \psi)|_{\Gamma_{i}(\rho)} = \frac{\oint_{\Gamma_i (\rho)} \frac{\psi}{|\nabla \psi_\ep|}}{\oint_{\Gamma_i (\rho)} \frac{1}{|\nabla \psi_\ep|}},\end{align}
where $\rho$ is in the range of $\psi_\ep$ and $\Gamma_i(\rho)$ is a branch of $\{\psi_\ep = \rho\}$. Noting that $\tilde{X}_{\ep, e}\subset X_{\ep, e}^*$, we  define the operator
$$\hat{A}_{\ep,e} = - \Delta - g'(\psi_\ep)(I - \hat{P}_{\ep,e}): \tilde{X}_{\ep, e} \rightarrow \tilde{X}^*_{\ep, e}.$$
Then we have the following lemma.
\begin{lemma}\label{L e-hat A} The number of unstable modes of \eqref{sep-hamiltonian} is
$$n^-\left( L_{\ep,e} |_{\overline{R(B_\ep)}} \right) = n^-\left(\hat{A}_{\ep,e}\right).$$
Consequently, if $n^-\left(\hat{A}_{\ep,e}\right)>0$, then $\omega_\ep$ is linearly unstable for $2m\pi$-periodic perturbations.
\end{lemma}
\begin{proof} Since $\hat{P}_{\ep,e}$  commutes with $f(\psi_\ep)\cdot$ for any function $f$,  $\omega \in \overline{R(B_\ep)}$ if and only if $\hat{P}_{\ep,e} \frac{\omega}{g'(\psi_\ep)} = 0$. Note that $\bar{P}_{\ep,e}$ is orthogonal in the $L^2$ sense.
For $\omega \in \overline{R(B_\ep)}\subset X_{\ep,e}$, there exists $\psi \in \tilde{X}_{\ep, e}$ such that $-\Delta\psi=\omega$ and
\begin{align*}
&\langle L_{\ep,e} \omega, \omega \rangle
= \iint_{\Omega_m} \left(\frac{\omega^2}{g'(\psi_\ep)} - \omega \psi\right) dxdy \\
= &\iint_{\Omega_m}\left(\bar{P}_{\ep,e}\left( \frac{\omega}{\sqrt{g'(\psi_\ep)}} - \psi \sqrt{g'(\psi_\ep)}\right)+(I-\bar{P}_{\ep,e})\left( \frac{\omega}{\sqrt{g'(\psi_\ep)}} - \psi \sqrt{g'(\psi_\ep)}\right) \right)^2 dxdy \\
&- \iint_{\Omega_m}\left(g'(\psi_\ep)\psi^2 -|\nabla \psi|^2\right) dxdy \\
=& \iint_{\Omega_m} \left(\left( \frac{\omega}{\sqrt{g'(\psi_\ep)}} - \sqrt{g'(\psi_\ep)} (I - \hat{P}_{\ep,e}) \psi \right)^2 + g'(\psi_\ep) (\hat{P}_{\ep,e} \psi)^2  - g'(\psi_\ep)\psi^2 + |\nabla \psi|^2 \right)dxdy \\
 \geq& \iint_{\Omega_m} \left(|\nabla \psi|^2 - g'(\psi_\ep)\psi^2 + g'(\psi_\ep) (\hat{P}_{\ep,e}\psi)^2 \right)dxdy = \langle\hat{A}_{\ep,e} \psi, \psi\rangle,
\end{align*}
where we used \eqref{Poincare inequality I-m} to ensure that $\sqrt{g'(\psi_\ep)}\psi\in L^2(\Omega_m)$.

For $\psi\in\tilde{X}_{\ep, e}$, we have $\tilde{\omega} \triangleq g'(\psi_\ep)(I - \hat{P}_{\ep,e})\psi \in \overline{R(B_\ep)}$. Let $\tilde{\psi} = (-\Delta)^{-1}\tilde{\omega}$. Then
\begin{align}\nonumber
\langle\hat{A}_{\ep,e} \psi, \psi\rangle
&= \iint_{\Omega_m} \left( |\nabla \psi|^2 - g'(\psi_\ep)((I - \hat{P}_{\ep,e})\psi)^2\right) dxdy \\\nonumber
&= \iint_{\Omega_m} \left( |\nabla \psi|^2 - \frac{\tilde{\omega}^2}{g'(\psi_\ep)}\right)dxdy \\\nonumber
&= \iint_{\Omega_m} \left(|\nabla \psi|^2 - 2 \tilde{\omega} \psi + \frac{\tilde{\omega}^2}{g'(\psi_\ep)}\right)dxdy \\\nonumber
& \geq \iint_{\Omega_m}  \left(\frac{\tilde{\omega}^2}{g'(\psi_\ep)} - |\nabla \tilde{\psi}|^2\right) dxdy = \langle L_{\ep,e}\tilde{\omega}, \tilde{\omega} \rangle,
\end{align}
where we used $\langle\tilde \omega, \hat{P}_{\ep,e}\psi\rangle=0$.
From the two inequalities above, we have $n^{\leq0}\left( L_{\ep,e} |_{\overline{R(B_\ep)}} \right) = n^{\leq0}\left(\hat{A}_{\ep,e}\right)$.
Similar to (11.60) in \cite{lin2022instability}, we have $\dim \ker \left(L_{\ep,e} |_{\overline{R(B_\ep)}} \right)=\dim \ker (\hat{A}_{\ep,e})$. Thus,
 $n^{-}\left( L_{\ep,e} |_{\overline{R(B_\ep)}} \right) = n^{-}\left(\hat{A}_{\ep,e}\right)$.
\end{proof}

\if0
From Proposition \ref{prop3}, we know that the eigenvalue problem
\begin{align}\label{poisson-eq}
-\Delta \psi = \lambda g'(\psi_\ep) \psi, \quad \psi \in \tilde{X}_{\ep, e},
\end{align}
is equivalent to
\begin{align}\label{poisson-new}
 -\frac{\phi_{\theta_\ep \theta_\ep}}{1-\gamma_\ep^2} - \left( (1-\gamma_\ep^2)\phi_{\gamma_\ep} \right)_{\gamma_\ep} = 2\lambda \phi, \quad \phi \in \tilde{Y}_{\ep, e},
\end{align}
where $\phi(\theta_\ep, \gamma_\ep) = \psi(x(\theta_\ep, \gamma_\ep), y(\theta_\ep, \gamma_\ep))$ under the transformation \eqref{transf1} and \eqref{transf2} with periodic extension.
\fi

 To study the linear instability of the Kelvin-Stuart vortex $\omega_\ep$ for multi-periodic perturbations, we will construct a specific test function $\psi \in \tilde{X}_{\ep, e}$ such that
 \begin{align*}
\langle\hat{A}_{\ep,e} \psi, \psi \rangle=  b_{\ep, 1}(\psi) + b_{\ep, 2}(\psi) < 0,
\end{align*}
where
\begin{align*}
  b_{\ep, 1}(\psi) = \iint_{\Omega_m} \left(|\nabla \psi|^2  - g'(\psi_\ep) \psi^2 \right)dxdy
\end{align*}
and
\begin{align*}
 b_{\ep, 2}(\psi) =  \iint_{\Omega_m} g'(\psi_\ep)( \hat{P}_{\ep,e}\psi)^2 dxdy =  \int_{\min \psi_\ep}^{\infty} g'(\rho) \sum_{i=1}^{n_\rho} \frac{\left|\oint_{\Gamma_i (\rho)} \frac{\psi}{|\nabla \psi_\ep|}\right|^2}{\oint_{\Gamma_i (\rho)} \frac{1}{|\nabla \psi_\ep|}} d\rho.
 \end{align*}
Here,  $\{ \Gamma_i (\rho), i = 1, \cdots, n_\rho\}$ is the set of all the disjoint closed level curves in the level set $\{(x, y)\in \Omega_m |\psi_\ep(x,y) = \rho\}$, where $\rho\in[\min\psi_\ep,\infty)$.
Then  by Lemma \ref{L e-hat A} we have $n^-\left( L_{\ep,e} |_{\overline{R(B_\ep)}} \right) \geq 1$, and the linear instability follows from Lemma \ref{indice-theorem-sep}.

\subsection{Proof of multi-periodic instability   (even multiple case)}
In this subsection, we prove the linear instability of the  Kelvin-Stuart vortex $\omega_\ep$ for $4k\pi$-periodic perturbations. We take the test function
\begin{align}\label{test-even}
\tilde{\psi}_\ep(x, y)  =\tilde{\Psi}_\ep(\theta_\ep, \gamma_\ep)= \cos\left(\frac{\theta_\ep}{2}\right)(1-\gamma_\ep^2)^{1\over4}
\end{align}
with $(\theta_\ep, \gep) \in \tilde{\Omega}_{2k} = \mathbb{T}_{4k\pi} \times [-1, 1]$. Then $\tilde{\Psi}_\ep\in\tilde Y_{\ep,e}\Longrightarrow\tilde{\psi}_\ep\in\tilde X_{\ep,e}$. By Theorem \ref{sol to eigenvalue problem varepsilon=0-pde}, $\tilde{\psi}_\ep(x, y)$ is exactly an eigenfunction of the principal  eigenvalue  $\lambda={3\over 8}$ for
\eqref{eigenvalue problem-ep-original-m}, and thus,
$$-(\Delta +g'(\psi_\ep)) \tilde{\psi}_\ep = - \frac 5 8 g'(\psi_\ep)\tilde{\psi}_\ep.$$
Then
\begin{align}\nonumber
b_{\ep, 1}(\tilde{\psi}_\ep)
= & \int_{-\infty}^{+\infty} \int_0^{4k\pi} \left(|\nabla \tilde{\psi}_\ep|^2 - g'(\psi_\epsilon)\tilde{\psi}_\ep^2 \right)dx dy
=  -{5\over8}\int_{-\infty}^{+\infty} \int_0^{4k\pi} g'(\psi_\ep)\tilde{\psi}_\ep^2dx dy \\\label{b1-even}
= & -{5\over4} \int_0^{4k\pi}  \cos^2\left(\frac{\theta_\ep}{2}\right)d \theta_\ep\int_{-1}^{1} (1-\gamma_\ep^2)^{1\over2} d\gep
=  -{5\over4}k\pi^2.
\end{align}
$b_{\ep, 2}(\tilde{\psi}_\ep)$ vanishes by symmetry as seen in the next lemma.
\begin{lemma}\label{b2-even}
$$ b_{\ep, 2}(\tilde{\psi}_\ep) =  \int_{\min \psi_\ep}^{\max \psi_\ep} g'(\rho) \sum_{i=1}^{n_\rho} \frac{\left|\oint_{\Gamma_i (\rho)} \frac{\tilde{\psi}_\ep}{|\nabla \psi_\ep|}\right|^2}{\oint_{\Gamma_i (\rho)} \frac{1}{|\nabla \psi_\ep|}} d\rho = 0.$$
\end{lemma}
\begin{proof} Since $\tilde{\psi}_\ep$ is   `odd'   symmetrical about the point $( (2j-1)\pi,0)$, $1\leq j\leq 2k$, we have $\hat{P}_{\ep,e}\tilde{\psi}_\ep \equiv0$ on $\mathbb{T}_{4k\pi}\times\mathbb{R}$, and thus, $b_{\ep, 2}(\tilde{\psi}_\ep)=0$.
 \if0
 Here we give a detailed computation for converting the curve integrals to  definite integrals, which will be used in studying the linear modulational instability of $\omega_\ep$.
 Consider $\psi_\ep$ on $\mathbb{T}_{2\pi}\times \mathbb{R}$.
By \eqref{steadyv}, $(0,0)$ and $(\pi,0)$ are critical points of $\psi_\ep$. The  Hessian matrix of $\psi_\ep$ is
\begin{align*}
\left( \begin{array}{cc} {-\ep^2-\ep\cos(x)\cosh(y)\over (\cosh(y)+\ep\cos(x))^2} & {\ep\sin(x)\sinh(y)\over (\cosh(y)+\ep\cos(x))^2} \\ {\ep\sin(x)\sinh(y)\over (\cosh(y)+\ep\cos(x))^2} & {1+\ep\cosh(y)\cos(x)\over (\cosh(y)+\ep\cos(x))^2} \end{array} \right).
\end{align*}
Then $(0,0)$ is a saddle point of $\psi_\ep$, and $(\pi,0)$ is the minimal  point of $\psi_\ep$, since $\psi_\ep(x,y)\to\infty$ as $y\to \pm\infty$ for $x\in\mathbb{T}_{2\pi}$.
Let
\begin{align}\label{def-rho0}
\rho_0 = \psi_\ep(0,0) = \ln\left(\sqrt{\frac{1+\ep}{1-\ep}}\right).
\end{align}
Then $\min \psi_\ep = \psi_\ep(\pi, 0) = - \rho_0$. We separate $b_{\ep, 2}(\tilde{\psi}_\ep)$ into two parts:
$$b_{\ep, 2}(\tilde{\psi}_\ep) =\int_{-\rho_0}^{\rho_0} g'(\rho) \sum_{i=1}^{n_\rho} \frac{\left|\oint_{\Gamma_i (\rho)} \frac{\tilde{\psi}_\ep}{|\nabla \psi_\ep|}\right|^2}{\oint_{\Gamma_i (\rho)} \frac{1}{|\nabla \psi_\ep|}} d\rho +\int_{\rho_0}^{\infty} g'(\rho) \sum_{i=1}^{n_\rho} \frac{\left|\oint_{\Gamma_i (\rho)} \frac{\tilde{\psi}_\ep}{|\nabla \psi_\ep|}\right|^2}{\oint_{\Gamma_i (\rho)} \frac{1}{|\nabla \psi_\ep|}} d\rho= I_1 + I_2. $$
For $\rho\in(-\rho_0, \rho_0]$, the streamlines are in the trapped regions and  the level set $\Gamma(\rho) = \{(x, y) \in \Omega_{2k} | \psi_\ep(x,y) = \rho\} $ has $n_\rho = 2k$ closed level curves, i.e. $\Gamma(\rho) =  \bigcup_{i=1}^{n_\rho} \Gamma_i(\rho)$, where $\Gamma_i(\rho)$ corresponds to a periodic orbit inside the $i$-th cat's eye. We divide each orbit into two parts, namely, the upper part
\begin{align}\label{curve-Gamma-i+rho1}
\Gamma_{i+} (\rho)
& = \{(x, y) \in [2(i-1)\pi, 2i\pi] \times \mathbb{R}  \;|\;  \psi_\ep(x,y) = \rho,  y \geq 0\},
\end{align}
and the lower part
\begin{align}\label{curve-Gamma-i+rho2}\Gamma_{i-} (\rho) = \{(x, y) \in [2(i-1)\pi, 2i\pi] \times \mathbb{R}\;|\;  \psi_\ep(x,y) = \rho,  y < 0\},\end{align}
where $i = 1, \cdots, n_\rho$. Using $x$ as the parameter,
we  represent $\Gamma_{i+}(\rho)$ and $\Gamma_{i-}(\rho)$ as follows:
$$\vec{r}_{i+} (x) = (x, \cosh^{-1}(\sqrt{1-\ep^2} e^{\rho} - \ep \cos(x))), \quad x \in [2(i-1)\pi + x_0, 2i\pi - x_0], $$
and
$$\vec{r}_{i-} (x) = (x, -\cosh^{-1}(\sqrt{1-\ep^2} e^{\rho} - \ep \cos(x))), \quad x \in (2(i-1)\pi + x_0, 2i\pi - x_0), $$
respectively. Here, $x_0 = \arccos\left( \frac{\sqrt{1-\ep^2} e^\rho - 1}{\ep} \right) $ is the point on $[0, \pi]$ such that $\psi_\ep(x_0,0)=\rho$.
Moreover, we have
\begin{align}\label{dr+}
\left|\frac{d \vec{r}_{i+}(x)}{dx}\right| = \sqrt{ 1 + \left(\frac{\ep \sin(x)}{\sinh(y(x))}\right)^2} = \sqrt{ 1 + \frac{\ep^2 \sin^2(x)}{ ( \sqrt{1-\ep^2} e^{\rho} - \ep \cos(x) )^2 - 1 }},
\end{align}
and
\begin{align}\label{dr-}
\left|\frac{d \vec{r}_{i-}(x)}{dx}\right|  = \sqrt{ 1 + \left(\frac{\ep \sin(x)}{\sinh(y(x))}\right)^2} = \sqrt{ 1 + \frac{\ep^2 \sin^2(x)}{ ( \sqrt{1-\ep^2} e^{\rho} - \ep \cos(x) )^2 - 1 }},\end{align}
where $y(x)=\cosh^{-1}(\sqrt{1-\ep^2} e^{\rho} - \ep \cos(x))$.
Then we  represent $\left| \nabla \psi_\ep \right| $ and $\tilde{\psi}_\ep$ on $\Gamma_{i+}(\rho)$ and $\Gamma_{i-}(\rho)$ in terms of the parameter $x$. Since
$\psi_\ep(x,y) = \rho$, we have $\cosh(y) + \ep \cos(x) = e^\rho \sqrt{1-\ep^2}$. So
\begin{align}\label{tilde psi-p-x}
\left| \nabla \psi_\ep \right|
= & \left| \left( - \frac{\ep \sin(x)}{e^\rho \sqrt{1-\ep^2}}, \frac{\sinh(y)}{e^\rho \sqrt{1-\ep^2}} \right) \right|
=  \frac{\sqrt{\ep^2 \sin^2(x) + \sinh^2(y)}}{e^\rho \sqrt{1-\ep^2}} \\\nonumber
=&  \sqrt{1-e^{-2\rho} - \frac{2\ep \cos(x)}{e^\rho \sqrt{1-\ep^2}}},
\end{align}
and
\begin{align}\label{psi-p-x}
\tilde{\psi}_\ep(x,y(x)) =& \cos\left(\frac{\theta_\ep}{2}\right)(1-\gep^2)^{1\over4} \\\nonumber
=& \left\{ \begin{array}{lll} \sqrt{\frac{1+\cos(\theta_\ep)}{2}} (1-\gep^2)^{1\over4} &\mbox{ for } & x \in [2(i-1)\pi + x_0, (2i - 1)\pi ],\\
 -\sqrt{\frac{1+\cos(\theta_\ep)}{2}} (1-\gep^2)^{1\over4} &\mbox{ for } & x \in ((2i - 1)\pi, 2i\pi - x_0],
 \end{array} \right.
\end{align}
where
\begin{align}\label{psi-p-x1}
1 - \gep^2 = 1 - \sinh^2(y)e^{-2\rho} = 1 - \left(\left( \sqrt{1-\ep^2} e^{\rho} - \ep \cos(x) \right)^2 - 1\right) e^{-2\rho},
\end{align}
and
\begin{align}\label{psi-p-x2}
\cos(\theta_\ep) = \frac{\xi_\ep}{\sqrt{1-\gep^2}} = \frac{\ep + \sqrt{1-\ep^2} \cos(x) e^{-\rho} }{\sqrt{1 - \left(\left( \sqrt{1-\ep^2} e^{\rho} - \ep \cos(x) \right)^2 - 1\right) e^{-2\rho}}}.
\end{align}
Thus, for $\rho\in(-\rho_0, \rho_0]$, we have
\begin{align*}
\oint_{\Gamma_i (\rho)} \frac{\tilde{\psi}_\ep}{|\nabla \psi_\ep|}
=& \int_{\Gamma_{i+} (\rho)} \frac{\tilde{\psi}_\ep}{|\nabla \psi_\ep|} + \int_{\Gamma_{i-} (\rho)} \frac{\tilde{\psi}_\ep}{|\nabla \psi_\ep|}\\
= & \int_{2(i-1)\pi + x_0}^{2i\pi - x_0} \frac{\tilde{\psi}_\ep}{|\nabla \psi_\ep|} \left|\frac{d \vec{r}_{i+}(x)}{dx}\right| dx+
 \int_{2(i-1)\pi + x_0}^{2i\pi - x_0} \frac{\tilde{\psi}_\ep}{|\nabla \psi_\ep|} \left|\frac{d \vec{r}_{i-}(x)}{dx}\right| dx.
\end{align*}
By \eqref{psi-p-x}-\eqref{psi-p-x2}, we have
\begin{align*}
\tilde{\psi}_\ep(x,\pm y(x))=-\tilde{\psi}_\ep((4i -2)\pi-x,\pm y((4i - 2)\pi-x)),\quad x \in [2(i-1)\pi + x_0, (2i - 1)\pi ].
\end{align*}
By \eqref{dr+}-\eqref{tilde psi-p-x}, we have
\begin{align*}
f(x)=f((4i - 2)\pi-x),\quad x \in [2(i-1)\pi + x_0, (2i - 1)\pi ]
\end{align*}
for $f(x)=|\nabla \psi_\ep|$, $\left|\frac{d \vec{r}_{i+}(x)}{dx}\right|$ or $\left|\frac{d \vec{r}_{i-}(x)}{dx}\right|$.
In other words, $\tilde{\psi}_\ep$ is an `odd'   function symmetrical about the point $( (2i-1)\pi,0)$, and
$|\nabla \psi_\ep|$, $\left|\frac{d \vec{r}_{i+}(x)}{dx}\right|$, $\left|\frac{d \vec{r}_{i-}(x)}{dx}\right|$ are `even'   functions symmetrical about the vertical line $x=(2i-1)\pi$.
Then
$$\int_{\Gamma_{i+} (\rho)} \frac{\tilde{\psi}_\ep}{|\nabla \psi_\ep|} = \int_{\Gamma_{i-} (\rho)} \frac{\tilde{\psi}_\ep}{|\nabla \psi_\ep|} = 0\Longrightarrow\oint_{\Gamma_i (\rho)} \frac{\tilde{\psi}_\ep}{|\nabla \psi_\ep|} = 0$$
for $i = 1, \cdots, n_\rho$.
Thus, $I_1 = 0$.

To calculate $I_2$, we note that the streamlines are not in the trapped regions for   $\rho\in(\rho_0,\infty)$, and there are only $n_\rho = 2$ closed curves $\Gamma_1(\rho), \Gamma_2(\rho)$ in the level set $\Gamma(\rho) = \{(x, y) \in \Omega_{2k} | \psi_\ep(x,y) = \rho\}$. The $x$ component of each curve runs from $0$ to  $4k\pi$. Let $i=1,2$. We divide the curve $\Gamma_i(\rho)$ into $2k$ pieces $\Gamma_{i,j}(\rho)$, where the $j$-th piece is contained in $[2(j-1)\pi,2j\pi]\times \mathbb{R}$, $j=1,\cdots, 2k$. We use the same  strategy to parameterize the curve $\Gamma_{i,j}(\rho)$ as $ \vec{r}_{i,j+}(x)$, $\vec{r}_{i,j-}(x)$, and  represent the functions $\tilde{\psi}_\ep,$ $|\nabla \psi_\ep|$, $\left|\frac{d \vec{r}_{i,j+}(x)}{dx}\right|$, $\left|\frac{d \vec{r}_{i,j-}(x)}{dx}\right|$ as above.  The main point is that
$\tilde{\psi}_\ep$ is still   `odd'   symmetrical about the point $( (2j-1)\pi,0)$, and
$|\nabla \psi_\ep|$, $\left|\frac{d \vec{r}_{i,j+}(x)}{dx}\right|$, $\left|\frac{d \vec{r}_{i,j-}(x)}{dx}\right|$ are `even'    symmetrical about the vertical line $x=(2j-1)\pi$. Thus, $I_2 = 0$.
\fi
\end{proof}
Now we get linear instability of $\omega_\ep$ for perturbations with even multiples of the period.
\begin{Theorem}\label{multi-even}
Let $\ep\in[0,1)$. Then the steady state $\omega_\ep$ is linearly unstable for $4k\pi$-periodic perturbations, where $k\geq1$ is an integer.
\end{Theorem}
\begin{proof}
With the test function $\tilde{\psi}_\ep$ defined in \eqref{test-even}, by \eqref{b1-even} and Lemma \ref{b2-even}, we have
$$\langle\hat{A}_{\ep,e} \tilde{\psi}_\ep, \tilde{\psi}_\ep \rangle = -{5\over4}k\pi^2 < 0.$$
Then we have
$n^-\left( L_{\ep,e} |_{\overline{R(B_\ep)}} \right) = n^-\left(\hat{A}_{\ep,e}\right)\geq1$ by Lemma \ref{L e-hat A}.  The conclusion follows from  Lemma  \ref{indice-theorem-sep}.
\end{proof}

\subsection{Proof of multi-periodic instability  (odd multiple case)}
In this subsection, we study linear instability of the steady state $\omega_\ep$ for   $(4k+2)\pi$-periodic  perturbations, where $k\geq1$ is an integer.
We divide our discussion into two cases in terms of the $\ep$ values.
\medskip

\noindent{\bf{Case 1. Test functions for $\ep\in[0,{4\over5}]$.}}
\medskip

In this case,
 we take the test function to be
 \begin{align}\nonumber
\hat{\psi}_{1,\ep}(x,y)= &\hat\Psi_{1,\ep}(\theta_\ep,\gamma_\ep)\\\label{test-odd}
=  &\left\{ \begin{array}{ll}
         \sin\left(\frac{\theta_\ep}{3}\right)(1-\gamma_\ep^2)^{1\over6} & \mbox{if $(\theta_\ep,\gamma_\ep) \in [0, 6\pi]\times[-1,1]$},\\
         \sin\left(\theta_\ep\right)(1-\gamma_\ep^2)^{1\over2} & \mbox{if $(\theta_\ep,\gamma_\ep) \in (6\pi, (4k + 2)\pi]\times[-1,1]$}.\\
        \end{array} \right.
 \end{align}
 To show that $ \hat{\psi}_{1,\ep}\in \tilde X_{\ep,e}$, it suffices  to prove that $\hat\Psi_{1,\ep}\in\tilde Y_{\ep,e}$, where $\tilde Y_{\ep,e}$ is defined in  \eqref{def-Y-ep-m}. Note that $\hat\Psi_{1,\ep}\in C^0(\tilde \Omega_{\ep,2k+1})$. By Theorem \ref{sol to eigenvalue problem varepsilon=0-pde}, $\sin\left(\frac{\theta_\ep}{3}\right)(1-\gamma_\ep^2)^{1\over6}$ is an eigenfunction of the principal eigenvalue $\lambda={2\over 9}$ for  \eqref{eigenvalue problem-ep-new-m} with $m=3$. By Theorems \ref{associate_ep0} and \ref{associate_ep-new-variable-original-variable},
$\sin\left(\theta_\ep\right)(1-\gamma_\ep^2)^{1\over2}$ is an eigenfunction of the principal eigenvalue $\lambda=1$ for   \eqref{eigenvalue problem-ep-new}. Thus,
 \begin{align*}
 &\|\hat\Psi_{1,\ep}\|_{\tilde Y_{\ep,e}}^2=\left(\int_{-1}^{1} \int_0^{6\pi}
+ \int_{-1}^{1} \int_{6\pi}^{(4k+2)\pi}\right)\left({1\over1-\gamma_\ep^2}|\partial_{\theta_\ep}\hat\Psi_{1,\ep}|^2+(1-\gamma_\ep^2)|\partial_{\gamma_\ep}\hat\Psi_{1,\ep}|^2\right)d \theta_\ep d\gamma_\ep\\
=&{4\over 9}\int_{-1}^{1} \int_0^{6\pi}\sin^2\left({1\over 3}\theta_{\ep}\right)(1-\gamma_{\ep}^2)^{1\over 3}d\theta_\ep d\gamma_\ep+
2(k-1)\times 2 \int_{-1}^{1} \int_{0}^{2\pi}\sin^2(\theta_\ep)(1-\gamma_{\ep}^2)d\theta_\ep d\gamma_\ep\\
\leq &
{8\over3}\pi+{16\over3}(k-1)\pi<\infty,
 \end{align*}
 and moreover,
 \begin{align*}
 \int_{0}^{(4k+2)\pi}\hat\Psi_{1,\ep}(\theta_\ep,0)d\theta_\ep=
 \int_0^{6\pi}\sin\left({1\over 3}\theta_{\ep}\right)d\theta_\ep +
 \int_{6\pi}^{(4k+2)\pi}\sin(\theta_\ep)d\theta_\ep=0.
 \end{align*}
\if0
  By Corollary  \ref{kernel of  the operator tilde A-ep and a decomposition of tilde Xep},
 $\cos\left(\theta_\ep\right)(1-\gamma_\ep^2)^{1\over2}\in \ker(\tilde A_\ep)$,
 and thus,
 \begin{align}\label{hat-psi-4kpi-4k+2pi}
 -\Delta \hat{\psi}_\ep = g'(\psi_\ep) \hat{\psi}_\ep\quad \text{for} \quad (x,y) \in [4k\pi, (4k+2)\pi]\times \mathbb{R}.
 \end{align}
  By Lemma
\ref{sol to eigenvalue problem non-zero modes varepsilon=0}, $(1-\gamma_\ep^2)^{1\over2}$ is an
 eigenfunction of the eigenvalue $1$ for \eqref{eigenvalue problem2 non-zero modes varepsilon=0} with $k=1$. This, along with \eqref{laplacian}, gives
$$-(\Delta + g'(\psi_\ep)) \hat{\psi}_\ep = - \frac 1 2 g'(\psi_\ep) \left( \frac 3 4 \frac{\hat{\psi}_\ep}{1-\gamma_\ep^2} \right),\;\;(x,y) \in [0, 4k\pi]\times \mathbb{R}.$$
Then
\begin{align}\label{b1-odd-term1}
& \int_{-\infty}^{+\infty} \int_0^{4k\pi} \left(|\nabla \hat{\psi}_\ep|^2 - g'(\psi_\epsilon)\hat{\psi}_\ep^2\right) dx dy  = \int_{-\infty}^{+\infty} \int_0^{4k\pi} - \frac 1 2 g'(\psi_\ep) \left( \frac 3 4 \frac{\hat{\psi}_\ep^2}{1-\gamma_\ep^2} \right)dx dy \\\nonumber
= & - \int_{-1}^{1} \int_0^{4k\pi}  \left( \frac 3 4 \frac{\hat{\psi}_\ep^2}{1-\gamma_\ep^2} \right)d \theta_\ep d\gep
= -3k\pi.
\end{align}
\fi
Again by Theorems \ref{associate_ep0}, \ref{associate_ep-new-variable-original-variable} and \ref{sol to eigenvalue problem varepsilon=0-pde},
\begin{align}\nonumber
b_{\ep, 1}(\hat{\psi}_{1,\ep})
& = \left(\int_{-\infty}^{+\infty} \int_0^{6\pi}
+ \int_{-\infty}^{+\infty} \int_{6\pi}^{(4k+2)\pi}\right) \left(|\nabla \hat{\psi}_{1,\ep}|^2 - g'(\psi_\epsilon)\hat{\psi}_{1,\ep}^2\right) dx dy\\\nonumber
&=\int_{-\infty}^{+\infty} \int_0^{6\pi}
 \left(|\nabla \hat{\psi}_{1,\ep}|^2 - g'(\psi_\epsilon)\hat{\psi}_{1,\ep}^2\right) dx dy\\\nonumber
 &=-{7\over 9}\int_{-\infty}^{+\infty} \int_0^{6\pi}
  g'(\psi_\epsilon)\hat{\psi}_{1,\ep}^2 dx dy\\\nonumber
 &= -{14\over 9}\int_{0}^{6\pi}\sin^2\left({1\over 3}\theta_{\ep}\right)d\theta_\ep\int_{-1}^1(1-\gamma_\ep^2)^{1\over3}d\gamma_\ep\\\label{b-ep-1-hat-psi}
 &\leq -{14\over 9}\times 3\pi\times {42\over 25}=-{196\pi\over 25}\leq -24.61,
\end{align}
where we used the fact that $\int_{-1}^1(1-\gamma_\ep^2)^{1\over3}d\gamma_\ep\geq {42\over 25}$.
By \eqref{steadyv}, $(2j\pi,0)$ and $((2j+1)\pi,0)$ are critical points of $\psi_\ep$  on $\mathbb{T}_{(4k+2)\pi}\times \mathbb{R}$, where $j=0,\cdots,2k$. The  Hessian matrix of $\psi_\ep$ is
\begin{align*}
\left( \begin{array}{cc} {-\ep^2-\ep\cos(x)\cosh(y)\over (\cosh(y)+\ep\cos(x))^2} & {\ep\sin(x)\sinh(y)\over (\cosh(y)+\ep\cos(x))^2} \\ {\ep\sin(x)\sinh(y)\over (\cosh(y)+\ep\cos(x))^2} & {1+\ep\cosh(y)\cos(x)\over (\cosh(y)+\ep\cos(x))^2} \end{array} \right).
\end{align*}
Then $(2j\pi,0)$ is a saddle point of $\psi_\ep$, and $((2j+1)\pi,0)$ is the minimal  point of $\psi_\ep$, since $\psi_\ep(x,y)\to\infty$ as $y\to \pm\infty$ for $x\in\mathbb{T}_{2\pi}$ and $j=0,\cdots,2k$.
Let
\begin{align}\label{def-rho0}
\rho_0 = \psi_\ep(0,0) = \ln\left(\sqrt{\frac{1+\ep}{1-\ep}}\right).
\end{align}
Then $\min \psi_\ep = \psi_\ep((2j+1)\pi, 0) = - \rho_0$.
For $\rho\in[-\rho_0, \rho_0]$, the streamlines are in the trapped regions and  the level set $\Gamma(\rho) = \{(x, y) \in \Omega_{2k+1} | \psi_\ep(x,y) = \rho\} $ has $n_\rho = 2k+1$ closed level curves, i.e.
\begin{align}\label{Gamma-rho}
\Gamma(\rho) =  \bigcup_{i=1}^{n_\rho} \Gamma_i(\rho),
\end{align}
 where $\Gamma_i(\rho)$ corresponds to a periodic orbit inside the $i$-th cat's eyes trapped region.
Since
$\sin\left({1\over 3}\theta_\ep\right)$ is  `odd'   symmetrical about the point $( 3\pi,0)$ and $\sin\left(\theta_\ep\right)$ is  `odd'   symmetrical about the points $( 6\pi+(2j-1)\pi,0)$ for $j=1,\cdots,2k-2$,  we have
$(\hat{P}_{\ep,e} \hat{\psi}_{1,\ep})(x,y)=0$ for $(x,y)$ in the untrapped regions of $\mathbb{T}_{(4k+2)\pi}\times \mathbb{R}$ and the $2$nd, $j$-th trapped regions for $4\leq j\leq 2k+1$, where $k\geq2$.
Now, we compute the projection term for $(x,y)$ in  the $1$st and $3$rd  trapped regions, denoted by $D_{\rm{in},1}$ and $D_{\rm{in},3}$.
Using $x$ as the parameter in the $1$st trapped region,
we  represent the upper  separatrix to be $y(x)=\cosh^{-1}(1+\ep-\ep\cos(x)), x\in[0,2\pi]$ and  the lower  separatrix to be $y(x)=-\cosh^{-1}(1+\ep-\ep\cos(x)), x\in[0,2\pi]$.
Then
\begin{align}\nonumber
b_{\ep, 2}(\hat{\psi}_{1,\ep})
& = \iint_{D_{\rm{in},1}} g'(\psi_\ep) |\hat{P}_{\ep,e} \hat{\psi}_{1,\ep}|^2 dxdy + \iint_{D_{\rm{in},3}} g'(\psi_\ep) |\hat{P}_{\ep,e} \hat{\psi}_{1,\ep}|^2 dxdy\\\nonumber
&=2\iint_{D_{\rm{in},1}} g'(\psi_\ep) |\hat{P}_{\ep,e} \hat{\psi}_{1,\ep}|^2 dxdy
=2\int_{-\rho_0}^{\rho_0} g'(\rho)  \frac{\left|\oint_{\Gamma_1 (\rho)} \frac{\hat{\psi}_{1,\ep}}{|\nabla \psi_\ep|}\right|^2}{\oint_{\Gamma_1 (\rho)} \frac{1}{|\nabla\psi_\ep|}} d\rho\\\nonumber
&\leq2\int_{-\rho_0}^{\rho_0} g'(\rho) \oint_{\Gamma_1 (\rho)} \frac{|\hat{\psi}_{1,\ep}|^2}{|\nabla \psi_\ep|}d\rho=2 \iint_{D_{\rm{in},1}} g'(\psi_\ep) | \hat{\psi}_{1,\ep}|^2 dxdy\\\nonumber
&=2\int_0^{2\pi}\int_{-\cosh^{-1}(1+\ep-\ep\cos(x))}^{\cosh^{-1}(1+\ep-\ep\cos(x))}g'(\psi_\ep)
\sin^2\left(\frac{\theta_\ep}{3}\right)(1-\gamma_\ep^2)^{1\over3}dydx\\\label{b-ep-2-hat-psi-1-ep}
&\triangleq b_{\ep,3}(\hat{\psi}_{1,\ep}).
\end{align}
To study the monotonicity of $ b_{\ep,3}(\hat{\psi}_{1,\ep})$ with respect to $\ep\in[0,1)$, we need the following lemma.
\begin{lemma}\label{D-theta-gamma-ep-nest}
Let
\begin{align*}
D_{xy,\ep}=&D_{\rm{in},1}=\{(x,y)|{-\cosh^{-1}(1+\ep-\ep\cos(x))}\leq y\leq {\cosh^{-1}(1+\ep-\ep\cos(x))}, x\in\mathbb{T}_{2\pi}\}\\
D_{\theta_\ep\gamma_\ep,\ep}=&\{(\theta_\ep,\gamma_{\ep})|\theta_\ep=\theta_\ep(x,y),\gamma_{\ep}=\gamma_{\ep}(x,y),(x,y)\in D_{xy,\ep}\}
\end{align*}
for $\ep\in[0,1)$. Then as subsets of $\mathbb{T}_{2\pi}\times[-1,1]$, we have
 \begin{align}\label{D1D2-constant}
 D_{\theta_{\ep_1}\gamma_{\ep_1},\ep_1}\subset D_{\theta_{\ep_2}\gamma_{\ep_2},\ep_2} \quad\text{ for }\quad0\leq \ep_1\leq \ep_2<1.
 \end{align}
\end{lemma}

\begin{proof} It suffices to consider the case $y\geq0\Longleftrightarrow\gamma_\ep\geq0$, since $D_{xy,\ep}$ (resp. $D_{\theta_\ep\gamma_\ep,\ep}$) is symmetric with respect to the line $y=0$ (resp. $\gamma_\ep=0$). Instead of using  $(\theta_\ep,\gamma_{\ep})$ directly, we choose the equivalent variables $(\xi_\ep,\eta_\ep)$ and define
\begin{align*}
D_{\xi_\ep\eta_\ep,\ep}=&\{(\xi_\ep,\eta_\ep)|
\eta_\ep = \sqrt{1-\gamma_\ep^2} \sin(\theta_\ep),
\xi_\ep = \sqrt{1-\gamma_\ep^2} \cos(\theta_\ep),
(\theta_\ep,\gamma_{\ep})\in D_{\theta_\ep\gamma_\ep,\ep}\}.
\end{align*}
To prove \eqref{D1D2-constant}, it is sufficient to show that as subsets of the closed unit disk $D_1=\{(\xi_{\ep},\eta_{\ep})|\xi_{\ep}^2+\eta_{\ep}^2\leq1\}$,
\begin{align}\label{D1D2-constant-eta-xi}
D_{\xi_{\ep_1}\eta_{\ep_1},\ep_1}\subset D_{\xi_{\ep_2}\eta_{\ep_2},\ep_2}\quad\text{ for }\quad0\leq \ep_1\leq \ep_2<1.
\end{align}
In the original variables, $D_{xy,\ep}$ consists of the level curves $\{\psi_\ep=\rho\}$ for $\rho\in\bigg[\ln\left(\sqrt{1-\ep\over 1+\ep}\right),$ $\ln\left(\sqrt{1+\ep\over 1-\ep}\right)\bigg]$. In the variables $(\xi_\ep,\eta_\ep)$, we study the level curves of $\omega_\ep$ for convenience. By the expression  \eqref{omega-xi-eta-gamma-ep} of $\omega_\ep$ in $(\xi_\ep,\eta_\ep)$, $D_{\xi_\ep\eta_\ep,\ep}$ consists of the level curves
\begin{align}\label{elliptical}
\left\{(\xi_\ep,\eta_\ep)\bigg|\frac{(\xi_\ep - \ep)^2}{1-\ep^2} + \eta_\ep^2=-c\right\}\bigcap D_1
\end{align}
for $c\in\left[c_\ep,1/c_\ep\right]$, where $c_\ep=-{1+\ep\over 1-\ep}$.
This is a family of ellipses, with the parameters $c$ ranging from $c_\ep$ to $1/c_\ep$, intersecting with the closed unit disk $D_1$. For fixed $c\in\left[c_\ep,1/c_\ep\right]$, the center, semi-major and semi-minor axes of the ellipse are $(\ep,0)$,  $\sqrt{-c}$ and $\sqrt{-c(1-\ep^2)}$. To study the nested relationship \eqref{D1D2-constant-eta-xi}, we use the variables $\xi,\eta\in[-1,1]$, which are independent of $\ep$. Note that as a subset of the closed unit disk $D_1$,
 the curve \eqref{elliptical} is the same one if we replace the variables $(\xi_\ep,\eta_\ep)$ by $(\xi,\eta)$. Thus, $D_{\xi_\ep\eta_\ep,\ep}$ can be written as
\begin{align*}
D_{\xi_\ep\eta_\ep,\ep}=\bigcup_{c\in\left[c_\ep,1/c_\ep\right]} \left(\Gamma_{c,\ep}\cap D_1\right)=
\left\{(\xi,\eta)\bigg|-1/c_\ep\leq\frac{(\xi - \ep)^2}{1-\ep^2} + \eta^2\leq -c_\ep\right\}\bigcap D_1,
\end{align*}
where
\begin{align*}
\Gamma_{c,\ep}=\left\{(\xi,\eta)\bigg|\frac{(\xi - \ep)^2}{1-\ep^2} + \eta^2=-c\right\}.
\end{align*}
To prove \eqref{D1D2-constant-eta-xi},
%we study the boundary curves of $D_{\xi_{\ep}\eta_{\ep},\ep}$.
%The inner boundary of $D_{\xi_{\ep}\eta_{\ep},\ep}$ is   the  ellipse  \eqref{elliptical} with  $c= -{1-\ep\over 1+\ep}$. The outer boundary consists of part of the ellipse  \eqref{elliptical} with  $c= -{1+\ep\over 1-\ep}$, and part of the boundary of the square $[-1,1]\times [-1,1]$.
 we divide our discussions into two steps.
\vspace{0.5mm}

\noindent{\bf{Step 1.}} For $\ep\in[0,1)$, we prove that
\begin{align}\label{out-in-nest}
\Gamma_{1/c_\ep,\ep}\text{  is  enclosed by } S_1, \text{ and }{S}_1\text{  is  enclosed by } \Gamma_{c_\ep,\ep},
\end{align}
where $c_\ep=-{1+\ep\over 1-\ep}$ and $S_1=\{(\xi,\eta)|\xi^2+\eta^2=1\}$ is the unit circle.
\eqref{out-in-nest} means that $\xi^2+\eta^2\leq1$ for $(\xi,\eta)\in\Gamma_{1/c_\ep,\ep}$ and $\frac{(\xi - \ep)^2}{1-\ep^2} + \eta^2\leq-c_\ep$ for $(\xi,\eta)\in S_1$.
See Figure \ref{fig3th:bdy22} for the curves  $\Gamma_{1/c_\ep,\ep}$, $S_1$ and $\Gamma_{c_\ep,\ep}$ with $\ep=0.5$.
Moreover, $\Gamma_{1/c_\ep,\ep}\cap S_1=\{(1,0)\}$ and ${S}_1\cap\Gamma_{c_\ep,\ep}=\{(-1,0)\}$ for $\ep>0$, while $\Gamma_{1/c_\ep,\ep}= S_1=\Gamma_{c_\ep,\ep}$ for $\ep=0$.
\begin{figure}[ht]
    \centering
	\includegraphics[width=0.7\textwidth]{Figure-bdy22.jpg}
	\caption{The curves $\Gamma_{1/c_\ep,\ep}$, $S_1$ and $\Gamma_{c_\ep,\ep}$ with $\ep=0.5$}
	\label{fig3th:bdy22}
\end{figure}


%To prove
%\eqref{out-in-nest}, is equivalent to show that
%\begin{align}
%\Gamma_{1/c_\ep,\ep}\subset D_1\quad\text{and}\quad S_1\subset D_{\xi_\ep\eta_\ep,\ep}.
%\end{align}



$\Gamma_{1/c_\ep,\ep}$ is given by the ellipse
\begin{align}\label{inner boundary}
\frac{(\xi - \ep)^2}{(1-\ep)^2} + {\eta^2\over {1-\ep\over 1+\ep}}=1.
\end{align}
Since the center and semi-minor axis of the ellipse \eqref{inner boundary} are $(\ep,0)$ and $1-\ep$, the right vertex of the ellipse is always $(1,0)$.
%Note that both the semi-major axis $\sqrt{1-\ep\over1+\ep}$  and semi-minor axis $1-\ep$ are decreasing on $\ep\in[0,1)$. Now, we prove that
%\begin{align}\label{Gamma12enclosed}
%\Gamma_{\text{in},\ep_2}\text{  is  enclosed by }\Gamma_{\text{in},\ep_1} \text{ if }0\leq \ep_1\leq \ep_2<1,
%\end{align}
%which means that $\frac{(\xi - \ep_1)^2}{(1-\ep_1)^2} + {\eta^2\over {1-\ep_1\over 1+\ep_1}}\leq1$ for $(\xi,\eta)\in\Gamma_{\text{in},\ep_2}$.
%See Figure \ref{figbdy1} for the curves  $\Gamma_{\text{in},\ep}$ with $\ep=0.4, 0.5$.
%\begin{figure}[ht]
 %   \centering
	%\includegraphics[width=0.7\textwidth]{Figure-bdy1.jpg}
	%\caption{The curves  $\Gamma_{\text{in},\ep}$ with $\ep=0.4, 0.5$}
	%\label{figbdy1}
%\end{figure}
Here, we only need to consider $\eta\geq0$ since $D_{\xi_{\ep}\eta_{\ep},\ep}$ is symmetric with respect to the line $\eta=0$.
For $(\xi,\eta)\in \Gamma_{1/c_\ep,\ep}$ with $\eta\geq0$, we rewrite $\eta$ by $\eta_{1/c_\ep,\ep}(\xi)$ to indicate its dependence on  $\ep$, $c_\ep$ and $\xi$. Then $\eta_{1/c_\ep,\ep}(\xi)^2={1-\ep\over 1+\ep}-\frac{(\xi - \ep)^2}{1-\ep^2}$ for $\xi\in[2\ep-1,1]$. For  $(\xi,\eta)\in S_1$, we rewrite $\eta$ by $\eta_{S_1}(\xi)$ to indicate its dependence on $\xi$. Then $\eta_{S_1}(\xi)^2=1-\xi^2$ for $\xi\in[-1,1]$.
To prove that $\Gamma_{1/c_\ep,\ep}\text{  is  enclosed by } S_1$ and
$\Gamma_{1/c_\ep,\ep}\cap S_1=\{(1,0)\}$  for $\ep>0$, it suffices to show that $\eta_{S_1}(\xi)^2>\eta_{1/c_\ep,\ep}(\xi)^2$ for $\xi\in[\ep,1)$.
Since the right vertex of both the ellipse $\Gamma_{1/c_\ep,\ep}$ and the unit circle $S_1$ is  $(1,0)$, it suffices to verify that $\left|\partial_{\xi}\left(\eta_{S_1}(\xi)^2\right)\right|>\left|\partial_{\xi}\left(\eta_{1/c_\ep,\ep}(\xi)^2\right)\right|$ for $\xi\in[\ep,1]$. In fact, direct computation gives
\begin{align*}
&\left|\partial_{\xi}\left(\eta_{1/c_\ep,\ep}(\xi)^2\right)\right|-\left|\partial_{\xi}\left(\eta_{S_1}(\xi)^2\right)\right|=2\left(\frac{\xi - \ep}{1-\ep^2}-\xi\right)
={-2\ep(1-\ep\xi)\over 1-\ep^2}<0
\end{align*}
 for $\xi\in[\ep,1]$ and $\ep>0$.

$\Gamma_{c_\ep,\ep}$ is given by the ellipse
\begin{align}\label{boundary2}
\frac{(\xi - \ep)^2}{(1+\ep)^2} + {\eta^2\over {1+\ep\over 1-\ep}}=1.
\end{align}
Since the center and semi-minor axis of the ellipse \eqref{boundary2} are $(\ep,0)$ and $1+\ep$, the left vertex of the ellipse is always $(-1,0)$.
%Note that both the semi-major axis $\sqrt{1-\ep\over1+\ep}$  and semi-minor axis $1-\ep$ are decreasing on $\ep\in[0,1)$. Now, we prove that
%\begin{align}\label{Gamma12enclosed}
%\Gamma_{\text{in},\ep_2}\text{  is  enclosed by }\Gamma_{\text{in},\ep_1} \text{ if }0\leq \ep_1\leq \ep_2<1,
%\end{align}
%which means that $\frac{(\xi - \ep_1)^2}{(1-\ep_1)^2} + {\eta^2\over {1-\ep_1\over 1+\ep_1}}\leq1$ for $(\xi,\eta)\in\Gamma_{\text{in},\ep_2}$.
%See Figure \ref{figbdy1} for the curves  $\Gamma_{\text{in},\ep}$ with $\ep=0.4, 0.5$.
%\begin{figure}[ht]
 %   \centering
	%\includegraphics[width=0.7\textwidth]{Figure-bdy1.jpg}
	%\caption{The curves  $\Gamma_{\text{in},\ep}$ with $\ep=0.4, 0.5$}
	%\label{figbdy1}
%\end{figure}
Here we only consider $\eta\geq0$ by  symmetry.
For $(\xi,\eta)\in \Gamma_{c_\ep,\ep}$ with $\eta\geq0$, we rewrite $\eta$ by $\eta_{c_\ep,\ep}(\xi)$. Then $\eta_{c_\ep,\ep}(\xi)^2={1+\ep\over 1-\ep}-\frac{(\xi - \ep)^2}{1-\ep^2}$ for $\xi\in[-1,1+2\ep]$. For  $(\xi,\eta)\in S_1$, $\eta_{S_1}(\xi)^2=1-\xi^2$ for $\xi\in[-1,1]$.
To prove that $S_1\text{  is  enclosed by } \Gamma_{c_\ep,\ep}$ and ${S}_1\cap\Gamma_{c_\ep,\ep}=\{(-1,0)\}$ for $\ep>0$, it suffices to show that $\eta_{c_\ep,\ep}(\xi)^2>\eta_{S_1}(\xi)^2$ for $\xi\in(-1,0]$.
Since the left vertex of both the ellipse $\Gamma_{c_\ep,\ep}$ and the unit circle $S_1$ is  $(-1,0)$, it suffices to verify that $\left|\partial_{\xi}\left(\eta_{c_\ep,\ep}(\xi)^2\right)\right|>\left|\partial_{\xi}\left(\eta_{S_1}(\xi)^2\right)\right|$ for $\xi\in[-1,0]$. Indeed,
\begin{align*}
&\left|\partial_{\xi}\left(\eta_{c_\ep,\ep}(\xi)^2\right)\right|-\left|\partial_{\xi}\left(\eta_{S_1}(\xi)^2\right)\right|=2\left(\frac{ \ep-\xi}{1-\ep^2}+\xi\right)
={2\ep(1-\ep\xi)\over 1-\ep^2}>0
\end{align*}
 for $\xi\in[-1,0]$ and $\ep>0$.

By Step 1,
\begin{align*}
D_{\xi_\ep\eta_\ep,\ep}=\left\{(\xi,\eta)\bigg|\xi^2+\eta^2\leq 1\leq\frac{(\xi - \ep)^2}{(1-\ep)^2} + {\eta^2\over {1-\ep\over 1+\ep}} \right\}.
\end{align*}
In other words, the outer boundary of $D_{\xi_\ep\eta_\ep,\ep}$ is always the unit circle $S_1$ and the inner boundary of $D_{\xi_\ep\eta_\ep,\ep}$ is the ellipse $\Gamma_{1/c_\ep,\ep}$. For $\ep=0.5$, see Figure \ref{Figure4th:orig-and-tran} for the upper trapped region $\{(x,y)\in D_{xy,\ep}|y\geq0\}$ in $(x,y)$ coordinate and  the corresponding region $D_{\xi_\ep\eta_\ep,\ep}$ in $(\xi,\eta)$ coordinate separately.

\begin{figure}[ht]
    \centering
\includegraphics[width=0.48\textwidth]{Figure-orig.jpg}
\includegraphics[width=0.42\textwidth]{Figure-tran.jpg}
\caption{Upper trapped region with $\ep=0.5$}
	\label{Figure4th:orig-and-tran}
\end{figure}
We point out the correspondence of  the streamlines and boundary of the upper trapped region  between the $(x,y)$ and $(\xi,\eta)$ coordinates.
\begin{itemize}
 \item For $\rho=\ln\left(\sqrt{1-\ep\over 1+\ep}\right)$, the streamline is the point $(\pi,0)$ in the $(x,y)$  coordinate, and is transformed to the  point $(-1,0)$ in the $(\xi,\eta)$ coordinate.
    \item    For $\rho=\ln\left(\sqrt{1+\ep\over 1-\ep}\right)$,  the upper
separatrix is transformed to the whole ellipse $\Gamma_{1/c_\ep,\ep}$ (the inner boundary of $D_{\xi_\ep\eta_\ep,\ep}$) in the $(\xi,\eta)$ coordinate.
\item For $\rho\in\left(\ln\left(\sqrt{1-\ep\over 1+\ep}\right),\ln\left(\sqrt{1+\ep\over 1-\ep}\right)\right)$, the upper part of the  streamline $\{\psi_\ep=\rho\}$ is transformed to the part of the ellipse $\Gamma_{-e^{-2\rho},\ep}\cap D_1$  in the $(\xi,\eta)$ coordinate, see the red curves in Figure \ref{Figure4th:orig-and-tran}.
    \item The boundary $\{y=0,x\in\mathbb{T}_{2\pi}\}$ in the $(x,y)$  coordinate is transformed to the unit circle $S_1$ (the outer boundary of $D_{\xi_\ep\eta_\ep,\ep}$) in the $(\xi,\eta)$ coordinate.
 \end{itemize}
 \vspace{0.5mm}

\noindent{\bf{Step 2.}} For $\ep\in[0,1)$, we prove the nested property for the inner boundary $\Gamma_{1/c_\ep,\ep}$ of $D_{\xi_{\ep}\eta_{\ep},\ep}$:
\begin{align}\label{Gamma12enclosed}
\Gamma_{1/c_{\ep_2},{\ep_2}}\text{  is  enclosed by }\Gamma_{1/c_{\ep_1},\ep_1} \quad\text{ if }\quad0\leq \ep_1< \ep_2<1.
\end{align}
See Figure \ref{fig5th:bdy1} for the curves  $\Gamma_{1/c_\ep,\ep}$ with $\ep=0.4, 0.5$.

%$\Gamma_{1/c_\ep,\ep}$ is given by the ellipse
%\begin{align}\label{inner boundary}
%\frac{(\xi - \ep)^2}{(1-\ep)^2} + {\eta^2\over {1-\ep\over 1+\ep}}=1.
%\end{align}
%Since the center and semi-minor axis of the ellipse \eqref{inner boundary} are $(\ep,0)$ and $1-\ep$, the right vertex of the ellipse is always $(1,0)$.
By \eqref{inner boundary}, both the semi-major axis $\sqrt{1-\ep\over1+\ep}$  and semi-minor axis $1-\ep$ of $\Gamma_{1/c_{\ep},{\ep}}$ are decreasing on $\ep\in[0,1)$. Here
we only need to consider $\eta\geq0$ by symmetry.
Recall that $\eta_{1/c_\ep,\ep}(\xi)^2={1-\ep\over 1+\ep}-\frac{(\xi - \ep)^2}{1-\ep^2}, \xi\in[2\ep-1,1]$ for $(\xi,\eta_{1/c_\ep,\ep}(\xi))\in \Gamma_{1/c_{\ep},{\ep}}$.
\begin{figure}[ht]
    \centering
	\includegraphics[width=0.7\textwidth]{Figure-bdy1.jpg}
	\caption{The curves  $\Gamma_{1/c_\ep,\ep}$ with $\ep=0.4, 0.5$}
	\label{fig5th:bdy1}
\end{figure}
To prove \eqref{Gamma12enclosed}, we will show that $\eta_{1/c_{\ep_1},\ep_1}(\xi)^2>\eta_{1/c_{\ep_2},\ep_2}(\xi)^2$ for $\xi\in[\ep_2,1)$.
Since the right vertex of the ellipse $\Gamma_{1/c_{\ep},{\ep}}$ is  $(1,0)$ for $\ep\in[0,1)$, it suffices to verify that $\left|\partial_{\xi}\left(\eta_{1/c_{\ep_1},\ep_1}(\xi)^2\right)\right|>\left|\partial_{\xi}\left(\eta_{1/c_{\ep_2},\ep_2}(\xi)^2\right)\right|$ for $\xi\in[\ep_2,1]$. In fact,
\begin{align*}
&\left|\partial_{\xi}\left(\eta_{1/c_{\ep_2},\ep_2}(\xi)^2\right)\right|-\left|\partial_{\xi}\left(\eta_{1/c_{\ep_1},\ep_1}(\xi)^2\right)\right|=2\left(\frac{\xi - \ep_2}{1-\ep_2^2}-\frac{\xi - \ep_1}{1-\ep_1^2}\right)\\
=&2{(\ep_2-\ep_1)\left((\ep_1+\ep_2)\xi-1-\ep_1\ep_2\right)\over (1-\ep_2^2)(1-\ep_1^2)}\leq 2{(\ep_2-\ep_1)\left(\ep_1+\ep_2-1-\ep_1\ep_2\right)\over (1-\ep_2^2)(1-\ep_1^2)}\\
=&2{(\ep_2-\ep_1)(\ep_1-1)(1-\ep_2)\over (1-\ep_2^2)(1-\ep_1^2)}<0
\end{align*}
 for $\xi\in[\ep_2,1]$ and $0\leq \ep_1< \ep_2<1$.

 \vspace{0.5mm}



\if0
The outer boundary of $D_{\xi_{\ep}\eta_{\ep},\ep}$ is given by the ellipse
\begin{align*}
\Gamma_{\text{out},\ep}=\left\{(\xi,\hat\eta_\ep(\xi))\bigg|\;\hat\eta_\ep(\xi)\in\{\pm\tilde \eta_\ep(\xi)\} \text{ if } |\tilde \eta_\ep(\xi)|<1, \hat\eta_\ep(\xi)\in\{\pm1\} \text{ if } |\tilde \eta_\ep(\xi)|\geq1, \xi\in[-1,1]\right\},
\end{align*}
where
$
\tilde \eta_\ep(\xi)=\sqrt{{1+\ep\over 1-\ep}-{(\xi-\ep)^2\over 1-\ep^2}}.
$
Instead of studying the outer boundary directly, we consider the ellipses
\begin{align}\label{tilde-Gamma-out}
\tilde \Gamma_{\text{out},\ep}=\left\{(\xi,\eta)\bigg|\;\frac{(\xi - \ep)^2}{(1+\ep)^2} + {\eta^2\over {1+\ep\over 1-\ep}}=1\right\}
\end{align}
for $\ep\in[0,1)$.
Then $\tilde \Gamma_{\text{out},\ep}=\left\{(\xi,\hat\eta_\ep(\xi))\bigg|\;\hat\eta_\ep(\xi)\in\{\pm\tilde \eta_\ep(\xi)\}, \xi\in[-1,1+2\ep]\right\}$.
Since the center and semi-minor axis of the ellipse \eqref{tilde-Gamma-out} are $(\ep,0)$ and $1+\ep$, the left vertex of the ellipse is always $(-1,0)$.
Note that both the semi-major axis $\sqrt{1+\ep\over1-\ep}$  and semi-minor axis $1+\ep$ are increasing on $\ep\in[0,1)$. Next, we prove that
\begin{align}\label{tilde-Gamma12enclosed}
\tilde\Gamma_{\text{in},\ep_1}\text{  is  enclosed by }\tilde\Gamma_{\text{in},\ep_2} \text{ if }0\leq \ep_1\leq \ep_2<1,
\end{align}
%which means that $\frac{(\xi - \ep_1)^2}{(1-\ep_1)^2} + {\eta^2\over {1-\ep_1\over 1+\ep_1}}\leq1$ for $(\xi,\eta)\in\Gamma_{\text{in},\ep_2}$.
 See Figure \ref{figbdy2} for the curves  $\tilde\Gamma_{\text{in},\ep}$ with $\ep=0.4, 0.5$.
\begin{figure}[ht]
    \centering
	\includegraphics[width=0.7\textwidth]{Figure-bdy2.jpg}
	\caption{The curves  $\tilde\Gamma_{\text{in},\ep}$ with $\ep=0.4, 0.5$}
	\label{figbdy2}
\end{figure}
We only  consider $\eta\geq0$ by symmetry.
To prove \eqref{tilde-Gamma12enclosed}, we will show that $\tilde\eta_{\ep_1}(\xi)^2\leq\tilde\eta_{\ep_2}(\xi)^2$ for $\xi\in[-1,\ep_1]$.
Since the left vertex of the ellipses is  $(-1,0)$ for both $\ep_1$ and $\ep_2$, it suffices to verify that $\left|\partial_{\xi}\left(\eta_{\ep_1}(\xi)^2\right)\right|\leq\left|\partial_{\xi}\left(\eta_{\ep_2}(\xi)^2\right)\right|$ for $\xi\in[-1,\ep_1]$. Indeed,
\begin{align*}
&\left|\partial_{\xi}\left(\eta_{\ep_2}(\xi)^2\right)\right|-\left|\partial_{\xi}\left(\eta_{\ep_1}(\xi)^2\right)\right|=2\left(\frac{ \ep_2-\xi }{1-\ep_2^2}-\frac{\ep_1-\xi}{1-\ep_1^2}\right)\\
=&2{(\ep_2-\ep_1)\left(-(\ep_1+\ep_2)x+1+\ep_1\ep_2\right)\over (1-\ep_2^2)(1-\ep_1^2)}\geq 2{(\ep_2-\ep_1)\left(-(\ep_1+\ep_2)\ep_1+1+\ep_1\ep_2\right)\over (1-\ep_2^2)(1-\ep_1^2)}\\
=&2{(\ep_2-\ep_1)(1-\ep_1^2)\over (1-\ep_2^2)(1-\ep_1^2)}=2{\ep_2-\ep_1\over 1-\ep_2^2}\geq0
\end{align*}
 for $\xi\in[-1,\ep_1]$ and $\ep_2\geq\ep_1$.
\fi

By Step 2, we get \eqref{D1D2-constant-eta-xi}, which implies \eqref{D1D2-constant}.
\end{proof}
\begin{Corollary}\label{b3-increasing}
$ b_{\ep,3}(\hat{\psi}_{1,\ep})$ is non-decreasing on $\ep\in[0,1)$.
\end{Corollary}
\begin{proof}
By the definition of $ b_{\ep,3}(\hat{\psi}_{1,\ep})$ in \eqref{b-ep-2-hat-psi-1-ep} and Lemma \ref{D-theta-gamma-ep-nest}, we have
\begin{align*}
b_{\ep_1,3}(\hat{\psi}_{1,\ep_1})=
&2\iint_{D_{xy,\ep_1}}g'(\psi_{\ep_1})
\sin^2\left(\frac{\theta_{\ep_1}}{3}\right)(1-\gamma_{\ep_1}^2)^{1\over3}dxdy\\
=&4\iint_{D_{\theta_{\ep_1}\gamma_{\ep_1},\ep_1}}\sin^2\left(\frac{\theta}{3}\right)(1-\gamma^2)^{1\over3}d\theta d\gamma\\
\leq &4\iint_{D_{\theta_{\ep_2}\gamma_{\ep_2},\ep_2}}\sin^2\left(\frac{\theta}{3}\right)(1-\gamma^2)^{1\over3}d\theta d\gamma\\
=&2\iint_{D_{xy,\ep_2}}g'(\psi_{\ep_2})
\sin^2\left(\frac{\theta_{\ep_2}}{3}\right)(1-\gamma_{\ep_2}^2)^{1\over3}dxdy=b_{\ep_2,3}(\hat{\psi}_{1,\ep_2})
\end{align*}
for $0\leq \ep_1\leq \ep_2<1$.
\end{proof}
%We use Python to find that $ b_{\ep,3}(\hat{\psi}_{1,\ep})$ is increasing on $\ep\in[0,1)$ and
By   splitting  the trapped regions and taking  approximate summation for the integral in $b_{\ep,3}(\hat{\psi}_{1,\ep})|_{\ep={4\over5}}$, we have
\begin{align*}
b_{\ep,3}(\hat{\psi}_{1,\ep})|_{\ep={4\over5}}<24.38.
\end{align*}
It then follows from Corollary \ref{b3-increasing} that
 \begin{align}\label{b-ep-2-hat-psi}
 b_{\ep, 2}(\hat{\psi}_{1,\ep})<24.38 \quad\text{for}\quad\ep\in\left[0,{4\over5}\right].
 \end{align}
 %The graph of  $ b_{\ep,3}(\hat{\psi}_{1,\ep})$ as a function of $\ep$ is given as follows.
%\begin{figure}[ht]
 %   \centering
	%\includegraphics[width=0.56\textwidth]{Figure-b3.jpg}
	%\caption{The values of $ b_{\ep,3}(\hat{\psi}_{1,\ep})$}
	%\label{figb3}
%\end{figure}
Combining \eqref{b-ep-1-hat-psi} and \eqref{b-ep-2-hat-psi}, we have
\begin{align}\label{test-odd-neg}
\langle\hat{A}_{\ep,e} \hat{\psi}_{1,\ep}, \hat{\psi}_{1,\ep} \rangle= b_{\ep, 1}(\hat{\psi}_{1,\ep}) + b_{\ep, 2}(\hat{\psi}_{1,\ep})<-24.61+24.38 =-0.23< 0.
\end{align}
\medskip

\noindent{\bf{Case 2. Test functions for $\ep\in\left({4\over5},1\right)$.}}
\medskip


Let
 \begin{align*}
&{\phi}_{2,\ep}(x,y)= \Phi_{2,\ep}(\theta_\ep,\gamma_\ep)\\\nonumber
=  &\left\{ \begin{array}{ll}
         \cos\left({1\over 2}\theta_\ep\right)(1-\gamma_\ep^2)^{1\over2} & \mbox{if $(\theta_\ep,\gamma_\ep) \in [0, 4k\pi]\times[-1,1]$},\\
         \cos\left(\theta_\ep\right)(1-\gamma_\ep^2)^{1\over2} & \mbox{if $(\theta_\ep,\gamma_\ep) \in \left((4k\pi,(4k+{1\over 2})\pi]\cup((4k+{3\over 2})\pi,(4k+2)\pi]\right)\times[-1,1]$},\\
         0& \mbox{if $(\theta_\ep,\gamma_\ep) \in ((4k+{1\over 2})\pi,(4k+{3\over 2})\pi]\times[-1,1]$}.
        \end{array} \right.
 \end{align*}
 Then
 \begin{align*}
 &\widehat{(\Phi_{2,\ep})}_0(0)={1\over (4k+2)\pi}\int_0^{(4k+2)\pi}\Phi_{2,\ep}(\theta_\ep,0)d\theta_{\ep}\\
 =&{1\over (4k+2)\pi}\left(\int_{4k\pi}^{(4k+{1\over2})\pi}+\int_{(4k+{3\over2})\pi}^{(4k+2)\pi}\right)\cos(\theta_{\ep})d\theta_{\ep}={1\over (2k+1)\pi}.
 \end{align*}
 We choose the test function
 \begin{align}\label{test-odd-2}
\hat{\psi}_{2,\ep}(x,y)= &\hat\Psi_{2,\ep}(\theta_\ep,\gamma_\ep)\triangleq \Phi_{2,\ep}(\theta_\ep,\gamma_\ep)-{1\over (2k+1)\pi}={\phi}_{2,\ep}(x,y)-{1\over (2k+1)\pi}
 \end{align}
 for $(\theta_\ep,\gamma_\ep) \in \mathbb{T}_{(4k+2)\pi}\times[-1,1]$.
 %To show that $ \hat{\psi}_{1,\ep}\in \tilde X_{\ep,e}$,
Then $\hat\Psi_{2,\ep}\in C^0(\tilde \Omega_{2k+1})$ and
 \begin{align*}
 \|\hat\Psi_{2,\ep}\|_{\tilde Y_{\ep,e}}^2=&\left(\int_{-1}^{1} \int_0^{4k\pi}
+ \int_{-1}^{1} \int_{4k\pi}^{(4k+2)\pi}\right)\left({1\over1-\gamma_\ep^2}|\partial_{\theta_\ep}\hat\Psi_{2,\ep}|^2+(1-\gamma_\ep^2)|\partial_{\gamma_\ep}\hat\Psi_{2,\ep}|^2\right)d \theta_\ep d\gamma_\ep\\
=&\left(\int_{-1}^{1} \int_0^{4k\pi}
+ \int_{-1}^{1} \int_{4k\pi}^{(4k+2)\pi}\right)\left({1\over1-\gamma_\ep^2}|\partial_{\theta_\ep}\Phi_{2,\ep}|^2+(1-\gamma_\ep^2)|\partial_{\gamma_\ep}\Phi_{2,\ep}|^2\right)d \theta_\ep d\gamma_\ep\\
= &
k\pi+{1\over3}\pi<\infty.
 \end{align*}
 Moreover,
 \begin{align*}
 \int_{0}^{(4k+2)\pi}\hat\Psi_{2,\ep}(\theta_\ep,0)d\theta_\ep=\int_{0}^{(4k+2)\pi}\left(\Phi_{2,\ep}(\theta_\ep,0)-{1\over (2k+1)\pi}\right)d\theta_\ep=2-2
 =0.
 \end{align*}
Thus, $\hat\Psi_{2,\ep}\in\tilde Y_{\ep,e}$, which implies $ \hat{\psi}_{2,\ep}\in \tilde X_{\ep,e}$. Since $\hat{P}_{\ep,e}{1\over (2k+1)\pi}={1\over (2k+1)\pi}$, we have
\begin{align}\nonumber
\langle\hat{A}_{\ep,e} \hat{\psi}_{2,\ep}, \hat{\psi}_{2,\ep}\rangle
&= \iint_{\Omega_{2k+1}} \left( |\nabla \hat{\psi}_{2,\ep}|^2 - g'(\psi_\ep)((I - \hat{P}_{\ep,e})\hat{\psi}_{2,\ep})^2\right) dxdy\\\nonumber
&=\iint_{\Omega_{2k+1}}\left( |\nabla {\phi}_{2,\ep}|^2 - g'(\psi_\ep)((I - \hat{P}_{\ep,e}){\phi}_{2,\ep})^2\right) dxdy\\\label{Aep-psi-2-ep-2}
&=b_{\ep, 1}({\phi}_{2,\ep}) + b_{\ep, 2}({\phi}_{2,\ep}).
\end{align}
 By Corollary  \ref{kernel of  the operator A-ep and a decomposition of tilde Xep},
 $\cos\left(\theta_\ep\right)(1-\gamma_\ep^2)^{1\over2}\in \ker( A_\ep)$,
 and thus,
 \begin{align}\label{hat-psi-4kpi-4k+2pi-2}
-{1\over1-\gamma_\ep^2}\pa_{\theta_\ep}^2\Phi_{2,\ep}-\pa_{\gamma_\ep}\left((1-\gamma_\ep^2)\pa_{\gamma_\ep}\Phi_{2,\ep}\right)
=2\Phi_{2,\ep}
 \end{align}
 for $(\theta_\ep,\gamma_\ep) \in ((4k\pi,(4k+{1\over 2})\pi]\cup((4k+{3\over 2})\pi,(4k+2)\pi])\times[-1,1]$.
  By Lemma
\ref{sol to eigenvalue problem non-zero modes varepsilon=0 original}, $(1-\gamma_\ep^2)^{1\over2}$ is an
 eigenfunction of the eigenvalue $1$ for \eqref{eigenvalue problem2 non-zero modes varepsilon=0} with $k=1$. This, along with \eqref{laplacian}, gives
$$-(\Delta + g'(\psi_\ep)) {\phi}_{2,\ep} = - \frac 1 2 g'(\psi_\ep) \left( \frac 3 4 \frac{{\Phi}_{2,\ep}}{1-\gamma_\ep^2} \right),\;\;(x,y) \in [0, 4k\pi]\times \mathbb{R}.$$
Then
\begin{align}\nonumber
& \int_{-\infty}^{+\infty} \int_0^{4k\pi} \left(|\nabla {\phi}_{2,\ep}|^2 - g'(\psi_\epsilon){\phi}_{2,\ep}^2\right) dx dy  = \int_{-\infty}^{+\infty} \int_0^{4k\pi} - \frac 1 2 g'(\psi_\ep) \left( \frac 3 4 \frac{{\Phi}_{2,\ep}^2}{1-\gamma_\ep^2} \right)dx dy \\\label{b1-odd-term1-2}
= & - \int_{-1}^{1} \int_0^{4k\pi}  \left( \frac 3 4 \frac{{\Phi}_{2,\ep}^2}{1-\gamma_\ep^2} \right)d \theta_\ep d\gep
= -3k\pi.
\end{align}
Combining \eqref{hat-psi-4kpi-4k+2pi-2} and \eqref{b1-odd-term1-2}, we have
\begin{align}\nonumber
b_{\ep,1}(\phi_{2,\ep})
=&  \left( \int_{-\infty}^{+\infty} \int_0^{4k\pi}+\int_{-\infty}^{+\infty} \int_{4k\pi}^{(4k+2)\pi}\right)\left(
|\nabla {\phi}_{2,\ep}|^2 - g'(\psi_\epsilon){\phi}_{2,\ep}^2\right) dx dy\\\nonumber
=& -3k\pi+\left(\int_{-1}^1\int_0^{{\pi\over2}}+\int_{-1}^1\int_{3\pi\over2}^{2\pi}\right)\bigg(
{1\over1-\gamma_\ep^2}|\pa_{\theta_\ep}\Phi_{2,\ep}|^2\\\nonumber
&+(1-\gamma_\ep^2)|\pa_{\gamma_\ep}\Phi_{2,\ep}|^2-2|\Phi_{2,\ep}|^2\bigg)d\theta_\ep d\gamma_\ep\\\label{com-b-ep-1-phi-2-ep}
=&-3k\pi.
\end{align}
Since
$\cos\left({1\over 2}\theta_\ep\right)$ is  `odd'   symmetrical about the points $((2j-1)\pi,0)$  for $j=1,\cdots,2k$,  we have
$\hat{P}_{\ep,e} \hat{\psi}_{2,\ep}(x,y)=0$ for $(x,y)$ in the $j$-th trapped region of
$\mathbb{T}_{(4k+2)\pi}\times \mathbb{R}$, where $1\leq j\leq2k$.
Next, we compute the projection term for $(x,y)$ in  the $(2k+1)$-th trapped region, denoted by $D_{{\rm{in}},2k+1}$.
Using $x$ as the parameter,
we  represent the upper and lower  separatrix to be $y(x)=\cosh^{-1}(1+\ep-\ep\cos(x)), x\in[4k\pi,(4k+2)\pi]$ and  $y(x)=-\cosh^{-1}(1+\ep-\ep\cos(x)), x\in[4k\pi,(4k+2)\pi]$, respectively.
Then
\begin{align*}
&\iint_{D_{{\rm{in}},2k+1}} g'(\psi_\ep) |\hat{P}_{\ep,e} {\phi}_{2,\ep}|^2 dxdy
=\int_{-\rho_0}^{\rho_0} g'(\rho)  \frac{\left|\oint_{\Gamma_{2k+1} (\rho)} \frac{{\phi}_{2,\ep}}{|\nabla \psi_\ep|}\right|^2}{\oint_{\Gamma_{2k+1} (\rho)} \frac{1}{|\nabla\psi_\ep|}} d\rho\\
\leq&\int_{-\rho_0}^{\rho_0} g'(\rho) \oint_{\Gamma_{2k+1} (\rho)} \frac{|{\phi}_{2,\ep}|^2}{|\nabla \psi_\ep|}d\rho= \iint_{D_{{\rm{in}},2k+1}} g'(\psi_\ep) | {\phi}_{2,\ep}|^2 dxdy\\
%=&\int_0^{2\pi}\int_{-\cosh^{-1}(1+\ep-\ep\cos(x))}^{\cosh^{-1}(1+\ep-\ep\cos(x))} g'(\psi_\ep) \cos^2\left(\theta_\ep\right)(1-\gamma_\ep^2) dydx\\
%\triangleq&b_{\ep,4}({\phi}_{2,\ep}).
\leq &\iint_{\Omega_{2k+1}\setminus\Omega_{2k}}g'(\psi_\ep)| {\phi}_{2,\ep}|^2dxdy
=2\int_{-1}^1\left(\int_0^{\pi\over 2}+\int_{3\pi\over2}^{2\pi}\right)\cos^2\left(\theta_\ep\right)(1-\gamma_\ep^2)d\theta_\ep d\gamma_\ep\\
=&{4\over3} \pi,
%&\int_{-\cosh^{-1}(1+\ep-\ep\cos(x))}^{\cosh^{-1}(1+\ep-\ep\cos(x))}\int_0^{2\pi}g'(\psi_\ep)
%\cos^2\left(\theta_\ep\right)(1-\gamma_\ep^2)dxdy\\
%\triangleq& b_{4,\ep}({\phi}_{2,\ep}).
\end{align*}
where $\rho_0$ and $\Gamma_{2k+1} (\rho)$ are defined in \eqref{def-rho0} and \eqref{Gamma-rho}.
Now, we compute the projection term for $(x,y)$ in  the untrapped region, denoted by $D_{c}$.
\begin{align*}
&\iint_{D_{c}} g'(\psi_\ep) |\hat{P}_{\ep,e} {\phi}_{2,\ep}|^2 dxdy
=(2k+1)\left(\iint_{\Omega_{2k+1}\setminus (\Omega_{2k}\cup D_{{\rm{in}},2k+1})} g'(\psi_\ep) |\hat{P}_{\ep,e} {\phi}_{2,\ep}|^2 dxdy\right)\\
\leq&(2k+1)\left(\iint_{\Omega_{2k+1}\setminus (\Omega_{2k}\cup D_{{\rm{in}},2k+1})} g'(\psi_\ep) | {\phi}_{2,\ep}|^2 dxdy\right)\\
\leq &(2k+1)\left(\iint_{\Omega_{2k+1}\setminus (\Omega_{2k}\cup D_{{\rm{in}},2k+1})} g'(\psi_\ep)\cos^2\left(\theta_\ep\right)(1-\gamma_\ep^2)dxdy\right)\\
= &(2k+1)\left({8\over3} \pi-\iint_{D_{{\rm{in}},2k+1}} g'(\psi_\ep) \cos^2\left(\theta_\ep\right)(1-\gamma_\ep^2)dxdy\right)\\
=&(2k+1)\left({8\over3} \pi-\int_0^{2\pi}\int_{-\cosh^{-1}(1+\ep-\ep\cos(x))}^{\cosh^{-1}(1+\ep-\ep\cos(x))} g'(\psi_\ep) \cos^2\left(\theta_\ep\right)(1-\gamma_\ep^2) dydx\right)\\
\triangleq&(2k+1)\left({8\over3} \pi-b_{\ep,4}({\phi}_{2,\ep})\right).
%&\int_{-\cosh^{-1}(1+\ep-\ep\cos(x))}^{\cosh^{-1}(1+\ep-\ep\cos(x))}\int_0^{2\pi}g'(\psi_\ep)
%\cos^2\left(\theta_\ep\right)(1-\gamma_\ep^2)dxdy\\
%\triangleq& b_{4,\ep}({\phi}_{2,\ep}).
\end{align*}
Thus,
\begin{align}\nonumber
 b_{\ep, 2}({\phi}_{2,\ep})=&\iint_{D_{{\rm{in}},2k+1}} g'(\psi_\ep) |\hat{P}_{\ep,e} {\phi}_{2,\ep}|^2 dxdy+\iint_{D_{c}} g'(\psi_\ep) |\hat{P}_{\ep,e} {\phi}_{2,\ep}|^2 dxdy\\\label{est-b-ep-2-phi-2-ep}
 \leq &{4\over3} \pi+(2k+1)\left({8\over3} \pi-b_{\ep,4}({\phi}_{2,\ep})\right).
 \end{align}
 \begin{Corollary}\label{b4-increasing}
 $b_{\ep,4}({\phi}_{2,\ep})$ is non-decreasing on $\ep\in[0,1)$.
\end{Corollary}
\begin{proof}
By the definition of $ b_{\ep,4}(\phi_{2,\ep})$ and Lemma \ref{D-theta-gamma-ep-nest}, we have
\begin{align*}
b_{\ep_1,4}(\phi_{2,\ep_1})=
&\iint_{D_{xy,\ep_1}}g'(\psi_{\ep_1}) \cos^2\left(\theta_{\ep_1}\right)(1-\gamma_{\ep_1}^2)dxdy\\
=&2\iint_{D_{\theta_{\ep_1}\gamma_{\ep_1},\ep_1}}\cos^2\left(\theta\right)(1-\gamma^2)d\theta d\gamma\\
\leq &2\iint_{D_{\theta_{\ep_2}\gamma_{\ep_2},\ep_2}}\cos^2\left(\theta\right)(1-\gamma^2)d\theta d\gamma\\
=&\iint_{D_{xy,\ep_2}}g'(\psi_{\ep_2})
\cos^2\left(\theta_{\ep_2}\right)(1-\gamma_{\ep_2}^2)dxdy=b_{\ep_2,4}(\phi_{2,\ep_2})
\end{align*}
for $0\leq \ep_1\leq \ep_2<1$.
\end{proof}
Since
\begin{align*}
b_{\ep,4}({\phi}_{2,\ep})|_{\ep={4\over5}}>6.94,
\end{align*}
by Corollary \ref{b4-increasing} we have $\min_{\ep\in[{4\over5},1)}b_{\ep,4}({\phi}_{2,\ep})>6.94.$
%The graph of  $ b_{\ep,4}({\phi}_{2,\ep})$ as a function of $\ep$ is given as follows.
%\begin{figure}[ht]
 %   \centering
	%\includegraphics[width=0.56\textwidth]{Figure-b4.png}
	%\caption{The values of $ b_{\ep,4}({\phi}_{2,\ep})$}
	%\label{figb3}
%\end{figure}
Then it follows from \eqref{est-b-ep-2-phi-2-ep} that
\begin{align}\label{est-b-ep-2-phi-2-ep-2}
 b_{\ep, 2}({\phi}_{2,\ep})
 \leq &{4\over3} \pi+(2k+1)\left({8\over3} \pi-6.94\right),\;\ep\in \left[{4\over5},1\right).
 \end{align}
By \eqref{Aep-psi-2-ep-2}, \eqref{com-b-ep-1-phi-2-ep} and \eqref{est-b-ep-2-phi-2-ep-2}, we have
 \begin{align}\nonumber
 &\langle\hat{A}_{\ep,e} \hat{\psi}_{2,\ep}, \hat{\psi}_{2,\ep}\rangle
=b_{\ep, 1}({\phi}_{2,\ep}) + b_{\ep, 2}({\phi}_{2,\ep})\leq -3k\pi+{4\over3} \pi+(2k+1)\left({8\over3} \pi-6.94\right)\\\label{A-ep-e-psi-2-neg}
=& \left({7\over 3}\pi-13.88\right)k+4\pi-6.94\leq {19\over 3}\pi-20.82<0
 \end{align}
 for $k\geq1$ and $\ep\in \left({4\over5},1\right)$.
 \medskip

 Combining Case 1  and Case 2, we obtain linear instability of $\omega_\ep$ for perturbations with odd multiples of the period.

\begin{Theorem}\label{multi-odd}
Let $\ep \in [0, 1)$. Then the steady state $\omega_\ep$ is linearly unstable for $(4k + 2)\pi$-periodic perturbations, where $k\geq1$ is an integer.
\end{Theorem}
\begin{proof}
For $\ep\in\left[0,{4\over5}\right]$, we define
 the test function to be $\hat{\psi}_{1,\ep}$   in \eqref{test-odd}.  By \eqref{test-odd-neg}, we have
$\langle\hat{A}_{\ep,e} \hat{\psi}_{1,\ep}, \hat{\psi}_{1,\ep} \rangle < 0.$
 For $\ep\in\left({4\over5},1\right)$, we define
 the test function to be $\hat{\psi}_{2,\ep}$   in \eqref{test-odd-2}.  By \eqref{A-ep-e-psi-2-neg}, we have
$\langle\hat{A}_{\ep,e} \hat{\psi}_{2,\ep}, \hat{\psi}_{2,\ep} \rangle < 0.$ Thus, $n^-\left( L_{\ep,e} |_{\overline{R(B_\ep)}} \right) = n^-\left(\hat{A}_{\ep,e}\right)\geq1$ for $\ep\in[0,1)$ by Lemma \ref{L e-hat A}. Then linear instability is obtained by applying Lemma  \ref{indice-theorem-sep}.
\end{proof}
\begin{remark}\label{number of unstable eigenvalues for multi-periodic perturbations}
%$(1)$ Although we use Python to estimate
%$b_{\ep,3}(\hat{\psi}_{1,\ep})|_{\ep={4\over5}}$ and
%$b_{\ep,4}({\phi}_{2,\ep})|_{\ep={4\over5}}$ in \eqref{bep3hatps1ep4over5} and \eqref{bep4hatps2ep4over5},
%the proof can be analytical by using  splitting  the trapped regions and taking  approximate summation.

%Thus, the proof of Theorem \ref{multi-odd} is almost analytical, and the  only computer assistant part is to check the monotonicity of
%$ b_{\ep,3}(\hat{\psi}_{1,\ep})$ and $ b_{\ep,4}({\phi}_{2,\ep})$ with respect to $\ep$, see Figures 3-4. The monotonicity is intuitively reasonable, since the area of the trapped region is increasing and the values of the main term $g'(\psi_\ep)$ in the integrand are larger
%near  the central point of the trapped region and smaller in the untrapped regions as $\ep $ increases.

%If we compute $ b_{\ep,2}(\hat{\psi}_{1,\ep})$ and $ b_{\ep,2}({\phi}_{2,\ep})$ directly, it turns to be very involved to handle the orbital integrals.
%We emphasize that in our method, there exist no projection terms in $ b_{\ep,3}(\hat{\psi}_{1,\ep})$ and $ b_{\ep,4}({\phi}_{2,\ep})$, which makes the computation more feasible and simpler.
$(1)$ For $\ep\in\left[0,{4\over5}\right]$, we use
 the test function $\hat{\psi}_{1,\ep}$ to get a negative direction of
$\hat{A}_{\ep,e}.$ A conjecture is that $\hat{\psi}_{1,\ep}$ is always a negative direction of
$\hat{A}_{\ep,e}$ for $\ep\in\left[0,1\right)$. The difficulty to prove or disprove this conjecture  is how to accurately compute or estimate the projection term in a rigorous way.


$(2)$
For $\ep=0$, the number of unstable eigenvalues of the linearized vorticity operator is $2(m-1)$. Indeed, on the one hand,
since
\begin{align*}
\langle\tilde A_{0,e}\psi,\psi\rangle=&\iint_{\Omega_m}\left(|\nabla\psi|^2-g'(\psi_0)\psi^2\right)dxdy+{\left(\iint_{\Omega_m}g'(\psi_0)\widehat\psi_0dxdy\right)^2\over \iint_{\Omega_m}g'(\psi_0)dxdy}\\
\leq &\iint_{\Omega_m}\left(|\nabla\psi|^2-g'(\psi_0)\psi^2\right)dxdy+{\iint_{\Omega_m}g'(\psi_0)\widehat\psi_0^2dxdy}\\
=&\iint_{\Omega_m}\left(|\nabla\psi|^2-g'(\psi_0)\psi^2\right)dxdy+{\iint_{\Omega_m}g'(\psi_0)(\hat P_{0,e}\psi)^2dxdy}=\langle\hat A_{0,e}\psi,\psi\rangle
 \end{align*}
 for $\psi\in\tilde X_{0,e}$, we have $n^-(\hat A_{0,e})\leq n^-(\tilde A_{0,e}).$
By Corollary \ref{A-L-dec-e}, $n^-\left(\hat{A}_{0,e}\right)\leq n^-\left(\tilde A_{0,e}\right)=2(m-1)$. On the other hand,
since $\ker(\vec{u}_0\cdot\nabla)=\{\phi(y)\in (\hat L_e^2(\Omega_m))^*\}$ and  $\hat{P}_{0,e}\psi=0$ for  $\psi\in\tilde X_{0,e-}$,
we have $\hat{A}_{0,e}|_{X_{0,e-}}=\tilde A_{0,e}|_{X_{0,e-}}$ and thus, $n^-\left(\hat{A}_{0,e}\right)=2(m-1)$. The conclusion is then a consequence of
Lemmas \ref{L e-hat A} and \ref{indice-theorem-sep}.
This  suggests that the number of unstable eigenvalues of the linearized vorticity operator is $2(m-1)$ for $\ep\ll1$.
\end{remark}
\section{Modulational instability}\label{modulational}
In this section, we study the linear stability of $\omega_\ep$ with respect to perturbations of the form
\begin{align}u(x, y) = \widetilde{u}(x, y)e^{i\alpha x},\nonumber\\
\omega(x, y) = \widetilde{\omega}(x, y)e^{i\alpha x},\label{perturbation-modulaitonal-form}\\
\psi(x, y) = \widetilde{\psi}(x, y)e^{i\alpha x},\nonumber\end{align}
where $\alpha \in (0, \frac 1 2]$, and $\widetilde{u}, \widetilde{\omega}, \widetilde{\psi}$ are complex-valued and defined on the domain $\Omega = \mathbb{T}_{2\pi} \times \mathbb{R}$.
\subsection{Complex Hamiltonian formulation}
Recall that the   linearized vorticity  operator has the form $ J_\epsilon L_\epsilon$, where
$J_\epsilon = -g'(\psi_\epsilon)\vec{u}_\epsilon\cdot\nabla$ and
 $L_\epsilon = \frac {1} {g'(\psi_\epsilon)} - (-\Delta)^{-1}$. We seek solutions of the form \eqref{perturbation-modulaitonal-form} for the linearized equations, where $\widetilde{\omega} \in L^2_{\frac{1}{g'(\psi_\ep)}}(\Omega)$.  Then we have $J_\epsilon L_\epsilon(e^{i\alpha x}\widetilde{\omega})=e^{i\alpha x}J_{\epsilon,\alpha} L_{\epsilon,\alpha}\widetilde{\omega}$, where
\begin{align}\label{def-J-ep-al}
J_{\epsilon, \alpha} =& g'(\psi_\epsilon)\vec{u}_\epsilon\cdot\nabla_\alpha:L^2_{g'(\psi_\ep)}(\Omega) \supset D(J_{\epsilon, \alpha}) \to L^2_{\frac{1}{g'(\psi_\ep)}}(\Omega),\\\label{def-L-ep-al}
 L_{\epsilon, \alpha} =& \frac {1} {g'(\psi_\epsilon)} - (-\Delta_\alpha)^{-1}:L^2_{\frac{1}{g'(\psi_\ep)}}(\Omega)\to L^2_{g'(\psi_\ep)}(\Omega),\end{align}
and
\begin{align}\label{nabla-alpha-Delta-alpha}
 \nabla_\alpha = (\partial_x + i\alpha, \partial_y)^T,\quad
 \Delta_\alpha  = (i\alpha + \partial_x)^2 + \partial_{yy}.
 \end{align}
To make it rigorous, we need to clarify  the solvability of the  $\alpha$-Poisson equation.
\begin{lemma}
For any $\widetilde{\omega} \in L^2_{\frac{1}{g'(\psi_\ep)}}(\Omega)$, the $\alpha$-Poisson equation
\begin{align}\label{a-Poisson}- \Delta_\alpha \widetilde{\psi} = \widetilde{\omega}\end{align}
has a unique weak solution $\widetilde{\psi}$ in the Hilbert space
$$H^1_\alpha(\Omega) := \{ \phi | \| \nabla_\alpha \phi \|^2_{L^2(\Omega)}< \infty  \}$$
equipped with the inner product
$$(\phi_1, \phi_2)_{H^1_\alpha(\Omega)} = \iint_\Omega \nabla_\alpha \phi_1 \cdot \overline{\nabla_\alpha \phi_2} dxdy.$$
\end{lemma}
\begin{remark}
Since $\mathbb{Z}\ni k\neq\alpha\in (0,{1\over2}]$, we have $c_0(k^2+\alpha^2)\leq (k+\alpha)^2$ for some $c_0>0$. Then
\begin{align*}
c_1\|  \phi \|_{H^1(\Omega)}^2\leq \| \nabla_\alpha \phi \|^2_{L^2(\Omega)}=\sum_{k\in\mathbb{Z}}\left((k+\alpha)^2\|\widehat{\phi}_k\|_{L^2(\mathbb{R})}^2 +\|\widehat{\phi}'_k\|_{L^2(\mathbb{R})}^2\right)\leq  c_2\|  \phi \|^2_{H^1(\Omega)}
\end{align*}
 for some $c_1, c_2>0$. Thus,  $H^1_\alpha(\Omega)\cong H^1(\Omega)$.
\end{remark}
\begin{proof}
 For $\widetilde{\omega} \in L^2_{\frac{1}{g'(\psi_\ep)}}(\Omega)$, we have
$$\iint_\Omega \phi \widetilde{\omega} dxdy \leq \iint_\Omega  \frac{|\widetilde{\omega}|^2}{g'(\psi_\ep)} dxdy \iint_\Omega g'(\psi_\ep) |\phi|^2 dxdy \leq C \|\widetilde{\omega}\|^2_{L^2_{\frac{1}{g'(\psi_\ep)}}(\Omega)} \|\phi\|^2_{H_\alpha^1(\Omega)}, \quad \phi \in H_\alpha^1(\Omega). $$
By the Riesz Representation Theorem, for any $\widetilde{\omega} \in L^2_{\frac{1}{g'(\psi_\ep)}}(\Omega)$, there exists a unique $\widetilde{\psi} \in H_\alpha^1(\Omega)$ such that
$$\iint_\Omega \widetilde{\omega} \phi dxdy = \langle \widetilde{\omega}, \phi \rangle = (\widetilde{\psi}, \phi)_{H_\alpha^1(\Omega)},\quad \phi \in H_\alpha^1(\Omega).$$
\end{proof}
For $\widetilde{\omega} \in L^2_{\frac{1}{g'(\psi_\ep)}}(\Omega)$, we denote $(- \Delta_\alpha)^{-1} \widetilde{\omega} \in H_\alpha^1(\Omega)$ to be the weak solution of the $\alpha$-Poisson equation
\eqref{a-Poisson}.
\if0
Thus, it is reasonable to consider
\begin{align*}
 &J_{\epsilon, \alpha} : D(J_{\epsilon, \alpha})\supset L^2_{g'(\psi_\ep)}(\Omega)\to L^2_{\frac{1}{g'(\psi_\ep)}}(\Omega),\quad
   L_{\epsilon, \alpha} :L^2_{\frac{1}{g'(\psi_\ep)}}(\Omega)\to L^2_{g'(\psi_\ep)}(\Omega).
 \end{align*}
 \fi
The linearized vorticity equation for $\widetilde{\omega}$ is  formulated as
 \begin{align}\label{complex Ham-modu}
 \partial_t \widetilde{\omega} = J_{\ep, \alpha} L_{\ep, \alpha} \widetilde{\omega}.
 \end{align}
$\omega_\ep$ is said to be linearly modulationally unstable for $\alpha\in(0,{1\over 2}]$ if the operator $J_{\ep, \alpha} L_{\ep, \alpha}$
has an unstable eigenvalue $\lambda$ with $Re(\lambda)>0$.
 \if0
 Moreover, we have
 \begin{equation}
 -\Delta_\alpha \widetilde{\psi} = \widetilde{\omega},
 \end{equation}
 and
 \begin{equation}
 \widetilde{u} =  (\partial_y \widetilde{\psi}, -(i\alpha+\partial_x) \widetilde{\psi}).
 \end{equation}
\fi


%\begin{lemma}\label{Lbounded}
%For  $\omega_1,\omega_2 \in X_0$, we have
%$\langle L_0 \omega_1, \omega_2 \rangle=\langle \omega_1, L_0 \omega_2 \rangle \leq C\|\omega_1\|_{X_0}\|\omega_2\|_{X_0}.$
%\end{lemma}
%\begin{proof}
For $\widetilde{\omega} \in L^2_{\frac{1}{g'(\psi_\ep)}}(\Omega)$, let $\widetilde{\psi}=(-\Delta_\alpha)^{-1}\widetilde{\omega}\in H_\alpha^1(\Omega)$, then
\begin{align*}
\|\widetilde{\psi}\|_{H_\alpha^1(\Omega)}^2=\iint_\Omega \widetilde{\omega} \overline{\widetilde{\psi}} dxdy
 \leq C\|\omega\|_{L^2_{\frac{1}{g'(\psi_\ep)}}(\Omega)} \|\widetilde{\psi}\|_{H_\alpha^1(\Omega)}.
\end{align*}
Thus, $\|\widetilde{\psi}\|_{H_\alpha^1(\Omega)}\leq C \|\widetilde{\omega}\|_{L^2_{\frac{1}{g'(\psi_\ep)}}(\Omega)}$. Let $\widetilde\omega_i\in L^2_{\frac{1}{g'(\psi_\ep)}}(\Omega)$ and $\widetilde{\psi}_i=(-\Delta_\alpha)^{-1}\widetilde{\omega}_i\in H_\alpha^1(\Omega)$ for $i=1,2$. Then
\begin{align}\label{L-alpha-bounded}
\langle L_{\ep,\alpha} \widetilde{\omega}_1, \widetilde{\omega}_2 \rangle=
 \langle  \widetilde{\omega}_1, L_{\ep,\alpha}\widetilde{\omega}_2 \rangle
 \leq C\|\widetilde{\omega}_1\|_{L^2_{\frac{1}{g'(\psi_\ep)}}(\Omega)}\|\widetilde{\omega}_2\|_{L^2_{\frac{1}{g'(\psi_\ep)}}(\Omega)}.
\end{align}
%\end{proof}
Thus, $
 \langle L_{\ep,\alpha}\cdot,\cdot\rangle$
is bounded and symmetric  on $L^2_{\frac{1}{g'(\psi_\ep)}}(\Omega)$.



\subsection{Exact solutions to the associated eigenvalue problems  for the  modulational case}

Define
 \begin{align*}
 \tilde{A}_{\ep,\alpha}=-\Delta_\alpha-g'(\psi_\ep): H_\alpha^1(\Omega) \rightarrow H_\alpha^{1}(\Omega)^*,
\end{align*}
where the negative $\alpha$-Laplacian operator is understood in the weak sense.
Then
$
 \langle\tilde{A}_{\ep,\alpha}\cdot,\cdot\rangle
$
defines a  bounded and symmetric bilinear form on $H_\alpha^{1}(\Omega)$. Noting that $\iint_{\Omega}g'(\psi_\ep)|\psi|^2dxdy\leq \|\psi\|_{H_\alpha^1(\Omega)}^2$ for $\psi\in H_\alpha^1(\Omega)$, a
similar argument to Lemma \ref{equal-indices0} implies
\begin{align*}
\dim\ker(L_{\ep,\alpha})=\dim\ker(\tilde{A}_{\ep,\alpha}) \quad {\rm{and}} \quad n^-(L_{\ep,\alpha})=n^-(\tilde{A}_{\ep,\alpha}).
\end{align*}

Since  $H_\alpha^{1}(\Omega)$ is compactly embedded in $L_{g'(\psi_\ep)}^2(\Omega)$,
 we can
inductively define $\lambda_n$, $n\geq1$, as follows:
\begin{align}\nonumber
\lambda_n(\ep,\alpha)=& \inf_{\widetilde{\psi} \in H_\alpha^{1}(\Omega), (\widetilde{\psi}, \widetilde{\psi}_{i})_{L_{g'(\psi_\ep)}^2(\Omega)} = 0, i = 1, 2, \cdots, n-1}{\iint_\Omega|\nabla_\alpha\widetilde{\psi}|^2dxdy\over\iint_\Omega g'(\psi_\ep)|\widetilde{\psi}|^2dxdy}\\\nonumber
=&\min_{\widetilde{\psi} \in H_\alpha^{1}(\Omega), (\widetilde{\psi}, \widetilde{\psi}_{i})_{L_{g'(\psi_\ep)}^2(\Omega)} = 0, i = 1, 2, \cdots, n-1}{\|\widetilde{\psi}\|_{H_\alpha^{1}(\Omega)}^2\over\|\widetilde{\psi}\|_{L_{g'(\psi_\ep)}^2(\Omega)}^2},
\end{align}
where the infimum for $\lambda_i(\ep,\alpha)$ is attained at  $\widetilde\psi_{i} \in H_\alpha^{1}(\Omega)$ and $\|\widetilde{\psi}_i\|_{L_{g'(\psi_\ep)}^2(\Omega)} = 1$, $1\leq i \leq n-1$.
A direct computation of the 1-order variation of $$G_{\ep,\alpha}(\widetilde{\psi})={\|\widetilde{\psi}\|_{H_\alpha^{1}(\Omega)}^2\over\|\widetilde{\psi}\|_{L_{g'(\psi_\ep)}^2(\Omega)}^2}$$
 at $\widetilde{\psi}_{n}$ gives
the corresponding Euler-Lagrangian equation
\begin{align}\label{elip02-alpha}
-\Delta_\alpha \widetilde\psi = \lambda g'(\psi_\ep)\widetilde\psi, \quad \widetilde\psi \in H_\alpha^{1}(\Omega).
\end{align}
To solve the associated eigenvalue problem \eqref{elip02-alpha}, at the first glance we try to use the new variables $(\theta_\ep,\gamma_\ep)$ directly, the transformed equation is however involved and difficult to handle. Instead, we consider  the full perturbation $\psi=\widetilde\psi e^{i\alpha x}$ and by  \eqref{elip02-alpha} it  satisfies
\begin{align}\label{elip02-alpha-full perturbation}
-\Delta(\widetilde\psi e^{i\alpha x}) = \lambda g'(\psi_\ep)(\widetilde\psi e^{i\alpha x}), \quad \widetilde\psi \in H_\alpha^{1}(\Omega).
\end{align}
Note that the full perturbation $\psi$ can also be written as $\widetilde\Psi(\theta_\ep,\gamma_\ep) e^{i\alpha \theta_\ep}$ in the new variables. This motivates us to introduce the following transformation
\begin{align}\label{transformation-modu}
\widetilde\Psi(\theta_\ep,\gamma_\ep)=\widetilde\psi(x,y) e^{i\alpha (x-\theta_\ep)}.
\end{align}
Since
$\widetilde\Psi(\theta_\ep+2\pi,\gamma_\ep)=e^{i\alpha(x(\theta_\ep+2\pi,\gamma_\ep)-\theta_\ep-2\pi)}
\widetilde\psi(x(\theta_\ep+2\pi,\gamma_\ep),y(\theta_\ep+2\pi,\gamma_\ep))=e^{i\alpha( x-\theta_\ep)}
\widetilde\psi(x,y)=\widetilde\Psi(\theta_\ep,\gamma_\ep)$, we know that  $\widetilde\Psi$ is $2\pi$-periodic  in $\theta_\ep$.
Moreover,
\begin{align*}
 \|\widetilde\psi\|_{{H}_\alpha^1(\Omega)}^2=\iint_{\tilde \Omega}\left({1\over1-\gamma_\ep^2}(|\widetilde\Psi_{\theta_\ep}+i\alpha\widetilde\Psi|^2)+(1-\gamma_\ep^2)|\widetilde\Psi_{\gamma_\ep}|^2\right)d \theta_\ep d\gamma_\ep\triangleq\|\widetilde\Psi\|_{Y_{\ep,\alpha}}^2,
\end{align*}
where $Y_{\ep,\alpha}=\{\Psi|\|\Psi\|_{Y_{\ep,\alpha}}<\infty\}$. By \eqref{elip02-alpha-full perturbation},
$\widetilde\Psi$ satisfies the eigenvalue problem
\begin{align}\label{eigenvalue problem-ep-new-alpha}
-\pa_{\gamma_\ep}\left((1-\gamma_\ep^2)\pa_{\gamma_\ep}\widetilde\Psi\right)-{1\over1-\gamma_\ep^2}(\pa_{\theta_\ep}+i\alpha)^2\widetilde\Psi
=2\lambda\widetilde\Psi, \quad \widetilde\Psi \in Y_{\ep,\alpha}.
\end{align}
Since $\widetilde\Psi$ is $2\pi$-periodic  in $\theta_\ep$, we separate it into the Fourier modes.
For the $k$ mode with $k\in\mathbb{Z}$, the eigenvalue problem \eqref{eigenvalue problem-ep-new-alpha}
is
\begin{equation}\label{eigenvalue problem2 non-zero modes varepsilon=0alpha}
-((1-\gamma_\ep^2)\varphi')'+{(k+\alpha)^2\over1-\gamma_\ep^2}\varphi =2\lambda\varphi \quad \text{on}\quad (-1,1),\quad\varphi\in \hat Y_1^\ep,
\end{equation}
where $
\hat Y_1^\ep
$ is defined in \eqref{Y-k-ep-def}.
To
 solve the eigenvalue problem \eqref{eigenvalue problem2 non-zero modes varepsilon=0alpha},
we use the transformation
\begin{align}\label{modulational-transformation}
\varphi=(1-\gamma_\ep^2)^{|k+\alpha|\over2}\phi.
\end{align}
Then \eqref{eigenvalue problem2 non-zero modes varepsilon=0alpha} is transformed to
\begin{equation}\label{eigenvalue problem2 non-zero modes varepsilon=0-transform-alpha}
(1-\gamma_\ep^2)\phi''-2\left(|k+\alpha|+1\right)\gamma_\ep\phi'+\left(-(k+\alpha)^2-|k+\alpha|+2\lambda\right)\phi =0 \quad \text{on}\quad (-1,1),
\end{equation}
where $\varphi\in W_{k+\alpha}=\{\phi|(1-\gamma_\ep^2)^{|k+\alpha|\over2}\phi\in\hat Y_1^\ep\}$.
Let
\begin{align*}
\beta=|k+\alpha|+{1\over2},\quad \lambda={1\over2}\left(n+|k+\alpha|\right)\left(n+|k+\alpha| +1\right)
\end{align*}
in \eqref{Gegenbauer differential equation} and \eqref{eigenvalue problem2 non-zero modes varepsilon=0-transform-alpha}, respectively.
Then the  equation \eqref{eigenvalue problem2 non-zero modes varepsilon=0-transform-alpha} and  the Gegenbauer differential equation \eqref{Gegenbauer differential equation} coincide. All the solutions of  \eqref{Gegenbauer differential equation} in $ L_{\hat g_\beta}^2(-1,1)$ are given by
Gegenbauer polynomials
$
C_n^\beta(\gamma_\ep), n\geq0$, in \eqref{Gegenbauer polynomials}. Since $\beta>{1\over2}$,  similar to \eqref{1-gammacnykep} we have $(1-\gamma_\ep^2)^{|k+\alpha|\over2}C_n^\beta\in\hat Y_1^\ep$ for $n\geq0$.
Thus,
\begin{align*}
&\varphi_{n,k+\alpha}(\gamma_\ep)\triangleq(1-\gamma_\ep^2)^{|k+\alpha|\over2}C_n^\beta(\gamma_\ep)\in\hat Y_1^\ep,\quad\lambda=\lambda_{n,k+\alpha}\triangleq{1\over2}\left(n+|k+\alpha|\right)\left(n+|k+\alpha| +1\right)
\end{align*}
solve \eqref{eigenvalue problem2 non-zero modes varepsilon=0alpha} for $n\geq0$.
Since $\beta>-{1\over2},$ $\{C_n^\beta\}_{n=0}^\infty$ is a complete and  orthogonal basis of $ L_{\hat g_\beta}^2(-1,1)$. This, along with the fact that   $\hat Y_1^\ep$ is embedded in $ L^2(-1,1)$, implies that
 $\{\varphi_{n,k+\alpha}\}_{n=0}^\infty$ is a complete and  orthogonal basis of $\hat Y_1^\ep$ under the inner product of $ L^2(-1,1)$.
Now, we solve the eigenvalue problem \eqref{eigenvalue problem2 non-zero modes varepsilon=0alpha} for the $k$ mode, $k\in\mathbb{Z}$.
\begin{lemma}\label{sol to eigenvalue problem non-zero modes varepsilon=0-alpha-k} Fix $\alpha\in(0,{1\over2}]$ and $k\in\mathbb{Z}.$ Then
all the eigenvalues  of the eigenvalue problem \eqref{eigenvalue problem2 non-zero modes varepsilon=0alpha}  are $\lambda_{n,k+\alpha}={1\over2}\left(n+|k+\alpha|\right)\left(n+|k+\alpha| +1\right)$, $n\geq 0$. For $n\geq0$, the eigenspace associated to $\lambda_{n,{k+\alpha}}$ is $\text{span}\{\varphi_{n,k+\alpha}(\gamma_\ep)\}=\text{span}\{(1-\gamma_\ep^2)^{|k+\alpha|\over2}C_n^{|k+\alpha|+{1\over2}}(\gamma_\ep)\}$.
\end{lemma}

Thus, we get the solutions of
the eigenvalue problem \eqref{eigenvalue problem-ep-new-alpha}.
\begin{Theorem}\label{sol to eigenvalue problem varepsilon=0-pde-alpha} Fix $\alpha\in(0,{1\over2}]$.

$(1)$
All the eigenvalues  of the eigenvalue problem \eqref{eigenvalue problem-ep-new-alpha} are
\begin{align}\label{km-eigenvalues-alpha}
{1\over2}\alpha\left(\alpha+1\right),\quad{1\over2}\left(n\pm\alpha\right)\left( n\pm\alpha+1\right), &\quad  n\geq1.
\end{align}
\if0
$(1)$ For $1\leq i\leq m-1$, the  eigenspace associated to the eigenvalue  ${i\over 2m}\left({i\over m}+1\right)$ is spanned by
 \begin{align*}
 (1-\gamma_\ep^2)^{i\over2m}C_0^{{i\over m}+{1\over2}}(\gamma_\ep)\cos\left({i\over m}\theta_\ep\right),\;\;(1-\gamma_\ep^2)^{i\over2m}C_0^{{i\over m}+{1\over2}}(\gamma_\ep)\sin\left({i\over m}\theta_\ep\right).
 \end{align*}
 \fi
 For  $n\geq0$,
the eigenspace associated to the eigenvalue  ${1\over2}\left(n+\alpha\right)\left( n+\alpha+1\right)$ is spanned by
\begin{align*}
&(1-\gamma_\ep^2)^{\alpha\over2}C_n^{\alpha+{1\over2}}(\gamma_\ep),\\
 &(1-\gamma_\ep^2)^{j+\alpha\over2}C_{n-j}^{{j+\alpha+{1\over2}}}(\gamma_\ep)e^{ij\theta_\ep},\;\;1\leq j\leq n.
 \end{align*}
 For  $n\geq1$,
the eigenspace associated to the eigenvalue  ${1\over2}\left(n-\alpha\right)\left( n-\alpha+1\right)$ is spanned by
\begin{align*}
 &(1-\gamma_\ep^2)^{j-\alpha\over2}C_{n-j}^{{j-\alpha+{1\over2}}}(\gamma_\ep)e^{-ij\theta_\ep},\;\;1\leq j\leq n.
 \end{align*}

\if0
$(2)$
All the eigenvalues  for the eigenvalue problem \eqref{elip02-alpha-full perturbation} are
given by \eqref{km-eigenvalues-alpha}.  For $n\geq0$,
the eigenspace associated to the eigenvalue  ${1\over2}\left(n+\alpha\right)\left( n+\alpha+1\right)$ is spanned by
\begin{align}\label{nonkm-eigenfunctions22}
&(1-\gamma_\ep^2)^{\alpha\over2}C_n^{\alpha+{1\over2}}(\gamma_\ep)e^{i\alpha\theta_\ep},\\\nonumber
 &(1-\gamma_\ep^2)^{j+\alpha\over2}C_{n-j}^{{j+\alpha+{1\over2}}}(\gamma_\ep)e^{ij\theta_\ep}e^{i\alpha\theta_\ep},\;\;1\leq j\leq n.
 \end{align}
Here $\theta_\ep(x,y)$ and  $\gamma_\ep(x,y)$ are defined in \eqref{transf1} and \eqref{transf2}.

For  $n\geq1$,
the eigenspace associated to the eigenvalue  ${1\over2}\left(n-\alpha\right)\left( n-\alpha+1\right)$ is spanned by
\begin{align}\label{nonkm-eigenfunctions22}
 &(1-\gamma_\ep^2)^{j-\alpha\over2}C_{n-j}^{{j-\alpha+{1\over2}}}(\gamma_\ep)e^{-ij\theta_\ep}e^{i\alpha\theta_\ep},\;\;1\leq j\leq n.
 \end{align}
\fi
$(2)$ All the eigenvalues  of the associated eigenvalue problem \eqref{elip02-alpha} are
given by \eqref{km-eigenvalues-alpha}.  For $n\geq0$,
the eigenspace associated to the eigenvalue  ${1\over2}\left(n+\alpha\right)\left( n+\alpha+1\right)$ is spanned by
\begin{align*}
&(1-\gamma_\ep^2)^{\alpha\over2}C_n^{\alpha+{1\over2}}(\gamma_\ep)e^{i\alpha(\theta_\ep-x)},\\\nonumber
 &(1-\gamma_\ep^2)^{j+\alpha\over2}C_{n-j}^{{j+\alpha+{1\over2}}}(\gamma_\ep)e^{ij\theta_\ep}e^{i\alpha(\theta_\ep-x)},\;\;1\leq j\leq n.
 \end{align*}
For  $n\geq1$,
the eigenspace associated to the eigenvalue  ${1\over2}\left(n-\alpha\right)\left( n-\alpha+1\right)$ is spanned by
\begin{align*}
 &(1-\gamma_\ep^2)^{j-\alpha\over2}C_{n-j}^{{j-\alpha+{1\over2}}}(\gamma_\ep)e^{-ij\theta_\ep}e^{i\alpha(\theta_\ep-x)},\;\;1\leq j\leq n.
 \end{align*}


 In particular,
the multiplicity of ${1\over2}\left(n+\alpha\right)\left( n+\alpha+1\right)$ is $n+1$ for  $n\geq0$, and the multiplicity of ${1\over2}\left(n-\alpha\right)\left( n-\alpha+1\right)$ is $n$ for  $n\geq1$.
\end{Theorem}

As an application, we give
 the explicit negative directions of  $\tilde A_{\ep,\alpha}$ and $L_{\ep,\alpha}$, confirm that the two operators are non-degenerate, as well as  provide   decompositions of
$H_\alpha^1(\Omega)$ and $L^2_{\frac{1}{g'(\psi_\ep)}}(\Omega)$  associated to  the two operators, respectively.

\begin{Corollary}\label{A-L-dec-e-alpha}
 Let $\alpha\in(0,{1\over2}]$. Then

$(1)$  the negative subspaces of  $H_\alpha^1(\Omega)$ and $L^2_{\frac{1}{g'(\psi_\ep)}}(\Omega)$  associated to $\tilde A_{\ep,\alpha}$ and $L_{\ep,\alpha}$ are
 \begin{align*} H_{\alpha-}^1(\Omega)&=\textup{span}\left\{(1-\gamma_\ep^2)^{\alpha\over2}e^{i\alpha(\theta_\ep-x)},
 (1-\gamma_\ep^2)^{1-\alpha\over2}e^{-i\theta_\ep}e^{i\alpha(\theta_\ep-x)}\right\},\\
 L^2_{\frac{1}{g'(\psi_\ep)}-}(\Omega)&=\textup{span}\left\{g'(\psi_\ep)(1-\gamma_\ep^2)^{\alpha\over2}e^{i\alpha(\theta_\ep-x)},
 g'(\psi_\ep)(1-\gamma_\ep^2)^{1-\alpha\over2}e^{-i\theta_\ep}e^{i\alpha(\theta_\ep-x)}\right\},
\end{align*}
respectively, where $\gamma_\ep=\gamma_\ep(x,y)$ and $\theta_\ep=\theta_\ep(x,y)$.
Thus,  $\dim H_{\alpha-}^1(\Omega)=\dim L^2_{\frac{1}{g'(\psi_\ep)}-}(\Omega)=2$.

$(2)$ $\ker (\tilde A_{\ep,\alpha})=\{0\}$ and $\ker (L_{\ep,\alpha})=\textup{span}\{0\}$.

$(3)$ Let  $ H_{\alpha+}^1(\Omega)= H_{\alpha}^1(\Omega) \ominus H_{\alpha-}^1(\Omega)$ and $L^2_{\frac{1}{g'(\psi_\ep)}+}(\Omega)=L^2_{\frac{1}{g'(\psi_\ep)}}(\Omega) \ominus L^2_{\frac{1}{g'(\psi_\ep)}-}(\Omega)$. Then
\begin{align*}
\langle \tilde A_{\ep,\alpha} \widetilde\psi,\widetilde\psi\rangle \geq \left(1-{2\over (\alpha+1)(\alpha+2)}\right) \| \widetilde\psi\|_{H_{\alpha}^1(\Omega)}^2, \quad\forall \widetilde\psi\in H_{\alpha+}^1(\Omega),
\end{align*}
and there exists $\delta>0$ such that
\begin{align*}
\langle L_{\ep,\alpha} \widetilde\omega,\widetilde\omega\rangle \geq \delta \| \widetilde\omega\|_{L^2_{\frac{1}{g'(\psi_\ep)}}(\Omega)}^2, \quad \forall \widetilde\omega\in L^2_{\frac{1}{g'(\psi_\ep)}+}(\Omega).
\end{align*}
\end{Corollary}
\begin{proof}
The proof is essentially due to the following three facts based on Theorem \ref{sol to eigenvalue problem varepsilon=0-pde-alpha}. First, the only eigenvalues, which are less than $1$, of \eqref{elip02-alpha} are ${1\over 2}\alpha(\alpha+1)$ and ${1\over 2}(1-\alpha)(2-\alpha)$. Second, $1$ is not an eigenvalue of \eqref{elip02-alpha}.
Finally, the minimal eigenvalue, which is larger than $1$, is ${1\over2}(1+\alpha)(2+\alpha)$.
\end{proof}
\subsection{A modulational  instability criterion}
Noting that $J_{\ep, \alpha}$ and  $L_{\ep, \alpha}$ are complex operators, we reformulate the linear modulational  problem in the real operators so that we can apply the index formula \eqref{index-formula-neg} for
the real separable Hamiltonian systems.

Let
\begin{align}\label{omega-triangle functions}
\omega(x,y)=\cos(\alpha x)\omega_1(x,y)+\sin(\alpha x)\omega_2(x,y),
\end{align}
where $\omega_1,\omega_2\in L^2_{\frac{1}{g'(\psi_\ep)}}(\Omega)$ are real-valued functions. We decompose
\begin{align*}
(-\Delta_\alpha)^{-1}=(-\Delta_\alpha)_1^{-1}+i(-\Delta_\alpha)_2^{-1},\quad(-\Delta_{-\alpha})^{-1}=(-\Delta_{\alpha})_1^{-1}-i(-\Delta_{\alpha})_2^{-1},
\end{align*}
where
\begin{align*}
(-\Delta_\alpha)_1^{-1}={1\over2} \left((-\Delta_\alpha)^{-1}+(-\Delta_{-\alpha})^{-1}\right),\quad
(-\Delta_{\alpha})_2^{-1}=-{i\over2} \left((-\Delta_\alpha)^{-1}-(-\Delta_{-\alpha})^{-1}\right).
\end{align*}
Here, $(-\Delta_\alpha)_1^{-1}$ is self-dual and $(-\Delta_{\alpha})_2^{-1}$ is anti-self-dual.
Since $\overline{(-\Delta_\alpha)^{-1}}=(-\Delta_{-\alpha})^{-1}$, $(-\Delta_\alpha)_1^{-1}$ and
$(-\Delta_{\alpha})_2^{-1}$ map real functions to real ones. By
\begin{align}\label{omega-i-1-2}
\omega={e^{i\alpha x}\over2}(\omega_1-i\omega_2)+{e^{-i\alpha x}\over 2}(\omega_1+i\omega_2),
\end{align}
we have
\begin{align}\nonumber
(-\Delta)^{-1}\omega=&\cos(\alpha x)\left((-\Delta_\alpha)_1^{-1}\omega_1+(-\Delta_\alpha)_2^{-1}\omega_2\right)\\\label{omega-neg-lap-triangle functions}
&+\sin(\alpha x)\left((-\Delta_\alpha)_1^{-1}\omega_2-(-\Delta_{\alpha})_2^{-1}\omega_1\right),
\end{align}
and
\begin{align}\nonumber
g'(\psi_\epsilon)\vec{u}_\epsilon\cdot\nabla \omega=&\cos(\alpha x)(g'(\psi_\epsilon)\vec{u}_\epsilon\cdot\nabla \omega_1+\alpha g'(\psi_\epsilon)u_{\ep,1}\omega_2)\\\label{g-der-nabla-triangle functions}
&+\sin(\alpha x)(g'(\psi_\epsilon)\vec{u}_\epsilon\cdot\nabla\omega_2-\alpha g'(\psi_\epsilon)u_{\ep,1}\omega_1).
\end{align}
We define the operators
\begin{align*}
\hat{J}_{\ep,\alpha}=&\left( \begin{array}{cc} g'(\psi_\epsilon)\vec{u}_\epsilon\cdot\nabla & \alpha g'(\psi_\epsilon)u_{\ep,1}\\ -\alpha g'(\psi_\epsilon)u_{\ep,1} & g'(\psi_\epsilon)\vec{u}_\epsilon\cdot\nabla \end{array} \right): \left(L^2_{g'(\psi_\ep)}(\Omega)\right)^2 \supset D(\hat{J}_{\epsilon, \alpha}) \to \left(L^2_{\frac{1}{g'(\psi_\ep)}}(\Omega)\right)^2,\\
 \hat {L}_{\epsilon, \alpha} =&
\left( \begin{array}{cc} \frac {1} {g'(\psi_\epsilon)} - (-\Delta_\alpha)_1^{-1} & -(-\Delta_\alpha)_2^{-1} \\ (-\Delta_\alpha)_2^{-1} & \frac {1} {g'(\psi_\epsilon)} - (-\Delta_\alpha)_1^{-1}  \end{array} \right)
 :\left(L^2_{\frac{1}{g'(\psi_\ep)}}(\Omega)\right)^2\to \left(L^2_{g'(\psi_\ep)}(\Omega)\right)^2.
\end{align*}
Then they are real operators,  $\hat{J}_{\ep,\alpha}$ is anti-self-dual and $\hat {L}_{\epsilon, \alpha}$ is self-dual. By \eqref{omega-triangle functions}, \eqref{omega-neg-lap-triangle functions} and \eqref{g-der-nabla-triangle functions}, $J_\epsilon L_\epsilon$ and $\hat{J}_{\ep,\alpha} \hat {L}_{\epsilon, \alpha}$ are related by
\begin{align*}
J_\epsilon L_\epsilon\omega=(\cos(\alpha x),\;\sin(\alpha x))\hat{J}_{\ep,\alpha} \hat {L}_{\epsilon, \alpha}\left( \begin{array}{cc} \omega_1 \\ \omega_2  \end{array} \right).
\end{align*}
By \eqref{omega-i-1-2}-\eqref{g-der-nabla-triangle functions},
 the  complex operators $J_{\ep,\alpha}, L_{\ep,\alpha}$ and the real  operators  $\hat{J}_{\ep,\alpha}, \hat {L}_{\epsilon, \alpha}$ are related by
\begin{align}\label{hat-J-L-alpha1}
\hat{J}_{\ep,\alpha}=&M^{-1}\left( \begin{array}{cc}J_{\ep,\alpha}  &0 \\0 & J_{\ep,-\alpha} \end{array} \right)M,\;\;
\hat{L}_{\ep,\alpha}=M^{-1}\left( \begin{array}{cc}L_{\ep,\alpha}  &0 \\0 & L_{\ep,-\alpha} \end{array} \right)M,\\\label{hat-J-L-alpha2}
\hat{J}_{\ep,\alpha}\hat{L}_{\ep,\alpha}=&M^{-1}\left( \begin{array}{cc}J_{\ep,\alpha}L_{\ep,\alpha}  &0 \\0 & J_{\ep,-\alpha}L_{\ep,-\alpha} \end{array} \right)M,
\end{align}
where
\begin{align*}
M={1\over2}\left( \begin{array}{cc}1  &-i \\1 & i \end{array} \right).
\end{align*}
By \eqref{def-J-ep-al}-\eqref{nabla-alpha-Delta-alpha}, we have
\begin{align}\label{L-JL-conjugate}
\overline{L_{\ep,\alpha}}=L_{\ep,-\alpha},\quad  \overline{J_{\ep,\alpha}L_{\ep,\alpha}}=J_{\ep,-\alpha}L_{\ep,-\alpha}.
\end{align}
By  \eqref{hat-J-L-alpha1} and \eqref{L-JL-conjugate}, we have
\begin{align*}
n^-(\hat L_{\ep,\alpha})=n^-( L_{\ep,\alpha})+n^-( L_{\ep,-\alpha})=2n^-(L_{\ep,\alpha}).
\end{align*}
For the real operator $\hat{J}_{\ep,\alpha}\hat{L}_{\ep,\alpha}$, let $k_{r,\ep,\alpha}, k_{c,\ep,\alpha}, k_{i,\ep,\alpha}^{\leq0}, k_{0,\ep,\alpha}^{\leq0}$ be the indices defined similarly as in Lemma
\ref{theorem-index}.
\if0
By Corollary \ref{A-L-dec-e}, $L_{\ep,\alpha}$
is non-degenerate on $L^2_{\frac{1}{g'(\psi_\ep)}}(\Omega)$, and thus,
\begin{align*}
k_{i,\alpha}^{\leq0}=k_{i,\alpha}^{-},\quad k_{0,\alpha}^{\leq0}=k_{0,\alpha}^{-},
\end{align*}
where $k_{i,\alpha}^-$ is the total number of negative dimensions of $\langle \hat{L}_{\ep,\alpha}\cdot, \cdot \rangle$ restricted to the generalized eigenspaces of purely imaginary eigenvalues of $\hat{J}_{\ep,\alpha}\hat{L}_{\ep,\alpha}$ with positive imaginary parts, and $k_{0,\alpha}^-$ is the number of negative dimensions of $\langle \hat{L}_{\ep,\alpha}\cdot, \cdot \rangle$ restricted to the generalized kernel of $\hat{J}_{\ep,\alpha}\hat{L}_{\ep,\alpha}$ modulo $\ker \hat{L}_{\ep,\alpha}$.
\fi
For the complex operator $J_{\ep,\alpha}L_{\ep,\alpha}$,
let $\tilde k_{r,\ep,\alpha}$ be  the sum of algebraic multiplicities of positive eigenvalues of $J_{\ep,\alpha}L_{\ep,\alpha}$,
 $\tilde k_{c,\ep,\alpha}$
be the sum of algebraic multiplicities of eigenvalues of $J_{\ep,\alpha}L_{\ep,\alpha}$ in the first
and the fourth quadrants,
$\tilde k_{i,\ep,\alpha}^{\leq0}$ be  the total number of non-positive dimensions
of $\langle L_{\ep,\alpha}\cdot,\cdot\rangle$  restricted to the  generalized eigenspaces of nonzero
pure imaginary eigenvalues of $J_{\ep,\alpha}L_{\ep,\alpha}$,
  and $\tilde k_{0,\ep,\alpha}^{\leq0}$ be the number of non-positive directions of $\langle L_{\ep,\alpha}\cdot, \cdot \rangle$ restricted to the generalized kernel of $J_{\ep,\alpha}L_{\ep,\alpha}$ modulo $\ker L_{\ep,\alpha}$.
By  \eqref{hat-J-L-alpha2}-\eqref{L-JL-conjugate}, we have
\begin{align}\label{index-k-tilde-k}
k_{r,\ep,\alpha}=2\tilde k_{r,\ep,\alpha}, \;\; k_{c,\ep,\alpha}=\tilde k_{c,\ep,\alpha},\;\;k_{i,\ep,\alpha}^{\leq0}=\tilde k_{i,\ep,\alpha}^{\leq0},\;\;k_{0,\ep,\alpha}^{\leq0}=2\tilde k_{0,\ep,\alpha}^{\leq0}.
\end{align}
Applying Lemma \ref{theorem-index} to the real operators  $\hat{J}_{\ep,\alpha}$ and $\hat{L}_{\ep,\alpha}$, by Corollary \ref{A-L-dec-e-alpha} we have
\begin{align}\label{index-formula-real-operator}
k_{r,\ep,\alpha}+ 2k_{c,\ep,\alpha}+2k_{i,\ep,\alpha}^{\leq0}+k_{0,\ep,\alpha}^{\leq0}=2n^-\left(\hat{L}_{\ep,\alpha}\right)=4.
\end{align}
Combining \eqref{index-k-tilde-k} and \eqref{index-formula-real-operator}, we get the index formula for the complex operators  $J_{\ep,\alpha}$ and $L_{\ep,\alpha}$:
\begin{align*}
\tilde k_{r,\ep,\alpha}+ \tilde k_{c,\ep,\alpha}+\tilde k_{i,\ep,\alpha}^{\leq0}+\tilde k_{0,\ep,\alpha}^{\leq0}=n^-\left(L_{\ep,\alpha}\right)=2.
\end{align*}
To study the linear modulational instability, one may try to prove that $\tilde k_{i,\ep,\alpha}^{\leq0}+\tilde k_{0,\ep,\alpha}^{\leq0}\leq 1$, it is however difficult to compute the two indices for the eigenvalues  of $J_{\ep,\alpha}L_{\ep,\alpha}$ in the imaginary axis.
Here, we use the separable Hamiltonian structure of the real operator $\hat{J}_{\ep,\alpha}\hat{L}_{\ep,\alpha}$.
Define two spaces
\begin{align*}
X_{\alpha, e} =& \left\{ \left( \begin{array}{c} \omega_1 \\ \omega_2 \end{array} \right) \in \left(L^2_{\frac{1}{g'(\psi_\ep)}}(\Omega)\right)^2 \bigg| \text{both } \omega_1 \text{ and } \omega_2 \text{ are even  in }y \right\},\\
X_{\alpha, o} =& \left\{  \left( \begin{array}{c} \omega_1 \\ \omega_2 \end{array} \right) \in \left( L^2_{\frac{1}{g'(\psi_\ep)}}(\Omega)\right)^2\bigg| \text{both } \omega_1 \text{ and } \omega_2 \text{ are odd in }y \right\}.
\end{align*}
Then $X_{\alpha, e}$ and  $X_{\alpha, o}$ are  Hilbert spaces.
The dual space of  $X_{\alpha, o}$ (resp. $X_{\alpha, e}$) restricted to the class of odd (resp. even) functions is denoted by $X_{\alpha, o}^*$ (resp. $X_{\alpha, e}^*$).
Let
\begin{align*}\hat{B}_\alpha = \hat{J}_{\ep,\alpha}|_{X_{\alpha, o}^*},\;\; \hat L_{\alpha,o} = \hat {L}_{\epsilon, \alpha}|_{X_{\alpha, o}},\;\;
\hat{L}_{\alpha,e} = \hat {L}_{\epsilon, \alpha}|_{X_{\alpha, e}}.
 \end{align*}
 Then
 \begin{align*}\hat{B}_\alpha: X_{\alpha, o}^*\supset D(B_\alpha) \rightarrow X_{\alpha, e},\;\;
 \hat L_{\alpha,o}:X_{\alpha, o} \rightarrow X_{\alpha, o}^*,\;\;
 \hat L_{\alpha,e} : X_{\alpha, e} \rightarrow X_{\alpha, e}^*.
 \end{align*}
The dual operator of $\hat{B}_\alpha$ is
$$\hat{B}_\alpha'= \left( \begin{array}{cc} -g'(\psi_\epsilon)\vec{u}_\epsilon\cdot\nabla & -\alpha g'(\psi_\epsilon)u_{\ep,1}\\ \alpha g'(\psi_\epsilon)u_{\ep,1} & -g'(\psi_\epsilon)\vec{u}_\epsilon\cdot\nabla \end{array} \right) : X_{\alpha, e}^* \supset D(B'_\alpha) \rightarrow X_{\alpha, o}.$$
We decompose $\left( \omega_1, \omega_2  \right)^T \in \left(L^2_{\frac{1}{g'(\psi_\ep)}}(\Omega)\right)^2$ as $ \left( \omega_{1,e}, \omega_{2,e}, \omega_{1,o}, \omega_{2,o} \right)^T $ such that $\left( \omega_1, \omega_2  \right)^T = \left( \omega_{1,e}, \omega_{2,e})+(\omega_{1,o},  \omega_{2,o} \right)^T$, where
$\vec{\omega}_{e}\triangleq\left( \omega_{1,e}, \omega_{2,e} \right)^T \in X_{\alpha, e}$ and $\vec{\omega}_{o}\triangleq\left(  \omega_{1,o}, \omega_{2,o} \right)^T  \in X_{\alpha, o}$. Then the linearized equation $\partial_t(\omega_1,\omega_2)^T=\hat{J}_{\ep,\alpha} \hat {L}_{\epsilon, \alpha}(\omega_1,\omega_2)^T$ can be written as  the following separable Hamiltonian system
\begin{align}\label{sep-hamiltonian-alpha}
\partial_t \left( \begin{array}{c}\vec{\omega}_{e} \\ \vec{\omega}_{o} \end{array} \right) = \left( \begin{array}{cc} 0 & \hat{B}_\alpha \\ -\hat{B}'_\alpha & 0 \end{array} \right)\left( \begin{array}{cc} \hat{L}_{\alpha,e} & 0 \\ 0 & \hat{L}_{\alpha,o} \end{array} \right) \left( \begin{array}{c} \vec{\omega}_{e} \\ \vec{\omega}_{o}\end{array} \right).
%= \mathbf{J}_{\ep,m} \mathbf{L}_{\ep,m} \left( \begin{array}{c} \omega_1 \\ \omega_2 \end{array} \right),
\end{align}
To apply the index formula  \eqref{index-formula-neg},
 we need to verify
{\textbf{(G1-4)}}  in Lemma \ref{indice-theorem-sep} for \eqref{sep-hamiltonian-alpha}. {\textbf{(G1)}} can be verified in a similar way as for \eqref{sep-hamiltonian}. Using \eqref{hat-J-L-alpha1},
{\textbf{(G2-4)}} can be verified by \eqref{L-alpha-bounded} and Corollary \ref{A-L-dec-e-alpha}. Then by Lemma \ref{indice-theorem-sep}, the number of unstable modes  for \eqref{sep-hamiltonian-alpha} is
$k_{r,\ep,\alpha}=n^-\left(\hat{L}_{\alpha,e}|_{\overline{R(\hat{B}_\alpha)}} \right)$ and $k_{c,\ep,\alpha}=0$. By \eqref{index-k-tilde-k} and  \eqref{hat-J-L-alpha1},
we have
\begin{align*}
2\tilde k_{r,\ep,\alpha}=k_{r,\ep,\alpha}=n^-\left(\hat{L}_{\alpha,e}|_{\overline{R(\hat{B}_\alpha)}} \right)=2n^-\left(L_{\alpha,e}|_{\overline{R(B_\alpha)}} \right)\Longrightarrow\tilde k_{r,\ep,\alpha}=n^-\left(L_{\alpha,e}|_{\overline{R(B_\alpha)}} \right),
\end{align*}
and
\begin{align}\label{index-formula-for-modulational-instabilitykc}\tilde k_{c,\ep,\alpha}=k_{c,\ep,\alpha}=0,
\end{align}
 where
 \begin{align}\label{L-alpha-e B-alpha}
 &L_{\alpha,e}=L_{\ep,\alpha}|_{L^2_{\frac{1}{g'(\psi_\ep)},e}(\Omega)},\quad B_\alpha=J_{\ep,\alpha}|_{L^2_{{g'(\psi_\ep)},o}(\Omega)}.
 \end{align}
Here,
$L^2_{\frac{1}{g'(\psi_\ep)},e}(\Omega)=\{\omega\in L^2_{\frac{1}{g'(\psi_\ep)}}(\Omega)|\omega \text{ is even in }y\},$
$L^2_{\frac{1}{g'(\psi_\ep)},o}(\Omega)=\{\omega\in L^2_{\frac{1}{g'(\psi_\ep)}}(\Omega)|\omega  \text{ is odd in }$ $ y\},$
$L^2_{{g'(\psi_\ep)},e}(\Omega)=\{\omega\in L^2_{{g'(\psi_\ep)}}(\Omega)|\omega \text{ is even in }y\}$
and $L^2_{{g'(\psi_\ep)},o}(\Omega)=\{\omega\in L^2_{{g'(\psi_\ep)}}(\Omega)|\omega$ $ \text{ is odd in }y\}$.
\if0
To study $n^-\left( L_{\ep,e} |_{\overline{R(B_\ep)}} \right)$,  we define
$\bar{P}_{\ep,e}$ to be the orthogonal   projection of the space $ L_e^2(\Omega_m)=\{\phi\in L^2(\Omega_m)|\phi \text{ is even in } y\}$ on $\ker(\vec{u}_\ep\cdot\nabla)$. It induces  a   projection
$\hat{P}_{\ep,e}$ of $X_{\ep, e}^*$ on $\ker (B_\ep')$ by $\hat{P}_{\ep,e}=(S_e')^{-1}\bar{P}_{\ep,e} S_e'$, where
$S_e: L_e^2(\Omega_m) \rightarrow X_{\ep,e},  S_e\omega = g'(\psi_\ep)^{1/2}\omega$
defines an isometry.
Similar to \cite{lin2004some}, it takes the form
$$\hat{P}_{\ep,e} \psi|_{\Gamma_{i}(\rho)} = \frac{\oint_{\Gamma_i (\rho)} \frac{\phi}{|\nabla \psi_\ep|}dl}{\oint_{\Gamma_i (\rho)} \frac{1}{|\nabla \psi_\ep|}dl},$$
where $\rho$ is in the range of $\psi_\ep$ and $\Gamma_i(\rho)$ is a branch of $\{\psi_\ep = \rho\}$. Noting that $\tilde{X}_{\ep, e}\subset X_{\ep, e}^*$, we  define the operator
$$\hat{A}_{\ep,e} = - \Delta - g'(\psi_\ep)(I - \hat{P}_{\ep,e}): \tilde{X}_{\ep, e} \rightarrow \tilde{X}^*_{\ep, e}.$$
Then we have the following lemma.
\begin{lemma}\label{L e-hat A}
$$n^-\left( L_{\ep,e} |_{\overline{R(B_\ep)}} \right) = n^-\left(\hat{A}_{\ep,e}\right).$$
\end{lemma}
\begin{proof} Since $\hat{P}_{\ep,e}$  commutes with $f(\psi_\ep)\cdot$ for any function $f$,  $\omega \in \overline{R(B_\ep)}$ if and only if $\hat{P}_{\ep,e} \frac{\omega}{g'(\psi_\ep)} = 0$. Note that $\bar{P}_{\ep,e}$ is orthogonal in the $L^2$ sense.
For $\omega \in \overline{R(B_\ep)}\subset X_{\ep,e}$, there exists $\psi \in \tilde{X}_{\ep, e}$ such that $-\Delta\psi=\omega$ and
\begin{align*}
&\langle L_{\ep,e} \omega, \omega \rangle
= \iint_{\Omega_m} \left(\frac{\omega^2}{g'(\psi_\ep)} - \omega \psi\right) dxdy \\
= &\iint_{\Omega_m}\left(\bar{P}_{\ep,e}\left( \frac{\omega}{\sqrt{g'(\psi_\ep)}} - \psi \sqrt{g'(\psi_\ep)}\right)+(I-\bar{P}_{\ep,e})\left( \frac{\omega}{\sqrt{g'(\psi_\ep)}} - \psi \sqrt{g'(\psi_\ep)}\right) \right)^2 dxdy \\
&- \iint_{\Omega_m}\left(g'(\psi_\ep)\psi^2 -|\nabla \psi|^2\right) dxdy \\
=& \iint_{\Omega_m} \left(\left( \frac{\omega}{\sqrt{g'(\psi_\ep)}} - \sqrt{g'(\psi_\ep)} (I - \hat{P}_{\ep,e}) \psi \right)^2 + g'(\psi_\ep) (\hat{P}_{\ep,e} \psi)^2  - g'(\psi_\ep)\psi^2 + |\nabla \psi|^2 \right)dxdy \\
 \geq& \iint_{\Omega_m} \left(|\nabla \psi|^2 - g'(\psi_\ep)\psi^2 + g'(\psi_\ep) (\hat{P}_{\ep,e}\psi)^2 \right)dxdy = \langle\hat{A}_{\ep,e} \psi, \psi\rangle.
\end{align*}

For $\psi\in\tilde{X}_{\ep, e}$, we have $\tilde{\omega} \triangleq g'(\psi_\ep)(I - \hat{P}_{\ep,e})\psi \in \overline{R(B_\ep)}$. Let $\tilde{\psi} = (-\Delta)^{-1}\tilde{\omega}$. Then
\begin{align*}
\langle\hat{A}_{\ep,e} \psi, \psi\rangle
&= \iint_{\Omega_m} \left( |\nabla \psi|^2 - g'(\psi_\ep)((I - \hat{P}_{\ep,e})\psi)^2\right) dxdy \\
&= \iint_{\Omega_m} \left( |\nabla \psi|^2 - \frac{\tilde{\omega}^2}{g'(\psi_\ep)}\right)dxdy \\
&= \iint_{\Omega_m} |\nabla \psi|^2 - 2 \tilde{\omega} \psi + \frac{\tilde{\omega}^2}{g'(\psi_\ep)}dxdy \\
& \geq \iint_{\Omega_m}  \frac{\tilde{\omega}^2}{g'(\psi_\ep)} - |\nabla \tilde{\psi}|^2 dxdy = \langle L_{\ep,e}\tilde{\omega}, \tilde{\omega} \rangle,
\end{align*}
where we used $\langle\tilde \omega, \hat{P}_{\ep,e}\psi\rangle=0$.
From the two inequalities above, we have $n^-\left( L_{\ep,e} |_{\overline{R(B_\ep)}} \right) = n^-\left(\hat{A}_{\ep,e}\right)$.
\end{proof}
\fi

In summary, we have the following criterion for modulational instability of $\omega_\ep$.
\begin{lemma}\label{modulational case:unstable modes}
The number of unstable  modes of $J_{\ep,\alpha}L_{\ep,\alpha}$ is $n^-\left(L_{\alpha,e}|_{\overline{R(B_\alpha)}} \right)$, where $L_{\alpha,e}$ and $B_\alpha$ are defined in  \eqref{L-alpha-e B-alpha}. Consequently, if $n^-\left(L_{\alpha,e}|_{\overline{R(B_\alpha)}} \right)\geq1$, then $\omega_\ep$ is linearly modulationally unstable.
\end{lemma}

Let $ L_e^2(\Omega)=\{\phi\in L^2(\Omega)|\phi \text{ is even in } y\}$. Since the dual space of $ L_e^2(\Omega)$ is restricted into the class of even functions, we have $ L_e^2(\Omega)=(L_e^2(\Omega))^*$. To study $n^-\left(L_{\alpha,e}|_{\overline{R(B_\alpha)}} \right)$,  we define
$\bar{P}_{\alpha,e}$ to be the orthogonal   projection of the space $(L_e^2(\Omega))^*=L_e^2(\Omega)$ on $\ker(\vec{u}_\ep\cdot\nabla_\alpha)$. For $\widetilde{\psi}\in\ker(\vec{u}_\ep\cdot\nabla_\alpha)$, we have
$(\vec{u}_\ep\cdot\nabla)(\widetilde{\psi}e^{i\alpha x})=0$ and thus, $\widetilde{\psi}e^{i\alpha x}|_{\Gamma(\rho)}\equiv c_0$, where $\Gamma(\rho)$ is a connected closed curve of the level set $\{\psi_\ep=\rho\}$.  Recall that $\rho_0$ is defined in \eqref{def-rho0}.
For
$\rho\in[\rho_0,\infty)$, $\Gamma(\rho)$ is in the un-trapped regions. Since  $\widetilde{\psi}(0,y)=c_0=\widetilde{\psi}(2\pi,y)e^{2\alpha\pi i  }$ and $\widetilde{\psi}(0,y)=\widetilde{\psi}(2\pi,y)$, we have
\begin{align}\label{untrapped region-tilde-psi-e-ialphax}
\widetilde{\psi}e^{i\alpha x}|_{\Gamma(\rho)}\equiv c_0=0,
\end{align}
 and thus, $\widetilde{\psi}\equiv0$ in  the un-trapped regions.
For
$\rho\in[-\rho_0,\rho_0)$,  the level set $\{\psi_\ep=\rho\}$ is in the trapped region and it is exactly  one closed  curve $\Gamma(\rho)$.
Let $(X(s;x_0,y_0), Y(s;x_0,y_0))$ be the solution to the equation
\begin{align}\label{characteristic equation}
\begin{cases}
\dot{X}(s)=\partial_y\psi_\ep(X(s), Y(s)),\\
\dot{Y}(s)=-\partial_x\psi_\ep(X(s), Y(s)),
\end{cases}
\end{align}
with the initial data $X(0)=x_0,Y(0)=y_0$, where $(x_0,y_0)\in\Gamma(\rho)$. Then $\psi_{\ep}$ is conserved  along $\Gamma(\rho)$.
Let  $l_\rho$ be the arc length variable on $\Gamma(\rho)$
and $L_\rho(\ep)$ be the length of  $\Gamma(\rho)$.
 Along the trajectory, the particle solves
\begin{align*}
{d l_\rho(s)\over ds}=|\nabla\psi_\ep |(X(s;x_0,y_0),Y(s;x_0,y_0))
\end{align*}
 and the period of the particle motion is
\begin{align*}
T_\ep(\rho)=\int_0^{L_\rho(\ep)}{1\over|\nabla\psi_\ep|}dl_\rho.
\end{align*}
Define the action and angle variables by
\begin{align*}
I_\ep(\rho)&={1\over 2\pi}\int_{-\rho_0}^\rho\left(\int_0^{L_{\tilde\rho}(\ep)}{1\over|\nabla\psi_\ep|}dl_{\tilde \rho}\right)d\tilde \rho,\;\;
\theta_\ep={2\pi\over T_\ep(\rho)} \int_{0}^{l_\rho}{1\over |\nabla\psi_\ep|}dl_{\tilde \rho}.
\end{align*}
 Then $I_\ep$ is increasing on $\rho\in[-\rho_0,\rho_0)$ and  $0\leq \theta_\ep\leq  2\pi$. We define the inverse map of $I_\ep(\rho)$ by $\rho(I_\ep)$.
Define the frequency by
\begin{align*}
\vartheta_\ep(I_\ep)={2\pi\over T_\ep(\rho(I_\ep))}.
\end{align*}
The action-angle transform $(x,y)\rightarrow(I_\ep,\theta_\ep)$ is a smooth diffeomorphism with Jacobian $-1$.
 The  characteristic equation \eqref{characteristic equation} becomes
\begin{align*}
\begin{cases}
\dot{I}_\ep=0,\\
\dot{\theta}_\ep=\vartheta_\ep(I_\ep).
\end{cases}
\end{align*}
The transport operator $\vec{u}_{\ep}\cdot\nabla$ becomes
\begin{align*}
\vec{u}_{\ep}\cdot\nabla=\partial_y\psi_\ep\partial_x-\partial_x\psi_\ep\partial_y=
\vartheta_\ep(I_\ep)\partial_{\theta_\ep}.
\end{align*}
Thus, $\ker(\vartheta_\ep(I_\ep)\partial_{\theta_\ep})=\{f(I_\ep):f(I_\ep)\in L^2(\Omega)\;\text{and}\; f(I_\ep(\rho))=0\text{ for }\rho\in[\rho_0,\infty)\}=\{h(\psi_\ep):h(\psi_\ep)\in L^2(\Omega)\;\text{and}\; h(\psi_\ep)=0\text{ for }\psi_\ep\geq\rho_0\}=\ker(\vec{u}_{\ep}\cdot\nabla)$. Thus, $\ker(\vec{u}_{\ep}\cdot\nabla_\alpha)=\{h(\psi_\ep)e^{-i\alpha x}:h(\psi_\ep)\in L^2(\Omega)\;\text{and}\; h(\psi_\ep)=0\text{ for }\psi_\ep\geq\rho_0\}$. Let $\phi\in L_e^2(\Omega)$.
For any $\varphi=h(\psi_\ep)e^{-i\alpha x}\in \ker(\vec{u}_{\ep}\cdot\nabla_\alpha)$, we have
\begin{align*}
(\phi-\bar{P}_{\alpha,e}\phi,\varphi)_{L^2(\Omega)}=&\iint_{\Omega}(\phi-\bar{P}_{\alpha,e}\phi)\overline{ h(\psi_\ep)}e^{i\alpha x} dxdy\\
=&\int_{-\rho_0}^{\rho_0}\left(\oint_{\Gamma(\rho)}{(\phi-\bar{P}_{\alpha,e}\phi)\overline{ h(\psi_\ep)}e^{i\alpha x}\over|\nabla\psi_\ep|}\right)d\rho\\
=&\int_{-\rho_0}^{\rho_0}\overline{ h(\rho)}\left(\oint_{\Gamma(\rho)}{\phi e^{i\alpha x}\over|\nabla\psi_\ep|}-(\bar{P}_{\alpha,e}\phi e^{i\alpha x})|_{\Gamma(\rho)}\oint_{\Gamma(\rho)}{1\over|\nabla\psi_\ep|}
\right)d\rho=0,
\end{align*}
where we used $\bar{P}_{\alpha,e}\phi e^{i\alpha x}$ takes constant on $\Gamma(\rho)$ since $\bar{P}_{\alpha,e}\phi \in \ker(\vec{u}_{\ep}\cdot\nabla_\alpha)$.
This gives
\begin{align*}
(\bar{P}_{\alpha,e}\phi) |_{\Gamma(\rho)}=
\left\{ \begin{array}{lll} {\oint_{\Gamma(\rho)}{\phi e^{i\alpha x}\over|\nabla\psi_\ep|}\over\oint_{\Gamma(\rho)}{1\over|\nabla\psi_\ep|}}e^{-i\alpha x} &\mbox{ for } & \rho \in [-\rho_0, \rho_0 ),\\
 0 &\mbox{ for } & \rho\in [\rho_0, \infty).
 \end{array} \right.
\end{align*}
 It induces  a   projection
$\hat{P}_{\alpha,e}$ of $(L_{{1\over g'(\psi_\ep)},e}^2(\Omega))^*=L_{g'(\psi_\ep),e}^2(\Omega)$  on $\ker (B_\alpha')$ by $\hat{P}_{\alpha,e}=(S_e')^{-1}\bar{P}_{\alpha,e} S_e'$, where
$S_e: L_e^2(\Omega) \rightarrow L^2_{\frac{1}{g'(\psi_\ep)},e}(\Omega),  S_e\omega = g'(\psi_\ep)^{1/2}\omega$
defines an isometry. The dual space $(L_{{1\over g'(\psi_\ep)},e}^2(\Omega))^*$ is restricted into the class of even functions.
Noting that $L^2_{{g'(\psi_\ep)},e}(\Omega)= (L^2_{\frac{1}{g'(\psi_\ep)},e}(\Omega))^*$, we  define the operator
$$\hat{A}_{\alpha,e} = - \Delta_\alpha - g'(\psi_\ep)(I - \hat{P}_{\alpha,e}): L^2_{{g'(\psi_\ep)},e}(\Omega) \rightarrow L^2_{\frac{1}{g'(\psi_\ep)},e}(\Omega).$$
Similar to Lemma \ref{L e-hat A}, we can estimate $n^-\left( L_{\alpha,e} |_{\overline{R(B_\alpha)}} \right)$ by studying the negative directions of $\langle\hat{A}_{\alpha,e}\cdot,\cdot\rangle$.
\begin{lemma}\label{L e-hat A-alpha}
$$n^-\left( L_{\alpha,e} |_{\overline{R(B_\alpha)}} \right) = n^-\left(\hat{A}_{\alpha,e}\right).$$
In particular, the number of unstable  modes of $J_{\ep,\alpha}L_{\ep,\alpha}$ is $n^-\left(\hat{A}_{\alpha,e}\right)$. If $n^-\left(\hat{A}_{\alpha,e}\right)\geq1$, then $\omega_\ep$ is linearly modulationally unstable.
\end{lemma}
\subsection{Proof of modulational instability}
 To study the linear modulational instability of the Kelvin-Stuart vortex $\omega_\ep$, we construct the test function to be
  \begin{align}\label{test-function-modulational-instability}\widetilde\psi_{\ep,\alpha} =(1-\gamma_\ep^2)^{\alpha\over2}e^{i\alpha(\theta_\ep-x)}\in L^2_{{g'(\psi_\ep)},e}(\Omega),
   \end{align}
 which is an eigenfunction of the eigenvalue ${1\over2}\alpha\left( \alpha+1\right)$ for the associated eigenvalue problem \eqref{elip02-alpha} in Theorem \ref{sol to eigenvalue problem varepsilon=0-pde-alpha}, and confirm that
 \begin{align*}
\langle\hat{A}_{\alpha,e} \widetilde\psi_{\ep,\alpha}, \widetilde\psi_{\ep,\alpha} \rangle=  b_{\alpha, 1}(\widetilde\psi_{\ep,\alpha}) + b_{\alpha, 2}(\widetilde\psi_{\ep,\alpha}) < 0,
\end{align*}
where
\begin{align}\label{func-b1-alpha}
  b_{\alpha, 1}(\widetilde\psi_{\ep,\alpha}) = \iint_{\Omega} \left(|\nabla_\alpha \widetilde\psi_{\ep,\alpha}|^2  - g'(\psi_\ep) |\widetilde\psi_{\ep,\alpha}|^2 \right)dxdy
\end{align}
and
\begin{align}\label{func-b2-alpha}
 b_{\alpha, 2}(\widetilde\psi_{\ep,\alpha}) =  \iint_{\Omega} g'(\psi_\ep)( \hat{P}_{\alpha,e}\widetilde\psi_{\ep,\alpha})^2 dxdy =  \int_{-\rho_0}^{\rho_0} g'(\rho) \frac{\left|\oint_{\Gamma (\rho)} \frac{\widetilde\psi_{\ep,\alpha} e^{i \alpha x}}{|\nabla \psi_\ep|}\right|^2}{\oint_{\Gamma (\rho)} \frac{1}{|\nabla \psi_\ep|}} d\rho,
 \end{align}
 where $\rho_0 $ is defined in \eqref{def-rho0}.
Here,  $\Gamma (\rho)=\{\psi_\ep=\rho\}$ for $\rho\in[-\rho_0,\rho_0)$. Since $\widetilde\psi_{\ep,\alpha} $ is an eigenfunction of the eigenvalue ${1\over2}\alpha\left( \alpha+1\right)$ for \eqref{elip02-alpha}, we have
\begin{align}\label{func-b1-alpha-2}
  b_{\alpha, 1}(\widetilde\psi_{\ep,\alpha}) = 2\pi(\alpha(\alpha+1)-2)\int_{-1}^1(1-\gamma_\ep^2)^\alpha d\gamma_\ep.
\end{align}
To compute $ b_{\alpha, 2}(\widetilde\psi_{\ep,\alpha})$, we  convert the curve integrals to  definite integrals.
Note that $\Gamma(\rho) = \{(x, y) \in \Omega| \psi_\ep(x,y) = \rho\} $ is a closed level curve in the trapped region for $\rho\in(-\rho_0, \rho_0]$.
We divide $\Gamma(\rho)$ into two parts, namely, the upper part
\begin{align*}
\Gamma_{+} (\rho)
& = \{(x, y) \in \mathbb{T}_{2\pi} \times \mathbb{R}  \;|\;  \psi_\ep(x,y) = \rho,  y \geq 0\},
\end{align*}
and the lower part
\begin{align*}
\Gamma_{-} (\rho) = \{(x, y) \in \mathbb{T}_{2\pi} \times \mathbb{R}\;|\;  \psi_\ep(x,y) = \rho,  y < 0\}.
\end{align*}
 Using $x$ as the parameter,
we  represent $\Gamma_{+}(\rho)$ and $\Gamma_{-}(\rho)$ as follows:
$$\vec{r}_{+} (x) = (x, \cosh^{-1}(\sqrt{1-\ep^2} e^{\rho} - \ep \cos(x))), \quad x \in [ x_0, 2\pi - x_0], $$
and
$$\vec{r}_{-} (x) = (x, -\cosh^{-1}(\sqrt{1-\ep^2} e^{\rho} - \ep \cos(x))), \quad x \in ( x_0, 2\pi - x_0), $$
respectively. Here, $x_0 = \arccos\left( \frac{\sqrt{1-\ep^2} e^\rho - 1}{\ep} \right) $ is the point on $[0, \pi]$ such that $\psi_\ep(x_0,0)=\rho$.
Moreover, we have
\begin{align}\label{dr+}
\left|\frac{d \vec{r}_{\pm}(x)}{dx}\right| = \sqrt{ 1 + \left(\frac{\ep \sin(x)}{\sinh(y(x))}\right)^2},
\end{align}
where
\begin{align}\label{sinhyx}
 \sinh(y(x))=\sqrt{( \sqrt{1-\ep^2} e^{\rho} - \ep \cos(x) )^2 - 1},\\\nonumber
 y(x)=\cosh^{-1}(\sqrt{1-\ep^2} e^{\rho} - \ep \cos(x)).
 \end{align}
Noting that $\sinh(y(x_0))=\sinh(y(2\pi-x_0))=0$, $\left|\frac{d \vec{r}_{\pm}(x)}{dx}\right|$ is singular near $x_0$ and $2\pi-x_0$. To avoid the singularity, one might represent $\Gamma(\rho)$ in terms of the parameter $y$ near the two points $(x_0,0)$ and $(2\pi-x_0,0)$ if necessary.
Then we  represent $\left| \nabla \psi_\ep \right| $ and $\widetilde{\psi}_{\ep,\alpha}$ on $\Gamma_{+}(\rho)$ and $\Gamma_{-}(\rho)$ in terms of the parameter $x$. Since
$\psi_\ep(x,y) = \rho$, we have $\cosh(y) + \ep \cos(x) = e^\rho \sqrt{1-\ep^2}$. So
\begin{align}\label{tilde psi-p-x}
\left| \nabla \psi_\ep \right|
= & \left| \left( - \frac{\ep \sin(x)}{e^\rho \sqrt{1-\ep^2}}, \frac{\sinh(y)}{e^\rho \sqrt{1-\ep^2}} \right) \right|
=  \frac{\sqrt{\ep^2 \sin^2(x) + \sinh^2(y)}}{e^\rho \sqrt{1-\ep^2}}.
%\\\nonumber
%=&  \sqrt{1-e^{-2\rho} - \frac{2\ep \cos(x)}{e^\rho \sqrt{1-\ep^2}}},
\end{align}
By \eqref{dr+}-\eqref{tilde psi-p-x}, we have
\begin{align}\nonumber
&\oint_{\Gamma (\rho)} \frac{1}{|\nabla \psi_\ep|}=2\oint_{\Gamma_+ (\rho)} \frac{1}{|\nabla \psi_\ep|}=2\int_{x_0}^{2\pi-x_0}\frac{1}{|\nabla \psi_\ep|}\left|\frac{d \vec{r}_{+}(x)}{dx}\right|dx\\\label{Gamma-rho-nabla-psi1}
=&2\int_{x_0}^{2\pi-x_0}\frac{e^\rho \sqrt{1-\ep^2}}{ \sinh(y(x))}dx
=2e^\rho \sqrt{1-\ep^2}\int_{x_0}^{2\pi-x_0}\frac{1}{ \sqrt{(e^\rho \sqrt{1-\ep^2}-\ep \cos(x))^2-1} }dx
\end{align}
and
\if0
\begin{align}\label{psi-p-x}
\tilde{\psi}_\ep(x,y(x)) =& \cos\left(\frac{\theta_\ep}{2}\right)(1-\gep^2)^{1\over4} \\\nonumber
=& \left\{ \begin{array}{lll} \sqrt{\frac{1+\cos(\theta_\ep)}{2}} (1-\gep^2)^{1\over4} &\mbox{ for } & x \in [2(i-1)\pi + x_0, (2i - 1)\pi ],\\
 -\sqrt{\frac{1+\cos(\theta_\ep)}{2}} (1-\gep^2)^{1\over4} &\mbox{ for } & x \in ((2i - 1)\pi, 2i\pi - x_0],
 \end{array} \right.
\end{align}
where
\begin{align}\label{psi-p-x1}
1 - \gep^2 = 1 - \sinh^2(y)e^{-2\rho} = 1 - \left(\left( \sqrt{1-\ep^2} e^{\rho} - \ep \cos(x) \right)^2 - 1\right) e^{-2\rho},
\end{align}
and
\begin{align}\label{psi-p-x2}
\cos(\theta_\ep) = \frac{\xi_\ep}{\sqrt{1-\gep^2}} = \frac{\ep + \sqrt{1-\ep^2} \cos(x) e^{-\rho} }{\sqrt{1 - \left(\left( \sqrt{1-\ep^2} e^{\rho} - \ep \cos(x) \right)^2 - 1\right) e^{-2\rho}}}.
\end{align}
\fi
%We divide it into  the upper part
%$
%\Gamma_{+} (\rho)$
%and the lower part
%$\Gamma_{-} (\rho)$ as in \eqref{curve-Gamma-i+rho1} and \eqref{curve-Gamma-i+rho2}.
%By \eqref{dr+}, \eqref{tilde psi-p-x} and the fact that $\cosh(y) + \ep \cos(x) = e^\rho \sqrt{1-\ep^2}$, we have
\begin{align}\nonumber
&\oint_{\Gamma (\rho)} \frac{\widetilde\psi_{\ep,\alpha} e^{i \alpha x}}{|\nabla \psi_\ep|}=2\oint_{\Gamma_+ (\rho)} \frac{\widetilde\psi_{\ep,\alpha} e^{i \alpha x}}{|\nabla \psi_\ep|}=2\int_{x_0}^{2\pi-x_0}\frac{e^\rho \sqrt{1-\ep^2}(1-\gamma_\ep^2)^{\alpha\over2}e^{i\alpha\theta_\ep}}{ \sinh(y(x))}dx\\\label{Gamma-rho-nabla-tilde-psi2}
=&2e^\rho \sqrt{1-\ep^2}\int_{x_0}^{2\pi-x_0}\frac{(1-\gamma_\ep^2)^{\alpha\over2}(\cos(\alpha\theta_\ep)+i\sin(\alpha\theta_\ep))}{ \sqrt{(e^\rho \sqrt{1-\ep^2}-\ep \cos(x))^2-1} }dx,
\end{align}
where $x_0 = \arccos\left( \frac{\sqrt{1-\ep^2} e^\rho - 1}{\ep} \right)$,
\begin{align}\nonumber
1-\gamma_\ep^2&=1-\sinh^2(y)e^{-2\rho}=1-\left((e^\rho \sqrt{1-\ep^2}-\ep \cos(x))^2-1\right)e^{-2\rho},\\\label{1-gamma-ep-2-expression}
\cos(\theta_\ep) &= \frac{\xi_\ep}{\sqrt{1-\gep^2}} = \frac{\ep + \sqrt{1-\ep^2} \cos(x) e^{-\rho} }{\sqrt{1 - \left(\left( \sqrt{1-\ep^2} e^{\rho} - \ep \cos(x) \right)^2 - 1\right) e^{-2\rho}}}.
\end{align}
 \eqref{func-b1-alpha-2}, \eqref{func-b2-alpha} and \eqref{Gamma-rho-nabla-psi1}-\eqref{Gamma-rho-nabla-tilde-psi2} give the explicit expression of $\langle\hat{A}_{\alpha,e} \widetilde\psi_{\ep,\alpha}, \widetilde\psi_{\ep,\alpha} \rangle=  b_{\alpha, 1}(\widetilde\psi_{\ep,\alpha}) + b_{\alpha, 2}(\widetilde\psi_{\ep,\alpha})$.
 \begin{figure}[ht]
    \centering
	\includegraphics[width=0.56\textwidth]{Figure-alpha.png}
	\caption{The value of $\langle\hat{A}_{\alpha,e} \widetilde\psi_{\ep,\alpha}, \widetilde\psi_{\ep,\alpha} \rangle$}
	\label{fig:3rdFig}
\end{figure}
 The integrals in the expression are computable, and  we compute $\langle\hat{A}_{\alpha,e} \widetilde\psi_{\ep,\alpha}, \widetilde\psi_{\ep,\alpha} \rangle$ as a real-valued function of $(\alpha,\epsilon)$ by Python.
The values of  $\langle\hat{A}_{\alpha,e} \widetilde\psi_{\ep,\alpha}, \widetilde\psi_{\ep,\alpha} \rangle$ are given in Figure \ref{fig:3rdFig}, and it reveals
 that
 \begin{align}\label{test-function-modulational-instability-quadratic form}
 \max_{\alpha\in(0,{1\over2}],\ep\in[0,1)}\langle\hat{A}_{\alpha,e} \widetilde\psi_{\ep,\alpha}, \widetilde\psi_{\ep,\alpha} \rangle=\langle\hat{A}_{\alpha,e} \widetilde\psi_{\ep,\alpha}, \widetilde\psi_{\ep,\alpha} \rangle|_{\alpha=0.01,\ep=0.99}=-0.78<0.
 \end{align}
Now, we are in a position to prove linear modulational instability for the family of steady states $\omega_\ep$, $\ep\in[0,1)$.

\begin{proof}[Proof of Theorem \ref{main result3-modulation-unstable}]
With the test function  $\widetilde\psi_{\ep,\alpha}$ defined in \eqref{test-function-modulational-instability},  we infer from \eqref{test-function-modulational-instability-quadratic form} that
$\langle\hat{A}_{\alpha,e} \widetilde\psi_{\ep,\alpha}, \widetilde\psi_{\ep,\alpha} \rangle<0$ for $\alpha\in(0,{1\over2}]$ and $\ep\in[0,1)$. Thus,
the number of unstable  modes of $J_{\ep,\alpha}L_{\ep,\alpha}$ is $n^-\left( L_{\alpha,e} |_{\overline{R(B_\alpha)}} \right) = n^-\left(\hat{A}_{\alpha,e}\right)\geq1$  by Lemma \ref{L e-hat A-alpha}. This proves  linear modulational instability of $\omega_\ep$.
\end{proof}
\begin{remark} \label{modulational-remark}
 %(1) The proof of Theorem \ref{main result3-modulation-unstable} is mostly theoretical, and the  only computer assistant part is  the calculation of the integral in \eqref{func-b2-alpha}-\eqref{func-b1-alpha-2}, where the integrand of the integral in \eqref{func-b2-alpha} is given  explicitly in \eqref{Gamma-rho-nabla-psi1}-\eqref{Gamma-rho-nabla-tilde-psi2}.

For the hyperbolic tangent shear flow ($\ep=0$), the trapped region vanishes and by \eqref{untrapped region-tilde-psi-e-ialphax}, we have $\ker(\vec{u}_0\cdot\nabla_\alpha)=\{0\}$ for $\alpha\in(0,{1\over2}]$. Thus,
 $ {\overline{R(B_\alpha)}}={L^2_{\frac{1}{g'(\psi_0)},e}(\Omega)}$. By Corollary $\ref{A-L-dec-e-alpha}$, $n^-( L_{\alpha,e} )|_{\ep=0}=n^-( L_{\ep,\alpha})|_{\ep=0}=2$. We infer from Lemma $\ref{modulational case:unstable modes}$ that for any modulational parameter $\alpha\in(0,{1\over2}]$, the number of   unstable modes in  the shear case is $2$. This also indicates that for fixed $\alpha\in(0,{1\over2}]$, the number of   unstable modes for the Kelvin-Stuart vortex $\omega_\ep$ with $\ep\ll1$ is $2$.


\if0
 (2) Let $\lambda_0$ be the most unstable eigenvalue of  $\partial_t \widetilde{\omega} = J_{\ep, \alpha} L_{\ep, \alpha} \widetilde{\omega}$. Then $\hat B_\alpha'\hat L_{\alpha,e}\hat B_\alpha \hat L_{\alpha,o}\vec{\omega}_\alpha=-\hat\lambda_\alpha^2\vec{\omega}_\alpha$ for some $\vec{\omega}_\alpha\in X_{\alpha,o}$ and
 \begin{align*}
 -\lambda_\alpha^2=\min\limits_{\langle \hat L_{\alpha,o}\vec{\omega}_o,\vec{\omega}_o\rangle=1,\vec{\omega}_o\in D( \hat L_{\alpha,o})}\langle\hat B_\alpha'\hat L_{\alpha,e}\hat B_\alpha \hat L_{\alpha,o}\vec{\omega}_o,\hat L_{\alpha,o}\vec{\omega}_o\rangle\approx\min\limits_{ \omega\in \overline{R(B_\alpha)}}{\langle\hat L_{\alpha,e}\omega,\omega\rangle\over \|\omega\|_{L^2_{\frac{1}{g'(\psi_\ep)}}(\Omega)}^2}.
 \end{align*}
By \eqref{test-function-modulational-instability-quadratic form}, $\langle\hat{A}_{\alpha,e} \widetilde\psi_{\ep,\alpha}, \widetilde\psi_{\ep,\alpha} \rangle\leq-0.78<0$   for $\alpha\in(0,{1\over2}]$ and $\ep\in[0,1)$.
Let $\widetilde \omega_{\ep,\alpha}=g'(\psi_\ep)(I - \hat{P}_{\alpha,e})\widetilde\psi_{\ep,\alpha}$. Then similar to \eqref{def-hat-A-ep-e}, we have
$\langle L_{\alpha,e}\widetilde \omega_{\ep,\alpha},\widetilde \omega_{\ep,\alpha}\rangle\leq \langle\hat{A}_{\alpha,e} \widetilde\psi_{\ep,\alpha}, \widetilde\psi_{\ep,\alpha} \rangle\leq-0.78<0$.
Moreover,
$$\|\widetilde \omega_{\ep,\alpha}\|_{L^2_{\frac{1}{g'(\psi_\ep)}}(\Omega)}^2=\iint_{\Omega}g'(\psi_\ep)|(I - \hat{P}_{\alpha,e})\widetilde\psi_{\ep,\alpha}|^2dxdy\leq \|\widetilde\psi_{\ep,\alpha}\|_{L^2_{{g'(\psi_\ep)}}(\Omega)}^2\leq C.$$
 Thus,  \begin{align*}
 -\lambda_\alpha^2\leq {\langle\hat L_{\alpha,e}\widetilde \omega_{\ep,\alpha},\widetilde \omega_{\ep,\alpha}\rangle\over \|\widetilde \omega_{\ep,\alpha}\|_{L^2_{\frac{1}{g'(\psi_\ep)}}(\Omega)}^2}\leq{-0.78\over C}<0.
 \end{align*}
This implies that  the growth rates of the most unstable direction has a positive lower bound even if $\alpha\to0^+$.
In the co-periodic case ($\alpha=0$), by Corollaries \ref{kernel of  the operator tilde A0 and a decomposition of tilde X0} and \ref{kernel of  the operator A-ep and a decomposition of tilde Xep}, we have $\langle A_\ep \psi,\psi\rangle\geq 0$ for all $\psi\in X_\ep$, $\ep\in[0,1)$.
Thus, the modulational case with $|\alpha|\ll1$ is a singular perturbation for the co-periodic case. Indeed,  the patterns  of the perturbed velocity  \eqref{perturbed  velocity} do not imply
  $\iint_\Omega \omega dxdy=0$ for  vorticity perturbations and $\tilde \psi\in H^1(\Omega)$   as long as $\alpha\neq0$, no matter how small it is.
By Theorem \ref{sol to eigenvalue problem varepsilon=0-pde-alpha}, the principal eigenvalue and a corresponding eigenfunction  of the eigenvalue problem \eqref{elip02-alpha} are
${1\over2}\alpha\left(\alpha+1\right)$ and
$(1-\gamma_\ep^2)^{\alpha\over2}$. Taking limits as $\alpha\to0^+$, we have ${1\over2}\alpha\left(\alpha+1\right)\to 0$ and
$(1-\gamma_\ep^2)^{\alpha\over2}\to1$.  Even though the limits solve the Legendre equation, they make no contribution to the spectra of the operator $ A_\ep$ in the co-periodic case due to  the projection terms \eqref{P0-psi-def} and \eqref{P-ep}, which can be traced back to the physical condition $\iint_\Omega \omega dxdy=0$.
\fi


\end{remark}
Finally, we give the relations between multi-periodic instability and modulational instability.
\begin{lemma}\label{multi-periodic-modulation} Let $\ep\in[0,1)$.
$(1)$ If the steady state $\omega_\ep$ is linearly  $2m\pi$-periodic unstable for some $m\geq2$, then there exists an integer  $1\leq\hat l\leq m-1$ such that $\omega_\ep$ is linearly modulationally unstable
for $\alpha={\hat l\over m}$.

$(2)$ If the steady state $\omega_\ep$ is linearly modulationally unstable
for some rational number $\alpha={p\over q}\in (0,{1\over2}]$ with $p,q\in\mathbb{Z}^+$, then $\omega_\ep$ is linearly  $2q\pi$-periodic unstable.
\end{lemma}
\begin{proof} (1) Let ${\lambda}_* $  be an unstable eigenvalue of $J_{\ep,m} L_{\ep,m} $ with an eigenfunction ${\omega}_*\in X_{\ep,m}$. Then
$${\omega}_*(x, y) = \sum_{k \in \mathbb{Z}} e^{\frac{ikx}{m}} \widehat{\omega}_{*,k}(y)  = \sum_{l = 0}^{m-1} e^{\frac{ilx}{m}} \omega_{*, l} (x, y), $$
where
$$\omega_{*, l} (x, y) = \sum_{n \in \mathbb{Z}} e^{inx} \widehat{\omega}_{*,mn + l}(y) \in L^2_{\frac {1} {g'(\psi_\epsilon)}}(\Omega),\quad 0\leq l\leq m-1.$$
Since $J_{\ep,m} L_{\ep,m}{\omega}_*={\lambda}_*{\omega}_*$, we have
\begin{align*}
J_\ep L_\ep {\omega}_{*, 0} + \sum_{l = 1}^{m-1} e^{\frac{ilx}{m}} J_{\ep, \frac{l}{m}}L_{\ep, \frac{l}{m}} {\omega}_{*, l} = {\lambda}_* \left({\omega}_{*, 0} + \sum_{l = 1}^{m-1} e^{\frac{ilx}{m}} {\omega}_{*, l}\right).
\end{align*}
By induction,
\begin{align*}
J_\ep L_\ep {\omega}_{*, 0} = {\lambda}_* {\omega}_{*, 0}\quad\text{and}\quad
J_{\ep, \frac{l}{m}}L_{\ep, \frac{l}{m}} {\omega}_{*, l}  = {\lambda}_* {\omega}_{*, l}  \quad\text{for} \quad l = 1, \cdots, m-1.
\end{align*}
By Theorem \ref{main result1-co-periodic perturbations},
 $\omega_\ep$ is spectrally stable for co-periodic perturbations. This, along with $Re({\lambda}_*)>0$,  implies that
${\omega}_{*, 0}\equiv 0$. Thus, there exists $1 \leq  \hat{l} \leq m - 1$ such that
$ {\omega}_{*,\hat{l}} \not\equiv 0$ and
$$J_{\ep, \frac{\hat{l}}{m}}L_{\ep, \frac{\hat{l}}{m}} {\omega}_{*, \hat{l}} = {\lambda}_*{\omega}_{*, \hat{l}},$$
which gives modulational instability of $\omega_\ep$ for $\alpha = \frac{\hat{l}}{m}$.

For $\alpha={p\over q}$, let $\lambda_\alpha$ be an unstable eigenvalue of $J_{\ep, \alpha}L_{\ep, \alpha}$ with an eigenfunction $\omega_\alpha$. Then  $e^{i\alpha x}\omega_\alpha$ is $2q\pi$-periodic in $x$ and
\begin{align}\label{modu-multi}J_{\ep,q}L_{\ep,q}(e^{i\alpha x}\omega_\alpha )=e^{i\alpha x}J_{\ep, \alpha}L_{\ep, \alpha}\omega_\alpha=\lambda_\alpha e^{i\alpha x}\omega_\alpha.
\end{align}
By \eqref{index-formula-for-modulational-instabilitykc}, $\lambda_\alpha$ is real-valued. By separating the real and imaginary parts in \eqref{modu-multi}, we know that $\lambda_\alpha$ is an unstable eigenvalue of $J_{\ep,q}L_{\ep,q}$.
\end{proof}
\begin{remark}
By Lemma $\ref{multi-periodic-modulation}$ $(1)$ for $m=2$ and Theorem $\ref{multi-even}$ for $k=1$,  again we confirm  that $\omega_\ep$ is linearly modulationally unstable
for $\alpha={1\over 2}$ and $\ep\in[0,1)$.

Motivated by the test function \eqref{test-even} for  $4\pi$-periodic perturbations, we give an alternative test function
 \begin{align*}
 \widetilde\phi_{\ep,{1\over2}} =\left({1+e^{-i\theta_\ep}\over 2}\right)(1-\gamma_\ep^2)^{1\over4}e^{{i\over2}(\theta_\ep-x)}\in L^2_{{g'(\psi_\ep)},e}(\Omega)
 \end{align*}
 for $\ep\in[0,1)$ and $\alpha={1\over2}$. The advantage of $ \widetilde\phi_{\ep,{1\over2}}$ is that  $b_{\alpha, 2}(\widetilde\phi_{\ep,\alpha})|_{\alpha={1\over2}}=0$ since $ \widetilde\phi_{\ep,{1\over2}} e^{{i\over2}x}=\cos\left({1\over2}\theta_\ep\right)(1-\gamma_\ep^2)^{1\over 4}$ is odd symmetrical about $(x,y)=(\pi,0)$.
 By \eqref{b1-even}, we have $b_{\alpha, 1}(\widetilde\phi_{\ep,\alpha})|_{\alpha={1\over2}}=-{5\over8}\pi^2$.
 Here, $b_{\alpha, 1}$ and $b_{\alpha, 2}$ are defined in \eqref{func-b1-alpha}-\eqref{func-b2-alpha}. Thus,
 $\langle\hat{A}_{\alpha,e} \widetilde\phi_{\ep,\alpha}, \widetilde\phi_{\ep,\alpha} \rangle|_{\alpha={1\over2}}=-{5\over8}\pi^2<0$ for $\ep\in[0,1)$.
\end{remark}






\section{Nonlinear orbital stability for co-periodic perturbations}\label{Sec-Nonlinear orbital stability for co-periodic perturbations}

In this section, we prove  nonlinear orbital stability for the Kelvin-Stuart vortices  $\omega_\ep$, $\ep\in(0,1)$.

\subsection{The pseudoenergy-Casimir  functional and the distance functional}
First, we separate the perturbed stream function $\tilde \psi=\psi_\ep+\psi$ in a combination of the steady part $\psi_{\ep}(x, y)$ and the perturbation part  $\psi(x,y)$,
where $\psi_{\ep}(x, y)=\ln \left(\frac{\cosh (y) + \epsilon \cos (x)}{\sqrt{1-\epsilon^2}}\right)$. Correspondingly, the perturbed  velocity and vorticity can be written as $\vec{u}_\ep+\vec{u}$ and $\tilde\omega=\omega_\ep + \omega$, respectively. Now, the nonlinear vorticity equation \eqref{vor} becomes
\begin{equation}\label{vorr}
\partial_t \omega + \{\omega_\ep + \omega, \psi_\ep + \psi\} = 0.
\end{equation}
By Proposition 4.4 in \cite{Milisic-Razafison13}, the Green function $G(x,y)$ solving
\begin{align*}
-\Delta \phi=\delta(0,0) \quad \text{on}\quad \Omega
\end{align*}
is
\begin{align}\label{green function}
G(x,y)=-{1\over 4\pi}\ln(\cosh(y)-\cos(x)),
\end{align}
which can also be obtained by   \eqref{catseye}-\eqref{steadyw} for the point vortex case ($\ep=-1$). Note that the total  energy ${1\over2}\iint_{\Omega}|\vec{u}_\ep+\vec{u}|^2dxdy$ is not finite since $\vec{u}_\ep\to(\pm1,0)$ as $y\to \pm \infty$.
Motivated by \cite{Majda-Bertozzi02}, we introduce an alternative bounded  functional called the pseudoenergy:
\begin{align}\label{def-pseudoenergy}
PE(\tilde \omega)={1\over2}\iint_{\Omega}(G\ast\tilde \omega)\tilde \omega dxdy,
\end{align}
where $\tilde\omega\in Y_{non}$ defined in \eqref{def-X-non-ep} and $G\ast\tilde \omega$ is the usual convolution of $G$ and $\tilde \omega$ on $\Omega$.
%Since $\cosh(y)-\cos(x)\leq e^{|y|}+1\leq 2e^{|y|}$, we have
By Proposition 4.4 in  \cite{Milisic-Razafison13}, $G=G_1+G_2$, where $G_1\in L^1\cap L^2(\Omega)$ and $G_2(x,y)=-{1\over4\pi} |y|$.
Then
\begin{align}\nonumber
|PE(\tilde \omega)|\leq& \left|{1\over2}\iint_{\Omega}(G_1\ast\tilde \omega)\tilde \omega dxdy\right|
+\left|{1\over2}\iint_{\Omega}(G_2\ast\tilde \omega)\tilde \omega dxdy\right|\\\nonumber
\leq &{1\over2}\|G_1\ast\tilde \omega\|_{L^2(\Omega)}\|\tilde \omega\|_{L^2(\Omega)}+{1\over8\pi}\iint_{\Omega}\left(\iint_{\Omega}(|y|+|\tilde y|)\tilde \omega(\tilde x,\tilde y)d\tilde x d\tilde y\right)\tilde \omega(x,y) dxdy\\\label{PE-finite}
\leq&{1\over2}\|G_1\|_{L^1(\Omega)}\|\tilde \omega\|_{L^2(\Omega)}^2+{1\over 4\pi}\|y\tilde\omega\|_{L^1(\Omega)}\|\tilde\omega\|_{L^1(\Omega)}<\infty
\end{align}
for $\tilde\omega\in Y_{non}$.
The relative  pseudoenergy (for the perturbation part) is
\begin{align*}
E_\ep(\omega)=PE(\tilde \omega)-PE(\omega_{\ep})={1\over2}\iint_{\Omega}\left((G\ast\tilde \omega)\tilde \omega -(G\ast\omega_{\ep}) \omega_{\ep}\right) dxdy,
 %= \frac 1 2  \iint_{\Omega} \left(|\vec{u}_\ep+\vec{u}|^2  - |\vec{u}_\ep|^2 \right)dxdy   = \frac 1 2 \iint_{\Omega} \left(|\nabla(\psi_\ep + \psi)|^2 -  |\nabla \psi_\ep|^2\right) dxdy\\
%& = \iint_{\Omega}\left( \frac 1 2  |\nabla \psi|^2 + \psi_\ep \omega \right)dxdy,
\end{align*}
where $\omega=\tilde \omega-\omega_{\ep}$.
%where $\psi=(-\Delta)^{-1}\omega$ is determined  in Lemma \ref{well-poseness-Poisson-equation-nonlinear-case} for $\omega\in X_{non,\ep}$.
%Since $\{\omega_\ep + \omega, \cdot\}$ is anti-symmetric, we have
%$$\frac{d E_\ep(\omega)}{dt} =  -\iint_{\Omega}(\psi_\ep+\psi)\{\omega_\ep + \omega,\psi_\ep+\psi\}dxdy = 0.$$
%Thus,  $E_\ep(\omega)$ is invariant.
To study the nonlinear stability of $\omega=0$,  we construct a Lyapunov functional for the evolved system \eqref{vorr}. Since  $\omega_\epsilon = g(\psi_\epsilon) = - e^{-2\psi_\epsilon}$, we have $\psi_\epsilon=g^{-1}(\omega_\epsilon)=-{1\over 2}\ln(-\omega_\ep)$.
Define $h(s)={1\over2}(s-s\ln(-s))$ for $s<0$. Then $h'(\omega_\ep)=-{1\over2}\ln(-\omega_\ep)=\psi_\ep$.
Following  Arnol$'$d \cite{Arnold65,Arnold69}, we use the pseudoenergy-Casimir (PEC) functional for the perturbation of vorticity
\begin{align}\nonumber H_\ep(\omega)&= \iint_{\Omega}h(\omega_\ep + \omega )dxdy - E_\ep(\omega)\\\nonumber
&={1\over2}\iint_{\Omega}\left(((\omega_\ep + \omega)-(\omega_\ep + \omega)\ln(-\omega_\ep - \omega))-(G\ast\tilde \omega)\tilde \omega +(G\ast\omega_{\ep}) \omega_{\ep}\right)dxdy.
\end{align}
Then $\omega=0$ is a critical point of $H_\ep$ since
$$H_\ep'(0)= h'(\omega_\epsilon)  - \psi_\epsilon = 0,$$
where $H_\ep'$ is the  variational derivative of the functional $H_\ep$.
The space of the perturbed vorticity is defined  in \eqref{def-X-non-ep}
and the space of vorticity perturbations is denoted by
\begin{align*}
X_{non,\ep}=\{\omega=\tilde \omega-\omega_\ep|\tilde \omega\in Y_{non}\}.
\end{align*}
The  PEC functional is well-defined in $X_{non,\ep}$ since
 $-\tilde \omega\ln(-\tilde \omega)\in L^1(\Omega)$ by Lemma \ref{tilde-omega0-kappa-properties} (8).
Note that the steady state  $\tilde \omega_\ep$ is pointwise negative, and  in the analysis of nonlinear stability, we consider the perturbed vorticity in the same fashion.
\if0
It is interesting to study nonlinear stability for  sign-changed  perturbed vorticity and the difficulty is to control the pseudoenergy for the positive part of vorticity by some non-weighted norm of vorticity.
 % since $|\tilde\omega|\ln|\tilde\omega|\leq |\tilde\omega|+|\tilde\omega|^2$.
 \fi
We prove the existence of weak solutions to the nonlinear 2D Euler equation with  vorticity in $Y_{non}$ in the appendix.
Now, we prove the existence and uniqueness of weak solutions to the Poisson equation.
\begin{lemma}\label{well-poseness-Poisson-equation-nonlinear-case}
For $\ep\in[0,1)$ and $\omega \in X_{non,\ep}$, the Poisson equation
$$-\Delta \psi = \omega$$
has a unique weak solution in $\tilde{X}_\ep$, which is defined in \eqref{tilde-X0} for $\ep=0$ and \eqref{tilde-X-e} for $\ep\in(0,1)$.
\end{lemma}
\begin{proof}
For $\phi \in \tilde{X}_\ep$, similar to \eqref{psi-m-dec} we split it into the shear part  $ \widehat\phi_0 $ and the non-shear part $\phi_{\neq0}$. Then
$\|\widehat{\phi}_0\|_{\dot{H}^1(\mathbb{R})}\leq \|\phi\|_{\tilde{X}_\ep}$ and   $\|\phi_{\neq0}\|_{H^1(\Omega)}\leq C\|\phi_{\neq0}\|_{\tilde{X}_\ep}.$
Since $\iint_{\Omega}\omega dxdy=0$, we have
\begin{align*}\iint_{\Omega} \omega\widehat{\phi}_0 dxdy& = \iint_{\Omega} \omega \left(\widehat{\phi}_0(y)-\widehat{\phi}_0(0)\right) dxdy \leq
\|\phi\|_{\tilde{X}_\ep}\iint_{\Omega} |\omega| \sqrt{|y|} dxdy\\
&\leq\|\phi\|_{\tilde{X}_\ep}\left(\iint_\Omega|\omega_\ep|\sqrt{|y|}dxdy+\|y\tilde \omega \|_{L^1(\Omega)}^{1\over2}\|\tilde \omega\|_{L^1(\Omega)}^{1\over2}\right)
\leq C\|\phi\|_{\tilde{X}_\ep},\\
\iint_{\Omega} \omega \phi dxdy
& = \iint_{\Omega} \omega \widehat\phi_0 dxdy + \iint_{\Omega} \omega \phi_{\neq0} dxdy \\
& \leq C\|\phi\|_{\tilde{X}_\ep} + \|\omega\|_{L^2(\Omega)}  \|\phi_{\neq0}\|_{L^2(\Omega)}\leq C\|\phi\|_{\tilde{X}_\ep}.
\end{align*}
By the Riesz Representation Theorem, there exists a unique $\psi \in \tilde{X}_\ep$ such that
$$\iint_{\Omega} \omega \phi dxdy = \iint_{\Omega}\nabla \psi \cdot \nabla\phi dxdy,\quad \phi \in \tilde{X}_\ep.$$
\end{proof}

For $\omega=\tilde \omega-\omega_{\ep}$, we give the relation between $G\ast\omega$ and the weak solution $\psi$  in Lemma \ref{well-poseness-Poisson-equation-nonlinear-case}.
\begin{lemma}\label{G-ast-omega-psi-constant}
$G\ast\omega-\psi$ is a constant for $\omega=\tilde \omega-\omega_{\ep}$, where $\ep\in[0,1)$, $\tilde \omega\in Y_{non}$ and $\psi\in\tilde{X}_\ep$ is the weak solution of $-\Delta \psi = \omega$.
\end{lemma}
\begin{proof}
Since $G=G_1+G_2$, $G_1\in L^1\cap L^2(\Omega)$ and $G_2(x,y)=-{1\over4\pi} |y|$, we have
\begin{align}\label{G-omega-conv}
|(G\ast\omega)(x, y)|\leq \|G_1\|_{L^2(\Omega)}\|\omega\|_{L^2(\Omega)}+{1\over4\pi}\left|\iint_{\Omega}|y-\tilde y|(\tilde\omega-\omega_{\ep})(\tilde x,\tilde y)d\tilde xd\tilde y\right|.
\end{align}
Let $B_R=\{x\in\mathbb{T}_{2\pi},y\in[-R,R]\}$. Note that $\iint_{\Omega}(\tilde \omega-\omega_{\ep})dxdy=0$ and $\tilde \omega-\omega_{\ep}\in L^1(\Omega)$. For any $\kappa>0$, there exists $R_{\kappa}>0$ such that
\begin{align*}
\left|\iint_{B_{R_\kappa}}(\tilde\omega-\omega_{\ep})dxdy\right|<\kappa\quad\text{and}\quad
\iint_{B_{R_\kappa}^c}|\tilde\omega-\omega_{\ep}|dxdy<\kappa.
\end{align*}
Thus, for $| y|>R_{\kappa}$, we have
\begin{align}\nonumber
&\left|\iint_{\Omega}|y-\tilde y|(\tilde\omega-\omega_{\ep})(\tilde x,\tilde y)d\tilde xd\tilde y\right|\\\nonumber
\leq&\left|\iint_{B_{R_\kappa}}(y-\tilde y)(\tilde\omega-\omega_{\ep})(\tilde x,\tilde y)d\tilde xd\tilde y\right|+\iint_{B_{R_\kappa}^c}|y-\tilde y||(\tilde\omega-\omega_{\ep})(\tilde x,\tilde y)|d\tilde xd\tilde y\\\nonumber
\leq &\kappa| y|+\|y(\tilde\omega-\omega_{\ep})\|_{L^1(B_{R_\kappa})} +\kappa| y|+\|y(\tilde\omega-\omega_{\ep})\|_{L^1(B_{R_\kappa}^c)}\\\label{G-omega-conv2}
\leq &2\kappa | y|+C.
\end{align}
Combining \eqref{G-omega-conv} and \eqref{G-omega-conv2}, we have for $| y|>R_{\kappa}$,
\begin{align}\label{G-omega-conv-sum}
|(G\ast\omega)( x, y)|\leq {\kappa\over2\pi} | y|+C.
\end{align}
Since $\psi=\widehat\psi_0+\psi_{\neq0}\in \tilde X_{\ep}$, we have
\begin{align}\label{widehat-psi-property}
|\widehat\psi_0(y)|\leq \|\widehat\psi_0'\|_{L^2(\mathbb{R})}|y|^{1\over2}+|\widehat\psi_0(0)|\leq C|y|^{1\over2}+C \text{ and }\psi_{\neq0}\in  H^1(\Omega),
\end{align}
where $\widehat\psi_0$ and $\psi_{\neq0}$  are the shear part  and  the non-shear part of $\psi$, respectively.
 Since $
-\Delta(G\ast \omega-\psi)=0$, we have $G\ast \omega-\psi=\sum_{j\neq0} e^{ijx}(d_{1j}e^{jy}+d_{2j}e^{-jy})+c_1y+c_2$, where  $d_{1j}, d_{2j}, c_1, c_2\in \mathbb{R}$ for $j\neq0$. By \eqref{G-omega-conv-sum}-\eqref{widehat-psi-property}, $d_{1j}, d_{2j}, c_1=0$ for $j\neq0$, and thus, $G\ast \omega-\psi=c_2$.
\end{proof}

Note that $\lim_{y\to\pm\infty}\partial_y\psi_{\ep}(x,y)=\pm1$ for fixed $x\in \mathbb{T}_{2\pi}$.
By a similar argument to \eqref{v-mu term1}, we have $\lim_{y\to\pm\infty}(\partial_y G* \omega_\ep)(x,y)=\pm1$ for fixed $x\in \mathbb{T}_{2\pi}$, and thus,
$G\ast\omega_{\ep}-\psi_{\ep}$ is a constant.
Since $\iint_{\Omega}(G\ast\omega_{\ep}) \tilde \omega dxdy=\iint_{\Omega}(G\ast\tilde \omega) \omega_{\ep} dxdy$, by Lemma \ref{G-ast-omega-psi-constant} we have
\begin{align*}
E_\ep(\omega)=&PE(\tilde \omega)-PE(\omega_{\ep})={1\over2}\iint_{\Omega}\left((G\ast\tilde \omega)\tilde \omega -(G\ast\omega_{\ep}) \omega_{\ep}\right) dxdy\\
=&{1\over2} \iint_{\Omega}\left((G\ast\tilde \omega)\tilde \omega -(G\ast\omega_{\ep}) \tilde \omega \right) dxdy+{1\over2} \iint_{\Omega}(G\ast\omega_{\ep}) (\tilde \omega-\omega_{\ep}) dxdy\\
=&{1\over2} \iint_{\Omega}(G\ast\tilde \omega)(\tilde \omega-\omega_{\ep}) dxdy+{1\over2} \iint_{\Omega}\psi_{\ep}  \omega dxdy\\
=&{1\over2} \iint_{\Omega}(\psi_{\ep}+\psi)\omega dxdy+{1\over2} \iint_{\Omega}\psi_{\ep}  \omega dxdy= \iint_{\Omega}\psi_{\ep}\omega dxdy+{1\over2}\iint_{\Omega}|\nabla \psi|^2dxdy,
 %= \frac 1 2  \iint_{\Omega} \left(|\vec{u}_\ep+\vec{u}|^2  - |\vec{u}_\ep|^2 \right)dxdy   = \frac 1 2 \iint_{\Omega} \left(|\nabla(\psi_\ep + \psi)|^2 -  |\nabla \psi_\ep|^2\right) dxdy\\
%& = \iint_{\Omega}\left( \frac 1 2  |\nabla \psi|^2 + \psi_\ep \omega \right)dxdy,
\end{align*}
where we used $\iint_{\Omega}\omega dxdy=0$, $\omega=\tilde \omega-\omega_{\ep}$ and $\psi$ is the weak solution of $-\Delta \psi = \omega$ in $\tilde{X}_\ep$.


\if0
\begin{lemma}
$$\frac {\delta^2 H(\omega)}{\delta^2 \omega} |_{\omega = 0} = L_\epsilon = \frac 1 {g' (\psi_\epsilon ) } - (-\Delta)^{-1}.$$
\end{lemma}
\begin{proof}
Since
\begin{align*}
H(\omega+ \lambda \delta \omega )
&= \iint_{\Omega}  h(\omega + \omega_\ep + \lambda \delta \omega)dxdy  - E(\omega + \lambda \delta \omega) \\
& = \iint_{\Omega}  h(\omega + \omega_\ep + \lambda \delta \omega) -\frac 1 2 (\delta + \lambda \delta \psi) (\omega + \lambda \delta \omega) - \psi_\ep(\omega + \lambda \delta \omega) dxdy,
\end{align*}
we have
\begin{align*}
 \frac {d^2}{d\lambda ^2} H(\omega + \lambda \delta \omega )|_{\lambda = 0}
& = \iint_{\Omega} h'' (\omega + \omega_\ep) (\delta \omega)^2 - \delta \psi \delta \omega dxdy\\
& = \iint_{\Omega} -\frac {1} {2(\omega +\omega_\epsilon)} (\delta \omega)^2 - \delta \psi \delta \omega dxdy \\
& = \iint_{\Omega}( -\frac {1} {2(\omega +\omega_\epsilon)} - (-\Delta)^{-1}) \delta \omega \cdot \delta \omega dxdy \\
& = \iint_{\Omega} \frac {\delta^2 H(\omega)}{\delta^2 \omega} \delta \omega \cdot \delta \omega dxdy \\
\end{align*}
with $\frac {\delta^2 H(\omega)}{\delta^2 \omega} = -\frac {1} {2(\omega +\omega_\epsilon)} - (-\Delta)^{-1}$ which is continuous at $\omega = 0$ and
$$\frac {\delta^2 H(\omega)}{\delta^2 \omega} |_{\omega = 0} = \frac 1 {g' (\psi_\epsilon ) } - (-\Delta)^{-1} = L_\ep.$$
\end{proof}

 \begin{lemma}\label{JLw0}
 $$span \{\frac {\partial \omega_\epsilon }{\partial x} , \frac {\partial \omega_\epsilon }{\partial y} , \frac {\partial \omega_\epsilon }{\partial\epsilon} \} \subseteq \ker( L_\epsilon).$$
 \end{lemma}
\begin{proof}
Since $$\frac {\delta H(\omega)}{\delta \omega} |_{\omega = 0} = h'(\omega_\ep) - \psi_\ep= 0,$$ we know
\begin{equation}\label{Lw}
\frac {\partial}{\partial x}\frac {\delta H(\omega)}{\delta \omega} |_{\omega = 0} = h''(\omega_\ep)\frac {\partial \omega_\epsilon }{\partial x} -  \frac {\partial \psi_\epsilon }{\partial x}=  ( h''(\omega_\ep) - (-\Delta)^{-1} )\frac {\partial \omega_\epsilon }{\partial x}= L_\epsilon \frac {\partial \omega_\epsilon }{\partial x} = 0.
\end{equation}
Similarly we have $L_\epsilon \frac {\partial \omega_\epsilon }{\partial y} = 0$ and $L_\epsilon \frac {\partial\omega_\epsilon }{\partial\epsilon} = 0$.
\end{proof}
%\newpage

With $$h(\omega_\epsilon) = \frac 1 2 \left[ \omega_\epsilon - \omega_\epsilon \ln(-\omega_\epsilon) \right],$$
\begin{align*}
h'(\omega_\epsilon) = \psi_\epsilon = -\frac 1 2 \ln(-\omega_\epsilon),
\end{align*}
$$H(\omega) = \iint_{\Omega}h(\omega_\ep + \omega )dxdy - E(\omega) = \iint_{\Omega}h(\omega_\ep + \omega ) - \frac 1 2\psi \omega - \psi_\ep \omega dxdy,$$
\fi
Since $h'(\omega_\ep)=\psi_\ep$,  we have
\begin{align*}
H_\ep(\omega) - H_\ep(0)
& =  \iint_{\Omega} f_{\omega_\ep}(\omega) dxdy-\frac 1 2\iint_\Omega  |\nabla\psi|^2 dxdy,
\end{align*}
where
\begin{align*}
f_{\omega_\ep}(\omega) = h(\omega_\ep + \omega ) - h(\omega_\ep)  - \psi_\ep \omega
\end{align*}
for  $\omega \in X_{non,\ep}$.
Define the distance functionals
\begin{align}\nonumber
d_{1}(\tilde\omega,\omega_\ep)&=\iint_{\Omega}f_{\omega_\ep}(\omega)dxdy,\quad d_2(\tilde\omega,\omega_\ep)=\iint_{\Omega}(G*\omega)\omega dxdy=\iint_{\Omega}|\nabla\psi|^2dxdy,\\\label{distance-euler case}
d(\tilde\omega,\omega_\ep)&=d_{1}(\tilde\omega,\omega_\ep)+d_{2}(\tilde\omega,\omega_\ep),
\end{align}
where $\tilde\omega\in Y_{non}$ is the perturbed vorticity. By Lemma \ref{well-poseness-Poisson-equation-nonlinear-case},  $d_2(\tilde\omega,\omega_\ep)$ is well-defined for $\tilde\omega\in Y_{non}$.
By Lemma \ref{tilde-omega0-kappa-properties} (7), we have $\psi_\ep\tilde \omega\in{L^1(\Omega)}$ for $\tilde\omega\in Y_{non}$, and thus,
 by Taylor's formula we have
\begin{align}\nonumber
0\leq&\int_0^1\iint_\Omega{(1-r)\big(\tilde \omega-\omega_{\ep}\big)^2\over 2|\omega^r|}dxdydr=d_1(\tilde \omega,\omega_{\ep})\\\nonumber
=&\iint_{\Omega}\left({1\over2}(\tilde \omega-\tilde \omega\ln(-\tilde \omega))-{1\over 2}\omega_\ep-\psi_\ep\tilde \omega \right) dxdy\\
\leq&\|\tilde \omega\|_{L^1(\Omega)}+\|\tilde \omega\|_{L^2(\Omega)}^2+\|\omega_\ep\|_{L^1(\Omega)}+\|\psi_\ep\tilde \omega\|_{L^1(\Omega)}<\infty,\label{d1-well-def}
\end{align}
where $\omega^r=r\tilde \omega+(1-r)\omega_{\ep}
%\leq|\tilde \omega(t_0;x,y)+\omega_{\ep_1(t_0)}(x+x_1(t_0),y+y_1(t_0))
$ for $r\in[0,1]$. Here, we used $s\ln s\leq s^2$ for $s>0$. Thus, $d_1(\tilde\omega,\omega_\ep)$ is well-defined for $\tilde\omega\in Y_{non}$.

\subsection{The dual functional and its regularity}
We   try to study the Taylor expansion of $H_\ep$ near $\omega=0$ directly, and use the positiveness of  $L_\ep$ in a finite co-dimensional subspace of $X_\ep$. However,
%the regularity of $H_\ep$ is only $C^0$, and even though in a sufficient regularity,
 $\|\omega\|_{L^3}$ can not be controlled by $\|\omega\|_{L_{1\over g'(\psi_\ep)}^2}$ in general. Our approach is  to transform $H_\ep$ to its dual functional and then study the Taylor expansion of the dual functional.
We observe that
\begin{align}\nonumber&H_\ep(\omega) - H_\ep(0)
 = d_{1}(\tilde\omega,\omega_\ep)-{1\over2}d_2(\tilde\omega,\omega_\ep)\\\nonumber
 =&{1\over2}\iint_{\Omega}|\nabla\psi|^2dxdy-\iint_{\Omega}(\psi\omega-f_{\omega_\ep}(\omega))dxdy\\\label{H-omega-H-0}
 \geq &  \iint_{\Omega} \left(\frac 1 2 |\nabla \psi|^2 - f_{\omega_\ep}^*(\psi)\right) dxdy
 \end{align}
 for $\omega \in X_{non,\ep}$, where $f_{\omega_\ep}^*$ is the Legendre transformation of $f_{\omega_\ep}$.
This gives  a lower bound of $d_{1}(\tilde\omega,\omega_\ep)-{1\over2}d_2(\tilde\omega,\omega_\ep)$. Then  we compute the pointwise expression of  $f_{\omega_\ep}^*$.
\begin{lemma}\label{Legendre transformation}
Let $\ep\in[0,1)$, $(x,y)\in\Omega$ and  $f_{\omega_\ep(x,y)}(z)= h(\omega_\ep(x,y) + z ) - h(\omega_\ep(x,y))  -  h'(\omega_\ep(x,y))z $ for  $z\in(-\infty,-\omega_\ep(x,y))$.
Then
 the Legendre transformation of $f_{\omega_\ep(x,y)}$ is
\begin{align*}
f_{\omega_\ep(x,y)}^*(s)= - \frac 1 2 \omega_\ep(x,y) (e^{-2s} + 2s - 1),\quad s\in\mathbb{R}.
\end{align*}
\end{lemma}
\begin{proof}
By its definition of  the Legendre transformation, $f_{\omega_\ep(x,y)}^*(s)=\sup\limits_{z<-\omega_\ep(x,y)}(sz-f_{\omega_\ep(x,y)}(z)),$ $ s\in\mathbb{R}$.
Let $F_{\omega_\ep(x,y),s}(z)=sz-f_{\omega_\ep(x,y)}(z)$ for $z<-\omega_\ep(x,y)$. Then
\begin{align*}
F_{\omega_\ep(x,y),s}'(z)= s - h'(\omega_\ep(x,y) +z) +h'(\omega_\ep(x,y)) =s+{1\over2}\ln|\omega_\ep(x,y)+z|+\psi_\ep(x,y).
\end{align*}
Thus, there exists a unique $z_{\omega_\ep(x,y)}(s)\triangleq\omega_\ep(x,y)(e^{-2s}-1)\in(-\infty,-\omega_\ep(x,y))$ such that $F_{\omega_\ep(x,y),s}'(z_{\omega_\ep(x,y)}(s))=0$ and $F_{\omega_\ep(x,y),s}''(z)={1\over 2(\omega_\ep(x,y)+z)}<0$ for $z\in(-\infty,-\omega_\ep(x,y))$, which implies
\begin{align*}
&f_{\omega_\ep(x,y)}^*(s)=F_{\omega_\ep(x,y),s}(z_{\omega_\ep(x,y)}(s))\\
 =& (s+\psi_\ep(x,y)) \omega_\ep(x,y)(e^{-2s}-1) - h(\omega_\ep(x,y)e^{-2s} ) + h(\omega_\ep)  \\
 =& - \frac 1 2 \omega_\ep(x,y) (e^{-2s} + 2s - 1),\quad s\in\mathbb{R}.
\end{align*}
\end{proof}
By \eqref{H-omega-H-0} and Lemma \ref{Legendre transformation}, we have
\begin{align*}
 d_{1}(\tilde\omega,\omega_\ep)-{1\over2}d_2(\tilde\omega,\omega_\ep)
 \geq &  \iint_{\Omega} \left(\frac 1 2 |\nabla \psi|^2 +\frac 1 2 \omega_\ep (e^{-2\psi} + 2\psi - 1)\right) dxdy.
 \end{align*}
To apply the Taylor formula of the  functional
\begin{align}\nonumber
\mathscr{B}_\ep( \psi)\triangleq &  \iint_{\Omega} \left(\frac 1 2 |\nabla \psi|^2 +\frac 1 2 \omega_\ep (e^{-2\psi} + 2\psi - 1)\right) dxdy\\\label{def-functional-B}
=  & \iint_{\Omega} \left(\frac 1 2 |\nabla \psi|^2 -\frac 1 4 g'(\psi_\ep)(e^{-2\psi} + 2\psi - 1)\right) dxdy,\quad \psi\in \tilde X_\ep,
\end{align}
we first study its regularity. To this end, we need the following inequalities.
\if0
For  $\omega \in  X_{non,\ep}$,
$$d_1(\omega,0)-{1\over2}d_2(\omega,0) \geq  \iint_{\Omega}\left(\frac 1 2 |\nabla\psi|^2- f^*(\psi) \right)dxdy.$$

Let $d_1(\omega) = \iint_{\Omega}f(\omega) dxdy$ and $d_2(\omega) = \iint_{\Omega} \psi \omega dxdy$, then for any $\tau \in [0, 1]$, we have
\begin{align*}
H(\omega) - H(0)
&= \frac 1 2  \iint_{\Omega} |\nabla \psi|^2 dxdy - \iint_{\Omega} \psi \omega - f(\omega) dxdy \\
&= \tau \left(\frac 1 2  \iint_{\Omega} |\nabla \psi|^2 dxdy - \iint_{\Omega} \psi \omega - f(\omega) dxdy \right) \\
&\qquad \qquad \qquad +  (1- \tau) \left(\frac 1 2  \iint_{\Omega} |\nabla \psi|^2 dxdy - \iint_{\Omega} \psi \omega - f(\omega) dxdy \right)\\
&\geq \tau \left(\frac 1 2  \iint_{\Omega} |\nabla \psi|^2 dxdy - \iint_{\Omega} \psi \omega - f(\omega) dxdy \right) \\
&\qquad \qquad \qquad  + (1-\tau)\left(\frac 1 2  \iint_{\Omega} |\nabla \psi|^2 dxdy - \iint_{\Omega} f^*(\psi) dxdy \right) \\
& = \tau \left(\frac 1 2  \iint_{\Omega} |\nabla \psi|^2 dxdy - \iint_{\Omega} \psi \omega - f(\omega) dxdy \right) + (1-\tau)B(\psi)\\
& = \tau\left(d_1(\omega) - \frac 1 2 d_2(\omega)\right) + (1-\tau)B(\psi)
\end{align*}
with
\begin{align}\label{funcB}
\begin{split}
B(\psi) &= \frac 1 2  \iint_{\Omega} |\nabla \psi|^2 dxdy - \iint_{\Omega} f^*(\psi) dxdy \\
& = \frac 1 2 \iint_\Omega \psi\omega + \omega_\ep ( e^{-2\psi} + 2\psi - 1) dxdy \\
& =  \frac 1 2 \iint_\Omega \psi\omega - \frac 1 2 g'(\psi_\ep) ( e^{-2\psi} + 2\psi - 1) dxdy.
\end{split}
\end{align}
for $\psi \in \tilde{X}_\ep$.
\fi
\begin{lemma}\label{Orlicz-type inequlity-lemma}
For $\ep\in[0,1)$ and $a\in\mathbb{R}$, we have
\begin{align}\label{Orlicz-type inequlity}
\iint_\Omega g'(\psi_\ep) e^ {a\psi} dxdy \leq\iint_\Omega g'(\psi_\ep) e^ {|a\psi|} dxdy \leq   C e^{Ca^2\|\psi\|_{\tilde{X}_\ep}^2},\quad\psi \in \tilde{X}_\ep.
\end{align}
In particular, for $p\in\mathbb{Z}^+$,
\begin{align*}\iint_\Omega g'(\psi_\ep) |\psi|^p dxdy \leq p!\iint_\Omega g'(\psi_\ep) e^{|\psi|} dxdy \leq  Cp! e^{C \|\psi\|_{\tilde{X}_\ep}^2},\quad\psi \in \tilde{X}_\ep.\end{align*}
\end{lemma}
\begin{proof} We first prove \eqref{Orlicz-type inequlity} for $\ep=0$.
Applying the similar decomposition \eqref{psi-m-dec} to $\psi \in \tilde{X}_\ep$, we have $\psi = \widehat\psi_0 +\psi_{\neq0}$, where
 $\psi_{\neq0} \in H^1(\Omega)$. Since $|a\widehat\psi_0(y)| \leq|a|\|\widehat\psi_0'\|_{L^2(\mathbb{R})}|y|^{\frac 1 2}\leq |a|\|\psi\|_{\tilde{X}_0}|y|^{\frac 1 2}\leq{a^2\over4}\|\psi\|_{\tilde{X}_0}^2+|y|$, we have
\begin{align}\label{0-mode-or-estimate}
\sqrt{g'(\psi_0)} e^{|a\widehat\psi_0(y)|} \leq \sqrt{g'(\psi_0)} e^{{a^2\over4}\|\psi\|_{\tilde{X}_0}^2 } e^{|y|} \leq C e^{{a^2\over4}\|\psi\|_{\tilde{X}_0}^2 }. \end{align}
Without loss of generality, assume that $\|\psi_{\neq0}\|_{\tilde{X}_0} \neq 0$. It follows from Subsection 8.26 in \cite{Adams75} that $H^1(\Omega)$ is embedded in the Orlicz space $L_{A_0}(\Omega)$ with $A_0(t) = e^{t^2} - 1$.
\if0
and the norm
\begin{align*}
\|\psi_{\neq0}\|_{L_{A_0}} &= \inf \left \{ k > 0 | \iint_{\Omega} A_0(\frac{|\psi_{\neq0}(x,y)|}{k}) dx dy \leq 1 \right \} \\
&= \inf \left \{ k > 0 | \iint_{\Omega} \left( e^{\left (\frac{|\psi_{\neq0}(x,y)|}{k}\right)^2} - 1\right)dx dy \leq 1 \right \}
\end{align*}
\fi
Since $\psi_{\neq0} \in H^1(\Omega)$, we have
   $\psi_{\neq0} \in L_{A_0}(\Omega)$ and
$\|\psi_{\neq0}\|_{L_{A_0}(\Omega)} \leq C \|\psi_{\neq0}\|_{H^1(\Omega)} \leq C \|\psi\|_{\tilde{X}_0}.$
Let $k_0 = \|\psi_{\neq0}\|_{L_{A_0}(\Omega)} + \|\psi_{\neq0}\|_{\tilde{X}_0}$. Then $k_0 \leq C\|\psi\|_{\tilde{X}_0}$. By the definition of the norm $\|\cdot\|_{L_{A_0}(\Omega)}$ (see (13) in Chapter VIII), we have
\begin{align*}
\|\psi_{\neq0}\|_{L_{A_0}(\Omega)}
&= \inf \left \{ k > 0 \bigg| \iint_{\Omega} \left( e^{\left (\frac{|\psi_{\neq0}|}{k}\right)^2} - 1\right)dx dy \leq 1 \right \},
\end{align*}
and thus, there exists $k_1\in[\|\psi_{\neq0}\|_{L_{A_0}(\Omega)}, k_0)$ such that
\begin{align}\label{Orlicz2}
\iint_{\Omega} \left( e^{\left(\frac{|\psi_{\neq0}|}{k_0}\right)^2} - 1 \right)dx dy \leq \iint_{\Omega} \left( e^{\left(\frac{|\psi_{\neq0}|}{k_1}\right)^2} - 1 \right)dx dy \leq  1.\end{align}
By \eqref{0-mode-or-estimate}, \eqref{Orlicz2} and the fact that $k_0 \leq C\|\psi\|_{\tilde{X}_0}$, we have
\begin{align*}
& \iint_\Omega g'(\psi_0)e^{|a\psi|}  dxdy
 \leq   \iint_\Omega \sqrt{g'(\psi_0)}e^{|a\widehat\psi_0|} \sqrt{g'(\psi_0)} e^{|a\psi_{\neq0}|} dxdy \\
 \leq & Ce^{\frac{a^2}{4} \|\psi\|^2_{\tilde{X}_0}} \iint_\Omega \sqrt{g'(\psi_0)}e^{\left| \frac{\psi_{\neq0}}{k_0}\right|^2} e^{\frac{a^2}{4}k_0^2}  dxdy \\
 = & Ce^{\frac{a^2}{4} \left(\|\psi\|^2_{\tilde{X}_0}+k_0^2\right)} \iint_\Omega \sqrt{g'(\psi_0)} \left( e^{\left| \frac{\psi_{\neq0}}{k_0}\right|^2} - 1\right)  dxdy +  Ce^{\frac{a^2}{4} \left(\|\psi\|^2_{\tilde{X}_0}+k_0^2\right)} \iint_\Omega \sqrt{g'(\psi_0)}  dxdy \\
\leq & Ce^{Ca^2 \|\psi\|^2_{\tilde{X}_0}}  \iint_\Omega \left( e^{\left| \frac{\psi_{\neq0}}{k_0}\right|^2} - 1\right)  dxdy +  Ce^{Ca^2 \|\psi\|^2_{\tilde{X}_0}} \\
\leq & Ce^{Ca^2 \|\psi\|^2_{\tilde{X}_0}}.
\end{align*}

Now, we consider the case $\epsilon\in(0,1)$. By
\eqref{Orlicz-type inequlity} for $\epsilon=0$, we have $\iint_{\tilde\Omega} e^{a\Psi}dxd\gamma_0\leq C e^{Ca^2\|\Psi\|_{\tilde Y_0}^2}$ for $\Psi\in\tilde Y_0$ in the new variables $(x,\gamma_0=\tanh(y))$. Then $\iint_{\tilde \Omega} e^{a\Psi}d\theta_\ep d\gamma_\ep\leq C e^{Ca^2\|\Psi\|_{\tilde Y_\ep}^2}$ for $\Psi\in\tilde Y_\ep$ in the new variables $(\theta_\ep,\gamma_\ep)$ for $\epsilon\in(0,1)$. Thus, \eqref{Orlicz-type inequlity} holds true  for $\epsilon\in(0,1)$.
\end{proof}

With the help of Lemma \ref{Orlicz-type inequlity-lemma}, we prove the $C^2$ regularity of $\mathscr{B}_\ep$ we need.



\begin{lemma}\label{B-C2}
$\mathscr{B}_\ep\in C^2(\tilde X_\ep)$, and for $ \psi\in \tilde X_\ep$,
\begin{align*}
\mathscr{B}_\ep'(\psi) &= -\Delta\psi +\frac{1}{2}g'(\psi_\ep)(e^{-2\psi}-1),\\
\langle \mathscr{B}_\ep''(\psi)\phi,\varphi \rangle&=  \iint_{\Omega}\left(\nabla\phi\cdot\nabla\varphi- g'(\psi_\ep) e^{-2\psi}\phi\varphi\right)dxdy,\quad \phi,\varphi\in\tilde X_\ep,
\end{align*}
 where   $\mathscr{B}_\ep$ is defined in \eqref{def-functional-B} and $\ep\in[0,1)$.
\end{lemma}
\begin{proof}
Let $\psi\in \tilde{X}_\ep$. For  $\phi \in \tilde{X}_\ep$, by Lemmas \ref{poincare1}, \ref{poincare1ep} and \ref{Orlicz-type inequlity-lemma} we have
\begin{align*}
|\partial_\lambda \mathscr{B}_\ep(\psi + \lambda \phi)|_{\lambda = 0}|
= & \iint_\Omega  \left(-\Delta \psi+{1\over2}  g'(\psi_\ep) (e^{-2\psi}-1)\right)\phi dxdy\\
\leq &\|\psi\|_{\tilde{X}_\ep}\|\phi\|_{\tilde{X}_\ep}+ C \left(\iint_\Omega    g'(\psi_\ep) (e^{-4\psi}-2e^{-2\psi}+1)dxdy\right)^{1\over2}\|\phi \|_{\tilde{X}_\ep}\\
\leq &\left(\|\psi\|_{\tilde{X}_\ep}+ C \left(Ce^{C\|\psi\|_{\tilde{X}_\ep}^2}+C\right)^{1\over2}\right)\|\phi \|_{\tilde{X}_\ep}.
\end{align*}
Thus,  $\mathscr{B}_\ep$  is G$\hat{\text{a}}$teaux differentiable at $\psi\in  \tilde{X}_\ep$.
To show that $\mathscr{B}_\ep\in C^1(\tilde X_\ep)$, we choose $\{\psi_n\}_{n=1}^\infty\in \tilde X_\ep$ such that $\psi_n\to\psi$ in $\tilde{X}_\ep$, and prove that for fixed $\phi\in\tilde X_\ep$,
\begin{align*}
\partial_\lambda \mathscr{B}_\ep(\psi_n + \lambda \phi)|_{\lambda = 0}\to\partial_\lambda \mathscr{B}_\ep(\psi + \lambda \phi)|_{\lambda = 0}
\end{align*}
as $n\to\infty$.
In fact, there exists $N>0$ such that $\|\psi_n\|_{\tilde X_\ep}\leq \|\psi\|_{\tilde X_\ep}+1$ for $n\geq N$, and  by Lemmas \ref{poincare1}, \ref{poincare1ep} and \ref{Orlicz-type inequlity-lemma} we have for $n\geq N$,
\begin{align*}
&|\partial_\lambda \mathscr{B}_\ep(\psi_n + \lambda \phi)|_{\lambda = 0}-\partial_\lambda \mathscr{B}_\ep(\psi + \lambda \phi)|_{\lambda = 0}| \\
= & \left|\iint_{\Omega}\left(\nabla(\psi_n-\psi)\cdot\nabla\phi +{1\over 2} g'(\psi_\ep)(e^{-2\psi_n} - e^{-2\psi})\phi\right) dxdy\right|\\
\leq&\|\psi_n-\psi\|_{\tilde X_\ep}\|\phi\|_{\tilde X_\ep}+\left|\int_0^1  \iint_{\Omega} g'(\psi_\ep) e^{-2(s\psi_n + (1-s)\psi)}  (\psi_n - \psi)  \phi dxdyds\right| \\
\leq&\|\psi_n-\psi\|_{\tilde X_\ep}\|\phi\|_{\tilde X_\ep}+\|\psi_n-\psi\|_{\tilde X_\ep}\|\phi\|_{L^4_{g'(\psi_\ep)}}\int_0^1 \left( \iint_{\Omega} g'(\psi_\ep) e^{-8(s\psi_n + (1-s)\psi)}  dxdy\right)^{1\over4}ds\\
\leq &  \|\psi_n-\psi\|_{\tilde X_\ep}\|\phi\|_{\tilde X_\ep}+
\|\psi_n-\psi\|_{\tilde X_\ep}
\left(Ce^{C\|\phi\|_{\tilde X_\ep}^2}\right)^{1\over4}
\int_0^1\left(Ce^{C\|s\psi_n + (1-s)\psi\|_{\tilde X_\ep}^2}\right)^{1\over4}ds\\
\leq &\left(\|\phi\|_{\tilde X_\ep}+C_{\|\phi\|_{\tilde X_\ep}}C_{\|\psi\|_{\tilde X_\ep}}\right)\|\psi_n-\psi\|_{\tilde X_\ep}\to 0\quad \text{as}\quad n\to\infty.
\end{align*}
This proves that $\mathscr{B}_\ep\in C^1(\tilde X_\ep)$.
Then we show that the 2-th order G$\hat{\text{a}}$teaux derivative of $\mathscr{B}_\ep$ exists at $\psi\in  \tilde{X}_\ep$. For
$\phi \in  \tilde{X}_\ep$ and $\varphi\in\tilde{X}_\ep$, by Lemma  \ref{Orlicz-type inequlity-lemma} we have
\begin{align*}
&\left|\partial_\tau\partial_\lambda \mathscr{B}_\ep(\psi + \lambda\phi+\tau\varphi)|_{\lambda =\tau= 0}\right|
= \left|\iint_{\Omega}\left(\nabla\phi\cdot\nabla\varphi- g'(\psi_\ep) e^{-2\psi}\phi\varphi\right)dxdy\right|\\
\leq&\|\phi\|_{\tilde{X}_\ep}\|\varphi\|_{\tilde{X}_\ep}+\left(\iint_{\Omega} g'(\psi_\ep) e^{-4\psi} dxdy\right)^{1\over2}
\|\phi\|_{L_{g'(\psi_\ep)}^4}\|\varphi\|_{L_{g'(\psi_\ep)}^4}\\
\leq&\|\phi\|_{\tilde{X}_\ep}\|\varphi\|_{\tilde{X}_\ep}+Ce^{C\left(\|\psi\|_{\tilde{X}_\ep}^2+\|\phi\|_{\tilde{X}_\ep}^2+\|\varphi\|_{\tilde{X}_\ep}^2\right)},
\end{align*}
which implies that   $\mathscr{B}_\ep$  is 2-order G$\hat{\text{a}}$teaux differentiable at $\psi\in  \tilde{X}_\ep$.
To show that $\mathscr{B}_\ep\in C^2(\tilde X_\ep)$, we use  $\{\psi_n\}_{n=1}^\infty\in \tilde X_\ep$  as above, and
  for  $\phi,\varphi\in\tilde X_\ep$ and $n\geq N$,
\begin{align*}
&|\partial_\tau\partial_\lambda \mathscr{B}_\ep(\psi_n + \lambda \phi+\tau\varphi)|_{\lambda =\tau= 0}-\partial_\tau\partial_\lambda \mathscr{B}_\ep(\psi + \lambda \phi+\tau\varphi)|_{\lambda =\tau= 0}|\\
=&\left|2\int_0^1\iint_{\Omega}g'(\psi_\ep) e^{-2(s\psi_n + (1-s)\psi)}(\psi_n-\psi)\phi\varphi dxdyds\right|\\
\leq&C\|\psi_n-\psi\|_{\tilde X_\ep}\|\phi\|_{L_{g'(\psi_\ep)}^6}\|\varphi\|_{L_{g'(\psi_\ep)}^6}\int_0^1\left(\iint_{\Omega}g'(\psi_\ep) e^{-12(s\psi_n + (1-s)\psi)}dxdy\right)^{1\over6}ds\\
\leq&C\|\psi_n-\psi\|_{\tilde X_\ep}
\left(Ce^{C\|\phi\|_{\tilde X_\ep}^2}\right)^{1\over6}\left(Ce^{C\|\varphi\|_{\tilde X_\ep}^2}\right)^{1\over6}
\int_0^1\left(Ce^{C\|s\psi_n + (1-s)\psi\|_{\tilde X_\ep}^2}\right)^{1\over6}ds\\
\leq &C_{\|\phi\|_{\tilde X_\ep}}C_{\|\varphi\|_{\tilde X_\ep}}C_{\|\psi\|_{\tilde X_\ep}}\|\psi_n-\psi\|_{\tilde X_\ep}\to 0\quad \text{as}\quad n\to\infty.
\end{align*}
This proves that $\mathscr{B}_\ep\in C^2(\tilde X_\ep)$.
\end{proof}

\begin{remark}
In view of Lemma \ref{Orlicz-type inequlity-lemma}, one can use a similar argument in the proof of Lemma \ref{B-C2} to show that  $\mathscr{B}_\ep\in C^\infty(\tilde X_\ep)$.
\end{remark}

By Lemma \ref{B-C2}, we have $\mathscr{B}_\ep'(0) = 0$, and
\begin{align*}
\langle \mathscr{B}_\ep''(0)\psi_1,\psi_2 \rangle&=  \iint_{\Omega}\left(\nabla\psi_1\cdot\nabla\psi_2- g'(\psi_\ep) \psi_1\psi_2\right)dxdy,\quad \psi_1,\psi_2\in\tilde X_\ep.
\end{align*}
Recall that
$A_\ep =-\Delta -g'(\psi_\ep):\tilde{X}_\ep \rightarrow \tilde{X}_\ep^*$ for $\ep\in[0,1)$. Then
\begin{align}\label{B-ep-A-ep}
\langle \mathscr{B}_\ep''(0)\psi_1,\psi_2 \rangle=\langle A_\ep \psi_1,\psi_2 \rangle,\quad \psi_1,\psi_2\in\tilde X_\ep.
\end{align}
By Corollaries \ref{kernel of  the operator A0 and a decomposition of tilde X0} and \ref{kernel of  the operator A-ep and a decomposition of tilde Xep}, we have
\begin{align*}
\ker ( A_\ep)={\rm{span}}\left\{\eta_\ep(x,y), \gamma_\ep(x,y), \xi_\ep(x,y)\right\}
\end{align*}
and
\begin{align}\label{A-ep-positive-lower-bound}
\langle  A_\ep \psi,\psi\rangle \geq C_0 \| \psi\|_{\tilde X_\ep}^2, \quad \quad \psi\in \tilde X_{\ep+}=\tilde X_\ep \ominus\ker ( A_\ep)
\end{align}
for some $C_0>0$ independent of $\ep\in[0,1)$.

\subsection{Removal of the kernel due to translations and change of parameters}
Let us first consider the 3 dimensional orbit
$$\Gamma = \{\omega_{\ep_1}(x+x_1, y + y_1)| \ep_1 \in (0, 1), x_1 \in \mathbb{T}_{2\pi}, y_1 \in \mathbb{R} \}.$$
To prove the nonlinear orbital stability of the steady states,  we need to carefully study the translations of the steady states  in the $x, y, \ep$ directions such that the perturbation of the stream function is perpendicular to the three kernel functions of $ A_\ep$.
\begin{lemma}\label{imp-vertical condition} Let $\ep_0\in(0,1)$. Then  there exists $\delta=\delta(\ep_0)>0$ such that for any  $(x_0,y_0)\in\Omega$ and  $\tilde \omega\in Y_{non}$ with $d_2(\tilde \omega,\omega_{\ep_0}(x+x_0,y+y_0))=\|\tilde\psi-\psi_{\ep_0}(x+x_0,y+y_0)\|_{\dot{H}^1(\Omega)}^2\leq \delta$, there exist $(\tilde x_0,\tilde y_0)\in\Omega$ and $\tilde \epsilon_0\in(a(\ep_0),b(\ep_0))$, depending continuously  on $(x_0,y_0)\in\Omega$ and  $\tilde \omega$, such that
\begin{align*}
\iint_{\Omega}\nabla\left(\tilde \psi(x,y)-\psi_{\tilde \ep_0}(x+\tilde x_0,y+\tilde y_0)\right)\cdot\nabla\eta_{\tilde \ep_0}\left(x+\tilde x_0,y+\tilde y_0\right)dxdy=0,\\
\iint_{\Omega}\nabla\left(\tilde \psi(x,y)-\psi_{\tilde \ep_0}(x+\tilde x_0,y+\tilde y_0)\right)\cdot\nabla\gamma_{\tilde \ep_0}\left(x+\tilde x_0,y+\tilde y_0\right)dxdy=0,\\
\iint_{\Omega}\nabla\left(\tilde \psi(x,y)-\psi_{\tilde \ep_0}(x+\tilde x_0,y+\tilde y_0)\right)\cdot\nabla\xi_{\tilde \ep_0}\left(x+\tilde x_0,y+\tilde y_0\right)dxdy=0,
\end{align*}
and
 \begin{align*}
|x_0-\tilde x_0|+|y_0-\tilde y_0|+|\ep_0-\tilde \ep_0|\leq C(\ep_0)\sqrt{\delta}
\end{align*}
for some $a(\ep_0)\in (0,\ep_0)$ and $b(\ep_0)\in(\ep_0,1)$, where $\tilde\psi=G*\tilde\omega$.
\end{lemma}

\begin{proof} For $\tilde \omega\in Y_{non}$, since $\tilde \psi-\psi_{\ep_0}=G*(\tilde \omega-\omega_{\ep_0})-c$ for  some constant $c$, by Lemma \ref{G-ast-omega-psi-constant} we have $\tilde \psi-\psi_{\ep_0}\in {\dot{H}^1(\Omega)}$.
 For $x_0=y_0=0$, we
define
\if0
the space $V=\{\tilde \omega|\tilde \psi-\psi_{\ep_0}\in\dot{H}^1(\Omega)\}$ and
 \fi
 the map $S=(S_1,S_2,S_3)$ from $Y_{non}\times \mathbb{T}_{2\pi}\times\mathbb{R}\times (0,1)$ to $\mathbb{R}^3$ by
\begin{align*}
S_1(\tilde\omega,x_1,y_1,\ep_1)
=&\iint_{\Omega}\nabla\left(\tilde \psi(x,y)-\psi_{\ep_1}(x+x_1,y+y_1)\right)\cdot\nabla\eta_{\ep_1}\left(x+x_1,y+y_1\right)dxdy,\\
S_2(\tilde\omega,x_1,y_1,\ep_1)
=&\iint_{\Omega}\nabla\left(\tilde \psi(x,y)-\psi_{\ep_1}(x+x_1,y+y_1)\right)\cdot\nabla\gamma_{\ep_1}\left(x+x_1,y+y_1\right)dxdy,\\
S_3(\tilde\omega,x_1,y_1,\ep_1)
=&\iint_{\Omega}\nabla\left(\tilde \psi(x,y)-\psi_{\ep_1}(x+x_1,y+y_1)\right)\cdot\nabla\xi_{\ep_1}\left(x+x_1,y+y_1\right)dxdy.
\end{align*}
\if0
where we used
\begin{align*}
0=&\partial_1\iint_{\Omega}\omega_\ep dxdy=\partial_1\iint_{\Omega}g(\psi_\ep) dxdy=\iint_{\Omega}g'(\psi_\ep)\partial_1\psi_\ep dxdy=-2\iint_{\Omega}g(\psi_\ep)\partial_1\psi_\ep dxdy\\
=&-2\iint_{\Omega}\omega_\ep\partial_1\psi_\ep dxdy=-2\iint_{\Omega}\nabla\psi_\ep\cdot\nabla\partial_1\psi_\ep dxdy
\end{align*}
since $\iint_{\Omega}\omega_\ep dxdy=-4\pi$, and $\partial_1=\partial_x, \partial_y$ or $\partial_\ep$.
\fi
Note that $S(\omega_{\ep_0},0,0,\ep_0)=(0,0,0)$ and
\begin{align*}
&{\partial(S_1,S_2,S_3)\over \partial(x_1,y_1,\ep_1)}\bigg|_{\tilde \omega=\omega_{\ep_0},x_1=0,y_1=0,\ep_1=\ep_0}\\
=&\left| \begin{array}{cccc} -\iint_\Omega\nabla\partial_x\psi_{\ep}\cdot\nabla\eta_{\ep}dxdy & -\iint_\Omega\nabla\partial_y\psi_{\ep}\cdot\nabla\eta_{\ep}dxdy
&-\iint_\Omega\nabla\partial_\ep\psi_{\ep}\cdot\nabla\eta_{\ep}dxdy\\ -\iint_\Omega\nabla\partial_x\psi_{\ep}\cdot\nabla\gamma_{\ep}dxdy& -\iint_\Omega\nabla\partial_y\psi_{\ep}\cdot \nabla\gamma_{\ep}dxdy &
 -\iint_\Omega\nabla\partial_\ep\psi_{\ep}\cdot\nabla\gamma_{\ep}dxdy\\
 -\iint_\Omega\nabla\partial_x\psi_{\ep}\cdot\nabla\xi_{\ep}dxdy& -\iint_\Omega\nabla\partial_y\psi_{\ep}\cdot\nabla\xi_{\ep}dxdy  &-\iint_\Omega\nabla\partial_\ep\psi_{\ep}\cdot \nabla\xi_{\ep}dxdy
\end{array} \right|_{\ep=\ep_0}.
\end{align*}
By \eqref{three-kers1}-\eqref{three-kers3}, \eqref{def-eta-ep}-\eqref{def-xi-ep} and Proposition \ref{prop1}, we have
\begin{align*}
\iint_\Omega\nabla\partial_x\psi_{\ep}\cdot\nabla\eta_{\ep}dxdy&={-\ep\over \sqrt{1-\ep^2}}\iint_\Omega|\nabla\eta_\ep|^2dxdy=
{-\ep\over \sqrt{1-\ep^2}}\int_{-1}^1\int_{0}^{2\pi}(1-\eta_\ep^2)d\theta_\ep d\gamma_\ep\\
&={-\ep\over \sqrt{1-\ep^2}}\int_{-1}^1\int_{0}^{2\pi}\left(\gamma_\ep^2\sin^2(\theta_\ep)+\cos^2(\theta_\ep)\right)d\theta_\ep d\gamma_\ep={-\ep\over \sqrt{1-\ep^2}}{8\over3}\pi,\\
\iint_\Omega\nabla\partial_y\psi_{\ep}\cdot\nabla\eta_{\ep}dxdy&={1\over \sqrt{1-\ep^2}}\iint_\Omega\nabla\gamma_\ep\cdot \nabla\eta_\ep dxdy=
{-1\over \sqrt{1-\ep^2}}\int_{-1}^1\int_{0}^{2\pi}\gamma_\ep\eta_\ep d\theta_\ep d\gamma_\ep\\
&={-1\over \sqrt{1-\ep^2}}\int_{-1}^1\int_{0}^{2\pi}\gamma_\ep(1-\gamma_\ep^2)^{1\over2}\sin(\theta_\ep) d\theta_\ep d\gamma_\ep=0,\\
\iint_\Omega\nabla\partial_\ep\psi_{\ep}\cdot\nabla\eta_{\ep}dxdy&={1\over 1-\ep^2}\iint_\Omega\nabla\xi_\ep\cdot \nabla\eta_\ep dxdy={-1\over 1-\ep^2}\int_{-1}^1\int_{0}^{2\pi}\xi_\ep\eta_\ep d\theta_\ep d\gamma_\ep\\
&={-1\over 1-\ep^2}\int_{-1}^1\int_{0}^{2\pi}(1-\gamma_\ep^2)\sin(\theta_\ep)\cos(\theta_\ep) d\theta_\ep d\gamma_\ep=0,\\
\iint_\Omega\nabla\partial_y\psi_{\ep}\cdot \nabla\gamma_{\ep}dxdy&={1\over \sqrt{1-\ep^2}}\iint_\Omega|\nabla\gamma_\ep|^2 dxdy={1\over \sqrt{1-\ep^2}}\int_{-1}^1\int_{0}^{2\pi}(1-\gamma_\ep^2)d\theta_\ep d\gamma_\ep\\
&={1\over \sqrt{1-\ep^2}}{8\over3}\pi,\\
\iint_\Omega\nabla\partial_\ep\psi_{\ep}\cdot\nabla\gamma_{\ep}dxdy&={1\over 1-\ep^2}\iint_\Omega\nabla\xi_\ep \cdot\nabla\gamma_{\ep} dxdy={-1\over 1-\ep^2}\int_{-1}^1\int_{0}^{2\pi}\xi_\ep \gamma_{\ep} d\theta_\ep d\gamma_\ep\\
&={-1\over 1-\ep^2}\int_{-1}^1\int_{0}^{2\pi}(1-\gamma_\ep^2)^{1\over2}\cos(\theta_\ep) \gamma_{\ep} d\theta_\ep d\gamma_\ep=0,\\
\iint_\Omega\nabla\partial_\ep\psi_{\ep}\cdot \nabla\xi_{\ep}dxdy&={1\over 1-\ep^2}\iint_\Omega\nabla\xi_\ep \cdot\nabla\xi_{\ep} dxdy={1\over 1-\ep^2}\int_{-1}^1\int_{0}^{2\pi}(1-\xi_\ep^2) d\theta_\ep d\gamma_\ep\\
&={1\over 1-\ep^2}\int_{-1}^1\int_{0}^{2\pi}\left(\gamma_\ep^2\cos^2(\theta_\ep)+\sin^2(\theta_\ep)\right) d\theta_\ep d\gamma_\ep={1\over 1-\ep^2}{8\over3}\pi.
\end{align*}
Then
\begin{align*}
\iint_\Omega\nabla\partial_x\psi_{\ep}\cdot\nabla\gamma_{\ep}dxdy=
 \iint_\Omega\nabla\partial_x\psi_{\ep}\cdot\nabla\xi_{\ep}dxdy=\iint_\Omega\nabla\partial_y\psi_{\ep}\cdot\nabla\xi_{\ep}dxdy =0.
 \end{align*}
Thus,
\begin{align*}
{\partial(S_1,S_2,S_3)\over \partial(x_1,y_1,\ep_1)}\bigg|_{\tilde \omega=\omega_{\ep_0},x_1=0,y_1=0,\ep_1=\ep_0}
=&\left| \begin{array}{cccc} {\ep_0\over \sqrt{1-\ep_0^2}}{8\over3}\pi &0
&0\\ 0& {-1\over \sqrt{1-\ep_0^2}}{8\over3}\pi &
0\\
0&0  &{-1\over 1-\ep_0^2}{8\over3}\pi
\end{array} \right|\\
=&{\ep_0\over (1-\ep_0^2)^2}\left({8\over3}\pi\right)^3\neq0.
\end{align*}
By the Implicit Function Theorem, there exists $\delta=\delta(\ep_0)>0$ such that
for any
$\tilde \omega\in Y_{non}$ with $d_2(\tilde \omega,\omega_{\ep_0})\leq \delta$, there exist $\tilde x_0=\tilde x_0(\tilde\omega)\in\mathbb{T}_{2\pi}$, $\tilde y_0=\tilde y_0(\tilde\omega)\in\mathbb{R}$ and $\tilde \ep_0=\tilde \epsilon_0(\tilde\omega)\in(a(\ep_0),b(\ep_0))\subset(0,1)$, depending continuously on $\tilde \omega$, such that
 $S_i(\tilde\omega,\tilde x_0(\tilde\omega),$ $\tilde y_0(\tilde\omega),\tilde \ep_0(\tilde\omega))=0$ for $i=1,2,3$.

Define a mapping $:\chi \mapsto \mathcal{T}\chi$ by
\begin{align*}
(\mathcal{T}\chi)(\tilde \omega):=\chi(\tilde \omega)-\left({\partial(S_1,S_2,S_3)\over \partial(x_1,y_1,\ep_1)}\bigg|_{\tilde \omega=\omega_{\ep_0},x_1=0,y_1=0,\ep_1=\ep_0}\right)^{-1}\vec{S}(\tilde \omega,\chi(\tilde \omega)^T),
\end{align*}
where $\chi\in C(\bar{B}_{d_2}(\omega_{\ep_0},\delta),\Omega\times (0,1))$, $\bar{B}_{d_2}(\omega_{\ep_0},\delta)$ is the closed ball in $Y_{non}$ centred at $\omega_{\ep_0}$ with semi-radius $\delta$ under the distance $d_2$, and  $\vec{S}=(S_1,S_2,S_3)^T$. The distance between $\chi_1$ and $\chi_2$ is given by  $\rho(\chi_1,\chi_2)=\max_{\tilde \omega\in\bar{B}_{d_2}(\omega_{\ep_0},\delta)}|\chi_1(\tilde \omega)-\chi_2(\tilde \omega)|.$
It is  standard that $\mathcal{T}$ is a contracting mapping with rate $\mu\in(0,1)$ on $\mathcal{H}=\{\chi\in C(\bar{B}_{d_2}(\omega_{\ep_0},\delta),\Omega\times (0,1))|\chi(\omega_{\ep_0})=(0,0,\ep_0)^T,|\chi(\tilde \omega)-(0,0,\ep_0)^T|\leq\nu\}$ for some $\nu>0$, and moreover, $\chi^*$, which is defined by $\chi^*(\tilde \omega)=(\tilde x_0(\tilde \omega),\tilde y_0(\tilde \omega),\tilde \ep_0(\tilde \omega))^T$ on $\bar{B}_{d_2}(\omega_{\ep_0},\delta)$, is the unique fixed point of $\mathcal{T}$. Then $\rho(\chi,\chi^*)= \rho(\chi,\mathcal{T}\chi^*)\leq \rho(\chi,\mathcal{T}\chi)+\rho(\mathcal{T}\chi,\mathcal{T}\chi^*)\leq \rho(\chi,\mathcal{T}\chi)+\mu\rho(\chi,\chi^*)$ for $\chi\in\mathcal{H}$, which implies that $\rho(\chi,\chi^*)\leq {1\over 1-\mu}\rho(\chi,\mathcal{T}\chi)$. By  choosing $\chi_0\equiv(0,0,\ep_0)^T$,  for any $\tilde \omega\in\bar{B}_{d_2}(\omega_{\ep_0},\delta)$ we have
\begin{align*}
&|\tilde x_0(\tilde \omega)|+|\tilde y_0(\tilde \omega)|+|\tilde \ep_0(\tilde \omega)-\ep_0|\leq \rho(\chi_0,\chi^*)\leq {1\over 1-\mu}\rho(\chi_0,\mathcal{T}\chi_0)\\
\leq& {C\over 1-\mu}\left\|\left({\partial(S_1,S_2,S_3)\over \partial(x_1,y_1,\ep_1)}\bigg|_{\tilde \omega=\omega_{\ep_0},x_1=0,y_1=0,\ep_1=\ep_0}\right)^{-1}\right\|\max_{\tilde \omega\in\bar{B}_{d_2}(\omega_{\ep_0},\delta)}|\vec{S}(\tilde \omega,(0,0,\ep_0))|\leq C(\ep_0)\sqrt{\delta},
\end{align*}
where $\|\cdot\|$ is a norm on $\mathbb{R}^{3\times 3}$.

Let $x_0\neq 0$ or $y_0\neq0$. For any $\tilde \omega\in Y_{non}$ with $d_2(\tilde \omega,\omega_{\ep_0}(x+x_0,y+y_0))=\|\tilde\psi(x,y)-\psi_{\ep_0}(x+x_0,y+y_0)\|_{\dot{H}^1(\Omega)}^2\leq \delta$, we define $\tilde \psi_1(x,y)=\tilde \psi(x-x_0,y-y_0)$ and $\tilde \omega_1=-\Delta\tilde \psi_1$. Then $d_2(\tilde \omega_1,\omega_{\ep_0})=\|\tilde\psi_1-\psi_{\ep_0}\|_{\dot{H}^1(\Omega)}^2\leq \delta$, and thus, there exist $\tilde x_0(\tilde\omega_1)\in\mathbb{T}_{2\pi}$, $\tilde y_0(\tilde\omega_1)\in\mathbb{R}$ and $\tilde \epsilon_0(\tilde\omega_1)\in(a(\ep_0),b(\ep_0))$ such that
\begin{align*}
S_i(\tilde\omega_1,\tilde x_0(\tilde\omega_1), \tilde y_0(\tilde\omega_1),\tilde \ep_0(\tilde\omega_1))=S_i(\tilde\omega,x_0+\tilde x_0(\tilde\omega_1), y_0+\tilde y_0(\tilde\omega_1),\tilde\ep_0(\tilde\omega_1))=0
\end{align*}
 for $i=1,2,3$. The conclusion follows from setting $\tilde x_0=x_0+\tilde x_0(\tilde \omega_1), \tilde y_0=y_0+\tilde y_0(\tilde \omega_1)$ and $\tilde\ep_0=\tilde \ep_0(\tilde \omega_1)$.
\end{proof}

Moreover, we prove that the following functional is not locally flat on the family of steady states $\omega_{\ep}, \ep\in[0,1)$. This is useful to control the distance between  the evolved solution and the given steady state  in the $\ep$ direction.

\begin{lemma}\label{intOmega2} As a function of $\ep$,
\begin{align}\label{additional functional-2deuler}I(\omega_\ep)\triangleq\iint_{\Omega} (-\omega_\ep)^{3\over2} dxdy\end{align}
can  not be a constant on any subinterval of $(-1, 1)$, where  $\omega_\ep = - \frac{1 - \ep^2}{(\cosh(y) + \ep \cos(x))^2}$.
%In particular, $\iint_{\Omega} \omega_\ep^2 dxdy$ is increasing on $\ep\in(0,1)$.
\end{lemma}
\begin{proof}
By \eqref{Jacobian of the transformation-ep}, we have  $$\frac{\partial (\theta_\ep, \gamma_\ep)}{\partial (x, y)} =\frac{1}{2} g'(\psi_\ep) = - \omega_\ep,$$
and thus,
\begin{align*}
\iint_{\Omega} (-\omega_\ep)^{3\over2}  dxdy = \int_{-1}^1 \int_{0}^{2\pi} (-\omega_\ep)^{1\over2}  d \theta_\ep d\gamma_\ep.
\end{align*}
\if0
We use $\eta_\ep, \gamma_\ep, \xi_\ep$ and $\ep$ to represent $-\omega_\ep = \frac{1 - \ep^2}{(\cosh(y) + \ep \cos(x))^2}$. Rewriting $$\xi_\ep = \frac{\ep \cosh(y) + \cos(x)}{\cosh(y) + \ep \cos(x)} = \frac{\ep \frac{\cosh(y)}{\cos(x)} + 1}{\frac{\cosh(y)}{\cos(x)} + \ep},$$
we obtain $$\frac{\cosh(y)}{\cos(x)} + \ep = \frac{1-\ep^2}{\xi_\ep - \ep}.$$
Similarly, we have
$$\eta_\ep = \frac{\sqrt{1-\ep^2} \sin(x)}{\cosh(y) + \ep \cos(x)} = \frac{\sqrt{1-\ep^2} \tan(x) }{\frac{\cosh(y)}{\cos(x)} + \ep}$$
and $$\tan(x) = \frac{\sqrt{1-\ep^2} \eta_\ep }{\xi_\ep - \ep}.$$
\fi
By \eqref{omega-xi-eta-gamma-ep}, we have
\begin{align*}
-\omega_\ep
 = \eta_\ep^2 + \frac{1}{1-\ep^2} (\xi_\ep - \ep)^2.
\end{align*}
Recall that $
\eta_\ep = \sqrt{1-\gamma_\ep^2} \sin(\theta_\ep)$ and $
\xi_\ep = \sqrt{1-\gamma_\ep^2} \cos(\theta_\ep)$. Then
we have
\begin{align*}
I(\omega_\ep)&=\iint_{\Omega} (-\omega_\ep)^{3\over2}  dxdy
 = \int_{-1}^1 \int_{0}^{2\pi} (-\omega_\ep)^{1\over2}  d \theta_\ep d\gamma_\ep \\
& = \int_{-1}^1 \int_{0}^{2\pi} \left(\eta_\ep^2 + \frac{1}{1-\ep^2} (\xi_\ep - \ep)^2\right)^{1\over2} d \theta_\ep d\gamma_\ep \\
& = \int_{-1}^1 \int_{0}^{2\pi} \left((1-\gamma_\ep^2) \sin^2(\theta_\ep) + \frac{1}{1-\ep^2} \left( \sqrt{1-\gamma_\ep^2} \cos(\theta_\ep) -\ep \right)^2\right)^{1\over2}  d \theta_\ep d\gamma_\ep \\
& \geq\frac{1}{\sqrt{1-\ep^2}}\int_{-1}^1 \int_{0}^{2\pi} \left| \sqrt{1-\gamma_\ep^2} \cos(\theta_\ep) -\ep \right|  d \theta_\ep d\gamma_\ep
\\
&\to\infty \quad\text{as}\quad \ep\to\pm1^{\mp}.
\end{align*}
Since $I(\omega_\ep)$, as a function of $\ep$, is real-analytic on $(-1,1)$, $I(\omega_\ep)$ can  not be a constant on any subinterval of $(-1, 1)$.
\end{proof}

\subsection{Proof of nonlinear orbital stability for co-periodic perturbations}
Now, we are in a position to prove Theorem \ref{main result4-nonlinear orbital stability}.

\begin{proof}[Proof of  Theorem \ref{main result4-nonlinear orbital stability}]
We prove  the existence of the weak solution to the 2D Euler equation
for the initial vorticity $\tilde\omega_0\in  Y_{non}$ in the Appendix. Indeed, we first construct a smoothly approximate solution sequence. Precisely, we define the mollified initial vorticity $\tilde\omega_0^{\mu}$ as in  \eqref{tilde-omega0-kappa-def} for $\mu>0$. In  Lemma \ref{lem-construction of an approximate solution sequence},
for
the initial velocity $\vec{v}_0^{\mu}=K\ast\tilde\omega_0^{\mu}$,  we prove that  there exists  a  smoothly strong solution $\vec{v}^{\mu}(t)\in H^q(\Omega)$  globally in time   to the 2D Euler equation for any $q\geq3$.
$\{\vec{v}^{\mu}\}$ forms an approximate solution sequence with $L^1$, $L^2$ vorticity control (see Definition \ref{Approximate solution sequence for the 2D Euler equation}).
In Lemma \ref{convergence of an approximate solution sequence} and Theorem \ref{existence of weak solution to the 2D Euler equation-thm},
we prove the convergence of  the approximate solution sequence $\{\vec{v}^{\mu}\}$ in $L^1\cap L^2(\Omega_{R,T})$ for any $R, T>0$, and that the limit function  $\vec{v}\in L^1\cap L^2(\Omega_{R,T})$ is a weak solution to the 2D Euler equation
    for the initial vorticity $\tilde \omega_0\in Y_{non}$, where $\Omega_{R,T}= [0,T]\times B_R$ and $B_R=\{x\in\mathbb{T}_{2\pi},y\in[-R,R]\}$. For the nonlinear orbital stability of $\omega_{\ep_0}$,
we divide the  proof  into two steps.
\vspace{0.5mm}

\noindent{\bf{Step 1.}} Prove the nonlinear orbital stability for the smoothly approximate solution $\omega^\mu(t)=\curl(\vec{v}^{\mu}(t))$. More precisely,
for any $\kappa>0$, there exists $\tilde\delta=\tilde\delta(\ep_0,\kappa)>0$ (independent of $\mu$) such that if
\begin{align}\nonumber&\inf_{(x_0,y_0)\in\Omega} d(\tilde \omega^\mu(0),\omega_{\ep_0}(x+x_0,y+y_0))\\
+&\inf_{(x_0,y_0)\in\Omega}\|\tilde \omega^\mu(0)-\omega_{\ep_0}(x+x_0,y+y_0)\|_{L^2(\Omega)}<\tilde\delta(\ep_0,\kappa),\label{initial-vorticity-app-small}
\end{align}
 then for any $t\geq0$, we have
\begin{align}\label{onlinear orbital stability-app}
\inf_{(x_0,y_0)\in\Omega}d(\tilde \omega^\mu(t),\omega_{\ep_0}(x+x_0,y+y_0))<\kappa.
\end{align}

By Lemma \ref{tilde-omega0-kappa-properties} (8), $\tilde \omega^\mu(0)\in Y_{non}$. It follows from Corollary \ref{y-tilde-omega-pseudoenergy-conserved} (1) that $\tilde \omega^\mu(t)\in Y_{non}$ for $t>0$. Thus, we infer from Lemma \ref{well-poseness-Poisson-equation-nonlinear-case} and \eqref{d1-well-def} that $d(\tilde \omega^\mu(t),\omega_{\ep_0}(x+x_0,y+y_0))$ is well-defined for $t>0$.
By Lemma \ref{imp-vertical condition},  there exists $\delta_0(\ep_0)>0$ such that for any  $(x_0,y_0)\in\Omega$ and  $\tilde \omega\in Y_{non}$ with $d_2(\tilde \omega,\omega_{\ep_0}(x+x_0,y+y_0))< \delta_0(\ep_0)$, there exist $(\tilde x_0,\tilde y_0)\in\Omega$ and $\tilde\epsilon_0\in(a(\ep_0),b(\ep_0))$, depending continuously on $\tilde\omega, x_0, y_0$, such that
\begin{align}\label{app-lemma-imp-vertical condition}
\tilde \psi\left(x-\tilde x_0,y-\tilde y_0\right)-\psi_{\tilde\ep_0}(x,y)\perp\ker \left( A_{\tilde\ep_0}\right)\quad \text{in}\quad \dot{H}^1(\Omega)
\end{align}
 and
$
|x_0-\tilde x_0|+|y_0-\tilde y_0|+|\ep_0-\tilde \ep_0|\leq C(\ep_0)\sqrt{\delta_0(\ep_0)}
$
for some $a(\ep_0)\in (0,\ep_0)$ and $b(\ep_0)\in(\ep_0,1)$.
For any $\kappa>0$, let $\tilde\delta=\tilde\delta(\ep_0,\kappa)<\min\big\{{\kappa^2\over8C_1C_2(\ep_0)^2C_3(\ep_0)^2},$ ${\delta_0(\ep_0)\over2},1\big\}$, where $C_1, C_2(\ep_0), C_3(\ep_0)>1$ are  determined by \eqref{de-c1}, \eqref{I-omegaep1t0-omegaep0} and \eqref{de-c3}.
For the initial data $\tilde \omega^\mu(0)$ satisfying
\eqref{initial-vorticity-app-small},
there exist $(x_0^\mu(0),y_0^\mu(0))\in\Omega$ and $(x_*^\mu(0),y_*^\mu(0))\in\Omega$ such that
\begin{align}\label{initial data distance} &d(\tilde \omega^\mu(0),\omega_{\ep_0}(x+x_0^\mu(0),y+y_0^\mu(0)))<\tilde\delta(\ep_0,\kappa),\\
&\|\tilde \omega^\mu(0)-\omega_{\ep_0}(x+x_*^\mu(0),y+y_*^\mu(0))\|_{L^2(\Omega)}<\tilde\delta(\ep_0,\kappa).\label{initial data}\end{align}

For $t\geq0$, we claim that if there exists $( x_0^\mu(t),y_0^\mu(t))\in\Omega$ such that $d(\tilde \omega^\mu(t),\omega_{\ep_0}(x+x_0^\mu(t),y+y_0^\mu(t)))<\delta_0(\ep_0)$, then there exist $(x_1^\mu(t),y_1^\mu(t))\in\Omega$ and $\ep_1^\mu(t)\in(a(\ep_0),b(\ep_0))$ such that
\begin{align}\label{a prior estimate}
d(\tilde \omega^\mu(t),\omega_{\ep_1^\mu(t)}(x+x_1^\mu(t),y+y_1^\mu(t)))<{\kappa^2\over4C_2(\ep_0)^2C_3(\ep_0)^2}.
\end{align}
In fact, by applying \eqref{app-lemma-imp-vertical condition} to  $\tilde \omega^\mu(t)$,  we can choose $(x_1^\mu(t),y_1^\mu(t))\in\Omega$ and $\ep_1^\mu(t)\in(a(\ep_0),b(\ep_0))$, depending continuously on $t$, such that
$\tilde \psi^\mu(x-x_1^\mu(t),$ $y-y_1^\mu(t))-\psi_{\ep_1^\mu(t)}(x,y)\perp\ker \left( A_{\ep_1^\mu(t)}\right)$
in $ \tilde X_{\ep_1^\mu(t)}$,
and
\begin{align}\label{translation distance}
|x_0^{\mu}(t)-x_1^{\mu}(t)|+|y_0^{\mu}(t)-y_1^{\mu}(t)|+|\ep_0-\ep_1^{\mu}(t)|\leq C(\ep_0)\sqrt{\delta_0(\ep_0)}.
\end{align}
By \eqref{initial data distance} and Lemma \ref{imp-vertical condition}, $\sqrt{\delta_0(\ep_0)}$ in \eqref{translation distance} can be replaced by $\sqrt{\tilde\delta(\ep_0,\kappa)}$ for $t=0$. By adding a constant if necessary, we have $\tilde \psi^\mu(x-x_1^\mu(t),$ $y-y_1^\mu(t))-\psi_{\ep_1^\mu(t)}(x,y)\in
\tilde X_{\ep_1^\mu(t)}$. Noting that if the constant is omitted, then the proof is the same since $\iint_\Omega\psi\omega dxdy=\iint_\Omega(\psi-c)\omega dxdy$ in  \eqref{H-omega-H-0} for any $c\in\mathbb{R}$ due to $\iint_\Omega\omega dxdy=0$. So in this proof, we write $\tilde \psi^\mu(x-x_1^\mu(t),$ $y-y_1^\mu(t))-\psi_{\ep_1^\mu(t)}(x,y)\in
\tilde X_{\ep_1^\mu(t)}$ in the sense that a constant difference is allowed.
By taking $\tilde\delta(\ep_0,\kappa)>0$ smaller, we infer from \eqref{translation distance} for $t=0$ that  $ d(\omega_{\ep_0}(x+x_0^\mu(0),y+y_0^\mu(0)),\omega_{\ep}(x+x_1^\mu(0),y+y_1^\mu(0)))<{\kappa^2\over8C_1C_2(\ep_0)^2C_3(\ep_0)^2}$, which along with \eqref{initial data distance}, implies
\begin{align*}
&d(\tilde \omega^\mu(0),\omega_{\ep}(x+x_1^\mu(0),y+y_1^\mu(0)))\\
\leq& d(\tilde \omega^\mu(0),\omega_{\ep_0}(x+x_0^\mu(0),y+y_0^\mu(0)))\\
&+d( \omega_{\ep_0}(x+x_0^\mu(0),y+y_0^\mu(0)), \omega_{\ep}(x+x_1^\mu(0), y+ y_1^\mu(0)))\\
\leq &{\kappa^2\over8C_1C_2(\ep_0)^2C_3(\ep_0)^2}+{\kappa^2\over8C_1C_2(\ep_0)^2C_3(\ep_0)^2}={\kappa^2\over4C_1C_2(\ep_0)^2C_3(\ep_0)^2},
\end{align*}
where $\ep=\ep_0$ or $\ep_1^\mu(0)$.
Take $\tau\in(0,1)$ small enough such that  $\left(  (1-\tau) C_0- \frac 1 2 \tau\right)>\tau$, where
$C_0>0$ is given in \eqref{A-ep-positive-lower-bound}.
By \eqref{H-omega-H-0}-\eqref{def-functional-B}, \eqref{B-ep-A-ep}-\eqref{A-ep-positive-lower-bound} and Lemma \ref{B-C2}, we have
\begin{align}\nonumber
& d(\tilde \omega^\mu(0),\omega_{\ep_1^\mu(0)}(x+x_1^\mu(0),y+y_1^\mu(0)))\\\nonumber
\geq&H_{\ep_1^\mu(0)}(\tilde \omega^\mu(0)-\omega_{\ep_1^\mu(0)}(x+x_1^\mu(0),y+y_1^\mu(0))) - H_{\ep_1^\mu(0)}(0)\\\nonumber
 =&H_{\ep_1^\mu(t)}(\tilde \omega_{tran}^\mu(t)-\omega_{\ep_1^\mu(t)}) - H_{\ep_1^\mu(t)}(0)\\\nonumber
 =&
 \tau d_1(\tilde \omega_{tran}^\mu(t),\omega_{\ep_1^\mu(t)}) - \frac 1 2 \tau d_2(\tilde \omega_{tran}^\mu(t),\omega_{\ep_1^\mu(t)}) \\\nonumber
 &+ (1-\tau) \left( d_1(\tilde \omega_{tran}^\mu(t),\omega_{\ep_1^\mu(t)}) - \frac 1 2  d_2(\tilde \omega_{tran}^\mu(t),\omega_{\ep_1^\mu(t)})\right) \\\nonumber
 \geq& \tau d_1(\tilde \omega_{tran}^\mu(t),\omega_{\ep_1^\mu(t)}) - \frac 1 2 \tau d_2(\tilde \omega_{tran}^\mu(t),\omega_{\ep_1^\mu(t)})
 + (1-\tau) \mathscr{B}_{\ep_1^\mu(t)}(\tilde \psi_{tran}^\mu(t)-\psi_{\ep_1^\mu(t)})\\\nonumber
 =&\tau d_1(\tilde \omega_{tran}^\mu(t),\omega_{\ep_1^\mu(t)}) - \frac 1 2 \tau d_2(\tilde \omega_{tran}^\mu(t),\omega_{\ep_1^\mu(t)})\\\nonumber
& + (1-\tau) \left(\langle A_{\ep_1^\mu(t)}(\tilde \psi_{tran}^\mu(t)-\psi_{\ep_1^\mu(t)}),(\tilde \psi_{tran}^\mu(t)-\psi_{\ep_1^\mu(t)})\rangle+o(d_2(\tilde \omega_{tran}^\mu(t),\omega_{\ep_1^\mu(t)}))\right)\\\nonumber
\geq&\tau d_1(\tilde \omega_{tran}^\mu(t),\omega_{\ep_1^\mu(t)})+\left(  (1-\tau) C_0- \frac 1 2 \tau\right) d_2(\tilde \omega_{tran}^\mu(t),\omega_{\ep_1^\mu(t)})\\\nonumber
&+o(d_2(\tilde \omega_{tran}^\mu(t),\omega_{\ep_1^\mu(t)}))\\\nonumber
\geq&\tau d(\tilde \omega_{tran}^\mu(t),\omega_{\ep_1^\mu(t)})+o(d(\tilde \omega_{tran}^\mu(t),\omega_{\ep_1^\mu(t)}))\\\label{t control by 0-Taylor of dual functional}
=&\tau d(\tilde \omega^\mu(t),\omega_{\ep_1^\mu(t)}(x+x_1^\mu(t),y+y_1^\mu(t)))
+o(d(\tilde \omega^\mu(t),\omega_{\ep_1^\mu(t)}(x+x_1^\mu(t),y+y_1^\mu(t)))),
\end{align}
where $\tilde \omega_{tran}^\mu(t)\triangleq\tilde \omega^\mu(t,x-x_1^\mu(t),y-y_1^\mu(t))$, $\tilde \psi_{tran}^\mu(t)\triangleq\tilde \psi^\mu(t,x-x_1^\mu(t),y-y_1^\mu(t))$, and we used the fact that $H_{\ep}(\tilde \omega^\mu(t)-\omega_{\ep}(x+x_1,y+y_1)) - H_{\ep}(0)$ is conserved for all $t, x_1, y_1, \ep$. Here the conservation for $t$ and $\ep$ can be deduced from Corollary \ref{y-tilde-omega-pseudoenergy-conserved} (2) and \eqref{H independent ep}, respectively. Then for $\kappa>0$ sufficiently small, by \eqref{t control by 0-Taylor of dual functional} and the continuity of $d(\tilde \omega^\mu(t),\omega_{\ep_1^\mu(t)}(x+x_1^\mu(t),y+y_1^\mu(t)))$ on $t$ we have
\begin{align}\nonumber
&d(\tilde \omega^\mu(t),\omega_{\ep_1^\mu(t)}(x+x_1^\mu(t),y+y_1^\mu(t)))\\\label{de-c1}
\leq& C_1 d(\tilde \omega^\mu(0),\omega_{\ep_1^\mu(0)}(x+x_1^\mu(0),y+y_1^\mu(0)))< {\kappa^2\over4C_2(\ep_0)^2C_3(\ep_0)^2},
\end{align}
where  $C_1={2\over\tau}>1$. This proves \eqref{a prior estimate}.

For any $\kappa\in(0,\min\{\delta_0(\ep_0), 1\})$, suppose that \eqref{onlinear orbital stability-app} is not true. Then there exists  $t_0>0$ such that $\inf_{(x_0,y_0)\in\Omega}d(\tilde \omega^\mu(t),\omega_{\ep_0}(x+x_0,y+y_0))<\kappa$ for $0\leq t<t_0$ and
\begin{align}\label{suppose inf=kappa}
\inf_{(x_0,y_0)\in\Omega}d(\tilde \omega^\mu(t_0),\omega_{\ep_0}(x+x_0,y+y_0))=\kappa.
\end{align}
Since $\kappa<\delta_0(\ep_0)$, there exists $( x_0^\mu(t),y_0^\mu(t))\in\Omega$, depending continuously on $t$, such that $ d(\tilde \omega^\mu(t),\omega_{\ep_0}(x+x_0^\mu(t),y+y_0^\mu(t)))<\delta_0(\ep_0)$ for $0\leq t\leq t_0$.
 By \eqref{a prior estimate},
 there exist $(x_1^\mu(t),y_1^\mu(t))\in\Omega$ and $\ep_1^\mu(t)\in(a(\ep_0),b(\ep_0))$ such that
\begin{align}\label{d-t0-ep1}
d(\tilde \omega^\mu(t),\omega_{\ep_1^\mu(t)}(x+x_1^\mu(t),y+y_1^\mu(t)))<{\kappa^2\over4C_2(\ep_0)^2C_3(\ep_0)^2}<{\kappa\over2},\quad 0\leq t\leq t_0.
\end{align}
We then show that
\begin{align}\label{ep1t0ep0}
d(\omega_{\ep_1^\mu(t_0)},\omega_{\ep_0})<{\kappa\over 2}.
\end{align}
Assume that \eqref{ep1t0ep0} is true. Then
$
d(\tilde \omega^\mu(t_0),\omega_{\ep_0}(x+x_1^\mu(t_0),y+y_1^\mu(t_0)))\leq d(\tilde \omega^\mu(t_0),\omega_{\ep_1^\mu(t_0)}(x+x_1^\mu(t_0),y+y_1^\mu(t_0)))+d(\omega_{\ep_1^\mu(t_0)}(x+x_1^\mu(t_0),y+y_1^\mu(t_0)),\omega_{\ep_0}(x+x_1^\mu(t_0),y+y_1^\mu(t_0))) <{\kappa\over2}+{\kappa\over2}=\kappa.
$
This contradicts \eqref{suppose inf=kappa}.

The rest is to prove \eqref{ep1t0ep0}. By the continuity of $
d(\omega_{\ep},\omega_{\ep_0})$ on $\ep$, it suffices to show that $|\ep_1^\mu(t_0)-\ep_0|<\delta_1(\ep_0)$ for some  $\delta_1(\ep_0)>0$ small enough.
Note that
$|\ep_1^{\mu}(0)-\ep_0|\leq C(\ep_0)\sqrt{\tilde\delta(\ep_0,\kappa)}$ by \eqref{translation distance} for $t=0$, and $\ep_1^{\mu}(t)$ is continuous on $t\in[0,t_0]$.
 By Lemma \ref{intOmega2} and taking $\tilde\delta(\ep_0,\kappa)>0$ smaller, we only need to prove that
\begin{align}\label{I-omegaep1t0-omegaep0}
|I(\omega_{\ep_1^\mu(t)})-I(\omega_{\ep_0})|<{\kappa\over C_2(\ep_0)},\quad 0\leq t\leq t_0
\end{align}
for some $C_2(\ep_0)>1$ large enough, where $I(\tilde \omega)=\iint_{\Omega}(-\tilde \omega)^{3\over2}dxdy$ for $\tilde \omega\in Y_{non}$. In fact, by Taylor's formula, we have
\begin{align}\nonumber
&d_1(\tilde \omega^\mu(t),\omega_{\ep_1^\mu(t)}(x+x_1^\mu(t),y+y_1^\mu(t)))\\\nonumber
=&\iint_{\Omega}\bigg(h(\tilde \omega^\mu(t))-h(\omega_{\ep_1^\mu(t)}(x+x_1^\mu(t),y+y_1^\mu(t)))\\\nonumber
&-h'(\omega_{\ep_1^\mu(t)}(x+x_1^\mu(t),y+y_1^\mu(t)))
(\tilde \omega^\mu(t)-\omega_{\ep_1^\mu(t)}(x+x_1^\mu(t),y+y_1^\mu(t)))\bigg)dxdy\\\nonumber
=&\int_0^1\iint_\Omega{(1-r)\big(\tilde \omega^\mu(t)-\omega_{\ep_1^\mu(t)}(x+x_1^\mu(t),y+y_1^\mu(t))\big)^2\over 2|\omega^{\mu,r}(t)|}dxdydr\\\label{d1 estimates}
\geq&\iint_\Omega{\big(\tilde \omega^\mu(t)-\omega_{\ep_1^\mu(t)}(x+x_1^\mu(t),y+y_1^\mu(t))\big)^2\over 4|\tilde \omega^\mu(t)+\omega_{\ep_1^\mu(t)}(x+x_1^\mu(t),y+y_1^\mu(t))|}dxdy,
\end{align}
where $0\leq t\leq t_0$ and $\omega^{\mu,r}(t,x,y)=r\tilde \omega^\mu(t,x,y)+(1-r)\omega_{\ep_1^\mu(t)}(x+x_1^\mu(t),y+y_1^\mu(t))
%\leq|\tilde \omega(t_0;x,y)+\omega_{\ep_1(t_0)}(x+x_1(t_0),y+y_1(t_0))
$ for $r\in[0,1]$.
%since $\omega(t_0)<0$ and $\omega_{\ep_1(t_0)}<0$.
Noting that $I(\tilde \omega^\mu(t))$ is conserved for all $t$, by \eqref{d1 estimates} and \eqref{d-t0-ep1}  we have
\begin{align}\nonumber
&|I(\tilde \omega^\mu(0))-I(\omega_{\ep_1^\mu(t)})|=|I(\tilde \omega^\mu(t))-I(\omega_{\ep_1^\mu(t)}(x+x_1^\mu(t),y+y_1^\mu(t)))|\\\nonumber
=&\left|\iint_\Omega\left((-\tilde \omega^\mu(t))^{3\over2}-(-\omega_{\ep_1^\mu(t)}(x+x_1^\mu(t),y+y_1^\mu(t)))^{3\over2}\right)dxdy\right|\\\nonumber
=&{3\over2}\left|\int_0^1\iint_\Omega|\omega^{\mu,r}(t)|^{1\over2}(\tilde \omega^\mu(t)-\omega_{\ep_1^\mu(t)}(x+x_1^\mu(t),y+y_1^\mu(t)))dxdydr\right|\\\nonumber
\leq&{3\over2}\bigg|\iint_\Omega|\tilde \omega^\mu(t)+\omega_{\ep_1^\mu(t)}(x+x_1^\mu(t),y+y_1^\mu(t))|^{1\over2}\cdot\\\nonumber
&|\tilde \omega^\mu(t)-\omega_{\ep_1^\mu(t)}(x+x_1^\mu(t),y+y_1^\mu(t))|dxdy\bigg|\\\nonumber
\leq&{3\over2}\left(\iint_\Omega{(\tilde \omega^\mu(t)-\omega_{\ep_1^\mu(t)}(x+x_1^\mu(t),y+y_1^\mu(t)))^2\over 4|\tilde \omega^\mu(t)+\omega_{\ep_1^\mu(t)}(x+x_1^\mu(t),y+y_1^\mu(t))|}dxdy\right)^{1\over2}\cdot\\\nonumber
&\left(\iint_\Omega4|\tilde \omega^\mu(t)+\omega_{\ep_1^\mu(t)}(x+x_1^\mu(t),y+y_1^\mu(t))|^2dxdy\right)^{1\over2}\\\nonumber
\leq &{3\sqrt{2}}d_1(\tilde \omega^\mu(t),\omega_{\ep_1^\mu(t)}(x+x_1^\mu(t),y+y_1^\mu(t)))^{1\over2}
\left(\|\tilde \omega^\mu(t)\|_{L^2(\Omega)}^2+\|\omega_{\ep_1^\mu(t)}\|_{L^2 (\Omega)}^2\right)^{1\over2}\\\nonumber
\leq&{3\sqrt{2}}d_1(\tilde \omega^\mu(t),\omega_{\ep_1^\mu(t)}(x+x_1^\mu(t),y+y_1^\mu(t)))^{1\over2}
\left(\|\tilde \omega^\mu(0)\|_{L^2(\Omega)}^2+\|\omega_{\ep_1^\mu(t)}\|_{L^2 (\Omega)}^2\right)^{1\over2}\\\nonumber
\leq&C_3(\ep_0)d_1(\tilde \omega^\mu(t),\omega_{\ep_1^\mu(t)}(x+x_1^\mu(t),y+y_1^\mu(t)))^{1\over2}\\\label{Iomega0Iomegat1}
< &{\kappa\over2C_2(\ep_0)},\quad 0\leq t\leq t_0,
\end{align}
where
\begin{align}\label{de-c3}
C_3(\ep_0)=&{3}\sqrt{2}\bigg(\left(1+\|\omega_{\ep_0}\|_{L^2(\Omega)}\right)^2+
\max_{\ep\in [a(\ep_0),b(\ep_0)]}\|\omega_{\ep}\|_{L^2 (\Omega)}^2
\bigg)^{1\over2}>1,
\end{align}
and we used
%$\|\omega_{\ep_1(t_0)}\|_{L^\infty(\Omega)}\leq  {1+\ep_1(t_0)\over 1-\ep_1(t_0)}\leq {1+b(\ep_0)\over 1-b(\ep_0)} \triangleq  C_4(\ep_0)$, $\|\tilde \omega(t_0)\|_{L^i(\Omega)}=\|\tilde \omega(0)\|_{L^i(\Omega)}$ for $i=1,2,3$,  $\|\tilde \omega(0)\|_{L^2(\Omega)}^2\leq \|\tilde \omega(0)\|_{L^1(\Omega)}+\|\tilde \omega(0)\|_{L^3(\Omega)}^3$, $\|\tilde \omega(0)\|_{L^1(\Omega)}=4\pi$ and
$\|\tilde \omega^\mu(0)\|_{L^2(\Omega)}\leq\|\tilde \omega^\mu(0)-\omega_{\ep_0}(x+x_*^\mu(0),y+y_*^\mu(0))\|_{L^2(\Omega)}+ \|\omega_{\ep_0}\|_{L^2(\Omega)}\leq \tilde\delta(\ep_0,\kappa)+\|\omega_{\ep_0}\|_{L^2(\Omega)}\leq 1+\|\omega_{\ep_0}\|_{L^2(\Omega)}$  due to \eqref{initial data}.
\if0
By Taylor's formula,
\begin{align*}
f(\omega) &= h(\omega_\ep + \omega) - h(\omega_\ep) - h'(\omega_\ep) \omega \\
& = \int_0^1 (1-r) h''(\omega_\ep + r\omega) \omega^2 dr\\
& = \int_0^1 -\frac{(1-r)  \omega^2}{2(\omega_\ep + r\omega)} dr.\\
\end{align*}
So,
$$d_1(\omega) = \iint_\Omega f(\omega) dxdy = \int_0^1 \iint_\Omega -\frac{(1-r)  \omega^2}{2(\omega_\ep + r\omega)} dxdydr.$$
\fi
Similar to \eqref{d1 estimates}-\eqref{Iomega0Iomegat1}, we have
\begin{align}\nonumber
&|I(\tilde \omega^\mu(0))-I(\omega_{\ep_0})|=|I(\tilde \omega^\mu(0))-I(\omega_{\ep_0}(x+x_1^\mu(0),y+y_1^\mu(0)))|\\\label{Itildeomega0Iomegaep0}
\leq& C_3(\ep_0)d_1(\tilde \omega^\mu(0),\omega_{\ep_0}(x+x_1^\mu(0),y+y_1^\mu(0)))^{1\over2}
\leq {\kappa\over2\sqrt{C_1}C_2(\ep_0)}< {\kappa\over2C_2(\ep_0)},
\end{align}
where we used \eqref{initial data distance}.
Combining \eqref{Iomega0Iomegat1} and \eqref{Itildeomega0Iomegaep0}, we have
\begin{align*}
|I(\omega_{\ep_1^\mu(t)})-I(\omega_{\ep_0})|\leq |I(\tilde \omega^\mu(0))-I(\omega_{\ep_1^\mu(t)})|+|I(\tilde \omega^\mu(0))-I(\omega_{\ep_0})|
<{\kappa\over C_2(\ep_0)}
\end{align*}
for $0\leq t\leq t_0$.
This proves \eqref{I-omegaep1t0-omegaep0}.

\vspace{0.5mm}

\noindent{\bf{Step 2.}} Prove the nonlinear orbital stability \eqref{onlinear orbital stability-goal} for the weak solution $\tilde \omega(t)$ by taking limits.

For any $\kappa>0$, let $\delta(\ep_0,\kappa)={1\over3}\tilde\delta\left(\ep_0,{1\over2}\kappa\right)$ and $\tilde \omega(0)\in Y_{non}$ such that
$$\inf_{(x_0,y_0)\in\Omega} d(\tilde \omega(0),\omega_{\ep_0}(x+x_0,y+y_0))+\inf_{(x_0,y_0)\in\Omega}\|\tilde \omega(0)-\omega_{\ep_0}(x+x_0,y+y_0)\|_{L^2(\Omega)}<\delta(\ep_0,\kappa).$$
Then there exist $(\tilde x_1, \tilde y_1), (\tilde x_2,\tilde y_2)\in\Omega$ such that
\begin{align}\label{tilde-omega0-ometa-ep0-distance}
 d(\tilde \omega(0),\omega_{\ep_0}(x+\tilde x_1,y+\tilde y_1))+\|\tilde \omega(0)-\omega_{\ep_0}(x+\tilde x_2,y+\tilde y_2)\|_{L^2(\Omega)}<\delta(\ep_0,\kappa).
\end{align}
By Lemma \ref{tilde-omega0-kappa-properties} (8),  $-\tilde \omega^\mu(0)\ln(-\tilde \omega^\mu(0))\to-\tilde \omega(0)\ln(-\tilde\omega(0))$ in $L^1(\Omega)$. Moreover, $\tilde \omega^\mu(0)\to \tilde \omega(0)$ in $L^1\cap L^2(\Omega)$ and $\psi_{\ep_0}\tilde \omega^\mu(0)\to\psi_{\ep_0}\tilde \omega(0)$ in $L^1(\Omega)$ by Lemma \ref{tilde-omega0-kappa-properties} (4) and (7). Since $\psi_{(\tilde x_1,\tilde y_1)}(0,x,y)=(-\Delta)^{-1}(\tilde \omega(0,x-\tilde x_1,y-\tilde y_1)-\omega_{\ep_0}(x,y))\in \dot{H}^1(\Omega)$ by Lemma \ref{well-poseness-Poisson-equation-nonlinear-case}, we have $\psi_{(\tilde x_1,\tilde y_1)}^\mu(0)=\hat J_{\mu}\star\psi_{(\tilde x_1,\tilde y_1)}(0)\in\dot{H}^1(\Omega)$ and $ \nabla\psi_{(\tilde x_1,\tilde y_1)}^\mu(0)\to\nabla\psi_{(\tilde x_1,\tilde y_1)}(0)$ in $(L^2(\Omega))^2$, where $\star$  is defined in \eqref{convolution-R2-def}. Thus,
\begin{align*}
&\iint_\Omega\bigg(|h(\tilde \omega^\mu(0))-h(\tilde \omega(0))|+|\psi_{\ep_0}(x+\tilde x_1,y+\tilde y_1)(\tilde \omega^\mu(0)-\tilde \omega(0))|\\
&+2|\nabla\psi_{(\tilde x_1,\tilde y_1)}^\mu(0)-\nabla\psi_{(\tilde x_1,\tilde y_1)}(0)|^2\bigg)dxdy
+
\|\tilde \omega^\mu(0)- \tilde \omega(0)\|_{L^2(\Omega)}\to0
\end{align*}
as $\mu\to0^+$. This, along with \eqref{tilde-omega0-ometa-ep0-distance}, implies
\begin{align*}
&\inf_{(x_0,y_0)\in\Omega} d(\tilde \omega^\mu(0),\omega_{\ep_0}(x+x_0,y+y_0))+\inf_{(x_0,y_0)\in\Omega}\|\tilde \omega^\mu(0)-\omega_{\ep_0}(x+x_0,y+y_0)\|_{L^2(\Omega)}\\
\leq &d(\tilde \omega^\mu(0),\omega_{\ep_0}(x+\tilde x_1,y+\tilde y_1))+\|\tilde \omega^\mu(0)-\omega_{\ep_0}(x+\tilde x_2,y+\tilde y_2)\|_{L^2(\Omega)}\\
\leq&\iint_\Omega\bigg(|h(\tilde \omega^\mu(0))-h(\tilde \omega(0))|+|\psi_{\ep_0}(x+\tilde x_1,y+\tilde y_1)(\tilde \omega^\mu(0)-\tilde \omega(0))|\\
&+2|\nabla\psi_{(\tilde x_1,\tilde y_1)}^\mu(0)-\nabla\psi_{(\tilde x_1,\tilde y_1)}(0)|^2\bigg)dxdy
+d_1(\tilde \omega(0),\omega_{\ep_0}(x+\tilde x_1,y+\tilde y_1))\\
&+2d_2(\tilde \omega(0),\omega_{\ep_0}(x+\tilde x_1,y+\tilde y_1))+\|\tilde \omega^\mu(0)- \tilde \omega(0)\|_{L^2(\Omega)}+
\|\tilde \omega(0)-\omega_{\ep_0}(x+\tilde x_2,y+\tilde y_2)\|_{L^2(\Omega)}\\
\leq &3\delta(\ep_0,\kappa)=\tilde\delta\left(\ep_0,{1\over2}\kappa\right)
\end{align*}
for $\mu>0$ sufficiently small.
For fixed  $t\geq0$, by applying  Step 1, there exists $(x_1^\mu(t),y_1^\mu(t))\in\Omega$ such that
\begin{align}\label{onlinear orbital stability-apply1}
d(\tilde \omega_{tran}^\mu(t),\omega_{\ep_0})=d(\tilde \omega^\mu(t),\omega_{\ep_0}(x+x_1^\mu(t),y+y_1^\mu(t)))<{1\over2}\kappa
\end{align}
for $\mu>0$ sufficiently small.
%Here, we use the notation $\tilde \omega_{tran}^\mu(t)\triangleq\tilde \omega^\mu(t,x-x_1^\mu(t),y-y_1^\mu(t))$.

Then we claim  that there exists $C(\ep_0,\tilde\omega(0))>0$ (independent of $\mu$) such that  $|y_1^\mu(t)|<C(\ep_0,\tilde\omega(0))$ for $\mu>0$ sufficiently small. Indeed,
by Corollary \ref{y-tilde-omega-pseudoenergy-conserved} (1) and Lemma \ref{tilde-omega0-kappa-properties} (6), we have
\begin{align}\label{y-bounded-first term}
\left|\iint_\Omega y\tilde\omega^\mu(t)dxdy\right|=\left|\iint_\Omega y\tilde\omega^\mu(0)dxdy\right|\leq\|y\tilde\omega^\mu(0)\|_{L^1(\Omega)}\leq \|y\tilde\omega(0)\|_{L^1(\Omega)}+1
\end{align}
 for $\mu>0$ small enough. For $|y|>\ln(4)$, we have
 \begin{align*}
 \psi_{\ep_0}(x,y)= \ln \left(\frac{\cosh (y) + \epsilon_0 \cos (x)}{\sqrt{1-\epsilon_0^2}} \right)\geq\ln \left(\frac{\cosh (y) -1}{\sqrt{1-\epsilon_0^2}} \right)\geq\ln \left({e^{|y|}\over4\sqrt{1-\epsilon_0^2}} \right)>0,
\end{align*}
and thus,
\begin{align}\label{y-bounded-second term}
|y|\leq \psi_{\ep_0}(x,y)+C_4(\ep_0), \quad y\in \mathbb{R},
\end{align}
where $C_4(\ep_0)=\left|\ln\left(4\sqrt{1-\epsilon_0^2}\right)\right|+\ln(4)+\max\limits_{x\in\mathbb{T}_{2\pi},y\in[-\ln(4),\ln(4)]}|\psi_{\ep_0}(x,y)|$.
By \eqref{y-bounded-first term}-\eqref{y-bounded-second term}, \eqref{d1-well-def} and \eqref{onlinear orbital stability-apply1}, we have
\begin{align}\nonumber
|4\pi y_1^\mu(t)|=&\left|\iint_\Omega(y-y_1^\mu(t))\tilde\omega_{tran}^\mu(t)dxdy-\iint_\Omega y\tilde\omega_{tran}^\mu(t)dxdy\right|\\\nonumber
\leq&\|y\tilde\omega(0)\|_{L^1(\Omega)}+1-\iint_\Omega\psi_{\ep_0}\tilde\omega_{tran}^\mu(t)dxdy
+C_4(\ep_0)\|\tilde\omega^\mu(t)\|_{L^1(\Omega)}\\\nonumber
\leq&\|y\tilde\omega(0)\|_{L^1(\Omega)}+1+
d_1(\tilde \omega_{tran}^\mu(t),\omega_{\ep_0})\\\nonumber
&+\iint_{\Omega}\left({1\over2}(-\tilde \omega^\mu(t)+\tilde \omega^\mu(t)\ln(-\tilde \omega^\mu(t)))+{1\over 2}\omega_{\ep_0} \right) dxdy+C_4(\ep_0)\|\tilde\omega^\mu(t)\|_{L^1(\Omega)}\\\nonumber
\leq&\|y\tilde\omega(0)\|_{L^1(\Omega)}+1+{\kappa\over2}+\left({1\over2}+C_4(\ep_0)\right)(\|\tilde\omega(0)\|_{L^1(\Omega)}+1)\\\nonumber
&+{1\over2}(\|\tilde \omega(0)\ln(-\tilde \omega(0))\|_{L^1(\Omega)}+1)+{1\over2}\|\omega_{\ep_0}\|_{L^1(\Omega)}\triangleq 4\pi C(\ep_0,\tilde\omega(0))
\end{align}
for $\mu>0$ small enough, where we used
\begin{align*}
\|\tilde\omega^\mu(t)\|_{L^1(\Omega)}=&\|\tilde\omega^\mu(0)\|_{L^1(\Omega)}\leq \|\tilde\omega(0)\|_{L^1(\Omega)}+1,\\
\|\tilde \omega^\mu(t)\ln(-\tilde \omega^\mu(t))\|_{L^1(\Omega)}=&\|\tilde \omega^\mu(0)\ln(-\tilde \omega^\mu(0))\|_{L^1(\Omega)}\leq
\|\tilde \omega(0)\ln(-\tilde \omega(0))\|_{L^1(\Omega)}+1
\end{align*}
by Lemma \ref{tilde-omega0-kappa-properties} (4) and (8).


Up to a subsequence,  $x_1^\mu(t)\to x_1(t)$ and $y_1^\mu(t)\to y_1(t)$ for some $(x_1(t), y_1(t))\in\Omega$ as $\mu\to0^+$. We denote $\tilde \omega_{tran}(t)\triangleq\tilde \omega(t,x-x_1(t),y-y_1(t))$.
 By \eqref{tilde-omega-kappa-weak convergence L1L2}, we have
\begin{align*}
&\left|\iint_\Omega\left(\tilde \omega_{tran}^\mu(t)-\tilde \omega_{tran}(t)\right)\varphi(x,y)dxdy\right|\\
%=&\iint_\Omega\left(\tilde \omega^\mu(t)\varphi(x+x_1^\mu(t),y+y_1^\mu(t))-\tilde \omega(t)\varphi(x+x_1(t),y+y_1(t))\right)dxdy\\
=&\bigg|\iint_\Omega\bigg(\tilde \omega^\mu(t)(\varphi(x+x_1^\mu(t),y+y_1^\mu(t))-\varphi(x+x_1(t),y+y_1(t)))+\\
&(\tilde \omega^\mu(t)-\tilde \omega(t))\varphi(x+x_1(t),y+y_1(t))\bigg)dxdy\bigg|\\
\leq&\|\tilde \omega^\mu(t)\|_{L^2(\Omega)}\|\varphi(x+x_1^\mu(t),y+y_1^\mu(t))-\varphi(x+x_1(t),y+y_1(t))\|_{L^2(\Omega)}\\
&+\bigg|\iint_\Omega
(\tilde \omega^\mu(t)-\tilde \omega(t))\varphi(x+x_1(t),y+y_1(t))dxdy\bigg|\to0\text{ as } \mu\to0^+
\end{align*}
for $\varphi\in L^2(\Omega)$,
where we used  $\|\tilde \omega^\mu(t)\|_{L^2(\Omega)}\leq C$ uniformly for $\mu>0$ small enough by Lemma \ref{lem-construction of an approximate solution sequence}.
Thus,
\begin{align}\label{tilde-omega-mu-translation-convergence L2}
\tilde \omega_{tran}^\mu(t)\rightharpoonup\tilde \omega_{tran}(t) \text{ in } L^2(\Omega).
\end{align}
Since $h(s)={1\over2}(s-s\ln(-s))$ is convex on $(-\infty,0]$, $\tilde \omega(t)\leq0$ a.e. on $\Omega$ by Corollary \ref{vorticity L123}, and $\psi_\ep \in L^2(B_R)$ for any $R>0$, it follows from  Theorem 1.1, Remark (iii) in \cite{Dacorogna} (see also \cite{Morrey1966})  and  \eqref{tilde-omega-mu-translation-convergence L2}  that
\begin{align}\nonumber
&\iint_{B_R}\left(h(\tilde \omega_{tran}(t)) - h(\omega_{\ep_0})  - \psi_{\ep_0} (\tilde \omega_{tran}(t)-\omega_{\ep_0})\right)dxdy\\\nonumber
\leq& \liminf_{\mu\to0^+}\iint_{B_R}(h(\tilde \omega_{tran}^\mu(t)) - h(\omega_{\ep_0}) - \psi_{\ep_0} (\tilde \omega_{tran}^\mu(t)
-\omega_{\ep_0}))dxdy\\\label{BRtrand1control}
\leq& \liminf_{\mu\to0^+} d_1(\tilde \omega_{tran}^\mu(t),\omega_{\ep_0}),
\end{align}
where $B_R=\mathbb{T}_{2\pi}\times [-R,R]$. By \eqref{limit-for-approximate solution-L2-t},  $x_1^\mu(t)\to x_1(t)$ and $y_1^\mu(t)\to y_1(t)$, we have
\begin{align}\label{BRtrand2control}
&\|\nabla\psi_{tran}(t)\|_{L^2(B_R)}=\lim_{\mu\to0^+}
\|\nabla\psi_{tran}^\mu(t)\|_{L^2(B_R)}
\leq\lim_{\mu\to0^+} d_2(\tilde \omega_{tran}^\mu(t),\omega_{\ep_0})
\end{align}
\if0
\begin{align*}
&\|\nabla\psi^\mu(t,x-x_1^\mu(t),y-y_1^\mu(t))-\nabla\psi(t,x-x_1(t),y-y_1(t))\|_{L^2(B_R)}\\
=&\|\nabla\tilde \psi^\mu(t,x-x_1^\mu(t),y-y_1^\mu(t))-\nabla\tilde \psi(t,x-x_1(t),y-y_1(t))\|_{L^2(B_R)}\\
\leq& \|\nabla\tilde \psi^\mu(t)-\nabla\tilde \psi(t)\|_{L^2(B_R)}\\
&+\|\nabla\tilde \psi(t,x-x_1^\mu(t),y-y_1^\mu(t))-\nabla\tilde \psi(t,x-x_1(t),y-y_1(t))\|_{L^2(B_R)}\to0
\end{align*}
\fi
 for any $R>0$, where
$\psi_{tran}^\mu(t)\triangleq(-\Delta)^{-1}(\tilde \omega^\mu(t,x-x_1^\mu(t),y-y_1^\mu(t))-\omega_{\ep_0})$ and $\psi_{tran}(t)\triangleq(-\Delta)^{-1}(\tilde \omega(t,x-x_1(t),y-y_1(t))-\omega_{\ep_0})$. Taking $R\to\infty$ in
\eqref{BRtrand1control}-\eqref{BRtrand2control}, up to a subsequence, we have
\begin{align*}
d(\tilde \omega(t),\omega_{\ep_0}(x+x_1(t),y+y_1(t)))=d(\tilde \omega_{tran}(t),\omega_{\ep_0})\leq \lim_{\mu\to0}d(\tilde \omega_{tran}^\mu(t),\omega_{\ep_0})\leq {1\over 2}\kappa<\kappa,
\end{align*}
where we used \eqref{onlinear orbital stability-apply1} in the second inequality.
\end{proof}
\begin{remark}
Another important approach to study nonlinear stability of the equilibria is to view the  equilibria as global minimizers of a suitable functional  and use the minimizing property (i.e. the variational approach). For the Kelvin-Stuart vortices, the functional could be chosen as the PEC functional for the perturbed vorticity
\begin{align*}
H(\tilde \omega)=\iint_{\Omega}\left({1\over 2}\tilde \omega-{1\over2}\tilde \omega\ln(-\tilde \omega)\right) dxdy-{1\over 2}\iint_\Omega(G*\tilde \omega)\tilde \omega dxdy
\end{align*}
over the constraint set $ Y_{non}$, which is defined in \eqref{def-X-non-ep}.
Direct computation gives $H'(\omega_\ep)=0$, and thus,
\begin{align}\label{H independent ep}
{d\over d\ep}H (\omega_\ep)=\langle H'(\omega_\ep),\partial_\ep\omega_\ep\rangle=0,
\end{align}
where we used  $\iint_\Omega\partial_\ep\omega_\ep dxdy=0$.
Our above  proof implies that $\omega_\ep$, $\ep\in(0,1)$, are, up to spatial translations,  local minimizers of the functional $H$, see \eqref{t control by 0-Taylor of dual functional}. Suppose that $\omega_{\ep_0}$ is a global minimizer of $H$ for some $\ep_0\in(0,1)$. Then by \eqref{H independent ep}, each member in  the whole family of equilibria $\omega_\ep$, $\ep\in(0,1)$, is a global minimizer of $H$. This also implies that $\omega_\ep$ is not an isolated global minimizer of $H$ for any fixed $\ep$, which causes difficulty in the variational approach. Note that the non-isolation of the global minimizer $\omega_\ep$ is not induced by spatial translations. Another difficulty is that the vortices $\omega_\ep$ becomes singular as $\ep\to1^-$, and thus, lack of compactness seems insufficient to ensure convergence of the minimizing sequence.
\end{remark}
\section{Numerical results}\label{Numerical Results}

% text of this chapter goes here
The numerical analysis consists of two parts. The first part is to approximate an eigenvalue with a corresponding  eigenfunction for the eigenvalue problem \eqref{elip02} in the co-periodic case, which motivates us to compute the first few eigenvalues with corresponding eigenfunctions for the $0$ mode in \eqref{eigen value-function}.
The second part shows that the number of unstable eigenvalues decreases as $\ep$ increases in the modulational case.
\if0
We use the Hermite functions to generate basis functions on $\tilde{X}_\ep$ and $H^1(\Omega)$. Then we compute approximation matrices for the operators $\tilde{A}_\ep$ and $J_{\ep\alpha} L_{\ep\alpha}(-\Delta_\alpha)$, and study the eigenvalues and eigenfunctions of the approximation matrices to simulate  the spectral results of the corresponding operators.
\fi
\if0
\subsection{Basis functions}
The Hermite polynomials $\{\hat{H}_n(y) | n \in \mathbb{N} \}$ were defined as
$$\hat{H}_n(y)=(-1)^ne^{y^2}\frac{d^n}{dy^n}e^{-y^2},\quad y \in \mathbb{R}, \quad n \in \mathbb{N}.$$
%The first three Hermite  polynomials are $\hat{H}_0(y) = 1$, $\hat{H}_1(y) = 2y$ and $\hat{H}_2(y) = 4y^2 - 2$.
Since $\{\hat{H}_n(y) | n \in \mathbb{N} \}$ are $L^2$-orthogonal under the Gaussian weight $e^{-y^2}$, i.e.,
$$\int_{\mathbb{R}} \hat{H}_{n_1}(y) \hat{H}_{n_2}(y) e^{-y^2} dy = \sqrt{\pi}2^n n! \delta_{n_1 n_2}, \quad\forall\;\; n_1, n_2 \in \mathbb{N}, $$
the Hermite functions defined as
\begin{align*}
H_n(y)= \frac{e^{-y^2/2}}{\pi^{1/4}\sqrt{2^nn!}}\hat{H}_n(y), \quad n \in \mathbb{N},
\end{align*}
are $L^2$-orthonormal.
%i.e.,
%$$\int_{\mathbb{R}} H_{n_1}(y) H_{n_2}(y) dy = \delta_{n_1n_2}, \quad\forall\;\; n_1, n_2 \in \mathbb{N}.$$
Indeed, the Hermite functions $\{H_n(y) | n \in \mathbb{N} \}$ form an orthonormal basis of $L^2(\mathbb{R})$. Moreover, it satisfies the recurrence relation
$$H''_n(y) = y^2 H_n(y) -(2n + 1)H_n(y),\quad n\in \mathbb{N}.$$
The above properties are used to simplify  computations in the following subsections.
\fi

 \subsection{An eigenfunction of the associated eigenvalue problem for the co-periodic case}\label{eigenfunction-motivation}
We simulate the eigenvalues and eigenfunctions of the operator $\tilde{A}_\ep$  by means of the spectral method in the co-periodic case.
We discretize the space $\tilde{X}_\ep$ with the following basis functions
 $$ \mathcal{B} =\left\{\psi_{n,k}(x,y) | n \in \mathbb{N}, k \in \mathbb{Z}\right\},$$
where
$$\psi_{n,k}(x,y) = \left\{ \begin{array}{cc} \frac{1}{\sqrt{2\pi}} \int_0^y H_n(\hat{y})d\hat{y}, & k = 0,  \\ \frac{1}{\sqrt{\pi}} H_n(y)\cos(kx), & k > 0, \\
\frac{1}{\sqrt{\pi}} H_n(y)\sin(kx), & k < 0,\end{array} \right.$$
$
H_n(y)= \frac{e^{-y^2/2}}{\pi^{1/4}\sqrt{2^nn!}}\hat{H}_n(y)$ and $\hat{H}_n(y)=(-1)^ne^{y^2}\frac{d^n}{dy^n}e^{-y^2}$, $n \in \mathbb{N}$,
are the Hermite functions and the Hermite polynomials, respectively.
Note that $\{H_n(y)|n \in \mathbb{N}\}$  form an orthonormal basis of $L^2(\mathbb{R})$. Moreover,
 $\{ \psi_{n,0}(y) = \frac{1}{\sqrt{2\pi}} \int_0^y H_n(\hat{y}) d\hat{y}| n\in \mathbb{N} \}$ is orthonormal in the sense that
\begin{align}\label{psi-nk-orthonormal}
(\psi_{n_1,0}, \psi_{n_2,0})_{\dot{H}^1(\Omega)} = \iint_{\Omega} \nabla \psi_{n_1,0} \cdot \nabla \psi_{n_2,0} dxdy = \delta_{n_1, n_2}.
\end{align}
 For any $\psi_{n_1, k_1},  \psi_{n_2, k_2} \in \mathcal{B}$, we have
 \begin{align*}
 \langle\tilde{A}_\ep \psi_{n_1, k_1}, \psi_{n_2, k_2}\rangle
 =& \iint_\Omega \nabla \psi_{n_1, k_1} \cdot {\nabla \psi_{n_2, k_2}} dxdy - \iint_\Omega g'(\psi_\ep)\psi_{n_1, k_1} {\psi_{n_2, k_2}} dxdy \\
 &  + \frac{1}{8\pi} \iint_\Omega g'(\psi_\ep)\psi_{n_1, k_1} dxdy \iint_\Omega g'(\psi_\ep)\psi_{n_2, k_2} dxdy.
 \end{align*}
 We  use the above  equality to find a finite dimensional matrix, which approximates the operator $\tilde{A}_\ep$, and obtain the spectral information of $\tilde{A}_\ep$ by studying the eigenvalues and eigenvectors of the approximate matrix.


The procedure to discretize the problem is summarized as follows:
\begin{enumerate}
 \item Choose a positive integer $N$.
 \item Truncate the basis $\mathcal{B}$ to $\mathcal{B}_N = \left\{\psi_{n,k}(x,y) | 0\leq n \leq2N,  -N \leq k \leq N\right\}$.
 \item Compute the $(2N + 1)^2 \times (2N+1)^2$ matrix $\mathbf{\tilde{A}}_\ep$ using
 $$(\mathbf{\tilde{A}}_\ep)_{(n_1, k_1), (n_2, k_2)} = \langle\tilde{A}_\ep \psi_{n_1, k_1}, \psi_{n_2, k_2}\rangle \text{ for } \psi_{n_1, k_1},  \psi_{n_2, k_2} \in \mathcal{B}_N.$$
 \item Calculate the eigenvalues $\lambda_i$ and eigenvectors $v_i$ of $\mathbf{\tilde{A}}_\ep$.
 \item Use the eigenvectors $v_i$ in (4) and the truncated basis $\mathcal{B}_N$ in (2) to compute the approximated eigenfunctions $f_i$ of $\tilde{A}_\ep$.
 \end{enumerate}


 We pick $N = 7$ and take different values for $\epsilon \in [0, 1)$. Then we  compute the $225\times225$ dimensional matrix $\mathbf{\tilde{A}}_\ep$ to approximate $\tilde{A}_\ep$ and calculate its eigenvalues. We summarize the first 10  eigenvalues of $\mathbf{\tilde{A}}_\ep$ in Table \ref{tbl:Tbl1}.
\begin{table}[ht]
\centering
\caption{The first 10  eigenvalues of $\mathbf{\tilde{A}}_\ep$}
\label{tbl:Tbl1}
\resizebox{0.8\textwidth}{!}{
\begin{tabular}{| l | r  r  r  r  r  r  r  r  r |}
\hline
  $\epsilon$ &    0.0 &    0.1 &    0.2 &    0.3 &    0.4 &    0.5 &    0.6 &    0.7 &    0.8 \\
\hline
$\lambda_1$ & 0.0000 & 0.0000 & 0.0000 & 0.0000 & 0.0001 & 0.0001 & 0.0002 & 0.0007 & 0.0041 \\
$\lambda_2$ & 0.0001 & 0.0001 & 0.0001 & 0.0001 & 0.0001 & 0.0002 & 0.0006 & 0.0024 & 0.0118 \\
$\lambda_3$ & 0.0001 & 0.0001 & 0.0001 & 0.0001 & 0.0001 & 0.0003 & 0.0008 & 0.0032 & 0.0169 \\
$\lambda_4$ & 0.6667 & 0.6682 & 0.6728 & 0.6807 & 0.6926 & 0.7094 & 0.7329 & 0.7662 & 0.8163 \\
$\lambda_5$ & 0.8336 & 0.8334 & 0.8329 & 0.8324 & 0.8322 & 0.8331 & 0.8361 & 0.8432 & 0.8588 \\
$\lambda_6$ & 0.9016 & 0.9018 & 0.9023 & 0.9034 & 0.9051 & 0.9078 & 0.9122 & 0.9192 & 0.9314 \\
$\lambda_7$ & 0.9367 & 0.9369 & 0.9375 & 0.9386 & 0.9404 & 0.9430 & 0.9468 & 0.9525 & 0.9612 \\
$\lambda_8$ & 0.9601 & 0.9603 & 0.9609 & 0.9620 & 0.9636 & 0.9659 & 0.9691 & 0.9733 & 0.9792 \\
$\lambda_9$ & 0.9738 & 0.9740 & 0.9745 & 0.9753 & 0.9766 & 0.9783 & 0.9806 & 0.9836 & 0.9875 \\
$\lambda_{10}$ & 0.9850 & 0.9851 & 0.9854 & 0.9860 & 0.9868 & 0.9879 & 0.9894 & 0.9912 & 0.9934 \\
\hline
\end{tabular}
}
\end{table}
\if0
\begin{figure}[ht]
    \centering
	\includegraphics[width=0.7\textwidth]{../graphs/eigenvalues_A.png}
	\caption{The first 10  eigenvalues of $\mathbf{\tilde{A}}_\ep$}
	\label{fig:3rdFig7}
\end{figure}
\fi
Even though the accuracy is affected for large $\epsilon$ values due to the singularity of the steady state at $\epsilon = 1$,
we could observe some interesting patterns from the numerical results.
\begin{itemize}
\item The eigenvalues $\lambda_i$ do not have a clear dependence on $\epsilon$.
\item For all $\epsilon$ values, $\mathbf{\tilde{A}}_\ep$ has three zero eigenvalues.
\item When $\ep = 0$, the first 3 eigenfunctions $f_1, f_2, f_3$ correspond to the three kernel functions of $\tilde{A}_0$, i.e.
$$f_1(x,y) = \tanh(y),\quad f_2(x,y) = \frac{\cos(x)}{\cosh(y)},\quad f_3(x,y) = \frac{\sin(x)}{\cosh(y)}.$$
\if0
The eigenfunctions with nonzero eigenvalues only depend on $y$ and are either odd or even as shown in the following Figure \ref{fig:8thFig}, where the change of colors represents the change of function's values in the domain.

\begin{figure}[ht]
    \centering
	\includegraphics[width=0.85\textwidth]{../graphs/eigenfuncs6.png}
	\caption{The heatmap of the first 6 eigenfunctions of $\mathbf{\tilde{A}}_\ep$ when $\ep = 0$}
	\label{fig:8thFig}
\end{figure}
\fi

\item The $4$-th eigenvalue $\lambda_4$ is a good approximation of the number $\frac 2 3$.
\item When $\epsilon = 0$, the $4$-th eigenfunction $f_4$ only depends on $y$ and has a bell shaped curve that matches the curve of $\tanh^2(y)$ perfectly after some linear transformation, see Figure \ref{fig:9thFig-co2}.
\begin{figure}[ht]
    \centering
	\includegraphics[width=0.85\textwidth]{eigen4.png}
	\caption{The 4-th eigenfunction $f_4$ of $\mathbf{\tilde{A}}_0$}
	\label{fig:9thFig-co2}
\end{figure}
\end{itemize}
The above observations  give a hint that
\begin{align}\label{eigen4}
\mathbf{\tilde{A}}_0 \vec{v}_4& = \lambda_4 \vec{v}_4 =  \frac 2 3 \vec{v}_4,\\\nonumber
v_{4,n,k}&=0\quad\text{for}\quad k\neq0\Longrightarrow
f_4=\sum\limits_{n=0}^{2N}\sum\limits_{k=-N}^Nv_{4,n,k}\psi_{n,k}=\sum\limits_{n=0}^{2N}v_{4,n,0} \psi_{n,0},
\end{align}
and $f_4$ might be $\tanh^2(y)$,
where
$\vec{v}_4=(v_{4,n,k})_{0\leq n\leq 2N,-N\leq k\leq N}$. By \eqref{psi-nk-orthonormal}, we have
$\|\vec{v}_4\|_{l^2}=\iint_{\Omega}|\nabla f_4|^2dxdy=\iint_\Omega(-\Delta f_4)f_4dxdy$.
By \eqref{eigen4}, $f_4$ approximately satisfies  $$\tilde{A}_0 f_4 = (-\Delta - g'(\psi_0)(I-P_0)) f_4 =  \frac 2 3 (-\Delta f_4),$$
which implies
$$-\Delta f_4 = 3g'(\psi_0)(I-P_0)f_4,$$
where $g'(\psi_0) = 2\sech^2(y)$.
This is exactly true when $f_4(x,y) = \tanh^2(y)$ since
$$-\Delta \tanh^2(y) = 2\sech^2(y)(3\tanh^2(y)-1) = 3g'(\psi_0)\left(\tanh^2(y) - \frac 1 3\right)$$
and $$P_0(\tanh^2(y)) = \frac{\int_0^{2\pi} \int_{-\infty}^{+\infty} g'(\psi_0) \tanh^2(y) dydx}{8\pi} = \frac1 2 \int_{-\infty}^{+\infty} \sech^2(y) \tanh^2(y)dy  = \frac 1 3.$$
By the above numerical simulation, $\tanh^2(y)$ is  an eigenfunction of the  eigenvalue $\lambda=3$ for \eqref{mode0}.  Recall that  $\tanh(y)$ is  an eigenfunction of the  eigenvalue $\lambda=1$ for \eqref{mode0}. Observing the form of these two eigenfunctions, our intuition is that all the eigenfunctions are possibly polynomials of $\tanh(y)$. This motivates us to compute
the first few eigenvalues and eigenfunctions as in \eqref{eigen value-function}, and inspires us to try
the change of variable $\gamma=\tanh(y)$ for the hyperbolic tangent shear flow. It is surprising and lucky to relate the eigenvalue problem \eqref{mode0} to the Legendre differential equations after the change of variable.

\subsection{The number of unstable modes in the modulational case}\label{The number of unstable modes in the modulational case}
In Section \ref{modulational}, we study the linear modulational instability analytically.  In this subsection, we obtain an interesting numerical phenomenon that there exists $\ep_0\in(0,1)$ such that  the number of unstable modes changes from $2$ to $1$ once $\ep$ passes through  $\ep_0$ increasingly for $\alpha={1\over 2}$ or ${1\over3}$.

To avoid solving the Poisson equation, we analyze the problem using the stream functions and solve the following generalized eigenvalue problem
 \begin{equation}\label{modeig}
 M_{\ep\alpha}\widetilde{\psi}=\sigma (-\Delta_\alpha) \widetilde{\psi}, \quad \widetilde{\psi} \in H^1(\Omega),
 \end{equation}
 where
$M_{\ep\alpha} = J_{\ep,\alpha}L_{\ep,\alpha} (-\Delta_\alpha)$, $J_{\ep,\alpha}$, $L_{\ep,\alpha}$ and $\Delta_\alpha$ are defined in \eqref{def-J-ep-al}-\eqref{nabla-alpha-Delta-alpha}.
The study of modulational instability is equivalent to the study the generalized eigenvalue problem in \eqref{modeig}. We  use spectral method to discretize this problem and study a generalized eigenvalue problem with two  approximation matrices. We take the basis
$$\tilde{\mathcal{B}} = \{\tilde{\psi}_{n,k}(x,y) | n \in \mathbb{N}, k \in \mathbb{Z}\},$$
where
$\tilde{\psi}_{n,k}(x,y) = \frac {1} {\sqrt{2\pi}}e^{ikx}H_n(y).$
We know that $\tilde{\mathcal{B}}$ is an orthornormal basis of $H^1(\Omega)$ and for any $\tilde{\psi}_{n_1,k_1}, \tilde{\psi}_{n_2,k_2} \in \tilde{\mathcal{B}}$,
\begin{align*}
\langle M_{\ep\alpha} \tilde{\psi}_{n_1,k_1} ,\tilde{\psi}_{n_2,k_2}\rangle
  =  \iint_\Omega M_{\ep\alpha} \tilde{\psi}_{n_1,k_1}(x,y) \overline{\tilde{\psi}_{n_2,k_2}(x,y)} dxdy
  \end{align*}
and
\begin{align*}
\langle-\Delta_\alpha \tilde{\psi}_{n_1,k_1} ,\tilde{\psi}_{n_2,k_2}\rangle
  = & \iint_\Omega -\Delta_\alpha \tilde{\psi}_{n_1,k_1}(x,y) \overline{\tilde{\psi}_{n_2,k_2}(x,y)} dxdy.\\
%  = & \left\{ \begin{array}{ccc} 0, & \text{ if }  & k_1 \neq k_2, \\
%  (\alpha + k_1)^2 \int_{\mathbb{R}} H_{n_1}(y) H_{n_2}(y) dy +  \int_{\mathbb{R}} H'_{n_1}(y) H'_{n_2}(y) dy, & \text{ if } &k_1 = k_2.\end{array} \right.
\end{align*}

 \subsubsection{Algorithm}
The procedure to discretize the problem is summarized as follows:
\begin{enumerate}
 \item Choose a positive integer $N$.
 \item Truncate the basis $\tilde{\mathcal{B}}$ to $\tilde{\mathcal{B}}_N = \left\{\tilde{\psi}_{n,k}(x,y) | 0\leq n \leq2N,  -N \leq k \leq N\right\}$.
 \item Compute the $(2N + 1)^2 \times (2N+1)^2$ matrices $\mathbf{M}_{\ep\alpha}$, $\mathbf{D}_{\alpha}$ with the entries
 $$(\mathbf{M}_{\ep\alpha})_{(n_1, k_1), (n_2, k_2)} = (M_{\ep\alpha} \tilde{\psi}_{n_1, k_1}, \tilde{\psi}_{n_2, k_2}) $$
 and
 $$(\mathbf{D}_{\alpha})_{(n_1, k_1), (n_2, k_2)} = (-\Delta_{\alpha} \tilde{\psi}_{n_1, k_1}, \tilde{\psi}_{n_2, k_2}) $$
 for $\tilde{\psi}_{n_1, k_1},  \tilde{\psi}_{n_2, k_2} \in \tilde{\mathcal{B}}_N.$
 \item Solve $\sigma$ from the generalized eigenvalue problem
\begin{align}\label{eigen-modu-app}
\mathbf{M}_{\ep\alpha}^* =\sigma \mathbf{D}_{\alpha}^*.
\end{align}
 \end{enumerate}
Here, $\mathbf{M}_{\ep\alpha}^*$ is the  conjugate transpose of $\mathbf{M}_{\ep\alpha}$.
 \subsubsection{Results}
 We pick $N = 7$ and take different values for $\epsilon \in (0, 1)$ and $\alpha \in (0, \frac 1 2]$. Then  we compute the $225 \times 225$ dimensional matrices $\mathbf{M}_{\ep\alpha}$, $\mathbf{D}_{\alpha}$ and calculate the generalized eigenvalues $\sigma$.
 \if0
\begin{figure}[ht]
    \centering
	\includegraphics[width=0.85\textwidth]{../graphs/multi-periodic23.png}
	\caption{Generalized eigenvalues of \eqref{eigen-modu-app} for  $\ep = 0.1$}
	\label{fig:tenthFig-mul}
\end{figure}

When $\epsilon = 0.1$, we obtain the generalized eigenvalues $\sigma$ for $\alpha = \frac 1 2$ and $\alpha = \frac 1 3$, and plot them in Figure \ref{fig:tenthFig-mul}. Since there exist two positive eigenvalues  of \eqref{eigen-modu-app}, it indicates that when $\ep = 0.1$, the steady state is linearly modulationally unstable for $\alpha = \frac 1 2$ and $\alpha = \frac 1 3$, which justifies our analytical results in Section 4.
More precisely, as explained in Remark \ref{modulational-remark},
 the number of modulational  unstable eigenvalues for  $\omega_\ep$ with $\ep\ll1$ is $2$, and our numerical results for $\ep=0.1$ coincide with the theoretical analysis.
\fi
\begin{figure}[ht]
    \centering
	\includegraphics[scale = 0.4]{double-periodic.png}
        \includegraphics[scale = 0.4]{three-periodic.png}
	\caption{Positive real parts of the generalized eigenvalues of \eqref{eigen-modu-app}  }
	\label{fig:eleventhFig}
\end{figure}


Our numerical results provide us  an interesting information.
Figure \ref{fig:eleventhFig} shows the correspondence between the positive real parts of the unstable eigenvalues and $\ep$ for  $\alpha={1\over2}, {1\over3}$. When $\alpha = \frac{1}{2}$, as $\ep$ grows from $0$ to $0.4$, there are two unstable directions with the same positive growth rates $0.186$ in the beginning, and then one of them decreases to $0$ at $\ep = 0.16$ while the other slowly increases up to $0.235$. This result compares well with the result in Figure 3 of \cite{pierrehumbert1982two}. Similarly, when $\alpha = \frac{1}{3}$, there are two unstable directions with positive growth rates. One of them decreases to $0$ at $\ep = 0.14$ and the other slowly increases up to $0.210$. This indicates that the number of unstable eigenvalues changes from $2$ to $1$ as $\ep$ grows far from $0$.
From the analytical perspective, the area of  the trapped region of the cat's eye is getting larger and the effect of  the projection term is increasing as $\ep$ grows. Thus, the value of  the quadratic form
$ b_{\alpha, 2}$ in \eqref{func-b2-alpha} increases,  which leads to a decrease in the number of negative directions of $L_{\alpha,e} |_{\overline{R(B_\alpha)}}  $ as well as the unstable eigenvalues.

If we take $\alpha$ close to $0$, then the numerical simulations could only give us one unstable eigenvalue for $\ep$ small enough. Indeed, there are exactly $2$  unstable eigenvalues in this case  by Remark \ref{modulational-remark}.
We explain why numerically there is only one unstable eigenvalue for $\ep$ small enough. Note that we use the Hermite functions as the basis of $\tilde X_\ep$, and these functions decay very fast (with a Gaussian rate $e^{-y^2/2}$) near $\pm\infty$.
As one of the negative direction of $\tilde A_{\ep,\alpha}$  is
$(1-\gamma_\ep^2)^{\alpha\over2}e^{i\alpha(\theta_\ep-x)}$ decaying like $\sech^{\alpha}(y)$ near $\pm\infty$ by Corollary
\ref{A-L-dec-e-alpha},
the eigenfunction of the unstable eigenvalue with lower growth rate might decay not so fast for $\alpha\ll1$, and our numerical simulations could only detect the low frequency part of the eigenfunctions (we pick $N=7$). If we take $N$ to be larger than 20, then the amount of computation will increase dramatically.

\section{Stability and instability of Kelvin-Stuart magnetic islands}\label{Stability and instability of magnetic islands0-sec}
Kelvin-Stuart cat's eyes are  a family of static equilibria of the planar ideal MHD equations.
 The equilibria are given by the magnetic island solutions $(\omega=0,\phi_{\ep})$, where $\phi_\ep$ is given in \eqref{Kelvin-Stuart cat's eyes-mhd-m-p}. In this section,  we prove  spectral stability and conditional  nonlinear orbital stability  for co-periodic perturbations, and coalescence instability of
 the Kelvin-Stuart magnetic islands $(\omega=0,\phi_{\ep})$.

 For the steady magnetic potential $\phi_{\ep}(x,y)=\ln \left(\frac{\cosh (y) + \epsilon \cos (x)}{\sqrt{1-\epsilon^2}} \right)$, we have
 \begin{align}\label{steady-mhd}
\phi_{\ep}=G*J^{\ep}-\ln\sqrt{1-\epsilon^2},
 \end{align}
 where $G$ is defined in \eqref{green function}.
 In fact, since
\begin{align*}
 &(G*J^{\ep})(x,y)-|y|={1\over 4\pi}\iint_{\Omega}\ln(\cosh(y-\tilde y)-\cos(x-\tilde x)){1\over 2}g'(\psi_\ep(\tilde x,\tilde y))d\tilde x d\tilde y-|y|\\
 =&
 {1\over 4\pi}\int_{-1}^1\int_0^{2\pi}\ln{\cosh(y-\tilde y)-\cos(x-\tilde x)\over e^{|y|}}d\tilde\theta_\ep d\tilde\gamma_{\ep}\to\ln{1\over2}
 \end{align*}
  and
  $\ln(\cosh(y)+\ep\cos(x))-|y|=\ln{\cosh(y)+\ep\cos(x)\over e^{|y|}}\to\ln{1\over2}$
  as $y\to\pm\infty$, we infer from $-\Delta(G*J^{\ep})=-\Delta\ln(\cosh(y)+\ep\cos(x))= J^{\ep}$ that
\begin{align*}
G*J^{\ep}(x,y)=\ln \left(\cosh (y) + \epsilon \cos (x) \right),
\end{align*}
where $\tilde\theta_\ep=\theta_\ep (\tilde x,\tilde y)$ and $\tilde \gamma_{\ep}=\gamma_{\ep}(\tilde x,\tilde y)$.
\subsection{Spectral stability for co-periodic perturbations}
We consider the co-periodic perturbations of the magnetic island solutions $(\omega=0,\phi_{\ep})$ for $\ep\in[0,1)$.
Linearizing \eqref{mhd} around $(\omega=0,\phi_{\ep})$, we have
\begin{align}\label{linearized mhd}
\left\{ \begin{array}{lll} \partial_t \phi=-\{\phi_\ep,\psi\},\\
 \partial_t \omega=-\{\phi_{\ep},(-\Delta-g'(\phi_{\ep}))\phi\}.
 \end{array} \right.
\end{align}
Unlike the linearized  2D Euler equation around the Kelvin-Stuart vortex,
the linearized equation \eqref{linearized mhd} has a different separable Hamiltonian structure
\begin{align}\label{linearized mhd-sep-hamiltonian}
\partial_t \left( \begin{array}{c} \phi \\ \omega \end{array} \right) = \left( \begin{array}{cc} 0 & D_\ep \\ -D_{\ep}' & 0 \end{array} \right)\left( \begin{array}{cc}-\Delta-g'(\phi_{\ep}) & 0 \\ 0 & (-\Delta)^{-1} \end{array} \right) \left( \begin{array}{c} \phi \\ \omega \end{array} \right),
\end{align}
where  $-\Delta-g'(\phi_{\ep}):\tilde W_{\ep}\to\tilde W_{\ep}^*$,
\begin{align*}
\tilde W_{\ep}=\left\{ \phi\in \dot{H}^1(\Omega) :\iint_\Omega g'(\phi_\ep)\phi dxdy=0 \right\},
\end{align*}
 $(-\Delta)^{-1}:\tilde Y\to\tilde Y^*$ is defined by
\begin{align}\label{tilde Y-space}(-\Delta)^{-1}\omega=G\ast\omega,\quad\omega\in \tilde Y=\left\{\omega\in L^1\cap L^3 (\Omega):\iint_{\Omega}\omega dxdy=0,y\omega\in L^1(\Omega)\right\},
\end{align}
  and $D_\ep=-\{\phi_{\ep},\cdot\}:\tilde Y^*\supset D(D_\ep)\to\tilde W_{\ep}$.
 Since $\iint_{\Omega}g'(\phi_\ep)\phi(t) dxdy$ is conserved for the linearized equation \eqref{linearized mhd},  it is reasonable to consider the perturbation of the magnetic potential  to satisfy $\iint_\Omega g'(\phi_\ep)\phi dxdy=0$ in the space $\tilde W_{\ep}$.
 Since $\omega\in L^1\cap L^3(\Omega)$ and $y\omega\in L^1(\Omega)$ for $\omega\in  \tilde Y$, by
\eqref{PE-finite} we have
$\iint_{\Omega}(G\ast\omega)\omega dxdy<\infty$. By a same argument to Lemma \ref{well-poseness-Poisson-equation-nonlinear-case},
 the Poisson equation
$-\Delta \psi = \omega\in \tilde Y$   has a  unique weak solution $\psi$ in $\tilde{X}_\ep$.
By Lemma \ref{G-ast-omega-psi-constant},
$G\ast\omega-\psi$ is a constant for $\omega\in \tilde Y$. Then $\iint_{\Omega}(G\ast\omega)\omega dxdy=\iint_{\Omega}\psi\omega dxdy=\iint_{\Omega}|\nabla\psi|^2 dxdy>0$ for $0\neq \omega\in\tilde Y$, where we used $\iint_{\Omega}\omega dxdy=0$. Thus, it is reasonable to equip
$\tilde Y$ with the inner product $(\omega_1, \omega_2) = \iint_{\Omega}(G\ast\omega_1)\omega_2 dxdy$ for $\omega_1, \omega_2 \in \tilde Y$.

Since $P_\ep\phi=0$ for $\phi\in \tilde W_\ep$, we have $-\Delta-g'(\phi_{\ep})=-\Delta-g'(\phi_{\ep})(I-P_\ep)=\tilde A_\ep: \tilde W_{\ep}\to\tilde W_{\ep}^*$, where $P_\ep$ takes the form  \eqref{P-ep}.
For any $\phi\in \tilde W_\ep$, there exist $\phi_*\in \tilde X_\ep$ and a constant $c_*$ such that $\phi-\phi_*=c_*$, and
\begin{align}\label{n-tilde A-w-x}
\langle\tilde A_\ep\phi,\phi\rangle=\langle\tilde A_\ep\phi_*,\phi_*\rangle.
\end{align}
Thus, the properties of the quadratic form $\langle\tilde A_\ep\cdot,\cdot\rangle|_{\tilde W_\ep}$ are equivalent to those of the quadratic form $\langle\tilde A_\ep\cdot,\cdot\rangle|_{\tilde X_\ep}$, which was studied in  Section \ref{co-periodic-linear}.
%In particular,
%\begin{align}\label{n-tilde A-w-x-a}
%\ker\left(\tilde A_\ep|_{\tilde W_\ep}\right)=\ker\left(\tilde A_\ep|_{\tilde X_\ep}\right)\quad \text{and}\quad
%n^-\left(\tilde A_\ep|_{\tilde W_\ep}\right)=n^-\left(\tilde A_\ep|_{\tilde X_\ep}\right).
%\end{align}

Now, we verify the assumptions $\textbf{(G1-4)}$ in Lemma \ref{indice-theorem-sep} for the separable Hamiltonian system
\eqref{linearized mhd-sep-hamiltonian}. By a similar argument as for $B_{\ep}$ $B_{\ep}'$ in
\eqref{sep-hamiltonian}, we infer that $D_\ep$ and $D_\ep'$ are densely defined and closed. This verifies $\textbf{(G1)}$. Since
\begin{align*}
\langle (-\Delta)^{-1}\omega_1,\omega_2\rangle=\iint_{\Omega}(G\ast \omega_1)\omega_2 dxdy=(\omega_1,\omega_2),
\end{align*}
we know that $(-\Delta)^{-1}$ is bounded and self-dual, $\ker((-\Delta)^{-1})=\{0\}$, $\langle(-\Delta)^{-1}\omega,\omega\rangle=\|\omega\|_{\tilde Y}^2$ for $\omega\in\tilde Y$, and thus,  $\textbf{(G2)}$ is verified.
 $\textbf{(G3-4)}$ are verified by \eqref{n-tilde A-w-x} and Corollaries \ref{kernel of  the operator tilde A0 and a decomposition of tilde X0}, \ref{kernel of  the operator tilde A-ep and a decomposition of tilde Xep}.
 By  Lemma \ref{indice-theorem-sep}, we obtain that
 \begin{align}\label{stab-criteria-co-mhd}
 (\omega=0,\phi_{\ep}) \text{ is spectrally stable if and only if } n^-\left(\tilde A_{\ep}|_{\overline{R(D_{\ep})}}\right)=0.
 \end{align}
Again by \eqref{n-tilde A-w-x} and Corollaries \ref{kernel of  the operator tilde A0 and a decomposition of tilde X0}, \ref{kernel of  the operator tilde A-ep and a decomposition of tilde Xep},  $\langle\tilde A_\ep\cdot,\cdot\rangle|_{\tilde W_\ep}\geq0$ and thus, $n^-\left(\tilde A_{\ep}|_{\overline{R(D_{\ep})}}\right)=0$  in the  co-periodic case for $\ep\in[0,1)$. This proves Theorem \ref{main result1-mhd-all1} (2).
 %which is restated as follows.
%\begin{Theorem}\label{main result1-mhd-co-periodic perturbations}
%Let $0 \leq \ep < 1$. Then
%the  magnetic island solution $(\omega=0,\phi_{\ep})$ is spectrally stable for co-periodic perturbations.
%\end{Theorem}
\subsection{Proof of coalescence instability}
In this subsection, we prove  coalescence instability  of the  magnetic island solutions $(\omega=0,\phi_{\ep})$, which means
 linear double-periodic instability of the whole family of steady states. Our proof  is based on the separable Hamiltonian structure of the linearized MHD equations and our study on linear double-periodic instability of the Kelvin-Stuart vortices  in the 2D Euler case.
%A direct consequence is  the  linear $2m\pi$-periodic instability of the magnetic island  $(\omega=0,\phi_{\ep})$ for any even $m$.
Let $\Omega_2 = \mathbb{T}_{4\pi} \times \mathbb{R}$.
The linearized equation around $(\omega=0,\phi_{\ep})$ is
\begin{align}\label{linearized mhd-sep-hamiltonian-2}
\partial_t \left( \begin{array}{c} \phi \\ \omega \end{array} \right) = \left( \begin{array}{cc} 0 & D_{\ep,2} \\ -D_{\ep,2}' & 0 \end{array} \right)\left( \begin{array}{cc}-\Delta-g'(\phi_{\ep}) & 0 \\ 0 & (-\Delta)^{-1} \end{array} \right) \left( \begin{array}{c} \phi \\ \omega \end{array} \right),
\end{align}
where  $-\Delta-g'(\phi_{\ep}):\tilde W_{\ep,2}\to\tilde W_{\ep,2}^*$,
\begin{align*}
\tilde W_{\ep,2}=\left\{ \phi \bigg| \|\nabla \phi\|_{L^2(\Omega_2)} < \infty\quad {\rm{ and }}\quad\iint_{\Omega_2} g'(\phi_\ep)\phi dxdy=0 \right\},
\end{align*}
$(-\Delta)^{-1}:\tilde Y_2\to\tilde Y_2^*$ is defined by
\begin{align*}(-\Delta)^{-1}\omega=G\ast\omega,\quad\omega\in \tilde Y_2=\left\{\omega\in L^1\cap L^3 (\Omega_2):\iint_{\Omega_2}\omega dxdy=0,y\omega\in L^1(\Omega_2)\right\},
\end{align*}
and $D_{\ep,2}=-\{\phi_{\ep},\cdot\}:\tilde Y_2^*\supset D(D_{\ep,2})\to\tilde W_{\ep,2}$. Here, $\tilde Y_2$ is equipped with the inner product $(\omega_1, \omega_2) = \iint_{\Omega_2}(G\ast\omega_1)\omega_2 dxdy$ for $\omega_1, \omega_2 \in \tilde Y_2$. Similar to \eqref{linearized mhd-sep-hamiltonian},
$\textbf{(G1-2)}$ in Lemma \ref{indice-theorem-sep} can be verified for \eqref{linearized mhd-sep-hamiltonian-2}.
%Since $\tilde X_{\ep,2}$ is compactly embedded in $L_{g'(\psi_\ep)}^2(\Omega_2)$,
Note that $-\Delta\phi-g'(\phi_{\ep})\phi=-\Delta\phi-g'(\phi_{\ep})(I-P_{\ep,2})\phi=\tilde A_{\ep,2}\phi$ due to $P_{\ep,2}\phi=0$ for $\phi\in \tilde W_{\ep,2}$.
\if0
Note that $\tilde A_{\ep,2}\psi-A_{\ep,2}\psi=g'(\psi_\ep)P_{\ep,2}\psi$ for $\psi\in\tilde X_{\ep,2}$, and the projection term $g'(\psi_\ep)P_{\ep,2}\psi$ does not vanish only for the $0$ mode, where the eigenvalue problem \eqref{eigenvalue problem for 0 mode-m} is the same one to the co-periodic case. Thus, a simple modification of the proof of Corollaries \ref{kernel of  the operator A0 and a decomposition of tilde X0} and \ref{kernel of  the operator A-ep and a decomposition of tilde Xep} implies that $n^-( A_{\ep,2})=2$, $\ker( A_{\ep,2})=3$ and $\langle  A_{\ep,2}\psi,\psi\rangle\geq C\|\psi\|_{\tilde X_{\ep,2}}^2$ for some $C>0$, where $\psi\in\tilde X_{\ep,2+}$.
\fi
By Corollaries \ref{A-L-dec-o} and \ref{A-L-dec-e}, a similar argument to \eqref{n-tilde A-w-x} implies
$n^-(\tilde A_{\ep,2}|_{\tilde W_{\ep,2}})=2$, $\ker(\tilde A_{\ep,2}|_{\tilde W_{\ep,2}})=3$ and $\langle \tilde A_{\ep,2}\phi,\phi\rangle\geq C\|\phi\|_{\tilde W_{\ep,2}}^2$ for some $C>0$, where $\phi\in\tilde W_{\ep,2+}$.
   This verifies
  $\textbf{(G3-4)}$ in Lemma \ref{indice-theorem-sep} for  \eqref{linearized mhd-sep-hamiltonian-2}.
 By  Lemma \ref{indice-theorem-sep}, we have
 \begin{align}\label{stab-criteria-coalescence-mhd}
 (\omega=0,\phi_{\ep}) \text{ is coalescence unstable if and only if } n^-\left(\tilde A_{\ep,2}|_{\overline{R(D_{\ep,2})}}\right)>0.
 \end{align}
We take the test function $\tilde{\psi}_\ep$ defined in \eqref{test-even}, where $(\theta_\ep, \gep) \in \tilde{\Omega}_{2} = \mathbb{T}_{4\pi} \times [-1, 1]$ are given in \eqref{transf1}-\eqref{transf2}. Noting that
\begin{align*}
\iint_{\Omega_2}g'(\phi_\ep)\tilde{\psi}_\ep dxdy  =2\int_{-1}^1\int_{0}^{4\pi}\cos\left(\frac{\theta_\ep}{2}\right)(1-\gamma_\ep^2)^{1\over4} d\theta_\ep d\gamma_\ep=0,
\end{align*}
we have $\tilde{\psi}_\ep\in\tilde W_{\ep,2}$.
Since $\tilde{\psi}_\ep$ is `odd' symmetrical about the point $(\pi,0)$, a similar argument to Lemma \ref{b2-even} implies that $
\tilde{\psi}_\ep\in \overline{R(D_{\ep,2})}$. It follows from \eqref{b1-even} that $\langle \tilde A_{\ep,2} \tilde{\psi}_\ep,\tilde{\psi}_\ep\rangle<0$, and thus, $n^-\left(\tilde A_{\ep,2}|_{\overline{R(D_{\ep,2})}}\right)>0$. This proves Theorem \ref{main result1-mhd-all1} (1).
%We  rewrite it here for convenience.
%\begin{Theorem}\label{main result1-mhd-double-periodic perturbations}
%Let $0 \leq \ep < 1$. Then
%the  magnetic island solution $(\omega=0,\phi_{\ep})$ is coalescence unstable. Consequently, $(\omega=0,\phi_{\ep})$ is linearly unstable for even-periodic perturbations.
%\end{Theorem}
\begin{Remark}\label{linear instability for odd periodic perturbations}
It is interesting  to prove that for an odd  $m>1$,  $(\omega=0,\phi_{\ep})$ is also linearly unstable for $2m\pi$-periodic perturbations.
We provide two potential methods to prove this conjecture. The first is based on the fact that $n^-\left(\hat{A}_{\ep,e}\right)\geq1$ due to \eqref{test-odd-neg} and \eqref{A-ep-e-psi-2-neg}, where
$\hat{A}_{\ep,e} = -\Delta - g'(\psi_\ep)(I-\hat P_{\ep,e}) : \tilde{X}_{\ep, e} \rightarrow \tilde{X}_{\ep, e}^*$ and $\hat P_{\ep,e}$  is given in \eqref{def-hat-P-ep-e}. One might try to study whether  $n^-\left(\hat{A}_{\ep,e}\right)\geq1$ implies $n^-\left(\tilde A_{\ep,m}|_{\overline{R(D_{\ep,m})}}\right)\geq1$, where $\tilde A_{\ep,m}=-\Delta-g'(\phi_{\ep})(I-P_{\ep,m}):\tilde W_{\ep,m}\to\tilde W_{\ep,m}^*$, $D_{\ep,m}=-\{\phi_{\ep},\cdot\}:\tilde Y_m^*\supset D(D_{\ep,m})\to\tilde W_{\ep,m}$, and $\tilde W_{\ep,m}$, $\tilde Y_m$ are defined similarly as $\tilde W_{\ep,2}$, $\tilde Y_2$. Another method is to use the eigenfunctions given in Theorem $\ref{sol to eigenvalue problem varepsilon=0-pde}$ to
 construct a concrete test function $\varphi_{\ep,m}$ inside $\overline{R(D_{\ep,m})}$ such that $\langle \tilde A_{\ep,m}\varphi_{\ep,m},\varphi_{\ep,m}\rangle<0$.
\end{Remark}
\subsection{Nonlinear orbital stability   for co-periodic perturbations}
%In this subsection, we consider the nonlinear stability  of the  magnetic island solutions for co-periodic perturbations.
Let $\tilde \omega$, $\tilde \psi$, $\tilde J$ and $\tilde \phi$ be  the perturbed vorticity, stream function, electrical current density and  magnetic potential, respectively. The perturbations of vorticity, stream function, electrical current density and magnetic potential are denoted by $\omega=\tilde \omega-0$,  $\psi=\tilde \psi-0$,
 $J=\tilde J-J^\ep$ and $\phi=\tilde \phi-\phi_\ep$, correspondingly. The perturbed stream function is determined by $
\tilde \psi=G*\tilde \omega$ for $\tilde \omega\in\tilde Y$.
Then $(\partial_y\tilde \psi(x,y),-\partial_x\tilde \psi(x,y))\to (0,0)$ as $y\to\pm\infty$ for $x\in\mathbb{T}_{2\pi}$, and $\vec{v}=(\partial_y\tilde \psi,-\partial_x\tilde \psi)$, where $\vec{v}$ is  the perturbed velocity field.
Since
%the magnetic field of the steady state has the asymptotic behavior  $\vec{B}^\ep(x,y)\to(\pm1,0)$ as $y\to \pm\infty$, it is reasonable to consider
 the perturbed magnetic field $\vec{B}$  satisfies $\vec{B}(x,y)\to(\pm1,0)$ as $y\to \pm\infty$ for $x\in\mathbb{T}_{2\pi}$, the electrical current density should satisfy $\iint_\Omega \tilde J dxdy=-4\pi$ and $\iint_\Omega  J dxdy=0$.

We  define the perturbed  magnetic potential by $\tilde \phi
=G*\tilde J-\ln\sqrt{1-\epsilon^2}$ for $\tilde J\in
W_{non}\triangleq \{ \tilde J \in L^1(\Omega)\cap L^3(\Omega)|
 %\tilde \omega\ln(-\tilde \omega)\in L^1(\Omega),
  y\tilde J\in L^1(\Omega),\iint_{\Omega}\tilde J dxdy=-4\pi\}.$ Similar to
  \eqref{v-mu term2}-\eqref{v-mu term1}, we have $(\partial_y\tilde \phi(x,y),-\partial_x\tilde \phi(x,y))\to (\pm1,0)$ as $y\to\pm\infty$ for $x\in\mathbb{T}_{2\pi}$. Then $\vec{B}=(\partial_y\tilde \phi,-\partial_x\tilde \phi)$.
Taking the curl of $\partial_t\vec{B}=-\text{curl}(\vec{E})$, we have $\partial_t\tilde J=-\Delta\{\tilde \psi,\tilde \phi\}$. This equation, taking convolution with $G$, gives $\partial_t(G*\tilde J)=\{\tilde \psi,G*\tilde J\}$. This implies that $\tilde \phi$ solves the equation $\partial_t\tilde \phi=\{\tilde \psi,\tilde \phi\}$.
The reason we add the constant $-\ln\sqrt{1-\epsilon^2}$ into the definition of  the perturbed  magnetic potential $\tilde \phi$ is that
 the steady states $
\phi_{\ep}=G*J^{\ep}-\ln\sqrt{1-\epsilon^2}$ in \eqref{steady-mhd} satisfy the same Liouville's equation \eqref{elip} for all $\ep\in[0,1)$. If we drop such a  constant, the function $g$ in \eqref{elip} changes and depends on $\ep$, which causes inconvenience.


Let $\hat h(s)=-{1\over2}e^{-2s}$. Then $\hat h'(\phi_\ep)=e^{-2\phi_\ep}=-g( \phi_\ep)=-J^{\ep}$, where $g(s)=-e^{-2s}$. For $\tilde \omega\in\tilde Y$ and
\begin{align}\label{def-Z-non-ep}
\tilde \phi\in \tilde Z_{non,\ep}\triangleq \{ \tilde \phi=G*\tilde J-\ln\sqrt{1-\epsilon^2}|\tilde J\in W_{non}\},
\end{align}
motivated by \cite{holm1985nonlinear}, we define the energy-Casimir (EC) functional
\begin{align}\nonumber
\hat H(\tilde \omega,\tilde \phi)=&{1\over2}\iint_{\Omega}\tilde\omega(-\Delta)^{-1}\tilde\omega dxdy+{1\over2}\iint_{\Omega}(G\ast\tilde J)\tilde J  dxdy+\iint_{\Omega}\hat h(\tilde\phi) dxdy\\\label{EC-functional-mhd}
=&{1\over2}\iint_{\Omega}(G\ast\tilde\omega)\tilde\omega dxdy+{1\over2}\iint_{\Omega}(G\ast \tilde J) \tilde J dxdy-\iint_{\Omega}{1\over2}e^{-2\tilde\phi} dxdy.
\end{align}
Similar to \eqref{PE-finite}, we have $|\iint_{\Omega}(G\ast\tilde\omega)\tilde\omega dxdy|<\infty$ and $|\iint_{\Omega}(G\ast\tilde J)\tilde J  dxdy|<\infty$.
For $\tilde \phi\in \tilde Z_{non,\ep}$, by \eqref{steady-mhd} we have $\tilde \phi-\phi_\ep=G*(\tilde J-J^{\ep})=G*J$.
 The space of perturbations of  magnetic potentials is
$ Z_{non,\ep}\triangleq\{\tilde \phi-\phi_\ep=G*J|\tilde \phi\in \tilde Z_{non,\ep}\}$. Similar to Lemmas \ref{well-poseness-Poisson-equation-nonlinear-case}-\ref{G-ast-omega-psi-constant},
there exist $\phi_*\in \tilde X_\ep$ and a constant $c_*$  such that $\phi-\phi_*=c_*$ for each $\phi=G*J\in Z_{non,\ep}$. Then for $\tilde \phi\in \tilde Z_{non,\ep}$, we have
 \begin{align*}
\iint_{\Omega}{1\over2}e^{-2\tilde\phi} dxdy=&\iint_{\Omega}{1\over2}e^{-2\phi_\ep}e^{-2\phi} dxdy={1\over4}\iint_{\Omega}g'( \phi_\ep)e^{-2(\phi_*+c_*)} dxdy
\leq C e^{C\|\phi_*\|_{\tilde{X}_\ep}^2}<\infty
\end{align*}
due to Lemma \ref{Orlicz-type inequlity-lemma} and $\phi_*\in \tilde X_\ep$. Thus, the EC functional \eqref{EC-functional-mhd} is well-defined. Then $\hat H'(0,\phi_{\ep})=-\Delta\phi_{\ep}+\hat h'(\phi_{\ep})=-\Delta\phi_{\ep}-g(\phi_{\ep})=0$ and
\begin{align}\nonumber
\hat H(\tilde \omega,\tilde \phi)-\hat H( 0,  \phi_{\ep})
=&{1\over2}\iint_{\Omega}(G\ast\omega)\omega dxdy+{1\over2}\iint_{\Omega}\left((G\ast\tilde J)\tilde J-(G\ast J^{\ep}) J^{\ep}\right)dxdy\\\nonumber
&+\iint_{\Omega}\left(\hat h(\tilde\phi)-\hat h(\phi_{\ep}) \right)dxdy\\\nonumber
=&{1\over2}\iint_{\Omega}(G\ast\omega)\omega dxdy+{1\over2}\iint_{\Omega}|\nabla\phi|^2dxdy\\\nonumber
&+\iint_{\Omega}\left(\hat h(\phi_\ep+\phi)-\hat h(\phi_\ep)-\hat h'(\phi_\ep)\phi \right)dxdy\\\nonumber
=&{1\over2}\iint_{\Omega}(G\ast\omega)\omega dxdy+\iint_{\Omega} \left(\frac 1 2 |\nabla \phi|^2 -\frac 1 4 g'(\phi_\ep)(e^{-2\phi} + 2\phi - 1)\right) dxdy\\\nonumber
=&{1\over2}\iint_{\Omega}(G\ast\omega)\omega dxdy\\\nonumber
&+\iint_{\Omega} \left(\frac 1 2 |\nabla \phi|^2 -\frac 1 4 g'(\phi_\ep)(e^{-2(\phi-P_\ep\phi)} + 2(\phi-P_\ep\phi) - 1)\right) dxdy\\\label{H-omega-phi}
&+\iint_{\Omega} \left(-\frac 1 2e^{-2\phi_\ep}(e^{-2\phi}-e^{-2(\phi-P_\ep\phi)} + 2P_\ep\phi)\right) dxdy,
\end{align}
where the expression of  $P_\ep$ is given in \eqref{P-ep}.
Define two  functionals by
\begin{align}\nonumber
S_\ep( \phi)\triangleq &
\iint_{\Omega} \left(\frac 1 2 |\nabla \phi|^2 -\frac 1 4 g'(\phi_\ep)(e^{-2(\phi-P_\ep\phi)} + 2(\phi-P_\ep\phi) - 1)\right) dxdy,\quad \phi\in \tilde X_\ep,\\\label{def-functional-S}
R_\ep( \phi)\triangleq &\iint_{\Omega} \left(-\frac 1 2e^{-2\phi_\ep}(e^{-2\phi}-e^{-2(\phi-P_\ep\phi)} + 2P_\ep\phi)\right) dxdy,\quad \phi\in Z_{non,\ep},
\end{align}
and the distance functionals by
\begin{align}\nonumber
&\hat d_{1}((\tilde\omega,\tilde \phi),(0,\phi_\ep))=\iint_{\Omega}(G\ast\omega)\omega dxdy,\quad
\hat d_{2}((\tilde\omega,\tilde \phi),(0,\phi_\ep))=\iint_{\Omega}|\nabla\phi|^2dxdy,\\\label{distance3-mhd}
&\hat d_{3}((\tilde\omega,\tilde \phi),(0,\phi_\ep))=-\iint_{\Omega}\left(\hat h(\phi_\ep+\phi)-\hat h(\phi_\ep)-\hat h'(\phi_\ep)\phi \right)dxdy,\\\label{distance-mhd}
&\hat d((\tilde\omega,\tilde \phi),(0,\phi_\ep))=\hat d_{1}((\tilde\omega,\tilde \phi),(0,\phi_\ep))+\hat d_{2}((\tilde\omega,\tilde \phi),(0,\phi_\ep))+\hat d_{3}((\tilde\omega,\tilde \phi),(0,\phi_\ep))
\end{align}
for $\tilde \omega\in \tilde Y$ and $\tilde \phi\in\tilde  Z_{non,\ep}$, where we used $e^{-2s} + 2s - 1>0$ for $s\neq 0$ to ensure that $\hat d_3$ is well-defined.
Then we study the $C^2$ regularity of   $S_\ep$ and prove that the remainder term $R_\ep$ is a high order term of the distance $\hat d$. We need the following inequalities.
\begin{lemma}\label{e-p-c-e-x}
For $\ep\in(0,1)$, $a\in\mathbb{R}$ and $p\in\mathbb{Z}^+$, we have $|P_\ep\phi|\leq C\|\phi\|_{\tilde X_\ep}$,
\begin{align*}
&\iint_\Omega g'(\phi_\ep) e^ {a|\phi-P_\ep\phi|} dxdy \leq  C e^{C(a)\left(\|\phi\|_{\tilde{X}_\ep}+\|\phi\|_{\tilde{X}_\ep}^2\right)},\\
&\iint_\Omega g'(\phi_\ep) |\phi-P_\ep\phi|^p dxdy  \leq  C(p) e^{C\left(\|\phi\|_{\tilde{X}_\ep}+\|\phi\|_{\tilde{X}_\ep}^2\right)}
\end{align*}
for $\phi \in \tilde{X}_\ep$.
\end{lemma}
\begin{proof}
$|P_\ep\phi|\leq C\|\phi\|_{\tilde X_\ep}$ follows from
\eqref{projection-controlled by-X-ep} for $\phi \in \tilde{X}_\ep$.
By Lemma  \ref{Orlicz-type inequlity-lemma}, we have
\begin{align*}
\iint_\Omega g'(\phi_\ep) e^ {a|\phi-P_\ep\phi|} dxdy \leq& e^ {|a||P_\ep\phi|} \iint_\Omega g'(\phi_\ep) e^ {|a||\phi|} dxdy\leq  Ce^ {C|a|\|\phi\|_{\tilde X_\ep}+Ca^2\|\phi\|_{\tilde{X}_\ep}^2},\\
\iint_\Omega g'(\phi_\ep) |\phi-P_\ep\phi|^p dxdy \leq& p!\iint_\Omega g'(\phi_\ep) e^{|\phi-P_\ep\phi|} dxdy \leq  Cp! e^ {C\|\phi\|_{\tilde X_\ep}+C\|\phi\|_{\tilde{X}_\ep}^2},\quad \phi \in \tilde{X}_\ep.
\end{align*}
\end{proof}
The $C^2$ regularity of   $S_\ep$ is proved as follows.
\begin{Lemma}\label{C2-mhd}
$S_\ep\in C^2(\tilde X_\ep)$, $S_\ep'(0) = 0$ and
\begin{align*}
\langle S_\ep''(0)\phi_1,\phi_2 \rangle&=  \iint_{\Omega}\left(\nabla\phi_1\cdot\nabla\phi_2- g'(\phi_\ep) (\phi_1-P_\ep\phi_1)(\phi_2-P_\ep\phi_2)\right)dxdy=\langle \tilde A_\ep \phi_1,\phi_2 \rangle
\end{align*}
for $\phi_1,\phi_2\in\tilde X_\ep$,
where   $\tilde{A}_\ep$ is defined in \eqref{tilde-A-ep-A-ep} and $\ep\in(0,1)$.
\end{Lemma}
\begin{proof}
Let $\phi\in \tilde{X}_\ep$. For  $\psi \in \tilde{X}_\ep$, by Lemmas  \ref{poincare2ep} and \ref{e-p-c-e-x} we have
\begin{align*}
|\partial_\lambda S_\ep(\phi + \lambda \psi)|_{\lambda = 0}|
= & \iint_\Omega  \left(\nabla\phi\cdot\nabla\psi+{1\over2}  g'(\phi_\ep) (e^{-2(\phi-P_\ep\phi)}-1)(\psi-P_\ep\psi) \right)dxdy\\
\leq &\|\phi\|_{\tilde{X}_\ep}\|\psi\|_{\tilde{X}_\ep}+ C \left(\iint_\Omega    g'(\phi_\ep) (e^{-4(\phi-P_\ep\phi)}-2e^{-2(\phi-P_\ep\phi)}+1)dxdy\right)^{1\over2}\|\psi \|_{\tilde{X}_\ep}\\
\leq &\left(\|\phi\|_{\tilde{X}_\ep}+ C \left(C e^{C\left(\|\phi\|_{\tilde{X}_\ep}+\|\phi\|_{\tilde{X}_\ep}^2\right)}+C\right)^{1\over2}\right)\|\psi \|_{\tilde{X}_\ep}.
\end{align*}
Thus,  $S_\ep$  is G$\hat{\text{a}}$teaux differentiable at $\phi\in  \tilde{X}_\ep$.
Let $\{\phi_n\}_{n=1}^\infty\in \tilde X_\ep$ such that $\phi_n\to\phi$ in $\tilde{X}_\ep$,
and choose $N>0$ such that $\|\phi_n\|_{\tilde X_\ep}\leq \|\phi\|_{\tilde X_\ep}+1$ for $n\geq N$.   By Lemmas  \ref{poincare2ep} and \ref{e-p-c-e-x} we have for $n\geq N$ and $\psi\in\tilde X_\ep$,
\begin{align*}
&|\partial_\lambda S_\ep(\phi_n + \lambda \psi)|_{\lambda = 0}-\partial_\lambda S_\ep(\phi + \lambda \psi)|_{\lambda = 0}| \\
= & \left|\iint_{\Omega}\left(\nabla(\phi_n-\phi)\cdot\nabla\psi +{1\over 2} g'(\phi_\ep)(e^{-2(\phi_n-P_\ep\phi_n)} - e^{-2(\phi-P_\ep\phi)})(\psi-P_\ep\psi)\right) dxdy\right|\\
\leq&\|\phi_n-\phi\|_{\tilde X_\ep}\|\psi\|_{\tilde X_\ep}\\
&+\left|\int_0^1  \iint_{\Omega} g'(\phi_\ep) e^{-2(s(\phi_n-P_\ep\phi_n) + (1-s)(\phi-P_\ep\phi))}  (\phi_n - \phi-P_\ep(\phi_n-\phi))  (\psi-P_\ep\psi) dxdyds\right| \\
\leq&\|\phi_n-\phi\|_{\tilde X_\ep}\|\psi\|_{\tilde X_\ep}\\
&+\|\phi_n-\phi\|_{\tilde X_\ep}\|\psi-P_\ep\psi\|_{L^4_{g'(\phi_\ep)}}\int_0^1 \left( \iint_{\Omega} g'(\phi_\ep) e^{-8(s(\phi_n-P_\ep\phi_n) + (1-s)(\phi-P_\ep\phi))}  dxdy\right)^{1\over4}ds\\
\leq &  \|\phi_n-\phi\|_{\tilde X_\ep}\|\psi\|_{\tilde X_\ep}\\
&+\|\phi_n-\phi\|_{\tilde X_\ep}
 C e^{C\left(\|\psi\|_{\tilde{X}_\ep}+\|\psi\|_{\tilde{X}_\ep}^2\right)}
\int_0^1e^{C\left(\|s\phi_n + (1-s)\phi\|_{\tilde X_\ep}+\|s\phi_n + (1-s)\phi\|_{\tilde X_\ep}^2\right)}ds\\
\leq &\left(\|\psi\|_{\tilde X_\ep}+C_{\|\psi\|_{\tilde X_\ep}}C_{\|\phi\|_{\tilde X_\ep}}\right)\|\phi_n-\phi\|_{\tilde X_\ep}\to 0\quad \text{as}\quad n\to\infty.
\end{align*}
Thus, $S_\ep\in C^1(\tilde X_\ep)$.
 For
$\psi \in  \tilde{X}_\ep$ and $\varphi\in\tilde{X}_\ep$, by Lemma  \ref{e-p-c-e-x} we have
\begin{align*}
&\left|\partial_\tau\partial_\lambda S_\ep(\phi + \lambda\psi+\tau\varphi)|_{\lambda =\tau= 0}\right|\\
=& \left|\iint_{\Omega}\left(\nabla\psi\cdot\nabla\varphi- g'(\phi_\ep) e^{-2(\phi-P_\ep\phi)}(\psi-P_\ep\psi)(\varphi-P_\ep\varphi)\right)dxdy\right|\\
\leq&\|\psi\|_{\tilde{X}_\ep}\|\varphi\|_{\tilde{X}_\ep}+\left(\iint_{\Omega} g'(\phi_\ep) e^{-4(\phi-P_\ep\phi)} dxdy\right)^{1\over2}
\|\psi-P_\ep\psi\|_{L_{g'(\phi_\ep)}^4}\|\varphi-P_\ep\varphi\|_{L_{g'(\phi_\ep)}^4}\\
\leq&\|\psi\|_{\tilde{X}_\ep}\|\varphi\|_{\tilde{X}_\ep}+Ce^{C\left(\|\phi\|_{\tilde{X}_\ep}+\|\psi\|_{\tilde{X}_\ep}+\|\varphi\|_{\tilde{X}_\ep}+
\|\phi\|_{\tilde{X}_\ep}^2+\|\psi\|_{\tilde{X}_\ep}^2+\|\varphi\|_{\tilde{X}_\ep}^2\right)}.
\end{align*}
Let  $\{\phi_n\}_{n=1}^\infty\in \tilde X_\ep$ be defined as above.
  For  $\psi,\varphi\in\tilde X_\ep$ and $n\geq N$, we have
\begin{align*}
&|\partial_\tau\partial_\lambda S_\ep(\phi_n + \lambda \psi+\tau\varphi)|_{\lambda =\tau= 0}-\partial_\tau\partial_\lambda S_\ep(\phi + \lambda \psi+\tau\varphi)|_{\lambda =\tau= 0}|\\
=&\left|2\int_0^1\iint_{\Omega}g'(\phi_\ep) e^{-2(s(\phi_n-P_\ep\phi_n) + (1-s)(\phi-P_\ep\phi))}(\phi_n - \phi-P_\ep(\phi_n-\phi))(\psi-P_\ep\psi)(\varphi-P_\ep\varphi) dxdyds\right|\\
\leq&C\|\phi_n-\phi\|_{\tilde X_\ep}\|\psi-P_\ep\psi\|_{L_{g'(\phi_\ep)}^6}\|\varphi-P_\ep\varphi\|_{L_{g'(\phi_\ep)}^6}\\
&\int_0^1\left(\iint_{\Omega}g'(\phi_\ep) e^{-12(s(\phi_n-P_\ep\phi_n) + (1-s)(\phi-P_\ep\phi))}dxdy\right)^{1\over6}ds\\
\leq&C\|\phi_n-\phi\|_{\tilde X_\ep}
e^{C(\|\psi\|_{\tilde X_\ep}+\|\psi\|_{\tilde X_\ep}^2)}e^{C(\|\varphi\|_{\tilde X_\ep}+\|\varphi\|_{\tilde X_\ep}^2)}
\int_0^1\left(Ce^{C(\|s\phi_n + (1-s)\phi\|_{\tilde X_\ep}+\|s\phi_n + (1-s)\phi\|_{\tilde X_\ep}^2)}\right)^{1\over6}ds\\
\leq &C_{\|\psi\|_{\tilde X_\ep}}C_{\|\varphi\|_{\tilde X_\ep}}C_{\|\phi\|_{\tilde X_\ep}}\|\phi_n-\phi\|_{\tilde X_\ep}\to 0\quad \text{as}\quad n\to\infty.
\end{align*}
Thus, $S_\ep\in C^2(\tilde X_\ep)$.
\end{proof}
Next, we estimate the remainder term $R_\ep$.
\begin{lemma} \label{remainder term R}
For $\phi\in Z_{non,\ep}$ and $\left|\iint_{\Omega}(e^{-2\tilde \phi}-e^{-2\phi_\ep})dxdy\right|<1$, we have
\begin{align}\label{remainder term-estimate}
|R_\ep(\phi)|\leq O(\hat d_{3}((\tilde\omega,\tilde \phi),(0,\phi_\ep))^2)+C\left|\iint_{\Omega}(e^{-2\tilde \phi}-e^{-2\phi_\ep})dxdy\right|
\end{align}
as $\hat d_{3}((\tilde\omega,\tilde \phi),(0,\phi_\ep))\to0$.
\end{lemma}
\begin{proof}
By \eqref{P-ep} and \eqref{distance3-mhd}, we have
\begin{align*}
P_\ep\phi={\iint_\Omega\hat h'(\phi_\ep)\phi dxdy\over4\pi}={1\over 4\pi}\left(\hat d_{3}((\tilde\omega,\tilde \phi),(0,\phi_\ep))-{1\over2}\iint_\Omega(e^{-2\tilde \phi}-e^{-2\phi_\ep})dxdy\right)
\end{align*}
for $\phi\in Z_{non,\ep}$. Then we infer from the definition \eqref{def-functional-S} of $R_\ep$ that
\begin{align*}
|R_\ep(\phi)|=&\left|-{1\over 2}\iint_{\Omega}\left(e^{-2\tilde \phi}-e^{-2(\tilde \phi-P_\ep \phi)}+2e^{-2\phi_\ep}P_\ep\phi\right)dxdy\right|\\
\leq&\left|{1\over 2}(e^{2P_\ep \phi}-1-2P_\ep\phi)\iint_{\Omega}e^{-2\phi_\ep}dxdy\right|
+\left|{1\over 2}(e^{2P_\ep\phi}-1)\iint_{\Omega}(e^{-2\tilde \phi}-e^{-2\phi_\ep})dxdy\right|\\
\leq&(P_\ep\phi)^2O(1)+|P_\ep\phi|\left|\iint_{\Omega}(e^{-2\tilde \phi}-e^{-2\phi_\ep})dxdy\right|O(1)\\
\leq&O(\hat d_{3}((\tilde\omega,\tilde \phi),(0,\phi_\ep))^2)+C\left(\iint_{\Omega}(e^{-2\tilde \phi}-e^{-2\phi_\ep})dxdy\right)^2,
\end{align*}
which gives \eqref{remainder term-estimate}.
\end{proof}
\if0
By Corollaries \ref{kernel of  the operator A0 and a decomposition of tilde X0} and \ref{kernel of  the operator A-ep and a decomposition of tilde Xep}, we have
\begin{align*}
\ker ( A_\ep)={\rm{span}}\left\{\eta_\ep(x,y), \gamma_\ep(x,y), \xi_\ep(x,y)\right\}
\end{align*}
and
\begin{align}\label{A-ep-positive-lower-bound}
\langle  A_\ep \psi,\psi\rangle \geq C_0 \| \psi\|_{\tilde X_\ep}^2, \quad \quad \psi\in \tilde X_{\ep+}=\tilde X_\ep \ominus\ker ( A_\ep)
\end{align}
for some $C_0>0$ independent of $\ep$.
\fi
Now, we  prove Theorem \ref{main result1-mhd-all}, that is,  the Kelvin-Stuart magnetic islands $(\omega=0,\phi_{\ep_0})$ are  conditionally  nonlinear orbital stable  for co-periodic perturbations, where $\ep_0\in(0,1)$.

%\begin{Theorem}\label{main result-nonlinear orbital stability-mhd}
%Assume that
 %$({\rm i})$-$({\rm iii})$ in Theorem \ref{main result1-mhd-all} hold true. For a given $\ep_0 \in (0, 1)$ and any $\kappa>0$, there exists $\delta=\delta(\ep_0,\kappa)>0$ such that if the initial data satisfies
%\eqref{initial data-mhd},
 % then for any $t\geq0$,
%\eqref{onlinear orbital stability-goal-mhd} holds true.
%\end{Theorem}
\begin{proof}By Lemma \ref{imp-vertical condition},  there exists $\delta_0(\ep_0)>0$ such that for any  $(x_0,y_0)\in\Omega$ and  $\tilde \phi$ with $\hat d_2((\tilde \omega,\tilde \phi),(0,\phi_{\ep_0}(x+x_0,y+y_0)))< \delta_0(\ep_0)$, there exist $(\tilde x_0,\tilde y_0)\in\Omega$ and $\tilde\epsilon_0\in(a(\ep_0),b(\ep_0))$, depending continuously on $\tilde\phi, x_0, y_0$, such that
\begin{align}\label{app-lemma-imp-vertical condition-mhd}
\tilde \phi\left(x-\tilde x_0,y-\tilde y_0\right)-\phi_{\tilde\ep_0}(x,y)\perp\ker \left( \tilde A_{\tilde\ep_0}\right)\quad \text{in}\quad \dot{H}^1(\Omega)
\end{align}
 and
$
|x_0-\tilde x_0|+|y_0-\tilde y_0|+|\ep_0-\tilde \ep_0|\leq C(\ep_0)\sqrt{\delta_0(\ep_0)}
$
for some $a(\ep_0)\in (0,\ep_0)$ and $b(\ep_0)\in(\ep_0,1)$.
For  $\kappa>0$, let $\delta=\delta(\ep_0,\kappa)<\min\big\{{\kappa^4\over32C_1C_2(\ep_0)^4C_3(\ep_0)^4},$ ${\delta_0(\ep_0)\over2}\big\}$, where $C_1, C_2(\ep_0), C_3(\ep_0)>1$ are  determined by \eqref{de-c1-mhd}, \eqref{I-omegaep1t0-omegaep0-mhd} and \eqref{Iomega0Iomegat1-mhd}.
For the initial data $(\tilde \omega(0)=\tilde \omega_0,\tilde \phi(0)=\tilde \phi_0)$ satisfying
\eqref{initial data-mhd},
there exists $(x_0(0),y_0(0))\in\Omega$  such that
\begin{align}\nonumber
&\hat d((\tilde\omega(0),\tilde \phi(0)),(0,\phi_{\ep_0}(x+x_0(0),y+y_0(0))))+\left|\iint_{\Omega}(e^{-2\tilde \phi(0)}-e^{-2\phi_{\ep_0}})dxdy\right|\\\label{initial data-translation-infimum}
<&\delta(\ep_0,\kappa)\leq {\kappa^4\over32C_1C_2(\ep_0)^4C_3(\ep_0)^4}.
\end{align}

For $t\geq0$, we claim that if there exists $( x_0(t),y_0(t))\in\Omega$ such that
$\hat d((\tilde\omega(t),\tilde \phi(t)),(0,\phi_{\ep_0}(x+x_0(t),y+y_0(t))))<\delta_0(\ep_0)$, then there exist $(x_1(t),y_1(t))\in\Omega$ and $\ep_1(t)\in(a(\ep_0),b(\ep_0))$ such that
\begin{align}\label{a prior estimate-mhd}\hat d((\tilde\omega(t),\tilde \phi(t)),(0,\phi_{\ep_1(t)}(x+x_1(t),y+y_1(t))))
<{\kappa^4\over16C_2(\ep_0)^4C_3(\ep_0)^4}.
\end{align}
In fact, by \eqref{app-lemma-imp-vertical condition-mhd},  there exist $(x_1(t),y_1(t))\in\Omega$ and $\ep_1(t)\in(a(\ep_0),b(\ep_0))$, depending continuously on $t$, such that
$\tilde \phi(x-x_1(t),$ $y-y_1(t))-\phi_{\ep_1(t)}(x,y)\perp\ker \left(\tilde  A_{\ep_1(t)}\right)$
in $ \dot{H}^1(\Omega)$,  $|x_0(t)-x_1(t)|+|y_0(t)-y_1(t)|+|\ep_0-\ep_1(t)|\leq C(\ep_0)\sqrt{\delta_0(\ep_0)}$ if $t>0$ and
\begin{align}\label{initial-x0x1-y0y1-ep0ep1}
|x_0(0)-x_1(0)|+|y_0(0)-y_1(0)|+|\ep_0-\ep_1(0)|\leq C(\ep_0)\sqrt{\delta(\ep_0,\kappa)}.
\end{align}
Note that
$
\langle  \tilde A_\ep \phi,\phi\rangle \geq C_0 \| \phi\|_{\tilde X_\ep}^2$ for  $\phi\in \tilde X_{\ep+}=\tilde X_\ep \ominus\ker ( \tilde A_\ep)
$, where $
\ker ( \tilde A_\ep)={\rm{span}}\left\{\eta_\ep, \gamma_\ep, \xi_\ep\right\}
$.
By taking $\delta(\ep_0,\kappa)>0$ smaller, it follows from \eqref{initial-x0x1-y0y1-ep0ep1} and \eqref{initial data-translation-infimum} that  $ \hat d((0,\phi_{\ep_0}(x+x_0(0),y+y_0(0))),(0,\phi_{\ep}(x+x_1(0),y+y_1(0))))<{\kappa^4\over32C_1C_2(\ep_0)^4C_3(\ep_0)^4}$ and
$
\hat d((\tilde \omega(0),\tilde \phi(0)),(0,\phi_{\ep}(x+x_1(0),y+y_1(0)))+\left|\iint_{\Omega}(e^{-2\tilde \phi(0)}-e^{-2\phi_{\ep_0}})dxdy\right|
\leq {\kappa^4\over16C_1C_2(\ep_0)^4C_3(\ep_0)^4}
$
for $\ep=\ep_0$ or $\ep_1(0)$.
Take $\tau\in\left(0,{1\over2}\right)$ small enough such that  $-{1\over2}\tau+(1+\tau)C_0$ $>\tau$.
By \eqref{H-omega-phi}-\eqref{def-functional-S} and Lemmas \ref{C2-mhd}-\ref{remainder term R} we have
\begin{align}\nonumber
& \hat d((\tilde \omega(0),\tilde \phi(0)),(0,\phi_{\ep_1(0)}(x+x_1(0),y+y_1(0)))\\\nonumber
\geq&\hat H(\tilde \omega(0),\tilde \phi(0))-\left(\hat H( 0, \phi_{\ep_1(0)}(x+x_1(0),y+y_1(0)))+4\pi\ln\sqrt{1-\ep_1(0)^2}\right)+4\pi\ln\sqrt{1-\ep_1(0)^2}\\\nonumber
 \geq& \hat H(\tilde \omega(t),\tilde \phi_{tran}(t))-\hat H( 0, \phi_{\ep_1(t)})-4\pi\ln\sqrt{1-\ep_1(t)^2}+4\pi\ln\sqrt{1-\ep_1(0)^2}\\\nonumber
=&{1\over2}\iint_\Omega (G*\tilde \omega(t))\tilde\omega(t) dxdy+{1\over2}\iint_\Omega (2(G*J^t)J^{\ep_1(t)}+(G*J^t)J^t) dxdy\\\nonumber
&+\iint_\Omega (\hat h(\phi_{\ep_1(t)}+\phi^t)-\hat h (\phi_{\ep_1(t)})) dxdy-4\pi\ln\sqrt{1-\ep_1(t)^2}+4\pi\ln\sqrt{1-\ep_1(0)^2}\\\nonumber
=&{1\over2}\iint_\Omega (G*\tilde \omega(t))\tilde\omega(t) dxdy+{1\over2}\iint_\Omega |\nabla\phi^t|^2 dxdy-4\pi\ln\sqrt{1-\ep_1(t)^2}+4\pi\ln\sqrt{1-\ep_1(0)^2}\\\nonumber
&+\iint_\Omega (\hat h(\phi_{\ep_1(t)}+\phi^t)-\hat h (\phi_{\ep_1(t)})-\hat h '(\phi_{\ep_1(t)})(G*J^t)) dxdy\\\nonumber
=&{1\over2}\iint_\Omega (G*\tilde \omega(t))\tilde\omega(t) dxdy+{1\over2}\iint_\Omega |\nabla\phi^t|^2 dxdy-4\pi\ln\sqrt{1-\ep_1(t)^2}+4\pi\ln\sqrt{1-\ep_1(0)^2}\\\nonumber
&+\iint_\Omega \left(\hat h(\phi_{\ep_1(t)}+\phi^t)-\hat h (\phi_{\ep_1(t)})-\hat h '(\phi_{\ep_1(t)})(\phi^t-\ln\sqrt{1-\ep_1(t)^2}+\ln\sqrt{1-\ep_0^2}) \right)dxdy\\\nonumber
=&{1\over2}\iint_\Omega (G*\tilde \omega(t))\tilde\omega(t) dxdy+{1\over2}\iint_\Omega |\nabla\phi^t|^2 dxdy-4\pi\ln\sqrt{1-\ep_0^2}+4\pi\ln\sqrt{1-\ep_1(0)^2}\\\nonumber
&+\iint_\Omega \left(\hat h(\phi_{\ep_1(t)}+\phi^t)-\hat h (\phi_{\ep_1(t)})-\hat h '(\phi_{\ep_1(t)})\phi^t \right)dxdy\\\nonumber
 =&\left({1\over2}\hat d_1+{1\over2}\hat d_2-\hat d_3\right)((\tilde \omega(t),\tilde \phi_{tran}(t)), ( 0, \phi_{\ep_1(t)}))-4\pi\ln\sqrt{1-\ep_0^2}+4\pi\ln\sqrt{1-\ep_1(0)^2}\\\nonumber
 =&{1\over2}\hat d_1((\tilde \omega(t),\tilde \phi_{tran}(t)), ( 0, \phi_{\ep_1(t)})) +
 \tau \left(\hat d_3-{1\over2}\hat d_2\right)((\tilde \omega(t),\tilde \phi_{tran}(t)), ( 0, \phi_{\ep_1(t)}))+\\\nonumber
 &(1+\tau) \left({1\over2}\hat d_2-\hat d_3\right)((\tilde \omega(t),\tilde \phi_{tran}(t)), ( 0, \phi_{\ep_1(t)}))-4\pi\ln\sqrt{1-\ep_0^2}+4\pi\ln\sqrt{1-\ep_1(0)^2} \\\nonumber
 =& \left( {1\over2}\hat d_1+\tau \left(\hat d_3-{1\over2}\hat d_2\right)\right)((\tilde \omega(t),\tilde \phi_{tran}(t)), ( 0, \phi_{\ep_1(t)}))
 + (1+\tau) S_{\ep_1(t)}( \phi^t-c_*(t))\\\nonumber
 &+(1+\tau)R_{\ep_1(t)}(\phi^t)-4\pi\ln\sqrt{1-\ep_0^2}+4\pi\ln\sqrt{1-\ep_1(0)^2}\\\nonumber
  \geq&\left( {1\over2}\hat d_1+\tau \left(\hat d_3-{1\over2}\hat d_2\right)\right)((\tilde \omega(t),\tilde \phi_{tran}(t)), ( 0, \phi_{\ep_1(t)}))+(1+\tau)\cdot
  \\\nonumber
 &\langle  \tilde A_{\ep_1(t)} (\phi^t-c_*(t)),\phi^t-c_*(t)\rangle+ o(\hat d_2((\tilde \omega(t),\tilde \phi_{tran}(t)),(0,\phi_{\ep_1(t)}))) \\\nonumber
 &-o(\hat d_{3}((\tilde \omega(t),\tilde \phi_{tran}(t)),(0,\phi_{\ep_1(t)})))-C\left|\iint_{\Omega}(e^{-2\tilde \phi_{tran}(t)}-e^{-2\phi_{\ep_1(t)}})dxdy\right|\\\nonumber
 &-4\pi\ln\sqrt{1-\ep_0^2}+4\pi\ln\sqrt{1-\ep_1(0)^2}\\\nonumber
 \geq&\left({1\over2} \hat d_1+\tau \hat d_3\right)((\tilde \omega(t),\tilde \phi_{tran}(t)), ( 0, \phi_{\ep_1(t)}))+\left(-{1\over2}\tau+(1+\tau)C_0\right)\hat d_2((\tilde \omega(t),\tilde \phi_{tran}(t)), ( 0, \phi_{\ep_1(t)}))\\\nonumber
 &
 + o(\hat d((\tilde \omega(t),\tilde \phi_{tran}(t)),(0,\phi_{\ep_1(t)})))-C\left|\iint_{\Omega}(e^{-2\tilde \phi(0)}-e^{-2\phi_{\ep_0}})dxdy\right|
 \\\nonumber
 &-4\pi\ln\sqrt{1-\ep_0^2}+4\pi\ln\sqrt{1-\ep_1(0)^2}\\\nonumber
\geq&\tau \hat d((\tilde \omega(t),\tilde \phi(t)), ( 0, \phi_{\ep_1(t)}(x+x_1(t),y+y_1(t))))\\\nonumber
&+o(\hat d((\tilde \omega(t),\tilde \phi(t)), ( 0, \phi_{\ep_1(t)}(x+x_1(t),y+y_1(t)))))-C\left|\iint_{\Omega}(e^{-2\tilde \phi(0)}-e^{-2\phi_{\ep_0}})dxdy\right|
\\\nonumber
 &-4\pi\ln\sqrt{1-\ep_0^2}+4\pi\ln\sqrt{1-\ep_1(0)^2},
\end{align}
where $\phi^t=\tilde \phi_{tran}(t)-\phi_{\ep_1(t)}$, $J^t=\tilde J_{tran}(t)-J^{\ep_1(t)}$,   $\tilde \phi_{tran}(t)=\tilde \phi(t;x-x_1(t),y-y_1(t))$, $\tilde J_{tran}(t)=\tilde J(t;x-x_1(t),y-y_1(t))$, $c_*(t)$ is chosen such that  $\phi^t-c_*(t)\in \tilde  X_{\ep_1(t)}$. Here, we used
$\tilde \phi(t)=G*\tilde J(t)-\ln\sqrt{1-\ep_0^2}$ for the initial data $\tilde \phi(0)=G*\tilde J(0)-\ln\sqrt{1-\ep_0^2}\in \tilde Z_{non,\ep_0}$,
\begin{align*}
&\tilde \phi_{tran}(t)=G*\tilde J_{tran}(t)-\ln\sqrt{1-\ep_0^2}\\
=&G*(J^{\ep_1(t)}+J^t)-\ln\sqrt{1-\ep_1(t)^2}+\ln\sqrt{1-\ep_1(t)^2}-\ln\sqrt{1-\ep_0^2}\\
=&\phi_{\ep_1(t)}+G*J^t+\ln\sqrt{1-\ep_1(t)^2}-\ln\sqrt{1-\ep_0^2},\\
\Longrightarrow\phi^t=&G*J^t+\ln\sqrt{1-\ep_1(t)^2}-\ln\sqrt{1-\ep_0^2},
\end{align*}
$S_{\ep_1(t)}(\phi^t)=S_{\ep_1(t)}(\phi^t-c_*(t))$, and $\hat H(0,\omega_\ep)+4\pi\ln\sqrt{1-\ep^2}$ is conserved for $\ep$, since
\begin{align*}
{d\over d\ep}\hat H(0,\phi_{\ep})=\iint_{\Omega}\partial_\ep(G* J^\ep)J^\ep dxdy=\iint_{\Omega}\partial_\ep(\phi_\ep+\ln\sqrt{1-\ep^2})J^\ep dxdy=-4\pi{d\over d\ep}\ln\sqrt{1-\ep^2}.
\end{align*}
Then for $\kappa>0$ sufficiently small, by assumption (ii) and  taking $\delta(\ep_0,\kappa)>0$ smaller, we have
\begin{align}\nonumber
&\hat d((\tilde \omega(t),\tilde \phi(t)), ( 0, \phi_{\ep_1(t)}(x+x_1(t),y+y_1(t))))\\\nonumber
\leq& C_1 \hat d((\tilde \omega(0),\tilde \phi(0)), ( 0, \phi_{\ep_1(0)}(x+x_1(0),y+y_1(0))))+ C_1 \left|\iint_{\Omega}(e^{-2\tilde \phi(0)}-e^{-2\phi_{\ep_0}})dxdy\right|\\\label{de-c1-mhd}
&+4\pi|\ln\sqrt{1-\ep_0^2}-\ln\sqrt{1-\ep_1(0)^2}|
< {\kappa^4\over16C_2(\ep_0)^4C_3(\ep_0)^4}
\end{align}
for some $C_1>1$.

For any $\kappa\in(0,\min\{\delta_0(\ep_0), 1\})$, suppose that \eqref{onlinear orbital stability-goal-mhd} is not true. Then there exist  $t_0>0$ and $( x_0(t),y_0(t))\in\Omega$, depending continuously on $t$, such that $  \hat d((\tilde\omega(t),\tilde \phi(t)),(0,\phi_{\ep_0}(x+x_0(t),y+y_0(t))))<\kappa<\delta_0(\ep_0)$ for $0\leq t< t_0$, and
\begin{align}\label{infimum-point-t0}
 \inf_{(x_0,y_0)\in\Omega}\hat d((\tilde\omega(t_0),\tilde \phi(t_0)),(0,\phi_{\ep_0}(x+x_0,y+y_0)))=\kappa.
\end{align}
 By \eqref{a prior estimate-mhd},
 there exist $(x_1(t),y_1(t))\in\Omega$ and $\ep_1(t)\in(a(\ep_0),b(\ep_0))$, depending continuously on $t$, such that
\begin{align}\label{d-t0-ep1-mhd}
\hat d((\tilde \omega(t),\tilde \phi(t)),( 0, \phi_{\ep_1(t)}(x+x_1(t),y+y_1(t))))<{\kappa^4\over16C_2(\ep_0)^4C_3(\ep_0)^4}<{\kappa\over2}
\end{align}
for $0\leq t\leq t_0$.
If we can prove that
$
\hat d((0,\phi_{\ep_1(t_0)}),(0,\phi_{\ep_0}))<{\kappa\over 2},
$
 then
$
\hat d((\tilde \omega(t_0),\tilde \phi(t_0)),( 0, \phi_{\ep_0}(x+x_1(t_0),y+y_1(t_0))))<\kappa,
$
which contradicts \eqref{infimum-point-t0}.

Now, we prove that $\hat d((0,\phi_{\ep_1(t_0)}),(0,\phi_{\ep_0}))<{\kappa\over 2}$.   By Lemma \ref{intOmega2}, \eqref{initial-x0x1-y0y1-ep0ep1} and taking $\delta(\ep_0,\kappa)>0$ smaller, it suffices to show that
\begin{align}\label{I-omegaep1t0-omegaep0-mhd}
\left|I\left(-e^{-2\phi_{\ep_1(t)}}\right)-I\left(-e^{-2\phi_{\ep_0}}\right)\right|<{\kappa\over C_2(\ep_0)}
\end{align}
for  some $C_2(\ep_0)>1$ large enough, where $0\leq t\leq t_0$ and $I(J)=\iint_{\Omega}(-J)^{3\over2}dxdy$. In fact,
\begin{align}\nonumber
&\hat d_3((\tilde \omega(t),\tilde \phi(t)),( 0, \phi_{\ep_1(t)}(x+x_1(t),y+y_1(t))))\\\nonumber
=&-\iint_{\Omega}\bigg(\hat h(\tilde \phi(t))-\hat h(\phi_{\ep_1(t)}(x+x_1(t),y+y_1(t)))\\\nonumber
&-\hat h'(\phi_{\ep_1(t)}(x+x_1(t),y+y_1(t)))
(\tilde \phi(t)-\phi_{\ep_1(t)}(x+x_1(t),y+y_1(t)))\bigg)dxdy\\\nonumber
=&\int_0^1\iint_\Omega2(1-r)e^{-2\phi^r(t)}\big(\tilde \phi(t)-\phi_{\ep_1(t)}(x+x_1(t),y+y_1(t))\big)^2dxdydr\\\nonumber
=&\int_0^1\iint_{\Omega}2(1-r)e^{-2\phi_{\ep_1(t)}}e^{-2r\phi^t}\left(\phi^t\right)^2dxdydr\\\nonumber
\geq&\int_0^1\iint_{\Omega}2(1-r)e^{-2\phi_{\ep_1(t)}}e^{-2\left|\phi^t\right|}\left(\phi^t\right)^2dxdydr\\\label{d1 estimates-mhd}
=&{1\over2}\iint_{\Omega}g'\left(\phi_{\ep_1(t)}\right)e^{-2\left|\phi^t\right|}\left(\phi^t\right)^2dxdy,
\end{align}
where $0\leq t\leq t_0$ and $\phi^r(t,x,y)=r\tilde \phi(t,x,y)+(1-r)\phi_{\ep_1(t)}(x+x_1(t),y+y_1(t))$ for $r\in[0,1]$.
Moreover, by Lemmas \ref{e-p-c-e-x}, \ref{poincare2ep}, \eqref{initial data-translation-infimum} and \eqref{d-t0-ep1-mhd} we have
\begin{align*}
&\iint_{\Omega}g'\left(\phi_{\ep_1(t)}\right)e^{7\left|\phi^t\right|}dxdy\\
\leq& e^{7\left|P_{\ep_1(t)}(\phi^t)\right|}\iint_{\Omega}g'\left(\phi_{\ep_1(t)}\right)e^{7\left|\phi^t-c_*(t)-P_{\ep_1(t)}(\phi^t-c_*(t))\right|}dxdy\\
\leq&Ce^{C|\iint_\Omega\hat h'(\phi_{\ep_1(t)})\phi^tdxdy|} e^{C\left(\|\phi^t\|_{\tilde{X}_\ep}+\|\phi^t\|_{\tilde{X}_\ep}^2\right)}\\
\leq &
Ce^{C\hat d_3((\tilde \omega(t),\tilde \phi(t)),( 0, \phi_{\ep_1(t)}(x+x_1(t),y+y_1(t))))+C\left|\iint_{\Omega}(e^{-2\tilde \phi(0)}-e^{-2\phi_{\ep_0}})dxdy\right|}\cdot\\
&e^{C\hat d_2((\tilde \omega(t),\tilde \phi(t)),( 0, \phi_{\ep_1(t)}(x+x_1(t),y+y_1(t))))^{1\over2}+C\hat d_2((\tilde \omega(t),\tilde \phi(t)),( 0, \phi_{\ep_1(t)}(x+x_1(t),y+y_1(t))))}\\
\leq &Ce^{C\kappa}e^{C\kappa^{1\over2}+C\kappa}\leq C,\\
&\iint_{\Omega}g'\left(\phi_{\ep_1(t)}\right)\left|\phi^t\right|^2dxdy\\
\leq&2\iint_{\Omega}g'\left(\phi_{\ep_1(t)}\right)\left|\phi^t-c_*(t)-P_{\ep_1(t)}(\phi^t-c_*(t))\right|^2dxdy+2\left|P_{\ep_1(t)}(\phi^t)\right|^2\iint_{\Omega}g'\left(\phi_{\ep_1(t)}\right)dxdy\\
\leq &\hat d_2((\tilde \omega(t),\tilde \phi(t)),( 0, \phi_{\ep_1(t)}(x+x_1(t),y+y_1(t))))\\
&+C\hat d_3((\tilde \omega(t),\tilde \phi(t)),( 0, \phi_{\ep_1(t)}(x+x_1(t),y+y_1(t))))^2+C\left|\iint_{\Omega}(e^{-2\tilde \phi(0)}-e^{-2\phi_{\ep_0}})dxdy\right|^2\leq C
\end{align*}
for $0\leq t\leq t_0$. Thus, by  \eqref{d-t0-ep1-mhd} and \eqref{d1 estimates-mhd}  we have
\begin{align}\nonumber
&\left|I\left(-e^{-2\tilde \phi(t)}\right)-I\left(-e^{-2\phi_{\ep_1(t)}}\right)\right|=\left|I\left(-e^{-2\tilde \phi(t)}\right)-I\left(-e^{-2\phi_{\ep_1(t)}(x+x_1(t),y+y_1(t))}\right)\right|\\\nonumber
=&\left|\iint_\Omega\left(e^{-3\tilde \phi(t)}-e^{-3\phi_{\ep_1(t)}(x+x_1(t),y+y_1(t))}\right)dxdy\right|\\\nonumber
=&3\bigg|\int_0^1\iint_\Omega e^{-3\phi^r(t)}\left(\tilde\phi(t)- \phi_{\ep_1(t)}(x+x_1(t),y+y_1(t)) \right)dxdydr\bigg|\\\nonumber
=&3\bigg|\int_0^1\iint_\Omega e^{-3\phi_{\ep_1(t)}}
e^{-3r\phi^t}\phi^t dxdydr\bigg|\\\nonumber
\leq&3\iint_\Omega e^{-3\phi_{\ep_1(t)}}
e^{3\left|\phi^t\right|}
\left|\phi^t \right|dxdy\\\nonumber
\leq&{3\over2}\left\|e^{-\phi_{\ep_1(t)}}\right\|_{L^\infty(\Omega)}
\iint_\Omega\left(\sqrt{2}e^{-\phi_{\ep_1(t)}}e^{{7\over2}\left|\phi^t\right|}\right)
\left(2^{1\over4}e^{-{1\over2}\phi_{\ep_1(t)}}e^{-{1\over2}\left|\phi^t\right|}\left|\phi^t\right|^{1\over2}\right)\\\nonumber
&\left(2^{1\over4}e^{-{1\over2}\phi_{\ep_1(t)}}\left|\phi^t\right|^{1\over2}\right)dxdy\\\nonumber
\leq &{3\over2}\left({1+b(\ep_0)\over1-b(\ep_0)}\right)^{1\over2}\left(\iint_{\Omega}g'\left(\phi_{\ep_1(t)}\right)e^{7\left|\phi^t\right|}dxdy\right)^{1\over2}
\left(\iint_{\Omega}g'\left(\phi_{\ep_1(t)}\right)e^{-2\left|\phi^t\right|}\left|\phi^t\right|^2dxdy\right)^{1\over4}\\\nonumber
&\left(\iint_{\Omega}g'\left(\phi_{\ep_1(t)}\right)\left|\phi^t\right|^2dxdy\right)^{1\over4}\\\nonumber
\leq&C_3(\ep_0)\hat d_3((\tilde \omega(t),\tilde \phi(t)),( 0, \phi_{\ep_1(t)}(x+x_1(t),y+y_1(t))))^{1\over4}\\\label{Iomega0Iomegat1-mhd}
< &{\kappa\over2C_2(\ep_0)},
\end{align}
where $0\leq t\leq t_0$ and we used $\left\|e^{-\phi_{\ep_1(t)}}\right\|_{L^\infty(\Omega)}\leq  \left({1+\ep_1(t)\over 1-\ep_1(t)}\right)^{1\over2}\leq \left({1+b(\ep_0)\over 1-b(\ep_0)}\right)^{1\over2}$.
Similar to \eqref{d1 estimates-mhd}-\eqref{Iomega0Iomegat1-mhd} and by the fact that $\hat d((\tilde\omega(0),\tilde \phi(0)),(0,\phi_{\ep_0}(x+x_1(0),y+y_1(0))))<{\kappa^4\over16C_1C_2(\ep_0)^4C_3(\ep_0)^4}$, we have
\begin{align}\nonumber
&\left|I\left(-e^{-2\tilde \phi(0)}\right)-I\left(-e^{-2\phi_{\ep_0}}\right)\right|=\left|I\left(-e^{-2\tilde \phi(0)}\right)-I\left(-e^{-2\phi_{\ep_0}(x+x_1(0),y+y_1(0))}\right)\right|\\\nonumber
\leq& C_3(\ep_0)\hat d_3((\tilde \omega(0),\tilde \phi(0)),( 0, \phi_{\ep_0}(x+x_1(0),y+y_1(0))))^{1\over4}\\\label{Itildeomega0Iomegaep0-mhd}
\leq& {\kappa\over2C_1^{1\over4}C_2(\ep_0)}< {\kappa\over2C_2(\ep_0)}.
\end{align}
By  \eqref{Iomega0Iomegat1-mhd}-\eqref{Itildeomega0Iomegaep0-mhd} and assumption (iii), we obtain \eqref{I-omegaep1t0-omegaep0-mhd}.
\end{proof}
\begin{appendix}
\section{Existence of weak solutions to 2D Euler equation with non-vanishing velocity at infinity}

In the Appendix, we prove the existence of weak solutions to the 2D Euler equation with vorticity  in $Y_{non}$, which is defined in \eqref{def-X-non-ep}.
Our method is motivated by Majda \cite{DiPerna-Majda87,Majda-Bertozzi02} for the region $\mathbb{R}^2$.
 At a first step, we construct an approximate solution sequence for the 2D Euler equation by smoothing the initial data. We carefully study
 the properties of the initial data of the approximate solution sequence and derive some elementary results concerning this sequence, which are useful in our nonlinear analysis in Section \ref{Sec-Nonlinear orbital stability for co-periodic perturbations}.
 Instead of the radial-energy decomposition of the velocity field in $\mathbb{R}^2$,   we use the shear-energy decomposition in $\Omega=\mathbb{T}_{2\pi}\times\mathbb{R}$ to prove the global existence of the  approximate solution sequence.
% and some alternative estimates for the approximate solution sequence to ensure their $L_{\text{loc}}^1$ convergence.
Then we prove the $L_{loc}^1\cap L_{loc}^2$ convergence of the approximate solution sequence, and construct the weak solution with the weak initial data by passing to the limit in the approximating parameter.

\subsection{Properties of the approximate initial data}

The definitions of a  weak solution and an approximate solution sequence for the 2D Euler equation are given as follows.

\begin{definition}
[Weak solution] A velocity field $\vec{u}(t,x,y)$ with initial data $\vec{u}_0$ is a weak solution of the 2D Euler equation if

$(\rm{i})$ $\vec{u}\in L^1(\Omega_{R,T})$ for any $T, R>0$,

$(\rm{ii})$ $u_iu_j\in L^1(\Omega_{R,T})$ for $i,j=1,2$,

$(\rm{iii})$ $\text{div}(\vec{u})=0$ in the sense of distributions, i.e. $\iint_{\Omega}\nabla\varphi\cdot\vec{u} dxdy=0$ for any $\varphi\in C([0,T],C_0^1(\Omega))$,

$(\rm{iv})$ for any $\vec{\Phi}=(\Phi_1,\Phi_2)\in C^1([0,T], C_0^1(\Omega))$ with $\text{div}(\vec{\Phi})=0$ in the sense of distributions,
\begin{align*}
\iint_{\Omega} (\vec{\Phi}\cdot\vec{u})(t,x,y)|_{t=0}^T dxdy=\int_0^T\iint_{\Omega}\left(\partial_t\vec{\Phi}\cdot \vec{u}+(\vec{u}\cdot\nabla)\vec{\Phi}\cdot\vec{u}\right)dxdydt,
\end{align*}
where $\Omega_{R,T}=[0,T]\times B_R$ and $B_R=\{x\in\mathbb{T}_{2\pi},y\in[-R,R]\}$.
\end{definition}
\begin{definition} [Approximate solution sequence for the 2D Euler equation]
\label{Approximate solution sequence for the 2D Euler equation} A sequence $\{\vec{u}^\mu\}$  is an approximate solution sequence for the 2D Euler equation if

$(\rm{i})$  $\vec{u}^\mu\in C([0,T],L_{\text{loc}}^2(\Omega))$, and   $\max_{0\leq t\leq T}\iint_{B_R}|\vec{u}^\mu(t,x,y)|^2dxdy\leq C(T, R) $ independent of $\mu$ for any $T, R>0$,

$(\rm{ii})$ $\text{div}(\vec{u}^\mu)=0$ in the sense of distributions,

$(\rm{iii})$
$\lim_{\mu\to0}\int_0^T\iint_\Omega\left(\partial_t\vec{\Phi}\cdot \vec{u}^\mu+(\vec{u}^\mu\cdot\nabla)\vec{\Phi}\cdot\vec{u}^\mu\right)dxdydt=0$ for any $\vec{\Phi}\in C_0^\infty([0,T]\times\Omega)$ with $div(\vec{\Phi})=0$.
\\
The approximate solution sequence  $\{\vec{u}^\mu\}$ is said to have $L^1$ vorticity control if, in addition,

 $(\rm{iv})$ $\max_{0\leq t\leq T}\iint_{\Omega}|\omega^\mu(t,x,y)|dxdy<C(T)$ for any $T>0$, where $\omega^\mu=\curl(\vec{u}^\mu)$.\\
The approximate solution sequence  $\{\vec{u}^\mu\}$ with $L^1$ vorticity control  is said to have $L^q$ vorticity control ($q>1$) if, in addition,

$(\rm{v})$ $\max_{0\leq t\leq T}\iint_{\Omega}|\omega^\mu(t,x,y)|^qdxdy<C(T)$ for any $T>0$.
\end{definition}

\begin{remark}
An approximate solution sequence  $\{\vec{u}^\mu\}$  for the 2D Euler equation satisfies
\begin{align*}
\|\varphi\vec{u}^\mu(t_1)-\varphi\vec{u}^\mu(t_2)\|_{H_{\text{loc}}^{-L}(\Omega)}\leq C|t_1-t_2|
\end{align*}
for $0\leq t_1,t_2\leq T$, $L>0$ and $\varphi\in C_0^\infty(\Omega)$, i.e. $\{\varphi\vec{u}^\mu\}$ is uniformly bounded in $Lip([0,T],$ $H_{\text{loc}}^{-L}(\Omega))$.
\end{remark}


To construct an approximate solution sequence $\{\vec{v}^{\mu}\}$ for the 2D Euler equation, we decompose the initial vorticity $\tilde\omega_0\in  Y_{non}$ into the shear part and the non-shear part:
\begin{align}\label{shear-energy decomposition}\tilde\omega_0(x,y)=\tilde\omega_{0,0}(y)+ \tilde\omega_{0,\neq0}(x,y),
 \end{align}
 where $\tilde\omega_{0,\neq0}(x,y)=\sum_{j\neq0}e^{ijx}\tilde\omega_{0,j}(y)$. Then $\iint_{\Omega}\tilde\omega_0 dxdy=2\pi\int_{-\infty}^\infty\tilde\omega_{0,0}dy =-4\pi$ and $\iint_{\Omega}\tilde\omega_{0,\neq0}$ $dxdy=0$.
By \eqref{green function}, we have $\tilde\psi_{0,\neq0}=G\ast \tilde\omega_{0,\neq0}$ solves $-\Delta\phi=\tilde\omega_{0,\neq0}$, and the non-shear initial velocity is defined by
 $\vec{v}_{0,\neq0}=\nabla^{\bot}\tilde \psi_{0,\neq0}=K\ast \tilde\omega_{0,\neq0}$, where \begin{align*}
K=\nabla^{\bot}G={1\over4\pi}\left({-\sinh(y)\over \cosh(y)-\cos(x)},{\sin(x)\over \cosh(y)-\cos(x)}\right).
\end{align*}
Since $\cosh(y)=1+{y^2\over2}+o(y^2)$ and $\cos(x)=1-{x^2\over2}+o(x^2)$, we have
\begin{align}\label{K-near-0}
|K(x,y)|\sqrt{x^2+y^2}={1\over4\pi}\sqrt{\cosh(y)+\cos(x)\over \cosh(y)-\cos(x)}\sqrt{x^2+y^2}\to{1\over2\pi}
\end{align}
as $(x,y)\to(0,0)$. On the other hand,
 \begin{align}\label{K-near-pm-infty}
 K(x,y)\to\left(\mp{1\over 4\pi},0\right) \text{ with exponential rate}
 \end{align}
 as $y\to\pm\infty$ uniformly for $x\in \mathbb{T}_{2\pi}$.


\eqref{shear-energy decomposition}  gives a shear-energy decomposition in the sense that $\vec{v}_{0,\neq0}=K*\tilde\omega_{0,\neq0}\in L^2(\Omega)$. In fact,
let
 \begin{align}\nonumber
 \rho\in C_0^\infty(\mathbb{R})& \text{ with }\rho(y)=1 \text{ for } |y|\leq 1, \rho(y)=0 \text{ for } |y|>2,\\\nonumber
 \rho_s(x,y)&=\rho\left({y\over s}\right) \text{ for }(x,y)\in \Omega \text{ and }s>0, \\\nonumber
(1-\rho_{s})_{>0}&\equiv(1-\rho_{s}) \text{ for } y>0 \text{ and } (1-\rho_{s})_{>0}\equiv0 \text{ for } y\leq0,\\\label{def-cut off functions}
 (1-\rho_{s})_{<0}&\equiv(1-\rho_{s}) \text{ for } y<0 \text{ and } (1-\rho_{s})_{<0}\equiv0 \text{ for } y\geq0.
 \end{align}
  By Young's inequality, we have
\begin{align*}
\|\vec{v}_{0,\neq0}\|_{L^2(\Omega)}\leq& \|(\rho_1K)\ast\tilde\omega_{0,\neq0}\|_{L^2(\Omega)}+\left\|\left((1-\rho_1)_{>0}\left(K+\left({1\over 4\pi},0\right)\right)\right)\ast\tilde\omega_{0,\neq0}\right\|_{L^2(\Omega)}\\
&+\left\|\left((1-\rho_1)_{<0}\left(K-\left({1\over 4\pi},0\right)\right)\right)\ast\tilde\omega_{0,\neq0}\right\|_{L^2(\Omega)}\\
\leq&\bigg( \|\rho_1K\|_{L^1(\Omega)}+\left\|(1-\rho_1)_{>0}\left(K+\left({1\over 4\pi},0\right)\right)\right\|_{L^1(\Omega)}\\
&+\left\|(1-\rho_1)_{<0}\left(K-\left({1\over 4\pi},0\right)\right)\right\|_{L^1(\Omega)}\bigg)\|\tilde\omega_{0,\neq0}\|_{L^2(\Omega)}\leq C\|\tilde\omega_{0}\|_{L^2(\Omega)},
\end{align*}
 where we used \eqref{K-near-pm-infty}, $(1-\rho_1)_{>0}\ast\tilde\omega_{0,\neq0}=0$ and $(1-\rho_1)_{<0}\ast\tilde\omega_{0,\neq0}=0$.

For $\tilde\omega_0\in  Y_{non}$ and $\mu>0$,
we extend $\tilde\omega_0$ from $\Omega$ to $\mathbb{R}^2$ by setting $\tilde\omega_0(x,y)=\tilde \omega_0(x-2k\pi,y)$ for $(x,y)\in[2k\pi,(2k+2)\pi)\times \mathbb{R}$, where $k\in\mathbb{Z}$ and $k\neq0$.
Then we
 define the initial data of the approximate solution sequence by
\begin{align}\label{tilde-omega0-kappa-def}
\tilde\omega_{0}^{\mu}(x,y)=(\hat J_{\mu}\star\tilde\omega_{0})(x,y)
\end{align}
 for $(x,y)\in\Omega$ and $\mu\in(0,1)$,
 where
 \begin{align}\label{convolution-R2-def}
 (\hat J_{\mu}\star\tilde\omega_{0})(x,y)\triangleq\iint_{\mathbb{R}^2}\hat J_{\mu}(x-\tilde x,y-\tilde y)\tilde\omega_{0}(\tilde x,\tilde y)d\tilde x d\tilde y,
 \end{align}
  $\hat J_{\mu}(x,y)=\mu^{-2}\hat J\left({x\over \mu},{y\over \mu}\right)$, $\hat J\in C_0^\infty(\mathbb{R}^2)$ satisfies that $\hat J\geq0$, $\hat J(x,y)=0$ if $x^2+y^2\geq1$ and $\iint_{\mathbb{R}^2}\hat J(x,y) dxdy=1$.
 Here, we use the notation $\star$ to avoid the confusion with the usual convolution  $*$ on $\Omega$.
Note that $\hat J_{\mu}(x,y)=0$ if $\sqrt{x^2+y^2}\geq\mu$ and $\iint_{\mathbb{R}^2}\hat J_{\mu}(x,y)dxdy=1$. Moreover, $\hat J_{\mu}\star\varpi\in C^{\infty}(\mathbb{R}^2)$ if $\varpi\in L_{loc}^1(\Omega)$.
To study  the inheritance and convergence  of the  approximate  initial data $\tilde\omega_{0}^{\mu}$,
we give some basic properties of $\hat J_{\mu}\star\varpi$, which are elementary to the proof of  Theorem \ref{main result4-nonlinear orbital stability}.

\begin{lemma} \label{tilde-omega0-kappa-properties}
Let $\mu>0$ and $\varpi\in L_{loc}^1(\Omega)$.

$(1)$ $\hat J_{\mu}\star\varpi$ is $2\pi$-periodic in $x$.

$(2)$ If $\varpi<0$ on $\Omega$, then $\hat J_{\mu}\star\varpi<0$ on $\Omega$.

$(3)$ If $\iint_{\Omega}\varpi dxdy=c$, then $\iint_{\Omega}\hat J_{\mu}\star\varpi dxdy=c$.

$(4)$ If $\varpi\in L^p(\Omega)$ for $1\leq p<\infty$, then $\hat J_\mu\star\varpi\in L^p(\Omega)$, $\|\hat J_\mu\star\varpi\|_{L^p(\Omega)}\leq \|\varpi\|_{L^p(\Omega)}$ and $\hat J_\mu\star\varpi \to\varpi$ in ${L^p(\Omega)}$.


$(5)$ If $\varpi\in L^2(\Omega)$, then $\|\hat J_{\mu}\star\varpi\|_{H^q(\Omega)}\leq C(\mu,q)\|\varpi\|_{L^2(\Omega)}$ and $\|D^q\hat J_{\mu}\star\varpi\|_{L^\infty(\Omega)}=\|\hat J_{\mu}\star D^q\varpi\|_{L^\infty(\Omega)}\leq C(\mu,q)\|\varpi\|_{L^2(\Omega)}$ for $q\in\mathbb{Z}^+\cup\{0\}$.

$(6)$ If $\varpi, y\varpi\in L^1(\Omega)$, then $y(\hat J_{\mu}\star\varpi)\in L^1(\Omega)$ and $y(\hat J_{\mu}\star\varpi)\to y\varpi$ in $L^1(\Omega)$.

$(7)$ If $\varpi, y\varpi\in L^1(\Omega)$, then $\psi_\ep\varpi, \psi_\ep(\hat J_{\mu}\star\varpi)\in L^1(\Omega)$ and $\psi_\ep(\hat J_{\mu}\star\varpi)\to \psi_\ep\varpi$ in $L^1(\Omega)$ for $\ep\in[0,1)$.

$(8)$ If $\varpi\in Y_{non}$, then $\hat J_{\mu}\star\varpi\in Y_{non}$, $-\varpi\ln(-\varpi),-(\hat J_{\mu}\star\varpi)\ln(-(\hat J_{\mu}\star\varpi))\in L^1(\Omega)$ and
\begin{align}\label{xlnx-convergence}
-(\hat J_{\mu}\star\varpi)\ln(-(\hat J_{\mu}\star\varpi))\to-\varpi\ln(-\varpi)\quad \text{in} \quad L^1(\Omega),
\end{align}
where $Y_{non}$ is defined in \eqref{def-X-non-ep}.
\end{lemma}
\begin{proof}
We extend $\varpi$ from $\Omega$ to $\mathbb{R}^2$ as above.
Since
\begin{align*}
(\hat J_{\mu}\star\varpi)(x,y)=&\iint_{\mathbb{R}^2}\hat J_\mu(\tilde x,\tilde y)\varpi(x-\tilde x,y-\tilde y)d\tilde x d\tilde y=\iint_{\mathbb{R}^2}\hat J_\mu(\tilde x,\tilde y)\varpi(x+2\pi-\tilde x,y-\tilde y)d\tilde x d\tilde y\\
=&\hat J_{\mu}\star\varpi(x+2\pi,y)
\end{align*}
for $(x,y)\in\mathbb{R}^2$, (1) holds true. (2) is trivially verified.

 (3) follows from
 \begin{align*}
 \iint_\Omega\hat J_{\mu}\star\varpi dxdy=\iint_{\mathbb{R}^2}\hat J_\mu(\tilde x,\tilde y)\left(\iint_\Omega\varpi(x-\tilde x,y-\tilde y) dxdy\right)d\tilde x d\tilde y=c\iint_{\mathbb{R}^2}\hat J_\mu(\tilde x,\tilde y)d\tilde x d\tilde y=c.
 \end{align*}

Next, we prove (4). For $1<p<\infty$,
\begin{align}\nonumber
|(\hat J_{\mu}\star\varpi)(x,y)|\leq& \left(\iint_{\mathbb{R}^2}\hat  J_{\mu}(\tilde x,\tilde y)d\tilde x d\tilde y\right)^{1\over p'}\left(\iint_{\mathbb{R}^2}\hat J_{\mu}(\tilde x,\tilde y)|\varpi(x-\tilde x,x-\tilde y)|^p d\tilde xd\tilde y\right)^{1\over p}\\\label{hat-J-kappa-star-omega-estimate}
=&\left(\iint_{\mathbb{R}^2}\hat J_{\mu}(\tilde x,\tilde y)|\varpi(x-\tilde x,x-\tilde y)|^p d\tilde xd\tilde y\right)^{1\over p},
\end{align}
where $p'={p\over p-1}$. Then
\begin{align}\nonumber
\|\hat J_{\mu}\star\varpi\|_{L^p(\Omega)}^p\leq& \iint_\Omega\iint_{\mathbb{R}^2} \hat  J_{\mu}(\tilde x,\tilde y)|\varpi(x-\tilde x,y-\tilde y)|^p d\tilde xd\tilde y dxdy\\\label{J-omega-Lp}
=&\iint_{\mathbb{R}^2} \hat J_{\mu}(\tilde x,\tilde y)d\tilde xd\tilde y \iint_\Omega|\varpi(x-\tilde x,y-\tilde y)|^p  dxdy=\|\varpi\|_{L^p(\Omega)}^p.
\end{align}
For $p=1$, $\eqref{J-omega-Lp}$ follows directly from the definition of $\hat J_{\mu}\star\varpi$.
Let $\delta>0$ and $1\leq p<\infty$.
Choose $\varpi_1\in C_0(\Omega)$ such that $\|\varpi-\varpi_1\|_{L^p(\Omega)}<{\delta\over 3}$. By \eqref{J-omega-Lp}, we have $\|\hat J_{\mu}\star\varpi-\hat J_{\mu}\star\varpi_1\|_{L^p(\Omega)}<{\delta\over 3}$. Since $|\hat J_{\mu}\star\varpi_1(x,y)-\varpi_1(x,y)|\leq \sup_{\sqrt{(x-\tilde x)^2+(y-\tilde y)^2}\leq \mu}|\varpi_1(\tilde x,\tilde y)-\varpi_1(x,y)|$, $\varpi_1$ is uniformly continuous on $\Omega$ and $\text{supp}(\varpi_1)$ is compact, we have $\|\hat J_{\mu}\star\varpi_1-\varpi_1\|_{L^p(\Omega)}\leq {\delta\over 3}$ for $\mu$ sufficiently small. Thus, $\|\hat J_{\mu}\star\varpi-\varpi\|_{L^p(\Omega)}\leq \delta$.

To  prove (5), we
 denote $D^j\hat J=\hat J^j$ for  $0\leq j\leq q$. Since
\begin{align*}
(D^j\hat J_{\mu}\star\varpi)(x,y)=\mu^{-j-2}\iint_{\mathbb{R}^2}\hat J^j\left({x-\tilde x\over \mu},{y-\tilde y\over \mu}\right)\varpi(\tilde x,\tilde y)d\tilde x d\tilde y,
\end{align*}
we have
\begin{align}\nonumber
|(D^j\hat J_{\mu}\star\varpi)(x,y)|^2
\leq &\mu^{-2j}\left(\mu^{-2}\iint_{\mathbb{R}^2}\hat J^j\left({x-\tilde x\over \mu},{y-\tilde y\over \mu}\right)d\tilde x d\tilde y\right)\\\nonumber
&\left(\mu^{-2}\iint_{\mathbb{R}^2}\hat J^j\left({x-\tilde x\over \mu},{y-\tilde y\over \mu}\right)\varpi(\tilde x,\tilde y)^2d\tilde xd\tilde y\right)\\\label{DjhatJkappaomegaxy}
\leq &{C_j\over \mu^{2j}}\mu^{-2}\iint_{\mathbb{R}^2}\hat J^j\left({x-\tilde x\over \mu},{y-\tilde y\over \mu}\right)\varpi(\tilde x,\tilde y)^2d\tilde xd\tilde y.
\end{align}
Thus,
\begin{align*}
\sum_{0\leq j\leq q}\|D^j\hat J_{\mu}\star\varpi\|_{L^2(\Omega)}^2\leq &\sum_{0\leq j\leq q}{C_j\over \mu^{2j}}\mu^{-2}\iint_{\mathbb{R}^2}\hat J^j\left({\tilde x\over \mu},{\tilde y\over \mu}\right)\left(\iint_\Omega\varpi(x-\tilde x,y-\tilde y)^2dxdy\right)d\tilde x d\tilde y\\
\leq &\sum_{0\leq j\leq q}{C_j\over \mu^{2j}}\|\varpi\|_{L^2(\Omega)}^2\leq C(\mu,q)\|\varpi\|_{L^2(\Omega)}^2.
\end{align*}
Since $\hat J^{q}\left({x-\tilde x\over \mu},{y-\tilde y\over \mu}\right)=0$ for $\sqrt{(x-\tilde x)^2+(y-\tilde y)^2}\geq\mu$ and $\hat J^{q}\in C_0^\infty(\mathbb{R}^2)$, by \eqref{DjhatJkappaomegaxy} for $j=q$ we have
$|(D^q\hat J_{\mu}\star\varpi)(x,y)|
\leq C(\mu,q)\|\varpi\|_{L^2(\Omega)}$ for any $(x,y)\in\Omega$ and $\mu>0$ sufficiently small.

Then we prove (6). Noting that
\begin{align}\nonumber
&\|y(\hat J_{\mu}\star\varpi)\|_{L^1(\Omega)}\leq\iint_{\mathbb{R}^2}\hat J_\mu(\tilde x,\tilde y)\iint_{\Omega}|y\varpi(x-\tilde x,y-\tilde y)|dxdyd\tilde x d\tilde y\\\nonumber
\leq&\iint_{\mathbb{R}^2}\hat J_\mu(\tilde x,\tilde y)\iint_{\Omega}(|y-\tilde y|+|\tilde y|) |\varpi(x-\tilde x,y-\tilde y)|dxdyd\tilde x d\tilde y\\\nonumber
\leq&\|y\varpi\|_{L^1(\Omega)}+\|\varpi\|_{L^1(\Omega)}\iint_{\mathbb{R}^2}\hat J_\mu(\tilde x,\tilde y)|\tilde y|d\tilde x d\tilde y,
\end{align}
we have $y(\hat J_{\mu}\star\varpi)\in L^1(\Omega)$. To prove that $y(\hat J_{\mu}\star\varpi)\to y\varpi$ in $L^1(\Omega)$, it suffices to show that
$\|y(\hat J_{\mu}\star\varpi)-\hat J_{\mu}\star(y\varpi)\|_{L^1(\Omega)}\to0$ by (4). In fact,
\begin{align*}
&\|y(\hat J_{\mu}\star\varpi)-\hat J_{\mu}\star(y\varpi)\|_{L^1(\Omega)}\leq\iint_{\mathbb{R}^2}\hat J_\mu(\tilde x,\tilde y)|\tilde y|\iint_{\Omega} |\varpi(x-\tilde x,y-\tilde y)|dxdyd\tilde xd\tilde y\\
=&\|\varpi\|_{L^1(\Omega)}\iint_{x^2+y^2\leq1}\hat J(x,y)\mu|y|dxdy\to0.
\end{align*}

Now, we prove (7). Direct computation gives
\begin{align}\nonumber
&\|\psi_\ep\varpi\|_{L^1(\Omega)}=\|(G*\omega_\ep) \varpi\|_{L^1(\Omega)}+C\| \varpi\|_{L^1(\Omega)}\\\nonumber
\leq&\|G_1\|_{L^2(\Omega)}\|\omega_\ep\|_{L^2(\Omega)}\|\varpi\|_{L^1(\Omega)}
+C\|\omega_\ep\|_{L^1(\Omega)}\|y\varpi\|_{L^1(\Omega)}\\\label{psi-tilde-omegaL1}
&+C\|y\omega_\ep\|_{L^1(\Omega)}\|\varpi\|_{L^1(\Omega)}+C\| \varpi\|_{L^1(\Omega)}<\infty.
\end{align}
By (4) and (6), $\hat J_{\mu}\star\varpi,y(\hat J_{\mu}\star\varpi)\in L^1(\Omega)$, and thus, $\psi_\ep(\hat J_{\mu}\star\varpi)\in L^1(\Omega)$. It follows again from (4) and (6) that $\hat J_{\mu}\star\varpi\to\varpi$ and $y(\hat J_{\mu}\star\varpi)\to y\varpi$ in $L^1(\Omega)$. Then
\begin{align*}
&\|\psi_\ep(\hat J_{\mu}\star\varpi-\varpi)\|_{L^1(\Omega)}\\
\leq&\|G_1\|_{L^2(\Omega)}\|\omega_\ep\|_{L^2(\Omega)}\|\hat J_{\mu}\star\varpi-\varpi\|_{L^1(\Omega)}+C\|\omega_\ep\|_{L^1(\Omega)}
\|y (\hat J_{\mu}\star\varpi-\varpi)\|_{L^1(\Omega)}\\
&+C\|y\omega_\ep\|_{L^1(\Omega)}\|\hat J_{\mu}\star\varpi-\varpi\|_{L^1(\Omega)}+C\|\hat J_{\mu}\star\varpi-\varpi\|_{L^1(\Omega)}\to 0.
\end{align*}

Finally, we prove (8).
If $-\varpi\geq1$, then $0\leq-\varpi\ln(-\varpi)\leq \varpi^2$ since $0\leq\ln (s)\leq s$ for $s\geq1$. If $0<-\varpi<1$, then $0\leq \int_0^1{(1-r)(\varpi-\omega_\ep)^2\over -2\varpi^r}dr={1\over2}\varpi-{1\over2}\varpi\ln(-\varpi)-{1\over2}\omega_\ep-\psi_\ep\varpi$, and thus, $0<\varpi\ln(-\varpi)\leq \varpi-\omega_\ep-2\psi_\ep\varpi$, where $\varpi^r=r\varpi+(1-r)\omega_\ep$.  This implies
\begin{align}\label{omega-ln-omega}|\varpi\ln(-\varpi)|\leq \varpi^2+|\varpi|+|\omega_\ep|+2|\psi_\ep\varpi|
\end{align}
for all $(x,y)\in\Omega$. By \eqref{psi-tilde-omegaL1}, we have $\psi_\ep\varpi\in L^1(\Omega)$. This, along with $\varpi\in L^1\cap L^2(\Omega)$, yields $\varpi\ln(-\varpi)\in L^1(\Omega)$.
Since $\varpi\in Y_{non}$, by  (1)-(4) and (6) we have $\hat J_{\mu}\star\varpi\in Y_{non}$. Thus,
$-(\hat J_{\mu}\star\varpi)\ln(-(\hat J_{\mu}\star\varpi))\in L^1(\Omega)$. Similar to \eqref{omega-ln-omega}, we have $|(\hat J_{\mu}\star\varpi)\ln(-(\hat J_{\mu}\star\varpi))|\leq (\hat J_{\mu}\star\varpi)^2+|(\hat J_{\mu}\star\varpi)|+|\omega_\ep|+2|\psi_\ep(\hat J_{\mu}\star\varpi)|$ for all $(x,y)\in\Omega$.
Let $B_R^c=\Omega\setminus B_R$. Then
\begin{align}\nonumber
&\iint_{B_R^c}|(-\hat J_{\mu}\star\varpi)\ln(-(\hat J_{\mu}\star\varpi))-(-\varpi)\ln(-\varpi)|dxdy\\\nonumber
\leq&
\iint_{B_R^c}\bigg((\hat J_{\mu}\star\varpi)^2+|\hat J_{\mu}\star\varpi|+|\omega_\ep|+2|\psi_\ep(\hat J_{\mu}\star\varpi)|\\\label{BRc1}
&+\varpi^2+|\varpi|+|\omega_\ep|+2|\psi_\ep\varpi|\bigg)dxdy
\end{align}
for $R>1$. By \eqref{hat-J-kappa-star-omega-estimate}, we have
\begin{align}\nonumber
\iint_{B_R^c}(\hat J_{\mu}\star\varpi)^2dxdy\leq &\iint_{\tilde x^2+\tilde y^2\leq \mu^2}\hat J_{\mu}(\tilde x,\tilde y)\iint_{B_R^c}|\varpi(x-\tilde x,y-\tilde y)|^2dxdyd\tilde x d\tilde y\\\nonumber
=&\iint_{\tilde x^2+\tilde y^2\leq \mu^2}\hat J_{\mu}(\tilde x,\tilde y)\iint_{B_R^c-(\tilde x,\tilde y)}|\varpi(\hat x,\hat y)|^2d\hat xd\hat yd\tilde x d\tilde y\\\label{BRc2}
\leq& \iint_{B_{R-1}^c}|\varpi(\hat x,\hat y)|^2d\hat xd\hat y=\|\varpi\|_{L^2(B_{R-1}^c)}^2,
\end{align}
 for $\mu\in(0,1)$ and $R>1$,
where $ B_R^c-(\tilde x,\tilde y)=\{(\hat x,\hat y)|\hat x=x-\tilde x,\hat y=y-\tilde y, (x,y)\in B_R^c\}$ and in the last inequality, we used $B_R^c-(\tilde x,\tilde y)\subset B_{R-1}^c$ since $\tilde y\in[-\mu,\mu]\subset (-1,1)$.
Similarly, we have
\begin{align}\label{BRc3}
\iint_{B_R^c}|\hat J_{\mu}\star\varpi|dxdy\leq \|\varpi\|_{L^1(B_{R-1}^c)}
\end{align}  for $\mu\in(0,1)$ and $R>1$.
Noting that
\begin{align*}
\|y(\hat J_{\mu}\star\varpi)\|_{L^1(B_R^c)}\leq &\iint_{\tilde x^2+\tilde y^2\leq \mu^2}\hat J_{\mu}(\tilde x,\tilde y)\iint_{B_R^c}\left(|y-\tilde y|+|\tilde y|\right)|\varpi(x-\tilde x,y-\tilde y)|dxdyd\tilde xd\tilde y\\
= &\iint_{\tilde x^2+\tilde y^2\leq \mu^2}\hat J_{\mu}(\tilde x,\tilde y)\iint_{B_R^c-(\tilde x,\tilde y)}\left(|\hat y\varpi(\hat x,\hat y)|+|\tilde y\varpi(\hat x,\hat y)|\right)d\hat xd\hat yd\tilde xd\tilde y\\
\leq& \|y\varpi\|_{L^1(B_{R-1}^c)}+C_0\|\varpi\|_{L^1(B_{R-1}^c)},
\end{align*}
we have
\begin{align}\nonumber
&\iint_{B_R^c}|\psi_\ep(\hat J_{\mu}\star\varpi)|dxdy\leq\iint_{B_R^c}\left(|((G_1+G_2)*\omega_\ep)(\hat J_{\mu}\star\varpi)|+C|\hat J_{\mu}\star\varpi|\right)dxdy\\\nonumber
\leq&\|G_1\|_{L^2(\Omega)} \|\omega_\ep\|_{L^2(\Omega)} \iint_{B_R^c}|\hat J_{\mu}\star\varpi|dxdy\\\nonumber
&+C\iint_{B_R^c}\left(\iint_\Omega|y-\tilde y||\varpi(\tilde x,\tilde y)|d\tilde x d\tilde y\right)|(\hat J_{\mu}\star\varpi)(x,y)|dxdy+C\|\varpi\|_{L^1(B_{R-1}^c)}\\\nonumber
\leq &\|G_1\|_{L^2(\Omega)} \|\omega_\ep\|_{L^2(\Omega)}\|\varpi\|_{L^1(B_{R-1}^c)}+C(\|\varpi\|_{L^1(\Omega)}\|y(\hat J_{\mu}\star\varpi)\|_{L^1(B_R^c)}\\\nonumber
&+\|y\varpi\|_{L^1(\Omega)}\|\hat J_{\mu}\star\varpi\|_{L^1(B_R^c)})+C\|\varpi\|_{L^1(B_{R-1}^c)}\\\nonumber
\leq &\|G_1\|_{L^2(\Omega)} \|\omega_\ep\|_{L^2(\Omega)}\|\varpi\|_{L^1(B_{R-1}^c)}+C\|\varpi\|_{L^1(\Omega)}(\|y\varpi\|_{L^1(B_{R-1}^c)}+C_0\|\varpi\|_{L^1(B_{R-1}^c)})\\\label{BRc4}
&+C\|y\varpi\|_{L^1(\Omega)}\|\varpi\|_{L^1(B_{R-1}^c)}+C\|\varpi\|_{L^1(B_{R-1}^c)}
\end{align}
for $\mu\in(0,1)$ and $R>1$. Combining \eqref{BRc1}-\eqref{BRc4}, we have
\begin{align}\nonumber
&\iint_{B_R^c}|(-\hat J_{\mu}\star\varpi)\ln(-(\hat J_{\mu}\star\varpi))-(-\varpi)\ln(-\varpi)|dxdy\\\nonumber
\leq&\|\varpi\|_{L^2(B_{R-1}^c)}^2+\|\varpi\|_{L^1(B_{R-1}^c)}+2\|\omega_\ep\|_{L^1(B_R^c)}+2\|G_1\|_{L^2(\Omega)} \|\omega_\ep\|_{L^2(\Omega)}\|\varpi\|_{L^1(B_{R-1}^c)}\\\nonumber
&+2C\|\varpi\|_{L^1(\Omega)}(\|y\varpi\|_{L^1(B_{R-1}^c)}+C_0\|\varpi\|_{L^1(B_{R-1}^c)})
+2C\|y\varpi\|_{L^1(\Omega)}\|\varpi\|_{L^1(B_{R-1}^c)}\\\label{BRc-sum}
&+2C\|\varpi\|_{L^1(B_{R-1}^c)}+\|\varpi\|_{L^2(B_{R}^c)}^2+\|\varpi\|_{L^1(B_{R}^c)}+2\|\psi_\ep\varpi\|_{L^1(B_{R}^c)}
\end{align}
for $\mu\in(0,1)$ and $R>1$. Thus, for any $\varepsilon>0$, we can choose  $R_0>1$ (independent of $\mu$) such that
\begin{align}\label{R-varepsilon}
\iint_{B_{R_0}^c}|(-\hat J_{\mu}\star\varpi)\ln(-(\hat J_{\mu}\star\varpi))-(-\varpi)\ln(-\varpi)|dxdy<{\varepsilon\over 4}.
\end{align}
Let $\nu_0>0$ small enough such that $(8+2\|G_1\|_{L^2(\Omega)} \|\omega_\ep\|_{L^2(\Omega)}+2C\|\varpi\|_{L^1(\Omega)}(1+C_0)
+2C\|y\varpi\|_{L^1(\Omega)}+C)\nu_0<\varepsilon/4$. Then there exists $\delta_0>0$ (depending on $\varepsilon$) such that for any subset $E\subset \Omega$ satisfying $|E|\leq\delta_0$, we have
\begin{align}\label{absolute continuity of integral}
\max\{\|\varpi\|_{L^2(E)}^2,\|\varpi\|_{L^1(E)},\|\omega_\ep\|_{L^1(E)},\|y\varpi\|_{L^1(E)},\|\psi_\ep\varpi\|_{L^1(E)}\}\leq \nu_0.
\end{align}
By \eqref{absolute continuity of integral} and the fact that $|E-(\tilde x,\tilde y)|=|E|$ for any $(\tilde x,\tilde y)\in \mathbb{R}^2$, a similar argument to \eqref{BRc1}-\eqref{BRc-sum} implies that
\begin{align*}
\iint_{E}(\hat J_{\mu}\star\varpi)^2dxdy\leq& \nu_0,\quad \iint_{E}|\hat J_{\mu}\star\varpi|dxdy\leq \nu_0,\\
\iint_{E}|\psi_\ep(\hat J_{\mu}\star\varpi)|dxdy
\leq& \|G_1\|_{L^2(\Omega)} \|\omega_\ep\|_{L^2(\Omega)}\nu_0+C\|\varpi\|_{L^1(\Omega)}(\nu_0
+C_0\nu_0)\\
&+C\|y\varpi\|_{L^1(\Omega)}\nu_0+C\nu_0,
\end{align*} and
\begin{align}\nonumber
&\iint_{E}|(-\hat J_{\mu}\star\varpi)\ln(-(\hat J_{\mu}\star\varpi))-(-\varpi)\ln(-\varpi)|dxdy\\\nonumber
\leq&\nu_0+\nu_0+2\nu_0+2\|G_1\|_{L^2(\Omega)} \|\omega_\ep\|_{L^2(\Omega)}\nu_0\\\label{E-small}
&+2C\|\varpi\|_{L^1(\Omega)}(\nu_0+C_0\nu_0)
+2C\|y\varpi\|_{L^1(\Omega)}\nu_0
+C\nu_0+\nu_0+\nu_0+2\nu_0
\leq {\varepsilon\over 4}
\end{align}
for   $E\subset \Omega$ satisfying $|E|\leq\delta_0$.  By Lusin's Theorem, there exists a closed subset $F\subset B_{R_0}$ such that $|B_{R_0}\setminus F|<\delta_0$ and $\varpi$ is continuous on $F$. Thus, $0< \min_{(x,y)\in F}|\varpi(x,y)|\leq \max_{(x,y)\in F}|\varpi(x,y)|<\infty$. Let  $a_F\triangleq \max_{(x,y)\in F}|\varpi(x,y)|+1$.
Since $s\ln(s)$ is uniformly continuous on $[0,a_F]$, there exists $\delta_1\in(0,\min\{\min_{(x,y)\in F}|\varpi(x,y)|,1\})$ (depending on $\varepsilon, R_0, F$) such that
\begin{align}\label{uniform continuity}
|s_2\ln(s_2)-s_1\ln(s_1)|<{\varepsilon\over 16\pi R_0} \text{ for }s_1,s_2\in[0,a_F] \text{ and }|s_2-s_1|\leq\delta_1.
\end{align}
We divide $F$ into two parts
\begin{align*}
B_{1,\delta_1}^{\mu}=\{(x,y)\in  F|\;|(\hat J_\mu\star\varpi)(x,y)-\varpi(x,y)|\leq \delta_1\},\\
B_{2,\delta_1}^{\mu}=\{(x,y)\in  F|\;|(\hat J_\mu\star\varpi)(x,y)-\varpi(x,y)|> \delta_1\}.
\end{align*}
Since $(\hat J_\mu\star\varpi)\to\varpi$ in $L^1(\Omega)$, we have
\begin{align*}
|B_{2,\delta_1}^{\mu}|\delta_1\leq\|(\hat J_\mu\star\varpi)-\varpi\|_{L^1(B_{2,\delta_1}^{\mu})}\leq \|(\hat J_\mu\star\varpi)-\varpi\|_{L^1(\Omega)}\leq \delta_0\delta_1\Longrightarrow|B_{2,\delta_1}^{\mu}|\leq \delta_0
\end{align*}
for $\mu>0$ small enough. By \eqref{uniform continuity}, we have
\begin{align}\label{B1delta1kappa-small}
&\iint_{B_{1,\delta_1}^{\mu}}|(-\hat J_{\mu}\star\varpi)\ln(-(\hat J_{\mu}\star\varpi))-(-\varpi)\ln(-\varpi)|dxdy
\leq {\varepsilon\over 16\pi R_0}|B_{1,\delta_1}^{\mu}|\leq{\varepsilon\over 4}.
\end{align}
Since $|B_{R_0}\setminus F|<\delta_0$ and
$|B_{2,\delta_1}^{\mu}|\leq \delta_0$,  we infer from \eqref{E-small} that
\begin{align}\label{BR0setminusF-small}
&\iint_{B_{R_0}\setminus F}|(-\hat J_{\mu}\star\varpi)\ln(-(\hat J_{\mu}\star\varpi))-(-\varpi)\ln(-\varpi)|dxdy
\leq {\varepsilon\over 4},\\\label{B2delta1kappa-small}
&\iint_{B_{2,\delta_1}^{\mu}}|(-\hat J_{\mu}\star\varpi)\ln(-(\hat J_{\mu}\star\varpi))-(-\varpi)\ln(-\varpi)|dxdy
\leq {\varepsilon\over 4}
\end{align}
for $\mu>0$ small enough. The conclusion \eqref{xlnx-convergence} then follows from
\eqref{R-varepsilon} and \eqref{B1delta1kappa-small}-\eqref{B2delta1kappa-small}.
\end{proof}

\subsection{Global existence  of the approximate solutions}
Now, we prove the global existence  of the approximate solutions.
\begin{lemma}\label{lem-construction of an approximate solution sequence}
Let $\tilde\omega_0\in  Y_{non}$ and $\tilde\omega_0^{\mu}$ be defined in  \eqref{tilde-omega0-kappa-def} for $\mu\in(0,1)$.
For the initial data $\vec{v}_0^{\mu}=K\ast\tilde\omega_0^{\mu}$, there exists  a smoothly strong solution $\vec{v}^{\mu}(t)=\vec{v}_{0,0}^{\mu}+\vec{v}_{\mu}(t)$  globally in time   to the 2D Euler equation such that
 $\vec{v}_{\mu}(t)\in H^q(\Omega)$ and $\vec{v}_{\mu}\in C^0([0,T], H^q(\Omega))$   for all $q\geq3$ and $T>0$, where $\vec{v}_{0,0}^{\mu}=K\ast\tilde\omega_{0,0}^{\mu}$. Moreover, $\lim_{y\to\pm\infty}\vec{v}^{\mu}(t,x,y)=(\pm1,0)$ for all $t\geq0$ and $x\in\mathbb{T}_{2\pi}$,
$\{\vec{v}^{\mu}\}$ forms an approximate solution sequence with $L^1$, $L^2$ vorticity control,
and $\tilde \omega_0^{\mu}\to\tilde\omega_0$ in $L^1\cap L^2(\Omega)$.
\end{lemma}

\begin{proof}
We decompose $\vec{v}_{0}^{\mu}$ into the shear-energy parts: $\vec{v}_{0}^{\mu}=K\ast\tilde\omega_0^{\mu}=K\ast\tilde\omega_{0,0}^{\mu}+K\ast\tilde\omega_{0,\neq0}^{\mu}\triangleq\vec{v}_{0,0}^{\mu}+\vec{v}_{0,\neq0}^{\mu}$. Then
by Lemma \ref{tilde-omega0-kappa-properties} (5), we have $\vec{v}_{0,\neq0}^{\mu}=K\ast(\hat J_{\mu}\star\tilde\omega_{0,\neq0})=\hat J_{\mu}\star\vec{v}_{0,\neq0}\in H^q(\Omega)$ for all $q\geq3$ since $\vec{v}_{0,\neq0}\in L^2(\Omega)$.
%Moreover, ${v}_{0,0,2}^{\mu}=-\partial_x G\ast\tilde\omega_{0,0}^{\mu}=- G\ast\partial_x \tilde\omega_{0,0}^{\mu}=0$, where ${v}_{0,0,2}^{\mu}$ is the second element of $\vec{v}_{0,0}^{\mu}$. Thus, $div(\vec{v}_{\mu})=div(\vec{v}^{\mu})-div(\vec{v}_{0,0}^{\mu})=0.$
Now we denote $\vec{v}_{\mu}$ to be the solution of the evolution equation
\begin{align}\label{euler-nonshear}
 \partial_t\vec{u}+(\vec{u}\cdot\nabla)\vec{u}+(\vec{v}_{0,0}^{\mu}\cdot\nabla)\vec{u}+(\vec{u}\cdot\nabla)\vec{v}_{0,0}^{\mu}=-\nabla p
 \end{align}
with the initial data
$\vec{v}_{\mu}(0)=\vec{v}_{0,\neq0}^{\mu}$.  Then similar to Subsection 3.2.4 in \cite{Majda-Bertozzi02}, the  solution $\vec{v}_{\mu}$ to equation \eqref{euler-nonshear} exists locally in time in $H^q(\Omega)$ for $q\geq3$ and can be continued in time provided that $\|\vec{v}_{\mu}(t)\|_{H^q(\Omega)}$ remains bounded.
We use the shear-energy decomposition to derive  the BKM-type estimate  \eqref{BKM} in the cylinder version,
which proves the global existence  of the  solution $\vec{v}_{\mu}$ to the 2D Euler equation in $H^q(\Omega)$ for $q\geq3$.
The BKM criterion was originally  obtained for the 3D Euler equation on  $\mathbb{R}^3$  in
\cite{BKM84} and extended to the $\mathbb{R}^2$ version using a radial-energy decomposition for the velocity field with infinite energy (see \cite{Majda-Bertozzi02} for example). We follow the line from \cite{Majda-Bertozzi02} and \cite{Kato1986nonlinear}.
Note that  $div(\vec{v}_{\mu}(t))=div(\vec{v}^{\mu}(t))-div(\vec{v}_{0,0}^{\mu})=0$ for $t\geq0$ since
${v}_{0,0,2}^{\mu}=- G\ast\partial_x \tilde\omega_{0,0}^{\mu}=0$, where ${v}_{0,0,2}^{\mu}$ is the second entry of $\vec{v}_{0,0}^{\mu}$. Then a basic energy estimate gives
\begin{align*}
{1\over 2} {d\over dt}\|\vec{v}_{\mu}(t)\|_{L^2(\Omega)}^2+\iint_{\Omega}(\vec{v}_{\mu}(t)\cdot\nabla)\vec{v}_{0,0}^{\mu}\cdot\vec{v}_{\mu}(t)dxdy=0.
 \end{align*}
Indeed, we
can first prove it for the regularized solution and then take the limit by a similar approach in Theorem 3.6 of \cite{Majda-Bertozzi02}.
\if0
used \eqref{v-mu-boundary term1}-\eqref{v-mu-boundary term2} to ensure that $\iint_{\Omega}(\vec{v}_{\mu}\cdot\nabla)\vec{v}_{\mu}\cdot\vec{v}_{\mu}dxdy
=\iint_{\Omega}\vec{v}_{\mu}\cdot \nabla({1\over2}|\vec{v}_{\mu}|^2) dxdy=-\iint_{\Omega}div(\vec{v}_{\mu})({1\over2}|\vec{v}_{\mu}|^2) dxdy=0$, $\iint_{\Omega}(\vec{v}_{0,0}^{\mu}\cdot\nabla)\vec{v}_{\mu}\cdot\vec{v}_{\mu}dxdy=0$ and
$\iint_{\Omega}\nabla p\cdot\vec{v}_{\mu}dxdy=0$. \eqref{v-mu-boundary term1}-\eqref{v-mu-boundary term2} are verified at the end of the proof.
\fi
Then
\begin{align}\label{basic energy estimate-eq}
{d\over dt}\|\vec{v}_{\mu}(t)\|_{L^2(\Omega)}\leq \|\vec{v}_{\mu}(t)\|_{L^2(\Omega)}\|\nabla\vec{v}_{0,0}^{\mu}\|_{L^\infty(\Omega)}
 \end{align}
and Gr\"{o}nwall's inequality implies
\begin{align}\label{basic energy estimate-gronwall}
\|\vec{v}_{\mu}(t)\|_{L^2(\Omega)}\leq \|\vec{v}_{\mu}(0)\|_{L^2(\Omega)}e^{\int_0^t\|\nabla \vec{v}_{0,0}^{\mu}\|_{L^\infty(\Omega)}ds},
 \end{align}
 where $\nabla \vec{v}_{0,0}^{\mu}$ is in the form of  $2\times2$ matrix.

 We prove that $\vec{v}_{0,0}^{\mu}\in W^{j,\infty}(\Omega)$ for $j\geq0$. Since $\tilde \omega_0\in L^2(\Omega)$, we have $\tilde \omega_{0,0}\in L^2(\Omega)$ and $\|D^j\tilde \omega_{0,0}^{\mu}\|_{L^\infty(\Omega)}\leq C(\mu,j)\|\tilde \omega_{0,0}\|_{L^2(\Omega)}$ by Lemma \ref{tilde-omega0-kappa-properties} (5). Noting that $\|D^j\tilde \omega_{0,0}^{\mu}\|_{L^1(\Omega)}=\iint_\Omega|(D^j\hat J_\mu)\star\tilde \omega_{0,0}|dxdy\leq \|D^j\hat J_\mu\|_{L^1(\mathbb{R}^2)}\|\tilde \omega_{0,0}\|_{L^1(\Omega)}\leq C(\mu,j)\|\tilde \omega_{0,0}\|_{L^1(\Omega)}$, we have
 \begin{align*}
 \|D^j\vec{v}_{0,0}^{\mu}\|_{L^\infty(\Omega)}=&\|K\ast D^j\tilde\omega_{0,0}^{\mu}\|_{L^\infty(\Omega)}\leq \|(\rho_1K)*D^j\tilde \omega_{0,0}^{\mu}\|_{L^\infty(\Omega)}+\|((1-\rho_1)K)*D^j\tilde \omega_{0,0}^{\mu}\|_{L^\infty(\Omega)}\\
 \leq&\|(\rho_1K)\|_{L^1(\Omega)}\|D^j\tilde \omega_{0,0}^{\mu}\|_{L^\infty(\Omega)}+\|((1-\rho_1)K)\|_{L^\infty(\Omega)}\|D^j\tilde \omega_{0,0}^{\mu}\|_{L^1(\Omega)}\\
 \leq&C(\mu,j)\|\tilde \omega_{0,0}\|_{L^2(\Omega)}+C(\mu,j)\|\tilde \omega_{0,0}\|_{L^1(\Omega)}.
 \end{align*}


 Taking derivative of \eqref{euler-nonshear} and similar to \eqref{basic energy estimate-eq}-\eqref{basic energy estimate-gronwall}, we get the high-order energy estimates ($q\geq1$):
 \begin{align}\nonumber
 {d\over dt}\|\vec{v}_{\mu}(t)\|_{H^q(\Omega)}&\leq C_q \|\vec{v}_{\mu}(t)\|_{H^q(\Omega)}\left(\|\nabla \vec{v}_{\mu}(t)\|_{L^\infty(\Omega)}+\|\vec{v}_{0,0}^{\mu}\|_{W^{q+1,\infty}(\Omega)}\right),
 \end{align}
 and
 \begin{align}\label{basic energy estimate-gronwall-high-order}
\|\vec{v}_{\mu}(t)\|_{H^q(\Omega)}&\leq \|\vec{v}_{\mu}(0)\|_{H^q(\Omega)}e^{\int_0^tC_q\left(\|\nabla \vec{v}_{\mu}(t)\|_{L^\infty(\Omega)}+\|\vec{v}_{0,0}^{\mu}\|_{W^{q+1,\infty}(\Omega)}\right)ds}.
 \end{align}
By the asymptotic behavior of $|K|$ near $(x,y)=(0,0)$ in  \eqref{K-near-0} and the exponential decay rate of $|\nabla K|$ as $|y|\to \infty$, a similar argument to Lemma A3 in \cite{Kato1986nonlinear} gives
\begin{align*}
&\|\nabla\vec{v}_{\mu}(t)\|_{L^\infty(\Omega)}\leq\|\nabla\vec{v}^{\mu}(t)\|_{L^\infty(\Omega)}+\|\nabla\vec{v}_{0,0}^{\mu}\|_{L^\infty(\Omega)}\\
\leq&
 C\bigg(\|\tilde\omega_{0}^\mu\|_{L^\infty(\Omega)}+\|\tilde\omega_{0}^\mu\|_{L^2(\Omega)}
+\|\tilde\omega_{0}^\mu\|_{L^\infty(\Omega)}
\ln\left(1+{\|\vec{v}^{\mu}(t)\|_{H^3(\Omega)}\over\|\tilde\omega_{0}^\mu\|_{L^\infty(\Omega)}}\right)\\
&+\|\tilde\omega_{0,0}\|_{L^2(\Omega)}+\|\tilde\omega_{0,0}\|_{L^1(\Omega)}\bigg)\\
\leq&C\bigg(\|\tilde\omega_{0}\|_{L^2(\Omega)}+\|\tilde\omega_{0,0}\|_{L^1(\Omega)}+\|\tilde\omega_{0}\|_{L^2(\Omega)}
\ln\bigg(1+{\|\vec{v}_{0,0}^{\mu}\|_{H^3(\Omega)}\over\|\tilde\omega_{0}^\mu\|_{L^\infty(\Omega)}}+
{\|\vec{v}_{\mu}(t)\|_{H^3(\Omega)}\over\|\tilde\omega_{0}^\mu\|_{L^\infty(\Omega)}}\bigg)\bigg),
\end{align*}
where we used \eqref{B-S law}.
 Then
\begin{align}\label{estimate-v-kappa-L-infty}
\|\nabla\vec{v}_{\mu}(t)\|_{L^\infty(\Omega)}\leq C_{\|\tilde\omega_{0}^\mu\|_{L^\infty(\Omega)},\|\tilde\omega_{0,0}\|_{L^1(\Omega)},\|\tilde\omega_0\|_{L^2(\Omega)},\|\vec{v}_{0,0}^{\mu}\|_{H^3(\Omega)}}\left(1+
\ln_+(\|\vec{v}_{\mu}(t)\|_{H^3(\Omega)})\right),
\end{align}
where $\ln_+(x)=\ln(x)$ for $x>1$ and $\ln_+(x)=0$ for $0<x\leq1$.
Plugging \eqref{basic energy estimate-gronwall-high-order} for $q=3$ into \eqref{estimate-v-kappa-L-infty}, we have
\begin{align*}
\|\nabla\vec{v}_{\mu}(t)\|_{L^\infty(\Omega)}\leq C_*\left(1+\|\vec{v}_{0,0}^{\mu}\|_{W^{4,\infty}(\Omega)}t+
\int_0^t\|\nabla \vec{v}_{\mu}(t)\|_{L^\infty(\Omega)}ds\right),
\end{align*}
where $C_*=C_{\|\tilde\omega_{0}^\mu\|_{L^\infty(\Omega)},\|\tilde\omega_{0,0}\|_{L^1(\Omega)},\|\tilde\omega_0\|_{L^2(\Omega)},
\|\vec{v}_{0,0}^{\mu}\|_{H^3(\Omega)},\|\vec{v}_{\mu}(0)\|_{H^3(\Omega)}}$
depends only on the initial data.
Then Gr\"{o}nwall's inequality implies
\begin{align*}
\|\nabla\vec{v}_{\mu}(t)\|_{L^\infty(\Omega)}\leq (C_*+\tilde C_* t)e^{C_*t},
\end{align*}
where $\tilde C_*=C_*\|\vec{v}_{0,0}^{\mu}\|_{W^{4,\infty}(\Omega)}$.
Inserting this into \eqref{basic energy estimate-gronwall-high-order}  gives an a priori bound for $\|\vec{v}_{\mu}\|_{H^q(\Omega)}$:
\begin{align}\label{BKM}
\|\vec{v}_{\mu}(t)\|_{H^q(\Omega)}\leq \|\vec{v}_{\mu}(0)\|_{H^q(\Omega)}e^{\int_0^tC_q\left( (C_*+\tilde C_* t)e^{C_*t}+\|\vec{v}_{0,0}^{\mu}\|_{W^{q+1,\infty}(\Omega)}\right)ds},
 \end{align}
 which proves the global existence  of the  solution $\vec{v}^{\mu}=\vec{v}_{0,0}^{\mu}+\vec{v}_{\mu}$ to 2D Euler equation in $H^q(\Omega)$ for $q\geq3$. This verifies (iii) of Definition \ref{Approximate solution sequence for the 2D Euler equation}. (ii) is trivially  verified. Then we prove that $\{\vec{v}^\mu\}$ has  $L^1$ and $L^2$ vorticity control.
  Let $\tilde \omega^\mu=\text{curl}(\vec{v}^\mu)$.
By Lemma \ref{tilde-omega0-kappa-properties} (4),
\begin{align}\label{L12 bound-approximate solution sequence}
\iint_{\Omega}|\tilde\omega^\mu(t)|^pdxdy=&\iint_{\Omega}|\tilde\omega_0^\mu|^pdxdy\leq \|\tilde\omega_{0}\|_{L^p(\Omega)}^p
\end{align}
for $t\geq0$, and
 $\tilde \omega_0^{\mu}\to\tilde\omega_0$ in $L^p(\Omega)$ for $p=1,2$. To  verify (i), we
 note that
 \begin{align}\nonumber
 &\|\vec{v}^\mu(t)\|_{L^2(B_R)}=\|(K*\tilde \omega^\mu)(t)\|_{L^2(B_R)}\\\nonumber
 \leq& \|((\rho_1K)*\tilde \omega^\mu)(t)\|_{L^2(\Omega)}+\|(((1-\rho_1)K)*\tilde \omega^\mu)(t)\|_{L^2(B_R)}\\\nonumber
 \leq&\|\rho_1K\|_{L^1(\Omega)}\|\tilde \omega^\mu(t)\|_{L^2(\Omega)}+C(R)\|(1-\rho_1)K\|_{L^\infty(\Omega)}\|\tilde \omega^\mu(t)\|_{L^2(\Omega)}\\\label{uniform L2 bound v kappa}
 \leq&C(R)\|\tilde \omega_0\|_{L^2(\Omega)}
 \end{align}
 for any $ R>0$,
where we used $\tilde \omega^\mu(t)=\text{curl}(\vec{v}^\mu(t))$ and \eqref{L12 bound-approximate solution sequence}.

We define the stream function by $\tilde \psi^\mu(t)=G*\tilde \omega^\mu(t)$, where $\omega^\mu(t)=curl(\vec{v}^{\mu}(t))$ is the vorticity. Then the velocity can be recovered from $\tilde \psi^\mu(t)$ by the Biot-Savart law
\begin{align}\label{B-S law}
\vec{v}^{\mu}(t)=\nabla^{\bot} (G*\tilde \omega^\mu(t))=K*\tilde \omega^\mu(t)\end{align}
in our setting. In fact, let
 $\vec{\vartheta}(t)=({\vartheta}_1(t),{\vartheta}_2(t) )\triangleq K*\tilde \omega^\mu(t)-\vec{v}^{\mu}(t)$ for  $\mu\in(0,1)$ and $t\geq0$.
 Since $div(\vec{\vartheta}(t))=0$ and $curl(\vec{\vartheta}(t))=0$, we have $ik\widehat{\vartheta}_{1,k}(y)+\widehat{\vartheta}_{2,k}'(y)=0$, $\widehat{\vartheta}_{1,k}'(y)-ik\widehat{\vartheta}_{2,k}(y)=0$ for $k\neq0$, $\widehat{\vartheta}_{1,0}'(y)=0$ and $\widehat{\vartheta}_{2,0}'(y)=0$.
Thus, $\widehat{\vartheta}_{1,k}''(y)-k^2\widehat{\vartheta}_{1,k}(y)=0$ and $\widehat{\vartheta}_{2,k}''(y)-k^2\widehat{\vartheta}_{2,k}(y)=0$ for $k\neq0$, which implies $
\widehat{\vartheta}_{1,k}(y)=c_{1,k}e^{ky}+\tilde c_{1,k}e^{-ky}$ and $
\widehat{\vartheta}_{2,k}(y)=c_{2,k}e^{ky}+\tilde c_{2,k}e^{-ky}$ for some $c_{1,k},\tilde c_{1,k},c_{2,k},\tilde c_{2,k}\in\mathbb{C}$. Noting that
$\vec{\vartheta}= ({\vartheta}_1,{\vartheta}_2 )=K*\tilde \omega_\mu(t)-\vec{v}_{\mu}(t)$, we have ${\vartheta}_2\in L^2 (\Omega)$, where $\tilde \omega_\mu(t)=\tilde \omega^\mu(t)-\tilde \omega_{0,0}^\mu$. Thus, $\widehat{\vartheta}_{2,k}(y)=0$ for $k\in\mathbb{Z}$, which implies $\widehat{\vartheta}_{1,k}(y)=0$ for $k\neq0$  since $ik\widehat{\vartheta}_{1,k}(y)+\widehat{\vartheta}_{2,k}'(y)=0$. By the first limit in \eqref{v-mu term1 shear decomposition} and $\vec{v}_{\mu}(t)\in L^2 (\Omega)$, we have  $\widehat{\vartheta}_{1,0}(y)=0$.

Finally, we prove that
\begin{align}\label{v-mu term2}
&\lim_{y\to\pm\infty}v^{\mu,2}(t,x,y)=-\lim_{y\to\pm\infty}\partial_x\tilde\psi^\mu(t,x,y)=-\lim_{y\to\pm\infty}(\partial_x G\ast\tilde \omega^\mu)(t,x,y)=0,\\
&\lim_{y\to\pm\infty}(\partial_y G\ast\tilde \omega_\mu)(t,x,y)=0,\;\;\lim_{y\to\pm\infty}(\partial_y G\ast\tilde \omega_{0,0}^\mu)(t,x,y)=\pm1,
\label{v-mu term1 shear decomposition}
\end{align}
which implies
\begin{align}
\label{v-mu term1}
&\lim_{y\to\pm\infty}v^{\mu,1}(t,x,y)=\lim_{y\to\pm\infty}\partial_y\tilde\psi^\mu(t,x,y)=\lim_{y\to\pm\infty}(\partial_y G\ast\tilde \omega^\mu)(t,x,y)=\pm1
\end{align}
for $t\geq0$ and $x\in\mathbb{T}_{2\pi}$, where $\vec{v}^{\mu}(t)=(v^{\mu,1}(t),v^{\mu,2}(t))$.
Indeed, $\|\tilde \omega^\mu(t)\|_{L^{p'}(\Omega)}=\|\tilde \omega^\mu(0)\|_{L^{p'}(\Omega)}\leq C\|\tilde \omega^\mu(0)\|_{H^{1}(\Omega)}$, and thus, for any $\varepsilon>0$,  there exists $R_1>0$ such that
\begin{align*}
\|\tilde \omega^\mu(t)\|_{L^{p'}(B_{R_1}^c)}<{\varepsilon\over 2\left\|\partial_x G\right\|_{L^{p}(\Omega)}},
\end{align*}
where $p\in(1,2)$ and ${1\over p}+{1\over p'}=1$. Then
\begin{align}\label{B-R1-c}
&\left|\iint_{B_{R_1}^c}\partial_x G(x-\tilde x,y-\tilde y)\tilde \omega^\mu(t,\tilde x,\tilde y)d\tilde xd\tilde y\right|
\leq\left\|\partial_x G\right\|_{L^{p}(\Omega)}
\|\tilde \omega^\mu(t)\|_{L^{p'}(B_{R_1}^c)}<{\varepsilon\over 2}
\end{align}
for $(x,y)\in\Omega$. Choose $M_1>0$ such that if $|y|>M_1$, then  $\left|\partial_x G(x-\tilde x,y-\tilde y)\right|<{\varepsilon\over 2\|\tilde \omega^\mu(t)\|_{L^1(\Omega)}}$ uniformly for $(\tilde x,\tilde y)\in B_{R_1}$. Then
\begin{align*}
&\left|\iint_{B_{R_1}}\partial_x G(x-\tilde x,y-\tilde y)\tilde \omega^\mu(t,\tilde x,\tilde y)d\tilde xd\tilde y\right|\leq{\varepsilon\over 2}
\end{align*}
for $|y|>M_1$. This, along with \eqref{B-R1-c}, gives   \eqref{v-mu term2}.
 To prove $\lim_{y\to\infty}(\partial_y G\ast\tilde \omega_\mu)(t,x,y)=0$ in \eqref{v-mu term1 shear decomposition}, we denote $C_0=\max_{x\in\mathbb{T}_{2\pi},|y|>1}(|\partial_y G|+1)<\infty$. For any $\varepsilon>0$, there exists $R_2>0$ such that
\begin{align*}
\|\tilde \omega_\mu(t)\|_{L^1(\{y>R_2\})}<{\varepsilon\over 4C_0},\;\;\|\tilde \omega_\mu(t)\|_{L^{p'}(\{y>R_2\})}<{\varepsilon\over 4\|\partial_yG\|_{L^p(B_1)}},
\end{align*}
where $p\in(1,2)$.
Since
$\iint_\Omega\tilde \omega_\mu(t)dxdy=\iint_\Omega\tilde \omega^\mu(t)dxdy-\iint_\Omega\tilde \omega_{0,0}^\mu dxdy=\iint_\Omega\tilde \omega^\mu(0)dxdy-\iint_\Omega\tilde \omega_{0,0}^\mu dxdy=0$, we have
\begin{align*}
&(\partial_y G\ast\tilde \omega_\mu)(t,x,y)=((\partial_y G+{1/ (4\pi)})\ast\tilde \omega_\mu)(t,x,y)\\
=&\iint_{\{\tilde y<R_2\}}(\partial_y G(x-\tilde x,y-\tilde y)+{1/ (4\pi)})\tilde \omega_\mu(t,\tilde x,\tilde y)d\tilde x d\tilde y\\
&+\iint_{\{\tilde y>R_2\}}(\partial_y G(x-\tilde x,y-\tilde y)+{1/ (4\pi)})\tilde \omega_\mu(t,\tilde x,\tilde y)d\tilde x d\tilde y= I +II.
\end{align*}
Choose $M_2>R_2$ such that if $y>M_2$, then $|\partial_y G(x-\tilde x,y-\tilde y)+{1/(4\pi)}|<{\varepsilon\over 4\|\tilde \omega_\mu(t)\|_{L^1(\Omega)}}$ uniformly for $\tilde y<R_2$. Then
$
|I|\leq {\varepsilon\over4}
$
for $y>M_2$. For $II$, we have
\begin{align*}
|II|=&\bigg|\iint_{\{\tilde y>R_2\}\cap\{|\tilde y-y|\leq1\}}\partial_y G(x-\tilde x,y-\tilde y)\tilde \omega_\mu(t,\tilde x,\tilde y)d\tilde x d\tilde y\\
&+\iint_{\{\tilde y>R_2\}\cap\{|\tilde y-y|\leq1\}}{1/ (4\pi)}\tilde \omega_\mu(t,\tilde x,\tilde y)d\tilde x d\tilde y\\
&+\iint_{\{\tilde y>R_2\}\cap\{|\tilde y-y|>1\}}(\partial_y G(x-\tilde x,y-\tilde y)+{1/ (4\pi)})\tilde \omega_\mu(t,\tilde x,\tilde y)d\tilde x d\tilde y\bigg|\\
\leq &\|\partial_yG\|_{L^p(B_1)}\|\tilde \omega_\mu(t)\|_{L^{p'}(\{y>R_2\})}+\|\tilde \omega_\mu(t)\|_{L^1(\{y>R_2\})}+C_0\|\tilde \omega_\mu(t)\|_{L^1(\{y>R_2\})}<{3\over4}\varepsilon
\end{align*}
for $y\in\mathbb{R}$. Combining the estimates for $I$ and $II$, we have $\lim_{y\to\infty}(\partial_y G\ast\tilde \omega_\mu)(t,x,y)=0$.
Similarly, we have $\lim_{y\to-\infty}(\partial_y G\ast\tilde \omega_\mu)(t,x,y)=0$ and $\lim_{y\to\pm\infty}(\partial_y G\ast\tilde \omega_{0,0}^\mu)(t,x,y)=\pm1$.
\if0
 For any $R>0$,
 \begin{align*}
|\partial_y G\ast(\tilde \omega^\mu(t,x,y)-\tilde\omega_{0,0}^{\mu}(x,y))|\leq\|\partial_y G\|_{L^1(B_R)}\|\tilde \omega^\mu(t)-\tilde\omega_{0,0}^{\mu}\|_{L^\infty(\Omega)}+C_R\|\tilde \omega^\mu(t)-\tilde\omega_{0,0}^{\mu}\|_{L^1(\Omega)},
\end{align*}
which proves \eqref{v-mu-boundary term2}.
\fi
\end{proof}
\begin{Corollary}\label{y-tilde-omega-pseudoenergy-conserved}
Let $\{\vec{v}^\mu\}$ be the approximate solution sequence constructed in Lemma \ref{lem-construction of an approximate solution sequence}.
Then

$(1)$ for any $T>0$, there exists $C(T)>0$ (independent of $\mu$) such that  $\max_{0\leq t\leq T} \|y\tilde\omega^{\mu}(t)\|_{ L^1(\Omega)}$ $\leq C(T)$, and thus, $\tilde\omega^{\mu}(t)\in Y_{non}$ for $t\geq0$;
$\iint_\Omega y\tilde\omega^{\mu}(t,x,y)dxdy$ is conserved for all $t\geq0$;

$(2)$ the pseudoenergy
$
PE(\tilde \omega^\mu(t))={1\over2}\iint_{\Omega}(G\ast\tilde \omega^\mu)(t)\tilde \omega^{\mu}(t) dxdy
$ is conserved for all $t\geq0$.
%$(3)$ $\varphi*\tilde \omega^\mu(t)$ is conserved for all $t\geq0$, where $\varphi\in W^{1,\infty}(\mathbb{R})$.
\end{Corollary}

\begin{proof}
(1) We change the variables $(x,y)$ to $(X^\mu(t),Y^\mu(t))$, where $(X^\mu(t),Y^\mu(t))$ is the solution to $\dot{X}^\mu(t)=\partial_y\tilde\psi^\mu(t,X^\mu(t), Y^\mu(t)),
\dot{Y}^\mu(t)=-\partial_x\tilde \psi^\mu(t,X^\mu(t), Y^\mu(t))$ with the initial data $(X^\mu(0),Y^\mu(0))=(x,y)$. Noting that the vorticity $\tilde\omega^\mu$ is conserved along particle trajectories and the Jacobian of the mapping $(x,y)\to(X^\mu(t),Y^\mu(t))$ is $1$, we have
\begin{align}\nonumber
{d\over dt}\iint_\Omega |y\tilde\omega^{\mu}(t,x,y)|dxdy=&\iint_\Omega \dot{Y}^\mu(t)\tilde\omega^{\mu}(t,X^\mu(t),Y^\mu(t))sign(-Y^\mu(t))dX^\mu(t)dY^\mu(t)\\\nonumber
\leq&\|\partial_x\tilde\psi^\mu(t)\|_{L^2(\Omega)}\|\tilde\omega^{\mu}(t)\|_{L^2(\Omega)}
\leq\|\partial_xG\|_{L^1(\Omega)}\|\tilde\omega^{\mu}(t)\|_{L^2(\Omega)}^2\\\nonumber
=&\|\partial_xG\|_{L^1(\Omega)}\|\tilde\omega_0^{\mu}\|_{L^2(\Omega)}^2\leq\|\partial_xG\|_{L^1(\Omega)}\|\tilde\omega_0\|_{L^2(\Omega)}^2,
\end{align}
which, along with $y\tilde \omega^{\mu}_0\to y\tilde \omega_0$, implies that $\max_{0\leq t\leq T} \|y\tilde\omega^{\mu}(t)\|_{ L^1(\Omega)}\leq C(T)$. Moreover,
\begin{align}\nonumber
{d\over dt}\iint_\Omega y\tilde\omega^{\mu}(t,x,y)dxdy=&\iint_\Omega \dot{Y}^\mu(t)\tilde\omega^{\mu}(t,X^\mu(t),Y^\mu(t))dX^\mu(t)dY^\mu(t)\\\nonumber
=&\iint_\Omega -\partial_x\tilde\psi^\mu(t,x,y)\tilde\omega^{\mu}(t,x,y)dxdy\\\nonumber
=&-{1\over2}\iint_\Omega \partial_x|\nabla\tilde\psi^\mu(t,x,y)|^2 dxdy+\int_0^{2\pi}(\partial_x\tilde\psi^\mu\partial_y\tilde\psi^\mu)(t,x,y)|_{y=-\infty}^\infty dx\\\nonumber
=&-{1\over2}\iint_\Omega \partial_x|\nabla\tilde\psi^\mu(t,x,y)|^2 dxdy=0,
\end{align}
where we used \eqref{v-mu term2} and \eqref{v-mu term1} to ensure that $\lim_{y\to\pm\infty}(\partial_x\tilde\psi^\mu\partial_y\tilde\psi^\mu)(t,x,y)=0$ for $t>0$ and $x\in\mathbb{T}_{2\pi}$.

(2) Since $\tilde \psi^\mu(t)=G*\tilde \omega^\mu(t)$, we have
\begin{align}\nonumber
{d\over dt}PE(\tilde \omega^\mu(t))=&{1\over2}\iint_{\Omega}\partial_t\tilde \psi^\mu(t,X^\mu(t),Y^\mu(t))\tilde \omega^{\mu}(t,X^\mu(t),Y^\mu(t) )dxdy\\\nonumber
&+{1\over2}\iint_{\Omega}\nabla\tilde \psi^\mu(t, X^\mu(t),Y^\mu(t) )\cdot\nabla^\bot\tilde \psi^\mu(t,X^\mu(t),Y^\mu(t))\tilde \omega^{\mu}(t,X^\mu(t),Y^\mu(t) ) dxdy\\
=&{1\over2}\iint_{\Omega}\partial_t\tilde \psi^\mu(t,x,y)\tilde \omega^{\mu}(t,x,y )dxdy.\label{PE-derivative1}
\end{align}
On  the other hand,
\begin{align}\nonumber
{d\over dt}PE(\tilde \omega^\mu(t))=&{1\over2}\iint_{\Omega}\bigg(\partial_t(G*\tilde \omega^\mu)(t,x,y)\tilde \omega^{\mu}(t,x,y )+(G*\tilde \omega^\mu)(t,x,y)\partial_t\tilde \omega^{\mu}(t,x,y )\bigg)dxdy\\\nonumber
=&{1\over2}\iint_{\Omega}\bigg(\partial_t\tilde \psi^\mu(t,x,y)\tilde \omega^{\mu}(t,x,y )+(G*\partial_t\tilde \omega^\mu)(t,x,y)\tilde \omega^{\mu}(t,x,y )\bigg)dxdy\\
=&\iint_{\Omega}\partial_t\tilde \psi^\mu(t,x,y)\tilde \omega^{\mu}(t,x,y )dxdy.\label{PE-derivative2}
\end{align}
By \eqref{PE-derivative1}-\eqref{PE-derivative2}, we have ${d\over dt}PE(\tilde \omega^\mu(t))=\iint_{\Omega}\partial_t\tilde \psi^\mu(t,x,y)\tilde \omega^{\mu}(t,x,y )dxdy=0$.
%(3)
%\begin{align}\nonumber
%{d\over dt}\varphi*\tilde \omega^\mu(t)={1\over2}\iint_{\Omega}\nabla\tilde \psi^\mu(t, X(t),Y(t) )\cdot\nabla^\bot\tilde \psi^\mu(t,X(t),Y(t))\tilde \omega^{\mu}(t,X(t),Y(t) ) dxdy=0.
%\end{align}
\end{proof}

\subsection{Convergence of the approximate solutions and existence of weak solutions}
First, we prove the $L_{loc}^1$ convergence of the approximate solution sequence with $L^1$ vorticity control.
\begin{lemma}\label{convergence of an approximate solution sequence}
Let $\{\vec{v}^\mu\}$ be the approximate solution sequence constructed in Lemma \ref{lem-construction of an approximate solution sequence}. Then for any $T>0$ and $R>0$, there exists $\vec{v}\in L^1(\Omega_{R,T})$ such that $\max_{0\leq t\leq T}\iint_{B_R}|\vec{v}(t)|^2dxdy$ $\leq C(R,T)$, $\text{div}(\vec{v})=0$, and up to a subsequence,
\begin{align}\label{limit-for-approximate solution}
\vec{v}^{\mu}\to\vec{v} \;\text{ in }\; L^1(\Omega_{R,T}),\end{align}
and
\begin{align}\label{limit-for-vorticity-approximate solution}
\textup{curl}(\vec{v}^{\mu})=\tilde\omega^{\mu}\stackrel{*}{\rightharpoonup}\tilde \omega=\textup{curl}(\vec{v}) \;\text{ in }\; \mathcal{M}(\Omega_{R,T}),\end{align}
where $\Omega_{R,T}=[0,T]\times B_R$ and $\mathcal{M}(\Omega_{R,T})=\{\mu| \mu \text{ is a Randon measure on }\Omega_{R,T} \text{ with } \mu(\Omega_{R,T})<\infty\}$. Moreover,  $\vec{v}^{\mu}(t)\in L^1(B_R)$ and
\begin{align}\label{limit-for-approximate solution-t}
\vec{v}^{\mu}(t)\to\vec{v}(t) \;\text{ in }\; L^1(B_R)\end{align}
for any $t\geq0$.
\end{lemma}
\begin{proof} By the $L^1$ vorticity control of $\{\vec{v}^\mu\}$, there exists $\tilde\omega\in \mathcal{M}(\Omega_{R,T})$ such that, up to a subsequence, \eqref{limit-for-vorticity-approximate solution} holds.
Similar to (10.33) in \cite{Majda-Bertozzi02}, $\tilde\omega\in C([0,T],H_{\text{loc}}^{-s}(\Omega))$ and
\begin{align}\label{H-s-estimate}
\max_{0\leq t\leq T}\|\varphi\tilde\omega^\mu(t)-\varphi\tilde\omega(t)\|_{H^{-s}(\Omega)}\to 0,\quad \forall\; s>1
\end{align}
for any $\varphi\in C_0^\infty(\Omega)$, where $\tilde \omega^\mu=\text{curl} (\vec{v}^\mu)$. By Lemma \ref{lem-construction of an approximate solution sequence}, we have $\tilde\omega(0)=\tilde \omega_0$.

To prove \eqref{limit-for-approximate solution}, it suffices to show that $\{\vec{v}^\mu\}$ is a Cauchy sequence in $L^1(\Omega_{R,T})$.
Let  $\rho,
 \rho_s,
(1-\rho_{s})_{>0}$ and
 $(1-\rho_{s})_{<0}$
be given in
\eqref{def-cut off functions}.
Define   $\tilde \rho_s(x,y)=\rho\left({\sqrt{x^2+y^2}\over s}\right)$ for $(x,y)\in \Omega$. Let $\delta\in(0,\pi)$ be small enough and $R'>\delta$. Then we split $\vec{v}^{\mu_1}-\vec{v}^{\mu_2}$ into five terms:
\begin{align}\nonumber
&\vec{v}^{\mu_1}-\vec{v}^{\mu_2}=K\ast(\tilde\omega^{\mu_1}-\tilde\omega^{\mu_2})\\\nonumber
=&
(\tilde\rho_\delta K)\ast(\tilde\omega^{\mu_1}-\tilde\omega^{\mu_2})+((\rho_{R'}-\tilde \rho_\delta)K)\ast(\tilde\omega^{\mu_1}-\tilde\omega^{\mu_2})\\\nonumber
&+\left((1-\rho_{R'})_{>0}\left(K+\left({1\over 4\pi},0\right)\right)\right)\ast(\tilde\omega^{\mu_1}-\tilde\omega^{\mu_2})\\\nonumber
&+\left((1-\rho_{R'})_{<0}\left(K-\left({1\over 4\pi},0\right)\right)\right)\ast(\tilde\omega^{\mu_1}-\tilde\omega^{\mu_2})\\\nonumber
&+\left(-(1-\rho_{R'})_{>0}+(1-\rho_{R'})_{<0}\right)\left({1\over 4\pi},0\right)\ast(\tilde\omega^{\mu_1}-\tilde\omega^{\mu_2})\\\label{vkappa1-vkappa2}
:=&I_1(\mu_1,\mu_2)+I_2(\mu_1,\mu_2)+I_3(\mu_1,\mu_2)+I_4(\mu_1,\mu_2)+I_5(\mu_1,\mu_2).
\end{align}
\if0
\begin{align}\label{vkappa1-vkappa2}
\vec{v}^{\mu_1}-\vec{v}^{\mu_2}=K\ast(\tilde\omega^{\mu_1}-\tilde\omega^{\mu_2})=&
(\tilde\rho_\delta K)\ast(\tilde\omega^{\mu_1}-\tilde\omega^{\mu_2})+((\rho_{R'}-\tilde \rho_\delta)K)\ast(\tilde\omega^{\mu_1}-\tilde\omega^{\mu_2})\\\nonumber
&+\left((1-\rho_{R'})_{>0}K+\left({1\over 4\pi},0\right)\right)\ast(\tilde\omega^{\mu_1}-\tilde\omega^{\mu_2})\\\nonumber
&+\left((1-\rho_{R'})_{<0}K-\left({1\over 4\pi},0\right)\right)\ast(\tilde\omega^{\mu_1}-\tilde\omega^{\mu_2})\\\nonumber
:=&I_1(\mu_1,\mu_2)+I_2(\mu_1,\mu_2)+I_3(\mu_1,\mu_2)+I_4(\mu_1,\mu_2).
\end{align}
\fi
%where we used $\iint_{\Omega}\tilde\omega_0^{\kappa}dxdy=2\pi\int_{\mathbb{R}}J_{\kappa}*\tilde\omega_{0,0}dy=-4\pi$, and thus
%$\iint_{\Omega}(\tilde\omega^{\kappa_1}-\tilde\omega^{\kappa_2})dxdy=\iint_{\Omega}(\tilde\omega_0^{\kappa_1}-\tilde\omega_0^{\kappa_2})dxdy=0$.
By \eqref{K-near-0} and
the  $L^1$  vorticity control of $\{\vec{v}^\mu\}$  in Lemma \ref{lem-construction of an approximate solution sequence}, we have
\begin{align}\nonumber
\|I_1(\mu_1,\mu_2)\|_{L^1(\Omega_{R,T})}\leq& \|(\tilde\rho_\delta K)\|_{L^1(\Omega)}\|\tilde\omega^{\mu_1}-\tilde\omega^{\mu_2}\|_{L^1(\Omega\times[0,T])}\\\label{I1-estimate}
\leq& C(T)\iint_{\sqrt{x^2+y^2}<2\delta}|K(x,y)|dxdy=C(T)\delta.
\end{align}
By \eqref{K-near-pm-infty} and the  $L^1$  vorticity control of $\{\vec{v}^\mu\}$, we have
\begin{align}\nonumber
&\|I_3(\mu_1,\mu_2)\|_{L^1(\Omega_{R,T})}\\\nonumber
\leq&C(R,T)\left\|\left((1-\rho_{R'})_{>0}\left(K+\left({1\over 4\pi},0\right)\right)\right)\ast(\tilde\omega^{\mu_1}(t)-\tilde\omega^{\mu_2}(t))\right\|_{L^{\infty}(\Omega)}\\\nonumber
\leq& C(R,T)\left\|(1-\rho_{R'})_{>0}\left(K+\left({1\over 4\pi},0\right)\right)\right\|_{L^{\infty}(\Omega)}\|\tilde\omega^{\mu_1}(t)-\tilde\omega^{\mu_2}(t)\|_{L^1(\Omega)}\\\label{I3-estimate}
\leq &C(R,T) R'^{-1}
\end{align}
for $R'>0$ (independent of $\mu_1, \mu_2$) sufficiently large.
Similarly,
\begin{align}\label{I4-estimate}
\|I_4(\mu_1,\mu_2)\|_{L^1(\Omega_{R,T})}\leq C(R,T) R'^{-1}
\end{align}
for $R'>0$ (independent of $\mu_1, \mu_2$) sufficiently large. Now, we fix $R'$.
To estimate $I_5(\mu_1,\mu_2)$, let $\varphi_{R'}=(-(1-\rho_{R'})_{>0}+$ $(1-\rho_{R'})_{<0})\left({1\over 4\pi},0\right)$. By the  $L^1$  vorticity control of $\{\vec{v}^\mu\}$ again, we have $\tilde \omega^{\mu}(t)\rightharpoonup\tilde \omega(t)$ in $L^1(\Omega)$ for $t>0$. This, along with the fact that  $\varphi_{R'}\in L^\infty (\Omega)$,   gives
\begin{align*}
I_5(\mu_1,\mu_2)=\iint_{\Omega}\varphi_{R'}(x-\tilde x, y-\tilde y)(\tilde\omega^{\mu_1}-\tilde\omega^{\mu_2})(t,\tilde x,\tilde y) d\tilde xd\tilde y\to0\quad \text{as}\quad \mu_1,\mu_2\to0^+
\end{align*}
 for fixed $R'$ and $(x,y,t)\in \Omega_{R,T}$. Since $|I_5(\mu_1,\mu_2)|\leq \|\tilde\omega^{\mu_1}(t)\|_{L^1(\Omega)}+\|\tilde\omega^{\mu_2}(t)\|_{L^1(\Omega)}\leq C$, by the Dominated Convergence Theorem  we have
 \begin{align}\label{I5-estimate}
\|I_5(\mu_1,\mu_2)\|_{L^1(\Omega_{R,T})}\to0 \;\;\text{as}\;\; \mu_1,\mu_2\to0^+.
\end{align}
\if0
let $\varphi=\left(-(1-\rho_{R'})_{>0}+(1-\rho_{R'})_{<0}\right)\left({1\over 4\pi},0\right)$. For any $\varepsilon>0$, there exists $R_*>0$ such that $\|\tilde\omega^{\mu_1}(0)\|_{L^1(B_{\tilde R}^c)}+\|\tilde\omega^{\mu_2}(0)\|_{L^1(B_{\tilde R}^c)}\leq 2\|\tilde\omega(0)\|_{L^1(B_{\tilde R}^c)}\leq \varepsilon/|\Omega_{R,T}|$ for $\tilde R>R_*$.
\fi

 \if0
 and similar to \eqref{y-omega-conserved}, we have
\begin{align}\nonumber
{d\over dt}\varphi*\tilde \omega^\mu(t)=&\iint_{\Omega}\varphi'(y-Y(t))\partial_x\psi^\mu(t,X(t),Y(t))\tilde\omega^{\mu}(t,X(t),Y(t))  dxdy\\\nonumber
=&{1\over2}\partial_x\iint_{\Omega}\varphi'(y-\tilde y)|\nabla\psi^\mu|^2(t,\tilde x,\tilde y)  d\tilde xd\tilde y=0.
\end{align}
Thus, $|(\varphi*(\tilde\omega^{\mu_1}-\tilde\omega^{\mu_2}))(t,x,y)|=|(\varphi*(\tilde\omega^{\mu_1}-\tilde\omega^{\mu_2}))(0,x,y)|\leq \|\varphi\|_{L^\infty(\Omega)}\|\tilde\omega^{\mu_1}(0)-\tilde\omega^{\mu_2}(0)\|_{L^1(\Omega)}$ for $(t,x,y)\in\Omega_{R,T}$, and
\begin{align}\label{I5-estimate}
&\|I_5(\mu_1,\mu_2)\|_{L^1(\Omega_{R,T})}\leq C(R,T)\|\varphi*(\tilde\omega^{\mu_1}-\tilde\omega^{\mu_2})\|_{L^\infty(\Omega_{R,T})}\\\nonumber
\leq& C(R,T)\|\varphi\|_{L^\infty(\Omega)}\|\tilde\omega^{\mu_1}(0)-\tilde\omega^{\mu_2}(0)\|_{L^1(\Omega)}\to0 \;\;\text{as}\;\; \mu_1,\mu_2\to0^+
\end{align}
 by Lemma \ref{tilde-omega0-kappa-properties} (4).
 \fi

 By \eqref{H-s-estimate}, for $(x,y,t)\in \Omega_{R,T}$ we have
\begin{align}\nonumber
&|I_2(\mu_1,\mu_2)|\leq \|(\rho_{R'}-\tilde \rho_\delta)K\|_{H^s(\Omega)}\|\rho_{2(R'+R)}(\tilde\omega^{\mu_1}(t)-\tilde\omega^{\mu_2}(t))\|_{H^{-s}(\Omega)}\to0\\\label{I2-estimate}
\Rightarrow&\|I_2(\mu_1,\mu_2)\|_{L^1(\Omega_{R,T})}\to0 \;\;\text{as}\;\; \mu_1,\mu_2\to0^+,
\end{align}
where $s>1$ and we used $(\rho_{R'}-\tilde \rho_\delta)K\in C_0^{\infty}(\Omega)$. Combining \eqref{vkappa1-vkappa2}-\eqref{I2-estimate}, taking $\delta>0$ sufficiently small and $R'>0$ sufficiently large, we obtain that $\{\vec{v}^\mu\}$ is a Cauchy sequence in $L^1(\Omega_{R,T})$. For any $t\geq0$, the proof of
$\vec{v}^{\mu}(t)\in L^1(B_R)$ and \eqref{limit-for-approximate solution-t} is the same as above.
 $\max_{0\leq t\leq T}\iint_{B_R}|\vec{v}(t)|^2dxdy\leq C(R,T)$ follows from \eqref{uniform L2 bound v kappa}.
\end{proof}
Now, we prove the existence of weak solution to the 2D Euler equation with initial vorticity $\tilde \omega_0\in Y_{non}$.
%Here, we essentially use the property that  $\tilde \omega_0\in L^1\cap L^2(\Omega)$.
\begin{Theorem}\label{existence of weak solution to the 2D Euler equation-thm}
Let $\{\vec{v}^\mu\}$ be the approximate solution sequence constructed in Lemma \ref{lem-construction of an approximate solution sequence}.
Then for any $R, T>0$, there exists $\vec{v}\in L^2(\Omega_{R,T})$ such that
\begin{align*}
\vec{v}^\mu\to \vec{v} \;\text{ in }\;L^2(\Omega_{R,T}),
\end{align*}
 and $\vec{v}$ is a weak solution to the 2D Euler equation.
   Moreover,  $\vec{v}^{\mu}(t)\in L^2(B_R)$ and
\begin{align}\label{limit-for-approximate solution-L2-t}
\vec{v}^{\mu}(t)\to\vec{v}(t) \;\text{ in }\; L^2(B_R)\end{align}
for any $t\geq0$.  Consequently, for any initial vorticity $\tilde \omega_0\in Y_{non}$, there exists  $\vec{v}\in L^2(\Omega_{R,T})$ such that
 $\textup{curl}(\vec{v}(0))=\tilde\omega_0$ and $\vec{v}$ is a weak solution to the 2D Euler equation.
\end{Theorem}
\begin{proof}
By Proposition 25 in \cite{Milisic-Razafison16} and the fact that $\tilde \omega^{\mu}(t)\in L^2(\Omega)$ for $t\geq0$, there exists $\varphi^{\mu}(t)\in W_0^{2,2}(\Omega)$ such that $\psi=\varphi^{\mu}(t)$ solves $-\Delta\psi=\tilde \omega^{\mu}(t)$, where $W_0^{2,2}(\Omega)=\{\phi|(1+|y|^2)^{-1}\phi\in L^2(\Omega), (1+|y|^2)^{-{1\over2}}\nabla\phi\in L^2(\Omega),D^2\phi\in L^2(\Omega)\}$.  Then there exists $c_1, c_2\in \mathbb{R}$ and $d_{1j}, d_{2j}\in \mathbb{C}$, $ j\neq0$, such that $
(G\ast\tilde \omega^{\mu})(t)=\varphi^{\mu}(t)+\sum_{j\neq0} e^{ijx}(d_{1j}e^{jy}+d_{2j}e^{-jy})+c_1y+c_2.
$
We claim that $d_{1j}, d_{2j}=0$ for $ j\neq0$. In fact,
 %by Proposition 4.4 in  \cite{Milisic-Razafison13}, $G=G_1+G_2$, where $G_1\in L^2(\Omega)$ and $G_2(x,y)=-{1\over2} |y|$. Then
\begin{align*}
&|(G\ast\tilde \omega^{\mu})(t)|=|(G_1\ast\tilde \omega^{\mu})(t)|+|(G_2\ast\tilde \omega^{\mu})(t)|\\
\leq& \|G_1\|_{L^2(\Omega)}\|\tilde  \omega^{\mu}(t)\|_{L^2(\Omega)}+C|y|\|\tilde  \omega^{\mu}(t)\|_{L^1(\Omega)}+C\|y\tilde  \omega^{\mu}(t)\|_{L^1(\Omega)}
\end{align*}
since $\tilde \omega^\mu(t)\in L^1\cap L^2(\Omega)$ and  $y\tilde \omega^\mu(t)\in L^1(\Omega)$ by Corollary \ref{y-tilde-omega-pseudoenergy-conserved} (1). Thus, $G\ast\tilde \omega^{\mu}(t)=\varphi^{\mu}(t)+c_1y+c_2.$
By the weighted Calderon-Zygmund inequality \cite{Sauer2014},  we have
\begin{align*}
\|\nabla \vec{v}^{\mu}(t)\|_{L^2(\Omega)}=\|D^2(G\ast\tilde \omega^{\mu})(t)\|_{L^2(\Omega)}=\|D^2\varphi^{\mu}(t)\|_{L^2(\Omega)}\leq C\|\tilde \omega^{\mu}(t)\|_{L^2(\Omega)}\leq C
\end{align*}
for $t\geq0$.
By \eqref{uniform L2 bound v kappa}, $\|\vec{v}^{\mu}(t)\|_{L^2(B_R)}\leq C(R)\|\tilde \omega_0\|_{L^2(\Omega)}\leq C(R)$ for $t\geq0$. By Sobolev embedding $H^1(B_R)\hookrightarrow L^q(B_R)$ for $2< q<\infty$, we have
\begin{align*}
\|\vec{v}^{\mu}(t)\|_{L^q(B_R)}\leq C\|\vec{v}^{\mu}(t)\|_{H^1(B_R)}\leq C(R)\Longrightarrow\|\vec{v}^{\mu}\|_{L^q(\Omega_{R,T})}\leq C(R,T).
\end{align*}
This, along with \eqref{limit-for-approximate solution}, implies that there exists $\lambda\in(0,1)$ such that
\begin{align*}
\|\vec{v}^{\mu}-\vec{v}\|_{L^2(\Omega_{R,T})}\leq C\|\vec{v}^{\mu}-\vec{v}\|_{L^1(\Omega_{R,T})}^{1-\lambda}
\|\vec{v}^{\mu}-\vec{v}\|_{L^q(\Omega_{R,T})}^{\lambda}\to 0
\end{align*}
as $\mu\to 0^+$.
Similarly, for any $t\geq0$, we have  by \eqref{limit-for-approximate solution-t} that there exists $\lambda'\in(0,1)$ such that
$
\|\vec{v}^{\mu}(t)-\vec{v}(t)\|_{L^2(B_R)}\leq C\|\vec{v}^{\mu}(t)-\vec{v}(t)\|_{L^1(B_R)}^{1-\lambda'}
\|\vec{v}^{\mu}(t)-\vec{v}(t)\|_{L^q(B_R)}^{\lambda'}\to 0.
$
With the $L^2$ convergence of $\{\vec{v}^{\mu}\}$, one can verify that $\vec{v}$ is a weak solution of the 2D Euler equation by a similar argument to (A)-(C) in the proof of Theorem 10.2 in \cite{Majda-Bertozzi02}.
\end{proof}
\begin{Corollary}\label{vorticity L123}
Let $\vec{v}$ be the weak solution (obtained in Theorem \ref{existence of weak solution to the 2D Euler equation-thm}) to the 2D Euler  equation with the initial data $\tilde \omega(0)=\tilde \omega_0\in Y_{non}$, and $\tilde \omega(t)=\curl(\vec{v}(t))$ for $t\geq0$.
Then up to a subsequence,
\begin{align}\label{tilde-omega-kappa-weak convergence L1L2}
\tilde \omega^{\mu}(t)\rightharpoonup\tilde \omega(t) \text{ in }L^j(\Omega),\;\;\;\;y\tilde \omega^{\mu}(t)\rightharpoonup y\tilde \omega(t) \text{ in }L^1(\Omega),
\end{align}
$
\|\tilde \omega(t)\|_{L^j(\Omega)}\leq \|\tilde\omega (0)\|_{L^j(\Omega)}
$, $\|y\tilde \omega(t)\|_{L^1(\Omega)}\leq C(t)$,
and $\tilde \omega(t)\leq 0$ almost everywhere on $\Omega$
for all $t\geq0$ and $j=1,2$.
\end{Corollary}
\begin{proof}
By Corollary \ref{y-tilde-omega-pseudoenergy-conserved} (1) and the $L^j$ vorticity control of the approximate solution sequence $\{\vec{v}^{\mu}\}$, we obtain \eqref{tilde-omega-kappa-weak convergence L1L2} for $t\geq0$. It then follows from Lemma \ref{tilde-omega0-kappa-properties} (4) that
\begin{align*}
\|\tilde \omega(t)\|_{L^j(\Omega)}\leq \liminf_{\mu\to0^+}\|\tilde\omega^{\mu} (t)\|_{L^j(\Omega)}=\liminf_{\mu\to0^+}\|\tilde\omega^{\mu} (0)\|_{L^j(\Omega)}=\|\tilde\omega (0)\|_{L^j(\Omega)}
\end{align*}
for $j=1,2$. By Corollary \ref{y-tilde-omega-pseudoenergy-conserved} (1),
$\|y\tilde \omega(t)\|_{L^1(\Omega)}\leq \liminf_{\mu\to0^+}\|y\tilde\omega^{\mu} (t)\|_{L^1(\Omega)}\leq C(t)$.
Suppose that there exist $t_0>0$ and $E_0\subset\Omega$ such that $|E_0|>0$ and   $\tilde \omega(t_0)>0$ on $E_0$. We assume that $|E_0|<\infty$ without loss of generality. Let $\varphi\equiv1 $ on $E_0$ and $\varphi\equiv0$ on $\Omega\setminus E_0$. Then $\varphi\in L^2(\Omega)$ and
\begin{align*}
0<\iint_{E_0}\tilde \omega(t_0)dxdy=&\iint_{\Omega}\tilde \omega(t_0)\varphi dxdy=\lim_{\mu\to0^+}\iint_{\Omega}\tilde \omega^\mu(t_0)\varphi dxdy\\
=&\lim_{\mu\to0^+}\iint_{E_0}\tilde \omega^\mu(t_0)dxdy\leq 0,
\end{align*}
which is a contradiction.
\end{proof}
\end{appendix}

\section*{Acknowledgement}
Z. Lin is partially supported by  the NSF under  Grants DMS-1715201 and DMS-2007457. H. Zhu is  partially supported by National Key R $\&$ D Program of China under Grant 2021YFA1002400, NSF
of China under Grant 12101306 and NSF of Jiangsu Province, China under Grant BK20210169.


\begin{thebibliography}{99}
\bibitem{Adams75} R. A. Adams,  Sobolev spaces, Pure and Applied Mathematics, Vol. 65. Academic Press [Harcourt Brace Jovanovich, Publishers], New York-London, 1975. xviii+268 pp.
\bibitem{Arnold65} V. I. Arnol$'$d,  Conditions for nonlinear stability of stationary plane curvilinear flows of an ideal
fluid, Sov. Math. Dokl., 6 (1965), 773-776.
\bibitem{Arnold69}  V. I. Arnol$'$d, On an a priori estimate in the theory of hydrodynamical stability, Am. Math. Soc.
Transl.: Series 2, 79 (1969), 267-269.
\bibitem{BD01} A. Barcilon, P. G. Drazin,  Nonlinear waves of vorticity, Stud. Appl. Math., 106 (2001),  437-479.
%\bibitem{Bateman32} H. Bateman,  Partial Differential Equations of Mathematical Physics, Cambridge University Press, 1932.
\bibitem{BKM84}  J. T. Beale, T. Kato, A. Majda,  Remarks on the breakdown of smooth solutions for the 3-D Euler equations, Comm. Math. Phys., 94 (1984),  61-66.
\bibitem{Benjamin-Feir1967} T. B. Benjamin, J. E. Feir,  The disintegration of wave trains on deep water. Part 1.
Theory, J. Fluid Mech. 27 (1967), 417-437.
\bibitem{Berti-Maspero-Ventura2022} M. Berti,  A. Maspero, P.  Ventura,  Full description of Benjamin-Feir instability of Stokes waves in deep water, Invent. Math., 230 (2022), 651-711.
 %\bibitem{Bieberrach1916}  L. Bieberrach,  $\Delta u = e^u$ und die automorphen Funktionen, Math. Ann., 77 (1916), 173-212.
\bibitem{Bogatov-Kichenassamy22} E. M. Bogatov, S. Kichenassamy, The solution of Liouville's equation (1850, 1853) and its impact, 	arXiv: 2205.04246.

\bibitem{Bondeson1983} A. Bondeson, Linear analysis of the coalescence instability,
 Phys. Fluids,  26  (1983), 1275-1278.
 \bibitem{Bridges-Mielke1995}
T. J. Bridges, A. Mielke,  A proof of the Benjamin-Feir instability, Arch. Rational Mech. Anal., 133 (1995),  145-198.
 %\bibitem{Brodetsky1924}
%S. Brodetsky,  Vortex Motion, Proc. 1st Int. Cong. Appl. Math., Delft,   374-379, 1924.
\bibitem{Bronski-Hur-Johnson2016}
 J. C. Bronski, V. M. Hur, M. A. Johnson, Modulational instability in equations of
KdV type. New approaches to nonlinear waves, pp. 83-133. Lecture Notes in Physics,
vol. 908. Springer, Cham, 2016.
\bibitem{Byerly59} W. E. Byerly,  An Elementary Treatise on Fourier's Series, and Spherical, Cylindrical, and Ellipsoidal Harmonics, with Applications to Problems in Mathematical Physics. New York: Dover, 1959.
\bibitem{Chen-Su2020}   G. Chen, Q. Su, Nonlinear modulational instabililty of the Stokes waves in 2d full water
waves, arXiv: 2012.15071.
\bibitem{Courant-Hilbert53}   R. Courant, D. Hilbert, Methods of Mathematical Physics, Interscience Publishers, New York, 1953.
\bibitem{Constantin-Crowdy-Krishnamurthy-Wheeler2021} A. Constantin, D. G. Crowdy, V. S. Krishnamurthy, M. H. Wheeler,  Stuart-type polar vortices on a rotating sphere, Discrete Contin. Dyn. Syst., 41 (2021),  201-215.
\bibitem{Constantin-Krishnamurthy2019}
 A. Constantin, V. S. Krishnamurthy,  Stuart-type vortices on a rotating sphere, J. Fluid Mech., 865 (2019), 1072-1084.

     \bibitem{Crowdy97}  D. G. Crowdy,  General solutions to the 2D Liouville equation, Internat. J. Engrg. Sci., 35 (1997), 141-149.
 \bibitem{Crowdy04} D. G. Crowdy,  Stuart vortices on a sphere, J. Fluid Mech., 498 (2004), 381-402.



 \bibitem{Dacorogna} B. Dacorogna, Weak continuity and weak lower semicontinuity of non-linear functionals, Vol. 922, Springer, 2006.
\bibitem{Dauxois-Fauve-Tuckerman1996} T. Dauxois, S. Fauve, L. Tuckerman,  Stability of periodic arrays of vortices, Phys. Fluids, 8 (1996), 487-495.
\bibitem{DiPerna-Majda87} R. J. DiPerna,  A. J. Majda, Concentrations in regularizations for 2-D incompressible flow, Comm. Pure Appl. Math., 40 (1987),  301-345.
 \bibitem{Dominguez-Heuer-Sayas11}  V. Dominguez, N. Heuer, F.-J. Sayas,  Hilbert scales and Sobolev spaces defined by associated Legendre functions,
    J. Comput. Appl. Math.,
235 (2011),  3481-3501.
\bibitem{Dunkerton-Montgomery-Wang2009}
T. J. Dunkerton, M. T. Montgomery, Z. Wang,  Tropical cyclogenesis in a tropical wave critical layer: easterly waves, Atmos. Chem. Phys., 9  (2009), 5587-5646.
%\bibitem{Emden1907} R. Emden, Gaskugeln: Anwendungen der mechanischen warmetheorie auf kosmologische und
%meteorologische Probleme, Leipzig und Berlin: B.G. Teubner, 1907, 497s.
\bibitem{Fadeev et al-1965} V. M. Fadeev, I. F. Kvabtskhava, N. N. Komarov,
 Self-focusing of local plasma currents,
Nucl. Fusion, 5 (1965), 202-209.
\bibitem{Finn-Kaw1977}  J. M. Finn, P. K. Kaw, Coalescence instability of magnetic islands, Phys. Fluids,  20 (1977), 72-78.
\bibitem{Fleischer1998} J. Fleischer, Nonlinear evolution of self-gravitating plasmas,  Phys. Scr.,  T74 (1998), 86-88.
%\bibitem{Granier-Tassi20}
%C. Granier, E. Tassi, Linear stability of magnetic vortex chains in a plasma in the presence of equilibrium electron temperature anisotropy, J. Phys. A, 53 (2020),  385702, 30 pp.
%\bibitem{Hilbert1900}
%D. Hilbert,  ``Mathematische Probleme", Nachrichten von der K$\ddot{o}$niglichen Gesellschaft der Wissenschaften zu G$\ddot{o}$ttingen, Mathematisch-Physikalische Klasse (in German) (3): 253-297, JFM 31.0068.03, 1900.

\bibitem{holm1986nonlinear}
D. D. Holm, J. E. Marsden, T. Ratiu, Nonlinear stability of the
Kelvin-Stuart cat's eyes flow, In: AMS. 1986.
\bibitem{holm1985nonlinear} D. D. Holm, J. E. Marsden, T. Ratiu, A. Weinstein,  Nonlinear stability of fluid and plasma equilibria, Phys. Rep., 123 (1985),  116 pp.
\bibitem{Jin-Liao-Lin2019} J. Jin, S. Liao, Z. Lin,  Nonlinear modulational instability of dispersive PDE models, Arch. Ration. Mech. Anal., 231 (2019),  1487-1530.
\bibitem{Kato1986nonlinear}  T. Kato, Remarks on the Euler and Navier-Stokes equations in $\mathbf{R}^2$, Nonlinear functional analysis and its applications, Part 2 (Berkeley, Calif., 1983), 1-7, Proc. Sympos. Pure Math., 45, Part 2, Amer. Math. Soc., Providence, RI, 1986.
 %\bibitem{Keller1957} J. B. Keller,  On solutions of $\Delta u=f(u)$, Comm. Pure Appl. Math., X (1957), 503-510.
\bibitem{kelly1967stability}
R. E. Kelly, On the stability of an inviscid shear layer which is periodic in space and
time,  J. Fluid Mech., 27.4 (1967),  657-689.
\bibitem{kelvin1880disturbing}
L. Kelvin, On a disturbing infinity in Lord Rayleigh's solution for waves in a
plane vortex stratum,  Nature, 23.1 (1880), 45-46.
\bibitem{Klaassen-Peltier1987}
G. P. Klaassen, W. R. Peltier,   Secondary instability and transition in finite amplitude
Kelvin-Helmholtz billows, In Proc. Third Intl Symp. on Stratified Flows, 3-5 Feb. 1987,
Pasadena, California, Vol. I.
\bibitem{Klaassen-Peltier1989}
 G. P. Klaassen, W. R. Peltier,  The role of transverse secondary instabilities in the evolution of free shear layers, J. Fluid Mech., 202 (1989), 367-402.
 \bibitem{Klaassen-Peltier1991}
 G. P. Klaassen, W. R. Peltier, The influence of stratification on secondary instability in
free shear layers, J. Fluid Mech., 227 (1991), 71-106.
 \bibitem{Krishnamurthy2019}
V. S. Krishnamurthy, M. H. Wheeler, D. G. Crowdy,  A. Constantin, Steady
point vortex pair in a field of Stuart-type vorticity, J. Fluid Mech., 874 (2019), R1, 11 pp.
 \bibitem{Krishnamurthy2021}
V. S. Krishnamurthy, M. H. Wheeler, D. G. Crowdy,  A. Constantin, Liouville chains: new hybrid vortex equilibria of the two-dimensional Euler equation, J. Fluid Mech., 921 (2021), Paper No. A1, 35 pp.
\bibitem{lamb1932hydrodynamics}
H. Lamb, Hydrodynamics, 6th edition, C.U.P, 1932.

\bibitem{Laugesen2011}
R. S. Laugesen, Spectural theory of partial differential equations, Lecture Notes, 2011.

\bibitem{Liouville1853} J.  Liouville, Sur l$'$\'{e}quation aux diff\'{e}rences partielles ${d^2\log\lambda\over du dv}\pm{\lambda\over 2a^2}=0$, J. Math. Pures Appl., 18 (1853), 71-72.
\bibitem{lin2003instability}
Z. Lin, Instability of some ideal plane flows, SIAM J. Math. Anal., 35 (2003),  318-356.

\bibitem{lin2004some}
Z. Lin, Some stability and instability criteria for ideal plane flows,  Comm. Math. Phys., 246 (2004),  87-112.
\bibitem{lin2021linear}
Z. Lin,  Linear instability of Vlasov-Maxwell systems revisited-a Hamiltonian
approach,   Kinet. Relat. Models, 15 (2022), 663-679.
\bibitem{lin2022instability}
Z. Lin, C. Zeng,  Instability, index theorem, and exponential trichotomy for linear Hamiltonian PDEs, Mem. Amer. Math. Soc., 275 (2022), no. 1347, v+136 pp.
\bibitem{lin2020separable}
Z. Lin, C. Zeng, Separable Hamiltonian PDEs and Turning point
principle for stability of gaseous stars,  Comm. Pure Appl. Math., 75 (2022),  2511-2572.
%\bibitem{lichtenstein1913}
%L. Lichtenstein,  Int\'egration de l$'$\'equation $\Delta^2u = ke^u$ sur une surface ferm\'ee,
%Comptes Rendus, 157 (1913), 1508-1511.
%\bibitem{lichtenstein1915}
%L. Lichtenstein, Integration der Differentialgleichungen $\Delta_2u = ke^u$ aub geschlossenen Fl$\ddot{a}$chen: Methode der unendlichvielen Variabeln, Acta Math., 140 (1915), 1-34.

%\bibitem{Longcope-Strauss1993}
%D. W. Longcope, H. R. Strauss,
%The coalescence instability and the
%development of current sheets in two-dimensional magnetohydrodynamics,
%Phys. Fluids B, 5 (1993), 2858-2869.
%\bibitem{Lutzen2012} J.  L$\ddot{u}$tzen, Joseph Liouville 1809-1882: Master of pure and applied mathematics, Springer
%Science \& Business Media, 2012, xix+884 p.
\bibitem{Majda-Bertozzi02}
A. J. Majda, A. L. Bertozzi,  Vorticity and incompressible flow, Cambridge Texts in Applied Mathematics, 27. Cambridge University Press, Cambridge, 2002. xii+545 pp.
\bibitem{Martin2018}
C. I. Martin,   On the vorticity of mesoscale ocean currents,
Oceanography, 31 (2018), 28-35.
\bibitem{Milisic-Razafison13}
V. Milisic, U. Razafison, Weighted Sobolev spaces for the Laplace equation in periodic infinite
strips, 2013, ffhal-00728408v2f.
\bibitem{Milisic-Razafison16}
V. Milisic, U. Razafison, Weighted $L^p$-theory for Poisson, biharmonic and Stokes problems on periodic
unbounded strips of $\mathbb{R}^n$, Ann. Univ. Ferrara Sez. VII Sci. Mat., 62 (2016),  117-142.
\bibitem{Morrey1966} C. B. Morrey, Multiple integrals in the calculus of variations,
Springer-Verlag, Berlin, 1966.
\bibitem{Nguyen-Strauss2023} H. Q. Nguyen, W. A.  Strauss,  Proof of modulational instability of Stokes waves in deep water, Comm. Pure Appl. Math., 76 (2023), 1035-1084.
%\bibitem{Picard1}
 %E. Picard,  De l$'$\'equation $\Delta u = ke^u$ sur une surface de Riemann ferm\'ee, J. Math. Pures Appl. (4), 9 (1893), 273-291.

%\bibitem{Picard3}
%E. Picard,  De l$'$int\'egration de l$'$\'equation $\Delta u = e^u$ sur une surface de Riemann
%ferm\'ee,  J. Reine Angew. Math., 130 (1905), 243-258.

\bibitem{pierrehumbert1982two}
R. T. Pierrehumbert, S. E. Widnall, The two- and three-dimensional instabilities of a spatially periodic shear layer, J. Fluid Mech., 114 (1982), 59-82.
%\bibitem{Poincare18998} H. Poincar\'e, Les fonctions fuchsiennes et l$'$\'equation $\Delta u = e^u$, J. Math. Pures Appl.
%(5), 5 (1898), 137-230.
 %\bibitem{Polyakov81} A. M. Polyakov,  Quantum geometry of bosonic strings, Phys. Lett. B, 103 (1981),  207-210.
 \bibitem{Pontin-Priest2022} D. I. Pontin, E. R. Priest,  Magnetic reconnection: MHD theory and modelling, Living Reviews in Solar Physics,  19.1 (2022), 1-202.
\bibitem{Pritchett-Wu1979} P. L. Pritchett, C. C. Wu, Coalescence of magnetic islands, Phys. Fluids,  22 (1979),  2140-2146.
\bibitem{Priest1985} E. R. Priest, The magnetohydrodynamics of current sheets, Rep. Prog. Phys., 48 (1985), 955-1090.
\bibitem{Priest-Forbes2000}  E. R. Priest, T. G. Forbes, Magnetic Reconnection: MHD Theory and Applications,  Cambridge University Press, 2000.
\bibitem{Richardson1921}
O. W. Richardson,  The Emission of Electricity from Hot Bodies, 2nd ed. London: Longmans,
Green and Co., 1921. 320 p.
\bibitem{Rossi-Doorly-Kustrin2013}
L. Rossi, D. Doorly, D. Kustrin,
Lamination, stretching, and mixing in cat's eyes flip sequences with varying periods, Phys. Fluids 25 (2013), 073604.
\bibitem{Sakajo2019}
T. Sakajo,   Exact solution to a Liouville equation with Stuart vortex distribution on the
surface of a torus, Proc. A., 475 (2019), 20180666.
\bibitem{Sauer2014} J. Sauer, Weighted resolvent estimates for the spatially periodic Stokes equations, Ann.
Univ. Ferrara, (2014), 1-22.
\bibitem{Schindler2006} K. Schindler, Physics of Space Plasma Activity, Cambridge University Press, 2006.
\bibitem{Schmid-Burgk1965}
J. Schmid-Burgk, Zweidimensionale selbstkonsistente L$\ddot{o}$sungen station$\ddot{a}$ren Wlassov-gleichung f$\ddot{u}$r Zweikomponentenplasmen, Ludwig-Maximilians-Universit$\ddot{a}$t, M$\ddot{u}$nchen, Diplomarbeit, 1965.
\bibitem{Shukla-Sen1996}
P. K. Shukla, A. Sen, Dusty vortex streets in Saturn's rings, Physica Scripta, T63 (1996), 275-276.
\bibitem{stuart1967finite}
J. T. Stuart, On finite amplitude oscillations in laminar mixing layers,  J. Fluid Mech., 29.3 (1967),  417-440.
\bibitem{Suetin2001}
 P. K. Suetin,   Ultraspherical polynomials, Encyclopedia of Mathematics, EMS Press, 2001.
 \bibitem{Tabeling-Perrin-Fauve1987}
P. Tabeling, B. Perrin, S. Fauve, Instability of a linear array of forced vortices, Europhys. Lett., 3 (1987),  459-465.
\bibitem{Tassi2022} E. Tassi, Formal stability in Hamiltonian fluid models for
plasmas, J. Phys. A: Math. Theor., 55 (2022), 413001 (82pp).
\bibitem{Taylor2018} M. Taylor, Curvature, conformal mapping, and 2D stationary fluid flows, Preprint, 2018, 1-7.
%\bibitem{Walker1915}
%G. W. Walker,   Some problems illustrating the forms of nebulae, Proc. Roy. Soc. A, 91 (1915), 410-420.

 \bibitem{Weidmann80}  J. Weidmann, Linear Operators in Hilbert Spaces, Grad. Texts in Math., Vol. 68, Springer-Verlag, Berlin, 1980.
 %\bibitem{Wittich44}
 %H. Wittich, Ganze L$\ddot{o}$sungen der Differentialgleiehung $\Delta u=e^u$, Math. Z., 49 (1944), 579-582.
 \bibitem{Yoon20} J. Yoon,  H. Yim,  S.-C. Kim,   Stuart vortices on a hyperbolic sphere, J. Math. Phys.,
61 (2020), 023103.
%\bibitem{yosida1995functional}
%K. Yosida, Functional Analysis, Reprint of the sixth: edition, Classics in Mathematics,  Springer, Berlin 11.501 (1995), p. 1980.

\end{thebibliography}
\end{CJK*}
\end{document}
