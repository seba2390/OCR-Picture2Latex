\documentclass{lmcs} %%% last changed 2014-08-20

%% mandatory lists of keywords 
\keywords{unification, convergent string-rewriting systems, fixed point problem, common term problem, common equation problem, conjugacy problem, common multiplier problem}

%% read in additional TeX-packages or personal macros here:
%% e.g. \usepackage{tikz}
\usepackage{hyperref}
%%\input{myMacros.tex}
%% define non-standard environments BEYOND the ones already supplied 
%% here, for example
\theoremstyle{plain}\newtheorem{satz}[thm]{Satz} %\crefname{satz}{Satz}{S\"atze}
%% Do NOT replace the proclamation environments lready provided by
%% your own.

\def\eg{{\em e.g.}}
\def\cf{{\em cf.}}


%OUR LIBRARIES HERE!
%\usepackage[utf8]{inputenc}


%\documentclass[11pt]{article}
%\pagestyle{empty}
%%%%%%%%
%\usepackage[latin1]{inputenc}
%\usepackage{epsfig, amssymb, amsmath, geometry}
%\usepackage{latexsym, graphics, graphicx}
%\usepackage{yfonts}
%\usepackage{proof}
%\usepackage{algorithmic}
\usepackage{algorithm}
\usepackage{algpseudocode}
%\usepackage{enumitem}
\usepackage{amsthm}
\usepackage{amssymb} % added for \smallsetminus in Definitions
\usepackage{amsmath} %added for aligned
\usepackage{wasysym} %added for \cent
\usepackage{epsfig, graphics, graphicx} %added for eps graphic in monadic cone theorem
\usepackage[toc,page]{appendix}%for multiple appendices
\usepackage{mathtools} % For \vdotswithin
%\usepackage{amsxtra}
%\usepackage{epsfig}
%\usepackage{color}
%\usepackage{float}
%\usepackage{authblk}
%\usepackage{url}
%\usepackage{mathptmx} 
%\usepackage[numbers]{natbib}
%\usepackage{fancyvrb} 
%\usepackage{marginnote}
%\usepackage{wasysym}
\usepackage{verbatim}%to add block comments \begin{comment}%

\usepackage{fancybox}

%\usepackage{centernot,datetime}

%\usepackage{setspace}

\usepackage{multirow}

%Tikz libraries for automata%
\usepackage{tikz}
\usetikzlibrary{automata,positioning,arrows}
\usetikzlibrary{shapes}
\newcommand\irregularcircle[2]{% radius, irregularity
  \pgfextra {\pgfmathsetmacro\len{(#1)+rand*(#2)}}
  +(0:\len pt)
  \foreach \a in {10,20,...,350}{
    \pgfextra {\pgfmathsetmacro\len{(#1)+rand*(#2)}}
    -- +(\a:\len pt)
  } -- cycle
}

\tikzset{set/.style={draw,circle,inner sep=0pt,align=center}}
%Subfigure%
\usepackage{caption}
\usepackage{subcaption}


%%%%%%%%
%\setlength{\textheight}{8.75in}
%\setlength{\columnsep}{2.0pc}
%\setlength{\textwidth}{6.0in}
%\setlength{\footheight}{0.0in}
%\setlength{\topmargin}{0.25in}
%\setlength{\headheight}{0.0in}
%\setlength{\headsep}{0.0in}
%\setlength{\parskip}{0.05in} This was uncommented
%\setlength{\oddsidemargin}{-.15in}
%\setlength{\parindent}{1pc}


\newcommand{\ignore}[1]{}
\newcommand{\undr}{\underline}
\newcommand{\ovr}{\overline}
\newcommand{\flr}{\rightarrow}
\newcommand{\fll}{\leftarrow}
\newcommand{\fflr}{\longrightarrow}
\newcommand{\fflrt}{\longleftrightarrow}
\newcommand{\thflr}{\twoheadrightarrow}

\newcommand{\lft}{\noindent}
\newcommand{\disp}[1]{\vspace{-0.5em}
   \begin{center} {#1} \end{center}\vspace{-0.5em}}

\newcommand{\Cross}{\mathbin{\tikz [x=1.4ex,y=1.4ex,line width=.2ex] \draw (0,0) -- (1,1) (0,1) -- (1,0);}}%

%\newcommand{\A}{\mathcal{A}}
%\newcommand{\E}{\mathcal{E}}
%\newcommand{\T}{\mathcal{T}}
%\newcommand{\X}{\mathcal{X}}
%\newcommand{\K}{\mathcal{K}}
%\newcommand{\R}{\mathcal{R}}
%\newcommand{\p}{\mathcal{P}}
%\newcommand{\h}{\mathcal{H}}
%\newcommand{\I}{\mathcal{I}}

\newcommand{\Var}{\mathcal{V}\!\mathit{ar}}
\newcommand{\dom}{\mathcal{D}om}
\newcommand{\ran}{\mathcal{R}an}
\newcommand{\vran}{\mathcal{V\!R}\!\mathit{an}}



\newcommand{\Prob}{\mathcal{EQ}}
\newcommand{\ueq}{=^?}

\newcommand{\Inf}[1]{{\mathcal{I}_\mathrm{#1}}}
\newcommand{\FAIL}{\mathrm{FAIL}}
\newcommand{\V}[2]{V^{\hspace{0pt}#1}_{#2}}

\newcommand{\Equiv}[2]{{[#1]}_{#2}}
\newcommand{\Proj}[2]{\succ_{#1, #2}}
\newcommand{\ProjEq}[2]{\sim_{#1, #2}}

\newcommand{\Root}{\mathrm{root}}

\newcommand{\AddFn}{{+}}
\newcommand{\MultFn}{{\times}}

\newcommand{\SubRoot}[1]{{#1}_{\mathrm{root}}}
\newcommand{\SubDep}[1]{{#1}_{\mathrm{dep}}}
\newcommand{\SubProp}[1]{{#1}_{\mathrm{prop}}}

\newcommand{\Graph}{\mathcal{G}}
\newcommand{\RootGraph}{\SubRoot{\Graph}}
\newcommand{\DepGraph}{\SubDep{\Graph}}
\newcommand{\PropGraph}{\SubProp{\Graph}}

\newcommand{\ITrivElim}{\text{\textsf{Triv-Elim}}}
\newcommand{\IVarElim}{\text{\textsf{Var-Elim}}}
\newcommand{\ISplitting}{\text{\textsf{Splitting}}}
\newcommand{\IFuncClash}{\text{\textsf{Func-Clash}}}
\newcommand{\IOccurCheck}{\text{\textsf{Occur-Check}}}
\newcommand{\IInfSplit}{\text{\textsf{Inf-Split}}}
\newcommand{\IDecomposition}{\text{\textsf{Decomposition}}}

% These are defined this way because kerning sucks for mathcal.
\newcommand{\EQ}{\mathcal{E\hspace{-1pt}Q}}

\newcommand{\Pos}{\mathcal{P}\!\mathit{os}}
\newcommand{\FPos}{\mathcal{FP}\!\mathit{os}}

\newcommand{\pa}{{\bf Pair}}
\newcommand{\enc}{{\bf Enc}}
\newcommand{\eq}{\mathcal{EQ}}
\newcommand{\bphi}{\ovr{\phi}}
\newcommand{\bh}{\ovr{h}}
\newcommand{\tdh}{\tilde{h}}

\newtheorem{propn}{Proposition}
\newtheorem{defn}{Definition}
\newtheorem{corol}{Corollary}[section]
\newtheorem{lemma}[section]{Lemma}
\newtheorem{theorem}{Theorem}

\newcommand{\vsp}{\vspace*{0.4em}}
\newcommand{\hsp}{\hspace*{2em}}
\newcommand{\hsps}{\hspace*{1em}}
\newcommand{\hspa}{\hspace*{1cm}}

\newcommand{\ssp}[1]{^{(#1)}}


\def\presuper#1#2%
  {\mathop{}%
   \mathopen{\vphantom{#2}}^{#1}%
   \kern-\scriptspace%
   #2}

\def\presub#1#2%
  {\mathop{}%
   \mathopen{\vphantom{#2}}_{#1}%
   \kern-\scriptspace%
   #2}

\def\presubsuper#1#2#3%
  {\mathop{}%
   \mathopen{\vphantom{#3}}_{#1}^{#2}%
   \kern-\scriptspace%
   #3}

%Roman Numerals%
\makeatletter
\newcommand*{\rom}[1]{\expandafter\@slowromancap\romannumeral #1@}
\makeatother
%OUR LIBRARIES END

%% due to the dependence on amsart.cls, \begin{document} has to occur
%%   BEFORE the title and author information:



\begin{document}

\title[On Problems Dual to Unification: The String-Rewriting Case]{On Problems Dual to Unification:\\
\large The String-Rewriting Case}
\titlecomment{{\lsuper*}A variant of the paper has also been published in \cite{zumrutDissertation, 2017arXiv, UNIF2017Akcam}. Some of the results reported here are a partial fulfillment of the Ph.D. requirements of the fourth author, and will be part of his dissertation.}

\author[Akcam]{Z\"{u}mr\"{u}t Ak{\c c}am}	%required
\address{The College of Saint Rose, Albany, NY, US}	%required
\email{akcamz@strose.edu}  %optional
\thanks{Thanks to Dr. Daniel J. Dougherty for his feedback}	%optional

\author[Hono]{Daniel S. Hono II}	%optional
\address{University at Albany, SUNY\\ Albany, NY, US}	%optional
\email{dhono@albany.edu}  %optional

\author[Narendran]{Paliath Narendran}	%optional
\address{University at Albany, SUNY\\ Albany, NY, US}	%optional
\email{pnarendran@albany.edu}
%\urladdr{name3@url3\quad\rm{(optionally, a web-page can be specified)}}  %optional

\author[Pulver]{Andrew Pulver}	%optional
\address{University at Albany, SUNY\\ Albany, NY, US}	%optional
\email{apulver@albany.edu}
%\urladdr{name3@url3\quad\rm{(optionally, a web-page can be specified)}}  %optional

%% etc.

%% required for running head on odd and even pages, use suitable
%% abbreviations in case of long titles and many authors:

%%%%%%%%%%%%%%%%%%%%%%%%%%%%%%%%%%%%%%%%%%%%%%%%%%%%%%%%%%%%%%%%%%%%%%%%%%%

%% the abstract has to PRECEDE the command \maketitle:
%% be sure not to issue the \maketitle command twice!

\begin{abstract}
  \noindent In this paper, we investigate problems which are dual to the
  unification problem, namely the Fixed Point~(FP) problem, Common Term~(CT) problem and
  the Common Equation~(CE) problem for string rewriting
  systems. Our main motivation is computing fixed points in
  systems, such as loop invariants in programming languages. We
  show that the fixed point~(FP) problem is reducible to the common term
  problem. Our new results are:
  (i) the fixed point problem is undecidable for finite convergent
  string rewriting systems whereas it is decidable in polynomial time for finite, convergent and dwindling string rewriting systems,
  (ii) the common term problem is undecidable
  for the class of dwindling string rewriting systems, and
  (iii) for the
  class of finite, monadic and convergent systems, the common equation
  problem is decidable in polynomial time but for the class of dwindling string rewriting systems, common equation problem is undecidable.
\end{abstract}

\maketitle

%% start the paper here:
\section*{Introduction}\label{S:one}

  Unification, with or without background theories such as
associativity and commutativity, is an area of great theoretical
and practical interest. The former problem, called \emph{equational}
or \emph{semantic} unification, has been studied from several different angles.
Here we investigate some problems that can clearly be viewed as \emph{dual}
to the unification problem.
Our main motivation for this work is theoretical, but, as explained below,
we begin with a practical application that is shared by
many fields.

In every major research field, there are variables or other parameters that change over
time. These variables are modified --- increased
or decreased --- as a
result of a change in the environment. 
Computing \emph{invariants,} or expressions whose values do not change
under a transformation, is very important in many areas 
such as Physics, e.g., invariance under the \emph{Lorentz} transformation.

In Computer Science, the issue of obtaining invariants arises
in \emph{axiomatic semantics} or \emph{Floyd-Hoare semantics},
in the context of formally proving a loop to be correct. A
\emph{loop invariant} is a
condition, over the program variables, 
that holds before and after each iteration. Our research
is partly motivated by the related question of
finding expressions, called \emph{fixed points}, whose values
will be the same before and after each iteration, i.e., will
remain unchanged as long as the iteration goes on. For instance,
for a loop whose body is \[ \text{\tt X = X + 2; Y = Y - 1;} \] the
value of the expression {\tt X + 2Y} is a fixed point.

We can formulate this problem in terms of properties of substitutions
\emph{modulo} a term rewriting system. One straightforward formulation
is as follows:\\

%\noindent
%{\large\bf Fixed Point Problem (FP)}
\subsection*{Fixed Point Problem (FP)}

\begin{description}[align=left]
\item[{\bf Input}] A substitution $\theta$ and an equational theory~$E$.
\item[{\bf Question}] Does there exist a non-ground term~$t \in T(Sig(E), \, {\dom}(\theta))$ 
such that~$\theta (t) \, \approx_E^{} t$?
\end{description}

\vspace{0.05in}
\noindent
\underline{Example 1}: Suppose~$E$ is a theory of integers which contains linear
arithmetic.\\ Let $\theta =\{x \mapsto x-2, \, y \mapsto y+1\}$ and we would like to find a term~$t$ such that 
$\theta (t) \approx_E^{} t$. Note that $x + 2y$ is such a term, since \[ \theta (x+2y) =
(x - 2) + 2*(y + 1) \approx_E^{} x+2y \] \\


\noindent
\underline{Example 2}: What is the fixed point/invariant of the given loop?
\begin{algorithm}
\caption{Fixed Point Loop}
	\begin{algorithmic}[1]
	    \State{} $\{x = X_0, y = Y_0\}$
	    \While{} $x > 0$
	    	\State{} $x = x -1$ \Comment \hspace{3pt} $\theta= \{x \mapsto x-1,\; y \mapsto y+x-1\}$
		\State{} $y = y + x $
	    \EndWhile
\end{algorithmic}
\end{algorithm}

Note that the value of the expression $y + \dfrac{x*(x-1)}{2}$ is unchanged, since

\begin{itemize}
\item Before Iteration: $y + \dfrac{x*(x-1)}{2}$
\item After Iteration: $$y + x - 1 + \dfrac{(x-1)*(x-2)}{2} = y + \dfrac{x^2 - 3x + 2 + 2(x-1)}{2} =  y + \dfrac{x*(x-1)}{2}$$
\end{itemize}

Thus $y + \dfrac{x*(x-1)}{2}$ is a fixed point of~$\theta$.

Note that fixed points may not be unique. Consider the term rewriting system \[ \{a(b(y)) \rightarrow a(y),\; c(b(z)) \rightarrow c(z) \} \] and
let $\theta = \{x \mapsto b(x)\}$. We can see that both $a(x)$ and $c(x)$ are fixed points of $\theta$.  

We plan to explore two related formulations, both of which can be 
viewed as \emph{dual} to the well-known unification problem. Unification
deals with solving symbolic equations: thus a typical input would be either two
terms, say $s$ and $t$, or an equation~$s \approx_{}^? t$. The task is to
find a substitution such that $\theta (s) \approx \theta(t)$.
For example, given two terms $s_1^{} = f(a, y)$ and $s_2^{} = f(x, b)$, where $f$~is
a binary function symbol, $a$ and $b$ are constants, and $x$ and $y$
are variables, the substitution $\sigma = \{ x \mapsto a, ~ y \mapsto
b\}$ unifies $s_1^{}$ and~$s_2^{}$, or equivalently, $\sigma$ is a unifier
for the equation~$s_1^{} =_{}^? s_2^{}$.

There are two ways to ``dualize'' the unification problem:\\

\noindent
{\large\bf Common Term Problem (CT)}:

\begin{description}[align=left]
\item [Input] Two ground\footnote{This may not be strictly necessary.} substitutions $\theta_1^{}$ and $\theta_2^{}$, and an
  equational theory~$E$. (i.e., $\vran(\theta_1) = \emptyset$ and $\vran(\theta_2) = \emptyset$ )
\item [Question] Does there exist a non-ground term~$t  \in 
T(Sig(E), \, \dom(\theta_1) \cup \dom(\theta_2))$ such that
\mbox{$\theta_1^{} (t) \, \approx_E^{} \theta_2^{} (t)$}?
\end{description} %$t \, \in \, T(Sig(E), \, \dom(\theta))$%

\ignore{
We will be considering equational theories that 
are decomposable into a set of identities~$Ax$ and
a set of rewrite rules such that \emph{equational} rewriting 
modulo~$Ax$ is convergent. The most widely used case of
equational rewriting is where $Ax$ consists of associativity and
commutativity axioms~(\emph{AC}).
The key concepts are defined in the next section.
}

\vspace{0.05in}
\noindent
\underline{Example 3}: Consider the two substitutions 
\[
\theta_{1} = \{ x
\mapsto p(a), \, y \mapsto p(b) \} \textrm{ and } \theta_{2} = \{ x \mapsto a,
\, y \mapsto b \}.
\]
If we take the term rewriting system~$R_1^{lin}$ in the Appendix A
as our background equational theory~$E$, then there exists a common term $t =
x-y$ that satisfies $\theta_{1}(t) \approx_E^{}
\theta_{2}(t)$. \[ \theta_1 ( x-y ) \approx_E^{} p(a)- p(b) \approx_E^{}
a-b \] and \[\theta_2 ( x-y ) \approx_E^{} a-b \]
\vspace{0.05in}

\noindent
\underline{Remark}: We can easily show that the fixed point problem can be reduced to the CT problem, as seen in Lemma~\ref{FPtoCTLemma} in Appendix A.\\[-5pt]

\noindent
{\large\bf Common Equation Problem (CE)}:

\begin{description}[align=left]
\item[Input] Two substitutions $\theta_1^{}$ and $\theta_2^{}$ with the 
 \emph{same domain}, and an
  equational theory~$E$.
\item[Question] Does there exist a non-ground, non-trivial $(t_1
  \not\approx_E^{} t_2)$ equation~$t_1^{} \approx_{}^? t_2^{}$, where
  $t_1^{}, t_2^{} \, \in \, T(Sig(E), \, \dom(\theta_1^{}))$ such that
  both~$\theta_1^{}$ and~$\theta_2^{}$ are \emph{E}-unifiers
  of~$t_1^{} \approx_{}^? t_2^{}$?\\
\end{description} %$t_1^{}, t_2^{} \, \in \, T(Sig(E), \, \dom(\theta_1^{}))$%
By trivial equations, we mean equations which are identities
in the equational theory~$E$, i.e., an equation $s \approx_{E}^? t$ is
trivial if and only if $s \approx_{E}^{} t$.
We exclude this type of trivial equations in the formulation of
this question.

%We called the first problem in this dualization as the \emph{common
% term (CT) problem}, while the latter is the \emph{common equation (CE)
% problem}.

%We will investigate these two problems for 3~cases, the $\omega\Delta$
%case, the string rewriting case and linear arithmetic.

\vspace{0.05in}
\noindent
\underline{Example 4}: Let $\, E ~ = ~ \{ p(s(x)) \approx x, \; s(p(x)) \approx x \}$.
Given two substitutions \[\theta_{1} = \{ x_1
\mapsto s(s(a)), \, x_2 \mapsto s(a) \}\textrm{ and }\theta_{2} = \{ x_1
\mapsto s(a), \, x_2 \mapsto a \},\] we can see that $\theta_{1}(t_1)
\approx_E^{} \theta_{1}(t_2)$ and $\theta_{2}(t_1) \approx_E^{}
\theta_{2}(t_2)$, with the equation $p(x_1) \approx_E^{} x_2$. However,
there is no term~$t$ on which the substitutions agree, i.e., there
aren't any solutions for the common term problem in this example.
Thus, CT and CE problems are not equivalent as we observe in the example above.

We explore the Fixed Point, Common Term and Common Equation problems in \emph{convergent} string rewriting systems, more specifically in the subclass of \emph{dwindling} string rewriting systems. Length-reducing and monadic string rewriting systems are thoroughly investigated by \cite{Otto1986}, \cite{NOW}, \cite{OND98}, \cite{NarendranOtto97}.

Dwindling convergent systems are especially important because they
are a special case of \emph{subterm-convergent theories} which are
widely used in the field of protocol analysis~\cite{abadi2006deciding,
  Baudet2005, ciocaba2009, cortier2009}. Tools such as TAMARIN
prover~\cite{Meier2013} and YAPA~\cite{Baudet2009} use
subterm-convergent theories since these theories have nice properties
(e.g., finite basis property~\cite{chevalier2010compiling}) and
decidability results~\cite{abadi2006deciding}.

In this document we will discuss (and survey) these three
problems for \emph{convergent} string rewriting systems. The current literature
and our discoveries are summarized in the
Table~\ref{complexityTable}, the \emph{rectangle} boxes indicating \emph{our own
results}:

\begin{table}[h]
\begin{center}
\begin{tabular}{|c|c|c|c|c|} \hline
	& Convergent & Length-reducing & Dwindling & Monadic \\[6pt] \hline
    &            &                 &           &         \\
FP  & \doublebox{\textbf{undecidable}} & NP-complete   & \doublebox{\textbf{P}}       &  P      \\
    &            &                 &           &         \\
CT  & undecidable & undecidable    & \doublebox{\textbf{undecidable}} & P \\
    &            &                 &           &         \\
CE  & undecidable & undecidable    & \doublebox{\textbf{undecidable}} & \doublebox{\textbf{P}} \\
    &            &                 &           &         \\ \hline
\end{tabular}
\end{center}
\caption{Complexity results of the problems in String Rewriting Systems.}
\label{complexityTable}
\end{table}

\section*{Definitions}\label{S:defns}
We start by presenting some notation and definitions on term rewriting
systems and particularly string rewriting systems. Only some
definitions are given in here, but for more details, refer to the
books~\cite{Term} for term rewriting systems and~\cite{Botto} for
string rewriting systems.

A signature $\Sigma$ consists of finitely many ranked function symbols.
Let $X$~be a (possibly infinite) set of variables. The set
of all terms over~$\Sigma$ and~$X$ is denoted~as $T(\Sigma, X)$. 
$\Var(t)$ shows the the set of variables for term $t$ and a term is a ground term iff $\Var(t) = \emptyset$. 
The set of \emph{ground terms}, or terms with no variables
is denoted~$T(\Sigma)$.
A term
rewriting system~(TRS) is a set of rewrite rules that are defined on
the signature~$\Sigma$, in the form of $l \rightarrow r$, 
where $l$ and $r$
are called the left- and right-hand-side (\emph{lhs} and \emph{rhs}) of
the rule, respectively. The rewrite relation induced by
a term rewriting system~$R$ is denoted by~$\rightarrow_R^{}$. 
The reflexive
and transitive closure of~$\rightarrow_R^{}$ is denoted
$\rightarrow_R^{*}$. A TRS~$R$
is called \emph{terminating} iff there is no infinite chain of terms.
A TRS $R$~is \emph{confluent} iff, for all terms $t$,
$s_1$, $s_2$, if $s_1$ and $s_2$ can be derived from~$t$, i.e.,
$s_1 \leftarrow_R^{*} t \rightarrow_R^{*} s_2$, then there exists a 
term~$t'$ such that $s_1 \rightarrow_R^{*} t' \leftarrow_R^{*} s_2$.  A TRS
$R$ is \emph{convergent\/} iff it is both terminating and confluent.

A term is \emph{irreducible} iff no rule of TRS $R$ can be
applied to that term. The set of terms that are irreducible modulo~$R$
is defined by $IRR(R)$ and also called as terms in their \emph{normal
  form}s.  A term~$t'$ is said to be an \emph{R-normal form} of a term
$t$, iff it is irreducible and reachable from~$t$ in a finite number
of steps; this is written as $t \rightarrow_{R}^{!} t'$. Let $t_\downarrow$
denote the normal form of $t$ when $R$ is understood from the context.

String rewriting systems are a restricted class of term rewriting
systems where all functions are unary. These unary operators, that are
defined by the symbols of a string, applied in the order in which
these symbols appear in the string, i.e., if $g, h \in \Sigma$, the
string $gh$ will be seen as the term $h(g(x))$.\footnote{
  It may be more common to view $gh$ as $g(h(x))$ with
  function application done in the reverse order.
  } The set of all strings
over the alphabet $\Sigma$ is denoted by $\Sigma^*$ and the empty
string is denoted by the symbol $\lambda$. Thus the term rewriting system
$\{ p(s(x)) \rightarrow x , \; s(p(x)) \rightarrow x \}$ is equivalent to
the string-rewriting system \[ \{ sp \rightarrow \lambda , \;
ps \rightarrow \lambda \} \]

If $R$ is a string
rewriting system (SRS) over alphabet $\Sigma$, then the single-step
reduction on $\Sigma^*$ can be written as:

For any $u,v \in \Sigma^*$, $u \rightarrow_R v$ iff there exists 
a rule~$l \rightarrow r
\in R$ such that 
$u = xly$ and $v = xry$ for some~$x, y \in \Sigma^*$; 
i.e., \[ {\rightarrow}_R^{} \; = \; \{ (xly, \, xry) \; \mid \; (l \rightarrow r) \in R, 
  \, x,y \in {{\Sigma}^*} \} \] 

For any string rewrite system~$R$ over $\Sigma$, the set of all
irreducible strings, $IRR(R)$, is a regular
language: in fact, $IRR(R) = \Sigma_{}^* \smallsetminus \{\Sigma_{}^*
l_1 \Sigma_{}^* \,\cup ... \cup \, \Sigma_{}^* l_n \Sigma_{}^*\}$,
where $l_1,\dots, l_n$ are the left-hand sides of the rules in~$T$.

Throughout the rest of the paper, 
$a, b, c, \dots, h$  will denote elements of the alphabet~$\Sigma$, 
and $l, r, u, v, w, x, y,z$ will denote strings over~$\Sigma$. Concepts such as {\em normal form,\/}  {\em terminating,\/} 
{\em confluent,\/} and {\em convergent\/} have the same definitions in the string rewriting
systems as they have for the term rewriting systems.
An SRS~$T$ is called {\em canonical\/} if and only if it is convergent and
{\em inter-reduced,\/} i.e., no lhs is a substring of another lhs.


For a string $x\in \Sigma^*$, the element at position $i$ is denoted
$x_{[i]}$, and the substring from position $i$ to position $j$
(inclusive) is denoted as $x_{[i:j]}$ where $i\leq j$ and this shall
denote the empty string when $i>j$. We will write $x_{[i:]}$ to denote
$x_{[i:|x|]}$ when it cumbersome to use $|x|$. Additionally, for an index sequence
$\beta = (\beta_1,\beta_2,\dots , \beta_c)$, we use $x_{[\beta]} := x_{[\beta_1]}x_{[\beta_2]}\dots x_{[\beta_c]}$.
Parenthesized superscripts shall be a general way to index elements in a sequence of
strings e.g.  $(x^{(1)},x^{(2)},x^{(3)},\dots)$.


A  string rewrite system $T$ is said to be:
\begin{itemize}
\item[-] {\em monadic\/} iff the rhs of each rule in $T$ is either a single
 symbol or the empty string, e.g., $abc \rightarrow b$. \par 
\item[-] {\em dwindling\/} iff, for every rule $l \flr r$ in $T$, the rhs $r$ 
  is a {\em proper prefix\/} of its lhs $l$, e.g., $abc \rightarrow ab$. \par
\item[-] \emph{length-reducing} iff $| l | > | r |$ for all rules~$l \flr r$ in~$T$, e.g., $abc \rightarrow ba$.
\end{itemize}

\begin{figure}[h]
\centering
\pagestyle{empty}
\def\firstcircle{(-1,0) ellipse (1.5cm and 1cm)}
\def\secondcircle{(60:0cm) ellipse (4cm and 2cm)}
\def\thirdcircle{(0:1cm) ellipse (1.5cm and 1cm)}
\begin{tikzpicture}
    \begin{scope}[shift={(3cm,-5cm)}]%, fill opacity=0.5]
        %\fill[green] \secondcircle;
        %\fill[red] \firstcircle;
        %\fill[blue] \thirdcircle;
        \node (M) at (0,0){}; 
        \draw \firstcircle node[left=0.5cm of M] {$Monadic$};
        \draw \secondcircle node [above=1.0cm of M] {$\text{\emph{Length-Reducing}}$};
        \draw \thirdcircle node [right=0.3cm of M] {$Dwindling$};
    \end{scope}
\end{tikzpicture}
\caption{Some of the classes of String Rewriting Systems \protect\footnotemark}
\label{SRSets}
\end{figure}

\footnotetext{The trivial forms of monadic rules such as $a \rightarrow b$ can be ignored. We can get rid of such rules by changing every occurrence of $a$ to $b$.}




%%%%%%%%%%%%%%%%%%%%%%%%%%%%%%%%%%%%%%%%%%%%%%%%%%%%%%%%%%%%%%%%%%%%%%%%%%%%%%%%%%%%%%%%%%%%%%%%%%%%%%%%%
\section{Fixed Point Problem}\label{S:three}
Note that for string rewriting systems the fixed point problem is equivalent to the following problem:

\begin{description}[align=left]
\item[Input] A string-rewriting system~$R$ on an alphabet~$\Sigma$, and 
a string~$\alpha \in \Sigma_{}^+$.
\item[Question] Does there exist a string~$W$ such that 
$\alpha W \; \stackrel{*}{{\longleftrightarrow}_R} \; W$?
\end{description}

Fixed Point Problem is a particular case of the Common Term Problem discussed in the next section and is thus decidable in polynomial time for finite, monadic and convergent string rewriting systems.

It is also a particular case of the \emph{conjugacy problem}.
Thus for finite, length-reducing and convergent systems it is
decidable in~\textbf{NP}~\cite{NOW}. The \textbf{NP}-hardness proof
in~\cite{NOW} also applies in our particular case: thus the problem
is~\textbf{NP}-complete for finite, length-reducing and convergent
systems.

\subsection{Fixed Point Problem for Finite and Convergent Systems:}

Theorem~\ref{FPConvergent} shows that fixed point problem is undecidable for finite and convergent string rewriting systems.

\begin{thm}
The following problem is undecidable: \begin{quote}
\begin{itemize}
\item[Input:] A non-looping, deterministic linear bounded automaton~$\mathcal{M}$ that
restores its input for \emph{accepted} strings.
\item[Question:] Is $\mathcal{L(M)}$ empty?
\end{itemize}
\end{quote}
\end{thm}

\begin{proof}

Suitably modifying the construction given in~\cite{caron1991linear} in a straightforward way, we can prove the undecidability of this problem with a reduction from the well-known
Post Correspondence Problem~(PCP). Recall that an instance of PCP is a
collection of pairs of strings over an alphabet~$\Sigma$ (``dominos'' in~\cite{Sipser}) and
the question is if there exists a sequence of
indices~$i_1, i_2, \ldots i_n$ such that
$x_{i_1}  x_{i_2} \ldots x_{i_n} = y_{i_1} y_{i_2} \ldots y_{i_n}$.
Alternatively, we can define the PCP in terms
of two homomorphisms
$\psi$ and $\phi$ from $C^*=\{c_1,\;\ldots\;c_n\}^*$ to $\Sigma^* =
\{a,b\}^*$. The question now is whether there exists a non-empty
string~$m_C^{}$ such that
$\psi(m_C^{}) = \phi(m_C^{})$.

The reduction proof is done by validating the ``solution string''~$w
\in \Sigma^*$, for the two homomorphisms using the DLBA $\mathcal{M}$
with the following configuration, $\mathcal{M} = (Q, \Sigma, \Gamma,
\delta, q_0, q_a, q_r)$. Assume $w=w_1 \; w_2$ where $w_1 \in C^*$ and
$w_2 \in \Sigma^*$. $\mathcal{M}$ will check the correctness of $w$ by
going over one symbol at a time in the string $w_1$ and its
corresponding mapping in~$w_2$. $\mathcal{M}$ uses a marking
technique, marking the symbols~$c_i$ and their correct mappings with
overlined symbols. If every letter in the string is overlined, then
clearly the string is ``okay'' according to the first homomorphism;
$\mathcal{M}$ will then replace the overlined letters with the same
non-overlined letters as before and repeat the same steps for the
second homomorphism. Thus the DLBA will accept and re-create~$w_1 \;
w_2$ if and only if it is a solution string. $\mathcal{L(M)}$ is empty if and only if the instance of PCP has no solution.
\end{proof}

We construct a string
rewriting system~$R$ from the above-mentioned DLBA~$\mathcal{M}$.  The construction of
$R$ is similar to the one in~\cite{bauer1984finite}. Let 
$\mathcal{M} = (Q, \Sigma, \Gamma, \delta, q_0, q_a, q_r)$. Here
$\Sigma$ denotes the input alphabet, $\Gamma$ is the tape alphabet, $Q$ is the set of states, $q_0 \in
Q$ is the initial state, $q_a \in Q$ the accepting state and
$q_r \in Q$ the rejecting state. We assume
that the tape has two end-markers: $\cent \in \Gamma$ denotes
the left end-marker and $\$ \in \Gamma$ is the right 
end-marker. We also assume that on acceptance
the DLBA comes to a halt at the left-end of the tape. Finally, 
$\delta:  Q \times \Gamma \rightarrow Q \times \Gamma \times \{L, R\}$
is the transition function of~$\mathcal{M}$. 

The alphabet of $R$ is $\Gamma = \Sigma \cup \Sigma^{\prime} \cup Q$
where $\Sigma^{\prime}$ is a replica of the alphabet
$\Sigma$ such that $\Sigma^{\prime} \cap \Sigma = \emptyset$.
$R$ has the rules
\begin{equation*}
\begin{aligned}
    q_i \, a_k \, &\rightarrow  a_k^{\prime}\, q_j 			&& \text{if }(q_i, a_k, q_j, R) \in \delta\\
    a_l^{\prime} \, q_i \, a_k \, &\rightarrow q_j \, a_l \, a_k	&& \text{for all } a_l^{\prime} \in \Sigma^{\prime} \text{, if }(q_i, a_k, q_j, L) \in \delta\\
    q_a \, \cent \, &\rightarrow \cent 					&& \text{ }
\end{aligned}
\end{equation*}

Since the linear bounded automaton is deterministic, $R$ is locally
confluent. Besides, $\mathcal{M}$ ultimately always halts, and that means
there will be no infinite chain of rewrites for~$R$, and thus $R$ is
terminating.

\begin{lem}\label{initialAccepting}
$\mathcal{M}$ accepts $w$ iff
$q_0\, \cent \, w \, \$ \rightarrow_R^{+} q_a\, \cent \, w\, \$ $.
\end{lem}
\begin{proof}
By inspection of the rules we can see that $\mathcal{M}$ makes the transition
$u_1\,q_1\,v_1 ~ \vdash_M^{} ~ {u_2}\, q_2\, v_2$ if and only if
$u_1^{\prime}\,q_1\,v_1 \rightarrow_R^{} u_2^{\prime}\, q_2\, v_2$.
\end{proof}

\begin{lem}
$\cent \,w\, \$$ is a fixed point for $q_0$ in $R$ iff
$\mathcal{M}$ accepts~$w$.
\end{lem}
\begin{proof}
Suppose $\mathcal{M}$ accepts~$w$.
Observe that $q_0\, \cent \, w \, \$ \rightarrow_R^{+} q_a\, \cent \,
w\, \$ $ by Lemma~\ref{initialAccepting}, and $R$ has the rule
$q_a \, \cent \, \rightarrow \cent$. Thus we get
$q_0\, \cent \, w \, \$ \rightarrow_R^{+} q_a\, \cent \, w\, \$
\rightarrow \cent \, w \,\$ $.

For the ``only if'' part, suppose $\cent \,w\, \$$ is a fixed point
for $q_0$, i.e., $q_0\, \cent \, w \, \$ \rightarrow_R^{+} \cent \, w
\,\$ $. Now note that the \emph{only} rule that can remove a
state-symbol from a string is the rule $q_a \, \cent \, \rightarrow \,
\cent$. But once that rule is applied, no other rules are
applicable. Therefore, there must be a reduction sequence such that $q_0\,
\cent \, w \, \$ \rightarrow_R^{+} q_a\, \cent \, w\, \$$. This
proves that $w$~is accepted by the~DLBA $\mathcal{M}$. 
\end{proof}

\begin{thm}\label{FPConvergent}
The fixed point problem is undecidable for finite and convergent
string rewriting systems.
\end{thm}

%%%%%%%%%%%%%%%%%%%%%%%%%%%%%%%%%%%%%%%%%%%%%%%%%%%%%%%%%%%%%%%%%%%%%%%%%%%%%%%%%%%%%%%%%%%%%%%%%%%%%%%%%%%%%%%%%%
\subsection{Fixed Point Problem for Dwindling Convergent Systems:}

Given a string $\alpha \in \Sigma^*$ and a dwindling, convergent, and
finite string rewriting system~$R$, does there exist a string $W\in
\Sigma^*$ such that $\alpha W \rightarrow^!_R W$?



%changes for formula breaking up to two lines \\
It will be notationally convenient to characterize dwindling rewrite
rules by a 3-tuple containing the LHS, the length of the LHS, and the length of the RHS.
In other words, for a dwindling rule $\ell \rightarrow \ell_{1:e} \in R$
where $|\ell| = d$ and $1 \leq e < d$, we have the tuple $(\ell,d,e)$.

\begin{lem}\label{dwindling_fp_l1} % \textbf{Lemma 1:} 
Let $A\in\Sigma^+$ irreducible, $X\in\Sigma^+$, and $R$ dwindling,
convergent. \\If $AX\rightarrow_R^* S$, then 
 $S=A_{[1:b]}X_{[\beta]}$ for some $b\leq|A|$ and index sequence \\$\beta
 = (\beta_1,\beta_2,\dots , \beta_c)$ where $1\leq \beta_1 < \beta_2 < \dots < \beta_c \leq |X|$. Additionally, $c < |X|$ if $S \neq AX$.
\end{lem}

%changes for formula breaking up to two lines $$
\begin{proof}
  This result follows closely from the irreducibility of the
individual strings $A$ and $X$, and the nature of dwindling
rules. Applying a dwindling rule to $AX$ either removes a substring
spanning the end of $A$ and the beginning of $X$ or a substring of
only $X$. The details are as follows:

We will prove Lemma \ref{dwindling_fp_l1} by showing for an
arbitrary sequence of terms $$(S\ssp{0},S\ssp{1},\dots,S\ssp{n})$$
where $S\ssp{0} = AX$ and $S\ssp{i-1} \rightarrow_R S\ssp{i}$, that for
each string $S\ssp{i}$ there exists integer $b$ and index sequence
$\beta$ such that $S^{(i)}=A_{[1:b]}X_{[\beta]}$. Clearly, the result
holds for $i=0$. Assume the result holds for some $i=k$. Since for any
$S^{(k)} := A_{[1:b]}X_{[\beta]}$ where $A_{[1:b]}$ irreducible, if
the LHS $\ell$ of any rewrite rule $(\ell,d,e)$ matches $S^{(k)}_{[p:p+d]}$,
we must have $b < p+d$. Therefore, applying $(\ell,d,e)$ and obtaining
$S^{(k+1)}$ we have


\begin{align} S^{(k+1)} &=
(A_{[1:b]}X_{[\beta]})_{[1:p+e]}(A_{[1:b]}X_{[\beta]})_{[p+d+1:]} \\ &=
(A_{[1:b]}X_{[\beta]})_{[1:\eta]}(A_{[1:b]}X_{[\beta]})_{[\eta+1:p+e]}(A_{[1:b]}X_{[\beta]})_{[p+d+1:]}
\end{align}

\noindent where $\eta=min(p+e,b)$. \\

%changes for formula breaking up to two lines \\
The first substring $(A_{[1:b]}X_{[\beta]})_{[1:\eta]}$ is a prefix of
$A_{[1:b]}$. The second substring \\$(A_{[1:b]}X_{[\beta]})_{[\eta+1:p+e]}$ is
the empty string if $p+e \leq b$ and a prefix of $X_{[\beta]}$
otherwise. Finally, $(A_{[1:b]}X_{[\beta]})_{[p+d+1:]}$ is a (strict)
right-substring of $X_{[\beta]}$. It is also clearly seen that
$|(A_{[1:b]}X_{[\beta]})_{[\eta+1:p+e]}(A_{[1:b]}X_{[\beta]})_{[p+d+1:]}| <
|X_{[\beta]}|$ 
\end{proof}

Note that there may be more than one decomposition $(b,\beta)$ for
Lemma (\ref{dwindling_fp_l1}). However, if one assigns unique labels
to each character instance and considers a specific sequence of
rewrites where each application of a (dwindling) rule ``deletes'' a
substring, then there is a specific coresponding tuple
$(b,\beta)$. Unless otherwise stated we shall assume maximum $b$ in
our application of (\ref{dwindling_fp_l1}).


\begin{lem} \label{dwindling_fp_l2}%\textbf{Lemma 2:} Assume $W$ is the smallest solution to
Assume $W$ is a minimal-length solution to an instance of the fixed point problem $(\alpha,R)$ for irreducible
$\alpha$ and dwindling, convergent string rewriting system $R$. Then
$W_{[1]} = \alpha_{[1]}$.
\end{lem}

\begin{proof}

Let $n=|W|$. We must have that $|(\alpha W)_\downarrow|=n$. By Lemma
\ref{dwindling_fp_l1}, we have that $(\alpha W)_\downarrow = \alpha_{[1:b]}W_{[\beta]}$ for
some $b\leq|\alpha|$ and $1\leq \beta_1 < \beta_2 < \dots < \beta_c
\leq |W|$ where $c < |W|$. If $\alpha_{[1:b]}$ not empty, then
$\alpha_{[1:b]}$ is a prefix of $(\alpha W)_\downarrow$. Otherwise
$|(\alpha W)_\downarrow| < n$ because $|W_{[\beta]}|<n$, contradicting
$n=|W|$.
\end{proof}



\begin{thm} \label{dwindling_fp_theorem1} Given an instance of the
fixed point problem $(\alpha,R)$ where $R$ is a dwindling, convergent
rewrite system and $\alpha$ is in normal form we can determine if
solutions exist in polynomial time.
\end{thm}

\begin{proof}

Since $R$ is length-reducing and convergent, the length of a smallest
solution (if solutions exist) is bounded above by $2\cdot max(\{|l| :
(l\rightarrow r) \in R \}) \cdot |\alpha|$ which follows from the
results in \cite{narendran1985complexity} and that the fixed point
problem is a special case of the conjugacy problem.  Let $n$ be an
upper bound on the length of minimal-length solutions.

From Lemma \ref{dwindling_fp_l2}, $W_{[1]}=\alpha_{[1]}$. Suppose the first $j$ symbols of a
minimal-length solution $W$ are known. Define $\alpha^{(j)} := (\alpha
W_{[1]} \dots W_{[j]})_\downarrow$. Then $\alpha^{(j)}W_{[j+1:|W|]}
\rightarrow_R^! W$, and so either $|W|=j$ and we know all the symbols
of $W$, or $|\alpha^{(j)}| > j$. Applying Lemma \ref{dwindling_fp_l1} to
$\alpha^{(j)}W_{[j+1:|W|]}$ by taking $A = \alpha^{(j)}$, $X =
 W_{[j+1:|W|]}$, we have that its normal form
$\alpha^{(j)}_{[1:b]}W_{[\beta]}$ must be such that $|W_{[\beta]}| <
|W_{[j+1:|W|]}|$ and so $\alpha^{(j)}_{[1:b]}$ is a prefix of
$\alpha^{(j)}$ with $b > j$. Thus, $W_{[j+1]} = \alpha^{(j)}_{[j+1]}$. If
there is a minimum-length solution, it will be found within $n$
iterations.
\end{proof}



\begin{thm} \textbf{Complexity:} If solutions exist, a minimal-length
solution $W$ can be found in $O(n \cdot |R|)$ time, where $n$ is the
upper bound on the length of minimal-length solutions for a given
instance of the fixed point problem $(\alpha,R)$ and $R$ is a
dwindling, convergent rewrite system. Otherwise, the algorithm will
terminiate in $O(n\cdot |R|)$ time.
\end{thm}

\begin{proof}
The procedure in Theorem 1 determines a minimal-length solution
if it exists. Since $\alpha^{(j)}$ can be defined recursively as:
$\alpha^{(0)}=\alpha$, $\alpha^{(j)} =
(\alpha^{(j-1)}\alpha_{[j]}^{(j-1)})_\downarrow$, and $\alpha^{(j-1)}$
is already in normal form, only one position needs to be matched to
check whether a rule in $R$ can be used to reduce
$\alpha^{(j-1)}\alpha_{[j]}^{(j-1)}$. Additionally, no more than one
reduction must be applied to fully reduce
$\alpha^{(j-1)}\alpha_{[j]}^{(j-1)}$.  At most $|\alpha|<n$ rewrites
need to be applied in total before $\alpha W$ is reduced to $W$ or
before the existence of a solution is ruled out by $|\alpha^{(j)}|\leq
j$. The overall time complexity is thus $O(n \cdot |R|)$. 
\end{proof}


\begin{cor} % \textbf{Corollary 2:} 
If the RHS of each rule in $R$ is also of length $\leq 1$ (i.e. the cases
where $R$ is a dwindling convergent system that is also special or
 monadic), then a minimal-length solution $W$ must be a prefix of
 $\alpha$.
 \end{cor} %changes for formula breaking up to two lines $$
 \begin{proof} In this case, we have $e \in \{0,1\}$ for each
application of a rule $(\ell,d,e)$.  Recall that in Lemma
\ref{dwindling_fp_l1}, if the LHS $\ell$ of any rewrite rule
 $(\ell,d,e)$ matches $S^{(k)}_{[p:p+d]}$, then $$S^{(k+1)}=
(A_{[1:b]}X_{[\beta]})_{[1:\eta]}(A_{[1:b]}X_{[\beta]})_{[\eta+1:p+e]}(A_{[1:b]}X_{[\beta]})_{[p+d+1:]}$$
for $\eta := min(p+e,b)$. In these cases, $p+e < \eta + 1$ and so the
middle substring $$(\alpha_{[1:b]}W_{[\beta_1,\beta_2,\dots,\beta_c]})_{[\eta+1:p+e]}$$ is empty.
It follows that every string reachable from $\alpha W$ consists of a prefix of $\alpha$
followed by a postfix of $W$, and so minimal-length solutions must be a prefix of $\alpha$. 
\end{proof}


%%%%%%%%%%%%%%%%%%%%%%%%%%%%%%%%%%%%%%%%%%%%%%%%%%%%%%%%%%%%%%%%%%%%%%%%%%%%%%%%%%%%%%%%%%%%%%%%%%%%%%%%%

% DH -- Common Term Problem
\section{Common Term Problem}

Note that for string rewriting systems the common term problem is
equivalent to the following problem: \medskip

\begin{description}[align=left]
  \item[Input] A string-rewriting system~$R$ on an alphabet~$\Sigma$,
and two strings~$\alpha, \beta \in \Sigma_{}^*$.
  \item[Question] Does there exist a string~$W$ such that $\alpha W \;
\stackrel{*}{{\longleftrightarrow}_R} \; \beta W$?
\end{description} \medskip

This is also known as \textit{Common Multiplier Problem} which has
been shown to be decidable in polynomial time for monadic and
convergent string-rewriting systems (see, e.g.,~\cite{OND98},
Lemma~3.7).  However, the CT problem is undecidable for convergent and
length-reducing string rewriting systems in
general~\cite{Otto1986}.\footnote{ In fact, Otto et al.~\cite{OND98}
showed that there is a \emph{fixed} convergent length-reducing string
rewriting system for which the CT problem is undecidable.}

In this section, we focus on the decidability of the CT problem for
\emph{convergent and dwindling} string rewriting systems. 
%The CT problem for Convergent and Dwindling systems is also %investigated in the work \cite{zumrutDissertation, 2017arXiv, UNIF2017Akcam}

\subsection{Common Term Problem for Convergent and Dwindling Systems}\label{ctdwindling} We show that
the CT (Common Term) problem is undecidable for string rewriting
systems that are dwindling and convergent. We define CT as the
following decision problem: \medskip

\begin{description}[align=left]
  \item[Given] A finite, non-empty alphabet $\Sigma$, strings $\alpha,
\beta \in \Sigma^*$ and a dwindling, convergent string rewriting
system $S$.
  \item[Question] Does there exist a string $W \in \Sigma^*$ such that
$\alpha W \approx_S^{} \beta W$?
\end{description} \medskip

Note that interpreting concatenation the other way, i.e., $ab$ as
$a(b(x))$, will make this a \emph{unification} problem.

We show that Generalized Post Correspondence Problem ($GPCP$) reduces
to the CT problem, where $GPCP$ stands for a variant of the modified
post correspondence problem such that we will provide the start and
finish dominoes in the problem instance. This slight change does not
effect the decidability of the problem in any way, i.e., $GPCP$ is
also undecidable~\cite{EKR82,GPCP}. \medskip

\begin{description}[align=left]
  \item[Given]~A finite set of tuples $\left\{(x_i,\; y_i)\right
\}^{n+1}_{i=0}$ such that each $x_i, y_i \in \Sigma^+$, i.e., for all
$i$, $|x_i|>0$, $|y_i|>0$, and $(x_0, y_0), (x_{n+1}, y_{n+1})$ are
the \emph{start} and \emph{end} dominoes, respectively.
  \item[Question]~Does there exist a sequence of indices $i_1, \ldots
,i_k$ such that \[ x_{0}\;x_{i_1}\; \ldots \;x_{i_k}\;x_{n+1} =
y_{0}\;y_{i_1}\; \ldots \; y_{i_k}\;y_{n+1} ? \]
\end{description} \medskip

We work towards showing that the CT problem defined above is
undecidable by a many-one reduction from $GPCP$. First, we show how to
construct a string-rewriting system that is dwindling and convergent
from a given instance of $GPCP$.

Let $\left\{(x_i,\; y_i)\right \}^{n}_{i=1}$ be the set of
``intermediate" dominoes and $(x_0, y_0), (x_{n+1}, y_{n+1})$, the
start and end dominoes respectively, be given. Suppose $\Sigma$ is the
alphabet given in the instance of $GPCP$. Without loss of generality,
we may assume $\Sigma = \{a,\, b\}$. Then set $\hat{\Sigma} := \{a,b\}
\cup \{c_0,\;\ldots\;c_{n+1}\} \cup \{\cent_1, \cent_2, B, a_1, a_2,
a_3, b_1, b_2, b_3\} $ which will be our alphabet for the instance of
CT.

Next we define a set of string homomorphisms used to simplify the
discussion of the reduction. Namely, we have the following:
\begin{equation*}
\begin{aligned}[c] h_1(a) & = & a_1 \, a_2 \, a_3,\\ h_1(b) & = &
b_1\, b_2 \, b_3,
\end{aligned} \qquad
\begin{aligned}[c] h_2(a) & = & a_1 \, a_2, \\ h_2(b) & = & b_1 \,
b_2,
\end{aligned} \qquad
\begin{aligned}[c] h_3(a) & = & a_1 \\ h_3(b) & = & b_1
\end{aligned}
\end{equation*} such that each $h_i : \Sigma \rightarrow
\hat{\Sigma}^+$ is a homomorphism.



\textbf{Reduction and form of solution:} For the convenience of the
reader, we state the form of the $CT$ problem instance and its
solutions before explaining the details. In particular, $\alpha =
\cent_1$, $\beta = \cent_2$. For instances such that a solution $Z$
exists, it shall take the form $h_1(Z_1)Z_2$, where $Z_1 \in \{a,b\}^+$ is
the string of the GPCP solution and $Z_2 = c_{n+1}Bc_{i_k}\dots Bc_{i_1}Bc_{0}$.
The sequence $c_{n+1},c_{i_k}, \dots, c_{i_1},c_{0}$ is the
(reversed) sequence of tuple indices, that is, the GPCP solution itself,
and they are separated by the symbol $B$. Informally, the purpose of the
homormorphisms and $\cent_1$, $\cent_2$ is to ensure that $\alpha Z$ and
$\beta Z$ are processed with two separate sets of rules, corresponding to
the sets of first and second tuple (tile, if the reader prefers) elements,
respectively. 


We are now in a position to construct the string rewriting system~$S$,
with the following collections of rules, named as the Class~D rules:

\begin{equation*}
\begin{aligned}[c] \cent_1 h_1(a) & \rightarrow & \cent_1 h_3(a),\\
\cent_1 h_1(b) & \rightarrow & \cent_1 h_3(b),
\end{aligned} \qquad
\begin{aligned}[c] \cent_2 h_1(a) & \rightarrow & \cent_2 h_2(a)\\
\cent_2 h_1(b) & \rightarrow & \cent_2 h_2(b)
\end{aligned}
\end{equation*}

\noindent and,
\begin{equation*}
\begin{aligned}[c] h_i(a)\,h_1(a) & \rightarrow & h_i(a)\,h_i(a),\\
h_i(b)\,h_1(a) & \rightarrow & h_i(b)\,h_i(a),
\end{aligned} \qquad
\begin{aligned}[c] h_i(a)\,h_1(b) & \rightarrow & h_i(a)\,h_i(b)\\
h_i(b)\,h_1(b) & \rightarrow & h_i(b)\,h_i(b)
\end{aligned}
\end{equation*}
\noindent for $i \in \{2, 3\}$.

The erasing rules of our system consists of three classes. Class
\rom{1} rules are defined as:
\begin{eqnarray*} \cent_1\,h_3(x_0)\,B\,c_0 & \rightarrow & \lambda \\
\cent_2\,h_2(y_0)\,c_0 & \rightarrow & \lambda
\end{eqnarray*}

and Class \rom{2} rules (for each $i = 1, 2, \ldots, n$),
\begin{eqnarray*} h_3(x_i)\,B\,c_i & \rightarrow & \lambda \\
h_2(y_i)\,c_i\,B & \rightarrow & \lambda
\end{eqnarray*}

and finally Class \rom{3} rules,
\begin{eqnarray*} h_3(x_{n+1})\,c_{n+1} & \rightarrow & \lambda \\
h_2(y_{n+1})\,c_{n+1}\,B & \rightarrow & \lambda
\end{eqnarray*}

Clearly given an instance of $GPCP$ the above set of rules can
effectively be constructed from the instance data. Also, by
inspection, we have that our system is confluent (there are no
overlaps among the left-hand sides of rules), terminating, and
dwindling.

We then set $\alpha = \cent_1$ and $\beta = \cent_2$ to complete the
constructed instance of $CT$ from~$GPCP$.

It remains to show that this instance of $CT$ is a ``yes" instance if
and only if the given instance of $GPCP$ is a ``yes" instance, i.e.,
the $CT$ has a solution if and only if the $GPCP$ does. In that
direction, we prove some results relating to~$S$.

\begin{lem}\label{FirstStep} Suppose $\cent_1 h_3(w_1) B \gamma
\rightarrow^{!} \lambda$ and $\cent_2 h_2(w_2) \gamma \rightarrow^{!}
\lambda$ for some $w_1, w_2 \, \in \, \{a,b\}_{}^*$, then $\gamma \in
\{c_1B,\;c_2B,\;...\;,c_nB\}_{}^{*} c_0$.
  \end{lem}

  \begin{proof} Suppose $\gamma$ is a minimal counter example with
respect to length and $\gamma \in IRR(R)$. In order for the terms to
be reducible, $\gamma = c_iB\; \gamma^\prime$ (this follows by
inspection of $S$).  After we replace the $\gamma$ at the equation in
the lemma, we get:
    \begin{eqnarray*} \cent_1 \;h_3(w_1)\; B\; c_i\; B\; \gamma^\prime
& \rightarrow & \cent_1 {h_3(w_1)}^\prime B\; \gamma^\prime
\rightarrow^{!} \lambda \\ \cent_2 \;h_2(w_2)\; c_i\; B\;
\gamma^\prime & \rightarrow & \cent_2 {h_2(w_2)}^\prime \;
\gamma^\prime \rightarrow^{!} \lambda
    \end{eqnarray*}
    \noindent by applying the Class \rom{2} rules and finally Class
\rom{1} rule to erase the $\cent$ signs.  Then, however, $\gamma'$ is
also a counterexample, and $| \gamma' | < | \gamma |$, which is a
contradiction.
  \end{proof}

    We are now in a position to state and prove the main result of
this section.

  \begin{thm} The CT problem is undecidable for dwindling convergent
string-rewriting systems.
  \end{thm}

  \begin{proof} We first complete the ``only if" direction. Suppose CT
has a solution such that $\cent_1 Z \downarrow \cent_2 Z$ where $Z$ is
a minimal solution. We show that $Z$ corresponds to a solution for
$GPCP$. Let $Z_1$ be the longest string such that $h_1(Z_1)$ is a prefix
of $Z$, and denote the rest of $Z$ by $Z_2$, so that $Z$ can be written $Z = h_1(Z_1)Z_2$.
 
%Let $Z = h_1(Z_1)Z_2$ such that $h_1(Z_1)$ is the longest
%prefix of $Z$ such that the following relationship holds: $Z =
%Z'\;Z_2$ and $Z' = h_1(Z_1)$ for some string $Z_1$.

    $h_1(Z_1)$ can be rewritten to $h_3(Z_1)$ and $h_2(Z_1)$ by
applying the Class D rules. Thus, we will get
\begin{eqnarray*} \cent_1 \;h_1(Z_1)\; Z_2 & \rightarrow^{*} & \cent_1
h_3(Z_1)\; Z_2 \\ \cent_2 \;h_1(Z_1)\; Z_2 & \rightarrow^{*} & \cent_2
h_2(Z_1)\; Z_2
\end{eqnarray*} In order for terms to be reducible simultaneously,
$Z_2$ must be of the form $Z_2 = c_{n+1}\;B\;Z_2^{\prime}$. Thus
\begin{eqnarray*} \cent_1 \;h_3(Z_1)\; Z_2 & = & \cent_1 h_3(Z_1)\;
c_{n+1}\;B\;Z_2^{\prime} \\ \cent_2 \;h_2(Z_1)\; Z_2 & = & \cent_2
h_2(Z_1)\; c_{n+1}\;B\;Z_2^{\prime}
\end{eqnarray*} i.e., $Z_1 = Z_1^\prime \: x_{n+1}$ and $Z_1 =
Z_1^{\prime\prime} \: y_{n+1}$.  By applying the Class \rom{3} rules,
these equations will reduce to:
\begin{eqnarray*} \cent_1 h_3(Z_1)\; c_{n+1}\;B\;Z_2^{\prime} &
\rightarrow & \cent_1 h_3(Z_1^\prime) \; B\;Z_2^{\prime}\\ \cent_2
h_2(Z_1)\; c_{n+1}\;B\;Z_2^{\prime} & \rightarrow & \cent_2
h_2(Z_1^{\prime\prime}) \; Z_2^{\prime}
\end{eqnarray*} We now apply Lemma~\ref{FirstStep} to conclude that
$Z_2^{\prime} \in \{c_1B,\;c_2B,\; \ldots,c_nB\}_{}^{*}c_0$.\\[-5pt]

At this point we have that:
\[ Z_2 = c_{n+1}Bc_{i_k}Bc_{i_{k-1}}{\cdots}Bc_{i_2}Bc_{i_1}Bc_0 \text{\:\:\: for
some } i_1, \ldots, i_k\] Then the sequence of dominoes
\[(x_0, y_0), (x_{i_1}, y_{i_1}), {\ldots} , (x_{i_k}, y_{i_k}),
(x_{n+1}, y_{n+1}) \] will be a solution to the given instance of
$GPCP$ with solution string~$Z_1$ since the left-hand sides of the
Class~\rom{1}, \rom{2}, \rom{3} rules consist of the images of domino
strings under $h_2$ and $h_3$. More specifically, there is a finite
number of $B$'s and $c_i$'s in~$Z_2$, so there must be a decomposition
of $h_1(Z_1)$:
\[ h_1(Z_1) = h_1(x_0)h_1(x_{i_1}) \cdots h_1(x_{i_k})h_1(x_{n+1}) \]
and
\[ h_1(Z_1) = h_1(y_0)h_1(y_{i_1}) \cdots h_1(y_{i_k})h_1(y_{n+1}) \]
Thus, we have the following reductions with Class~D rules:
\[ \cent_1\;h_1(Z_1)\;Z_2 \rightarrow^{*} \cent_1\;h_3(Z_1)\;Z_2 \]
\[ \cent_2\;h_1(Z_1)\;Z_2 \rightarrow^{*} \cent_2\;h_2(Z_1)\;Z_2 \]
Finally, by Class \rom{1}, \rom{2}, \rom{3} rules:
\[\cent_1\;h_3(Z_1)\;Z_2 \rightarrow^{*} \cent_1\; h_3(x_0)\;B\;c_0
\rightarrow \lambda \]
\[\cent_2\;h_2(Z_1)\;Z_2 \rightarrow^{*} \cent_2\; h_2(y_0)\;c_0
\rightarrow \lambda \] and $Z_1$ is a solution to the instance of
the~$GPCP$.

We next prove the ``if" direction. Assume that the given instance of
$GPCP$ has a solution. Let $Z_1$ be the string corresponding to the
matching dominoes, and let
\[(x_0, y_0), (x_{i_1}, y_{i_1}), {\ldots} , (x_{i_k}, y_{i_k}),
(x_{n+1}, y_{n+1}) \] be the sequence of tiles that induces the match.
Let $Z_2 = c_{n+1}Bc_{i_k}Bc_{i_{k-1}}{\cdots}Bc_{i_2}Bc_{i_1}Bc_0$. We show that
$\cent_1 h_1(Z_1)Z_2\; \downarrow\; \cent_2 h_1(Z_1)Z_2$.  First apply the
Class $D$ rules to get:
\[ {\cent_1}h_1(Z_1)Z_2 \rightarrow^{*} {\cent_1}h_3(Z_1)Z_2 \]
\[ {\cent_2}h_1(Z_1)Z_2 \rightarrow^{*} {\cent_2}h_2(Z_1)Z_2 \] but then we
can apply Class~\rom{1}, \rom{2}, \rom{3} rules to reduce both of the
above terms to $\lambda$.
\end{proof}

This result strengthens the earlier undecidability result of
Otto~\cite{Otto1986} for string-rewriting systems that are
\emph{length-reducing} and convergent.


%%%%%%%%%%%%%%%%%%%%%%%%%%%%%%%%%%%%%%%%%%%%%%%%%%%%%%%%%%%%%%%%%%%%%%%%%%%%%%%%%%%%%%%%%%%%%%%%%%%%%%%%%

\section{Common Equation Problem}\label{S:CE}
To clarify the CE problem for string rewriting systems, let us consider two substitutions 
$\theta^{}_1$ and $\theta^{}_2$ such that
\begin{eqnarray*}
\theta^{}_1 = \{ x_1 \mapsto \alpha^{}_1, \; x_2 \mapsto \alpha^{}_2\} \\
\theta^{}_2 = \{ x_1 \mapsto \beta^{}_1,  \; x_2 \mapsto \beta^{}_2\}
\end{eqnarray*}
Think of the letters of the alphabet as monadic function symbols as
mentioned in the section of Definitions.%~\ref{defns}. 
We have two cases for the equation
$e_1 = e_2$: (i)~both $e_1$ and $e_2$ have the same variable in them, or
(ii)~they have different variables, i.e., one has~$x_1$ and the other~$x_2$.
Thus in the former, which we call the ``one-mapping'' case, we are looking for
\emph{different} irreducible strings~$W_1^{}$ and~$W_2^{}$ such that either
\begin{enumerate}
\item $\alpha_1 W_1 \stackrel{*}{{\longleftrightarrow}_R} \alpha_1 W_2 \; \text{ and } \;
  \beta_1 W_1 \stackrel{*}{{\longleftrightarrow}_R} \beta_1 W_2, \; \; \; \text{or}$

\item $\alpha_2 W_1 \stackrel{*}{{\longleftrightarrow}_R} \alpha_2 W_2 \; \text{ and } \;
  \beta_2 W_1 \stackrel{*}{{\longleftrightarrow}_R} \beta_2 W_2$.
\end{enumerate}
In the latter (``two-mappings case'') case, we want to find strings~$W_1^{}$ and~$W_2^{}$,
\emph{not necessarily distinct,} such that \[
\alpha_1 W_1 \stackrel{*}{{\longleftrightarrow}_R} \alpha_2 W_2 \; \text{ and } \;
 \beta_1 W_1 \stackrel{*}{{\longleftrightarrow}_R} \beta_2 W_2 \]

%changes for formula breaking up to two lines -erased \; in the formula
The one-mapping case can be illustrated by an example. Consider the
term rewriting system $\{ a(a(b(z))) \, \rightarrow \, a(b(z)) \}$ and
two substitutions $\theta_1^{} = \{ x \mapsto b(c) \}$ and
$\theta_2^{} = \{ x \mapsto b(b(c)) \}$. Now
$a(a(x)) \; = \; a(x)$ is a common equation. Considering this
in the string rewriting setting, we have  $R ~ = ~ \{ baa \, \rightarrow \, ba \}$,
$\alpha = b$ and $\beta = bb$. Now $W_1 ~ = ~ aa$ and $W_2 ~ = ~ a$ is a solution.\\[-4pt]

Hence we define CE as the following decision problem:

\begin{description}[align=left]
\item[Input] A string-rewriting system~$R$ on an alphabet~$\Sigma$, and 
strings~$\alpha_1, \alpha_2, \beta_1, \beta_2 \in \Sigma_{}^*$.
\item[Question] Do there exist 
\emph{irreducible} 
strings~$W_1, W_2 \in \Sigma^*$ such that one of the following conditions is satisfied?
\end{description}
\begin{tabular}{ c c r }
i. & $\alpha_1 W_1 \stackrel{*}{{\longleftrightarrow}_R} \alpha_2 W_2$, & \multirow{2}{20em}{$\;\;\;\;\;\;\;$$\alpha_1 \neq \alpha_2$ $\lor$ $\beta_1 \neq \beta_2$} \\[5pt]
& $\beta_1 W_1 \; \stackrel{*}{{\longleftrightarrow}_R} \; \beta_2 W_2$, & \\[10pt]
ii. & $\alpha_i W_1 \stackrel{*}{{\longleftrightarrow}_R} \alpha_i W_2$, & \multirow{2}{20em}{$\;\;\;\;\;\;\;$ $i \in \{1, 2\} $ $\land$ $W_1 \neq W_2$} \\[5pt]
& $ \beta_i  W_1 \; \stackrel{*}{{\longleftrightarrow}_R} \; \beta_i W_2$,  & \\[10pt]
\end{tabular}

CE is also undecidable for dwindling systems, since
in the string-rewriting case CT is a particular case of~CE. To see this,
consider the case where $\alpha_1 \, \neq \, \alpha_2$ and
$\beta_1 = \beta_2 = \lambda$, i.e., consider
the substitutions
\begin{eqnarray*}
\theta^{}_1 & = & \{ x_1 \mapsto \alpha^{}_1, \; x_2 \mapsto \alpha^{}_2\} \\
\theta^{}_2 & = & \{ x_1 \mapsto \lambda ,  \; x_2 \mapsto \lambda \}
\end{eqnarray*}
where $\alpha_1 \, \neq \, \alpha_2$.
This has a solution if and only if there are irreducible strings~$W_1, W_2 \in \Sigma^*$ such that either

\begin{tabular}{ c c r }
i. & $\alpha_1 W_1 \stackrel{*}{{\longleftrightarrow}_R} \alpha_2 W_2$, & \multirow{2}{20em}{$\;\;\;\;\;\;\;$} \\[5pt]
& $ W_1 \; \stackrel{*}{{\longleftrightarrow}_R} \; W_2$, & $\qquad$ or \\[10pt]
ii. & $\alpha_i W_1 \stackrel{*}{{\longleftrightarrow}_R} \alpha_i W_2$, & \multirow{2}{20em}{$\;\;\;\;\;\;\;$ $i \in \{1, 2\} $ $\land$ $W_1 \neq W_2$} \\[5pt]
& $   W_1 \; \stackrel{*}{{\longleftrightarrow}_R} \;  W_2$,  & \\[10pt]
\end{tabular}
\ignore{
if $\beta_1 = \beta_2 = \lambda$, then \[
\alpha_1 W_1 \; \stackrel{*}{{\longleftrightarrow}_R} \; \alpha_2 W_2 \; \; ~ \text{and} ~ \; \;
W_1 \; \stackrel{*}{{\longleftrightarrow}_R} \; W_2 \]
}

%\todo{CT-CE reduction}
Since $W_1$ and $W_2$ are irreducible strings, $W_1 \;
\stackrel{*}{{\longleftrightarrow}_R} \; W_2$ makes $W_1$ equal to
$W_2$. Thus, we eliminate the second condition. 
%We call this new version of the problem as \emph{restricted CE} problem.

With the second condition being out of the picture, we only consider the first condition which shows a similarity with the definition of Common Term(CT) problem. The CE problem for $\theta_1$ and $\theta_2$ has a solution
if and only if the CT problem for $\alpha_1$ and $\alpha_2$ has a solution. Therefore, CT is reducible to CE problem.

It can also be shown, using the construction
from~\cite{NarendranOtto97} that there are theories for which
CT is decidable whereas CE is not\footnote{We can use Corollary~4.4
  on page~101 of~\cite{NarendranOtto97}}. In Appendix C, 
  we provided a simpler construction than the one in ~\cite{NarendranOtto97} for reader's ease of understanding.
  
We now show that CE is decidable for monadic string rewriting
systems.  We start by plunging into the \emph{two-mappings} case
first, since the solution for \emph{one-mapping} case is similar
to the \emph{two-mapping} case with a slightly simpler approach.

\subsection{Two-mapping CE Problem for Monadic Systems}
For monadic and convergent string rewriting systems, the \emph{two-mappings} case of Common
Equation (CE) problem is decidable. This can be
shown using Lemma~3.6 in~\cite{OND98}. (See also
Theorem~3.11 of~\cite{OND98}.) In fact, the algorithm runs
in polynomial time as explained below:%\todo{This has to be elaborated.}

%changes for formula breaking up to two lines \\
\begin{thm}
{\label{CEFirstTheorem}}
Common Equation (CE) problem, given below, is decidable in polynomial
time for monadic,
finite and convergent string rewriting systems.
\begin{description}[align=left]
\item[Input] A string-rewriting system~$R$ on an alphabet~$\Sigma$, and 
strings~$\alpha_1, \alpha_2, \beta_1, \beta_2 \in \Sigma_{}^*$.
\item[Question] Do there exist strings~$X, Y \in \Sigma^*$ such that 
$\alpha_1 X \; \stackrel{*}{{\longleftrightarrow}_R} \; \alpha_2 Y$ and \\
$\beta_1 X \; \stackrel{*}{{\longleftrightarrow}_R} \; \beta_2 Y$?
\end{description}
\end{thm}

\begin{proof}
  The CE problem is a particular case of the simultaneous
  E-unification problem of \cite{OND98}, but with a slight
  difference: CE consists of \emph{only two} equations, while
  simultaneous E-unification problem is defined for an arbitrary
  number of equations.  Besides, simultaneous E-unification problem is
  PSPACE-hard. We will use their construction, but we will modify
  it to obtain our polynomial time result.
  
%changes for formula breaking up to two lines $$
Given a monadic, finite and convergent string rewriting system~$R$
and irreducible strings~$x$ and~$y$, let
$RF(x, y)$ define the set of \emph{right factors} needed to derive~$y$, i.e.,
$$RF(x, y) = \{ z \in IRR(R) ~ | ~ xz \rightarrow_R^{!} y \}.$$
$RF(x, y)$ is a regular language for
all~$x$,~$y$~\cite{OND98} and a DFA
for it can be constructed in time polynomial
in $|R|$,~$|x|$ and~$|y|$. 
We can characterize the solutions of the equation
$\alpha_1 X \; \downarrow_R^{} \; \alpha_2 Y$ by an
analysis similar to that used in~Lemma~3.6 of~\cite{OND98}
and its proof. 
Since $R$ is
monadic, there exist $a,b \in \Sigma \cup \{\lambda\}$ and partitions
of the strings $\alpha_1 = {\alpha_{11}} \; {\alpha_{12}}$, $\alpha_2 =
{\alpha_{21}} \; {\alpha_{22}}$, $X = {X_1}\;{X_2}$ and $Y =
{Y_1} \; {Y_2}$, such that 
\begin{tabbing}
$\qquad$ \= $\alpha_{12} X_1 \; \rightarrow_{}^! \; a$,\\
         \> $\alpha_{22} Y_1 \; \rightarrow_{}^! \; b, \qquad$ and \\[+5pt]
         \> ${\alpha_{11}}\; a\; {X_2} = {\alpha_{21}}\; b\; {Y_2}$.\\
\end{tabbing}
Now there are 2 main cases:

\begin{itemize}
\item[(a)] ${X_2}$ is a suffix of ${Y_2}$: \\[+5pt]
 \indent \hspace{12pt} ${Y_2} = Z\; {X_2}$\\
 \indent \hspace{12pt} ${\alpha_{11}}\;a = {\alpha_{21}} \;b\; Z$ \hfill $RF({\alpha_{12}}, a) \times RF({\alpha_{22}}, b) \cdot Z$\\
\item[(b)] ${Y_2}$ is a proper suffix of  ${X_2}$: \\[+5pt]
 \indent \hspace{12pt} ${X_2} = Z''\; {Y_2}$\\
 \indent \hspace{12pt} ${\alpha_{11}} \; a \; Z'' = {\alpha_{21}} b$\\
 \indent \hspace{12pt} ${\alpha_{11}} \; a \; U = {\alpha_{21}}$, $\; Z'' = U\,b$
	\hfill $RF({\alpha_{12}}, a) \cdot Z'' \times RF({\alpha_{22}}, b)$
\end{itemize}

Similar partitioning can be done for the second equation. 

Let $Sol(\alpha_1, \alpha_2)$ stand for a set of `minimal' solutions:

\begin{equation*}
\begin{aligned}
Sol(\alpha_1, \alpha_2) \; = & \; \; \; \; \,
 \bigcup_{\substack{a,b \; \in \; \Sigma \; \cup \; \{\lambda\} \\[+3pt] {\alpha_{11}^{} a} = {\alpha_{21}^{}}\,b\,Z}} RF({\alpha_{12}^{}}, a) \; \Cross \; RF({\alpha_{22}^{}}, b) \cdot Z \; \\[+5pt]
 & \cup 
\bigcup_{\substack{a,b \; \in \; \Sigma \; \cup \; \{\lambda\} \\[+3pt] {\alpha_{21}^{} b} = {\alpha_{11}^{}}\,a\,Z''}} RF({\alpha_{12}^{}}, a) \cdot Z'' \; \Cross \; RF({\alpha_{22}^{}}, b) \\
\end{aligned}
\end{equation*}

\noindent
Note that this is a finite union of cartesian products of regular languages. More precisely,
it is an expression of the form \[
(L_{11}^{} \Cross L_{12}^{}) \; \cup \; \ldots \; \cup \; 
(L_{m1}^{} \Cross L_{m2}^{}) \] where $m$~is a polynomial over~$| \alpha_1^{} |$,
$| \alpha_2^{} |$ and~$| \Sigma |$ and each~$L_{ij}^{}$ has a DFA of size
polynomial in~$|R|$ and~$\text{max}( |\alpha_1|, \, |\alpha_2|)$.\\
To find the complexity of a DFA concatenation with a letter or string, 
check the Lemma~\ref{DFAConcatLetter} for the former and Lemma~\ref{DFAConcatString} for the latter.

\noindent
The set of all solutions for the equation $\alpha_1^{} X \; \downarrow_R^{} \;
\alpha_2^{} Y$ is \[
\Delta(\alpha_1, \alpha_2) ~ = ~ \left\{ \vphantom{b^b} \, (w_1 x_1, \, z_1 x_1) ~ \; | \; ~ 
(w_1, z_1) \in Sol(\alpha_1, \alpha_2) \text{ and }
x_1 \in IRR(R) \right\} \]

The minimal solutions for the second
equation with $\beta_1$ and $\beta_2$, $Sol(\beta_1 , \beta_2)$, 
can be found by following the
same steps.  Thus $Sol(\beta_1 , \beta_2)$ can also be expressed
as 
the union of cartesian products of regular languages:  \[
(L_{11}^{\prime} \Cross L_{12}^{\prime}) \; \cup \; \ldots \; \cup \; 
(L_{n1}^{\prime} \Cross L_{n2}^{\prime}) \] where 
$n$ is also a polynomial over $|\beta_1|$, $|\beta_2|$ and $|\Sigma|$.
The set of all solutions for $\beta_1 X = \beta_2 Y$ equals to \[
\Delta(\beta_1, \beta_2) ~ = ~ \left\{ \vphantom{b^b} \, (w_2 x_2, \, z_2 x_2) ~ \; | \; ~ 
(w_2, z_2) \in Sol(\beta_1, \beta_2) \text{ and }
x_2 \in IRR(R) \right\} \]

%changes for formula breaking up to two lines $$ and put "and" into \text
The solutions for both the equations are the tuples $(w, z)
\in \; \Delta(\alpha_1^{}, \alpha_2^{})\; \cap \;\Delta(\beta_1^{},
\beta_2^{})$. That is, there must be
$w_1,\;w_2,\;z_1,\;z_2,\;x_1,\;x_2$ such that $$(w_1,\;z_1) \in
Sol(\alpha_1^{}, \alpha_2^{}), (w_2,\;z_2) \in Sol(\beta_1^{},
\beta_2^{}) \text{ and}$$ 
\begin{equation*}
\begin{aligned}
w &= w_1\; x_1= w_2 \; x_2 \\
z &= z_1\; x_1 = z_2 \; x_2
\end{aligned}
\end{equation*}
If $x_1$ is a suffix of $x_2$, i.e., $x_2 = {x_2}^{\prime} x_1$, then
\begin{equation*}
\begin{aligned}
w_1 &= w_2 \; {x_2}^{\prime}\\ 
z_1 &= z_2 \; {x_2}^{\prime}
\end{aligned}
\end{equation*}
(Similarly we repeat the same steps when $x_2$ is a suffix of~$x_1$.) 

Recall that $(w_1, \;z_1) \in L_{i1}^{} \times L_{i2}^{}$ for some
$i>0$, and  $(w_2, \;z_2) \in L_{j1}^{\prime} \times L_{j2}^{\prime}$
for some $j>0$. Thus 
$x_2^{\prime} \; \in \; L_{i1}^{} \, \backslash \, L_{j1}^{\prime}$
and $x_2^{\prime} \; \in \; L_{i2}^{} \, \backslash \, L_{j2}^{\prime}$
where $\backslash$ stands for the left quotient operation on languages,
defined as 
$A \backslash B := \{v \in \Sigma^{*} ~ | ~ \exists u \in B : uv \in A \}$ (See Lemma~\ref{FindMEFA} for more detail).  
Thus there is a solution if
the intersection of $L_{i1}^{} \, \backslash \, L_{j1}^{\prime}$
and $L_{i2}^{} \, \backslash \, L_{j2}^{\prime}$ is nonempty. This check
has to be repeated for every~$i, \, j$. The process of finding the intersection of two languages is explained in Lemma~\ref{MEFAIntersection} 
and to be able to find the strings in the intersection and the quotient, you need to follow the steps in Lemma~\ref{FindingStringsThruIntersection}.
\end{proof}

\subsection{One-mapping CE Problem for Monadic Systems}
One-mapping case of the CE problem is decidable for monadic and
convergent string rewriting systems.  We can show it using a
construction similar to the two-mappings case. However it will be a
slightly simpler approach since we only have two input
strings~$\alpha$ and~$\beta$ as opposed to four.  The algorithm
for the one-mapping case also runs in polynomial time as explained below:

\begin{thm}{\label{CESecondTheorem}}
The following CE problem is decidable in polynomial time for monadic, finite and convergent string rewriting systems.
\begin{description}[align=left]
\item[Input] A string-rewriting system~$R$ on an alphabet~$\Sigma$, and irreducible
strings~$\alpha, \beta \in \Sigma_{}^*$.
\item[Question] Do there exist distinct irreducible strings~$X, Y \in IRR(R)$ such that 
\begin{enumerate}
\item[] $\alpha X \downarrow_R^{} \alpha Y \; \text{ and } \;
  \beta X \downarrow_R^{} \beta Y$?
\end{enumerate}
\end{description}
\end{thm}
\begin{proof}
This follows from Lemma~\ref{CEOneMappingMonadicLemma4},
since we only need to check whether there are
strings $X^{\prime}, Y^{\prime}, V, \gamma_1^{}, \gamma_2^{}$ such that
$X^{\prime} \text{ and } Y^{\prime}$ are distinct,
$\gamma_1^{}  \text{ and } \gamma_2^{}$ belong to
$ MP(\alpha) \text{ and } MP(\beta)$ respectively,
and one of the following symmetric cases hold:
\begin{enumerate}
  
  \item[(a)] $X^{\prime} V, \, Y^{\prime} V \, \in \, RF(\alpha, \gamma_1^{}) $ \emph{and}

  \item[(b)] $X^{\prime} , \, Y^{\prime} \, \in \, RF(\beta, \gamma_2^{})$
  
\end{enumerate} or \begin{enumerate}

  \item[(c)] $X^{\prime} , \, Y^{\prime} \, \in \, RF(\alpha, \gamma_1^{})$ \emph{and}

  \item[(d)] $X^{\prime} V, \, Y^{\prime} V \, \in \, RF(\beta, \gamma_2^{}) $

\end{enumerate}
First of all there are only polynomially many strings
in~$MP(\alpha)$ and $MP(\beta)$. The two cases can be
checked in polynomial time by Lemma~\ref{FindingStringsThruIntersection}. \end{proof}

\ignore{
we can check in polynomial time if there exist strings $X^\prime$, $Y^\prime$ and $V$.
}

\ignore{
  Since ~$R$ is a monadic rewriting system, there exist
  $a,b \in \Sigma \cup \{\lambda\}$, and strings
  $\alpha_1 , \alpha_2 , \alpha_3 , \alpha_4$, $X_1, X_2 , Y_1 , Y_2$ such that
  $\alpha = \alpha_1\; \alpha_2 = \alpha_3\; \alpha_4$, $X = X_1\, X_2$, $Y = Y_1\, Y_2$
  and \[ \alpha_2 X_1 \stackrel{*}{{\longrightarrow}_R} a , \; \;
  \alpha_4 Y_1 \stackrel{*}{{\longrightarrow}_R} b , \; \;
  \alpha_1 a X_2 ~ = ~ \alpha_3 b Y_2
  \]

  Now there are two cases (as before):

  \begin{enumerate}

  \item $X_2$ is a proper suffix of $Y_2$: Let $Y_2 \; = \; Z
    X_2$. Then $\alpha_1 a ~ = ~ \alpha_3 b Z$ and both $X_1$ and
    $Y_1 Z$ belong to~$RF( \alpha , \alpha_1 a )$.

  \item $Y_2$ is a suffix of $X_2$: Let $X_2 \; = \; UY_2$.
   Then $\alpha_1 a U ~ = ~ \alpha_3 b$. Thus both
   $X_1 U$ and $Y_1$ belong to~$RF( \alpha , \alpha_3 b )$.

  \end{enumerate}

\begin{itemize}
\item $\alpha_1\;a$, $\qquad \alpha_1 \in Prefix (\alpha)$
\item $\beta_1\;c$, $\qquad \beta_1 \in Prefix (\beta)$
\end{itemize}
The Lemmata ~\ref{CEOneMappingMonadicLemma1}, \ref{CEOneMappingMonadicLemma2}, 
\ref{CEOneMappingMonadicLemma3} and \ref{CEOneMappingMonadicLemma4} show that there exist different combinations
of $X$ and $Y$. To be able to find a solution with the given options, we should apply Lemma~\ref{FindMEFA} to be able to find the quotient of the strings 
as in $RF(\alpha, \gamma_1^{}) $.

As the second step, we should intersect the set of solutions with Lemma~\ref{MEFAIntersection} to find different strings for $X^\prime$ and $Y^\prime$. Then given a set of $X^\prime$ and $Y^\prime$s and
$X^{\prime} V, \, Y^{\prime} V$s, we apply Lemma~\ref{FindingStringsThruIntersection} to find the strings $X^{\prime}$, $Y^{\prime}$ and $V$.

By applying all those steps above, we find if there exists $X, Y \in \Sigma^*$ such that $\alpha X \stackrel{*}{{\longleftrightarrow}_R} \alpha Y \; \text{ and } \;
  \beta X \stackrel{*}{{\longleftrightarrow}_R} \beta Y$ holds.
}


\section{Future Work}
For the sake of brevity (and clarity), we only discussed string
rewriting systems in this paper. Our future work will include
the investigation of these two problems for general term rewriting systems.

\bibliographystyle{alpha}
\bibliography{unifs-duals-journal}

\begin{appendices}

%\appendix
\section{Appendix}
\noindent
The following term rewriting system $R_1^{{lin}}$ specifies a fragment of linear arithmetic using
\emph{successor} and \emph{predecessor} operators:
\begin{eqnarray*}
 x - 0 & \rightarrow & x \\
 x - x & \rightarrow & 0 \\
 s(x) - y & \rightarrow & s(x-y)\\
 p(x) - y & \rightarrow & p(x-y)\\
 x - p(y) & \rightarrow & s(x-y)\\
 x - s(y) & \rightarrow & p(x-y)\\
 p(s(x)) & \rightarrow & x\\
 s(p(x)) & \rightarrow & x
\end{eqnarray*}
We checked convergence of these rules on RRL (Rewrite Rule Laboratory)~\cite{RRL}. 

\begin{lem}\label{FPtoCTLemma}
The fixed point problem is reducible to the common term problem.
\end{lem}
\begin{proof}
Let $\theta_2$ be
the identity substitution. 
Assume that the fixed point problem has a solution, i.e., there exists
a term $t$ such that $\theta (t) \, \approx_E^{} t$. Then the CT problem
for $\theta$ and $\theta_2$ has a solution since
$\theta_2(t) \approx_E^{} t$ (because $\theta_2 (s) = s$ for all~$s$). The ``only if'' part is trivial, again
because $\theta_2 (s) = s$ for all~$s$.

Alternatively, suppose
that $\dom(\theta)$ consists of $n$~variables, where $n \geq 1$. If we
map all the variables in $\vran(\theta)$ to new constants, this will
create a ground substitution \[\theta_1 = \{x_1
\mapsto a_1,\; x_2 \mapsto a_2,\; ..., \;x_n \mapsto
a_n\}.\] $\theta_1$ will be the one of the substitutions for the CT
problem. The other substitution, $\theta_2$, is the composition of the
substitutions $\theta$ and $\theta_1$. The substitution $\theta_1$
will replace all of the variables in~$\vran(\theta)$ with the new
constants, thus making $\theta_2$ a ground substitution. Now
if $\theta (t) \, \approx_E^{} \, t$, then
$\theta_2 (t) = \theta_1 ( \theta (t) ) \approx_E^{} \theta_1( t )$; in other words,
$t$ is a solution to the common term problem.

The ``only if'' part can also be explained in terms of the composition
above. Suppose that $\theta_1(s)$ and $\theta_2(s)$ are equivalent,
i.e., $\theta_1(s) \, \approx_E^{} \, \theta_2(s)$ for some~$s$. Since
$\theta_2 = \theta_1 \circ \theta$, the equation can be rewritten as
$\theta_1(\theta(s)) \approx_E^{} \theta_1(s)$.
Since $a_1, \, \ldots , \, a_n$ are new constants and are not included in the signature of
the theory, for all $t_1$ and $t_2$, $\theta_1(t_1) \, \approx_E^{}
\theta_1(t_2)$ holds if and only if $t_1\approx_E^{} t_2$
(See~\cite{Term}, Section~4.1, page~60)
Thus $\theta_1(\theta(s)) \approx_E^{} \theta_1(s)$ implies that $\theta(s) \approx_E^{} s$, making
$s$~a fixed point. 
\end{proof}



%\appendix
\section{Appendix}
\begin{lem}{\label{DFAConcatLetter}}
Let $\, M \; = \; (Q, \Sigma, \delta, q_0^{}, F)$ be a DFA and $a \in \Sigma$. Then
there exists a DFA\\
$\, M_{}^{\prime} \; = \; (Q_{}^{\prime}, \Sigma, \delta_{}^{\prime}, q_0^{\prime}, F_{}^{\prime})$
that recognizes $\mathcal{L}(M) \, \circ \, \{ a \}$ such that
$| F_{}^{\prime} | \leq |F|$ and $| Q_{}^{\prime} | \; \le \; | Q | + | F |$.
\end{lem}
\begin{proof}
The concatenation of a letter with a DFA can be easily achieved by
adding extra transitions from each final state $q_{fi}$ to a new
state $p_i$ for the symbol~$a$. However, that turns the DFA $M$ into a
non-deterministic finite automaton (NFA). We claim that there exists a
DFA $M_{}^{\prime}$ with $|F|$ as the upper bound for accepting states.

Consider Figure~\ref{fig:test}: $q_{f1}$ through $q_{fn}$
are the $n$ final states of the given DFA~$M$. Suppose
$\delta (q_{fi}, a) = r_i$. Then the subset construction
gives us the new transition
$\delta{\prime} (\{ q_{fi} \}, a) = \{ r_i, p_i \}$.
The new accepting states for the
new DFA $M'$ will be $\{ r_i, p_i \}$, such that $1 \leq i \leq n$.
\begin{figure}[h]
\centering
\begin{tikzpicture}[->,>=stealth',shorten >=1pt,auto,node distance=1.8cm,
                    semithick, scale=0.5]
         \node[state, accepting] 		(F1)                   		{$q_{f1}$};
         \node[state, draw=none]	     	(Fi) [below=0.2cm of F1] 	{$\vdots$};
         \node[state, accepting]	     	(Fn) [below=0.2cm of Fi] 	{$q_{fn}$}; 
         \node[state]			     	(P1) [right of=F1] 		{$p_1$};
         \node[state]			     	(Pn) [right of=Fn] 		{$p_n$};
         \node[state] 				(R1) [below left of=F1]       {$r_1$};
         \node[state] 				(Rn) [below left of=Fn]       {$r_n$};
         \node[state]				(R2) [above left of=R1]	{$r_2$};
         \node[state]				(Rn2) [above left of=Rn]	{$r_{n2}$};
	 \node[state, draw=none]	     	(Info) [below right of=Rn]	{(a) $M$};
\path       
         (F1) edge node{$a$} (P1)
         	edge node{$a$} (R1)
	 (R1) edge node{$b$} (R2)	
         (Fn) edge node{$a$} (Pn)
         	edge node{$a$} (Rn)
	(Rn) edge node{$b$} (Rn2);
\end{tikzpicture} 
\hspace{3cm}
\begin{tikzpicture}[->,>=stealth',shorten >=1pt,auto,node distance=1.8cm,
                    semithick, scale=0.5]
         \node[state] 		(F1)                    {$q_{f1}$};
         \node[state, draw=none]	     	(Fi) [below=0.2cm of F1] {$\vdots$};
         \node[state]	     	(Fn) [below=0.2cm of Fi] {$q_{fn}$}; 
         \node[state, accepting]			     	(P1) [right of=F1] {$r_1, p_1$};
         \node[state, accepting]			     	(Pn) [right of=Fn] {$r_n, p_n$};
         \node[state] 				(R1) [below left of=F1]           {$r_1$};
         \node[state] 				(Rn) [below left of=Fn]           {$r_n$};
          \node[state]				(R2) [above left of=R1]	{$r_2$};
          \node[state]				(Rn2) [above left of=Rn]	{$r_{n2}$}; 
         \node[state, draw=none]	     	(Info)[below right of=Rn]{(b) $M'$};

\path       
         (F1) edge node{$a$} (P1)
         (Fn) edge node{$a$} (Pn)
	(P1) edge[bend right=40] node[swap]{$b$} (R2)
	(R1) edge node{$b$} (R2)
	(Pn) edge[bend left] node {$b$} (Rn2)
	(Rn) edge node{$b$} (Rn2);
\end{tikzpicture}
\caption{DFA $M$ concatenation with a single letter $a$.}
\label{fig:test}
\end{figure}

\ignore{
However, if the letter we want to concatenate does not have a
transition from the final states of $M$, the state of the string, $p_1
\ldots p_n$ will be the new final states of the DFA $M_{}^{\prime}$.
}

Besides, if the transitions for the letter~$a$ from
two earlier accepting states have the same destination state,
we can combine the new accepting states that were created.
Thus in
Figure~\ref{fig:sameLetterTransition}, the state
$\{ r_1, p_1 \}$ can be assigned to~$\delta^{\prime} (\{q_{fn}\}, a)$,
avoiding needless duplication.

Thus the number of final states, $|F_{}^{\prime}|$
for the DFA $M_{}^{\prime}$ is less than or equal to the original
number of final states, $|F|$, in DFA~$M$.

Total number of states $|Q_{}^{\prime}|$ for $M_{}^{\prime}$ is
bounded by the number of the final states $|F|$ in $M$ as well as the
number of total states, $|Q|$, in $M$.  Therefore, the
number of states for $M_{}^{\prime}$ can be less than or equal to the
both of the factors, i.e., $|Q_{}^{\prime}| \leq |Q|+|F|$.
\end{proof}


\begin{figure}%[h]
\centering
\resizebox{.55\linewidth}{!}{
\begin{tikzpicture}[->,>=stealth',shorten >=1pt,auto,node distance=1.8cm,
                    semithick, scale=0.5]
         \node[state, accepting] 		(F1)                   		{$q_{f1}$};
         \node[state, draw=none]	     	(Fi) [below=0.2cm of F1] 	{$\vdots$};
         \node[state, accepting]	     	(Fn) [below=0.2cm of Fi] 	{$q_{fn}$}; 
         \node[state]			     	(P1) [below right of=F1] 		{$p_1$};
         \node[state] 				(R1) [below left of=F1]       	{$r_1$};
	 \node[state, draw=none]	     	(Info) [below of=Fn]	{(a) $M$};
\path       
         (F1) edge node{$a$} (P1)
         	edge node[swap]{$a$} (R1)	
         (Fn) edge node[swap]{$a$} (P1)
         	edge node{$a$} (R1);
\end{tikzpicture} 
\hspace{3cm}
\begin{tikzpicture}[->,>=stealth',shorten >=1pt,auto,node distance=1.8cm,
                    semithick, scale=0.5]
         \node[state] 				(F1)                    		{$q_{f1}$};
         \node[state, draw=none]	     	(Fi) [below=0.2cm of F1] 	{$\vdots$};
         \node[state]	     			(Fn) [below=0.2cm of Fi] 	{$q_{fn}$}; 
         \node[state, accepting]		(P1) [below right of=F1] 		{$r_1, p_1$};
         \node[state, draw=none]	     	(Info)[below of=Fn]	{(b) $M_{}^{\prime}$};

\path       
         (F1) edge node{$a$} (P1)
         (Fn) edge node{$a$} (P1);
\end{tikzpicture}
}
\caption{DFA $M_{}^{\prime}$ can have less than $|F|$ states.}
\label{fig:sameLetterTransition}
\end{figure}

\begin{lem}{\label{DFAConcatString}}
Concatenation of a deterministic finite automaton (DFA) with a single
string has the time complexity $O(|F| \ast |Z| \ast
|\Sigma|)$, where $|F|$ is the number of final states in the DFA, $|Z|$
is the length of the string and $|\Sigma|$ is the size of
the alphabet.
\end{lem}
\begin{proof}
Recall the previous lemma proved that the number of states 
in the new DFA after
the concatenation of one letter 
is at most~$|Q| + |F|$ and that the number of 
(new) accepting states is at most~$F$. Thus repeatedly
applying this operation will result in a DFA with
at most~$|Q| + |Z| \ast |F|$ states and at most~$|F|$ accepting states.
The number of new edges will be at most~$|F| \ast |Z| \ast
|\Sigma|$. Thus the overall complexity is
polynomial in the size of the original~DFA. The Figure~\ref{fig:cloudDFA} shows an example of DFA $M$'s concatenation with the symbols in string $Z$. 
\end{proof}

\begin{figure}
\begin{tikzpicture}[->,>=stealth',shorten >=1pt,auto,node distance=1.8cm,
                    semithick]
  \coordinate (c) at (2,0);
  \draw[black,ultra thick,rounded corners=1mm] (c) \irregularcircle{3cm}{2mm};
  %\node[draw=blue, ultra thick, rectangle, minimum width=5cm, minimum height=5cm] at (1.8, 0){};
  \node[initial,state, circle] (A)                    {$q_0$};
  	
  \node[state, draw=none]         (H) [right of=A] {$\hdots$};
  \node[state, draw=none]	     (Fi) [right of=H] {$\vdots$};
  \node[state, accepting]	     (F1) [above of=Fi] {$q_{a1}$};
  \node[state, accepting]	     (Fn) [below of=Fi] {$q_{an}$}; 
  \node[state]			     (P1) [right =2.1 cm of Fi] {$p_1$}; %Extra distance for this node, to not include epsilons
  \node[state]			     (P2) [right of=P1] {$p_2$};
  \node[state, draw=none]	     (Pi) [right of=P2] {$\hdots$};
  \node[state]			     (Pn) [right of=Pi] {$p_n$};
  
\path (A) edge (H)
	(H) edge [bend left] (F1)
	      edge (Fi)
	      edge [bend right] (Fn)
	(F1) edge [bend left, blue] node{$a_1$} (P1)
	(Fi)  edge [blue] node{$a_1$} (P1)
	(Fn)  edge [bend right, blue] node{$a_1$} (P1)
	(P1) edge node{$a_2$} (P2)
	(P2) edge node{$a_3$} (Pi)
	(Pi) edge node{$a_n$} (Pn);
  
\end{tikzpicture}
\caption{DFA $M$ concatenation with a string $Z = a_1 a_2 \ldots a_n$.}
\label{fig:cloudDFA}
\end{figure}

%changes for formula breaking up to two lines \\
\begin{lem}\label{FindMEFA}
Let $M_1^{}$ and $M_2^{}$ be DFAs. Then a multiple-entry DFA (MEFA)
\\for $\mathcal{L}(M_2^{}) \, \backslash \, \mathcal{L}(M_1^{})$ can
be computed in polynomial time.
\end{lem}

%changes for formula breaking up to two lines - removed \; around the = signs
\begin{proof}
Let $M_1^{} = (Q_1^{}, \Sigma, \delta_1^{}, q_{01}^{}, F_1^{})$ and
$M_2^{} = (Q_2^{}, \Sigma, \delta_2^{}, q_{02}^{}, F_2^{})$ and 
let $L_{quo}^{} = \mathcal{L}(M_2^{}) \backslash \mathcal{L}(M_1^{})$.
A string~$y$ belongs to~$L_{quo}^{}$ if and only if
there exists a string~$x$ and states $p \in Q_1^{}$, $p' \in F_1^{}$
and $q \in F_2^{}$ such that
$\delta_{2}^* (q_{02}^{}, \, x) ~ = ~ q$,
$\delta_{1}^* (q_{01}^{}, \, x) ~ = ~ p$ and 
$\delta_{1}^* (p, \, y) ~ = ~ p'$.

%changes for formula breaking up to two lines $$
Let $P$ be the product transition system of the two automata, i.e., $$P
\; = \; (Q_1^{} \times Q_2^{}, \Sigma, \delta, (q_{01}^{} ,
q_{02}^{}))$$ where $\delta$ is defined as \[ \delta((r, r'), c) = (
\delta_1^{}(r, c) , \, \delta_2^{}(r', c) ) \] for all~$c \in \Sigma$,
$r \in Q_1^{}$ and $r' \in Q_2^{}$. We can assume that states which
are not reachable from~$(q_{01}^{} , q_{02}^{})$ have been removed.
This can be done in time linear in the size of the transition graph.
Now in terms of the transition system we can say that a string~$y$
belongs to~$L_{quo}^{}$ if and only if there exists a string~$x$ and
states $p \in Q_1^{}$, $p' \in F_1^{}$, $q \in F_2^{}$ and $q' \in
Q_2^{}$ such that \[ \delta_{}^* ((q_{01}^{} , q_{02}^{}), \, x) ~ = ~
(p, q) \text{  and  } \delta_{}^*((p, q), y) ~ = ~ (p', q') \] We can now
convert $P$
into a multiple-entry DFA~(MEFA). In the above case, $(p,q)$ has to be
one of the initial states of the new MEFA, and $(p', q')$ one
of its final states. Therefore, the states that are reachable
from $(q_{01}^{} , q_{02}^{})$ that are in $F_2^{} \times Q_1^{}$ will be
the initial states of the MEFA and $F_1^{} \times Q_2^{}$ will be the
final states of the~MEFA.
\end{proof}

\begin{lem}{\label{MEFAIntersection}}
Given two MEFAs, we can check whether their intersection is empty in polynomial time.
\end{lem}

%changes for formula breaking up to two lines $$ and \text{and}
\begin{proof}
Consider two MEFAs $A_1^{} \; = \; (Q_1^{}, \Sigma, \delta_1^{},
Q_{s_1}^{}, F_{A_1}^{})$ and $A_2^{} \; = \; (Q_2^{}, \Sigma,
\delta_2^{}, Q_{s_2}^{}, F_{A_2}^{})$. $Q_{s_1}^{}$ and
$Q_{s_2}^{}$ may include more than one initial state.
A string~$w$ is accepted by both MEFAs if and only if
there exist states 
$q_{init}^1 \; \in \; Q_{s_1}^{}$ and
$q_{init}^2 \; \in \; Q_{s_2}^{}$ such that
$$\delta_1^*( q_{init}^1 , w ) \; \in \; F_{A1}^{} \text{ and}
\delta_2^*( q_{init}^2 , w ) \; \in \; F_{A2}^{}.$$

To find such a string~$w$, we take the product transition system of the
two MEFAs, named as $T$, i.e., $T \; = \; (Q_1^{} \times Q_2^{},
\Sigma, \delta, (Q_{s_1}^{} \times Q_{s_2}^{}))$ where
\[ \delta((r, r'), c) = (
\delta_1^{}(r, c) , \, \delta_2^{}(r', c) ) \] for all~$c \in \Sigma$,
$r \in Q_1^{}$ and $r' \in Q_2^{}$.  A
string~$w$ is accepted by both of the MEFAs $A_1^{}$ and $A_2^{}$
if and only if there exist states $p, q, p', q'$ such that
$p \; \in \; Q_{s_1}^{}$, $q \; \in \; Q_{s_2}^{}$,
$p' \; \in \; F_{A_1}$ and $q' \; \in \; F_{A_2}$, and \[ \delta_{}^*((p , q) , w) ~ = ~ (p', q')
\] We can now apply \emph{depth-first search (DFS)} to check, in time
linear in the size of~$T$, 
if there exists a path from some state in $Q_{s_1}^{} \times Q_{s_2}^{}$ to a state 
in~$F_{A_1} \times F_{A_2}$.
\end{proof}

\begin{lem}{\label{FindingStringsThruIntersection}}
The following problem is decidable in polynomial time:

  \begin{quote}
    \begin{quote}
  \begin{description}[align=right]
  \item[{\bf Input}] DFAs $M^{}$ and $N^{}$.\\

  \item[{\bf Question}] Do there exist strings $x, y, z$ such that $x \neq y$,
    $x, y \in \mathcal{L}( M^{} )$, \emph{and}
    \mbox{$xz, yz \in \mathcal{L}( N^{} )$}?
  \end{description}
    \end{quote}
  \end{quote}
\end{lem}

\begin{proof}
Suppose there exist strings $x, y, z$ such that $x \neq y$, $x, y \in
\mathcal{L}( M^{} )$, \emph{and} \mbox{$xz, yz \in \mathcal{L}( N^{}
  )$}. We call the triple $(x, y, z)$ a \emph{solution}. Thus, we have two cases:

\begin{enumerate}[label=(\roman*)]
\item Both $x$ and $y$ start from an initial state $q_0$ and reach the
  same state, $q$, in $N$, i.e., \[ \exists\; q: \delta_{}^*(q_0^{N},
  x) = \delta_{}^*(q_0^{N}, y) = q \; \; \mathrm{and} \; \; \delta_{}^*(q, z) \in
  F^{N}. \]
\item $x$ and $y$ reach different states, say $q^{\prime}$~and~$q^{\prime\prime}$,
  in~$N$, i.e., \[ \exists\; q^{\prime}, q^{\prime\prime}: \delta_{}^*(q_0^{N},
  x) = q^{\prime} \neq q^{\prime\prime} = \delta_{}^*(q_0^{N}, y) \; \; \mathrm{and} \; \;
  \delta_{}^*(q^{\prime}, z) \in F^{N} \land \delta_{}^*(q^{\prime\prime}, z) \in
  F^{N}. \]
\end{enumerate}

Let $A=(Q, \Sigma, \delta, s, F)$ be a DFA and $p$ be a state in
$A$. By $A^{F = \{ p \}}$, we denote a replication of $A$, with
the sole difference of $p$ being the only accepting state of
$A$. Thus $N^{F = \{ q \}}$ denotes a replication of
$N$, with $q$ being the accepting state of $N$. Then, we classify
these states of~$N$ which \emph{are not dead states} into GREEN,
ORANGE and BLUE states. Note that confirming the status of~$q$
being a dead state can be done in linear time w.r.t.\ to the size
of graph.
%The following options give us two unequal strings, $x$ and $y$:
\begin{itemize}
\item \emph{GREEN states:} $\big\{ q \mid \; \scalebox{1.2}{$\mid$}  \mathcal{L}( N^{F=\{ q \}}) \cap \mathcal{L}( M^{}) \scalebox{1.2}{$\mid$} > 1 \big\}$. \\
  
  The state~$q$ mentioned in case~(i) is a GREEN state. (See Figure~\ref{fig:sub1})

\item \emph{ORANGE states:} $\big\{ q \mid \; \scalebox{1.2}{$\mid$}  \mathcal{L}( N^{F=\{ q \}}) \cap \mathcal{L}( M^{}) \scalebox{1.2}{$\mid$} = 1 \big\}$. \\
  
Suppose that case (i) does not apply, i.e., there are no GREEN states in~$N$. 
Then case (ii) must apply and the states $q^{\prime}$
and $q^{\prime\prime}$ must be ORANGE states; in other words, the intersection of
$\mathcal{L}( M^{})$ individually with the two DFAs, $\mathcal{L}(
N^{F=\{ q^{\prime} \}})$ and $\mathcal{L}( N^{F=\{ q^{\prime\prime} \}})$ gives us
exactly 1 string for each. Note also that $x$ and $y$ are two
strings in~$\mathcal{L}( M^{})$ which are not equal to each other since
$q_{}^{\prime} \; \neq \; q_{}^{\prime\prime}$. (See Figure~\ref{fig:sub2})

\item \emph{BLUE states:} $\big\{ q \mid \; \scalebox{1.2}{$\mid$}
  \mathcal{L}( N^{F=\{ q \}}) \cap \mathcal{L}( M^{})
  \scalebox{1.2}{$\mid$} = 0 \big\}$. \\
\end{itemize}

%\pagebreak

\begin{figure}[h]
\centering
\begin{subfigure}{.5\textwidth}
\centering
\begin{tikzpicture}[->,>=stealth',shorten >=1pt,auto,node distance=1.8cm,
                    semithick]
  \coordinate (c) at (2.5,0);
  \draw[black,ultra thick,rounded corners=1mm] (c) \irregularcircle{3.1cm}{2mm};
  %\node[draw=blue, ultra thick, rectangle, minimum width=5cm, minimum height=5cm] at (1.8, 0){};
  \node[initial,state, circle] (A)                    {$q_0$};
  	
  \node[state, draw=none]	     (Fi) [right of=A] {};
  \node[state, green]	     (F1) [above right=1 cm of Fi] {$q$};
  \node[state, accepting]			     (P1) [right=2.1 cm of Fi] {}; 
  \node[state, accepting]	     (P2) at (4.3,-1.60) {} ; 
  \node[state, draw=none]	     (D) at (4.5, -0.7) {$\vdots$};%Extra distance for this node, to not include epsilons%Extra distance for this node, to not include epsilons
  
\path 
	(A) edge [bend left] node{$x$} (F1)
	      edge [bend right] node{$y$} (F1)
	(F1) edge[dashed] node{$z$} (P1);
  
\end{tikzpicture}
\caption{Green States}
  \label{fig:sub1}
\end{subfigure}%
\begin{subfigure}{.5\textwidth}
\centering
\begin{tikzpicture}[->,>=stealth',shorten >=1pt,auto,node distance=1.8cm,
                    semithick]
  \coordinate (c) at (2.5,0);
  \draw[black,ultra thick,rounded corners=1mm] (c) \irregularcircle{3.1cm}{2mm};
  %\node[draw=blue, ultra thick, rectangle, minimum width=5cm, minimum height=5cm] at (1.8, 0){};
  \node[initial,state, circle] (A)                    {$q_0$};
  	
  \node[state, draw=none]	     (Fi) [right of=A] {};
  \node[state, orange]	     (F1) [above right=1 cm of Fi] {$q$};
  \node[state, orange]	     (Fn) [below right=1 cm of Fi] {$q^{\prime}$}; 
  \node[state, accepting]			     (P1) [right=2.1 cm of Fi] {}; 
  \node[state, accepting]	     (P2) at (4.3,-1.60) {} ; 
  \node[state, draw=none]	     (D) at (4.5, -0.7) {$\vdots$};%Extra distance for this node, to not include epsilons
  
\path 
	(A) edge [bend left] node{$x$} (F1)
	      edge [bend right] node{$y$} (Fn)
	(F1) edge[dashed] node{$z$} (P1)
	(Fn)  edge [dashed] node{$z$} (P1);
  
\end{tikzpicture}
\caption{Orange States}
  \label{fig:sub2}
\end{subfigure}
\caption{Different Structures of Green and Orange States.}
\label{fig:GOStates}
\end{figure}

The algorithm for finding the triple $(x, y, z)$ is constructed
as follows. First, we identify the green and orange states. If
there exists a green state, then we have a solution. Otherwise we
explore whether there exists a $z$ such that $\delta_{}^*(q^{\prime},
z) \in F^{N} \land \delta_{}^*(q^{\prime\prime}, z) \in F^{N}$
for orange states~$q_{}^{\prime}$
and~$q_{}^{\prime\prime}$, i.e., we check whether \[ \big\{ \, z \; \big| \; \exists (q^{\prime}, q^{\prime\prime}) : \, q^{\prime}, q^{\prime\prime} \text{ are
  orange states } \land \; \delta_{}^*(q^{\prime},z) \in F^{N} \; \land \; 
  \delta_{}^*(q^{\prime\prime},z) \in F^{N} \big\} \] is empty.

Given orange states
$q^{\prime}$ and $q^{\prime\prime}$, we use DFA intersection to check whether there is
a string $z$ that takes both to an accepting state.
Let $N_{s=\{q^{\prime}\}}$ denote a
replication of $N$, with the difference of $q^{\prime}$ being the initial state
of~$N$. $N_{s=\{q^{\prime\prime}\}}$ is similar to the $N_{s=\{q^{\prime}\}}$, but
this time $q^{\prime\prime}$ is the initial state. After creating these two
DFAs, we can find if there exists a string $z$ by intersecting the
DFA~$N_{s=\{q^{\prime}\}}$ with~$N_{s=\{q^{\prime\prime}\}}$. 
This process may have to be repeated for every tuple $(q^{\prime},
q^{\prime\prime})$ of orange states. \qedhere
\end{proof}

  \ignore{
        Given $x, y \in \mathcal{L}( M_1^{} )$, \emph{and} \mbox{$xz, yz \in \mathcal{L}( M_2^{} )$}, we can extract $z$ by taking the quotient of languages $\mathcal{L}( M_1^{} )$ and $\mathcal{L}( M_2^{} )$ , such that $z \in 		\mathcal{L}( M_2^{} ) \backslash \mathcal{L}( M_1^{} )$.
        
        The quotient of the languages $\mathcal{L}( M_1^{} )$ and $\mathcal{L}( M_2^{} )$ can be find by using \emph{Lemma~\ref{FindMEFA}}. Pay attention to the fact that the quotient of the languages creates a \emph{MEFA}, 		which has more than 1 initial state.
        
        For such a $z$, there exists different cases to discover $x$ and $y$'s, such that $x \neq y$:
        \begin{enumerate}[label=\roman*.]
        \item $\exists x_1,\; x_2 \in \mathcal{L}( M_1^{} )$ such that $x_1 z \in \mathcal{L}( M_2^{} )$ and $x_2 z \in \mathcal{L}( M_2^{} )$. 
        
        Therefore, $| \mathcal{L}( M_2^{} ) \cap \mathcal{L}( M_1^{} ) \circ \{ z\} | > 1$. \ignore{cdot}
        \item $\exists y_1,\; y_2 \in \mathcal{L}( M_1^{} )$ such that $y_1 z \in \mathcal{L}( M_2^{} )$ and $y_2 z \in \mathcal{L}( M_2^{} )$. 
        
        Same construction with the previous case, with the difference of $x$ changed with $y$. Preserves the result of $| \mathcal{L}( M_2^{} ) \cap \mathcal{L}( M_1^{} ) \circ \{ z\} | > 1$. 
        \item ...
        \end{enumerate}
        }

%Monadic-cone-theorem.tex inserted here
\begin{figure}
\centering
\epsfig{file=monadic-cone.eps, width=4in}
\caption{How the monadic cone looks for $\alpha$}
\label{monadicCone}
\end{figure}

The Figure~\ref{monadicCone} illustrates how $\alpha X$ reduces to its normal form $\alpha^{}_1 a X_2$. ($\alpha$
and $X$ are irreducible strings.)

Let $R$ be a convergent monadic SRS. For an irreducible string~$\alpha$, let \[ MP(\alpha) 
~ = ~ \Big\{ \, w\!\big\downarrow \; ~ \Big| ~ \; w \in PREF(\alpha) \circ (\Sigma \cup \{ \epsilon \})
      \Big\} \]

\emph{MP} stands for the term \emph{Minimal Product} and \emph{PREF}
is the set of \emph{prefixes} of given string.

%\pagebreak

\begin{lem}{\label{CEOneMappingMonadicLemma1}}
Let $\mu, \omega, X, Y \in IRR(R)$. Then $\mu X  \downarrow  \omega Y$ if and only
if there exist strings~$X^{\prime}, Y^{\prime}, W, \gamma$ such that
\begin{enumerate}

\item $\gamma \in MP(\mu) \cup MP(\omega)$,

\item $X = X^{\prime} W$, $Y = Y^{\prime} W$, \emph{and}

\item $\mu X^{\prime} \stackrel{!}{{\longrightarrow}_{R}^{}} ~ \gamma \; \; ~ {}_R^{}{\!{\stackrel{!}{\longleftarrow}}} \; \omega Y^{\prime}$.

\end{enumerate}


%$\presubsuper{!}{R}{\longleftarrow}$

%${}_R^{!}{\longleftarrow}$
\end{lem}
\begin{proof}
This proof as well as the proof for
Lemma~\ref{CEOneMappingMonadicLemma2} follows from~\cite{OND98}
(see Lemma~3.6).
\end{proof}

%\pagebreak

\begin{lem}{\label{CEOneMappingMonadicLemma2}}
\openup 0.5em
Let $\alpha_1^{}, \alpha_2^{}, \beta_1^{} , \beta_2^{} , X, Y \in IRR(R)$. Then $\alpha_1^{} X  \downarrow  \alpha_2^{} Y$
and $\beta_1^{} X  \downarrow  \beta_2^{} Y$ if and only
if there exist strings~$X^{\prime}, Y^{\prime}, V, W, \gamma_1^{}, \gamma_2^{}$ such that

\openup -0.5em
\begin{enumerate}

\item $\gamma_1^{} \in MP(\alpha_1^{}) \cup MP(\alpha_2^{})$,

\item $\gamma_2^{} \in MP(\beta_1^{}) \cup MP(\beta_2^{})$,

\item $X = X^{\prime} V W$, $Y = Y^{\prime} V W, \; \; $ \emph{and}

\item either \begin{itemize}

  \item[(a)] $\alpha_1^{} X^{\prime} V \stackrel{!}{{\longrightarrow}_{R}^{}} ~ \gamma_1^{} \; \; ~ {}_R^{}{\!{\stackrel{!}{\longleftarrow}}} \;
  \alpha_2^{} Y^{\prime} V \; \; $ \emph{and}

  \item[(b)] $\; \; \; \beta_1^{} X^{\prime} \stackrel{!}{{\longrightarrow}_{R}^{}} ~ \gamma_2^{} \; \; ~ {}_R^{}{\!{\stackrel{!}{\longleftarrow}}} \;
  \beta_2^{} Y^{\prime}$.
  
\end{itemize} or \begin{itemize}

  \item[(c)] $\; \; \; \alpha_1^{} X^{\prime} \stackrel{!}{{\longrightarrow}_{R}^{}} ~ \gamma_1^{} \; \; ~ {}_R^{}{\!{\stackrel{!}{\longleftarrow}}} \;
  \alpha_2^{} Y^{\prime} \; \; $ \emph{and}

  \item[(d)] $\beta_1^{} X^{\prime} V \stackrel{!}{{\longrightarrow}_{R}^{}} ~ \gamma_2^{} \; \; ~ {}_R^{}{\!{\stackrel{!}{\longleftarrow}}} \;
  \beta_2^{} Y^{\prime} V$.
  \end{itemize}

\end{enumerate}


%$\presubsuper{!}{R}{\longleftarrow}$

%${}_R^{!}{\longleftarrow}$
\end{lem}

\vspace{0.2in}

\begin{cor}
\openup 0.5em
  Let $\alpha_1^{}, \alpha_2^{}, \beta_1^{} , \beta_2^{} \in
  IRR(R)$. Then there exist irreducible strings~$X$~and~$Y$ such that
  $\alpha_1^{} X \downarrow \alpha_2^{} Y$ and $\beta_1^{} X
  \downarrow \beta_2^{} Y$ if and only if there exist
  strings~$X^{\prime}, Y^{\prime}, V, \gamma_1^{}, \gamma_2^{}$
  such that

\openup -0.5em
\begin{enumerate}

\item $\gamma_1^{} \in MP(\alpha_1^{}) \cup MP(\alpha_2^{})$,

\item $\gamma_2^{} \in MP(\beta_1^{}) \cup MP(\beta_2^{})$,

\item $X^{\prime} V$ and $Y^{\prime} V$ are irreducible, \emph{and} %$X = X^{\prime} V W$, $Y = Y^{\prime} V W, \; \; $ \emph{and}

\item either \begin{itemize}

  \item[(a)] $\alpha_1^{} X^{\prime} V \stackrel{!}{{\longrightarrow}_{R}^{}} ~ \gamma_1^{} \; \; ~ {}_R^{}{\!{\stackrel{!}{\longleftarrow}}} \;
  \alpha_2^{} Y^{\prime} V \; \; $ \emph{and}

  \item[(b)] $\; \; \; \beta_1^{} X^{\prime} \stackrel{!}{{\longrightarrow}_{R}^{}} ~ \gamma_2^{} \; \; ~ {}_R^{}{\!{\stackrel{!}{\longleftarrow}}} \;
  \beta_2^{} Y^{\prime}$.
  
\end{itemize} or \begin{itemize}

  \item[(c)] $\; \; \; \alpha_1^{} X^{\prime} \stackrel{!}{{\longrightarrow}_{R}^{}} ~ \gamma_1^{} \; \; ~ {}_R^{}{\!{\stackrel{!}{\longleftarrow}}} \;
  \alpha_2^{} Y^{\prime} \; \; $ \emph{and}

  \item[(d)] $\beta_1^{} X^{\prime} V \stackrel{!}{{\longrightarrow}_{R}^{}} ~ \gamma_2^{} \; \; ~ {}_R^{}{\!{\stackrel{!}{\longleftarrow}}} \;
  \beta_2^{} Y^{\prime} V$.
  \end{itemize}

\end{enumerate}
\end{cor}

%\pagebreak

\begin{lem}{\label{CEOneMappingMonadicLemma4}}
\openup 0.5em
  Let $\alpha_1^{}, \alpha_2^{}, \beta_1^{} , \beta_2^{} \in
  IRR(R)$. Then there exist irreducible strings~$X$~and~$Y$ such that
  $\alpha_1^{} X \downarrow \alpha_2^{} Y$ and $\beta_1^{} X
  \downarrow \beta_2^{} Y$ if and only if there exist
  strings~$X^{\prime}, Y^{\prime}, V, \gamma_1^{}, \gamma_2^{}$
  such that

\openup -0.5em
\begin{enumerate}

\item $\gamma_1^{} \in MP(\alpha_1^{}) \cup MP(\alpha_2^{})$,

\item $\gamma_2^{} \in MP(\beta_1^{}) \cup MP(\beta_2^{})$,

\item either \begin{itemize}
  
  \item[(a)] $X^{\prime} V \, \in \, RF(\alpha_1, \gamma_1^{}) $, $~ Y^{\prime} V \, \in \, RF(\alpha_2, \gamma_1^{}) $, 
             $~ X^{\prime} \, \in \, RF(\beta_1, \gamma_2^{})$ \emph{and} $~ Y^{\prime} \, \in \, RF(\beta_2, \gamma_2^{})$
  
\end{itemize} or \begin{itemize}

  \item[(b)] $X^{\prime} \, \in \, RF(\alpha_1, \gamma_1^{}) $, $~ Y^{\prime} \, \in \, RF(\alpha_2, \gamma_1^{}) $, 
             $~ X^{\prime} V  \, \in \, RF(\beta_1, \gamma_2^{})$ \emph{and} $~ Y^{\prime} V \, \in \, RF(\beta_2, \gamma_2^{})$

\end{itemize}

\end{enumerate}

\end{lem}


%\pagebreak

%%%%%%%%%%%%% One-Mapping Proofs%%%%%%%%%%%%%


\begin{lem}{\label{CEOneMappingMonadicLemma5}}
Let $\mu, X, Y \in IRR(R)$ where $X \neq Y$. Then $\mu X  \downarrow  \mu Y$ if and only
if there exist strings~$X^{\prime}, Y^{\prime}, W, \gamma$ such that
\begin{enumerate}

\item $\gamma \in MP(\mu)$,

\item $X = X^{\prime} W$, $Y = Y^{\prime} W$, $X^{\prime} \neq Y^{\prime} \; $ \emph{and}

\item $\mu X^{\prime} \stackrel{!}{{\longrightarrow}_{R}^{}} ~ \gamma \; \; ~ {}_R^{}{\!{\stackrel{!}{\longleftarrow}}} \; \mu Y^{\prime}$.

\end{enumerate}


%$\presubsuper{!}{R}{\longleftarrow}$

%${}_R^{!}{\longleftarrow}$
\end{lem}

%changes for formula breaking up to two lines \\
\begin{proof}
Let $Z$ be the normal form of $\mu X$ and $\mu Y$. \\
Then there exists strings
$\mu_1, \mu_2, \mu_3, \mu_4, X_1, X_2, Y_1, Y_2$ such that
$\mu = \mu_1\mu_2 = \mu_3\mu_4$, $X=X_1X_2$, $Y=Y_1Y_2$ and
\begin{tabbing}
$\qquad$ \= $\mu_{2} X_1 \; \rightarrow_{}^! \; a$,\\
         \> $\mu_{4} Y_1 \; \rightarrow_{}^! \; b, \qquad$ and \\[+5pt]
         \> $Z = {\mu_{1}}\, a \, {X_2} = {\mu_{3}} \, b \, {Y_2}$.\\
\end{tabbing}
where $a,b \in \Sigma \cup \{\lambda\}$.
If $X_1 = Y_1$, then the same reduction can
be applied on both sides, i.e., $\mu_{2} = \mu_{4}$ and~$a = b$.
But the rest of the string $X_2 \neq Y_2$
since $X \neq Y$. Therefore, we conclude that $X_1 \neq Y_1$.  It can
also be seen that $\mu_1 a, \, \mu_3 b \in MP(\alpha)$.

We now consider two cases:
\begin{enumerate}
  \item[(a)] $X_2$ is a suffix of $Y_2$: Let $Y_2 = Y_2^{\prime} X_2$.
  Then ${\mu_{1}}\, a \; = \; {\mu_{3}} \, b \, {Y_2^{\prime}}$. We can take
  $\gamma = {\mu_{1}}\, a$, $X^{\prime} = X_1$, $Y^{\prime} = Y_1 Y_2^{\prime}$
  and $W = X_2$.

  \item[(b)] $Y_2$ is a suffix of $X_2$: Let $X_2 = X_2^{\prime} Y_2$.
  Then ${\mu_{1}}\, a X_2^{\prime} \; = \; {\mu_{3}} \, b$.
  In this case we can take
  $\gamma = {\mu_{3}}\, b$, $X^{\prime} = X_1 X_2^{\prime}$, $Y^{\prime} = Y_1$
  and $W = X_2$. \qedhere
\end{enumerate}
\end{proof}

\vspace{0.1in}

%\pagebreak

\begin{lem}%{\label{CEOneMappingMonadicLemma3}}
\openup 0.5em
  Let $\alpha, \beta \in IRR(R)$. Then there exist \emph{distinct}
  irreducible strings~$X$~and~$Y$ such that $\alpha X \downarrow
  \alpha Y$ and $\beta X \downarrow \beta Y$ if and only if there
  exist irreducible strings~$X^{\prime}, Y^{\prime}, V, W, \gamma_1^{}, \gamma_2^{}$ such that

\openup -0.5em
\begin{enumerate}

\item $X^{\prime} \, \neq \, Y^{\prime}$,

\item $\gamma_1^{} \in MP(\alpha)$,

\item $\gamma_2^{} \in MP(\beta)$,

\item $X = X^{\prime} V W$, $Y = Y^{\prime} VW$, and

\item either \begin{itemize}
  
  \item[(a)] $\alpha X^{\prime} V \stackrel{!}{{\longrightarrow}_{R}^{}} ~ \gamma_1^{} \; \; ~ {}_R^{}{\!{\stackrel{!}{\longleftarrow}}} \;
  \alpha Y^{\prime} V \; \; $ \emph{and}

  \item[(b)] $\; \; \; \beta X^{\prime} \stackrel{!}{{\longrightarrow}_{R}^{}} ~ \gamma_2^{} \; \; ~ {}_R^{}{\!{\stackrel{!}{\longleftarrow}}} \;
  \beta Y^{\prime}$.
  
\end{itemize} or \begin{itemize}

  \item[(c)] $\; \; \; \alpha X^{\prime} \stackrel{!}{{\longrightarrow}_{R}^{}} ~ \gamma_1^{} \; \; ~ {}_R^{}{\!{\stackrel{!}{\longleftarrow}}} \;
  \alpha Y^{\prime} \; \; $ \emph{and}

  \item[(d)] $\beta X^{\prime} V \stackrel{!}{{\longrightarrow}_{R}^{}} ~ \gamma_2^{} \; \; ~ {}_R^{}{\!{\stackrel{!}{\longleftarrow}}} \;
  \beta Y^{\prime} V$.
\end{itemize}

\end{enumerate}

\end{lem}
\begin{proof}
Assume that there exist strings~$X^{\prime}, Y^{\prime}, V,
\gamma_1^{}, \gamma_2^{}$ that satisfy the properties above. Let us
consider the fifth property. It shows that $\alpha X^{\prime} V
\downarrow \alpha Y^{\prime} V$ as well as $\beta X^{\prime} V
\downarrow \beta Y^{\prime} V$.  Now suppose $X = X^{\prime} V$ and $Y
= Y^{\prime} V$. Therefore, we can see that $\alpha X \downarrow
\alpha Y$ as well as $\beta X \downarrow \beta Y$ and $X$ and $Y$ are
distinct irreducible strings.

Conversely, assume that there exist distinct irreducible strings $X$
and $Y$ such that $\alpha X \downarrow \alpha Y$ and $\beta X
\downarrow \beta Y$. We start by considering the case $\alpha X
\downarrow \alpha Y$ such that $X \neq Y$.
By Lemma~\ref{CEOneMappingMonadicLemma5}, there must be strings
$X^{\prime}$, $Y^{\prime}$, $Z$ and~$\gamma_1$ such that $X=X^{\prime} Z$,
$Y=Y^{\prime} Z$ and 
$\alpha X^{\prime} \; \stackrel{!}{{\longrightarrow}_{R}^{}} ~ \gamma_1 ~ \; \; {}_R^{}{\!{\stackrel{!}{\longleftarrow}}} \; \alpha Y^{\prime}$
where~$\gamma_1 \in MP(\alpha)$.
Similarly, for the case $\beta X \downarrow \beta
Y$, $X$ can be written as $X^{\prime\prime} Z^{\prime}$ and $Y$
can be written as $Y^{\prime\prime} Z^{\prime}$ such that 
$\beta X^{\prime\prime} \; \stackrel{!}{{\longrightarrow}_{R}^{}} ~ \gamma_2 \; \; ~ {}_R^{}{\!{\stackrel{!}{\longleftarrow}}} \; \beta Y^{\prime\prime}$ for some $\gamma_2 \in MP(\beta)$.

%$\beta X^{\prime\prime} \downarrow \beta Y^{\prime\prime}$.

We have to consider two cases depending on whether $X^{\prime}$ is a
prefix of $X^{\prime\prime}$ or vice versa. It is not hard to see that
they correspond to the two cases in condition~5.
\end{proof}

%\pagebreak

\begin{cor}{\label{CEOneMappingMonadicLemma3}}
\openup 0.5em
  Let $\alpha, \beta \in IRR(R)$. Then there exist \emph{distinct}
  irreducible strings~$X$~and~$Y$ such that $\alpha X \downarrow
  \alpha Y$ and $\beta X \downarrow \beta Y$ if and only if there
  exist irreducible strings~$X^{\prime}, Y^{\prime}, V, \gamma_1^{}, \gamma_2^{}$ such that

\openup -0.5em
\begin{enumerate}

\item $X^{\prime} \, \neq \, Y^{\prime}$,

\item $\gamma_1^{} \in MP(\alpha)$,

\item $\gamma_2^{} \in MP(\beta)$,

\item $X^{\prime} V$ and $Y^{\prime} V$ are irreducible, and

\item either \begin{itemize}
  
  \item[(a)] $\alpha X^{\prime} V \stackrel{!}{{\longrightarrow}_{R}^{}} ~ \gamma_1^{} \; \; ~ {}_R^{}{\!{\stackrel{!}{\longleftarrow}}} \;
  \alpha Y^{\prime} V \; \; $ \emph{and}

  \item[(b)] $\; \; \; \beta X^{\prime} \stackrel{!}{{\longrightarrow}_{R}^{}} ~ \gamma_2^{} \; \; ~ {}_R^{}{\!{\stackrel{!}{\longleftarrow}}} \;
  \beta Y^{\prime}$.
  
\end{itemize} or \begin{itemize}

  \item[(c)] $\; \; \; \alpha X^{\prime} \stackrel{!}{{\longrightarrow}_{R}^{}} ~ \gamma_1^{} \; \; ~ {}_R^{}{\!{\stackrel{!}{\longleftarrow}}} \;
  \alpha Y^{\prime} \; \; $ \emph{and}

  \item[(d)] $\beta X^{\prime} V \stackrel{!}{{\longrightarrow}_{R}^{}} ~ \gamma_2^{} \; \; ~ {}_R^{}{\!{\stackrel{!}{\longleftarrow}}} \;
  \beta Y^{\prime} V$.
\end{itemize}

\end{enumerate}

\end{cor}

\ignore{
Since the
rewrite system $R$ is monadic, there exist $a,b \in \Sigma \cup
\{\lambda\}$ such that
\begin{tabbing}
$\qquad$ \= $\alpha_{2} X_1 \; \rightarrow_{}^! \; a$,\\
         \> $\alpha_{4} Y_1 \; \rightarrow_{}^! \; b, \qquad$ and \\[+5pt]
         \> ${\alpha_{1}}\; a\; {X_2} = {\alpha_{3}}\; b\; {Y_2}$.\\
\end{tabbing}
\vspace{-15pt}The first reduction is shown in the
Figure~\ref{monadicCone}. If $X_1 = Y_1$, then the same reduction can
be applied on both sides, but the rest of the string $X_2 \neq Y_2$
since $X \neq Y$. Therefore, we conclude that $X_1 \neq Y_1$.  It can
also be seen that $\alpha_1\;a, \alpha_3\;b \in MP(\alpha)$.

The same analogy can be applied to the $\beta$ equation. Therefore we
say that $\alpha_1 \& \alpha_2$ works with $X$ and $Y$ such that $X=
X^{\prime}_1\;W_1$ and $Y = Y^{\prime}_1\;W_1$. Then for the both
sides of $\beta$ reduction, we can say $X= X^{\prime}_2\;W_2$ and $Y =
Y^{\prime}_2\;W_2$. We consider two cases such that:

\begin{itemize}
\item[(a)] ${W_2}$ is a suffix of ${W_1}$: \\[+5pt]
 \indent \hspace{12pt} ${W_1} = Z\; {W_2}$\\
 \indent \hspace{12pt} ${\alpha}\; X^{\prime}_1\;Z\; W_2 \stackrel{*}{{\longleftrightarrow}_R}  {\alpha} \; Y^{\prime}_1\;Z\;W_2$ and ${\beta}\;X^{\prime}_2\;W_2 \stackrel{*}{{\longleftrightarrow}_R}  {\beta} \;Y^{\prime}_2\; W_2$\\
\item[(b)] ${W_1}$ is a suffix of  ${W_2}$: \\[+5pt]
 \indent \hspace{12pt} ${W_2} = Z^{\prime}\; {W_1}$\\
  \indent \hspace{12pt} ${\alpha}\; X^{\prime}_1\;W_1 \stackrel{*}{{\longleftrightarrow}_R}  {\alpha} \; Y^{\prime}_1\; W_1$ and ${\beta}\;X^{\prime}_2\;Z^{\prime}\;W_1 \stackrel{*}{{\longleftrightarrow}_R}  {\beta}\;Y^{\prime}_2\; Z^{\prime}\;W_1$
\end{itemize}
}

%\pagebreak


\begin{lem}{\label{CEOneMappingMonadicLemma4}}
\openup 0.5em
  Let $\alpha, \beta \in IRR(R)$. Then there exist \emph{distinct}
  irreducible strings~$X$~and~$Y$ such that $\alpha X \downarrow
  \alpha Y$ and $\beta X \downarrow \beta Y$ if and only if there
  exist strings~$X^{\prime}, Y^{\prime}, V, \gamma_1^{}, \gamma_2^{}$ such that

\openup -0.5em
\begin{enumerate}

\item $X^{\prime} \, \neq \, Y^{\prime}$,

\item $\gamma_1^{} \in MP(\alpha)$,

\item $\gamma_2^{} \in MP(\beta)$,

\item either \begin{itemize}
  
  \item[(a)] $X^{\prime} V, \, Y^{\prime} V \, \in \, RF(\alpha, \gamma_1^{}) $ \emph{and}

  \item[(b)] $X^{\prime} , \, Y^{\prime} \, \in \, RF(\beta, \gamma_2^{})$
  
\end{itemize} or \begin{itemize}

  \item[(c)] $X^{\prime} , \, Y^{\prime} \, \in \, RF(\alpha, \gamma_1^{})$ \emph{and}

  \item[(d)] $X^{\prime} V, \, Y^{\prime} V \, \in \, RF(\beta, \gamma_2^{}) $

\end{itemize}

\end{enumerate}

\end{lem}


%\appendix
\section{Appendix}\label{CTdecidableCEnot}
This corollary is a slightly simpler construction than the one
in~\cite{NarendranOtto97} for the reader's ease of understanding.

\begin{cor}
There is a finite and convergent string system for which the common
term (CT) problem is decidable, while the common equation (CE) problem
is undecidable.
\end{cor}

We show that CE is undecidable by a reduction from GPCP. (See Section~\ref{ctdwindling}.)
For notational convenience, we represent an instance of the GPCP as a 3-tuple
\begin{equation*}
\Bigg \langle \bigg[ \dfrac{x}{y} \bigg], \, S, \, \bigg[ \dfrac{u}{v} \bigg] \Bigg \rangle
\end{equation*}
where $(x, y)$ is the start domino, $(u, v)$ is the end domino and
$S$ the set of intermediate dominos.

\begin{thm}[\cite{HopcroftMotwaniUllman}]
There exist strings $\alpha$, $\beta$, and a set of tuples of
strings~$S$ such that following problem is undecidable:

\begin{description}[align=left]
\item[Input] Strings $x_0, y_0$.
\item[Question] Does the GPCP $\Bigg \langle \bigg[ \dfrac{x_0}{y_0} \bigg], \, S, \, \bigg[ \dfrac{\alpha}{\beta} \bigg] \Bigg \rangle$ have a solution?
\end{description}
\end{thm}

%changes for formula breaking up to two lines $$
Let $\Bigg \langle \bigg[ \dfrac{x_0}{y_0} \bigg], \, \left\{(x_i,\; y_i)\right \}^{n}_{i=1} \,
\bigg[ \dfrac{x_{n+1}}{y_{n+1}} \bigg] \Bigg \rangle$ be an instance of the GPCP, i.e.,
$\mathop{\left\{(x_i,\; y_i)\right \}}^{n}_{i=1}$ is the set of
intermediate dominoes and $(x_0, y_0), (x_{n+1}, y_{n+1})$, the
start and end dominoes respectively, where the strings
are over the alphabet~$\Sigma = \{a,\, b\}$.
Let $$\widehat{\Sigma}= \{a,b\} \, \cup \, \{c_0,\ldots, c_n\} \, \cup \, \{ \cent_1, \cent_2, \$, \#_1, \#_2 \}$$ be
the new alphabet for the instance of~CE.

From the given instance, we construct a string-rewriting system $R_1^{}$ with the following rules:

\begin{equation*}
\begin{aligned}[c]
\cent_1 \;c_1 & \rightarrow & x_1 \;\cent_1, \\
\vdotswithin{\cent_1 \;c_n} & \rightarrow & \vdotswithin{x_n \cent_1}\\
\cent_1 \;c_{n+1} & \rightarrow & \#_1 \;x_{n+1} \$ \\[20pt]
\cent_2 \;c_1 & \rightarrow & y_1 \;\cent_2, \\
\vdotswithin{\cent_2 \;c_n} & \rightarrow & \vdotswithin{y_n \cent_2}\\
\cent_2 \;c_{n+1} & \rightarrow & \#_2 \;y_{n+1} \$
\end{aligned}
\qquad
\qquad
\qquad
\begin{aligned}[c]
a\; \#_1 & \rightarrow & \#_1\; a\\
b\; \#_1 & \rightarrow &  \#_1\; b\\[20pt]
a\; \#_2 & \rightarrow & \#_2\; a\\
b\; \#_2 & \rightarrow &  \#_2\; b\\
\end{aligned}
\end{equation*}

\begin{lem}
$R_1^{}$ is convergent.
\end{lem}

\begin{lem}
GPCP has a solution if and only if there exist strings $w_1, w_2$ such that
$x_0 \cent_1\; w_1 \rightarrow^{!} \#_1 w_2$ and $y_0 \cent_2\; w_1
\rightarrow^{!} \#_2 w_2$.
\end{lem}

\begin{proof}
We prove the ``if" direction first. Assume GPCP has a solution. 
Let $i_1, \ldots, i_k$ be the indices of the intermediate dominoes used, i.e., there are~$k+2$ dominoes 
in all
including the start and end~dominoes. This will result in
$x_0\;x_{i_1}, \ldots, x_{i_k}\; x_{n+1} =
y_0\;y_{i_1}, \ldots, y_{i_k}\; y_{n+1}$.
Let
$w_1 = c_{i_1} \ldots c_{i_k} c_{n+1}^{}$ and $w_2 = x_0\;x_{i_1} \ldots x_{i_k}\;x_{n+1} \$$, then
\begin{equation*}
\begin{aligned}[c]
x_0\; \cent_1 c_{i_1} \ldots c_{i_k} c_{n+1} & \rightarrow x_0\; x_{i_1} \cent_1 \; \ldots \; c_{i_k} c_{n+1} \rightarrow^{!} \#_1 \;x_0\;x_{i_1} \ldots x_{i_k}\;x_{n+1} \$\\
y_0\; \cent_2 c_{i_1} \ldots c_{i_k} c_{n+1} & \rightarrow y_0\; y_{i_1} \cent_2 \; \ldots \; c_{i_k} c_{n+1} \rightarrow^{!} \#_1 \;y_0\;y_{i_1} \ldots y_{i_k}\;y_{n+1} \$
\end{aligned}
\end{equation*}

%changes for formula breaking up to two lines $$ and \text{and}
For the ``only if" direction suppose there exist strings
$w_1$ and $w_2$ such that $$x_0\; \cent_1\; w_1 \rightarrow^{!} \#_1 w_2 \text{ and}
y_0\; \cent_2\; w_1 \rightarrow^{!} \#_2 w_2.$$
Without loss of generality, we can assume that $w_1$ and $w_2$ are irreducible.
The observation of the rules on the both sides shows that $\#_1 w_2$ and $ \#_2 w_2$ can be derived
if and only if $c_{n+1}$ occurs in~$w_1$ since only the rules with~$c_{n+1}$
has a $\cent$ on the left-hand side (LHS) and a
$\#$~symbol on the RHS.

Thus, we can write $w_1$ as $w_1 = {w_1^{\prime}} \; c_{n+1}\;
{w_1^{\prime\prime}}$ such that ${w_1^{\prime}} c_{n+1}$ is the shortest
prefix of $w_1$ that contains~$c_{n+1}$.  To be able to apply the
rules with $\cent$ signs, ${w_1^{\prime}}$ should be in~$\{c_1,
\ldots, c_n\}_{}^{*}$. Let
${w_1}^{\prime} \, = \, c_{i_1}^{} \ldots c_{i_k}^{}$ where $k = | {w_1}^{\prime} |$.
\begin{equation*}
\begin{aligned}[c]
x_0\;\cent_1\; {w_1}^{\prime}\;c_{n+1} & \rightarrow_{}^* \#_1 x_{0}\;x_{i_1}\; \ldots \;x_{i_k}\;x_{n+1} \$ \\
y_0\;\cent_2\; {w_1}^{\prime}\;c_{n+1} & \rightarrow_{}^* \#_2 y_{0}\;y_{i_1}\; \ldots \;y_{i_k}\;y_{n+1} \$
\end{aligned}
\end{equation*}

Since $\$$ does not occur on the left hand side of the rules, the string after the $\$$ sign, i.e., ${w_1^{\prime\prime}}$,
does not take part in the reductions. Thus

\begin{equation*}
\begin{aligned}[c]
x_0\;\cent_1\; {w_1}^{\prime}\;c_{n+1}\; {w_1^{\prime\prime}} & \rightarrow_{}^! \#_1 x_{0}\;x_{i_1}\; \ldots \;x_{i_k}\;x_{n+1} \$\; {w_1^{\prime\prime}}\\
y_0\;\cent_2\; {w_1}^{\prime}\;c_{n+1}\; {w_1^{\prime\prime}} & \rightarrow_{}^! \#_2 y_{0}\;y_{i_1}\; \ldots \;y_{i_k}\;y_{n+1} \$\; {w_1^{\prime\prime}}
\end{aligned}
\end{equation*}

\ignore{$\ldots$ \todo{Finish this.}
$w_2$ is in the form of domino strings, thus, there exists a solution for GPCP if and only if $w_1$ and $w_2$ exist such that $x_0 \cent_1\; w_1 \rightarrow^{!} \#_1 w_2$ and $y_0 \cent_2\; w_1
  \rightarrow^{!} \#_2 w_2$.
}

Thus it must be that $x_{0}\;x_{i_1}\; \ldots \;x_{i_k}\;x_{n+1} \; = \; y_{0}\;y_{i_1}\; \ldots \;y_{i_k}\;y_{n+1}$,
which is a solution to the~GPCP.
\end{proof}

\begin{thm}
The CE problem is undecidable for the finite and convergent string rewriting system~$R_1^{}$.
\end{thm}

\begin{lem}
For all $a \in \widehat{\Sigma}$ and strings $Z_1$ and $Z_2$,
$~ Z_1 a \, \downarrow \, Z_2 a ~$ if and only if $~ Z_1 \, \downarrow \, Z_2$.
\end{lem}

\begin{proof}
  Since $R_1^{}$ is convergent, all we need to prove is that
  if $Z_1, Z_2 \in IRR(R_1^{})$ and $a \in \widehat{\Sigma}$, then
  $Z_1 a \; \downarrow \; Z_2 a$ if and only if $Z_1 = Z_2$.

%changes for formula breaking up to two lines \\
  Let $c \in \widehat{\Sigma}$ such that $Z_1 c \, \downarrow \, Z_2 c$
  and $Z_1 \neq Z_2$ where $Z_1$ and $Z_2$ are irreducible strings.
  Clearly either $Z_1 c$ or $Z_2 c$ must be reducible.
  \\Thus it has to be that
  $c \in \left\{ c_1, \ldots , c_{n+1} \right\} \cup \{ \#_1 , \#_2 \}$.
  We now need to consider three cases:

  \begin{enumerate}[label=\roman*.]

  \item $c \in \left\{ c_1, \ldots , c_{n} \right\}$:\\
  The observation of the rules and the set $c$ belongs shows that, $Z_1$ and $Z_2$ should end with $\cent_1$ or $\cent_2$ to make $Z_1 c \; \downarrow \; Z_2 c$.
  Let $i$ be an index, such that $0 \leq i \leq n$ and $Z_1$ and $Z_2$ can be written as $Z_1 = {Z_1^{\prime}}\; \cent_1$ and $Z_2 = {Z_2^{\prime}}\; \cent_1$: 
  \begin{equation}
	\begin{aligned}[c]
		{Z_1^{\prime}}\;\cent_1\;c_i &\rightarrow {Z_1^{\prime}}\;x_i\;\cent_1 \\
		{Z_2^{\prime}}\;\cent_1\;c_i &\rightarrow {Z_2^{\prime}}\;x_i\;\cent_1
	\end{aligned}
  \end{equation}
  \noindent Since ${Z_1^{\prime}}, {Z_2^{\prime}} \in IRR(R_1^{})$, so are ${Z_1^{\prime}}\;x_i$ and ${Z_2^{\prime}}\;x_i$, since no left-hand sides end with either $a$ or $b$. 
  Hence, ${Z_1^{\prime}} = {Z_2^{\prime}}$.
    
  \item $c = \#_1$:\\
  Suppose $Z_1 = {Z_1^{\prime}}\;{Z_1^{\prime\prime}}\;$  and $Z_2 = {Z_2^{\prime}}\;{Z_2^{\prime\prime}}\;$ such that ${Z_1^{\prime\prime}}$ and ${Z_2^{\prime\prime}}$
  are the longest suffixes of $Z_1$ and $Z_2$ that belong to $\{a,b\}^{*}$. Thus:
   \begin{equation}
	\begin{aligned}[c]
		{Z_1^{\prime}}\;{Z_1^{\prime\prime}}\; \#_1&\rightarrow^{!} {Z_1^{\prime}}\; \#_1 \; {Z_1^{\prime\prime}} \\
		{Z_2^{\prime}}\;{Z_2^{\prime\prime}}\; \#_1&\rightarrow^{!} {Z_2^{\prime}}\; \#_1 \; {Z_2^{\prime\prime}}
	\end{aligned}
  \end{equation}
  Since ${Z_1^{\prime}} \#_1 {Z_1^{\prime\prime}}$ and ${Z_2^{\prime}} \#_1 {Z_2^{\prime\prime}}$ are irreducible, ${Z_1^{\prime}} = {Z_2^{\prime}}$ and 
  ${Z_1^{\prime\prime}} = {Z_2^{\prime\prime}}$.
  
  \item $c = \#_2$:\\  
  Follows the same construction as in the previous case, with the only difference being $\#_2$ instead of $\#_1$.
  
  \item $c = c_{n+1}$:\\
  Thus,
  \ignore{The similar construction to the case~(i) above, such that $\cent_1$ will be changed with $\#_1$ and $\$$ on the right-hand side of the reduction:
         }
  \begin{equation}
	\begin{aligned}[c]
		{Z_1^{\prime}}\;\cent_1\;c_{n+1} &\rightarrow {Z_1^{\prime}}\; \#_1 \;x_{n+1}\;\$  \\
		{Z_2^{\prime}}\;\cent_1\;c_{n+1} &\rightarrow {Z_2^{\prime}}\; \#_1 \;x_{n+1}\;\$
	\end{aligned}
  \end{equation}
  Since there are no left-hand sides that end with $a$, $b$
  or {\textdollar}, we have ${Z_1^{\prime}}\; \#_1 \downarrow {Z_2^{\prime}}\; \#_1$.
  By the case~(ii) above, we get ${Z_1^{\prime}} = {Z_2^{\prime}}$. \qedhere
  \end{enumerate}    
\end{proof}

\begin{thm}
The CT problem is decidable for the finite and convergent string rewriting system~$R_1^{}$.
\end{thm}

\end{appendices}

\end{document}