% \documentclass[twocolumn]{svjour3}
\documentclass{svjour3}

\usepackage{inputenc}

\usepackage[letterpaper,centering,margin=1in]{geometry}
%\usepackage[colorlinks=false]{hyperref}
\usepackage{natbib} % loaded by svjour3 with natbib option

\usepackage{amsfonts} 
\usepackage{amsmath}
\usepackage{amssymb}
\usepackage{mathtools}
\usepackage{bm}
\usepackage{url}

\usepackage[color=blue!30!white,textsize=tiny]{todonotes}
\setlength{\marginparwidth}{2.1cm}

\usepackage{algorithm}
\usepackage[noend]{algpseudocode}
%\usepackage[linesnumbered,noend]{algorithmic}
%\usepackage{algorithmicx}

\usepackage{graphicx}
%\usepackage{subcaption}

\DeclareMathOperator*{\argmin}{arg\,min}
\DeclareMathOperator*{\argmax}{arg\,max}
\DeclareMathOperator*{\diag}{diag}
\DeclareMathOperator*{\var}{Var}
\DeclareMathOperator*{\vol}{Vol}
\DeclareMathOperator*{\E}{\mathbb{E}}
\DeclareMathOperator*{\sign}{sign}

\newtheorem{assumption}{Assumption}

\smartqed  % flush right qed marks, e.g. at end of proof

% \usepackage{mathptmx}      % use Times fonts if available on your TeX system
%
% insert here the call for the packages your document requires
%\usepackage{latexsym}
% etc.
%
% please place your own definitions here and don't use \def but
% \newcommand{}{}
%
% Insert the name of "your journal" with
% \journalname{myjournal}

% \newcommand{\edits}[1]{{\color{blue!80!black}{#1}}}
\newcommand{\edits}[1]{{\color{black}{#1}}}

\newcommand{\moreedits}[1]{{\color{black}{#1}}}

\DeclareUnicodeCharacter{2212}{\textendash}

\begin{document}

\title{Rate-optimal refinement strategies for local approximation MCMC
  \thanks{AS was supported by NSERC. AD and YM were supported by the SciDAC program of the DOE Office of Advanced Scientific Computing Research.}
% [Grants or other notes about the article that should go on the front page should be placed here. General acknowledgments should be placed at the end of the article.]}
}

\subtitle{}

\author{Andrew D. Davis \and Youssef Marzouk \and Aaron Smith \and Natesh Pillai}
\institute{A.~D.~Davis \at
                Courant Institute of Mathematical Sciences,
                New York, NY USA
              \email{davisad@alum.mit.edu}           %  \\
%             \emph{Present address:} of F. Author  %  if needed
           \and
          Y.~M.~Marzouk \at
            Massachusetts Institute of Technology,
            Cambridge, MA USA
            \email{ymarz@mit.edu}
          \and
          A.~M.~Smith \at
            University of Ottawa,
            Ottawa, Canada
            \email{asmi28@uottawa.ca}
            \and
          N.~Pillai \at
            Harvard University,
            Cambridge, MA USA
            \email{pillai@fas.harvard.edu}
}

\titlerunning{Rate-optimal refinement for local approximation MCMC}        % if too long for running head

\authorrunning{Davis et al.} % if too long for running head

\maketitle

\begin{abstract}
Many Bayesian inference problems involve target distributions whose density functions are computationally expensive to evaluate. Replacing the target density with a local approximation based on a small number of carefully chosen density evaluations can significantly reduce the computational expense of Markov chain Monte Carlo (MCMC) sampling. Moreover, continual refinement of the local approximation can guarantee asymptotically exact sampling. We devise a new strategy for balancing the decay rate of the bias due to the approximation with that of the MCMC variance. We prove that the error of the resulting local approximation MCMC (LA-MCMC) algorithm decays at roughly the expected $1/\sqrt{T}$ rate, and we demonstrate this rate numerically. We also introduce an algorithmic parameter that guarantees convergence given very weak tail bounds, significantly strengthening previous convergence results. Finally, we apply LA-MCMC to a computationally intensive Bayesian inverse problem arising in groundwater hydrology.

\keywords{Markov chain Monte Carlo \and local regression \and Bayesian inference \and surrogate models \and sampling methods}
\PACS{02.50.−r \and 02.50.Ng \and 02.50.Tt \and 02.70.Uu}
% \subclass{MSC code1 \and MSC code2 \and more}
\end{abstract}

% \leavevmode
% \\
% \\
% \\
% \\
% \\
\section{Introduction}
\label{introduction}

AutoML is the process by which machine learning models are built automatically for a new dataset. Given a dataset, AutoML systems perform a search over valid data transformations and learners, along with hyper-parameter optimization for each learner~\cite{VolcanoML}. Choosing the transformations and learners over which to search is our focus.
A significant number of systems mine from prior runs of pipelines over a set of datasets to choose transformers and learners that are effective with different types of datasets (e.g. \cite{NEURIPS2018_b59a51a3}, \cite{10.14778/3415478.3415542}, \cite{autosklearn}). Thus, they build a database by actually running different pipelines with a diverse set of datasets to estimate the accuracy of potential pipelines. Hence, they can be used to effectively reduce the search space. A new dataset, based on a set of features (meta-features) is then matched to this database to find the most plausible candidates for both learner selection and hyper-parameter tuning. This process of choosing starting points in the search space is called meta-learning for the cold start problem.  

Other meta-learning approaches include mining existing data science code and their associated datasets to learn from human expertise. The AL~\cite{al} system mined existing Kaggle notebooks using dynamic analysis, i.e., actually running the scripts, and showed that such a system has promise.  However, this meta-learning approach does not scale because it is onerous to execute a large number of pipeline scripts on datasets, preprocessing datasets is never trivial, and older scripts cease to run at all as software evolves. It is not surprising that AL therefore performed dynamic analysis on just nine datasets.

Our system, {\sysname}, provides a scalable meta-learning approach to leverage human expertise, using static analysis to mine pipelines from large repositories of scripts. Static analysis has the advantage of scaling to thousands or millions of scripts \cite{graph4code} easily, but lacks the performance data gathered by dynamic analysis. The {\sysname} meta-learning approach guides the learning process by a scalable dataset similarity search, based on dataset embeddings, to find the most similar datasets and the semantics of ML pipelines applied on them.  Many existing systems, such as Auto-Sklearn \cite{autosklearn} and AL \cite{al}, compute a set of meta-features for each dataset. We developed a deep neural network model to generate embeddings at the granularity of a dataset, e.g., a table or CSV file, to capture similarity at the level of an entire dataset rather than relying on a set of meta-features.
 
Because we use static analysis to capture the semantics of the meta-learning process, we have no mechanism to choose the \textbf{best} pipeline from many seen pipelines, unlike the dynamic execution case where one can rely on runtime to choose the best performing pipeline.  Observing that pipelines are basically workflow graphs, we use graph generator neural models to succinctly capture the statically-observed pipelines for a single dataset. In {\sysname}, we formulate learner selection as a graph generation problem to predict optimized pipelines based on pipelines seen in actual notebooks.

%. This formulation enables {\sysname} for effective pruning of the AutoML search space to predict optimized pipelines based on pipelines seen in actual notebooks.}
%We note that increasingly, state-of-the-art performance in AutoML systems is being generated by more complex pipelines such as Directed Acyclic Graphs (DAGs) \cite{piper} rather than the linear pipelines used in earlier systems.  
 
{\sysname} does learner and transformation selection, and hence is a component of an AutoML systems. To evaluate this component, we integrated it into two existing AutoML systems, FLAML \cite{flaml} and Auto-Sklearn \cite{autosklearn}.  
% We evaluate each system with and without {\sysname}.  
We chose FLAML because it does not yet have any meta-learning component for the cold start problem and instead allows user selection of learners and transformers. The authors of FLAML explicitly pointed to the fact that FLAML might benefit from a meta-learning component and pointed to it as a possibility for future work. For FLAML, if mining historical pipelines provides an advantage, we should improve its performance. We also picked Auto-Sklearn as it does have a learner selection component based on meta-features, as described earlier~\cite{autosklearn2}. For Auto-Sklearn, we should at least match performance if our static mining of pipelines can match their extensive database. For context, we also compared {\sysname} with the recent VolcanoML~\cite{VolcanoML}, which provides an efficient decomposition and execution strategy for the AutoML search space. In contrast, {\sysname} prunes the search space using our meta-learning model to perform hyperparameter optimization only for the most promising candidates. 

The contributions of this paper are the following:
\begin{itemize}
    \item Section ~\ref{sec:mining} defines a scalable meta-learning approach based on representation learning of mined ML pipeline semantics and datasets for over 100 datasets and ~11K Python scripts.  
    \newline
    \item Sections~\ref{sec:kgpipGen} formulates AutoML pipeline generation as a graph generation problem. {\sysname} predicts efficiently an optimized ML pipeline for an unseen dataset based on our meta-learning model.  To the best of our knowledge, {\sysname} is the first approach to formulate  AutoML pipeline generation in such a way.
    \newline
    \item Section~\ref{sec:eval} presents a comprehensive evaluation using a large collection of 121 datasets from major AutoML benchmarks and Kaggle. Our experimental results show that {\sysname} outperforms all existing AutoML systems and achieves state-of-the-art results on the majority of these datasets. {\sysname} significantly improves the performance of both FLAML and Auto-Sklearn in classification and regression tasks. We also outperformed AL in 75 out of 77 datasets and VolcanoML in 75  out of 121 datasets, including 44 datasets used only by VolcanoML~\cite{VolcanoML}.  On average, {\sysname} achieves scores that are statistically better than the means of all other systems. 
\end{itemize}


%This approach does not need to apply cleaning or transformation methods to handle different variances among datasets. Moreover, we do not need to deal with complex analysis, such as dynamic code analysis. Thus, our approach proved to be scalable, as discussed in Sections~\ref{sec:mining}.

\section{Numerical implementation and solution}
In this section, we explain how to implement and solve the minimization problem \eqref{eq:dynamic_recon} numerically which, depending on the amount of time steps $T$, can be challenging. 
We derive a general primal-dual algorithm for its solution, before we line out some strategies to reduce computational costs and speed up the implementation at the end of this section. 

\subsection{Gradients and sampling operators}
\label{subsec:implementation of operators}

In order to use the discrete total variation already defined in Section \ref{subsubsec:TV}, we need a discrete gradient operator that maps an image $u \in \C^N$ to its gradient $\nabla u \in \C^{N \times 2}$. 
Following \cite{ChambollePock}, we implement the gradient by standard forward differences. Moreover, we  will use its discrete adjoint, the negative divergence $-\diverg$, defined by the identity $\langle \nabla u, w \rangle_{\C^{N \times 2}} = - \langle u, \diverg(w) \rangle_{\C^N}$.
The inner product on the gradient space $\C^{N\times 2}$ is defined in a straightforward way as 
\begin{align*}
	\langle v,w \rangle_{\C^{N \times 2}} = \Real(v_1^* w_1) + \Real(v_2^* w_2),
\end{align*}
%
for $v,w \in \C^{N \times 2}$.
For $v \in \C^{N \times 2}$, the (isotropic) 1-norm is defined by 
\begin{align*}
	\|v \|_1 := \sum_{i=1}^N \sqrt{|(v_i)_1|^2 + |(v_i)_2|^2},
\end{align*}
and accordingly the dual $\infty$-norm for $w \in \C^{N \times 2}$ is given by 
\begin{align*}
	\| w \|_\infty := \max_{i=1,\cdots,N} |w_i| = \max_{i=1,\cdots,N} ~ \sqrt{|(w_i)_1|^2 + |(w_i)_2|^2}.
\end{align*}

The sampling operators $\Kcal_t \colon \C^N \to \C^{M_t}$ (and analogously for $\Kcal_0 \colon \C^{N_0} \to \C^{M_0}$) we consider are either a standard fast Fourier transform (FFT) on a Cartesian grid, followed by a projection onto the sampled frequencies, or a non-uniform fast Fourier transform (NUFFT), in case the sampled frequencies are not located on a Cartesian grid \cite{Fessler:NUFFT}. \\

\noindent {\it Fourier transform on a Cartesian grid - the simulated data case}\\
\noindent
In the numerical study on artificial data we use a simple version of the Fourier transform and sampling operator (the same can also be found in e.g. \cite{Ehrhardt2016,Rasch2017}). 
We discretize the image domain on the unit square using an (equi-spaced) Cartesian grid with $N_1 \times N_2$ pixels such that the discrete grid points are given by 
\begin{align*}
 \Omega_N = \left\{ \left(\frac{n_1}{N_1-1}, \frac{n_2}{N_2-1} \right) ~\Big|~ n_1 = 0, \dots, N_1-1, \; n_2 = 0, \dots N_2-1 \right\}.
\end{align*}
We proceed analogously with the $k$-space, i.e. the location of the $(m_1,m_2)$-th Fourier coefficient is given by $(m_1/(N_1-1), m_2/(N_2-1))$. Then, we arrive at the following formula for the (standard) Fourier transform $\Fcal$ applied to $u \in \C^{N_1 \times N_2}$:
\begin{align*}
 (\Fcal u)_{m_1,m_2} = \frac{1}{N_1 N_2} \sum_{n_1=0}^{N_1-1} \sum_{n_2=0}^{N_2-1} u_{n_1,n_2} e^{-2\pi i \left(\frac{n_1 m_1}{N_1} + \frac{n_2 m_2}{N_2} \right)},
\end{align*}
where $ m_1 = 0, \dots, N_1-1, m_2 = 0, \dots, N_2 -1$.
For simplicity, we use a vectorized version such that $\Fcal \colon \C^N \to \C^N$ with $N = N_1 \cdot N_2$.
We then employ a simple sampling operator $\Scal_t \colon \C^N \to \C^{M_t}$ which discards all Fourier frequencies which are not located on the desired sampling geometry at time $t$ (i.e. the chosen spokes). 
More precisely, following \cite{Ehrhardt2016}, if we let $\Pcal_t \colon \{1,\dots,M_t \} \to \{1,\dots,N\}$ be an injective mapping which chooses $M_t$ Fourier coefficients from the $N$ coefficients available, we can define the sampling operator $\Scal$ applied to $f \in \C^N$ as
\begin{align*}
 \Scal_t \colon \C^N \to \C^{M_t}, \quad (\Scal_t f)_k = f_{\Pcal_t(k)}.
\end{align*}
The full forward operator $\Kcal_t$ can hence be expressed as 
\begin{align}\label{eq:forward_op_art}
 \Kcal_t \colon \C^N \xrightarrow{\Fcal} \C^N \xrightarrow{\Scal_t} \C^{M_t}.
\end{align}
The corresponding adjoint operator of $\Kcal_t$ is given by  
\begin{align*}
 \Kcal_t^* \colon \C^{M_t} \xrightarrow{\Scal_t^*} \C^N \xrightarrow{\Fcal^{-1}} \C^N,
\end{align*}
where $\Fcal^{-1}$ denotes the standard inverse Fourier transform and $\Scal_t^*$ `fills' the missing frequencies with zeros, i.e. 
\begin{align*}
 (\Scal_t^*z)_l = \sum_{k=1}^{M_t} z_k \delta_{l,\Pcal_t(k)}, \qquad \text{for } l = 1, \dots, N.
\end{align*}
For the prior $u_0$, we choose a full Cartesian sampling, which corresponds to $\Pcal_0$ being the identity. For the subsequent dynamic scan, we set up $\Pcal_t$ such that it chooses the frequencies located on (discrete) spokes through the center of the $k$-space.     
It is important to notice that this implies that the locations of the (discretized) spokes are still located on a Cartesian grid, which allows to employ a standard fast Fourier transform (FFT) followed by the above projection onto the desired frequencies. 
This is not the case for the operators we use for real data. \\

\noindent {\it Non-uniform Fourier transform - the real data case}\\ 
\noindent 
In contrast to the above (simplified) setup for artificial data, in many real world application the measured $k$-space frequencies $\xi_m$ in \eqref{eq:fourier_transform} are {\it not} located on a Cartesian grid. 
While this is not a problem with respect to the formula itself, it however excludes the possibility to employ a fast Fourier transform, 
%numerically
which usually reduces the computational costs of an $N$-point Fourier transform from an order of $O(N^2)$ to $O(N \log N)$.
To get to a similar order of convergence also for non-Cartesian samplings, it is necessary to employ the concept of non-uniform fast Fourier transforms (NUFFT) \cite{Fessler:NUFFT,Fessler:code,Matej2004,Nguyen:1999,Strohmer2000}. 
We only give a quick intuition here and for further information we refer the reader to the literature listed above. 
The main idea is to use a (weighted) and oversampled standard Cartesian $K$-point FFT $\Fcal$, $K \geq N$ followed by an interpolation $\Scal$ in $k$-space onto the desired frequencies $\xi_m$. 
Note that the oversampling takes place in $k$-space.
The operator $\Kcal_t$ for time $t$ can hence again be expressed as a concatenation of a $K$-point FFT and a sampling operator 
\textbf{\begin{align*}
 \Kcal_t \colon \C^N \xrightarrow{\Fcal} \C^N \xrightarrow{\Scal_t} \C^{M_t}.
\end{align*}}
For our numerical experiments with the experimental DCE-MRI data, the sampling operator $\Scal_t$ and its adjoint were taken from the NUFFT package \cite{Fessler:code}.

\subsection{Numerical solution}
%
Due to the nondifferentiablity and the involved operators we apply a primal-dual method \cite{ChambollePock} to solve the minimization problem \eqref{eq:dynamic_recon}. 
We first line out how to solve the (simple) TV-regularized problem for the prior (\ref{tvu0}) and then extend the approach to the dynamic problem. 
Interestingly, the problem for the prior already provides all the ingredients needed for the numerical solution of the dynamic problem, which can then be done in a very straightforward way. 
We consider the problem 
\begin{equation} \label{tvu0}
	\min_{u_0} ~ \frac{\alpha_0}{2} \| \Kcal_0 u_0 - f_0 \|_{\C^{M_0}}^2 + \| \nabla u_0 \|_1, 
\end{equation}
with $u_0 \in \C^{N_0}$.
Dualizing both terms leads to its primal-dual formulation 
\begin{align}\label{eq:tv_pd}
	\min_{u_0} \max_{y_1,y_2} ~ \langle y_1, \Kcal_0 u_0 - f_0 \rangle_{C^{M_0}} - \frac{1}{2 \alpha_0} \|y_1 \|_{\C^{M_0}}^2 + \langle y_2, \nabla u_0 \rangle_{\C^{N_0 \times 2}} + \chi_{C}(y_2),
\end{align}
where $y_1 \in \C^{M_0}$ and $\chi_{C}$ denotes the characteristic function of the set  
\begin{align*}
	C := \{ y \in \C^{N_0 \times 2} ~|~ \|y \|_\infty \leq 1 \}.
\end{align*}
% 
The primal-dual algorithm in \cite{ChambollePock} now essentially consists in performing a proximal gradient descent on the primal variable $u_0$ and a proximal gradient ascent on the dual variables $y_1$ and $y_2$, where the gradients are taken with respect to the linear part, the proximum with respect to the nonlinear part. 
We hence need to compute the proximal operators for the nonlinear parts in \eqref{eq:tv_pd} to obtain the update steps for $u_0$ and $y_1,y_2$. 
It is easy to see that the proximal operator for $\phi(y_1) = \frac{1}{2 \alpha} \|y_1\|_{\C^{M_0}}^2$ is given by 
\begin{align}\label{eq:prox_dual_l2}
	y_1 = \prox_{\sigma \phi} (r) \Leftrightarrow y_1 = \frac{\alpha r}{\alpha + \sigma}.
\end{align}
The proximal operators for the update of $y_2$ are given by a simple projection onto the set $C$, i.e. 
\begin{align}\label{eq:prox_proj}
	y_2 = \proj_C (r) \Leftrightarrow (y_2)_i = r_i / \max (|r_i|,1) \quad \text{for all } i.
\end{align}
Putting everything together leads to Algorithm \ref{alg:prior}.
\begin{algorithm}[t!] 
\caption{\textbf{Reconstruction of the prior}}
{
\begin{algorithmic}[1]
\Require step sizes $\tau,\sigma > 0$, data $f_0$, parameter $\alpha_0$
\Ensure $u_0^0 = \bar{u}_0^0 = \Kcal_0^*f_0, ~ y_1^0 = y_2^0 = 0$
	\While{$\sim$ stop crit}
    	\State {\it Dual updates}
        \State $y_1^{k+1} = (\alpha_0 \left[ y_1^k + \sigma (\Kcal_0 \bar{u}_0^k - f_0)\right]) / (\alpha_0 + \sigma)$
          \State $y_2^{k+1} = \proj_C \left(y_2^k + \sigma \nabla \bar{u}_0^k\right)$
          \State {\it Primal updates}
          \State $u_0^{k+1} =  u_0^k - \tau \left[ \Kcal_0^* y_1^{k+1} - \diverg(y_2^{k+1}) \right]$
          \State {\it Overrelaxation}
          \State $\bar{u}_0^{k+1}= 2 u_0^{k+1} - u_0^k$
	\EndWhile\\
\Return $u_0 = u_0^k$
\end{algorithmic}
}
\label{alg:prior}
\end{algorithm}
\ \\

The numerical realization of the dynamic problem is now straightforward.
In order to deal with the infimal convolution, we use its definition and introduce an additional auxiliary variable yielding 
\begin{alignat*}{4}
	&\min_{\ubold}&& ~ &&\sum_{t=1}^T \frac{\alpha_t}{2} \| \Kcal_t u_t - f_t \|_{\C^{M_t}}^2 + \sum_{t=1}^{T-1} \frac{\gamma_t}{2} \|u_{t+1} - u_t \|_{\C_N}^2 + \sum_{t=1}^T w_t \TV(u_t) \\
	& && + && \sum_{t=1}^T (1-w_t) \ICBTV^{p_0}(u_t,u_0) \\    
   = &\min_{\ubold,\zbold}&& ~ &&\sum_{t=1}^T \frac{\alpha_t}{2} \| \Kcal_t u_t - f_t \|_{\C^{M_t}}^2 + \sum_{t=1}^{T-1} \frac{\gamma_t}{2} \|u_{t+1} - u_t \|_{\C_N}^2 +\sum_{t=1}^T w_t \| \nabla u_t \|_1\\
   & && + &&\sum_{t=1}^T (1-w_t) \left[ \| \nabla (u_t - z_t) \|_1 + \| \nabla z_t \|_1  - \langle p_0,u_t \rangle_{\C^N} + \langle 2 p_0, z_t \rangle_{\C^N} \right]
\end{alignat*}
where $\ubold = [u_1, \dots, u_T] \in \C^{N \times T}$ and $\zbold = [z_1, \dots, z_T] \in \C^{N \times T}$.
Introducing a dual variable $\ybold$ for all the terms containing an operator, leads to the primal-dual formulation 
\begin{alignat}{4}
\label{eq:primal_dual}
	&\min_{\ubold,\zbold} \max_{\ybold} && ~ && \sum_{t=1}^T \left(\langle y_{t,1}, \Kcal_t u_t - f_t \rangle_{\C^{M_t}} - \frac{1}{2 \alpha_t} \|y_{t,1} \|_{\C^M}^2 \right) + \sum_{t=1}^{T-1} \frac{\gamma_t}{2} \| u_{t+1} - u_t \|_{\C^N}^2 \notag \\
    & && + && \sum_{t=1}^T \left(\langle y_{t,2}, \nabla u_t \rangle_{\C^{N \times 2}} + \langle y_{t,3}, \nabla (u_t - z_t) _{\C^{N \times 2}} + \langle y_{t,4}, \nabla z_t \rangle_{\C^{N \times 2}}\right) \notag \\
    & && - && \sum_{t=1}^T \langle (1-w_t)p_0,u_t \rangle_{\C^N} + \sum_{t=1}^T \langle 2(1-w_t)p_0,z_t \rangle_{\C^N}  \notag \\
    & && + && \sum_{t=1}^T \left(\chi_{C_{t,2}}(y_{t,2}) + \chi_{C_{t,3}}(y_{t,3}) + \chi_{C_{t,4}}(y_{t,4})\right)
\end{alignat}
where $\ybold = [\ybold_1, \dots, \ybold_T]$, $\ybold_t = [y_{t,1}, \dots, y_{t,4}]$, and for all $t = 1, \dots, T$, $u_{t} \in \C^{M_t}$ and
\begin{align*}
	&C_{t,2} := \{ y \in \C^{M \times 2} ~|~ \|y \|_{\infty} \leq w_t \}, \\
    &C_{t,3} := \{ y \in \C^{M \times 2} ~|~ \|y \|_{\infty} \leq (1-w_t) \}, \\
    &C_{t,4} := \{ y \in \C^{M \times 2} ~|~ \|y \|_{\infty} \leq (1-w_t) \}. \\
\end{align*}
%
\begin{algorithm}[t!]
\caption{\textbf{Dynamic reconstruction with structural prior}}
{
\begin{algorithmic}[1]
\Require step sizes $\tau,\sigma > 0$, subgradient $p_0$, for all $t=1,\dots,T$: data $f_t$, parameters $\alpha_t$, $w_t$, $\gamma_t$
\Ensure for all $t=1,\dots,T$: $u_t^0 = \bar{u}_t^0 = \Kcal_t^*f_t, ~ z_t^0 = \bar{z}_t^0 = 0, ~ y_{t,1}^0 = y_{t,2}^0 = y_{t,3}^0 = y_{t,4}^0 = 0$
	\While{$\sim$ stop crit}
    	\For{t=1,\dots,T} 
          \State {\it Dual updates}
          \State $y_{t,1}^{k+1} = \frac{\alpha_t \left[ y_{t,1} + \sigma (\Kcal_t \bar{u}_t^k - f_t)\right]}{\alpha_t + \sigma}$
          \State $y_{t,2}^{k+1} = \proj_{C_2}\left(y_{t,2}^k + \sigma \nabla \bar{u}_t^k\right)$
          \State $y_{t,3}^{k+1} = \proj_{C_3}\left(y_{t,3}^k + \sigma \nabla (\bar{u}_t^k - \bar{z}_t^k) \right)$
          \State $y_{t,4}^{k+1} = \proj_{C_4}\left(y_{t,4}^k + \sigma \nabla \bar{z}_t^k \right)$
          \State {\it Primal updates}
          \State $u_t^{k+1} =  \frac{u_t^k - \tau \left[ \Kcal_t^* y_{t,1}^{k+1} - \diverg(y_{t,2}^{k+1}) - \diverg(y_{t,3}^{k+1}) - (1-w_t)p_0 \right] + \tau \gamma_t u_{t+1}^k + \tau \gamma_{t-1} u_{t-1}^k}{\tau (\gamma_t + \gamma_{t+1}) +1}$
          \State $z_t^{k+1} - \tau \left[ 2(1-w_t) p_0 + \diverg(y_{t,3}^{k+1}) - \diverg(y_{t,4}^{k+1}) \right]$
          \State {\it Overrelaxation}
          \State $(\bar{u}_t^{k+1}, \bar{z}_t^{k+1}) = 2 (u_t^{k+1}, z_t^{k+1}) - (u_t^k,z_t^k)$
    	\EndFor
	\EndWhile\\
\Return for all $t = 1,\dots,T$: $u_t = u_t^k$
\end{algorithmic}
}
\label{alg:fmri}
\end{algorithm}
%
To solve the problem, we again perform a proximal gradient descent on the primal variables $\ubold$ and $\zbold$, and a proximal gradient ascent on the dual variables $\ybold$, where the gradients are taken with respect to the linear parts, the proximum with respect to the nonlinear parts. 
We hence need to compute the proximal operators for the nonlinear parts in \eqref{eq:primal_dual} to obtain the update steps for $u_t,z_t$ and $\ybold_t$ for every $t = 1, \dots, T$. 
The proximal operators for $\phi_t(y_{t,1}) = \frac{1}{2 \alpha_t} \| y_{t,1} \|_{\C^{M_t}}^2$ can be computed exactly as in \eqref{eq:prox_dual_l2}.
The proximal operators for the updates of $y_{t,j}$, $j = 2,3,4$, are given by projections onto the sets $C_{t,j}$ similar to \eqref{eq:prox_proj}.
For the squared norm related to the time regularization, we notice that for every $1< t < T$, $u_t$ only interacts with the previous and the following time step, i.e. $u_{t-1}$ and  $u_{t+1}$. 
Hence, analogously to $\phi_t$, the proximum for 
\begin{align*}
	\psi_t(u_t) = \frac{\gamma_{t-1}}{2} \|u_t - u_{t-1}\|_{\C^N}^2 + \frac{\gamma_t}{2} \|u_{t+1} - u_t\|_{\C^N}^2
\end{align*}
is given by 
\begin{align*}
	u_t = \prox_{\tau \psi_t} (r) \Leftrightarrow u_t = \frac{r + \tau \gamma_t u_{t+1} + \tau \gamma_{t-1} u_{t-1}}{\tau (\gamma_t + \gamma_{t-1}) + 1}. 
\end{align*}
The two odd updates for $t = 1$ and $t = T$ can be obtained by the same formula by simply setting $\gamma_0 = 0$ and $\gamma_T = 0$, respectively.
Putting everything together, we obtain Algorithm \ref{alg:fmri}.\\

\subsection{Step sizes and stopping criteria}
We quickly discuss the choice of the step sizes $\tau, \sigma$ and stopping criteria for Algorithm \ref{alg:fmri}. 
In most standard applications it stands to reason to choose the step sizes according to the condition $\tau \sigma \| L \|^2 < 1$ ($L$ denotes the collection of all operators) such that convergence of the algorithm is guaranteed \cite{ChambollePock}.
However, depending on $T$, i.e. the number of time frames we consider, the norm of the operator $L$ 
can be very costly to compute, or too large such that the condition $\tau \sigma \| L \|^2 < 1$ only permits extremely small step sizes. 
For practical use, we instead simply choose $\tau$ and $\sigma$ reasonably ''small`` and track both the energy of the problem and the primal-dual residual \cite{Goldstein:Adaptive} to monitor convergence. 
For the sake of brevity, we do not write down the primal-dual residual for Algorithm \ref{alg:fmri} and instead refer the reader to \cite{Goldstein:Adaptive} for its definition. 
The implementation is then straightforward.
We hence stop the algorithm if both, the relative change in energy between consecutive iterates and the primal-dual residual, have dropped below a certain threshold.

\subsection{Practical considerations}
It is clear that for a large number of time frames $T$ Algorithm \ref{alg:fmri} starts to require an increasing amount of time to return reliable results and for reasonably ''large`` step sizes $\tau$ and $\sigma$ it is even doubtful whether we can obtain convergence. 
In practice, it is hence necessary to divide the time series $\Tbold = \{1, \dots, T\}$ into $l$ smaller bits of consecutive time frames. More precisely, choose numbers $1 \leq T_1 < \dots < T_l = T$ such that $\Tbold = \Tbold_1 \cup \Tbold_2 \cup \dots \cup \Tbold_l$ with $\Tbold = \{1, \dots, T_1, T_1+1, \dots, T_2, \dots, T_{l-1}+1, \dots, T_l\}$.
We can then perform the reconstruction separately for all $\Tbold_i$. 
In order to keep the ''continuity`` between $\Tbold_i$ and $\Tbold_{i+1}$, we can include the last frame of $\Tbold_i$ into the reconstruction of $\Tbold_{i+1}$ by letting $\gamma_{T_i} \neq 0$ and choosing $u_{T_i}$ as the respective last frame of $\Tbold_i$.
This divides the overall problem into smaller and easier subproblems, which can be solved faster.
In practice, we observed that a size of five to ten frames per subset $\Tbold_i$ is a reasonable choice, which essentially gives very similar results as doing a reconstruction for the entire time series $\Tbold$.








\section{Theory}
In this section, we give guarantees on our grid-based approach. Suppose there is some underlying distribution $\mathcal{P}$ with corresponding density function $p : \mathbb{R}^d \rightarrow \mathbb{R}_{\ge 0}$ from which our data points $X_{[n]} = \{x_1,...,x_n\}$ are drawn i.i.d. We show guarantees on the density estimator based on the grid cell counts.

We need the following regularity assumptions on the density function. The first ensures that the density function has compact support with smooth boundaries and is lower bounded by some positive quantity, and the other ensures that the density function has smoothness. These are standard assumptions in analyses on density estimation e.g. \cite{gine2002rates,jiang2017uniform,chen2017tutorial,singh2009adaptive}.
\begin{assumption}\label{assumption1}
$p$ has compact support $\mathcal{X} \in \mathbb{R}^d$ and there exists $\lambda_0, r_0, C_0 > 0$ such that $p(x) \ge \lambda_0$ for all $x \in \mathcal{X}$ and $\text{Vol}(B(x, r) \cap \mathcal{X}) \ge C_0 \cdot \text{Vol}(B(x, r))$ for all $x \in \mathcal{X}$ and $0 < r \le r_0$, where $B(x, r) := \{x' \in \mathbb{R}^d: |x-x'| \le r\}$.
\end{assumption}
\begin{assumption}\label{assumption2}
$p$ is $\alpha$-Hölder continuous for some $0 < \alpha \le 1$: i.e. there exists $C_\alpha > 0$ such that $|p(x) - p(x')| \le C_\alpha \cdot |x - x'|^\alpha$ for all $x, x' \in \mathbb{R}^d$.
\end{assumption}

We now give the result, which says that for $h$ sufficiently small depending on $p$ (if $h$ is too large, then the grid is too coarse to learn a statistically consistent density estimator), and $n$ sufficiently large, there will be a high probability finite-sample uniform bound on the difference between the density estimator and the true density. The proof can be found in the Appendix.
\begin{theorem}\label{theorem}
Suppose Assumption~\ref{assumption1} and~\ref{assumption2} hold. Then there exists constants $C, C_{1} > 0$ depending on $p$ such that the following holds.
Let $0 < \delta < 1$, $0 < h < \text{min}\{\left(\frac{\lambda_0}{2\cdot C_\alpha}\right)^{1/\alpha}, r_0\}$, $nh^d \ge C_1$. Let $\mathcal{G}_h$ be a partitioning of $\mathbb{R}^d$ into grid cells of edge-length $h$ and for $x \in \mathbb{R}^d$. Let $G(x)$ denote the cell in $\mathcal{G}_h$ that $x$ belongs to.  Then, define the corresponding density estimator $\widehat{p}_h$ as:
\begin{align*}
    \widehat{p}_h(x) := \frac{|X_{[n]} \cap G(x)|}{n\cdot h^d}.
\end{align*}
Then, with probability at least $1 - \delta$:
\begin{align*}
    \sup_{x \in \mathbb{R}^d} |\widehat{p}_h(x)  - p(x)| \le C\cdot \left( h^\alpha + \frac{\sqrt{\log(1/(h\delta)}}{\sqrt{n\cdot h^d}} \right).
\end{align*}
\end{theorem}


\begin{remark}
In the above result, choosing $h \approx n^{-1/(2\alpha+d)}$ optimizes the convergence rate to $\tilde{O}(n^{-\alpha/(2\alpha+d)})$, which is the minimax optimal convergence up to logarithmic factors for the density estimation problem as established by Tsybakov \cite{tsybakov1997nonparametric,tsybakov2008introduction}.
\end{remark}
In other words, the grid-based approach statistically performs at least as well as any estimator of the density function, including the density estimator used by MeanShift. It is worth noting that while our results only provide results for the density estimation portion of MeanShift++ (i.e. the grid-cell binning technique), we prove the near-minimax optimality of this estimation. This implies that the information contained in the density estimation portion serves as an approximately sufficient statistic for the rest of the procedure, which behaves similarly to MeanShift, which operates on another, also nearly-optimal density estimator. Thus, existing analyses of MeanShift e.g. \cite{arias2016estimation,chen2015convergence,xiang2005convergence,li2007note,ghassabeh2015sufficient,ghassabeh2013convergence,subbarao2009nonlinear} can be adapted here; however, it is known that MeanShift is very difficult to analyze \cite{dasgupta2014optimal} and a complete analysis is beyond the scope of this paper.


\section{Numerical examples}
\label{sec:examples}
We present three numerical experiments. The first focuses on understanding the convergence of the LA-MCMC algorithm and the impact of various algorithmic parameters controlling the approximation. The second experiment illustrates the tail correction approach described in Section~\ref{sec:tailcorrection}. The third then demonstrates the practical performance of LA-MCMC in a computationally challenging large-scale application: an inverse problem arising in groundwater hydrology.


\subsection{One-dimensional toy example} \label{sec:1d-example}

We first use the one-dimensional density
\begin{equation}
    \log{\pi(x)} \propto -0.5 x^2 + \sin{(4 \pi x)}, \ x \in \mathbb{R}
\end{equation}
to demonstrate how changing various algorithmic parameters affects the performance of LA-MCMC. Figure \ref{fig:1d-example-density} shows binned MCMC samples computed with a random-walk Metropolis algorithm that uses exact target evaluations, compared with binned samples from an LA-MCMC algorithm that uses the same proposal. The sample histograms match very closely. Figure \ref{fig:1d-example-error} shows the error indicator $\Delta(x)^{p+1}$ and error threshold \eqref{eq:error-threshold} computed in a single run of LA-MCMC. Both the error indicator and threshold depend on the current state $X_t$: the indicator depends on the local ball size, and the threshold depends on the Lyapunov function $V(X_t) = \exp{(\|X_t\|)}$. Intuitively, the Lyapunov function relaxes the refinement threshold in the tails of the distribution and thus prevents excessive refinement in low probability regions, where the ball size $\Delta(x)$ tends to be large. 

LA-MCMC algorithmic parameters for the runs described here and below are $\gamma_0=0.1$,  $\bar{\Lambda} = \infty$, $\tau_0=1$, $\eta = 0$, $k=2(p+1)$, and $V(x) = \exp{(\|x\|)}$ (see Algorithm \ref{alg:la-mcmc}). We use local polynomial degree $p=2$ and $\gamma_1 = 1$ unless otherwise indicated.

Let $\sigma^2_t$ and $\breve{\sigma}^2_t$ be running $t$-sample estimates of the target variance computed using sequences of samples $\{X_t\}_{t>0}$ generated by MCMC with exact evaluations (which we refer to as ``exact MCMC'' for shorthand) and LA-MCMC, respectively. As a baseline for comparison, we also compute a `high fidelity' approximation of the variance from 50 realizations of exact MCMC: $\bar{\sigma}^2 = \sum_{i=1}^{50} \sigma_{T,i}^2$, with $T = 10^{6}$. Then we evaluate the error in variance estimates produced by exact MCMC and LA-MCMC,
\begin{equation}
    \begin{array}{ccc}
        e_t = \vert \bar{\sigma}^2 - \sigma_t^2 \vert & \mbox{and} & \breve{e}_t = \vert \bar{\sigma}^2 - \breve{\sigma}_t^2 \vert.
    \end{array}
\end{equation}
We compute the expectations of these errors by averaging $e_t$ and $\breve{e}_t$ over multiple independent realizations of each chain.  

The bias-variance trade-off used to construct our algorithm (see Section \ref{sec:bias-variance-trade-off}) ensures that the error in an expectation computed with LA-MCMC decays at essentially the same rate as the error in an expectation computed with exact MCMC. However, we need to tune the initial error threshold $\gamma_0$ and initial level length $\tau_0$ to ensure that the expected errors are of the same magnitude. In general, we set $\tau_0=1$. The initial error threshold also determines the initial 
% \todo{why ``expected?'' AD: I changed this to the initial local radius. Although, this isn't quite right. If we where to freeze the error threshold and run a long chain this would be an upper bound on the radius at each point $x$.}
local radius $\Delta(x) = \gamma_0^{1/(p+1)} V(x)$. As a heuristic, we choose $\gamma_0$ so that the initial radius is smaller than the radius of a ball containing the non-trivial support of the target density. 

\begin{figure}[h!]
\centering
    \includegraphics[width=0.475\textwidth]{fig_LA-MCMC.pdf}
    \caption{Binned MCMC samples computed with exact evaluations (grey line) and using local approximations (red line).}
    \label{fig:1d-example-density}
\end{figure}

\begin{figure}[h!]
\centering
    \includegraphics[width=0.475\textwidth]{fig_LA-error.pdf}
    \caption{The local error indicator $\Delta(x)^{p+1}$ (blue line) and error threshold $\gamma_\ell(x)$ (Equation \eqref{eq:error-threshold}---red line) used to trigger refinement. The sharp lower bound for the error threshold (bottom border of the red region) is the error threshold with $V(x) = 1$. We, however, allow $V(x) = \exp(\|x\|)$ to be larger in the tails, which relaxes the error threshold in low probability regions. This example run triggered $472$ refinements. Many more (unnecessary) refinements would be required if the Lyapunov function did not relax the allowable error in the tails.}
    \label{fig:1d-example-error}
\end{figure}

The bias-variance trade-off renders LA-MCMC insensitive to the error decay rate $\gamma_1$, {as long as} $\gamma_1 \geq 0.5$. This is borne out in Figure \ref{fig:1d-example-gamma}(a), which shows that the error in the variance estimate decays at the same ($1/\sqrt{t}$) rate  as in the exact evaluation case for all values of $\gamma_1$ except $\gamma_1 = 0.25$. 
%
% \todo{Is what follows the explanation of why we need $\gamma_1 > 0.5$? Maybe this should move into Section 2.2.1. AD: This is why we need $\gamma_1>0.5$. I added a similar discussion in section 2.2.1---maybe we can remove it here now? Although there is some redundancy, I don't really mind having both. thoughts?} 
%
Recall that we impose a piecewise constant error threshold $\gamma_0 \ell^{-\gamma_1}$, fixed for each level $\ell(t)$. If $\gamma_1 < 0.5$, then the level lengths $\tau_l$ decrease as $t \rightarrow \infty$; see Section~\ref{sec:bias-variance-trade-off}. In our practical implementation, we can increase $\ell$ at most once per MCMC step and, therefore, when the level length is less than one step, the error threshold cannot decay quickly enough. In this case, the surrogate bias dominates the error and, as we see in Figure \ref{fig:1d-example-gamma}(a), the error decays more slowly as a function of MCMC steps. 

Figure \ref{fig:1d-example-gamma}(b) shows that, since we do not evaluate the target density every MCMC step, the convergence of LA-MCMC as a function of the number of \emph{density evaluations} $n$ is in general much \emph{faster} than in the exact evaluation case.

\begin{figure}
  \centering
  \begin{tabular}{@{}c@{}}
    \includegraphics[width=0.475\textwidth]{fig_Convergence_MCMCSteps_gamma.pdf} \\[\abovecaptionskip]
    \small (a)
  \end{tabular}

  \vspace{\floatsep}

  \begin{tabular}{@{}c@{}}
    \includegraphics[width=0.475\textwidth]{fig_Convergence_TargetEvals_gamma.pdf} \\[\abovecaptionskip]
    \small (b)
  \end{tabular}

  \caption{The expected errors in variance ($e_t$ and $\breve{e}_t$) averaged over $50$ MCMC chains as a function of (a) MCMC steps $t$ and (b) the number of target density evaluations $n$.  The LA-MCMC construction ensures that, if $\gamma_1  \geq 0.5$, the expected error decays at essentially the same $1/\sqrt{t}$ rate as in the exact evaluation case (Theorem~\ref{thm:convergence-rate}). Since we do not need to evaluate the target density at every MCMC step, and in fact evaluate the target much less frequently over time (see Figure~\ref{fig:1d-example-expected-refinements}), the error decays much more quickly as a function of the number of target density evaluations.}
  \label{fig:1d-example-gamma}
\end{figure}

Approximating the target density with polynomials of higher degree $p$ increases the efficiency of LA-MCMC; we explore this in Figure \ref{fig:1d-example-order}. Figure \ref{fig:1d-example-order}(a) shows that controlling the bias-variance trade-off ensures the error decay rate---as a function of the number of MCMC steps $t$---is the same regardless of $p$. Since the local error indicator is $\Delta(x)^{p+1}$, larger values of $p$ achieve the same error threshold with larger radius $\Delta(x) < 1$. Using higher-degree polynomials therefore requires \emph{fewer} target density evaluations, as shown in Figure \ref{fig:1d-example-order}(b). Yet higher-degree polynomials also require more evaluated points inside the local ball $\mathcal{B}_k(x)$. In this one-dimensional example, the local polynomial requires $p+1$ points to interpolate, and here we choose $k=2(p+1)$ nearest neighbors to solve the regression problem \eqref{eq:local-polynomial-estimate}. We thus see diminishing returns as $p$ increases: higher order polynomials achieve the same accuracy with larger $\Delta(x)$ but require more target density evaluations within each ball $\mathcal{B}_k(x)$. 
% \todo{Say anything about how this tradeoff will evolve with dimension $d$? Interesting to discuss, but maybe skip it now. AD: I had a similar thought but at this point it feels like wild speculation since we have only discussed a 1D example so far. We somewhat address this with the different types of expansions in the tracer sample, but not completely. I think this fits in with a way more general problem of how do we choose/adapt the polynomial basis? So, probably skip for now.}

\begin{figure}
  \centering
  \begin{tabular}{@{}c@{}}
    \includegraphics[width=0.475\textwidth]{fig_Convergence_MCMCSteps_order.pdf} \\[\abovecaptionskip]
    \small (a)
  \end{tabular}

  \vspace{\floatsep}

  \begin{tabular}{@{}c@{}}
    \includegraphics[width=0.475\textwidth]{fig_Convergence_TargetEvals_order.pdf} \\[\abovecaptionskip]
    \small (b)
  \end{tabular}

  \caption{The expected errors $e_t$ and $\breve{e}_t$ averaged over $50$ MCMC chains as a function of (a) MCMC steps $t$, and (b) the number of target density evaluations $n$. Controlling the bias-variance trade-off ensures that error decays at essentially the same $1/\sqrt{t}$ rate as in the exact evaluation case. As we increase the order of the local polynomial approximation, however, LA-MCMC requires fewer target density evaluations $n$ to achieve the same error.}
  \label{fig:1d-example-order}
\end{figure}

As the number of MCMC steps $t \rightarrow \infty$, the rate at which LA-MCMC requires new target density evaluations---i.e., the refinement rate---slows significantly. Figure \ref{fig:1d-example-expected-refinements} illustrates this pattern: the number of target density evaluations increases much more slowly than $t$. While the bias-variance trade-off ensures that the error decay rate remains $1/\sqrt{t}$, viewed in terms of the number of target density evaluations $n$, the picture is different. If target density evaluations are the dominant computational expense---as is typical in many applications---then LA-MCMC generates samples more efficiently as $t \rightarrow \infty$.
% as a function of $n$. 

\begin{figure}[h!]
\centering
    \includegraphics[width=0.475\textwidth]{fig_ExpectedRefinements.pdf}
    \caption{The expected number of refinements $n(t)$, computed from $50$ independent LA-MCMC chains, given different local polynomial degrees $p$.
    % LA-MCMC parameters are $\gamma_0=0.1$, $\gamma_1 = 1$, $\bar{\Lambda}=\infty$, $\tau_0=1$, $\eta = 0$, $k=2(p+1)$, and $V(x) = \exp{(\|x\|)}$. 
    The refinement rate decreases as $t \rightarrow \infty$, making MCMC more efficient as $t$ increases.}
    \label{fig:1d-example-expected-refinements}
\end{figure}


\subsection{Controlling tail behavior} \label{sec:example-banana}

The algorithmic parameter $\eta$ (see \eqref{EqQvDef}) controls how quickly LA-MCMC explores the tails of the target distribution. In the previous example, the tails  of the log-target density decayed quadratically---like a Gaussian---and therefore we set $\eta=0$. Now we consider the ``banana-shaped'' density
\begin{equation}
    \log{\pi(x)} \propto -x_1^2 - (x_2-5x_1^2)^2 
    \label{eq:banana-density}
\end{equation}
for $x \in \mathbb{R}^2$. Figure \ref{fig:banana-density-exact}  illustrates this target distribution; note the long tails in the $x_2$ direction. The two traces in Figure \ref{fig:banana-density-exact-mixing} show the mixing of an adaptive Metropolis \citep{Haarioetal2001} chain that uses exact density evaluations. We see that the chain does explore the tails in $x_2$, but always returns to high probability regions; this is the expected and desired behavior of an MCMC algorithm for this target.

\begin{figure}
\centering
  \includegraphics[width=0.475\textwidth]{fig_true_denisty.pdf} \\[\abovecaptionskip]
  \caption{The two dimensional density defined in \eqref{eq:banana-density}.}
  \label{fig:banana-density-exact}
\end{figure}

\begin{figure}
\centering
  \includegraphics[width=0.475\textwidth]{fig_MH-MIXING.pdf} \\[\abovecaptionskip]
  \caption{Trace plots of an MCMC chain targeting \eqref{eq:banana-density}, using exact evaluations.}
  \label{fig:banana-density-exact-mixing}
\end{figure}

We can control how quickly LA-MCMC explores the tails of the distribution by varying $\eta$. Here, the local polynomial approximation of $\mathcal{L} = \log \pi$ might not (at any finite time) correctly capture the tail behavior; the surrogate model is therefore, in general, not an unnormalized probability density. 
% \todo{AS, can you check that these are reasonable things to say? AD: I think this might be too strong---the chain could return if the tails decay sufficiently fast or if the domain is compact? But, in general, maybe we should set $\eta>0$ to be safe? YMM: I softened it a bit.} 
Using LA-MCMC with no corrections ($\eta=0$) thus allows the chain to wander into the tail without returning to the high-probability region, as shown in Figure \ref{fig:banana-density-la-mixing}(c). Increasing the tail correction parameter $\eta$ biases the acceptance probability so that proposed points close to the centroid are more likely to be accepted and those that are farther are more likely to be rejected. As the number of MCMC steps $t \rightarrow \infty$, this biasing diminishes. The traces in Figure \ref{fig:banana-density-la-mixing}(a) show that setting $\eta > 0$ (here $\eta = 0.01$)  prevents the chain from wandering too far into the tails, and Figure \ref{fig:banana-density-la}(a) shows that the resulting samples correctly characterize the target distribution. 

If $\eta$ is too large, however, then the correction will reduce the efficiency with which the chain explores the distribution's tail. The trace plot in Figure \ref{fig:banana-density-la-mixing}(b) shows that for $\eta=5$, the chain appears to be mixing well. When we compare this chain to Figure~\ref{fig:banana-density-exact-mixing} or Figure~\ref{fig:banana-density-la-mixing}(a), though, we see that the chain is not spending any time in the tail of the distribution. Indeed, the density estimate in Figure \ref{fig:banana-density-la}(b) shows that the tails of the distribution are missing. Asymptotically, the tail correction decays and the algorithm \emph{will} correctly characterize the target distribution for any $\eta > 0$. But too large an $\eta$ can have an impact at finite time. In general, we choose the smallest $\eta$ that prevents the chain from wandering into the distribution's tail.

\begin{figure}
  \centering

  \begin{tabular}{@{}c@{}}
    \includegraphics[width=0.475\textwidth]{fig_LA-MIXING-eta1e-2.pdf} \\[\abovecaptionskip]
    \small (a) $\eta=0.01$
  \end{tabular}
  
  \begin{tabular}{@{}c@{}}
    \includegraphics[width=0.475\textwidth]{fig_LA-MIXING-eta5.pdf} \\[\abovecaptionskip]
    \small (b) $\eta=5$
  \end{tabular}
  
  \begin{tabular}{@{}c@{}}
    \includegraphics[width=0.475\textwidth]{fig_LA-MIXING-eta0.pdf} \\[\abovecaptionskip]
    \small (c) $\eta=0$
  \end{tabular}
  
  \caption{Trace plots of LA-MCMC chains targeting \eqref{eq:banana-density} using different values of the tail correction parameter $\eta$. Other algorithmic parameters are $\gamma_0 = 2$, $\gamma_1 = 1$, $\bar{\Lambda} = \infty$, $\tau_0 = 1$, $k = 15$, $p = 2$, and $V(x) = \exp{(0.25 \|x\|^{0.75})}$.}
  \label{fig:banana-density-la-mixing}
\end{figure}

\begin{figure}
  \centering
  
  \begin{tabular}{@{}c@{}}
    \includegraphics[width=0.475\textwidth]{fig_LA-MCMC-eta1e-2.pdf} \\[\abovecaptionskip]
    \small (a) $\eta=0.01$
  \end{tabular}

  \begin{tabular}{@{}c@{}}
    \includegraphics[width=0.475\textwidth]{fig_LA-MCMC-eta5.pdf} \\[\abovecaptionskip]
    \small (b) $\eta=5$
  \end{tabular}
  
  \caption{Estimates of the target density in \eqref{eq:banana-density} constructed from $10^6$ LA-MCMC samples. We vary $\eta \in \{0.01, 5\}$ and set $\gamma_0 = 2$, $\gamma_1 = 1$, $\bar{\Lambda} = \infty$, $\tau_0 = 1$, $k = 15$, $p = 2$, and $V(x) = \exp{(0.25 \|x\|^{0.75})}$.}
  \label{fig:banana-density-la}
\end{figure}


\subsection{Inferring aquifer transmissivity} \label{sec:tracer}

% Now we demonstrate the usefulness of LA-MCMC in a more computationally demanding example: inferring the spatially heterogeneous transmissivity of an unconfined aquifer. The goal of this example is to demonstrate the performance of LA-MCMC for a model that is \emph{computationally} similar to models used by practitioners. While our model is physically motivated and similar to that used in the study \citet{matott2012screening}, it is simplified in various ways. CONTINUE

Now we demonstrate the usefulness of LA-MCMC in \edits{a more computationally demanding example: inferring the spatially heterogeneous transmissivity of an unconfined aquifer. Although this example is physically motivated, our main goal is to highlight important aspects of the LA-MCMC algorithm and its performance. We choose this example because it is related to examples used in previous work \citep{Conradetal2018} and because similar models are used in the groundwater literature---see, for example, \citet{matott2012screening,janettietal2010,al2018,jardani2012,willmann2007,pooletal2015,govTransmissivity200340}. Though our model is idealized, it is not unreasonably different from many models that are used in practice. Moreover, explaining state-of-the-art groundwater models is well beyond the scope of this paper. We refer to existing work (e.g., \citet{janettietal2010}) for a detailed discussion of inferring transmissivity fields in hydrological applications, and here we focus on the computational demonstration of LA-MCMC.
}
%Now we demonstrate the usefulness of LA-MCMC in a large-scale example: inferring the spatially heterogeneous transmissivity of an unconfined aquifer. 
In this problem, the likelihood contains a set of coupled partial differential equations that model transport of a nonreactive tracer through the aquifer; the tracer concentration is then observed, with noise, at selected locations in the domain. Each likelihood evaluation is thus computationally intensive. 
% The transmissivity is a spatially distributed field that is described with $d=10$ parameters. 

More specifically, the tracer is advected and diffused through the unconfined aquifer (a groundwater resource whose top boundary is not capped by an impermeable layer of rock/soil) by a steady \edits{state} velocity field. 
\edits{
We model the aquifer's log-transmissivity, which determines the permeability of the soil, as a random field on the unit square $\mathcal{D} = [0,1]^2$, parameterized as
\begin{equation}
    \log{ \kappa(z) } = \sum_{i=1}^{d} \kappa_i \sqrt{\lambda_i} e_i(z),
    \label{eq:log-transmissivity}
\end{equation}
where $\kappa_i$ are scalar coefficients, $z \in \mathcal{D}$, and $\{ (\lambda_i, e_i(z))\}_{i=1}^d$ are the $d=9$ leading eigenvalues/eigenfunctions of the integral operator associated with the squared exponential kernel $k(z, z^{\prime}) = \exp{(-\|z-z^{\prime}\|^2/ 2 L^2 )}$, with $L=0.1$. In other words, we have 
\begin{equation}
     \int_{\mathcal{D}} k(z, z^{\prime}) e_i(z^{\prime}) \, dz^\prime = \lambda_i e_i(z) \, .
\end{equation}
Figure~\ref{fig:tracer-transport-domain} shows the ``true'' log-transmissivity, where the true $\kappa_i$ are drawn from a standard normal distribution.}
%The aquifer is divided into ten irregularly shaped regions with a river running along the right-hand side (Figure \ref{fig:tracer-transport-domain}). 
An ``injection'' well is located at \edits{$(x_w, y_w) = (0.45, 0.6)$}. Given a value of the field $\kappa(z)$, we first compute the steady state hydraulic head $h$ by solving
\begin{subequations}
\begin{equation}
    - \nabla \cdot (\kappa h \nabla h) = f_h
\end{equation}
on the domain $\mathcal{D} \ni z \equiv (x,y)$, with Dirichlet boundary conditions \edits{$h(x,y)=1$ for $x=0$, $x=1$, $y=0$, or $y=1$.} We set the source term above to:
\edits{
\begin{equation*}
    f_h(x,y) = 250 \exp{(-((x-x_w)^2+(y-y_w)^2)/0.005)},
\end{equation*}
}
\label{eq:hydraulic-head}
\end{subequations}
which models hydraulic forcing due to well pumping. The velocity is then
\begin{equation}
    u = -\kappa h \nabla h.
    \label{eq:tracer-velocity}
\end{equation}
The hydraulic head and velocity resulting from the ``true'' parameter values (Figure~\ref{fig:tracer-transport-domain}) are illustrated in Figure~\ref{fig:tracer-transport-hydraulic-head}. Given the steady-state velocity field $u$ and an initial tracer concentration $c(x, y, 0) = 0$, advection and diffusion of the tracer throughout the domain is modeled by a time-dependent concentration field $c(x,y,t)$ that obeys the following transport equation:
\begin{equation}
    \frac{\partial c}{\partial t} + \nabla \cdot ((d_m \mathbf{I} + d_{\ell} u u^T) \nabla c) - u^T \nabla c = -f_t
    \label{eq:tracer-concentration}
\end{equation}
with \edits{$d_m=0.05$, $d_{\ell}=0.001$, and} 
\begin{equation}
    f_t(x,y) = \exp{(-((x-x_w)^2+(y-y_w)^2)/0.005)}.
\end{equation}
The tracer forcing $f_t(x,y)$ models tracer leakage into the domain. The concentration field at time $t = 1$ is shown in \moreedits{Figure \ref{fig:tracer-transport-concentration}}. 
%We numerically implement this tracer-transport model using the finite element method, via the software package FEniCS \citep{FENICS}. We evolve the tracer concentration in time using an adaptive time integration scheme implemented in Sundials \citep{sundials}. 

\begin{figure}[h!]
\centering
    \includegraphics[width=0.475\textwidth]{log-transmissivity.pdf}
    \caption{The ``true'' log-transmissivity for an unconfined aquifer model. The coefficients in \eqref{eq:log-transmissivity} are sampled from a standard Gaussian distribution. The diamond denotes a well location.}
    \label{fig:tracer-transport-domain}
\end{figure}

\begin{figure}[h!]
\centering
    \includegraphics[width=0.475\textwidth]{head.pdf}
    \caption{The steady-state hydraulic head and velocity field computed by solving \eqref{eq:hydraulic-head} and \eqref{eq:tracer-velocity} given the ``true'' parameter values are sampled from a standard Gaussian distribution. The white diamond represents the location of a well in the aquifer.}
    \label{fig:tracer-transport-hydraulic-head}
\end{figure}

\begin{figure}[h!]
\centering
    \includegraphics[width=0.475\textwidth]{concentration-64.pdf}
    \caption{Tracer concentration computed at time $t=1$ by solving \eqref{eq:tracer-concentration} given the ``true'' parameter values in Figure~\ref{fig:tracer-transport-domain} and the steady state velocity field shown in Figure \ref{fig:tracer-transport-hydraulic-head}. The white diamond represents the location of a well in the aquifer, and the circles are sensor locations used in the Bayesian inference problem.}
    \label{fig:tracer-transport-concentration}
\end{figure}

Our goal is to infer the log-transmissivity parameters $(\kappa_i)_{i=1}^d$ given observations of the tracer concentration. We endow each $\kappa_i$  with an independent standard Gaussian prior distribution, $\kappa_i \sim N(0, 1)$. 
%
\edits{We could also infer other model parameters (e.g., $d_m$, $d_l$, or the forcing functions $f_h$ and $f_t$) by including them in the set of inferred parameters. For the sake of this example, however, we assume that these parameters are known and therefore fixed.}
%
Data are collected on a $10 \times 10$ array of points evenly spaced in $x \in [0, 1]$ and $y \in [0, 1]$. These locations are marked in Figure~\ref{fig:tracer-transport-concentration}. We make observations at $10$ evenly spaced times $t \in [0, 1]$. 
\edits{This is a relatively large amount of data---typical observation arrays only partially observe the aquifer, although there are often still hundreds of observations (see, e.g., \citet{al2018,janettietal2010}). The large data set here is relatively informative, such that the posterior distribution is quite concentrated relative to the prior and the inference problem is thus more computationally challenging.}
The observations are modeled as
\begin{equation}
    y = f(\kappa) + \varepsilon,
\end{equation}
where $f(\kappa)$ is the \textit{forward model} induced by the partial differential equations above, mapping the log-transmissivity parameters $\kappa \equiv (\kappa_i)_{i=1}^d$ to a prediction of the tracer concentration at the chosen sensor locations/times. The sensor noise is $\varepsilon \sim N(0, 10^{-4} {I})$, yielding the conditional distribution 
%\todo{is the noise variance $10^{-5}$ or $10^{-2}$? Not consistent above and below. AD: It should be $10^{-2}$}
\begin{equation}
y \vert \kappa \sim  N(f(\kappa), 10^{-4} I).
\end{equation}
The posterior density then follows from Bayes' rule:
\begin{eqnarray}
    \pi(\kappa \vert y) &\propto& \pi(y \vert \kappa) \pi(\kappa) \, .
    % \\ &=& N(y; f(\kappa), 10^{-2} \mathbf{I}) N(\kappa; \bar{\kappa}, 0.25 \mathbf{I}).
    \label{eq:tracer-posterior}
\end{eqnarray}
To avoid a so-called ``inverse crime,'' \edits{where data are generated from the same numerical model used to perform inference \citep{kaipiosomersalo2006},} we generate the data $y$ by solving the forward model with a well-refined $65 \times 65$ numerical discretization using the ``true'' parameters, and then perform the inference using a $25 \times 25$ numerical discretization of the forward model.

Evaluating the forward model at every MCMC step is computationally prohibitive. The run time of the model is \edits{$\mathcal{O}(1)$ seconds on an Intel Core i7-7700 CPU at 3.60GHz, with some variability depending on the value of log-transmissivity parameters $\kappa$.} Generating \edits{$10^6$ samples thus takes 5--10 days} 
%\todo{This number should be consistent with what we claim for the exact chain below. Roughly 10 days? Just under 10 days? 6--10 days? 5--10 days? AD: 5--10 is about what it is taking in practice.} 
%\todo{Also, the hardware should be consistent with the i7-7700 mentioned below.} 
of computation time. LA-MCMC is essential to making this Bayesian inference problem computationally feasible. 
%Here, since prior density evaluations are trivial, we apply the approximation capability of LA-MCMC to the log-likelihood function, $\log p(y \vert \kappa)$. 
\edits{We note that high performance and parallel computing resources could certainly reduce the computational cost of each model evaluation. Parallel computing can even enable the shared construction of surrogate models using concurrent chains, as described in our previous work \citep{Conradetal2018}. Since our goal here is to demonstrate the impact of new local approximation strategies, however, we focus on a serial implementation. A useful hardware- and implementation-independent measure of computational cost is the number of refinements (i.e., expensive model/target density evaluations) in a given run, which we discuss below.
} % \todo{YM: What do you think of the last sentence? Wanted to blunt some discussion of less impressive wallclock time/etc. AS: I like it. AD: agreed.}

% The error bounds in Section~\ref{sec:erroranalysis} assume that the local polynomial approximations are chosen from a polynomial space $\mathcal{P}$ over $\mathbb{R}^d$ of \emph{total degree} $p$. We implement this choice here, but we also compare it empirically with other choices of polynomial approximation space. 

Recall that the local polynomial surrogate \eqref{eq:local-polynomial-estimate} requires computing $q = \text{dim}(\mathcal{P})$ coefficients, and in the total-degree setting of our theory, we have $q = {{d+p}\choose{p}}$. This quantity grows rapidly with the dimension $d$ of the parameters for $p>1$.
Retaining a total-degree construction, we could set $p=0$ and thus include only a constant term in $\mathcal{P}$. This results in a constant surrogate model within each ball; now the number of terms in the local polynomial approximation is always one, independent of the parameter dimension. Similarly, setting $p=1$ lets $q$ grow only linearly with $d$. Yet in the one-dimensional example of Section~\ref{sec:1d-example}, we showed that including higher-degree terms in the approximation can reduce the number of expensive target density evaluations required to achieve a given accuracy (see Figure \ref{fig:1d-example-order}).

We implement these total-degree local polynomial approximations but also compare them empirically with other choices of polynomial approximation space---in particular, truncations that retain higher-order terms more selectively. A common practice in high-dimensional  approximation is to employ \emph{sparse truncations} of the relevant index set \citep{BlatmanSudret2011}. Let $\psi_s(\kappa_i)$ be a univariate polynomial of degree $s$ in $\kappa_i$. (Typically, we choose an orthogonal polynomial family, but \moreedits{this choice} is immaterial.) Each $d$-variate polynomial basis function $\phi_{\bm{\alpha}}(\kappa) \coloneqq \prod_{i=1}^{d} \psi_{\alpha_i}(\kappa_i)$ is defined by a multi-index $\bm{\alpha} \in \mathbb{N}_0^d$, where the integers $\alpha_i \geq 0$ are components of $\bm{\alpha}$. Now define the $\ell^\nu$ norm for $\nu > 0$ (a quasi-norm for $0 < \nu < 1$): 
% \todo{potential notation overload: we used $\nu_0$ and $\nu_1$ to define the Lyapunov function earlier. Is this $\nu$ without subscript okay? AD: I did not notice this, but am probably too familiar with this notation to be a good judge. Perhaps we change it to be safe? This $\nu$ is pretty isolated to this section though so maybe we are okay?}
\begin{equation}
    \|\bm{\alpha}\|_{\nu} = \left(\sum_{i=1}^{d} \alpha_i^{\nu}\right)^{1/\nu}.
\end{equation}
Given a maximum degree $p$, we can build our \emph{local} polynomial approximations in the polynomial space $\mathcal{P}_\nu^{p,d}$ spanned by basis functions with multi-indices $\|\bm{\alpha}\|_{\nu} \leq p$. When $\nu=1$, this recovers a total-degree expansion, which is consistent with the error bounds in Section~\ref{sec:erroranalysis}. However, the dimensionality of $\mathcal{P}_\nu^{p,d}$ decreases as $\nu \rightarrow 0$. We define the special case of $\nu=0$ to be a polynomial space with no cross terms: each multi-index has one nonzero element, and that element is at most $p$. 
%We can think of the polynomial approximations with the truncation $\nu < 1$ as being sparse in that they are ``missing'' cross terms.

\begin{figure}[h!]
\centering
    \includegraphics[width=0.475\textwidth]{marginals.pdf} 
    \caption{The diagonal and lower diagonal subplots show the one- and two-dimensional marginal distributions of the posterior distribution \eqref{eq:tracer-posterior} estimated using LA-MCMC with \edits{$\nu=0.75$, $\gamma_0=2$, $\gamma_1=0.5$, $\bar{\Lambda}=\infty$, $\tau_0 = 1$, $\eta = 0$, $k=75$, $p=3$, and $V(x) = \exp{(\|\kappa-\kappa_{\text{MAP}}\|)}$. Here, $\kappa_{\text{MAP}} = \argmax_{\kappa}{\pi(\kappa \vert y)}$ is the posterior mode. The subplot on the upper right shows the two-dimensional prior marginal for any pair of parameters.}}
    \label{fig:tracer-transport-marginals}
\end{figure}

We now apply the LA-MCMC algorithm, \edits{as well as an exact MCMC algorithm for comparison,} to the posterior distribution of the tracer transport problem. \edits{All MCMC chains use the same (fixed) random-walk proposal $q_t = \mathcal{N}(\kappa_t, 0.005 I)$, where $\kappa_t$ is the current state of the chain.} We tested different configurations of LA-MCMC, parameters of which are given in Table~\ref{tab:tracer-mcmc-parameters}. The Lyapunov function $V$ used in the tail correction is defined using the posterior mode, $\bar{\kappa} = \argmax_{\kappa}{\pi(\kappa \vert y)}$, \edits{which we obtain by maximizing the log-posterior density using optimization algorithms in \texttt{nlopt} \citep{johnson2014nlopt}}. The rows in Table~\ref{tab:tracer-mcmc-parameters} with $\nu =1$ correspond to using total-degree linear and quadratic local polynomial approximations (consistent with our theory). Setting $\nu < 1$ defines a sparse approximation. The required number of nearest neighbors $k$ is chosen to be slightly more than the number of points $q = \text{dim}(\mathcal{P}_\nu^{p,d}) $ required to interpolate. 
%\todo{Is there a rationale or formula for fixing $k$, given $q$? Also, should we add a column of $q$ values to Table~\ref{tab:tracer-mcmc-parameters}? Also, why is $k$ bold? I think perhaps $\nu$ and then $p$ should be the first two columns, then $q$ and $k$.}
Reducing $\nu$ reduces $q$ and $k$. Chains produced by all these algorithmic configurations yield essentially identical posterior estimates. A trace plot of selected states from the LA-MCMC chain corresponding to the $p=3$ configuration is shown in  Figure \ref{fig:tracer-mixing}, \edits{and compared to a trace plot of the exact-model chain}. One- and two-dimensional marginals of the posterior distribution are shown in Figure \ref{fig:tracer-transport-marginals}. While this figure is generated with $\nu=0.75$ and $p=3$, the results are essentially identical for the other algorithmic configurations. \edits{An exact chain of $10^6$ steps took about $6$ days to complete, whereas the LA-MCMC chains of the same length with quadratic and cubic surrogate models took roughly $6$ to $10$ hours, and with linear models took just over one day. These timings are somewhat imprecise, however, as other computational tasks were competing for resources on the same workstation (a Intel Core i7-7700 CPU at 3.60GHz) used for our runs.} %\todo{Check this and edit as needed. I removed the table column, and made the timings vague since they are not ``clean.'' Upon re-reading the reviewer comments, I feel like we should at least mention wallclock times.}
% \todo{Same number of MCMC steps in each case? AD: Yes, always $10^6$} 
% \todo{How is 3 days for the exact model consistent with the comment above, which is 1 or 2 seconds per model evaluation, so that $10^6$ seconds = 10 days? AD: There are three things besides the number of model evaluations that can affect this: (i) The overhead from the algorithm (the nearest neighbor search is the most expensive part) (ii) The model evaluation is $\mathcal{O}(1)$ but varies quite a lot depending how the specific transmissivity field and (iii) other things happening on my computer that affect its memory. Bullet (iii) really explains most of this since I'm running multiple MCMC chains at once for this project and writing code for another project at the same time. I really should have put this on a cluster with a proper scheduler but I didn't ... if the reviewers/editors really care I can rerun them properly.}
% \todo{Is the fact that the difference in wall clock times doesn't equal the ratio of number of model evaluations simply due to overhead of the local approximation construction? AD: Partially (see previous comment)}

%Figures \ref{fig:tracer-mixing-linear}, \ref{fig:tracer-mixing-quadratic}, and \ref{fig:tracer-mixing-cubic} show that all of these scenarios explore the posterior distribution.


% IN PROGRESS HERE

% Carefully choosing higher-order terms to include the local polynomial surrogate model can drastically affect the performance of LA-MCMC. We generate five chains using LA-MCMC---one using a linear surrogate model, two with quadratic surrogate models, and two with cubic surrogate models. Table \ref{tab:tracer-mcmc-parameters} shows the parameters supplied to Algorithm \ref{alg:la-mcmc}; the bold columns show how varying the parameter $\eta$, which determines how to truncate the polynomial expansion, affects the required number of nearest neighbors $k$. We choose $k$ to be slightly more than the number of points $q$ required to interpolate. \todo{any particular rationale or formula for fixing $k$, given $q$? Also, should we add a column of $q$ values to the table? }
% For larger $\eta$, the number of terms in the expansion must also increase. Figure \ref{fig:tracer-mixing-linear} shows that LA-MCMC with a locally linear surrogate model successfully explores the posterior distribution---similar mixing plots for LA-MCMC with higher order local surrogate models show these chains exploring the same distribution.
% %Figures \ref{fig:tracer-mixing-linear}, \ref{fig:tracer-mixing-quadratic}, and \ref{fig:tracer-mixing-cubic} show that all of these scenarios explore the posterior distribution.

\begin{table}[h!]
    \centering
    % \begin{tabular}{c|c|c|c|c|c}
    %     degree $p$ & $\nu$ & $k$ & \# to interpolate ($q$) & $\gamma_0$ & Run time (days) \\ \hline \hline
    %     Exact & --- & --- & --- & --- & $2.89$ \\
    %     $1$ & $1$ & $20$ & $10$ & $1.9$ & $1.25$ \\
    %     %$2$ & $0.1$ & $25$ & $19$ & $0.6$ & $2.13$ \\
    %     $2$ & $1$ & $65$ & $55$ & $1$ & 0.33 \\
    %     %$3$ & $0.1$ & $40$ & $28$ & $0.1$ & 2.10 \\
    %     $3$ & $0.75$ & $75$ & $64$ & $1$ & 0.48
    % \end{tabular}
    \begin{tabular}{c|c|c|c|c}
        degree $p$ & $\nu$ & $k$ & \# to interpolate ($q$) & $\gamma_0$ \\ \hline \hline
        Exact & --- & --- & --- & ---\\
        $1$ & $1$ & $20$ & $10$ & $1.9$ \\
        $2$ & $1$ & $65$ & $55$ & $1$ \\
        $3$ & $0.75$ & $75$ & $64$ & $1$ 
    \end{tabular}
    \caption{Parameter configurations for Algorithm \ref{alg:la-mcmc}, used to generate samples from the posterior distribution \eqref{eq:tracer-posterior} of the PDE/tracer transport problem. In all cases $\gamma_1 = 0.5$, $\eta = 0$, $\log{V(\kappa)} = \|\kappa-\kappa_{\text{MAP}}\|$, and $\bar{\Lambda} = \infty$. \edits{Here, $\kappa_{\text{MAP}} = \argmax_{\kappa}{\pi(\kappa \vert y)}$ is the posterior mode, which we obtain by maximizing the log-posterior density using \texttt{nlopt} \citep{johnson2014nlopt}}.}
    \label{tab:tracer-mcmc-parameters}
\end{table}

\begin{figure}
  \centering
  
  \begin{tabular}{@{}c@{}}
    \includegraphics[width=0.475\textwidth]{mixing-exact.pdf} \\[\abovecaptionskip]
    \small (a) Exact MCMC mixing
  \end{tabular}
  
  \begin{tabular}{@{}c@{}}
    \includegraphics[width=0.475\textwidth]{mixing.pdf} \\[\abovecaptionskip]
    \small (a) LA-MCMC mixing
  \end{tabular}
  
  \caption{PDE/tracer transport problem: trace plot of \edits{three parameters of (a) an exact MCMC chain and (b) an LA-MCMC chain using a locally cubic ($p=3$) surrogate model with $\nu = 0.75$.} Additional parameters are defined on the last row of Table \ref{tab:tracer-mcmc-parameters}.}
  \label{fig:tracer-mixing}
\end{figure}

% \begin{figure}
%   \centering
  
%   \begin{tabular}{@{}c@{}}
%     \includegraphics[width=0.475\textwidth]{fig_MixingQuadratic.png} \\[\abovecaptionskip]
%     \small (a) $\nu = 1$
%   \end{tabular}
  
%   \begin{tabular}{@{}c@{}}
%     \includegraphics[width=0.475\textwidth]{fig_MixingQuadraticDiagonal.png} \\[\abovecaptionskip]
%     \small (b) $\nu = 0.5$
%   \end{tabular}
  
%   \caption{Trace plot of the LA-MCMC chain using quadratic surrogate models with parameters defined in Table \ref{tab:tracer-mcmc-parameters}. The vertical line marks the half-way point in the chain---we typically discard the first half as ``burn-in.''}
%   \label{fig:tracer-mixing-quadratic}
% \end{figure}

% \begin{figure}
%   \centering
%   \begin{tabular}{@{}c@{}}
%     \includegraphics[width=0.475\textwidth]{fig_MixingCubicDiagonal.png} \\[\abovecaptionskip]
%     \small (a) $\nu = 0.5$
%   \end{tabular}
  
%   \begin{tabular}{@{}c@{}}
%     \includegraphics[width=0.475\textwidth]{fig_MixingCubicReduced.png} \\[\abovecaptionskip]
%     \small (a) $\nu = 0.75$
%   \end{tabular}
  
%   \caption{Trace plot of the LA-MCMC chain using cubic surrogate models with parameters defined in Table \ref{tab:tracer-mcmc-parameters}. The vertical line marks the half-way point in the chain---we typically discard the first half as ``burn-in.''}
%   \label{fig:tracer-mixing-cubic}
% \end{figure}

\begin{figure}
  \centering
  
  \begin{tabular}{@{}c@{}}
    \includegraphics[width=0.475\textwidth]{refinements.pdf} \\[\abovecaptionskip]
    \small (a) Refinements per MCMC step 
  \end{tabular}
  
  \begin{tabular}{@{}c@{}}
    \includegraphics[width=0.475\textwidth]{refinements-chopped.pdf} \\[\abovecaptionskip]
    \small (b) Refinements per MCMC step (rescaled)
  \end{tabular}
  
  \caption{PDE/tracer transport problem: number of likelihood evaluations (i.e., refinements) as a function of MCMC steps, for LA-MCMC chains with parameters defined in Table \ref{tab:tracer-mcmc-parameters}. Panel (a) shows the total number of refinements for all chains, and panel (b) rescales the vertical axis to emphasize how many fewer model evaluations LA-MCMC requires than the exact case. For local surrogate models with $p \geq 2$, we see significant reductions (of three to four orders of magnitude) over the number of likelihood evaluations required by exact MCMC.}
  \label{fig:tracer-refinements}
\end{figure}

Though all these chains are successful in characterizing the posterior distribution, the required number of (expensive) likelihood evaluations---and hence the overall computational cost---depend on a non-trivial relationship between the accuracy of the local approximations and the number of nearest neighbors required to define them. 
\edits{As noted earlier, though, the number of such likelihood evaluations is a robust and translatable measure of computational cost, independent of implementation and hardware, and also most meaningful in the setting where model evaluations dominate all other costs of the MCMC machinery.} 
Figure~\ref{fig:tracer-refinements} shows the number of likelihood evaluations $n$ as a function of the number of MCMC steps $t$ for each experiment in Table \ref{tab:tracer-mcmc-parameters}. \edits{We see that the quadratic and cubic local approximations (purple and blue lines) require far fewer expensive likelihood evaluations. In all cases, we see a reduction relative to the number of likelihood evaluations that would be required by exact MCMC. However, this improvement is drastically increased, by orders of magnitude, when choosing $p>1$.}
%We see that total-degree quadratic approximations (red line) require more likelihood evaluations for a given $t$ than linear approximations (purple line). Sparse ($\nu = 0.5$) quadratic models require fewer likelihood evaluations at a given $t$, however. And while cubic models with $\nu = 0.75$ require more likelihood evaluations than the other cases, reducing $\nu$ to 0.5 in the cubic case makes the number of  evaluations commensurate with those of the total-degree quadratic approximations. In all cases, we see significant improvements---between three and four orders of magnitude---over the $10^6$ likelihood evaluations that would be required by exact MCMC (at $t =10^6$, i.e., the right boundary of the horizontal axis).

It is important to note that our approach to controlling the bias-variance tradeoff, and thus the error thresholds for triggering refinement via \eqref{eq:local-error-bound} and \eqref{eq:error-threshold}, are derived for expansions of total degree $p$ and thus in principle applicable only for $\nu=1$. We use the same rules here for $\nu<1$, but this approach cannot guarantee that bias and variance decay at the same rate. Indeed, a full analysis of the local sparse approximations may depend on understanding the magnitudes of mixed derivatives of the target function $g$; this can become quite problem-specific, and we defer such an investigation to future work. The empirical results for $\nu<1$ here are intended to be practical and exploratory.

% We see that replacing a linear surrogate model with a quadratic one (purple and red lines) increases the total number of likelihood evaluations because the number of terms in the expansion grows exponentially. However, this can be offset by carefully choosing the terms in the polynomial expansion (blue line). When we replace the surrogate model with a truncated cubic polynomial expansion we see that the number of likelihood evaluations is substantially decreased. Here, we require that the error decay as if the expansion where third-order but do not include all of the terms in this expansion. Strictly speaking, this heuristic no longer guarantees that the squared bias and variance decay at the same rate.

The choice of local approximation not only affects the overall computational cost of each chain, but also its mixing. Figure~\ref{fig:tracer-autocorrelation-ess}(a) shows the number of effectively independent samples (i.e., the effective sample size (ESS) \citep{Wolffetal2004}) produced by each configuration, as a function of the number of MCMC steps $t$. \edits{The chain using exact density evaluations generates effectively independent samples fastest \textit{as a function of $t$}. Mixing is slightly faster with cubic local approximations than with local quadratic or linear approximations. Figures \ref{fig:tracer-autocorrelation-ess}(b)--(c), however, show a non-trivial relationship between complexity of the surrogate model and how quickly it generates ESS \textit{as a function of the number of model evaluations $n$}.} Chains employing local linear approximations ($p=1$) \edits{(red line) do not mix as efficiently as the sparse local cubic approximation or the total-degree quadratic approximation (blue and purple lines). Because the number of model evaluations in the $p=1$ case is not so drastically reduced over the exact case (see Figure~\ref{fig:tracer-refinements}(a)), poorer mixing leads to worse performance in the overall metric of ESS per model evaluation. On the other hand, the $p=3$ case achieves the same ESS with 5000 model evaluations that the exact chain achieves with $5\times 10^5$ evaluations, an improvement of two orders of magnitude.

Recall that Figure~\ref{fig:tracer-transport-marginals} shows strong posterior correlations for this problem. We speculate that log-likelihood approximations that include cross terms (i.e., $p>1$) can more easily approximate the true posterior density and, therefore, significantly outperform local approximations that do not include these factors.} 
%
% \todo{Side note, just for us: this may also related to the choice of whether to approximate the forward model or the log-likelihood. If we did a $p=1$ approximation of the forward model, we would have cross terms in the log-likelihood, as the approximation would be locally quadratic! On the other hand, we'd need to build an approximation for each model output/observation, which is expensive. AD: This is an interesting point. Definitely beyond scope to explore here, but maybe there is a way to use this. True, we would have to build a surrogate for each output but we could use the same nearest neighbors for each component. Since this is the expensive part and it doesn't grow with dimension maybe we would be okay ...}
%
In theory, even a locally constant model will yield convergence as $\Delta \to 0$. \edits{In practice, we find that the choice of basis for the local approximation may have a substantial impact. While $p=2$ or $p=3$ are good default choices, a method for adaptively selecting polynomial basis functions as we learn about correlations in the posterior could possibly improve the algorithm. We leave this to future work.
}


\begin{figure}
  \centering

%   \begin{tabular}{@{}c@{}}
%     \includegraphics[width=0.475\textwidth]{fig_Autocorrelation.pdf} \\[\abovecaptionskip]
%     \small (a) Autocorrelation of $\kappa_0$
%   \end{tabular}
  
  \begin{tabular}{@{}c@{}}
    \includegraphics[width=0.475\textwidth]{ess_samples.pdf} \\[\abovecaptionskip]
    \small (a) ESS per MCMC step 
  \end{tabular}
  
  \begin{tabular}{@{}c@{}}
    \includegraphics[width=0.475\textwidth]{ess_refinements.pdf} \\[\abovecaptionskip]
    \small (b) ESS per model evaluation
  \end{tabular}

    \begin{tabular}{@{}c@{}}
    \includegraphics[width=0.475\textwidth]{ess_refinements-chopped.pdf} \\[\abovecaptionskip]
    \small (c) ESS per model evaluation (rescaled)
  \end{tabular}
  
  \caption{PDE/tracer transport problem, computational efficiency for LA-MCMC chains with parameters defined in Table \ref{tab:tracer-mcmc-parameters}. (a) effective sample size (ESS) as a function of the number of MCMC steps; (b-c) ESS as a function of the number of likelihood evaluations (refinements). Panel (c) rescales the horizontal axis to more clearly show how quickly LA-MCMC with cubic and quadratic surrogate models generates independent samples as a function of the number of refinements.}
  \label{fig:tracer-autocorrelation-ess}
\end{figure}

% In all cases, we see drastic improvements over MCMC with exact evaluations: LA-MCMC requires fewer than $1500$ likelihood evaluations to generate $10^6$ samples and, in some cases requires fewer than $500$. This is an improvement of three to four orders of magnitude. Furthermore, the ratio of generated samples to likelihood evaluations increases as the chain gets longer. Surrogates built using higher-order polynomial expansions are more accurate given a fixed ball containing all of the nearest neighbors. However, such surrogates also require more nearest neighbors. 



% \vspace{-0.5em}
\section{Conclusion}
% \vspace{-0.5em}
Recent advances in multimodal single-cell technology have enabled the simultaneous profiling of the transcriptome alongside other cellular modalities, leading to an increase in the availability of multimodal single-cell data. In this paper, we present \method{}, a multimodal transformer model for single-cell surface protein abundance from gene expression measurements. We combined the data with prior biological interaction knowledge from the STRING database into a richly connected heterogeneous graph and leveraged the transformer architectures to learn an accurate mapping between gene expression and surface protein abundance. Remarkably, \method{} achieves superior and more stable performance than other baselines on both 2021 and 2022 NeurIPS single-cell datasets.

\noindent\textbf{Future Work.}
% Our work is an extension of the model we implemented in the NeurIPS 2022 competition. 
Our framework of multimodal transformers with the cross-modality heterogeneous graph goes far beyond the specific downstream task of modality prediction, and there are lots of potentials to be further explored. Our graph contains three types of nodes. While the cell embeddings are used for predictions, the remaining protein embeddings and gene embeddings may be further interpreted for other tasks. The similarities between proteins may show data-specific protein-protein relationships, while the attention matrix of the gene transformer may help to identify marker genes of each cell type. Additionally, we may achieve gene interaction prediction using the attention mechanism.
% under adequate regulations. 
% We expect \method{} to be capable of much more than just modality prediction. Note that currently, we fuse information from different transformers with message-passing GNNs. 
To extend more on transformers, a potential next step is implementing cross-attention cross-modalities. Ideally, all three types of nodes, namely genes, proteins, and cells, would be jointly modeled using a large transformer that includes specific regulations for each modality. 

% insight of protein and gene embedding (diff task)

% all in one transformer

% \noindent\textbf{Limitations and future work}
% Despite the noticeable performance improvement by utilizing transformers with the cross-modality heterogeneous graph, there are still bottlenecks in the current settings. To begin with, we noticed that the performance variations of all methods are consistently higher in the ``CITE'' dataset compared to the ``GEX2ADT'' dataset. We hypothesized that the increased variability in ``CITE'' was due to both less number of training samples (43k vs. 66k cells) and a significantly more number of testing samples used (28k vs. 1k cells). One straightforward solution to alleviate the high variation issue is to include more training samples, which is not always possible given the training data availability. Nevertheless, publicly available single-cell datasets have been accumulated over the past decades and are still being collected on an ever-increasing scale. Taking advantage of these large-scale atlases is the key to a more stable and well-performing model, as some of the intra-cell variations could be common across different datasets. For example, reference-based methods are commonly used to identify the cell identity of a single cell, or cell-type compositions of a mixture of cells. (other examples for pretrained, e.g., scbert)


%\noindent\textbf{Future work.}
% Our work is an extension of the model we implemented in the NeurIPS 2022 competition. Now our framework of multimodal transformers with the cross-modality heterogeneous graph goes far beyond the specific downstream task of modality prediction, and there are lots of potentials to be further explored. Our graph contains three types of nodes. while the cell embeddings are used for predictions, the remaining protein embeddings and gene embeddings may be further interpreted for other tasks. The similarities between proteins may show data-specific protein-protein relationships, while the attention matrix of the gene transformer may help to identify marker genes of each cell type. Additionally, we may achieve gene interaction prediction using the attention mechanism under adequate regulations. We expect \method{} to be capable of much more than just modality prediction. Note that currently, we fuse information from different transformers with message-passing GNNs. To extend more on transformers, a potential next step is implementing cross-attention cross-modalities. Ideally, all three types of nodes, namely genes, proteins, and cells, would be jointly modeled using a large transformer that includes specific regulations for each modality. The self-attention within each modality would reconstruct the prior interaction network, while the cross-attention between modalities would be supervised by the data observations. Then, The attention matrix will provide insights into all the internal interactions and cross-relationships. With the linearized transformer, this idea would be both practical and versatile.

% \begin{acks}
% This research is supported by the National Science Foundation (NSF) and Johnson \& Johnson.
% \end{acks}

\appendix

\section{Theoretical results} \label{app:theory}

We include all theoretical results from the paper. Throughout this section, we only deal with the case that the underlying proposal distribution $q_{t}$ does not change with $t$. To deal with typical small adaptations, we believe that the following framework can be combined with, e.g., the approach of \citet{roberts2007coupling}, but this would result in a significantly longer paper and these adaptations are not central to our approach.

\subsection{General bounds on non-Markovian approximate MCMC algorithms}

Proceeding more formally, let $\{\hat{X}_t, \hat{K}_t, \mathcal{F}_t\}_{t \geq 0}$ be a triple satisfying:
\begin{enumerate}
    \item $\{\hat{X}_t\}_{t \geq 0}$ is a sequence of \textit{random variables} on $\mathbb{R}^{d}$;
    \item $\{\hat{K}_t\}_{t \geq 0}$ is a (typically random) sequence of \textit{transition kernels} on $\mathbb{R}^{d}$;
    \item $\{\mathcal{F}_t\}_{t \geq 0}$ is a \textit{filtration}, and $\{\hat{X}_t, \hat{K}_t\}_{t \geq 0}$ is \textit{adapted} to this filtration;
    \item the three agree in the sense that 
    \begin{equation} 
        \mathbb{P}[X_{s+1} \in A \vert \mathcal{F}_{s}] = \hat{K}_s(\hat{X}_s, A)
    \end{equation}
    for all $s \geq 0$ and all measurable $A$. Note that, in particular, both left- and right-hand sides are $\mathcal{F}_s$-measurable random variables in $[0, 1]$.
\end{enumerate}
In practice, $\mathcal{F}_{s}$ is generated by our sequence of approximations to the true log-target. We use the following quantitative assumptions:
\begin{assumption}
(Lyapunov inequality). There exists $V: \mathbb{R}^{d} \to [1, \infty)$ and constants $0 < \alpha \leq 1$ and $0 \leq \beta < \infty$ so that 
\begin{equation}
    (\hat{K}_s V)(\hat{X}_{s}) \leq (1-\alpha) V(\hat{X}_{s}) + \beta
\end{equation} 
and
\begin{equation}
    (KV)(x) \leq (1-\alpha) V(x) + \beta
\end{equation} 
for all $s \geq 0$. The second inequality should hold deterministically; note that this is an $\mathcal{F}_{s}$-measurable event.
\label{assumption:lyapunov-inequality}
\end{assumption}
\begin{assumption}
(Good approximation). Let Assumption \ref{assumption:lyapunov-inequality-simple} or \ref{assumption:lyapunov-inequality}  hold. There exists a monotonically decreasing function $\delta:[0, \infty) \to [0, 0.5)$ so that 
\begin{equation}
    \|K(\hat{X}_{s}, \cdot) - \hat{K}_s(\hat{X}_{s}, \cdot)\|_{TV} \leq \delta(s) V(x)
\end{equation}
for all $s \geq 0$ and $x \in \mathbb{R}^{d}$. Again, this inequality should hold deterministically, which is an $\mathcal{F}_{s}$-measurable event. For notational convenience, we define $\delta(s) = \delta(0)$ for all $s < 0$.
\label{assumption:good-approximation}
\end{assumption}

\subsubsection{Initial coupling bounds}

The following is our main technical lemma. It is \textit{not} monotone in the time $s$; this will be remedied in applications.
\begin{lemma}
 Let Assumptions \ref{assumption:geometric-ergodicity}, \ref{assumption:lyapunov-inequality}, and \ref{assumption:good-approximation} hold. There exists a constant $0 < C < \infty$ depending only on $\alpha$, $\beta$, $R$, and $\gamma$ so that for $x \in \mathbb{R}^{d}$ and triple $\{\hat{X}_t, \hat{K}_t, \mathcal{F}_t\}_{t \geq 0}$ started at $\hat{X}_0 = x$, we have 
 \begin{equation*}
\| \mathbb{P}[\hat{X}_s \in \cdot] - \pi(\cdot) \|_{TV} \leq \begin{cases}
1 & \mbox{if } s \leq C_0 \\
 C \delta(0) s & \mbox{if } s > C_0,
\end{cases}
 \end{equation*}
 where $C_0 = C \log{(\delta(0)^{-1}V(x))}$.
 \label{lem:approx-chain-bound}
\end{lemma}

\begin{proof}
Define the ``small set'' 
\begin{equation} \label{EqSmallSet}
    \mathcal{C} = \left\{y : V(y) \leq \frac{4 \beta}{\alpha}\right\}    
\end{equation}
and the associated hitting time 
\begin{equation} \label{EqSmallSetHitting}
    \tau_{\mathcal{C}} = \min{\{t : \hat{X}_t \in \mathcal{C}\}}.
\end{equation}
Denote by $T_b \geq 0$ a ``burn-in'' time whose value will be fixed toward the end of the proof. 

By the triangle inequality, for all measurable $A \subset \mathbb{R}^{d}$
\begin{eqnarray}
\vert \mathbb{P}[\hat{X}_s \in A] - \pi(A) \vert &\leq& \vert \mathbb{P}[\hat{X}_s \in A, \tau_{\mathcal{C}} \leq T_b] - \pi(A) \mathbb{P}[\tau_{\mathcal{C}} \leq T_b] \vert \nonumber  + \vert \mathbb{P}[\hat{X}_s \in A, \tau_{\mathcal{C}} > T_b] - \pi(A) \mathbb{P}[\tau_{\mathcal{C}} > T_b] \vert \nonumber \\
&\leq& \vert \mathbb{P}[\hat{X}_s \in A, \tau_{\mathcal{C}} \leq T_b] - \pi(A) \mathbb{P}[\tau_{\mathcal{C}} \leq T_b] \vert \nonumber  + \mathbb{P}[\tau_{\mathcal{C}} > T_b] \label{eq:chain-inequality}
\end{eqnarray}
To bound the first term, note that
\begin{eqnarray}
    \vert \mathbb{P}[\hat{X}_s \in A, \tau_{\mathcal{C}} \leq T_b] - \pi(A) \mathbb{P}[\tau_{\mathcal{C}} \leq T_b] \vert \nonumber  &\leq& \sup_{y \in \mathcal{C}, 0 \leq u \leq T_b}{\vert \mathbb{P}[\hat{X}_s\in A \vert \hat{X}_u = y, \tau_{\mathcal{C}}=u] - \pi(A) \vert} \nonumber \\
    &\leq& \sup_{y \in \mathcal{C}, 0 \leq u \leq T_b}{\vert \mathbb{P}[\hat{X}_s\in A \vert \hat{X}_u = y, \tau_{\mathcal{C}}=u] - K^{s-u-1}(y, A) \vert} \nonumber \\ && + \sup_{y \in \mathcal{C}, 0 \leq u \leq T_b}{\vert K^{s-u-1}(y, A) - \pi(A) \vert}.
    \label{eq:chain-inequality-first-term}
\end{eqnarray}
Assumption \ref{assumption:geometric-ergodicity} gives 
\begin{equation*}
    \sup_{y \in \mathcal{C}, 0 \leq u \leq T_b}{\left| K^{s-u-1}(y, A) - \pi(A)\right|} \leq R \gamma^{s-T_b-1}    
\end{equation*}
and Assumption \ref{assumption:good-approximation} gives 
% \todo{AARON PLEASE CHECK THIS NOW.}
\begin{equation*}
  \sup_{y \in \mathcal{C}, 0 \leq u \leq T_b}{\vert 
\mathbb{P}[\hat{X}_s\in A \vert \hat{X}_u = y, \tau_{\mathcal{C}}=u] - K^{s-u-1}(y, A) \vert} \leq \sup_{y \in \mathcal{C}, 0 \leq u \leq T_b}{\sum_{t=u}^{s-1} \delta(t) (K^{s-t-1} V)(y)}.
\end{equation*}
%
Furthermore,
\begin{equation*}
    \sup_{y \in \mathcal{C}, 0 \leq u \leq T_b}{\sum_{t=u}^{s-1} \delta(t) (K^{s-t-1} V)(y)} +  R \gamma^{s-T_b-1}  \leq \sup_{y \in \mathcal{C}}{\sum_{t=0}^{s-1} \delta(t) (K^{s-t-1} V)(y)} +  R \gamma^{s-T_b-1}.
\end{equation*}
Substituting this back into \eqref{eq:chain-inequality-first-term} and by Assumption \ref{assumption:lyapunov-inequality}, 
\begin{subequations}
\begin{eqnarray}
    \vert \mathbb{P}[\hat{X}_s \in A, \tau_{\mathcal{C}} \leq T_b] - \pi(A) \mathbb{P}[\tau_{\mathcal{C}} \leq T_b] \vert  &\leq& \sup_{y \in \mathcal{C}}{\sum_{t=0}^{s-1} \delta(t) ((1-\alpha)^{s-t-1} V(y) + \beta/\alpha)} +  R \gamma^{s-T_b-1} \\
    &\leq& \sum_{t=0}^{s-1} \delta(t) ((1-\alpha)^{s-t-1} 4 \beta/\alpha + \beta/\alpha) +  R \gamma^{s-T_b-1} \\
    &\leq& \delta(0) 5 s \beta /\alpha +  R \gamma^{s-T_b-1}.
\end{eqnarray}
\label{eq:chain-inequality-first-term-bound}
\end{subequations}

To bound the second term in \eqref{eq:chain-inequality}, recall from Assumption \ref{assumption:lyapunov-inequality} that 
\begin{eqnarray*}
    \mathbb{E}[V(\hat{X}_{t+1}) \mathbf{1}_{\tau_{\mathcal{C}} > t} \vert \mathcal{F}_t] &\leq& ((1-\alpha)V(\hat{X}_t) + \beta) \mathbf{1}_{V(\hat{X}_t)>4 \beta/\alpha} \\
    &\leq& \left(1-\frac{3\alpha}{4}\right)V(\hat{X}_t) + \left(\beta - \frac{\alpha}{4} V(\hat{X}_t)\right) \mathbf{1}_{V(\hat{X}_t)>4 \beta/\alpha} \\
    &\leq& \left(1-\frac{3\alpha}{4}\right)V(\hat{X}_t) + \left(\beta - \frac{\alpha}{4} \frac{4 \beta}{\alpha}\right) \mathbf{1}_{V(\hat{X}_t)>4 \beta/\alpha} \\
    &\leq& \left(1-\frac{3\alpha}{4}\right)V(\hat{X}_t)
\end{eqnarray*}
for all $t \geq 0$. Iterating, we find by induction on $t$ that 
\begin{equation*}
    \mathbb{E}[V(\hat{X}_{t}) \mathbf{1}_{\tau_{\mathcal{C}} > t}] \leq \left(1-\frac{3\alpha}{4}\right)^{t} V(\hat{X}_0) = \left(1-\frac{3\alpha}{4}\right)^{t} V(x).   
\end{equation*}
Thus, by Markov's inequality,
\begin{equation}
    \mathbb{P}[\tau_{\mathcal{C}}>T_b] = \mathbb{P}\left[V(\hat{X}_{\tau_{\mathcal{C}}}) \mathbf{1}_{\tau_{\mathcal{C}} > T_b}>\frac{4 \beta}{\alpha} \right] \leq \frac{\alpha}{4 \beta}\left(1-\frac{3\alpha}{4}\right)^{T_b} V(x).
    \label{eq:chain-inequality-second-term-bound}
\end{equation}
Combining \eqref{eq:chain-inequality-first-term-bound} and \eqref{eq:chain-inequality-second-term-bound}, we have shown that
\begin{equation}
\vert \mathbb{P}[\hat{X}_s \in A] - \pi(A) \vert \leq \delta(0) \frac{5 s \beta}{\alpha} +  R \gamma^{s-T_b-1} + \frac{\alpha}{4 \beta}\left(1-\frac{3\alpha}{4}\right)^{T_b} V(x).
\end{equation}
Finally, we can choose $T_b$. Set 
\begin{equation*}
    T(s) = \max{\{t : R \gamma^{s-t-1}, \frac{\alpha}{4\beta}\left(1-\frac{3\alpha}{4}\right)^{t} V(x) \leq \frac{1}{2}\delta(0)\}}    
\end{equation*}
and define $T_b = \lfloor T(s) \rfloor$ when $0 < T(s) < \infty$ and $T_b = 0$ otherwise. Define $S=\min{\{s : T(s) \in (0, \infty)\}}$. Noting that $S = \Theta(\log{(\delta(0)^{-1})}+\log{(V(x))})$ for fixed $\alpha$, $\beta$, $R$, and $\gamma$ completes the proof.
\begin{flushright}$\qed$\end{flushright}
\end{proof}

We strengthen Lemma \ref{lem:approx-chain-bound} by first observing that, if $\hat{X}_0$ satisfies $V(\hat{X}_0) \leq 4 \beta / \alpha$, then
\begin{equation}
    \mathbb{E}[V(\hat{X}_1)] \leq (1-\alpha) \mathbb{E}[ V(\hat{X}_0)]+ \beta \leq (1-\alpha) \frac{4 \beta}{\alpha} + \beta \leq \frac{4 \beta}{\alpha}.
\end{equation}
Thus, by induction, if $\mathbb{E}[V(\hat{X}_{0})] \leq 4 \beta / \alpha$, then $\mathbb{E}[V(\hat{X}_s)] \leq 4 \beta / \alpha$ for \textit{all} time $s \geq 0$. Using this (and possibly relabelling the starting time to the quantity denoted by $T_{0}(s)$), Lemma \ref{lem:approx-chain-bound} has the immediate slight strengthening:
\begin{lemma}
  Let Assumptions \ref{assumption:geometric-ergodicity}, \ref{assumption:lyapunov-inequality}, and \ref{assumption:good-approximation} hold. There exists a constant $0 < C < \infty$ depending only on $\alpha$, $\beta$, $R$, and $\gamma$ so that for all starting distributions $\mu$ on $\mathbb{R}^{d}$ with $\mu(V) \leq 4 \beta / \alpha$ and triple $\{\hat{X}_t, \hat{K}_t, \mathcal{F}_t\}_{t \geq 0}$ started at $\hat{X}_0 \sim \mu$, we have 
  \begin{equation*}
\| \mathbb{P}[\hat{X}_s \in \cdot] - \pi(\cdot) \|_{TV} \leq \begin{cases}
1, & s \leq C_0 \\
  C \delta( T_{0}(s) ) \log{(\delta(T_{0}(s))^{-1})}, & s > C_0,
\end{cases}
  \end{equation*}
where $C_0 = C \delta(0) \log{(\delta(0)^{-1})}$ and 
\begin{equation}
T_{0}(s) = s - C \delta(0) \log{(\delta(0)^{-1})}.
\label{EqT0Def}
\end{equation}
  \label{lem:approx-chain-bound-distribution}
\end{lemma}

\subsubsection{Application to bounds on mean-squared error}

Recall that $f : \mathbb{R}^{d} \to [-1,1]$ with $\pi(f) = 0$. We apply Lemma \ref{lem:approx-chain-bound-distribution} to obtain the following bound on the Monte Carlo bias:
\begin{lemma}
  (Bias estimate). Let Assumptions \ref{assumption:geometric-ergodicity}, \ref{assumption:lyapunov-inequality}, and \ref{assumption:good-approximation} hold. There exists a constant $0 < C < \infty$ depending only on $\alpha$, $\beta$, $R$, and $\gamma$ so that for all starting points $x \in \mathbb{R}^{d}$ with $V(x) \leq 4 \beta / \alpha$ and triple $\{\hat{X}_t, \hat{K}_t, \mathcal{F}_t\}_{t \geq 0}$ started at $\hat{X}_0 = x$ we have 
  \begin{equation*}
    \left| \mathbb{E}\left[ \frac{1}{T} \sum_{t=1}^{T} f(\hat{X}_t) \right] \right|  \ \leq \ \begin{cases}
    1, & T \leq C_0 \\
    \frac{C_0}{T} + \frac{C}{T} \displaystyle\sum_{s=C_0}^{T} \delta(T_{0}(s)) \log{(\delta(T_{0}(s))^{-1})} & T > C_0,
    \end{cases}
  \end{equation*}
where $C_0 = C \delta(0) \log{(\delta(0)^{-1})}$.
\label{lem:bias-estimate}
\end{lemma}
\begin{proof}
 In the notation of Lemma \ref{lem:approx-chain-bound-distribution}, we have for $T > C_0$ sufficiently large
\begin{eqnarray*} 
 \left|\mathbb{E}\left[ \frac{1}{T} \sum_{t=1}^{T} f(\hat{X}_t) \right] \right| & \leq &  \frac{1}{T} \sum_{t=1}^{C_0} |\mathbb{E}[f(\hat{X}_{t})]| + \frac{1}{T} \sum_{t=C_0}^{T} |\mathbb{E}[f(\hat{X}_{t})]| \\
&\leq& \frac{C_0}{T} + \frac{C}{T} \sum_{s=C_0}^{T} \delta(T_{0}(s)) \log(\delta(T_{0}(s))^{-1}). 
\end{eqnarray*}
\begin{flushright}$\qed$\end{flushright}
\end{proof}
We have a similar bound for the Monte Carlo variance: 
\begin{lemma}
  (Covariance estimate). Let Assumptions \ref{assumption:geometric-ergodicity}, \ref{assumption:lyapunov-inequality}, and \ref{assumption:good-approximation} hold. There exists a constant $0 < C < \infty$ depending only on $\alpha$, $\beta$, $R$, and $\gamma$ so that for all starting points $x \in \mathbb{R}^{d}$ with $V(x) \leq 4 \beta / \alpha$ and triple $\{\hat{X}_t, \hat{K}_t, \mathcal{F}_t\}_{t \geq 0}$ started at $\hat{X}_0 = x$ we have 
  \begin{equation*}
      \sqrt{\vert\mathbb{E}[f(\hat{X}_s) f(\hat{X}_t)] - \mathbb{E}[f(\hat{X}_s)] \mathbb{E}[f(\hat{X}_t)] \vert} \  \leq \ \begin{cases}
1, & m(s,t) \leq C_0 \\
  C \delta( T_{0} ) \log{(\delta(T_{0})^{-1})}, & m(s,t) > C_0,
\end{cases}
  \end{equation*}
where $m(s,t) = \min(s,t,|t-s|)$, $T_{0} = T_{0}(m(s,t))$ is as in \eqref{EqT0Def}, and $C_0 = C \delta(0) \log{(\delta(0)^{-1})}$ as before.
\label{lem:covariance-estimate}
\end{lemma}
\begin{proof}
 By the triangle inequality 
 \begin{equation*}
    \vert\mathbb{E}[f(\hat{X}_s) f(\hat{X}_t)] - \mathbb{E}[f(\hat{X}_s)] \mathbb{E}[f(\hat{X}_t)] \vert \leq \vert\mathbb{E}[f(\hat{X}_s) f(\hat{X}_t)] \vert + \vert \mathbb{E}[f(\hat{X}_s)] \mathbb{E}[f(\hat{X}_t)] \vert.
 \end{equation*}
 As above, applying Lemma \ref{lem:approx-chain-bound-distribution} completes the proof.
 \begin{flushright}$\qed$\end{flushright}
\end{proof}

The above bias and variance estimates immediate imply our main theorem on the total error of the Monte Carlo estimator: 

\begin{theorem} \label{thm_main_mce}
Let Assumptions \ref{assumption:geometric-ergodicity}, \ref{assumption:lyapunov-inequality}, and \ref{assumption:good-approximation} hold. There exists a constant $0 < C < \infty$ depending only on $\alpha$, $\beta$, $R$, and $\gamma$ so that for all starting points $x \in \mathbb{R}^{d}$ with $V(x) \leq 4 \beta / \alpha$ and triple $\{\hat{X}_t, \hat{K}_t, \mathcal{F}_t\}_{t \geq 0}$ started at $\hat{X}_0 = x$ we have 
\begin{equation*}
\mathbb{E}\left[ \left( \frac{1}{T} \sum_{t=1}^{T} f(\hat{X}_t) \right)^2 \right]  \leq \frac{2C_{0}}{T^{2}}\sum_{s=1}^{T} C(s) + \frac{3}{T} \sum_{s=1}^{T} C(s)^{2},
\end{equation*}
where $C(s) = C \delta(T_{0}(s)) \log(\delta(T_{0}(s))^{-1})$ and we write $C_{0} = C_{0}(0)$.
\label{thm:expected-error}
\end{theorem}
\begin{proof}
We calculate
\begin{eqnarray*}
\mathbb{E}\left[ \left( \frac{1}{T} \sum_{t=1}^{T} f(\hat{X}_t) \right)^2 \right]
&=& T^{-2}\left[\sum_{t=1}^{T} \mathbb{E}[f(\hat{X}_{t})^{2}] \ + \sum_{s,t \, : \, m(s,t) < C_{0}} \mathbb{E}[f(\hat{X}_{s}) f(\hat{X}_{t})]  \ + \sum_{s,t \, : \, m(s,t) \geq C_{0}} \mathbb{E}[f(\hat{X}_{s}) f(\hat{X}_{t})] \right] \\
&\leq& T^{-2} \left[\sum_{s=1}^{T} C_{0}(s) + C_{0} \sum_{s=1}^{T} C_{0}(s) + 3 T \sum_{s=1}^{T} C_{0}(s)^{2} \right] \\
&\leq& \frac{2C_{0}}{T^{2}}\sum_{s=1}^{T} C_{0}(s) + \frac{3}{T} \sum_{s=1}^{T} C_{0}(s)^{2}.
\end{eqnarray*}
\begin{flushright}$\qed$\end{flushright}
\end{proof}

\subsection{Inheriting Lyapunov conditions} \label{SubsecInheritLyap}

Observe that each step of the main ``for'' loop in Algorithm \ref{alg:la-mcmc} determines an entire transition kernel from \textit{any} starting point; denote the kernel in step $t$ by $\mathcal{K}_{t}$. Finally, let $\mathcal{F}_{t}$ be the associated filtration.

\begin{lemma} \label{LemmaLyapSimpleGen}
Let Assumptions \ref{assumption:lyapunov-inequality-simple} and \ref{assumption:good-appr} hold. Then in fact Assumption \ref{assumption:lyapunov-inequality} holds as well. 
\label{lem:lyapunov-correction}
\end{lemma}
\begin{proof}
Under Assumption \ref{assumption:good-appr}, all proposals that \textit{decrease} $V$ are \textit{more} likely to be accepted under $\hat{K}_{t}$ than under $K$, while all proposals that \textit{increase} $V$ are \textit{less} likely to be accepted under $\hat{K}_{t}$ than under $K$. Thus, for all $x$ and all $t$,
\begin{equation}
(\hat{K}_{t}V)(\hat{X}_{t}) \leq (KV)(\hat{X}_{t}) \leq (1 - \alpha) V(\hat{X}_{t}) + \beta,
\end{equation}
which completes the proof.
\begin{flushright}$\qed$\end{flushright}
\end{proof}

\subsection{Final estimates}

We combine the theoretical results in the previous sections to obtain a final estimate on the error of our algorithm. Continuing the notation as above, we have our main theoretical result: Theorem \ref{thm:convergence-rate}, whose proof we give here.

\begin{proof}
We set some notation. Define
\begin{equation}
    E(T) \equiv \mathbb{E}\left[ \left( \frac{1}{T} \sum_{t=1}^{T} f(\hat{X}_t) \right)^2 \right]. 
\end{equation}
Also define the ``burn-in" time $T_{b} = \log(T)^{2}$, and define the hitting time $\tau_{\mathcal{C}}$ as in Equation \eqref{EqSmallSetHitting}.


By Lemma \ref{LemmaLyapSimpleGen}, Assumption \ref{assumption:lyapunov-inequality} in fact holds. Note that our assumptions also immediately give Assumption \ref{assumption:good-approximation} with
\[
\delta(t) \leq 2 \gamma_{0} \sqrt{\frac{\tau_{0}}{t}}.
\]
Thus, applying Theorem \ref{thm_main_mce} in line 3 and then Assumption  \ref{assumption:lyapunov-inequality} and Markov's inequality in line 4, we have (in the notation of that theorem and assumption): 

\begin{align*}
E(T) &\leq \mathbb{E}\left[ \left( \frac{1}{T} \sum_{t=1}^{T_{b}} f(\hat{X}_t) \right)^2 \right] + \mathbb{E}\left[ \left( \frac{1}{T} \sum_{t=T_{b}+1}^{T} f(\hat{X}_t) \right)^2 \right] + 2 \mathbb{E}\left[ \frac{1}{T^{2}} \left( \sum_{t=1}^{T_{b}} f(\hat{X}_t) \right) \left( \sum_{t=T_{b}+1}^{T} f(\hat{X}_t) \right) \right]\\
&\leq \mathbb{E}\left[ \left( \frac{1}{T} \sum_{t=1}^{T_{b}} f(\hat{X}_t) \right)^2 \right] + \frac{T_{b}^{2}}{T^{2}} + \frac{2T_{b}}{T} \\
&\leq \frac{2C_{0}}{T^{2}}\sum_{s=1}^{T} C_{0}(s) + \frac{3}{T} \sum_{s=1}^{T} C_{0}(s)^{2} + \mathbb{P}[\tau_{\mathcal{C}} > T_{b}] + \frac{T_{b}^{2}}{T^{2}} + \frac{2T_{b}}{T} \\
&\leq \frac{2C_{0}}{T^{2}}\sum_{s=1}^{T} C_{0}(s) + \frac{3}{T} \sum_{s=1}^{T} C_{0}(s)^{2} + \frac{\alpha}{4 \beta} (1-\alpha)^{T_{b}} V(x) + \frac{T_{b}^{2}}{T^{2}} + \frac{2T_{b}}{T} \\
&=O\left ( \frac{1}{T^{2}} \sum_{s=1}^{T} \sqrt{\frac{\tau_{0}}{s}} \log\left (\sqrt{\frac{\tau_{0}}{s}} \right )  + \frac{1}{T} \sum_{s=1}^{T}\frac{\tau_{0}}{s} \log\left (\sqrt{\frac{\tau_{0}}{s}}\right )^{2} + \frac{\log(T)^{2}}{T} \right )\\
&= O \left ( \frac{\log(T)^{3}}{T} \right ).
\end{align*}
\begin{flushright}$\qed$\end{flushright}
\end{proof}



% \begin{acknowledgements}
% If you'd like to thank anyone, place your comments here.
% \todo{Fill this in! I think we already acknowledged funding on the title page.}
% \end{acknowledgements}

% BibTeX users please use one of
%\bibliographystyle{spbasic}      % basic style, author-year citations
% \bibliographystyle{spmpsci}      % mathematics and physical sciences
% \bibliographystyle{spphys}       % APS-like style for physics
%\bibliography{workscited}   % name your BibTeX data base

\pdfoutput=1
\documentclass{article}
\usepackage[final]{pdfpages}
\begin{document}
\includepdf[pages=1-9]{CVPR18VOlearner.pdf}
\includepdf[pages=1-last]{supp.pdf}
\end{document}

\end{document}
