\subsection{Controlling tail behavior} \label{sec:example-banana}

The algorithmic parameter $\eta$ (see \eqref{EqQvDef}) controls how quickly LA-MCMC explores the tails of the target distribution. In the previous example, the tails  of the log-target density decayed quadratically---like a Gaussian---and therefore we set $\eta=0$. Now we consider the ``banana-shaped'' density
\begin{equation}
    \log{\pi(x)} \propto -x_1^2 - (x_2-5x_1^2)^2 
    \label{eq:banana-density}
\end{equation}
for $x \in \mathbb{R}^2$. Figure \ref{fig:banana-density-exact}  illustrates this target distribution; note the long tails in the $x_2$ direction. The two traces in Figure \ref{fig:banana-density-exact-mixing} show the mixing of an adaptive Metropolis \citep{Haarioetal2001} chain that uses exact density evaluations. We see that the chain does explore the tails in $x_2$, but always returns to high probability regions; this is the expected and desired behavior of an MCMC algorithm for this target.

\begin{figure}
\centering
  \includegraphics[width=0.475\textwidth]{fig_true_denisty.pdf} \\[\abovecaptionskip]
  \caption{The two dimensional density defined in \eqref{eq:banana-density}.}
  \label{fig:banana-density-exact}
\end{figure}

\begin{figure}
\centering
  \includegraphics[width=0.475\textwidth]{fig_MH-MIXING.pdf} \\[\abovecaptionskip]
  \caption{Trace plots of an MCMC chain targeting \eqref{eq:banana-density}, using exact evaluations.}
  \label{fig:banana-density-exact-mixing}
\end{figure}

We can control how quickly LA-MCMC explores the tails of the distribution by varying $\eta$. Here, the local polynomial approximation of $\mathcal{L} = \log \pi$ might not (at any finite time) correctly capture the tail behavior; the surrogate model is therefore, in general, not an unnormalized probability density. 
% \todo{AS, can you check that these are reasonable things to say? AD: I think this might be too strong---the chain could return if the tails decay sufficiently fast or if the domain is compact? But, in general, maybe we should set $\eta>0$ to be safe? YMM: I softened it a bit.} 
Using LA-MCMC with no corrections ($\eta=0$) thus allows the chain to wander into the tail without returning to the high-probability region, as shown in Figure \ref{fig:banana-density-la-mixing}(c). Increasing the tail correction parameter $\eta$ biases the acceptance probability so that proposed points close to the centroid are more likely to be accepted and those that are farther are more likely to be rejected. As the number of MCMC steps $t \rightarrow \infty$, this biasing diminishes. The traces in Figure \ref{fig:banana-density-la-mixing}(a) show that setting $\eta > 0$ (here $\eta = 0.01$)  prevents the chain from wandering too far into the tails, and Figure \ref{fig:banana-density-la}(a) shows that the resulting samples correctly characterize the target distribution. 

If $\eta$ is too large, however, then the correction will reduce the efficiency with which the chain explores the distribution's tail. The trace plot in Figure \ref{fig:banana-density-la-mixing}(b) shows that for $\eta=5$, the chain appears to be mixing well. When we compare this chain to Figure~\ref{fig:banana-density-exact-mixing} or Figure~\ref{fig:banana-density-la-mixing}(a), though, we see that the chain is not spending any time in the tail of the distribution. Indeed, the density estimate in Figure \ref{fig:banana-density-la}(b) shows that the tails of the distribution are missing. Asymptotically, the tail correction decays and the algorithm \emph{will} correctly characterize the target distribution for any $\eta > 0$. But too large an $\eta$ can have an impact at finite time. In general, we choose the smallest $\eta$ that prevents the chain from wandering into the distribution's tail.

\begin{figure}
  \centering

  \begin{tabular}{@{}c@{}}
    \includegraphics[width=0.475\textwidth]{fig_LA-MIXING-eta1e-2.pdf} \\[\abovecaptionskip]
    \small (a) $\eta=0.01$
  \end{tabular}
  
  \begin{tabular}{@{}c@{}}
    \includegraphics[width=0.475\textwidth]{fig_LA-MIXING-eta5.pdf} \\[\abovecaptionskip]
    \small (b) $\eta=5$
  \end{tabular}
  
  \begin{tabular}{@{}c@{}}
    \includegraphics[width=0.475\textwidth]{fig_LA-MIXING-eta0.pdf} \\[\abovecaptionskip]
    \small (c) $\eta=0$
  \end{tabular}
  
  \caption{Trace plots of LA-MCMC chains targeting \eqref{eq:banana-density} using different values of the tail correction parameter $\eta$. Other algorithmic parameters are $\gamma_0 = 2$, $\gamma_1 = 1$, $\bar{\Lambda} = \infty$, $\tau_0 = 1$, $k = 15$, $p = 2$, and $V(x) = \exp{(0.25 \|x\|^{0.75})}$.}
  \label{fig:banana-density-la-mixing}
\end{figure}

\begin{figure}
  \centering
  
  \begin{tabular}{@{}c@{}}
    \includegraphics[width=0.475\textwidth]{fig_LA-MCMC-eta1e-2.pdf} \\[\abovecaptionskip]
    \small (a) $\eta=0.01$
  \end{tabular}

  \begin{tabular}{@{}c@{}}
    \includegraphics[width=0.475\textwidth]{fig_LA-MCMC-eta5.pdf} \\[\abovecaptionskip]
    \small (b) $\eta=5$
  \end{tabular}
  
  \caption{Estimates of the target density in \eqref{eq:banana-density} constructed from $10^6$ LA-MCMC samples. We vary $\eta \in \{0.01, 5\}$ and set $\gamma_0 = 2$, $\gamma_1 = 1$, $\bar{\Lambda} = \infty$, $\tau_0 = 1$, $k = 15$, $p = 2$, and $V(x) = \exp{(0.25 \|x\|^{0.75})}$.}
  \label{fig:banana-density-la}
\end{figure}
