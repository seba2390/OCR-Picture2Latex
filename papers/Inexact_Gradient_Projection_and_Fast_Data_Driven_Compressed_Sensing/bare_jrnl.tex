
%% bare_jrnl.tex
%% V1.4b
%% 2015/08/26
%% by Michael Shell
%% see http://www.michaelshell.org/
%% for current contact information.
%%
%% This is a skeleton file demonstrating the use of IEEEtran.cls
%% (requires IEEEtran.cls version 1.8b or later) with an IEEE
%% journal paper.
%%
%% Support sites:
%% http://www.michaelshell.org/tex/ieeetran/
%% http://www.ctan.org/pkg/ieeetran
%% and
%% http://www.ieee.org/

%%*************************************************************************
%% Legal Notice:
%% This code is offered as-is without any warranty either expressed or
%% implied; without even the implied warranty of MERCHANTABILITY or
%% FITNESS FOR A PARTICULAR PURPOSE! 
%% User assumes all risk.
%% In no event shall the IEEE or any contributor to this code be liable for
%% any damages or losses, including, but not limited to, incidental,
%% consequential, or any other damages, resulting from the use or misuse
%% of any information contained here.
%%
%% All comments are the opinions of their respective authors and are not
%% necessarily endorsed by the IEEE.
%%
%% This work is distributed under the LaTeX Project Public License (LPPL)
%% ( http://www.latex-project.org/ ) version 1.3, and may be freely used,
%% distributed and modified. A copy of the LPPL, version 1.3, is included
%% in the base LaTeX documentation of all distributions of LaTeX released
%% 2003/12/01 or later.
%% Retain all contribution notices and credits.
%% ** Modified files should be clearly indicated as such, including  **
%% ** renaming them and changing author support contact information. **
%%*************************************************************************


% *** Authors should verify (and, if needed, correct) their LaTeX system  ***
% *** with the testflow diagnostic prior to trusting their LaTeX platform ***
% *** with production work. The IEEE's font choices and paper sizes can   ***
% *** trigger bugs that do not appear when using other class files.       ***                          ***
% The testflow support page is at:
% http://www.michaelshell.org/tex/testflow/



%\documentclass[journal]{IEEEtran}
\documentclass[11pt,draftcls,onecolumn]{IEEEtran}
%
% If IEEEtran.cls has not been installed into the LaTeX system files,
% manually specify the path to it like:
% \documentclass[journal]{../sty/IEEEtran}





% Some very useful LaTeX packages include:
% (uncomment the ones you want to load)


% *** MISC UTILITY PACKAGES ***
%
%\usepackage{ifpdf}
% Heiko Oberdiek's ifpdf.sty is very useful if you need conditional
% compilation based on whether the output is pdf or dvi.
% usage:
% \ifpdf
%   % pdf code
% \else
%   % dvi code
% \fi
% The latest version of ifpdf.sty can be obtained from:
% http://www.ctan.org/pkg/ifpdf
% Also, note that IEEEtran.cls V1.7 and later provides a builtin
% \ifCLASSINFOpdf conditional that works the same way.
% When switching from latex to pdflatex and vice-versa, the compiler may
% have to be run twice to clear warning/error messages.





% *** CITATION PACKAGES ***
%
\usepackage{cite}
% cite.sty was written by Donald Arseneau
% V1.6 and later of IEEEtran pre-defines the format of the cite.sty package
% \cite{} output to follow that of the IEEE. Loading the cite package will
% result in citation numbers being automatically sorted and properly
% "compressed/ranged". e.g., [1], [9], [2], [7], [5], [6] without using
% cite.sty will become [1], [2], [5]--[7], [9] using cite.sty. cite.sty's
% \cite will automatically add leading space, if needed. Use cite.sty's
% noadjust option (cite.sty V3.8 and later) if you want to turn this off
% such as if a citation ever needs to be enclosed in parenthesis.
% cite.sty is already installed on most LaTeX systems. Be sure and use
% version 5.0 (2009-03-20) and later if using hyperref.sty.
% The latest version can be obtained at:
% http://www.ctan.org/pkg/cite
% The documentation is contained in the cite.sty file itself.






% *** GRAPHICS RELATED PACKAGES ***
%
\ifCLASSINFOpdf
  % \usepackage[pdftex]{graphicx}
  % declare the path(s) where your graphic files are
  % \graphicspath{{../pdf/}{../jpeg/}}
  % and their extensions so you won't have to specify these with
  % every instance of \includegraphics
  % \DeclareGraphicsExtensions{.pdf,.jpeg,.png}
\else
  % or other class option (dvipsone, dvipdf, if not using dvips). graphicx
  % will default to the driver specified in the system graphics.cfg if no
  % driver is specified.
  % \usepackage[dvips]{graphicx}
  % declare the path(s) where your graphic files are
  % \graphicspath{{../eps/}}
  % and their extensions so you won't have to specify these with
  % every instance of \includegraphics
  % \DeclareGraphicsExtensions{.eps}
\fi
% graphicx was written by David Carlisle and Sebastian Rahtz. It is
% required if you want graphics, photos, etc. graphicx.sty is already
% installed on most LaTeX systems. The latest version and documentation
% can be obtained at: 
% http://www.ctan.org/pkg/graphicx
% Another good source of documentation is "Using Imported Graphics in
% LaTeX2e" by Keith Reckdahl which can be found at:
% http://www.ctan.org/pkg/epslatex
%
% latex, and pdflatex in dvi mode, support graphics in encapsulated
% postscript (.eps) format. pdflatex in pdf mode supports graphics
% in .pdf, .jpeg, .png and .mps (metapost) formats. Users should ensure
% that all non-photo figures use a vector format (.eps, .pdf, .mps) and
% not a bitmapped formats (.jpeg, .png). The IEEE frowns on bitmapped formats
% which can result in "jaggedy"/blurry rendering of lines and letters as
% well as large increases in file sizes.
%
% You can find documentation about the pdfTeX application at:
% http://www.tug.org/applications/pdftex





% *** MATH PACKAGES ***
%
%\usepackage{amsmath}
% A popular package from the American Mathematical Society that provides
% many useful and powerful commands for dealing with mathematics.
%
% Note that the amsmath package sets \interdisplaylinepenalty to 10000
% thus preventing page breaks from occurring within multiline equations. Use:
%\interdisplaylinepenalty=2500
% after loading amsmath to restore such page breaks as IEEEtran.cls normally
% does. amsmath.sty is already installed on most LaTeX systems. The latest
% version and documentation can be obtained at:
% http://www.ctan.org/pkg/amsmath





% *** SPECIALIZED LIST PACKAGES ***
%
%\usepackage{algorithmic}
% algorithmic.sty was written by Peter Williams and Rogerio Brito.
% This package provides an algorithmic environment fo describing algorithms.
% You can use the algorithmic environment in-text or within a figure
% environment to provide for a floating algorithm. Do NOT use the algorithm
% floating environment provided by algorithm.sty (by the same authors) or
% algorithm2e.sty (by Christophe Fiorio) as the IEEE does not use dedicated
% algorithm float types and packages that provide these will not provide
% correct IEEE style captions. The latest version and documentation of
% algorithmic.sty can be obtained at:
% http://www.ctan.org/pkg/algorithms
% Also of interest may be the (relatively newer and more customizable)
% algorithmicx.sty package by Szasz Janos:
% http://www.ctan.org/pkg/algorithmicx




% *** ALIGNMENT PACKAGES ***
%
%\usepackage{array}
% Frank Mittelbach's and David Carlisle's array.sty patches and improves
% the standard LaTeX2e array and tabular environments to provide better
% appearance and additional user controls. As the default LaTeX2e table
% generation code is lacking to the point of almost being broken with
% respect to the quality of the end results, all users are strongly
% advised to use an enhanced (at the very least that provided by array.sty)
% set of table tools. array.sty is already installed on most systems. The
% latest version and documentation can be obtained at:
% http://www.ctan.org/pkg/array


% IEEEtran contains the IEEEeqnarray family of commands that can be used to
% generate multiline equations as well as matrices, tables, etc., of high
% quality.




% *** SUBFIGURE PACKAGES ***
%\ifCLASSOPTIONcompsoc
%  \usepackage[caption=false,font=normalsize,labelfont=sf,textfont=sf]{subfig}
%\else
%  \usepackage[caption=false,font=footnotesize]{subfig}
%\fi
% subfig.sty, written by Steven Douglas Cochran, is the modern replacement
% for subfigure.sty, the latter of which is no longer maintained and is
% incompatible with some LaTeX packages including fixltx2e. However,
% subfig.sty requires and automatically loads Axel Sommerfeldt's caption.sty
% which will override IEEEtran.cls' handling of captions and this will result
% in non-IEEE style figure/table captions. To prevent this problem, be sure
% and invoke subfig.sty's "caption=false" package option (available since
% subfig.sty version 1.3, 2005/06/28) as this is will preserve IEEEtran.cls
% handling of captions.
% Note that the Computer Society format requires a larger sans serif font
% than the serif footnote size font used in traditional IEEE formatting
% and thus the need to invoke different subfig.sty package options depending
% on whether compsoc mode has been enabled.
%
% The latest version and documentation of subfig.sty can be obtained at:
% http://www.ctan.org/pkg/subfig




% *** FLOAT PACKAGES ***
%
%\usepackage{fixltx2e}
% fixltx2e, the successor to the earlier fix2col.sty, was written by
% Frank Mittelbach and David Carlisle. This package corrects a few problems
% in the LaTeX2e kernel, the most notable of which is that in current
% LaTeX2e releases, the ordering of single and double column floats is not
% guaranteed to be preserved. Thus, an unpatched LaTeX2e can allow a
% single column figure to be placed prior to an earlier double column
% figure.
% Be aware that LaTeX2e kernels dated 2015 and later have fixltx2e.sty's
% corrections already built into the system in which case a warning will
% be issued if an attempt is made to load fixltx2e.sty as it is no longer
% needed.
% The latest version and documentation can be found at:
% http://www.ctan.org/pkg/fixltx2e


%\usepackage{stfloats}
% stfloats.sty was written by Sigitas Tolusis. This package gives LaTeX2e
% the ability to do double column floats at the bottom of the page as well
% as the top. (e.g., "\begin{figure*}[!b]" is not normally possible in
% LaTeX2e). It also provides a command:
%\fnbelowfloat
% to enable the placement of footnotes below bottom floats (the standard
% LaTeX2e kernel puts them above bottom floats). This is an invasive package
% which rewrites many portions of the LaTeX2e float routines. It may not work
% with other packages that modify the LaTeX2e float routines. The latest
% version and documentation can be obtained at:
% http://www.ctan.org/pkg/stfloats
% Do not use the stfloats baselinefloat ability as the IEEE does not allow
% \baselineskip to stretch. Authors submitting work to the IEEE should note
% that the IEEE rarely uses double column equations and that authors should try
% to avoid such use. Do not be tempted to use the cuted.sty or midfloat.sty
% packages (also by Sigitas Tolusis) as the IEEE does not format its papers in
% such ways.
% Do not attempt to use stfloats with fixltx2e as they are incompatible.
% Instead, use Morten Hogholm'a dblfloatfix which combines the features
% of both fixltx2e and stfloats:
%
% \usepackage{dblfloatfix}
% The latest version can be found at:
% http://www.ctan.org/pkg/dblfloatfix




%\ifCLASSOPTIONcaptionsoff
%  \usepackage[nomarkers]{endfloat}
% \let\MYoriglatexcaption\caption
% \renewcommand{\caption}[2][\relax]{\MYoriglatexcaption[#2]{#2}}
%\fi
% endfloat.sty was written by James Darrell McCauley, Jeff Goldberg and 
% Axel Sommerfeldt. This package may be useful when used in conjunction with 
% IEEEtran.cls'  captionsoff option. Some IEEE journals/societies require that
% submissions have lists of figures/tables at the end of the paper and that
% figures/tables without any captions are placed on a page by themselves at
% the end of the document. If needed, the draftcls IEEEtran class option or
% \CLASSINPUTbaselinestretch interface can be used to increase the line
% spacing as well. Be sure and use the nomarkers option of endfloat to
% prevent endfloat from "marking" where the figures would have been placed
% in the text. The two hack lines of code above are a slight modification of
% that suggested by in the endfloat docs (section 8.4.1) to ensure that
% the full captions always appear in the list of figures/tables - even if
% the user used the short optional argument of \caption[]{}.
% IEEE papers do not typically make use of \caption[]'s optional argument,
% so this should not be an issue. A similar trick can be used to disable
% captions of packages such as subfig.sty that lack options to turn off
% the subcaptions:
% For subfig.sty:
% \let\MYorigsubfloat\subfloat
% \renewcommand{\subfloat}[2][\relax]{\MYorigsubfloat[]{#2}}
% However, the above trick will not work if both optional arguments of
% the \subfloat command are used. Furthermore, there needs to be a
% description of each subfigure *somewhere* and endfloat does not add
% subfigure captions to its list of figures. Thus, the best approach is to
% avoid the use of subfigure captions (many IEEE journals avoid them anyway)
% and instead reference/explain all the subfigures within the main caption.
% The latest version of endfloat.sty and its documentation can obtained at:
% http://www.ctan.org/pkg/endfloat
%
% The IEEEtran \ifCLASSOPTIONcaptionsoff conditional can also be used
% later in the document, say, to conditionally put the References on a 
% page by themselves.




% *** PDF, URL AND HYPERLINK PACKAGES ***
%
%\usepackage{url}
% url.sty was written by Donald Arseneau. It provides better support for
% handling and breaking URLs. url.sty is already installed on most LaTeX
% systems. The latest version and documentation can be obtained at:
% http://www.ctan.org/pkg/url
% Basically, \url{my_url_here}.




% *** Do not adjust lengths that control margins, column widths, etc. ***
% *** Do not use packages that alter fonts (such as pslatex).         ***
% There should be no need to do such things with IEEEtran.cls V1.6 and later.
% (Unless specifically asked to do so by the journal or conference you plan
% to submit to, of course. )

\usepackage{amsmath,graphicx,mystyle,epstopdf,hyperref,url,cite,booktabs,amsfonts}
\usepackage{cite}
\usepackage[ruled]{algorithm2e}%,subfigure}
%\usepackage{fancyref}
\usepackage{mathtools}
\DeclarePairedDelimiter{\ceil}{\lceil}{\rceil}
% Example definitions.
% --------------------
\def\x{{\mathbf x}}
\def\L{{\cal L}}
\def\m{{\mathbf m}}
\def\Mb{{\mathbf M}}
\def\dist{{\text{dist}}}

\newcommand{\pp}{\mathcal{P}}
\newcommand{\MM}{\beta}
\newcommand{\mm}{\alpha}
\newcommand{\rsc}{+}
\newcommand{\gt}{*}
\newcommand{\mmx}{\mm_{x^\gt}}
\newcommand{\ep}{e_p}
\newcommand{\eg}{e_g}
\newcommand{\nup}{\nu_p}
\newcommand{\nug}{\nu_g}
\newcommand{\nut}{\mathbf{e}}
\newcommand{\ppp}{\widetilde{\pp}}
\newcommand{\nablaa}{\widetilde{\nabla}}
\newcommand{\g}{\widetilde g}
\newcommand{\n}{\widetilde n}

\newcommand{\vertiii}[1]{{\left\vert\kern-0.25ex\left\vert\kern-0.25ex\left\vert #1 
		\right\vert\kern-0.25ex\right\vert\kern-0.25ex\right\vert}}
\newcommand{\red}[1]{{\textcolor{red}{#1}}}
% correct bad hyphenation here
%\hyphenation{op-tical net-works semi-conduc-tor}



%==================================
\def\mylinewidth{8cm}% figure width
\def\mytextwidth{8cm}
\newlength \myfigwidth
\newlength \myfigwidthll
\newlength \myfigwidthllI

\newlength \descriptionwidth


\if@twocolumn
\setlength \myfigwidth {0.9\columnwidth}
\else
\setlength \myfigwidth {0.5\textwidth}
\fi
\setlength \myfigwidthll {0.75 \myfigwidth}
\setlength \myfigwidthllI {0.15 \textwidth}

\if@twocolumn
\setlength \descriptionwidth {12.5cm}
\else
\setlength \descriptionwidth {0.8 \textwidth}
\fi

\if@twocolumn
\else
\renewcommand{\arraystretch}{0.60}\addtolength{\tabcolsep}{+0pt} %reduce the space in the tab
\fi
%==================================

\begin{document}

\title{Inexact Gradient Projection 
	and Fast Data Driven Compressed Sensing}
%
%
% author names and IEEE memberships
% note positions of commas and nonbreaking spaces ( ~ ) LaTeX will not break
% a structure at a ~ so this keeps an author's name from being broken across
% two lines.
% use \thanks{} to gain access to the first footnote area
% a separate \thanks must be used for each paragraph as LaTeX2e's \thanks
% was not built to handle multiple paragraphs
%

\author{Mohammad~Golbabaee,~Mike~E.~Davies,~\IEEEmembership{Fellow,~IEEE,}
        

\thanks{MG and MED are with the Institute for Digital Communications (IDCOM), School of Engineering, University of Edinburgh, EH9 3JL, United Kingdom. E-mail: \{m.golbabaee, mike.davies\}@ed.ac.uk. This work is partly funded by the EPSRC grant EP/M019802/1 and the ERC C-SENSE project (ERC-ADG-2015-694888). MED is also supported by the  Royal Society Wolfson Research Merit Award.
	}% <-this % stops a space
%\thanks{This work is partly funded by EPSRC grant EP/M019802/1 and ERC C-SENSE project (ERC-ADG-2015-694888). MED is also supported by a Royal Society Wolfson Research Merit Award.}
%\thanks{Manuscript received April 19, 2005; revised August 26, 2015.}
}

% note the % following the last \IEEEmembership and also \thanks - 
% these prevent an unwanted space from occurring between the last author name
% and the end of the author line. i.e., if you had this:
% 
% \author{....lastname \thanks{...} \thanks{...} }
%                     ^------------^------------^----Do not want these spaces!
%
% a space would be appended to the last name and could cause every name on that
% line to be shifted left slightly. This is one of those "LaTeX things". For
% instance, "\textbf{A} \textbf{B}" will typeset as "A B" not "AB". To get
% "AB" then you have to do: "\textbf{A}\textbf{B}"
% \thanks is no different in this regard, so shield the last } of each \thanks
% that ends a line with a % and do not let a space in before the next \thanks.
% Spaces after \IEEEmembership other than the last one are OK (and needed) as
% you are supposed to have spaces between the names. For what it is worth,
% this is a minor point as most people would not even notice if the said evil
% space somehow managed to creep in.



% The paper headers
%\markboth{Journal of \LaTeX\ Class Files,~Vol.~14, No.~8, August~2015}%
%{Golbabaee \MakeLowercase{\textit{et al.}}: Bare Demo of IEEEtran.cls for IEEE Journals}
% The only time the second header will appear is for the odd numbered pages
% after the title page when using the twoside option.
% 
% *** Note that you probably will NOT want to include the author's ***
% *** name in the headers of peer review papers.                   ***
% You can use \ifCLASSOPTIONpeerreview for conditional compilation here if
% you desire.



% If you want to put a publisher's ID mark on the page you can do it like
% this:
%\IEEEpubid{0000--0000/00\$00.00~\copyright~2015 IEEE}
% Remember, if you use this you must call \IEEEpubidadjcol in the second
% column for its text to clear the IEEEpubid mark.

% use for special paper notices
%\IEEEspecialpapernotice{(Invited Paper)}


\maketitle
\begin{abstract}
%We consider solving optimizations with differentiable objective function constrained to an arbitrary (possibly non-convex) set.
%We show that the simple iterative projected gradient (IPG) algorithm can achieve a near-global-optimum solution. 
We study convergence of the iterative projected gradient (IPG) algorithm for arbitrary (possibly non-convex) sets and when both the gradient and projection oracles are computed approximately. We consider different notions of  approximation of which we show that 
%three classes of approximate oracles: fixed-precision,
the Progressive Fixed Precision (PFP) and the $(1+\epsilon)$-optimal oracles can achieve the same  accuracy as for the exact IPG algorithm. We show that the former scheme is also able to maintain the (linear) rate of convergence of the exact algorithm, under the same embedding assumption. In contrast, the $(1+\epsilon)$-approximate oracle requires a stronger embedding condition, moderate compression ratios and it typically slows down the convergence. 

We apply our results to accelerate solving a class of data driven compressed sensing problems, where we replace iterative 
%Finally we study an application of the inexact IPG in a data driven compressed sensing problem. We shortcut the computations involved with (iteratively) 
 exhaustive searches over large datasets by fast approximate nearest neighbour search strategies based on the cover tree data structure. For datasets with low intrinsic dimensions our proposed algorithm achieves a complexity logarithmic  in terms of the dataset population as opposed to the linear complexity of a brute force search. 
By running several numerical experiments we conclude similar observations as predicted by our theoretical analysis.


% approximation errors Under certain assumptions on 
\end{abstract}

% Note that keywords are not normally used for peerreview papers.
%\begin{IEEEkeywords}
%Iterative projected gradient, signal model, approximate oracles, linear convergence, data driven, compressed sensing, cover tree, approximate nearest neighbour search.
%\end{IEEEkeywords}






% For peer review papers, you can put extra information on the cover
% page as needed:
% \ifCLASSOPTIONpeerreview
% \begin{center} \bfseries EDICS Category: 3-BBND \end{center}
% \fi
%
% For peerreview papers, this IEEEtran command inserts a page break and
% creates the second title. It will be ignored for other modes.
\IEEEpeerreviewmaketitle

\section{Introduction}  
Signal inference under limited and noisy observations is a major line of research in signal processing, machine learning and statistics and it has a wide application ranging from biomedical imaging, astrophysics, remote sensing  to data mining. %\todo{maybe better apps to list?}.
Incorporating the structure of signals is proven to significantly help with an accurate inference since natural datasets often have limited degrees of freedom as compared to their original ambient dimensionality. This fact has been invoked in Compressed sensing (CS)  literature by adopting efficient signal models to achieve accurate signal reconstruction given near-minimal number of measurements i.e. much smaller than the signal ambient dimension (see \cite{DonohoCS, CRT:CS,Tropp:SOMP1,Candes:EMC,BW:manifold,modelbasedCS} and e.g. \cite{RichCSreview} for an overview on different CS models). 
%\todo{This for instance in sensory scenarios such as in MRI, video, high resHSI saves a lot of energy cost time...} 
%Compressed sensing (CS) algorithms adopt efficient signal models to achieve accurate reconstruction given small number of measurements. 
CS consists of a linear sampling protocol:
\eql{
	\label{eq:CSsampling}
	y \approx Ax^\gt,
}
where a linear mapping $A$ samples a $m$-dimensional vector $y$ of noisy measurements from a ground truth signal $x^\gt$ which typically lives in a high ambient dimension $n\gg m$. Natural signals often have efficient compact representations using nonlinear models such as low dimensional smooth manifolds, low-rank matrices or the Union of Subspaces (UoS) that itself includes sparse (unstructured) or structured sparse (e.g. group, tree or analysis sparsity) representations in properly chosen orthobases or redundant dictionaries~\cite{RichCSreview}.
%An efficient CS scheme incorporates such non-linearity into the reconstruction phase to achieve robustness and better compression rates. 
CS reconstruction algorithms for estimating $x^\gt$ from $y$ are in general more computationally complex (as opposed to the simple linear acquisition of CS) as they typically require solving a nonlinear optimization problem based around a prior signal model.  % signal models.  
 A proper model should be carefully chosen in order to efficiently promote the low-dimensional structure of signal meanwhile not bringing a huge computational burden to the reconstruction algorithm.
 
%\subsection{Iterative projected gradient}
Consider the following constrained least square problem for CS reconstruction:
\eql{\label{eq:p1}
	\min_{x\in \Cc} \{f(x):= \frac{1}{2}\norm{y-Ax}^2\},
}
where, the constraint set $\Cc\in \RR^n$ represents the signal model. First order algorithms in the form of projected Landweber iteration a.k.a. iterative projected gradient (IPG) descent or Forward-Backward are very popular for solving \eqref{eq:p1}. 
% in both convex and nonconvex settings. 
%In this case IPG iterates between calculating the gradient and projection onto the model 
% i.e. for positive integers $k$ and an initialization $x^0$ we have:
%\eql{\label{eq:IP}x^{k} = \pp_\Cc\left(x^{k-1}-\mu \nabla f(x^{k-1})\right) }
%where, $\nabla f(x)=A^T(Ax-y)$. It has been shown that under certain conditions and for a well chosen step size $\mu$ IPG converges to a solution of \eqref{eq:p1}.
%We do not make a particular assumption on the initialization of \eqref{eq:IP} and we set $x^0=0$ throughout. 
Interesting features of IPG include flexibility of handling various and often complicated signal models, e.g. $\Cc$ might be convex, nonconvex or/and semi-algebraic such as sparsity or rank constraints (these last models result in challenging combinatorial optimization problems but with tractable projection operators).  
%, as long as local oracles gradient (the forward model) and Euclidean projection onto $\Cc $ are available. 
%As long as the Euclidean projection\footnote{\todo{Could be generalize by proximal operators to solve sum of the LS with a non-smooth function}} operator is well-defined.
Also IPG (and more generally the proximal-gradient methods) has been considered to be particularly useful for big data applications \cite{Volkan:bigdata}. It is memory efficient due to using only first order local oracles e.g., the gradient and the projection onto $\Cc$, it can be implemented in a distributed/parallel fashion, and it is also robust to using cheap statistical estimates e.g. in stochastic descend methods~\cite{Bottou:SGD} to shortcut heavy gradient computations.% and as its rate of convergence is nearly independent of data size.


In this regard a major challenge that IPG may encounter is the computational burden of performing an \emph{exact} projection step onto certain complex models (or equivalently, performing an exact but complicated gradient step). In many interesting inverse problems the model projection amounts to solving another optimization within each iteration of IPG. This includes important cases in practice such as the total variation penalized least squares \cite{Chambolle2011,TVprojGabirel}, low-rank matrix completion \cite{Ma2011} or tree sparse model-based CS \cite{modelbasedCS}. Another example is the convex inclusion constraints $\Cc=\bigcap_i \Cc_i$, appearing in multi constrained problems e.g.~\cite{SPCA,LRJS}, where one might be required to perform  a Djkstra type feasibility algorithm at each iteration~\cite{Dykstra,Dykstra2}. Also, for data driven signal models the projection will typically involve some form of search through potentially large datasets. 
In all these cases accessing an exact oracle could be either computationally inefficient or even not possible (e.g. in analysis sparse recovery~\cite{GiryIPGaprox} or  tensor rank minimization~\cite{Holger:tensor} where the exact projection is NP hard), and therefore a natural line of thought is to carry those steps with cheaper \emph{approximate} oracles.

\subsection{Contributions}
In this paper we feature an important property of the IPG algorithm; that \emph{it is robust against deterministic  errors in calculation of the projection and gradient steps.} 
We cover different types of oracles: i) A \emph{fixed precision} (FP) oracle which compared to the exact one has an additive bounded approximation error.  ii) A \emph{progressive fixed precision} (PFP) oracle which allows for larger (additive) approximations in the earlier iterations and refines the precision as the algorithm  progresses. iii) A $(1+\epsilon)$-approximate oracle which introduces a notion of relatively optimal approximation with a multiplicative error (as compared to the exact oracle). 

Our analysis uses a notion of model restricted bi-Lipschitz \emph{embedding} similar to e.g. \cite{Blumen}, however in a more local form and with an improved conditioning (we discuss this in more details in Section~\ref{sec:main}). 
%with improved  
%recovery condition in two ways: first by relaxing the purely uniform restrictions to a more local form which can avoid worst-case scenarios unnecessarily restricting the analysis, and second, by improving the requirement on the embedding conditioning by a factor (we discuss these in more details in Section~\ref{sec:main}). 
With that respect, our analysis differs from the previous related works in the convex settings as the embedding  enables us for instance to prove a globally optimal recovery result for nonconvex models, as well as establishing linear rate of convergences for the inexact IPG applied for solving CS problems (e.g. results of \cite{BachinexactIPG} on linear convergence of the inexact IPG assumes strong convexity which does not hold in solving underdetermined least squares such as CS). 

In summary, we show that the FP type oracles restrict the final accuracy of the main reconstruction problem. 
% to be proportional to the approximation level introduced within each iteration. 
This limitation can be overcome by increasing the precision at an appropriate rate using the PFP type oracles where one could achieve the same solution accuracy as for the exact IPG algorithm under the same embedding assumptions (and even with the convergence rate). We show that the $(1+\epsilon)$-approximate projection can also achieve the accuracy of exact IPG however under a stronger embedding assumption, moderate compression ratios and using possibly more iterations (since using this type of oracle typically decreases the rate of convergence). In all the cases above we study conditions that provide us with linear convergence results. 

Finally we apply this theory to a stylized data driven compressed sensing application that requires a nearest neighbour search order to calculate the model projection. We shortcut the computations involved,  (iteratively) performing exhaustive searches over large datasets, by using approximate nearest neighbour search strategies corresponding to the aforementioned oracles and motivated by the cover tree structure introduced in \cite{beygelzimer2006cover}. Our proposed algorithm achieves a complexity logarithmic  in terms of the dataset population (as opposed to the linear complexity of a brute force search). 
By running several numerical experiments on different  datasets we conclude similar observations as predicted by our theoretical results.
\subsection{Paper organization}
The rest of this paper is organized as follows: In Section~\ref{sec:SOA} we review and compare our results to the previous related works on inexact IPG. In Section~\ref{sec:prelim} we define the inexact IPG algorithm for three types of approximate oracles. Section~\ref{sec:main} includes our main theoretical results on robustness and linear convergence of the inexact IPG for solving CS reconstruction problem. %In this section we also define a hybrid (local-uniform) notion of bi-Lipschitz embeddingg.
 In  Section~\ref{sec:datadrivenCS} we discuss an application of the proposed inexact algorithms to accelerate solving data driven CS problems. We also briefly discuss the cover tree data structure and the associated exact/approximate search strategies.  Section~\ref{sec:expe} is dedicated to the numerical experiments on using inexact IPG for data driven CS. 
And finally we discuss and conclude our results in Section~\ref{sec:conclusion}.

%This problem has been partly studied in both convex and nonconvex setting for which we provide more details in the next section. 

% This problem a.k.a. the \emph{inexact} IPG, has been studied in convex settings (and even for more general objectives than the CS problem under consideration here). A \emph{fixed precision} (FP) approximate oracles could be often obtained thanks to invoking the duality-gap information and stopping earlier the auxiliary optimization w.r.t. solving the projection. In this case the solution accuracy of the main  problem can not get better than the introduced approximation level. More recently \cite{BachinexactIPG} proposed and studied a \emph{progressive fixed precision} (FPF) type approximation scheme which refines the inexactness levels at an appropriate rate and achieves the same accuracy as for the exact IPG algorithm. However this result is only customized for convex problems and does not prove a linear convergence for CS related problems with non strong convexity.
%
%In nonconvex settings this has been tried for certain cases e.g. lowrank, tree-sparse, earth mover distance CS and by using $1+\epsilon$ type approximations.  This type of approximation introduces a notion of relatively optimal oracles with  a multiplicative error (as compared to the exact oracle).




%In this paper we show another important feature of IPG; that it is robust against deterministic (worst case) errors in calculation of the projection and gradient steps. This has straightforward implications \todo{e.g. for robust computation in distributed systems with errors}. However and of particular interest, this feature implies that one can  approximately conduct both steps to save in computations. For instance, this opens the possibility of using more complex models (which offer efficient representations for better CS reconstruction) with computationally exhaustive or even NP-hard projections, however approximable!







%. Moreover, we show that for certain decaying approximation errors, IPG solution maintain  the same accuracy as for the approximation-free (exact IPG) case.


%A general convergence result holds for both convex and non-convex constraints $\Cc$, that is by choosing the step size small enough $\gamma\leq 1/\norm{A}$, IPGD converges to a critical point of the problem \eqref{eq:p1} which for convex $\Cc$ is also a global convergence i.e., $\lim{x^{t}}\rightarrow x^{opt}$.  \nref ATOUCH. 
%
%Another class of results, guarantee near-optimal linear convergence of IPGD for non-convex $\Cc$, when $A$ drawn at random from certain distributions (e.g. element-wise i.i.d. subgaussian, or random orthoprojector,...) and when number of the measurements $m > \dim(\Cc)$. \marginpar{can we be more regorous about dim here?} 


  

%\section{Introduction}
Signal inference under limited and noisy observations is a major line of research in signal processing, machine learning and statistics\todo{and it has a wide application ranging from biomedical imaging, astrophysics, remote sensing  to data mining..ask mike to complete}. Exploring the structure of signals is proven to significantly help with an accurate inference since natural datasets often have limited degrees of freedom as compared to their original ambient dimensionality. This fact has been invoked in Compressed sensing (CS)  literature by adopting efficient signal models to achieve accurate signal reconstruction given near-minimal number of measurements i.e. much smaller than the signal original dimension (see \cite{DonohoCS, CRT:CS,Tropp:SOMP1,Candes:EMC,BW:manifold,modelbasedCS} and e.g. \cite{RichCSreview} for an overview on different CS models).
\todo{This for instance in sensory scenarios such as in MRI, video, high resHSI saves a lot of energy cost time...} 
%Compressed sensing (CS) algorithms adopt efficient signal models to achieve accurate reconstruction given small number of measurements. 
CS consists of a linear sampling protocol:
\eql{
	\label{eq:CSsampling}
	y \approx Ax^\gt,
}
where a linear mapping $A$ samples a $m$-dimensional vector $y$ of noisy measurements from a ground truth signal $x^\gt$ which typically lives in a high ambient dimension $n\gg m$. Natural signals often have efficiently compact representations using nonlinear models such as low dimensional smooth manifolds, lowrank matrices or the Union of Subspaces (UoS) that itself includes sparse (unstructured) or structured sparse (e.g. group, tree or analysis sparsity) representations in properly chosen orthobases or redundant dictionaries\nref.
%An efficient CS scheme incorporates such non-linearity into the reconstruction phase to achieve robustness and better compression rates. 
CS reconstruction algorithms for estimating $x^\gt$ from $y$ are in general more computationally complex (as opposed to the CS simple linear acquisition) as they require to handle  mostly non-linear and complicated models. % signal models.  
 A proper model should be carefully chosen in order to efficiently promote the low-dimensional structure of signal meanwhile not bringing a huge computational burden to the reconstruction algorithm.

Consider the following constrained least square problem for CS reconstruction:
\eql{\label{eq:p1}
	\min_{x\in \Cc'} \{f(x):= \frac{1}{2}\norm{y-Ax}^2\},
}
where, the constraint set $\Cc'\in \RR^n$ represents the signal model. First order algorithms in the form of projected Landweber iteration a.k.a. iterative projected gradient (IPG) descend or Forward-Backward are very popular for solving \eqref{eq:p1}. 
% in both convex and nonconvex settings. 
%In this case IPG iterates between calculating the gradient and projection onto the model 
% i.e. for positive integers $k$ and an initialization $x^0$ we have:
%\eql{\label{eq:IP}x^{k} = \pp_\Cc'\left(x^{k-1}-\mu \nabla f(x^{k-1})\right) }
%where, $\nabla f(x)=A^T(Ax-y)$. It has been shown that under certain conditions and for a well chosen step size $\mu$ IPG converges to a solution of \eqref{eq:p1}.
%We do not make a particular assumption on the initialization of \eqref{eq:IP} and we set $x^0=0$ throughout. 
Interesting features of IPG include flexibility of handling various and often complicated models, e.g. $\Cc'$ might be convex, nonconvex or/and semi-algebraic such as sparsity or rank constraints which for the latter IPG deals with solving a combinatorial problem.
%, as long as local oracles gradient (the forward model) and Euclidean projection onto $\Cc' $ are available. 
%As long as the Euclidean projection\footnote{\todo{Could be generalize by proximal operators to solve sum of the LS with a non-smooth function}} operator is well-defined.
Also IPG (and more generally the proximal-gradient methods) has been considered to be particularly useful for big data applications \cite{Volkan:bigdata}\todo{or a better ref?}; It is memory efficient due to using only first order local oracles e.g., the gradient and the Euclidean projection onto $\Cc'$, it can be implemented in a distributed/parallel fashion, it is robust to using cheap statistical estimates, e.g. in stochastic descend methods\nref, to shortcut heavy gradient computations and as its rate of convergence is nearly independent of data size.


In this regard a major challenge that IPG encounters is the computational burden of performing an \emph{exact} projection step onto certain complex models. In many interesting inverse problems the model projection amounts to solving another optimization within each iteration. This includes important cases in practice such as the total variation penalized least squares \cite{Chambolle2011,TVprojGabirel}, lowrank matrix completion \cite{Ma2011} or tree sparse model-based CS \cite{modelbasedCS} to name a few. Another example is the convex inclusion constraints $\Cc'=\bigcap_i \Cc'_i$, appearing in \nref[spca][meLRJS], where one might be required to perform e.g. a Djkstra-type feasibility algorithm at each iteration\nref[combette pesqe].
In all these cases accessing to an exact oracle could be either computationally inefficient or even not possible (e.g. in analysis sparsity \nref), and therefore a natural line of thought is to carry those steps with cheaper \emph{approximate} oracles.

In this paper we feature an important property of the IPG algorithm; that \emph{it is robust against deterministic (worst case) errors in calculation of the projection and gradient steps.} 
We cover different types of oracles: i) A \emph{fixed precision} (FP) oracle which compared to the exact one has an additive bounded approximation error.  ii) A \emph{progressive fixed precision} (PFP) oracle which allows for larger (additive) approximations in the earlier iterations and refines the precision at an appropriate rate as the IPG iterations progress. iii) A $(1+\epsilon)$-approximate oracle which introduces a notion of relatively optimal approximation with a multiplicative error (as compared to the exact oracle). Our analysis uses a notion of model restricted bi-Lipschitz embedding similar as in e.g. \cite{Blumen}, however with improving the 
%(w.r.t. the forward operator $A$ and the signal model $\Cc'$)
%However improve the Blumensath's 
recovery condition in two folds: first by relaxing the purely uniform restrictions to a more local form which can avoid worst-case scenarios unnecessarily restricting the analysis, and second, by improving the requirement on the embedding conditioning by a factor (we discuss these in more details in Section X). With that respect, our analysis also slightly differentiates from the previous related works in convex settings as it enables us for instance to prove a globally optimal recovery result for nonconvex models, as well as establishing linear rate of convergences for the inexact IPG applied to the CS problems (e.g., results of \cite{BachinexactIPG} on linear convergence of the inexact IPG assumes strong convexity which does not hold in solving underdetermined least squares such as CS). In summary, we show that the FP type oracles restrict the final accuracy of the main reconstruction problem to be proportional to the approximation level introduced within each iteration. This limitation can be overcome by increasing the precision at an appropriate rate and one could achieve the same solution accuracy as for the exact IPG algorithm (and even within the same number of iterations) using the PFP type oracles. We show that the $(1+\epsilon)$-approximate projection can also achieve the accuracy of exact IPG however for moderately conditioned systems $A$ and using possibly more iterations (since using this type of oracle typically decreases the rate of convergence). In all cases above we study conditions that provide us with linear convergence results. Finally we study application of the inexact IPG  in data-driven compressed sensing problem. We shortcut the computations involved with (iteratively) performing exhaustive searches over large datasets by using approximate nearest neighbour search strategies corresponding to the aforementioned oracles and motivated by the cover tree structure introduced in \cite{beygelzimer2006cover}. By running several numerical experiments on a number of synthetic and real datasets we conclude similar observations as predicted by  our theoretical results.

The rest of this paper is organized as follows: In Section~\ref{sec:SOA} we review the previous related works on inexact IPG and compare our results/contributions to them. In Section~\ref{sec:prelim} we define the inexact IPG algorithm for three types of approximate oracles. Section~\ref{sec:main} includes our main theoretical bounds for robust CS recovery and linear convergence of the inexact IPG. %In this section we also define a hybrid (local-uniform) notion of bi-Lipschitz embeddingg.
We discuss an application of the proposed inexact IPG algorithms for data driven CS problems in Section~\ref{sec:datadrivenCS}. Section~\ref{sec:expe} is dedicated to the numerical experiments on using inexact IPG for data driven CS. 
And finally we discuss and conclude our results in Section XX\nref.

%This problem has been partly studied in both convex and nonconvex setting for which we provide more details in the next section. 

% This problem a.k.a. the \emph{inexact} IPG, has been studied in convex settings (and even for more general objectives than the CS problem under consideration here). A \emph{fixed precision} (FP) approximate oracles could be often obtained thanks to invoking the duality-gap information and stopping earlier the auxiliary optimization w.r.t. solving the projection. In this case the solution accuracy of the main  problem can not get better than the introduced approximation level. More recently \cite{BachinexactIPG} proposed and studied a \emph{progressive fixed precision} (FPF) type approximation scheme which refines the inexactness levels at an appropriate rate and achieves the same accuracy as for the exact IPG algorithm. However this result is only customized for convex problems and does not prove a linear convergence for CS related problems with non strong convexity.
%
%In nonconvex settings this has been tried for certain cases e.g. lowrank, tree-sparse, earth mover distance CS and by using $1+\epsilon$ type approximations.  This type of approximation introduces a notion of relatively optimal oracles with  a multiplicative error (as compared to the exact oracle).




%In this paper we show another important feature of IPG; that it is robust against deterministic (worst case) errors in calculation of the projection and gradient steps. This has straightforward implications \todo{e.g. for robust computation in distributed systems with errors}. However and of particular interest, this feature implies that one can  approximately conduct both steps to save in computations. For instance, this opens the possibility of using more complex models (which offer efficient representations for better CS reconstruction) with computationally exhaustive or even NP-hard projections, however approximable!







%. Moreover, we show that for certain decaying approximation errors, IPG solution maintain  the same accuracy as for the approximation-free (exact IPG) case.


%A general convergence result holds for both convex and non-convex constraints $\Cc'$, that is by choosing the step size small enough $\gamma\leq 1/\norm{A}$, IPGD converges to a critical point of the problem \eqref{eq:p1} which for convex $\Cc'$ is also a global convergence i.e., $\lim{x^{t}}\rightarrow x^{opt}$.  \nref ATOUCH. 
%
%Another class of results, guarantee near-optimal linear convergence of IPGD for non-convex $\Cc'$, when $A$ drawn at random from certain distributions (e.g. element-wise i.i.d. subgaussian, or random orthoprojector,...) and when number of the measurements $m > \dim(\Cc')$. \marginpar{can we be more regorous about dim here?} 


  

\section{Related Work}\label{sec:related}
 
The authors in \cite{humphreys2007noncontact} showed that it is possible to extract the PPG signal from the video using a complementary metal-oxide semiconductor camera by illuminating a region of tissue using through external light-emitting diodes at dual-wavelength (760nm and 880nm).  Further, the authors of  \cite{verkruysse2008remote} demonstrated that the PPG signal can be estimated by just using ambient light as a source of illumination along with a simple digital camera.  Further in \cite{poh2011advancements}, the PPG waveform was estimated from the videos recorded using a low-cost webcam. The red, green, and blue channels of the images were decomposed into independent sources using independent component analysis. One of the independent sources was selected to estimate PPG and further calculate HR, and HRV. All these works showed the possibility of extracting PPG signals from the videos and proved the similarity of this signal with the one obtained using a contact device. Further, the authors in \cite{10.1109/CVPR.2013.440} showed that heart rate can be extracted from features from the head as well by capturing the subtle head movements that happen due to blood flow.

%
The authors of \cite{kumar2015distanceppg} proposed a methodology that overcomes a challenge in extracting PPG for people with darker skin tones. The challenge due to slight movement and low lighting conditions during recording a video was also addressed. They implemented the method where PPG signal is extracted from different regions of the face and signal from each region is combined using their weighted average making weights different for different people depending on their skin color. 
%

There are other attempts where authors of \cite{6523142,6909939, 7410772, 7412627} have introduced different methodologies to make algorithms for estimating pulse rate robust to illumination variation and motion of the subjects. The paper \cite{6523142} introduces a chrominance-based method to reduce the effect of motion in estimating pulse rate. The authors of \cite{6909939} used a technique in which face tracking and normalized least square adaptive filtering is used to counter the effects of variations due to illumination and subject movement. 
The paper \cite{7410772} resolves the issue of subject movement by choosing the rectangular ROI's on the face relative to the facial landmarks and facial landmarks are tracked in the video using pose-free facial landmark fitting tracker discussed in \cite{yu2016face} followed by the removal of noise due to illumination to extract noise-free PPG signal for estimating pulse rate. 

Recently, the use of machine learning in the prediction of health parameters have gained attention. The paper \cite{osman2015supervised} used a supervised learning methodology to predict the pulse rate from the videos taken from any off-the-shelf camera. Their model showed the possibility of using machine learning methods to estimate the pulse rate. However, our method outperforms their results when the root mean squared error of the predicted pulse rate is compared. The authors in \cite{hsu2017deep} proposed a deep learning methodology to predict the pulse rate from the facial videos. The researchers trained a convolutional neural network (CNN) on the images generated using Short-Time Fourier Transform (STFT) applied on the R, G, \& B channels from the facial region of interests.
The authors of \cite{osman2015supervised, hsu2017deep} only predicted pulse rate, and we extended our work in predicting variance in the pulse rate measurements as well.

All the related work discussed above utilizes filtering and digital signal processing to extract PPG signals from the video which is further used to estimate the PR and PRV.  %
The method proposed in \cite{kumar2015distanceppg} is person dependent since the weights will be different for people with different skin tone. In contrast, we propose a deep learning model to predict the PR which is independent of the person who is being trained. Thus, the model would work even if there is no prior training model built for that individual and hence, making our model robust. 

%
%!TEX ROOT = ../centralized_vs_distributed.tex

\section{Problem Setup}\label{sec:setup}

We consider {an undirected network} with $ N $ agents 
{in which the state of the $ i $th agent at time $ t $ is given by $ \xbar{i}{t}\in\Real{} $ with the control input $ \u{i}{t}\in\Real{} $.}
For {notational} convenience,
we {introduce} the aggregate {state of the} system  $ \xbar{}{t} $ and {the}
aggregate control input $ \u{}{t} $ {by stacking states and control inputs of each subsystem} $ \xbar{i}{t} $ and $ \u{i}{t} $, 
respectively.

\iffalse
\begin{rem}[Network topology]
	While in the first part of the paper we focus on circular formations for the sake of analysis and ease of presentation,
	the control design can be readily extended to generic undirected topologies.
	We discuss theoretical guarantees in~\autoref{sec:generic-topology} and observe with computational experiments in~\autoref{sec:numerical-results}
	that the \tradeoff holds regardless of the specific topology. % at hand.
\end{rem}
\fi

\myParagraph{Problem Statement}
The agents aim to reach consensus towards a common state trajectory. 
The $i$th component of the vector $ \x{}{t} \doteq \Omega\xbar{}{t} $ represents 
the mismatch between the state of agent $ i $ and the average network state at time $ t $\revision{\cite{bamjovmitpat12}},
where
\begin{equation}\label{eq:error-matrix}
	\Omega \doteq I_{N}-\consMatrix
\end{equation}
and $ \mathds{1}_N \in\Real{N} $ is the vector of all ones,
such that $ \Omega\mathds{1}_N=0 $.
%\red{The target consensus vector is defined as $ \x{m}{t} \doteq \xbar{}{t} - \x{}{t} $.}

\done{
	\tcb{\myParagraph{Ring Topology}
	We focus on ring topology to obtain analytical insights about 
	optimal control design and fundamental performance trade-offs in the presence of communication delays. 
	While some of our notation is tailored to such topology (\eg see equations~\eqref{eq:meas} and~\eqref{eq:feedback-matrix}), 
	in~\autoref{sec:generic-topology} we discuss extension of the optimal control design to generic undirected networks 
	and complement these developments with computational experiments in~\autoref{sec:numerical-results}.}
}

%\myParagraph{\titlecap{Communication model}}
%The agents communicate through a {shared wireless channel}.
%Data are exchanged through a {shared wireless channel} in a symmetric fashion.
%Agent $ i $ communicates with
%\red{$ n $ pairs of agents,
%both agents in each such pair being at equal distance from $ i $}
%\tcb{its} $ 2n $ closest neighbors \tcb{in ring topology.}
%Also, we make the following assumption
%to address channel constraints.

\begin{ass}[Communication model]\label{ass:hypothesis}
	Data are exchanged through a shared wireless channel in a symmetric fashion.
%	\tcb{its} $ 2n $ closest neighbors \tcb{in ring topology.}
	\revisiontwo{Agent $ i $ receives state measurements from
	all agents within $ n $ communication hops.}
	All measurements are received with delay $ \taun \doteq f(n) $
	where $ f(\cdot) $ is a positive increasing sequence.
	\revisiontwo{In particular,
		in ring topology,
		agent $ i $ receives state measurements from the
		$ 2n $ closest agents,
		that is,
		from the $ n $ pairs of agents at distance $ \ell = 1,\dots,n $,
		with $ 1\le n<\nicefrac{N}{2} $.}\footnote{\revision{
%	where both agents in each such pair are at equal distance $ \ell $ from $ i $. % in the ring topology.
%%	located $ \ell $ positions ahead and behind in the formation,.
%	In what follows,
%	without loss of generality,
%	we assume that such $ n $ agent pairs coincide with the
%	$ 2n $ closest agents in ring topology,
%	and that each pair is at distance $ \ell = 1,\dots,n<\nicefrac{N}{2} $.
	\revisiontwo{For example,
	$ n = 1 $ corresponds to nearest-neighbor interaction in ring topology
	and $ n = \floor{\nicefrac{(N-1)}{2}} $ to all-to-all communication topology.}}}
%	Also, each agent measures its own state instantaneously.
\end{ass}

\revision{\begin{rem}[Architecture parametrization]\label{rem:architecture-param}
	Parameter $ n $ will play a crucial role throughout our discussion. 
	In particular,
	we will use it to (i) evaluate the optimal performance %that can be attained 
	for a given
	budget of links
	\revisiontwo{(see~\cref{prob:variance-minimization})};
	and to (ii) compare optimal performance of different control architectures.
	In the first part of the paper, 
	we examine circular formations and
	$ n $ represents how many neighbor pairs communicate with each agent.
	For \linebreak general undirected networks,
	$ n $ determines the number of communication hops for each agent.
	In general,
	$ n $ characterizes sparsity of a controller architecture:
	sparse controllers correspond to small $n$ while highly connected ones to 
	large $ n $.
\end{rem}}

%\begin{rem}
%	The time $ \delayn $ embeds both the communication delay,
%	due to channel constraints,
%	and the computation delay,
%%	Even though the rate $ f(n) = n $ may seem natural,
%%	other rates are possible, 
%	arising if the agents preprocess the acquired measurements.
%	In practice, $ f(n) $ is to be estimated or learned from data.
%\end{rem}

\myParagraph{\titlecap{Feedback control}}
Agent $ i $ uses the received information to compute the
state mismatches $ \meas{i}{\ell^\pm}{t} $ {relative to its} neighbors,
\begin{equation}\label{eq:meas}
	\meas{i}{\ell^\pm}{t} = 
	\begin{cases}
		\xbar{i}{t} - \xbar{i\pm\ell}{t}, & 0<i\pm\ell\le N\\
		\xbar{i}{t} - \xbar{i\pm\ell\mp N}{t}, & \mbox{otherwise},
	\end{cases}
\end{equation}
%Such mismatches are exploited to compute the 
{and} the proportional control input is {given by}
\begin{equation}\label{eq:prop-control}
	\u{P,i}{t} = -\sum_{\ell=1}^{n}k_\ell\left(\meas{i}{\ell^+}{t-\taun}+\meas{i}{\ell^-}{t-\taun}\right),
\end{equation}
where measurements are delayed according to~\cref{ass:hypothesis}.

For networks with double integrator agents,
the control input $u_i(t)$ may also include a derivative term,
%Depending on the agent dynamics, the control input $ \u{i}{t} $
%may be purely proportional or include a derivative term, such as
\begin{equation}\label{eq:control-input-PD}
	\u{i}{t} = \gvel\u{P,i}{t} - \gvel\dfrac{d\xbar{i}{t}}{dt} = \gvel\u{P,i}{t} -\gvel\dfrac{d\x{i}{t}}{dt}.
\end{equation}
The derivative term in~\eqref{eq:control-input-PD} is delay free
because it only requires measurements coming from the agent itself,
which we assume {to be} available instantaneously. 
%The latter will be defined in due time.
The proportional input can be compactly written as $ \u{P}{t} = -K\xbar{}{t-\taun}=-K\x{}{t-\taun} $.
\revision{With ring topology, the feedback gain matrix is}
\begin{equation}\label{eq:feedback-matrix}
%	\begin{array}{c}
%		K \doteq K_f + K_f^\top \\
		K = \mathrm{circ}
		\left(\sum_{\ell=1}^nk_\ell, -k_1, \dots, -k_n, 0,  \dots, 0, -k_n, \dots, -k_1\right),
%	\end{array}
\end{equation}
where $ \mathrm{circ}\left(a_1,\dots,a_n\right) $ denotes the circulant matrix in $ \Real{n\times n} $
with elements $ a_1,\dots,a_n $ in the first row.

\revisiontwo{For agents with additive stochastic disturbances
	(see Sections \ref{sec:cont-time} and~\ref{sec:disc-time}),
	we consider the following problem for each $ n $.}

\begin{prob}\label{prob:variance-minimization}
	Design the feedback gains in order to minimize the steady-state variance of the consensus error,
%	\marginpar{\tiny Added both problems to highlight the optimization variables in the two cases.}
	\blue{\begin{subequations}\label{eq:problem}
		\begin{equation}\label{eq:variance-minimization-P}
		\mbox{P control:} \qquad \argmin_{K} \; \var(K),
		\end{equation}
		\begin{equation}\label{eq:variance-minimization-PD}
		\mbox{PD control:} \qquad \argmin_{\gvel,K} \; \var(\gvel,K),
		\end{equation}
	\end{subequations}}
	where
	\begin{equation}\label{}
	\var \doteq \lim_{t\rightarrow+\infty} \mathbb{E}\left[\lVert\x{}{t}\rVert^2\right]
	\end{equation}
	and w.l.o.g. we assume $ \mathbb{E}\left[\x{}{\cdot}\right] \equiv \mathbb{E}\left[\x{}{0}\right] = 0 $.
	%	and $ \varx{x} \stackrel{!}{=} $ if the system is mean-square unstable.
\end{prob}
%\section{Preliminaries}\label{sec:prelim}
Iterative projected gradient iterates between calculating the gradient and projection onto the model 
i.e. for positive integers $k$ the \emph{exact} form of IPG follows:
\eql{\label{eq:IP}
	x^{k} = \pp_{\Cc'}\left(x^{k-1}-\mu \nabla f(x^{k-1})\right) }
where, $\nabla f(x)=A^T(Ax-y)$ and $\pp_\Cc'$ denote the exact gradient and the Euclidean projection oracles, respectively.  We do not make a particular assumption on  the initialization and we set $x^0=\mathbf{0}$ throughout (also for the inexact form of IPG which we shall define next). The exact IPG requires   
%problem \eqref{eq:p1} has solution(s) and	
the constraint set $\Cc'$ 
 to have a well defined, not necessarily unique but  computationally tractable Euclidean projection $\pp_{\Cc'}:\RR^n\rightarrow \Cc'$ 
% \footnote{Closeness of $\Cc$ %and a lower bound on $f$ 
% 	implies the existence of such minimum i.e. the exact projection. %$x^*$ for \eqref{eq:p1}. 
% 	%One might drop the closeness condition but rather use a FP type approximate projection as discussed in \cite{Blumen}.
% 	}:
\eq{
	\pp_{\Cc'}(x)\in \argmin_{u\in\Cc'}\norm{u-x}.}
Throughout we use $\norm{.}$ as a shorthand for  the Euclidean norm $\norm{.}_{\ell_2}$. 
%Closeness of $\Cc'$ 
%implies the existence of a minimum, however we do not assume uniqueness for the exact projection. 

In the following we define three types of approximate oracles which frequently appear in the literature and could be incorporated within the IPG iterations. We also briefly discuss their applications. 
\newline
\textbf{Important note:} We incorporate throughout the flexibility of considering an approximate projection onto a (possibly) larger set $C$ including the original signal model $x^\gt \in C'\subseteq C$ , which for instance finds practical implications in tree-sparse signal or lowrank matrix CS recovery, see \cite{HegdeISIT,MatrixAlpsapprox}.
\subsection{Fixed Precision (FP) approximate oracles}
\label{sec:FP}
We first consider 
%fixed precision (FP) 
approximate oracles with \emph{additive} bounded-norm 
%varying 
errors, namely the fixed precision gradient oracle $\nablaa^{\nug} f(x):\RR^n\rightarrow \RR^n$ where:
\begin{align}
\norm{\nablaa^{\nug} f(x) -\nabla f(x)}\leq \nug, \label{eq:grad} 
\end{align}
and the fixed precision projection oracle $\pp_\Cc^{\nup}:\RR^n\rightarrow\Cc$ where:
\begin{align}
\pp_\Cc^{\nup}(x) \in \Big\{ u\in \Cc :\,	\norm{u-x}^2 \leq \inf_{u'\in \Cc'}\norm{u'-x}^2 +\nup  \Big\}.\label{eq:proj1}
\end{align} 
The values of $\nug,\nup$ denote the levels of inaccuracy in calculating the gradient and projection steps respectively. 
The corresponding \emph{inexact} IPG
%for the gradient IPG with a fixed-precision accuracy 
iterates as follows:
%\eql{\label{eq:inIP} x^k = \pp^{\nup^k}_{\Cc}\left(x^{k-1}-\mu( \nabla f(x^{k-1}) + \eg^k)\right) + \ep^k.}
\eql{\label{eq:inIP} x^k = \pp^{\nup^k}_{\Cc}\left(x^{k-1}
	%-\mu\left( \nabla f(x^{k-1}) + \eg^k\right)\right), }	
-\mu \nablaa^{\nug^k} f(x^{k-1})\right).}
Note that in this formulation we allow for variations in the inexactness levels at different stages of IPG. % \todo{however we assume neither $\ep^k$ or $\eg^k$ depend on $x^{k-1}$ or the previous solution updates.}  
The case where the accuracy levels are bounded by a constant threshold $\nup^k=\nup$ and $\nug^k=\nug$, $\forall k$, refers to an inexact IPG algorithm with \emph{fixed precision} (FP) approximate oracles.

% which could apply for instance in  distributed network

%In this regard a \emph{Fixed precision} (FP) inexact IPG refers to having approximate oracles with accuracy levels bounded by a constant threshold i.e.  $\nup^k=\nup$ and $\nug^k=\nug$, $\forall k$. 
\subsubsection*{Examples}
Such errors can occur for instance in distributed network
optimizations where the gradient calculations could be noisy during the communication on the network\nref[.], or in CS under UoS models with infinite subspaces \cite{Blumen} where an exact projection might not exist by definition when e.g. $\Cc=\Cc'$ is an open set, however a FP type approximation could be achievable. It also has application in finite super resolution\nref, source localization and separation \cite{TASL14,SCOOP}, and data driven CS problems e.g., in Magnetic Resonance Fingerprinting \cite{MRF,BLIPsiam},  where typically a continuous manifold is first discretized into a finite precision and large dictionary which is eventually used for e.g. sparse recovery.\todo{but this is $C\subseteq C'$?}
\subsection{Progressive Fixed Precision (PFP) approximate oracles}\label{sec:PFP}
One obtains a \emph{Progressive Fixed Precision} (PFP) approximate IPG algorithm by refining the FP type precisions thorough the course of iterations. Therefore any FP gradient or projection oracle which has control on tuning its accuracy could be used in this setting and follows \eqref{eq:inIP} with decaying sequences $\nup^k,\nug^k$.
\subsubsection*{Examples}
For instance this case includes projection schemes which require iteratively solving an auxiliary optimization (e.g. the total variation ball \cite{TVprojGabirel}, sparse CUR factorization \cite{BachinexactIPG} or the multi-constraint inclusions\nref, etc) and can progress (at an appropriate rate) on the accuracy of their solutions by adding more subiterations. We also discuss in Section~\ref{sec:covertree} another example 
%of the PFP type projection oracles 
in this form that is customized for fast approximate nearest neighbour searches and its 
application to the data driven CS framework.

\subsection{$(1+\epsilon)$-approximate projection}\label{sec:epsproj}
For some constraint sets obtaining a fixed precision (and thus PFP) accuracy in projections might be still computationally exhaustive, whereas a notion of relative optimality 
%oracles with multiplicative errors are 
could be more efficient to implement. The $(1+\epsilon)$-approximate projection is defined as follows and for a given $\epsilon\geq0$:
\eql{\label{eq:eproj}
	\pp_\Cc^{\epsilon}(x) \in \Big\{ u\in \Cc :\,	\norm{u-x} \leq (1+\epsilon)\inf_{u'\in \Cc'}\norm{u'-x}  \Big\}. 
}
We note that $\pp_\Cc^{\epsilon}(x)$ might not be unique. In this regard, the inexact IPG algorithm with a $(1+\epsilon)$-approximate projection takes the following form:
\eql{\label{eq:inIP2}
	x^k = \pp_\Cc^{\epsilon}\left(x^{k-1}
	%-\mu( \nabla f(x^{k-1}) + \eg^k)\right).
	-\mu \nablaa^{\nug^k} f(x^{k-1})\right).}
Note that we still assume a fixed precision gradient oracle with flexible accuracies $\nug^k$.  
%$\norm{\eg^k}\leq \nug^k$. 
One could also consider a $(1+\epsilon)$-approximate gradient oracle and in combination with those aforementioned inexact projections however for brevity and since the analysis is quite  similar to the relative approximate projection case we skip more details on this case (one might be able to link this with d'aspremond works on subselecting row of $A$ for implementing cheaper statistical gradients).  
\subsubsection*{Examples}
The tree $s$-sparse projection in $\RR^n$ and in the exact form requires solving a dynamical programming with $O(ns)$ running time \cite{DRthompson} whereas solving this problem approximately with $1+\epsilon$ accuracy requires the time complexity of $O(n\log(n))$ \cite{HegdeISIT} which suits better most imaging problems in practice with a Wavelet sparsity level of typically $s=\Omega(\log(n))$. Also \cite{StoIHT,MatrixAlpsapprox} show one can reduce the cost of lowrank matrix completion problem by using randomized linear algebra methods, e.g. see \cite{Deshpande2006,HalkoTropp}, and carry out fast lowrank factorizations with a $1+\epsilon$ type approximation. \todo{can we say something here about submodular optimization [Krause] with greedy algos and their 1+e guaranty?}
We also discuss in Section~\ref{sec:covertree} another example 
%of the PFP type projection oracles 
in this form that is customized for fast approximate nearest neighbour searches and its 
application to the data driven CS framework.


%In a similar fashion the fixed precision (FP) approximate projection oracle is defined:

 

 
%It has been shown that under certain conditions and for a well chosen step size $\mu$ IPG converges to a solution of \eqref{eq:p1}.


\section{Main results}
\label{sec:main}
\subsection{Uniform linear embeddings}
The success of CS paradigm heavily relies  on 
%the low (intrinsic) dimensionality of most natural signals which can be captured by an efficient model, as well as 
the embedding property of certain random sampling matrices which preserves signal information for low dimensional but often complicated/combinatorial models. It has been shown that IPG can stably predict the true signal $x^\gt$ from noisy CS measurements provided that $A$ satisfies the so called Restricted Isometry Property (RIP): 
\eql{\label{eq:RIP}(1-\theta)\norm{x-x'}^2 \leq \norm{A(x-x')}^2\leq (1+\theta)\norm{x-x'}^2, \quad \forall x,x' \in \Cc
}
for a small constant  $0<\theta<1$.
This has been shown for models such as sparse, low-rank and low-dimensional smooth manifold signals and by using 
IPG type reconstruction 
algorithms which in the nonconvex settings are also known as Iterative Hard Thresholding \cite{IHTCS, Ma2011, AIHT,MIP, modelbasedCS}. Interestingly these results indicate that under the RIP condition (and without any assumption on the initialization) the first order IPG algorithms with cheap local oracles can globally solve
nonconvex optimization problems. 

For instance random orthoprojectors and i.i.d. subgaussian matrices $A$ satisfy RIP when the number of measurements $m$ is proportional to the intrinsic dimension of the model (i.e. signal sparsity level, rank of a data matrix or the dimension of a smooth signal manifold, see e.g. \cite{RichCSreview} for a review on comparing different CS models and their measurement complexities) and sublinearly scales with the ambient dimension $n$. 

A more recent work generalizes the theory of IPG to arbitrary \emph{bi-Lipschitz embeddable} models \cite{Blumen}, that is for given $\Cc$ and $A$ it holds
\eq{\mm \norm{x-x'}^2\leq \norm{A(x-x')}^2\leq \MM \norm{x-x'}^2 \quad \forall x,x'\in \Cc.}
for some constants $\mm, \MM >0$. Similar to the RIP these constants are defined \emph{uniformly} over the constraint set i.e. $\forall x,x'\in \Cc$. There Blumensath shows that if \eq{\MM<1.5\mm,} then IPG robustly solves the corresponding noisy CS reconstruction problem \emph{for all} $x^\gt\in \Cc$. This result also relaxes the RIP requirement to a nonsymmetric and unnormalised notion of linear embedding whose implication in deriving sharper recovery bounds is previously studied by \cite{JaredJeff}. 

\subsection{Hybrid (local-uniform) linear embeddings}
Similarly the notion of restricted embedding plays a key role in our analysis. However we adopt a more local form of embedding and show that it is still able to guarantee stable CS reconstruction. 
{\ass{ \label{def:Lip}
		Given $(x_0\in \Cc, \Cc, A)$ there exists constants  $\MM,\mm_{x_0}>0$ for which the following inequalities hold:%\todo{$x_0$ or $x*$}
		%$A$ satisfies the following inequalities w.r.t. $\Cc$ and a point $x_0\in \Cc$:
		\begin{itemize}
			\item Uniform Upper Lipschitz Embedding (ULE)
			\begin{align*}
			\norm{A(x-x')}^2\leq \MM \norm{x-x'}^2 \quad \forall x,x'\in \Cc
			\end{align*}
			\item Local Lower Lipschitz Embedding (LLE)
			\begin{align*}
			\norm{A(x-x_0)}^2 \geq \mm_{x_0} \norm{x-x_0}^2 \quad \forall x\in \Cc
			\end{align*}
		\end{itemize}	
		Upon existence, $\MM$ and $\mm_{x_0}$ denote  respectively the smallest and largest constants for which the inequalities above hold.		
	}}
	\newline

This is a weaker assumption compared to RIP or the uniform bi-Lipschitz embedding. Note that for any $x_0\in \Cc$ we have: \eq{
\mm\leq\mm_{x_0}\leq\MM\leq \vertiii{A}^2
} 
(where $\vertiii{.}$ denotes the matrix spectral norm i.e. the largest singular value). 
However with such an assumption one has to sacrifice the \emph{universality} of the RIP-dependent results for a signal $x^\gt$ dependent analysis. Depending on the study, local analysis could be very useful to avoid e.g. worst-case scenarios that might unnecessarily restrict the recovery analysis~\cite{me:modelselecion}. Similar local assumptions in the convex settings are shown   to improve the measurement bound and the speed of convergence up to very sharp constants~\cite{recht:GW, Oymak:tradeoff}. 

Unfortunately we are currently unable to make the analysis fully local as we require the uniform ULE constraint. Nonetheless, one can always plug the stronger bi-Liptchitz assumption into our results throughout (i.e. replacing $\mmx$ with $\mm$) and regain the universality.  


%Our results on the exact and inexact FP approximate IPG improve the Blumensath's recovery condition in two folds: first by relaxing the uniform lower Lipschitz constant $\mm$ to a local form $\mmx$ which avoids worst-case scenarios that might unnecessarily restrict the recovery analysis (see Definition X), and second, by improving the factor in the Lipschitz embedding condition i.e. $\MM < 2\mmx$ for CS recovery (see e.g. Theorem 1). 






%In addition, if there exist a uniform constant $\mm>0$ such that 
%\eq{\mm\leq \mmx, \quad \forall x_0\in \Cc}
%then $A$ is a bi-Lipschitz embedding with constants $\mm,\MM$ i.e.
%\eq{\mm \norm{x-x'}^2\leq \norm{A(x-x')}^2\leq \MM \norm{x-x'}^2 \quad \forall x,x'\in \Cc.}
%Note that for any $x_0\in \Cc$ it holds \eq{\mm\leq\mm_{x_0}\leq\MM\leq \norm{A}^2.} 

%this assumption is weaker than  the so called Restricted Isometry Property (RIP) which guarantees performance of many CS reconstruction algorithms:
%{\defn{$A$ satisfies RIP w.r.t. a set $\Cc$ and a constant $0<\delta<1$, if $\forall x,x'\in \Cc$ it holds :
%		\eq{
%			(1-\delta)\norm{x-x'}^2\leq\norm{A(x-x')}^2\leq (1+\delta)\norm{x-x'}^2.
%			}
%}}
	
	
	
	

\subsection{Linear convergence of (P)FP inexact IPG for CS recovery}
In this section we show that IPG is robust against deterministic (worst case) errors. Moreover, we show that for certain decaying approximation errors, the IPG solution maintain  the same accuracy as for the approximation-free  algorithm. %In this part we assume neither $\ep^k$ or $\eg^k$ depend on $x^{k-1}$ or the previous updates.

{\thm{\label{th:inexactLS1} Assume $(x^\gt\in \Cc, \Cc,A)$ satisfy the main Lipschitz assumption  with constants $\MM< 2\mmx$. Set the step size $(2 \mmx )^{-1}<\mu\leq\MM^{-1}$. The sequence generated by Algorithm \eqref{eq:inIP} obeys the following bound:
		\eql{\label{eq:errbound}
			\norm{x^{k}-x^\gt}\leq  \rho^k \left(\norm{x^\gt}+\sum_{i=1}^k \rho^{-i} \nut^i \right)+ \frac{2\sqrt{\MM}}{\mmx(1-\rho)}w
		}
		where 
		\begin{align*}
		\rho=\sqrt{\frac{1}{\mu \mmx} -1} \qandq
		\nut^i=%\sum_{i=1}^k \rho^{-i}
		\frac{2\nug^i}{\mmx} + \frac{\nup^i}{\sqrt{\mu \mmx}},			
		\end{align*} 
		and $w=\norm{y-Ax^\gt}$.
	}} 

{\rem{Theorem \ref{th:inexactLS1} implications for the exact IPG (i.e. $\nup^k=\nug^k=0$) and inexact FP approximate IPG (i.e. $\nup^k=\nup,\nug^k=\nug, \forall k$)  improve  \cite[Theorem~2]{Blumen} in three ways: first by relaxing the uniform lower Lipschitz constant $\mm$ to a local form $\mmx\geq \mm$ with the possibility of conducting  a local recovery/convergence analysis. Second, by improving the embedding condition for CS stable recovery to 
\eql{\label{eq:cond}\MM < 2\mmx,
	} 
or $\MM < 2\mm$ for a uniform recovery $\forall x^\gt\in\Cc$. And third, by improving 
%(i.e. twice faster for the uniform case) 
the rate $\rho$ of convergence.}}

The following corollary is an immediate consequence of the linear convergence result established in Theorem~\ref{th:inexactLS1} for which we do not provide a proof:
{\cor{\label{cor:FP}With assumptions in Theorem \ref{th:inexactLS1} the IPG algorithm with FP approximate oracles achieves the solution accuracy
\eq{\norm{x^K-x^\gt}\leq 
		 	\frac{1}{1-\rho}\left(\frac{2\nug}{\mmx} + \frac{\nup}{\sqrt{\mu \mmx}}+ \frac{2\sqrt{\MM}}{\mmx}w\right) +\tau}
for any $\tau>0$ and in a finite number of iterations
\eq{
	K=\left\lceil\frac{1}{\log(\rho^{-1})} \log\left( \frac{ \norm{x^\gt}}{\tau}\right)\right\rceil
	}			
}}
As it turns out in our experiments and aligned with the result of Corollary~\ref{cor:FP}, the solution accuracy of IPG can not exceed the precision level introduced by a PF oracle. In this sense Corollary~\ref{cor:FP} is tight as a trivial converse example would be that IPG starts from the optimal solution $x^\gt$ but an adversarial FP scheme projects it to another point within a fixed distance. 

Interestingly one can deduce another implication from  Theorem~\ref{th:inexactLS1} and overcome such limitation by  using a PFP type oracle.
%Here comes 
%an interesting part of Theorem \ref{th:inexactLS1} analysing the behaviour of the PFP type oracles. 
Remarkably 
one achieves a linear convergence to a solution with the same accuracy as for the exact IPG, 
as long as $\nut^k$ geometrically decays. 	
The following corollary makes this statement explicit:
	
%{\cor{\label{cor:decay}Assume $\nut^k= O(r^k)$ for some error decay rate $0<r<1$ and a constant $C$. We have 
%			%After a finite number $k\geq K$ of iterations 
%			\eq{
%				\norm{x^{k}-x^*}\leq %\rho^k \norm{x^0-x'} +
%				\frac{2\sqrt{\MM}}{\mm_{x^\gt}(1-\rho)} w+ O({\bar \rho}^k), 
%			}
%			where $\rho$ is the same as in Theorem \ref{th:inexactLS1}, and
%			\begin{align*}
%			%&x^\rsc_{\min}\in \argmin_{x^\rsc\in \Omega}\norm{\nabla f(x^\rsc)},\\
%			%&K=\left\lceil\log\left( \frac{2\norm{\nabla f(x^\rsc_{\min})}}{m(1-\rho) \norm{x^\rsc_{\min}}}\right)/\log(\bar \rho)\right\rceil,\\
%			\bar \rho = \choice{\max(\rho,r)\quad r\neq\rho \\
%				r+\xi\qquad\quad r=\rho}
%			\end{align*}
%			for an arbitrary small $\xi>0$.	
%}
%}
	
{\cor{\label{cor:decay}Assume $\nut^k\leq Cr^k$ for some error decay rate $0<r<1$ and a constant $C$. Under the assumptions of Theorem~\ref{th:inexactLS1} the solution updates $\norm{x^{k}-x^*}$ of the IPG algorithm with PFP approximate oracles is bounded above by:
		%After a finite number $k\geq K$ of iterations 
		\begin{align*}
			%\norm{x^{k}-x^*}\leq 
			&\max(\rho,r)^k \left(\norm{x^\gt}+\frac{C}{1-\frac{\min(\rho,r)}{\max(\rho,r)}}\right) +
			\frac{2\sqrt{\MM}}{\mm_{x^\gt}(1-\rho)} w,  &r\neq \rho \\
			&\rho^k \Big(\norm{x^\gt}+Ck\Big) +
			\frac{2\sqrt{\MM}}{\mm_{x^\gt}(1-\rho)} w, &r=\rho 
		\end{align*}
		Which implies a linear convergence at rate 
%		$\bar \rho$:
%		\eq{
%		\norm{x^{k}-x^*}\leq %\rho^k \norm{x^0-x'} +
%		\frac{2\sqrt{\MM}}{\mm_{x^\gt}(1-\rho)} w+ O({\bar \rho}^k), 
%		}
%		where,
		\begin{align*}
		%&x^\rsc_{\min}\in \argmin_{x^\rsc\in \Omega}\norm{\nabla f(x^\rsc)},\\
		%&K=\left\lceil\log\left( \frac{2\norm{\nabla f(x^\rsc_{\min})}}{m(1-\rho) \norm{x^\rsc_{\min}}}\right)/\log(\bar \rho)\right\rceil,\\
		\bar \rho = \choice{\max(\rho,r)\quad r\neq\rho \\
			\rho+\xi\qquad\quad r=\rho}
		\end{align*}
		for an arbitrary small $\xi>0$.	
	}
}
{\rem{Similar to Corollary \ref{cor:FP} 
%and due to the linear convergence 
one can increase the final solution precision of the FPF type  IPG with logarithmically more iterations i.e. in a finite number $K=O(\log(\tau^{-1}))$ of iterations one achieves $\norm{x^{K}-x^*}\leq O(w)+\tau$. Therefore in contrast with the FP oracles one achieves an accuracy within the noise level $O(w)$ that is the precision of an approximation-free IPG.
}}

{\rem{Using the PFP type oracles can also maintain the rate of linear convergence identical as for the exact IPG. For this the approximation errors suffice to follow a geometric decaying rate of  $r<\rho$. % \todo{discuss suitable oracle complexity within PFP.}
}}

{\rem{The embedding condition~\eqref{eq:cond} sufficient to guarantee our stability results is invariant to the  precisions of the FP/PFP oracles and it is the same as for an exact IPG.}}

\subsection{Linear convergence of inexact IPG with $(1+\epsilon)$-approximate projection for CS recovery}
\label{sec:relative}
In this part we focus on the inexact  algorithm~\eqref{eq:inIP2} with a $(1+\epsilon)$-approximate projection. As it turns out by the following theorem we require a stronger embedding condition to guarantee the CS stability compared to the previous algorithms.
%the exact or the FP/PFP type inexact IPG. 


{\thm{\label{th:inexactLS2}  Assume $(x^\gt\in \Cc, \Cc,A)$ satisfy the main Lipschitz assumption and that
		\eq{\sqrt{2\epsilon+\epsilon^2}\leq \delta\frac{\sqrt{\mmx}}{\vertiii{A}} \qandq \MM < (2-2\delta+\delta^2) \mmx} 	
		for $\epsilon\geq 0$ and some constant $\delta \in [0,1)$.
		Set the step size $\left((2-2\delta+\delta^2) \mmx\right)^{-1}<\mu\leq\MM^{-1}$. The sequence generated by Algorithm \eqref{eq:inIP2} obeys the following bound:
		\eq{
			\norm{x^{k}-x^\gt}\leq  \rho^k \left(\norm{x^\gt}+\kappa_g \sum_{i=1}^k \rho^{-i} \nug^i \right)+ 
			%\frac{ \left( 2\frac{\sqrt{M}}{m}+\frac{\sqrt\mu}{\norm{A}}\delta \right)}
			\frac{\kappa_w}{1-\rho}w
		}
		where 
		\begin{align*}
		&\rho=\sqrt{\frac{1}{\mu \mmx} -1}+ \delta, \quad
		\kappa_g = \frac{2}{\mmx}+\frac{\sqrt\mu}{\vertiii{A}}\delta, \\		
		&\kappa_w= 2\frac{\sqrt{\MM}}{\mmx}+\sqrt\mu\delta, \qandq w=\norm{y-Ax^\gt}.
		\end{align*}		
		
	}} 	
{\rem{Similar conclusions follow as in Corollaries~\ref{cor:FP} and \ref{cor:decay} on the linear convergence, logarithmic number of iterations vs. final level of accuracy (depending whether the gradient oracle is exact or FP/PFP) however with a stronger requirement than \eqref{eq:cond} on the embedding; increasing $\epsilon$ i.e. consequently $\delta$, limits the recovery guarantee and slows down the convergence (compare $\rho$ in Theorems~\ref{th:inexactLS1} and \ref{th:inexactLS2}). Also approximations of this type result in amplifying distortions (i.e. constants $\kappa_w, \kappa_g$) due to the measurement noise and gradient errors. For example for an exact or a geometrically decaying PFP gradient updates and for a fixed $\epsilon$ chosen according to the Theorem~\ref{th:inexactLS2}  assumptions, Algorithm~\eqref{eq:inIP2} achieves a full precision accuracy $\norm{x^{K}-x^*}\leq O(w)+\tau$ (similar to the exact IPG) in a finite number $K=O(\log(\tau^{-1}))$ of iterations.
		}}



%(i.e. the LLE condition is not sufficient and we need to replace it with a uniform lower Lipschitz constant $\mm>0$ which holds $\forall x_0\in\Cc$). With such formalism one can update the noise term  in both (theorems) bounds i.e. $w:=\norm{y-Ax^{opt}}\leq \norm{y-Ax^\gt}+\norm{A(x^\gt-x^{opt})} $.


%One can also 
%incorporate throughout the flexibility of considering an approximate projection onto a possibly larger set $\Cc$ including the original signal model $\Cc'$ i.e. $x^\gt \in \Cc'\subseteq \Cc$ , which for instance finds practical implications in tree-sparse signal or lowrank matrix CS recovery, see \cite{HegdeISIT,MatrixAlpsapprox}. Such a distinction  modifies our earlier definitions \eqref{eq:proj1} and \eqref{eq:eproj}; an approximate FP projection reads, 
%\begin{align}
%\pp_\Cc^{\nup}(x) \in \Big\{ u\in \Cc :\,	\norm{u-x}^2 \leq \inf_{u'\in \Cc'}\norm{u'-x}^2 +\nup  \Big\},\label{eq:proj1-1}
%\end{align} 
%and a $(1+\epsilon)$-approximate projection is defined as
%\eql{\label{eq:eproj-1}
%	\pp_\Cc^{\epsilon}(x) \in \Big\{ u\in \Cc :\,	\norm{u-x} \leq (1+\epsilon)\inf_{u'\in \Cc'}\norm{u'-x}  \Big\}. 
%}
% Note that with respect to our distinction Theorems~\ref{th:inexactLS1} and \ref{th:inexactLS2} still hold.
% between the projection set $C$ and the original signal model $C'\subseteq C$ defined in \eqref{eq:proj1} and \eqref{eq:eproj}, our embedding assumption is on the projection set.

{\rem{\label{rem:stringe}The assumptions of Theorem~\ref{th:inexactLS2} impose a stringent requirement on the scaling of the approximation parameter i.e. $\epsilon=O\left(\sqrt{\frac{\mmx}{\vertiii{A}}}\right)$ which is not purely dependent on the model-restricted embedding condition but also on the spectral norm  $\vertiii{A}$. In this sense since  $\vertiii{A}$ ignores the structure $\Cc$ of the problem it might scale very differently than  the corresponding embedding constants $\mmx,\mm$ and $\MM$. For instance a $m\times n$  i.i.d. Gaussian matrix has w.h.p. $\vertiii{A}=\Theta(n)$ (when $m\ll n$) whereas, e.g. for  sparse signals, the embedding constants $\mm,\MM$ w.h.p. scale as $O(m)$. A similar gap exists for other low dimensional signal models and for other compressed sensing matrices e.g. random orthoprojectors. 
This indicates that the $1+\epsilon$ oracles may be
sensitive to the CS sampling ratio i.e. for $m\ll n $
we may be limited to use very small approximations  $\epsilon = O(\sqrt{ \frac{m}{n}} )$. 

%which indicates that the $1+\epsilon$ oracles are also sensitive to the CS sampling ratio i.e. for $m\ll n$ where $\sqrt{\frac{\mmx}{\vertiii{A}}}\approx \sqrt{\frac{m}{n}}$ becomes very small one should use proportionally small  approximations $\epsilon$. 

In the following we show by a deterministic example that this requirement is indeed tight. We also empirically observe in Section \ref{sec:expe} that such a limitation indeed holds in randomized settings (e.g. i.i.d. Gaussian $A$) and on average. 
Although it would be desirable to modify the IPG algorithm to avoid such restriction, as was done in~\cite{Hegde15} for specific structured sparse models, we note that this is the same term that appears due to 'noise folding' when the signal model is not exact or when there is noise in the signal domain (see the discussion in Section~\ref{sec:inexactmodel}). As such most practical CS systems will inevitably have to avoid regimes of extreme undersampling.
}}

%which is unfavourable for applying large approximation of this type to the extreme compressed sensing settings where $\sqrt{\frac{\mmx}{\vertiii{A}}}\approx \sqrt{\frac{m}{n}}\rightarrow 0$. \marginpar{\todo{rewrite it positively + relation to noise folding}}
%}}



\subsubsection*{A converse example}
Consider a noiseless CS recovery problem where $n =2$, $m=1$ and the sampling matrix (i.e. here a row vector) is
\eq{A=[\cos(\gamma)\quad -\sin(\gamma)]}
for some parameter $0\leq\gamma<\pi/2$. Consider the following one-dimensional signal model along the first coordinate:
\eq{
\Cc = \{x\in \RR^2: x(1)\in \RR,\, x(2)=0\}.
}
We have indeed $\vertiii{A}=1$. It is easy to verify that both of the embedding constants w.r.t. to $\Cc$ are
\eq{ \mmx=\MM = \cos(\gamma)^2.}
Therefore one can tune $\gamma\rightarrow \pi/2$ to obtain arbitrary small ratios for $\sqrt{\frac{\mmx}{\vertiii{A}}}=\cos(\gamma)$. 
%$\vertiii{A}$ and $\mmx$ the 

Assume the true signal, the corresponding CS measurement,  and the initialization point are
\eq{
x^\gt=[1 \quad 0]^T,\quad y=Ax^\gt=\cos(\gamma) \qandq x^0=[0 \quad 0]^T.
}
%and the initialization point $x^0=(0,0)^T$. 
Consider an adversarial $(1+\epsilon)$-approximate projection oracle which performs the following step for any given $x\in \RR^2$:
\eq{
		\pp_\Cc^{\epsilon}(x) := [x(1)+\epsilon x(2) \quad 0]^T.
	}
For simplicity we assume no errors on the gradient step. 	
By setting $\mu=1/\cos(\gamma)^2$, the corresponding inexact IPG updates as 
\eq{	
	x^k(1)= 1+\epsilon\tan(\gamma) \left(x^{k-1}(1)-1\right)  
}
and only along the first dimension (we note that due to the choice of oracle $x^k(2)=0, \forall k$). Therefore we have
\eq{	
	x^k(1)= 1-\left(\epsilon\tan(\gamma)\right)^k.
}
which requires $\epsilon<\tan^{-1}(\gamma)=O(\cos(\gamma))$ for convergence, and it diverges otherwise. As we can see for $\gamma\rightarrow \pi/2$  (i.e. where $A$ becomes extremely unstable w.r.t. sampling the first dimension) the range of admissible $\epsilon$ shrinks, regardless of the fact that the restricted embedding $\mmx=\MM$ exhibits a  perfect isometry; which is an ideal situation for solving a noiseless CS (i.e. in this case an exact IPG takes only one iteration to converge).
\subsection{When the projection is not onto the signal model}\label{sec:inexactmodel}

 One can also make a distinction between  the projection set $\Cc$ and the signal model here denoted as $\Cc'$ (i.e. $x^\gt\in\Cc'$) by modifying our earlier definitions \eqref{eq:proj1} and \eqref{eq:eproj} in the following ways:
an approximate FP projection reads, 
\begin{align}
\pp_\Cc^{\nup}(x) \in \Big\{ u\in \Cc :\,	\norm{u-x}^2 \leq \inf_{u'\in \Cc'}\norm{u'-x}^2 +\nup^2  \Big\},\label{eq:proj1-1}
\end{align} 
and a $(1+\epsilon)$-approximate projection reads
\eql{\label{eq:eproj-1}
	\pp_\Cc^{\epsilon}(x) \in \Big\{ u\in \Cc :\,	\norm{u-x} \leq (1+\epsilon)\inf_{u'\in \Cc'}\norm{u'-x}  \Big\}. 
}
With respect to such a distinction,  Theorems~\ref{th:inexactLS1} and \ref{th:inexactLS2} still hold (with the same embedding assumption/constants on the projection set $\Cc$), conditioned that $x^\gt\in \Cc$.
Indeed this can be verified by following identical steps as in the proof of both theorems.
This allows throughout the flexibility of considering an approximate projection onto a possibly larger set $\Cc$ including the original signal model $\Cc'$ i.e. $x^\gt \in \Cc'\subseteq \Cc$, which for instance finds application in fast tree-sparse signal or low-rank matrix CS recovery, see \cite{HegdeISIT,MatrixAlpsapprox}. Such an inclusion is also important to derive a uniform recovery result \emph{for all} $x^*\in\Cc'$. 

The case where $x^\gt \notin \Cc$ can also be bounded in a similar fashion as in \cite{Blumen}. We first consider a proximity point in $\Cc$ i.e. $x^{o}:=\argmin_{u\in\Cc} \norm{x^\gt-u}$  and update the noise term  to $w:=\norm{y-Ax^{o}}\leq \norm{y-Ax^\gt}+\norm{A(x^\gt-x^{o})} $. We then use Theorems~\ref{th:inexactLS1} and \ref{th:inexactLS2} to derive an error bound, here on $\norm{x^k-x^{o}}$. For this we assume the embedding condition \emph{uniformly} holds over the projection set $\Cc$ (which includes $x^{o}$). As a result we get a bound on the error $\norm{x^k-x^\gt} \leq \norm{x^\gt-x^{o}}+\norm{x^k-x^{o}}$ which includes a bias term with respect to the distance of $x^\gt$ to $\Cc$. Note that since here $w$ also includes a signal (and not only measurement) noise term introduced by  $\norm{A(x^\gt-x^{o})}$, the results are subjected to \emph{noise folding} i.e. a noise amplification 
%$\sim O(\sqrt{\frac{n}{m}})$ 
with a similar unfavourable scaling (when $m \ll n$) to our discussion in Remark~\ref{rem:stringe} (for more details on CS noise folding see e.g.~\cite{eldar:noisefolding,daven:noisefolding}). 



%\section{Main results}
\label{sec:main}
\subsection{Uniform linear embeddings}
The success of CS paradigm relies heavily on 
%the low (intrinsic) dimensionality of most natural signals which can be captured by an efficient model, as well as 
the embedding property of certain random sampling matrices which preserves signal information for low dimensional but often complicated/combinatorial models. It has been shown that the exact IPG can stably predict the true signal $x^\gt$ from noisy CS measurements provided that $A$ satisfies the so called Restricted Isometry Property (RIP): 
\eql{\label{eq:RIP}(1-\theta)\norm{x-x'}^2 \leq \norm{A(x-x')}^2\leq (1+\theta)\norm{x-x'}^2, \quad \forall x,x' \in \Cc'
}
for a small constant  $0<\theta<1$.
This has been shown for models such as sparse, low-rank and low-dimensional smooth manifold signals and by using 
IPG type reconstruction 
algorithms which in the nonconvex settings are also known as Iterative Hard Thresholding \cite{IHTCS, Ma2011, AIHT,MIP, modelbasedCS}. Interestingly these results indicate that under the RIP condition (and without any assumption on the initialization) the first order IPG algorithms with cheap local oracles can globally solve
nonconvex optimization problems. 

For instance random orthoprojectors and i.i.d. subgaussian matrices $A$ satisfy RIP when the number of measurements $m$ is proportional to the intrinsic dimension of the model (i.e. signal sparsity level, rank of a data matrix or the dimension of a smooth signal manifold, see e.g. \cite{RichCSreview} for a review on comparing different CS models and their measurement complexities) and sublinearly scales with the ambient dimension $n$. 

A more recent work generalizes the theory of IPG to arbitrary \emph{bi-Lipschitz embeddable} models \cite{Blumen}, that is for given $\Cc'$ and $A$ it holds
\eq{\mm \norm{x-x'}^2\leq \norm{A(x-x')}^2\leq \MM \norm{x-x'}^2 \quad \forall x,x'\in \Cc'.}
for some constants $\mm, \MM >0$. Similar as for the RIP these constants are defined \emph{uniformly} over the constraint set i.e. $\forall x,x'\in \Cc'$. There Blumensath shows that if \eq{\MM<1.5\mm,} then IPG robustly solves the corresponding noisy CS reconstruction problem \emph{for all} $x^\gt\in \Cc'$. This result also relaxes the RIP requirement to a nonsymmetric and unnormalised notion of linear embedding whose implication in deriving sharper recovery bounds is previously studied by \cite{JaredJeff}. 

\subsection{Hybrid (local-uniform) linear embeddings}
Similarly the notion of restricted embedding plays a key role in our analysis. However we adopt a more local form of embedding and show that it is still able to guaranty stable CS reconstruction.  \marginpar{\todo{x0 or x*? C' deleted?}}

{\ass{ \label{def:Lip}
 Given $(x_0\in \Cc'\subseteq \Cc, A)$ there exists constants  $\MM,\mm_{x_0}>0$ for which the following inequalities hold:%\todo{$x_0$ or $x*$}
		%$A$ satisfies the following inequalities w.r.t. $\Cc$ and a point $x_0\in \Cc$:
		\begin{itemize}
			\item Uniform Upper Lipschitz Embedding (ULE)
			\begin{align*}
			\norm{A(x-x')}^2\leq \MM \norm{x-x'}^2 \quad \forall x,x'\in \Cc
			\end{align*}
			\item Local Lower Lipschitz Embedding (LLE)
			\begin{align*}
			\norm{A(x-x_0)}^2 \geq \mm_{x_0} \norm{x-x_0}^2 \quad \forall x\in \Cc
			\end{align*}
		\end{itemize}	
		Upon existence, $\MM$ and $\mm_{x_0}$ denote  respectively the smallest and largest constants for which the inequalities above hold.		
	}}
\newline
	
 Note that with respect to our distinction between the projection set $C$ and the original signal model $C'\subseteq C$ defined in \eqref{eq:proj1} and \eqref{eq:eproj}, our embedding assumption is on the projection set.
 
To compare with the previous results we shall set $\Cc=\Cc'$; which implies a weaker assumption compared to RIP or the uniform bi-Lipschitz embedding. Note that for any $x_0\in\Cc'$ and $\Cc=\Cc'$ we have: \eq{
\mm\leq\mm_{x_0}\leq\MM\leq \vertiii{A}^2
} 
(where $\vertiii{.}$ denotes the matrix spectral norm i.e. the largest singular value). 
In this case one has to sacrifice the \emph{universality} of the RIP dependant results for a signal $x^\gt$ dependent analysis. Depending on the study, local analysis could be very useful as for instance allows for avoiding worst-case scenarios that might unnecessarily restrict the recovery analysis~\cite{me:modelselecion}. In the convex settings it has been shown that local embeddings can improve the measurement bound and the speed of convergence up to sharp constants~\cite{recht:GW, Oymak:tradeoff}. 

Unfortunately we are currently unable to make the analysis fully local as we require the uniform RLS constraint. Nonetheless, one can always plug the stronger bi-Liptchitz assumption into our results throughout (i.e. replacing $\mmx$ with $\mm$) and regain the universality.  


%Our results on the exact and inexact FP approximate IPG improve the Blumensath's recovery condition in two folds: first by relaxing the uniform lower Lipschitz constant $\mm$ to a local form $\mmx$ which avoids worst-case scenarios that might unnecessarily restrict the recovery analysis (see Definition X), and second, by improving the factor in the Lipschitz embedding condition i.e. $\MM < 2\mmx$ for CS recovery (see e.g. Theorem 1). 






%In addition, if there exist a uniform constant $\mm>0$ such that 
%\eq{\mm\leq \mmx, \quad \forall x_0\in \Cc}
%then $A$ is a bi-Lipschitz embedding with constants $\mm,\MM$ i.e.
%\eq{\mm \norm{x-x'}^2\leq \norm{A(x-x')}^2\leq \MM \norm{x-x'}^2 \quad \forall x,x'\in \Cc.}
%Note that for any $x_0\in \Cc$ it holds \eq{\mm\leq\mm_{x_0}\leq\MM\leq \norm{A}^2.} 

%this assumption is weaker than  the so called Restricted Isometry Property (RIP) which guarantees performance of many CS reconstruction algorithms:
%{\defn{$A$ satisfies RIP w.r.t. a set $\Cc$ and a constant $0<\delta<1$, if $\forall x,x'\in \Cc$ it holds :
%		\eq{
%			(1-\delta)\norm{x-x'}^2\leq\norm{A(x-x')}^2\leq (1+\delta)\norm{x-x'}^2.
%			}
%}}
	
	
	
	

\subsection{Linear convergence of (P)FP inexact IPG for CS recovery}
In this section we show that IPG is robust against deterministic (worst case) errors. Moreover, we show that for certain decaying approximation errors, IPG solution maintain  the same accuracy as for the approximation-free (exact IPG) case.


In this part we assume neither $\ep^k$ or $\eg^k$ depend on $x^{k-1}$ or the previous updates.

{\thm{\label{th:inexactLS1} Assume $(x^\gt\in \Cc, \Cc,A)$ satisfy the main Lipschitz assumption  with constants $\MM< 2\mmx$. Set the step size $(2 \mmx )^{-1}<\mu\leq\MM^{-1}$. The sequence generated by Algorithm \eqref{eq:inIP} obeys the following bound:
		\eql{\label{eq:errbound}
			\norm{x^{k}-x^\gt}\leq  \rho^k \left(\norm{x^\gt}+\sum_{i=1}^k \rho^{-i} \nut^i \right)+ \frac{2\sqrt{\MM}}{\mmx(1-\rho)}w
		}
		where 
		\begin{align*}
		\rho=\sqrt{\frac{1}{\mu \mmx} -1} \qandq
		\nut^i=%\sum_{i=1}^k \rho^{-i}
		\frac{2\nug^i}{\mmx} + \sqrt{\frac{\nup^i}{\mu \mmx}},			
		\end{align*} 
		and $w=\norm{y-Ax^\gt}$.
	}} 

{\rem{Theorem \ref{th:inexactLS1} implications for the exact IPG (i.e. $\nup^k=\nug^k=0$) and inexact FP approximate IPG (i.e. $\nup^k=\nup,\nug^k=\nug, \forall k$)  improve  \cite[Theorem~2]{Blumen} in three folds: first by relaxing the uniform lower Lipschitz constant $\mm$ to a local form $\mmx\geq \mm$ and conduct a local recovery/convergence analysis. Second, by improving the embedding condition for CS stable recovery to 
\eql{\label{eq:cond}\MM < 2\mmx,
	} 
or $\MM < 2\mm$ for a uniform recovery $\forall x^\gt\in\Cc$. And third, by improving  twice faster the rate $\rho$ of convergence.}}

The following corollary is an immediate consequence of linear convergence established in Theorem \ref{th:inexactLS1} for which we do not provide proof:
{\cor{\label{cor:FP}With assumptions in Theorem \ref{th:inexactLS1} the IPG algorithm with FP approximate oracles achieves the solution accuracy
\eq{\norm{x^K-x^\gt}\leq 
		 	\frac{1}{1-\rho}\left(\frac{2\nug}{\mmx} + \sqrt{\frac{\nup}{\mu \mmx}},+ \frac{2\sqrt{\MM}}{\mmx}w\right) +\tau}
for a $\tau>0$ and in a finite number of iterations
\eq{
	K=\left\lceil\frac{1}{\log(\rho^{-1})} \log\left( \frac{ \norm{x^\gt}}{\tau}\right)\right\rceil
	}			
}}
As it turns out in our experiments and aligned with the result of Corollary~\ref{cor:FP}, the solution accuracy of IPG can not exceed the precision levels introduced by PF oracles. In this sense Corollary~\ref{cor:FP} is tight as a trivial converse example would be that IPG starts from the optimal solution $x^\gt$ but an adversarial FP scheme projects it to another point within a fixed distance. 

Interestingly one can make another implication of Theorem~\ref{th:inexactLS1} and overcome such limitation using the PFP type oracles.
%Here comes 
%an interesting part of Theorem \ref{th:inexactLS1} analysing the behaviour of the PFP type oracles. 
Remarkably 
one achieves a linear convergence to a solution with the same accuracy as for the exact IPG, 
as long as $\nut^k$ geometrically decays. 	
The following corollary makes this statement explicit:
	
%{\cor{\label{cor:decay}Assume $\nut^k= O(r^k)$ for some error decay rate $0<r<1$ and a constant $C$. We have 
%			%After a finite number $k\geq K$ of iterations 
%			\eq{
%				\norm{x^{k}-x^*}\leq %\rho^k \norm{x^0-x'} +
%				\frac{2\sqrt{\MM}}{\mm_{x^\gt}(1-\rho)} w+ O({\bar \rho}^k), 
%			}
%			where $\rho$ is the same as in Theorem \ref{th:inexactLS1}, and
%			\begin{align*}
%			%&x^\rsc_{\min}\in \argmin_{x^\rsc\in \Omega}\norm{\nabla f(x^\rsc)},\\
%			%&K=\left\lceil\log\left( \frac{2\norm{\nabla f(x^\rsc_{\min})}}{m(1-\rho) \norm{x^\rsc_{\min}}}\right)/\log(\bar \rho)\right\rceil,\\
%			\bar \rho = \choice{\max(\rho,r)\quad r\neq\rho \\
%				r+\xi\qquad\quad r=\rho}
%			\end{align*}
%			for an arbitrary small $\xi>0$.	
%}
%}
	
{\cor{\label{cor:decay}Assume $\nut^k\leq Cr^k$ for some error decay rate $0<r<1$ and a constant $C$. Under the assumptions of Theorem~\ref{th:inexactLS1} the solution updates $\norm{x^{k}-x^*}$ of the IPG algorithm with PFP approximate oracles is bounded above by:
		%After a finite number $k\geq K$ of iterations 
		\begin{align*}
			%\norm{x^{k}-x^*}\leq 
			&\max(\rho,r)^k \left(\norm{x^\gt}+\frac{C}{1-\frac{\min(\rho,r)}{\max(\rho,r)}}\right) +
			\frac{2\sqrt{\MM}}{\mm_{x^\gt}(1-\rho)} w,  &r\neq \rho \\
			&\rho^k \Big(\norm{x^\gt}+Ck\Big) +
			\frac{2\sqrt{\MM}}{\mm_{x^\gt}(1-\rho)} w, &r=\rho 
		\end{align*}
		Which implies a linear convergence at rate 
%		$\bar \rho$:
%		\eq{
%		\norm{x^{k}-x^*}\leq %\rho^k \norm{x^0-x'} +
%		\frac{2\sqrt{\MM}}{\mm_{x^\gt}(1-\rho)} w+ O({\bar \rho}^k), 
%		}
%		where,
		\begin{align*}
		%&x^\rsc_{\min}\in \argmin_{x^\rsc\in \Omega}\norm{\nabla f(x^\rsc)},\\
		%&K=\left\lceil\log\left( \frac{2\norm{\nabla f(x^\rsc_{\min})}}{m(1-\rho) \norm{x^\rsc_{\min}}}\right)/\log(\bar \rho)\right\rceil,\\
		\bar \rho = \choice{\max(\rho,r)\quad r\neq\rho \\
			\rho+\xi\qquad\quad r=\rho}
		\end{align*}
		for an arbitrary small $\xi>0$.	
	}
}
{\rem{Similar to Corollary \ref{cor:FP} 
%and due to the linear convergence 
one can increase the final solution precision of the FPF type  IPG with logarithmically more iterations i.e. in a finite number $K=O(\log(\tau^{-1}))$ of iterations one achieves $\norm{x^{k}-x^*}\leq O(w)+\tau$. As we can see (and in contrast with the FP oracles) one could achieve an accuracy within the noise level $O(w)$ that is the precision of an approximate-free IPG.
}}

{\rem{Using the PFP type oracles can also maintain the rate of linear convergence identical as for the exact IPG. For this the approximation errors suffice to follow a geometric decaying rate of  $r<\rho$. \todo{discuss suitable oracle complexity within PFP.}
}}

\subsection{Linear convergence of inexact IPG with $(1+\epsilon)$-approximate projection for CS recovery}
\label{sec:relative}
In this part we focus on analysing the inexact IPG algorithm with a $(1+\epsilon)$-approximate projection. As it turns out in the following theorem we require a stronger condition on the embedding to guaranty the CS stability as compared to that \eqref{eq:cond} for the previous cases.
%the exact or the FP/PFP type inexact IPG. 


{\thm{\label{th:inexactLS2} Assume $(x^\gt\in \Cc, \Cc,A)$ satisfy the main Lipschitz assumption and that
		\eq{\sqrt{\epsilon+\epsilon^2}\leq \delta\frac{\sqrt{\mmx}}{\vertiii{A}} \qandq \MM < (2-2\delta+\delta^2) \mmx} 	
		for $\epsilon\geq 0$ and some constant $\delta \in [0,1)$.
		Set the step size $\left((2-2\delta+\delta^2) \mmx\right)^{-1}<\mu\leq\MM^{-1}$. The sequence generated by Algorithm \eqref{eq:inIP2} obeys the following bound:
		\eq{
			\norm{x^{k}-x^\gt}\leq  \rho^k \left(\norm{x^\gt}+\kappa_g \sum_{i=1}^k \rho^{-i} \nug^i \right)+ 
			%\frac{ \left( 2\frac{\sqrt{M}}{m}+\frac{\sqrt\mu}{\norm{A}}\delta \right)}
			\frac{\kappa_w}{1-\rho}w
		}
		where 
		\begin{align*}
		&\rho=\sqrt{\frac{1}{\mu \mmx} -1}+ \delta, \quad
		\kappa_g = \frac{2}{\mmx}+\frac{\sqrt\mu}{\vertiii{A}}\delta, \\		
		&\kappa_w= 2\frac{\sqrt{\MM}}{\mmx}+\sqrt\mu\delta, \qandq w=\norm{y-Ax^\gt}.
		\end{align*}		
		
	}} 	
{\rem{Similar conclusions can be made as in Cor1,2 on the linear convergence, logarithmic number of iterations vs final level of accuracy (depending whether the gradient oracle is exact or FP/PFP) however with a stronger requirement than \eqref{eq:cond} on the embedding: increasing $\epsilon$ limits the recovery guaranty and slows down the convergence (by comparing $\rho$ in Theorems 1 and 2) as $\epsilon$ increases.}}

{\rem{Assumptions of Theorem \ref{th:inexactLS2} impose a stringent requirement on the scaling of the approximation parameter i.e. $\epsilon=O\left(\sqrt{\frac{\mmx}{\vertiii{A}}}\right)$ which is not purely dependant to the model restricted embedding conditioning but also the spectral norm of $\vertiii{A}$. In this sense since  $\vertiii{A}$ ignores the structure $\Cc$ of the problem it might scale very differently as compared to the corresponding embedding constants $\mmx,\mm$ and $\MM$. For instance an $m\times n$  i.i.d. Gaussian matrix have with high probability $\vertiii{A}=\Theta(n)$ (when $m\ll n$) whereas, e.g. for $s$-sparse signals, its embedding constants $\mm,\MM$ w.h.p. scale as $O(m)$. A similar gap exists for other low dimensional signal models and for other compressed sensing matrices e.g. random orthoprojectors, 
%for$(s\ll m)$ sufficiently large $m$ hav This is the case for e.g. orthoprojectors and subgaussians matrices often 
which is unfavourable for applying large approximation of this type to the extreme compressed sensing settings where $\sqrt{\frac{\mmx}{\vertiii{A}}}\approx \sqrt{\frac{m}{n}}\rightarrow 0$. 
}}

In the following we show by a deterministic example that this requirement is indeed tight. We also observe empirically in Section \ref{sec:expe} that such limitation indeed holds in randomized settings (e.g. i.i.d. Gaussian $A$) and on average. 

\subsubsection*{A converse example}
Consider a noiseless CS recovery problem where $n =2$, $m=1$ and the sampling matrix (i.e. here a row vector) is
\eq{A=[\cos(\gamma)\quad -\sin(\gamma)]}
for some parameter $0\leq\gamma\leq\pi/2$. Consider the following signal model which is a line segment across the first dimension:
\eq{
\Cc = \{x\in \RR^2: 0\leq x(1)\leq 1,\, x(2)=0\}.
}
We have indeed $\vertiii{A}=1$. It is easy to verify that both of the embedding constants w.r.t. to $\Cc$ are
\eq{ \mmx=\MM = \cos(\gamma)^2.}
Therefore one can tune $\gamma\rightarrow \pi/2$ to obtain arbitrary small ratios for $\sqrt{\frac{\mmx}{\vertiii{A}}}=\cos(\gamma)$. 
%$\vertiii{A}$ and $\mmx$ the 

Assume the true signal and the initialization point are at the either ends of $\Cc$ i.e.
\eq{
x^\gt=[1 \quad 0]^T,\quad y=Ax^\gt=\cos(\beta) \qandq x^0=[0 \quad 0]^T.
}
%and the initialization point $x^0=(0,0)^T$. 
Consider an adversarial $(1+\epsilon)$-approximate projection oracle which performs the following step for any given $x\in \RR^2$: \marginpar{\todo{verify + this part modified $\epsilon$}}
\eq{
		\pp_\Cc^{\epsilon}(x) := [x(1)+\epsilon x(2) \quad 0]^T.
	}
For simplicity we assume no errors on the gradient step. 	
By setting $\mu=1/\cos(\gamma)^2$, the corresponding inexact IPG updates as 
\eq{	
	x^k(1)= \epsilon\tan(\gamma) \left(x^{k-1}(1)-1\right) + 1 
}
and only along the first dimension (we note that due to the choice of oracle $x^k(2)=0, \forall k$). Therefore
\eq{	
	x^k(1)= \left(\epsilon\tan(\gamma)\right)^k+1.
}
which requires $\epsilon<\tan^{-1}(\gamma)=O(\cos(\gamma))$ for convergence, and it diverges otherwise. As we can see for $\gamma\rightarrow \pi/2$  (i.e. where $A$ becomes extremely unstable w.r.t. sampling the first dimension) the range of admissible $\epsilon$ shrinks, irrespective of the fact that the restricted embedding $\mmx=\MM$ exhibits an perfect isometry; which is an ideal situation for solving a noiseless CS with the exact IPG (i.e. in this case IPG takes only one iteration to converge).




\section{Proofs from \secref{sec:qkd}}
\label{app:proofs}

In \secref{sec:qkd} we show how to define the security of QKD in a
composable framework and relate this to the trace distance security
criterion introduced in \textcite{Ren05}. This composable treatment of
the security of QKD follows the literature \cite{BHLMO05,MR09}, and
the results presented in \secref{sec:qkd} may be found in
\textcite{BHLMO05,MR09} as well. The formulation of the statements
differs however from those works, since we use here the Abstract
Cryptography framework of \textcite{MR11}. So for completeness, we
provide here proofs of the main results from \secref{sec:qkd}.

\begin{proof}[Proof of \thmref{thm:qkd}]
  Recall that in \secref{sec:security.simulator} we fixed the
  simulator and show that to satisfy \eqnref{eq:qkd.security} it is
  sufficient for \eqnref{eq:qkd.security.2} to hold. Here, we will
  break \eqnref{eq:qkd.security.2} into security [\eqnref{eq:qkd.cor}]
  and correctness [\eqnref{eq:qkd.sec}], thus proving the theorem.

  Let us define $\gamma_{ABE}$ to be a state obtained from
  $\rho^{\top}_{ABE}$ [\eqnref{eq:qkd.security.tmp}] by throwing away
  the $B$ system and replacing it with a copy of $A$, i.e., \[
  \gamma_{ABE} = \frac{1}{1-p^\bot} \sum_{k_A,k_B \in \cK} p_{k_A,k_B}
  \proj{k_A,k_A} \otimes \rho^{k_A,k_B}_E.\] From the triangle
  inequality we get \begin{multline*} D(\rho^\top_{ABE},\tau_{AB} \otimes
  \rho^\top_{E}) \leq \\ D(\rho^\top_{ABE},\gamma_{ABE}) +
  D(\gamma_{ABE},\tau_{AB} \otimes \rho^\top_{E}) .\end{multline*}

Since in the states $\gamma_{ABE}$ and
$\tau_{AB} \otimes \rho^\top_{E}$ the $B$ system is a copy of the $A$
system, it does not modify the distance. Furthermore,
$\trace[B]{\gamma_{ABE}} =
\trace[B]{\rho^{\top}_{ABE}}$. Hence
\[D(\gamma_{ABE},\tau_{AB} \otimes \rho^\top_{E}) =
  D(\gamma_{AE},\tau_{A} \otimes \rho^\top_{E}) =
  D(\rho^\top_{AE},\tau_{A} \otimes \rho^\top_{E}).\]

For the other term note that
\begin{align*}
  & D(\rho^\top_{ABE},\gamma_{ABE}) \\
  & \qquad \leq \sum_{k_A,k_B} \frac{p_{k_A,k_B}}{1-p^{\bot}}
    D\left(\proj{k_A,k_B} \otimes \rho^{k_A,k_B}_E,\right. \\
  & \qquad \qquad \qquad \qquad \qquad \qquad \left.\proj{k_A,k_A} \otimes \rho^{k_A,k_B}_E \right)\\
  & \qquad = \sum_{k_A \neq k_B} \frac{p_{k_A,k_B}}{1-p^{\bot}} = \frac{1}{1-p^{\bot}}\Pr
  \left[ K_A \neq K_B \right].
\end{align*}
Putting the above together with \eqnref{eq:qkd.security.2}, we get
\begin{align*} & D(\rho_{ABE},\tilde{\rho}_{ABE}) \\
  & \qquad = (1-p^\bot)
  D(\rho^\top_{ABE},\tau_{AB} \otimes \rho^\bot_{E}) \\ & \qquad \leq \Pr
  \left[ K_A \neq K_B \right] + (1-p^\bot) D(\rho^\top_{AE},\tau_{A}
  \otimes \rho^\top_{E}). \qedhere \end{align*}
\end{proof}

\begin{proof}[Proof of \lemref{lem:robustness}]
  By construction, $\aK_\delta$ aborts with exactly the same
  probability as the real system. And because $\sigma^{\qkd}_E$
  simulates the real protocols, if we plug a converter $\pi_E$ in
  $\aK\sigma^{\qkd}_E$ which emulates the noisy channel $\aQ_q$ and
  blogs the output of the simulated authentic channel, then
  $\aK_\delta = \aK\sigma^{\qkd}_E\pi_E$. Also note that by
  construction we have
  $\aQ_q \| \aA' = \left(\aQ \| \aA\right) \pi_E$. Thus
  \begin{multline*} d\left( \pi_A^{\qkd}\pi_B^{\qkd}(\aQ_q \| \aA')
      ,\aK_\delta\right) \\ = d\left( \pi_A^{\qkd}\pi_B^{\qkd}\left(\aQ
        \| \aA\right) \pi_E , \aK\sigma^{\qkd}_E\pi_E\right). \end{multline*}

  Finally, because the converter $\pi_E$ on both the real and ideal
  systems can only decrease their distance (see
  \secref{sec:ac.systems}), the result follows.
\end{proof}


%%% Local Variables:
%%% TeX-master: "main.tex"
%%% End:

\section{Application in Data driven compressed sensing}
\label{sec:datadrivenCS}
Many CS reconstruction programs resort to signal models promoted by certain (semi) algebraic functions $h(x):\RR^n\rightarrow \RR_+\cup\{ 0\}$. For example we can have
\eq{\Cc := \big\{x\in \RR^n : h(x)\leq \zeta\big\},
	}
where $h(x)$ may be chosen as the $\ell_{0}$ or $\ell_{0-2}$ semi-norms or as the $\rank(x)$  which promotes sparse, group-sparse or low-rank (for matrix spaces) solutions, respectively. One might also replace those penalties with their corresponding  %algebraic and 
convex relaxations namely, the $\ell_1$, $\ell_{1-2}$ norms or the nuclear norm. 
%For $R_i(x),\tau_i$ and $i=1,...,b$, one can define a $b$-dimensional linear model. Also the constraint $R(x)\leq \tau$ may correspond to a parametric manifold. In this case the projection is often algebraic.  
%\todo{ask mike for this part}

\emph{Data driven} compressed sensing however corresponds to cases where in the absence of an algebraic physical model one resorts to collecting a large number of data samples in a dictionary and use it as a \emph{point cloud} model for CS reconstruction~\cite{RichCSreview}.  
%In this case the model is $\Cc = \bigcup_{i=1}^{d}\{\psi_i\}$ where $\psi_i\in\RR^n$s are atoms of a $n\times d$ dictionary $\Psi$. %\{\psi_i\}_{i=1}^{d}$
% of $d$ points (atoms) in $\RR^n$ the model becomes.  
Data driven CS finds numerous applications e.g. in Hyperspectral imagery \cite{TIPHSI}, Mass spectroscopy (MALDI imaging) \cite{Kobarg2014}, Raman Imaging~\cite{ramanCS} and Magnetic Resonance Fingerprinting (MRF)~\cite{MRF,BLIPsiam} just to name a few. For instance the USGS Hyperspectral library \footnote{\url{http://speclab.cr.usgs.gov}} contains the spectral signatures (reflectance) of thousands of substances measured across a few hundred frequency bands. This side information is shown to be useful for CS reconstruction and classification in both convex and nonconvex settings (see e.g. \cite{TIPHSI} for more details and relations to sparse approximation in redundant dictionaries).  
%Recently MR Fingerprinting~\cite{MRF} proposed a fast CS acquisition scheme for quantifying the NMR properties (such as the $T1,T2$ relaxation times, off-resonance frequencies, proton density) of tissues. Since random excitation sequences generate magnetization responses in non-analytic forms, \cite{MRF} proposed build a very large dictionary of fingerprints driven by the Bloch dynamic equations for all combination of those parameters. 
Data driven CS may also apply to algebraic models with non trivial projections. For example in the MRF reconstruction problem one first constructs a huge  dictionary of fingerprints i.e. the magnetization responses (across the readout times) for many $T1,T2$ relaxation values (i.e. spin-spin and spin-echo) presented in normal tissues~\cite{MRF}. This corresponds to sampling a two-dimensional manifold associated with the solutions of the \emph{Bloch dynamic  equations}~\cite{BLIPsiam}, which in the MRF settings neither the response nor the projection has an analytic closed-form solution.
%the set $\Cc$ may also be a dense collection of samples from a continuous algebraic manifold with a non trivial projection e.g. Magnetic resonance fingerprinting \nref. 
% As the projection step might not be trivial or computationally easy for many complex manifolds we consider a data driven approach which consists of collecting discrete samples of the manifold in a dictionary and approximate 


 

%Recently MR Fingerprinting \cite{MRF} proposed a fast CS acquisition scheme for quantifying the NMR properties (namely the $T1,T2$ relaxation times) of tissues. Small number of  excitations in form of rotating the magnetic field applies to the tissue, and between each two excitations, the \emph{partial} k-space information is measured. An iterative exact projection algorithm (BLIP) is proposed for the MRF problem which achieves a great parameter estimation accuracy \cite{BLIPsiam}. % versus compression trade-off $\nref$.
%The projection consists of NN searches on a densely sampled manifold $\Mb$ of fingerprints driven by the Bloch dynamic equations.
\subsection{A data driven CS in product space}
To explore how our theoretical results can be used to accelerate CS reconstruction we consider a stylized data driven application for which we explain how one can obtain each of the aforementioned approximate projections. Consider a multi-dimensional image such as HSI, MALDI or MRF that can be represented by a $\n\times J$ matrix $X$, where $n=\n J$ is the total number of spatio-spectral pixels, $J$ is the spatial resolution and $\n$ is the number of spectral bands 
e.g. $\n=3$ for an RGB image, $\n\approx 400$ for an HSI acquired by NASA's AVIRIS spectrometer, $\n\approx 5000$ for a MALDI image~\cite{Kobarg2014}.
 In the simplest form we assume that 
%e.g., when the acquisition resolution is high enough so that the spectral contents do not merge spatially, one has pixel purity and thus 
each spatial pixel corresponds to a certain material with a specific signature, i.e. 
\eq{
	X_{j}\in \widetilde \Cc, \quad \forall j=1,\ldots,J,
} 
where $X_j$ denotes the $j$th column of $X$ and 
\eq{
	\widetilde \Cc:=\bigcup_{i=1}^{d}\{\psi_i\} \in \RR^J}
is the point cloud of a large number $d$ of signatures $\psi_i$ e.g. in a customized spectral library for HSI or MALDI data.

The CS sampling model follows \eqref{eq:CSsampling} by setting $x^*:=X_\text{vec}$, where by $X_\text{vec}\in \RR^n$ we denote the vector-rearranged form of the matrix $X$.
%\eq{y\approx AX^*(:),}  
%where in this set up the number of A's columns is $n=dN$. 
The CS reconstruction reads
\eql{\label{eq:datadrivenCS}
	\min_{\substack{X_j\in \widetilde\Cc,\\ \forall j=1,...,J}} \big\{f(x):= \frac{1}{2}\norm{y-AX_\text{vec}}^2\big\}.
}
or equivalently and similar to problem \eqref{eq:p1} 
\eq{
	\min_{x\in \prod_{j=1}^J \widetilde\Cc} \big\{f(x):= \frac{1}{2}\norm{y-Ax}^2\big\}.
}
The only update w.r.t. problem \eqref{eq:p1} 
%are the dimensionality of $A,x$ and 
is the fact that now the solution(s) $x$ lives in a product space of the same model i.e. 
\eql{\label{eq:prod}\Cc:=\prod_{j=1}^J \widetilde\Cc
} 
(see also~\cite{kronCS} on product/Kronecker space CS however using a sparsity inducing semi-algebraic model). We note that solving directly this problem for a general $A$ (e.g. sampling models which non trivially combine columns/spatial pixels of $X$)  is \emph{exponentially hard} $O(d^J)$ because of the combinatorial nature of the product space constraints. In this regard, a tractable scheme which has been frequently considered for this problem e.g. in \cite{BLIPsiam} would be the application of an IPG type algorithm 
%(or its proximal-gradient equivalences e.g. studied in \cite{TIPHSI,BLIPsiam}) 
in order to break down the cost into the gradient and projection computations (here the projection requires $O(Jd)$ computations to search the closest signatures  to the current solution) to locally solve the problem at each iteration. 
%where at each iteration one requires to perform $O(Jd)$ computations to locally solve/update the problem. 


%However this linear complexity in $d$ can still be a serious bottleneck for solving problems with very large size datasets. We address this issue by replacing the exhaustive search with fast approximate searches.


\subsection{Measurement bound}
The classic Johnson-Lindenstrauss lemma says that one can use random linear transforms to stably embed point clouds into a lower dimension of size $O(\log(\#\widetilde \Cc))$ where $\#$ stands for the set cardinality~\cite{JL}.
{\thm{\label{th:JL} Let $\widetilde \Cc$ be a finite set of points in $\RR^{\n}$. For $A$ drawn at random from the i.i.d. normal distribution and a positive constant $\widetilde \theta$, with very high probability one has
		\eq{(1-\widetilde \theta)\norm{(x-x')} \leq \norm{A(x-x')}\leq (1+\widetilde \theta)\norm{(x-x')}, \quad \forall x,x' \in \widetilde \Cc
			}
			provided $m=O(\log(\#\widetilde \Cc)/\widetilde \theta^2)$.
		}}
		
Note that this definition implies an RIP embedding for the point cloud model according to \eqref{eq:RIP} with a constant $\theta<3\widetilde \theta$ which in turn (and for small enough $\widetilde \theta$) implies the sufficient embedding condition for a stable CS recovery using the exact or approximate IPG. % (see for more details Section~\ref{sec:main}). 
This bound considers an arbitrary point cloud and could be improved when data points $\widetilde \Cc\subseteq \Mm$ are derived from a low-dimensional structure, e.g. a smooth manifold $\Mm$ such as the MR Fingerprints, for which one can alternatively use $m=\dim_{E}(\Mm)$ for the corresponding RIP type \emph{embedding dimension} listed e.g. in \cite{RichCSreview}.

%$m \propto \dim_{E}(\widetilde \Cc)\leq \n$ where $\dim_{E}(.)$ denotes the embedding dimension to satisfy RIP type embedding \eqref{eq:RIP}.

% e.g. points that are living on a smooth manifold 
%A more recent result \cite{Indyk:NNembedding} also considers the low dimensional structures within data points and weakened the measurement requirement to $m \propto \dim(\widetilde \Cc)\leq \n$ under certain conditions and notions of data intrinsic dimensionality. \todo{add embedding dims papers eg richs overview}

%This $\theta$-stable embedding has similarities with the bi-Liptchitz embedding that we discussed in section x that 
%
%defined above, also known as the Restricted Isometry Property (RIP), is the key to guaranty the success of many CS recovery programs. For certain infinite models such as sparse (or in general the union of linear models), low-rank or smooth manifold signals the measurement bound of JL lemma can be further milden to $m \propto \dim(\Cc)$, where $\dim$ denotes the intrinsic dimensionality of the corresponding model e.g. sparsity level, rank of a matrix or manifold (intrinsic) dimension \nref [classiCS]. 
%In relation to Definition \eqref{def:Lip}, we note that RIP is a uniform (stronger) guaranty whereas ours is local on RCS i.e.  we have $1-3\delta\leq \mmx$ and w.r.t. RLS it holds $L=1+3\delta$.
		
We note that such a measurement bound for a product space model  \eqref{eq:prod}  without considering any structure between spaces turns into 
\eq{
	m=O\left(J\min\left\{\log(d), \dim_E(\Mm)\right\}\right).
	}




\subsection{Cover tree for fast nearest neighbour search}
\label{sec:covertree}
With the data driven CS formalism and discretization of the model  %$\Cc = \prod_{i=1}^N \Mb$, 
the projection step of IPG reduces to searching for the nearest signature in each of the product spaces, however in a potentially very large $d$ size dictionary. %Therefore the linear complexity of an exact projection in $d$ can still be a bottneck  an extract IPG %which brings a serious bottleneck to searches with a linear complexity in term of $d$.  
And thus search strategies with linear complexity in $d$ e.g. an exhaustive search, can be a serious bottleneck for solving such problems. %We address this issue by replacing the exhaustive search with fast approximate searches. 
A very well-established approach to overcome the complexity of an exhaustive nearest neighbour (NN) search on a large dataset consists of hierarchically partitioning the solution space and forming a \emph{tree} whose nodes represents those partitions, and then using branch-and-bound methods on the resulting tree for a fast Approximate NN (ANN) search with $o(d)$ complexity e.g. see~\cite{Navigating,beygelzimer2006cover}.

In this regard, we address the computational shortcoming of the projection step in the exact IPG by preprocessing $\widetilde \Cc$ and form a \emph{cover tree} structure suitable for fast ANN searches \cite{beygelzimer2006cover}. 
A cover tree is a levelled tree whose nodes at different scales form covering nets for data points at multiple resolutions; 
%(i.e., coarse-to-fine dyadic coverage levels). 
if $\sigma:=\max_{\psi\in\widetilde \Cc}\norm{\psi_\text{root}-\psi}$ corresponds to the maximal coverage by the root, then  
nodes appearing at any finer scale $l>0$ form a $(\sigma 2^{-l})$-covering net for their descendants i.e. as we descend down the tree the covering resolution refines in a dyadic coarse-to-fine fashion.

 %Two approximate search strategies can be adopted with respect to the cover tree structure:
 We consider three possible search strategies 
 %(one exact and two approximate) 
 using such a tree structure:
 \begin{itemize}
 	\item \textbf{Exact NN:} which is based on the branch-and-bound algorithm proposed in \cite[Section 3.2]{beygelzimer2006cover}.	Note that we should distinguish between this strategy and performing a brute force search. Although they both perform an exact NN search, the complexity of the proposed algorithm in~\cite{beygelzimer2006cover} is shown to be way less in practical datasets.% (thanks to pre-processing a cover tree structure).
 	\item \textbf{$(1+\epsilon)$-ANN:} this search is also based on the branch-and-bound algorithm proposed in \cite[Section 3.2]{beygelzimer2006cover} which has includes an early search termination criteria (depending on the accuracy level $\epsilon$) 
 	%	that if for a parameter $\gamma=\epsilon$ the search stops as soon as having
 	%	\eq{(1+\frac{1}{\epsilon-1})\sigma2^{-i+1}\leq \argmin_{q\in Q_i} \norm{p-q}
 	%	}
 	%	then 
 	for which one obtains an approximate  oracle of type defined by \eqref{eq:eproj}. %that could be used for solving data driven CS problems \eqref{eq:datadrivenCS} with an inexact IPG.  
 	Note that the case $\epsilon=0$ refers to the exact tree NN search described above. %and performing a brute-force search. Although they both perform an exact NN search, the complexity of the proposed algorithm in~\cite{beygelzimer2006cover} is shown to be way less in practical datasets.
 	
 	\item \textbf{FP-ANN:} that is traversing down the tree up to a scale $l=\lceil \log(\frac{\nup}{\sigma})\rceil$ for which the covering resolution falls below a threshold $\nup$ on the search accuracy. This search results in a fixed precision type approximate oracle as described in Section~\ref{sec:FP} and in a sense it is similar to performing the former search with $\epsilon=0$, however on a truncated (low-resolution) cover tree. 
 	%	 Given the knowledge of $maxdist$ (or the bound in Lemma \ref{lem:maxdist}) we stop the search when
 	%	%the stoppage can be refined to 
 	%	\eq{ \max_{q\in Q_i} \{maxdist(q)\} < \gamma. }
 	%	This is similar as performing exact NN searches on a truncated (low-resolution) cover tree. 
 	%	\item \textbf{PFP ANN decaying at rate $r$:} as defined in Corollary \ref{cor:decay}. For this case we perform the stoppage procedure above for an updating accuracy parameter 
 	%	\eq{ 
 	%		\gamma := \nup^k =r^k
 	%	}
 	%	that geometrically decays at a certain rate $r<1$ through the iterations.
 	
 \end{itemize}
 All strategies could be applied to accelerate the projection step of an exact or inexact IPG (with variations discussed in Sections~\ref{sec:epsproj} and \ref{sec:FP}) to tackle the data driven CS problem~\eqref{eq:datadrivenCS}.  
 %with the inexact IPG variations discussed in Sections~\ref{sec:epsproj} and \ref{sec:FP}.  
 In addition one can iteratively refine the accuracy of the FP-ANN search (e.g. $\nup^k =r^k$ for a certain decay rate $r<1$ and IPG iteration number $k$) and obtain a PFP type approximate IPG discussed in Section~\ref{sec:PFP}.  
 
Note that while the cover tree construction is blind to the explicit structure of the data, several key growth properties such as the tree's explicit depth, the number of children per node, and importantly the overall search complexity are characterized by the intrinsic dimension of the model, called the \emph{doubling dimension},
% i.e.  a notion which is referred to as the \emph{doubling constant} or \emph{doubling dimension}
and defined as follows~\cite{assouad,heinonen}:

{\defn{\label{def:doub} Let $B(q,r)$ denotes a ball of radius $r$ centred at a point $q$ in some metric space. The doubling dimension $\dim_D(\Mm)$ of a set $\Mm$ is the smallest integer such that every ball of $\Mm$ (i.e. $\forall r>0$, $\forall q\in\Mm$, $B(q,2r)\cap \Mm$) can be covered by $2^{\dim_D(\Mm)}$  balls of half radius i.e. $B(q',r)\cap \Mm$, $q'\in \Mm$. %Where $B(q,r)$ denotes a ball of radius $r$ centred at $q$ in some metric space. 
		}}
		
The doubling dimension has several 
appealing properties e.g. $\dim_D(\RR^n)=\Theta(n)$, $\dim_D(\Mm_1)\leq \dim_D(\Mm_2)$ when $\Mm_1$ is a subspace of $\Mm_2$, and $\dim(\cup_{i=1}^I \Mm_i)\leq \max_i \dim_D(\Mm_i)+\log(I)$~\cite{heinonen,Navigating}. 
Practical datasets are often assumed to have small doubling dimensions e.g. when $\widetilde \Cc \subseteq \Mm$  %uniformly
 samples a low $K$-dimensional manifold $\Mm$ with certain smoothness and regularity one has $\dim_D(\widetilde \Cc)\leq \dim_D(\Mm)=O(K)$ \cite{dasgupta2008}.\footnote{Although the two notions of embedding $\dim_E$ and doubling $\dim_D$ dimensions scale similarly for certain sets e.g. linear subspaces, UoS, smooth manifolds..., this does not generally hold in $\RR^n$ and one needs to distinguish between them,  
%these two notions are different 
see for more discussions~\cite{Indyk:NNembedding,dasgupta2012}.}

Equipped with such a notion of dimensionality, the following theorem bounds the complexity of a  $(1+\epsilon)$-ANN cover tree search~\cite{Navigating,beygelzimer2006cover}: 

{\thm{\label{thm:NNcomp2}Given a query which might not belong to $\widetilde \Cc$, the approximate $(1+\epsilon)$-ANN search on a cover tree takes at most 
		\eql{
			2^{O(\dim_D(\widetilde \Cc))}\log \Delta+(1/\epsilon)^{O(\dim_D(\widetilde \Cc))} 
		}
		computations in time with $O(\#\widetilde \Cc)$ memory requirement, where  $\Delta$ is the aspect ratio of $\widetilde \Cc$.}	
}

For most applications $\log (\Delta) = O(\log(d))$~\cite{Navigating} and thus for datasets with low dimensional structures i.e. $\dim_D=O(1)$ and by using  moderate approximations one achieves a logarithmic search complexity in $d$, as opposed to the linear complexity of a brute force search.

Note that the complexity of an \emph{exact} cover tree search could be arbitrarily high 
%i.e. linear in $d$, at least in theory (unless the query belongs to the dataset~\cite{beygelzimer2006cover} which does not generally apply e.g. for the intermediate steps of the IPG). 
and thus the same applies to the FP and PFP type ANN searches since they are also based on performing exact NN (on a truncated tree). However in the next section we empirically observe  that the complexity of an exact cover tree NN (and also the FP and PFP type ANN) is much lower than performing an exhaustive search.

%(i.e. by a notion which is referred to as the \emph{doubling constant}~\cite{beygelzimer2006cover,Navigating,Indyk:NNembedding})  
%which in practical datasets is often a small number e.g. when $\widetilde \Cc$ uniformly samples of low dimensional manifolds.


%This structure hierarchically partitions the metric space and enables using branch-and-bound methods for fast NN search. 
%We denote by $\widetilde \Cc_i$ the set of nodes appear at scale $i=0,\dots,L_{\max}$.







%
%
%Let $\Tt_\Mb$ denotes a cover tree defined in order to arrange dataset $\Mb$.  %Cover tree is a leveled tree with multiple scales $i=0,\dots,L_{\max}$ where its nodes are associated with data points in  $\Mb$. 
%We denote by $\widetilde \Cc_i$ the set of nodes appear at scale $i=0,\dots,L_{\max}$.
%The lowest scale $i=0$ correspond to the root $\widetilde \Cc_0$ which is a point covering all data. We denote by $\sigma:=\max_{q\in\Mb}\dist(\widetilde \Cc_0,q)$ the maximum coverage i.e. the maximum distance between the root and any other node on the tree. As we descend down the tree (i.e. incrementing the scale) the nodes present at higher scales cover their descendants at finer resolutions. More precisely, a cover tree structure must have the following  three properties:
%% node $i$ denoted by $\Nn_{ji}$ represents a section of the space denoted by$\widetilde \Cc_{ji}$. Nodes and represented partitions must satisfy the following three properties.
%\begin{enumerate}
%	\item Nesting: $\widetilde \Cc_i \subseteq \widetilde \Cc_{i+1}$, 
%	that is once a point $p$ appears as a node in $\widetilde \Cc_i$, then every lower level in the tree has a node associated with $p$.
%	\item Covering: every node $q\in \widetilde \Cc_{i+1}$ has a parent node  $p\in \widetilde \Cc_{i}$, where $\dist(p,q)\leq \sigma2^{-i} $. As a result, covering becomes finer at higher scales in a dyadic fashion. 
%	%and For every $p\in \widetilde \Cc_{i−1}$, there exists
%	%a q 2 Ci such that d(p, q) < 2i and the node in
%	%level i associated with q is a parent of the node
%	%in level i − 1 associated with p.
%	
%	\item Separation: nodes belonging to the same scale are separated by a minimal distance which dyadically shrinks at higher scales i.e. $\forall q,q'\in\widetilde \Cc_i$ we have $\dist(q,q')>\sigma2^{-i}$.   
%\end{enumerate}  
%Each node $p$ also keeps the maximum distance to its descendants denoted by
%\eq{maxdist(q):= \max_{q'\in \text{descendant}(q)} \dist(q,q'),
%}
%which will be useful for the fast NN search.  Note that one might decide to avoid saving $maxdist$ values and use the following upper bound instead.
%
%
%{\lem{\label{lem:maxdist}For any $q\in \widetilde \Cc_i$ and due to the covering property we have:
%\begin{align}\label{eq:maxdist}
%maxdist(q) \leq \sigma \left( 2^{-i}+2^{-i-1}	+ 2^{-i-2}+\dots \right)
%< \sigma 2^{-i+1}.
%\end{align}
%}}
%
%Depth of the \emph{implicit} cover tree (constructed w.r.t. the three constraints above) might grow very large for arbitrary datasets. Indeed we can easily verify that $L_{\max}\leq \log(\Delta(\Mb))$, where 
%\eq{
%	\Delta(\Mb):= \frac{\max_{p,q \in \Mb}\dist(p,q)}{\min_{p\neq q \in \Mb}\dist(p,q)}.
%}
%is the aspect ratio of $\Mb$. In practice however we only keep one copy of the nodes which do not have either parent or a child other than themselves. This \emph{explicit} representation efficiently reduces the required storage space to $O(n)$. 
%
%\begin{algorithm}[t]
%	\label{alg:findNN}
%	\SetAlgoLined
%	$Q_0 = \{\widetilde \Cc_0\}$, where $\widetilde \Cc_0$ is the root of $\Tt_{\Mb}$\\
%	$d_{\min}=\norm{p-q_c}$\\
%	$i=0$\\
%	\While{$i<L_{\max}$ \& $! \textbf{stoppage}(\gamma)$}
%	{
%		$Q=\left\{\text{children}(q):\, q\in Q_i \right\} $\\
%		$q^* = \argmin_{q\in Q} \norm{p-q}$, \quad $d = \norm{p-q^*}$ \\
%		\If{$d<d_{\min}$}{$d_{\min}=d, \quad q_c=q^*$}
%		$Q_{i+1} = \left\{q\in Q: \norm{p-q}\leq d_{\min} + maxdist(q) \right\}$\\
%		$i  = i+1$\\
%	}	
%	\Return $q_c$\\%=\argmin_{q\in Q_{L_{\max}}} \norm{p-q}$\;
%	\caption{\label{alg:NN} \textbf{ANN}(cover tree $\Tt_\Mb$, query point $p$, current estimate $q_c \in \Mb$, accuracy parameter $\gamma$)}
%\end{algorithm}
%
%Algorithm \ref{alg:NN} details the procedure for approximate nearest neighbour (ANN) search on a given cover tree. 
%In short, we iteratively traverse down the cover tree and at each scale we populate the set of candidates $Q_i$ with nodes that can be ancestors of the closest NN point and discard others (this refinement uses the triangular inequality and the lower bound on the distance of the grandchildren of $Q$ to $p$, based on  $maxdist(q)$). In the next iteration, children of these candidates are similarly refined and added to the updated candidate set $Q_{i+1}$. At the finest scale (before stoppage) we search the whole set of final candidates and report an ANN point. Not that at each scale we only compute distances for non self-parent nodes (we pass, without computation, distance information of the self-parent children to finer scales).



\section{Numerical experiments}
\label{sec:expe}

We test the performance of the exact/inexact IPG algorithm for our product-space data driven CS reconstruction using the four datasets described in Table \ref{tab:data}. The datasets are uniformly sampled (populated) from 2-dimensional continuous manifolds embedded in a higher ambient dimension, see also  Figure~\ref{fig:datasets}\footnote{The S-manifold, Swiss roll and Oscillating wave are synthetic machine learning  datasets available e.g. in \cite{GMRA12}. The Magnetic Resonance Fingerprints (MRF) is generated by solving the Bloch dynamic equation for a uniform grid of relaxation times $T1,T2$ and for an external magnetic excitation pattern, discussed and implemented in~\cite{MRF}.}. 

To proceed with fast ANN searches within IPG, we separately build a cover tree structure per dataset i.e. a preprocessing step. As illustrated for the MRF manifold in Figure~\ref{fig:CT} the coverage levels 
%(highlighted in colours for segments associated with tree nodes at certain scale) 
decrease in a coarse-to-fine manner as we traverse down the tree i.e. increasing the scale.

%============TABLE DATA=========================
\ifCLASSOPTIONtwocolumn
\begin{table}[t!]
	\centering
	\scalebox{.91}{
	\begin{tabular}{ccccc}
		%\hline
		\toprule[.2em]
		Dataset & Population ($d$) & Ambient dim. ($\n$)&CT depth&CT res.\\
		\midrule[.1em]
		S-Manifold & 5000 & 200& 14&2.43E-4 \\
		%\hline
		Swiss roll & 5000 & 200 &14&1.70E-4\\
		%\hline
		Oscillating wave & 5000 & 200 &14&1.86E-4\\
		%\hline
		MR Fingerprints & 29760 & 512& 13&3.44E-4\\
		%\hline
		\bottomrule[.2em]
	\end{tabular}}
	\caption{Datasets for data-driven CS evaluations; a cover tree (CT) structure is formed for each dataset. The last two columns respectively report the number of scales and the finest covering resolution of each tree.  }
	\label{tab:data}	
\end{table}
\else
\begin{table}[t!]
	\centering
	\scalebox{1.1}{
		\begin{tabular}{ccccc}
			%\hline
			\toprule[.2em]
			Dataset & Population ($d$) & Ambient dim. ($\n$)&CT depth&CT res.\\
			\midrule[.1em]
			S-Manifold & 5000 & 200& 14&2.43E-4 \\
			%\hline
			Swiss roll & 5000 & 200 &14&1.70E-4\\
			%\hline
			Oscillating wave & 5000 & 200 &14&1.86E-4\\
			%\hline
			MR Fingerprints & 29760 & 512& 13&3.44E-4\\
			%\hline
			\bottomrule[.2em]
		\end{tabular}}
		\caption{Datasets for data-driven CS evaluations; a cover tree (CT) structure is formed for each dataset. The last two columns respectively report the number of scales and the finest covering resolution of each tree.  }
		\label{tab:data}	
	\end{table}
	\fi
%========== Manifolds illustrations ===========================
\ifCLASSOPTIONtwocolumn
\begin{figure}[t!]
	\centering
	\begin{minipage}{\linewidth}
		\centering
		\subfloat[S-Manifold]{\includegraphics[width=.46\textwidth]{dict_1_3manifold.png} }	
		\quad	
		\subfloat[Swiss roll]{\includegraphics[width=.46\textwidth]{dict_2_3manifold.png} }	
		\quad	
		\subfloat[Oscillating wave]{\includegraphics[width=.46\textwidth]{dict_3_3manifold.png} }	
		\quad	
		\subfloat[MR Fingerprints]{\includegraphics[width=.46\textwidth]{dict_4_2manifold.png} }	
	\caption{Illustration of the low dimensional structures of  datasets presented in Table~\ref{tab:data}. Points are depicted along the first three principal components of each dataset.\label{fig:datasets}}
\end{minipage}
\end{figure}
\else
\begin{figure}[t!]
	\centering
	\begin{minipage}{\linewidth}
		\centering
		\subfloat[S-Manifold]{\includegraphics[width=.35\textwidth]{dict_1_3manifold.png} }	
		\quad	
		\subfloat[Swiss roll]{\includegraphics[width=.35\textwidth]{dict_2_3manifold.png} }	
		\quad	
		\subfloat[Oscillating wave]{\includegraphics[width=.35\textwidth]{dict_3_3manifold.png} }	
		\quad	
		\subfloat[MR Fingerprints]{\includegraphics[width=.35\textwidth]{dict_4_2manifold.png} }	
		\caption{Illustration of the low dimensional structures of  datasets presented in Table~\ref{tab:data}. Points are depicted along the first three principal components of each dataset.\label{fig:datasets}}
	\end{minipage}
\end{figure}
\fi
		
		
%----------------CT levels-------------
\ifCLASSOPTIONtwocolumn
\begin{figure}[t!]
\centering
\begin{minipage}{\linewidth}
		\subfloat[Scale 2]{\includegraphics[width=.46\textwidth]{dict_4_2CT_scale_2.png} }	
		\quad	
		\subfloat[Scale 3]{\includegraphics[width=.46\textwidth]{dict_4_2CT_scale_3.png} }	
		\\
		\subfloat[Scale 4]{\includegraphics[width=.46\textwidth]{dict_4_2CT_scale_4.png} }	
		\quad
		\subfloat[Scale 5]{\includegraphics[width=.46\textwidth]{dict_4_2CT_scale_5.png} }		
		\caption{A cover tree is built on MR Fingerprints dataset: (a-d) data partitions i.e. descendants covered with parent nodes appearing at scales 2-5
			are highlighted in different colours. The coverage resolution refines 
			%Low scale partitions divide into finer segments 
			by increasing the scale.\label{fig:CT}}
\end{minipage}
\end{figure}
\else
\begin{figure}[t!]
	\centering
	\begin{minipage}{\linewidth}
		\centering
		\subfloat[Scale 2]{\includegraphics[width=.35\textwidth]{dict_4_2CT_scale_2.png} }	
		\quad	
		\subfloat[Scale 3]{\includegraphics[width=.35\textwidth]{dict_4_2CT_scale_3.png} }	
		\\
		\subfloat[Scale 4]{\includegraphics[width=.35\textwidth]{dict_4_2CT_scale_4.png} }	
		\quad
		\subfloat[Scale 5]{\includegraphics[width=.35\textwidth]{dict_4_2CT_scale_5.png} }		
		\caption{A cover tree is built on MR Fingerprints dataset: (a-d) data partitions i.e. descendants covered with parent nodes appearing at scales 2-5
			are highlighted in different colours. The coverage resolution refines 
			%Low scale partitions divide into finer segments 
			by increasing the scale.\label{fig:CT}}
	\end{minipage}
\end{figure}
\fi

%-------wide

%\begin{figure*}[t!]
%	\centering
%	\begin{minipage}{\textwidth}
%		\centering
%		\subfloat[Scale 2]{\includegraphics[width=.22\textwidth]{./figs/manifolds/dict_4_2CT_scale_2.png} }	
%		\quad	
%		\subfloat[Scale 3]{\includegraphics[width=.22\textwidth]{./figs/manifolds/dict_4_2CT_scale_3.png} }	
%		\quad
%		\subfloat[Scale 4]{\includegraphics[width=.22\textwidth]{./figs/manifolds/dict_4_2CT_scale_4.png} }	
%		\quad
%		\subfloat[Scale 5]{\includegraphics[width=.22\textwidth]{./figs/manifolds/dict_4_2CT_scale_5.png} }		
%		\caption{A cover tree is built on MR Fingerprints dataset: (a-d) data partitions i.e. descendants covered with parent nodes appearing at scales 2-5
%			are highlighted in different colours. The coverage resolution refines 
%			%Low scale partitions divide into finer segments 
%			by increasing the scale.\label{fig:CT}}
%	\end{minipage}
%\end{figure*}
%=================MSE vs. iter Decays===============
\ifCLASSOPTIONtwocolumn
\begin{figure*}[t]
	\centering
	\begin{minipage}{\textwidth}
		\centering
		\subfloat{\includegraphics[width=.225\textwidth]{Sol_iter_data_1_alg_4_compr_4} }	
		\quad	
		\subfloat{\includegraphics[width=.225\textwidth]{Sol_iter_data_2_alg_4_compr_4} }	
		\quad
		\subfloat{\includegraphics[width=.225\textwidth]{Sol_iter_data_3_alg_4_compr_4} }	
		\quad	
		\subfloat{\includegraphics[width=.225\textwidth]{Sol_iter_data_4_alg_4_compr_4} }	
		\quad
		\subfloat{\includegraphics[width=.225\textwidth]{Sol_iter_data_1_alg_3_compr_4} }	
		\quad	
		\subfloat{\includegraphics[width=.225\textwidth]{Sol_iter_data_2_alg_3_compr_4} }	
		\quad
		\subfloat{\includegraphics[width=.225\textwidth]{Sol_iter_data_3_alg_3_compr_4} }	
		\quad	
		\subfloat{\includegraphics[width=.225\textwidth]{Sol_iter_data_4_alg_3_compr_4} }	
		\quad
		\subfloat{\includegraphics[width=.225\textwidth]{Sol_iter_data_1_alg_2_compr_4} }	
		\quad	
		\subfloat{\includegraphics[width=.225\textwidth]{Sol_iter_data_2_alg_2_compr_4} }	
		\quad
		\subfloat{\includegraphics[width=.225\textwidth]{Sol_iter_data_3_alg_2_compr_4} }	
		\quad	
		\subfloat{\includegraphics[width=.225\textwidth]{Sol_iter_data_4_alg_2_compr_4} }	
		\caption{Convergence of the exact/inexact IPG for subsampling ratio $\frac{m}{n}=0.2$. 
			Rows from top to bottom correspond to inexact algorithms with FP, PFP and $1+\epsilon$ ANN searches, respectively (legends for the plots in each row are identical and included in the last column). Columns from left to right correspond to  S-Manifold, Swiss roll, Oscillating wave and MR Fingerprints datasets, respectively. \label{fig:Decays}}
	\end{minipage}
\end{figure*}
\else
\begin{figure*}[t]
	\centering
	\begin{minipage}{\textwidth}
		\centering
		\subfloat{\includegraphics[width=.3\textwidth]{Sol_iter_data_1_alg_4_compr_4} }	
		\quad
		\subfloat{\includegraphics[width=.3\textwidth]{Sol_iter_data_1_alg_3_compr_4} }	
		\quad
		\subfloat{\includegraphics[width=.3\textwidth]{Sol_iter_data_1_alg_2_compr_4} }
		\quad			
		\subfloat{\includegraphics[width=.3\textwidth]{Sol_iter_data_2_alg_4_compr_4} }	
		\quad	
		\subfloat{\includegraphics[width=.3\textwidth]{Sol_iter_data_2_alg_3_compr_4} }
		\quad	
		\subfloat{\includegraphics[width=.3\textwidth]{Sol_iter_data_2_alg_2_compr_4} }
		\quad
		\subfloat{\includegraphics[width=.3\textwidth]{Sol_iter_data_3_alg_4_compr_4} }	
		\quad
		\subfloat{\includegraphics[width=.3\textwidth]{Sol_iter_data_3_alg_3_compr_4} }
		\quad
		\subfloat{\includegraphics[width=.3\textwidth]{Sol_iter_data_3_alg_2_compr_4} }		
		\quad	
		\subfloat{\includegraphics[width=.3\textwidth]{Sol_iter_data_4_alg_4_compr_4} }		
		\quad	
		\subfloat{\includegraphics[width=.3\textwidth]{Sol_iter_data_4_alg_3_compr_4} }	
		\quad	
		\subfloat{\includegraphics[width=.3\textwidth]{Sol_iter_data_4_alg_2_compr_4} }	
		\caption{Convergence of the exact/inexact IPG for subsampling ratio $\frac{m}{n}=0.2$. 			 
			Columns from left to right correspond to inexact algorithms with FP, PFP and $1+\epsilon$ ANN searches, respectively (legends for the plots in each column are identical and included in the last row). Rows from top to bottom correspond to   S-Manifold, Swiss roll, Oscillating wave and MR Fingerprints datasets, respectively. \label{fig:Decays}}
	\end{minipage}
\end{figure*}
\fi

%================Phase Transitions=====================
\ifCLASSOPTIONtwocolumn
\begin{figure*}
	\centering
	\begin{minipage}{\textwidth}
		\centering
		\subfloat
		%[S-Manifold, $(1+\epsilon)$-IPG]
		{\includegraphics[width=.225\textwidth]{TITPTiter_data_1_alg_2} }
		\quad
		\subfloat
		%[Swiss roll,  $(1+\epsilon)$-IPG]
		{\includegraphics[width=.225\textwidth]{TITPTiter_data_2_alg_2} }
		\quad
		\subfloat
		%[Oscillating wave,  $(1+\epsilon)$-IPG]
		{\includegraphics[width=.225\textwidth]{TITPTiter_data_3_alg_2} }
		\quad
		\subfloat
		%[MR Fingerprints,  $(1+\epsilon)$-IPG]
		{\includegraphics[width=.225\textwidth]{TITPTiter_data_4_alg_2} }
		\quad
		\subfloat
		%[S-Manifold, PFP-IPG]
		{\includegraphics[width=.225\textwidth]{TITPTiter_data_1_alg_3} }
		\quad		
		\subfloat
		%[Swissroll, PFP-IPG]
		{\includegraphics[width=.225\textwidth]{TITPTiter_data_2_alg_3} }		
		\quad
		\subfloat
		%[Oscillating wave, PFP-IPG]
		{\includegraphics[width=.225\textwidth]{TITPTiter_data_3_alg_3} }
		\quad		
		\subfloat
		%[MR Fingerprints, PFP-IPG]
		{\includegraphics[width=.225\textwidth]{TITPTiter_data_4_alg_3} }
		
		
		\caption{Recovery phase transitions for IPG with approximate projection (i.e. ANN search). Image intensities correspond to the normalized solution MSE for a search parameter and a given subsampling ratio (ranging between 5-100$\%$).
			% and search parameter, and darker pixels indicate higher solution accuracies.
			 Intensities in all plots are identically set with a logarithmic scale: black pixels correspond to accurate points with MSE $\leq 10^{-6}$, white pixels represent points with MSE $\geq 1$, and the region below the red curve is defined as the exact recovery region with MSE $\leq10^{-4}$. 			
			The columns from left to right correspond to the phase transitions of S-Manifold, Swiss roll, Oscillating wave and MR Fingerprints datasets. The top and the bottom rows correspond to two cover tree based ANN searches namely, the $(1+\epsilon)$-ANN and the PFP-ANN with decay parameter $r$.   \label{fig:PT}}
	\end{minipage}
\end{figure*}
\else
\begin{figure*}
	\centering
	\begin{minipage}{\textwidth}
		\centering
		\subfloat
		%[S-Manifold, $(1+\epsilon)$-IPG]
		{\includegraphics[width=.35\textwidth]{TITPTiter_data_1_alg_2} }
		\quad
		\subfloat
		%[S-Manifold, PFP-IPG]
		{\includegraphics[width=.35\textwidth]{TITPTiter_data_1_alg_3} }
		\\
		\subfloat
		%[Swiss roll,  $(1+\epsilon)$-IPG]
		{\includegraphics[width=.35\textwidth]{TITPTiter_data_2_alg_2} }		
		\quad		
		\subfloat
		%[Swissroll, PFP-IPG]
		{\includegraphics[width=.35\textwidth]{TITPTiter_data_2_alg_3} }
		\\
		\subfloat
		%[Oscillating wave,  $(1+\epsilon)$-IPG]
		{\includegraphics[width=.35\textwidth]{TITPTiter_data_3_alg_2} }		
		\quad
		\subfloat
		%[Oscillating wave, PFP-IPG]
		{\includegraphics[width=.35\textwidth]{TITPTiter_data_3_alg_3} }
		\\
		\subfloat
		%[MR Fingerprints,  $(1+\epsilon)$-IPG]
		{\includegraphics[width=.35\textwidth]{TITPTiter_data_4_alg_2} }
		\quad		
		\subfloat
		%[MR Fingerprints, PFP-IPG]
		{\includegraphics[width=.35\textwidth]{TITPTiter_data_4_alg_3} }

		\caption{Recovery phase transitions for IPG with approximate projection (i.e. ANN search). Image intensities correspond to the normalized solution MSE for a search parameter and a given subsampling ratio (ranging between 5-100$\%$).
			% and search parameter, and darker pixels indicate higher solution accuracies.
			Intensities in all plots are identically set with a logarithmic scale: black pixels correspond to accurate points with MSE $\leq 10^{-6}$, white pixels represent points with MSE $\geq 1$, and the region below the red curve is defined as the exact recovery region with MSE $\leq10^{-4}$. 			
			Rows from top to bottom 
			 correspond to the phase transitions of S-Manifold, Swiss roll, Oscillating wave and MR Fingerprints datasets. The left and the right columns correspond to two cover tree based ANN searches namely, the $(1+\epsilon)$-ANN and the PFP-ANN with decay parameter $r$.   \label{fig:PT}}
	\end{minipage}
\end{figure*}
\fi

%\begin{figure*}
%	\centering
%	\vspace{-2cm}
%	\begin{minipage}{\textwidth}
%		\centering
%		\subfloat[S-Manifold, IPG with $(1+\epsilon)$-ANN  ]{\includegraphics[width=.4\textwidth]{./figs/PTiter_data_1_alg_2} }
%		\quad
%		\subfloat[S-Manifold, IPG with PFP $r$-ANN]{\includegraphics[width=.4\textwidth]{./figs/PTiter_data_1_alg_3} }
%		\quad
%		\subfloat[Swissroll, IPG with $(1+\epsilon)$-ANN]{\includegraphics[width=.4\textwidth]{./figs/PTiter_data_2_alg_2} }
%		\quad
%		\subfloat[Swissroll, IPG with PFP $r$-ANN]{\includegraphics[width=.4\textwidth]{./figs/PTiter_data_2_alg_3} }
%		
%		\subfloat[Oscillating wave, IPG with $(1+\epsilon)$-ANN]{\includegraphics[width=.4\textwidth]{./figs/PTiter_data_3_alg_2} }
%		\quad
%		\subfloat[Oscillating wave, IPG with PFP $r$-ANN]{\includegraphics[width=.4\textwidth]{./figs/PTiter_data_3_alg_3} }
%		\quad
%		\subfloat[MR Fingerprints, IPG with $(1+\epsilon)$-ANN]{\includegraphics[width=.4\textwidth]{./figs/PTiter_data_4_alg_2} }
%		\quad
%		\subfloat[MR Fingerprints, IPG with PFP $r$-ANN]{\includegraphics[width=.4\textwidth]{./figs/PTiter_data_4_alg_3} }
%		
%		
%		\caption{Average IPG iterations for four studied datasets, two covertree-based ANN algorithms ($(1+\epsilon)$-ANN and PFP $r$-ANN), and for different compression ratios. In each plot, region below the red curve corresponds to exact CS reconstruction. Intensities in all plots are identically set: black pixels correspond to points with $\leq 4$ iterations, and white pixels represent points with $\geq 20$ iterations.  }
%	\end{minipage}
%\end{figure*}



Along with a brute-force exact search,  three cover tree based ANN search strategies are investigated as described in the previous section:
\begin{itemize}
	\item FP-ANN for precision parameters $\nu_p= \{0.1, 0.05, .01, 0.001\}$.
	\item PFP-ANN for varying precision errors $\nu_p^k=r^k$ decaying at rates $r=\{0.05, 0.1,0.15,\ldots,0.95\}$.
	\item $(1+\epsilon)$-ANN for near optimality parameters $\epsilon=\{0, 0.2,0.4,\ldots,4\}$. 
	The case $\epsilon=0$ corresponds to an exact NN search, however by using the branch-and-bound algorithm on the cover tree proposed in~\cite{beygelzimer2006cover}.
	%\footnote{The reader should distinguish this case with performing a brute-force search. Although both perform an exact NN search, the complexity of the former is shown to be way less in practical datasets.}
\end{itemize}


\subsubsection*{Gaussian CS sampling}
From each dataset we select $J=50$ points at random and populate our signal matrix $X\in \RR^{\n\times J}$. We then subsample the signal using the linear noiseless model discussed  in \eqref{eq:datadrivenCS}, where the sampling matrix $A\in \RR^{m\times \n J}$ is drawn at random from the i.i.d. 
%(zero mean, unite variance) Gaussian 
normal distribution. We denote by $\frac{m}{n}\leq 1$ (where, $n=\n J$) as the subsampling ratio used in each experiment.
 %matrix $A\in \RR^{M J\times \n J}$, where $M\leq N$ is the number of CS measurements per signal. The ratio $\frac{M}{N}$ measures the overall compression. 
 
%\subsubsection*{The recovery algorithms}
Throughout we set the maximum number of IPG iterations to $30$. The step size is set to $\mu = 1/m\approx 1/\MM$ which is a theoretical value satisfying the restricted Lipschitz smoothness condition for the i.i.d. Normal sampling ensembles in our theorems and related works on iterative hard thresholding algorithms e.g. see~\cite{IHTCS,AIHT,MIP}.

Figure~\ref{fig:Decays} shows the normalized solution MSE measured by $\frac{\norm{x^k-x^\gt}}{\norm{x^\gt}}$ at each iteration of the exact and inexact IPG algorithms, and
for a given random realization of the sampling matrix $A$ and selected signals $X$. 
For the FP-ANN IPG the convergence rate is unchanged from the exact IPG algorithm but the reconstruction accuracy depends on the chosen precision parameter and for lower precisions the algorithm
stops at an earlier iteration with reduced accuracy, but with the benefit of requiring a smaller search tree. 

The PFP-ANN IPG ultimately achieves the same accuracy of the exact algorithm. 
%As we observe, for the FP-ANN IPG the reconstruction accuracy depends on the chosen precision parameter and for lower precisions the algorithm stops at its early iterations with poor solution accuracy. The PFP-ANN IPG ultimately achieves the accuracy of the exact algorithm. 
Refining the approximations at a slow rate slows down the convergence of the algorithm (i.e. the staircase effect visible in the plots correspond to  $r=\{0.7,0.9\}$), whereas choosing too fast error decays, e.g. $r=0.1$, does not improve the convergence rate beyond the exact algorithm and thus potentially leads to computational inefficiency. The $(1+\epsilon)$-ANN IPG algorithm can also achieve the precision of an exact recovery for moderately chosen approximation parameters. The case $\epsilon=0$ (unsurprisingly) iterates the same steps as for the IPG with brute-force search. Increasing $\epsilon$ slows down the convergence and for a very large parameter, e.g. $\epsilon=\{3,4\}$, the algorithm diverges.

Figure~\ref{fig:PT} illustrates the recovery phase transitions for the inexact IPG using the PFP-ANN and $(1+\epsilon)$-ANN searches. %The image intensities correspond to the normalized solution MSEs (image intensities are scaled logarithmically i.e. $\log_{10}\left(\frac{\norm{x^k-x^\gt}}{\norm{x^\gt}}\right)$ and darker pixels indicate higher solution accuracies) for  various approximation parameters versus the subsampling ratios ranging between $5-100\%$. 
The normalized MSE is averaged over $10$ random realizations of the sampling matrix $A$ and $20$ randomly subselected signal matrices $X$ for a given $A$. 
In each image the area below the red curve has the solution MSE less than $10^{-4}$ and is chosen as the recovery region. We can observe that the PFP-ANN oracle results in a  recovery region which is almost invariant to the chosen decay parameter $r$ (except for the slow converging case $r \gtrsim 0.6$, due to the limit on the maximum number of iterations). 

In the case of the $1+\epsilon$ oracle we see a different behaviour; smaller values of $\epsilon$ allow for a larger recovery region and larger approximations are restricted to work only 
%for CS recovery 
in high sampling regimes. This observation is in agreement with our theoretical  bounds on recovery and it shows that the $(1+\epsilon)$-approximate oracles are sensitive to the compression ratio, even though an exact (or a better-chosen approximate) 
IPG might still report recovery in the same sampling regime.

Finally in Table~\ref{tab:comp} we report the total cost of projections for each iterative scheme. The cost is measured as the total number of pairwise distances calculated for performing the NN or ANN searches, and it is averaged over the same  trials as previously described\footnote{In our evaluations, we exclude the computation costs of the gradient updates, i.e. the forward and backward operators, which can become dominant when datasets are not very large and the sampling matrix is dense e.g. a Gaussian matrix. For structured embedding matrices such as the fast Johnson-Lindenstrauss transform~\cite{FJLT1} or randomized orthoprojectors e.g. in MRI applications the cost of gradient updates becomes a tiny fraction of the search step, particularly when dealing with a large size dataset.}. For a better evaluation we set the algorithm to terminate earlier (than 30 iterations) when the objective function does not progress more than a tolerance level $tol = 10^{-8}$.  
 For each scheme the reported parameter achieves an average normalized solution MSE $\leq10^{-4}$ in the smallest amount of computations. For comparison we also include the cost of exact IPG implemented with the brute-force and exact ($\epsilon=0$) cover tree NN searches.  When using a brute-force NN search the cost per iteration is fixed and it is equal to the whole dataset population; as a result the corresponding exact IPG reports the highest computation. Replacing the brute-force search with a cover tree based exact NN search significantly reduces the computations. This is related to the low dimensionality of the manifolds in our experiments for which a cover tree search, even for performing an exact NN,  turns out to require many fewer pairwise distances evaluations.  
  Remarkably, the approximate algorithm $(1+\epsilon)$-ANN IPG 
  %and for a well chosen parameter (here mostly $\epsilon=.4$) 
  consistently outperforms all other schemes by reporting 4-10 times acceleration compared to the exact algorithm with $\epsilon=0$, and about (or sometimes more than) 2 orders of magnitude acceleration compared to the IPG with an exact brute-force search; in fact for larger  datasets the gap becomes wider as the $(1+\epsilon)$-ANN complexity stays relatively invariant to the population size.   
  %We recall that both inexact schemes based on FP-ANN and PFP-ANN are individually performing exact searches however on the truncated tree. 
The FP-ANN IPG reports similar computations as for the exact tree search ($\epsilon=0$) algorithm because in order to achieve the desired accuracy the (exact) search is performed up to a very fine level of the tree. 
%Despite its robustness against approximation, 
%and a fast convergence in number of iterations, 
  A gradual progress along the tree levels by the PFP-ANN IPG however improves the search time and reports a comparable computation cost to the $(1+\epsilon)$-ANN. 
%also does not much reduce the overall computations compared to the exact IPG.
Also it can be observed that by taking more samples the overall projection cost reduces which is related to the fast convergence (i.e. less iterations) of the algorithm once more measurements are available.
%Another remarkable observation is that the precision of the PFP and $(1+\epsilon)$ approximate schemes are about 2 orders of magnitude better than the exact IPG. Despite our theoretical results do not cover such observation, we shall relate it to a common practical knowledge that using relaxations, e.g. here approximations,  generally improves the performance of nonconvex algorithms compared to making hard decisions, and introduces a notion of robustness against undesirable local minima in such settings.     
  





%The performance is measured in terms of: 
%\begin{itemize}
%	\item The relative solution MSE measured as 
%	\eq{\log_{10}\left( \frac{\norm{\widehat{x}-x^\gt}}{\norm{x^\gt}}\right).} Solutions with log-MSE below $-4$ are reported as instances of exact recovery.
%	\item Number of iterations before termination. 
%	%\item The overall NN complexity measured as the total number of the cover tree nodes visited before termination.
%\end{itemize}
%
%
%For each case (dataset, compression ratio, algorithm), we repeat this experiment 25 times for random independent realizations of $A,X$ and report the mean value of the measures defied above. 

%\subsection{Conclusion on experiments}
%The $(1+\epsilon)$ approximation limits the recovery regime; For high compression ratios one can not afford for large  approximations of this type. However, the PFP type approximation is more robust in that sense and iterates less.
%\todo{Do we really need a conclusion here?}

%===============TABLE COMP=================
\ifCLASSOPTIONtwocolumn
\begin{table*}[t!]
	\vspace{0cm}
	\centering
	\scalebox{.95}{%	
		\begin{tabular}{lccccccccccccccc}
			\toprule[0.2em]
			& \multicolumn{14}{c}{ Total NN/ANN cost $(\times 10^4)$ }   \\
			\midrule[0.05em]
			Subsampling ratio ($\frac{m}{n}$)  & \multicolumn{5}{c}{ $10\%$ } &  \multicolumn{4}{c}{ $20\%$ } &  \multicolumn{5}{c}{ $30\%$ }  \\
			\midrule[0.05em]
			Datasets   & SM & SR & OW & MRF & & SM & SR & OW & MRF & & SM &  SR & OW & MRF  \\
			\midrule[0.2em]
			Brute-force NN& 194.23 & 193.67 & 215.10 & 923.34 &&  130.80 & 127.19 & 140.89 & 744.23 & & 113.55 &  109.34 & 123.06 & 699.48\\
			\midrule[.05em]
			CT's exact NN $(\epsilon=0)$  & 8.11 & 8.90 & 15.47 & 33.05 & & 4.90 & 5.19 & 8.99 & 24.74 & &  3.87 & 4.08 & 7.19 & 20.91\\
			\midrule[.05em]
			FP-ANN  & 8.11 & 8.90 & 15.47 & - & & 4.90 & 5.19 & 9.00 & - & & 3.88 & 4.07 & 7.21 & - \\
			Parameter $\nup$  & 1E-3 & 1E-3 & 1E-3 &  & & 1E-3 & 1E-3 & 1E-3 & & & 1E-3 & 1E-3 & 1E-3 &  \\
			\midrule[.05em]
			PFP-ANN & 2.94 & 3.50 & 7.10 & 3.41 & & 1.96 &  2.41 & 3.94 & 2.84 & & 1.78 &  1.99 & 3.38 & 2.52\\
			Parameter $r$ & 4E-1 & 5E-1 & 5E-1 & 4E-1 & & 3E-1 &  3E-1 & 4E-1 & 4E-1 & & 4E-1 &  3E-1 & 4E-1 & 2E-1\\
			\midrule[.05em]
			$(1+\epsilon)$-ANN &  \textbf{2.36} & \textbf{2.77} & \textbf{4.54} & \textbf{2.78} & & \textbf{1.54} & \textbf{1.86} & \textbf{2.91} & \textbf{2.21} & & \textbf{1.31} & \textbf{1.60} & \textbf{2.46} & \textbf{1.92}\\	
			Parameter $\epsilon$ &   4E-1 & 4E-1 & 4E-1 & 4E-1 & & 4E-1 & 4E-1 & 4E-1 & 4E-1 & & 4E-1 & 4E-1 & 6E-1 & 4E-1\\							
			\bottomrule[0.2em]\\
		\end{tabular}}
		\caption{Average computational complexity of the exact/inexact IPG measured by the total number of pairwise distances (in the ambient dimension) calculated within the NN/ANN steps to achieve an average solution MSE $\leq 10^{-4}$ (algorithms with less accuracies are marked as '-'). For each ANN scheme the lowest cost and the associated parameter is reported.  
			%The marker '-' indicates poor accuracy i.e. MSE $>10^{-4}$. 
			SM, SR, OW and MRF abbreviate S-Manifold, Swiss roll, Oscillating wave and the MR Fingerprints datasets, respectively.}\label{tab:comp}
	\end{table*}
\else
\begin{table*}[t!]
	\vspace{0cm}
	\centering
	\scalebox{.83}{%	
		\begin{tabular}{lccccccccccccccc}
			\toprule[0.2em]
			& \multicolumn{14}{c}{ Total NN/ANN cost $(\times 10^4)$ }   \\
			\midrule[0.05em]
			Subsampling ratio ($\frac{m}{n}$)  & \multicolumn{5}{c}{ $10\%$ } &  \multicolumn{4}{c}{ $20\%$ } &  \multicolumn{5}{c}{ $30\%$ }  \\
			\midrule[0.05em]
			Datasets   & SM & SR & OW & MRF & & SM & SR & OW & MRF & & SM &  SR & OW & MRF  \\
			\midrule[0.2em]
			Brute-force NN& 194.23 & 193.67 & 215.10 & 923.34 &&  130.80 & 127.19 & 140.89 & 744.23 & & 113.55 &  109.34 & 123.06 & 699.48\\
			\midrule[.05em]
			CT's exact NN $(\epsilon=0)$  & 8.11 & 8.90 & 15.47 & 33.05 & & 4.90 & 5.19 & 8.99 & 24.74 & &  3.87 & 4.08 & 7.19 & 20.91\\
			\midrule[.05em]
			FP-ANN  & 8.11 & 8.90 & 15.47 & - & & 4.90 & 5.19 & 9.00 & - & & 3.88 & 4.07 & 7.21 & - \\
			Parameter $\nup$  & 1E-3 & 1E-3 & 1E-3 &  & & 1E-3 & 1E-3 & 1E-3 & & & 1E-3 & 1E-3 & 1E-3 &  \\
			\midrule[.05em]
			PFP-ANN & 2.94 & 3.50 & 7.10 & 3.41 & & 1.96 &  2.41 & 3.94 & 2.84 & & 1.78 &  1.99 & 3.38 & 2.52\\
			Parameter $r$ & 4E-1 & 5E-1 & 5E-1 & 4E-1 & & 3E-1 &  3E-1 & 4E-1 & 4E-1 & & 4E-1 &  3E-1 & 4E-1 & 2E-1\\
			\midrule[.05em]
			$(1+\epsilon)$-ANN &  \textbf{2.36} & \textbf{2.77} & \textbf{4.54} & \textbf{2.78} & & \textbf{1.54} & \textbf{1.86} & \textbf{2.91} & \textbf{2.21} & & \textbf{1.31} & \textbf{1.60} & \textbf{2.46} & \textbf{1.92}\\	
			Parameter $\epsilon$ &   4E-1 & 4E-1 & 4E-1 & 4E-1 & & 4E-1 & 4E-1 & 4E-1 & 4E-1 & & 4E-1 & 4E-1 & 6E-1 & 4E-1\\							
			\bottomrule[0.2em]\\
		\end{tabular}}
		\caption{Average computational complexity of the exact/inexact IPG measured by the total number of pairwise distances (in the ambient dimension) calculated within the NN/ANN steps to achieve an average solution MSE $\leq 10^{-4}$ (algorithms with less accuracies are marked as '-'). For each ANN scheme the lowest cost and the associated parameter is reported.  
			%The marker '-' indicates poor accuracy i.e. MSE $>10^{-4}$. 
			SM, SR, OW and MRF abbreviate S-Manifold, Swiss roll, Oscillating wave and the MR Fingerprints datasets, respectively.}\label{tab:comp}
	\end{table*}
\fi	

% \vspace{-0.5em}
\section{Conclusion}
% \vspace{-0.5em}
Recent advances in multimodal single-cell technology have enabled the simultaneous profiling of the transcriptome alongside other cellular modalities, leading to an increase in the availability of multimodal single-cell data. In this paper, we present \method{}, a multimodal transformer model for single-cell surface protein abundance from gene expression measurements. We combined the data with prior biological interaction knowledge from the STRING database into a richly connected heterogeneous graph and leveraged the transformer architectures to learn an accurate mapping between gene expression and surface protein abundance. Remarkably, \method{} achieves superior and more stable performance than other baselines on both 2021 and 2022 NeurIPS single-cell datasets.

\noindent\textbf{Future Work.}
% Our work is an extension of the model we implemented in the NeurIPS 2022 competition. 
Our framework of multimodal transformers with the cross-modality heterogeneous graph goes far beyond the specific downstream task of modality prediction, and there are lots of potentials to be further explored. Our graph contains three types of nodes. While the cell embeddings are used for predictions, the remaining protein embeddings and gene embeddings may be further interpreted for other tasks. The similarities between proteins may show data-specific protein-protein relationships, while the attention matrix of the gene transformer may help to identify marker genes of each cell type. Additionally, we may achieve gene interaction prediction using the attention mechanism.
% under adequate regulations. 
% We expect \method{} to be capable of much more than just modality prediction. Note that currently, we fuse information from different transformers with message-passing GNNs. 
To extend more on transformers, a potential next step is implementing cross-attention cross-modalities. Ideally, all three types of nodes, namely genes, proteins, and cells, would be jointly modeled using a large transformer that includes specific regulations for each modality. 

% insight of protein and gene embedding (diff task)

% all in one transformer

% \noindent\textbf{Limitations and future work}
% Despite the noticeable performance improvement by utilizing transformers with the cross-modality heterogeneous graph, there are still bottlenecks in the current settings. To begin with, we noticed that the performance variations of all methods are consistently higher in the ``CITE'' dataset compared to the ``GEX2ADT'' dataset. We hypothesized that the increased variability in ``CITE'' was due to both less number of training samples (43k vs. 66k cells) and a significantly more number of testing samples used (28k vs. 1k cells). One straightforward solution to alleviate the high variation issue is to include more training samples, which is not always possible given the training data availability. Nevertheless, publicly available single-cell datasets have been accumulated over the past decades and are still being collected on an ever-increasing scale. Taking advantage of these large-scale atlases is the key to a more stable and well-performing model, as some of the intra-cell variations could be common across different datasets. For example, reference-based methods are commonly used to identify the cell identity of a single cell, or cell-type compositions of a mixture of cells. (other examples for pretrained, e.g., scbert)


%\noindent\textbf{Future work.}
% Our work is an extension of the model we implemented in the NeurIPS 2022 competition. Now our framework of multimodal transformers with the cross-modality heterogeneous graph goes far beyond the specific downstream task of modality prediction, and there are lots of potentials to be further explored. Our graph contains three types of nodes. while the cell embeddings are used for predictions, the remaining protein embeddings and gene embeddings may be further interpreted for other tasks. The similarities between proteins may show data-specific protein-protein relationships, while the attention matrix of the gene transformer may help to identify marker genes of each cell type. Additionally, we may achieve gene interaction prediction using the attention mechanism under adequate regulations. We expect \method{} to be capable of much more than just modality prediction. Note that currently, we fuse information from different transformers with message-passing GNNs. To extend more on transformers, a potential next step is implementing cross-attention cross-modalities. Ideally, all three types of nodes, namely genes, proteins, and cells, would be jointly modeled using a large transformer that includes specific regulations for each modality. The self-attention within each modality would reconstruct the prior interaction network, while the cross-attention between modalities would be supervised by the data observations. Then, The attention matrix will provide insights into all the internal interactions and cross-relationships. With the linearized transformer, this idea would be both practical and versatile.

% \begin{acks}
% This research is supported by the National Science Foundation (NSF) and Johnson \& Johnson.
% \end{acks}

\bibliographystyle{IEEEtran}
\bibliography{mybiblio}

% You can push biographies down or up by placing
% a \vfill before or after them. The appropriate
% use of \vfill depends on what kind of text is
% on the last page and whether or not the columns
% are being equalized.

%\vfill

% Can be used to pull up biographies so that the bottom of the last one
% is flush with the other column.
%\enlargethispage{-5in}



% that's all folks
\end{document}


