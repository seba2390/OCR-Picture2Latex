\section{Numerical experiments}
\label{sec:expe}

We test the performance of the exact/inexact IPG algorithm for our product-space data driven CS reconstruction using the four datasets described in Table \ref{tab:data}. The datasets are uniformly sampled (populated) from 2-dimensional continuous manifolds embedded in a higher ambient dimension, see also  Figure~\ref{fig:datasets}\footnote{The S-manifold, Swiss roll and Oscillating wave are synthetic machine learning  datasets available e.g. in \cite{GMRA12}. The Magnetic Resonance Fingerprints (MRF) is generated by solving the Bloch dynamic equation for a uniform grid of relaxation times $T1,T2$ and for an external magnetic excitation pattern, discussed and implemented in~\cite{MRF}.}. 

To proceed with fast ANN searches within IPG, we separately build a cover tree structure per dataset i.e. a preprocessing step. As illustrated for the MRF manifold in Figure~\ref{fig:CT} the coverage levels 
%(highlighted in colours for segments associated with tree nodes at certain scale) 
decrease in a coarse-to-fine manner as we traverse down the tree i.e. increasing the scale.

%============TABLE DATA=========================
\ifCLASSOPTIONtwocolumn
\begin{table}[t!]
	\centering
	\scalebox{.91}{
	\begin{tabular}{ccccc}
		%\hline
		\toprule[.2em]
		Dataset & Population ($d$) & Ambient dim. ($\n$)&CT depth&CT res.\\
		\midrule[.1em]
		S-Manifold & 5000 & 200& 14&2.43E-4 \\
		%\hline
		Swiss roll & 5000 & 200 &14&1.70E-4\\
		%\hline
		Oscillating wave & 5000 & 200 &14&1.86E-4\\
		%\hline
		MR Fingerprints & 29760 & 512& 13&3.44E-4\\
		%\hline
		\bottomrule[.2em]
	\end{tabular}}
	\caption{Datasets for data-driven CS evaluations; a cover tree (CT) structure is formed for each dataset. The last two columns respectively report the number of scales and the finest covering resolution of each tree.  }
	\label{tab:data}	
\end{table}
\else
\begin{table}[t!]
	\centering
	\scalebox{1.1}{
		\begin{tabular}{ccccc}
			%\hline
			\toprule[.2em]
			Dataset & Population ($d$) & Ambient dim. ($\n$)&CT depth&CT res.\\
			\midrule[.1em]
			S-Manifold & 5000 & 200& 14&2.43E-4 \\
			%\hline
			Swiss roll & 5000 & 200 &14&1.70E-4\\
			%\hline
			Oscillating wave & 5000 & 200 &14&1.86E-4\\
			%\hline
			MR Fingerprints & 29760 & 512& 13&3.44E-4\\
			%\hline
			\bottomrule[.2em]
		\end{tabular}}
		\caption{Datasets for data-driven CS evaluations; a cover tree (CT) structure is formed for each dataset. The last two columns respectively report the number of scales and the finest covering resolution of each tree.  }
		\label{tab:data}	
	\end{table}
	\fi
%========== Manifolds illustrations ===========================
\ifCLASSOPTIONtwocolumn
\begin{figure}[t!]
	\centering
	\begin{minipage}{\linewidth}
		\centering
		\subfloat[S-Manifold]{\includegraphics[width=.46\textwidth]{dict_1_3manifold.png} }	
		\quad	
		\subfloat[Swiss roll]{\includegraphics[width=.46\textwidth]{dict_2_3manifold.png} }	
		\quad	
		\subfloat[Oscillating wave]{\includegraphics[width=.46\textwidth]{dict_3_3manifold.png} }	
		\quad	
		\subfloat[MR Fingerprints]{\includegraphics[width=.46\textwidth]{dict_4_2manifold.png} }	
	\caption{Illustration of the low dimensional structures of  datasets presented in Table~\ref{tab:data}. Points are depicted along the first three principal components of each dataset.\label{fig:datasets}}
\end{minipage}
\end{figure}
\else
\begin{figure}[t!]
	\centering
	\begin{minipage}{\linewidth}
		\centering
		\subfloat[S-Manifold]{\includegraphics[width=.35\textwidth]{dict_1_3manifold.png} }	
		\quad	
		\subfloat[Swiss roll]{\includegraphics[width=.35\textwidth]{dict_2_3manifold.png} }	
		\quad	
		\subfloat[Oscillating wave]{\includegraphics[width=.35\textwidth]{dict_3_3manifold.png} }	
		\quad	
		\subfloat[MR Fingerprints]{\includegraphics[width=.35\textwidth]{dict_4_2manifold.png} }	
		\caption{Illustration of the low dimensional structures of  datasets presented in Table~\ref{tab:data}. Points are depicted along the first three principal components of each dataset.\label{fig:datasets}}
	\end{minipage}
\end{figure}
\fi
		
		
%----------------CT levels-------------
\ifCLASSOPTIONtwocolumn
\begin{figure}[t!]
\centering
\begin{minipage}{\linewidth}
		\subfloat[Scale 2]{\includegraphics[width=.46\textwidth]{dict_4_2CT_scale_2.png} }	
		\quad	
		\subfloat[Scale 3]{\includegraphics[width=.46\textwidth]{dict_4_2CT_scale_3.png} }	
		\\
		\subfloat[Scale 4]{\includegraphics[width=.46\textwidth]{dict_4_2CT_scale_4.png} }	
		\quad
		\subfloat[Scale 5]{\includegraphics[width=.46\textwidth]{dict_4_2CT_scale_5.png} }		
		\caption{A cover tree is built on MR Fingerprints dataset: (a-d) data partitions i.e. descendants covered with parent nodes appearing at scales 2-5
			are highlighted in different colours. The coverage resolution refines 
			%Low scale partitions divide into finer segments 
			by increasing the scale.\label{fig:CT}}
\end{minipage}
\end{figure}
\else
\begin{figure}[t!]
	\centering
	\begin{minipage}{\linewidth}
		\centering
		\subfloat[Scale 2]{\includegraphics[width=.35\textwidth]{dict_4_2CT_scale_2.png} }	
		\quad	
		\subfloat[Scale 3]{\includegraphics[width=.35\textwidth]{dict_4_2CT_scale_3.png} }	
		\\
		\subfloat[Scale 4]{\includegraphics[width=.35\textwidth]{dict_4_2CT_scale_4.png} }	
		\quad
		\subfloat[Scale 5]{\includegraphics[width=.35\textwidth]{dict_4_2CT_scale_5.png} }		
		\caption{A cover tree is built on MR Fingerprints dataset: (a-d) data partitions i.e. descendants covered with parent nodes appearing at scales 2-5
			are highlighted in different colours. The coverage resolution refines 
			%Low scale partitions divide into finer segments 
			by increasing the scale.\label{fig:CT}}
	\end{minipage}
\end{figure}
\fi

%-------wide

%\begin{figure*}[t!]
%	\centering
%	\begin{minipage}{\textwidth}
%		\centering
%		\subfloat[Scale 2]{\includegraphics[width=.22\textwidth]{./figs/manifolds/dict_4_2CT_scale_2.png} }	
%		\quad	
%		\subfloat[Scale 3]{\includegraphics[width=.22\textwidth]{./figs/manifolds/dict_4_2CT_scale_3.png} }	
%		\quad
%		\subfloat[Scale 4]{\includegraphics[width=.22\textwidth]{./figs/manifolds/dict_4_2CT_scale_4.png} }	
%		\quad
%		\subfloat[Scale 5]{\includegraphics[width=.22\textwidth]{./figs/manifolds/dict_4_2CT_scale_5.png} }		
%		\caption{A cover tree is built on MR Fingerprints dataset: (a-d) data partitions i.e. descendants covered with parent nodes appearing at scales 2-5
%			are highlighted in different colours. The coverage resolution refines 
%			%Low scale partitions divide into finer segments 
%			by increasing the scale.\label{fig:CT}}
%	\end{minipage}
%\end{figure*}
%=================MSE vs. iter Decays===============
\ifCLASSOPTIONtwocolumn
\begin{figure*}[t]
	\centering
	\begin{minipage}{\textwidth}
		\centering
		\subfloat{\includegraphics[width=.225\textwidth]{Sol_iter_data_1_alg_4_compr_4} }	
		\quad	
		\subfloat{\includegraphics[width=.225\textwidth]{Sol_iter_data_2_alg_4_compr_4} }	
		\quad
		\subfloat{\includegraphics[width=.225\textwidth]{Sol_iter_data_3_alg_4_compr_4} }	
		\quad	
		\subfloat{\includegraphics[width=.225\textwidth]{Sol_iter_data_4_alg_4_compr_4} }	
		\quad
		\subfloat{\includegraphics[width=.225\textwidth]{Sol_iter_data_1_alg_3_compr_4} }	
		\quad	
		\subfloat{\includegraphics[width=.225\textwidth]{Sol_iter_data_2_alg_3_compr_4} }	
		\quad
		\subfloat{\includegraphics[width=.225\textwidth]{Sol_iter_data_3_alg_3_compr_4} }	
		\quad	
		\subfloat{\includegraphics[width=.225\textwidth]{Sol_iter_data_4_alg_3_compr_4} }	
		\quad
		\subfloat{\includegraphics[width=.225\textwidth]{Sol_iter_data_1_alg_2_compr_4} }	
		\quad	
		\subfloat{\includegraphics[width=.225\textwidth]{Sol_iter_data_2_alg_2_compr_4} }	
		\quad
		\subfloat{\includegraphics[width=.225\textwidth]{Sol_iter_data_3_alg_2_compr_4} }	
		\quad	
		\subfloat{\includegraphics[width=.225\textwidth]{Sol_iter_data_4_alg_2_compr_4} }	
		\caption{Convergence of the exact/inexact IPG for subsampling ratio $\frac{m}{n}=0.2$. 
			Rows from top to bottom correspond to inexact algorithms with FP, PFP and $1+\epsilon$ ANN searches, respectively (legends for the plots in each row are identical and included in the last column). Columns from left to right correspond to  S-Manifold, Swiss roll, Oscillating wave and MR Fingerprints datasets, respectively. \label{fig:Decays}}
	\end{minipage}
\end{figure*}
\else
\begin{figure*}[t]
	\centering
	\begin{minipage}{\textwidth}
		\centering
		\subfloat{\includegraphics[width=.3\textwidth]{Sol_iter_data_1_alg_4_compr_4} }	
		\quad
		\subfloat{\includegraphics[width=.3\textwidth]{Sol_iter_data_1_alg_3_compr_4} }	
		\quad
		\subfloat{\includegraphics[width=.3\textwidth]{Sol_iter_data_1_alg_2_compr_4} }
		\quad			
		\subfloat{\includegraphics[width=.3\textwidth]{Sol_iter_data_2_alg_4_compr_4} }	
		\quad	
		\subfloat{\includegraphics[width=.3\textwidth]{Sol_iter_data_2_alg_3_compr_4} }
		\quad	
		\subfloat{\includegraphics[width=.3\textwidth]{Sol_iter_data_2_alg_2_compr_4} }
		\quad
		\subfloat{\includegraphics[width=.3\textwidth]{Sol_iter_data_3_alg_4_compr_4} }	
		\quad
		\subfloat{\includegraphics[width=.3\textwidth]{Sol_iter_data_3_alg_3_compr_4} }
		\quad
		\subfloat{\includegraphics[width=.3\textwidth]{Sol_iter_data_3_alg_2_compr_4} }		
		\quad	
		\subfloat{\includegraphics[width=.3\textwidth]{Sol_iter_data_4_alg_4_compr_4} }		
		\quad	
		\subfloat{\includegraphics[width=.3\textwidth]{Sol_iter_data_4_alg_3_compr_4} }	
		\quad	
		\subfloat{\includegraphics[width=.3\textwidth]{Sol_iter_data_4_alg_2_compr_4} }	
		\caption{Convergence of the exact/inexact IPG for subsampling ratio $\frac{m}{n}=0.2$. 			 
			Columns from left to right correspond to inexact algorithms with FP, PFP and $1+\epsilon$ ANN searches, respectively (legends for the plots in each column are identical and included in the last row). Rows from top to bottom correspond to   S-Manifold, Swiss roll, Oscillating wave and MR Fingerprints datasets, respectively. \label{fig:Decays}}
	\end{minipage}
\end{figure*}
\fi

%================Phase Transitions=====================
\ifCLASSOPTIONtwocolumn
\begin{figure*}
	\centering
	\begin{minipage}{\textwidth}
		\centering
		\subfloat
		%[S-Manifold, $(1+\epsilon)$-IPG]
		{\includegraphics[width=.225\textwidth]{TITPTiter_data_1_alg_2} }
		\quad
		\subfloat
		%[Swiss roll,  $(1+\epsilon)$-IPG]
		{\includegraphics[width=.225\textwidth]{TITPTiter_data_2_alg_2} }
		\quad
		\subfloat
		%[Oscillating wave,  $(1+\epsilon)$-IPG]
		{\includegraphics[width=.225\textwidth]{TITPTiter_data_3_alg_2} }
		\quad
		\subfloat
		%[MR Fingerprints,  $(1+\epsilon)$-IPG]
		{\includegraphics[width=.225\textwidth]{TITPTiter_data_4_alg_2} }
		\quad
		\subfloat
		%[S-Manifold, PFP-IPG]
		{\includegraphics[width=.225\textwidth]{TITPTiter_data_1_alg_3} }
		\quad		
		\subfloat
		%[Swissroll, PFP-IPG]
		{\includegraphics[width=.225\textwidth]{TITPTiter_data_2_alg_3} }		
		\quad
		\subfloat
		%[Oscillating wave, PFP-IPG]
		{\includegraphics[width=.225\textwidth]{TITPTiter_data_3_alg_3} }
		\quad		
		\subfloat
		%[MR Fingerprints, PFP-IPG]
		{\includegraphics[width=.225\textwidth]{TITPTiter_data_4_alg_3} }
		
		
		\caption{Recovery phase transitions for IPG with approximate projection (i.e. ANN search). Image intensities correspond to the normalized solution MSE for a search parameter and a given subsampling ratio (ranging between 5-100$\%$).
			% and search parameter, and darker pixels indicate higher solution accuracies.
			 Intensities in all plots are identically set with a logarithmic scale: black pixels correspond to accurate points with MSE $\leq 10^{-6}$, white pixels represent points with MSE $\geq 1$, and the region below the red curve is defined as the exact recovery region with MSE $\leq10^{-4}$. 			
			The columns from left to right correspond to the phase transitions of S-Manifold, Swiss roll, Oscillating wave and MR Fingerprints datasets. The top and the bottom rows correspond to two cover tree based ANN searches namely, the $(1+\epsilon)$-ANN and the PFP-ANN with decay parameter $r$.   \label{fig:PT}}
	\end{minipage}
\end{figure*}
\else
\begin{figure*}
	\centering
	\begin{minipage}{\textwidth}
		\centering
		\subfloat
		%[S-Manifold, $(1+\epsilon)$-IPG]
		{\includegraphics[width=.35\textwidth]{TITPTiter_data_1_alg_2} }
		\quad
		\subfloat
		%[S-Manifold, PFP-IPG]
		{\includegraphics[width=.35\textwidth]{TITPTiter_data_1_alg_3} }
		\\
		\subfloat
		%[Swiss roll,  $(1+\epsilon)$-IPG]
		{\includegraphics[width=.35\textwidth]{TITPTiter_data_2_alg_2} }		
		\quad		
		\subfloat
		%[Swissroll, PFP-IPG]
		{\includegraphics[width=.35\textwidth]{TITPTiter_data_2_alg_3} }
		\\
		\subfloat
		%[Oscillating wave,  $(1+\epsilon)$-IPG]
		{\includegraphics[width=.35\textwidth]{TITPTiter_data_3_alg_2} }		
		\quad
		\subfloat
		%[Oscillating wave, PFP-IPG]
		{\includegraphics[width=.35\textwidth]{TITPTiter_data_3_alg_3} }
		\\
		\subfloat
		%[MR Fingerprints,  $(1+\epsilon)$-IPG]
		{\includegraphics[width=.35\textwidth]{TITPTiter_data_4_alg_2} }
		\quad		
		\subfloat
		%[MR Fingerprints, PFP-IPG]
		{\includegraphics[width=.35\textwidth]{TITPTiter_data_4_alg_3} }

		\caption{Recovery phase transitions for IPG with approximate projection (i.e. ANN search). Image intensities correspond to the normalized solution MSE for a search parameter and a given subsampling ratio (ranging between 5-100$\%$).
			% and search parameter, and darker pixels indicate higher solution accuracies.
			Intensities in all plots are identically set with a logarithmic scale: black pixels correspond to accurate points with MSE $\leq 10^{-6}$, white pixels represent points with MSE $\geq 1$, and the region below the red curve is defined as the exact recovery region with MSE $\leq10^{-4}$. 			
			Rows from top to bottom 
			 correspond to the phase transitions of S-Manifold, Swiss roll, Oscillating wave and MR Fingerprints datasets. The left and the right columns correspond to two cover tree based ANN searches namely, the $(1+\epsilon)$-ANN and the PFP-ANN with decay parameter $r$.   \label{fig:PT}}
	\end{minipage}
\end{figure*}
\fi

%\begin{figure*}
%	\centering
%	\vspace{-2cm}
%	\begin{minipage}{\textwidth}
%		\centering
%		\subfloat[S-Manifold, IPG with $(1+\epsilon)$-ANN  ]{\includegraphics[width=.4\textwidth]{./figs/PTiter_data_1_alg_2} }
%		\quad
%		\subfloat[S-Manifold, IPG with PFP $r$-ANN]{\includegraphics[width=.4\textwidth]{./figs/PTiter_data_1_alg_3} }
%		\quad
%		\subfloat[Swissroll, IPG with $(1+\epsilon)$-ANN]{\includegraphics[width=.4\textwidth]{./figs/PTiter_data_2_alg_2} }
%		\quad
%		\subfloat[Swissroll, IPG with PFP $r$-ANN]{\includegraphics[width=.4\textwidth]{./figs/PTiter_data_2_alg_3} }
%		
%		\subfloat[Oscillating wave, IPG with $(1+\epsilon)$-ANN]{\includegraphics[width=.4\textwidth]{./figs/PTiter_data_3_alg_2} }
%		\quad
%		\subfloat[Oscillating wave, IPG with PFP $r$-ANN]{\includegraphics[width=.4\textwidth]{./figs/PTiter_data_3_alg_3} }
%		\quad
%		\subfloat[MR Fingerprints, IPG with $(1+\epsilon)$-ANN]{\includegraphics[width=.4\textwidth]{./figs/PTiter_data_4_alg_2} }
%		\quad
%		\subfloat[MR Fingerprints, IPG with PFP $r$-ANN]{\includegraphics[width=.4\textwidth]{./figs/PTiter_data_4_alg_3} }
%		
%		
%		\caption{Average IPG iterations for four studied datasets, two covertree-based ANN algorithms ($(1+\epsilon)$-ANN and PFP $r$-ANN), and for different compression ratios. In each plot, region below the red curve corresponds to exact CS reconstruction. Intensities in all plots are identically set: black pixels correspond to points with $\leq 4$ iterations, and white pixels represent points with $\geq 20$ iterations.  }
%	\end{minipage}
%\end{figure*}



Along with a brute-force exact search,  three cover tree based ANN search strategies are investigated as described in the previous section:
\begin{itemize}
	\item FP-ANN for precision parameters $\nu_p= \{0.1, 0.05, .01, 0.001\}$.
	\item PFP-ANN for varying precision errors $\nu_p^k=r^k$ decaying at rates $r=\{0.05, 0.1,0.15,\ldots,0.95\}$.
	\item $(1+\epsilon)$-ANN for near optimality parameters $\epsilon=\{0, 0.2,0.4,\ldots,4\}$. 
	The case $\epsilon=0$ corresponds to an exact NN search, however by using the branch-and-bound algorithm on the cover tree proposed in~\cite{beygelzimer2006cover}.
	%\footnote{The reader should distinguish this case with performing a brute-force search. Although both perform an exact NN search, the complexity of the former is shown to be way less in practical datasets.}
\end{itemize}


\subsubsection*{Gaussian CS sampling}
From each dataset we select $J=50$ points at random and populate our signal matrix $X\in \RR^{\n\times J}$. We then subsample the signal using the linear noiseless model discussed  in \eqref{eq:datadrivenCS}, where the sampling matrix $A\in \RR^{m\times \n J}$ is drawn at random from the i.i.d. 
%(zero mean, unite variance) Gaussian 
normal distribution. We denote by $\frac{m}{n}\leq 1$ (where, $n=\n J$) as the subsampling ratio used in each experiment.
 %matrix $A\in \RR^{M J\times \n J}$, where $M\leq N$ is the number of CS measurements per signal. The ratio $\frac{M}{N}$ measures the overall compression. 
 
%\subsubsection*{The recovery algorithms}
Throughout we set the maximum number of IPG iterations to $30$. The step size is set to $\mu = 1/m\approx 1/\MM$ which is a theoretical value satisfying the restricted Lipschitz smoothness condition for the i.i.d. Normal sampling ensembles in our theorems and related works on iterative hard thresholding algorithms e.g. see~\cite{IHTCS,AIHT,MIP}.

Figure~\ref{fig:Decays} shows the normalized solution MSE measured by $\frac{\norm{x^k-x^\gt}}{\norm{x^\gt}}$ at each iteration of the exact and inexact IPG algorithms, and
for a given random realization of the sampling matrix $A$ and selected signals $X$. 
For the FP-ANN IPG the convergence rate is unchanged from the exact IPG algorithm but the reconstruction accuracy depends on the chosen precision parameter and for lower precisions the algorithm
stops at an earlier iteration with reduced accuracy, but with the benefit of requiring a smaller search tree. 

The PFP-ANN IPG ultimately achieves the same accuracy of the exact algorithm. 
%As we observe, for the FP-ANN IPG the reconstruction accuracy depends on the chosen precision parameter and for lower precisions the algorithm stops at its early iterations with poor solution accuracy. The PFP-ANN IPG ultimately achieves the accuracy of the exact algorithm. 
Refining the approximations at a slow rate slows down the convergence of the algorithm (i.e. the staircase effect visible in the plots correspond to  $r=\{0.7,0.9\}$), whereas choosing too fast error decays, e.g. $r=0.1$, does not improve the convergence rate beyond the exact algorithm and thus potentially leads to computational inefficiency. The $(1+\epsilon)$-ANN IPG algorithm can also achieve the precision of an exact recovery for moderately chosen approximation parameters. The case $\epsilon=0$ (unsurprisingly) iterates the same steps as for the IPG with brute-force search. Increasing $\epsilon$ slows down the convergence and for a very large parameter, e.g. $\epsilon=\{3,4\}$, the algorithm diverges.

Figure~\ref{fig:PT} illustrates the recovery phase transitions for the inexact IPG using the PFP-ANN and $(1+\epsilon)$-ANN searches. %The image intensities correspond to the normalized solution MSEs (image intensities are scaled logarithmically i.e. $\log_{10}\left(\frac{\norm{x^k-x^\gt}}{\norm{x^\gt}}\right)$ and darker pixels indicate higher solution accuracies) for  various approximation parameters versus the subsampling ratios ranging between $5-100\%$. 
The normalized MSE is averaged over $10$ random realizations of the sampling matrix $A$ and $20$ randomly subselected signal matrices $X$ for a given $A$. 
In each image the area below the red curve has the solution MSE less than $10^{-4}$ and is chosen as the recovery region. We can observe that the PFP-ANN oracle results in a  recovery region which is almost invariant to the chosen decay parameter $r$ (except for the slow converging case $r \gtrsim 0.6$, due to the limit on the maximum number of iterations). 

In the case of the $1+\epsilon$ oracle we see a different behaviour; smaller values of $\epsilon$ allow for a larger recovery region and larger approximations are restricted to work only 
%for CS recovery 
in high sampling regimes. This observation is in agreement with our theoretical  bounds on recovery and it shows that the $(1+\epsilon)$-approximate oracles are sensitive to the compression ratio, even though an exact (or a better-chosen approximate) 
IPG might still report recovery in the same sampling regime.

Finally in Table~\ref{tab:comp} we report the total cost of projections for each iterative scheme. The cost is measured as the total number of pairwise distances calculated for performing the NN or ANN searches, and it is averaged over the same  trials as previously described\footnote{In our evaluations, we exclude the computation costs of the gradient updates, i.e. the forward and backward operators, which can become dominant when datasets are not very large and the sampling matrix is dense e.g. a Gaussian matrix. For structured embedding matrices such as the fast Johnson-Lindenstrauss transform~\cite{FJLT1} or randomized orthoprojectors e.g. in MRI applications the cost of gradient updates becomes a tiny fraction of the search step, particularly when dealing with a large size dataset.}. For a better evaluation we set the algorithm to terminate earlier (than 30 iterations) when the objective function does not progress more than a tolerance level $tol = 10^{-8}$.  
 For each scheme the reported parameter achieves an average normalized solution MSE $\leq10^{-4}$ in the smallest amount of computations. For comparison we also include the cost of exact IPG implemented with the brute-force and exact ($\epsilon=0$) cover tree NN searches.  When using a brute-force NN search the cost per iteration is fixed and it is equal to the whole dataset population; as a result the corresponding exact IPG reports the highest computation. Replacing the brute-force search with a cover tree based exact NN search significantly reduces the computations. This is related to the low dimensionality of the manifolds in our experiments for which a cover tree search, even for performing an exact NN,  turns out to require many fewer pairwise distances evaluations.  
  Remarkably, the approximate algorithm $(1+\epsilon)$-ANN IPG 
  %and for a well chosen parameter (here mostly $\epsilon=.4$) 
  consistently outperforms all other schemes by reporting 4-10 times acceleration compared to the exact algorithm with $\epsilon=0$, and about (or sometimes more than) 2 orders of magnitude acceleration compared to the IPG with an exact brute-force search; in fact for larger  datasets the gap becomes wider as the $(1+\epsilon)$-ANN complexity stays relatively invariant to the population size.   
  %We recall that both inexact schemes based on FP-ANN and PFP-ANN are individually performing exact searches however on the truncated tree. 
The FP-ANN IPG reports similar computations as for the exact tree search ($\epsilon=0$) algorithm because in order to achieve the desired accuracy the (exact) search is performed up to a very fine level of the tree. 
%Despite its robustness against approximation, 
%and a fast convergence in number of iterations, 
  A gradual progress along the tree levels by the PFP-ANN IPG however improves the search time and reports a comparable computation cost to the $(1+\epsilon)$-ANN. 
%also does not much reduce the overall computations compared to the exact IPG.
Also it can be observed that by taking more samples the overall projection cost reduces which is related to the fast convergence (i.e. less iterations) of the algorithm once more measurements are available.
%Another remarkable observation is that the precision of the PFP and $(1+\epsilon)$ approximate schemes are about 2 orders of magnitude better than the exact IPG. Despite our theoretical results do not cover such observation, we shall relate it to a common practical knowledge that using relaxations, e.g. here approximations,  generally improves the performance of nonconvex algorithms compared to making hard decisions, and introduces a notion of robustness against undesirable local minima in such settings.     
  





%The performance is measured in terms of: 
%\begin{itemize}
%	\item The relative solution MSE measured as 
%	\eq{\log_{10}\left( \frac{\norm{\widehat{x}-x^\gt}}{\norm{x^\gt}}\right).} Solutions with log-MSE below $-4$ are reported as instances of exact recovery.
%	\item Number of iterations before termination. 
%	%\item The overall NN complexity measured as the total number of the cover tree nodes visited before termination.
%\end{itemize}
%
%
%For each case (dataset, compression ratio, algorithm), we repeat this experiment 25 times for random independent realizations of $A,X$ and report the mean value of the measures defied above. 

%\subsection{Conclusion on experiments}
%The $(1+\epsilon)$ approximation limits the recovery regime; For high compression ratios one can not afford for large  approximations of this type. However, the PFP type approximation is more robust in that sense and iterates less.
%\todo{Do we really need a conclusion here?}

%===============TABLE COMP=================
\ifCLASSOPTIONtwocolumn
\begin{table*}[t!]
	\vspace{0cm}
	\centering
	\scalebox{.95}{%	
		\begin{tabular}{lccccccccccccccc}
			\toprule[0.2em]
			& \multicolumn{14}{c}{ Total NN/ANN cost $(\times 10^4)$ }   \\
			\midrule[0.05em]
			Subsampling ratio ($\frac{m}{n}$)  & \multicolumn{5}{c}{ $10\%$ } &  \multicolumn{4}{c}{ $20\%$ } &  \multicolumn{5}{c}{ $30\%$ }  \\
			\midrule[0.05em]
			Datasets   & SM & SR & OW & MRF & & SM & SR & OW & MRF & & SM &  SR & OW & MRF  \\
			\midrule[0.2em]
			Brute-force NN& 194.23 & 193.67 & 215.10 & 923.34 &&  130.80 & 127.19 & 140.89 & 744.23 & & 113.55 &  109.34 & 123.06 & 699.48\\
			\midrule[.05em]
			CT's exact NN $(\epsilon=0)$  & 8.11 & 8.90 & 15.47 & 33.05 & & 4.90 & 5.19 & 8.99 & 24.74 & &  3.87 & 4.08 & 7.19 & 20.91\\
			\midrule[.05em]
			FP-ANN  & 8.11 & 8.90 & 15.47 & - & & 4.90 & 5.19 & 9.00 & - & & 3.88 & 4.07 & 7.21 & - \\
			Parameter $\nup$  & 1E-3 & 1E-3 & 1E-3 &  & & 1E-3 & 1E-3 & 1E-3 & & & 1E-3 & 1E-3 & 1E-3 &  \\
			\midrule[.05em]
			PFP-ANN & 2.94 & 3.50 & 7.10 & 3.41 & & 1.96 &  2.41 & 3.94 & 2.84 & & 1.78 &  1.99 & 3.38 & 2.52\\
			Parameter $r$ & 4E-1 & 5E-1 & 5E-1 & 4E-1 & & 3E-1 &  3E-1 & 4E-1 & 4E-1 & & 4E-1 &  3E-1 & 4E-1 & 2E-1\\
			\midrule[.05em]
			$(1+\epsilon)$-ANN &  \textbf{2.36} & \textbf{2.77} & \textbf{4.54} & \textbf{2.78} & & \textbf{1.54} & \textbf{1.86} & \textbf{2.91} & \textbf{2.21} & & \textbf{1.31} & \textbf{1.60} & \textbf{2.46} & \textbf{1.92}\\	
			Parameter $\epsilon$ &   4E-1 & 4E-1 & 4E-1 & 4E-1 & & 4E-1 & 4E-1 & 4E-1 & 4E-1 & & 4E-1 & 4E-1 & 6E-1 & 4E-1\\							
			\bottomrule[0.2em]\\
		\end{tabular}}
		\caption{Average computational complexity of the exact/inexact IPG measured by the total number of pairwise distances (in the ambient dimension) calculated within the NN/ANN steps to achieve an average solution MSE $\leq 10^{-4}$ (algorithms with less accuracies are marked as '-'). For each ANN scheme the lowest cost and the associated parameter is reported.  
			%The marker '-' indicates poor accuracy i.e. MSE $>10^{-4}$. 
			SM, SR, OW and MRF abbreviate S-Manifold, Swiss roll, Oscillating wave and the MR Fingerprints datasets, respectively.}\label{tab:comp}
	\end{table*}
\else
\begin{table*}[t!]
	\vspace{0cm}
	\centering
	\scalebox{.83}{%	
		\begin{tabular}{lccccccccccccccc}
			\toprule[0.2em]
			& \multicolumn{14}{c}{ Total NN/ANN cost $(\times 10^4)$ }   \\
			\midrule[0.05em]
			Subsampling ratio ($\frac{m}{n}$)  & \multicolumn{5}{c}{ $10\%$ } &  \multicolumn{4}{c}{ $20\%$ } &  \multicolumn{5}{c}{ $30\%$ }  \\
			\midrule[0.05em]
			Datasets   & SM & SR & OW & MRF & & SM & SR & OW & MRF & & SM &  SR & OW & MRF  \\
			\midrule[0.2em]
			Brute-force NN& 194.23 & 193.67 & 215.10 & 923.34 &&  130.80 & 127.19 & 140.89 & 744.23 & & 113.55 &  109.34 & 123.06 & 699.48\\
			\midrule[.05em]
			CT's exact NN $(\epsilon=0)$  & 8.11 & 8.90 & 15.47 & 33.05 & & 4.90 & 5.19 & 8.99 & 24.74 & &  3.87 & 4.08 & 7.19 & 20.91\\
			\midrule[.05em]
			FP-ANN  & 8.11 & 8.90 & 15.47 & - & & 4.90 & 5.19 & 9.00 & - & & 3.88 & 4.07 & 7.21 & - \\
			Parameter $\nup$  & 1E-3 & 1E-3 & 1E-3 &  & & 1E-3 & 1E-3 & 1E-3 & & & 1E-3 & 1E-3 & 1E-3 &  \\
			\midrule[.05em]
			PFP-ANN & 2.94 & 3.50 & 7.10 & 3.41 & & 1.96 &  2.41 & 3.94 & 2.84 & & 1.78 &  1.99 & 3.38 & 2.52\\
			Parameter $r$ & 4E-1 & 5E-1 & 5E-1 & 4E-1 & & 3E-1 &  3E-1 & 4E-1 & 4E-1 & & 4E-1 &  3E-1 & 4E-1 & 2E-1\\
			\midrule[.05em]
			$(1+\epsilon)$-ANN &  \textbf{2.36} & \textbf{2.77} & \textbf{4.54} & \textbf{2.78} & & \textbf{1.54} & \textbf{1.86} & \textbf{2.91} & \textbf{2.21} & & \textbf{1.31} & \textbf{1.60} & \textbf{2.46} & \textbf{1.92}\\	
			Parameter $\epsilon$ &   4E-1 & 4E-1 & 4E-1 & 4E-1 & & 4E-1 & 4E-1 & 4E-1 & 4E-1 & & 4E-1 & 4E-1 & 6E-1 & 4E-1\\							
			\bottomrule[0.2em]\\
		\end{tabular}}
		\caption{Average computational complexity of the exact/inexact IPG measured by the total number of pairwise distances (in the ambient dimension) calculated within the NN/ANN steps to achieve an average solution MSE $\leq 10^{-4}$ (algorithms with less accuracies are marked as '-'). For each ANN scheme the lowest cost and the associated parameter is reported.  
			%The marker '-' indicates poor accuracy i.e. MSE $>10^{-4}$. 
			SM, SR, OW and MRF abbreviate S-Manifold, Swiss roll, Oscillating wave and the MR Fingerprints datasets, respectively.}\label{tab:comp}
	\end{table*}
\fi	
