% Template for Elsevier CRC journal article
% version 1.2 dated 08 January 2015

% This file (c) 2009-15 Elsevier Ltd.  Modifications may be freely made,
% provided the edited file is saved under a different name

% This file contains modifications for Nuclear and Particle Physics Proceedings

% Changes since version 1.0
% - elsarticle class option changed from 1p to 3p (to better reflect CRC layout)
%
% version 1.2
% - Journal name changed to "Nuclear and Particle Physics Proceedings"

%-----------------------------------------------------------------------------------

%% This template uses the elsarticle.cls document class and the extension package ecrc.sty
%% For full documentation on usage of elsarticle.cls, consult the documentation "elsdoc.pdf"
%% Further resources available at http://www.elsevier.com/latex

%-----------------------------------------------------------------------------------

%%%%%%%%%%%%%%%%%%%%%%%%%%%%%%%%%%%%%%%%%%%%%%
%%%%%%%%%%%%%%%%%%%%%%%%%%%%%%%%%%%%%%%%%%%%%%
%%                                          %%
%% Important note on usage                  %%
%% -----------------------                  %%
%% This file must be compiled with PDFLaTeX %%
%% Using standard LaTeX will not work!      %%
%%                                          %%
%%%%%%%%%%%%%%%%%%%%%%%%%%%%%%%%%%%%%%%%%%%%%%
%%%%%%%%%%%%%%%%%%%%%%%%%%%%%%%%%%%%%%%%%%%%%%

%% The '3p' and 'times' class options of elsarticle are used for Elsevier CRC
\documentclass[3p,times,twocolumn]{elsarticle}

%% The `ecrc' package must be called to make the CRC functionality available
\usepackage{ecrc}

%% The ecrc package defines commands needed for running heads and logos.
%% For running heads, you can set the journal name, the volume, the starting page and the authors

%% set the volume if you know. Otherwise `00'
\volume{00}

%% set the starting page if not 1
\firstpage{1}

%% Give the name of the journal
\journalname{Nuclear and Particle Physics Proceedings}

%% Give the author list to appear in the running head
%% Example \runauth{C.V. Radhakrishnan et al.}
\runauth{Mirta Duman\v{c}i\'{c}}

%% The choice of journal logo is determined by the \jid and \jnltitlelogo commands.
%% A user-supplied logo with the name <\jid>logo.pdf will be inserted if present.
%% e.g. if \jid{yspmi} the system will look for a file yspmilogo.pdf
%% Otherwise the content of \jnltitlelogo will be set between horizontal lines as a default logo

%% Give the abbreviation of the Journal.
\jid{nppp}

%% Give a short journal name for the dummy logo (if needed)
\jnltitlelogo{Nuclear and Particle Physics Proceedings}

%% Hereafter the template follows `elsarticle'.
%% For more details see the existing template files elsarticle-template-harv.tex and elsarticle-template-num.tex.

%% Elsevier CRC generally uses a numbered reference style
%% For this, the conventions of elsarticle-template-num.tex should be followed (included below)
%% If using BibTeX, use the style file elsarticle-num.bst

%% End of ecrc-specific commands
%%%%%%%%%%%%%%%%%%%%%%%%%%%%%%%%%%%%%%%%%%%%%%%%%%%%%%%%%%%%%%%%%%%%%%%%%%

%% The amssymb package provides various useful mathematical symbols
\usepackage{amssymb}
\usepackage{amsmath}
%% The amsthm package provides extended theorem environments
%% \usepackage{amsthm}

%% The lineno packages adds line numbers. Start line numbering with
%% \begin{linenumbers}, end it with \end{linenumbers}. Or switch it on
%% for the whole article with \linenumbers after \end{frontmatter}.
\usepackage{lineno}

%% natbib.sty is loaded by default. However, natbib options can be
%% provided with \biboptions{...} command. Following options are
%% valid:

%%   round  -  round parentheses are used (default)
%%   square -  square brackets are used   [option]
%%   curly  -  curly braces are used      {option}
%%   angle  -  angle brackets are used    <option>
%%   semicolon  -  multiple citations separated by semi-colon
%%   colon  - same as semicolon, an earlier confusion
%%   comma  -  separated by comma
%%   numbers-  selects numerical citations
%%   super  -  numerical citations as superscripts
%%   sort   -  sorts multiple citations according to order in ref. list
%%   sort&compress   -  like sort, but also compresses numerical citations
%%   compress - compresses without sorting
%%
%% \biboptions{comma,round}

% \biboptions{}

% if you have landscape tables
\usepackage[figuresright]{rotating}
\usepackage{xspace}

% put your own definitions here:
%\newcommand*{\ATLASLATEXPATH}{../comm/latex/} -> did not work

\newcommand*{\antibar}[1]{\ensuremath{#1\bar{#1}}\xspace}

\newcommand{\cZ}{\cal{Z}}
\newcommand*{\pPb}{\ensuremath{p}+Pb\xspace}
\newcommand*{\PbPb}{Pb+Pb\xspace}
\newcommand*{\Zboson}{\ensuremath{Z}\xspace}
\newcommand*{\Wboson}{\ensuremath{W}\xspace}
\newcommand*{\nn}{NN\xspace}
\newcommand*{\pp}{\ensuremath{pp}\xspace}
\newcommand*{\pn}{\ensuremath{pn}\xspace}
\newcommand*{\np}{\ensuremath{np}\xspace}
\newcommand*{\Zmm}{\ensuremath{Z\rightarrow\mu\mu}\xspace}
\newcommand*{\Ztautau}{\ensuremath{Z\rightarrow\tau\tau}\xspace}
\newcommand*{\Wmn}{\ensuremath{W\rightarrow\mu\nu}\xspace}
\newcommand*{\ttbar}{\antibar{t}}
\newcommand*{\sqn}{\ensuremath{\sqrt{s_{_{\mathrm{NN}}}}}\xspace}
\newcommand*{\sqs}{\ensuremath{\sqrt{s}}\xspace}
\newcommand*{\pT}{\ensuremath{p_{\text{T}}}\xspace}
\newcommand{\ptZ}{\mbox{$p_{\mathrm{T}}^{Z}$}\xspace}
\newcommand{\yZ}{\mbox{$y^{Z}$}\xspace}
\newcommand*{\TeV}{\mbox{TeV}\xspace}
\newcommand*{\GeV}{\mbox{GeV}\xspace}

\newcommand*{\RAA}{\ensuremath{R_{\mathrm{AA}}}\xspace}
\newcommand*{\rAA}{\ensuremath{\RAA}\xspace}
\newcommand*{\raa}{\ensuremath{\RAA}\xspace}
\newcommand*{\RCP}{\ensuremath{R_{\mathrm{CP}}}\xspace}
\newcommand*{\rCP}{\ensuremath{\RCP}\xspace}
\newcommand*{\rcp}{\ensuremath{\RCP}\xspace}
\newcommand*{\RpA}{\ensuremath{R_{\mathrm{pA}}}\xspace}
\newcommand*{\rpa}{\ensuremath{\RpA}\xspace}
\newcommand*{\Taa}{\ensuremath{T_{\mathrm{AA}}}\xspace}
\newcommand*{\Tppb}{\ensuremath{T_{p\mathrm{Pb}}}\xspace}
\newcommand*{\avgTppb}{\ensuremath{\langle\Tppb\rangle}\xspace}

\newcommand*{\RpPb}{\ensuremath{R_{p\mathrm{Pb}}}\xspace}
\newcommand*{\rpPb}{\ensuremath{\RpPb}\xspace}
\newcommand*{\rppb}{\ensuremath{\RpPb}\xspace}
\newcommand*{\rpA}{\ensuremath{\RpPb}\xspace}

\newcommand*{\ifb}{\mbox{fb$^{-1}$}}
\newcommand*{\ipb}{\mbox{pb$^{-1}$}}
\newcommand*{\inb}{\mbox{nb$^{-1}$}}

\newcommand*{\GEANT}{\textsc{Geant}\xspace}

%   \newtheorem{def}{Definition}[section]
%   ...
%\usepackage{\ATLASLATEXPATH atlasheavyion}

% add words to TeX's hyphenation exception list
%\hyphenation{author another created financial paper re-commend-ed Post-Script}

% declarations for front matter

\begin{document}

%\linenumbers

\begin{frontmatter}

%% Title, authors and addresses

%% use the tnoteref command within \title for footnotes;
%% use the tnotetext command for the associated footnote;
%% use the fnref command within \author or \address for footnotes;
%% use the fntext command for the associated footnote;
%% use the corref command within \author for corresponding author footnotes;
%% use the cortext command for the associated footnote;
%% use the ead command for the email address,
%% and the form \ead[url] for the home page:
%%
%% \title{Title\tnoteref{label1}}
%% \tnotetext[label1]{}
%% \author{Name\corref{cor1}\fnref{label2}}
%% \ead{email address}
%% \ead[url]{home page}
%% \fntext[label2]{}
%% \cortext[cor1]{}
%% \address{Address\fnref{label3}}
%% \fntext[label3]{}

\dochead{}
%% Use \dochead if there is an article header, e.g. \dochead{Short communication}

\title{\textit{W} and \textit{Z} boson production in 5.02 TeV \textit{pp} and \textit{p}+Pb collisions with the ATLAS detector}

%% use optional labels to link authors explicitly to addresses:
%% \author[label1,label2]{<author name>}
%% \address[label1]{<address>}
%% \address[label2]{<address>}

\author{Mirta Duman\v{c}i\'{c}}
\author{on behalf of the ATLAS Collaboration}

\address{Department of Particle Physics and Astrophysics, Weizmann Institute of Science}
%\address[WIS]{Department of Particle Physics and Astrophysics, Weizmann Institute of Science}


\begin{abstract}
% Sasha corrected text of the abstract

This proceeding reports the ATLAS results of measuring vector boson production in \pp and \pPb collisions at the center-of-mass energy of 5.02~TeV per nucleon. \Zboson bosons are reconstructed via leptonic decays and results are presented as a function of $Z$-boson rapidity. \Wboson bosons are identified by measuring leptons coming from \Wboson decays and the results are presented as a function of lepton pseudorapidity. In \pp collisions \Zboson boson cross section in $|y_{Z}|<2.5$ is 590$~\pm$~9 (stat.)~$\pm$~12 (syst.)~$\pm$~32 (lumi.) pb. In \pPb system kinematic distributions of \Wboson and \Zboson bosons are compared to models based on NLO and NNLO QCD calculations using different PDF sets. Using PDF modification shows better agreement of the model and the data. Measurements done in different centrality intervals show that the PDF modification in \pPb may have centrality dependence. 

\end{abstract}

\begin{keyword}
%% keywords here, in the form: keyword \sep keyword
ATLAS \sep heavy ion \sep \Wboson boson \sep \Zboson boson \sep nPDF \sep centrality
%% MSC codes here, in the form: \MSC code \sep code
%% or \MSC[2008] code \sep code (2000 is the default)

\end{keyword}

\end{frontmatter}

%%
%% Start line numbering here if you want
%%

%% main text
%\section{Introduction}
%\label{Intro}
Experimental study of electroweak (EW) bosons in relativistic heavy ion (HI) collisions is an integral part of the physics program carried out by all four detectors taking data at the Large Hadron Collider (LHC). The results of these programs have demonstrated several important phenomena. Since EW bosons and their leptonic decay products do not interact strongly with the hot and dense matter created in the HI collision, their production rates are expected to be sensitive to the effective overlap area of colliding nuclear matter. This has been confirmed in measurements performed by the ATLAS and CMS experiments with \Zboson and \Wboson bosons decaying leptonically or semileptonically where it has been shown that the production rate of non-strongly interacting particles in \PbPb collisions scales with the nuclear thickness function~\cite{Aad:2012ew,Aad:2014bha,Aad:2015lcb,Chatrchyan:2011ua,Chatrchyan:2012nt}. \par
EW bosons are also used as an outstanding tool to study nuclear modifications to parton distribution functions (PDF). These effects include nuclear shadowing, anti-shadowing and the EMC effect. In particular, rapidity distributions of \Zboson and \Wboson bosons determined by the Bjorken $x$ of the interacting partons are sensitive to the presence of nuclear modification. The existing \PbPb measurements due to their limited precision cannot exclude the presence of the nuclear modification~\cite{Aad:2012ew,Aad:2014bha,Aad:2015lcb,Chatrchyan:2011ua,Chatrchyan:2012nt}. \par 
Study of asymmetric collisions systems, such as proton-lead (\pPb) can be used to differentiate between initial and final state effects in HI collisions. The results using 2013 \pPb data at the center-of-mass energy, \sqn ~=~5.02~\TeV on EW boson production have been published by all LHC experiments ~\cite{Aad:2015gta,Khachatryan:2015pzs,Khachatryan:2015hha,Alice:2016wka,Aaij:2014pvu}. %They show that calculations including nuclear PDF modification describe data better than those which use \pp PDF sets. ATLAS results \cite{Aad:2015gta, Markus:2015} suggest that the modification may also have centrality dependence.
%The rates of \Zboson boson production in \pPb collisions help to understand the problem of centrality definitions in \pPb collisions and the contribution of colour charge fluctuations to Glauber model calculations~\cite{Guzey:2005tk, Alvioli:2013vk}. 
%Nuclear modifications were previously estimated using predictions based on existing PDF sets~\cite{Aad:2015gta}, making the estimates model dependent. 
Using preliminary result~\cite{me:2016} on \Zboson boson production in \pp collisions at the same center-of-mass energy as in \pPb it becomes possible to study modifications observed in that system with respect to a physics measurement rather than to calculated prediction. \par

%%%%%%% we don't have space in tis proceeding to write an outline, skip below %%%%%%%%

%Recent results from measurements of \Wboson and \Zboson boson production in \pPb system at \sqn ~=~5.02~\TeV are discussed in the proceeding. The preliminary results on measuring \Zboson boson production in the \Zmm decay channel in \pp collisions at \sqs = 5.02~\TeV with the ATLAS detector at the LHC are presented~\cite{Aad:2008zzm,me:2016}. Preliminary data on \Zboson boson measurements in \pp, combined with published ATLAS \pPb results at \sqn=5.02 TeV \cite{Aad:2015gta}, allow constructing nuclear modification factor for Z boson in proton-lead collisions.
%These new data combined with the 2013 published ATLAS \pPb results at \sqn~=~5.02~\TeV\ allow the first measurement of the nuclear modification factor, \RpPb, for \Zboson bosons in proton-lead collisions. \par

%%%%%%%%%%%%%%%%%%%%%%%%%%%%%%%%%%%%%%%%%%%%%%%%%%%%%%%%%%%%%%%%%%%%%%%%%%%%%%%%%%%%%%%%%%%%%%%%%%%%%%%%%%%%%%%%%%%%
% let's start this proceeding 
% outline in short is physics-wise story telling: pp result, nPDF in pPb, centrality dependence of nPDF modification pPb 
%%%%%%%%%%%%%%%%%%%%%%%%%%%%%%%%%%%%%%%%%%%%%%%%%%%%%%%%%%%%%%%%%%%%%%%%%%%%%%%%%%%%%%%%%%%%%%%%%%%%%%%%%%%%%%%%%%%%

%\section{\Zboson boson in 5.02 TeV \pp collisions}
%\label{pp_section}
%\newline

%%%%%% here now we want to describe Z and W boson cross section measurements %%%%%%%
%%% it makes sense to do it for pPb first that is old and then bring new pp measurement %%%

The \pPb data obtained by the ATLAS experiment at \sqn=5.02~TeV corresponding to integrated luminosity of 29~\inb \xspace has been used to measure \Zboson boson production in muon and electron decay channels~\cite{Aad:2015gta}. Events for analysis in the electron channel were selected by high-level trigger requiring an electron with at least 15 GeV transverse momentum that also passes loose identification criteria. Similarly, events in the muon channel were selected with the high-level trigger requiring a muon with transverse momentum of at least 8 GeV. In total 1647 (2032) \Zboson boson candidates were reconstructed in electron (muon) decay channel. In particular, electrons within the range $2.5<|\eta|<4.9$ are reconstructed based on the energy deposited in the forward calorimeter that allows for the reconstruction of \Zboson boson candidates up to $|y^{Z}|<3.5$. This yields additional 264 \Zboson boson candidates. Total measured \Zboson boson fiducial cross section of 139.8~$\pm$~4.8 (stat.)~$\pm$~6.2 (syst.)~$\pm$~3.8 (lumi.)~nb is found to be slightly higher compared to model predictions based on perturbative QCD (pQCD) calculations that include CT10 PDF set~\cite{CT10ref}. \par

%%%%%%%%%%%%%%%%%%%%%%%%%%%%%%%%%%%%%%%%%%%%%%%%%%%%%%
% for now leave W out of this because we don't have number of candidates in the note or the cross section%
%%%%%%%%%%%%%%%%%%%%%%%%%%%%%%%%%%%%%%%%%%%%%%%%%%%%%%

%In the same dataset, $\Wboson \rightarrow \mu\nu$ production and the dependence of the cross section on the pseudorapidity of the muons was studied. The trigger used for the event selection was the same as described in the \Zboson boson analysis. Total of  \par

Preliminary measurement of the \Zboson boson cross section has been obtained from the \pp data sample corresponding to the integrated luminosity of 24.7 $\pm$ 1.3 ~\ipb\  at  \sqs~=~5.02~\TeV. Events for the analysis have been selected with the high-level trigger requiring a muon with transverse momentum of at least 14 GeV. In total 7293 \Zboson boson candidates passed all the analysis selection. %with invariant mass of $66<m_{\mu\mu}<116$~\GeV\  were obtained. 
The \Zboson boson fiducial cross section is measured to be 590~$\pm$~9~(stat.)~$\pm$~12~(syst.)~$\pm$~32~(lumi.)~pb. Model with CT10 PDF set predicts a significantly lower cross section of 537~pb. The NNLO prediction using the CT14 PDF set~\cite{CT14ref} and calculated using a version of DYNNLO 1.5~\cite{dyn1,dyn2}  yields a cross section of $573.77^{+13.94}_{-15.96}$~pb which agrees well with the measurement within its uncertainties.  




%~\cite{CT14ref} 
%~\cite{dyn1,dyn2} 

%%%%%%%%%%%%%%%%%%%%%%%%%%%%%%%%%%%%%%%%%%%
%%%%% skip fully the analysis description, except for trigger and statistics %%%%%
%%%%%%%%%%%%%%%%%%%%%%%%%%%%%%%%%%%%%%%%%%%

%Events are selected for further analysis if they contain at least two reconstructed muons with at least one of them passing the trigger requiring at least 14 GeV of the muon transverse momentum.  Both muons must satisfy high quality level of reconstruction quality and pass kinematic requirements: $\pT^{\mu}>20$~\GeV\ and $|\eta^{\mu}|<2.4$. The muons are also required to be isolated, i.e. the energy measured by the ATLAS calorimeters in the vicinity of the muon should not exceed a certain value, which depends on the muon momentum. Opposite charge muon pairs with invariant mass of $66<m_{\mu\mu}<116$~\GeV\ are selected for further analysis. %Monte Carlo simulation of different physics processes were used for background subtraction and to calculate efficiency and acceptance of the ATLAS detector.  %Different physics processes have been simulated using three Monte Carlo~(MC) generators and traced through the ATLAS detector using \GEANT4.  \Zmm events were generated in order to calculate the efficiency and acceptance of the ATLAS detector in measuring \Zboson bosons while electroweak~(EW) background processes have been used for the background subtraction. \par

%The main sources of systematic uncertainty are associated with the background subtraction and the correction factors used in bin-by-bin unfolding. The second contribution arises from muon reconstruction correction factors (reconstruction, isolation, and trigger) and the statistical uncertainties of the MC simulation used for the evaluation of these correction factors. Each uncertainty source is calculated differentially in \yZ and it does not exceed 4\%. In addition there is a 5.4\% systematic uncertainty associated with measuring the integrated luminosity of the \pp collision dataset. \par




\begin{figure}[ht!]
\centering
\includegraphics[width=0.42 \textwidth]{Xsection_y_22092016_newSys_Prelim.pdf}
\caption{The measured rapidity differential cross section in \pp data.  The data is compared to the simulated MC generated with the CT10 PDF at NLO, scaled to the integrated cross section calculated with DYNNLO at NNLO using the CT14 PDF set~\cite{me:2016}.}
\label{fig:Xsection}
\end{figure}

%%%%%%%%% pp ::  xsection in y %%%%%%%%%%%%

In addition to the integrated cross section, \Zboson analyses in \pPb and \pp datasets include measurements of the \yZ\ differential cross section. Figure~\ref{fig:Xsection} shows differential cross section $\textrm{d}\sigma/\textrm{d}y^{Z}$ measured in \pp collisions.  Because of the symmetry in \yZ, the data is shown in bins of $|\yZ|$.  The data is compared to the simulated MC generated with the CT10 PDF at NLO, scaled to the integrated cross section calculated at NNLO using the CT14 PDF set.  The scaled simulation is in good agreement with the data. \par


%\section{Nuclear PDF modification in \pPb collisions}
%\label{pPb_section}
\begin{figure}[ht!]
\centering
\includegraphics[width=0.4 \textwidth]{ZpPb}
\caption{ (a) The $d\sigma/dy^{Z}$ distribution from \Zboson boson decays measured in \pPb data, shown along with several model calculations in the upper panel. The bars indicate statistical uncertainty and the shaded boxes systematic uncertainty on the data; the uncertainties on the model calculations are not shown. (b-d) Ratios of the data to the models. The uncertainties of the model calculations (scale and PDF uncertainties added in quadrature) are shown as bands around unity in each panel. An additional 2.7\% luminosity uncertainty on the cross section is not shown~\cite{Aad:2015gta}.}
\label{fig:pPbZ}
\end{figure}

%%%%%%%%% pPb ::  xsection in y%%%%%%%%%%%%

The rapidity differential cross section measured in \pPb data is presented in Figure~\ref{fig:pPbZ} and compared to model calculations. The data shows a strong asymmetry about $y^{Z}=0$ compared to the model prediction with CT10 PDF set and is better described by the models containing nuclear modification such as EPS09.  

\begin{figure}[ht!]
\centering
\includegraphics[width=0.42 \textwidth]{WpPb}
\caption{The upper panel shows the cross section of the $W^{+}$ and $W^{-}$ production measured in \pPb data as a function of the lepton pseudorapidity compared to model prediction based on CT10 PDF set. The middle panel shows the data-to-model ratios for and the lower panel shows the lepton charge asymmetry compared to the same model~\cite{Markus:2015}.}
\label{fig:WpPb}
\end{figure}

In the same \pPb dataset the $\Wboson \rightarrow \mu\nu$ production and the dependence of the cross section on the pseudorapidity of the muons has also been studied~\cite{Markus:2015}. Prediction based on pQCD calculations reproduce the data well, except for the $W^{-}$ boson in the lead-going direction where there is an excess above the model, which is consistent with the similar observation in the \Zboson boson measurement. In order to study the difference in production of $W^{+}$ and $W^{-}$ bosons, the observable called lepton charge asymmetry  $A_{\mu}(\eta_{\mu})$ has also been measured. Measurement shows deviation from the CT10 prediction on the lead-going side. Upper panel of Figure~\ref{fig:WpPb} shows $W^{+}$ and $W^{-}$ cross section for all centrality classes of events as a function of the lepton pseudorapidity compared to model prediction based on CT10 PDF set. Middle panel shows the ratio to the model for different charges and the lower panel shows the lepton charge asymmetry compared to the model.  %In addition to both \Wboson and \Zboson boson measurements showing a slight deviation from the prediction on the lead-going side, the measurements both indicate that the observed effect slightly increases with the collision centrality.
%Measurements of both EW bosons show deviation from prediction in $-2.5<\eta<0$ pseudorapidity range, corresponding to the lead-going side. 
\par


\begin{figure}[h!]
\centering
\includegraphics[width=0.45 \textwidth]{Rpa_Z_blue.pdf}
\caption{The nuclear modification factor as a function of \Zboson boson rapidity. The uncertainties on each point include statistical and systematic uncertainties from \pPb and \pp measurements.  The band around unity represents the uncertainty of the luminosities. The blue line shows an expected \RpPb based on simulation~\cite{me:2016}.}
\label{fig:Rpa}
\end{figure}


The \Zboson boson cross section measured in \pp collisions may be used as a baseline to study nuclear modifications which may be present in the previously measured cross section in \pPb collisions~\cite{Aad:2015gta}. 
The nuclear modification factor, \RpPb, is defined as the ratio of the cross sections measured in \pPb and \pp systems. The exact definition of \RpPb can be found in~\cite{me:2016}. Figure~\ref{fig:Rpa} presents the nuclear modification factor as a function of \yZ for all centralities of events in \pPb data. It shows enhancement in the lead-going direction and suppression in the proton-going direction. The blue curve which describes the expected contribution from the isospin effect is not sufficient to describe the shape of the measured distribution. This difference can be explained by nuclear PDF modification present inside the lead nucleus and effects observed in \pPb measurements as discussed above.  \par

%\section{Centrality dependence of nPDF modification}
%\label{centrality_section}

Measurements suggest that the nuclear modification factor has dependance on collision centrality~\cite{Aad:2015gta, Markus:2015}. Figure~\ref{fig:WpPb_010} shows the $W^{+}$ and $W^{-}$ boson differential yields in lepton pseudorapidity (upper panel) together with ratio to the model prediction (middle panel) and lepton charge asymmetry (bottom panel)  for events in the 0-10\% centrality class. There appears to be a dependence of the shape of the pseudorapidity distributions of both positively and negatively charged muons from \Wboson bosons on centrality. The data shown in the middle panel suggests the presence of a slope in most central collisions. The asymmetry shown in the lower panel also shows deviation from the prediction. This is similar to the trend observed in the \Zboson boson measurement as discussed in~\cite{Aad:2015gta}.

\begin{figure}[ht!]
\centering
\includegraphics[width=0.42 \textwidth]{fig_03c}
\caption{The upper panel shows the cross section of the $W^{+}$ and $W^{-}$ production as a function of the lepton pseudorapidity compared to model prediction based on CT10 PDF set measured in 0-10\% centrality interval of the \pPb data. The middle panel shows the data-to-model ratios for and the lower panel shows the lepton charge asymmetry compared to the same model~\cite{Markus:2015}.}
\label{fig:WpPb_010}
\end{figure} 


\begin{figure}[ht!] %[ht!] 
\centering
\includegraphics[width=0.45 \textwidth]{RpA_Z_4090_Prelim.pdf}
%\includegraphics[width=0.45 \textwidth]{RpA_Z_1040_Prelim.pdf}
\includegraphics[width=0.45 \textwidth]{RpA_Z_010_Prelim.pdf}
%\caption{The nuclear modification factor in most peripheral (40-90\%), mid-central (10-40\%) and most central events (0-10\%). The uncertainties on each point include statistical and systematic uncertainties from \pPb and \pp measurements.  The band around unity represents the uncertainty of \avgTppb and luminosity~\cite{me:2016}.}
\caption{The nuclear modification factor in most peripheral (40-90\%) and most central events (0-10\%). The uncertainties on each point include statistical and systematic uncertainties from \pPb and \pp measurements.  The band around unity represents the uncertainty of \avgTppb and luminosity~\cite{me:2016}.}
\label{fig:Rpa_centrality}
\end{figure}

%The dependance of the nuclear modification on the centrality in \pPb events has been studied using the measurement of \Zboson boson cross section in \pp collisions. 
Nuclear modification factor as a function of \yZ was measured for three different centrality classes of \pPb events. Results suggest that the asymmetry observed in the rapidity distribution of the \RpPb is more pronounced for more central collisions as shown in Figure~\ref{fig:Rpa_centrality}.  To quantify the change in asymmetry in different centrality classes, each $\RpPb^{\text{cent}}$ distribution is fitted to a linear function.  The resultant slopes for 40-90\%, 10-40\% and 0-10\% centrality bins are $0.02 \pm 0.04$, $-0.05 \pm 0.03$ and $-0.14 \pm 0.04$, respectively.  The uncertainties come from the fitting and are dominated by the uncertainties of the \pPb data. \par

%\vspace*{\fill}
%\newpage

\section*{Acknowledgements}
%\label{Acknowledgements}
This research is supported by the Israel Science Foundation (grant 1065/15) and by the MINERVA Stiftung with the funds from the BMBF of the Federal Republic of Germany.

%% The Appendices part is started with the command \appendix;
%% appendix sections are then done as normal sections
%% \appendix

%% \section{}
%% \label{}

%% References
%%
%% Following citation commands can be used in the body text:
%% Usage of \cite is as follows:
%%   \cite{key}         ==>>  [#]
%%   \cite[chap. 2]{key} ==>> [#, chap. 2]
%%

%% References with BibTeX database:
%\nocite{*}
\bibliographystyle{elsarticle-num}
%\bibliography{jos}

\bibliography{Z_2015_paper,ATLAS}


%% Authors are advised to use a BibTeX database file for their reference list.
%% The provided style file elsarticle-num.bst formats references in the required Procedia style

%% For references without a BibTeX database:

% \begin{thebibliography}{00}

%% \bibitem must have the following form:
%%   \bibitem{key}...
%%

% \bibitem{}

% \end{thebibliography}

\end{document}
%Acknowlegdement
%This research is supported by the Israel Science Foundation (grant 1065/15) and by the MINERVA Stiftung with the funds from the BMBF of the Federal Republic of Germany. 

%%
%% End of file `nuphbp-template.tex'. 