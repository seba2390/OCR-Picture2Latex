\documentclass[runningheads]{llncs}



\usepackage{amsmath,amssymb}
\usepackage{bussproofs}
\usepackage{mathtools}
\usepackage{stmaryrd}
\usepackage{multicol}
\usepackage{color}
\usepackage[normalem]{ulem}
\usepackage{tikz}
\usetikzlibrary{shapes,arrows}
\usepackage{bcprules}
\usepackage{afterpage,array,rotating}
\usepackage{ifthen}
\newboolean{printBibInSubfiles}
\setboolean{printBibInSubfiles}{true} 
\def\bib{\ifthenelse{\boolean{printBibInSubfiles}}
        {\bibliographystyle{splncs04}
         \bibliography{ref,abbrv,koba}}
        {}
    }

%!TEX root = hopfwright.tex

\subsection{Constructing a Newton-like operator}
\label{s:newtonlike}

In this section and in the appendices we often suppress the subscript in $F=F_\epsilon$.
We will find solutions to the equation $F(\alpha ,\omega , c)=0$ by the
constructing a Newton-like operator $T$ such that fixed points of $T$
corresponds precisely to zeros of $F$. In order to construct the map $T$ we
need an operator $A^{\dagger}$ which is an approximate inverse of 
$DF(\bx_\epsilon)$. 
% Since
% $\bx_\epsilon$ is an approximate zero of $F_\epsilon$ up to order
% $\cO(\epsilon^2)$ correction terms,
We will use an approximation $A$ of 
$DF( \bx_\epsilon )$ that is linear in~$\epsilon$ and correct up to $\cO(\epsilon^2)$.
% (recall that $F(\bx_\epsilon)=\cO(\eps^2)$). 
Likewise, we define $A^{\dagger}$ to be linear in $\epsilon$ (and again correct up to $\cO(\epsilon^2)$). 

It will be convenient to use the usual identification $i_\C : \R^2 \to \C$ given by $i_\C (x,y) = x+iy $. We also use $\omega_0 := \pi/2$.
 % order
% Since $x(\epsilon)$ is only correct up to order $\cO(\epsilon^2)$, then it only makes sense to compute our approximate derivative up to order $\cO( \epsilon^2)$.

% \marginpar{Jonathan: I tried to be careful about the spaces here, but it all seems a bit of a distraction since everything is explicit in coordinates}
\begin{definition}\label{def:A}
We introduce the linear maps $A:  \R^2 \times \ell^K_0 \to \ell^1$ and 
$ A^{\dagger}:  \ell^1 \to  \R^2 \times \ell^K_0 $ by
\begin{alignat*}{1}
A &:= A_0 + \epsilon A_1 \, , \\
A^{\dagger} &:= A_0^{-1} - \epsilon A_0^{-1} A_1 A_0^{-1} \,  ,
%\label{eq:ADagger}
\end{alignat*}
where the linear maps $ A_0 , A_1 : \R^2 \times \ell^K_0 \to \ell^1$  are defined below. Writing $x=(\alpha,\omega,c)$, we set
\begin{alignat*}{1}
A_0	x = A_0 (\alpha,\omega,c) & := i_\C A_{0,1} 
\!\left[\!\! \begin{array}{c} \alpha \\ \omega \end{array} \!\!\right]  \e_1
 + A_{0,*}  c , \\
A_1 x =	A_1 (\alpha,\omega,c) & := i_\C  A_{1,2}
\!\left[\!\! \begin{array}{c} \alpha \\ \omega \end{array} \!\!\right]  \e_2
 + A_{1,*}  c .
%\label{eq:ApproxDFdef}
\end{alignat*}
Here the matrices $A_{0,1}$ and $A_{1,2}$ are given by
\begin{equation}
A_{0,1} := 
\left[
\begin{matrix}
0 & - \pp \\
-1  & 1 
\end{matrix} 
\right]
\qquad\text{and}\qquad
A_{1,2} := \frac{1}{5}
\left[
\begin{matrix}
-2 & 2-\tfrac{3 \pi}{2} \\
-4  & 2(2+\pi) 
\end{matrix}  
\right]  ,
\label{eq:defA12}
\end{equation}
and the linear maps $A_{0,*} : \ell^K_0 \to \ell^1_0$ and
$A_{1,*} : \ell^K_0 \to \ell^1$
are given by
\begin{equation*}
% A_{0,*} :& \ell^1_0 \to \ell^1_0
% &
% A_{1,*} :& \ell^1_0 \to \ell^1  \\
% %%%%%%
% A_{0,1} :& \{ \alpha, \omega\} \to \{ Re \, F_1 , Im\, F_1 \}
% &
% A_{1,2} :& \{ \alpha, \omega\} \to \{ Re \,F_2 , Im \, F_2 \}
% \end{align*}
% and given by the equations below, taking $ \omega_0 = \pp$.
% \begin{align}
A_{0,*} 	 := \tfrac{\pi}{2} ( i K^{-1} + U_{\omega_0}) 
\qquad\text{and}\qquad
A_{1,*} 	:= \tfrac{\pi}{2} L_{\omega_0} .
\end{equation*}
%%%%%%%%%%%%%%%%%%%%
% A_{0,1} := &
% \left[
% \begin{matrix}
% 0 & - \pp \\
% -1  & 1
% \end{matrix}
% \right]
% &
% A_{1,2} :=& \frac{1}{5}
% \left[
% \begin{matrix}
% -2 & 2-\tfrac{3 \pi}{2} \\
% -4  & 2(2+\pi)
% \end{matrix}
% \right]
% \label{eq:defA12}
% \end{align}
\end{definition}

Since $K$ and $U_{\omega_0}$ both act as diagonal operators, the inverse 
$A_{0,*}^{-1} : \ell^1_0 \to \ell^K_0$ of $A_{0,*}$ is given by
\begin{equation*}
	  (A_{0,*}^{-1} a)_k = \frac{2}{\pi} \frac{a_k}{ik+e^{-ik\omega_0}} 
	  \qquad\text{for all } k \geq 2.
\end{equation*} 
An explicit computation, which we leave to the reader, shows that these approximations are indeed correct up to $\cO(\epsilon^2)$. 
In particular, $A^{\dagger} = \left[ DF( \bx_\epsilon ) \right]^{-1} + \cO(\epsilon^2)$.
In Appendix~\ref{sec:OperatorNorms} several additional properties of these operators are derived. The most important one is the following.
% \note[J]{I've tried to make this change for the new injectivity bound throughout.} \note[JB]{Seems fine, but wouldn't it be nicer to write $\tfrac{\sqrt{10}}{4}$ instead of $\tfrac{5}{2 \sqrt{10}}$?} \note[J]{Yes it would, made changes below. }
\begin{proposition}
	\label{prop:Injective}
	For 
%\change[J]{$0 \leq \epsilon < \tfrac{5}{2} ( 4 + \sqrt{10})^{-1} \approx 0.349$}
	$0 \leq \epsilon < \tfrac{\sqrt{10}}{4} \approx 0.790$
	 the operator $ A^{\dagger}$ is injective. 
\end{proposition}
\begin{proof}
	In order to show that $ A^{\dagger}$ is injective we show that 
	it has a left inverse. 
	Note that $ A A^{\dagger} = I - \epsilon^2 ( A_1 A_0^{-1})^2$. 
	By Proposition \ref{prop:A1A0} it follows that 
%	\change[J]{$ \| A_1 A_0^{-1} \| \leq \tfrac{2}{5} ( 4 + \sqrt{10})$}
	 $ \| A_1 A_0^{-1} \| \leq \tfrac{2 \sqrt{10}}{5} $.  
	By choosing 
%\change[J]{$ \epsilon < \tfrac{5}{2} ( 4 + \sqrt{10})^{-1}$}
$ \epsilon < \tfrac{\sqrt{10}}{4}$ 
we obtain 
	$\|  \epsilon^2 ( A_1 A_0^{-1})^2 \| < 1$, whereby $ A A^{\dagger}$ is 
	invertible, and so $ A^{\dagger}$ is injective. 
\end{proof}


\begin{definition}
We define the operator $ T: \R^2 \times \ell^K_0 \to \R^2 \times \ell^K_0 $ by
\begin{equation*}
	T(x) :=  x - A^{\dagger} F(x) ,
\end{equation*}
	where  $F$ is defined in Equation~\eqref{eq:FDefinition}  and $A^{\dagger}$ in Definition~\ref{def:A}.
	We note that $F$, $A^{\dagger}$ and $T$ depend on the parameter $\epsilon \geq 0$, although we suppress this in the notation.
\end{definition}

% \note[J]{When we got the better bound on $\|A_1 A_{0}^{-1}\|$, then $A^{\dagger}$ being injective ceased to be a bottleneck for doing a Hopf bifurcation. I don't think we'd lose much if we just delete this remark.  }
% \begin{remark}
% 	\label{r:Injective}
% \remove[J]{
% 	If $A^{\dagger}$ is injective, which is true for
% 	$0 \le \epsilon <  \tfrac{5}{2} ( 4 + \sqrt{10})^{-1}$ by Proposition 3.2, then the fixed points of $T$ correspond bijectively with the zeros of $F$.
% 	Since the periodic solution having $ \epsilon_0 = \tfrac{5}{2} ( 4 + \sqrt{10})^{-1}$ corresponds approximately to $\bar{\alpha}_{\epsilon_0} = \pp + 0.090$, above this value we cannot use the Newton-like operator $T$ to reliably study the SOPS to Wright's equation.
% 	Hence $ \alpha = \pp + 0.09$ represents an upper bound for doing an $\cO(\epsilon^2)$ Hopf bifurcation analysis.}
% \end{remark}
%


\subsection{Explicit contraction bounds}
\label{s:contraction}


The map $T$ is not continuous on all of $\R^2 \times \ell^K_0$,
since $ U_{\omega} c $ is not continuous in $\omega$.
While continuity is ``recovered'' for terms of the form $A^{\dagger} U_{\omega} c$,  this is not the case for the nonlinear part $ - \alpha \epsilon A^{\dagger} [ U_{\omega} c ] * c$.  
% The problem is that while in the $ U_{\omega} c$ term and that  $ \tfrac{\partial}{\partial \omega} U_{\omega} = - i K^{-1} U_{\omega}$.
% Since  the map $ A^{\dagger}$ is approximately $\tfrac{2 }{\pi i} K$, then the  $  A^{\dagger}  U_{\omega} c$ component of $ T$ is continuous in $\omega$.
%
%For any $ \omega_1, \omega_2\in \R$  then $ \| U_{\omega_1} - U_{\omega_2} \|  = 2$ when $ 2 \pi $ does not divide $ \omega_1 - \omega_2$.  
%This is not a problem for the $ A^{\dagger} ( i \omega K^{-1} + \alpha U_{\omega} ) c$ component of $T$; 
%the and then $ A^{\dagger } U_{\omega}$ is continuous in $\omega$.  
% However, since $ U_{\omega}$ is inside a convolution in the nonlinear part, this type of simplification cannot happen.
% 
We overcome this difficulty by fixing some $ \rho > 0$ and restricting the domain of $T$ to sets of the form 
%$\R^2 \times X_\rho$, where
\[
  \R^2 \times  \{ c \in \ell^K_0 : \|K^{-1} c\| \leq \rho \} = \R^2 \times \ell_\rho.
\]
Since we wish to center the domain of $T$ about the approximate solution~$\bx_\epsilon$, we introduce the following definition, which uses a triple of radii $r \in \R^3_+$, for which it will be convenient to use two different notations:
\[
  r = ( r_{\alpha } , r_{\omega} , r_c) = (r_1,r_2,r_3).
\]
\begin{definition}
	Fix   $ r \in \R^3_+$ and $ \rho > 0$ and let  $ \bx_\epsilon = ( \balpha_\epsilon , \bomega_\epsilon , \bc_\epsilon )$ be as defined in Definition~\ref{def:xepsilon}. 
    We define the $\rho$-ball $B_\epsilon(r,\rho) \subset \R^2 \times \ell^1_0$
    of radius $r$ centered at $\bx_\epsilon$ to be the set of points satisfying 
\begin{alignat*}{1}
	|  \alpha -\balpha_\epsilon | & \leq  r_\alpha  \\
	| \omega - \bomega_\epsilon  | & \leq  r_{\omega} \\
	\| c - \bc_\epsilon  \| & \leq r_c \\
	\|K^{-1} c\| & \leq  \rho .
\end{alignat*}
\end{definition}

We want to show that $T$ is a contraction map on some $\rho$-ball 
$B_\epsilon(r,\rho) \subset \R^2 \times \ell^1_0$ using a Newton-Kantorovich argument. 
This will require us to develop a bound on $DT$ using some norm on  $ X$.  
Unfortunately there is no natural choice of norm on the product space $ X$. 
Furthermore, it will not become apparent if one norm is better than another until after  significant calculation.  
For this reason, we use a notion of an ``upper bound'' which allows us to delay our choice of norm. 
We first introduce the operator $\zeta:  X  \to \R^3_{+}$
which consists of the norms of the three components:
\[
  \LL(x) :=   ( |\pi_\alpha x|, |\pi_\omega x|, \|\pi_c x\| )^T \in \R^3_{+}
  \qquad\text{for any } x \in X.
\]
% \note[JB]{Propose to revert to using single overlines for the upper bound; the double overlines I had introduced just look ridiculous to me now. There is a potential for confusion with $\bx_\epsilon$, but I can live with it.} \note[J]{I would also prefer using only one overline. }
\begin{definition}[upper bound]\label{def:upperbound}
We call $\upperbound{x} \in \R^3_+$ an upper bound on $x$ if $\LL(x) \leq \upperbound{x}$, where the inequality is interpreted componentwise in $\R^3$. 
Let $X'$ be a subspace of $X$ and let $X''$ be a subset of $X'$.   
An upper bound on a linear operator $A' : X' \to X $ over $X''$  is 
a $3 \times 3$ matrix $\upperbound{A'} \in \text{\textup{Mat}}(\R^3 , \R^3)$ such that
\[
   \LL(A' x ) \leq \upperbound{A'} \cdot \LL(x)  
     \qquad\text{for any }  x \in X'',
\]
where the inequality is again interpreted componentwise in $\R^3$. 
The notion of upper bound conveniently encapsulates bounds on the different components of the operator $A'$ on the product space $X$. Clearly the components of the matrix $\upperbound{A'}$ are nonnegative.


		% 	Let $ (\alpha , \omega , c) \in \R^2 \times \ell^1_0$ and $  \upperbound{x} = ( x_{ \alpha } , x_{\omega} , x_{c}) \in \R^3_+$.
		% 	Then $ \upperbound{x}$ is an \emph{upper bound} on $(\alpha , \omega , c)$ if $ | \alpha | \leq x_\alpha$ and $ | \omega | \leq x_{\omega}$ and $ \| c\|\leq x_{c}$.
		% Similarly, suppose that $ A' : X \to \R^2 \times \ell^1_0$ is a linear operator, defined on some domain $ X \subset \R^2 \times \ell^1_0$.
		% Then $ \upperbound{A'} \in Mat(\R^3 , \R^3)$ is an \emph{upper bound } on $ A'$  if $ \upperbound{A'} \cdot \upperbound{x} $ is an upper bound on $ A' x$ whenever $ \upperbound{x} \in \R^3_+$ is an upper bound on all $ x \in X$.
\end{definition}
		% The notion of upper bounds commutes with vector addition and matrix multiplication.
		% That is, if $\upperbound{x} , \upperbound{y} \in \R^3_+ $ are upper bounds on $ x,y \in \R^2 \times \ell^1_0$, then $ \upperbound{x} + \upperbound{y}$ is an upper bound on $x +y$.
		% Similarly, if we have two linear operators $A'$ and $A''$ with upper bounds
		% $\upperbound{A'}$ and $\upperbound{A''}$, respectively, then $ \upperbound{A'} \cdot \upperbound{A''}$ is an upper bound for $ A' \circ A''$.
		% Furthermore, if $\upperbound{A'} \in Mat(\R^3,\R^3)$ is an upper bound, then the entries of this matrix are necessarily non-negative.
For example, in Proposition \ref{prop:A0A1} we calculate an upper  bound on the map $A_0^{-1} A_1$.  
As for the domain of definition of $T$, in practice we use $X' = \R^2 \times  \ell^K_0  $ and  $X'' = \R^2 \times  \ell_\rho  $.
The subset $X''$ does not always affect the upper bound calculation (such as in Proposition \ref{prop:A0A1}). 
However, operators such as $U_{\omega} - U_{\omega_0}$ have upper bounds which contain $\rho$-terms (see for example Proposition \ref{prop:OmegaDerivatives}).

Using this terminology, we state a ``radii polynomial'' theorem, which allows us to check whether $T$ is a contraction map. This technique has been used frequently in a computer-assisted setting in the past decade. Early application include~\cite{daylessardmischaikow,lessardvandenberg}, while a previous implementation in the context of Wright's delay equation can be found in~\cite{lessard2010recent}. 
Although we use radii polynomials as well, our approach differs significantly from the computer-assisted setting mentioned above. 
While we do engage a computer (namely the Mathematica file~\cite{mathematicafile}) to optimize our quantitative results, the analysis is performed essentially in terms of pencil-and-paper mathematics (in particular, our operators do not involve any floating point numbers).
In our current setup we employ \emph{three} radii as a priori unknown variables,
which builds on an idea introduced in~\cite{vandenberg}.
We note that in most of the papers mentioned above the notation of $A$ and $A^\dagger$ is reversed compared to the current paper.

As preparation, the following lemma (of which the proof can be found in Appendix~\ref{sec:CompactDomain})  provides an explicit choice for $\rho$, as a function of $\epsilon$ and $r$, for which we have proper control on the image of $B_\epsilon(r,\rho)$ under $T$.
\begin{lemma}\label{lem:Crho}
For any $\epsilon \geq 0$ and $r \in\R^3_+$, let $C=C(\epsilon,r)$ be given  by Equation~\eqref{eq:RhoConstant}. 
If $C(\epsilon,r) >0$  then 
% Proposition~\ref{prop:DerivativeEndo} states that
\begin{equation}\label{e:Cepsr} 
  \| K^{-1} \pi_c  T(x) \| \leq \rho 
  \quad\text{whenever } x \in B_\epsilon(r,\rho) \text{ and } \rho \geq C(\epsilon,r).
\end{equation}
%%%%\marginpar{Jonathan: please fix appendix to reflect this (and define $C$ there)}
Moreover, $C(\epsilon,r)$ is nondecreasing in $\epsilon$ and $r$. 
\end{lemma}

\begin{proof}
See Proposition~\ref{prop:DerivativeEndo}.
\end{proof}


\begin{theorem}
	\label{thm:RadPoly}
	Let  
%\change[J]{$0 \leq \epsilon < \tfrac{5}{2} ( 4 + \sqrt{10})^{-1} $}
 $0 \leq \epsilon < \tfrac{\sqrt{10}}{4} $  
 and fix $r = (r_\alpha, r_\omega, r_c) \in \R^3_+$. Fix $\rho > 0$ such that $ \rho \geq C(\epsilon,r)$, as given by Lemma~\ref{lem:Crho}.
 % as in Proposition \ref{prop:DerivativeEndo} (REFORMULATE TO POINT TO THE PREPARATION ABOVE).
%
Suppose that $Y(\epsilon) $ is an upper bound on $ T(\bx_\epsilon) - \bx_\epsilon$ and $Z(\epsilon , r ,\rho) $ a (uniform) upper bound on $ DT(x) $ for all $ x \in B_\epsilon(r,\rho)$. 
Define the \emph{radii polynomials}
$P :\R^5_+ \to \R^3 $  by 
 \begin{equation}
 \label{eq:RadPolyDef}
  P(\epsilon,r,\rho) := Y(\epsilon) - \left[ I - Z( \epsilon,r,\rho) \right] \cdot r  \,  .
 \end{equation}
If each component of $P(\epsilon,r,\rho)$ is negative, then there is  a unique $\hat{x}_\epsilon \in B_\epsilon( r , \rho)$ such that $F(\hat{x}_\epsilon) =0$. 
\end{theorem}

\begin{proof}    
Let $r \in \R^3_+$ be a triple such that $P(\epsilon,r,\rho)<0$.
By Proposition \ref{prop:Injective}, if 
%\change[J]{$\epsilon < \tfrac{5}{2} ( 4 + \sqrt{10})^{-1} $}
$\epsilon <\tfrac{\sqrt{10}}{4} $
then $ A^{\dagger}$ is injective. 
Hence $ \hat{x}_{\epsilon} $ is a fixed point of $T$ if and only if $ F( \hat{x}_{\epsilon}) = 0$.  
In order to show  there is a unique fixed point $ \hat{x}_{\epsilon}$, we show that $T$ maps  $ B_{\epsilon}(r,\rho) $ into itself and that $ T $ is a contraction mapping. 

We first show that $T: B_\epsilon(r,\rho) \to B_\epsilon(r,\rho)$. 
Since $ \rho \geq C(\epsilon,r)$ then by Equation~\eqref{e:Cepsr} it follows that $ \| K^{-1} \pi_c T( x) \| \leq \rho$ for all $ x \in B_\epsilon(r,\rho)$.
In order to show that $T(x) \in B_\epsilon(r,\rho)$, it suffices to show that $ r=(r_\alpha , r_\omega, r_c)$ is an upper bound on $ T(x) - \bx_\epsilon$
for all $ x \in B_\epsilon(r,\rho)$.  
We decompose 
%by breaking $T(x) - \bx_\epsilon$ into two parts: 
\begin{equation}\label{e:Tsplit}
	T(x) - \bx_\epsilon = [T(\bx_\epsilon) -\bx_\epsilon] +
	[T(x) - T(\bx_\epsilon)],
\end{equation}
and estimate each part separately. Concerning the first term,
by assumption, $Y(\epsilon)$ is an upper bound on $T(\bx_\epsilon) - \bx_\epsilon$. 
%
Concerning the second term, we claim that $ Z(\epsilon,r,\rho) \cdot r$ is an upper bound on $T(x) - T(\bx_\epsilon)$.
Indeed, we have the following somewhat stronger bound: 
% if $ x,y\in B_\epsilon(r,\rho)$ and $\upperbound{\xi}$ is an upper bound on $y-x$, then
% $Z(\epsilon,r,\rho) \cdot \upperbound{\xi}$ is an upper bound on $T(y) - T(x)$,
% i.e.,
\begin{equation}\label{e:DTisboundedbyZ}
	\LL(T(y) - T(x)) \leq Z(\epsilon,r,\rho) \cdot \LL(y-x)
	\qquad\text{for all } x,y \in B_\epsilon(r,\rho) .
\end{equation}
The latter follows from the mean value theorem, since 
$T$ is continuously Fr\'echet differentiable on $B_\epsilon(r,\rho)$.
%
% MORE DETAILED ARGUMENT PROBABLY NOT NEEDED?
% \begin{equation}
% \label{eq:ZIntegrationBound}
% 	T( y) - T(x) \leq Z(\epsilon,r,\rho) \cdot \upperbound{\xi}
% \end{equation}
% Since $T$ is continuously Frechet differentiable on $B_\epsilon(r,\rho)$,
% then it follows that $T(y) - T(x) = \int_x^y DT( z) dz  $.
% %Since $ B_\epsilon(r,\rho)$ is convex, then $z= x+ s(y-x) \in B_\epsilon(r,\rho)$ for all $s \in [0,1]$.
% By assumption $Z(\epsilon,r,\rho)$ is an upper bound on $DT(z)$ for all $z \in B_\epsilon(r,\rho)$.
% If $ \upperbound{\xi} $ is an upper bound on $ y-x$, then we obtain the inequality $ T(y) - T(x) \leq \int_0^1 Z(\epsilon,r,\rho) \cdot \upperbound{\xi} \, ds$, from which Equation \ref{eq:ZIntegrationBound} follows.
%
Since $r$ is an upper bound on $x - \bx_\epsilon$ for all $ x \in B_\epsilon(r,\rho)$, we find, by using~\eqref{e:Tsplit}, that  
% have obtained that
% \begin{eqnarray}
% 	T(\bx_\epsilon) - \bx_\epsilon &=& \left[ T(\bx_\epsilon) - \bx_\epsilon \right] +
% 	\left[ T(x) - T(\bx_\epsilon) \right] \\
% 	&\leq& Y(\epsilon) + Z(\epsilon,r,\rho) \cdot r
% \end{eqnarray}
% By assumption each component of Equation \ref{eq:RadPolyDef} is negative, so
$Y(\epsilon) + Z(\epsilon,r,\rho) \cdot r \leq r$ (with the inequality, interpreted componentwise, following from $P(\epsilon,r,\rho)<0$) is an upper bound on $T(x) - \bx_\epsilon$
for all $ x \in B_\epsilon(r,\rho)$.  
%It follows that $r$ is an upper bound on $T(x) - \bx_\epsilon $. 
That is to  say, if all of the  radii polynomials are negative, 
then  $T$ maps $B_\epsilon(r,\rho) $ into itself.

To finish the proof we show that $T$ is a contraction mapping. 
We abbreviate $Z=Z(\epsilon,r,\rho)$ and  recall that $r=(r_\alpha,r_\omega,r_c)=(r_1,r_2,r_3) \in \R^3_+$
is such that $Z \cdot r < r$, hence for some $\kappa <1$ we have
\begin{equation}\label{e:defkappa}
  \frac{(Z \cdot r)_i}{r_i} \leq \kappa  \qquad\text{for } i=1,2,3.
\end{equation}

We now need to choose a norm on $X$. 
We define a norm $ \| \cdot \|_r$ on elements $x = (\alpha,\omega,c) \in X$
by
\[  
\| (\alpha, \omega, c) \|_r := \max 
\left\{  		  
	 \frac{|\alpha|}{r_\alpha},
	 \frac{|\omega|}{r_\omega},
	 \frac{\|c\|}{r_{c}} \right\} , 
\]
or
\[
  \|x\|_r = \max_{i=1,2,3} \frac{ \LL(x)_i}{r_i}
  \qquad \text{for all } x \in X.
\]
% We also introduce the compatible norm $ \| \cdot \|_{\tilde{r}}$ on $\R^3$ by
% $ \| (y_1, y_2, y_3) \|_{\tilde{r}} = \max_{i=1,2,3 }
% \{\frac{|y_i|}/{r_i} \}$, so that $\|x\|_r = \| \LL(x) \|_{\tilde{r}}$ for all $x \in X$.
%
By using the upper bound $Z$, we bound the Lipschitz constant of $T$ on $B_\epsilon(r, \rho)$ as follows:
\begin{alignat*}{1}
 \| T(y) - T(x) \|_r 
 % &= \|\LL(T(y) - T(x)) \|_{\tilde{r} } \\
    &= \max_{i=1,2,3} \frac{\LL(T(y) - T(x))_i} {r_i} \\
    &\leq  \max_{i=1,2,3}  \frac{(Z \cdot \LL(y-x))_i}{r_i} \\
    &\leq  \max_{i=1,2,3} \max_{j=1,2,3}\frac{\LL(y-x)_j}{r_j}  
				   \frac{(Z \cdot r )_i}{r_i} \\
    & = \| y-x \|_r \max_{i=1,2,3} \frac{(Z \cdot r )_i}{r_i} \\
    & \leq \kappa \| y-x \|_r,
\end{alignat*}
where we have used~\eqref{e:DTisboundedbyZ} and~\eqref{e:defkappa} with $\kappa<1$.
% \sup_{x,y \in B_\epsilon(r,\rho)} \frac{\|T(y) - T(x) \|_r}{\| y- x\|_r}
% \leq
% \sup_{ x,y \in B_\epsilon(r,\rho)}
% \frac{\left \|
% Z(\epsilon,r,\rho) \cdot \upperbound{\xi}
% \right\|_{\tilde{r}} }{  \| y -x\|_{r}} ,
% \]
% where $\upperbound{\xi}$ is any upper bound on $y-x$, as before.
% \marginpar{THIS IS STILL NOT ENTIRELY CLEAR!}
% If $u \in \R^3$ and $ \|u\|_{\tilde{r}} =1$, then $ \| Z \cdot u\|_{\tilde{r}}$ is maximized when $u=r$.
% Hence $ Lip(T) \leq \| Z(\epsilon,r,\rho) \cdot r \|_r$.
% Since all of the radii polynomials are negative, then $ Z \cdot r < r$component wise, thus proving that $ \|Z \cdot r\|_r <1$ and
Hence $T:B_{\epsilon}(r,\rho) \to B_{\epsilon}(r,\rho)$ is a contraction with respect to the $\| \cdot \|_r$ norm.

% We have thereby proved that  $T:B_{\epsilon}(r,\rho) \to B_{\epsilon}(r,\rho)$ is a contraction mapping.
Since $B_\epsilon(r,\rho)$ with this norm is a complete metric space, by the Banach fixed point theorem $T$~has a unique fixed point $ \hat{x}_\epsilon \in B_\epsilon(r,\rho)$. 
Since $A^\dagger$ is injective,  it follows that $ \hat{x}_\epsilon$   is the unique point in $B_\epsilon(r,\rho)$ for which $ F(\hat{x}_\epsilon) =0$. 
\end{proof}

\begin{remark}\label{r:boundDT}
Under the assumptions in Theorem~\ref{thm:RadPoly},
essentially the same calculation as in the proof above
leads to the estimate
\[
  \| DT(x) y \|_r \leq \kappa \|y\|_r 
  \qquad \text{for all } y \in \R^2 \times \ell^K_0 , 
  \, x \in B_\epsilon(r,\rho),
\]
where $\kappa := \max_{i=1,2,3} (Z\cdot r)_i / r_i$.
\end{remark}


In Appendix \ref{sec:YBoundingFunctions} and Appendix \ref{sec:BoundingFunctions} we construct explicit upper bounds 
$Y(\epsilon)$ and $ Z(\epsilon,r,\rho)$, respectively.  
These functions are constructed such that their components are (multivariate) polynomials in $\epsilon$, $r$ and $ \rho$ with nonnegative coefficients, hence they are increasing in these variables. 
This construction enables us to make use of the uniform contraction principle. 

\begin{corollary}\label{cor:eps0}
Let 
%\change[J]{$0 <\epsilon_0 < \tfrac{5}{2} ( 4 + \sqrt{10})^{-1} $}
 $0 < \epsilon_0 < \tfrac{\sqrt{10}}{4} $ 
and fix some $r = (r_\alpha, r_\omega, r_c) \in \R^3_+$.  
Fix $\rho > 0$ such that $ \rho \geq C(\epsilon_0,  r)$, as given by Lemma~\ref{lem:Crho}.
% as in Proposition \ref{prop:DerivativeEndo}. 
%
Let $Y(\epsilon)$ and $Z(\epsilon,r,\rho)$ be the upper bounds as given in  Propositions~\ref{prop:Ydef} and~\ref{prop:Zdef}. 
Let the radii polynomials $P$ be defined by Equation~\eqref{eq:RadPolyDef}.


If each component of  $P(\epsilon_0, r,\rho)$ is negative, 
then for all $ 0 \leq \epsilon \leq \epsilon_0$ there exists a unique $ \hat{x}_\epsilon \in B_\epsilon(  r , \rho)$ such that $ F(\hat{x}_\epsilon) =0$.  
The solution $\hat{x}_\epsilon$ depends smoothly on $\epsilon$.
\end{corollary}
\begin{proof} 
	Let $0 \leq  \epsilon \leq \epsilon_0$ be arbitrary.
	Because $\rho \geq C(\epsilon_0, r) \geq C(\epsilon, r)$ by Lemma~\ref{lem:Crho},
	Theorem~\ref{thm:RadPoly} implies that it suffices to show that $ P(\epsilon, r ,\rho) <0$. 	
Since  the bounds 
$Y(\epsilon)$ and $ Z(\epsilon,r,\rho)$ are monotonically increasing in their arguments, it follows that $ P(\epsilon,r,\rho) \leq P(\epsilon_0,r,\rho) <0$.  
Continuous and smooth dependence on $\epsilon$ of the fixed point follows from the uniform contraction principle (see for example~\cite{ChowHale}). 
\end{proof}


Given the upper bounds $ Y(\epsilon)$ and $ Z( \epsilon ,r , \rho)$, 
trying to apply Corollary~\ref{cor:eps0} amounts to finding values of $ \epsilon, r_\alpha, r_\omega, r_c,\rho$ for which the radii polynomials are negative.
Selecting a value for $ \rho$ is straightforward: all estimates improve with smaller values of $\rho$, and Proposition \ref{prop:DerivativeEndo} (see also Lemma~\ref{lem:Crho}) explicitly describes the smallest allowable choice of $\rho$ in terms of $ \epsilon,r_\alpha,r_\omega,r_c$. 

Beyond selecting a value for $ \rho$, it is difficult to pinpoint what constitutes an ``optimal'' choice of these variables. 
In general it is interesting to find such  viable radii (i.e.\ radii such that $P(r)<0$) which are both large and small.  
The smaller radius tells us how close the true solution is to our approximate solution. 
The larger radius tells us in how large a neighborhood our solution is unique.  With regard to $\epsilon$, larger values allow us to describe functions whose first Fourier mode is large. However this will ``grow'' the smallest viable radius and ``shrink'' the largest viable radius. 

Proposition \ref{prop:bigboxes} presents two selections of variables which satisfy the hypothesis of Corollary~\ref{cor:eps0}.  
We check the hypothesis is indeed satisfied by using interval arithmetic.
All details are provided in the Mathematica file~\cite{mathematicafile}. 
While the specific numbers used may appear to be somewhat arbitrary (see also the discussion in Remark~\ref{r:largeradii})  they have been chosen to be used later in Theorem 
\ref{thm:WrightConjecture} and Theorem \ref{thm:UniqunessNbd}.  


%%%
%%%BY DOING SOME CHOICES THAT HAVE NO MOTIVATION AT THIS POINT, BUT THAT WILL TURN OUT TO BE USEFUL IN SECTION~\ref{s:global} WE PROVE THE FOLLOWING USING MATHEMATICA  FILES.\marginpar{todo}

\begin{proposition}
		\label{prop:bigboxes}
Fix the constants $ \epsilon_0$, $(r_\alpha, r_\omega,r_c)$  and $\rho$ according to one of the following choices:
% \begin{enumerate}
% \item[\textup{(a)}]  $ \epsilon_0 = 0.029 $ and $ (r_\alpha , r_ \omega , r_c) = (  0.21, \, 0.16 , \, 0.09 ) $ and $\rho = 1.01$;
% \item[\textup{(b)}]  $ \epsilon_0 = 0.087 $ and $ (r_\alpha , r_ \omega , r_c) = (  0.1501, \, 0.0626 , \, 0.2092 ) $ and $\rho = 0.5672$.
% \end{enumerate}
% \note[J]{Version with new numbers below}
\begin{enumerate}
	\item[\textup{(a)}]  $ \epsilon_0 = 0.029 $ and $ (r_\alpha , r_ \omega , r_c) = (  0.13, \, 0.17 , \, 0.17 ) $ and $\rho = 1.78$; 
	\item[\textup{(b)}]  $ \epsilon_0 = 0.09 $ and $ (r_\alpha , r_ \omega , r_c) = (  0.1753, \, 0.0941 , \, 0.3829 ) $ and $\rho = 1.5940$. 
\end{enumerate}
For either of the choices (a) and (b) we have the following: 
for all $0 \leq \epsilon \leq \epsilon_0$ there exists a unique point 
$(\hat{\alpha}_\epsilon,\hat{\omega}_\epsilon,\hat{c}_\epsilon) \in B_{\epsilon}(r,\rho)$ 
satisfying $F_\epsilon(\hat{\alpha}_\epsilon,\hat{\omega}_\epsilon,\hat{c}_\epsilon) = 0$ and 
\[ 	
 | \hat{\alpha}_\epsilon - \balpha_\epsilon| \leq r_\alpha , 
 \quad
 |\hat{\omega}_\epsilon - \bomega_\epsilon| \leq  r_\omega  ,
 \quad
 \| \hat{c}_\epsilon - \bc_\epsilon\| \leq r_c     ,
 \quad
 \| K^{-1} \hat{c}_\epsilon \| \leq \rho  .
\]
\end{proposition}
\begin{proof}
In the Mathematica file~\cite{mathematicafile}  we check, using interval arithmetic, that  $\rho \geq C(\epsilon_0, r)$ and  the radii polynomials $P(\epsilon_0,r,\rho)$ are negative for the choices (a) and (b). The result then follows from Corollary~\ref{cor:eps0}.	
\end{proof}


\begin{remark}\label{r:largeradii}	
In Proposition~\ref{prop:bigboxes} we aimed for large balls on which the solution is unique.
Even for a fixed value of $ \epsilon$, it is not immediately obvious how to find a ``largest'' viable radius $r$, 
since $r$ has three components. In particular, there is a trade-off between the different components of $r$. On the other hand, as explained in Remark~\ref{r:smallradii}, no such difficulty arises when looking for a ``smallest'' viable radius.
\end{remark}




We will also need a rescaled version of the radii polynomials, which takes into account the asymptotic behavior of the bound $Y$ on the residue $T(\bar{x}_\epsilon) -\bar{x}_\epsilon = - A^\dagger F(\bx_\epsilon)$  as $\epsilon \to 0$, namely it is of the form
$Y(\epsilon)= \epsilon^2 \tilde{Y}(\epsilon)$,
see Proposition~\ref{prop:Ydef}.
The proofs of the following monotonicity properties can be found in 
Appendices~\ref{sec:YBoundingFunctions} and~\ref{sec:BoundingFunctions}. 
\begin{lemma}\label{lem:YZ}
Let $\epsilon \geq 0$, $\rho >0$ 
and $r \in\R^3_+$. 
Then there are upper bounds
$Y(\epsilon) =\epsilon^2 \tilde{Y}(\epsilon)$ on $ T(\bx_\epsilon) - \bx_\epsilon$ and a (uniform) upper bound 
$Z(\epsilon , r ,\rho) $  on $ DT(x) $ for all $ x \in B_\epsilon(r,\rho)$.
These bounds are given explicitly by Propositions~\ref{prop:Ydef} and~\ref{prop:Zdef}, respectively. Moreover, $\tilde{Y}(\epsilon)$ is nondecreasing in $\epsilon$,
while $Z(\epsilon , r ,\rho) $ is nondecreasing in $\epsilon$, $r$ and $\rho$.
\end{lemma}

This implies, roughly speaking, that if we are able to show that $T$ is a contraction map on 
$B_{\epsilon_0}( \epsilon_0^2 \rr,\rho)$ for a particular choice of $ \epsilon_0$, then it will be a contraction map on $B_\epsilon( \epsilon^2 \rr,\rho)$ for all $ 0 \leq \epsilon \leq \epsilon_0$. Here, and in what follows, we use the notation $r = \epsilon^2 \rr$ for the $\epsilon$-scaled version of the radii. 



\begin{corollary}
	\label{cor:RPUniformEpsilon}
	Let  
	 $0 < \epsilon_0 < \tfrac{\sqrt{10}}{4} $ 
	and fix some $\rr = (\rr_\alpha, \rr_\omega, \rr_c) \in \R^3_+$. 
	Fix $\rho > 0$ such that $ \rho \geq C(\epsilon_0, \epsilon_0^2 \rr)$, as given by Lemma~\ref{lem:Crho}. 
	Let $Y(\epsilon)$ and $Z(\epsilon,r,\rho)$ be the upper bounds as given by Lemma~\ref{lem:YZ}.  
Let the radii polynomials $P$ be defined by~\eqref{eq:RadPolyDef}. 

	If each component of  $P(\epsilon_0,\epsilon_0^2 \rr,\rho)$ is negative, 
	then for all $ 0 \leq \epsilon \leq \epsilon_0$ 
	there exists a unique $ \hat{x}_\epsilon \in B_\epsilon(\epsilon^2  \rr , \rho)$ 
	such that $ F(\hat{x}_\epsilon) =0$. 
	Furthermore, $\hat{x}_\epsilon$ depends smoothly on $\epsilon$.
\end{corollary}

\begin{proof}
	 Let $0 \leq  \epsilon < \epsilon_0$ be arbitrary.
	 Because $\rho \geq C(\epsilon_0,\epsilon_0^2 \rr) \geq C(\epsilon,\epsilon^2 \rr)$ by Lemma~\ref{lem:Crho},
	Theorem~\ref{thm:RadPoly} implies that it suffices to show that $ P(\epsilon,\epsilon^2 \rr ,\rho) <0$. 
	By using the monotonicity provided by Lemma~\ref{lem:YZ}, we obtain
	\begin{alignat*}{1}
		P(\epsilon,\epsilon^2 \rr ,\rho) &= Y(\epsilon) 
- \left[ I - Z(\epsilon,\epsilon^2 \rr,\rho)\right] \cdot \epsilon^2 \rr \\
		&=  (\epsilon / \epsilon_0)^{2} \left[ \epsilon_0^2   
		  \tilde{Y}(\epsilon) - \epsilon_0^2 \rr 
		+  Z(\epsilon,\epsilon^2 \rr,\rho) \cdot \epsilon_0^2 \rr  \right] \\
		&\leq  (\epsilon / \epsilon_0)^{2} \left[ \epsilon_0^2  
		  \tilde{Y}(\epsilon_0)  - \epsilon_0^2 \rr 
   +  Z(\epsilon_0,\epsilon_0^2 \rr,\rho) \cdot \epsilon_0^2 \rr  \right] \\
		&= (\epsilon / \epsilon_0)^{ 2} P(\epsilon_0 , \epsilon_0^2 \rr,\rho) \\
		& < 0,
	\end{alignat*}
where inequalities are interpreted componentwise in $\R^3$, as usual.
\end{proof}




%%%%%
%%%%%		THIS IS THE OLD VERSION OF THE UNIFORM \EPSILON^2 THEOREM
%%%%%
%%%%%\begin{corollary}
%%%%%	\label{prop:RPUniformEpsilon}
%%%%%	Let $ 0 < \epsilon_0 < \tfrac{5}{2} ( 4 + \sqrt{10})^{-1}$ and fix some $r = (r_\alpha, r_\omega, r_c) \in \R^3_+$ and 
%%%%%	fix $ k \in \{ 0,1,2\}$.  
%%%%%	Fix $\rho > 0$ such that $ \rho \geq C(\epsilon_0, (\epsilon_0)^2 r)$, as given by Lemma~\ref{lem:Crho}. 
%%%%%	Let $Y(\epsilon)$ and $Z(\epsilon,r,\rho)$ be the upper bounds as given by~\ref{lem:YZ}.  
%%%%%	Let the radii polynomials $P$ be defined by~\eqref{eq:RadPolyDef}.
%%%%%	If each component of  $P(\epsilon_0,{\epsilon_0}^k r,\rho)$ is negative, 
%%%%%	then for all $ 0 \leq \epsilon \leq \epsilon_0$ there exists a unique $ \hat{x}_\epsilon \in B_\epsilon(\epsilon^k  r , \rho)$ such that $ F(\hat{x}_\epsilon) =0$. Furthermore, $\hat{x}_\epsilon$ depends smoothly on $\epsilon$.
%%%%%\end{corollary}
%%%%%
%%%%%\begin{proof}
%%%%%	Let $0 \leq  \epsilon < \epsilon_0$ be arbitrary.
%%%%%	Because $\rho \geq C(\epsilon_0,\epsilon_0^k r) \geq C(\epsilon_0,\epsilon_0^k r)$ by Lemma~\ref{lem:Crho},
%%%%%	Theorem~\ref{thm:RadPoly} implies that it suffices to show that $ P(\epsilon,\epsilon^k r ,\rho) <0$. 
%%%%%	By using the monotonicity provided by Lemma~\ref{lem:YZ}, we obtain
%%%%%	\begin{alignat*}{1}
%%%%%	P(\epsilon,\epsilon^k r ,\rho) &= Y(\epsilon) - \left[ I - Z(\epsilon,\epsilon^k r,\rho)\right] \cdot \epsilon^k r \\
%%%%%	&=  (\epsilon / \epsilon_0)^{k} \left[ \epsilon_0^k  \epsilon^{2-k}  \tilde{Y}(\epsilon) - \epsilon_0^k r +  Z(\epsilon,\epsilon^k r,\rho) \cdot \epsilon_0^k r  \right] \\
%%%%%	&\leq  (\epsilon / \epsilon_0)^{k} \left[ \epsilon_0^k  \epsilon_0^{2-k}  \tilde{Y}(\epsilon_0)  - \epsilon_0^k r +  Z(\epsilon_0,\epsilon_0^k r,\rho) \cdot \epsilon_0^k r  \right] \\
%%%%%	&= (\epsilon / \epsilon_0)^{ k} P(\epsilon_0 , \epsilon_0^k r,\rho) \\
%%%%%	& < 0,
%%%%%	\end{alignat*}
%%%%%	where inequalities are interpreted componentwise in $\R^3$, as usual.
%%%%%\end{proof}
%%%%%




%%%%%%%%%%%%%%%%%%%%%%%%%%%%%%%%%%%%%%%%%%%%%%%%%%%%%%%%%%%%%%%%%%%%%%%%%%%%
%\subsection{Application of Radii Polynomials}

%\begin{remark}


These $\epsilon$-rescaled variables are used in
Proposition~\ref{prop:TightEstimate} below to derive \emph{tight} bounds on the
solution (in particular, tight enough to conclude that the bifurcation is
supercritical). The following remark explains that the monotonicity properties of
the bounds $Y$ and $Z$ imply that looking for small(est) radii which satisfy $P(r)<0$, is
a well-defined problem.


\begin{remark}\label{r:smallradii}
The set $R$ of radii for which the radii polynomials are negative is given by 
\[
  R := \{ r \in \R^3_+ : r_j > 0,  P_i(r) < 0 \text{ for } i,j=1,2,3 \} .
\] 
This set has the property that if
	$r,r' \in R$, then $r''\in R$, where $r''_j=\min\{ r_j,r'_j\}$.
Namely, the main observation is that we can write 
	$P_i(r)= \tilde{P}_i(r)-r_i$, where $\partial_{r_j} \tilde{P}_i \geq 0$ for all $i,j=1,2,3$.
Now fix any $i$; we want to show that $P_i(r'') < 0$.	
We have either $r''_i=r_i$ or $r''_i=r'_i$, hence assume $r''_i=r_i$ (otherwise
just exchange the roles of $r$ and $r'$). We infer that $P_i(r'') \leq P_i(r) <
0$, since $\partial_{r_j} P_i \geq 0$ for $j \neq i$.
We conclude that there are no trade-offs in looking for minimal/tight radii, as
opposed to looking for large radii, see Remark~\ref{r:largeradii}.
\end{remark}

%%%
%%%The optimization problem is simplified to a degree because the region $ P(\epsilon_0,r,\rho_0) <0$ is convex for fixed $\epsilon_0$ and $ \rho_0 $.  
%%%This is because the function $Z(\epsilon,r,\rho)$ is constructed out of polynomials with non-negative coefficients, whereby $\tfrac{\partial}{ \partial r_i} \tfrac{\partial }{\partial r_j} P_{r_k}(\epsilon_0,r,\rho_0) >0$ for all $ i,j,k \in \{ \alpha, \omega,c\}$. 
%%%\marginpar{I believe this is true, right? -JJ}


\begin{proposition}
		\label{prop:TightEstimate}
	Fix $ \epsilon_0 = 0.10$ and 
%\change[J]{$ (\rr_\alpha , \rr_ \omega , \rr_c) = (  0.1149, \, 0.0470 , \, 0.4711 ) $}
$ (\rr_\alpha , \rr_ \omega , \rr_c) = (  0.0594, \, 0.0260 , \, 0.4929 ) $ 
and 
%\change[J]{$\rho = 0.0279$}
$\rho = 0.3191$. 
	For all $0< \epsilon \leq \epsilon_0$ there exists a unique point $\hat{x}_\epsilon = (\hat{\alpha}_\epsilon,\hat{\omega}_\epsilon,\hat{c}_\epsilon)$ 
	satisfying $F(\hat{x}_\epsilon) = 0$ and 
	\begin{align}
	\label{eq:TightBound}
 | \hat{\alpha}_\epsilon - \balpha_\epsilon| <& \rr_\alpha \epsilon^2 , 
 %
 &|\hat{\omega}_\epsilon - \bomega_\epsilon| <&  \rr_\omega \epsilon^2 ,
 %
 &
 \| \hat{c}_\epsilon - \bc_\epsilon\| <& \rr_c  \epsilon^2   ,
  %
  &
  \| K^{-1} \hat{c}_\epsilon \| <& \rho  .
	\end{align}
Furthermore, $\hat{\alpha}_\epsilon > \pp$ for $ 0 < \epsilon < \epsilon_0$.
\end{proposition}

\begin{proof}
	In the Mathematica file~\cite{mathematicafile}  we check, using interval arithmetic, that  $\rho \geq C(\epsilon_0, \epsilon_0^2 \rr)$ and  the radii polynomials $P(\epsilon_0,\epsilon_0^2 \rr,\rho)$ are negative.  
	%I DO NOT UNDERSTAND THE NEXT SENTENCE
 The inequalities in Equation~\eqref{eq:TightBound} follow from Corollary~\ref{cor:RPUniformEpsilon}. 
 Since $\hat{\alpha}_\epsilon \geq \balpha_\epsilon - \rr_\alpha \epsilon^2
 = \pp +\frac{1}{5}(\frac{3\pi}{2}-1)\epsilon^2 - \rr_\alpha \epsilon^2$ and $ \rr_\alpha < \tfrac{1}{5} ( \tfrac{3 \pi}{2} -1) $, it follows that $ \hat{\alpha}_\epsilon > \pp $ for all $ 0 < \epsilon \leq \epsilon_0$. 
%%
%%STILL NEEDS AN EXPLANATION
%%\marginpar{Jonathan: I am not sure what the argument is \dots}
%% WHY IT IS UNIQUE IN THE BALL GIVEN BY~\eqref{eq:TightBound}. CLEARLY IT IS UNIQUE IN $B_\epsilon(r,\rho)$
%%WITH $\rho= C( \epsilon_0,\epsilon_0^2 r)$. WHY CAN THERE BE NO SOLUTIONS WITH
%%$\| K^{-1} c \| > \rho$ SATISFYING~\eqref{eq:TightBound} ? 
\end{proof}

\begin{remark}\label{r:nested}
% The pivotal result in Proposition~\ref{prop:TightEstimate} is that $\hat{\alpha}_\epsilon > \pp$, which implies that the bifurcation is subcritical.
Since $\epsilon_0^2\rr < r$ for the choices (a) and (b) in Proposition~\ref{prop:bigboxes},
and the choices of $\rho$ and $\epsilon_0$ are compatible as well, the solutions found in Proposition~\ref{prop:bigboxes} are the same as those described by Proposition~\ref{prop:TightEstimate}. While the former proposition provides large isolation/uniqueness neighborhoods for the solutions,
the latter provides tight bounds and confirms the  supercriticality of the bifurcation suggested in Definition \ref{def:xepsilon}.

% The bifurcation is supercritical (see eg.  \cite{faria2006normal} p 252
	
	
	
%		We note that for each (appropriate) $\epsilon$, the ball 
%	$ B_{\epsilon}(r,\rho)$ from Proposition \ref{prop:TightEstimate} is contained within the balls   
%	$ B_{\epsilon}(r_a,\rho_a)$  and 
%	$ B_{\epsilon}(r_b,\rho_b)$ from Proposition \ref{prop:bigboxes}. 
%	This means that the fixed points $  \hat{x}_{\epsilon} \in B_{\epsilon}(r,\rho)$ is the same fixed point $\hat{x}_{\epsilon} \in B_{\epsilon}(r_a,\rho_a)$ .

\end{remark}


%
%The method of radii polynomials is versatile. 
%With the goal of later proving Corollary \ref{prop:UniqunessNbd}, we added additional constraints to \emph{Mathematica}'s function \emph{NMaximize} to find the parameters for Proposition \ref{prop:WideEstimate}.
%
%When searching for the largest viable radius we add an additional constraint. 
%In Proposition \ref{prop:Cone}, we showed that for a given selection of $ \epsilon$, $r_\alpha$ and $ r_\omega$, then the unscaled variable $ \|c\|$ is either $\cO(1)$ or $\cO(\epsilon^2)$. 
%When we scale $c \to \epsilon c$, then we are only able to prove uniqueness of our solution in an $ \epsilon-$cone about the approximate solution. 
%We use this Proposition to select $r_c = ????$ in terms of $\epsilon$, $r_\alpha$ and $r_\omega$ so that any unscaled  solution $c$ is either $\cO(1)$ or $ c \in B_{\epsilon}(r,\rho)$.
%
%Even still, the larger we choose $ \epsilon$, the smaller we will need to take $ r_\omega$ in order to have a proof. 
%For the following theorem, we fixed $ \epsilon_0 =0.085$ and used \emph{NMaximize} to find a choice of variables $(\epsilon,r_\alpha,r_\omega,r_c)$  which maximized the objective function $r_w$ and for which all the radii polynomials were negative. 
%By slightly shrinking the estimate for the optimal radii, we obtain the following theorem.



\makeatletter
\newcommand{\len}[1]{\text{len}(#1)}
\newcommand\ty\iota
\newcommand\chty\kappa
\newcommand\env\Gamma
\newcommand\predenv\Phi
\newcommand\chenv\Delta
\newcommand{\sch}[3][\reg]{\Chty{#1}{#2}{#3}}
\newcommand\schan{\sch{\reg}{\seq{\ty}}{\seq{\chty}}}
\newcommand\srchan\schan
\newcommand\pred\phi
\newcommand\predvar{P}
\newcommand{\rch}[4]{\textbf{ch}_{#1}(#2; #3; #4)}
\newcommand{\ioch}[6]{\textbf{ch}_{#1}(#2; #3; #4; #5; #6)}
\newcommand{\fdef}[3]{#1(#2) = #3}
\newcommand{\ndint}{\star}
\newcommand{\ndletb@se}[3]{\textbf{let }#1 = #2 \textbf{ in } #3}
\newcommand{\ndletst@r}[2]{\ndletb@se{#1}{\ndint}{#2}}
\newcommand{\ndletnost@r}[2]{\ndletb@se{\seq{#1}}{\seq{\ndint}}{#2}}
\newcommand{\ndlet}{\@ifstar{\ndletst@r}{\ndletnost@r}}
\newcommand{\ifexp}[3]{\textbf{if}\ #1 \allowbreak \ \textbf{then}\ #2 \allowbreak \ \textbf{else}\ #3}
\newcommand{\op}{\mathit{op}}
\newcommand{\cname}[1]{\mathit{#1}}
\newcommand{\fname}[1]{\mathit{#1}}
\newcommand{\sequiv}{\equiv}
\newcommand\piequiv{\sequiv_{\pi}}
\newcommand{\expequiv}{\sequiv_{\text{E}}}
\newcommand\seqequiv{\sequiv_{\text{S}}}
\newcommand{\fdefequiv}{\sequiv_{D}}
\newcommand\seqto{\rightsquigarrow}
\newcommand{\subdef}{\trianglelefteq}
\newif\if@draft
\@draftfalse
\newcommand{\sk}[1]
{
\if@draft%
{\small\textcolor{blue}{[#1 -sk]}}%
\else\ignorespaces%
\fi
}
\newcommand{\skchanged}[1]
{\if@draft\textcolor{blue}{#1}\else#1\fi}
\newcommand{\changed}[1]
{\if@draft\textcolor{red}{#1}\else#1\fi}
\newcommand{\nk}[1]
{
\if@draft%
{\small\textcolor{red}{[#1 -nk]}}%
\else\ignorespaces%
\fi
}
\newcommand{\sh}[1]
{
\if@draft%
{\small\textcolor{magenta}{[#1 -sh]}}%
\else\ignorespaces%
\fi
}
\newcommand{\shchanged}[1]
{\if@draft\textcolor{magenta}{#1}\else#1\fi}
\newcommand{\ry}[1]
{
\if@draft%
{\small\textcolor{green}{[#1 -ry]}}%
\else\ignorespaces%
\fi
}

\newif\if@aplas
\@aplasfalse
\newcommand{\ifaplas}[2]
{\if@aplas#1\else#2\fi}
\makeatother
\makeatletter
\RequirePackage[bookmarks,unicode,colorlinks=true]{hyperref}%
   \def\@citecolor{blue}%
   \def\@urlcolor{blue}%
   \def\@linkcolor{blue}%
\def\UrlFont{\rmfamily}
\def\orcidID#1{\smash{\href{http://orcid.org/#1}{\protect\raisebox{-1.25pt}{\protect\includegraphics{ORCID_Color.eps}}}}}
\makeatother

\begin{document}
\setboolean{printBibInSubfiles}{false}
\title{Termination Analysis for the $\pi$-Calculus by Reduction to Sequential Program Termination}
\titlerunning{Termination Analysis for the \texorpdfstring{$\pi$}{Pi}-Calculus}
\author{
  Tsubasa Shoshi\inst{1}\orcidID{0000-0002-8164-0995} \and 
  Takuma Ishikawa\inst{1} \and
  Naoki Kobayashi\inst{1}\orcidID{0000-0002-0537-0604} \and
  Ken Sakayori\inst{1}\orcidID{0000-0003-3238-9279} \and
  Ryosuke Sato\inst{1}\orcidID{0000-0001-8679-2747} \and
  Takeshi Tsukada\inst{2}\orcidID{0000-0002-2824-8708}
}
\authorrunning{T. Shoshi et al.}
\institute{
  The University of Tokyo, Japan
  \and
  Chiba University, Japan
}
\maketitle              %
  In this paper, we explore the connection between secret key agreement and secure omniscience within the setting of the multiterminal source model with a wiretapper who has side information. While the secret key agreement problem considers the generation of a maximum-rate secret key through public discussion, the secure omniscience problem is concerned with communication protocols for omniscience that minimize the rate of information leakage to the wiretapper. The starting point of our work is a lower bound on the minimum leakage rate for omniscience, $\rl$, in terms of the wiretap secret key capacity, $\wskc$. Our interest is in identifying broad classes of sources for which this lower bound is met with equality, in which case we say that there is a duality between secure omniscience and secret key agreement. We show that this duality holds in the case of certain finite linear source (FLS) models, such as two-terminal FLS models and pairwise independent network models on trees with a linear wiretapper. Duality also holds for any FLS model in which $\wskc$ is achieved by a perfect linear secret key agreement scheme. We conjecture that the duality in fact holds unconditionally for any FLS model. On the negative side, we give an example of a (non-FLS) source model for which duality does not hold if we limit ourselves to communication-for-omniscience protocols with at most two (interactive) communications.  We also address the secure function computation problem and explore the connection between the minimum leakage rate for computing a function and the wiretap secret key capacity.
  
%   Finally, we demonstrate the usefulness of our lower bound on $\rl$ by using it to derive equivalent conditions for the positivity of $\wskc$ in the multiterminal model. This extends a recent result of Gohari, G\"{u}nl\"{u} and Kramer (2020) obtained for the two-user setting.
  
   
%   In this paper, we study the problem of secret key generation through an omniscience achieving communication that minimizes the 
%   leakage rate to a wiretapper who has side information in the setting of multiterminal source model.  We explore this problem by deriving a lower bound on the wiretap secret key capacity $\wskc$ in terms of the minimum leakage rate for omniscience, $\rl$. 
%   %The former quantity is defined to be the maximum secret key rate achievable, and the latter one is defined as the minimum possible leakage rate about the source through an omniscience scheme to a wiretapper. 
%   The main focus of our work is the characterization of the sources for which the lower bound holds with equality \textemdash it is referred to as a duality between secure omniscience and wiretap secret key agreement. For general source models, we show that duality need not hold if we limit to the communication protocols with at most two (interactive) communications. In the case when there is no restriction on the number of communications, whether the duality holds or not is still unknown. However, we resolve this question affirmatively for two-user finite linear sources (FLS) and pairwise independent networks (PIN) defined on trees, a subclass of FLS. Moreover, for these sources, we give a single-letter expression for $\wskc$. Furthermore, in the direction of proving the conjecture that duality holds for all FLS, we show that if $\wskc$ is achieved by a \emph{perfect} secret key agreement scheme for FLS then the duality must hold. All these results mount up the evidence in favor of the conjecture on FLS. Moreover, we demonstrate the usefulness of our lower bound on $\wskc$ in terms of $\rl$ by deriving some equivalent conditions on the positivity of secret key capacity for multiterminal source model. Our result indeed extends the work of Gohari, G\"{u}nl\"{u} and Kramer in two-user case.
% \leavevmode
% \\
% \\
% \\
% \\
% \\
\section{Introduction}
\label{introduction}

AutoML is the process by which machine learning models are built automatically for a new dataset. Given a dataset, AutoML systems perform a search over valid data transformations and learners, along with hyper-parameter optimization for each learner~\cite{VolcanoML}. Choosing the transformations and learners over which to search is our focus.
A significant number of systems mine from prior runs of pipelines over a set of datasets to choose transformers and learners that are effective with different types of datasets (e.g. \cite{NEURIPS2018_b59a51a3}, \cite{10.14778/3415478.3415542}, \cite{autosklearn}). Thus, they build a database by actually running different pipelines with a diverse set of datasets to estimate the accuracy of potential pipelines. Hence, they can be used to effectively reduce the search space. A new dataset, based on a set of features (meta-features) is then matched to this database to find the most plausible candidates for both learner selection and hyper-parameter tuning. This process of choosing starting points in the search space is called meta-learning for the cold start problem.  

Other meta-learning approaches include mining existing data science code and their associated datasets to learn from human expertise. The AL~\cite{al} system mined existing Kaggle notebooks using dynamic analysis, i.e., actually running the scripts, and showed that such a system has promise.  However, this meta-learning approach does not scale because it is onerous to execute a large number of pipeline scripts on datasets, preprocessing datasets is never trivial, and older scripts cease to run at all as software evolves. It is not surprising that AL therefore performed dynamic analysis on just nine datasets.

Our system, {\sysname}, provides a scalable meta-learning approach to leverage human expertise, using static analysis to mine pipelines from large repositories of scripts. Static analysis has the advantage of scaling to thousands or millions of scripts \cite{graph4code} easily, but lacks the performance data gathered by dynamic analysis. The {\sysname} meta-learning approach guides the learning process by a scalable dataset similarity search, based on dataset embeddings, to find the most similar datasets and the semantics of ML pipelines applied on them.  Many existing systems, such as Auto-Sklearn \cite{autosklearn} and AL \cite{al}, compute a set of meta-features for each dataset. We developed a deep neural network model to generate embeddings at the granularity of a dataset, e.g., a table or CSV file, to capture similarity at the level of an entire dataset rather than relying on a set of meta-features.
 
Because we use static analysis to capture the semantics of the meta-learning process, we have no mechanism to choose the \textbf{best} pipeline from many seen pipelines, unlike the dynamic execution case where one can rely on runtime to choose the best performing pipeline.  Observing that pipelines are basically workflow graphs, we use graph generator neural models to succinctly capture the statically-observed pipelines for a single dataset. In {\sysname}, we formulate learner selection as a graph generation problem to predict optimized pipelines based on pipelines seen in actual notebooks.

%. This formulation enables {\sysname} for effective pruning of the AutoML search space to predict optimized pipelines based on pipelines seen in actual notebooks.}
%We note that increasingly, state-of-the-art performance in AutoML systems is being generated by more complex pipelines such as Directed Acyclic Graphs (DAGs) \cite{piper} rather than the linear pipelines used in earlier systems.  
 
{\sysname} does learner and transformation selection, and hence is a component of an AutoML systems. To evaluate this component, we integrated it into two existing AutoML systems, FLAML \cite{flaml} and Auto-Sklearn \cite{autosklearn}.  
% We evaluate each system with and without {\sysname}.  
We chose FLAML because it does not yet have any meta-learning component for the cold start problem and instead allows user selection of learners and transformers. The authors of FLAML explicitly pointed to the fact that FLAML might benefit from a meta-learning component and pointed to it as a possibility for future work. For FLAML, if mining historical pipelines provides an advantage, we should improve its performance. We also picked Auto-Sklearn as it does have a learner selection component based on meta-features, as described earlier~\cite{autosklearn2}. For Auto-Sklearn, we should at least match performance if our static mining of pipelines can match their extensive database. For context, we also compared {\sysname} with the recent VolcanoML~\cite{VolcanoML}, which provides an efficient decomposition and execution strategy for the AutoML search space. In contrast, {\sysname} prunes the search space using our meta-learning model to perform hyperparameter optimization only for the most promising candidates. 

The contributions of this paper are the following:
\begin{itemize}
    \item Section ~\ref{sec:mining} defines a scalable meta-learning approach based on representation learning of mined ML pipeline semantics and datasets for over 100 datasets and ~11K Python scripts.  
    \newline
    \item Sections~\ref{sec:kgpipGen} formulates AutoML pipeline generation as a graph generation problem. {\sysname} predicts efficiently an optimized ML pipeline for an unseen dataset based on our meta-learning model.  To the best of our knowledge, {\sysname} is the first approach to formulate  AutoML pipeline generation in such a way.
    \newline
    \item Section~\ref{sec:eval} presents a comprehensive evaluation using a large collection of 121 datasets from major AutoML benchmarks and Kaggle. Our experimental results show that {\sysname} outperforms all existing AutoML systems and achieves state-of-the-art results on the majority of these datasets. {\sysname} significantly improves the performance of both FLAML and Auto-Sklearn in classification and regression tasks. We also outperformed AL in 75 out of 77 datasets and VolcanoML in 75  out of 121 datasets, including 44 datasets used only by VolcanoML~\cite{VolcanoML}.  On average, {\sysname} achieves scores that are statistically better than the means of all other systems. 
\end{itemize}


%This approach does not need to apply cleaning or transformation methods to handle different variances among datasets. Moreover, we do not need to deal with complex analysis, such as dynamic code analysis. Thus, our approach proved to be scalable, as discussed in Sections~\ref{sec:mining}.
\section{Source and Target Languages}  \label{sec:targetlanguage}

This section introduces the source and target languages for our reduction.
The source language is the
polyadic \(\pi\)-calculus~\cite{milner1993polyadic} extended with
integers and conditional expressions,
and the target language is a first-order functional language with non-determinism.


\subsection{$\pi$-Calculus}
\subsubsection{Syntax}

Below we assume
a countable set of variables ranged over by \(x, y, z, w,\!\ldots\)
and write \( \mathbb{Z} \) for the set of integers, ranged
over by \( i \).  We write $\tilde{\cdot}$ for (possibly empty) finite
sequences; for example, $\tilde{x}$ abbreviates a sequence
$x_1,\dots,x_n$.  We write \( \len{\tilde{x}} \) for the length of \(\seq{x}\) and
\(\epsilon\) for the empty sequence.

The sets of \emph{processes} and \emph{simple expressions},
ranged over by $P$ and $v$ respectively, are defined inductively
by: %
\begin{align*}
    &P \mbox{ (processes) }::= \ \zeroexp \mid \outexp{x}{\seq{v}}{\seq{w}}P \mid \inexp{x}{\seq{y}}{\seq{z}}P \mid \rinexp{x}{\seq{y}}{\seq{z}}P \mid (P_1 \PAR P_2) \mid \nuexp{x \COL \chty} P \\
    &\hphantom{P \mbox{ (processes) }::=}     \mid \ifexp{v}{P_1}{P_2} \mid \ndlet{x}{P} \\
    &v \mbox{ (simple expressions) }::= \ x \mid i \mid \op(\tilde{v})
\end{align*}
The syntax of processes on the first line is fairly standard, except that
the values sent along each channel consist of two parts: \(\seq{v}\) for integers,
and \(\seq{w}\) for channels; this is for the sake of technical convenience in
presenting the translation to sequential programs. The process \(\zeroexp\)
denotes an inaction, \(\outexp{x}{\tilde{v}}{\tilde{w}}P\) sends
a tuple \((\tilde{v},\tilde{w})\) along the channel \(x\) and behaves like \(P\),
and the process \(\inexp{x}{\tilde{y}}{\tilde{z}}P\) receives
a tuple \((\tilde{v},\tilde{w})\) along the channel \(x\), and behaves like
\([\seq{v}/\seq{y}, \seq{w}/\seq{z}]P\). We often just write \(\seq{v}\) for
\(\seq{v};\epsilon\) or \(\epsilon;\seq{v}\).
The process \(\rinexp{x}{\seq{y}}{\seq{z}}P\) represents infinitely many copies
of \(\inexp{x}{\seq{y}}{\seq{z}}P\) running in parallel.
The process \(P_1\PAR P_2\) runs \(P_1\) and \(P_2\) in parallel,
and \(\nuexp{x\COL\chty}{P}\) creates a fresh channel \shchanged{$x$} of type \(\chty\) (where types
will be introduced shortly) and behaves like \(P\).
The process \(\ifexp{v}{P_1}{P_2}\) executes \(P_1\) if the value of \(v \) is
non-zero, and \(P_2\) otherwise.
The process \(\ndlet{x}{P}\) instantiates the variables \(\seq{x}\) to
some integer values in a non-deterministic manner, and then behaves like \(P\).
The meta-variable \( \op \) ranges over integer operations such as \( + \) or \( \le \).


The free and bound variables are defined as usual.
The only binders are \(\nuexp{x\COL\chty}\)
(which binds \(x\)), \(\ndletsatom{x}\) (which binds \(\seq{x}\)),
\(\inexp{x}{\seq{y}}{\seq{z}}\)
and \(\rinexp{x}{\seq{y}}{\seq{z}}\) (which bind \(\seq{y}\) and \(\seq{z}\)).
Processes are identified up to renaming of bound variables,
and we implicitly apply \( \alpha \)-conversions as necessary.

We write \(P\red Q\) for the standard one-step reduction relation on processes.
The base cases of the communication are given by:
\begin{align*}
    \inexp{x}{\seq{y}}{\seq{z}}P_1 \PAR  \outexp{x}{\seq{v}}{\seq{w}}P_2 &\red [\seq{i}/ \seq{y}, \seq{w} / \seq{z} ]P_1 \PAR P_2 \\
    \rinexp{x}{\seq{y}}{\seq{z}}P_1 \PAR  \outexp{x}{\seq{v}}{\seq{w}}P_2 &\red \rinexp{x}{\seq{y}}{\seq{z}}P_1 \PAR [\seq{i}/ \seq{y}, \seq{w} / \seq{z}]P_1 \PAR P_2
\end{align*}
provided that \( \seq{v} \) evaluates to \( \seq{i} \).
\ifaplas{
  The full definition is given in
  the extended version~\cite{fullversion}}{The full definition is given in Appendix~\ref{sec:operational_semantics}}.
We say that a process \(P\) is \emph{terminating} if there is no infinite
reduction sequence \(P\red P_1\red P_2\red \cdots\).

In the rest of the paper, we consider only well-typed processes.
We write \(\ty\) for the type of integers.
The set of channel types, ranged over by \(\chty\), is given by:
\begin{align*}
  \chty ::= \Chty{\reg}{\tilde{\ty}}{\tilde{\chty}}
\end{align*}
The type $\Chty{\reg}{\tilde{\ty}}{\tilde{\chty}}$
describes channels used for transmitting a tuple \((\seq{v};\seq{w})\)
of integers \(\seq{v}\) and channels \(\seq{w}\) of types \(\seq{\chty}\).
Below we will just write \( \seq{\ty} \) for \( \seq{\ty}; \epsilon \) and \( \seq{\chty} \) for \( \epsilon ;\seq{\chty}\).
The subscript \(\reg\), called a \emph{region}, is a symbol that
abstracts channels; it is used in the translation to sequential programs.
For example, \(\Chty{\reg_1}{\ty}{\sChty{\reg_2}{\ty}}\) is the type of channels
that belong to the region \(\reg_1\) and are
used for transmitting a pair \((i,r)\) where
\(r\) is a channel of region \(\reg_2\) used for transmitting integers.
We use a meta-variable \(\mty\) for an integer or channel type.

Type judgments for processes and simple expressions are of the form \(\env;\chenv\p P\)
and \(\env;\chenv\p v:\mty\), where \(\env\) and \(\chenv\)
are sequences of bindings of the form
\(x\COL\ty\) and \(x\COL\chty\), respectively.
The typing rules are shown in Figure~\ref{fig:simple_type_system}.
Here \( \env; \chenv \vdash \seq{v} \COL \seq{\mty} \) means \( \env; \chenv \p v_i : \mty_i \) holds for each \( i \in \{ 1, \ldots, \len{\seq{v}} \} \).
We omit the explanation of the typing rules as they are standard.
\begin{figure}[tb]
    \centering
    \small
    \begin{minipage}{0.3\linewidth}
        \centering
        \begin{prooftree}
            \AxiomC{}
            \UnaryInfC{$\env; \chenv \vdash \zeroexp$}
        \end{prooftree}
    \end{minipage}
    \begin{minipage}{0.65\linewidth}
        \centering
        \begin{prooftree}
            \AxiomC{$\env; \chenv \vdash v \COL \ty$}
            \AxiomC{$\env; \chenv \vdash P_1$}
            \AxiomC{$\env; \chenv \vdash P_2$}
            \TrinaryInfC{$\env; \chenv \vdash \ifexp{v}{P_1}{P_2}$}
        \end{prooftree}
    \end{minipage}
    \\\vspace*{1ex}
    \begin{minipage}{0.38\linewidth}
        \centering
        \begin{prooftree}
            \AxiomC{$\env; \chenv \vdash P_1$}
            \AxiomC{$\env; \chenv \vdash P_2$}
            \BinaryInfC{$\env; \chenv \vdash P_1 \PAR P_2$}
        \end{prooftree}
    \end{minipage}
    \begin{minipage}{0.25\linewidth}
        \centering
        \begin{prooftree}
            \AxiomC{$\env; \chenv, x\COL\chty \vdash P$}
            \UnaryInfC{$\env; \chenv \vdash \nuexp{x \COL \chty}P$}
        \end{prooftree}
    \end{minipage}
    \begin{minipage}{.3\linewidth}
        \begin{prooftree}
            \AxiomC{$\env, \seq{x}\COL \seq{\ty}; \chenv \vdash P$}
            \UnaryInfC{$\env; \chenv \vdash \ndlet{x}{P}$}
        \end{prooftree}
    \end{minipage}
    \\\vspace*{1ex}
    \begin{minipage}{\linewidth}
        \centering
        \begin{prooftree}
            \AxiomC{$\env; \chenv \vdash x\COL \Chty{\reg}{\seq{\ty}}{\seq{\chty}}$}
            \AxiomC{$\env, \seq{y}\COL \seq{\ty}; \chenv, \tilde{z}\COL \tilde{\chty} \vdash P$}
            \BinaryInfC{$\env; \chenv \vdash \inexp{x}{\seq{y}}{\seq{z}}P $}
        \end{prooftree}
    \end{minipage}
    \\\vspace*{1ex}
    \begin{minipage}{\linewidth}
        \centering
        \begin{prooftree}
            \AxiomC{$\env; \chenv \vdash x\COL \Chty{\reg}{\seq{\ty}}{\seq{\chty}}$}
            \AxiomC{$\env; \chenv \vdash \seq{v}\COL \seq{\ty}$}
            \AxiomC{$\env; \chenv \vdash \seq{w}\COL \seq{\chty}$}
            \AxiomC{$\env; \chenv \vdash P$}
            \QuaternaryInfC{$\env; \chenv \vdash \outexp{x}{\seq{v}}{\seq{w}}P$}
        \end{prooftree}
    \end{minipage}
    \\\vspace*{1ex}
    \begin{minipage}{\linewidth}
        \begin{prooftree}
            \AxiomC{$\env; \chenv \vdash x\COL \Chty{\reg}{\seq{\ty}}{\seq{\chty}}$}
            \AxiomC{$\env, \seq{y}\COL \seq{\ty}; \chenv, \seq{z}\COL \seq{\chty} \vdash P$}
            \BinaryInfC{$\env; \chenv \vdash \rinexp{x}{\seq{y}}{\seq{z}}P$}
        \end{prooftree}
    \end{minipage}
    \\\vspace*{1ex}
    \begin{minipage}{0.2\linewidth}
        \centering
        \begin{prooftree}
            \AxiomC{$x\COL\ty \in \env$}
            \UnaryInfC{$\env; \chenv \vdash x\COL \ty$}
        \end{prooftree}
    \end{minipage}
    \begin{minipage}{0.2\linewidth}
        \centering
        \begin{prooftree}
            \AxiomC{$x\COL\chty \in \chenv$}
            \UnaryInfC{$\env; \chenv \vdash x\COL \chty$}
        \end{prooftree}
    \end{minipage}
    \begin{minipage}{0.18\linewidth}
        \centering
        \begin{prooftree}
            \AxiomC{}
            \UnaryInfC{$\env; \chenv \vdash i\COL \ty$}
        \end{prooftree}
    \end{minipage}
    \begin{minipage}{0.25\linewidth}
        \centering
        \begin{prooftree}
            \AxiomC{$\env; \chenv \vdash \tilde{v}\COL \seq{\ty}$}
            \UnaryInfC{$\env; \chenv \vdash \op(\seq{v})\COL \ty$}
        \end{prooftree}
    \end{minipage}
    \normalsize
    \caption{The typing rules of the simple type system for the $\pi$-calculus}
    \label{fig:simple_type_system}
\end{figure}





\subsection{Sequential Language}
We define the target language of our translation, which
is a first-order functional language with non-determinism.

A \emph{program} is a pair \((\Def, \Exp)\) consisting of (a set of)
\emph{function definitions} \(\Def\) and
an \emph{expression} \(\Exp\), defined by:
\begin{align*}
    \mathcal{D} \ \text{(function definitions)}        \, ::=& \ \{ \fdef{f_1}{\tilde{x}_1}{E_1}, \ldots, \fdef{f_n}{\tilde{x}_n}{E_n} \} \\
    E           \ \text{(expression)}                  \, ::=& \ \skipexp
    \mid \ndlet{x}{E} \mid f(\tilde{v}) \mid E_1 \nondet E_2 \\
                                                             & \ \mid \textbf{if } v \textbf{ then } E_1 \textbf{ else } E_2 \mid \textbf{Assume}(v); E \\
    v           \ \text{(simple expressions)}                       \, ::=& \ x \mid i \mid \op(\tilde{v})
\end{align*}
In a function definition
\( \fdef{f_i}{x_1, \ldots , x_{k_i}}{\Exp_i} \),
the variables \( x_1, \ldots, x_{k_i} \)
are bound in \( \Exp_i \); we identify function definitions up to renaming of bound
variables, and implicitly apply \(\alpha\)-conversions.
The function names \( f_1, \ldots, f_n \) need not be distinct from each other.
  If there are more than one definition for \( f \), then one of the definitions
  will be non-deterministically used when \( f \) is called.
We explain the informal meanings of the nonstandard expressions.
The expression \( \ndlet{x}{E} \)
instantiates \(\seq{x}\) to some integers in a non-deterministic manner.
The expression \(E_1\nondet E_2\) non-deterministically evaluates to
\(E_1\) or \(E_2\).
The expression
\(\textbf{Assume}(v); E \) evaluates to \(E\) if \(v\) is non-zero;
otherwise the whole program is aborted.
The other expressions are standard and their meanings should be clear.

We write \((\Def,\Exp)\sred (\Def,\Exp')\)
for the one-step reduction relation, whose definition is given
in \ifaplas{the extended version~\cite{fullversion}}{Appendix~\ref{sec:operational_semantics}}.
We say that a program is terminating if there is no infinite
reduction sequence.

\section{Basic Transformation}  \label{sec:approach}

This section presents our transformation from
a  \(\pi\)-calculus process to a sequential program,
so that
if the transformed program is terminating then the original process is terminating.

As explained in Section~\ref{sec:introduction}, the idea is to transform
an infinite chain of message passing on replicated input channels to
an infinite chain of recursive function calls.
Table~\ref{tab:trans} summarizes the correspondence between
processes and  sequential programs.
As shown in the table, a replicated input process is transformed to
a function definition, whereas a non-replicated input process is
just ignored, and integer bound variables are non-deterministically instantiated.
Note that channel arguments \(\seq{z}\) are ignored in both cases.
Instead, we prepare a global function name \(\Fname{\reg}\) for each
region \(\reg\); \(\regof{x}\) in the table indicates the region assigned to
the channel type of \(x\).\footnote{Thus, 
  the simple type system with
  ``regions'' introduced in the previous section is used here as a simple
  may-alias analysis.
  If \(x\) and \(y\) may be bound to the same channel during reductions,
  the type system assigns the same region to \(x\) and \(y\),
  hence \(x\) and \(y\) are
  mapped to the same function name \(\Fname{\regof{x}}\) by our transformation.}
  

\begin{table}[tbp]
  \caption{Correspondence between processes and sequential programs}
  \label{tab:trans}
\begin{tabular}{|l|l|}
  \hline
  processes & sequential programs \\
  \hline
  \hline
  replicated input (\(\rinexp{x}{\seq{y}}{\seq{z}}\cdots\)) & function definition  \(\Fname{\regof{x}}(\seq{y})=\cdots\)\\
  \hline
  non-replicated input (\(\inexp{x}{\seq{y}}{\seq{z}}\cdots\)) &
  non-deterministic instantiation (\(\ndlet{y}\cdots\))\\
  \hline
  output (\(\outexp{x}{\seq{v}}{\seq{w}}\cdots\)) &
    function call (\(\Fname{\regof{x}}(\seq{v})\nondet \cdots\))\\
    \hline
    parallel composition (\(\cdots\PAR\cdots\)) &
    non-deterministic choice (\(\cdots\nondet \cdots\))\\
    \hline
\end{tabular}
\end{table}

We define the transformation relation
$\env; \chenv \vdash P \Rightarrow \prog{\Def}{\Exp}$,
which means that the \(\pi\)-calculus process \(P\) well-typed
under \(\env;\chenv\) is transformed to the sequential program \((\Def,\Exp)\).
The relation is defined by the rules in Figure~\ref{fig:program_transformation}.
\begin{figure}[tb]
  \typicallabel{SX-RIn}
  \infrule[SX-Nil]{}
          {\env; \chenv \vdash \textbf{0} \Rightarrow
            \prog{\set{ \fdef{\Fname{\reg}}{\seq{y}}{\skipexp}
              \mid x\COL\Chty{\reg}{\seq{\ty}}{\seq{\chty}} \in \chenv, 
                   \len{\seq{y}} = \len{\seq{\ty}} }}{\skipexp}}
          \vspace*{1ex}
          \infrule[SX-In]
            {\env; \chenv \vdash x : \Chty{\reg}{\seq{\ty}}{\seq{\chty}}\andalso
            \env, \seq{y} : \seq{\ty}; \chenv, \seq{z} : \seq{\chty} \vdash P \Rightarrow \prog{\Def}{\Exp}}
            {\env; \chenv \vdash \inexp{x}{\seq{y}}{\seq{z}}P \Rightarrow
              \prog{\ndlet{y}{\Def}}{\ndlet{y}{\Exp}}}
          \vspace*{1ex}
\infrule[SX-RIn]
{\env; \chenv \vdash x : \Chty{\reg}{\seq{\ty}}{\seq{\chty}}\andalso
  \env, \seq{y} : \seq{\ty}; \chenv, \seq{z} : \seq{\chty} \vdash P \Rightarrow
  \prog{\Def}{\Exp}}
{\env; \chenv \vdash \rinexp{x}{\seq{y}}{\seq{z}}P \Rightarrow
  \prog{\set{ \fdef{\Fname{\reg}}{\seq{y}}{\Exp} } \mrg (\ndlet{y}{\Def})} {\skipexp}}
\vspace*{1ex}
\infrule[SX-Out]
{\env; \chenv \vdash x : \Chty{\reg}{\seq{\ty}}{ \seq{\chty}}\andalso
\env; \chenv \vdash \seq{v} : \seq{\ty}\andalso
\env; \chenv \vdash \seq{w} : \seq{\chty}\andalso
\env; \chenv \vdash P \Rightarrow \prog{\Def}{\Exp}}
{\env; \chenv \vdash \outexp{x}{\seq{v}}{\seq{w}}P \Rightarrow
  \prog{\mathcal{D}}{\Fname{\reg}(\seq{v}) \nondet \Exp}}

\infrule[SX-Par]
{\env; \chenv \vdash P_1 \Rightarrow \prog{\Def_1}{\Exp_1}\andalso
\env; \chenv \vdash P_2 \Rightarrow \prog{\Def_2}{\Exp_2}}
{\env; \chenv \vdash P_1 \mid P_2 \Rightarrow \prog{\Def_1 \mrg \Def_2}{\Exp_1 \nondet \Exp_2}}

\vspace*{1ex}
\infrule[SX-Nu]
        {\env; \chenv, x : \chty \vdash P \Rightarrow \prog{\Def}{\Exp}}
        {\env; \chenv \vdash \nuexp{x \COL \chty}P \Rightarrow \prog{\Def}{\Exp}}
          \vspace*{1ex}

\infrule[SX-If]
{\env; \chenv \vdash v : \ty\andalso
 \env; \chenv \vdash P_1 \Rightarrow \prog{\Def_1}{\Exp_1}\andalso
 \env; \chenv \vdash P_2 \Rightarrow \prog{\Def_2}{\Exp_2}}
{\env; \chenv \vdash \ifexp{v}{P_1}{P_2} \Rightarrow
  \prog{\Def_1 \mrg \Def_2}{\ifexp{v}{\Exp_1}{\Exp_2}}}

          \vspace*{1ex}

\infrule[SX-LetND]
{\env, \seq{x}: \seq{\ty}; \chenv \vdash P \Rightarrow \prog{\Def}{\Exp}}
{\env; \chenv \vdash \ndlet{x}{P} \Rightarrow
  \prog{\ndlet{x}{\Def}}{\ndlet{x}{\Exp}}}

    \begin{align*}
        \ndlet{x}{\Def} \coloneqq
& \{ \fdef{f}{\tilde{y}}{(\ndlet{x}{\Exp})} \mid \fdef{f}{\tilde{y}}{E} \in \Def \} 
    \end{align*}
    \normalsize
    \caption{The rules of simple type-based program transformation}
    \label{fig:program_transformation}
\end{figure}

We explain some key rules.
In \rn{SX-Nil}, \(\zeroexp\) is translated to \((\Def,\skipexp)\),
where \(\Def\) is the set of trivial function definitions.
In \rn{SX-In}, a (non-replicated) input is just removed, 
and the bound variables are instantiated to non-deterministic integers;
this is because we have no information about \(\seq{y}\); this will be refined
in Section~\ref{sec:refinement}. In contrast,
in \rn{SX-RIn}, a replicated input is converted to a function definition.
Since \(\Def\) generated from \(P\) may contain \(\seq{y}\), they are
 bound to non-deterministic integers and merged with the new definition for \(\Fname{\reg}\).
In \rn{SX-Out}, an output is replaced by a function call.
In \rn{SX-Par}, parallel composition is replaced by non-deterministic choice.



\begin{example}
\label{ex:fib}
\newcommand{\FIB}{P_\text{fib}}
    Let us revisit the Fibonacci example used in the introduction to explain the actual translation.
    Using the syntax we introduced, the Fibonacci process \( \FIB \) can now be defined as:
    \begin{align*}
        & \nuexp{\fib\COL \Chty{\reg_1}{\ty}{\sChty{\reg_2}{\ty}}}\rinexp{\fib}{n}{r} \\
        & \quad \ifexp{n<2}{\soutatom{r}{1}}{(\nu r_1\COL \sChty{\reg_2}{\ty})(\nu r_2\COL \sChty{\reg_2}{\ty})\\
        & \quad (\outatom{\fib}{n-1}{r_1} \PAR \outatom{\fib}{n-2}{r_2} \PAR \sinexp{r_1}{x} \sinexp{r_2}{y} \soutatom{r}{x+y})} \\
        & \PAR \letexp{m}{\ndint}{(\nu r\COL \sChty{\reg_2}{\ty}) \outatom{\fib}{m}{r}}
    \end{align*}
    Note that \( \nuexp{\fib} \) and \( \ndletatom{m} \) have been added to close the process.
    %
    We can derive $\emptyset; \emptyset \vdash \FIB \Rightarrow \prog{\Def}{\Exp}$,
    where $\Def$ and $\Exp$ are given as follows:\footnotemark
    \footnotetext{The program written here has been simplified for the sake of readability.
    For instance, we removed some redundant \( \skipexp \), trivial function definitions, and unused non-deterministic integers.
    The other examples that will appear in this paper are also simplified in the same way.
    }
\begin{align*}
        \Def &=  \{ \fdef{\Fname{\reg_1}}{z}{\ifexp{z<2}{\Fname{\reg_2}(1)}{( \Fname{\reg_1}(z-1) \nondet \Fname{\reg_1}(z-2) \\ 
               &\qquad\qquad\qquad \nondet \letexp{x}{\ndint}{\letexp{y}{\ndint}{\Fname{\reg_2}(x+y)}} )}}, \\
               &\ \quad  \fdef{\Fname{\reg_2}}{z}{\skipexp} \} \\ 
        \Exp &= \letexp{m}{\ndint}{\Fname{\reg_1}(m)}
    \end{align*}
    Here \( \Fname{\reg_1} \) is the ``Fibonacci function'' because \( \reg_1 \) is the region assigned to the channel \( \cname{fib}\) in \( \FIB \).
    The function call \( \Fname{\reg_2} (x + y)\) corresponds to the output \( \soutatom{r}{x + y} \); the argument of the function call is actually a nondeterministic integer because \( \sinatom{r}{x} \) and \( \sinatom{r}{y} \) are translated to non-deterministic instantiations.
    Since the program $(\Def, \Exp)$ is terminating,
    we can verify that $\FIB$ is also terminating.
    \qed
\end{example}


\begin{example}
\label{ex:nested_rep}
To help readers understand the rule \rn{SX-RIn}, we consider the following process, which contains a nested input:
\begin{align*}
 \rinexp{f}{x}{r}\srinexp{g}{y, z}(\ifexp{y \leq 0}{\soutatom{r}{z}}{\soutatom{g}{y - 1, x + z}} ) \PAR \outexp{f}{2}{r}\soutatom{g}{3,0}
\end{align*}
%
where \( f \COL \Chty{\reg_1}{\ty}{\sChty{\reg_2}{\ty}} \) and \( g \COL \sChty{\reg_3}{\ty, \ty} \).
This process computes \( x * y + z \) (which is \( 6 \) in this case) and returns that value using \( r \).
This program is translated to:
\begin{align*}
  &\fdef{\Fname{\reg_1}}{x}{\skipexp}\qquad \fdef{\Fname{\reg_2}}{z}{\skipexp} \\
  &\fdef{\Fname{\reg_3}}{y, z}{\ndlet*{x}{\ifexp{y \leq 0}{\Fname{\reg_2}(z)}{} } \Fname{\reg_3}(y - 1, x + z)}
\end{align*}
with the main expression  \( \Fname{\reg_1}(2) \nondet \Fname{\reg_3}(3, 0) \).
Note that the body of \( \Fname{\reg_1}\), which is the function corresponding to \( f \), is \( \skipexp \).
This is because when the rule \rn{SX-RIN} is applied to 
\( \srinatom{g}{y, z}\ldots\),
the main expression of the translated program becomes \( \skipexp \).
Observe that the function definition for \( \Fname{\reg_3} \) still contains a free variable \( x \) at this moment.
Then \( \Fname{\reg_3} \) is closed by \( \ndletatom{x} \) when we apply the rule \rn{SX-RIn} to \( \srinatom{f}{x; r}\ldots \).
%
We can check that the above program is terminating, and thus we can verify that the original process is terminating.
%
Note that some precision is lost in the application of \rn{SX-RIn} above
since we cannot track the relation between the argument of \( \Fname{\reg_1} \)
and the value of \( x \) used inside \( \Fname{\reg_3}\). This loss causes a problem if, for example, the condition \(y\le 0\) in
  the process above is replaced with \(y\le x\). The body of
  \(\Fname{\reg_3}\) would then become
  \(\ndlet*{x}{\textbf{if } y\le x\ \cdots}\), hence the sequential program would be
    non-terminating.
 \qed
\end{example}

\begin{remark}
A reader may wonder why
a non-replicated input is removed in \rn{SX-In},
rather than translated to a function definition as done for a replicated input.
It is actually possible to obtain a sound transformation even if we treat
non-replicated inputs in the same manner as replicated inputs,
but we expect that our approach of removing non-replicated inputs often works
better.
For example,
  consider $\sinexp{x}{y}\soutatom{x}{y} \PAR \soutatom{x}{0}$.
  Our translation generates
  $\prog{\{\fdef{\Fname{\reg_x}}{z}{\skipexp}\}}{(\ndlet*{y}{\Fname{\reg_x}(y)}) \nondet \Fname{\reg_x}(0)}$
  which is terminating,
  whereas if we treat the input in the same way as a replicated input,
  we would obtain
  $\prog{\{\fdef{\Fname{\reg_x}}{z}{\Fname{\reg_x}(z)}\}}{\Fname{\reg_x}(0)}$
  which is not terminating.
 Our approach also has some defect.
  For example, consider
  $\soutatom{x}{0} \PAR \sinexp{x}{y} \ifexp{y=0}{\zeroexp}{\Omega}$
  where $\Omega$ is a diverging process. 
  Our translation yields
  $\prog{\{\fdef{\Fname{\reg_x}}{z}{\skipexp}\}}{\Fname{\reg_x}(0) \nondet \ndlet*{y}{\ifexp{y=0}{\skipexp}{\Omega'}}}$
  which is non-terminating.
  On the other hand, if we treat the input like a replicated input,
  we would obtain
  $\prog{\{\fdef{\Fname{\reg_x}}{z}{\ifexp{z=0}{\skipexp}{\Omega'}}\}}{\Fname{\reg_x}(0)}$
  which is terminating.
  This issue can, however,
  be mitigated by the extension with refinement types in Section~\ref{sec:refinement}.
Our choice of removing non-replicated inputs is also 
consistent with Deng and Sangiorgi's type system~\cite{Deng06IC}, which
prevents an infinite chain of communications on replicated input channels by using types
and ignores non-replicated inputs.
\qed
\end{remark}


The following theorem states the soundness of our transformation.
\begin{theorem}[soundness]  \label{thm:soundness}
  Suppose  $\emptyset; \emptyset \vdash P \Rightarrow \prog{\Def}{\Exp}$.
    If $(\Def, \Exp)$ is terminating, then so is $P$.
\end{theorem}
We briefly explain the proof strategy; see \ifaplas{the extended version~\cite{fullversion}}{Appendix~\ref{sec:soundness}}
for the actual proof.
Basically, our idea is to show that the translated program simulates the original process.
Then we can conclude that if the original process is non-terminating then so is the sequential program.
However, there is a slight mismatch between the reduction of a process and that of the sequential program that we need to overcome.
Recall that \( \srinexp{f}{x}P \PAR \soutatom{f}{1} \PAR \soutatom{f}{2} \) is translated to \( \Fname{\reg_f}(1) \nondet \Fname{\reg_f}(2) \) with a function definition for \( \Fname{\reg_f} \).
In the sequential program, we need to make a ``choice'', e.g.~if \( \Fname{\reg_f}(1) \) is called, we cannot call \( \Fname{\reg_f}(2) \) anymore.
On the other hand, the output \( \soutatom{f}{2} \) can be used even if \( \soutatom{f}{1} \) has been used before.
To fill this gap, we introduce a non-standard reduction relation, which does not discard branches of non-deterministic choices and show the simulation relation using that non-standard semantics.
Then we show that if there is an infinite non-standard reduction sequence, then there is an infinite subsequence that corresponds to a reduction along a certain choice of non-deterministic branches.
This step is essentially a corollary of the K\"onig's Lemma.
This is because the infinite non-standard reduction sequence can be reformulated as an infinite tree in which branches correspond to non-deterministic choices \( \oplus \) (thus the tree is finitely branching) and paths correspond to reduction sequences.


The following example indicates that the basic transformation is
sometimes too conservative.
\begin{example}
  \label{ex:weakeness-of-basic-transformation}
Let us consider the following process \( \DEC \):
\begin{align*}
    &\rinexp{\pre}{n}{r} \soutatom{r}{n-1} \\
    & \PAR \rinexp{f}{n}{r} \ifexp{n<0}{ \soutatom{r}{1} }{ \nuexp{s \COL \sChty{\reg_2}{\ty} } (\outatom{\pre}{n}{s} \PAR \sinexp{s}{x}\outatom{f}{x}{r}) } \\
    &\PAR \outatom{f}{m}{r}
\end{align*}
where
\( \pre\COL \Chty{\reg_1}{\ty}{\sChty{\reg_2}{\ty}} \), \(\cname{f} \COL \Chty{\reg_3}{\ty}{\sChty{\reg_4}{\ty}} \) and \( r \COL \sChty{\reg_4}{\ty}\).
This process, which also appeared in the introduction, keeps on decrementing the integer \( m \) until it gets negative and then returns \( 1 \) via \( r \).
We can turn this process into a closed process \( \DECclosed \) by restricting the names \( \pre \), \( \cname{f} \), \( \cname{r} \) and adding \( \ndletatom{m} \) in front of the process.
Note that \( \DECclosed \) is terminating.

The process \( \DECclosed \) is translated to:
\begin{align*}
  &\fdef{\Fname{\reg_1}}{n}{\Fname{\reg_2}(n-1)}, 
  \qquad \fdef{\Fname{\reg_2}}{x}{\skipexp},
  \qquad \fdef{\Fname{\reg_4}}{x}{\skipexp}, \\
  & \fdef{\Fname{\reg_3}}{n}{\ifexp{n<0}{\Fname{\reg_4}(1) \\ &\qquad\qquad }
            {(\Fname{\reg_1}(n) \nondet \letexp{x}{\ndint}{\Fname{\reg_3}(x)}})}
\end{align*}
with the main expression \( \ndlet*{m}{\Fname{\reg_3}(m)} \).
Observe that the function \( f_{\reg_3}\) is applied to a non-deterministic integer, not \( n - 1\).
Thus, this program is not terminating, meaning that we fail to verify that the original process is terminating.
This is due to the shortcoming of our transformation that all the integer   values received by non-replicated inputs are replaced by non-deterministic integers.
This problem is addressed in the next section.
\qed
\end{example}


\section{Improving Transformation Using Refinement Types}  \label{sec:refinement}

In this section, we refine the basic transformation in the previous section
by using a refinement type system.

Recall that in Example~\ref{ex:weakeness-of-basic-transformation},
the problem was that information about values received by non-replicated
inputs was completely lost.
By using a refinement type system for the \(\pi\)-calculus,
we can statically infer that \(x<n\) holds between \(x\) and \(n\) in
the process in Example~\ref{ex:weakeness-of-basic-transformation}.
Using that information, we can transform the process in Example~\ref{ex:weakeness-of-basic-transformation} and obtain
\[
    \fdef{\Fname{\reg_3}}{n}{\ifexp{n<0}{\cdots}
            {(\Fname{\reg_1}(n) \nondet \letexp{x}{\ndint}{\assexp{x < n}\Fname{\reg_3}(x)}})}
\]
for the definition of \( \Fname{\reg_3}\).
This is sufficient to conclude that the resulting program is terminating.

In the rest of this section, we first introduce a refinement type system
in Section~\ref{sec:rtype} and explain the refined transformation in Section~\ref{sec:rx}.
We then discuss how to automatically infer refinement types and
achieve the refined transformation in Section~\ref{sec:inference}.



\subsection{Refinement Type System}
\label{sec:rtype}


The set of \emph{refinement channel types}, ranged over by $\chty$, is given by:
\begin{align*}
    \chty ::= \rch{\reg}{\seq{x}}{\pred}{\seq{\chty}}
\end{align*}
Here, \(\pred\) is a formula of integer arithmetic.
We sometimes write just
\(\rchepsilon{\reg}{\seq{x}}{\pred}\) for
\(\rch{\reg}{\seq{x}}{\pred}{\epsilon}\).
Intuitively, 
$\rch{\reg}{\tilde{x}}{\pred}{\tilde{\chty}}$
describes channels that are used for transmitting
a tuple \((\seq{x};\seq{y})\) such that (i) \(\seq{x}\) are integers
that satisfy \(\pred\), and (ii) \(\seq{y}\) are channels of types \(\seq{\chty}\).
For example, the type
\(\rch{\reg_1}{x}{\TRUE}{\rchepsilon{\reg_2}{z}{z<x}}\)
describes channels used for transmitting a pair
\((x, y)\), where \(x\) may be any integer, and \(y\) must be a channel of
type \(\rchepsilon{\reg_2}{z}{z<x}\), i.e.,
a channel used for passing an integer \(z\) smaller than \(x\).%
Thus, if \(u\) has type \(\rch{\reg_1}{x}{\TRUE}{\rchepsilon{\reg_2}{z}{z<x}}\),
then the process
\(\inexp{u}{n}{r}{\soutatom{r}{n-1}}\) is allowed
but \(\inexp{u}{n}{r}{\soutatom{r}{n}}\) is not.


Type judgments for processes and expressions are
now of the form \( \env;\predenv;\chenv \p P \) and \( \env;\predenv;\chenv \p v\COL\mty \), where \( \predenv \) is a sequence of formulas.
Intuitively, \( \env;\predenv;\chenv \p P \) means that \( P \) is well-typed under the environments \( \env \) and \( \chenv \) assuming that all the formulas in \( \predenv \) holds.

The selected typing rules are shown in Figure~\ref{fig:refinement_type_system}.
The rules for the other constructs are identical to that of the simple type system; 
the complete list of typing rules appears in \ifaplas{the extended version~\cite{fullversion}}{Appendix~\ref{sec:refinement-apx}}. %
The rules shown in Figure~\ref{fig:refinement_type_system} are fairly standard rules for refinement type systems.
In \rn{RT-Out}, the notation
\(\predenv \vDash \pred\) means that \(\pred\) is a logical consequence of
\(\predenv\); for example, \(x<y, y<z \vDash x<z\) holds.
In the typing rules, we implicitly require that
all the type judgments are well-formed, in the sense that
all the integer variables occurring in a formula is
properly declared in \(\env\) or bound by a channel type constructor;
see \ifaplas{the extended version~\cite{fullversion}}{Appendix~\ref{sec:refinement-apx}}
for the well-formedness condition.
\begin{figure}[tb]
    \centering
    \small
    \begin{minipage}{\linewidth}
        \centering
        \begin{prooftree}
            \AxiomC{$\env;\predenv;\chenv\p x\COL\rch{\reg}{\seq{y}}{\pred}{\seq{\chty}}$}
            \AxiomC{$\env,\seq{y}\COL\seq{\ty}; \predenv,\pred; \chenv,\seq{z}\COL\seq{\chty} \p P$}
            \RightLabel{\textsc{(RT-In)}}
            \BinaryInfC{$\env;\predenv;\chenv\p \inexp{x}{\seq{y}}{\seq{z}}P$}
        \end{prooftree}
    \end{minipage}
    \\
    \begin{minipage}{0.93\linewidth}
        \infrule[RT-Out]
        {\env;\predenv;\chenv\p x\COL\rch{\reg}{\seq{y}}{\pred}{\seq{\chty}}\andalso
        \env;\predenv;\chenv\p \seq{v}\COL\seq{\ty}\andalso
        \predenv \vDash [\seq{v}/\seq{y}]\pred\\
        \env;\predenv;\chenv\p \seq{w}\COL[\seq{v}/\seq{y}]\seq{\chty}\andalso
        \env;\predenv;\chenv\p P}
        {\env;\predenv;\chenv \vdash \outexp{x}{\seq{v}}{\seq{w}}P}
    \end{minipage}
    \\
    \begin{minipage}{\linewidth}
        \centering
        \begin{prooftree}
            \AxiomC{$\env;\predenv;\chenv\p x\COL\rch{\reg}{\seq{y}}{\pred}{\seq{\chty}}$}
            \AxiomC{$\env,\seq{y}\COL\seq{\ty}; \predenv,\pred; \chenv,\seq{z}\COL\seq{\chty} \p P$}
            \RightLabel{\textsc{(RT-RIn)}}
            \BinaryInfC{$\env;\predenv;\chenv\p \rinexp{x}{\seq{y}}{\seq{z}}P$}
        \end{prooftree}
    \end{minipage}
    \\
    \begin{minipage}{\linewidth}
        \centering
        \begin{prooftree}
            \AxiomC{$\env; \predenv; \chenv \p v\COL\ty$}
            \AxiomC{$\env; \predenv, v \neq 0; \chenv \p P_1$}
            \AxiomC{$\env; \predenv, v =    0; \chenv \p P_2$}
            \RightLabel{\textsc{(RT-If)}}
            \TrinaryInfC{$\env;\predenv;\chenv\p \ifexp{v}{P_1}{P_2}$}
        \end{prooftree}
    \end{minipage}
    \\
    \begin{minipage}{0.45\linewidth}
        \centering
        \begin{prooftree}
            \AxiomC{$x \COL \chty \in \chenv$}
            \RightLabel{\textsc{(RT-Var-Ch)}}
            \UnaryInfC{$\env;\predenv;\chenv \p x\COL\chty$}
        \end{prooftree}
    \end{minipage}
    \normalsize
    \caption{\skchanged{Selected} typing rules of the refinement type system for the $\pi$-calculus}
    \label{fig:refinement_type_system}
\end{figure}



\subsection{Program Transformation}
\label{sec:rx}



Based on the refinement type system above,
we refine the transformation relation to
$\env; \predenv; \chenv \vdash P \Rightarrow \prog{\Def}{\Exp}$.
The only change is in the following rule for non-replicated inputs.\footnote{The rule for replicated inputs is also modified in a similar manner.}
\infrule[RX-In]
        {\env;\predenv;\chenv\p x\COL\rch{\reg}{\tilde{y}}{\pred}{\tilde{\chty}}
          \andalso
          \env,\seq{y}\COL\seq{\ty}; \predenv,\pred; \chenv,\seq{z}\COL\seq{\chty} \vdash P \Rightarrow \prog{\Def}{\Exp}}
        {\env;\predenv;\chenv \vdash \inexp{x}{\seq{y}}{\seq{z}}P
          \qquad\qquad\qquad\qquad\qquad\qquad\qquad\qquad         \\
          \Rightarrow
          \prog{\ndlet{y}{\textbf{Assume}(\pred);\Def}}{\ndlet{y}{\assexp{\pred}\Exp}}}
        Here, we insert \(\assatom{\pred}\), based on the refinement type
        of \(x\).
        The expression \(\ndlet{y}{\textbf{Assume}(\pred);\Exp}\) first
        instantiates \(\seq{y}\) to some integers in a non-deterministic manner,
        but proceeds to evaluate \(E\) only if the values of \(\seq{y}\) satisfy
        \(\pred\). Thus, the termination analysis for the target sequential program may assume that
       \(\seq{y}\) satisfies \(\pred\) in \(\Exp\).
        



\begin{example}  \label{ex:refined-transformation-for-f}
Let us explain how the process \( \DEC \) introduced in Example~\ref{ex:weakeness-of-basic-transformation} is translated by the refined translation.
Recall that the following simple types were assigned to the channels:
\begin{align*}
    \pre\COL \Chty{\reg_1}{\ty}{\sChty{\reg_2}{\ty}}, \quad
    \cname{f} \COL \Chty{\reg_3}{\ty}{\sChty{\reg_4}{\ty}}, \quad
    r \COL \sChty{\reg_4}{\ty}, \quad
    s \COL \sChty{\reg_2}{\ty}.
\end{align*}
By the refinement type system, the above types can be refined as:
\begin{align*}
    &\pre\COL \rch{\reg_1}{n}{\TRUE}{\sChty{\reg_2}{x; x < n}}, \quad
    \cname{f} \COL \rch{\reg_3}{n}{\TRUE}{\sChty{\reg_4}{x; \TRUE}}, \\
    &r \COL \sChty{\reg_4}{x; \TRUE}, \quad
    s \COL \sChty{\reg_2}{x; x < n}.
\end{align*}
For example, one can check that the output \( \soutatom{r}{n - 1} \) on the first line of \( \DEC \) is well-typed because \( \models [n - 1/x]x < n \) holds.
Note that this \( r \) is the variable bound by \( \inatom{\pre}{n}{r} \) and thus has the type \( \sChty{\reg_2}{x; x < n} \).

Therefore, by the rule \rn{RX-In}, the input \( \sinexp{s}{x}\outatom{f}{x}{r} \) is now translated as follows:
\infrule{\env;\predenv;\chenv\p s\COL\sChty{\reg_2}{x; x < n}
          \quad
          \env, x \COL \ty; \predenv,x < n; \chenv \vdash \outatom{f}{x}{r} \Rightarrow \prog{\Def}{\Fname{\reg_3}(x)}}
{\env;\predenv;\chenv \vdash \sinexp{s}{x}\outatom{f}{x}{r} \hspace{7cm}
          \qquad \\
          \Rightarrow
          \prog{(\ndlet*{x}{\textbf{Assume}(x < n);\Def})}{(\ndlet*{x}{\textbf{Assume}(x < n);\Fname{\reg_3}(x)})}}
with suitable \( \env \), \( \predenv \) and \( \chenv \).
By translating the whole process, we obtain
\begin{align*}
\Fname{\reg_3}(n) = \ &\textbf{if }{n<0} \textbf{ then } \Fname{\reg_4}(1) \\
    &\textbf{else }(\Fname{\reg_1}(n) \nondet \ndlet*{x}{\assexp{x < n}{\Fname{\reg_3}(x)}})
\end{align*}
as desired.
The other function definitions are given as in the case of Example~\ref{ex:weakeness-of-basic-transformation} (except for the fact that some redundant assertions \( \ndlet*{x}{\assatom{x < n}} \) are added).
\end{example}

%
\shchanged{
The soundness of the refined translation is obtained from the following argument.
We first extend the \(\pi\)-calculus with the $\ASSUME$ statement.
Then the refined translation can be decomposed into the following two steps:
(a) given a \(\pi\)-calculus process $P$,
insert $\ASSUME$ statements based on refinement types 
and obtain a process $P'$; and
(b) apply the translation of Section~\ref{sec:approach} to $P'$ (where $\ASSUME$ is just mapped to itself)
and obtain a sequential program $S$.
The soundness of step (b) follows by an easy modification of the proof\ifaplas{}{ in Appendix~\ref{sec:soundness}} 
for the basic transformation (just add the case for $\ASSUME$).
So, the termination of \(S\) would imply that of \(P'\).
Now, from the soundness of the refinement type system (which follows from a standard
argument on type preservation and progress), it follows that the $\ASSUME$ statements
inserted in step (a) always succeed. Thus,
the termination of \(P'\) would imply that of \(P\).
We can, therefore, conclude that if \(S\) is terminating, so is \(P\).
}
\newcommand{\constr}{\mathbb{C}}
\newcommand{\Inf}[4]{\fname{Inf}(#1; #2; #3; #4)}
\newcommand{\newty}[1]{\fname{NewTy}(#1)}
\newcommand{\newtys}[1]{\fname{NewTys}(#1)}
\newcommand{\subty}[3]{\fname{SubTy}(#1; #2; #3)}
\newcommand{\subtys}[3]{\fname{SubTys}(#1; #2; #3)}

\subsection{Type Inference}  \label{sec:inference}

This section discusses how to infer refinement types automatically
to automatically achieve the transformation.
As in refinement type inference for functional programs~\cite{Jhala08,Unno09PPDP,DBLP:journals/jar/ChampionCKS20},
we can reduce refinement type inference for the \(\pi\)-calculus to
the problem of CHC (Constrained Horn Clauses) solving~\cite{Bjorner15}.

We explain the procedure through an example.
Once again, we use the process \( \DEC \) introduced in Example~\ref{ex:weakeness-of-basic-transformation}.
We first perform type inference for the simple type system in Section~\ref{sec:targetlanguage}, and (as we have seen) obtain the following simple types for \( \pre \) and \( f \):
\begin{align*}
    \pre\COL \Chty{\reg_1}{\ty}{\sChty{\reg_2}{\ty}}, \quad
    \cname{f} \COL \Chty{\reg_3}{\ty}{\sChty{\reg_4}{\ty}}
\end{align*}
Here, we have omitted the types for other (bound) channels \(r,s,y\),
as they can be determined based on those of \( \pre \) and \( f \).
Based on the simple types, we prepare the following templates for refinement types.
\begin{align*}
  \pre\COL \rch{\reg_1}{n}{P_1(n)}{\sChty{\reg_2}{x; P_2(n,x)}}, \quad
  f\COL \rch{\reg_3}{n}{P_3(n)}{\sChty{\reg_4}{x; P_4(n,x)}}.
\end{align*}
Here, \(P_i\) (\(i\in\set{1,\ldots,4}\)) is a predicate variable that represents
unknown conditions.

Based on the refinement type system, we can generate the following constraints on
the predicate variables.
\[
\begin{array}{l}
  \forall n.(P_1(n) \imp P_2(n, n-1))\qquad
  \forall n.(P_3(n)\land n<0 \imp P_4(n, 1)) \\
  \forall n.(P_3(n)\land n\ge 0 \imp P_1(n-1))\\
  \forall n,x.(P_3(n)\land n\ge 0\land P_2(n-1,x) \imp P_3(x))\\
  \forall m.(\TRUE \imp P_3(m))
\end{array}
\]
Here, the first constraint comes from the first line of the process,
and the second constraint (the third and fourth constraints, resp.)
comes from the then-part (the else-part, resp.)
of the second line of the process. The last constraint
comes from \(\outatom{f}{m}{r}\).

The generated constraints are in general a set of \emph{Constrained Horn Clauses}
(CHCs)~\cite{Bjorner15} of the form
\(\forall \seq{x}.( P_1(\seq{v}_1)\land \cdots \land P_k(\seq{v}_k)\land \pred \imp H)\),
where \(P_1,\ldots,P_k\) are predicate variables, \(\pred\) is a formula
of integer arithmetic (without predicate variables),
and \(H\) is either of the form \(P(\seq{v})\) or \(\pred'\).
The problem of finding a solution (i.e. an assignment of predicates to
predicate variables) of a set of CHCs is undecidable in general,
but there are various automated tools (called CHC solvers)
for solving the problem~\cite{DBLP:journals/fmsd/KomuravelliGC16,DBLP:journals/jar/ChampionCKS20}.
Thus, by using such a CHC solver, we can solve the constraints on predicate variables,
and obtain refinement types by substituting the solution for the templates of
refinement types.

For the example above, the following is a solution.
\[
\begin{array}{l}
  P_1(n)\equiv \TRUE\qquad P_2(n,x)\equiv x < n\qquad 
P_3(x) \equiv \TRUE \qquad P_4(n,x)\equiv \TRUE.
\end{array}
\]
This is exactly the predicates we used in Example~\ref{ex:refined-transformation-for-f} to translate \( \DEC \) using the refined approach.

\subsubsection*{Adding extra CHCs.}
Actually, a further twist is necessary in the step of CHC solving.
As in the example above, all the CHCs generated based on the refinement typing rules
are of the form \(\cdots \imp P_i(\seq{v})\) (i.e., the head of every CHC is
an atomic formula on a predicate variable).
Thus, there always exists a trivial solution for the CHCs, which instantiates
all the predicate variables to \(\TRUE\).
For the example above,
\[
\begin{array}{l}
  P_1(n)\equiv \TRUE\qquad P_2(n,x)\equiv \TRUE\qquad 
P_3(n) \equiv \TRUE \qquad P_4(n,x)\equiv \TRUE
\end{array}
\]
is also a solution,
but using the trivial solution,
our transformation yields the non-terminating program.
This program is essentially the same as the one in Example~\ref{ex:weakeness-of-basic-transformation} since \( \ndlet*{x}{\assexp{\TRUE}}\Exp \) is equivalent to \( \ndlet*{x}\Exp \).
Typical CHC solvers indeed tend to find the trivial solution.

To remedy the problem above, in addition to the CHCs generated from the typing rules,
we add extra constraints that prevent infinite loops.
For the example above, the definition of \(\Fname{\reg_3} \) (which corresponds to the channel \( f \)) in the translated program is of the form
\[  \Fname{\reg_3}(n) = \ifexp{\!n<0\!}{\!\skipexp\!}{\Fname{\reg_1}(n) \nondet  (\letexp{x}{\ndint}\assexp{P_2(n,x)}\Fname{\reg_3}(x))}.
\]
Thus we add the clause:
\[ P_2(n,x) \imp n\ne x\]
to prevent an infinite loop \(\Fname{\reg_3}(m)\red \Fname{\reg_3}(m) \red \cdots\).
With the added clause, a CHC solver \hoice{}~\cite{DBLP:journals/jar/ChampionCKS20}
indeed returns \(n<x\) as the solution for \(P_2(n,x)\).

In general, we can add the extra CHCs in the following, counter-example-guided manner.
\begin{enumerate}
\item \(\mathcal{C} := \) the CHCs generated from the typing rules
\item \(\theta := \mathit{callCHCsolver}(\mathcal{C})\)
\item \(S := \) the sequential program generated based on the solution \(\theta\)
\item if \(S\) is terminating then return OK; otherwise, 
 analyze \( S \) to find an infinite reduction sequence, add an extra clause to \(\mathcal{C}\) to disable the infinite sequence, and go back to 2.
\end{enumerate}
\changed{
More precisely, in the last step, the backend termination analysis tool generates a lasso
as a certificate of non-termination. We extract 
a chain $f(\seq{x}) \to \dots \to f(\seq{x}')$ of recursive calls
from the lasso, and
add an extra clause requiring $\seq{x} \neq \seq{x}'$
to \(\mathcal{C}\). This is naive and insufficient
for excluding out an infinite sequence like \(f(1)\to f(2) \to f(3) \to \cdots\).
We plan to refine the method by incorporating more sophisticated techniques
developed for sequential programs~\cite{hashimoto2015refinement}.
}
\section{Implementation: Ring Abstraction}
\label{sec:implement}
\subsection{Distributed \mbox{$G_t$} in QMC Solver}
\label{distributedG4}
Before introducing the communication phase of the ring abstraction layer,
it is important to understand how the authors distributed the large device array $G_t$ across MPI ranks.
%
Original $G_t$ was compared, and $G^d_t$ versions were distributed
(Figure~\ref{fig:compare_original_distributed_g4}). 


In the original $G_t$ implementation, the measurements---which were computed by matrix-matrix multiplication---are distributed statically and independently over the MPI ranks to avoid
inter-node communications. Each MPI rank keeps its partial copy of $G_{t,i}$ to accumulate 
measurements within a rank, where $i$ is the rank index. 
After all the measurements are finished, a reduction step is 
taken to accumulate $G_{t,i}$ across all MPI ranks into a final and complete
$G_t$ in the root MPI rank. The size of the $G_{t,i}$ in each rank is 
the same size as the final and complete $G_t$. 

With the distributed $G^d_t$ implementation, this large device array 
$G_t$ was evenly partitioned across all MPI ranks; each portion of it is local to each MPI rank.
%
Instead of keeping its partial copy of $G_t$, 
each rank now keeps an instance of $G^d_{t,i}$ to accumulate measurements 
of a portion or sub-slice of the final and complete $G_t$, where the notation
$d$ in $G^d_t$  refers to the distributed version, and $i$ means the $i$-th rank.
%
The $G^d_{t,i}$ size in each rank is 
reduced to $1/p$ of the size of the final and complete $G_t$, comparing the same configuration 
in original $G_t$ implementation, where $p$ is the number of MPI ranks used. 
%
For example, in Figure~\ref{fig:distributed_g4}, there are four ranks, and rank $i$
now only keeps $G^d_{t,i}$, which is one-fourth the size of the original $G_t$ array size.
% and contains values indexing from range of $[0, ..., N/4)$ in original $G_t$ array where $N$ is the 
% number of entries in  $G_t$  when viewed as a one-dimensional array.

To compute the final and complete $G^d_{t,i}$ for the distributed $G^d_t$ implementation, 
each rank must see every $G_{\sigma,i}$ from all ranks. 
%
In other words, each rank must broadcast the
locally generated $G_{\sigma,i}$ to the remainder of the other ranks at every measurement step. 
%
To efficiently perform this ``all-to-all'' broadcast, a ring abstraction layer was built (Section. \ref{section:ring_algorithm}), which circulates
all $G_{\sigma,i}$ across all ranks.

% over high-speed GPUs interconnect (GPUDirect RDMA) to facilitate the communication phase.

% \begin{figure}
% \centering
% \subfloat[Original $G_t$ implementation.]
%     {\includegraphics[width=\columnwidth]{original_g4.pdf}}\label{fig:original_g4}

% \subfloat[Distributed $G_t$ implementation.]
%     {\includegraphics[width=0.99\columnwidth]{distributed_g4.pdf} \label{fig:distributed_g4}}

% \caption{Comparison of the original $G_t$ vs. the distributed $G^d_t$ implementation. Each rank contains one GPU resource.}
% \label{fig:compare_original_distributed_g4} 
% \end{figure} 

\begin{figure}
\centering
     \begin{subfigure}[b]{\columnwidth}
         \centering
         \includegraphics[width=\textwidth]{images/original_g4.pdf}
         \caption{Original $G_t$ implementation.}
         \label{fig:original_g4}
     \end{subfigure}
     
    \begin{subfigure}[b]{\columnwidth}
         \centering
         \includegraphics[width=\textwidth]{images/distributed_g4.pdf}
         \caption{Distributed $G_t$ implementation.}
         \label{fig:distributed_g4}
     \end{subfigure}
     
\caption{Comparison of the original $G_t$ vs. the distributed $G^d_t$ implementation. Each rank contains one GPU resource.}
\label{fig:compare_original_distributed_g4}
\end{figure}

\subsection{Pipeline Ring Algorithm}
\label{section:ring_algorithm}
A pipeline ring algorithm was implemented that broadcasts the $G_{\sigma}$ 
array circularly during every measurement. 
%
The algorithm (Algorithm \ref{alg:ring_algorithm_code}) is 
visualized in Figure~\ref{fig:ring_algorithm_figure}.

\begin{algorithm}
\SetAlgoLined
    generateGSigma(gSigmaBuf)\; \label{lst:line:generateG2}
    updateG4(gSigmaBuf)\;       \label{lst:line:updateG4}
    %\texttt{\\}
    {$i\leftarrow 0$}\;         \label{lst:line:initStart}
    {$myRank \leftarrow worldRank$}\;          \label{lst:line:initRankId}
    {$ringSize \leftarrow mpiWorldSize$}\;      \label{lst:line:initRingSize}
    {$leftRank \leftarrow (myRank - 1 + ringSize) \: \% \: ringSize $}\;
    {$rightRank \leftarrow (myRank + 1 + ringSize) \: \% \: ringSize $}\;
    sendBuf.swap(gSigmaBuf)\;           \label{lst:line:initEnd}
    \While{$i< ringSize$}{
        MPI\_Irecv(recvBuf, source=leftRank, tag = recvTag, recvRequest)\; \label{lst:line:Irecv}
        MPI\_Isend(sendBuf, source=rightRank, tag = sendTag, sendRequest)\; \label{lst:line:Isend}
        MPI\_Wait(recvRequest)\;        \label{lst:line:recevBuffWait}
        
        updateG4(recvBuf)\;             \label{lst:line:updateG4_loop}
        
        MPI\_Wait(sendRequest)\;        \label{lst:line:sendBuffWait}
        
        sendBuf.swap(recvBuf)\;         \label{lst:line:swapBuff}
        i++\;
    }
\caption{Pipeline ring algorithm}
\label{alg:ring_algorithm_code}
\end{algorithm}

\begin{figure}
	\centering
	\includegraphics[width=\columnwidth, trim=0 5cm 0 0, clip]{images/ring_algorithm.pdf}
	\caption{Workflow of ring algorithm per iteration. }
	\label{fig:ring_algorithm_figure}
\end{figure}

At the start of every new measurement, a single-particle Green's function $G_{\sigma}$
 (Line~\ref{lst:line:generateG2}) is generated 
and then used to update $G^d_{t,i}$ (Line~\ref{lst:line:updateG4})
via the formula in Eq.~(\ref{eq:G4}).
%
% Different from original method that performs 
% full matrix-matrix multiplication (Equation~(\ref{eq:G4})), the current ring algorithm only performs partial
% matrix-matrix multiplication that contributes to $G^d_{t,i}$, a subslice of the final and complete $G_t$.
%
Between Lines \ref{lst:line:initStart} to \ref{lst:line:initEnd}, the algorithm 
initializes the indices
of left and right neighbors and prepares the sending message buffer from the
previously generated $G_{\sigma}$ buffer. 
%
The processes are organized as a ring so that the first and last rank are considered to be neighbors to each other. 
%
A \textit{swap} operation is used to avoid unnecessary memory copies for \textit{sendBuf} preparation.
%
A walker-accumulator thread allocates an additional \textit{recvBuf} buffer of the same size 
as \textit{gSigmaBuf} to hold incoming 
\textit{gSigmaBuf} buffer from \textit{leftRank}. 

The \textit{while} loop is the core part of the pipeline ring algorithm. 
%
For every iteration, each thread in a rank 
receives a $G_{\sigma}$ buffer from its left neighbor rank and sends a $G_{\sigma}$ buffer to its right neighbor rank. 
A synchronization step (Line~\ref{lst:line:recevBuffWait}) is performed
afterward to ensure that each rank receives a new buffer to update the 
local $G^d_{t,i}$ (Line~\ref{lst:line:updateG4_loop}). 
%
Another synchronization step
follows to ensure that all send requests are finalized 
(Line~\ref{lst:line:sendBuffWait}). Lastly, another \textit{swap} operation is used to exchange
content pointers between \textit{sendBuf} and \textit{recvBuf} to avoid unnecessary memory copy and prepare
for the next iteration of communication.
%
In the multi-threaded version (Section~\ref{subsec:multi-thread}), the thread of index, \textit{i}, only communicates with
	threads of index, \textit{i}, in neighbor ranks, and each thread allocates two buffers: \textit{sendBuff} and \textit{recvBuff}.

The \textit{while} loop will be terminated after $\mbox{\textit{ringSize}} - 1$ steps. By that time, 
each locally generated $G_{\sigma,i}$ will have traveled across all MPI ranks and
updated $G^d_{t,i}$ in all ranks. Eventually, each $G_{\sigma,i}$ reaches
to the left neighbor of its birth rank. For example, $G_{\sigma,0}$ generated from rank $0$ will end 
in last rank in the ring communicator.

Additionally, if the $G_t$ is too large to be stored in one node, 
it is optional to accumulate all $G^d_{t,i}$
at the end of all measurements. 
%
Instead, a parallel write into the file system could be taken.

\subsubsection{Sub-Ring Optimization.}

A sub-ring optimization strategy is further proposed to reduce message communication
times if the large device array $G_t$ can fit in fewer devices. 
%
The sub-ring algorithm is visualized in Figure~\ref{fig:subring_algorithm_figure}.

For the ring algorithm (Section~\ref{section:ring_algorithm}), the size of the ring communicator
(\textit{mpiWorldSize}) is set to the same size of the global \mbox{\texttt{MPI\_COMM\_WORLD}}, and thus the size of $G_t$ is equally 
distributed across all MPI ranks.

However, to complete the update to $G^d_{t,i}$ in one measurement, 
one $G_{\sigma,i}$
must travel \textit{mpiWorldSize} ranks. In total, 
there are \textit{mpiWorldSize} numbers of $G_{\sigma,i}$
being sent and received concurrently in one measurement 
in the global
\mbox{\texttt{MPI\_COMM\_WORLD}} 
communicator. If the size of $G^d_{t,i}$ is relatively small per rank, then this will cause high communication overhead.

If $G_t$ can be distributed and fitted in fewer devices, then a shorter travel distance is required 
for $G_{\sigma,i}$, thus reducing the communication overhead. One reduction
step was performed at the end of all measurements to accumulate $G^d_{t,s_i}$, 
where $s_i$ means $i$-th rank on the $s$-th sub-ring.

At the beginning of MPI initialization, the global \mbox{\texttt{MPI\_COMM\_WORLD}} was partitioned  into several new sub-ring communicators by using \mbox{\texttt{MPI\_Comm\_split}}. 
% where each new communicator represents conceptually a subring. 
The new
communicator information was passed to the DCA++ concurrency class by substituting the original global 
\mbox{\texttt{MPI\_COMM\_WORLD}} with this new communicator. Now, only a few minor modifications
are needed to transform the ring algorithm (Algorithm~\ref{alg:ring_algorithm_code})
to sub-ring Algorithm~\ref{alg:sub_ring_algorithm}. In Line~\ref{lst:line:initRankId}, \textit{myRank} is 
initialized to \textit{subRingRank} instead of \textit{worldRank}, where 
\textit{subRingRank} is the rank index in the local sub-ring communicator. 
%
In Line~\ref{lst:line:initRingSize}, \textit{ringSize} is initialized to \textit{subRingSize}
instead of \textit{mpiWorldSize}, where \textit{subRingSize} is the
size of the new communicator.
%
The general ring algorithm is a special case for the sub-ring algorithm because the
\textit{subRingSize} of the general ring algorithm is equal to \textit{mpiWorldSize}, and
there is only one sub-ring group throughout all MPI ranks.


\LinesNumberedHidden
\begin{algorithm}
    {$\mbox{\textit{myRank}} \leftarrow \mbox{\textit{subRingRank}}$}\;         
    {$\mbox{\textit{ringSize}} \leftarrow \mbox{\textit{subRingSize}}$}\;      
\caption{Modified ring algorithm to support sub-ring communication}
\label{alg:sub_ring_algorithm}
\end{algorithm}


\begin{figure}
	\centering
	\includegraphics[width=\columnwidth, trim=0 5cm 0 0, clip]{images/subring_alg.pdf}
	\caption{Workflow of sub-ring algorithm per iteration. Every consecutive $S$ rank forms a sub-ring communicator, 
	and no communication occurs between sub-ring communicators until all measurements are finished. Here, $S$ is the number of ranks in a sub-ring.}
	\label{fig:subring_algorithm_figure}
\end{figure}

\subsubsection{Multi-Threaded Ring Communication.}
\label{subsec:multi-thread}
To take advantage of the multi-threaded QMC model already in DCA++, 
multi-threaded ring communication support was further implemented in the ring algorithm.
%
Figure~\ref{fig:dca_overview} shows that in the original DCA++ method,
each walker-accumulator
thread in a rank is independent of each other, and all the threads in a 
rank synchronize only after all rank-local measurements are finished.
%
Moreover, during every measurement, each walker-accumulator thread
generates its own thread-private $G_{\sigma, i}$ to update $G_t$. 
%

The multi-threaded ring algorithm now allows concurrent message exchange so that threads of same rank-local thread index exchange their thread-private $G_{\sigma, i}$. 
%
Conceptually, there are $k$ parallel and independent rings, where $k$ 
is number of threads per rank, because threads of the same local thread ID
form a closed ring. 
%
For example, a thread of index $0$ in rank $0$ will send its $G_\sigma$ to 
the thread of index $0$ in rank $1$ and receive another $G_\sigma$ from thread index of $0$ 
from last rank in the ring algorithm.
%

The only changes in the ring algorithm are offsetting the tag values 
(\texttt{recvTag} and \texttt{sendTag}) by the thread index value. For example,
Lines~\ref{lst:line:Irecv} and ~\ref{lst:line:Isend} from 
Algorithm~\ref{alg:ring_algorithm_code} are modified to Algorithm~\ref{alg:multi_threaded_ring}.

\LinesNumberedHidden
\begin{algorithm}
        MPI\_Irecv(recvBuf, source=leftRank, tag = recvTag + threadId, recvRequest)\; 
        MPI\_Isend(sendBuf, source=rightRank, tag = sendTag + threadId, sendRequest)\;
\caption{Modified ring algorithm to support multi-threaded ring}
\label{alg:multi_threaded_ring}
\end{algorithm}

To efficiently send and receive $G_\sigma$, each thread
will allocate one additional \textit{recvBuff} to hold incoming 
\textit{gSigmaBuf} buffer from \textit{leftRank} and perform send/receive efficiently.
%
In the original DCA++ method, there are $k$ numbers of buffers of $G_\sigma$ 
size per rank, and in the multi-threaded ring method, there are $2k$
numbers of buffers of $G_\sigma$ size per rank, where $k$ is number of 
threads per rank.

\section{Related Work}
\label{sec:related_work}
We now provide a brief overview of related work in the areas of language grounding and transfer for reinforcement learning.
%There has been work on learning to make optimal local decisions for structured prediction problems~\cite{daume2006searn}.
%
%\newcite{branavan2010reading} looked at a similar task of building a partial model of the environment while following instructions. The differences with our work are (1) the text in their case is instructions, while we only have text describing the environment, and (2) their environment is deterministic, hence the transition function can be learned more easily. 
%
%TODO - model-based RL, value iteration, predictron.


\subsection{Grounding Language in Interactive Environments}
In recent years, there has been increasing interest in systems that can utilize textual knowledge to learn control policies. Such applications include interpreting help documentation~\fullcite{branavan2010reading}, instruction following~\fullcite{vogel2010learning,kollar2010toward,artzi2013weakly,matuszek2013learning,Andreas15Instructions} and learning to play computer games~\fullcite{branavan2011nonlinear,branavan2012learning,narasimhan2015language,he2016deep}. In all these applications, the models are trained and tested on the same domain.

Our work represents two departures from prior work on grounding. First, rather than optimizing control performance for a single domain,
we are interested in the multi-domain transfer scenario, where language 
descriptions drive generalization. Second, prior work used text in the form of strategy advice to directly learn the policy. Since the policies are typically optimized for a specific task, they may be harder to transfer across domains. Instead, we utilize text to bootstrap the induction of the environment dynamics, moving beyond task-specific strategies. 

%Previous work has explored the use of text manuals in game playing, %ranging from constructing useful features by mining patterns in %text~\cite{eisenstein2009reading}, learning a semantic interpreter %with access to limited gameplay examples~\cite{goldwasser2014learning} %to learning through reinforcement from in-game %rewards~\cite{branavan2011learning}. These efforts have demonstrated %the usefulness of exploiting domain knowledge encoded in text to learn %effective policies. However, these methods use the text to infer %directly the best strategy to perform a task. In contrast, our goal is %to learn mappings from the text to the dynamics of an environment and %separate out the learning of the strategy/motives. 

Another related line of work consists of systems that learn spatial and topographical maps of the environment for robot navigation using natural language descriptions~\fullcite{walter2013learning,hemachandra2014learning}. These approaches use text mainly containing appearance and positional information, and integrate it with other semantic sources (such as appearance models) to obtain more accurate maps. In contrast, our work uses language describing the dynamics of the environment, such as entity movements and interactions, which 
is complementary to static positional information received through state observations. Further, our goal is to help an agent learn policies that generalize over different stochastic domains, while their works consider a single domain.

%karthik: I don't see the direct relevance
%Another line of work explores using textual interactive %environments~\cite{narasimhan2015language,he2016deep} to ground %language understanding into actions taken by the system in the %environment. In these tasks, understanding of language is crucial, %without which a system would not be able to take reasonable actions. %Our motivation is different -- we take an existing set of tasks and %domains which are amenable to learning through reinforcement, and %demonstrate how to utilize textual knowledge to learn faster and more %optimal policies in both multitask and transfer setups.

\subsection{Transfer in Reinforcement Learning}
Transferring policies across domains is a challenging problem in reinforcement learning~\fullcite{konidaris2006framework,taylor2009transfer}. The main hurdle lies in learning a good mapping between the state and action spaces of different domains to enable effective transfer. Most previous approaches have either explored skill transfer~\fullcite{konidaris2007building,konidaris2012transfer} or value function/policy transfer~\fullcite{liu2006value,taylor2007transfer,taylor2007cross}. There have also been attempts at model-based transfer for RL~\fullcite{taylor2008transferring,nguyen2012transferring,gavsic2013pomdp,wang2015learning,joshi2018cross} but these methods either rely on hand-coded inter-task mappings for state and actions spaces or require significant interactions in the target task to learn an effective mapping. Our approach doesn't use any explicit mappings and can learn to predict the dynamics of a target task using its descriptions.

% Work by \newcite{konidaris2006autonomous} look at knowledge transfer by learning a mapping from sensory signals to reward functions.

A closely related line of work concerns transfer methods for deep reinforcement learning. \citeA{parisotto2016actor}  train a deep network to mimic pre-trained experts on source tasks using policy distillation. The learned parameters are then used to initialize a network on a target task to perform transfer. Rusu et al.~\citeyear{rusu2016progressive} facilitate transfer by freezing parameters learned on source tasks and adding a new set of parameters for every new target task, while using both sets to learn the new policy. Work by Rajendran et al.~\citeyear{rajendran20172t} uses attention networks to selectively transfer from a set of expert policies to a new task. \textcolor{black}{Barreto et al.~\citeyear{barreto2017successor} use features based on successor representations~\fullcite{dayan1993improving} for transfer across tasks in the same domain. Kansky~et~al.~\citeyear{kansky2017schema} learn a generative model of causal physics in order to help zero-shot transfer learning.} Our approach is orthogonal to all these directions since we use text to bootstrap transfer, and can potentially be combined with these methods to achieve more effective transfer. 

\textcolor{black}{There has also been prior work on zero-shot policy generalization for tasks with input goal specifications. \fullciteA{schaul2015universal} learn a universal value function approximator that can generalize across both states and goals. \fullcite{andreas2016modular} use policy sketches, which are annotated sequences of subgoals, in order to learn a hierarchical policy that can generalize to new goals. \fullciteA{oh2017zero} investigate zero-shot transfer for instruction following tasks, aiming to generalize to unseen instructions in the same domain. The main departure of our work compared to these is in the use of environment descriptions for generalization across domains rather than generalizing across text instructions.}

Perhaps closest in spirit to our hypothesis is the recent work by~\fullcite{harrison2017guiding}. Their approach makes use of paired instances of text descriptions and state-action information from human gameplay to learn a machine translation model. This model is incorporated into a policy shaping algorithm to better guide agent exploration. Although the motivation of language-guided control policies is similar to ours, their work considers transfer across tasks in a single domain, and requires human demonstrations to learn a policy.

\textcolor{black}{
\subsection{Using Task Features for Transfer}
The idea of using task features/dictionaries for zero-shot generalization has been explored previously in the context of image classification. \fullciteA{kodirov2015unsupervised} learn a joint feature embedding space between domains and also induce the corresponding projections onto this space from different class labels. 
\fullciteA{kolouri2018joint} learn a joint dictionary across visual features and class attributes using sparse coding techniques. \fullciteA{romera2015embarrassingly} model the relationship between input features, task attributes and classes as a linear model to achieve efficient yet simple zero-shot transfer for classification. \fullciteA{socher2013zero} learn a joint semantic representation space for images and associated words to perform zero-shot transfer.}

\textcolor{black}{
Task descriptors have also been explored in zero-shot generalization for control policies. \fullciteA{sinapov2015learning} use task meta-data as features to learn a mapping between pairs of tasks. This mapping is then used to select the most relevant source task to transfer a policy from. \fullciteA{isele2016using} build on the ELLA framework~\fullcite{ruvolo2013ella,ammar2014online}, and their key idea is to maintain two shared linear bases across tasks -- one for the policy ($L$) and the other for task descriptors ($D$). Once these bases are learned on a set of source tasks, they can be used to predict policy parameters for a new task given its corresponding descriptor. 
% The training scheme is similar to Actor-mimic scheme~\cite{parisotto2016actor} -- for each task, an expert policy is trained separately and then distilled into policy parameters dependent on the shared basis $L$. 
In these lines of work, the task features were either manually engineered or directly taken from the underlying system parameters defining the dynamics of the environment. In contrast, our framework only requires access to crowd-sourced textual descriptions, alleviating the need for expert domain knowledge.}





% A major difference in our work is that we utilize natural language descriptions of different environments to bootstrap transfer, requiring less exploration in the new task.

% using a policy distillation~\cite{parisotto2016actor,rusu2016progressive,yin2017knowledge} or selective attention over expert networks learnt in the source tasks~\cite{rajendran20172t}. Though these approaches provide some benefits, they still suffer from the requirement of efficiently exploring the new environment to learn how to transfer their existing policies. In contrast, we utilize natural language descriptions of different environments to bootstrap transfer, leading to more focused exploration in the target task. 


% Describe amn in detail





% \vspace{-0.5em}
\section{Conclusion}
% \vspace{-0.5em}
Recent advances in multimodal single-cell technology have enabled the simultaneous profiling of the transcriptome alongside other cellular modalities, leading to an increase in the availability of multimodal single-cell data. In this paper, we present \method{}, a multimodal transformer model for single-cell surface protein abundance from gene expression measurements. We combined the data with prior biological interaction knowledge from the STRING database into a richly connected heterogeneous graph and leveraged the transformer architectures to learn an accurate mapping between gene expression and surface protein abundance. Remarkably, \method{} achieves superior and more stable performance than other baselines on both 2021 and 2022 NeurIPS single-cell datasets.

\noindent\textbf{Future Work.}
% Our work is an extension of the model we implemented in the NeurIPS 2022 competition. 
Our framework of multimodal transformers with the cross-modality heterogeneous graph goes far beyond the specific downstream task of modality prediction, and there are lots of potentials to be further explored. Our graph contains three types of nodes. While the cell embeddings are used for predictions, the remaining protein embeddings and gene embeddings may be further interpreted for other tasks. The similarities between proteins may show data-specific protein-protein relationships, while the attention matrix of the gene transformer may help to identify marker genes of each cell type. Additionally, we may achieve gene interaction prediction using the attention mechanism.
% under adequate regulations. 
% We expect \method{} to be capable of much more than just modality prediction. Note that currently, we fuse information from different transformers with message-passing GNNs. 
To extend more on transformers, a potential next step is implementing cross-attention cross-modalities. Ideally, all three types of nodes, namely genes, proteins, and cells, would be jointly modeled using a large transformer that includes specific regulations for each modality. 

% insight of protein and gene embedding (diff task)

% all in one transformer

% \noindent\textbf{Limitations and future work}
% Despite the noticeable performance improvement by utilizing transformers with the cross-modality heterogeneous graph, there are still bottlenecks in the current settings. To begin with, we noticed that the performance variations of all methods are consistently higher in the ``CITE'' dataset compared to the ``GEX2ADT'' dataset. We hypothesized that the increased variability in ``CITE'' was due to both less number of training samples (43k vs. 66k cells) and a significantly more number of testing samples used (28k vs. 1k cells). One straightforward solution to alleviate the high variation issue is to include more training samples, which is not always possible given the training data availability. Nevertheless, publicly available single-cell datasets have been accumulated over the past decades and are still being collected on an ever-increasing scale. Taking advantage of these large-scale atlases is the key to a more stable and well-performing model, as some of the intra-cell variations could be common across different datasets. For example, reference-based methods are commonly used to identify the cell identity of a single cell, or cell-type compositions of a mixture of cells. (other examples for pretrained, e.g., scbert)


%\noindent\textbf{Future work.}
% Our work is an extension of the model we implemented in the NeurIPS 2022 competition. Now our framework of multimodal transformers with the cross-modality heterogeneous graph goes far beyond the specific downstream task of modality prediction, and there are lots of potentials to be further explored. Our graph contains three types of nodes. while the cell embeddings are used for predictions, the remaining protein embeddings and gene embeddings may be further interpreted for other tasks. The similarities between proteins may show data-specific protein-protein relationships, while the attention matrix of the gene transformer may help to identify marker genes of each cell type. Additionally, we may achieve gene interaction prediction using the attention mechanism under adequate regulations. We expect \method{} to be capable of much more than just modality prediction. Note that currently, we fuse information from different transformers with message-passing GNNs. To extend more on transformers, a potential next step is implementing cross-attention cross-modalities. Ideally, all three types of nodes, namely genes, proteins, and cells, would be jointly modeled using a large transformer that includes specific regulations for each modality. The self-attention within each modality would reconstruct the prior interaction network, while the cross-attention between modalities would be supervised by the data observations. Then, The attention matrix will provide insights into all the internal interactions and cross-relationships. With the linearized transformer, this idea would be both practical and versatile.

% \begin{acks}
% This research is supported by the National Science Foundation (NSF) and Johnson \& Johnson.
% \end{acks}

\subsubsection*{Acknowledgments}
We would like to thank anonymous referees for useful comments.
This work was supported by JSPS KAKENHI Grant Number JP20H05703.

%
\bibliographystyle{splncs04}
\bibliography{ref,abbrv,koba}
%
\ifaplas{}{
\clearpage
\appendix
\section{Operational Semantics} \label{sec:operational_semantics}
\subsection{Reduction Semantics of the \( \pi \)-Calculus}
We define a reduction relation~\cite{milner1993polyadic} as the operational semantics of the $\pi$-calculus.

As usual, we first define the structural congruence relation \( \piequiv \) on the set of processes.
\begin{definition}[structural congruence for processes]
  The \emph{structural congruence relation} $\piequiv$ on %
  $\pi$-calculus processes is defined as the least congruence relation that satisfies the following %
  rules.
    \begin{gather*}
        P_1 \mid P_2 \piequiv P_2 \mid P_1
        \qquad (P_1 \mid P_2) \mid P_3 \piequiv P_1 \mid (P_2 \mid P_3) 
        \\     P \mid \textbf{0} \piequiv P 
        \qquad (\nu x)\,\textbf{0} \piequiv \textbf{0}
        \qquad (\nu x)(\nu y)P \piequiv (\nu y)(\nu x)P
        \\     (\nu x)(P_1 \mid P_2) \piequiv P_1 \mid (\nu x)P_2 \quad \text{if $x$ does not freely occur in $P_1$}
    \end{gather*}
\end{definition}

Next, we define the reduction relation on processes. %
\begin{definition}
  The \emph{reduction relation $\to$ on %
    processes} is defined by the set of rules in Figure~\ref{fig:pi_reduction}.
    We write $\to^*$ and \( \to^+ \)for the reflexive transitive closure and the transitive closure of the reduction relation $\to$, respectively.
\end{definition}
\begin{figure}[tbp]
    \centering
    \small
    \begin{minipage}{\linewidth}
        \centering
        \begin{prooftree}
            \AxiomC{$\len{\seq{y}} = \len{\seq{v}}$}
            \AxiomC{$\len{\seq{z}} = \len{\seq{w}}$}
            \AxiomC{$\seq{v} \Downarrow \seq{i}$}
            \RightLabel{\textsc{(R-Comm)}}
            \TrinaryInfC{$\inexp{x}{\seq{y}}{\seq{z}}P_1 \PAR  \outexp{x}{\seq{v}}{\seq{w}}P_2 \red [\seq{i}/ \seq{y}, \seq{w} / \seq{z} ]P_1 \PAR P_2$}
        \end{prooftree}
    \end{minipage}
    \begin{minipage}{0.48\linewidth}
        \centering
        \begin{prooftree}
            \AxiomC{$P_1 \red P_1'$}
            \RightLabel{\textsc{(R-Par)}}
            \UnaryInfC{$P_1 \PAR P_2 \red P_1' \PAR P_2$}
        \end{prooftree}
    \end{minipage}
    \begin{minipage}{0.5\linewidth}
        \centering
        \begin{prooftree}
            \AxiomC{$P \red P'$}
            \RightLabel{\textsc{(R-Nu)}}
            \UnaryInfC{$\nuexp{x \COL \chty} P \red \nuexp{x \COL \chty}P' $}
        \end{prooftree}
    \end{minipage}
    \begin{minipage}{\linewidth}
        \centering
        \begin{prooftree}
            \AxiomC{$\len{\seq{y}} = \len{\seq{v}}$}
            \AxiomC{$\len{\seq{z}} = \len{\seq{w}}$}
            \AxiomC{$\seq{v} \Downarrow \seq{i}$}
            \RightLabel{\textsc{(R-RComm)}}
            \TrinaryInfC{$\rinexp{x}{\seq{y}}{\seq{z}}P_1 \PAR  \outexp{x}{\seq{v}}{\seq{w}}P_2 \red \rinexp{x}{\seq{y}}{\seq{z}}P_1 \PAR [\seq{i}/ \seq{y}, \seq{w} / \seq{z}]P_1 \PAR P_2$}
        \end{prooftree}
    \end{minipage}
    \begin{minipage}{\linewidth}
        \centering
        \begin{prooftree}
            \AxiomC{$v \Downarrow i \neq 0$}
            \RightLabel{\textsc{(R-If-T)}}
            \UnaryInfC{$\ifexp{v}{P_1}{P_2} \red P_1$}
        \end{prooftree}
    \end{minipage}
    \begin{minipage}{\linewidth}
        \centering
        \begin{prooftree}
            \AxiomC{$v \Downarrow 0$}
            \RightLabel{\textsc{(R-If-F)}}
            \UnaryInfC{$\ifexp{v}{P_1}{P_2} \red P_2$}
        \end{prooftree}
    \end{minipage}
    \begin{minipage}{\linewidth}
        \centering
        \begin{prooftree}
            \AxiomC{\( \len{\seq{x}} = \len{\seq{i}}\)}
            \RightLabel{\textsc{(R-LetND)}}
            \UnaryInfC{$\ndlet{x}{P} \red [\seq{i} / \seq{x}]P$}
        \end{prooftree}
    \end{minipage}
    \begin{minipage}{\linewidth}
        \centering
        \begin{prooftree}
            \AxiomC{$P \piequiv P_1 \red P_1' \piequiv P'$}
            \RightLabel{\textsc{(R-Cong)}}
            \UnaryInfC{$P \red P'$}
        \end{prooftree}
    \end{minipage}
    \begin{minipage}{0.3\linewidth}
        \centering
        \begin{prooftree}
            \AxiomC{}
            \RightLabel{\textsc{(R-Int)}}
            \UnaryInfC{$i \Downarrow i$}
        \end{prooftree}
    \end{minipage}
    \begin{minipage}{0.38\linewidth}
        \centering
        \begin{prooftree}
            \AxiomC{$\seq{v} \Downarrow \seq{i}$}
            \RightLabel{\textsc{(R-Op)}}
            \UnaryInfC{$\op(\seq{v}) \Downarrow \llbracket \op\rrbracket (\seq{i})$.}
        \end{prooftree}
    \end{minipage}
    \normalsize
    \caption{The reduction rules of the $\pi$-calculus. Here \( \llbracket \op \rrbracket \colon \mathbb{Z}^n \to \mathbb{Z} \) represents the interpretation of the operation \( \op \) whose arity is \( n \). }
    \label{fig:pi_reduction}
\end{figure}






\subsection{Reduction Semantics of the Sequential Language}
Here, we define the reduction semantics for the sequential language.
We actually define two kinds of semantics:
one is a standard reduction relation
\((\Def,\Exp)\sred (\Def',\Exp')\), which evaluates \( \Exp_1 \nondet \Exp_2\) to either \( \Exp_1 \) or \( \Exp_2\); the other is a non-standard reduction relation
\((\Def,\Exp)\nsred(\Def',\Exp')\), which
 does not discard branches of non-deterministic choices.

\begin{definition}
  The \emph{reduction relation $\seqto$ on %
    sequential programs} is defined by the set of rules in Figure~\ref{fig:seq_reduction}.
    In the rule \rn{SR-App} we are considering \(\Def\) as a map that maps \( f \) to \( \Def(f) =  \{ \lambda \seq{x}. \Exp \mid \fdef{f}{\seq{x}}{\Exp} \in \Def \}\).
\end{definition}
\begin{figure}[tb]
    \centering
    \small
    \begin{minipage}{\linewidth}
        \centering
        \begin{prooftree}
            \AxiomC{$\len{\seq{x}} = \len{\seq{i}}$}
            \RightLabel{\textsc{(SR-LetND)}}
            \UnaryInfC{$(\Def, \ndlet{x}{\Exp} )  \sred (\Def, [\seq{i}/\seq{x}]\Exp)$}
        \end{prooftree}
    \end{minipage}
    \begin{minipage}{\linewidth}
        \centering
        \begin{prooftree}
            \AxiomC{\( (\lambda \seq{y}.\Exp) \in  \Def(f)  \)}
            \AxiomC{$\len{\seq{y}} = \len{\seq{v}}$}
            \AxiomC{$\seq{v} \Downarrow \seq{i}$}
            \RightLabel{\textsc{(SR-App)}}
            \TrinaryInfC{$(\Def, f(\seq{v})) \sred (\Def, [\seq{i}/\tilde{y}] \Exp)$}
        \end{prooftree}
    \end{minipage}
    \begin{minipage}{\linewidth}
        \centering
        \begin{prooftree}
            \AxiomC{$v \Downarrow i$ \qquad \( i \neq 0\)}
            \RightLabel{\textsc{(SR-If-T)}}
            \UnaryInfC{$(\Def, \ifexp{v}{\Exp_1}{\Exp_2})  \sred (\Def, \Exp_1)$}
        \end{prooftree}
    \end{minipage}
    \begin{minipage}{\linewidth}
        \centering
        \begin{prooftree}
            \AxiomC{$v \Downarrow 0$}
            \RightLabel{\textsc{(SR-If-F)}}
            \UnaryInfC{$(\Def, \ifexp{v}{\Exp_1}{\Exp_2}) \sred (\Def, \Exp_2)$}
        \end{prooftree}
    \end{minipage}
    \\[.3cm]
    \begin{minipage}{\linewidth}
        \centering
        \begin{prooftree}
            \AxiomC{}
            \RightLabel{\textsc{(SR-Cho-L)}}
            \UnaryInfC{$(\Def, \Exp_1 \nondet \Exp_2) \sred (\Def, \Exp_1)$}
        \end{prooftree}
    \end{minipage}
    \\[.3cm]
    \begin{minipage}{\linewidth}
        \centering
        \begin{prooftree}
            \AxiomC{}
            \RightLabel{\textsc{(SR-Cho-R)}}
            \UnaryInfC{$(\Def, \Exp_1 \nondet \Exp_2) \sred (\Def, \Exp_2)$}
        \end{prooftree}
    \end{minipage}
    \begin{minipage}{\linewidth}
        \centering
        \begin{prooftree}
            \AxiomC{$v \Downarrow i$ \qquad \( i \neq 0\)}
            \RightLabel{\textsc{(SR-Ass-T)}}
            \UnaryInfC{$(\Def, \textbf{Assume}(v);E) \sred (\Def, \Exp)$}
        \end{prooftree}
    \end{minipage}
    \begin{minipage}{\linewidth}
        \centering
        \begin{prooftree}
            \AxiomC{$v \Downarrow 0$}
            \RightLabel{\textsc{(SR-Ass-F)}}
            \UnaryInfC{$(\Def, \textbf{Assume}(v);\Exp) \sred (\Def, \skipexp)$}
        \end{prooftree}
    \end{minipage}
    \normalsize
    \caption{Reduction rules of the sequential language}
    \label{fig:seq_reduction}
\end{figure}

We now define a non-standard reduction relation that keeps
all the non-deterministic branches \shchanged{during} the reduction.
This non-standard reduction relation has a better match with the reduction of processes.
Since processes have structural rules, we also introduce structural rules on expressions.

\begin{definition}[structural congruence for sequential expressions]
    The \emph{structural congruence relation for expressions}, written \( \Exp_1 \expequiv \Exp_2 \), is defined as the least congruence relation that satisfies the following rules.
    \begin{gather*}
        \Exp_1 \nondet \Exp_2 \expequiv \Exp_2 \nondet \Exp_1
        \qquad (\Exp_1 \nondet \Exp_2) \nondet \Exp_3 \expequiv \Exp_1 \nondet (\Exp_2 \nondet \Exp_3)
        \qquad \Exp \nondet \skipexp \expequiv \Exp
        \qquad
    \end{gather*}


\end{definition}




\begin{definition}
    The \emph{non-standard reduction relation} $\nsred$ on the set of sequential programs is defined
    by the set of rules in Figure~\ref{fig:seq_reduction2} together with all the rules in
    Figure~\ref{fig:seq_reduction} (with $\sred$ replaced by $\nsred$), except for \rn{SR-Cho-L} and  \rn{SR-Cho-R}.
    To simplify the notation, we may write \( \Exp \nsred_\Def \Exp' \) for \( (\Def, \Exp) \nsred (\Def, \Exp') \) or even \( \Exp \nsred \Exp' \) if \( \Def \) is clear from the context.
\end{definition}

\begin{figure}[tb]
    \centering
    \small
    \begin{minipage}{\linewidth}
        \centering
        \begin{prooftree}
            \AxiomC{\( \Exp \expequiv \Exp_1 \quad (\Def, \Exp_1) \nsred (\Def, \Exp'_1) \quad \Exp'_1 \expequiv \Exp' \)}
            \RightLabel{\textsc{(SR-Cong)}}
            \UnaryInfC{$(\Def, \Exp) \nsred (\Def, \Exp')$}
        \end{prooftree}
    \end{minipage}
    \begin{minipage}{\linewidth}
        \centering
        \begin{prooftree}
            \AxiomC{$(\Def, \Exp_1) \nsred (\Def, \Exp_1')$}
            \RightLabel{\textsc{(SR-ChoBody-L)}}
            \UnaryInfC{$(\Def, \Exp_1 \nondet \Exp_2) \nsred (\Def, \Exp_1' \nondet \Exp_2)$}
        \end{prooftree}
    \end{minipage}
    \begin{minipage}{\linewidth}
        \centering
        \begin{prooftree}
            \AxiomC{$(\Def, \Exp_2) \nsred (\Def, \Exp_2')$}
            \RightLabel{\textsc{(SR-ChoBody-R)}}
            \UnaryInfC{$(\Def, \Exp_1 \nondet \Exp_2) \nsred (\Def, \Exp_1 \nondet \Exp_2')$}
        \end{prooftree}
    \end{minipage}

    \normalsize
    \caption{Additional rules for the  non-standard reduction relation}
    \label{fig:seq_reduction2}
\end{figure}

For the proof of the soundness of our transformation
(given in Appendix~\ref{sec:soundness}),
we also prepare a relation \( \Def \subdef \Def' \), which intuitively
means that \(\Def\) can simulate \(\Def'\) so that if \((\Def,\Exp)\) is terminating,
so is \((\Def',\Exp)\)
(cf.\ Lemma~\ref{lem:subdef}).

\begin{figure}[tb]
    \centering
    \small
    \begin{minipage}{.4\linewidth}
        \centering
        \begin{prooftree}
            \AxiomC{ }
            \RightLabel{\textsc{(D-Id)}}
            \UnaryInfC{\( \Def \subdef \Def\)}
        \end{prooftree}
    \end{minipage}
    \begin{minipage}{.4\linewidth}
        \centering
        \begin{prooftree}
            \AxiomC{\( \Def = \Def_1 \mrg \Def_2 \)}
            \RightLabel{\textsc{(D-Splt)}}
            \UnaryInfC{\( \Def \subdef \Def_1 \)}
        \end{prooftree}
    \end{minipage}
    \begin{minipage}{\linewidth}
        \centering
        \begin{prooftree}
            \AxiomC{\( \Def = (\ndlet{x}{\Def'}) \) \qquad \( \len{\seq{x}} = \len{\seq{v}} \)}
            \RightLabel{\textsc{(D-ND)}}
            \UnaryInfC{\( \Def \subdef [\seq{v}/\seq{x}]\Def' \)}
        \end{prooftree}
    \end{minipage}
    \begin{minipage}{.4\linewidth}
        \centering
        \begin{prooftree}
            \AxiomC{ \( \Def_1 \subdef \Def_1' \)}
            \RightLabel{\textsc{(D-Mrg)}}
            \UnaryInfC{\( \Def_1 \mrg \Def_2 \subdef \Def_1' \mrg \Def_2 \)}
        \end{prooftree}
    \end{minipage}
    \begin{minipage}{.4\linewidth}
        \centering
        \begin{prooftree}
            \AxiomC{ \( \Def_1 \subdef \Def_2 \) \qquad \( \Def_2 \subdef \Def_3 \)}
            \RightLabel{\textsc{(D-Trns)}}
            \UnaryInfC{\( \Def_1 \subdef \Def_3 \)}
        \end{prooftree}
    \end{minipage}
    \normalsize
    \caption{Preorder on function definitions}
    \label{fig:subdef}
\end{figure}

\begin{lemma}
    \label{lem:subdef}
    Suppose that \( \Def \subdef \Def' \) and \( (\Def', \Exp ) \nsred ( \Def', \Exp') \).
    Then \( (\Def, \Exp ) \nsred^+ ( \Def, \Exp') \).
\end{lemma}
\begin{proof}
    By induction on the derivation of \( \Def \subdef \Def' \).
    \qed
\end{proof}
\section{Proof of the Soundness}  \label{sec:soundness}
Here we prove the soundness of the translation (Theorem~\ref{thm:soundness}) saying that if the sequential program \( (\Def, \Exp) \) obtained by translating \( P \) is terminating, \( P \) is also terminating.
The proof is split into two steps.
First, we show that reductions from \( P \) can be simulated by non-standard reductions from \( (\Def, \Exp ) \) (Lemma~\ref{lem:simulate}).
This implies that if \( (\Def, \Exp) \) is terminating with respect to the non-standard reduction, then \( P \) is terminating.
Then we show that if  \( (\Def, \Exp) \) is terminating with respect to the standard reduction, then \( (\Def, \Exp) \) is terminating with respect to the non-standard reduction (Lemma~\ref{lem:konig}).




We start by preparing some auxiliary lemmas that are used to show the simulation relation.

\begin{lemma}[substitution]
    \label{lem:subst}
    If \( \env; \chenv \vdash \seq{v} : \seq{\ty} \),
       \( \env; \chenv \vdash \seq{w} : \seq{\chty} \)
       and \( \env, \seq{y}:\seq{\ty}; \chenv, \seq{z}:\seq{\chty} \vdash P \Rightarrow
       \prog{\Def}{\Exp} \),
       then \( \env; \chenv \vdash [\seq{v}/\seq{y}, \seq{w} / \seq{z}]P \Rightarrow
       \prog{[\seq{v}/\seq{y}]\Def}{[\seq{v}/\seq{y}] \Exp} \).
\end{lemma}
\begin{proof}
    By induction on the derivation of $\env, \seq{y}:\seq{\ty}; \chenv, \seq{z}:\seq{\chty} \vdash P \Rightarrow \prog{\Def}{\Exp}$.
    \qed
\end{proof}



\begin{lemma} \label{lem:cong}
    If $P \piequiv P'$ and $\env; \chenv \vdash P \Rightarrow \prog{\Def}{\Exp}$,
    then there exists $\Exp'$ such that
    $\Exp \expequiv \Exp'$
    and $\env; \chenv \vdash P' \Rightarrow \prog{\Def}{\Exp'}$.
\end{lemma}
\begin{proof}
    By induction on the construction of $P \piequiv P'$.
    \qed
\end{proof}

Now we prove the simulation relation.

\begin{lemma} \label{lem:simulate}
    If $P \red P'$ and $\env; \chenv \vdash P \Rightarrow \prog{\Def}{\Exp}$,
    then there exist $\Def'$, $\Exp'$ such that \( \Def \subdef \Def' \),
    \( (\Def', \Exp) \nsred^+ (\Def', \Exp') \)
    and $\env; \chenv \vdash P' \Rightarrow \prog{\Def'}{\Exp'}$.
\end{lemma}
\begin{proof}
    By induction on the construction of \( P \red P' \).
    We only give detailed proofs for  interesting cases; the other cases are sketched.

\begin{description}
    \item[Case \rn{R-Comm}:]
    In this case $P \to P'$ must be of the form
    \begin{align*}
    \inexp{x}{\seq{y}}{\seq{z}}P_1 \PAR \outexp{x}{\seq{v}}{\seq{w}}P_2 \red [\seq{i}/ \seq{y} ,\seq{w} / \seq{z}]P_1 \PAR P_2,
    \end{align*}
    where
    $\len{\seq{y}} = \len{\seq{v}}$,
    $\len{\seq{z}} = \len{\seq{w}}$
    and $\seq{v} \Downarrow \seq{i}$.
    Also $\env; \chenv \vdash P \Rightarrow \prog{\Def}{\Exp}$ must be the form of
    \begin{align*}
      \env; \chenv \vdash P \Rightarrow
      \prog{(\ndlet{y}{\Def_1}) \mrg \Def_2}{(\ndlet{y}{\Exp_1}) \nondet (f_\reg(\seq{v}) \oplus \Exp_2)},
    \end{align*}
    where
    \begin{align}
    &\env; \chenv \vdash x : \Chty{\reg}{\seq{\ty}}{\seq{\chty}}
    \qquad \env; \chenv \vdash \seq{v} : \seq{\ty}
    \qquad \env; \chenv \vdash \seq{w} : \seq{\chty} \nonumber \\
    &\env, \seq{y} : \seq{\ty}; \chenv, \seq{z} : \seq{\chty} \vdash P_1 \Rightarrow
    \prog{\Def_1}{\Exp_1} \label{eq:sim:com:trans-p1} \\
    & \env; \chenv \vdash P_2 \Rightarrow \prog{\Def_2}{\Exp_2}. \label{eq:sim:com:trans-p2}
    \end{align}
    By applying Lemma~\ref{lem:subst} to \eqref{eq:sim:com:trans-p1} with
    $\env; \chenv \vdash \seq{i} : \seq{\ty}$,
    $\env; \chenv \vdash \seq{w} : \seq{\chty}$,
    we obtain
    $\env; \chenv \vdash [\seq{i} / \seq{y}, \seq{w})/\seq{z}]P_1 \Rightarrow
    \prog{[\seq{i}/\seq{y}]\Def_1}{[\seq{i}/\seq{y}]\Exp_1}$.
    From this and \eqref{eq:sim:com:trans-p2}, we have
    \begin{align*}
      \env; \chenv \vdash [\seq{i}/ \seq{y}, \seq{w}/ \seq{z}]P_1 \PAR P_2 \Rightarrow
      \prog{[\seq{i}/\seq{y}]\Def_1 \mrg \Def_2}
           {[\seq{i}/\seq{y}]\Exp_1 \nondet \Exp_2}
    \end{align*}
    by applying the rule \rn{SX-Par}.
    Observe that we also have
    \begin{align*}
      \Def = (\ndlet{y}{\Def_1}) \mrg \Def_2 \subdef [\seq{i}/\seq{y}]\Def_1 \mrg \Def_2.
    \end{align*}
    Therefore, for \( (\Def', \Exp')\) we can take \( ([\seq{i}/\seq{y}]\Def_1 \mrg \Def_2, [\seq{i}/\seq{y}]\Exp_1 \nondet \Exp_2 )\) with the following matching reduction sequence:
    \begin{align*}
      \Exp
      &=  (\ndlet{y}{\Exp_1}) \nondet f_\reg(\seq{v}) \nondet \Exp_2 \\
      &\nsred_{\Def'} [\seq{i}/\seq{y}]\Exp_1 \nondet f_\reg(\tilde{v}) \oplus E_2 \tag{\rn{SR-LetND}} \\
        &\nsred_{\Def'}, [\seq{i}/\seq{y}]\Exp_1 \nondet \skipexp \nondet \Exp_2 \tag{by (\rn{SR-App}) and \( \lambda \seq{y}.\skipexp \in \Def'(f_\reg) \) } \\
        &\expequiv [\seq{i}/\seq{y}]\Exp_1 \nondet \Exp_2.
    \end{align*}

    \item[Case \rn{R-RComm}:]
    In this case  $P \to P'$ is of the form
    \begin{align*}
      \rinexp{x}{\seq{y}}{\seq{z}}P_1 \PAR \outexp{x}{\seq{v}}{\seq{w}}P_2 \red \rinexp{x}{\seq{y}}{\seq{z}}P_1 \PAR [(\seq{i},\seq{w})/(\seq{y},\seq{z})]P_1 \PAR P_2,
    \end{align*}
    where
    $\len{\seq{y}} = \len{\seq{v}}$,
    $\len{\seq{z}} = \len{\seq{w}}$
    and $\seq{v} \Downarrow \seq{i}$.
    Moreover, the judgment $\env; \chenv \vdash P \Rightarrow
    \prog{\Def}{\Exp}$ must be of the form
    \begin{align*}
      \env; \chenv \vdash P \Rightarrow
      \prog{\{ \fdef{f_\reg}{\seq{y}}{\Exp_1} \} \mrg (\ndlet{y}{\Def_1}) \mrg \Def_2}
      {\skipexp \nondet f_\reg (\seq{v}) \nondet \Exp_2},
    \end{align*}
    where
    \begin{align}
      &\env; \chenv \vdash x : \Chty{\reg}{\seq{\ty}}{\seq{\chty}}
      \qquad \env; \chenv \vdash \seq{v} : \seq{\ty}
      \qquad \env; \chenv \vdash \seq{w} : \seq{\chty} \nonumber \\
      &\env; \chenv \vdash \rinexp{x}{\seq{y}}{\seq{z}}P_1 \Rightarrow
      \prog{\{ \fdef{f_\reg}{\seq{y}}{\Exp_1} \} \mrg (\ndlet{y}{\Def_1})}{
        \skipexp} \label{eq:sim:rcom:trans-in} \\
      &\env, \seq{y} : \seq{\ty}; \chenv, \seq{z} : \seq{\chty} \vdash P_1 \Rightarrow \Def_1; \Exp_1 \label{eq:sim:rcom:trans-p1}\\
      &\env; \chenv \vdash P_2 \Rightarrow \prog{\Def_2}{\Exp_2}. \label{eq:sim:rcom:trans-p2}
    \end{align}
    Since
    $\env; \chenv \vdash \seq{i} : \tilde{\ty}$ and
    $\env; \chenv \vdash \tilde{w} : \tilde{\chty}$, we can apply the substitution lemma (Lemma~\ref{lem:subst}) to \eqref{eq:sim:rcom:trans-p1} and obtain
    \begin{align*}
      \env; \chenv \vdash [\seq{i} / \seq{y}, \seq{w}/\seq{z}]P_1 \Rightarrow
          \prog{[\seq{i}/\seq{y}]\Def_1}{[\seq{i}/\seq{y}]\Exp_1}.
    \end{align*}
    From this, \eqref{eq:sim:rcom:trans-in} and \eqref{eq:sim:rcom:trans-p2}, we have
    \begin{align*}
      \env; \chenv \vdash P' \Rightarrow
      \begin{aligned}
        &(\{ \fdef{f_\reg}{\seq{y}}{\Exp_1} \} \mrg (\ndlet{y}{\Def_1}) \mrg  [\seq{i}/\seq{y}]\Def_1 \mrg \Def_2, \\
        &\skipexp \nondet [\seq{i}/\seq{y}]\Exp_1 \nondet \Exp_2)
      \end{aligned}
    \end{align*}
    So we can take \( \{ \fdef{f_\reg}{\seq{y}}{\Exp_1} \} \mrg (\ndlet{y}{\Def_1}) \mrg  [\seq{i}/\seq{y}]\Def_1 \mrg \Def_2 \) as  \( \Def' \) and \( \skipexp \nondet [\seq{i}/\seq{y}]\Exp_1 \nondet \Exp_2 \) as \( \Exp' \).
    Now it remains to show that \( \Def \subdef \Def' \) and that there is a reduction sequence from \( (\Def', \Exp) \) to \( (\Def', \Exp') \).
    The relation \( \Def \subdef \Def' \) holds because
    \begin{align*}
      \Def
      &= (\{ \fdef{f_\reg}{\seq{y}}{\Exp_1} \} \mrg (\ndlet{y}{\Def_1}) \mrg \Def_2 \\
      &= \{ \fdef{f_\reg}{\seq{y}}{\Exp_1} \} \mrg (\ndlet{y}{\Def_1}) \mrg (\ndlet{y}{\Def_1}) \mrg \Def_2 \\
      &\subdef \{ \fdef{f_\reg}{\seq{y}}{\Exp_1} \} \mrg (\ndlet{y}{\Def_1}) \mrg [\seq{i} / \seq{y}]{\Def_1} \mrg \Def_2 \tag{\rn{D-ND}} \\
      &=\Def'
    \end{align*}

    Finally, by \rn{SR-App}, we obtain
    \begin{align*}
      \Exp &= \skipexp \nondet f_\reg (\seq{v}) \nondet \Exp_2  \nsred_{\Def'} \skipexp \nondet [\seq{i}/\seq{y}]\Exp_1 \nondet \Exp_2  = \Exp'
    \end{align*}
    as desired.
    \item[Case \rn{R-If-T}:]
      In this case \( P \red P' \) and \( \env; \chenv \vdash P \Rightarrow
      \prog{\Def}{\Exp} \) must be of the form
      \begin{align*}
      &\ifexp{v}{P_1}{P_2} \red P_1 \\
        &\env; \chenv \vdash \ifexp{v} {P_1}{P_2} \Rightarrow
        \prog{\Def_1 \mrg \Def_2}{\ifexp{v}{\Exp_1}{\Exp_2}}
      \end{align*}
      where
      \begin{align*}
        v \Downarrow i \neq 0  \qquad \env; \chenv \vdash v : \ty \\
        \env; \chenv \vdash P_1 \Rightarrow \prog{\Def_1}{\Exp_1} \\
        \env; \chenv \vdash P_2 \Rightarrow \prog{\Def_2}{\Exp_2}.
      \end{align*}
    We can take \( (\Def_1, \Exp_1) \) for \( ( \Def', \Exp' )\) because \( \Def_1 \mrg \Def_2 \subdef \Def_1 \), and \( \Exp \nsred_{\Def_1} \Exp_1 \), which is trivial from \rn{SR-If-T}.

    \item[Case \rn{R-If-F}:]
    Similar to the previous case.

    \item[Case \rn{R-Cong}:]
    In this case \( P \red P' \) must be of the form
    \begin{align*}
      P \piequiv P_1 \red P_1' \piequiv P'.
    \end{align*}
    By Lemma~\ref{lem:cong}, we have
    \begin{align*}
      \env, \chenv \vdash P_1 \Rightarrow \prog{\Def}{\Exp_1} \text{ and } \Exp \expequiv \Exp_1
    \end{align*}
    for some \( \Exp_1 \).
    Thus, by the induction hypothesis, we have
    \begin{align}
      &\env, \chenv \vdash P_1' \Rightarrow \prog{\Def'}{\Exp_1'} \label{eq:sim:cong-P1prime}\\
      & (\Def',  \Exp_1) \nsred^+ (\Def', \Exp_1') \label{eq:sim:cong:red-seq}
    \end{align}
    where \( \Def \subdef \Def' \).
    By applying Lemma~\ref{lem:cong} to \eqref{eq:sim:cong-P1prime}, we obtain
    \begin{align*}
      \env, \chenv \vdash P' \Rightarrow \prog{\Def'}{\Exp'} \text{ and } \Exp_1' \expequiv \Exp'
    \end{align*}
    for some \( \Exp' \).
    It remains to show that \( (\Def', \Exp) \nsred^+  (\Def', \Exp') \), but this is easily shown by repeatedly applying the rule \rn{SR-Cong} along the reduction sequence \eqref{eq:sim:cong:red-seq}.

    \item[Case \rn{R-Par}, \rn{R-Nu} and \rn{R-LetND}:]
    Similar to the previous case, i.e.~follows from the definition of the translation and the induction hypothesis together with Lemma~\ref{lem:subdef}.
\end{description}







\leavevmode\qed
\end{proof}


\begin{lemma}  \label{lem:infinitechain}
    Suppose that \( \emptyset; \emptyset \vdash P \Rightarrow \prog{\Def}{\Exp} \).
    If \( (\mathcal{D}, E) \) is terminating with respect to $\nsred$, then \( P \) is terminating.
\end{lemma}
\begin{proof}
    We show the contraposition.
    Assume that \( P \) is not terminating, i.e.~assume that there exists an infinite reduction sequence $P = P_0 \red P_1 \red \cdots$.
    Let \( \Def_0 = \Def \) and \( \Exp_0 = \Exp \).
    By applying Lemma~\ref{lem:simulate}, for each natural number \( k \ge 1 \), we obtain \( \Def_k \), \( \Exp_k \) such that \( \emptyset; \emptyset \vdash P_k \Rightarrow
    \prog{\Def_k}{\Exp_k} \), \( (\Def_{k}, \Exp_{k-1}) \nsred^+ (\Def_{k}, \Exp_{k}) \) and \( \Def \subdef \Def_k \).
    Hence, by Lemma~\ref{lem:subdef} there exists an infinite reduction sequence
    $(\Def, \Exp) = (\Def, \Exp_0)\allowbreak \nsred^+ (\Def, \Exp_1) \nsred^+ \cdots$.
    \qed
\end{proof}

We now show the relation between standard and non-standard reductions.
\begin{lemma}  \label{lem:konig}
    Assume that $\emptyset; \emptyset \vdash P \Rightarrow \prog{\Def}{\Exp}$.
    If \( (\Def, \Exp) \) is terminating with respect to the standard reduction \( \sred \), then \( (\Def, \Exp)\) is also terminating with respect to the non-standard reduction relation \( \nsred \).
\end{lemma}

To prove the lemma above, we introduce a slight variation of the
non-standard reduction relation:
\((\Def,\Exp) \nsredv{\gamma} (\Def',\Exp')\) where
\(\gamma\in \set{1,2}^*\). (Actually, \(\Def\) does not change during the reduction.)
It is defined by the rules in Figure~\ref{fig:nsredv}.

\begin{figure}[tbp]

  \infrule[NSR-LetND]
          {\len{\seq{x}} = \len{\seq{i}}}
          {(\Def, \ndlet{x}{\Exp} )  \nsredv{\epsilon} (\Def, [\seq{i}/\seq{x}]\Exp)}
\vspace*{1ex}
          \infrule[NSR-App]
{ (\lambda \seq{y}.\Exp) \in  \Def(f)\andalso
 \len{\seq{y}} = \len{\seq{v}}\andalso
 \seq{v} \Downarrow \seq{i}}
{(\Def, f(\seq{v})) \nsredv{\epsilon} (\Def, [\seq{i}/\tilde{y}] \Exp)}

\vspace*{1ex}
\infrule[NSR-If-T]{v \Downarrow i \andalso i \neq 0}
 {(\Def, \ifexp{v}{\Exp_1}{\Exp_2})  \nsredv{\epsilon} (\Def, \Exp_1)}
\vspace*{1ex}
\infrule[NSR-If-F]{v \Downarrow 0}
 {(\Def, \ifexp{v}{\Exp_1}{\Exp_2})  \nsredv{\epsilon} (\Def, \Exp_2)}
\vspace*{1ex}

 \infrule[NSR-ChoBody-L]
 {(\Def, \Exp_1) \nsredv{\gamma} (\Def, \Exp_1')}
 {(\Def, \Exp_1 \nondet \Exp_2) \nsredv{1\cdot \gamma} (\Def, \Exp_1' \nondet \Exp_2)}
\vspace*{1ex}
 \infrule[NSR-ChoBody-R]
 {(\Def, \Exp_2) \nsredv{\gamma} (\Def, \Exp_2')}
 {(\Def, \Exp_1 \nondet \Exp_2) \nsredv{2\cdot \gamma} (\Def, \Exp_1 \nondet \Exp_2')}

\vspace*{1ex}
 \infrule[NSR-Ass-T]
  {v \Downarrow i\andalso i \neq 0}
  {(\Def, \textbf{Assume}(v);E) \nsredv{\epsilon} (\Def, \Exp)}
\vspace*{1ex}
 \infrule[NSR-Ass-F]
  {v \Downarrow 0}
  {(\Def, \textbf{Assume}(v);E) \nsredv{\epsilon} (\Def, \skipexp)}

\caption{A variation of the non-standard reduction relation}
\label{fig:nsredv}
\end{figure}

The only differences of
\((\Def,\Exp) \nsredv{\gamma} (\Def',\Exp')\) from
\((\Def,\Exp) \nsred (\Def',\Exp')\) are that
the reduction is annotated with the position \(\gamma\) that indicates where the reduction occurs,
and that the rule \rn{SR-Cong} for shuffling expressions is forbidden.
Since the rule \rn{SR-Cong} does not affect the reducibility, we can easily
observe the following property. (We omit the proof since it is trivial.)
\begin{lemma}
  \label{lem:nsred-vs-nsredv}
  If \((\Def,\Exp)\) has an infinite reduction sequence with respect to \(\nsred\),
  \((\Def,\Exp)\) has an infinite reduction sequence also with respect to \(\nsredv{\gamma}\).
\end{lemma}

It remains to show that
if \((\Def,\Exp)\) has an infinite reduction sequence
\[(\Def,\Exp)\nsredv{\gamma_1} (\Def,\Exp_1)\nsredv{\gamma_2}
(\Def,\Exp_2)\nsredv{\gamma_3}(\Def,\Exp_3)\nsredv{\gamma_4}\cdots,\]
then
\((\Def,\Exp)\) has an infinite reduction sequence also with respect to \(\sred\).

We write \(\gamma \preceq \gamma'\) if \(\gamma\) is a prefix of \(\gamma'\).
We have the following property.
\begin{lemma}
  \label{lem:inf-nsredv}
  If
\[(\Def,\Exp)\nsredv{\gamma_1} (\Def,\Exp_1)\nsredv{\gamma_2}
(\Def,\Exp_2)\nsredv{\gamma_3}(\Def,\Exp_3)\nsredv{\gamma_4}\cdots,\]
then there exists an infinite sequence
\(i_1 < i_2 < i_3< \cdots\)
such that \(\gamma_{i_j} \preceq \gamma_{i_k}\) for any \(j<k\).
\end{lemma}
\begin{proof}
  The required property obviously holds if   the set \(\set{\gamma_i\mid i\ge 1}\) is finite.
  So, assume that \(\set{\gamma_i\mid i\ge 1}\) is infinite.
  Let \(T\) be the least binary tree that contains, for every \(\gamma_i\),
  the node whose path from the root is \(\gamma_i\).
  By the assumption that \(\set{\gamma_i\mid i\ge 1}\) is infinite,
  \(T\) is an infinite tree. Thus, by K\"onig's lemma,
  \(T\) must have an infinite path, which implies that
  there exists an infinite sequence
  \[\gamma_{i_1} \preceq \gamma_{i_2} \preceq \gamma_{i_3} \preceq \cdots, \]
  as required. \qed
\end{proof}


For an expression \(\Exp\) and a position \(\gamma\in\set{1,2}^*\), we write
\(\Proj{\Exp}{\gamma}\) for the subexpression at \(\gamma\). It is inductively defined by:
\[
\begin{array}{l}
\Proj{\Exp}{\epsilon} = \Exp\\
\Proj{\Exp}{i\cdot \gamma} =
\left\{\begin{array}{ll}
  \Proj{\Exp_i}{\gamma} & \mbox{if $\Exp$ is of the form \(\Exp_1\nondet \Exp_2\)}\\
  \mbox{undefined}\hspace*{2em} & \mbox{otherwise}
\end{array}\right.
\end{array}
\]
The following lemma states the correspondence between \(\nsredv{\gamma}\) and \(\sred\).

\begin{lemma}
\label{lem:nsredv-vs-sred}
  \begin{enumerate}
\item  If \((\Def,\Exp)\nsredv{\gamma}(\Def,\Exp')\),
  then \((\Def, \Proj{\Exp}{\gamma})\sred (\Def,\Proj{\Exp'}{\gamma})\).
\item Suppose \(\Proj{\Exp}{\gamma'}\) is defined and \(\gamma'\not\preceq \gamma\).
  If \((\Def,\Exp)\nsredv{\gamma}(\Def,\Exp')\), then
  \(\Proj{\Exp}{\gamma'}=\Proj{\Exp'}{\gamma'}\).
\item
  If \((\Def,\Exp)\nsredv{\gamma}(\Def,\Exp')\), and \(\gamma'\preceq \gamma\),
  then \((\Def, \Proj{\Exp}{\gamma'} ) \sred^* (\Def, \Proj{\Exp}{\gamma})\).
\end{enumerate}  
\end{lemma}
\begin{proof}
  The properties follow by a straightforward induction on the derivation of
  \((\Def,\Exp)\nsredv{\gamma}(\Def,\Exp')\). \qed
\end{proof}

We are now ready to prove Lemma~\ref{lem:konig}.

\begin{proof}[of Lemma~\ref{lem:konig}]
  We show the contraposition.
  Suppose \((\Def,\Exp)\) has an infinite reduction sequence with respect to \(\nsred\).
  By Lemma~\ref{lem:nsred-vs-nsredv}, there exists an infinite reduction sequence
  \[(\Def,\Exp)\nsredv{\gamma_1} (\Def,\Exp_1)\nsredv{\gamma_2}
(\Def,\Exp_2)\nsredv{\gamma_3}(\Def,\Exp_3)\nsredv{\gamma_4}\cdots.\]
  By Lemma~\ref{lem:inf-nsredv},
 there exists an infinite sequence:
  \[\gamma_{i_1}\preceq \gamma_{i_2}\preceq \gamma_{i_3}\preceq \cdots.\]
  such that \(i_1<i_2<i_3<\cdots\).
  Let us choose a maximal one among such sequences, i.e.,
  a sequence
  \[\gamma_{i_1}\preceq \gamma_{i_2}\preceq \gamma_{i_3}\preceq \cdots.\]
  such that, for any \(i_j\), 
  \(\gamma_{k}\preceq \gamma_{i_{j}}\) implies \(k=i_{j'}\) for some \(j'\le j\).
  Consider the fragment of the infinite reduction sequence:
  \[
  (\Def,\Exp_{i_{\ell-1}})\nsredv{\gamma_{i_{\ell-1}+1}} (\Def,\Exp_{i_{\ell-1}+1})
  \nsredv{\gamma_{i_{\ell-1}+2}}\cdots
  \nsredv{\gamma_{i_{\ell}-1}}
(\Def,\Exp_{i_{\ell}-1})\nsredv{\gamma_{i_\ell}}(\Def,\Exp_{i_\ell})
  \]
  for each \(\ell>0\). 
  (Here, we define \(\gamma_0 = \epsilon\), \(i_0=0\) and \(E_0 = E\).)
  By Lemma~\ref{lem:nsredv-vs-sred} (1)
  and \((\Def,\Exp_{i_{\ell}-1})\nsredv{\gamma_{i_\ell}}(\Def,\Exp_{i_\ell})\),
  we have
  \[(\Def,\Proj{\Exp_{i_{\ell}-1}}{\gamma_{i_\ell}})\sred
  (\Def,\Proj{\Exp_{i_{\ell}}}{\gamma_{i_\ell}}).\]
  By Lemma~\ref{lem:nsredv-vs-sred} (2) (note that
  since none of \(\gamma_{i_{\ell-1}+1},\ldots,\gamma_{i_{\ell}-1}\) is a
  prefix of \(\gamma_{i_{\ell}}\) by the assumption on maximality,
  \(\Proj{\Exp_{i_{\ell-1}}}{\gamma_{i_\ell}}\) is defined),
  we have
  \[\Proj{\Exp_{i_{\ell-1}}}{\gamma_{i_\ell}} =
  \Proj{\Exp_{i_{\ell-1}+1}}{\gamma_{i_\ell}} = \cdots =
  \Proj{\Exp_{i_{\ell}-1}}{\gamma_{i_\ell}}.\]
  Thus, together with Lemma~\ref{lem:nsredv-vs-sred} (3), we obtain:
  \[(\Def,\Proj{\Exp_{i_{\ell-1}}}{\gamma_{i_{\ell-1}}})\sred^*
  (\Def,\Proj{\Exp_{i_{\ell-1}}}{\gamma_{i_{\ell}}})\sred
  (\Def,\Proj{\Exp_{i_{\ell}}}{\gamma_{i_\ell}}).\]
  Therefore,
  we have an infinite reduction sequence
  \[(\Def,\Exp)=(\Def,\Proj{\Exp_{i_0}}{\gamma_{i_0}})\sred^+ (\Def,\Proj{\Exp_{i_1}}{\gamma_{i_1}})
  \sred^+ (\Def,\Proj{\Exp_{i_2}}{\gamma_{i_2}})
  \sred^+ (\Def,\Proj{\Exp_{i_3}}{\gamma_{i_3}})
  \sred^+ \cdots,\]
  as required. \qed
\end{proof}




    
    



Finally, the soundness (Theorem~\ref{thm:soundness}) follows from Lemmas~\ref{lem:infinitechain} and \ref{lem:konig}.

\section{Complete Definition of the Refinement Type System}
\label{sec:refinement-apx}
This section shows the complete definition of the refinement type system we discussed in Section~\ref{sec:refinement}.

First, we define the well-formedness conditions for types and type environments.
We write \(\FV(\pred)\) (\(\FV(\predenv)\), resp.)
for the set of variables occurring in \(\pred\) (\(\predenv\), resp.),
and \(\dom(\env)\) for the domain of \(\env\), i.e.,
\(\set{x \mid x\COL\ty\in\env}\).
The relations \(\tyok{\env}{\chty}\)
and \(\tyenvok{\env}{\predenv}{\chenv}\)
are defined by:

\infrule{\FV(\pred)\subseteq \dom(\env)\cup \set{\seq{x}}\\
  \tyok{\env, \seq{x}\COL\seq{\ty}}{\chty_i}\mbox{ for each $i\in\set{1,\ldots,k}$}}
{\tyok{\env}{\rch{\reg}{\seq{x}}{\pred}{\chty_1,\ldots,\chty_k}}}

\infrule{\tyok{\env}{\chty} \mbox{ for every $x\COL\chty\in\chenv$}\\
   \FV(\predenv) \subseteq \dom(\env)}
{\tyenvok{\env}{\predenv}{\chenv}}
  For example, \(x\COL \ty \p \rch{\reg}{y}{y<x}{\epsilon}:\OK\)
  holds but
  \(\emptyset \p \rch{\reg}{y}{y<x}{\epsilon}:\OK\) does not.

For every type judgment of the form \(\env;\predenv;\chenv \p P\),
we implicitly require that \(\tyenvok{\env}{\predenv}{\chenv}\) holds.
Similarly,  for $\env;\predenv;\chenv \p v\COL\chty$,
we require
that \(\tyenvok{\env}{\predenv}{\chenv}\) and \(\tyok{\env}{\chty}\) hold.

The complete list of typing rules is given in Figure~\ref{fig:refinement_type_system-complete}.

\begin{figure}[tb]
    \centering
    \small
    \begin{minipage}{0.4\linewidth}
        \centering
        \begin{prooftree}
            \AxiomC{}
            \RightLabel{\textsc{(RT-Nil)}}
            \UnaryInfC{$\env;\predenv;\chenv\p \textbf{0}$}
        \end{prooftree}
    \end{minipage}
    \begin{minipage}{0.55\linewidth}
        \centering
        \begin{prooftree}
            \AxiomC{$\env;\predenv;\chenv\p P_1$}
            \AxiomC{$\env;\predenv;\chenv\p P_2$}
            \RightLabel{\textsc{(RT-Par)}}
            \BinaryInfC{$\env;\predenv;\chenv\p P_1 \PAR P_2$}
        \end{prooftree}
    \end{minipage}
    \\
    \begin{minipage}{\linewidth}
        \centering
        \begin{prooftree}
            \AxiomC{$\env;\predenv;\chenv\p x\COL\rch{\reg}{\seq{y}}{\pred}{\seq{\chty}}$}
            \AxiomC{$\env,\seq{y}\COL\seq{\ty}; \predenv,\pred; \chenv,\seq{z}\COL\seq{\chty} \p P$}
            \RightLabel{\textsc{(RT-In)}}
            \BinaryInfC{$\env;\predenv;\chenv\p \inexp{x}{\seq{y}}{\seq{z}}P$}
        \end{prooftree}
    \end{minipage}
    \\
    \begin{minipage}{0.93\linewidth}
        \centering
        \infrule[RT-Out]
        {\env;\predenv;\chenv\p x\COL\rch{\reg}{\seq{y}}{\pred}{\seq{\chty}}\andalso
        \env;\predenv;\chenv\p \seq{v}\COL\seq{\ty}\andalso
        \predenv \vDash [\seq{v}/\seq{y}]\pred\\
        \env;\predenv;\chenv\p \seq{w}\COL[\seq{v}/\seq{y}]\seq{\chty}\andalso
        \env;\predenv;\chenv\p P}
        {\env;\predenv;\chenv \vdash \outexp{x}{\seq{v}}{\seq{w}}P}
    \end{minipage}
    \\
    \begin{minipage}{\linewidth}
        \centering
        \begin{prooftree}
            \AxiomC{$\env;\predenv;\chenv,x\COL\chty \p P$}
            \RightLabel{\textsc{(RT-Nu)}}
            \UnaryInfC{$\env;\predenv;\chenv \p (\nu x\COL\chty)P$}
        \end{prooftree}
    \end{minipage}
    \\
    \begin{minipage}{\linewidth}
        \centering
        \begin{prooftree}
            \AxiomC{$\env;\predenv;\chenv\p x\COL\rch{\reg}{\seq{y}}{\pred}{\seq{\chty}}$}
            \AxiomC{$\env,\seq{y}\COL\seq{\ty}; \predenv,\pred; \chenv,\seq{z}\COL\seq{\chty} \p P$}
            \RightLabel{\textsc{(RT-RIn)}}
            \BinaryInfC{$\env;\predenv;\chenv\p \rinexp{x}{\seq{y}}{\seq{z}}P$}
        \end{prooftree}
    \end{minipage}
    \\
    \begin{minipage}{\linewidth}
        \centering
        \begin{prooftree}
            \AxiomC{$\env; \predenv; \chenv \p v\COL\ty$}
            \AxiomC{$\env; \predenv, v \neq 0; \chenv \p P_1$}
            \AxiomC{$\env; \predenv, v =    0; \chenv \p P_2$}
            \RightLabel{\textsc{(RT-If)}}
            \TrinaryInfC{$\env;\predenv;\chenv\p \ifexp{v}{P_1}{P_2}$}
        \end{prooftree}
    \end{minipage}
    \\
    \begin{minipage}{\linewidth}
        \begin{prooftree}
            \AxiomC{$\env,\seq{x}\COL\seq{\ty}; \predenv; \chenv \vdash P$}
            \RightLabel{\textsc{(RT-LetND)}}
            \UnaryInfC{$\env; \predenv; \chenv \vdash \ndlet{x}{P}$}
        \end{prooftree}
    \end{minipage}
    \\
    \begin{minipage}{0.45\linewidth}
        \centering
        \begin{prooftree}
            \AxiomC{$x \COL \ty \in \env$}
            \RightLabel{\textsc{(RT-Var-Int)}}
            \UnaryInfC{$\env;\predenv;\chenv \p x\COL\ty$}
        \end{prooftree}
    \end{minipage}
    \begin{minipage}{0.45\linewidth}
        \centering
        \begin{prooftree}
            \AxiomC{$x \COL \chty \in \chenv$}
            \RightLabel{\textsc{(RT-Var-Ch)}}
            \UnaryInfC{$\env;\predenv;\chenv \p x\COL\chty$}
        \end{prooftree}
    \end{minipage}
    \\
    \begin{minipage}{0.45\linewidth}
        \centering
        \begin{prooftree}
            \AxiomC{}
            \RightLabel{\textsc{(RT-Int)}}
            \UnaryInfC{$\env;\predenv;\chenv \p i\COL\ty$}
        \end{prooftree}
    \end{minipage}
    \begin{minipage}{0.45\linewidth}
        \centering
        \begin{prooftree}
            \AxiomC{$\env;\predenv;\chenv \p \seq{v}\COL\seq{\ty}$}
            \RightLabel{\textsc{(RT-Op)}}
            \UnaryInfC{$\env;\predenv;\chenv \p \op(\seq{v})\COL\ty$}
        \end{prooftree}
    \end{minipage}
    \normalsize
    \caption{Typing rules of the refinement type system for the $\pi$-calculus}
    \label{fig:refinement_type_system-complete}
\end{figure}

\section{Refinement Type System with Subtyping}
\label{sec:subtyping}

As mentioned in Section~\ref{sec:implementation},
the implementation is based on the following extension of the refinement type system in Section~\ref{sec:rtype}.

The set of \emph{refinement i/o channel types}, ranged over by $\chty$, is given by:
\[ 
    \chty ::= \ioch{\reg}{\seq{x}}{\inty{\pred}}{\inty{\seq{\chty}}}{\outy{\pred}}{\outy{\seq{\chty}}}
\]
Here, \(\ioch{\reg}{\seq{x}}{\inty{\pred}}{\inty{\seq{\chty}}}{\outy{\pred}}{\outy{\seq{\chty}}}\)
is the type of channels used for \emph{receiving} tuples \((\seq{x};\seq{y})\)
such that \(\seq{x}\) satisfies \(\pred_I\) and \(\seq{y}\) have types
\(\inty{\seq{\chty}}\),
and for \emph{sending} tuples \((\seq{x};\seq{y})\)
such that \(\seq{x}\) satisfies \(\pred_O\) and \(\seq{y}\) have types
\(\outy{\seq{\chty}}\).
The distinction between the types of input (i.e. received) values  and those of
output (i.e. sent) values has been inspired by
the type system of Yoshida and Hennessy~\cite{DBLP:conf/concur/YoshidaH99}.
It leads to a more precise type system than Pierce and Sangiorgi's subtyping,
and is convenient for automatic refinement type inference~\cite{Pierce96MSCS}
(because we need not infer input/output modes and perform case analysis on the modes).

The subtyping relation on channel types is defined by:
\infrule[RT-Sub-Ch]
{\predenv,\inty{\pred} \vDash \inty{\pred}' \andalso
 \env,\seq{x}\COL\seq{\ty};\predenv,\inty{\pred}\p  \inty{\seq{\chty}} \subtype \inty{\seq{\chty}}' \\
 \predenv,\outy{\pred}' \vDash \outy{\pred} \andalso
 \env,\seq{x}\COL\seq{\ty};\predenv,\outy{\pred}'\p  \outy{\seq{\chty}}' \subtype \outy{\seq{\chty}}
}
{\env;\predenv\p 
    \ioch{\reg}{\seq{x}}{\inty{\pred}}{\inty{\seq{\chty}}}{\outy{\pred}}{\outy{\seq{\chty}}} 
    \subtype 
    \ioch{\reg}{\seq{x}}{\inty{\pred}'}{\inty{\seq{\chty}}'}{\outy{\pred}'}{\outy{\seq{\chty}}'}
}
Note that the 
 channel type $\ioch{\reg}{\seq{x}}{\inty{\pred}}{\inty{\seq{\chty}}}{\outy{\pred}}{\outy{\seq{\chty}}}$
is covariant on $\inty{\pred}$ and $\inty{\seq{\chty}}$, 
and contravariant on $\outy{\pred}$ and $\outy{\seq{\chty}}$.

We make the following two modifications to the typing rules.
\begin{enumerate}
\item
  We add the following subsumption rule.
        \infrule[RT-Sub]
                {\env;\predenv;\chenv\p v\COL \chty\andalso \env;\predenv\p \chty \subtype \chty'}
                {\env;\predenv;\chenv\p v\COL \chty'}

  \item
    We refine the well-formedness condition on types and type environments by:
\infrule{\FV(\pred)\subseteq \dom(\env)\cup \set{\seq{x}}\\
  \tyok{\env, \seq{x}\COL\seq{\ty}}{\inty{\seq{\chty}}}\\
  \tyok{\env, \seq{x}\COL\seq{\ty}}{\outy{\seq{\chty}}}
}
{\tyok{\env}{\ioch{\reg}{\seq{x}}{\inty{\pred}}{\inty{\seq{\chty}}}{\outy{\pred}}{\outy{\seq{\chty}}}}}

\infrule{
   \FV(\predenv) \subseteq \dom(\env)
}
{\tyenvok{\env}{\predenv}{\emptyset}}


\infrule{\tyok{\env}{\ioch{\reg}{\seq{x}}{\inty{\pred}}{\inty{\seq{\chty}}}{\outy{\pred}}{\outy{\seq{\chty}}}}\\
  \predenv, \outy{\pred} \models \inty{\pred}
  \andalso
  \env,\seq{x}\COL\seq{\ty};\predenv,\outy{\pred}\p \outy{\seq{\chty}} \subtype
  \inty{\seq{\chty}}}
{\tyenvok{\env}{\predenv}{\chenv,y\COL\ioch{\reg}{\seq{x}}{\inty{\pred}}{\inty{\seq{\chty}}}{\outy{\pred}}{\outy{\seq{\chty}}}}}

The requirement for the subtyping relation in the last rule
 ensures the consistency between the types of
values expected by a receiver process and those actually output by a sender process;
for example, the channel type
\(\ioch{\reg}{x}{x>0}{\epsilon}{x<0}{\epsilon}\) is judged to be ill-formed,
because the type indicates that a receiver process expects a positive value \(x\) but 
a sender will output a negative value on the channel.
\end{enumerate}

The following example demonstrates the usefulness of subtyping for refinement channel
types.
\begin{example}
  \label{ex:subtyping}
  Let us consider the following process:
  \[
  \begin{array}{l}
    \rinexp{\pre}{x}{r}\soutatom{r}{x-1}\\
\PAR
\srinexp{f}{x}\ifexp{x<0}{\zeroexp}
        {\nuexp{s}(\outatom{\pre}{x}{s}\PAR \sinexp{s}{y}\soutatom{f}{y})}\\
\PAR \soutatom{f}{100}\\
\PAR \rinexp{c}{x}{r}\letexp{y}{\ndint}\soutatom{r}{y}\\
\PAR \soutatom{d}{\pre}\PAR \soutatom{d}{c}
  \end{array}
  \]
  The process consisting of the first three lines
  is a variation of the process in Example~\ref{ex:weakeness-of-basic-transformation},
  which is obviously terminating.
  Without the fourth and fifth lines, we would be able to assign
  the type \(\rch{\reg_1}{x}{\TRUE}{\rch{\reg_2}{y}{y<x}{\epsilon}}\)
  in the refinement type system in Section~\ref{sec:refinement},
  and reduce the process to a terminating program.

  The processes on the \skchanged{fifth} line, however, force us to assign the same type
  to \(\pre\) and \(c\) in the refinement type system in Section~\ref{sec:refinement},
  and thus we can assign only \(\rch{\reg_1}{x}{\TRUE}{\rch{\reg_2}{y}{\TRUE}{\epsilon}}\)
  to \(\pre\), failing to transform the process to a non-terminating program.

  With subtyping, we can assign the following types to \(\pre\), \(c\), and \(d\):
  \[
  \begin{array}{l}
    \pre\COL
    \ioch{\reg_1}{x}{\TRUE}{\ioch{\reg_2}{y}{\TRUE}{\epsilon}{y<x}{\epsilon}}{\TRUE}{\ioch{\reg_2}{y}{\TRUE}{\epsilon}{y<x}{\epsilon}}\\
    c\COL
    \ioch{\reg_1}{x}{\TRUE}{\ioch{\reg_2}{y}{\TRUE}{\epsilon}{\TRUE}{\epsilon}}{\TRUE}{\ioch{\reg_2}{y}{\TRUE}{\epsilon}{\TRUE}{\epsilon}}\\
    d\COL
    \ioch{\reg_0}{\epsilon}{\TRUE}{\chty}{\TRUE}{\chty}\\
    \mbox{where}\\
    \chty =
    \ioch{\reg_1}{x}{\TRUE}{\ioch{\reg_2}{y}{\TRUE}{\epsilon}{y<x}{\epsilon}}{\TRUE}{\ioch{\reg_2}{y}{\TRUE}{\epsilon}{\TRUE}{\epsilon}}\\
  \end{array}
  \]
  Note that the types of \(\pre\) and \(c\) are
  subtypes of \(\chty\).
  Here, the type of \(\pre\) indicates that
  the value \(y\) sent along the second argument \(r\) should be smaller than
  the first argument \(x\).
  Thus, the process on the second line is translated to the following
  function definition:
  \[
 \begin{array}{l}
\Fname{\reg_f}(x)=\ifexp{x<0}{\skipexp}\\\qquad\qquad{
  (\Fname{\pre}(x)\nondet (\letexp{y}{\ndint}\assexp{y<x}\Fname{\reg_f}(y)))}
\end{array}
\]

\end{example}

}

\end{document}
