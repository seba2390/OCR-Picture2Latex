\section{Operational Semantics} \label{sec:operational_semantics}
\subsection{Reduction Semantics of the \( \pi \)-Calculus}
We define a reduction relation~\cite{milner1993polyadic} as the operational semantics of the $\pi$-calculus.

As usual, we first define the structural congruence relation \( \piequiv \) on the set of processes.
\begin{definition}[structural congruence for processes]
  The \emph{structural congruence relation} $\piequiv$ on %
  $\pi$-calculus processes is defined as the least congruence relation that satisfies the following %
  rules.
    \begin{gather*}
        P_1 \mid P_2 \piequiv P_2 \mid P_1
        \qquad (P_1 \mid P_2) \mid P_3 \piequiv P_1 \mid (P_2 \mid P_3) 
        \\     P \mid \textbf{0} \piequiv P 
        \qquad (\nu x)\,\textbf{0} \piequiv \textbf{0}
        \qquad (\nu x)(\nu y)P \piequiv (\nu y)(\nu x)P
        \\     (\nu x)(P_1 \mid P_2) \piequiv P_1 \mid (\nu x)P_2 \quad \text{if $x$ does not freely occur in $P_1$}
    \end{gather*}
\end{definition}

Next, we define the reduction relation on processes. %
\begin{definition}
  The \emph{reduction relation $\to$ on %
    processes} is defined by the set of rules in Figure~\ref{fig:pi_reduction}.
    We write $\to^*$ and \( \to^+ \)for the reflexive transitive closure and the transitive closure of the reduction relation $\to$, respectively.
\end{definition}
\begin{figure}[tbp]
    \centering
    \small
    \begin{minipage}{\linewidth}
        \centering
        \begin{prooftree}
            \AxiomC{$\len{\seq{y}} = \len{\seq{v}}$}
            \AxiomC{$\len{\seq{z}} = \len{\seq{w}}$}
            \AxiomC{$\seq{v} \Downarrow \seq{i}$}
            \RightLabel{\textsc{(R-Comm)}}
            \TrinaryInfC{$\inexp{x}{\seq{y}}{\seq{z}}P_1 \PAR  \outexp{x}{\seq{v}}{\seq{w}}P_2 \red [\seq{i}/ \seq{y}, \seq{w} / \seq{z} ]P_1 \PAR P_2$}
        \end{prooftree}
    \end{minipage}
    \begin{minipage}{0.48\linewidth}
        \centering
        \begin{prooftree}
            \AxiomC{$P_1 \red P_1'$}
            \RightLabel{\textsc{(R-Par)}}
            \UnaryInfC{$P_1 \PAR P_2 \red P_1' \PAR P_2$}
        \end{prooftree}
    \end{minipage}
    \begin{minipage}{0.5\linewidth}
        \centering
        \begin{prooftree}
            \AxiomC{$P \red P'$}
            \RightLabel{\textsc{(R-Nu)}}
            \UnaryInfC{$\nuexp{x \COL \chty} P \red \nuexp{x \COL \chty}P' $}
        \end{prooftree}
    \end{minipage}
    \begin{minipage}{\linewidth}
        \centering
        \begin{prooftree}
            \AxiomC{$\len{\seq{y}} = \len{\seq{v}}$}
            \AxiomC{$\len{\seq{z}} = \len{\seq{w}}$}
            \AxiomC{$\seq{v} \Downarrow \seq{i}$}
            \RightLabel{\textsc{(R-RComm)}}
            \TrinaryInfC{$\rinexp{x}{\seq{y}}{\seq{z}}P_1 \PAR  \outexp{x}{\seq{v}}{\seq{w}}P_2 \red \rinexp{x}{\seq{y}}{\seq{z}}P_1 \PAR [\seq{i}/ \seq{y}, \seq{w} / \seq{z}]P_1 \PAR P_2$}
        \end{prooftree}
    \end{minipage}
    \begin{minipage}{\linewidth}
        \centering
        \begin{prooftree}
            \AxiomC{$v \Downarrow i \neq 0$}
            \RightLabel{\textsc{(R-If-T)}}
            \UnaryInfC{$\ifexp{v}{P_1}{P_2} \red P_1$}
        \end{prooftree}
    \end{minipage}
    \begin{minipage}{\linewidth}
        \centering
        \begin{prooftree}
            \AxiomC{$v \Downarrow 0$}
            \RightLabel{\textsc{(R-If-F)}}
            \UnaryInfC{$\ifexp{v}{P_1}{P_2} \red P_2$}
        \end{prooftree}
    \end{minipage}
    \begin{minipage}{\linewidth}
        \centering
        \begin{prooftree}
            \AxiomC{\( \len{\seq{x}} = \len{\seq{i}}\)}
            \RightLabel{\textsc{(R-LetND)}}
            \UnaryInfC{$\ndlet{x}{P} \red [\seq{i} / \seq{x}]P$}
        \end{prooftree}
    \end{minipage}
    \begin{minipage}{\linewidth}
        \centering
        \begin{prooftree}
            \AxiomC{$P \piequiv P_1 \red P_1' \piequiv P'$}
            \RightLabel{\textsc{(R-Cong)}}
            \UnaryInfC{$P \red P'$}
        \end{prooftree}
    \end{minipage}
    \begin{minipage}{0.3\linewidth}
        \centering
        \begin{prooftree}
            \AxiomC{}
            \RightLabel{\textsc{(R-Int)}}
            \UnaryInfC{$i \Downarrow i$}
        \end{prooftree}
    \end{minipage}
    \begin{minipage}{0.38\linewidth}
        \centering
        \begin{prooftree}
            \AxiomC{$\seq{v} \Downarrow \seq{i}$}
            \RightLabel{\textsc{(R-Op)}}
            \UnaryInfC{$\op(\seq{v}) \Downarrow \llbracket \op\rrbracket (\seq{i})$.}
        \end{prooftree}
    \end{minipage}
    \normalsize
    \caption{The reduction rules of the $\pi$-calculus. Here \( \llbracket \op \rrbracket \colon \mathbb{Z}^n \to \mathbb{Z} \) represents the interpretation of the operation \( \op \) whose arity is \( n \). }
    \label{fig:pi_reduction}
\end{figure}






\subsection{Reduction Semantics of the Sequential Language}
Here, we define the reduction semantics for the sequential language.
We actually define two kinds of semantics:
one is a standard reduction relation
\((\Def,\Exp)\sred (\Def',\Exp')\), which evaluates \( \Exp_1 \nondet \Exp_2\) to either \( \Exp_1 \) or \( \Exp_2\); the other is a non-standard reduction relation
\((\Def,\Exp)\nsred(\Def',\Exp')\), which
 does not discard branches of non-deterministic choices.

\begin{definition}
  The \emph{reduction relation $\seqto$ on %
    sequential programs} is defined by the set of rules in Figure~\ref{fig:seq_reduction}.
    In the rule \rn{SR-App} we are considering \(\Def\) as a map that maps \( f \) to \( \Def(f) =  \{ \lambda \seq{x}. \Exp \mid \fdef{f}{\seq{x}}{\Exp} \in \Def \}\).
\end{definition}
\begin{figure}[tb]
    \centering
    \small
    \begin{minipage}{\linewidth}
        \centering
        \begin{prooftree}
            \AxiomC{$\len{\seq{x}} = \len{\seq{i}}$}
            \RightLabel{\textsc{(SR-LetND)}}
            \UnaryInfC{$(\Def, \ndlet{x}{\Exp} )  \sred (\Def, [\seq{i}/\seq{x}]\Exp)$}
        \end{prooftree}
    \end{minipage}
    \begin{minipage}{\linewidth}
        \centering
        \begin{prooftree}
            \AxiomC{\( (\lambda \seq{y}.\Exp) \in  \Def(f)  \)}
            \AxiomC{$\len{\seq{y}} = \len{\seq{v}}$}
            \AxiomC{$\seq{v} \Downarrow \seq{i}$}
            \RightLabel{\textsc{(SR-App)}}
            \TrinaryInfC{$(\Def, f(\seq{v})) \sred (\Def, [\seq{i}/\tilde{y}] \Exp)$}
        \end{prooftree}
    \end{minipage}
    \begin{minipage}{\linewidth}
        \centering
        \begin{prooftree}
            \AxiomC{$v \Downarrow i$ \qquad \( i \neq 0\)}
            \RightLabel{\textsc{(SR-If-T)}}
            \UnaryInfC{$(\Def, \ifexp{v}{\Exp_1}{\Exp_2})  \sred (\Def, \Exp_1)$}
        \end{prooftree}
    \end{minipage}
    \begin{minipage}{\linewidth}
        \centering
        \begin{prooftree}
            \AxiomC{$v \Downarrow 0$}
            \RightLabel{\textsc{(SR-If-F)}}
            \UnaryInfC{$(\Def, \ifexp{v}{\Exp_1}{\Exp_2}) \sred (\Def, \Exp_2)$}
        \end{prooftree}
    \end{minipage}
    \\[.3cm]
    \begin{minipage}{\linewidth}
        \centering
        \begin{prooftree}
            \AxiomC{}
            \RightLabel{\textsc{(SR-Cho-L)}}
            \UnaryInfC{$(\Def, \Exp_1 \nondet \Exp_2) \sred (\Def, \Exp_1)$}
        \end{prooftree}
    \end{minipage}
    \\[.3cm]
    \begin{minipage}{\linewidth}
        \centering
        \begin{prooftree}
            \AxiomC{}
            \RightLabel{\textsc{(SR-Cho-R)}}
            \UnaryInfC{$(\Def, \Exp_1 \nondet \Exp_2) \sred (\Def, \Exp_2)$}
        \end{prooftree}
    \end{minipage}
    \begin{minipage}{\linewidth}
        \centering
        \begin{prooftree}
            \AxiomC{$v \Downarrow i$ \qquad \( i \neq 0\)}
            \RightLabel{\textsc{(SR-Ass-T)}}
            \UnaryInfC{$(\Def, \textbf{Assume}(v);E) \sred (\Def, \Exp)$}
        \end{prooftree}
    \end{minipage}
    \begin{minipage}{\linewidth}
        \centering
        \begin{prooftree}
            \AxiomC{$v \Downarrow 0$}
            \RightLabel{\textsc{(SR-Ass-F)}}
            \UnaryInfC{$(\Def, \textbf{Assume}(v);\Exp) \sred (\Def, \skipexp)$}
        \end{prooftree}
    \end{minipage}
    \normalsize
    \caption{Reduction rules of the sequential language}
    \label{fig:seq_reduction}
\end{figure}

We now define a non-standard reduction relation that keeps
all the non-deterministic branches \shchanged{during} the reduction.
This non-standard reduction relation has a better match with the reduction of processes.
Since processes have structural rules, we also introduce structural rules on expressions.

\begin{definition}[structural congruence for sequential expressions]
    The \emph{structural congruence relation for expressions}, written \( \Exp_1 \expequiv \Exp_2 \), is defined as the least congruence relation that satisfies the following rules.
    \begin{gather*}
        \Exp_1 \nondet \Exp_2 \expequiv \Exp_2 \nondet \Exp_1
        \qquad (\Exp_1 \nondet \Exp_2) \nondet \Exp_3 \expequiv \Exp_1 \nondet (\Exp_2 \nondet \Exp_3)
        \qquad \Exp \nondet \skipexp \expequiv \Exp
        \qquad
    \end{gather*}


\end{definition}




\begin{definition}
    The \emph{non-standard reduction relation} $\nsred$ on the set of sequential programs is defined
    by the set of rules in Figure~\ref{fig:seq_reduction2} together with all the rules in
    Figure~\ref{fig:seq_reduction} (with $\sred$ replaced by $\nsred$), except for \rn{SR-Cho-L} and  \rn{SR-Cho-R}.
    To simplify the notation, we may write \( \Exp \nsred_\Def \Exp' \) for \( (\Def, \Exp) \nsred (\Def, \Exp') \) or even \( \Exp \nsred \Exp' \) if \( \Def \) is clear from the context.
\end{definition}

\begin{figure}[tb]
    \centering
    \small
    \begin{minipage}{\linewidth}
        \centering
        \begin{prooftree}
            \AxiomC{\( \Exp \expequiv \Exp_1 \quad (\Def, \Exp_1) \nsred (\Def, \Exp'_1) \quad \Exp'_1 \expequiv \Exp' \)}
            \RightLabel{\textsc{(SR-Cong)}}
            \UnaryInfC{$(\Def, \Exp) \nsred (\Def, \Exp')$}
        \end{prooftree}
    \end{minipage}
    \begin{minipage}{\linewidth}
        \centering
        \begin{prooftree}
            \AxiomC{$(\Def, \Exp_1) \nsred (\Def, \Exp_1')$}
            \RightLabel{\textsc{(SR-ChoBody-L)}}
            \UnaryInfC{$(\Def, \Exp_1 \nondet \Exp_2) \nsred (\Def, \Exp_1' \nondet \Exp_2)$}
        \end{prooftree}
    \end{minipage}
    \begin{minipage}{\linewidth}
        \centering
        \begin{prooftree}
            \AxiomC{$(\Def, \Exp_2) \nsred (\Def, \Exp_2')$}
            \RightLabel{\textsc{(SR-ChoBody-R)}}
            \UnaryInfC{$(\Def, \Exp_1 \nondet \Exp_2) \nsred (\Def, \Exp_1 \nondet \Exp_2')$}
        \end{prooftree}
    \end{minipage}

    \normalsize
    \caption{Additional rules for the  non-standard reduction relation}
    \label{fig:seq_reduction2}
\end{figure}

For the proof of the soundness of our transformation
(given in Appendix~\ref{sec:soundness}),
we also prepare a relation \( \Def \subdef \Def' \), which intuitively
means that \(\Def\) can simulate \(\Def'\) so that if \((\Def,\Exp)\) is terminating,
so is \((\Def',\Exp)\)
(cf.\ Lemma~\ref{lem:subdef}).

\begin{figure}[tb]
    \centering
    \small
    \begin{minipage}{.4\linewidth}
        \centering
        \begin{prooftree}
            \AxiomC{ }
            \RightLabel{\textsc{(D-Id)}}
            \UnaryInfC{\( \Def \subdef \Def\)}
        \end{prooftree}
    \end{minipage}
    \begin{minipage}{.4\linewidth}
        \centering
        \begin{prooftree}
            \AxiomC{\( \Def = \Def_1 \mrg \Def_2 \)}
            \RightLabel{\textsc{(D-Splt)}}
            \UnaryInfC{\( \Def \subdef \Def_1 \)}
        \end{prooftree}
    \end{minipage}
    \begin{minipage}{\linewidth}
        \centering
        \begin{prooftree}
            \AxiomC{\( \Def = (\ndlet{x}{\Def'}) \) \qquad \( \len{\seq{x}} = \len{\seq{v}} \)}
            \RightLabel{\textsc{(D-ND)}}
            \UnaryInfC{\( \Def \subdef [\seq{v}/\seq{x}]\Def' \)}
        \end{prooftree}
    \end{minipage}
    \begin{minipage}{.4\linewidth}
        \centering
        \begin{prooftree}
            \AxiomC{ \( \Def_1 \subdef \Def_1' \)}
            \RightLabel{\textsc{(D-Mrg)}}
            \UnaryInfC{\( \Def_1 \mrg \Def_2 \subdef \Def_1' \mrg \Def_2 \)}
        \end{prooftree}
    \end{minipage}
    \begin{minipage}{.4\linewidth}
        \centering
        \begin{prooftree}
            \AxiomC{ \( \Def_1 \subdef \Def_2 \) \qquad \( \Def_2 \subdef \Def_3 \)}
            \RightLabel{\textsc{(D-Trns)}}
            \UnaryInfC{\( \Def_1 \subdef \Def_3 \)}
        \end{prooftree}
    \end{minipage}
    \normalsize
    \caption{Preorder on function definitions}
    \label{fig:subdef}
\end{figure}

\begin{lemma}
    \label{lem:subdef}
    Suppose that \( \Def \subdef \Def' \) and \( (\Def', \Exp ) \nsred ( \Def', \Exp') \).
    Then \( (\Def, \Exp ) \nsred^+ ( \Def, \Exp') \).
\end{lemma}
\begin{proof}
    By induction on the derivation of \( \Def \subdef \Def' \).
    \qed
\end{proof}