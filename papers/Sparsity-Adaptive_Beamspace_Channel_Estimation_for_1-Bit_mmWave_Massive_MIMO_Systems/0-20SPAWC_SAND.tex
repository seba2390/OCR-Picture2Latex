\documentclass[conference,10pt,twocolumn,letter]{IEEEtran}
\IEEEoverridecommandlockouts
\usepackage{cite}
\usepackage[T1]{fontenc}
\usepackage{graphicx}

\usepackage{amssymb}
\usepackage{amsmath}
\usepackage{paralist}
\usepackage{subfigure}
\usepackage{booktabs} 
\usepackage{algorithm}
\usepackage{algorithmic}
\usepackage{amsthm}
\usepackage{multirow}
\usepackage{microtype} 
\usepackage{tablefootnote}
\usepackage{balance}
\usepackage[paper=letterpaper,top=0.7in,bottom=1.0in,right=0.68in,left=0.68in]{geometry}

\input{vmr-symbols-vecbold}
\input{standard-macros}

%\linespread{0.99396}
\usepackage{color}


\setlength{\textfloatsep}{14pt}

\allowdisplaybreaks 

\setlength{\columnsep}{0.24 in} 


\newcommand{\rcomment}[1]{\textcolor{red}{#1}}
\newcommand{\revision}[1]{#1}


\newtheorem{remark}{Remark}
\newtheorem{theorem}{Theorem}
\newtheorem{corollary}{Corollary}



\newcommand{\rhat}[0]{\hat{\bmr}} 
\newcommand{\yhat}[0]{\hat{\bmy}} 
\newcommand{\rshat}[0]{\hat{\bmr}^s} 
\newcommand{\hhat}[0]{\hat{\bmh}} 
\newcommand{\dhat}[0]{\hat{\bmd}} 
\newcommand{\ehat}[0]{\hat{\bme}} 
\newcommand{\qhat}[0]{\hat{\bmq}} 
\newcommand{\hhatstar}[0]{\hat{\bmh}^\star} 
\newcommand{\hhattick}[0]{\hat{\bmh}^\prime} 
\newcommand{\rhattick}[0]{\hat{\bmr}^\prime} 

\newcommand{\mur}[0]{\mu(\rhat)} 
\newcommand{\mure}[1]{\mu(\rhate{#1})} 


\newcommand{\rhate}[1]{\hat{r}_{#1}} 
\newcommand{\yhate}[1]{\hat{y}_{#1}}
\newcommand{\rshate}[1]{\hat{r}^s_{#1}}
\newcommand{\hhate}[1]{\hat{h}_{#1}}
\newcommand{\dhate}[1]{\hat{d}_{#1}}
\newcommand{\ehate}[1]{\hat{e}_{#1}}
\newcommand{\qhate}[1]{\hat{q}_{#1}}
\newcommand{\hhatstare}[1]{\hat{h}^\star_{#1}}
\newcommand{\hhatticke}[1]{\hat{h}^\prime_{#1}}
\newcommand{\rhatticke}[1]{\hat{r}^\prime_{#1}} 

\newcommand{\E}[1]{\mathbb{E}\!\left[#1\right]} 
\newcommand{\smolE}[1]{\mathbb{E}[#1]} 

\newcommand{\realindex}[1]{[#1]_{\mathcal{R}}}
\newcommand{\imagindex}[1]{[#1]_{\mathcal{I}}}
\newcommand{\realpart}[1]{\mathcal{R}\{#1\}}
\newcommand{\imagpart}[1]{\mathcal{I}\{#1\}}

\safemath{\SURE}{\textit{SURE}}
\safemath{\MSE}{\textit{MSE}}
\safemath{\EVM}{\textit{EVM}}


\safemath{\Eo}{E_0}
\safemath{\Eh}{E_h}
\safemath{\Ep}{E_p}
\safemath{\Do}{D_0}
\safemath{\Qo}{Q_0}
\safemath{\Dbeta}{D_\beta}
\safemath{\dalpha}{{\bmd}_{\alpha}}
\safemath{\dhatalpha}{\hat{\bmd}_{\alpha}}
\safemath{\dhatbeta}{\hat{\bmd}_{\beta}}

\safemath{\Tran}{\textnormal{T}}
\safemath{\Herm}{\textnormal{H}}

\safemath{\CN}{\mathcal{CN}}
\safemath{\N}{\mathcal{N}}

\safemath{\diag}{\textnormal{diag}}
\safemath{\trace}{\textnormal{trace}}

\safemath{\sumabsysquared}{a}
\safemath{\sumabsy}{b}
\safemath{\sumone}{(B-k)}
\safemath{\sumabsyinverse}{c}


\begin{document}

\title{Sparsity-Adaptive Beamspace Channel Estimation for 1-Bit mmWave Massive MIMO Systems}



\author{\IEEEauthorblockN{Alexandra Gallyas-Sanhueza$^\text{1}$, Seyed Hadi Mirfarshbafan$^\text{1}$, Ramina Ghods$^\text{2}$, and Christoph Studer$^\text{1}$} \\[-0.1cm]
\IEEEauthorblockA{
\textit{$^\text{1}$Cornell Tech, New York, NY;} \textit{$^\text{2}$Carnegie Mellon University, Pittsburgh, PA}
} 
%
\thanks{The work of RG and CS was supported in part by the US NSF under grants ECCS-1408006, CCF-1535897,  CCF-1652065, CNS-1717559, and ECCS-1824379. The work of AGS, SHM, and CS was supported in part by Xilinx, Inc.\ and by ComSenTer, one of six centers in JUMP, a Semiconductor Research Corporation (SRC) program sponsored by DARPA.  
%
}\\[-0.2cm]
}

\maketitle
\begin{abstract}
We propose sparsity-adaptive beamspace channel estimation algorithms that improve accuracy for 1-bit data converters in all-digital millimeter-wave (mmWave) massive multiple-input multiple-output (MIMO)  basestations. Our algorithms include a tuning stage based on Stein's unbiased risk estimate (SURE) that automatically selects optimal denoising parameters depending on the instantaneous channel conditions. Simulation results with line-of-sight (LoS) and non-LoS mmWave massive MIMO channel models show that our algorithms improve channel estimation accuracy with 1-bit measurements in a computationally-efficient manner.
\end{abstract}

% !TEX root = 0-20SPAWC_SAND.tex
% DO NOT REMOVE THE ABOVE COMMENT!
\section{Introduction}
%
Millimeter-wave (mmWave) and massive multi-user (MU) multiple-input multiple-output (MIMO) will be  core technologies for future wireless systems~\cite{larsson14a, rappaport15a}.
%
The combination of these technologies enables simultaneous communication to multiple user equipments (UEs) at unprecedentedly high data rates. 
%
These advantages come at the cost of significantly increased power consumption, implementation complexity, and system costs. A viable solution to address these challenges is the use of low-resolution data converters combined with sophisticated but efficient baseband processing algorithms in all-digital basestations (BS) architectures~\cite{dutta2019case,jacobsson17b,li17b,mo16b,panagiotis20}. 

\subsection{Channel Estimation with Low-Resolution Data Converters}
%
Coarse quantization of the received baseband samples, due to the use of low-resolution analog-to-digital converters (ADCs) at the BS, together with the high path loss at mmWave or terahertz (THz) frequencies~\cite{rappaport15b, gao16}, renders the acquisition of accurate channel estimates a challenging task.
%
Fortunately,  wave propagation at mmWave or THz frequencies is directional~\cite{akdeniz14a} and channels typically consist only of a few dominant propagation paths~\cite{rappaport13a,rappaport15a}. Both of these properties cause the channel vectors to be sparse in the beamspace domain, which can be exploited to perform denoising that improves reliability of data transmission~\cite{alkhateeb14a,mo14b,tang13,brady13,ghods19a}. 

Practical sparsity-exploiting channel denoising methods for mmWave massive MU-MIMO systems must exhibit low computational complexity due to the large number of BS antenna elements and the potentially large number of UEs that commmunicate simultaneously.
%
A low-complexity mmWave channel denoising algorithm called BEACHES (short for beamspace channel estimation) has been proposed recently in~\cite{ghods19a}. This method has orders-of-magnitude lower complexity than state-of-the-art denoising methods, such as atomic norm minimization (ANM)~\cite{bhaskar13} and Newtonized orthogonal matching pursuit (NOMP) \cite{mamandipoor16}. 
%
However, all of these existing denoising methods perform poorly when denoising channel vectors that were acquired through low-resolution data converters. 
%
Channel estimation with 1-bit ADCs has been analyzed in~\cite{li17b,li16a,jacobsson17b,mollen16c,studer16a}. 
%
Beamspace sparsity of mmWave channels has been exploited to denoise channel vectors from 1-bit measurements in \cite{mo16b, huang19,kaushik18}.
%
However, all of these denoising methods exhibit high complexity, ignore beamspace sparsity, and/or require a number of parameters that must be adapted to the instantaneous propagation conditions, such as the number of dominant propagation paths.  


\subsection{Contributions}
%
We propose low-complexity channel estimation algorithms for mmWave massive MU-MIMO systems that operate with 1-bit data converters.
%
By using a Bussgang-like decomposition~\cite{bussgang52a} of the 1-bit measurement process, our methods adapt the optimal denoising parameters to the channel's instantaneous sparsity via Stein's unbiased risk estimate (SURE).
%
We propose two methods that build upon BEACHES put forward in~\cite{ghods19a} and a novel method, referred to as Sparsity-Adaptive oNe-bit Denoiser (SAND), which automatically tunes two algorithm parameters to minimize the channel estimation mean-square error (MSE). 
%
To demonstrate the efficacy of our channel estimation algorithms, we perform  MSE and bit error rate (BER) simulations with  line-of-sight (LoS) and non-LoS mmWave channels in a massive MU-MIMO system. 

\subsection{Notation}
%
Lowercase and uppercase boldface letters denote column vectors and matrices, respectively. 
%
The $k$th entry of the vector~$\bma$ is~$a_k$; 
the real and imaginary parts are $\realindex{\bma} = \realpart{\bma}$ and $\imagindex{\bma} = \imagpart{\bma}$, respectively. 
For a matrix $\bA$, 
its transpose and Hermitian transpose are $\bA^\Tran$ and $\bA^\Herm$, respectively. 
A complex Gaussian vector $\bma$ with mean $\bmm$ and covariance $\bK$ is written as $\bma\sim\CN(\bmm, \bK)$.
%
Expectation is denoted by $\smolE{\cdot}$. 
%
% !TEX root = 0-20SPAWC_SAND.tex
% DO NOT REMOVE THE ABOVE COMMENT!
\section{1-bit Quantized System Model}
\label{sec:systemmodel}

%
We consider a mmWave massive MU-MIMO uplink system in which $U$ single-antenna UEs transmit data to a $B$-antenna BS equipped with a uniform linear array (ULA). 
%
We assume that each of the $B$ radio-frequency (RF) chains at the BS contains a pair of 1-bit ADCs that separately quantize the in-phase and quadrature signals.
%
A widely-used, yet simplistic channel vector model for such systems is as follows \cite{tse05a}:
\begin{align} \label{eq:antenna_domain_channel}
\bmh =  \textstyle \sum_{\ell=1}^{L}{\!\kappa_\ell\bma(\Omega_\ell)}, \, \bma(\Omega)\! =\! [e^{j0\Omega}, e^{j1\Omega}, \ldots, e^{j(B-1)\Omega}]^\Tran\!\!.
\end{align}
%
Here, $L$ stands for the number of propagation paths arriving at the BS, $\kappa_\ell\in\complexset$ is the channel gain of the $\ell$th path, $\bma(\Omega_\ell)\in\complexset^B$ contains the relative phases between BS antennas, and $\Omega_\ell\in[0,2\pi)$ is determined by the $\ell$th path's incident angle.
%
We emphasize that our simulation results in \fref{sec:simulations} will use more realistic mmWave channel models. 

We consider orthogonal training-based channel estimation, where only one UE transmits a pilot at a time---a generalization to other training schemes is part of ongoing work. To model 1-bit ADCs, we define $Q(z) = \sign(\realpart{z}) + j \sign(\imagpart{z})$, which is applied element-wise to vectors. The 1-bit quantized channel vector for a given UE can be modeled as follows \cite{li17b}:
\begin{align} \label{eq:quantized_antenna_domain}
\bmr = & Q\!\left(\varrho\bmh+\bmn\right)\!,
\end{align}
where $\bmn\sim\CN(\boldsymbol0,\No\bI_B)$ models thermal noise. Without loss of generality, we assume  $\varrho=1$ for the rest of the paper. 

All of the above vectors are in the antenna domain, where each entry is associated with one of the $B$ BS antennas. 
%
By taking the discrete Fourier transform (DFT) across the antenna array, we can transform these vectors into the beamspace domain, where each entry corresponds to an incident angle. 
%
From \fref{eq:antenna_domain_channel} we see that $\bmh$ is a superposition of~$L$ complex sinusoids. Consequently, the beamspace domain representation $\hhat = \bF\bmh$, where~$\bF$ is the $B\times B$ unitary DFT matrix, will be  sparse assuming that $L\ll B$. In what follows, all beamspace domain quantities are designated with a $\hat{\text{hat}}$.

\fref{fig:sparse_channel} shows examples for LoS and non-LoS channel vectors in the beamspace domain without and with 1-bit quantization. 
%
Clearly, the unquantized beamspace vectors $\hhat$ exhibit sparsity; the 1-bit quantized beamspace vectors, which are obtained from $\rhat = \bF\bmr$, also exhibit sparsity but, in addition, are distorted by quantization artifacts.
%
We also observe that the quantization artifacts differ significantly between the LoS and non-LoS channels, which exhibit different levels of sparsity. 
%
In what follows, we develop algorithms that exploit beamspace sparsity to denoise 1-bit quantized channel vectors while adapting the denoising parameters to the instantaneous channel sparsity. 


\begin{figure}[tp]
\subfigure[LoS channel]{\includegraphics[width=0.495\columnwidth]{fig/mmMagic_los_beamspace.pdf}}
\subfigure[Non-LoS channel]{\includegraphics[width=0.495\columnwidth]{fig/mmMagic_nlos_beamspace.pdf}}
\vspace{-0.35cm}
\caption{Beamspace representation of an unquantized and a 1-bit quantized channel vector for a line-of-sight (LoS) and non-LoS scenario. The channel vectors are generated with the QuaDRiGa mmMAGIC UMi model \cite{jaeckel2019quadriga} at $60$\,GHz for a $256$ uniform-linear array (ULA) with $\lambda/2$ antenna spacing. The average energy has been normalized to $1$ and no noise is present.} 
\label{fig:sparse_channel}
\end{figure}


% !TEX root = 0-20SPAWC_SAND.tex
% DO NOT REMOVE THE ABOVE COMMENT!

\section{BEACHES-Based 1-Bit Denoising}
\label{sec:one_bit_denoising}
Before we discuss denoising methods for 1-bit measurements, we briefly review the BEACHES algorithm in~\cite{ghods19a}, which was developed for systems with high-resolution data converters. Assume that we observe a noisy measurement of the channel vector $\bmh$ in the beamspace domain as
\begin{align} \label{eq:noisy_measurement}
\yhat = \hhat + \ehat,
\end{align}
where $\ehat\sim\setC\setN(\bZero,\Eo\bI)$ models channel estimation errors.
BEACHES denoises $\yhat$ by applying the soft-thresholding function $\hhattick = \eta(\yhat,\tau)$ defined as 
\begin{align}
\textstyle [\eta(\yhat,\tau)]_b = \frac{\yhate{b}}{|\yhate{b}|}\max\{|\yhate{b}|-\tau,0\},  \quad b=1,\ldots,B,
\end{align}  
where we define ${\yhate{b}}/{|\yhate{b}|} = 0$ for $\yhate{b} = 0$ and the parameter $\tau$ is the denoising threshold. 
%
For an optimally-chosen denoising threshold~$\tau^\star$, the soft-thresholding function suppresses noise (which is typically weak) while leaving the sparse components that pertain to the channel vector mostly intact.  We define $\tau^\star$ as the denoising threshold that minimizes the MSE  
%
\begin{align} \label{eq:MSE}
\textstyle \MSE = \frac{1}{B}\smolE{\|\hhattick-\hhat\|^2},
\end{align}
%
which determines the optimally-denoised channel vector $\hhatstar = \eta(\yhat,\tau^\star)$ in the beamspace domain. However, the MSE expression depends on the unknown vector $\hhat$. BEACHES circumvents this issue by using Stein's unbiased risk estimator (SURE) \cite{donoho95}, which is an unbiased estimate of the MSE in~\fref{eq:MSE} that does \emph{not} depend on $\hhat$. 
%
BEACHES requires (i) the channel estimation error $\hat\bme$ to be i.i.d.\ Gaussian
and (ii) knowledge of the channel estimation variance $\Eo$ to optimally denoise the channel vector at a complexity that scales only with $B\log(B)$. 


%
\subsection{The $\textit{1}$-BEACHES Algorithm} \label{sec:1BEACHES}
%
We now present $1$-BEACHES, which denoises the received 1-bit channel measurements using BEACHES.
%
For this method, we model the 1-bit received vector $\bmr$  in \fref{eq:quantized_antenna_domain}   as 
\begin{align} \label{eq:1beaches_linearization}
{\bmr}=  Q(\bmh+\bmn) = \bmh + \bmq,
\end{align}
where the vector $\bmq$ depends on $\bmh$ and~$\bmn$, and models quantization errors and noise.
%
By transforming $\bmr$ into the beamspace domain, we have 
\begin{align} \label{eq:beamspace_domain_bussgang}
\hat{\bmr}= \bF \bmr = \hhat + \qhat.
\end{align}
Even though the vector $\bmq$ is not Gaussian distributed, the beamspace version $\qhat=\bF\bmq$ is well-approximated by a Gaussian random vector as each entry is a sum of all entries of $\bmq$ with different phases. 
%
To denoise the system in~\fref{eq:beamspace_domain_bussgang} with BEACHES, we need knowledge of the variance~$\Qo$ of the entries in $\qhat$.
%
By assuming that $\hhat=\bF\bmh$ is circularly-symmetric, which is reasonable as $\bmh$ is a sum of complex sinusoids as modeled in~\fref{eq:antenna_domain_channel}, we obtain   
\begin{align} \label{eq:expression0}
\textstyle \Qo =  \frac{1}{B}\smolE{\|\qhat\|^2} =  \frac{1}{B}\smolE{\|\bmr\|^2+\|\bmh\|^2-2\realpart{\bmh^\Herm\bmr}}.
\end{align}
%
In order to obtain a closed-form expression of $\Qo$ with a minimal number of parameters, we further assume\footnote{This assumption is accurate if the number of propagation paths $L$ in \fref{eq:antenna_domain_channel} is large. As shown in \fref{sec:simulations}, this assumption is simplistic for LoS channels.} that $\bmh\sim\CN(\boldsymbol0,\Eh\bI_B)$, which leads to $\Qo = 2+\Eh-{{4\Eh}/{\sqrt{\pi({\Eh}+{\No})}}}$,
where  $\Eh=\frac{1}{B}\smolE{\|\bmh\|^2}$ and~$\smolE{\bmh^\Herm\bmr}$ in~\fref{eq:expression0} is computed in  \fref{app:bussgang_gain}. 
%
Under these assumptions, the beamspace representation \fref{eq:beamspace_domain_bussgang} has the same form as \fref{eq:noisy_measurement}, where 
$\yhat=\rhat$ and we model $\ehat=\qhat\sim\CN(\boldsymbol0,\Qo\bI_B)$, which allows us to (i) apply BEACHES to find the optimal denoising threshold~$\tau^\star$ given~$\rhat$ and the variance~$\Qo$, and (ii) compute $\hhatstar = \eta(\rhat,\tau^\star).$ We call this procedure $1$-BEACHES.


\subsection{The $\alpha$-BEACHES Algorithm} \label{sec:alphaBEACHES}
In the model \fref{eq:1beaches_linearization}, the error $\bmq$ will be large if the power of~$\hhat$ differs from the power of $\rhat$.
%
We now derive $\alpha$-BEACHES which addresses this aspect.
%
To this end, we use a Bussgang-like decomposition \cite{bussgang52a} that models the 1-bit ADCs as 
\begin{align} \label{eq:bussgang_decomposition}
\bmr = Q(\bmh+\bmn) = \alpha \bmh + \bmd,
\end{align}
where $\alpha$ is a scalar that minimizes the distortion variance $\smolE{\|\bmd\|^2}$ and also ensures $\smolE{\bmd^\Herm\bmh}=0$.
%
By assuming that the vector~$\bmh$ is circularly symmetric, we have
\begin{align} \label{eq:alpha_bussgang_general}
\alpha = & 
\argmin_{\alpha^\prime\in\complexset}\smolE{\|\bmr-\alpha^\prime\bmh\|^2} 
= \textstyle \frac{\smolE{\bmh^\Herm \bmr}}{\smolE{\|\bmh\|^2}}.
\end{align}
To obtain a closed-form expression for $\alpha$, we assume $\bmh\sim\setC\setN(\bZero,\Eh\bI)$ as in 1-BEACHES
and use the derivation of $\smolE{\bmh^\Herm\bmr}$ in \fref{app:bussgang_gain},  which yields 
$\alpha = {2}/{\!\sqrt{\pi({\Eh}+{\No})}}$.


In \fref{eq:bussgang_decomposition}, the distortion $\bmd$ is not Gaussian. By transforming into beamspace domain and dividing the result by $\alpha$, we get
\begin{align} \label{eq:bussgang_decomposition_beamspace}
\textstyle \frac{1}{\alpha}\rhat = \frac{1}{\alpha}\bF \bmr =  \hhat + \frac{1}{\alpha}\dhat,
\end{align}
in which the distortion $ {\dhat}/{\alpha}$ is well-approximated by a  Gaussian, as each entry is a scaled and phase-shifted sum of all of the entries of $\bmd$. The distortion variance ${\Do}/{\alpha^2}$ is 
\begin{align}
\textstyle \frac{1}{B}\frac{1}{\alpha^2} \smolE{\|\dhat\|^2}
= \frac{1}{B}\frac{1}{\alpha^2}\smolE{\|\bmr\|^2\!-\alpha^2\|\bmh\|^2}\! = \frac{2}{\alpha^2}-E_h.
\end{align}
%
The model \fref{eq:bussgang_decomposition_beamspace}, enables us to apply BEACHES to $\rhat/{\alpha}$ in order to determine the denoising threshold $\tau^\star$ given $\rhat/\alpha$ and $\Do/\alpha^2$. Finally, $\alpha$-BEACHES computes $\hhatstar = \eta(\frac{\rhat}{\alpha},\tau^\star).$

% !TEX root = 0-20SPAWC_SAND.tex
% DO NOT REMOVE THE ABOVE COMMENT!

\section{SAND: Sparsity-Adaptive oNe-bit Denoiser} \label{sec:SAND}

As a generalized variant of $\alpha$-BEACHES, we next develop a sparsity-adaptive method that \emph{simultaneously} learns a prefactor~$\gamma$ and a denoising threshold~$\tau$ in order to minimize the MSE.
%
By defining our two-parameter estimator\footnote{This estimator is equivalent to $\hhattick = \eta(\gamma'\rhat,\tau')$ for $\gamma = \gamma'$ and $\tau = \frac{\tau'}{\gamma}$.} as $\hhattick = \gamma\eta(\rhat,\tau)$, we aim to find the parameters $\gamma^\star$ and $\tau^\star$ that minimize the MSE in \fref{eq:MSE}. 
%
Since the MSE depends on the unknown vector~$\hhat$, we select the optimal parameters $\gamma^\star$ and $\tau^\star$ that minimize SURE, which (i) is an unbiased estimator of the MSE so that $\E{\SURE}=\MSE$ and $\lim_{B\to\infty}\SURE = \MSE$ (see \cite{ghods19a} for the details) and (ii) does \emph{not}  depend on $\hhat$. 
%
For any weakly differentiable estimator $\mur$, using the decomposition \fref{eq:bussgang_decomposition_beamspace} and assuming that $\dhat$ is i.i.d.\ Gaussian,  SURE is given by  
\begin{align}
\SURE = \, & \textstyle \frac{1}{B} {\|\mur\|^2}+ \frac{2-\Do}{\alpha^2}  -  \frac{1}{B} {\frac{2}{\alpha}\realpart{{\rhat}^\Herm \mur}} \notag\\
& + \textstyle \frac{1}{B} \sum_{b=1}^B \frac{\Do}{\alpha}\left(\frac{\partial\realindex{\mure{b}}}{\partial\realindex{\rhate{b}}}+\frac{\partial\imagindex{\mure{b}}}{\partial\imagindex{\rhate{b}}}\right)\!.  \label{eq:sure}
\end{align}
Refer to \fref{app:sand_sure} for the proof.
%
%
Since SURE is an unbiased estimator of the MSE, we use SURE in \fref{eq:sure} with $\mur = \gamma\eta(\rhat,\tau)$, in order to find the optimal parameters $\gamma^\star$ and $\tau^\star$.
%
While a na\"ive approach could perform a two-dimensional grid search over the tuple $(\gamma,\tau)$, we next show that we can efficiently find $\gamma^\star$ and $\tau^\star$ with $\setO(B\log(B))$ complexity.

%
Let $\rshat$ be a vector containing the absolute values of $\rhat$ sorted in ascending order. For a given $\tau$, let $k$ be the number of entries in $\rshat$ that are smaller than $\tau$.
For the denoiser $\mur = \gamma\eta(\rhat,\tau)$, following the derivations in \cite[App.~B]{ghods19a}, SURE in \fref{eq:sure} is 
\begin{align} \label{eq:SureShrink1}
\SURE = \, & \textstyle \frac{1}{B} \gamma^2\sum_{b=k+1}^{B}{(\rshate{b}-\tau)^2} + \frac{2-\Do}{\alpha^2} \\
& - \textstyle \frac{1}{B} \frac{\gamma}{\alpha}\sum_{b=k+1}^{B}\left({{2}{\rshate{b}(\rshate{b}-\tau)}}-{\Do}\left(2-\textstyle\frac{\tau}{\rshate{b}}\right)\right) \notag.
\end{align}
%
By defining the quantities $\sumabsysquared = \sum_{b=k+1}^{B} (\rshate{b})^2$, $\sumabsy = \sum_{b=k+1}^{B} \rshate{b}$ and $\sumabsyinverse = \sum_{b=k+1}^{B} (\rshate{b})^{-1}$, we can rewrite \fref{eq:SureShrink1} as
\begin{align} 
\SURE = \, & \textstyle \frac{1}{B}\gamma^2\left(\sumabsysquared-2\tau\sumabsy+\tau^2\sumone\right) + \frac{2-\Do}{\alpha^2} \notag \\
& - \textstyle \frac{1}{B}\frac{\gamma}{\alpha}\left(2\left(\sumabsysquared-\tau\sumabsy\right)  
- \Do\left(2\sumone-\tau\sumabsyinverse\right)\right)\!.
\label{eq:SureShrinkabc}
\end{align}
%
For a fixed $\tau$, the optimal $\gamma^\star\in\reals_{\geq0}$ that minimizes \fref{eq:SureShrinkabc} is% given by
\begin{align} \label{eq:gammastar}
\gamma^\star = \max\{0,\textstyle \frac{2\left(\sumabsysquared-\tau\sumabsy\right)  
	- \Do\left(2\sumone-\tau\sumabsyinverse\right)}{2\alpha\left(\sumabsysquared-2\tau\sumabsy+\tau^2\sumone\right)}\}.
\end{align}
%
The optimal threshold $\tau^\star$ could take any value between $0$ and~$\rshate{B}$. However, as in the derivation of BEACHES~\cite{mirfarshbafan19a}, we restrict the search to values in $\rshat$, as it significantly reduces the complexity, without sacrificing performance. We also set an upper limit for $\tau$ of $\sqrt{2\Do\log(B)}$, which ensures (with high probability) that the threshold is lower than the largest noise realization~\cite{donoho95}.
%
For each $\tau = \rshate{k}$, $k = 0,\ldots,B$ (with $\rshate{0} = 0$), and for its associated $\gamma^\star$ given by \fref{eq:gammastar}, we evaluate SURE in \fref{eq:SureShrinkabc}, and then pick $\gamma^\star$ and $\tau^\star$ that result in the minimum value of SURE. We call the resulting algorithm Sparsity-Adaptive oNe-bit Denoiser (SAND), which is summarized in \fref{alg:SAND}. 
Since the complexity of a fast Fourier transform (FFT) and sorting scale with $\mathcal{O}(B\log(B))$, and the operations in each iteration (lines 6 to 11) have complexity $\setO(1)$, the overall complexity of SAND scales with $\mathcal{O}(B\log(B))$.


% !TEX root = 0-20SPAWC_SAND.tex
% DO NOT REMOVE THE ABOVE COMMENT!
\makeatletter
\newcommand\fs@betterruled{%
  \def\@fs@cfont{\bfseries}\let\@fs@capt\floatc@ruled
  \def\@fs@pre{\vspace*{5pt}\hrule height.8pt depth0pt \kern2pt}%
  \def\@fs@post{\kern2pt\hrule\relax}%
  \def\@fs@mid{\kern2pt\hrule\kern2pt}%
  \let\@fs@iftopcapt\iftrue}
\floatstyle{betterruled}
\restylefloat{algorithm}
\makeatother
\setlength{\textfloatsep}{5pt}
\begin{algorithm}[tp]
\caption{\strut SAND: Sparsity-Adaptive oNe-bit Denoiser \label{alg:SAND}}
\begin{algorithmic}[1]
%
\STATE {\bf input} $\bmr$, $\alpha$ and $\Do$ 
\STATE $\rhat = \text{FFT}(\bmr)$, $\SURE_{\text{min}}=\infty$	, $\tau=0$			
\STATE $\rshat=\text{sort}\{|\rhat|,\text{`ascend'}\}$, $\rshate{B+1}=\rshate{B+2}=\infty$
\STATE $\sumabsysquared=\sum_{k=1}^{B} {(\rshate{k})^{2}}$, $\sumabsy=\sum_{k=1}^{B} {\rshate{k}}$, $\sumabsyinverse=\sum_{k=1}^{B} {(\rshate{k})^{-1}}$
%
\FOR{$k=0,\ldots,B+1$}
\STATE $\gamma =\max\{0, \frac{2(\sumabsysquared-\tau \sumabsy) - \Do(2\sumone-\tau \sumabsyinverse)}{2\alpha(\sumabsysquared-2\tau \sumabsy+\tau^2 \sumone)}\}$
\STATE $\SURE = \frac{1}{B}\gamma^2\left(\sumabsysquared-2\tau\sumabsy+\tau^2\sumone\right) + \frac{2-\Do}{\alpha^2}
- \frac{1}{B}\frac{\gamma}{\alpha}\left(2\left(\sumabsysquared-\tau\sumabsy\right) 
- \Do\left(2\sumone-\tau\sumabsyinverse\right)\right)$ 					
%
\IF{$\SURE<\SURE_{\text{min}}$ \AND $\tau<\sqrt{2\Do\log(B)}$}		
\STATE $\SURE_{\text{min}} = \SURE$, $\tau^\star = \tau$, $\gamma^\star = \gamma$
\ENDIF
%	
\STATE $\tau \!=\! \rshate{k+1}$, $\sumabsysquared \!=\! \sumabsysquared \!-\! (\rshate{k+1})^2, \sumabsy \!=\! \sumabsy \!-\! \rshate{k+1}, \sumabsyinverse \!=\! \sumabsyinverse \!-\! (\rshate{k+1})^{-1}$
%		
\ENDFOR																				
\STATE $\hhatstare{k}=\gamma^\star\frac{\rhate{k}}{|\rhate{k}|}\max{\{|\rhate{k}|-\tau^\star,0\}}$, $k=1,\ldots,B$
\STATE  {\bf return} $\bmh^\star = \text{IFFT}(\hhatstar)$ 
%
\end{algorithmic}
\end{algorithm}





% !TEX root = 0-20SPAWC_SAND.tex
% DO NOT REMOVE THE ABOVE COMMENT!
\section{Results}
\label{sec:simulations}

We now demonstrate the efficacy of 1-BEACHES, $\alpha$-BEACHES, and SAND. 
%
As reference methods, we consider perfect channel state information (CSI), where $\bmh^\star= \bmh$,  BEACHES \cite{ghods19a}, which denoises the infinite-resolution (unquantized) measurements $\bmy = \bmh+\bmn$, and 1-bit maximum-likelihood (ML) channel estimation, where $\bmh^\star = \bmr$ is the 1-bit observation in~\fref{eq:quantized_antenna_domain}. In addition, we compare the performance to state-of-the-art denoising methods, including (i) Newtonized orthogonal matching pursuit (NOMP) \cite{mamandipoor16} with an equivalent noise variance $\Qo$  and a false alarm rate $P_\text{fa}=0.5$ (using $\Eo$ results in poor performance; $P_\text{fa}$ has been tuned to achieve low MSE at low and high SNR)
%
and (ii) the 1-bit Bussgang linear MMSE  estimator (BLMMSE)~\cite{li16a,li17b}, which corresponds to $\bmh^\star = \frac{\Eh}{\sqrt{\pi(\Eh+\No)}}\bmr$ for the used orthogonal pilots.




\begin{figure}[tp]
	\centering
	\subfigure[LoS channel]{\includegraphics[width=0.235\textwidth]{fig/MSE_256x16_QuadMMLoS_QPSK_BussgangLMMSE_QuantizedChannel_QuantizedSignal_Eh1_Norm-0p5to2_Trials10000}}
	\subfigure[non-LoS channel]{\includegraphics[width=0.235\textwidth]{fig/MSE_256x16_QuadMMnLoS_QPSK_BussgangLMMSE_QuantizedChannel_QuantizedSignal_Eh1_Norm-0p5to2_Trials10000.pdf}}
	\vspace{-0.1cm}
	\caption{Mean square error (MSE) of the considered channel denoising methods for mmWave LoS and non-LoS channels. The proposed sparsity-adaptive denoising methods significantly outperform na\"ive 1-bit ML channel estimation.}
	\label{fig:MSE}
\end{figure}


\subsection{Simulation Setup}
We simulate a mmWave massive MIMO system with $B=256$ BS antennas and $U=16$ single-antenna UEs. 
%
We generate LoS and non-LoS channel matrices using the QuaDRiGa mmMAGIC UMi model \cite{jaeckel2019quadriga} at a carrier frequency of $60$\,GHz for a BS with $\lambda/2$-spaced antennas arranged in a ULA. 
%
The UEs are placed randomly in a $120^\circ$ circular sector around the BS between a distance of $10$\,m and $110$\,m, and the UEs are separated by at least $4^\circ$. 
%
We model UE-side power control to ensure that the highest receive power is at most $6$\,dB higher than that of the weakest UE. 



\begin{figure*}[tp]
	\centering
	\subfigure[LoS, QPSK]{\includegraphics[width=0.24\textwidth]{fig/BER_256x16_QuadMMLoS_QPSK_BussgangLMMSE_QuantizedChannel_QuantizedSignal_Eh1_Norm-0p5to2_Trials10000}}
	\hfill
	\subfigure[non-LoS, QPSK]{\includegraphics[width=0.24\textwidth]{fig/BER_256x16_QuadMMnLoS_QPSK_BussgangLMMSE_QuantizedChannel_QuantizedSignal_Eh1_Norm-0p5to2_Trials10000}}
	\hfill
	\subfigure[LoS, 16-QAM]{\includegraphics[width=0.24\textwidth]{fig/BER_256x16_QuadMMLoS_16QAM_BussgangLMMSE_QuantizedChannel_QuantizedSignal_Eh1_Norm-0p5to2_Trials10000.pdf}}
	\hfill
	\subfigure[non-LoS, 16-QAM]{\includegraphics[width=0.24\textwidth]{fig/BER_256x16_QuadMMnLoS_16QAM_BussgangLMMSE_QuantizedChannel_QuantizedSignal_Eh1_Norm-0p5to2_Trials10000.pdf}}
	\vspace{-0.1cm}
	\caption{Uncoded bit error rate (BER) of 1-bit channel estimation and 1-bit data detection in mmWave LoS and non-LoS channels. We see that $\alpha$-BEACHES and SAND outperform $1$-BEACHES and 1-bit ML for LoS and non-LoS channel conditions for 16-QAM transmission. }
	\label{fig:BER}
\vspace{-0.37cm}	
\end{figure*}
%


\subsection{Mean-Square Error (MSE) Performance}
%
\fref{fig:MSE} shows the channel estimation MSE of the proposed 1-bit denoising algorithms and the considered baseline methods. We observe that the three proposed methods, $1$-BEACHES, $\alpha$-BEACHES, and SAND significantly outperform 1-bit ML channel estimation. Furthermore, we see that $\alpha$-BEACHES and SAND have a slight advantage over $1$-BEACHES in LoS scenarios. Surprisingly, SAND has a slightly higher MSE than $\alpha$-BEACHES, which we attribute to the fact that SAND has to learn two parameters, whereas $ \alpha$-BEACHES only learns the optimal denoising threshold. For that reason, SAND is more sensitive to the assumptions made in footnote~1.
%
NOMP and BLMMSE also outperform 1-bit ML, but their MSE is higher than that of our algorithms, especially at high SNR. 

\subsection{Bit Error Rate (BER) Performance}
%
To assess the impact of the proposed 1-bit denoising algorithms on the uncoded BER performance during the data detection phase, we use the 1-bit Bussgang linear MMSE equalizer proposed in \cite{nguyen19}, which operates on the 1-bit quantized received data using the channel estimates provided by our denoising methods and the considered baseline algorithms. We consider QPSK and $16$-QAM transmission.


\fref{fig:BER} shows that the proposed sparsity-adaptive denoising algorithms significantly outperform na\"ive 1-bit ML channel estimation. We furthermore see that for QPSK, all three methods,  $1$-BEACHES, $\alpha$-BEACHES, and SAND, perform equally well under both LoS and non-LoS scenarios. For 16-QAM, where it is important to get an accurate estimate of the channel gain, $\alpha$-BEACHES and SAND outperform $1$-BEACHES and NOMP, which directly operate with the received 1-bit measurements. 
Hence, correcting the scale of the received data is critical for higher-order constellation sets.
%
While BLMMSE adjusts for the scale, it is unable to exploit sparsity which results in rather poor BER performance.
%
For non-LoS channels, NOMP performs inferior to the proposed methods. 
In addition, NOMP requires high complexity~\cite{mirfarshbafan19a}.
%
Since the propagation conditions (such as the number of propagation paths $L$) are typically unknown in practice, SAND and $\alpha$-BEACHES are the preferred denoising methods. 
%


% !TEX root = 0-20SPAWC_SAND.tex
% DO NOT REMOVE THE ABOVE COMMENT!

\section{Conclusions}
\label{sec:conclusions}


 
We have presented three sparsity-adaptive channel vector denoising algorithms for 1-bit mmWave massive MIMO systems.
%
Two of our algorithms denoise 1-bit measurements of the channel estimates using BEACHES~\cite{ghods19a} in order to automatically adapt the denoising parameter to the instantaneous channel realization. 
%
While 1-BEACHES applies  BEACHES to the 1-bit measurements using the effective noise variance (which also includes the quantization noise variance), $\alpha$-BEACHES uses a Bussgang-like scaling factor~\cite{bussgang52a}, which results in superior performance.
%
We have also introduced SAND (short for Sparsity-Adaptive oNe-bit Denoiser), a novel denoising algorithm with $\setO(B\log(B))$ complexity, which jointly optimizes the thresholding parameter and the scaling factor in a nonparametric fashion. 
%
Our simulations have shown that $\alpha$-BEACHES and SAND perform equally well under the considered LoS and non-LoS mmWave channels and outperform $1$-BEACHES as well as other considered baseline methods in the case of 16-QAM transmission.
\appendices 
% !TEX root = 0-20SPAWC_SAND.tex
% DO NOT REMOVE THE ABOVE COMMENT!

\refstepcounter{section} \label{app:bussgang_gain}
\section*{\fref{app:bussgang_gain}: Derivation of $\frac{1}{B}\E{\bmh^\Herm\bmr}$} 
Since $\bmn\sim\CN(\boldsymbol0,\No\bI_B)$ and $\bmh$ is assumed circularly symmetric, the imaginary part of $\E{\bmh^\Herm\bmr}$ is zero, and
\begin{align}
\textstyle\frac{1}{B}\E{\bmh^\Herm\bmr} = 
\frac{1}{B}\E{\realindex{\bmh}\realindex{\bmr}} + \frac{1}{B}\E{\imagindex{\bmh}\imagindex{\bmr}}\!.
\end{align}
By assuming $\bmh\sim\CN(\boldsymbol0,\Eh\bI_B)$, $\frac{1}{B} \E{\realindex{\bmh}\realindex{\bmr}}$ becomes
\begin{align}
\textstyle\frac{1}{B} & \textstyle\sum_{b=1}^{B}\E{\int_{-\infty}^{-\realindex{n_b}}\frac{-\realindex{h_b}}{\sqrt{\pi\Eh}}e^{-\frac{(\realindex{h_b})^2}{\Eh}}d\realindex{h_b}} \\
\textstyle + & \textstyle\frac{1}{B}\sum_{b=1}^{B}\E{\int_{-\realindex{n_b}}^{-\infty}\!\frac{\realindex{h_b}}{\sqrt{\pi\Eh}}e^{-\frac{(\realindex{h_b})^2}{\Eh}}d\realindex{h_b}}
\!=\! \textstyle\frac{\Eh}{\sqrt{\pi(\Eh+\No)}}. \notag 
\end{align}
Following the same procedure for the imaginary part, we get
\begin{align}
\textstyle\frac{1}{B}\E{\bmh^\Herm\bmr} = \textstyle\frac{2\Eh}{\sqrt{\pi({\Eh}+{\No})}}.
\end{align}
\vspace{-0.4cm}
% !TEX root = 0-20SPAWC_SAND.tex
% DO NOT REMOVE THE ABOVE COMMENT!

\refstepcounter{section} \label{app:sand_sure}
\section*{\fref{app:sand_sure}: Derivation of SURE in \fref{eq:sure}} 

For deriving SURE as in \fref{eq:sure}, we follow the procedure in \cite[App.~A]{ghods19a}, with the following modifications: Instead of $\yhat = \hhat+\ehat$, we use $\rhat = \alpha\hhat + \dhat$. In other words, where~\cite{ghods19a} uses $\yhat\sim\CN(\hhat,\Eo\bI_B)$, we replace it by $\rhat\sim\CN(\alpha\hhat,\Do\bI_B)$. Instead of $g(\yhat) = \mu(\yhat) - \yhat$, we use $g(\rhat) = \mu(\rhat) - \rhat/\alpha$.



\balance
\bibliographystyle{IEEEtran}
\bibliography{bib/confs-jrnls,bib/IEEEabrv,bib/publishers,bib/vipbib,bib/sm_ref}
\balance

\end{document}