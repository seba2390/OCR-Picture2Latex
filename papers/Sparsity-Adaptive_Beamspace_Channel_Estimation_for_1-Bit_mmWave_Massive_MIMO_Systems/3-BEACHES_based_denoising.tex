% !TEX root = 0-20SPAWC_SAND.tex
% DO NOT REMOVE THE ABOVE COMMENT!

\section{BEACHES-Based 1-Bit Denoising}
\label{sec:one_bit_denoising}
Before we discuss denoising methods for 1-bit measurements, we briefly review the BEACHES algorithm in~\cite{ghods19a}, which was developed for systems with high-resolution data converters. Assume that we observe a noisy measurement of the channel vector $\bmh$ in the beamspace domain as
\begin{align} \label{eq:noisy_measurement}
\yhat = \hhat + \ehat,
\end{align}
where $\ehat\sim\setC\setN(\bZero,\Eo\bI)$ models channel estimation errors.
BEACHES denoises $\yhat$ by applying the soft-thresholding function $\hhattick = \eta(\yhat,\tau)$ defined as 
\begin{align}
\textstyle [\eta(\yhat,\tau)]_b = \frac{\yhate{b}}{|\yhate{b}|}\max\{|\yhate{b}|-\tau,0\},  \quad b=1,\ldots,B,
\end{align}  
where we define ${\yhate{b}}/{|\yhate{b}|} = 0$ for $\yhate{b} = 0$ and the parameter $\tau$ is the denoising threshold. 
%
For an optimally-chosen denoising threshold~$\tau^\star$, the soft-thresholding function suppresses noise (which is typically weak) while leaving the sparse components that pertain to the channel vector mostly intact.  We define $\tau^\star$ as the denoising threshold that minimizes the MSE  
%
\begin{align} \label{eq:MSE}
\textstyle \MSE = \frac{1}{B}\smolE{\|\hhattick-\hhat\|^2},
\end{align}
%
which determines the optimally-denoised channel vector $\hhatstar = \eta(\yhat,\tau^\star)$ in the beamspace domain. However, the MSE expression depends on the unknown vector $\hhat$. BEACHES circumvents this issue by using Stein's unbiased risk estimator (SURE) \cite{donoho95}, which is an unbiased estimate of the MSE in~\fref{eq:MSE} that does \emph{not} depend on $\hhat$. 
%
BEACHES requires (i) the channel estimation error $\hat\bme$ to be i.i.d.\ Gaussian
and (ii) knowledge of the channel estimation variance $\Eo$ to optimally denoise the channel vector at a complexity that scales only with $B\log(B)$. 


%
\subsection{The $\textit{1}$-BEACHES Algorithm} \label{sec:1BEACHES}
%
We now present $1$-BEACHES, which denoises the received 1-bit channel measurements using BEACHES.
%
For this method, we model the 1-bit received vector $\bmr$  in \fref{eq:quantized_antenna_domain}   as 
\begin{align} \label{eq:1beaches_linearization}
{\bmr}=  Q(\bmh+\bmn) = \bmh + \bmq,
\end{align}
where the vector $\bmq$ depends on $\bmh$ and~$\bmn$, and models quantization errors and noise.
%
By transforming $\bmr$ into the beamspace domain, we have 
\begin{align} \label{eq:beamspace_domain_bussgang}
\hat{\bmr}= \bF \bmr = \hhat + \qhat.
\end{align}
Even though the vector $\bmq$ is not Gaussian distributed, the beamspace version $\qhat=\bF\bmq$ is well-approximated by a Gaussian random vector as each entry is a sum of all entries of $\bmq$ with different phases. 
%
To denoise the system in~\fref{eq:beamspace_domain_bussgang} with BEACHES, we need knowledge of the variance~$\Qo$ of the entries in $\qhat$.
%
By assuming that $\hhat=\bF\bmh$ is circularly-symmetric, which is reasonable as $\bmh$ is a sum of complex sinusoids as modeled in~\fref{eq:antenna_domain_channel}, we obtain   
\begin{align} \label{eq:expression0}
\textstyle \Qo =  \frac{1}{B}\smolE{\|\qhat\|^2} =  \frac{1}{B}\smolE{\|\bmr\|^2+\|\bmh\|^2-2\realpart{\bmh^\Herm\bmr}}.
\end{align}
%
In order to obtain a closed-form expression of $\Qo$ with a minimal number of parameters, we further assume\footnote{This assumption is accurate if the number of propagation paths $L$ in \fref{eq:antenna_domain_channel} is large. As shown in \fref{sec:simulations}, this assumption is simplistic for LoS channels.} that $\bmh\sim\CN(\boldsymbol0,\Eh\bI_B)$, which leads to $\Qo = 2+\Eh-{{4\Eh}/{\sqrt{\pi({\Eh}+{\No})}}}$,
where  $\Eh=\frac{1}{B}\smolE{\|\bmh\|^2}$ and~$\smolE{\bmh^\Herm\bmr}$ in~\fref{eq:expression0} is computed in  \fref{app:bussgang_gain}. 
%
Under these assumptions, the beamspace representation \fref{eq:beamspace_domain_bussgang} has the same form as \fref{eq:noisy_measurement}, where 
$\yhat=\rhat$ and we model $\ehat=\qhat\sim\CN(\boldsymbol0,\Qo\bI_B)$, which allows us to (i) apply BEACHES to find the optimal denoising threshold~$\tau^\star$ given~$\rhat$ and the variance~$\Qo$, and (ii) compute $\hhatstar = \eta(\rhat,\tau^\star).$ We call this procedure $1$-BEACHES.


\subsection{The $\alpha$-BEACHES Algorithm} \label{sec:alphaBEACHES}
In the model \fref{eq:1beaches_linearization}, the error $\bmq$ will be large if the power of~$\hhat$ differs from the power of $\rhat$.
%
We now derive $\alpha$-BEACHES which addresses this aspect.
%
To this end, we use a Bussgang-like decomposition \cite{bussgang52a} that models the 1-bit ADCs as 
\begin{align} \label{eq:bussgang_decomposition}
\bmr = Q(\bmh+\bmn) = \alpha \bmh + \bmd,
\end{align}
where $\alpha$ is a scalar that minimizes the distortion variance $\smolE{\|\bmd\|^2}$ and also ensures $\smolE{\bmd^\Herm\bmh}=0$.
%
By assuming that the vector~$\bmh$ is circularly symmetric, we have
\begin{align} \label{eq:alpha_bussgang_general}
\alpha = & 
\argmin_{\alpha^\prime\in\complexset}\smolE{\|\bmr-\alpha^\prime\bmh\|^2} 
= \textstyle \frac{\smolE{\bmh^\Herm \bmr}}{\smolE{\|\bmh\|^2}}.
\end{align}
To obtain a closed-form expression for $\alpha$, we assume $\bmh\sim\setC\setN(\bZero,\Eh\bI)$ as in 1-BEACHES
and use the derivation of $\smolE{\bmh^\Herm\bmr}$ in \fref{app:bussgang_gain},  which yields 
$\alpha = {2}/{\!\sqrt{\pi({\Eh}+{\No})}}$.


In \fref{eq:bussgang_decomposition}, the distortion $\bmd$ is not Gaussian. By transforming into beamspace domain and dividing the result by $\alpha$, we get
\begin{align} \label{eq:bussgang_decomposition_beamspace}
\textstyle \frac{1}{\alpha}\rhat = \frac{1}{\alpha}\bF \bmr =  \hhat + \frac{1}{\alpha}\dhat,
\end{align}
in which the distortion $ {\dhat}/{\alpha}$ is well-approximated by a  Gaussian, as each entry is a scaled and phase-shifted sum of all of the entries of $\bmd$. The distortion variance ${\Do}/{\alpha^2}$ is 
\begin{align}
\textstyle \frac{1}{B}\frac{1}{\alpha^2} \smolE{\|\dhat\|^2}
= \frac{1}{B}\frac{1}{\alpha^2}\smolE{\|\bmr\|^2\!-\alpha^2\|\bmh\|^2}\! = \frac{2}{\alpha^2}-E_h.
\end{align}
%
The model \fref{eq:bussgang_decomposition_beamspace}, enables us to apply BEACHES to $\rhat/{\alpha}$ in order to determine the denoising threshold $\tau^\star$ given $\rhat/\alpha$ and $\Do/\alpha^2$. Finally, $\alpha$-BEACHES computes $\hhatstar = \eta(\frac{\rhat}{\alpha},\tau^\star).$
