% !TEX root = 0-20SPAWC_SAND.tex
% DO NOT REMOVE THE ABOVE COMMENT!
\section{1-bit Quantized System Model}
\label{sec:systemmodel}

%
We consider a mmWave massive MU-MIMO uplink system in which $U$ single-antenna UEs transmit data to a $B$-antenna BS equipped with a uniform linear array (ULA). 
%
We assume that each of the $B$ radio-frequency (RF) chains at the BS contains a pair of 1-bit ADCs that separately quantize the in-phase and quadrature signals.
%
A widely-used, yet simplistic channel vector model for such systems is as follows \cite{tse05a}:
\begin{align} \label{eq:antenna_domain_channel}
\bmh =  \textstyle \sum_{\ell=1}^{L}{\!\kappa_\ell\bma(\Omega_\ell)}, \, \bma(\Omega)\! =\! [e^{j0\Omega}, e^{j1\Omega}, \ldots, e^{j(B-1)\Omega}]^\Tran\!\!.
\end{align}
%
Here, $L$ stands for the number of propagation paths arriving at the BS, $\kappa_\ell\in\complexset$ is the channel gain of the $\ell$th path, $\bma(\Omega_\ell)\in\complexset^B$ contains the relative phases between BS antennas, and $\Omega_\ell\in[0,2\pi)$ is determined by the $\ell$th path's incident angle.
%
We emphasize that our simulation results in \fref{sec:simulations} will use more realistic mmWave channel models. 

We consider orthogonal training-based channel estimation, where only one UE transmits a pilot at a time---a generalization to other training schemes is part of ongoing work. To model 1-bit ADCs, we define $Q(z) = \sign(\realpart{z}) + j \sign(\imagpart{z})$, which is applied element-wise to vectors. The 1-bit quantized channel vector for a given UE can be modeled as follows \cite{li17b}:
\begin{align} \label{eq:quantized_antenna_domain}
\bmr = & Q\!\left(\varrho\bmh+\bmn\right)\!,
\end{align}
where $\bmn\sim\CN(\boldsymbol0,\No\bI_B)$ models thermal noise. Without loss of generality, we assume  $\varrho=1$ for the rest of the paper. 

All of the above vectors are in the antenna domain, where each entry is associated with one of the $B$ BS antennas. 
%
By taking the discrete Fourier transform (DFT) across the antenna array, we can transform these vectors into the beamspace domain, where each entry corresponds to an incident angle. 
%
From \fref{eq:antenna_domain_channel} we see that $\bmh$ is a superposition of~$L$ complex sinusoids. Consequently, the beamspace domain representation $\hhat = \bF\bmh$, where~$\bF$ is the $B\times B$ unitary DFT matrix, will be  sparse assuming that $L\ll B$. In what follows, all beamspace domain quantities are designated with a $\hat{\text{hat}}$.

\fref{fig:sparse_channel} shows examples for LoS and non-LoS channel vectors in the beamspace domain without and with 1-bit quantization. 
%
Clearly, the unquantized beamspace vectors $\hhat$ exhibit sparsity; the 1-bit quantized beamspace vectors, which are obtained from $\rhat = \bF\bmr$, also exhibit sparsity but, in addition, are distorted by quantization artifacts.
%
We also observe that the quantization artifacts differ significantly between the LoS and non-LoS channels, which exhibit different levels of sparsity. 
%
In what follows, we develop algorithms that exploit beamspace sparsity to denoise 1-bit quantized channel vectors while adapting the denoising parameters to the instantaneous channel sparsity. 


\begin{figure}[tp]
\subfigure[LoS channel]{\includegraphics[width=0.495\columnwidth]{fig/mmMagic_los_beamspace.pdf}}
\subfigure[Non-LoS channel]{\includegraphics[width=0.495\columnwidth]{fig/mmMagic_nlos_beamspace.pdf}}
\vspace{-0.35cm}
\caption{Beamspace representation of an unquantized and a 1-bit quantized channel vector for a line-of-sight (LoS) and non-LoS scenario. The channel vectors are generated with the QuaDRiGa mmMAGIC UMi model \cite{jaeckel2019quadriga} at $60$\,GHz for a $256$ uniform-linear array (ULA) with $\lambda/2$ antenna spacing. The average energy has been normalized to $1$ and no noise is present.} 
\label{fig:sparse_channel}
\end{figure}

