% !TEX root = 0-20SPAWC_SAND.tex
% DO NOT REMOVE THE ABOVE COMMENT!

\refstepcounter{section} \label{app:bussgang_gain}
\section*{\fref{app:bussgang_gain}: Derivation of $\frac{1}{B}\E{\bmh^\Herm\bmr}$} 
Since $\bmn\sim\CN(\boldsymbol0,\No\bI_B)$ and $\bmh$ is assumed circularly symmetric, the imaginary part of $\E{\bmh^\Herm\bmr}$ is zero, and
\begin{align}
\textstyle\frac{1}{B}\E{\bmh^\Herm\bmr} = 
\frac{1}{B}\E{\realindex{\bmh}\realindex{\bmr}} + \frac{1}{B}\E{\imagindex{\bmh}\imagindex{\bmr}}\!.
\end{align}
By assuming $\bmh\sim\CN(\boldsymbol0,\Eh\bI_B)$, $\frac{1}{B} \E{\realindex{\bmh}\realindex{\bmr}}$ becomes
\begin{align}
\textstyle\frac{1}{B} & \textstyle\sum_{b=1}^{B}\E{\int_{-\infty}^{-\realindex{n_b}}\frac{-\realindex{h_b}}{\sqrt{\pi\Eh}}e^{-\frac{(\realindex{h_b})^2}{\Eh}}d\realindex{h_b}} \\
\textstyle + & \textstyle\frac{1}{B}\sum_{b=1}^{B}\E{\int_{-\realindex{n_b}}^{-\infty}\!\frac{\realindex{h_b}}{\sqrt{\pi\Eh}}e^{-\frac{(\realindex{h_b})^2}{\Eh}}d\realindex{h_b}}
\!=\! \textstyle\frac{\Eh}{\sqrt{\pi(\Eh+\No)}}. \notag 
\end{align}
Following the same procedure for the imaginary part, we get
\begin{align}
\textstyle\frac{1}{B}\E{\bmh^\Herm\bmr} = \textstyle\frac{2\Eh}{\sqrt{\pi({\Eh}+{\No})}}.
\end{align}
\vspace{-0.4cm}