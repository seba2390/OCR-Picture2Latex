\section*{Methods}
Molecules were extracted from the ChEMBL database~\cite{mendez2019chembl} (version 27). Conformers were generated using RDKit~\cite{rdkit} and GFN2-xTB~\cite{grimme2017robust, bannwarth2019gfn2, grimme2019exploration, bannwarth2020extended}. DFT ($\omega$B97X-D/def2-SVP) calculations were carried out via Psi4~\cite{smith2020psi4}. A similar approach was adopted in a previous study on transition-metal complexes.~\cite{balcells2020tmqm} An overview of the data processing pipeline is given in Figure~\ref{fig:pipeline}, while individual steps are described in more detail in the following subsections.

In chemical terminology, the term ``conformation'' refers to any arrangement of atoms in space, whereas ``conformer'' refers to a conformation that is a local minimum on the potential energy surface of the molecule.~\cite{moss1996basic} In the analyses that follow, the term ``conformation'' is loosely used to refer to both, unless explicitly mentioned otherwise.


\subsection*{Data extraction and SMILES processing}

Single-protein targets with assay information for at least $10$ compounds with unique internal identifiers were extracted from the ChEMBL database. Several activity and annotation filters were subsequently applied to these compounds (see ESI for a detailed query description). This procedure resulted in $685,917$ molecules with unique external identifiers (ChEMBL-IDs), represented by their Simplified Molecular Input Line Entry Specification (SMILES)~\cite{weininger1988smiles}. Molecules were neutralized, and salts and solvents were removed using the ChEMBL Structure Pipeline package~\cite{Bento2020, chembl_structure_pipeline_repo}. For compounds consisting of multiple separate fragments after this "washing" procedure, all except the one with the highest number of heavy atoms were discarded. Additionally, molecules containing fewer than $3$ or more than $100$ heavy atoms, as well as radical species and molecules with a net charge different from zero after the attempted neutralization, were removed. Atom types included in the QMugs dataset are hydrogen, carbon, nitrogen, oxygen, fluorine, phosphorus, sulfur, chloride, bromide, and iodine.


\subsection*{Conformer generation and optimization}

With the procedure described herein, a compromise between efficient molecular conformational search and practical computational expense considerations was sought. 

The RDKit~\cite{rdkit} implementation of the Experimental-Torsion Knowledge Distance Geometry (ETKDG) method~\cite{riniker2015better} was used to generate up to $100$ conformers for each molecule, with a maximum of $1000$ embedding attempts and an initial coordinate assignment using distance-matrix eigenvalues and default settings (\texttt{boxSizeMult=2.0}, \texttt{force-field tolerance=1e-3}) . Upon no successful conformer generation, it was re-attempted via random assignment of the starting coordinates. The resulting conformers were further minimized using the Merck molecular force field~\cite{tosco2014bringing} (MMFF94s) for a maximum of $1000$ iterations, with default settings (\texttt{nonBondedThresh=100.0}). The lowest-energy conformer (according to the selected force field) for each structure was then used as a starting point for meta-dynamics (MTD) simulations. Stereocenters that were previously undefined in the SMILES extraction procedure were assigned in this conformer generation process. 


For each generated conformer, an MTD simulation was performed with the xTB software package~\cite{grimme2019exploration} for a duration of $5$ ps with time steps of $1$ fs, at a temperature of $300$ K. The biasing root-mean-square deviation (RMSD) potential used for all MTD simulations is given by $E_{\mathrm{bias}}^{\mathrm{RMSD}} = \sum_{i=1}^{N} k_{i} \exp(\alpha \Delta^{2}_{i}$), where $N$ is the number of reference structures, $k_i$ the pushing strength, $\Delta_i$ the collective variable (\textit{i.e.}, the RMSD between structure $i$ and a reference structure), and $\alpha$ the width of the Gaussian potential used in the RMSD criterion. Simulations were carried out with $\alpha=$\SI{1.2}{\bohr}$^{-1}$ and $k_{i}=$\SI{0.2}{\milli\hartree} with snapshots taken every $50$ fs, resulting in $100$ conformations stored with their corresponding energies. To obtain conformationally diverse samples, these structures were subsequently clustered into three groups via the $k$-means \cite{lloyd1982least} algorithm, as implemented in the scikit-learn~\cite{scikit-learn} (version 0.23.1) Python package using the pairwise RMSD of the aligned structures as molecular features. The conformation with the lowest-energy value from each cluster was then selected for further processing. The three resulting conformations for each molecule were then optimized using the GFN2-xTB~\cite{grimme2017robust, bannwarth2019gfn2, grimme2019exploration, bannwarth2020extended} method using energy and gradient convergence criteria of $5\times 10^{-6}$ \si{\hartree} and $1\times 10^{-3}$ \si{\hartree} $\alpha^{-1}$, respectively, and the approximate normal coordinate rational function optimizer (ANCopt). Harmonic frequencies, entropies, enthalpies and heat capacities at $298.15$ K were extracted at the end of the geometry optimization process. Structures for which vibrational frequencies with imaginary wave numbers were obtained --- indicative of failure to reach energy minima --- were subjected to additional optimizations until no significant ones remained, up to a maximum of $100$ attempts. 


\subsection*{Quantum mechanical calculations}

Single-point electronic calculations were performed for the optimized geometries using the $\omega$B97X-D quantum functional and the def2-SVP basis set as implemented in the open-source quantum-chemistry software suite Psi4~\cite{smith2020psi4}. Single-point properties such as formation and orbital energies, dipole moments, rotational constants, partial charges, bond orders, valence numbers, as well as wave functions including $\alpha$ and $\beta$ DFT-density matrices, orbital matrices, and the atomic-orbital-to-symmetry-orbital transformer matrix were obtained. For practical reasons, $52$ structures whose DFT calculations required computational resources that exceeded empirically determined limits, or for which calculations were unsuccessful, were discarded (see ESI for details). 
