\newpage
\section{Independent terms of the Schrödinger equation}
The Born Oppenheimer approximation \cite{born1927quantentheorie} defines the molecular energy of the electronic Schrödinger equation as a sum of four independent terms, namely (i) nuclear repulsion energy $\hat{V}_{NN}$, (ii) exchange correlation energy $\hat{V}_{eN}$, (iii) kinetic electron energy also known as one electron energy $\hat{T}_{e}$, and (iv) electron repulsion energy also known as two electron energy $\hat{V}_{ee}$. The distribution of the four terms in QMugs are visualized in Figue \ref{fig:qm_se}.

\begin{figure}[H]
\centering
\includegraphics[width=\linewidth]{figures_si/qm_non_delta_props.pdf}
\caption{Terms of the molecular hamiltonian $\hat{H}$ calculated on the $\omega$B97X-D/def2-SVP level-of-theory fo moelcules in QMugs. (\textbf{A}) Nuclear repulsion energy $\hat{V}_{NN}$ in $E_{H}$. (\textbf{B}) Exchange correlation energy $\hat{V}_{eN}$ in $E_{H}$. (\textbf{C}) One electron energy $\hat{T}_{e}$ in $E_{H}$. (\textbf{D}) Two electron energy $\hat{V}_{ee}$ in $E_{H}$. 
}
\label{fig:qm_se}
\end{figure}

