\section*{Background \& Summary}

Machine learning methodologies are increasingly becoming well-established tools in many chemistry-related disciplines, such as drug discovery~\cite{gawehn2016deep}, material science~\cite{schmidt2019recent}, and physical chemistry~\cite{von2018quantum}. In recent years, significant progress has been made in quantum-based machine learning (QML) methods~\cite{von2020exploring}, which aim to accurately and computationally inexpensively predict the governing properties of atomistic systems, such as energies and forces~\cite{satorras2021n, schutt2021equivariant, huang2020quantum, christensen2020fchl, heinen2020quantum, heinen2020machine, christensen2019operators, faber2018alchemical}, dipole moments~\cite{balcells2020tmqm}, wave functions~\cite{unke2021se3equivariant, schutt2019unifying} and electron densities~\cite{grisafi2018transferable, fabrizio2019electron}. Despite the success and promise surrounding the applicability of such approaches, several challenges remain for QML. Arguably, one of the most important challenges is the increasing need for curated, comprehensive datasets.~\cite{balcells2020tmqm} While several options, such as the QM9~\cite{ramakrishnan2014quantum} or ANI-1~\cite{smith2017anid} sets have paved the way for the development of current-generation QML methods~\cite{satorras2021n, schutt2021equivariant, huang2020quantum, christensen2020fchl, smith2017ani, smith2020ani, nakata2020pubchemqc}, the computational cost entailed in their generation limits both the scope of the explored chemical space (\textit{e.g.}, molecule size, atom-type diversity), and prospective modeling applicability~\cite{balcells2020tmqm, glavatskikh2019dataset}. 

There has been a recent surge in interest in the delta-learning ($\Delta$-learning) of chemical properties, which aims to use a machine learning model to predict a physically relevant quantity, such as those generated by density-functional theory (DFT) by utilizing information extracted with a computationally cheaper method~\cite{qiao2020orbnet, smith2020ani} (\textit{e.g.}, semi-empirical approaches such as GFN2-xTB~\cite{grimme2017robust, bannwarth2019gfn2, pracht2019robust, grimme2019exploration, bannwarth2020extended} and PM6~\cite{rezac2009semiempirical}). Datasets that enable this type of learning are scarce and could promote the development of accurate models at potentially a fraction of the computational cost of more precise alternatives~\cite{folmsbee2021assessing}. Furthermore, datasets that provide three-dimensional conformational data, for a wide variety of chemical space, at levels of theory higher than classical force fields~\cite{bolton2011pubchem3d, axelrod2020geom}, could boost the performance of machine learning methods in predicting properties from ensembles as well as generative models of conformations. Relevant examples include the PubChemQC-PM6~\cite{nakata2020pubchemqc} and GEOM~\cite{axelrod2020geom} datasets, which include molecules with properties computed using different semi-empirical levels of theory. Finally, there is a clear potential to open up new lines of research by combining biological annotations (\textit{e.g.}, from molecular databases such as ChEMBL~\cite{mendez2019chembl}), and additional QM-derived physical information.

This work introduces QMugs (\textbf{Q}uantum-\textbf{M}echanical Properties of Dr\textbf{ug}-like Molecule\textbf{s}), a data collection of over $665$k curated molecular structures extracted from the ChEMBL database, with accompanying computed quantum mechanical properties. Different levels of theory were combined in these calculations. Per compound, three conformers were generated, and their geometries were optimized using the semi-empirical GFN2-xTB method~\cite{grimme2017robust, bannwarth2019gfn2, pracht2019robust, grimme2019exploration, bannwarth2020extended}, whereas a comprehensive array of quantum properties was computed at the DFT level of theory using the $\omega$B97X-D functional~\cite{chai2008long} and the def2-SVP Karlsruhe basis set~\cite{weigend2005balanced}. The data collection presented herein is put in the context of other sets that also feature DFT-level properties. A descriptive evaluation against QM9~\cite{ramakrishnan2014quantum}, PubchemQC~\cite{nakata2017pubchemqc}, and the ANI-1~\cite{smith2017anid} datasets is provided in Figure~\ref{fig:rod_disk_sphere_venn}B, as well as in Table~\ref{tbl:desc}. As previously reported for the ChEMBL database~\cite{meyers2016origins}, most of the considered drug-like molecules in this study fall within the rod-disk axis in the principal moments of inertia plot~\cite{sauer2003molecular} (Figure~\ref{fig:rod_disk_sphere_venn}A). Furthermore, the vast majority of the included compounds ($\sim 641$k, $96.3\%$) were previously unreported in other DFT data collections, while also providing equivalent information at additional levels of theory, namely GFN2-xTB. With an average of $30.6$ and a maximum of $100$ heavy atoms per compound (Table~\ref{tbl:desc} \& Figure~\ref{fig:molecule_props}), QMugs also features molecular samples that are considerably larger than those provided by other datasets. To the best of our knowledge, this work is the first to provide a large and diverse dataset of quantum mechanical wave functions represented as local bases of atomic orbitals (\textit{i.e.}, DFT density and orbital matrices). Single-point properties as well as wave functions were computed with the Psi4 software suite~\cite{smith2020psi4} for all the conformers ($\sim 2.0$M) present in the database, totaling over $7$ terabytes of supplied quantum mechanical data.

Overall, the utility of the presented dataset is fourfold: (i) it will provide researchers with the largest-to-date dataset to either directly predict the quantum chemical properties, or learn a property mapping between two popular quantum mechanical levels of theory (\textit{i.e.}, GFN2-xTB and $\omega$B97X-D/def2-SVP); (ii) it will facilitate the development of novel machine-learning methodologies for the generation of molecular conformations and molecular property predictions via their ensembles; (iii) it will enable the development of novel deep learning frameworks for the prediction of the quantum mechanical wave function in a local basis of atomic orbitals from which all ground-state properties as well as electron densities can be derived; and (iv) it will enable research towards the exploration of QML methods quantum featurization in the context of pharmacologically relevant, annotated biological data.

