\section*{Technical Validation}


\subsection*{Optimized geometry sanity checks}
Four consecutive geometry checks were performed to filter out structures for which the geometry optimization procedure converged to unrealistic conformations. To determine suitable thresholds for removing a structure from our dataset, the generated geometries were compared to experimental reference values and to DFT-optimized geometries extracted from the PubChemQC dataset~\cite{nakata2017pubchemqc}. Specifically, we investigated (i) the deviation of bond lengths from experimental reference values, (ii) isomorphism between the initial molecular graphs and those obtained after geometry optimization, (iii) linearity of triple bonds, and (iv) planarity of aromatic rings. Structures were removed from the dataset if they failed any test of these tests. In total, $10,986$ ($0.55$\%) conformations were discarded from the dataset. Each test is briefly described in the following subsections, with further technical details reported in the ESI.

\subsubsection*{Deviation of bond lengths from experimental reference values}
Bond lengths in the optimized structures were compared to average experimental reference values for bonds of the same bond type (single, double, triple, or aromatic) and between the same atoms. Reference values were obtained from the Computational Chemistry Comparison and Benchmark DataBase (CCCBDB)~\cite{nist_database}, and the largest absolute bond-length deviation from reference values was recorded per molecule. Bonds for which no reference value was available ($0.75\%$) were omitted. The same analysis was carried out for molecules from the PubChemQC dataset containing the same atom types as QMugs, in order to obtain a comparable set with respect to the present atom types. The PubChemQC set ($3,834,382$ conformations with reference bond lengths) showed a deviation of $0.06$~$\pm$~\SI{0.04}{\angstrom} (median $\pm$ $1$ standard deviation), whereas the QMugs dataset ($2,004,003$ conformations with reference bond lengths) showed a deviation of $0.07$~$\pm$~\SI{0.03}{\angstrom}. Based on the observed distribution of bond-length deviations from experimental reference values (Figure~S1 in ESI) and manual investigation of example structures, \SI{0.2}{\angstrom} was determined to be a suitable threshold for a conformation to be removed from the dataset, which included $6,131$ ($0.31$\%) examples.

\subsubsection*{Molecular graph isomorphism}
It was investigated whether atom connectivity could be reconstructed after removing bond information from the generated SDFs. To this end, molecular graphs constructed exclusively from atom positions and types were compared to those obtained using the original atom connectivity (see ESI for details). $1,568$ ($0.08$\%) conformations for which the resulting molecular graphs were non-isomorphic failed this test.

\subsubsection*{Deviation of triple bonds from linear geometry}
The deviation of triple bonds from their ideal linear geometry was examined. In this investigation, ring triple bonds were not considered owing to routinely-occurring deviations from linear geometry in systems with high ring strain~\cite{bach2009ring}. The largest deviation from a \SI{180}{\degree} (linear) bond angle was recorded for each molecule containing at least one non-ring triple bond. The same analysis was performed on the PubChemQC dataset~\cite{nakata2017pubchemqc}. Triple-bond-containing molecules from PubchemQC and QMugs ($273,320$ and $165,101$ samples, respectively) show deviations of $1.38$~$\pm$~\SI{1.46}{\degree} (median $\pm$ $1$ standard deviation), and $1.46$~$\pm$~\SI{2.13}{\degree}, respectively. Based on the observed distribution of triple bond angles from a linear geometry (Figure~S2 in ESI) and manual inspection of structures, a \SI{10}{\degree} deviation was identified as a suitable threshold. $1,147$ ($0.06$\%) conformations failed this test.

\subsubsection*{Deviation of aromatic rings from planar geometry}
The planarity of carbon-containing aromatic rings was also investigated. For each molecule containing aromatic carbon atoms, the largest dihedral angle between the two planes spanned by each aromatic carbon atom and its three neighbors was recorded (see ESI for details). The same analysis was performed on the PubChemQC dataset~\cite{nakata2017pubchemqc}. Molecules from PubchemQC and QMugs ($2,391,589$ and $1,950,929$ conformations with aromatic carbons, respectively) showed median dihedral angles (~$\pm$~$1$ standard deviation) of $1.70$~$\pm$~\SI{1.85}{\degree} and $2.99$~$\pm$~\SI{2.20}{\degree}, respectively. Based on the observed distribution of dihedral angles from planar geometries (Figure~S3 in ESI) and manual inspection of structures, $2,769$ ($0.14$\%) conformations with aromatic carbon dihedral angles above \SI{15}{\degree} were discarded.



\subsection*{Further geometrical assessment}

The changes in the molecular geometries along the applied pipeline were examined in order to evaluate the effects of the applied steps. Figure~\ref{fig:geometry_validation}A shows the mean pairwise RMSD of atom positions between the conformations of each molecule at different steps along the pipeline. Conformations sampled during MTD simulations show a mean pairwise RMSD of $2.40$ $\pm$ $0.52$ \si{\angstrom} (median $\pm$ $1$ standard deviation). The $k$-means clustering procedure accomplishes the envisaged task of sampling conformations with higher geometric diversity ($2.67$ $\pm$ $0.74$ \si{\angstrom}). During the geometry optimization process, conformational diversity decreases ($2.48$ $\pm$ $0.86$ \si{\angstrom}). Unsurprisingly, for some molecules featuring rigid structures, conformations tend to converge toward the same energy minimum ($0.09$ \% of molecules show a mean pairwise RMSD $<$~\SI{0.01}{\angstrom} between their optimized conformers).


The degree to which the molecular geometries changed during the final optimization step was further analyzed. Molecules with initially more diverse conformations (higher mean pairwise RMSD of pre-optimized conformations) were shown to undergo a greater chance in atom positions (mean RMSD of pre- vs. post-optimized conformations) during optimization with the GFN2-xTB method (Figure~\ref{fig:geometry_validation}B). The observed heteroscedastic behavior of these two properties indicates that while the mean RMSD of pre- vs. post-optimized conformations tends to increase with higher mean pairwise RMSD of pre-optimized conformations, its variance also increases. 


Finally, the suitability of GFN2-xTB as a lower-cost surrogate for DFT-level geometry optimization (Figure~\ref{fig:geometry_validation}C) was confirmed. $500$ randomly-chosen structures prior to semi-empirical geometry optimization from the QMugs dataset were further subjected to DFT-level geometry optimization ($\omega$B97X-D/def2-SVP), discarding structures that could not be converged in 100 iterations or with the computational resources described in the ESI. The RMSDs between the structures independently optimized at both levels of theory were then measured. The pairs of structures show RMSDs of $0.47$ $\pm$ $0.63$ \si{\angstrom} (median $\pm$ $1$ standard deviation), indicating that the chosen semi-empirical method obtains similar geometries to those obtained with more expensive first-principle calculations. Large RMSDs in some example pairs (Figure~\ref{fig:geometry_validation}C) could be interpreted as indicative of convergence to distinct local minima. 

For $2,067$ molecules, their individual conformations have different SMILES describing two different \textit{(E)/(Z)} isomers. Those structures are either $\alpha$-$\beta$-unsaturated ketones, $\alpha$-$\beta$-unsaturated nitriles, imine, or azo compounds, for which isomerization might be plausible~\cite{goulet2017electrocatalytic, roca2014dft}. In part due to the applied washing procedure, $17,176$ molecules can be represented with a SMILES string that is shared with at least one other ChEMBL-ID. 


\subsection*{Validation of single-point properties}

To validate the general agreement between the two methods employed in this work, the correlation between a series of single-point properties computed on both levels of theory was analyzed. Both global, molecular (Figure~\ref{fig:delta_molecular_props}) and local, atomic/bond properties (Figures~\ref{fig:partial}~\&~\ref{fig:bond}) were considered. All single-point molecular properties showed a high degree of correlation. Formation energies $E_\mathrm{Form}$ (Figure~\ref{fig:delta_molecular_props}A), which were obtained by subtracting atomic energies $E_{\mathrm{Atom}}$ from total internal energies $U_{RT}$, show a Pearson correlation coefficient (PCC) of $0.998$. Dipole moments $\mu$ and rotational constants $A$ (excl. $22$ small structures with very high rotational constants; Figure~\ref{fig:delta_molecular_props}B,C) display PCCs of $0.969$ and $0.999$, respectively. Orbital energies, namely the energies for highest occupied (HOMO)  $E_{\mathrm{HOMO}}$ and lowest unoccupied molecular orbitals (LUMO) $E_{\mathrm{LUMO}}$ and HOMO-LUMO gaps $E_{\mathrm{Gap}}$ show PCCs of $0.769$, $0.924$ and $0.830$, respectively (Figures~\ref{fig:delta_molecular_props}~D,~E~\&~F). The observed PCCs for all six single-point molecular properties indicate good agreement between the two methods. Atom-type-specific partial charges for the 10 atom-types in QMugs (Figure~\ref{fig:partial}, ESI Table~1) as well as the 15 most abundant covalent bonds orders (Figure~\ref{fig:bond}, ESI Table~2) also show high correlations between the two methods used herein. Regarding partial charges, $7$ out of the $10$ atom types considered in QMugs were observed to have PCCs $>0.8$, with the remaining carbon, nitrogen, and oxygen atom-types resulting in lower PCCs of $0.574$, $0.124$, and $0.274$, respectively. Regarding bond orders, $10$ out of the $15$ show PCCs $>0.9$ and $14$ out of $15$ display PCCs $>0.75$ (see ESI Table~2 for additional metrics). Notably only carbon-fluorine bonds display a larger discrepancy between both levels of theory, with an observed PCC of $0.153$. The observed correlations in both molecular and atomic single-point properties between GFN2-xTB and $\omega$B97X-D/def2-SVP confirm the suitability of the former method as a computationally affordable starting point for $\Delta$-learning of DFT-level properties.
