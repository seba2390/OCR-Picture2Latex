\section{Optimized geometry sanity checks}
We performed four consecutive geometry checks (Sections~\ref{sec:bond_length_check} -- \ref{sec:planar_rings}) to filter out structures for which geometry optimization had converged to unrealistic conformations. Structures failing any of these tests were removed from the dataset. Note that stated numbers of investigated molecules for the QMugs dataset are larger than the size of the final dataset, as only molecules that passed all geometry checks were included in the final dataset. Note that the reported numbers of conformations failing the tests described in the following subsections do not add up to $10,986$ (the number of conformations removed from the dataset) since some conformations failed in multiple tests.

\subsection{Deviation of bond lengths from experimental reference values}
\label{sec:bond_length_check}

We searched for bond-length outliers in the optimized structures. As reference values, average bond-length values between atom types from ``Standard Reference Database, Computational Chemistry Comparison and Benchmark DataBase''~\cite{nist_database} were extracted. For each bond in the optimized structures, the absolute deviation from its respective reference value was calculated. For each conformation, the largest absolute deviation was recorded. If no reference value for a specific bond type between two atom types was found, the bond was not considered for further analysis. This was the case for $876,113$ ($0.75$\%) of all bonds in the investigated molecules from the QMugs dataset. 

The same procedure was carried out on molecules from the PubChemQC dataset~\cite{nakata2017pubchemqc} containing the same atom types as the QMugs dataset (Figure~\ref{fig:bond_length_histograms}). The restriction of atom types was made in order to rule out effects of atom types on bond length deviations. This restriction removed $65,316$ ($1.64$\%) molecules from the assessment. Further $82,708$ ($2.08$\%) molecules which could not be successfully read with RDKit~\cite{rdkit} were ignored. Based on the distribution of bond-length deviations from experimental reference values, we discarded molecules above the \SI{0.2}{\angstrom} threshold. Based on manual inspection of structures at different bond length deviation values, this threshold appeared as a suitable, conservative threshold. $6,131$ ($0.31$\%) conformations did not pass this test.

\begin{figure}[H]
\centering
\includegraphics[width=\textwidth]{figures_si/bond_length_checks_qmugs_final.pdf}
\caption{
\textit{(Top)} Distribution of largest absolute bond-length deviation from experimental reference values per conformation (histogram bin size $5\times 10^{-4}$\si{\angstrom}). PubChemQC ($3,834,382$ conformations with reference bond lengths) shows a deviation of $0.0580$~$\pm$~\SI{0.0419}{\angstrom} (median $\pm$ $1$ standard deviation), whereas QMugs ($2,004,003$ conformations with reference bond lengths) exhibits a deviation of $0.0687$~$\pm$~\SI{0.0317}{\angstrom}. $8,492$ ($0.22$\%) and $926$ ($0.05$\%) conformations in the PubChemQC and QMugs sets, respectively, have higher deviations than \SI{0.30}{\angstrom} and are not shown. \textit{(Bottom)} Distribution of the total number of atoms per conformation in both datasets (histogram bin size $1$), showing that molecules in the QMugs sample are significantly larger on average. $765$ ($0.04$\%) conformations in the QMugs dataset have more than $200$ atoms and are not shown. Potentially-arising greater steric clashes in larger molecules may contribute to the slightly higher bond length deviations in the QMugs dataset, compared to the PubChemQC dataset.} 
\label{fig:bond_length_histograms}
\end{figure}
\noindent


\subsection{Molecular graph isomorphism}
We investigated whether heavy-atom connectivity can be reconstructed when removing all bond information from the generated structure-data files (SDF) in the database. SDFs were converted to the .xyz file format (which does not contain bond information) using OpenBabel~\cite{o2011open, openbabel} (version 3.1.1). We then attempted to perform a conversion from .xyz to InChI~\cite{heller2015inchi}, or to SMILES~\cite{weininger1988smiles} upon failure of the former. If both were unsuccessful, we considered the graph isomorphism check as failed. If either succeeded, however, we compared the molecular graph of generated molecular strings (as read by RDKit) to the one originally obtained from the SDF (which includes bond information). The isomorphism of the molecular graphs was then checked using the NetworkX~\cite{SciPyProceedings_11} Python package (version 2.5), considering nodes (representing atoms) in each graph labelled with their respective atom types. We did not use bond types in the previous comparison due to observed high false negative rates related to mislabelled bonds in nitro and other functional groups with multiple resonance structures, among further reasons. $1,568$ ($0.08$\%) conformations failed this test. 

\subsection{Deviation of triple bonds from linear geometry}
For each molecule containing a triple bond which is not part of a ring, the deviation of bond angles $\gamma$ from the ideal 180\textdegree ~(linear) geometry, denoted as $\Delta \gamma$, was assessed. For each non-terminal atom in a non-ring triple bond, the angle between the bonds to its two neighbors was computed. The largest deviation $\Delta \gamma$ from a perfectly linear triple bond was recorded per molecule. We limit this investigation to triple bonds outside a ring as triple bonds in rings can show substantial deviations from a linear geometry due to high ring strain (\textit{e.g.,} cyclooctyne, the smallest stable, cyclic hydrocarbon accommodating a triple bond, deviates by $\Delta \gamma =$ \SI{17}{\degree} from a linear geometry~\cite{bach2009ring}). For reference, we performed the same study on all molecules from the PubChemQC dataset~\cite{nakata2017pubchemqc} which include at least one non-ring triple bond (Figure~\ref{fig:linear_triple_bonds}), skipping molecules with could not be read with RDKit. From visual inspection of the distribution of triple bond angle deviations and manual inspection of example structures at different deviation levels, we decided to discard structures with a triple bond angle deviation of $\Delta \gamma >$ \SI{10}{\degree}. $1,147$ ($0.06$\%) conformations failed this test. Structures for which GFN2-xTB~\cite{grimme2017robust, bannwarth2019gfn2, grimme2019exploration, bannwarth2020extended} still indicated significant negative wavenumbers after 100 iterations ($17,502$ conformations, $0.88$\%) are denoted in the \texttt{summary.csv} file.

\begin{figure}[H]
\centering
\includegraphics[width=\textwidth]{figures_si/triple_bond_angles_qmugs_final.pdf}
\caption{Distribution of triple bond angle deviations from an ideal \SI{180}{\degree} angle (histogram bin size \SI{0.05}{\degree}). Triple bond-containing conformations from PubchemQC ($273,320$ conformations) and QMugs ($165,101$ conformations) show a deviation of $1.38$~$\pm$~\SI{1.46}{\degree} (median $\pm$ $1$ standard deviation), and $1.46$~$\pm$~\SI{2.13}{\degree}, respectively. $104$ conformations ($0.04$\%) in the PubChemQC and $179$ molecules ($0.11$\%) in the QMugs sample with higher deviations than \SI{14}{\degree} are not shown.} 
\label{fig:linear_triple_bonds}
\end{figure}
\noindent

\subsection{Deviation of aromatic rings from planar geometry}
\label{sec:planar_rings}
We furthermore investigated the planarity of carbon-containing aromatic rings. To do so, we assessed the dihedral angle between the two planes spanned by each aromatic carbon atom and its three neighbors. Angles greater than $90$\textdegree ~were corrected to 180\textdegree-<angle> to remove directional dependency. Calculations were performed for all order permutations of the aromatic carbon atom and its three neighbors. The largest dihedral angle (and hence the largest deviation from a perfectly planar aromatic ring) was recorded per conformation. We performed the same study on the molecules from the PubChemQC dataset~\cite{nakata2017pubchemqc} described in Section~\ref{sec:bond_length_check} (Figure~\ref{fig:carbon_C_angle}). From visual inspection of the distribution of aromatic carbon dihedral angle deviations
and manual inspection of example structures at different deviation levels, we decided to discard structures with an aromatic carbon dihedral angle deviation of \SI{15}{\degree} or more. $2,769$ ($0.14$\%) conformations failed this test.

\begin{figure}[H]
\centering
\includegraphics[width=\textwidth]{figures_si/arom_C_angles_qmugs_final.pdf}
\caption{Distribution of dihedral angle around aromatic carbons (histogram bin size \SI{0.05}{\degree}). Molecules with aromatic carbons from PubchemQC ($2,391,589$ conformations) and QMugs ($1,950,929$ conformations) show a deviation of $1.70$~$\pm$~\SI{1.85}{\degree} (median $\pm$ $1$ standard deviation) and $2.99$~$\pm$~\SI{2.20}{\degree}, respectively. $1050$ ($0.04$\%) molecules in the PubChemQC dataset and $564$ ($0.03$\%) molecules in the QMugs dataset with deviations greater than \SI{20}{\degree} are not shown.} 
\label{fig:carbon_C_angle}
\end{figure}
\noindent


