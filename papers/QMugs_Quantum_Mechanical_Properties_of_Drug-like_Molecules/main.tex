\documentclass[fleqn,10pt]{wlscirep}
\usepackage[utf8]{inputenc}
\usepackage[T1]{fontenc}
\usepackage{siunitx}
\newcolumntype{M}[1]{>{\raggedright\arraybackslash}p{#1cm}}
\newcommand{\mD}[2]{\multicolumn{1}{M{#1}}{#2}}
\usepackage{pifont}
\newcommand{\cmark}{\ding{51}}%
\newcommand{\xmark}{\ding{55}}%

\usepackage{array}
\newcolumntype{P}[1]{>{\centering\arraybackslash}p{#1}}


%\We recommend that you encode all DOIs in your bibtex database as full URLs, e.g. https://doi.org/10.1007/s12110-009-9068-2 and then include the following command to remove the default DOI prefix.
%\newcommand{\doiprefix}{}

\title{QMugs: Quantum Mechanical Properties of Drug-like Molecules}

\author[1,$\dag$]{Clemens Isert}
\author[1,$\dag$]{Kenneth Atz}
\author[1,2,*]{José Jiménez-Luna}
\author[1,3,*]{Gisbert Schneider}
\affil[1]{Department of Chemistry and Applied Biosciences, RETHINK, ETH Zurich, 8093 Zurich, Switzerland.}
\affil[2]{Department of Medicinal Chemistry, Boehringer Ingelheim Pharma GmbH \& Co. KG, Birkendorfer Straße 65, 88397 Biberach an der Riss, Germany.}
\affil[3]{ETH Singapore SEC Ltd, 1 CREATE Way, $\#$06-01 CREATE Tower, Singapore 138602, Singapore.}

\affil[*]{corresponding authors: Gisbert Schneider (gisbert@ethz.ch), José Jiménez-Luna (jose.jimenez@rethink.ethz.ch)}
\affil[$\dag$]{these authors contributed equally to this work}

\begin{abstract}


%\href{https://v1.overleaf.com/latex/templates/template-for-data-descriptor-submissions-to-scientific-data/ygdzkjcqzmbp.pdf}{Original template for Sci. Data as reference}\\

Machine learning approaches in drug discovery, as well as in other areas of the chemical sciences, benefit from curated datasets of physical molecular properties. However, there is a lack of sufficiently large data collections that include first-principle quantum chemical information on bioactive molecules, such as single-point electronic properties, quantum mechanical wave functions and density-functional theory (DFT) matrices. The open-access QMugs (Quantum-Mechanical Properties of Drug-like Molecules) dataset fills this void. The QMugs collection comprises quantum mechanical properties of more than 665k biologically and pharmacologically relevant molecules extracted from the ChEMBL database, totaling $\sim$2M conformers. QMugs contains optimized molecular geometries and thermodynamic data obtained via the semi-empirical method GFN2-xTB. Atomic and molecular properties (\textit{e.g.}, partial charges, energies, and rotational constants) are provided on both the GFN2-xTB and on the DFT ($\omega$B97X-D/def2-SVP) levels of theory. QMugs also comprises the respective quantum mechanical wave functions, including DFT density and orbital matrices, totaling over 7 terabytes of uncompressed data. This dataset is intended to facilitate the development of models that learn from molecular data on different levels of theory while also providing insight into the corresponding relationships between molecular structure and biological activity.

\end{abstract}
\begin{document}

\flushbottom
\maketitle
\thispagestyle{empty}
\section*{Background \& Summary}

Machine learning methodologies are increasingly becoming well-established tools in many chemistry-related disciplines, such as drug discovery~\cite{gawehn2016deep}, material science~\cite{schmidt2019recent}, and physical chemistry~\cite{von2018quantum}. In recent years, significant progress has been made in quantum-based machine learning (QML) methods~\cite{von2020exploring}, which aim to accurately and computationally inexpensively predict the governing properties of atomistic systems, such as energies and forces~\cite{satorras2021n, schutt2021equivariant, huang2020quantum, christensen2020fchl, heinen2020quantum, heinen2020machine, christensen2019operators, faber2018alchemical}, dipole moments~\cite{balcells2020tmqm}, wave functions~\cite{unke2021se3equivariant, schutt2019unifying} and electron densities~\cite{grisafi2018transferable, fabrizio2019electron}. Despite the success and promise surrounding the applicability of such approaches, several challenges remain for QML. Arguably, one of the most important challenges is the increasing need for curated, comprehensive datasets.~\cite{balcells2020tmqm} While several options, such as the QM9~\cite{ramakrishnan2014quantum} or ANI-1~\cite{smith2017anid} sets have paved the way for the development of current-generation QML methods~\cite{satorras2021n, schutt2021equivariant, huang2020quantum, christensen2020fchl, smith2017ani, smith2020ani, nakata2020pubchemqc}, the computational cost entailed in their generation limits both the scope of the explored chemical space (\textit{e.g.}, molecule size, atom-type diversity), and prospective modeling applicability~\cite{balcells2020tmqm, glavatskikh2019dataset}. 

There has been a recent surge in interest in the delta-learning ($\Delta$-learning) of chemical properties, which aims to use a machine learning model to predict a physically relevant quantity, such as those generated by density-functional theory (DFT) by utilizing information extracted with a computationally cheaper method~\cite{qiao2020orbnet, smith2020ani} (\textit{e.g.}, semi-empirical approaches such as GFN2-xTB~\cite{grimme2017robust, bannwarth2019gfn2, pracht2019robust, grimme2019exploration, bannwarth2020extended} and PM6~\cite{rezac2009semiempirical}). Datasets that enable this type of learning are scarce and could promote the development of accurate models at potentially a fraction of the computational cost of more precise alternatives~\cite{folmsbee2021assessing}. Furthermore, datasets that provide three-dimensional conformational data, for a wide variety of chemical space, at levels of theory higher than classical force fields~\cite{bolton2011pubchem3d, axelrod2020geom}, could boost the performance of machine learning methods in predicting properties from ensembles as well as generative models of conformations. Relevant examples include the PubChemQC-PM6~\cite{nakata2020pubchemqc} and GEOM~\cite{axelrod2020geom} datasets, which include molecules with properties computed using different semi-empirical levels of theory. Finally, there is a clear potential to open up new lines of research by combining biological annotations (\textit{e.g.}, from molecular databases such as ChEMBL~\cite{mendez2019chembl}), and additional QM-derived physical information.

This work introduces QMugs (\textbf{Q}uantum-\textbf{M}echanical Properties of Dr\textbf{ug}-like Molecule\textbf{s}), a data collection of over $665$k curated molecular structures extracted from the ChEMBL database, with accompanying computed quantum mechanical properties. Different levels of theory were combined in these calculations. Per compound, three conformers were generated, and their geometries were optimized using the semi-empirical GFN2-xTB method~\cite{grimme2017robust, bannwarth2019gfn2, pracht2019robust, grimme2019exploration, bannwarth2020extended}, whereas a comprehensive array of quantum properties was computed at the DFT level of theory using the $\omega$B97X-D functional~\cite{chai2008long} and the def2-SVP Karlsruhe basis set~\cite{weigend2005balanced}. The data collection presented herein is put in the context of other sets that also feature DFT-level properties. A descriptive evaluation against QM9~\cite{ramakrishnan2014quantum}, PubchemQC~\cite{nakata2017pubchemqc}, and the ANI-1~\cite{smith2017anid} datasets is provided in Figure~\ref{fig:rod_disk_sphere_venn}B, as well as in Table~\ref{tbl:desc}. As previously reported for the ChEMBL database~\cite{meyers2016origins}, most of the considered drug-like molecules in this study fall within the rod-disk axis in the principal moments of inertia plot~\cite{sauer2003molecular} (Figure~\ref{fig:rod_disk_sphere_venn}A). Furthermore, the vast majority of the included compounds ($\sim 641$k, $96.3\%$) were previously unreported in other DFT data collections, while also providing equivalent information at additional levels of theory, namely GFN2-xTB. With an average of $30.6$ and a maximum of $100$ heavy atoms per compound (Table~\ref{tbl:desc} \& Figure~\ref{fig:molecule_props}), QMugs also features molecular samples that are considerably larger than those provided by other datasets. To the best of our knowledge, this work is the first to provide a large and diverse dataset of quantum mechanical wave functions represented as local bases of atomic orbitals (\textit{i.e.}, DFT density and orbital matrices). Single-point properties as well as wave functions were computed with the Psi4 software suite~\cite{smith2020psi4} for all the conformers ($\sim 2.0$M) present in the database, totaling over $7$ terabytes of supplied quantum mechanical data.

Overall, the utility of the presented dataset is fourfold: (i) it will provide researchers with the largest-to-date dataset to either directly predict the quantum chemical properties, or learn a property mapping between two popular quantum mechanical levels of theory (\textit{i.e.}, GFN2-xTB and $\omega$B97X-D/def2-SVP); (ii) it will facilitate the development of novel machine-learning methodologies for the generation of molecular conformations and molecular property predictions via their ensembles; (iii) it will enable the development of novel deep learning frameworks for the prediction of the quantum mechanical wave function in a local basis of atomic orbitals from which all ground-state properties as well as electron densities can be derived; and (iv) it will enable research towards the exploration of QML methods quantum featurization in the context of pharmacologically relevant, annotated biological data.


\section{SYSTEM OVERVIEW}
\begin{figure}
\centering

\def\picScale{0.08}    % define variable for scaling all pictures evenly
\def\colWidth{0.5\linewidth}

\begin{tikzpicture}
\matrix [row sep=0.25cm, column sep=0cm, style={align=center}] (my matrix) at (0,0) %(2,1)
{
\node[style={anchor=center}] (FREEhand) {\includegraphics[width=0.85\linewidth]{figures/FREEhand.pdf}}; %\fill[blue] (0,0) circle (2pt);
\\
\node[style={anchor=center}] (rigid_v_soft) {\includegraphics[width=0.75\linewidth]{figures/FREE_vs_rigid-v8.pdf}}; %\fill[blue] (0,0) circle (2pt);
\\
};
\node[above] (FREEhand) at ($ (FREEhand.south west)  !0.05! (FREEhand.south east) + (0, 0.1)$) {(a)};
\node[below] (a) at ($ (rigid_v_soft.south west) !0.20! (rigid_v_soft.south east) $) {(b)};
\node[below] (b) at ($ (rigid_v_soft.south west) !0.75! (rigid_v_soft.south east) $) {(c)};
\end{tikzpicture}


% \begin{tikzpicture} %[every node/.style={draw=black}]
% % \draw[help lines] (0,0) grid (4,2);
% \matrix [row sep=0cm, column sep=0cm, style={align=center}] (my matrix) at (0,0) %(2,1)
% {
% \node[style={anchor=center}] {\includegraphics[width=\colWidth]{figures/photos/labFREEs3.jpg}}; %\fill[blue] (0,0) circle (2pt)
% &
% \node[style={anchor=center}] {\includegraphics[width=\colWidth, height=160pt]{figures/stewartRender.png}}; %\fill[blue] (0,0) circle (2pt);
% \\
% };

% %\node[style={anchor=center}] at (0,-5) (FREEstate) {\includegraphics[width=0.7\linewidth]{figures/FREEstate_noLabels2.pdf}};

% \end{tikzpicture}

\caption{\revcomment{2.3}{(a) A fiber-reinforced elastomerc enclosure (FREE) is a soft fluid-driven actuator composed of an elastomer tube with fibers wound around it to impose specific deformations under an increase in volume, such as extension and torsion. (b) A linear actuator and motor combined in \emph{series} has the ability to generate 2 dimensional forces at the end effector (shown in red), but is constrained to motions only in the directions of these forces. (b) Three FREEs combined in \emph{parallel} can generate the same 2 dimensional forces at the end effector (shown in red), without imposing kinematic constraints that prohibit motion in other directions (shown in blue).}}

% \caption{A fiber-reinforced elastomeric enclosure (FREE) (top) is a soft fluid-driven actuator composed of an elastomer tube with fibers wound around it to impose deformation in specific directions upon pressurization, such as extension and torsion. \revcomment{2.3}{In this paper we explore the potential of combining multiple FREEs in parallel to generate fully controllable multi-dimensional spacial forces}, such as in a parallel arrangement around a flexible spine element (bottom-left), or a Stewart Platform arrangement (bottom-right).}

\label{fig:overview}
\end{figure}


We now give an overview of our learning framework as illustrated in Figure~\ref{fig:overview}. Our framework splits athletic jumps into two phases: a run-up phase and a jump phase. The {\em take-off state} marks the transition between these two phases, and consists of a time instant midway through the last support phase before becoming airborne. The take-off state is key to our exploration strategy, as it is a strong determinant of the resulting jump strategy. We characterize the take-off state by a feature vector that captures key aspects of the state, such as the net angular velocity and body orientation. This defines a low-dimensional take-off feature space that we can sample in order to explore and evaluate a variety of motion strategies. While random sampling of take-off state features is straightforward, it is computationally impractical as evaluating one sample involves an expensive DRL learning process that takes hours even on modern machines. Therefore, we introduce a sample-efficient Bayesian Diversity Search (BDS) algorithm as a key part of our Stage~1 optimization process.

Given a specific sampled take-off state, we then need to produce an optimized run-up controller and a jump controller that result in the best possible corresponding jumps. This process has several steps. We first train a {\em }run-up controller, using deep reinforcement learning, that imitates a single generic run-up motion capture clip while also targeting the desired take-off state. For simplicity, the run-up controller and its training are not shown in Figure~\ref{fig:overview}. These are discussed in Section~\ref{sec:Experiments-Runup}. The main challenge lies with the synthesis of the actual jump controller which governs the remainder of the motion, and for which we wish to discover strategies without any recourse to known solutions.

The jump controller begins from the take-off state and needs to control the body during take-off, over the bar, and to prepare for landing. This poses a challenging learning problem because of the demanding nature of the task, the sparse fail/success rewards, and the difficulty of also achieving natural human-like movement. We apply two key insights to make this task learnable using deep reinforcement learning. First, we employ an action space defined by a subspace of natural human poses as modeled with a Pose Variational Autoencoder (P-VAE). Given an action parameterized as a target body pose, individual joint torques are then realized using PD-controllers. We additionally allow for regularized {\em offset} PD-targets that are added to the P-VAE targets to enable strong takeoff forces. Second, we employ a curriculum that progressively increases the task difficulty, i.e., the height of the bar, based on current performance.

A diverse set of strategies can already emerge after the Stage 1 BDS optimization. To achieve further strategy variations, we reuse the take-off states of the existing discovered strategies for another stage of optimization. The diversity is explicitly incentivized during this Stage 2 optimization via a novelty reward, which is focused specifically on features of the body pose at the peak height of the jump. As shown in Figure~\ref{fig:overview}, Stage~2 makes use of the same overall DRL learning procedure as in Stage~1, albeit with a slightly different reward structure.



\section*{Data Records}

All computed molecular structures, as well as their corresponding properties and wave functions are accessible through the ETH Library Collection service (\href{https://doi.org/10.3929/ethz-b-000482129}{Data Citation 1})~\cite{qmugs_repository}.

\subsection*{Format specification}
A \texttt{summary.csv} comma-separated file contains computed molecular-level properties and additional annotations. A compressed tarball file (\texttt{structures.tar.gz}) of $\sim7$~gigabytes (GB) contains plain MDL structure-data files~\cite{dalby1992description} (SDFs) with embedded atomic and molecular properties, grouped in sub-directories according to their respective ChEMBL identifiers. These SDFs include single-point electronic properties calculated on the GFN2-xTB and $\omega$B97X-D/def-SVP levels of theory, as described in Table~\ref{tbl:properties}. A second compressed tarball file (\texttt{vibspectra.tar.gz}, $\sim3$~GB) contains vibrational spectra.

Wave function files, including additional properties such as DFT-density and orbital matrices, as described in Table~\ref{tbl:wfn}, are split into $100$ compressed tarballs (\texttt{wfns\_xx.tar.gz}) of $\sim50$~GB each for easier management and downloading. These are supplied as NumPy~\cite{harris2020array} (\texttt{.npy}) binary files, which can be read using the Psi4 software package. Molecules (with all conformers grouped together) were assigned at random to the tarballs to enable easy use of subsets of the QMugs dataset without having to download all the files. The assignment of ChEMBL identifiers to tarballs is described in a \texttt{tarball\_assignment.csv} file.



\section*{Technical Validation}


\subsection*{Optimized geometry sanity checks}
Four consecutive geometry checks were performed to filter out structures for which the geometry optimization procedure converged to unrealistic conformations. To determine suitable thresholds for removing a structure from our dataset, the generated geometries were compared to experimental reference values and to DFT-optimized geometries extracted from the PubChemQC dataset~\cite{nakata2017pubchemqc}. Specifically, we investigated (i) the deviation of bond lengths from experimental reference values, (ii) isomorphism between the initial molecular graphs and those obtained after geometry optimization, (iii) linearity of triple bonds, and (iv) planarity of aromatic rings. Structures were removed from the dataset if they failed any test of these tests. In total, $10,986$ ($0.55$\%) conformations were discarded from the dataset. Each test is briefly described in the following subsections, with further technical details reported in the ESI.

\subsubsection*{Deviation of bond lengths from experimental reference values}
Bond lengths in the optimized structures were compared to average experimental reference values for bonds of the same bond type (single, double, triple, or aromatic) and between the same atoms. Reference values were obtained from the Computational Chemistry Comparison and Benchmark DataBase (CCCBDB)~\cite{nist_database}, and the largest absolute bond-length deviation from reference values was recorded per molecule. Bonds for which no reference value was available ($0.75\%$) were omitted. The same analysis was carried out for molecules from the PubChemQC dataset containing the same atom types as QMugs, in order to obtain a comparable set with respect to the present atom types. The PubChemQC set ($3,834,382$ conformations with reference bond lengths) showed a deviation of $0.06$~$\pm$~\SI{0.04}{\angstrom} (median $\pm$ $1$ standard deviation), whereas the QMugs dataset ($2,004,003$ conformations with reference bond lengths) showed a deviation of $0.07$~$\pm$~\SI{0.03}{\angstrom}. Based on the observed distribution of bond-length deviations from experimental reference values (Figure~S1 in ESI) and manual investigation of example structures, \SI{0.2}{\angstrom} was determined to be a suitable threshold for a conformation to be removed from the dataset, which included $6,131$ ($0.31$\%) examples.

\subsubsection*{Molecular graph isomorphism}
It was investigated whether atom connectivity could be reconstructed after removing bond information from the generated SDFs. To this end, molecular graphs constructed exclusively from atom positions and types were compared to those obtained using the original atom connectivity (see ESI for details). $1,568$ ($0.08$\%) conformations for which the resulting molecular graphs were non-isomorphic failed this test.

\subsubsection*{Deviation of triple bonds from linear geometry}
The deviation of triple bonds from their ideal linear geometry was examined. In this investigation, ring triple bonds were not considered owing to routinely-occurring deviations from linear geometry in systems with high ring strain~\cite{bach2009ring}. The largest deviation from a \SI{180}{\degree} (linear) bond angle was recorded for each molecule containing at least one non-ring triple bond. The same analysis was performed on the PubChemQC dataset~\cite{nakata2017pubchemqc}. Triple-bond-containing molecules from PubchemQC and QMugs ($273,320$ and $165,101$ samples, respectively) show deviations of $1.38$~$\pm$~\SI{1.46}{\degree} (median $\pm$ $1$ standard deviation), and $1.46$~$\pm$~\SI{2.13}{\degree}, respectively. Based on the observed distribution of triple bond angles from a linear geometry (Figure~S2 in ESI) and manual inspection of structures, a \SI{10}{\degree} deviation was identified as a suitable threshold. $1,147$ ($0.06$\%) conformations failed this test.

\subsubsection*{Deviation of aromatic rings from planar geometry}
The planarity of carbon-containing aromatic rings was also investigated. For each molecule containing aromatic carbon atoms, the largest dihedral angle between the two planes spanned by each aromatic carbon atom and its three neighbors was recorded (see ESI for details). The same analysis was performed on the PubChemQC dataset~\cite{nakata2017pubchemqc}. Molecules from PubchemQC and QMugs ($2,391,589$ and $1,950,929$ conformations with aromatic carbons, respectively) showed median dihedral angles (~$\pm$~$1$ standard deviation) of $1.70$~$\pm$~\SI{1.85}{\degree} and $2.99$~$\pm$~\SI{2.20}{\degree}, respectively. Based on the observed distribution of dihedral angles from planar geometries (Figure~S3 in ESI) and manual inspection of structures, $2,769$ ($0.14$\%) conformations with aromatic carbon dihedral angles above \SI{15}{\degree} were discarded.



\subsection*{Further geometrical assessment}

The changes in the molecular geometries along the applied pipeline were examined in order to evaluate the effects of the applied steps. Figure~\ref{fig:geometry_validation}A shows the mean pairwise RMSD of atom positions between the conformations of each molecule at different steps along the pipeline. Conformations sampled during MTD simulations show a mean pairwise RMSD of $2.40$ $\pm$ $0.52$ \si{\angstrom} (median $\pm$ $1$ standard deviation). The $k$-means clustering procedure accomplishes the envisaged task of sampling conformations with higher geometric diversity ($2.67$ $\pm$ $0.74$ \si{\angstrom}). During the geometry optimization process, conformational diversity decreases ($2.48$ $\pm$ $0.86$ \si{\angstrom}). Unsurprisingly, for some molecules featuring rigid structures, conformations tend to converge toward the same energy minimum ($0.09$ \% of molecules show a mean pairwise RMSD $<$~\SI{0.01}{\angstrom} between their optimized conformers).


The degree to which the molecular geometries changed during the final optimization step was further analyzed. Molecules with initially more diverse conformations (higher mean pairwise RMSD of pre-optimized conformations) were shown to undergo a greater chance in atom positions (mean RMSD of pre- vs. post-optimized conformations) during optimization with the GFN2-xTB method (Figure~\ref{fig:geometry_validation}B). The observed heteroscedastic behavior of these two properties indicates that while the mean RMSD of pre- vs. post-optimized conformations tends to increase with higher mean pairwise RMSD of pre-optimized conformations, its variance also increases. 


Finally, the suitability of GFN2-xTB as a lower-cost surrogate for DFT-level geometry optimization (Figure~\ref{fig:geometry_validation}C) was confirmed. $500$ randomly-chosen structures prior to semi-empirical geometry optimization from the QMugs dataset were further subjected to DFT-level geometry optimization ($\omega$B97X-D/def2-SVP), discarding structures that could not be converged in 100 iterations or with the computational resources described in the ESI. The RMSDs between the structures independently optimized at both levels of theory were then measured. The pairs of structures show RMSDs of $0.47$ $\pm$ $0.63$ \si{\angstrom} (median $\pm$ $1$ standard deviation), indicating that the chosen semi-empirical method obtains similar geometries to those obtained with more expensive first-principle calculations. Large RMSDs in some example pairs (Figure~\ref{fig:geometry_validation}C) could be interpreted as indicative of convergence to distinct local minima. 

For $2,067$ molecules, their individual conformations have different SMILES describing two different \textit{(E)/(Z)} isomers. Those structures are either $\alpha$-$\beta$-unsaturated ketones, $\alpha$-$\beta$-unsaturated nitriles, imine, or azo compounds, for which isomerization might be plausible~\cite{goulet2017electrocatalytic, roca2014dft}. In part due to the applied washing procedure, $17,176$ molecules can be represented with a SMILES string that is shared with at least one other ChEMBL-ID. 


\subsection*{Validation of single-point properties}

To validate the general agreement between the two methods employed in this work, the correlation between a series of single-point properties computed on both levels of theory was analyzed. Both global, molecular (Figure~\ref{fig:delta_molecular_props}) and local, atomic/bond properties (Figures~\ref{fig:partial}~\&~\ref{fig:bond}) were considered. All single-point molecular properties showed a high degree of correlation. Formation energies $E_\mathrm{Form}$ (Figure~\ref{fig:delta_molecular_props}A), which were obtained by subtracting atomic energies $E_{\mathrm{Atom}}$ from total internal energies $U_{RT}$, show a Pearson correlation coefficient (PCC) of $0.998$. Dipole moments $\mu$ and rotational constants $A$ (excl. $22$ small structures with very high rotational constants; Figure~\ref{fig:delta_molecular_props}B,C) display PCCs of $0.969$ and $0.999$, respectively. Orbital energies, namely the energies for highest occupied (HOMO)  $E_{\mathrm{HOMO}}$ and lowest unoccupied molecular orbitals (LUMO) $E_{\mathrm{LUMO}}$ and HOMO-LUMO gaps $E_{\mathrm{Gap}}$ show PCCs of $0.769$, $0.924$ and $0.830$, respectively (Figures~\ref{fig:delta_molecular_props}~D,~E~\&~F). The observed PCCs for all six single-point molecular properties indicate good agreement between the two methods. Atom-type-specific partial charges for the 10 atom-types in QMugs (Figure~\ref{fig:partial}, ESI Table~1) as well as the 15 most abundant covalent bonds orders (Figure~\ref{fig:bond}, ESI Table~2) also show high correlations between the two methods used herein. Regarding partial charges, $7$ out of the $10$ atom types considered in QMugs were observed to have PCCs $>0.8$, with the remaining carbon, nitrogen, and oxygen atom-types resulting in lower PCCs of $0.574$, $0.124$, and $0.274$, respectively. Regarding bond orders, $10$ out of the $15$ show PCCs $>0.9$ and $14$ out of $15$ display PCCs $>0.75$ (see ESI Table~2 for additional metrics). Notably only carbon-fluorine bonds display a larger discrepancy between both levels of theory, with an observed PCC of $0.153$. The observed correlations in both molecular and atomic single-point properties between GFN2-xTB and $\omega$B97X-D/def2-SVP confirm the suitability of the former method as a computationally affordable starting point for $\Delta$-learning of DFT-level properties.

\section*{Usage Notes}

All data files can be accessed via any modern web browser, and can be programmatically downloaded using the provided instructions in the archive's readme. The provided SDFs can be processed using standard cheminformatics software (\textit{e.g.}, RDKit~\cite{rdkit}, KNIME~\cite{knime}), and wave function files using the Psi4~\cite{smith2020psi4} software package or directly using Numpy~\cite{harris2020array}. 

\section*{Code availability}

All analyses were supported by the Python programming language (version 3.7.7) and its scientific software stack~\cite{harris2020array}. Molecular conformations were generated using RDKit~\cite{rdkit} (version 2020.03.3) and GFN2-xTB~\cite{grimme2017robust, bannwarth2019gfn2, pracht2019robust, grimme2019exploration, bannwarth2020extended} (version 6.3.1). All quantum mechanical calculations were carried out with Psi4~\cite{smith2020psi4} (version 1.3.2). Molecular structure visualizations were created using PyMol~\cite{PyMOL} (version 2.3.5) and ChemDraw (version 19.1.1.32). The rclone (\url{https://rclone.org}) WebDAV client was used for all data uploading purposes.



\documentclass{article}

\begin{document}

\section{Acknowledgements}

The author would like to thank B. J. Hiley, M. Hajtanian, and D. Nellist for their insightful conversations and support.

\end{document}
\section*{Author contributions statement}
\textbf{Clemens Isert}: Methodology, Formal Analyses, Writing. \textbf{Kenneth Atz}: Methodology, Formal Analyses, Writing. \textbf{José Jiménez-Luna}: Conceptualization, Methodology, Writing. \textbf{Gisbert Schneider}: Supervision, Writing. 


%(check \url{https://www.elsevier.com/authors/policies-and-guidelines/credit-author-statement} for examples)  

\section*{Competing interests}
G.S. is a cofounder of inSili.com LLC, Zurich, and a consultant to the pharmaceutical industry.


\bibliography{references}

\section*{Figures \& Tables}

\begin{figure}[ht]
\centering
\includegraphics[width=\linewidth]{figures/rod_shere_disk_example_with_structures}
\caption{\textbf{A})~Principal-moments-of-inertia plot~\cite{sauer2003molecular} for molecules in the QMugs dataset. $NPR_x$ = $x$-th normalized principal moment, $I_x$ = $x$-th smallest principal moment of inertia.
\textbf{B})~Venn diagram showing overlap between QMugs and other well-known datasets with DFT-level computed properties: QM9~\cite{ramakrishnan2014quantum}, PubChemQC~\cite{nakata2017pubchemqc}, and ANI-1~\cite{smith2017anid}. Overlap was computed based on the uniqueness of the InChI representations of the contained molecules. Numbers do not add up to those reported in Table~\ref{tbl:desc} because of InChI strings that occur multiple times.}
\label{fig:rod_disk_sphere_venn}
\end{figure}


\begin{figure}[ht]
\centering
\includegraphics[width=\linewidth]{figures/rdkit_props}
\caption{Distribution of properties for the molecules contained in the QMugs dataset.}
\label{fig:molecule_props}
\end{figure}


\begin{figure}[ht]
\centering
\includegraphics[width=\linewidth]{figures/qmugs_pipeline}
\caption{Overview of the data generation process. Molecules were extracted from the ChEMBL database, standardized, and filtered, and starting conformers were generated using the RDKit software package. Metadynamics (MTD) simulations were performed using the GFN2-xTB semi-empirical method to generate three diverse conformations before final geometry optimization. Molecules that did not pass a series of geometric sanity checks were removed. DFT-level properties ($\omega$B97X-D/def2-SVP) were computed using the Psi4 software.}
\label{fig:pipeline}
\end{figure}


\begin{figure}[ht]
\centering
\includegraphics[width=\linewidth]{figures/geometry_validation}
\caption{
    (\textbf{A})~Distributions of mean pairwise RMSD of atom positions between conformations of each molecule in the QMugs dataset at different stages along the pipeline. While the $k$-means sampling process selects conformations that are, on average, more geometrically diverse than the average pair of structures generated by MTD simulations, geometry optimization reduces the geometrical diversity between the optimized conformers.
    (\textbf{B})~Change in atom positions during geometry optimization vs. mean pairwise RMSD of conformations before optimization. Molecules with initially more diverse conformations displayed a greater change in atom positions than those with initially less diverse conformations. 
    (\textbf{C})~Distribution of RMSD of structures prior to and after optimization with the semi-empirical GFN2-xTB method, and of structures optimized with the same approach vs. with $\omega$B97X-D/def2-SVP. The structures of three molecules with varying differences between the two methods are shown as illustrative examples (black and gray correspond to GFN2-xTB and $\omega$B97X-D/def2-SVP-optimized structures, respectively). For illustrative purposes, the example molecules are aligned on their substructures.
}
\label{fig:geometry_validation}
\end{figure}


\begin{figure}[ht]
\centering
\includegraphics[width=\linewidth]{figures/delta_molecular}
\caption{Comparison of molecular properties computed at the two levels of theory considered herein (GFN2-xTB, $\omega$B97X-D/def2-SVP) for the molecules contained in QMugs. The molecular formation energy $E_{\mathrm{form}}$ in (\textbf{A}) was calculated by subtracting the atomic $U_{\mathrm{Atom}}$ contributions from the  total molecular energies $U_{RT}$. Only the rotational constants $A$ are shown in (\textbf{C}) as their $B$ and $C$ counterparts showed highly similar values. $22$ conformations of small molecules show very large rotational constants and are not shown. RMSE and PCC for rotational constant $A$ are $845.834$ cm$^{-1}$ and $0.091$ respectively, if those structures are included. Abbreviations: RMSE, root mean squared error; PCC, Pearson's correlation coefficient.}
\label{fig:delta_molecular_props}
\end{figure}


\begin{figure}[ht]
\centering
\includegraphics[width=\linewidth]{figures/partial_charges}
\caption{Atom-type-specific partial charge correlations (GFN2-xTB, $\omega$B97X-D/def2-SVP) for the QMugs dataset (see ESI Table~1 for additional metrics)}
\label{fig:partial}
\end{figure}


\begin{figure}[ht]
\centering
\includegraphics[width=\linewidth]{figures/bond_orders}
\caption{Comparison of Wiberg bond orders between GFN2-xTB and $\omega$B97X-D/def2-SVP for the 15 most frequently occurring bond types in the QMugs dataset. The latter level of theory uses L\"owdin-orthogonalization. See ESI Table~2 for additional metrics. For bond types which occurred $>1$M times in the dataset, a randomly chosen sample of $1$M bonds is plotted.}
\label{fig:bond}
\end{figure}




\begin{table}[ht]
\caption{Descriptive statistics of the dataset reported herein in the context of other  DFT-level molecular datasets and the information provided by each. The number of molecules for PubChemQC corresponds to that available on the website of the project.~\cite{pubchemqc_website} Heavy atom averages are weighted by the number of conformations.}
\label{tbl:desc}
\centering
\resizebox{\textwidth}{!}{%
\begin{tabular}{@{}l>{\raggedleft\arraybackslash}p{2cm}>{\raggedleft\arraybackslash}p{2cm}>{\raggedleft\arraybackslash}p{2cm}p{5cm}P{2cm}P{2cm}p{3cm}@{}}
\toprule
\textbf{Dataset} &
  \textbf{Unique compounds} &
  \textbf{Total conformations} &
  \textbf{Heavy atoms max (mean)} &
  \textbf{Method} &
  \textbf{$\Delta$-learning possible} &
  \textbf{Wave functions} \\ \midrule
QM9       & $133,885$  & $133,885$   & $9$~~~($8.8$) & B3LYP/6-31G(2df,p)               & \xmark &  \xmark \\
ANI-1     & $57,462$   & $22,057,374$ & $8$~~~($7.1$) & $\omega$B97X/6-31G(d)              & \xmark  & \xmark \\
PubChemQC & $3,982,436$ & $3,982,436$  & $51$ ($14.1$) & B3LYP/6-31G(d)              & \xmark  & \xmark \\
QMugs     & $665,911$  & $1,992,984$  & $100$ ($30.6$) & GFN2-xTB + $\omega$B97X-D/def2-SVP & \cmark  & \cmark \\ \bottomrule
\end{tabular}
}
\end{table}


\begin{table}[ht]
\footnotesize
\caption{Calculated properties as stored in the SDFs of the QMugs data collection. Abbreviations: a.u., atomic units; vib., vibrational; rot., rotational; transl., translational. Properties that enable $\Delta$ machine learning are labelled with $\blacklozenge$.}
\label{tbl:properties}
\centering
\begin{tabular}{llllll}
\toprule
\textbf{Property}                                        & \textbf{Symbol}                               & \textbf{Unit}           & \textbf{Key}  & $\Delta$-ML        \\  \midrule
ChEMBL identifier                                               &         -                      &           -              & \texttt{CHEMBL\_ID} &                \\
Conformer identifier                                                 &     -                                          &                    -     & \texttt{CONF\_ID}      &              \\
Total energy                                             & $U_{RT}$            & \si{\hartree}                      & \texttt{GFN2:TOTAL\_ENERGY} & $\blacklozenge$                  \\
Internal atomic energy                             & $E_\mathrm{Atom}$                             & \si{\hartree}                      & \texttt{GFN2:ATOMIC\_ENERGY} &       \\
Formation energy                             & $E_\mathrm{Form}$           & \si{\hartree}                      & \texttt{GFN2:FORMATION\_ENERGY}   & $\blacklozenge$          \\
Total enthalpy                                           & $H_{RT}$                                      & \si{\hartree}                      & \texttt{GFN2:TOTAL\_ENTHALPY}    &        \\
Total free energy                                        & $G_{RT}$                                      & \si{\hartree}                      & \texttt{GFN2:TOTAL\_FREE\_ENERGY} &        \\
Dipole ($x$, $y$, $z$, total)                  & $\mu$                                         & D                       & \texttt{GFN2:DIPOLE}      & $\blacklozenge$                    \\
Quadrupole ($xx$, $xy$, $yy$, $xz$, $yz$, $zz$) & $Q_{ij}$                                      & D \si{\angstrom}                 & \texttt{GFN2:QUADRUPOLE}      &           \\
Rotational constants ($A$, $B$, $C$)           & $A$, $B$, $C$                              & \si{\centi\meter}$^{-1}$                     & \texttt{GFN2:ROT\_RONSTANTS}    & $\blacklozenge$              \\
Enthalpy (vib., rot., transl., total)          & $\Delta H$                                    & cal mol$^{-1}$          & \texttt{GFN2:ENTHALPY}     &               \\
Heat capacity (vib., rot., transl., total)    & $C_{V}$                                       & cal K$^{-1}$ mol$^{-1}$ & \texttt{GFN2:HEAT\_CAPACITY}  &             \\
Entropy (vib., rot., transl., and total)       & $\Delta S$                                    & cal K$^{-1}$ mol$^{-1}$ & \texttt{GFN2:ENTROPY}     &                \\
HOMO energy                                              & $E_\mathrm{HOMO}$                             & \si{\hartree}                      & \texttt{GFN2:HOMO\_ENERGY}       & $\blacklozenge$             \\
LUMO energy                                              & $E_\mathrm{LUMO}$                             & \si{\hartree}                      & \texttt{GFN2:LUMO\_ENERGY}     & $\blacklozenge$           \\
HOMO-LUMO gap                                            & $E_\mathrm{Gap}$                              & \si{\hartree}                      & \texttt{GFN2:HOMO\_LUMO\_GAP}  & $\blacklozenge$               \\
Fermi level                                              & $E_{\mathrm{Fermi}}$                                   & \si{\hartree}                      & \texttt{GFN2:FERMI\_LEVEL}   &             \\
Mulliken partial charges                                 & $\delta_{M}$                                  & \si{\elementarycharge}                       & \texttt{GFN2:MULLIKEN\_CHARGES} & $\blacklozenge$              \\
Covalent coordination number                           & $N_{\textrm{coord}}$                  & -                   &\texttt{GFN2:COVALENT\_COORDINATION\_NUMBER}  & \\
Molecular dispersion coefficient                           & $C_6$                                                & a.u.                            &\texttt{GFN2:DISPERSION\_COEFFICIENT\_MOLECULAR} & \\
Atomic dispersion coefficients                         & $C_6$                                                & a.u.                                &\texttt{GFN2:DISPERSION\_COEFFICIENT\_ATOMIC} & \\
Molecular polarizability                           & $\alpha(0)$                                                 & a.u.                                        &\texttt{GFN2:POLARIZABILITY\_MOLECULAR} &  \\
Atomic polarizabilities                            & $\alpha(0)$                                                 & a.u.                                        &\texttt{GFN2:POLARIZABILITY\_ATOMIC} &  \\
Wiberg bond orders                                    & $M_{AB}$                                 &             -            &\texttt{GFN2:WIBERG\_BOND\_ORDER}     & $\blacklozenge$            \\
Total Wiberg bond orders                               & $\sum_{A (A \neq B)} M_{AB}$                      &       -                  &\texttt{GFN2:TOTAL\_WIBERG\_BOND\_ORDER}  & $\blacklozenge$              \\
Total energy                                             & $U_{RT}$                                      & \si{\hartree}                      & \texttt{DFT:TOTAL\_ENERGY}     & $\blacklozenge$             \\
Total internal atomic energy                             & $E_\mathrm{Atom}$                             & \si{\hartree}                      & \texttt{DFT:ATOMIC\_ENERGY}    &         \\
Formation energy                             & $E_\mathrm{Form}$                             & \si{\hartree}                      & \texttt{DFT:FORMATION\_ENERGY}  & $\blacklozenge$           \\
Electrostatic potential                                  & $V_{ESP}$                                     & \si{\volt}                       & \texttt{DFT:ESP\_AT\_NUCLEI}   &         \\
L\"owdin partial charges                                 & $\delta_{L}$                                  & \si{\elementarycharge}                       & \texttt{DFT:LOWDIN\_CHARGES}     &       \\
Mulliken partial charges                                 & $\delta_{M}$                                  & \si{\elementarycharge}                       & \texttt{DFT:MULLIKEN\_CHARGES}   & $\blacklozenge$           \\
Rotational constants  ($A$, $B$, $C$)          & $A$, $B$, $C$                              & \si{\centi\meter}$^{-1}$                     & \texttt{DFT:ROT\_CONSTANTS}   & $\blacklozenge$              \\
Dipole ($x$, $y$, $z$, total)                  & $\mu$                                         & D                       & \texttt{DFT:DIPOLE}                     \\
Exchange correlation energy                              & $\hat{V}_{eN}$                                & \si{\hartree}                      & \texttt{DFT:XC\_ENERGY}   &              \\
Nuclear repulsion energy                                 & $\hat{V}_{eN}$                                & \si{\hartree}                      & \texttt{DFT:NUCLEAR\_REPULSION\_ENERGY} & \\
One-electron energy                               & $\hat{T}_{e}$                                 & \si{\hartree}                      & \texttt{DFT:ONE\_ELECTRON\_ENERGY}   &    \\
Two-electron energy                                      & $\hat{V}_{ee}$                                & \si{\hartree}                      & \texttt{DFT:TWO\_ELECTRON\_ENERGY} &     \\
HOMO energy                                              & $E_\mathrm{HOMO}$                             & \si{\hartree}                      & \texttt{DFT:HOMO\_ENERGY}    & $\blacklozenge$               \\
LUMO energy                                              & $E_\mathrm{LUMO}$                             & \si{\hartree}                      & \texttt{DFT:LUMO\_ENERGY}    & $\blacklozenge$               \\
HOMO-LUMO gap                                            & $E_\mathrm{Gap}$                              & \si{\hartree}                      & \texttt{DFT:HOMO\_LUMO\_GAP}   & $\blacklozenge$             \\
Mayer bond orders                                        & $M_{AB}$                                 &                -         & \texttt{DFT:MAYER\_BOND\_ORDER}   &           \\
Wiberg-L\"owdin bond orders                              & $W_{AB}$                                 &            -             & \texttt{DFT:WIBERG\_LOWDIN\_BOND\_ORDER}  & $\blacklozenge$      \\
Total Mayer bond orders                              & $\sum_{A (A \neq B)} M_{AB}$                      &          -               & \texttt{DFT:TOTAL\_MAYER\_BOND\_ORDER}      &      \\
Total Wiberg-L\"owdin bond orders                          & $\sum_{A (A \neq B)} W_{AB}$        &      -                   & \texttt{DFT:TOTAL\_WIBERG\_LOWDIN\_BOND\_ORDER}  & $\blacklozenge$     \\ \bottomrule
\end{tabular}
\end{table}

\begin{table}[ht]
\caption{Calculated molecular properties stored in the wave function files provided in the QMugs data collection. Mayer and Wiberg-L\"owdin bond orders included here represent a superset of the bond orders in the SDFs which additionally comprise bond orders for non-covalent bonds.}
\label{tbl:wfn}
\centering
\begin{tabular}{llll}
\toprule
\textbf{Property}                                        & \textbf{Symbol}                               & \textbf{Key}                            \\ \midrule
Alpha density matrix                                     & $\mathrm{D}_{\alpha}$                         & \texttt{matrix, Ca}                              \\
Beta density matrix                                      & $\mathrm{D}_{\beta}$                          & \texttt{matrix, Cb}                              \\
Alpha orbitals                                           & $\mathrm{C}_{\alpha}$                         & \texttt{matrix, Da}                              \\
Beta orbitals                                            & $\mathrm{C}_{\beta}$                          & \texttt{matrix, Db}                              \\
Atomic-orbital-to-symmetry-orbital transformer           & $\mathrm{C}_{\mathrm{AOTOSO}}$                & \texttt{matrix, aotoso}                          \\
Mayer bond orders                                        & $M_{AB}$                                      & \texttt{MAYER\_INDICES}                          \\
Wiberg-L\"owdin bond orders                              & $W_{AB}$                                      & \texttt{WIBERG\_LOWDIN\_INDICES}                 \\ \bottomrule
\end{tabular}
\end{table}


\end{document}