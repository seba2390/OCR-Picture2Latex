
\documentclass[twocolumn,floats,superscriptaddress]{revtex4}
%\documentclass[sor,graphicx]{revtex4-1}
%\documentclass[onecolumn,showpacs,floats,superscriptaddress]{revtex4}
%\usepackage{geometry}
%\usepackage{multicol}
\usepackage{graphicx}
\usepackage{appendix}
\usepackage{amsmath}
\usepackage{amssymb}
\usepackage[colorlinks=true,citecolor=blue,linkcolor=blue]{hyperref}
\usepackage{amsfonts}
\usepackage[latin9]{inputenc}
\usepackage{color}
\usepackage{bm}
\usepackage{mathrsfs}
\usepackage{float}
\hyphenpenalty=5000
\tolerance=1000

\usepackage{graphicx}% Include figure files
\usepackage{dcolumn}% Align table columns on decimal point
\usepackage{bm}% bold math
%\usepackage[mathlines]{lineno}% Enable numbering of text and display math
%\linenumbers\relax % Commence numbering lines


%\makeatletter
%
%\@addtoreset{equation}{section}
%
%\makeatother
%\oddsidemargin=-1.54cm
%    \evensidemargin=-1.54cm
%\textwidth=19cm
%\geometry{left=2.5cm,right=2.5cm,top=2.5cm,bottom=2.5cm}
\begin{document}
\title{ Exact solutions and degenerate properties of spin chains with reducible Hamiltonians}
\date{\today }
\author{Shiung Fan}
\email{shiungfan@gmail.com}
\affiliation{Beijing Computational Science Research Center, Beijing 100193, China}


\begin{abstract}
The Jordan--Wigner transformation plays an important role in spin models.
However, the non-locality of the transformation implies that a periodic chain of {$N$}
%should it be italics?
 spins is not mapped to a periodic or an anti-periodic chain of lattice fermions. Since only the $N-1$ bond is different, the effect is negligible for large systems, while it is significant for small systems. In this paper, it is interesting to find that a class of periodic spin chains can be exactly mapped to a periodic chain and an anti-periodic chain of lattice fermions without redundancy when the Jordan--Wigner transformation is implemented. For these systems, possible high degeneracy is found to appear in not only the ground state, but also the excitation states. {Further, we take the one-dimensional compass model and a new XY-XY model ($\sigma_x\sigma_y-\sigma_x\sigma_y$) as examples to demonstrate our {proposition}. Except for the well-known one-dimensional compass model, we will see that in the XY-XY model, the degeneracy also grows exponentially with the number of sites.}

\end{abstract}


\maketitle
%keywords: Periodic spin chains; Jordan-Wigner transformation; Degeneracy;

\section{Introduction}
The Jordan--Wigner (JW) transformation establishes a connection between spin-1/2 operators and spinless fermion operators \cite{Jordan1928}, and it has become a powerful tool for solving one-dimensional (1D) spin models and {a few two-dimensional Ising models} \cite{Mattis,Lieb1961,Schultz1964}.
Besides, it is remarkable that {the} JW transformation has been generalized to higher dimensions in recent decades \cite{Fradkin1989,Wang1992-1,Wang1992-2,Wang1992-3,Azzouz1993,Huerta1993,Kochmanski1995,Batista2001,Bock2001}. {Typical examples of applications are provided in \cite{Lieb1961}. In that paper, using the JW transformation, Lieb et al. studied the ground states, excitations and the order of the one-dimensional XY model and Heisenberg--Ising model and concluded that both models have no long-range order for the isotropic case, but long-range order for any anisotropic cases.}

 {Generally speaking, when the JW transformation is applied under the periodic boundary condition, theoretical physicists working on related fields always encounter that it introduces a phase term, and consequently causes redundant solutions. The work in \cite{Mattis} mentions this problem when the author introduces the JW transformation. As an example relevant to this paper, in \cite{Brzezicki2007}, in order to exclude redundancy or to find the physical spectrum, Brzezicki et al. needed to distinguish the Bogoliubov vacuum by its parity and to judge whether operators change the parity. With regard to the Bogoliubov vacuum and fermion parity, \cite{Bertsch2009} provided impressive clarification. The work in \cite{Bertsch2009} not only introduced the basic concepts of the Bogoliubov quasiparticles and the Bogoliubov vacuum, but also rigorously discussed the choice of the Bogoliubov vacuum in general situations, the relationship between the particle-number parity and the Bogoliubov matrix transformation and the applications in systems owning the signature symmetry \cite{Bohr1975}.} As for periodic spin chains, since the $N-1$ bond takes an additional phase term {$\exp{(i\pi n)}$ with $n=\sum_{l=1}^{N}{a^\dag_{l}a_l}$ after the JW transformation}, {solutions depend on the evenness and oddness of the total number of {occupancies}}, {i.e., $n$}, which is called the ``\emph{a}-cyclic'' problem. To remove the redundancy, some effort is spent in doing projections; or {approximate results are} adopted for large systems {by directly dropping this phase}, i.e., the ``\emph{c}-cyclic'' problem \cite{Lieb1961}.

In this paper, we find a class of systems in which the JW transformation does not introduce redundancy. {In addition, it is discovered that the ``holistic degeneracy'' (or which can be interpreted as the degeneracy of the subspace of the Hamiltonian) exists in these systems, and it must be {$2^x$-fold, in which $x$ is a positive integer}. In some cases, the holistic degeneracy grows exponentially with the size of chains, and two representatives, the 1D compass model and a new XY-XY model, are given in Sections~\ref{exa} and \ref{exb}, respectively. Taking into consideration that a common feature of spin liquid states is the high degeneracy \cite{Balents2010}, the finding of this class of systems may help the research concerning spin liquid.}

\section{ {Spin-Fermion Mappings in Ordinary and Reducible Systems}}

For completeness, we first introduce how redundancy occurs in solutions, then give {an} abstract discussion on the solutions of reducible systems, and further, these systems are classified by the newly-defined holistic degeneracy.

Considering a 1D spin-$1/2$ system with $N$ sites labeled by $1$, $2$, $\cdots$, $N$, its Hamiltonian is $H_{s}$. We assume that $H_{s}$ does not change the parity of the number of spin-up {or -down states}, and $H_s$ includes merely nearest-neighbor interactions. {The restriction of nearest-neighbor interactions simplifies the problem, because merely the $N-1$ bond needs exceptional attention; otherwise, it becomes more complicated.} Let the parity be $P_{s}=(-1)^{N_{up}}$ where $N_{up}$ is the number of spin-up states; we have:
\begin{equation}\label{pspin}
 [P_{s}, H_{s}]=0.
\end{equation}
Equation (\ref{pspin}) indicates that the eigenstates of $H_{s}$ can be divided into two sets according to different eigenvalues of $P_s$. In one set, $P_{s}=1$, and in another set, $P_{s}=-1$. Basic vectors in $H_{s}$'s Hilbert space $\mathcal{M}$ are described as:
\begin{equation}\label{basicv}
 \nu_{i}=\alpha_1\otimes\alpha_2\otimes\alpha_3 \cdots \otimes\alpha_{N},\, i={1,2,3,\ldots,2^N},
\end{equation}
where $\alpha$ is a spin-up or -down state. An eigenstate $\varphi$ of $H_{s}$ is {the} linear superposition of basic vectors, which can be expressed by:
\begin{equation}\label{eigenstate}
 \varphi=\sum_{i=1}^{2^N}\rho_i\nu_i,
\end{equation}
where $\rho$ is the corresponding coefficient. Let $P_{s}$ act on $\varphi$; we have:
\begin{equation}\label{pvariphi1}
 P_{s}\varphi=\sum_{i=1}^{2^N}\rho_i P_{s}\nu_i.
\end{equation}
Since $P_{s}\varphi=\pm\varphi$, we have:
\begin{equation}\label{pvariphi2}
\sum_{i=1}^{2^N}\rho_i P_{s}\nu_i=\pm\sum_{i=1}^{2^N}\rho_i\nu_i.
\end{equation}
Hence, $\nu$'s with nonzero $\rho$'s have the same parity with $\varphi$. For zero $\rho_i$, the parity of $\nu_i$ is not certain, yet this does not make sense. Accordingly, we divide $\mathcal{M}$ into two parts: $\mathcal{M}=\mathcal{M}_{ o}\bigoplus\mathcal{M}_{e}$, in which $\mathcal{M}_{o}$ {($\mathcal{M}_{e}$) consists of basic vectors with odd-(even-)parity}. By the assumptions, we have $H_s=H_s^{o}\bigoplus H_s^{e}$ ({in matrix form}), and $\mathcal{M}_{o}$ and $\mathcal{M}_{e}$ are the Hilbert spaces of $H_s^{o}$ and $H_s^{e}$, respectively. Obviously, $\mathcal{M}_{o}$ and $\mathcal{M}_{e}$ have equal dimensions $2^{N-1}$, i.e., $D(\mathcal{M}_{o})=D(\mathcal{M}_{e})=2^{N-1}$. Independently diagonalizing $H_s^{o}$ and $H_s^{e}$, $2^{N-1}$ eigenvalues will be obtained for either one.

Now, we apply the JW transformation to $H_s$. The Pauli matrices in $H_s$ are transformed by {the following relationships},
\begin{gather}\label{pauli}
\sigma_l^x=\frac{\sigma_l^+ +\sigma_l^-}{2},\,
\sigma_l^y=\frac{\sigma_l^+ -\sigma_l^-}{2i},\,
\sigma_l^z=\frac{\sigma_l^+ \sigma_l^-}{4},
\end{gather}
where $\sigma^{\pm}=\sigma^x\pm i \sigma^y$, and the subscripts $l$ is the {label} of sites.
The JW transformation is as follows:
\begin{gather}
\sigma_l^+=2a_l^\dag \exp{(i\pi\sum_{j<l}{a^\dag_{j}a_j})},\notag
\\
\sigma_l^-=2a_l \exp{(-i\pi\sum_{j<l}{a^\dag_{j}a_j})}.\label{JWT}
\end{gather}
A one-to-one mapping between the spin-up (-down) state and the occupation (non-occupation) state of a fermion has been built, and meanwhile, the commutation relation of spin operators and anticommutation relation of fermion operators are preserved. We use the fermion operators to substitute for $\sigma^\pm$ in the spin Hamiltonian and obtain $H_f$. The purpose of implementing the JW transformation is to take advantage of {the diagonalizable} quadratic form of {the} fermion Hamiltonian $H_f$. A problem here that used to {be faced on} is the boundary condition. $H_s$ is considered to have the periodic boundary condition, i.e., $\sigma_{N}^{\pm}\sigma_{N+1}^{\pm}=\sigma_{N}^{\pm}\sigma_{1}^{\pm}$. {Nevertheless}, the equation $\sigma_{N}^{\pm}\sigma_{N+1}^{\pm}=\sigma_{N}^{\pm}\sigma_{1}^{\pm}$ may not be valid for $H_f$, {because it depends on} the parity of the number of occupation states. {To clarify this statement}, utilizing Equation (\ref{JWT}), we have
$\sigma_{N}^{+}\sigma_{N+1}^{+}=4a_N^{\dag}a_{N+1}^{\dag}$ and $\sigma_{N}^{+}\sigma_{1}^{+}=-4\exp{(i\pi n)}a_N^{\dag}a_{1}^{\dag}$ with $n=\sum_{l=1}^{N}{a^\dag_{l}a_l}$. Apparently for even $n$, $a_{N+1}^{\dag}=-a_{1}^{\dag}$, namely the anti-periodic boundary condition (APBC); for odd $n$, $a_{N+1}^{\dag}=a_{1}^{\dag}$, namely, the periodic boundary condition (PBC).

The same as $H_s$, $H_f$ does not change the parity of occupancies; we are able to divide the Hilbert space of $H_f$ into two subspaces: $\mathcal{M}^f=\mathcal{M}_{ o}^f\bigoplus\mathcal{M}_{e}^f$. The dimensions of each space are:
\begin{equation}\label{dspace}
D(\mathcal{M}^f)=2^N,\,\, D(\mathcal{M}_{o}^f)=D(\mathcal{M}_{e}^f)=\frac{1}{2}\times2^N.
\end{equation}
To obtain exact results, {we need to treat the Hamiltonian within the physical subspaces $\mathcal{M}_{e}^f$ and $\mathcal{M}_{o}^f$.} Commonly, the dimensions of the Hamiltonian with a fixed boundary condition are two-times as large as the subspace, that is to say, $D(H_r^A)>D(\mathcal{M}_{e}^f)$ and $D(H_r^P)>D(\mathcal{M}_{o}^f)$, and consequently, redundancy is inevitable {for solutions}. In order to remove the redundancy, further projections are necessary.

 {Hereafter, the discussion turns to the contents we are focused on} in this paper. We consider a {class} of systems in which the fermion Hamiltonians are reducible when the JW transformation is implemented. The reducible Hamiltonian means that the fermion Hamiltonian can be reduced to lower dimensions by appropriate methods, and such reductions always imply some symmetries in these systems. {In Sections \ref{exa} and \ref{exb} and Appendices \ref{Appa} and \ref{Appb}, we give two examples to show how the dimensions of the Hamiltonians are reduced, and we will see that in both models, the Majorana fermion operators are of cruciality in the reduction.}
For these reducible systems, we denote the reduced Hamiltonian by $H_r$, and $H_r$ is represented by fermion operators. We further assume that $H_r$ has $Q$ quasiparticle states when it is diagonalized, which indicates that $D(H_r)=2^Q$. $Q$ is definitely less than $N$. Let $H_r$ be limited to a fixed boundary condition, which explicitly does not change the dimensions of $H_r$, then we have $D(H_r^A)=2^Q$ and $D(H_r^P)=2^Q$. Defining $d=\frac{D(H_{s/f})}{D(H_r^A)+D(H_r^P)}$, we have two possible situations {of}~$d$: 1.~$d=1$; 2. $d>1$. When $d=1$, we have $D(H_r^A)=D(\mathcal{M}_{e}^f)$ and $D(H_r^P)=D(\mathcal{M}_{o}^f)$; hence, $H_r$ can be exactly diagonalized in the physical subspaces, and redundancy is {avoided}.
%In other words, it can be considered that $H_r$ with a fixed boundary condition has the same solution in two subspaces, nevertheless the redundancy is automatically removed according to the parity relation, thus the redundancy is avoided at no cost.
%%%%%%%%%%%%%%%%%%%%%%%%%%%%%%%%%%%%%%%%%%%%%%%%%%%%%%%%%%%%%%%%%%%%%%%%%%%%%%%%%%%%%%%%%%%%%%%%%%%%%%%%%revise
Therefore, $H_r$ with two boundary conditions exactly gives all solutions of $H_s$, and $N$ and $Q$ satisfy the following relation:
\begin{equation}\label{d1}
 2^N=2\times2^Q.
\end{equation}
Otherwise, $d>1$, and we find $D(H_r^A)<D(\mathcal{M}_{e}^f)$ and $D(H_r^P)<D(\mathcal{M}_{o}^f)$. Since $H_r$ is {equivalent} to $H_f$ except for the distinction of dimensions, it can be deduced that these systems own some kind of symmetries, such that a subspace can be divided into several smaller spaces that are all {equivalent} for $H_r$. Thereby, we have $H_f=dH_r^A\bigoplus dH_r^P$, where $d$ is referred as the multiplicity in the group theory. It deserves to be mentioned that the degeneracy of the ground state of $H_s$ is already determined simply through the {dimensions} of $H_r$. Now that the subspace is divided, regarding each smaller space as an element, we are able to find quantum numbers like $q_1$, $q_2$, $q_3$, etc., and construct a complete set $\{H_r, q's\}$ to describe each smaller space. {Basic vectors in each subspace are not always like these states defined in Equation (\ref{basicv}), because the states in Equation (\ref{basicv}) have been mixed by the Hamiltonian.} Clearly, by deduction, the latter case $d>1$ indicates that the degeneracy of all the eigenvalues given by $H_r$ is exactly $d$-fold (the degeneracy {inside} $H_r$ is not {counted} here). Since $d$ describes the degeneracy of a group of energy levels, it {is} more appropriate to call such degeneracy as {holistic degeneracy}. Similarly, {for $d>1$}, we have:
\begin{equation}\label{dn}
 2^N=2d\times2^Q.
\end{equation}
Equation (\ref{d1}) can be {regarded} as a special case of Equation (\ref{dn}). From Equation (\ref{dn}), we easily find the relation between $d$, $N$, and $Q$,
\begin{equation}\label{dNq}
d=2^{N-Q-1},
\end{equation}
i.e., the holistic degeneracy {increases exponentially with the degree of reduction}. Further, it can be known that the total degeneracy for each energy level must be even-fold, and at least $d$-fold (here, the degeneracy within $H_r$ is {taken into account}).

{At the end of this section, we conspicuously show our conclusions in Figure \ref{mapping}. In the upper mapping diagram of Figure \ref{mapping}a, it is seen that for ordinary systems, they are mapped to two half ranges belonging to the periodic and anti-periodic fermion chains, and each other half range is redundancy denoted by shadow areas. In contrast, in the lower mapping diagram of Figure \ref{mapping}b, reducible systems are mapped to full ranges of the periodic and anti-periodic fermion chains, and degeneracy exists.}

\begin{figure}[H]
\centering
\includegraphics[width= 8cm]{mapping}
\caption{ {Mapping diagrams for ({\bf a}) ordinary and ({\bf b}) reducible systems. In ({a}), shadow areas mean the nonphysical states, i.e., the redundancy; in ({\bf b}), marks ``$\times d$'' mean that the ranges are degenerate and the degeneracy is $d$-fold.}}
\label{mapping}
\end{figure}


%%%%%%%%%%%%%%%%%%%%%%%%%%%%%%%%%%%%%%%%%%
\section{ {Example A: 1D Compass Model}}\label{exa}

In this part, we show an excellent example of our {proposition}. Generally speaking, in light of the degeneracy of these systems, it is in spin systems with special symmetries that our {proposition} is most possibly realized. At present, the compass model is known to own various symmetries \cite{Nussinov2015}; hence, we naturally search a case in the compass model. Indeed, the 1D compass model \cite{Brzezicki2007,Brzezicki2009} {(for one's interest in recent progress, see \cite{Jafari2017-1,Jafari2017-2,Jafari2017-3})}, which is also referred to as the reduced Kitaev model \cite{Feng2007}, is found to be a case of our {proposition}.

The 1D compass model has the following Hamiltonian:
\begin{equation}\label{1dhamito}
H_s=J_x\sum_{l=1}^{N/2}{\sigma^x_{2l-1}\sigma^x_{2l}}+J_y\sum_{l=1}^{N/2}{\sigma^y_{2l}\sigma^y_{2l+1}},
\end{equation}
in which $J_x$ and $J_y$ are interacting parameters for odd and even bonds, respectively. We treat Equation~($\ref{1dhamito}$) with the JW transformation and then substitute Majorana fermions for normal fermions. The Hamiltonian in the Majorana representation is:
\begin{equation}\label{mfermion}
H_{mf}=-i\sum_{l=1}^{N/2}{(J_x c_{2l-1}c_{2l}-J_yc_{2l}c_{2l+1})}, %
\end{equation}
and the diagonalization of Equation (\ref{mfermion}) has already been given in Appendix \ref{Appa}. {In Equation (\ref{mfermion}), the dimensions of the Hamiltonian are already reduced, because each site owns only one Majorana fermion with $\sqrt{2}$ degrees of freedom.} Here, we stress that although the calculations in Appendix \ref{Appa} are standard and the redundancy {of ordinary systems} can be discarded by carefully handling the parity of the states as mentioned in \cite{Lieb1961}, our focus is {on} whether the JW transformation brings redundancy compared {with} conventional models.

Considering the PBC in Equation (\ref{1dhamito}), since each Majorana fermion takes merely $\sqrt{2}$ degrees of freedom, $H_{mf}$ with a certain boundary condition has $Q=N/2$ quasiparticle states when it is diagonalized. Therefore, the dimensions of the Hamiltonian have been reduced from $2^N$ ($H_s$) to $2^{N/2}$ ($H_{mf}$). By the definition $d=\frac{D(H_{s})}{D(H_{mf}^A)+D(H_{mf}^P)}$, we have two situations. If $d=1$ (corresponding to the two-site case), $H_{mf}$ with the APBC and {$H_{mf}$ with} the PBC {give} eigenvalues in two physical subspaces, respectively, and they constitute all the eigenvalues of $H_s$ with no redundancy. Otherwise, $d>1$, and {under two boundary conditions}, $H_{mf}$ gives a part of the eigenvalues of $H_s$, while no redundancy is {introduced. Instead,} the solutions are not complete. The approach of obtaining complete solutions is to duplicate the solutions of $H_{mf}^A$ and $H_{mf}^P$ for $d$ times. By Equation (\ref{dNq}), we are able to find that $d=2^{N/2-1}$. $d$ is called {holistic degeneracy} in this paper, and {meanwhile,} it represents the {minimum} degeneracy of the system.
 {Therefore, it is straightforward to conclude that the ground state of the 1D compass model is $2^{N/2-1}$-fold degenerate, which is identical to the result obtained by the reflection positivity technique \cite{You2008} and by mapping to the quantum Ising models \cite{Brzezicki2009}.}

 {To illustrate how the holistic degeneracy appears {in the momentum space}, utilizing the methods of \cite{Sun2009}, we elaborate the approach of finding all $2^N$ eigenvalues of the spin Hamiltonian in Appendix \ref{Appb}.}

%As is shown in Appendix \ref{Appb}, the Hamiltonian of the 1D compass model is decoupled in the momentum space,
%\begin{equation}\label{mhwk}
%H=H_0+H_\pi+\sum_{0<k<\pi}W(k).
%\end{equation}
%For $W(k)$, $H_0$, and $H_\pi$, they have degenerate eigenvalues corresponding to different parity in their %independent spaces. Eigenvalues of $H$ are obtained through the combination of eigenvalues of independent terms, %and different combinations possibly lead to the even or odd parity of the whole eigenstate. Considering both the %degeneracy in each independent terms and equal combinations, by rigorous algebra, we are able to prove that the %holistic degeneracy of the 1D compass model is $2^{N/2-1}$. In addition, it is straightforward to find that the %ground state of the 1D compass model is $2^{N/2-1}$-fold degenerate, which is identical with the result obtained by %the reflection positivity technique \cite{You2008}, {and by mapping to the quantum %Ising models \cite{Brzezicki2009}.}

 {Exploiting results in \cite{Brzezicki2007}, we are able to analyze the symmetric characters in the real space.} Rotating Equation (\ref{1dhamito}) about the $x$-axis through $\pi/2$, i.e., $\sigma_y\rightarrow \sigma_z$, and $J_y\rightarrow J_z$, then the Hamiltonian in \cite{Brzezicki2007} is obtained,
\begin{equation}\label{1dhamitor}
H_s'=J_x\sum_{l=1}^{N/2}{\sigma^x_{2l-1}\sigma^x_{2l}}+J_z\sum_{l=1}^{N/2}{\sigma^z_{2l}\sigma^z_{2l+1}}.
\end{equation}
Using the $z$-axis as the quantization axis, obviously the quantization axis is currently parallel with one interacting direction. Note that although the Hamiltonians of Equations (\ref{1dhamito}) and (\ref{1dhamitor}) are {equivalent, the methods} in \cite{Brzezicki2007} are out of our formalism. {By comparing the order of applying different transformations between the methods {of} \cite{Brzezicki2007} and our formalism, one would find the difference.} Transforming this Hamiltonian to the dual space by dividing the $N$-site chain into $N/2$ odd pairs, i.e., sites $2l-1$ and $2l$ constitute a unit. There are four states for each pair: $|\uparrow\uparrow \rangle, |\downarrow\downarrow\rangle, |\uparrow\downarrow\rangle, |\downarrow\uparrow\rangle$. Then, introducing a set of quantum numbers $\{s_1,s_2, \cdots, s_{N/2-1},s_{N/2}\}$, $s_l$ corresponds to the $l$-th pair, and $s_l=1$ for parallel states, while $s_l=0$ for antiparallel states. Now, the Hilbert space can be divided equally into $2^{N/2}$ subspaces by giving the set with {distinct} values.
% Further, transforming Equation (\ref{1dhamitor}) to a quantum Ising model, it can be solved in the dual subspace, %although redundancy exists.
 {The key} point here is that the Hamiltonian in each subspace has the same {solutions} when $\sum_l s_l$ {owns} identical parity.
Now, we {think about} symmetries in the dual space. First, for a certain $\sum_l s_l$, the set owns a permutation symmetry. For instance, when $\sum_l s_l=1$, the Hamiltonians are the same {wherever} $s=1$ is placed. Second, the Hamiltonians have no difference when $\sum_l s_l$ has the same parity. {Thus, the condition that $\sum_l s_l$ is conserved modulo two for partial subspaces can be considered as a kind of symmetry here.} {To sum up, in the dual space, both the permutation and the modulo-two symmetries together result in the holistic degeneracy of the 1D compass model.}

\section{ {Example B: XY-XY Model}}\label{exb}
Except for the 1D compass model, it is easy to find another example of our {proposition}, which is as follows:
\begin{equation}\label{hexb}
H=J\sum_{j=1}^{N} \sigma_j^x\sigma_{j+1}^y.
\end{equation}
This model is {named} the {XY-XY model} here according to its form. {The degenerate property is} almost the same as that of the 1D compass model, except that it depends on the evenness and oddness of the number of sites. {Besides, the method of solving the 1D compass model can be used on this Hamiltonian.}

 {Define $c_j=i(a_j^\dag-a_j)$. Applying the JW transformation, Equation (\ref{hexb}) has the form:
\begin{equation}\label{hexbma}
H=iJ\sum_{j=1}^{N} c_j c_{j+1}.
\end{equation}
Then, applying the Fourier transformation, we obtain:
\begin{equation}\label{hexbck}
H=2J\sum_{k} \sin k c_k^\dag c_{k},
\end{equation}
where $c_k^\dag=\frac{1}{\sqrt {2N}}\sum_j{\exp{(ikj)}c_{j}}$, and $k$ has the values in Equation (\ref{kpbc}), but with $N'=N$. Besides, we have the constraints $c_k^\dag c_{k}+c_{-k}^\dag c_{-k}=1$ and $c_\pi^\dag c_{\pi}+c_{0}^\dag c_{0}=1$. The spectrum of the XY-XY model is gapless and identical to the $J_x=J_y$ case of the 1D compass model. However, a subtle difference exists. The number of sites is even for the 1D compass model; in contrast, that can be even or odd for the XY-XY model. For even $N$, the holistic degeneracy is $d=2^{N/2-1}$. For odd $N$, the $k=\pi/k=0$ state under the APBC/PBC can be directly eliminated since $\sin k=0$; and the spectra under the APBC and PBC are the same according to the trigonometric function $\sin k_{APBC}=\sin (\pi-k_{APBC})=\sin k'_{PBC}$, which means that the state $k_{APBC}$ corresponds to a state $k'_{PBC}$. Therefore, for odd $N$, the holistic degeneracy is $d=2^{(N+1)/2}$. According to Appendix \ref{Appb}, the same results can be obtained using normal fermion operators.}


 {For the purpose of analyzing the symmetry, like the 1D compass model, the XY-XY chain can be mapped to the quantum Ising model \cite{Lieb1961,Katsura1962,Brzezicki2009}. We utilize the eigenstates of $\sigma^y$ and $\sigma^x$ to represent states in sites $2l-1$ and $2l$, respectively. For example, $|Y(X)_1\rangle$ and $|Y(X)_{-1}\rangle$ represent spin-up and -down states in the $y(x)$-axis. For the odd pairs, i.e., {bonds $(2l-1)$--$2l$}, %please confirm if this change is correct. please pay attention to the use of en dash (--) and minus sign ($-$)
 we label the states of $|Y_1, X_1\rangle$ and $|Y_{-1}, X_{-1}\rangle$ with $s_l=1$, and other states with $s_l=0$. Then, define pseudospin operators for odd pairs with $s=1$,

\begin{eqnarray}
\Gamma_l^x&=&|Y_1,X_{1}\rangle\langle Y_{-1},X_{-1}|+| Y_{-1},X_{-1}\rangle\langle Y_1,X_{1}|,\notag
\\
\Gamma_l^z&=&|Y_1,X_{1}\rangle\langle Y_{1},X_{1}|-| Y_{-1},X_{-1}\rangle\langle Y_{-1},X_{-1}|.\label{newspin1}
\end{eqnarray}
By analogy, for odd pairs with $s=0$,
\begin{eqnarray}
\Gamma_l^x&=&|Y_1,X_{-1}\rangle\langle Y_{-1},X_{1}|+| Y_{-1},X_{1}\rangle\langle Y_1,X_{-1}|,\notag
\\
\Gamma_l^z&=&|Y_1,X_{-1}\rangle\langle Y_{1},X_{-1}|-| Y_{-1},X_{1}\rangle\langle Y_{-1},X_{1}|.\label{newspin2}
\end{eqnarray}
}
As is seen in Figure \ref{sites}, we divide all sites into odd pairs; however, in Figure \ref{sites}b, the end site $N$ is isolated when $N$ is odd. Its operators can be individually defined,
\begin{gather}
\Gamma_{\frac{N-1}{2}+1}^x=|Y_1\rangle\langle Y_{-1}|+| Y_{-1}\rangle\langle Y_1|,\notag
\\
\Gamma_{\frac{N-1}{2}+1}^z=|Y_1\rangle\langle Y_{1}|-| Y_{-1}\rangle\langle Y_{-1}|.\label{newspin2}
\end{gather}
Since the isolated site $N$ already has two degrees of freedom, we do not assign the label $s$ to it. Now, each subspace can be labeled by a set $\{s_1,...,s_{N/2\, or\, (N-1)/2}\}$.


\begin{figure}[ H]
\centering
\includegraphics[width= 8.5cm]{sites}
\caption{ {Depictions of the XY-XY model with ({\bf a}) an even number of sites and ({\bf b}) an odd number of sites. Circles around every two sites denote the odd pairs in the text. When the number of sites is odd, the end site $N$ does not form pairs.}}
\label{sites}
\end{figure}


 {By our definitions, the Hamiltonian can be transformed to:
\begin{gather}
H_{even}=\sum_{l=1}^{N/2}(-1)^{s_l}\Gamma_{l}^x+(-1)^{s_l+1}\Gamma_{l}^z\Gamma_{l+1}^z,\notag
\\
H_{odd}=\sum_{l=1}^{(N-1)/2-1}(-1)^{s_l}\Gamma_{l}^x+\sum_{l=1}^{(N-1)/2-1}(-1)^{s_l+1}\Gamma_{l}^z\Gamma_{l+1}^z\\ \notag
+\Gamma_{(N-1)/2+1}^x\Gamma_{1}^z,
\label{gammah}
\end{gather}
where the subscripts ``even'' and ``odd'' denote the parity of the number of sites. Except for the boundary term, other sign factors can be removed by canonical transformations, as a result,
\begin{gather}
H_{even}=\sum_{l=1}^{N/2}\Gamma_{l}^x+\sum_{l=1}^{N/2-1}\Gamma_{l}^z\Gamma_{l+1}^z+(-1)^{N/2+2s_{N/2}-\sum s}\Gamma_{N/2}^z\Gamma_{1}^z,\notag
\\
H_{odd}=\sum_{l=1}^{(N-1)/2-1}\Gamma_{l}^x+\Gamma_{l}^z\Gamma_{l+1}^z+\Gamma_{(N-1)/2}^z\Gamma_{(N-1)/2+1}^z\\ \notag
+\Gamma_{(N-1)/2+1}^x\Gamma_{1}^z.
\label{gammah}
\end{gather}
 Then, $H_{even}$ is exactly solvable, and the calculations can be found in \cite{Brzezicki2007}. Nevertheless, when we apply the JW transformation, the quadratic form of $H_{odd}$ is not accessible because of the $\Gamma^x\Gamma^z$ term. Hence, in this situation, our approach with the help of Majorana fermions shows its priority.}

 { In addition, it is noticed that the boundary term of $H_{odd}$ does not take a sign factor, and at present $d=2^{(N-1)/2}$, according to the previous result $d=2^{(N+1)/2}$, each subspace must own extra two-fold holistic degeneracy. It is not hard to figure out the absent holistic degeneracy. To construct new basic vectors that reduce the size of the subspace, we need to label each basic vector. When $\sigma_x^{2l}\sigma_y^{2l+1}|X_{2l}, Y_{2l+1}\rangle=1\,\, (-1)$, we label the corresponding basic vectors with $t_l=1 \,\,(-1)$, then each basic vector of a specific subspace is labeled by a set $\{n_1,...,n_{(N-1)/2}\}$.
It is straightforward to realize that two basic vectors share a common set. We now construct new basic vectors by linearly combining the two old vectors with the same label,
\begin{gather}
\nu_{even}^{new}=\nu_{1}^{old}+\nu_{2}^{old},\notag
\\
\nu_{odd}^{new}=\nu_{1}^{old}-\nu_{2}^{old},
\label{newv}
\end{gather}
where the subscripts ``even'' and ``odd'' denote the parity of new basic vectors. Letting $H_{odd}$ act on new vectors, it is found that the subspace is divided by the parity of new vectors, and the Hamiltonian matrices are the same. We take the three-site system as an example to display the effectiveness of our approach. One subspace of the three-site model includes the following basic vectors:
\begin{equation}\label{}
|Y_1, X_1, Y_1\rangle,\, |Y_1, X_1, Y_{-1}\rangle,\, |Y_{-1}, X_{-1}, Y_1\rangle,\, |Y_{-1}, X_{-1}, Y_{-1}\rangle.
\end{equation}
A new smaller subspace is made up of:
\begin{equation}\label{}
|Y_1, X_1, Y_1\rangle+|Y_{-1}, X_{-1}, Y_{-1}\rangle,\, |Y_{-1}, X_{-1}, Y_1\rangle+|Y_1, X_1, Y_{-1}\rangle.
\end{equation}
Obviously, $H_{odd}$ does not mix both vectors with outside vectors. Finally, the absent holistic degeneracy is located.
 }

 {In a word, similar to the 1D compass model, the holistic degeneracy of the XY-XY model mostly comes from the different configurations of odd pairs or the parity of $s_l$. In particular, when $N$ is odd, two-fold holistic degeneracy is from the parity of new basic vectors.}

\section{Conclusions}
To sum up, we have theoretically proposed that, {with no redundancy,} the JW transformation can exactly map a periodic spin chain to a periodic, and an anti-periodic {chain} of lattice fermions when the Hamiltonians in the fermion representation can be reduced to lower dimensions. The conditions include that the Hamiltonian merely involves the nearest-neighbor interactions and does not change the parity of the number of fermions. In these systems, the holistic degeneracy is defined, and it is found to be {$2^x$-fold where $x$ is a positive integer}. These systems are further classified according to the folds of holistic degeneracy, and possible high degeneracy exists in these systems. In addition, we take the 1D compass model {and the XY-XY model} as the examples to demonstrate the degenerate properties of these systems. {In both models, their holistic degeneracy grows exponentially with the size of systems.} It is {remarkable that by our work, complete energy spectra can be totally determined by the reduced fermion Hamiltonian with little effort,} although other advantages are not clear yet.

%%%%%%%%%%%%%%%%%%%%%%%%%%%%%%%%%%%%%%%%%%
%\section{Patents}
%This section is not mandatory, but may be added if there are patents resulting from the work reported in this manuscript.

%%%%%%%%%%%%%%%%%%%%%%%%%%%%%%%%%%%%%%%%%%
\vspace{6pt}

%%%%%%%%%%%%%%%%%%%%%%%%%%%%%%%%%%%%%%%%%%
%% optional
%\supplementary{The following are available online at \linksupplementary{s1}, Figure S1: title, Table S1: title, Video S1: title.}

% Only for the journal Methods and Protocols:
% If you wish to submit a video article, please do so with any other supplementary material.
% \supplementary{The following are available at \linksupplementary, Figure S1: title, Table S1: title, Video S1: title. A supporting video article is available at doi: link.}

%%%%%%%%%%%%%%%%%%%%%%%%%%%%%%%%%%%%%%%%%%
%\authorcontributions{conceptualization, F. L. Ning; formal analysis, Cui Ding; investigation, Cui Ding, Kai Wang, Haojie Zhang, Shengli Guo, Guoxiang Zhi, Yilun Gu and LiCheng Fu; writing—original draft preparation, Cui Ding; writing—review and editing, F. L. Ning; visualization, Cui Ding and Guoxiang Zhi; supervision, F. L. Ning.}

%%%%%%%%%%%%%%%%%%%%%%%%%%%%%%%%%%%%%%%%%%
%\funding{This research was funded by Key Projects for Research and Development of China (No. 2016FYA0300402), National Natural Science Foundation of China (No.11574265), Natural Science Foundation of Zhejiang Province (No. LR15A040001), and the Fundamental Research Funds for the Central Universities (No. 2017FZA3003).}


%%%%%%%%%%%%%%%%%%%%%%%%%%%%%%%%%%%%%%%%%%
\acknowledgments{S.F. thanks {Prof. Hai-Qing Lin and Mr. Jian Lee} for useful discussions. S.F. acknowledges the support from {Beijing Computational Science Research Center} and {Prof. Hai-Qing Lin}.}

%%%%%%%%%%%%%%%%%%%%%%%%%%%%%%%%%%%%%%%%%%


%%%%%%%%%%%%%%%%%%%%%%%%%%%%%%%%%%%%%%%%%%
%% optional
%\abbreviations{The following abbreviations are used in this manuscript:\\
%
%\noindent
%\begin{tabular}{@{}ll}
%DMSs & Diluted Magnetic Semiconductors\\
%$\mu$SR & Muon Spin Relaxation\\
%NMR & Nuclear Magnetic Resonance\\
%RKKY & Ruderman - Kittel - Kasuya - Yosida\\
%BMP & Bound Magnetic Polaron\\
%FC & Field Cooling\\
%ZFC & Zero Field Cooling
%\end{tabular}}

%% optional
%\appendixtitles{yes} %Leave argument "no" if all appendix headings stay EMPTY (then no dot is printed after "Appendix A"). If the appendix sections contain a heading then change the argument to "yes".
%\appendixsections{multiple} %Leave argument "multiple" if there are multiple sections. Then a counter is printed ("Appendix A"). If there is only one appendix section then change the argument to "one" and no counter is printed ("Appendix").
\appendix
\section{Diagonalization of Equation (\ref{mfermion})}\label{Appa}


The Hamiltonian for the 1D compass model is:
\begin{equation}\label{A1dhamito}
H=J_x\sum_{l=1}^{N/2}{\sigma^x_{2l-1}\sigma^x_{2l}}+J_y\sum_{l=1}^{N/2}{\sigma^y_{2l}\sigma^y_{2l+1}},
\end{equation}
where $J_x$ is the interacting strength {for the odd bonds and $J_y$ is for the other half {of} bonds}.
A toy model for this Hamiltonian is depicted in Figure \ref{figtoy1}a. We use the JW transformation to transform the spin Hamiltonian and have:
% The JW transformation is as follows
%\begin{gather}\label{AJWT}
%\sigma_m^+=2a_m^\dag \exp{(i\pi\sum_{m>n}{a^\dag_{n}a_n})},\notag
%\\
%\sigma_m^-=2a_m \exp{(-i\pi\sum_{m>n}{a^\dag_{n}a_n})},
%\end{gather}
%
%where $\sigma^{\pm}=\sigma^x\pm i \sigma^y$. Now we have
\begin{align}\label{nfermion}
H=&J_x\sum_{l=1}^{N/2}{(a_{2l-1}^{\dag}a_{2l}^\dag+a_{2l-1}^{\dag}a_{2l}+H.c.)}\notag
\\
&+J_y\sum_{l=1}^{N/2}{(-a_{2l}^{\dag}a_{2l+1}^\dag+a_{2l}^{\dag}a_{2l+1}+H.c.)}. %
\end{align}
Taking advantage of the Majorana fermion operators, the Hamiltonian has a more concise form:
\begin{equation}\label{Amfermion}
H=-i\sum_{l=1}^{N/2}{(J_x c_{2l-1}c_{2l}-J_yc_{2l}c_{2l+1})}, %
\end{equation}
where $c_{2l-1}=i(a^\dag_{2l-1}-a_{2l-1})$ and $c_{2l}=a^\dag_{2l}+a_{2l}$ are Majorana fermion operators. The corresponding illustration is given in Figure \ref{figtoy1}b. It is known that a normal fermion can be described by a pair of Majorana fermions, and accordingly, each Majorana fermion has $\sqrt2$ degrees of freedom. Interestingly, the Hamiltonian here does not include the other half Majorana fermions d%should it be italics? please check the conventions for the math notations throughout
's which pair with c's. {Through counting the number of the Majorana fermions, it is found that the dimensions of the Hamiltonian have been reduced compared with the original spin Hamiltonian.}

%For a $N$-site chain, we consider the periodic boundary condition $\bm{\sigma}_{N+1}=\bm{\sigma}_1$ in the spin %Hamiltonian. Under this condition, we have $a^\dag_{N+1}=-a^\dag_1$ and $a_{N+1}=-a_1$ in Equation %(\ref{nfermion}), $c_{N+1}=-c_1$ in Equation (\ref{Amfermion}) for states with even normal fermions; and %$a^\dag_{N+1}=a^\dag_1$ and $a_{N+1}=a_1$ in Equation (\ref{nfermion}), $c_{N+1}=c_1$ in Equation (\ref{Amfermion}) %for states with odd normal fermions. The former condition is anti-periodic boundary condition (APBC), and the %latter one is periodic boundary condition (PBC) for the fermion Hamiltonian. Besides, the character that the %Hamiltonian Equation (\ref{nfermion}) changes the number of normal fermions in pairs guarantees no interaction %between even- and odd-occupation basic vectors, therefore, basic vectors with even normal fermions construct a %subspace with $N/2$ dimensions, and so do the other basic vectors. Thus, the Hamiltonian with APBC (PBC) should be %solved in the subspace with even (odd) normal fermions. Normally the dimension of the subspace does not match that %of the corresponding Hamiltonian, thus redundancy is introduced. For small systems, the redundancy can be dropped %off by a project operator; for large systems, the approximate result is always adopted in either APBC or PBC.


\begin{figure}[H]
\centering
\includegraphics[width= 8.5cm]{1dmodel}
\caption{ ({\bf a}) A toy model for the 1D compass model and ({\bf b}) the Majorana representation of this model. $J_{x}$ and $J_{y}$ are the interacting strength between two nearest sites, and $J_x$ and $J_y$ are for the $x$-axis and $y$-axis interactions, respectively. {c's and d's} are paired Majorana states; remarkably, c's are exclusively involved in the Majorana representation.} % please confirm the italic format
\label{figtoy1}
\end{figure}


To diagonalize the Hamiltonian, we mark odd sites {with A's}, and even sites {B's}. Thus, $N$ sites are divided into $N/2$ cells. Equation (\ref{Amfermion}) becomes:
\begin{align}\label{mferab}
H=&-i\sum_{l=1}^{N/2}{(J_x c_{l, A}c_{l, B}-J_yc_{l, B}c_{l+1, A})}\notag
\\ %
=&-\frac{i}{2}\sum_{l=1}^{N/2}{(J_x c_{l, A}c_{l, B}-J_x c_{l, B}c_{l, A}-J_yc_{l, B}c_{l+1, A} }\notag \\
&\quad+J_y c_{l+1, A}c_{l, B}),
\end{align}
where $l$ is the {label} of cells. Equation (\ref{mferab}) can be described by:
\begin{equation}\label{}
 \Psi^\dag H_{ij}\Psi,
\end{equation}
where $\Psi^\dag=\left(
 \begin{array}{cccc}
 \cdot\cdot\cdot & c_{jA} & c_{jB} & \cdot\cdot\cdot \\
 \end{array}
\right)$, $\Psi=\left(
   \begin{array}{c}
   \vdots \\
   c_{jA} \\
   c_{jB}\\
   \vdots\\
   \end{array}
  \right)
$, and $H_{ij}$ is a skew symmetric matrix.
Further, transform Equation (\ref{mferab}) to the momentum space by the Fourier transformation:
\begin{align}\label{phi}
 \Phi^\dag =&\Psi^\dag \frac{1}{\sqrt 2}F^\dag, \notag\\
 \Phi =&\frac{1}{\sqrt 2}F\Psi,
\end{align}
where $\Phi^\dag=\left(
 \begin{array}{cccc}
 \cdot\cdot\cdot & c_{kA}^\dag & c_{kB}^\dag & \cdot\cdot\cdot \\
 \end{array}
\right)$, $\Phi=\left(
   \begin{array}{c}
   \vdots \\
   c_{kA} \\
   c_{kB}\\
   \vdots\\
   \end{array}
  \right)
$ and $F^\dag/F$ is the matrix of a normal Fourier transformation from the real space to the momentum space. The purpose of multiplying a factor $\frac{1}{\sqrt2}$ before $F^\dag/F$ is to make the operators in the momentum space satisfy the anti-commutation relation. $c_{k,A/B}^\dag$ and $c_{k,A/B}$ are as follows:
\begin{align}\label{ck}
 c_{k,A/B}^\dag=\frac{1}{\sqrt {N}}\sum_j{\exp{(ikR_j)}c_{j,A/B}} ,\notag \\
 c_{k,A/B}=\frac{1}{\sqrt {N}}\sum_j{\exp{(-ikR_j)}c_{j,A/B}} .
\end{align}
It can be verified that $c_{k,A/B}^\dag= c_{-k,A/B}$ and $\{c_{k\nu}^\dag,c_{k'\nu'}\}=\delta_{kk'}\delta_{\nu\nu'}$ ($\nu=A,B$). In the momentum space, the Hamiltonian becomes:
\begin{equation}\label{hk}
 H= \Phi^\dag F2H_{ij}F^\dag\Phi,
\end{equation}
which can be divided into blocks. Each block is written as:
\begin{equation}\label{hkb}
H_k=\left(
  \begin{array}{cc}
  c_{kA}^\dag & c_{kB}^\dag \\
  \end{array}
 \right)
 \left(
  \begin{array}{cc}
  0 & f(k) \\
  f(k)^\ast & 0 \\
  \end{array}
 \right)
 \left(
  \begin{array}{c}
  c_{k A } \\
  c_{kB} \\
  \end{array}
 \right)
\end{equation}
where $f(k)=-iJ_x-iJ_y\exp{(-ik)}$.
Then, exploiting the Bogoliubov transformation to diagonalize $H_k$, we obtain:
\begin{equation}\label{hkd}
H_k=|f(k)|\alpha_{k}^\dag \alpha_{k}-|f(k)|\beta_{k}^{\dag}\beta_{k},
\end{equation}
where $\alpha_{k}^\dag=\frac{f(k)}{\sqrt2|f(k)|}c_{kA}^\dag+\frac{1}{\sqrt2}c_{kB}^\dag$ and $\beta_{k}^\dag=\frac{f(k)}{\sqrt2|f(k)|}c_{kA}^\dag-\frac{1}{\sqrt2}c_{kB}^\dag$. Meanwhile, considering the relation $c_{k,A/B}^\dag= c_{-k,A/B}$, we have $\alpha_{k}^\dag \alpha_{k}+\beta_{-k}^\dag \beta_{-k}=1$ and $\beta_{k}^\dag \beta_{k}+\alpha_{-k}^\dag \alpha_{-k}=1$. {In particular}, when $k=0$ or $\pi$, we have $\alpha_{k}^\dag \alpha_{k}+\beta_{k}^\dag \beta_{k}=1$. These constraints are radically attributed to the Majorana fermions' own symmetry, and {each} constraint cut off a half {of} solutions of the Hamiltonian. With the constraints, the diagonal Hamiltonian is:
\begin{equation}\label{hkd1}
H=\sum_{k}2E_k(\alpha_{k}^\dag \alpha_{k}-\frac{1}{2}),
\end{equation}
where $E_k=|f(k)|=\sqrt{J_x^2+J_y^2+2\cos{k}J_xJ_y}$ is the energy of $k$-mode quasiparticle states. Under the adopted form, $-E_k$ corresponds to $-k$-mode quasiparticle states, except for $k=0$ and $\pi$.

Till Equation (\ref{hkd1}), although we transform the Hamiltonian under the periodic condition $\bm{\sigma}_{N+1}=\bm{\sigma}_1$, the boundary conditions for the fermion Hamiltonian are not fixed yet. Now, we assign values to $k$ to {identify} the boundary conditions. For the APBC, the momentum $k$ has values:
\begin{equation}\label{kapbc}
\pm \frac{1}{N'}\pi,\, \pm \frac{3}{N'}\pi,\, \pm \frac{5}{N'}\pi,\cdots,\pm \frac{N'-1}{N'}\pi (\text{even}\,N'),\,
 \pi \,(\text{odd}\, N');\notag
\end{equation}
for the PBC, $k$ has values:
\begin{equation}\label{kpbc}
0,\,\pm \frac{2}{N'}\pi,\, \pm \frac{4}{N'}\pi
%,\, \pm \frac{6}{N'}\pi
,\cdots,\pm \frac{N'-1}{N'}\pi\,(\text{odd}\, N'),\,
 \pi \,(\text{even}\, N'),
\end{equation}
where $N'=N/2$. For each boundary condition, there must be $N'$ quasiparticle states, and Equation~(\ref{hkd1}) will give $2^{N'}$ energy levels for a $N$-site system.

\section{Proof of the {Holistic Degeneracy} in the 1D Compass Model}\label{Appb}
Except for the discussion on degeneracy, the following method is {from} \cite{Sun2009}. Similar to Appendix \ref{Appa}, the odd site is marked {with} A, and the even site B. Equation (\ref{nfermion}) becomes:
\begin{align}\label{Bnfermion}
H=&J_x\sum_{l=1}^{N/2}{(a_{l,A}^{\dag}a_{l,B}^\dag+a_{l,A}^{\dag}a_{l,B}+H.c.)}\notag
\\
&+J_y\sum_{l=1}^{N/2}{(-a_{l,B}^{\dag}a_{l+1,A}^\dag+a_{l,B}^{\dag}a_{l+1,A}+H.c.)}. %
\end{align}
Then, using the Fourier transformation $c_{l, A(B)}^\dag=\frac{1}{\sqrt {N/2}}\sum_k{\exp{(-ikR_l)}c_{k,A/B}}^\dag$, Equation (\ref{Bnfermion}) is directly transformed to the momentum space:
\begin{align}\label{Bnfermionk}
&\quad H=J_x\sum_{k}{(a_{k,A}^{\dag}a_{-k,B}^\dag+a_{k,A}^{\dag}a_{k,B}+H.c.)}\notag
\\
&+J_y\sum_{k}{(-\exp{(ik)}a_{k,B}^{\dag}a_{-k,A}^\dag+\exp{(ik)}a_{k,B}^{\dag}a_{k,A}+H.c.)}, %
\end{align}
in which $-k$ should be modified to $k$ when $k=\pi$.
Define:
\begin{align}\label{Bhk}
&H_k =J_x(a_{k,A}^{\dag}a_{-k,B}^\dag+a_{k,A}^{\dag}a_{k,B}+H.c.)\notag
\\
&+J_y(-\exp{(ik)}a_{k,B}^{\dag}a_{-k,A}^\dag+\exp{(ik)}a_{k,B}^{\dag}a_{k,A}+H.c.); %
\end{align}
and:
\begin{equation}\label{Wk}
W(k)=H_k+H_{-k}\, (0<k<\pi).
\end{equation}

Since the Hamiltonian has already been decoupled in the $k$ representation, we have:
\begin{equation}\label{Hwk}
H=H_0+H_\pi+\sum_{0<k<\pi}W(k),
\end{equation}
and each part can be solved independently. For each $W(k)$, its Hilbert space has sixteen dimensions, and the Hilbert space can be divided into two subspaces with the same dimensions.

Firstly, we solve $W(k)$ in the subspace with even parity. The subspace with even parity has the following basic vectors:
\begin{align}\label{BBv}
&|0\rangle,\quad a_{k,A}^\dag a_{-k,A}^\dag|0\rangle,\quad a_{k,B}^\dag a_{-k,B}^\dag|0\rangle, \notag
\\
& a_{k,A}^\dag a_{k,B}^\dag|0\rangle, \quad a_{k,A}^\dag a_{-k,B}^\dag|0\rangle, \quad a_{-k,A}^\dag a_{k,B}^\dag|0\rangle, \notag
\\
&a_{-k,A}^\dag a_{-k,B}^\dag|0\rangle,\quad a_{k,A}^\dag a_{-k,A}^\dag a_{k,B}^\dag a_{-k,B}^\dag|0\rangle.
\end{align}
The eigenvalues of $W(k)$ in this subspace are:
\begin{align}\label{BEWK1}
& \lambda_1^e=2E(k), \lambda_2^e=-2E(k), \lambda_3^e=0, \lambda_4^e=0,\notag
\\
& \lambda_5^e=2E(k), \lambda_6^e=-2E(k), \lambda_7^e=0, \lambda_8^e=0,
\end{align}
where $E(k)$ is identical to that in Appendix \ref{Appa}.
In the subspace with odd parity, $W(k)$ has eigenvalues:
\begin{align}\label{BEWK2}
& \lambda_1^o=2E(k), \lambda_2^o=-2E(k), \lambda_3^o=0, \lambda_4^o=0,\notag
\\
& \lambda_5^o=2E(k), \lambda_6^o=-2E(k), \lambda_7^o=0, \lambda_8^o=0.
\end{align}
One needs to be careful with the point that it is the subspace of $W(k)$ rather than $H$. Therefore, Equations (\ref{BEWK1}) and (\ref{BEWK2}) are valid for both boundary conditions. Similarly, for $H_0$,
\begin{align}\label{BEWK20}
& \lambda_1^e=E(0), \lambda_2^e=-E(0),\notag
\\
& \lambda_1^o=E(0), \lambda_2^o=-E(0);
\end{align}
for $H_\pi$,
\begin{align}\label{BEWK2pi}
& \lambda_1^e=E(\pi), \lambda_2^e=-E(\pi),\notag
\\
& \lambda_1^o=E(\pi), \lambda_2^o=-E(\pi).
\end{align}
Whether the terms $H_0$ and $H_\pi$ exist in the Hamiltonian depends on the boundary conditions (see Equation (\ref{kpbc}) in Appendix \ref{Appa}).

Now, we show how to find all eigenvalues of the 1D compass model. Defining a set of quantum numbers
$\{q_0,\cdots ,q_k ,\cdots,q_\pi\}$, we let $q_k=1$ when selecting an eigenvalue corresponding to odd parity ($\lambda^o$) from $W(k)$ and let $q_k=0$ when selecting a $\lambda^e$. {For the case of the APBC and even $N/2$}, we need to satisfy the condition of the even-parity eigenstate; hence, we must have:
\begin{equation}\label{Bsumq}
 \mod(\sum_k q_k,2)=0.
\end{equation}
There are $N/4$ $q$'s in the set $\{q_k\}$ $(0\leq k\leq\pi)$; thus, $2^{N/4-1}$ out of $2^{N/4}$ cases are appropriate, like the case $\sum_k q_k=0$, which means that {always} choosing a $\lambda^e$ from {each} $W(k)$. Besides, it is {also} noticed that each eigenvalue of $W(k)$ in a certain subspace has two-fold holistic degeneracy; therefore, we obtain $2^{N/4}$-fold holistic degeneracy for each appropriate case of $\{q_k\}$. {Incorporating} both factors leads to the $2^{N/2-1}$-fold holistic degeneracy. The analysis above can be extended to other situations straightforwardly, and the {conclusions are} consistent.


%%%%%%%%%%%%%%%%%%%%%%%%%%%%%%%%%%%%%%%%%%
% Citations and References in Supplementary files are permitted provided that they also appear in the reference list here.

%=====================================
% References, variant A: internal bibliography
%=====================================
%\reftitle{References}
\begin{thebibliography}{999}
% Reference 1
%\bibitem[Author1(year)]{ref-journal}
%Author1, T. The title of the cited article. {\em Journal Abbreviation} {\bf 2008}, {\em 10}, 142-149, doi:xxxxx.
% Reference 2
%\bibitem[Author2(year)]{ref-book}
%Author2, L. The title of the cited contribution. In {\em The Book Title}; Editor1, F., Editor2, A., Eds.; Publishing House: City, Country, 2007; pp. 32-58, ISBN.
%\textcolor[rgb]{1,0,0}{}
\bibitem{Jordan1928}Jordan, P.; Wigner, E. \"{U}ber das Paulische \"{A}quivalenzverbot. \emph{Z. Phys.} {\bf 1928}, {\it 47}, 631--651. {(In German)} % please confirm

\bibitem{Lieb1961}Lieb, E.; Schultz, T.; Mattis, D. Two soluble models of an antiferromagnetic chain. \emph{Ann. Phys.} {\bf 1961}, {\it 16}, 407--466.

\bibitem{Schultz1964}Schultz, T.; Mattis, D.; Lieb, E. Two-Dimensional Ising Model as a Soluble Problem of Many Fermions. \emph{Rev. Mod. Phys.} {\bf 1964}, {\it 36}, 856--871.

\bibitem{Mattis}Mattis, D.C. \emph{The Theory of Magnetism Made Simple}; World Scientific: Beijing, China, 2006.

\bibitem{Fradkin1989}Fradkin, E. Jordan--Wigner transformation for quantum-spin systems in two dimensions and fractional statistics. \emph{Phys. Rev. Lett.} {\bf 1989}, {\it 63}, 322--325.

\bibitem{Wang1992-1}{Wang, Y.R. Low-dimensional quantum antiferromagnetic Heisenberg model studied using Wigner-Jordan transformations}. \emph{Phys. Rev. B} {\bf 1992}, {\it 46}, 151--161.
% accorrding to our layout style, this one reference should be splited into three, please check

 \bibitem{Wang1992-2}Wang, Y.R. Spin-liquid state of Wigner-Jordan fermions of the quantum antiferromagnetic Heisenberg model on a triangular lattice. \emph{Phys. Rev. B} {\bf 1992}, {\it 45}, 12604--12607.

 \bibitem{Wang1992-3}Wang, Y.R. Ground state of the two-dimensional antiferromagnetic Heisenberg model studied using an extended Wigner-Jordon transformation. \emph{Phys. Rev. B} {\bf 1991}, {\it 43}, 3786--3789.

\bibitem{Azzouz1993}Azzouz, M. Interchain-coupling effect on the one-dimensional spin-1/2 antiferromagnetic Heisenberg model. \emph{Phys. Rev. B} {\bf 1993}, {\it 48}, 6136--6140.

\bibitem{Huerta1993}Huerta L.; Zanelli, J. Bose-Fermi transformation in three-dimensional space. \emph{Phys. Rev. Lett.} {\bf 1993}, {\it 71}, 3622--3624.

\bibitem{Kochmanski1995}Kochma\'{n}ski, M.S. Jordan--Wigner transformations and their generalizations for multidimensional systems. \emph{arXiv} {\bf1998}, arXiv:cond-mat/9807388.
% J. Tech. Phys. {\bf36}, 485 (1995). the website is not available

\bibitem{Batista2001}Batista C.D.; Ortiz, G. Generalized Jordan--Wigner Transformations. \emph{Phys. Rev. Lett.} {\bf 2001}, {\it 86}, 1082--1085.

\bibitem{Bock2001}Bock, B.; Azzouz, M. Generalization of the Jordan--Wigner transformation in three dimensions and its application to the Heisenberg bilayer antiferromagnet. \emph{Phys. Rev. B} {\bf2001}, {\it 64}, 054410.

%\bibitem{NN} We limit the problem to the nearest neighbor interaction, such that merely the $N-1$ bond is concerned.
%\bibitem{sub} The basic vectors in each subspace are always not like that defined in Equation (\ref{basicv}), because basic vectors in Equation (\ref{basicv}) have been mixed by the Hamiltonian.
\bibitem{Brzezicki2007}Brzezicki, W.; Dziarmaga, J.; Ole\'{s}, A.M. Quantum phase transition in the one-dimensional compass model. \emph{Phys. Rev. B} {\bf 2007}, {\it 75}, 134415.

\bibitem{Bertsch2009} {Bertsch, G.; Dobaczewski, J.; Nazarewicz, W.; Pei, J. Hartree-Fock-Bogoliubov theory of polarized Fermi systems. \emph{Phys. Rev. A} {\bf 2009}, {\it 79}, 043602.}

\bibitem{Bohr1975} {Bohr, A.; Mottelson, B.R. \emph{Nuclear Structure, Volume II}; W.A. Benjamin: New York, NY, USA, 1975.}

\bibitem{Balents2010} {Balents, L. Spin liquids in frustrated magnets. \emph{Nature} {\bf2010}, {\it464}, 199-208.}

\bibitem{Nussinov2015}Nussinov, Z.; van den Brink, J. Compass models: Theory and physical motivations. \emph{Rev. Mod. Phys.} {\bf 2015}, {\it87}, 1--59.

\bibitem{Brzezicki2009} {Brzezicki, W.; Ole\'{s}, A.M. Quantum Phase Transition in the One-Dimensional XZ Model. \emph{Acta Phys. Polon. A} {\bf2009}, {\it115}, 162--164.}

\bibitem{Jafari2017-1}{Jafari, R.; Johannesson, H. Loschmidt Echo Revivals: Critical and Noncritical.} \emph{Phys. Rev. Lett.} {\bf 2017 }, {\it118}, 015701.
% we splited this ref into three, please check

\bibitem{Jafari2017-2}Jafari, R.; Johannesson, H. Decoherence from spin environments: Loschmidt echo and quasiparticle excitations. \emph{Phys. Rev. B.} {\bf 2017}, {\it96}, 224302.

\bibitem{Jafari2017-3} Jafari, R. Quench dynamics and ground state fidelity of the one-dimensional extended quantum compass model in a transverse field. \emph{J. Phys. A: Math. Theor.} {\bf 2016}, {\it49}, 185004.


\bibitem{Feng2007}Feng, X.-Y.; Zhang, G.-M.; Xiang, T. Topological Characterization of Quantum Phase Transitions in a Spin-1/2 Model. \emph{Phys. Rev. Lett.} {\bf 2007}, {\it98}, 087204.


\bibitem{You2008} You, W.-L.; Tian, G.-S. Quantum phase transition in the one-dimensional compass model using the pseudospin approach. \emph{Phys. Rev. B} {\bf 2008}, {\it78}, 184406.

\bibitem{Sun2009} Sun, K.-W.; Chen, Q.-H. Quantum phase transition of the one-dimensional transverse-field compass model. \emph{Phys. Rev. B} {\bf 2009}, {\it80}, 174417.


\bibitem{Katsura1962} {Katsura, S. Statistical Mechanics of the Anisotropic Linear Heisenberg Model. \emph{Phys. Rev.} {\bf1962}, {\it127}, 1508--1518.}


%\bibitem{order} Comparing the order of applying different transformations between the method in Reference \cite{Brzezicki2007} and our formalism, one would find the difference.

%\bibitem{Urban}Urban P., and Sperlich G. (1975) Solid State Commun. \textbf{16}, 927.

\end{thebibliography}
\end{document}
