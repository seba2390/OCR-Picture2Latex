
% ********************************************************************************************

\section{The Riesz transform on the tractable trees: transference estimates}\label{sec99}

 In all this section we assume that $R\in\MDW$ is such that $R\in\Trc$, i.e., $\TT(e'(R))$ is tractable, and we consider the measure $\eta$ constructed in the previous section.
Essentially, our objective is to transfer the lower estimate we obtained for $\RR\eta$ in Lemma \ref{lemrieszeta} to 
the Haar coefficients of
$\RR\mu$ associated with cubes close to $\TT(e'(R))$.

\subsection{The operators \texorpdfstring{$\RR_{\TT_\Reg}$, $\RR_{\wt \TT}$,  and $\Delta_{\wt \TT}\RR$}{R\_{TReg}, R\_{T},  and Delta\_{T}R}}\label{sec7.1}

To simplify notation, in this section we will write
$$\End=\End(e'(R)), \quad \;\Reg = \Reg(e'(R)),\; \quad \;\TT = \TT(e'(R)),\quad\,\text{and}\quad\; \TT_\Reg = \TT_\Reg(e'(R)).$$
We need to consider an enlarged version of the generalized tree $\TT$, due to some technical difficulties that arise because the cubes from $\Neg:= \Neg(e'(R))\cap\End$ are not $\PP$-doubling.
To this end, denote by $\Reg_\Neg$ the family of the cubes from $\Reg$ which are contained in some cube from $\Neg$.
Let $\sD_\Neg$ be the subfamily of the cubes $P\in\Reg_\Neg$ for which there exists some
$\PP$-doubling cube $S\in\TT_\Reg$ that contains $P$. By the definition of $\Neg$, such cube $S$ is contained in the cube from $Q\in\Neg$ such that $P\subset Q$. 
We also denote by $\sM_\Neg$ the family of maximal $\PP$-doubling cubes which belong to $\TT_\Reg$ and are contained in some
cube from $\Neg$, so that, in particular, any cube from $\sD_\Neg$ is contained in some cube from $\sM_\Neg$.

We define
$$\wt\End = (\End \setminus \Neg) \cup \sM_\Neg,$$
and we let $\wt \TT=\wt \TT(e'(R))$ be the family of cubes which belong to $\TT_\Reg$ and are not strictly contained in any cube from $\wt\End$.
Further, we write
\begin{equation}\label{eqdef*f}
Z = Z(e'(R)) = e'(R)\setminus \bigcup_{Q\in\End} Q\quad \text{ and }\quad\wt Z = \wt Z(e'(R)) = e'(R)\setminus \bigcup_{Q\in\wt\End} Q
\end{equation}
Then we denote
$$\RR_{\TT_\Reg}\mu(x) = \sum_{Q\in\Reg} \chi_Q(x)\,\RR(\chi_{2R\setminus 2Q}\mu)(x),$$
$$\RR_{\wt \TT}\mu(x) = \sum_{Q\in\wt\End} \chi_Q(x)\,\RR(\chi_{2R\setminus 2Q}\mu)(x),$$
and
$$\Delta_{\wt\TT}\RR\mu(x) = \sum_{Q\in\wt\End} \chi_Q(x)\,\big(m_{\mu,Q}(\RR\mu) - m_{\mu,2R}(\RR\mu)\big)
+ \chi_{Z}(x) \big(\RR\mu(x) -  m_{\mu,2R}(\RR\mu)\big)
.$$
Here $\RR\mu(x)$ is understood in the principal value sense (which exists $\mu$-a.e.\ because we assume that
$\mu$ has polynomial growth and that $\RR_*\mu<\infty$ $\mu$-a.e.

Remark that the cubes from $\Reg$ have the advantage over the cubes from $\wt \End$ that their size changes smoothly so that, for example, neighboring cubes have comparable side lengths. However, they need not be $\PP$-doubling or doubling, unlike the cubes from $\wt \End$. 

We define
$$\QQ_\Reg(Q) = \sum_{P\in\Reg} \frac{\ell(P)}{D(P,Q)^{n+1}}\,\mu(P),$$
where 
$$D(P,Q) = \ell(P) + \dist(P,Q) + \ell(Q).$$
The coefficient $\QQ_\Reg(Q)$ will be used to bound some ``error terms'' in our transference 
arguments. We will see later how they can be estimated in terms of the coefficients $\PP(Q)$ by duality. Notice that, unlike $\PP(Q)$, the coefficient $\QQ_\Reg(Q)$ depends on the family
$\Reg$.

\vv


\begin{lemma}\label{lemaprox1}
For any $Q\in\Reg$ such that $(Q\cup \frac12 B(Q))\cap V_4\neq\varnothing$ and $x\in Q$, $y\in \frac12B(Q)$,
$$\big|\RR_{\TT_\Reg}\mu(x) - \RR\eta(y)\big| \lesssim \Theta(R) + \PP(Q) + \QQ_\Reg(Q).$$
\end{lemma}

\begin{proof}
By the triangle inequality, for $x$, $y$ and $Q$ as in the lemma, we have
\begin{align*}
\big|\RR_{\TT_\Reg}\mu(x) - \RR\eta(y)\big| & \leq \big|\RR_\mu\chi_{2R\setminus e'(R)}(x)\big| 
+ \big|\RR_\mu\chi_{e'(R)\setminus 2Q}(x) - \RR_\mu\chi_{e'(R)\setminus 2Q}(x_Q)\big|\\
& \quad 
+ \big|\RR_\mu\chi_{2Q\setminus Q}(x_Q)\big| 
+\big|\RR_\mu\chi_{e'(R)\setminus Q}(x_Q) - \RR_\eta \chi_{(\frac12B(Q))^c}(x_Q)\big|\\
&\quad + \big|\RR_\eta \chi_{(\frac12B(Q))^c}(x_Q) - \RR_\eta \chi_{(\frac12B(Q))^c}(y)\big| + \big|\RR_\eta \chi_{\frac12B(Q)}(y)\big|\\
& = I_1+ \ldots + I_6.
\end{align*}
To estimate $I_1$, notice that, by \rf{eqtec732}, $\dist(Q,\supp\mu\setminus e'(R))\gtrsim\ell(R)$.
Then it follows that
$$I_1\lesssim \frac{\mu(2R)}{\ell(R)^n} \lesssim \Theta(R).$$
Arguing similarly one also gets 
\begin{equation*}
	I_3\lesssim \frac{\mu(2Q)}{\ell(Q)^n}\lesssim \PP(Q).
\end{equation*}
By standard arguments involving continuity of the Riesz kernel, we also deduce that
$$I_2\lesssim\PP(Q),\quad I_5\lesssim\PP(Q).$$
Concerning the term $I_6$, using that $\eta$ is a constant multiple of $\LL^{n+1}$ on 
$\frac12B(Q)$, we get
$$I_6\leq \int_{\frac12B(Q)}\frac1{|x-y|^n}\,d\eta(y)\lesssim \frac{\eta(\tfrac12B(Q))}{\ell(Q)^n}
\lesssim \PP(Q).$$

Finally we deal with the term $I_4$. To this end, recall that $\mu(P) = \eta(\frac12B(P))$ for all 
$P\in \Reg$, and so
\begin{align*}
I_4 & \leq \sum_{P\in\Reg:P\neq Q} \left| \int K(x_Q-z)\,d\big(\eta\rest_{\frac12B(P)} - \mu\rest_P\big)(z)\right|\\
& \leq \sum_{P\in\Reg:P\neq Q}  \int |K(x_Q-z)-K(x_Q-x_P)|\,d\big(\eta\rest_{\frac12B(P)} + \mu\rest_P\big)(z)\\
& \lesssim \sum_{P\in\Reg:P\neq Q} \frac{\ell(P)}{|x_Q-x_P|^{n+1}}\,\mu(P)\approx \sum_{P\in\Reg:P\neq Q} \frac{\ell(P)}{D(Q,P)^{n+1}}\,\mu(P) \leq \QQ_\Reg(Q). 
\end{align*}
For the estimates in the last line we took into account the properties of the family $\Reg$ described 
in Lemma \ref{lem74}.
Gathering the estimates above, the lemma follows.
\end{proof}


\vv
\begin{lemma}\label{lemaprox2}
For any $Q\in\wt\End$ and $x,y\in Q$,
$$\big|\RR_{\wt\TT}\mu(x) - \Delta_{\wt\TT}\RR\mu(y)\big| \lesssim \PP(R) + \left(\frac{\EE(4R)}{\mu(R)}\right)^{1/2} + \PP(Q) +  \left(\frac{\EE(2Q)}{\mu(Q)}\right)^{1/2}.$$
\end{lemma}

\begin{proof}
For $Q\in\wt\End$ and $x,y\in Q$, we have
\begin{align}\label{eqal842}
\RR_{\wt\TT}\mu(x) - \Delta_{\wt\TT}\RR\mu(y) & = \RR(\chi_{2R\setminus 2Q}\mu)(x) - 
\big(m_{\mu,Q}(\RR\mu) - m_{\mu,2R}(\RR\mu)\big)\\
& = \big(\RR(\chi_{2R\setminus 2Q}\mu)(x) - m_{\mu,Q}(\RR\chi_{2R\setminus 2Q}\mu) \big) - m_{\mu,Q}(\RR(\chi_{2Q}\mu)))\nonumber\\
&\quad -
\big(m_{\mu,Q}(\RR(\chi_{2R^c}\mu))- m_{\mu,2R}(\RR\mu)\big).\nonumber
\end{align}
To estimate the first term on the right hand side, notice that for all $x,z\in Q$, by standard arguments we have
$$\big|\RR(\chi_{2R\setminus 2Q}\mu)(x) - \RR(\chi_{2R\setminus 2Q}\mu)(z)\big|\lesssim \PP(Q).$$
Averaging over $z\in Q$, we get
$$\big|\RR(\chi_{2R\setminus 2Q}\mu)(x) - m_{\mu,Q}(\RR\chi_{2R\setminus 2Q}\mu) \big|
\lesssim \PP(Q).$$
To estimate $m_{\mu,Q}(\RR(\chi_{2Q}\mu))$ we use the fact that, by the antisymmetry of the Riesz kernel, $m_{\mu,Q}(\RR(\chi_{Q}\mu))=0$, and thus
$$\big|m_{\mu,Q}(\RR(\chi_{2Q}\mu))\big| = \big|m_{\mu,Q}(\RR(\chi_{2Q\setminus Q}\mu))\big| 
\leq \avint_{z\in Q}\int_{\xi\in 2Q\setminus Q}\frac1{|z-\xi|^n}\,d\mu(\xi)d\mu(z).$$
By Cauchy-Schwarz and Lemma \ref{lemDMimproved}, we have
\begin{multline}\label{eqboundar3}
\avint_{z\in Q}\int_{\xi\in 2Q\setminus Q}\frac1{|z-\xi|^n}\,d\mu(\xi)d\mu(z)\\
= \avint_{z\in Q}\int_{\xi\in 2Q\setminus 2B_Q}\frac1{|z-\xi|^n}\,d\mu(\xi)d\mu(z) + \avint_{z\in Q}\int_{\xi\in 2B_Q\setminus Q}\frac1{|z-\xi|^n}\,d\mu(\xi)d\mu(z)\\
\leq
  \PP(Q)+\bigg(\avint_{z\in Q}\bigg(\int_{\xi\in 2B_Q\setminus Q}\frac1{|z-\xi|^n}\,d\mu(\xi)\bigg)^2d\mu(z)\bigg)^{1/2}
   \lesssim \PP(Q) + \left(\frac{\EE(2Q)}{\mu(Q)}\right)^{1/2}.
\end{multline}

Finally we deal with the last term on the right hand side of \rf{eqal842}. We split it:
\begin{align*}
\big|m_{\mu,Q}(\RR(\chi_{2R^c}\mu))- m_{\mu,2R}(\RR\mu)\big|
& \leq \big|m_{\mu,Q}(\RR(\chi_{3R\setminus 2R}\mu))| + \big|m_{\mu,2R}(\RR(\chi_{3R}\mu))\big|\\
&\quad +
\big|m_{\mu,Q}(\RR(\chi_{3R^c}\mu))- m_{\mu,2R}(\RR(\chi_{3R^c}\mu))\big| \\
& = J_1 + J_2 + J_3.
\end{align*}
Clearly,
$$J_1\lesssim \frac{\mu(3R)}{\ell(R)^n}\lesssim \PP(R).$$
Also, by the antisymmetry of the Riesz kernel and arguing as in \rf{eqboundar3},
\begin{align*}
J_2=& \big|m_{\mu,2R}(\RR(\chi_{3R}\mu))\big| = \big|m_{\mu,2R}(\RR(\chi_{3R\setminus 2R}\mu))\big| 
\\
& \leq  \avint_{z\in 2R}\int_{\xi\in 3R\setminus 2R}\frac1{|z-\xi|^n}\,d\mu(\xi)d\mu(z)
\lesssim \PP(R)+\left(\frac{\EE(4R)}{\mu(R)}\right)^{1/2}.
\end{align*}
Regarding $J_3$, by standard methods, for all $z,z'\in 2Q$, 
$$\big|\RR(\chi_{3R^c}\mu)(z) - \RR(\chi_{3R^c}\mu)(z')\big| \lesssim \PP(R).$$
Hence averaging for $z\in Q$, $z'\in 2R$, we obtain
$$J_3\lesssim \PP(R).$$
\end{proof}

\vv


\begin{lemma}\label{lemaprox3}
Let $1<p\leq2$.
For any $Q\in\wt\End$ such that $\ell_0\leq \ell(Q)$,
$$\int_Q \big|\RR_{\wt\TT}\mu - \RR_{\TT_\Reg}\mu\big|^p\,d\mu \lesssim \EE(2Q)^{\frac p2} \,\mu(Q)^{1-\frac p2}.$$
\end{lemma}

\begin{proof}
The lemma follows from H\"older's inequality and the estimate
\begin{equation}\label{eqf932}
\int_Q \big|\RR_{\wt\TT}\mu - \RR_{\TT_\Reg}\mu\big|^2\,d\mu \lesssim \EE(2Q),
\end{equation}
which we prove below.

Notice first that the cube $Q$ as above is covered by the cubes $P\in\DD_{\mu}(Q)\cap\Reg$. Indeed, if $Q\in \cM_\Neg$, then $Q\in\TT_{\Reg}$ and this is trivially true. On the other hand, if $Q\in\wt\End\setminus\cM_\Neg\subset\End$, then we have $Q\in\TT$, and the condition $\ell_0\leq \ell(Q)$ implies that $d_{R,\ell_0}(x)\leq\ell(Q)$. Hence, 
$Q$ is covered by the cubes $P\in\DD_\mu(Q)\cap\Reg$.

Observe now that, for $Q\in\wt\End$, $P\in\Reg$ such that $P\subset Q$, and $x\in P$,
$$\RR_{\TT_\Reg}\mu(x) - \RR_{\wt\TT}\mu(x) = \RR(\chi_{2R\setminus 2P}\mu)(x) -
\RR(\chi_{2R\setminus 2Q}\mu)(x) = \RR(\chi_{2Q\setminus 2P}\mu)(x).$$
Thus,
$$\big|\RR_{\TT_\Reg}\mu(x) - \RR_{\wt\TT}\mu(x)\big| \lesssim \sum_{S\in\DD_\mu:P\subset S\subset Q}
\theta_\mu(2B_S).$$
Notice now that, if $\wh Q$ denotes the cube from $\End$ that contains $Q$ (which coincides with
$Q$ when $P\not\in\Reg_\Neg$), we have
$$d_R(x) \gtrsim \dist(P,\supp\mu\setminus \wh Q) \geq \dist(P,\supp\mu\setminus Q) 
\quad \mbox{ for all $x\in P$.}$$
So by Lemma \ref{lem74} we also have
\begin{equation}\label{eqreg71}
\ell(P) \gtrsim \dist(P,\supp\mu\setminus Q)
\end{equation}
Thus,
$$\sum_{S\in\DD_\mu:P\subset S\subset Q}
\theta_\mu(2B_S) \lesssim \sum_{S\in\wt\DD_\mu^{int}(Q):S\supset P}
\theta_\mu(2B_S),$$
with $\wt\DD_\mu^{int}(Q)$ defined in \rf{eqdmuint}.

So we have
\begin{align*}
\int_Q \big|\RR_{\wt\TT}\mu - \RR_{\TT_\Reg}\mu\big|^2\,d\mu &=\sum_{P\in\Reg:P\subset Q}
\int_P \big|\RR_{\wt\TT}\mu - \RR_{\TT_\Reg}\mu\big|^2\,d\mu\\
& \lesssim \sum_{P\in\Reg:P\subset Q}
\bigg(\sum_{S\in\wt\DD_\mu^{int}(Q):S\supset P}
\theta_\mu(2B_S)\bigg)^2\,\mu(P).
\end{align*}
By H\"older's inequality, for any $\alpha>0$,
\begin{align*}
\bigg(\sum_{S\in\wt\DD_\mu^{int}(Q):S\supset P} &
\theta_\mu(2B_S)\bigg)^2\\
&\leq \bigg(\sum_{S\in\wt\DD_\mu^{int}(Q):S\supset P}\left(\frac{\ell(Q)}{\ell(S)}\right)^\alpha
\theta_\mu(2B_S)^2\bigg) \bigg(\sum_{S\in\wt\DD_\mu(Q):S\supset P}\left(\frac{\ell(S)}{\ell(Q)}\right)^{\alpha}\bigg)\\
& \lesssim_\alpha \sum_{S\in\wt\DD_\mu^{int}(Q):S\supset P}\left(\frac{\ell(Q)}{\ell(S)}\right)^\alpha
\theta_\mu(2B_S)^2.
\end{align*}
Therefore,
\begin{align*}
\int_Q \big|\RR_{\wt\TT}\mu - \RR_{\TT_\Reg}\mu\big|^2\,d\mu & \lesssim_\alpha 
\sum_{P\in\Reg:P\subset Q}\,\sum_{S\in\wt\DD_\mu^{int}(Q):S\supset P}\left(\frac{\ell(Q)}{\ell(S)}\right)^\alpha
\theta_\mu(2B_S)^2\mu(P)\\
& =
\sum_{S\in\wt\DD_\mu^{int}(Q)}\left(\frac{\ell(Q)}{\ell(S)}\right)^\alpha
\theta_\mu(2B_S)^2 \sum_{P\in\Reg:P\subset S}\mu(P)\\
& \le \sum_{S\in\wt\DD_\mu^{int}(Q)}\left(\frac{\ell(Q)}{\ell(S)}\right)^\alpha
\theta_\mu(2B_S)^2\,\mu(S).
\end{align*}
By Lemma \ref{lemdmutot}, for $\alpha$ small enough, the right hand side is bounded above by 
$C\EE(2Q)$, so that \rf{eqf932} holds.
\end{proof}

\vv

% ********************************************************************************************


\subsection{Estimates for the \texorpdfstring{$\PP$ and $\QQ_\Reg$}{P and Q\_Reg} coefficients}

We will transfer the lower estimate obtained for the $L^p(\eta)$ norm of $\RR\eta$ in Lemma
\ref{lemrieszeta} to $\RR_{\TT}\mu$, $\RR_{\TT_\Reg}\mu$, and $\Delta_{\TT}\RR\mu$ by means of
Lemmas \ref{lemaprox1}, \ref{lemaprox2}, and \ref{lemaprox3}. To this end, we will need careful 
estimates for the $\PP$ and $\QQ_\Reg$ coefficients of cubes from $\wt\End$ and  $\Reg$. This is the
task we will perform in this section.
\vv

%\subsection{Notation}

Given $R\in\MDW$, recall that
$\HD_1(e'(R))=\HD_*(R)\cap\sss_*(e'(R))$. 
To shorten notation, we will write $\HD_1=\HD_1(e'(R))$ in this section.
Notice also that, by \rf{eqstop2}, we have
\begin{equation}\label{eqsplit71}
\wt\End =  \LD_1 \cup \LD_2  \cup \HD_2\cup \sM_\Neg,
\end{equation}
where we introduced the following notations:
\begin{itemize}
%\item $\Neg=\Neg(e'(R))$.
\item $\LD_1$ is the subfamily of $\wt\End$ of those maximal $\PP$-doubling cubes which are contained both in $e'(R)$ and in some cube from $\LD(R)\cap\sss_1(e'(R))\setminus \Neg$.

\item $\LD_2$ is the subfamily of $\wt\End$ of those maximal $\PP$-doubling cubes which are contained in some cube
$Q\in \LD(Q')\cap \sss_*(Q')\setminus \Neg$ for some $Q'\in \HD_1$.
\item $\HD_2= \bigcup_{Q'\in\HD_1}(\HD_*(Q')\cap\sss_*(Q'))$.
\end{itemize}
Remark that the splitting in \rf{eqsplit71} is disjoint. Indeed, notice that, by the definition
of $\sM_\Neg$, the cubes from $\HD_2$ do not belong to $\sM_\Neg$, since they are strictly contained in some cube from $\HD_1$, which is $\PP$-doubling, in particular.

For $i=1,2$, we also denote by $\Reg_{\LD_i}$ the subfamily of the cubes from $\Reg$ which are contained in some cube from $\LD_i$, and we define $\Reg_{\HD_2}$, $\Reg_\Neg$, and
$\Reg_{\sM_\Neg}$ analogously.\footnote{Notice that $\sD_\Neg=\Reg_{\sM_\Neg}$.}
We let $\Reg_\Ot$ be the ``other'' cubes from $\Reg$: the ones which are not contained in any cube from $\End$ (which, in particular, have side length comparable to $\ell_0$ and are contained in $\TT$).
Also, we let $\Reg_{\HE}$ be the subfamily of the cubes from $\Reg$ which are contained in some cube from $\wt\End\cap\HE$. Notice that we have the splitting
\begin{equation}\label{eqsplitreg0}
\Reg = \Reg_{\LD_1} \cup  \Reg_{\LD_2}\cup \Reg_{\HD_2}\cup \Reg_{\Neg}\cup \Reg_{\Ot}.
\end{equation}
The families above may intersect the family $\Reg_{\HE}$.

Given a family $I\in\DD_\mu$ and $1<p\leq2$, we denote
$$\Sigma_p^\PP(I) = \sum_{Q\in I} \PP(Q)^p\,\mu(Q),\qquad \Sigma_p^\QQ(I) = \sum_{Q\in I} \QQ_\Reg(Q)^p\,\mu(Q).$$
We also write $\Sigma^\PP(I)= \Sigma_2^\PP(I)$, $\Sigma^\QQ(I)= \Sigma_2^\QQ(I)$.

\vv

\begin{lemma}\label{lemenereg}
For any $Q\in\wt\End$, 
$$\Sigma^\PP(\Reg\cap\DD_\mu(Q)) \lesssim \PP(Q)^2\,\mu(Q) + \EE(2Q).$$
\end{lemma}

\begin{proof}
 For all $S\in\Reg\cap \DD_\mu(Q)$, by H\"older's inequality, we have
\begin{align*}
\PP(S)^2 &\lesssim \bigg(\sum_{P:S\subset P\subset Q}\frac{\ell(S)}{\ell(P)}\,\Theta(P)\bigg)^2
+ \bigg(\frac{\ell(S)}{\ell(Q)}\,\PP(Q)\bigg)^2\\
& \leq \bigg(\sum_{P:S\subset P\subset Q}\frac{\ell(S)}{\ell(P)}\,\Theta(P)^2\bigg)
\bigg(\sum_{P:S\subset P\subset Q}\frac{\ell(S)}{\ell(P)}\bigg) +
\PP(Q)^2\\
& \lesssim \sum_{P:S\subset P\subset Q}\frac{\ell(S)}{\ell(P)}\,\Theta(P)^2+ 
\PP(Q)^2.
\end{align*}
We deduce that
\begin{align*}
\Sigma^\PP(\Reg\cap\DD_\mu(Q)) &=
\sum_{S\in\Reg\cap \DD_\mu(Q)} \PP(S)^2\,\mu(S)\\
& \lesssim
\sum_{S\in\Reg\cap \DD_\mu(Q)} 
\sum_{P:S\subset P\subset Q}\frac{\ell(S)}{\ell(P)}\,\Theta(P)^2\,\mu(S) +
\sum_{S\in\Reg\cap \DD_\mu(Q)} \PP(Q)^2\,\mu(S).
\end{align*}
Clearly, the last sum is bounded by $\PP(Q)^2\,\mu(Q)$. Concerning the first term,
arguing as in \rf{eqreg71}, we deduce that the cubes $P$ in the sum belong to 
$\wt\DD_\mu^{int}(Q)$. Thus, by Fubini,
\begin{align*}
\Sigma^\PP(\Reg\cap\DD_\mu(Q)) &\lesssim 
\sum_{P\in\wt \DD_\mu^{int}(Q)} \Theta(P)^2 \sum_{S\in\Reg:S\subset P}\frac{\ell(S)}{\ell(P)}\,\mu(S)
+ \PP(Q)^2\,\mu(Q)\\
& \leq \sum_{P\in\wt \DD_\mu^{int}(Q)}\! \Theta(P)^2 \,\mu(P) + \PP(Q)^2\,\mu(Q)\\ 
&\lesssim \EE(2Q)
+ \PP(Q)^2\,\mu(Q).
\end{align*}
\end{proof}
%\vv

%\begin{rem}\label{remtambe44}
%Let $Q\in\wt\End$, $P\in\Reg$, and $S\in\DD_\mu$ such that $P\subset S\subset Q$. By the same arguments
%as in Lemma \ref{lemenereg}, it follows that
%$$\Sigma^\PP(\Reg\cap\DD_\mu(S)) \lesssim \PP(S)^2\,\mu(S) + \EE(2S).$$
%\end{rem}

\vv
\begin{lemma}\label{lempoiss} 
We have:
\begin{itemize}
\item[(i)] If $Q\in\LD_1$, then $\PP(Q)\leq \delta_0\,\Theta(R)$.
%\vspace{1mm}
\item[(ii)] If $Q\in\LD_2$, then $\PP(Q)\lesssim \delta_0\,\Lambda_*\,\Theta(R)$.
\vspace{1mm}

\item[(iii)] If $Q\in \HD_2$, then $\PP(Q)\lesssim \Lambda_*^2\,\Theta(R)$.

\vspace{1mm}

\item[(iv)] If $Q\in\Neg\cup\sM_\Neg$, then $\PP(Q)\lesssim \left(\frac{\ell(Q)}{\ell(R)}\right)^{1/3}\,\Theta(R)$.
\end{itemize}
\end{lemma}

\begin{proof}
The statements (i),  (ii), and (iii) follow from the stopping conditions used to define $\LD(\,\cdot\,)$
and $\HD_*(\,\cdot\,)$ and Lemma \ref{lemdobpp}.
%The statement (ii) follows in the same way as (i), replacing $\Theta(R)$ by $\Lambda\Theta(R)$.


Regarding the property (iv), by the definitions of $\Neg$ and $\sM_\Neg$ and Lemma \ref{lemdobpp}, for all
$S\in\DD_\mu$ such that $Q\subset S\subset R$, we have
$$\Theta(S)\lesssim \left(\frac{\ell(S)}{\ell(R)}\right)^{1/2}\,\PP(R)\lesssim \left(\frac{\ell(S)}{\ell(R)}\right)^{1/2}\,\Theta(R).$$
Thus,
\begin{align*}
\PP(Q) &\approx \frac{\ell(Q)}{\ell(R)}\,\PP(R) + \sum_{S:Q\subset S\subset R}\frac{\ell(Q)}{\ell(S)}\,\Theta(S)\\
& \lesssim \frac{\ell(Q)}{\ell(R)}\,\Theta(R) + \sum_{S:Q\subset S\subset R}\frac{\ell(Q)}{\ell(S)}
\,\left(\frac{\ell(S)}{\ell(R)}\right)^{1/2}\,\Theta(R)\\
& \lesssim \frac{\ell(Q)}{\ell(R)}\,\Theta(R) + \sum_{S:Q\subset S\subset R}\,\left(\frac{\ell(Q)}{\ell(R)}\right)^{1/2}\,\Theta(R)\\
& \lesssim \frac{\ell(Q)}{\ell(R)}\,\Theta(R) + \log\left(2+\frac{\ell(R)}{\ell(Q)}\right)\,\left(\frac{\ell(Q)}{\ell(R)}\right)^{1/2}\,\Theta(R) \lesssim \left(\frac{\ell(Q)}{\ell(R)}\right)^{1/3}\,\Theta(R).
\end{align*}

\end{proof}
\vv



\begin{rem}\label{rem9.7}
Since $\delta_0 = \Lambda^{-N_0 - \frac1{2N}}\le \Lambda_*^{-N_0 - \frac1{2N}}$, it follows that $\delta_0\,\Lambda_*
\leq \delta_0^{1/2}$, so that by the preceding lemma
 $$\PP(Q)\lesssim \delta_0^{1/2}\,\Theta(R)\quad \mbox{ for all $Q\in\LD_1\cup\LD_2$.}$$
\end{rem}

\vv
Given $Q\in\DD_\mu$, we write $Q\sim \TT$ if there exists some
$Q'\in\TT$ such that 
\begin{equation}\label{defsim0}
A_0^{-2}\ell(Q)\leq \ell(Q')\leq A_0^2\ell(Q)\quad \text{ and }\quad 20Q'\cap20Q\neq\varnothing.
\end{equation}
Given $\gamma\in(0,1)$, we say that the tree $\TT$ is $\gamma$-nice if
$$\sum_{Q\in\HE:Q\sim\TT} \EE(4Q)\leq\gamma \,\sigma(\HD_1).$$

% *********************************************************************************

\vv
\begin{lemma}\label{lemregmolt}
Let $R\in\MDW$ and suppose that $\TT=\TT(e'(R))$ is tractable and $\gamma$-nice.
Then

%\begin{equation}\label{eqlem*1}
%\Sigma^\PP(\wt\End)\approx \sigma(\wt\End) \lesssim B\,\sigma(\HD_1),
%\end{equation}

\begin{equation}\label{eqlem*1}
\Sigma^\PP(\Reg_\HE) \lesssim \sum_{Q\in\wt \End\cap \HE}\EE(4Q)\leq
\sum_{Q\in\HE:Q\sim \TT}\EE(4Q)
\leq \gamma\,\sigma(\HD_1),
\end{equation}

\begin{equation}\label{eqlem*2}
\Sigma^\PP(\Reg_{\LD_1}\cup \Reg_{\LD_2}) \lesssim \sum_{Q\in
\LD_1\cup \LD_2} \EE(4Q)\lesssim 
\big(B\,M_0^2\,\delta_0 + \gamma\big)\,\sigma(\HD_1),
\end{equation}

\begin{equation}\label{eqlem*2.5}
 \sum_{Q\in\HD_2} \EE(4Q)\lesssim \big(B\,M_0^2 + \gamma\big)\,\sigma(\HD_1).
\end{equation}

\noi Also, for $1<p\leq2$,

\begin{equation}\label{eqlem*3}
\Sigma_p^\PP(\Reg_\HE)\lesssim  \gamma^{\frac 12}\,B\,\Lambda_*^2\,\sigma_p(\HD_1),
\end{equation}

\begin{equation}\label{eqlem*4}
\Sigma_p^\PP(\Reg_{\LD_1}\cup\Reg_{\LD_2})\lesssim B\,\Lambda_*^2
\big(M_0^2\,\delta_0 + \gamma\big)\,\sigma_p(\HD_1),
\end{equation}

\begin{equation}\label{eqlem*5}
\Sigma_p^\PP(\Reg_{\HD_2})\lesssim \Sigma_p^\PP(\HD_2) \approx \sigma_p(\HD_2) \lesssim B\,\Lambda_*^{p-2}\,\sigma_p(\HD_1).
\end{equation}
\end{lemma}
\vv

\begin{proof}
To get \rf{eqlem*1} note that all the cubes in $\wt\End$ are $\PP$-doubling. Thus, by Lemma \ref{lemenereg} and the definition of $\HE$
\begin{multline*}
\Sigma^\PP(\Reg_\HE) \lesssim \Sigma^\PP(\wt\End\cap\HE) + \sum_{Q\in\wt \End\cap \HE}\EE(4Q)\\
 \lesssim \sum_{Q\in\wt \End\cap \HE}\Theta(Q)^2\mu(Q) + \sum_{Q\in\wt \End\cap \HE}\EE(4Q) \approx \sum_{Q\in\wt \End\cap \HE}\EE(4Q).
\end{multline*}
Observe that for all $Q\in\TT_{\Reg}$ (in particular, for all $Q\in\wt\End$) we have $Q\sim \TT$. Together with the fact that $\TT$ is $\gamma$-nice, \rf{eqlem*1} follows.

To see \rf{eqlem*3}, we first use H\"older's inequality together with \rf{eqlem*1}:
$$
\Sigma_p^\PP(\Reg_\HE)\leq \Sigma^\PP(\Reg_\HE)^{\frac p2}\,\mu(e'(R))^{1-\frac p2}\lesssim \gamma^{\frac 12}\,\sigma(\HD_1)^{\frac p2}
\mu(R)^{1-\frac p2}.
$$
Observe now that, writing $HD_i=\bigcup_{Q\in\HD_i}Q$,
\begin{equation}\label{eqmuhd1}
\mu(HD_1) = \frac1{\Lambda_*^2\,\Theta(R)^2}\,\sigma(\HD_1) \geq \frac1{B\,\Lambda_*^2\,\Theta(R)^2}\,\sigma(R) = \frac1{B\,\Lambda_*^2}\,\mu(R).
\end{equation}
Therefore,
$$
\Sigma_p^\PP(\Reg_\HE)\lesssim \gamma^{\frac 12}\,B\,\Lambda_*^2\,\sigma(\HD_1)^{\frac p2}\,\mu(HD_1)^{1-\frac p2}
= \gamma^{\frac 12}\,B\,\Lambda_*^2\,\sigma_p(\HD_1),$$
which proves \rf{eqlem*3}.

To show \rf{eqlem*2} observe first that, by Lemma \ref{lempoiss} and Remark \ref{rem9.7},
$$\sigma(\LD_1\cup\LD_2) \lesssim \delta_0\,\Theta(R)^2\,\mu(R)\lesssim \delta_0\,B\,\sigma(\HD_1).$$
Then, by Lemma \ref{lemenereg} again, 
\begin{multline*}
\Sigma^\PP(\Reg_{\LD_1}\cup \Reg_{\LD_2}) \lesssim \Sigma^\PP(\LD_1\cup\LD_2)+\sum_{Q\in
\LD_1\cup \LD_2} \EE(4Q)\\
 \leq \sigma(\LD_1\cup\LD_2)+ \sum_{Q\in
(\LD_1\cup \LD_2)\setminus\HE} \!\!\EE(4Q) + \sum_{Q\in\HE:Q\sim\TT} \EE(4Q)\\
\lesssim \sigma(\LD_1\cup\LD_2) + M_0^2\,\sigma\big((\LD_1\cup \LD_2)\setminus\HE\big) + \gamma\,\sigma(\HD_1)
 \\
 \lesssim
\big(B\,M_0^2\,\delta_0 + \gamma\big)\,\sigma(\HD_1).
\end{multline*}
So, by H\"older's inequality and \rf{eqmuhd1},
\begin{align*}
\Sigma_p^\PP(\Reg_{\LD_1}\cup \Reg_{\LD_2})& \leq \Sigma^\PP(\Reg_{\LD_1}\cup \Reg_{\LD_2})^{\frac p2}\,\mu(e'(R))^{1-\frac p2}\\
& \lesssim \big(B\,M_0^2\,\delta_0 + \gamma\big)^{\frac 12}\,\sigma(\HD_1)^{\frac p2}\,\mu(R)^{1-\frac p2}\\
&\lesssim
\big(B\,M_0^2\,\delta_0 + \gamma\big)^{\frac p2} (B\Lambda_*^2)^{1-\frac p2}
\,\sigma_p(\HD_1)\\
& \leq \big(B\,\Lambda_*^2\,M_0^2\,\delta_0 + B\,\Lambda_*^2\gamma\big)\,\sigma_p(\HD_1),
\end{align*}
which yields \rf{eqlem*4}.

To prove \rf{eqlem*2.5}, we just write
\begin{align*}
 \sum_{Q\in\HD_2} \EE(4Q) &\leq  \sum_{Q\in\HD_2\setminus \HE} \EE(4Q) +  \sum_{Q\in \HE:Q\sim\TT} \EE(4Q)\\
&\leq M_0^2\,\sigma(\HD_2) + \gamma\,\sigma(\HD_1)\leq M_0^2\,B\,\sigma(\HD_1) + \gamma\,\sigma(\HD_1).
\end{align*}
Finally, regarding
 \rf{eqlem*5}, recall that $\Theta(Q) \leq\Lambda_*^2\,\Theta(R)$ for all $Q\in\TT$.
This implies that also $\PP(Q) \lesssim\Lambda_*^2\,\Theta(R)$ for all $Q\in\TT$, by Lemma \ref{lemdobpp}. Arguing as in Remark \ref{rem:Hjempty} one gets
\begin{equation}\label{eq:Regdens}
\PP(Q) \lesssim\Lambda_*^2\,\Theta(R)\quad \text{for all $Q\in\Reg$.}
\end{equation}
Consequently,
$$\Sigma_p^\PP(\Reg_{\HD_2})\lesssim \Lambda_*^{2p}\,\Theta(R)^p\,\mu(HD_2) = \sigma_p(\HD_2) \approx
\Sigma_p^\PP(\HD_2).$$
On the other hand, since $R\in\Trc$,
\begin{align*}
\sigma_p(\HD_2) & = \Theta(\HD_2)^{p-2}\,\sigma(\HD_2)\leq B\,\Theta(\HD_2)^{p-2}\,\sigma(\HD_1)
\\ &= B\,\frac{\Theta(\HD_2)^{p-2}}{\Theta(\HD_1)^{p-2}}\,\sigma_p(\HD_1) = B\,\Lambda_*^{p-2}\,\sigma_p(\HD_1),
 \end{align*}
 which completes the proof of \rf{eqlem*5}.
\end{proof}

\vv



\begin{lemma}\label{lemneg3}
Let $R\in\MDW$ and suppose that $\TT=\TT(e'(R))$ is tractable and $\gamma$-nice.
 Then
\begin{equation}\label{eqlemneg01}
\Sigma^\PP(\sM_\Neg)\lesssim\Sigma^\PP(\Neg) \lesssim \big(\delta_0\,B\,\Lambda_*^6 + B\,\Lambda_*^{-4}\big)\,\sigma(\HD_1),
\end{equation}
and
\begin{equation}\label{eqlemneg02}
\Sigma^\PP(\Reg_\Neg) + \sum_{S\in\sM_\Neg} \EE(4S)\lesssim \big(\delta_0\,B\,M_0^2\,\Lambda_*^6 + B\,\Lambda_*^{-4}\,M_0^2+ \gamma\big)\,\sigma(\HD_1)
\end{equation}
Also, for $1<p\leq 2$,
\begin{equation}\label{eqlemneg03}
\Sigma_p^\PP(\Reg_\Neg) \lesssim \big(\delta_0^{\frac12}\,B\,M_0^2\,\Lambda_*^6 + B\,\Lambda_*^{-1}\,M_0+ B\Lambda_*^2\gamma^{\frac12}\big)
\,\sigma_p(\HD_1).
\end{equation}
\end{lemma}

\begin{proof}
Recall that $\Neg = \Neg(e'(R))\cap\End$. The first inequality in \eqref{eqlemneg01} follows from Lemma \ref{eqlemneg01} and the definition of $\sM_\Neg$. 

By Lemma \ref{lemnegs}, the cubes from $\Neg$ belong to $\LD(R)$. Recall also they are not $\PP$-doubling and that $\PP(Q)\lesssim \left(\frac{\ell(Q)}{\ell(R)}\right)^{1/3}\,\Theta(R)$ for all $Q\in\Neg$.
To estimate $\Sigma^\PP(\Neg)$, we split $\Neg$ into two subfamilies $I$ and $J$ so that the 
cubes from $I$ have side length at least $\Lambda_*^{-6}\ell(R)$, opposite to the ones from $J$. We have
\begin{align*}
\Sigma^\PP(I) &= \sum_{Q\in I}\PP(Q)^2\,\mu(Q) \lesssim \Theta(R)^2\sum_{Q\in I}\mu(Q)\\
& \lesssim \Theta(R)^2\sum_{Q\in I}\Theta(Q)\,\ell(Q)^n \lesssim \delta_0\Theta(R)^3\sum_{Q\in I}\ell(Q)^n.
\end{align*}
Using now that the balls $\frac12B(Q)$, with $Q\in I$, are disjoint and that $\ell(Q)\geq \Lambda_*^{-6}\ell(R)$, we get
$$\sum_{Q\in I}\ell(Q)^n \leq \sum_{Q\in I}\frac{\Lambda_*^6\,\ell(Q)^{n+1} }{\ell(R)}\lesssim
\frac{\Lambda_*^6\,\ell(R)^{n+1} }{\ell(R)}=\Lambda_*^6\,\ell(R)^n.$$
Therefore,
$$\Sigma^\PP(I)\lesssim \delta_0\,\Lambda_*^6\,\Theta(R)^3\,\ell(R)^n\approx  \delta_0\,\Lambda_*^6\,\sigma(R)\lesssim \delta_0\,B\,\Lambda_*^6\,\sigma(\HD_1).$$
In connection with the family $J$, we have
\begin{align*}
\Sigma^\PP(J) &= \sum_{Q\in J}\PP(Q)^2\,\mu(Q)\lesssim \Theta(R)^2\sum_{Q\in J}\left(\frac{\ell(Q)}{\ell(R)}\right)^{2/3}\,\mu(Q)\\
&\leq \Lambda_*^{-12/3}\,\Theta(R)^2\sum_{Q\in J}\mu(Q) \leq \Lambda_*^{-4}\,\sigma(R)\leq B\,\Lambda_*^{-4}\,\sigma(\HD_1).
\end{align*}
Hence,
$$\Sigma^\PP(\Neg)\lesssim \big(\delta_0\,B\,\Lambda_*^6 + B\,\Lambda_*^{-4}\big)\,\sigma(\HD_1).$$

\vv
To deal with \rf{eqlemneg02}, we split
$$\Sigma^\PP(\Reg_\Neg) = \Sigma^\PP(\Reg_{\sM_\Neg}) + \Sigma^\PP(\Reg_\Neg\setminus \Reg_{\sM_\Neg}).$$
Notice that, for all $P\in\Reg_\Neg\setminus \Reg_{\sM_\Neg}$ such that $P\subset Q\in\Neg$, by Lemma \ref{lemdobpp}, 
$\PP(P)\lesssim\PP(Q)$. Then it follows that
$$\Sigma^\PP(\Reg_\Neg\setminus \Reg_{\sM_\Neg})\lesssim \Sigma^\PP(\Neg).$$

Next we estimate $\Sigma^\PP(\Reg_{\sM_\Neg})$. By Lemma \ref{lemenereg}, we have 
\begin{align}\label{eq83e}
\Sigma^\PP(\Reg_{\sM_\Neg})  & =\sum_{S\in\sM_\Neg} \Sigma^\PP(\Reg_{\sM_\Neg}\cap\DD_\mu(S))
 \lesssim \sum_{S\in\sM_\Neg} \EE(4S) + \Sigma^\PP(\sM_\Neg)\\
 & \leq  \sum_{S\in\sM_\Neg\setminus\HE} \EE(4S) + \sum_{S\in\HE\cap\wt \End} \EE(4S) + \Sigma^\PP(\sM_\Neg).\nonumber
\end{align}
By the definition of $\HE$ and the fact that $\PP(S)\lesssim\PP(Q)$ whenever $S\subset Q\in\Neg$,
we derive
$$\sum_{S\in\sM_\Neg\setminus\HE} \EE(4S) + \Sigma^\PP(\sM_\Neg)\lesssim M_0^2\sum_{S\in\sM_\Neg\setminus\HE} \sigma(S) + \Sigma^\PP(\sM_\Neg)\lesssim
M_0^2\,\Sigma^\PP(\Neg).$$
On the other hand, by \rf{eqlem*1},
$$\sum_{S\in\HE\cap\wt \End} \EE(4S) \leq\gamma\,\sigma(\HD_1).
$$
 Therefore,
$$\Sigma^\PP(\Reg_{\sM_\Neg})+ \sum_{S\in\sM_\Neg} \EE(4S)\lesssim \sum_{S\in\sM_\Neg} \EE(4S) \lesssim M_0^2\,\Sigma^\PP(\Neg) + \gamma\,\sigma(\HD_1).$$
Gathering the estimates above, we deduce
\begin{align*}
\Sigma^\PP(\Reg_\Neg)&\lesssim M_0^2\,\Sigma^\PP(\Neg) + \gamma\,\sigma(\HD_1)\\
& \lesssim
\big(\delta_0\,B\,M_0^2\,\Lambda_*^6 + B\,\Lambda_*^{-4}\,M_0^2+ \gamma\big)\,\sigma(\HD_1).
\end{align*}

Finally, to prove \rf{eqlemneg03} we apply H\"older's inequality and \rf{eqmuhd1}:
\begin{align*}
\Sigma_p^\PP(\Reg_\Neg)& \leq \Sigma^\PP(\Reg_\Neg)^{\frac p2}\,\mu(e'(R))^{1-\frac p2}\\
& \lesssim \big(\delta_0\,B\,M_0^2\,\Lambda_*^6 + B\,\Lambda_*^{-4}\,M_0^2+ \gamma\big)^{\frac p2}\,\sigma(\HD_1)^{\frac p2}\,\mu(R)^{1-\frac p2}\\
&\lesssim 
\big(\delta_0\,B\,M_0^2\,\Lambda_*^6 + B\,\Lambda_*^{-4}\,M_0^2+ \gamma\big)^{\frac p2} (B\Lambda_*^2)^{1-\frac p2}
\,\sigma_p(\HD_1)\\
 & \lesssim
 \big(\delta_0^{\frac12}\,B\,M_0^2\,\Lambda_*^6 + B\,\Lambda_*^{-p}\,M_0^p+ B\,\Lambda_*^2\,\gamma^{\frac12}\big)
\,\sigma_p(\HD_1).
\end{align*} 
Assuming $\Lambda_*$ big enough, we have $\Lambda_*^{-1}\,M_0<1$. Then, it is clear that $\Lambda_*^{-p}\,M_0^p\leq \Lambda_*^{-1}\,M_0$, and so 
\rf{eqlemneg03} follows.
\end{proof}
\vv



Next lemma deals with $\Sigma_p^\PP(\Reg_\Ot)$. Below we write $\Reg_\Ot(\ell_0)$ to recall the dependence of the family $\Reg_\Ot$ on the parameter
$\ell_0$ in \rf{eql00*23}.

\begin{lemma}\label{lemregot}
For $1\leq p\leq2$, we have
\begin{equation}\label{eqgaga34}
\limsup_{\ell_0\to0} \Sigma_p^\PP(\Reg_\Ot(\ell_0)) \lesssim \Lambda_*^{2p}\,\Theta(R)^p\,\mu(Z).
\end{equation}
Consequently, if $\mu(Z)\leq \ve_Z\,\mu(R)$, then
\begin{equation}\label{eqgaga35}
\limsup_{\ell_0\to0} \Sigma_p^\PP(\Reg_\Ot(\ell_0)) \lesssim B\,\Lambda_*^4\,\ve_Z\,\sigma_p(\HD_1).
\end{equation}
\end{lemma}
\begin{proof}
Notice that if $x\in Q\in\End$ with $\ell(Q)\geq \ell_0$, then 
$$d_{R,\ell_0}(x)\leq \max(\ell_0,\ell(Q)) = \ell(Q)$$
(recall that $d_{R,\ell_0}$ is defined in \rf{eql00*23}), and thus $x$ is contained in some cube $Q'\in\Reg$ with $\ell(Q')\leq \ell(Q)$, by the definition of the family $\Reg$. So $Q'\subset Q$ and then $Q'\in\Reg\setminus \Reg_\Ot$. Therefore,
\begin{equation}\label{eqregotin}
\bigcup_{P\in\Reg_\Ot} P \subset e'(R) \setminus \bigcup_{Q\in\End:\ell(Q)>\ell_0} Q,
\end{equation}
and thus
\begin{equation}\label{eqlimot62}
\limsup_{\ell_0\to0} \mu\bigg(\bigcup_{P\in\Reg_\Ot(\ell_0)} P\bigg) \leq \mu\bigg(e'(R) \setminus \bigcup_{Q\in\End} Q\bigg) = \mu(Z).
\end{equation}
To complete the proof of \rf{eqgaga34} it just remains to notice that by \eqref{eq:Regdens} $\PP(P)\lesssim \Lambda_*^2\,\Theta(R)$ for all $P\in\Reg$, and so
$$\Sigma_p^\PP(\Reg_\Ot(\ell_0)) \lesssim \Lambda_*^{2p}\,\Theta(R)^p \,\mu\bigg(\bigcup_{P\in\Reg_\Ot(\ell_0)} P\bigg).$$

Regarding the second statement of the lemma recall that, as in \rf{eqmuhd1},
$\mu(HD_1) \geq  \frac1{B\,\Lambda_*^2}\,\mu(R)$, which implies that
$$\mu(Z) \leq B\,\Lambda_*^2\,\ve_Z\,\mu(HD_1).$$
Plugging this estimate into \rf{eqgaga34} and taking into account that $\Theta(\HD_1)=\Lambda_*\Theta(R)$, we get 
$$\limsup_{\ell_0\to0} \Sigma_p^\PP(\Reg_\Ot(\ell_0)) \lesssim B\,\Lambda_*^{p+2}\,\ve_Z\,\Theta(\HD_1)^p\,\mu(HD_1) \leq B\,\Lambda_*^4\,\ve_Z\,\sigma_p(\HD_1).$$
\end{proof}

\vv
\begin{rem}\label{remregot}
From \rf{eqregotin}, it is immediate to check that for all $x\in P\in\Reg_\Ot$, we have $d_R(x)\lesssim\ell_0$, and so $d_{R,\ell_0}(x)\approx \ell_0$, which implies that
$$\ell(P)\approx \ell_0.$$
\end{rem}

\vv

\begin{lemma}\label{lemregtot*}
Let $R\in\MDW$ and suppose that $\TT=\TT(e'(R))$ is tractable and $\gamma$-nice.
For $1\leq p\leq2$,  if $\mu(Z)\leq \ve_Z\,\mu(R)$ and $\ell_0$ is small enough, then
\begin{equation}\label{eqlemtot*}
\Sigma_p^\PP(\Reg) \lesssim \big(B\,M_0^2\,\Lambda_*^6 \delta_0^{\frac12}+ B\,\Lambda_*^{-1}M_0+ B\,\Lambda_*^2\,\gamma^{\frac12} + B\,\Lambda_*^4\,\ve_Z + B\,\Lambda_*^{p-2}\big)
\,\sigma_p(\HD_1).
\end{equation}
\end{lemma}

\begin{proof}
This follows by gathering the estimates obtained in Lemmas \ref{lemregmolt},
\ref{lemneg3}, and \ref{lemregot}.
\end{proof}

\vv



\begin{rem}\label{rem9.12}
Recall that $\delta_0 = \Lambda^{-N_0 - \frac1{2N}}\le \Lambda_*^{-N_0 - \frac1{2N}}$, and also that $\Lambda_* = A_0^{k_{\Lambda} (1-1/N) n}$ and $B = \Lambda_*^{\frac{1}{100n}}$. Assuming that $N_0>20$, that $k_{\Lambda}$ is big enough (depending on $M_0$), and that $\gamma$ and $\ve_Z$ are small enough (depending on $k_\Lambda$ and $M_0$), we have
$$B\,M_0^2\,\Lambda_*^6 \delta_0^{\frac12}+  B\,\Lambda_*^2\,\gamma^{\frac12} + B\, M_0^2\, \Lambda_*^4\,
\ve_Z^{\frac12} \leq \Lambda_*^{-1}.$$
In this way, for $\ell_0$ small enough, under the assumptions of Lemma \ref{lemregtot*} we have
$$\Sigma_p^\PP(\Reg) \lesssim \big(\Lambda_*^{-1}+ B\,\Lambda_*^{-1}M_0+ B\,\Lambda_*^{p-2}\big)
\,\sigma_p(\HD_1).
$$

Assuming again $k_{\Lambda}$ big enough (depending on $M_0$), and taking $p\in(1,3/2]$, we have
\begin{equation}\label{eqregtot45}
\Sigma_p^\PP(\Reg) \lesssim  B\,\Lambda_*^{-1/2}M_0
\,\sigma_p(\HD_1)\leq \Lambda_*^{-1/4}\,\sigma_p(\HD_1).
\end{equation}
\end{rem}

\vv


Next lemma shows how to estimate the $\QQ_\Reg$ coefficients in terms of the $\PP$ coefficients.
\vv

\begin{lemma}\label{lemregpq}
For all $p\in(1,\infty)$,
$$\Sigma_p^\QQ(\Reg)\lesssim \Sigma_p^\PP(\Reg).$$
\end{lemma}

%Remark that the lemma does not assert that $\Sigma_p^\QQ(I)\lesssim\Sigma_p^\PP(I)$ for any family $I\subset \Reg$. 
%As far as we know this is not correct.

\begin{proof}
By duality,
\begin{equation}\label{eqdual912}
\Sigma_p^\QQ(\Reg)^{1/p} = \bigg(\sum_{Q\in\Reg}\QQ_\Reg(Q)^p\,\mu(Q)\bigg)^{1/p}
= \sup \sum_{Q\in\Reg} \QQ_\Reg(Q)\,g_Q\,\mu(Q),
\end{equation}
where the supremum is taken over all sequences $g=\{g_Q\}_{Q\in\Reg}$ such that 
\begin{equation}\label{eqdual912b}
\sum_{Q\in\Reg} |g_Q|^{p'}\,\mu(Q)\leq1.
\end{equation}
We will identify the sequence $g$ with the function 
$$\wt g= \sum_{Q\in\Reg}g_Q\,\chi_Q,$$
so that the sum in \rf{eqdual912b} equals $\|\wt g \|_{L^{p'}(\mu)}^{p'}$.
 By the definition of $\QQ_\Reg$ and Fubini we have
\begin{align}\label{eqdjkq44}
\sum_{Q\in\Reg} \QQ_\Reg(Q)\,g_Q\,\mu(Q) &= \sum_{Q\in\Reg} \sum_{P\in\Reg} \frac{\ell(P)}{D(P,Q)^{n+1}}\,\mu(P)\,g_Q\,\mu(Q)\\
& = \sum_{P\in\Reg} \bigg( \sum_{Q\in\Reg} \frac{\ell(P)}{D(P,Q)^{n+1}}\,g_Q\,\mu(Q)\bigg)\,\mu(P).
\nonumber
\end{align}
For each $P\in\Reg$, we have
\begin{align}\label{eqalg539}
\sum_{Q\in\Reg} \frac{\ell(P)}{D(P,Q)^{n+1}}\,|g_Q|\,\mu(Q) & = \sum_{j\geq0}\;
\sum_{Q\in\Reg:2^j\ell(P)\leq D(P,Q)\leq 2^{j+1}\ell(P)} \frac{\ell(P)}{D(P,Q)^{n+1}}\,|g_Q|\,\mu(Q)\\
& \leq \sum_{j\geq0}\;
\sum_{Q\in\Reg:D(P,Q)\leq 2^{j+1}\ell(P)} \frac{2^{-j}}{(2^j\ell(P))^{n}}\,|g_Q|\,\mu(Q).\nonumber
\end{align}

Observe now that the condition
$$\ell(Q) + \dist(P,Q)\leq D(P,Q)\leq 2^{j+1}\ell(P),$$
implies that 
$$Q\subset B(x_Q,\ell(Q))\subset B(x_P,2^{j+3}\ell(P)).$$
%the cubes $Q\in\Reg$ such that $D(P,Q)\leq 2^{j+1}\ell(P)$ are contained in 
%$B(x_P,C2^j\ell(P))$ for some absolute constant $C$. Indeed, the condition
%$$\ell(Q) + \dist(P,Q)\leq D(P,Q)\leq 2^{j+1}\ell(P),$$
%implies that 
%$$Q\subset B(x_Q,\ell(Q))\subset B(x_P,2^{j+3}\ell(P)).$$
%In case that 
%$$B(x_P,3\ell(P))\cap B(x_Q,3\ell(Q))\neq\varnothing,$$
%then $\ell(P)\approx \ell(Q)$ by Lemma \ref{lem74} (b), and so we also have
%$$Q\subset B(x_Q,\ell(Q))\subset B(x_P,C\ell(P)),$$
%which concludes the proof of the claim.
From \rf{eqalg539} and this fact, we infer that
\begin{align*}
\sum_{Q\in\Reg} \frac{\ell(P)}{D(P,Q)^{n+1}}\,|g_Q|\,\mu(Q) &
\leq  \sum_{j\geq0}\;
\sum_{Q\in\Reg:Q\subset B(x_P,C2^j\ell(P))} \frac{2^{-j}}{(2^j\ell(P))^{n}}\,|g_Q|\,\mu(Q)\\
& \leq\sum_{j\geq0}
 \frac{2^{-j}}{(2^j\ell(P))^{n}}\,\int_{B(x_P,C2^j\ell(P))}|\wt g|\,d\mu
\end{align*}
Notice now that, for all $x\in P$,
\begin{align*}
\int_{B(x_P,C2^j\ell(P))}|\wt g|\,d\mu & \leq \int_{B(x,C'2^j\ell(P))}|\wt g|\,d\mu\\
&\leq \mu(B(x,C'2^j\ell(P)))\,\cM_\mu \wt g(x)\leq \mu(B(x_P,C''2^j\ell(P)))\,\cM_\mu \wt g(x),
\end{align*}
where $\cM_\mu$ is the centered Hardy-Littlewood maximal operator. Thus,
\begin{align*}
\sum_{Q\in\Reg} \frac{\ell(P)}{D(P,Q)^{n+1}}\,|g_Q|\,\mu(Q)
& \leq\sum_{j\geq0}
 \frac{2^{-j}\mu(B(x_P,C''2^j\ell(P)))}{(2^j\ell(P))^{n}}\,\cM_\mu \wt g(x)\\
 &\approx
 \sum_{k\geq0}
 2^{-k}\theta_\mu(2^kB_P)\,\cM_\mu \wt g(x) \approx \PP(P)\,\,\cM_\mu \wt g(x).
 \end{align*}
Plugging this estimate into \rf{eqdjkq44}, and taking the infimum for $x\in P$, we get
\begin{align*}
\sum_{Q\in\Reg} \QQ_\Reg(Q)\,|g_Q|\,\mu(Q) & \lesssim \sum_{P\in\Reg} \PP(P)\,\inf_{x\in P}\cM_\mu \wt g(x)\,\mu(P)\\
& \leq \bigg(\sum_{P\in\Reg} \PP(P)^p\,\mu(P)\bigg)^{1/p}
\bigg(\sum_{P\in\Reg} \int_P |\cM_\mu \wt g|^{p'}\,d\mu\bigg)^{1/p'}\\
& = \Sigma_p^\PP(\Reg)^{1/p}\,\|\cM_\mu\wt g\|_{L^{p'}(\mu\rest_{e'(R)})}\\
&\lesssim
\Sigma_p^\PP(\Reg)^{1/p}\,\|\wt g\|_{L^{p'}(\mu)}\leq \Sigma_p^\PP(\Reg)^{1/p},
\end{align*}
which concludes the proof of the lemma, by \rf{eqdual912}.
\end{proof}

\vv





% ********************************************************************************************


\subsection{Transference of the lower estimates for \texorpdfstring{$\RR\eta$}{R eta} to \texorpdfstring{$\Delta_{\wt \TT}\RR\mu$}{Delta\_T R mu}}

We denote by $\wt V_4$ the union of the balls $\frac12B(Q)$, with $Q\in\Reg$, that intersect $V_4$.

\vv

\begin{lemma}\label{lemalter*}
Let $R\in\MDW$ and suppose that $\TT=\TT(e'(R))$ is tractable and $\gamma$-nice, with $\gamma$ small enough.
Suppose that $\mu(Z)\leq \ve_Z\,\mu(R)$, and take $\ve_Z,\gamma,M_0,\Lambda_*,B$ and $\ell_0$ as in Remark \ref{rem9.12}. 
Then 
$$\|\Delta_{\wt \TT} \RR\mu\|_{L^2(\mu)}^2\geq \Lambda_*^{-2}\,\sigma(\HD_1).$$
\end{lemma}

\begin{proof}
Recall that in Lemma \ref{lemrieszeta} we showed that  
\begin{equation}\label{eqI0038}
I_0:=\int_{V_4} \big|(|\RR\eta(x)| - \frac{c_3}2\,\Theta(\HD_1))_+\big|^p\,d\eta(x)
 \gtrsim \Lambda_*^{-p'\ve_n}\sigma_p(\HD_1),
\end{equation}
for any $p\in (1,\infty)$, with $c_3$ as in Lemma \ref{lemvar}.
The appropriate values of $\ve_n$ and $p$ will be chosen at the end of the proof.


By Lemma \ref{lemaprox1}, for all $Q\in\Reg$ such that $\frac12B(Q)\subset\wt V_4$, all $x\in \frac12B(Q)$, and all $y\in Q$,
$$|\RR\eta(x)| \leq |\RR_{\TT_\Reg}\mu(y)| + C\Theta(R) + C\PP(Q) + C\QQ_\Reg(Q).$$
Thus, for $\Lambda_*$ big enough, since $\Theta(R)=\Lambda_*^{-1}\Theta(\HD_1)<\frac{c_3}4 \Theta(\HD_1)$,
$$(|\RR\eta(x)| - \frac{c_3}2 \,\Theta(\HD_1))_+\leq
(|\RR_{\TT_\Reg}\mu(y)| - \frac{c_3}4\, \Theta(\HD_1))_+  + C\PP(Q) + C\QQ_\Reg(Q).$$
Therefore,
$$
I_0  \lesssim \sum_{Q\in\Reg} \int_Q \big|(|\RR_{\TT_\Reg}\mu(y)| - 
\frac{c_3}4 \Theta(\HD_1))_+\big|^{p}\,d\mu(y) + \sum_{Q\in\Reg} (\PP(Q)^{p} + \QQ_\Reg(Q)^{p})\,\mu(Q).$$
By \rf{eqregtot45} and Lemma \ref{lemregpq}, we have
\begin{align*}
\sum_{Q\in\Reg} (\PP(Q)^{p} + \QQ_\Reg(Q)^{p})\,\mu(Q) & = 
\Sigma_p^\PP(\Reg) +\Sigma_p^\QQ(\Reg) \lesssim \Lambda_*^{-1/4}\,\sigma_p(\HD_1).
\end{align*}

Taking into account that any cube from $\Reg\setminus \Reg_{\Ot}$ 
is contained in some cube $S\in\End$, we derive
\begin{align}\label{eqI0**}
I_0  & \lesssim \sum_{S\in\End} \int_S \big|(|\RR_{\TT_\Reg}\mu(y)| - 
\frac{c_3}4 \Theta(\HD_1))_+\big|^{p}\,d\mu(y) \\
&\quad
+ \sum_{Q\in\Reg_{\Ot}} \int_Q \big|(|\RR_{\TT_\Reg}\mu(y)| - 
\frac{c_3}4 \Theta(\HD_1))_+\big|^{p}\,d\mu(y) + \Lambda_*^{-1/4}\,\sigma_p(\HD_1)\nonumber\\
& =:I_{\End} + I_\Ot + \Lambda_*^{-1/4}\,\sigma_p(\HD_1).\nonumber
\end{align}

\vv
\noi {\bf Estimate of $I_\End$.}
Recall that
$$\wt\End =  \LD_1 \cup \LD_2 \cup \HD_2  \cup\sM_\Neg.$$
We split
\begin{align}\label{eqfkzx23}
I_{\End} &= \sum_{S\in\wt\End} \int_S \big|(|\RR_{\TT_\Reg}\mu(y)| - 
\frac{c_3}4 \Theta(\HD_1))_+\big|^{p}\,d\mu(y)\\
&\quad + \sum_{Q\in\Reg_\Neg\setminus \Reg_{\sM_\Neg}} \int_Q \big|(|\RR_{\TT_\Reg}\mu(y)| - 
\frac{c_3}4 \Theta(\HD_1))_+\big|^{p}\,d\mu(y).\nonumber
\end{align}
We claim that the second sum on the right hand side vanishes. Indeed, by \eqref{eqcad35}, given $Q\in\Reg_\Neg\setminus \Reg_{\sM_\Neg}$, all the cubes $S$ such that $Q\subset S\subset R$ satisfy
$$\Theta(S)\lesssim \left(\frac{\ell(S)}{\ell(R)}\right)^{1/2}\,\Theta(R),$$
and so, for any $x\in Q$,
\begin{align*}
|\RR_{\TT_\Reg}\mu(x)| &\lesssim \sum_{S:Q\subset S\subset R} \Theta(S)  \lesssim \sum_{S:Q\subset S\subset R}\left(\frac{\ell(S)}{\ell(R)}\right)^{1/2}\,\Theta(R) \lesssim \Theta(R).
\end{align*}
Hence, for $\Lambda_*$ big enough, $(|\RR_{\TT_\Reg}\mu(x)| - \frac{c_3}4 \Theta(\HD_1))_+=0$, which proves our claim.

To estimate the first sum on right hand side of \rf{eqfkzx23} notice that, for each $S\in\wt\End$, by the triangle inequality and the fact that $(\;\cdot\;)_+$ is a $1$-Lipschitz function, we have 
\begin{align*}
\int_S \big|(|\RR_{\TT_\Reg}\mu| - 
\frac{c_3}4 \Theta(\HD_1))_+\big|^{p}\,d\mu &\lesssim 
\int_S \big|(|\RR_{\wt \TT}\mu| - 
\frac{c_3}4 \Theta(\HD_1))_+\big|^{p}\,d\mu \\
&\quad+ \int_S\big|\RR_{\wt\TT}\mu - \RR_{\TT_\Reg}\mu\big|^{p}\,d\mu
\end{align*}
By Lemma \ref{lemaprox3}, the last integral does not exceed 
$C\EE(2S)^{\frac{p}2} \,\mu(S)^{1-\frac{p}2}$, and thus we deduce that
\begin{equation}\label{eqIa1}
I_\End\lesssim
\sum_{S\in\wt\End}\int_S \big|(|\RR_{\wt\TT}\mu| - 
\frac{c_3}4 \Theta(\HD_1))_+\big|^{p}\,d\mu + \sum_{S\in\wt\End}\EE(2S)^{\frac{p}2} \,\mu(S)^{1-\frac {p}2}.
\end{equation}

Next we apply Lemma \ref{lemaprox2}, which ensures that
for any $S\in\wt\End$ and all $x\in S$,
\begin{equation}\label{eqIa2}
|\RR_{\wt\TT}\mu(x)| \leq |\Delta_{\wt\TT}\RR\mu(x)| + C\left(\frac{\EE(4R)}{\mu(R)}\right)^{1/2} +  C\left(\frac{\EE(2S)}{\mu(S)}\right)^{1/2} + C\PP(R) + C\PP(S).
\end{equation}
Note that since $R$ and $S$ are $\PP$-doubling, $\PP(R)$ is bounded from above by the second term, and $\PP(S)$ is bounded by the third term.

In  case that $R\not\in\HE$,
\begin{equation}\label{eqEr4}
C\left(\frac{\EE(4R)}{\mu(R)}\right)^{1/2} \leq C\,M_0\,\Theta(R)\leq \frac{c_3}{20} \Theta(\HD_1),
\end{equation}
When $R\in\HE$, 
since $\TT$ is $\gamma$-nice, for $\gamma$ small enough we have
\begin{equation}\label{eqEr5}
\frac{\EE(4R)}{\mu(R)} \leq \frac1{\mu(R)}\sum_{Q\in\HE:Q\sim\TT} \EE(4Q)\leq\frac\gamma{\mu(R)} \,\sigma(\HD_1)
\leq \gamma\,\Theta(\HD_1)^2\ll \left(\frac{c_3}{20} \Theta(\HD_1)\right)^2.
\end{equation}
So in any case we deduce
$$(|\RR_{\wt\TT}\mu(x)| - \frac{c_3}4 \Theta(\HD_1))_+ \leq 
|\Delta_{\wt\TT}\RR\mu(x)| + C\left(\frac{\EE(2S)}{\mu(S)}\right)^{1/2}.$$
Plugging this estimate into \rf{eqIa1} and applying H\"older's inequality, we get
\begin{align}\label{eqIa99}
I_\End &\lesssim
\sum_{S\in\wt\End}\int_S |\Delta_{\wt\TT}\RR\mu|^{p}\,d\mu + \sum_{S\in\wt\End}\EE(2S)^{\frac {p}2} \,\mu(S)^{1-\frac {p}2}\\
& \leq \|\Delta_{\wt\TT}\RR\mu\|_{L^2(\mu)}^{p}\,
\mu(R)^{1-\frac{p}2} + \sum_{S\in\wt\End}\EE(2S)^{\frac {p}2} \,\mu(S)^{1-\frac {p}2}.\nonumber
\end{align}

Regarding the second term in \rf{eqIa99}, by H\"older's inequality again,
\begin{align*}
\sum_{S\in\wt\End}\EE(2S)^{\frac{p}2} \,\mu(S)^{1-\frac {p}2} &= 
\sum_{S\in\LD_1\cup \LD_2\cup\sM_\Neg}
\EE(2S)^{\frac{p}2} \,\mu(S)^{1-\frac {p}2} + \sum_{S\in\HD_2}
\EE(2S)^{\frac{p}2} \,\mu(S)^{1-\frac {p}2}\\
& \leq \bigg(\sum_{S\in\LD_1\cup \LD_2\cup\sM_\Neg}\EE(2S)\bigg)^{\frac{p}2} \,\bigg(\sum_{S\in\wt\End} \mu(S)\bigg)^{1-\frac {p}2} \\
&\quad + 
\bigg(\sum_{S\in\HD_2}\EE(2S)\bigg)^{\frac{p}2} \,\bigg(\sum_{S\in\HD_2} \mu(S)\bigg)^{1-\frac {p}2} 
.
\end{align*}
We estimate the first summand on the right hand side using \rf{eqlem*2}, \rf{eqlemneg02}, \rf{eqmuhd1},
and the choice of the constants $\Lambda_*$, $M_0$, $B$, and $\gamma$ in Remark \ref{rem9.12}:
\begin{align*}
\bigg(\sum_{S\in\LD_1\cup \LD_2\cup\sM_\Neg}\EE(2S)&\bigg)^{\frac{p}2} \,\bigg(\sum_{S\in\wt\End} \mu(S)\bigg)^{1-\frac {p}2} \\
& \leq \bigg(
\sum_{S\in
\LD_1\cup \LD_2} \EE(2S)  + \sum_{S\in\sM_\Neg} \EE(2S)\bigg)^{\frac{p}2}
\mu(R)^{1-\frac p2}
 \\
& \lesssim \big(B\,M_0^2\,\Lambda_*^6\,\delta_0 + B\,M_0^2\,\Lambda_*^{-4}
+\gamma\big)^{\frac{p}2}\,\sigma(\HD_1)^{\frac{p}2}\,\big(B\Lambda_*^2\,\mu(HD_1)\big)^{1-\frac p2}\\
& \lesssim \big(B\,M_0^2\,\Lambda_*^2\,\delta_0 + B\,M_0^2\,\Lambda_*^{-4}
+\gamma\big)^{\frac{p}2}\,\big(B\Lambda_*^2\big)^{1-\frac p2}\sigma_p(\HD_1)\\
& \lesssim
\big(B\,M_0^2\,\Lambda_*^2\,\delta_0^{\frac12}  + BM_0^2\Lambda_*^{-1}
+B\Lambda_*^2\gamma^{\frac12}\big) \sigma_p(\HD_1)\leq \Lambda_*^{\frac{-1}2}\,\sigma_p(\HD_1).
\end{align*}

On the other hand, by \rf{eqlem*2.5} and using that $\mu(HD_2)\leq B\,\Lambda_*^{-2}\mu(HD_1)$ (by the definition of the tractable trees), we get
\begin{align*}
\bigg(\sum_{S\in\HD_2}\EE(2S)\bigg)^{\frac{p}2} \,\bigg(\sum_{S\in\HD_2} \mu(S)\bigg)^{1-\frac {p}2} 
&\lesssim
\big(B\,M_0^2 + \gamma\big)^{\frac p2}\,\sigma(\HD_1)^{\frac p2}\,\mu(HD_2)^{1-\frac p2}\\
&\leq
\big(B\,M_0^2 + \gamma\big)^{\frac p2}\,(B\,\Lambda_*^{-2})^{1-\frac p2}\,\sigma_p(\HD_1)\\
& \lesssim \big(BM_0^p\Lambda_*^{p-2} + \gamma^{\frac12}\big)\,\sigma_p(\HD_1).
\end{align*}
Assuming $p\leq3/2$, the right hand side is at most $\Lambda_*^{\frac{-1}4}\,\sigma_p(\HD_1)$,

From the last estimates and \rf{eqIa99}, we infer that
$$I_\End \lesssim
 \|\Delta_{\wt\TT}\RR\mu\|_{L^2(\mu)}^{p}\,
\mu(R)^{1-\frac{p}2} + \Lambda_*^{\frac{-1}4}\,\sigma_p(\HD_1).$$

\vv
\noi {\bf Estimate of $I_\Ot$.} 
By Remark \rf{remregot}, every $Q\in\Reg_\Ot$ satisfies $\ell(Q)\approx\ell_0$. On the other hand, by Lemma \ref{lemnegs}, the cubes $P\in\Neg(e'(R))$, satisfy $\ell(P) \gtrsim \delta_0^{2}\,\ell(R)$.
Thus, assuming $\ell_0\ll\delta_0^2$, we have $Q\not\in\Neg(e'(R))$.

To estimate $I_{\Ot}$,
denote by $\sM_\Ot$ the family of maximal $\PP$-doubling cubes which are contained in some cube from $\Reg_{\Ot}$
and let
$$N_\Ot = \bigcup_{Q\in \Reg_{\Ot}} Q\setminus \bigcup_{P\in \sM_{\Ot}} P.$$
We claim that 
\begin{equation}\label{eqinclu827}
\sM_\Ot\subset \TT\quad \text{ and }\quad N_\Ot\subset Z.
\end{equation}
To check this, for a given $P\in\sM_\Ot$ with $P\subset Q\in\Reg_{\Ot}$, suppose there exists $S\in\End$ such that $S\supset P$. We have $S\subsetneq Q$ because $Q\in\Reg_{\Ot}$ implies that $Q\not\subset S$. Since $Q\in\Reg_{\Ot}\setminus \Neg(e'(R))$, we  have
$S\not\in \Neg$. 
As $S$ is $\PP$-doubling, we deduce that $P\supset S$, by the maximality of $P$ as $\PP$-doubling cube contained in $Q$. An analogous argument shows that $N_\Ot\subset Z$.


By H\"older's inequality and \rf{eqlimot62}, for $\ell_0$ small enough we have
\begin{align*}
I_\Ot &\leq \bigg(\sum_{Q\in\Reg_{\Ot}} \int_Q \big|(|\RR_{\TT_\Reg}\mu(x)| - 
\frac{c_3}4 \Theta(\HD_1))_+\big|^{2}\,d\mu(x)\bigg)^{\frac p2} \bigg(\sum_{Q\in\Reg_{\Ot}}\mu(Q)\bigg)^{1-\frac p2}\\
& \leq
\bigg(\sum_{P\in\sM_{\Ot}} \int_P \big|\RR_{\TT_\Reg}\mu\big|^{2}\,d\mu + \int_{N_\Ot} \big|\RR_{\TT_\Reg}\mu\big|^{2}\,d\mu
\bigg)^{\frac p2} \bigg(\mu(Z) + o(\ell_0)\bigg)^{1-\frac p2},
\end{align*}
with $o(\ell_0)\to 0$ as $\ell_0\to0$. 

Denote
$$\RR_{\sM_\Ot}\mu(x) = \sum_{P\in\sM_\Ot} \chi_P(x)\,\RR(\chi_{2R\setminus 2P}\mu)(x)$$
and
$$\Delta_{\sM_\Ot}\RR\mu(x) =
\sum_{P\in\sM_\Ot} \chi_P(x)\,\big(m_{\mu,P}(\RR\mu) - m_{\mu,2R}(\RR\mu)\big)
+ \chi_{Z}(x) \big(\RR\mu(x) -  m_{\mu,2R}(\RR\mu)\big)
.$$
Notice that, for $x\in P\in\sM_\Ot$ and $Q\in \Reg_\Ot\setminus \Neg(e'(R))$ such that $Q\supset P$, since there are no $\PP$-doubling cubes $P'$ such that $P\subsetneq P'\subset Q$,
$$\big|\RR_{\sM_\Ot}\mu(x) - \RR_{\TT_\Reg}\mu(x)\big| = |
\RR(\chi_{2 Q\setminus 2P}\mu)(x)|\lesssim \sum_{P: P\subset P'\subset Q} \Theta(P') \overset{\eqref{eqcad35}}{\lesssim} \PP(Q)\overset{\eqref{eq:Regdens}}{\lesssim}
\Lambda_*\,\Theta(\HD_1).$$
Almost the same argument shows also that, for $x\in N_\Ot$,
$$\big|\RR(\chi_{2R}\mu)(x) - \RR_{\TT_\Reg}\mu(x)\big| \lesssim
\Lambda_*\,\Theta(\HD_1).$$
%Therefore,
%$$\int_{N_\Ot} \big|\RR(\chi_{2R}\mu)\big|^{2}\,d\mu \leq 
%\int_Z \big|\RR(\chi_{2R}\mu)\big|^{2}\,d\mu.$$
Then, for $\ell_0$ small enough, we deduce that
$$I_\Ot \lesssim \bigg(\sum_{P\in\sM_{\Ot}} \int_P \big|\RR_{\sM_\Ot}\mu\big|^{2}\,d\mu + \int_Z \big|\RR(\chi_{2R}\mu)\big|^{2}\,d\mu + \Lambda_*^2\,\Theta(\HD_1)^2\,\mu(R)
\bigg)^{\frac p2} \big(\ve_Z\,\mu(R)\big)^{1-\frac p2}.$$ 


Almost the same arguments as in Lemma \ref{lemaprox2} show that for $x\in P\in\sM_\Ot$,
$$\big|\RR_{\sM_\Ot}\mu(x) - \Delta_{\sM_\Ot}\RR\mu(x)\big| \lesssim \PP(R) + \left(\frac{\EE(4R)}{\mu(R)}\right)^{1/2} + \PP(P) +  \left(\frac{\EE(2P)}{\mu(P)}\right)^{1/2}$$
and that, for $x\in Z$,
$$\big|\RR(\chi_{2R}\mu)(x) - \Delta_{\sM_\Ot}\RR\mu(x)\big| \lesssim \PP(R) + \left(\frac{\EE(4R)}{\mu(R)}\right)^{1/2}.$$
Therefore, by \rf{eqEr4} and \rf{eqEr5} and the fact that $\PP(P)\lesssim\Lambda_*\,\Theta(\HD_1)$ for 
$P\in\sM_\Ot$, we deduce
$$I_\Ot \lesssim \bigg(\int |\Delta_{\sM_\Ot}\RR\mu|^2\,d\mu +
\sum_{P\in\sM_{\Ot}} \EE(2P) + \Lambda_*^2\,\Theta(\HD_1)^2\,\mu(R)
\bigg)^{\frac p2} \big(\ve_Z\,\mu(R)\big)^{1-\frac p2}.$$ 

By the orthogonality of the functions $\Delta_Q\RR\mu$, $Q\in\DD_\mu$ and \rf{eqinclu827}, it is clear that
$$\int |\Delta_{\sM_\Ot}\RR\mu|^2\,d\mu\leq \|\Delta_{\wt\TT}\RR\mu\|_{L^2(\mu)}^2.$$
On the other hand, since the tree $\TT$ is $\gamma$-nice and $\sM_\Ot\subset\TT$,
\begin{align*}
\sum_{P\in\sM_{\Ot}} \EE(2P) & \leq \sum_{P\in\sM_{\Ot}\setminus\HE} \EE(2P) + \sum_{P\in \TT\cap\HE} \EE(2P)
\leq M_0^2\,\sigma(\sM_{\Ot}) + \gamma\,\sigma(\HD_1)\\
&
\leq M_0^2\Lambda_*^2\,\Theta(\HD_1)^2\,\mu(R)+ \gamma\,\sigma(\HD_1)\lesssim M_0^2\Lambda_*^2\,\Theta(\HD_1)^2\,\mu(R).
\end{align*}
Thus, using also \rf{eqmuhd1},
\begin{align*}
I_\Ot & \lesssim \|\Delta_{\wt\TT}\RR\mu\|_{L^2(\mu)}^p \,\mu(R)^{1-\frac p2}+ \ve_Z^{1-\frac p2}\,M_0^p\,\Lambda_*^p\,\Theta(\HD_1)^p\,\mu(R)\\
&
\lesssim \|\Delta_{\wt\TT}\RR\mu\|_{L^2(\mu)}^p \,\mu(R)^{1-\frac p2}+ \ve_Z^{1-\frac p2}\,M_0^p\,B\,\Lambda_*^{2+p}\,\sigma_p(\HD_1)\\
&\lesssim \|\Delta_{\wt\TT}\RR\mu\|_{L^2(\mu)}^p\, \mu(R)^{1-\frac p2}+ \ve_Z^{\frac 12}\,M_0^2\,B\,\Lambda_*^{4}\,\sigma_p(\HD_1).
\end{align*}
Remark that, by the choice of $\ve_Z$ in Remark \ref{rem9.12}, we have $\ve_Z^{\frac 12}\,M_0^2\,B\,\Lambda_*^{4}\leq \Lambda_*^{-1}$.
\vv

From \rf{eqI0038}, \rf{eqI0**}, and the estimates for $I_\End$ and $I_\Ot$, with $p=3/2$, we derive
$$\Lambda_*^{-3\ve_n}\sigma_{3/2}(\HD_1)\lesssim I_0 \lesssim
\|\Delta_{\wt\TT}\RR\mu\|_{L^2(\mu)}^{3/2} \mu(R)^{1/4}+ \Lambda_*^{{-1}/4}\,\sigma_{3/2}(\HD_1).$$
Thus, taking
$$\ve_n=\frac1{15},$$
we get
$$\Lambda_*^{-1/5}\sigma_{3/2}(\HD_1) \lesssim
\|\Delta_{\wt\TT}\RR\mu\|_{L^2(\mu)}^{3/2} \mu(R)^{1/4}.$$
Hence,
\begin{align*}
\|\Delta_{\wt\TT}\RR\mu\|_{L^2(\mu)}^2 &\gtrsim \Big(\Lambda_*^{-1/5}\sigma_{3/2}(\HD_1)\,\mu(R)^{-1/4}\Big)^{4/3}\\
& \overset{\eqref{eqmuhd1}}{\gtrsim} \Big(\Lambda_*^{-1/5}\sigma_{3/2}(\HD_1)\,\big(B\Lambda_*^2\,\mu(HD_1)\big)^{-1/4}\Big)^{4/3}
\gtrsim \Lambda_*^{-2}\,\sigma(\HD_1),
\end{align*}
which proves the lemma.
\end{proof}



\vv
