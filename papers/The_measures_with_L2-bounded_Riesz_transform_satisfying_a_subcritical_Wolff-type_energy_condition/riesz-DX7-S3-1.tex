\section{\texorpdfstring{$\PP$}{P}-doubling cubes, the Main Theorem, the corona decomposition, and the Main Lemma}\label{sec:Pdoubling}

In the rest of the paper we assume that $\mu$ is a compactly supported Radon measure in $\R^{n+1}$ with polynomial 
growth of degree $n$ and
that $\DD_\mu$ is a David-Mattila dyadic lattice satisfying the properties 
described in the preceding section, in particular, the ones in Lemmas \ref{lemcubs}
and \ref{lemDMimproved}, with $\gamma=9/10$.
By rescaling, we assume that $\DD_{\mu,k}$ is defined for all $k\geq k_0$, with $A_0^{-k_0}\approx
\diam(\supp\mu)$, and we also assume that there is a unique cube in $\DD_{\mu,k_0}$ which coincides
with the whole $\supp\mu$.
Further, from now on, we allow all the constants $C$ and all implicit constants in the relations
``$\lesssim$'', ``$\approx$", to depend on the parameters $C_0, A_0$ of the dyadic lattice of David-Mattila.


\subsection{\texorpdfstring{$\PP$}{P}-doubling cubes and the family \texorpdfstring{$\hd^k(Q)$}{hdk(Q)}}\label{subsec:Pdoubling}

We denote
$$\PP(Q) = \sum_{R\in\DD_\mu:R\supset Q} \frac{\ell(Q)}{\ell(R)^{n+1}} \,\mu(2B_R).$$
We say that a cube $Q$ is $\PP$-doubling if
$$\PP(Q) \leq C_d\,\frac{\mu(2B_Q)}{\ell(Q)^n},$$
for  $C_d =4A_0^n$. We write
$$Q\in\DD_\mu^\PP.$$
Notice that
$$\PP(Q) \approx_{C_0} \sum_{R\in\DD_\mu:R\supset Q} \frac{\ell(Q)}{\ell(R)} \,\theta_\mu(2B_R).$$
and thus saying that $Q$ is $\PP$-doubling implies that 
$$\sum_{R\in\DD_\mu:R\supset Q} \frac{\ell(Q)}{\ell(R)} \,\theta_\mu(2B_R)\leq C_d'\,\theta_\mu(2B_Q)$$
for some $C_d'$ depending on $C_d$. Conversely, the latter condition implies that $Q$ is $\PP$-doubling with another constant $C_d$ depending on $C_d'$.

From the properties of the David-Mattila lattice, we deduce the following.

\begin{lemma}\label{lempois00}
	Suppose that $C_0$ and $A_0$ are chosen suitably. If $Q$ is $\PP$-doubling, then $Q\in\DD_\mu^{db}$.
	Also, any cube $R\in\DD_\mu$ such that $R\cap 2Q\neq\varnothing$ and $\ell(R)=A_0\ell(Q)$ belongs to $\DD_\mu^{db}$.
\end{lemma}

\begin{proof}
	Let $Q\in\DD_\mu^\PP$. Regarding the fist statement of the lemma, if $Q\not\in\DD_\mu^{db}$, by \rf{eqdob23} we have
	$$\mu(2B_Q)\leq \mu(100B(Q))\leq C_0^{-l}\,\mu(100^{l+1}B(Q))\quad
	\mbox{for all $l\geq1$ with $100^l\leq C_0$}.$$
	In particular, if $Q'$ denotes the parent of $Q$,
	$$\mu(2B_Q)\leq C_0^{-l}\,\mu(2B_{Q'})\quad
	\mbox{for all $l\geq1$ with $100^l\leq C_0$}.$$
	So,
	\begin{equation}\label{eqigu8208}
		\mu(2B_Q)\leq C_0^{-c\log C_0}\,\mu(2B_{Q'})
	\end{equation}
	for some $c>0$. Using now that $Q$ is $\PP$-doubling, we get
	$$\mu(2B_Q)\leq C_0^{-c\log C_0}\,C_d\,\frac{\ell(Q')^{n+1}}{\ell(Q)^{n+1}} \mu(2B_Q)
	= 4C_0^{-c\log C_0}\,A_0^{2n+1} \mu(2B_Q)
	.$$
	Recall now that, as explained in Remark \ref{rema00}, we assume that $A_0=C_0^{C(n)}$, for some constant
	$C(n)$ depending just on $n$. Then, clearly, the preceding inequality fails if $C_0$ is big enough, which gives the desired contradiction.
	\vv
	
	To prove the second statement of the lemma, suppose that $Q\in\DD_{\mu,k}^\PP$ and let $R\in\DD_\mu$ be such that $R\cap 2Q\neq\varnothing$ and $\ell(R)=A_0\ell(Q)$.
	By the definition and the fact that $R\subset B_R$ we get
	$$|x_Q-x_R|\leq 3\ell(Q) + r(B_R).$$
	Since $$r(B_R)=28\,r(R) \geq 28A_0^{-k+1} = \frac12\,C_0^{-1}A_0\,\ell(Q)\geq 2500\,\ell(Q),$$
	we deduce that 
	$$2B_Q\subset 2B_R.$$
	If $R'$ denotes the parent of $R$ and $Q'''$ the great-grandparent of $Q$ (so that $\ell(Q''')=A_0^3\ell(Q)=A_0\ell(R')$), by an analogous argument we infer that
	$$2B_{R'}\subset 2B_{Q'''}.$$
	Then, using also that $Q$ is $\PP$-doubling, we obtain
	$$\mu(2B_{R'})\leq \mu(2B_{Q'''}) \leq C_d\,\left(\frac{\ell(Q''')}{\ell(Q)}\right)^{n+1}\!\mu(2B_Q)
	\leq C_d\,A_0^{3(n+1)}\,\mu(2B_R) = 4A_0^{4n+3}\,\mu(2B_R).$$
	If $R\not\in\DD_\mu^{db}$, arguing as in \rf{eqigu8208}, we infer that
	$$\mu(2B_R)\leq C_0^{-c\log C_0}\,\mu(2B_{R'}),$$
	which contradicts the previous statement if $C_0$ is big enough (recalling that $A_0=C_0^{C(n)}$).
\end{proof}

\vv
Notice that, by the preceding lemma, if $Q$ is $\PP$-doubling, then 
$$\sum_{R\in\DD_\mu:R\supset Q} \frac{\ell(Q)^{n+1}}{\ell(R)^{n+1}} \,\mu(2B_R) \lesssim_{C_d} \mu(Q).$$



For technical reasons that will be more evident below, it is appropriate to consider a discrete version of the density $\theta_\mu$. Given  $Q\in\DD_\mu$, we let
$$\Theta(Q) = A_0^{kn} \quad \mbox{ if\, $\dfrac{\mu(2B_Q)}{\ell(Q)^n}\in [A_0^{kn},A_0^{(k+1)n})$}.$$
Clearly, $\Theta(Q)\approx \theta_\mu(2B_Q)$.
Notice also that if $\Theta(Q) = A_0^k$ and $P$ is a son of $Q$, then
$$
\frac{\mu(2 B_P)}{\ell(P)^n} \leq \frac{\mu(2 B_Q)}{\ell(P)^n} = A_0^n\,\frac{\mu(2 B_Q)}{\ell(Q)^n}.
$$
Thus,
\begin{equation}\label{eqson1}
	\Theta(P)\leq A_0^n\,\Theta(Q)\quad \mbox{  for every son $P$ of $Q$.}
\end{equation}


Given $Q\in\DD_\mu$ and $k\geq1$, we denote by $\hd^k(Q)$ the family of maximal cubes $P\in\DD_\mu$ satisfying
\begin{equation}\label{a0tilde}
	\ell(P)<\ell(Q), \qquad \Theta(P)\geq  A_0^{kn}\Theta(Q).
\end{equation}
%From \rf{eqson1}, it follows that if $P\in\hd^k(Q)$, then $\Theta(P)= A_0^{kn}\Theta(Q)$.



\vv
\begin{lemma}\label{lempdoubling}
	Let $Q\in\DD_\mu$ be $\PP$-doubling. Then, for $k\geq4$, every $P\in\hd^k(Q)\cap\DD_\mu(4Q)$ is also $\PP$-doubling
	and moreover $\Theta(P)=A_0^{kn}\Theta(Q)$.
\end{lemma}

Remark that this lemma implies that, under the assumptions in the lemma,
$$\Theta(P)\approx A_0^{kn}\,\Theta(Q)\quad\mbox{ for all $k\geq 1$.}$$

\begin{proof}
	First we show that $\Theta(P)=A_0^{kn}\Theta(Q)$. 
	The fact that $\Theta(P)\geq A_0^{kn}\,\Theta(Q)$ is clear. To see the converse inequality,
	denote by $\wh Q$ the parent of $Q$. Notice that any cube $S\subset 4Q$ with $\ell(S)=\ell(Q)$
	satisfies
	$$\frac{\mu(2B_S)}{\ell(S)^n}\leq \frac{\mu\big(2B_{\wh Q}\big)}{\ell(Q)^n} = A_0^{n}\frac{\mu(2B_{\wh Q})}{\ell(\wh Q)^n} \leq A_0^{n+1}\,\PP(Q) \leq C_d\,A_0^{n+1}\,\frac{\mu(2B_Q)}{\ell(Q)^n} < A_0^{3n}\,\frac{\mu(2B_Q)}{\ell(Q)^n}.$$
	Therefore,
	$$\Theta(S)\leq A_0^{3n}\,\Theta(Q).$$
	As a consequence, if $P\in\hd^k(Q)\cap\DD_\mu(4Q)$ with $k\geq4$, then its parent $\wh P$ satisfies
	$\Theta(\wh P)<A_0^{kn}\,\Theta(Q)$, which implies that $\Theta(P)\leq A_0^{kn}\,\Theta(Q)$.
	
	\vv
	To see that $P$ is $\PP$-doubling,
	we split
	\begin{equation}\label{eqvdx1}
		\PP(P) =\sum_{\substack{R\in\DD_\mu:R\supset P\\ \ell(R)\leq \ell(Q)}} \frac{\ell(P)}{\ell(R)^{n+1}} \,\mu(2B_R)+
		\sum_{\substack{R\in\DD_\mu:R\supset P\\ \ell(R)> \ell(Q)}}\frac{\ell(P)}{\ell(R)^{n+1}} \,\mu(2B_R).
	\end{equation}
	The cubes $R$ in the first sum on the right hand side satisfy
	$\Theta(R)\leq \Theta(P)$, by the definition of $\hd^k(Q)$. Thus,
	$$\sum_{\substack{R\in\DD_\mu:R\supset P\\ \ell(R)\leq \ell(Q)}} \frac{\ell(P)}{\ell(R)^{n+1}} \,\mu(2B_R)\leq 
	A_0^n 
	\sum_{\substack{R\in\DD_\mu:R\supset P\\ \ell(R)\leq \ell(Q)}}  \frac{\ell(P)}{\ell(R)} \,\Theta(R)\leq 
	2A_0^n\,\Theta(P)\leq 
	2A_0^n\,\frac{\mu(2B_P)}{\ell(P)^n}.$$
	Concerning the last sum in \rf{eqvdx1}, notice that the cubes $R$ in that sum satisfy $\ell(R)
	>\ell(Q)$.
	Using that $A_0\gg1$, it follows easily that $2B_R\subset 2B_{R'}$, where $R'$ is the cube containing $Q$ such that $\ell(R')=A_0\,\ell(R)$.
	Consequently, denoting by $\wh Q$ the parent of $Q$,
	\begin{align*}
		\sum_{\substack{R\in\DD_\mu:R\supset P\\ \ell(R)> \ell(Q)}}\frac{\ell(P)}{\ell(R)^{n+1}} \,\mu(2B_R)
		& \leq \sum_{R'\in\DD_\mu:R'\supset \wh Q} \frac{\ell(P)}{A_0^{-1}\ell(R')} \,\frac{\mu(2B_{R'})}{(A_0^{-1} \ell(R'))^n}\\
		& = A_0^{n+1} \frac{\ell(P)}{\ell(Q)} 
		\sum_{R'\in\DD_\mu:R'\supset \wh Q} \frac{\ell(Q)}{\ell(R')^{n+1}} \,\mu(2B_{R'})\\
		& \leq A_0^{n}\,\PP(Q) \leq A_0^{n}C_d\,\frac{\mu(2Q)}{\ell(Q)^n} \leq  A_0^{2n}C_d\,\Theta(Q)\\
		& 
		\leq \frac{A_0^{2n}C_d}{A_0^{4n}}\,\Theta(P)\leq \frac{C_d}{A_0^{2n}}\,\frac{\mu(2B_P)}{\ell(P)^n},
	\end{align*}
	where in the last two lines we took into account that $\ell(P)\leq A_0^{-1}\ell(Q)$ (because $P\in\hd^k(Q)$ for some $k\geq4$), that 
	$Q$ is $\PP$-doubling, and again that $P\in\hd^k(Q)$ for some $k\geq4$.
	
	From the estimates above, we infer that
	$$\PP(P) \leq \bigg(2A_0^n + \frac{C_d}{A_0^{2n}}\bigg) \,\frac{\mu(2B_P)}{\ell(P)^n} \leq C_d\,\frac{\mu(2B_P)}{\ell(P)^n},$$
	since $C_d= 4A_0^n$.
\end{proof}
\vv


\begin{lemma}\label{lemdobpp}
	Let $Q_0,Q_1,\ldots,Q_m$ be a family of cubes from $\DD_\mu$ such that $Q_j$ is a child of $Q_{j-1}$ for $1\leq j\leq 
	m$. Suppose that $Q_j$ is not $\PP$-doubling for $1\leq j\leq m$.
	Then
	\begin{equation}\label{eqcad35}
		\frac{\mu(2B_{Q_m})}{\ell(Q_m)^n}\leq A_0^{-m/2}\,\PP(Q_0).
	\end{equation}
	and
	\begin{equation}\label{eqcad35'}
		\PP(Q_m)\leq 2A_0^{-m/2}\,\PP(Q_0).
	\end{equation}
	
\end{lemma}


\begin{proof}
	Let us denote
	$\wt\Theta(R) = \frac{\mu(2B_R)}{\ell(R)^n},$ so that
	$$\PP(Q) = \sum_{R\in\DD_\mu:R\supset Q} \frac{\ell(Q)}{\ell(R)} \,\wt\Theta(R).$$
	For $1\leq j \leq m$,
	the fact that $Q_j$ is not $\PP$-doubling implies that
	\begin{equation}\label{eqsak33}
		\wt \Theta(Q_j) \leq \frac1{C_d}\,\PP(Q_j) = \frac1{C_d}\Biggl (\sum_{k=0}^{j-1} \frac{\ell(Q_j)}{\ell(Q_{j-k})}\,
		\wt\Theta(Q_{j-k})+ \frac{\ell(Q_j)}{\ell(Q_0)}\,\PP(Q_0)\Biggr).
	\end{equation}
	We will prove \rf{eqcad35} by induction on $j$. For $j=0$ this is in an immediate consequence of the
	definition of $\PP(Q_0)$. Suppose that \rf{eqcad35} holds for $0\leq h\leq j$, with $j\leq m-1$, and let us 
	consider the case $j+1$. From \rf{eqsak33} and the induction hypothesis we get
	\begin{align*}
		\wt\Theta(Q_{j+1}) & \leq  \frac1{C_d}\Biggl (\wt\Theta(Q_{j+1}) + \sum_{k=1}^j \frac{\ell(Q_{j+1})}{\ell(Q_{j+1-k})}\,
		\wt\Theta(Q_{j+1-k})+ \frac{\ell(Q_{j+1})}{\ell(Q_0)}\,\PP(Q_0)\Biggr)\\
		&= \frac1{C_d}\Biggl (\wt\Theta(Q_{j+1}) + \sum_{k=1}^j A_0^{-k}\,
		\wt\Theta(Q_{j+1-k})+ A_0^{-j-1}\,\PP(Q_0)\Biggr)\\
		&\leq \frac1{C_d}\Biggl (\wt\Theta(Q_{j+1}) + \sum_{k=1}^j A_0^{-k}\,A_0^{(-j-1+k)/2}\PP(Q_0)
		+ A_0^{-j-1}\,\PP(Q_0)\Biggr)
	\end{align*}
	Since 
	$$\sum_{k=1}^j A_0^{-k}\,A_0^{(-j-1+k)/2} = A_0^{(-j-1)/2}\sum_{k=1}^j A_0^{-k/2}\leq A_0^{-j/2},$$
	we obtain
	\begin{align*}
		\wt\Theta(Q_{j+1})  &\leq  \frac1{C_d}\bigl (\wt\Theta(Q_{j+1}) + A_0^{-j/2}\,\PP(Q_0)+ A_0^{-j-1}\,\PP(Q_0)\bigr)\\
		&\leq \frac1{C_d}\bigl (\wt\Theta(Q_{j+1}) + 2\,A_0^{-j/2}\,\PP(Q_0)\bigr) \\
	\end{align*}
	It is straightforward to check that this yields $\wt\Theta(Q_{j+1})\leq A_0^{-(j+1)/2}\,\PP(Q_0)$.
	
	The estimate \rf{eqcad35'} follows easily from \rf{eqcad35}:
	\begin{align*}
		\PP(Q_m) &= \sum_{k=0}^{m-1} \frac{\ell(Q_m)}{\ell(Q_{m-k})}\,
		\wt\Theta(Q_{m-k})+ \frac{\ell(Q_m)}{\ell(Q_0)}\,\PP(Q_0)\\
		&\leq \sum_{k=0}^{m-1} A_0^{-k}\,A_0^{-(m-k)/2}\,\PP(Q_0)+ A_0^{-m}\,\PP(Q_0)\\
		&\leq A^{-m/2}\sum_{k=0}^{m-1} A_0^{-k/2}\,\PP(Q_0)+ A_0^{-m}\,\PP(Q_0)\leq 2\,A_0^{-m/2}\,\PP(Q_0).
	\end{align*}
\end{proof}



\vv

% ********************************************************************************************


\subsection{The Main Theorem}\label{subsec:main thm}


For a given $\lambda\geq1$ and $Q\in\DD_\mu$, we  consider the energy
$$\EE(\lambda Q) = \sum_{P\in\DD_\mu(\lambda Q)} \left(\frac{\ell(P)}{\ell(Q)}\right)^{3/4}\Theta(P)^2\,\mu(P),$$
Given a fixed constant $M_0\gg1$, we write $Q\in \HE$ (which stands for ``high energy'') if
$$\EE(4Q)\geq M_0^2\,\Theta(Q)^2\,\mu(Q).$$

\begin{theorem}[Main Theorem]\label{propomain}
	Let $\mu$ be a Radon measure in $\R^{n+1}$ such that
	$$
	\mu(B(x,r))\leq \theta_0\,r^n\quad \mbox{ for all $x\in\supp\mu$ and all $r>0$}.
	$$
	Then, for any choice of $M_0>1$,
	\begin{equation}\label{eqpropo*}
		\sum_{Q\in\DD_\mu} \beta_{\mu,2}(2B_Q)^2\,\Theta(Q)\,\mu(Q)\leq C\,\big(\|\RR\mu\|_{L^2(\mu)}^2 + \theta_0^2\,\|\mu\|
		+ \sum_{Q\in\DD_\mu^\PP\cap\HE}\EE(4Q)\big),
	\end{equation}
	with $C$ depending on $M_0$.
\end{theorem}


\vv
Notice that Theorem \ref{teomain} follows from the previous result. Indeed, the assumptions in Theorem \ref{teomain}
imply that there are no $\PP$-doubling cubes from $\HE$ if $M_0$ is big enough since,  for any $Q\in\DD_\mu^\PP$,
\begin{align*}
	\EE(4Q) &= \sum_{P\in\DD_\mu(4 Q)} \left(\frac{\ell(P)}{\ell(Q)}\right)^{3/4}\Theta(P)^2\,\mu(P)\\
	&
	\lesssim \int_{B(x_Q,20\ell(Q))} \int_0^\infty \left(\frac{\mu(B(x,r)\cap B(x_Q,20\ell(Q))}{r^n}\right)^2   \,\left(\frac r{\ell(Q)}\right)^{3/{4}}\,\frac{dr}r\,d\mu(x)\\
	& = \frac{\mathbb E(\mu\rest_{B(x_Q,20\ell(Q))})}{\ell(Q)^{3/4}} \lesssim M\,\theta_\mu(B(x_Q,40\ell(Q)))^2\,\mu(B(x_Q,40\ell(Q)))
	\approx M\,\Theta(Q)^2\,\mu(Q).
\end{align*}
So the last sum on the right hand side of \rf{eqpropo*} vanishes in this case.
Further, it is also easy to check that
$$\int \!\!\int_0^\infty \beta_{\mu,2}(x,r)^2\,\theta_\mu(B(x,r))\,
\frac{dr}r\,d\mu(x) \lesssim \sum_{Q\in\DD_\mu} \beta_{\mu,2}(2B_Q)^2\,\Theta(Q)\,\mu(Q).$$
Putting altogether, from \rf{eqpropo*} we get \rf{eqteo*}.

\vv
% ********************************************************************************************
% ********************************************************************************************
% ********************************************************************************************


\subsection{The corona decomposition and the Main Lemma}\label{sec3.3}

In order to prove Theorem \ref{propomain} we have to use a suitable corona decomposition which splits
the lattice $\DD_\mu$ into appropriate trees. We need first to introduce some notation.

Given a big integer $k_\Lambda>10$ to be fixed below and $R\in\DD_\mu^\PP$, we denote 
$$\Lambda=A_0^{k_\Lambda\,n},\qquad \HD(R) = \hd^{k_\Lambda}(R).$$
Also, we consider a small constant $\delta_0 \in (0,\Lambda^{-C(n)})$ which will be chosen below too, with $C(n)>2$.
We let $\LD(R)$ be the family of cubes $Q\in\DD_{\mu}$
which
%\footnote{Notice we ask cubes from $\LD(R)$ to be $\PP$-doubling. It has to be checked this is OK.} 
are maximal and satisfy
$$\ell(Q)< \ell(R)\quad \mbox{ and }\quad\PP(Q) \leq \delta_0\,\Theta(R).$$
We denote by $\sss(R)$ the family of maximal cubes from $\HD(R)\cup\LD(R)$ which are contained in $R$.
Also, we let $\End(R)$ be the family of maximal $\PP$-doubling cubes which are contained in some cube
from $\sss(R)$. Notice that, by Lemma \ref{lempdoubling}, the cubes from $\HD(R)\cap \DD_\mu(4R)$ are $\PP$-doubling, and thus
any cube from $\sss(R)\cap\HD(R)$ belongs to $\End(R)$. Finally, we let $\tree(R)$ denote the subfamily of the cubes from $\DD_\mu(R)$ which are not strictly contained in any cube
from $\End(R)$, and we say that $R$ is the root of the tree.


Next we define the family $\ttt$ inductively. 
We assume that $\supp\mu$ coincides with a cube $S_0$, and then
we set $\ttt_0=\{S_0\}$. Assuming $\ttt_k$ to be defined, we let
$$\ttt_{k+1} = \bigcup_{R\in\ttt_k} \End(R).$$
Then we let
$$\ttt =\bigcup_{k\geq0} \ttt_k.$$
Notice that we have 
$$\DD_\mu=\bigcup_{R\in\ttt} \tree(R).$$
Two trees $\tree(R)$, $\tree(R')$, with $R,R'\in\ttt$, $R\neq R'$ can only intersect if one
of the roots is and ending cube of the other, i.e., $R'\in\End(R)$ or $R\in\End(R')$.


\vv

The main step for the proof of Theorem \ref{propomain} consists of proving the following.

\begin{mlemma}\label{mainlemma}
	Let $\mu$ be a Radon measure in $\R^{n+1}$ such that
	$$
	\mu(B(x,r))\leq \theta_0\,r^n\quad \mbox{ for all $x\in\supp\mu$ and all $r>0$}.
	$$
	Then, for any choice of $M_0>1$,
	\begin{equation}\label{eqmainlemma*}
		\sum_{R\in\ttt} \Theta(R)^2\,\mu(R)\leq C\,\big(\|\RR\mu\|_{L^2(\mu)}^2 + \theta_0^2\,\|\mu\|
		+ \sum_{Q\in\DD_\mu^\PP\cap\HE}\EE(4Q)\big),
	\end{equation}
	with $C$ depending on $M_0$.
\end{mlemma}

The next Sections \ref{sec4}-\ref{sec8} are devoted to the proof of this lemma. Later, in Section \ref{sec9} we will complete the 
proof of Theorem \ref{propomain}.

\vv