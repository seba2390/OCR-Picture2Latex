

\section{The cubes with moderate decrement of Wolff energy and the associated tractable trees} \label{sec4}

From now on we assume that we are under the assumptions of Main Lemma \ref{mainlemma}. Further, 
for a family $I\subset\DD_\mu$, we denote
$$\sigma(I) = \sum_{P\in I}\Theta(P)^2\,\mu(P).$$

\subsection{The family \texorpdfstring{$\MDW$}{MDW}}\label{secMDW}

For technical reasons, we need to define a kind of variant of the family $\HD(R)$. Given $R\in\DD_\mu^\PP$,
we denote
$$\Lambda_*= \Lambda^{1-\frac1N} = A_0^{k_\Lambda (1-\frac1N)n},\qquad
\HD_*(R) = \hd^{k_\Lambda (1-\frac1N)},$$
where $N$ is some big integer depending on $n$ that will be fixed below. Regarding the constant $\delta_0$ in the definition of $\LD(R)$ in Section \ref{sec3.3}, we will choose $\delta_0$ of the form
$$\delta_0 = \Lambda^{-N_0 - \frac1{2N}} = A_0^{-n\,k_\Lambda(N_0 + \frac1{2N})}$$
for some big integer $N_0$ depending on $n$, chosen later on.
Further we assume that $k_\Lambda$ is a multiple of $2N$, so that $k_\Lambda(N_0 + \frac1{2N})$ and $k_\Lambda (1-\frac1N)$ are integers.

We let $\bad(R)$ be the family of maximal cubes from $\LD(R)\cup\HD_*(R)$ (not necessarily  contained 
in $R$) and we denote
$$\sss_*(R)= \bad(R)\cap\DD_\mu(R).$$


We say that a cube $R\in\DD_{\mu}$ has moderate decrement of Wolff energy, and we write
$R\in\MDW$, if $R$ is $\PP$-doubling and
\begin{equation}\label{eq:MDWdef}
\sigma(\HD_*(R)\cap\sss_*(R))\geq B^{-1}\,\sigma(R),
\end{equation}
where
$$B= \Lambda_*^{\frac1{100n}}.$$  


\vv
\begin{lemma}
For $R\in\ttt\setminus \MDW$, we have
\begin{equation}\label{eqmdw81}
\sigma(\End(R))\leq 2\Lambda^{2/N}B^{-1}\,\sigma(R).
\end{equation}
\end{lemma}

 
\begin{proof}
We have
$$\sigma(\End(R)) = \sigma(\HD(R)\cap\sss(R)) + \sum_{Q\in\LD(R)\cap\sss(R)}\sum_{P\in\End(R):P\subset Q}
\sigma(Q).$$
Clearly, since any cube from $\HD(R)$ is contained in some cube from $\HD_*(R)$, we infer that
\begin{align*}
\sigma(\HD(R)\cap\End(R)) & =\Lambda^2\,\Theta(R)^2\!\sum_{Q\in \HD(R)\cap\End(R)}\!\!\mu(Q)
\leq \Lambda^2\,\Theta(R)^2\!\sum_{Q\in \HD_*(R)\cap\sss_*(R)}\!\!\mu(Q) \\
& = \frac{\Lambda^2}{\Lambda_*^2}\,\sigma
(\HD_*(R)\cap\sss_*(R))\leq \Lambda^{2/N}B^{-1}\,\sigma(R).
\end{align*}
For $P\in\End(R)\setminus\HD(R)$, there exists some $Q$ such that $P\subset Q\in\LD(R)\cap\sss(R)$.  By \eqref{eqcad35} we have 
$$\Theta(P)\lesssim\PP(Q)\leq \delta_0\,\Theta(R),$$
and thus
$$
\sigma(\End(R)) \leq \Lambda^{2/N} B^{-1}\,\sigma(R) + C\delta_0^2\,\sigma(R) \leq 2\Lambda^{2/N} B^{-1}\,\sigma(R),
$$
since $\delta_0\ll B^{-1}$.
\end{proof}
\vv

We will take $N$ big enough so that 
$$2\Lambda^{2/N}\,B^{-1}\leq \frac12.$$


\begin{lemma}\label{lemtoptop}
We have
$$\sigma(\ttt)  \lesssim \sigma(\ttt\cap\MDW)+ \theta_0^2\,\|\mu\|.$$
\end{lemma}

\begin{proof}
For each $R\in \{S_0\}\cup (\ttt\cap\MDW)$, we denote
$I_0(R)=\{R\}$, and for $k\geq0$,
$$I_{k+1}(R) = \bigcup_{Q\in I_k(R)} \End(Q)\setminus \MDW.$$ 
In this way, we have
\begin{equation}\label{eqtttot}
\ttt = \bigcup_{R\in \{S_0\}\cup (\ttt\cap\MDW)} \;\bigcup_{k\geq0} I_k(R).
\end{equation}
Indeed, for each $Q\in \ttt$, let $R$ be the minimal cube from $\ttt\cap\MDW$ that contains $Q$, and in case this does not exists, let $R=S_0$. Then it follows that
$$Q\in \bigcup_{k\geq0} I_k(R).$$

Given $R\in\{S_0\}\cup (\ttt\cap\MDW)$, for each $k\geq 1$ and $Q\in I_k(R)$, 
since $I_k(R)\subset \ttt\setminus \MDW$, by \rf{eqmdw81} we have 
$$\sigma(\End(Q)\setminus \MDW)\leq 2 \Lambda^{2/N}B^{-1}\,\sigma(Q)\leq \frac12\,\sigma(Q).$$
Then we deduce that
$$\sigma(I_{k+1}(R)) = \sum_{Q\in I_k(R)} \sigma(\End(Q)\setminus \MDW) \leq \frac12 \sum_{Q\in I_k(R)}\sigma(Q)
= \frac12\, \sigma(I_k(R)).$$
So 
$$\sigma(I_{k}(R))\leq \frac1{2^k}\,\sigma(R)\quad \mbox{ for each $k\geq0$.}$$
Then, by \rf{eqtttot}, 
\begin{align*}
\sigma(\ttt) & = \sum_{R\in \{S_0\}\cup (\ttt\cap\MDW)} \;\sum_{k\geq0} \sigma(I_k(R))
\leq \sum_{R\in \{S_0\}\cup (\ttt\cap\MDW)}\;\sum_{k\geq0} \frac1{2^k}\,\sigma(R) \\
& \approx \sigma(S_0) + \sigma(\ttt\cap\MDW)
\lesssim \theta_0^2\,\|\mu\|+\sigma(\ttt\cap\MDW).
\end{align*}
\end{proof}

\vv

%In this section we show how to associate to each $R\in\MDW$ a suitable family of tractable trees (to be defined later).

\subsection{The enlarged cubes}\label{subsec:enlar}

For $R\in\MDW$, the fact that the cubes from family $\HD_*(R)\cap\sss_*(R)$ may be located close to $\supp\mu\setminus R$ 
may cause problems when trying to obtain estimates involving the Riesz transform. For this reason we need to
introduce some ``enlarged cubes''.

Given $j\geq0$ and $R\in\DD_{\mu,k}$, 
we let 
$$e_j(R) = R \cup \bigcup Q,$$
where the last union runs over the cubes $Q\in\DD_{\mu,k+1}$ such that
\begin{equation}\label{eqxrq83}
\dist(x_R,Q)< \frac{\ell(R)}2 + 2j\ell(Q).
\end{equation}
We say that $e_j(R)$ is an enlarged cube.
Notice that, since $\diam(Q)\leq \ell(Q)$,
\begin{equation}\label{eqxrq84}
\supp\mu\cap B\big(x_R,\tfrac12\ell(R) + 2j\ell(Q)\big)\subset e_j(R) \subset B\big(x_R,\tfrac12\ell(R) + (2j+1)\ell(Q)\big).
\end{equation}
Also,  we have
\begin{equation}\label{eqqj8d}
e_j(R)\subset 2R\quad \mbox{ for $0\leq j\leq \frac34 A_0$,}
\end{equation}
since, for any $Q\in\DD_{\mu,k+1}$ satisfying \rf{eqxrq83}, its parent satisfies $\wh Q$
$$\dist(x_R,\wh Q)< \frac{\ell(R)}2 + 2j A_0^{-1}\ell(\wh Q) \leq 2\ell(R).$$

%We denote by $\NDB$ the family of cubes $Q$ for which there exists another cube $Q'\in\DB$ of the same generation as $Q$ such that $Q'\subset 9Q$. 
For $R\in\MDW$, we let
$$\sss_*(e_j(R)) = \bad(R) \cap\DD_\mu(e_j(R)),$$
where $\DD_\mu(e_j(R))$ stands for the family of cubes from $\DD_\mu$ which are contained in
$e_j(R)$ and have side length at most $\ell(R)$. Notice that we are not assuming $R\in\ttt$.



\vv
\begin{lemma}\label{lem:43}
For any $R\in\MDW$ there exists some $j$, with $10\leq j\leq A_0/4$ such that
\begin{equation}\label{eqsigmaj}
\sigma(\HD_*(R)\cap\sss_*(e_{j}(R))) \leq B^{1/4} \sigma(\HD_*(R)\cap\sss_*(e_{j-10}(R))),
\end{equation}
 assuming $A_0$ big enough, depending just on $n$. 
\end{lemma}

\begin{proof}
Given $R\in\MDW$, suppose that such $j$ does not exist. Let $j_0$ be the largest integer which is multiple of $10$ and smaller that $A_0/4$.
Then we get
\begin{align*}
\sigma(\HD_*(R)\cap\sss_*(e_{j_0}(R))) & \geq B^{\frac14}\sigma(\HD_*(R)\cap\sss_*(e_{j_0-10}(R)))\\
&\geq
\ldots \geq \big(B^{\frac14}\big)^{\frac{j_0}{10}-1}\sigma(\HD_*(R)\cap\sss_*(R)) \overset{\eqref{eq:MDWdef}}{\geq} B^{\frac{j_0}{40}-\frac54}\sigma(R).
\end{align*}
By \rf{eqqj8d}, we have $e_{j_0}(R)\subset 2R$ and thus
$$\sigma(\HD_*(R)\cap\sss_*(e_{j_0}(R)))= \sum_{Q\in\HD_*(R)\cap \sss_*(e_{j_0}(R))} \Lambda_*^2\Theta(R)^2\mu(Q)\leq 
\Lambda_*^2\Theta(R)^2\mu(2R).$$
Since $R$ is $\PP$-doubling (and in particular $R\in\DD_\mu^{db}$), denoting by $\wh R$ the parent of $R$,  we derive 
\begin{equation}\label{eqdoub*11}
\mu(2R)\leq \mu(2B_{\wh R}) \leq \frac{\ell(\wh R)^{n+1}}{\ell(R)}\,\PP(R) \leq C_d\,A_0^{n+1}
\mu(2B_R)\leq C_0\,C_d\,A_0^{n+1}
\mu(R).
\end{equation}
So we deduce that
$$B^{\frac{j_0}{40}-\frac54}\sigma(R)\leq C_0\,C_d\,A_0^{n+1}\,\Lambda_*^2\sigma(R),$$
or equivalently, recalling the choice of $B$ and $C_d$,
$$\Lambda_*^{\frac{1}{100n}\left(\frac{j_0}{40}-\frac54\right) -2} \leq 4 C_0\,A_0^{2n+1}.$$
Since $\Lambda_*\geq A_0^n$ and $j_0\approx A_0$, it is clear that this inequality is violated if $A_0$ is big enough, depending just on $n$.
\end{proof}
\vv

Given $R\in\MDW$,  let $j\geq 10$ be minimal such that \rf{eqsigmaj} holds. We denote
$h(R)=j-10$ and we write
$$e(R) = e_{h(R)}(R),\qquad e'(R) = e_{h(R)+1}(R), \quad e''(R)=e_{h(R)+2}(R), \quad e^{(k)}(R) = e_{h(R)+k}(R),$$
for $k\geq 1$. 
We let
\begin{align*}
B(e(R)) &= B\big(x_R,(\tfrac12 + 2A_0^{-1}h(R))\ell(R)\big),\\
B(e'(R)) & = B\big(x_R,(\tfrac12 + 2A_0^{-1}(h(R)+1))\ell(R)\big),\\
B(e''(R)) & = B\big(x_R,(\tfrac12 + 2A_0^{-1}(h(R)+2))\ell(R)\big),\\
B(e^{(k)}(R)) & = B\big(x_R,(\tfrac12 + 2A_0^{-1}(h(R)+k))\ell(R)\big).
\end{align*}
By construction (see \rf{eqxrq84}) we have 
$$B(e'(R))\cap\supp\mu\subset e'(R),$$
and analogously replacing $e'(R)$ by $e(R)$ or $e''(R)$.
Remark also that
$$e(R)\subset B(e'(R))\quad \text{ and }\quad \dist(e(R),\partial B(e'(R))) \geq A_0^{-1}\ell(R),$$
and, analogously,
$$e'(R)\subset B(e''(R))\quad \text{ and }\quad \dist(e'(R),\partial B(e''(R))) \geq A_0^{-1}\ell(R).$$


\vv


\begin{lemma}\label{lem-calcf}
For each $R\in\MDW$  we have
$$B(e''(R)) \subset (1+8A_0^{-1})\,\,B(e(R)) \subset B(e^{(6)}(R)),$$
and more generally, for $k\geq 2$ such that $h(R)+k-2\leq A_0/2$,
$$B(e^{(k)}(R)) \subset (1+8A_0^{-1})\,\,B(e^{(k-2)}(R)) \subset B(e^{(k+4)}(R)).$$
Also,
$$B(e^{(10)}(R))\subset B\big(x_R,\tfrac32 \ell(R)\big).$$
\end{lemma}

\begin{proof}
This follows from straightforward calculations. 
Indeed,
\begin{multline*}
r(B(e^{(k)}(R)))  = \frac{(\tfrac12 + 2A_0^{-1}(h(R)+k))\ell(R)}{(\tfrac12 + 2A_0^{-1}(h(R)+k-2))\ell(R)}
\,r(B(e^{(k-2)}(R)))\\
 = 1 + \frac{8A_0^{-1}}{1 + 4A_0^{-1}(h(R)+k-2)}\,r(B(e^{(k-2)}(R)))\leq (1+8A_0^{-1})\,r(B(e^{(k-2)}(R))).
\end{multline*}
Also, using that $h(R)+k-2\leq A_0/2$,
\begin{align*}
(1+8A_0^{-1})\,r(B(e^{(k-2)}(R))) & = (1+8A_0^{-1})\,\big(\tfrac12 + 2A_0^{-1}(h(R)+k-2)\big)\,\ell(R)\\
& \leq \big(\tfrac12 + 2A_0^{-1}(h(R)+k-2) + 4A_0^{-1} + 8A_0^{-1}
\big)\,\ell(R)\\
& =
r(B(e^{(k+4)}(R))).
\end{align*}

The last statement of the lemma follows from the fact that $h(R)+10\leq A_0/4<A_0/2$:
$$
B(e^{(10)}(R)) = B\big(x_R,(\tfrac12 + 2A_0^{-1}(h(R)+10))\ell(R)\big) \subset
B\big(x_R,(\tfrac12 + 2)\ell(R)\big) = B\big(x_R,\tfrac32 \ell(R)\big).$$
\end{proof}
\vv

\subsection{Generalized trees and negligible cubes}\label{subsec:generalized}

Next we need to define some families that can be considered as ``generalized trees''. First, we introduce some additional notation regarding the stopping cubes. For $R\in\DD_{\mu}^\PP$ we set
$$\HD_{1}(R) = \sss_*(R)\cap \HD_*(R).$$ 
Assume additionally that $R\in\MDW$. We write $\sss_*(e(R))=\sss_*(e_{h(R)}(R))$ and $\sss_*(e'(R))=\sss_*(e_{h(R)+1}(R))$. Furthermore,
$$\HD_1(e(R)) = \sss_*(e(R))\cap \HD_*(R),$$
and
$$\HD_{1}(e'(R)) = \sss_*(e'(R))\cap \HD_*(R).$$
We define $\HD_{1}(e^{(k)}(R))$ for $2\le k\le 10$ analogously. Also, we set 
$$\HD_2(e'(R)) = \bigcup_{Q\in \HD_1(e'(R))} (\sss_*(Q)\cap \HD_*(Q))$$
and
\begin{equation}\label{eqstop2}
\sss_2(e'(R)) = \big(\sss_*(e'(R)) \setminus \HD_1(e'(R))\big) \cup \bigcup_{Q\in \HD_1(e'(R))} \sss_*(Q).
\end{equation}
We let $\TT_\sss(e'(R))$ be the family of cubes made up of $R$ and all the cubes of the next generations which are contained in $e'(R)$ but are not 
strictly contained in any cube from $\sss_2(e'(R))$.

Observe that the defining property of $\MDW$ \eqref{eq:MDWdef} can now be rewritten as 
\begin{equation}\label{eq:MDWdef2}
	\sigma(R)\le B\, \sigma(\HD_1(R)).
\end{equation}
Moreover, by \eqref{eqsigmaj} and the definition of $e(R)$ we have
\begin{equation}\label{eq:sigmae'lesigmae}
\sigma(\HD_1(e^{(10)}(R)))\le B^{1/4}\sigma(\HD_1(e(R))).
\end{equation}

We define now the family of negligible cubes. We say that a cube $Q\in\TT_\sss(e'(R))$ is negligible for $\TT_\sss(e'(R))$, and we write $Q\in\Neg(e'(R))$ if
there does not exist any cube from $\TT_\sss(e'(R))$ that contains $Q$ and is $\PP$-doubling.  

\vv
\begin{lemma}\label{lemnegs}
Let $R\in\MDW$. If $Q\in\Neg(e'(R))$, then $Q\subset e'(R)\setminus R$, $Q$ is not contained in any cube from $\HD_1(e'(R))$, and
\begin{equation}\label{eqcostat}
\ell(Q) \gtrsim \delta_0^{2}\,\ell(R).
\end{equation}
\end{lemma}

\begin{proof}
Let $Q\in\Neg(e'(R))$. We have $Q\subset e'(R)\setminus R$ due to the fact that $R$ is $\PP$-doubling.
For the same reason, $Q$ is not contained in any cube from $\HD_1(e'(R))$.


To prove \rf{eqcostat}, assume that $\ell(Q)\leq A_0^{-2}\ell(R)$. Otherwise the inequality is immediate.
By Lemma \ref{lemdobpp}, since all the ancestors $Q_1,\ldots,Q_m$ of $Q$ that are contained in $e'(R)$ are not $\PP$-doubling, it follows that $Q_1$ (the parent of $Q$) satisfies
$$\PP(Q_1)\lesssim A_0^{-m/2-1}\,\PP(Q_m).$$
Because $Q_m\subset e'(R)\subset 2R$ and $\ell(Q_m)=A_0^{-1}\ell(R)$, it is easy to see that $\PP(Q_m)\lesssim \PP(R)\lesssim C_{d}\,\Theta(R)$, and so
\begin{equation*}
\PP(Q_1)\lesssim A_0^{-m/2}\,\Theta(R)\approx\left(\frac{\ell(Q)}{\ell(R)}\right)^{1/2}\,\Theta(R).
\end{equation*}
By the definition of $\LD(R)$, we know that $\PP(Q_1)\geq\delta_0\,\Theta(R)$, which together with 
the previous estimate yields \rf{eqcostat}.
\end{proof}
\vv


The cubes from $\sss_2(e'(R))$ need not be $\PP$-doubling, which is problematic for some of the estimates involving the Riesz transform localized around the
trees $\TT_\sss(e'(R))$ that will be required later. For this reason, we need to consider enlarged versions of them. For $R\in\MDW$, we let $\End(e'(R))$ be the family made up of the following cubes:
\begin{itemize}
\item the cubes from $\sss_2(e'(R))\cap \Neg(e'(R))$,
\item the cubes that are contained in any cube from $\sss_2(e'(R))\setminus \Neg(e'(R))$ which are $\PP$-doubling and, moreover, are maximal.
\end{itemize}
Notice that all the cubes from $\End(e'(R))$ are $\PP$-doubling, with the exception of the ones from $\Neg(e'(R))$.
We let $\TT(e'(R))$ be the family of cubes that are contained in $e'(R)$ and are not 
strictly contained in any cube from $\End(e'(R))$. 


 \vv
 
\subsection{Tractable trees}\label{subsec:trc} 
Given $R\in\MDW$, we say that $\TT(e'(R))$ is tractable (or that $R$ is tractable) if 
$$\sigma(\HD_2(e'(R)))\leq B\,\sigma(\HD_1(e(R))).$$
In this case we write $R\in\Trc$.

 Our next objective consists in showing how we can associate a family of tractable trees to any $R\in\MDW\cap\ttt$, so that we can reduce the estimate of $\sigma(\ttt)$ to estimating the Haar coefficients of $\RR\mu$ from below on such family of tractable trees.
First we need the following lemma.

\begin{lemma}\label{lemalg1}
Let $R\in\MDW$ be such that $\TT(e'(R))$ is not tractable. Then there exists a family $\GH(R)\subset
\HD_1(e'(R))\cap\MDW$ satisfying:
\begin{itemize}
\item[(a)] The balls $B(e''(Q))$, with $Q\in\GH(R)$ are pairwise disjoint.
\item[(b)] For every $Q\in\GH(R)$, $\sigma(\HD_1(e(Q)))\geq \sigma(\HD_1(Q))\geq B^{1/2}\sigma(Q)$.
\item[(c)] $$B^{1/4} \sum_{Q\in\GH(R)} \sigma(\HD_1(e(Q))) \gtrsim \sigma(\HD_2(e'(R))).$$
\end{itemize}
\end{lemma}

The name ``$\GH$'' stands for ``good high (density)''. Remark that the property (c) and the fact that $R\not\in\Trc$ yield
\begin{equation}\label{eqiter582}
\sum_{Q\in\GH(R)} \sigma(\HD_1(e(Q))) \gtrsim B^{3/4}\,\sigma(\HD_1(e(R))),
\end{equation}
which is suitable for iteration.

\begin{proof}[Proof of Lemma \ref{lemalg1}]
Let $R\in\MDW$ be such that $\TT(e'(R))$ is not tractable. Notice first that
$$\sigma(\HD_1(e'(R)))\overset{\eqref{eq:sigmae'lesigmae}}{\leq} B^{1/4}
\sigma(\HD_1(e(R)))\leq B^{-3/4}\sigma(\HD_2(e'(R))) .$$
Let $I\subset \HD_1(e'(R))$ be the subfamily of the cubes $Q$ such that
$$\sigma(\HD_1(Q))< B^{1/2}\sigma(Q).$$
Then we have 
\begin{align*}
\sum_{Q\in I} \sigma(\HD_2(e'(R))\cap\DD_\mu(Q)) & \leq B^{1/2} \sum_{Q\in I} \sigma(Q)
\leq B^{1/2} \sigma(\HD_1(e'(R))) \\
&\leq \frac{B^{1/2}}{B^{3/4}}\,\sigma(\HD_2(e'(R)))\leq
 \frac12\,\sigma(\HD_2(e'(R))).
\end{align*}
Therefore,
\begin{align}\label{eqdj723}
\sum_{Q\in \HD_1(e'(R))\setminus I} \sigma(\HD_2(e'(R))\cap\DD_\mu(Q)) & = \sigma(\HD_2(e'(R))) -
\sum_{Q\in I} \sigma(\HD_2(e'(R))\cap\DD_\mu(Q))\\
& \geq \frac12\,\sigma(\HD_2(e'(R))).\nonumber
\end{align}

Next we will choose a family $J\subset \HD_1(e'(R))\setminus I$ satisfying
\begin{itemize}
\item[(i)] The balls $B(e''(Q))$, with $Q\in J$, are pairwise disjoint.
\item[(ii)] $$B^{1/4}\sum_{Q\in J}\sigma(\HD_1(e(Q)))
\gtrsim \sum_{Q\in \HD_1(e'(R))\setminus I} \sigma(\HD_2(e'(R))\cap\DD_\mu(Q)).$$
\end{itemize}
Then, choosing $\GH(R) = J$ we will be done. Indeed, the  property (a) in the statement of the lemma is the same as (i), and the property (b) is a consequence of the fact that
$J\subset I^c$ and the definition of $I$. This also implies that $\GH(R)\subset\MDW$. Finally, the property (c) follows from \rf{eqdj723} and (ii).


Let us see how $J$ can be constructed. By the covering Theorem 9.31 from \cite{Tolsa-llibre}, there
is a family $J_0\subset \HD_1(e'(R))\setminus I$ such that
\begin{itemize}
\item[1)] The balls $B(e''(Q))$, with $Q\in J_0$, have finite superposition, that is, 
$$\sum_{Q\in J_0}\chi_{B(e''(Q))}\leq C,$$
and
\item[2)] 
$$\bigcup_{Q\in \HD_1(e'(R))\setminus I} B(e''(Q)) \subset \bigcup_{Q\in J_0} (1+8A_0^{-1})\,B(e''(Q)),$$
\end{itemize}
Actually, in Theorem 9.31 from \cite{Tolsa-llibre} the result above is stated for a finite family of
balls. However, it is easy to check that the same arguments work as soon as the family $\HD_1(e'(R))\setminus I$ is countable and can be ordered so that $\HD_1(e'(R))\setminus I=\{Q_1,Q_2,\ldots\}$,
with $\ell(Q_1)\geq \ell(Q_2)\geq\ldots$. Further, one can check that the constant $C$ in 1)
does not exceed some absolute constant times $A_0^{n+1}$.

From the finite superposition property 1), by rather standard arguments which are analogous to the
ones in the proof of Besicovitch's covering theorem in \cite[Theorem 2.7]{Mattila-llibre}, say, 
one deduces that $J_0$ can be split into $m_0$ subfamilies $J_1,\ldots, J_{m_0}$ so that, for each $k$,  the balls $\{B(e''(Q)): Q\in J_k\}$  are pairwise disjoint, with $m_0\leq C(A_0)$.

Notice that the condition 2) and Lemma \ref{lem-calcf} applied to $Q$ ensure that
\begin{equation}\label{equni98-1}
\bigcup_{Q\in \HD_1(e'(R))\setminus I} Q\subset \bigcup_{Q\in \HD_1(e'(R))\setminus I} B(e''(Q) )\subset \bigcup_{Q\in J_0} (1+8A_0^{-1})\,B(e''(Q)) \subset \bigcup_{Q\in J_0} B(e^{(8)}(Q)).
\end{equation}
Next we choose $J:=J_k$ to be the family such that
$$\sum_{Q\in J_k}\sigma(\HD_1(e(Q)))$$
is maximal among $J_1,\ldots,J_{m_0}$, so that
\begin{align*}
\sum_{Q\in J}\sigma(\HD_1(e(Q))) & \geq \frac1{m_0}\,
\sum_{Q\in J_0}\sigma(\HD_1(e(Q)))\\
& \overset{\eqref{eq:sigmae'lesigmae}}{\geq} \frac{1}{m_0\,B^{1/4}} \sum_{Q\in J_0}\sigma(\HD_1(e^{(8)}(Q)))\\
& \overset{\rf{equni98-1}}{\ge} \frac{1}{m_0\,B^{1/4}} \sum_{Q\in \HD_1(e'(R))\setminus I} \sigma(\HD_1(Q))\\
& = \frac{1}{m_0\,B^{1/4}}\sum_{Q\in \HD_1(e'(R))\setminus I} \sigma(\HD_2(e'(R))\cap\DD_\mu(Q)).
\end{align*}
This proves (ii).
\end{proof}
\vv

Given $R\in\ttt\cap \MDW$, we will construct now a subfamily of cubes from $\MDW$ generated by $R$,
which we will denote $\Gen(R)$, by iterating the construction of Lemma \ref{lemalg1}.
The algorithm goes as follows.
Given $R\in\ttt\cap \MDW$, we denote 
$$\Gen_0(R) = \{R\}.$$
If $R\in\Trc$, we set $\Gen_1(R)=\varnothing$, and otherwise
$$\Gen_1(R) = \GH(R),$$
where $\GH(R)$ is defined in Lemma \ref{lemalg1}.
For $j\geq 2$, we set
$$\Gen_{j}(R) = \bigcup_{Q\in\Gen_{j-1}(R)\setminus \Trc} \GH(Q).$$
For $j\geq0$, we also set
$$\Trc_j(R) = \Gen_j(R)\cap\Trc,$$
and
$$\Gen(R) = \bigcup_{j\geq0}\Gen_j(R),\qquad\Trc(R) = \bigcup_{j\geq0}\Trc_j(R).$$
\vv


\begin{lemma}\label{eqtec74}
For $R\in\ttt\cap \MDW$, we have
\begin{equation}\label{eqtec741}
\bigcup_{Q\in\Trc(R)}Q\subset\bigcup_{Q\in\Gen(R)}Q \subset B(e''(R)).
\end{equation}
Also,
\begin{equation}\label{eqiter*44}
\sigma(\HD_1(e(R)))\leq \sum_{j\geq0} B^{-j/2}\sum_{Q\in\Trc_j(R)}\sigma(\HD_1(e(Q))).
\end{equation}
\end{lemma}



\begin{proof}
%Obviously, in the case $R\in\Trc$, \rf{eqiter*44} holds.

The first inclusion in \eqref{eqtec741} holds because $\Trc(R)\subset\Gen(R)$. So we only have to show the second inclusion.

By construction, for any $R'\in\MDW$, $\GH(R')\subset \HD_1(e'(R'))$, and thus any $Q\in \GH(R')$ is contained in $e'(R')$. This implies that 
$$|x_{R'}-x_Q|\leq r(B(e'(R'))) + \frac12\,\ell(Q)\leq  \Big(1+ 2A_0^{-1}+ \frac12\,A_0^{-1}\Big)\ell(R') \leq 1.1\,\ell(R').$$
Then, given $Q\in\Gen_j(R)$, $x\in Q$, and $0\leq k\leq j$, if we denote by $R_k$ the cube from $\Gen_k(R)$ such that $Q\in\Gen_{j-k}(R_k)$, we have
\begin{align*}
|x_R-x|& \leq |x_R- x_{R_1}|+ \sum_{k=1}^{j-1} |x_{R_k}- x_{R_{k+1}}| + |x_Q-x|\\
& \leq r(B(e'(R)))+ \frac12\,A_0^{-1}\,\ell(R) + \sum_{k=1}^{j-1}1.1\,A_0^{-k}\ell(R) + \frac12\,A_0^{-1}\,\ell(R)\\
& \leq r(B(e'(R)))+ 2\,A_0^{-1}\,\ell(R),
\end{align*}
which shows that $Q\subset B(e''(R))$.

To prove the second statement in the lemma, observe that, for $Q\in\Gen_{j-1}(R)\setminus \Trc$,
by \rf{eqiter582} applied to $Q$ we have
$$\sum_{P\in\GH(Q)} \sigma(\HD_1(e(P))) \geq c\,B^{3/4}\,\sigma(\HD_1(e(Q)))\geq 
B^{1/2}\,\sigma(\HD_1(e(Q))),
$$
assuming $\Lambda_*$, and thus $B$, big enough. Therefore,
\begin{align*}
\sum_{P\in\Gen_j(R)}\sigma(\HD_1(e(P))
& = 
\sum_{Q\in\Gen_{j-1}(R)\setminus \Trc} \,\sum_{P\in\GH(Q)}\sigma(\HD_1(e(P)))\\
& \geq 
B^{1/2}\sum_{Q\in\Gen_{j-1}(R)\setminus \Trc}\sigma(\HD_1(e(Q)))
\end{align*}
So,
$$\sum_{Q\in\Gen_{j-1}(R)}\sigma(\HD_1(e(Q)))\leq 
\sum_{Q\in\Trc_{j-1}(R)}\sigma(\HD_1(e(Q))) +
B^{-1/2}\sum_{P\in\Gen_j(R)}\sigma(\HD_1(e(P))).
$$
Iterating this estimate, and taking into account that, by the polynomial growth of $\mu$,
$\Gen_{j-1}(R)=\varnothing$ for some large $j$, we get \rf{eqiter*44}.
\end{proof}

\vv


