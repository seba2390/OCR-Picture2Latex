% ********************************************************************************************

\section{The Riesz transform on the tractable trees: the approximating measures \texorpdfstring{$\eta$}{eta}, \texorpdfstring{$\nu$}{nu}, and the variational argument}\label{sec6}

In this section,
for a given $R\in\MDW$ such that $\TT(e'(R))$ is tractable (i.e., $R\in\Trc$),
we will define a suitable measure $\eta$ that approximates $\mu$ at the level of the cubes from
$\TT(e'(R))$ and we will estimate $\|\RR\eta\|_{L^p(\eta)}$ from below. To this end,
we will apply a variational argument in $L^p$ by techniques inspired by 
\cite{Reguera-Tolsa} and \cite{JNRT}.  In the next section we will transfer these
estimates to $\RR\mu$.


%  *************************************************************************************************


\subsection{The suppressed Riesz transform and a Cotlar type inequality}\label{sec6.1}

%Millor posar aixo cap al prinicipi.

Let $\Phi:\R^{n+1}\to[0,\infty)$ be a $1$-Lipschitz function.
Below we will need to work with the suppressed Riesz kernel
\begin{equation}\label{eqsuppressed}
K_\Phi(x,y) = \frac{x-y}{\bigl (|x-y|^2+\Phi(x)\Phi(y)\bigr)^{(n+1)/2}}
\end{equation}
and the associated operator 
$$\RR_\Phi\alpha(x) =\int K_\Phi(x,y)\,d\alpha(y),$$
where $\alpha$ is a signed measure in $\R^{n+1}$.
For a positive measure $\omega$ and $f\in L^1_{loc}(\omega)$, we write $\RR_{\Phi,\omega} f = \RR_\Phi (f\,\omega)$.
The kernel $K_\Phi$ (or a variant of this) appeared for the first
time in the work of Nazarov, Treil and Volberg in connection with Vitushkin's conjecture (see
\cite{Volberg}). 
This is a Calder\'on-Zygmund kernel which satisfies the properties:
\begin{equation}\label{eqkafi1}
|K_\Phi(x,y)|\lesssim \frac1{\big(|x-y| + \Phi(x) + \Phi(y)\big)^n}
\end{equation}
and
\begin{equation}\label{eqkafi2}
|\nabla_x K_\Phi(x,y)|+ |\nabla_y K_\Phi(x,y)|
\lesssim \frac1{\big(|x-y| + \Phi(x) + \Phi(y)\big)^{n+1}}
\end{equation}
for all $x,y\in\R^{n+1}$.

Also, if $\ve\approx\Phi(x)$, then we have
\begin{equation}
\label{e.compsup''}
\bigl|\RR_{\ve}\alpha(x) - \RR_{\Phi}\alpha(x)\bigr|\lesssim  \sup_{r> \Phi(x)}\frac{|\alpha|(B(x,r))}{r^n},
\end{equation}
with the implicit constant in the inequality depending on the implicit constant in the comparability $\ve\approx\Phi(x)$.
See Lemmas 5.4 and 5.5 in \cite{Tolsa-llibre}. 

The following result is an easy consequence of a $Tb$ theorem of Nazarov, Treil and Volberg.
See Chapter 5 of \cite{Tolsa-llibre}, for example. We will use this to prove \rf{eqacpsi}.

\begin{theorem}\label{teontv}
Let $\omega$ be a Radon measure in $\R^{n+1}$ and let $\Phi:\R^{n+1}\to[0,\infty)$ be a $1$-Lipschitz function. Suppose that
\begin{itemize}
\item[(a)] $\omega(B(x,r))\leq c_0\,r^n$ for all $r\geq \Phi(x)$, and
\item[(b)] $\sup_{\ve>\Phi(x)}|\RR_\ve\omega(x)|\leq c_1$.
\end{itemize}
Then $\RR_{\Phi,\omega}$ is bounded in $L^p(\omega)$, for $1<p<\infty$, with a bound on its norm depending only on $p$, $c_0$ and
$c_1$. In particular, $\RR_\omega$ is bounded in $L^p(\omega)$ on the set $\{x:\Phi(x)=0\}$.
\end{theorem}
\vv


%\subsection{A Cotlar type inequality}

We define the energy $W_\omega$ (with respect to $\omega$) of a set $F\subset \R^{n+1}$ as
$$W_\omega(F) = \iint_{F\times F} \frac1{\diam(F)\,|x-y|^{n-1}}\,d\omega(x)d\omega(y).$$
%For $\omega=\mu$ and a ball $B$, $W_\mu(B)$ can be estimated in terms of $\EE(B)$. 
%This fact will used in Section \ref{sec99} but not in the current section. See Lemma \ref{lemw-e} for more details.
We say that a ball $B\subset \R^{n+1}$ is $(a,b)$-doubling $$\omega(aB)\leq b\,\omega(B).$$
We denote by $\cM_\omega f$ the usual centered maximal Hardy-Littlewood operator applied to $f$:
$$\cM_\omega f(x) = \sup_{r>0}\,\frac1{\omega(B(x,r))}\int_{B(x,r)}|f|\,d\omega,$$
and by $\cM_\omega^{(r_0,a,b)} f$ the version
$$\cM_\omega^{(r_0,a,b)} f(x) = \sup\frac1{\omega(B(x,r))}\int_{B(x,r)}|f|\,d\omega,$$
where the $\sup$ is taken over all radii $r>r_0$ such that the ball $B(x,r)$ is $(a,b)$-doubling.

\vv

\begin{lemma}\label{lemcotlar1}
Let $x\in R$, $r_0>0$, and $\theta_1>0$.
Suppose that for all $r\geq r_0$ 
$$\theta_\omega(x,r)\leq \theta_1$$
and 
$$W_\omega(B(x,r))\leq \theta_1\,\omega(B(x,r))\qquad \mbox{if $B(x,r)$ is $(16,128^{n+2})$-doubling.}$$
Then
\begin{equation}\label{eqcot99}
\sup_{\ve\geq r_0} |\RR_\ve\omega(x)|\lesssim \cM_\omega^{(r_0,16,128^{n+2})}(\RR\omega)(x) + \theta_1.
\end{equation}
\end{lemma}

\begin{proof}
For a given $\ve\geq r_0$, let $k\geq0$ be minimal so that, for $r=128^k\,\ve$, the ball $B(x,r)$ is $(128,128^{n+2})$-doubling (in particular, this implies that $B(x,8r)$ is $(16,128^{n+2})$-doubling). It is easy to see that such $k$ exists using the assumption $\theta_\omega(x,r)\leq \theta_1$.
By a standard estimate (see Lemma 2.20 from \cite{Tolsa-llibre}), it follows that
$$|\RR_\ve \omega(x)|\leq |\RR_{8r} \omega(x)| + C\,\frac{\omega(B(x,8r))}{(8r)^n} \leq |\RR_{8r} \omega(x)| + C\,\theta_1.$$
It is immediate to check that for any $x'\in B(x,2r)$,
\begin{equation}\label{eqss1}
|\RR_{8r}\omega(x) - \RR\chi_{B(x,4r)^c}\omega(x')|\lesssim \theta_1.
\end{equation}
Consider radial $C^1$ functions $\psi_1$ and $\psi_2$ such that
$$\chi_{B(x,4r)}\leq \psi_1\leq \chi_{B(x,8r)}\qquad \text{and} \qquad
\chi_{B(x,r)}\leq \psi_2\leq \chi_{B(x,2r)},$$
and $\text{Lip}(\psi_1)\le r,\ \text{Lip}(\psi_2)\le r.$ Given a function $f\in L^1_{loc}(\omega)$, denote by $m_{\psi_2\omega}f$ the $(\psi_2\,\omega)$-mean of $f$, i.e.,
$$m_{\psi_2\omega}f = \frac1{\|\psi_2\|_{L^1(\omega)}} \int f\,\psi_2\,d\omega.$$
Notice that
$$\bigl|m_{\psi_2\omega}(\RR\omega)\bigr|\leq \frac1{\omega(B(x,r))} \int_{B(x,2r)}
|\RR\omega|\,d\omega \approx\avint_{B(x,2r)}
|\RR\omega|\,d\omega 
\leq \cM_\omega^{(r_0,16,128^{n+2})}(\RR\omega)(x).$$
From \rf{eqss1} we deduce that
$$\big|\RR_{8r}\omega(x) - m_{\psi_2\omega}\bigl(\RR(\chi_{B(x,4r)^c}\omega)\bigr)\big| \leq m_{\psi_2\omega}\bigl(|\RR_{8r}\omega(x) - \RR(\chi_{B(x,4r)^c}\omega)|\bigr)
\lesssim\theta_1.$$
Then we have
\begin{align*}
|\RR_{8r}\omega(x)|&\lesssim\theta_1 + |m_{\psi_2\omega}(\RR(\chi_{B(x,4r)^c}\omega)|\\
& \lesssim\theta_1 + \bigl|m_{\psi_2\omega}\bigl(\RR(\chi_{B(x,4r)^c}\omega) - \RR((1-\psi_1)\omega)\bigr)\bigr| +\bigl|m_{\psi_2\omega}\bigl(\RR(\psi_1\omega)\bigr)\bigr|+ \bigl|m_{\psi_2\omega}(\RR\omega)\bigr|\\
& \lesssim\theta_1 +  \bigl|m_{\psi_2\omega}\bigl(\RR(\psi_1\omega)\bigr)\bigr|
+ \cM_\omega^{(r_0,16,128^{n+2})} (\RR\omega)(x).
\end{align*}
To estimate the middle term on the right hand side we use the antisymmetry of $\RR$:
\begin{align*}
\bigl|m_{\psi_2}\bigl(\RR(\psi_1\omega)\bigr)\bigr|
& =\frac1{\|\psi_2\|_{L^1(\omega)}} \left|\iint K(y-z)\,\psi_1(y)\,\psi_2(z)\,d\omega(y)\,d\omega(z)\right|\\
& =\frac1{2\|\psi_2\|_{L^1(\omega)}} \left|\iint K(y-z)\,\bigl(\psi_1(y)\,\psi_2(z) - \psi_1(z)\,\psi_2(y)\bigr)\,d\omega(y)\,d\omega(z)\right|\\
& \lesssim \frac1{\omega(B(x,r))} \iint_{B(x,8r)\times B(x,8r)} \!\frac1{r\,|y-z|^{n-1}}\,d\omega(y)\,d\omega(z)
 \approx\frac{W_\omega(B(x,8r))}{\omega(B(x,8r))} \lesssim \theta_1.
\end{align*}
\end{proof}

\vv
%\begin{rem}
%For every $a\geq2$ the lemma is also valid replacing $(128,1128^{n+2})$-doubling balls by $(a,b)$-doubling balls, with $b\geq(8a)^{n+2}$, say, and
%5$\cM_\omega^{(r_0,2,16^{d+1})}$ by $\cM_\omega^{(r_0,a,b)}$.
%\end{rem}


\vv
\begin{rem}\label{rem**}
In fact, the proof of the preceding lemma shows that, given any measure $\omega$, $x\in\R^{n+1}$ and
$\ve>0$,
\begin{equation}\label{eqcotlar*99}
|\RR_\ve\omega(x)|\lesssim \avint_{B(x,2\ve')} |\RR\omega|\,d\omega + \sup_{r\geq \ve} \frac{\omega(B(x,r))}{r^n} + \frac{W_\omega(B(x,8\ve'))}{\omega(B(x,8\ve'))},
\end{equation}
where $\ve'=2^{7k}\ve$, with $k\geq0$ minimal such that the ball $B(x,\ve')$ is $(128,128^{n+2})$-doubling. 
%Of course, a similar estimate holds if we replace the $(16,16^{n+2})$-doubling condition
%by $(16,C)$-doubling with any constant $C>1$, and then the implicit constant in \rf{eqcotlar*99}
%depends on $C$.
%Further, a quick inspection of the proof shows that in \rf{eqcotlar*99} one can replace the term 
%$\ds \sup_{r\geq \ve} \frac{\omega(B(x,r))}{r^n}$ by\footnote{Perhaps this requires more details.}
%$$P_\omega(x,\ve') := \int \frac{\ve'}{\big(|x-y|+\ve'\big)^{n+1}}\,d\omega(y).$$
\end{rem}


\vv

% ********************************************************************************************

%\subsection{The families $\End(e'(R))$ and $\Reg(e'(R))$ and the approximating measure $\eta$}
\subsection{The family \texorpdfstring{$\Reg(e'(R))$}{Reg(e'(R))} and the approximating measure \texorpdfstring{$\eta$}{eta}}\label{sec6.2*}

In the remaining of this section we fix a cube $R\in\MDW$ such that $\TT(e'(R))$ is tractable.
%Recall that the family $\TT(e'(R))$ was constructed by stopping time conditions involving the families $\LD(\,\cdot\,)$ and $\HD_*(\,\cdot\,)$.
% , and that $\sss(e'(R))$ is the family of maximal $\PP$-doubling cubes which contained both in $e'(R)$ and in some cube from $\LD(R)\cup\HD_*(R)\cup\NDB$.

We need to define some regularized family of ending cubes for $\TT(e'(R))$. 
First, let 
$$d_R(x) = \inf_{Q\in\TT(e'(R))}\big(\dist(x,Q) + \ell(Q)\big).$$
Notice that $d_R$ is a $1$-Lipschitz function. 
%We denote by $Z(e'(R))$ the subset of the points $x\in e'(R)$ such that $d_R(x)=0$.
Given $0<\ell_0\ll\ell(R)$, we denote
\begin{equation}\label{eql00*23}
d_{R,\ell_0}(x) = \max\big(\ell_0,d_R(x)\big),
\end{equation}
which is also $1$-Lipschitz.
For each $x\in e'(R)$ we take the largest cube $Q_x\in\DD_\mu$ 
such that $x\in Q_x$ with
\begin{equation}\label{eqdefqx}
\ell(Q_x) \leq \frac1{60}\,\inf_{y\in Q_x} d_{R,\ell_0}(y).
\end{equation}
We consider the collection of the different cubes $Q_x$, $x\in e'(R)$, and we denote it by $\Reg(e'(R))$ (this stands for ``regularized cubes''). 
%Also, we let $\qgood$ (this stands for ``quite good'') be
%the family of cubes $Q\in \DD$ such that $Q$ is contained in $B(x_0,2 K r_0)$ and $Q$ is not strictly contained in any
%cube of the family $\Reg$. 
%Observe that $\reg\subset \qgood$.

The constant $\ell_0$ is just an auxiliary parameter that prevents $\ell(Q_x)$ from vanishing.  Eventually $\ell_0$ will be taken extremely small. In particular, we assume $\ell_0$ small enough
so that 
\begin{equation}\label{eql00}
\mu\bigg(\bigcup_{Q\in\HD_1(e(R)):\ell(Q)\geq \ell_0} Q \bigg)\geq \frac12\,\mu\bigg(\bigcup_{Q\in\HD_1(e(R))} Q\bigg).
\end{equation}

We let $\TT_\Reg(e'(R))$ be the family of cubes made up of $R$ and all the cubes of the next generations which are contained in $e'(R)$ but are not 
strictly contained in any cube from $\Reg(e'(R))$.



\vv

\begin{lemma}\label{lem74}
The cubes from $\Reg(e'(R))$ are pairwise disjoint and satisfy the following properties:
\begin{itemize}
\item[(a)] If $P\in\Reg(e'(R))$ and $x\in B(x_{P},50\ell(P))$, then $10\,\ell(P)\leq d_{R,\ell_0}(x) \leq c\,\ell(P)$,
where $c$ is some constant depending only on $n$. 
%In particular, $B(z_{P},50\ell(P))\cap W_0=\varnothing$.

\item[(b)] There exists some absolute constant $c>0$ such that if $P,\,P'\in\Reg(e'(R))$ satisfy $B(x_{P},50\ell(P))\cap B(x_{P'},50\ell(P'))
\neq\varnothing$, then
$$c^{-1}\ell(P)\leq \ell(P')\leq c\,\ell(P).$$
\item[(c)] For each $P\in \Reg(e'(R))$, there are at most $C_3$ cubes $P'\in\Reg(e'(R))$ such that
$$B(x_{P},50\ell(P))\cap B(x_{P'},50\ell(P'))
\neq\varnothing,$$
 where $C_3$ is some absolute constant.
 
 %\item[(d)] If $x\not\in B(x_0,\frac1{8} K r_0)$, then $d(x)\approx |x-x_0|$. Thus, if $P\in\reg$ and $B(z_{P},50\ell(P))\not\subset  B(x_0,\frac1{8} K r_0)$, then $\ell(P)\gtrsim  K r_0$.
\end{itemize}
\end{lemma}

The proof of this lemma is standard. See for example \cite[Lemma 6.6]{Tolsa-memo}.

\vvv

Next we define a measure $\eta$ which, in a sense, approximates $\mu\rest_{e'(R)}$ at the level of the family $\Reg(e'(R))$. 
We let
$$\eta = \sum_{P\in\Reg(e'(R))} \mu(P) \frac{\LL^{n+1}\rest_{\tfrac12B(P)}}{\LL^{n+1}(\tfrac12B(P))},
$$
where $\LL^{n+1}$ stands for the Lebesgue measure in $\R^{n+1}$.
With each $Q\in\TT_{\Reg}(e'(R))$ we associate another ``cube'' $Q^{(\eta)}$ defined as follows:
$$Q^{(\eta)}= \bigcup_{P\in\Reg(e'(R)):P\subset Q} \tfrac12B(P).$$
Further, we consider a lattice $\DD_\eta$ associated with the measure $\eta$ which is made up of the cubes
$Q^{(\eta)}$ with $Q\in \TT_{\Reg}(e'(R))$ and other cubes which are descendants of the cubes from $\Reg(e'(R))$.
We assume that $\DD_\eta$ satisfies the first two properties of Lemma \ref{lemcubs} with the same parameters $A_0$ and $C_0$ as $\DD_\mu$. It is straightforward to check that $\DD_\eta$ can be constructed in this way.
For $S\in\DD_\eta$ such that $S=Q^{(\eta)}$ with $Q\in\TT_{\Reg}(e'(R))$, we let $Q=S^{(\mu)}$. Further, we write
$\ell(S):=\ell(Q)$, $B_S:=B_Q$, and $\Theta(S):=\Theta(Q)$.


\vv

%We let\footnote{Perhaps it is appropriate to define the smaller cubes contained in cubes from $\Reg$.}
%$$\DD_\eta = \{s(Q)\}_{Q\in\DD_\mu, Q\subset e'(R)}.$$







% ********************************************************************************************

\subsection{The auxiliary family \texorpdfstring{$\sH$}{H}}\label{subsec:H}

Given $p\geq1$ and a family $I\subset\DD_\mu$, we denote
$$\sigma_p(I) = \sum_{P\in I}\Theta(P)^p\,\mu(P),$$
so that $\sigma(I)=\sigma_2(I)$. 
Recall that  
$$\HD_*(R) = \hd^{k_{\Lambda_*}}(R),$$
with $k_{\Lambda_*}=k_\Lambda(1-\frac1N)$,
and that
$$\HD_1(e(R)) = \sss_*(e(R))\cap \HD_*(R),\qquad  \HD_1(e'(R)) = \sss_*(e'(R))\cap \HD_*(R).
$$
For $R\in\MDW$ and $j\geq0$, denote 
$$\sH_j(e'(R))= \TT_\Reg(e'(R)) \cap \hd^{k_{\Lambda_*} + j}(R),$$
so that $\sH_0(e'(R))=\HD_1(e'(R))\cap\TT_\Reg(e'(R))$. 
\brem\label{rem:Hjempty}
Note that for $j> k_{\Lambda_*}+2$ we have $\sH_j(e'(R))=\varnothing$. Indeed, by the definition of $\Reg(e'(R))$ and $\TT_\Reg(e'(R))$, for each $Q\in \TT_\Reg(e'(R))$ there exists $P\in \TT(e'(R))$ such that $2B_Q\subset 2B_P$ and $\ell(Q)\le \ell(P)\le A_0^2\ell(Q)$. Thus,
\begin{equation*}
\frac{\mu(2B_Q)}{\ell(Q)^n}\le \frac{\mu(2B_P)}{\ell(Q)^n}\le A_0^{2n}\frac{\mu(2B_P)}{\ell(P)^n}.
\end{equation*}
Since for each $P\in \TT(e'(R))$ we have $\Theta(P)\le \Lambda_*^2\Theta(R)$, it follows that $\Theta(Q)\le A_0^{2n}\Lambda_*^2\Theta(R)$.
\erem

%In the case $R\in\Red$,  we set 
%$$\sH_j(e'(R))= \TT_0(e'(R)) \cap \hd^{k_\Lambda-k_1 + j}(R).$$
%Notice that it would have been more appropriate to use a notation different from $\sH_j(e'(R))$, such as $\sH_{0,j}(e'(R))$. The reason for our abuse of notation is the wish simplify the exposition below.

The fact that $\max_{j\geq0} \sigma_p(\sH_j(e'(R)))$ may be much larger than $\sigma_p(\HD_1(e'(R))$ may cause problems in some estimates. For this reason,
we need to introduce an auxiliary family $\sH$. We deal with this issue in this section.

Recall that, for $R\in\DD_{\mu,k}$,
$$e(R) = e_{h(R)}(R) \quad \mbox{ and }\quad e'(R) = e_{h(R)+1},$$
where
$$e_i(R) = R \cup \bigcup Q,$$
with the union running over the cubes $Q\in\DD_{\mu,k+1}$ such that
$$
\dist(x_R,Q)< \frac{\ell(R)}2 + 2i\ell(Q).
$$
For $j\geq0$, we set 
$$e_{i,j}(R) = \bigcup_{Q\in\DD_{\mu,k+1}:Q\subset e_i(R)} e_j(Q),$$
and we let $\sH_k(e_{i,j}(R))$ be the subfamily of the cubes from $\sH_k(e'(R))$ which are contained
in $e_{i,j}(R)$.


From now on, we let $\ve_n$ be some positive constant depending just on $n$. In the present paper, later on, we will simply
take $\ve_n=1/15$. However, for another application of the results from this section in \cite{Tolsa-riesz} it
is convenient to allow $\ve_n$ to depend on $n$.

\vv
\begin{lemma}\label{lem:66}
Let $p\in (1,2]$.
For any $R\in\MDW$ there exist some $j,k$, with $10\leq j\leq A_0/4$ and  $0\leq k\leq k_{\Lambda_*}+2$ such that
\begin{equation}\label{eqsigmah}
\sigma_p(\sH_{m}(e_{h(R),j+1}(R))) \leq \Lambda_*^{\ve_n} \,\sigma_p(\sH_{k}(e_{h(R),j}(R)))
\quad \mbox{for all $m\geq0$,}
\end{equation}
 assuming $A_0$ big enough (possibly depending on $n$). 
\end{lemma}

\begin{proof}
For each $j$, we denote by $0\le k_j\le k_{\Lambda_*}+2$ the integer such that
$$\sigma_p(\sH_{k_j}(e_{h(R),j}(R))) = \max_{0\leq k\leq k_{\Lambda_*}} \sigma_p(\sH_{k}(e_{h(R),j}(R))).$$
The lemma can be rephrased in the following way: there exists $10\leq j\leq A_0/4$ such that
\begin{equation*}
\sigma_p(\sH_{k_{j+1}}(e_{h(R),j+1}(R))) \leq \Lambda_*^{\ve_n} \,\sigma_p(\sH_{k_j}(e_{h(R),j}(R))).
\end{equation*}
We prove this by contradiction. Suppose the estimate above fails for all $10\leq j\leq A_0/4$, and let $j_0$ be the largest integer smaller than $A_0/4$.
Then we have
\begin{multline}\label{eq:651}
\sigma_p(\sH_{k_{j_0}}(e_{h(R),j_0}(R)))
  \geq \Lambda_*^{\ve_n}\,\sigma_p(\sH_{k_{j_0-1}}(e_{h(R),j_0-1}(R)))\\
\geq
\ldots\geq \Lambda_*^{\ve_n(j_0-10)}\sigma_p(\sH_{k_{10}}(e_{h(R),10}(R))) \geq \Lambda_*^{\ve_n(j_0-10)}\sigma_p(\sH_{0}(e_{h(R),10}(R)))\\
\geq  \Lambda_*^{\ve_n(j_0-10)}\sigma_p(\sH_{0}(e_{h(R)}(R))) \overset{\rf{eql00}}{\geq} \frac12\,\Lambda_*^{\ve_n(j_0-10)}
\sigma_p(\HD_1(e(R))),
\end{multline}

Concerning the left hand side of the inequality above, since $e_{h(R),j_0}(R)\subset 2R$ and $k_{j_0}\le k_{\Lambda_*}+2$, we have
$$
\sigma_p(\sH_{k_{j_0}}(e_{h(R),j_0}(R)))\leq A_0^{2np}
\Lambda_*^{2p}\,\Theta(R)^p\mu(2R).$$
Due to the fact that $R$ is $\PP$-doubling, as in \rf{eqdoub*11} we have
$\mu(2R)\leq C_0\,C_d\,A_0^{n+1}
\mu(R).$ Thus,
\begin{equation}\label{eq:652}
\sigma_p(\sH_{k_{j_0}}(e_{h(R),j_0}(R)))\leq C_0\,C_d\,A_0^{2np+n+1}
\Lambda_*^{2p}\sigma_p(R).
\end{equation}

Concerning the right hand side of \eqref{eq:651}, observe that, denoting $\Theta(\HD_1)=\Lambda_*\Theta(R)=\Theta(Q)$ for any $Q\in\HD_1(e(R))$, we have
$$\sigma_p(\HD_1(e(R))) = \Theta(\HD_1)^{p-2}\sigma(\HD_1(e(R)))\overset{\eqref{eq:MDWdef2}}{\ge} B^{-1}\Theta(\HD_1)^{p-2}\sigma(R) \geq \Lambda_*^{-1}\Lambda_*^{p-2}\sigma_p(R).
$$
We deduce from \eqref{eq:651}, \eqref{eq:652}, and the above, that
$$\frac12\,\Lambda_*^{\ve_n(j_0-10)}\Lambda_*^{p-3}
\sigma_p(R)\leq C_0\,C_d\,A_0^{2np+n+1}
\Lambda_*^{2p}\sigma_p(R).$$
Since $\Lambda_*\geq A_0^n$ and $j_0\approx A_0$, it is clear that this inequality is violated if $A_0$ is big 
enough, depending just on $n$.
\end{proof}
\vv


% ********************************************************************************************

\subsection{Some technical lemmas}\label{subsec9.5}


Let $j(R),\,k(R)$ be such that $10\leq j(R)\leq A_0/4$,  $0\leq k(R)\leq k_{\Lambda_*}+2$ and
$$
\sigma_p(\sH_{m}(e_{h(R),j(R)+1}(R))) \leq \Lambda_*^{\ve_n} \,\sigma_p(\sH_{k(R)}(e_{h(R),j(R)}(R)))
\quad \mbox{for all $m\geq0$,}
$$
We denote
$$\cS_\mu = \bigcup_{Q\in\Reg:Q\subset e_{h(R),j(R)}(R)} Q,\qquad
\cS_\eta = \bigcup_{Q\in\Reg:Q\subset e_{h(R),j(R)}(R)} \tfrac12B(Q)$$
and
$$\cS_\mu' = \bigcup_{Q\in\Reg:Q\subset e_{h(R),j(R)+1}(R)} Q,\qquad
\cS_\eta' = \bigcup_{Q\in\Reg:Q\subset e_{h(R),j(R)+1}(R)} \tfrac12B(Q).$$
Notice that, by construction,
\begin{equation}\label{eqtec732}
\dist(\supp\mu \setminus e'(R),\cS_\mu')\geq cA_0^{-1} \ell(R),
\end{equation}
where $c>0$ is an absolute constant.

For $m=1,2,3,4$, denote by $V_m$ the $m A_0^{-3}\ell(R)$-neighborhood of $\cS_\eta$.
Let $\vphi_R$ be a $C^1$ function which equals $1$ in $V_3$, vanishes out of $V_4$, and such that $\|\vphi_R\|_\infty\leq 1$ and
$\|\nabla\vphi_R\|_\infty\leq 2A_0^3 \ell(R)^{-1}$.
Observe that, for $x\in \supp\mu\setminus e'(R)$, $\dist(x,V_4)\gtrsim \ell(R)$.
In fact, from \rf{eqtec732} one can derive that
\begin{equation}\label{eqtec733}
\dist(Q,\supp\mu\setminus e'(R)) \gtrsim \ell(R)\quad\mbox{ for all $Q\in\Reg(e'(R))$ such that
$B_Q\cap V_4\neq\varnothing$,}
\end{equation}
taking into account that $\ell(Q)\leq \frac{A_0^{-1}}{60}\,\ell(R)$ for every $Q\in\Reg(e'(R))$.

 We consider the measure
$$\nu = \vphi_R\,\eta$$
and the function
$$G(x) = 2A_0^3\int_{\cS_\eta'\setminus V_2} 
 \frac1{\ell(R)\,|x-y|^{n-1}}\,d\eta(y).$$
 Notice that
$$G(x)\lesssim \Theta(R)\quad\mbox{ for all $x\in V_1$.}$$ 


To shorten notation, we write
$$\sH= \sH_{k(R)}(e_{h(R),j(R)}(R)),\qquad \sH'= \sH_{k(R)}(e_{h(R),j(R)+1}(R))$$
and
$$H= \bigcup_{Q\in\sH} Q^{(\eta)},\qquad  H'= \bigcup_{Q\in\sH'}Q^{(\eta)}.$$
We also set
$$\Theta(\sH) = A_0^{(k_{\Lambda_*}+k(R))n}\,\Theta(R),$$
so that for any $Q\in\sH$ we have
$\Theta(Q) = \Theta(\sH).$


\vv
\begin{lemma}\label{lem*921}
Let $A\subset \R^{n+1}$ be the set of those $x\in\R^{n+1}$ which belong to some 
$(16,128^{n+2})$-doubling (with respect to $\nu$) ball $B\subset\R^{n+1}$ such that
$$W_\nu(B)\geq M\,\Theta(\sH)\,\nu(B)\quad \mbox{ and } \quad
\theta_\eta( \gamma B)\leq c_2\,\Theta(\sH) \quad\mbox{for all $\gamma\geq1$.}$$
Then, for $M\geq1$ big enough, 
$$\nu(A) \lesssim \frac{\Lambda_*^{\ve_n}}{M} \,\nu(H).$$
\end{lemma}

\begin{proof}
Observe first that, for any ball $B\subset\R^{n+1}$ with $r(B)\in[A_0^{-k-1},A_0^{-k}]$,
\begin{align*}
W_\nu(B) &\lesssim
%\theta_\nu(B)\,\nu(B) + \sum_{j\geq k+1} \iint_{\begin{subarray}\,(x,y)\in B\times B\\
%A_0^{-j-2}<|x-y|\leq A_0^{-j-1} \end{subarray}}
%\frac1{r(B)\,|x-y|^{n-1}}\, d\nu(x)\,d\nu(y)\\
%& \lesssim
\theta_\nu(B)\,\nu(B) \\
&\quad+ \sum_{j\geq k+1} \sum_{Q\in\DD_{\eta,j}:Q\subset 2B}\int_{x\in Q}\int_{y:
A_0^{-j-2}<|x-y|\leq A_0^{-j-1}}
\frac1{r(B)\,|x-y|^{n-1}}\, d\nu(x)\,d\nu(y)\\
& \lesssim \sum_{Q\in\DD_{\eta}:Q\subset 2B} \frac{\ell(Q)}{r(B)}\,\theta_\nu(2B_Q)\,\nu(Q).
\end{align*}

To prove the lemma, we apply Vitali's $5r$-covering lemma to get a family of $(16,128^{n+2})$-doubling balls $B_i$, $i\in I$,
which satisfy the following:
\begin{itemize}
\item the balls $2B_i$, $i\in I$, are pairwise disjoint,
\item $A\subset \bigcup_{i\in I} 10B_i$,
\item 
$W_\nu(B_i)\geq M\,\Theta(\sH)\,\nu(B_i)$ and $\theta_\eta(\gamma B_i)\leq c_2\,\Theta(\sH)$ and  for all $i\in I$ and $\gamma\geq1$.
\end{itemize}
Then we deduce
\begin{align}\label{eqplug6}
\nu(A) &\leq \sum_{i\in I} \nu(10B_i) \lesssim \sum_{i\in I} \nu(B_i)
\leq \frac1{M\,\Theta(\sH)} \sum_{i\in I} W_\nu(B_i)\\
& \lesssim \frac1{M\,\Theta(\sH)} \sum_{i\in I}\sum_{Q\in\DD_{\eta}:Q\subset 2B_i} \frac{\ell(Q)}{r(B_i)}\,\theta_\nu(2B_Q)\,\nu(Q).\nonumber
\end{align}
Now we take into account that all the cubes $Q$ which are not contained in any cube $P$ with $P^{(\mu)}\in\sH'$
satisfy $\theta_\nu(2B_Q)\leq \theta_\eta(2B_Q)\lesssim \Theta(\sH)$. Then, for each $i\in I$,
\begin{align*}
\sum_{Q\in\DD_{\eta}:Q\subset 2B_i} \frac{\ell(Q)}{r(B_i)}\,\theta_\nu(2B_Q)\,\nu(Q) & \leq
\sum_{P^{(\mu)}\in \sH'}\,
\sum_{Q\in\DD_{\eta}:Q\subset 2B_i\cap P} \frac{\ell(Q)}{r(B_i)}\,\theta_\eta(2B_Q)\,\eta(Q)\\
&\quad + \Theta(\sH)\sum_{Q\in\DD_{\eta}:Q\subset 2B_i} \frac{\ell(Q)}{r(B_i)}\, \nu(Q)\\
\end{align*}
Observe that the last term on the right hand side is bounded above by $\Theta(\sH)\nu(2B_i)\approx
\Theta(\sH)\nu(B_i)$. 
So plugging the previous estimate into \rf{eqplug6}, we get
\begin{align*}
\nu(A) \lesssim \frac{1}{M\,\Theta(\sH)} \sum_{i\in I}
\sum_{P^{(\mu)}\in \sH'}\,
\sum_{Q\in\DD_{\eta}:Q\subset 2B_i\cap P} \frac{\ell(Q)}{r(B_i)}\,\theta_\eta(2B_Q)\,\eta(Q)
+ \frac{1}{M} \sum_{i\in I} \nu(B_i).
\end{align*}
Since $B_i\subset A$ for each $i$, the last term is at most $\frac12\nu(A)$ for $M$ big enough. Thus,
\begin{equation}\label{eqnua12}
\nu(A) \lesssim  \frac{1}{M\,\Theta(\sH)} \sum_{i\in I}
\sum_{P^{(\mu)}\in \sH'}\,
\sum_{Q\in\DD_{\eta}:Q\subset 2B_i\cap P} \frac{\ell(Q)}{r(B_i)}\,\theta_\eta(2B_Q)\,\eta(Q).
\end{equation}

To estimate the term on the right hand side above, we denote by $\sF_k$ the family of cubes from
$\DD_\eta$ which are contained in some cube $Q^{(\eta)}$ with $Q\in\sH_{k}':=\sH_{k}(e_{h(R),j(R)+1}(R))$ and are not contained
in any cube $P^{(\eta)}$ with $P\in\sH_{k+1}':=\sH_{k+1}(e_{h(R),j(R)+1}(R))$. Notice that
$$\theta_\eta(2B_Q)\lesssim \Theta(\sH_{k+1})\approx \Theta(\sH_{k}),$$
where $\Theta(\sH_{k})=\Theta(Q')$ for $Q'\in\sH_{k}'$ (this does not depend on the specific cube $Q'$). Then, for each $i\in I$, we have
\begin{align}\label{eqalj4}
\sum_{P^{(\mu)}\in \sH'}
\sum_{Q\in\DD_{\eta}:Q\subset 2B_i\cap P} \frac{\ell(Q)}{r(B_i)}\,\theta_\eta(2B_Q)\eta(Q) &=\!\!
\sum_{k\geq k(R)}\sum_{P^{(\mu)}\in \sH'_k}
\sum_{Q\in\sF_k:Q\subset 2B_i\cap P} \frac{\ell(Q)}{r(B_i)}\,\theta_\eta(2B_Q)\eta(Q)\\
& \lesssim 
\sum_{k\geq k(R)}\sum_{P^{(\mu)}\in \sH'_k} \Theta(\sH_k)
\sum_{Q\in\sF_k:Q\subset 2B_i\cap P} \frac{\ell(Q)}{r(B_i)}\,\eta(Q).\nonumber
\end{align}
We claim now that for $Q$ in the last sum, we have
\begin{equation}\label{eqcond*492}
\ell(Q)\lesssim A_0^{-(k-k(R))}\,r(B_i).
\end{equation}
To check this, let $P(Q,k)\in \sH_k'$ be such that $Q\subset P(Q,k)$. From the condition 
\begin{equation}\label{eqcond492}
\theta_\eta( \gamma B_i)\leq c_2\,\Theta(\sH) \quad\mbox{for all $\gamma\geq1$}
\end{equation}
and the fact that $P(Q,k)\cap 2B_i\neq\varnothing$ (because $Q\subset 2B_i$) we infer that 
$r(B_i)\geq \ell(P(Q,k))$ for $k$ big enough. Otherwise we would find a ball $\gamma B_i$ containing $P(Q,k)$ with radius comparable to $\ell(P(Q,k))$, so that
$$\theta_\eta(\gamma B_i) \geq c\,\Theta(P(Q,k)) >c_2\,\Theta(\sH)$$
for $k$ big enough (depending on $c_2$), contradicting \rf{eqcond492}. So we have $P(Q,k)\subset 6B_i$ and thus
$$c_2\Theta(\sH)\geq \theta_\eta(6B_i)\gtrsim \frac{\ell(P(Q,k))^n}{r(B_i)^n}\,\Theta(P(Q,k))= 
\frac{\ell(P(Q,k))^n}{r(B_i)^n}\,A_0^{n(k-k(R))}\,\Theta(\sH).$$
This gives 
$$\ell(Q)\leq \ell(P(Q,k))\lesssim A_0^{-(k-k(R))}\,r(B_i)$$
and proves \rf{eqcond*492} for $k$ big enough, and thus for all $k$ if we choose the implicit constant in \rf{eqcond*492} suitably.

From \rf{eqcond*492} we deduce that the right hand side of \rf{eqalj4}
does not exceed
\begin{multline*}
\sum_{k\geq k(R)}A_0^{-(k-k(R))/2} \Theta(\sH_k)\sum_{P^{(\mu)}\in \sH'_k} 
\sum_{Q\subset 2B_i\cap P} \left(\frac{\ell(Q)}{r(B_i)}\right)^{1/2}\,\eta(Q)\\
\lesssim \sum_{k\geq k(R)}A_0^{-(k-k(R))/2} \,\Theta(\sH_k)\sum_{P^{(\mu)}\in \sH'_k} 
\eta(2B_i\cap P).
\end{multline*}
Plugging this estimate into \rf{eqnua12} and recalling that the balls $2B_i$ are disjoint, we get
\begin{align*}
\nu(A) & \lesssim \frac{1}{M\,\Theta(\sH)} \sum_{i\in I}
\sum_{k\geq k(R)}A_0^{-(k-k(R))/2} \sum_{P^{(\mu)}\in \sH'_k} \Theta(\sH_k)\, \eta(2B_i\cap P)\\
& \leq
\frac{1}{M\,\Theta(\sH)}
\sum_{k\geq k(R)}A_0^{-(k-k(R))/2} \sum_{P^{(\mu)}\in \sH'_k} \Theta(\sH_k) \,\eta(P).
\end{align*}
By H\"older's inequality, for each $k\geq k(R)$ we have
$$\sum_{P^{(\mu)}\in \sH'_k} \Theta(\sH_k) \eta(P) \leq \sigma_p(\sH'_k)^{1/p}\,\eta(H')^{1/p'}.
$$
We can estimate the right hand side using \rf{eqsigmah}:
\begin{align*}
\sigma_p(\sH'_k)&\le \Lambda_*^{\ve_n}\,\sigma_p(\sH),\\ \eta(H')&=\frac{\sigma_p(\sH')}{\Theta(\sH)^p} \le \Lambda_*^{\ve_n}\,\frac{\sigma_p(\sH)}{\Theta(\sH)^p} = \Lambda_*^{\ve_n}\,\eta(H).
\end{align*}
Thus,
$$\nu(A) \lesssim \frac{\Lambda_*^{\ve_n}}{M\,\Theta(\sH)} \sum_{k\geq k(R)}A_0^{-(k-k(R))/2} \,\sigma_p(\sH)^{1/p}\,\eta(H)^{1/p'}
\approx
\frac{\Lambda_*^{\ve_n}}{M}\,\eta(H)\approx \frac{\Lambda_*^{\ve_n}}{M}\,\nu(H).$$
\end{proof}
\vv


Let $\TT_{\sH'}$ denote the family of all cubes from $\DD_\mu$ 
 made up of $R$ and all the cubes of the next generations which are contained in $e_{h(R),j(R)+1}(R)$ but are not 
strictly contained in any cube from from $\sH'$. We consider the function
\begin{equation}\label{eqdefPsi1}
\Psi(x) = \inf_{Q\in \TT_{\sH'}} \big(\ell(Q) +\dist(x,Q)\big).
\end{equation}
Notice that $\Psi$ is a $1$-Lipschitz function. We also have the following result, which will be proven
by quite standard arguments.

\begin{lemma}\label{lem6.77}
For all $x\in\R^{n+1}$, we have
\begin{equation}\label{ecreixnupsi}
\sup_{r\geq \Psi(x)} \frac{\nu(B(x,r))}{r^n} \leq \sup_{r\geq \Psi(x)} \frac{\eta(B(x,r))}{r^n} \lesssim \Theta(\sH)\quad\mbox{ for all $x\in \R^{n+1}$.}
\end{equation}
\end{lemma}

\begin{proof}
The first inequality in \rf{ecreixnupsi} is trivial. Concerning the second one,
in the case $r>\ell(R)/10$ we just use that
$$\eta(B(x,r))\leq \mu(e'(R))\lesssim\mu(R)\lesssim \Theta(R)\,\ell(R)^n\lesssim \Theta(\sH)\,r^n.$$
So we may assume that $\Psi(x)<r\leq \ell(R)/10$.
By the definition of $\Psi(x)$, there exists $Q\in\TT_{\sH'}$ such that
$$\ell(Q) + \dist(x,Q)\leq r.$$
Therefore, $B_Q\subset B(x,4r)$ and so there exists an ancestor $Q'\supset Q$ which belongs to $\TT_{\sH'}$ such that $B(x,r)\subset  2B_{Q'}$, with $\ell(Q')\approx r$. 
Then, $\Theta(Q')\leq \Theta(\sH)$ and
$$\eta(B(x,r))\leq \sum_{P\in\Reg(e'(R)):P\cap 2B_{Q'}\neq\varnothing} \eta(P^{(\eta)}) = \sum_{P\in\Reg(e'(R)):P\cap 2B_{Q'}\neq\varnothing} \mu(P).
$$
Observe now that if $P\in\Reg(e'(R))$ satisfies $P\cap 2B_{Q'}\neq\varnothing$, then $\ell(P)\lesssim \ell(Q')$ (this is an easy consequence of Lemma \ref{lem74}(b) and the fact that $Q'$ contains some cube from $\Reg(e'(R))$). So we deduce that
$P\subset CB_{Q'}$. Hence,
$$\eta(B(x,r))\leq \mu(CB_{Q'}) \lesssim \Theta(\sH)\,\ell(Q')^n \approx \Theta(\sH)\,r^n.$$
\end{proof}

\vv

In the next subsection we are going to use a variational argument to show that for some constant $c_3$, depending at most\footnote{The constant $c_3$
	will be chosen of the form $c_3=A_0^{C(n)}$, and it will not depend on $\Lambda_*$, $\ve_n$, or other
	parameters of the construction.} on $n$ and $A_0$, we have
$$\int \big|(|\RR\nu(x)| - G(x)- c_3\,\Theta(\sH))_+\big|^p\,d\nu(x) \gtrsim \Lambda_*^{-p'\ve_n}\sigma_p(\sH).$$
See Lemma \ref{lemvar} for details. We prove this estimate by contradiction, so that in particular we will assume that
\begin{equation}\label{eqassu8}
\int \big|(|\RR\nu(x)| - G(x)- c_3\,\Theta(\sH))_+\big|^p\,d\nu(x) \leq \sigma_p(\sH).
\end{equation}
Below we prove a few consequences of \eqref{eqassu8} that will be useful in the proof of Lemma \ref{lemvar}.

We denote
$$\RR_{*,\Psi}\nu(x) = \sup_{\ve>\Psi(x)} \big|\RR_\ve\nu(x)\big|.$$

\begin{lemma}\label{lemrieszpetit}
Suppose that \eqref{eqassu8} holds. Then,
\begin{equation}\label{eqassu9}
\nu\big(\big\{x\in \cS_\eta:\,\RR_{*,\Psi}\nu(x) > M\,\Theta(\sH)\big\}\big) \leq C_4\, 
\frac{\Lambda_*^{\ve_n}}{M} \,\nu(H).
\end{equation}
\end{lemma}

\begin{proof}
Recall that we denote by $V_1$ the $A_0^{-3}\ell(R)$-neighborhood of $\cS_\eta$
and that
$$G(x)\lesssim \Theta(R)\quad\mbox{ for all $x\in V_1$.}$$ 
%In fact, the last estimate also holds in the $\frac34c_4\ell(R)$-neighborhood of 
%$e_{h(R),j(R)}(R)$, which we will denote by $\wt V_1$.
Then the assumption in the lemma implies that
\begin{equation}\label{eqsig84}
\int_{V_1} \big|\RR\nu(x)|^p\,d\nu(x) \lesssim \sigma_p(\sH).
\end{equation}

By Remark \ref{rem**}, for any $x\in \cS_\eta$ and $\ve>\Psi(x)$,
$$|\RR_\ve\nu(x)|\lesssim \avint_{B(x,2\ve')} |\RR\nu|\,d\nu + \sup_{r\geq \ve} \frac{\nu(B(x,r))}{r^n} + \frac{W_\nu(B(x,8\ve'))}{\nu(B(x,8\ve'))},$$
where $\ve'=2^{7k}\ve$, with $k\geq0$ minimal such that the ball $B(x,\ve')$ is $(128,128^{n+2})$-doubling. 
In the case $\ve'\geq \frac12 A_0^{-3}\,\ell(R)$, by standard arguments,
$$
|\RR_\ve \nu(x)|\leq |\RR_{\ve'} \nu(x)| + C\,\frac{\nu(B(x,\ve'))}{(\ve')^n} \leq C\,
\Theta(R)< M\,\Theta(\sH),$$
for $\Lambda_*$ (or $M$) big enough.

In the case $\ve'< \frac12 A_0^{-3}\ell(R)$, we have $B(x,2\ve')\subset V_1$ and thus
$$|\RR_\ve\nu(x)|\lesssim \cM_\nu^{(\Psi(x),16,128^{n+2})}(\chi_{V_1}\RR\nu)(x) 
+ \sup_{r\geq \Psi(x)} \frac{\nu(B(x,r))}{r^n} + \sup_{r\in D(x):r\geq \Psi(x)} \\
 \frac{W_\nu(B(x,r))}{\nu(B(x,r))},
$$
where $D(x)$ denotes the set of radii $r>0$ such that $B(x,r)$ is $(16,128^{n+2})$-doubling.
Also, as shown in \rf{ecreixnupsi},
$$\sup_{r\geq \Psi(x)} \frac{\nu(B(x,r))}{r^n} \lesssim \Theta(\sH).$$

We deduce that in any case (i.e., for any $\ve'$), assuming $M$ larger than some absolute constant,
\begin{align*}
\nu\big(\big\{x\in \cS_\eta:\,\RR_{*,\Psi}\nu(x) &> M\,\Theta(\sH)\big\}\big) \\& \leq
\nu\Big(\Big\{x\in \cS_\eta:\,\cM_\nu^{(\Psi(x),16,128^{n+2})}(\chi_{V_1}\RR\nu)(x) > \frac{M\,\Theta(\sH)}3\Big\}\Big)\\
& \!\!+
\nu\Big(\Big\{x\in \cS_\eta:\!\!\!\sup_{r\in D(x):r\geq \Psi(x)} \!
 \frac{W_\nu(B(x,r))}{\nu(B(x,r))} > \frac{M\,\Theta(\sH)}3\Big\}\Big)\\
 & =: T_1 + T_2.
\end{align*}

To deal with the term $T_1$, we use the weak $L^p(\nu)$ boundedness of $\cM_\nu^{(\Psi(x),16,128^{n+2})}$ and \rf{eqsig84} to obtain
$$T_1\lesssim \frac1{(M\Theta(\sH))^p} \int_{V_1}|\RR\nu|^p\,d\nu
\lesssim \frac1{M^p}\,\eta(H).
$$
Concerning $T_2$, by Lemma \ref{lem*921},
$$T_2 \lesssim \frac{\Lambda_*^{\ve_n}}{M} \,\nu(H).$$
So we have
$$\nu\big(\big\{x\in \cS_\eta:\,\RR_{*,\Psi}\nu(x) > M\,\Theta(\sH)\big\}\big)\lesssim 
\frac1{M^p}\,\eta(H) + \frac{\Lambda_*^{\ve_n}}{M} \,\nu(H)\lesssim \frac{\Lambda_*^{\ve_n}}{M} \,\nu(H).$$
\end{proof}
\vv



For $\gamma>1$, we let
$$Z(\gamma) = \big\{x\in \cS_\eta': \RR_{*,\Psi}\nu(x) > \gamma \Lambda_*^{\ve_n}\,\Theta(\sH)\big\}.$$
Notice that, under the assumption  \rf{eqassu8}, by Lemma \ref{lemrieszpetit}
 (with $M= \gamma \Lambda_*^{\ve_n}$), we have
\begin{equation}\label{eqzgam1}
\nu(Z(\gamma)\cap \cS_\eta) \leq C_4\gamma^{-1}\,\nu(H).
\end{equation}
For $x\in Z(\gamma)$, let
\begin{equation}\label{eqe3194}
e(x,\gamma) = \sup\{\ve: \ve>\Psi(x),\,|\RR_\ve\nu(x)|>\gamma \Lambda_*^{\ve_n}\,\Theta(\sH)\}
\end{equation}
%For $x\in S_\eta\setminus Z_0$, we set $e(x)=0$.
We define the exceptional set $Z'(\gamma)$ as
$$Z'(\gamma) = \bigcup_{x\in Z(\gamma)} B(x,e(x,\gamma)).$$

The next lemma shows that $\nu(Z'(\gamma)\cap \cS_\eta)$ is small if $\gamma$ is taken big enough.

\vv
\begin{lemma} \label{tamt}
If $y\in Z'(\gamma)$, then $\RR_{*,\Psi}\nu(y) > (\gamma \Lambda_*^{\ve_n}- c_4)\Theta(\sH)$,
for some $c_4>0$.
Thus, under the assumption  \rf{eqassu8},
if $\gamma\geq 2c_4$, then
\begin{equation} \label{wert1}
\nu(Z'(\gamma)\cap \cS_\eta) \leq \frac{2C_4}{\gamma}\,\nu(H).
\end{equation}
\end{lemma}

\begin{proof}
The arguments are similar to the ones in Lemma 5.2 from
\cite{Tolsa-llibre}. However, for the reader's convenience we will explain here the basic details.

By standard arguments (which just use the fact that the Riesz kernel is a Calder\'on-Zygmund kernel),
for all $y\in B(x,e(x,\gamma))$, with $x\in Z(\gamma)$,
 we have
$$|\RR_{e(x,\gamma)}\nu(x) -\RR_{2e(x,\gamma)}\nu(y)|\lesssim \sup_{r\geq e(x,\gamma)} \frac{\nu(B(x,r))}{r^n} \lesssim \Theta(\sH),$$
taking into account that $e(x,\gamma)\geq\Psi(x)$ and recalling \rf{ecreixnupsi} for the last estimate.
So we have
$$|\RR_{2e(x,\gamma)}\nu(y)| \geq |\RR_{e(x,\gamma)}\nu(x)|- c_4\,\Theta(\sH) \geq  \gamma\,\Lambda_*^{\ve_n}\,\Theta(\sH) -  c_4\,\Theta(\sH).$$
Observe now that
$$\Psi(y)\leq \Psi(x) + |x-y|< 2e(x,\gamma).$$ 
So 
$$\RR_{*,\Psi}\nu(y) \geq |\RR_{2e(x,\gamma)}\nu(y)|> (\gamma \Lambda_*^{\ve_n}- c_4)\Theta(\sH),$$
which proves the first statement of the lemma.

In particular, taking $\gamma\geq 2c_4$, from the last estimate we derive 
$$\RR_{*,\Psi}\nu(y) \geq \frac\gamma2\,\Lambda_*^{\ve_n}\,\Theta(\sH),$$
and so $y\in Z(\gamma/2)$, which shows that $Z'(\gamma)\subset Z(\gamma/2)$. Thus,
by \rf{eqzgam1},
$$\nu(Z'(\gamma)\cap \cS_\eta) \leq \nu(Z(\gamma/2)\cap \cS_\eta) 
\leq 2C_4\gamma^{-1}\,\nu(H).$$
\end{proof}
\vv

We choose $\gamma=\max(1,2c_4,4C_4)$, so that, under the assumption  \rf{eqassu8},
\begin{equation}\label{eqnugam*}
\nu(Z'(\gamma)\cap \cS_\eta) \leq \frac12\,\nu(H).
\end{equation}
Also we define
$$\Phi(x) = \max(\Psi(x),\dist(x,\R^{n+1}\setminus Z'(\gamma)).$$
Notice that $\Phi$ is a $1$-Lipschitz function which coincides with $\Psi(x)$ away from $Z'(\gamma)$ and
that 
\begin{equation}\label{eqphigam}
\Phi(x)\geq e(x,\gamma)\quad \mbox{ for all $x\in Z(\gamma)$.}
\end{equation}


\vv

\begin{lemma}\label{lemrieszphi4}
The suppressed operator $\RR_\Phi$ is bounded in $L^p(\nu)$, with
$\|\RR_\Phi\|_{L^p(\nu)\to L^p(\nu)}\lesssim \Lambda_*^{\ve_n}\Theta(\sH).$
\end{lemma}

\begin{proof}
This is an immediate consequence of Theorem \ref{teontv} and the construction of $\Phi$ above.
Indeed, by \rf{ecreixnupsi},
$$\nu(B(x,r)) \lesssim \Theta(\sH)\,r^n \quad\mbox{ for all $r\geq\Phi(x)$.}$$
Also, by \rf{eqphigam},
$$\sup_{\ve>\Phi(x)}|\RR_\ve\nu(x)|\leq \sup_{\ve>e(x,\gamma)}|\RR_\ve\nu(x)|
\leq \gamma \Lambda_*^{\ve_n}\,\Theta(\sH)
\quad \mbox{ for all $x\in Z(\gamma)$.}$$
On the other hand, by the definition of $Z(\gamma)$ we also have
$$\sup_{\ve>\Phi(x)}|\RR_\ve\nu(x)|\leq \sup_{\ve>\Psi(x)}|\RR_\ve\nu(x)|\leq
\gamma \Lambda_*^{\ve_n}\,\Theta(\sH) \quad \mbox{ for $x\in \cS_\eta'\setminus Z(\gamma)$.} 
$$
So we can apply Theorem \ref{teontv}, taking 
$\omega= (C\Lambda_*^{\ve_n}\Theta(\sH))^{-1}\,\nu$ with an appropriate absolute constant $C$, and
then the lemma follows.
\end{proof}

\vv

\begin{lemma}\label{lemH0}
Under the assumption \rf{eqassu8},
there exists a subset $H_0\subset H$ such that:
\begin{itemize}
\item[(a)] $\eta(H_0)\geq \frac12\,\eta(H)$,
\item[(b)] $\RR_{\Psi,\eta}$ is bounded from $L^p(\eta\rest_{H_0})$ to $L^p(\eta\rest_{\cS'_\eta})$, with 
$\|\RR_\Psi\|_{L^p(\eta\rest_{H_0})\to L^p(\eta\rest_{\cS'_\eta})}\lesssim \Lambda_*^{\ve_n}\Theta(\sH).$
\end{itemize}
\end{lemma}

\begin{proof}
We let 
$$H_0= H\setminus Z'(\gamma),$$
so that, by \rf{eqnugam*},
$$\eta(H_0) = \nu(H_0) \geq \nu(H) - \nu(Z'(\gamma)\cap \cS_\eta) \geq \frac12 \,\nu(H) = \frac12 \,\eta(H).$$

To prove (b), take $f\in L^p(\eta\rest_{H_0})$ and observe that for $x\in \cS'_\eta$, by \rf{e.compsup''},
$$|\RR_{\Psi} (f\eta)(x)| = |\RR_{\Psi} (f\nu)(x)|\leq |\RR_{\Psi(x)} (f\nu)(x)| + 
\cM_{\Psi,n}(f\nu)(x),$$
where $\RR_{\Psi(x)}$ is the $\Psi(x)$-truncated Riesz transform and
\begin{equation}\label{eq:maximaloppsi}
\cM_{\Psi,n} \alpha(x) = \sup_{r>\Psi(x)}\frac{|\alpha|(B(x,r))}{r^n}
\end{equation}
for any signed Radon measure $\alpha$. Taking into account that $\Phi(x)\geq\Psi(x)$, we can split
$$\RR_{\Psi(x)} (f\nu)(x) = \RR_{\Phi(x)} (f\nu)(x) + \int_{y\in H_0:\Psi(x)<|x-y|\leq \Phi(x)} 
\frac{x-y}{|x-y|^{n+1}}\,f(y)\,d\nu(y).$$
To estimate the last integral, notice that for $y\in H_0$, $\Psi(y) = \Phi(y)$, and then the condition $\Psi(x)<|x-y|$ implies that
$$\Phi(x) \leq \Phi(y) + |x-y| = \Psi(y) + |x-y| \leq \Psi(x) + 2|x-y|<3|x-y|.$$
So the last integral above is bounded by
$$\int_{\Phi(x)/3<|x-y|\leq \Phi(x)} \frac1{|x-y|^n}\,|f(y)|\,d\nu(y)\lesssim \cM_{\Psi,n}(f\nu)(x).$$
From the preceding estimate and \rf{e.compsup''} we derive that
$$|\RR_{\Psi} (f\eta)(x)| \leq |\RR_{\Phi(x)} (f\nu)(x)| + C\,\cM_{\Psi,n}(f\nu)(x) \leq
|\RR_{\Phi} (f\nu)(x)| + C\,\cM_{\Psi,n}(f\nu)(x).$$

From the last inequality and Lemma \ref{lemrieszphi4}, taking into account that $\eta$ coincides with $\nu$ on $V_1$, we deduce that 
$$\|\RR_\Psi\|_{L^p(\eta\rest_{H_0})\to L^p(\eta\rest_{V_1})}\lesssim \Lambda_*^{\ve_n}\Theta(\sH) 
+ \|\cM_{\Psi,n}\|_{L^p(\eta)\to L^p(\eta)}.$$
From the growth condition \rf{ecreixnupsi} and standard covering lemmas, it follows easily that
$$\|\cM_{\Psi,n}\|_{L^p(\eta)\to L^p(\eta)}\lesssim\Theta(\sH).$$
On the other hand, using the fact that $\dist(H_0,\R^{n+1}\setminus V_1)\gtrsim \ell(R)$, it is immediate (by Schur's criterion, for example) to check that also
$$\|\RR_\Psi\|_{L^p(\eta\rest_{H_0})\to L^p(\eta\rest_{\cS_\eta'\setminus V_1})}\lesssim 
\Theta(R)\leq \Theta(\sH).$$
\end{proof}


\vv

% ********************************************************************************************

\subsection{The variational argument}


\begin{lemma}\label{lemvar}
There is a constant $c_3>0$, depending at most on $n$ and $A_0$, such that for any $p\in (1,2]$, if $\Lambda_*$ is big enough, we have
$$\int \big|(|\RR\nu(x)| - G(x)- c_3\,\Theta(\sH))_+\big|^p\,d\nu(x) \gtrsim \Lambda_*^{-p'\ve_n}\sigma_p(\sH),$$
where $p'=p/(p-1)$ and the implicit constant depends on $p$.
\end{lemma}


\begin{proof}
Suppose that, for a very small $0<\lambda<1$ to be fixed below,
\begin{equation}\label{eqsupp1}
\int \bigl|\bigl(|\RR\nu| - G - c_3\,\Theta(\sH)\bigr)_+\bigr|^p\,d\nu\leq \lambda\,\sigma_p(\sH).
\end{equation}
%We consider the subfamily $\sH_0\subset \sH$ of the cubes $Q$ from $\sH$ that are contained in $S_\eta$ and satisfy  
%$$\frac1{\eta(Q)} \int_Q |\RR\nu|\leq \Theta(\sH).$$
%We also denote
%$$H_0=\bigcup_{Q\in\sH_0}Q.$$ 

%Millor dir aixo cert per maximal de HL doblant. Cubs de H que contenen algun punt en que el maximal 


Let $H_0\subset H$ be the set found in Lemma \ref{lemH0}.
 Consider the measures of the form
$\tau=a\,\nu$, with $a\in L^\infty(\nu)$, $a\geq 0$, and let $F$ be the functional
\begin{align*}\label{e.functional}
F(\tau) & = 
\int \bigl|\bigl(|\RR\tau| - G- c_3\,\Theta(\sH)\bigr)_+\bigr|^p\,d\tau
 + \lambda\,\|a\|_{L^\infty(\nu)} \,\sigma_p(\sH) + \lambda \,\frac{\nu(H_0)}{\tau(H_0)}\,\sigma_p(\sH).\nonumber
\end{align*}
Let
$$m= \inf F(\tau),$$ where the infimum is taken over all the measures $\tau=a\,\nu$, with $a\in L^\infty(\nu)$. We call such measures admissible. 
Since $\nu$ is admissible we infer that
\begin{equation}
\label{e.admis}
m \leq F(\nu) \leq 3\lambda\,\sigma_p(\sH).
\end{equation}
So the infimum $m$ is attained over the collection of
measures $\tau=a\nu$ such that $\|a\|_{L^\infty(\nu)}\leq 3$ and $
\tau(H_0)\geq\frac13\,\nu(H_0)$. In particular, by taking a weak $*$ limit in $L^\infty(\nu)$, this guaranties
the existence of a minimizer.

Let $\tau$ be an admissible measure such that 
\begin{equation}\label{e.admis2}
	m=F(\tau)\leq 3\lambda\,\sigma_p(\sH).
\end{equation}
To simplify notation, we denote
$$f(x) = G + c_3\,\Theta(\sH).$$
We claim that
\begin{equation}\label{eqclaim*}
\bigl|\bigl(|\RR\tau(x)| - f(x)\bigr)_+|^p + 
p \,\RR_\tau^*\Bigl[\bigl|\bigl(|\RR\tau| - f\bigr)_+|^{p-1} |\RR\tau|^{-1} \RR\tau \Bigr](x)
 \lesssim \lambda\,\Theta(\sH)^p \quad \text{\!\!in $\supp\tau$,}
\end{equation}
where for a vector field $U$, we wrote
$$\RR_\tau^*U =\RR^*(U\tau) = -\sum_{i=1}^{n+1}\RR_i(U_i\,\tau).$$
Assume \rf{eqclaim*} for the moment. Since the function on the left hand side is subharmonic
in $\R^{n+1}\setminus\supp\tau$ 
and continuous in $\R^{n+1}$ (recall that $\tau$ and $\eta$ are absolutely continuous with respect to Lebesgue measure, with bounded densities), we deduce that the same estimate holds in the whole $\R^{n+1}$.

Next we need to construct an auxiliary function $\vphi$. First we claim that 
there exists a subfamily of cubes $\sH^0\subset \sH$ such that
\begin{itemize}
\item[(i)] the balls $3B_Q\equiv3B_{Q^{(\mu)}}$, with $Q^{(\mu)}\in\sH^0$, are disjoint,
\item[(ii)] $\frac1{12}\nu(Q)\leq \tau(Q\cap H_0)\leq 3 \nu(Q)$ for all $Q^{(\mu)}\in\sH^0$, and
\item[(iii)] $\sum_{Q^{(\mu)}\in\sH^0}\tau(Q\cap H_0)\approx \nu(H)$.
\end{itemize}
The existence of the family $\sH^0$ follows easily from the fact that
 $\tau(H_0)\geq\frac13\nu(H_0)\geq \frac16\,\nu(H)$ and $\tau=a\,\nu$ with $\|a\|_{L^\infty(\nu)}\leq 3$. Indeed,
 notice that the family $I$ of cubes $Q^{(\mu)}\in \sH$ such that 
$\tau(Q\cap H_0)\leq \frac1{12}\nu(Q)$ satisfies
$$\sum_{Q^{(\mu)}\in I}\tau(Q\cap H_0)\leq \frac1{12}\sum_{Q^{(\mu)}\in I}\nu(Q)\leq \frac1{12}\,\nu(H)
\leq \frac12\,\tau(H_0).$$
So
$$\sum_{Q^{(\mu)}\in \sH\setminus I} \tau(Q)\geq \sum_{Q^{(\mu)}\in \sH\setminus I} \tau(Q\cap H_0)\geq \frac12\,\tau(H_0).$$
By Vitali's 5r-covering lemma, there exists a subfamily $ \sH^0\subset \sH\setminus I$ 
such that the balls $\{3B_{Q}\}_{Q^{(\mu)} \in\sH^0}$ are disjoint and the balls $\{15B_{Q}\}_{Q^{(\mu)}\in\sH^0}$ cover
$\bigcup_{Q^{(\mu)}\in \sH\setminus I}Q$. From the fact that the cubes from $\sH^0$ are $\PP$-doubling and the properties of the family $\Reg(e'(R))$ in Lemma \ref{lem74}, we have
$\nu(Q) \approx \nu(15B_Q)$ if $Q^{(\mu)}\in\sH^0$, and thus
\begin{align*}
\sum_{Q^{(\mu)}\in\sH^0} \tau(Q\cap H_0) & \geq \frac1{12}\sum_{Q^{(\mu)}\in\sH^0} \nu(Q) \approx
\sum_{Q^{(\mu)}\in\sH^0} \nu(15B_Q) \\
& \geq \sum_{Q^{(\mu)}\in \sH\setminus I} \nu(Q)\geq \frac13\,
\sum_{Q^{(\mu)}\in \sH\setminus I} \tau(Q)\geq \frac16\,\tau(H_0)\geq \frac1{36}\,\nu(H).
\end{align*}
The converse estimate $\sum_{Q^{(\mu)}\in\sH^0}\tau(Q\cap H_0)\lesssim \nu(H)$ follows trivially from  $\|a\|_{L^\infty(\nu)}\leq3$.

We consider the function
\begin{equation}\label{eqvphi99}
\vphi = \sum_{Q^{(\mu)}\in\sH^0} \Theta(Q)\,\vphi_Q,
\end{equation}
where $\chi_{B_Q}\leq \vphi_Q\leq \chi_{1.1B_Q}$, $\|\nabla\vphi_Q\|_\infty\lesssim\ell(Q)^{-1}$. 
Remark that
$$\RR^*(\nabla\vphi\,\LL^{n+1}) = \tilde c\,\vphi,$$
where $\tilde c$ is some non-zero absolute constant.
Notice also that
\begin{equation}\label{eqnormpsi}
\|\nabla\vphi\|_1\lesssim \sum_{Q^{(\mu)}\in\sH^0} \Theta(Q)\,\ell(Q)^n\approx \tau(H_0).
\end{equation}
Multiplying the left hand side of \rf{eqclaim*} by $|\nabla\vphi|$, integrating with respect to the Lebesgue measure $\LL^{n+1}$, taking into account that the estimate in
\rf{eqclaim*} holds on the whole $\R^{n+1}$, and using \rf{eqnormpsi}, we get
\begin{multline}\label{eqpss3}
\int \bigl|\bigl(|\RR\tau| - f\bigr)_+|^p\,|\nabla\vphi|\,d\LL^{n+1} + p\! \int \RR_\tau^*\Bigl[\bigl|\bigl(|\RR\tau| - f\bigr)_+|^{p-1} |\RR\tau|^{-1} \RR\tau \Bigr]\,
|\nabla\vphi|\,d\LL^{n+1}\\
\lesssim \lambda\,\sigma_p(\sH).
\end{multline}

Next we estimate the second integral on the left hand side of the preceding inequality, which we denote by $I$.  For that purpose we will also need the estimate 
\begin{equation}\label{eqacpsi}
\int |\RR(|\nabla \vphi|\,d\LL^{n+1})|^p\,d\tau\lesssim \Lambda_*^{p\ve_n}\sigma_p(\sH),
\end{equation}
which we defer to Lemma \ref{lemtech79}.
Using this inequality we get
\begin{multline*}
|I| = \left|\int 
\bigl|\bigl(|\RR\tau| - f\bigr)_+|^{p-1} |\RR\tau|^{-1} \RR\tau \cdot
\RR(|\nabla \vphi|\,\LL^{n+1})\,d\tau\right| \\
\leq \left(\int \bigl|\bigl(|\RR\tau| - f\bigr)_+|^{p} \,d\tau\right)^{1/p'}
\left(\int |\RR(|\nabla \vphi|\,\LL^{n+1})|^p\,d\tau\right)^{1/p}\\
\le (F(\tau))^{1/p'}\left(\int |\RR(|\nabla \vphi|\,\LL^{n+1})|^p\,d\tau\right)^{1/p}\\
 \overset{\rf{e.admis2},\eqref{eqacpsi}}{\lesssim} \bigl (\lambda\,\sigma_p(\sH)\bigr)^{1/p'} \bigl (\Lambda_*^{p\ve_n}\sigma_p(\sH)\bigr)^{1/p} = \lambda^{1/p'}\,\Lambda_*^{\ve_n}\sigma_p(\sH).
\end{multline*}
From \rf{eqpss3} and the preceding estimate we deduce
that
$$\int \bigl|\bigl(|\RR\tau| - f\bigr)_+|^p\,|\nabla\vphi|\,d\LL^{n+1}\lesssim  \lambda^{1/p'}\,\Lambda_*^{\ve_n}
\sigma_p(\sH).
$$

Notice now that, for all $x\in \supp\vphi$,
$$G(x) = 2A_0^3\int_{\cS_\eta'\setminus V_2} 
 \frac1{\ell(R)\,|x-y|^{n-1}}\,d\eta(y) \lesssim \frac{\eta(\cS_\eta')}{\ell(R)\,\dist(\cS_\eta'\setminus V_2, \cS_\eta)^{n-1}} \lesssim \Theta(R),$$
and so
$$|f(x)|\leq C\, \Theta(R) + c_3\,\Theta(\sH)\leq 2c_3\,\Theta(\sH) \qquad\mbox{for all $x\in \supp\vphi$,}$$ 
taking into account that $\Lambda_*\gg1$.
So we have
\begin{align}\label{eqcont4}
\int |\RR\tau|^p\,|\nabla\vphi|\,d\LL^{n+1} &\lesssim \int \bigl|\bigl(|\RR\tau| - f\bigr)_+|^p\,|\nabla\vphi|\,d\LL^{n+1}
+ \int |f|^p\,|\nabla\vphi|\,d\LL^{n+1}\\
& \lesssim 
  \bigl(\lambda^{1/p'}\Lambda_*^{\ve_n} + c_3^p\bigr)\,\sigma_p(\sH).\nonumber
\end{align}

To get a contradiction, note that by the construction of $\vphi$ and by the properties of $\tau$, we have
\begin{multline}\label{eqprev}
\left|\int \RR\tau\cdot\nabla\vphi\,d\LL^{n+1}\right| = \left|\int \RR^*(\nabla\vphi\,\LL^{n+1})\,d\tau \right|=  
|\tilde c| \int \vphi\,d\tau\\
\ge |\tilde c|\,
\Theta(\sH) \sum_{Q^{(\mu)}\in\sH^0} \tau(B_Q) \gtrsim |\tilde c|\,
\Theta(\sH) \tau(H_0)\geq \frac{|\tilde c|}6\,
 \Theta(\sH) \nu(H).
 \end{multline}
However, by H\"older's inequality, \eqref{eqnormpsi}, and \rf{eqcont4}, we have
\begin{align*}%\label{eqa1}
\left|\int \RR\tau\cdot\nabla\vphi\,d\LL^{n+1}\right| & \leq \left(\int |\RR\tau|^p \,|\nabla\vphi|\,d\LL^{n+1} \right)^{1/p}
\left(\int |\nabla\vphi|\,d\LL^{n+1}\right)^{1/p'} \\
&\lesssim \bigl(\lambda^{1/(p\,p')} \Lambda_*^{\ve_n/p}+ c_3\bigr)
\, \Theta(\sH)\nu(H),
\end{align*}
which contradicts \rf{eqprev} if $c_3$ is chosen small enough and 
$$\lambda\leq c\,\Lambda_*^{-p'\ve_n},$$
with $c$ small enough.
\end{proof}


\vv

\begin{proof}[\bf Proof of \rf{eqclaim*}]
Recall that $\tau=a\,\nu$ is a minimizer for $F(\cdot)$ that in particular satisfies
$F(\tau)\leq 3\lambda\,\sigma_p(\sH),$ by \rf{e.admis2}.
We have to show that, for all $x\in\supp\tau$,
$$
\bigl|\bigl(|\RR\tau(x)| - f(x)\bigr)_+|^p + 
p \,\RR_\tau^*\Bigl[\bigl|\bigl(|\RR\tau| - f\bigr)_+|^{p-1} |\RR\tau|^{-1} \RR\tau \Bigr](x)
 \lesssim \lambda\,\Theta(\sH)^p.
$$

Let $x_0\in\supp\tau$ 
and let $B=B(x_0,\rho)$, with $\rho>0$ small. For $0<t<1$ we consider the competing measure $\tau_t =
a_t\,\tau$, where $a_t$ is defined as follows:
$$a_t = (1-t\,\chi_B)a.$$
Notice that, for each $0<t<1$, $a_t$ is a non-negative function such that $$\|a_t\|_{L^\infty(\nu)}\leq \|a\|_{L^\infty(\nu)}.$$
Taking the above into account we deduce that
\begin{align*}
F(\tau_t) & = 
\int \bigl|\bigl(|\RR\tau_t| - f\bigr)_+\bigr|^p\,d\tau_t
 + \lambda\,\|a_t\|_{L^\infty(\nu)} \,\sigma_p(\sH) + \lambda \,\frac{\nu(H_0)}{\tau_t(H_0)}\,\sigma_p(\sH)\\
& \leq
\int \bigl|\bigl(|\RR\tau_t| - f\bigr)_+\bigr|^p\,d\tau_t
 + \lambda\,\|a\|_{L^\infty(\nu)} \,\sigma_p(\sH) + \lambda \,\frac{\nu(H_0)}{\tau_t(H_0)}\,\sigma_p(\sH)
\\& =:\wt F(\tau_t).
\end{align*}
Since $\wt F(\tau) = F(\tau) \leq F(\tau_t)\leq \wt F(\tau_t)$ for $t\geq 0$, we infer that 
\begin{equation}\label{eqvar49}
\frac1{\tau(B)}\,\frac{d}{dt}\,\wt F(\tau_t)\biggr|_{t=0+} \geq 0.
\end{equation}
Using that
$$
\frac d{dt}|\RR\tau_t(x)|\biggr|_{t=0} = \frac1{|\RR\tau(x)|} \RR\tau(x)\cdot \RR (-\chi_B\tau )(x)
,$$
an easy computation gives 
\begin{align*}
\frac{d}{dt}\,\wt F(\tau_t)\biggr|_{t=0} & = -\int_B \bigl|\bigl(|\RR\tau| - f\bigr)_+|^p \,d\tau\\
& \quad +
p \int \bigl|\bigl(|\RR\tau| - f\bigr)_+|^{p-1}\, \frac1{|\RR\tau|} \RR\tau\cdot \RR (-\chi_B\tau)\,
d\tau \\
&\quad +  
\lambda \,\frac{\nu(H_0)}{\tau(H_0)^2}\,\sigma_p(\sH)\,\tau(B).
\end{align*}
Recalling that $\tau(H_0)\geq\frac13\,\nu(H_0)\ge \frac16\,\nu(H)$, from \rf{eqvar49} and the preceding calculation we derive
\begin{multline}\label{eqmult82}
 \frac1{\tau(B)} 
\int_B \bigl|\bigl(|\RR\tau| - f\bigr)_+|^p \,d\tau 
 +
 \frac p{\tau(B)} \int \bigl|\bigl(|\RR\tau| - f\bigr)_+|^{p-1}\, \frac1{|\RR\tau|} \RR\tau\cdot \RR (\chi_B\tau )
\,
d\tau
 \\ 
\leq 
3\lambda \,\frac{\sigma_p(\sH)}{\tau(H_0)} =  3\lambda\, \Theta(\sH)^p \,\frac{\nu(H)}{\tau(H_0)} \le 18 \lambda \, \Theta(\sH)^p.
\end{multline}


 We rewrite the left hand side of \rf{eqmult82} as
$$ 
 \frac 1{\tau(B)} \int_B \bigl|\bigl(|\RR\tau| - f\bigr)_+|^p \,d\tau 
 +
 \frac p{\tau(B)} \int_B \RR_\tau^*\Bigl[\bigl|\bigl(|\RR\tau| - f\bigr)_+|^{p-1}\, |\RR\tau|^{-1} \RR\tau \Bigr]
\,
d\tau
.$$
Taking into account that the functions in the integrands are continuous on $\supp(\tau)$, 
letting the radius $\rho$ of $B$ tend to $0$, it turns out that the above expression converges to
$$\bigl|\bigl(|\RR\tau(x_0)| - f(x_0)\bigr)_+|^p 
 +
p\, \RR_\tau^*\Bigl[\bigl|\bigl(|\RR\tau| - f\bigr)_+|^{p-1}\, |\RR\tau|^{-1} \RR\tau \Bigr](x_0).$$
The desired estimate \rf{eqclaim*} follows from the above and \rf{eqmult82}.
\end{proof}
\vv

In order to complete the proof of Lemma \ref{lemvar} we need the following technical result.

\begin{lemma}\label{lemtech79}
Suppose that \rf{eqsupp1} holds with $\lambda\leq1$ and let $\vphi$ be as in \rf{eqvphi99}.
Then,
$$
\int |\RR(|\nabla \vphi|\,d\LL^{n+1})|^p\,d\nu\lesssim \Lambda_*^{p\ve_n}\sigma_p(\sH),
$$
\end{lemma}

\begin{proof}
Recall that 
$$\vphi = \sum_{Q^{(\mu)}\in\sH^0} \Theta(Q)\,\vphi_Q,$$
with $\chi_{B_Q}\leq \vphi_Q\leq \chi_{1.1B_Q}$, $\|\nabla\vphi_Q\|_\infty\lesssim\ell(Q)^{-1}$.
We consider the function
$$g= \sum_{Q^{(\mu)}\in\sH^0} g_Q,$$
where $g_Q = c_Q\,\chi_{Q\cap H_0}$, with $c_Q=\Theta(Q)\int|\nabla\vphi_Q|\,d\LL^{n+1}\,\eta(Q\cap H_0)^{-1}$.
Observe that
\begin{equation}\label{eq:gQintegral}
\int g_Q\,d\eta = \Theta(Q) \int|\nabla\vphi_Q|\,d\LL^{n+1}
\end{equation}
and that
$$0\leq c_Q \lesssim \frac{\Theta(Q) \,\ell(Q)^n}{\eta(Q\cap H_0)}\approx \frac{\mu(Q^{(\mu)})}{\eta(Q)} =1.$$


The first step of our arguments consists in comparing $\RR(|\nabla\vphi|\,d\LL^{n+1})(x)$ to 
$\RR_{\Psi(x)}(g\,\eta)(x)$, with $\Psi$ given by \rf{eqdefPsi1}.
 We will prove that, for each $Q^{(\mu)}\in \sH^0$,
 \begin{align}\label{eqtech4}
|\RR(\Theta(Q)|\nabla\vphi_Q|&\,\LL^{n+1})(x) - \RR_{\Psi(x)}(g_Q\,\eta)(x)|\\
& \lesssim \frac{\Theta(Q)\,\ell(Q)^{n+1}}{\dist(x,Q)^{n+1}+
\ell(Q)^{n+1}} + \chi_{2B_Q}(x)\, |\RR_{\Psi(x)}(g_Q\,\eta)(x)|\nonumber \\
&\quad+ \int_{c\Psi(x)\leq |x-y|\leq\Psi(x)} \frac{g_Q(y)}{|x-y|^{n}}\,d\eta(y),\nonumber
\end{align}
for some fixed $c>0$.
 The arguments to prove this estimate are quite standard.
 
 Suppose first that $x\in 2B_Q$. In this case, we have
 \begin{align*}
|\RR(\Theta(Q)|\nabla\vphi_Q|\,\LL^{n+1})(x)| &\lesssim \Theta(Q)\int_{1.1B_Q} \frac{1}{\ell(Q)\,|x-y|^n}\,d\LL^{n+1}(y) \lesssim \Theta(Q),
\end{align*}
which yields
$$|\RR(\Theta(Q)|\nabla\vphi_Q|\,\LL^{n+1})(x) - \RR_{\Psi(x)}(g_Q\,\eta)(x)|
 \lesssim \Theta(Q) + |\RR_{\Psi(x)}(g_Q\,\eta)(x)|,$$
 and shows that \rf{eqtech4} holds in this situation.
 
In the case $x\not\in 2B_Q$ we write
\begin{align*}
\RR(\Theta(Q)|&\nabla\vphi_Q|\,\LL^{n+1})(x) - \RR_{\Psi(x)}(g_Q\,\eta)(x)\\  &=
\RR\big(\Theta(Q)|\nabla\vphi_Q|\,\LL^{n+1} - g_Q\,\eta\big)(x) +
\int_{|x-y|\leq\Psi(x)} \frac{x-y}{|x-y|^{n+1}} g_Q(y)\,d\eta(y)\\
& \overset{\eqref{eq:gQintegral}}{=} 
\int \left(\frac{x-y}{|x-y|^{n+1}} - \frac{x-x_Q}{|x-x_Q|^{n+1}}\right) \bigl[\Theta(Q)|\nabla\vphi_Q(y)|\,d\LL^{n+1}(y) - g_Q(y)\,d\eta(y)\bigr]\\
& \quad\quad\quad + \int_{|x-y|\leq\Psi(x)} \frac{x-y}{|x-y|^{n+1}} g_Q(y)\,d\eta(y)\\
& =: I_1(x)+ I_2(x).
\end{align*}
Concerning the term $I_1(x)$, recalling that $\supp\vphi_Q \cup\supp g_Q\subset 1.1\overline{ B_Q}$,
we obtain
\begin{align*}
|I_1(x)|
&\lesssim \int \frac{\ell(Q)}{|x-x_Q|^{n+1}} \bigl[\Theta(Q)|\nabla\vphi_Q(y)|\,d\LL^{n+1}(y) + g_Q(y)\,d\eta(y)\bigr]
\lesssim \frac{\ell(Q)}{|x-x_Q|^{n+1}}\,\eta(Q).
\end{align*}
Regarding $I_2(x)$, we write
$$
I_2(x)\leq \int_{y\in Q:|x-y|\leq\Psi(x)} \frac1{|x-y|^{n}} \,g_Q(y)\,d\eta(y).$$
Notice that for $y\in Q$, since $x\notin 2B_Q$, we have
$$C|x-y|\geq \ell(Q)\overset{\eqref{eqdefPsi1}}{\ge}\Psi(y) \geq \Psi(x) - |x-y|.$$
Thus, $|x-y|\geq c\,\Psi(x)$, and so
$$I_2(x)\leq \int_{c\Psi(x)\leq |x-y|\leq\Psi(x)} \frac1{|x-y|^{n}}\, g_Q(y)\,d\eta(y).
$$
Gathering the estimates for $I_1(x)$ and $I_2(x)$ we see that \rf{eqtech4} also holds in this case.


From \rf{eqtech4} we infer that
\begin{align*}
|\RR(|\nabla\vphi|\,\LL^{n+1})&(x) - \RR_{\Psi(x)}(g\,\eta)(x)|\\
& \lesssim \sum_{Q^{(\mu)}\in \sH^0}\frac{\Theta(Q)\,\ell(Q)^{n+1}}{\dist(x,Q)^{n+1}+
\ell(Q)^{n+1}} + 
\sum_{Q^{(\mu)}\in \sH^0} \chi_{2B_Q}(x)\, |\RR_{\Psi(x)}(g_Q\,\eta)(x)|\\
&\quad + \int_{c\Psi(x)\leq |x-y|\leq\Psi(x)} \frac1{|x-y|^{n}}\, g(y)\,d\eta(y)\\
& = S_1(x) + S_2(x) + S_3(x).
\end{align*}
We split
\begin{equation}\label{eqs123}
\int |\RR(|\nabla\vphi|\,\LL^{n+1})(x) - \RR_{\Psi(x)}(g\,\eta)(x)|^p\,d\eta(x)
\lesssim \sum_{i=1}^3 \int |S_i(x)|^p\,d\eta(x).
\end{equation}

We estimate the first summand by duality. Consider a function $h\in L^{p'}(\eta)$. Then, 
\begin{equation}\label{eqmul429**}
\int S_1(x)\,h(x)\,d\eta(x)= 
  \sum_{Q^{(\mu)}\in \sH^0} \eta(Q) \int\frac{\ell(Q)}{\dist(x,Q)^{n+1}+
\ell(Q)^{n+1}}\,h(x)\,d\eta(x).
\end{equation}
For each $Q^{(\mu)}\in\sH^0$,
 using the fact that $\eta(\lambda Q)\leq C\,\Theta(\sH)\,\ell(\lambda Q)^n$ for every $\lambda\geq 1$,
 we obtain 
$$\int  \frac{\ell(Q)}{\dist(x,Q)^{n+1}+
\ell(Q)^{n+1}}\,h(x)\,d\eta(x)\lesssim C\,\Theta(\sH)\,\inf_{y\in Q} \cM_\eta h(y).$$
Therefore, the right side of \rf{eqmul429**} does not exceed
\begin{align*}
C\Theta(\sH)\sum_{Q^{(\mu)}\in \sH^0} \eta(Q)\,\inf_{y\in Q} \cM_\eta h(y) & \lesssim
\Theta(\sH) \int_{H}\cM_\eta h(y)\,d\eta(y) \\ 
& \le \Theta(\sH)\,\eta(H)^{1/p}\,\|\cM_\eta h\|_{L^{p'}(\eta)}
\lesssim \Theta(\sH)\,\eta(H)^{1/p}\,\|h\|_{L^{p'}(\eta)}.
\end{align*}
So we deduce that
$$\int |S_1(x)|^p\,d\eta(x)\lesssim \sigma_p(\sH).
$$

Regarding the summand in \rf{eqs123} involving $S_2$, since the balls $2B_Q$ are disjoint, we have
$$\int |S_2(x)|^p\,d\eta(x) = 
\sum_{Q^{(\mu)}\in \sH^0} \int_{2B_Q} |\RR_{\Psi(x)}(g_Q\,\eta)(x)|^p\,d\eta(x) 
\leq \sum_{Q^{(\mu)}\in \sH^0}\| \RR_{\Psi(\cdot)}(g_Q\,\eta)\|_{L^p(\eta)}^p,$$
where $\RR_{\Psi(\cdot)}(g_Q\,\eta)(x) = \RR_{\Psi(x)}(g_Q\,\eta)(x)$.
Finally, to estimate the last summand in \rf{eqs123}, we take into account that 
$|S_3(x)|\lesssim \cM_{\Psi,n}(g\,\eta)(x),$ where $\cM_{\Psi,n}(g\,\eta)$ is the maximal operator from \eqref{eq:maximaloppsi}.
Hence,
$$\int |S_3(x)|^p\,d\eta(x)  \lesssim \int | \cM_{\Psi,n}(g\,\eta)|^p\,d\eta \lesssim \Theta(\sH)^p\,
\|g\|_{L^p(\eta)}^p \lesssim \sigma_p(\sH).$$

Gathering the estimates obtained for $S_1$, $S_2$, $S_3$, by \rf{eqs123} we get
$$\|\RR(|\nabla\vphi|\,\LL^{n+1})\|_{L^p(\eta)}^p \lesssim \sigma_p(\sH) +\sum_{Q^{(\mu)}\in \sH^0}
\|\RR_{\Psi(\cdot)}(g_Q\,\eta)\|_{L^p(\eta)}^p + \|\RR_{\Psi(\cdot)}(g\eta)\|_{L^p(\eta)}^p.$$
From \rf{e.compsup''}, we deduce that
\begin{align*}
\|\RR(|\nabla\vphi|\,\LL^{n+1})\|_{L^p(\eta)}^p & \lesssim \sigma_p(\sH) +\sum_{Q^{(\mu)}\in \sH^0}
\|\RR_{\Psi}(g_Q\,\eta)\|_{L^p(\eta)}^p + \|\RR_{\Psi}(g\eta)\|_{L^p(\eta)}^p\\
&\quad + \sum_{Q^{(\mu)}\in \sH^0}
\|\cM_{\Psi,n}(g_Q\,\eta)\|_{L^p(\eta)}^p  + \|\cM_{\Psi,n}(g\,\eta)\|_{L^p(\eta)}^p.
\end{align*}
Using now that, by Lemma \ref{lemH0}, 
 $\RR_{\Psi,\eta}$ is bounded from $L^p(\eta\rest_{H_0})$ to $L^p(\eta\rest_{\cS'_\eta})$ with 
$$\|\RR_\Psi\|_{L^p(\eta\rest_{H_0})\to L^p(\eta\rest_{\cS'_\eta})}\lesssim \Lambda_*^{\ve_n}\Theta(\sH),$$
and that the same happens for $\cM_{\Psi,n}$, we get
$$\|\RR(|\nabla\vphi|\,\LL^{n+1})\|_{L^p(\eta)}^p \lesssim \sigma_p(\sH) + \Lambda_*^{p\ve_n}\Theta(\sH)^p\,\|g\|_{L^p(\eta)}^p \lesssim \Lambda_*^{p\ve_n}\sigma_p(\sH).$$
\end{proof}


\vv

\subsection{Lower estimates for \texorpdfstring{$\RR\eta$}{R\_eta}}

\begin{lemma}\label{lemrieszeta}
Let $R\in\MDW$ be such that $R\in\Trc$ and let $V_4$ and $\eta$ be as in Section \ref{subsec9.5}.
Assume $\Lambda_*>0$ big enough and let $c_3$ be as in Lemma \ref{lemvar}.
Then we have
$$\int_{V_4} \big|(|\RR\eta(x)| - \frac{c_3}2\,\Theta(\HD_1))_+\big|^p\,d\eta(x) \gtrsim \Lambda_*^{-p'\ve_n}\sigma_p(\HD_1(e(R))),$$
for any $p\in (1,\infty)$, with the implicit constant depending on $p$.
\end{lemma}

\begin{proof}
For all $x\in \supp\eta$, using that $\|\nabla\vphi_R\|_\infty\leq 2A_0^{-3}\ell(R)^{-1}$, we obtain
\begin{align*}
\big|\vphi_R(x)\,\RR\eta(x) - \RR\nu(x)\big| &= \big|\vphi_R(x)\,\RR\eta(x) - \RR(\vphi_R\eta)(x)\big|\\
& = \left|\int \frac{(\vphi_R(x) - \vphi_R(y))\,(x-y)}{|x-y|^{n+1}}\,d\eta(y)\right|\\
& \leq 2A_0^3\int_{y\in \cS_\eta\setminus V_2} \frac1{\ell(R)\,|x-y|^{n-1}}\,d\eta(y)\\
&\quad +
2A_0^3\int_{y\in V_2:\vphi_R(y)\neq \vphi_R(x)} \frac1{\ell(R)\,|x-y|^{n-1}}\,d\eta(y).
\end{align*}
Recall that $\vphi_R$ equals $1$ on $V_3$ and vanishes out of $V_4$.
Then it is clear that the last integral on the right hand side above vanishes if $x\in V_3$ and
it does not exceed $C\,\Theta(R)$ if $x\in V_3^c$.
So in any case
$$\big|\vphi_R(x)\,\RR\eta(x) - \RR\nu(x)\big|\leq G(x) + C_5\,\Theta(R).$$

From the preceding estimate we infer that
\begin{align*}
|\vphi_R(x)\,\RR\eta(x)| - \frac{c_3}2\,\Theta(\HD_1) & \geq 
|\RR\nu(x)| - G(x) - C_5\,\Theta(R)
- \frac{c_3}2\,\Theta(\HD_1)\\
& \geq |\RR\nu(x)| - G(x) 
- c_3\,\Theta(\HD_1).
\end{align*}
Therefore, since $|(\;\cdot\;)_+|^p$ is non-decreasing,
\begin{align*}
\int_{V_4} \big|(|\RR\eta| - \frac{c_3}2\,\Theta(\HD_1))_+\big|^p\,d\eta & 
\geq \int \big|(|\vphi_R\,\RR\eta| - \frac{c_3}2\,\Theta(\HD_1))_+\big|^p\,d\eta\\
& \geq \int  \big|(|\RR\nu| - G 
- c_3\,\Theta(\HD_1))_+\big|^p\,d\eta\\
& \geq \int  \big|(|\RR\nu| - G 
- c_3\,\Theta(\sH))_+\big|^p\,d\nu.
\end{align*}
By Lemma \ref{lemvar}, %the right hand above is bounded from below by
$$\int  \big|(|\RR\nu| - G 
- c_3\,\Theta(\sH))_+\big|^p\,d\nu\geq c \Lambda_*^{-p'\ve_n}\sigma_p(\sH)\geq c\Lambda_*^{-p'\ve_n}\sigma_p(\HD_1(e(R))),$$
and so the lemma follows.
\end{proof}
\vv






