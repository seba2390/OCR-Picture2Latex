% ********************************************************************************************

\section{The proof of the Main Theorem \ref{propomain}} \label{sec9}

Recall that, for a cube $R\in\ttt$, $\tree(R)$ denotes the subfamily of the cubes from $\DD_\mu(R)$ which are not strictly contained in any cube
from $\End(R)$.
In this section we will prove the following result. 

\begin{lemma}\label{lemtreebeta}
For each $R\in\ttt$, the following holds:
\begin{align*}
\sum_{Q\in\tree(R)} \!\beta_{\mu,2}(2B_Q)^2\,\Theta(Q)\,\mu(Q) & \lesssim_{\Lambda,\delta_0}\!\sum_{Q\in\tree(R)}\!\|\Delta_Q\RR\mu\|_{L^2(\mu)}^2\\&\quad \;+ \!\sum_{Q\in\tree(R)\cap\HE}\! \EE(4Q) +\Theta(R)^2\,\mu(R).
\end{align*}
\end{lemma}

It is clear that together with Main Lemma \ref{mainlemma}, this yields Main
Theorem \ref{propomain}.


% ********************************************************************************************

\vv
\subsection{The approximating measure \texorpdfstring{$\eta$}{eta} on a subtree \texorpdfstring{$\wh\tree_0(R)$}{Tree\_0(R)}}\label{subsec:91}

To prove Lemma \ref{lemtreebeta} for a given cube $R_0\in\ttt$ (in place of $R$), we will consider a corona 
decomposition of $\tree(R_0)$ into subtrees by introducing appropriate new stopping conditions. 
In this section we will deal with the construction of each subtree and an associated AD-regular measure
which approximates $\mu$ in that subtree.
To this end we need some additional notation. 

First, for a cube 
$R\in\tree(R_0)\cap\DD_\mu^\PP$, we write $Q\in \BR(R)$ (which stands for ``big Riesz transform'') if $Q$ is a $\PP$-doubling maximal cube which does not belong to $\HD(R_0)\cup\LD(R_0)$ and satisfies
$$|\RR_\mu\chi_{2R\setminus 2Q}(x_Q)|\geq K\,\Theta(R),$$
where $K$ is some big constant to be fixed below, depending on $\Lambda$, $\delta_0$, and $M_0$. Also, for a cube $Q\in\tree(R_0)$,
we denote by $\wh \Ch(Q)$ the family of maximal cubes $P\in\DD_\mu(Q)\setminus\{Q\}$ that satisfy one of the following conditions:
\begin{itemize} 
\item $P\in\DD_\mu^\PP$, i.e.\ $P$ is $\PP$-doubling, or
\item $P\in\LD(R_0)$.
\end{itemize}
From Lemma \ref{lemdobpp}, it is immediate to check that if $Q$ is not contained in any cube from $\LD(R_0)$, then the cubes from $\wh\Ch(Q)$
cover $Q$, and also 
\begin{equation}\label{eqcompa492}
\ell(P)\approx_{\Lambda,\delta_0}\ell(Q)\quad \mbox{ for each $P\in\wh\Ch(P)$,} 
\end{equation}

Given $R\in\DD_\mu^\PP\in\tree(R_0)\setminus \End(R_0)$, we will construct a tree $\wh \tree_0(R)$ inductively, consisting just of $\PP$-doubling cubes and stopping cubes from $\LD(R_0)$. At the
same time we will construct an approximating AD-regular measure for this tree. We will do this by ``spreading''
the measure of the cubes from $\wh\tree_0(R)\cap \LD(R_0)$ among the other cubes from $\wh\tree_0(R)$. To this end, we will consider some coefficients $s(Q)$, $Q\in\DD_\mu^\PP\cap\wh\tree_0(R)$, that, in a sense, quantify the additional measure $\mu$ spreaded on $Q$
due to the presence of close cubes from $\LD(R_0)$. The algorithm is the following.

First we choose $R$ as the root of $\wh \tree_0(R)$, and we set $s(R)=0$. Next, suppose that $Q\in\wh \tree_0(R)$
(in particular, this implies that $Q\in\tree(R_0)$), and assume that we have not 
decided yet if the cubes from $\wh\Ch(Q)$ belong to $\wh \tree_0(R)$. 
First we decide that $Q\in\wh\sss(R)$ if one of the following conditions hold:
\begin{itemize}
\item[(i)] $Q\in\HD(R_0)\cup\LD(R_0)\cup\BR(R)$, or\vspace{1mm}

\item[(ii)] $s(Q)\geq \mu(Q)$ and (i) does not hold, or \vspace{1mm}

\item[(iii)] $\sum_{P\in\wh\Ch(Q)\cap \LD(R_0)}\mu(P) \geq \frac12\,\mu(Q)$ and neither (i) nor (ii) hold.
\end{itemize}
\vspace{1mm}

\noi If $Q\in\wh\sss(R)$, no descendants of $Q$ are allowed to belong to $\wh\tree_0(R)$.
Otherwise, all the cubes from $\wh\Ch(Q)$ are chosen to belong to $\wh\tree_0(R)$, and for each $P\in\wh\Ch(Q)$, we define
$$s(P)=-\mu(P) \quad \mbox{ if $P\in\LD(R_0)$,}$$
and, otherwise, we set 
\begin{equation}\label{eqdqaq14}
t(Q) = \sum_{S\in\wh\Ch(Q)\cap\LD(R_0)}\mu(S)\quad \mbox{ and }\quad
s(P) = \big(s(Q) + t(Q)\big)\, 
\frac{\mu(P)}{\mu(Q)-t(Q)}.
\end{equation}
Observe that
\begin{align*}
\sum_{P\in\wh\Ch(Q)}s(P) & = \sum_{P\in\wh\Ch(Q)\cap \LD(R_0)} s(P) + \sum_{P\in\wh\Ch(Q)\setminus \LD(R_0)}s(P) \\
& = -t(Q) + \big(s(Q) + t(Q)\big) = s(Q).
\end{align*}

By induction, the coefficients $s(\cdot)$ satisfy the following.
If $Q\in\wh \tree_0(R)$ and $\cI$ is some finite family of cubes from $\wh\tree_0(R)\cap\DD_\mu(Q)$
which cover $Q$ and are disjoint, then
\begin{equation}\label{eqaq89}
\sum_{P\in \cI} s(P) = s(Q).
\end{equation}
Further, $s(Q)\geq0$ for all $Q\in\wh\tree_0(R)\setminus \LD(R_0)$.

Now we are ready to define an approximating measure $\eta$ associated with $\wh\tree_0(R)$. 
First, we denote
$$\wh \sG(R) = R\,\setminus \bigcup_{Q\in\wh\sss(R)} Q,$$
and for each $Q\in\DD_\mu$ we let $D_Q$ be an $n$-dimensional disk passing through $x_Q$
with radius $\frac12 \,r(Q)$ (recall that $r(Q)$ is the radius of $B(Q)$).
In  case that $\mu(\wh\sG(R))=0$, we define
$$\eta = \sum_{Q\in\wh \sss(R)} \big(s(Q) + \mu(Q)\big)\,\frac{\HH^n\rest_{D_Q}}{\HH^n(D_Q)}.$$
Observe that $\eta(D_Q)=\mu(Q)+ s(Q)$ for all $Q\in\wh\sss(R)$ and, in particular $\eta(D_Q)=0$
if $Q\in\wh\sss(R)\cap\LD(R_0)$.

In case that $\mu(\wh\sG(R))\neq0$, we have to be a little more careful. For a given
$N\geq1$ we let $\wh\sss_N(R)$ be the family consisting of all the cubes from 
$\wh\sss(R)$ with side length larger that $A_0^{-N}\,\ell(R)$, and we let $\cI_N$ be the family of the cubes from $
\wh\tree_0(R)$ which have side length smaller than $A_0^{-N}\,\ell(R)$ and are maximal.
We denote 
\begin{equation}\label{eqdefetaN}
\eta_N = \sum_{Q\in\wh \sss_N(R)} \big(s(Q) + \mu(Q)\big)\,\frac{\HH^n\rest_{D_Q}}{\HH^n(D_Q)}+
\sum_{Q\in \cI_N(R)} \big(s(Q) + \mu(Q)\big)\,\frac{\mu\rest_Q}{\mu(Q)}
,
\end{equation}
and we let $\eta$ be a weak limit of $\eta_N$ as $N\to\infty$.
%In this case, it is immediate to check that \rf{eqetq5} holds too.

As in Section \ref{sec6.2*}, we use the following notation.
To each $Q\in\wh\tree_0(R)$ we associate another ``cube'' $Q^{(\eta)}$ defined as follows:
$$Q^{(\eta)}= (\sG(R)\cap Q)\cup \bigcup_{P\in\wh \sss(R):P\subset Q} D_P.$$
We let
$$\wh\tree_0^{(\eta)}(R) \equiv \wh\tree_0^{(\eta)}(R^{(\eta)}):= \{Q^{(\eta)}: Q\in\wh\tree_0(R)\}.$$
%Further, we consider a lattice $\DD_\eta$ associated with the measure $\eta$ which is made up of the cubes
%$Q^{(\eta)}$ with $Q\in \TT_{\Reg}(e'(R))$ and other cubes which are descendants of the cubes from $\Reg(e'(R))$.
%We assume that $\DD_\eta$ satisfies the first two properties of Lemma \ref{lemcubs} with the same parameters $A_0$ and $C_0$ as $\DD_\mu$. It is straightforward to check that $\DD_\eta$ can be constructed in this way.
For $S=Q^{(\eta)}\in \wh\tree_0^{(\eta)}(R)$ with $Q\in\wh\tree_0(R)$, we denote $Q=S^{(\mu)}$ and we write
$\ell(S):=\ell(Q)$.

Observe now that, from \rf{eqaq89} and the definition of $\eta$, we have the key property
\begin{equation}\label{eqetq5}
\eta(Q^{(\eta)})= \mu(Q) + s(Q)\quad \mbox{ for all $Q\in\wh\tree_0(R)$}.
\end{equation}
So $s(Q)$ is the measure added to $\mu(Q)$ to obtain $\eta(Q^{(\eta)})$.

\vv

\begin{lemma}\label{lem9.2}
The measure $\Theta(R_0)^{-1}\eta$ is AD-regular (with a constant depending on $\Lambda$ and $\delta_0$), and 
$\eta(Q^{(\eta)})=0$ for all $Q\in\LD(R_0)\cap\wh\tree_0(R)$.
\end{lemma}

\begin{proof}
The fact that $\eta(Q^{(\eta)})=0$ for all $Q\in\LD(R_0)\cap\wh\tree_0(R)$ follows by construction and has already been mentioned above.
To prove the AD-regularity of $\eta$, by standard arguments, it is enough to show that
$$\eta(Q^{(\eta)})\approx_{\Lambda,\delta_0} \Theta(R_0)\,\ell(Q)^n\quad\mbox{ for all $Q\in\wh\tree_0(R)\setminus
\LD(R_0)$,}$$
taking into account \rf{eqcompa492}.
Given such a cube $Q$, the fact that $Q\not\in\LD(R_0)$ ensures that
$$\eta(Q^{(\eta)})\geq \mu(Q)\gtrsim\delta_0\,\Theta(R_0)\,\ell(Q)^n.
$$
To show the converse estimate we can assume $Q\neq R$. By
the condition (ii) in the definition of $\wh \sss(\wh Q)$, where $\wh Q$ is the first ancestor of $Q$ in $\wh \tree_0(R)$
(i.e., $\wh Q$ is the smallest cube from $\wh\tree_0(R)$ that strictly contains $Q$), we have
$$s(\wh Q)\leq \mu(\wh Q).$$
Also, by (iii) (which does not hold for $\wh Q$), the coefficient $t(\wh Q)$ in \rf{eqdqaq14} satisfies
$$t(\wh Q) = \sum_{P\in\wh\Ch(\wh Q)\cap \LD(R_0)}\mu(P) < \frac12\,\mu(\wh Q).$$
Therefore,
\begin{equation}\label{eqaq745}
s(Q) = \big(s(\wh Q) + t(\wh Q)\big)\, 
\frac{\mu(Q)}{\mu(\wh Q)-t(\wh Q)} \leq 2 \big(\mu(\wh Q) + \frac12\,\mu(\wh Q)\big)\, 
\frac{\mu(Q)}{\mu(\wh Q)} = 3\,\mu(Q),
\end{equation}
and so
$$\eta(Q^{(\eta)}) = s(Q) + \mu(Q)\leq 4\,\mu(Q)\leq 4\,\mu(\wh Q)\lesssim \Lambda\, \Theta(R_0)\,\ell(\wh Q)^n
\approx_{\Lambda,\delta_0} \Theta(R_0)\,\ell(Q)^n,
$$
taking into account that $\wh{Q}\not \in\HD(R_0)$, by (i).
\end{proof}
\vv

\begin{rem}
For the record, notice that from \rf{eqaq745} it follows that, for all $Q\in\wh\tree_0(R)$, either
\begin{equation}\label{eq:aQmuQ}
0\leq s(Q)\leq 3\,\mu(Q),
\end{equation}
or 
$$s(Q)=-\mu(Q).$$
The latter case happens if and only if $Q\in\LD(R_0)$.
\end{rem}
\vv

\begin{rem}
Consider the measure defined by
$$\eta' = \sum_{Q\in\wh \sss(R)\setminus\LD(R_0)} \mu(Q)\,\frac{\HH^n\rest_{D_Q}}{\HH^n(D_Q)} + \mu\rest_{\wh\sG(R)}.$$
This measure is mutually absolutely continuous with $\eta$. Further, since the
coefficients $s(Q)$, with $Q\in\wh\tree_0(R)$, are uniformly bounded (by the previous remark), it turns out that
$$\eta'= \rho\,\eta,$$
for some function $\rho\in L^\infty(\eta)$ satisfying $\rho\approx1$.
Consequently, by Lemma \ref{lem9.2}, $\eta'$ is also AD-regular.
\end{rem}



\vv

For a family of cubes $\cI\subset \tree(R_0)$, we denote 
$$\wh\Ch(\cI)=\bigcup_{Q\in \cI} \wh\Ch(Q).$$
For $Q\in\wh\tree_0(R)$, we write $Q\in (i)_R$, if $Q\in\wh\sss(R)$ and the condition (i) in the definition of 
$\wh\sss(R)$ holds for $Q$, and analogously regarding the notations $Q\in (ii)_R$ and $Q\in (iii)_R$.


\begin{lemma}\label{lem9.5*}
The following holds:
$$\sum_{Q\in\wh\sss(R)\cap(\LD(R_0)\cup\HD(R_0)\cup\BR(R))}\mu(Q) 
+\sum_{Q\in\wh\Ch((iii)_R)\cap \LD(R_0)}\mu(Q) + \mu(\wh\sG(R))
\approx \mu(R).$$
\end{lemma}

\begin{proof}
It is clear that the left hand side above is bounded by $\mu(R)$.
For the converse estimate,  we write
\begin{equation}\label{eqsplit63}
\mu(R) = \sum_{Q\in\wh\sss(R)}\mu(Q) + \mu(\wh\sG(R)) = 
\sum_{Q\in (i)_R} \mu(Q) + \sum_{Q\in (ii)_R} \mu(Q) +\sum_{Q\in (iii)_R} \mu(Q)  
 + \mu(\wh\sG(R)).
 \end{equation}
 By construction,
\begin{equation}\label{eqsplit64}
\sum_{Q\in (i)_R} \mu(Q) =\sum_{Q\in \wh\sss(R)\cap(
\HD(R_0)\cup\LD(R_0)\cup\BR(R))}\mu(Q).
\end{equation}
Also, if $Q\in (iii)_R$, then
$$\mu(Q)\leq 2\sum_{P\in\wh\Ch(Q)\cap \LD(R_0)}\mu(P),$$
and thus
\begin{equation}\label{eqsplit65}
\sum_{Q\in (iii)_R} \mu(Q)\leq 2 \sum_{P\in\wh\Ch(\wh\sss(R))\cap \LD(R_0)}\mu(P).
\end{equation}

On the other hand, if $Q\in (ii)_R$, then $0\leq \mu(Q)\leq s(Q)$, and so
$$\sum_{Q\in (ii)_R} \mu(Q)\leq \sum_{Q\in \wh\sss(Q):s(Q)\geq 0} s(Q) = \sum_{Q\in \wh\sss(R)\setminus \LD(R_0)} s(Q).$$
For a given $N\geq 1$, consider the families $\wh\sss_N(R)$ and $\cI_N$ defined just above \rf{eqdefetaN}.
Notice that $\cJ_N:= \wh\sss_N(R)\cup\cI_N$ is a finite family of cubes which cover $R$, and thus, 
from the property \rf{eqaq89},
it follows that
$$0 = s(R) = \sum_{Q\in \cJ_N}s(Q) = \sum_{Q\in \cJ_N\cap \LD(R_0)} s(Q) + 
\sum_{Q\in \cJ_N\setminus \LD(R_0)} s(Q). $$
Since $s(Q) = -\mu(Q)$ for all $Q\in \cJ_N\cap \LD(R_0)$, we deduce
$$\sum_{Q\in \cJ_N\setminus \LD(R_0)} s(Q) = \sum_{Q\in \cJ_N\cap \LD(R_0)} \mu(Q).$$ 
Letting $N\to\infty$ and taking into account that $s(Q)\geq 0$ for all
$Q\in \cJ_N\setminus \LD(R_0)$, we get
$$\sum_{Q\in \wh\sss(R)\setminus \LD(R_0)} s(Q)\leq \sum_{Q\in \wh\sss(R)\cap \LD(R_0)} \mu(Q),$$
and thus
\begin{equation}\label{eqsplit66}
\sum_{Q\in (ii)_R} \mu(Q)\leq \sum_{Q\in \wh\sss(R)\cap \LD(R_0)} \mu(Q).
\end{equation}

The lemma follows from the splitting \rf{eqsplit63} and the inequalities \rf{eqsplit64}, \rf{eqsplit65}, \rf{eqsplit66}.
\end{proof}


\vv

\begin{lemma}\label{lemriesz*eta}
The operator $\RR_\eta$ is bounded in $L^2(\eta)$, with
$$\|\RR_\eta\|_{L^2(\eta)\to L^2(\eta)}\lesssim_{\Lambda,\delta_0,K}\Theta(R).$$
\end{lemma}

\begin{proof}
To prove this lemma we will use the suppressed kernel $K_\Phi$ introduced in Section \ref{sec6.1},
with the following $1$-Lipchitz function:
$$\Phi(x) =\inf_{Q\in\wh\tree_0(R)} \big(\ell(Q) + \dist(x,Q)\big).$$
We will prove first that $\RR_{\Phi,\mu\rest_{2R}}$ is bounded in $L^2(\mu\rest_{2R})$ by applying Theorem
\ref{teontv}, and later on we will 
show that $\RR_\eta$ is bounded in $L^2(\eta)$ by approximation. 

In order to apply Theorem \ref{teontv}, we will show that
\begin{itemize}
\item[(a)] $\mu(B(x,r)\cap 2R)\leq C\,\Theta(R)\,r^n$ for all $r\geq \Phi(x)$, and
\item[(b)] $\sup_{r>\Phi(x)}\big|\RR_r(\chi_{2R}\mu)(x)\big|\leq C\Theta(R)$,
\end{itemize}
with $C$ possibly depending on $\Lambda$, $\delta_0$, and $K$. Once these conditions are proven, then Theorem \ref{teontv} applied to the measure $\Theta(R)^{-1}\,\mu\rest_{2R}$ ensures that 
\begin{equation}\label{eqphi894}
\|\RR_{\Phi,\mu\rest_{2R}}\|_{L^2(\mu\rest_{2R})\to L^2(\mu\rest_{2R})}\lesssim_{\Lambda,\delta_0,K}\Theta(R).
\end{equation}

The proof of (a) is quite similar to the proof of Lemma \ref{lem6.77}. However, we repeat
here the arguments for the reader's convenience.
In the case $r>\ell(R)/10$ we just use that
$$\mu(B(x,r)\cap 2R)\leq \mu(2R)\lesssim \Theta(R)\,\ell(R)^n\lesssim \Theta(R)\,r^n.$$
So we may assume that $\Phi(x)<r\leq \ell(R)/10$.
By the definition of $\Phi(x)$, there exists $Q\in\wh\tree_0(R)$ such that
$$\ell(Q) + \dist(x,Q)\leq r.$$
Therefore, $B_Q\subset B(x,4r)$ and so there exists an ancestor $Q'\supset Q$ which belongs to $\wh\tree_0(R)$ such that $B(x,r)\subset  2B_{Q'}$, with $\ell(Q')\approx r$.  
Then,
$$\mu(B(x,r)\cap 2R)\leq \mu(2B_{Q'}) \lesssim \Lambda\,\Theta(R_0)\,\ell(Q')^n\approx_{\Lambda,\delta_0} \Theta(R)\,r^n,$$
as wished.

Let us turn our attention to the property (b).
In the case $r>\ell(R)/10$ we have
$$\big|\RR_r(\chi_{2R}\mu)(x)\big|\leq \frac{\mu(2R)}{r^n}\lesssim \Theta(R).$$
In the case $\Phi(x)<r\leq \ell(R)/10$ we consider the same cube $Q'\in\wh\tree_0(R)$ as above, 
which satisfies $B(x,r)\subset  2B_{Q'}$ and $\ell(Q')\approx r$. Further, by replacing
$Q'$ by the first ancestor in $\wh\tree_0(R)$ if necessary, we may assume that
$$|\RR_\mu\chi_{2R\setminus 2Q'}(x_{Q'})|\lesssim K\,\Theta(R).$$
Since $|x-x_{Q'}|\lesssim\ell(Q')$, by standard arguments which use the fact that $K_\Phi$ is a Calder\'on-Zygmund kernel (see \rf{eqkafi1}
and \rf{eqkafi2}),
it follows that
$$\big|\RR_\mu\chi_{2R\setminus 2Q'}(x_{Q'}) - \RR_r(\chi_{2R}\mu)(x)\big|\lesssim \PP(Q')\lesssim
\Lambda\,\Theta(R_0)\approx_{\Lambda,\delta_0}\Theta(R).$$
Thus,
$$\big|\RR_r(\chi_{2R}\mu)(x)\big|\leq 
\big|\RR_\mu\chi_{2R\setminus 2Q'}(x_{Q'})\big| + 
\big|\RR_\mu\chi_{2R\setminus 2Q'}(x_{Q'}) - \RR_r(\chi_{2R}\mu)(x)\big|\lesssim_{\Lambda,\delta_0,K}\Theta(R).$$
So both (a) and (b) hold, and then \rf{eqphi894} follows.
\vv

Next we deal with the $L^2(\eta)$ boundedness of $\RR_\eta$. 
First notice that  $\RR_{\mu\rest_{\wh\sG(R)}}$ is bounded in $L^2(\mu\rest_{\wh\sG(R)})$ with norm
at most $C\Theta(R)$ because $\Phi(x)=0$ on $\wh\sG(R)$. Since $\eta\rest_{\wh\sG(R)} = \rho\,\mu\rest_{\wh\sG(R)}$ for some function $\rho\approx1$, $\RR_{\eta\rest_{\wh\sG(R)}}$ is also bounded in $L^2(\eta\rest_{\wh\sG(R)})$ with norm bounded by $C\Theta(R)$.
So it suffices to show that $\RR_{\eta\rest_{\wh\sG(R)^c}}$ is bounded in $L^2(\eta\rest_{\wh\sG(R)^c})$.
This follows from the fact that if $\alpha$ and $\beta$ are Radon measures with polynomial growth of degree $n$ such that $\RR_\alpha$ is bounded in $L^2(\alpha)$ and 
$\RR_\beta$ is bounded in $L^2(\beta)$, then $\RR_{\alpha+\beta}$ is bounded in $L^2(\alpha+\beta)$,
and then choosing $\alpha= \Theta(R)^{-1}\mu\rest_{\wh\sG(R)}$ and $\beta= \Theta(R)^{-1}\mu\rest_{\wh\sG(R)^c}$.
See for example Proposition 3.1 from \cite{NToV2}.
 
It remains to show that $\RR_{\eta\rest_{\wh\sG(R)^c}}$ is bounded in $L^2(\eta\rest_{\wh\sG(R)^c})$
with norm bounded above by $C\,\Theta(R)$.
Notice first that, by \rf{eqfk490}, there exists some constant $b>0$ depending at most on 
$C_0,A_0,n$ such that
$$\mu\bigl(\{x\in Q:\dist(x,\supp\mu\setminus Q)\leq b\,\ell(Q)\}\bigr) \leq\frac12\,\mu(Q).$$
We denote
\begin{equation}\label{eqsepbb*}
Q^{(0)}= \{x\in Q:\dist(x,\supp\mu\setminus Q)> b\,\ell(Q)\},
\end{equation}
so that $\mu(Q^{(0)})\geq \frac12\,\mu(Q)$.

We have to show that
\begin{equation}\label{eqbeta8430}
\|\RR (g\,\eta\rest_{\wh\sG(R)^c})\|_{L^2(\eta\rest_{\wh\sG(R)^c})} \lesssim_{\Lambda,\delta_0,K}\Theta(R)\, \|g\|_{L^2(\eta)}
\end{equation}
for any given $g\in L^2(\eta\rest_{\wh\sG(R)^c})$, with $\RR (g\,\eta\rest_{\wh\sG(R)^c})$ 
understood in the principal value sense. To this end, we take
the function $f\in L^2(\mu\rest_{2R})$ defined as follows:
\begin{equation}\label{deff99}
f = \sum_{Q\in\wh\sss(R)} \int_{D_Q}\! g\,d\eta \,\,\frac{\chi_{Q^{(0)}}}{\mu(Q^{(0)})}.
\end{equation}
We also consider the signed measures
$$\alpha = f\,\mu,\qquad \beta = g\,\eta,$$
so that $\alpha(Q) = \alpha(Q^{(0)}) = \beta(D_Q)$ for all $Q\in\wh\sss(R)$.

As a preliminary step to obtain \rf{eqbeta8430}, we will show first 
\begin{equation}\label{eqbeta843}
\|\RR_\Phi \beta\|_{L^2(\eta\rest_{\wh\sG(R)^c})} \lesssim_{\Lambda,\delta_0,K}\Theta(R)\, \|g\|_{L^2(\eta)}.
\end{equation}
For that purpose, first we will estimate the term  $|\RR_\Phi \alpha(x) -\RR_\Phi\beta(y)|$,
with $x\in Q^{(0)}$, $y\in D_Q$, for $Q\in\wh\sss(R)$, in terms of the coefficients
$$\PP_\alpha(Q) := \sum_{P\in\DD_\mu:P\supset Q} \frac{\ell(Q)}{\ell(P)^{n+1}} \,|\alpha|(2B_P)
\quad \mbox{ and }\quad
\QQ_\alpha(Q) := \sum_{P\in\wh\sss(R)} \frac{\ell(P)}{D(P,Q)^{n+1}}\,|\alpha|(P),
$$
and
$$\PP_\beta(Q) := \sum_{P\in\DD_\mu:P\supset Q} \frac{\ell(Q)}{\ell(P)^{n+1}} \,|\beta|(2B_P)
\quad \mbox{ and }\quad
\QQ_\beta(Q) := \sum_{P\in\wh\sss(R)} \frac{\ell(P)}{D(P,Q)^{n+1}}\,|\beta|(D_P).
$$
We claim that
\begin{equation}\label{eqclaim7284}
|\RR_\Phi \alpha(x) -\RR_\Phi\beta(y)| \lesssim \PP_\alpha(Q) + \QQ_\alpha(Q)+\PP_\beta(Q) + \QQ_\beta(Q)
\end{equation}
for all $x\in Q^{(0)}$, $y\in D_Q$, with $Q\in\wh\sss(R)$.
The arguments to prove this are quite similar to the ones in Lemma \ref{lemaprox1}, but we 
will show the details for completeness.
By the triangle inequality, for $x$, $y$ and $Q$ as above, we have
\begin{align*}
\big|\RR_\Phi \alpha(x) - \RR_\Phi \beta(y)\big| & \leq 
\big|\RR_\Phi \alpha(x) - \RR_\Phi \alpha(x_Q)\big|\\
&\quad + 
\big|\RR_\Phi \alpha(x_Q) - \RR_\Phi \beta(x_Q)\big| 
+ \big|\RR_\Phi \beta(x_Q) - \RR_\Phi \beta(y)\big| \\
& =: I_1 + I_2 + I_3.
\end{align*}

First we estimate $I_1$ using the properties of the kernel $K_\Phi$ in \rf{eqkafi1}
and \rf{eqkafi2}, and taking into account that for $x\in Q^{(0)}$ (and thus for $x_Q$)
$\Phi(x)\approx_b \ell(Q)$, because of the separation condition in the definition of
$Q^{(0)}$ in \rf{eqsepbb*}. Then we get
\begin{align*}
|I_1| &\leq \int |K_\Phi(x,z) -  K_\Phi(x_Q,z)|\,d|\alpha|(z)\\
& = \bigg(\int_{2B_Q} +
\sum_{P\in\DD_\mu:P\supset Q} \int_{2B_{\wh P}\setminus 2B_P}\bigg) |K_\Phi(x,z) -  K_\Phi(x_Q,z)|\,d|\alpha|(z)\\
&\lesssim \sum_{P\in\DD_\mu:P\supset Q} \frac{\ell(Q)}{\ell(P)^{n+1}} \,|\alpha|(2B_{\wh P}) \lesssim \PP_\alpha(Q),
\end{align*}
where above we denoted by $\wh P$ the parent of $P$.
The same estimate holds for the term $I_3$ (with $\alpha$ replaced by $\beta$), using that $\Phi(x)\approx \ell(Q)$ for all $x\in D_Q$,
since $D_Q\subset\frac12B(Q)$ and $B(Q)\cap\supp\mu\subset Q$. So
$$|I_3| \lesssim \PP_\beta(Q).$$
Finally we deal with the term $I_2$.
Since $\alpha(P^{(0)})=\beta(D_P)$ for all $P\in\wh\sss(R)$, we have
\begin{align*}
I_2 & \leq \sum_{P\in\wh\sss(R)} \left| \int K_\Phi(x_Q-z)\,d\big(\alpha\rest_{P^{(0)}} - 
\beta\rest_{D_P} \big)(z)\right|\\
& \leq \sum_{P\in\wh\sss(R)}  \int |K_\Phi(x_Q-z)-K_\Phi(x_Q-x_P)|\,d\big(|\alpha|\rest_{P^{(0)}} + 
|\beta|\rest_{D_P} \big)(z)
\end{align*}
From the separation condition in \rf{eqsepbb*} and the fact that $D_P\subset\frac12B(P)$, we infer that, for $P,Q\in\wh\sss(R)$ with $P\neq Q$ and $z\in P^{(0)}\cup D_P$,
$$|x_Q-z|\approx |x_Q - x_P|\gtrsim \ell(Q) + \ell(P).$$
Hence, in the case $P\neq Q$,
$$\int |K_\Phi(x_Q-z)-K_\Phi(x_Q-x_P)|\,d\big(|\alpha|\rest_{P^{(0)}} + 
|\beta|\rest_{D_P} \big)(z)\lesssim \frac{\ell(P)}{D(P,Q)^{n+1}}\,\big(|\alpha|(P^{(0)}) + |\beta|(D_P)\big).$$
The same inequality holds in the case $P=Q$ using \rf{eqkafi1} and the fact that $\Phi(x_Q) 
\approx\ell(Q)$.
So we deduce that
$$I_2\lesssim \QQ_\alpha(Q) + \QQ_\beta(Q).$$
Gathering the estimates obtained for $I_1$, $I_2$, $I_3$, the claim \rf{eqclaim7284} follows.

Now we are ready to show \rf{eqbeta843}. By the claim just proven and using that
$\eta(B_Q)\lesssim\mu(Q)$, we obtain
\begin{align}\label{eqdhvn43}
\|\RR_\Phi \beta\|_{L^2(\eta\rest_{\wh\sG(R)^c})}^2 & =
\sum_{Q\in\wh\sss(R)} \int_{B_Q} |\RR_\Phi \beta|^2\,d\eta\\
& \lesssim \sum_{Q\in\wh\sss(R)} \inf_{x\in Q^{(0)}}|\RR_\Phi \alpha(x)|^2\,\mu(Q) + 
\sum_{Q\in\wh\sss(R)}\big(\PP_\alpha(Q)^2 + \QQ_\alpha(Q)\big)^2\,\mu(Q) \nonumber\\
 &\quad+ 
\sum_{Q\in\wh\sss(R)}\big(\PP_\beta(Q)^2 + \QQ_\beta(Q)\big)^2\,\eta(D_Q).\nonumber
\end{align}
Since $\RR_{\Phi,\mu\rest_{2R}}$ is bounded in $L^2(\mu\rest_{2R})$ with norm bounded by 
$C(\Lambda,\delta_0,K)\,\Theta(R)$, we infer that
\begin{align*}
\sum_{Q\in\wh\sss(R)} \inf_{x\in Q^{(0)}}|\RR_\Phi \alpha(x)|^2\,\mu(Q)
\leq \int_R |\RR_\Phi \alpha|^2d\mu \leq \|\RR_\Phi (f\mu)\|^2_{L^2(\mu\rest_{2R})}
\lesssim_{\Lambda,\delta_0,K} \Theta(R)^2\,\|f\|_{L^2(\mu)}^2.
\end{align*}

To estimate the second sum on the right hand side of \rf{eqdhvn43} we use the fact
that
$$\sum_{Q\in\wh\sss(R)}\QQ_\alpha(Q)^2\,\mu(Q) \lesssim 
\sum_{Q\in\wh\sss(R)}\PP_\alpha(Q)^2\,\mu(Q).$$
This follows by the same argument as in Lemma \ref{lemregpq}, with $p=2$. Indeed, in that lemma
one does not use any specific property of the measure $\mu$ or the family $\Reg$, apart from the
fact the cubes from $\Reg$ are pairwise disjoint. So the lemma applies to $\wh\sss(R)$ too.
Observe also that, for every $x\in Q$, $2B_P\subset B(x,2\ell(P))\subset CB_P$, for some $C>1$, and thus
\begin{align*}
\PP_\alpha(Q) &\leq \sum_{P\in\DD_\mu:P\supset Q} \frac{\ell(Q)}{\ell(P)^{n+1}} \,
\frac{|\alpha|(B(x,2\ell(P)))}{\mu(B(x,2\ell(P)))}\,\mu(CB_P) \\
&\lesssim \PP(Q)\,\cM_\mu f(x)\lesssim_{\Lambda,\delta_0}\Theta(R)\,\cM_\eta f(x).
\end{align*}
Consequently,
\begin{align*}
\sum_{Q\in\wh\sss(R)}\big(\PP_\alpha(Q)^2 + \QQ_\alpha(Q)\big)^2\,\mu(Q) &
\lesssim \sum_{Q\in\wh\sss(R)}\PP_\alpha(Q)^2\,\mu(Q) \\
&\lesssim_{\Lambda,\delta_0}\Theta(R)^2
\int |\cM_\mu f|^2\,d\mu\lesssim_{\Lambda,\delta_0}\Theta(R)^2\,\|f\|_{L^2(\mu)}^2.
\end{align*}

The last sum on the right hand side of \rf{eqdhvn43} is estimated similarly. Indeed, 
by Lemma \ref{lemregpq} we also have
$$\sum_{Q\in\wh\sss(R)}\QQ_\beta(Q)^2\,\eta(D_Q) \lesssim 
\sum_{Q\in\wh\sss(R)}\PP_\beta(Q)^2\,\eta(D_Q),$$
and as above,
$$\PP_\beta(Q) \lesssim_{\Lambda,\delta_0}\Theta(R)\,\cM_\eta g(x).
$$
Then it follows that
$$\sum_{Q\in\wh\sss(R)}\big(\PP_\beta(Q)^2 + \QQ_\beta(Q)\big)^2\,\mu(Q) \lesssim_{\Lambda,\delta_0}\Theta(R)^2\,\|g\|_{L^2(\eta)}^2.$$
By \rf{eqdhvn43} and the preceding esitmates, to complete the proof of \rf{eqbeta843} it just remains
to notice that, by the definition of $f$ in \rf{deff99} and Cauchy-Schwarz, $\|f\|_{L^2(\mu)}\lesssim
\|g\|_{L^2(\eta)}$.

In order to prove \rf{eqbeta8430}, we denote
$$\RR^{r(Q)/4}\beta(x) = \RR\beta(x) - \RR_{r(Q)/4}\beta(x),$$
and we split
\begin{align}\label{eqalgud77}
\|\RR (g\,\eta\rest_{\wh\sG(R)^c})\|_{L^2(\eta\rest_{\wh\sG(R)^c})}^2 
 & = \sum_{Q\in\wh\sss(R)} \int_{D_Q} |\RR \beta|^2\,d\eta\\
& \lesssim \!\sum_{Q\in\wh\sss(R)}\! \bigg(\int_{D_Q} |\RR^{r(Q)/4}\beta|^2\,d\eta +\int_{D_Q} 
|\RR_{r(Q)/4}\beta - \RR_\Phi \beta|^2\,d\eta\bigg)\nonumber\\
& \quad + \|\RR_\Phi \beta\|_{L^2(\eta\rest_{\wh\sG(R)^c})}^2.\nonumber
\end{align}
Using the fact that $\RR_{\eta\rest_{D_Q}}$ is bounded in $L^2(\eta\rest_{D_Q})$ with norm 
comparable to $\eta(B_Q)/r(Q^n)$ (because $D_Q$
is an $n$-dimensional disk), we deduce that
$$\int_{D_Q} |\RR^{r(Q)/4}\beta|^2\,d\eta = \int_{D_Q}
|\RR^{r(Q)/4}(\chi_{D_Q}g\,\eta)|^2\,d\eta \lesssim_{\Lambda,\delta_0} \Theta(R)^2\,\|\chi_{D_Q}g\|_{
L^2(\eta)}^2.$$
Regarding the second integral in \rf{eqalgud77}, 
observe that, by \rf{e.compsup''}, for all $x\in D_Q$ with
$Q\in \wh\sss(R)$,
$$\bigl|\RR_{r(Q)/4}\beta(x) - \RR_{\Phi}\beta(x)\bigr|\lesssim  \sup_{r> \Phi(x)}\frac{|\beta|(B(x,r))}{r^n} \lesssim_{\Lambda,\delta_0} \Theta(R)\,\cM_\mu g(x).$$
Then, by the last estimates and \rf{eqbeta843}, we get
\begin{align*}%\label{eqalgud77}
\|\RR_\Phi (g\,\eta\rest_{\wh\sG(R)^c})\|_{L^2(\eta\rest_{\wh\sG(R)^c})}^2 
& \lesssim_{\Lambda,\delta_0}\Theta(R)^2 \sum_{Q\in\wh\sss(R)} \int_{D_Q} \big(|g|^2 + |\cM_\eta g|^2\big)\,d\eta + \|\RR_\Phi \beta\|_{L^2(\eta\rest_{\wh\sG(R)^c})}^2\\
&\lesssim_{\Lambda,\delta_0,K}\Theta(R)^2\,
\|g\|_{L^2(\eta)}^2,
\end{align*}
which concludes the proof of the lemma.
\end{proof}

\vv


Next, for each $R\in\DD_\mu^\PP\setminus\End(R_0)$, we define the family $\wh\End(R)$ and $\wh\tree(R)$, which can be considered as an enlarged version of $\wh\tree_0(R)$.
First we define 
\begin{equation*}
\wh{\Stop_*}(R) = (i)_R\cup (ii)_R\cup \wh\Ch((iii)_R).
\end{equation*}
Let $\wh\End(R)$ be the family of maximal $\PP$-doubling cubes which are contained in the cubes from $\wh\Stop_*(R)$.
%\begin{itemize}
%\item the maximal $\PP$-doubling cubes which are contained in the cubes $Q\in (i)_R\cup (ii)_R$, and
%\item the maximal $\PP$-doubling cubes which are contained in any cube $P\in\wh\Ch(Q)$ for some $Q\in (iii)_R$.
%\end{itemize}
Notice that $R\not\in\wh\End(R)$. Further, the cubes from $\wh\sss(R)\cap(\HD(R_0)\cup\BR(R)\cup(ii)_R)$
belong to $\wh\End(R)$ because they are $\PP$-doubling.
Then we let $\wh\tree(R)$ be the family of cubes from $\DD_\mu$ which are contained in $R$ and are not strictly contained in any cube from $\wh\End(R)$. Observe that we do not ask the cubes from 
$\wh\tree(R)$ to be $\PP$-doubling. Similarly, we define $\wh\tree_*(R)$ as the family of cubes from $\DD_\mu$ which are contained in $R$ and are not strictly contained in any cube from $\wh\sss_*(R)$.

\vv
\subsection{Estimating the \texorpdfstring{$\beta$}{beta} numbers on \texorpdfstring{$\wh\tree(R)$}{Tree(R)}}

Our goal is now to prove the following estimate.
\begin{lemma}\label{lembetas99}
	For each $R\in\DD_\mu^\PP\setminus\End(R_0)$, we have
	$$\sum_{Q\in\wh \tree(R)}\beta_{\mu,2}(2B_Q)^2\,\mu(Q)\lesssim_{\Lambda,\delta_0,K}\Theta(R_0)\,\mu(R).$$
\end{lemma}
We split the proof into several steps. Fix $R\in\DD_\mu^\PP\setminus\End(R_0)$. First we deal with cubes in $\wh\tree(R)\setminus\wh\tree_*(R)$.
\begin{lemma}
	We have
	\begin{equation*}
	\sum_{Q\in\wh\tree(R)\setminus\wh\tree_*(R)}\beta_{\mu,2}(2B_Q)^2\mu(Q)\lesssim_{\Lambda}\Theta(R_0)\mu(R).
	\end{equation*}
\end{lemma}
\begin{proof}
	We use the trivial bound $\beta_{\mu,2}(2B_Q)^2\lesssim\theta_\mu(2B_Q)$ and Lemma \ref{lemdobpp} to get
	\begin{align*}
	\sum_{Q\in\wh\tree(R)\setminus\wh\tree_*(R)}\beta_{\mu,2}(2B_Q)^2&\mu(Q)
	\lesssim\sum_{Q\in\wh\tree(R)\setminus\wh\tree_*(R)}\theta_{\mu}(2B_Q)\mu(Q)\\
	&=\sum_{P\in\wh\Stop_*(R)}\sum_{Q\in\wh\tree(R):\, Q\subset P }\theta_{\mu}(2B_Q)\mu(Q)\\
	&= \sum_{m\ge 0} \sum_{P\in\wh\Stop_*(R)}\sum_{\substack{Q\in\wh\tree(R)\\ Q\subset P,\, \ell(Q)=A_0^{-m}\ell(P) }}\theta_{\mu}(2B_Q)\mu(Q)\\
	&\le \sum_{m\ge 0} \sum_{P\in\wh\Stop_*(R)}\sum_{\substack{Q\in\wh\tree(R)\\ Q\subset P,\, \ell(Q)=A_0^{-m}\ell(P) }}A_0^{-m/2}\PP(P)\mu(Q)\\
	&\le \sum_{m\ge 0} \sum_{P\in\wh\Stop_*(R)}A_0^{-m/2}\PP(P)\mu(P)\approx  \sum_{P\in\wh\Stop_*(R)}\PP(P)\mu(P).
	\end{align*}
	Recall that for $Q\in\tree(R_0)$ we have $\PP(Q)\lesssim_{\Lambda}\Theta(R_0)$, and so
	\begin{equation*}
	\sum_{P\in\wh\Stop_*(R)}\PP(P)\mu(P)\lesssim_{\Lambda}\Theta(R)\sum_{P\in\wh\Stop_*(R)}\mu(P) \le \Theta(R_0)\mu(R).
	\end{equation*}
\end{proof}
It remains to prove
\begin{equation*}
\sum_{Q\in\wh \tree_*(R)}\beta_{\mu,2}(2B_Q)^2\,\mu(Q)\lesssim_{\Lambda,\delta_0,K}\Theta(R_0)\,\mu(R).
\end{equation*}
Consider the set
\begin{equation*}
\Gamma = \supp\eta = \wh\GG(R) \cup \bigcup_{Q\in\wh\sss(R)\setminus\LD(R_0)} D_Q .
\end{equation*}
Denote $\nu = \HH^n|_{\Gamma}$. We showed in Lemma \ref{lem9.2} that $\Theta(R_0)^{-1}\eta$ is an AD-regular measure, and so it follows by standard arguments (using e.g. \cite[Theorem 6.9]{Mattila-llibre}) that $\Gamma$ is an AD-regular set, and that $\eta = \rho \nu$ for some density $\rho$ satisfying $\rho\approx_{\Lambda,\delta_0}\Theta(R_0)$. It is also immediate to check that Lemma \ref{lemriesz*eta} implies that $\RR_\nu$ is bounded in $L^2(\nu)$, with
\begin{equation*}
\|\RR_\nu\|_{L^2(\nu)\to L^2(\nu)}\lesssim_{\Lambda,\delta_0,K}1.
\end{equation*}
Hence, by the main result of \cite{NToV1} we know that $\Gamma$ is uniformly $n$-rectifiable. This allows us to use the $\beta$ numbers characterization of uniform rectifiability \cite{DS1} to get
\begin{equation}\label{eq:betasUR}
\int_{B(z,r_0)}\int_0^{r_0}\beta_{\nu,2}(x,r)^2\, \frac{dr}{r}d\nu(x)\lesssim_{\Lambda,\delta_0,K}r_0^n,\quad \text{for }z\in\supp\nu, r_0\in(0,\diam(\supp\nu)).
\end{equation}

To transfer these estimates back to the measure $\mu$ we will argue similarly as in Section 7 of \cite{Azzam-Tolsa}. It will be convenient to work with regularized cubes, as we did in Section \ref{sec6.2*}. Consider a function
\begin{equation*}
d_{R,*}(x) = \inf_{Q\in\wh\tree_*(R)} (\dist(x,Q)+\ell(Q)).
\end{equation*}
Note that $d_{R,*}(x)=0$ for $x$ in the closure of  $\wh\GG(R),$ and $d_{R,*}(x)>0$ everywhere else. Moreover, $d_{R,*}$ is 1-Lipschitz. For each $x\in  R\setminus\overline{\wh\GG(R)}$ we define $Q_x$ to be the maximal cube from $\DD_\mu$ that contains $x$ and satisfies
\begin{equation*}
\ell(Q_x)\le \frac{1}{60}\inf_{y\in Q_x} d_{R,*}(y).
\end{equation*}
The family of all the cubes $Q_x,\ x\in R\setminus\overline{\wh\GG(R)},$ will be denoted by $\Reg_*(R)$. We define also the regularized tree $\treg_*(R)$ consisting of the cubes from $\DD_\mu$ that are contained in $R$ and are not strictly contained in any of the $\Reg_*(R)$ cubes. 

It follows easily from the definition of $\Reg_*(R)$ that $\wh\tree_*(R)\subset\treg_*(R)$.
Observe also that $\Reg_*(R)$ consists of pairwise disjoint cubes and satisfies
\begin{equation*}
\mu\bigg(R\setminus \bigg(\bigcup_{P\in\Reg_*(R)}P\cup \wh\GG(R)\bigg)\bigg)=0.
\end{equation*}

The following is an analogue of Lemma \ref{lem74}.
\begin{lemma}\label{lem:reg prop}
	The cubes from $\Reg_*(R)$ satisfy the following properties:
	\begin{itemize}
		\item[(a)] If $P\in\Reg_*(R)$ and $x\in B(x_{P},50\ell(P))$, then $10\,\ell(P)\leq d_{R,*}(x) \leq c\,\ell(P)$,
		where $c$ is some constant depending only on $n$. 
		%In particular, $B(z_{P},50\ell(P))\cap W_0=\varnothing$.		
		\item[(b)] There exists some absolute constant $c>0$ such that if $P,\,P'\in\Reg_*(R)$ satisfy $B(x_{P},50\ell(P))\cap B(x_{P'},50\ell(P'))
		\neq\varnothing$, then
		$$c^{-1}\ell(P)\leq \ell(P')\leq c\,\ell(P).$$
		\item[(c)] For each $P\in \Reg_*(R)$, there are at most $N$ cubes $P'\in\Reg_*(R)$ such that
		$$B(x_{P},50\ell(P))\cap B(x_{P'},50\ell(P'))
		\neq\varnothing,$$
		where $N$ is some absolute constant.		
		%\item[(d)] If $x\not\in B(x_0,\frac1{8} K r_0)$, then $d(x)\approx |x-x_0|$. Thus, if $P\in\reg$ and $B(z_{P},50\ell(P))\not\subset  B(x_0,\frac1{8} K r_0)$, then $\ell(P)\gtrsim  K r_0$.
	\end{itemize}
\end{lemma}
As before, we omit the proof.

\begin{lemma}\label{lem:QtregPtree}
	For all $Q\in\treg_*(R)$ there exists some $P\in\wh\tree_*(R)$ such that $\ell(Q)\approx\ell(P)$ and $2B_Q\subset C B_{P}\subset C' B_{Q}$, where $C$ and $C'$ are some absolute constants. In consequence,
	\begin{equation}\label{eq:treg dens bdd}
	\PP(Q)\lesssim_{\Lambda}\Theta(R_0).
	\end{equation}
\end{lemma}
\begin{proof}
	Let $Q\in\treg_*(R)$. If $Q\cap\wh\GG(R)\neq\varnothing,$ then $Q\in\wh\tree_*(R)$ and we can take $P=Q$. If $Q\cap\wh\GG(R)=\varnothing,$ then there exists some $Q_0\in\Reg_*(R)$ such that $Q_0\subset Q$. By the definition of $d_{R,*}$ and Lemma \ref{lem:reg prop} (a), there exists $P_0\in\wh\tree_*(R)$ such that
	\begin{equation*}
	\dist(x_{Q_0},P_0)+\ell(P_0)\le 2 d_{R,*}(x_{Q_0})\approx \ell(Q_0).
	\end{equation*}
	In particular, $\ell(P_0)\lesssim \ell(Q_0)\le\ell(Q)$. If $\ell(P_0)\ge\ell(Q),$ set $P=P_0$, otherwise let $P$ be the ancestor of $P_0$ with $\ell(P)= \ell(Q)$. Clearly, $\ell(P)\approx\ell(Q)$, and moreover
	\begin{equation*}
	\dist(x_Q,x_P)\le \dist(x_{Q_0},P_0)+\ell(Q)+\ell(P)\lesssim\ell(Q_0)+\ell(Q)+\ell(P)\approx \ell(Q)\approx\ell(P),
	\end{equation*}
	which implies $2B_Q\subset C B_{P}\subset C' B_{Q}$ for some absolute $C$ and $C'$.
	
	Finally, to see $\PP(Q)\lesssim_{\Lambda}\Theta(R_0)$ recall that $\PP(P)\lesssim \Lambda\Theta(R_0)$ for all $P\in\wh\tree_*(R)\subset\tree(R_0)$, and we have $\PP(Q)\lesssim\PP(P)$ because $B_Q\subset C B_P$ and $\ell(Q)\approx \ell(P)$.
\end{proof}

The following lemma states that the uniformly rectifiable set $\Gamma$ lies relatively close to all the cubes from $\Reg_*(R)$. This property will be crucial in our subsequent estimates.
\begin{lemma}\label{lem:RegGamma}
	There exists $C_*=C_*(\Lambda,\delta_0)$ such that for all $Q\in\Reg_*(R)$ we have
	\begin{equation}\label{eq:RegGamma}
	\frac{C_*}{2}B_Q\cap\Gamma\neq\varnothing.
	\end{equation}
\end{lemma}
\begin{proof}
	Let $Q\in\Reg_*(R)$ and let $P\in\wh\tree_*(R)$ be the cube from Lemma \ref{lem:QtregPtree}.
	In particular, we have
	\begin{equation}\label{eq:BPCBQ}
	2B_P\subset C B_Q
	\end{equation}
	for some absolute constant $C$.
	
	If $P$ contains some cube $P_1\in\wh\Stop(R)\setminus\LD(R_0)$, then we are done, because in that case $D_{P_1}\subset 2B_P\subset C B_Q$, and $D_{P_1}\subset\Gamma$. Similarly, if $\wh\GG(R)\cap P\neq\varnothing$, then there is nothing to prove.
	
	Now suppose that $P\cap\wh\GG(R)=\varnothing$ and $P$ does not contain any cube from $\wh\Stop(R)\setminus\LD(R_0)$. Since $P\in\wh\tree_*(R)$ and $P\cap\wh\GG(R)=\varnothing$, there exists some $P_1\in\wh\sss_*(R)$ such that $P_1\subset P$. By our assumption $P_1\notin\wh\sss(R)\setminus\LD(R_0)$, and so
	\begin{equation*}
	P_1\in \wh\sss_*(R)\setminus \left(\wh\sss(R)\setminus\LD(R_0)\right) = \wh\Ch((iii)_R) \cup \left(\wh\sss(R)\cap\LD(R_0)\right).
	\end{equation*}
	
	There are two cases to consider. Suppose that $P_1\in\wh\Ch((iii)_R)$. Let $S\in (iii)_R$ be such that $P_1\in\wh\Ch(S)$. Since $S\notin\LD(R_0)$ (otherwise we'd have $S\in (i)_R$), \eqref{eqcompa492} gives $\ell(P_1)\approx_{\Lambda,\delta_0}\ell(S)$. Thus, there exists some constant  $C(\Lambda,\delta_0)$ such that
	\begin{equation*}
	D_S\subset C(\Lambda,\delta_0)B_{P_1}\subset C(\Lambda,\delta_0)B_{P}\overset{\eqref{eq:BPCBQ}}{\subset}C\, C(\Lambda,\delta_0) B_Q.
	\end{equation*}
	Since $D_S\subset\Gamma$, we get \eqref{eq:RegGamma} as soon as $C_*\ge 2C\, C(\Lambda,\delta_0)$.
	
	Finally, suppose that $P_1\in\wh\sss(R)\cap\LD(R_0)$. Let $P_0\in\wh\tree_0(R)\setminus\wh\sss(R)$ be the unique cube such that $P_1\in\wh\Ch(P_0)$. By \eqref{eqcompa492} we have $\ell(P_0)\approx_{\Lambda,\delta_0}\ell(P_1)$, and so
	\begin{equation*}
	2B_{P_0}\subset C(\Lambda,\delta_0) B_{P_1}\subset C(\Lambda,\delta_0) B_{P}\subset C\, C(\Lambda,\delta_0) B_{Q}.
	\end{equation*}
	We claim that $2B_{P_0}\cap\Gamma\neq\varnothing$, and so \eqref{eq:RegGamma} is satisfied if we assume $C_*\ge 2C\, C(\Lambda,\delta_0)$. First, if $2B_{P_0}\cap\wh\GG(R)\neq\varnothing$, then there is nothing to prove. Assume the contrary. In that case $P_0$ is covered by cubes from $\wh\Stop(R)$. We claim that there exists some $S\in\wh\sss(R)\setminus\LD(R_0)$ such that $S\subset P_0$. Indeed, otherwise $P_0$ would be covered by cubes from $\wh\sss(R)\cap\LD(R_0)$, but then
	\begin{equation*}
	-\mu(P_0)=\sum_{P'\in\wh\sss(R)\cap\LD(R_0)}-\mu(P')=\sum_{P'\in\wh\sss(R)\cap\LD(R_0)}s(P') \overset{\eqref{eqaq89}}{=}s(P_0)\overset{\eqref{eq:aQmuQ}}{\ge}0,
	\end{equation*}
	which is a contradiction. Thus, there exists $S\in\wh\sss(R)\setminus\LD(R_0)$ such that $S\subset P_0$, which implies $D_{S}\subset 2B_{P_0}$. Since $D_{S}\subset\Gamma$, we are done.
\end{proof}

In the following lemma we define functions supported on $\Gamma$ that approximate $\mu$ at the level of $\Reg_*(R)$.

\begin{lemma}\label{lem:gP}
	There exist functions $g_P:\Gamma\to\R,\ P\in\Reg_*(R),$ such that each $g_P$ is supported in $\Gamma\cap \overline{C_* B_P}$,
	\begin{equation}\label{eq:intgP}
	\int_{\Gamma} g_P\ d\nu = \mu(P),
	\end{equation}
	and
	\begin{equation}\label{eq:sumgP}
	\sum_{P\in\Reg_*(R)} g_P \lesssim_{\Lambda,\delta_0} \Theta(R).
	\end{equation}
\end{lemma}
\begin{proof}
	Assume first that the family $\Reg_*(R)$ is finite.
	%(which is equivalent to $\wh\GG(R)=\varnothing$)
	 We label the cubes from $\Reg_*(R)$ in the order of increasing sidelength, that is we let $P_1$ be a cube with the minimal sidelength, and then we label all the remaining cubes so that $\ell(P_i)\le\ell(P_{i+1})$.
	
	The functions $g_{i}:=g_{P_i}$ will be of the form $g_i = \alpha_i\chi_{A_i}$ where $\alpha_i\ge 0 $ and $A_i\subset \Gamma\cap C_* B_{P_i}$. We begin by setting $\alpha_1 = \mu(P_1)/\nu(C_* B_{P_1})$ and $A_1 = C_* B_{P_1}\cap\Gamma$. Clearly, \eqref{eq:intgP} holds for $P_1$. Moreover, using the fact that $\nu$ is AD-regular and $\frac{C_*}{2} B_{P_1}\cap\Gamma\neq\varnothing$ we get
	\begin{equation*}
	\|g_1\|_\infty = \alpha_1 = \frac{\mu(P_1)}{\nu(C_* B_{P_1})}\approx_{\Lambda,\delta_0}\frac{\mu(P_1)}{\ell(P_1)^n}\overset{\eqref{eq:treg dens bdd}}{\lesssim_{\Lambda}}\Theta(R_0).
	\end{equation*}
	
	We define the remaining $g_k,\, k\ge 2$ inductively. Suppose that $g_1,\dots,g_{k-1}$ have already been constructed, and they satisfy
	\begin{equation}\label{eq:sumgi}
	\sum_{i=1}^{k-1} g_i \le C' \Theta(R)
	\end{equation}
	for some constant $C'=C'(\Lambda,\delta_0)$ to be fixed below. Let $P_{i_1},\dots,P_{i_m}$ be the subfamily of $P_1,\dots,P_{k-1}$ consisting of cubes such that $C_* B_{P_k}\cap C_* B_{P_{i_j}}\neq\varnothing$. Due to the non-decreasing sizes of $P_i$'s we have $P_{i_j}\subset C_* B_{P_{i_j}}\subset 3C_* B_{P_k}$. Hence, applying \eqref{eq:intgP} to $g_{i_j}$ we get
	\begin{equation*}
	\sum_j \int_{\Gamma} g_{i_j}\ d\nu = \sum_j\mu(P_{i_j})\le\mu(3C_* B_{P_k})\overset{\eqref{eq:treg dens bdd}}{\le}C({\Lambda,\delta_0})\Theta(R_0)\ell(P_k)^n\le C''\Theta(R_0)\nu(\Gamma\cap C_* B_{P_k}),
	\end{equation*}
	for some $C''$ depending on $\Lambda,\delta_0$. By the Chebyshev's inequality
	\begin{equation*}
	\nu\left(\Gamma\cap\big\{\textstyle{\sum_j}\, g_{i_j}\ge 2C''\Theta(R_0)\big\}\right)\le \frac{1}{2}\nu(\Gamma\cap C_* B_{P_k}).
	\end{equation*}
	Set
	\begin{equation*}
	A_k = \Gamma\cap C_* B_{P_k}\cap \big\{\textstyle{\sum_j}\, g_{i_j}\le 2C''\Theta(R_0)\big\},
	\end{equation*}
	and then by the preceding estimate $\nu(A_k)\ge \nu(\Gamma\cap C_* B_{P_k})/2.$ We define
	\begin{equation*}
	\alpha_k = \frac{\mu(P_k)}{\nu(A_k)},
	\end{equation*}
	so that for $g_k=\alpha_k\chi_{A_k}$ we have $\int g_k\, d\nu = \mu(P_k)$. Moreover, using AD-regularity of $\nu$ and the fact that $\frac{C_*}{2} B_{P_k}\cap\Gamma\neq\varnothing$
	\begin{equation*}
	\alpha_k \le 2 \frac{\mu(P_k)}{\nu(C_* B_{P_k})} \le C({\Lambda,\delta_0})\frac{\mu(P_k)}{\ell(P_k)^n}\overset{\eqref{eq:treg dens bdd}}{\le} C'''\Theta(R_0)
	\end{equation*}
	for some $C'''$ depending on $\Lambda,\delta_0$. Hence, by the definition of $A_k$
	\begin{equation*}
	g_k(x) + \sum_j g_{i_j}(x) \le C'''\Theta(R_0) + 2C''\Theta(R_0), \quad\text{for $x\in A_k$.}
	\end{equation*}
	For $x\not\in A_k$ we have $g_k=0$, and so it follows from the above and the inductive assumption \eqref{eq:sumgi} that for $C' = C''' + 2C''$ we have
	\begin{equation*}
	\sum_{i=1}^{k} g_i \le C' \Theta(R),
	\end{equation*}
	which closes the induction.
	
	Suppose now that the family $\Reg_*(R)$ is infinite. We can relabel it so that $\Reg_*(R)=\{P^i\}_{i\in\N}$. For each $N$ we consider the family $\{P^i\}_{1\le i\le N}$. We construct functions $g_{P^1}^N,\dots, g_{P^N}^N$ as above, so that they satisfy
	\begin{equation*}
	\int g_{P^1}^N\ d\nu = \mu(P^1),\qquad \sum_{i=1}^{N} g_{P^i}^N \le C' \Theta(R).
	\end{equation*}
	There exists a subsequence $I_1\subset\N$ such that $\{g_{P^1}^k\}_{k\in I_1}$ is convergent in the weak-$*$ topology of $L^\infty(\nu)$ to some function $g_{P^1}\in L^{\infty}(\nu)$. We take another subsequence $I_2\subset I_1$ such that $\{g_{P^2}^k\}_{k\in I_2}$ is convergent in the weak-$*$ topology of $L^\infty(\nu)$ to some $g_{P^2}\in L^{\infty}(\nu)$. Proceeding in this fashion we obtain a family $\{g_{P^i}\}_{i\in\N}$ such that $\supp g_{P^i}\subset \overline{C_* B_{P^i}}$, and the properties \eqref{eq:intgP}, \eqref{eq:sumgP} are preserved (because of the weak-$*$ convergence).
\end{proof}

Recall that by the uniform rectifiability of $\nu$ we have a good estimate on the $\beta_{\nu,2}$ numbers \eqref{eq:betasUR}. We will now use Lemmas \ref{lem:RegGamma} and \ref{lem:gP} to transfer these estimates to the measure $\mu$ and obtain
\begin{equation*}
	\sum_{Q\in\wh \tree_*(R)}\beta_{\mu,2}(2B_Q)^2\,\mu(Q)\lesssim_{\Lambda,\delta_0,K}\Theta(R_0)\,\mu(R).
\end{equation*}
In fact, we will show that
\begin{equation*}
	\sum_{Q\in\treg_*(R)}\beta_{\mu,2}(2B_Q)^2\,\mu(Q)\lesssim_{\Lambda,\delta_0,K}\Theta(R_0)\,\mu(R),
\end{equation*}
and the former estimate will follow, since $\wh \tree_*(R)\subset\treg_*(R)$.

Let $Q\in\treg_*(R)$, and let $L_Q$ be an $n$-plane minimizing $\beta_{\nu,2}(C_*'B_Q)$, where $C_*'>2$ is some constant depending on $C_*$, to be chosen in Lemma \ref{lem:PsizeQ}. We estimate
\begin{multline}\label{eq:bet1}
\beta_{\mu,2}(2B_Q)^2\mu(Q)\lesssim \frac{\mu(Q)}{\ell(Q)^n}\int_{2B_Q} \left(\frac{\dist(x,L_Q)}{\ell(Q)}\right)^2\,d\mu(x)\\
\overset{\eqref{eq:treg dens bdd}}{\lesssim_\Lambda}\Theta(R_0)\int_{2B_Q} \left(\frac{\dist(x,L_Q)}{\ell(Q)}\right)^2\,d\mu(x)\\
= \Theta(R_0)\left(\int_{2B_Q\cap R\setminus \wh\GG(R)}\dots\,d\mu(x) + \int_{2B_Q\cap\wh\GG(R)} \dots\,d\mu(x) + \int_{2B_Q\setminus R} \dots\,d\mu(x)\right)\\
=:\Theta(R_0)(I_1+I_2+I_3).
\end{multline}
Concerning $I_3$, we use the trivial estimate
\begin{equation}\label{eq:bet2}
I_3\lesssim\mu(2B_Q\setminus R).
\end{equation}
Estimating $I_2$ is simple because on $\wh\GG(R)$ we have $\mu=\rho'\nu$ for some $\rho'\lesssim_{\Lambda} \Theta(R_0)$, and so
\begin{multline}\label{eq:bet3}
I_2 \lesssim_{\Lambda} \Theta(R_0)\int_{2B_Q\cap\wh\GG(R)} \left(\frac{\dist(x,L_Q)}{\ell(Q)}\right)^2\,d\nu(x)\le \Theta(R_0)\int_{C_*'B_Q} \left(\frac{\dist(x,L_Q)}{\ell(Q)}\right)^2\,d\nu(x)\\
\approx_{C_*'} \Theta(R_0)\beta_{\nu,2}(C_*'B_Q)^2\ell(Q)^n.
\end{multline}

Bounding $I_1$ requires more work. First, we use the fact that $R\setminus\wh\GG(R)$ is covered $\mu$-a.e. by $\Reg_*(R)$:
\begin{align*}
I_1 &\le \sum_{P\in\Reg_*(R) :\,  P\cap 2B_Q\neq\varnothing} \int_{P} \left(\frac{\dist(x,L_Q)}{\ell(Q)}\right)^2\,d\mu(x) \\
&= \sum_{P\in\Reg_*(R) :\,  P\cap 2B_Q\neq\varnothing} 
\bigg(\int_{\Gamma} \left(\frac{\dist(x,L_Q)}{\ell(Q)}\right)^2 g_P(x)\,d\nu(x)\\
&\quad+ \int \left(\frac{\dist(x,L_Q)}{\ell(Q)}\right)^2 (\chi_{P}(x)d\mu(x) - g_P(x)d\nu(x))\bigg)
\\
&=: \sum_{P\in\Reg_*(R) :\,  P\cap 2B_Q\neq\varnothing} \left(I_{11}(P) + I_{12}(P)\right).
\end{align*}
We need the following auxiliary result.

\begin{lemma}\label{lem:PsizeQ}
	If $P\in\Reg_*(R)$ is such that $P\cap 2B_Q\neq\varnothing$, then $\ell(P)\lesssim\ell(Q)$ and in consequence $\overline{C_*B_P}\subset C_*'B_Q$ for some $C_*'=C_*'(C_*)\ge C_*$.
\end{lemma}
\begin{proof}
	If $\ell(P)\le\ell(Q)$ then there is nothing to prove, so suppose $\ell(P)>\ell(Q)$ (in particular $\ell(P)\ge A_0\ell(Q))$. In that case we have $2B_Q\subset 2B_P$. 
	
	Note that if we had $Q\setminus\wh\GG(R)=\varnothing$, then $d_{R,*}(x_Q)=0$, but by Lemma \ref{lem:reg prop} (a) we know that $d_{R,*}(x_Q)\ge 10\ell(P)$.
	Hence, there exists some $S\in\Reg_*(R)$ such that $S\subset Q$. Together with the fact that $2B_Q\subset 2B_P$ this implies $B_S\cap 2B_P\neq\varnothing$. By Lemma \ref{lem:reg prop} (b) this gives
	\begin{equation*}
	\ell(P)\approx \ell(S)\le\ell(Q).
	\end{equation*}
\end{proof}
By the lemma above and the preceding estimate we get
\begin{equation}\label{eq:bet4}
I_1 \le \sum_{P\in\Reg_*(R) :\,  C_*B_P\subset C_*'B_Q} I_{11}(P) + \sum_{P\in\Reg_*(R) :\,  C_*B_P\subset C_*'B_Q} I_{12}(P).
\end{equation}
We estimate the first sum as follows:
\begin{multline}\label{eq:bet5}
\sum_{P\in\Reg_*(R) :\,  C_*B_P\subset C_*'B_Q} \int_{\Gamma} \left(\frac{\dist(x,L_Q)}{\ell(Q)}\right)^2 g_P(x)\,d\nu(x)\\
=  \int_{\Gamma} \left(\frac{\dist(x,L_Q)}{\ell(Q)}\right)^2 \sum_{P\in\Reg_*(R) :\,  C_*B_P\subset C_*'B_Q}  g_P(x)\,d\nu(x)\\
\overset{{\supp g_P\subset C_* B_P}}{\le} \int_{\Gamma\cap C_*'B_Q} \left(\frac{\dist(x,L_Q)}{\ell(Q)}\right)^2 \sum_{P\in\Reg_*(R)}  g_P(x)\,d\nu(x)\\
\overset{\eqref{eq:sumgP}}{\lesssim}_{\Lambda,\delta_0}\Theta(R_0)\int_{\Gamma\cap C_*'B_Q} \left(\frac{\dist(x,L_Q)}{\ell(Q)}\right)^2\,d\nu(x)\approx_{C_*'} \Theta(R_0)\beta_{\nu,2}(C_*'B_Q)^2\ell(Q)^n.
\end{multline}
Concerning $I_{12}(P)$, observe that since $\int g_P\, d\nu=\mu(P)$ by \eqref{eq:intgP}, we have
\begin{equation*}
I_{12}(P) = \int \left(\left(\frac{\dist(x,L_Q)}{\ell(Q)}\right)^2 - \left(\frac{\dist(x_P,L_Q)}{\ell(Q)}\right)^2\right) \left(\chi_{P}(x)d\mu(x) - g_P(x)d\nu(x)\right).
\end{equation*}
For $x\in\supp(\chi_{P}(x)d\mu(x) - g_P(x)d\nu(x))\subset C_*B_P\subset C_*'B_Q$ we have 
\begin{multline*}
\left|\left(\frac{\dist(x,L_Q)}{\ell(Q)}\right)^2 - \left(\frac{\dist(x_P,L_Q)}{\ell(Q)}\right)^2\right| \le \frac{|x-x_P|}{\ell(Q)}\cdot\frac{\dist(x,L_Q)+\dist(x_P,L_Q)}{\ell(Q)}\\
\lesssim\frac{C_*\ell(P)}{\ell(Q)}\cdot\frac{C_*'\ell(Q)}{\ell(Q)} \approx_{C_*,C_*'} \frac{\ell(P)}{\ell(Q)}.
\end{multline*}
Hence,
\begin{equation}\label{eq:bet6}
I_{12}(P) \lesssim_{C_*,C_*'}\frac{\ell(P)}{\ell(Q)}\mu(P).
\end{equation}
Recall that $C_*$ depends on $\Lambda,\delta_0$, and $C_*'$ depends on $C_*$. Thus, putting together the estimates \eqref{eq:bet4}, \eqref{eq:bet5}, and \eqref{eq:bet6} yields
\begin{equation*}
I_1\lesssim_{\Lambda,\delta_0} \Theta(R_0)\beta_{\nu,2}(C_*'B_Q)^2\ell(Q)^n + \sum_{\substack{P\in\Reg_*(R):\\  C_*B_P\subset C_*'B_Q}}\frac{\ell(P)}{\ell(Q)}\mu(P).
\end{equation*}
Together with \eqref{eq:bet1}, \eqref{eq:bet2}, and \eqref{eq:bet3} this gives
\begin{multline*}	
\beta_{\mu,2}(2B_Q)^2\mu(Q)\lesssim_{\Lambda,\delta_0} \Theta(R_0)^2\beta_{\nu,2}(C'_* B_Q)^2\ell(Q)^n \\ +
\Theta(R_0)\sum_{\substack{P\in\Reg_*(R):\\  C_*B_P\subset C_*'B_Q}}\frac{\ell(P)}{\ell(Q)}\mu(P) + \Theta(R_0)\mu(2B_Q\setminus R).
\end{multline*}
Summing over $Q\in\treg_*(R)$ we get
\begin{multline*}	
\sum_{Q\in\treg_*(R)}\beta_{\mu,2}(2B_Q)^2\mu(Q)\lesssim_{\Lambda,\delta_0} \Theta(R_0)^2\sum_{Q\in\treg_*(R)}\beta_{\nu,2}(C'_* B_Q)^2\ell(Q)^n \\+
\Theta(R_0)\sum_{Q\in\treg_*(R)}\sum_{\substack{P\in\Reg_*(R):\\  C_*B_P\subset C_*'B_Q}}\frac{\ell(P)}{\ell(Q)}\mu(P) + \Theta(R_0)\sum_{Q\in\treg_*(R)}\mu(2B_Q\setminus R)\\
=:\Theta(R_0)^2 S_1+\Theta(R_0)S_2+\Theta(R_0)S_3.
\end{multline*}

Concerning $S_1$, note that by \eqref{eq:RegGamma} we know that if $Q\in\treg_*(R)$, then $\nu(C_* B_Q\cap\Gamma)\approx_{\Lambda,\delta_0}\ell(Q)^n$ and for all $x\in C_* B_Q\cap\Gamma$ we have $C'_* B_Q\subset B(x,2C_*' \ell(Q))$. Thus, $\beta_{\nu,2}(C'_* B_Q)\lesssim \beta_{\nu,2}(x,r)$ for $2C_*' \ell(Q)<r<3C_*' \ell(Q)$. Observe also that the sets $C_* B_Q\cap\Gamma$ corresponding to cubes of the same generation have bounded intersection. It follows easily that
\begin{multline*}
S_1=\sum_{Q\in\treg_*(R)}\beta_{\nu,2}(C'_* B_Q)^2\ell(Q)^n \lesssim_{\Lambda,\delta_0} \int_{5C_*' B_{R}}\int_0^{5C_*'\ell(R)}\beta_{\nu,2}(x,r)^2\, \frac{dr}{r}d\nu(x)\\
\overset{\eqref{eq:betasUR}}{\lesssim}_{\Lambda,\delta_0,K}\ell(R)^n.
\end{multline*}

To estimate $S_2$ we change the order of summation:
\begin{equation*}
S_2 = \sum_{P\in\Reg_*(R)}\mu(P)\sum_{\substack{Q\in\treg_*(R):\\  C_*B_P\subset C_*'B_Q}}\frac{\ell(P)}{\ell(Q)}.
\end{equation*}
Note that the inner sum is essentially a geometric series, and so
\begin{equation*}
S_2 \lesssim_{C_*,C_*'} \sum_{P\in\Reg_*(R)}\mu(P)\le \mu(R).
\end{equation*}

Finally, we can bound $S_3$ using the small boundaries property of the David-Mattila lattice \eqref{eqsmb2}. To be more precise, note that for $Q\in\treg_*(R)$ if $2B_Q\setminus R\neq\varnothing$ and $\ell(Q)=A_0^{-k}\ell(R)$, then necessarily $Q\subset N_{k-1}(R)$, and even $2B_Q\subset N_{k-1}(R)$. Furthermore, the balls $2B_Q$ for cubes of the same generation have only bounded intersection. Thus,
\begin{equation*}
S_3\le \sum_{k\ge 1}\sum_{\substack{Q\subset N_{k-1}(R),\\ \ell(Q)=A_0^{-k}\ell(R)}}\mu(2B_Q)\lesssim \sum_{k\ge 1}\mu(N_{k-1}(R))\overset{\eqref{eqsmb2}}{\lesssim}\mu(90B(R))\approx\mu(R).
\end{equation*}

Putting the estimates for $S_1,\ S_2$ and $S_3$ together we arrive at
\begin{equation*}
\sum_{Q\in\treg_*(R)}\beta_{\mu,2}(2B_Q)^2\mu(Q)\lesssim_{\Lambda,\delta_0,K} \Theta(R_0)^2\ell(R)^n + \Theta(R_0)\mu(R)\lesssim_{\delta_0}\Theta(R_0)\mu(R),
\end{equation*}
where in the last estimate we used the fact that $\Theta(R_0)\lesssim\delta_0^{-1}\Theta(R)$ (note that $R\not\in\LD(R_0)$ because $\DD_\mu^\PP\cap\LD(R_0)\subset\End(R_0)$ and we assume $R\in\DD_\mu^\PP\setminus\End(R_0)$). This finishes the proof of Lemma \ref{lembetas99}.


% ********************************************************************************************


\vv

\subsection{The corona decompostion and the proof of Lemma \ref{lemtreebeta}}

Now we 
 define $\wh \ttt= \wh\ttt(R_0)$ inductively.
We set $\wh \ttt_0=\{R_0\}$ and, assuming $\wh\ttt_k$ to be defined, we let
$$\wh\ttt_{k+1} = \bigcup_{R\in\wh\ttt_k} (\wh \End(R)\setminus\End(R_0)).$$
Then we let
$$\wh\ttt =\bigcup_{k\geq0} \wh\ttt_k.$$
In this way, we have
$$\tree(R_0)=\bigcup_{R\in\wh\ttt} \wh\tree(R).$$

\vv


\begin{lemma}\label{lemtop8}
We have
$$\sigma(\wh \ttt)  \lesssim_{\Lambda,\delta_0}\sigma(R_0) +\!\sum_{Q\in\tree(R_0)}\!\|\Delta_Q\RR\mu\|_{L^2(\mu)}^2+ \sum_{Q\in\tree(R_0)\cap\HE}\! \EE(4Q).$$
\end{lemma}

\begin{proof} 
By Lemma \ref{lem9.5*}, we have
$$\sum_{Q\in\wh\sss(R)\cap(\LD(R_0)\cup\HD(R_0)\cup\BR(R))}\mu(Q) 
+\sum_{Q\in\wh\Ch((iii)_R)\cap \LD(R_0)}\mu(Q) + \mu(\wh\sG(R))
\approx \mu(R).$$
By the construction of $\wh\tree(R)$, the cubes from 
$\wh\sss(R)\cup \wh\Ch((iii)_R)$ belong to $\wh\tree(R)$ and the ones from $\wh\sss(R)\cap\BR(R)$ belong to $\wh \End(R)$, and so
$$\mu(R)\approx \sum_{Q\in\wh\tree(R)\cap(\LD(R_0)\cup\HD(R_0))}\mu(Q) 
+ \sum_{Q\in\wh\End(R)\cap\BR(R)}\!\!\mu(Q)
+ \mu(\wh\sG(R)).$$
Notice that the families $\wh\tree(R)$, with $R\in\wh\ttt$, are disjoint, with the possible exception of
the roots and ending cubes of the trees $\wh\tree(\cdot)$, which may belong to two different trees. 
Then we deduce that
\begin{align*}
\sum_{R\in\wh\ttt} \mu(R) & \approx \sum_{R\in\wh\ttt} \bigg(
\sum_{Q\in\wh\tree(R)\cap(\LD(R_0)\cup\HD(R_0))}\!\!\!\!\mu(Q) + \sum_{Q\in\wh\End(R)\cap\BR(R)}\!\!\!\mu(Q)\bigg)
+\sum_{R\in\wh\ttt}\! \mu(\wh\sG(R))\\
& \lesssim \mu(R_0) + \sum_{R\in\wh\ttt}\,\sum_{Q\in\wh\End(R)\cap\BR(R)}\!\!\mu(Q).
\end{align*}
Since the cubes $Q\in\BR(R)$ do not belong to $\LD(R_0)$, we have $\Theta(Q)\approx_{\Lambda,\delta_0} \Theta(R_0)$ for such cubes. The same happens for $R\in\wh\ttt$, and thus
\begin{equation}\label{eqthus883}
\sigma(\wh \ttt)  \lesssim_{\Lambda,\delta_0} \sigma(R_0) + \sum_{R\in\wh\ttt} \Theta(R)^2\sum_{Q\in\wh\End(R)\cap\BR(R)}\mu(Q).
\end{equation}

To estimate the last sum above, we claim that for a given $Q\in\BR(R)\cap\wh\End(R)$ we have
$$\big|\RR(\chi_{2R\setminus 2Q}\mu)(x_Q) - 
\big(m_{\mu,Q}(\RR\mu) - m_{\mu,R}(\RR\mu)\big)\big|
\lesssim \PP(R) + \left(\frac{\EE(4R)}{\mu(R)}\right)^{1/2} + \PP(Q) +  \left(\frac{\EE(2Q)}{\mu(Q)}\right)^{1/2}.$$
This is proved exactly in the same way as Lemma \ref{lemaprox2} (see also \rf{eqal842}) and so we omit the arguments. 
In case that both $R,Q\not\in \HE$, then 
$$\left(\frac{\EE(4R)}{\mu(R)}\right)^{1/2}\leq M_0\,\Theta(R)\quad \text{ and }\quad   \left(\frac{\EE(2Q)}{\mu(Q)}\right)^{1/2}\leq M_0\,\Theta(Q),$$
and so
we get
$$\big|\RR(\chi_{2R\setminus 2Q}\mu)(x_Q) - 
\big(m_{\mu,Q}(\RR\mu) - m_{\mu,R}(\RR\mu)\big)\big|
\leq C(\Lambda,\delta_0,M_0) \Theta(R).$$
Thus, by the $\BR(R)$ condition,
$$K\,\Theta(R) \leq |\RR(\chi_{2R\setminus 2Q}\mu)(x_Q)| \leq 
\big|m_{\mu,Q}(\RR\mu) - m_{\mu,R}(\RR\mu)\big| + C(\Lambda,\delta_0,M_0) \Theta(R).$$
Hence, for $K\geq 2\,C(\Lambda,\delta_0,M_0)$, we obtain
$$\frac12K\,\Theta(R) \leq 
\big|m_{\mu,Q}(\RR\mu) - m_{\mu,R}(\RR\mu)\big|.$$

In the general case where $Q$ and $R$ may belong to $\HE$, by analogous arguments, we get
$$\frac12K\,\Theta(R) \leq 
\big|m_{\mu,Q}(\RR\mu) - m_{\mu,R}(\RR\mu)\big| + \chi_\HE(R) \left(\frac{\EE(4R)}{\mu(R)}\right)^{1/2}
+ \chi_\HE(Q) \left(\frac{\EE(2Q)}{\mu(Q)}\right)^{1/2},$$
where $\chi_\HE(P)=1$ if $P\in\HE$ and $\chi_\HE(P)=0$ otherwise. Since
$$m_{\mu,Q}(\RR\mu) - m_{\mu,R}(\RR\mu) = \chi_Q \sum_{P\in\wh\tree(R)\setminus\wh\End(R)}\Delta_P(\RR\mu),$$
assuming $K\geq1$, we get
\begin{align}\label{eqfje42}
\Theta(R)^2\sum_{Q\in\wh\End(R)\cap\BR(R)}\mu(Q) &\lesssim 
\sum_{Q\in\wh\End(R)\cap\BR(R)} \int_Q \Big|\sum_{P\in\wh\tree(R)\setminus\wh\End(R)}\Delta_P(\RR\mu)
\Big|^2\,d\mu \\
&\quad + \chi_\HE(R)\!\!\sum_{Q\in\wh\End(R)\cap\BR(R)}  \frac{\EE(4R)}{\mu(R)}\,\mu(Q) +\!
\sum_{Q\in\wh\End(R)\cap\HE} \! \EE(2Q).\nonumber
\end{align}
By orthogonality, the first sum on the right hand is bounded by
$$\int \Big|\sum_{P\in\wh\tree(R)\setminus\wh\End(R)}\Delta_P(\RR\mu)
\Big|^2\,d\mu = \sum_{P\in\wh\tree(R)\setminus\wh\End(R)}\|\Delta_P(\RR\mu)\|_{L^2(\mu)}^2.$$
Also, it is clear that the second sum on the right had side of \rf{eqfje42} does not exceed 
$\chi_\HE(R)\,\EE(4R)$. Therefore,
\begin{multline*}
\Theta(R)^2\!\!\sum_{Q\in\wh\End(R)\cap\BR(R)}\mu(Q) \\
\lesssim
\sum_{P\in\wh\tree(R)\setminus\wh\End(R)}\|\Delta_P(\RR\mu)\|_{L^2(\mu)}^2
+ \chi_\HE(R)\,\EE(4R) + \sum_{Q\in\wh\End(R)\cap\HE} \! \EE(2Q).
\end{multline*}

Plugging the previous estimate into \rf{eqthus883}, we obtain
\begin{align*}
\sigma(\wh \ttt)  & \lesssim_{\Lambda,\delta_0} \sigma(R_0) + \sum_{R\in\wh\ttt} 
\sum_{P\in\wh\tree(R)\setminus\wh\End(R)}\|\Delta_P(\RR\mu)\|_{L^2(\mu)}^2\\
&\quad
+ \sum_{R\in\wh\ttt\cap\HE}\EE(4R) +\sum_{R\in\wh\ttt}\,\sum_{Q\in\wh\End(R)\cap\HE} \! \EE(2Q)\\
& \lesssim_{\Lambda,\delta_0} \sigma(R_0) + \sum_{P\in\tree(R_0)} 
\|\Delta_P(\RR\mu)\|_{L^2(\mu)}^2 
+ \sum_{Q\in\tree(R_0)\cap\HE}\EE(4Q),
\end{align*}
as wished.
\end{proof}
\vv


\begin{proof}[\bf Proof of Lemma \ref{lemtreebeta}]
Given $R_0\in\ttt$, combining Lemmas \ref{lembetas99} and \ref{lemtop8}, we obtain
\begin{align*}
\sum_{Q\in\tree(R_0)} \!\!\!\beta_{\mu,2}(2B_Q)^2\,\Theta(Q)\,\mu(Q)  & 
\lesssim_{\Lambda,\delta_0}  \sum_{R\in \wh\ttt} \Theta(R)\sum_{Q\in\wh\tree(R)} \beta_{\mu,2}(2B_Q)^2\,\mu(Q)\\
& \lesssim_{\Lambda,\delta_0}  \sum_{R\in \wh\ttt} \Theta(R)^2 \mu(R)\\
& 
\lesssim_{\Lambda,\delta_0}\sigma(R_0) +\!\!\sum_{Q\in\tree(R_0)}\!\|\Delta_Q\RR\mu\|_{L^2(\mu)}^2+ \!\!\sum_{Q\in\tree(R_0)\cap\HE}\! \!\EE(4Q).
\end{align*}
\end{proof}


