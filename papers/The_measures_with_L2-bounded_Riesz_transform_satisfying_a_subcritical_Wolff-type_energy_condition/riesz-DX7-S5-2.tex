
\section{The layers \texorpdfstring{$\sF_j^h$ and $\sL_j^h$}{Fjh and Ljh}, and the tractable trees}\label{sec-layers}

We denote
$$\sF_j= \big\{R\in \ttt\cap \MDW:\Theta(R)=A_0^{nj}\big\},$$
so that 
$$\ttt\cap \MDW= \bigcup_{j\in\Z} \sF_j.$$
Next we split $\sF_j$ into layers $\sF_j^h$, $h\geq1$, which are defined as follows:
$\sF_j^1$ is the family of maximal cubes from $\sF_j$, and by induction
$\sF_j^h$ is the family of maximal cubes from $\sF_j\setminus \bigcup_{k=1}^{h-1} \sF_j^{h-1}$.
So we have the splitting
$$\ttt\cap \MDW= \bigcup_{j\in\Z}\,\bigcup_{h\geq1} \sF_j^h.$$


Our next objective is to choose a suitable subfamily $\sL_j^h\subset \sF_j^h$, for each $j,h$.
By the covering Theorem 9.31 from \cite{Tolsa-llibre}, there
is a family $J_0\subset \sF_j^h$ such that\footnote{Actually the property 1) is not stated in that theorem, however this can be obtained by preselecting a subfamily of maximal balls from $\sF_j^h$
with respect to inclusion and then applying the theorem to the maximal subfamily.}
\begin{itemize}
\item[1)] no ball $B(e^{(4)}(Q))$, with $Q\in J_0$, is contained in any other ball  
$B(e^{(4)}(Q'))$, with $Q'\in \sF_j^h$, $Q'\neq Q$,
\item[2)] the balls $B(e^{(4)}(Q))$, with $Q\in J_0$, have finite superposition, 
and
\item[3)] 
every ball $B(e^{(4)}(Q))$, with $Q\in\sF_j^h$, is contained in some ball 
$(1+8A_0^{-1})\,B(e^{(4)}(R))$, with $R\in J_0$. Consequently,
$$\bigcup_{Q\in\sF_j^h} B(e^{(4)}(Q)) \subset \bigcup_{R\in J_0} (1+8A_0^{-1})\,B(e^{(4)}(R)).$$
\end{itemize}


From the finite superposition property 2), as in the proof of Lemma \ref{lemalg1}, the family 
$J_0$ can be split into $m_0$ subfamilies $J_1,\ldots, J_{m_0}$ so that, for each $k$,  the balls $\{B(e^{(4)}(Q)): Q\in J_k\}$  are pairwise disjoint, with $m_0\leq C(A_0)$.
The condition 3) and Lemma \ref{lem-calcf} applied to $Q$ ensure that
\begin{equation}\label{equni98}
\bigcup_{Q\in\sF_j^h} Q\subset \bigcup_{Q\in\sF_j^h} B(e^{(4)}(Q) )\subset \bigcup_{Q\in J_0} (1+8A_0^{-1})\,B(e^{(4)}(Q)) \subset \bigcup_{Q\in J_0} B(e^{(10)}(Q)).
\end{equation}
Next we choose $\sL_j^h=J_k$ to be the family such that
$$\sum_{Q\in J_k}\sigma(\HD_1(e(Q)))$$
is maximal among $J_1,\ldots,J_{m_0}$, so that
\begin{align*}
\sum_{Q\in \sL_j^h}\sigma(\HD_1(e(Q))) & \geq \frac1{m_0}\,
\sum_{Q\in J_0}\sigma(\HD_1(e(Q)))\\
& \overset{\eqref{eq:sigmae'lesigmae}}{\geq} \frac{1}{m_0\,B^{1/4}} \sum_{Q\in J_0}\sigma(\HD_1(e^{(10)}(Q)))\\
& \overset{\rf{equni98}}{\geq} \frac{1}{m_0\,B^{1/4}} \sum_{Q\in \sF_j^h} \sigma(\HD_1(Q)).
\end{align*}

So we have:

\begin{lemma}\label{lemljh}
The family $\sL_j^h$ satisfies:
\begin{itemize}
\item[(i)] no ball $B(e^{(4)}(Q))$, with $Q\in \sL_j^h$, is contained in any other ball  
$B(e^{(4)}(Q'))$, with $Q'\in \sF_j^h$, $Q'\neq Q$,
\item[(ii)] the balls $B(e^{(4)}(Q))$, with $Q\in \sL_j^h$, are pairwise disjoint, 
and
\item[(iii)] 
 $$\sum_{Q\in \sF_j^h} \sigma(\HD_1(Q)) \lesssim B^{1/4} 
\sum_{Q\in \sL_j^h}\sigma(\HD_1(e(Q))).$$
\end{itemize}
\end{lemma}

\vv
We denote 
$$\sL_j= \bigcup_{h\geq 1}\sL_j^h,\qquad \sL= \bigcup_{j\in\Z}\sL_j =
\bigcup_{j\in\Z}\,\bigcup_{h\geq 1}\sL_j^h.$$
By the property (iii) in the lemma, we have
\begin{align}\label{eqover5}
\sum_{R\in \ttt\cap \MDW}\!\!\sigma(\HD_1(R)) & = \sum_{j\in\Z, \,h\geq0}\,
\sum_{R\in\sF_j^h} \sigma(\HD_1(R)) \\ 
& \lesssim B^{1/4} \sum_{j\in\Z, \,h\geq0}\,\sum_{R\in \sL_j^h}\sigma(\HD_1(e(R))) = B^{1/4}\sum_{R\in \sL}\sigma(\HD_1(e(R))).\nonumber
\end{align}

\vv

\begin{lemma}\label{lemsuper**9}
We have
$$\sigma(\ttt) \lesssim B^{5/4} \sum_{R\in \sL} \sum_{k\geq0} B^{-k/2}\sum_{Q\in\Trc_k(R)}\sigma(\HD_1(e(Q))) + \theta_0^2\,\|\mu\|.$$
\end{lemma}


\begin{proof}
This is an immediate consequence of Lemma \ref{lemtoptop} and our earlier estimates: \todo{the estimate $\sigma(R)\le B\sigma(\HD_1(R))$ was missing, so the constant $B^{1/4}$ had to be changed to $B^{5/4}$}
\begin{align*}
\sigma(\ttt)  & \lesssim \sigma(\ttt\cap\MDW)+ \theta_0^2\,\|\mu\|\\
& \overset{\eqref{eq:MDWdef2}}{\lesssim} B\sum_{R\in \ttt\cap\MDW}\sigma(\HD_1(R)) + \theta_0^2\,\|\mu\|\\
& \overset{\rf{eqover5}}{\lesssim} B^{5/4}\sum_{R\in \sL}\sigma(\HD_1(e(R))) + \theta_0^2\,\|\mu\|\\
& \overset{\eqref{eqiter*44}}{\lesssim} B^{5/4} \sum_{R\in \sL} \sum_{k\geq0} B^{-k/2}\sum_{Q\in\Trc_k(R)}\sigma(\HD_1(e(Q))) + \theta_0^2\,\|\mu\|.
\end{align*}
%Remark that the implicit constants above depend on $m_0$, which in turn depends just on $A_0$ and $n$.
\end{proof}

\vv

To be able to apply later the preceding lemma, we need to get an estimate for $\#\sL(P,k)$, where $P\in\DD_\mu,\, k\ge 0$ and
\begin{equation*}
	\sL(P,k)= \big\{R\in\sL:\exists \,Q\in\Trc_k(R) \mbox{ such that } P\in\TT(e'(Q))\big\}.
\end{equation*}
For $j\in\Z$ set also 
$$\sL_j(P,k)= \big\{R\in\sL_j:\exists \,Q\in\Trc_k(R) \mbox{ such that } P\in\TT(e'(Q))\big\},$$
so that $\sL(P,k) = \bigcup_j \sL_j(P,k)$. The following important technical result is the main achievement in this section.

\begin{lemma}\label{lemimp9}
There exists some constant $C_1$ such that, for all $P\in\DD_\mu$ and all $k\geq0$,
$$\#\sL(P,k)\leq C_1\,\log\Lambda_*.$$
More precisely, for each $P\in\DD_\mu$ and $k\geq0$
\begin{equation}\label{eq:lemimp91}
	\#\{j\in\Z:\sL_j(P,k)\neq\varnothing\}\lesssim \log\Lambda_*,
\end{equation}
and for each $j\in\Z$, $P\in\DD_\mu$, $k\geq0$,
\begin{equation}\label{eqlj83}
	\#\sL_j(P,k) \leq C_2.
\end{equation}
\end{lemma}

\vv

We prove first \eqref{eq:lemimp91}.
\begin{proof}[Proof of \eqref{eq:lemimp91}]
	Let $\wt P_1$ be the smallest $\PP$-doubling cube containing $P$, and let $\wt P_2$ be be the smallest $\PP$-doubling cube strictly containing $\wt P_1$. Suppose that  $R\in\sL_j(P,k)$. There are two cases to consider.
	\vv
	
	\emph{Case 1.} There exists $Q\in\Trc_k(R)$ such that $P\in\TT(e'(Q))\setminus\Neg(e'(Q))$. We claim that in this case we have $\wt P_i\in \TT_\sss(e'(Q))$ for some $i\in\{1,2\}$. Indeed, if $P\in\TT_\sss(e'(Q))\setminus\Neg(e'(Q))$, then  necessarily $\wt P_1\in \TT_\sss(e'(Q))$, by the definition of
	the family $\Neg(e'(Q))$. If $P\notin\TT_\sss(e'(Q)),$ then either $P=\wt P_1\in\End(e'(Q))$, in which case $\wt P_2\in \TT_\sss(e'(Q))$, or $P\neq \wt P_1$ and we have $\wt P_1\in \TT_\sss(e'(Q))$, again by the definition of $\Neg(e'(Q))$.
	
	Choosing $i\in\{1,2\}$ such that $\wt P_i\in \TT_\sss(e'(Q))$ we see by the definition of $\TT_\sss(e'(Q))$ that
	$$\delta_0\,\Theta(Q)\lesssim \Theta(\wt P_i)\leq \Lambda_*^2\,\Theta(Q).$$
	Since $\Theta(Q)=\Lambda_*^k\Theta(R)$ (by the definition of $\Trc_k(R)$), the above is equivalent to
	\begin{equation*}
		\Lambda_*^{-2}\Theta(\wt P_i)\le \Lambda_*^k\Theta(R)\le C\delta_0^{-1}\Theta(\wt P_i)
	\end{equation*}
	We have $\Theta(R)=A_0^{nj}$ because $R\in \sL_j(P,k)$, and so it follows that 
	$$-C\log\Lambda_*\leq j + c\,k\log\Lambda_* - c'\log\Theta(\wt P_i)\leq C|\log\delta_0| = C'\log\Lambda_*.$$
	Recall that $k\ge 0$ is fixed, and $\Theta(\wt P_i)$ is equal to either $\Theta(\wt P_1)$ or $\Theta(\wt P_2)$, where both of these values depend only on $P$, which is fixed. Thus, there are at most $C''\log\Lambda_*$ integers $j$ such that there exists  $R\in\sL_j(P,k)$ and $Q\in\Trc_k(R)$ for which $P\in\TT(e'(Q))\setminus\Neg(e'(Q))$.
	
	\vv
	
	\emph{Case 2.}
	Suppose now that there exists $Q\in\Trc_k(R)$ such that $P\in\Neg(e'(Q))\subset\TT(e'(Q))$.
	In this case, by Lemma \ref{lemnegs}, $\ell(P) \gtrsim \delta_0^{-2}\,\ell(Q)$. Hence, there
	are at most $C\,|\log\delta_0|\approx \log\Lambda$ cubes $Q$ such that 
	$P\in\TT(e'(Q))\cap\Neg(e'(Q))$. 
	
	For each such cube we have $\Theta(Q)=\Lambda_*^k\Theta(R) = \Lambda_*^k A_0^{nj}$. Thus, for each cube $Q$ as above there is exactly one value of $j$ such that there may exist $R\in\sL_j$ with $Q\in\Trc_k(R)$. It follows that there are at most 
	$C'''\log\Lambda_*$ values of $j$ such that
	there exists $R\in\sL_j(P,k)$ and $Q\in\Trc_k(R)$ for which $P\in\TT(e'(Q))\cap\Neg(e'(Q))$.
	
	\vv
	Putting the estimates from both cases together we get that $\sL_j(P,k)$ is non-empty for at most $(C''+C''')\log\Lambda_*$ integers $j$.
\end{proof}

The proof of \eqref{eqlj83} is more involved. Its key ingredient is the following auxiliary result.

\begin{lemma}\label{lemtrucguai}
There exists some positive integer
$N_1$ depending on $n$ (with $N_1\leq C\,N_0 N$) such that the following holds.
For a given $\theta>0$, consider the interval
$$I_\theta = \big(\theta\,\Lambda^{-\frac1{4N}}\delta_0,\, \theta\,\Lambda^{\frac1{4N}}\Lambda_*\big).$$
Let $R_1,R_2,\ldots,R_{N_1}$ be cubes from $\ttt$ such that $R_{k+1}\in\End(R_{k})$ for $k\geq1$. 
Then at least one of the cubes $R_k$, with $1\leq k\le N_1$, satisfies
$$\Theta(R_k)\not \in I_\theta.$$
\end{lemma}

\begin{proof}
Recall that
$$\Lambda_*= \Lambda^{1-\frac1N}\quad\mbox{ and }\quad\delta_0 = \Lambda^{-N_0 - \frac1{2N}},$$
so that
$$I_\theta = \big(\theta\,\Lambda^{-N_0-\frac3{4N}},\, \theta\,\Lambda^{1-\frac3{4N}}\big).$$
Consider a sequence $R_1,R_2,\ldots,R_{N_1}$ of cubes from $\ttt$ such that $R_{k+1}\in\End(R_{k})$ for $k\geq1$. By the definition of $\End(R_k)$ there are only two possibilities: either $R_{k+1}\in\HD(R_k)$, or $R_{k+1}$ is a maximal $\PP$-doubling cube contained in some cube from $\LD(R_k)$. Note that in the former case we have $\Theta(R_{k+1})=\Lambda\Theta(R_k)$, and in the latter case we have $\Theta(R_{k+1})\le C\delta_0\Theta(R_k)$, by \eqref{eqcad35} and the definition of $\LD(R_k)$.

The key observation is the following: 
\begin{equation}\label{eqsist1}
\Theta(R_k)\leq \theta\,\Lambda^{\frac{-1}{3N}} \quad \Rightarrow \quad \mbox{either \;$\Theta(R_{k+1})\not\in I_\theta$ \;or\;
$\Theta(R_{k+1})=\Lambda\,\Theta(R_k)$.}
\end{equation}
This follows from the fact that, in the case $\Theta(R_{k+1})\in I_\theta$, we have 
$R_{k+1}\in\HD(R_k)$ because otherwise 
$$\Theta(R_{k+1})\leq C\delta_0\,\Theta(R_k) \leq C\,\theta\,\delta_0\,\Lambda^{\frac{-1}{3N}}
\leq \theta\,\delta_0\,\Lambda^{\frac{-1}{4N}}
.$$
Analogously, 
\begin{equation}\label{eqsist2}
\Theta(R_k)\geq \theta\,\Lambda^{\frac{-3}{4N}} \quad \Rightarrow \quad \mbox{either \;$\Theta(R_{k+1})\not\in I_\theta$ \;or\;
$\Theta(R_{k+1})\leq C\delta_0\Theta(R_k)$,}
\end{equation}
because, in the case $\Theta(R_{k+1})\in I_\theta$, we have 
$R_{k+1}\not\in\HD(R_k)$,
since otherwise
$$\Theta(R_{k+1}) = \Lambda\,\Theta(R_k) \geq \theta\,\Lambda^{1-\frac3{4N}}
.$$

To prove the lemma, suppose that $\Theta(R_1)\in I_\theta$. Otherwise we are done. 
By applying \rf{eqsist1} $N_0+1$ times, we deduce that either
$\Theta(R_{k})\not\in I_\theta$ for some $k\in(1,N_0+2]$, or there exists some $k_1\in[1,N_0+1]$ such that
$$\Theta(R_{k_1})\geq \theta\,\Lambda^{\frac{-1}{3N}}.$$
Then, from \rf{eqsist2}, we deduce that either 
$\Theta(R_{k_1 + 1})\not\in I_\theta$, or
\begin{equation}\label{eqk111}
\Theta(R_{k_1+1})\leq C\delta_0\,\Theta(R_{k_1})\leq C\,\delta_0\,\theta\,\Lambda^{1-\frac3{4N}} \leq \delta_0\,\theta\,\Lambda^{1-\frac1{3N}}.
\end{equation}

Now we have:

\begin{claim}
Let $k\geq1$ and $a\in (0,1)$. Suppose that
$$\Theta(R_k) \in (\delta_0\,\theta\,\Lambda^{-\frac1{3N}},\,\delta_0\,\theta\,\Lambda^{a}).$$
Then, either there exists some
$k_2\in [k+1,k+N_0+1]$ such that $\Theta(R_{k_2})\not\in I_\theta$, or 
$$\Theta(R_{k+N_0+1}) \in 
(\delta_0\,\theta\,\Lambda^{-\frac1{4N}},\,\delta_0\,\theta\,\Lambda^{a-\frac1{3N}}).$$
\end{claim}

In  case that $a\leq\frac1{12N}$, one should understand that the second alternative is not possible.

\begin{proof}
Suppose that the first alternative in the claim does not hold.
Then we deduce that
$$\Theta(R_{k+j}) = \Lambda^j\,\Theta(R_k)\quad \mbox{ for $j=1,\ldots,N_0$,}$$
because, for all $j=1,\ldots,N_0-1$,
$$\Theta(R_{k+j}) \leq \Lambda^j\,\Theta(R_k)
\leq \delta_0\,\theta\,\Lambda^{N_0-1+a} = \theta\,\Lambda^{\frac{-1}{2N}-1+a}\leq \theta\,\Lambda^{\frac{-1}{2N}},
$$
and then \rf{eqsist1} implies that $\Theta(R_{k+j+1})=\Lambda\,\Theta(R_{k+j})$.
So we infer that
$$\Theta(R_{k+N_0}) = \Lambda^{N_0}\,\Theta(R_k)\geq\delta_0\,\theta\,\Lambda^{-\frac1{4N} + N_0}
= \theta\,\Lambda^{-\frac3{4N}}.
$$
Then, by \rf{eqsist2} we have
\begin{align*}
\Theta(R_{k+N_0+1})& \leq C\delta_0\,\Theta(R_{k+N_0}) = C\delta_0\Lambda^{N_0}\,\Theta(R_k)\\
&
= C\Lambda^{\frac{-1}{2N}}\,\Theta(R_k)
\leq C\Lambda^{\frac{-1}{2N}}\,\delta_0\,\theta\,\Lambda^{a} \leq \delta_0\,\theta\,\Lambda^{a-\frac1{3N}}.
\end{align*}
\end{proof}
\vv

To complete the proof of the lemma observe that, by \rf{eqk111} and a repeated application of the preceding claim, we infer that
 there exists some $k\in [k_1+2,k_1+C N_0\,N]$ such that $\Theta(R_{k})\not\in I_\theta$, since after $CN$ iterations the second alternative in the lemma is not possible. This concludes the proof of the lemma.
\end{proof}

%*********************************************************************


\vvv
We proceed with the proof of \rf{eqlj83} from Lemma \ref{lemimp9}, that is, the estimate $\#\sL_j(P,k) \leq C_2$.
\begin{proof}[Proof of \rf{eqlj83}]
%To shorten notation, we will write here  $\sL$, $\sL_j$ and $\sL_j^h$ instead of $\sL(\GDF)$, $\sL_j(\GDF)$.

%Next, to shorten notation, we write, we will write here  $\sL$, $\sL_j$ and $\sL_j^h$ instead of $\sL(\GDF)$, $\sL_j(\GDF)$.
For $h\ge 1$ set
\begin{equation*}
\sL_j^{h}(P,k):= \sL_j(P,k)\cap \sL_j^{h}.
\end{equation*}
Notice that each family $\sL_j^h(P,k)$ consists of a single cube, at most.
Indeed, we have
\begin{equation}\label{eq:ljh-inclusion}
R\in\sL_j(P,k)\quad\Rightarrow\quad P\subset B(e^{(3)}(R))
\end{equation}
because $P\in e'(Q)$ for some $Q\in\Trc_k(R)$ and $Q\subset B(e''(R))$ by Lemma \ref{eqtec74}. Thus, if $R,R'\in\sL_j^h(P,k)$,
then $B(e^{(3)}(R))\cap B(e^{(3)}(R'))\neq\varnothing$, which can only happen if $R=R'$ (by Lemma \ref{lemljh} (ii)).

Let $R_0$ be a cube in $\sL_j(P,k)$ with maximal side length, and let $h_0$ be such that $R_0\in \sL_j^{h_0}(P,k)$. We will show that $\sL_j^{h_1}(P,k)\neq\varnothing$ implies $h_0\le h_1\le h_0+C_2$. Together with the observation $\#\sL_j^h(P,k)\le 1$ this will conclude the proof of \rf{eqlj83}.

\begin{claim}
Let $R_1\in\sL_j(P,k)\setminus\{R_0\}$, and let $h_1$ be such that $R_1\in \sL_j^{h_1}(P,k)$. 
Then $h_1\geq h_0$.
\end{claim}

\begin{proof}
Suppose that $h_1< h_0$.
Let $R_0^{h_1}$ be the cube that contains $R_0$ and belongs to
$\sF_j^{h_1}$. Observe that
\begin{multline*}
P \overset{ \eqref{eq:ljh-inclusion}}{\subset}  B(e^{(3)}(R_0)) \cap B(e^{(3)}(R_1)) \subset B(x_{R_0},\tfrac32\ell(R_0)) \cap
B(x_{R_1},\tfrac32\ell(R_1))\\
 \subset B(x_{R_0^{h_1}},\tfrac12\ell(R^{h_1}_0) + \tfrac32\ell(R_0)) \cap
B(x_{R_1},\tfrac32\ell(R_1)).
\end{multline*}
So the two balls $B(x_{R_0^{h_1}},\tfrac12\ell(R^{h_1}_0) + \tfrac32\ell(R_0))$ and 
$B(x_{R_1},\tfrac32\ell(R_1))$ have non-empty intersection. Since $\ell(R_0^{h_1})\ge A_0\,\ell(R_0)\ge A_0\,\ell(R_1)$ (the last inequality follows by the choice of $R_0$), we deduce that
\begin{equation*}
B(e^{(4)}(R_1))\subset B(x_{R_1},\tfrac32\ell(R_1)) \subset B(x_{R_0^{h_1}},\tfrac12\ell(R^{h_1}_0) + 5\ell(R_0))
\subset B(e^{(4)}(R_0^{h_1})).
\end{equation*}
However, these inclusions contradict the property (i) of the family $\sL_j^{h_1}$ in Lemma 
\ref{lemljh} because $R_1\neq R_0^{h_1}$.
\end{proof}

\begin{claim}
Let $R_1\in\sL_j(P,k)\setminus\{R_0\}$, and let $h_1$ be such that $R_1\in \sL_j^{h_1}(P,k)$. 
Then 
\begin{equation}\label{eqclaimh1}
h_1\leq h_0+C.
\end{equation}
% or
%\begin{equation}\label{eqsec85}
%\ell(R_1)\approx\ell(P)\quad\mbox{ and }\quad \dist(R_1,P)\lesssim \ell(P).
%\end{equation}
\end{claim}

\begin{proof}
Suppose that $h_1> h_0+1$. This implies that there are cubes
$\{R_1^h\}_{h_0+1\leq h \leq h_1-1}$, with $R_1^h\in\sF_j^h$ for each $h$, such that
$$R_1^{h_0+1}\supsetneq R_1^{h_0+2}\supsetneq\ldots \supsetneq R_1^{h_1-1}\supsetneq R_1^{h_1}=R_1.$$
Observe now that $\ell(R_1^{h_0+1})< \ell(R_0)$. Otherwise, there exists some cube
$R_1^{h_0}\in\sF_j^{h_0}$ that contains $R_1^{h_0+1}$ with 
$$\ell(R_1^{h_0})\geq A_0\,\ell(R_1^{h_0+1})\geq A_0\,\ell(R_0).$$
Since $P \subset  B(e^{(3)}(R_0)) \cap B(e^{(3)}(R_1))$, arguing as in the previous claim, we deduce that
$B(e^{(4)}(R_0))\subset B(e^{(4)}(R_1^{h_0}))$, which contradicts again the property (i) of the family $\sL_j^{h_0}$ in Lemma \ref{lemljh}, as above. So we have
$$\ell(R_1^h)\leq \ell(R_1^{h_0+1})< \ell(R_0)\quad \mbox{ for $h\geq h_0+1$.}$$

By the construction of $\Trc_k(R_0)$, there exists a sequence of cubes
$S_0=R_0, S_1, S_2, \ldots, S_k=Q$ such that 
$$ S_{i+1}\in \GH(S_i)\; \mbox{ for $i=0,\ldots,k-1$,}$$
and $P\in\TT(e'(S_{k}))$. In case that $P$ is contained in some $Q'\in\HD_1(e'(Q))=\HD_1(e'(S_k))$, we write $S_{k+1}=Q'$, and we let $\tilde k:=k+1$. Otherwise, we let $\tilde k:=k$. All in all, we have
\begin{equation}\label{eq:bla1}
	S_{i+1}\in\HD_1(e'(S_i))\quad\text{for $i=0,\dots,\tilde k-1$},
\end{equation}
and $S_{\tilde{k}+1}:=P\subset e'(S_{\tilde k})$ is not strictly contained in any cube from $\HD_1(e'(S_{\tilde k}))$.

Obviously we have  $\ell(S_{i+1})<\ell(S_i)$ for all $i$.
So, for each $h$ with $h_0+1\leq h\leq h_1$ there is some $i=i(h)$ such that 
\begin{equation}\label{eq0asd}
\ell(S_i)>\ell(R_1^h)\geq \ell(S_{i+1}),
\end{equation}
with $0\leq i \leq \tilde k$. 
We claim that either $i\lesssim 1$ or $i=\wt k$, with the implicit constant depending on $n$. 
Indeed, in the case $i<\wt k$, let $T\in\DD_\mu$ be such that $T\supset S_{i+1}$ and $\ell(T)=\ell(R_1^h)$.
Notice that, since $2R_1^h\cap 2T\neq \varnothing$ (because both $2R_1^h$ and $2T$ contain $P$) and 
$\ell(R_1^h)=\ell(T)$, we have
\begin{equation}\label{eq1asd}
\PP(T) \approx \PP(R_1^h)\approx \Theta(R_1^h)=\Theta(R_0),
\end{equation}
where in the last equality we used the definition of $\sL_j$.
On the other hand, 
\begin{equation}\label{eq2asd}
\PP(T)\geq \delta_0\,\Theta(S_i)
\end{equation} 
because otherwise 
$T$ is contained in some cube from $\LD(S_i)$, which would imply that $S_{i+1}$ does not belong to $\HD_1(e'(S_i))$.
Thus, from \rf{eq1asd} and \rf{eq2asd}
we derive that
$$\Theta(R_0)\gtrsim \delta_0\,\Theta(S_i) =  \delta_0\,\Lambda^i\,\Theta(R_0).$$
Hence  $\Lambda^i\lesssim\delta_0^{-1}$, which yields $i\lesssim_n 1$ if $i<\wt k$, as claimed.

The preceding discussion implies that, in order to prove \rf{eqclaimh1}, it suffices to show that, for each fixed 
$i=0,\ldots,\tilde k$, there are at most $C=C(n)$ cubes $R_1^h$ satisfying 
\rf{eq0asd} with this fixed $i$. 

\emph{Case $i<\tilde k$}. Assume first that $i<\tilde k$. Recall that $N_1$ is the constant given by Lemma \ref{lemtrucguai}, and suppose that there exist more than $N_1$ cubes $R_1^h$ satisfying 
\rf{eq0asd}. Since $\{R_1^h\}$ is a nested sequence of cubes, this is equivalent to saying that there exists some $s\in [h_0+1,\, h_1-N_1]$ such that
\begin{equation}\label{eqfam11}
\text{for $h\in[s,s+N_1]$ the cubes $R_1^h$ satisfy \rf{eq0asd}.}
\end{equation}
Taking $\theta=\Theta(S_{i})$, Lemma \ref{lemtrucguai} ensures
that there exists some cube $T\in\ttt$ such that $R_1^s\supset T\supset R_1^{s+N_1}$ which satisfies either
\begin{equation}\label{eqdisju9}
\Theta(T)\leq \Lambda^{-\frac1{4N}}\delta_0\,\Theta(S_i) \qquad\mbox{ or }\qquad 
\Theta(T)\geq \Lambda^{\frac1{4N}}\Lambda_*\,\Theta(S_i).
\end{equation}
Now, let $T'\in\DD_\mu$ be such that $S_{i+1}\subset T'\subset e'(S_{i})$ and $\ell(T')=\ell(T)$, where we use the fact that $\ell(S_i)>\ell(R_1^s)\ge \ell(T)\ge\ell(R_1^{s+N_1})\ge\ell(S_{i+1})$ and $S_{i+1}\subset e'(S_i)$. Notice that
\begin{equation}\label{eqdisju99}
\PP(T')\approx\PP(T)\approx \Theta(T),
\end{equation}
because $2T\cap 2T'\neq\varnothing$.

If the first option in \rf{eqdisju9} holds, we deduce that
$$\PP(T')\leq C \Theta(T)\leq C\Lambda^{-\frac1{4N}}\delta_0\,\Theta(S_i) < \delta_0\,\Theta(S_i).$$
This implies that 
  $T'$ is contained in some cube from $\LD(S_i)\cap\sss_*(e'(S_i))$, which ensures that 
  $S_{i+1}\not\in\HD_1(e'(S_i))$ (notice that we are using the fact that $i<\tilde k$), which is a contradiction with \eqref{eq:bla1}. Thus, $T$ must satisfy the second estimate of \eqref{eqdisju9}. But in this case \rf{eqdisju99} yields $\PP(T')> \Lambda_*\Theta(S_i)$, and so $T'$ is strictly contained in some cube from $\HD_1(e'(S_i))$. Hence, $S_{i+1}\not\in\HD_1(e'(S_i))$, which again gives a contradiction.
  In consequence, if $i<\tilde k$, then \rf{eqfam11} does not hold for any $s$.
 \vv
 
 
\emph{Case $i=\tilde k$}. Assume again that
\rf{eqfam11} holds for some $s\in [h_0+1,\, h_1-N_1]$, and let $s$ be the smallest possible such that \rf{eqfam11} holds. The same argument as above shows that the cube $T'$ from the preceding paragraph is contained in some cube $T''\in\sss_*(e'(S_i))$. Since we assumed that $s$ is minimal, $R_1^s\supset T\supset R_1^{s+N_1}$, and $\ell(T'')\ge\ell(T')=\ell(T)$, we get that there are at most $N_1$ cubes $R_1^h$ satisfying \rf{eq0asd} such that 
$\ell(S_i)>\ell(R_1^h)\geq \ell(T'')$.
We claim that there is also a bounded number of cubes $R_1^h$ such that 
\begin{equation}\label{eqlastclaim5}
\ell(T'')\geq \ell(R_1^h)\geq \ell(P).
\end{equation}
Indeed, by the definition of the family $\End(e'(R))$ and Lemma
 \ref{lemdobpp}, if we denote by $T_m$ the $m$-th descendant of $T''$ which contains $P$, for $m'\geq m\geq 0$,
 it follows that 
$$\PP(T_{m'})\leq A_0^{-|m-m'|/2}\,\PP(T_m)\leq  A_0^{-m/2}\,\PP(T'').$$
Suppose then that there are two cubes $R_1^h$, $R_1^{h'}$ such that  $\ell(R_1^{h'})\leq \ell(R_1^h)\leq \ell(T'')$.
Let $T_m$ and $T_{m'}$ be such that $\ell(R_1^h)=\ell(T_m)$ and $\ell(R_1^{h'})= \ell(T_{m'})$.
By arguments analogous to the ones in \rf{eqdisju99}, we derive that 
$$\PP(T_m)\approx \PP(R_1^h) \approx \Theta(R_0)\quad \text{ and }\quad
\PP(T_{m'})\approx \PP(R_1^{h'}) \approx \Theta(R_0),$$
where we also used the fact that $\Theta(R_1^{h})=\Theta(R_1^{h'})=\Theta(R_0)$.
On the other hand, since $|m-m'|\geq |h-h'|$, we have
$$\Theta(R_0)\approx\PP(T_{m'})\leq A_0^{-|h-h'|/2}\,\PP(T_m)\approx A_0^{-|h-h'|/2}\,\Theta(R_0),$$
which implies that $|h-h'|\lesssim1$. From this fact it follows that there is a bounded number of cubes $R_1^h$ satisfying \rf{eqlastclaim5}, as claimed. Putting all together, we get \rf{eqclaimh1}.
\end{proof}
This ends the proof of Lemma \ref{lemimp9}.
\end{proof}

\vv



%%%%%%%%%%%%%%%%%%%%%%%%%%%%%%%%%%%%%%%%%%%%%%%%%%%%%%%%%%%%%%%%%%%%%%%%%%%%%%%%%%%%%%%%
