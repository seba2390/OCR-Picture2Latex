\section{Introduction}

Given a Radon measure $\mu$ in $\R^{n+1}$, its 
$n$-dimensional Riesz transform at $x\in\R^{n+1}$ is defined by
$$\RR\mu (x) = \int \frac{x-y}{|x-y|^{n+1}}\,d\mu(y),$$
whenever the integral makes sense. For $f\in L^1_{loc}(\mu)$,
we write $\RR_\mu f(x) = \RR(f\mu)(x)$.
Given $\ve>0$, the $\ve$-truncated Riesz transform of $\mu$ equals
$$\RR_\ve\mu (x) = \int_{|x-y|>\ve} \frac{x-y}{|x-y|^{n+1}}\,d\mu(y),$$
and we also write $\RR_{\mu,\ve}f(x) = \RR_\ve(f\mu)(x)$. We say that $\RR_\mu$ is bounded in $L^2(\mu)$ if
the operators $\RR_{\mu,\ve}$ are bounded on $L^2(\mu)$ uniformly in $\ve$, and then we denote
$$\|\RR_\mu\|_{L^2(\mu)\to L^2(\mu)} = \sup_{\ve>0} \|\RR_{\mu,\ve}\|_{L^2(\mu)\to L^2(\mu)}.$$
We also write
$$\RR_*\mu(x) = \sup_{\ve>0} |\RR_{\ve}\mu(x)|, \qquad  \pv\RR\mu(x) = \lim_{\ve>0} \RR_{\ve}\mu(x),$$
in case that the latter limit exists. Remark that, sometimes, abusing notation we will
write $\RR\mu$ instead of $\pv\RR\mu$.

This is the first of a series of two papers where we obtain a geometric characterization
of the measures $\mu$ in $\R^{n+1}$ such that 
the Riesz operator $\RR_\mu$ is bounded in $L^2(\mu)$. 
One of the main motivations for such characterization is the application to the Painlev\'e problem for Lipschitz harmonic functions (i.e., the geometric description of the removable singularities for
Lipschitz harmonic functions).  Also, there may be other
applications regarding 
questions on the approximation by Lipschitz or $C^1$-harmonic functions, as well as to
the study of harmonic measure, which is a field where the Riesz transform has played an important role in recent advances, such as \cite{AHM3TV} and \cite{AMTV}, for example.


To state our results in detail we need
to introduce additional notation.
For a ball $B$ with radius $r(B)$, we consider the density 
$$\theta_\mu(B)=\frac{\mu(B)}{r(B)^n}.$$
Also we define
$$\beta_{\mu,2}(B) = \left(\inf_L \frac1{r(B)^n}\int_{B} \left(\frac{\dist(y,L)}r\right)^2\,d\mu(y)
\right)^{1/2},$$
where the infimum is taken over all affine $n$-planes $L\subset\R^{n+1}$.
For $B=B(x,r)$ we may also write $\theta_\mu(x,r)$ and $\beta_{\mu,2}(x,r)$ instead of $\theta_\mu(B)$
and $\beta_{\mu,2}(B)$.

We consider the following Wolff type energy
\begin{equation}\label{eqEmu0}
\mathbb E(\mu) = \int\!\! \int_0^\infty \left(\frac{\mu(B(x,r))}{r^{n-\frac38}}\right)^2\,\frac{dr}r\,d\mu(x)
= \int\!\! \int_0^\infty r^{\frac3{4}}\,\theta_\mu(x,r)^2\,\frac{dr}{r}\,d\mu(x).
\end{equation}


In this paper we prove the following result.

\begin{theorem}\label{teomain}
	Let $\mu$ be a Radon measure in $\R^{n+1}$ satisfying the polynomial growth condition
	\begin{equation}\label{eqgrow00}
		\mu(B(x,r))\leq \theta_0\,r^n\quad \mbox{ for all $x\in\supp\mu$ and all $r>0$}.
	\end{equation}
	Suppose that there exists some
	constant $M$ such that
	\begin{equation}\label{eqwolff99}
		\mathbb E(\mu\rest_B) \leq M\,r(B)^{\frac3{4}}\,\theta_\mu(2B)^2\,\mu(2B)\quad \mbox{ for any ball $B\subset\R^{n+1}$}.
	\end{equation}
	Suppose also that $\RR_*\mu(x)<\infty$ $\mu$-a.e. Then
	\begin{equation}\label{eqteo*}
		\int \!\!\int_0^\infty \beta_{\mu,2}(x,r)^2\,\theta_\mu(x,r)\,
		\frac{dr}r\,d\mu(x) \leq C\,\big(\|\pv\RR\mu\|_{L^2(\mu)}^2 + \theta_0^2\,\|\mu\|\big),
	\end{equation}
	with $C$ depending on $M$.
\end{theorem}

Remark that, in this theorem, since $\mu$ has polynomial growth of degree $n$ and
$\RR_*\mu(x)<\infty$ $\mu$-a.e., then $\pv\RR\mu(x)$ exists
$\mu$-a.e.\ by \cite{NToV2}. 

A converse to the inequality \rf{eqteo*} also holds: if $\mu$ satisfies the growth condition
\rf{eqgrow00}, then
\begin{equation}\label{eqbetawolff'}
	\|\pv\RR\mu\|_{L^2(\mu)}^2\leq C\,\int\!\!\int_0^\infty \beta_{\mu,2}(x,r)^2\,\theta_\mu(x,r)\,\frac{dr}r\,d\mu(x) 
	+C\,\theta_0^2\,\|\mu\|,
\end{equation}
where $C$ is an absolute constant. This was proved in \cite{Azzam-Tolsa} in the case $n=1$, and in \cite{Girela} in full generality. 

From \rf{eqbetawolff'}, Theorem \ref{teomain}, and a direct application of the $T1$ theorem for non-doubling measures (\cite{NTrV1}, \cite{NTrV2}) we obtain the following.

\begin{theorem}\label{teomain2}
	Let $\mu$ be a Radon measure in $\R^{n+1}$ such that
	\begin{equation}\label{eqwolff99'}
		\mathbb E(\mu\rest_B) \leq M\,r(B)^{\frac3{4}}\,\theta_\mu(2B)^2\,\mu(2B)\quad \mbox{ for any ball $B\subset\R^{n+1}$}
	\end{equation}
	for some fixed constant $M$.
	Then $\RR_\mu$ is bounded in $L^2(\mu)$ if and
	only if it satisfies the polynomial growth condition
	\begin{equation}\label{eqgrow01}
		\mu(B(x,r))\leq C\,r^n\quad \mbox{ for all $x\in\supp\mu$ and all $r>0$}
	\end{equation}
	and
	\begin{equation}\label{eqbetawolff2}
		\int_B\int_0^{r(B)} \beta_{\mu,2}(x,r)^2\,\theta_\mu(x,r)\,\frac{dr}r\,d\mu(x)\leq C^2\,\mu(B)\quad\mbox{ for any ball
			$B\subset\R^{n+1}$.}
	\end{equation}
	Further, $\|\RR_\mu\|_{L^2(\mu)\to L^2(\mu)}$ is bounded above by some constant depending just on $C$ and $M$.
\end{theorem}



Some remarks are in order.
\begin{rem}\label{rem:exp}
	The same results (Theorems \ref{teomain} and \ref{teomain2}) are valid if one replaces the constants $3/8$ and $3/4$ in
	the definition of $\mathbb E(\mu)$ and in \rf{eqwolff99}, \rf{eqwolff99'} by $\alpha/2$ and  $\alpha$, with $\alpha\in(0,1)$. We have chosen $3/8$ and $3/4$ for simplicity.
	
On the other hand, in the case $\alpha=0$, Theorem \ref{teomain} and the ``if'' direction of
Theorem \ref{teomain2} hold trivially because
$$\int \!\!\int_0^\infty \beta_{\mu,2}(x,r)^2\,\theta_\mu(x,r)\,
		\frac{dr}r\,d\mu(x) \leq \int\!\! \int_0^\infty \left(\frac{\mu(B(x,r))}{r^{n}}\right)^2\,\frac{dr}r\,d\mu(x) =: \mathbb E_0(\mu),$$
so that the analog of \rf{eqwolff99}, namely
\begin{equation}\label{eqE000}
\mathbb E_0(\mu\rest_B) \leq M\,\theta_\mu(2B)^2\,\mu(2B)\quad \mbox{ for any ball $B\subset\R^{n+1}$},
\end{equation}
and the polynomial growth condition \rf{eqwolff99'} yield \rf{eqteo*}. The estimate \rf{eqbetawolff2} follows by the same argument. However, the condition \rf{eqE000} is much stronger than \rf{eqwolff99}.
In fact, this does not hold even in the case when $\mu$ is AD-regular (see also Remark \ref{rem333} below). 

One should think of $\mathbb E_0(\mu)$ as the critical Wolff energy in connection with the $L^2(\mu)$ boundedness of $\RR_\mu$, while the energy $\mathbb E(\mu)$ in \rf{eqEmu0} should be considered as a subcrtical energy. 
\end{rem}


\begin{rem}
	In a sense, the condition \rf{eqwolff99} ensures that the
	density of the balls $B(x,r)$ centered in $B$ does not grow too fast as the radius $r$ becomes
	smaller than $r(B)$. One can check easily that the condition
	\begin{equation}\label{eq:denscontr}
	\theta_{\mu}(x,r)\leq C\left(\frac{R}{r}\right)^{3/8}\theta_{\mu}(x,R)\quad\text{for $0<r<R$},
	\end{equation}
	implies \rf{eqwolff99} (with a suitable $M$). As noted in Remark~\ref{rem:exp}, the exponent $3/8$ could be replaced by any parameter strictly smaller than $1/2$, and the results of the paper would still hold for such measures. 
	
	We point out that for domains satisfying the so called capacity density condition (see \cite{AH}) the associated harmonic measure satisfies a similar condition, namely \eqref{eq:denscontr} with exponent $3/8$ replaced by $1-\delta$, for some small $\delta$. Since this exponent is larger than $1/2$, and increasing the exponent makes the condition \eqref{eq:denscontr} weaker, one cannot conclude that this class of measures is covered by our assumption \eqref{eqwolff99}.
\end{rem}

\begin{rem}\label{rem333}
	It is easy to check that the polynomial growth condition on $\mu$ implies that
	$$\mathbb E(\mu\rest_{B})\lesssim \theta_0^2\,r(B)^{3/4}\,\mu(2B).$$
	In particular, if $\mu$ is an AD-regular measure, i.e.\
	$$\mu(B(x,r))\approx r^n\quad \mbox{ for all $x\in\supp\mu$ and $0<r\leq\diam(\supp\mu)$},$$
	then \rf{eqwolff99} holds for a suitable $M$. Further, in this case the statement \rf{eqbetawolff2}
	is equivalent to saying that $\mu$ is uniformly rectifiable, by \cite{DS1}. So in the AD-regular case Theorem \ref{teomain2} reduces to
	the solution of the David-Semmes problem in \cite{NToV1}.
\end{rem}

In fact, we will prove a result more general than Theorem \ref{teomain} which does not
require the condition \rf{eqwolff99}, and instead asserts that, 
under the condition \rf{eqgrow00} and the assumption that $\RR_*\mu(x)<\infty$ $\mu$-a.e.,
\begin{equation}\label{eqhe53}
	\int \!\!\int_0^\infty \beta_{\mu,2}(x,r)^2\,\theta_\mu(x,r)\,
	\frac{dr}r\,d\mu(x) \leq C\,\big(\|\RR\mu\|_{L^2(\mu)}^2 + \theta_0^2\,\|\mu\|
	+ \sum_{Q\in\DD_\mu^\PP\cap\HE}\EE(4Q)\big),
\end{equation}
where $\DD_\mu^\PP\cap\HE$ is a family of ``$\PP$-doubling cubes'' $Q$ from a suitable lattice $\DD_\mu$
with a large Wolff type energy $\EE(4Q)$. See Theorem \ref{propomain} for more details.
In the companion paper \cite{Tolsa-riesz} it is shown that the last sum on right hand side of 
\rf{eqhe53} can be estimated in terms of $\|\pv\RR\mu\|_{L^2(\mu)}$. More precisely,
\begin{equation}\label{eqhe6d3}
	\sum_{Q\in\DD_\mu^\PP\cap\HE}\EE(4Q)\lesssim \|\pv\RR\mu\|_{L^2(\mu)}^2 + \theta_0^2\,\|\mu\|,
\end{equation}
so that combining the results of both papers, one gets:

\begin{theorem*}
	Let $\mu$ be a Radon measure in $\R^{n+1}$ satisfying the polynomial growth condition
	\begin{equation*}
		\mu(B(x,r))\leq \theta_0\,r^n\quad \mbox{ for all $x\in\supp\mu$ and all $r>0$}
	\end{equation*}
	and such that $\RR_*\mu(x)<\infty$ $\mu$-a.e.
	Then
	\begin{equation*}
		\int\!\!\int_0^\infty \beta_{\mu,2}(x,r)^2\,\theta_\mu(x,r)\,\frac{dr}r\,d\mu(x)\leq C\,(\big\|\pv\RR\mu\|_{L^2(\mu)}^2
		+\theta_0^2\,\|\mu\|\big),
	\end{equation*}
	where $C$ is an absolute constant.
\end{theorem*}

Combining also the estimate \rf{eqhe6d3} with Theorem \ref{teomain2}, it turns out that the assumption
\rf{eqwolff99'} can be eliminated in that theorem and then one gets a complete geometric characterization of the measures $\mu$ such that $\RR_\mu$ is bounded in $L^2(\mu)$.

\begin{theorem*}
	Let $\mu$ be a Radon measure in $\R^{n+1}$. Then $\RR_\mu$ is bounded in $L^2(\mu)$ if and
	only if it satisfies the polynomial growth condition
	$$
	\mu(B(x,r))\leq C\,r^n\quad \mbox{ for all $x\in\supp\mu$ and all $r>0$}
	$$
	and
	$$
	\int_B\int_0^{r(B)} \beta_{\mu,2}(x,r)^2\,\theta_\mu(x,r)\,\frac{dr}r\,d\mu(x)\leq C^2\,\mu(B)\quad\mbox{ for any ball
		$B\subset\R^{n+1}$.}
	$$
	Further, the optimal constant $C$ is comparable to $\|\RR_\mu\|_{L^2(\mu)\to L^2(\mu)}$.\end{theorem*}

The preceding result has some important applications. For example, it implies that the class 
of measures $\mu$ such that $\RR_\mu$ is bounded in $L^2(\mu)$ is invariant by bilipschitz maps. That is, given a bilipschitz $T:\R^n\to\R^n$, if $\RR_\mu$ is bounded in $L^2(\mu)$, then $\RR_{T\#\mu}$ is bounded in $L^2(T\#\mu)$, where 
$T\#\mu$ is the image measure of $\mu$ by $T$. As another corollary, one obtains a description of the removable singularities for Lipschitz harmonic functions in terms a metric-geometric potential
involving the $\beta_{\mu,2}$ coefficients, and one deduces that the class of sets which are removable
is invariant by bilipschitz mappings. 
These results can be considered the extension to higher dimensions of the results from 
\cite{Tolsa-bilip} in connection with analytic capacity.
See \cite{Tolsa-riesz} for more details.

Next we will describe the main ideas involved in the proof of Theorem \ref{teomain}, as well as the 
main difficulties and innovations.
The strategy consists in performing a corona decomposition of the dyadic lattice $\DD_\mu$ into trees of cubes where the density of the cubes does not oscillate too much. Then, roughly speaking,  in each tree the measure
$\mu$ behaves as if it were AD-regular, and from the $L^2(\mu)$ boundedness of $\RR_\mu$ and
\cite{NToV1}, one should expect that $\mu$ is close to some uniformly rectifiable measure at the locations and scales of the cubes in the tree, so that one can obtain a good packing condition
for the $\beta_{\mu,2}$ coefficients of the cubes in the tree.
For this strategy to work, we need to show that the roots of the trees where the density does
not oscillate too much satisfy a suitable packing condition. This is the content of the Main Lemma
\ref{mainlemma}, whose proof takes most of the paper (Sections \ref{sec4}-\ref{sec8}).

To prove the desired packing condition, we reduce matters to obtaining good lower estimates for 
the Haar coefficients of $\RR\mu$ for the cubes of some suitable trees
(the so-called tractable trees) where, in a sense the density of many cubes in some intermediate generations increases (with respect to the density of the root), and later in many stopping cubes the density decreases. These lower estimates are obtained by a variational argument applied to some measure $\eta$ that approximates $\mu$ at the locations and scales of the cubes in the tractable tree.
By that variational argument one obtains some lower bounds for $\|\RR\eta\|_{L^2(\eta)}$
that later are transferred to $\RR\mu$ (i.e., to the Haar coefficients of $\RR\mu$ 
for the cubes in the tree). The idea of applying a variational argument like this
originates from the work \cite{ENVo} by Eiderman, Nazarov, and Volberg and the reduction
to the tractable trees comes from the work \cite{Reguera-Tolsa} by Reguera and the second author of 
this paper. The article \cite{JNRT} includes an improved version of that variational argument.
Unlike the present paper, \cite{JNRT} makes an extensive use of compactness arguments, which
do not work so well in our situation, where the geometry plays a more important role.

The implementation of the variational argument and the transference of the estimates from the approximating measure $\eta$ to $\mu$ is more difficult in the present
paper than in \cite{Reguera-Tolsa} or in other related works such as \cite{JNRT}. Some of the
difficulties arise from the fact that, for technical reasons (essentially, we need that many cubes of 
the intermediate generations with high density are located far from the boundary of the root of the tree), we have to consider trees of ``enlarged cubes''. This causes an overlapping between different trees that
has to be quantified carefully (this is done in Section \ref{sec-layers}). 
On the other hand, the transference of the lower estimate for $\|\RR\eta\|_{L^2(\eta)}$ 
to the Haar coefficients of $\RR\mu$ for the cubes in the tree originates many error terms. Roughly speaking, in
order to be able to transfer that lower bound for $\|\RR\eta\|_{L^2(\eta)}$ to $\mu$ we need
the error terms to be smaller than the lower bound of $\|\RR\eta\|_{L^2(\eta)}$. Some of these
error terms are difficult to handle and we only can show that they are small under the condition \rf{eqwolff99}. If this condition is not assumed, then we can bound them in terms of the 
energies $\EE(4Q)$ that appear in \rf{eqhe6d3}. For this to work, we need 
an enhanced version of the dyadic lattice of David and Mattila that is obtained in Section
\ref{sec5}. This is an essential tool for our arguments.

As explained above, the last stage of the proof of Theorem \ref{teomain} consists of estimating the 
$\beta_{\mu,2}$ coefficients in each tree where the density does not oscillate too much.
This step, which requires a delicate approximation by an AD-regular measure which has its own interest,
is performed in Section~\ref{sec9}.

Throughout the proof a large number of parameters and families of cubes is defined. To help the reader navigate the paper and keep track of different objects, we list most of them in Appendices~\ref{app:param} and \ref{app:fam}.

\vv

In the whole paper we denote by $C$ or $c$ some constants that may depend on the dimension and perhaps other fixed parameters. Their value may change at different occurrences. On the contrary, constants with subscripts, like $C_0$, retain their values.
For $a,b\geq 0$, we write $a\lesssim b$ if there is $C>0$ such that $a\leq Cb$. We write $a\approx b$ to mean $a\lesssim b\lesssim a$. 

\vv

% ***************************************************************************