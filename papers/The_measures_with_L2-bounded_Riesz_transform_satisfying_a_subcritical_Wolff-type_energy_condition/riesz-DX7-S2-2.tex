\section{The modified dyadic lattice of David and Mattila}\label{sec:DMlatt}\label{sec5}

Next we will introduce the dyadic lattice of cubes
with small boundaries of David-Mattila \cite{David-Mattila} associated with a Radon measure $\mu$. The properties of the lattice are summarized in the next lemma. Later on we will show how its construction
can be modified in order to obtain addtional properties relevant for our arguments.



\begin{lemma}[David, Mattila]
	\label{lemcubs}
	Let $\mu$ be a compactly supported Radon measure in $\R^{d}$.
	Consider two constants $C_0>1$ and $A_0>5000\,C_0$ and denote $E=\supp\mu$. 
	Then there exists a sequence of partitions of $E$ into
	Borel subsets $Q$, $Q\in \DD_{\mu,k}$, with the following properties:
	\begin{itemize}
		\item For each integer $k\geq0$, $E$ is the disjoint union of the ``cubes'' $Q$, $Q\in\DD_{\mu,k}$, and
		if $k<l$, $Q\in\DD_{\mu,l}$, and $R\in\DD_{\mu,k}$, then either $Q\cap R=\varnothing$ or else $Q\subset R$.
		\vv
		
		\item The general position of the cubes $Q$ can be described as follows. For each $k\geq0$ and each cube $Q\in\DD_{\mu,k}$, there is a ball $B(Q)=B(x_Q,r(Q))$ such that
		$$x_Q\in E, \qquad A_0^{-k}\leq r(Q)\leq C_0\,A_0^{-k},$$
		$$E\cap B(Q)\subset Q\subset E\cap 28\,B(Q)=E \cap B(x_Q,28r(Q)),$$
		and
		$$\mbox{the balls\, $5B(Q)$, $Q\in\DD_{\mu,k}$, are disjoint.}$$
		
		\vv
		\item The cubes $Q\in\DD_{\mu,k}$ have small boundaries. That is, for each $Q\in\DD_{\mu,k}$ and each
		integer $l\geq0$, set
		$$N_l^{ext}(Q)= \{x\in E\setminus Q:\,\dist(x,Q)< A_0^{-k-l}\},$$
		$$N_l^{int}(Q)= \{x\in Q:\,\dist(x,E\setminus Q)< A_0^{-k-l}\},$$
		and
		$$N_l(Q)= N_l^{ext}(Q) \cup N_l^{int}(Q).$$
		Then
		\begin{equation}\label{eqsmb2}
			\mu(N_l(Q))\leq (C^{-1}C_0^{-3d-1}A_0)^{-l}\,\mu(90B(Q)).
		\end{equation}
		\vv
		
		\item Denote by $\DD_{\mu,k}^{db}$ the family of cubes $Q\in\DD_{\mu,k}$ for which
		\begin{equation}\label{eqdob22}
			\mu(100B(Q))\leq C_0\,\mu(B(Q)).
		\end{equation}
		We have that $r(Q)=A_0^{-k}$ when $Q\in\DD_{\mu,k}\setminus \DD_{\mu,k}^{db}$
		and
		\begin{equation}\label{eqdob23}
			\mu(100B(Q))\leq C_0^{-l}\,\mu(100^{l+1}B(Q))\quad
			\mbox{for all $l\geq1$ with $100^l\leq C_0$ and $Q\in\DD_{\mu,k}\setminus \DD_{\mu,k}^{db}$.}
		\end{equation}
	\end{itemize}
\end{lemma}

\vv

\begin{rem}\label{rema00}
	The constants $C_0$ and $A_0$ are chosen so that
	$$A_0 = C_0^{C(d)},$$
	where $C(d)$ depends
	%\footnote{Probably $C(d)=10d^{20}$ works, but this should be checked.}
	just on $d$ and $C_0$ is big enough.
\end{rem}

We use the notation $\DD_\mu=\bigcup_{k\geq0}\DD_{\mu,k}$. Observe that the families $\DD_{\mu,k}$ are only defined for $k\geq0$. So the diameter of the cubes from $\DD_\mu$ are uniformly
bounded from above.
%For $Q\in\DD$, we set $\DD(Q) =\{P\in\DD:P\subset Q\}$.
%Given $Q\in\DD_k$, we denote $J(Q)=k$. 
We set
$\ell(Q)= 56\,C_0\,A_0^{-k}$ and we call it the side length of $Q$. Notice that 
$$C_0^{-1}\ell(Q)\leq \diam(28B(Q))\leq\ell(Q).$$
Observe that $r(Q)\approx\diam(Q)\approx\ell(Q)$.
Also we call $x_Q$ the center of $Q$, and the cube $Q'\in \DD_{\mu,k-1}$ such that $Q'\supset Q$ the parent of $Q$.
We denote the family of cubes from $\DD_{\mu,k+1}$ which are contained in $Q$ by $\Ch(Q)$, and we call their elements children or sons of $Q$.
We set
$B_Q=28 B(Q)=B(x_Q,28\,r(Q))$, so that 
$$E\cap \tfrac1{28}B_Q\subset Q\subset B_Q\subset B(x_Q,\ell(Q)/2).$$

For a given $\gamma\in(0,1)$, let $A_0$ be big enough so that the constant $C^{-1}C_0^{-3d-1}A_0$ in 
\rf{eqsmb2} satisfies 
$$C^{-1}C_0^{-3d-1}A_0>A_0^{\gamma}>10.$$
Then we deduce that, for all $0<\lambda\leq1$,
\begin{align}\label{eqfk490}
	\mu\bigl(\{x\in Q:\dist(x,E\setminus Q)\leq \lambda\,\ell(Q)\}\bigr) + 
	\mu\bigl(\bigl\{x\in 3.5B_Q\setminus Q:\dist&(x,Q)\leq \lambda\,\ell(Q)\}\bigr)\\
	&\leq_\gamma
	c\,\lambda^{\gamma}\,\mu(3.5B_Q).\nonumber
\end{align}

We denote
$\DD_\mu^{db}=\bigcup_{k\geq0}\DD_{\mu,k}^{db}$.
Note that, in particular, from \rf{eqdob22} it follows that
\begin{equation}\label{eqdob*}
	\mu(3B_{Q})\leq \mu(100B(Q))\leq C_0\,\mu(Q)\qquad\mbox{if $Q\in\DD_\mu^{db}.$}
\end{equation}
For this reason we will call the cubes from $\DD_\mu^{db}$ doubling. 
Given $Q\in\DD_\mu$, we denote by $\DD_\mu(Q)$
the family of cubes from $\DD_\mu$ which are contained in $Q$. Analogously,
we write $\DD_\mu^{db}(Q) = \DD^{db}_\mu\cap\DD(Q)$. 


As shown in \cite[Lemma 5.28]{David-Mattila}, every cube $R\in\DD_\mu$ can be covered $\mu$-a.e.\
by a family of doubling cubes:
\vv

\begin{lemma}\label{lemcobdob}
	Let $R\in\DD_\mu$. Suppose that the constants $A_0$ and $C_0$ in Lemma \ref{lemcubs} are
	chosen as in Remark \ref{rema00}. Then there exists a family of
	doubling cubes $\{Q_i\}_{i\in I}\subset \DD_\mu^{db}$, with
	$Q_i\subset R$ for all $i$, such that their union covers $\mu$-almost all $R$.
\end{lemma}

The following result is proved in \cite[Lemma 5.31]{David-Mattila}.
\vv

\begin{lemma}\label{lemcad22}
	Let $R\in\DD_\mu$ and let $Q\subset R$ be a cube such that all the intermediate cubes $S$,
	$Q\subsetneq S\subsetneq R$ are non-doubling (i.e.\ belong to $\DD_\mu\setminus \DD_\mu^{db}$).
	Suppose that the constants $A_0$ and $C_0$ in Lemma \ref{lemcubs} are
	chosen as in Remark \ref{rema00}. 
	Then
	\begin{equation}\label{eqdk88}
		\mu(100B(Q))\leq A_0^{-10d(J(Q)-J(R)-1)}\mu(100B(R)).
	\end{equation}
\end{lemma}
\vv
%Let us remark that the constant $10$ in \rf{eqdk88} can be replaced by any other positive 
%constant if $A_0$ and $C_0$ are chosen suitably in Lemma \ref{lemcubs}, as shown in (5.30) of
%\cite{David-Mattila}.

%Recall that, given a ball (or an arbitrary set) $B\subset \R^{n+1}$, we consider its $n$-dimensional density:
%$$\theta_\mu(B)= \frac{\mu(B)}{\diam(B)^n}.$$
%We will also write $\theta_\mu(x,r)$ instead of $\theta_\mu(B(x,r))$.

From this lemma we deduce:

\vv
\begin{lemma}\label{lemcad23}
	Let $Q,R\in\DD_\mu$ be as in Lemma \ref{lemcad22}.
	Then
	$$\theta_\mu(100B(Q))\leq (C_0A_0)^{n+1}\,A_0^{-9d(J(Q)-J(R)-1)}\,\theta_\mu(100B(R))$$
	and
	$$\sum_{S\in\DD_\mu:Q\subset S\subset R}\theta_\mu(100B(S))\leq c\,\theta_\mu(100B(R)),$$
	with $c$ depending on $C_0$ and $A_0$.
\end{lemma}

For the easy proof, see
\cite[Lemma 4.4]{Tolsa-memo}, for example.

For $f\in L^2(\mu)$ and $Q\in\DD_\mu$ we define
\begin{equation}\label{eqdq1}
	\Delta_Q f=\sum_{S\in\Ch(Q)}m_{\mu,S}(f)\chi_S-m_{\mu,Q}(f)\chi_Q,
\end{equation}
where $m_{\mu,S}(f)$ stands for the average of $f$ on $S$ with respect to $\mu$.
Then we have the orthogonal expansion, for any cube $R\in\DD_\mu$,
$$\chi_{R} \bigl(f - m_{\mu,R}(f)\bigr) = \sum_{Q\in\DD_\mu(R)}\Delta_Q f,$$
in the $L^2(\mu)$-sense, so that
$$\|\chi_{R} \bigl(f - m_{\mu,R}(f)\|_{L^2(\mu)}^2 = \sum_{Q\in\DD_\mu(R)}\|\Delta_Q f\|_{L^2(\mu)}^2.$$

In this paper we will have to estimate terms such as $\|\RR(\chi_{Q}\mu)\|_{L^2(\mu\rest_{2B_Q\setminus Q})}$, 
which leads to deal with integrals of the form
$$\int_{2B_Q\setminus Q}\left(\int_Q \frac1{|x-y|^n}\,d\mu(y)\right)^2 d\mu(x).$$
Our next objective is to show that integrals such as this one can be estimated in terms of the Wolff type energy $\EE(2Q)$, to be defined soon.

We need some additional notation.\todo{the definition of $\lambda Q$ was above Lemma 2.8, but $2Q$ appears already in Lemma 2.6, so I moved the definition}
Given $Q\in\DD_\mu$ and $\lambda>1$, we denote by $\lambda Q$ the union of cubes $P$ from the same
generation as $Q$ such that $\dist(x_Q,P)\leq \lambda \,\ell(Q)$. Notice that
\begin{equation}\label{eqlambq12}
	\lambda Q\subset B(x_Q,(\lambda+\tfrac12)\ell(Q)).
\end{equation}
Also, we let
$$\DD_\mu(\lambda Q)=\{P\in\DD_\mu:P\subset \lambda Q,\,\ell(P)\leq \ell(Q)\},$$
and, for $k\geq0$,
$$\DD_{\mu,k}(\lambda Q) =\{P\in\DD_\mu:P\subset \lambda Q,\,\ell(P)=A_0^{-k} \ell(Q)\},\qquad
\DD_\mu^k(\lambda Q) = \bigcup_{j\geq k} \DD_{\mu,j}(\lambda Q).
$$
\vv

%For a given $Q\in\DD_\mu$ and $\lambda\geq1$, we denote $\ell(\lambda Q) = \lambda\,\ell(Q)$ and
%we consider the ``density''
%$$\Theta_\mu(Q)=\frac{\mu(2 B_Q)}{\ell(Q)^n}.$$
%Clearly, $\Theta_\mu(Q)\approx_{A_0,C_0}\theta_\mu(2B_Q)$.

\begin{lemma}\label{lemDMimproved}
	Let $\mu$ be a compactly supported Radon measure in $\R^{d}$.
	Assume that $\mu$ has polynomial growth of degree $n$ and let $\gamma\in(0,1)$. The lattice $\DD_\mu$ from Lemma
	\ref{lemcubs} can be constructed so that the following holds for all
	all $Q\in\DD_{\mu}$:
	\begin{align*}
		\int_{2B_Q\setminus Q}\left(\int_Q \frac1{|x-y|^n}\,d\mu(y)\right)^2 d\mu(x) 
		+ &\int_{Q}\left(\int_{2B_Q\setminus Q} \frac1{|x-y|^n}\,d\mu(y)\right)^2 d\mu(x)\\
		&\leq C(\gamma)\sum_{P\in\DD_\mu: P\subset 2Q} \left(\frac{\ell(P)}{\ell(Q)}\right)^\gamma\theta_\mu(2B_P)^2\mu(P).
	\end{align*}
\end{lemma}


Remark that the polynomial growth assumption is just necessary to ensure that some of the integrals above are finite. In fact, the constant $C(\gamma)$ does not depend on the polynomial growth constant. 

To prove the lemma, we denote
\begin{equation}\label{eqdmuint}
	\wt \DD_\mu^{int}(Q) = \big\{P\in\DD_\mu(Q):2B_P\cap (\supp\mu\setminus Q)\neq \varnothing\big\}
\end{equation}
and
\begin{equation}\label{eqdmuext}
	\wt \DD_\mu^{ext}(Q) = \big\{P\in\DD_\mu:\ell(P)\leq \ell(Q),P\subset \R^{n+1}\setminus Q,\,2B_P\cap Q\neq \varnothing\big\}.
\end{equation}
Also,
\begin{equation}\label{eqdmutot}
	\wt \DD_\mu(Q) = \wt \DD_\mu^{int}(Q) \cup \wt \DD_\mu^{ext}(Q),
\end{equation}
and, for $k\geq0$,\todo{I removed "$P\subset \lambda Q$" from the definition of $\wt \DD_{\mu,k}$, I think this was an artefact}
$$\wt \DD_{\mu,k}(Q) = 
\{P\in\wt \DD_\mu:\ell(P)=  A_0^{-k}\ell(Q)\}.$$

We need some auxiliary results. The first one is the following.

\begin{lemma}\label{lemDMimproved2}
	Let $\mu$ be a compactly supported Radon measure in $\R^{d}$ and $Q\in\DD_\mu$. For any  $\alpha\in(0,1)$, we have
	\begin{align}\label{eqdosint}
		\int_{2B_Q\setminus Q}\left(\int_Q \frac1{|x-y|^n}\,d\mu(y)\right)^2d\mu(x) \,
		+ &\int_{Q}\left(\int_{2B_Q\setminus Q} \frac1{|x-y|^n}\,d\mu(y)\right)^2d\mu(x) \\
		& \lesssim_{\alpha,A_0}
		\sum_{P\in\wt \DD_\mu(Q)} \left(\frac{\ell(Q)}{\ell(P)}\right)^\alpha\theta_\mu(2B_P)^2
		\,\mu(P).\nonumber
	\end{align}
\end{lemma}


\begin{proof}
	Observe that, for $x\in 2B_Q\setminus Q$,
	\begin{align*}
		\int_Q\frac1{|x-y|^n}\,d\mu(y) & = \left(\int_{y\in Q:|x-y|\geq r(B_Q)/2} + \int_{y\in Q:|x-y|< r(B_Q)/2}\right)
		\frac1{|x-y|^n}\,d\mu(y)\\
		&\lesssim_{A_0} \frac{\mu(Q)}{r(B_Q)^n} + \sum_{P\in\wt \DD_\mu^{ext}(Q):x\in P} \theta_\mu(2B_P).
	\end{align*}
	Thus,
	\begin{multline*}
		\int_{2B_Q\setminus Q}\left(\int_Q \frac1{|x-y|^n}\,d\mu(y)\right)^2d\mu(x) \\
		\lesssim_{A_0} \left(\frac{\mu(Q)}{r(B_Q)^n}\right)^2\,\mu(2B_Q) +
		\int_{2B_Q\setminus Q}\bigg(\sum_{P\in\wt \DD_\mu^{ext}(Q):x\in P} \theta_\mu(2B_P)\bigg)^2d\mu(x).
	\end{multline*}
	By H\"older's inequality, for any $\alpha>0$,
	\begin{align*}
		\bigg(\sum_{P\in\wt \DD_\mu^{ext}(Q):x\in P} &\theta_\mu(2B_P)\bigg)^2\\
		& \leq \bigg(\sum_{P\in\wt \DD_\mu^{ext}(Q):x\in P} \left(\frac{\ell(Q)}{\ell(P)}\right)^\alpha\theta_\mu(2B_P)^2\bigg) \cdot \bigg(\sum_{P\in\wt \DD_\mu^{ext}(Q):x\in P} \left(\frac{\ell(P)}{\ell(Q)}\right)^{\alpha}\bigg).
	\end{align*}
	The last sum above is bounded above by
	$$\sum_{P\in\DD_\mu:x\in P,P\subset 2Q} \left(\frac{\ell(P)}{\ell(Q)}\right)^{\alpha}\leq C(\alpha).
	$$
	Therefore,
	\begin{align*}
		\int_{2B_Q\setminus Q}& \left(\int_Q \frac1{|x-y|^n}\,d\mu(y)\right)^2d\mu(x) \\
		& \lesssim_{\alpha,A_0} \frac{\mu(Q)^2\,\mu(2B_Q)}{r(B_Q)^{2n}}+
		\int_{2B_Q\setminus Q} \sum_{P\in\wt \DD_\mu^{ext}(Q):x\in P} \left(\frac{\ell(Q)}{\ell(P)}\right)^\alpha\theta_\mu(2B_P)^2\,d\mu(x)\\
		& \lesssim  \theta_\mu(2B_Q)^2\,\mu(Q)  +
		\sum_{P\in\wt \DD_\mu^{ext}(Q)} \left(\frac{\ell(Q)}{\ell(P)}\right)^\alpha\theta_\mu(2B_P)^2
		\,\mu(P).
	\end{align*}
	
	The estimate of the second integral on the left hand side of \rf{eqdosint} is analogous.
\end{proof}
\vv

\begin{lemma}\label{lemrecur5}
	Let $\mu$ be a compactly supported Radon measure in $\R^{d}$ and let $\gamma\in (0,1)$.
	Assume that $\mu$ has polynomial growth of degree $n$ and let $\gamma\in(0,1)$. The lattice $\DD_\mu$ from Lemma
	\ref{lemcubs} can be constructed so that the following holds for all
	all $Q\in\DD_{\mu}$:
	\begin{equation}\label{eqfir5}
		\sum_{S\in\wt \DD_{\mu,1}(Q)} \sum_{P\in\DD_\mu(2S)} \left(\frac{\ell(P)}{\ell(Q)}\right)^\gamma\theta_\mu(2B_P)^2\,\mu(P)\lesssim
		C_0^{6d+1}A_0^{-1}\!\!
		\sum_{P\in \DD_\mu^1(2Q)}\left(\frac{\ell(P)}{\ell(Q)}\right)^\gamma \theta_\mu(2B_P)^2\,\mu(P).
	\end{equation} 
\end{lemma}


\begin{proof}
	We will describe the relevant changes required in the arguments in \cite[Theorem 3.2]{David-Mattila} in order to get the estimate \rf{eqfir5}. We will
	use the same notation as in that theorem, with the exception of the constant $A$ in \cite[Theorem 3.2]{David-Mattila}, which here we denote by $A_0$. 
	
	Denote $E=\supp\mu$.
	For each generation $k\geq 0$, the starting point to construct $\DD_{\mu,k}$
	consists of choosing, for each $x\in E$, a suitable radius $r^k(x)$ such that
	\begin{equation}\label{eqrk1}
		A_0^{-k}\leq r^k(x)\leq C_0 A_0^{-k}
	\end{equation}
	depending on the doubling properties of the ball $B(x,r^k(x))$ (see \cite[(3.17)-(3.20)]{David-Mattila}).
	Next, one chooses two auxiliary radii $r_1^k(x)$ and $r_2^k(x)$ such that
	$$\frac{11}{10}\,r^k(x) < r_1^k(x)<\frac{12}{10}\,r^k(x),$$
	$$25\,r^k(x) < r_2^k(x)<26\,r^k(x),$$
	and such that the following small boundary conditions hold:
	\begin{equation}\label{eqthin1*}
		\mu\big(\big\{y\in\R^{d}\!: \dist(y,\partial B(x,r_1^k(x)))\leq \tau\,r^k(x)\big\}\big)
		\leq C\tau\,\mu\big(B(x,\tfrac{13}{10}r^k(x))\big)\quad
		\mbox{for $0<\tau < \tfrac1{10}$,}
	\end{equation}
	and
	\begin{equation}\label{eqthin2*}
		\mu\big(\big\{y\in\R^{d}\!: \dist(y,\partial B(x,r_2^k(x)))\leq \tau\,r^k(x)\big\}\big)
		\leq C\tau\,\mu\big(B(x,27r^k(x))\big)\quad
		\mbox{for $0<\tau < 1$.}
	\end{equation}
	
	At this point we will require the auxiliary radii $r_1^k(x)$ and $r_2^k(x)$ to be chosen so that an additional condition holds. Set $A(x,r,R) = B(x,R)\setminus B(x,r).$ 
Observe first that
\begin{align}\label{eqcalr45}
\int_{\tfrac{11}{10}r^k(x)}^{\tfrac{12}{10}r^k(x)}  &\sum_{j\geq0}  A_0^{-\gamma(j+1)} \int_{A(x,\,t-300C_0A_0^{-k-1},\,t+300C_0A_0^{-k-1})}
 \theta_\mu(y,112C_0A_0^{-k-j-1})^2\,d\mu(y)\,dt\\
& \leq\sum_{j\geq0}  A_0^{-\gamma(j+1)} \int_{B(x,\tfrac{12}{10}r^k(x)+300C_0A_0^{-k-1})}
 \theta_\mu(y,112C_0A_0^{-k-j-1})^2\,\LL^1(I_{x,y,k})\,d\mu(y),\nonumber
\end{align} 
where we applied Fubini and we denoted by $I_{x,y,k}$ the interval
\begin{align*}
I_{x,y,k} & = \{t\in\R:t-300C_0A_0^{-k-1}\leq |x-y|\leq t+300C_0A_0^{-k-1}\}\\
&= \big[|x-y|-300C_0A_0^{-k-1},|x-y|+300C_0A_0^{-k-1}\big].
\end{align*}
Obviously, its Lebesgue measure is $\LL^1(I_{x,y,k})= 600C_0A_0^{-k-1}$, and so the left hand side of \rf{eqcalr45}
is bounded above by
$$600 \,C_0\,A_0^{-k-1}\sum_{j\geq0} A_0^{-\gamma(j+1)} \!\! \int_{B(x,\tfrac{13}{10}r^k(x))}
 \theta_\mu(y,112C_0A_0^{-k-j-1})^2 \,d\mu(y).
$$  
	Thus, by Chebyshev, the set $U_{1}^k\subset\R$ of those $t\in [\tfrac{11}{10}r^k(x),\tfrac{12}{10}r^k(x)]$ such that
	\begin{align*}
		\sum_{j\geq0}  A_0^{-\gamma(j+1)} &\int_{A(x,t-300C_0A_0^{-k-1},t+300C_0A_0^{-k-1})}
		\theta_\mu(y,112C_0A_0^{-k-j-1})^2\,d\mu(y)\\
		&> \frac{10^5\,C_0\,A_0^{-k-1}}{r^k(x)}
		\sum_{j\geq0} A_0^{-\gamma(j+1)} 
		\int_{B(x,\tfrac{13}{10}r^k(x))}
		\theta_\mu(y,112C_0A_0^{-k-j-1})^2 \,d\mu(y)
	\end{align*}
	satisfies 
	$$|U_{1}^k|\leq \frac1{100}\,r^k(x).$$
	By a standard argument involving the  boundedness of the maximal Hardy-Littlewood operator 
	from $L^1(\R)$ to $L^{1,\infty}(\R)$, one can deduce that there exists some 
	$$r_1^k(x)\in [\tfrac{11}{10}r^k(x),\tfrac{12}{10}r^k(x)]\setminus
	U_{1}^k
	$$ 
	such that \rf{eqthin1*} holds. The fact that $r_1^k(x)\not\in\ U_{1}^k$
	ensures that
	\begin{align}\label{eqal848}
		\sum_{j\geq0}  A_0^{-\gamma(j+1)} &\int_{A(x,r_1^k(x)-300C_0A_0^{-k-1},r_1^k(x)+300C_0A_0^{-k-1})}
		\theta_\mu(y,112C_0A_0^{-k-j-1})^2\,d\mu(y)\\
		&\leq \frac{10^5\,C_0\,A_0^{-k-1}}{r^k(x)}
		\sum_{j\geq0} A_0^{-\gamma(j+1)} 
		\int_{B(x,\tfrac{13}{10}r^k(x))}
		\theta_\mu(y,112C_0A_0^{-k-j-1})^2 \,d\mu(y).
		\nonumber
	\end{align}
	An analogous argument shows that 
	$r^k_2(x)$ can be taken such that, besides \rf{eqthin2*}, the 
	preceding estimate also holds with $r_1^k(x)$ replaced
	by $r_2^k(x)$ and $B(x,\tfrac{13}{10}r^k(x))$ replaced by $B(x,27r^k(x))$.
	
	As in \cite[Theorem 3.2]{David-Mattila}, we denote $B_1^k(x) = B(x,r_1^k(x))$ and $B_2^k(x) = B(x,r_2^k(x))$, and by a Vitali type covering lemma we select a family of points $x\in I^k$
	such that the balls $\{B(x,5r^k(x))\}_{x\in I^k}$ are disjoint, while the balls $\{B(x,25r^k(x))\}_{x\in I^k}$ cover $E$. We also denote
	$$B_3^k(x) =B_2^k(x)\setminus \bigg(\bigcup_{y\in I^k\setminus\{x\}} B_1^k(y)\bigg).$$
	For $x\in I_k$, let $J(x)$ be the family of those
	$y\in I^k\setminus\{x\}$ such that $B_1^k(y)\cap B_2^k(x)\neq
	\varnothing$. As explained in \cite[Theorem 3.2]{David-Mattila}, using \rf{eqrk1} it is easy to check
	$\# J(x) \leq C C_0^d$.
	
	Next we consider an order in $I^k$ such that
	$$\mbox{$y<x$ in $I^k$ whenever $\mu(B(x,90r^k(x)))<\mu(B(y,90r^k(y)))$}$$
	and we define
	$$B_4^k(x) = B_3^k(x) \setminus \bigg(\bigcup_{y\in I^k:y<x} B_3^k(y)\bigg).$$
	Again, as explained in \cite[Theorem 3.2]{David-Mattila}, using \rf{eqrk1} it is easy to check
	that, for each $x\in I^k$, there are at most $C C_0^{n+1}$ sets $B_3^k(y)$ that intersect $B_3^k(x)$, with 
	$y\in I^k$.
	
	The family $\{B_4^k(x)\}_{x\in I^k}$ is a first approximation to $\{Q\}_{Q\in\DD_\mu^k}$.
	Indeed, by the arguments in \cite[Theorem 3.2]{David-Mattila}, for each $x\in I^k$ one constructs 
	a set $Q^k(x)\subset E$
	such that, denoting
	$\DD_{\mu,k} = \{Q^k(x)\}_{x\in I^k},$
	the properties stated in Lemma \ref{lemcubs} hold, with $r(Q^k(x))=r^k(x)$ and $B(Q^k(x))=B(x,r^k(x))$.
	In particular, 
	$$B(x,r^k(x))\cap E\subset Q^k(x) \subset B(x,28r^k(x)).$$
	Also, as shown in \cite[(3.61)]{David-Mattila}, it holds 
	\begin{equation}\label{eqdisb4}
		\dist(y,\partial B_4^k(x)) \leq 51C_0A_0^{-k-1}\quad \mbox{ for all $y\in N_1(Q(x))$.}
	\end{equation}
	
	For a cube $Q=Q^k(x)\in\DD_{\mu,k}$, we write $r(Q)=r^k(x)$, $B(Q)=B(x,r^k(x))$ and
	$B_i(Q) = B_i^k(x)$ for $i=1,\ldots,4$. 
	By an argument quite similar to the one used in \cite[Theorem 3.2]{David-Mattila} to prove \rf{eqdisb4}, we will show now that
	\begin{equation}\label{eqdisb4'}
		2S\subset \UU_{5A_0^{-1}\ell(Q)}(\partial B_4(Q))\quad \mbox{ for any $S\in\wt \DD_{\mu,1}(Q)$,}
	\end{equation}
	where $\UU_\ell(A)$ stands for the $\ell$-neighborhood of $A$.
	This will be needed below to prove 
	\rf{eqfir5}. The condition $S\in\wt \DD_{\mu,1}(Q)$ tells us that either $S\subset Q$ and 
	$2B_S\cap (E\setminus Q)\neq \varnothing$, or $S\subset E\setminus Q$ and 
	$2B_S\cap  Q\neq \varnothing$. Assume the first option (the arguments for the second one are analogous). So there exists some point $z\in E\setminus Q$ such that $|x_S-z|\leq 2r(B_S)=56r(S)$.
	Let $x\in I^k$ be such that $z\in Q^k(x)$. Then we have $\dist(z,B_4^k(x))\leq 50C_0A_0^{-k-1}$
	and also $\dist(x_S,B_4^k(x_Q))\leq 50C_0A_0^{-k-1}$, by 
	\cite[(3.50)]{David-Mattila} (see also the first paragraph after \cite[(3.61)]{David-Mattila}). 
	Since the sets $B_4^k(x_Q)$, $B_4^k(x)$ are disjoint, we deduce that
	$$\dist(x_S,\partial B_4^k(x))\leq 50C_0A_0^{-k-1} + 56\,r(S) \leq 106C_0A_0^{-k-1}< 2A_0^{-1}\ell(Q).
	$$
	Together with the fact that $2S\subset B\big(x_S,\tfrac52\ell(S)\big)$, this gives \rf{eqdisb4'}.
	
	Notice that, for each $j\geq0$,
	$$
	\sum_{P\in\DD_{\mu,j}(2S)} \left(\frac{\ell(P)}{\ell(Q)}\right)^\gamma\theta_\mu(2B_P)^2\,\mu(P)
	\lesssim C_0^{2n}A_0^{-\gamma(j+1)} \int_{2S} \theta_\mu(x,2A_0^{-j}\ell(S))^2\,d\mu(x).$$
	Then we obtain
	\begin{align}\label{eqhfk2}
		\sum_{S\in\wt \DD_{\mu,1}(Q)} \sum_{P\in\DD_\mu(2S)}& \left(\frac{\ell(P)}{\ell(Q)}\right)^\gamma\theta_\mu(2B_P)^2\,\mu(P) \\
		&\lesssim C_0^{2n}
		\sum_{S\in\wt \DD_{\mu,1}(Q)} \sum_{j\geq0} A_0^{-\gamma(j+1)} \int_{2S} \theta_\mu(x,2A_0^{-j}\ell(S))^2\,d\mu(x)\nonumber\\
		& \lesssim C_0^{2n+d}
		\sum_{j\geq0} A_0^{-\gamma(j+1)} \int_{\UU_{5A_0^{-1}\ell(Q)}(\partial B_4(Q))}
		\theta_\mu(x,2A_0^{-j-1}\ell(Q))^2\,d\mu(x).\nonumber
	\end{align}
	Denote by $\wt J(Q)$ the family of cubes $R\in\DD_{\mu,k}$ such that $B_2(R)\cap B_4(Q)\neq\varnothing$, so that, by the above construction we have
	$$\partial B_4(Q)\subset \bigcup_{R\in \wt J(Q)} (\partial B_1(R) 
	\cup \partial B_2(R)).$$
	Also, notice that $\#\wt J(Q)\leq C\,C_0^d$.
	From \rf{eqal848} we deduce that, for each $R\in\wt J(Q)$ and $i=1,2$,
	\begin{align}\label{eqrig56}
		\sum_{j\geq0} A_0^{-\gamma(j+1)} &\int_{\UU_{5A_0^{-1}\ell(R)}(\partial B_i(R))}
		\theta_\mu(x,2A_0^{-j-1}\ell(R))^2\,d\mu(x)\\
		& \leq C\,C_0A_0^{-1}
		\sum_{j\geq0} A_0^{-\gamma(j+1)} \int_{27B(R)}
		\theta_\mu(x,2A_0^{-j-1}\ell(R))^2\,d\mu(x)\nonumber \\
		& \leq C\,C_0A_0^{-1}
		\sum_{P\in \DD_\mu^{k+1}:P\cap 27B(R)\neq\varnothing}\left(\frac{\ell(P)}{\ell(Q)}\right)^\gamma\, \theta_\mu(x_P,3\ell(P))^2\,\mu(P),\nonumber
	\end{align}
	Notice that for $Q,R\in\DD_{\mu,k}$ as above, the condition $B_2(R)\cap B_4(Q)\neq\varnothing$ implies that $26B(Q) \cap 26B(R)\neq\varnothing$. Then, if
	$P\in \DD_\mu^{k+1}$ is such that $P\cap 27B(R)\neq\varnothing$, we derive
	%$|x_Q-x_R|\leq 26(r(Q) + r(R))$ and so
	$$\dist(x_Q,P)\leq |x_Q-x_R| + 27 r(R)
	\leq 26(r(Q) + r(R)) + 27 r(R) \leq 
	79C_0A_0^{-k} = \frac{79}{56}\,\ell(Q).$$
	Then, since $\ell(P)\leq A_0^{-1}\ell(Q)$, we infer that
	\begin{equation}\label{eqinc732}
		B(x_P,3\ell(P))\cap\supp\mu\subset 2Q.
	\end{equation}
	Also, we can write
	\begin{align*}
		\theta_\mu(x_P,3\ell(P))^2\,\mu(P) &\lesssim \frac{\mu(B(x_P,3\ell(P)))^3}{\ell(P)^{2n}}
		\lesssim \frac1{\ell(P)^{2n}}
		\bigg(\sum_{\substack{P'\in\DD_\mu:\ell(P')=\ell(P),\\
				P'\cap  B(x_P,3\ell(P))\neq\varnothing}}
		\mu(P')\bigg)^3\\
		& \lesssim C_0^{2d} \!\!\!\sum_{\substack{P'\in\DD_\mu:\ell(P')=\ell(P),\\
				P'\cap  B(x_P,3\ell(P))\neq\varnothing}}\!\frac{\mu(P')^3}{\ell(P')^{2n}}\lesssim
		C_0^{2d} \!\!\!\sum_{\substack{P'\in\DD_\mu:\ell(P')=\ell(P),\\
				P'\cap  B(x_P,3\ell(P))\neq\varnothing}}\!\!\theta_\mu(2B_{P'})^2\,\mu(P'),
	\end{align*}
	where we used the fact that the sums above are only over $CC_0^d$ terms at most.
	Together with \rf{eqinc732}, this implies that
	the right hand side of \rf{eqrig56} does not exceed
	\begin{multline*}
		C\,C_0^{2d+1}A_0^{-1} \sum_{P\in \DD_\mu^{k+1}:P\cap 27B(R)\neq\varnothing}\left(\frac{\ell(P)}{\ell(Q)}\right)^\gamma  \!\!\!\!\!\sum_{\substack{P'\in\DD_\mu(2Q):\ell(P')=\ell(P),\\
				P'\cap  B(x_P,3\ell(P))\neq\varnothing}}\!\!\!\theta_\mu(2B_{P'})^2\,\mu(P')
		\\ \leq 
		C\,C_0^{3d+1}A_0^{-1}
		\sum_{P'\in \DD_\mu^{1}(2Q)}\left(\frac{\ell(P')}{\ell(Q)}\right)^\gamma\, \theta_\mu(2B_{P'})^2\,\mu(P').
	\end{multline*}
	By this estimate and \rf{eqhfk2}, summing over all $R\in\wt J(Q)$, we get
	$$\sum_{S\in\wt \DD_{\mu,1}(Q)} \sum_{P\in\DD_\mu(2S)} \left(\frac{\ell(P)}{\ell(Q)}\right)^\gamma\theta_\mu(2B_P)^2\,\mu(P)\lesssim
	C_0^{6d+1}A_0^{-1}\!\!
	\sum_{P\in \DD_\mu^1(2Q)}\left(\frac{\ell(P)}{\ell(Q)}\right)^\gamma\, \theta_\mu(2B_P)^2\,\mu(P),
	$$%\end{align*}
	as wished.
\end{proof}

\vv



By Lemma \ref{lemDMimproved2}, it is clear that to complete the proof of Lemma \ref{lemDMimproved} it suffices to show the following result.

\begin{lemma}\label{lemdmutot}
	Let $\mu$ be a compactly supported Radon measure in $\R^{d}$.
	Assume that $\mu$ has polynomial growth of degree $n$ and let $\gamma\in(0,1)$. The lattice $\DD_\mu$ from Lemma \ref{lemcubs} can be constructed so that the following holds for all
	all $Q\in\DD_{\mu}$:
	\begin{equation}\label{eqfhq29}
		\sum_{P\in\wt \DD_\mu(Q)} \left(\frac{\ell(Q)}{\ell(P)}\right)^{\frac{1-\gamma}2} \theta_\mu(2B_P)^2
		\,\mu(P) \lesssim_{A_0,\gamma}\sum_{P\in\DD_\mu: P\subset 2Q} \left(\frac{\ell(P)}{\ell(Q)}\right)^\gamma\theta_\mu(2B_P)^2\mu(P).
	\end{equation}
\end{lemma}


\begin{proof}
	To prove \rf{eqfhq29} notice that, for each $k\geq1$, by Lemma \ref{lemrecur5} we have
	\begin{align*}
		\sum_{R\in\wt \DD_{\mu,k}(Q)}
		\sum_{P\in \DD_{\mu}^1(2R)} &\left(\frac{\ell(P)}{\ell(R)}\right)^\gamma \theta_\mu(2B_P)^2 \,\mu(P)\\
		&\leq \sum_{R\in\wt \DD_{\mu,k-1}(Q)}
		\sum_{S\in\wt \DD_{\mu,1}(R)} 
		\sum_{P\in \DD_{\mu}(2S)}  \left(\frac{\ell(P)}{\ell(S)}\right)^\gamma\theta_\mu(2B_P)^2 \,\mu(P)\\
		& = A_0^{\gamma} \sum_{R\in\wt \DD_{\mu,k-1}(Q)}
		\sum_{S\in\wt \DD_{\mu,1}(R)} 
		\sum_{P\in \DD_{\mu}(2S)}  \left(\frac{\ell(P)}{\ell(R)}\right)^\gamma\theta_\mu(2B_P)^2 \,\mu(P)\\
		& \leq C C_0^{6d+1} A_0^{\gamma-1} \sum_{R\in\wt \DD_{\mu,k-1}(Q)}
		\sum_{P\in \DD_\mu^1(2R)}\left(\frac{\ell(P)}{\ell(R)}\right)^\gamma\, \theta_\mu(2B_P)^2\,\mu(P).
	\end{align*}
	Iterating the preceding estimate, we obtain
	\begin{align*}
		\sum_{P\in\wt \DD_{\mu,k+1}(Q)} 
		\theta_\mu(2B_P)^2 \,\mu(P) & \leq 
		\sum_{R\in\wt \DD_{\mu,k}(Q)}
		\sum_{P\in \DD_{\mu,1}(2R)} \theta_\mu(2B_P)^2 \,\mu(P)\\
		& \leq A_0^\gamma\sum_{R\in\wt \DD_{\mu,k}(Q)}
		\sum_{P\in \DD_{\mu}^1(2R)} \left(\frac{\ell(P)}{\ell(R)}\right)^\gamma \theta_\mu(2B_P)^2 \,\mu(P)\\
		& \leq A_0^\gamma\,(C C_0^{6d+1} A_0^{\gamma-1})^k \sum_{P\in \DD_\mu^1(2Q)}\left(\frac{\ell(P)}{\ell(Q)}\right)^\gamma\, \theta_\mu(2B_P)^2\,\mu(P).
	\end{align*}
	Therefore,
	\begin{align*}
		&\sum_{P\in\wt \DD_\mu(Q)} \left(\frac{\ell(Q)}{\ell(P)}\right)^{(1-\gamma)/2}\theta_\mu(2B_P)^2
		\,\mu(P)\\
		& \quad\leq \theta_\mu(2B_Q)^2\,\mu(Q) + \sum_{k\geq1} A_0^{(1-\gamma) k/2}\sum_{P\in\wt \DD_{\mu,k}(Q)} \theta_\mu(2B_P)^2
		\,\mu(P)\\
		& \quad\leq \theta_\mu(2B_Q)^2\,\mu(Q) + A_0^\gamma\sum_{k\geq1} A_0^{(1-\gamma) k/2}(C C_0^{6d+1} A_0^{\gamma-1})^{k-1}\!\!\!\!
		\sum_{P\in \DD_\mu^1(2Q)}\!\left(\frac{\ell(P)}{\ell(Q)}\right)^\gamma\, \theta_\mu(2B_P)^2\,\mu(P)\\
		& \quad\lesssim_{A_0,\gamma} \sum_{P\in \DD_\mu(2Q)}\left(\frac{\ell(P)}{\ell(Q)}\right)^\gamma\, \theta_\mu(2B_P)^2\,\mu(P),
	\end{align*}
	taking $A_0$ big enough for the last estimate. This yields \rf{eqfhq29}.
\end{proof}




\vv
