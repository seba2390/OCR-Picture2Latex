% Workflow Analysis
\newpage
\section{Workflow Analysis} \label{sec:workflow}

The runtime of quantum programs is mainly determined by queuing delays, execution modes through cloud services (which vary across providers), and embedding time, rather than the execution time on quantum hardware (microseconds to milliseconds). 
% These factors are critical in determining the overall duration of a quantum program.
To offer a holistic examination of the runtime between the proposed techniques and the baseline, we employ the following analytical model:
\begin{equation}
	T = T_{\text{emb}} + N_{\text{QMI}} \left({ T_{\text{queue}} + T_{\text{QMI}} + T_{\text{net}} }\right) + T_{\text{classical}},
	\label{eq:overall_runtime}
\end{equation}
where $T_{\text{emb}}$ is the embedding time, $N_{\text{QMI}}$ is the number of quantum executables, $T_{\text{queue}}$ is the job wait time, $T_{\text{QMI}}$ is the QMI execution time, $T_{\text{net}}$ is the network delay, and $T_{\text{classical}}$ is the classical pre/post-processing time.

For $r$ trials, $T_{QMI} = t_p + \Delta + r \times t_s$, where $t_p$ is the raw signal preparation time, $\Delta$ is the 10ms QA initialization time, and $t_s$ is the single annealing/readout time. 
Given that D-Wave limits $T_{QMI}$ to two seconds, we assume $T_{QMI} = 2$ in all cases. 
We assume one second for $T_{\text{net}}$ for each job.


We assume a baseline embedding time of 30 minutes, decreasing proportionally with skipped chains (as discussed in section~\ref{subsec:m1_reliability}). 
For example, pruning ten chains reduces the embedding time to three minutes. 
All embeddings can be computed in parallel, making $T_{\text{emb}}$ in Skipper-G the maximum time for individual embeddings. 
Additionally, we allocate one second each for pre- and post-processing.

We examine two access scenarios: \emph{shared} and \emph{dedicated}, with one and zero-second queuing times, respectively. 
Figure~\ref{fig:latency} compares the end-to-end runtime of the baseline and our proposed techniques with $c=11$, resulting in up to 1024 QA runs. 
Skipper shows significantly greater advantages over others in the dedicated access mode.

\begin{figure}[h]    
    \centering    
	\includegraphics[width=\columnwidth]{./figures/workflow.pdf}
    \caption{  
Overall Runtime comparison. 
    }
    \label{fig:latency} 
\end{figure}   




\ignore{
Table~\ref{tbl:methods} offers a comprehensive overview of the proposed techniques. 

\begin{table}[htbp]
    \centering
    \caption{
        Skipper family characteristics relative to the baseline. 
        }
    \centering
    \label{tbl:methods} 
    \begin{tabularx}{\columnwidth}{|l|l|l|}
      \cline{2-3}
      \multicolumn{1}{c|}{} & \textbf{Skipper} & \textbf{Skipper-G} \\
      \cline{2-3}      
      \cline{1-1}      
      \textbf{Quantum Cost} & Exponential & Linear \\
      \cline{1-3}
      \textbf{Embedding Cost} & No Cost& Linear \\
      \cline{1-3}
      \textbf{Avg Fidelity Improvement} & 33.08\% & 29.19\% \\
      \cline{1-3}
      \textbf{Max Fidelity Improvement} & 44.4\% & 40.75\% \\
      \cline{1-3}
      \textbf{Avg Capacity Improvement} & 28.26\% & NA \\
      \cline{1-3}
      \textbf{Max Capacity Improvement} & 59.61\% & NA \\
      \cline{1-3}
   \end{tabularx}
  \end{table}


}