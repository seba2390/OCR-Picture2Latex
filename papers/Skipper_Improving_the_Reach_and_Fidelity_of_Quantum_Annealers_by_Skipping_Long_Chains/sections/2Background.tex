% background 
\clearpage

\section{Background and Motivation}


%++++++++++++++++++++++++++++++++++++++++++++++++++
\subsection{Quantum Computers: Digital vs. Analog}

% Quantum computers (QCs) utilize qubits as the fundamental units of computation, 
% where a qubit's state $|\psi\rangle$ can be expressed as a linear combination or superposition of its basis states $|0\rangle$ and $|1\rangle$, represented by $\alpha |0\rangle + \beta|1\rangle$ with complex probability amplitudes $\alpha$ and $\beta$. 
% As the number of qubits increases, the state space expands exponentially, providing the potential for quantum advantage~\cite{nielsen2010quantum}.

QCs fall into two categories: digital and analog. 
Digital QCs, like those from IBM and Google, apply precise quantum operations---defined by the quantum algorithm---to qubits in order to directly manipulate their state~\cite{nielsen2010quantum}. 
Conversely, analog QCs, like those from D-Wave and QuEra, do not directly manipulate the state of qubits. 
Instead, they apply precise changes---defined by the quantum program---to the environment in which the qubits reside, allowing the qubits to evolve and change their states naturally~\cite{albash2018adiabatic,ayanzadeh2022equal}.
% , thereby guiding the qubits along predetermined paths~\cite{albash2018adiabatic,ayanzadeh2022equal}.
 


%++++++++++++++++++++++++++++++++++++++++++++++++++
\subsection{Quantum Annealers: Analog Quantum Accelerators}

Quantum annealing is a meta-heuristic for tackling optimization problems that runs on classical computers. 
\emph{Quantum Annealers} (\emph{QAs}) are a form of analog QCs that can sample from the ground state (the configuration with the lowest energy value) of a physical system, called Hamiltonian~\cite{albash2018adiabatic,ayanzadeh2021multi,ayanzadeh2022equal}. 
QAs by D-Wave are single-instruction optimization accelerators that can only sample from the ground state of the following problem Hamiltonian (or Ising model): 
\begin{equation}
    \mathcal{H}_p := \sum_{i}{\mathbf{h}_i \mathbf{z}_i} + \sum_{I \neq j}{J_{ij}\mathbf{z}_i \mathbf{z}_j}
    \label{eq:QA_H_p}
\end{equation}
acting on spin variables $\mathbf{z}_i \in {-1, +1}$, where $\mathbf{h}_i \in \mathbb{R}$  and $J_{ij} \in \mathbb{R}$ 
are linear and quadratic coefficients, respectively~\cite{ayanzadeh2022equal}. 


%++++++++++++++++++++++++++++++++++++++++++++++++++
\subsection{Operation Model of Single-Instruction QAs}

QAs operate as single-instruction computers, and during each execution trial, they only draw a single sample to approximate the global minimum of~\eqref{eq:QA_H_p}.
Therefore, we \emph{cast} real-world problems into Hamiltonians, where $\mathbf{h}$ and ${J}$ are defined in such a way that its global minimum represents the optimal solution to the problem at hand~\cite{albash2018adiabatic,ayanzadeh2022equal}.
The abstract problem Hamiltonian is then \emph{embedded} into the connectivity map of the QA hardware to generate an executable Quantum Machine Instruction (QMI)~\cite{minorminerGithub,cai2014practical}. 
Casting and embedding in QAs are akin to designing and compiling quantum circuits in digital QCs, respectively (Fig.~\ref{fig:QC_operation_models}). 
The QMI is executed for several trials, and the outcome with the lowest objective value is deemed as the ultimate result~\cite{ayanzadeh2022equal}. 



%++++++++++++++++++++++++++++++++++++++++++++++++++
\subsection{Anneal Time: Current Technological Barriers}

As the energy gap between the global minimum and the adjacent higher state diminishes linearly, the required annealing time for successful adiabaticity grows exponentially~\cite{albash2018adiabatic,das2008colloquium}, surpassing the limits of contemporary QAs~\cite{yan2022analytical}. 
Nonetheless, QAs, akin to other QCs, are advancing; subsequent generations are expected to bypass present technological constraints. 
Specifically, incorporating $XX$ terms into the time-dependent Hamiltonian can ebb the annealing time scaling from exponential to linear~\cite{nishimori2017exponential}.



\begin{figure}[t]
    \centering
    \includegraphics[width=1\columnwidth]{./figures/background_embedding_v2_cropped.pdf}
    \caption{
        Embedding example.
}       
    \label{fig:background_embedding}
\end{figure}  


%++++++++++++++++++++++++++++++++++++++++++++++++++
\subsection{Embedding for QAs} \label{sec:embedding}    

The connectivity of QA qubits is sparse, thereby limiting users to only specify $J_{ij}$ for those qubits that are physically connected. 
Thus, the abstract problem Hamiltonian is \emph{embedded} into QA hardware where a program qubit ($Q_i$) with higher connectivity is represented by multiple physical qubits ($q_i$) called \emph{chain} (Fig.~\ref{fig:background_embedding}). 
Satisfying the following conditions is sufficient to guarantee that both the abstract Hamiltonian and the embedded Hamiltonian executed on the QA hardware have identical ground states:

\begin{enumerate}[ leftmargin=0.5cm,itemindent=0.cm,labelwidth=.5cm,labelsep=0cm,align=left, itemsep=0.2 cm, listparindent=0.5cm]

    \item 
All chains representing program qubits must be a connected component graph---i.e., there must be a path between any two qubits within a chain.

\item
There must be at least one connection between chains whose corresponding program qubits are connected.

\item
The quadratic coefficient $J_{ij}$ is distributed equally among the couplers connecting $Q_i$ and $Q_j$.

\item
The linear coefficient $\mathbf{h}_i$ is distributed equally among all physical qubits of the corresponding chain.

\item
Inter-chain quadratic coefficients must be large enough to guarantee that all qubits within a chain take an identical value---i.e., a very high penalty for broken chains.

\end{enumerate}



%++++++++++++++++++++++++++++++++++++++++++++++++++
\subsection{Prior Work Limitations}

% ----------------------------------------
\subsubsection{Circuit Cutting in Digital QCs}

Circuit cutting techniques, namely CutQC~\cite{tang2021cutqc}, partition quantum circuits into smaller sub-circuits, enabling larger quantum circuits to be run on smaller QCs. 
However, a similar approach is infeasible in the analog quantum realm because:
(a) analog QAs do not incorporate quantum circuits to cut its wires; 
and (b) partitioning graphs by edge/node removal is nontrivial (e.g., highly dense graphs are non-partitionable).


% ----------------------------------------
\subsubsection{Solving Larger Problems on Smalle QAs}

Previous methods for solving larger problems on smaller QAs ~\cite{pelofske2022solving, okada2019improving} employ iterative or alternating approaches involving approximations, leading to reduced reliability as problem size increases. 
Additionally, convergence---even to a local optimum---is not guaranteed with these techniques.
Conversely, Skipper explores the entire search space comprehensively without resorting to approximations, and since it is not iterative, it does not face convergence issues.


\newpage
% ----------------------------------------
\subsubsection{Application-Specific Policies}

Recent studies have proposed methods for tackling larger instances in various domains, such as Boolean Satisfiability (SAT)~\cite{tan2023hyqsat}, Max-Clique~\cite{pelofske2023solving, pelofske2019solving, pelofske2022parallel,pelofske2021decomposition}, and compressive sensing with matrix uncertainty~\cite{ayanzadeh2019quantum,mousavi2019survey}. 
However, these techniques are tailored to their specific applications and cannot be easily adapted to other domains. 
In contrast, Skipper is versatile and can  be applied to any problem Hamiltonian. 
Moreover, reduction to SAT and Max-Clique often leads to a polynomial increase in program qubits, expanding the problem size.
% Furthermore, both Skipper and other decomposition methods incur exponential quantum overhead. 
% However, unlike these decomposition techniques, the cut-size in Skipper is user-defined and does not scale with problem size . 

% ----------------------------------------
\subsubsection{FrozenQubits}

Skipper is inspired by FrozenQubits~\cite{ayanzadeh2023frozenqubits}, with both methods aiming to eliminate high-degree program qubits. 
While the impact of FrozenQubits on addressing larger problems in digital QCs is minimal due to the one-to-one correspondence between program and physical qubits, 
Skipper, on the other hand, is capable of solving larger problems on QAs and enhancing QA fidelity. 
Moreover, unlike FrozenQubits, whose performance declines with increasing graph density, Skipper maintains effectiveness across a spectrum of graph densities, from sparse to dense structures.



%++++++++++++++++++++++++++++++++++++++++++++++++++
\subsection{Goal of This Paper}

Figure~\ref{fig:observation}(a) shows the maximum and average chain lengths for different graph topologies embedded on a 5761-qubit QA. 
A few dominant chains contain over 7.9x as many qubits as the average chain lengths.
Furthermore, Fig.~\ref{fig:observation}(b) displays the number of unused qubits when embedding the largest possible graphs on a 5761-qubit QA for different graph topologies, 
indicating that more than 25\% of physical qubits remain unutilized, primarily due to dominant chains.

The severe underutilization of QA qubits, along with utilizing several physical qubits to represent a single program qubit, severely diminishes the capacity of QAs by up to 33x.
For instance, while current D-Wave QAs boast over 5,700 qubits, they can accommodate at most 177 program qubits with full connectivity.
The aim of this paper is to enable QAs to tackle larger problems by pruning dominant chains, while also enhancing the fidelity of the QAs.


\begin{figure}[h]
    \captionsetup[subfigure]{position=top} %, singlelinecheck=off,justification=raggedright	
    \centering
    % (a) 
    \subfloat[]{
        \includegraphics[width=0.23\textwidth]{./figures/current_chain_length_advantage.pdf}
	}%\hspace*{-0.8em}        
    % (b)     
    \subfloat[]{	
        \includegraphics[width=0.24\textwidth]{./figures/current_qubit_utilization_advantage.pdf}
	}%\hspace*{-0.8em}    
    \caption{
Maximum embeddable BA graphs on 5761-qubit QA: (a) Avg and Max chain lengths, and (b) Number of unutilized qubits. }    
    \label{fig:observation}    
\end{figure}  
