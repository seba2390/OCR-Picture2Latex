% Conclusion
% \newpage
\section{Conclusion}

We propose \emph{Skipper}, a software scheme designed to enhance the capacity and fidelity of QAs. 
Observing that chain lengths in QAs follow a ``Power-Law'' distribution, with a few \emph{dominant chains} containing significantly more qubits than others, Skipper prunes these chains. 
This approach replaces their corresponding program qubits with two possible measurement outcomes, freeing all qubits in the dominant chains and an additional 25\% of isolated qubits previously entrapped in chains. 
Our experiments on a 5761-qubit QA by D-Wave show that Skipper allows QAs to solve problems up to 59\% larger (Avg. 28.3\%) when up to eleven chains are skipped. 
Additionally, by removing five chains, Skipper substantially improves QA fidelity by up to 44.4\% (Avg. 33.1\%). 

The number of chain cuts in Skipper is user-defined; users can trim up to eleven chains, which necessitates running an average of 1024 (and up to 2048) distinct quantum executables. However, 
this may lead to affordability concerns for some users. 
To mitigate this, we introduce \emph{Skipper-G}, a greedy scheme that prioritizes examining sub-spaces more likely to contain the global optimum. 
When up to eleven chains are pruned, Skipper-G runs a maximum of 23 quantum executables. 
Our experiments show that Skipper-G enhances QA fidelity by up to 40.8\% (Avg. 29.2\%), requiring only 11 quantum executable runs for up to five chain cuts, compared to Skipper's 32 runs.

% Our experiments indicate that Skipper-G boosts QA fidelity by up to 40.8\% (Avg. 29.2\%), 
% with cutting up to five chains and running 11 quantum executables, compared to 32 runs in Skipper.

