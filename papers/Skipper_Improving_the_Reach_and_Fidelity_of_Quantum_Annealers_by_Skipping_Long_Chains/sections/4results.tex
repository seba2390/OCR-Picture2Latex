%  Skipper results 
\newpage
\section{Skipper Evaluation Results}
We evaluate Skipper using Barabasi--Albert (BA) graphs~\cite{barabasi1999emergence} 
with different preferential attachment factor values: $m=1$ (BA-1) to $m=6$ (BA-6). 
% These graphs represent most real-world applications~\cite{clauset2016colorado}, spanning from sparse (BA-1) to highly connected (BA-6) topologies.


% ++++++++++++++++++++++++++++++++++++++++++++++++++
\subsection{Solving Larger Problems}
% We will now discuss the effectiveness of Skipper in enhancing the capacity of QAs to solve larger problems.

% --------------------------------------------------
\subsubsection{Impact on Chain Length}

Figure~\ref{fig:future_relative_avg_chaining_cost}(a) illustrates that increasing the number of chain cuts ($c$) in Skipper leads to a reduction in the average chain length of the embeddings.
Figure~\ref{fig:future_relative_avg_chaining_cost}(b) demonstrates that Skipper decreases the mean chain length by up to 1.32x (with an average of 1.22x) when cutting up to eleven chains.

Figure~\ref{fig:future_relative_max_chaining_cost}(a) shows that the maximum chain length of the embeddings decreases as $c$ in Skipper increases.
Figure~\ref{fig:future_relative_max_chaining_cost}(b) shows that cutting up to $c=11$ chains in Skipper reduces the maximum chain length by up to 9.14x (average 1.86x).
Our observations indicate that long chains are the primary contributing factor to the underutilization of physical qubits.
% These results highlight the effectiveness of Skipper in optimizing qubit utilization and lowering program qubit costs of QA embeddings.

\begin{figure}[h]
    \captionsetup[subfigure]{position=top} %, singlelinecheck=off,justification=raggedright	
    \centering
    % (a)
    \subfloat[]{
		\includegraphics[width=0.49\columnwidth]{./figures/future_relative_avg_chain_lengthh_k136_advantage.pdf}
	}%\hspace*{-0.8em}        
    % (b)
    \subfloat[]{
		\includegraphics[width=0.48\columnwidth]{./figures/future_relative_avg_chain_length_advantage.pdf}
	}%\hspace*{-0.8em}        	    
    \caption{        
        Relative Avg. chain length in Skipper compared to baseline (lower is better). 
        (a) Relative Avg. chain length for different graphs as cut size ($c$) increase. 
        (b) Overall relative mean chain lengths for up to 11 chain cuts. 
}    
    \label{fig:future_relative_avg_chaining_cost} 
\end{figure}  


\begin{figure}[t]
    \captionsetup[subfigure]{position=top} %, singlelinecheck=off,justification=raggedright	
    \centering
    % (a)
    \subfloat[]{
		\includegraphics[width=0.49\columnwidth]{./figures/future_relative_max_chain_lengthh_k136_advantage.pdf}
		%\label{subfig:future_qubit_utilization}
	}%\hspace*{-0.8em}        
    % (b)
    \subfloat[]{
		\includegraphics[width=0.49\columnwidth]{./figures/future_relative_max_chain_length_advantage.pdf}
		%\label{subfig:future_relative_unused_qubits}
	}%\hspace*{-0.8em}        	    
    \caption{        
        Relative Max chain length in Skipper compared to the baseline (lower is better). 
        (a) Relative Max chain length for different graphs as $c$ increases.
        (b) Overall relative max chain lengths for up to 11 chain cuts.
    }    
    \label{fig:future_relative_max_chaining_cost}
\end{figure}  


% --------------------------------------------------
\subsubsection{Impact on Qubit Utilization}

Figure~\ref{fig:future_qubit_utilization}(a) displays the average and maximum number of physical qubits when up to eleven chains are pruned. 
In Fig.~\ref{fig:future_qubit_utilization}(b), Skipper reduces underutilization of QA qubits by up to 57\% (average 22.14\%) with up to eleven trimmed chains. 

\begin{figure}[h]
    \captionsetup[subfigure]{position=top} %, singlelinecheck=off,justification=raggedright	
    \centering
    % (a)
    \subfloat[]{
		\includegraphics[width=0.50\columnwidth]{./figures/future_qubit_utilization_advantage.pdf}
		%\label{subfig:future_qubit_utilization}
	}%\hspace*{-0.8em}        
    % (b)
    \subfloat[]{
		\includegraphics[width=0.48\columnwidth]{./figures/future_relative_unused_qubits_advantage.pdf}
		\label{subfig:future_relative_unused_qubits}
	}%\hspace*{-0.8em}        	    
    \caption{         
        (a) Utilization of Physical Qubits in Skipper across Different Graph Types.
        (b) Relative Number of Unused Physical Qubits in Skipper for up to 11 Chain Cuts, Compared to the Baseline.
Lower is better. 
    }    
    \label{fig:future_qubit_utilization} 
\end{figure}  



% --------------------------------------------------
\subsubsection{Impact on Capacity of QAs}

The QA capacity to accommodate specific graph types, from BA-1 to BA-6, is determined by the largest number of program qubits of each type that can be embedded on the QA.
Figure~\ref{fig:future_relative_capacity}(a) shows that QA capacity in Skipper improves with increasing $c$ across various graph topologies.


Figure~\ref{fig:future_relative_capacity}(b) demonstrates that Skipper enables the embedding of larger problems onto current QAs, with an increase of up to 59.61\% (average 28.26\%). 
It is important to note that this growth in the number of program qubits necessitates a substantial increase in the number of physical qubits, as one program qubit is represented by multiple physical qubits. 
% It is important to emphasize that Skipper's performance remains consistent regardless of the increasing density of problem graphs (from BA2 to BA6). 
% , in contrast to FrozenQubits~\cite{ayanzadeh2023frozenqubits}, whose performance diminishes as graph density increases.


\begin{tcolorbox}[colback=blue!12]
	Skipper's performance remains consistent regardless of the increasing density of problem graphs (from BA2 to BA6). 
\end{tcolorbox}

%This observation suggests that Skipper can be leveraged to emulate the forthcoming generation of QAs using presently available QA devices.


\begin{figure}[]
    \captionsetup[subfigure]{position=top} %, singlelinecheck=off,justification=raggedright	
    \centering
    % (a)
    \subfloat[]{
		\includegraphics[width=0.48\columnwidth]{./figures/future_relative_capacity_k136_advantage.pdf}
		%\label{subfig:future_relative_capacity_k136}
	}%\hspace*{-0.8em}        
    % (b)
    \subfloat[]{
		\includegraphics[width=0.49\columnwidth]{./figures/future_relative_capacity_advantage.pdf}
		\label{subfig:future_relative_capacity}
	}%\hspace*{-0.8em}        	    
    \caption{        
        Relative QA capacity in Skipper compared to baseline. 
         (a) Relative capacity for different graphs as cuts increase. 
         (b) Overall relative capacity for up to 11 chain cuts.
         Higher is better.
    }    
    \label{fig:future_relative_capacity} 
\end{figure}  



% +++++++++++++++++++++++++++++++++++++++++++
% +++++++++++++++++++++++++++++++++++++++++++
% +++++++++++++++++++++++++++++++++++++++++++
\newpage 
\subsection{Boosting QA Reliability} \label{subsec:m1_reliability}
In addition to enhancing QA capacity, Skipper can be employed to improve the reliability of QAs. 


% --------------------------------------------------
\subsubsection{Impact on Embedding Quality }

QAs do not incorporate circuits, thus precluding the use of the Probability of Successful Trials metric commonly employed to assess compilation quality in digital QCs~\cite{ayanzadeh2023frozenqubits, alam2020circuit, nishio, tannu2022hammer}.
Prior studies suggest that embeddings with similar chain lengths can produce better solutions~\cite{boothby2016fast,venturelli2015quantum,rieffel2015case,choi2008minor}.
% , as uniform chains are more resilient against breaking
Figure~\ref{fig:future_embedding_performance}(a) demonstrates that trimming up to eleven chains in Skipper reduces the average variance in chain lengths by 2.93x (up to 70.19x).


\begin{figure}[ht]
    \captionsetup[subfigure]{position=top} %, singlelinecheck=off,justification=raggedright	
    \centering
    % (a)
    \subfloat[]{
		\includegraphics[width=0.48\columnwidth]{./figures/future_relative_var_chain_length_advantage.pdf}
		%\label{subfig:future_relative_capacity_k136}
	}%\hspace*{-0.8em}        
    % (b)
    \subfloat[]{
		\includegraphics[width=0.48\columnwidth]{./figures/future_relative_time_chain_length_advantage.pdf}
		\label{subfig:future_relative_capacity}
	}%\hspace*{-0.8em}        	    
    \caption{ 
(a) Relative variance of chain lengths and (b) relative embedding time in Skipper compared to the baseline when trimming up to eleven chains. 
Lower is better. 
}
    \label{fig:future_embedding_performance} 
\end{figure}  



% --------------------------------------------------
\subsubsection{Impact on Embedding Time}

Figure~\ref{fig:future_embedding_performance}(b) demonstrates that pruning up to eleven chains in Skipper leads to a significant reduction in embedding time, with a maximum improvement of 17.13x (average improvement of 7.12x).


% --------------------------------------------------
\subsubsection{Impact on Fidelity}

Figure~\ref{fig:current_fidelity}(a) shows that as Skipper skips more chains, the Energy Residual (ER) decreases, indicating a progressive approach towards the global optimum.
Additionally, Fig.~\ref{fig:current_fidelity}(b) demonstrates a significant reduction in ER by up to 44.4\% (average 33.08\%), when up to five chains are cut using Skipper, compared to the baseline.


% Figure~\ref{fig:current_fidelity}(a) shows that increasing the number of skipped chains in Skipper reduces the Energy Residual (ER), indicating a progressive approach towards the global optimum.
% Moreover, Figure~\ref{fig:current_fidelity}(b) shows a remarkable maximum reduction of 44.4\% (average 33.08\%) in the gap between the global optimum and the best solution achieved by QAs using Skipper, when up to five chains are pruned, compared to the baseline.

%QAs are widely employed for optimization tasks, and even relatively small improvements can yield substantial impacts, potentially resulting in significant cost savings, enhanced efficiency, or other valuable outcomes.


\begin{figure}[h]
    \captionsetup[subfigure]{position=top} %, singlelinecheck=off,justification=raggedright	
    \centering
    % (a)
    \subfloat[]{
		\includegraphics[width=0.48\columnwidth]{./figures/current_relative_ER_k136_advantage.pdf}		
	}%\hspace*{-0.8em}        
    % (b)
    \subfloat[]{
		\includegraphics[width=0.48\columnwidth]{./figures/current_relative_ER_advantage.pdf}
	}%\hspace*{-0.8em}        	    
    \caption{  
        Relative Energy Residual (ER) in Skipper compared to baseline (lower is better). 
        (a) Relative ER as $c$ increase. 
        (b) Overall relative ER for up to five chain cuts.
    }
    \label{fig:current_fidelity} 
\end{figure}  



% >>how many qubits for same problem size 
% impact on used couplers
%  reduced leakage  between local field and couplers fewer truncatuion of coeffs so rducing the probability of generating infeasible QMI
% when we have fewer h and J, we have lower probability of having infeasible QMI whos searchspae or landscape is different .
% probability of broken chain 
% approximation ratio or energy residual


%\subsection{Performance--Cost Trade-Off}


