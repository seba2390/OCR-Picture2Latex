\begin{abstract}

% Analog Quantum Computers (QCs), such as D-Wave's Quantum Annealers (QAs) and QuEra's neutral atom platform, rival their digital counterparts in computing power, but their operation model differs significantly.
Quantum Annealers (QAs) operate as single-instruction machines, lacking a SWAP operation to overcome limited qubit connectivity. 
Consequently, multiple physical qubits are \emph{chained} to form a program qubit with higher connectivity, resulting in a drastically diminished effective QA capacity by up to 33x.
We observe that in QAs: (a) chain lengths exhibit a power-law distribution, a few \emph{dominant chains} holding substantially more qubits than others; and (b) about 25\% of physical qubits remain unused, getting isolated between these chains.
We propose \emph{Skipper}, a software technique that enhances the capacity and fidelity of QAs by skipping dominant chains and substituting their program qubit with two readout results.
Using a 5761-qubit QA, we demonstrate that Skipper can tackle up to 59\% (Avg. 28\%) larger problems when eleven chains are skipped. 
Additionally, Skipper can improve QA fidelity by up to 44\% (Avg. 33\%) when cutting five chains (32 runs).
Users can specify up to eleven chain cuts in Skipper, necessitating about 2,000 distinct quantum executable runs. 
To mitigate this, we introduce \emph{Skipper-G}, a greedy scheme that skips sub-problems less likely to hold the global optimum, executing a maximum of 23 quantum executables with eleven chain trims.
Skipper-G can boost QA fidelity by up to 41\% (Avg. 29\%) when cutting five chains (11 runs).
        
\end{abstract}
  