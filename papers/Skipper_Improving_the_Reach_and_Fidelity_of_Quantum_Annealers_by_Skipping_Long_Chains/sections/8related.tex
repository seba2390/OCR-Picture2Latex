% Related Work 
% \newpage
\section{Related Work} \label{sec:related_work}

Prior studies can be broadly classified into two categories: 
(a) techniques for solving larger problems on QAs, which are relevant primarily to Skipper; 
and (b) approaches for enhancing QA fidelity, which are considered related work to both Skipper and Skipper-G.

%Solving Larger Problems
Prior research on solving larger problems with smaller QAs~\cite{pelofske2022solving, okada2019improving} are iterative schemes, which tend to lose reliability as problem size increases due to reliance on approximations.
Conversely, Skipper explores the entire search space without resorting to approximations.
Recent studies have introduced schemes for addressing larger instances of Boolean Satisfiability (SAT)~\cite{tan2023hyqsat}, Max-Clique~\cite{pelofske2023solving, pelofske2019solving, pelofske2022parallel, pelofske2021decomposition}, and compressive sensing with matrix uncertainty~\cite{ayanzadeh2019quantum} problems. 
However, these methods are specific to their respective applications and are not transferable to other domains, whereas Skipper is versatile and applicable to any application. 


Policies for improving the fidelity of QAs can be classified as: 
(a) Preprocessing~\cite{pelofske2019optimizing,ayanzadeh2022equal,ayanzadeh2020reinforcement}, modifying QMIs before submission;
(b) Postprocessing~\cite{ayanzadeh2021multi}, enhancing outcomes using heuristics;
(c) Hybrid strategies~\cite{ayanzadeh2022quantum,ayanzadeh2020leveraging}, combining heuristics and QAs for reliability; 
(d) Logical analog qubits~\cite{jordan2006error,sarovar2013error,young2013error,pudenz2014error,vinci2015quantum,venturelli2015quantum,vinci2016nested,matsuura2016mean,mishra2016performance,matsuura2017quantum,vinci2018scalable,pearson2019analog,matsuura2019nested,mohseni2021error}, spreading qubit information over multiple physical qubits;
and ensembling policies~\cite{ ayanzadeh2022equal,ayanzadeh2020reinforcement,ayanzadeh2020ensemble}, subjecting the quantum program to different noise profiles to suppress the bias. 
These proposals are orthogonal  to Skipper and Skipper-G and can effectively boost the reliability of our proposed techniques. 


Skipper is inspired by FrozenQubits~\cite{ayanzadeh2023frozenqubits} in digital QCs. 
While FrozenQubits enhances fidelity of optimization applications in digital QCs, Skipper excels in addressing larger problems and enhancing QA fidelity. 
More importantly, while FrozenQubits' performance diminishes with increased problem graph density, Skipper and Skipper-G maintain their performance, demonstrating the effectiveness of our proposal in handling sparse to dense graphs.




\ignore{
\subsection{Improving the Performance of Embedding techniques}
Finding the optimal embedding is an NP-hard problem, and therefore current techniques rely on heuristics to identify an embedding within a reasonable timeframe. 
However, these heuristics often fail when applied on a larger scale, resulting in timeouts. Consequently, previous studies have primarily focused on reducing the embedding time and increasing the likelihood of successfully identifying an embedding, particularly for problems at scale~\cite{boothby2016fast,klymko2014adiabatic,date2019efficiently,serra2022template,bernal2020integer,choi2008minor,zbinden2020embedding,pelofske20234,pelofske2019solving,pelofske2022solving,barbosa2021optimizing}.
These techniques are orthogonal to Skipper, and can be employed to further enhance the performance of our proposed policies.

\subsection{Related Techniques from Digital QCs}


% \newpage
Quantum Approximate Optimization Algorithm (QAOA)~\cite{basso2021quantum,farhi2014quantum} can be viewed as a digital implementation of quantum annealing on digital QCs. 
QAOA is a leading candidate for showcasing quantum advantage using near-term quantum computers, and various techniques have been developed to enhance its performance~\cite{lao20222qan,li2022paulihedral,alam2020circuit,xie2022suppressing,gokhale2019partial}. 
However, unfortunatly, most error mitigation methods for QAOA focus on circuit compilation or pulse-level optimization and cannot be applied to improve the fidelity of single-instruction QAs.

Edge cutting has been proposed for solving larger QAOA problems on smaller Quantum Computers (QCs)~\cite{li2021large}. However, as high-degree nodes appear in the adjacency list of most nodes, employing this technique for real-world problems that typically follow a Power-Law distribution and have some hotspots is nontrivial.

Circuit cutting techniques, such as CutQC~\cite{tang2021cutqc}, enable the management of larger quantum circuits on smaller devices. 
However, adopting these techniques for QAs is nontrivial due to the specific operational model of QAs. 
Similarly, fidelity improvement heuristics~\cite{bravyi2020mitigating,matrixmeasurementmitigation,patel2020veritas,tannu2022hammer} proposed for digital QCs cannot be directly employed for QAs. 


}


