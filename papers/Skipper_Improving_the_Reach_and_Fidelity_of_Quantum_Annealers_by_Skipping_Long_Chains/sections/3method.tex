\newpage
\section{Methodology}
% This section delves into the evaluation infrastructure employed in the paper.


% ++++++++++++++++++++++++++++++++++++++++++++++++++
\subsection{Hardware Platform}

For our evaluations, we utilize the D-Wave Advantage System (version 6.2), which features over 5,760 qubits and more than 40,000 couplers, accessed via the D-Wave Leap cloud service~\cite{D-Wave}.
We employ the default annealing time of 20 microseconds and adhere to the anneal schedule recommended for this device. 
Each problem is run for 4,000 trials to comply with the two seconds maximum job duration limit.


% ++++++++++++++++++++++++++++++++++++++++++++++++++
\subsection{Software Platform}

We utilize the \emph{minorminer} tool~\cite{minorminerGithub,cai2014practical} to find embeddings for arbitrary problem Hamiltonians on QA hardware. 
In our experiments, we set a timeout of 1,000 seconds, a maximum of 20 failed attempts for improvement, and conduct 20 trials. 
To program the D-Wave QAs, we employ the Ocean SDK~\cite{dwave_ocean_github}.


% ++++++++++++++++++++++++++++++++++++++++++++++++++
\subsection{Benchmarking}

Although current QAs feature over 5,700 qubits, their single-instruction operation model limits them to a few hundred program qubits with higher degrees, which is far below the number of variables required for real-world applications. 
Consequently, in this study, we employ synthetic benchmarks instead of real-world problems. 
In many real-world applications, graphs often exhibit a ``Power-Law'' distribution~\cite{agler2016microbial,clauset2016colorado,gamermann2019comprehensive,goh2002classification,house2015testing,mislove2007measurement,pastor2015epidemic}, 
and the \emph{Barabasi--Albert} (BA) algorithm~\cite{albert2005scale,barabasi1999emergence} is considered representative of these real-world graph structures~\cite{ayanzadeh2023frozenqubits,barabasi2000scale,gray2018super,kim2022sparsity,lusseau2003emergent,wang2019complex,zadorozhnyi2012structural,zbinden2020embedding}. 
The BA graphs are generated with a preferential attachment factor $m$, enabling us to vary the density of the graphs by adjusting $m$---with higher values of $m$ yielding denser graphs. 
We generate BA graphs with $m$ values ranging from $m=1$ (BA-1)  to $m=6$ (BA-6) to capture a broad spectrum of topologies, from sparse to nearly fully connected networks, thus effectively representing the dynamics of various real-world systems~\cite{clauset2016colorado}. 
Edge weights are assigned randomly following a standard normal distribution, which is a common approach in QA benchmarking~\cite{das2008colloquium,ayanzadeh2022equal,ayanzadeh2021multi}.



% ++++++++++++++++++++++++++++++++++++++++++++++++++ 
\subsection{Figure of merit}

In our evaluations, we use the \emph{Energy Residual} (\emph{ER}) to assess the fidelity of QA as 
\begin{equation}	
	\downarrow \textrm{Energy Residual (ER)} = \left| E_{min} - E_{global} \right|,
	\label{eq:ER}
\end{equation}	
where $E_{global}$ represents the global minimum of the benchmark problem, and $E_{min}$ corresponds to the best solution obtained by the QA.
% A lower ER is desirable. 
Ideally, an ER value closer to zero is desirable as it indicates a solution that closely aligns with the ground state of the problem Hamiltonian.
We conducted intensive classical computations using state-of-the-art tools~\cite{ayanzadeh_ramin_2021_5142230} to determine the global minima of the benchmarks.
