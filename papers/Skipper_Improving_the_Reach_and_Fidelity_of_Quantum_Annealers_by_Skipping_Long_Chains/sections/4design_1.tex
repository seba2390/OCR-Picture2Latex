%  Skipper design 
\newpage

\section{Skipper: Skipping Dominant Chains}
We introduce \emph{Skipper}, a software technique to enhance the capacity and fidelity of QAs through strategically skipping dominant qubit chains. 


% ++++++++++++++++++++++++++++++++++++++++++++++++++
\subsection{Key Insights} \label{subsec:method_insights}

% --------------------------------------------------
\subsubsection{Not All Program Qubits are Equal}

In digital QCs, the individual fabrication of physical qubits, such as superconducting ones, results in inevitable performance variations~\cite{tannu2019not}. 
Compilers, therefore, aim to prioritizing high-quality ones and limit the reliance on those of lower quality~\cite{tannu2019not,li2018tackling,noiseadaptive}.
However, in analog QCs, our observations reveal a significant variability at the level of program qubits.

Figure~\ref{fig:not_equal_qubits}(a) shows the histogram of chain lengths (in log-scale) for the BA-3 graph type after embedding onto a 5761-qubit QA device, revealing a \emph{Power-Law} distribution with some notably longer \emph{dominant chains} and a majority of considerably shorter chains.
Figure~\ref{fig:not_equal_qubits}(b) presents the maximum and average chain lengths as the number of nodes in BA-3 graphs increases, notably magnifying the variability in chain lengths with the increase in problem size.

These intriguing observations extends beyond the BA-3 graph type, and we observe it in all benchmark graphs, including BA-1 to BA-6, spanning from sparse to nearly fully connected graphs. 
Additionally, we observe the nonuniformity of chain lengths in regular and fully connected graphs. 


% \begin{tcolorbox}[colback=blue!12]
%     % \textbf{Insight:} 
%     Chain lengths in QAs follow a ``Power-Law'' distribution, with a few \emph{dominant chains} being significantly longer. 
% \end{tcolorbox}

\begin{figure}[h]
    % \vspace{--0.1in}
    \captionsetup[subfigure]{position=top} %, singlelinecheck=off,justification=raggedright 
    \centering
    % (a)
    \subfloat[]{
        \includegraphics[width=0.49\columnwidth]{./figures/ba3_chain_length_hist.pdf}
    
    }%\hspace*{-0.8em}        
    % (b)
    \subfloat[]{
        \includegraphics[width=0.45\columnwidth]{./figures/ba3_chain_length_scale.pdf}
    }%\hspace*{-0.8em}              
    \caption{  
(a) Histogram of chain lengths for a 600-node BA-3 graph (log-scale), indicating a ``Power-Law`` distribution of chain lengths.
 (b) Max and Avg chain lengths of BA-3 graphs, embedded on a 5761-qubit QA.
    }
    \label{fig:not_equal_qubits} 
\end{figure}  


% --------------------------------------------------
\subsubsection{QA Qubits are significantly Underutilized}

We observe that, on average, 25\% of physical qubits remain unused as they get trapped by chains.  
Additionally, we observe that the dominant chains significantly contribute to this qubit isolation.
This underutilization of QA qubits, along with utilizing several physical qubits to represent a single program qubit, severely diminishes the capacity of QAs by up to 33x.
For instance, the 2048-qubit and 5760-qubit QAs by D-Wave can accommodate a maximum of 64 and 177 fully connected program qubits, respectively.


% \begin{tcolorbox}[colback=blue!12]
%     \textbf{Insight:}
% More than 25\% of QA qubits remain unused, primarily due to dominant chains trapping them.
% \end{tcolorbox}



% --------------------------------------------------
\subsubsection{Diminishing Returns with Increased QA Trials}

QAs are noisy and prone to errors, leading to a systematic bias during the execution of quantum programs.
This bias causes deviations from the global optimum, reducing the reliability of QAs~\cite{albash2018adiabatic,mcgeoch2020theory,ayanzadeh2022equal,ayanzadeh2021multi}.
The bias arises from repeating the same quantum program across multiple iterations, exposing all trials to a similar noise profile~\cite{ayanzadeh2022equal}. 

Figure~\ref{fig:ER_plateau} shows that when the number of trials in QA is increased, the output distribution reaches saturation. 
This indicates that the gap between the ideal solution and the QA does not reduce despite drawing more samples.
Moreover, due to the operation model of QAs as single-instruction computers, strategies commonly used in gate-based QCs~\cite{tannu2019ensemble,patel2020veritas} to address this bias are not applicable.

% \begin{tcolorbox}[colback=blue!12]
%     \textbf{Insight:}
% Increasing trials do not necessarily enhance QA fidelity due to the law of diminishing returns, whereas Skipper benefits from utilizing more quantum resources.
% \end{tcolorbox}

\begin{figure}[h]    
    % \vspace{-0.05in}
    \centering
    % (a)    
    \includegraphics[width=\columnwidth]{./figures/ensemble_ER_reduction_k136_advantage.pdf}    
    \caption{        
        The Energy Residual (ER) in QAs tends to plateau with an increasing number of trials, and the global minimum often remains unreachable by QAs.
}    
    \label{fig:ER_plateau} 
\end{figure}  


\begin{figure*}[t]    
    \centering    
    \includegraphics[width=1\textwidth]{./figures/m1_overview_v2_cropped.pdf}
    \caption{
    Overview of Skipper. 
    }
    \label{fig:m1_overview} 
\end{figure*}   


% ++++++++++++++++++++++++++++++++++++++++++++++++++
\subsection{Overview of Skipper}

Figure~\ref{fig:m1_overview} shows the overview of Skipper. 
Skipper leverages insights into the distribution of chain lengths and the severe underutilization of qubits in QAs, employing a strategic approach to prune dominant chains and replace the corresponding program qubit with two potential measurement outcomes (+1 and -1). 
This process involves eliminating each chain, which partitions the search space into two sub-spaces. 
Skipper explores all sub-spaces, guaranteeing an exact recovery of the optimum solutions. 

Eliminating a dominant chain accomplishes two significant objectives: firstly, it frees up physical qubits previously used within pruned chains, and secondly, it eliminates the isolation of solitary qubits resulting from dominant chains.
As a result, Skipper enables the handling of larger problems by accommodating a significantly higher number of program qubits. 
Additionally, Skipper significantly enhances QA fidelity by substantially mitigating the impact of dominant chains, a primary factor in compromising QA reliability.
While Skipper utilizes more quantum resources due to the need to execute $2^c$ unique quantum programs for the removal of $c$ chains, it doesn't correspondingly enhance QAs (baseline) performance, as demonstrated in Fig.~\ref{fig:ER_plateau}. 


% \newpage
% ++++++++++++++++++++++++++++++++++++++++++++++++++
\subsection{Chain Skips: How, Where, and When to Skip?}

% Skipping a chain in QAs is akin to freezing a qubit in digital QCs~\cite{ayanzadeh2023frozenqubits}. 
Figure~\ref{fig:chain_cut_binary_tree} illustrates the elimination of two chains from a problem with five variables, creating two and four \emph{independent} sub-problems, respectively. 
To skip a chain, the program qubit is replaced with +1 and -1 (two measurement outcomes), removing the node and its connected edges from the problem graph.
Unlike digital QCs, where removing one program qubit results in reducing the physical qubit utilization by one, in QAs, removing one program qubit liberates all the physical qubits involved in its corresponding chain. 


\begin{figure}[h]
    \centering
    \includegraphics[width=\columnwidth]{./figures/chain_cut_binary_tree_cropped.pdf}
    \caption{        
        By replacing $Q_0$ with +1 and -1 among five spin variables (baseline), two sub-problems each with four spin variables are obtained ($c=1$). 
        Fixing $Q_1$ in these two sub-problems with +1 and -1 results in four sub-problems with three spin variables ($c=2$).
        The same embedding is utilized for all sub-problems at each level of the binary tree. 
}           
    \label{fig:chain_cut_binary_tree} 
\end{figure}  




Identifying dominant chains to trim in Skipper is nontrivial.
In digital QCs, high-degree program qubits necessitate more CNOT gates, enabling direct identification prior to circuit compilation. 
However, in QAs, it is not feasible to directly recognize program qubits linked to longer chains, thus requiring embedding techniques to identify them.
Furthermore, it is not always optimal to prune the dominant chain. 
In Fig.~\ref{fig:where_to_cut}(a), the dominant chain is $Q_0$ and consists of ten physical qubits. 
Pruning $Q_0$ (Fig.~\ref{fig:where_to_cut}(b)), liberates all ten physical qubits, leaving the other chains intact. 
However, as shown in Fig.~\ref{fig:where_to_cut}(c), removing $Q_2$ and re-embedding the problem not only releases the five physical qubits associated with $Q_2$ but also effectively reduces $Q_0$ to a singleton chain, totaling fourteen qubits released.


Skipper adopts a greedy approach to prune $c$ chains by sorting program qubits based on their degree and removing the top $c$ qubits simultaneously. The removal of a single program qubit can have a substantial impact on other chains, as shown in Fig.~\ref{fig:where_to_cut}(c). 
In the context of irregular graphs that often follow the Power-Law distribution in real-world applications, this greedy approach exhibits a desirable, near-optimal behavior for $c \geq 5$ chain cuts.

\begin{figure}[ht]
    \centering
    \includegraphics[width=\columnwidth]{./figures/where_to_cut_cropped.pdf}
    \caption{  
(a) Embedding of five program qubits on a grid. 
(b) Freeing ten qubits by pruning the dominant chain $Q_0$. 
(c) Fourteen qubits freed by pruning $Q_2$. 
    }
    \label{fig:where_to_cut} 
\end{figure}  



% ++++++++++++++++++++++++++++++++++++++++++++++++++
\subsection{Skip Count: A Cost-Performance Tradeoff}

Skipper permits users to trim up to eleven chains. 
Each chain skipped bifurcates the search space; therefore, trimming up to eleven chains can lead to a maximum of 2048 sub-problems. 
Skipper executes all corresponding QMIs to ensure exact solution recovery.
However, the nontrivial embedding process and the need to execute up to 2048 embeddings can create a bottleneck for Skipper. 
Fortunately, the identical structure of all sub-problems at the $c$-th level in the binary tree enables sharing the same embedding across them (Fig.~\ref{fig:chain_cut_binary_tree}).



% ++++++++++++++++++++++++++++++++++++++++++++++++++
\subsection{Unembedding: Remediating Broken Chains}

The sub-problem resulting from cutting chains consists of $n-c$ program qubits.
However, after embedding, the problem executed on the QA hardware encompasses $N$ physical qubits, where $c \ll n \ll N$.
As a result, the QA produces outcomes as $N$-bit strings, with each program qubit collectively represented by multiple bits in a chain.
Therefore, Skipper \emph{unembeds} these outcome samples, converting them back into the space of program qubits.

Ideally, all physical qubits within a chain should have identical values in a given QA sample. 
The value of the associated program qubit is then determined by observing any one of the physical qubits within it (e.g., program qubit $Q_0$ in Fig.~\ref{fig:unembedding_example}).
However, QAs are inherently open systems, as interactions with the environment are unavoidable in QCs, and the annealing process tends to be diabatic since truly adiabatic processes are often unfeasible~\cite{ayanzadeh2021multi}.
As a result, qubits within a chain can take different values, an issue known as \emph{broken chains}~\cite{grant2022benchmarking,pelofske2020advanced,king2014algorithm,barbosa2021optimizing}.
% However, since QAs are noisy and prone to errors, qubits within a chain can take different values, an issue known as \emph{broken chains}~\cite{grant2022benchmarking,pelofske2020advanced,king2014algorithm,barbosa2021optimizing}.

To remediate broken chains, Skipper employ the \emph{majority voting} approach. % (e.g., $Q_1$ in Fig.~\ref{fig:unembedding_example}). 
For instance, in Fig.~\ref{fig:unembedding_example}, although $Q_1$ exhibits a broken chain with varying qubit values, the unembedding process assigns a value of -1, reflecting the majority of -1 values within the chain (4 versus 1).
However, not all chains have an odd length, and forcing the embedding to produce odd chain lengths is nontrivial.
Unembedding even length chains with mostly identical qubit values (e.g., $Q_2$ in Fig.~\ref{fig:unembedding_example}) is not challenging, as majority voting can effectively determine the value of the program qubit.
However, as demonstrated by $Q_3$ in Fig.~\ref{fig:unembedding_example}, a chain of even length can contain an equal number of -1 and +1 values, referred to as \emph{balanced chains}, a condition where majority voting fails.
Skipper manages balanced chains by counting them and implementing distinct strategies based on their quantity. 
For problems with fewer than ten balanced chains, Skipper discards their qubit values and uses a brute-force approach (with up to 1024 possible configurations), selecting the configuration that yields the lowest energy value. 
If the number of balanced chains exceeds ten, Skipper randomly assigns values to the corresponding program qubits. 
When a broken chain occurs, Skipper can optionally apply Single-Qubit Correction (SQC)~\cite{ayanzadeh2021multi,ayanzadeh2022equal} postprocessing to maintain a feasible solution for the original problem.


\begin{figure}[h]
    \centering
    \includegraphics[width=\columnwidth]{./figures/unembedding_cropped.pdf}
    \caption{  
Unembedding examples
    }
    \label{fig:unembedding_example} 
\end{figure}  



% ++++++++++++++++++++++++++++++++++++++++++++++++++
\subsection{Decoding Sub-Problem Results}

After unembedding, each sample will encompass $n-c$ bits, while the original problem includes $n$ variables. 
The decoding process reintroduces the values of the $c$ pruned program qubits,
which were fixed during the sub-problem formulation by assigning fixed values to these variables. 
% These values were determined during the formation of the sub-problems, where these $c$ variables were assigned fixed values. 


% ++++++++++++++++++++++++++++++++++++++++++++++++++
\subsection{Postprocessing}

Theoretically, QAs sample from a Boltzmann distribution, exponentially favoring lower energy values, and thus should locate the global optimum in few attempts. 
However, like other QCs, QAs are vulnerable to noise and various error sources that degrade their fidelity. 
To enhance the reliability of QAs, we can optionally apply postprocessing heuristics to the resulting samples~\cite{ayanzadeh2021multi}.

\subsection{Deriving the Final Output}
In Skipper, all sub-problems are executed independently, each one corresponding to a separate sub-space of the primary problem. 
Consequently, in Skipper, the sample with the lowest energy or objective value is deemed as the ultimate output, with the originating sub-space of this global optimum being of no consequence.



\subsection{Overhead of Skipper}

Let ${c}$ represent the number of skipped chains, $e$ denote the edges in the problem graph, $r$ symbolize the number of trials on the QA, while $n$ and $N$ correspond to the number of program and physical qubits, respectively.

\vspace{0.05in}
\noindent \textbf{Quantum overhead:}
Skipper allows for up to eleven chain cuts, necessitating the execution of at most 2048 distinct quantum executables, each running independently.

\vspace{0.05in}
\noindent \textbf{Classical overhead:}
We separate the embedding overhead of Skipper from all other classical pre/post-processing modules, as the embedding is the primary factor influencing the end-to-end runtime of the proposed techniques in this paper (refer to section~\ref{sec:workflow}). 
Given the fact that $c \ll n \ll N \ll r$, Skipper demonstrates a classical time complexity of $O\left( 2^c \left(rN + c \right) \right)$. 
This is representative of the unembedding and decoding processes for outcome samples of sub-problems.

\vspace{0.05in}
\noindent \textbf{Embedding overhead:}
% Embedding techniques are heuristics, and analyzing their complexity can be a nontrivial task. In this paper, we evaluate the embedding complexity of the proposed techniques by examining the total number of distinct embeddings that each method requires.
In Skipper, all sub-problems of the binary tree at the $c$-th level share a single embedding, leading to $O(1)$ embedding complexity. 
Note that we assume all sub-problems are executed on the same QA hardware or that all devices have the same working graph topology, allowing them to share the embedding.

\vspace{0.05in}
\noindent \textbf{Memory utilization:}
The memory utilization in Skipper scales according to $O(rN2^c)$.



