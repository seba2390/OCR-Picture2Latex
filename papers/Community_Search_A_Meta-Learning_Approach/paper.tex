\documentclass[conference]{IEEEtran}
\IEEEoverridecommandlockouts
% The preceding line is only needed to identify funding in the first footnote. If that is unneeded, please comment it out.
\newcommand{\shadd}[1]{\textcolor{blue}{#1}}
\makeatother
\newtheorem{example}{Example}
\usepackage{cite}
\usepackage{amsmath,amssymb,amsfonts}
\usepackage{algorithmic}
\usepackage{graphicx}
\usepackage{textcomp}
\usepackage{xcolor}
\def\BibTeX{{\rm B\kern-.05em{\sc i\kern-.025em b}\kern-.08em
    T\kern-.1667em\lower.7ex\hbox{E}\kern-.125emX}}
\usepackage{enumitem}
\usepackage{cleveref}
\usepackage{pifont}
\definecolor{LightCyan}{rgb}{0.88,1,1}
\definecolor{LightRed}{rgb}{1,0.88,1}
\definecolor{LightYellow}{rgb}{1,1,0.88}
\definecolor{LightGray}{gray}{0.8}
%\usepackage[font=scriptsize]{caption}
\usepackage{graphicx}
\usepackage{balance}  % for  \balance command ON LAST PAGE  (only there!)
\usepackage[linesnumbered,vlined,ruled]{algorithm2e}
\usepackage{lipsum}
\usepackage{amsmath}
% \usepackage{amssymb}
\usepackage{physics}
\usepackage{subfigure}
\usepackage{tabularx}
\usepackage{color}
\usepackage{colortbl}
%\usepackage{hyperref}
\usepackage{float}
\usepackage{listings}
\usepackage{multirow}
\usepackage[normalem]{ulem}
 \usepackage{longtable}
\usepackage{enumitem}
\usepackage{multirow}
\usepackage{booktabs}
\usepackage{capt-of}
\usepackage{pifont}
\usepackage{ulem}
\usepackage{soul}
\usepackage{cancel}
\usepackage{bm}
\usepackage{url}
\usepackage{caption}
\usepackage{tikz}


%\usepackage[colorlinks=true,allcolors=black]{hyperref}
% \usepackage{amssymb}% http://ctan.org/pkg/amssymb


\def\BibTeX{{\rm B\kern-.05em{\sc i\kern-.025em b}\kern-.08em
    T\kern-.1667em\lower.7ex\hbox{E}\kern-.125emX}}

%speed up compiliation
%\pdfcompresslevel=0
%\pdfobjcompresslevel=0

 \graphicspath{{./Graph/}, {./Fig/}, {./Legend/}}
 
\def\ojoin{\setbox0=\hbox{$\Join$}%
\rule[0.1ex]{.27em}{.4pt}\llap{\rule[1.3ex]{.27em}{.4pt}}}
\def\leftouterjoin{\mathbin{\ojoin\mkern-5.8mu\Join}}
\def\rightouterjoin{\mathbin{\Join\mkern-5.8mu\ojoin}}
\def\fullouterjoin{\mathbin{\ojoin\mkern-5.8mu\Join\mkern-5.8mu\ojoin}}

\long\def\comment#1{}


\newcounter{definition}[section]
\renewcommand{\thedefinition}{\nthesection.\arabic{definition}}
\newenvironment{definition}{
     \refstepcounter{definition}
     {\vspace{1ex} \noindent\bf  Definition  \thedefinition:}}{
     \vspace{1ex}} %\hspace*{\fill}\vspace*{1ex}}

\newcounter{theorem}[section]
\renewcommand{\thetheorem}{\nthesection.\arabic{theorem}}
\newenvironment{theorem}{\begin{em}
        \refstepcounter{theorem}
        {\vspace{1ex} \noindent\bf  Theorem  \thetheorem:}}{
        \end{em}\vspace{1ex}} %\hspace*{\fill}\vspace*{1ex}}

\newcounter{lemma}[section]
\renewcommand{\thelemma}{\nthesection.\arabic{lemma}}
\newenvironment{lemma}{\begin{em}
        \refstepcounter{lemma}
        {\vspace{1ex}\noindent\bf Lemma \thelemma:}}{
        \end{em}\vspace{1ex}} %\hspace*{\fill}\vspace*{1ex}}
        
\newcounter{proposition}[section]
\renewcommand{\theproposition}{\nthesection.\arabic{proposition}}
\newenvironment{proposition}{\begin{em}
        \refstepcounter{proposition}
        {\vspace{1ex}\noindent\bf Proposition \theproposition:}}{
        \end{em}\vspace{1ex}} %\hspace*{\fill}\vspace*{1ex}}

\newcounter{remark}[section]
\renewcommand{\theremark}{\nthesection.\arabic{remark}}
\newenvironment{remark}{\begin{em}
        \refstepcounter{remark}
        {\vspace{1ex}\noindent\bf Remark \theremark:}}{
        \end{em}\vspace{1ex}} %\hspace*{\fill}\vspace*{1ex}}


\newcommand{\proofsketch}{\noindent{\bf Proof Sketch: }}
\newcommand{\myproof}{\noindent{\bf Proof: }}
\newcommand{\nthesection}{\arabic{section}}


\newcommand{\eop}{\hspace*{\fill}\mbox{$\Box$}\vspace*{1ex}}

% Small Title
\newcommand{\stitle}[1]{\vspace{1ex} \noindent{\bf #1}}
\newcommand{\utitle}[1]{\vspace{1ex} \noindent{\underline{#1}}}
\newtheorem{rule1}{Rule}

%Testing
\newcommand{\red}[1]{\textcolor{red}{#1}}
\newcommand{\redD}[1]{\red{\sout{#1}}}
\newcommand{\blue}[1]{\textcolor{blue}{#1}}
\newcommand{\green}[1]{\textcolor{green}{}}

%Release
%\newcommand{\red}[1]{#1}
%\newcommand{\redD}[1]{#1}
%\newcommand{\blue}[1]{#1}
%\newcommand{\green}[1]{#1}


%Macro for custom styles
\newcommand{\kw}[1]{{\ensuremath {\mathsf{#1}}}\xspace}
\newcommand{\kwnospace}[1]{{\ensuremath {\mathsf{#1}}}}
\newcommand{\kwtt}[1]{{\ensuremath {\texttt{#1}}}\xspace}
\newcommand{\kwttnospace}[1]{{\ensuremath {\texttt{#1}}}}
\newcommand{\vgapbefore}{\vspace{0.05cm}}
\newcommand{\vgap}{\vspace{0.1cm}}
\newcommand{\vfiggap}{\vspace{-0.2cm}}
\newcommand{\mc}[1]{{\ensuremath {\mathcal{#1}}}\xspace}
\newcommand{\td}[1]{{\ensuremath {\widetilde{#1}}}\xspace}
\newcommand{\lessgap}{\vspace{-0.2cm}}



%Macro for Dolphin
\def\ojoin{\setbox0=\hbox{$\bowtie$}%
  \rule[-.02ex]{.25em}{.4pt}\llap{\rule[\ht0]{.25em}{.4pt}}}
\def\leftouterjoin{\mathbin{\ojoin\mkern-5.8mu\bowtie}}
\def\rightouterjoin{\mathbin{\bowtie\mkern-5.8mu\ojoin}}
\def\fullouterjoin{\mathbin{\ojoin\mkern-5.8mu\bowtie\mkern-5.8mu\ojoin}}

\newcommand{\dom}[1]{\kw{Dom}(#1)}
\newcommand{\attrs}[1]{\kw{Attrs}(#1)}
\newcommand*{\defeq}{\stackrel{\text{def}}{=}}
\newcommand{\exec}[2]{\kw{exec}_{#1,#2}}
\newcommand{\exchange}[1]{\kw{exchange}_{#1}}
\newcommand{\map}[1]{\kw{map}_{#1}}
\newcommand{\renest}[1]{\kw{renest}_{#1}}
\newcommand{\shuffle}[1]{\kw{shuffle}_{#1}}



\newcommand{\partition}{\Psi}
\newcommand{\pair}{\kw{P}}
\newcommand{\comp}{\kw{C}}
\newcommand{\cloop}{\kw{Loop}}
\newcommand{\assign}{\kw{Assign}}
\newcommand{\update}{\kw{Update}}
\newcommand{\cache}{\kw{Cache}}




%convention name definition

\newcommand{\PJoin}{\overset{P}{\bowtie}}
\newcommand{\FJoin}{\overset{F}{\bowtie}}
\newcommand{\FCJoin}{\overset{FC}{\bowtie}}
\newcommand{\FAJoin}{\overset{FA}{\bowtie}}

\newcommand{\ghd}{\kw{GHD}} %GHD
\newcommand{\hdfs}{\kw{HDFS}} %HDFS
\newcommand{\spark}{\kw{Spark}} %Spark
\newcommand{\lss}{\kw{LSSMatch}} %LSS-Match
\newcommand{\lssest}{\kw{LSS}} %NueralEstimator
\newcommand{\lf}{\kw{LeapFrog}} %LeapFrog


\newcommand{\dolphin}{\kw{Dolphin}} %Our method
\newcommand{\benu}{\kw{BENU}} %BENU method
\newcommand{\graphx}{\kw{GraphX}} %GraphX method
\newcommand{\sparksql}{\kw{SparkSQL}} %SparkSQL method
\newcommand{\dbx}{\kw{SQL-G}} %DBX method
\newcommand{\graphframe}{\kw{GraphFrame}} %GraphFrame method
\newcommand{\rasql}{\kw{RASQL}} %GraphFrame method

\newcommand{\WB}{\kw{WB}}
\newcommand{\AS}{\kw{AS}}
\newcommand{\LJ}{\kw{LJ}}
\newcommand{\OK}{\kw{OK}}
\newcommand{\UK}{\kw{UK}}
\newcommand{\TW}{\kw{TW}}
\newcommand{\IMDB}{\kw{IMDB}}







% Datasets
\newcommand{\aids}{\kw{aids}}
\newcommand{\hprd}{\kw{hprd}}
\newcommand{\yeast}{\kw{yeast}}
\newcommand{\wordnet}{\kw{wordnet}}
\newcommand{\youtube}{\kw{youtube}}
\newcommand{\eu}{\kw{eu2005}}
\newcommand{\yago}{\kw{yago}}













%\newcommand{\disc}{\kw{DISC}} %Our method
%\newcommand{\benu}{\kw{BENU}} %BENU method
%\newcommand{\dist}{\kw{DIST}} %4-DIST method
%\newcommand{\eclog}{\kw{ECLOG}} % 5-EClog method
%\newcommand{\jesse}{\kw{JESSE}} %General node orbit method
%\newcommand{\evoke}{\kw{EVOKE}} %Evoke method
%\newcommand{\pte}{\kw{PTE}} %Evoke method
%\newcommand{\orca}{\kw{ORCA}} %Evoke method
%
%\newcommand{\WB}{\kw{WB}} %Evoke method
%\newcommand{\AS}{\kw{AS}} %Evoke method
%\newcommand{\LJ}{\kw{LJ}} %Evoke method
%\newcommand{\OK}{\kw{OK}} %Evoke method
%\newcommand{\UK}{\kw{UK}} %Evoke method
%
%\newcommand{\AC}{\kw{AC}}
%\newcommand{\TP}{\kw{TP}}
%\newcommand{\RC}{\kw{RC}}
%\newcommand{\FB}{\kw{FB}}
%
\newcommand{\cmark}{\ding{51}}%
\newcommand{\xmark}{\ding{55}}%
\def\halfcheckmark{\tikz\draw[scale=0.4,fill=black](0,.35) -- (.25,0) -- (1,.7) -- (.25,.15) -- cycle (0.75,0.2) -- (0.77,0.2)  -- (0.6,0.7) -- cycle;}
%
%
%
%\newcommand\bigutimes{\mathop{\ooalign{$\bigcup$\cr%
%   \hfil\raise0.36ex\hbox{$\scriptscriptstyle\boldsymbol{\times}$}\hfil\cr}}}
%\newcommand\bigumius{\mathop{\ooalign{$\bigcup$\cr%
%   \hfil\raise0.36ex\hbox{$\scriptscriptstyle\boldsymbol{-}$}\hfil\cr}}}


%
%
%
%%Macro for graph
%\newcommand{\Vset}[1]{V(#1)} % vertex set of the graph
%\newcommand{\Eset}[1]{E(#1)} % edge set of the graph
%\newcommand{\dataGraph}{D} % data graph
%\newcommand{\pattern}{p} % pattern
%\newcommand{\core}{c} % core
%
%
%
%\newcommand{\dset}[1]{\mc{I}_{#1}^{s}} % the instance set of pattern #1
%\newcommand{\iset}[1]{\mc{I}_{#1}^{i}} % the induced instance set of pattern #1
%\newcommand{\pset}[1]{\mc{I}_{#1}} % the sub-instance set of pattern #1
%\newcommand{\ppset}[1]{\mc{I}_{#1}'} % the sub-instance set of pattern without considering automorphism #1
%
%\newcommand{\q}{q} % a query
%\newcommand{\CT}[1]{T^{s}_{#1}} % subgraph count
%\newcommand{\OT}[1]{T_{#1}} % partial subgraph count
%\newcommand{\NT}[1]{T^{i}_{#1}} % induced subgraph count
%
%\newcommand{\PCT}[1]{T^{s'}_{#1}} % subgraph count
%\newcommand{\POT}[1]{T'_{#1}} % partial subgraph count
%\newcommand{\PNT}[1]{T^{i'}_{#1}} % induced subgraph count
%
%\newcommand{\Collap}[1]{\kw{Collap}(#1)} % the collapsed pattern set of #1
%\newcommand{\Ext}[1]{\kw{Ext}(#1)} % the collapsed pattern set of #1
%\newcommand{\bag}[1]{\kw{B}(#1)} % the bag of the hyperedge
%%\newcommand{\attr}[1]{\kw{Attr}(#1)} % attribute set
%\newcommand{\attr}{\kw{Attr}} % attribute set
%% \newcommand{\rel}[1]{\kw{Rel}(#1)} % relation set
%\newcommand{\inputRel}[1]{\kw{In}(#1)} % relation set
%\newcommand{\idx}[1]{\kw{Idx}(#1)} % index of an attribute
%\newcommand{\size}[1]{\kw{Size}(#1)} % size of the relation
%\newcommand{\out}[1]{\kw{Out}(#1)} % output of the query #1
%\newcommand{\wocj}[1]{\kw{Wocj}(#1)} % size of the relation
%\newcommand{\incid}[2]{\kw{Incid}(#1, #2)} % incident instance set of #2 on #1
%\newcommand{\ord}[1]{\kw{Ord}_{#1}} % symmetry breaking orders





%\newcommand{\tset}[1]{\mc{T}_{#1}} % the projection of instance set #1
%\newcommand{\result}{c} % core
%\newcommand{\superP}{\kw{sup}} % core
%\newcommand{\superPset}{\kw{Sup}} % core
%\newcommand{\subP}{\kw{sub}} % core
%\newcommand{\subPset}{\kw{Sub}} % core
%\newcommand{\enumP}{\kw{enum}} % core
%\newcommand{\enumPset}{\kw{Enum}} % core
%
%\newcommand{\coef}{\kw{coef}} % core
%
%\newcommand{\ItoA}{\kw{ItoA}} % core
%\newcommand{\joinPlan}{\kw{joinPlan}} % core
%\newcommand{\joinEval}{\kw{joinEval}} % core
%\newcommand{\joinEvalAll}{\kw{joinEvalAll}} % core
%\newcommand{\countQ}{\kw{count}} % core
%\newcommand{\countQAll}{\kw{countAll}} % core
%
%\newcommand{\fhw}{\kw{fhw}}
%\newcommand{\GHD}{\kw{GHD}} % core
%\newcommand{\Spark}{{\sl Spark}\xspace} % core
%\newcommand{\SparkSQL}{{\sl Spark-SQL}\xspace} % core
%
%
%\newcommand{\hcubej}{\kw{HCubeJoin}} % HyperCube Join
%\newcommand{\hcube}{\kw{HCube}} %HyperCube Shuffle
%\newcommand{\lf}{\kw{Leapfrog}} % LeapFrog Algorithm
%\newcommand{\attrs}{\kw{attrs}}
%\newcommand{\rel}{\kw{rel}}
%\newcommand{\hroot}{\kw{root}}
%
%\newcommand{\Rs}{\mc{R}} % All relations
%% \newcommand{\attrs}[1]{\kw{attrs}(#1)} % attribute set of any objects, i.e., \attrs{R}, the attribute set of a relation
%\newcommand{\R}{R\xspace} % symbol for a relation
%\newcommand{\A}{A\xspace} % symbol for a attributes
%\newcommand{\extension}[2]{\kw{val}(#1 \rightarrow #2)} % The extension list from a tuple \ti{i} to a list of values of \A_{i+1}
%\newcommand{\ti}[1]{t^{#1}} % i-th tuples
%\newcommand{\p}{p\xspace} % share vector
%\newcommand{\pos}[1]{\kw{pos}(#1)} % position of an attributes in attribute order
%\newcommand{\ghd}{\kw{GHD}} %Generalized Hypertree Decomposition
%\newcommand{\ghdj}{\kw{GHDjoin}} %Generalized Hypertree Decomposition
%
%\newcommand{\subg}{\kw{SubG}}
%\newcommand{\insubg}{\kw{InSubG}}
%\newcommand{\iso}{\kw{ISO}}
%\newcommand{\iniso}{\kw{InISO}}
%\newcommand{\noniniso}{\kw{\bcancel{\kw{In}}ISO}}
%\newcommand{\homo}{\kw{HOM}}
%\newcommand{\noninjhomo}{\kw{\bcancel{\kw{Inj}}HOM}}

\def\subfigcapskip{2pt}
\def\subfigtopskip{2pt}
\def\subfigbottomskip{4pt}

% \newcommand{\lscount}{{\sl LS}-counting\xspace}
% \newcommand{\liscount}{{\sl LIS}-counting\xspace}
% \newcommand{\lscount}{local subgraph counting\xspace}
% \newcommand{\liscount}{local induced subgraph counting\xspace}


\pagenumbering{gobble}
\usepackage{geometry}
\geometry{verbose,tmargin=1in,bmargin=1in,lmargin=1in,rmargin=1in}
\usepackage{setspace}
\usepackage{amsmath, amssymb, amsfonts, bm, mathtools}
\usepackage{amsthm}
\usepackage[dvipsnames]{xcolor}
\definecolor{darkblue}{rgb}{0,0,.5}
\usepackage{graphicx}
\usepackage{subfigure}

\usepackage[numbers]{natbib}

\usepackage[colorlinks=true,allcolors=darkblue]{hyperref}       % hyperlinks
\usepackage{url}            % simple URL typesetting
\usepackage{booktabs}       % professional-quality tables
\usepackage{amsfonts}       % blackboard math symbols
\usepackage{nicefrac}       % compact symbols for 1/2, etc.
\usepackage{microtype}      % microtypography

\allowdisplaybreaks

\usepackage{float}
\usepackage{multirow}
\usepackage{footnote}
\usepackage{dsfont}
% \usepackage{mathabx}

\usepackage{algorithm}
\usepackage{algorithmic}
\usepackage{nicefrac}

\usepackage{tikz}
\usepackage{overpic}

\usepackage{dsfont}
\usepackage{hyperref}
\usepackage[capitalize]{cleveref}
\usepackage{crossreftools}

\newcommand{\ind}{\mathds{1}}
\newcommand{\var}{\mathsf{Var}}
\newcommand{\E}{\mathbb{E}}
\newcommand{\mc}[1]{\mathcal{#1}}
\newcommand{\brc}[1]{\left\{{#1}\right\}}
\newcommand{\paren}[1]{\left({#1}\right)} % parentheses
\newcommand{\brk}[1]{\left[{#1}\right]} % bracket
\newcommand{\norm}[1]{\left\|{#1}\right\|} % norm
\newcommand{\abs}[1]{\left|{#1}\right|} % norm
\newcommand{\what}[1]{\widehat{#1}}
\newcommand{\sgn}{\mathsf{sign}}
\newcommand{\normal}{\mathsf{N}}
\newcommand{\bindist}{\mathsf{Binomial}}
\newcommand{\matrixnorm}[1]{\left|\!\left|\!\left|{#1}
	\right|\!\right|\!\right|} % Matrix norm with three bars

\newcommand{\opnorm}[1]{\matrixnorm{#1}_{\rm op}} % Operator norm with three bars
\newcommand{\<}{\langle} % Angle brackets
\renewcommand{\>}{\rangle}

\DeclareMathOperator*{\argmax}{argmax}
\DeclareMathOperator*{\argmin}{argmin}

\newcommand{\simiid}{\stackrel{\textup{iid}}{\sim}}

\newcommand{\cd}{\stackrel{d}{\rightarrow}}
\newcommand{\cp}{\stackrel{p}{\rightarrow}}
\newcommand{\cas}{\stackrel{a.s.}{\rightarrow}}

%%%%% NEW MATH DEFINITIONS %%%%%

\def\ceil#1{\lceil #1 \rceil}
\def\floor#1{\lfloor #1 \rfloor}

% Graph
\def\gA{{\mathcal{A}}}
\def\gB{{\mathcal{B}}}
\def\gC{{\mathcal{C}}}
\def\gD{{\mathcal{D}}}
\def\gE{{\mathcal{E}}}
\def\gF{{\mathcal{F}}}
\def\gG{{\mathcal{G}}}
\def\gH{{\mathcal{H}}}
\def\gI{{\mathcal{I}}}
\def\gJ{{\mathcal{J}}}
\def\gK{{\mathcal{K}}}
\def\gL{{\mathcal{L}}}
\def\gM{{\mathcal{M}}}
\def\gN{{\mathcal{N}}}
\def\gO{{\mathcal{O}}}
\def\gP{{\mathcal{P}}}
\def\gQ{{\mathcal{Q}}}
\def\gR{{\mathcal{R}}}
\def\gS{{\mathcal{S}}}
\def\gT{{\mathcal{T}}}
\def\gU{{\mathcal{U}}}
\def\gV{{\mathcal{V}}}
\def\gW{{\mathcal{W}}}
\def\gX{{\mathcal{X}}}
\def\gY{{\mathcal{Y}}}
\def\gZ{{\mathcal{Z}}}

\def\CB{{\mathbb C}}
\def\RB{{\mathbb R}}
\def\EB{{\mathbb E}}
\def\ZB{{\mathbb Z}}
\def\PB{{\mathbb P}}
\def\OB{{\mathbb O}}

\def\st{\mathsf{s.t.}}
\def\vect{\mathsf{vec}}
\def\etal{{\em et al.\/}\,}
\def\ie{{\em i.e.\/}}


\newcommand{\KL}{D_{\textup{KL}}}
% Wolfram Mathworld says $L^2$ is for function spaces and $\ell^2$ is for vectors
% But then they seem to use $L^2$ for vectors throughout the site, and so does
% wikipedia.

\makeatletter
\long\def\@makecaption#1#2{
  \vskip 0.8ex
  \setbox\@tempboxa\hbox{\small {\bf #1:} #2}
  \parindent 1.5em  %% How can we use the global value of this???
  \dimen0=\hsize
  \advance\dimen0 by -3em
  \ifdim \wd\@tempboxa >\dimen0
  \hbox to \hsize{
    \parindent 0em
    \hfil 
    \parbox{\dimen0}{\def\baselinestretch{0.96}\small
      {\bf #1.} #2
      %%\unhbox\@tempboxa
    } 
    \hfil}
  \else \hbox to \hsize{\hfil \box\@tempboxa \hfil}
  \fi
}
\makeatother

\newcommand{\lipnorm}[1]{\norm{#1}_{\textup{Lip}}}

\newcommand{\half}{\frac{1}{2}}

\newcommand{\conv}{\textup{Conv}}

\newcommand{\red}[1]{\textcolor{red}{#1}}
\newcommand{\blue}[1]{\textcolor{blue}{#1}}
\newcommand{\darkblue}[1]{\textcolor{darkblue}{#1}}

\newcommand{\Var}{\mathrm{Var}}

\begin{document}

\title{Community Search: A Meta-Learning Approach}

\author{%
	% author names are typeset in 11pt, which is the default size in the author block
	{Shuheng Fang\IEEEauthorrefmark{1}, Kangfei Zhao\thanks{\IEEEauthorrefmark{2} Corresponding author.}\IEEEauthorrefmark{2}, Guanghua Li\IEEEauthorrefmark{3}, Jeffrey Xu Yu\IEEEauthorrefmark{1}} 
	% add some space between author names and affils
	\vspace{1.6mm}\\
	\fontsize{10}{10}\selectfont\itshape
	% 20080211 CAUSAL PRODUCTIONS
	% separate superscript on following line from affiliation using narrow space}%
	% add some space between author names and affils
	\IEEEauthorrefmark{1}The Chinese University of Hong Kong,
	\IEEEauthorrefmark{2}Beijing Institute of Technology\\
	\IEEEauthorrefmark{3}The Hong Kong University of Science and Technology (Guangzhou)\\
	
	\fontsize{9}{9}\selectfont\ttfamily\upshape
	
	\IEEEauthorrefmark{1}{\{shfang,yu\}}@se.cuhk.edu.hk, \IEEEauthorrefmark{2}zkf1105@gmail.com,
	\IEEEauthorrefmark{3}gli945@connect.hkust-gz.edu.cn
	\fontsize{9}{9}\selectfont\ttfamily\upshape
	\vspace{-0.6cm}
}
\vspace{-0.6cm}


\maketitle
\thispagestyle{plain}



\comment{
\author{\IEEEauthorblockN{Shuheng Fang}
\IEEEauthorblockA{\textit{The Chinese University of Hong Kong} \\
%Hong Kong, China\\
shfang@se.cuhk.edu.hk}

\and
\IEEEauthorblockN{Kangfei Zhao}
\IEEEauthorblockA{\textit{Beijing Institute of Technology} \\
%Shenzhen, China\\
zkf1105@gmail.com}

\and
\IEEEauthorblockN{Guanghua Li}
\IEEEauthorblockA{\textit{Wuhan University} \\
%Wuhan, China \\
guanghli@whu.edu.cn}


\and
\IEEEauthorblockN{Jeffrey Xu Yu}
\IEEEauthorblockA{\textit{The Chinese University of Hong Kong} \\
%Hong Kong, China \\
yu@se.cuhk.edu.hk}
}
\maketitle}

\begin{abstract}
Community Search (CS) is one of the fundamental graph analysis tasks, which is a building block of various real applications.
Given any query nodes, CS aims to find cohesive subgraphs that query nodes belong to. 
Recently,  a large number of CS algorithms are designed. 
These algorithms adopt pre-defined subgraph patterns to model the communities, which cannot find ground-truth communities that do not have such pre-defined patterns in real-world graphs.
Thereby, machine learning (ML) and deep learning (DL) based approaches are proposed to capture flexible community structures by learning from ground-truth communities in a data-driven fashion.
These approaches rely on sufficient training data to provide enough generalization for ML models, however, the ground-truth cannot be comprehensively collected beforehand.
%

In this paper, we study ML/DL-based approaches for CS, under the
circumstance of small training data.  Instead of directly fitting the
small data, we extract prior knowledge which is shared across multiple
CS tasks via learning a meta model. Each CS task is a graph with
several queries that possess corresponding partial ground-truth.  The
meta model can be swiftly adapted to a task to be predicted by feeding
a few task-specific training data.  We find that trivially applying multiple
classical meta-learning algorithms to CS suffers from problems
regarding prediction effectiveness, generalization capability and
efficiency.  To address such problems, we propose a novel
meta-learning based framework, Conditional Graph Neural Process
(CGNP), to fulfill the prior extraction and adaptation procedure.  A
meta CGNP model is a task-common node embedding function for
clustering, learned by metric-based graph learning, which
fully exploits the characteristics of CS.  We compare CGNP with
CS algorithms and ML baselines on real graphs with
ground-truth communities.
% \kfadd{
Our experiments verify that CGNP outperforms the other native
graph algorithms and ML/DL baselines 0.33 and 0.26 on \Fone score by average.
%}
The source code has been made available at
\textbf{https://github.com/FangShuheng/CGNP}.
\end{abstract}

\begin{IEEEkeywords}
Community search, Meta-learning, Neural process
\end{IEEEkeywords}

\section{Introduction}

Community is a cohesive subgraph that is densely intra-connected and
loosely inter-connected in a graph. Given any query nodes, community
search (CS) aims at finding communities covering the query nodes,
i.e., local query-dependent communities, which has a wide range of
real applications, e.g., friend recommendation, advertisement in
e-commence and protein complex
identification~\cite{DBLP:journals/vldb/FangHQZZCL20,
  DBLP:series/synthesis/2019Huang}.
%
%  Recently, a large number of CS
%  algorithms~\cite{DBLP:journals/vldb/FangHQZZCL20,
%  DBLP:series/synthesis/2019Huang} have been proposed to efficiently
%  search communities from large graphs in real-time.
% 
In the literature, to model structural cohesiveness, various
community models are adopted, including $k$-core~\cite{cs3,cs4,cs6},
$k$-truss~\cite{cs2,cs7}, $k$-clique \cite{cs1,cs8} and $k$-edge
connected component~\cite{cs9,cs10}. Such models can be computed
efficiently by CS algorithms.
%
But such models are designed based on some pre-defined community
patterns which are too rigid to be used to find ground-truth
communities in real applications.  We show a DBLP example in
Fig.~\ref{fig:case} in which nodes represent researchers
%
% in the
% field of computer science
%
and edges represent their collaboration.
% 
% The ground-truth community
% are identified by the researchers' interest and high-frequency
% keywords used in the papers.
%
The ground-truth community of 'Jure Leskovec', i.e., the orange and
white nodes in Fig.~\ref{fig:case} are with the researchers who have
collaborations and share the common interest of 'social
networks'. Such a community cannot be accurately found with any
$k$-related subgraph patterns. For example, in the community, some
nodes (e.g., Michael W. M.) have one neighbor, which can only be found
by $1$-core that may result in accommodating the whole graph.

\comment{
To this end,
it is quite difficult to choose a proper $k$ value as well as one
community metric to pursue high accuracy.

In other words, in real-world
graphs, it is difficult to have some universal community constraints
for us to

which is the overall best. One constraint may be either too loose or
too tight.


Even for one graph, the topology is diverse in different
subgraphs so that a fixed community constraint may not be consistently
applicable for different local queries.
}

\begin{figure}[t] 
	\centering 
	\includegraphics[width=0.34\textwidth]{fig/caseJure.pdf} 
	\vspace{-0.4cm}
	\caption{An Example on DBLP: Query `Jure Leskovec'} 
	\vspace{-0.4cm}
	\label{fig:case} 
\end{figure}


To tackle the structural inflexibility of CS algorithms, ML/DL-based
solutions~\cite{ICSGNN, AQDGNN} are arising as an attractive research
direction.  They build ML/DL models from given ground-truth
communities and expect the models to generalize to unknown
community-member relationships.
% Compared with traditional algorithms,
Such ML/DL-based approaches have achieved success in finding
high-quality solutions due to two reasons.  For one thing, these
data-driven approaches get rid of the inflexible constraints and adapt
to implicit structural patterns from data. For another thing, the
models can learn via error feedback from its predictions on the query
nodes in the ground-truth communities.
%
%It has been shown that compared with algorithmic approaches, the
%learning-based approaches are more successful in finding high-quality
%solutions.
%
But, effective error feedback heavily relies on sufficient
ground-truth communities to train, which are hard to collect and
label. On the one hand, they are labor-intensive, on the other hand, such
ground-truth communities for different query nodes can be very
different.

\comment{
Specifically, \cite{ICSGNN}
collects user feedback to update a model incrementally and
interactively. The model is trained for a specific query node, whose
effectiveness is fully determined by the quality of the given
ground-truth of that query node.
%
%The effectiveness will be degraded if the user cannot provide enough
%high-quality ground-truth for that query node.
%
\cite{AQDGNN} proposes a graph neural network based model that is
trained by a collection of query nodes with their ground-truth, and
makes predictions for unseen query nodes.  For one graph, a large
volume of query nodes with ground-truth communities are necessary for
training, which ensure the model well generalizes to other local
queries.
}


To deal with this problem, an effective solution is to inject prior
knowledge extracted from multiple CS tasks into the ML model,
%
% instead
% of training one model from scratch over the insufficient training data
% of one task.
%
where one CS task is a subgraph with a small number of
query nodes with partial ground-truth community membership.
% Fortunately,
The implicit prior knowledge of the CS tasks is rather intuitive,
i.e., for any query node of an arbitrary graph, its communities are
the nearby densely connected nodes that share similar attributes with
the query node. Such prior is shared by different CS tasks for
different query nodes in any real-world graphs.
%
% and also serves as the design principle of existing CS algorithms,
% in different forms, e.g., different $k$-related subgraph patterns.
%
%
%For example, in Fig.~\ref{fig:case}, the meta-model that predicts for
%`Jure' is trained by 10 tasks (graphs) drawn from DBLP, which are
%ego-network of `Jeff. Dean, xxxx, etc.', involving communities of
%various research interest. For each graph with more than 500 nodes,
%aaa query nodes with bbb positive and ccc negative samples are used
%for training.
%
%
\comment{
For example, the meta-model is constructed by multiple tasks, which
are ego-networks of researchers in different fields, possessing
several query nodes with ground-truth as training data.  Then, given
several query nodes, e.g., `U. Kang', with partial ground-truth for
the adaptation, the model can be adapted to the task of
Fig.~\ref{fig:case} to search communities for new query nodes, e.g.,
`Jure'.
}
%
%
And the prior knowledge is capable of synthesizing similar or
complementary inductive bias across different CS tasks to compensate
the insufficient knowledge from small training data, thus can be
swiftly adapted to a new task to test.
%
In this paper, we concentrate ourselves on learning a meta model to
capture this prior by meta-learning.

There are existing meta-learning algorithms, e.g., simple feature
transfer and model-agnostic meta-learning. However, trivial
adaptations to CS tasks fail to achieve high performance since they do
not exploit the intrinsic characteristic of the CS tasks.  For CS,
what a model needs to justify for each node in a graph is whether or
not it has its community membership with any given query node.  To
facilitate such binary justification, we propose a novel model,
Conditional Graph Neural Process (CGNP), to generate node embeddings
conditioned on the small training data, where the distance between a
node embedding to that of the query node explicitly indicates their
community membership.  Furthermore, as a graph specification of
Conditional Neural Process (CNP)~\cite{CNP}, CGNP inherits the main
ideas of CNP that implicitly learns a kernel function between a
training query node and a query node to be predicted.  \comment{ In a
  nutshell, our learned CGNP is a common embedding function that
  transforms the nodes of every graph into a distance-aware hidden
  space, as well as a common kernel function between query nodes
  across different graphs, which captures the CS prior knowledge.  }
In a nutshell, the learned CGNP is not only a \emph{common embedding
  function} but also a \emph{common kernel function}, shared across
different graphs. The embedding function transforms the nodes of each
graph into a distance-aware hidden space, while the kernel function
memorizes the small training data of each task as a hidden
representation.  Compared with optimization-based meta-learning
approaches whose parameters are easy to overfit, the metric learning
and memorization mechanisms are more suitable for classification tasks
with small training data, especially for imbalanced labels.
%
% Compared
% with graph algorithms for CS, CGNP learns some patterns that are
% adaptive to the graph data via error feedback with given
% ground-truth.
%
%
%Combining with Graph Neural Network, the metric-based learning
%approach, CGNP is particularly suitable for classification tasks over
%graphs with small training data. 

The contributions of this paper are summarized as follows:
%
%\begin{itemize}[noitemsep,topsep=0pt,parsep=5pt,partopsep=0pt,leftmargin=*]
\ding{172} We formulate the problem of learning a meta model to answer CS queries, where the meta model is to absorb the prior knowledge from multiple CS tasks and adapt to a specific task with only a few training data.  We generalize three CS task scenarios that represent comprehensive query cases. To the best of our knowledge, our study is the first attempt at meta model/algorithm for CS.
%
\ding{173} We explore three Graph Neural Network based solutions, i.e., feature transfer, model-agnostic meta-learning and Graph Prototypical Network, which are trivial adaptations of existing transfer/meta-learning algorithms to CS. 
We identify their individual limitations regarding prediction effectiveness, generalization capability and efficiency.
%
\ding{174} We propose a novel framework, \emph{Conditional Graph Neural Process} (CGNP) on the basis of conceptual CNP and learn the meta model in an efficient, metric-based learning perspective. We design and explore model variants with different model complexities and different options for the core components. To the best of our knowledge, we made the first effort to explore how to solve CS problem by meta-learning.
%We propose a novel framework, \emph{Conditional Graph Neural Process} (CGNP) that learns the meta-model in an efficient, metric-based learning perspective. We design and explore three CGNP models with different model complexities and different options for the core components.
%
% \kfadd{
\ding{175} We conduct extensive experiments on 6 real-world datasets
with ground-truth communities for performance evaluation. Compared
with 3 CS algorithms, 4 naive approaches, and 3 supervised
learning validates our CGNP outperforms the others
%algorithmic and
%ML-based approaches 0.33 and 0.26 on the \Fone on average,
with small training and prediction cost.
% }
%\end{itemize}

%1. study ml for community search, meta model. Discuss  2 native solution.. 
%2. propose CNP differnet variants 
%3. extensive experiment, compare with xxx, achieve xxx performance



%\textbf{Motivation.} 
%ICS-GNN is a learning-based model proposed for interactive community search. It begins from building a candidate subgraph from the query node. Then it apply a GNN model on subgraph and learns the node representation. Finally, a BFS-based algorithm is deployed to find the $k$-sized maximum GNN scores and locate the target community. 

%However, it has some inevitable drawbacks. The first thing is about the pre-defined $k$-sized community. From the perspective of interactive community search, it is convenient for users to adjust $k$ according to their requirements. However, some applications require us to find entire community in subgraphs. For example, we need to figure out the size of interest groups of a student participated in or to find out the circles that contains a specific users in the online social network. For ICS-GNN, the searched community is dependent of size $k$, and it's difficult for us to set a proper $k$ to find the complete community that contains the query node. Second, a model of ICS-GNN can only applies to a specified query node. Each time we encounter a new graph or even a new query node, it has to retrain a new model. That's lack of efficiency. 

%Consider these challenges above, we come up to a meta learning method for community search without a pre-defined community size. Our model can be applied to different graphs and different query nodes after it trained. 


\stitle{Roadmap:} 
The rest of the paper is organized as follows. 
\cref{sec:related} reviews the relative work. In \cref{sec:problem}, we give the problem statement followed by three naive solutions introduced in \cref{sec:naive}.
We introduce the core idea of our approach, CGNP in \cref{sec:metric}. We elaborate on its architecture design and present the learning algorithms of CGNP in \cref{sec:CGNP}. 
We present our comprehensive experimental studies in \cref{sec:exp} and conclude the paper in \cref{sec:conclusion}.

\section{Related Work}\label{sec:related}
 
The authors in \cite{humphreys2007noncontact} showed that it is possible to extract the PPG signal from the video using a complementary metal-oxide semiconductor camera by illuminating a region of tissue using through external light-emitting diodes at dual-wavelength (760nm and 880nm).  Further, the authors of  \cite{verkruysse2008remote} demonstrated that the PPG signal can be estimated by just using ambient light as a source of illumination along with a simple digital camera.  Further in \cite{poh2011advancements}, the PPG waveform was estimated from the videos recorded using a low-cost webcam. The red, green, and blue channels of the images were decomposed into independent sources using independent component analysis. One of the independent sources was selected to estimate PPG and further calculate HR, and HRV. All these works showed the possibility of extracting PPG signals from the videos and proved the similarity of this signal with the one obtained using a contact device. Further, the authors in \cite{10.1109/CVPR.2013.440} showed that heart rate can be extracted from features from the head as well by capturing the subtle head movements that happen due to blood flow.

%
The authors of \cite{kumar2015distanceppg} proposed a methodology that overcomes a challenge in extracting PPG for people with darker skin tones. The challenge due to slight movement and low lighting conditions during recording a video was also addressed. They implemented the method where PPG signal is extracted from different regions of the face and signal from each region is combined using their weighted average making weights different for different people depending on their skin color. 
%

There are other attempts where authors of \cite{6523142,6909939, 7410772, 7412627} have introduced different methodologies to make algorithms for estimating pulse rate robust to illumination variation and motion of the subjects. The paper \cite{6523142} introduces a chrominance-based method to reduce the effect of motion in estimating pulse rate. The authors of \cite{6909939} used a technique in which face tracking and normalized least square adaptive filtering is used to counter the effects of variations due to illumination and subject movement. 
The paper \cite{7410772} resolves the issue of subject movement by choosing the rectangular ROI's on the face relative to the facial landmarks and facial landmarks are tracked in the video using pose-free facial landmark fitting tracker discussed in \cite{yu2016face} followed by the removal of noise due to illumination to extract noise-free PPG signal for estimating pulse rate. 

Recently, the use of machine learning in the prediction of health parameters have gained attention. The paper \cite{osman2015supervised} used a supervised learning methodology to predict the pulse rate from the videos taken from any off-the-shelf camera. Their model showed the possibility of using machine learning methods to estimate the pulse rate. However, our method outperforms their results when the root mean squared error of the predicted pulse rate is compared. The authors in \cite{hsu2017deep} proposed a deep learning methodology to predict the pulse rate from the facial videos. The researchers trained a convolutional neural network (CNN) on the images generated using Short-Time Fourier Transform (STFT) applied on the R, G, \& B channels from the facial region of interests.
The authors of \cite{osman2015supervised, hsu2017deep} only predicted pulse rate, and we extended our work in predicting variance in the pulse rate measurements as well.

All the related work discussed above utilizes filtering and digital signal processing to extract PPG signals from the video which is further used to estimate the PR and PRV.  %
The method proposed in \cite{kumar2015distanceppg} is person dependent since the weights will be different for people with different skin tone. In contrast, we propose a deep learning model to predict the PR which is independent of the person who is being trained. Thus, the model would work even if there is no prior training model built for that individual and hence, making our model robust. 

%
\pdfoutput=1
\documentclass{article}
\usepackage[final]{pdfpages}
\begin{document}
\includepdf[pages=1-9]{CVPR18VOlearner.pdf}
\includepdf[pages=1-last]{supp.pdf}
\end{document}
\section{Experimental Evaluation}
\label{sec:experiment}
To demonstrate the viability of our modeling methodology, we show experimentally how through the deliberate combination and configuration of parallel FREEs, full control over 2DOF spacial forces can be achieved by using only the minimum combination of three FREEs.
To this end, we carefully chose the fiber angle $\Gamma$ of each of these actuators to achieve a well-balanced force zonotope (Fig.~\ref{fig:rigDiagram}).
We combined a contracting and counterclockwise twisting FREE with a fiber angle of $\Gamma = 48^\circ$, a contracting and clockwise twisting FREE with $\Gamma = -48^\circ$, and an extending FREE with $\Gamma = -85^\circ$.
All three FREEs were designed with a nominal radius of $R$ = \unit[5]{mm} and a length of $L$ = \unit[100]{mm}.
%
\begin{figure}
    \centering
    \includegraphics[width=0.75\linewidth]{figures/rigDiagram_wlabels10.pdf}
    \caption{In the experimental evaluation, we employed a parallel combination of three FREEs (top) to yield forces along and moments about the $z$-axis of an end effector.
    The FREEs were carefully selected to yield a well-balanced force zonotope (bottom) to gain full control authority over $F^{\hat{z}_e}$ and $M^{\hat{z}_e}$.
    To this end, we used one extending FREE, and two contracting FREEs which generate antagonistic moments about the end effector $z$-axis.}
    \label{fig:rigDiagram}
\end{figure}


\subsection{Experimental Setup}
To measure the forces generated by this actuator combination under a varying state $\vec{x}$ and pressure input $\vec{p}$, we developed a custom built test platform (Fig.~\ref{fig:rig}). 
%
\begin{figure}
    \centering
    \includegraphics[width=0.9\linewidth]{figures/photos/rig_labeled.pdf}
    \caption{\revcomment{1.3}{This experimental platform is used to generate a targeted displacement (extension and twist) of the end effector and to measure the forces and torques created by a parallel combination of three FREEs. A linear actuator and servomotor impose an extension and a twist, respectively, while the net force and moment generated by the FREEs is measured with a force load cell and moment load cell mounted in series.}}
    \label{fig:rig}
\end{figure}
%
In the test platform, a linear actuator (ServoCity HDA 6-50) and a rotational servomotor (Hitec HS-645mg) were used to impose a 2-dimensional displacement on the end effector. 
A force load cell (LoadStar  RAS1-25lb) and a moment load cell (LoadStar RST1-6Nm) measured the end-effector forces $F^{\hat{z_e}}$ and moments $M^{\hat{z_e}}$, respectively.
During the experiments, the pressures inside the FREEs were varied using pneumatic pressure regulators (Enfield TR-010-g10-s). 

The FREE attachment points (measured from the end effector origin) were measured to be:
\begin{align}
    \vec{d}_1 &= \bmx 0.013 & 0 & 0 \emx^T  \text{m}\\
    \vec{d}_2 &= \bmx -0.006 & 0.011 & 0 \emx^T  \text{m}\\
    \vec{d}_3 &= \bmx -0.006 & -0.011 & 0 \emx^T \text{m}
%    \vec{d}_i &= \bmx 0 & 0 & 0 \emx^T , && \text{for } i = 1,2,3
\end{align}
All three FREEs were oriented parallel to the end effector $z$-axis:
\begin{align}
    \hat{a}_i &= \bmx 0 & 0 & 1 \emx^T, \hspace{20pt} \text{for } i = 1,2,3
\end{align}
Based on this geometry, the transformation matrices $\bar{\mathcal{D}}_i$ were given by:
\begin{align}
    \bar{\mathcal{D}}_1 &= \bmx 0 & 0 & 1 & 0 & -0.013 & 0 \\ 0 & 0 & 0 & 0 & 0 & 1 \emx^T  \\
    \bar{\mathcal{D}}_2 &= \bmx 0 & 0 & 1 & 0.011 & 0.006 & 0 \\ 0 & 0 & 0 & 0 & 0 & 1 \emx^T  \\
    \bar{\mathcal{D}}_3 &= \bmx 0 & 0 & 1 & -0.011 & 0.006 & 0 \\ 0 & 0 & 0 & 0 & 0 & 1 \emx^T 
%    \bar{\mathcal{D}}_i &= \bmx 0 & 0 & 1 & 0 & 0 & 0 \\ 0 & 0 & 0 & 0 & 0 & 1 \emx^T , && \text{for } i = 1,2,3
\end{align}
These matrices were used in equation \eqref{eq:zeta} to yield the state-dependent fluid Jacobian $\bar{J}_x$ and to compute the resulting force zontopes.
%while using measured values of $\vec{\zeta}^{\,\text{meas}} (\vec{q}, \vec{P})$ and $\vec{\zeta}^{\,\text{meas}} (\vec{q}, 0)$ in \eqref{eq:fiberIso} yields the empirical measurements of the active force.



\subsection{Isolating the Active Force}
To compare our model force predictions (which focus only on the active forces induced by the fibers)
to those measured empirically on a physical system, we had to remove the elastic force components attributed to the elastomer. 
Under the assumption that the elastomer force is merely a function of the displacement $\vec{x}$ and independent of pressure $\vec{p}$ \cite{bruder2017model}, this force component can be approximated by the measured force at a pressure of $\vec{p}=0$. 
That is: 
\begin{align}
    \vec{f}_{\text{elast}} (\vec{x}) = \vec{f}_{\text{\,meas}} (\vec{x}, 0)
\end{align}
With this, the active generalized forces were measured indirectly by subtracting off the force generated at zero pressure:
\begin{align}
    \vec{f} (\vec{x}, \vec{p})  &= \vec{f}_{\text{meas}} (\vec{x}, \vec{p}) - \vec{f}_{\text{meas}} (\vec{x}, 0)     \label{eq:fiberIso}
\end{align}


%To validate our parallel force model, we compare its force predictions, $\vec{\zeta}_{\text{pred}}$, to those measured empirically on a physical system, $\vec{\zeta}_\text{meas}$. 
%From \eqref{eq:Z} and \eqref{eq:zeta}, the force at the end effector is given by:
%\begin{align}
%    \vec{\zeta}(\vec{q}, \vec{P}) &= \sum_{i=1}^n \bar{\mathcal{D}}_i \left( {\bar{J}_V}_i^T(\vec{q_i}) P_i + \vec{Z}_i^{\text{elast}} (\vec{q_i}) \right) \\
%    &= \underbrace{\sum_{i=1}^n \bar{\mathcal{D}}_i {\bar{J}_V}_i^T(\vec{q_i}) P_i}_{\vec{\zeta}^{\,\text{fiber}} (\vec{q}, \vec{P})} + \underbrace{\sum_{i=1}^n \bar{\mathcal{D}}_i \vec{Z}_i^{\text{elast}} (\vec{q_i})}_{\vec{\zeta}^{\text{elast}} (\vec{q})}   \label{eq:zetaSplit}
%     &= \vec{\zeta}^{\,\text{fiber}} (\vec{q}, \vec{P}) + \vec{\zeta}^{\text{elast}} (\vec{q})
%\end{align}
%\Dan{These will need to reflect changes made to previous section.}
%The model presented in this paper does not specify the elastomer forces, $\vec{\zeta}^{\text{elast}}$, therefore we only validate its predictions %of the fiber forces, $\vec{\zeta}^{\,\text{fiber}}$. 
%We isolate the fiber forces by noting that $\vec{\zeta}^{\text{elast}} (\vec{q}) = \vec{\zeta}(\vec{q}, 0)$ and rearranging \eqref{eq:zetaSplit}
%\begin{align}
%    \vec{\zeta}^{\,\text{fiber}} (\vec{q}, \vec{P})  &= \vec{\zeta}(\vec{q}, \vec{P}) - \vec{\zeta}(\vec{q}, 0)     \label{eq:fiberIso}
%%    \vec{\zeta}^{\,\text{fiber}}_{\text{emp}} (\vec{q}, \vec{P})  &= \vec{\zeta}_{\text{emp}}(\vec{q}, \vec{P}) - %\vec{\zeta}_{\text{emp}}(\vec{q}, 0)
%\end{align}
%Thus we measure the fiber forces indirectly by subtracting off the forces generated at zero pressure.  


\subsection{Experimental Protocol}
The force and moment generated by the parallel combination of FREEs about the end effector $z$-axis  was measured in four different geometric configurations: neutral, extended, twisted, and simultaneously extended and twisted (see Table \ref{table:RMSE} for the exact deformation amounts). 
At each of these configurations, the forces were measured at all pressure combinations in the set
\begin{align}
    \mathcal{P} &= \left\{ \bmx \alpha_1 & \alpha_2 & \alpha_3 \emx^T p^{\text{max}} \, : \, \alpha_i = \left\{ 0, \frac{1}{4}, \frac{1}{2}, \frac{3}{4}, 1 \right\} \right\}
\end{align}
with $p^{\text{max}}$ = \unit[103.4]{kPa}. 
\revcomment{3.2}{The experiment was performed twice using two different sets of FREEs to observe how fabrication variability might affect performance. The results from Trial 1 are displayed in Fig.~\ref{fig:results} and the error for both trials is given in Table \ref{table:RMSE}.}



\subsection{Results}

\begin{figure*}[ht]
\centering

\def\picScale{0.08}    % define variable for scaling all pictures evenly
\def\plotScale{0.2}    % define variable for scaling all plots evenly
\def\colWidth{0.22\linewidth}

\begin{tikzpicture} %[every node/.style={draw=black}]
% \draw[help lines] (0,0) grid (4,2);
\matrix [row sep=0cm, column sep=0cm, style={align=center}] (my matrix) at (0,0) %(2,1)
{
& \node (q1) {(a) $\Delta l = 0, \Delta \phi = 0$}; & \node (q2) {(b) $\Delta l = 5\text{mm}, \Delta \phi = 0$}; & \node (q3) {(c) $\Delta l = 0, \Delta \phi = 20^\circ$}; & \node (q4) {(d) $\Delta l = 5\text{mm}, \Delta \phi = 20^\circ$};

\\

&
\node[style={anchor=center}] {\includegraphics[width=\colWidth]{figures/photos/s0w0pic_colored.pdf}}; %\fill[blue] (0,0) circle (2pt);
&
\node[style={anchor=center}] {\includegraphics[width=\colWidth]{figures/photos/s5w0pic_colored.pdf}}; %\fill[blue] (0,0) circle (2pt);
&
\node[style={anchor=center}] {\includegraphics[width=\colWidth]{figures/photos/s0w20pic_colored.pdf}}; %\fill[blue] (0,0) circle (2pt);
&
\node[style={anchor=center}] {\includegraphics[width=\colWidth]{figures/photos/s5w20pic_colored.pdf}}; %\fill[blue] (0,0) circle (2pt);

\\

\node[rotate=90] (ylabel) {Moment, $M^{\hat{z}_e}$ (N-m)};
&
\node[style={anchor=center}] {\includegraphics[width=\colWidth]{figures/plots3/s0w0.pdf}}; %\fill[blue] (0,0) circle (2pt);
&
\node[style={anchor=center}] {\includegraphics[width=\colWidth]{figures/plots3/s5w0.pdf}}; %\fill[blue] (0,0) circle (2pt);
&
\node[style={anchor=center}] {\includegraphics[width=\colWidth]{figures/plots3/s0w20.pdf}}; %\fill[blue] (0,0) circle (2pt);
&
\node[style={anchor=center}] {\includegraphics[width=\colWidth]{figures/plots3/s5w20.pdf}}; %\fill[blue] (0,0) circle (2pt);

\\

& \node (xlabel1) {Force, $F^{\hat{z}_e}$ (N)}; & \node (xlabel2) {Force, $F^{\hat{z}_e}$ (N)}; & \node (xlabel3) {Force, $F^{\hat{z}_e}$ (N)}; & \node (xlabel4) {Force, $F^{\hat{z}_e}$ (N)};

\\
};
\end{tikzpicture}

\caption{For four different deformed configurations (top row), we compare the predicted and the measured forces for the parallel combination of three FREEs (bottom row). 
\revcomment{2.6}{Data points and predictions corresponding to the same input pressures are connected by a thin line, and the convex hull of the measured data points is outlined in black.}
The Trial 1 data is overlaid on top of the theoretical force zonotopes (grey areas) for each of the four configurations.
Identical colors indicate correspondence between a FREE and its resulting force/torque direction.}
\label{fig:results}
\end{figure*}






% & \node (a) {(a)}; & \node (b) {(b)}; & \node (c) {(c)}; & \node (d) {(d)};


For comparison, the measured forces are superimposed over the force zonotope generated by our model in Fig.~\ref{fig:results}a-~\ref{fig:results}d.
To quantify the accuracy of the model, we defined the error at the $j^{th}$ evaluation point as the difference between the modeled and measured forces
\begin{align}
%    \vec{e}_j &= \left( {\vec{\zeta}_{\,\text{mod}}} - {\vec{\zeta}_{\,\text{emp}}} \right)_j
%    e_j &= \left( F/M_{\,\text{mod}} - F/M_{\,\text{emp}} \right)_j
    e^F_j &= \left( F^{\hat{z}_e}_{\text{pred}, j} - F^{\hat{z}_e}_{\text{meas}, j} \right) \\
    e^M_j &= \left( M^{\hat{z}_e}_{\text{pred}, j} - M^{\hat{z}_e}_{\text{meas}, j} \right)
\end{align}
and evaluated the error across all $N = 125$ trials of a given end effector configuration.
% using the following metrics:
% \begin{align}
%     \text{RMSE} &= \sqrt{ \frac{\sum_{j=1}^{N} e_j^2}{N} } \\
%     \text{Max Error} &= \max \{ \left| e_j \right| : j = 1, ... , N \}
% \end{align}
As shown in Table \ref{table:RMSE}, the root-mean-square error (RMSE) is less than \unit[1.5]{N} (\unit[${8 \times 10^{-3}}$]{Nm}), and the maximum error is less than \unit[3]{N}  (\unit[${19 \times 10^{-3}}$]{Nm}) across all trials and configurations.

\begin{table}[H]
\centering
\caption{Root-mean-square error and maximum error}
\begin{tabular}{| c | c || c | c | c | c|}
    \hline
     & \rule{0pt}{2ex} \textbf{Disp.} & \multicolumn{2}{c |}{\textbf{RMSE}} & \multicolumn{2}{c |}{\textbf{Max Error}} \\ 
     \cline{2-6}
     & \rule{0pt}{2ex} (mm, $^\circ$) & F (N) & M (Nm) & F (N) & M (Nm) \\
     \hline
     \multirow{4}{*}{\rotatebox[origin=c]{90}{\textbf{Trial 1}}}
     & 0, 0 & 1.13 & $3.8 \times 10^{-3}$ & 2.96 & $7.8 \times 10^{-3}$ \\
     & 5, 0 & 0.74 & $3.2 \times 10^{-3}$ & 2.31 & $7.4 \times 10^{-3}$ \\
     & 0, 20 & 1.47 & $6.3 \times 10^{-3}$ & 2.52 & $15.6 \times 10^{-3}$\\
     & 5, 20 & 1.18 & $4.6 \times 10^{-3}$ & 2.85 & $12.4 \times 10^{-3}$ \\  
     \hline
     \multirow{4}{*}{\rotatebox[origin=c]{90}{\textbf{Trial 2}}}
     & 0, 0 & 0.93 & $6.0 \times 10^{-3}$ & 1.90 & $13.3 \times 10^{-3}$ \\
     & 5, 0 & 1.00 & $7.7 \times 10^{-3}$ & 2.97 & $19.0 \times 10^{-3}$ \\
     & 0, 20 & 0.77 & $6.9 \times 10^{-3}$ & 2.89 & $15.7 \times 10^{-3}$\\
     & 5, 20 & 0.95 & $5.3 \times 10^{-3}$ & 2.22 & $13.3 \times 10^{-3}$ \\  
     \hline
\end{tabular}
\label{table:RMSE}
\end{table}

\begin{figure}
    \centering
    \includegraphics[width=\linewidth]{figures/photos/buckling.pdf}
    \caption{At high fluid pressure the FREE with fiber angle of $-85^\circ$ started to buckle.  This effect was less pronounced when the system was extended along the $z$-axis.}
    \label{fig:buckling}
\end{figure}

%Experimental precision was limited by unmodeled material defects in the FREEs, as well as sensor inaccuracy. While the commercial force and moment sensors used have a quoted accuracy of 0.02\% for the force sensor and 0.2\% for the moment sensor (LoadStar Sensors, 2015), a drifting of up to 0.5 N away from zero was noticed on the force sensor during testing.

It should be noted, that throughout the experiments, the FREE with a fiber angle of $-85^\circ$ exhibited noticeable buckling behavior at pressures above $\approx$ \unit[50]{kPa} (Fig.~\ref{fig:buckling}). 
This behavior was more pronounced during testing in the non-extended configurations (Fig.~\ref{fig:results}a~and~\ref{fig:results}c). 
The buckling might explain the noticeable leftward offset of many of the points in Fig.~\ref{fig:results}a and Fig.~\ref{fig:results}c, since it is reasonable to assume that buckling reduces the efficacy of of the FREE to exert force in the direction normal to the force sensor. 

\begin{figure}
    \centering
    \includegraphics[width=\linewidth]{figures/zntp_vs_x4.pdf}
    \caption{A visualization of how the \emph{force zonotope} of the parallel combination of three FREEs (see Fig.~\ref{fig:rig}) changes as a function of the end effector state $x$. One can observe that the change in the zonotope ultimately limits the work-space of such a system.  In particular the zonotope will collapse for compressions of more than \unit[-10]{mm}.  For \revcomment{2.5}{scale and comparison, the convex hulls of the measured points from Fig.~\ref{fig:results}} are superimposed over their corresponding zonotope at the configurations that were evaluated experimentally.}
    % \marginnote{\#2.5}
    \label{fig:zntp_vs_x}
\end{figure}


\section{Conclusion}
\label{sec:conclusion}
In this paper, we study leveraging ML/DL approaches for community search (CS), under the circumstance that the training data is scarce. 
We propose a metric-based meta-learning framework, Conditional Graph Neural Process (CGNP) to learn a meta model to capture the prior knowledge of CS.
The meta model is adapted to a new task swiftly to make predictions of the community membership, where a task is a graph with only a few given ground-truth. 
To the best of our knowledge, CGNP is the first meta-learning model for CS that utilizes the generalization ability of neural networks to the greatest extent.
Compared with algorithmic approaches, CGNP supports flexible community structures learned from the data. Compared with general meta-learning algorithms, CGNP further exploits the characteristic of CS. Our extensive experiments demonstrate that CGNP outperforms the two lines of approaches significantly regarding accuracy and efficiency.

\section*{Acknowledgment}
This work was supported by the Research Grants Council of Hong Kong, China, No. 14203618, No. 14202919 and No. 14205520.


\bibliographystyle{IEEEtran}
\bibliography{ref}


\end{document}
\endinput
