
%=================================================================================================
\section{Dead time and time-over-threshold calibration}
%=================================================================================================
\label{sec:deadtime}

% Dead time
An important consideration for a highly efficient particle detector is the event loss due to the dead time, especially at high luminosity and high pileup.
As explained in Section~\ref{sec:requirements} the dead time caused by the \gls{afe} depends on the \gls{tot} calibration, and a charge resolution of \num{3000}~e${^{-}}$/TOT$_{40}$ is necessary to achieve the \SI{1}{\percent} dead time required for the innermost layer of the \gls{it} barrel.
The \gls{tot} response can be set by adjusting the discharge current of the \gls{pa}. When the \gls{pa} discharge current increases, the \gls{pa} output returns faster to the baseline and the corresponding \gls{tot} is smaller, as illustrated in Figure~\ref{fig:discharge-schetch}. Therefore, a faster \gls{pa} discharge leads to a reduced detector dead time. In the following the required charge resolution of \num{3000}~e${^{-}}$/TOT$_{40}$ is also referred to as the \emph{fast discharge}.

% PA Discharge sketch:
\begin{figure}[ht]
    \centering
    \includegraphics[width=0.55\textwidth]{Figures/pa_bias.pdf}
    \caption{Sketch of the influence of the discharge current on the signal shape at the output of the \gls{pa} and on the corresponding \gls{tot}.}
    \label{fig:discharge-schetch}
\end{figure}

The \gls{tot} charge resolution of the three \gls{afe}s was measured with a constant charge injection of \num{6000}~e${^{-}}$ for different \gls{pa} discharge currents. First, the charge resolution of all three \gls{afe}s was set to about \num{1100}~e${^{-}}$/TOT$_{40}$, as can be observed in Figure~\ref{fig:dt}. This resolution is not reached for the same current in different \gls{afe}s. %The \gls{sync} \gls{afe} needs more current than the other two, which contributes to a higher power consumption. 
In the next step, the \gls{pa} discharge current was increased to verify the front-end compliance with the dead time requirement. As expected, when the discharge current increases the \gls{pa} discharges faster and the charge resolution is coarser.
All three \gls{afe}s can reach the required charge resolution indicated by the red line. The \gls{sync} and \gls{lin} \glspl{afe} can also discharge faster, while the \gls{diff} \gls{afe} shows a saturation of the \gls{pa} discharge current \gls{dac} and would be operated at its limit to reach the dead time required for the inner layers.

\begin{figure}[ht]
    \centering
    \begin{subfigure}{0.49\textwidth}
        \includegraphics[width=\textwidth]{Figures/discharge_speed.pdf}
        \caption{}
        \label{fig:dt}
    \end{subfigure}
    \hfill
    \begin{subfigure}{0.49\textwidth}
        \includegraphics[width=\textwidth]{Figures/discharge_diff.pdf}
        \caption{}
        \label{fig:dtdiff}
    \end{subfigure}  
    \caption{The charge resolution as a function of the \gls{pa} discharge current (a) measured with a constant charge injection of \num{6000}~e${^{-}}$ for the three RD53A \glspl{afe} and (b) measured for different input charges for the \gls{diff} \gls{afe} only.}
\end{figure}

A dedicated measurement was carried out on the \gls{diff} \gls{afe}, to better understand the observed saturation effect.
The charge resolution of the \gls{diff} \gls{afe} versus the discharge current was measured for different input charges, ranging from \num{3} to \num{20}~ke$^{-}$. The result, presented in Figure~\ref{fig:dtdiff}, confirms the saturation of the discharge current \gls{dac} in this \gls{afe}, occurring at \SI{30}{\percent} of the \gls{dac} range, regardless of the input charge. This implies a marginal operation of this particular \gls{afe} to reach the dead time requirement.

Increasing the discharge current reduces the dead time, as mentioned above, but it also reduces the \gls{afe} stability, and therefore it is likely to induce more noise. Hence the noise was re-evaluated for the fast discharge operation.
The noise was measured for two detection thresholds, \num{1000}~e${^{-}}$ and \num{1200}~e${^{-}}$, and two charge resolutions, \num{1100}~e${^{-}}$/TOT$_{40}$ and the required \num{3000}~e${^{-}}$/TOT$_{40}$.
The combination of these four parameters defined four measurement scenarios for which the average noise occupancy was measured. The measurement method was the same as in Section~\ref{sec:noise}. Pixels with more than \num{100} hits in \num{e6} triggers were declared noisy and masked, then the average noise occupancy of non-masked pixels was defined as the number of noise hits per pixel and per trigger, measured over \num{e6} events.

The fraction of masked pixels is shown in Figure~\ref{fig:dt-noisy-pixels} and the average noise occupancy in Figure~\ref{fig:dt-noise-occ} for the four considered scenarios.
The average noise occupancy of all three \gls{afe}s is higher at fast discharge and is the highest at fast discharge and low threshold, as expected.
The \gls{diff} \gls{afe} demonstrates again excellent noise performance, with almost no noisy pixels and the average noise occupancy well below the requirement, even at fast discharge. At slow discharge the noise in this \gls{afe} was so low that only an upper limit was estimated. The \gls{lin} \gls{afe} has few noisy pixels and the average noise occupancy satisfies the requirement for any scenario. The \gls{sync} \gls{afe} appears to be the noisiest of the three, reaching almost \SI{3.8}{\percent} of noisy pixels when operated at fast discharge and low threshold. The higher noise in this \gls{afe}, significantly increasing with more aggressive chip settings, was considered a critical aspect for the operation in the innermost layer of the CMS Inner Tracker.

\begin{figure}[t]
    \centering
    \begin{subfigure}{0.49\textwidth}
        \centering
        \includegraphics[width=\textwidth]{Figures/speed_noise_1.pdf}
        \caption{}
        \label{fig:dt-noisy-pixels}
    \end{subfigure}
    \hfill
    \begin{subfigure}{0.49\textwidth}
        \centering
        \includegraphics[width=\textwidth]{Figures/speed_noise_2.pdf}
        \caption{}
        \label{fig:dt-noise-occ}
    \end{subfigure}
    \caption{Fraction of masked noisy pixels~(a) and the average noise occupancy after masking~(b) of the three \gls{afe}s in the RD53A chip. The blue colours represent the charge calibration of \num{1100}~e${^{-}}$/TOT$_{40}$ and the red colours represent the charge calibration of \num{3000}~e${^{-}}$/TOT$_{40}$. The darker colours are used for the threshold of \SI{1000}{e^-} and the lighter colours are used for the threshold of \SI{1200}{e^-}.}
\end{figure}