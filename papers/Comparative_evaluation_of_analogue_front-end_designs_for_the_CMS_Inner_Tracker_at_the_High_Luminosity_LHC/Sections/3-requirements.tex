%=================================================================================================
\section{CMS requirements for the analogue front-end}
%=================================================================================================
\label{sec:requirements}

The first step towards the choice of the \gls{afe} for the \gls{cms} final chip was the establishment of the evaluation criteria. The most relevant detector parameters were used to derive the following \gls{cms} requirements:

\paragraph{Optimal threshold.}
The new \gls{cms} readout chip will feature \SI[product-units = power]{50x50}{\micro\meter} pixels, while the pixel size is \SI[product-units = power]{100x150}{\micro\meter} in the present \gls{cms} pixel detector~\citep{p1_tdr}. The readout chip can be bump-bonded either to sensors with square pixels of the same size or to rectangular pixels of \SI[product-units = power]{100x25}{\micro\meter}, thanks to electrode routing in the sensor~\citep{p2_tdr}. Silicon sensors with a thickness of \SI{150}{\micro\meter} will be used. This is about half the thickness of the current \SI{285}{\micro\meter} thick sensors~\citep{p1_tdr}. The main advantage of thin sensors is better radiation tolerance, but the collected signal charge is smaller. The charge distribution obtained with \SI{120}{\giga\electronvolt} protons from a test beam collected in a \SI{130}{\micro\meter} thick sensor with \SI[product-units = power]{100x150}{\micro\meter} pixels is shown in Figure~\ref{fig:mip}. The \gls{mpv} is about \num{7900}~e${^{-}}$ before irradiation. While this number is about \SI{10}{\percent} lower than the expectation, it is well compatible with it within the measurement uncertainties (e.g. due to the charge calibration). The \gls{mpv} decreases by about \num{2000}~e${^{-}}$ after irradiation to \SI{1.2e15}{n_{eq}\per\centi\meter\squared}~\citep{p2_tdr}. Based on the expected signal, a detection threshold of \num{1000}~e${^{-}}$ is required by \gls{cms} for the innermost layer of the \gls{it} to ensure sufficient detection efficiency, especially with irradiated sensors. A threshold of \num{1200}~e${^{-}}$ is sufficient for the outer layers of the detector, where the fluence is lower. 

% Landau:
\begin{figure}[!b]
    \centering
    \includegraphics[width=0.49\textwidth]{Figures/landau_not_irrad_anotated.pdf}
    \includegraphics[width=0.49\textwidth]{Figures/landau_irrad_anotate.pdf}
    \caption{Test beam measurement of the collected charge before (left) and after (right) irradiation to \SI{1.2e15}{n_{eq}\per\centi\meter\squared}, using single pixel clusters, in a \SI{130}{\micro\meter} thick pixel sensor with \SI[product-units = power]{100x150}{\micro\meter} pixels. The red line represents a fit to a Landau distribution convoluted with a Gaussian~\cite{p2_tdr}.}
    \label{fig:mip}
\end{figure}

The number of pixels hit increases with the incidence angle of the particle, giving clusters with large hit multiplicity in particular in the high-$\eta$ part of the barrel. In this specific part of the detector, the charge collection path in a pixel is similar to the pixel dimension in the $z$ direction, hence $\gtrsim$~\SI{50}{\micro\meter} for square pixels and $\gtrsim$~\SI{100}{\micro\meter} for rectangular pixels, to be compared with a charge collection path of \SI{150}{\micro\meter} at normal incidence. For this reason, in the high-$\eta$ region of the barrel square pixels are disfavoured, being more prone to remain below threshold, notably after irradiation.


\paragraph{Radiation tolerance.}
The \gls{it} is the \gls{cms} subdetector closest to the \gls{lhc} interaction point and therefore it is exposed to the highest radiation levels. Two scenarios are envisaged for the HL-LHC: in the "nominal" scenario, the accelerator would deliver a maximum of \num{140} proton-proton (pp) collisions per bunch crossing, to reach a total integrated luminosity of \SI{3000}{\per\femto\barn} by the end of the physics program. In the "ultimate" scenario, the number of pp collisions per bunch crossing would be pushed up to 200, reaching an integrated luminosity of \SI{4000}{\per\femto\barn}.
A fluence reaching \SI{2.6e16}{n_{eq}\per\centi\meter\squared} and a \gls{tid} up to \SI{1.4}{\giga\rad} are expected in the innermost layer in the nominal scenario, while the figures would scale up to \SI{3.4e16}{n_{eq}\per\centi\meter\squared} and \SI{1.9}{\giga\rad} in the ultimate scenario.
The RD53A chip was designed to withstand a \gls{tid} of at least \SI{500}{\mega\rad} and an average leakage current up to \SI{10}{\nano\ampere\per pixel}~\citep{rd53a_specs}. However, with this specification the radiation levels expected in \gls{cms}, reaching \SI{1.9}{\giga\rad} in the ultimate luminosity scenario, would imply a replacement of the innermost layer of the \gls{it} barrel after every two years of operation. The \gls{cms} Collaboration aims for a single replacement of the innermost layer during the ten-year lifetime of the detector, hence a higher radiation tolerance is necessary.

\begin{figure}[b]
    \centering
    \includegraphics[width=0.6\textwidth]{Figures/occupancy.pdf}
    \caption{Simulation of the hit occupancy as a function of pseudorapidity for all layers and double-discs of the \gls{it} for simulated top quark pair production events with a pileup of 200 events~\citep{p2_tdr}.}
    \label{fig:occ}
\end{figure}

\paragraph{Noise occupancy.}
For a stable operation at low threshold it is important to minimize the front-end noise to have an acceptable fraction of spurious hits in the data. Single pixels that are too noisy can be disabled, to keep the overall noise occupancy low, but their fraction must be low in order not to significantly affect the detector efficiency.
%The average noise occupancy has to be much smaller than the hit occupancy. 
Based on the occupancy simulation for different parts of the detector, shown in Figure~\ref{fig:occ}, the average noise occupancy of the new front-end is required to be below \num{e-6}, i.e.~two orders of magnitude below the lowest expected occupancy.




\paragraph{Dead time.} \gls{cms} requires a maximum dead time of \SI{1}{\percent} in the innermost layer of the \gls{it} barrel to ensure high detection efficiency even at the highest expected hit rate. This requirement translates to a maximum efficiency loss of \SI{1}{\percent} at maximum hit rate caused by the total dead time (digital + analogue). The dead time in the RD53A chip has a minor contribution from the digital buffering and a major contribution from the \gls{csa} of the \gls{afe}. While the digital contribution is due to the limited hit buffer size and cannot be reduced with the chip settings, the \gls{afe} dead time depends on the \gls{tot} response calibration. The \gls{tot} response to a given input charge can be set in the chip to a certain number of TOT\textsubscript{40} units (one TOT\textsubscript{40} unit corresponds to one \SI{40}{\mega\hertz} clock cycle, i.e.~to~\SI{25}{\nano\second}). The charge resolution is obtained by dividing the input charge by the corresponding number of clock cycles and can therefore be expressed in e${^{-}}$/TOT\textsubscript{40} units.

% Dead time requirement simulation: 
\begin{figure}[!b]
    \centering
    \includegraphics[width=0.45\textwidth]{Figures/dt_sim_sara.pdf}
    \caption{Hit efficiency losses due to the digital buffering (green) and analogue dead time (blue) simulated in the 200 pileup scenario for two charge resolutions: \num{1500}~e${^{-}}$/TOT$_{40}$ and \num{3000}~e${^{-}}$/TOT$_{40}$. The simulation was done for the centre (c) and edge (e) of the innermost layer (L1) of the \gls{it} barrel and for two pixel geometries: the solid bins represent the \SI[product-units = power]{100x25}{\micro\meter} pixels and hashed bins represent the \SI[product-units = power]{50x50}{\micro\meter} pixels. The red line represents the CMS requirement.}
    \label{fig:dt-simulation}
\end{figure}

A Monte Carlo simulation of hit efficiency losses due to the digital, analogue, and total dead time is shown in Figure~\ref{fig:dt-simulation} for two charge resolutions: \num{1500}~e${^{-}}$/TOT$_{40}$ and \num{3000}~e${^{-}}$/TOT$_{40}$. The simulation was performed for two pixel module positions in the innermost layer of the \gls{it} barrel: the centre ($z=0$), denoted L1c, and the edge, denoted L1e.
For each position, both pixel geometries were simulated. The rectangular pixels are represented with solid bins and the square pixels with hashed bins. The square pixels have a slightly higher inefficiency. 
As expected, the \gls{tot} charge resolution has no influence on the digital dead time, and the hit losses caused by the \gls{afe} are smaller with the coarser charge resolution of \num{3000}~e${^{-}}$/TOT$_{40}$. 
The efficiency losses are higher in the centre making the dead time requirement difficult to meet. With the charge resolution of \num{1500}~e${^{-}}$/TOT$_{40}$ the requirement is not satisfied in any of the two module positions, while with \num{3000}~e${^{-}}$/TOT$_{40}$ the requirement is satisfied on average. The hit efficiency losses are slightly above the requirement in the centre and slightly below at the edge. The charge resolution of \num{3000}~e${^{-}}$/TOT$_{40}$ was therefore taken as the \gls{tot} calibration requirement for the \gls{afe} evaluation.

The impact of charge resolution on tracking performance was also evaluated.
Simulation of the tracking performance for the reconstruction of single muons with a transverse momentum of \SI{10}{\giga\electronvolt} was performed with planar \SI{150}{\micro\meter}-thick sensors, with both sensor pixel geometries and two different thresholds: \num{1200}~e${^{-}}$ and \num{2400}~e${^{-}}$.
The resolution on the transverse ($d$\textsubscript{0}) and longitudinal ($z$\textsubscript{0}) impact parameters, denoted $\sigma(d$\textsubscript{0}) and $\sigma(z$\textsubscript{0}) respectively, integrated over the full $\eta$ range are shown in Figure~\ref{fig:dt-tot-resolution} for three charge resolutions: \num{600}~e${^{-}}$/TOT$_{40}$, \num{3000}~e${^{-}}$/TOT$_{40}$ and \num{6000}~e${^{-}}$/TOT$_{40}$.
The impact parameter resolution deteriorates for a higher threshold and appears to be insensitive to the charge resolution. Since a higher charge resolution does not affect the tracking performance, a charge resolution of \num{3000}~e${^{-}}$/TOT$_{40}$ was taken as the baseline calibration for the inner regions of the Inner Tracker.


% The Tracking resolution versus the discharge speed:
\begin{figure}[htb]
    \centering
    \begin{subfigure}{0.49\textwidth}
        \includegraphics[width=\textwidth]{Figures/dt_sim_ernesto11.pdf}
    \end{subfigure}
    \hfill
    \begin{subfigure}{0.49\textwidth}
        \includegraphics[width=\textwidth]{Figures/dt_sim_ernesto12.pdf}
    \end{subfigure}  
    \caption{Influence of the charge resolution on the transverse (left) and longitudinal (right) impact parameter resolution obtained from simulation. The \SI[product-units = power]{100x25}{\micro\meter} pixels are represented in blue and the \SI[product-units = power]{50x50}{\micro\meter} pixels are represented in red. The full markers and solid lines indicate the threshold of \SI{1200}{e^-} and the open markers and dashed lines indicate the threshold of \SI{2400}{e^-}.}
    \label{fig:dt-tot-resolution}
\end{figure}