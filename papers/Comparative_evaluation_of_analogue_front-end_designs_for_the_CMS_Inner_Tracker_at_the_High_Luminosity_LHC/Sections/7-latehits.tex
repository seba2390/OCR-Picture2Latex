
%=================================================================================================
\section{Late-detected hits}
%=================================================================================================
\label{sec:late}

% Intro:
The time response of the pixel readout chip is important to assign detected hits to their corresponding \gls{lhc} \glspl{bx} and to limit the number of spurious hits from out-of-time pileup interactions.
% Signal amplitude:
The time response of the \gls{afe}, i.e.~the combination of the \gls{pa} rise time and the discriminator speed, is a function of the input charge. Pulses with the same peaking time but different amplitude pass the discriminator threshold at different times. High amplitude signals, depicted in light blue in Figure~\ref{fig:late_hits_sketch}, pass the threshold within one \gls{bx}, i.e.~within \SI{25}{\nano\second}. If the deposited charge is just above the threshold instead, the signal rises more slowly and is detected later by the discriminator. Such a hit, shown in red, might be assigned to the following \gls{bx}, and appears as a spurious hit in another event. The smallest charge ($\mathrm{Q_{min}}$) that can be detected within the correct \gls{bx} (dark blue signal) is equivalent to the so-called "in-time threshold", which is higher than the threshold of the discriminator. 
% Define time walk:
The time behaviour of an \gls{afe} is typically described by its time walk curve, i.e.~the response delay of the discriminator as a function of the input charge. An example of a simulated time walk curve of the \gls{lin} \gls{afe} is shown in Figure~\ref{fig:time_walk}. Given that the discriminator of the \gls{sync} \gls{afe} is locked to the clock, the following method is used to compare the timing of the three \glspl{afe}.

%Given that the discriminator of one of the three \gls{afe}s is synchronous to the clock, the time walk of this \gls{afe} cannot be directly measured and another method had to be used to compare the timing of the three \glspl{afe}.

% Sketch of signal amplitude vs. time walk + time walk curve:
\begin{figure}[t]
    \centering
    \captionsetup{justification=centering}
    \begin{minipage}{0.49\textwidth}
        \centering
        \includegraphics[width=0.89\textwidth]{Figures/late_hits_sketch.pdf}
        \caption{Illustration of the discriminator time response for different signal amplitudes.}
        \label{fig:late_hits_sketch}
    \end{minipage}
    \hfill
    \begin{minipage}{0.49\textwidth}
        \centering
        \includegraphics[width=0.89\textwidth]{Figures/time_walk.pdf}
        \caption{Simulated time walk curve of the linear front-end in the RD53A chip.}
        \label{fig:time_walk}
    \end{minipage}
\end{figure}




%-------------------------------------------------------------------------------------------------
\subsection*{Front-end time response measurement}

The charge injection in the RD53A chip can be delayed with respect to the rising edge of the clock with a step size of \SI{1.5625}{\nano\second}~\citep{rd53a_manual}. The front-end time response was measured by injecting calibration pulses with different amplitudes and with different time delays. The detection threshold was set to \num{1000}~e${^{-}}$ and the full range of available charges up to \num{35}~ke${^{-}}$ was scanned, using a finer charge step for low charges where timing is critical. For every pixel the charge was injected \num{50} times for each time delay. Figure~\ref{fig:3tor} shows the two-dimensional plot of charge versus time for all three \glspl{afe}. %, also called a "tornado plot". 
The \hbox{$x$ axis} represents time in nanoseconds and $t=0$ indicates the time when the highest charge is detected. The \hbox{$y$ axis}, showing the injected charge in electrons, is limited to \num{10}~ke${^{-}}$ in this figure. The colour code indicates the detection probability for a given \gls{bx}, for each combination of charge and injection delay. The yellow zone corresponds to \SI{100}{\percent} detection efficiency, while in the white-coloured region no hit is detected. The left edge of the coloured region corresponds to the time walk curve.
In the upper part of the plot, the coloured region is a straight rectangle with a time width of \SI{25}{\nano\second}, which confirms that high charges are always detected within one \gls{bx}. Small charges instead are detected later, resulting in a tail in the detection region. This tail extends up to about \SI{40}{\nano\second} in the \gls{sync} and \gls{diff} \gls{afe}, indicating that these two \glspl{afe} have a comparable time response. The \gls{diff} \gls{afe} is able to correctly assign slightly smaller charges than the \gls{sync} \gls{afe}. %, but it has a less sharp transition on the edges, due to a slower rise time of the comparator. 
The \gls{lin} \gls{afe} appears to be the slowest of the three, with the largest time walk of more than two \glspl{bx}.
%the longest tail, up to two \gls{bx}s later.

% 3 Tornados:
\begin{figure}[ht]
    \centering
    \includegraphics[width=\textwidth]{Figures/tornado_slow_special_threeFEs.pdf}
    \caption{The measured time response of the \gls{sync} \gls{afe} (left), the \gls{lin} \gls{afe} (middle), and the \gls{diff} \gls{afe} (right) obtained with one RD53A chip.}
    \label{fig:3tor}
\end{figure}



%-------------------------------------------------------------------------------------------------
\subsection*{Combination with time of arrival simulation}

% Simulation:
A Monte Carlo simulation was performed within the standard \gls{cms} simulation and reconstruction software framework called CMSSW~\citep{cmssw} to evaluate the influence of the time response of each \gls{afe} on the detector performance and to estimate the resulting fraction of spurious hits. The time of arrival of particles was simulated for different locations in the \gls{it} detector, given that it depends on the position of the pixel module with respect to the interaction point. Sixteen different locations were studied: the centre ($z=0$) and edge module of each barrel layer, the innermost and outermost module of the first and last small disc, as well as of the first and last large disc (Figure~\ref{fig:itlayout}). For each location about \num{2000} minimum bias QCD events, without pileup and without a transverse momentum cut, were simulated. For the simulation of the beam spot a Gaussian distribution with a width ($\sigma$) in $z$ of \SI{4}{\centi\meter}, corresponding to a Gaussian width of \SI{130}{\pico\second} in time, was simulated.
The simulated pixel hits, corresponding to single pixels with deposited charge, were sorted by released charge, ranging from \num{600}~e${^{-}}$ \footnote{Given that charges smaller than \SI{600}{e^-} are not expected to be detected because of the threshold, the simulation started at this charge to avoid overloading the computing time.} to \num{50}~ke${^{-}}$, with a granularity of \num{150}~e${^{-}}$ and a time resolution of \SI{0.25}{\nano\second}. The simulated time of arrival versus charge distribution for the central module ($z=0$) of the innermost layer of the \gls{it} barrel is presented in Figure~\ref{fig:mc}.

% Tornado transform to probability:
Such a distribution was combined with the time response measurement introduced in the previous section.
The \hbox{$x$ axis} of the time response is reversed, obtaining the acceptance region, in time and charge, giving the probability of a charge to be detected in the correct \gls{bx}. This way, instead of showing when a hit is detected by the electronics, the figure indicates when a hit has to occur to be detected in a given \gls{bx}.
The \hbox{$y$ axis} has to be extended to match the charge range in the simulation. Assuming that the time response remains constant for very large signals, the yellow region with sharp edges is extended up to \num{50}~ke${^{-}}$.
For illustration, the time response of the \gls{lin} \gls{afe}, after such modifications, is presented in Figure~\ref{fig:torproba}.

% Figure: probability tornado AND simulation:
\begin{figure}[ht]
    \centering
    \captionsetup{justification=centering}
    \begin{subfigure}{0.49\textwidth}
        \centering
        \includegraphics[width=\textwidth]{Figures/toa_14.pdf}
        \caption{Time of arrival simulation.}
        \label{fig:mc}
    \end{subfigure}
    \hfill
    \begin{subfigure}{0.49\textwidth}
        \centering
        \includegraphics[width=\textwidth]{Figures/tornado_14.pdf}
        \caption{\gls{afe} acceptance region for a given \gls{bx}.}
        \label{fig:torproba}
    \end{subfigure}
    \begin{subfigure}{0.49\textwidth}
        %\centering
        \includegraphics[width=\textwidth]{Figures/overlap_14.pdf}
        \caption{Overlay between measurement and simulation.}
        \label{fig:overlap}
    \end{subfigure}
    \hspace{2pt}
    \begin{subfigure}{0.49\textwidth}
        \centering
        \includegraphics[width=\textwidth]{Figures/beta_14.pdf}
        \caption{Late-detected hits.}
        \label{fig:beta}
    \end{subfigure}
    \caption{Different steps of the time response evaluation method.}
\end{figure}

% Combine Measurement and simulation:
When the acceptance region of the front-end is superimposed with the hit distribution from simulation, as indicated in Figure~\ref{fig:overlap}, the hits that are inside the yellow part of the acceptance region have a \SI{100}{\percent} probability to be assigned to the correct \gls{bx}. On the other hand, hits that are outside of the detection region have zero probability to be detected in time. Figure~\ref{fig:beta} shows the hits that will be assigned to a wrong \gls{bx}, obtained from the exclusion of the two overlaid plots. The integral of the exclusion plot divided by the total number of hits gives the fraction of late-detected hits in a given location of the future detector.

% Time alignment:
An important part of this method is the time alignment of the two overlapping plots. The origin of the time axis of both the measurement and the simulation have to be correctly aligned.
The $t=0$ of the simulation corresponds to the time when the two proton bunches overlap in the interaction region, corrected with the expected time-of-flight from the interaction point to the given module. The zero of the chip acceptance can be shifted to maximize the overlap, as it would be done in the detector by calibration. For this measurement, the peak of the simulation is placed three fully efficient bins from the left edge of the acceptance region, i.e.~\SI{4.625}{\nano\second}, as it is indicated in Figure~\ref{fig:overlap}. This estimate of about \SI{5}{\nano\second} was used to account for the imperfect time alignment in the detector due to the variations in the length of the electrical links, jitter, and other contributions, and also including some margin.




%-------------------------------------------------------------------------------------------------
\subsection*{Fraction of late hits}

The method described above was used to evaluate the fraction of hits detected late by the three \gls{afe} designs. The result is shown in Figure~\ref{fig:tor_late_hits} for the selected detector locations.
The left half of the histogram corresponds to the \gls{it} barrel layers, numbered from the centre outwards L1 to L4. For each layer the study was done for two pixel modules, one at the edge (e) and the one in the centre (c) of the barrel. The fraction of late hits increases with the distance from the interaction point. 
The right half of the histogram is dedicated to the discs, numbered D1 to D12 with increasing distance from the interaction point. For each disc one module on the innermost (i) and one on the outermost (o) ring is presented. For any given disc the fraction of late detected hits is higher on the outer ring.

% Square tornado:
An ideal front-end with infinitely fast time response was also simulated %by creating a time response plot with perfectly rectangular shape. 
and the fraction of hits detected late was estimated using the same method described above. Results are shown in grey in Figure~\ref{fig:tor_late_hits} overlaid to the estimates of the actual \acrlong{afe}s, because they represent the irreducible background. For the considered positions, this fraction is between \SI{0.38}{\percent} and \SI{7.26}{\percent}.
These are hits generated by particles whose travel time up to the sensor is more than \SI{25}{\nano\second} longer than the minimum, for which the detector is tuned. 
%In fact, some particles take more time than expected to arrive to the pixel module, as it is the case for instance with the so-called loopers, i.e.~low $p_T$ particles with helical trajectory. 
%The three \glspl{afe} introduce more late hits than the ideal \gls{afe}. 
The \gls{sync} and \gls{diff} \gls{afe} have similar performance, causing few percent of misassigned hits on top of the background. The \gls{diff} is slightly faster.
The \gls{lin} \gls{afe} instead is significantly slower, causing up to additional \SI{11}{\percent} of late hits in the detector, on top of the irreducible \SI{7}{\percent} in the worst case.

\begin{figure}[t]
    \centering
    \includegraphics[width=\textwidth]{Figures/latehits.pdf}
    \caption{Fraction of hits detected late by the three RD53A \glspl{afe} for 16 pixel module positions. The \gls{it} barrel layers are numbered from the centre outwards L1 to L4 and for each layer the module at the edge is denoted "e" and the one in the centre is denoted "c". The \gls{it} discs are numbered D1 to D12 with increasing distance from the interaction point and for each disc the innermost ring is denoted "i" and the outermost one is denoted "o".}
    \label{fig:tor_late_hits}
\end{figure}

%-------------------------------------------------------------------------------------------------
\subsection*{LIN AFE slow time response mitigation}

Following the outcome of the previous measurement, a modification of the discriminator circuit was proposed by the design team to improve the time response of the \gls{lin} \gls{afe}. 
The discriminator is composed of two stages: a transconductance stage and a \acrfull{tia}. In the \gls{tia} two diode-connected transistors, initially introduced to minimise the static current consumption at the output of the discriminator, were forcing other transistors to operate in the deep sub-threshold regime, consequently making them slower. A significant improvement in time walk at the cost of a marginal increase in static current consumption was achieved by removing those two transistors. This led to a simpler \gls{tia} stage in the new design of the \gls{lin} \gls{afe}~\citep{new_lin}, for the next version of the chip, called RD53B~\citep{rd53b_manual}.

Circuit simulations were used to extract the time walk curves of both the original and the improved \gls{lin} \gls{afe} designs. They were transformed into time response plots and combined with the time of arrival simulations to estimate the fraction of late hits for the simulated designs.
The simulated RD53A design was compared to the measurement and the difference in late hits is shown in Figure~\ref{fig:meas_vs_sim}. The simulated \gls{afe} gives a slightly higher number of late hits. Nevertheless, the simulation demonstrates a very good agreement with the measurement, the difference in late hits being below \SI{1}{\percent}. This confirms the validity of the simulation, which can therefore be used to predict the fraction of late hits in the improved design. The difference in late hits between the original design (RD53A) and the new one (RD53B) is also shown in Figure~\ref{fig:meas_vs_sim}. The new \gls{lin} \gls{afe} demonstrates on average \SI{5}{\percent} less misassigned hits. 
The improved design of the \gls{lin} \gls{afe} was also implemented in a test chip and verified before and after irradiation. The simulation and measurement results after an irradiation up to \SI{1}{\giga\rad} confirmed the improvement in time walk, which remains below \SI{20}{\nano\second}, whereas it increases to around \SI{30}{\nano\second} in the RD53A version~\citep{new_lin}.
%show that the degradation of time walk after irradiation is minor~\citep{new_lin}.

% Histogram New vs Old LIN:
\begin{figure}[t]
    \centering
    \includegraphics[width=0.4\textwidth]{Figures/lin_late_hits_legend.pdf}
    \caption{Difference between the fraction of hits detected late by the simulated and measured RD53A design of the \gls{lin} \gls{afe} (light green) and difference between the fraction of hits detected late by the simulated RD53A version of the \gls{lin} \gls{afe} and the improved RD53B version (dark green). }
    \label{fig:meas_vs_sim}
\end{figure}

%-------------------------------------------------------------------------------------------------
\subsection*{Late-hit occupancy}

The fraction of late-detected hits was converted to the occupancy due to late hits, using the simulated hit occupancies extracted from Figure~\ref{fig:occ}. The result is shown in Figure~\ref{fig:late-hit-occ} for all the positions in the detector. The irreducible background of misassigned hits is almost uniform in the tracker and amounts to between \num{e-5} and \num{e-4}, regardless of the \gls{afe} design. The late-hit occupancy levels are at least one order of magnitude above the required noise level, indicated by the red line in the figure. Hence the spurious hit rate in the detector is dominated by the time response of the \gls{afe}, not by the noise.
Moreover, the performance of the improved design of the \gls{lin} \gls{afe} is comparable to the other two \glspl{afe}, although it remains slightly higher. 

% Late hit occupancy:
\begin{figure}[ht]
    \centering
    \includegraphics[width=\textwidth]{Figures/late_hit_occ.pdf}
    \caption{The occupancy due to hits detected late by the RD53A \glspl{afe} for different module positions in the detector. The \gls{it} barrel layers are numbered from the centre outwards L1 to L4 and for each layer the module at the edge is denoted "e" and the one in the centre is denoted "c". The \gls{it} discs are numbered D1 to D12 with increasing distance from the interaction point and for each disc the innermost ring is denoted "i" and the outermost one is denoted "o".}
    \label{fig:late-hit-occ}
\end{figure}