%=================================================================================================
\section{CMS pixel detector upgrade for the High Luminosity LHC}
%=================================================================================================
\label{sec:intro}

% High luminosity LHC:
The High Luminosity upgrade~\citep{hllhc} of the CERN \gls{lhc}~\citep{lhc} will boost its potential for physics discoveries, but also impose extreme operating conditions for the experiments. Along with the accelerator, the \gls{cms}~\citep{cms_exp} detector will be substantially upgraded during \acrlong{ls3}, starting in \num{2025}~\citep{hllhc_web}. This upgrade is referred to as the \gls{cms} Phase-2 Upgrade~\citep{cms_p2}.
% CMS Tracker:
The silicon tracking system, located at the heart of \gls{cms}, detects trajectories of charged particles. It will be entirely replaced during the \acrlong{ls3} because of the accumulated radiation damage and to take advantage of the increased luminosity. The goal of the upgrade is to maintain or improve the tracking and vertex reconstruction performance of the detector in the harsh environment of the \gls{hllhc}. The \gls{cms} Phase-2 tracker will consist of the \acrlong{ot}, made of silicon modules with strip and macro-pixel sensors, and the \gls{it}, based on silicon pixel modules~\citep{p2_tdr}.

% CMS Inner Tracker
The high granularity of the \gls{it} offers excellent spatial resolution, which is important for a precise three-dimensional reconstruction of particle trajectories, as well as the identification of primary interaction vertices and secondary decay vertices. 
% IT Layout:
One quarter of the latest layout of the Phase-2 \gls{it} in the $r$-$z$\footnote{\gls{cms} adopts a right-handed coordinate system with the origin centred at the nominal collision point inside the experiment. The $x$ axis points towards the centre of the \gls{lhc}, the $y$ axis points vertically upwards and the $z$ axis points along the beam direction. The azimuthal angle $\phi$ is measured from the $x$ axis in the $x$-$y$ plane, the radial coordinate in this plane is denoted by $r$ and the polar angle $\theta$ is measured from the $z$ axis. The pseudorapidity $\eta$ is defined as $\eta = -ln\tan (\theta/2)$~\citep{p2_tdr}.} 
view is shown in Figure~\ref{fig:itlayout}. The \gls{it} will consist of a barrel component with four cylindrical layers, referred to as the \gls{tbpx}. Eight smaller double-discs forming the \gls{tfpx} and four larger double-discs forming the \gls{tepx} will be placed in the forward direction on each side. The forward acceptance will be extended up to a pseudorapidity of $|\eta| = 4$~\citep{p2_tdr}, as indicated by the red line in Figure~\ref{fig:itlayout}.

% IT Layout figure
\begin{figure}[t]
    \centering
    \includegraphics[width=0.8\textwidth]{Figures/ITlayout_annot.pdf}
    \caption{Layout of one quarter of the Phase-2 Inner Tracker in the $r$-$z$ view. Green lines correspond to pixel modules with two readout chips and orange lines represent modules with four chips. The modules shown in brown correspond to the innermost ring of the last \gls{tepx} disc ($z=\SI{2650}{\milli\meter}$), which will be used by the Beam Radiation Instrumentation and Luminosity (BRIL) project~\citep{bril_cdr} for dedicated luminosity and background measurements. The grey line represents the beam pipe envelope.}
    \label{fig:itlayout}
\end{figure}

% IT System:
The \gls{it} detector will have an active area of \SI{4.9}{\meter\squared} and it will be composed of \num{3892} pixel modules. Two pixel sizes are currently being considered for the Phase-2 \gls{it}: \SI[product-units = power]{100x25}{\micro\meter} pixels and \SI[product-units = power]{50x50}{\micro\meter} pixels. With a pixel size of \SI{2500}{\micro\meter\squared} there would be about \num{2}~billion readout channels. The detector design strives for a minimal mass of the detector to avoid degradation of the tracking performance due to the interactions of particles with the detector material. Therefore, lightweight mechanical structures made of carbon fiber, two-phase CO\textsubscript{2} cooling~\citep{p2_tdr} and a low voltage powering scheme based on serial powering~\citep{sp_vertex19} will be used. The data will be transmitted through low-mass electrical links and optical fibers~\citep{p2_tdr} to further reduce the detector mass.

% Hybrid pixel module and detection principle:
The main building block of the \gls{it} system is a hybrid pixel module, shown in Figure~\ref{fig:module}. It is composed of a silicon sensor bump-bonded to two or four readout chips. Pixel modules with two readout chips are indicated in green in Figure~\ref{fig:itlayout} and modules with four chips are indicated in orange.
The readout chips of the module are wire-bonded to a flexible printed circuit board with passive components and connectors, called the \acrlong{hdi}, that distributes power (low voltage to power the pixel chips and high voltage to bias the sensor), clock and control signals and collects data from the chips. Signals produced in the sensor are transmitted to the front-end electronics, where they are processed and the data are stored during the planned \SI{12.8}{\micro\second} trigger latency interval. The hit information is sent to the back-end \acrlong{daq} system of the experiment only after receipt of a Level-1 trigger~\citep{L1_trigger} signal. In the innermost layer of the \gls{it}, the hit rate will reach \SI{3.5}{\giga\hertz\per\centi\meter\squared}, while the \gls{cms} Level-1 accept rate will increase from \num{100} to \SI{750}{kHz}~\citep{L1_trigger}.

% Pixel module
\begin{figure}[!ht]
    \centering
    \includegraphics[width=0.6\textwidth]{Figures/module.pdf}
    \caption{A 3D exploded view of the Phase-2 Inner Tracker pixel module with four readout chips~\citep{bane_module}. From top to bottom the following components can be seen: the high density interconnect, the silicon pixel sensor, the RD53 readout chips, and the rails to mount the module on a support structure and to ensure the electrical isolation of the module.}
    \label{fig:module}
\end{figure}