
%=================================================================================================
\section{Equalization of threshold dispersion}
%=================================================================================================
\label{sec:rad}

The calibration injection circuit is used to inject a range of charges to measure the threshold of each pixel and the threshold dispersion across the matrix. The occupancy versus charge of a pixel is a sigmoid from \num{0} to \SI{100}{\percent} occupancy, commonly called an S-curve. An example of an S-curve plot including more than \num{26000} pixels is shown in Figure~\ref{fig:scurves}. 
The calibration charge at which \SI{50}{\percent} occupancy is reached is taken as a measurement of the charge equivalent of the threshold of each pixel. 
The mean value of the pixel threshold distribution represents the global threshold and the \gls{rms} is the threshold dispersion.
Typically, pixel-to-pixel variations result in a threshold dispersion of several hundred electrons, which can be reduced to less than hundred electrons after setting optimal trim bits for each pixel with a dedicated tuning algorithm. Examples of untuned and tuned threshold distributions are shown in Figure~\ref{fig:uthrdist} and Figure~\ref{fig:tthrdist}, respectively.

% Example of scurve and thr. distributions:
\begin{figure}[ht]
    \centering
    \begin{subfigure}{0.37\textwidth}
        \centering
        \captionsetup{justification=centering}
        \includegraphics[width=\textwidth]{Figures/scurve.pdf}
        \caption{S-curves.\newline}
        \label{fig:scurves}
    \end{subfigure}
    \hfill
    \begin{subfigure}{0.29\textwidth}
        \centering
        \captionsetup{justification=centering}
        \includegraphics[width=\textwidth]{Figures/thr_dist_untuned.pdf}
        \caption{Untuned threshold distribution.}
        \label{fig:uthrdist}
    \end{subfigure}
    \hfill
    \begin{subfigure}{0.29\textwidth}
        \centering
        \captionsetup{justification=centering}
        \includegraphics[width=\textwidth]{Figures/thr_dist_tuned.pdf}
        \caption{Tuned threshold distribution.}
        \label{fig:tthrdist}
    \end{subfigure}
    \label{fig:thrdist}
    \caption{S-curves (a) and threshold distribution before (b) and after (c) tuning obtained with all pixels of the \gls{lin} \gls{afe} of one RD53A chip. The red lines in the threshold distributions represent fits to Gaussian distributions and the mean ($\mu$) and width ($\sigma$) of the fit functions are given.}
\end{figure}

The threshold tuning capability of the three \gls{afe} designs was tested and proven to be functional in many samples, some of which were irradiated and re-evaluated afterwards. An assembly of an RD53A chip and a sensor with \SI[product-units = power]{50x50}{\micro\meter\squared} pixels was irradiated at Karlsruhe Institute of Technology~\citep{kit} with \SI{23}{\mega\electronvolt} protons up to a fluence of \SI{3e15}{n_{eq}\per\centi\meter\squared}, corresponding to a \gls{tid} reaching \SI{350}{\mega\rad}. The RD53A chip was not powered during irradiation. The sample was irradiated at room temperature and maintained at cold temperature after irradiation to avoid annealing. It was tested at \SI{-10}{\celsius} in a dry environment. The sensor bias voltage was adjusted to reach an average leakage current of \SI{10}{\nano\ampere}$/$pixel, which is the maximum specified value for the RD53A chip \cite{rd53a_specs}.
The pixels with a noise occupancy higher than \num{e-4} were considered noisy and masked, based on the lowest hit occupancy expected from the simulation presented in Figure~\ref{fig:occ}. The remaining pixels were tuned to a threshold of \num{1000}~e${^{-}}$. Pixels with an anomalously high threshold that could not be adjusted with the range of the trim bits were masked in the tuning procedure. The threshold distributions of the three \gls{afe}s after tuning are shown in Figure~\ref{fig:irrad_3afes_thr}.

\begin{figure}[t]
    \centering
    \begin{subfigure}{0.32\textwidth}
        \centering
        \includegraphics[width=\textwidth]{Figures/Irrad_SYNC.pdf}
        \caption{SYNC AFE.}
    \end{subfigure}
    \hfill
    \begin{subfigure}{0.32\textwidth}
        \centering
        \includegraphics[width=\textwidth]{Figures/Irrad_LIN.pdf}
        \caption{LIN AFE.}
    \end{subfigure}
    \hfill
    \begin{subfigure}{0.32\textwidth}
        \centering
        \includegraphics[width=\textwidth]{Figures/Irrad_DIFF.pdf}
        \caption{DIFF AFE.}
    \end{subfigure}
    \caption{Threshold distributions of the \gls{sync} \gls{afe} (a), the \gls{lin} \gls{afe} (b), and the \gls{diff} \gls{afe}~(c) obtained with an RD53A chip after irradiation. The mean and \gls{rms} were calculated using all non-masked pixels in each \gls{afe}.}
    \label{fig:irrad_3afes_thr}
\end{figure}

All three \gls{afe}s were functional after irradiation and could reach the required threshold with a threshold dispersion of about \num{100}~e${^{-}}$. The threshold tuning of the \gls{lin} \gls{afe} worked well and only \SI{0.1}{\percent} of pixels were masked. The auto-zeroing in the \gls{sync} \gls{afe} worked well too, however the leakage current caused a higher noise in this front-end and \SI{3}{\percent} of the pixels were masked. The threshold distribution of the \gls{diff} \gls{afe} features a narrow core and long tails, and  \SI{11.2}{\percent} of pixels were masked.
The large fraction of masked pixels in the \gls{diff} \gls{afe} is the consequence of a long tail in the untuned threshold distribution, shown in Figure~\ref{fig:irrad_diff_untuned}. Pixels with a too high threshold cannot be tuned to the desired threshold value because the range covered by the trim bits is a global setting for the whole chip. As the threshold dispersion depends on several \gls{afe} parameters, many parameter combinations were tried to mitigate the problem, and the \SI{11.2}{\percent} of masked pixels was the best result that could be achieved with this irradiated sample.

\begin{figure}[t]
    \centering
    \includegraphics[width=0.33\textwidth]{Figures/irrad_diff_untuned.pdf}
    \caption{Untuned threshold distribution of the \gls{diff} \gls{afe} in the irradiated RD53A chip.}
    \label{fig:irrad_diff_untuned}
\end{figure}

This study triggered an investigation and the design team discovered that the combined effect of irradiation and cold temperature resulted in a PMOS threshold increase in the \gls{diff} precomparator, resulting in a small voltage margin.
Simulations showed that the voltage margin is smaller at cold temperature and decreases with irradiation, reaching a value close to zero for the \gls{diff} \gls{afe} design implemented in the RD53A chip after irradiation to \SI{200}{\mega\rad}, which explains the problematic threshold tuning observed after irradiation to \SI{350}{\mega\rad}.
A design improvement of the \gls{diff} precomparator was proposed and simulated, obtaining an extension of the expected operation range up to \SI{500}{\mega\rad}, although this radiation level is still low compared to the dose expected in the \gls{cms} detector. With such operation range, replacements of the innermost layer would be required every two years once the ultimate luminosity is reached, while \gls{cms} is aiming at a single replacement during the whole high-luminosity program. For this reason the attention turned to the other two \gls{afe}s, which seem promising candidates for a higher radiation tolerance.