\section{Conclusions}

A new generation pixel readout chip is being designed for the upgrade of the \gls{cms} \acrlong{it} to cope with stringent requirements imposed by unprecedented radiation levels and hit rates. Three different analogue front-ends were designed by the RD53 Collaboration and implemented in a large scale demonstrator chip (RD53A). The three designs were characterized and the expected detector performance was evaluated against the requirements to choose the most suitable option for \gls{cms}.

The \acrlong{diff} \acrlong{afe} showed the best noise performance, with the noise occupancy several orders of magnitude below the requirement, as well as a very good time response. Nevertheless, this \acrlong{afe} showed a problematic threshold tuning after irradiation, at cold temperature (\SI{-10}{\celsius}) and with high leakage current. An improved design was proposed, expected to extend the operation with effective threshold tuning up to \SI{500}{\mega\rad}, according to the simulation results. %However, given the radiation levels expected in \gls{cms}, such an operation range would require several replacements of the innermost layer of the \acrlong{it} barrel during the physics program. In addition, a
A saturation in the preamplifier feedback requires operation at the limits of the \acrlong{tot} response in order to match the dead time requirement for the innermost layer of the \acrlong{it}. 

The \acrlong{sync} \acrlong{afe} features an automatic threshold tuning performed periodically by the auto-zeroing circuit and offers a very good timing performance. However, it appeared to be the noisiest of the three \acrlong{afe}s. The noise increased for lower thresholds and fast preamplifier return to baseline, becoming critical for the operation settings of Layer~\num{1}.

The \acrlong{lin} \acrlong{afe} satisfied all the requirements, but featured a slower time response.
However, an improved design was developed and is expected from simulation to reach a timing performance almost equivalent to the other two \acrlongpl{afe}. Since all the performance parameters of this \acrlong{afe} satisfy \gls{cms} requirements and the main drawback was addressed and mitigated, the \acrlong{lin} \acrlong{afe} was identified as the lowest-risk option for the future pixel detector. \gls{cms} selected the \acrlong{lin} \acrlong{afe} with an improved design for the integration into the next version of the RD53 pixel chip for \gls{cms} (the C-ROC). A prototype is expected to become available in 2021.


\section*{Acknowledgments}

The tracker groups gratefully acknowledge financial support from the following funding agencies: BMWFW and FWF (Austria); FNRS and FWO (Belgium); CERN; MSE and CSF (Croatia); Academy of Finland, MEC, and HIP (Finland); CEA and CNRS/IN2P3 (France); BMBF, DFG, and HGF (Germany); GSRT (Greece); NKFIA K124850, and Bolyai Fellowship of the Hungarian Academy of Sciences (Hungary); DAE and DST (India); IPM (Iran); INFN (Italy); PAEC (Pakistan); SEIDI, CPAN, PCTI and FEDER (Spain); Swiss Funding Agencies (Switzerland); MST (Taipei); STFC (United Kingdom); DOE and NSF (U.S.A.).

Individuals have received support from HFRI (Greece).