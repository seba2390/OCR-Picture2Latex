
%=================================================================================================
\section{Noise evaluation}
%=================================================================================================
\label{sec:noise}

% Noisy pixel masking:
The evaluation of the noise levels in the RD53A chip was done by sending triggers, without any charge injection, so that each recorded hit was induced by the noise. The average noise occupancy is then defined as the number of noise hits per pixel and per trigger. It was measured for the three \gls{afe}s, using a non-irradiated RD53A chip with a sensor with the highest capacitance, i.e.~rectangular pixels, operated at a temperature of about \SI{-10}{\celsius}.
Single noisy pixels can be disabled to reduce the rate of noise hits. 
However, the fraction of disabled pixels should not significantly increase the detection inefficiency.
A single pixel was considered noisy if its noise occupancy was above \num{e-4} based on the lowest occupancy expected in the \gls{it} detector from simulation (Figure~\ref{fig:occ}). Therefore, as a first step of the noise evaluation, every pixel with more than \num{100} hits in \num{e6} triggers was disabled at a threshold of \num{1200}~e${^{-}}$. 
% Noise occupancy
As a second step, a new set of \num{e6} triggers was sent to each front-end to measure the noise occupancy of the non-masked pixels. In case of very low noise more triggers were sent. Results of noise occupancy measurements are presented in Figure~\ref{fig:noise}. The fraction of masked noisy pixels is indicated in the legend and the maximum noise occupancy of \num{e-6}, required by \gls{cms}, is indicated by the red line. The \gls{tot} was calibrated to \num{1100}~e$^{-}$/TOT\textsubscript{40}.

% Noise occupancy results:
\begin{figure}[t]
    \centering
    \begin{subfigure}{0.49\textwidth}
        \centering
        \includegraphics[width=\textwidth]{Figures/noise_thr.pdf}
        \caption{Influence of the threshold.}
        \label{fig:noise_thr}
    \end{subfigure}
    \hfill
    \begin{subfigure}{0.49\textwidth}
        \centering
        \includegraphics[width=\textwidth]{Figures/noise_pa.pdf}
        \caption{Influence of the \gls{pa} bias current \cite{luigi_vertex}.}
        \label{fig:noise_pabias}
    \end{subfigure}
    \caption{Noise occupancy measurement of the three \gls{afe}s implemented in the RD53A chip as a function of the threshold (a) and the \gls{pa} bias current (b). The \gls{sync} \gls{afe} is shown in blue, the \gls{lin} \gls{afe} in green and the \gls{diff} \gls{afe} in violet. The \gls{cms} requirement for the maximum noise occupancy is indicated in red. The number of masked noisy pixels is given for each \gls{afe}.}
    \label{fig:noise}
\end{figure}

% Noise vs. Threshold:
First, the influence of the threshold on the average noise occupancy was evaluated. The result is shown in Figure~\ref{fig:noise_thr}. The threshold was gradually decreased from \num{1200}~e${^{-}}$, keeping the same noisy pixels disabled. As expected, the average noise occupancy decreases with increasing threshold, regardless of the front-end design. The \gls{diff} front-end shows very good noise performance, with the average noise occupancy several orders of magnitude below the requirement, even for low thresholds. No hits were found in this front-end in \num{e6} triggers at higher thresholds, hence a higher number of triggers was sent to evaluate the average noise occupancy. The other two \gls{afe}s satisfy the noise requirement down to a threshold of \num{1000}~e${^{-}}$, which is consistent with the requirement on the detection threshold. Nonetheless, it can be noticed that the fraction of masked pixels is higher in the \gls{sync} \gls{afe}.

% Noise vs. PA bias:
The influence of the \gls{pa} bias current on the noise was also studied. When this current increases, the transconductance of the input transistor is increased, which results in lower noise with a penalty of an increase in the analogue current consumption. The average noise occupancy was measured for different \gls{pa} bias currents and is presented in Figure~\ref{fig:noise_pabias} as a function of the measured analogue current consumption per pixel. All the other front-end settings that could contribute to the current consumption were kept constant during this measurement. 
As expected the noise in all three \gls{afe}s decreases when more current is provided.
The \gls{diff} \gls{afe} shows again a very good noise performance, with the average noise occupancy well below the requirement, even when operated with low \gls{pa} bias. The \gls{lin} and the \gls{sync} \gls{afe} need \SI{3.5}{\micro\ampere} and \SI{4.5}{\micro\ampere} per pixel, respectively, to reach the required noise level. All three \gls{afe}s can meet the \gls{cms} noise requirement if the \gls{pa} bias current is adjusted, hence this parameter is a handle to reduce the front-end noise at the price of an increase in the power consumption.