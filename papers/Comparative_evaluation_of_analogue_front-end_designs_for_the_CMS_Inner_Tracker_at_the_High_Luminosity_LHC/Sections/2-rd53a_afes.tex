%=================================================================================================
\section{RD53A analogue front-ends}
%=================================================================================================
\label{sec:pixelchip}

% RD53A chip
A high-performance radiation tolerant pixel readout chip is essential for good tracking performance of the \gls{it} operating during the \gls{hllhc} era. Such a readout chip is being designed in \acrshort{tsmc}~\cite{tsmc_web} \SI{65}{\nano\meter} CMOS technology by the RD53 Collaboration~\citep{rd53}, a joint effort between the ATLAS and \gls{cms} experiments.
A large-scale demonstrator chip called RD53A~\citep{rd53a_manual} containing design variations was produced. Its purpose is to demonstrate the suitability of the chosen technology for low threshold, low noise, and low power operation at high hit rates, to verify sufficient radiation tolerance~\citep{rd53a_specs}, and to select the most suitable design for the final readout chip.
It is a mixed signal chip, having both analogue and digital circuits. It features custom-designed \acrlong{ip} blocks, such as \acrlong{cdr} and \acrlong{pll} blocks~\citep{rd53a_cdr} for the clock recovery from the command stream running at \SI{160}{\mega b\per\second}; a high speed output transmitter with a \acrlong{cml} cable driver~\citep{rd53a_transmitter} sending data at \SI{1.28}{\giga b\per\second} on up to four output lanes; and a \acrlong{shldo} regulator~\citep{shldo} for serial powering of the pixel modules.
The chip size is \SI[product-units = power]{20.0 x 11.8}{\milli\meter}, which is about half the size of the final chip, as it shares the chip reticle with \gls{cms} \acrlong{ot} chips. The pixel matrix is composed of \num{400 x 192} square pixels with \SI{50}{\micro\metre} pitch. 
All the common analogue and digital circuitry needed to bias, configure, monitor, and read out the chip is placed at the bottom chip periphery~\citep{rd53a_manual}.

The analogue-to-digital conversion is performed by the \gls{afe}, whose basic structure (shown in Figure~\ref{fig:afesteps}) includes a \gls{csa}, usually referred to as \gls{pa}, a feedback circuit taking care of the signal return to baseline and leakage current compensation, a threshold discriminator, a threshold trimming circuit to address pixel-to-pixel variation of the threshold voltage, and a \gls{tot} counting of the input signal amplitude. In the RD53A chip the \gls{tot}$_{40}$ digitization with 4-bit resolution is done with respect to rising edges of the \SI{40}{\mega\hertz} \gls{lhc} clock\footnote{In the final pixel chip, the counting will be performed on both the rising and the falling edge of the clock, resulting in a finer \gls{tot}$_{80}$ counting at \SI{80}{\mega\hertz} with one \gls{tot}$_{80}$ unit equal to \SI{12.5}{\nano\second} \cite{rd53b_manual}.}. Therefore, one \gls{tot}$_{40}$ unit corresponds to \SI{25}{\nano\second} \cite{rd53a_manual}.
The chip also features a circuit for the generation of internal calibration charge injection signals. The circuit, connected to the input of the \gls{pa}, enables the injection of a well-defined and programmable charge to test the front-end functionalities and calibrate the chip response. Every pixel in the RD53A chip contains the same circuit based on two switches that generate voltage steps fed to an injection capacitor~\citep{rd53a_manual}.

% Schematic of a generic AFE:
\begin{figure}[t]
    \centering
    \includegraphics[width=\textwidth]{Figures/afe.pdf}
    \caption{Signal processing steps in different stages of a generic analogue front-end, from signal collection to digitization.}
    \label{fig:afesteps}
\end{figure}

The RD53A \gls{afe}s are grouped by four, i.e.~$2\times2$ pixels, into analogue "islands", which are embedded in a synthesized digital “sea”, as shown in Figure~\ref{fig:ana_island}.
Three different \gls{afe} designs have been proposed within the RD53 project to explore different options for the ATLAS and \gls{cms} experiments leaving open the possibility that the experiments might make different choices.
%to allow for comparison and to provide a choice of the most suitable option for each of the two experiments. 
The chip is divided horizontally into three sections, each one having one \gls{afe} design, as indicated in Figure~\ref{fig:rd53achip}. The \gls{sync} \gls{afe} is implemented between columns \num{0} and \num{127}, the \gls{lin} \gls{afe} between columns \num{128} and \num{263}, and the \gls{diff} \gls{afe} between columns \num{264} and \num{399}. It was not possible to have an equal area for all three designs because the 400-pixels wide matrix is built of \num{8x8} pixel cores~\citep{rd53a_manual}.
The three \glspl{afe} share the digital logic and the chip periphery in the RD53A chip~\citep{rd53a_manual}.
All three \gls{afe}s are based on a \gls{csa} with a feedback loop ensuring the return to baseline of the \gls{pa} output after each hit. The gain of the \gls{pa} can be chosen globally thanks to different feedback capacitors (C\textsubscript{F}) present in each \gls{afe}. The specific features of each \gls{afe} are discussed in the following paragraphs. 

% Analogue islands and RD53A chip
\begin{figure}[ht]
    \captionsetup{justification=centering}
    \begin{minipage}{0.45\textwidth}
        \centering
        \includegraphics[trim=150 0 100 0, clip, height=3.9cm]{Figures/analog_island.jpg}
        \caption{RD53A layout of four analogue islands, i.e.~sixteen pixels, surrounded by the fully synthesized digital “sea”~\citep{rd53a_manual}.}
        \label{fig:ana_island}
    \end{minipage}
    \hfill
    \begin{minipage}{0.5\textwidth}
        \centering
        \includegraphics[width=\textwidth]{Figures/RD53A.png}
        \caption{Photograph of the RD53A chip, wire-bonded to a test card, indicating the placement of the three \acrlong{afe}s.}
        \label{fig:rd53achip}
    \end{minipage}
\end{figure}

\paragraph{Synchronous front-end.}
The schematic of the Synchronous front-end is shown in Figure~\ref{fig:sync}. It features a single-stage \gls{csa} with a Krummenacher feedback (I\textsubscript{Krum}, V\textsubscript{\texttt{REF\_Krum}})~\citep{krum}, which ensures both the sensor leakage current compensation and the constant current discharge of the feedback capacitor. The Krummenacher current (I\textsubscript{Krum}) drives the speed of the \gls{pa} output return to baseline.
The \gls{pa} is AC-coupled (C\textsubscript{AC}) to a synchronous discriminator composed of a \acrlong{da}, providing a further small gain, and a positive feedback latch, which performs the signal comparison with a threshold (V\textsubscript{th}) and generates the discriminator output. The latter can also be switched to a local oscillator with a selectable frequency higher than the standard \gls{lhc} clock, in order to perform a fast \gls{tot} counting.
The distinctive feature of this \gls{afe} is a so-called "auto-zero" functionality. In traditional designs, the transistor mismatch causing pixel-to-pixel variations of the threshold is compensated with a trimming \gls{dac}. In the \gls{sync} \gls{afe} instead, internal capacitors (C\textsubscript{az}) are used to compensate voltage offsets automatically. A periodic acquisition of a baseline (V\textsubscript{BL}) is required, which can be done during \gls{lhc} abort gaps~\citep{rd53a_manual, sync}. 

\paragraph{Linear front-end.}
The Linear front-end %is the most traditional design of the three. It 
implements a linear pulse amplification in front of the discriminator. %, which compares the pulse to a threshold voltage. 
The schematic of this \gls{afe} is shown in Figure~\ref{fig:lin}. As for the \gls{sync} \gls{afe}, the \gls{pa} of the \gls{lin} \gls{afe} is based on a \gls{csa} featuring a Krummenacher feedback (I\textsubscript{Krum}, V\textsubscript{\texttt{REF\_Krum}}). The signal from the \gls{csa} is fed to a low power threshold discriminator based on current comparison, which compares the signal with the threshold (V\textsubscript{th}). It is composed of a transconductance stage followed by a \gls{tia} providing a low impedance path for fast switching. A 4-bit binary weighted trimming \gls{dac} with adjustable range (I\textsubscript{DAC}) allows for a reduction in the threshold dispersion across the pixel matrix~\citep{rd53a_manual, lin}.

% Schematics
\begin{figure}[p]
    \centering
    \includegraphics[width=\textwidth]{Figures/sync_updated.pdf}
    \caption{Schematic of the Synchronous front-end implemented in the RD53A chip~\citep{rd53a_manual}.}
    \label{fig:sync}
    \vspace{0.5cm} 
    \includegraphics[width=1.1\textwidth]{Figures/lin_fe_modified.pdf}
    \caption{Schematic of the Linear front-end implemented in the RD53A chip~\citep{rd53a_manual}.}
    \label{fig:lin}
    \vspace{0.5cm} 
    \includegraphics[width=1.1\textwidth]{Figures/diff_fe_modified.pdf}
    \caption{Schematic of the Differential front-end implemented in the RD53A chip~\citep{rd53a_manual}.}
    \label{fig:diff}
\end{figure}


\paragraph{Differential front-end.}
The \gls{pa} of the Differential front-end, shown in Figure~\ref{fig:diff}, has a continuous reset (I\textsubscript{ff}), unlike the other two designs, which use the Krummenacher feedback with constant current reset. This continuous feedback is able to prevent the input from saturation for a leakage current of up to \SI{2}{\nano\ampere}~\cite{rd53b_manual}. For higher currents, a dedicated \gls{lcc} circuit can be enabled. The \gls{lcc} is disconnected from the input when disabled, which improves the \gls{afe} stability and noise performance. The DC-coupled precomparator provides additional gain in front of the comparator and acts as a differential threshold circuit, i.e.~the global threshold is adjustable through two distributed threshold voltages (V\textsubscript{th1} and V\textsubscript{th2}) instead of one. The precomparator stage is followed by a classic time-continuous  comparator. The threshold is trimmed in each pixel using a local 5-bit trimming \gls{dac} (TDAC)~\citep{rd53a_manual}.


% PURPOSE OF THE PAPER:

{The basic functionalities of the RD53A chip and each of the three \gls{afe}s were previously verified and reported~\citep{timon_hiroshima, luigi_vertex, aleksandra_vci, ennio_twepp, luigi_nima}. The objective of this work was to evaluate the three \gls{afe} designs against the \gls{cms} requirements, in terms of spurious hit rate, dead time, and radiation tolerance and to compare their performance.
A dedicated evaluation program was established and the most relevant detector performance parameters were studied. The key measurements that enabled \gls{cms} to identify the most suitable option for integration into the \gls{cms} pixel detector are presented in this paper. All presented test results were obtained with the BDAQ53 test system~\citep{bdaq53}, using the calibration injection circuit, with RD53A chips bump-bonded to sensors with rectangular pixels, i.e.~\SI[product-units = power]{100x25}{\micro\meter}, if not otherwise stated, and operated at cold temperature (T $\approx$ \SI{-10}{\celsius}), which is the lowest temperature that could be achieved with the cooling systems available for the lab setups.}