\section{Related work}
\label{sec:epidemics}

Multiple approaches have been proposed to overcome the overhead (number of messages, 
number of rounds and number of transmitted bits) of epidemic dissemination algorithms, especially the two lower bounds of Karp et al.~\cite{DBLP:conf/focs/KarpSSV00}. By allowing direct addressing in the random phone call model, Avin and Elsässer~\cite{avin2013faster} presented an algorithm requiring $\mathcal{O}(\sqrt{\ln n})$ rounds by building
a virtual topology between processes, at the cost of transmitting a larger number of more complex
messages. Haeupler and Malkhi~\cite{haeupler2014optimal} generalized the work with a gossip algorithm running
in $\mathcal{O}(\ln \ln n)$ rounds and sending $\mathcal{O}(1)$ messages per node with $\mathcal{O}(\ln n)$
bits per message, all of which are optimal. The main insight of their algorithm is the careful construction and
manipulation of clusters. Panagiotou et al.~\cite{panagiotou2013faster} removed the uniform
assumption of the random phone call model and presented a push-pull protocol using $\Theta(\ln \ln n)$
rounds. The number of calls per process is fixed for each process, but follows a power law distribution with
exponent $\beta \in (2, 3)$. This distribution has infinite variance and causes uneven load balancing, with
some processes that must call $\mathcal{O}(n)$ processes at every round. Doerr and
Fouz~\cite{doerr2011asymptotically} presented a push-only protocol spreading a rumor in $(1 + o(1))\log_2 n$
rounds and using $\mathcal{O}(nf(n))$ messages for an arbitrary function $f \in \omega(1)$. It assumes that each process possesses a permutation of all the processes. Doerr et al. \cite{doerr2016simple} disseminate information by randomizing the whispering protocols of \cite{Gasieniec96adaptivebroadcasting,Diks2000}.  Alistarh et al.~\cite{DBLP:conf/icalp/AlistarhGGZ10} designed a gossip protocol with a $\mathcal{O}(n)$ message complexity by randomly selecting a set of coordinators that collect and disseminate the rumors using overlay structures. Their algorithm is robust against oblivious failures. Processes are allowed to keep a communication line open over multiple rounds and can call $\mathcal{O}(n)$ processes per round.

Work on epidemic dissemination was done in other contexts and with different constraints, such as topologies other than the complete graph~\cite{fountoulakis2010rumor,DBLP:conf/stacs/Giakkoupis11}, communication with latency~\cite{DBLP:conf/podc/GilbertRS17} and asynchronicity~\cite{acan2015push}. 
