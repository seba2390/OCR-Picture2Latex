\documentclass[format=acmsmall,review=false,screen=true,authorversion=true]{acmart}
\pdfoutput=1

\settopmatter{printccs=false,printacmref=false,printfolios=true}
\setcopyright{none}
\acmDOI{}

% \acmJournal{JACM}

\usepackage{multirow}
\usepackage[utf8]{inputenc}
\usepackage[T1]{fontenc}
\usepackage[australian,american]{babel}
\usepackage{graphicx}
\usepackage{thmtools, mathdots}
\usepackage[framemethod=tikz]{mdframed}
\usepackage{pgfplots}
\pgfplotsset{compat=1.12}
\usepackage{array}
\usepackage[binary-units]{siunitx}
\usepackage{booktabs}
\usepackage{multirow}
\usepackage{pifont}
\usepackage{tikz}
\usepackage{csquotes}
%\usepackage{natbib}
\setcitestyle{square,aysep={},yysep={;}}
%\usepackage{url}
\usepackage{color}
\usepackage{todonotes}
\usepackage{lipsum}
\usepackage{subfigure}
\usepackage{pgfplotstable}
%\usepackage{amsmath, amssymb, amsthm}

\newcommand{\N}{\mathbb{N}}
\newcommand{\Z}{\mathbb{Z}}
\newcommand{\Q}{\mathbb{Q}} 
\newcommand{\R}{\mathbb{R}}
\renewcommand{\C}{\mathbb{C}}
\newcommand{\E}{\mathbb{E}}
\renewcommand{\P}{\mathbb{P}}
\renewcommand{\epsilon}{\varepsilon} 
\renewcommand{\phi}{\varphi}
\renewcommand{\emph}{\textbf}

\newcommand{\fanin}{f_\text{in}}
\newcommand{\fanout}{f_\text{out}}
\newcommand{\view}{viewSize}
\newcommand{\rpushonly}{r}
\newcommand{\rpull}{r_\text{pull}}
\newcommand{\rpush}{r_\text{push}}
\newcommand{\uninformed}[1]{u_{#1}}
\newcommand{\informed}[1]{i_{#1}}
\newcommand{\new}[1]{X_{#1}}
\newcommand{\Ppull}[2]{\P_{\text{pull}} \left( #1 \mid #2 \right)}
\newcommand{\Ppush}[2]{\P_{\text{push}} \left( #1 \mid #2 \right)}
\newcommand{\pscale}{p_\text{scale}}

% Package to generate and customize Algorithm as per ACM style
\usepackage[ruled]{algorithm2e}
% \SetAlFnt{\algofont}
% \SetAlCapFnt{\algofont}
% \SetAlCapNameFnt{\algofont}
% \SetAlCapHSkip{0pt}
% \IncMargin{-\parindent}
% \renewcommand{\algorithmcfname}{ALGORITHM}

% Add only one number to align* environment
\newcommand\numberthis{\addtocounter{equation}{1}\tag{\theequation}}

% Command for displaying stirling numbers of the second kind
\DeclareRobustCommand{\stir}{\genfrac\{\}{0pt}{}}

%% Environment for comments: Set the boolean to false to produce a comment-free version.
\newboolean{showcomments}
\setboolean{showcomments}{false}
\ifthenelse{\boolean{showcomments}}
{ \newcommand{\mynote}[3]{
    \fbox{\bfseries\sffamily\scriptsize#1}
    {\small$\blacktriangleright$\textsf{\textit{\color{#3}{#2}}}$\blacktriangleleft$}}}
{ \newcommand{\mynote}[3]{}}
% One command per author:
\newcommand{\hm}[1]{\mynote{Hugues}{#1}{orange}}
\newcommand{\lh}[1]{\mynote{Laurent}{#1}{ForestGreen}}
\newcommand{\mm}[1]{\mynote{Miguel}{#1}{blue}}
\newcommand{\jt}[1]{\mynote{Jocelyn}{#1}{SaddleBrown}}

% %% algorithm2e
\DontPrintSemicolon

\SetFuncSty{sc}
\SetDataSty{em}
\SetCommentSty{em}

\SetKwFor{func}{procedure}{}{end}
\SetKwFor{upon}{upon}{}{end}
\SetKwFor{struct}{struct}{}{end}
\SetKwFor{periodically}{task every $\delta$ time units}{}{end}
\SetKwFor{initially}{initially}{}{end}

\SetKw{Break}{break}
\SetKw{return}{return}

\SetKwFunction{BALL}{BALL}
\SetKwFunction{PULL}{PULL}
\SetKwFunction{broadcast}{Broadcast}
\SetKwFunction{receive}{receive}
\SetKwFunction{deliver}{deliver}
\SetKwFunction{Random}{Random}
\SetKwFunction{send}{send}
\SetKwFunction{dest}{to}
\SetKwFunction{orderEvents}{orderEvents}
\SetKwFunction{receiveEvents}{receiveEvents}
\SetKwFunction{isDeliverable}{isDeliverable}
\SetKwFunction{isPush}{\underline{isPush}}
\SetKwFunction{isPull}{\underline{isPull}}
\SetKwFunction{getClock}{getClock}
\SetKwFunction{updateClock}{updateClock}

\SetKwBlock{Block}{}{}
% %% algorithm2e
%%%%%%%%%%%%%%%%%%%%%%%%%%%%%%%%%%%%%%%%%%%%%%%%%%%%%%%%%%%%%%%%%%%%%%



% For syncronisation with skim
\synctex = 1

%\setcopyright{none}

% Page heads
%\markboth{L. Hayez, M. Matos, H. Mercier}{Title here}

% Title portion

\title{Optimal epidemic dissemination}

\author{Hugues Mercier}
\email{hugues.mercier@unine.ch}
\author{Laurent Hayez}
\email{laurent.hayez@unine.ch}
\affiliation{%
  \institution{Université de Neuchâtel}
  \city{Neuchâtel}
  \country{Switzerland}
}
\author{Miguel Matos}
\affiliation{%
  \institution{INESC-ID \& IST, Universidade de Lisboa}
  \city{Lisboa}
  \country{Portugal}
}

\begin{abstract}
{
  In this paper, we explore the connection between secret key agreement and secure omniscience within the setting of the multiterminal source model with a wiretapper who has side information. While the secret key agreement problem considers the generation of a maximum-rate secret key through public discussion, the secure omniscience problem is concerned with communication protocols for omniscience that minimize the rate of information leakage to the wiretapper. The starting point of our work is a lower bound on the minimum leakage rate for omniscience, $\rl$, in terms of the wiretap secret key capacity, $\wskc$. Our interest is in identifying broad classes of sources for which this lower bound is met with equality, in which case we say that there is a duality between secure omniscience and secret key agreement. We show that this duality holds in the case of certain finite linear source (FLS) models, such as two-terminal FLS models and pairwise independent network models on trees with a linear wiretapper. Duality also holds for any FLS model in which $\wskc$ is achieved by a perfect linear secret key agreement scheme. We conjecture that the duality in fact holds unconditionally for any FLS model. On the negative side, we give an example of a (non-FLS) source model for which duality does not hold if we limit ourselves to communication-for-omniscience protocols with at most two (interactive) communications.  We also address the secure function computation problem and explore the connection between the minimum leakage rate for computing a function and the wiretap secret key capacity.
  
%   Finally, we demonstrate the usefulness of our lower bound on $\rl$ by using it to derive equivalent conditions for the positivity of $\wskc$ in the multiterminal model. This extends a recent result of Gohari, G\"{u}nl\"{u} and Kramer (2020) obtained for the two-user setting.
  
   
%   In this paper, we study the problem of secret key generation through an omniscience achieving communication that minimizes the 
%   leakage rate to a wiretapper who has side information in the setting of multiterminal source model.  We explore this problem by deriving a lower bound on the wiretap secret key capacity $\wskc$ in terms of the minimum leakage rate for omniscience, $\rl$. 
%   %The former quantity is defined to be the maximum secret key rate achievable, and the latter one is defined as the minimum possible leakage rate about the source through an omniscience scheme to a wiretapper. 
%   The main focus of our work is the characterization of the sources for which the lower bound holds with equality \textemdash it is referred to as a duality between secure omniscience and wiretap secret key agreement. For general source models, we show that duality need not hold if we limit to the communication protocols with at most two (interactive) communications. In the case when there is no restriction on the number of communications, whether the duality holds or not is still unknown. However, we resolve this question affirmatively for two-user finite linear sources (FLS) and pairwise independent networks (PIN) defined on trees, a subclass of FLS. Moreover, for these sources, we give a single-letter expression for $\wskc$. Furthermore, in the direction of proving the conjecture that duality holds for all FLS, we show that if $\wskc$ is achieved by a \emph{perfect} secret key agreement scheme for FLS then the duality must hold. All these results mount up the evidence in favor of the conjecture on FLS. Moreover, we demonstrate the usefulness of our lower bound on $\wskc$ in terms of $\rl$ by deriving some equivalent conditions on the positivity of secret key capacity for multiterminal source model. Our result indeed extends the work of Gohari, G\"{u}nl\"{u} and Kramer in two-user case.
}
\end{abstract}

\thanks{A brief announcement of this work was presented at PODC 2017.}

\begin{document}

\maketitle

% \leavevmode
% \\
% \\
% \\
% \\
% \\
\section{Introduction}
\label{introduction}

AutoML is the process by which machine learning models are built automatically for a new dataset. Given a dataset, AutoML systems perform a search over valid data transformations and learners, along with hyper-parameter optimization for each learner~\cite{VolcanoML}. Choosing the transformations and learners over which to search is our focus.
A significant number of systems mine from prior runs of pipelines over a set of datasets to choose transformers and learners that are effective with different types of datasets (e.g. \cite{NEURIPS2018_b59a51a3}, \cite{10.14778/3415478.3415542}, \cite{autosklearn}). Thus, they build a database by actually running different pipelines with a diverse set of datasets to estimate the accuracy of potential pipelines. Hence, they can be used to effectively reduce the search space. A new dataset, based on a set of features (meta-features) is then matched to this database to find the most plausible candidates for both learner selection and hyper-parameter tuning. This process of choosing starting points in the search space is called meta-learning for the cold start problem.  

Other meta-learning approaches include mining existing data science code and their associated datasets to learn from human expertise. The AL~\cite{al} system mined existing Kaggle notebooks using dynamic analysis, i.e., actually running the scripts, and showed that such a system has promise.  However, this meta-learning approach does not scale because it is onerous to execute a large number of pipeline scripts on datasets, preprocessing datasets is never trivial, and older scripts cease to run at all as software evolves. It is not surprising that AL therefore performed dynamic analysis on just nine datasets.

Our system, {\sysname}, provides a scalable meta-learning approach to leverage human expertise, using static analysis to mine pipelines from large repositories of scripts. Static analysis has the advantage of scaling to thousands or millions of scripts \cite{graph4code} easily, but lacks the performance data gathered by dynamic analysis. The {\sysname} meta-learning approach guides the learning process by a scalable dataset similarity search, based on dataset embeddings, to find the most similar datasets and the semantics of ML pipelines applied on them.  Many existing systems, such as Auto-Sklearn \cite{autosklearn} and AL \cite{al}, compute a set of meta-features for each dataset. We developed a deep neural network model to generate embeddings at the granularity of a dataset, e.g., a table or CSV file, to capture similarity at the level of an entire dataset rather than relying on a set of meta-features.
 
Because we use static analysis to capture the semantics of the meta-learning process, we have no mechanism to choose the \textbf{best} pipeline from many seen pipelines, unlike the dynamic execution case where one can rely on runtime to choose the best performing pipeline.  Observing that pipelines are basically workflow graphs, we use graph generator neural models to succinctly capture the statically-observed pipelines for a single dataset. In {\sysname}, we formulate learner selection as a graph generation problem to predict optimized pipelines based on pipelines seen in actual notebooks.

%. This formulation enables {\sysname} for effective pruning of the AutoML search space to predict optimized pipelines based on pipelines seen in actual notebooks.}
%We note that increasingly, state-of-the-art performance in AutoML systems is being generated by more complex pipelines such as Directed Acyclic Graphs (DAGs) \cite{piper} rather than the linear pipelines used in earlier systems.  
 
{\sysname} does learner and transformation selection, and hence is a component of an AutoML systems. To evaluate this component, we integrated it into two existing AutoML systems, FLAML \cite{flaml} and Auto-Sklearn \cite{autosklearn}.  
% We evaluate each system with and without {\sysname}.  
We chose FLAML because it does not yet have any meta-learning component for the cold start problem and instead allows user selection of learners and transformers. The authors of FLAML explicitly pointed to the fact that FLAML might benefit from a meta-learning component and pointed to it as a possibility for future work. For FLAML, if mining historical pipelines provides an advantage, we should improve its performance. We also picked Auto-Sklearn as it does have a learner selection component based on meta-features, as described earlier~\cite{autosklearn2}. For Auto-Sklearn, we should at least match performance if our static mining of pipelines can match their extensive database. For context, we also compared {\sysname} with the recent VolcanoML~\cite{VolcanoML}, which provides an efficient decomposition and execution strategy for the AutoML search space. In contrast, {\sysname} prunes the search space using our meta-learning model to perform hyperparameter optimization only for the most promising candidates. 

The contributions of this paper are the following:
\begin{itemize}
    \item Section ~\ref{sec:mining} defines a scalable meta-learning approach based on representation learning of mined ML pipeline semantics and datasets for over 100 datasets and ~11K Python scripts.  
    \newline
    \item Sections~\ref{sec:kgpipGen} formulates AutoML pipeline generation as a graph generation problem. {\sysname} predicts efficiently an optimized ML pipeline for an unseen dataset based on our meta-learning model.  To the best of our knowledge, {\sysname} is the first approach to formulate  AutoML pipeline generation in such a way.
    \newline
    \item Section~\ref{sec:eval} presents a comprehensive evaluation using a large collection of 121 datasets from major AutoML benchmarks and Kaggle. Our experimental results show that {\sysname} outperforms all existing AutoML systems and achieves state-of-the-art results on the majority of these datasets. {\sysname} significantly improves the performance of both FLAML and Auto-Sklearn in classification and regression tasks. We also outperformed AL in 75 out of 77 datasets and VolcanoML in 75  out of 121 datasets, including 44 datasets used only by VolcanoML~\cite{VolcanoML}.  On average, {\sysname} achieves scores that are statistically better than the means of all other systems. 
\end{itemize}


%This approach does not need to apply cleaning or transformation methods to handle different variances among datasets. Moreover, we do not need to deal with complex analysis, such as dynamic code analysis. Thus, our approach proved to be scalable, as discussed in Sections~\ref{sec:mining}.
\section{Related Work}
\label{sec:related_work}
We now provide a brief overview of related work in the areas of language grounding and transfer for reinforcement learning.
%There has been work on learning to make optimal local decisions for structured prediction problems~\cite{daume2006searn}.
%
%\newcite{branavan2010reading} looked at a similar task of building a partial model of the environment while following instructions. The differences with our work are (1) the text in their case is instructions, while we only have text describing the environment, and (2) their environment is deterministic, hence the transition function can be learned more easily. 
%
%TODO - model-based RL, value iteration, predictron.


\subsection{Grounding Language in Interactive Environments}
In recent years, there has been increasing interest in systems that can utilize textual knowledge to learn control policies. Such applications include interpreting help documentation~\fullcite{branavan2010reading}, instruction following~\fullcite{vogel2010learning,kollar2010toward,artzi2013weakly,matuszek2013learning,Andreas15Instructions} and learning to play computer games~\fullcite{branavan2011nonlinear,branavan2012learning,narasimhan2015language,he2016deep}. In all these applications, the models are trained and tested on the same domain.

Our work represents two departures from prior work on grounding. First, rather than optimizing control performance for a single domain,
we are interested in the multi-domain transfer scenario, where language 
descriptions drive generalization. Second, prior work used text in the form of strategy advice to directly learn the policy. Since the policies are typically optimized for a specific task, they may be harder to transfer across domains. Instead, we utilize text to bootstrap the induction of the environment dynamics, moving beyond task-specific strategies. 

%Previous work has explored the use of text manuals in game playing, %ranging from constructing useful features by mining patterns in %text~\cite{eisenstein2009reading}, learning a semantic interpreter %with access to limited gameplay examples~\cite{goldwasser2014learning} %to learning through reinforcement from in-game %rewards~\cite{branavan2011learning}. These efforts have demonstrated %the usefulness of exploiting domain knowledge encoded in text to learn %effective policies. However, these methods use the text to infer %directly the best strategy to perform a task. In contrast, our goal is %to learn mappings from the text to the dynamics of an environment and %separate out the learning of the strategy/motives. 

Another related line of work consists of systems that learn spatial and topographical maps of the environment for robot navigation using natural language descriptions~\fullcite{walter2013learning,hemachandra2014learning}. These approaches use text mainly containing appearance and positional information, and integrate it with other semantic sources (such as appearance models) to obtain more accurate maps. In contrast, our work uses language describing the dynamics of the environment, such as entity movements and interactions, which 
is complementary to static positional information received through state observations. Further, our goal is to help an agent learn policies that generalize over different stochastic domains, while their works consider a single domain.

%karthik: I don't see the direct relevance
%Another line of work explores using textual interactive %environments~\cite{narasimhan2015language,he2016deep} to ground %language understanding into actions taken by the system in the %environment. In these tasks, understanding of language is crucial, %without which a system would not be able to take reasonable actions. %Our motivation is different -- we take an existing set of tasks and %domains which are amenable to learning through reinforcement, and %demonstrate how to utilize textual knowledge to learn faster and more %optimal policies in both multitask and transfer setups.

\subsection{Transfer in Reinforcement Learning}
Transferring policies across domains is a challenging problem in reinforcement learning~\fullcite{konidaris2006framework,taylor2009transfer}. The main hurdle lies in learning a good mapping between the state and action spaces of different domains to enable effective transfer. Most previous approaches have either explored skill transfer~\fullcite{konidaris2007building,konidaris2012transfer} or value function/policy transfer~\fullcite{liu2006value,taylor2007transfer,taylor2007cross}. There have also been attempts at model-based transfer for RL~\fullcite{taylor2008transferring,nguyen2012transferring,gavsic2013pomdp,wang2015learning,joshi2018cross} but these methods either rely on hand-coded inter-task mappings for state and actions spaces or require significant interactions in the target task to learn an effective mapping. Our approach doesn't use any explicit mappings and can learn to predict the dynamics of a target task using its descriptions.

% Work by \newcite{konidaris2006autonomous} look at knowledge transfer by learning a mapping from sensory signals to reward functions.

A closely related line of work concerns transfer methods for deep reinforcement learning. \citeA{parisotto2016actor}  train a deep network to mimic pre-trained experts on source tasks using policy distillation. The learned parameters are then used to initialize a network on a target task to perform transfer. Rusu et al.~\citeyear{rusu2016progressive} facilitate transfer by freezing parameters learned on source tasks and adding a new set of parameters for every new target task, while using both sets to learn the new policy. Work by Rajendran et al.~\citeyear{rajendran20172t} uses attention networks to selectively transfer from a set of expert policies to a new task. \textcolor{black}{Barreto et al.~\citeyear{barreto2017successor} use features based on successor representations~\fullcite{dayan1993improving} for transfer across tasks in the same domain. Kansky~et~al.~\citeyear{kansky2017schema} learn a generative model of causal physics in order to help zero-shot transfer learning.} Our approach is orthogonal to all these directions since we use text to bootstrap transfer, and can potentially be combined with these methods to achieve more effective transfer. 

\textcolor{black}{There has also been prior work on zero-shot policy generalization for tasks with input goal specifications. \fullciteA{schaul2015universal} learn a universal value function approximator that can generalize across both states and goals. \fullcite{andreas2016modular} use policy sketches, which are annotated sequences of subgoals, in order to learn a hierarchical policy that can generalize to new goals. \fullciteA{oh2017zero} investigate zero-shot transfer for instruction following tasks, aiming to generalize to unseen instructions in the same domain. The main departure of our work compared to these is in the use of environment descriptions for generalization across domains rather than generalizing across text instructions.}

Perhaps closest in spirit to our hypothesis is the recent work by~\fullcite{harrison2017guiding}. Their approach makes use of paired instances of text descriptions and state-action information from human gameplay to learn a machine translation model. This model is incorporated into a policy shaping algorithm to better guide agent exploration. Although the motivation of language-guided control policies is similar to ours, their work considers transfer across tasks in a single domain, and requires human demonstrations to learn a policy.

\textcolor{black}{
\subsection{Using Task Features for Transfer}
The idea of using task features/dictionaries for zero-shot generalization has been explored previously in the context of image classification. \fullciteA{kodirov2015unsupervised} learn a joint feature embedding space between domains and also induce the corresponding projections onto this space from different class labels. 
\fullciteA{kolouri2018joint} learn a joint dictionary across visual features and class attributes using sparse coding techniques. \fullciteA{romera2015embarrassingly} model the relationship between input features, task attributes and classes as a linear model to achieve efficient yet simple zero-shot transfer for classification. \fullciteA{socher2013zero} learn a joint semantic representation space for images and associated words to perform zero-shot transfer.}

\textcolor{black}{
Task descriptors have also been explored in zero-shot generalization for control policies. \fullciteA{sinapov2015learning} use task meta-data as features to learn a mapping between pairs of tasks. This mapping is then used to select the most relevant source task to transfer a policy from. \fullciteA{isele2016using} build on the ELLA framework~\fullcite{ruvolo2013ella,ammar2014online}, and their key idea is to maintain two shared linear bases across tasks -- one for the policy ($L$) and the other for task descriptors ($D$). Once these bases are learned on a set of source tasks, they can be used to predict policy parameters for a new task given its corresponding descriptor. 
% The training scheme is similar to Actor-mimic scheme~\cite{parisotto2016actor} -- for each task, an expert policy is trained separately and then distilled into policy parameters dependent on the shared basis $L$. 
In these lines of work, the task features were either manually engineered or directly taken from the underlying system parameters defining the dynamics of the environment. In contrast, our framework only requires access to crowd-sourced textual descriptions, alleviating the need for expert domain knowledge.}





% A major difference in our work is that we utilize natural language descriptions of different environments to bootstrap transfer, requiring less exploration in the new task.

% using a policy distillation~\cite{parisotto2016actor,rusu2016progressive,yin2017knowledge} or selective attention over expert networks learnt in the source tasks~\cite{rajendran20172t}. Though these approaches provide some benefits, they still suffer from the requirement of efficiently exploring the new environment to learn how to transfer their existing policies. In contrast, we utilize natural language descriptions of different environments to bootstrap transfer, leading to more focused exploration in the target task. 


% Describe amn in detail





% ----------------------------------------------------------------------------------------------

\section{The regular pull algorithm is asymptotically optimal}
\label{sec:pull}

In this section, we focus on pull-only algorithms. 
Our first observation is that on expectation, pulling is always at least as good as pushing, although the higher variance of pull at the early stage of the dissemination makes pulling less efficient when the rumor is new. For instance, starting with one informed process and $\fanin=\fanout=1$, it takes $\Theta(\ln n)$ pull rounds to inform a second process with high probability, whereas a single push round suffices. The behavior reverses when the rumor is old: if $n-1$ processes are already informed, a single pull round informs the last process with high probability but $\Theta(\ln n)$ push rounds are needed. Despite these differences, our second observation is that pulling and pushing have the same asymptotic round complexity. Our third observation is that the regular pull algorithm is asymptotically optimal, thus pushing is not required. 
Our fourth observation is that the regular pull algorithm asymptotically requires the same round, bit, and message complexity even in the presence of a large number of adversarial and stochastic failures.

Note that in the generalized random phone call model, processes push and pull requests uniformly at random but independently (i.e., with replacement), thus they can push the rumor to themselves, call themselves, and have multiple push messages and/or pull requests colliding in the same round. Of course in practice, in a given round, a process will not send multiple pull requests or multiple push messages to the same process, nor will it call itself. Instead, it will select a uniform random sample among the other processes in the network. Our reason for this definition is twofold. First, choosing interlocutors independently and uniformly at random is more amenable to mathematical analysis, especially upper bounds. Second, we prove that choosing $f$ processes uniformly at random with replacement, or choosing a uniform random sample of size $f$ without replacement among the other $n-1$ processes, are asymptotically equivalent when $f \in \mathcal{O}(n)$. We prove this by matching lower bounds obtained from random samples with upper bounds obtained with interlocutors selected independently and uniformly at random.
\begin{definition}
  Let $0 \leq \uninformed{r} \leq n$ be the number of uninformed processes at round $r$,  $\E_{\text{pull}}[\uninformed{r}]$ the expected number of uninformed processes at round $r$ with the regular pull algorithm, and $\E_{\text{push}}[\uninformed{r}]$ the expected number of uninformed processes at round $r$ with the regular push algorithm. For the number of informed processes at round $r$, we similarly define $\informed{r}$, $\E_{\text{pull}}[\informed{r}]$ and $\E_{\text{push}}[\informed{r}]$. It is clear that $n=\uninformed{r}+\informed{r}=\E_{\text{pull}}[\uninformed{r}]+\E_{\text{pull}}[\informed{r}]=\E_{\text{push}}[\uninformed{r}]+\E_{\text{push}}[\informed{r}]$.
\end{definition}
If processes send pull requests independently and uniformly at random, $\P(\uninformed{r+1} \mid \uninformed{r})$ follows a binomial distribution with mean
\begin{align}
  \label{eq:meanpull1}
  \E_\text{pull}[\uninformed{r+1} \mid \uninformed{r}]&=\uninformed{r}\cdot\left(\frac{\uninformed{r}}{n}\right)^{\fanin}
\end{align}
whereas if they select a uniform random sample without replacement among the other $(n-1)$ processes we obtain
\begin{align}
  \label{eq:meanpull2}
  \E_\text{pull}[\uninformed{r+1} \mid \uninformed{r}]&=\uninformed{r}\cdot\frac{{\uninformed{r}\choose\fanin}}{{n-1 \choose \fanin}}=n-\uninformed{r}\frac{\uninformed{r}(\uninformed{r}-1)\dots(\uninformed{r}-\fanin+1)}{n(n-1)\dots(n-\fanin+1)} = n-\uninformed{r} \frac{(\uninformed{r})_{\fanin}}{(n-1)_{\fanin}} 
\end{align}
where $(\boldsymbol{\cdot})_{\boldsymbol{\cdot}} $ is the falling factorial notation.


\begin{lemma}
  \label{lem:pullbetterthanpush}
If $\fanout = \fanin$, then 
    $\E_{\text{pull}}[\uninformed{r+1}|\uninformed{r}] \leq \E_{\text{push}}[\uninformed{r+1}|\uninformed{r}]$.
\end{lemma}

\begin{proof}
We prove the lemma with processes chosen independently and uniformly at random.  Let  $f=\fanin=\fanout$. For the pull version, we saw that
\begin{align}
  \label{eq:meanpull1repeat}
  \E_\text{pull}[\uninformed{r+1} \mid \uninformed{r}]=\uninformed{r}\cdot\left(\frac{\uninformed{r}}{n}\right)^{f}
\end{align}
whereas for the push version we can show that 
  \begin{equation}
    \label{eq:meanpush}
   \E_{\text{push}}[\uninformed{r+1} \mid \uninformed{r}] = \uninformed{r}\left(1 - \frac{1}{n}\right)^{f(n-\uninformed{r})}.
 \end{equation}
 From Eq.~\eqref{eq:meanpull1repeat} and \eqref{eq:meanpush}, it is clear that the lemma holds when $\uninformed{r}=0$, $\uninformed{r}=n-1$, and $\uninformed{r}=n$. For the other values of $\uninformed{r}$, we prove that
 \begin{align}
\label{eq:comp}
      & \left( \frac{\uninformed{r}}{n}\right)^{f} \leq \left( \left(1 - \frac{1}{n}\right)^{n - \uninformed{r}}\right)^{f} 
      \Leftrightarrow \left(\frac{n-1}{n}\right)^{n - \uninformed{r}} - \frac{\uninformed{r}}{n} \geq 0.
 \end{align}
  Let $g(x) \triangleq \left(\frac{n-1}{n}\right)^{n - x} - \frac{x}{n}$. Since $g(0) \geq 0$ and $g(n-1) = 0$, we prove that $g(x) \geq 0$ for every $x \in \{0,1,\dots, n-1\}$ by showing that $g'(x) \leq 0$ over the interval $[0,n-1]$. We have
  \begin{equation}
     \begin{split}
      g'(x) 
        &= - \left( \frac{n-1}{n} \right)^{n-x} \ln \left(\frac{n-1}{n}\right) - \frac{1}{n} \\
        &= \left(\frac{n}{n-1}\right)^x \left( \frac{n-1}{n} \right)^{n} \ln \left(\frac{n}{n-1}\right) - \frac{1}{n} \\
      \end{split}
    \end{equation}
    which is an increasing function with respect to $x$. To complete the proof, we verify that $g'(n-1) \leq 0$:
     \begin{equation}
    \begin{split}
      g'(n-1) & =  \left(\frac{n}{n-1}\right)^{(n-1)} \left( \frac{n-1}{n} \right)^{n} \ln \left(\frac{n}{n-1}\right) - \frac{1}{n} \\
      & \leq \frac{n-1}{n} \left(\frac{n}{n-1} -1\right)  - \frac{1}{n} \\ & = 0.
    \end{split}
  \end{equation}
\end{proof}


 We now bound the expected progression of the regular pull algorithm, and later use it to derive lower bounds on its round complexity.
 
 \begin{lemma}
   \label{lem:pullasymptotic}
$\E_\text{pull}[\informed{r+1}\mid\informed{r}] \leq \informed{r} \cdot(\fanin+1)$.
 \end{lemma}

\begin{proof}
  We prove the lemma with processes chosen from a uniform random sample using Eq.~\eqref{eq:meanpull2}. We fix $n$ and $\uninformed{r}$ and prove the lemma by induction on $\fanin$.

  \paragraph{Basis step.} The lemma is clearly true for $x=\fanin=0$.
  \paragraph{Inductive step.} Let $0 \leq x \leq n-2$ be an integer. We assume that $n-\uninformed{r} \frac{(\uninformed{r})_x}{(n-1)_x} \leq \informed{r}(x+1)$, which is equivalent to
  \begin{align}
    \label{eq:ind1}
     \uninformed{r} \frac{(\uninformed{r})_x}{(n-1)_x} \geq n - \informed{r}(x+1)
  \end{align}
  and must show that
  \begin{align}
    \label{eq:ind2}
    n-\uninformed{r} \frac{(\uninformed{r})_x(\uninformed{r}-x)}{(n-1)_x(n-1-x)} \leq \informed{r}(x+2) \Leftrightarrow  S \triangleq n-\uninformed{r} \frac{(\uninformed{r})_x(\uninformed{r}-x)}{(n-1)_x(n-1-x)} - \informed{r}(x+2) \leq 0.
     \end{align}
Substituting the left side of Eq.~(\ref{eq:ind1}) for its right side in Eq.~(\ref{eq:ind2}), and replacing $\uninformed{r}$ by $n - \informed{r}$, we have

\begin{equation}
  \begin{split}
  S & \leq n- (n - \informed{r}(x+1)) \frac{n-\informed{r}-x}{n-1-x} - \informed{r}(x+2) \\
    & \leq \frac{n(\informed{r}-1) - \informed{r} (x+1) (\informed{r}  - 1)}{n-x-1} - \informed{r} \\ & \leq \informed{r}-1 - \frac{(x+1) (\informed{r}  - 1)^2}{n-x-1} - \informed{r} \\
    & \leq - \frac{(x+1) (\informed{r}  - 1)^2}{n-x-1} \\
  & \leq 0.
  \end{split}
\end{equation}
  
 \end{proof}


 
\begin{lemma}
    \label{lem:loglog}
        If $\fanin\in\mathcal{O}(\ln n)$, the regular pull algorithm starting with $\frac{n}{\ln n}$ informed processes informs all processes with high probability in $\Theta(\log_{\fanin+1} \ln n)$ rounds.
  \end{lemma}

  \begin{proof}
For the lower bound, it is clear from Lemma~\ref{lem:pullasymptotic} that $\Omega(\log_{\fanin+1} \ln n)$ are required to reach all processes on expectation, thus required to inform all processes with high probability. For the upper bound, the proof for $\fanin=1$ consists of the points 3 and 4 in the proof of Theorem 2.1 of Karp et
    al.~\cite{DBLP:conf/focs/KarpSSV00}. We generalize their proof for an arbitrary $\fanin$.

    Recall that $\E_{\text{pull}}[\uninformed{t}\mid\uninformed{t-1}] =  \frac{(\uninformed{t-1})^{\fanin+1}}{n^{\fanin}}$ and that we start with at most $u_0 = n - \frac{n}{\ln n}$ uninformed processes. We use the following Chernoff bound from \cite{mitzenmacher2005probability}: 
    \[ \P(X \geq (1 + \delta) \mu) \leq e^{-\frac{\delta^2 \mu}{3}},\ 0 < \delta < 1. \]
If $u_{t-1} \geq (\ln n)^{\frac{4}{\fanin+1}}n^\frac{\fanin}{\fanin+1}$, it follows that
    \begin{align*}
      \P\left(u_t \geq \left(1 + \frac{1}{\ln n}\right) \frac{(\uninformed{t-1})^{\fanin+1}}{n^{\fanin}}\right) 
        & \leq  e^{- \frac{1}{3} \ln^2 n} \\ & \in o\left(n^{-c}\right) \text{ for any constant $c$}
    \end{align*}
and we can deduce that
    \begin{equation}
      \label{eq:uninformed-whp}
      \uninformed{t} \leq \left(1 + \frac{1}{\ln n}\right) \frac{(\uninformed{t-1})^{\fanin+1}}{n^{\fanin}}
    \end{equation}
    with high probability. Applying Eq. (\ref{eq:uninformed-whp}) recursively, we obtain
    \begin{align}
      \uninformed{t} & \leq (\uninformed{0})^{{(\fanin+1)^t}}\left(\frac{1 + \frac{1}{\ln n}}{n^{\fanin}}\right)^\frac{(\fanin+1)^t-1}{\fanin} 
    \end{align}
Replacing $u_0$ by $n - \frac{n }{\ln n}$, and $t$ by $4 \log_{\fanin+1}\ln n$ we obtain 
\begin{equation}
  \begin{split}
   \uninformed{t} & \leq \left(n-\frac{n}{\ln n}\right)^{{(\fanin+1)^t}}\left(\frac{1 + \frac{1}{\ln n}}{n^{\fanin}}\right)^\frac{(\fanin+1)^t-1}{\fanin} \\ & \leq n \left(1-\frac{1}{\ln n}\right)^{\ln^4 n} \left(1+\frac{1}{\ln n}\right)^{\ln^4 n} \\ & \leq n \left(1-\frac{1}{\ln^2 n}\right)^{\ln^4 n} \\ & \in o(1)
 \end{split}
\end{equation}
which shows that we need $O(\log_{\fanin+1} \ln n)$ rounds to reach the point where there are at most $(\ln n)^{\frac{4}{\fanin+1}}n^\frac{\fanin}{\fanin+1}$ uninformed processes with high probability. Note that this step is unnecessary if $\fanin$ is large enough with respect to $n$ since $(\ln n)^{\frac{4}{\fanin+1}}n^\frac{\fanin}{\fanin+1} \geq n-\frac{n}{\ln n}$.

At this stage, the probability that an uninformed process remains uninformed after each subsequent round is at most 
    \begin{align}
      \label{eq:21}
      \left(\frac{\uninformed{r}}{n}\right)^{\fanin} & \leq \left(\frac{(\ln n)^{\frac{4}{\fanin+1}}n^\frac{\fanin}{\fanin+1}}{n}\right)^{\fanin} \leq \frac{(\ln n)^4}{\sqrt{n}}.
    \end{align}
    Hence after a constant number of additional rounds, we inform every remaining uninformed process with high probability. 

  \end{proof}

% ----------------------------------------------------------------------------------------------

  \begin{corollary}
 If $\fanin\in\Omega(\ln n)$, the regular pull algorithm starting with $\frac{n}{\ln n}$ informed processes informs all processes with high probability in $\Theta(1)$ rounds.
  \end{corollary}

  
% ----------------------------------------------------------------------------------------------

\begin{theorem}
  \label{thm:pull}
    The regular pull algorithm disseminates a rumor to all processes with high probability in $\Theta(\log_{\fanin + 1} n)$ rounds of communication.
\end{theorem}

\begin{proof}
  For the lower bound, it is clear from Lemma~\ref{lem:pullasymptotic} that $\Omega(\log_{\fanin+1} n)$ rounds are required in expectation to inform all processes, and thus necessary to inform all processes with high probability.

  We now show that $\mathcal{O}(\log_{\fanin + 1} n)$ rounds suffice when
  $\fanin \in \mathcal{O}(\ln n)$ (the statement for $\fanin = 1$ is implicitly discussed without proof in \cite{DBLP:conf/focs/KarpSSV00}). 

  In a first phase, we show that $\mathcal{O}(\log_{\fanin+1}n)$ rounds are sufficient to inform $\ln n$ processes with high probability. Let $c_0 \geq 1$ be a constant. In this case, we show that for stages $k \in \{0,1,2,\dots,\ln\ln n\}$, if $\informed{r} = 2^k$ processes are informed, then after $\rho_k \triangleq c_0\left\lceil \frac{\log_{\fanin + 1} n}{2^k} \right \rceil$ rounds, the number of informed processes doubles with high probability, i.e., $\informed{r+\rho_k} \geq 2^{k+1}$ with high probability. At every round of stage $k$, each pull request has a probability at least $\frac{2^k}{n}$ of reaching an informed process, thus after $\rho_k$ rounds and $\rho_k \cdot \fanin$ pull requests, the probability that an uninformed process learns the rumor is bounded by
  \begin{align}
  p \geq 1 - \left( 1 - \frac{2^k}{n} \right)^{\rho_k \cdot \fanin} \geq \frac{2^k\rho_k\fanin}{n}-\frac{2^{2k}\rho_k^2\fanin^2}{n^2}.    
  \end{align}
  The probability $T$ to inform $l=\informed{r}=2^k$ processes or less in stage $k$ is upper bounded by the left tail of the binomial distribution with parameters $p$ and $N = \uninformed{r} = n-2^k$. We can bound this tail using the Chernoff bound
  \begin{align}
    \label{eq:chernoffbinomial}
       T & \leq \exp\left(-\frac{(Np-l)^2}{2Np}\right)
  \end{align}
  which is valid when $l \leq N p$. We can indeed apply this bound by showing that $Np \geq \frac{c_0 \fanin}{\ln(\fanin + 1)} \ln n + o(1)$, which is greater than $2^k$ when $c_0 \geq 1$. The Chernoff bound gives
  \begin{equation}
       \begin{split}
     \label{eq:3}
      T 
     & \leq \exp\left(-\frac{Np}{2}+l\right) \\
   %  \\
        & \leq \exp \left(-\frac{(n-2^k)p}{2} + 2^{k}\right) \\     
    & \leq \exp\left(\left(1-\frac{c_0\fanin}{2\ln(\fanin+1)}\right) \ln n +o(1)\right) \\
  & \in \mathcal{O}\left(n^{  1-\frac{c_0\fanin}{2\ln(\fanin+1)}}  \right)
\end{split}
\end{equation}
and for any constant $c>0$ we can find $c_0$ such that $T \in \mathcal{O}\left(n^{-c}\right)$. This first phase, with the $k$ stages, requires $\sum\limits_{k=0}^{\ln\ln n} \rho_k \leq c_0 \log_{\fanin + 1} n \cdot \sum\limits_{k=0}^{\ln\ln n} 2^{-k} + c_0(\ln\ln n + 1) \sim 2c_0 \log_{\fanin + 1} n$ rounds of communication to inform $1 + 2^0 + 2^1 + \dots + 2^{\ln\ln n} \approx 2 \ln n$ processes with high probability. 

In a second phase, when $\ln n \leq \informed{r} \leq \frac{n}{(\ln n)^2}$, we show that a constant number of rounds $c_1$ is sufficient to multiply the number of informed processes by $\fanin+1$ with high probability. We use the Chernoff bound of Eq.~\eqref{eq:chernoffbinomial} with $l=\fanin \cdot\informed{r}$, $n-\frac{n}{\ln n}\leq N \leq n-\ln n$ and $p \geq 1-\left(1-\frac{\informed{r}}{n}\right)^{c_1 \fanin} \geq \frac{\informed{r}c_1\fanin}{n} -\frac{\informed{r}^2 c_1^2 \fanin^2}{2n^2}.$ We obtain
\begin{equation}
  \begin{split}
  T & \leq \exp\left(-\frac{Np}{2}+l\right) \\
    & \leq \exp \left( -\frac{\informed{r}c_1\fanin}{2}\left(1-o(1)\right)+\informed{r}\fanin\right) \\
    & \leq \exp \left(\ln n\left(1-\frac{c_1}{2}+o(1)\right) \right) \\
    & \in \mathcal{O}\left(n^{  1-\frac{c_1}{2}}  \right)
  \end{split}
\end{equation}
and for any constant $c>0$ we can find $c_1$ such that $T \in \mathcal{O}\left(n^{-c}\right)$.  This second phase requires $\mathcal{O}(\log_{\fanin + 1} n)$ rounds of communication.

In a third phase, we can go from $\frac{n}{(\ln n)^2}$ to $\frac{n}{\ln n}$ informed processes in $\mathcal{O}(\log_{\fanin + 1} n)$ rounds of communication since multiplying the number of informed processes by $\ln n$ at this stage cannot be slower than during the first phase.  Finally, in a fourth phase we saw in Lemma~\ref{lem:loglog} that we can go from $\frac{n}{\ln n}$ to $n$ informed processes with high probability with $\Theta\left(\log_{\fanin+1}\ln n\right)$ rounds of communication.

We now summarize the proof of the upper bound when $\fanin\in \omega(\ln n)$ and $\fanin\in\mathcal{O}(n)$. The different cases must me handled with care, but we omit the details for simplicity purposes. In a first phase, we show that  $\mathcal{O}(\log_{\fanin + 1} n)$  rounds are sufficient to inform $\ln n$ processes with high probability. In a second phase, if $\fanin \cdot \informed{r} \in o(n)$, we apply the Chernoff bound of Eq.~\eqref{eq:chernoffbinomial} during $\mathcal{O}(\log_{\fanin + 1} n)$  rounds to reach either $\frac{n}{\ln n}$ informed process with high probability, or $\fanin \cdot \informed{r} \in \Theta(n)$ (the Chernoff bound must be changed when $\fanin \cdot \informed{r} \in \Theta(n)$). If $\fanin \cdot i \in \Theta(n)$, we again apply Eq.~\eqref{eq:chernoffbinomial} during a constant number of rounds to reach $c_2 \cdot n$ informed processes with $c_2<1$ with high probability. Finally, in a last phase, we go from $c_2 \cdot  n$ or $\frac{n}{\ln n}$ to $n$ informed processes with high probability using Lemma~\ref{lem:loglog}.

\end{proof}

% ----------------------------------------------------------------------------------------------

\begin{corollary}
  If $\fanin \in \mathcal{O}(1)$, then the total number of messages (replies to pull requests) required by the regular pull algorithm is in $\Theta(n)$. In particular, the communication overhead is 0 when $\fanin=1$.
\end{corollary}

\begin{proof}
  It is clear that a process cannot pull a rumor more than $\fanin$ times since it stops requesting it in the rounds that follow its reception.
\end{proof}

% ----------------------------------------------------------------------------------------------

We now prove that the round complexity of the regular pull algorithm is asymptotically optimal for the generalized random phone call model.

\begin{theorem}
  \label{thm:pushpullround}
If $f=\fanin=\fanout$, any protocol in the generalized random phone call model requires $\Omega(\log_{f+1} n)$ rounds of communication to disseminate a rumor to all processes with high probability.
\end{theorem}

\begin{proof}
  % The bit complexity and the message complexity of the regular pull algorithm are trivially optimal. For the round complexity,
Let $f=\max(\fanin,\fanout)$. 
If we only push messages, it is clear that the number of informed processes increases at most by a factor of $(\fanout + 1)$ per round. If we only pull messages, we saw in Lemma~\ref{lem:pullasymptotic}
  that the number of informed processes increases at most by a factor of $(\fanin + 1)$ per round in
  expectation. If all processes simultaneously push and pull at every round, the number of informed processes
  increases at most by a factor of $(\fanin + 1)(\fanout + 1)$ per round in expectation, thus the number or
  rounds required to informed all processes is at least
  $\log_{(\fanout+1)(\fanin+1)} n \geq \log_{(f+1)^2} n \in \Omega(\log_{f+1} n)$.

\end{proof}

% ----------------------------------------------------------------------------------------------

We now show that the regular pull algorithm is robust against adversarial and stochastic failures.  First, consider an adversary that fails $\epsilon \cdot n$ processes for $0 \leq \epsilon < 1$, excluding the process starting the rumor. Before the execution of the algorithm, the adversary decides which processes fail, and for each failed process during which round it fails. Once a process fails, it stops participating until the end of the execution, although it may still be uselessly called by active processes. We also consider stochastic failures, in the sense that each phone call fails with probability $\delta$ for $0 \leq \delta < 1$. Note that both types of failures are independent of the execution.%\mm{i do not understand the non byzantine model and independence. A non byzantine adversary could simply stop processes based on the state of the execution}. 

The main difference introduced by the failures is that we can no longer go from $\frac{n}{\ln n}$ to $n$ informed processes in $\mathcal{O}(\log_{f+1} \ln n)$ rounds because there is a non-vanishing probability that pull requests either target failed processes or result in failed phone calls. We nevertheless show that the regular pull algorithm can disseminate a rumor to all $(1-\epsilon)n$ good (i.e., non-failed) processes with high probability with the same asymptotic round complexity. 

\begin{theorem}
  \label{thm:failures}
  Let $0 \leq \epsilon < 1$, and let $0 \leq \delta < 1$. If $\epsilon \cdot n$ processes, excluding the initial process with the rumor, fail adversarially, and if phone calls fail with probability $\delta$, then the regular pull algorithm still disseminates a rumor to all $(1-\epsilon)n$ good processes with high probability in $\Theta(\log_{\fanin + 1} n)$ rounds of communication.
\end{theorem}

\begin{proof}
  It is clear that the lower bound remains valid when there are failures. We prove the upper bound for $\fanin\in\mathcal{O}(\ln n)$, but as we mentioned for Theorem~\ref{thm:pull} we can adapt the proof for $\fanin\in\omega(\ln n)$ by carefully applying Chernoff bounds in different phases.

Note that the earlier a process fails, the more damage it causes. We thus assume that the $\epsilon \cdot n$ processes fail at the beginning of the execution, which is the worst possible scenario. We can use the first three phases of the proof of Theorem \ref{thm:pull} with minor modifications (only multiplicative constants change) and prove that $\mathcal{O}(\log_{\fanin + 1} n)$ rounds are sufficient to go from 1 to $\frac{n}{\ln n}$ informed processes with high probability. 

We now show that we need $c_2\log_{\fanin + 1} n$ rounds to go from $\frac{n}{\ln n}$ to $c_1 \cdot n$ informed processes with high probability for some arbitrary $c_1 < 1-\epsilon$. We again use the Chernoff bound of Eq.~\eqref{eq:chernoffbinomial} with $\informed{r}=\frac{n}{\ln n}$, $l = c_1 \cdot n$ and $N = (1-\epsilon) n - \frac{n}{\ln n}$. If $c_2$ is a large enough constant, the probability that a process learns a rumor during that phase is
  \begin{align}
  p \geq 1-\left(1-\frac{(1-\delta)\informed{r}}{n}\right)^{\fanin c_2\log_{\fanin + 1} n} \geq 1-\left(1-\frac{1-\delta}{\ln n}\right)^{\frac{\fanin c_2\ln n}{\ln(\fanin+1)}} \geq 1-e^{-c_2(1-\delta)} \triangleq c_3.
  \end{align}
The Chernoff bound gives
\begin{equation}
 \begin{split}
    T & \leq \exp\left(-\frac{Np}{2}+l\right) \\
      & \leq \exp \left( -c_3n\left(1-\epsilon-\frac{1}{\ln n}\right)+c_1 n \right)\\
     & \leq \exp \left( n\left(-c_3+c_3\epsilon+c_1+o(1) \right)\right)\\
%    & \leq \exp \left( -\frac{\informed{r}c_1\fanin}{2}\left(1-o(1)\right)+\informed{r}\fanin\right) \\
%    & \leq \exp \left(\ln n\left(1-\frac{c_1}{2}+o(1)\right) \right) \\
 %   & \in \mathcal{O}\left(n^{  1-\frac{c_1}{2}}  \right)
   \end{split}
 \end{equation}
and we can choose $c_2$ such that $T \leq e^{c_4 n}$ with $c_4 < 0$. This guarantees $T \in \mathcal{O}\left(n^{-c}\right)$ for any~$c>0$.

Starting from $c_1 n$ informed processes, the probability that a process is informed in any subsequent round is bounded by $p \geq 1-\left(\frac{n-c_1(1-\delta)n}{n}\right)^{\fanin} \geq 1 - (1-c_1(1-\delta))^{\fanin}$.  After $r$ such rounds, the probability that a process remains uninformed is thus upper bounded by $(1-c_1(1-\delta))^{\fanin r}$, and for this probability to be bounded by $n^{-c}$ we need
  \begin{align}
    (1-c_1(1-\delta))^{\fanin r} \leq n^{-c} \Leftrightarrow
    r \geq  \frac{c \ln n}{\ln{\frac{1}{1-c_1(1-\delta)}}\fanin} \geq c_4 \log_{f+1} n \text{ for some constant $c_4$.}
  \end{align}
Hence, $\mathcal{O}(\log_{f+1} n)$ rounds are sufficient to go from $c_1n$ to $(1-\epsilon) n$ informed processes with high probability.

\end{proof}

Note that adversarial and stochastic failures do not increase the message complexity of the regular pull algorithm: uninformed processes that fail decrease the number of rumor transmissions, and failed phone calls do not exchange the rumor. We could, however, consider that messages containing the rumor are dropped with probability $0 \leq \gamma < 1$. Theorem~\ref{thm:failures} also holds in this instance, but the number of messages increases by an unavoidable factor of $\frac{1}{1-\gamma}$.

% ----------------------------------------------------------------------------------------------

% ----------------------------------------------------------------------------------------------

% ----------------------------------------------------------------------------------------------

%%% Local Variables:
%%% mode: latex
%%% TeX-master: "main"
%%% End:

\mySection{Related Works and Discussion}{}
\label{chap3:sec:discussion}

In this section we briefly discuss the similarities and differences of the model presented in this chapter, comparing it with some related work presented earlier (Chapter \ref{chap1:artifact-centric-bpm}). We will mention a few related studies and discuss directly; a more formal comparative study using qualitative and quantitative metrics should be the subject of future work.

Hull et al. \citeyearpar{hull2009facilitating} provide an interoperation framework in which, data are hosted on central infrastructures named \textit{artifact-centric hubs}. As in the work presented in this chapter, they propose mechanisms (including user views) for controlling access to these data. Compared to choreography-like approach as the one presented in this chapter, their settings has the advantage of providing a conceptual rendezvous point to exchange status information. The same purpose can be replicated in this chapter's approach by introducing a new type of agent called "\textit{monitor}", which will serve as a rendezvous point; the behaviour of the agents will therefore have to be slightly adapted to take into account the monitor and to preserve as much as possible the autonomy of agents.

Lohmann and Wolf \citeyearpar{lohmann2010artifact} abandon the concept of having a single artifact hub \cite{hull2009facilitating} and they introduce the idea of having several agents which operate on artifacts. Some of those artifacts are mobile; thus, the authors provide a systematic approach for modelling artifact location and its impact on the accessibility of actions using a Petri net. Even though we also manipulate mobile artifacts, we do not model artifact location; rather, our agents are equipped with capabilities that allow them to manipulate the artifacts appropriately (taking into account their location). Moreover, our approach considers that artifacts can not be remotely accessed, this increases the autonomy of agents.

The process design approach presented in this chapter, has some conceptual similarities with the concept of \textit{proclets} proposed by Wil M. P. van der Aalst et al. \citeyearpar{van2001proclets, van2009workflow}: they both split the process when designing it. In the model presented in this chapter, the process is split into execution scenarios and its specification consists in the diagramming of each of them. Proclets \cite{van2001proclets, van2009workflow} uses the concept of \textit{proclet-class} to model different levels of granularity and cardinality of processes. Additionally, proclets act like agents and are autonomous enough to decide how to interact with each other.

The model presented in this chapter uses an attributed grammar as its mathematical foundation. This is also the case of the AWGAG model by Badouel et al. \citeyearpar{badouel14, badouel2015active}. However, their model puts stress on modelling process data and users as first class citizens and it is designed for Adaptive Case Management.

To summarise, the proposed approach in this chapter allows the modelling and decentralized execution of administrative processes using autonomous agents. In it, process management is very simply done in two steps. The designer only needs to focus on modelling the artifacts in the form of task trees and the rest is easily deduced. Moreover, we propose a simple but powerful mechanism for securing data based on the notion of accreditation; this mechanism is perfectly composed with that of artifacts. The main strengths of our model are therefore : 
\begin{itemize}
	\item The simplicity of its syntax (process specification language), which moreover (well helped by the accreditation model), is suitable for administrative processes;
	\item The simplicity of its execution model; the latter is very close to the blockchain's execution model \cite{hull2017blockchain, mendling2018blockchains}. On condition of a formal study, the latter could possess the same qualities (fault tolerance, distributivity, security, peer autonomy, etc.) that emanate from the blockchain;
	\item Its formal character, which makes it verifiable using appropriate mathematical tools;
	\item The conformity of its execution model with the agent paradigm and service technology.
\end{itemize}
In view of all these benefits, we can say that the objectives set for this thesis have indeed been achieved. However, the proposed model is perfectible. For example, it can be modified to permit agents to respond incrementally to incoming requests as soon as any prefix of the extension of a bud is produced. This makes it possible to avoid the situation observed on figure \ref{chap3:fig:execution-figure-4} where the associated editor is informed of the evolution of the subtree resulting from $C$ only when this one is closed. All the criticisms we can make of the proposed model in particular, and of this thesis in general, have been introduced in the general conclusion (page \pageref{chap5:general-conclusion}) of this manuscript.





% Bibliography
% \printbibliography
\bibliographystyle{unsrtnat}
\bibliography{bib.bib}

\pagebreak


\end{document}

%%% Local Variables:
%%% mode: latex
%%% TeX-master: t 
%%% End:
