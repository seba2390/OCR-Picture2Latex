\documentclass[format=acmsmall,review=false,screen=true,authorversion=true]{acmart}
\pdfoutput=1

\settopmatter{printccs=false,printacmref=false,printfolios=true}
\setcopyright{none}
\acmDOI{}

% \acmJournal{JACM}

\usepackage{multirow}
\usepackage[utf8]{inputenc}
\usepackage[T1]{fontenc}
\usepackage[australian,american]{babel}
\usepackage{graphicx}
\usepackage{thmtools, mathdots}
\usepackage[framemethod=tikz]{mdframed}
\usepackage{pgfplots}
\pgfplotsset{compat=1.12}
\usepackage{array}
\usepackage[binary-units]{siunitx}
\usepackage{booktabs}
\usepackage{multirow}
\usepackage{pifont}
\usepackage{tikz}
\usepackage{csquotes}
%\usepackage{natbib}
\setcitestyle{square,aysep={},yysep={;}}
%\usepackage{url}
\usepackage{color}
\usepackage{todonotes}
\usepackage{lipsum}
\usepackage{subfigure}
\usepackage{pgfplotstable}
%\usepackage{amsmath, amssymb, amsthm}

\newcommand{\N}{\mathbb{N}}
\newcommand{\Z}{\mathbb{Z}}
\newcommand{\Q}{\mathbb{Q}} 
\newcommand{\R}{\mathbb{R}}
\renewcommand{\C}{\mathbb{C}}
\newcommand{\E}{\mathbb{E}}
\renewcommand{\P}{\mathbb{P}}
\renewcommand{\epsilon}{\varepsilon} 
\renewcommand{\phi}{\varphi}
\renewcommand{\emph}{\textbf}

\newcommand{\fanin}{f_\text{in}}
\newcommand{\fanout}{f_\text{out}}
\newcommand{\view}{viewSize}
\newcommand{\rpushonly}{r}
\newcommand{\rpull}{r_\text{pull}}
\newcommand{\rpush}{r_\text{push}}
\newcommand{\uninformed}[1]{u_{#1}}
\newcommand{\informed}[1]{i_{#1}}
\newcommand{\new}[1]{X_{#1}}
\newcommand{\Ppull}[2]{\P_{\text{pull}} \left( #1 \mid #2 \right)}
\newcommand{\Ppush}[2]{\P_{\text{push}} \left( #1 \mid #2 \right)}
\newcommand{\pscale}{p_\text{scale}}

% Package to generate and customize Algorithm as per ACM style
\usepackage[ruled]{algorithm2e}
% \SetAlFnt{\algofont}
% \SetAlCapFnt{\algofont}
% \SetAlCapNameFnt{\algofont}
% \SetAlCapHSkip{0pt}
% \IncMargin{-\parindent}
% \renewcommand{\algorithmcfname}{ALGORITHM}

% Add only one number to align* environment
\newcommand\numberthis{\addtocounter{equation}{1}\tag{\theequation}}

% Command for displaying stirling numbers of the second kind
\DeclareRobustCommand{\stir}{\genfrac\{\}{0pt}{}}

%% Environment for comments: Set the boolean to false to produce a comment-free version.
\newboolean{showcomments}
\setboolean{showcomments}{false}
\ifthenelse{\boolean{showcomments}}
{ \newcommand{\mynote}[3]{
    \fbox{\bfseries\sffamily\scriptsize#1}
    {\small$\blacktriangleright$\textsf{\textit{\color{#3}{#2}}}$\blacktriangleleft$}}}
{ \newcommand{\mynote}[3]{}}
% One command per author:
\newcommand{\hm}[1]{\mynote{Hugues}{#1}{orange}}
\newcommand{\lh}[1]{\mynote{Laurent}{#1}{ForestGreen}}
\newcommand{\mm}[1]{\mynote{Miguel}{#1}{blue}}
\newcommand{\jt}[1]{\mynote{Jocelyn}{#1}{SaddleBrown}}

% %% algorithm2e
\DontPrintSemicolon

\SetFuncSty{sc}
\SetDataSty{em}
\SetCommentSty{em}

\SetKwFor{func}{procedure}{}{end}
\SetKwFor{upon}{upon}{}{end}
\SetKwFor{struct}{struct}{}{end}
\SetKwFor{periodically}{task every $\delta$ time units}{}{end}
\SetKwFor{initially}{initially}{}{end}

\SetKw{Break}{break}
\SetKw{return}{return}

\SetKwFunction{BALL}{BALL}
\SetKwFunction{PULL}{PULL}
\SetKwFunction{broadcast}{Broadcast}
\SetKwFunction{receive}{receive}
\SetKwFunction{deliver}{deliver}
\SetKwFunction{Random}{Random}
\SetKwFunction{send}{send}
\SetKwFunction{dest}{to}
\SetKwFunction{orderEvents}{orderEvents}
\SetKwFunction{receiveEvents}{receiveEvents}
\SetKwFunction{isDeliverable}{isDeliverable}
\SetKwFunction{isPush}{\underline{isPush}}
\SetKwFunction{isPull}{\underline{isPull}}
\SetKwFunction{getClock}{getClock}
\SetKwFunction{updateClock}{updateClock}

\SetKwBlock{Block}{}{}
% %% algorithm2e
%%%%%%%%%%%%%%%%%%%%%%%%%%%%%%%%%%%%%%%%%%%%%%%%%%%%%%%%%%%%%%%%%%%%%%



% For syncronisation with skim
\synctex = 1

%\setcopyright{none}

% Page heads
%\markboth{L. Hayez, M. Matos, H. Mercier}{Title here}

% Title portion

\title{Optimal epidemic dissemination}

\author{Hugues Mercier}
\email{hugues.mercier@unine.ch}
\author{Laurent Hayez}
\email{laurent.hayez@unine.ch}
\affiliation{%
  \institution{Université de Neuchâtel}
  \city{Neuchâtel}
  \country{Switzerland}
}
\author{Miguel Matos}
\affiliation{%
  \institution{INESC-ID \& IST, Universidade de Lisboa}
  \city{Lisboa}
  \country{Portugal}
}

\begin{abstract}
{
\begin{abstract}
\label{sec:abstract}

%% 1. what is the problem 
Scientific applications that run on leadership computing facilities often face the challenge 
of being unable to fit leading science cases onto accelerator devices due to memory constraints 
(memory-bound applications).
%
% 2. what is your solution 
In this work, the authors studied one such US Department of Energy mission-critical condensed matter 
physics application, Dynamical Cluster Approximation (DCA++), and this paper discusses how device memory-bound challenges were successfully reduced  by proposing an effective 
``all-to-all'' communication method---a ring communication algorithm. 
%
This implementation takes advantage of acceleration on GPUs and remote direct memory access (RDMA) for fast data exchange between GPUs. 
%
\\Additionally, the ring algorithm was optimized with sub-ring communicators
and multi-threaded support to further reduce communication overhead and 
expose more concurrency, respectively.
%
% 3. What's the cherry-picked evaluation result you want to mention
The computation and communication were also analyzed 
by using the Autonomic Performance Environment for Exascale 
(APEX) profiling tool,  and this paper further discusses the 
performance trade-off for the ring algorithm implementation. 
%
The memory analysis on the ring algorithm shows that the allocation size for the authors' most 
memory-intensive data structure per GPU is now reduced to $1/p$ of the original size, where $p$ is the number of GPUs in the ring communicator.
%
The communication analysis suggests that 
the distributed Quantum Monte Carlo execution time grows linearly as sub-ring size increases, and the cost of messages passing through the network interface connector could be a limiting factor.


%
% \todoRed{Ronnie: Next sentence needs rewrite, too much information about Green's function that no one knows in the abstract; recommend generalizing.} \emph {However, DCA++ is currently facing memory-bound challenge as 
% a larger device array $G_t$ is limited by device memory size, where
% $G_t$ is a two-particle Green's function that allows condensed matter
% scientists to explore larger and more complex (higher fidelity)
% physics cases.}

\end{abstract}

\keywords{DCA++, Quantum Monte Carlo, GPU Remote Direct Memory Access, memory-bound issue, exascale machines}

}
\end{abstract}

\thanks{A brief announcement of this work was presented at PODC 2017.}

\begin{document}

\maketitle

\section{Introduction}  \label{sec:introduction}

\newcommand\inexpIntro[3]{#1?(#2,#3).}
\newcommand\rinexpIntro[3]{*#1?(#2,#3).}
\newcommand\outexpIntro[3]{#1!(#2,#3).}
\newcommand\outatomIntro[3]{#1!(#2,#3)}

We propose a fully automated method for proving termination of \(\pi\)-calculus processes.
Although there have been a lot of studies on termination analysis for the \(\pi\)-calculus
and related calculi~\cite{Deng06IC,Demangeon07,SangiorgiTermination,KobayashiHybrid,Yoshida04IC,DBLP:journals/jlp/DemangeonHS10,Venet98SAS}, most of them have been rather theoretical,
and there have been surprisingly little efforts in developing  fully automated termination
verification methods and tools based on them. To our knowledge,
Kobayashi's \typical{}~\cite{TyPiCal,KobayashiHybrid} is the only exception that
can prove termination of \(\pi\)-calculus processes (extended with natural numbers)
fully automatically, but its termination analysis is quite limited (see Section~\ref{sec:relatedwork}).

Our method is based on a reduction to termination analysis for sequential programs:
we translate a \(\pi\)-calculus process \(P\) to a sequential program \(S_P\), so that
if \(S_P\) is terminating, so is \(P\). The reduction allows us to use
powerful, mature methods and tools
for termination analysis of sequential programs~\cite{heizmann2016ultimate,freqterm,DBLP:conf/lics/PodelskiR04,Kuwahara2014Termination,DBLP:journals/cacm/CookPR11}.

The idea of the translation is to convert a chain of communications on replicated input
channels to a chain of recursive function calls of the target sequential program.
Let us consider the following Fibonacci process:
\begin{align*}
    & \rinexpIntro{\fib}{n}{r}
        \ifexp{n<2}{ \soutatom{r}{1} \\ &\quad}
                   { \nuexp{s_1} \nuexp{s_2} (\outatomIntro{\fib}{n-1}{s_1} \PAR \outatomIntro{\fib}{n-2}{s_2} \PAR \sinexp{s_1}{x}\sinexp{s_2}{y}\soutatom{r}{x+y}) \\}
    & \PAR \outatomIntro{\fib}{m}{r}
\end{align*}
Here, the process
$\rinexpIntro{\fib}{n}{r} \ldots$ is a function server that computes the \(n\)-th Fibonacci number
in parallel and returns the result to \(r\),
and $\outatom{\fib}{m}{r}$ sends a request for computing the \(m\)-th Fibonacci number;
those who are not familiar with the syntax of the \(\pi\)-calculus may wish to consult
Section~\ref{sec:targetlanguage} first.
To prove that the process above is terminating for any integer \(m\),
it suffices to show that there is no infinite chain of communications on $\fib$:
\[
    \fib(m,r) \to \fib(m_1,r_1) \to \fib(m_2,r_2) \to \cdots.
\]
We convert the process above to the following program:\footnote{The actual translation
  given later is a little more complex.}
\begin{verbatim}
 let rec fib(n) = if n<2 then () else (fib(n-1) [] fib(n-2)) in
 fib(m)
\end{verbatim}
Here, \texttt{[]} represents the non-deterministic choice.
Note that, although the calculation of Fibonacci numbers is not preserved,
for each chain of communications on \texttt{fib}, there is a corresponding
sequence of recursive calls:
\[
\mathtt{fib}(m) \to \mathtt{fib}(m_1) \to \mathtt{fib}(m_2) \to \cdots.
\]
Thus, the termination of the sequential program above implies the termination of
the original process.
As shown in the example above, (i) each communication on a replicated input channel
is converted to a function call, (ii) each communication on a non-replicated input
channel is just removed (or, in the actual translation, replaced by a call of
a trivial function defined by \(f(\seq{x})=(\,)\)), and (iii) parallel composition
is replaced by a non-deterministic choice.
We formalize the translation outlined above and prove its correctness.

The basic translation sketched above sometimes loses too much information.
For example, consider the following process:
\begin{align*}
    & \rinexpIntro{\pre}{n}{r} \soutatom{r}{n-1} \\
    & \PAR \rinexpIntro{f}{n}{r} \ifexp{n<0}{ \soutatom{r}{1} }
                                       { \nuexp{s} (\outatomIntro{\pre}{n}{s} \PAR \sinexp{s}{x}\outatomIntro{f}{x}{r}) } \\
    & \PAR \outatomIntro{f}{m}{r}
\end{align*}
The translation sketched above would yield:
\begin{verbatim}
  let pred(n) = n-1 in
  let rec f(n) = if n<0 then () else (pred(n) [] f(*)) in
  f(m)
\end{verbatim}
Here, \texttt{*} represents a non-deterministic integer: since we have removed
the input $\sinatom{s}{x}$, we do not have information about the value of \( x \).
As a result, the sequential program above is non-terminating, although the original
process is terminating.
To remedy this problem, we also refine the basic translation above by using a refinement
type system for the \(\pi\)-calculus. Using the refinement type system,
we can infer that the value of \(x\) in the original process is less than \(n\),
so that we can refine the definition of \texttt{f} to:
\begin{verbatim}
 let rec f(n) = ... else (pred(n) [] let x=* in assume(x<n);f(x))
\end{verbatim}
The target program is now terminating, from which
we can deduce that the original process is also terminating.
We have implemented an automated tool based on the refined translation above.

The contributions of this paper are summarized as follows.
\begin{itemize}
\item The formalization of the basic translation from the \(\pi\)-calculus
  (extended with integers) to sequential programs, and a proof of its correctness.
\item The formalization of a refined translation based on a refinement type system.
\item An implementation of the refined translation, including automated refinement type
  inference based on CHC solving, and experiments to evaluate the effectiveness of
  our method.
\end{itemize}

The rest of this paper is structured as follows.
Section~\ref{sec:targetlanguage} introduces the source and target languages
of our translation.
Section~\ref{sec:approach} 
formalizes the basic translation, and proves its correctness.
Section~\ref{sec:refinement} refines the basic translation by using a refinement type system.
Section~\ref{sec:implementation} reports an implementation and experiments.
Section~\ref{sec:relatedwork} discusses related work,
and Section~\ref{sec:conclusion} concludes the paper.

\subsection{Multitask Learning}

MTL has been succesfully used in different domains, including CV \cite{UberNet,MaskRCNN}. Some challenges appear when applying it \cite{Caruana}: \textit{learning speed} differences between tasks and  deciding \textit{what to share} according to the \textit{relatedness} between tasks in the multitask architecture \cite{Stitch, AdaptiveFeatureSharing}.

\subsection{Semantic Segmentation}

Semantic segmentation aims at partitioning parts of images belonging to the same semantic class, typically via pixel-wise classification. Fully convolutional networks (FCN) \cite{FCN} have improved both accuracy and speed for dense prediction problems by using only convolutional layers. Upsampling layers allow a segmentation output size equal to the input and skip connections add finer details. Other approaches add post-processing steps \cite{DeeplabCRF}, learnable \textit{deconvolution} layers \cite{ Deconv} or global context \cite{ParseNet}.

\subsection{Object Detection}

Object detection aims at finding in an image all instances of objects and classifying them in a number of classes. Faster R-CNN \cite{FasterRCNN} was the first to give close to real-time performance. YOLO \cite{YOLO} avoids the generation of region proposals for increased speed. SSD \cite{SSD} avoids fully-connected layers for speed and takes features at different levels for improved accuracy. 

%\cite{SpeedAccuracy} reviews the speed vs. accuracy trade-off for different object detectors.
% ----------------------------------------------------------------------------------------------

\section{The regular pull algorithm is asymptotically optimal}
\label{sec:pull}

In this section, we focus on pull-only algorithms. 
Our first observation is that on expectation, pulling is always at least as good as pushing, although the higher variance of pull at the early stage of the dissemination makes pulling less efficient when the rumor is new. For instance, starting with one informed process and $\fanin=\fanout=1$, it takes $\Theta(\ln n)$ pull rounds to inform a second process with high probability, whereas a single push round suffices. The behavior reverses when the rumor is old: if $n-1$ processes are already informed, a single pull round informs the last process with high probability but $\Theta(\ln n)$ push rounds are needed. Despite these differences, our second observation is that pulling and pushing have the same asymptotic round complexity. Our third observation is that the regular pull algorithm is asymptotically optimal, thus pushing is not required. 
Our fourth observation is that the regular pull algorithm asymptotically requires the same round, bit, and message complexity even in the presence of a large number of adversarial and stochastic failures.

Note that in the generalized random phone call model, processes push and pull requests uniformly at random but independently (i.e., with replacement), thus they can push the rumor to themselves, call themselves, and have multiple push messages and/or pull requests colliding in the same round. Of course in practice, in a given round, a process will not send multiple pull requests or multiple push messages to the same process, nor will it call itself. Instead, it will select a uniform random sample among the other processes in the network. Our reason for this definition is twofold. First, choosing interlocutors independently and uniformly at random is more amenable to mathematical analysis, especially upper bounds. Second, we prove that choosing $f$ processes uniformly at random with replacement, or choosing a uniform random sample of size $f$ without replacement among the other $n-1$ processes, are asymptotically equivalent when $f \in \mathcal{O}(n)$. We prove this by matching lower bounds obtained from random samples with upper bounds obtained with interlocutors selected independently and uniformly at random.
\begin{definition}
  Let $0 \leq \uninformed{r} \leq n$ be the number of uninformed processes at round $r$,  $\E_{\text{pull}}[\uninformed{r}]$ the expected number of uninformed processes at round $r$ with the regular pull algorithm, and $\E_{\text{push}}[\uninformed{r}]$ the expected number of uninformed processes at round $r$ with the regular push algorithm. For the number of informed processes at round $r$, we similarly define $\informed{r}$, $\E_{\text{pull}}[\informed{r}]$ and $\E_{\text{push}}[\informed{r}]$. It is clear that $n=\uninformed{r}+\informed{r}=\E_{\text{pull}}[\uninformed{r}]+\E_{\text{pull}}[\informed{r}]=\E_{\text{push}}[\uninformed{r}]+\E_{\text{push}}[\informed{r}]$.
\end{definition}
If processes send pull requests independently and uniformly at random, $\P(\uninformed{r+1} \mid \uninformed{r})$ follows a binomial distribution with mean
\begin{align}
  \label{eq:meanpull1}
  \E_\text{pull}[\uninformed{r+1} \mid \uninformed{r}]&=\uninformed{r}\cdot\left(\frac{\uninformed{r}}{n}\right)^{\fanin}
\end{align}
whereas if they select a uniform random sample without replacement among the other $(n-1)$ processes we obtain
\begin{align}
  \label{eq:meanpull2}
  \E_\text{pull}[\uninformed{r+1} \mid \uninformed{r}]&=\uninformed{r}\cdot\frac{{\uninformed{r}\choose\fanin}}{{n-1 \choose \fanin}}=n-\uninformed{r}\frac{\uninformed{r}(\uninformed{r}-1)\dots(\uninformed{r}-\fanin+1)}{n(n-1)\dots(n-\fanin+1)} = n-\uninformed{r} \frac{(\uninformed{r})_{\fanin}}{(n-1)_{\fanin}} 
\end{align}
where $(\boldsymbol{\cdot})_{\boldsymbol{\cdot}} $ is the falling factorial notation.


\begin{lemma}
  \label{lem:pullbetterthanpush}
If $\fanout = \fanin$, then 
    $\E_{\text{pull}}[\uninformed{r+1}|\uninformed{r}] \leq \E_{\text{push}}[\uninformed{r+1}|\uninformed{r}]$.
\end{lemma}

\begin{proof}
We prove the lemma with processes chosen independently and uniformly at random.  Let  $f=\fanin=\fanout$. For the pull version, we saw that
\begin{align}
  \label{eq:meanpull1repeat}
  \E_\text{pull}[\uninformed{r+1} \mid \uninformed{r}]=\uninformed{r}\cdot\left(\frac{\uninformed{r}}{n}\right)^{f}
\end{align}
whereas for the push version we can show that 
  \begin{equation}
    \label{eq:meanpush}
   \E_{\text{push}}[\uninformed{r+1} \mid \uninformed{r}] = \uninformed{r}\left(1 - \frac{1}{n}\right)^{f(n-\uninformed{r})}.
 \end{equation}
 From Eq.~\eqref{eq:meanpull1repeat} and \eqref{eq:meanpush}, it is clear that the lemma holds when $\uninformed{r}=0$, $\uninformed{r}=n-1$, and $\uninformed{r}=n$. For the other values of $\uninformed{r}$, we prove that
 \begin{align}
\label{eq:comp}
      & \left( \frac{\uninformed{r}}{n}\right)^{f} \leq \left( \left(1 - \frac{1}{n}\right)^{n - \uninformed{r}}\right)^{f} 
      \Leftrightarrow \left(\frac{n-1}{n}\right)^{n - \uninformed{r}} - \frac{\uninformed{r}}{n} \geq 0.
 \end{align}
  Let $g(x) \triangleq \left(\frac{n-1}{n}\right)^{n - x} - \frac{x}{n}$. Since $g(0) \geq 0$ and $g(n-1) = 0$, we prove that $g(x) \geq 0$ for every $x \in \{0,1,\dots, n-1\}$ by showing that $g'(x) \leq 0$ over the interval $[0,n-1]$. We have
  \begin{equation}
     \begin{split}
      g'(x) 
        &= - \left( \frac{n-1}{n} \right)^{n-x} \ln \left(\frac{n-1}{n}\right) - \frac{1}{n} \\
        &= \left(\frac{n}{n-1}\right)^x \left( \frac{n-1}{n} \right)^{n} \ln \left(\frac{n}{n-1}\right) - \frac{1}{n} \\
      \end{split}
    \end{equation}
    which is an increasing function with respect to $x$. To complete the proof, we verify that $g'(n-1) \leq 0$:
     \begin{equation}
    \begin{split}
      g'(n-1) & =  \left(\frac{n}{n-1}\right)^{(n-1)} \left( \frac{n-1}{n} \right)^{n} \ln \left(\frac{n}{n-1}\right) - \frac{1}{n} \\
      & \leq \frac{n-1}{n} \left(\frac{n}{n-1} -1\right)  - \frac{1}{n} \\ & = 0.
    \end{split}
  \end{equation}
\end{proof}


 We now bound the expected progression of the regular pull algorithm, and later use it to derive lower bounds on its round complexity.
 
 \begin{lemma}
   \label{lem:pullasymptotic}
$\E_\text{pull}[\informed{r+1}\mid\informed{r}] \leq \informed{r} \cdot(\fanin+1)$.
 \end{lemma}

\begin{proof}
  We prove the lemma with processes chosen from a uniform random sample using Eq.~\eqref{eq:meanpull2}. We fix $n$ and $\uninformed{r}$ and prove the lemma by induction on $\fanin$.

  \paragraph{Basis step.} The lemma is clearly true for $x=\fanin=0$.
  \paragraph{Inductive step.} Let $0 \leq x \leq n-2$ be an integer. We assume that $n-\uninformed{r} \frac{(\uninformed{r})_x}{(n-1)_x} \leq \informed{r}(x+1)$, which is equivalent to
  \begin{align}
    \label{eq:ind1}
     \uninformed{r} \frac{(\uninformed{r})_x}{(n-1)_x} \geq n - \informed{r}(x+1)
  \end{align}
  and must show that
  \begin{align}
    \label{eq:ind2}
    n-\uninformed{r} \frac{(\uninformed{r})_x(\uninformed{r}-x)}{(n-1)_x(n-1-x)} \leq \informed{r}(x+2) \Leftrightarrow  S \triangleq n-\uninformed{r} \frac{(\uninformed{r})_x(\uninformed{r}-x)}{(n-1)_x(n-1-x)} - \informed{r}(x+2) \leq 0.
     \end{align}
Substituting the left side of Eq.~(\ref{eq:ind1}) for its right side in Eq.~(\ref{eq:ind2}), and replacing $\uninformed{r}$ by $n - \informed{r}$, we have

\begin{equation}
  \begin{split}
  S & \leq n- (n - \informed{r}(x+1)) \frac{n-\informed{r}-x}{n-1-x} - \informed{r}(x+2) \\
    & \leq \frac{n(\informed{r}-1) - \informed{r} (x+1) (\informed{r}  - 1)}{n-x-1} - \informed{r} \\ & \leq \informed{r}-1 - \frac{(x+1) (\informed{r}  - 1)^2}{n-x-1} - \informed{r} \\
    & \leq - \frac{(x+1) (\informed{r}  - 1)^2}{n-x-1} \\
  & \leq 0.
  \end{split}
\end{equation}
  
 \end{proof}


 
\begin{lemma}
    \label{lem:loglog}
        If $\fanin\in\mathcal{O}(\ln n)$, the regular pull algorithm starting with $\frac{n}{\ln n}$ informed processes informs all processes with high probability in $\Theta(\log_{\fanin+1} \ln n)$ rounds.
  \end{lemma}

  \begin{proof}
For the lower bound, it is clear from Lemma~\ref{lem:pullasymptotic} that $\Omega(\log_{\fanin+1} \ln n)$ are required to reach all processes on expectation, thus required to inform all processes with high probability. For the upper bound, the proof for $\fanin=1$ consists of the points 3 and 4 in the proof of Theorem 2.1 of Karp et
    al.~\cite{DBLP:conf/focs/KarpSSV00}. We generalize their proof for an arbitrary $\fanin$.

    Recall that $\E_{\text{pull}}[\uninformed{t}\mid\uninformed{t-1}] =  \frac{(\uninformed{t-1})^{\fanin+1}}{n^{\fanin}}$ and that we start with at most $u_0 = n - \frac{n}{\ln n}$ uninformed processes. We use the following Chernoff bound from \cite{mitzenmacher2005probability}: 
    \[ \P(X \geq (1 + \delta) \mu) \leq e^{-\frac{\delta^2 \mu}{3}},\ 0 < \delta < 1. \]
If $u_{t-1} \geq (\ln n)^{\frac{4}{\fanin+1}}n^\frac{\fanin}{\fanin+1}$, it follows that
    \begin{align*}
      \P\left(u_t \geq \left(1 + \frac{1}{\ln n}\right) \frac{(\uninformed{t-1})^{\fanin+1}}{n^{\fanin}}\right) 
        & \leq  e^{- \frac{1}{3} \ln^2 n} \\ & \in o\left(n^{-c}\right) \text{ for any constant $c$}
    \end{align*}
and we can deduce that
    \begin{equation}
      \label{eq:uninformed-whp}
      \uninformed{t} \leq \left(1 + \frac{1}{\ln n}\right) \frac{(\uninformed{t-1})^{\fanin+1}}{n^{\fanin}}
    \end{equation}
    with high probability. Applying Eq. (\ref{eq:uninformed-whp}) recursively, we obtain
    \begin{align}
      \uninformed{t} & \leq (\uninformed{0})^{{(\fanin+1)^t}}\left(\frac{1 + \frac{1}{\ln n}}{n^{\fanin}}\right)^\frac{(\fanin+1)^t-1}{\fanin} 
    \end{align}
Replacing $u_0$ by $n - \frac{n }{\ln n}$, and $t$ by $4 \log_{\fanin+1}\ln n$ we obtain 
\begin{equation}
  \begin{split}
   \uninformed{t} & \leq \left(n-\frac{n}{\ln n}\right)^{{(\fanin+1)^t}}\left(\frac{1 + \frac{1}{\ln n}}{n^{\fanin}}\right)^\frac{(\fanin+1)^t-1}{\fanin} \\ & \leq n \left(1-\frac{1}{\ln n}\right)^{\ln^4 n} \left(1+\frac{1}{\ln n}\right)^{\ln^4 n} \\ & \leq n \left(1-\frac{1}{\ln^2 n}\right)^{\ln^4 n} \\ & \in o(1)
 \end{split}
\end{equation}
which shows that we need $O(\log_{\fanin+1} \ln n)$ rounds to reach the point where there are at most $(\ln n)^{\frac{4}{\fanin+1}}n^\frac{\fanin}{\fanin+1}$ uninformed processes with high probability. Note that this step is unnecessary if $\fanin$ is large enough with respect to $n$ since $(\ln n)^{\frac{4}{\fanin+1}}n^\frac{\fanin}{\fanin+1} \geq n-\frac{n}{\ln n}$.

At this stage, the probability that an uninformed process remains uninformed after each subsequent round is at most 
    \begin{align}
      \label{eq:21}
      \left(\frac{\uninformed{r}}{n}\right)^{\fanin} & \leq \left(\frac{(\ln n)^{\frac{4}{\fanin+1}}n^\frac{\fanin}{\fanin+1}}{n}\right)^{\fanin} \leq \frac{(\ln n)^4}{\sqrt{n}}.
    \end{align}
    Hence after a constant number of additional rounds, we inform every remaining uninformed process with high probability. 

  \end{proof}

% ----------------------------------------------------------------------------------------------

  \begin{corollary}
 If $\fanin\in\Omega(\ln n)$, the regular pull algorithm starting with $\frac{n}{\ln n}$ informed processes informs all processes with high probability in $\Theta(1)$ rounds.
  \end{corollary}

  
% ----------------------------------------------------------------------------------------------

\begin{theorem}
  \label{thm:pull}
    The regular pull algorithm disseminates a rumor to all processes with high probability in $\Theta(\log_{\fanin + 1} n)$ rounds of communication.
\end{theorem}

\begin{proof}
  For the lower bound, it is clear from Lemma~\ref{lem:pullasymptotic} that $\Omega(\log_{\fanin+1} n)$ rounds are required in expectation to inform all processes, and thus necessary to inform all processes with high probability.

  We now show that $\mathcal{O}(\log_{\fanin + 1} n)$ rounds suffice when
  $\fanin \in \mathcal{O}(\ln n)$ (the statement for $\fanin = 1$ is implicitly discussed without proof in \cite{DBLP:conf/focs/KarpSSV00}). 

  In a first phase, we show that $\mathcal{O}(\log_{\fanin+1}n)$ rounds are sufficient to inform $\ln n$ processes with high probability. Let $c_0 \geq 1$ be a constant. In this case, we show that for stages $k \in \{0,1,2,\dots,\ln\ln n\}$, if $\informed{r} = 2^k$ processes are informed, then after $\rho_k \triangleq c_0\left\lceil \frac{\log_{\fanin + 1} n}{2^k} \right \rceil$ rounds, the number of informed processes doubles with high probability, i.e., $\informed{r+\rho_k} \geq 2^{k+1}$ with high probability. At every round of stage $k$, each pull request has a probability at least $\frac{2^k}{n}$ of reaching an informed process, thus after $\rho_k$ rounds and $\rho_k \cdot \fanin$ pull requests, the probability that an uninformed process learns the rumor is bounded by
  \begin{align}
  p \geq 1 - \left( 1 - \frac{2^k}{n} \right)^{\rho_k \cdot \fanin} \geq \frac{2^k\rho_k\fanin}{n}-\frac{2^{2k}\rho_k^2\fanin^2}{n^2}.    
  \end{align}
  The probability $T$ to inform $l=\informed{r}=2^k$ processes or less in stage $k$ is upper bounded by the left tail of the binomial distribution with parameters $p$ and $N = \uninformed{r} = n-2^k$. We can bound this tail using the Chernoff bound
  \begin{align}
    \label{eq:chernoffbinomial}
       T & \leq \exp\left(-\frac{(Np-l)^2}{2Np}\right)
  \end{align}
  which is valid when $l \leq N p$. We can indeed apply this bound by showing that $Np \geq \frac{c_0 \fanin}{\ln(\fanin + 1)} \ln n + o(1)$, which is greater than $2^k$ when $c_0 \geq 1$. The Chernoff bound gives
  \begin{equation}
       \begin{split}
     \label{eq:3}
      T 
     & \leq \exp\left(-\frac{Np}{2}+l\right) \\
   %  \\
        & \leq \exp \left(-\frac{(n-2^k)p}{2} + 2^{k}\right) \\     
    & \leq \exp\left(\left(1-\frac{c_0\fanin}{2\ln(\fanin+1)}\right) \ln n +o(1)\right) \\
  & \in \mathcal{O}\left(n^{  1-\frac{c_0\fanin}{2\ln(\fanin+1)}}  \right)
\end{split}
\end{equation}
and for any constant $c>0$ we can find $c_0$ such that $T \in \mathcal{O}\left(n^{-c}\right)$. This first phase, with the $k$ stages, requires $\sum\limits_{k=0}^{\ln\ln n} \rho_k \leq c_0 \log_{\fanin + 1} n \cdot \sum\limits_{k=0}^{\ln\ln n} 2^{-k} + c_0(\ln\ln n + 1) \sim 2c_0 \log_{\fanin + 1} n$ rounds of communication to inform $1 + 2^0 + 2^1 + \dots + 2^{\ln\ln n} \approx 2 \ln n$ processes with high probability. 

In a second phase, when $\ln n \leq \informed{r} \leq \frac{n}{(\ln n)^2}$, we show that a constant number of rounds $c_1$ is sufficient to multiply the number of informed processes by $\fanin+1$ with high probability. We use the Chernoff bound of Eq.~\eqref{eq:chernoffbinomial} with $l=\fanin \cdot\informed{r}$, $n-\frac{n}{\ln n}\leq N \leq n-\ln n$ and $p \geq 1-\left(1-\frac{\informed{r}}{n}\right)^{c_1 \fanin} \geq \frac{\informed{r}c_1\fanin}{n} -\frac{\informed{r}^2 c_1^2 \fanin^2}{2n^2}.$ We obtain
\begin{equation}
  \begin{split}
  T & \leq \exp\left(-\frac{Np}{2}+l\right) \\
    & \leq \exp \left( -\frac{\informed{r}c_1\fanin}{2}\left(1-o(1)\right)+\informed{r}\fanin\right) \\
    & \leq \exp \left(\ln n\left(1-\frac{c_1}{2}+o(1)\right) \right) \\
    & \in \mathcal{O}\left(n^{  1-\frac{c_1}{2}}  \right)
  \end{split}
\end{equation}
and for any constant $c>0$ we can find $c_1$ such that $T \in \mathcal{O}\left(n^{-c}\right)$.  This second phase requires $\mathcal{O}(\log_{\fanin + 1} n)$ rounds of communication.

In a third phase, we can go from $\frac{n}{(\ln n)^2}$ to $\frac{n}{\ln n}$ informed processes in $\mathcal{O}(\log_{\fanin + 1} n)$ rounds of communication since multiplying the number of informed processes by $\ln n$ at this stage cannot be slower than during the first phase.  Finally, in a fourth phase we saw in Lemma~\ref{lem:loglog} that we can go from $\frac{n}{\ln n}$ to $n$ informed processes with high probability with $\Theta\left(\log_{\fanin+1}\ln n\right)$ rounds of communication.

We now summarize the proof of the upper bound when $\fanin\in \omega(\ln n)$ and $\fanin\in\mathcal{O}(n)$. The different cases must me handled with care, but we omit the details for simplicity purposes. In a first phase, we show that  $\mathcal{O}(\log_{\fanin + 1} n)$  rounds are sufficient to inform $\ln n$ processes with high probability. In a second phase, if $\fanin \cdot \informed{r} \in o(n)$, we apply the Chernoff bound of Eq.~\eqref{eq:chernoffbinomial} during $\mathcal{O}(\log_{\fanin + 1} n)$  rounds to reach either $\frac{n}{\ln n}$ informed process with high probability, or $\fanin \cdot \informed{r} \in \Theta(n)$ (the Chernoff bound must be changed when $\fanin \cdot \informed{r} \in \Theta(n)$). If $\fanin \cdot i \in \Theta(n)$, we again apply Eq.~\eqref{eq:chernoffbinomial} during a constant number of rounds to reach $c_2 \cdot n$ informed processes with $c_2<1$ with high probability. Finally, in a last phase, we go from $c_2 \cdot  n$ or $\frac{n}{\ln n}$ to $n$ informed processes with high probability using Lemma~\ref{lem:loglog}.

\end{proof}

% ----------------------------------------------------------------------------------------------

\begin{corollary}
  If $\fanin \in \mathcal{O}(1)$, then the total number of messages (replies to pull requests) required by the regular pull algorithm is in $\Theta(n)$. In particular, the communication overhead is 0 when $\fanin=1$.
\end{corollary}

\begin{proof}
  It is clear that a process cannot pull a rumor more than $\fanin$ times since it stops requesting it in the rounds that follow its reception.
\end{proof}

% ----------------------------------------------------------------------------------------------

We now prove that the round complexity of the regular pull algorithm is asymptotically optimal for the generalized random phone call model.

\begin{theorem}
  \label{thm:pushpullround}
If $f=\fanin=\fanout$, any protocol in the generalized random phone call model requires $\Omega(\log_{f+1} n)$ rounds of communication to disseminate a rumor to all processes with high probability.
\end{theorem}

\begin{proof}
  % The bit complexity and the message complexity of the regular pull algorithm are trivially optimal. For the round complexity,
Let $f=\max(\fanin,\fanout)$. 
If we only push messages, it is clear that the number of informed processes increases at most by a factor of $(\fanout + 1)$ per round. If we only pull messages, we saw in Lemma~\ref{lem:pullasymptotic}
  that the number of informed processes increases at most by a factor of $(\fanin + 1)$ per round in
  expectation. If all processes simultaneously push and pull at every round, the number of informed processes
  increases at most by a factor of $(\fanin + 1)(\fanout + 1)$ per round in expectation, thus the number or
  rounds required to informed all processes is at least
  $\log_{(\fanout+1)(\fanin+1)} n \geq \log_{(f+1)^2} n \in \Omega(\log_{f+1} n)$.

\end{proof}

% ----------------------------------------------------------------------------------------------

We now show that the regular pull algorithm is robust against adversarial and stochastic failures.  First, consider an adversary that fails $\epsilon \cdot n$ processes for $0 \leq \epsilon < 1$, excluding the process starting the rumor. Before the execution of the algorithm, the adversary decides which processes fail, and for each failed process during which round it fails. Once a process fails, it stops participating until the end of the execution, although it may still be uselessly called by active processes. We also consider stochastic failures, in the sense that each phone call fails with probability $\delta$ for $0 \leq \delta < 1$. Note that both types of failures are independent of the execution.%\mm{i do not understand the non byzantine model and independence. A non byzantine adversary could simply stop processes based on the state of the execution}. 

The main difference introduced by the failures is that we can no longer go from $\frac{n}{\ln n}$ to $n$ informed processes in $\mathcal{O}(\log_{f+1} \ln n)$ rounds because there is a non-vanishing probability that pull requests either target failed processes or result in failed phone calls. We nevertheless show that the regular pull algorithm can disseminate a rumor to all $(1-\epsilon)n$ good (i.e., non-failed) processes with high probability with the same asymptotic round complexity. 

\begin{theorem}
  \label{thm:failures}
  Let $0 \leq \epsilon < 1$, and let $0 \leq \delta < 1$. If $\epsilon \cdot n$ processes, excluding the initial process with the rumor, fail adversarially, and if phone calls fail with probability $\delta$, then the regular pull algorithm still disseminates a rumor to all $(1-\epsilon)n$ good processes with high probability in $\Theta(\log_{\fanin + 1} n)$ rounds of communication.
\end{theorem}

\begin{proof}
  It is clear that the lower bound remains valid when there are failures. We prove the upper bound for $\fanin\in\mathcal{O}(\ln n)$, but as we mentioned for Theorem~\ref{thm:pull} we can adapt the proof for $\fanin\in\omega(\ln n)$ by carefully applying Chernoff bounds in different phases.

Note that the earlier a process fails, the more damage it causes. We thus assume that the $\epsilon \cdot n$ processes fail at the beginning of the execution, which is the worst possible scenario. We can use the first three phases of the proof of Theorem \ref{thm:pull} with minor modifications (only multiplicative constants change) and prove that $\mathcal{O}(\log_{\fanin + 1} n)$ rounds are sufficient to go from 1 to $\frac{n}{\ln n}$ informed processes with high probability. 

We now show that we need $c_2\log_{\fanin + 1} n$ rounds to go from $\frac{n}{\ln n}$ to $c_1 \cdot n$ informed processes with high probability for some arbitrary $c_1 < 1-\epsilon$. We again use the Chernoff bound of Eq.~\eqref{eq:chernoffbinomial} with $\informed{r}=\frac{n}{\ln n}$, $l = c_1 \cdot n$ and $N = (1-\epsilon) n - \frac{n}{\ln n}$. If $c_2$ is a large enough constant, the probability that a process learns a rumor during that phase is
  \begin{align}
  p \geq 1-\left(1-\frac{(1-\delta)\informed{r}}{n}\right)^{\fanin c_2\log_{\fanin + 1} n} \geq 1-\left(1-\frac{1-\delta}{\ln n}\right)^{\frac{\fanin c_2\ln n}{\ln(\fanin+1)}} \geq 1-e^{-c_2(1-\delta)} \triangleq c_3.
  \end{align}
The Chernoff bound gives
\begin{equation}
 \begin{split}
    T & \leq \exp\left(-\frac{Np}{2}+l\right) \\
      & \leq \exp \left( -c_3n\left(1-\epsilon-\frac{1}{\ln n}\right)+c_1 n \right)\\
     & \leq \exp \left( n\left(-c_3+c_3\epsilon+c_1+o(1) \right)\right)\\
%    & \leq \exp \left( -\frac{\informed{r}c_1\fanin}{2}\left(1-o(1)\right)+\informed{r}\fanin\right) \\
%    & \leq \exp \left(\ln n\left(1-\frac{c_1}{2}+o(1)\right) \right) \\
 %   & \in \mathcal{O}\left(n^{  1-\frac{c_1}{2}}  \right)
   \end{split}
 \end{equation}
and we can choose $c_2$ such that $T \leq e^{c_4 n}$ with $c_4 < 0$. This guarantees $T \in \mathcal{O}\left(n^{-c}\right)$ for any~$c>0$.

Starting from $c_1 n$ informed processes, the probability that a process is informed in any subsequent round is bounded by $p \geq 1-\left(\frac{n-c_1(1-\delta)n}{n}\right)^{\fanin} \geq 1 - (1-c_1(1-\delta))^{\fanin}$.  After $r$ such rounds, the probability that a process remains uninformed is thus upper bounded by $(1-c_1(1-\delta))^{\fanin r}$, and for this probability to be bounded by $n^{-c}$ we need
  \begin{align}
    (1-c_1(1-\delta))^{\fanin r} \leq n^{-c} \Leftrightarrow
    r \geq  \frac{c \ln n}{\ln{\frac{1}{1-c_1(1-\delta)}}\fanin} \geq c_4 \log_{f+1} n \text{ for some constant $c_4$.}
  \end{align}
Hence, $\mathcal{O}(\log_{f+1} n)$ rounds are sufficient to go from $c_1n$ to $(1-\epsilon) n$ informed processes with high probability.

\end{proof}

Note that adversarial and stochastic failures do not increase the message complexity of the regular pull algorithm: uninformed processes that fail decrease the number of rumor transmissions, and failed phone calls do not exchange the rumor. We could, however, consider that messages containing the rumor are dropped with probability $0 \leq \gamma < 1$. Theorem~\ref{thm:failures} also holds in this instance, but the number of messages increases by an unavoidable factor of $\frac{1}{1-\gamma}$.

% ----------------------------------------------------------------------------------------------

% ----------------------------------------------------------------------------------------------

% ----------------------------------------------------------------------------------------------

%%% Local Variables:
%%% mode: latex
%%% TeX-master: "main"
%%% End:

In this paper, 2D and 3D CNN models were used to generate pelvic sCTs from T1-weighted MR images. Our sCT generation methods were fully automated, requiring no deformable registration or manual segmentation of bone tissues. As shown in Figure~\ref{fig3}, the 2D and 3D CNN models generated high quality sCTs. MAE curves shown in Figure~\ref{fig4} indicated that both models could precisely estimate soft-tissue HU values but had difficulty in reproducing air and high-density bone tissues. 

The MAEs within the body contour across all patients were 40.5 $\pm$ 5.4 HU and 37.6 $\pm$ 5.1 HU for the 2D and 3D models, respectively. The time required for generating a pelvic sCT using our CNN models was about 5.5 s. Our MAE results are comparable to previous studies. Kim $et \ al.$\cite{RN41} presented a voxel-based weighted summation method that produced an MAE of 74.3 $\pm$ 3.9 HU. However, manual contouring of bone tissues required for this method can be tedious and time-consuming. An MAE of 40.5 $\pm$ 8.2 HU was achieved by Dowling $et \ al.$\cite{RN11} using an average MRI-CT atlas from 38 patients. Andreasen $et \ al.$\cite{RN42} reported an MAE of 54 $\pm$ 8 HU using an atlas-based method with pattern recognition, and its prediction time was about 20.8 min. Another random forest model proposed by Andreasen $et \ al.$\cite{RN43} generated sCTs with an MAE of 58 $pm$ 9 HU. A hybrid method suggested by Siversson $et \ al.$ \cite{RN45} obtained an MAE of 36.5 $\pm$ 4.1 HU when ignoring errors introduced by gas cavities. This hybrid method was implemented in the cloud-based commercial software MriPlanner (Spectronic Medical AB, Helsingborg, Sweden), which required 50 to 80 min to generate a sCT.\cite{RN45} The patch-based 3D context-aware generative adversarial network presented by Nie $et \ al.$\cite{RN26} achieved an MAE of 39.0 $\pm$ 4.6 HU. 

Our CNN models reproduced low-density bone as shown in Figure ~\ref{fig4}. The bone-region DSCs were 0.81 $\pm$ 0.04 and 0.82 $\pm$ 0.04 from the 2D and 3D models, respectively. These results are comparable to reported DSC results of 0.79 $\pm$ 0.12\cite{RN10} and 0.91$\pm$0.03{\cite{RN11}}, where the authors compared bone contours manually drawn on the sCT and CT.

It was feasible to train the proposed 3D model with 16 image volumes from scratch. Results of the Wilcoxon signed-rank tests shown in Table~\ref{tab1} demonstrated a statistically significant improvement in overall MAE, bone DSC, and bone precision of the 3D model compared to the 2D model. However, as shown in Figure~\ref{fig4}, the 2D model seemed to perform better in estimating the high-density bone HU values. It should be noted that smaller overall MAEs do not guarantee improved sCT dose calculation and patient positioning performance. While the models performed well, we will continue to acquire more patient data to potentially improve model accuracy and further test model differences.

As this was a retrospective study, the MR image voxel sizes were not matched, resulting in different voxel intensities between images. This may have affected the sCT generation accuracy although we applied intensity normalization. A potential study could examine how voxel size variations affects sCT estimation. 

The proposed 3D model can be implemented on a 12 GB GPU to process volumetric images with dimensions of 256 $\times$ 256 $\times$ 30. More GPU memory would be required to process higher resolution 3D images. Considering the limited access to multi-GPU systems, a 3D architecture with fewer convolutional layers could be considered to deal with higher resolutions. However, the performance could be affected by the reduced parameters and smaller receptive fields of the less complex model. Another approach would be to extract 30-slice sub-volumes from CT and MR images for training the 3D model. The sCT could then be generated by averaging 30-slice sCT sub-volumes produced by the model. 

A number of techniques could be investigated for improving model performance.  Nie $et \ al.$\cite{RN26} showed that introducing an additional adversarial discriminator improved overall sCT quality. The same approach could be adapted in our proposed 2D and 3D CNN models.  Non-rigid deformation\cite{RN44} could also be applied to both CT and MR images in the process of the on-the-fly data augmentation to produce more training pairs. Multiple MR images acquired with different sequences could be fed into models to provide more information for distinguishing different tissues. Multi-GPU systems with more memory would enable the exploration of larger batch sizes for training CNN models, which could reduce variances in gradient estimation and accelerate the training. 



% Bibliography
% \printbibliography
\bibliographystyle{unsrtnat}
\bibliography{bib.bib}

\pagebreak


\end{document}

%%% Local Variables:
%%% mode: latex
%%% TeX-master: t 
%%% End:
