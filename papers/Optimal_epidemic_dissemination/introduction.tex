\section{Introduction}
\label{sec:introduction}

We consider the problem of reliable epidemic/gossip dissemination of a rumor in a fully connected network of $n$ processes using address-oblivious algorithms. In this class of algorithms, the local decisions taken locally by each process are oblivious to the addresses of the other processes. Besides dissemination~\cite{bimodal}, epidemic/gossip-based algorithms have been proposed to address a wide variety of problems such as replicated database maintenance~\cite{Demers:1987:EAR:41840.41841}, failure detection~\cite{vanRenesse:1998:GFD:1659232.1659238},
aggregation~\cite{Kempe:2003:GCA:946243.946317}, code propagation and maintenance~\cite{Levis:2004:TSA:1251175.1251177}, modeling of computer virus propagation~\cite{Berger:2005:SVI:1070432.1070475}, membership~\cite{Jelasity2007},  publish-subscribe~\cite{10.1109/TPDS.2013.6}, total ordering~\cite{matos2015epto}, and numerous distributed signal processing tasks~\cite{journals/pieee/DimakisKMRS10}. The randomness inherent to the selection of the communication peers makes epidemic algorithms particularly robust to all kinds of failures such as message loss and process failures, which tend to be the norm rather than the exception in large systems. Their appeal also stems from their simplicity and highly distributed nature. 
The amount of work studying theoretical models of epidemic dissemination is vast and mainly focuses on establishing bounds for different dissemination models, which we briefly describe below.

\paragraph{\textbf{Push algorithms.}}

The simplest epidemic dissemination algorithms are push-based, where processes that know the rumor propagate
it to other processes.  Consider the following ``infect forever'' push algorithm first introduced by Frieze and Grimmett~\cite{DBLP:journals/dam/FriezeG85}. The algorithm starts with a
single process knowing a rumor, and at every round, every informed process chooses $\fanout$ processes uniformly at random and forwards the rumor to them. Pittel~\cite{Pittel:1987:SR:37387.37400} showed that for a network of size $n$,
$\log_{\fanout+1}n+\frac{1}{\fanout}\ln n + \mathcal{O}(1)$ rounds of communication are necessary and
sufficient in probability for every process to learn the rumor.
There are other flavors of push algorithms~\cite{EGMM2004,koldehofe2004simple}, although in all cases,
reaching the last few uninformed processes becomes increasingly costly as most messages are sent to processes already
informed.  Push algorithms must transmit $\Theta(n \ln n)$ messages if every process is to learn a rumor with high probability\footnote{With high probability (w.h.p)
  means with probability at least $1-\mathcal{O}\left(n^{-c}\right)$ for a constant $c>0$.}.

\paragraph{\textbf{Pull algorithms}}

Instead of pushing a rumor, a different strategy is for an uninformed process to ask another process chosen at
random to convey the rumor if it is already in its possession. Pulling rumors was first proposed and studied
by Demers et al.~\cite{Demers:1987:EAR:41840.41841}, and further studied by Karp et
al.~\cite{DBLP:conf/focs/KarpSSV00}. 
Pulling algorithms are advantageous when rumors are frequently created because pull requests will more often than not reach processes with new rumors to share. However, issuing pull requests in systems with little activity result in useless traffic. 

\paragraph{\textbf{Push-pull algorithms and the (polite) random phone call model}}

The idea to push and pull rumors simultaneously was first considered by Demers et al.~\cite{Demers:1987:EAR:41840.41841}, and further studied in the seminal work of Karp et al.~\cite{DBLP:conf/focs/KarpSSV00} who considered the following random phone call model. At each round, each process randomly chooses an interlocutor and calls it.  If, say, Alice calls Bob, Alice pushes the rumor to Bob if she has it, and pulls the rumor from Bob if he has it. Establishing communication (the phone call itself) is free, and only messages that include the rumor are counted. It is paramount to note that in the original work~\cite{DBLP:conf/focs/KarpSSV00}, processes do not have to share the rumor once the communication is established, although it is implicitly assumed that they always do in the analysis of their lower bounds. We thus define the \emph{polite random phone call} model as it is used in the analysis of \cite{DBLP:conf/focs/KarpSSV00}, i.e., assuming that processes always share the rumor.
We generalize this model, including the right not to share the rumor, in the next section.

Using the polite random phone call model, Karp et al.~\cite{DBLP:conf/focs/KarpSSV00} presented an algorithm that transmits a rumor to every process with high probability using $\mathcal{O}(\ln n)$ rounds of communication and $\mathcal{O}(n \ln \ln n)$ messages. The idea is that the number of informed processes increases exponentially at each round until approximately $\frac{n}{2}$ processes are informed due to the push operations, after which the number of uninformed processes shrinks quadratically at each round due to the pull operations. The authors also prove that any algorithm in the polite random phone call model running in $\mathcal{O}(\ln n)$ rounds with communication peers chosen uniformly at random requires at least $\omega(n)$ messages, and that any address-oblivious algorithm needs $\Omega(n \ln \ln n)$ messages to disseminate a rumor regardless of the number of communication rounds. Even though these lower bounds are valid in this polite random phone call model, the authors imply that they are valid in the more general model that they defined, which is false. We break both lower bounds in this article.

The work of Karp et al.~\cite{DBLP:conf/focs/KarpSSV00} is widely cited. Their push-pull algorithm is leveraged as a primitive block in numerous settings, but more worrisome, their lower bounds are wrongly used as fundamental limits of epidemic dissemination algorithms, which sometimes lead to cascaded errors. A relevant example here is the work of Fraigniaud and Giakkoupis~\cite{DBLP:conf/spaa/FraigniaudG10} on the total number of bits exchanged in the random phone call model. The authors presented a push-pull algorithm with concise feedback that requires $\mathcal{O}(\ln n)$ rounds and $\mathcal{O}(n(b + \ln \ln n \ln b))$ bits to disseminate a rumor of size $b$, as well as a lower bound of  $\Omega(nb + n \ln \ln n))$ bits when the number of rounds is in $\mathcal{O}(n)$. They proved the $nb$ term of the lower bound for $n \in \omega(\ln \ln n)$, but relied on the false $\Omega(n \ln \ln n)$ bound of ~\cite{DBLP:conf/focs/KarpSSV00} for the other term. Their correct lower bound is therefore $\Omega(nb)$, and only valid for $n \in \omega(\ln \ln n)$. 

\subsection{Our contributions}

\paragraph{\textbf{Generalized (impolite) random phone call model.}}

In the proofs of the original random phone call model, rumors are transmitted in both directions whenever both players on the line have the rumor. Our generalized model removes this restriction, and also allows multiple push and pull phone calls per round. Let $\fanout \geq 1$ and $\fanin \geq 1$. At each communication round, each process: i) calls between 0 and $\fanin$ processes uniformly at random to request a rumor, ii) calls between 0 and $\fanout$ processes uniformly at random to push a rumor, and iii) has the option not to answer pull requests. To keep the phone call analogy, our generalized model allows impolite parties: each player can call multiple players, refuse to reply to pull requests, refuse to push a rumor, and refuse to request a rumor at any given round. 

We assume, like for the original model, that establishing the communication is free, and we only count the number of messages that contain the rumor. The practical rationale behind this assumption is that the cost of establishing the communication is negligible if the rumor is large or if there are multiple rumors that can be transmitted in a single communication. 
We also assume that the network is a complete graph, 
that the rounds are synchronous, and that processes can reply to pull requests in the same round. Finally, we assume that a single process has a rumor to share at the start of the dissemination process\footnote{We handle multiple rumors over a long period of time in Section~\ref{sec:multiplerumors}.}. 

We define three regular algorithms, all defined to halt after an agreed upon number of dissemination rounds. 
In the \emph{regular pull algorithm} uninformed processes send exactly $\fanin$ pull requests per round, whereas informed processes never push, never send pull requests but always reply to pull requests.  In the \emph{regular push algorithm} informed processes push the rumor to exactly $\fanout$ processes per round, whereas uninformed processes never send pull requests. Finally, the \emph{regular push-then-pull algorithm} consists of a regular push algorithm followed by a regular pull algorithm. Note that the best protocols for the generalized random phone call model are strikingly simple and do not require, for instance, to define a complicated probability distribution that determines who replies to what: we prove that the regular pull algorithm and the regular push-then-pull algorithm are asymptotically optimal.

\paragraph{\textbf{Breaking the lower bounds from~\cite{DBLP:conf/focs/KarpSSV00}.}}

The confusion from the lower bounds of Karp et al.~\cite{DBLP:conf/focs/KarpSSV00} stems from the fact that their model definition allows impolite behavior, but the proofs of their lower bounds implicitly assume that processes always behave politely. More precisely, one the one hand, (1) they define the model such that processes do not have to share the rumor once the communication is established: ``Whenever a connection is established between two players, each one of them (if holding the rumor) has to decide whether to transmit the rumor to the other player, typically without knowing whether this player has received the rumor already.'' and (2) state their lower bounds as such: ``[...] any address-oblivious algorithm [...] needs to send $\Omega(n \ln \ln n)$ messages for each rumor regardless of the number of rounds. Furthermore, we give a general lower bound showing that time- and communication-optimality cannot be achieved simultaneously using random phone calls, that is, every algorithm that distributes a rumor
in $\mathcal{O}(\ln n)$ rounds needs $\omega(n)$ transmissions.'' On the other hand, in the proofs of their lower bounds in Theorems 4.1 and 5.1 it is implicitly assumed that processes always pull and push the rumor each time a communication is established. This is not optimal and allows us to break both lower bounds. The idea that selectively not replying and not pushing might be beneficial is never discussed.

\paragraph{\textbf{Optimal algorithms with  $\mathcal{O}(\ln n)$ rounds and $n+o(n)$ messages of size $b$}}

If we discount the cost of establishing the communication (the phone call), it is natural to let processes choose whether or not to call, and whether or not to reply when called. This generalization makes a huge difference: we show that the regular pull and push-then-pull algorithms disseminate a rumor of size $b$ to all processes with high probability in $\mathcal{O}(\ln n)$ rounds of communication using only $n+o(n)$ messages of size $b$. The idea is simple: we do not push old rumors because doing so results in a large communication overhead. 

Consider the regular pull algorithm. We prove that this algorithm requires $\Theta(\log_{\fanin+1} n)$ rounds of communication, $n-1$ messages of size $b$ when $\fanin=1$, and $\mathcal{O}(n)$ messages if $\fanin \in \mathcal{O}(1)$. This algorithm is optimal for the generalized phone call model. First, its message complexity is optimal since any algorithm requires at least $n-1$ messages. Second, its bit complexity is optimal for $b\in\omega(\ln \ln n)$ from the (corrected) $\Omega(nb)$ lower bound of Fraigniaud and Giakkoupis~\cite{DBLP:conf/spaa/FraigniaudG10}. 
Third, if $f=\fanin = \fanout$, we prove that its round complexity is asymptotically optimal by showing that pushing and pulling at the same time using potentially complex rules is unnecessary: any algorithm in the generalized random phone call model requires $\Omega(\log_{f+1} n)$ rounds of communication to disseminate a rumor with high probability.

Despite its utter simplicity, the regular pull algorithm exhibits strong robustness against adversarial and stochastic failures. Let $\delta$ be the probability that a phone call fails, and let $\epsilon \cdot n$ be a set of processes, excluding the process initiating the rumor, initially chosen by an adversary to fail at any point during the execution of the algorithm. We prove that for any $0 \leq \epsilon < 1$ and  $0 \leq \delta < 1$, $\mathcal{O}(\log_{\fanin+1} n)$ rounds of communication remain sufficient to inform all processes that do not fail with high probability. The number of transmitted messages when failures occur remains asymptotically optimal.

Although pushing is never required asymptotically, in practice the best approach is to push when the rumor is young until the expected communication overhead reaches an agreed upon threshold, and then pull until all processes learn the rumor with the desired probability. The regular  push-then-pull algorithm is thus asymptotically optimal when $\fanin \in \mathcal{O}(1)$ as long as the number of messages transmitted during the push phase is in $\mathcal{O}(n)$.

We also prove that when $b \in \omega(\ln n \ln \ln n)$, the regular pull and push-then-pull algorithms can be modified to handle multiple and possibly concurrent rumors over a long period of time with $nb + o(nb)$ bits of communication per rumor. This is optimal as it matches the $\Omega(nb)$ lower bound of~\cite{DBLP:conf/spaa/FraigniaudG10}. 


The rest of this article is organized as follows. We present related work in Section~\ref{sec:epidemics}, followed by an analysis of pull algorithms in Section~\ref{sec:pull}.
We discuss push--pull algorithms in Section~\ref{sec:pushthenpull} and handle multiple rumors in Section~\ref{sec:multiplerumors}. 


%%% local Variables:
%%% mode: latex
%%% TeX-master: "./main.tex"
%%% End:
