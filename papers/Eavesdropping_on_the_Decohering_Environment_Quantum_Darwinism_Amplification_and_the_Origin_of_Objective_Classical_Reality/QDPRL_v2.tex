\documentclass[aps,prl,showpacs,amsmath,amssymb,amsfonts,lengthcheck,twocolumn,longbibliography,superscriptaddress]{revtex4-2}
%%%%{revtex4-2}

\usepackage{graphicx}
\usepackage{dsfont}
\usepackage{subfigure}
\usepackage{verbatim}
\usepackage{dcolumn}% Align table columns on decimal point
\usepackage{bm}% bold math
\usepackage{epsf}
\usepackage{color}
\usepackage[toc]{appendix}
\usepackage{stmaryrd}
\usepackage{amsthm}
\usepackage{hhline}
\usepackage{amssymb}
\usepackage{mwe,tikz}\usepackage[percent]{overpic}
\usepackage{tikz}
\usetikzlibrary{arrows,shapes,calc,matrix}

\expandafter\let\csname equation*\endcsname\relax
\expandafter\let\csname endequation*\endcsname\relax
\usepackage{amsmath}
\usepackage{xstring}

\newcommand{\mymacro}[1]{%
	\StrRight{#1}{1}[\lastletter]%
\lastletter
}

\makeatletter
\newcommand\footnoteref[1]{\protected@xdef\@thefnmark{\ref{#1}}\@footnotemark}
\makeatother

\newcommand\myeqref[1]{
	Eq. (\textup{\ref{#1}})
}
\newcommand\invisiblesection[1]{%
	\refstepcounter{section}%
	\addcontentsline{toc}{section}{\protect\numberline{\thesection}#1}%
	\sectionmark{#1}}
%%%%\newcommand{\bra}[1]{\left\langle #1\right|}
%%%%\newcommand{\ket}[1]{\left|#1\right\rangle}
\newcommand{\braket}[2]{\left\langle #1|#2\right\rangle}
\newcommand{\ketbrad}[1]{|#1\rangle\!\langle #1|}
\newcommand{\trace}[1]{\mathrm{tr}\left\{#1\right\}}
\newcommand{\ptr}[2]{\mathrm{tr_{#1}}\left\{#2\right\}}
\newcommand{\thickbar}[1]{\mathbf{\bar{\text{$#1$}}}}

%%%Wojciech's commands start
\def\II{1\!\mathrm{l}}
\newcommand{\Tr}        {\mathrm{Tr}}
%\newcommand{\Id}        {I}
\newcommand{\bra}[1]    {\langle #1|}
\newcommand{\ket}[1]    {| #1 \rangle}
\newcommand{\bk}[2]     {\langle #1 | #2 \rangle}
\newcommand{\kb}[2]     {| #1 \rangle \! \langle #2 |}
\newcommand{\cH}        {{\mathcal H}}
\newcommand{\cS}        {{\mathcal S}}
\newcommand{\cA}        {{\mathcal A}}
\newcommand{\cE}        {{\mathcal E}}
\newcommand{\cN}        {{\mathcal N}}
\newcommand{\cO}        {{\mathcal O}}
\newcommand{\cI}        {{\mathcal I}}
\newcommand{\cJ}        {{\mathcal J}}
\newcommand{\+}         {\dagger}
\newcommand{\eend}      {\hspace{\stretch{1}}\rule{1ex}{1ex}}
\def\bA{{\bf A}}
\def\bB{{\bf B}}
\def\bC{{\bf C}}
\def\bX{{\bf X}}
\def\bY{{\bf Y}}
\def\bZ{{\bf Z}}
\def\bH{{\bf H}}
\newcommand\cF{{\mathcal F}}
\newcommand\hocom[1]{}%\marginpar{\center \small $\triangleright$HO}}

\newcommand{\ba}{\begin{eqnarray}}
\newcommand{\ea}{\end{eqnarray}}
\newcommand{\bmath}{\begin{mathletters}}
\newcommand{\emath}{\end{mathletters}}
\newcommand{\ban}{\begin{eqnarray*}}
\newcommand{\ean}{\end{eqnarray*}}
\newcommand{\tl}{\tilde{\ell}}

%%%Wojciech's commands end

%%%to align chi
\DeclareRobustCommand{\rchi}{{\mathpalette\irchi\relax}}
\newcommand{\irchi}[2]{\raisebox{\depth}{$#1\chi$}}
%%% voila

\newcommand{\di}[1]{\mathrm{div}\left\{#1\right\}}
\newcommand{\la}{\left\langle}
\newcommand{\ra}{\right\rangle}
\newcommand{\pd}{\partial}
\newcommand{\de}[1]{\delta\left(#1\right)}
\newcommand{\td}{\mathrm{d}}
\newcommand{\ma}[1]{\max{\left\{#1\right\}}}
\newcommand{\mi}[1]{\min{\left\{#1\right\}}}
\newcommand{\etals}{\textit{et al. }}
\newcommand{\ex}[1]{\exp{\left(#1\right)}}
\newcommand{\loge}[1]{\ln{\left(#1\right)}}
\newcommand{\tr}[1]{\mathrm{tr}\left\{#1\right\}}
\newcommand{\id}{\mathbb{I}}
\newcommand{\com}[2]{\left[#1,\,#2\right]}
\newcommand{\acom}[2]{\left\{#1,\,#2\right\}}
\newcommand{\co}[1]{\cos{\left(#1\right)}}
\newcommand{\si}[1]{\sin{\left(#1\right)}}
\newcommand{\sh}[1]{\sinh{\left(#1\right)}}
\newcommand{\ch}[1]{\cosh{\left(#1\right)}}
\newcommand{\shi}[1]{\mathrm{shi}{\left(#1\right)}}
\newcommand{\cohi}[1]{\mathrm{chi}{\left(#1\right)}}
\newcommand{\ct}[1]{\coth{\left(#1\right)}}
\newcommand{\bla}{bla\\bla\\bla\\bla\\bla}
%\newcommand{\eqref}[1]{(\ref{#1})}
%\newcommand{\PR}{Phys. Rev.}
\newcommand{\PRA}{Phys. Rev. A }
\newcommand{\PRB}{Phys. Rev. B }
\newcommand{\PRD}{Phys. Rev. D }
\newcommand{\PRE}{Phys. Rev. E }
%\newcommand{\PRL}{Phys. Rev. Lett. }
\newcommand{\PRX}{Phys. Rev. X }
%\newcommand{\EPL}{EPL (Europhys. Lett.) }
%\newcommand{\RMP}{Rev. Mod. Phys. }
%\newcommand{\NJP}{New. J. Phys. }
\usepackage{fmtcount}



\newcommand{\mb}[1]{\mbox{\boldmath$#1$}}
\newcommand{\mc}[1]{\mathcal{#1}}
\newcommand{\mbb}[1]{\mathbb{#1}}
\newcommand{\mf}[1]{\mathfrak{#1}}
\newcommand{\mrm}[1]{\mathrm{#1}}
\newcommand{\clr}{\color{red}}


\newcommand{\ptl}[3]{\left( \frac{\partial {#1}}{\partial {#2}} \right)_{#3}}

\def\dbar{{\mathchar'26\mkern-12mu {\rm d}}}


% Editing colors
\newcommand{\draftmode}{1}    %to control draft colors below
\newcommand{\notetoself}[1]{\ifnum \draftmode=1 {\color[rgb]{0,0,0.8} [#1]} \fi}  %notes to self in blue when \draftmode==1.  invisible otherwise
\newcommand{\cuttext}[1]{\ifnum \draftmode=1 {\color[rgb]{0,0.5,0} [#1]} \fi}  %cut out text in green when \draftmode==1.  invisible otherwise
\newcommand{\warntext}[1]{\ifnum \draftmode=1 {\color[rgb]{0.9,0.6,0} #1} \else {#1} \color{black} \fi}
%warning text in orange when \draftmode==1.  regular black otherwise
\newcommand{\siren}[1]{{\color[rgb]{0.8,0.0,0} [#1]}}  %for loud error-type text that prints red *always*
\newcommand{\aref}[1]{{Appendix~\hyperref[#1]{A}}}
\newcommand{\bref}[1]{{Appendix~\hyperref[#1]{B}}}
\newcommand{\dref}[1]{{Appendix~\hyperref[#1]{C}}}
\usepackage[colorlinks=true,citecolor=blue,linkcolor=blue,urlcolor=blue]{hyperref}
%\usepackage{hyperref}
\usepackage{cleveref}



\begin{document}


\title{Eavesdropping on the Decohering Environment:\\ Quantum Darwinism, Amplification, and the Origin of Objective Classical Reality}

\author{Akram Touil}
\email{akramt1@umbc.edu}
\affiliation{Department of Physics, University of Maryland, Baltimore County, Baltimore, MD 21250, USA}
\affiliation{Center for Nonlinear Studies, Los Alamos National Laboratory, Los Alamos, New Mexico 87545}

\author{Bin Yan}
\affiliation{Center for Nonlinear Studies, Los Alamos National Laboratory, Los Alamos, New Mexico 87545}
\affiliation{Theoretical Division, Los Alamos National Laboratory, Los Alamos, New Mexico 87545}

\author{Davide Girolami}
\affiliation{Politecnico di Torino, Corso Duca degli Abruzzi 24, Torino, 10129, Italy}

\author{Sebastian Deffner}
\affiliation{Department of Physics, University of Maryland, Baltimore County, Baltimore, MD 21250, USA}
\affiliation{Instituto de F\'isica `Gleb Wataghin', Universidade Estadual de Campinas, 13083-859, Campinas, S\~{a}o Paulo, Brazil}

\author{Wojciech Hubert Zurek}
\affiliation{Theoretical Division, Los Alamos National Laboratory, Los Alamos, New Mexico 87545}


	
\begin{abstract} 
``How much information about a system $\cS$ can one extract from a fragment $\cF$ of the environment $\cE$ that decohered it?'' is the central question of Quantum Darwinism. To date, most answers relied on the quantum mutual information of $\cS\cF$, or on the Holevo bound on the channel capacity of $\mathcal{F}$ to communicate the classical information encoded in $\mathcal{S}$. These are reasonable upper bounds on what is really needed but much harder to calculate -- the accessible information in the fragment $\cF$ about $\cS$. We consider a model based on imperfect {\tt c-not} gates where all the above can be computed, and discuss its implications for the emergence of objective classical reality. We find that all relevant quantities, such as the quantum mutual information as well as various bounds on the accessible information exhibit similar behavior. In the regime relevant for the emergence of objective classical reality this includes scaling independent of the quality 
of the imperfect {\tt c-not} gates or the size of $\cE$, and even nearly independent of the initial state of $\cS$. 
%We also show that redundancy of the record of the pointer states in the environment depends only weakly on which estimate of the channel capacity of $\cF$ is used.
\end{abstract}


\maketitle

Quantum Darwinism \cite{Zurek2000AP,Zurek2003RMP,Ollivier2004PRL,Ollivier2005PRA,Zurek2009NP} explains the emergence of objective classical reality in our quantum Universe: The decohering environment $\cal E$ is a ``witness’’  who monitors and can reveal the state of the system $\cS$. Agents like us never measure systems of interest directly. Rather, we accesses fragments $\cal F$ of $\cal E$ that carry information about them. Since its inception \cite{Zurek2000AP}, Quantum Darwinism has advanced on both theory \cite{Giorgi2015PRA,Balaneskovic2015EPJD,Balaneskovic2016EPJD,Knott2018PRL,Milazzo2019PRA,Campbell2019PRA,Ryan2020,Garcia2020PRR,Lorenzo2020PRR,Qdc1,Qdc2,Qdc3,Qdc4,Qdc5,Qdc6,Qdc7,Qdc8,Qdc9,Qdc10,Qdc11} and experimental fronts \cite{Ciampini2018PRA,Chen2019SB,Unden2019PRL,Garcia2020NPJQI}.

Quantum mutual information $I({\cal S} : {\cal F})$ between an environment fragment and the system yields an upper bound on what $\cF$ can reveal about $\cS$. It has been used to estimate the capacity of the environment as a communication channel. We analyze a solvable model based on imperfect tunable {\tt c-not} (or {\tt c-maybe}) gates that couple $\cS$ to the subsystems of $\cE$. We compute the mutual information $I({\cal S}: {\cal E})$ as well as the Holevo $\chi({\cal S} : {\cal F})$  \cite{Holevo,nielsen2002quantum} -- that characterize the accessible information
%of either $\cS$ and $\cF$ -- 
in our {\tt c-maybe} - based model. We also compute the quantum discord \cite{Zurek2000AP,Ollivier2001PRL,Henderson2001JPA,Giorda2010PRL,Shi2011JPA,Zwolak2013SR,ZRZ2016SR} -- the difference of $I({\cal S} : {\cal F})$ and $\chi({\cal S} : {\cal F})$ that quantifies the genuinely quantum correlations between $\mc{S}$ and $\mc{F}$ \cite{Brodutch2011JPCS,Adesso2016,Bera2017RPP,discordc1}.

%%%The central quantity in the study of Quantum Darwinisim is the quantum mutual information $I({\cal S} : {\cal F})$ between the environment fragment and the system. It yields an upper bound on what $\cF$ can reveal about $\cS$ and has been used to assess the capacity of the environment as a communication channel. The relevant Holevo quantity -- a better estimate of channel capacity --  is significantly more involved to compute in closed form.  To the very best of our knowledge, this has not been accomplished in a scenario relevant for Quantum Darwinism.

%We propose and analysis a exactky solvable model, in which the interaction of $\mc{S}$ with the qubits of $\mc{E}$ is described by imperfect, tunable {\tt cnot} (or {\tt c-maybe}) gates.  For this model, we obtain analytical formulas not only for the mutual information $I({\cal S}: {\cal E})$, but also the Holevo bound $\chi({\cal S} : {\cal F})$, a tight upper bound on the channel capacity of either $\cS$ or $\cF$ \cite{Holevo,nielsen2002quantum}.  We also compute the Quantum Discord \cite{Zurek2000AP,Ollivier2001PRL,Henderson2001JPA,Giorda2010PRL,Shi2011JPA,Zwolak2013SR}, which is the difference of $I({\cal S} : {\cal F})$ and $\chi({\cal S} : {\cal F})$, and which quantifies the genuinely quantum correlations between $\mc{S}$ and $\mc{F}$ \cite{Brodutch2011JPCS,Adesso2016,Bera2017RPP}.

We find that $I({\cal S} : {\cal F})$ and $\chi({\cal S}, {\cal F})$ exhibit strikingly similar dependence on the size of $\cF$, with the initial steep rise followed by the classical plateau where, at the level set by the entropy $H_\cS$ of the system, the information $\cF$ has about $\cS$ saturates: Enlarging $\cF$ 
%does not provide additional data -- it 
only confirms what is already known.  This behavior is universal and nearly independent of the initial state of $\mc{S}$ and the size of $\cE$.

\paragraph*{The model.}

%We begin by describing the model and establishing notions and notations.  
The system $\mc{S}$ is a qubit coupled to $N$ independent non-interacting qubits of the environment  $\mc{E}$ via a {\tt c-maybe} gate,
\begin{equation}
U_{\oslash}=\begin{pmatrix}
1 & 0 & 0 & 0 \\
0 & 1 & 0 & 0 \\
0 & 0 & s & c \\
0 & 0 & c & -s
\end{pmatrix}.
\label{ope}
\end{equation}
The parameters $c=\cos(a)$ and $s=\sin(a)$ (where $a$ is the rotation angle of the target qubit) 
%($c^2 + s^2=1$) 
quantify the imperfection.
%, and the {\tt c-not} is recovered for $c=1$ and $s=0$. \textcolor{red}{Note that these parameters are real and belong to the interval $[0,1]$}.

Our Quantum Universe $\cS\cE$ starts in a pure state:
%\footnote{We shall (on occasion) relax this convenient but unrealistic assumption -- key conclusions remain unchanged.}: 
\begin{equation}
\ket{\Psi_\mathcal{SE}^0} = \left(\sqrt{p}\, \ket { 0_\mathcal{S}} + \sqrt{q}\, \ket { 1_\mathcal{S}} \right) \bigotimes_{i=1}^{N} \ket {0^{i}} ,
\end{equation}
where $p+q=1$. The unitary $U_{\oslash}$ correlates each qubit in $\cE$ with $\mc{S}$,  and we obtain a branching state \cite{blume2005simple},
\begin{equation}
\ket{ \Psi^{\oslash}_\mathcal{SE}} = \sqrt{p}\,  \ket {0_\mathcal{S}} \bigotimes_{i=1}^{N} \ket { 0_{\cE_{i}}} + \sqrt{q}\,  \ket{1 _\mathcal{S}} \bigotimes_{i=1}^{N} \ket { 1_{\mathcal{E}_i}} \, .
\label{bstate}
\end{equation}
By construction $\ket{0_\mathcal{S}}$ and $\ket{1_\mathcal{S}}$ are the pointer states \cite{basis1,basis2}. They are orthogonal and immune to decoherence. The corresponding record states of $\mc{E}$ are
\begin{equation}
\ket { 0_{\cE_{i}}}  \equiv \ket{0^i}\quad\text{and}\quad \ket { 1_{\mathcal{E}_i}} \equiv s \ket{0^i}+ c \ket{1^i}\,,
\end{equation}
in terms of the orthogonal basis $\ket{0^i}$ and $\ket{1^i}$ of the $i$th qubit that defines $U_\oslash$, so that $\bk { 0_{\cE_{i}}} { 1_{\mathcal{E}_i}}= s$.

We will be interested in the correlations between the fragment $\cal F$ and $\cal S$. The marginal states of ${\cal S}$, an $m$-qubit fragment ${\cal F}_{m}$, and a bipartition ${\cal SF}_m$ are rank-two density matrices~\footnote{Deducing the reduced quantum states $\rho_\mathcal{S}$ and $\rho_{\mathcal{S}\mathcal{F}_m}$ in Eqs.~\eqref{eq:rho_S} and \eqref{eq:rho_SF} is straightforward, whereas $\rho_{\mathcal{F}_m}$ is slightly more involved, see e.g. \cite{ZQZ10}.}:
\begin{equation}
\label{eq:rho_S}
\rho_\mathcal{S}\equiv\ptr{\mc{E}}{\ket{ \Psi^{\oslash}_\mathcal{SE}}\bra{\Psi^{\oslash}_\mathcal{SE}} }=
\begin{pmatrix}
p & s^{N}\sqrt{pq}  \\
s^{N}\sqrt{pq}  & q 
\end{pmatrix} ,
\end{equation}

\begin{equation}
\label{eq:rho_F}
\rho_{\mathcal{F}_m}=
\begin{pmatrix}
p & s^{m}\sqrt{pq}  \\
s^{m}\sqrt{pq}  & q
\end{pmatrix},
\end{equation} 

\begin{equation}
\label{eq:rho_SF}
\rho_{\mathcal{S}\mathcal{F}_m}=
\begin{pmatrix}
p & s^{N-m}\sqrt{pq}  \\
s^{N-m}\sqrt{pq}  & q
\end{pmatrix}\,.
\end{equation}
%Note that Eqs.~\eqref{eq:rho_S} -- \eqref{eq:rho_SF} 
%They demonstrate how much $\mc{E}$ decoheres $\mc{S}$.

{\it Symmetric quantum mutual information}
%${I}(\mc{S}:\mc{F}_m)$.
is often used to estimate the accessible information in $\cF$ in
 Quantum Darwinism \cite{Zurek2003RMP,Ollivier2004PRL,Ollivier2005PRA,blume2006quantum,Riedel2010PRL,Riedel2011NJP,Campbell2019PRA,Ryan2020,Garcia2020PRR}. It is defined using the von Neumann entropy, $H(\rho)=-\tr{\rho \log_{2}(\rho)}$ as;
\begin{equation}
{I}(\mc{S}:\mc{F}_m)=H_{\mc{S}}+H_{\mc{F}_m}-H_{\mc{S},\mc{F}_m}.
\label{MUTI}
\end{equation}
Joint entropy $H_{\mc{S},\mc{F}_m}$ quantifies the ignorance about the state of ${\mc{S}\mc{F}_m}$ in the tensor product of the Hilbert spaces of $\cS$ and $\cF$. 

In our model ${I}(\mc{S}:\mc{F}_m)$ can be computed exactly \cite{ZQZ10};
\begin{equation}
{I}(\mc{S}:\mc{F}_m)={h}(\lambda^{+}_{N,p})+{h}(\lambda^{+}_{m,p})-{h}(\lambda^{+}_{N-m,p}),
\label{mutinfo}
\end{equation}
where ${h}(x)=-x \log_2(x)-(1-x) \log_2(1-x)$ and $\lambda^{\pm}_{k,p}$ are the eigenvalues of the density matrices \eqref{eq:rho_S} -- \eqref{eq:rho_SF},
\begin{equation}
\lambda^{\pm}_{k,p}=\frac{1}{2}\left(1 \pm \sqrt{\left(q-p\right)^{2}+4s^{2k}pq}\right)\,.
\label{root}
\end{equation}
We thus have a closed expression for the mutual information ${I}(\mc{S}:\mc{F}_m)$.


As seen in Fig.~\ref{fig:mutI}, symmetric mutual information ${I}(\mc{S}:\mc{F}_m)$ exhibits a steep initial rise with increasing fragment size $m$, as a larger $\cF_m$ provides more data about $\cS$. This initial rise is followed by a long \emph{classical plateau}, where the additional information imprinted on the environment is redundant.  

Note that, when $\cS \cE$ is in a pure state,  the entropy of a fragment $\cF$ is equal to $H_{\cS d\cF}$, that is the entropy $\cS$ would have if it was decohered only by the fragment $\cF$.  When we further assume {\it good decoherence} \cite{blume2005simple,ZQZ10} -- i.e., that the off-diagonal terms of $\rho_\cS$ and $\rho_{\cS \cF_m} $ are negligible (which in our model corresponds to $s^{N-m} \ll  s^m$) -- we obtain an approximate equality; 
\begin{equation}
I(\cS : \cF_m) = H_{\cF_m} = H_{\cS d \cF_m}, 
\label{gooddeco}
\end{equation}
since $H_\cS=H_{\cS \cF_m}$ cancel one another in Eq. ~\eqref{MUTI}.  Furthermore,  when the environment fragments are typical~\cite{CoverThomas} (and in our model all fragments of the same size are identical -- hence, each is typical) the plot of ${I}(\mc{S}:\mc{F}_m)$ is antisymmetric around ${I}(\mc{S}:\mc{F}_m) = H_{\cal S}$ and $m=N/2$ \cite{blume2005simple}.  

We will see that the behavior of  ${I}(\mc{S}:\mc{F}_m)$ is approximately universal. This means that after suitable re-scaling its functional form is nearly independent of the size of the environment $N$, of the quality of the {\tt c-maybe} gate $U_\oslash$, and almost independent of the initial state of $\cS$.

%Before we continue we note that agents tolerate a finite {\it information deficit} $\delta$.  that quantifies the lack of information an agent receives from any single fragment $\mc{F}_m$, and we write  $I(\cS : \cF_{m_\delta}) \geq (1-\delta) H_\cS$. However,  if size of such deficient fragments is small, $m_\delta \ll N$, then the number of fragments $1/f_\delta\equiv N/m_\delta$ is very large. This motivates the definition of the redundancy of information, $\mc{R}_\delta=1/f_\delta$,
Agents generally do not insist on knowing the state of $\cS$ completely, but tolerate a finite {\it information deficit} $\delta$. When $I(\cS : \cF_{m_\delta}) \geq (1-\delta) H_\cS$ is attained already for a fragment with $m_\delta \ll N$ subsystems, a fraction $f_\delta=m_\delta /N$ of the environment, then there are many ($1/f_\delta$) such fragments. We define redundancy of the information about $\cS$ in $\cE$ via:
%We note that observers will generally not insist on knowing the state of $\cS$ completely, but will tolerate a finite {\it information deficit} $\delta$. We will be interested in the redundancy ${\cal R}_\delta$ -- the number of the fragments of the environment that can supply all by $\delta$ of the information about $\cS$. Definition of ${\cal R}_\delta$ is illustrated in Fig. XX. They can independently provide the same information about the system. Indeed, we have arrived -- to within one additional refinement – at the definition of redundancy: We define redundancy as the number of fragments that can independently supply all but $\delta$ of the missing information about the system:
%To quantify this abundance of the records of $\cS$ in $\cE$ it is helpful to consider redundancy -- the number of copies -- of the information contained in the environment as a whole. 
\begin{equation}
\label{eq:redun}
{\cal R}_{\delta} \equiv N/m_\delta \quad \text{with}\quad I(\cS : \cF_{m_\delta}) = (1-\delta) H_\cS\,.
\end{equation}
Redundancy ${\cal R}_{\delta}$ is the length of the classical plateau in the units set by $m_\delta$, see Fig.~\eqref{fig:mutI}. The beginning of the plateau is determined by the smallest $m_\delta$ such that $I(\cS : \cF_{m_\delta}) \geq (1-\delta) H_\cS$. 

%The other reason for introducing $\delta$ is that %(unless perfect {\tt c-not}'s are used) 
In realistic models $I(\cS : \cF)=H_\cS$ only when $ f = 1/2$ (see \cite{blume2005simple}). Thus, significant redundancy appears only when the requirement of completeness of the information about $\cS$ that can be extracted from $\cF$ is relaxed. Moreover,  Eq.~\eqref{eq:redun} is an overestimate since $I(\cS : \cF_{m_\delta}) $ is only an upper bound of what can be found out about $\cS$ from $\cF$ \footnote{Note that $H_{\cS}$ -- in the circumstances of interest to us -- is not the thermodynamic entropy of $\cS$. Rather, it is the missing information about the few {\it relevant} degrees of freedom of $\cS$. The thermodynamic entropy of a cat, for instance, will vastly exceed the information an observer is most interested in -- e.g., the one crucial bit in the `diabolical contraption' envisaged by Schr{\"o}dinger.}.

We will now consider better estimates: Inset in Fig.~\ref{fig:mutI} compares $I(\cS : \cF_{m_\delta}) $ with the two Holevo - like $\chi$'s we are about to discuss and illustrates resulting fragment sizes (hence, redundancies) they imply.

%This is an overestimate of ${\cal R}_{\delta}$: As noted earlier, $I(\cS : \cF_{m_\delta}) $ is an upper bound on the information that can be extracted. Our aim is to provide a better (Holevo bound - based) estimate of what $\cF$ can reveal about $\cS$.

%In previous work \cite{blume2005simple}, $\delta$ was introduced since $I(\cS : \cF)$ can reach $H_\cS$ only when $ f = 1/2$ in realistic models \footnote{It is interesting to note that $H_\cS$ is not the thermodynamic entropy of $\mc{S}$.  Whereas in thermodynamics the entropy quantifies the occurrence of all accessible microstates \cite{Deffner2019book}, in the present context an observes is typically only interested in a few salient features.}.  Hence,  redundancy is only exhibited if the requirement of extracting complete information about $\cS$ from $\cF$ is relaxed.  

\begin{figure}
	\includegraphics[width=0.482\textwidth]{Figure1.png}
	\caption{\label{fig:mutI}{\it Approximate universality of mutual information:} Symmetric $I(\cS : \cF_m)$ and Holevo bound $\chi(\check \cS : \cF_m)$ coincide until the fragment $\cF_m$ becomes almost as large as $\cE$. Renormalized $I(\cS : \cF)/H_\cS$ and $\chi(\check \cS : \cF)/H_\cS$ depend only weakly on the probabilities of the outcomes (see inset). Their difference -- quantum discord $D(\check \cS : \cF)$ -- vanishes until $\cF_m$ begins to encompass almost all of $\cE$, $m \sim N/R_\delta$. The inset also compares the ratios $\chi(\check \cS : \cF)/H_\cS$ and $\chi( \cS : \check \cF)/H_\cS$ computed for several probabilities $p$ of the pointer state $\ket {0_\cS}$ in Eq.~\eqref{bstate}. Note that the fragment sizes $m_\delta$ that supply $\sim$80\% of information about $\cS$ are only modestly affected by $p$ and quite similar for these two different information measures.} 
\end{figure}



\paragraph*{Asymmetric mutual information} 
%In realistic situations is it is unreasonable to assume that an agent has access to full state tomography of $\cS\cF$. Hence,  Quantum Darwinism needs to account for the local nature of measurements, rather than working with the full mutual information ${I}(\mc{S}:\mc{F}_m)$ \eqref{MUTI}. In particular,  this means one has to account for the measurement back action \cite{nielsen2002quantum}.  We will denote the subsystem upon which the measurement is taken by  an inverted ``hat'', and if the measurement is taken on $\check{\mc{S}}$ we consider the asymmetric mutual information
is defined using conditional entropy. We mark the system whose states are used for such conditioning by an inverted ``hat'', so when it is $\check{\mc{S}}$ we consider the asymmetric mutual information:
\begin{equation}
J(\check \cS : \mc{F}_m)_{\{ \ket {s_k} \}}  = H_{\cF} - H_{{\cF}|\check \cS_{\{ \ket {s_k} \}}} \ .
\label{Asmutinfo}
\end{equation}
Above, $H_{{\cF_m} | {\check \cS_{\{ \ket {s_k} \}}}}$  is the conditional entropy \cite{nielsen2002quantum} that quantifies the missing information about $\cF$ remaining after the observable with the eigenstates $\{ \ket {s_k} \}$ was measured. Accordingly, the joint entropy in Eq.~\eqref{MUTI} is replaced by;
\begin{equation}
H_{{\cF_m}, \check \cS_{\{ \ket {s_k}\}}}= H_{{\cF_m} | {\check \cS_{\{ \ket {s_k} \}}}} + H_{\check \cS_{\{ \ket {s_k}\}}}\,.
\label{CondS}
\end{equation}
The asymmetric joint entropy  depends on whether $\cS$ or $\cF$ are measured and on the measurements that are used. The entropy increase associated with the wavepacket reduction means that the asymmetric entropy \eqref{CondS} is typically larger than the symmetric version $H_{\cS, {\cF_m}} $ in Eq.~\eqref{MUTI}: Local measurements cannot extract information encoded in the quantum correlations between $\mc{S}$ and $\mc{F}_m$, which is why the asymmetric $J(\check \cS : \mc{F}_m)$ is needed, \cite{nielsen2002quantum}; see also \cite{touil2021quantum}. 

For  optimal  measurements the asymmetric $J(\check \cS : \mc{F}_m)$ defines the Holevo bound \cite{Holevo},
\begin{equation}
\label{eq:holevo}
J(\check \cS:\mc{F}_m)=\max_{\{ \ket {s_k} \}} J(\check \cS:\mc{F}_m)_{\{ \ket {s_k} \}} \equiv \chi (\check \cS:\mc{F}_m) .
\end{equation}
%Here, ``optimal'' stands for the measurements (possibly, involving POVM's) that maximize the asymmetric mutual information. The basis subscript will be eliminated from now on as optimization determines the measurement.
%measurement, and they will be our focus of attention).
%For ease of notation, we can also drop the subscript as from here on we will focus on such optimal measurements.
In our model,
%it is natural to focus on measurements of the pointer basis of $\cS$. In particular, 
%in case of good decoherence, 
measurement of the pointer observable of $\cS$ is optimal~\cite{ZQZ10}. Indeed,~\myeqref{bstate} shows that in the pointer basis $\{\ket {0_\cS}, \ket {1_\cS}\}$ the conditional entropy disappears, $H_{{\cF_m} | {\check \cS}} = 0$, as states of $\cF_m$ correlated with pointer states of $\cS$ are pure.
%Allowing POVM's (rather than just projective measurements) leads to Holevo information (or Holevo bound) $\chi$. 

The limit of large $\cE$ ($N \ge N-m \gg m$) reflects typical situation of agents (who do not even know the size of $\cE$, and only access ``their ${\cal F}_m$'', with $m \ll N$). This is  good decoherence, $s^N \le s^{N-m} \ll s^m$, and equations simplify:  
\hocom{As already noted, $H_{\cS,\cF_m} = H_{\cS}$.
Therefore;
\ba
I(\mc{S}:\mc{F}_m) \approx H_{\mc{F}_m}={h}(\lambda^{+}_{m,p}) = 
J(\check \cS : \mc{F}_m)=
\chi(\check \cS : \mc{F}_m) . \ \ \ \ \ \ 
%= J(\check{\mc{S}}:\mc{F}_m)
\label{Mutinfo}
\ea
%\end{equation}
The equality $J(\check{\mc{S}}:\mc{F}_m)\approx I(\mc{S}:\mc{F}_m)$ assumes optimal measurements on $\check \cS$. 
In case of good decoherence
%, $N \rightarrow \infty$, and 
measurements of the pointer states are optimal \cite{ZQZ10}.

Indeed, for model and for good decoherence \cite{blume2005simple,ZQZ10} measuring the pointer observable of $\cS$ is actually optimal \cite{ZQZ10}. Observe in Eq.~\eqref{bstate} that in the pointer basis $\{\ket {0_\cS}, \ket {1_\cS}\}$ the conditional entropy disappears, $H_{{\cF_m} | {\check \cS}} = 0$, since all states of $\cF_m$ correlated with the pointer states of $\cS$ are pure. Moreover, good decoherence, $s^N \le s^{N-m} \ll s^m$, is equivalent to large environments, $N \ge N-m \gg m$. 
}
Using $H_{\cS,\cF_m} = H_{\cS}$ and Eq.~\eqref{gooddeco} we can thus write,
\begin{equation}
I(\mc{S}:\mc{F}_m) \approx H_{\mc{F}_m}={h}(\lambda^{+}_{m,p}) =\chi(\check \cS : \mc{F}_m). 
\label{Mutinfo}
\end{equation}
%Note that identifying $\chi(\check{\mc{S}}:\mc{F}_m)\approx I(\mc{S}:\mc{F}_m)$ is only true for measurements in the pointer basis of $\mc{S}$.
An immediate important consequence is that  $H_{\cF_m}$ determines both the symmetric $I(\mc{S}:\mc{F}_m)$ (except for the final rise) as well as the asymmetric (optimal) $J(\check \cS : \mc{F}_m)=\chi(\check \cS : \mc{F}_m)$. We have:
\begin{widetext}
\begin{equation}
\chi(\check \cS : \mc{F}_m)=-\frac{1}{2}\,\log _{2}\left( p q\left(1-s^{2 m}\right) \right)-\sqrt{1-4p q\left(1-s^{2 m}\right)}\, \operatorname{Arctanh}_{2}\left(\sqrt{1-4p q\left(1-s^{2 m}\right)}\right)\,,
\label{ChiS}
\end{equation}
\end{widetext}
where ``$\operatorname{Arctanh}_2$'' denotes $\operatorname{Arctanh}/\ln(2)$.

Fig.~\ref{fig:mutI} compares  $\chi(\check{\mc{S}}:\mc{F}_m)$ with ${I}(\mc{S}:\mc{F}_m)$ for finite  and infinite $N$ and for different values of $s$ and $p$. 
As it shows, Eq. (\ref{ChiS}) matches $I(\mc{S}:\mc{F}_m)$ until the far end ($N-m \ll m$) of the classical plateau. This is a consequence of two scalings: (i) ``vertically'', the plateau appears at $H_\cS = -p\log_2( p) - q \log_2 (q)$, and it is easy to see that for  $s^N \ll s^m \ll 1$ we have $\chi(\check{\mc{S}}:\mc{F}_m)=H_\cS$ in Eq.~\eqref{ChiS}; (ii) ``horizontally",  $H_{\mc{F}_m}$ depends on $s^m$, so weakly entangling gates can be compensated by using more of them -- larger $m$. What is surprising is how insensitive are these plots to $p$, the probability of the outcome.

This remarkably universal behavior is a consequence of \emph{good decoherence} \cite{ZQZ10}.  Both,  $\rho_{\mathcal{S}}$ and $\rho_{\mathcal{S}\mathcal{F}_m}$, Eqs.~\eqref{eq:rho_S} and \eqref{eq:rho_SF}, become diagonal in the pointer basis. Moreover, the quality of $U_{\oslash}$ (set by $c$ and $s$) determines the `` information flow rate'' from $\cal S$ to $\cal F$. Thus, when (at a fixed $p$) one demands the same $H_{\mc{F}_m}$, this translates into identical $\rho_{\cF_m}$  when $s_1^{m_1} \simeq s_2^{m_2}$. Therefore, less efficiently entangling gates can be compensated by relying on more of them -- on a suitably enlarged $\cF$, with $m_2 = m_1 \,\log(s_1)/\log(s_2)$. 

\paragraph{Environment as a communication channel.}
While the mutual information $I({\mc{S}}:\mc{F}_m)$ is easier to compute and a safe upper bound on the accessible information in $\cF_m$, it is important to verify it is also a reasonable estimate of that accessible information (as generally assumed in much of the Quantum Darwinism literature). 
The asymmetric mutual information extracted by optimal measurements on the environment fragment $\cF_m$ is:
%The corresponding asymmetric mutual information extracted by optimal POVMs on the environment fragment $\cF_m$ then reads
\begin{equation}
 J(\cS : \check \cF_m) =  H_\cS - H_{\cS|\check{\cF}_m} = \chi(\cS : \check \cF_m)\,.
 \label{JF}
\end{equation}
The joint entropy given in terms of the conditional entropy $H_{\cS | {\check \cF}_m}$ now becomes,
\begin{equation}
H_{\cS, \check \cF_m} = H_{\cS | {\check \cF}_m} + H_{{\check \cF}_m}\, .
\label{CondF}
\end{equation}
As in Eq.~\eqref{CondS} above, all terms in  Eq.~\eqref{CondF} depend on how $\cF$ is measured. However, while measuring $\mc{S}$ in the pointer basis simplified the analysis (since e.g. $H_{{\cF}|{\check \cS}_{\{ \ket {s_k}\}}}=0$) this is no longer the case when $\cF_m$ is measured. 

To compute $\chi(\cS : \check \cF_m)$ we rely on the Koashi-Winter monogamy relation~\cite{KW}. Details of that calculation are relegated to the supplementary material~\cite{SM}.

We focus again on the limit of large $\cE$ ($N \ge N-m \gg m$): Agents only access ``their ${\cal F}_m$'', a small fraction of $\cE$ with $m \ll N$.  Assuming good decoherence we obtain
\begin{widetext}
\begin{equation}
\label{eq:holevo_F}
\chi(\mc{S}:\check \cF_m)= H_\cS +\frac{1}{2}\, \log_{2}\left(pqs^{2m}\right) +\sqrt{1-4pqs^{2m}}\,\operatorname{Arctanh}_2\left(\sqrt{1-4pqs^{2m}}\right)\,.
\end{equation}
\end{widetext}
Equation~\eqref{eq:holevo_F} constitutes our main result. We have decomposed the Holevo-like quantity $\chi(\mc{S}:\check \cF_m)$ into the plateau entropy $H_\cS$ and $H_{\cS|{\check \cF_m}}$ -- the ignorance about $\cS$ remaining in spite of the optimal measurement on $\cF_m$~\cite{elaborate}. Rather remarkably,  $H_{\cS|{\check \cF_m}}= H_\cS - \chi(\mc{S}: \check {\mc{F}}_m) $ -- the conditional entropy -- scales {\it exactly} with $pqs^{2m}$. What remains to do is to quantify the differences of $I({\mc{S}}:\mc{F}_m)$ and  $\chi (\check \cS:\mc{F}_m)$ with  $\chi(\mc{S}:\check \cF_m)$. In Fig.~\ref{fig:D} we compare it with these other, easier to compute, quantities. 

{\it Redundancy} of the information about $\cS$ in the channel $\cF_m$ can be now computed using $\chi(\mc{S}:\check \cF_m)$, Eq. \eqref{eq:holevo_F}, and compared with the estimates based on $I(\cS:\cF_m)$. The fragment $\cF_{m_\delta}$ can carry all but the deficit $\delta$ of the classical information about the pointer state of $\cS$ when $\chi(\mc{S}:\check \cF_{m_\delta})\ge(1-\delta) H_\cS$. This leads to a transcendental equation for $m_\delta$ that we solve numerically: $R_\delta = N / m_{\delta}$, where $N$ is the number of subsystems in $\cE$.

The inset in Fig.~\ref{fig:mutI} shows that -- while $m_\delta$ deduced using $I(\cS:\cF_m)\approx \chi (\check \cS:\mc{F}_m)$ do not coincide with those obtained using $\chi(\mc{S}:\check \cF_m)$ -- the difference is modest, unlikely to materially affect conclusions about the emergence of objective classical reality. Indeed, in the supplementary materials we estimate that the redundancy estimates based on $I(\cS:\cF_m)$ and $\chi(\mc{S}:\check \cF_m)$ differ at most by $\sim 37\%$ for $\delta \leq 0.2$,
and by much less in the regime where $\delta \rightarrow 0$.

In situations relevant for observers who rely on photons, $R_{\delta=0.1} \simeq 10^8$ is amassed when sunlight illuminates a $1\mu m$ dust grain in a superposition with a $1\mu m$ spatial separation for $1 \mu s$ ~\cite{Riedel2010PRL,Riedel2011NJP}. It may seem like we are stretching the applicability of our {\tt c-maybe} model too far, but the equations for $I({\mc{S}}:\mc{F}_m)$ and $\chi (\check \cS:\mc{F}_m)$ derived for photon scattering {\it coincide } with our Eq. \eqref{ChiS}, see supplement~\cite{SM}. Thus, it appears that the information transfer from $\cS$ to $\cE$ leading to the buildup of redundancy has universal features captured by our model. 


{\it Quantum discord} is the difference between symmetric \eqref{mutinfo} and asymmetric quantum mutual information \cite{Ollivier2001PRL,Henderson2001JPA,Giorda2010PRL,Shi2011JPA,Zwolak2013SR,ZRZ2016SR,Brodutch2011JPCS,Adesso2016,Bera2017RPP,discordc1}. The \emph{systemic discord} is defined as;
\begin{equation}
D(\check{\mc{S}}:\mc{F}_m) = I({\mc{S}}:\mc{F}_m)  - \chi (\check \cS:\mc{F}_m)  \,.
\label{Discord}
\end{equation}
The measurements on pointer observables of $\mc{S}$ are optimal. 

Mutual information for pure decoherence induced by non-interacting subsystems of $\cE$ can be written as \cite{ZQZ10,Z07b}:
\begin{equation}
I(\cS:\cF)=\stackrel {local/classical} {\bigl(H_\cF-H_\cF(0)\bigr)} + \stackrel {global/quantum} {\bigl(H_{\cS d \cE} - H_{\cS d \cE_{\backslash\cF}}\bigr)}\,.
\label{4.20}
\end{equation}
As $\cS\cE$ is a pure product state, the initial entropy of $\mc{F}$ is zero, $H_\cF(0)=0$. Assuming good decoherence and conditioning on the pointer basis (hence, 
$\chi(\check \cS :\mc{F}_m)=H_{\mc{F}_m}$), Eq. (\ref{Mutinfo}) we have
\begin{equation}
I({\mc{S}}:{\mc{F}}_m) - J(\check \cS : {\mc{F}}_m) = H_{\cS d \cE} - H_{\cS d \cE_{\backslash{\cF}_m}},
\label{discord}
\end{equation}
where $H_{\cS d \cE}$ ($H_{\cS d \cE_{\backslash\cF}}$) is the entropy of the system \textit{decohered} by $\cE$ (or just by $\cE_{\backslash\cF}$ -- i.e., $\cE$ less the fragment $\cF$).   

The global/quantum term represents quantum discord  in the pointer basis of $\cS$ \cite{ZQZ10}. Good decoherence implies $H_{\cS d \cE} \approx H_{\cS d \cE_{\backslash\cF}}$, so $D(\check{\mc{S}}:\mc{F}_m) \approx 0$. As long as $\cE_{\backslash\cF}$ is large enough to induce good decoherence, Eq. (\ref{Mutinfo}) holds, and, hence,  the systemic discord \eqref{Discord} vanishes~\footnote{Some have advocated ``strong Quantum Darwinism'' \cite{LeCastro} where the Holevo information  $\chi (\check \cS:\mc{F}_m)$ based on the measurement of $\cS$ (rather than $I({\mc{S}}:\mc{F}_m)$) would play a key role.  At least in case of good decoherence, and in view of Eq. ~\eqref{Mutinfo} which implies $\chi (\check \cS: \mc{F}_m) \approx I({\mc{S}}:\mc{F}_m)$ when $m \ll N - m<N$, such distinctions appear unnecessary.}.

Systemic quantum discord can become large again when $\cF_m$ encompasses almost all $\cE$, as in this case $H_{\cS \cF_m}$ approaches $H_{\cS \cE}=0$ (given our assumption of a pure $\cS \cE$). In this (unphysical) limit $I(\cS : \cF_m)$ climbs to $H_{\cF_m} + H_\cS = 2 H_{\cS}$, while $\chi (\check \cS:\mc{F}_m) \leq H_{\cF_m}$. As good decoherence implies $\chi (\check \cS:{\mc{F}}_m) \approx H_{\cF_m}$, $D(\check {\mc{S}}:\mc{F}_m)$ can reach $H_{\cF_m}$. Indeed, when $\cS\cE$ is pure, $\chi (\check \cS: \mc{F}_m)$ and $D(\check{\mc{S}}:\mc{F}_m)$ -- classical and quantum content of the correlation -- are complementary~\cite{Zwolak2013SR},  see Fig.~\ref{fig:mutI}.

\begin{figure}
	\includegraphics[width=0.482\textwidth]{Figure2.png}
	\caption{\label{fig:D}{\it Accessible information in the environment fragment.} Fragment $\cF_m$ carries at most $\chi(\cS : \check \cF)$ of classical information about the system it helped decohere. As seen above, this Holevo - like quantity is less than the symmetric mutual information $I(\cS : \cF)$ or the Holevo bound $\chi(\check{\mc{S}}:\mc{F}_m)$. Their difference (quantum discord $D(\cS : \check \cF)$) is significant already early on (in contrast to $D(\check \cS :  \cF)$), but disappears as the plateau is reached. It reappears again (as did $D(\check \cS :  \cF)$) when $\cF_m$ begins to encompass almost all of $\cE$. }
\end{figure}

%We conclude the analysis by returning to measurements on the environment fragment $\mc{F}$. 
The \emph{fragmentary discord} is the difference between the mutual information $I({\mc{S}}:\mc{F})$ and what can be extracted from $\mc{SF}$ by measuring only the fragment $\cF$:
\ba
I({\mc{S}}:\mc{F})  - \chi (\cS: \check \cF) \approx \chi (\check \cS:\mc{F}) - \chi(\cS : \check \cF).
\ea  
It can be evaluated:
\begin{equation}
\begin{split}
D(\cS : \check \cF_m) &\approx H_{\check \cF_m} - \left( H_\cS - H_{\cS|{\check \cF_m}}\right),\\ 
& = \left(H_{\cS|{\check \cF_m}}+H_{\check \cF_m}\right)- H_\cS\, .
\end{split}
\end{equation}
The bracketed terms in the last two expressions represent different quantities. The difference between the symmetric and asymmetric mutual information  $H_{\check \cF_m} - \left( H_\cS - H_{\cS|{\check \cF_m}}\right)$ is the original definition of discord.

Note that initially decoherence does not suppress fragmentary discord $D(\cS : \check \cF_m)$. This is because the states of $\cF_m$ that are correlated with the pointer states of $\cS$ are not orthogonal: The scalar product of the branch fragments $\cF_m$ corresponding to $\ket {0_\cS}$ and $\ket {1_\cS}$ is $s^m$. Thus, while the symmetric mutual information increases with $m$, orthogonality is approached gradually, also as $m$ increases.  Perfect distinguishability, i.e., orthogonality of record states of $\cF$ is needed to pass on all the information about $\cS$ \cite{Z07a, Z13,Gardas2016PRA}. See again Fig.~\ref{fig:D} for an illustration of these findings.

\paragraph*{Concluding remarks.}
We found that in the pre-plateau regime relevant for emergence of objective reality (where $I({\cal S} : {\cal F})$ increases with the size of $\cF$) the mutual information as well as the Holevo bound $\chi(\check {\cal S} : {\cal F})$ coincide and exhibit universal scaling behaviors independent of the size of $\cE$, of how imperfect are the {\tt c-maybe}'s, and only weakly dependent on the probabilities of pointer states. The corresponding Holevo $\chi(\check {\cal S} : {\cal F})$ and $I({\cal S} : {\cal F})$ coincide until $\cF$ encompasses almost all of $\cE$. 

However, the accessible information $\chi({\cal S} : \check {\cal F})$ in the environment fragments $\cF$ differs somewhat from $I({\cal S} : {\cal F})$ in the pre-plateau region. This difference tends to be small compared to, e.g., the level of the plateau, and disappears as the plateau is reached for larger fragments. This behavior -- generic when many copies of the information about $\cS$ are deposited in the environment -- facilitates estimates of the redundancy of the information about the system in the environment, as the differences between $I({\cal S} : {\cal F}) \approx \chi(\check {\cal S} : {\cal F})$ or $\chi({\cal S}: \check {\cal F})$ are noticeable but inconsequential.

To sum up, sensible measures of information flow lead to compatible conclusions about $R_\delta$. The differences in the estimates of redundancy based on these quantities are insignificant for the emergence of objective classical reality -- the overarching goal of Quantum Darwinism.  The functional dependence of the symmetric mutual information in the photon scattering model~\cite{Riedel2010PRL,Riedel2011NJP} is the same as in our model. Thus, the universality we noted in scaling with $s$ and $p$ (approximate for $I({\cal S} : {\cal F})=\chi (\check \cS: \cF_m) $, exact for $\chi (\cS: \check \cF_m) $) may be a common attribute of the information that reaches us, human observers.

\acknowledgements{We acknowledge several discussions with Michael Zwolak, who has provided us with extensive and perceptive comments that have greatly improved presentation of our results. This research was supported by grants FQXiRFP-1808 and FQXiRFP-2020-224322 from the Foundational Questions Institute and Fetzer Franklin Fund, a donor advised fund of Silicon Valley Community Foundation (SD and WHZ, respectively), as well as by the Department of Energy under the LDRD program in Los Alamos. A.T., B.Y. and W.H.Z. also acknowledge support from U.S. Department of Energy, Office of Science, Basic Energy Sciences, Materials Sciences and Engineering Division, Condensed Matter Theory Program, and the Center for Nonlinear Studies. D. G. acknowledges financial support from the Italian Ministry of Research and Education (MIUR), grant number 54$\_$AI20GD01.}


\bibliography{opm}

\end{document}
