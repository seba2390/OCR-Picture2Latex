%%%%XXX\documentclass[aps,pre,twocolumn,notitlepage,superscriptaddress]{revtex4-1}
%\documentclass[aps,prl,superscriptaddress,reprint,groupedaddress]{revtex4-2}


%\documentclass[reprint, superscriptaddress,
%groupedaddress,
%unsortedaddress,
%runinaddress,
%frontmatterverbose, 
%preprint,
%showpacs,
%preprintnumbers,
%nofootinbib,
%nobibnotes,
%bibnotes,
%amsmath,amssymb,
%aps,
%prx
%prb,
%rmp,
%prstab,
%prstper,
%floatfix,
%longbibliography
%]{revtex4-1}

\documentclass[aps,prl,showpacs,amsmath,amssymb,amsfonts,lengthcheck,onecolumn,longbibliography,superscriptaddress]{revtex4-2}

%%%%%\documentclass[aps,pre,showpacs,amsmath,amssymb,amsfonts,superscriptaddress,lengthcheck,onecolumn,preprint]{revtex4-1}
%\documentclass[aps,pre,showpacs,amsmath,amssymb,amsfonts,superscriptaddress,lengthcheck,twocolumn]{revtex4-1}
%\documentclass[10pt]{iopart}
%\documentclass[doublecol]{epl2}
%\usepackage[doublespacing]{setspace}

\usepackage{graphicx}
\usepackage{dsfont}
\usepackage{subfigure}
\usepackage{verbatim}
\usepackage{dcolumn}% Align table columns on decimal point
\usepackage{bm}% bold math
\usepackage{epsf}
\usepackage{color}
\usepackage[toc]{appendix}
\usepackage{stmaryrd}
\usepackage{amsthm}
\usepackage{hhline}
\usepackage{amssymb}
\usepackage{mwe,tikz}\usepackage[percent]{overpic}
%\usepackage{iopams} uncomment these when switching to IOP
%\usepackage{cite}
%\usepackage{doi}
\usepackage{tikz}
\usetikzlibrary{arrows,shapes,calc,matrix}

\expandafter\let\csname equation*\endcsname\relax
\expandafter\let\csname endequation*\endcsname\relax
\usepackage{amsmath}
\usepackage{xstring}

\newcommand{\mymacro}[1]{%
	\StrRight{#1}{1}[\lastletter]%
\lastletter
}

\makeatletter
\newcommand\footnoteref[1]{\protected@xdef\@thefnmark{\ref{#1}}\@footnotemark}
\makeatother

\newcommand\myeqref[1]{
	Eq. (\textup{\ref{#1}})
}
%\newcommand\invisiblesection[1]{%
%	\refstepcounter{section}%
%	\addcontentsline{toc}{section}{\protect\numberline{\thesection}#1}%
%	\sectionmark{#1}}
%%%%\newcommand{\bra}[1]{\left\langle #1\right|}
%%%%\newcommand{\ket}[1]{\left|#1\right\rangle}
\newcommand{\braket}[2]{\left\langle #1|#2\right\rangle}
\newcommand{\ketbrad}[1]{|#1\rangle\!\langle #1|}
\newcommand{\trace}[1]{\mathrm{tr}\left\{#1\right\}}
\newcommand{\ptr}[2]{\mathrm{tr_{#1}}\left\{#2\right\}}
\newcommand{\thickbar}[1]{\mathbf{\bar{\text{$#1$}}}}
\expandafter\let\csname equation*\endcsname\relax
\expandafter\let\csname endequation*\endcsname\relax

%%%Wojciech's commands start
\def\II{1\!\mathrm{l}}
\newcommand{\Tr}        {\mathrm{Tr}}
%\newcommand{\Id}        {I}
\newcommand{\bra}[1]    {\langle #1|}
\newcommand{\ket}[1]    {| #1 \rangle}
\newcommand{\bk}[2]     {\langle #1 | #2 \rangle}
\newcommand{\kb}[2]     {| #1 \rangle \! \langle #2 |}
\newcommand{\cH}        {{\mathcal H}}
\newcommand{\cS}        {{\mathcal S}}
\newcommand{\cA}        {{\mathcal A}}
\newcommand{\cE}        {{\mathcal E}}
\newcommand{\cN}        {{\mathcal N}}
\newcommand{\cO}        {{\mathcal O}}
\newcommand{\cI}        {{\mathcal I}}
\newcommand{\cJ}        {{\mathcal J}}
\newcommand{\+}         {\dagger}
\newcommand{\eend}      {\hspace{\stretch{1}}\rule{1ex}{1ex}}
\def\bA{{\bf A}}
\def\bB{{\bf B}}
\def\bC{{\bf C}}
\def\bX{{\bf X}}
\def\bY{{\bf Y}}
\def\bZ{{\bf Z}}
\def\bH{{\bf H}}
\newcommand\cF{{\mathcal F}}
\newcommand\hocom[1]{}%\marginpar{\center \small $\triangleright$HO}}

\newcommand{\ba}{\begin{eqnarray}}
\newcommand{\ea}{\end{eqnarray}}
\newcommand{\bmath}{\begin{mathletters}}
\newcommand{\emath}{\end{mathletters}}
\newcommand{\ban}{\begin{eqnarray*}}
\newcommand{\ean}{\end{eqnarray*}}
\newcommand{\tl}{\tilde{\ell}}

%%%Wojciech's commands end

%%%to align chi
\DeclareRobustCommand{\rchi}{{\mathpalette\irchi\relax}}
\newcommand{\irchi}[2]{\raisebox{\depth}{$#1\chi$}}
%%% voila

\newcommand{\di}[1]{\mathrm{div}\left\{#1\right\}}
\newcommand{\la}{\left\langle}
\newcommand{\ra}{\right\rangle}
\newcommand{\pd}{\partial}
\newcommand{\de}[1]{\delta\left(#1\right)}
\newcommand{\td}{\mathrm{d}}
\newcommand{\ma}[1]{\max{\left\{#1\right\}}}
\newcommand{\mi}[1]{\min{\left\{#1\right\}}}
\newcommand{\etals}{\textit{et al. }}
\newcommand{\ex}[1]{\exp{\left(#1\right)}}
\newcommand{\loge}[1]{\ln{\left(#1\right)}}
\newcommand{\id}{\mathbb{I}}
\newcommand{\com}[2]{\left[#1,\,#2\right]}
\newcommand{\acom}[2]{\left\{#1,\,#2\right\}}
\newcommand{\co}[1]{\cos{\left(#1\right)}}
\newcommand{\si}[1]{\sin{\left(#1\right)}}
\newcommand{\sh}[1]{\sinh{\left(#1\right)}}
\newcommand{\ch}[1]{\cosh{\left(#1\right)}}
\newcommand{\shi}[1]{\mathrm{shi}{\left(#1\right)}}
\newcommand{\cohi}[1]{\mathrm{chi}{\left(#1\right)}}
\newcommand{\ct}[1]{\coth{\left(#1\right)}}
\newcommand{\bla}{bla\\bla\\bla\\bla\\bla}
%\newcommand{\eqref}[1]{(\ref{#1})}
%\newcommand{\PR}{Phys. Rev.}
\newcommand{\PRA}{Phys. Rev. A }
\newcommand{\PRB}{Phys. Rev. B }
\newcommand{\PRD}{Phys. Rev. D }
\newcommand{\PRE}{Phys. Rev. E }
%\newcommand{\PRL}{Phys. Rev. Lett. }
\newcommand{\PRX}{Phys. Rev. X }
%\newcommand{\EPL}{EPL (Europhys. Lett.) }
%\newcommand{\RMP}{Rev. Mod. Phys. }
%\newcommand{\NJP}{New. J. Phys. }
\usepackage{fmtcount}



\newcommand{\mb}[1]{\mbox{\boldmath$#1$}}
\newcommand{\mc}[1]{\mathcal{#1}}
\newcommand{\mbb}[1]{\mathbb{#1}}
\newcommand{\mf}[1]{\mathfrak{#1}}
\newcommand{\mrm}[1]{\mathrm{#1}}
\newcommand{\clr}{\color{red}}


\newcommand{\ptl}[3]{\left( \frac{\partial {#1}}{\partial {#2}} \right)_{#3}}

\def\dbar{{\mathchar'26\mkern-12mu {\rm d}}}

%\DeclareMathOperator*{\sumint}{%
%\mathchoice%
%  {\ooalign{$\displaystyle\sum$\cr\hidewidth$\displaystyle\int$\hidewidth\cr}}
%  {\ooalign{\raisebox{.14\height}{\scalebox{.7}{$\textstyle\sum$}}\cr\hidewidth$\textstyle\int$\hidewidth\cr}}
%  {\ooalign{\raisebox{.2\height}{\scalebox{.6}{$\scriptstyle\sum$}}\cr$\scriptstyle\int$\cr}}
%  {\ooalign{\raisebox{.2\height}{\scalebox{.6}{$\scriptstyle\sum$}}\cr$\scriptstyle\int$\cr}}
%}
%\renewcommand{\appendixname}{}
%\renewcommand{\appendix}{
%}


% Editing colors
\newcommand{\draftmode}{1}    %to control draft colors below
\newcommand{\notetoself}[1]{\ifnum \draftmode=1 {\color[rgb]{0,0,0.8} [#1]} \fi}  %notes to self in blue when \draftmode==1.  invisible otherwise
\newcommand{\cuttext}[1]{\ifnum \draftmode=1 {\color[rgb]{0,0.5,0} [#1]} \fi}  %cut out text in green when \draftmode==1.  invisible otherwise
\newcommand{\warntext}[1]{\ifnum \draftmode=1 {\color[rgb]{0.9,0.6,0} #1} \else {#1} \color{black} \fi}
%warning text in orange when \draftmode==1.  regular black otherwise
\newcommand{\siren}[1]{{\color[rgb]{0.8,0.0,0} [#1]}}  %for loud error-type text that prints red *always*
\newcommand{\aref}[1]{{Appendix~\hyperref[#1]{A}}}
\newcommand{\bref}[1]{{Appendix~\hyperref[#1]{B}}}
\newcommand{\dref}[1]{{Appendix~\hyperref[#1]{C}}}
\usepackage[colorlinks=true,citecolor=blue,linkcolor=blue,urlcolor=blue]{hyperref}
%\renewcommand{\thesection}{\Roman{section}} 


%\usepackage{hyperref}
\usepackage[capitalise]{cleveref}

%\titlespacing*{\section}{0pt}{1.1\baselineskip}{\baselineskip}

\begin{document}
%\begin{widetext}
\title{Supplemental Material\\
	 Eavesdropping on the Decohering Environment:\\ Quantum Darwinism, Amplification, and the Origin of Objective Classical Reality}

%\title{Watching the watcher --- inevitable classicality of open systems}
\author{Akram Touil}
\email{akramt1@umbc.edu}
\affiliation{Department of Physics, University of Maryland, Baltimore County, Baltimore, MD 21250, USA}
\affiliation{Center for Nonlinear Studies, Los Alamos National Laboratory, Los Alamos, New Mexico 87545}

\author{Bin Yan}
\affiliation{Center for Nonlinear Studies, Los Alamos National Laboratory, Los Alamos, New Mexico 87545}
\affiliation{Theoretical Division, Los Alamos National Laboratory, Los Alamos, New Mexico 87545}

\author{Davide Girolami}
\affiliation{Politecnico di Torino, Corso Duca degli Abruzzi 24, Torino, 10129, Italy}

\author{Sebastian Deffner}
\affiliation{Department of Physics, University of Maryland, Baltimore County, Baltimore, MD 21250, USA}
\affiliation{Instituto de F\'isica `Gleb Wataghin', Universidade Estadual de Campinas, 13083-859, Campinas, S\~{a}o Paulo, Brazil}

\author{Wojciech Hubert Zurek}
\affiliation{Theoretical Division, Los Alamos National Laboratory, Los Alamos, New Mexico 87545}
	% \affiliation command applies to all authors since the last
	% \affiliation command. The \affiliation command should follow the
	% other information
	% \affiliation can be followed by \email, \homepage, \thanks as well.
	%\author{}
	%\email[]{Your e-mail address}
	%\homepage[]{Your web page}
	%\thanks{}
	%\altaffiliation{}
	%\affiliation{}
	
	%Collaboration name if desired (requires use of superscriptaddress
	%option in \documentclass). \noaffiliation is required (may also be
	%used with the \author command).
	%\collaboration can be followed by \email, \homepage, \thanks as well.	%\collaboration{}
	%\noaffiliation
	%\date{\today}
	


\maketitle

%\end{widetext}


%\section*{}
%\bibliographystyle{unsrt}
%\bibliography{opm}
%\begin{thebibliography}{99}
%%%%%%%%%% Merge with supplemental materials %%%%%%%%%%
%\begin{center}
%		\textbf{\Large Eavesdropping on the Decohering Environment: Quantum Darwinism, Amplification, and the Origin of Objective Classical Reality} \vspace{3.5mm}
		
 %   \text{\Large Supplemental Material} \vspace{1.2mm}
    
%	Akram Touil, Bin Yan, Davide Girolami, Sebastian Deffner, Wojciech Hubert Zurek
%\end{center}
%%%%%%%%%% Merge with supplemental materials %%%%%%%%%%
%%%%%%%%%% Prefix a "S" to all equations, figures, tables and reset the counter %%%%%%%%%%
\setcounter{equation}{0}
\setcounter{figure}{0}
\setcounter{table}{0}
\setcounter{page}{1}
\makeatletter
\renewcommand{\theequation}{S\arabic{equation}}
\renewcommand{\thefigure}{S\arabic{figure}}
\renewcommand{\bibnumfmt}[1]{[R#1]}
\renewcommand{\citenumfont}[1]{R#1}
%%%%%%%%%% Prefix a "S" to all equations, figures, tables and reset the counter %%%%%%%
Within the framework of Quantum Darwinism, the emergence of classicality from the quantum substrate is a direct consequence of the redundant imprinting of classical information, about the system of interest $\mc{S}$, in different fragments $\mc{F}$ of the environment. This classical information, obtained by eavesdropping on environmental fragments, is upper bounded by the Holevo quantity $\rchi(\mc{S}:\check{\mc{F}})$ (aka the Holevo bound), which can be expressed as
\begin{equation}
\rchi(\cS : \check \cF)= H_\cS - H_{\cS|\check\cF},
\end{equation}
such that $H_i=-\trace{\rho_i\log_2(\rho_i)}$ stands for the von Neumann entropy, and $H_{\cS|\check\cF}$ is the conditional von Neumann entropy~\cite{nielsen2002quantum} that accounts for the missing information about the system $\cS$ after optimal measurements are applied on the fragment $\cF$.

In the present supplemental material, we provide the technical details leading to analytic expressions of $\rchi(\mc{S}:\check{\mc{F}})$, in the physical model we described in the manuscript~\cite{darwint1}. We separate the supplemental material into five sections. First, we present a general overview of the Koashi-Winter relation as the main tool of our derivation. We then show how it applies to our specific physical model of a central qubit undergoing decoherence in a many-qubit environment. The second part of the supplemental material details the necessary steps leading to analytic expressions of the Holevo bound (in the general case and in the limit of good decoherence). Using this result, we illustrate the statements made in the manuscript regarding the redundancy of information and the breaking of the universal behavior of both the mutual information and the Holevo bound. Finally, in the final section we explicitly explore the connection, through the quantum mutual information, between our many-qubit model and the photon scattering model. The latter connection shows that our analysis captures universal attributes in cases when decoherence is caused by environments composed of noninteracting subsystems such as photons.

\section{The Koashi-Winter relation}
For any tripartite system $ABC$, such that $\rho_{ABC}$ is pure, the Koashi-Winter relation can be written as~\cite{KW}
\begin{equation}
	\rchi(A:\check{B})+E(A:C)=H_A,
	\label{kw}
\end{equation}
where ``$E(A:C)$'' refers to the entanglement of formation~\cite{concurrence1,concurrence2,wootters2001entanglement} of $\rho_{AC}$, which is an entanglement measure for bipartite mixed quantum states (defined as an extension of the entanglement entropy to mixed states). To elaborate, the entanglement of formation ``$E(A:C)$'' quantifies the amount of entanglement in the mixed state $\rho_{AC}$, since for such cases (i.e. for mixed states) the entanglement entropy is no longer a viable measure of entanglement. The latter is due to the fact that the von Neumann entropy is not symmetric for general mixed states~\cite{nielsen2002quantum}.

The Koashi-Winter relation~(\ref{kw}) represents a monogamy relation between the entanglement of formation and the classical correlations in an extended Hilbert space. In other words, this relation captures the trade-off between entanglement and classical correlations. For a given value of the von Neumann entropy $H_{A}$, the higher the entanglement subsystem ``$A$'' shares with ``$C$'', the less classical information about ``$A$'' is accessible through optimal measurements on ``$B$''.

For two-qubit states $\rho_{AC}$, the entanglement of formation can be expressed as a function of the concurrence~\cite{wootters2001entanglement}
\begin{equation}
	E\left(A:C\right)=h\left(\frac{1+\sqrt{1-\left(\mathrm{Con}\left(A:C\right)\right)^{2}}}{2}\right),
\end{equation}
where ``$\mathrm{Con}\left(A:C\right)$'' stands for the concurrence of the state $\rho_{A C}$, and ${h}(x)=-x \log_2(x)-(1-x) \log_2(1-x)$. Furthermore, the entanglement measure~\cite{emeasure,wootters2001entanglement} ``$\mathrm{Con}\left(A:C\right)$'' can be analytically determined by computing the eigenvalues $\mu_i$ of the 4x4 density matrix $\rho_{AC}\tilde{\rho}_{AC}$, where $\tilde{\rho}_{AC}=\left(\sigma_y \otimes \sigma_y \right)  \rho^{*}_{AC} \left(\sigma_y \otimes \sigma_y \right)$. For two-qubit states where $\mu_i \geq \mu_{i+1}$, the analytic formula reads~\cite{concurrence2}
\begin{equation}
	\mathrm{Con}(A:C)=\max \left\{0, \sqrt{\mu_{1}}-\sqrt{\mu_{2}}-\sqrt{\mu_{3}}-\sqrt{\mu_{4}}\right\}.
	\label{con1}
\end{equation}

In the physical model of a central qubit $\cS$ undergoing decoherence in a many-qubit environment $\cE$, assuming that the state of the quantum universe $\mc{S}\mc{E}$ is pure implies that, by decomposing the environment into typical fragments, the state $\rho_{\mc{S}\mc{F}_m\mc{F}_{N-m}}$ is pure. Therefore, in this model, the Koashi-Winter relation is written as
\begin{equation}
	\rchi(\mc{S}:\check{\mc{F}}_m)+E(\mc{S}:\mc{F}_{N-m})=H_{\mc{S}}.
	\label{kwm}
\end{equation}
Based on the above expression, in order to determine the Holevo bound we need to evaluate the entanglement of formation. In fact, given the structure of the branching state of the dynamics~\cite{darwint1}
\begin{equation}
	| \Psi_\mathcal{SE} \rangle= \sqrt{p}  \ket { 0_\mathcal{S}} \bigotimes_{i=1}^{N} \ket { 0_{\cE_{i}}} + \sqrt{q} | 1 _\mathcal{S}\rangle \bigotimes_{i=1}^{N} \ket { 1_{\mathcal{E}_i}} \ ,
\end{equation}
tracing out the degrees of freedom of a given partition of the environment, or the system of interest, results in a density matrix of rank-two at most, i.e. the resulting density matrices are regarded as virtual qubits~\footnote{Here, the notion of virtual qubit is a mathematical construct where the corresponding states can be mapped to a Hilbert space of dimension equal to two.}. This implies that the bipartite system ``$\mc{S}\mc{F}_{N-m}$'' is a qubit-virtual qubit pair. Therefore, we have
\begin{equation}
	E\left(\mc{S}:\mc{F}_{N-m}\right)=h\left(\frac{1+\sqrt{1-\left(\mathrm{Con}\left(\mc{S}:\mc{F}_{N-m}\right)\right)^{2}}}{2}\right).
	\label{enof}
\end{equation}
Additionally, since we are dealing with rank-two (at most) density matrices, when computing the concurrence there are only two nonzero eigenvalues ($\mu_1$ and $\mu_2$), which implies that~\myeqref{con1} simplifies to
\begin{equation}
	\mathrm{Con}\left(\mc{S}:\mc{F}_{N-m}\right)=|\sqrt{\mu_1}-\sqrt{\mu_2}|.
	\label{con2}
\end{equation}

In summary, to determine the Holevo bound $\rchi(\mc{S}:\check{\mc{F}}_m)$ we first need to evaluate $\mu_1$ and $\mu_2$ from the expression of the 4x4 density matrix $\rho_{\mc{S}\mc{F}_{N-m}}$. From these eigenvalues we can determine the concurrence (cf.~\myeqref{con2}), which directly results in an analytic expression for the entanglement of formation (cf.~\myeqref{enof}). Thus, we can infer the general form of the Holevo bound $\rchi(\mc{S}:\check{\mc{F}}_m)$, and the corresponding expression in the good decoherence limit (presented in Equation (20) of the manuscript~\cite{darwint1}).

In what follows, we present the technical details that led to the expressions of $\mu_1$ and $\mu_2$. We start with instructive case of $m=1$, and then generalize to arbitrary values of $m$ such that $1\leq m < N$.

\section{The Holevo bound}
\paragraph{Single qubit of the environment $\mc{F}_1$:}
We split the branching state $| \Psi_\mathcal{SE} \rangle$ into two partitions. The first is composed of the central qubit $\mc{S}$ and a single qubit of $\mc{E}$, while the second partition is the rest of the environment. We adopt the following notations,
\begin{equation}
	\begin{split}
		| \boldsymbol{0} \rangle &\equiv \bigotimes_{i=1}^{N-1} | 0_{\cE_{i}} \rangle,\ \ | \boldsymbol{1} \rangle \equiv \frac{1}{\mc{N}}\left(\bigotimes_{i=1}^{N-1} |  1_{\cE_{i}} \rangle-s^{N-1}\bigotimes_{i=1}^{N-1} |  0_{\cE_{i}} \rangle\right),\\
		| \thickbar{0} \rangle &\equiv|00\rangle ,\ \ | \thickbar{1} \rangle \equiv s|10\rangle+c|11\rangle,
		\label{basis1}
	\end{split}
\end{equation}
with $\mc{N}=\sqrt{1-s^{2(N-1)}}$, and
\begin{equation}
(\forall i \in \llbracket 1, N\rrbracket); 
\ \	|  0_{\cE_{i}} \rangle \equiv | 0 \rangle ,\ | 1_{\cE_{i}} \rangle \equiv s | 0 \rangle + c | 1 \rangle.
\end{equation}
The real parameters $c$ and $s$, quantify the degree with which the environmental qubits monitor the central qubit $\cS$. From the above definitions, we have $\langle \boldsymbol{i}|\boldsymbol{j} \rangle=\delta_{i,j}$ and $\langle \thickbar{i}|\thickbar{j} \rangle=\delta_{i,j}$ for $(i,j) \in \{0,1\}^2$, and the state of the universe $\mc{S}\mc{E}$ can be written as
\begin{equation}
		| \Psi_\mathcal{SE} \rangle = \sqrt{p} |\thickbar{0}\boldsymbol{0}\rangle+\sqrt{q}s^{N-1}|\thickbar{1}\boldsymbol{0}\rangle+\sqrt{q}\sqrt{1-s^{2(N-1)}}|\thickbar{1}\boldsymbol{1}\rangle.
\end{equation}
Therefore, the rank-two density matrix representing the composite state of the central qubit $\mc{S}$ and a single qubit of the environment $\mc{F}_1$ is
\begin{equation}
	\rho_{\mc{S}\mc{F}_1}=\begin{pmatrix}
		p && 0 && s^{N}\sqrt{pq} && s^{N-1}c\sqrt{pq} \\
		0 && 0 && 0 && 0\\
		s^{N}\sqrt{pq}&& 0 && s^2 q && s c q\\
		s^{N-1}c\sqrt{pq}&& 0 && s c q && c^2 q
	\end{pmatrix}.
\end{equation}
Now we generalize to the case of an arbitrary environmental fragment $\cF_m$ (with $m$ qubits) in order to determine the density matrix $\rho_{\mc{S}\mc{F}_m}$, for any $1\leq m < N$, which directly implies the expression of $\rho_{\mc{S}\mc{F}_{N-m}}$ and the corresponding eigenvalues $\mu_1$ and $\mu_2$.

\paragraph{Environmental fragment $\mc{F}_m$:}
Considering a partition of the environment (with $m$ qubits), and similar to the previous single qubit analysis (cf.~\myeqref{basis1}), we group the central qubit with ``$m$'' qubits from the environment to get
\begin{equation}
	| \thickbar{0} \rangle \equiv|0\rangle \bigotimes_{i=1}^{m} | 0_{\cE_{i}} \rangle,\ | \thickbar{1} \rangle \equiv |1\rangle \bigotimes_{i=1}^{m} |  1_{\cE_{i}} \rangle,
\end{equation}
which in turn can be written as a basis for ``the qubit-virtual qubit'' pair
\begin{equation}
	| \thickbar{0} \rangle \equiv|0\mathfrak{0}\rangle ,\ | \thickbar{1} \rangle \equiv |1\rangle\left( s^{m}|\mathfrak{0}\rangle+\sqrt{1-s^{2m}}|\mathfrak{1}\rangle\right),
	\label{basism}
\end{equation}
with
\begin{equation}
	|\mathfrak{0} \rangle \equiv \bigotimes_{i=1}^{m} | 0_{\cE_{i}} \rangle,\  | \mathfrak{1} \rangle \equiv \frac{1}{\mc{M}}\left(\bigotimes_{i=1}^{m} |  1_{\cE_{i}} \rangle-s^{m}\bigotimes_{i=1}^{m} |  0_{\cE_{i}} \rangle\right),
\end{equation}
such that $\mc{M}=\sqrt{1-s^{2m}}$ and $\langle \mathfrak{i}|\mathfrak{j} \rangle=\delta_{\mathfrak{i},\mathfrak{j}}$. Following this ``qubit-virtual qubit'' decomposition, we can infer the expression of the rank-two density matrix $\rho_{\mc{S}\mc{F}_{N-m}}$,
\begin{equation}
	\rho_{\mc{S}\mc{F}_{N-m}}=\begin{pmatrix}
		p && 0 && s^{N}\sqrt{pq} && s^{m}\sqrt{1-s^{2(N-m)}}\sqrt{pq} \\
		0 && 0 && 0 && 0\\
		s^{N}\sqrt{pq}&& 0 &&  qs^{2(N-m)} && qs^{N-m}\sqrt{1-s^{2(N-m)}}\\
		s^{m}\sqrt{1-s^{2(N-m)}}\sqrt{pq}&& 0 && qs^{N-m}\sqrt{1-s^{2(N-m)}} && q(1-s^{2(N-m)})
	\end{pmatrix}.
\end{equation}
The eigenvalues, $\mu_1$ and $\mu_2$, of the matrix $\rho_{\mc{S}\mc{F}_{N-m}}\tilde{\rho}_{\mc{S}\mc{F}_{N-m}}$, are
\begin{equation}
	\mu_1= pq\left(1-s^{m}\right)^{2}(1-s^{2(N-m)}), \ \ \mu_2= pq\left(1+s^{m}\right)^{2}(1-s^{2(N-m)}).
\end{equation}
From the above eigenvalues, and based on~\cref{con2,enof,kwm}, we get the general analytic expression for the Holevo bound,
\begin{equation}
	\rchi(\mc{S}:\check{\mc{F}}_m)= h\left(r_p\right)-h\left(t_{p,m}\right),
	\label{chi1}
\end{equation}
where
\begin{equation}
	r_p= \frac{1}{2}\left(1+\sqrt{1-4pq\left(1-s^{2N}\right)}\right) \ \ \text{and} \ \ t_{p,m}=\frac{1}{2}\left(1+\sqrt{1-4pq\left(s^{2m}-s^{2N}\right)}\right).
\end{equation}
From this expression, it is a simple exercise to show that for good decoherence (i.e. in the limit $s^{N-m} \ll  s^m$) we have the following simplification
\ba
\rchi(\mc{S}:\check \cF_m)= H_\cS +\frac 1 2 \log_{2}\left(pqs^{2m}\right) +\sqrt{1-4pqs^{2m}}\operatorname{Arctanh}_2\left(\sqrt{1-4pqs^{2m}}\right),
\ea
such that $H_{\mc{S}}=-p \log_2(p)-q \log_2(q)$ is the von Neumann entropy of the decohered state of the central qubit $\mc{S}$.

In the subsequent two sections, based on the above result, we will illustrate the following statements made in the manuscript. The first statement is that the redundancy evaluated through the quantum mutual information $\cI$ overestimates the redundancy from the viewpoint of observers eavesdropping on fragments $\cF$ of the environment via optimal measurements. The difference between the two quantities can be computed numerically. The second statement concerns the weak dependence of the re-scaled quantum mutual information and Holevo bounds on the state preparation of the central qubit $\cS$, i.e. on the probability of the pointer states ``$p$''.
 
\section{Redundancy}

In~\cref{Red}, we illustrate the behavior of the redundancy $R_{\delta}=N/m_{\delta}$ (with information deficit $\delta=0.1$), evaluated using $I$ (red curve) and $\rchi$ (black curve), as a function of the parameter $c$. We observe that the redundancy is monotonically increasing as a function of $c$, such that the maximum redundancy is achieved when $c \rightarrow 1$. The latter is due to the fact that in the limit where $c \rightarrow 1$ the coupling between the central qubit and each qubit from the environment is now modeled by the {\tt c-not} gate, and the branching state is a GHZ state. We also observe that for almost all values of $c \in \ [0,1]$ the redundancy computed using the quantum mutual information is an overestimate to the redundancy observed by eavesdropping on fragments of the environment through optimal measurements.

\begin{figure}
	\includegraphics[width=0.58\textwidth]{redund.pdf}
	\caption{\label{Red}Plots of the redundancy $R_{0.1}$ computed through the mutual information $I$ (red curve) and the Holevo bound $\chi$ (black curve), as a function of $c$, for $p=1/2$ and $N=100$.}
\end{figure}

Using the analytic expressions of the mutual information and the Holevo bound, we can numerically determine the degree with which the redundancy obtained from the mutual information overestimates the actual redundancy. We can define $\Delta m= m^{\chi}-m^{I}$, such that $m^{\chi}$ ($m^{I}$) represents the minimum number of environmental qubits needed in order for the Holevo bound (mutual information) to reflect the missing information about the central qubit $\mc{S}$, up to an information deficit $\delta$, i.e., $(1-\delta)H_{\mc{S}}$. Therefore, the quantity $\Delta m/m^{\chi}$ reflects the difference between the redundancies obtained from the two different information theoretic measures ($I$ and $\chi$). In Fig.~\ref{deltam2}, we illustrate, in the limit of large $N$ ($N \rightarrow \infty$), $\Delta m/m^{\chi}$ as a function of $\delta$ and $p$. In panel (a), we observe that the maximum difference, for a given value of $\delta$, is attained for $p=1/2$. From panel (b), in the limit where $\delta \rightarrow 0$ we conclude that the redundancies differ by at most $\sim 13\%$, and for $\delta=0.2$ (similar to the case presented in the inset of Figure 1 of the manuscript~\cite{darwint1}) the difference is at most $\sim 37\%$.
\begin{figure}[h!]
	\centering
	\subfigure[]{
		\includegraphics[width=.485\textwidth]{Deltam3D.pdf}
	}
	\subfigure[]{
		\includegraphics[width=.485\textwidth]{Deltam2D.pdf}
	}
	\caption[]{\label{deltam2}Plots of $\Delta m /m^{\chi}$ as a function of $\delta \in [0,0.2]$ and $p$. In panel (a), we plot $\Delta m /m^{\chi}$ for all values of $p \in [0.01,0.99]$, and in panel (b) we focus on the case of $p=1/2$.}
\end{figure}


\section{Breaking of universality}
In the manuscript, we show that the quantum mutual information and the Holevo bound display a universal scaling behavior which is weakly dependent on the state preparation of the central qubit $\cS$. Namely, through appropriate re-scaling, the two information theoretic quantities exhibit a universal rising behavior insensitive to the value of the parameter $p$, up to some limitations. Here, our goal is to quantify and illustrate these limitations, which shows the breaking of the aforementioned universality when the parameter $p$ approaches a value of zero or one. To this end, we choose the case of the initial equal superposition ($p=1/2$) as a reference, and we compare its corresponding expressions of the mutual information and the Holevo bound with those for arbitrary $p^{\prime}$, which leads to defining $\Delta_{{I}}$ and $\Delta_{\rchi}$ as follows.
\begin{equation}
	\Delta_{{I}}={I}_{p=1/2}(\mc{S}:\mc{F}_m)-\frac{{I}_{p^{\prime}}(\mc{S}:\mc{F}_m)}{H_{p^{\prime}}(\mc{S})},
	\label{}
\end{equation}
such that ${I}_{p=1/2}(\mc{S}:\mc{F}_m)$ and ${I}_{p^{\prime}}(\mc{S}:\mc{F}_m)$ are the values of the mutual information for $p=1/2$ and arbitrary $p^{\prime}\in [0,1]$, respectively. The quantity $H_{p^{\prime}}(\mc{S})$ is the maximum von Neumann entropy of the system $\mc{S}$ corresponding to $p^{\prime}$ (i.e. the plateau of the curve of ${I}_{p^{\prime}}(\mc{S}:\mc{F}_m)$). Similarly, for the Holevo bound we define the quantity $\Delta_{\rchi}$ such that
\begin{equation}
	\Delta_{\rchi}=-\rchi_{p=1/2}(\mc{S}:\check{\mc{F}}_m)+\frac{\rchi_{p^{\prime}}(\mc{S}:\check{\mc{F}}_m)}{H_{p^{\prime}}(\mc{S})}.
	\label{}
\end{equation}
In~\cref{delta}, we plot $\Delta_{{I}}$ and $\Delta_{\rchi}$ as functions of both $m$ and $p^{\prime}$. For small $m$, the breaking of the universal rise of both the mutual information and the Holevo bound is observed when $p^{\prime}$ is either close to zero or one, otherwise $\Delta_{{I}} \approx 0$ and $\Delta_{\rchi} \approx 0$. For completeness, we also plot our quantities for large values of $m$ (i.e. when the environment fraction approaches one). In this limit, $\Delta_{\rchi} = 0$ for all $p^{\prime}$ while $\Delta_{{I}} \neq 0$ for $p^{\prime}$ close to zero or one. This is due to the fact that the quantum mutual information rises beyond the plateau when we capture almost all of the environment, which is a direct consequence of the purity of the state of the universe $\mc{S}\mc{E}$.


\begin{figure}[h!]
	\centering
	\subfigure[]{
		\includegraphics[width=.6\textwidth]{break1.pdf}
	}
	\subfigure[]{
		\includegraphics[width=.6\textwidth]{break2.pdf}
	}
	\caption[]{\label{delta}Plots of $\Delta_{{I}}$ (panel (a)) and $\Delta_{\rchi}$ (panel (b)), as a function of $m$ and $p^{\prime}$, for $N=100$ and $c=\sqrt{0.4}$. Note that changing the value of the parameter $c$ for a fixed total number of qubits $N$ is equivalent to a straightforward re-scaling of the x-axis of our plots (displaying the number of qubits ``$m$'' in a typical environmental fragment), with $\log(s)$ as the re-scaling factor~\cite{darwint1}. It is also noteworthy that the initial rise of the mutual information or the Holevo bound is independent of the total number of qubits $N$ in our environment~\cite{darwint1}.}
\end{figure}

\section{Connection to the photon model}
As mentioned towards the end of the manuscript~\cite{darwint1}, our results naturally extend to the realistic model of a photon environment, as studied in Refs.~\cite{Riedel2010PRL,Riedel2011NJP}. In fact, the quantum mutual information expression is the same as the one derived in the photon scattering model. More specifically, if we consider a dielectric sphere, as our system of interest $\mc{S}$, initially in a spatial superposition such that $| \psi_{\mc{S}} \rangle =\sqrt{p} \delta\left(\vec{x}-\vec{x}_{1}\right)+\sqrt{q} \delta\left(\vec{x}-\vec{x}_{2}\right)$. The environment is composed of $N$ photons, originally emitted from a point source (for simplicity we assume point source illumination). In this case, it was shown~\cite{Riedel2010PRL,Riedel2011NJP} that the quantum mutual information has the following form,
\begin{equation}
	I (\mathcal{S}: \mathcal{F}_m)=1+\frac{1}{\log(2)}\sum_{i=1}^{\infty} \frac{\Gamma^{(1-f) i}-\Gamma^{f i}-\Gamma^{i}}{2 i(2 i-1)},
	\label{phot}
\end{equation}
where $\Gamma^{x}= (p-q)^{2}+4pq \exp(-tx/\tau_D)$ such that $x \in \{1,f,1-f\}$, $\tau_D$ is the decoherence time, and $f=m/N$ is the fraction of photons we access. The above expression is in direct agreement with our the results~\cite{darwint1}. In particular, from the specific form of equations (5-7) of the manuscript we can compute the analytic expression of the von Neumann entropies of the states $\rho_{\mc{S}}$, $\rho_{\mc{F}_m}$, and $\rho_{\mc{S}\mc{F}_m}$,
\begin{equation}
	H_{\mc{S}}=-\lambda_{-}\left(N, p\right) \log_{2} \left(\lambda_{-}(N,p)\right)-\lambda_{+}(N,p) \log_{2} \left(\lambda_{+}(N,p)\right),
\end{equation}
\begin{equation}
	H_{\mc{F}_m}=-\lambda_{-}(m,p) \log_{2} \left(\lambda_{-}(m,p)\right)-\lambda_{+}(m,p) \log_{2} \left(\lambda_{+}(m,p)\right),
\end{equation}
\begin{equation}
	H_{\mc{S}\mc{F}_m}=-\lambda_{-}(N-m,p) \log_{2} \left(\lambda_{-}(N-m,p)\right)-\lambda_{+}(N-m,p) \log_{2} \left(\lambda_{+}(N-m,p)\right),
\end{equation}
with $\lambda_{\pm}(k,p)=\frac{1}{2}\left(1 \pm \sqrt{\left(q-p\right)^{2}+4s^{2k}pq}\right)$. Generally, we have, for $|x|<1$,
\begin{equation}
	\log_{2}(1+x)=\frac{1}{\log(2)}\sum_{i=1}^{\infty}(-1)^{i+1} \frac{x^{i}}{i}, \ \ \text{and} \ \ \log_{2}(1-x)=-\frac{1}{\log(2)}\sum_{i=1}^{\infty} \frac{x^{i}}{i}.
\end{equation}
Therefore,
\begin{equation}
	-(1-x)\log_{2}(1-x)-(1+x)\log_{2}(1+x)=-\frac{2}{\log(2)}\sum_{i=1}^{\infty} \frac{x^{2i}}{2i(2i-1)}.
\end{equation}
It is, then, a straightforward exercise to get
\begin{equation}
	I (\mathcal{S}: \mathcal{F}_m)=1+\frac{1}{\log(2)}\left(\sum_{i=1}^{\infty} \frac{X_{N-m}^{2i}-X_{m}^{2i}-X_{N}^{2i}}{2i(2i-1)}\right),
	\label{series}
\end{equation}
such that
\begin{equation}
	X_i=\sqrt{\left(p-q\right)^2+4s^{2i}pq} \leq 1.
\end{equation}
Comparing this expression of the quantum mutual information with that of the photon model (cf.~\myeqref{phot}) we recognize the exact similarity such that $s^{N} \equiv \exp(-t/2\tau_D)$. Therefore, the universal rise (one of our main results) of the quantum mutual information also applies to the realistic photon scattering model.

%\newpage
%\pagebreak
%\section*{}
\bibliography{opm2}

\end{document}
