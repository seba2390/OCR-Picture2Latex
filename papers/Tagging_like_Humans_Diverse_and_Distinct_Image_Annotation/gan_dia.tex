\documentclass[10pt,twocolumn,letterpaper]{article}

\usepackage{cvpr}
\usepackage{times}
\usepackage{epsfig}
\usepackage{graphicx}
\usepackage{amsmath}
\usepackage{amssymb}

\usepackage{multirow}

\renewcommand{\u}{\mathbf{u}}
\newcommand{\w}{\mathbf{w}}
\newcommand{\x}{\mathbf{x}}
\newcommand{\y}{\mathbf{y}}
\newcommand{\z}{\mathbf{z}}
\renewcommand{\d}{\mathbf{d}}
\newcommand{\E}{\mathcal{E}}
\newcommand{\A}{\mathbf{A}}
\newcommand{\B}{\mathbf{B}}
\newcommand{\C}{\mathbf{C}}
\newcommand{\I}{\mathbf{I}}
\newcommand{\Y}{\mathbf{Y}}
\newcommand{\X}{\mathbf{X}}
\newcommand{\Z}{\mathbf{Z}}
\newcommand{\W}{\mathbf{W}}
\newcommand{\U}{\mathbf{U}}
\newcommand{\V}{\mathbf{V}}
\newcommand{\Q}{\mathbf{Q}}

\newcommand{\D}{\mathcal{D}}
\newcommand{\G}{\mathcal{G}}

\newcommand{\tr}{\mathrm{tr}}

\newcommand{\red}[1]{\textcolor{red} {#1}}
\newcommand{\blue}[1]{\textcolor{blue} {#1}}

\long\def\comment#1{}

% Include other packages here, before hyperref.

% If you comment hyperref and then uncomment it, you should delete
% egpaper.aux before re-running latex.  (Or just hit 'q' on the first latex
% run, let it finish, and you should be clear).
%\usepackage[pagebackref=true,breaklinks=true,letterpaper=true,colorlinks,bookmarks=false]{hyperref}
%
%\cvprfinalcopy % *** Uncomment this line for the final submission
%
%%\def\cvprPaperID{1182} % *** Enter the CVPR Paper ID here
%%\def\httilde{\mbox{\tt\raisebox{-.5ex}{\symbol{126}}}}
%
%% Pages are numbered in submission mode, and unnumbered in camera-ready
%\ifcvprfinal\pagestyle{empty}\fi
%
%\usepackage[pagebackref=true,breaklinks=true,letterpaper=true,colorlinks,bookmarks=false]{hyperref}

\cvprfinalcopy % *** Uncomment this line for the final submission

\def\cvprPaperID{1182} % *** Enter the CVPR Paper ID here
\def\httilde{\mbox{\tt\raisebox{-.5ex}{\symbol{126}}}}

% Pages are numbered in submission mode, and unnumbered in camera-ready
\ifcvprfinal\pagestyle{empty}\fi

\newcommand\Mark[1]{\textsuperscript#1}
\usepackage{authblk}

\begin{document}

%%%%%%%%% TITLE
\title{Tagging like Humans: Diverse and Distinct Image Annotation}

%\begingroup
%\centering
%{\LARGE The Title \\[1.5em]
%\large First Author\Mark{1}, Second Author\Mark{2}, Third Author\Mark{1}, Fourth Author\Mark{2} and Fifth Author\Mark{3}}\\[1em]
%\begin{tabular}{*{3}{>{\centering}p{.25\textwidth}}}
%\Mark{1}Department1 & \Mark{2}Department2 & \Mark{3}Department3 \tabularnewline
%School1 & School2 & School3 \tabularnewline
%\url{email1} & \url{email2} & \url{email3}
%\end{tabular}\par
%\endgroup

%\begingroup
%Baoyuan Wu\Mark{1}, Weidong Chen\Mark{1}, Peng  Sun\Mark{2}
%%\begin{tabular}{*{1}{>{\centering}p{.9\textwidth}}}
%%%\textsuperscript#1 Tencent AI Lab, \textsuperscript#2 KAUST
%% Tencent AI Lab, KAUST
%%\end{tabular}\par
%\begin{tabular}{*{3}{>{\centering}p{.25\textwidth}}}
%\Mark{1}Department1 & \Mark{2}Department2 & \Mark{3}Department3 \tabularnewline
%School1 & School2 & School3 \tabularnewline
%\url{email1} & \url{email2} & \url{email3}
%\end{tabular}\par
%\endgroup


\author[1]{Baoyuan Wu}
\author[1]{Weidong Chen}
\author[1]{Peng  Sun}
\author[1]{Wei Liu}
\author[2]{Bernard  Ghanem}
\author[3]{Siwei Lyu}

\affil[1]{Tencent AI Lab ~ \Mark{2}KAUST ~ \Mark{3}University at Albany, SUNY 
\authorcr{\tt\small  wubaoyuan1987@gmail.com, powerchen@tencent.com,  pengsun000@gmail.com, wliu@ee.columbia.edu, bernard.ghanem@kaust.edu.sa, slyu@albany.edu}}
%\affil[2]{KAUST  {\tt\small } }
%\affil[3]{University at Albany, SUNY \authorcr{\tt\small  wubaoyuan1987@gmail.com, powerchen@tencent.com,  pengsun000@gmail.com, wliu@ee.columbia.edu, bernard.ghanem@kaust.edu.sa, slyu@albany.edu}}


\comment{
\author{Baoyuan Wu,  Weidong Chen, Peng  Sun, Wei Liu, Bernard  Ghanem, Siwei Lyu}
\affil{Tencent AI Lab,  Tencent AI Lab, Tencent AI Lab, Tencent AI Lab, KAUST, University at Albany, SUNY \authorcr{wubaoyuan1987@gmail.com}
}
}


%
%\author{Baoyuan Wu\\
%Tencent AI Lab\\
%Shenzhen, China\\
%{\tt\small wubaoyuan1987@gmail.com}
%% For a paper whose authors are all at the same institution,
%% omit the following lines up until the closing ``}''.
%% Additional authors and addresses can be added with ``\and'',
%% just like the second author.
%% To save space, use either the email address or home page, not both
%\and
%Weidong Chen\\
%Tencent AI Lab\\
%Shenzhen, China\\
%{\tt\small powerchen@tencent.com}
%\and
%Peng  Sun\\
%Tencent AI Lab\\
%Shenzhen, China\\
%{\tt\small pengsun000@gmail.com}
%\and
%Wei Liu\\
%Tencent AI Lab\\
%Shenzhen, China\\
%{\tt\small wliu@ee.columbia.edu}
%\and
%Bernard  Ghanem\\
%KAUST\\
%Saudi Arabia\\
%{\tt\small bernard.ghanem@kaust.edu.sa}
%\and
%Siwei  Lyu\\
%University at Albany, SUNY\\
%Albany, NY, USA\\
%{\tt\small slyu@albany.edu}
%}

\maketitle
\thispagestyle{empty}

%  In this paper, we explore the connection between secret key agreement and secure omniscience within the setting of the multiterminal source model with a wiretapper who has side information. While the secret key agreement problem considers the generation of a maximum-rate secret key through public discussion, the secure omniscience problem is concerned with communication protocols for omniscience that minimize the rate of information leakage to the wiretapper. The starting point of our work is a lower bound on the minimum leakage rate for omniscience, $\rl$, in terms of the wiretap secret key capacity, $\wskc$. Our interest is in identifying broad classes of sources for which this lower bound is met with equality, in which case we say that there is a duality between secure omniscience and secret key agreement. We show that this duality holds in the case of certain finite linear source (FLS) models, such as two-terminal FLS models and pairwise independent network models on trees with a linear wiretapper. Duality also holds for any FLS model in which $\wskc$ is achieved by a perfect linear secret key agreement scheme. We conjecture that the duality in fact holds unconditionally for any FLS model. On the negative side, we give an example of a (non-FLS) source model for which duality does not hold if we limit ourselves to communication-for-omniscience protocols with at most two (interactive) communications.  We also address the secure function computation problem and explore the connection between the minimum leakage rate for computing a function and the wiretap secret key capacity.
  
%   Finally, we demonstrate the usefulness of our lower bound on $\rl$ by using it to derive equivalent conditions for the positivity of $\wskc$ in the multiterminal model. This extends a recent result of Gohari, G\"{u}nl\"{u} and Kramer (2020) obtained for the two-user setting.
  
   
%   In this paper, we study the problem of secret key generation through an omniscience achieving communication that minimizes the 
%   leakage rate to a wiretapper who has side information in the setting of multiterminal source model.  We explore this problem by deriving a lower bound on the wiretap secret key capacity $\wskc$ in terms of the minimum leakage rate for omniscience, $\rl$. 
%   %The former quantity is defined to be the maximum secret key rate achievable, and the latter one is defined as the minimum possible leakage rate about the source through an omniscience scheme to a wiretapper. 
%   The main focus of our work is the characterization of the sources for which the lower bound holds with equality \textemdash it is referred to as a duality between secure omniscience and wiretap secret key agreement. For general source models, we show that duality need not hold if we limit to the communication protocols with at most two (interactive) communications. In the case when there is no restriction on the number of communications, whether the duality holds or not is still unknown. However, we resolve this question affirmatively for two-user finite linear sources (FLS) and pairwise independent networks (PIN) defined on trees, a subclass of FLS. Moreover, for these sources, we give a single-letter expression for $\wskc$. Furthermore, in the direction of proving the conjecture that duality holds for all FLS, we show that if $\wskc$ is achieved by a \emph{perfect} secret key agreement scheme for FLS then the duality must hold. All these results mount up the evidence in favor of the conjecture on FLS. Moreover, we demonstrate the usefulness of our lower bound on $\wskc$ in terms of $\rl$ by deriving some equivalent conditions on the positivity of secret key capacity for multiterminal source model. Our result indeed extends the work of Gohari, G\"{u}nl\"{u} and Kramer in two-user case.

%\section{Introduction}

A powerful, model-independent framework to constrain, identify,  and parametrise potential
deviations with respect to the predictions of the Standard
Model (SM) is
provided by the Standard Model Effective Field
Theory (SMEFT)~\cite{Weinberg:1979sa,Buchmuller:1985jz,Grzadkowski:2010es}, see also~\cite{Brivio:2017vri}
for a review.
%
A particularly attractive feature of the SMEFT is its capability to systematically correlate
deviations from the SM between different processes, for example between  Higgs 
and top quark cross-sections, or between high-$p_T$ and flavor observables.

A direct consequence of this model independence  is the high dimensionality
of the parameter space spanned by the relevant higher-dimensional EFT operators.
%
Indeed, the number of Wilson coefficients
constrained in  typical SMEFT analyses can vary between
just a few up to the several tens or even hundreds, depending
on the specific  assumptions adopted concerning
the flavour, family (non-)universality of the couplings, and CP-symmetry
structure (among others) of the UV-complete theory.
%
For this reason, the full exploitation of the SMEFT potential
for indirect New Physics searches from precision measurements requires
combining the information provided by the broadest possible dataset.

The phenomenology of the SMEFT has attracted significant
 attention, with most analyses focusing on specific sectors
of the parameter space and groups of processes.
%
Some of these recent studies have targeted the top quark
properties~\cite{Buckley:2016cfg,Buckley:2015lku,Hartland:2019bjb,Brivio:2019ius},
the Higgs and electroweak gauge sector~\cite{Biekotter:2018rhp,Ellis:2018gqa,Almeida:2018cld},
single and double gauge boson
production~\cite{Baglio:2020ibv,Alioli:2018ljm,Ethier:2021ydt,Greljo:2017vvb},
vector-boson scattering~\cite{Gomez-Ambrosio:2018pnl,Ethier:2021ydt,Dedes:2020xmo},
and flavour and low-energy
observables~\cite{Aebischer:2018iyb,Falkowski:2019xoe,Falkowski:2017pss}, among
several others.
%
Furthermore, analyses that combine the constraints of different
groups of processes in the EFT parameter space, such as the Higgs and electroweak sector with
the top quark one~\cite{Ellis:2020unq} or 
top quark data with $B$-meson observables~\cite{Bissmann:2020mfi,Bruggisser:2021duo},
have also been presented.
%
These and related studies demonstrate that a
global interpretation of the SMEFT is unavoidable and makes possible
benefiting from hitherto unexpected connections, such as the  correlation
of the LHCb flavour anomalies~\cite{Pich:2019pzg,Aaij:2021vac}
at the $B$-meson scale with the high-$p_T$ tails
at the LHC~\cite{Greljo:2017vvb,Fuentes-Martin:2020lea}.

With the ultimate motivation of performing a truly global EFT interpretation
of particle physics data, the {\tt SMEFiT} fitting framework was developed in~\cite{Hartland:2019bjb}
and applied to the analysis of the top quark properties at the LHC as a proof-of-concept.
%
This novel EFT fitting methodology, inspired by techniques deployed by the NNPDF
Collaboration to
determine the proton's parton distribution
functions (PDFs)~\cite{Ball:2008by,Ball:2010de,Ball:2014uwa,Rojo:2018qdd,Forte:2020yip},
made possible constraining the Wilson coefficients
associated to 34 independent dimension-six
operators that modify the production cross-sections of top
quarks.
%
Our results improved over
existing bounds~\cite{AguilarSaavedra:2018nen} for the wide majority of
directions in the SMEFT parameter space and in several
cases the associated Wilson coefficients were constrained for the first time.
%
Subsequently, {\tt SMEFiT} was extended with the Bayesian reweighting method~\cite{vanBeek:2019evb}
developed for PDFs~\cite{Ball:2011gg,Ball:2010gb} which  allows one
constraining the EFT parameter space {\it a posteriori}
with novel measurements without requiring a dedicated fit.
%
 {\tt SMEFiT} has also been recently applied for the first SMEFT interpretation of vector boson
 scattering data~\cite{Ethier:2021ydt} from the full Run II dataset.

In this work, we complement and
extend the {\tt SMEFiT} analysis framework of~\cite{Hartland:2019bjb} in several directions.
%
First and foremost, we extend the dimension-six EFT operator basis
in order to simultaneously describe top-quark
measurements together with Higgs boson production and decay cross-sections,
as well as with weak gauge boson pair production from LEP and the LHC.
%
Specifically, we consider Higgs signal strengths, differential distributions,
and simplified template cross-section (STXS) measurements from 
ATLAS and CMS taken at Runs I and II.
%
Furthermore, we account for the most recent top-quark observables
from the Run II dataset, such as updated measurements of four-top,
top quark pair in association with a $Z$ boson, and differential
single-top and top quark pair production.
%
We also include the differential distributions
in gauge boson pair production from LEP
and the LHC, which constrain complementary directions in the EFT space.
 %
In addition, we account in an indirect manner for the information
provided by electroweak precision observables (EWPO) from
LEP~\cite{ALEPH:2005ab}
by means of imposing restrictions on specific combinations
of the EFT coefficients.

A second improvement as compared to~\cite{Hartland:2019bjb}
concerns the fitting methodology.
%
On the one hand, the  Monte Carlo replica fitting method has been
upgraded by means of more efficient optimizers and the imposition of
post-fit quality selection criteria for the replicas.
%
On the other hand, we have implemented a novel, independent
approach to constrain the  parameter space based on
Nested Sampling (NS) by means of the MultiNest algorithm~\cite{Feroz:2013hea}.
%
As opposed to the replica fitting method, which is an
optimisation problem,
NS aims to reconstruct the posterior probability distribution given
the model and the data by means of Bayesian inference.
%
We have cross-validated the performance
of  the two methods and demonstrated that they
lead to equivalent results.
%
The availability of two orthogonal fitting strategies
strengthens the robustness of {\tt SMEFiT} and facilitates the
combined interpretation of data from different processes.

From the combination of the  improved fitting framework
and the extensive input dataset, we 
derive individual, two-dimensional, and global (marginalised) bounds for  36
independent directions (and 14 dependent ones) in the EFT parameter space.
%
The EFT cross-sections used in this analysis account for
either only the linear
or for both linear and quadratic effects, $\mathcal{O}\lp \Lambda^{-2}\rp$ and
$\mathcal{O}\lp \Lambda^{-4}\rp$ respectively, and
include NLO QCD corrections whenever available.
%
We demonstrate in detail how the inclusion of NLO QCD
and $\mathcal{O}\lp \Lambda^{-4}\rp$ corrections
in the EFT calculations
is instrumental in order to accurately pin down the  posterior distributions associated
to the fitted Wilson coefficients.

By means of information geometry and principal component analysis techniques,
we quantify the sensitivity of each
of the input datasets to the various Wilson coefficients.
%
We validate these statistical diagnosis tools by means of a series of  fits restricted
to subsets of processes, such as Higgs-only and top-only EFT analyses.
%
Specifically, we quantify the interplay between the top-quark and Higgs measurements
in the determination of EFT degrees of freedom sensitive to both processes,
such as the modifications of the top Yukawa coupling.
%
Furthermore, we explore how the EFT fit results are modified when additional, UV-inspired theory restrictions
are imposed in the parameter space, and present results
for the case of a top-philic model.

The paper is organised as follows.
%
First of all, Sect.~\ref{sec:smefttheory} discusses the operator basis,
flavour assumptions, the fitted degrees of freedom,
and the top-philic scenario.
%
Then Sect.~\ref{sec:settings_expdata} describes the top-quark, Higgs,
and diboson datasets that are used as input to the analysis together
with the corresponding SM and EFT calculations.
%
The methodological improvements in {\tt SMEFiT}, together with the description of  the fit settings,
are presented in Sect.~\ref{sec:fitsettings}.
%
The main results of this work, namely the combined SMEFT
interpretation of top-quark, Higgs, and diboson measurements at the LHC,
are presented and discussed in Sect.~\ref{sec:results}.
%
Finally, in Sect.~\ref{sec:summary} we summarise and discuss future steps
in this project.

Supplementary information is provided in three appendices.
%
In App.~\ref{sect:app_comparison_data} we present
the comparison between the SM and SMEFT theory predictions
with the experimental datasets used as input to
the fit;
in App.~\ref{sec:signalstrenghts} we describe the implementation
of  the Higgs signal strength measurements;
{  in App.~\ref{sec:fullcovmat} we present the 
  correlation matrices for the complete set of operators
considered in the analysis; and}
%
then
in App.~\ref{sec:delivery} we discuss how
the results of this work are rendered publicly available
and provide usage instruction.


%\input{sections/sec-related-work}
%
%\input{sections/sec-background}
%
%%!TEX root = main.tex
\section{Proposed CGIR Method}
\label{sec:method}
In this section, we first formally define the notation and the gradient item retrieval problem. Then we introduce the proposed framework, followed by discussions about how the proposed method can learn the disentangled item representations with semantic meanings. After that, we show that our weakly-supervised method can achieve disentangled representation with consistency and restrictiveness theoretically.


\begin{figure*}[h!]
\centering
\includegraphics[width=0.9\textwidth]{figures/model.png}
\caption{Overview of our proposed \CGIR~ framework. It includes three major parts -- the left is for disentangled item representation, the right part aims at enforcing representation of attributes to be sparse, and the middle part is for aligning the disentangled item representation space and the sparse word representation space. They are trained in an end-to-end manner.}
\label{fig:model}
\end{figure*}

\subsection{Notation and Problem Formulation}
\textbf{Notation} In this problem, we are provided with a set of users $\mathcal{U}$, a set of items $\mathcal{I}$, a set of attribute strings $\mathcal{T}$, interaction data $\mathbf{X}$ between users and items, and item-attribute relation data $\mathbf{A}$ between attributes and items. Specifically, the interaction data $\mathbf{X}$ consists of the interactions between $N$ users and $M$ items. An interaction between user $u$ and item $i$ is denoted by $x_{u,i} \in \{0,1\}$, where $x_{u,i}=1$ indicates that user $u$ adopts item $i$, whereas $x_{u,i}=0$ means there is no recorded interaction between them. For convenience, we use $\mathbf{x}_{u,:}$ to represent the items adopted by user $u$ and $\mathbf{x}_{:,i}$ to denote the users who interacted with item $i$. The item-attribute relation data $\mathbf{A}$ consists of relations between $M$ items and $T$ attributes, $T = |\mathcal{T}|$. If item $i$ has attribute $t$, then $a_{i,t} = 1$, otherwise $a_{i,t} = 0$. The attribute vector of item $i$ is denoted as $\mathbf{a}_{i,:}$. Besides, the attribute difference data $\mathbf{Y}$ is composed of attribute difference vector $\mathbf{y}_{i,i'} = \mathbf{a}_{i,:} - \mathbf{a}_{i',:}$ , $\mathbf{y}_{i,i'} \in R^{T}$. Each element of the difference vector $y_{i,i'}^t \in \{-1,0,1\}$ indicates the difference between item $i$ and $i'$ on a certain attribute $t$. Triple data $\mathcal{D}$ is constructed using previously mentioned data and it is composed of $(i, \mathbf{y}_{i,i'}, i')$ triples where $i$ denotes reference item, $\mathbf{y}_{i,i'}$ denotes modification and $i'$ is the desired target item.

\textbf{Problem Definition} We define the gradient item retrieval problem as follows: \textit{based on a reference item and a modification, retrieve a sequence of items in which relevance for a certain desired attribute is in increasing or decreasing order, and relevance for other attributes remains the same}. To make it simple, we consider that a query consists of a reference item and a modification about only one attribute. Note that, if multiple attributes are required to be modified, we can apply the atomic modification several times. Mathematically, we define the query as $(i, \alpha t)$ where $i$ indicates the reference item, $\alpha \in \{1,-1\}$ is the modification action and $t$ is the desired modification attribute. Note that there is a bijection between $\alpha$ and the modification words "more" and "less". For the gradient item retrieval problem, it can be defined as: for a query $(i, \alpha t)$ and its corresponding retrieval sequence $Seq\text{-}i$, we want to maximize the probability of the sequence satisfying the constraint: $\alpha \cdot relevance(Seq\text{-}i@k,t) < \alpha \cdot relevance(Seq\text{-}i@k+1,t)$ and $ relevance(Seq\text{-}i@k,t') = relevance(Seq\text{-}i@k+1,t')$, for any other $ t' \in \mathcal{T}, t' \neq t$, where the $relevance$ function measures the relevance score between a retrieved item $Seq\text{-}i@k$ and a certain attribute.


% Need add more detail
\subsection{Proposed Framework of CGIR}
The general framework of the proposed method is shown in Figure \ref{fig:model}. It includes three major parts. 

The left part is designed based on Variational Autoencoder framework~\cite{KingmaW13VAE}, which learns a disentangled item representation from user activities. For each user $u$, we encode the interaction vector $x_{u,:}$ to the user hidden representation $\mathbf{z}_{u} \in R^{D}$. After calculating the interaction probability between user $u$ and all items $\mathbf{H} \in R^{M \times D}$, we reconstruct the interaction vector $x'_{u,:} \in R^{M}$. The reconstruction loss can be calculated between user interaction vector $x_{u,:}$ and its relevant reconstructed vector $x'_{u,:}$. The disentanglement loss is computed using the mean $\mu_{u}$ and variance $\sigma_{u}$.
%of normal distribution of user $u$'s hidden representation.\he{The previous sentence is confusing.} 
We keep the dimensionality of $\mu_{u}$ and  $\sigma_{u}$ the same as $\mathbf{z}_{u}$.

The right part aims to encode attribute strings to a space where attribute representations are sparse. In that space, each representation of an attribute string has only a few activated dimensions. Our intuition is, for each item, the information of its disentangled representation includes the information of all its attributes. Therefore, each attribute representation should only correspond to some dimensions of the disentangled representation. The input of the sparse encoder model is pre-trained word vectors. We use GloVe~\cite{Pennington14glove} as initial features for English words and pre-trained Chinese Word Vectors~\cite{Shen18weibo} as initial features for Chinese characters~(attribute data of Alishop-attribute dataset is in Chinese). If an attribute only has one word or phrase, we use the relevant sparse word representation as the attribute representation. For attributes including multiple words, a sum pooling is applied over sparse representations of words to obtain the attribute representation.

The middle part is for aligning the disentangled item representation space and the sparse attribute representation space. By leveraging the VAE framework, representations are factorized, where dimensions tend to be independent~\cite{Locatello18ChallengeDisentangle}. However, the meaning of each dimension or the composition of some dimensions remains unclear. The goal of this part is to ground the semantic meanings of attributes to dimensions of factorized item representations. 

To achieve the goal, one direct way is to leverage the item-attribute relation data $\mathbf{A}$, which adopted by some existing GAN-based methods \cite{CycleGAN2017, lee2018diverse, shen2020interpreting, jahanian2020steerability, zhou2020ganbased}. However, those methods ignore the relationship between items, which contradicted with the essence of item retrieval. Instead, we implicitly align the two space by minimizing the distance between the target item $i'$ and the modification result which computed by adding a correct modification $\mathbf{y}_{i,i'}$ on the reference item $i$. Note, overlapping is allowed between corresponding dimension sets of two attributes, because two attributes may have the same semantic primitives which are separately encoded into different dimensions of item representation. And to keep a linear relationship in the hidden space, we directly add an item representation and an attribute representation without any non-linear transformation. The coefficient $\gamma$ controls the strength of modification. In the training stage, we set $\gamma = 1$ since we only use the information of whether one item has a certain attribute or not. During the inference stage, in order to retrieve a sequence of items in a gradient manner, we change the strength coefficient $\gamma$ by increasing a fraction number at each step and keep the top one retrieved item for each step to form the retrieval sequence. The three parts are trained in an end-to-end manner.

\subsection{Weakly-Supervised Disentangled Representation Learning with Semantic Meaning}
% To achieve gradient item retrieval and align semantic meaning to dimensions of item representations, we need to achieve disentangled item representations with consistency and restrictiveness. Besides, each attribute representation should only be grounded to some dimensions of its relevant item representations, due to the information of item representation should conclude all its attribute's information.
\textbf{Weakly-Supervised Variational Auto-Encoder.} 
we leverage the VAE framework \cite{KingmaW13VAE} to  enforce item representations to be factorized. And, to involve the information of attribute data, as we stated in the previous section, we model the relation between item pairs and attributes instead of item-attribute data. Specifically, we model the joint distribution of observed variables $\mathbf{X}$ and $\mathbf{Y}$ by joint distribution $p_{\theta}(\mathbf{X}, \mathbf{Y})$ where $\theta$ denotes parameters of \CGIR~. Our generative model assumes that the observed data are generated from the following distribution:
\begin{equation}
\begin{split}
p_{\theta}(\Tilde{\mathbf{X}}, \Tilde{\mathbf{Y}}) &= \iint p_{\theta}(\mathbf{\Tilde{X},\Tilde{Y}}|\mathbf{Z,H}) p_{\theta}(\mathbf{Z,H})  \,d\mathbf{Z} \,d\mathbf{H}
\end{split}
\label{equ:joint}
\end{equation}
$\mathbf{\Tilde{X}}$ and $\mathbf{\Tilde{Y}}$ are variables sampled from a distribution parameterized by hidden variables $\mathbf{Z} \in R^{N \times D}$ and $\mathbf{H} \in R^{M \times D}$. The meanings of $\mathbf{Z}$ and  $\mathbf{H}$ are described in the previous subsection. As shown in Figure \ref{fig:pgm1}, $\mathbf{\Tilde{X},\Tilde{Y}}$ are independent when conditional on $\mathbf{Z}$ and $\mathbf{H}$. Therefore, we have, 
\begin{equation}
\begin{split}
p_{\theta}(\Tilde{\mathbf{X}}, \Tilde{\mathbf{Y}}) = \iint p_{\theta}(\mathbf{\Tilde{X}}|\mathbf{Z,H}) p_{\theta}(\mathbf{\Tilde{Y}}|\mathbf{Z,H})  \,d\mathbf{Z} \,d\mathbf{H}
\end{split}
\label{equ:joint_split}
\end{equation}
We assume interactions between users and items are independent and identically distributed ($i.i.d.$), and vectors in attribute difference data are also $i.i.d.$. Therefore, for the two terms in equation \ref{equ:joint_split}, we have $p_{\theta}(\mathbf{\Tilde{X}}|\mathbf{Z,H}) = \prod_{u,i}p_{\theta}(\Tilde{x}_{u,i}|\mathbf{z}_u, \mathbf{h}_i)$ and  $p_{\theta}(\Tilde{\mathbf{Y}}|\mathbf{Z,H}) = \prod_{i,i'}p_{\theta}(\mathbf{\Tilde{y}}_{i,i'}|\mathbf{h}_i, \mathbf{h}_{i'})$ separately. Following the paradigm of variational autoencoder (VAE) \cite{ChenB20PairwiseVAE, mazhou0Y019MacridVAE}, we introduce a variational dsitribution to alleviate computational burden of integral of equation\ref{equ:joint_split} and maximize the lower bound of $\ln p_{\theta}(\Tilde{x}_{u,i}, \mathbf{\Tilde{y}}_{i,i'}) $ by:

\begin{equation}
\begin{aligned}
\ln p_{\theta}(\Tilde{x}_{u,i}, & \mathbf{\Tilde{y}}_{i,i'}) \geq \mathbb{E}_{q_{\theta}{(\mathbf{z}_{u}, \mathbf{h}_{i}|{x}_{u,i}})} \big[\ln p_{\theta}(\Tilde{x}_{u,i} |\mathbf{z}_{u}, \mathbf{h}_{i})\big] \\
& - \mathcal{D}_{KL}\big(q_{\theta}(\mathbf{z}_{u}, \mathbf{h}_{i}|{x}_{u,i}) || p(\mathbf{z}_{u}, \mathbf{h}_{i}) \big) \\
&+ \mathbb{E}_{q_{\theta}{(\mathbf{z}_{u}, \mathbf{h}_{i}|{x}_{u,i}}), q_{\theta}{(\mathbf{z}_{u}, \mathbf{h}_{i'}|{x}_{u,i'}})} \big[\ln p_{\theta}(\mathbf{\Tilde{y}}_{i,i'} |\mathbf{h}_{i}, \mathbf{h}_{i'})\big]. 
\end{aligned}
\label{equ:vae_lb}
\end{equation}

The expectation $\mathbb{E}_{q_{\theta}{(\mathbf{z}_{u}, \mathbf{h}_{i}|x_{u,i}})}[\cdot]$ is still intractable. As shown in figure \ref{fig:pgm2}, we have $\mathbf{z}_{u} \perp \mathbf{h}_{i} | x_{u,i}$, according to the Common cause decomposition of graphical models\cite{Buntine11PGM}. Therefore, we have the following decomposition:

\begin{equation}
\begin{aligned}
q_{\theta}(\mathbf{z}_u,\mathbf{h}_i,|x_{u,i}) = q_{\theta}(\mathbf{z}_u,|x_{u,i}) q_{\theta}(\mathbf{h}_i,|x_{u,i}). \end{aligned}
\label{equ:demopose1}
\end{equation}

Instead of computing $\mathbb{E}_{q_{\theta}{(\mathbf{z}_{u}, \mathbf{h}_{i}|x_{u,i}})}[\cdot]$ directly, we use the Gaussian re-parameterization trick\cite{KingmaW13VAE} to solve $\mathbb{E}_{q_{\theta}(\mathbf{z}_u,|x_{u,i}) q_{\theta}(\mathbf{h}_i,|x_{u,i})}[\cdot]$.


\textbf{Factorization via Regularization.} A natural strategy to encourage factorization is to force statistical independence between dimensions. 
% so that each dimension describes an isolated factor, i.e. , to force $q_{\theta}(\mathbf{z}_u) \approx \prod_{d=1}^D q_{\theta}(\mathbf{z}_u^{(d)})$ and $q_{\theta}(\mathbf{h}_i) \approx \prod_{d=1}^D q_{\theta}(\mathbf{h}_i^{(d)})$. 
As demonstrate in the previous work \cite{Higgins17betaVAE}, if the prior satisfies factorization, penalizing the Kullback-Leibler term of equation \ref{equ:vae_lb} would encourage independence between the dimensions. In here, we choose two standard multivariate normal distributions as priors for $\mathbf{z}_u$ and $\mathbf{h}_i$. For the Kullback-Leibler divergence part of equation \ref{equ:vae_lb},  we can decompose it as:

\begin{equation}
\begin{split}
&\mathcal{D}_{KL}\big(q_{\theta}(\mathbf{z}_{u}, \mathbf{h}_{i}|x_{u,i}) || p(\mathbf{z}_{u}, \mathbf{h}_{i}) \big) \\
&= \mathcal{D}_{KL}\big(q_{\theta}(\mathbf{z}_{u}|x_{u,i}) q_{\theta}(\mathbf{h}_{i}|x_{u,i})|| p(\mathbf{z}_{u}) p ({ \mathbf{h}_{i}})\big)\\
&= \mathcal{D}_{KL}\big(q_{\theta}(\mathbf{z}_{u}|x_{u,i})||p(\mathbf{z}_{u})\big) + \mathcal{D}_{KL}\big(q_{\theta}(\mathbf{h}_{i}|x_{u,i})||p(\mathbf{h}_{i})\big) 
\end{split}
\label{equ:kl_decompose1}
\end{equation}

The two KL terms in equation \ref{equ:kl_decompose1} aim at enforcing factorization of user and item representations separately.
Due to the time-efficient requirement of recommendation system, we keep a representation table for items, instead of inferring them from interaction matrix at each time. Therefore, we only keep the first term of equation \ref{equ:kl_decompose1} in the final objective. Although this simplification has been used in the previous work\cite{mazhou0Y019MacridVAE}, we also empirically show  that this simplification can enforce item representations to be factorized in our experiments. Besides, We follow $\beta$-VAE\cite{Higgins17betaVAE} to strengthen the KL divergence by a factor of $\beta$.

\textbf{Geometric Relationship of Item Representation.} As shown in the middle part of Figure \ref{fig:model}, to implicitly align item space and attribute space, we leverage the geometric relationship between items. For a reference-target item pair, their distance will be minimized when a correct modification is added on the reference item. Based on the intuition, we define the third term of equation \ref{equ:vae_lb} as:

\begin{equation}
\begin{aligned}
&p_{\theta}(\mathbf{\Tilde{y}}_{i,i'}|\mathbf{h}_{i}, \mathbf{h}_{i'}) =\\
&\frac{q_{\theta}(\mathbf{h}_{i'}|\mathbf{x}_{:,i'}) \big(q_{\theta}(\mathbf{h}_{i}|\mathbf{x}_{:,i}) + \gamma \cdot \sum_{t \in \mathcal{T}} \Tilde{y}_{i,i'}^{t} \cdot F_{\theta}(t)\big)}{\sum_{j' \in [1,M]} q_{\theta}(\mathbf{h}_{i'}|\mathbf{x}_{:,i'})   \big(q_{\theta}(\mathbf{h}_{i}|\mathbf{x}_{:,i}) + \gamma \cdot \sum_{t \in \mathcal{T}} \Tilde{y}_{i,j'}^{t} \cdot F_{\theta}(t)\big)}
\label{equ:contrastive}
\end{aligned}
\end{equation}

In whole, $\gamma \cdot \sum_{t \in \mathcal{T}} \Tilde{y}_{i,i'}^{t} \cdot F_{\theta}(t)$ represents the modification $\mathbf{y}_{i,i'}$ scaled by a factor $\gamma$. During training stage, we set the modification strengthen coefficient $\gamma$ equals one. And during inference, $\gamma$ will be gradually changed to retrieve item in gradient manner. The $\Tilde{y}_{i,i'}^{t}$ indicates the modification direction for attribute $t$, $F_{\theta}(\cdot):R^{K} \rightarrow R^{D}$ is the sparse attribute encoder which encode the attribute $t$ to a sparse representation. The equation \ref{equ:contrastive} represents the probability of one triple $(i, \mathbf{y}_{i,i'}, i')$ in $\mathcal{D}$. To align two representation spaces, we maximize the equation \ref{equ:contrastive}.

\begin{figure}[h]
\centering
\includegraphics[width=0.35\textwidth]{figures/PGM1.png}
\caption{\textbf{The decoder model,} $p(\mathbf{X,Y}|\mathbf{Z,H})$.}
\label{fig:pgm1}
 \end{figure}


\begin{figure}[h]
\centering
\includegraphics[width=0.20\textwidth]{figures/PGM2.png}
\caption{\textbf{The encoder model,} $p(\mathbf{Z,H}|\mathbf{X})$.}
\label{fig:pgm2}
\end{figure}


\textbf{Sparse Attribute Representation} Following our intuition that one item's attribute has less information than the whole item and should only be grounded to part of disentangled item representations, we enforce the attribute representation to be sparse before the alignment of attribute and item representation. Function $F_{\theta}(\cdot)$ is an attribute encoder which maps a attribute string to a sparse representation space. Specifically,

\begin{equation}
\begin{aligned}
F_{\theta}(t) = \sum_{w \in \mathcal{W}(t)} f_{\theta}(w)
\end{aligned}
\end{equation}

where $\mathcal{W}(t)$ represents the set of words used in attribute string $t$. Function $f_{\theta}(\cdot)$ upscale the word representation to another representation space. To enforce the word representation has only a few activated dimensions a sparse loss ($\textit{SL}$) is applied:

\begin{equation}
\begin{aligned}
\textit{SL} = \frac{1}{D} \sum_{d=1}^D &\Big( \text{max}  (\frac{1}{|\mathcal{W}|} \sum_{w \in \mathcal{W}} f^{d}_{\theta}(w) - \rho, 0) ^2 \\
&+ \frac{1}{|\mathcal{T}|} \sum_{w \in \mathcal{W}}  f^{d}_{\theta}(w) \times (1- f^{d}_{\theta}(w))  \Big).
\label{equ:sparseloss}
\end{aligned}
\end{equation}

The first term is an Average Sparsity Loss (ASL) which penalizes any deviation of the observed average activation value $f^{d}_{\theta}(w)$ from the desired average activation value $\rho$ which is usually set to a small value. The second term is a Partial Sparsity Loss (PSL) that facilitates the value of each dimension of $f_{\theta}(w)$ to be close to either 0 or 1\cite{Subramanian18SPINE}.

\textbf{Overall Objective Function} The above equations bring us to the following training objective. Parameter $\theta$ is optimized by maximizing the objective:

\begin{equation}
\begin{split}
\mathbb{E}_{ p_{data}(\mathbf{X})} & \Big[ \mathbb{E}_{q_{\theta}{(\mathbf{z}_{u}, \mathbf{h}_{i}|x_{u,i}})} \big[\ln p_{\theta}(\Tilde{x}_{u,i} |\mathbf{z}_{u}, \mathbf{h}_{i})\big] \\
& - \mathcal{D}_{KL}\big(q_{\theta}(\mathbf{z}_{u}|{x}_{u,i})||p(\mathbf{z}_{u})\big) \\
&+ \mathbb{E}_{q_{\theta}{(\mathbf{z}_{u}, \mathbf{h}_{i}|{x}_{u,i}}), q_{\theta}{(\mathbf{z}_{u}, \mathbf{h}_{i'}|{x}_{u,i'}})} \big[\ln p_{\theta}(\mathbf{\Tilde{y}}_{i,i'} |\mathbf{h}_{i}, \mathbf{h}_{i'})\big] \Big] \\
&-\frac{1}{D} \sum_{d=1}^D \Big( \text{max}  \big(\frac{1}{|\mathcal{W}|} \textstyle\sum_{w \in \mathcal{W}} f^{d}_{\theta}(w) - \rho, 0\big)^2 \\
&+ \frac{1}{|\mathcal{T}|} \textstyle\sum_{w \in \mathcal{W}}  f^{d}_{\theta}(w) \times \big(1- f^{d}_{\theta}(w) \big)  \Big).
\end{split}
\label{equ:whole_obj}
\end{equation}



\subsection{Disentanglement with Guarantee}
As demonstrated by Locatello et al.\cite{Locatello19challengedisentangle}, VAE-based unsupervised learning methods fundamentally cannot achieve disentanglement without model inductive biases. Therefore, a natural question is can our method deliver a disentanglement without the help of model inductive bias? Shu et al.\cite{shu20disentangleguarantee} gives a theoretical analysis which shows disentanglement can be achieved with guarantee under proper weak supervision. Within their analysis framework, three types of weakly supervised settings were considered, which are restricted labeling, matching pairing, and rank pairing. In our case, attributes of items are considered as hidden factors. For item $i$ and item $i'$, we construct the attribute difference vector $\mathbf{y}_{i,i'}$ by comparing them under each attribute $t$, $y_{i,i'}^t = a_{i,t} - a_{i',t}$. If attribute $t$ belongs to item $i$ but not for item $i'$, then the ranking of $i$ is higher than $i'$ and $y_{i,i'}^t$ equals $1$. Therefore, the $(i,\mathbf{y}_{i,i'},i')$ triple data can be understood as a special type of ranking-pair where the ranking is binarized and semantic meaningful. Then according to the \textit{Weak Supervision Disentanglement Theorem}\cite{shu20disentangleguarantee}, the disentangled representation learned under three types of weak supervision is distribution-matching an oracle disentangled representation in which the consistency property of hidden factors, considered by weak supervision signal, can be guaranteed. In our setting, we consider the ranking of one attribute between two items at each triple, because of the restriction $\sum_{t\in\mathcal{T}} |{y}^t_{i,i'}| = 1$. Further, empirically we have $\sum_{i,i' \in [1,N]} |{y}^t_{i,i'}| > 1, \forall t \in \mathcal{T}$ which means all attributes are considered by the weak supervision signal. Further, the consistency of all attributes can be guaranteed. By the Full Disentanglement Rule\cite{shu20disentangleguarantee}, the consistency of all factors further implies the restrictiveness property is guaranteed in disentangled representation.

\begin{equation}
\begin{split}
\bigwedge_{t \in \mathcal{T}} C(t) \iff  \bigwedge_{t \in  \mathcal{T}} D(t), \bigwedge_{t \in \mathcal{T}} D(t) \iff  \bigwedge_{t \in  \mathcal{T}} R(t)
\end{split}
\end{equation}

where $C(t)$ denotes the consistency of hidden factor $t$, $R(t)$ denotes restrictiveness of hidden factor $t$ and $D(t)$ denotes the disentanglement of hidden factor $t$. 
%
%\input{sections/sec-experiment}
%
%%!TEX root = main.tex
\section{conclusion}
\label{sec:conclusion}
In this paper, we identify and study a new problem -- gradient item retrieval. It is defined as retrieving a sequence of items with gradual change with respect to a certain attribute indicated by a modification text. To solve this problem, we proposed a novel method Controllable Gradient Item Retrieval \CGIR. Our method takes a product and a modification text, which indicates what attributes to change and how to change, as a query and retrieves a sequence of items with gradual change on the relevance between the indicated tag and items in the sequence. To achieve the gradient effect, our method learns a disentangled item representation with weak supervision and grounds semantic meanings to dimensions of the representation. We show that our method can achieve consistency and restrictiveness under a previously proposed theoretical framework. Empirically, we demonstrate that our method can retrieve items in a gradient manner; and in item retrieval tasks, our method outperforms existing approaches on three different datasets.

\begin{abstract}
In this work we propose a new automatic image annotation model, dubbed {\bf diverse and distinct image annotation} (D$^2$IA). The generative model D$^2$IA is inspired by the ensemble of human annotations, which create semantically relevant, yet distinct and diverse tags. 
In D$^2$IA, we generate a relevant and distinct tag subset, in which the tags are relevant to the image contents and semantically distinct to each other, using sequential sampling from a determinantal point process (DPP) model.
Multiple such tag subsets that cover diverse semantic aspects or diverse semantic levels of the image contents are generated by randomly perturbing the DPP sampling process.
We leverage a generative adversarial network (GAN) model to train D$^2$IA. 
Extensive experiments including quantitative and qualitative comparisons, as well as human subject studies, on two benchmark datasets demonstrate that the proposed model can produce more diverse and distinct tags than the state-of-the-arts. 
\end{abstract}

\section{Introduction}
\label{sec: introduction}

Image annotation is one of the fundamental tasks of computer vision with many applications in image retrieval, caption generation and visual recognition. Given an input image, an image annotator outputs a set of keywords (tags) that are relevant to the content of the image.  Albeit an impressive progress has been made by current image annotation algorithms, to date, most of them \cite{LEML-ICML-2014, my-iccv-2015, pairwise-ranking-jiebo-cvpr-2017} focus on the {\em relevancy} of the obtained tags to the image with little consideration to their inter-dependencies. As a result, algorithmically generated tags for an image are relevant but at the same time less informative, with redundancy among the obtained tags, {\it e.g.}, one state-of-the-art image annotation algorithm ML-MG \cite{my-iccv-2015} generates tautology {\it`people'} and {\it `person'} for the image in Fig. \ref{fig-1}(f). 

This is different from how human annotators work. We illustrate this using an annotation task involving three human annotators (identified as A1,A2 and A3). Each annotator was asked to independently annotate the first $1,000$ test images in the IAPRTC-12 dataset \cite{iaprtc-12-data-2006} with the requirement of ``describing the main contents of one image using as few tags as possible".  
% 
One example of the annotation results is presented in Fig. \ref{fig-1}. Note that individual human annotators tend to use semantically {\em distinct} tags (see Fig.  \ref{fig-1} (b)-(d)), and the semantic redundancy among tags is lower than that among the tags generated by the annotation algorithm ML-MG \cite{my-iccv-2015} (see Fig. \ref{fig-1}(f)).  
%
Improving the semantic distinctiveness of generated tags has been studied in recent work \cite{my-cvpr-2017-dia}, which uses a determinant point process (DPP) model \cite{dpp-for-machine-learning-2012} to produce tags with less semantic redundancies.
The annotation result of running this algorithm on the example image is shown in Fig. \ref{fig-1}(g).

\begin{figure*}[t]
\centering
\includegraphics[width=0.97\textwidth,height=3.3in]{fig1-human_annotation_one_image_7_circles_green_shaded_small.png}
\caption{{\bf An example illustrating the diversity and distinctiveness in image annotation}. The image \textbf{(a)} is from IAPRTC-12 \cite{iaprtc-12-data-2006}.
We present the tagging results from 3 independent human annotators \textbf{(v)}-\textbf{(d)}, identified as A1, A2, A3, respectively, as well as their ensemble result \textbf{(e)}. 
We also present the results of some automatic annotation methods. ML-MG \cite{my-iccv-2015} \textbf{(f)} is a standard annotation method that requires the relevant tags. 
DIA (ensemble) \cite{my-cvpr-2017-dia} \textbf{(g)} indicates that we repeat the sampling of DIA for 3 times, with the requirement that each subset includes at most 5 tags, and then combine these 3 subsets to one ensemble subset. 
Similarly, we obtain the ensemble subset of our method \textbf{(h)}.   
In each graph, nodes are candidate tags and the arrows connect parent and child tags in the semantic hierarchy. 
%
This figure is better viewed in color.}
\label{fig-1}
\vspace{-1em}
\end{figure*}

However, such results still lack in one aspect when comparing with the annotations from the ensemble of human annotators (see Fig. \ref{fig-1}(e)). The collective annotations from human annotators also tend to be {\em diverse}, consisting of tags that cover more semantic elements of the image. For instance, different human annotators tend to use tags across different abstract levels, such as {\it `church'} vs. {\it `building'}, to describe the image. Furthermore, different human annotators usually focus on different parts or elements of the image. For example, A1 describes the scene as {\it `square'}, A2 notices the {\it `yellow'} color of the building, while A3 finds the {\it `camera'} worn on the chest of people. 



\begin{figure}[t]
\centering
\includegraphics[width=0.47\textwidth,height=1.5in]{fig2-gan-model-structure-single-column-1115-new-small.png}
\caption{A schematic illustration of the structure of the proposed D$^2$IA-GAN model. $S_{DD-I}$ indicates the ground-truth set of diverse and distinct tag subsets for the image $I$, which will be defined in the Section \ref{sec: background}.}
\label{fig-2}
\vspace{-.2in}
\end{figure}


In this work, we propose a novel image annotation model, namely diverse and distinct image annotation (D$^2$IA), which aims to improve the diversity and distinctiveness of the tags for an image by learning a generative model of tags from multiple human annotators. 
%Our approach is to simulate tag generations by introducing random noise perturbations to the image. 
The distinctiveness enforces the semantic redundancy among the tags in the same subset to be small, while the diversity encourages different tag subsets to cover different aspects or different semantic levels of the image contents. 
Specifically, this generative model first maps the concatenation of the image feature vector and a random noise vector to a posterior probability with respect to all candidate tags, and then incorporates it into a determinantal point process (DPP) model \cite{dpp-for-machine-learning-2012} to generate a distinct tag subset by sequential sampling. 
Utilizing multiple random noise vectors for the same image, multiple diverse tag subsets are sampled. 

We train D$^2$IA as the generator in a generative adversarial network (GAN) model \cite{gan-nips-2014} given a large amount of human annotation data, which is subsequently referred to as 
D$^2$IA-GAN. 
The discriminator of D$^2$IA-GAN is a neural network measuring the relevance between the image feature and the tag subset that aims to distinguish the generated tag subsets and the ground-truth tag subsets from human annotators. 
The general structure of D$^2$IA-GAN model is shown in Fig. \ref{fig-2}.
The proposed D$^2$IA-GAN is trained by alternative optimization of the generator and discriminator while fixing the other until convergence.

One characteristic of the D$^2$IA-GAN model is that its generator includes a sampling step which is not easy to optimize directly using gradient based optimization methods. Inspired by reinforcement learning algorithms, we develop a method based on the {\it policy gradient} (PG) algorithm, where we model the discrete sampling with a differentiable policy function (a neural network), and devise a {\it reward} to encourage the generated tag subset to match the image content as close as possible. 
Incorporating the policy gradient algorithm in the training of D$^2$IA-GAN, we can effectively obtain the generative model for tags conditioned on the image. 
%
As shown in Fig. \ref{fig-1}(h), using the trained generator of D$^2$IA-GAN can produce 
diverse and distinct tags that are closer to those generated from the ensemble of multiple human annotators (Fig. \ref{fig-1}(e)).

The main contributions of this work are four-fold. 
{\bf (1)} We develop a new image annotation method, namely diverse and distinct image annotator (D$^2$IA), to create relevant, yet distinct and diverse annotations for an image, which are more similar to tags provided by different human annotators for the same image;
{\bf (2)} we formulate the problem as learning a probabilistic generative model of tags conditioned on the image content, which exploits a DPP model to ensure distinctiveness and conducts random perturbations to improve diversity of the generated tags; 
{\bf (3)} the generative model is adversarially trained using a specially designed GAN model that we term as D$^2$IA-GAN;
{\bf (4)} in the training of D$^2$IA-GAN we use the policy gradient algorithm to handle the discrete sampling process in the generative model. 
We perform experimental evaluations on ESP Game \cite{espgame-2004} and IAPRTC-12 \cite{iaprtc-12-data-2006} image annotation datasets, and subject studies based on human annotators for the quality of the generated tags. The evaluation results show that the tag set produced by D$^2$IA-GAN is more diverse and distinct when comparing with those generated by the state-of-the-art methods.

\section{Related Work}
\label{sec: related work}

Existing image annotation methods fall into two general categories: they either generate all tags simultaneously using multi-label learning, or predict tags sequentially using sequence generation.
The majority of existing image annotation methods are in the first category.
They mainly differ in designing different loss functions or exploring different class dependencies. 
Typical loss functions include square loss \cite{LEML-ICML-2014, muitilabel-attention-cvpr-2017},  
ranking loss \cite{cnn-image-annotation-arxiv-2013, pairwise-ranking-jiebo-cvpr-2017}, cross-entropy loss \cite{spatial-regularization-cvpr-2017}), {\it etc.} 
Commonly used class dependencies include class co-occurrence \cite{my-icpr-2014,my-pr-2015,li2016facial}, mutual exclusion \cite{xiaotong-multi-label-exclusive-2011, deep-dpp-cvpr-2017}, class cardinality \cite{my-aaai-2016-imbalance}, sparse and low rank \cite{my-ijcv-2018}, and semantic hierarchy \cite{my-iccv-2015}. 
Besides, some multi-label learning methods consider different learning settings, such as multi-label learning with missing labels \cite{my-icpr-2014, my-pr-2015}, label propagation in semi-supervised learning \cite{ssl-wei-liu-icml-2010, teach-to-learn-wei-liu-aaai-2016, teach-to-learn-wei-liu-tnnls-2017} and transfer learning \cite{multi-label-transfer-wei-liu-pami-2017} settings. 
A thorough review of multi-label learning based image annotation methods can be found in      
 \cite{review-image-annotation-pr-2012}.

Our method falls into the second category, which generates tags in a sequential manner. This can better employ the inter-dependencies of the tags. 
Many methods in this category are built on sequential models, such as recurrent neural networks (RNNs), which work in coordination with convolutional neural networks (CNNs) to exploit their representation power for images. The main difference of these works lies in designing an interface between CNN and RNN.
In \cite{rnn-image-annotation-icpr-2016}, features extracted by a CNN model were used as the hidden states of a RNN. 
In \cite{rnn-cnn-image-annotation-cvpr-2016}, the CNN features were integrated with the output of a RNN. 
In \cite{rnn-semantic-regularization-cvpr-2017}, the predictions of a CNN were used as the hidden states of a RNN, and the ground-truth tags of images were used to supervise the training of the CNN. 
Not directly using the output layer of a RNN, the work in \cite{rnn-fisher-vector-eccv-2016} utilized the Fisher vector derived from the gradient of the RNN, as the feature representation.

Although RNN is a suitable model for the sequential image annotation
task for its ability to implicitly encode the dependencies
among tags, it is not easy to explicitly embed
some prior knowledge about the tag dependencies like semantic
hierarchy \cite{my-iccv-2015} or mutual exclusion \cite{xiaotong-multi-label-exclusive-2011} in the RNN model. 
%
To remedy this issue, the recent work of DIA \cite{my-cvpr-2017-dia} formulated the sequential prediction as a sampling process based on a determinantal point process (DPP) \cite{dpp-for-machine-learning-2012}. 
DIA encodes the class co-occurrence into the learning process, and incorporates the semantic hierarchy into the sampling process. 
Another important difference between
DIA and the RNN-based methods is that the former explicitly
embeds the negative correlations among tags {\it i.e.}, avoiding using semantically similar tags for the same image, while RNN-based methods typically ignore such negative corrlations.
%
The main reason is that the objective of DIA is
to describe an image with a few diverse and relevant tags,
while most other methods tend to predict most relevant tags.


Our proposed model D$^2$IA-GAN is inspired by DIA, and both are developed based on the observations of human annotations. Yet, there are several significant differences between them. 
The most important difference is in their objectives. 
DIA aims to simulate a single human annotator to use semantically distinct
tags for an image, while D$^2$IA-GAN aims to simulate multiple human annotators simultaneously to capture the diversity among human annotators. 
They are also different in the training process, which will be reviewed in the Section \ref{sec: model}. 
Besides, in DIA \cite{my-cvpr-2017-dia}, `diverse/diversity' refers to the semantic difference between tags in the same tag subset, to which we use the word `distinct/distinctiveness' for the same meaning in this work. We use `diverse/diversity' to indicate the semantic difference between multiple tag subsets for the same image. 


\section{Background}
\label{sec: background}


\noindent
{\bf Weighted semantic paths.} Weighted semantic paths \cite{my-cvpr-2017-dia} are constructed based on the semantic hierarchy and synonyms \cite{my-iccv-2015} among all candidate tags. To construct a weighted semantic path, we treat each tag as a node, and the synonyms are merged into one node. Then, starting from each leaf node in the semantic hierarchy, we connect its direct parent node and repeat this connection process, until the root node is achieved. All tags that are visited in this process form the weighted semantic path of the leaf tag. The weight of each tag in the semantic path is computed inversely proportional to the node layer (the layer number starts from 0 at leaf nodes) and the number of descendants of each node. 
As such, the weight of the tag with more specified information will be larger.
A brief example of the weighted semantic paths is shown in Fig. \ref{fig-3-weighted-path}. 
We use $SP_{\mathcal{T}}$ to denote the semantic paths of set $\mathcal{T}$ of all candidate tags.  
$SP_T$ indicates the semantic paths of the tag subset $T$. 
$SP_I$ represents the weighted semantic paths of all ground-truth tags of image $I$. 

\vspace{3pt}
\noindent
{\bf Diverse and distinct tag subsets.} Given an image $I$ and its ground-truth semantic paths $SP_I$, a tag subset is distinct if there are no tags being sampled from the same semantic path. 
%if we pick at most one tag from each path, then the formed tag subset is distinct. 
An example of the distinct tag subset is shown in Fig. \ref{fig-3-weighted-path}:  $SP_I = \{ lady \rightarrow woman \rightarrow (people, person), cactus \rightarrow plant \}$ includes 3 semantic paths with 7 tags, such as $\{lady, plant\}$ or $\{women, cactus\}$. 
A tag set is diverse if it includes multiple distinct tag subsets. These subsets cover different contents of the image, due to two possible reasons, including 1) they describe  different contents of the image, and 2) they describe the same content but at different semantic levels.  
As shown in Fig. \ref{fig-3-weighted-path}, we can construct a diverse set of distinct tag subsets like $\{ \{lady, cactus \}, \{ plant, cat \}, \{woman, plant, animal\}  \}$. 
Furthermore, we can construct all possible distinct tag subsets (ignoring the subset with a single  tag) to obtain the complete diverse set of distinct tag subsets, referred to as $S_{DD-I}$. 
Specifically, for the subset with 2 tags, we will pick 2 paths out of 3 and sample one tag from each picked path. Then we obtain in total 16 distinct subsets. For the subset with 3 tags, we sample one tag from each semantic path, leading to 12 distinct subsets. 
$S_{DD-I}$ will be used as the ground-truth to train the proposed model.  

\begin{figure}[t]
%\vspace{-.2in}
\centering
\includegraphics[width=0.43\textwidth,height=1.0in]{fig-semantic-paths-new-3-paths-box.png}
\caption{A brief example of the weighted semantic paths. The word in box indicates the tag. The arrow $a \rightarrow b$ tells that tag $a$ is the semantic parent of tag $b$. The bracket close to each box denotes the corresponding (node layer, number of descendants, tag weight). Boxes connected by arrows construct a semantic path.}
\label{fig-3-weighted-path}
\vspace{-.2in}
\end{figure}

\vspace{3pt}
\noindent
{\bf Conditional DPP.} 
We use a conditional determinantal point process (DPP) model to measure the probability of the tag subset $T$, derived from the ground set $\mathcal{T}$ given a feature $\x$ of the image $I$. The DPP model is formulated as 
\begin{flalign}
\vspace{-0.1in}
 \mathcal{P}(T|I) = \frac{\text{det}\big(\mathbf{L}_{T}(I)\big)}{\text{det}\big(\mathbf{L}_{\mathcal{T}}(I) + \I\big)},
 \label{eq: formulation of conditional DPP}
 \vspace{-0.13in}
\end{flalign}
where $\mathbf{L}_{\mathcal{T}}(I) \in \mathbb{R}^{|\mathcal{T}| \times |\mathcal{T}|}$ is a positive semi-definite kernel matrix. 
$\I$ indicates the identity matrix. 
For clarity, the parameters of $\mathbf{L}_{\mathcal{T}}(I)$ and (\ref{eq: formulation of conditional DPP}) have been omitted. 
The sub-matrix $\mathbf{L}_{T}(I) \in \mathbb{R}^{|T| \times |T|}$ is constructed by extracting the rows and columns corresponding to the tag indexes in $T$. 
For example, assuming $\mathbf{L}_{\mathcal{T}}(I) = [ a_{ij} ]_{i,j = 1, 2, 3, 4}$ and $T = \{2,4\}$, then $\mathbf{L}_{T}(I) = [a_{22}, a_{24}; a_{42}, a_{44}]$. 
$\text{det}\big(\mathbf{L}_{T}(I)\big)$ indicates the determinant of $\mathbf{L}_{\mathcal{T}}(I)$. It encodes the negative correlations among the tags in the subset $T$. 

Learning the kernel matrix $\mathbf{L}_{\mathcal{T}}(I)$ directly is often difficult, especially when $|\mathcal{T}|$ is large. To alleviate this problem, we decompose $\mathbf{L}_{\mathcal{T}}(I)$ as $\mathbf{L}_{\mathcal{T}}(i,j) = v_i \boldsymbol{\phi}_i^\top \boldsymbol{\phi}_i v_j$, 
%as follows:
%\begin{flalign}
%\mathbf{L}_{\mathcal{T}}(i,j) = v_i \phi_i^\top \phi_i v_j, 
%\end{flalign}
where the scalar $v_i$ indicates the individual score with respect to tag $i$, and $\mathbf{v}_{\mathcal{T}} = [ v_1, \ldots, v_i, \ldots, v_{\mathcal{T}} ]$. 
The vector $\boldsymbol{\phi}_i \in \mathbb{R}^{d'}$ corresponds to the direction of tag $i$, with $\parallel\boldsymbol{\phi}_i\parallel = 1$, and can be used to construct the semantic similarity matrix $\mathbf{S}_{\mathcal{T}} \in \mathbb{R}^{|\mathcal{T}| \times |\mathcal{T}|}$ with $\mathbf{S}_{\mathcal{T}}(i,j) = \boldsymbol{\phi}_i^\top \boldsymbol{\phi}_j$.  
With this decomposition, we can learn $\mathbf{v}_{\mathcal{T}}$ and  $\mathbf{S}_{\mathcal{T}}$ separately. 
More details of DPP can be found in \cite{dpp-for-machine-learning-2012}. 
%
In this work, $\mathbf{S}_{\mathcal{T}}$ is pre-computed as:
\begin{flalign}
\vspace{-0.13in}
\mathbf{S}_{\mathcal{T}}(i,j) = \frac{1}{2} + \frac{\langle \mathbf{t}_i, \mathbf{t}_j\rangle }{ 2\| \mathbf{t}_i \|_2 \|\mathbf{t}_j\|_2} \in [0, 1]~~\forall~i,j\in\mathcal{T},
\vspace{-0.13in}
\end{flalign}
where the tag representation $\mathbf{t}_i \in \mathbb{R}^{50}$ is derived from the GloVe algorithm \cite{glove-2014}. $\langle \cdot, \cdot \rangle$ indicates the inner product of two vectors, while $\|\cdot\|_2$ denotes the $\ell_2$ norm of a vector.

\vspace{3pt}
\noindent
{\bf k-DPP sampling with weighted semantic paths.}
k-DPP sampling \cite{dpp-for-machine-learning-2012} is a sequential sampling process to obtain a tag subset $T$ with at most $k$ tags, according to the distribution (\ref{eq: formulation of conditional DPP}) and the weighted semantic paths $SP_{\mathcal{T}}$. It is denoted as $\mathcal{S}_{\text{k-DPP}, SP_{\mathcal{T}}}(\mathbf{v}_{\mathcal{T}}, \mathbf{S}_{\mathcal{T}})$ subsequently. 
Specifically, in each sampling step, the newly sampled tag will be checked whether it is from the same semantic path with any previously sampled tags. If not, it is included into the tag subset; if yes, it is abandoned and we go on sampling the next tag, until $k$ tags are obtained. 
The whole sampling process is repeated multiple times to obtain different tag subsets. 
Then the subset with the largest tag weight summation is picked as the final output. 
Note that a larger weight summation indicates more semantic information. Since the tag weight is pre-defined when introducing the weighted semantic paths, it is an objective criterion to pick the subset.




\section{D$^2$IA-GAN Model}
\label{sec: model}

%Given an image $I$, we want to generate multiple diverse and distinct tag subsets. 
Given an image $I$, we aim to generate a diverse tag set including multiple distinct tag subsets relevant to the image content, as well as an ensemble tag subset of these distinct subsets, which could provide a comprehensive description of $I$.
These tags are sampled from a generative model conditioned on the image, and we use a  conditional GAN (CGAN) \cite{cgan-2014, wenhan-cvpr-2018, face-gan} to train it, with the generator part $\G$ being our model and a discriminator $\D$, as shown in Fig. \ref{fig-2}.
Specifically, conditioned on $I$, $\G$ projects one noise vector $\z$ to one distinct tag subset $T$, and uses different noise vectors to ensure diverse/different tag subsets. 
$\D$ serves as an adversary of $\G$, aiming to distinguish the generated tag subsets using $\G$ from the ground-truth ones $S_{DD-I}$.  

\subsection{Generator}

The tag subset $T \subset \mathcal{T} = \{1, 2, \ldots, m\}$ with $|T| \leq k$ can be generated from the generator $G_{\boldsymbol{\theta}}(I, \mathbf{z})$, according to the input image $I$ and a noise vector $\z$, as follows:
\begin{flalign}
\hspace{-0.7em} \G_{\boldsymbol{\theta}}(I, \mathbf{z}; \mathbf{S}_{\mathcal{T}}, SP_{\mathcal{T}}, k) 
\sim
 \mathcal{S}_{\text{k-DPP}, SP_{\mathcal{T}}}\big( \sqrt{\mathbf{q}_{\mathcal{T}}(I, \z)}, \mathbf{S}_{\mathcal{T}} \big).
\label{eq: G model} 
\end{flalign}
%where $\boldsymbol{\theta}$ includes $\W, \mathbf{b}_{\G}$ and the parameters of $f_{\G}$. 
%
The above generator is a composite function with two parts. 
The {\bf inner} part $\mathbf{q}_{\mathcal{T}}(I, \z) = \sigma\big(\W^\top [f_{\G}(I); \z]+\mathbf{b}_{\G} \big) \in [0,1]^{|\mathcal{T}|}$ is a CNN based soft classifier. 
$f_{\G}(I)$ represents the output vector of the fully-connected layer of a CNN model, and $[\mathbf{a}_1; \mathbf{a}_2]$ denotes the concatenation of two vectors $\mathbf{a}_1$ and $\mathbf{a}_2$. $\sigma(\mathbf{a}) = \frac{1}{1+\exp(-\mathbf{a})}$ is the sigmoid function. $\sqrt{\mathbf{a}}$ indicates the element-wise square root of vector $\mathbf{a}$.
The parameter matrix $\W = [\w_1, \ldots, \w_i, \ldots, \w_m] \in \mathbb{R}^{m \times d}$ and the bias parameter $\mathbf{b}_{\G} \in \mathbb{R}^{d}$ map the feature vector $[f_{\G}(I); \z] \in \mathbb{R}^{d}$ to the logit vector.  
The trainable parameter $\boldsymbol{\theta}$ includes $\W, \mathbf{b}_{\G}$ and the parameters of $f_{\G}$.
The noise vector $\z$ is sampled from the uniform distribution $\text{U}[-1,1]$. 
%
The {\bf outer} part $\mathcal{S}_{\text{k-DPP}, SP_{\mathcal{T}}}(\sqrt{\mathbf{q}_{\mathcal{T}}(I, \z)}, \mathbf{S}_{\mathcal{T}})$ is the k-DPP sampling with weighted semantic paths $SP_{\mathcal{T}}$ (see Section \ref{sec: background}). Using $\sqrt{\mathbf{q}_{\mathcal{T}}(I, \z)}$ as the quality term and utilizing the pre-defined similarity matrix $\mathbf{S}_{\mathcal{T}}$, then a conditional DPP model can be constructed as described in Section \ref{sec: background}. 



\subsection{Discriminator}

$\D_{\boldsymbol{\eta}}(I, T)$ evaluates the relevance of image $I$ and tag subset $T$: it outputs a value in $[0,1]$, with $1$ meaning the highest relevance and $0$ being the least relevant. 
Specifically, $\D_{\boldsymbol{\eta}}$ is constructed as follows: first, as described in Section \ref{sec: background}, each tag $i \in T$ is represented by a vector $\boldsymbol{t}_i \in \mathbb{R}^{50}$ derived from the GloVe algorithm \cite{glove-2014}. Then, we formulate $\D_{\boldsymbol{\eta}}(I, T)$ as
\vspace{-0.7em}
\begin{eqnarray}
\vspace{-8em}
\D_{\boldsymbol{\eta}}(I, T) = \frac{1}{|T|}\sum_{i \in T} \sigma\left( \w_{\D}^\top [ f_{\D}(I) ; \boldsymbol{t}_i ] + b_{\D} \right),
\vspace{-8em}
\label{eq: D model}
\vspace{-8em}
\end{eqnarray}
where $f_{\D}(I)$ denotes the output vector of the fully-connected layer of a CNN model  (different from that used in the generator). 
$\boldsymbol{\eta}$ includes $\w_{\D} \in \mathbb{R}^{|f_{\D}(I)|+50}, b_{\D} \in \mathbb{R}$ and the parameters of $f_{\D}(I)$ in the CNN model. 



\subsection{Conditional GAN}
\label{sec: subsec conditional GAN}

Following the general training procedure, we learn D$^2$IA-GAN by iterating two steps until convergence: 
(1) fixing the discriminator $\D_{\boldsymbol{\eta}}$ and optimizing the generator $\G_{\boldsymbol{\theta}}$ using (\ref{eq: sub-problem G}), as shown in Section \ref{sec: subsec optimizing G}; 
(2) fixing $\G_{\boldsymbol{\theta}}$ and optimizing $\D_{\boldsymbol{\eta}}$ using (\ref{eq: maximizing discriminator with F1}), as shown in Section \ref{sec: subsec optimizing D}. 


\vspace{-0.6em}
\subsubsection{Optimizing $\G_{\boldsymbol{\theta}}$} % Using Policy Gradient}
\label{sec: subsec optimizing G}

Given $\D_{\boldsymbol{\eta}}$, we learn $\G_{\boldsymbol{\theta}}$ by 
\vspace{-0.6em}
\begin{eqnarray}
\vspace{-0.8em}
\underset{\boldsymbol{\theta}}{\min} ~
\mathbb{E}_{\mathbf{z} \sim \text{U}[-1,1]} \left[\log \bigg( 1 -  D_{\boldsymbol{\eta}}\big(I, \G_{\boldsymbol{\theta}}(I, \mathbf{z})\big) \bigg) \right].
\vspace{-0.8em}
\label{eq: sub-problem G}
\vspace{-0.8em}
\end{eqnarray}
For clarity, we only show the case with one training image $I$ in the above formulation.
Due to the discrete sampling process $\mathcal{S}(\mathbf{v}_{\mathcal{T}}, \mathbf{S}_{\mathcal{T}})$ in $\G_{\boldsymbol{\theta}}(I, \mathbf{z})$, we cannot optimize (\ref{eq: sub-problem G}) using any existing continuous optimization algorithm. 
%
To address this issue, we view the sequential generation of tags as controlled by a continuous policy function, which weighs different choices of the next tag based on the image and tags already generated. As such, we can use the policy gradient (PG) algorithm in reinforcement learning for its optimization.
Given a sampled tag subset $T_{\G}$ from $\mathcal{S}(\mathbf{v}_{\mathcal{T}}, \mathbf{S}_{\mathcal{T}})$, the original objective function of (\ref{eq: sub-problem G}) is approximated by a continuous function. 
%
Specifically, we denote $T_{\G} = \{y_{[1]}, y_{[2]}, \ldots, y_{[k]}\}$, where $[i]$ indicates the sampling order, and its subset $T_{\G-i} = \{y_{[1]}, \ldots, y_{[i]}\}, i \leq k$ includes the first $i$ tags in $T_{\G}$. 
Then, with an instantialized $\z$ sampled from $[-1,1]$, the approximated function is formulated as
\vspace{-0.8em}
\begin{flalign}
\hspace{-0.8em}\mathcal{J}_{\boldsymbol{\theta}}(T_{\G}) & =  
\sum_{i=1}^k \mathcal{R}(I, T_{\G-i}) \log\bigg( \hspace{-0.3em} \prod_{t_1 \in T_{\G-i}} \hspace{-0.75em} q_{t_1}^1 \hspace{-0.25em} \prod_{t_2 \in \mathcal{T} \setminus T_{\G-i}} \hspace{-0.8em} q_{t_2}^0  \bigg), 
\vspace{-0.8em}
\label{eq: objective of policy gradient}
\vspace{-0.8em}
\end{flalign}
where $\mathcal{T} \setminus T_{\G-i}$ denotes the relative complement of $T_{\G-i}$ with respect to $\mathcal{T}$. 
$q_{t}^1 = \sigma\big(\w_t^\top [f_{\G}(I); \z] + b_{\G}(t)\big)$ indicates the posterior probability, and $q_{t}^0 = 1- q_{t}^1$. 
The reward function $\mathcal{R}(I, T_{\G})$ encourages the content of $I$ and the tags $T_{\G}$ to be consistent, and is defined as
\vspace{-0.8em}
\begin{flalign}
\mathcal{R}(I, T_{\G}) & = -\log\big( 1 -  \D_{\boldsymbol{\eta}}(I, T_{\G}) \big).
\vspace{-0.8em}
\label{eq: reward}
\vspace{-0.8em}
\end{flalign}
%
Compared to a full PG objective function, in (\ref{eq: objective of policy gradient}) we have replaced the \emph{return} with the \emph{immediate reward} $\mathcal{R}(I, T_{\G})$, and the \emph{policy} probability with the decomposed likelihood $\prod_{t_1 \in T_{\G-i}}  q_{t_1}^1 \prod_{t_2 \in \mathcal{T} \setminus T_{\G-i}} q_{t_2}^0$.
Consequently, it is easy to compute the gradient $\frac{\partial \mathcal{J}_{\boldsymbol{\theta}}(T_{\G})}{\partial \boldsymbol{\theta}}$, which will be used in the stochastic gradient ascent algorithm and back-propagation \cite{back-propagation-hinton-1986} to update $\boldsymbol{\theta}$. 

When generating $T_{\G}$ during training, we repeat the sampling process multiple times to obtain different subsets. Then, as the ground-truth set $S_{DD-I}$ for each training image is available, the semantic F$_{1-sp}$ score (see Section \ref{sec: experiments}) for each generated subset can  be computed, and the one with the largest F$_{1-sp}$ score will be used to update parameters. This process encourages the model to generate tag subsets more consistent with the evaluation metric. 


\subsubsection{Optimizing $\D_{\boldsymbol{\eta}}$}
\label{sec: subsec optimizing D}

Utilizing the generated tag subset $T_{\G}$ from the fixed generator 
$\G_{\boldsymbol{\theta}}(I, \mathbf{z})$, we learn $\D_{\boldsymbol{\eta}}$ by  
\vspace{-0.6em}
\begin{flalign}
& \underset{\boldsymbol{\eta}}{\max}  ~
\frac{1}{|S_{DD-I}|}\sum_{T \in S_{DD-I} } \bigg[ \beta \log \D_{\boldsymbol{\eta}}(I, T) - (1-\beta) \cdot
\label{eq: maximizing discriminator with F1}
\\
&  \big(\D_{\boldsymbol{\eta}}(I, T) - F_{1-sp}(I, T) \big)^2 \bigg] + \beta \log\big( 1 -  \D_{\boldsymbol{\eta}}(I, T_{\G}) \big) -
\nonumber
\\
&  (1-\beta) \left(\D_{\boldsymbol{\eta}}(I, T_{\G}) - \text{F}_{1-sp}(I, T_{\G})\right)^2,
\nonumber
\vspace{-0.8em}
\end{flalign}
where semantic score $\text{F}_{1-sp}(I, T)$ measures the relevance between the tag subset $T$ and the content of $I$. 
If we set the trade-off parameter $\beta=1$, then (\ref{eq: maximizing discriminator with F1}) is equivalent to the objective used in the standard GAN model. 
For $\beta \in (0,1)$, (\ref{eq: maximizing discriminator with F1}) also encourages the updated $\D_{\boldsymbol{\eta}}$ to be close to the semantic score   $F_{1-sp}(I, T)$. 
%
We can then compute the gradient of (\ref{eq: maximizing discriminator with F1}) with respect to $\boldsymbol{\eta}$, and use the stochastic gradient ascent algorithm and back-propagation \cite{back-propagation-hinton-1986} to update $\boldsymbol{\eta}$.



\section{Experiments}
\label{sec: experiments}

\subsection{Experimental Settings}

\noindent
{\bf Datasets.} 
%Since the weighted semantic paths play the important role in the proposed model, in experiments 
We adopt two benchmark datasets, ESP Game \cite{espgame-2004} and IAPRTC-12 \cite{iaprtc-12-data-2006} for evaluation. One important reason for choosing these two datasets is that they have complete weighted semantic paths of all candidate tags $SP_{\mathcal{T}}$, the ground-truth weighted semantic paths of each image $SP_{I}$, the image features and the trained DIA model, provided by the authors of \cite{my-cvpr-2017-dia} and available on GitHub\footnote{Downloaded from {\it https://github.com/wubaoyuan/DIA}}. 
%
%We borrow the table from \cite{my-cvpr-2017-dia} to demonstrate the basic data statistics, as shown in Table \ref{table: dataset}. 
%Table \ref{table: dataset} gives the basic data statistics of these two datasets. 
Since the weighted semantic paths are important to our method, these two datasets facilitate its evaluation.
Specifically, in ESP Game, there are 18689 train images, 2081 test images, 268 candidate classes, 106 semantic paths corresponding to all candidate tags, and the feature dimension is 597; in IAPRTC-12, there are 17495 train images, 1957 test images, 291 candidate classes, 139 semantic paths of all candidate tags, and the feature dimension is 536.


\vspace{3pt}
\noindent
{\bf Model training.} 
We firstly fix the CNN models in both $\G_{\boldsymbol{\theta}}$ and $\D_{\boldsymbol{\eta}}$ as the VGG-F model\footnote{Downloaded from {\it http://www.vlfeat.org/matconvnet/pretrained/}} pre-trained on ImageNet \cite{imagenet-cvpr-2009}. Then we initialize the columns of the fully-connected parameter matrix $\W$ (see Eq. (\ref{eq: G model})) that corresponds to the image feature $f_{\G}(I)$ using the trained DIA model, while the columns corresponding to the noise vector $\z$ and the bias parameter $\mathbf{b}_{\G}$ are randomly initialized. 
%
We pre-train $\D_{\boldsymbol{\eta}}$ by setting $\beta=0$ in Eq. (\ref{eq: maximizing discriminator with F1}), {\it i.e.}, only using the F$_{1-sp}$ scores of ground-truth subsets $S_{DD-I}$ and the fake subsets generated by the initialized $\G_{\boldsymbol{\theta}}$ with $\z$ being the zero vector. The corresponding pre-training parameters are: batch size $=256$, epochs $=20$, learning rate $=1$, $\ell_2$ weight decay $=0.0001$. 
%
With the initialized $\G_{\boldsymbol{\theta}}$ and the pre-trained $\D_{\boldsymbol{\eta}}$, we fine-tune the D$^2$IA-GAN model using the following parameters: batch size $=256$, epochs $=50$,  the learning rates of $\W$ and $\boldsymbol{\eta}$  are set to $0.0001$ and $0.00005$ respectively, both learning rates are decayed by $0.1$ in every 10 epochs, $\ell_2$ weight decay $=0.0001$, and $\beta=0.5$. 
%
Besides, if there are a few long paths ({\it i.e.}, many tags in a semantic path) in $SP_{I}$, the number of subsets in $SP_{I}$, {\it i.e.}, $|S_{DD-I}|$, could be very large. In ESP Game and IAPRTC-12, the largest $|SP_{I}|$ is up to 4000, though $|S_{DD-I}|$ for most images are smaller than 30. If $|S_{DD-I}|$ is too large, the training of the discriminator $\D_{\boldsymbol{\eta}}$ (see Eq. (\ref{eq: maximizing discriminator with F1})) will be slow. Thus, we set a upper bound $10$ for $|S_{DD-I}|$ in training, if $|S_{DD-I}| > 10$, then we randomly choose 10 subsets from $S_{DD-I}$ to update $\D_{\boldsymbol{\eta}}$.
The implementation adopts Tensorflow 1.2.0 and Python 2.7.  

\vspace{3pt}
\noindent
{\bf Evaluation metrics.}  
To evaluate the distinctiveness and relevance of the predicted tag subset, three semantic metrics, including semantic precision, recall and F1, are proposed in \cite{my-cvpr-2017-dia}, according to the weighted semantic paths. They are denoted as P$_{sp}$,  R$_{sp}$ and F$_{1-sp}$ respectively. Specifically, given a predicted subset $T$, the corresponding semantic paths $SP_T$ and the ground-truth semantic paths $SP_I$, P$_{sp}$ computes the proportion of the true semantic paths in $SP_T$, and R$_{sp}$ computes the proportion of the true semantic paths in $SP_I$ that are also included in $SP_T$, and F$_{1-sp} = 2(\text{P}_{sp} \cdot \text{R}_{sp}) / (\text{P}_{sp} + \text{R}_{sp})$. The tag weight in each path is also considered when computes the proportion.  
Please refer to \cite{my-cvpr-2017-dia} for the detailed definition.

\vspace{3pt}
\noindent
{\bf Comparisons.} 
We compare with two state-of-the-art image annotation methods, including ML-MG\footnote{Downloaded from {\it https://sites.google.com/site/baoyuanwu2015/home}} \cite{my-iccv-2015} and DIA\footnote{Downloaded from {\it https://github.com/wubaoyuan/DIA}} \cite{my-cvpr-2017-dia}. 
The reason we compare with them is that both of them and the proposed method utilize the semantic hierarchy and the weighted semantic paths, but with different usages. 
We also compare with another state-of-the-art multi-label learning method, called LEML\footnote{Downloaded from {\it http://www.cs.utexas.edu/~rofuyu/}} \cite{LEML-ICML-2014}, which doesn't utilize the semantic hierarchy.
Since both ML-MG and LEML do not consider the semantic distinctiveness among tags, their predicted tag subsets are likely to include semantic redundancies. As reported in \cite{my-cvpr-2017-dia}, the evaluation scores using the semantic metrics ({\it i.e.}, P$_{sp}$, R$_{sp}$ and F$_{1-sp}$) of ML-MG and LEML's predictions are much lower than DIA. Hence it is not relevant to compare with the original results of ML-MG and LEML. Instead, we combine the predictions of ML-MG and LEML with the DPP-sampling that is also used in DIA and our method. Specifically, the square root of posterior probabilities with respect to all candidate tags produced by ML-MG are used as the quality vector (see Section \ref{sec: background}); as there are negative scores in the predictions of LEML, we normalize all predicted scores to $[0,1]$ to obtain the posterior probabilities. Then combining with the similarity matrix $\mathbf{S}$, a DPP distribution is constructed to sampling a distinct tag subset. The obtained results denoted as MLMG-DPP and LEML-DPP respectively. 


\subsection{Quantitative Results}
\label{sec: subsec experimental results}

As all compared methods (MLMG-DPP, LEML-DPP and DIA) and the proposed method D$^2$IA-GAN sample DPP models to generate tag subsets, we can generate multiple tag subsets using each method for each image. 
Specifically, MLMG-DPP and DIA generates 10 random tag subsets for each image. The weight of each tag subset is computed by summing the weights of all tags in the subset. 
Then we construct two outputs: the {\it single subset}, which picks the subset with the largest weight from these 10 subsets; and the {\it ensemble subset}, which merges 5 tag subsets with top-5 largest weights among 10 subsets into one unique tag subset. 
The evaluations of the single subset reflect the performance of distinctiveness of the compared methods. 
The evaluations of the ensemble subset measure the performance of both diversity and distinctiveness. Larger distinctiveness of the ensemble subset indicates higher diversity among the consisting subsets of this ensemble subset. Besides, we present two cases by limiting the size of each tag subset to 3 and 5, respectively. 

\renewcommand{\arraystretch}{0.9}
\begin{table}[t] %\scriptsize
%\vspace{-0.1in}
\begin{center}
%\tabcolsep 0.06in
\scalebox{0.74}{
\begin{tabular}{|p{.075\textwidth}|p{.105\textwidth}| p{.045\textwidth} p{.045\textwidth} p{.045\textwidth} | p{.045\textwidth} p{.045\textwidth} p{.045\textwidth} |}
\hline
evaluation & metric$\rightarrow$ &
\multicolumn{3}{|c|}{3 tags} & \multicolumn{3}{|c|}{5 tags}
\\
target & method$\downarrow$ & P$_{sp}$ & R$_{sp}$ & F$_{1-sp}$ & P$_{sp}$ & R$_{sp}$ & F$_{1-sp}$
 \\
 \hline \hline
 & \scalebox{0.8}{LEML-DPP \cite{LEML-ICML-2014}} & 34.64 & 25.21 & 27.76 & 29.24 & 35.05 & 30.29
\\ 
 \scalebox{1}{\multirow{1}{*}{single}} & \scalebox{0.8}{MLMG-DPP \cite{my-iccv-2015}} & 37.18 & 27.71 & 30.05 & 33.85 & 38.91 & 34.30
 \\
 \scalebox{1}{\multirow{1}{*}{subset}} & \scalebox{0.9}{DIA \cite{my-cvpr-2017-dia}} & 41.44 & 31.00 & 33.61 & 34.99 & 40.92 & 35.78
 \\
 & \scalebox{0.9}{D$^2$-GAN} & \textbf{42.96}  & \textbf{32.34} & \textbf{34.93} & \textbf{35.04} & \textbf{41.50} & \textbf{36.06}
 \\
 \hline \hline
%
%
 & \scalebox{0.8}{LEML-DPP \cite{LEML-ICML-2014}} & 34.62 & 38.09 & 34.32 & 29.04 & 46.61 & 34.02
\\
 \scalebox{1}{\multirow{1}{*}{ensemble}} & \scalebox{0.8}{MLMG-DPP \cite{my-iccv-2015}} & 30.44 &  34.88 & 30.70 & 28.99 & 43.46 & 33.05 		
 \\
 \scalebox{1}{\multirow{1}{*}{subset}} & \scalebox{0.9}{DIA \cite{my-cvpr-2017-dia}} & 35.73 & 33.53 & 32.39 & \textbf{32.62} & 40.86 & 34.31	
 \\
 & \scalebox{0.9}{D$^2$-GAN}  & \textbf{36.73} & \textbf{42.44} & \textbf{36.71} & 31.28 & \textbf{48.74} & \textbf{35.82}	
 \\
 \hline
\end{tabular}
}
\end{center}
\vspace{-0.05in}
\caption{ Results ($\%$) evaluated by semantic metrics on ESP Game. The higher value indicates the better performance, and the best result in each column is highlighted in bold.}
\label{table: result on espgame}
\vspace{-0.1in}
\end{table}


\renewcommand{\arraystretch}{0.9}
\begin{table}[t] %\scriptsize
%\vspace{-0.1in}
\begin{center}
%\tabcolsep 0.06in
\scalebox{0.74}{
\begin{tabular}{|p{.075\textwidth}|p{.105\textwidth}| p{.045\textwidth} p{.045\textwidth} p{.045\textwidth} | p{.045\textwidth} p{.045\textwidth} p{.045\textwidth} |}
\hline
evaluation & metric$\rightarrow$ &
\multicolumn{3}{|c|}{3 tags} & \multicolumn{3}{|c|}{5 tags}
\\
target & method$\downarrow$ & P$_{sp}$ & R$_{sp}$ & F$_{1-sp}$ & P$_{sp}$ & R$_{sp}$ & F$_{1-sp}$
 \\
 \hline \hline
  & \scalebox{0.8}{LEML-DPP \cite{LEML-ICML-2014}} & 41.42 & 24.39 & 29.00 & 37.06 & 32.86 & 32.98
 \\
 single & \scalebox{0.8}{MLMG-DPP \cite{my-iccv-2015}} & 40.93 & 24.29 & 28.61 & 37.06 & 33.68 & 33.29
 \\
 subset & \scalebox{0.9}{DIA \cite{my-cvpr-2017-dia}} & 42.65 & 25.07 & 29.87 & \textbf{37.83} & 34.62 & 34.11
 \\
 & \scalebox{0.9}{D$^2$-GAN}  & \textbf{43.57} & \textbf{26.22} & \textbf{31.04} & 37.31 & \textbf{35.35} & \textbf{34.41}
 \\
 \hline \hline
%
%
 & \scalebox{0.8}{LEML-DPP \cite{LEML-ICML-2014}} & 35.22 & 32.75 & 31.86 & 32.28 & 39.89 & 33.74
\\
 ensemble & \scalebox{0.8}{MLMG-DPP \cite{my-iccv-2015}} & 33.71 & 32.00 & 30.64 & 31.91 & 40.11 & 33.49	
 \\
 subset & \scalebox{0.9}{DIA \cite{my-cvpr-2017-dia}} & \textbf{35.73} & 33.53 & 32.39 & \textbf{32.62} & 40.86 & 34.31 
 \\
 & \scalebox{0.9}{D$^2$-GAN} & 35.49 & \textbf{39.06} & \textbf{34.44} & 32.50 & \textbf{44.98} & \textbf{35.34}
 \\
 \hline
\end{tabular}
}
\end{center}
\vspace{-0.05in}
\caption{ Results ($\%$) evaluated by semantic metrics on IAPRTC-12. The higher value indicates the better performance, and the best result in each column is highlighted in bold.}
\label{table: result on iaprtc12}
\vspace{-0.2in}
\end{table}



The quantitative results on ESP Game are shown in Table \ref{table: result on espgame}. For  single subset evaluations, D$^2$IA-GAN shows the best performance evaluated by all metrics for both 3 and 5 tags, while MLMG-DPP and LEML-DPP perform worst in all cases. The reason is that the learning of ML-MG/LEML and the DPP sampling are independent. For ML-MG, it enforces the ancestor tags to be ranked before its descendant tags, while the distinctiveness is not considered. There is much semantic redundancy in the top-k tags of ML-MG, which is likely to include fewer semantic paths than the ones of DIA and D$^2$IA-GAN. Hence, although DPP sampling can produce a distinct tag subset from the top-k candidate tags, it covers fewer semantic concept (remember that one semantic path represents one semantic concept) than DIA and D$^2$IA-GAN. 
For LEML, it treats each tag equally when training, totally ignoring the semantic distinctiveness. It is not surprising that LEML-DPP also covers fewer semantic concepts than DIA and D$^2$IA-GAN. 
%
In contrast, both DIA and D$^2$IA-GAN take into account the semantic distinctiveness in learning. 
However, there are several significant differences between their training processes. 
Firstly, the DPP sampling is independent with the model training in DIA, while the generated subset by DPP sampling is used to updated the model parameter in D$^2$IA-GAN. 
Secondly, DIA learns from the ground-truth complete tag list, and the semantic distinctiveness is indirectly embedded into the learning process through the similarity matrix $\mathbf{S}$. In contrast, D$^2$IA-GAN learns from the ground-truth distinct tag subsets. 
Thirdly, the model training of DIA is independent of the evaluation metric F$_{1-sp}$, which plays the important role in the training process of D$^2$IA-GAN. 
These differences are the causes that D$^2$IA-GAN produces more semantically distinct tag subsets than DIA. 
Specifically, in the case of 3 tags, the relative improvements of D$^2$IA-GAN over DIA are $3.67\%, 4.32\%, 3.93\%$ at P$_{sp}$, R$_{sp}$ and F$_{1-sp}$,  respectively; while being $0.14\%, 3.86\%$ and 
 $0.78\%$ in the case of 5 tags. 
In addition, the improvement decreases as the size limit of tag subset increases. The reason is that D$^2$IA-GAN may include more irrelevant tags, as the random noise combined with the image feature not only brings in diversity, but also uncertainty. 
Note that due to the randomness of sampling, the results of single subset by DIA presented here are slightly different with those reported in \cite{my-cvpr-2017-dia}.

In terms of the evaluation of the ensemble subsets, the improvement of D$^2$IA-GAN over three compared methods is more significant. 
This is because all three compared methods sample multiple tag subsets from a fixed DPP distribution, while D$^2$IA-GAN generates multiple tag subsets from different DPP distributions with the random perturbations. 
As such, the diversity among the tag subsets generated by D$^2$IA-GAN is expected to be higher than those corresponding to three compared methods. 
Subsequently, the ensemble subset of D$^2$IA-GAN is likely to cover more relevant semantic paths than those of other methods.
%The reason is that different random noise vectors D$^2$IA-GAN will generate more diverse tag subsets, then the ensemble subset is likely to cover more relevant semantic paths than MLMG-DPP, LEML-DPP and DIA. 
It is supported by the comparison through the evaluation by R$_{sp}$: the relative improvement of D$^2$IA-GAN over DIA is $26.57\%$ in the case of 3 tags, while $19.29\%$ in the case of 5 tags. 
It is encouraging that the P$_{sp}$ scores of D$^2$IA-GAN are also comparable with those of DIA. It demonstrates that training using GAN reduces the likelihood to include irrelevant semantic paths due to the uncertainty of the noise vector $\z$, because GAN encourages the generated tag subsets to be close to the ground-truth diverse and distinct tag subsets. 
Specifically, in the case of 3 tags, the relative improvements of D$^2$IA-GAN over DIA are $2.80\%, 26.57\%, 11.77\%$ for P$_{sp}$, R$_{sp}$ and F$_{1-sp}$, respectively; the corresponding improvements are $-4.11\%, 19.29\%, 4.40\%$ in the case of 5 tags. 
%The negative improvement $-4.11\%$ tells that D$^2$IA-GAN includes more irrelevant semantic concepts than DIA in this case.

The results on IAPRTC-12 are summarized in Table \ref{table: result on iaprtc12}. 
In the case of single subset with 3 tags, the relative improvements of D$^2$IA-GAN over DIA are $2.16\%, 4.59\%, 3.92\%$ for P$_{sp}$, R$_{sp}$ and F$_{1-sp}$, respectively; 
In the case of single subset with 5 tags, the corresponding improvements are $-1.37\%, 2.11\%, 0.88\%$.
In the case of ensemble subset and 3 tags, the corresponding improvements are $-0.67\%, 16.49\%, 6.33\%$. 
In the case of ensemble subset and 5 tags, the corresponding improvements are $-0.37\%, 10.08\%, 3.0\%$. 
%The relative performance of three methods through this comparison is similar with the comparison on ESP Game. 
The comparisons on above two benchmark datasets verify that D$^2$IA-GAN produces more semantically diverse and distinct tag subsets than the compared MLMG-DPP and DIA methods.  
%Due to the space limit, 
Some qualitative results will be presented in the {\bf supplementary material}.


\vspace{-0.1in}
\subsection{Subject Study}
\label{sec: subsec subjet study}
\vspace{-0.05in}

Since the diversity and distinctiveness are subjective concepts,  
we also conduct human subject studies to compare the results of DIA and D$^2$-GAN on these two criterion. 
Specifically, for each test image, we run DIA 10 times to obtain 10 tag subsets, and then the set including 3 subsets with the largest weights are picked as the final output. 
%Note that if two subsets are completely same at both the included tags and the tag order, then one of them will be removed from the set. 
For D$^2$-GAN, we firstly generate 10 random noise vectors $\z$. With each noise vector, we conduct the DPP sampling in $\G_{\boldsymbol{\theta}}$ for 10 times to obtain 10 subsets, out of which we pick the one with the largest weight as the tag subset corresponding to this noise vector. Then from the obtained 10 subsets, we again pick 3 subsets with the largest weights to form the output set of D$^2$-GAN. 
%Such a set of at most three tag subsets aims to simulate crowd-sourcing image annotations. 
%We use the randomized perturbation to simulate multiple human annotators.
For each test image, we present these two sets of tag subsets with the corresponding image to 5 human evaluators. The only instruction to the subjects is to determine ``which set  describes this image more comprehensively". 
Besides, we notice that if two sets are very similar, or if they both are irrelevant to the image content, human evaluators may pick one randomly. To reduce such randomness, we filter the test images using the following criterion: firstly we combine the subsets in each set to an ensemble subset; if the F$_{1-sp}$ scores of both ensemble subsets are larger than 0.2, and the gap between this two scores is larger than $0.15$, then this image is used in subject studies. 
Finally, the numbers of test images used in subject studies are: ESP Game, $375$ in the case of 3 tags, and $324$ in the case of of 5 tags; IAPRTC-12, $342$ in the case of 3 tags, and $306$ in the case of of 5 tags. 
%
We also present the comparison results using the F$_{1-sp}$ to evaluate the compared two ensemble subsets. The consistency between the F$_{1-sp}$ evaluation and the human evaluation is also computed. 
%
The subject study results on ESP Game are summarized in Table \ref{table: subject study on espgame}. With human evaluation, D$^2$IA-GAN is judged better at $\frac{240}{375}=64\%$ of all evaluated images over DIA in the case of 3 tags, and $\frac{204}{324}=62.96\%$ in the case of 5 tags. 
With F$_{1-sp}$ evaluation, D$^2$IA-GAN outperforms DIA at $\frac{250}{375}=66.67\%$ in the case of 3 tags, and $\frac{212}{324}=65.43\%$ in the case of5 tags. 
Both evaluation results suggest the improvement of D$^2$IA-GAN over DIA. 
Besides, the results of these two evaluations are consistent ({\it i.e.}, their decisions of which set is better are same) at $\frac{239}{375}=63.73\%$ of all evaluated images of the case of 3 tags, while $\frac{222}{324}=68.52\%$ of the case of 5 tags. 
It demonstrates that the evaluation using F$_{1-sp}$ is relatively reliable. 
%
%The subject study results on IAPRTC-12 are shown in Table \ref{table: subject study on iaprtc12}. The superior performance of D$^2$IA-GAN to DIA, as well as the consistency between human evaluation and F$_{1-sp}$ evaluation, are again verified. 
The same trend is also observed for the results obtained on the IAPRTC-12 dataset (Table \ref{table: subject study on iaprtc12}). 

Moreover, in the {\bf supplementary material}, we will present a detailed analysis about human annotations conducted on partial images of IAPRTC-12. 
It not only shows that D$^2$IA-GAN produces more human-like tags than DIA, but also discusses the difference between D$^2$IA-GAN and human annotators, and how to shrink that difference.


\renewcommand{\arraystretch}{1}
\begin{table}[t] %\scriptsize
%\vspace{-0.1in}
\begin{center}
%\tabcolsep 0.06in
\scalebox{0.87}{
\begin{tabular}{|p{.106\textwidth}|p{.03\textwidth} p{.05\textwidth} p{.053\textwidth} | p{.03\textwidth} p{.05\textwidth} p{.053\textwidth} |}
\hline
\scalebox{0.7}{$\#$} tags $\rightarrow$ & \multicolumn{3}{c|}{3 tags} & \multicolumn{3}{c|}{5 tags} 
\\
\hline
\multirow{2}{*}{metric $\downarrow$} 
& \scalebox{0.8}{DIA}  & \scalebox{0.7}{D$^2$IA-GAN}  & \hspace{0.5em} \multirow{2}{*}{\scalebox{1}{total}} 
& \scalebox{0.8}{DIA}  & \scalebox{0.7}{D$^2$IA-GAN}  & \hspace{0.5em} \multirow{2}{*}{\scalebox{1}{total}}
 \\
  & \scalebox{0.75}{wins} & \hspace{0.55em} \scalebox{0.75}{wins} &  & \scalebox{0.75}{wins} & \hspace{0.55em} \scalebox{0.75}{wins} &
 \\
 \hline
 \scalebox{0.8}{human evaluation} 
 & 135 & \hspace{0.5em} 240 & \hspace{0.55em} 375 
 & 120 & \hspace{0.5em} 204 & \hspace{0.55em} 324
\\
  F$_{1-sp}$ & 125 & \hspace{0.5em} 250 & \hspace{0.55em} 375  
  & 112 & \hspace{0.5em} 212 & \hspace{0.55em} 324  
 \\
 consistency & 62 & \hspace{0.5em} 177 & $63.73\%$
 & 65 & \hspace{0.5em} 157 & $68.52\%$
 \\
 \hline
\end{tabular}
}
\end{center}
\vspace{-0.06in}
\caption{ \small{\small{Subject study results on ESP Game. Note that the entry `62' corresponding to the row `consistency' and the column `DIA wins' indicates that both human evaluation and F$_{1-sp}$ evaluation decide that the predicted tags of DIA are better than those of D$^2$IA-GAN at 62 images. Similarly, human evaluation and F$_{1-sp}$ evaluation have the same decision that the results of D$^2$IA-GAN are better than those of DIA at 177 images. Hence, two evaluations have the same decision ({\it i.e.}, consistent) on $62+177=239$ images, and the consistency rate among all evaluated images are $239/372=63.73\%$.}}
%The proportion $63.73\%$ tells that two evaluations have the consistent decision at $63.73\%$ of all evaluated images.
}
\label{table: subject study on espgame}
\vspace{-0.08in}
\end{table}


\vspace{-0.05in}
\renewcommand{\arraystretch}{1}
\begin{table}[t] %\scriptsize
%\vspace{-0.1in}
\begin{center}
%\tabcolsep 0.06in
\scalebox{0.87}{
\begin{tabular}{|p{.106\textwidth}|p{.03\textwidth} p{.05\textwidth} p{.053\textwidth} | p{.03\textwidth} p{.05\textwidth} p{.053\textwidth} |}
\hline
\scalebox{0.7}{$\#$} tags $\rightarrow$ & \multicolumn{3}{c|}{3 tags} & \multicolumn{3}{c|}{5 tags} 
\\
\hline
\multirow{2}{*}{metric $\downarrow$} 
& \scalebox{0.8}{DIA}  & \scalebox{0.7}{D$^2$IA-GAN}  & \hspace{0.5em} \multirow{2}{*}{\scalebox{1}{total}} 
& \scalebox{0.8}{DIA}  & \scalebox{0.7}{D$^2$IA-GAN}  & \hspace{0.5em} \multirow{2}{*}{\scalebox{1}{total}}
 \\
  & \scalebox{0.75}{wins} & \hspace{0.55em} \scalebox{0.75}{wins} &  & \scalebox{0.75}{wins} & \hspace{0.55em} \scalebox{0.75}{wins} &
 \\
 \hline
 \scalebox{0.8}{human evaluation} 
 & 129 & \hspace{0.5em} 213 & \hspace{0.52em} 342 
 & 123 & \hspace{0.5em} 183 & \hspace{0.52em} 306
\\
  F$_{1-sp}$ & 141 & \hspace{0.5em} 201 & \hspace{0.52em} 342  
  & 123 & \hspace{0.5em} 183 & \hspace{0.52em} 306  
 \\
 consistency & 82 & \hspace{0.5em} 154 & $69.01\%$
 & 58 & \hspace{0.5em} 118 & $57.52\%$
 \\
 \hline
\end{tabular}
}
\end{center}
\vspace{-0.07in}
\caption{ Subject study results on IAPRTC-12.
}
\label{table: subject study on iaprtc12}
\vspace{-0.2in}
\end{table}
\vspace{-0.05in}

\section{Conclusion}
\label{sec: conclusion}
\vspace{-0.05in}

In this work, we have proposed a new image annotation method, called {\it diverse and distinct image annotation} (D$^2$IA), to simulate the diversity and distinctiveness of the tags generated by human annotators. 
D$^2$IA is formulated as a sequential generative model, in which the image feature is firstly incorporated into a determinantal point process (DPP) model that also encodes the weighted semantic paths, from which a sequence of distinct tags are generated by sampling. 
The diversity among the generated multiple tag subsets is ensured by sampling the DPP model with random noise perturbations to the image feature.
In addition, we adopt the generative adversarial network (GAN) model to train the generative model D$^2$IA, and employ the policy gradient algorithm to handle the training difficulty due to the discrete DPP sampling in D$^2$IA. 
Experimental results and human subject studies on benchmark datasets demonstrate that the diverse and distinct tag subsets generated by the proposed method can provide more comprehensive descriptions of the image contents than those generated by the state-of-the-art methods.


\vspace{0.05in}
\noindent
\textbf{Acknowledgements}:
This work is supported by Tencent AI Lab. The participation of Bernard Ghanem is supported by the King Abdullah University of Science and Technology (KAUST) Office of Sponsored Research.  
The participation of Siwei Lyu is partially supported by National Science Foundation National Robotics Initiative (NRI) Grant (IIS-1537257) and National Science Foundation of China Project Number 61771341. 


{\small
\bibliographystyle{ieee}
\bibliography{bywu_bib}
}

\end{document}
