\documentclass[sigconf,10pt]{acmart}

%\usepackage[•]{•}{microtype}

\pdfoutput=1
\usepackage{booktabs} % For formal tables
%\usepackage[subtle]{savetrees}
%\settopmatter{printacmref=true}
% Copyright
%\setcopyright{none}
%\setcopyright{acmcopyright}
\setcopyright{acmlicensed}
%\setcopyright{rightsretained}
%\setcopyright{usgov}
%\setcopyright{usgovmixed}
%\setcopyright{cagov}
%\setcopyright{cagovmixed}
\newcommand{\minihead}[1]{{\vspace{.5em}\noindent\textbf{#1} }}
\newcommand{\red}[1]{{\color{black}#1}}
\newcommand{\mvar}{\red{d}}
\newcommand{\dvar}{\red{n}}

\newcommand\code[1]{\lstinline$#1$}

%%%%%%%%%%%%
\renewcommand\paragraph{\@startsection{paragraph}{4}{\z@}%
                                    {0.5ex \@plus 0.5ex \@minus .2ex}%
                                    {-0.5em}%
                                    {\normalfont\normalsize\bfseries}}
\usepackage[labelfont=bf]{caption}
\setlength{\intextsep}{5pt plus 1.0pt minus 2.0pt}
\setlength{\abovedisplayskip}{1pt}
\setlength{\belowdisplayskip}{1pt}
%\usepackage[small,compact]{titlesec}

%\newenvironment{denseitemize}{
%\begin{itemize}[topsep=2pt, partopsep=0pt, leftmargin=1.5em]
%  \setlength{\itemsep}{4pt}
%  \setlength{\parskip}{0pt}
%  \setlength{\parsep}{0pt}
%}{\end{itemize}}
%
%\newenvironment{denseenum}{
%\begin{enumerate}[topsep=2pt, partopsep=0pt, leftmargin=1.5em]
%  \setlength{\itemsep}{4pt}
%  \setlength{\parskip}{0pt}
%  \setlength{\parsep}{0pt}
%}{\end{enumerate}}

%\newcommand{\subparagraph}{}
%\usepackage[small,compact]{titlesec}
%\renewcommand{\paragraph}[1]{\vspace{1mm}\noindent \textbf{#1}}
%\usepackage[labelfont=bf,skip=2pt,belowskip=2pt]{caption}
%%%%%%%%%%%

\usepackage{enumitem}


\usepackage{algorithm}
\usepackage[noend]{algpseudocode}
\algdef{SE}[DOWHILE]{Do}{doWhile}{\algorithmicdo}[1]{\algorithmicwhile\ #1}%

\theoremstyle{problem}
\newtheorem{problem}{Problem}[section]




\begin{document}
\title{DROP: A Workload-Aware Optimizer for Dimensionality Reduction}


\author{Sahaana Suri, Peter Bailis}
\affiliation{
  \institution{Stanford University}
}

\renewcommand{\shortauthors}{S. Suri and P. Bailis}


\copyrightyear{2019} 
\acmYear{2019} 
\setcopyright{acmlicensed}
\acmConference[DEEM'19]{International Workshop on Data Management for End-to-End Machine Learning}{June 30, 2019}{Amsterdam, Netherlands}
\acmBooktitle{International Workshop on Data Management for End-to-End Machine Learning (DEEM'30), June 30, 2019, Amsterdam, Netherlands}
\acmPrice{15.00}
\acmDOI{10.1145/3329486.3329490}
\acmISBN{978-1-4503-6797-4/19/06}


\begin{abstract}
Dimensionality reduction (DR) is critical in scaling machine learning pipelines: by reducing input dimensionality in exchange for a preprocessing overhead, DR enables faster end-to-end runtime. Principal component analysis (PCA) is a DR standard, but can be computationally expensive: classically $O(dn^2 + n^3)$ for an $n$-dimensional dataset of $d$ points. 
Theoretical work has optimized PCA via iterative, sample-based stochastic methods. 
However, these methods execute for a fixed number of iterations or to convergence, sampling too many or too few datapoints for end-to-end runtime improvements. 
We show how accounting for downstream analytics operations during DR via PCA allows stochastic methods to efficiently terminate after processing small (e.g., 1\%) samples of data. 
Leveraging this, we propose DROP, a DR optimizer that enables speedups of up to \red{$5\times$} over \red{Singular-Value-Decomposition (SVD)-based} PCA, and \red{$16\times$} over conventional DR methods in end-to-end nearest neighbor workloads.
\end{abstract}


\begin{CCSXML}
<ccs2012>
<concept>
<concept_id>10010147.10010257.10010258.10010262.10010277</concept_id>
<concept_desc>Computing methodologies~Transfer learning</concept_desc>
<concept_significance>500</concept_significance>
</concept>
<concept>
<concept_id>10002951.10003227.10003351</concept_id>
<concept_desc>Information systems~Data mining</concept_desc>
<concept_significance>300</concept_significance>
</concept>
</ccs2012>
\end{CCSXML}

\maketitle

%\begin{abstract}
\label{sec:abstract}

%% 1. what is the problem 
Scientific applications that run on leadership computing facilities often face the challenge 
of being unable to fit leading science cases onto accelerator devices due to memory constraints 
(memory-bound applications).
%
% 2. what is your solution 
In this work, the authors studied one such US Department of Energy mission-critical condensed matter 
physics application, Dynamical Cluster Approximation (DCA++), and this paper discusses how device memory-bound challenges were successfully reduced  by proposing an effective 
``all-to-all'' communication method---a ring communication algorithm. 
%
This implementation takes advantage of acceleration on GPUs and remote direct memory access (RDMA) for fast data exchange between GPUs. 
%
\\Additionally, the ring algorithm was optimized with sub-ring communicators
and multi-threaded support to further reduce communication overhead and 
expose more concurrency, respectively.
%
% 3. What's the cherry-picked evaluation result you want to mention
The computation and communication were also analyzed 
by using the Autonomic Performance Environment for Exascale 
(APEX) profiling tool,  and this paper further discusses the 
performance trade-off for the ring algorithm implementation. 
%
The memory analysis on the ring algorithm shows that the allocation size for the authors' most 
memory-intensive data structure per GPU is now reduced to $1/p$ of the original size, where $p$ is the number of GPUs in the ring communicator.
%
The communication analysis suggests that 
the distributed Quantum Monte Carlo execution time grows linearly as sub-ring size increases, and the cost of messages passing through the network interface connector could be a limiting factor.


%
% \todoRed{Ronnie: Next sentence needs rewrite, too much information about Green's function that no one knows in the abstract; recommend generalizing.} \emph {However, DCA++ is currently facing memory-bound challenge as 
% a larger device array $G_t$ is limited by device memory size, where
% $G_t$ is a two-particle Green's function that allows condensed matter
% scientists to explore larger and more complex (higher fidelity)
% physics cases.}

\end{abstract}

\keywords{DCA++, Quantum Monte Carlo, GPU Remote Direct Memory Access, memory-bound issue, exascale machines}

Reinforcement learning has achieved great success in areas such as Game-playing \citep{silver2018general,vinyals2019grandmaster}, robotics \cite{kober2013reinforcement}, large language models \citep{ouyang2022training}, etc.
However, due to safety concerns or physical limitations, in some real-world reinforcement learning problems, we must consider additional constraints that may influence the optimal policy and the learning process \citep{garcia2015comprehensive}.
% For example, a robotic arm must not take actions that may cause harm to itself or the environments.
A standard framework to handle such cases is the constrained Markov Decision Process (CMDP) \citep{altman1999constrained}.
Within the CMDP framework, the agent has to maximize
the expected cumulative reward while
obeying a finite number of constraints, which are usually in the form of expected cumulative cost criteria.

However, we are sometimes concerned with the problem with a continuum of constraints.
For example,
the constraints we meet might be time-evolving or subject to uncertain parameters, which
cannot be formulated as an ordinary CMDP
(see Examples \ref{Example_Time_Evolving} and  \ref{Example_Uncertain}).
In this paper we would study a generalized CMDP  
to address the above problem.  Because the constraints are not only infinite-number but also lie
in a continuous set,
the generalization is not trivial. Fortunately, we find that we can borrow the idea behind semi-infinite programming (SIP) \citep{remez1934determination, hettich1993semi} to deal with the semi-infinite constraints.
Accordingly, we propose \emph{semi-infinitely constrained Markov decision processes} (SICMDPs)
as a novel complement to the ordinary CMDP framework.
%More specifically,  an SICMDP model %, we consider 
%contains a continuum of constraints whereas an ordinary CMDP contains a finite number of constraints. 

%This generalization is natural but not trivial. However, we can brows the idea  
%The idea is quite natural and can be backtracked
%to the practice of extending linear programming to linear semi-infinite programming (LSIP) %\cite{remez1934determination, GobernaLSIO1998}.
%In addition, 
%As a complementary approach to the ordinary CMDP framework, 
%SICMDP can be used to model these problems  which cannot be described by a finite number of constraints
%that are not covered by .
%For example,
%the restrictions we consider can be time-evolving or subject to uncertain parameters
%, thus
%cannot be described by a finite number of constraints but a continuum of constraints 
%(see Examples \ref{Example_Time_Evolving} and  \ref{Example_Uncertain}).

We also present two reinforcement learning algorithms to solve SICMDPs called SI-CRL and SI-CPO, respectively.
SI-CRL is a model-based reinforcement learning algorithm designed for tabular cases, and SI-CPO is a policy optimization algorithm for non-tabular cases.
% and analyze its performance both theoretically and empirically.
The main challenge is that we need to deal with a continuum of constraints, thus reinforcement learning algorithms for ordinary CMDPs do not work anymore.
In SI-CRL, we tackle this difficulty by first transforming the reinforcement learning problem to an equivalent LSIP problem, which can then be solved using methods in the LSIP literature like the dual exchange methods \citep{Hu1990,reemtsen1998numerical}.
In SI-CPO, we resort to the idea of cooperative stochastic approximation developed in \cite{lan2020algorithms, wei2020comirror}.
As far as we know, we are the first to introduce tools from semi-infinitely programming (SIP) into the reinforcement learning community for solving constrained reinforcement learning problems.

% To the best of our knowledge, we are the first to apply tools from semi-infinitely programming (SIP) to solve reinforcement learning problems.
Furthermore, we give theoretical analysis for both SI-CRL and SI-CPO.
We decompose the error of SI-CRL into two parts: the statistical error from approximating the true SICMDP with an offline dataset and the optimization error due to the fact that the solution of the LSIP problem obtained by the dual exchange method is inexact.
On the optimization side, we show that the iteration complexity of SI-CRL is $O\left(\left\{\mathrm{diam}(Y)L\sqrt{|\gS|^2|\gA|m}/\left[(1-\gamma)\epsilon\right]\right\}^m\right)$.
On the statistical side, we show that the sample complexity of SI-CRL is $\widetilde O\left(\frac{|S|^2|A|^2}{\epsilon^2(1-\gamma)^3}\right)$ if the offline dataset is generated by a generative model, and $\widetilde O\left(\frac{|S||A|}{\nu_{\min} \epsilon^2(1-\gamma)^3}\right)$ if the dataset is generated by a probability measure $\nu$ as considered in \cite{chen2019information}.
Here $\widetilde O$ means that all logarithm terms are discarded.
For SI-CPO, things become a little more complicated because other than the statistical error and the optimization error, we also need to consider the function approximation error, which comes from imperfect policy parametrizations.
It is shown if the function approximation error can be controlled to $O(\epsilon)$ order, the iteration complexity of SI-CPO is $\widetilde{O}\left(\frac{1}{\epsilon^2(1-\gamma)^6}\right)$ and the sample complexity of SI-CPO is $\widetilde{O}(\frac{1}{\epsilon^4(1-\gamma)^{10}})$.
Here our iteration complexity bound is equivalent to a typical $\widetilde O(1/\sqrt{T})$ global convergence rate.

We perform a set of numerical experiments to illustrate the SICMDP model and validate our proposed algorithms.
Specifically, we examine two numerical examples, namely the discharge of sewage and ship route planning.
Through the discharge of sewage example, we show the advantage of the SICMDP framework over the CMDP baseline obtained by naive discretization in modeling realistic sequential decision-making problems.
Moreover, we demonstrate the effectiveness of the SI-CRL and SI-CPO algorithms in such tabular environments. 
In the ship route planning example, we illustrate the benefits of the SICMDP framework and the ability of the SI-CPO algorithm to address complex continuous control tasks involving continuous state spaces with modern deep reinforcement learning techniques.

% In summary, our contributions are listed as follows.
% First, we present the SICMDP model, which can be viewed as a generalization of the ordinary CMDP model.
% Second, we propose an algorithm to perform reinforcement learning for SICMDPs, which is called SI-CRL, and we believe that we are the first to apply tools from SIP
% to solve reinforcement learning problems.
% Third, we give a theoretical analysis of SI-CRL and identify both its sample complexity and iteration complexity.
% In addition, we perform numerical experiments to illustrate the SICMDP model and validate the SI-CRL algorithm.
% \{This paragraph can be removed!!! \}





\section{Background and Related Work}
\label{sec:relwork}

\subsection{Watermarking}

Watermarking of textual data has been extensively studied~\citep{kamaruddin18,rizzo17,ahvanooey19}
It can be viewed as a form of steganography~\citep{cox_information_2005,majeed_review_2021} with a one-bit message.
Steganography imposes the stronger requirement that, without knowledge of the secret key, the distribution of watermarked and unwatermarked outputs be indistinguishable; however, indistinguishability may be useful in some settings, as it implies that the mark does not harm the quality of the generated text.
Watermarking the output of a large language model can be viewed as the problem of embedding a mark when we can sample from the output distribution with limited additional control~\citep{hopper2008stego,ziegler_neural_2019a,xiang_novel_2020,kang_generative_2020,cao_generative_2022}. 

\subsection{AI Detectors}

Another approach is to use train a classifier to detect LLM-generated text.
This approach avoids the need to modify how text is generated by the LLM.
Many schemes can be found in the literature,
including OpenAI's classifiers~\citep{openai_gpt2outputdataset_2021,openai_new_2023}, GLTR~\citep{gehrmann_gltr_2019a}, DetectGPT~\citep{mitchell_detectgpt_2023},
and others~\citep{bakhtin_real_2019,zellers_defending_2019a,ippolito_automatic_2020,uchendu_authorship_2020,fagni_tweepfake_2021}.
Unfortunately, current LLMs have gotten so good at generating natural-looking text that these detectors have become unreliable, and as LLMs improve, these detectors will perform even worse.

For proprietary LLMs, another alternative is for the vendor to keep a copy of all generated outputs from their LLM, and provide an API that can be used to look up any text to determine whether it was previously produced with their LLM~\citep{krishna_paraphrasing_2023}.


% learned classifiers:
% - https://arxiv.org/abs/1905.12616
% - https://arxiv.org/abs/1906.03351
% - https://arxiv.org/abs/1911.00650
% - https://aclanthology.org/2020.emnlp-main.673/
% - TweepFake: https://journals.plos.org/plosone/article?id=10.1371/journal.pone.0251415

% zero-shot methods:
% - https://arxiv.org/abs/1906.04043
% - https://arxiv.org/abs/2301.11305v1

% database lookup:
% - https://arxiv.org/abs/2303.13408

% \subsection{Attacks}

% \chawin{Overlap with section 5; remove?}
% Various potential attacks against watermarking methods have also been examined, and of particular interest are paraphrasing attacks (Kirchenbauer et al., Krishna et al., Sadasivan et al.).


% Panoptic segmentation

% 3D segmentation

% Multi-object tracking

% Online 3D panoptic:

% PanopticFusion: (IROS 2019)
% https://arxiv.org/pdf/1903.01177.pdf
%
% - most similar to ours
% - PSPNet + M-RCNN + 2D fusion
% - volumetric mapping, 
% - greedy matching with IoU -> optimal only with 0.5 threshold
% - voxel & class weighting
% - CRF regularisation
%
% - good:
%
% - bad:
%  - CRF post-processing step
%  - greedy data-association
%    - can't be tuned for lower overlap ratios -> has to have high framerate, large changes in viewpoint could break this
%    - IoU: sensitive to 2D labels projecting over object borders (CRF and voxel weighting seem to alleviate this)

% Voxblox++: (Robotics & automation letters 2019)
% https://arxiv.org/pdf/1903.00268.pdf
% https://github.com/ethz-asl/voxblox-plusplus
%
% - M-RCNN + geometric segmentation + fusion 
% - data association of geometric segments with 3D overlap (no. points inside volume), fixed threshold for min number of points
% - instance label is assigned to a segment based on highest overlap
% - only one detected segment per reference label, as in PanopticFusion and Ours
% - TSDF Integration 
%
% good: 
% - because of geometric segmentation objects with no associated semantic class can also be segmented
% bad:
% - two different object segment types -> confusing, overly complicated ?
% - quite inaccurate (fixed below)

% Reconstructing Interactive 3D Scenes by Panoptic Mapping and CAD Model Alignments (ICRA 2021)
% https://arxiv.org/pdf/2103.16095.pdf
% https://github.com/hmz-15/Interactive-Scene-Reconstruction
%
% - based heavily on Voxblox++, much more accurate
% - Scene-graph ("contact graph") for mapping object relations
% - Search & replace voxels with CAD models, with geometrical and physical constraints
% - Object 6D pose
% - Format for robot interaction
%
% - Segmentation: bilateral fusion of geomatric and semantic segments -> reduce segmentation noise compared to Voxblox++
% - Fusion: triplet count improves consistency over Voxblox++ pairwise count strategy (take semantic label into account in addition to instance and geometry)
% - Fusion: instance labels are also combined if there is enough overlap with common geometric label for long enough time
%   - this means multiple detections can match the same reference unlike ours, voxblox++ and PanopticFusion ?
%

% Panoptic-MOPE: (ROBOTICS AND AUTOMATION LETTERS 2020)
% https://ieeexplore.ieee.org/stamp/stamp.jsp?tp=&arnumber=8977356
% https://github.com/hoangcuongbk80/Object-RPE/tree/panoptic-mope
%
% - novel RGB-D semantic segmentation model + M-RCNN
% - camera tracking based on "addaptively weighted optimization of geometric, appearance, and semantic cues"
% - surfel map: 
%   - how does it scale ? authors satate they tested on room-sized environments, but could be applied in larger scale as well ...
%     - could maybe be applied as VO in a SLAM algorithm ...
%   - demo only on a small pallet + surroundings, might not be applicable in large-scale SLAM

% US VS THEM:
%
% - based heavily on PanopticFusion, with modifications:
%   - instead of greedy data-association (which seems to be the case in others as well), we solve LAP (JPDA?)
%     - overlap threshold can be tuned, which renders the algorithm more flexible
%     - could be extended to dynamic tracking ?
%   - multiple options for association likelihood
%   - outlier rejection (either clustering or probabilistic)
%   - test different options for decreasing processing time
%   - no post-processing
%
% - model-agnostic:
%   - completely separated from segmentation
%   - does not care how point clouds are obtained -> applicable for LIDAR segmentation (e.g. EfficientLPS) as well
%
% - also agnostic to localisation method
%   - could, however, be utilised to find landmark locations / poses

% More compact version of this paragraph to introduction to save space?
%Panoptic segmentation -- proposed in \cite{panoptic_segmentation} -- aims to solve the unified task of semantic- and instance segmentation. Semantic classes are separated to \textit{stuff} -- amorphous, unquantifiable regions like sky, road or floor -- and \textit{things} -- quantifiable objects. The distinction between the two can vary depending on the application, but a semantic class can only belong to one or another. The article also proposes a unified panoptic evaluation metric, coined \textbf{Panoptic Quality} (PQ). Many 2D approaches to panoptic segmentation -- \textit{e.g.} \cite{panopticfpn,seamless,panoptic_deeplab,efficientps} -- have since been proposed. Deep neural networks for performing semantic- or instance segmentation directly on the 3D reconstruction -- \textit{e.g.} on \cite{scannet,s3dis,paris_lille_3d} -- have also been proposed, but since they require the reconstructed 3D scene, they are mostly offline approaches and therefore out of scope for this work. Some recent works also apply panoptic segmentation to point clouds -- \textit{e.g.} methods in the SemanticKITTI panoptic segmentation competition \cite{semantic_kitti} -- mostly aimed at segmenting LiDAR output. They are suitable for online processing, but similar to RGB-D images require a method for tracking object instances persistent in both time and space. In fact, our proposed method, as well as some others mentioned in this work, could use segmented LiDAR point clouds as an input similarly to RGB-D images.

PanopticFusion \cite{panopticfusion} is the first work to propose online integration of panoptic image segmentations to a 3D reconstruction. They integrate point clouds generated from segmented images to a TSDF voxel volume \cite{tsdf,voxblox} by greedily matching detected segments with the reconstruction and regulating each voxel's corresponding instance with a weighting function. Semantic labels are inferred in a bayesian manner based on confidence scores provided by the segmentation model. They also apply a Conditional Random Field (CRF) to regularise the reconstruction, improving results significantly. Voxblox++ \cite{voxblox++} -- introduced later the same year -- is a similar approach that also integrates segmented RGB-D images into a TSDF volume. It leverages geometric segmentation of depth images to improve instance segmentation accuracy. Both geometric and semantic segments are used to compute a pair-wise weight, which is used to greedily match them with segments in the reconstruction. Because of the geometric segmentation, the method allows segmentation of objects with no known semantic class in addition to objects recognised by the instance segmentation model. 

Recently, \cite{interactive_3d_scenes} built upon the idea of Voxblox++. They apply Voxblox++ for 3D instance integration, with two small but effective modifications: the pair-wise weight is replaced by a triplet weight that also takes semantic labels into account in the fusion, and -- in addition to geometric segments -- instance segments are fused if they overlap by a significant amount. The article introduces a method for searching and aligning CAD models to reconstructed objects based on geometry and semantic class, as well as geometrical and physical rules. With the CAD models, a contact graph and interactive virtual scene are reconstructed to allow a robot to simulate its interaction with the environment. SceneGraphFusion \cite{scenegraphfusion} is another approach that forms a scene graph online from a stream of RGB-D images, but unlike the above-mentioned approach, it generates the graph with a deep neural network, after which the panoptic labels for geometrically segmented portions of the 3D reconstruction are produced a side product.

Panoptic-MOPE \cite{panoptic_mope} is another recent approach, which integrates sequences of RGB-D images into a surfel reconstruction. Unlike other mentioned approaches -- which assume the camera pose either known or estimated elsewhere -- it also tracks camera movements based on geometric-, appearance- and semantic cues. The method also applies a novel RGB-D panoptic segmentation model. Although it is only tested on room-sized environments, the authors claim it could be scaled to larger environments as well.
%\section{Sample-Based PCA}
\label{sec:sampling}

To tackle workload-aware DR, we demonstrate how sample-based PCA can bridge the performance , but that the number of samples required varies per dataset.
Finally, we show how dynamically increasing the sampling rate can help identify how much to sample a given dataset, providing a foundation for workload-aware DR.

\begin{figure}
\includegraphics[width=\linewidth]{figs/progressive.pdf}
\caption[]{ Improvement in representation size for  $TLB = 0.80$ across three datasets. Higher sampling rates improve quality until reaching a state equivalent to running PCA over the full dataset ("convergence")}
\label{fig:progressive}
\end{figure}

\begin{comment}
\subsection{PCA Speed vs. Quality}

While improved quality provides faster repeated query execution, the cost of DR via PCA dominates this speedup, encouraging the use of faster, lower-quality alternatives~\cite{keogh-study}. 

To briefly quantify this trade-off, we augment a widely-cited time series similarity search DR study from VLDB 2008~\cite{keogh-study} by evaluating PCA---which the authors did not benchmark due to it being ``untenable for large data sets." 
We compare PCA via SVD to baseline techniques based on runtime and DR performance with respect to $TLB$ over the largest datasets from~\cite{keogh-study}. 
We use their two fastest methods as our baselines as they show the remainder exhibited ``very little difference'': Fast Fourier Transform (FFT) and Piecewise Aggregate Approximation (PAA).

\minihead{TLB Performance Comparison}
We compute the minimum dimensionality achieved by each technique subject to a $TLB$ constraint. 
On average, PCA provides bases that are $2.3\times$ (up to $3.9\times$) and $3.7\times$ (up to $26\times$)  smaller than PAA and FFT for $TLB = 0.75$, and $2.9\times$ (up to $8.3\times$) and $1.8\times$ (up to $5.1\times$) smaller for $TLB = 0.99$.
While the margin between PCA and alternatives is dataset-dependent, PCA almost always preserves $TLB$ with a lower dimensional representation.

%\section{Additional End-to-End Plots}
%\input{endendplots}

\minihead{Runtime Performance Comparison} 
PCA implemented via out-of-the-box SVD is on average over \red{$26\times$ (up to $56\times$)} slower than PAA and over \red{$4.6\times$ (up to $9.7\times$)} times slower than FFT when computing the smallest $TLB$-preserving basis.
%This substantiates the observation that classic PCA is incredibly slow compared to alternatives~\cite{keogh-study}. 

\end{comment}

\subsection{Incremental, Progressive Sampling}
To bridge this performance-runtime gap, we turn to data sampling. 
Many real-world \red{datasets} are intrinsically low-dimensional, as evidenced by their rapid falloff in their eigenvalue spectrum.
A data sample thus captures much of the dataset's ``interesting'' behavior, so fitting models over data samples generalize well. 
We verify this by varying the target $TLB$ and examining the minimum number of uniformly selected samples required to obtain a $TLB$-preserving transform with output dimension $k$ equal to input dimension $\dvar$.

On average, a sample of under $0.64\%$ $(\text{up to } 5.5\%)$ of the input is sufficient for $TLB = 0.75$, and under $4.2\%$ $(\text{up to } 38.6\%)$ is sufficient for $TLB=0.99$.  
If this sample rate is known, we obtain up to \red{$91\times$ speedup} compared to a na\"ive implementation of PCA via SVD---with no algorithmic improvement. 

However, this benefit is dataset-dependent, and unknown a priori.
We thus turn to progressive sampling (gradually increasing the sample size) to identify how large a sample suffices.
Figure~\ref{fig:progressive} shows how the dimensionality required to attain a given $TLB$ changes when we vary dataset and proportion of data sampled.
Increasing the number of samples provides lower dimensional transformations for the same quality.
However, this decrease in dimension plateaus as the number of samples increases.
Thus, while progressive sampling would allow us to tune the amount of time spent on DR, we must determine when the downstream value of decreased dimension is overpowered by the cost of additional DR---that is, whether to sample to convergence (evaluated in \S\ref{subsec:lesion}) or terminate early (e.g., at $0.3$ proportion of data sampled for SmallKitchenAppliances). 






\section{Efficiently Recognizing {\em Cyclic Hyper Degrees}}
This section consists of two parts. In the first part will culminate with Theorem~\ref{thm:shiftrange} which provides an efficiently computable closed form formula for the range of values taken by contiguous sum of $N$ elements in a list $c_{i,n}$, for any $i$. In the second part we will show how to use Theorem~\ref{thm:shiftrange} to decide if a given degree sequence is a {\em cyclic hyper degree}.

The elements in the columns of $T_n$ do not change their relative position after application of a cyclic permutation when seen as a cyclic list. We shall use this property to efficiently search for possible bit subsets which may sum up to a given input degree sequence.

\subsection{Contiguous Sum of Bit Lists}

\begin{definition}[Contiguous Sum]
\label{def:csum}
 Given a list $L$ of length $m$, the contiguous sum of $N$ elements in $L$ starting at the index $i\in [m]$ is defined to be
 $$\mathcal{S}(L,i,N):=\sum_{j=0}^{N-1} L(1+ ((i+j-1)\mod m)).$$
\end{definition}

The summation above treats the list $L$ as a cyclic list. Next, we prove that the contiguous sum function is a `continuous' function, this property will allow us to specify the range of sum by stating the minimum and the maximum value taken by it.
 Note that if $L$ is a $0$-$1$ list, for any index $\ell \in [m]$, 
 we have $\vert \mathcal{S}(L,\ell,N)-\mathcal{S}(L,\ell+1,N) \vert \in \{0,1\}$. This fact gives us the following property.

\begin{observation}
\label{obs:continuity}
 Let $L$ be a size $m$ list having $0$-$1$ entries and $N\in \mathbb{Z}_+$.
 If $v_i=\mathcal{S}(L,i,N)$ and $v_j=\mathcal{S}(L,j,N)$, for some $i,j\in[m]$, then 
 for every $v\in \mathbb{Z}_+$ contained between  $v_i$ and $v_j$
 there exists a $k\in [m]$  such that $\mathcal{S}(L,k,N)=v$.
\end{observation}

As the lists $c_{i,n}$ are over $0$-$1$ we get an easy relation between the maximum and minimum values taken by the contiguous sum as follows.

\begin{lemma}
\label{lem:sumcomplement}
  Let  $j\in\{0,\dots,n\}$, $i \in [n]$ and $N\in[2^n]$. The minimum of the sum of $N$ contiguous bits in a bit list $c_{i,n}$ is $m$ if and only if its maximum is $N-m$.
\end{lemma}
\begin{proof}
 Let $\overline{c}_{i,n}$ be the bit list obtained from the list $c_{i,n}$ by flipping each zero to one and vice versa. Let $\sigma_{2^{i-1}}$ be an order $2^{i-1}$ cyclic permutation, observe that $\overline{c}_{i,n}$ is equal to $\sigma_{2^{i-1}}(c_{i,n})$.
 If the minimum value is obtained at the contiguous segment which starts at the index $j$ in $c_{i,n}$, then the value $N-m$ can be obtained by the contiguous sum starting at index $\sigma_{2^{i-1}}(j)$. Finally, note that $m$ is the minimum value if and only if $N-m$ is the maximum value.
\end{proof}

Combining Observation~\ref{obs:continuity} and Lemma~\ref{lem:sumcomplement} we get the following.

\begin{lemma}
\label{lem:minmax}
 Let $N\in[2^n]$ and $m=\min_{j \in [2^n]} \mathcal{S}(c_{i,n}, j, N)$. For every value $v$ in the range $\{m,\dots, N-m\}$ there exists a $j\in [2^n]$ such that $\mathcal{S}(c_{i,n}, j, N)=v$.
\end{lemma}

The lemma above allows us to find the range of values taken by the contiguous sum by just finding the minimum value taken by it.
Next we prove a simpler lemma about the range of values taken. Using that, in Theorem~\ref{thm:shiftrange}, we will find the range of values taken by the contiguous sum of $N$ elements in any list $c_{i,n}$.

\begin{lemma}
\label{lem:shiftpower}
 For  $j\in\{0,\dots,n\}$ and $i \in [n]$, the sum of $2^j$ contiguous bits in a bit list $c_{i,n}$ takes the following values.
\begin{enumerate}
\item
\label{enum:shiftpower-one}
If $j \leq (i-1)$, then the range is $\{0, \dots,2^j \}$, and
\item
\label{enum:shiftpower-two}
If $j\geq i$, then the sum is exactly $2^{j-1}$.
\end{enumerate}
\end{lemma}
\begin{proof}
 By Lemma~\ref{lem:minmax}, it suffices to find the minimum value of contiguous sum function. Notice that we have, $c_{i,n}= (0_{\times 2^{i-1}} \cdot 1_{\times 2^{i-1}})_{\times 2^{n-i}}$, by Observation~\ref{obs:zero-one-power}.
 \begin{enumerate}
 \item
 When $j\leq (i-1)$, we can pick a block of $2^j$ zeros giving a total of zero, which is the minimum possible value.
 \item
 When $j\geq i$, let $L_k$ be a list of $2^j$ contiguous bits of $c_{i,n}$ starting at the index $k$ in $c_{i,n}$. To prove that $\sum L_k = \sum L_{k+1} $, it suffices to show that $c_{i,n}(k)=c_{i,n}(k+2^j)$. Rewriting
 $c_{i,n}=((0_{\times 2^{i-1}}\cdot 1_{\times 2^{i-1}})_{\times 2^{j-i}})_{\times 2^{n-j}}$ shows that any two indices with difference equal to $2^j$ store the same value. As the choice of $k$ was arbitrary, the contiguous sum is equal to $2^{j-1}$.
 \end{enumerate} 
\end{proof}

\begin{theorem}
\label{thm:shiftrange}
 For $i \in [n]$, $N\in[2^{n}]$ and $p=2^i$, the sum of $N$ contiguous bits in a bit list $c_{i,n}$ takes values in the range, ${\rm range}(i,N) \triangleq$
 $$ 
 \left\{
	\floor[\Big]{\frac{N}{p}} \frac{p}{2} + \max \left( (N\mod p) - \frac{p}{2}, 0 \right),
  	\cdots,
   	\floor[\Big]{\frac{N}{p}} \frac{p}{2} + \min \left( N\mod p, \frac{p}{2} \right)
 \right\}.
 $$
\end{theorem}
\begin{proof}
 For a fixed $i\in[n]$ consider the list $c_{i,n}$. Assuming that the minimum value of the range is as claimed, by Lemma~\ref{lem:minmax}, the maximum value is
\begin{align*}
   \max_{j \in [2^n]} \mathcal{S}(c_{i,n}, j, N)
   &= N-\min_{j \in [2^n]} \mathcal{S}(c_{i,n}, j, N)\\
   &=N- \left( \floor[\Big]{\frac{N}{p}} \frac{p}{2} + \max \left( (N\mod p) - \frac{p}{2}, 0 \right)\right)\\
 &= \floor[\Big]{\frac{N}{p}}p + (N\mod p) - 
 \left( 
 \floor [\Big] {\frac{N}{p}} \frac{p}{2} + \max \left( (N\mod p) - \frac{p}{2}, 0 \right)
 \right)	\\
 &= \floor[\Big]{\frac{N}{p}} \frac{p}{2} +(N\mod p) - 
 \max	\left(		(N \mod p) - \frac{p}{2}, 	0 	\right)	\\
 &= \floor[\Big]{\frac{N}{p}} \frac{p}{2} + \min \left( (N\mod p)-(N \mod p) + \frac{p}{2}, N\mod p\right)\\
 &=  \floor[\Big]{\frac{N}{p}} \frac{p}{2} + \min \left(\frac{p}{2}, N\mod p\right).
\end{align*}
 
 As proved in case~\ref{enum:shiftpower-two} of Lemma~\ref{lem:shiftpower}, the sum of $\floor{\frac{N}{p}}p$ contiguous bits is equal to $\floor{\frac{N}{p}}\frac{p}{2}$  irrespective of the starting index.
 Therefore, it suffices to find the minimum sum of $R=(N\mod 2^i)$ contiguous bits. Let $L_k$ be a list of $R$ bits occurring contiguously in $c_{i,n}$ starting at index $k$. If the first bit of $L_k$ is $1$, then $\sum L_{k+1}\leq \sum L_k$. Therefore, we can keep on increasing the value of $k$ until the first bit is zero,  without increasing the value of the contiguous sum. On the other hand, if $c_{i,n}(k-1)=0$, then $\sum L_{k-1}\leq \sum L_k$. Therefore, we can keep on decreasing the value of $k$ one at a time until $c_{i,n}(k-1)=1$, without increasing the value of the contiguous sum. Thus the minimum value of the contiguous sum is achieved when the index $k$ points to the start of any block $0_{\times 2^{i-1}}$ contained in $c_{i,n}$. The value of minimum is  $\max(R-\frac{p}{2},0)$ as the ones start appearing after $\frac{p}{2}$ indices from the start of a list $0_{\times 2^{i-1}}\cdot 1_{\times 2^{i-1}}$.  Adding it to $\floor{\frac{N}{p}}\frac{p}{2}$ gives the required minimum value.
\end{proof}

\subsection{Algorithm}
We next state a theorem which gives an equivalent definition of {\em cyclic hyper degrees}.
\begin{theorem}
\label{thm:chd}
A list $w=\{w_1,\dots,w_n\}\in \mathbb{Z}_+^n$ is a {\em cyclic hyper degree} if and only if there exist $N\in[2^n]$ and a permutation $\pi$, such that for each $i\in [n]$, $w_{\pi(i)} \in {\rm range}(i,N)$.% (see Theorem~\ref{thm:shiftrange}).
\end{theorem}

\begin{proof}
 Forward direction is a direct consequence of the definition.
 
 Using Definitions~\ref{def:chd} and~\ref{def:csum}, we get that there exist numbers $s_1,\dots,s_n \in [2^n]$ such that for each $i\in [n]$, we have $w_{\pi(i)}=\mathcal{S}(c_{i,n},s_i,N)$. Let $\Pi^{-1}=(\sigma_{s_1}^{-1},\dots,\sigma_{s_n}^{-1})$ be the list of cyclic permutations, where for each $i\in [n]$, $\sigma_{s_i}^{-1}$ is the inverse of the cyclic permutation of order $s_i$. Consider the table $\Pi^{-1}(T_n)$, by
 Theorem~\ref{thm:rotatebits}, all its rows are distinct. In particular, the first $N$ rows are distinct and their sum is $\pi(w)$. Finally, $w\in H_n$ if and only if $\pi(w)\in H_n$.
\end{proof}

Theorem~\ref{thm:shiftrange} gives us a way to efficiently find the number of bits in a contiguous sum of $N$ bits. If we know the number of distinct bit sequences that can sum up to a given vector $w\in \mathbb{Z}_+^n$, then using Theorem~\ref{thm:shiftrange} we can generate all the possible ranges of values which can be taken by each coordinate of the sum. Finally, we need to check if each coordinate of $w$ is contained in different ranges, this corresponds to finding the permutation $\pi$ in Theorem~\ref{thm:chd}. In the next lemma, we will find the number of possible  distinct bit-sequences which can sum up to a given $w$ using cyclic shifts, this corresponds to finding $N$ in Theorem~\ref{thm:chd}.
\begin{lemma}
\label{lem:set-size}
If $w=\{w_1,\dots,w_n\}\in \mathbb{Z}_+^n$ is a {\em cyclic hyper degree}, then the number of bit sequences which sum up to $w$ is an element of the set
$$\mathcal{N}_w\triangleq \{2w_i+j~:~i\in[n], j\in \{-1,0,1\}\}.$$
\end{lemma}
\begin{proof}
 As one of the coordinates of $w$, say $w_k$, is the contiguous sum of $c_{1,n}$, we need to find the number of bits which sum up to $w_k$. From the structure of $c_{1,n}$, it is easily seen that there are just three values viz. $2w_k-1,2w_k,2w_k+1$ which contain $w_k$ in their range of sums. Conversely, for any number $x$ not contained in $\{2w_i+j~:~i\in[n], j\in \{-1,0,1\}\}$, the sum of $x$ contiguous bits $c_{1,n}$ will not contain any of $w_i$, for $i\in [n]$.
\end{proof}

\begin{lemma}
\label{lem:embed}
 Given $w\in\mathbb{Z}_+^n$ and a list of integer intervals $R_1,\dots,R_n \subset \mathbb{Z}_+^2$. There exists an algorithm running in time polynomial in $n$ which correctly answers if there exists a permutation $\pi$ such that  for each $i\in [n]$, $w_{\pi(i)}\in R_i$. 
\end{lemma}
\begin{proof}
 Construct a bipartite graph $G=(A,B,E)$ on $2n$ vertices. Let $A=B=[n]$ and $(i,j)\in E$ if and only if $w_i \in R_j$. Using a polynomial time algorithm one can find if there exists a perfect matching in $G$. If there is a perfect matching then the answer is \YES, otherwise it is \NO.
\end{proof}

\begin{theorem}
 \label{thm:chd-poly}
 There is a polynomial time algorithm in $n$ which decides if a given $w\in\mathbb{Z}_+^n$ is a {\em cyclic hyper degree}.
\end{theorem}

\begin{proof}
 For each $N\in\mathcal{N}_w$, given by Lemma~\ref{lem:set-size}, and $i\in[n]$ compute ${\rm range}(i,N)$ as given by Theorem~\ref{thm:shiftrange}. Now, use Lemma~\ref{lem:embed} on these ranges of numbers and decide if $w$ is a {\em cyclic hyper degree}, if it is not then try the next number from the set $\mathcal{N}_w$. If it succeeds for at least one element of $\mathcal{N}_w$, we answer \YES, otherwise we answer \NO. Finally, note that $\vert N_w \vert\leq 3n$ and all the other steps can be performed in time which is a polynomial function of $n$. 
\end{proof}

\begin{figure*}[t!]
\includegraphics[width=\linewidth]{figs/KNNraw-revision.pdf}
\caption[]{End-to-End DR and k-NN runtime (top three) and returned lower dimension (bottom) over the largest UCR datasets for three different indexing routines. DROP consistently returns lower dimensional representations than conventional alternatives (FFT, PAA), and is on average faster than PAA and FFT.}
\label{fig:knnAll}
\end{figure*}

\section{Experimental Evaluation}
\label{sec:experiments}

We evaluate DROP's runtime, accuracy, and extensibility. We demonstrate that (1) DROP outperforms PAA and FFT in end-to-end workloads, (2) DROP's optimizations each contribute to performance,  and (3) DROP extends beyond time series.

\begin{comment}
\begin{enumerate}[itemsep=0.5em]
\item{DROP outperforms PAA and FFT in end-to-end, repetitive-query workloads (\S\ref{subsec:runtime}).}	
\item{DROP's optimizations for sampling, downstream task and work reuse contribute to performance (\S\ref{subsec:lesion}).}
\item{DROP's DR runtime scales with intrinsic dimensionality, independently of data size (\S\ref{subsec:scale}).}
\item{DROP extends beyond our time series case study (\S\ref{subsec:nonts}).}	
\end{enumerate}
\end{comment}

\subsection{Experimental Setup}
\label{subsec:setup}
\minihead{Implementation} We implement DROP\footnote{\href{https://github.com/stanford-futuredata/DROP}{https://github.com/stanford-futuredata/DROP}} in Java using \red{the multi-threaded Matrix-Toolkits-Java (MTJ) library~\cite{mtj}, and netlib-java~\cite{netlib} linked against Intel MKL~\cite{mkl} for compute-intensive linear algebra operations. 
We use multi-threaded JTransforms~\cite{jtransforms} for FFT, and implement multi-threaded PAA from scratch.}
%To provide an apples-to-apples comparison with our single-core PAA implementation, we disable multithreading in MTJ---enabling multi-core will equally improve all PCA-based methods across the board, including DROP, relative to PAA.
We use \red{the} Statistical Machine Intelligence and Learning Engine (SMILE) library~\cite{smile} for k-NN {and k-means}. 

\begin{comment}
\minihead{Environment} We use a server with \red{two Intel Xeon E5-2690v4 @ 2.60Ghz CPUs, each with 14 physical and 28 virtual cores (with hyper-threading). The server contains 512GB of RAM.}
We exclude data loading and parsing time.
\end{comment}

\minihead{Datasets} 
%To showcase DROP's performance in an end-to-end setting and contributions from each optimization, we use several real world datasets.
We first consider the UCR Time Series Classification Archive~\cite{ucr}, excluding datasets with fewer than 1 million entries, and fewer datapoints than dimensionality, leaving 14 datasets. 
%We also consider a larger, labeled earthquake dataset from XXX~\cite{quake}.
As these are all relatively small time series, we consider four additional datasets to showcase DROP's scalability and generalizability: the MNIST digits dataset~\cite{mnist}, the FMA featurized music dataset~\cite{fma}, a sentiment analysis IMDb dataset~\cite{imdb}, and the fashion MNIST dataset~\cite{fashion}. 

\minihead{DROP Configuration} We use a runtime model for k-NN and k-means computed via polynomial interpolation on data of varying dimension.
While the model is an optional input parameter, any function estimation routine can estimate it given black-box access to the downstream workload.
To evaluate sensitivity to runtime model, we report on the effect of operating without it (i.e., sample until convergence).
We set $TLB$ constraints such that k-NN accuracy remains unchanged, corresponding to $B = 0.99$  for the UCR data.
We use a default sampling schedule that begins with and increases by $1\%$ of the input.
It is possible to optimize (and perhaps overfit) this schedule in future work (\S\ref{subsec:disc}), but we provide a conservative, general schedule as a proof of concept.
%%%%We further discuss sampling schedules and properties that make a dataset amenable to DROP in \S\ref{subsec:disc}. 


\minihead{Baselines} We report runtime, accuracy, and dimensionality compared to FFT, PAA, PCA via SVD-Halko, and PCA via SVD. 
Each computes a transformation over all the data, then performs binary search to identify the lowest dimensionality that satisfies the target $TLB$. 
%%%%We further discuss choice of PCA subroutine in \S\ref{subsec:pcaexp}. 

\minihead{Similarity Search/k-NN Setup} 
We primarily consider k-NN in our evaluation as in~\cite{keogh-study}, but also briefly validate k-means performance.
%\red{Further, adopting k-NN (and, k-means in \S\ref{subsec:nonts}), which is not a classically supported relational operator, as our target task demonstrates that simple runtime estimation routines can be extended to time-series-specific  operators.}
To evaluate DR performance when used with downstream indexes, we vary k-NN's multidimensional index structure: cover trees~\cite{ctree}, K-D trees~\cite{kdtree}, or no index. 

End-to-end performance depends on the number of queries in the workload, and DROP is optimized for the repeated-query use case. 
Due to the small size of the UCR datasets, we choose a 1:50 ratio of data indexed to number of query points, and vary this index-query ratio in later microbenchmarks and experiments. 
We also provide a cost model for assessing the break-even point that balances the cost of a given DR technique against its indexing benefits.

%%%IT WAS HERE BEFORE


\subsection{DROP Performance}
\label{subsec:runtime}


We first evaluate DROP's performance compared to PAA and FFT using the time series case study extended from~\cite{keogh-study}. 

\minihead{k-NN Performance} We summarize DROP's results on a 1-Nearest Neighbor classification in Figure~\ref{fig:knnAll}.
We display the end-to-end runtime of DROP, PAA, and FFT for each of the considered index structures: no index, K-D trees, cover trees. 
We display the size of the returned dimension for the no indexing scenario, as the other two scenarios return near \red{identical values.
This occurs as many of the datasets used in this experiment are small and possess low intrinsic dimensionality that DROP quickly identifies
}
We do not display k-NN accuracy as all techniques meet the $TLB$ constraint, and achieve the same accuracy within $1\%$.

On average, DROP returns transformations that are $2.3\times$ and  $1.4\times$ smaller than PAA and FFT, translating to significantly smaller k-NN query time. 
End-to-end runtime with DROP is on average \red{$2.2\times$ and $1.4\times$ (up to $10\times$ and $3.9\times$)} faster than PAA and FFT, respectively, when using brute force linear search,  \red{$2.3\times$ and $1.2\times$ (up to $16\times$ and $3.6\times$)}  faster when using K-D trees, and \red{$1.9\times$ and $1.2\times$ (up to $5.8\times$ and $2.6\times$)} faster when using cover trees.
When evaluating Figure~\ref{fig:knnAll}, it becomes clear that DROP's runtime improvement is data dependent for both smaller datasets, and for datasets that do not possess a low intrinsic dimension (such as Phoneme, elaborated on in \S\ref{subsec:lesion})
Determining if DROP is a good fit for a dataset is an exciting area for future work (\S\ref{subsec:disc}).

%We demonstrate in our lesion study in \S\ref{subsec:lesion} that DROP also outperforms our baseline PCA via SVD implementation, as well as our SVD-Halko implementation. 

%When evaluating Figure~\ref{fig:knnAll}, it becomes clear that DROP's runtime improvement is data dependent for both smaller datasets, and for datasets that do not possess a low intrinsic dimension (such as Phoneme, elaborated on in \S\ref{subsec:lesion}). 
%Thus, in the end of the evaluation section, we provide guidelines on how to determine if DROP is a good fit for a dataset. 

\minihead{Varying Index-Query Ratio} DROP is optimized for a low index-query ratio, as in many streaming and/or high-volume data use cases.
If there are many more data points queried than used for constructing an index, but not enough such that expensive, na\"ive PCA is justified, DROP will outperform alternatives. 
A natural question that arises is: at what scale is it beneficial to use DROP?
While domain experts are typically aware of the scale of their workloads, we provide a heuristic to answer this question given rough runtime and cardinality estimates of the downstream task and the alternative DR technique in consideration.

Let $x_d$ and $x_a$ be the per-query runtime of running a downstream task with the output of DROP and a given alternative method, respectively. 
Let $r_d$ and $r_a$ denote the amortized per-datapoint runtime of DROP and the alternative method, respectively. 
Let $n_i$ and $n_q$ the number of indexed and queried points. 
DROP is faster when $n_q x_d + n_i r_d < n_q x_a + n_i r_a$.

To verify, we obtained estimates of the above and empirically validate when running k-NN using cover trees (Figure~\ref{fig:query}).
We first found that in the 1:1 index-query ratio setting, DROP should be slower than PAA and FFT, as observed. 
However, as we decrease the ratio, DROP becomes faster, with a break-even point of slightly lower than 1:3. 
We show that DROP does indeed outperform PAA \red{and FFT} in the 1:5 index-query ratio case, where it is is on average \red{$1.51\times$} faster than PAA and \red{$1.03\times$} faster than FFT. 
As the ratio decreases to 1:50, DROP is up to \red{$1.9\times$} faster than alternatives.  


\begin{figure}
\includegraphics[width=\linewidth]{figs/query-rev.pdf}
\caption[]{Effect of decreasing the index-query ratio. As an index is queried more frequently, DROP's relative runtime benefit  increases.}
\label{fig:query}
\end{figure}

\red{
\minihead{Time Series Similarity Search Extensions}
Given the breadth of research in time series indexing, we evaluate how DROP, a general operator for PCA, compares to time series indexes. 
As a preliminary evaluation, we consider iSAX2+~\cite{isax}, a state-of-the-art indexing tool, in a 1:1 index-query ratio setting, using a publicly available Java implementation~\cite{isaxcode}. 
While these indexing techniques also optimize for the low index-query ratio setting, we find index construction to be a large bottleneck in these workloads. 
For iSax2+, index construction is on average $143\times$ (up to $389\times$) slower than DR via DROP, but is on average only $11.3\times$ faster than k-NN on the reduced space.  However, given high enough query workload, these specialized techniques will surpass DROP.


We also verify that DROP is able to perform well when using downstream similarity search tasks relying on alternative distance metrics, namely, Dynamic Time Warping (DTW)---a commonly used distance measure in the literature~\cite{isaxorig}. 
As proof-of-concept, we implement a 1-NN task using DTW with a 1:1 index-query ratio, and find that even with this high ratio, DROP provides on average $1.2\times$ and $1.3\times$ runtime improvement over PAA and FFT, respectively.
%As DTW is known to be incredibly slow~\cite{dtwslow}, it is unsurprising that DROP provides large runtime benefits for tasks using DTW without additional pruning---in terms of absolute runtime, DROP saves 2.8 minutes on the FordA dataset compared to PAA, and 2.2 minutes on the wafer dataset compared to FFT.

%Finally, as the considered time series are fairly short, we perform the same experiment over a standard gaussian random walk synthetic dataset~\cite{coconut,ssh} consisting of 50,000 time series of dimension 10,000. 
%Each time series is generated by, for each time step, generating a random value distributed via standard normal distribution, and adding it to the running sum of all previous time steps. 
%We find that DROP takes 4150ms to complete, whereas PAA and FFT take 5523ms (1.3$\times$ faster) and 15329ms (3.7$\times$ faster), respectively. 
}


\subsection{Ablation Study}
\label{subsec:lesion}


We perform an ablation study of the runtime contributions of each of DROP's components compared to baseline SVD methods. 
We only display the results of k-NN with cover trees; the results hold for the other indexes.
We use a 1:1 index-query ratio \red{with data inflated by 5$\times$} to better highlight the effects of each contribution to the DR routine.%, and display average results over the UCR datasets in Figure~\ref{fig:lesion}, excluding Phoneme.


\begin{figure}
\includegraphics[width=\linewidth]{figs/lesion-all.pdf}
\caption[]{Ablation Study demonstrating average optimization improvement (a), and sample datasets that are amenable to (b) and operate poorly (c) with DROP}
\label{fig:lesion}
\end{figure}

Figure~\ref{fig:lesion} first demonstrates the boost from using SVD-Halko over a na\"ive implementation of PCA via SVD, which comes from not computing the full transformation a priori, incrementally binary searching as needed. 
It then shows the runtime boost obtained from running on samples until convergence, where DROP samples and terminates after the returned lower dimension from each iteration plateaus.
This represents the na\"ive sampling-until-convergence approach that DROP defaults to sans user-specified cost model.
We finally introduce cost based optimization and work reuse.
Each of these optimizations improves runtime, with the exception of work reuse, which has a negligible impact on average but disproportionately impacts certain datasets. 

\begin{comment}
\red{On average, DROP is $2.1\times$ faster (up to $41\times$) than PCA via SVD, and $1.6\times$ faster than SVD-Halko (up to $3.3\times$). DROP with cost-based optimization is faster than  sampling to convergence by $1.2\times$ on average, but this default strategy is still $1.4\times$ faster than SVD-Halko on average.}


\begin{comment}
\begin{figure}
\includegraphics[width=\linewidth]{figs/lesion-rev.pdf}
\caption[]{Average result of lesion study over the UCR datasets.}
\label{fig:lesion}
\end{figure}

\begin{figure}
\includegraphics[width=\linewidth]{figs/phoneme-rev.pdf}
\caption[]{Lesion study of the UCR phoneme, a dataset with high intrinsic dimensionality, meaning sampling to convergence is orders of magnitude slower than a batch SVD. DROP's cost function enables it to terminate in advance, returning a higher dimensional basis to minimize reduce overall compute.}
\label{fig:phoneme_lesion}
\end{figure}

\begin{figure}
\includegraphics[width=\linewidth]{figs/yoga-rev.pdf}
\caption[]{Lesion study over the UCR yoga dataset. Work reuse provides a $15\%$ runtime improvement.}
\label{fig:yoga-lesion}
\end{figure}
\end{comment}

Work reuse here typically slightly affects end-to-end runtime as it is useful primarily when a large number of DROP iterations are required.
We also observe this behavior on certain small datasets with moderate intrinsic dimensionality, such as the yoga dataset in Figure~\ref{fig:lesion}b. 
Work reuse provides a $15\%$ improvement over cost based optimization.

DROP's sampling operates on the premise that the dataset has data-point-level redundancy. 
However, datasets without this structure are more difficult to reduce the dimensionality of.
Phoneme is an example of one such dataset (Figure~\ref{fig:lesion}c).  
In this setting, DROP \red{incrementally examines a large proportion of data before enabling cost-based optimization,} resulting in a performance penalty.
%We discuss extensions to DROP to mitigate this in the extended version of this manuscript. 

%\S\ref{subsec:disc}.%, and provide all lesion studies in the Appendix.



\begin{comment}
\subsection{Scalability}
\label{subsec:scale}
Data generated by automated processes such as time series often grows much faster in size than intrinsic dimensionality.
DROP can exploit this intrinsic dimensionality to compute PCA faster than traditional methods as it only processes an \emph{entire} dataset if a low intrinsic dimensionality does not exist. 

To demonstrate this, we fix intrinsic dimensionality of a synthetic dataset generated via random projections to 8 as we grow the number of datapoints from 5K to 135K. 
Hence, the sample size an algorithm requires to uncover this dataset's intrinsic dimensionality is constant regardless of the full dataset size. 
In this experiment, we enable DROP's fixed-size sampling schedule set to increase by 500 datapoints at each iteration. 
As Figure~\ref{fig:increasingdata} shows, DROP is able to find a 8-dimensional basis that preserves $TLB$ to $0.99$ within \red{145ms} for dataset sizes up to 135K data points, and is \red{$12\times$} faster than binary search with SVD-Halko. 
Runtime is near constant as dataset size increases, with small overhead due to sampling from larger datasets.
This near-constant runtime contrasts with PCA via SVD and SVD-Halko as they do not exploit the intrinsic dimensionality of the dataset and process all provided points, further illustrating the scalability and utility of sample-based DR.


\begin{figure}
\includegraphics[width=\linewidth]{figs/increasing-revision.pdf}
\caption[]{Effect of dataset size on time and output dimension ($k$), with constant intrinsic data dimensionality of 8. DROP runtime with a fixed schedule remains near constant.}
\label{fig:increasingdata}
\end{figure}

\end{comment}


\begin{figure}
\includegraphics[width=\linewidth]{figs/nonts-revision.pdf}
\caption[]{End-to-End k-NN runtime (top) and returned dimension $k$ (bottom) over four non-time-series datasets spanning text, image, and music }
\label{fig:beyond}
\end{figure}

\red{
\subsection{Beyond Time Series}
\label{subsec:nonts}

We consider generalizability beyond our initial case study along two axes: data domain and downstream workload. %These preliminary results show promise in extension to additional domains and target tasks.  

\subsubsection*{Data Domain}
We examine classification/similarity search workloads across image classification, music analysis, and natural language processing. 
}
We repeat the k-NN retrieval experiments with a 1:1 index-query ratio.
We use the MNIST hand-written digit image dataset of 70,000 images of dimension 784 (obtained by flattening each $28 \times 28$-dimensional image into a single vector~\cite{mnist}, combining both the training and testing datasets); FMA's featurized music dataset, providing 518 features across 106,574 music tracks; a bag-of-words representation of an IMDb sentiment analysis dataset across 25,000 movies with 5000 features~\cite{imdb}; \red{Fashion MNIST's 70,000 images of dimension 784~\cite{fashion}}.  
We present our results in Figure~\ref{fig:beyond}.
As these datasets are larger than those in~\cite{ucr}, DROP's ability to find a $TLB$-preserving low dimensional basis is more valuable as this more directly translates to significant reduction in end-to-end runtime---up to \red{a 7.6 minute wall-clock improvement in MNIST, 42 second improvement in Fashion MNIST, 1.2 minute improvement in music features, and 8 minute improvement in IMDb compared to PAA. 
These runtime effects will only be amplified as the index-query ratio decreases, to be more typical of the repeated-query setting. 
For instance, when we decrease the ratio to 1:5 on the music features dataset, DROP provides a 6.1 and 4.5 minute improvement compared to PAA and FFT, respectively. 
}

\red{
\subsubsection*{Downstream Workload}
To demonstrate the generalizability of  DROP's pipeline as well as black-box runtime cost-model estimation routines, we extend our pipeline to perform a k-means task over the MNIST digits dataset. 
We fit a downstream workload runtime model as we did with k-NN, and operate under a 1:1 index-query ratio. 
DROP terminates in 1488ms, which is 16.5$\times$ and 6.5$\times$ faster than PAA and FFT. 
}



\begin{comment}
\subsection{PCA Subroutine Evaluation}
\label{subsec:pcaexp}

PCA algorithms  are optimized for different purposes, with varying convergence, runtime, and communication complexity guarantees. 
DROP is agnostic to choice of PCA subroutine, and improvements to said routine provide complementary runtime benefits. 
\red{To implement DROP's default algorithm, we use MTJ and netlib-java linked against Intel MKL. 
Our SVD subroutine is competitive with the commonly used SciPy library~\cite{scipy} in Python linked against Intel MKL. 
We provide a plot of the runtimes over the UCR datasets (original, and number of datapoints inflated by $5\times$).

We also provide implementations of PCA via SMILE, Probabilistic PCA via SMILE's implementation, and PCA via (stochastic) Oja's method (not linked against Intel MKL) as a proof-of-concept of DROP's modularity, but they perform orders of magnitude slower than the optimized default.}



\begin{figure}
\includegraphics[width=\linewidth]{figs/runtime-revision.pdf}
\caption[]{Comparison of DROP's java PCA implementation with Python (SciPy) over the UCR datasets. }
\label{fig:pca_comp}
\end{figure}
\end{comment}






     

 

%\red{
\section{Conclusion and Future Work}
\label{subsec:disc}

%\subsection{Generalization}
%The techniques introduced via DROP can benefit any repeated-query setting where PCA is the method of choice, so long as users are willing to sacrifice small amounts of accuracy for improved running time and a metric of interest can be defined for the application (i.e., $TLB$ for similarity search, or loss function estimates for more general tasks). 
%For instance, examining a recent natural language processing application of PCA as a word vector post-processing step prior to downstream workloads~\cite{allbut} is exciting future work. 
%While having an exact runtime model is not common a priori, many common analytics workloads for clustering, classification, or regression, black-box techniques (as we used for k-NN and k-means) can be applied as downstream tasks can be performed as a series of matrix decompositions and multiplies (i.e., techniques that make use of gradient descent). 
%}
DROP provides a first step in bridging the gap between quality and efficiency in DR for downstream \red{analytics}.
However, there are several avenues to explore for future work, such as sophisticated sampling methods and streaming execution:

%We demonstrated that workload-aware approximate PCA can provide large end-to-end speedups in \red{our time series case study} and for non-time series datasets with low intrinsic dimensionality.

DROP's efficiency is determined by the dataset's spectrum; MALLAT, with the sharpest drop-off, performs extremely well, and Phoneme, with a near uniform distribution, does not.
Datasets such as Phoneme perform poorly under the default configuration as we enable cost-based optimization after reaching a feasible point.
Thus, DROP spends a disproportionate time sampling (Fig.~\ref{fig:lesion}c). 
Extending DROP to determine if a dataset is amenable to aggressive sampling is an exciting area of future work. 
For instance, recent theoretical results that use sampling to estimate spectrum, even when the number of samples is small in comparison to the input dimensionality~\cite{estspec}, can be run alongside DROP.

%To combat this, we provide an alternate sampling schedule that aggressively increases the sampling rate to quickly reach a $TLB$-achieving state if DROP repeatedly fails to meet the target $TLB$. 

%\vspace{.2cm}
%\includegraphics[width= .9\linewidth]{figs/spectrum-rev.pdf}

%\begin{figure}
%\includegraphics[width=\linewidth]{figs/spectrum-rev.pdf}
%\caption[]{Spectrum of UCR data highlighting MALLAT (performs well) and Phoneme (performs poorly).}
%\label{fig:spectrum}
%\end{figure}

%DROP can be extended to repeated-query scenarios that occur in a streaming context, where users wish to query incoming data against historical data.
%For instance, time series for similarity search are often generated as via systems that continuously monitor and obtain data from processes over a large span of time.
%Users wish to process this data as it arrives to identify anomalous or interesting behavior.
% (e.g., to identify repetitive seismic activity/earthquakes~\cite{quakes}).

In a streaming setting, with a stationary input distribution, users can extract fixed-length sliding windows from the source and apply DROP's transformation over these segments as they arrive. 
Should the data distribution not be stationary, DROP can be retrained in one of two ways. 
First, DROP can make use of the wide body of work in changepoint or feature drift detection~\cite{cp1} to determine when to retrain. 
Alternatively, DROP can maintain a reservoir sample of incoming data~\cite{reservoir}, tuned to the specific application, and retrain if the metric of interest no longer satisfies user-specified constraints. 
Due to DROP's default termination condition, cost-based optimization must be disabled until the metric constraint is achieved to prevent early termination.





%
\begin{comment}
\begin{figure}
\includegraphics[width=\linewidth]{figs/beyond_tss_lesion.pdf}
\caption[]{End-to-End runtime lesion study of the entire MNIST dataset and the FMA featurized music dataset. Each of DROP's contributions provides a runtime improvement.}
\label{fig:beyond_lesion}
\end{figure}
\end{comment}



\section{Conclusion}
\label{sec:conclusion}

Advanced data analytics techniques must scale to rising data volumes. 
DR techniques offer a powerful toolkit when processing these datasets, with PCA frequently outperforming popular techniques in exchange for high computational cost. 
In response, we propose DROP, a new dimensionality reduction optimizer. 
DROP combines progressive sampling, progress estimation, and online aggregation to identify high quality low dimensional bases via PCA without processing the entire dataset by balancing the runtime of downstream tasks and achieved dimensionality. 
Thus, DROP provides a first step in bridging the gap between quality and efficiency in end-to-end DR for downstream \red{analytics}. 

%We revisit canonical operators for time series dimensionality reduction and the measurement study of~\cite{keogh-study}, and show that PCA is more effective than popular alternatives in the data mining literature often by a margin of over $2\times$ on average on gold-standard time series benchmark data sets with respect to output data dimension. More surprisingly, we empirically demonstrate that a small number of samples are sufficient to accurately characterize directions of maximum variance and obtain a high-quality low-dimensional transformation.




\section*{Acknowledgements}
We thank the members of the Stanford InfoLab as well as Aaron Sidford, Mary Wootters, and Moses Charikar for valuable feedback.  
We also thank the creators of the UCR classification archive for their diverse set of time series.
This research was supported in part by affiliate members and other supporters of the Stanford DAWN project---Ant Financial, Facebook, Google, Intel, Microsoft, NEC, SAP, Teradata, and VMware---as well as Toyota Research Institute, Keysight Technologies, Northrop Grumman, Hitachi, and the NSF Graduate Research Fellowship grant DGE-1656518.


\bibliographystyle{ACM-Reference-Format}
{\footnotesize
\bibliography{drop}}
%
	\section{Proof of Proposition~\ref{prop-quasi-order}}
\begin{proof}
	We will prove the result for relation $\sqsubseteq$, the proof for $\preceq$ being similar. We need to prove that $\sqsubseteq$ is reflexive and transitive. For reflexivity, it is obvious that since $G\subseteq G$ for any goal $G$, we have $G\sqsubseteq_\theta G$ for the empty substitution $\theta$. For transitivity, suppose that for goals $G_1$, $G_2$ and $G_3$, it holds that $G_1 \sqsubseteq_{\theta_1} G_2$ and $G_2 \sqsubseteq_{\theta_2} G_3$. Then by Definition~\ref{def-generalization}, there exist sets of atoms $\Delta_1$ and $\Delta_2$ such that $G_1\theta_1 \cup \Delta_1 = G_2$ and $G_2\theta_2\cup\Delta_2 = G_3$. In other words it holds that $(G_1\theta_1\cup\Delta_1)\theta_2\cup\Delta_2 = G_3$ or equivalently, $(G_1\theta_1)\theta_2 \cup (\Delta_1\theta_2 \cup\Delta_2) = G_3$. As the composition of two substitutions is a substitution, by defining $\theta_3 = \theta_2\circ\theta_1$ and $\Delta_3 = \Delta_1\theta_2 \cup\Delta_2$, we have $G_1\theta_3 \cup \Delta_3 = G_3$, so $G_1\sqsubseteq_{\theta_3} G_3$, which concludes the proof.
\end{proof}
	

	\section{Proof of Proposition~\ref{prop-msg-lcg}}
	
		First, observe the following property that holds for both relations, essentially stating that a common generalization that is not a lcg has a direct extension obtained by the addition of one atom. 
		
		\begin{proposition}\label{prop-lcg-extensible}
			Let $G_1, \dots, G_n$ and $G$ be goals such that $G$ is a $\leqslant$-common generalization, but not a $\leqslant$-lcg, of $\{G_1, \dots, G_n\}$. Then there exists an atom $A\notin G$ such that $G\cup\{A\}$ is a $\leqslant$-common generalization of $\{G_1, \dots, G_n\}$.
		\end{proposition} 
	
	
		\begin{proof}
		
		Let us suppose the existence of some goal $G$, a $\leqslant$-common generalization that is not a $\leqslant$-lcg of $\{G_1, \dots, G_n\}$, and let us try and extend $G$ into a $\leqslant$-common generalization $G\cup\{A\}$ with $A\notin G$ an atom. As $G$ is not a lcg, there must exist another goal $G'$ being a $\leqslant$-lcg of $G_1$ and $G_2$ and obviously we have $|G'|>|G|$. As a consequence at this point there are three groups of atoms that can be identified: let us denote by $\hat{A}_1, \dots, \hat{A}_p$ the $p (\ge 0)$ atom(s) that are both in $G$ and in $G'$; by $A_1, \dots, A_m$ the $m (\ge 0)$ atom(s) that are part of $G$ but not of $G'$; and by $B_1, \dots, B_l$ the $l (\ge 1)$ atom(s) that are part of $G'$ but not of $G$. For an element $A$ of any of these sets, we denote by $A^1, \dots, A_n$ the atom in respectively $G_1, \dots, G_n$ whose anti-unification led to having $A$ as part of the generalizations.
		
		From the fact that $|G'|>|G|$ it follows that $l>m$. Now each $A_i (i \in 1..m)$ is such that $\exists  h\in 1..n : A_i^h\in \{B_i^h|i\in 1..l\}$: if not, it would be possible to add an atom generalizing $\{A_i^1, \dots, A_i^n\}$ (such as $A_i$) in $G'$ and get a larger generalization, which is impossible given that $G'$ is a lcg. Also note that for two atoms $B_i$ and $B_j (1\le i < j \le l)$, for $g, h \in 1..n : g\neq h$, if $B_i^g$ is anti-unifiable with an atom $B_j^h$ then $B_j^g$ is also anti-unifiable with $B_i^h$ (as it means that all four base atoms are a call to one and the same predicate (with relation $\sqsubseteq$) or have the exact same inner structure save for variables (with relation $\preceq$)), so that it is possible to switch the atoms $B_i^g$ and $B_j^h$, compute the anti-unification of $\{B_i^g, B_j^h)$ and $(B_j^g, B_i^h)$, and get an equally valid anti-unification. Thanks to this we can, where necessary, perform switches so as to rearrange the atoms $B_i (1\le i \le l)$ into $\{\hat{B}_i|i \in 1..l\}$ in such a way that $\{A_i|i\in 1..m\} \subset \{\hat{B}_i|i \in 1..l\}$ and for each atom $\hat{B}_i$, either $\hat{B}_i \in \{A_k|k\in 1..m\}$ or $\exists g, h \in 1..n : g\neq h \wedge \hat{B}_i^g\notin \{A_k^g|k\in 1..m\}\wedge \hat{B}_i^h\notin \{A_k^h|k\in 1..m\}$. We can now define a new generalization $\hat{G}$ defined as the union of these rearranged atoms and those that are common to $G$ and $G'$, i.e. $\hat{G} = \{\hat{A_i}|i\in 1..p\}\cup\{\hat{B}_i|i\in 1..l\}$. Since $|\hat{G}| = |G'|$ and $G\subset \hat{G}$, it suffices to add one of the atoms $A \in \hat{G}\setminus G$ to $G$ in order to obtain $G\cup\{A\}$, a $\leqslant$-common generalization of $\{G_1, \dots, G_n\}$ by construction.
	\end{proof}


		Next, we prove Proposition~\ref{prop-msg-lcg}.
		
	\begin{proof}
	We prove that any $\leqslant$-msg is a $\leqslant$-lcg by contradiction. Let us suppose that some goal $G$ is both a $\leqslant$-msg and not a $\leqslant$-lcg of the set of $\{G_1, \dots, G_n\}$. According to Proposition~\ref{prop-lcg-extensible} it must then be possible to select an atom $A \notin G$ such that $G\cup\{A\}$ is a $\leqslant$-common generalization of $\{G_1, \dots, G_n\}$. Since $A\notin G$ and any atom has a $\tau$-value of at least 1, it follows that $|\tau(G\cup\{A\})|>|\tau(G)|$. Consequently $G$ cannot be a $\leqslant$-most specific generalization of $G_1$ and $G_2$: a contradiction.
	
	As for the fact that any $\preceq$-lcg is a $\preceq$-msg, we prove this also by contradiction. Let $G$ represent a $\preceq$-lcg of the set of goals $\{G_1, \dots, G_n\}$ and let us suppose that $G$ is not a $\preceq$-msg. Then there must exist another goal that is a $\preceq$-msg of $\{G_1, \dots, G_n\}$, say $G'$, such that $|\tau(G')|>|\tau(G)|$ and, according to the first part of the proposition, $|G'|=|G|$. 
	Now, observe that for a set of atoms $\{A_1, \dots, A_n\}$ to be anti-unified with $\preceq$ into an atom $A$, necessarily all $A_i (1\le i\le n)$ must have the same $\terms$-value. Indeed relation $\preceq$ is defined upon renamings so that only variables (having a $\terms$-value of zero) are impacted by the generalization process. Therefore, the only possibility for the inequality $|\tau(G')|>|\tau(G)|$ to be true is that some atoms $B_1, \dots, B_n$ of respective goals $G_1,\dots,G_n$ appear in a generalized form (say $B$) in $G'$, while these atoms have not been generalized in $G$. This means that it is possible to add a (possibly renamed) version of $B$ in $G$ and obtain $G\cup\{B\}$, also a $\preceq$-common generalization and larger than $G$: a contradiction.
	\end{proof}

	
	\section{Detailed proof of Lemma~\ref{lemma-au-op}}
	\begin{proof}
The lemma will be shown correct by the definition of three anti-unification operators. A first anti-unification operator, based on $\sqsubseteq$ is the following. 

\begin{definition}%[Simple anti-unification operator]
	\label{def-atoms-au}
	Given a variabilization function $\Phi$, let $\au^\Phi_\sqsubseteq$ (or simply $\au_\sqsubseteq$ if $\Phi$ is clear from the context) denote the anti-unification operator such that for any two atoms $A = a(t^A_1, \dots, t^A_n)$ and $B = b(t^B_1, \dots, t^B_m)$, it holds that \[\au^{\Phi}_\sqsubseteq(A,B)=\left\{\begin{array}{l}
		a\big(\Phi(t^A_1, t^B_1), \dots, \Phi(t^A_n, t^B_n)\big) \\ \qquad  \mbox{if } a = b \mbox{ and } n = m\\
		\bot \\
		\qquad \mbox{otherwise}\\
	\end{array}\right. \]
\end{definition} 

\begin{example}\label{ex-au-sq}
	In Table~\ref{table:sqsubseteq}, we show three atomic anti-unification results obtained by the application of $\au_\sqsubseteq^\Phi$ with $\Phi$ a given variabilization function. Note how in the first example, the predicates used in $A_1$ and $A_2$ differ (resp. $p/2$ and $p/3$), leading to an impossible anti-unification.
\end{example}

\begin{table*}
	\caption{Example results for $au_\sqsubseteq^\Phi$}
	\label{table:sqsubseteq}
	\centering
	\begin{tabular}{l|l|l}
		%\hline 
		$\bm{A_1}$ & $\bm{A_2}$ & $\bm{\au_\sqsubseteq^\Phi(A_1, A_2)}$\\\hline 
		$p(X, 5, q(Y,4))$ & $p(W,t(Z))$ & $\bot$\\\hline 
		$p(r(X,3), t(5))$ & $p(W, t(Z))$ & $p(\Phi(r(X,3),W), \Phi(t(5), t(Z)))$\\\hline 
		$p(r(X,3), t(Y))$ & $p(r(W,3),t(Z))$ & $p(\Phi(r(X,3),r(W,3)), \Phi(t(Y),t(Z)))$ %\\\hline 
	\end{tabular} 
\end{table*}

Note that the anti-unification operator defined in Definition~\ref{def-atoms-au} differs from the traditional subsumption operator in the ordered case (i.e. when goals are ordered sequences of atoms). The difference comes from the fact that our goals being sets, all the possible couples of atoms have to be considered, whereas traditional subsumption must handle one atom at the time, making the anti-unification operator more straigtforward.     

Let us now introduce a second anti-unification operator that will allow to compute a $\preceq$-lcg. 
%and to prove Theorem~\ref{thm-preceq-lcg}. 
Since the result of this operator should be a $\preceq$-common generalization, the operator need only to anti-unify the \textit{variables} occurring at the corresponding positions in the atoms under investigation. 
The operator must thus go deeper into the term structure of the atoms than $\au_\sqsubseteq$ does, as it needs to only anti-unify those atoms that harbor the exact same structure at the level of their non-variable terms.
\begin{definition}%[Variable anti-unification operator]
	\label{def-term-au-through-variables}
	Given some variabilization function $\Phi$, let $\au^\Phi_\preceq$ (or simply $\au_\preceq$ if $\Phi$ is clear from the context) denote the function such that for any two terms $T = t(t_1, \dots, t_n)$ and $U = u(u_1, \dots, u_m)$ it holds that
	\[\au^\Phi_\preceq(T,U)=\left\{\begin{array}{l}
		\Phi(T,U) 
		\\ \qquad \mbox{if } T\in\mathcal{V}\mbox{ and } U\in\mathcal{V}
		\\t\big(\au^\Phi_\preceq(t_1,u_1), \dots, \au^\Phi_\preceq(t_n, u_n)\big) 
		\\ \qquad \mbox{if } t = u \mbox{ and } n = m 
		\\ \qquad \mbox{and } \forall i \in 1..n: \au^\Phi_\preceq(t_i,u_i)\neq\bot
		\\ \bot
		\\ \qquad  \mbox{otherwise}
	\end{array}\right.\]
	and for any two atoms $A = a(t^A_1, \dots, t^A_n)$ and $B = b(u^B_1, \dots, u^B_m)$, it holds that
	\[\au^\Phi_\preceq(A,B)=\left\{\begin{array}{l}
		a\big(\au^\Phi_\preceq(t^A_1, u^B_1),\dots, \au^\Phi_\preceq(t^A_n, u^B_n)\big) 
		\\ \qquad \mbox{if } a = b \mbox{ and } n = m 
		\\ \qquad \mbox{and } \forall i \in 1..n: \au^\Phi_\preceq(t^A_i, u^B_i) \neq\bot
		\\ \bot  
		\\ \qquad \mbox{otherwise}
	\end{array}\right.\]
\end{definition}

\begin{example}
	In Table~\ref{table:preceq}, we treat the anti-unification of the same atoms as above, this time with the use of $\au_\preceq^\Phi$ with $\Phi$ a given variabilization function. Note how $\au_\preceq$ behaves differently than $\au_\sqsubseteq$ on the second and third couple of atoms as it requires its arguments to exhibit a similar structure in order to be anti-unifiable.
\end{example}

\begin{table*}
	\caption{Example results for $au_\preceq^\Phi$}
	\label{table:preceq}
	\centering
	\begin{tabular}{l|l|l}
		%\hline
		$\bm{A_1}$ & $\bm{A_2}$ & $\bm{\au_\preceq^\Phi(A_1, A_2)}$\\\hline 
		$p(X, 5, q(Y,4))$ & $p(W,t(Z))$ & $\bot$\\\hline 
		$p(r(X,3), t(5))$ & $p(W, t(Z))$ & $\bot$\\\hline 
		$p(r(X,3), t(Y))$ & $p(r(W,3),t(Z))$ & $p(r(\Phi(X,W),3), t(\Phi(Y,Z)))$ %\\\hline
	\end{tabular}
\end{table*}

Now, in order to compute $\sqsubseteq$-msgs, we need a more precise anti-unification operator: one that goes deeper into detail when comparing atoms so as not to miss their maximal common structure. 
\begin{definition} %[Deep anti-unification operator]\label{def-deep-operator}
	Given some variabilization function $\Phi$, let $\dau^\Phi_\sqsubseteq$ (or simply $\dau_\sqsubseteq$ if $\Phi$ is clear from the context) denote the function such that for any two terms $T = t(t_1, \dots, t_n)$ and $U = u(u_1, \dots, u_m)$ it holds that 
	\[\dau^\Phi_\sqsubseteq(T,U)=\left\{\begin{array}{l}
		
		t\big(\dau^\Phi_\sqsubseteq(t_1,u_1), \dots, \dau^\Phi_\sqsubseteq(t_n, u_n)\big) 
		\\ \qquad \mbox{if } t = u \mbox{ and } n = m 
		\\ \qquad \mbox{and } T \notin \mathcal{V} \mbox{ and } U \notin \mathcal{V}
		\\ \Phi(T,U) 
		\\ \qquad \mbox{otherwise}
	\end{array}\right.\]
	and for any two atoms $A = a(t^A_1, \dots, t^A_n)$ and $B = b(u^B_1, \dots, u^B_m)$, it holds that
	\[\dau^\Phi_\sqsubseteq(A,B)=\left\{\begin{array}{l}
		a\big(\dau^\Phi_\sqsubseteq(t^A_1, u^B_1),\dots, \dau^\Phi_\sqsubseteq(t^A_n, u^B_n)\big) 
		\\ \qquad \mbox{if } a = b \mbox{ and } n = m 
		\\ \bot
		\\ \qquad \mbox{otherwise}
	\end{array}\right.\]
\end{definition}

When applied on atoms, it is easy to see that $\dau_\sqsubseteq$ is an anti-unification operator based on relation $\sqsubseteq$.

\begin{example}
	Let us once more consider the anti-unification of the atoms introduced in Example~\ref{ex-au-sq}. This time we make use of $\dau_\sqsubseteq^\Phi$ with $\Phi$ a given variabilization function, to anti-unify the three pairs of atoms. The result is shown in Table~\ref{table:dau}. Notice how the operator preserves as much non-variable atomic structure as possible in the process.
\end{example}
\begin{table*}
	\caption{Example results for $dau_\sqsubseteq^\Phi$}
	\label{table:dau}
	\centering
	\begin{tabular}{l|l|l}
		%\hline 
		$\bm{A_1}$ & $\bm{A_2}$ & $\bm{\dau_\sqsubseteq^\Phi(A_1, A_2)}$\\\hline 
		$p(X, 5, q(Y,4))$ & $p(W,t(Z))$ & $\bot$\\\hline 
		$p(r(X,3), t(5))$ & $p(W, t(Z))$ & $p(\Phi(r(X,3),W),t(\Phi(5,Z)))$\\\hline 
		$p(r(X,3), t(Y))$ & $p(r(W,3),t(Z))$ & $p(r(\Phi(X,W),3), t(\Phi(Y,Z)))$ %\\\hline 
	\end{tabular} 
\end{table*}
The existence of these operators proves Lemma~\ref{lemma-au-op}.
	\end{proof}
	
	\section{Proof of Theorem~\ref{thm-ausqsubseteq}}
	\begin{proof}
	Obviously the $\au_\sqsubseteq(A_1,A_2)$ operation can be achieved in a time linear with respect to the arity $n$ of $A_1$. In the worst case, the operation needs to be performed for each atom in $G_1$ with respect to each atom in $G_2$. Hence the first result.
	
	
	It is also easy to see that the $\au_\preceq(A_1,A_2)$ operation can be achieved in linear time with respect to the maximum number of function applications in the argument terms of the atom $A_1$ under scrutiny. In the worst case, the operation needs to be performed for each atom in $G_1$ with respect to each atom in $G_2$. Hence the second result.
	\end{proof}
		
%	
%	\section{Maximum Weight Matching of Example~\ref{example-mwm}}
%	See Fig.~\ref{fig:mwm}.
	
	
	\section{Proof of Theorem~\ref{thm-sqsubseteq-msg}}
\begin{proof}
	First note how the atomic anti-unifications and the weights of the associated bipartite graph's edges can be computed simultaneously, by working out $\dau_\sqsubseteq(A_1,A_2)$ for each possible couple $(A_1,A_2)$ in $G_1\times G_2$ and keeping account of the number of non-variable terms encountered during the operation (or $-1$). Given that $\dau_\sqsubseteq(A_1,A_2)$ can obviously operate linearly in the number of terms appearing in $A_1$ (denoted $N$), the computation of all weights is carried out in a time not exceeding $\mathcal{O}(|G_1|.|G_2|.N)$.
	
	Now the obtained assignment problem can be solved by existing algorithms (such as the Hungarian method~\cite{assignment}) that compute a MWM in $\mathcal{O}(n^3)$, where $n$ is the number of vertexes appearing on the side of the bipartite graph that has the most vertexes. In our case, there are $|G_1|$ left vertexes and $|G_2|$ right vertexes so that a MWM algorithm can be ran in $\mathcal{O}(max(|G_1|,|G_2|)^3)$.
\end{proof}
	
	\section{Proof of Theorem~\ref{thm-dataflow-np-complete}}
\begin{proof}
	First, let us consider MSG-MIN. It clearly belongs to NP. Indeed, given an arbitrary generalization $G$, we can verify in polynomial time whether it is a most specific generalization. The procedure is as follows. We can compute at least one $\leqslant$-msg, say $G'$, in polynomial time (see Theorem~\ref{thm-sqsubseteq-msg}). It suffices then to compare the $\tau$-value of $G'$ with that of $G$ in order to decide whether $G$ is a msg. Next, verifying whether the number of variables in $G$ is bounded by a constant is obviously achieved in polynomial time as well.
	
	In order to prove NP-hardness, we will construct a reduction from the well-known set cover problem (known to be NP-complete~\cite{karp}) to MSG-MIN. The set cover problem in its decision-problem version (denoted SCP), can be formulated as follows. Given a constant $p \in \mathbb{N}_0$, a universe $U$ of values and a collection $S$ composed of $n$ sets $\{S_1, \dots, S_n\}$ that cover $U$, i.e. $U = \underset{i=1}{\overset{n}{\cup}}S_i$, the problem is to decide whether there exists $p$ subsets from $S$ that still cover $U$.
	
	We can transform an arbitrary instance of SCP into MSG-MIN as follows. Let us consider without loss of generality a universe $U$ where the elements are lowercase strings and $p \in \mathbb{N}_0$ a constant. Given a collection of sets $S=\{S_1, \dots, S_n\}$ we construct an instance of MSG-MIN as follows. In our construction we use $n+1$ different variables, namely $V$ and $(W_i)_{i\in1..n}$. We use $x_j$ to denote some element of $U$; these elements being strings, we can easily use them as predicate names. The construction of goals $G_1$ and $G_2$ proceeds then as follows:
	
	\begin{algorithmic}
		\State $G_1 = \{\}$ 
		\State $G_2 = \{\}$ 
		\For {each ($S_i \in S$)}
		\For {each ($x_j \in S_i$)}
		\State $G_1 \gets G_1\cup \{x_j(V)\}$				
		\State $G_2 \gets G_2\cup \{x_j(W_i)\}$
		\EndFor
		\EndFor
	\end{algorithmic}
	Note that all the atoms in $G_1$ have the same argument (namely the variable $V$) and there are as many atoms in $G_1$ as there are distinct elements in $S$. In $G_2$, however, there is an atom of the form $x_j(W_i)$ for each element $x_j$ occurring in $S_i$.
	
	The construction is such that any $\leqslant$-msg of $G_1$ and $G_2$ will be a version of $G_1$ where each occurrence of a variable $V$ is replaced by $\Phi(V, W_k)$ for some $W_k\in\vars(G_2)$ (where $\Phi$ is a variabilization function). Now, introducing such a variable $\Phi(V, W_k)$ in the generalization will allow to reuse the same variable for all the atoms $x_j(V)$ in $G_1$ that have a corresponding $x_j(W_k)$ in $G_2$. In other words, choosing to have variable $\Phi(V,W_k)$ in the $\leqslant$-msg is the same as selecting the subset $S_k$ to be part of the solution of the set cover problem. Consequently, using this transformation MSG-MIN can be used to decide SCP. Since the transformation can clearly be done in polynomial time, and since SCP is known to be NP-complete, we conclude that MSG-MIN is NP-complete as well.
	
	Now let us prove the result for LCG-MIN. We know that a $\leqslant$-lcg can be computed in polynomial time, so that a positive instance of LCG-MIN can be verified just like it can be for MSG-MIN. Moreover, the absence of non-variable terms in the transformation from SCP to MSG-MIN above allows us to reuse said transformation as-is to prove that LCG-MIN is NP-hard. Indeed, since the obtained anti-unification problem doesn't harbor terms other than variables, it is both an instance of MSG-MIN and LCG-MIN. LCG-MIN is therefore also NP-complete.
\end{proof}
	
	\section{Proof of Theorem~\ref{thm-inj-np-complete}}
\begin{proof}
	INJ is in NP: given a relation $\leqslant^\iota$, goals $G_1$ and $G_2$ and a substitution (or renaming) $\theta$, it is possible to verify in polynomial time whether the application of $\theta$ on $G_1$ results on a subset of $G_2$ or not.
	As for the proof of NP-hardness, we refer to~\cite{gen} in which the problem ``is $G_1$ a $\preceq^\iota$-lcg of $G_1$ and $G_2$?'' has been proved to be NP-complete using a polynomial reduction from the Induced Subgraph Isomorphism Problem~\cite{SYSLO198291}. The same reduction can be used for the other cases, leading to the conclusion that INJ is NP-complete.
\end{proof}
	
\end{document}
