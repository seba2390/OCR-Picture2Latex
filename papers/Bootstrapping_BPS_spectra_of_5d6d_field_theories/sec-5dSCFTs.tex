\section{\texorpdfstring{5d theories on $\Omega$-background}{5d theories on Omega-background}}\label{sec:5dthyOmega}

In this section, we review some basic properties of 5d $\mathcal{N}=1$ QFTs that have UV completions. 


\subsection{Gauge theories and effective prepotential}

A large class of 5d SQFTs admits mass deformations that lead to non-Abelian gauge theory descriptions at low energy. Consider a 5d $\mathcal{N}=1$ gauge theory with a non-Abelian gauge group $G$. The theory consists of the vector multiplet $\Phi$ for the gauge group $G$ and charged matter hypermultiplets. The vector multiplet contains a real scalar field $\phi$ as well as the vector field $A_\mu$. On the Coulomb branch of the moduli space, the scalar field $\phi$ gets the expectation value in the Cartan subalgebra of the gauge group $G$. This breaks the gauge group to its Abelian subgroup $U(1)^r$, with $r={\rm rank}(G)$. Then the low-energy theory is described by an effective theory with the Abelian groups. The scalar expectation values $\phi^i, i=1,\cdots, r$ for the Abelian gauge groups parameterize the Coulomb branch of the moduli space. 

The effective Abelian theory is characterized by a prepotential $\mathcal{F}(\Phi)$ which is a cubic polynomial in the Abelian vector multiplets $\Phi^I$ for both the dynamical gauge symmetry and the non-dynamical flavor symmetry where the index $I$ labels both the dynamical and the background vector multiplets. The exact prepotential can be computed by 1-loop calculations. For a general gauge group $G$ and matter hypermultiplets in generic representations, the cubic prepotential in terms of the scalar components is given by \cite{Witten:1996qb, Intriligator:1997pq}
\begin{equation}\label{eq:preF}
	\mathcal{F} \!=\! \sum_a\!\Big(\frac{m_a}{2}K^a_{ij}\phi^a_i\phi^a_j + \frac{\kappa_a}{6}d^a_{ijk}\phi^a_i\phi^a_j\phi^a_k\Big) +\frac{1}{12}\bigg(\sum_{e\in{\bf R}}|e\cdot \phi|^3-\!\sum_f\!\sum_{w\in{\bf w}_f}|w\cdot\phi+m_f|^3\!\bigg),
\end{equation}
where $a$ runs over all non-Abelian subgroups $G_a\subset G$. Here, $m_a= 1/g_a^2$ is the inverse gauge couplings squared and $\kappa_a$ is the classical Chern-Simons level, which is non-zero only for $G_a=SU(N)$ with $N\ge3$, for the group $G_a$. $K^a_{ij}={\rm Tr} (T^a_iT^a_j)$ is the Killing form of $G_a$ and $d^a_{ijk}=\frac{1}{2}{\rm Tr}T^a_i\{T^a_j,T^a_k\}$ with the generator $T^a_i$ in the fundamental representation of $G_a$. ${\bf R}$ and ${\bf w}_f$ are the roots and the weights for the $f$-th hypermultiplet with mass $m_f$ of $G$, respectively. The mass parameters $m_a$ and $m_f$ can be regarded as the scalar components in the background vector multiplets for the topological symmetries and the flavor symmetries rotating hypermultiplets respectively. We note from the prepotential that the Coulomb branch is divided into distinct sub-chambers (or phases) distinguished by the signs of masses inside absolute values in \eqref{eq:preF}, and accordingly the prepotential takes different values in the different sub-chambers.

The prepotential $\mathcal{F}$ determines the gauge kinetic terms in the effective action with the gauge coupling 
\begin{equation}
	\tau_{ij}^{\rm eff} = (g^{-2}_{\rm eff})_{ij} = \partial_i \partial_j \mathcal{F} \ , 
\end{equation}
which sets the metric on the Coulomb branch and the cubic Chern-Simons terms of the form
\begin{equation}\label{eq:cubic-CS}
	S_{CS}=\frac{C_{IJK}}{24\pi^2}\int A^I\wedge F^J\wedge F^K \ , \quad C_{IJK} = \partial_I\partial_J\partial_K \mathcal{F} \ ,
\end{equation}
with the level $C_{IJK}$ quantized as $C_{IJK}\in \mathbb{Z}$ due to gauge invariance of the Abelian symmetries \cite{Witten:1996qb}.

Other topological terms in the effective action are also important in our discussion later on the blowup equations. First, the effective action contains the mixed gauge/gravitational Chern-Simons terms of the form, \cite{Bonetti:2011mw,Bonetti:2013ela,Grimm:2015zea}
\begin{equation}\label{eq:grav-CS}
	S_{\rm grav} = -\frac{1}{48}\int C^G_i A^i\wedge p_1(T) \ ,
\end{equation}
where $p_1(T)$ is the first Pontryagin class of the tangent bundle on the 5d spacetime. Here $C_i^G$ is the level for the mixed Chern-Simons term and it is quantized as $C_i^G\in \mathbb{Z}$ \cite{Chang:2019uag,Katz:2020ewz}. The mixed Chern-Simons terms are induced at low energy by integrating out the charged fermions. The induced level from the fermion 1-loop calculations is \cite{Bonetti:2013ela,Grimm:2015zea}
\begin{equation}
	C_i^G = -\partial_i\bigg(\sum_{e\in {\bf R}}|e\cdot \phi| - \sum_f\sum_{w\in{\bf w}_f}|w\cdot\phi+m_f|\bigg) \ .
\end{equation}

There also exists the mixed gauge/$SU(2)_R$ Chern-Simons terms of the form,
\begin{equation}\label{eq:R-CS}
	S_{R} = \frac{1}{2}\int C_i^R A^i\wedge c_2(R)\ ,
\end{equation}
where $c_2(R)$ is the second Chern class of the $SU(2)_R$ R-symmetry bundle. Due to the gauge invariance, the level $C^R_i$ is quantized as $C^R_i\in 2\mathbb{Z}$. Note that the gauginos in the vector multiplets are doublets, while the matter fermions are singlets under the $SU(2)_R$ R-symmetry. Thus this term receives 1-loop contributions only from the charged gauginos and therefore it is independent of the number of hypermultiplets. The mixed gauge/$SU(2)_R$ Chern-Simons level induced from the gaugino 1-loop calculation is \cite{BenettiGenolini:2019zth}
\begin{align}\label{eq:CS-R}
	C^R_i &= \frac{1}{2} \partial_i \sum_{e\in {\bf R}}|e\cdot \phi| \ .
\end{align}
We remark here that, in the {\it Dynkin basis} where the rows of the Cartan matrix $A_{ij}$ for a gauge group $G$ are given by the simple roots, this level in the low-energy effective theory is fixed to be $C^R_i=2$ for all $i$'s.

In this paper, we are interested in the partition functions of 5d $\mathcal{N}=1$ theories on $\Omega$-deformed $\mathbb{R}^4\times S^1$. This partition function is a Witten index counting BPS states in the 5d theory, which is defined as\footnote{We can also define another Witten index as $\hat{Z}(\phi,m;\epsilon_1,\epsilon_2) \equiv Z(\phi,m;\epsilon_1,\epsilon_2)|_{(-1)^F\rightarrow (-1)^{2J_R}}$ by replacing $(-1)^F$ in $Z$ by $(-1)^{2J_R}$. This index will be used later when a 5d theory is put on the blowup $\hat{\mathbb{C}}^2$.} \cite{Nekrasov:2002qd}
\begin{equation}\label{eq:Z}
	Z(\phi,m;\epsilon_1,\epsilon_2) = {\rm Tr}\left[(-1)^Fe^{-\beta\{Q,Q^\dagger\}}e^{-\epsilon_1(J_1+J_R)}e^{-\epsilon_2(J_2+J_R)}e^{-\phi\cdot \Pi}e^{-m\cdot H}\right] \ ,
\end{equation} 
where $J_1,J_2$ are the Cartan generators of the $SO(4)$ Lorentz rotation and $J_R$ is the Cartan of the $SU(2)_R$ R-symmetry, and $\Pi$ and $H$ are the gauge and the flavor charges respectively. $Q$ is a supercharge commuting with $J_1+J_R$ and $J_2+J_R$, and $Q^\dagger$ is its conjugate. $\beta$ is the radius of $S^1$ and $\epsilon_1,\epsilon_2$ are the $\Omega$-deformation parameters. We denote by $\phi$ and $m$ the chemical potentials for the gauge and the flavor symmetries, respectively. The index computes the BPS spectrum annihilated by the supercharges $Q$ and $Q^\dagger$. So the index is independent of $\beta$.

The index can be represented by a path integral of the 5d theory on the $\Omega$-background, which can be evaluated using localization \cite{Kim:2011mv,Kim:2012gu,Hwang:2014uwa}. We will call this path integral representation of the Witten index the partition function on the $\Omega$-background or just {\it BPS partition function}.

We compute the path integral on a vacuum on the Coulomb branch specified by the expectation values $\phi$, which are now complexified by combining the scalar vevs with the gauge holonomies around $S^1$ at infinity of $\mathbb{R}^4$. The chemical potential $\phi$ in the above index is identified with the complexified expectation value $\phi$ in the path integral. Similarly, we identify the chemical potential $m$ with the complexified background gauge field for a flavor symmetry. In the following discussions, however, we shall take the chemical potentials $\phi$ and $m$ to be pure real values.

In the localization, the BPS partition function receives perturbative and non-perturbative instanton contributions which factorizes as
\begin{equation}
	Z=Z_{\rm pert}\cdot Z_{\rm inst} \ .
\end{equation}
The perturbative partition function $Z_{\rm pert}$ consists of the classical action contribution and the 1-loop contribution. Actually, it depends on the boundary condition at infinity of $\mathbb{R}^4$. We need to consider boundary conditions preserving two supercharges $Q,Q^\dagger$ and being compatible with the vacuum on the Coulomb branch. We shall choose the following boundary condition at infinity: the vector multiplets associated with the positive roots of gauge group $G$ survives, and the chiral halves of hypermultiplets with positive masses, {i.e.,} $w\cdot \phi +m_f>0$ survive. With this boundary condition, the perturbative partition function can be written as
\begin{align}\label{eq:Zpert=cx1-loop}
	Z_{\rm pert} &= Z_{\rm class}\cdot Z_{\text{1-loop}} \\
	&= e^{\mathcal{E}} \cdot {\rm PE}\bigg[-\frac{1+p_1p_2}{(1-p_1)(1-p_2)}\!\sum_{e\in{\bf R}^+}\!e^{-e\cdot\phi}+\frac{(p_1p_2)^{1/2}}{(1-p_1)(1-p_2)}\sum_f\!\sum_{w\in{\bf w}_f}e^{-|w\cdot\phi+m_f|}\bigg], \nonumber
\end{align}
where $p_{1,2}\equiv e^{-\epsilon_{1,2}}$ and ${\bf R}^+$ denotes the positive roots for the gauge group, and PE means the Plethystic exponential of a letter index $f(\mu)$ with a chemical potential $\mu$ defined as
\begin{equation}
	\PE\left[f(\mu)\right] \equiv \exp\bigg(\sum_{n=1}^\infty \frac{1}{n}f(n\mu)\bigg) \ .\label{eq:PE}
\end{equation}

In \eqref{eq:Zpert=cx1-loop}, $\mathcal{E}=\mathcal{E}(\phi,m;\epsilon_1,\epsilon_2)$ in the prefactor is a combination of the classical contribution and the Casimir energy contribution coming from regularization of infinite products in the 1-loop part. In fact, $\mathcal{E}$ is {\it effective prepotential}, which includes the cubic and mixed Chern-Simons terms \eqref{eq:cubic-CS}, \eqref{eq:grav-CS}, \eqref{eq:R-CS}, and their SUSY completions, evaluated on the $\Omega$-background. We find, for the boundary condition we chose above,
\begin{align}\label{eq:E-func}
	\mathcal{E}(\phi,m;\epsilon_1,\epsilon_2)&=i\left(S_{CS}+S_{\rm grav}+S_R + \cdots \right)|_{\phi, m,\epsilon_1,\epsilon_2} \nonumber\\
	&= \frac{1}{\epsilon_1\epsilon_2}\left[\mathcal{F}+\frac{1}{48}C_i^G \phi^i (\epsilon_1^2+\epsilon_2^2)+\frac{1}{2}C^R_i \phi^i \epsilon_+^2\right] \ ,
\end{align}
where $\cdots$ denotes the SUSY completions of the Chern-Simons terms and $\epsilon_+\equiv \frac{\epsilon_1+\epsilon_2}{2}$. The first term $\mathcal{F}$ in the bracket is the cubic prepotential \eqref{eq:preF} on the Coulomb branch, and the other two terms are the contributions from the mixed gauge/gravitational CS terms and the mixed gauge/$SU(2)_R$ CS terms, respectively. 	This factor $\mathcal{E}$ can also be considered as an equivariant integral of effective Chern-Simons terms by making the replacements
\begin{equation}
	p_1(T)\rightarrow -(\epsilon_1^2+\epsilon_2^2) \ , \qquad c_2(R) \rightarrow \epsilon_+^2 \ ,
\end{equation}
with the equivariant parameters $\epsilon_{1,2}$ and $\phi,m$. A similar interpretation for the Casimir energy of superconformal indices has been proposed in \cite{Bobev:2015kza}.

The instanton contribution $Z_{\rm inst}$ is in general given by a power series expansion by
the instanton numbers $k_a$ for each non-Abelian gauge group factor. It can thus be written as
\begin{equation}
	Z_{\rm inst} = \sum_{k_a=0}^\infty \prod_aq_a^{k_a} Z_{\{k_a\}} \ ,
\end{equation}
where $q_a=e^{-m{}_a}$ is the instanton fugacity for the $a$-th gauge group, and $Z_{\{k_a\}}$ denotes the path integral over instanton moduli space with instanton numbers $\{k_a\}$. When a UV completion for the instanton moduli space is known, for example by using ADHM construction (see \cite{Nekrasov:2002qd,Nekrasov:2003rj,Nekrasov:2004vw,Shadchin:2005mx} for some early works), we can use it to compute the instanton partition function $Z_{\{k_a\}}$ by localization. However, unfortunately, such ADHM constructions for general gauge group $G$ and matter representations are yet unknown. 

Recently, there has been some progress on computation of the instanton partition functions for more general gauge groups by using Nakajima-Yoshioka's blowup equations \cite{Nakajima:2003pg}. See also \cite{Huang:2017mis,Gu:2018gmy,Kim:2019uqw} and the references therein. Still, this method is applicable only for very limited cases. In this paper, we will propose a new and simple strategy to compute the instanton partition functions for arbitrary gauge groups and matter representations based on the blowup formula. We expect that our strategy can be applied to {\it all} the 5d gauge theories that have UV completions as 5d SCFTs or 5d KK theories coming from 6d $\mathcal{N}=(1,0)$ theories on a circle with/without twists.


\subsection{Geometric engineering}

Many examples of 5d $\mathcal{N}=1$ theories have been engineered in M-theory on local Calabi-Yau threefolds \cite{Morrison:1996xf,Douglas:1996xp,Intriligator:1997pq}. In this subsection we review some basic features of M-theory compactification on a smooth non-compact 3-fold $X$ which gives rise to a 5d SCFT or a 6d SCFT on $S^1$ (possibly with twists) in a singular limit. See \cite{Jefferson:2018irk,Bhardwaj:2019fzv} for more details.

A smooth 3-fold $X$ can be locally described as a neighborhood of a collection of K\"ahler surfaces $S_i$. A K\"ahler surface $S_i$ inside $X$ is represented by either a local $\mathbb{P}^2$ or a Hirzebruch surface with blowups $\mathbb{F}_{n}^b$ where $n$ is the degree of the Hirzebruch surface and $b$ is the number of blowups. More explanation on the local $\mathbb{P}^2$ and Hirzebruch surfaces can be found in Appendix A of \cite{Bhardwaj:2019fzv}.

The volumes of complex $p$-cycles in $X$ are controlled by K\"ahler deformations. There are normalizable K\"ahler deformations parameterized by dynamical K\"ahler parameters $\phi_i$ assigned for each compact surface $S_i$. The number $r$ of independent compact surfaces is the rank of the CY 3-fold $X$. Upon M-theory compactification, the K\"ahler moduli space of the dynamical parameters $\phi_{i=1,\cdots,r}$ is identified with the Coulomb branch moduli in the low-energy 5d theory, where $r$ is the rank of the gauge group in the field theory.

There are also non-compact K\"ahler deformations parameterized by non-dynamical K\"ahler parameters $m_{j=1,\cdots,r_F}$ where $r_F=h^{1,1}(X)-r$. These non-dynamical parameters are identified with the mass parameters in the 5d theory. For a given basis $S_i,N_j \in H^{1,1}(X)$, one can then express the K\"ahler form $J$ of $X$ as a linear sum of the K\"ahler parameters
\begin{equation}
	J = \sum_{I=1}^{h^{1,1}(X)}\phi_I D_I= \sum_{i=1}^r\phi_i S_i + \sum_{j=1}^{r_F} m_j N_j \ ,
\end{equation}
where $D_{I=1,\cdots,r}=S_{i=1,\cdots,r}$ and $D_{I=r+1,\cdots, h^{1,1}(X) }=N_{j=1,\cdots,r_F}$ are the divisors for the compact and the non-compact 4-cycles inside $X$, respectively. In particular, the K\"ahler parameters $\phi_i$ for elementary surfaces $S_i$ in this geometric basis are directly mapped to the Coulomb branch parameters $\phi_i$ in the Dynkin basis of the associated gauge group $G$. We will only use the Dynkin basis for gauge groups in the following discussions.

The volumes of $p$-cycles in $X$ are measured with respect to the K\"ahler form $J$. The total volume of the 3-fold $X$ is 
\begin{equation}
	{\rm vol}(X) = \frac{1}{3!}\int_X J^3  \ .
\end{equation}
This is identified with the 5d cubic prepotential given in \eqref{eq:preF}, i.e. $\mathcal{F}={\rm vol}(X)$. Therefore the cubic Chern-Simons coefficients in the 5d theory are geometrically determined by the triple intersections of divisors in $X$ \cite{Cadavid:1995bk,Ferrara:1996hh,Witten:1996qb,Ferrara:1996wv} (See also \cite{Grimm:2011fx,Bonetti:2011mw,Bonetti:2013ela})
\begin{equation}
	C_{IJK} = D_I\cdot D_J\cdot D_K \equiv \int_X D_I\wedge D_J\wedge D_K \ .
\end{equation}

The low-energy effective action in the 5d theory is also characterized by the coefficients of the mixed Chern-Simons terms. The mixed gauge/gravitational Chern-Simons level for a divisor $S_i$ is determined by its intersection with $c_2(X)\in H^4(X,\mathbb{Z})$, the 2nd Chern-class of the 3-fold $X$, as \cite{Cadavid:1995bk,Ferrara:1996hh,Bonetti:2011mw,Bonetti:2013ela,Katz:2020ewz}
\begin{equation}
	C_i^G = c_2(X)\cdot S_i \ .\label{eq:C_i}
\end{equation}
For a local $\mathbb{P}^2$ and a Hirzebruch surface $\mathbb{F}_n^{b}$ with $b$ blowups,
% and $s$ pairs of self-glued curves,
\begin{align}
	c_2(X)\cdot \mathbb{P}^2 = -6, \qquad c_2(X)\cdot \mathbb{F}_n^{b} = -4+2b \, .
\end{align}

In addition, we {\it propose} that the level $C^R_i$ of the mixed gauge/$SU(2)_R$ Chern-Simons term is always
\begin{equation}
	C_i^R = 2 \ ,
\end{equation}
for {\it all} the basis surfaces $S_i$ represented by either a local $\mathbb{P}^2$ or a Hirzebruch surface with blowups, in a Calabi-Yau 3-fold. This is motivated by the field theory result in \eqref{eq:CS-R}; the level $C^R_i$ is always $2$ in the Dynkin basis of the gauge groups.

Therefore the cubic and the mixed Chern-Simons terms in the low-energy effective field theory are fully expressed in terms of the topological data in the CY $3$-fold $X$. The effective prepotential $\mathcal{E}$ in \eqref{eq:E-func} for a 5d field theory can then be readily computed from the associated 3-fold $X$. As we will see below, this effective prepotential $\mathcal{E}$ together with few more information about primitive 2-cycles in a CY $3$-fold allows us to compute the BPS partition function of the corresponding 5d theory by solving the blowup equations.

The BPS spectrum in the 5d SQFT involves electric particles and (dual) magnetic strings charged under the gauge groups. In the M-theory compactification on $X$, these states arise from M2-branes and M5-branes wrapping holomorphic curves and holomorphic surfaces respectively. Their masses and tensions are determined by the volumes of the corresponding $p$-cycles. The volume of a 2-cycle (or a curve) $C$ is
\begin{equation}
	{\rm vol}(C) = -J\cdot C \ , 
\end{equation}
and the volume of a 4-cycle $S_i$ is given by
\begin{equation}
	{\rm vol}(S_i) = J\cdot J\cdot S_i \equiv \partial_i\mathcal{F}= \frac{1}{2}\int_X J^2\wedge S_i \ .
\end{equation}

The K\"ahler surfaces are glued to each other by identifying a pair (or multiple pairs) of holomorphic curves at the intersections as
\begin{equation}
	C_{ij}^\alpha \sim C_{ji}^\alpha \ ,
\end{equation}
where $C_{ij}^\alpha$ is a curve in $S_i$ and $C_{ji}^\alpha$ is a curve in $S_j$ at the intersection of two adjacent surfaces $S_i\cap S_j$, and $\alpha$ labels a pair of gluing curves. In order to be consistent with the Calabi-Yau structure of $X$, a pair of gluing curves should satisfy the condition
\begin{equation}\label{eq:CY-cond}
	(C^\alpha_{ij})^2 + (C^\alpha_{ji})^2 = 2g-2 \ ,
\end{equation} 
where $g$ is the genus of curve $C_{ij}^\alpha$ and $C_{ji}^\alpha$. 

It is also possible that two curves in a single surface are glued together while satisfying the Calabi-Yau condition \eqref{eq:CY-cond}. Such gluing is often called {\it self-gluing} \cite{Jefferson:2018irk}. A surface can have multiple self-glued curves. With $s$ self-gluings, the canonical class $K_S$ of a surface $S$ changes to
\begin{equation}
	K_S' = K_S + \sum_{i=1}^s (x_i+y_i) \ ,
\end{equation}
where $(x_i,y_i)$ is $i$-th pair of self-glued curves.

The volume of a 2-cycle can in fact be written in terms of its intersection with the canonical classes $K'_S$ as
\begin{equation}
	{\rm vol}(C) = -J\cdot C = -\sum_I\phi_I D_I\cdot C = -\sum_I\phi_I K_{D_I}' \cdot C \ .
\end{equation}
Also, the genus $g$ of a curve $C$ in a surface $S$ can be determined by the modified adjunction formula \cite{Bhardwaj:2019fzv,Bhardwaj:2020gyu}:
\begin{equation}
	C\cdot (K_S +C)+\sum_{i=1}^s{\rm min}(C\cdot x_i,C\cdot y_i)=2g-2\ ,
\end{equation}
where $K_S$ is the original canonical divisor class of $S$.

We can now easily compute the triple intersection product $S_i\cdot S_j \cdot S_k$ among three surfaces in a 3-fold $X$. For three distinct surfaces, their intersection product is given by
\begin{equation}
	S_i\cdot S_j\cdot S_k = C_{ij} \cdot C_{ik} = C_{ji} \cdot C_{jk} = C_{ki} \cdot C_{kj} \ \quad \ {\rm for}\ i\neq j\neq k \ .
\end{equation}
The triple intersection product of two distinct surfaces is 
\begin{equation}
	S_i\cdot S_i \cdot S_j = K'_i\cdot C_{ij} \ \quad \ {\rm for}\ i\neq j \ .
\end{equation}
Lastly, the triple intersection of a single surface $S_i$ is given by
\begin{equation}
	S_i^3 = K_i'^2\ .\label{eq:triple intersection}
\end{equation}
It is now straightforward to obtain the full effective prepotential $\mathcal{E}$ on the $\Omega$-background for the low-energy theory from a geometric construction.


\subsection{\texorpdfstring{6d SCFTs on $S^1$ with/without twists}{6d SCFTs on S1 with/without twists}}

Compactification of 6d SCFTs on a circle with/without outer automorphism twists provides concrete UV completions of a large class of 5d field theories. These 5d theories are often called 5d Kaluza-Klein (KK) theories. The effective prepotential of such a 5d theory can be easily obtained from the 6d classical action on the tensor branch and the action of outer automorphism as well as matter content. The detailed procedure has been introduced in \cite{Bhardwaj:2019fzv}. We will now generalize this and propose full effective prepotentials for 5d KK theories on the $\Omega$-background including the contributions from mixed gauge/gravitational and gauge/$SU(2)_R$ Chern-Simons terms.

\paragraph{5d reductions without twist}
Let us first consider the compactification of a 6d SCFT without a twist on its tensor branch. We shall also consider a generic point on the Coulomb branch of the resulting 5d theory where both tensor scalar fields and gauge holonomies are turned on. The effective prepotential of the 5d theory can be written in terms of the Chern-Simons coefficients which can be exactly calculable from the classical action of the original 6d SCFT and 1-loop computations for charged fermions.

The 6d tree-level action on the tensor branch is given by
\begin{equation}
	S_{\rm tree}=\int -\frac{1}{4}\Omega^{\alpha\beta} G_\alpha \wedge * G_\beta - \Omega^{\alpha\beta}B_{\alpha} \wedge X_{4\beta} \ , \label{eq:S_tree-GS}
\end{equation}
and supersymmetric completions, where $\Omega^{\alpha\beta}$ is the negative-definite, symmetric bilinear form of $T$ tensor fields and $G_\alpha$ is the gauge-invariant field strength for the 2-form tensor field $B_\alpha$. The second term in \eqref{eq:S_tree-GS} is the Green-Schwarz term with the 4-form $X_{4\beta}$ defined as
\begin{equation}
	dG_\alpha = X_{4\alpha} \ , \qquad X_{4\alpha} = -\frac{1}{4}a_\alpha p_1(T) +\frac{1}{4}\sum_a b_{a,\alpha}{\rm Tr}F_{a}^2 +c_\alpha c_2(R) \ ,
\end{equation}
where $a_\alpha,b_{a,\alpha}, c_\alpha$ are fixed to cancel 1-loop gauge anomalies via the Green-Schwarz-Sagnotti mechanism \cite{Sagnotti:1992qw}, the summation index $a$ runs over all the gauge and the flavor groups, and $F_a$ is the field strength for the $a$-th symmetry group. The circle reduction of the classical action then gives rise to the following terms
\begin{equation}
	S_{\rm tree} = \int -\frac{\tau}{4} \Omega^{\alpha\beta} F_{\alpha}\wedge *F_\beta - \Omega^{\alpha\beta}A_{0,\alpha} \wedge X_{4\beta} + \cdots \ ,
\end{equation}
where $\tau\equiv 1/R$ is the inverse radius of the 6d circle and $F_{\alpha}$ denotes the $U(1)_\alpha$ field strength for the gauge field $A_{0,\alpha}$ obtained by reducing the tensor field $B_\alpha$ on the circle.

In order to compute the matter contribution to the effective Chern-Simons terms, we need to perform fermion 1-loop computations including Kaluza-Klein momentum states \cite{Bonetti:2011mw,Bonetti:2013ela,Grimm:2015zea}. The cubic Chern-Simons terms are captured by the 1-loop prepotential which is given by
\begin{equation}
	\mathcal{F}_{\text{1-loop}} = \frac{1}{12}\sum_{n \in \mathbb{Z}}\left(\sum_{e \in{\bf R}} |n\tau + e\cdot \phi|^3 - \sum_f \sum_{w\in {\bf w}_f}|n\tau + w\cdot \phi + m_f|^3\right) \ ,
\end{equation}
where the sum for an integer $n$ is performed over all KK charges $n$, ${\bf R}$ means collectively the roots of the 6d gauge groups, ${\bf w}_f$ is the weight and $m_f$ is the mass parameter for the $f$-th hypermultiplet in 6d. $\phi$'s are the gauge holonomies that become the scalar fields in the 5d vector multiplets. The infinite sums in the prepotential can be regularized using the zeta function regularization. In the 5d limit where $\tau\gg \phi_i,m_f$, the regularized cubic prepotential is given by \cite{Bonetti:2013ela,Grimm:2015zea}
\begin{align}
	\mathcal{F}_{\text{1-loop}} &= \frac{1}{12}\left(\sum_{e \in {\bf R}}|e\cdot \phi|^3 - \sum_f \sum_{w\in{\bf w}_f} |w\cdot \phi + m_f|^3\right) \nonumber \\
	& \ \ \ \ - \frac{\tau}{24}\left(\sum_{e \in {\bf R}}(e\cdot \phi)^2 - \sum_f \sum_{w\in{\bf w}_f} (w \cdot \phi + m_f)^2\right) \ .
\end{align}
The first line comes from the zero KK momentum modes and the second line is the contribution from the KK momentum states after the regularization. We omitted the terms independent of the dynamical parameters $\phi$. 

The mixed gauge/gravitational and gauge/$SU(2)_R$ Chern-Simons terms can be computed in a similar manner. Note here that the contributions from the positive and the negative KK momentum states cancel each other. So these Chern-Simons terms receive the contribution only from the zero modes on a circle.
Therefore, the coefficients for the mixed Chern-Simons terms are, respectively, \cite{Bonetti:2013ela,Grimm:2015zea}
\begin{align}
	C^G_i =& -\partial_i\left(\sum_{e\in {\bf R}}|e\cdot \phi| - \sum_f\sum_{w\in{\bf w}_f}|w\cdot\phi+m_f|\right) \ , \crcr
	C^R_i =&~ \frac{1}{2} \partial_i \sum_{e\in {\bf R}}|e\cdot \phi| \ .
\end{align}
Here, $e$ and $w$ run over all the zero modes of the vectors and the hypers, respectively, from the 6d theory.

The effective action of a 5d KK theory on the Coulomb branch can be obtained by collecting the above tree-level and the 1-loop Chern-Simons terms. On the $\Omega$-background, the full effective action is then given by
\begin{align}\label{eq:6dE}
	\mathcal{E}_{6d} &= i(S_{\rm tree} + S_{\text{1-loop}} + S_{\rm grav} + S_R) |_{\epsilon_{1,2},\phi,m,\tau} \nonumber \\
	&= \frac{1}{\epsilon_1\epsilon_2}\left(\mathcal{E}_{\rm tree}+\mathcal{F}_{\text{1-loop}}+\frac{1}{48} C^G_i\phi^i (\epsilon_1^2+\epsilon_2^2) + \frac{1}{2}C_i^R \phi^i\epsilon_+^2\right) \ ,  \\
	\mathcal{E}_{\rm tree} &\equiv iS_{\rm tree}|_{\epsilon_{1,2},\phi,m,\tau}\cr
	&= -\frac{\tau}{2}\Omega^{\alpha\beta}\phi_{\alpha,0}\phi_{\beta,0} - \Omega^{\alpha\beta}\phi_{\alpha,0}\left(\frac{a_\beta}{4}(\epsilon_1^2+\epsilon_2^2)+\frac{b_{a,\beta}}{2}K_{a,ij}\phi_{a,i}\phi_{a,j}+c_\beta \epsilon_+^2\right) \nonumber \ ,
\end{align}
where $K_{a,ij}$ is the Killing form for the $a$-th symmetry group $G_a$. If $G_a$ is a gauge group, then $\phi_{a,i}$ is the Coulomb branch parameter for it, otherwise, it is the mass parameter for the flavor symmetry. Consequently, we can compute the effective prepotentials on the Coulomb branch evaluated on the $\Omega$-background directly from the knowledge of the 6d SCFTs.

As discussed in \cite{Bhardwaj:2019fzv}, when we compare this effective prepotential of the 5d KK theory with geometry, in which $b_{a,\beta}=\delta_{a,\beta}$ for gauge groups $G_a$, we need to shift the Coulomb branch parameters $\phi_{\alpha,i}$ as
\begin{equation}\label{eq:shift}
	\phi_{\alpha,i} \rightarrow \phi_{\alpha,i} - d_i^{\vee} \phi_{\alpha,0} \ ,
\end{equation}
for all $1\le i \le r_\alpha$ as well as for all $\alpha$, where $d_i^{\vee}$ is the dual Coxeter label for the gauge group $G_\alpha$. After this shift, the tensor parameter $\phi_{\alpha,0}$ becomes the K\"ahler parameter for the affine node of the associated affine gauge algebra $\hat{\mathfrak{g}}_\alpha$.

We claim that the function $\mathcal{E}_{6d}$ in \eqref{eq:6dE} with the shift \eqref{eq:shift} is the full effective prepotential on the Coulomb branch in the 5d reduction of the 6d SCFT without a twist. This contains all the terms of the dynamical Coulomb branch parameters that do not vanish on the $\Omega$-background.

As an example, consider the 6d minimal $SU(3)$ gauge theory consists of a tensor multiplet (so $\alpha=1$) coupled to an $SU(3)$ vector multiplet. The 1-loop anomalies are cancelled by the Green-Schwarz-Sagnotti mechanism with the data
\begin{equation}\label{eq:GS-SU3}
	\Omega^{11} = -3 \ , \quad a_1 = -\frac{1}{3} \ , \quad b_1 = 1 \ , \quad c_1 = 1 \ .
\end{equation}
Thus, the effective prepotential on the $\Omega$-background before the shift \eqref{eq:shift} is given by
\begin{align}
\epsilon_1\epsilon_2\,\mathcal{E} 
=&~ \frac{3}{2} \,\tau\phi_0^2 +3\phi_0\Big(-\frac{1}{12}(\epsilon_1^2+\epsilon_2^2)+\frac{1}{2}K_{ij}\phi_i\phi_j+\epsilon_+^2\Big)\cr
&+\frac{1}{6}\Big(8\phi_1^3+8\phi_2^3-3\phi_1\phi_2(\phi_1+\phi_2)\Big) -\frac{\tau}{4}K_{ij}\phi_i\phi_j\cr	
&-\frac{1}{12}(\phi_1+\phi_2)(\epsilon_1^2+\epsilon_2^2) + (\phi_1+\phi_2)\epsilon_+^2 \ ,
\end{align}
where $K_{ij}$ is the Killing form of the $SU(3)$ group. To compare with its geometric construction, we first shift $\phi_0\rightarrow \phi_0+\tau/6$ and then perform the shift \eqref{eq:shift}, $\phi_{1,2}\rightarrow \phi_{1,2}-\phi_0$. The result is then
\begin{align}
	\epsilon_1\epsilon_2 \,\mathcal{E}' =& ~\frac{1}{6}\bigg(9\tau\phi_0^2+3\tau^2\phi_0 + 8(\phi_0^3+\phi_1^3+\phi_2^3)\cr
	&\quad -3\phi_0^2(\phi_1+\phi_2)-3\phi_1^2(\phi_0+\phi_2)-3\phi_2^2(\phi_0+\phi_1)-6\phi_0\phi_1\phi_2\bigg) \nonumber \\
	& -\frac{1}{12}(\phi_0+\phi_1+\phi_2)(\epsilon_1^2+\epsilon_2^2) + (\phi_0+\phi_1+\phi_2)\epsilon_+^2 \ .
\end{align}
This is in perfect agreement with the effective prepotential for the geometry for the minimal $SU(3)$ gauge theory that consists of three $\mathbb{F}_1$ surfaces glued along their $-1$ curves and one non-compact surface with K\"ahler parameter `$-\tau$' glued to the $\mathbb{F}_1$ surface for $\phi_0$ along the base $+1$ curve as constructed in \cite{DelZotto:2017pti}.

\paragraph{5d reductions with twist}
When a 6d SCFT has a discrete global symmetry $\Gamma$, we can compactify the theory on a circle with a discrete holonomy $\gamma$ for the background gauge field of the symmetry $\Gamma$. This is often called automorphism twist on $\Gamma$ around the circle. Such twists for 6d SCFTs are well described in section 3 of \cite{Bhardwaj:2019fzv}. We will employ this prescription to compute the effective prepotentials and the partition functions of 6d SCFTs compactified on a circle with twists.

There are two kinds of discrete symmetries in 6d SCFTs. The first kind is the symmetry arising from outer automorphism of gauge algebra $\mathfrak{g}$ which permutes matter representations of the gauge algebra. The second one is the symmetry from permutations of tensor fields. A general discrete symmetry is generated by a combination of these two kinds of discrete symmetries.

\begin{table}[t]
\centering
\begin{tabular}{|c|c|c|c|c|c|}
	\hline
	$\mathfrak{\hat{g}}$ & $A_{2\ell}^{(2)}$ & $A_{2\ell-1}^{(2)}$ & $D^{(2)}_{\ell+1}$ & $E_6^{(2)}$ & $D_4^{(3)}$  \\
	\hline 
	$\mathfrak{h}$ & $C_\ell$ & $C_\ell$ & $B_\ell$ & $F_4$ & $G_2$  \\ 
	\hline
\end{tabular}
\caption{Twisted affine Lie algebra $\mathfrak{\hat{g}}$ and invariant subalgebra $\mathfrak{h}$ under outer automorphism. The superscript ${}^{(p)}$ denotes order $p=2,3$ of twist.}\label{tb:g-h}
\end{table}

The twist of the first kind on a gauge algebra $\mathfrak{g}$ splits representations of  $\mathfrak{g}$ into representations of the invariant subalgebra $\mathfrak{h}$, listed in Table \ref{tb:g-h}, under the outer automorphism. (If the twist does not act on a gauge algebra, then the invariant subalgebra is the same as the original gauge algebra, i.e. $\mathfrak{h}=\mathfrak{g}$.) A 6d state in a representation of $\mathfrak{g}$ is then decomposed into several representations of the subalgebra $\mathfrak{h}$. The decomposed states will carry shifted KK momentum as $n\rightarrow n+r$ depending on their charge under the discrete symmetry, where $n\in \mathbb{Z}$ and $r$ is a fractional KK charge. See Appendix \ref{appendix:1-loop} for more details.

For example, a 5d KK state from twisting a 6d state in the adjoint representation of the original gauge algebra will carry the following charges \cite{Kac:1990gs}
\begin{align}
	A_{2\ell}^{(2)}	\ \ &: \ \ {\bf Adj} \ {\rm of} \ A_{2\ell} \ \rightarrow \ {\bf Adj}_0\oplus {\bf F}_{1/4}\oplus {\bf F}_{3/4} \oplus {\bf \Lambda}^2_{1/2} \oplus {\bf 1}_{1/2} \ \ {\rm of} \  C_{\ell} \ ,\nonumber \\
	A_{2\ell-1}^{(2)} \ \ &:  \ \ {\bf Adj} \ {\rm of} \  A_{2\ell-1} \ \rightarrow \ {\bf Adj}_0 \oplus {\bf \Lambda}^2_{1/2}  \ \ {\rm of} \  C_{\ell} \ , \nonumber \\
	D_{\ell+1}^{(2)} \ \ &:  \ \ {\bf Adj} \ {\rm of} \  D_{\ell+1} \ \rightarrow \ {\bf Adj}_0 \oplus {\bf F}_{1/2} \ \ {\rm of} \ B_\ell \ , \nonumber \\
	E_6^{(2)} \ \ &:  \ \ {\bf Adj} \ {\rm of} \  E_6 \ \rightarrow \ {\bf Adj}_0\oplus{\bf F}_{1/2}  \ \ {\rm of} \ F_4 \ , \nonumber \\
	D_{4}^{(3)} \ \ &:  \ \ {\bf Adj} \ {\rm of} \  D_4 \ \rightarrow \  {\bf Adj}_0 \oplus{\bf F}_{1/3} \oplus {\bf F}_{2/3} \ \ {\rm of} \ G_2 \ ,
\end{align}
where $\mathbf{Adj}_r$, $\mathbf{F}_r$, and $\mathbf{\Lambda}^2_r$ refer to the adjoint, fundamental, and 2nd rank antisymmetric representation of subalgebra $\mathfrak{h}$, carrying shifted KK charge $r$ for the KK state.

The second kind of discrete symmetries is generated by a permutation $S$ acting on tensor nodes as $\alpha \rightarrow S(\alpha)$. The twist with $S$ then identifies the tensor nodes $\alpha$ and $S(\alpha)$. This brings the intersection form $\Omega^{\alpha\beta}$ into another matrix $\Omega_S^{\alpha'\beta'}$ where $\alpha',\beta'$ parametrize orbits of tensor nodes permuted by the action of $S$. The intersection matrix after the twist is determined by 
\begin{equation}
	\Omega_S^{\alpha'\beta'} = \sum_{\beta\in \beta'}\Omega^{\alpha\beta} \ ,
\end{equation}
where $\alpha$ is any node in a given orbit $\alpha'$. See \cite{Bhardwaj:2019fzv} for more details.

Now consider a 5d KK theory obtained by a twisted compactification of a 6d SCFT. The 5d gauge group and matter content as well as their KK charges are now fully fixed by the action of the twists discussed above. Knowing this, we can compute the effective prepotential for any twisted KK theory. The tree-level part $\tilde{\mathcal{E}}_{\rm tree}$ after a certain twist is given by
\begin{align}
	\tilde{\mathcal{E}}_{\rm tree} &\equiv i\tilde{S}_{\rm tree}|_{\epsilon_{1,2},\phi,m,\tau} \nonumber \\
	&= -\frac{\tau}{2}K^{\alpha'\beta'}_S\phi_{\alpha',0}\phi_{\beta',0} - \Omega^{\alpha'\beta'}_S\phi_{\alpha',0}\left(\frac{a_{\beta'}}{4}(\epsilon_1^2+\epsilon_2^2)+\frac{b_{a,\beta'}}{2}\tilde{K}_{a,ij}\phi_{a,i}\phi_{a,j}+c_{\beta'} \epsilon_+^2\right) \nonumber \ .
\end{align}
Here $K_S^{\alpha'\beta'}\equiv\sum_{\alpha\in\alpha',\beta\in\beta'}\Omega^{\alpha\beta}$ and $\tilde{K}_{a,ij}$ is the Killing form for an invariant subalgebra $\mathfrak{h}_a$ and $a$ runs over the gauge and the flavor symmetry groups.

The cubic prepotential receives contributions from the KK momentum states. We compute, in the parameter regime $\tau\gg \phi,m_f$,
\begin{align}
\tilde{\mathcal{F}}_{\text{1-loop}} 
&= \frac{1}{12}\sum_{n\in \mathbb{Z}} \bigg(\sum_{e\in \oplus{\bf R}_r} \big|(n+r)\tau + e\cdot \phi\big|^3 - \sum_f \sum_{w\in \oplus{\bf w}_{r,f}} \big|(n+r)\tau +w\cdot \phi+m_f\big|^3\bigg) \nonumber \\
&= \frac{1}{12}\bigg(\sum_{e\in {\bf R}_0} | e\cdot \phi|^3 - \sum_f \sum_{w\in {\bf w}_{0,f} } |w\cdot \phi+m_f|^3\bigg) \\
&\quad -\frac{\tau}{24}\bigg(\sum_{e \in \oplus{\bf R}_r} \!\big(6r(r\!-\!1)\!+\!1\big)(e\cdot \phi)^2 - \sum_f\!\!\sum_{w \in \oplus {\bf w}_{r,f}}\!\!\big(6r(r\!-\!1)\!+\!1\big)(w\cdot \phi+m_f)^2\bigg) \ .\nonumber
\end{align}
Here in the first line, the first term comes from the 6d vector multiplets decomposed into $\oplus {\bf R}_r$ representations of invariant subalgebras $\otimes_a \mathfrak{h}_a$, and the second term corresponds to the 6d hypermultiplets decomposed into $\oplus {\bf w}_{r,f}$ representations of $\otimes_a \mathfrak{h}_a$. We used the zeta function regularization in the second and third lines. Note that the cubic terms in the Coulomb branch parameters $\phi$ receive contributions only from the zero KK momentum states because the contributions from other higher KK towers are all cancelled each other. Similarly, the mixed gauge/gravitational and the mixed gauge/$SU(2)_R$ Chern-Simons coefficients receive contributions only from zero KK momentum states. We compute
\begin{align}
	\tilde{C}^G_i = ~&-\partial_i\bigg(\sum_{e\in {\bf R_0}}|e\cdot \phi| - \sum_f\sum_{w\in{\bf w}_{0,f}}|w\cdot\phi+m_f|\bigg) \ , \nonumber \\
	\tilde{C}^R_i =~& \frac{1}{2}\, \partial_i \sum_{e\in {\bf R}_0}|e\cdot \phi| \ .
\end{align}

Then the effective prepotential on the $\Omega$-background for a 6d SCFT on a circle with general outer automorphism twists is given by
\begin{align}\label{eq:E-twist}
\mathcal{E}_{{\rm twist}} &= i(\tilde{S}_{\rm tree} + \tilde{S}_{\text{1-loop}} + \tilde{S}_{\rm grav} + \tilde{S}_R) |_{\epsilon_{1,2},\phi,m,\tau} \nonumber \\
	&= \frac{1}{\epsilon_1\epsilon_2}\left(\tilde{\mathcal{E}}_{\rm tree}+\tilde{\mathcal{F}}_{\text{1-loop}}+\frac{1}{48} \tilde{C}^G_i\phi^i (\epsilon_1^2+\epsilon_2^2) + \frac{1}{2}\tilde{C}_i^R \phi^i\epsilon_+^2\right) \ .
\end{align}
Again, we need to perform the shift in \eqref{eq:shift} for the Coulomb branch parameters when we compare this 6d result with the effective prepotential of a geometric description or with the effective prepotentials of other 5d dual gauge theories. 

As an example, consider the 6d minimal $SU(3)$ theory and its twisted compactification. This theory has discrete $\mathbb{Z}_2$ global symmetry acting on $SU(3)$ representations. We can therefore compactify this theory on a circle with twist of the $\mathbb{Z}_2$ symmetry. The invariant subalgebra under the twist is $\mathfrak{h}=\mathfrak{su}(2)$ and the KK states in the 5d theory take representations of $\mathfrak{su}(2)$. The vector multiplet of the $\mathfrak{su}(3)$ algebra in 6d reduces to the following combination of KK momentum states.
\begin{equation}
	{\bf 8} \ {\rm of} \ \mathfrak{su}(3) \ \rightarrow \ {\bf 3}_0 \oplus {\bf 2}_{1/4} \oplus {\bf 2}_{3/4} \oplus {\bf 1}_{1/2} \ {\rm of} \ \mathfrak{su}(2) \ .
\end{equation}
From this together with the data \eqref{eq:GS-SU3}, we can compute the tree-level and the loop contributions to the effective prepotential as
\begin{align}
	\tilde{\mathcal{E}}_{\rm tree} =~& \frac{3}{2}\tau \phi_0^2 +\phi_0\left(3\phi_1^2 -\frac{1}{4}(\epsilon_1^2+\epsilon_2^2) + 3\epsilon_+^2\right) \ , \nonumber \\
	\tilde{\mathcal{F}}_{\text{1-loop}} = ~&\frac{4}{3}\phi_1^3 -\frac{5}{16}\tau \phi_1^2 \ ,
\end{align}
and the mixed Chern-Simons coefficients as
\begin{equation}
	\tilde{C}^G_1 = -4 \ , \qquad \tilde{C}^R_1 = 2 \ .
\end{equation}
Plugging all these terms into the formula \eqref{eq:E-twist}, one obtains the effective prepotential for the twisted minimal $SU(3)$ SCFT as
\begin{align}\label{eq:E-SU3-Z2}
	\epsilon_1\epsilon_2\, \mathcal{E} = &~\frac{3}{2}\tau \phi_0^2 +\phi_0\left(3\phi_1^2 -\frac{1}{4}(\epsilon_1^2+\epsilon_2^2) + 3\epsilon_+^2\right)+\frac{4}{3}\phi_1^3 -\frac{5}{16}\tau \phi_1^2 \nonumber \\
	&-\frac{1}{12}\phi_1(\epsilon_1^2+\epsilon_2^2)+\phi_1\epsilon_+^2 \ .
\end{align}
As this theory is dual to the 5d pure $SU(3)$ gauge theory at the Chern-Simons level $\kappa=9$, we can compare the effective prepotential $\mathcal{E}$ in \eqref{eq:E-SU3-Z2} with that for the dual 5d theory. For the comparison, we should shift the Coulomb branch parameters as $\phi_0 \rightarrow \phi_0 -\frac{1}{16}\tau$ and consecutively as $\phi_1 \rightarrow \phi_1 - 2\phi_0+\frac{\tau}{4}$ as guided by \eqref{eq:shift}. Then the shifted effective prepotential $\mathcal{E}'$ becomes
\begin{align}
\epsilon_1\epsilon_2\, \mathcal{E}' =&~ \frac{\tau}{2}(\phi_0^2-\phi_0\phi_1+\phi_1^2)
+\frac{1}{6}\left(8\phi_0^3+8\phi_1^3 +24\phi_0^2\phi_1 - 30\phi_0\phi_1^2\right) \nonumber \\
& -\Big(\frac{1}{12}(\epsilon_1^2+\epsilon_2^2)-\epsilon_+^2 \Big) (\phi_0+\phi_1) \ ,
\end{align}
up to terms independent of $\phi_i$. The result is precisely the effective prepotential for the pure $SU(3)_9$ theory with gauge coupling $\frac{1}{g^2} = \tfrac{\tau}{2}$.