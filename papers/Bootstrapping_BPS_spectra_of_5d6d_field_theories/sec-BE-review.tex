\subsection{Blowup equation review}\label{sec:BlowupEqReview}

To obtain the partition function $Z$ defined in \eqref{eq:Z}, we first consider the partition function $\hat{Z}$ on blowup $\hat{\mathbb{C}}^2$, where the origin of the $\mathbb{C}^2$ is replaced by a compact 2-cycle $\mathbb{P}^1$. The $\hat{\mathbb{C}}^2$ can be parametrized by the projective coordinates $(z_0,z_1,z_2)\sim(\lambda^{-1}z_0,\lambda z_1,\lambda z_2)$ for $\lambda\in\mathbb{C}\setminus\{0\}$. The Lorentz generators $J_{1,2}$ act on the $\hat{\mathbb{C}}^2$ by 
\begin{align}
(z_0,z_1,z_2)\mapsto(z_0,e^{\epsilon_1}z_1,e^{\epsilon_2}z_2)\ ,
\end{align}
with parameters $\epsilon_{1,2}$ for the Cartans of the $SO(4)$ rotations. There are now two fixed points of the Lorentz rotations, the North pole and the South pole of the coordinates $(0,1,0)$ and $(0,0,1)$ respectively on the $\mathbb{P}^1$. Around these fixed points, the local coordinates are given as $(z_0z_1,z_2/z_1)$ and $(z_1/z_2, z_0z_2)$, and thus their weights under $J_{1,2}$ actions can be represented as $(\epsilon_1,\epsilon_2-\epsilon_1)$ and $(\epsilon_1-\epsilon_2,\epsilon_2)$ at the North and South poles respectively. 

By performing the localization the partition function will be given by a sum over magnetic fluxes $\vec{n}$ on the $\mathbb{P}^1$, which is an $r$-dimensional vector $\vec{n}=(n_1,n_2,\cdots,n_r)\in \mathbb{Q}^r$ (running over the coweight lattices $\Gamma^\vee$ of gauge algebras), for the maximal torus $U(1)^r$ of the gauge symmetry group $G$. In geometry, such magnetic flux sum is performed for each compact 4-cycle. Also, background magnetic fluxes $\vec{B}$ for the Abelian subgroup $U(1)^{r_F}$ of global symmetries can be turned on, but we note that they are fixed and not summed over. Quantization conditions for each set of magnetic fluxes $(\vec{n},\vec{B})$ will be discussed shortly. In this paper, a flux set $(\vec{n},\vec{B})$ represents a set of all allowed dynamical magnetic flux vectors $\vec{n}_i$ and a fixed background flux vector $\vec{B}$.

At each flux background labelled by $\vec{n}$ and $\vec{B}$, the partition function is localized at two fixed points on the $\mathbb{P}^1$ and the path integral near each fixed point reduces to that of local $\Omega$-deformed $\hat{\mathbb{C}}^2$ with shifted chemical potentials due to the magnetic fluxes. As a consequence, the partition function $\hat{Z}$ can be written as \cite{Nakajima:2003pg,Nakajima:2005fg,Gottsche:2006bm}
\begin{align}\label{eq:bleq}
&\Lambda(m_j;\epsilon_1,\epsilon_2)\hat{Z}(\phi_i, m_j,B_j; \epsilon_1,\epsilon_2)  \\
&=\sum_{\vec{n}}(-1)^{|\vec{n}|}\hat{Z}^{(N)}(\phi_i\!+\!n_i\epsilon_1,m_j\!+\!B_j\epsilon_1;\epsilon_1,\epsilon_2\!-\!\epsilon_1) \cdot \hat{Z}^{(S)}(\phi_i\!+\!n_i\epsilon_2,m_j\!+\!B_j\epsilon_2;\epsilon_1\!-\!\epsilon_2,\epsilon_2)\ , \nonumber
\end{align}
where $|\vec{n}|=\sum_i n_i$. Here $\hat{Z}^{(N)}$ and $\hat{Z}^{(S)}$ are the partition function $\hat{Z}$ with shifted chemical potentials evaluated near the North and South poles respectively. The shifts in the chemical potentials $\phi_i$ and $m_j$ reflect the fact that magnetically charged states experience angular momentum shifts under the flux background. The prefactor $\Lambda(m_j;\epsilon_1,\epsilon_2)$ does not depend on dynamical parameters $\phi_i$, but depends only on mass parameters $m_j$ as well as $\epsilon_{1,2}$.

Now we will smoothly blow down the $\mathbb{P}^1$ at the origin. This is a smooth transition bringing the blowup geometry $\hat{\mathbb{C}}^2$ back to the flat $\mathbb{C}^2$ (with $\Omega$-deformation) without the $\mathbb{P}^1$ at the origin. The claim in \cite{Nakajima:2003pg} was that for certain theories, the partition function $\hat{Z}$ after the blowdown procedure reduces to the usual BPS partition function $Z$ on $\mathbb{C}^2$. In particular, the final partition function is independent of the background fluxes $\vec{B}$ on the $\mathbb{P}^1$. The reason for this is the following. The magnetic fluxes were supported on the $\mathbb{P}^1$ at the origin, but the $\mathbb{P}^1$ has been blown down and disappeared. Then, nothing remains to support these fluxes and moreover, there's nowhere these fluxes can flow on the flat $\mathbb{C}^2$. Therefore, we do not expect any remnant of the fluxes after the transition. 

We would like to make a remark on a subtle point in the presence of magnetic fluxes about the fermion number operator and some modifications of $Z$ associated with it. Since the partition function $\hat{Z}$ was defined with magnetic fluxes on $\hat{\mathbb{C}}^2$, the angular momentum for a state with electric charge $\sf{e}$ is shifted by ${\sf e}\cdot n$ where $n$ is the magnetic flux on $\mathbb{P}^1$ at the origin. Recall that the fermion number operator $(-1)^F$ in the index in \eqref{eq:Z} can be also defined as $(-1)^{2J_1}$. In the presence of the magnetic flux $n$, this should change as $(-1)^{2J_1} \rightarrow (-1)^{2J_1+{\sf e}\cdot n}$. This is formally equivalent to the following replacement in the index\footnote{Similar replacements $(-1)^F\rightarrow (-1)^{2J_R}$ in the superconformal index and in the holomorphic block for 3d SCFTs were discussed in \cite{Dimofte:2011py,Beem:2012mb}.}
\begin{equation}\label{eq:F-JR}
	(-1)^F \ \ \rightarrow \ \ (-1)^{2J_R} \ ,
\end{equation}
with the Cartan $J_R$ of the $SU(2)_R$ charge. This indicates that the partition function $\hat{Z}$ with magnetic fluxes on $\hat{\mathbb{C}}^2$ is in fact defined with the operator $(-1)^{2J_R}$ instead of $(-1)^F$. Moreover, since blowing down the $\mathbb{P}^1$ is a smooth transition, this definition is still valid even after the transition. Thus, the partition function (or the Witten index) in \eqref{eq:bleq} before and after the transition is defined with respect to the operator $(-1)^{2J_R}$.

One also finds that the replacement \eqref{eq:F-JR} can be implemented by a simple redefinition of the angular momentum chemical potential as $\epsilon_1\rightarrow \epsilon_1+2\pi i$. Therefore, after the transition from $\hat{\mathbb{C}}^2$ to $\mathbb{C}^2$, the partition function in the equation \eqref{eq:bleq} can be written as
\begin{align}\label{eq:hat-Z-Z}
	\hat{Z}(\phi,m;\epsilon_1,\epsilon_2) &= e^{\mathcal{E}(\phi,m;\epsilon_1,\epsilon_2)} \cdot \hat{Z}_{GV}(\phi,m;\epsilon_1,\epsilon_2) \ , \nonumber \\ 
	\hat{Z}_{GV}(\phi,m;\epsilon_1,\epsilon_2) &\equiv Z_{GV}(\phi,m;\epsilon_1+2\pi i,\epsilon_2) \ .
\end{align}
Note that the prefactor $\mathcal{E}$ in the first equation is the same as the prefactor before the replacement of $\epsilon_1$ because the redefinition doesn't affect the regularization factor in the path integral computation. Here $Z_{GV}$ is the index part of the BPS partition function, which is actually the refined Gopakumar-Vafa (GV) invariant \cite{Gopakumar:1998ii,Gopakumar:1998jq}, defined as
\begin{equation}\label{eq:GV-inv}
	Z_{GV}(\phi,m;\epsilon_1,\epsilon_2) = {\rm PE}\left[\sum_{j_l,j_r,{\bf d}}(-1)^{2(j_l+j_r)} N^{\bf d}_{j_l,j_r} \frac{\chi^{SU(2)}_{j_l}(p_1/p_2)\,\chi^{SU(2)}_{j_r}(p_1p_2)}{(p_1^{1/2}-p_1^{-1/2})(p_2^{1/2}-p_2^{-1/2})}e^{-{\bf d}\cdot {\bf m}}\right] \ ,
\end{equation}
where ${\bf d}$ denotes the charge of a BPS state, ${\bf m}$ stands for the chemical potentials (or K\"ahler parameters) $\phi,m$, and $N^{\bf d}_{j_l,j_r}$ is the degeneracy of a single-particle BPS state with spin $(j_l,j_r)$ and charge ${\bf d}$, and $\chi_j^{SU(2)}$ is the $SU(2)$ character of spin $j$. Also, $j_l\equiv \tfrac{J_1-J_2}{2}$ and $j_r\equiv \tfrac{J_1+J_2}{2}$. For example, the GV-invariants for a hypermultiplet with K\"ahler parameter $\phi$ providing a BPS state with spin $(0,0)$ are given by
\begin{align}
	Z_{GV}^{\rm hyper} &= {\rm PE}\left[\frac{\sqrt{p_1p_2}}{(1-p_1)(1-p_2)}e^{-\phi}\right] = \prod_{i,j=0}^\infty \frac{1}{1-p_1^{i+1/2}p_2^{j+1/2}e^{-\phi}} \ , \quad  \nonumber \\
	\hat{Z}_{GV}^{\rm hyper} &= Z^{\rm hyper}_{GV}(\phi;\epsilon_1+2\pi i,\epsilon_2)=\prod_{i,j=0}^\infty \frac{1}{1+p_1^{i+1/2}p_2^{j+1/2}e^{-\phi}} \ .
\end{align}

The equation \eqref{eq:bleq} with the identification \eqref{eq:hat-Z-Z} is the celebrated blowup equation for instanton partition functions on the $\Omega$-background introduced in \cite{Nakajima:2003pg,Nakajima:2005fg,Gottsche:2006bm}. See also \cite{Huang:2017mis} for a geometric generalization of the blowup formula. The blowup equation with non-trivial $\Lambda$ is called a {\it unity blowup equation}. The prefactor $\Lambda$ can also be trivial, i.e. $\Lambda=0$, for certain choices of fluxes, and the blowup equation in this case is called a {\it vanishing blowup equation} \cite{Nakajima:2005fg,Huang:2017mis}.

The purpose of this paper is to further generalize the above blowup formula such that it can cover all the 5d supersymmetric theories which have consistent UV completions. In addition, we will provide a systematic way to compute the BPS partition function $Z$ for any 5d supersymmetric theories using the blowup formula. More precisely, we propose the following conjecture:
\begin{framed}
\noindent {\bf Conjecture:} The partition function $Z$ on the $\Omega$-background in \eqref{eq:Z} for any 5d $\mathcal{N}=1$ field theory can be computed by solving the blowup equations \eqref{eq:bleq} with (i) {\it  consistent magnetic fluxes} $\vec{n}$ and $\vec{B}$, up to (ii) {\it flop transitions}.
\end{framed}

Based on this conjecture, we will present in this paper how to {\it bootstrap} BPS spectra of 5d field theories by solving the blowup equations. The seeds for this bootstrapping are the effective prepotential $\mathcal{E}$ on the $\Omega$-background and a set (or multiple sets) of consistent magnetic fluxes $\vec{n}$ and $\vec{B}$. We have already introduced how to compute the effective prepotential for every 5d field theory having either a gauge theory description in 5d or a 6d field theory origin or a geometric realization in a local CY 3-fold. We will now discuss how to choose consistent magnetic fluxes $\vec{n}$ and $\vec{B}$, basically the points (i), (ii) in the conjecture.


\subsubsection{Solving blowup equations}

Let us explain how to compute the partition function $Z$ using the blowup formula. We first remark that the index part of the partition function $Z$ of any 5d/6d SQFT must take the form of the GV-invariant  $Z_{GV}$ in \eqref{eq:GV-inv}. Now consider a power series expansion of the GV-invariant with respect to the fugacities $e^{-{\bf d}\cdot {\bf m}}$. The BPS states captured by the GV-invariant satisfy the BPS mass formula $|M|={\bf d}\cdot {\bf m}$ and at a generic point on the Coulomb branch (with mass parameters for global symmetries turned on) they have {\it positive masses} ${\bf d}\cdot {\bf m}>0$. Therefore the series expansion of $Z_{GV}$ in terms of the fugacities $e^{-{\bf d}\cdot {\bf m}}$ on the Coulomb branch is well-defined.

The blowup equation \eqref{eq:bleq} can be expressed in terms of power series in the fugacities and can be solved iteratively. Practically, we first recast the blowup equation as
\begin{align}\label{eq:bleq-GV}
	&\Lambda(m_j;\epsilon_{1},\epsilon_2) \hat{Z}_{GV}(\phi_i,m_j;\epsilon_1,\epsilon_2) = \sum_{\vec{n}}(-1)^{|\vec{n}|}e^{-V(\phi_i,m_j,\vec{n},\vec{B};\epsilon_1,\epsilon_2) } \nonumber \\
	&\times \hat{Z}_{GV}^{(N)} (\phi_i\!+\!n_i\epsilon_1,m_j\!+\!B_j\epsilon_1;\epsilon_1,\epsilon_2\!-\!\epsilon_1) \cdot \hat{Z}_{GV}^{(S)}(\phi_i\!+\!n_i\epsilon_2,m_j\!+\!B_j\epsilon_2;\epsilon_1\!-\!\epsilon_2,\epsilon_2) \ ,
\end{align}
where 
\begin{align}\label{eq:GV-V}
	&V(\phi_i,m_j,\vec{n},\vec{B};\epsilon_1,\epsilon_2) \equiv \  \mathcal{E}(\phi_i,m_j;\epsilon_1,\epsilon_2) \\
	& \qquad \quad - \mathcal{E}^{(N)}(\phi_i\!+\!n_i\epsilon_1,m_j\!+\!B_j\epsilon_1;\epsilon_1,\epsilon_2-\epsilon_1) - \mathcal{E}^{(S)}(\phi_i\!+\!n_i\epsilon_2,m_j\!+\!B_j\epsilon_2;\epsilon_1-\epsilon_2,\epsilon_2)\nonumber \ .
\end{align}
We expand both sides of the blowup equation \eqref{eq:bleq-GV} and then try to find an iterative solution of $\hat{Z}_{GV}$.

Importantly, we can use the fact that the GV-invariant should take a special form in \eqref{eq:GV-inv}. Also, spins of states at each order are bounded by the maximum spin $(j_l^{\rm max},j_r^{\rm max})$ in the series expansion, and the characters $\chi_j^{SU(2)}$ with different spins are all orthogonal to each other. Plugging the ansatz of the GV-invariant  with a finite number of trial states for a given charge ${\bf d}$ into the blowup equation and expanding it, we can iteratively solve the equations to evaluate multiplicities $N_{j_l,j_r}^{\bf d}$ of BPS states.

We conjecture that every 5d field theory enjoys enough number of independent blowup equations, enabling one to compute full BPS spectrum. It appears that a generic 5d/6d SQFT admits at least one unity blowup equation which suffices to determine all BPS degeneracies as shown in \cite{Huang:2017mis}. For instance, all 5d and 6d gauge theories with only full hypermultiplets (without any unpaired half hypermultiplets) have a number of unity blowup equations. We expect that at least one of those unity equations in this case is formulated with a set of {\it consistent magnetic fluxes}, whose definition will be given shortly, and thus the solution to such unity equation will produce the correct BPS spectrum of the gauge theory.

There are some theories having only vanishing blowup equations, though. The 6d theories involving unpaired half hypermultiplets on a circle with/without twists are such theories\footnote{On the other hand, we note that  5d gauge theories with half hypermultiplets can have unity blowup equations.} \cite{Gu:2020fem}. An analysis of small $\epsilon_{1,2}$ expansion in \cite{Huang:2017mis} suggests that a single vanishing blowup equation without other information may not be sufficient to determine all BPS degeneracies. Nevertheless, we propose that those theories in fact have enough number of vanishing blowup equations so that we can compute their BPS spectra by solving all the vanishing equations together with other supplementary consistency conditions, like the positivity of BPS degeneracies $N_{j_l,j_r}^{\bf d} \ge 0$, conformity to geometric realizations and dualities, and the KK tower structure of KK theories, etc. As a concrete example for this, in section~\ref{sec:instructive}, we will show that a single vanishing blowup equation for the pure $SU(2)_\theta$ gauge theories can be solved with the aid of additional consistency conditions so that BPS spectra  can be obtained, though the partition function for the $SU(2)_\theta$ theory can be also obtained from unity blowup equations as well. 

Though it seems trivial, the condition $N_{j_l,j_r}^{\bf d} \ge 0$ turns out to be quite powerful. While solving the blowup equations, it usually happens that degeneracies of BPS states captured in higher orders in the expansion are fixed by BPS degeneracies appearing in lower orders. Accordingly, the non-negativity of degeneracies $N_{j_l,j_r}^{\bf d} \ge 0$ for all the BPS states in higher orders puts constraints on the possible lower order BPS degeneracies. When taking into account higher expansion orders, one finds more and more additional constraints on the lower order BPS degeneracies, which would strongly restrict the allowed lower order degeneracies and hence the BPS degeneracies in a given order may be completely fixed at a certain stage in the iteration procedure\footnote{For example, the BPS spectrum of the 6d $E_7$ gauge theory with a half hypermultiplet in the fundamental representation was computed in Table 24 in \cite{Gu:2020fem} by solving vanishing blowup equations. Several undetermined BPS degeneracies (denoted by symbol `?') can actually be fixed by the condition $N_{j_l,j_r}^{\bf d} \ge 0$. For instance, from the BPS spectrum given in Table 24 in \cite{Gu:2020fem}, we could fix many degeneracies as $2(0,1)$ for $\beta=(2,1,1,0,0,1,2,1,0)$, $(0,1)$ for $\beta=(2,0,1,0,0,1,2,1,0)$, $(0,0)\oplus(0,1)$ for $\beta=(1,0,1,0,0,1,2,1,0)$, $2(0,0)\oplus(0,1)$ for $\beta=(1,0,1,0,0,2,2,1,0)$, and etc. All other undetermined degeneracies, but $\beta= (0, 3, 3, 0, 1, 0, 0, 0, 0)$, are strongly constrained and actually have only few possibilities. It seems that some higher degree computations can fix all the lower order degeneracies in the table.}.

Dualities and geometric realizations can also be useful for computation. When a theory enjoys a geometric construction or dualities, we can extract from them yet another supplementary information about the BPS states. In particular, when blowup equations have more than one distinct solutions, one can use a geometric construction or dualities to pick up the right solution for a given theory, which we will see with concrete examples below. 

Consequently, we conjecture that one can compute BPS spectra of all 5d/6d field theories using the blowup equations formulated from their geometric realizations or gauge theory descriptions, or RG-flows thereof,\footnote{We are assuming that every 5d field theory admitting a UV completion has either a geometric realization or gauge theory descriptions in 5d or in 6d on a circle possibly with twists, or can be obtained by an RG-flow from a UV complete theory.} even for the cases equipped with only vanishing blowup equations. A similar conjecture for refined BPS invariants of a local CY 3-fold was given in \cite{Huang:2017mis}. We will provide evidences for our conjecture by explicitly solving the blowup equations for all rank-2 theories and some interesting higher rank theories in sections~\ref{sec:rank2 theories} and \ref{sec:higher rank theories}.


\subsubsection{Magnetic flux quantization}\label{sec:magnetic flux quantization}

The magnetic fluxes on the $\mathbb{P}^1$ in the blowup $\hat{\mathbb{C}}^2$ cannot be arbitrary. They should satisfy suitable quantization conditions. Let us explain the quantization conditions for the magnetic fluxes in three different perspectives: the geometric perspective, the 5d gauge theoretic perspective, and the 6d gauge theoretic perspective.

\paragraph{Geometries}
In geometry, the magnetic flux $n_i$ (or $B_j$) can be turned on for each (non-)compact divisor $D_I$ in a 3-fold.
The flux quantization depends on charges and spins $(j_l,j_r)$ of wrapped M2-branes on holomorphic 2-cycles. Consider an M2-brane wrapping a primitive curve $C_i$. Here the primitive curve is a Mori cone generator and every holomorphic 2-cycle can be expressed as a linear combination of primitive curves $C_i$ as $C=\sum_i p_i C_i$ with non-negative integers $p_i$. The curve $C_i$ can consistently couple to a magnetic flux ${\sf F}$ if the following condition is satisfied \cite{Huang:2017mis}
\begin{equation}\label{eq:quantization}
	{\sf F}\cdot C_i \ \text{ is integral/half-integral when } C_i^2 \ {\rm is \ even/odd } \ ,
\end{equation}
where $C_i^2$ is the self-intersection number of $C_i$. The flux ${\sf F}$ is defined in geometry as ${\sf F}\equiv\sum_{i=1}^r n_i S_i + \sum_{j=1}^{r_F} B_j N_j$,  where $S_i, N_j\in H^{1,1}(X)$ are the basis of the compact and the non-compact surfaces inside a 3-fold $X$, respectively. The above condition \eqref{eq:quantization} is equivalent to the condition that the magnetic flux on a charged M2-brane state of spin $(j_l,j_r)$ wrapping on $C_i$ satisfies  
\begin{align}
{\sf F}\cdot C_i ~~&\text{is integral/half-integral, when $2(j_l+j_r)$ is odd/even}. 	
\end{align}
This is because the spin of a curve $C_i$ is related to the self-intersection as $2(j_l+j_r) = C_i^2+1$ mod 2.  From this we claim that the flux ${\sf F}$, or equivalently $(\vec{n},\vec{B})$, must be quantized such that all the primitive curves $C_i$ in a 3-fold satisfy the condition \eqref{eq:quantization}. In general the solution to this quantization condition is not unique. The proper choices of magnetic fluxes associated to spectrum of unitary BPS states will be discussed below.

\paragraph{5d Gauge Theories}
We can easily translate the above geometric quantization condition to physical conditions in 5d gauge theories. W-bosons and hypermultiplet states in a 5d gauge theory correspond to primitive curves with $C^2=0$ and $C^2=-1$, respectively. The charges of these elementary fields in the classical Lagrangian are all known. Based on the classical information, we can first quantize the magnetic fluxes $\vec{n}$ and $\vec{B}$ coupled to the elementary fields. The geometric quantization condition implies that the total fluxes on the W-bosons of the gauge group should be integral and those on the perturbative hypermultiplets should be half-integral. Namely,
\begin{align}\label{eq:n-B}
	\vec{n}\cdot e \in \mathbb{Z} \ , \qquad \vec{n}\cdot w_f + B_f \in \mathbb{Z}+\frac{1}{2} \ ,
\end{align}
for all roots $e$, the weights $w_f$ associated with all hypermultiplets $f$, and the fluxes $B_f$ for flavor symmetries. 

We will use the first condition of \eqref{eq:n-B} to fix the quantization of fluxes $ \vec{n}$, and use the second condition to quantize the background fluxes $B_f$.\footnote{There is one exception. A half hypermultiplet does not admit its mass parameter, so there is no corresponding $ B_f$. In this case, the second condition of \eqref{eq:n-B} associated with the half-hyper further constrains the quantization of $ \vec{n}$.} Depending on the gauge algebra, there are several possibilities of magnetic fluxes satisfying \eqref{eq:n-B}. Since a state in a given representation is obtained by subtracting roots from its highest weight state, the quantization of $ \vec{n}\cdot w_f $ is the same for every $ w_f $ in a fixed representation.

To find all possible quantizations for $\vec{n}$, it is convenient to introduce the fundamental weight $\mu_i$ which is a dual basis of the coroots $ \alpha_i^\vee $ of the gauge algebra.\footnote{For a root $ \alpha $ in Euclidean space $ E $, the coroot $\alpha^{vee}$ is a map from $ E $ to $ \mathbb{R} $, defined to be $ \inner{\alpha^\vee}{x} = 2(x, \alpha)/(\alpha, \alpha) $, where $ (\cdot, \cdot) $ is an inner product in $ E $. The fundamental weight $ \mu_i $ is dual basis of $ \alpha_i^\vee $, i.e., $ \inner{\alpha_i^\vee}{\mu_j} = \delta_{ij} $. The general weight vector can be written as a linear combination of $ \mu_i $. For example, write a simple root $ \alpha_i = \sum_k a_{ij} \mu_j$. Then $ \inner{\alpha_j^\vee}{\alpha_i} = a_{ij} $ so that $ a_{ij} $ is an element of the Cartan matrix. The details can be found in many textbooks about Lie algebras, for example, \cite{humphreys_introduction_1972} or \cite{bump_lie_2013}.} Consider a lattice $ \Lambda = \oplus \mathbb{Z}\, \mu_i $ and the corresponding root lattice $ \Lambda_r $ which is a sublattice of $ \Lambda $. The lattice $ \Lambda $ contains not only root lattice but also all possible weight vectors. Thus, the number of possible quantizations of $ \vec{n} $ with $ \vec{n} \cdot e \in \mathbb{Z} $ is counted by the number of elements in $ \Lambda / \Lambda_r $. This group is isomorphic to the center of simply connected Lie group corresponding to gauge algebra \cite{humphreys_introduction_1972, bump_lie_2013}:
\begin{align}
\begin{array}{cccccccccc}
A_r & \hspace{1ex} B_\ell & \hspace{1ex} C_\ell & \hspace{1ex} D_{\ell, \text{odd}} & \hspace{1ex} D_{\ell,\text{even}} & \hspace{1ex} E_6 & \hspace{1ex} E_7 & \hspace{1ex} E_8 & \hspace{1ex} F_4 & \hspace{1ex} G_2 \\
\mathbb{Z}_{r + 1} & \hspace{1ex} \mathbb{Z}_2 & \hspace{1ex} \mathbb{Z}_2& \hspace{1ex} \mathbb{Z}_4  & \hspace{1ex} \mathbb{Z}_2 \times \mathbb{Z}_2 & \hspace{1ex} \mathbb{Z}_3 & \hspace{1ex} \mathbb{Z}_2 & \hspace{1ex} \{1\} & \hspace{1ex} \{1\} & \hspace{1ex} \{1\}
\end{array} \ .
\end{align}

More explicitly, for the gauge algebra  of type $ A_\ell $, the possible quantization for $ \vec{n} $ is given by
\begin{align}
n_i \in \mathbb{Z} + \frac{h}{\ell+1}i \quad (1 \leq i \leq \ell)\ ,
\end{align}
where $n_i\equiv \vec{n} \cdot \mu_i$ and $ h $ is a fixed integer subject to  $0 \leq h \leq \ell$. 
For the gauge algebra of type $ B_\ell $, there are two quantizations,
\begin{align}
n_i&\in \mathbb{Z} \quad ( 1 \leq i \leq \ell-1 ) \ , \qquad
n_\ell\in \mathbb{Z} + \frac{h}{2} \ ,
\end{align}
where $ h = 0, 1 $. For type $ C_\ell $,
\begin{align}
n_i \in \mathbb{Z} + \frac{h}{2} \quad \text{for $i$ odd} \ , \qquad
n_i\in \mathbb{Z}  \quad \text{for $i$ even},
\end{align}
where $ h = 0, 1 $. The set of the quantizations of types $ B_\ell $ and $ C_\ell $ has a $ \mathbb{Z}_2 $ structure. For  
type $ D_{\ell} $ with $\ell$ odd,
\begin{align}
n_i &\in \mathbb{Z} + \frac{h}{2} \qquad (i = 1, 3, \cdots, \ell-3) \ , \quad\cr
n_i&\in \mathbb{Z}  \qquad (i=2, 4, \cdots, \ell-2) \ , \quad\\
n_{\ell-1}&\in \mathbb{Z} + \frac{h}{4}, \qquad  n_\ell \in \mathbb{Z} + \frac{h+2}{4}  \ , \nonumber
\end{align}
where $ h = 0, 1, 2, 3 $. The set of the quantization for $D_{\ell,\text{odd}}$ hence has a $ \mathbb{Z}_4 $ structure. For  
type $ D_{\ell} $ with $\ell$ even, there are four quantizations:
\begin{align}
n_i  &\in \mathbb{Z} + \frac{h_1 + h_2}{2} \qquad (i = 1, 3, \cdots, \ell-3) \ ,\cr
n_i  &\in \mathbb{Z}  \qquad (i=2, 4, \cdots, \ell-2) \ , \\
n_{\ell-1}&\in \mathbb{Z}+ \frac{h_1}{2},\qquad n_\ell \in \mathbb{Z} + \frac{h_2}{2}\ , \nonumber
\end{align}
where $ (h_1, h_2) \in \mathbb{Z}_2 \times \mathbb{Z}_2 $. For type $ E_6 $,
\begin{align}
(n_1, n_2, n_3, n_4, n_5, n_6) \in \mathbb{Z}^6 + \frac{h}{3}(2, 1, 0, 2, 1, 0)\ ,
\end{align}
with $ h = 0, 1, 2 $, which has a $\mathbb{Z}_3 $ structure. For type $ E_7 $,
\begin{align}
(n_1, n_2, n_3, n_4, n_5, n_6, n_7) \in \mathbb{Z}^7 + \frac{h}{2}(0, 0, 0, 1, 0, 1, 1) \ ,
\end{align}
with $h=0,1$, which has a $ \mathbb{Z}_2 $ structure. The remaining exceptional algebras $ E_8$, $ F_4 $ and $ G_2 $ admit only {\it canonical flux quantization} defined as
\begin{equation}\label{eq:n-B-canon}
	n_i \in \mathbb{Z} \ , \quad B_f \in \mathbb{Z}+\frac{1}{2} \ .
\end{equation}
This canonical flux quantization holds for any gauge and flavor symmetry groups except for the cases where the theory contains unpaired half hypermultiplets. Most of the examples we will discuss below use the canonical flux quantization. 

The quantization of fluxes $B_a$ for topological symmetries of gauge groups is more involved because in general we do not know spins of instanton states. Let us first discuss the quantization condition on $B_a$ when fluxes for the gauge group $G_a$ satisfy the canonical flux quantization. In this case, a single instanton state carries a unit charge under $U(1)_a$ topological symmetry for the $a$-th gauge group. The charges of the instanton state under other global symmetries can be computed by summing over contributions from the zero modes of charged hypermultiplets on the instanton background. For a hypermultiplet in a representation ${\bf r}$ of a gauge group $G_a$, the zero mode contribution to the flavor charge of a 1-instanton state on the Coulomb branch is given by its Dynkin index, i.e. $-T_a({\bf r})$. Therefore, it follows from \eqref{eq:quantization} that the ground state of a single instanton for gauge group $G_a$ should induce a quantization condition
\begin{equation}\label{eq:inst-flux}
	B_a-\sum_f B_f T_a({\bf r}_f) \  \in \ \mathbb{Z} \ \ {\rm or} \ \ \mathbb{Z}+1/2 \ ,
\end{equation} 
depending on the spin of the state. This will quantize the flux $B_a$ provided that $B_f$'s have already been quantized by \eqref{eq:n-B-canon}. Here, we denote by $f$ all hypermultiplets in the representation ${\bf r}_f$ of the gauge group $G_a$ collectively. Since the canonical gauge fluxes $n_i$ bring about only integral shifts, the gauge fluxes do not affect the above quantization condition.

When $n_i^a$'s for a gauge group $G_a$ are not canonical, we can fix $B_a$ by requiring the corresponding blowup equation to be solvable.
We first expand the blowup equation in terms of the instanton fugacity $e^{-m_a}$ for a gauge group $G_a$, assuming that it is a unity equation. The terms in the leading order in this expansion are given by a set of magnetic flux vectors $\vec{n}$ that minimize
\begin{equation}
	\partial_{m_a}V = \frac{1}{2}(\vec{n},\vec{n})^a\equiv \frac{1}{2} K^a_{ij}n^a_i n^a_j \ ,
\end{equation}
for all gauge groups, where $V$ is defined in \eqref{eq:GV-V}. We call this set ${\rm Min}(\vec{n})$. Since the GV-invariant $\hat{Z}_{GV}$ starts with $1$ in the fugacity expansion and the prefactor $\Lambda$ is independent of the Coulomb branch parameters $\phi_i$, the right-hand side in the blowup equation given in \eqref{eq:bleq-GV} must start with terms independent of $\phi_i$. This means that for a given $B_a$, there must exist at least a magnetic flux vector $\vec{n}\in {\rm Min}(\vec{n})$ such that
\begin{equation}\label{eq:Bj-cond1}
	\partial_{\phi^a_i}V = 0 \ \ {\rm at} \ (\vec{n},\vec{B}) \  {\rm for} \ {\rm all} \ i \ {\rm of} \ G_a \ .
\end{equation}
This condition then fixes the background flux $B_a$ when the gauge fluxes $n^a_i$ are non-canonical ones. We note that when the fluxes $n^a_i$ are all canonical ones, this condition is trivially satisfied since $n^a_i\in$ Min($\vec{n})$ are all zero and $V$ is independent of $\phi^a_i$ when $n^a_i=0$.

On the other hand, a vanishing blowup equation can be solved if there exists at least a pair of minimal flux vectors $\vec{n}_1,\vec{n}_2\in {\rm Min}(\vec{n})$ with $|\vec{n}_1|-|\vec{n}_2| = 1$, which is a necessary condition for that two leading terms cancel each other in the expansion, satisfying
\begin{equation}\label{eq:Bj-cond2}
	\partial_{\phi^a_i}V \ {\rm at} \ (\vec{n}_1 , \vec{B}) \quad = \quad 
	\partial_{\phi^a_i} V \ {\rm at} \ (\vec{n}_2, \vec{B}) \ ,
\end{equation}
for all $\phi_i^a$.

The conditions \eqref{eq:Bj-cond1} and \eqref{eq:Bj-cond2} will then leave only a finite number of allowed fluxes $B_a$ for the topological symmetry of $G_a$ in the unity and  vanishing blowup equations, respectively, for non-canonical magnetic fluxes $n^a_i$.


\paragraph{6d Gauge Theories with/without Twists}
Lastly, we will discuss the flux quantization conditions for the 5d KK theories in terms of the associated 6d field theory data. Upon a circle compactification, the 6d multiplets reduce to 5d vector multiplets and hypermultiplets taking representations of the 6d gauge algebra $\mathfrak{g}$, or the invariant subalgebra $\mathfrak{h}$ when $\mathfrak{g}$ is twisted. In the 5d reduction of a 6d gauge theory, the magnetic fluxes for the gauge group and those for the flavor group acting on 5d gauge and matter fields satisfy the quantization condition in \eqref{eq:n-B}. Also, the magnetic flux $B_\tau$ of the KK $U(1)$ symmetry should be an integer because all KK states from a single 6d state should have the same (half-)integrality of fluxes. So it is natural to fix
\begin{equation}
	B_\tau = 0 \ .
\end{equation}

We then need to determine the quantization conditions on the tensor fluxes $n_{\alpha}$ (or $n_{\alpha'}$ when the $\alpha$-th tensor node is twisted) for tensor symmetries. Self-dual strings are charged under the tensor symmetries. Their tensor charges are determined by the classical Green-Schwarz terms evaluated on self-dual string backgrounds. The ground state of a single string associated to the $\beta'$-th (or $\beta$-th without twist) tensor field carries charges for the $\alpha'$-th tensor symmetry as
\begin{equation}
	\Omega^{\alpha'\beta'}_S b_{a,\beta'} \ ,
\end{equation}
when the string is an instantonic string of the $a$-th gauge group. If the $\beta'$-th tensor supports no gauge algebra, then $b_{a,\beta'}=1$. 

Let us first consider the cases where magnetic fluxes for a gauge group $G_a$ coupled to the $\alpha'$-th tensor fields, so when $b_{a,\alpha'}\neq0$, satisfy the canonical flux quantization condition in \eqref{eq:n-B-canon}. The flavor and the KK charges for a single string can be computed by counting the zero mode contributions from KK states on the string background, which is the same computation we did above for instanton states in 5d gauge theories. Collecting all the zero mode contributions, we claim that the tensor flux, which we denote by $n_{\alpha'}\equiv \tilde{n}_{\alpha'}+B_{\alpha'}$ with $\tilde{n}_{\alpha'}\in \mathbb{Z}$, should satisfy the following quantization condition: given the spin $(j_l,j_r)$ of the ground state, 
\begin{align}
	-\Omega^{\alpha'\beta'}_S\!B_{\alpha'}b_{a,\beta'} &\!- \!\sum_{f_0}B_{f_0}T_a({\bf r}_{f_0}) \!+ \!\frac{1}6B_\tau\! \sum_{r}(6r(r\!-\!1)\!+\!1)\Big(T_a({\bf R}_r) \!-\! \sum_f T_a({\bf w}_{r,f})\Big) \nonumber \\
	&\in	\left\{\begin{array}{ll}
		\mathbb{Z}& ~\text{for~}2(j_l+j_r)~\text{odd}\ ,\\
		\mathbb{Z}+\frac12 & ~\text{for~}2(j_l+j_r)~\text{even}\ .
	\end{array} \right.
\end{align}
This is essentially the quantization condition for $B_{\alpha'}$.
Here, $f_0$ runs over hypermultiplets with KK-charge $0$, and $r$ runs over all fractional KK-charge shifts including $r=0$. ${\bf R}_r$ and ${\bf w}_{r,f}$ denote representations of the vector multiplets  and the hypermultiplets with fractional shift $r$ of KK charges, respectively. Note that since gauge and tensor charges of the ground state on a self-dual string are quantized to be integers and the associated fluxes $n_a$ and $\tilde{n}_{\alpha'}$ are all integers as well, they have no effect on this quantization condition.

Now consider non-canonical magnetic fluxes for a gauge group $G_a$. Let us first expand the blowup equation in terms of the fugacities $e^{-\phi_{\alpha'}}$ of tensor symmetries. The leading power of these fugacities in this expansion is determined by the minimum of
\begin{equation}
	\partial_{\phi_{\alpha'}}V=\Omega_S^{\alpha'\beta'}b_{a,\beta'}\tilde{K}_{a,ij}n^a_in^a_j  \ .
\end{equation}
This is non-zero with non-integral gauge fluxes $n^a_i$. This means that when gauge fluxes $n^a_i$ do not satisfy the canonical flux quantization condition for any $G_a$, then the blowup equation written in terms of a 6d field theory description is always a vanishing-type, i.e. $\Lambda=0$. 

As we discussed for 5d gauge theories above, the vanishing blowup equation can be solved only if we can find at least a pair of minimal flux vectors $\vec{n}_1,\vec{n}_2\in {\rm Min}(\vec{n})$ and background fluxes $\vec{B}$ subject to the condition given in \eqref{eq:Bj-cond2}, where ${\rm Min}(\vec{n})$ is a set of magnetic flux vectors minimizing $(\vec{n},\vec{n})^a$ for all $G_a$ and $\vec{B}$ here involves the tensor fluxes $B_{\alpha'}$ as well as other flavor fluxes. We can use this fact to fix the background magnetic fluxes $B_{\alpha'}$ of tensor symmetries.

We remark that if a 6d gauge theory involves unpaired half hypermultiplets and thus the background flavor flux acting on the hypermultiplet cannot be turned on, then we cannot choose the canonical flux quantization \eqref{eq:n-B-canon} for the fluxes of the gauge group $G_a$ coupled to the half hypermultiplet. Therefore, we have only vanishing blowup equations in such cases, as discussed in \cite{Gu:2020fem}.


\subsubsection{Consistent magnetic fluxes}

The above quantization conditions are a necessary condition but not a sufficient one for the BPS partition functions $Z$ of a 5d theory to satisfy the blowup equation \eqref{eq:bleq} with chosen magnetic fluxes. Among the magnetic fluxes satisfying the quantization conditions, only few of them can provide consistent blowup equations whose solution correctly produces BPS spectrum of a 5d field theory. We call such magnetic fluxes leading to the consistent blowup equations {\it consistent magnetic fluxes}. With a wrong set of fluxes, one would find an inconsistent blowup equation: the blowup equation has no solution or the solution to the blowup equation does not take the form of a GV-invariant, or the solution involves unphysical states and thus differs from the desired BPS spectrum for a 5d theory. Therefore, in order to correctly compute BPS spectrum using the blowup equations, it is crucial to know how to choose proper consistent magnetic fluxes. We now present a set of criteria for the consistent magnetic fluxes.

As discussed, all the BPS particles on the Coulomb branch must satisfy the BPS mass formula. It is expressed as $|M|= \sum_i {\sf{e}}_i \phi_i$, when we turned off mass parameters for global symmetries in strong coupling limit near the UV fixed point. Here ${\sf{e}}_i$ denotes the charge of a state under the $i$-th $U(1)$ gauge group. In geometry, these masses are identified with the volumes of holomorphic 2-cycles for the BPS states as $\sum_i {\sf{e}}_i \phi_i={\rm vol}(C)$ measured with respect to the normalizable K\"ahler parameters $\phi_i$. It then follows that the mass $\sum_i {\sf{e}}_i \phi_i$ (or ${\rm vol}(C)$) must be non-negative on the Coulomb branch, i.e. $\sum_i {\sf{e}}_i \phi_i \ge 0$ for all BPS states. In fact, the  Coulomb branch $\mathcal{C}$ is the space of the dynamical K\"ahler moduli $\phi_i$ defined as \cite{Jefferson:2017ahm,Jefferson:2018irk}
\begin{equation}\label{eq:C-Coulomb}
	\mathcal{C} = \bigg\{\phi_i,i=1,\cdots,r\, | \,{\sf{e}}\cdot \phi\equiv\sum_i {\sf{e}}_i \phi_i \ge 0\bigg\} \ ,
\end{equation}
with $\phi_i>0$ when all mass parameters for global symmetries are switched off. The Coulomb branch $\mathcal{C}$ must exist for a UV finite 5d theory equipped with dynamical K\"ahler parameters. Otherwise, unitarity of the theory will be violated at some point on the Coulomb branch and the theory cannot have a consistent UV completion.

The Coulomb branch $\mathcal{C}$ is a collection of sub-chambers $\mathcal{C}_i$ that are connected by so-called {\it flop transitions}. It is possible that a BPS hypermultiplet becomes massless $M=0$ at the boundary between two sub-chambers. If then, two sub-chambers are connected by a flop transition flipping the mass sign of the hypermultiplet as $M>0\rightarrow M<0$. The effective Chern-Simons terms in the low energy theory take different expressions in the two sub-chambers because the CS terms depend on mass signs of hypermultiplets. More precisely, the prepotential and the gauge/gravitational Chern-Simons coefficients alter their forms under a flop transition related to a hypermultiplet as
\begin{equation}
	\mathcal{F} \ \ \rightarrow \ \ \mathcal{F} + \frac16 M^3 \ , \qquad C^G_i \ \ \rightarrow \ \ C^G_i -\partial_iM \ , \qquad C_i^R \ \ \rightarrow \ \ C_i^R \ ,
\end{equation}
where $M$ here is the mass for the hypermultiplet before the flop transition. In geometry, the flop transition corresponds to a geometric transition $X \rightarrow X'$ between two CY 3-folds $X$ and $X'$  described by blowing down a $-1$ curve $C\subset X$ and blowing up a different $-1$ curve $C'\subset X'$.

The partition function $Z$ computes degeneracies of BPS states on the Coulomb branch of a  5d theory. Suppose that we solve the blowup equations for $Z$ in the fugacity expansion and the BPS states are identified up to a certain order. One can easily read off the masses for the BPS states up to that order. For a consistent 5d field theory having dynamical K\"ahler parameters, there must exist a non-vanishing Coulomb branch $\mathcal{C}$ defined in \eqref{eq:C-Coulomb}, possibly after a finite number of flop transitions, where the masses of the BPS states are all non-negative, i.e. ${\sf e}\cdot\phi\ge0$, when all non-dynamical parameters are turned off. If one cannot find a non-trivial Coulomb branch, then it implies that the 5d theory is inconsistent in UV.

Note however that the partition function $Z$ is defined and computed in a particular sub-chamber on the (extended) Coulomb branch with mass parameters $m_j$ for global symmetries turned on. In the limit $m_j\rightarrow0$, some of hypermultiplets captured in the partition function can be massless and then flop transitions should take place before their masses become negative. Since this can generically happen, we should carefully examine the existence of Coulomb branch by testing ${\sf e}\cdot\phi\ge0$ for all the states in $Z$, except for hypermultiplets (or states with spin $(j_l,j_r)=(0,0)$). We note that for hypermultiplets, either ${\sf e}\cdot\phi \ge 0$ or ${\sf e}\cdot\phi< 0$ is allowed, as flop transitions can happen. If one finds a non-trivial Coulomb branch where ${\sf e}\cdot\phi\ge0$ for all the states in $Z$ except for a hypermultiplet which is of ${\sf e}\cdot\phi < 0$, then this means that there must be a flop transition associated to the hypermultiplet as we approach the UV fixed point. In this case we should first perform flop transitions for the hypermultiplets with ${\sf e}\cdot\phi < 0$ and then test again ${\sf e}\cdot\phi\ge0$ for all states including the hypermultiplets.

If one finds a chamber with ${\sf e}\cdot\phi\ge0$ for all the states including hypermultiplets, after a sequence of such flop transitions, it is a strong indication that the theory has a UV completion with non-trivial Coulomb branch and the solution $Z$ (or $\hat{Z}$) of the blowup equations computes the BPS spectrum on the Coulomb branch around the UV fixed point. In this case, we will refer to the set of magnetic fluxes as {\it consistent magnetic fluxes} for the 5d theory on the blowup $\hat{\mathbb{C}}^2$. We conjecture that there exist enough sets of consistent magnetic fluxes for every UV finite 5d theory and therefore we can compute its BPS spectrum using the blowup equations.

Examining ${\sf e}\cdot\phi \ge 0$ for all BPS states is a formidable task as there are infinitely many BPS states but the partition function $Z$ is computed up to a certain order in the fugacity expansion. Practically we can check this non-negativity of masses of BPS states only up to a certain higher order. This however would not be very harmful from the perspective of geometry. Recall that all holomorphic 2-cycles ${C}=\sum_i n_i C_i$  can be written as a linear sum of primitive curves $C_i$ with non-negative integers $n_i$ and the number of the primitive curves is finite for a local CY 3-fold. The primitive curves (or associated BPS states) are usually captured at some lower orders in the expansion. We then need to check non-negativity of volumes only for such primitive curves in lower orders. We expect this holds also for general 5d theories. Hence one can in principle identify consistent magnetic fluxes by computing the BPS partition function up to certain leading orders.

We now have all the ingredients for solving the blowup equations and thus are ready for bootstrapping BPS spectrum of any 5d field theory. In the followings we will explicitly illustrate the bootstrap procedure with a large number of interesting examples.