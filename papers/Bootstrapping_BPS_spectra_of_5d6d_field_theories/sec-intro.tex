\section{Introduction}\label{sec:intro}

Supersymmetric theories with eight supercharges in five and six dimensions are a very rich subject that has been investigated over the past decades. Lots of recent progress in this subject have been made in the classification of 5d and 6d superconformal theories (SCFTs) \cite{Heckman:2013pva, Heckman:2015bfa, Bhardwaj:2015xxa} and 6d little string theories (LSTs) \cite{Bhardwaj:2015oru}. Such classifications have been carried out based on geometric properties of F-theory compactified on local elliptic Calabi-Yau (CY) three-folds. Also, a large class of 5d SCFTs has been classified using gauge theoretic constructions \cite{Seiberg:1996bd, Intriligator:1997pq, Jefferson:2017ahm} and M-theory compactified on local CY 3-folds \cite{Douglas:1996xp, Intriligator:1997pq, DelZotto:2017pti, Xie:2017pfl, Jefferson:2018irk, Bhardwaj:2018yhy, Bhardwaj:2018vuu, Apruzzi:2018nre,Apruzzi:2019vpe, Apruzzi:2019opn, Bhardwaj:2019fzv, Apruzzi:2019kgb, Apruzzi:2019enx}. These higher dimensional theories turn out to exhibit several fascinating features of quantum field theories (QFTs) such as the existence of tensionless strings, dualities, and symmetry enhancements. Moreover, they have played a pivotal role in constructing and studying lower dimensional QFTs via compactifications.

Supersymmetric partition functions have provided unexpectedly powerful methods for exploring such features of higher dimensional field theories. Of particular interest are the partition functions on $\mathbb{R}^4\times S^1$ in 5d and $\mathbb{R}^4\times T^2$ (or the elliptic genera) in 6d, which are the Witten indices on the $\Omega$-deformed $\mathbb{R}^4$ computing the number of the BPS states weighted by their electric charges and angular momenta. Throughout the paper, we refer to them as the BPS partition functions, though it may be an abuse of notation. Protected by the supersymmetry, the BPS spectra encoded in the partition function can be used to confirm non-trivial dualities and symmetry enhancements at various points on the moduli space \cite{Bao:2011rc,Hayashi:2013qwa,Taki:2014pba,Mitev:2014jza,Hayashi:2014wfa,Gaiotto:2015una,Hayashi:2015xla,Hayashi:2016abm,Yun:2016yzw,Hayashi:2017jze,Kim:2018gjo,Chen:2020jla}. Also, the Seiberg-Witten prepotential in the Coulomb phase can be reproduced from the BPS partition function by taking the limit $\epsilon_1,\epsilon_2\rightarrow 0$ of two $\Omega$-deformation parameters \cite{Nekrasov:2002qd,Nekrasov:2003rj}. In particular, other supersymmetric observables such as the superconformal index \cite{Bhattacharya:2008zy,Kim:2012gu,Kim:2013nva,Iqbal:2012xm} and the $S^5$ partition function \cite{Kallen:2012va,Kim:2012ava,Kallen:2012zn,Lockhart:2012vp,Imamura:2013xna,Kim:2012qf} can be factorized into a product of several BPS partition functions under localization, which signifies importance of the partition function on the $\Omega$-background as a building block for other observables in 5d and 6d.

Various computational tools for the 5d and 6d BPS partition functions have been developed. First, the partition function on the $\Omega$-background for 5d $\mathcal{N}=1$ gauge theories is closely related to Nekrasov instanton partition function counting the BPS bound states of instantons with other charged particles on the Coulomb branch of the moduli space. The Nekrasov instanton partition functions for gauge theories with the classical gauge groups are computed through localization based on the ADHM constructions of the instanton moduli space \cite{Atiyah:1978ri,Nekrasov:2002qd,Nekrasov:2003rj}. (See also \cite{Marino:2004cn,Nekrasov:2004vw,Fucito:2004gi,Hwang:2014uwa} for various generalizations). The ADHM constructions have also been used to compute the elliptic genera of the self-dual strings in 6d SCFTs (in the tensor branch) \cite{Haghighat:2013tka,Kim:2014dza, Haghighat:2014vxa,Gadde:2015tra}. Even though the ADHM method is systematic and applicable for many 5d and 6d gauge theories with classical gauge groups, ADHM construction for the exceptional gauge groups is however still missing.%
\footnote{The ADHM-like construction for the moduli space of instantons in the $G_2$ gauge theory with fundamental matters was proposed in \cite{Kim:2016foj,Kim:2018gjo}.}
Moreover, the ADHM method is not applicable when a gauge theory of classical group is coupled to a large number of matter fields in generic representations. Hence, there are still challenges and difficulties in the ADHM method when computing the BPS partition functions for other more generic gauge theories in 5d and in 6d.

Topological vertex method \cite{Aganagic:2003db,Iqbal:2007ii,Awata:2008ed}\footnote{See also the Ding-Iohara-Miki (DIM) algebra \cite{Ding:1996mq, doi:10.1063/1.2823979} and its relation to topological vertex \cite{Aganagic:2012hs, Awata_2012,Bourgine:2017jsi}.} provides yet another systematic way of computing the BPS partition functions when the 5d/6d theories are realized by Type IIB 5-brane webs that are toric or toric-like \cite{Aharony:1997bh, Benini:2009gi}, by which we mean those 5-brane webs which can be constructed from toric 5-brane webs through Higgsing procedures \cite{Hayashi:2013qwa}. Topological vertex has been further developed so that it is also applicable for 5-brane webs with $O5$-planes \cite{Kim:2017jqn, Hayashi:2020hhb} or $ON$-planes \cite{Bourgine:2017rik, Kim-Wei2020}. Though fairly many 5-brane webs are known or discovered for 5d theories and for 6d theories on a circle \cite{Kim:2015jba, Hayashi:2015fsa, Bergman:2015dpa, Zafrir:2015rga, Hayashi:2015zka, Hayashi:2015vhy, Zafrir:2015ftn, Zafrir:2016jpu, Hayashi:2018lyv, Hayashi:2019yxj, Kim:2019dqn}, there still remain challenges when the number of hypermultiplets is large or the Chern-Simons (CS) level is high. For instance, 5-brane webs for the 5d $SU(3)_8$ and $SU(4)_8$ gauge theories are unknown, and 5-brane web realizations for gauge theories of exceptional groups are still far from clear except for $G_2$ gauge theories \cite{Hayashi:2018bkd,Kim:2019dqn}. 

In \cite{Nakajima:2003pg}, Nakajima-Yoshioka formulated the so-called blowup equations to compute the Nekrasov instanton partition functions for four-dimensional $\mathcal{N} = 2\ SU(N)$ gauge theories. The K-theoretic blowup equations were established soon in \cite{Nakajima:2005fg, Gottsche:2006bm} for five-dimensional $\mathcal{N} = 1\ SU(N)$ gauge theories. This blowup equation approach has been further generalized to compute the instanton partition functions for other simple gauge groups including exceptional gauge groups in 5d in \cite{Keller:2012da,Kim:2019uqw}. Moreover, a geometric generalization of the blowup equation approach was formulated for certain local Calabi-Yau 3-folds \cite{Gu:2017ccq, Huang:2017mis}. This geometric formulation has been extensively studied recently for computing refined BPS invariants of various 5d SCFTs and elliptic genera of 6d SCFTs admitting geometric constructions in M-/F-theory on local (toric or elliptic) CY 3-folds \cite{Gu:2017ccq, Huang:2017mis, Gu:2018gmy, Gu:2019dan, Gu:2019pqj, Gu:2020fem}.

Despite the fact that the blowup method is a very powerful tool for computing BPS spectra of a broad class of 5d/6d SCFTs even beyond the scope of other computational methods, there is as yet no complete formulation of the blowup equations that can be applicable for all supersymmetric theories in five and six dimensions. For example, though the blowup equations obtained in \cite{Kim:2019uqw} cover a large set of 5d gauge theories with simple gauge group including some theories whose ADHM descriptions are not known, it still requires further studies for plenty of other interesting theories such as 5d quiver gauge theories, 5d theories with half-hypermultiplets, and 5d Kaluza-Klein (KK) theories arising from 6d SCFTs compactified on a circle with outer-automorphism twists. In particular, there exist some 5d/6d gauge theories, for instance, the 5d $SU(3)_8$ gauge theory, that currently have neither ADHM constructions nor shrinkable geometric descriptions nor associated 5-brane webs, so that all the computational methods introduced above including the blowup method cannot be used to compute their BPS partition functions.

The main aim of this paper is to devise a complete blowup formalism that enables one to compute BPS spectra of {\it all} supersymmetric field theories having UV completions in five or six dimensions. In this paper, we will generalize the Nakajima-Yoshioka's blowup equations in \cite{Nakajima:2003pg} to arbitrary 5d/6d gauge theories (including quiver gauge theories and twisted circle compactifications of 6d theories) with matter fields in arbitrary representations, and also extend the work of \cite{Gu:2017ccq,Huang:2017mis} for a geometric application of the blowup equations to any local Calabi-Yau 3-folds (including elliptic and non-toric ones) based on novel geometric constructions of 5d SCFTs/KK theories introduced in \cite{Jefferson:2018irk,Bhardwaj:2019fzv}.

The blowup equation is a functional equation identifying two partition functions on different backgrounds, one is the $\Omega$-deformed $\mathbb{C}^2$ and another one is the one point blow-up $\hat{\mathbb{C}}^2$ of the $\mathbb{C}^2$, that are related to each other by a smooth blow-up or blow-down transition. Each 5d/6d field theory can have a number of blowup equations depending on background magnetic fluxes residing on the blown-up $\mathbb{P}^1$ in $\hat{\mathbb{C}}^2$. As we will discuss in section~\ref{sec:BlowupEq}, such blowup equations can be solved recursively by expanding them in terms of K\"ahler parameters of the theory.

The main input in our recursion process is the {\it effective prepotential $\mathcal{E}$} of a given 5d/6d SQFT evaluated on the $\Omega$-background. The effective prepotential, as we will illustrate more precisely in section~\ref{sec:5dthyOmega}, is fully determined by effective cubic and mixed Chern-Simons terms in the low energy theory on the Coulomb branch which can be systematically calculated by computing the induced Chern-Simons terms after integrating out charged fermions \cite{Witten:1996qb} (and also by collecting classical Green-Schwarz contributions for a 6d theory on a circle with/without a twist \cite{Bhardwaj:2019fzv}). See \cite{Maldacena:1997de,Katz:2020ewz} for the geometric counterparts of the effective Chern-Simons terms. Therefore, the effective prepotential, which is one of the main ingredients for our blowup formula, can be systematically computed for every 5d/6d field theory having a UV completion. Here we assume that every UV finite 5d/6d theory either has a gauge theory description in 5d or in 6d on a circle with/without twist, or has a geometric description as a local (elliptic) Calabi-Yau 3-fold, or can be obtained by RG-flows thereof.

By seeding the effective prepotential as well as a consistent choice of background magnetic fluxes on the blown-up $\mathbb{P}^1$ into the blowup equations, we can {\it bootstrap} the spectrum of charged BPS states in a given 5d/6d theory. Since the effective prepotential can be easily prepared for any arbitrary 5d/6d field theory, we now make a bold conjecture that we can compute BPS spectra of all 5d/6d field theories by employing our bootstrap method to formulate and solve blowup equations in those theories.

As concrete examples for our conjecture, we will apply our method to all rank-1 and rank-2 5d supersymmetric field theories including also KK theories to obtain their BPS spectra explicitly. This will involve the $SU(3)_{15/2}+1{\bf F}$ theory (dual to the $\mathcal{N}=2$ $G_2$ gauge theory), the $SU(3)_8$ theory, and new rank-1 and rank-2 5d SCFTs, which we call the local $\mathbb{P}^2+1{\bf Adj}$ and the local $\mathbb{P}^2\cup \mathbb{F}_3+1{\bf Sym}$, obtained by mass deformations of the 5d $\mathcal{N}=2$ $Sp(N)_\pi$ gauge theories with $N=1,2$ respectively first introduced in \cite{Bhardwaj:2019jtr}. We emphasize that the partition functions of these theories cannot be obtained by other means since they have none of ADHM constructions, conventional geometric constructions, and also brane webs (but we will introduce brane webs for the new rank-1 and rank-2 theories in this paper).

We will also compute BPS spectra of some higher rank theories. In particular, we will show that the blowup equations for the $SU(5)_8$ theory can be solved. The result shows that this theory may have no physical Coulomb branch and thus be inconsistent in UV limit. This theory was `undetermined' to exist in \cite{Bhardwaj:2020gyu} because its existence was neither confirmed nor ruled out with currently known techniques. Our computation provides a supporting evidence that the $SU(5)_8$ theory has no UV completion. In this sense, our bootstrap approach can be used to confirm or disprove the existence of certain 5d and 6d QFTs. \bigskip

The organization of the paper is as follows. In section~\ref{sec:5dthyOmega}, we review salient features of 5d/6d supersymmetric gauge theories and their geometry engineering. In section~\ref{sec:BlowupEq}, we explain the blowup equation as a tool for bootstrapping BPS spectra of 5d/6d supersymmetric theories, and discuss our main conjecture with instructive examples. Section~\ref{sec:rank1theories} and section~\ref{sec:rank2 theories} are devoted to cover all rank-1 and rank-2 5d theories including KK theories. We also discuss some interesting higher rank theories, in section~\ref{sec:higher rank theories}, including $SU(4)_8$ and $SU(5)_8$. We then conclude with some subtle issues. In Appendix~\ref{appendix:1-loop}, we further discuss 1-loop partition functions of 6d SCFTs on a circle with twists. In Appendix~\ref{sec:appendix2}, some new 5-brane webs associated with frozen singularity are presented. \bigskip

\paragraph{Notation:} To avoid the cluttering of theories, we denote by $G+N_{\mathbf{r}}\,\mathbf{r}$, the theory of gauge group $G$ with $N_\mathbf{r}$ number of hypermultiplets in the representation $\mathbf{r}$. Gauge group $G$ can be any classical groups, $SU(N)$, $SO(N)$, $Sp(N)$, and exceptional groups $G_2$, $F_4$, $E_6$, $E_7$, $E_8$ as well as a quiver gauge group. For hypermultiplets in the $\mathbf{r}$ representation of $G$, we use the following shorthand notation: $\mathbf{F}$ for fundamental, $\mathbf{bi\text{-}F}$ for bi-fundamental, $\mathbf{\Lambda}^n$ for rank-$n$ antisymmetric, $\mathbf{Sym}$ for symmetric, $\mathbf{Adj}$ for adjoint, $\mathbf{S}$ for spinor, and $\mathbf{C}$ for conjugate spinor. For example, $SU(2)+8\mathbf{F}$ means $SU(2)$ gauge theory with 8 hypermultiplets in the fundamental representation. For the $\Omega$-deformation parameters $\epsilon_{1}, \epsilon_{2}$, we frequently use $\epsilon_+=\frac{\epsilon_1+\epsilon_2}{2}$ and the fugacities associated with them are denoted by $p_1 =e^{-\epsilon_1}$ and $p_2= e^{-\epsilon_2}$. We also denote the set of complex, real, rational, integer numbers by $\mathbb{C}$, $\mathbb{R}$, $\mathbb{Q}$, $\mathbb{Z}$, respectively. 
