\section{5-brane webs for theories of frozen singularities}\label{sec:appendix2}

In this section, we present 5-brane webs for some theories of frozen singularities, which contain an $O7^+$-plane. 

\subsection{\texorpdfstring{5-brane webs for $SU(2)_\pi+1\mathbf{Adj}$ and local $\mathbb{P}^2+``1\mathbf{Adj}"$}{5-brane webs for SU(2)pi + 1Adj and local P2 + 1Adj}}

A 5-brane configuration for $SU(2)_\pi+1\mathbf{Adj}$ is depicted in Figure~\ref{fig:su2+1adj}(a). A little bit of deformation of this 5-brane web leads to a 5-brane web given in Figure~\ref{fig:su2+1adj}(b). By taking the $(1,-1)$ 7-brane painted in red through the cut of an $O7^+$-plane, one gets a 5-brane web in Figure~\ref{fig:su2+1adj}(c). Notice here that on the left hand side, the 5-brane configuration looks like ${\rm d}\mathbb{P}_1$ geometry locally. Hence, if one decouples the adjoint hypermultiplet by taking an $O7^+$ far away, then one obtains a 5-brane web for the pure $SU(2)_\pi$ theory as expected. On the other hand, recalling that ${\rm d}\mathbb{P}_1$ has an $\mathcal{O}(-1)$ curve, we can flop this $-1$ curve, which gives a 5-brane web given in Figure~\ref{fig:su2+1adj}(d). By taking this flopped part away, one finds that the remaining part is a local $\mathbb{P}^2$ with the adjoint hypermultiplet inherited from $SU(2)_\pi+1\mathbf{Adj}$, as depicted in Figure \ref{fig:su2+1adj}(e). which we have referred to as $\mathbb{P}^2+``1\mathbf{Adj}"$. One can also view this decoupling as decoupling the `instantonic' hypermultiplet as discussed in the main text.
\begin{figure}
	\includegraphics[width=5cm]{fig-su2+Adj.pdf}
	\centering	
	\caption{(a) A 5-brane web for $SU(2)_\pi+1\mathbf{Adj}$. (b) An equivalent 5-brane web for $SU(2)_\pi+1\mathbf{Adj}$ where the D5 brane located on the right hand side of an $O7^+$-plane in figure (a) is moved to the left hand side. (c) Hanany-Witten transition associated with the (1,1) 7-brane painted in blue in figure (b), going through the cut of the $O7^+$-plane. (d) A flop transition. (e) Decoupling the flopped `instantonic' hypermultiplet, giving rise to a 5-brane web for the local $\mathbb{P}^2+``1\mathbf{Adj}"$ theory. Clearly putting an $O7^+$-plane far away leads to a local $\mathbb{P}^2$ which corresponds to decoupling ``$1\mathbf{Adj}$". }
	\label{fig:su2+1adj}
\end{figure}


\subsection{\texorpdfstring{5-brane web for $SU(2)_0+1\mathbf{Adj}$}{5-brane web for SU(2)0 + 1Adj}}

$SU(2)_0+1\mathbf{Adj}$ is a KK theory which is also referred to as the M-string. The corresponding web, depicted in Figure~\ref{fig:su2_0+1adj}(c), was discussed in \cite{Haghighat:2013gba}. We note that one also depicts it with an $O7^+$-plane as given in Figure~\ref{fig:su2_0+1adj}(a). It is not difficult to see that the K\"ahler parameters of these two 5-brane webs are equivalent as shown in Figure~\ref{fig:su2_0+1adj}(b) and Figure \ref{fig:su2_0+1adj}(c).
\begin{figure}[t]
\includegraphics[width=8cm]{fig-SU2_0+1Adj.pdf}
\centering	
\caption{(a) A 5-brane web for $SU(2)_0+1\mathbf{Adj}$. (b) The reflected image of (a) is included.  (c) A 5-brane configuration for the M-string.}
\label{fig:su2_0+1adj}
\end{figure}


\subsection{\texorpdfstring{5-brane web for $Sp(2)_{0}+1\mathbf{Adj}$}{5-brane web for Sp(2)0 + 1Adj}}

A 5-brane web for 5d $Sp(2)_0+1\mathbf{Adj}$ can be obtained by a $\mathbb{Z}_2$ twisting of the 6d $\mathcal{N}=(2,0)$ $A_3$ theory whose brane configuration can be realized as a $D6$ brane suspended between 4 $NS5$ branes, as depicted in Figure~\ref{fig:Sp2_0+1Adj}(a).
\begin{figure}[t]
	\includegraphics[width=7cm]{fig-Sp2_0+1Adj.pdf}
	\centering	
	\caption{(a) Type IIA configuration for 6d $SU(1)-SU(1)-SU(1)$ theory or equivalently $SU(4)$ theory. (b) A $\mathbb{Z}_2$ twisted compactification of (a), where 1/2 $D5$-branes are stuck along the cut of an $O7^+$-plane. (c) Type IIB 5-brane configuration after resolving an $O7^-$ into two 7-branes of charge $(1,1)$ and $(1, -1)$. Here a blue dot refers to a $(1,1)$ 7-brane while a red dot refers to a $(1,-1)$ 7-brane.}
	\label{fig:Sp2_0+1Adj}
\end{figure} 
The procedure of a $\mathbb{Z}_2$ twisting of the 6d $\mathcal{N}=(2,0)$ $A_N$ theory is already considered in \cite{Hayashi:2015vhy}. Such twisted compactification on 5-brane gives rise to a pair of $O7^-$- and $O7^+$-planes and the number of $D6$ branes in Type IIA brane configuration is halved to yield half the number of $D5$-branes. For this $A_3$ case, the resulting 5-brane configuration is given in Figure~\ref{fig:Sp2_0+1Adj}(b), where a half $D5$ is stuck along the cut of an $O7^+$-plane. By resolving an $O7^-$ into two 7-branes of charges $(1,1)$ and $(1, -1)$, one gets a 5-brane configuration for $Sp(2)_0+1\mathbf{Adj}$ as depicted in Figure~\ref{fig:Sp2_0+1Adj}(c). We note that in Figure~\ref{fig:Sp2_0+1Adj}(c), we moved two half $D5$-branes on the right hand side of an $O7^+$ to the left to form a full $D5$-brane so that they then can be away from the $O7^+$ cut at the bottom, which allows us to give the mass of the adjoint hypermultiplet.


\subsection{\texorpdfstring{5-brane web for $Sp(2)_\pi+1\mathbf{Adj},~ SU(3)_{\frac32}+1\mathbf{Sym}$}{5-brane web for Sp(2)pi + 1Adj}}

A 5-brane web for $Sp(2)_\pi + 1\mathbf{Adj}$ can be obtained a $\mathbb{Z}_2$ twisted compactification of the 6d $\mathcal{N}=(2,0)$ $A_4$ theory. The resulting web is given in Figure~\ref{fig:Sp2_pi+1Adj}, where two half $D5$ branes are suspended either between an $O7^+$ and $(1,1)$ 5-brane or between an $O7^+$ and $(-2, 1)$ 5-brane. 
\begin{figure}
\centering	
\includegraphics[width=6cm]{fig-Sp2_pi+1Adj.pdf}
\caption{A 5-brane web for $Sp(2)_\pi + 1\mathbf{Adj}$ or $SU(3)_\frac32+1\mathbf{Sym}$.}
\label{fig:Sp2_pi+1Adj}
\end{figure}
As before, two half $D5$-branes can be put together to form a full $D5$-brane. It can be then easily recognized that the resulting 5-brane configuration is nothing but that for $SU(3)_\frac32+1\mathbf{Sym}$ as given in Figure~\ref{fig:F3+P2+1Sym}(a).

\begin{figure}[t]
	\includegraphics[width=7cm]{fig-F3+P2+1Sym.pdf}
	\centering	
	\caption{(a) A 5-brane web for $SU(3)_\frac32+1\mathbf{Sym}$ which has a $-1$ curve on the left hand side of an $O7^+$-plane. (b) A flop transition and allocating a D5-brane from the right to the left. (c)  Decoupling the `instantonic' hypermultiplet, which leads to a 5-brane configuration for a non-Lagrangian theory, $\mathbb{P}^2\cup\mathbb{F}_3+``1\mathbf{Sym}."$}
	\label{fig:F3+P2+1Sym}
\end{figure}
\subsection{\texorpdfstring{5-brane web for $\mathbb{P}^2\cup\mathbb{F}_3+``1\mathbf{Sym}"$ or $ ``SU(3)_{\frac32}+1\mathbf{Sym}-1\mathbf{F}"$}{5-brane web for P2 U F3 + 1Sym}}\label{sec:app-P2 U F3+1Sym}
In a similar way as done for the 5-brane configuration for $\mathbb{P}^2+1\mathbf{Adj}$, one can easily get a 5-brane configuration for $\mathbb{P}^2\cup\mathbb{F}_3+``1\mathbf{Sym}"$ or $ ``SU(3)_{\frac32}+1\mathbf{Sym}-1\mathbf{F}"$ as depicted in Figure~\ref{fig:F3+P2+1Sym}. Given a 5-brane configuration for $SU(3)_\frac32+1\mathbf{Sym}$ in Figure~\ref{fig:F3+P2+1Sym}(a), which has a $-1$ curve, one performs a flop transition to yield 5-brane web in Figure~\ref{fig:F3+P2+1Sym}(b). Decoupling this `instantonic' hypermultiplet gives rise to $\mathbb{P}^2\cup\mathbb{F}_3+``1\mathbf{Sym}"$ whose 5-brane web is given in Figure~\ref{fig:F3+P2+1Sym}(c), which we may be referred to as $ ``SU(3)_{\frac32}+1\mathbf{Sym}-1\mathbf{F}."$



\subsection{5-brane web for \texorpdfstring{$SU(3)_{0}+1\mathbf{Sym}+1\mathbf{F}$}{SU(3)0 + 1Sym + 1F}}
A 5-brane configuration for $SU(3)_{0}+1\mathbf{Sym}+1\mathbf{F}$ is depicted in Figure~\ref{fig:SU3_0+1Sym+1F}, which is discussed in detail in \cite{Hayashi:2018lyv}.
\begin{figure}[t]
\includegraphics[width=6cm]{fig-SU3_0+1Sym+1F.pdf}
\centering	
\caption{A 5-brane configuration for $SU(3)_0+1\mathbf{Sym}+1\mathbf{F}$.}
\label{fig:SU3_0+1Sym+1F}
\end{figure}


\subsection{5-brane web for \texorpdfstring{local $\mathbb{P}^2\cup \mathbb{F}_6 +``1\mathbf{Sym}"$}{P2 U F6 + 1Sym}}\label{sec:app-P^2F_6+1Sym}

A 5-brane web for $\mathbb{P}^2\cup \mathbb{F}_6 +``1\mathbf{Sym}"$ is depicted in Figure~\ref{fig:P2-F6+1Sym}(e). It is worthy of noting that this 5-brane web can be obtained from a non-perturbative Higgsing of $SU(4)_0+1\mathbf{Sym}$ in Figure~\ref{fig:SU4_0+1Sym}.
\begin{figure}
	\includegraphics[width=14cm]{fig-SU4_0+1Sym.pdf}
	\centering	
	\caption{(a) A 5-brane web for $SU(4)_0+1\mathbf{Sym}$. (b) A non-perturbative Higgsing by putting two $NS5$-branes together and perform a Higgsing such that the $NS5$-brane ending two $(0,1)$ 7-branes is taken away along the $x^{7,8,9}$-directions.}
	\label{fig:SU4_0+1Sym}
\end{figure}
By taking a non-perturbative Higgsing\footnote{A simple non-perturbative Higgsing procedure is from $SU(3)_0+5\mathbf{F}$ to $SU(2)+5\mathbf{F}$ where two parallel $NS5$-branes are bound together in such a way that the resulting 5-brane web preserves the S-rule \cite{Hayashi:2013qwa, Kim:2014nqa}.} shown in Figure~\ref{fig:SU4_0+1Sym}(b), we get a new rank-2 non-Lagrangian theory, $\mathbb{P}^2\cup \mathbb{F}_6 +``1\mathbf{Sym}"$. To see local geometry apart from the frozen singularity, consider a series of deformations of the 5-brane web given in Figure~\ref{fig:P2-F6+1Sym}. 
Lowering the $D5$-brane on the right side of an $O7^+$-plane leads to a transition like Figure \ref{fig:P2-F6+1Sym}(a) discussed in \cite{Hayashi:2017btw} giving rise to a 5-brane web in Figure \ref{fig:P2-F6+1Sym}(b). A further lowering of the $D5$-brane allocates the $D5$-brane from the right to the bottom left as depicted in Figure \ref{fig:P2-F6+1Sym}(c). The resulting 5-brane configuration then looks like locally $\mathbb{P}^2\cup \mathbb{F}_6$. Taking into account $1\mathbf{Sym}$ inherited from $SU(4)_0+1\mathbf{Sym}$, we call this $ \mathbb{P}^2\cup \mathbb{F}_6 +``1\mathbf{Sym}"$.
\begin{figure}
	\includegraphics[width=8cm]{fig-P2-F6+1Sym.pdf}
	\centering	
	\caption{Deformations of the 5-brane web given in Figure \ref{fig:SU4_0+1Sym}(b) leading to a 5-brane web for $\mathbb{P}^2\cup \mathbb{F}_6 +``1\mathbf{Sym}"$.}
	\label{fig:P2-F6+1Sym}
\end{figure}

We note that it is straightforward to implement this kind of non-perturbative Higgsing to $SU(N)_\frac{4-N}{2}+1\mathbf{Sym}$ to yield rank $N-1$ non-Lagrangian theory with a frozen singularity, $\mathbb{P}^2\cup \mathbb{F}_6 \cup \mathbb{F}_8 \cup \cdots \cup \mathbb{F}_{2N}+``1\mathbf{Sym}"$, as shown in Figure~\ref{fig:P2F682N}.
\begin{figure}[t]
\includegraphics[width=15cm]{fig-P2F682N.pdf}
\centering	
\caption{(a) A 5-brane web for $SU(N)_\frac{4-N}{2}+1\mathbf{Sym}$, where there are $N$ D5-branes in total. (b) A non-perturbative Higgsing. (c) 5-brane web for $\mathbb{P}^2\cup \mathbb{F}_6 \cup \mathbb{F}_8 \cup \cdots \cup \mathbb{F}_{2N}+``1\mathbf{Sym}"$ which is rank of $N-2$.}
\label{fig:P2F682N}
\end{figure}
\begin{figure}[t]
\includegraphics[width=15cm]{fig-localP2+1Adj-2.pdf}
\centering	
\caption{Various phases of 5-brane web for local $\mathbb{P}^2 +``1\mathbf{Adj}"$.}
\label{fig:localP2+1Adj-2}
\end{figure}
In particular, when $N=2$, it provides yet another way of obtaining a 5-brane web for the local $\mathbb{P}^2+``1\mathbf{Adj}"$ theory giving rise to various phases for the theory as depicted in Figure~\ref{fig:localP2+1Adj-2}(a)-(d). Note that a 5-brane configuration in Figure~\ref{fig:localP2+1Adj-2}(a) looks as if all W-bosons are legitimate so that the resulting theory is a Lagrangian theory, but it appears that some of the W-bosons would be annihilated so that the corresponding system is that of non-Lagrangian theory. \medskip \bigskip

\begin{figure}[t]
	\includegraphics[width=9cm]{fig-localP2+1Adj+2F.pdf}
	\centering	
	\caption{From 5-brane web for local $\mathbb{P}^2 \cup \mathbb{F}_6+``1\mathbf{Sym}+2\mathbf{F}"$ to that for $SU(3)_0 +1\mathbf{Sym}+1\mathbf{F}$. (a) A 5-brane web for local $\mathbb{P}^2 \cup \mathbb{F}_6+``1\mathbf{Sym}+2\mathbf{F}"$. (b) Locating D5-brane on the right-hand side of $O7^+$ to the left. (c) Pull down the $(0,1)$ 7-brane (in red) in figure (b) toward the cut of $O7^+$ and perform an $SL(2,\mathbb{Z})$ transformation. (d) With some mass deformations, Hanany-Witten move with the $(-3, 1)$ 7-brane in figure (c) which gives rise to an 5-brane web for $SU(3)_0 +1\mathbf{Sym}+1\mathbf{F}$. (d) A little deformation leading to the same web for $SU(3)_0 +1\mathbf{Sym}+1\mathbf{F}$ given in Figure~\ref{fig:SU3_0+1Sym+1F}.}
	\label{fig:localP2+1Adj+2F}
\end{figure}

\noindent\underline{Equivalence between $\mathbb{P}^2\cup \mathbb{F}_6 +``1\mathbf{Sym}+2\mathbf{F}"$ and $SU(3)_{0}+1\mathbf{Sym}+1\mathbf{F}$} \\
We can show that $\mathbb{P}^2\cup \mathbb{F}_6 +``1\mathbf{Sym}+2\mathbf{F}"$ and $SU(3)_{0}+1\mathbf{Sym}+1\mathbf{F}$ are equivalent by transforming a 5-brane web for $\mathbb{P}^2\cup \mathbb{F}_6 +``1\mathbf{Sym}+2\mathbf{F}"$ into that of $SU(3)_{0}+1\mathbf{Sym}+1\mathbf{F}$ that is given in Figure \ref{fig:SU3_0+1Sym+1F}. Consider a 5-brane web for $\mathbb{P}^2\cup \mathbb{F}_6 +``1\mathbf{Sym}+2\mathbf{F}"$ which is to add two flavor $D7$-branes to $\mathbb{P}^2\cup \mathbb{F}_6 +``1\mathbf{Sym}"$ in Figure~\ref{fig:SU4_0+1Sym}(b). This resulting 5-brane web is depicted in Figure~\ref{fig:localP2+1Adj+2F}(a). The deformations given in Figure~\ref{fig:localP2+1Adj+2F}(b) to \ref{fig:localP2+1Adj+2F}(d) lead to a 5-brane web for $SU(3)_0 +1\mathbf{Adj}+1\mathbf{F}$ in Figure~\ref{fig:localP2+1Adj+2F}(e) which is the same as the one in Figure~\ref{fig:SU3_0+1Sym+1F}. This clearly suggests that there is a new RG flow from $SU(3)_0+1\mathbf{Sym}+1\mathbf{F}$ to local $\mathbb{P}^2 \cup \mathbb{F}_6+``1\mathbf{Sym}+1\mathbf{F}"$.

The above equivalence relation is readily generalized to the following $SU(N)_0$ KK theory,
\begin{align}
&SU(N)_0 +1\mathbf{Sym}+(N-2)\mathbf{F} \cr
&~~~~\text{is equivalent to }	\cr
&\text{local }\mathbb{P}^2 \cup \mathbb{F}_6 \cup \mathbb{F}_8 \cup \cdots \cup \mathbb{F}_{2N-2} \cup \mathbb{F}_{2N} +``1\mathbf{Sym}+(N-1)\mathbf{F}".
\end{align}
We note that when $N=2$, the equivalence reads
\begin{align}
	SU(2)_\pi+ 1 \mathbf{Adj}\quad \Longleftrightarrow \quad
	\text{local }\mathbb{P}^2+ 1 \mathbf{Adj} + 1\mathbf{F}.
\end{align}

