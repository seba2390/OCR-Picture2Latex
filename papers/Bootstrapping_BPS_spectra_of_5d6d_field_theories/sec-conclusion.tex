\section{Conclusion}\label{sec:Conclusion}

In this paper, we proposed a systematic bootstrap method for BPS spectra of 5d $\mathcal{N} = 1$ field theories including KK theories, based on the Nakajima-Yoshioka's blowup equation. As the main input, we introduced the effective prepotential $\mathcal{E}$ and the consistent magnetic fluxes. The effective prepotential incorporates the usual cubic prepotential, the mixed gauge/gravitational, and the mixed gauge/$SU(2)_R$ Chern-Simons terms, which can be readily obtained for every 5d SQFT, as discussed in section~\ref{sec:5dthyOmega}. The consistent magnetic fluxes that one can turn on should satisfy the quantization condition. We discussed possible quantization conditions in section~\ref{sec:BlowupEqReview}. Equipped with these inputs, one can formulate blowup equations and solve them recursively to obtain BPS spectrum for a 5d QFT. We conjecture that our method applies to all the 5d $\mathcal{N}=1$ theories and 6d theories on a circle with/without a twist. To support this we explicitly showed how to bootstrap for all rank-1 and rank-2 theories as well as some of interesting higher rank theories. In particular, we computed BPS spectra of the theories whose partition functions still remain as challenges from other methods such as the ADHM or topological vertex method. For instance, $SU(3)_8$, $SU(4)_8$, and some non-Lagrangian theories.

There are some open questions that beg to be resolved. The first question concerns our main conjecture which asserts that BPS spectra of all UV finite theories in 5d and 6d can be obtained by solving blowup equations. The bootstrapping method proposed in this paper allows us to build a collection of blowup equations for any arbitrary supersymmetric field theory in 5d or 6d. However, physical or mathematical proofs for our conjecture that the blowup equations must be solved to compute the correct BPS spectrum of a QFT are currently lacking. The proof may require a physical derivation of blowup equations and further studies on the structure of blowup equations for general 5d and 6d field theories. We would like to provide more discussions about this in the future.

A related question is whether the blowup equations can always be solved. As shown in \cite{Huang:2017mis}, the blowup equations are solvable if there exists at least one unity blowup equation. There is however a class of theories only admitting blowup equations of the vanishing type, for example, 6d SCFTs with half hypermultiplets. Unfortunately, it is not proven yet whether the vanishing blowup equations are enough to compute full BPS spectra of those theories. Nevertheless, we claimed without proof that though vanishing blowup equations may not be solved by themselves, we can still compute full BPS spectra for those theories from them provided that they are supplemented with other additional constraints such as non-negativity of BPS degeneracies and KK tower structure for 6d theories as well as dualities, geometric descriptions. We have checked this claim for several examples including the 6d minimal $E_7$ SCFT with a half hypermultiplet up to certain lower orders, but we leave the proof of solvability of vanishing blowup equations to future study.

In section~\ref{sec:higher rank theories}, we saw that the $SU(5)_8$ theory has a solvable blowup equation, but its solution shows that this theory has inconsistent BPS spectrum, by which we mean that the theory does not have physical Coulomb branch where all BPS states have non-negative masses in UV. This theory was classified as an `undetermined' theory in \cite{Bhardwaj:2020gyu} as its existence couldn't be proved or disproved yet. Our computation for this theory supports that this theory has no UV completion. Similarly, one can try to compute BPS spectra for other `undetermined' theories and see if they can have non-trivial Coulomb branches. If not, such theories are inconsistent in UV with singularities on their Coulomb branches. Thus we suggest that BPS spectra computed by our bootstrap method for such `undetermined' theories can be used to distinguish UV completable theories.

Partition functions on other supersymmetric backgrounds are also important observables in 5d/6d field theories. Many of them can be factorized into a product of partition functions on the $\Omega$-background localized at fixed points of spatial Lorentz rotations under supersymmetric localization. For instance, a superconformal index which counts BPS operators in a 5d CFT can be understood as a partition function on $S^4\times S^1$ and a supersymmetric localization factorizes it into a product of two  partition functions on $\mathbb{R}^4\times S^1$, each of which we can compute by solving the blowup equations, around the north and the south poles of $S^4$ \cite{Kim:2012gu}. This implies that in principle the blowup equations can be used to compute the superconformal index as well. We notice however that the superconformal index is usually expressed as a power series expansion of the fugacity $x\equiv e^{-\epsilon_+}$ related to conformal dimensions of local operators, and thus it is calculated in the parameter regime where the $\Omega$-parameter $\epsilon_+$ is much bigger than K\"ahler parameters $\phi_i$ and $m_j$ in the theory. On the other hand, we have solved the blowup equations in a different parameter regime where all the K\"ahler parameters are bigger than the $\Omega$-deformation parameters. It would be interesting to see if the blowup equations can be solved in the other parameter regimes, in particular where $\epsilon_+\gg \phi_i,m_j$, so that we can also compute superconformal indices and other observables having similar factorization structures by applying the blowup method.

Lastly, it is intriguing to ask whether the blowup equations can capture asymptotics of BPS partition functions in the large $N$ limit (or in the large rank limit) of 5d and 6d field theories. Asymptotic behavior of supersymmetric partition functions at a large number of degrees of freedom has been playing a crucial role in studying holographic dual theories. Holographic interpretation of the blowup formula in the large $N$ limit, if well-defined, can give us a new implication on physics of dual gravity theories.

\acknowledgments
We would like to thank Lakshya Bhardwaj, Hirotaka Hayashi, Chiung Hwang, Seok Kim, Kimyeong Lee, Houri-Christina Tarazi and Futoshi Yagi for valuable discussions and comments. SSK thanks APCTP, KIAS and POSTECH for his visit where part of this work is done. The research of HK and MK is supported by the POSCO Science Fellowship of POSCO TJ Park Foundation and the National Research Foundation of Korea (NRF) Grant 2018R1D1A1B07042934.