In this section, we study the behaviour of the solutions of the system (\ref{eq:PVlog-system}) near the set $\mathcal{I}$, where the vector field is infinite.
We prove that $\mathcal{I}$ is a repeller for the solutions and that each solution which comes sufficiently close to $\mathcal{I}$ at a certain point $t$ will have a pole in a neighbourhood of $t$.

\begin{lemma}\label{lemma:L2}
For every $\epsilon_1>0$ there exists a neighbourhood $U$ of $\mathcal{L}_2^p$ such that
$$
\left|
\frac{E'}{E}+\theta_1 e^t
\right|
<\epsilon_1
\quad
\text{in}\ U.
$$
\end{lemma}

\begin{proof}
In the corresponding charts near $\mathcal{L}_2^p$ (see Section \ref{a2-blow}), the function
$$
r=\frac{E'}{E}+\theta_1 e^t
$$
is equal to:
$$
\begin{aligned}
r_{31} &= \frac{\theta_1 e^t y_{31} z_{31} (\eta y_{31}-2 (\theta_0 y_{31}-1) (y_{31} z_{31}-1))}
{\eta y_{31} (y_{31} z_{31}-1)+F y_{31}^2-\theta_0 y_{31} (y_{31} z_{31}-1)^2+y_{31}^2 z_{31}^2-2 y_{31} z_{31}+1},
\\
r_{32} &= \frac{\theta_1 e^t y_{32} z_{32} (\eta-2 (y_{32}-1) (\theta_0-z_{32}))}
{(y_{32}-1) z_{32} (\eta-(y_{32}-1) (\theta_0-z_{32}))+F}.
\end{aligned}
$$
The statement of the lemma follows immediately from these expressions, since  $\mathcal{L}_2^p$ is given by $z_{31}=0$ and $y_{32}=0$.
\end{proof}


\begin{lemma}\label{lemma:L1}
For each compact subset $K$ of
$\mathcal{L}_1^p\setminus\mathcal{L}_8^p$
there exists a neighbourhood $V$ of $K$ and a constant $C>0$ such that:
$$
\left| e^{-t}\frac{E'}{E} \right|<C
\quad
\text{in}\ V\ \text{for all}\ t.
$$
\end{lemma}

\begin{proof}
Near $\mathcal{L}_1^p$, in the respective coordinate charts (see Section \ref{a1-blow}), we have:
$$
e^{-t}\frac{E'}{E}\sim
\begin{cases}
-\dfrac{\theta_1 (y_{21}+1)}{y_{21}-1},
\\
\dfrac{ \theta_1 (z_{22}+1)}{z_{22}-1}.
\end{cases}
$$
Since the projection of $\mathcal{L}_8^p$ to these two charts is the point on $\mathcal{L}_1^p$ given by coordinates $y_{21}=1$ and $z_{22}=0$, the statement is proved.
\end{proof}

The modulus $|d|$ of the function $d$ from the next lemma will serve as a measure for the distance from the infinity set $\mathcal{I}$.
Throughout the paper, we denote by $J_{n}$ the Jacobian of the coordinate change from $(y,z)$ to $(y_{n},z_{n})$:
$$
J_{n}=\frac{\partial y_{n}}{\partial y}\frac{\partial z_{n}}{\partial z}-\frac{\partial y_{n}}{\partial z}\frac{\partial z_{n}}{\partial y}.
$$

\begin{lemma}
There exists a continuous complex valued function $d$ on a neighbourhood of the infinity set $\mathcal{I}$ in the Okamoto space, such that:
$$
d=
\begin{cases}
\frac1E, & \text{in a neighbourhood of}\ \ \mathcal{I}\setminus(\mathcal{L}_3^p\cup\mathcal{L}_4^p\cup\mathcal{L}_8^p\cup\mathcal{L}_9^p),
\\
%J_{62}, & \text{in a neighbourhood of}\ \ \mathcal{L}_2^p\setminus\mathcal{L}_0^p
%\\
-J_{82}, & \text{in a neighbourhood of}\ \ \mathcal{L}_3^p\setminus\mathcal{L}_1^p,
%\\
%(\theta_1e^{t})^{-1} J_{91}, & \text{in a neighbourhood of}\ \ \mathcal{L}_4^p\setminus\mathcal{L}_8^p,
\\
-J_{102}, & \text{in a neighbourhood of}\ \ (\mathcal{L}_4^p\cup\mathcal{L}_8^p)\setminus\mathcal{L}_1^p,
\\
-J_{112}, & \text{in a neighbourhood of}\ \ \mathcal{L}_9^p\setminus\mathcal{L}_8^p.
\end{cases}
$$
\end{lemma}

\begin{proof}
%Assume $d$ is defined as $\frac1E$, in a neighbourhood of $\mathcal{I}\setminus(\mathcal{L}_3^p\cup\mathcal{L}_4^p\cup\mathcal{L}_8^p\cup\mathcal{L}_9^p)$.

%From Section \ref{a5-blow}, the line $\mathcal{L}_{2}^p$ is given by $z_{62}=0$ in the $(y_{62},z_{62})$ chart.
%Thus as we approach $\mathcal{L}_{2}^p$, i.e., as $z_{62}\to0$, we have
%$$EJ_{62}\sim$$

From Section \ref{a7-blow}, the line $\mathcal{L}_{3}^p$ is given by $z_{82}=0$ in the $(y_{82},z_{82})$ chart.
Thus as we approach $\mathcal{L}_{3}^p$, i.e., as $z_{82}\to0$, we have
$$
EJ_{82}\sim-1.
$$

%From Section \ref{a8-blow}, the line $\mathcal{L}_4^p$ is given by $y_{91}=0$ in the $(y_{91},z_{91})$ chart.
%Thus as we approach $\mathcal{L}_4^p$, i.e., as $y_{91}\to0$, we have
%$$\frac{J_{91}}{J_{102}}\sim-\theta_1e^t.$$

From Section \ref{a9-blow}, the line $\mathcal{L}_{8}^p$ is given by $z_{102}=0$ in the $(y_{102},z_{102})$ chart.
Thus as we approach $\mathcal{L}_{8}^p$, i.e., as $z_{102}\to0$, we have
$$
EJ_{102}\sim-1-\frac{\theta_1 e^t}{y_{102}}.
$$
In the same chart, the line $\mathcal{L}_{4}^p$ is given by $y_{102}=-\theta_1e^t$.

From Section \ref{a10-blow}, the line $\mathcal{L}_9^p$ is given by $z_{112}=0$ in the  $(y_{112},z_{112})$ chart.
We have:
$$
\frac{J_{112}}{J_{102}}=1+\frac{(1+\eta)\theta_1 e^t}{y_{112}}.
$$
 \end{proof}
 
 


 \begin{lemma}[Behaviour near $\mathcal{L}_3^p\setminus\mathcal{L}_1^p$]\label{lemma:L3}
 If a solution at a complex time $t$ is sufficiently close to  $\mathcal{L}_3^p\setminus\mathcal{L}_1^p$, then there exists unique $\tau\in\mathbf{C}$ such that $(y(\tau),z(\tau))$ belongs to the line $\mathcal{L}_7$.
  In other words, the solution has a pole at $t=\tau$.
  
Moreover
$|t-\tau|=O(|d(t)||y_{82}(t)|)$
 for sufficiently small $d(t)$ and bounded $|y_{82}|$.
 
 For large $R_3>0$, consider the set $\{ t\in\mathbf{C} \mid |y_{82}|\le R_3\}$.
 Its connected component containing $\tau$ is an approximate disk $D_3$ with centre $\tau$ and radius $|d(\tau)|R_3$,
 and $t\mapsto y_{82}(t)$ is a complex analytic diffeomorphism from that approximate disk onto $\{y\in\mathbf{C}\mid|y|\le R_3\}$.
 \end{lemma}
 
 \begin{proof}
For the study of the solutions near $\mathcal{L}_3^p\setminus\mathcal{L}_1^p$, we use coordinates $(y_{82},z_{82})$, see Section \ref{a7-blow}.
In this chart, the line $\mathcal{L}_3^p\setminus\mathcal{L}_1^p$ is given by the equation $z_{82}=0$ and parametrised by $y_{82}\in\mathbf{C}$.
Moreover, $\mathcal{L}_7^p$ is given by $y_{82}=0$ and parametrised by $z_{82}\in\mathbf{C}$.
 
Asymptotically, for $z_{82}\to0$ and bounded $y_{82}$, $e^{-t}$, we have:
\begin{subequations}
\begin{align}
y_{82}' &\sim\frac1{z_{82}},\label{eq:y82'}
\\
z_{82}' &\sim4y_{82}^3z_{82}^2,\label{eq:z82'}
\\
J_{82} &=-z_{82},\label{eq:J82}
\\
\frac{J'_{82}}{J_{82}} & = \eta-2\theta_0-\theta_1e^t-4y_{82}+O(z_{82})=\eta-2\theta_0-\theta_1e^t-4y_{82}+O(J_{82}),\label{eq:J'/J82}
\\
EJ_{82} & \sim -1.\label{eq:EJ82}
\end{align}
\end{subequations}
%From the coordinate change $(y,z)\to(y_{82},z_{82})$ in the chart (see Section \ref{a7-blow}) and the fact that $y$ and $z$ approach respectively $0$ and the infinity near $\mathcal{L}_3^p\setminus\mathcal{L}_1^p$, we conclude from (\ref{eq:z82'}) that $z_{82}'\sim0$ near $\mathcal{L}_3$, i.e.~that $z_{82}$ is approximately a small constant.
%Then (\ref{eq:y82'}) implies that $y_{82}\sim Ct$, where C is a large constant.

Integrating (\ref{eq:J'/J82}) from $\tau$ to $t$, we get
\begin{gather*}
J_{82}(t)=J_{82}(\tau)e^{K(t-\tau)}e^{-\theta_1(e^t-e^{\tau})}(1+o(1)),
\\
K=\eta-2\theta_0-4y_{82}(\tilde{\tau}),
\end{gather*}
where $\tilde{\tau}$ is on the integration path.

Because of (\ref{eq:z82'}), $z_{82}$ is approximately equal to a small constant, and from (\ref{eq:y82'}) follows that:
$$
y_{82}\sim y_{82}(\tau)+\frac{t-\tau}{z_{82}}.
$$
Thus, if $t$ runs over an approximate disk $D$ centered at $\tau$ with radius $|z_{82}|R$, then $y_{82}$ fills and approximate disk centered at $y_{82}(\tau)$ with radius $R$.
Therefore, if $z_{82}(\tau)\ll\tau$, the solution has the following properties for $t\in D$:
$$
\frac{z_{82}(t)}{z_{82}(\tau)}\sim1,
 $$ 
 and $y_{82}$ is a complex analytic diffeomorphism from $D$ onto an approximate disk with centre $y_{82}(\tau)$ and radius $R$.
 If $R$ is sufficiently large, we will have $0\in y_{82}(D)$, i.e.~the solution of the Painlev\'e equation will have a pole at a unique point in $D$.
 
 Now, it is possible to take $\tau$ to be the pole point.
 For $|t-\tau|\ll|\tau|$, we have:
 \begin{gather*}
 \frac{d(t)}{d(\tau)}\sim1,
 \quad\text{i.e.}\quad
 \frac{z_{82}(t)}{d(\tau)}\sim-\frac{J_{82}(t)}{d(\tau)}\sim1,
 \\
 y_{82}(t)\sim\frac{t-\tau}{z_{82}}\sim\frac{t-\tau}{d(\tau)}.
 \end{gather*}
 Let $R_3$ be a large positive real number.
 Then the equation $|y_{82}(t)|=R_3$ corresponds to $|t-\tau|\sim|d(\tau)|R_3$, which is still small compared to $|\tau|$ if $|d(\tau)|$ is sufficiently small.
 Denote by $D_3$ the connected component of the set of all $t\in\mathbf{C}$ such that $\{t\mid |y_{82}(t)|\le R_3\}$ is an approximate disk with centre $\tau$ and radius $2|d(\tau)|R_3$.
More precisely, $y_{82}$ is a complex analytic diffeomorphism from $D_3$ onto $\{y\in\mathbf{C}\mid|y|\le R_3\}$, and
$$
\frac{d(t)}{d(\tau)}\sim1
\quad\text{for all}\quad
t\in D_3.
$$
We have $E(t)J_{82}(t)\sim-1$ when $|z_{82}|\ll1$.
Thus $E(t)J_{82}(t)\sim-1$ for the annular disk $t\in D_3\setminus D_3'$, where $D_3'$ is a disk centered at $\tau$ with small radius compared to radius of $D_3$.
\end{proof}

\begin{lemma}[Behaviour near $\mathcal{L}_9^p\setminus\mathcal{L}_8^p$]\label{lemma:L9}
 If a solution at a complex time $t$ is sufficiently close to  $\mathcal{L}_9^p\setminus\mathcal{L}_8^p$, then there exists unique $\tau\in\mathbf{C}$ such that $(y(\tau),z(\tau))$ belongs to the line $\mathcal{L}_{10}$.
  In other words, the solution has a pole at $t=\tau$.
  
Moreover
$|t-\tau|=O(|d(t)||y_{112}(t)|)$
 for sufficiently small $d(t)$ and bounded $|y_{112}|$.
 
 For large $R_9>0$, consider the set $\{ t\in\mathbf{C} \mid |y_{112}|\le R_9\}$.
 Its connected component containing $\tau$ is an approximate disk $D_9$ with centre $\tau$ and radius $|d(\tau)|R_9$,
 and $t\mapsto y_{112}(t)$ is a complex analytic diffeomorphism from that approximate disk onto $\{y\in\mathbf{C}\mid|y|\le R_9\}$.
 \end{lemma}
 
 \begin{proof}
For the study of the solutions near $\mathcal{L}_9^p\setminus\mathcal{L}_8^p$, we use coordinates $(y_{112},z_{112})$, see Section \ref{a10-blow}.
In this chart, the line $\mathcal{L}_9^p\setminus\mathcal{L}_8^p$ is given by the equation $z_{112}=0$ and parametrised by $y_{112}\in\mathbf{C}$.
Moreover, $\mathcal{L}_{10}^p$ is given by $y_{112}=0$ and parametrised by $z_{112}\in\mathbf{C}$.
 
Asymptotically, for $z_{112}\to0$ and bounded $y_{112}$, $e^{-t}$, we have:
\begin{subequations}
\begin{align}
y_{112}' &\sim\frac{1}{z_{112}},\label{eq:y112'}
\\
z_{112}' &\sim-3(1+\eta)z_{112}-\frac{2}{\theta_1 e^t}y_{112}z_{112},\label{eq:z112'}
\\
J_{112} &\sim-\theta_1 e^t z_{112},\label{eq:J112}
\\
\frac{J'_{112}}{J_{112}} & = -2-\eta+\theta_1e^t+O(z_{112})=-2-\eta+\theta_1e^t+O(J_{112}),\label{eq:J'/J112}
\\
\frac{J_{112}}{J_{102}} & = 1+\frac{(1+\eta)\theta_1 e^t}{y_{112}}.\label{eq:J112/J102}
\end{align}
\end{subequations}
%From the coordinate change $(y,z)\to(y_{82},z_{82})$ in the chart (see Section \ref{a7-blow}) and the fact that $y$ and $z$ approach respectively $0$ and the infinity near $\mathcal{L}_3^p\setminus\mathcal{L}_1^p$, we conclude from (\ref{eq:z82'}) that $z_{82}'\sim0$ near $\mathcal{L}_3$, i.e.~that $z_{82}$ is approximately a small constant.
%Then (\ref{eq:y82'}) implies that $y_{82}\sim Ct$, where C is a large constant.

Integrating (\ref{eq:J'/J112}) from $\tau$ to $t$, we get
\begin{gather*}
J_{112}(t)=J_{112}(\tau)e^{-(2+\eta)(t-\tau)}e^{\theta_1(e^t-e^{\tau})}(1+o(1)).
\end{gather*}

Because of (\ref{eq:J112}), $z_{112}$ is approximately equal to a small constant, and from (\ref{eq:y112'}) follows that:
$$
y_{112}\sim y_{112}(\tau)+\frac{t-\tau}{z_{112}}.
$$
Thus, if $t$ runs over an approximate disk $D$ centered at $\tau$ with radius $|z_{112}|R$, then $y_{112}$ fills and approximate disk centered at $y_{112}(\tau)$ with radius $R$.
Therefore, if $z_{112}(\tau)\ll\tau$, the solution has the following properties for $t\in D$:
$$
\frac{z_{112}(t)}{z_{112}(\tau)}\sim1,
 $$ 
 and $y_{112}$ is a complex analytic diffeomorphism from $D$ onto an approximate disk with centre $y_{112}(\tau)$ and radius $R$.
 If $R$ is sufficiently large, we will have $0\in y_{112}(D)$, i.e.~the solution of the Painlev\'e equation will have a pole at a unique point in $D$.
 
 Now, it is possible to take $\tau$ to be the pole point.
 For $|t-\tau|\ll|\tau|$, we have:
 \begin{gather*}
 \frac{d(t)}{d(\tau)}\sim1,
 \quad\text{i.e.}\quad
 \frac{z_{112}(t)}{d(\tau)}\sim-\frac{1}{\theta_1 e^t}\cdot\frac{J_{112}(t)}{d(\tau)}\sim\frac{1}{\theta_1 e^t},
 \\
 y_{112}(t)\sim\frac{t-\tau}{z_{112}}\sim\frac{t-\tau}{d(\tau)}\cdot \theta_1e^t.
 \end{gather*}
 Let $R_9$ be a large positive real number.
 Then the equation $|y_{112}(t)|=R_9$ corresponds to $|(t-\tau)\theta_1 e^t|\sim|d(\tau)|R_3$, which is still small compared to $|\tau|$ if $|d(\tau)|$ is sufficiently small.
 Denote by $D_9$ the connected component of the set of all $t\in\mathbf{C}$ such that $\{t\mid |y_{112}(t)|\le R_9\}$ is an approximate disk with centre $\tau$ and radius $2|d(\tau)|R_3$.
More precisely, $y_{112}$ is a complex analytic diffeomorphism from $D_9$ onto $\{y\in\mathbf{C}\mid|y|\le R_9\}$, and
$$
\frac{d(t)}{d(\tau)}\sim1
\quad\text{for all}\quad
t\in D_9.
$$
From (\ref{eq:J112/J102}), we have:
$$
\frac{J_{112}}{J_{102}} \sim 1
\quad
\text{when}
\quad
1
\gg
\left|\frac{(1+\eta)\theta_1 e^t}{y_{112}(t)}\right|
\sim
\left| \frac{(1+\eta) d(\tau)}{t-\tau}   \right|,
$$
that is, when
$$
|t-\tau|\gg|d(\tau)|.
$$
Since $R_9\gg1$, we have
$$
|t-\tau|\sim|d(\tau)|R_9\gg|d(\tau)|.
$$
Thus $\dfrac{J_{112}}{J_{102}}\sim1$ for the annular disk $t\in D_9\setminus D_9'$, where $D_9'$ is a disk centered at $\tau$ with small radius compared to the radius of $D_9$.
\end{proof}

\begin{lemma}[Behaviour near $\mathcal{L}_8^p\setminus\mathcal{L}_1^p$]\label{lemma:L8}
For large finite $R_8>0$, consider the set of all $t\in\mathbf{C}$, such that the solution at complex time $t$ is close to 
$\mathcal{L}_8^p\setminus\mathcal{L}_1^p$, with $|y_{102}(t)|\le R_8$, but not close to $\mathcal{L}_9^p$.
Then this set is the complement of $D_9$ in an approximate disk $D_8$ with centre $\tau$ and radius $\sim\sqrt{d(\tau)}R_8$.
More precisely, $t\mapsto y_{102}$ defines a covering from the annular domain $D_8\setminus D_9$ onto the complement in
$\{ z\in\mathbf{C}\mid |z|\le R_8\}$ of an approximate disk with centre at the origin and small radius $\sim |d(\tau)|R_9$, where
$y_{102}(t)\sim d(\tau)^{-1/2}(t-\tau)$.
 \end{lemma}
 
 \begin{proof}
 Set $\mathcal{L}_8^p\setminus\mathcal{L}_1^p$ is visible in the chart $(y_{102},z_{102})$, where it is given by the equation $z_{102}=0$ and parametrized by $y_{102}\in\mathbf{C}$, see Section \ref{a9-blow}.
 In that chart, the line $\mathcal{L}_9^p$ (without one point) is given by the equation $y_{102}=0$ and parametrized by $z_{102}\in\mathbf{C}$.
 The line $\mathcal{L}_4^p$ (without one point) is given by the equation $y_{102}=-\theta_1 e^t$ and also parametrized by $z_{102}\in\mathbf{C}$.
 
For $z_{102}\to0$, bounded $e^t$, and $y_{102}$ bounded and bounded away from $-\theta_1 e^t$, we have:
\begin{subequations}
\begin{align}
z_{102}' &\sim-\frac{1}{y_{102}}-\frac{1}{\theta_1e^t+y_{102}}-\theta_1e^t-2y_{102},\label{eq:z102'}
\\
y_{102}' &\sim\frac{2}{z_{102}},\label{eq:y102'}
\\
J_{102} &=-y_{102}(\theta_1 e^t +y_{102}) z_{102}^2,\label{eq:J102}
\\
\frac{J'_{102}}{J_{102}} & \sim \frac{(1+\eta)\theta_1 e^t}{y_{102}}+\theta_1 e^t+2y_{102},\label{eq:J'/J102}
\\
E J_{102} & \sim 1-\frac{\theta_1 e^t}{y_{102}}.\label{eq:EJ102}
%\\
%\frac{E'}{E} & \sim -\frac{2\theta_1 e^t}{y_{102}z_{102}(\theta_1 e^t+y_{102})}.
\end{align}
\end{subequations}

%From (\ref{eq:J'/J102}) and (\ref{eq:z102'}), we have:
%$$\frac{J'_{102}}{J_{102}} \sim -(1+\eta)(\theta_1e^t+y_{102})z_{102}'-2-(\theta_1e^t+2y_{102})(\theta_1e^t+y_{102}-1).$$

From (\ref{eq:J'/J102}):
\begin{gather*}
\log\frac{J_{102}(t_1)}{J_{102}(t_0)}
\sim
\theta_1(e^{t_1}-e^{t_0})+(t_1-t_0)K,
\\
K=\frac{(1+\eta)\theta_1 e^{\tilde\tau}}{y_{102}(\tilde\tau)}+2y_{102}(\tilde\tau),
\end{gather*}
where $\tilde\tau$ is on the integration path.

Therefore $J_{102}(t_1)/J_{102}(t_0)\sim1$, if for all $t$ on the segment from $t_0$ to $t_1$ we have $|t-t_0|\ll|t_0|$ and
$$
\left|\frac{\theta_1 e^t}{y_{102}(t)}\right|\ll\frac1{|t_0|},
\qquad
|y_{102}(t)|\ll\frac1{|t_0|}.
$$
We choose $t_0$ on the boundary of $D_9$ from Lemma \ref{lemma:L9}.
Then we have
$$
\frac{d(\tau)}{d(t_0)}\sim \frac{J_{112}(\tau)}{J_{102}(t_0)}\sim1
\quad\text{and}\quad
|y_{112}(t_0)|=R_9,
$$
which implies that
$$
|z_{102}|=\left|\frac{1}{y_{112}+(1+\eta)\theta_1 e^t}\right|\sim\frac1{R_9}\ll 1.
$$


Since $D_9$ is an approximate disk with centre $\tau$ and small radius $\sim|d(\tau)|R_9$, and $R_9\gg|\tau|^{-1}$, we have that
$|y_{112}(t)|\ge R_9\gg 1$ hence:
$$
|y_{102}|\ll1
\quad\text{if}\quad
t=\tau+r(t_0-\tau),
\ r\ge1,
$$
and
$$
\frac{|t-t_0|}{|t_0|}=(r-1)\left|1-\frac{\tau}{t_0}\right|\ll1
\quad\text{if}\quad
r-1\ll\frac{1}{|1-\frac{\tau}{t_0}|}.
$$

Then equation (\ref{eq:J102}) and $J_{102}\sim-d(\tau)$ yield
$$
z_{102}^{-1}
\sim
\left(\frac{y_{102}(\theta_1e^t+y_{102})}{d(\tau)}\right)^{1/2}
\sim
\left(\frac{y_{102}^2}{d(\tau)}\right)^{1/2},
$$
which in combination with (\ref{eq:z102'}) leads to
$$
\frac{d y_{102}}{dt}\sim d(\tau)^{-1/2}.
$$
Hence
$$
y_{102}(t)\sim y_{102}(t_0)+d(\tau)^{-1/2}(t-t_0),
$$
and therefore
$$
y_{102}(t) \sim d(\tau)^{-1/2}(t-t_0)
\quad\text{if}\quad
|t-t_0|\gg|y_{102}(t_0)|.
$$
For large finite $R_8>0$, the equation $|y_{102}|=R_8$ corresponds to $|t-t_0|\sim\sqrt{d(\tau)}R_8$, which is still small compared to 
$|t_0|\sim|\tau|$, therefore $|t-\tau|\le|t-t_0|+|t_0-\tau|\le|\tau|$.
This proves the statement of the lemma.
\end{proof}

\begin{lemma}[Behaviour near $\mathcal{L}_4^p$]\label{lemma:L4}
For large finite $R_4>0$, consider the set of all $t\in\mathbf{C}$, such that the solution at complex time $t$ is close to 
$\mathcal{L}_4^p$, with $|z_{91}(t)|\le R_4$, but not close to $\mathcal{L}_8^p$.
Then this set is the complement of $D_8$ in an approximate disk $D_4$ with centre $\tau$ and radius $\sim R_4/d(\tau)$.
More precisely, $t\mapsto z_{91}$ defines a covering from the annular domain $D_4\setminus D_8$ onto the complement in
$\{ z\in\mathbf{C}\mid |z|\le R_4\}$ of an approximate disk with centre at the origin and small radius $\sim |d(\tau)|/R_8$, where
$z_{91}(t)\sim d(\tau)/(t-\tau)$.
\end{lemma}
\begin{proof}
$\mathcal{L}_4^p$ without one point is visible in the chart $(y_{91},z_{91})$, where it is given by the equation $y_{91}=0$ and parametrized by $z_{91}\in\mathbf{C}$, see Section \ref{a8-blow}.
In that chart, the line $\mathcal{L}_1$ is not visible, while $\mathcal{L}_8$ is given by the equation $z_{91}=0$.

For $y_{91}\to0$ and bounded $z_{91}$ and $e^t$, we have:
\begin{subequations}
\begin{align}
y_{91}' &\sim-\frac{2\theta_1 e^t}{z_{91}},\label{eq:y91'}
\\
z_{91}' &\sim\frac{\theta_1 e^t}{y_{91}},\label{eq:z91'}
\\
J_{102}& = \frac{y_{91}z_{91}^2}{\theta_1 e^t-y_{91}},\label{eq:J91}
\\
\frac{J_{102}'}{J_{102}} & \sim -1-\eta -\theta_1 e^t
.\label{eq:J'/J91}
\end{align}
\end{subequations}
From (\ref{eq:J'/J91}):
$$
\log\frac{J_{102}(t_1)}{J_{102}(t_0)}\sim-(1+\eta)(t_1-t_0)-\theta_1(e^{t_1}-e^{t_0}).
$$
Therefore $J_{102}(t_1)/J_{102}(t_0)\sim1$ if for all $t$ on the segment from $t_0$ to $t_1$ we have $|t-t_0|\ll t_0$.
We choose $t_0$ on the boundary of $D_8$ from Lemma \ref{lemma:L8}.
Then we have
$$
\frac{d(\tau)}{d(t_0)}\sim\frac{J_{102}(\tau)}{J_{102}(t_0)}\sim1
\quad\text{and}\quad
|y_{102}(t_0)|=R_8,
$$
which implies that
$$
|\theta_1e^t|=|y_{91}-y_{102}|\sim R_8\gg1.
$$
Hence:
$$
|y_{91}|\ll 1
\quad\text{if}\quad
|\theta_1e^t|\sim R_8.
$$
Then equation (\ref{eq:J'/J91}) and $J_{102}\sim-d(\tau)$ yield
$$
y_{91}^{-1}\sim-\frac{z_{91}^2}{\theta_1e^t d(\tau)},
$$
then, using (\ref{eq:z91'}) we get:
$$
\frac{d(z_{91}^{-1})}{dt}\sim\frac1{d(\tau)}.
$$
It follows that
$$
z_{91}^{-1}\sim z_{91}(t_0)^{-1}+\frac{t-t_0}{d(\tau)},
$$
and therefore
$$
z_{91}\sim\frac{d(\tau)}{t-t_0}
\quad\text{if}\quad
|t-t_0|\gg|z_{91}(t_0)^{-1}|.
$$
For large finite $R_4>0$, the equation $|z_{91}|=R_4$ corresponds to $|t-t_0|\sim d(\tau)/R_4$, which is small compared to 
$|t_0|\sim|\tau|$, and therefore $|t-\tau|\le|t-t_0|+|t_0-\tau|\le|\tau|$.
This proves the statement of the lemma.
\end{proof}

\begin{theorem}\label{th:estimates}
Let $\epsilon_1$, $\epsilon_2$, $\epsilon_3$ be given such that $\epsilon_1>0$, $0<\epsilon_2<|\theta_1|$, $0<\epsilon_3<1$.
Then there exists $\delta_1>0$ such that if $|e^{t_0}|<\epsilon_1$ and $|d(t_0)|<\delta_1$, then:
$$
\rho=\inf\{ r<|e^{t_0}| \ \text{such that}\ |d(t)|<\delta_1\ \text{whenever}\ |e^{t_0}|\ge|e^{t}|\ge r\}
$$
satisfies:
\begin{itemize}
\item[(i)]
$\delta_1\ge|d(t_0)|\left(\rho^{-1}|e^{t_0}|\right)^{\theta_1-\epsilon_2}(1-\epsilon_3)$;
\item[(ii)]
if $|e^{t_0}|\ge|e^{t}|\ge\rho$ then $d(t)=d(t_0)e^{\theta_1(t-t_0)+\epsilon_2(t)}(1+\epsilon_3(t))$;
\item[(iii)]
if $|e^{t}|\le\rho$ then $d(t)\ge\delta_1(1-\epsilon_3)$.
\end{itemize}
\end{theorem}

\begin{proof}
Suppose a solution of the system (\ref{eq:PVlog-system}) is close to the infinity set at times $t_0$ and $t_1$.
It follows from Lemmas \ref{lemma:L3}--\ref{lemma:L4} that for every solution close to $\mathcal{I}$, the set of complex time $t$ such that the solution is not close to $\mathcal{L}_{1}^p\cup\mathcal{L}_{2}^p$ is the union of approximate disks of radius $\sim|d|$.
Hence if the solution is near $\mathcal{I}$ for all complex times $t$ such that $|e^{t_0}|\ge|e^{t}|\ge|e^{t_1}|$, then there exists a path $\mathcal{P}$ from $t_0$ to $t_1$, such that the solution is close to $\mathcal{L}_{1}^p\cup\mathcal{L}_{2}^p$ for all $t\in\mathcal{P}$ and $\mathcal{P}$ is $C^1$-close to the path: $s\mapsto t_1^s t_0^{1-s}$, $s\in[0,1]$.

Then Lemmas \ref{lemma:L2} and \ref{lemma:L1} imply that:
$$
\log\frac{E(t)}{E(t_0)}=-\theta_1(t-t_0)\int_0^1dt+o(1).
$$
Therefore
$$
E(t)=E(t_0)e^{-\theta_1(t-t_0)+o(1)}(1+o(1)),
$$
and
\begin{equation}\label{eq:d(t)d(t0)}
d(t)=d(t_0)e^{\theta_1(t-t_0)+o(1)}(1+o(1)).
\end{equation}
From Lemmas \ref{lemma:L3}--\ref{lemma:L4}, we then have that, as long as the solution is close to $\mathcal{I}$, the ratio of $d$ remains close to $1$.

For the first statement of the theorem, we have:
$$
\delta_1>|d(t)|\ge |d(t_0)| \left| e^{\theta_1(t-t_0)-\epsilon_2}\right|(1-\epsilon_3)
$$
and so
$$
\delta_1\ge\sup_{\{t\mid |d(t)|<\delta_1\}}  |d(t_0)| \left| e^{\theta_1(t-t_0)-\epsilon_2}\right|(1-\epsilon_3).
$$
The second statement follows from (\ref{eq:d(t)d(t0)}), while the third follows by the assumption on $t$.
\end{proof}

In the following corollary, we summarise the results obtained in this section.

\begin{corollary}\label{cor:infinity}
No solution of $(\ref{eq:PVlog-system})$ intersects $\mathcal{I}$.
A solution that approached $\mathcal{I}$ will stay in its vicinity for a limited range of the independent variable $t$.
Moreover, if a solution is sufficiently close to $\mathcal{I}$ at a point $t$, then it will have a pole in a neighbourhood of $t$. 
\end{corollary}

\begin{proof}
The first two statements follow from Theorem \ref{th:estimates}, and the last one from Lemmas \ref{lemma:L2}--\ref{lemma:L4}.
\end{proof}
