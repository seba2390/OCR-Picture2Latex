

The pencil of curves arising from the Hamiltonian  of the autonomous system (\ref{eq:PVlog-system-auto}) is given by the zero set of the one-parameter family of polynomials (parametrised by $c$):
\begin{equation}\label{eq:pencil}
%\begin{split}
h_c(y,z)=y(y-1)^2z^2-(\theta_0+\eta)y^2z+(2\theta_0+\eta)yz-\theta_0z+\frac12\epsilon(\theta_0+\eta-\theta_{\infty})y-c.
%\end{split}
\end{equation}
For each $c$, the curve $h_c(y,z)=0$ is birationally equivalent to
$$
h_{\tilde c}^1(y_1,z_1)=\frac{z_1^2-\left( \theta_0^2 +\tilde{c} y_1  + \theta_{\infty}^2  y_1^2 \right)}{4y_1}=0,
$$
where $\tilde{c}=4c-2\theta_0(\eta+\theta_0)$ and the birational equivalence is given by
$$
y_1=y,
\quad
z_1=2y(y-1)z-(\theta_0+\eta)y+\theta_0.
$$
The family of level curves $h_{\tilde c}^1=0$ forms a pencil of conics.
The singular conics in the pencil are:
\begin{align*}
\tilde{c}&=&2\theta_0\theta_{\infty},
&&(z_1+\theta_{\infty} y_1+\theta_0)(z_1-\theta_{\infty}y_1-\theta_0)=0,
\\
\tilde{c}&=&-2\theta_0\theta_{\infty},
&&(z_1-\theta_{\infty} y_1+\theta_0)(z_1+\theta_{\infty}y_1-\theta_0)=0,
\\
\tilde{c}&=&\infty,
&&
y_1w_1=0,
\end{align*}
where $[y_1:z_1:w_1]$ are homogeneous coordinates.
Moreover, the base points of this pencil are:
$$
[0:\theta_0:1],
\quad
[0:-\theta_0:1],
\quad
[1:\theta_{\infty}:0],
\quad
[1:-\theta_{\infty}:0].
$$

Recalling that the Hamiltonian $E$ of the autonomous system, given by Equation \eqref{eq:E}, has the time derivative 
$$
E'=-\theta_1 e^t\big( z(2\theta_0-(\eta+2\theta_0)y+2y(y-1)z ) +E\big),
$$
we can also search for conditions under which all successive derivatives of $E$ are zero. This happens if and only if
$$
z=0
\quad\text{or}\quad
2\theta_0-(\eta+2\theta_0)y+2y(y-1)z =0.
$$
These cases are investigated in further detail below.

\subsubsection*{Case $z=0$}
From the second equation of (\ref{eq:PVlog-system}) we have $\epsilon(\theta_0+\eta-\theta_{\infty})=0$, i.e.~ $\theta_0+\eta=\pm\theta_{\infty}$.
In this case $z_1=\pm\theta_{\infty}y_1+\theta_0$, and so this case corresponds to lines in the pencil of conics containing the base point $[0:\theta_0:1]$.

The first equation of (\ref{eq:PVlog-system}) is then a Riccati equation:
$$
\frac{dy}{dt}=\mp\theta_{\infty} y^2+(\theta_0\pm\theta_{\infty}-\theta_1 e^t)y-\theta_0,
$$
which is equivalent to 
\begin{equation}\label{eq:riccati1}
x\frac{dy}{dx}=\mp\theta_{\infty} y^2+(\theta_0\pm\theta_{\infty}-\theta_1 x)y-\theta_0.
\end{equation}
For $\theta_{\infty}\theta_1\neq0$, the solutions of this equation can be expressed in terms of the Whittaker functions \cite{NISThandbook}.
Note that $\theta_1\neq0$ is equivalent to $\delta\neq0$, when the $\PV$ can be renormalised to $\delta=-1/2$.
Then, the solutions of (\ref{eq:riccati1}) are given by
$$
y=-\frac{z\phi'(x)}{\theta_{\infty}\phi(x)},
$$
where
$$
\phi(x)=\frac{C_1M_{\kappa,\mu}(\xi)+C_2W_{\kappa,\mu}(\xi)}{\xi^{\kappa}}\, e^{\xi/2},
$$
with $\xi=\pm x$, $\kappa=(\mp\theta_{\infty}-\theta_0+1)/2$, $\mu=\mp\theta_{\infty}+\theta_0$, and $C_1$, $C_2$ being arbitrary constants.

Solutions from this class intersect only the pole lines $\mathcal{L}_5$ and $\mathcal{L}_6$.

\subsubsection*{Case $2\theta_0-(\eta+2\theta_0)y+2y(y-1)z =0$}
The first equation of (\ref{eq:PVlog-system}) is then:
$$
\frac{dy}{dt}=\theta_0 y^2-(2\theta_0+\theta_1 e^t)y+\theta_0,
$$
which is again a Riccati equation that can be solved analogously to the previous case of (\ref{eq:riccati1}).
Since in this case $z_1=\theta_0 y-\theta_0$, the condition on the constants is $\theta_0=\pm\theta_{\infty}$.
The solutions belong to the lines from the pencil of conics that contain the base point $[0:-\theta_0:1]$.

Solutions from this class intersect only the pole line $\mathcal{L}_7$.