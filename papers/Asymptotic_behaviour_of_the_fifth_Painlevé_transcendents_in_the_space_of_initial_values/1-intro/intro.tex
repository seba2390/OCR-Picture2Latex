Following the construction of initial value spaces of Painlev\'e equations by \citeauthor{Okamoto1979} \ycites{Okamoto1979}, and more recent asymptotic analysis of the solutions of the first, second and fourth Painlev\'e equations in such spaces, we investigate the solutions of the fifth Painlev\'e equation:
\begin{equation}\label{eq:PV}
\begin{split}
\PV\ : \ \frac{d^2y}{dx^2}=&\left(\frac{1}{2y}+\frac{1}{y-1}\right) \left(\frac{dy}{dx}\right)^2-\frac{1}{x}\frac{dy}{dx}+\frac{(y-1)^2}{x^2}\left(\alpha y+\frac{\beta}{y}\right)\\
&\qquad+\frac{\gamma y}{x}+\frac{\delta y(y+1)}{y-1},
\end{split}
\end{equation}
in an asymptotic limit in its initial value space. Complete information about the limit sets of transcendental solutions and their behaviours near the infinity set are found. Unlike earlier asymptotic investigations of $\PV$, we do not impose any reality constraints; here $y$ is a function of the complex variable $x$, and $\alpha$, $\beta$, $\gamma$, $\delta$ are given complex constants.

Noting that $\PV$ has (fixed) essential singularities only when the independent variable $x$ takes the values $0$ and $\infty$, we investigate the behaviour of the solutions near $x=0$. A similar analysis can be carried out near $\infty$ and we also include an outline of the main results for this limit.  We show that each solution that is singular at $x=0$ has infinitely many poles and zeroes in every neighbourhood of this point. Similarly, each solution singular at $x=\infty$ has infinitely many poles and, moreover, takes the value $1$ infinitely many times in each neighbourhood of infinity.

The starting point for our analysis is the compactification and regularisation of the initial-value space. To make explicit analytic estimates possible, we calculate detailed information about the Painlev\'e vector field after each resolution (or blow-up) of this space. A similar approach was carried out for the first, second, and fourth Painlev\'e equations respectively by \fullocite{DJ2011}, \fullocite{HJ2014}, and \fullocite{JR2016}. However, the construction of the initial-value spaces in each of these earlier works consisted of exactly nine blow-ups, while in the present paper, we will see that eleven blow-ups are needed, followed by two blow-downs.
The initial-value space is then obtained by removing the set, denoted by $\mathcal{I}$, of points which are not attained by any solution.

The main results obtained in this paper fall into four parts:
\vspace{2pt}
\begin{list}{}
  {\usecounter{enumi}
    \setlength{\parsep}{2pt}
    \setlength{\leftmargin}{12pt}\setlength{\rightmargin}{12pt}
    \setlength{\itemindent=-12pt}
  }

\item {\em Existence of a repeller set:} Theorem \ref{th:estimates} in Section \ref{sec:infinity} shows that $\mathcal I$ is a repeller for the flow. The theorem also provides the range of the independent variable for which a solution may remain in the vicinity of $\mathcal{I}$.
\item {\em Numbers of poles and zeroes:} In Corollary \ref{cor:infinity}, we prove that each solution that is sufficiently close to $\mathcal{I}$ has a pole in a neighbourhood of the corresponding value of the independent variable. Moreover, Theorem \ref{th:zeroespoles} shows that each solution with essential singularity at $x=0$ has infinitely many poles and infinitely many zeroes in each neighbourhood of that point.
\item {\em The complex limit set:} We prove in Theorem \ref{th:limit} that the limit set for each solution is non-empty, compact, connected, and invariant under the flow of the autonomous equation obtained as $x\to0$.
\item {\em Asymptotic behaviour as $x\to\infty$:} We show in Section \ref{sec:x_infinity} that each solution with an essential singularity at $x=\infty$ has infinitely many poles and takes the value unity infinitely many times in each neighbourhood of that point.
\end{list}
\vspace{2pt}

The asymptotic analysis of the fifth Painlev\'e transcendent has been studied by many authors, including
\citeauthor{AK2000} \ycites{AK1997a,AK1997b,AK2000}, \ocites{ZZ2016,BP2012,QS2006}, \citeauthor{LM1999} \ycites{LM1999,LM1999b}, \citeauthor{McCoyTang1986} \ycites{McCoyTang1986,McCoyTang1986b,McCoyTang1986c}, and \ocite{Jimbo1982}.
However, the literature on the asymptotic behaviours of the fifth Painlev\'e transcendent concentrates on behaviours on the real line, often focusing on special behaviours or solutions, while we consider all solution behaviours for $x\in\mathbb C$.
For other mathematical results related to $\PV$, see \cites{BFSVZ2013,Shimomura2011,KO2007,Sasaki2007,Clark2005b,LS2004,GJP2001,GJP2001b},
while for applications in physics see \cites{JMMS1980,Dyson1995,  Schief1994}, and references therein.

This paper is organised as follows.
In Section \ref{sec:space}, we construct and describe the space of the initial values for equation (\ref{eq:PV}), with
complete details of all the necessary calculations provided in Appendix \ref{sec:resolution}.
In Section \ref{sec:special}, we consider the special solutions of $\PV$.
Section \ref{sec:infinity} contains the analysis of the behaviours of solutions near the infinity set in the space of initial values. Results on the complex limit sets of solutions when the independent variable approaches $0$ are provided in Section \ref{sec:limit}.
The behaviours of the fifth Painlev\'e transcendent in the limit $x\to\infty$ are outlined in Section \ref{sec:x_infinity}.
A summary of the notation used in this paper is given in Appendix \ref{sec:notation}.

