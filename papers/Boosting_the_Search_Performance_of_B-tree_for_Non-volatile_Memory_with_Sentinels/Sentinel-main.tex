\documentclass[conference]{IEEEtran}
\IEEEoverridecommandlockouts
\usepackage[noadjust]{cite}
\usepackage{amsmath,amssymb,amsfonts}
\usepackage{algorithm}
\usepackage{algorithmic}
\usepackage{graphicx}
\usepackage{textcomp}
\usepackage{xcolor}
\usepackage{comment}
\usepackage{subfigure}
\usepackage{flushend}

\newcommand{\todo}[1]{\textcolor{red}{\textbf{TODO: }\emph{#1}}}
\renewcommand{\algorithmicrequire}{\textbf{Input:}} 
\renewcommand{\algorithmicensure}{\textbf{Output:}}

\def\BibTeX{{\rm B\kern-.05em{\sc i\kern-.025em b}\kern-.08em
    T\kern-.1667em\lower.7ex\hbox{E}\kern-.125emX}}
\begin{document}

\title{Boosting the Search Performance of B+-tree for Non-volatile Memory with Sentinels}


\author{\IEEEauthorblockN{Chongnan Ye}
\IEEEauthorblockA{\textit{School of Information Science and Technology}\\
\textit{ShanghaiTech University}\\
yechn@shanghaitech.edu.cn
}
\and
\IEEEauthorblockN{Chundong Wang*\thanks{*Corresponding Author}}
\IEEEauthorblockA{\textit{School of Information Science and Technology}\\ 
\textit{ShanghaiTech University}\\
wangchd@shanghaitech.edu.cn
}
}

\maketitle

\begin{abstract}
The next-generation non-volatile memory (NVM) is striding into computer systems as a new tier 
as it incorporates both DRAM's byte-addressability and disk's persistency.
Researchers and practitioners have considered building {\em persistent memory} 
by placing NVM on the memory bus for CPU to directly load and store data.
As a result, cache-friendly data structures have been developed for NVM.
One of them is the prevalent B+-tree. State-of-the-art in-NVM B+-trees
mainly focus on the optimization of write operations (insertion and deletion).
However, search is of vital importance for B+-tree.
Not only search-intensive workloads benefit from an optimized search, but
insertion and deletion also rely on a preceding search operation to proceed.
In this paper, we attentively study a sorted B+-tree node that spans over
contiguous cache lines. 
Such cache lines
exhibit a monotonically increasing trend and searching a target key across them
can be accelerated by estimating a range the key falls into.
To do so, we construct a probing {\em Sentinel Array} in which a sentinel stands for
each cache line of B+-tree node. Checking the Sentinel Array 
avoids scanning unnecessary cache lines and hence
significantly reduces cache misses for a search.
A quantitative evaluation shows that
using Sentinel Arrays boosts the search performance of state-of-the-art in-NVM B+-trees
by up to 48.4\% while the cost of maintaining of Sentinel Array is low.
\end{abstract}

\begin{IEEEkeywords}
Non-volatile memory, B+-tree, Key-value Database, Cache
\end{IEEEkeywords}

% !TEX root = ../arxiv.tex

Unsupervised domain adaptation (UDA) is a variant of semi-supervised learning \cite{blum1998combining}, where the available unlabelled data comes from a different distribution than the annotated dataset \cite{Ben-DavidBCP06}.
A case in point is to exploit synthetic data, where annotation is more accessible compared to the costly labelling of real-world images \cite{RichterVRK16,RosSMVL16}.
Along with some success in addressing UDA for semantic segmentation \cite{TsaiHSS0C18,VuJBCP19,0001S20,ZouYKW18}, the developed methods are growing increasingly sophisticated and often combine style transfer networks, adversarial training or network ensembles \cite{KimB20a,LiYV19,TsaiSSC19,Yang_2020_ECCV}.
This increase in model complexity impedes reproducibility, potentially slowing further progress.

In this work, we propose a UDA framework reaching state-of-the-art segmentation accuracy (measured by the Intersection-over-Union, IoU) without incurring substantial training efforts.
Toward this goal, we adopt a simple semi-supervised approach, \emph{self-training} \cite{ChenWB11,lee2013pseudo,ZouYKW18}, used in recent works only in conjunction with adversarial training or network ensembles \cite{ChoiKK19,KimB20a,Mei_2020_ECCV,Wang_2020_ECCV,0001S20,Zheng_2020_IJCV,ZhengY20}.
By contrast, we use self-training \emph{standalone}.
Compared to previous self-training methods \cite{ChenLCCCZAS20,Li_2020_ECCV,subhani2020learning,ZouYKW18,ZouYLKW19}, our approach also sidesteps the inconvenience of multiple training rounds, as they often require expert intervention between consecutive rounds.
We train our model using co-evolving pseudo labels end-to-end without such need.

\begin{figure}[t]%
    \centering
    \def\svgwidth{\linewidth}
    \input{figures/preview/bars.pdf_tex}
    \caption{\textbf{Results preview.} Unlike much recent work that combines multiple training paradigms, such as adversarial training and style transfer, our approach retains the modest single-round training complexity of self-training, yet improves the state of the art for adapting semantic segmentation by a significant margin.}
    \label{fig:preview}
\end{figure}

Our method leverages the ubiquitous \emph{data augmentation} techniques from fully supervised learning \cite{deeplabv3plus2018,ZhaoSQWJ17}: photometric jitter, flipping and multi-scale cropping.
We enforce \emph{consistency} of the semantic maps produced by the model across these image perturbations.
The following assumption formalises the key premise:

\myparagraph{Assumption 1.}
Let $f: \mathcal{I} \rightarrow \mathcal{M}$ represent a pixelwise mapping from images $\mathcal{I}$ to semantic output $\mathcal{M}$.
Denote $\rho_{\bm{\epsilon}}: \mathcal{I} \rightarrow \mathcal{I}$ a photometric image transform and, similarly, $\tau_{\bm{\epsilon}'}: \mathcal{I} \rightarrow \mathcal{I}$ a spatial similarity transformation, where $\bm{\epsilon},\bm{\epsilon}'\sim p(\cdot)$ are control variables following some pre-defined density (\eg, $p \equiv \mathcal{N}(0, 1)$).
Then, for any image $I \in \mathcal{I}$, $f$ is \emph{invariant} under $\rho_{\bm{\epsilon}}$ and \emph{equivariant} under $\tau_{\bm{\epsilon}'}$, \ie~$f(\rho_{\bm{\epsilon}}(I)) = f(I)$ and $f(\tau_{\bm{\epsilon}'}(I)) = \tau_{\bm{\epsilon}'}(f(I))$.

\smallskip
\noindent Next, we introduce a training framework using a \emph{momentum network} -- a slowly advancing copy of the original model.
The momentum network provides stable, yet recent targets for model updates, as opposed to the fixed supervision in model distillation \cite{Chen0G18,Zheng_2020_IJCV,ZhengY20}.
We also re-visit the problem of long-tail recognition in the context of generating pseudo labels for self-supervision.
In particular, we maintain an \emph{exponentially moving class prior} used to discount the confidence thresholds for those classes with few samples and increase their relative contribution to the training loss.
Our framework is simple to train, adds moderate computational overhead compared to a fully supervised setup, yet sets a new state of the art on established benchmarks (\cf \cref{fig:preview}).

\section{Background}%
\label{s:bg}

The goal of metadata-private messaging systems~\cite{vandenhoof15vuvuzela,
  kwon17atom, alexopoulos17mcmix, tyagi17stadium, angel16unobservable}
  is to allow a pair (or group) of \emph{friends} to exchange bidirectional 
  messages without leaking metadata to any party besides the sender and the 
  recipient.
A pair of users are friends if they have previously shared a secret, either
  out-of-band (e.g., in person at a coffee shop), or in-band with 
  an \emph{add friend} protocol~\cite{lazar16alpenhorn}.
Users in these systems exchange a fixed number of messages with their friends 
  in discrete time epochs called rounds; users participate in every round even 
  if they are idle.
This ensures that an attacker that monitors the network cannot tell when users
  are actively communicating with their friends or starting/stopping 
  conversations.
This also places a bound on the number of active conversations that a user can 
  have at any time; we refer to this as the client's 
  \emph{communication capacity}.

Once a client reaches its communication capacity, it cannot send 
  messages to other friends until it ends an existing conversation.
As a result, clients use a separate \emph{dialing protocol} to 
  coordinate the start and end of conversations.
In a dialing protocol, a client sends a short message (a few bits) 
  to a friend regardless of whether the friend's client has reached its communication 
  capacity.
The dialing message is sufficient to notify a user that one of their friends 
  wishes to communicate, and to agree on a round to start the 
  conversation~\cite{lazar16alpenhorn}.
There are multiple ways in which a client can react to a dialing message.
Some natural choices are:

\begin{myitemize}
\item If the client has not reached its communication capacity, it can 
  automatically accept the call and start a new conversation.
\item The client could prompt the user (similar to calling a friend in Skype), 
  who can choose to accept or reject the call.
\item If at capacity, the client could randomly end an existing conversation 
  to make room for a new one.
\end{myitemize}

Each of these choices is problematic. 
If the client's communication capacity is $1$ (as in some of the existing
  systems~\cite{vandenhoof15vuvuzela, tyagi17stadium}) and the client 
  automatically accepts calls, then any of the client's friends can easily 
  learn when the client is \emph{not} active in a conversation simply by 
  calling.
Leaving the choice to the users is slightly better since the user can choose to 
  ignore or delay accepting some calls, but their choices can still 
  inadvertently lead to intersection attacks.
Ending conversations randomly hurts usability and might still leak information.
The goal of the next section is to formalize the desired properties of 
  the client's answering mechanism.

\section{Designing a Watermark}
\label{sec:taxonomy}

After the quick overview of watermarking schemes in \cref{sec:background}, we now provide more details 
about the watermarking design space. We created a unifying taxonomy under which all previous schemes 
can be expressed. We first discuss the requirements then the building blocks of a text watermark. 
%
%We provide a modular implementation of all schemes, so any of the building blocks can be combined.
%
\cref{fig:design-figure} summarizes the current design space.

\subsection{Requirements}

A useful watermarking scheme must detect watermarked texts, without falsely flagging human-generated text and without impairing the original model's performance.
%
More precisely, we want watermarks to have the following properties.
% \begin{itemize}[leftmargin=\itemlm,itemsep=2pt]
\begin{enumerate}[leftmargin=\itemlm,itemsep=2pt]
    \item \textbf{High Recall}. $\Pr[\mathcal{V}_k(T) = \texttt{True}]$ is large if $T$ is a watermarked text generated using the marking procedure $\mathcal{W}$ and secret key $k$.
    %
    \item \textbf{High Precision}. For a random key $k$, $\Pr[\mathcal{V}_k(\Tilde{T}) = \texttt{False}]$ is large if $\Tilde{T}$ is a human-generated (\emph{non-watermarked}) text.
    %
    \item \textbf{Quality}. The watermarked model should perform similarly to the original model. 
    It should be useful for the same tasks and generate similar quality text.
    %
    \item \textbf{Robustness}. A good watermark should be robust to small changes to the watermarked text (potentially caused by an adversary), 
    meaning if a sample $T$ is watermarked with key $k$, then for any text $\Tilde{T}$ that is semantically close to $T$, $\mathcal{V}_k(\Tilde{T})$ should evaluate to \text{True}.
\end{enumerate}

\noindent
A desireable (but optional) property for watermarks is diversity. 
In some settings, such as creative tasks like story-telling, users might want the model to have the ability to generate 
multiple different outputs in response to the same prompt (so they can select their favorite).
We would like watermarked outputs to preserve this capability.
% \noindent
% In addition to these properties, another desirable property for a watermark is to 
% preserve a model's diversity. Language models tend to have diverse generated text distributions: 
% they are able to generate different responses to a same prompt. This is useful in many settings, 
% such as creative tasks like story telling, so the user can  their favorite output.

% The notion of \emph{undetectability} has been defined in previous work~\citep{christ_undetectable_2023}:
Another useful property is \emph{undetectability}, also called \emph{indistinguishability}:
%
no feasible adversary should be able to distinguish watermarked text from non-watermarked text, without knowledge of the secret key~\citep{christ_undetectable_2023}. 
%
A watermark is considered undetectable if the maximum advantage at distinguishing is very small.
%
This notion is appealing; for instance, undetectability implies that watermarking does not degrade the model's quality.
%
However, we find in practice that undetectability is not necessary and may be overly restrictive:
%
minor changes to the model's output distribution are not always detrimental to its quality.

In this paper we focus on symmetric-key watermarking, where both the watermarking and verification procedures share a secret key.
%
This is most suitable for proprietary language models that served via an API.
%
We imagine that the vendor would watermark all outputs, and also provide a second API to query the verification procedure.
%
Alternatively, one could publish the key, enabling anyone to run the verification procedure.
%
\begin{figure*}
    \begin{center}
    \begin{tikzpicture}
    
    \draw[draw=black] (0,15) rectangle ++(17.5,1) node[pos=0.5, align=center] {\Large{Watermarking Taxonomy}};
    \draw[draw=black] (0,12.75) rectangle ++(8.375,2) node[pos=0.5, align=left] 
    {\\
    \\
    \textbf{Parameters:} Key $k$, Sampling $\mathcal{C}$, Randomness $\mathcal{R}$\\
    \textbf{Inputs:} Probs $\mathcal{D}_n = \{\lambda^n_1,\, \cdots, \lambda^n_d\}$, Tokens $\{T_i\}_{i < n}$\\
    \textbf{Output:} Next token 
    $T_n \leftarrow \mathcal{C}(\mathcal{R}_k( \{T_i\}_{i < n}), \mathcal{D}_n)$};
    \draw[draw=black] (9.125,12.75) rectangle ++(8.375,2) node[pos=0.5, align=left] 
    {\\
    \\
    \textbf{Parameters:} Key $k$, Score $\mathcal{S}$, Threshold $p$\\
    \textbf{Inputs:} Text $T$\\
    \textbf{Output:} Decision $\mathcal{V} \leftarrow \text{P}_{0}\left( \mathcal{S} < \mathcal{S}_k(T)\right) < p$};
    \draw (8.75,13.75) circle (0.25) node {+};
    \draw[draw=none] (0,14.25) rectangle ++(8.375,.5) node[pos=0.5, align=left] {\large{Marking $\mathcal{W}$}};
    \draw[draw=none] (9.125,14.25) rectangle ++(8.375,.5) node[pos=0.5, align=left] {\large{Verification $\mathcal{V}$}};
    
    %%%
    
    \draw[draw=black,dashed] (0,8.75) rectangle ++(17.5,3.75);
    \draw[draw=none] (0,11) rectangle ++(17.5,1.75) node[pos=0.5, align=center] {\large{Randomness Source $\mathcal{R}$}\\
    \textbf{Inputs:} Tokens $\{T_i\}_{i < n}$\,
    \textbf{Output:} Random value $r_n = \mathcal{R}_k(\{T_i\}_{i < n})$};
    \draw[draw=black] (0.25,9) rectangle ++(11.25,2.35) node[pos=0, anchor=south west] {\textbf{Text-dependent.} Hash function $h$. Context length H};
    \draw[draw=black] (0.5,9.6) rectangle ++(10.75,0.625) node[anchor=north west] at (0.5, 10.225) {\textbf{(R2) Min Hash}} node[pos=1, anchor=north east, align=left] {
    $r_n = \text{min} \left( h\left( T_{n-1} \mathbin\Vert k\right), \, \cdots, h\left( T_{n-H} \mathbin\Vert k\right) \right)$\\
    };
    \draw[draw=black] (0.5,10.475) rectangle ++(10.75,0.625) node[anchor=north west] at (0.5, 11.1) {\textbf{(R1) Sliding Window}} node[pos=1, anchor=north east, align=left] {
    $r_n = h\left( T_{n-1} \mathbin\Vert \, \cdots \mathbin\Vert T_{n-H} \mathbin\Vert k\right)$\\
    };
    \draw[draw=black] (11.75,9) rectangle ++(5.5,2.35) node[pos=0, anchor=south west] {\textbf{(R3) Fixed}} node[pos=0.5, align=left] {Key length L. Expand $k$ to\\ pseudo-random sequence $\{r^k_i\}_{i<L}$.\\ 
    $r_n = r^k_{n \text{ (mod L)}}$ \\ \\ };
    
    %%%
    
    \draw[draw=black,dashed] (0,3.25) rectangle ++(17.5,5.25);
    \draw[draw=none] (0,6.85) rectangle ++(17.5,1.75) node[pos=0.5, align=center] {\large{Sampling algorithm $\mathcal{C}$ \& Per-token statistic $s$}\\
    \textbf{Inputs:} Random value $r_n = \mathcal{R}_k( \{T_i\}_{i < n})$, Probabilities $\mathcal{D}_n = \{\lambda^n_1,\, \cdots, \lambda^n_d\}$, Logits $\mathcal{L}_n = \{l^n_1,\,\cdots,l^n_d\}$\\};
    
    %
    
    \draw[draw=black] (11,4.75) rectangle ++(6.25,2.5) node[pos=0, anchor=south west] {\textbf{(C3) Binary}} node[pos=0.5, align=left] {Binary alphabet.\\ 
    $T_n \leftarrow 0$ if $r_n < \lambda^n_0$, else $1$. \\
    $s(T_n, r) = \begin{cases} -\log(r) \text{ if } T_n = 1\\
          -\log(1-r) \text{ if } T_n = 0\\\end{cases} $};
    
    \draw[draw=black] (5,4.75) rectangle ++(5.75,2.5) node[pos=0, anchor=south west] {\textbf{(C2) Inverse Transform}} node[pos=0.5, align=left] 
    {$\pi$ keyed permutation. $\eta$ scaling func.\\
    $T_n \leftarrow \pi_k \left( \min\limits_{ j \leq d } \sum\limits_{i=1}^j \lambda^n_{\pi_k (i)} \geq r_n \right)$ \\
    $s(T_n, r) = | r - \eta \left( \pi^{-1}_k(T_n) \right) | $\\};
    
    \draw[draw=black] (0.25,4.75) rectangle ++(4.5,2.5) node[pos=0, anchor=south west] {\textbf{(C1) Exponential}} node[pos=0.5, align=left] 
    {$h$ keyed hash function. \\
    $T_n \leftarrow \argmax\limits_{i \leq d} \left\{ \frac{\log \left( h_{r_n}\left( i \right) \right)}{\lambda^n_i} \right\}$ \\
    $s(T_n, r) = -\log(1 \! - \! h_r(T_n))$\\};
    
    % 
    
    \draw[draw=black] (0.25,3.5) rectangle ++(17,1) node[pos=0, anchor=south west] {\textbf{(C4) Distribution-shift}} node[pos=0.5, align=right] {Bias $\delta$, Greenlist size $\gamma$. Keyed permutation $\pi$. $T_n$ sampled from $\widetilde{\mathcal{L}}_n = \{l^n_i + \delta \text{ if } \pi_{r_n}(i) < \gamma d \text{ else } l^n_i\, , 1 \leq i \leq d\}$\\
    $s(T_n, r) = 1 \text{ if } \pi_{r}(T_n) < \gamma d \text{ else } 0$};
    
    %%% 
    
    \draw[draw=black,dashed] (0,0) rectangle ++(17.5,3);
    \draw[draw=none] (0,1.75) rectangle ++(17.5,1.25) node[pos=0.5, align=center] {\large{Score $\mathcal{S}$}\\
    \textbf{Inputs:} Per-token statistics $s_{i,j} = s(T_i, r_j)$, where $r_j = \mathcal{R}_k( \{T_l\}_{l < j}))$. \# Tokens $N$.};
    
    % 
    
    \draw[draw=black] (8.15,0.25) rectangle ++(9.1,1.5) node[pos=0, anchor=south west] {\textbf{(S3) Edit Score}}
    
    node[pos=0.5, align=left] {
    $\mathcal{S}_{\text{edit}}^\psi = s^\psi(N,N)$,
    $
        s^\psi (i,j) = \min \begin{cases}
          s^\psi (i-1, j-1) + s_{i,j}\\
          s^\psi (i-1, j) + \psi\\
          s^\psi (i, j-1) + \psi\\
        \end{cases} 
    $};
    \draw[draw=black] (0.25,0.25) rectangle ++(2.6,1.5) node[pos=0, anchor=south west] {\textbf{(S1) Sum Score}} node[pos=0.5, align=left] {$\mathcal{S}_{\text{sum}}\! = \! \sum_{i=1}^N s_{i,i}$ \\};
    \draw[draw=black] (3.1,0.25) rectangle ++(4.8,1.5) node[pos=0, anchor=south west] {\textbf{(S2) Align Score}} node[pos=0.5, align=left] {$\mathcal{S}_{\text{align}} \!= \!\min\limits_{0 \leq j < N} \sum\limits_{i=1}^N s_{i, (i+j) \text{ mod}(N)}$ \\ \\ };
    
    \end{tikzpicture}
    \caption{Watermarking design blocks. There are three main components: randomness source, sampling algorithm (and associated per-token statistics), and score function. Each solid box within each of these three components (dashed) denotes a design choice. The choice for each component is independent and offers different trade-offs.}\label{fig:design-figure}
    \end{center}
    \end{figure*}

\subsection{Watermark Design Space}
\label{sec:watermark-design}

Designing a good watermark is a balancing act.
% 
For instance, replacing every word of the output with [WATERMARK] would achieve high recall but destroy the utility of the model.
%
%Conversely, sampling from the original distribution preserves quality but makes it impossible to watermark. 

Existing proposals have cleverly crafted marking procedures that are meant to preserve quality, provide high precision and recall, and achieve a degree of robustness.
%
Despite their apparent differences, we realized they can all be expressed within a unified framework:

\begin{itemize}[leftmargin=\itemlm,itemsep=2pt]
    \item The marking procedure $\mathcal{W}$ contains a randomness source $\mathcal{R}$ and a sampling algorithm $\mathcal{C}$.
    %
    The randomness source $\mathcal{R}$ produces a (pseudo-random) value $r_n$ for each new token, based on the secret key $k$ and the previous tokens $T_0,\cdots,T_{n-1}$.
    %
    The sampling algorithm $\mathcal{C}$ uses $r_n$ and the model's next token distribution $\mathcal{D}$ to  a token.
    \item The verification procedure $\mathcal{V}$ is a one-tailed significance test that computes a $p$-value for the null hypothesis that the text is not watermarked.
    %
    The procedure compares this $p$-value to a threshold, which enables control over the watermark's precision and recall.
    %
    % This test is done using a \emph{score function} $\mathcal{S}$ based on a per-token variable that depends on the ed sampling algorithm.
    % We call the value of this per-token test statistic $s_n$, which only depends on the random value $r_n$ and the ed token $T_n$: $s_n = s(T_n, r_n)$.
    In particular, we compute a per-token score $s_{n,m} \coloneqq s(T_n, r_m)$ for each token $T_n$ and randomness $r_m$, aggregate them to obtain an overall score $\mathcal{S}$, and compute a $p$-value from this score.
    We consider all overlaps $s_{n,m}$ instead of only $s_{n,n}$ to support scores that consider misaligned randomness and text after perturbation. 
    %the test computes \emph{score function} $\mathcal{S}$ which takes as input per-token test statistics $s_{n,m} \coloneqq s(T_n, r_m)$ for a token $T_n$ and a random value $r_m$, $\forall n,m \in [N]$.
    %
    %$s_{n,m}$ depends on the sampling algorithm (see \cref{fig:design-figure} for examples).
    %
    % \dave{I believe $s(T_n, r_m)$ is incorrect and it should be $s(T_n, r_n)$.  Also I think the score should be $s_n$ rather than $s_{n,m}$.}
    % \jp{Depending on the alignment between the key string and the text, there are times we want to refer to the score for key at position m and token at poistion n (for instance, for both the align and edit scores). I'll add some explanation for this.}
    
\end{itemize}
% \dave{I find the sheer number of fonts inelegant (blackboard bold, mathcal, mathbf, typewritter, italics, bold, etc.). In some places, algorithms are denoted by mathcal (W,V), in other places by mathbf (R,C,S).  I suggest picking one and being consistent.  I prefer mathcal.  Lots of bold feels distracting to my eyes, as does lots of font changes.}
% \jp{I changed a bunch of fonts to make it more consistent, and removed bold fonts}

Next, we show how each scheme we consider falls within this framework, each with its own choices for $\mathcal{R},\mathcal{C},\mathcal{S}$.
%Given this template, previous work introduced their own variants of the building blocks, which we will now detail. 
% \chawin{I would have liked to see a summary of which design choices belong to which paper. Maybe we can add a shorthand notation denoting each paper in \cref{fig:design-figure} or have a separate table.}
% \jp{I agree that's a good idea. A table is probably the right way to represent this.}

\subsubsection{Randomness source $\mathcal{R}$}\label{ssec:randomness}
% \textbf{Randomness source $\mathcal{R}$.}
%
% \chawin{Maybe others?} \jp{Yeah but all the other papers i've seen seem to attribute it to one of these two.}
We distinguish two main ways of generating the random values $r_n$, \emph{text-dependent} (computed as a deterministic function of the prior tokens) vs \emph{fixed} (computed as a function of the token index).
Both approaches use the standard heuristic of applying a keyed function (typically, a PRF) to some data, to produce pseudorandom values that can be treated as effectively random but can also be reproduced by the verification procedure.

\citet{aaronson_watermarking_2022} and \citet{kirchenbauer_watermark_2023}
use text-dependent randomness: $r_n = f\left(T_0,\,\cdots,T_{n-1},k\right)$.
%
This scheme has two parameters: the length of the token context window (which we call the window size H) and the aggregation function $f$.
%
\citet{aaronson_watermarking_2022} proposed using the hash of the concatenation of previous tokens, $f := h\left( T_{n-1} \mathbin\Vert \, \cdots \mathbin\Vert T_{n-H} \mathbin\Vert k\right)$; we call this (R1) sliding window.
%
\citet{kirchenbauer_watermark_2023} used this with a window size of $ H = 1$ and also introduced an alternate aggregation function $f := \text{min} \left( h\left( T_{n-1} \mathbin\Vert k\right), \, \cdots, h\left( T_{n-H} \mathbin\Vert k\right) \right)$.
%
We call this last aggregation function (R2) min hash.
%
While these two schemes propose specific choices of $H$, other values are possible. 
We use \benchmarkname{} to evaluate a range of values of $H$ with each candidate aggregation function.

% \smallskip\noindent\textbf{(R3) Fixed}
\citet{kuditipudi_robust_2023} use fixed randomness:
$r_n = f_k(n)$, where $n$ is the index (position) of the token.
We call this (R3) fixed.
%
In practice, they propose using a fixed string of length $L$ (the key length), which is repeated across the generation.
% r_n = f_k(n \bmod L)$ where $L$ is the key length.
% \dave{I don't think we need this level of detail.  I suggest deleting the preceding sentence.}
% \jp{Since we look at the impact of the key length on generations we still need to introduce the idea that the key is repeated, but I canwrite that in english for it to be more digestable}
%
We test the choice of key length in ~\cref{ssec:param_tuning}
%
In the extreme case where $L=1$ or $H=0$, both sources are identical, as $r_n$ will be the same value for every token. \citet{zhao2023provable} explored this option using the same sampling algorithm as~\citet{kirchenbauer_watermark_2023}.

\label{ssec:binary}
\citet{christ_undetectable_2023} proposed setting a target entropy for the context window instead of fixing a window size.
%
This allows to set a lower bound on the security parameter for the model's undetectability.
%
However, setting a fixed entropy makes for a less efficient detector since all context window lengths must be tried in order to detect a watermark.
%
Furthermore, in practice, provable undetectability is not needed to achieve optimal quality: we chose to keep using a fixed-size window for increased efficiency.

\subsubsection{Sampling algorithm \(\mathcal{C}\)}\label{ssec:sampling}
% \textbf{sampling algorithm $\mathcal{C}$.}
%
\noindent
We now give more details about the four sampling algorithms initially presented in~\cref{tab:marking-algorithms}.

\smallskip\noindent\textbf{(C1) Exponential}.
%
Introduced by \citet{aaronson_watermarking_2022} and also used by \citet{kuditipudi_robust_2023}. It relies on the Gumbel-max trick.
%
Let $\mathcal{D}_n = \left\{\lambda^n_i\,, 1 \leq i \leq d\right\}$ be the distribution of the language model over the next token. %(obtained after passing the logits through a softmax and applying a temperature adjustment).
%
The exponential scheme will select the next token as:
\begin{align}
    T_{n} = \argmax\limits_{i \leq d}\left\{ \frac{\log \left( h_{r_n}\left( i \right) \right)}{\lambda^n_i} \right\}
\end{align}
where $h$ is a keyed hash function using $r_n$ as its key.
%
The per-token variable used in the statistical test is either $s_n = h_{r_n}(T_n)$ or $s_n = -\log \left( 1-h_{r_n}(T_n)\right)$.
%
\citet{aaronson_watermarking_2022} and \citet{kuditipudi_robust_2023} both use the latter quantity.
%
We argue the first variable provides the same results, and unlike the log-based variable, the distribution of watermarked variables can be expressed analytically (see~\cref{app:ssec:pseudorandom-proofs} for more details).
%
We align with previous work and use the $\log$ for \benchmarkname{}.

\smallskip\noindent\textbf{(C2) Inverse transform}.
%
\citet{kuditipudi_robust_2023} introduce inverse transform sampling.
%
They derive a random permutation using the secret key $\pi_k$. The next token is selected as follows:
\begin{align}
    T_{n} = \pi_k \left( \min\limits_{ j \leq d } \sum\limits_{i=1}^j \lambda^n_{\pi_k (i)} \geq r_n \right)
\end{align}
which is the smallest index in the inverse permutation such that the CDF of the next token distribution is at least $r_n$.
%
\citet{kuditipudi_robust_2023} propose to use $s_n = | r_n - \eta \left( \pi^{-1}_k(T_n) \right) |$ as a the test variable, where $\eta$ normalizes the token index to the $[0,1]$ range.
%
% We call this scheme the \textit{inverse transform} scheme.

\smallskip\noindent\textbf{(C3) Binary}.
%
\citet{christ_undetectable_2023} propose a different sampling scheme for binary token alphabets --- however, it can be applied to any model by using a bit encoding of the tokens.
%
In our implementation, we rely on a Huffman encoding of the token set, using frequencies derived from a large corpus of natural text.
%
In this case, the distribution over the next token reduces to a single probability $p_n$ that token ``0'' is ed next, and $1-p$ that ``1'' is ed.
%
The sampling rule s 0 if $r_n < p$, and 1 otherwise. The test variable for this case is $s_n = -\log \left( T_n r_n + (1-T_n) (1-r_n) \right)$.
%
% We call this scheme the \textit{binary} scheme.
%
% At first glance, it can seem like this scheme is identical to the exponential scheme. However, because it uses a binary alphabet, the distribution of the test variable is different for both schemes.
%
% However, we show in Appendix \jp{ref} that this is not the case: the distribution of the test variable is different for both schemes.
% %
% \jp{Maybe I'll remove this if I don't have time to show it.}

\smallskip\noindent\textbf{(C4) Distribution-shift}.
%
\citet{kirchenbauer_watermark_2023} propose the distribution-shift scheme. 
%
It produces a modified distribution $D_n$ from which the next token is sampled.
%
Let $\delta > 0$ and $\gamma \in [0,1]$ be two system parameters, and $d$ be the number of tokens.
%
The scheme constructs a permutation $\pi_{r_n}$, seeded by the random value $r_n$, which is used to define a ``green list,'' containing tokens $T$ such that $\pi_{r_n} (T) < \delta d$. It then adds $\delta$ to green-list logits.
%
This modified distribution is then used by the model to sample the next token. The test variable $s_n$ is a bit equal to ``1'' if $T_n$ is in the green list defined by $\pi_{r_n}$, and ``0'' if not.
%
% We call this scheme the \textit{distribution-shift} scheme.

The advantage of this last scheme over the others is that it preserves the model's diversity: 
for a given key, the model will still generate diverse outputs.
In contrast, for a given secret key and a given prompt, the first three sampling strategies 
will always produce the same result, since the randomness value $r_n$ will be the same.
\citet{kuditipudi_robust_2023} tackles this by randomly offseting the key sequence of 
fixed randomness for each generation. We introcude a skip probability $p$ for the 
same effect on text-dependent randomness. Each token is selected without the marking 
strategy with probability $p$. In the interest of space, we leave a detailed discussion 
of generation diversity in~\cref{app:ssec:diverse}.

Another advantage of the distribution-shift scheme is that it can also be used 
at any temperature, by applying the temperature scaling \emph{after} using the 
scheme to modify the logits. Other models apply temperature before watermarking.

However the distribution-shift scheme is not indistinguishable from the original model, 
as discussed earlier in~\cref{ssec:watermark-design}.

\subsubsection{Score Function $\mathcal{S}$}\label{ssec:score}

% \paragraph{Verification procedure $\mathcal{V}.$}

% The distribution of the per-token test statistic is different for watermarked text and non-watermarked text: this is what makes detection possible. Depending on the scheme, it is either higher or lower on average in the watermarked case. Without loss of generality, we assume it is always lower for this discussion.

To determine whether an $N$-token text is watermarked, we compute a score over per-token statistics.
%
This score is then subject to a one-tailed statistical test where the null hypothesis is that the text is not watermarked.
%
In other words, if its $p$-value is under a fixed threshold, the text is watermarked.
%
Different works propose different scores.

\smallskip\noindent\textbf{(S1) Sum score}.
%
\citet{aaronson_watermarking_2022} and \citet{kirchenbauer_watermark_2023} take the sum of all individual per-token statistics:
\begin{align}
    \mathcal{S}_{\text{sum}}=\sum_{i=1}^N s_i = \sum_{i=1}^N s(T_i, r_i).
\end{align}
%
This score requires the random values $r_i$ and the tokens $T_i$ to be aligned.
%
% \chawin{Maybe this goes into limitation or discussion or appendix}
This is not a problem when using text-dependent randomness, since the random values are directly obtained from the tokens.
%
However, this score is not suited for fixed randomness: removing one token at the start of the text will offset the values of $r_i$ for the rest of the text and remove the watermark.
%
The use of the randomness shift to increase diversity will have the same effect. 

\smallskip\noindent\textbf{(S2) Alignment score}.
Proposed by \citet{kuditipudi_robust_2023}, the alignment score aims to mitigate the misalignment issue mentioned earlier.
% \citet{kuditipudi_robust_2023} proposes two alternative scores to deal with this issue.
%
% In keeping with their work, we name these scores the alignment score and the edit score.
Given the sequence of random values $r_i$ and the sequence of tokens $T_i$, the verification process now computes different versions of the per-token test statistic for each possible overlap of both sequences $s_{i,j} = s(T_i, r_j)$.
%
The alignment score is defined as:
\begin{align}
   \mathcal{S}_{\text{align}}  = \min\limits_{0 \leq j < N} \sum\limits_{i=1}^N s_{i, (i+j) \text{ mod}(N)}
\end{align}

\smallskip\noindent\textbf{(S3) Edit score}.
Similar to the alignment score, \citet{kuditipudi_robust_2023} propose the edit score as an alternative for dealing with the misalignment issue.
%
It comes with an additional parameter $\psi$ and is defined as $\mathcal{S}_{\text{edit}}^\psi = s^\psi(N,N)$, where
\begin{align}
    s^\psi (i,j) &= \min \begin{cases}
      s^\psi (i-1, j-1) + s_{i,j}\\
      s^\psi (i-1, j) + \psi\\
      s^\psi (i, j-1) + \psi\\
    \end{cases} 
\end{align}

In all three cases, the average value of the score for watermarked text will be lower than for non-watermarked text.
%
% In the case of the sum score, we can often derive the distribution of the score under the null hypothesis, allowing us to use a $z$-test to determine if the text is watermarked.
In the case of the sum score, the previous works use the $z$-test on the score to determine whether the text is watermarked, but it is also possible, or even better in certain situations, to use a different statistical test according to \citet{fernandez_three_2023}.
%
When possible, we derive the exact distribution of the scores under the null hypothesis (see \cref{app:ssec:exact_dist}) which is more precise than the $z$-test. When it is not, we rely on an empirical T-test, as proposed by \citet{kuditipudi_robust_2023}
%
% This allows one to compute 

\subsection{Limitations of the Building Blocks}\label{ssec:limit_blocks}

While we design the blocks to be as independent as possible, some combinations of the scheme and specific parameters are obviously sub-optimal.
%
Here, we list a few of these subpar block combinations as a guide for practitioners.
% Even though any of the three scores can be used with any scheme and randomness source, in practice not all combinations are useful.
\begin{itemize}[leftmargin=\itemlm,itemsep=2pt]
    \item The sum score (S1) is not robust for fixed randomness (R3).
    \item The alignment score (S2) does not make sense for the text-dependent randomness (R1, R2) since misalignment is not an issue.
    \item The edit score (S3) has a robustness benefit since it can support local misalignment caused by token insertion, deletion, or swapping. However, using it with text-dependent randomness (R1, R2) only makes sense for a window size of 1: for longer window sizes, swapping, adding, or removing tokens would actually change the random values themselves, and not just misalign them.
    \item Finally, in the corner case when a window size of 0 for the text-dependent randomness (R1, R2) or when a random sequence length of 1 for the fixed randomness (R3), both the alignment score (S2) and the edit score (S3) are unnecessary since all random values are the same and misalignment is not possible.
\end{itemize}

In our experiments (\cref{sec:experiments}), we test all reasonable configurations of the randomness source, 
the sampling protocol, and the verification score, along with their parameters. 
We list the evaluated combinations in~\cref{tab:design_space_combinations}. 
The edit score is too inefficient 
to be run on all configurations, instead we rely on the sum and align scores.
%
We hope to not only fairly compare the prior works but also investigate previously unexplored combinations in the 
design space that can produce a better result.

% \chawin{We need a table or a tree that lists all the combinations we test.}\

\begin{table}[h!]
    \centering
    \caption{Tested combinations in the design space, using notations from~\cref{fig:design-figure}.\\
    We only tested the edit score {\bf S3} on a subset of watermarks.\\
    The distribution of non-watermarked scores is known for \textcolor{orange}{orange} configurations and 
    unknown for \textcolor{blue}{blue} configuration. We rely on empirical T-tests~\cite{kuditipudi_robust_2023} for blue configurations.
    }
    \label{tab:design_space_combinations}
    \normalsize
    \begin{tabular}{|l||c|c|c|c|} 
    \hline
     & \makecell[tc]{{\bf C4}\\{\small Distribution}\\{\small Shift}} & \makecell[tc]{{\bf C1}\\{\small Exponential}} & \makecell[tc]{{\bf C2}\\{\small Binary}} & \makecell[tc]{{\bf C3}\\{\small Inverse}\\{\small Transform}} \\
    \hline
    \hline
    \makecell{{\bf S1}+{\bf R1}}  & \textcolor{orange}{X} & \textcolor{orange}{X} & \textcolor{orange}{X} & \textcolor{blue}{X} \\
    \hline
    \makecell{{\bf S1}+{\bf R2}}  & \textcolor{orange}{X} & \textcolor{orange}{X} & \textcolor{orange}{X} & \textcolor{blue}{X} \\
    \hline
    \makecell{{\bf S2}+{\bf R3}}  & \textcolor{blue}{X} & \textcolor{blue}{X} & \textcolor{blue}{X} & \textcolor{blue}{X} \\
    \hline
    \makecell{{\bf S3}+{\bf R3}}  & \textcolor{blue}{X} &  &  &  \\
    \hline
    \end{tabular}
\end{table}
    

\subsection{Analysis of the edit score.} 
\label{ssec:editscore}
We analyzed the tamper-resistance of the edit score on a subset of watermarks 
(distribution-shift with $\delta=2.5$ at a temperature of 1, for key lengths between 1 and 1024). 
We tried various $\psi$ values between 0 and 1 for the edit distance, and compared the tamper-resistance 
and watermark size of the resulting verification procedures to the align score. 
Using an edit distance does improve tamper-resistance for key lengths under 32, but at a large efficiency cost: 
for key lengths above 8, the edit score size is at least twice that of the align score. 
We do not recommend using an edit score on low entropy models such as Llama-2 chat.



\subsection{Benchmark and Evaluation Setup}
\label{subsec:bench}

\myparagraph{Benchmark}
We have conducted a preliminary evaluation of \tool. The benchmarks used in the 
evaluation are shown in Table~\ref{tab:benchmark}. We used 8 examples 
from different application domains with \neval web assembly files in total. 
The size of benchmark files varies from 2KB to 840KB. We also counted the number of 
\wasm instructions in all files, which manifested a relatively large difference across 
different test cases. The largest one (Zxing) has over 381K instructions, while the smallest 
(Snake) only includes 528 instructions. The benchmarks and results are publicly available at \dataset.

%!TEX ROOT = ../../centralized_vs_distributed.tex

\section{{\titlecap{the centralized-distributed trade-off}}}\label{sec:numerical-results}

\revision{In the previous sections we formulated the optimal control problem for a given controller architecture
(\ie the number of links) parametrized by $ n $
and showed how to compute minimum-variance objective function and the corresponding constraints.
In this section, we present our main result:
%\red{for a ring topology with multiple options for the parameter $ n $},
we solve the optimal control problem for each $ n $ and compare the best achievable closed-loop performance with different control architectures.\footnote{
\revision{Recall that small (large) values of $ n $ mean sparse (dense) architectures.}}
For delays that increase linearly with $n$,
\ie $ f(n) \propto n $, 
we demonstrate that distributed controllers with} {few communication links outperform controllers with larger number of communication links.}

\textcolor{subsectioncolor}{Figure~\ref{fig:cont-time-single-int-opt-var}} shows the steady-state variances
obtained with single-integrator dynamics~\eqref{eq:cont-time-single-int-variance-minimization}
%where we compare the standard multi-parameter design 
%with a simplified version \tcb{that utilizes spatially-constant feedback gains
and the quadratic approximation~\eqref{eq:quadratic-approximation} for \revision{ring topology}
with $ N = 50 $ nodes. % and $ n\in\{1,\dots,10\} $.
%with $ N = 50 $, $ f(n) = n $ and $ \tau_{\textit{min}} = 0.1 $.
%\autoref{fig:cont-time-single-int-err} shows the relative error, defined as
%\begin{equation}\label{eq:relative-error}
%	e \doteq \dfrac{\optvarx-\optvar}{\optvar}
%\end{equation}
%where $ \optvar $ and $ \optvarx $ denote the the optimal and sub-optimal scalar variances, respectively.
%The performance gap is small
%and becomes negligible for large $ n $.
{The best performance is achieved for a sparse architecture with  $ n = 2 $ 
in which each agent communicates with the two closest pairs of neighboring nodes. 
This should be compared and contrasted to nearest-neighbor and all-to-all 
communication topologies which induce higher closed-loop variances. 
Thus, 
the advantage of introducing additional communication links diminishes 
beyond}
{a certain threshold because of communication delays.}

%For a linear increase in the delay,
\textcolor{subsectioncolor}{Figure~\ref{fig:cont-time-double-int-opt-var}} shows that the use of approximation~\eqref{eq:cont-time-double-int-min-var-simplified} with $ \tilde{\gvel}^* = 70 $
identifies nearest-neighbor information exchange as the {near-optimal} architecture for a double-integrator model
with ring topology. 
This can be explained by noting that the variance of the process noise $ n(t) $
in the reduced model~\eqref{eq:x-dynamics-1st-order-approximation}
is proportional to $ \nicefrac{1}{\gvel} $ and thereby to $ \taun $,
according to~\eqref{eq:substitutions-4-normalization},
making the variance scale with the delay.

%\mjmargin{i feel that we need to comment about different results that we obtained for CT and DT double-intergrator dynamics (monotonic deterioration of performance for the former and oscillations for the latter)}
\revision{\textcolor{subsectioncolor}{Figures~\ref{fig:disc-time-single-int-opt-var}--\ref{fig:disc-time-double-int-opt-var}}
show the results obtained by solving the optimal control problem for discrete-time dynamics.
%which exhibit similar trade-offs.
The oscillations about the minimum in~\autoref{fig:disc-time-double-int-opt-var}
are compatible with the investigated \tradeoff~\eqref{eq:trade-off}:
in general, 
the sum of two monotone functions does not have a unique local minimum.
Details about discrete-time systems are deferred to~\autoref{sec:disc-time}.
Interestingly,
double integrators with continuous- (\autoref{fig:cont-time-double-int-opt-var}) ad discrete-time (\autoref{fig:disc-time-double-int-opt-var}) dynamics
exhibits very different trade-off curves,
whereby performance monotonically deteriorates for the former and oscillates for the latter.
While a clear interpretation is difficult because there is no explicit expression of the variance as a function of $ n $,
one possible explanation might be the first-order approximation used to compute gains in the continuous-time case.
%which reinforce our thesis exposed in~\autoref{sec:contribution}.

%\begin{figure}
%	\centering
%	\includegraphics[width=.6\linewidth]{cont-time-double-int-opt-var-n}
%	\caption{Steady-state scalar variance for continuous-time double integrators with $ \taun = 0.1n $.
%		Here, the \tradeoff is optimized by nearest-neighbor interaction.
%	}
%	\label{fig:cont-time-double-int-opt-var-lin}
%\end{figure}
}

\begin{figure}
	\centering
	\begin{minipage}[l]{.5\linewidth}
		\centering
		\includegraphics[width=\linewidth]{random-graph}
	\end{minipage}%
	\begin{minipage}[r]{.5\linewidth}
		\centering
		\includegraphics[width=\linewidth]{disc-time-single-int-random-graph-opt-var}
	\end{minipage}
	\caption{Network topology and its optimal {closed-loop} variance.}
	\label{fig:general-graph}
\end{figure}

Finally,
\autoref{fig:general-graph} shows the optimization results for a random graph topology with discrete-time single integrator agents. % with a linear increase in the delay, $ \taun = n $.
Here, $ n $ denotes the number of communication hops in the ``original" network, shown in~\autoref{fig:general-graph}:
as $ n $ increases, each agent can first communicate with its nearest neighbors,
then with its neighbors' neighbors, and so on. For a control architecture that utilizes different feedback gains for each communication link
	(\ie we only require $ K = K^\top $) we demonstrate that, in this case, two communication hops provide optimal closed-loop performance. % of the system.}

Additional computational experiments performed with different rates $ f(\cdot) $ show that the optimal number of links increases for slower rates: 
for example, 
the optimal number of links is larger for $ f(n) = \sqrt{n} $ than for $ f(n) = n $. 
\revision{These results are not reported because of space limitations.}

\myparagraph{Evaluation Setup}
All the experiments were performed on two environments, \ie, with and without \tee, respectively. 
The \tee environment was set up on Google Cloud confidential computing 
platform\footnote{https://cloud.google.com/confidential-computing}, which uses 
the AMD Epyc processor and SEV-ES\footnote{https://developer.amd.com/sev/} as the underlying TEE setting. 
The cloud machine was configured with dual 2.25GHz cores, 8GB memory and a Ubuntu 18.04 
operating system. Moreover, the non-\tee environment carried Intel i9 processor with dual 2.3GHz 
cores, 8GB memory and a Ubuntu 18.04 operating system. All the computation was executed by only 
one core of the processor in our evaluation.

\subsection{Evaluation Results}
\label{subsec:results}
In the evaluation, we ran \tool with and without the support of TEE (\ie, AMD SEV 
in our case) on all benchmarks. For a test file $f$, \tool performs a systematic 
program analysis based on symbolic execution~\cite{king1976symbolic} to explore the 
state space of $f$ and create a semantic abstraction as well. That said, the execution 
did not include detection of specific types of bugs or errors, as in many existing program 
analyzers. The goal of this evaluation was to understand the performance trade-off with the 
design of \tcpa, rather than assessing the effectiveness of a certain bug-detection algorithm.
With a framework such as \tool, the implementation of a detector is a straightforward task 
even in the context of \tcpa.




\begin{table}[htbp]
\centering
\caption{Time cost. s: second.}\label{tab:time}
\begin{tabular}{c r r r}
\toprule
%\multicolumn{1}{c}{\bf DeFi} & \multicolumn{1}{c}{\bf Codefi Inspect} & \multicolumn{1}{c}{\bf \tool}\\
\textbf{File} & \textbf{TEE (s)} & \textbf{Non-TEE (s)} & \textbf{Overhead} \\
\midrule

module1 & 67.7 & 36.3 & 85.4\% \\

avif\_dec & 1,823.6 & 559.6 & 225.9\% \\

imagequant & 493.0 & 105.4 & 367.5\% \\

mozjpeg\_enc & 659.7 & 338.2 & 95.1\% \\

rotate & 358.4 & 173.1 & 107.0\% \\

zxing & 1,427.5 & 530.4 & 169.3\% \\

tfjs-backend & 712.9 & 661.1 & 7.7\% \\

stdio & 53.5 & 20.4 & 162.3\% \\

string & 26.5 & 10.5 & 152.4\% \\

memory & 117.9 & 46.0 & 156.3\% \\

asm-dom & 382.5 & 117.5 & 225.5\% \\

binjgb & 155.5 & 136.7 & 13.8\% \\

maze & 2.2 & 1.0 & 120.0\%\\

snake & 0.5 & 0.3 & 61.3\%\\

\midrule
average & $\ast$ & $\ast$ & 139.3\% \\

%module1 & 164.5 & 111.7 & 40.2\% \\
%
%avif\_dec & 3,775.7 & 2,067.5 & 82.6\% \\
%
%imagequant & 278.7 & 195.8 & 42.3\% \\
%
%mozjpeg\_enc & 323.4 & 246.1 & 31.3\% \\
%
%rotate & 181.8 & 142.4 & 27.5\% \\
%
%Zxing & 1,687.4 & 903.3 & 86,8\% \\
%
%tfjs-backend & 19,573.3 & 11,079.7 & 76.7\% \\
%
%stdio & 171.2 & 123.8 & 38.3\% \\
%
%asm-dom & 2,791.5 & 2,189.1 & 27.4\% \\
%
%binjgb & 274.0 & 214.2 & 27.9\% \\
%
%maze & 2.1 & 1.8 & 16.7\%\\
%
%snake & 6.8 & 5.9 & 15.3\%\\

\bottomrule
\end{tabular}
\end{table}

\myparagraph{Time Overhead}
The time cost with and without a TEE is described in Table~\ref{tab:time}. 
In the evaluation, we observed the smallest 7.7\% overhead in the case of \texttt{tfjs-backend}, 
while the largest case was 367.5\% for \texttt{imagequant}. The average time overhead was 
139.3\% on all benchmark files. For a subset of the test cases, files with larger sizes introduced 
bigger time overhead as expected. For instance, \texttt{binjgb}, \texttt{module1}, \texttt{avif\_dec} 
led to an increasing level of overhead with a growing size and number of instructions. However, 
there were exceptions in the evaluation where big files manifested small overheads. For example, 
in the case of \texttt{mozjpeg\_enc} (217 KB), running \tool is 95.1\% slower than the non-TEE version. 
For the case of \texttt{rotate} (14 KB) which is only 6.5\% as large as \texttt{mozjpeg\_enc}, the 
overhead was 107.0\% that amounts to a relative 12.5\% growth. Further discussions on root causes of 
the overhead can be found below.

%Specifically, 
%as the size of the benchmark file increased, the overhead introduced by \tool grows 
%accordingly. In our evaluation, the smallest overhead was 15.3\%, while the largest 
%was 86.8\%. The average overhead of \tool was 42.8\%, which is acceptable in practical 
%applications.

\begin{table}[htbp]
\centering
\caption{Memory cost. MB: megabyte.}\label{tab:memory}
\begin{tabular}{c r r r}
\toprule
%\multicolumn{1}{c}{\bf DeFi} & \multicolumn{1}{c}{\bf Codefi Inspect} & \multicolumn{1}{c}{\bf \tool}\\
\textbf{File} & \textbf{TEE (MB)} & \textbf{Non-TEE (MB)} & \textbf{Overhead} \\
\midrule

module1 & 6.4 & 5.7 & 12.3\% \\

avif\_dec & 80.2 & 59.5 & 34.8\% \\

imagequant & 51.2 & 10.8 & 374.1\% \\

mozjpeg\_enc & 78.5 & 75.9 & 3.4\% \\

rotate & 15.7 & 14.8 & 6.1\% \\

zxing & 106.0 & 36.4 & 191.2\% \\

tfjs-backend & 65.4 & 65.6 & -0.3\% \\

stdio & 11.7 & 10.9 & 7.3\% \\

string & 6.1 & 5.7 & 7.0\% \\

memory & 15.2 & 14.2 & 7.0\% \\

asm-dom & 52.3 & 49.4 & 5.9\% \\

binjgb & 20.6 & 14.2 & 45.1 \% \\

maze & 2.8 & 2.0 & 40.0\% \\

snake & 0.8 & 0.6 & 33.3\% \\

\midrule
average & $\ast$ & $\ast$ & 54.8\% \\

\bottomrule
\end{tabular}
\end{table}


\myparagraph{Memory Overhead}
In addition to time cost, we analyzed the memory overhead with \tool in our evaluation. 
Similarly, the analysis was conducted with and without the support of TEE, 
as shown in Table~\ref{tab:memory}. In general, majority of the overheads were below 45\%. 
More specifically, the overheads for half of the test cases were even less than 8\%, which 
we believe is highly acceptable in practical scenarios. On the other hand, there were two cases 
that manifested a 2X and 4X overheads, although the actual memory used were not big, \ie, 10.8MB 
and 36.4MB respectively. The \texttt{tfjs-backend} file was particularly interesting due to the 
fact that \tool consumed less memory with a TEE than the non-TEE version. Detailed explanations 
are given in the following section.


\subsection{Discussions}
\label{subsec:discuss}
We now describe a further discussion on the empirical results with \tool to help understand its 
performance manifested in the evaluation. 

\myparagraph{General Explanation}
First of all, the runtime overheads in general introduced by \tool in our evaluation are easy to understand. 
In the case of time overhead, the execution of \tcpa protocol was encapsulated in a \tee environment, 
which encrypts and decrypts memory accessed by the running program, \ie, in our case the \tool implementation, 
therefore should last longer than running \tcpa without a \tee (139.3\% as described in Table~\ref{tab:time}), 
depending on how efficient the \tee is realized. 
On the other hand, \tool did not manifest a higher level of memory consumption than a non-\tee implementation 
for the majority of test cases used in the evaluation as shown in Table~\ref{tab:memory}, due to the fact that 
encryption and decryption of memory in a \tee are not memory-intensive procedures thus commonly require little 
extra memory in running \tcpa.

\myparagraph{Special Cases}
Despite the general analysis of evaluation results, we did observe that there were exceptions that seemed not to
be consistent with other cases. As shown in Table~\ref{tab:time}, it was much slower for \tool to process 
\texttt{avif\_dec}, \texttt{imagequant} and \texttt{asm-dom} than other test files. The average time overhead for 
the three is 273.0\% which almost doubles the number of total average. As explained above, the time overhead is 
mainly resulted from encryption and decryption of memory used by \tool. More specifically, the overhead is closely 
correlated to the memory complexity of analysis (\eg, the amount of memory used and the frequency to access it) 
adopted in the \tcpa realization, \ie, a symbolic-execution based analysis. Like many other well-designed symbolic 
engines, \tool uses a variety of specific data structures to store intermediate information of program analysis, 
\eg, states of analysis, symbolic contexts, path conditions, \etc Particularly, \tool introduced a graph-based structure to 
separate the modeling of a given program and its symbolic execution process. While the advantage of such design is 
to have better composability via integration with different symbolic execution engines and backend analyzers, it 
inevitably increases the level of memory consumption and access frequency. Moreover, abnormal time overheads were 
partially attributed to the evaluation setting as well. We use an illustrative example in Figure~\ref{fig:setting} 
to explain the cause. 

\begin{figure}[h]
\centering
\includegraphics[width=.85\linewidth]{coverage.pdf}
\caption{\label{fig:setting}An evaluation setting of covering two paths with a specified timeout on each path.}	
\end{figure}

Figure~\ref{fig:setting} describes an evaluation setting to cover two paths in the given program and the exploration 
of each path is bounded within a specified timeout, \eg, one second. Although settings may vary across different 
program analyzers to deal with specific use cases, they commonly share similar fundamental parameters, \eg, level of 
coverage and timeout for SMT solving. In particular, Figure~\ref{fig:setting} demonstrates a scenario where setting 
overhead is introduced. Specifically, a program analyzer without \tee (left) manages to cover the first and second 
paths of a given program and then finishes the analysis without exploring the remaining paths. However, in the case 
on the right, the program analyzer with \tee (right) manages to cover the first path of a given program, fail at 
the second and third due to timeouts of SMT solving, and then cover the last. In such cases, although overheads 
on two visited paths are relatively small, the total overhead becomes much bigger because of unfinished explorations 
on the other two paths.




\begin{table*}[htbp]
\centering
\caption{The preliminary performance validation with simple programs.}\label{tab:validate}
\begin{tabular}{c r r r r r r}
\toprule
%\multicolumn{1}{c}{\bf DeFi} & \multicolumn{1}{c}{\bf Codefi Inspect} & \multicolumn{1}{c}{\bf \tool}\\
\multirow{2}{*}{\textbf{File}} & \multicolumn{3}{c}{\textbf{Time (second)}} & \multicolumn{3}{c}{\textbf{Memory (MB)}} \\

 & \textbf{TEE} & \textbf{Non-TEE} & \textbf{Overhead} & \textbf{TEE} & \textbf{Non-TEE} & \textbf{Overhead}\\
\midrule

self\_addition & 14.2 & 12.5 & 13.6\% & 0.7 & 0.5 & 40.0\%\\

array\_addition & 3.6 & 0.8 & 350.0\% & 389.3 & 389.2 & 0.03\% \\

quick\_sort & 134.4 & 32.4 & 314.8\% & 387.4 & 392.5 & -1.3\%\\

constraint\_addition & 0.4 & 0.3 & 33.3\% & 128.6 & 128.2 & 0.3\%\\

constraint\_division & 5.0 & 2.6 & 92.3\% & 300.1 & 302.8 & -0.9\%\\

\bottomrule
\end{tabular}
\end{table*}

In terms of memory overhead, the evaluation manifested abnormal results as well. Specifically, \texttt{imagequant} 
and \texttt{zxing} introduced a large overhead while \texttt{tfjs-backend} even showed a negative overhead, \ie, 
\tool was faster than the non-\tee version. Commonly, \tcpa does not introduce a high level of memory overhead because 
the encryption process, \eg, AES as used in our case with AMD SEV, often generates ciphertexts with similar sizes as 
plaintexts. However, there might be cases as well where ciphertexts are bigger with specific padding strategies. Another 
factor to potentially affect the measurement of memory overhead is garbage collection in virtual machines. For cases where 
memory consumption is measured right after a garbage collection process, we might have a much smaller number than expected. 
Further investigation on such cases is left as future work.

\myparagraph{Preliminary Validation}
Since the implementation of \tool is non-trivial, we conducted a preliminary validation with a small group of 
simple test cases to justify the root cause analysis as described above. The validation is shown in Table~\ref{tab:validate}.



Specifically, the test cases used in the validation included the following programs:
\begin{itemize}[leftmargin=*]
\item \texttt{self\_addition}: increment a variable $10^7$ times with a specific value

\item \texttt{array\_addition}: add to $10^7$ elements in a given array

\item \texttt{quick\_sort}: quick sort a given list

\item \texttt{constraint\_addition}: a program with an addition constraint for SMT solving

\item \texttt{constraint\_division}: a program with an division constraint for SMT solving
\end{itemize}

As shown in the first two rows of Table~\ref{tab:validate}, \texttt{self\_addition} manifested 
a relatively lower level of time overhead (13.6\%) compared to \texttt{array\_addition} (350.0\%). The gap 
is resulted from different structures of memory accessed by both programs. While 
\texttt{self\_addition} only manipulated a single unit of memory, \texttt{array\_addition} 
is allocated with a consecutive memory space therefore each access to it requires addressing 
with the starting point and the offset. As a result, running \texttt{array\_addition} with \tee 
was much slower than without \tee due to encryption of a more complicated memory. In the case 
of \texttt{self\_addition}, \tee did not slow down too much of the execution. Furthermore, a similar 
explanation can apply to \texttt{quick\_sort}, \ie, the third row of Table~\ref{tab:validate}. Since 
the memory used by a quick sorting algorithm commonly includes a pivot and sub-lists of a given list, 
it introduced a high level of time overhead (314.8\%) as \texttt{array\_addition}. Moreover, the last 
two rows of Table~\ref{tab:validate} demonstrated two cases with simple and complicated path 
constraints for SMT solving, respectively. While \texttt{constraint\_addition} generated a 
constraint with addition, \texttt{constraint\_division} was a division constraint. Therefore, it took 
longer for \tool to solve \texttt{constraint\_division} than \texttt{constraint\_addition}. As 
explained in Figure~\ref{fig:setting}, \tool managed to solve the division constraint without \tee 
but failed with \tee due to timeout (which could be verified based on runtime logs). 
Therefore, the time overhead in the forth row is larger, \ie, 92.3\%. On the 
other hand, the addition constraint can be solved with and without \tee thus did not manifest a large 
overhead in the last row. In terms of memory, all cases introduced slight overheads, which can be 
explained by the fact that the memory encryption process enforced by \tee (\ie, AMD SEV) 
did not require much extra memory space.
% \vspace{-0.5em}
\section{Conclusion}
% \vspace{-0.5em}
Recent advances in multimodal single-cell technology have enabled the simultaneous profiling of the transcriptome alongside other cellular modalities, leading to an increase in the availability of multimodal single-cell data. In this paper, we present \method{}, a multimodal transformer model for single-cell surface protein abundance from gene expression measurements. We combined the data with prior biological interaction knowledge from the STRING database into a richly connected heterogeneous graph and leveraged the transformer architectures to learn an accurate mapping between gene expression and surface protein abundance. Remarkably, \method{} achieves superior and more stable performance than other baselines on both 2021 and 2022 NeurIPS single-cell datasets.

\noindent\textbf{Future Work.}
% Our work is an extension of the model we implemented in the NeurIPS 2022 competition. 
Our framework of multimodal transformers with the cross-modality heterogeneous graph goes far beyond the specific downstream task of modality prediction, and there are lots of potentials to be further explored. Our graph contains three types of nodes. While the cell embeddings are used for predictions, the remaining protein embeddings and gene embeddings may be further interpreted for other tasks. The similarities between proteins may show data-specific protein-protein relationships, while the attention matrix of the gene transformer may help to identify marker genes of each cell type. Additionally, we may achieve gene interaction prediction using the attention mechanism.
% under adequate regulations. 
% We expect \method{} to be capable of much more than just modality prediction. Note that currently, we fuse information from different transformers with message-passing GNNs. 
To extend more on transformers, a potential next step is implementing cross-attention cross-modalities. Ideally, all three types of nodes, namely genes, proteins, and cells, would be jointly modeled using a large transformer that includes specific regulations for each modality. 

% insight of protein and gene embedding (diff task)

% all in one transformer

% \noindent\textbf{Limitations and future work}
% Despite the noticeable performance improvement by utilizing transformers with the cross-modality heterogeneous graph, there are still bottlenecks in the current settings. To begin with, we noticed that the performance variations of all methods are consistently higher in the ``CITE'' dataset compared to the ``GEX2ADT'' dataset. We hypothesized that the increased variability in ``CITE'' was due to both less number of training samples (43k vs. 66k cells) and a significantly more number of testing samples used (28k vs. 1k cells). One straightforward solution to alleviate the high variation issue is to include more training samples, which is not always possible given the training data availability. Nevertheless, publicly available single-cell datasets have been accumulated over the past decades and are still being collected on an ever-increasing scale. Taking advantage of these large-scale atlases is the key to a more stable and well-performing model, as some of the intra-cell variations could be common across different datasets. For example, reference-based methods are commonly used to identify the cell identity of a single cell, or cell-type compositions of a mixture of cells. (other examples for pretrained, e.g., scbert)


%\noindent\textbf{Future work.}
% Our work is an extension of the model we implemented in the NeurIPS 2022 competition. Now our framework of multimodal transformers with the cross-modality heterogeneous graph goes far beyond the specific downstream task of modality prediction, and there are lots of potentials to be further explored. Our graph contains three types of nodes. while the cell embeddings are used for predictions, the remaining protein embeddings and gene embeddings may be further interpreted for other tasks. The similarities between proteins may show data-specific protein-protein relationships, while the attention matrix of the gene transformer may help to identify marker genes of each cell type. Additionally, we may achieve gene interaction prediction using the attention mechanism under adequate regulations. We expect \method{} to be capable of much more than just modality prediction. Note that currently, we fuse information from different transformers with message-passing GNNs. To extend more on transformers, a potential next step is implementing cross-attention cross-modalities. Ideally, all three types of nodes, namely genes, proteins, and cells, would be jointly modeled using a large transformer that includes specific regulations for each modality. The self-attention within each modality would reconstruct the prior interaction network, while the cross-attention between modalities would be supervised by the data observations. Then, The attention matrix will provide insights into all the internal interactions and cross-relationships. With the linearized transformer, this idea would be both practical and versatile.

% \begin{acks}
% This research is supported by the National Science Foundation (NSF) and Johnson \& Johnson.
% \end{acks}

{
\bibliographystyle{IEEEtran}	
\bibliography{sentinel}
}


\end{document}
