\documentclass[conference]{IEEEtran}
\IEEEoverridecommandlockouts
\usepackage[noadjust]{cite}
\usepackage{amsmath,amssymb,amsfonts}
\usepackage{algorithm}
\usepackage{algorithmic}
\usepackage{graphicx}
\usepackage{textcomp}
\usepackage{xcolor}
\usepackage{comment}
\usepackage{subfigure}
\usepackage{flushend}

\newcommand{\todo}[1]{\textcolor{red}{\textbf{TODO: }\emph{#1}}}
\renewcommand{\algorithmicrequire}{\textbf{Input:}} 
\renewcommand{\algorithmicensure}{\textbf{Output:}}

\def\BibTeX{{\rm B\kern-.05em{\sc i\kern-.025em b}\kern-.08em
    T\kern-.1667em\lower.7ex\hbox{E}\kern-.125emX}}
\begin{document}

\title{Boosting the Search Performance of B+-tree for Non-volatile Memory with Sentinels}


\author{\IEEEauthorblockN{Chongnan Ye}
\IEEEauthorblockA{\textit{School of Information Science and Technology}\\
\textit{ShanghaiTech University}\\
yechn@shanghaitech.edu.cn
}
\and
\IEEEauthorblockN{Chundong Wang*\thanks{*Corresponding Author}}
\IEEEauthorblockA{\textit{School of Information Science and Technology}\\ 
\textit{ShanghaiTech University}\\
wangchd@shanghaitech.edu.cn
}
}

\maketitle

\begin{abstract}
The next-generation non-volatile memory (NVM) is striding into computer systems as a new tier 
as it incorporates both DRAM's byte-addressability and disk's persistency.
Researchers and practitioners have considered building {\em persistent memory} 
by placing NVM on the memory bus for CPU to directly load and store data.
As a result, cache-friendly data structures have been developed for NVM.
One of them is the prevalent B+-tree. State-of-the-art in-NVM B+-trees
mainly focus on the optimization of write operations (insertion and deletion).
However, search is of vital importance for B+-tree.
Not only search-intensive workloads benefit from an optimized search, but
insertion and deletion also rely on a preceding search operation to proceed.
In this paper, we attentively study a sorted B+-tree node that spans over
contiguous cache lines. 
Such cache lines
exhibit a monotonically increasing trend and searching a target key across them
can be accelerated by estimating a range the key falls into.
To do so, we construct a probing {\em Sentinel Array} in which a sentinel stands for
each cache line of B+-tree node. Checking the Sentinel Array 
avoids scanning unnecessary cache lines and hence
significantly reduces cache misses for a search.
A quantitative evaluation shows that
using Sentinel Arrays boosts the search performance of state-of-the-art in-NVM B+-trees
by up to 48.4\% while the cost of maintaining of Sentinel Array is low.
\end{abstract}

\begin{IEEEkeywords}
Non-volatile memory, B+-tree, Key-value Database, Cache
\end{IEEEkeywords}

Reinforcement learning has achieved great success in areas such as Game-playing \citep{silver2018general,vinyals2019grandmaster}, robotics \cite{kober2013reinforcement}, large language models \citep{ouyang2022training}, etc.
However, due to safety concerns or physical limitations, in some real-world reinforcement learning problems, we must consider additional constraints that may influence the optimal policy and the learning process \citep{garcia2015comprehensive}.
% For example, a robotic arm must not take actions that may cause harm to itself or the environments.
A standard framework to handle such cases is the constrained Markov Decision Process (CMDP) \citep{altman1999constrained}.
Within the CMDP framework, the agent has to maximize
the expected cumulative reward while
obeying a finite number of constraints, which are usually in the form of expected cumulative cost criteria.

However, we are sometimes concerned with the problem with a continuum of constraints.
For example,
the constraints we meet might be time-evolving or subject to uncertain parameters, which
cannot be formulated as an ordinary CMDP
(see Examples \ref{Example_Time_Evolving} and  \ref{Example_Uncertain}).
In this paper we would study a generalized CMDP  
to address the above problem.  Because the constraints are not only infinite-number but also lie
in a continuous set,
the generalization is not trivial. Fortunately, we find that we can borrow the idea behind semi-infinite programming (SIP) \citep{remez1934determination, hettich1993semi} to deal with the semi-infinite constraints.
Accordingly, we propose \emph{semi-infinitely constrained Markov decision processes} (SICMDPs)
as a novel complement to the ordinary CMDP framework.
%More specifically,  an SICMDP model %, we consider 
%contains a continuum of constraints whereas an ordinary CMDP contains a finite number of constraints. 

%This generalization is natural but not trivial. However, we can brows the idea  
%The idea is quite natural and can be backtracked
%to the practice of extending linear programming to linear semi-infinite programming (LSIP) %\cite{remez1934determination, GobernaLSIO1998}.
%In addition, 
%As a complementary approach to the ordinary CMDP framework, 
%SICMDP can be used to model these problems  which cannot be described by a finite number of constraints
%that are not covered by .
%For example,
%the restrictions we consider can be time-evolving or subject to uncertain parameters
%, thus
%cannot be described by a finite number of constraints but a continuum of constraints 
%(see Examples \ref{Example_Time_Evolving} and  \ref{Example_Uncertain}).

We also present two reinforcement learning algorithms to solve SICMDPs called SI-CRL and SI-CPO, respectively.
SI-CRL is a model-based reinforcement learning algorithm designed for tabular cases, and SI-CPO is a policy optimization algorithm for non-tabular cases.
% and analyze its performance both theoretically and empirically.
The main challenge is that we need to deal with a continuum of constraints, thus reinforcement learning algorithms for ordinary CMDPs do not work anymore.
In SI-CRL, we tackle this difficulty by first transforming the reinforcement learning problem to an equivalent LSIP problem, which can then be solved using methods in the LSIP literature like the dual exchange methods \citep{Hu1990,reemtsen1998numerical}.
In SI-CPO, we resort to the idea of cooperative stochastic approximation developed in \cite{lan2020algorithms, wei2020comirror}.
As far as we know, we are the first to introduce tools from semi-infinitely programming (SIP) into the reinforcement learning community for solving constrained reinforcement learning problems.

% To the best of our knowledge, we are the first to apply tools from semi-infinitely programming (SIP) to solve reinforcement learning problems.
Furthermore, we give theoretical analysis for both SI-CRL and SI-CPO.
We decompose the error of SI-CRL into two parts: the statistical error from approximating the true SICMDP with an offline dataset and the optimization error due to the fact that the solution of the LSIP problem obtained by the dual exchange method is inexact.
On the optimization side, we show that the iteration complexity of SI-CRL is $O\left(\left\{\mathrm{diam}(Y)L\sqrt{|\gS|^2|\gA|m}/\left[(1-\gamma)\epsilon\right]\right\}^m\right)$.
On the statistical side, we show that the sample complexity of SI-CRL is $\widetilde O\left(\frac{|S|^2|A|^2}{\epsilon^2(1-\gamma)^3}\right)$ if the offline dataset is generated by a generative model, and $\widetilde O\left(\frac{|S||A|}{\nu_{\min} \epsilon^2(1-\gamma)^3}\right)$ if the dataset is generated by a probability measure $\nu$ as considered in \cite{chen2019information}.
Here $\widetilde O$ means that all logarithm terms are discarded.
For SI-CPO, things become a little more complicated because other than the statistical error and the optimization error, we also need to consider the function approximation error, which comes from imperfect policy parametrizations.
It is shown if the function approximation error can be controlled to $O(\epsilon)$ order, the iteration complexity of SI-CPO is $\widetilde{O}\left(\frac{1}{\epsilon^2(1-\gamma)^6}\right)$ and the sample complexity of SI-CPO is $\widetilde{O}(\frac{1}{\epsilon^4(1-\gamma)^{10}})$.
Here our iteration complexity bound is equivalent to a typical $\widetilde O(1/\sqrt{T})$ global convergence rate.

We perform a set of numerical experiments to illustrate the SICMDP model and validate our proposed algorithms.
Specifically, we examine two numerical examples, namely the discharge of sewage and ship route planning.
Through the discharge of sewage example, we show the advantage of the SICMDP framework over the CMDP baseline obtained by naive discretization in modeling realistic sequential decision-making problems.
Moreover, we demonstrate the effectiveness of the SI-CRL and SI-CPO algorithms in such tabular environments. 
In the ship route planning example, we illustrate the benefits of the SICMDP framework and the ability of the SI-CPO algorithm to address complex continuous control tasks involving continuous state spaces with modern deep reinforcement learning techniques.

% In summary, our contributions are listed as follows.
% First, we present the SICMDP model, which can be viewed as a generalization of the ordinary CMDP model.
% Second, we propose an algorithm to perform reinforcement learning for SICMDPs, which is called SI-CRL, and we believe that we are the first to apply tools from SIP
% to solve reinforcement learning problems.
% Third, we give a theoretical analysis of SI-CRL and identify both its sample complexity and iteration complexity.
% In addition, we perform numerical experiments to illustrate the SICMDP model and validate the SI-CRL algorithm.
% \{This paragraph can be removed!!! \}






% not yet covered:
% DRAM pricing is flatlining and DDR5 more expensive

\begin{figure}[t!]
  \centering
%\includegraphics[width=.99\columnwidth]{F/sketches/cxl_read_write}
\includegraphics[width=\columnwidth]{F/sketches/cxl_read_write_hcl}
%\vspace{-4mm}
\vminten

\mycaption{fig-cxlreadwrite}{CXL Request Flow
    (\sec\ref{sec-bg})}{\newtxt{CPU cache misses and write-backs to
    addresses mapped to CXL devices are translated to requests on a CXL
    port by the HDM decoder. Intel measures the round-trip port latency
    to be 25ns.}}

\end{figure}




\section{Background}
\label{sec-bg}

{\bf Hypervisor memory management.}
Public cloud workloads are virtualized~\cite{firecracker.nsdi20}.
To maximize performance and minimize overheads, hypervisors perform minimal memory management and rely on
virtualization accelerators to improve I/O performance~\cite{intelvtd.web20,nicpagefault.asplos17,leapio.asplos20,sriov}.
Examples of common accelerators are direct I/O device assignment
(DDA)~\cite{intelvtd.web20,nicpagefault.asplos17} and Single Root I/O
Virtualization (SR-IOV)~\cite{leapio.asplos20,sriov}.
Accelerated networking is enabled by default on AWS and Azure~\cite{awsaccelnet,azureaccelnet}.
As pointed out in prior work, virtualization acceleration requires statically preallocating (or ``pinning'') a VM's entire address space~\cite{nicpagefault.asplos17,tian2020coiommu,yassour2010dma,willmann2008protection,amit2011viommu,ben2010turtles}.

\myparagraph{Memory stranding} Cloud VMs demand a vector of resources
(\eg, CPUs, memory, \etc)
~\cite{resourcecentral.sosp17, hadary2020protean,
googlejobpacking.cluster14, borg.eurosys15}.
Scheduling VMs thus leads to a multi-dimensional bin-packing
problem~\cite{binpackheuristics.web11, tetris.sigcomm14, hadary2020protean, bpbounds.stoc13}
which is complicated by constraints such as spreading VMs across multiple failure
domains.
Consequently, it is difficult to provision servers that closely
match the resource demands of the incoming VM mix.
When the DRAM-to-core ratio of VM arrivals and the server resources
do not match, tight packing becomes more difficult.
%
We define a resource as \emph{stranded} when
it is technically available to be rented to a customer, but is
practically unavailable as some other resource has exhausted. The typical scenario for {\em
memory stranding} is that all cores have been
rented, but there is still memory available in the server.
%Memory stranding is a lower bound on stranding because the availability of one or a few cores
%still leave memory stranded for VMs requiring more cores.  For example,
%even if there were one core still to be rented, we could not rent the
%resources given a workload mix where VMs demand two or more cores.



\myparagraph{Reducing stranding} Multiple
techniques can reduce memory stranding. For example,
oversubscribing cores~\cite{smartharvest.eurosys21,harvestslo.osdi20} enables more memory to be rented.
However, oversubscription only applies to a subset of VMs for performance reasons.
%Oversubscription also requires a few unrented cores and, thus, cannot address memory stranding.
Our measurements at \azure (\sec\ref{sec-strand}) include
clusters that enable oversubscription and still show significant memory stranding.
%An alternative is to reduce the cost of stranded memory by leveraging
%cheaper memory technologies, such as Intel Optane Persistent Memory
%(PMem)~\cite{intelpmem.web21}. Unfortunately, PMem's high latency
%requires performance mitigations such as page
%migrations~\cite{raybuck2021hemem,nimblepage.asplos19,zswap.web20,softfarmem.asplos19,thermostat.%asplos17,optaneval.memsys19,scalepm.fast20} which are not compatible with virtualization acceleration (\sec\ref{sec-des}).
% Additionally, new memory technologies such as PMem are only available
% from a single vendor, which entails sourcing risks for cloud
% providers.

The approach we target is to disaggregate a
portion of memory into a pool that is accessible by multiple
hosts~\cite{resdisagg.osdi16, fastnetdisagg.socc17,
thymesisflow.micro20}. This breaks the fixed hardware
configuration of servers.
By dynamically reassigning memory to different
hosts at different times, we can shift memory resources to where they
are needed, instead of relying on each individual server to be configured
for all cases pessimistically.
Thus, we can provision servers close to the average
DRAM-to-core ratios and tackle deviations via the memory pool.


%-----------------------------------------------------------------------
%\begin{figure}[t!]
\begin{center}
%\includegraphics[width=\columnwidth]{F/stranding/stranding_bars}
\includegraphics[width=\columnwidth]{F/stranding/stranding_bars_hcl}
\vminfifteen
%\vspace{-4mm}
    \mycaption{fig-stranding-bars}{Memory stranding (\sec\ref{sec-strand})}{Stranding increases significantly as
more CPU cores are scheduled.}
\vspace{-1mm}
\vminten
\end{center}
\end{figure}


%-----------------------------------------------------------------------
% NOT USED
%-----------------------------------------------------------------------
\begin{comment}
\begin{figure*}[t]
\begin{center}

\begin{subfigure}[b]{.4\linewidth}
\includegraphics[width=\columnwidth]{F/stranding/stranding_bars}
\caption{Stranding averages and outliers}
\label{fig-stranding-bars}
\end{subfigure}
%
%\begin{subfigure}[b]{.42\linewidth}
%\includegraphics[width=\columnwidth]{F/stranding/stranding_cpu_correlation.png}
%\caption{Hourly Stranding Snapshots.}
%\label{fig-stranding}
%\end{subfigure}
%
%\begin{subfigure}[b]{.29\linewidth}
%\includegraphics[width=\columnwidth]{F/stranding/stranding_bitmap1.png}
%\caption{TBD}
%\label{fig-stranding-bitmap1}
%\end{subfigure}
%
\hspace*{12pt}
\begin{subfigure}[b]{.4\linewidth}
\includegraphics[width=\columnwidth]{F/stranding/stranding_bitmap2.png}
\caption{Stranding over time}
\label{fig-stranding-bitmap2}
\end{subfigure}
\newline
\hspace*{-20pt}
\begin{subfigure}[b]{.4\linewidth}
\includegraphics[width=\columnwidth]{F/stranding/pooling_static}
\caption{Pooling savings}
\label{fig-stranding-pooling}
\end{subfigure}
\hspace*{12pt}
\begin{subfigure}[b]{.38\linewidth}
\includegraphics[width=\columnwidth]{F/stranding/memory_util_cluster}
\caption{Memory usage}
\label{fig-memoryutil}
\end{subfigure}
%  in VMs observed at \azure. Most VMs use only a fraction of their allocated DRAM
%\vminfive
%
% VM trace analysis from \azure shows that over thousands of daily
% samples
\mycaption{fig-stranding-all}{Memory stranding}{(a) Stranding increases significantly as
more CPU cores are scheduled, the error bars indicate the 5\th\ and
95\th\ percentile (outliers in red); (b) stranding dynamics change
over time and can affect a broad set of servers, (c) small pools of
around 32 sockets are sufficient to significantly reduce overall memory
needs; and (d) VMs have a significant amount of frigid
memory.}

%
\vminten
%
\end{center}
\end{figure*}
\end{comment}

\input{fig-stranding2}


\myparagraph{Pooling via CXL}
\newtxt{CXL contains multiple protocols including \texttt{ld/st} memory semantics (CXL.mem) and I/O semantics (CXL.io).
CXL.mem maps device memory to the system address space.
Last-level cache (LLC) misses to CXL memory addresses translate into requests on a CXL port whose reponses bring the missing cachelines (Figure~\ref{fig-cxlreadwrite}).
Similarly, LLC write-backs translate into CXL data writes.
Neither action involves page faults or DMAs.
CXL memory is virtualized using hypervisor page tables and the memory-management unit and is thus compatible with virtualization acceleration.
The CXL.io protocol facilitates device discovery and configuration.
CXL 1.1 targets directly-attached devices, 2.0~\cite{cxl2spec.web20,cxl2whitepaper.web21} adds switch-based pooling, and 3.0~\cite{debendra2022fms,cxl3spec} standardizes switch-less pooling (\S\ref{sec-des}) and higher bandwidth.}

\newtxt{CXL.mem uses PCIe's eletrical interface with custom link and transaction layers for low latency.
%  protocol Muxing at the PHY level (vs higher level of the stack) helps deliver a low latency path for CXL.$Mem traffic. This eliminates the higher latency in the PCIe/ CXL.io path due to the support for variable packet size, ordering rules, access rights checks, etc., conforming the fundamental principle guiding the protocol choice made for CXL specifications
With PCIe 5.0, the bandwidth of a birectional \tms{8}-CXL port at a typical 2:1 read:write-ratio matches a DDR5-4800 channel.
CXL request latencies are largely determined by the CXL port.
Intel measures round-trip CXL port traversals at 25ns~\cite{debendra2022hoti} which, when combined with expected controller-side latencies, leads to an end-to-end overhead of 70ns for CXL reads, compared to NUMA-local DRAM reads.
While FPGA-based prototypes report higher latency~\cite{maruf2022tpp,gouk2022direct}, Intel's measurements match industry-expectations for ASIC-based memory controllers~\cite{maruf2022tpp,debendra2022hoti,cxl3spec}.}
%Latency targets in the upcoming CXL 3.0 specification~\cite[Table 13-2 in \S13]{cxl3spec} are 80ns for reads (Req-DRS) and 40ns for writes (RwD-NDR).}




\section{Scalable Representations for Communication Patterns}
\label{sec:design}

Using the lessons learned in our preliminary studies, along with existing case studies~\cite{isaacs2014combing, Isaacs2016} using idealized unit time, we design a set of strategies for representing communication patterns when there are too many PEs to draw distinct communication lines in Gantt charts. We first describe our design goals. Then, we present our designs. Finally, we discuss initial feedback from experts familiar trace analysis in HPC.


\subsection{Design Goals}

Our goal is to design a representation of communication in execution traces that (1) aids users in recognizing and understanding what communication is occurring in that temporal and logical position in the Gantt chart and (2) is agnostic to the number of processing elements, thereby scaling to larger traces. These goals are derived from usage and scalability limitations noted in prior work~\cite{isaacs2014combing}. 

We limit our focus to scaling in PEs (y-axis) rather than time. Traces are typically explored using a time window, so we focus on that case. Adapting a design or creating a new one for compressed time settings we leave for future work.

Based on our preliminary study (\autoref{sec:prelim}), we chose to focus on offset, ring, and exchange pattern types as stencils require more design consideration even at small scales. 

\subsection{Visualization Design}

Our design process began with open brainstorming on paper, which we include in the supplemental material. We tried a variety of strategies, including linked views and added channels to the traditional Gantt chart encoding rules. However, most of these retained scaling problems, leading us to focus on designs centering on glyphs.

In designing the representation, we considered the saliency of what was to be encoded (e.g., temporal range, pattern type, grouping, stride) and efficacy of available channels, taking into account that the design needs to be incorporated in a Gantt chart. For example, temporal range is set to a horizontal position matching where a pattern would be drawn in a full chart. See supplemental materials for a table containing discussion of channel considerations.

We prioritize the type of pattern before the grouping factor or stride. The rationale is that the pattern type is fixed by the source code while the grouping and stride are often computed from the problem size and number of resources. Therefore, a user will recognize pattern type first before considering other factors. \autoref{fig:abstract_designs} shows the resulting designs.


\begin{figure*}
    \centering
    \begin{subfigure}{0.18\textwidth}
         \centering
         \includegraphics[width=\textwidth]{figures/new-basic-offset.png}
         \caption{Continuous offset pattern}
         \label{fig:noc}
    \end{subfigure}
    \begin{subfigure}{0.18\textwidth}
         \centering
         \includegraphics[width=\textwidth]{figures/new-basic-offset-grouped.png}
         \caption{Grouped offset pattern}
         \label{fig:nog}
    \end{subfigure}
    \begin{subfigure}{0.18\textwidth}
         \centering
         \includegraphics[width=\textwidth]{figures/new-basic-ring.png}
         \caption{Continuous ring pattern}
         \label{fig:nrc}
    \end{subfigure}
    \begin{subfigure}{0.18\textwidth}
         \centering
         \includegraphics[width=\textwidth]{figures/new-basic-ring-grouped.png}
         \caption{Grouped ring pattern}
         \label{fig:nrg}
    \end{subfigure}
    \begin{subfigure}{0.18\textwidth}
         \centering
         \includegraphics[width=\textwidth]{figures/new-basic-exchange.png}
         \caption{Exchange pattern}
         \label{fig:neg}
    \end{subfigure}
    \caption{Examples of our designs for five communication patterns. They are reminiscent of the underlying communication pattern encoding, but not aligned to the underlying chart and agnostic to the number of rows the underlying pattern repeats over. 
    %Note that the angle for our ring pattern is slightly shallower than the angle for offset, this reflects the difference in stride between the two patterns. 
    Grouped representations fill the vertical space to indicate that the repetition continues from the top of row to the bottom.}
    \label{fig:abstract_designs}
\end{figure*}

\vspace{1ex}

\textbf{Encoding Pattern Type.} To encode the pattern type, we started with the overall shape of of the pattern when drawn at small scale with a small stride. Offsets are drawn with angled repeating lines forming a rhombus-like shape. We use a fixed distance between lines and draw as many will fit in the relevant area.

Rings add indicators of the ``wrap-around'' communication. However, unlike fully drawn rings, we only render the protruding segments at the ends of the shape. There are two main rationales for only drawing protruding segments: (1) we want to indicate this is an abstraction and (2) participants in our interviews found the crossing lines difficult to disambiguate. The number of protruding segments is proportional to the stride of a ring. 
%For a ring with a small stride, one segment is added. This increases to a max of four for a very large stride.

Exchange patterns are drawn as a series of symmetrical ``x'' shapes and avoid direct crossings for the same reason as rings. The number of lines in each cross is proportional to stride of the exchange. Short stride exchanges will exchange between only a few PEs, a long stride exchange spans many PEs. Our glyphs approximate this by increasing the number of crossing lines as stride increases.

\vspace{1ex}

\textbf{Encoding Grouping Factor.} To represent grouping, we partition the available area vertically and repeat the pattern type drawing in those partitions. More formally, the encoding rule to show ``grouping" is repetition and alignment on a non-common scale. The number of partitions is determined by the available vertical space in a chart.


\vspace{1ex}

\textbf{Encoding Stride.} We express the notion of stride through the angle of lines used in our pattern types. As people had difficulty with steeply angled lines in our preliminary study, we limit the angles to a range of 15 degrees to 60 degrees. Therefore, these do not match the encoding of a full view. Instead, they hint at the magnitude of distance over which communication is occurring. This allows users to see that there are differences in stride between glyphs, but not necessarily calculate the exact stride visually.

\vspace{1ex}

\textbf{Temporal range.} Rather than show the exact range, we place the glyphs on the x-axis so they are centered in their range. If two structures overlap, they are placed alongside one another. 

\vspace{1ex}

\textbf{Incorporation in Gantt Charts.} These are designed to be used in Gantt charts when exact lines would be too dense to be interpreted. The underlying interval rectangles will still be drawn. The color encoding of these intervals was shown to be a secondary indicator in our preliminary study, so we preserve them. We add a slight blur effect to the background as another signifier that the glyphs are an abstraction and should not be confused for exact lines.



\subsection{Expert Feedback}
\label{sec:expertfeedback}

We sent our designs to two HPC experts for feedback regarding both the designs themselves and the overall approach. Both experts were familiar with idealized unit time representations of traces. The first expert, E1, had previously collaborated on this strategy with the authors but was not involved in any of the work presented here. The second expert, E2, had managed an integration of the strategy into an HPC center's performance tools, referencing the open-source research code~\cite{isaacs2014combing} but using an alternate calculation method and front-end technology.

We sent both experts a short email with a PDF describing the visualizations with comparisons to fully drawn traces and showing how they might be applied in practice, including a few complicated examples such as zoomed-out time and idle processes. (See supplemental materials.) We asked if and how the strategy would be useful and if there were any suggestions or concerns. E1 responded the designs ``definitely look helpful,'' noted the trade off in exactness, and then pointed out figures which led to ambiguities in his view. He also identified a error where the mock-up did not match the underlying trace. E2 noted that stride is less important and wondered how the translation from data to glyph would be calculated. He suggested the strategy might also be helpful for collective communications (e.g., broadcasts, all-to-all, reductions), a set of patterns we did not consider in this work.

We interpreted these responses to suggest the designs were worth further study, particularly E1's ability to interpret well enough to detect an error and E2's interest in further patterns. However, there are design decisions in applying these glyphs in some scenarios, particularly in zoomed-out time, that require refinement. We leave these cases for future iterations and instead focus on how the base designs could be interpreted by a wider range of users in a controlled study.
\section{Evaluation}

We evaluate \system in two aspects: (1) \emph{effectiveness (accuracy)}, which assesses how accurate \system is in  detecting and ranking root causes, and (2) \emph{efficiency}, which assesses how long it takes for \system to derive root causes and conduct end-to-end analysis in action. Particularly, we intend to address the following research questions:
    %\item \emph{User Experience}, which represents how two groups of SREs experience while using \system: domain SREs which use \system to find the root cause to investigate and mitigate, infrastructure SREs which maintain \system to facilitate new requirements.


\begin{itemize}
    \item \textbf{RQ1.} What are the accuracy and efficiency of \system when applied on the collected dataset?
    \item \textbf{RQ2.} How does \system compare with baseline approaches in terms of accuracy?
    \item \textbf{RQ3.} What are the accuracy and efficiency of \system in an end-to-end scenario?
\end{itemize}

% RQ1 aims to see if given a clear triggering event, how would \system perform. RQ2 aims to see whether using adaptive event-driven approach has an advantage compared to baseline in this scenario. RQ3 puts \system in the real time end-to-end scenario, evaluates the \system's efficiency as a whole. 
%2) Conduct a user study to figure out whether \system is actually helpful and user-friendly to the end SRE users.

\subsection{Evaluation Setup}
\label{sec:evalset}
%To measure whether \system can deal with the challenges with industrial settings, we evaluate \system in an environment that is able to represent the real-time large scale distributed system that is running in real world
To evaluate \system in a real-world scenario, we deploy and apply \system  in eBay's e-commerce system that serves more than 159 million active buyers. In particular, we apply \system upon a microservice ecosystem that contains over 5,000 services on three data centers. These services are built on different tech stacks with different programming languages, including Java, Python, Node.js, etc. Furthermore, these services interact with each other by using different types of service protocols, including HTTP, gRPC,  and Message Queue. The distributed tracing of the ecosystem generates 147B traces on average per day.% with 2.8T spans~\cite{opentracing}%The most busy services in the system have more than 50,000 TPS (transactions per second) every day. It is both time-consuming and troublesome for site reliability engineers to manually solve different incidents without the support of tools like \system.  
\subsubsection{Data Set}
\label{sec:dataset}
The SRE teams at eBay help collect a labeled data set containing 952 incidents over 15 months (Jan 2020 - Apr 2021). Each incident data contains the input required by \system (e.g., dependency snapshot and events with details) and the root cause manually labeled by the SRE teams. %To create this data set, 
These incidents are grouped into two categories: 
%we evaluate over 1,500 incidents grouped into two categories: %and around ${40\%}$ cases are missing supported events or caused by external issues (e.g, regional network provider failures). These incidents are categorized as:
\begin{itemize}
\item \emph{Business domain incidents.} These incidents are detected mainly due to their business impact. For example, end users encounter failed interactions, and business or customer experience is impacted, similar to the example in  Figure~\ref{fig:example1}. 
\item \emph{Service-based incidents.} These incidents are detected mainly due to their impact on the service level, similar to the example in Figure~\ref{fig:dynamic_example}.
\end{itemize}

An internal incident may get detected early, and then likely get categorized as a service-based incident or even solved directly by owners without records. On the other hand, infrastructure-level issues or issues of external service providers (e.g., checkout and shipping services) may not get detected until business impact is caused. 

There are 782 business domain incidents and 170 service-based incidents in the data set. For each incident, the root cause is manually labeled, validated, and collected by the SRE teams, who handle the site incidents everyday. For a case with multiple interacting causes, only the most actionable/influential event is labelled as the root cause for the case. These actual root causes and incident contexts serve as the ground truth in our evaluation.


%To trigger the root cause analysis, w
%\begin{enumerate}
%\item\textbf{Business Monitoring}: Business metrics alerts are collected from here. There are around 200+ business time series are detected  by  anomaly detection service every minute.
%\item\textbf{Application Monitoring}:Different application alerts including TPS alerts, Error Spiking alerts, Latency alerts are collected from here. There are around several million time series are detected by  anomaly detection service every minute. 
%\item\textbf{Load Balance Monitoring}: VIP connection number spiking alerts are collected from here. There are around 60k time series are detected by  anomaly detection service every 5 secs. 
%\item\textbf{Deployment system}: Deployment events are collected from here.
%\item\textbf{Site Event system}: Application mark down events and DB mark down events  are collected from here. When a micro-service is not able to talk to its downstream service after retries, it will mark down the downstream service and send a service mark down event to site event system.  Similar to this, when a micro-service is not able to talk to Database, it will mark down the Database and send a DB mark down event to site event system.
%\end{enumerate}

%\subsubsection{Root cause distribution of collected cases}

%\begin{figure}[t]
%\centering
%  \includegraphics[width=\columnwidth]{figures/app_rootcause.png}
%  \caption{Ground-truth RCA distribution of Application-based anomalies}
%  \label{fig:app_rootcause}
%\end{figure}
%\begin{figure}[t]
%\centering
%  \includegraphics[width=\columnwidth]{figures/biz_rootcause.png}
%  \caption{Ground-truth RCA distribution of Business domain anomalies}
%  \label{fig:biz_rootcause}
%\end{figure}

%Figure ~\ref{fig:app_rootcause} and Figure ~\ref{fig:biz_rootcause} shows the root cause distribution of application-based and business domain anomalies in the dataset. From these two charts, we can see that distribution among application-based anomalies is quite diverse while in business domain anomalies, more than 70\% of the anomalies are due to third party errors. We think this imbalance of the distribution among two categories are due to the reason that we cannot monitor third party applications like we do on other parts of the system. So that in the well-monitored applications, the anomalies would be detected earlier by the abnormal metrics before resulting in severe business impact.



\subsubsection{\system Setup}

The \system production system is deployed as three microservices and federated in three data centers with nine 8-core CPUs, 20GB RAM pods each on Kubernetes.

% They are hosted by Gunicorn to scale up. Nginx is used as our routing manager to serve static files and reverse proxying to Gunicorn, and Supervisor to watch the Gunicorn servers in the background and start the processes on reboot.

\subsubsection{Baseline Approaches} 
\label{sec:baseline}
In order to compare \system with other related approaches, we design and implement two baseline approaches for the evaluation:  
\begin{itemize}
    \item \emph{Naive Approach.} This approach directly uses the constructed service dependency graph (Section~\ref{sec:appgraph}). The events are assigned a score by the severeness of the associated anomaly. Then a normalized score for each service is calculated summarizing all the events related to the service. Lastly, the PageRank algorithm is used to calculate the root cause ranking. 
    \item \emph{Non-adaptive Approach.} This approach is not context-aware. It replaces all special rules (i.e., conditional and dynamic ones) with their basic rule versions. Its other parts are identical to \system.
\end{itemize}
The non-adaptive approach can be seen as a baseline for reflecting a group of graph-based approaches (e.g.,  CauseInfer~\cite{chen2014causeinfer} and Microscope~\cite{lin2018microscope}). These approaches also specify certain service-level metrics but lack the context-aware capabilities of \system. Because the tools for these approaches are not publicly available, we implement the non-adaptive approach to approximate these approaches.%of existing graph-based approaches.

\subsection{Evaluation Results}

%Under the settings described in Section~\ref{sec:evalset}, we would be able to evaluate the efficiency and effectiveness of \system in real world scenarios. 

\subsubsection{RQ1}

\label{sec:rq1}

\begin{table}[t]
\centering
\caption{Accuracy of RCA by \system and baselines}
\resizebox{0.9\linewidth}{!}{ 
\begin{tabular}{l|r|r|r|r|r|r|}
\cline{2-7}
                                    & \multicolumn{2}{c|}{\system} & \multicolumn{2}{c|}{Naive} & \multicolumn{2}{c|}{Non-adaptive} \\ \cline{2-7} 
                                    & Top 3        & Top 1       & Top 3         & Top 1      & Top 3         & Top 1 \\ \hline
\multicolumn{1}{|c|}{Service-based}    & 92\%         & 74\%         & 25\%          & 16\%   & 84\% & 62\%       \\ \hline
\multicolumn{1}{|c|}{Business domain} & 96\%         & 81\%        & 2\%          & 1\%   & 28\% & 26\%       \\ \hline
\multicolumn{1}{|c|}{Combined} & 95\%         & 78\%        & 6\%          & 3\%   & 38\% & 33\%       \\ \hline
\end{tabular}
}
 \label{tab:accuracy}
  \vspace{-3.0ex} 
\end{table}

%\begin{figure}[t]
%\centering
%  \includegraphics[width=0.5\textwidth]{figures/app_wrongcause.png}
%  \caption{Ground-truth RCA distribution of Application-based anomalies which \system gives incorrect top-1 result}
%  \label{fig:app_wrongcause}
%\end{figure}
%\begin{figure}[t]
%\centering
%  \includegraphics[width=\columnwidth]{figures/biz_wrongcause.png}
%  \caption{Ground-truth RCA distribution of Business domain anomalies which \system gives incorrect top-1 result}
%  \label{fig:biz_wrongcause}
%\end{figure}

Table~\ref{tab:accuracy} shows the results of applying \system on the collected data set. We measure both top-1 and top-3 accuracy. The top-1 and top-3 accuracy is calculated as the percentage of cases where their ground-truth root cause is ranked within top 1 and top 3, respectively, in \system's results. %All joint rankings must be untied first for fairness. 
\system achieves high accuracy on both incident categories. For example, for business domain incidents, \system achieves 96\% top-3 accuracy.

% While after the causal links construction, the average number of services that are potentially targeted as root cause is reduced to 11.2 and 7.3 for service-based and business domain incidents respectively.
The unsuccessful cases that \system ranks the root cause after top 3 are mostly caused by missing event(s). %Therefore the causality graph is short of necessary causal link(s) or the root cause itself. 
More than one-third of these unsuccessful cases have been addressed by adding necessary events and corresponding rules over time. For example, initially, we had only an event type of general error spike, which mixes different categories of errors and thus causes high false-positive rate. We then have designed different event types for each category of the error metrics (including various internal and client API errors). %We include these cases to reflect a fair setting in production. 
In many cases that \system ranks the root cause after top 1, the labeled root cause is just one of the multiple co-existing root causes. But for fairness,  the SRE teams label only a single root cause in each case. According to the feedback from the SRE teams, \system still facilitates the RCA process for these cases.   %the RCA distribution of these incidents is quite different from the overall ground-truth RCA distribution. More specifically, we find that 

%Every second of an incident RCA process is valuable. 
Our results show that the runtime cost of applying \system is relatively low. For a service-based incident, the average runtime cost of \system is 1.06s while the maximum is 1.69s. For a business domain incident, the average runtime cost is 0.98s while the maximum is 1.14s. %We designed every step as lightweight as possible, for providing timely results in action.
%In order to answer RQ1, we take 219 real production incidents in eBay: 90 App-Search and 129 Domain-Search cases. These cases are handled by the site reliability engineering team or the technical duty officers of eBay. The app-search cases are from the production environment where the trigger anomaly was firstly observed at the application level. The domain-search is built for the most critical business domain - checkout flow. The trigger anomaly of domain search are generated by our ML anomaly detection systems which calculated from different business metrics and customer interaction metrics. Every root cause is manually labeled, and also collect application graph and event context for each ticket. 

%The results are present in table \ref{tab:accuracy}: The \system detects and ranked 100\% and 86\% correct root cause as Top Rank or Top 3 for app-search use cases. For Domain-search are 91\% (Top 1) and 87\% (Top 3). Obviously, our approach is significantly better than the baseline approach. Moreover, the validation for the baseline approach is less strict, the detection is at the application level. For \system accuracy we also restrict on providing the correct root cause event (e.g., code deploy).      

\subsubsection{RQ2}

\label{sec:rq2}

We additionally apply the baseline approaches on the  data set. Table~\ref{tab:accuracy} also shows the evaluation results. The results show that the accuracy of \system is substantially higher than that of the baseline approaches. In terms of the top-1 accuracy, \system achieves 78\% compared with 3\% and 33\% of the naive and non-adaptive approaches, respectively.  In terms of the top-3 accuracy, \system achieves 95\% compared with 6\% and 38\% of the naive and non-adaptive approaches, respectively. 
%In general, \system is 75\% and 45\% more, respectively, in the top-1 accuracy compared with the naive and non-adaptive baselines. \system has even larger advantages in the top-3 accuracy, being 89\% and 57\% more, respectively, compared with the two baselines. For example, \system achieves 81\% in the top-1 accuracy compared with 26\% accuracy of the non-adaptive approach for business domain incidents.%, so that SREs can save time from investigating unrelated services or events.

The naive approach performs worst in all settings, because it blindly propagates the score at service levels.
The accuracy of the non-adaptive approach is much worse for business domain incidents. The reason is that for a business domain incident, it often takes a longer propagation path since the incident is triggered by a group of services, and new dynamic dependencies may be introduced during the event collection, causing more inaccuracy for the non-adaptive approach. %As showed in Section~\ref{sec:rq1}, the causality construction helps to locate the root cause more effectively.  
There can be many non-critical or irrelevant error events in an actual production scenario, aka ``soft'' errors. We suspect that these non-critical or irrelevant events may be ranked higher by the non-adaptive approach since they are similar to injected faults and hard to be distinguished from the actual ones. \system uses dynamic and conditional rules to discover the actual causal links, building fewer links related to such non-critical or irrelevant events for leading to higher accuracy.  

\subsubsection{RQ3}

\begin{table}[]
\centering
\caption{Comparison of \system results on the dataset and end-to-end scenario}
\resizebox{0.95\linewidth}{!}{ 
\begin{tabular}{l|r|r|r|r|}
\cline{2-5}
                                             & \multicolumn{2}{c|}{Service-based} & \multicolumn{2}{c|}{Business Domain} \\ \cline{2-5} 
                                             & Dataset          & End-to-End          & Dataset         & End-to-End         \\ \hline
\multicolumn{1}{|l|}{Top-1 Accuracy}         & 74\%             & 73\%                & 81\%            & 73\%               \\ \hline
\multicolumn{1}{|l|}{Top-3 Accuracy}         & 92\%              & 91\%                & 96\%            & 87\%               \\ \hline
\multicolumn{1}{|l|}{Average Runtime Cost} & 1.06s            & 3.16s               & 0.98s           & 2.98s              \\ \hline
\multicolumn{1}{|l|}{Maximum Runtime Cost} & 1.69s            & 4.56s               & 1.14s           & 3.61s              \\ \hline
\end{tabular}
}
\label{tab:compare}
 \vspace{-3.0ex} 
\end{table}

To evaluate \system under an end-to-end scenario, we apply \system  upon actual incidents in action. Table~\ref{tab:compare} shows the results. The accuracy has a decrease of up to 9 percentage points in the end-to-end scenario, with some failures caused by production issues such as missing data and service/storage failures. In addition, the runtime cost is increased by up to nearly 3 seconds due to the time spent on fetching data from different data sources, e.g., querying the events for a certain time period.

%\system is currently deployed in production, and it helps to detect/confirm the root causes and boost the triaging speed in action for many cases by the time we submit this paper. There is one great case to share: around mid July 2020, there was an issue happened in one of core DB which impacted multiple services across multiple data centers. This caused more than 100 events including more than 40 database mark down events, along with many VIP connection number spike alerts, latency alerts, and error spike alerts happened around the same time. \system was successfully analyzing the root cause to be one problematic DB within a few seconds.

%\subsection{Case Study}
%During the time we put \system integrated into the production environment to work in the real time scenario, we found an interesting case that we would like to share. In mid July 2020, there was an issue happened in one of core DB which impacted multiple applications across multiple data centers. This caused more than 100 events including more than 40 database mark down events, along with many VIP connection number spike alerts, latency alerts, and error spike alerts happened around the same time. \system was successfully analyzing the root cause to be one problematic DB within a few seconds. 
%\system is currently deployed in production and being used for site reliability teams. According to engineers monitoring platform architect, \system explores a new approach to analyze the alerts and events in order to improve observability efficiency. 
%Another comment on \system comes from a senior reliability engineer who now uses \system everyday, "\system fasten our site issue triage a lot. For example, in connection stacking cases, \system clearly shows error chain (connection stacking chain) and points to the last unhealthy pool, which greatly saves our time to check logs of every pool on this chain to find out which pool is the next. And thanks to the \system feature that gathering almost all related events, we can identify which pools or events are related to the ongoing site issue instead of checking every dashboard first to collect pools health information first. Every second improving of issue resolving speed saves thousands of dollars."  



\begin{comment}
\begin{figure}
\includegraphics[width=\linewidth]{figs/beyond_tss_lesion.pdf}
\caption[]{End-to-End runtime lesion study of the entire MNIST dataset and the FMA featurized music dataset. Each of DROP's contributions provides a runtime improvement.}
\label{fig:beyond_lesion}
\end{figure}
\end{comment}



\section{Conclusion}
\label{sec:conclusion}

Advanced data analytics techniques must scale to rising data volumes. 
DR techniques offer a powerful toolkit when processing these datasets, with PCA frequently outperforming popular techniques in exchange for high computational cost. 
In response, we propose DROP, a new dimensionality reduction optimizer. 
DROP combines progressive sampling, progress estimation, and online aggregation to identify high quality low dimensional bases via PCA without processing the entire dataset by balancing the runtime of downstream tasks and achieved dimensionality. 
Thus, DROP provides a first step in bridging the gap between quality and efficiency in end-to-end DR for downstream \red{analytics}. 

%We revisit canonical operators for time series dimensionality reduction and the measurement study of~\cite{keogh-study}, and show that PCA is more effective than popular alternatives in the data mining literature often by a margin of over $2\times$ on average on gold-standard time series benchmark data sets with respect to output data dimension. More surprisingly, we empirically demonstrate that a small number of samples are sufficient to accurately characterize directions of maximum variance and obtain a high-quality low-dimensional transformation.




{
\bibliographystyle{IEEEtran}	
\bibliography{sentinel}
}


\end{document}
