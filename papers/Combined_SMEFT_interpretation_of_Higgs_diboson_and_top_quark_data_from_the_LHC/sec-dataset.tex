\section{Experimental data and theoretical calculations}
\label{sec:settings_expdata}

In this section we present the experimental measurements and the theoretical
computations used to constrain the SMEFT operators
introduced in
Sect.~\ref{sec:smefttheory}.
%
We focus in turn on each of the three
groups of LHC processes that we consider in the current analysis:
top quark, Higgs boson,
and gauge boson pair production.

\subsection{Top-quark production data}
\label{sec:exptop}

The top-quark production measurements included in this analysis belong to four
different categories: inclusive top-quark pair production, top-quark pair
production in association with vector bosons or heavy quarks, inclusive single
top-quark production, and single top-quark production in association with
vector bosons.
%
In the following we present the datasets that belong to each of
these categories. Top-quark pair production in association
with a Higgs boson is discussed in Sect.~\ref{sec:HiggsProductionDecay}.

\paragraph{Inclusive top-quark pair production.}
%
The experimental measurements of inclusive top-quark pair production
included in this analysis are summarised in Table~\ref{eq:input_datasets}. 
For each of them, we indicate the dataset label, the center of mass energy
$\sqrt{s}$, the integrated luminosity $\mathcal{L}$, the final state or the
specific production mechanism, the physical observable, the number of data 
points $n_{\rm dat}$, and the publication reference. Measurements indicated with
a {\bf (*)} were not included in our earlier analysis~\cite{Hartland:2019bjb}.

The bulk of the measurements correspond to datasets already included
in~\cite{Hartland:2019bjb}: at 8~TeV,
the ATLAS top-quark pair invariant mass distribution~\cite{Aad:2015mbv} and
the CMS top-quark pair normalized invariant rapidity
distribution~\cite{Khachatryan:2015oqa}, both in the lepton+jets final state,
the CMS top-quark pair normalized invariant mass and rapidity two-dimensional
distribution in the dilepton final state~\cite{Sirunyan:2017azo},
and the ATLAS and CMS $W$ helicity fractions~\cite{Aaboud:2016hsq,
  Khachatryan:2016fky}; at 13~TeV, the CMS top-quark pair invariant mass
distributions in the lepton+jets and dilepton final states based on integrated
luminosities of up to $\mathcal{L}=35.8$~fb$^{-1}$~\cite{Khachatryan:2016mnb,
  Sirunyan:2018wem,Sirunyan:2017mzl}.
%
In addition to these, we now  consider further top-quark pair
invariant mass distributions: at 8~TeV, the ATLAS measurement in the dilepton
final state~\cite{Aaboud:2016iot}; at 13~TeV, and the ATLAS and CMS
measurements, respectively in the lepton+jets and dilepton final states, 
corresponding to an integrated luminosity of
$\mathcal{L}=35.8$~fb$^{-1}$~\cite{Aad:2019ntk,Sirunyan:2018ucr}.
%
We also include top-quark pair charge asymmetry measurements:
the ATLAS and CMS combined dataset at 8~TeV~\cite{Sirunyan:2017lvd},
and the ATLAS dataset at 13~TeV~\cite{ATLAS:2019czt}.

%-------------------------------------------------------------------------------
\begin{table}[t]
  \centering
  \scriptsize
   \renewcommand{\arraystretch}{1.80}
  \begin{tabular}{c|c|c|c|c|c}
 Dataset   &  $\sqrt{s}$, $\mathcal{L}$  & Info  &  Observables  & $n_{\rm dat}$ & Ref   \\
    \toprule
      {\tt ATLAS\_tt\_8TeV\_ljets}
      & { \bf 8 TeV, 20.3~fb$^{-1}$}
      & lepton+jets
      & $d\sigma/dm_{t\bar{t}}$
      & 7
      & \cite{Aad:2015mbv} \\
%-------------------------------------------------------------------------------
    \midrule
      {\tt CMS\_tt\_8TeV\_ljets}
      & {\bf 8 TeV, 20.3~fb$^{-1}$}
      & lepton+jets
      & $1/\sigma d\sigma/dy_{t\bar{t}}$
      & 10
      & \cite{Khachatryan:2015oqa} \\
%-------------------------------------------------------------------------------
    \midrule
      {\tt CMS\_tt2D\_8TeV\_dilep}
      & {\bf 8 TeV, 20.3~fb$^{-1}$}
      & dileptons
      & $1/\sigma d^2\sigma/dy_{t\bar{t}}dm_{t\bar{t}}$
      & 16
      & \cite{Sirunyan:2017azo} \\
%-------------------------------------------------------------------------------
    \midrule
      {\tt ATLAS\_tt\_8TeV\_dilep} {\bf(*)}
      & {\bf 8 TeV, 20.3~fb$^{-1}$}
      & dileptons
      & $d\sigma/dm_{t\bar{t}}$
      & 6
      & \cite{Aaboud:2016iot}  \\
%-------------------------------------------------------------------------------
    \midrule
    \midrule
      {\tt CMS\_tt\_13TeV\_ljets\_2015 }
      & {\bf 13 TeV, 2.3~fb$^{-1}$}
      & lepton+jets
      & $d\sigma/dm_{t\bar{t}}$
      & 8
      & \cite{Khachatryan:2016mnb}  \\
%-------------------------------------------------------------------------------
    \midrule
      {\tt CMS\_tt\_13TeV\_dilep\_2015 }
      & {\bf 13 TeV, 2.1~fb$^{-1}$}
      & dileptons
      & $d\sigma/dm_{t\bar{t}}$
      & 6
      & \cite{Sirunyan:2017mzl}  \\
%-------------------------------------------------------------------------------
    \midrule
      {\tt CMS\_tt\_13TeV\_ljets\_2016 }
      & {\bf 13 TeV, 35.8~fb$^{-1}$}
      & lepton+jets
      & $d\sigma/dm_{t\bar{t}}$
      & 10
      & \cite{Sirunyan:2018wem}  \\
%-------------------------------------------------------------------------------
    \midrule
      {\tt CMS\_tt\_13TeV\_dilep\_2016 } {\bf(*)}
      & {\bf 13 TeV, 35.8~fb$^{-1}$}
      & dileptons
      & $d\sigma/dm_{t\bar{t}}$
      & 7
      & \cite{Sirunyan:2018ucr}  \\
%-------------------------------------------------------------------------------
    \midrule
      {\tt ATLAS\_tt\_13TeV\_ljets\_2016 } {\bf (*)}
      & {\bf 13 TeV, 35.8~fb$^{-1}$}
      & lepton+jets
      & $d\sigma/dm_{t\bar{t}}$
      & 9
      & \cite{Aad:2019ntk}  \\
%-------------------------------------------------------------------------------
   \midrule
%-------------------------------------------------------------------------------
\midrule
 {\tt ATLAS\_WhelF\_8TeV}  & {\bf 8 TeV, 20.3~fb$^{-1}$}  & $W$ hel. fract &
$F_0, F_L, F_R$  &  3  &  \cite{Aaboud:2016hsq}  \\
%-------------------------------------------------------------------------------
\midrule
{\tt CMS\_WhelF\_8TeV}  & {\bf 8 TeV, 20.3~fb$^{-1}$}  & $W$ hel. fract &
$F_0, F_L, F_R$  &  3  &  \cite{Khachatryan:2016fky}  \\
%-------------------------------------------------------------------------------
\midrule
\midrule
    {\tt ATLAS\_CMS\_tt\_AC\_8TeV}  {\bf(*)} & {\bf 8 TeV, 20.3~fb$^{-1}$} & charge asymmetry &
    $ A_C$ & 6 & \cite{Sirunyan:2017lvd}\\
\midrule   
    {\tt ATLAS\_tt\_AC\_13TeV
}  {\bf(*)} & {\bf 13 TeV, 139~fb$^{-1}$} & charge asymmetry &
    $ A_C$ & 5 & \cite{ATLAS:2019czt}\\
\bottomrule
  \end{tabular}
  \caption{\small The experimental measurements of inclusive top-quark pair 
    production at the LHC considered in the present analysis.
    %
    For each dataset we indicate the label, 
    the center of mass energy $\sqrt{s}$, the
    integrated luminosity $\mathcal{L}$,
    the final state or the specific 
    production mechanism, the physical observable, the number of data 
    points $n_{\rm dat}$, and the publication reference.
    %
    Measurements indicated with {\bf (*)} were not included 
    in~\cite{Hartland:2019bjb}.
    %
    We also include in this category the $W$ helicity fractions
    from top quark decay and the charge asymmetries.
\label{eq:input_datasets}
}
\end{table}
%%%%%%%%%%%%%%%%%%%%%%%%%%%%%%%%%%%%%%%%%%%%%%%%%%%%%%%%%%%%%%%%%%%%%%%%%%%%%%%%



%-------------------------------------------------------------------------------

Although several distributions differential in various kinematic variables are
available for the measurements presented in~\cite{Khachatryan:2015oqa,
  Sirunyan:2017azo,Khachatryan:2016mnb,Sirunyan:2018wem,
  Sirunyan:2017mzl,Aaboud:2016iot,Aad:2019ntk,Sirunyan:2018ucr}, only one of
them can typically be included in the fit at a time.
%
The reason is that
experimental correlations between pairs of distributions are unknown:
including more than one distribution at a time will therefore result in
a double counting.
%
An exception to this state of affairs is represented
by the ATLAS measurement of~\cite{Aad:2015mbv}, which is provided with the
correlations among differential distributions.
%
Unfortunately, they significantly deteriorate the fit quality when
an analysis of all the available distributions is attempted, a fact that
questions their reliability (see also~\cite{Amoroso:2020lgh,Bailey:2019yze}).
We therefore include only one distribution also in this case.
%
In general, we
include the invariant mass distribution $m_{t\bar{t}}$, whose high-energy
tail 
is known to be particularly
sensitive to deviations from the SM expectations.
%
For~\cite{Khachatryan:2015oqa}
we include instead the invariant rapidity distribution as in our earlier
analysis~\cite{Hartland:2019bjb}, due to difficulties in achieving an
acceptable fit quality to $m_{t\bar{t}}$.

The additional top-quark pair measurements considered
in this work do not expand the kinematic coverage
in the EFT parameter space in comparison to those already included
in~\cite{Hartland:2019bjb}.
%
Nevertheless, they provide additional weight for
the inclusive top-quark pair differential distributions in the global fit,
which are known to provide the dominant constraints on several of the EFT
coefficients.
%
All in all, we end up with $n_{\rm dat}=94$ data points in this category.

Additional sensitivity to EFT effects could be achieved
by means of LHC Run-II measurements with an extended coverage in the
invariant mass or transverse momentum.
%
However, differential distributions based on luminosities larger than
$\mathcal{L}\simeq 36$~fb$^{-1}$ are not available yet: the statistical
precision of the data, and consequently their constraining power, remain
therefore limited.
%
For instance, the ATLAS fully hadronic final state
measurement~\cite{Aad:2020nsf} is available, but it exhibits
larger uncertainties than in the cleaner lepton+jets and dilepton final states.
Furthermore, some measurements are not reconstructed at the parton level,
as required in our analysis.
%
This is the case of the ATLAS and CMS measurements
at high top-quark
transverse momentum~\cite{Aad:2020nsf,Sirunyan:2020vwa}, that are
based on reconstructing boosted topologies, and of the dilepton distributions
from ATLAS~\cite{Aad:2019hzw}, that are restricted to the particle level.

Concerning theoretical calculations, the SM cross-sections are evaluated at NLO
using {\tt MadGraph5\_aMC@NLO}~\cite{Alwall:2014hca} and supplemented with
NNLO $K$-factors~\cite{Czakon:2016dgf,Czakon:2016olj}.
The input PDF set is NNPDF3.1NNLO no-top~\cite{Ball:2017nwa},
to avoid possible contamination between PDF and EFT
effects.\footnote{See~\cite{Greljo:2021kvv,Carrazza:2019sec} for a detailed
  discussion of the interplay between PDF and EFT fits.}
%
The EFT cross-sections are evaluated with
{\tt MadGraph5\_aMC@NLO}~\cite{Alwall:2014hca}
combined with the {\tt SMEFT@NLO} model~\cite{Degrande:2020evl}.
Unless otherwise specified, the same EFT settings will be used also for the
other processes considered in this analysis.
%
Specifically, NLO QCD effects to
the EFT corrections are accounted systematically whenever available.

\paragraph{Associated top-quark pair production.}
%
Table~\ref{eq:input_datasets2} lists, in the same format as
Table~\ref{eq:input_datasets}, the experimental measurements for top
quark pair production in association with heavy quarks or weak vector bosons.
%
The dataset considered in~\cite{Hartland:2019bjb} consisted of
the CMS measurements of total cross-sections for $t\bar{t}t\bar{t}$
and $b\bar{b}b\bar{b}$ at 13~TeV~\cite{Sirunyan:2017snr,Sirunyan:2017roi},
and in the ATLAS and CMS measurements of inclusive $tW$ and $tZ$ production at
8~TeV and 13~TeV~\cite{Khachatryan:2015sha,Sirunyan:2017uzs,Aad:2015eua,
  Aaboud:2016xve}.
%
In the present analysis, we augment this dataset with the most updated
measurements of total cross-sections for $t\bar{t}t\bar{t}$ and
$t\bar{t}b\bar{b}$ production at 13~TeV: for $t\bar{t}b\bar{b}$, with the ATLAS
and CMS measurements based on
$\mathcal{L}=137$~fb$^{-1}$~\cite{Sirunyan:2019wxt,Aad:2020klt};
for $\sigma_{\rm tot}(t\bar{t}b\bar{b})$, with the ATLAS and CMS measurements
based on $\mathcal{L}=36$~fb$^{-1}$~\cite{Aaboud:2018eki,Sirunyan:2019jud}.
These measurements are comparatively more precise than the measurements
already included in~\cite{Hartland:2019bjb} thanks to the increased luminosity.

Concerning top-quark pair production in association with an electroweak gauge
boson, we include here the ATLAS total cross-section measurements of
$t\bar{t}W$ and $t\bar{t}Z$ based on
$\mathcal{L}=36$~fb$^{-1}$~\cite{Aaboud:2019njj},
as well as the CMS differential
measurements of $d\sigma(t\bar{t}Z)/dp_T^Z$ based on
$\mathcal{L}=78$~fb$^{-1}$~\cite{CMS:2019too}, which
is the first differential measurement of $t\bar{t}V$ associated production
presented at the LHC. We do not include the still preliminary
ATLAS measurement of $\sigma_{\rm tot}(t\bar{t}Z)$ based on
$\mathcal{L}=139$~fb$^{-1}$~\cite{ATLAS-CONF-2020-028}.
%
The $t\bar{t}V$ measurements are
especially useful to constrain EFT effects that modify the electroweak
couplings of the top-quark.
%
In total, we include $n_{\rm dat}=20$
data points in the category of $t\bar{t}$ associated production with heavy quark pairs
or weak vector bosons.

%-------------------------------------------------------------------------------
\begin{table}[t]
  \centering
  \scriptsize
   \renewcommand{\arraystretch}{1.40}
  \begin{tabular}{c|c|c|c|c|c}
 Dataset   &  $\sqrt{s}, \mathcal{L}$ & Info  &  Observables  & $N_{\rm dat}$ & Ref   \\
    \toprule
%-----------------------------------------------------------------------------------------
 {\tt CMS\_ttbb\_13TeV}  & {\bf 13 TeV, 2.3~fb$^{-1}$}  & total xsec & $\sigma_{\rm tot}(t\bar{t}b\bar{b})$  &  1  &  \cite{Sirunyan:2017snr}  \\
\midrule
%-----------------------------------------------------------------------------------------
{\tt CMS\_ttbb\_13TeV\_2016}  {\bf (*)}  & {\bf 13 TeV, 35.9~fb$^{-1}$}  & total xsec & $\sigma_{\rm tot}(t\bar{t}b\bar{b})$  &  1  &  \cite{Sirunyan:2019jud}  \\
\midrule
%-----------------------------------------------------------------------------------------
{\tt ATLAS\_ttbb\_13TeV\_2016}  {\bf (*)}  & {\bf 13 TeV, 35.9~fb$^{-1}$}  & total xsec & $\sigma_{\rm tot}(t\bar{t}b\bar{b})$  &  1  &  \cite{Aaboud:2018eki}  \\
\midrule
%-----------------------------------------------------------------------------------------
 {\tt CMS\_tttt\_13TeV}  & {\bf 13 TeV, 35.9~fb$^{-1}$}  & total xsec & $\sigma_{\rm tot}(t\bar{t}t\bar{t})$  &  1  &  \cite{Sirunyan:2017roi}  \\
%-----------------------------------------------------------------------------------------
\midrule
%-----------------------------------------------------------------------------------------
 {\tt CMS\_tttt\_13TeV\_run2} {\bf (*)} & {\bf 13 TeV, 137~fb$^{-1}$}  & total xsec & $\sigma_{\rm tot}(t\bar{t}t\bar{t})$  &  1  &  \cite{Sirunyan:2019wxt}  \\
 %-----------------------------------------------------------------------------------------
 \midrule
%-----------------------------------------------------------------------------------------
 {\tt ATLAS\_tttt\_13TeV\_run2} {\bf (*)} & {\bf 13 TeV, 137~fb$^{-1}$}  & total xsec & $\sigma_{\rm tot}(t\bar{t}t\bar{t})$  &  1  &  \cite{Aad:2020klt}  \\
 %-----------------------------------------------------------------------------------------
\midrule
\midrule
  {\tt CMS\_ttZ\_8TeV}  & {\bf 8 TeV, 19.5~fb$^{-1}$}  & total xsec & $\sigma_{\rm tot}(t\bar{t}Z)$  &  1  &  \cite{Khachatryan:2015sha}  \\ \midrule
  %-----------------------------------------------------------------------------------------
   {\tt CMS\_ttZ\_13TeV}  & {\bf 13 TeV, 35.9~fb$^{-1}$ }  & total xsec & $\sigma_{\rm tot}(t\bar{t}Z)$  &  1  &  \cite{Sirunyan:2017uzs}  \\
%-----------------------------------------------------------------------------------------
   \midrule
    {\tt CMS\_ttZ\_ptZ\_13TeV}  {\bf (*)} & {\bf 13 TeV, 77.5~fb$^{-1}$ }  & total xsec & $d\sigma(t\bar{t}Z)/dp_T^Z $  &  4  &  \cite{CMS:2019too}  \\
%-----------------------------------------------------------------------------------------
   \midrule
  {\tt ATLAS\_ttZ\_8TeV}  & {\bf 8 TeV, 20.3~fb$^{-1}$}  & total xsec & $\sigma_{\rm tot}(t\bar{t}Z)$  &  1  &  \cite{Aad:2015eua}  \\
  %-----------------------------------------------------------------------------------------
    \midrule
        {\tt ATLAS\_ttZ\_13TeV}  & {\bf 13 TeV, 3.2~fb$^{-1}$}  & total xsec & $\sigma_{\rm tot}(t\bar{t}Z)$  &  1  &  \cite{Aaboud:2016xve}  \\
 %-----------------------------------------------------------------------------------------
     \midrule
   {\tt ATLAS\_ttZ\_13TeV\_2016} {\bf (*)} & {\bf 13 TeV, 36~fb$^{-1}$}  & total xsec & $\sigma_{\rm tot}(t\bar{t}Z)$  &  1  &  \cite{Aaboud:2019njj}  \\
%-----------------------------------------------------------------------------------------
  \midrule
  \midrule
    {\tt CMS\_ttW\_8\_TeV}  & {\bf 8 TeV, 19.5~fb$^{-1}$}  & total xsec & $\sigma_{\rm tot}(t\bar{t}W)$  &  1  &  \cite{Khachatryan:2015sha}  \\ \midrule
    %-----------------------------------------------------------------------------------------
     {\tt CMS\_ttW\_13TeV}  & {\bf 13 TeV, 35.9~fb$^{-1}$}  & total xsec & $\sigma_{\rm tot}(t\bar{t}W)$  &  1  &  \cite{Sirunyan:2017uzs}  \\
%-----------------------------------------------------------------------------------------
   \midrule
   {\tt ATLAS\_ttW\_8TeV}  & {\bf 8 TeV, 20.3~fb$^{-1}$}  & total xsec & $\sigma_{\rm tot}(t\bar{t}W)$  &  1  &  \cite{Aad:2015eua}  \\
   %-----------------------------------------------------------------------------------------
    \midrule
   {\tt ATLAS\_ttW\_13TeV}  & {\bf 13 TeV, 3.2~fb$^{-1}$}  & total xsec & $\sigma_{\rm tot}(t\bar{t}W)$  &  1  &  \cite{Aaboud:2016xve}  \\
   %-----------------------------------------------------------------------------------------
     \midrule
   {\tt ATLAS\_ttW\_13TeV\_2016} {\bf (*) } & {\bf 13 TeV, 36~fb$^{-1}$} & total xsec & $\sigma_{\rm tot}(t\bar{t}W)$  &  1  &  \cite{Aaboud:2019njj}  \\
%-----------------------------------------------------------------------------------------
\bottomrule
  \end{tabular}
  \caption{\small Same as Table~\ref{eq:input_datasets}, now for the production
    of top quark pairs in association with
    heavy quarks and with weak vector bosons.
     \label{eq:input_datasets2}
  }
\end{table}
%%%%%%%%%%%%%%%%%%%%%%%%%%%%%%%%%%%%%%%%%%%%%%%%%%%%%%%%%%%%%%%%%%%%%%%%%%%%%%%%%%%%%%%%%%%%%%%%%%%%


%-------------------------------------------------------------------------------

Theoretical predictions are computed at NLO both in the SM and in the EFT.
%
We use {\tt MCFM} for the SM cross-sections and {\tt SMEFT@NLO} for the EFT corrections,
with NLO QCD effects accounted for exactly for the 2-fermion operators. 
%
The exception is the $p_T^Z$ distribution in $t\bar{t}Z$ events, for which
{\tt MadGraph5\_aMC@NLO} is used instead to evaluate the SM cross-section
at NLO.

\paragraph{Inclusive single top-quark production.}
We now move to consider the inclusive production of single top-quarks,
both in the $t$-channel and in the $s$-channel ($tW$ associated production is
discussed separately below).
%
Table~\ref{eq:input_datasets3} displays the information on
the experimental data for these processes that is being considered in the
present analysis.
%
The dataset in this category that was already included in our
previous analysis~\cite{Hartland:2019bjb} consisted, at 8 TeV, of the
$t$-channel total cross-sections and in the top-quark rapidity differential
distributions from CMS~\cite{Khachatryan:2014iya,CMS-PAS-TOP-14-004}
and from ATLAS~\cite{Aaboud:2017pdi}, and in the $s$-channel total
cross-sections from ATLAS~\cite{Aad:2015upn} and CMS~\cite{Khachatryan:2016ewo};
at 13 TeV, in  the $t$-channel total cross-sections and top-quark rapidity
differential distributions from ATLAS~\cite{Aaboud:2016ymp} and
CMS~\cite{CMS:2016xnv,Sirunyan:2016cdg}.

Here we augment this dataset with one additional measurement, namely the CMS
top-quark rapidity differential cross-section for $t$-channel single top-quark
production at 13 TeV based on
$\mathcal{L}=35.9$~fb$^{-1}$~\cite{Sirunyan:2019hqb}.
%
As customary, we consider the distribution reconstructed at parton level for
consistency with the theoretical predictions.
No differential measurements of single top-quark production
based on the Run II dataset have been presented by ATLAS so far.
Furthermore, while the ATLAS and CMS combination of total cross-sections for
single top-quark production at 7 TeV and 8 TeV has been presented
in~\cite{Aaboud:2019pkc}, here we include instead the original individual
measurements. We end up with $n_{\rm dat}=27$ data points in this category.

%-------------------------------------------------------------------------------
\begin{table}[t]
  \centering
  \scriptsize
   \renewcommand{\arraystretch}{1.60}
  \begin{tabular}{c|c|c|c|c|c}
 Dataset   &  $\sqrt{s}, \mathcal{L}$ & Info  &  Observables  & $N_{\rm dat}$ & Ref   \\
\toprule
    {\tt CMS\_t\_tch\_8TeV\_inc}
    & {\bf 8 TeV, 19.7~{\rm \bf fb}$^{-1}$ }
    & $t$-channel
    & $\sigma_{\rm tot}(t),\sigma_{\rm tot}(\bar{t})$  & 2
    & \cite{Khachatryan:2014iya}  \\
%--------------------------------------------------------------------------
\midrule
    {\tt ATLAS\_t\_tch\_8TeV}
    & {\bf 8 TeV, 20.2~{\rm \bf fb}$^{-1}$}
    & $t$-channel
    & $d\sigma(tq)/dy_t$
    & 4
    & \cite{Aaboud:2017pdi}  \\
%----------------------------------------------------------------------------
\midrule
    {\tt CMS\_t\_tch\_8TeV\_dif}
    & {\bf 8 TeV, 19.7~{\rm \bf fb}$^{-1}$}
    & $t$-channel
    & $d\sigma/d|y^{(t+\bar{t})}|$
    & 6
    & \cite{CMS-PAS-TOP-14-004}  \\
%-----------------------------------------------------------------------------
\midrule
    {\tt CMS\_t\_sch\_8TeV}
    & {\bf 8 TeV, 19.7~{\rm \bf fb}$^{-1}$}
    & $s$-channel
    & $\sigma_{\rm tot}(t+\bar{t})$
    & 1
    & \cite{Khachatryan:2016ewo}  \\
%------------------------------------------------------------------------------
\midrule
    {\tt ATLAS\_t\_sch\_8TeV}
    & {\bf 8 TeV, 20.3~{\rm \bf fb}$^{-1}$ }
    & $s$-channel
    & $\sigma_{\rm tot}(t+\bar{t})$
    & 1
    & \cite{Aad:2015upn}  \\
%------------------------------------------------------------------------------
\midrule
\midrule
    {\tt ATLAS\_t\_tch\_13TeV}
    & {\bf 13 TeV, 3.2~{\rm \bf fb}$^{-1}$}
    & $t$-channel
    & $\sigma_{\rm tot}(t),\sigma_{\rm tot}(\bar{t})$
    & 2
    & \cite{Aaboud:2016ymp}  \\
%------------------------------------------------------------------------------
    \midrule
    {\tt CMS\_t\_tch\_13TeV\_inc}
    & {\bf 13 TeV, 2.2~{\rm \bf fb}$^{-1}$}
    & $t$-channel
    & $\sigma_{\rm tot}(t),\sigma_{\rm tot}(\bar{t})$
    & 2
    & \cite{Sirunyan:2016cdg}  \\
%------------------------------------------------------------------------------
    \midrule
    {\tt CMS\_t\_tch\_13TeV\_dif}
    & {\bf 13 TeV, 2.3~{\rm \bf fb}$^{-1}$}
    & $t$-channel
    & $d\sigma/d|y^{(t+\bar{t})}|$
    & 4
    & \cite{CMS:2016xnv}  \\
    \midrule
    {\tt CMS\_t\_tch\_13TeV\_2016} {\bf (*)}
    & {\bf 13 TeV, 35.9~{\rm \bf fb}$^{-1}$}
    & $t$-channel
    & $d\sigma/d|y^{(t)}|$
    & 5
    & \cite{Sirunyan:2019hqb}  \\
%------------------------------------------------------------------------------
\bottomrule
  \end{tabular}
  \caption{\small Same as Table~\ref{eq:input_datasets},
    now for inclusive single $t$ production both in the $t$- and the $s$-channels.
     \label{eq:input_datasets3}
  }
\end{table}

%-------------------------------------------------------------------------------

The calculation of the SM and EFT cross-sections has been carried out with the
same settings as for inclusive $t\bar{t}$ production. Note that for single top
we work with a 5-flavour number scheme (5FNS) where the bottom quark is
considered as massless, and thus enters the initial state of the reaction,
see~\cite{Nocera:2019wyk} for details.
%
The NNLO QCD $K$-factors in the 5FNS are
obtained from the calculation of~\cite{Berger:2016oht}.

\paragraph{Associated single top-quark production with weak bosons.}
Finally, in Table~\ref{eq:input_datasets4} we consider the experimental
measurements on the associated production of single top-quarks together with
a weak gauge boson $V$.
%
The dataset in this category that was already part of our
original analysis~\cite{Hartland:2019bjb} consisted of the total inclusive
cross-sections for $tW$ production by ATLAS and CMS at
8~TeV~\cite{Aad:2015eto,Chatrchyan:2014tua}
and at 13~TeV~\cite{Aaboud:2016lpj,Sirunyan:2018lcp},
as well as in the ATLAS and CMS measurements of the $tZ$ total
cross-sections at 13~TeV~\cite{Sirunyan:2017nbr,Aaboud:2017ylb},
in the latter case restricted to the fiducial region
in the $Wb\ell^+\ell^-q$ final state.

%-------------------------------------------------------------------------------

%%%%%%%%%%%%%%%%%%%%%%%%%%%%%%%%%%%%%%%%%%%%%%%%%%%%%%%%%%%%%%%%%%%%%%%%%%%%%%%%%%%%%%%
%\begin{table}[htbp]
\begin{table}[t]
  \centering
  \scriptsize
   \renewcommand{\arraystretch}{1.60}
  \begin{tabular}{c|c|c|c|c|c}
 Dataset   &  $\sqrt{s}, \mathcal{L}$ & Info  &  Observables  & $N_{\rm dat}$ & Ref   \\
\toprule
\multirow{2}{*}{ {\tt ATLAS\_tW\_8TeV\_inc}}      &
\multirow{2}{*}{{\bf 8 TeV, 20.2}~{\rm \bf fb}$^{-1}$}   & \multirow{1}{*}{inclusive}   &
\multirow{2}{*}{$\sigma_{\rm tot}(tW)$}  &  1  &
\multirow{2}{*}{\cite{Aad:2015eto}}  \\
    &
  & \multirow{1}{*}{(dilepton)}   &
  &   &\\
%-----------------------------------------------------------------------------------------
\toprule
\multirow{2}{*}{ {\tt ATLAS\_tW\_inc\_slep\_8TeV} {\bf (*)}}      &
\multirow{2}{*}{{\bf 8 TeV, 20.2}~{\rm \bf fb}$^{-1}$}   & \multirow{1}{*}{inclusive}   &
\multirow{2}{*}{$\sigma_{\rm tot}(tW)$}  &  1  &
\multirow{2}{*}{\cite{Aad:2020zhd}}  \\
    &
  & \multirow{1}{*}{(single lepton)}   &
  &   &\\
%-----------------------------------------------------------------------------------------
\midrule
      \multirow{1}{*}{ {\tt CMS\_tW\_8TeV\_inc}}      &
 \multirow{1}{*}{{\bf 8 TeV, 19.7}~{\rm \bf fb}$^{-1}$}   & \multirow{1}{*}{inclusive}   &
\multirow{1}{*}{$\sigma_{\rm tot}(tW)$}  &  1  &
\multirow{1}{*}{\cite{Chatrchyan:2014tua}}  \\
%-----------------------------------------------------------------------------------------
\midrule
        \multirow{1}{*}{ {\tt ATLAS\_tW\_inc\_13TeV}}      &
 \multirow{1}{*}{{\bf 13 TeV, 3.2}~{\rm \bf fb}$^{-1}$}   & \multirow{1}{*}{inclusive}   &
\multirow{1}{*}{$\sigma_{\rm tot}(tW)$}  &  1  &
\multirow{1}{*}{\cite{Aaboud:2016lpj}}  \\
%-----------------------------------------------------------------------------------------
\midrule
     \multirow{1}{*}{ {\tt CMS\_tW\_13TeV\_inc}}      &
 \multirow{1}{*}{{\bf 13 TeV, 35.9}~{\rm \bf fb}$^{-1}$}   & \multirow{1}{*}{inclusive}   &
\multirow{1}{*}{$\sigma_{\rm tot}(tW)$}  &  1  &
\multirow{1}{*}{\cite{Sirunyan:2018lcp}}  \\
%-----------------------------------------------------------------------------------------
\midrule
\midrule
      \multirow{1}{*}{ {\tt ATLAS\_tZ\_13TeV\_inc}}      &
 \multirow{1}{*}{{\bf 13 TeV, 36.1}~{\rm \bf fb}$^{-1}$}    & \multirow{1}{*}{inclusive}   &
\multirow{1}{*}{$\sigma_{\rm tot}(tZq)$}  &  1  &
\multirow{1}{*}{\cite{Aaboud:2017ylb}}  \\
%-----------------------------------------------------------------------------------------
\midrule
 \multirow{1}{*}{ {\tt ATLAS\_tZ\_13TeV\_run2\_inc} {\bf (*)}}      &
 \multirow{1}{*}{{\bf 13 TeV, 139.1}~{\rm \bf fb}$^{-1}$}    & \multirow{1}{*}{inclusive}   &
\multirow{1}{*}{$\sigma_{\rm fid}(t\ell^+\ell^-q)$}  &  1  &
\multirow{1}{*}{\cite{Aad:2020wog}}  \\
%-----------------------------------------------------------------------------------------
\midrule
       \multirow{1}{*}{ {\tt CMS\_tZ\_13TeV\_inc}}      &
 \multirow{1}{*}{{\bf 13 TeV, 35.9}~{\rm \bf fb}$^{-1}$}   & \multirow{1}{*}{inclusive}   &
\multirow{1}{*}{$\sigma_{\rm fid}(Wb\ell^+\ell^-q)$}  &  1  &
\multirow{1}{*}{\cite{Sirunyan:2017nbr}}  \\
%-----------------------------------------------------------------------------------------
\midrule
       \multirow{1}{*}{ {\tt CMS\_tZ\_13TeV\_2016\_inc}  {\bf (*)}}      &
 \multirow{1}{*}{{\bf 13 TeV, 77.4}~{\rm \bf fb}$^{-1}$ }   & \multirow{1}{*}{inclusive}   &
\multirow{1}{*}{$\sigma_{\rm fid}(t\ell^+\ell^-q)$}  &  1  &
\multirow{1}{*}{\cite{Sirunyan:2018zgs}}  \\
%-----------------------------------------------------------------------------------------
\bottomrule
  \end{tabular}
  \caption{\small Same as Table~\ref{eq:input_datasets},
    now for single top quark production in association with
    electroweak gauge bosons.
     \label{eq:input_datasets4}
  }
\end{table}
%%%%%%%%%%%%%%%%%%%%%%%%%%%%%%%%%%%%%%%%%%%%%%%%%%%%%%%%%%%%%%%%%%%%%%%%%%%%%%%%%%%%%%%%%%%%%%%%%%%%

%-------------------------------------------------------------------------------

In addition to these datasets, we include here several new measurements
of $tW$ and $tZ$ production.
First of all, we include a new total cross-section measurement of
$tW$ production by ATLAS at 8 TeV~\cite{Aad:2020zhd}.
This measurement is carried out in the single lepton channel, and thus does
not overlap with~\cite{Aad:2015eto}, which instead was obtained
in the two leptons with one central $b$-jet channel.
Then we include the ATLAS measurement of the fiducial cross-section
for $tZ$ production~\cite{Aad:2020wog} using the $t \ell^+\ell^- q$ final state
(in the tri-lepton channel) based on the full Run II luminosity of
$\mathcal{L}=139$ fb$^{-1}$. In this analysis, the cross-section measurement
differs from the background-only hypothesis
(dominated by $t\bar{t}Z$ and dibosons) by more than five sigma and thus
corresponds to an observation of this process.
%
We also consider the corresponding measurement from CMS,
where the observation of $tZ$ associated production
is reported by reconstructing the $t\ell^+\ell^-q$
final state~\cite{Sirunyan:2018zgs}
based on a luminosity of $\mathcal{L}=77.4$ fb$^{-1}$.
No differential distributions for $tZ$ have been reported so far.
The settings of the theoretical calculations for these
$n_{\rm dat}=9$ data points are the same as of~\cite{Hartland:2019bjb}.

In addition to these measurements, both ATLAS and CMS have
measured differential distributions in $tW$ production
at 13~TeV based on a luminosity of
$\mathcal{L}=35.9$~fb$^{-1}$~\cite{Aaboud:2017qyi,CMS-PAS-TOP-19-003}.
%
However, these measurements are reported
at the particle rather than
at the parton level, and therefore they are not suitable for inclusion in the
present analysis, which is restricted to top-quark level observables.
%
We also note that CMS has reported on the EFT interpretation of the associated
production of top-quarks, including with vector bosons, in an analysis
based on a luminosity of $\mathcal{L}=41.5$ fb$^{-1}$~\cite{CMS-PAS-TOP-19-001}.
\\
[-0.3cm]

Combining the four categories discussed above, the present analysis contains
$n_{\rm dat}=150$ top-quark cross-sections, to be compared with
$n_{\rm dat}=103$ in~\cite{Hartland:2019bjb}.
%
In Sect.~\ref{sec:dataset_dependence} we will quantify the
impact of the new top-quark
measurements by comparing two fits, one based on the dataset
of~\cite{Hartland:2019bjb} and one based on the extended top-quark dataset
included here.

\subsection{Higgs production and decay}
\label{sec:HiggsProductionDecay}

We now turn to the Higgs boson production and decay measurements.
We consider first inclusive cross-section measurements, presented as signal
strengths normalised to the SM predictions, and then differential
distributions and STXS measurements.

\paragraph{Signal strengths.} 
First of all, we consider the inclusive Higgs boson production signal strengths
$\mu_i^f$ measured by ATLAS and CMS from LHC Run I and Run II.
These  signal strengths are defined for each combination
of production and decay channels in terms of cross-section $\sigma_i$
and the branching fraction $B_f$ as
\be
\mu_i^f \equiv \frac{\sigma_i \times B_f}{\lp \sigma_i\rp_{\rm SM} \times
  \lp B_f \rp_{\rm SM}} = \mu_i \cdot \mu^f = \lp \frac{\sigma_i }{\lp \sigma_i\rp_{\rm SM} } \rp
\lp  \frac{  B_f}{ \lp B_f \rp_{\rm SM}}\rp \, ,
\ee
that is, as the ratio of the experimentally measured production cross-sections
in specific decay channels to the corresponding (state-of-the-art)
SM predictions.
%
These inclusive signal strengths can also be expressed as
\be
\label{eq:signal_strength_def}
\mu_i^f = \lp \frac{\sigma_i }{\lp \sigma_i\rp_{\rm SM} } \rp
\lp  \frac{  \Gamma(h \to f) }{  \Gamma(h \to f)\big|_{\rm SM} }\rp
\lp  \frac{  \Gamma(h \to {\rm all}) }{  \Gamma(h \to {\rm all})\big|_{\rm SM} }\rp^{-1} \, ,
\ee
in terms of the partial and total decay widths. The measurements of
signal strengths that we consider in the present analysis
are collected in Table~\ref{eq:input_datasets_higgsSS}.
In contrast to the differential distributions and STXS discussed below,
these signal strengths are typically extrapolated to the full phase space and
do not include selection or acceptance cuts.

%-------------------------------------------------------------------------------

%%%%%%%%%%%%%%%%%%%%%%%%%%%%%%%%%%%%%%%%%%%%%%%%%%%%%%%%%%%%%%%%%%%%%%%%%%%%%%%%
\begin{table}[t]
  \centering
  \scriptsize
   \renewcommand{\arraystretch}{1.65}
  \begin{tabular}{c|c|c|c|c|c}
 Dataset   &  $\sqrt{s},~\mathcal{L}$ & Info  &  Observables  & $n_{\rm dat}$ & Ref.   \\
    \toprule
    \multirow{2}{*}{ {\tt ATLAS\_CMS\_SSinc\_RunI} {\bf (*)}}  &\multirow{2}{*}{ {\bf 7+8 TeV, 20~fb$^{-1}$}}  &
    \multirow{2}{*}{Incl. $\mu_i^f$} &  $gg$F, VBF, $Vh$, $t\bar{t}h$
    &  \multirow{2}{*}{20}    &  \multirow{2}{*}{\cite{Khachatryan:2016vau} } \\
    &   &     & $h\to \gamma\gamma, VV, \tau\tau, b\bar{b}$   &  &    \\ \midrule
    %-----------------------------------------------------------------------------------------
    {\tt ATLAS\_SSinc\_RunI} {\bf (*)}  & {\bf 8 TeV, 20~fb$^{-1}$}  &
    Incl. $\mu^f_i$ &  $h\to Z\gamma, \mu\mu$
    & 2    & \cite{Aad:2015gba} \\ \midrule
    \midrule
    %-----------------------------------------------------------------------------------------
 \multirow{2}{*}{ {\tt ATLAS\_SSinc\_RunII} {\bf (*)}}  &\multirow{2}{*}{ {\bf 13 TeV, 80~fb$^{-1}$}}  &
 \multirow{2}{*}{Incl. $\mu_i^f$} &  $gg$F, VBF, $Vh$, $t\bar{t}h$
 &  \multirow{2}{*}{16}    &  \multirow{2}{*}{\cite{Aad:2019mbh} } \\
 &   &     & $h\to \gamma\gamma, WW, ZZ, \tau\tau, b\bar{b}$   &  &    \\ \midrule
 %-----------------------------------------------------------------------------------------
  %-----------------------------------------------------------------------------------------
 \multirow{2}{*}{ {\tt CMS\_SSinc\_RunII} {\bf (*)}}  &\multirow{2}{*}{ {\bf 13 TeV, 36.9~fb$^{-1}$}}  &
 \multirow{2}{*}{Incl. $\mu_i^f$} &  $gg$F, VBF, $Wh$, $Zh$ $t\bar{t}h$
 &  \multirow{2}{*}{24}    &  \multirow{2}{*}{\cite{Sirunyan:2018koj} } \\
 &   &     & $h\to \gamma\gamma, WW, ZZ, \tau\tau, b\bar{b}$   &  &    \\ 
    %-----------------------------------------------------------------------------------------
    \bottomrule
    \end{tabular}
  \caption{\small Same as Table~\ref{eq:input_datasets} now for
    the  measurements of the inclusive
    signal strenghts, Eq.~(\ref{eq:signal_strength_def}),
    in Higgs production and decay from the LHC Run I and Run II.
     \label{eq:input_datasets_higgsSS}
  }
\end{table}
%%%%%%%%%%%%%%%%%%%%%%%%%%%%%%%%%%%%%%%%%%%%%%%%%%%%%%%%%%%%%%%%%%%%%%%%%%%%%%%%%%%%%%%%%%%%%%%%%%%%


%-------------------------------------------------------------------------------

For the LHC Run I, we take into account the inclusive
measurements of  Higgs boson production and decay rates from the ATLAS and CMS
combination based on the full 7 and 8~TeV datasets~\cite{Khachatryan:2016vau}.
Specifically, we include the 20 measurements presented in Table~8
of~\cite{Khachatryan:2016vau}. These measurements correspond to five different
production channels ($gg{\rm F}$, VBF, $Wh$, $Zh$, $tth$) for five final
states ($\gamma\gamma$, $ZZ$, $WW$, $\tau\tau$, $b\bar{b}$), 
excluding those combinations that are either not measured with
a meaningful precision or not measured at all.
We account for the experimental correlations between the measured
signal strengths using the information provided in~\cite{Khachatryan:2016vau}.
In addition to these ATLAS+CMS combination results from Run I, we also include
two more signal strengths measurements from Run I, namely the ATLAS
constraints on the $Z\gamma$ and $\mu\mu$ decays from~\cite{Aad:2015gba}.

For the LHC Run II, we consider the ATLAS measurement of signal strengths
corresponding to an integrated luminosity of
$\mathcal{L}=80$~fb$^{-1}$~\cite{Aad:2019mbh}, and the CMS measurement
corresponding to an integrated luminosity of
$\mathcal{L}=35.9$~fb$^{-1}$~\cite{Sirunyan:2018koj}.
As in the case of the Run I signal strengths, we keep into account
correlations between the various production and final state combinations.
The ATLAS combination contains 16 signal strengths
for the $gg$F, VBF, $Vh$ and $t\bar{t}h$ production channels
and the $\gamma\gamma$, $ZZ$, $WW$, $\tau\tau$ and $b\bar{b}$
final states. As in the case of Run I, measurements are sometimes not available
for all final states for a given production channel, for example the
$h\to b\bar{b}$ decay is not available for $gg$F while $\tau\tau$ is not
provided in the case of $Vh$ associate production.
The CMS analysis contains 24 signal strengths measurements
in the $gg$F, VBF, $Wh$, $Zh$, and $t\bar{t}h$ production channels
for the same final states as in the ATLAS case. Results for the
$WW$, $ZZ$ and$\gamma\gamma$ final states are available for all production
channels, while for the other final states, $\mu\mu$, $\tau\tau$,
and $b\bar{b}$, signal strength measurements
are only available for specific production channels.
In total, we have $n_{\rm dat}=62$ measurements
of Higgs inclusive signal strengths from Runs I and II.

Concerning the theoretical calculations corresponding to these
measurements, the SM production cross-sections and decay branching fractions
are obtained from the associated experimental publications.
%
In turn,
these represent the
most updated available predictions, and are provided in the LHC Higgs
Cross-Section Working Group (HXSWG)
reports~\cite{Heinemeyer:2013tqa,deFlorian:2016spz,Dittmaier:2012vm}.
As in the case of top-quark production processes,
EFT calculations are obtained at NLO QCD
using {\tt MadGraph5\_aMC@NLO}~\cite{Alwall:2014hca} with the
{\tt SMEFT@NLO} model.
%
Additional details about the implementation of EFT
corrections to the Higgs signal strengths are provided
in App.~\ref{sec:signalstrenghts}.

\paragraph{Differential distributions and STXS.}
Table~\ref{eq:input_datasets_higgs} summarizes the experimental measurements of
differential distributions and STXS for Higgs boson production and decay at the
LHC considered in the present analysis.
%
Whenever one has a potential double counting between a
signal strength measurement and the corresponding differential distribution or
STXS measurement, we always select the latter, which provides more information
on the EFT parameter space due to its enhanced
kinematical sensitivity.

%-------------------------------------------------------------------------------

%%%%%%%%%%%%%%%%%%%%%%%%%%%%%%%%%%%%%%%%%%%%%%%%%%%%%%%%%%%%%%%%%%%%%%%%%%%%%%%%
\begin{table}[t]
  \centering
  \scriptsize
   \renewcommand{\arraystretch}{1.95}
  \begin{tabular}{c|c|c|c|c|c}
 Dataset   &  $\sqrt{s}, \mathcal{L}$ & Info  &  Observables  & $N_{\rm dat}$ & Ref   \\
 \toprule
  \multirow{2}{*}{ {\tt CMS\_H\_13TeV\_2015} {\bf (*)}}  &\multirow{2}{*}{ {\bf 13 TeV, 35.9~fb$^{-1}$}}  &
  $gg$F, VBF, $Vh$, $t\bar{t}h$ & \multirow{2}{*}{$d\sigma/dp_T^h$}    &   \multirow{2}{*}{9}    &  \multirow{2}{*}{ \cite{Sirunyan:2018sgc} } \\
     &  & $h\to ZZ,\gamma\gamma,b\bar{b}$
  &      &     &    \\
  \midrule
   \multirow{2}{*}{ {\tt ATLAS\_ggF\_13TeV\_2015} {\bf (*)}}  &\multirow{2}{*}{ {\bf 13 TeV, 36.1~fb$^{-1}$}}  &
  $gg$F, VBF, $Vh$, $t\bar{t}h$ &  \multirow{2}{*}{$d\sigma/dp_T^h$}    &   \multirow{2}{*}{9}    &  \multirow{2}{*}{ \cite{Aaboud:2018ezd} } \\
     &  & $h\to ZZ(\to 4l)$
  &      &     &    \\
  \midrule
   \midrule
    %-----------------------------------------------------------------------------------------
    \multirow{2}{*}{ {\tt ATLAS\_Vh\_hbb\_13TeV} {\bf (*)}}  &\multirow{2}{*}{ {\bf 13 TeV, 79.8~fb$^{-1}$}}  &
    \multirow{2}{*}{$Wh,~Zh$} &  $d\sigma^{\rm (fid)}/dp_T^W$   &  2    &  \multirow{2}{*}{\cite{Aaboud:2019nan} } \\
    &   &     & $d\sigma^{\rm (fid)}/dp_T^Z$   & 3  &    \\ \midrule
    %-----------------------------------------------------------------------------------------
     {\tt ATLAS\_ggF\_ZZ\_13TeV} {\bf (*)}  &{\bf 13 TeV, 79.8~fb$^{-1}$}  &
     $gg$F, $h\to ZZ$ &  $\sigma_{\rm ggF}(p_T^h,N_{\rm jets})$    &  6    &  \cite{Aad:2019mbh}  \\
     \midrule
    %-----------------------------------------------------------------------------------------
    {\tt CMS\_ggF\_aa\_13TeV} {\bf (*)}  &{\bf 13 TeV, 77.4~fb$^{-1}$}  &
    $gg$F, $h\to \gamma\gamma$ &   $\sigma_{\rm ggF}(p_T^h,N_{\rm jets})$    &  6    &  \cite{CMS:1900lgv}  \\
    %-----------------------------------------------------------------------------------------
%    \midrule
    %-----------------------------------------------------------------------------------------
%     \multirow{1}{*}{ {\tt CMS\_ggF\_tautau\_13TeV} {\bf (*)}}  &\multirow{1}{*}{ {\bf 13 TeV, 77.4~fb$^{-1}$}}  &
%    \multirow{1}{*}{$gg$F, $h\to \tau\tau$} &   $\sigma_{\rm ggF}(p_T^h,N_{\rm jets})$    &  5    &  \multirow{1}{*}{\cite{CMS:2019pyn} } \\
    %-----------------------------------------------------------------------------------------
%    \midrule
%    \midrule
%     \multirow{1}{*}{ {\tt ATLAS\_h\_ZZ\_13TeV\_RunII} {\bf (*)}}  &\multirow{1}{*}{ {\bf 13 TeV, 139~fb$^{-1}$}}  &
%     \multirow{1}{*}{$h\to ZZ\to 4l$} &  \multirow{1}{*}{$d\sigma^{\rm (fid)}/dp_T^{4l}$}   &  10    &  \multirow{1}{*}{\cite{ATLAS:2020wny} } \\
     %-----------------------------------------------------------------------------------------
%      \multirow{2}{*}{ {\tt ATLAS\_h\_aa\_13TeV\_RunII} {\bf (*)}}  &\multirow{2}{*}{ {\bf 13 TeV, 139~fb$^{-1}$}}  &
%  \multirow{2}{*}{$h\to \gamma\gamma$} &  $d\sigma/d|y_{\gamma\gamma}|$    &   \multirow{1}{*}{9}    &  \multirow{2}{*}{ \cite{ATLAS:2019jst} } \\
%     &  & 
%  &  $d\sigma/dp_T^{\gamma\gamma}$    &  17   &    \\
    \bottomrule
    \end{tabular}
  \caption{\small Same as Table~\ref{eq:input_datasets} for
   differential distributions and STXS for Higgs production and decay.
     \label{eq:input_datasets_higgs}
  }
\end{table}
%%%%%%%%%%%%%%%%%%%%%%%%%%%%%%%%%%%%%%%%%%%%%%%%%%%%%%%%%%%%%%%%%%%%%%%%%%%%%%%%%%%%%%%%%%%%%%%%%%%%


%-------------------------------------------------------------------------------

To being with, we consider the ATLAS and CMS differential distributions
in the Higgs boson kinematic variables obtained from the combination of the
$h\to \gamma\gamma$, $h\to ZZ$, and (in the CMS case)
$h \to b\bar{b}$ final states at 13~TeV based
on $\mathcal{L}=36$~fb$^{-1}$~\cite{Aaboud:2018ezd,Sirunyan:2018sgc}.
Specifically, we consider the differential distributions in the Higgs boson
transverse momentum $p_T^h$.
%
These distributions are particularly
sensitive probes of new physics, for instance through new particles circulating
in the gluon-fusion loop.

We also include the ATLAS measurement of the associated production of Higgs
bosons, $Vh$, in the $h\to b\bar{b}$ final state at
13~TeV~\cite{Aaboud:2019nan}.
%
These measurements, performed in kinematic
fiducial volumes defined in the simplified template cross-section framework,
correspond to an integrated luminosity of $\mathcal{L}=79.8$ fb$^{-1}$.
Specifically, here we include the data corresponding to the 5-POI
(parameters of interest) category, which refers to three cross-sections
for $Zh$ production in the bins $75 < p_T^Z < 150$ GeV,
$150 < p_T^Z < 250$ GeV, and $p_T^Z > 250$ GeV, and two cross-sections
for $Wh$ production, one for $150 < p_T^W < 250$ GeV and the other for
$p_T^W > 250$ GeV. Gauge bosons are reconstructed by means of their leptonic
decays.

Then we also include selected differential measurements presented in the ATLAS
Run II Higgs combination paper~\cite{Aad:2019mbh}.
Specifically, we include the measurements of Higgs production
in gluon fusion, $gg \to h$, in different bins of
$p_T^h$ and in the number of jets in the event.
These measurements are presented as $\sigma_i \times B_{ZZ}/B_{ZZ}^{\rm (SM)}$,
since the $ZZ$ branching fraction is used to normalise the data.
We include the 0-jet cross-section, the 1-jet cross-section
for $p_T^h < 60$ GeV, $60 \le p_T^h \le 120$ GeV, and
$120 \le p_T^h \le 200$ GeV, and the $\le 1$ jet and $\le 2$ jet cross-sections
for $p_T^h \ge 200$ GeV and $p_T^h<200$ GeV respectively.

Furthermore, we consider the differential Higgs boson production measurements
presented by CMS at 13~TeV based on an integrated luminosity of
$\mathcal{L}=77.4$~fb$^{-1}$ and corresponding to the final state
$\gamma\gamma$~\cite{CMS:1900lgv}.
The STXS measurements associated to different final-state topologies
and kinematic values such as $p_T^h$ are presented.
These inclusive measurements are dominated by the gluon-fusion production
channel. Note that the CMS diphoton measurement of~\cite{CMS:1900lgv}
supersedes~\cite{Aaboud:2018xdt}, which was based on the 2016 dataset only.

Whenever available, the
information on the experimental correlated systematic uncertainties is included.
As mentioned above, the SM theoretical predictions are taken
from the HXSWG reports~\cite{deFlorian:2016spz,
  Heinemeyer:2013tqa,Dittmaier:2012vm}.
%
In total, we include
$n_{\rm dat}=35$ measurements of differential cross-sections and STXS
on Higgs production and decay from the LHC Run II.

We note that additional Higgs production
and decay measurements have been recently presented by ATLAS and CMS based
on the full Run II luminosity of $\mathcal{L}=139$ fb$^{-1}$.
Two examples of these are the CMS measurement of the $p_T^h$ distribution in the
$h\to WW$ fully leptonic final state~\cite{Sirunyan:2020tzo} and the updated
ATLAS measurement of $Vh$ associated production in the $b\bar{b}$ final
state~\cite{Aad:2020jym}.
%
These measurements are however not expected to modify
significantly the results of the present analysis,
since the constraints they provide on the EFT parameter space are already
covered by other measurements, and  their inclusion is left for future work.

\subsection{Diboson production from LEP and the LHC}

We complement the constraints provided by the Higgs data with those
provided by diboson production cross-sections measured by LEP and the LHC.
The dataset is summarised in Table~\ref{eq:input_datasets_diboson}.
%
To begin with, we consider the LEP-2
legacy measurements of $WW$ production~\cite{Schael:2013ita}.
Specifically, we include the cross-sections differential
in $\cos\theta_W$ in four different bins in the center of
mass energy, from $\sqrt{s}=182$ GeV up to $\sqrt{s}=206$ GeV.
%
Here $\theta_{W}$ is defined as the polar angle of the produced
$W^-$ boson with respect to the incoming electron beam direction.
Each set of bins with a different center-of-mass energy
correspond to a different integrated luminosity, ranging
between $\mathcal{L}=163.9$ pb$^{-1}$ and 630.5 pb$^{-1}$.
For each value of $\sqrt{s}$, there are 10 bins in $\cos\theta_W$,
adding up to a total of $n_{\rm dat}=40$ data points.
The theoretical calculations of the SM predictions,
which include higher-order electroweak but not NLO QCD
corrections, are also taken from~\cite{Schael:2013ita}.
%
For this process, the squared terms in the EFT
proportional to
$c_ic_j/\Lambda^{-4}$ are small and will be neglected.

%-------------------------------------------------------------------------------

%%%%%%%%%%%%%%%%%%%%%%%%%%%%%%%%%%%%%%%%%%%%%%%%%%%%%%%%%%%%%%%%%%%%%%%%%%%%%%%%
\begin{table}[t]
  \centering
  \scriptsize
   \renewcommand{\arraystretch}{1.90}
  \begin{tabular}{c|c|c|c|c|c}
 Dataset   &  $\sqrt{s}, ~\mathcal{L}$ & Info  &  Observables  & $N_{\rm dat}$ & Ref   \\
    \toprule
    %-----------------------------------------------------------------------------------------
        {\tt LEP2\_WW\_diff} {\bf (*)}   & {\bf [182,~296] GeV}   & LEP-2 comb   & $d^2\sigma(WW)/dE_{\rm cm}d\cos\theta_{W}$  & 40  &  \cite{Schael:2013ita} \\ \toprule
    %-----------------------------------------------------------------------------------------
    {\tt ATLAS\_WZ\_13TeV\_2016} {\bf (*)}  & {\bf 13 TeV, 36.1~fb$^{-1}$}  &
    fully leptonic &  $d\sigma^{\rm (fid)}/dm_T^{WZ}$  &  6    & \cite{Aaboud:2019gxl}  \\
    \midrule
     %-----------------------------------------------------------------------------------------
    {\tt ATLAS\_WW\_13TeV\_2016} {\bf (*)}  &{\bf 13 TeV, 36.1~fb$^{-1}$}  &
    fully leptonic &  $d\sigma^{\rm (fid)}/dm_{e\mu}$  &  13    &  \cite{Aaboud:2019nkz} \\
     \midrule
   %-----------------------------------------------------------------------------------------
    {\tt CMS\_WZ\_13TeV\_2016} {\bf (*)}  & {\bf 13 TeV, 35.9~fb$^{-1}$}  &
    fully leptonic &  $d\sigma^{\rm (fid)}/dp_T^{Z}$  &  11    &  \cite{Sirunyan:2019bez} \\
    \bottomrule
    \end{tabular}
  \caption{\small Same as Table~\ref{eq:input_datasets} for
    the differential distributions of gauge boson pair
    production from LEP-2 and the LHC.
     \label{eq:input_datasets_diboson}
  }
\end{table}
%%%%%%%%%%%%%%%%%%%%%%%%%%%%%%%%%%%%%%%%%%%%%%%%%%%%%%%%%%%%%%%%%%%%%%%%%%%%%%%%%%%%%%%%%%%%%%%%%%%%


%-------------------------------------------------------------------------------

Concerning the LHC datasets, we include measurements of the differential
distributions for $W^{\pm}Z$ production at 13~TeV from
ATLAS~\cite{ATLAS-CONF-2018-034} and CMS~\cite{Sirunyan:2019bez} based on a
luminosity of $\mathcal{L}=36.1$ fb$^{-1}$. In both cases, the two gauge bosons
are reconstructed by means of the fully leptonic final state, whereby events
of the type  $WZ \to \ell^+ \ell^- \ell^{(')\pm}$ are selected.
%
The different leptonic final states are then combined into an inclusive
measurement. For the ATLAS measurement three fiducial distributions are
presented, respectively differential in $p_T^W$, $p_T^Z$ and $m_T^{WZ}$.
As indicated in Table~\ref{eq:input_datasets_diboson},
in this analysis, our baseline choice will be to include the $m_{T}^{WZ}$
distribution, which extends up to transverse masses of $m_{T}^{WZ}=600$ GeV.
In the case of the corresponding CMS measurement, normalised
differential distributions in $p_T^Z$, $m_{WZ}$,
$p_T^{W}$, and $p_T^{\rm jet,lead}$ are available.
Here the baseline will be the $p^Z_{T}$ distribution.

In addition to these measurements, we also consider
the differential distributions for $WW $production from ATLAS at 13~TeV
based on a luminosity of $\mathcal{L}=36.1$ fb$^{-1}$~\cite{Aaboud:2019nkz}.
Events are selected by requiring one electron
and one muon in the final state, corresponding to the decay
mode $WW \to e^\pm \nu \mu^\pm \nu$.
Several differential distributions in the fiducial
region are provided, including  $m_{e\mu}$,
$p_T^{e\mu}$ and $|y_{e\mu}|$.
%
Here our baseline choice will be the $m_{e\mu}$ distribution,
the invariant mass of the dilepton system, which reaches values
of up to $m_{e\mu}\simeq 1$~TeV.
%
The total number of  data points in the
LHC diboson category is $n_{\rm dat}=30$.

Other diboson measurements from the LHC have been presented but their EFT
interpretation is left for future work.
For instance, the data for the CMS differential distributions of $WW$ production
at 13 TeV based on $\mathcal{L}=36.1$ fb$^{-1}$~\cite{CMS-PAS-SMP-18-004} is
still preliminary.
ATLAS has presented recent measurements
of the differential cross-sections in four-lepton events in 13~TeV
based on $\mathcal{L}=139$~fb$^{-1}$~\cite{ATLAS:2020xtq}, though here the
measured
distributions receive contributions from single $Z$ and Higgs boson production,
in addition to those from $ZZ$ production.

The theoretical predictions for the SM cross-sections of these LHC diboson
processes are accurate to NNLO QCD and were computed with
{\tt MATRIX}~\cite{Grazzini:2017mhc}.
The EFT contributions for this process include NLO QCD corrections
and take into account the constraints from Eq.~(\ref{LEPconstraints})
to express the calculation in terms of only three Wilson
coefficients, one being the triple-gauge operator $c_{WWW}$
and the other two the purely bosonic coefficients
$c_{\varphi D}$ and $c_{\varphi W B}$.
%
This choice is ultimately arbitrary and has no physical implications;
any other two coefficients
out of Eq.~(\ref{eq:LEPconstrainedDoFs}) would lead to
the same results. Its only motivation is to facilitate the event generation of
the diboson processes.

\subsection{Dataset and theory overview and EFT sensitivity}

We conclude this section by presenting an overview of the datasets
considered (and of the corresponding theoretical
calculations), summarizing their dependence
on the EFT coefficients defined in Sect.~\ref{sec:smefttheory},
and
quantifying the sensitivity that each process
has on these coefficients by means of information geometry.

\paragraph{Dataset overview.}
%
In Table~\ref{eq:table_dataset_overview} we summarise the number of data points in our baseline dataset
for each of the data categories and processes considered in this analysis, as well
as the total per category and the overall total.
%
We include 150, 97, and 70 cross-sections from top-quark production, Higgs boson production
and decay, and diboson production
cross-sections from LEP and the LHC respectively in the baseline dataset,
for a total of 317 cross-section measurements.

%%%%%%%%%%%%%%%%%%%%%%%%%%%%%%%%%%%%%%%%%%%%%%%%%%%%%%%%
%%%%%%%%%%%%%%%%%%%%%%%%%%%%%%%%%%%%%%%%%%%%%%%%%%%%%%%%%%%%%%%%%%%%%%%%%%%%%%%%%%%%%%%
\begin{table}[t]
  \centering
  \small
   \renewcommand{\arraystretch}{1.30}
  \begin{tabular}{C{3.9cm}|C{6.6cm}|C{2cm}}
 Category   & Processes    &  $n_{\rm dat}$     \\
    \toprule
    \multirow{6}{*}{Top quark production}   &  $t\bar{t}$ (inclusive)   &  94  \\
    &  $t\bar{t}Z$, $t\bar{t}W$    & 14 \\
    &   single top (inclusive)   & 27 \\
    &  $tZ, tW$   &  9\\
    &  $t\bar{t}t\bar{t}$, $t\bar{t}b\bar{b}$    & 6 \\
    &  {\bf Total}    & {\bf 150 }  \\
    \midrule
    \multirow{3.3}{*}{Higgs production} & Run I signal strengths  &22   \\
    \multirow{3.1}{*}{and decay} & Run II  signal strengths  & 40  \\
    & Run II, differential distributions \& STXS  & 35  \\
    &  {\bf Total}    & {\bf 97}  \\
    \midrule
    \multirow{3}{*}{Diboson production} & LEP-2 &40   \\
     & LHC & 30  \\
    &  {\bf Total}    & {\bf 70}  \\
    \bottomrule
   Baseline dataset     & {\bf Total}      & {\bf 317}  \\
\bottomrule
  \end{tabular}
  \caption{\small The number of data points $n_{\rm dat}$ in our baseline dataset
    for each of the categories of processes considered here.
 \label{eq:table_dataset_overview}
}
\end{table}
%%%%%%%%%%%%%%%%%%%%%%%%%%%%%%%%%%%%%%%%%%%%%%%%%%%%%%%%%%%%%%%%%%%%%%%%%%%%%%%%



%%%%%%%%%%%%%%%%%%%%%%%%%%%%%%%%%%%%%%%%%%%%%%%%%%%%%%%%

\paragraph{Overview of theoretical calculations.}
%
Table~\ref{eq:table-processes-theory} displays a
summary of the theoretical calculations used for the 
description various datasets included in the 
present analysis. We indicate, for both the SM and the SMEFT contributions 
to the cross-sections, the perturbative accuracy and the codes used to 
produce the corresponding theoretical predictions.

%%%%%%%%%%%%%%%%%%%%%%%%%%%%%%%%%%%%%%%%%%%%%%%%%%%%%%%%
%-------------------------------------------------------------------------------
\begin{table}[htbp]
  \centering
  \footnotesize
  \renewcommand{\arraystretch}{1.80}
  \begin{tabular}{c|c|c|c|c}
  Category & Process 
  & SM   
  & Code/Ref  
  & SMEFT 
  \\
  \toprule
 \multirow{10}{*}{Top quark}  &   \multirow{2}{*}{$t\bar{t}$ (incl)}  
  & \multirow{2}{*}{NNLO QCD}   
  & {\tt MG5\_aMC} NLO  
  & \multirow{2}{*}{NLO QCD}  
  \\
\multirow{10}{*}{production}   &    
&  & + NNLO $K$-fact  
  &    
\\
\cmidrule(lr{0.7em}){2-5}
%-------------------------------------------------------
  &  \multirow{2}{*}{$t\bar{t}+V$} 
  & \multirow{2}{*}{NLO QCD} 
  & \multirow{2}{*}{{\tt MG5\_aMC} NLO}  
  & LO QCD   
  \\
  &  
  &   
  & & + NLO SM $K$-fact  
  \\
 \cmidrule(lr{0.7em}){2-5}
%-------------------------------------------------------
&    \multirow{2}{*}{single-$t$ (incl)} 
  & \multirow{2}{*}{NNLO QCD} 
  & {\tt MG5\_aMC} NLO  
  & \multirow{2}{*}{NLO QCD}  
  \\
  &   
 & & + NNLO $K$-fact     
  &
  \\
  \cmidrule(lr{0.7em}){2-5}
%-------------------------------------------------------
  &  \multirow{2}{*}{$t+V$} 
  & \multirow{2}{*}{NLO QCD} 
  & \multirow{2}{*}{{\tt MG5\_aMC} NLO}  
  & LO QCD   
  \\
  &  
  &   
 & & + NLO SM $K$-fact  
  \\
  \cmidrule(lr{0.7em}){2-5}
 %----------------------------------------------
 &   \multirow{2}{*}{$t\bar{t}t\bar{t},~t\bar{b}t\bar{b}$} 
  & \multirow{2}{*}{NLO QCD} 
  & \multirow{2}{*}{{\tt MG5\_aMC} NLO}  
  & LO QCD  
  \\
  &  
  &   
 & & + NLO SM $K$-fact  
  \\
  %---------------------------------------------------------
   \midrule
 \multirow{10}{*}{Higgs production}  &   \multirow{2}{*}{$gg\to h$}  
  & \multirow{1}{*}{NNLO QCD +}   
  &\multirow{2}{*}{ HXSWG}
  & \multirow{2}{*}{NLO QCD}  
  \\
\multirow{10}{*}{and decay}   &    
&  \multirow{1}{*}{NLO EW}   & 
  &    
\\
 \cmidrule(lr{0.7em}){2-5}
%-------------------------------------------------------
   &   \multirow{2}{*}{VBF}  
  & \multirow{1}{*}{NNLO QCD +}   
  &\multirow{2}{*}{ HXSWG}
  & \multirow{2}{*}{LO QCD}  
 \\
 &    
&  \multirow{1}{*}{NLO EW}   & 
  &    
 \\
  \cmidrule(lr{0.7em}){2-5}
%-------------------------------------------------------
   &   \multirow{2}{*}{$h+V$}  
  & \multirow{1}{*}{NNLO QCD +}   
  &\multirow{2}{*}{ HXSWG}
  & \multirow{2}{*}{NLO QCD}  
  \\
   &    
&  \multirow{1}{*}{NLO EW}   & 
  &    
\\
 \cmidrule(lr{0.7em}){2-5}
 %-------------------------------------------------------
  &   \multirow{2}{*}{$ht\bar{t}$}  
  & \multirow{1}{*}{NNLO QCD +}   
  &\multirow{2}{*}{ HXSWG}
  & \multirow{2}{*}{NLO QCD}  
  \\
   &    
&  \multirow{1}{*}{NLO EW}   & 
  &    
\\
 \cmidrule(lr{0.7em}){2-5}
 %-------------------------------------------------------
  %-------------------------------------------------------
  &   \multirow{2}{*}{$h\to X$}  
  & \multirow{1}{*}{NNLO QCD +}   
  &\multirow{2}{*}{ HXSWG}
  & \multirow{1}{*}{NLO QCD ($X=b\bar{b}$)}  
 \\
 &    
&  \multirow{1}{*}{NLO EW}   & 
  &    \multirow{1}{*}{LO QCD ($X\ne b \bar{b} $)} 
\\
%-------------------------------------------------------
 \midrule
 \multirow{4}{*}{Diboson}  &   \multirow{2}{*}{$e^+e^- \to W^+W^-$}  
  & \multirow{1}{*}{NNLO QCD +}   
  &\multirow{2}{*}{ LEP EWWG}
  & \multirow{2}{*}{LO QCD}  
  \\
\multirow{4}{*}{production}   &    
&  \multirow{1}{*}{NLO EW}   & 
  &    
\\
\cmidrule(lr{0.7em}){2-5}
\multirow{4}{*}{}  &   \multirow{2}{*}{$pp \to VV'$}  
  & \multirow{2}{*}{NNLO QCD}   
& 
\multirow{2}{*}{ {\tt MATRIX}
}
  & \multirow{2}{*}{NLO QCD}  
  \\
\multirow{4}{*}{}   &    
&     &  
  &    
\\
  \bottomrule
 \end{tabular}
 \caption{\small Summary of the theoretical calculations used for the 
   description various datasets included in the 
   present analysis. We indicate, for both the SM and the SMEFT contributions 
   to the cross-sections, the perturbative accuracy and the codes used to 
   produce the corresponding theoretical predictions.
   %
   In all cases, the EFT cross-sections are evaluated with {\tt MG5\_aMC} interfaced
   to {\tt SMEFT@NLO}.
   %
   See the text for more details and the corresponding references.
 }
  \label{eq:table-processes-theory}
\end{table}
%-------------------------------------------------------------------------------

%%%%%%%%%%%%%%%%%%%%%%%%%%%%%%%%%%%%%%%%%%%%%%%%%%%%%%%%

{ 
\paragraph{Electroweak scheme.}
%
The theoretical calculations presented
in this work are based on
the $(G_F, m_Z, m_W)$ electroweak scheme,
which is the default in the
{\tt SMEFTatNLO} model.
%
The corresponding values of the SM parameters are
set to be the following:
  \begin{equation}
\begin{aligned}
     m_\mathrm{W}  &= 80.352~{\rm GeV} \text{,} &
\Gamma_\mathrm{W}  &= 2.084~{\rm GeV} \text{,} &
     m_t           &= 172.5~{\rm GeV} \text{,} \\
m_\mathrm{Z}       &= 91.1535~{\rm GeV} \text{,} &
\Gamma_\mathrm{Z}  &= 2.4943~{\rm GeV} \text{,} &
\Gamma_t           &= 1.37758~{\rm GeV} \text{,} \\
m_\mathrm{H}       &= 125.0~{\rm GeV} \text{,} &
\Gamma_\mathrm{H}  &= 4.07468 \times 10^{-3}~{\rm GeV} \text{,} &
G_\mu              &= 1.166378 \times 10^{-5}~{\rm GeV}^{-2} \text{.}
\end{aligned}
\end{equation}
}

\paragraph{Dependence on the EFT coefficients.}
%
In order to interpret the results of the global EFT analyses
which will be presented in Sect.~\ref{sec:results},
it is useful to collect the dependence
of the various datasets described
in this section with respect to the degrees
of freedom defined in Sect.~\ref{sec:smefttheory}.
%
Table~\ref{table:operatorprocess} indicates
which EFT coefficients contribute to the theoretical description of each of the
processes considered in this analysis.
%
Recall that the 16 coefficients listed in Eq.~(\ref{eq:LEPconstrainedDoFs}) are related
among them by the EWPO relations, and that only two of them are independent.

In Table~\ref{table:operatorprocess} we display
from top to bottom the coefficients associated to
the  two-light-two-heavy, four-heavy, four-lepton, two-fermion plus bosonic,
and purely bosonic dimension-six operators.
%
The Higgs measurements are separated between the Run I and Run II datasets,
and in the latter case also between signal strengths and differential
distributions and STXS.
%
A check mark outside (inside) brackets indicates that a given
process constrains the corresponding coefficients
starting at $\mathcal{O}( \Lambda^{-2})$ 
($\mathcal{O}( \Lambda^{-4})$) at LO.
%
Entries labelled with (b) indicate that the sensitivity
to the associated coefficients enters via bottom-initiated
processes, which arise due to contributions from the $b$-PDF
in the 5FNS adopted here.

%%%%%%%%%%%%%%%%%%%%%%%%%%%%%%%%%%%%%%%%%%%%%%%%%%%%%%%%%%%%%
%-------------------------------------------------------------------------------
\begin{table}[p]
 \centering
 \scriptsize
 \renewcommand{\arraystretch}{1.25}
 \begin{tabular}{l|l||C{0.8cm}|C{0.8cm}|C{0.8cm}|C{0.8cm}|C{0.8cm}||C{1.1cm}|C{1.1cm}|C{1.3cm}|C{0.8cm}}
   Class  &
   DoF &
   $\,t\bar{t}\,$
 & $t\bar{t}V$   
 & $\,\, t\,\,$ 
 & $tV$ 
 & $t\bar{t}Q\bar{Q}$ 
 & $h$ ($\mu_i^{f}$, Run-I)
 & $h$ ($\mu_i^{f}$, Run-II)
 & $h$ (STXS, Run-II)
 & $VV$
 \\
 \toprule
 %---------------------------------------------------------------------------------------------------
\multirow{13.5}{*}{2-heavy-} &  $c_{Qq}^{1,8}$ &  \checkmark  &  \checkmark  &    &    & \checkmark    & \checkmark     &  \checkmark     &    \checkmark  &       \\
\multirow{13.5}{*}{2-light} &  $c_{Qq}^{1,1}$ & (\checkmark)   & (\checkmark)  &    &     &  \checkmark   &  (\checkmark)   & (\checkmark)     & (\checkmark)    &       \\
 &  $c_{Qq}^{3,8}$ &   \checkmark  &  \checkmark  &  (\checkmark)   &  (\checkmark)    &   \checkmark   &  \checkmark    &   \checkmark    &   \checkmark   &       \\
 &  $c_{Qq}^{3,1}$ &   (\checkmark)  &    (\checkmark)   &  \checkmark    &  \checkmark    &  \checkmark   & ( \checkmark)      &   (\checkmark)   &  ( \checkmark)   &  \\
 &  $c_{tq}^{8}$ &  \checkmark   &  \checkmark  &    &      & \checkmark    &  \checkmark    &  \checkmark     &  \checkmark    &       \\
 &   $c_{tq}^{1}$&   (\checkmark)  &  (\checkmark)  &    &     & \checkmark & ( \checkmark)    &   (\checkmark)   & ( \checkmark)     &           \\
 &  $c_{tu}^{8}$ &   \checkmark  &   \checkmark   &     &     &  \checkmark   &    \checkmark   &  \checkmark    &  \checkmark &     \\
 &  $c_{tu}^{1}$ &    (\checkmark) &  (\checkmark)  &    &     &   \checkmark  &  ( \checkmark)   &   (\checkmark)    & ( \checkmark)     &       \\
 &  $c_{Qu}^{8}$ &   \checkmark  &  \checkmark  &    &     &   \checkmark  &  \checkmark    &    \checkmark   &  \checkmark    &       \\
 &  $c_{Qu}^{1}$&   (\checkmark)  &  (\checkmark)  &    &     &   \checkmark  & ( \checkmark)    &  (\checkmark)     &  ( \checkmark)    &       \\
 &  $c_{td}^{8}$ &   \checkmark  &  \checkmark  &    &     &   \checkmark  &   \checkmark   &  \checkmark     &  \checkmark    &       \\
 &  $c_{td}^{1}$ &   (\checkmark)  &   (\checkmark) &    &     &   \checkmark  &  ( \checkmark)   &  (\checkmark)     & ( \checkmark)    &       \\
 &  $c_{Qd}^{8}$ &   \checkmark  &  \checkmark  &    &     &   \checkmark  &   \checkmark   &    \checkmark   &  \checkmark    &       \\
&  $c_{Qd}^{1}$ &   (\checkmark)  &  (\checkmark)  &    &     &   \checkmark  &  ( \checkmark)   &    (\checkmark)   &  ( \checkmark)    &       \\
\midrule
%---------------------------------------------------------------------------------------------------
\multirow{5}{*}{4-heavy} & $c_{QQ}^1$ &    &   &    &     &  \checkmark   &     &      &     &       \\
 &  $c_{QQ}^8$ &    &   &    &     &  \checkmark   &     &      &     &       \\
 &  $c_{Qt}^1$ &    &   &    &     &  \checkmark   &     &      &     &       \\
 &  $c_{Qt}^8$ &    &   &    &     &  \checkmark   &     &      &     &       \\
&  $c_{tt}^1$ &    &   &    &     &  \checkmark   &     &      &     &       \\
\midrule
%---------------------------------------------------------------------------------------------------
4-lepton &  $c_{ll}$ &    &   & \checkmark   & \checkmark    &    &  \checkmark   & \checkmark     & \checkmark    & \checkmark      \\
\midrule
%---------------------------------------------------------------------------------------------------
\multirow{18.5}{*}{2-fermion} &  $c_{t\varphi}$ &    &   &    &     &     &   \checkmark   &   \checkmark    &   \checkmark   &       \\
\multirow{18.5}{*}{+bosonic} &  $c_{tG}$ & \checkmark  & \checkmark   &    &     & \checkmark    &   \checkmark   &   \checkmark    &  \checkmark    &       \\
&  $c_{b\varphi}$ &   &   &    &     &     &  \checkmark   &   \checkmark    & \checkmark (b)    &       \\
&  $c_{c\varphi}$ &    &   &    &     &     &   \checkmark  &  \checkmark    &     &       \\
&  $c_{\tau\varphi}$ &    &   &    &     &     &   \checkmark  &   \checkmark   &     &       \\
&  $c_{tW}$ & \checkmark   &  & \checkmark   &  \checkmark   &     & \checkmark    &  \checkmark    &     &       \\
&  $c_{tZ}$ &    &  \checkmark &    &  \checkmark   &     &   \checkmark  &  \checkmark    &     &       \\[0.1cm]
&  $c_{\varphi Q}^{(3)}$ &    & \checkmark (b)  &  \checkmark  & \checkmark    &     & \checkmark (b)    &  \checkmark (b)    & \checkmark (b)   &       \\[0.1cm]
&  $c_{\varphi Q}^{(-)}$ &    & \checkmark  &    &  \checkmark   &     &  \checkmark   &   \checkmark   &   \checkmark (b)  &       \\
&  $c_{\varphi t}$ &    &  \checkmark   &    & \checkmark    &     & \checkmark    & \checkmark     &     &       \\[0.1cm]
&  $c_{\varphi l_i}^{(1)}$ &    &   &    &     &    &  \checkmark   &  \checkmark    &     & \checkmark      \\[0.1cm]
&  $c_{\varphi l_i}^{(3)}$ &    &   &\checkmark    & \checkmark    &    & \checkmark    &  \checkmark    & \checkmark    &  \checkmark     \\
&  $c_{\varphi e}$ &    &   &    &     &    &   \checkmark  &  \checkmark    &     &    \checkmark   \\
& $c_{\varphi \mu}$  &    &   &    &     &    &  \checkmark    & \checkmark     &    &       \\
& $c_{\varphi \tau}$  &    &   &    &     &    &    \checkmark   &  \checkmark    &   &       \\[0.1cm]
& $c_{\varphi q}^{(3)}$  &    &  \checkmark & \checkmark   &  \checkmark   &    &  \checkmark   &  \checkmark    & \checkmark    &   \checkmark    \\[0.1cm]
&  $c_{\varphi q}^{(-)}$ &    & \checkmark  &    & \checkmark    &    &  \checkmark   &  \checkmark    & \checkmark   &  \checkmark     \\
& $c_{\varphi u}$  &    &  \checkmark &    &  \checkmark   &    &    \checkmark &  \checkmark    &  \checkmark   &   \checkmark    \\
& $c_{\varphi d}$  &    &  \checkmark &    &  \checkmark   &    &  \checkmark   &  \checkmark    &   \checkmark  &    \checkmark   \\
\midrule
%---------------------------------------------------------------------------------------------------
\multirow{6.5}{*}{purely} &  $c_{\varphi G}$ &    &   &    &     &     &  \checkmark   & \checkmark     &  \checkmark   &       \\
\multirow{6.5}{*}{bosonic} &  $c_{\varphi B}$ &    &   &    &     &     &   \checkmark  &  \checkmark    &  \checkmark   &       \\
&  $c_{\varphi W}$ &    &   &    &     &     &  \checkmark   &  \checkmark    & \checkmark    &       \\
&  $c_{\varphi d}$ &    &   &    &     &     &   \checkmark  &   \checkmark   &  \checkmark   &       \\
&  $c_{\varphi D}$ &    & \checkmark  & \checkmark   &   \checkmark  &     &  \checkmark   &  \checkmark    &  \checkmark   &    \checkmark   \\
&  $c_{\varphi W B}$ &    & \checkmark  &  \checkmark  &   \checkmark  &     &   \checkmark  & \checkmark     & \checkmark    &   \checkmark    \\
&  $c_{WWW}$  &    &   &    &     &     &     &      &     & \checkmark      \\
%---------------------------------------------------------------------------------------------------
 \bottomrule
 \end{tabular}
 \caption{\small Overview indicating which EFT coefficients
   contribute to the theoretical description of each of the processes
   considered in this global analysis.
}
\label{table:operatorprocess}	
\end{table}
%-------------------------------------------------------------------------------

%%%%%%%%%%%%%%%%%%%%%%%%%%%%%%%%%%%%%%%%%%%%%%%%%%%%%%%%%%%%%

Several observations can be drawn from this table.
%
First of all, we observe that the four-heavy coefficients are constrained only by the $t\bar{t}Q\bar{Q}$ production
data, either $t\bar{t}t\bar{t}$ or $t\bar{t}b\bar{b}$.
%
Such measurements also depend on the 2-light-2-heavy operators, as well as on $c_{tG}$, although in practice
this correlation is  small.
%
Furthermore, the four-heavy coefficients are essentially left undetermined at
$\mathcal{O}\lp \Lambda^{-2}\rp$, and can only be meaningfully constrained only
the quadratic corrections are accounted for.
%
One can also note how the two-light-two-heavy operators are constrained by top-quark pair production
(inclusive and in association with vector bosons) as well as by the $t\bar{t}h$ production measurements.
%
As will be shown below, by far the dominant constraints on these coefficients
arise from the differential
distributions in inclusive top
quark pair production.

Concerning the two-fermion operators, most of them are constrained both by top and by Higgs production
process.
%
Recall that the top and Higgs sectors are connected, among others,
by means of the gluon-fusion production process (with its virtual
top-quark loop) as well as by $t\bar{t}h$ associated production.
%
In particular, we note that $c_{t\varphi}$, which modifies the top
Yukawa coupling,
is constrained by these Higgs production measurements.
%
The purely bosonic operators exhibit sensitivity only to Higgs and diboson processes, since
these do not affect the properties of top quarks.
%
The diboson data is uniquely sensitive to the triple-gauge coefficient
$c_{WWW}$, which modifies the triple (and quartic) electroweak gauge couplings,
as well as to $c_{\varphi D}$ and $c_{\varphi WB}$, which  are also constrained by Higgs data.

\paragraph{The Fisher matrix and information geometry.}
\label{sec:information}
The information presented in Table~\ref{table:operatorprocess} does not allow one to compare
the sensitivity brought in by different datasets on a given EFT coefficient.
%
To achieve this, here we adopt the ideas underlying information geometry~\cite{Brehmer:2017lrt}
and define the Fisher information  matrix $I_{ij}$ as
\be
\label{eq:FisherDef}
I_{ij}\lp {\boldsymbol c} \rp = -{\rm E}\lc \frac{\partial^2 \ln f \lp {\boldsymbol \sigma}_{\rm exp}|
{\boldsymbol c} \rp}{\partial c_i \partial c_j} \rc \, , \qquad i,j=1,\ldots,n_{\rm op} \, ,
\ee
where ${\rm E}\lc~\rc$ indicates the expectation value and
$ f \lp {\boldsymbol \sigma}_{\rm exp}|{\boldsymbol c} \rp$ indicates the relation
between a set of experimental measurements and the assumed true values of
the EFT coefficients
$ {\boldsymbol c}$.
%
The covariance matrix in the EFT parameter space, $C_{ij} \lp {\boldsymbol c} \rp$,
is then bounded by the Fisher information matrix:
\be
C_{ij} \ge \lp I^{-1}\rp_{ij} \, ,
\ee
which is known as the Cramer-Rao bound.
%
The diagonal entries of the Cramer-Rao bound are
$C_{ii} =\lp \delta c_i\rp^2 \ge \lp I^{-1}\rp_{ii}$ and indicate that the smallest possible
uncertainty achievable on the coefficient $c_i$  given the input
data is $\delta{c_i}|^{\rm (best)}=\sqrt{ (I^{-1})_{ii}}$.

For a given set of EFT
coefficients, comparing the values of $I_{ij}$ between different datasets highlights
those which provide the highest information.
%
The larger the entries of the Fisher matrix, the better (smaller uncertainty) that
these  coefficients can be constrained by the data considered.
%
The Fisher information matrix can also be understood as a metric in
model space.
%
If one has two sets of coefficients $\boldsymbol{c}_a$ and $\boldsymbol{c}_b$,
corresponding to two different points in the EFT parameter space,
then the local distance between them is defined as
\be
d_{\rm loc}\lp \boldsymbol{c}_a,\boldsymbol{c}_b \rp =  \lc \sum_{i,j}
(\boldsymbol{c}_a-\boldsymbol{c}_b)_i I_{ij}(\boldsymbol{c}_a)
(\boldsymbol{c}_a-\boldsymbol{c}_b)_j\rc^{1/2} \, ,
\ee
a feature which provides a robust method to quantify how (di)similar
are two points in this model space.

If one has $n_{\rm dat}$ experimental measurements
$\sigma_m^{\rm (exp)}$ whose theoretical predictions depend on $n_{\rm op}$ coefficients
${\boldsymbol c}$,
assuming that these measurements are Gaussianly distributed, one has
\be
\label{eq:fisherf}
f \lp {\boldsymbol \sigma}_{\rm exp})|
{\boldsymbol c} \rp = \prod_{m=1}^{n_{\rm dat}}\frac{1}{\sqrt{2\pi \delta_{{\rm exp},m}^2}}
\exp \lp -\frac{ \lp \sigma_m^{\rm (exp)}-\sigma_m^{\rm (th)}({\boldsymbol c})\rp^2  }{ 2\delta_{{\rm exp},m}^2}\rp \, ,
\ee
where $\delta_{{\rm exp},m}$ stands for the total experimental uncertainty associated to this
cross-section measurement.
%
Here we neglect for simplicity the point-by-point correlations, the extension
to the full correlation covariance matrix is straightforward.
%
The theoretical predictions that enter Eq.~(\ref{eq:fisherf})  include the  SM contribution
as well as the terms linear and quadratic on the Wilson coefficients,
\be
\label{eq:quadraticTHform}
\sigma_m^{\rm (th)}({\boldsymbol c})= \sigma_m^{\rm (sm)} + \sum_{i=1}^{n_{\rm op}}c_i\sigma^{(\rm eft)}_{m,i} +
\sum_{i<j}^{n_{\rm op}}c_ic_j \sigma^{(\rm eft)}_{m,ij} \, ,
\ee
where we assume $\Lambda=1$ TeV, so that one can write
\be
-\ln f \lp {\boldsymbol \sigma}_{\rm exp}|
{\boldsymbol c} \rp = \sum_{m=1}^{n_{\rm dat}} \frac{1}{2\delta_{{\rm exp},m}^2} \lp
\lp \sigma_m^{\rm (exp)} - \sigma_m^{\rm (sm)}\rp - \sum_{i=1}^{n_{\rm op}}c_i\sigma^{(\rm eft)}_{m,i} -
\sum_{i<j}^{n_{\rm op}}c_ic_j \sigma^{(\rm eft)}_{m,ij} \rp^2 + A \, ,
\ee
where $A$ is a constant that does not depend on the value of the coefficients,
and thus the Fisher information matrix can be evaluated using Eq.~(\ref{eq:FisherDef}) to yield
\be
\label{eq:fisherinformation}
I_{ij} = {\rm E}\Bigg[ \sum_{m=1}^{n_{\rm dat}} \frac{1}{\delta_{{\rm exp},m}^2}  \Bigg( \sigma_{m,ij}\lp
  \sigma_m^{\rm (th)}-\sigma_m^{\rm (exp)} \rp  
  + \lp  \sigma^{\rm (eft)}_{m,i} + \sum_{l=1}^{n_{\rm op}}
  c_l \sigma^{\rm (eft)}_{m,il} \rp
  \lp \sigma^{\rm (eft)}_{m,j}+ \sum_{l'=1}^{n_{\rm op}} 
 c_{l'}  \sigma_{m,jl'} \rp \Bigg)\Bigg] \, ,
\ee
where this expectation value can be evaluated  by averaging over the
$N_{\rm rep}$ replicas (or $N_{\rm spl}$ samples) that provide a sampling of the probability density in the space
of coefficients within our approach, see also Sect.~\ref{sec:fitsettings}.
 
The Fisher information matrix becomes specially simple if we restrict
 ourselves to the linear approximation~\cite{Ellis:2018gqa}, i.e. $\sigma_{m,ij}=0$, since in this case
\be
\label{eq:fisherinformation2}
I_{ij} = \sum_{m=1}^{n_{\rm dat}} \frac{\sigma^{\rm (eft)}_{m,i}\sigma^{\rm (eft)}_{m,j}}{\delta_{{\rm exp},m}^2} \, ,
\ee
which is independent of the values of the coefficients $\boldsymbol{c}$
and therefore of the actual fit results.
%
The diagonal entries of the Fisher matrix $I_{ii}$ are then given by the
sum over a given dataset (or group of processes) of the square of
the linear
EFT cross-sections over the total experimental uncertainty.
%
In the specific case of one-parameter fits, the Cramer-Rao bound reads
\be
\label{cramerraobound}
\delta c_i  \ge \lp \sum_{m=1}^{n_{\rm dat}} \frac{\sigma^{\rm (eft)}_{m,i}\sigma^{\rm (eft)}_{m,j}}{\delta_{{\rm exp},m}^2}   \rp ^{-1/2} \,,\qquad  i=1,\ldots, n_{\rm op}\, ,
\ee
which, provided that the sum is over the global dataset, can be used to cross-check
the results of individual (one-parameter) fits.

One should emphasize that the absolute size of the entries of the Fisher matrix does not
contain physical information: one is always allowed to redefine the overall normalisation
of an operator such that $c_i\sigma^{(\rm eft)}_{m,i} = c_i'\sigma'^{(\rm eft)}_{m,i}$, with
$c_i' = B_i c_i$ and $\sigma^{(\rm eft)}_{m,i} =\sigma'^{(\rm eft)}_{m,i}/B_i$
with $B_i$ being arbitrary constants.
%
However, for a given operator the relative value of $I_{ii}$ between two groups of processes is independent
of this choice of normalisation and
thus conveys meaningful information.
%
For this reason, in the following we present results for the Fisher information matrix normalised
such that 
the sum of the diagonal entries associated to a given EFT coefficient adds up to a fixed reference value
which is taken to be 100.

Fig.~\ref{fig:FisherMatrix} displays the values of the diagonal entries of the Fisher
information matrix, Eq.~(\ref{eq:fisherinformation}), evaluated for
the same groups of processes as in Table~\ref{table:operatorprocess}.
%
The normalisation is such that the sum of the entries associated to each coefficients
adds up to 100.
%
We show results for the Fisher information both at the linear level, Eq.~(\ref{eq:fisherinformation2}),
and with the quadratic corrections included, Eq.~(\ref{eq:fisherinformation}),
in the left and right panels respectively.
%
The entries in blue indicate those groups of processes
which provide more than 75\% of the information on the corresponding
EFT coefficient.
%
Entries in grey indicate relative contributions
of less than 10\%.
%
As mentioned above, the sum of the entries over columns does not contain a physical
      interpretation.
  
%%%%%%%%%%%%%%%%%%%%%%%%%%%%%%%%%%%%%%%%%%%%%%%%%%%%%%%%%%%%%%%%%%%%%
\begin{figure}[htbp]
  \begin{center}
    \includegraphics[width=0.99\linewidth]{plots_v2/Fisher_heat.pdf}
    \caption{\small The values of the diagonal entries of the Fisher
      information matrix, Eq.~(\ref{eq:fisherinformation}), evaluated for
      the same groups of processes as in Table~\ref{table:operatorprocess} (except
      with the charge asymmetry $A_C$ data considered separately).
      %
      The normalisation here is such that the sum of the entries associated to each EFT
      coefficient adds up to 100.
      %
      We show results for the Fisher information matrix
      both at the linear level, Eq.~(\ref{eq:fisherinformation2}),
      and with the quadratic corrections included,  left and right panels respectively.
      %
      For entries in the heat map larger than 10, we also indicate the corresponding
      numerical values.
      %
      {  The quadratic Fisher information matrix (right panel)
        is evaluated
        using the best-fit values of the corresponding global baseline
      fit, to be presented in Sect.~5.}
     \label{fig:FisherMatrix} }
  \end{center}
\end{figure}
%%%%%%%%%%%%%%%%%%%%%%%%%%%%%%%%%%%%%%%%%%%%%%%%%%%%%%%%%%%%%%%%%%%%%%

The information contained in Fig.~\ref{fig:FisherMatrix} is consistent with that of
Table~\ref{table:operatorprocess}, but now we can identify, for each coefficient, which datasets
provide the dominant constraints.
%
For instance, one observes that the two-light-two-heavy operators are overwhelmingly constrained
by inclusive top quark pair production data, except for $c_{Qq}^{3,1}$ for
which single top is the most important set of processes.
%
At the linear level, the information on the two-light-two-heavy coefficients provided
by the differential distributions and by the charge asymmetry $A_C$ data is comparable,
while the latter is less important in the quadratic fits.
%
In the case of the two-fermion operators, the leading constraints typically arise from Higgs data, in particular
from the Run II signal strengths measurements, and then to a lesser extent from the Run I data
and the Run II differential distributions.
%
Two  exceptions are $c_{\varphi t}$, which at the linear level (but not at the quadratic one)
is dominated by $t\bar{t}V$, and the chromo-magnetic operator $c_{tG}$, for which inclusive
$t\bar{t}$ production is most important.
%
Also for the purely bosonic operators the Higgs data provides most of the information,
except for $c_{WWW}$, as expected since this operator is only accessible in
diboson processes.
%
Furthermore, one observes that the $\mathcal{O}\lp \Lambda^{-4}\rp$ corrections induce in most
cases a moderate change in the Fisher information
matrix, but in others they can significantly alter
the balance between processes.
%
As a representative example, the two-fermion operators
such as $c_{\varphi t}$ and $c_{\varphi Q}^{(3)}$
become dominated by the Higgs data only once quadratic corrections are accounted for.

Another relevant application of the Fisher information matrix is the determination
of optimal directions in the EFT parameter space by means of principal component
analysis (PCA), and in particular the assessment of whether or not the coefficients basis adopted
for the fit contains flat directions.
%
We will discuss this related application in the Sect.~\ref{sec:pca}.
