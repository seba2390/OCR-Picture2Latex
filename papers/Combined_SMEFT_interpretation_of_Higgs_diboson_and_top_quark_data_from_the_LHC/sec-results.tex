\section{Results}
\label{sec:results}

We now present the main results of this work:
the determination of the best-fit values,
confidence level intervals, and posterior probability distributions
associated to the $n_{\rm op}=50$ EFT coefficients (of which 36 are independent) listed in
Table~\ref{tab:operatorbasis} from the global
interpretation of Higgs, top quark, and diboson cross-section measurements.
%
As motivated in Sect.~\ref{sec:fitsettings}, the results
shown here have been obtained with the NS approach, and we have verified that
equivalent results are obtained with the MCfit method.

First of all, we discuss the quality of the fit,
both for the total and for individual
datasets.
%
Second, we present the bounds and posterior probability distributions
for the various EFT coefficients, assess their consistency with the
SM hypothesis, and determine their pattern of cross-correlations.
%
Third, we study the dependence of our results on the choice of input dataset,
in particular with fits based only on top or Higgs data, as well as on that of the theory settings,
where the impact of the NLO QCD corrections to the EFT cross-sections is quantified.
%
Finally, we present EFT fits  in the top-philic scenario, where the parameter space
is restricted by constraints motivated by specific UV-complete models.
%
The comparison between SM and SMEFT theory predictions
with the experimental dataset used as input to
the fit is then collected in App.~\ref{sect:app_comparison_data}.

\subsection{Fit quality}

To begin with, we investigate the quality of the fit in terms
of the $\chi^2$ values for the individual
datasets as well as for the global one.
%
The values that will be provided here correspond to
a modified version of Eq.~(\ref{eq:chi2definition2}),
specifically
\begin{equation}
  \chi^2 \equiv \frac{1}{n_{\rm dat}}\sum_{i,j=1}^{n_{\rm dat}}\lp 
  \la \sigma^{(\rm th)}_i\lp {\boldsymbol c}^{(k)} \rp \ra
  -\sigma^{(\rm exp)}_i\rp ({\rm cov}^{-1})_{ij}
\lp 
 \la \sigma^{(\rm th)}_j\lp {\boldsymbol c}^{(k)} \rp \ra
  -\sigma^{(\rm exp)}_j\rp
 \label{eq:chi2definition3}
    \; ,
\end{equation}
where the average over the theory predictions is evaluated over the the $N_{\rm spl}$ samples
provided by NS,
and the covariance matrix is evaluated with the  experimental definition~\cite{Ball:2012wy}.
%
Note that in general the average over theory predictions does not correspond to the theory
prediction evaluated using the average value of the Wilson coefficients,
\be
\la \sigma^{(\rm th)}_i\lp {\boldsymbol c}^{(k)} \rp \ra \ne
\sigma^{(\rm th)}_i\lp \la  {\boldsymbol c}^{(k)}  \ra \rp \, , 
\ee
due to the presence of the quadratic corrections to the EFT cross-sections.

With the figure of merit defined in Eq.~(\ref{eq:chi2definition3}),
we collect in Tables~\ref{eq:chi2-baseline} and~\ref{eq:chi2-baseline2}
the values of the $\chi^2$ per data point corresponding to
the baseline settings of our analysis.
%
We display both the values based on the SM theory predictions
as well as the best-fit SMEFT results obtained with $\mathcal{O}\lp \Lambda^{-2}\rp$ and
$\mathcal{O}\lp \Lambda^{-4}\rp$ calculations.
%
Note that, for ease of reference, in these tables
each dataset has associated a hyperlink pointing to the original publication.
%
For those datasets for which more than one differential distribution is available, we indicate the specific one used in the fit.
%
Then Table~\ref{eq:chi2-baseline-grouped} presents the
summary of these $\chi^2$ values now indicating the total values for each group of processes
as well as for the global dataset.
%
Furthermore, the results of Tables~\ref{eq:chi2-baseline}
and~\ref{eq:chi2-baseline2} are graphically represented in Fig.~\ref{fig:chi2_barplot}.

%%%%%%%%%%%%%%%%%%%%%%%%%%%%%%%%%%%%%%%%%%%%%%%%%%%%%%%%%%%%%%
%%%%%%%%%%%%%%%%%%%%%%%%%%%%%%%%%%%%%%%%%%%%%%%%%%%%%%%%%%%%%%%%%%%%%%%%%%%%%%%%%%%%%%%
\begin{table}[htbp]
  \centering
  \footnotesize
   \renewcommand{\arraystretch}{1.55}
  \begin{tabular}{l|C{1.1cm}|C{2.1cm}|C{2.5cm}|C{2.5cm}}
   \multirow{2}{*}{ Dataset}   & \multirow{2}{*}{$ n_{\rm dat}$} & \multirow{2}{*}{ $\chi^2_{\rm SM}$} &  $\chi^2_{\rm EFT}$   & $\chi^2_{\rm EFT}$     \\
      &   &   & $\mathcal{O}\lp \Lambda^{-2}\rp$ &  $\mathcal{O}\lp \Lambda^{-4}\rp$  \\
 \toprule
%-----------------------------------------------------------------------------------------
 \href{https://arxiv.org/abs/1511.04716}{\tt ATLAS\_tt\_8TeV\_ljets\_mtt} {{\bf (*)}}
 & 7 &  2.95   &  2.46  &  2.71     \\
%-----------------------------------------------------------------------------------------
 \href{https://arxiv.org/abs/1607.07281}{\tt ATLAS\_tt\_8TeV\_dilep\_mtt} & 6 &  0.09   &  0.12  &  0.12   \\
%-----------------------------------------------------------------------------------------
\href{https://arxiv.org/abs/1505.04480}{\tt CMS\_tt\_8TeV\_ljets\_ytt} & 10 &  0.91   &  1.19   &  1.05     \\
%-----------------------------------------------------------------------------------------
\href{https://arxiv.org/abs/1703.01630}{\tt CMS\_tt2D\_8TeV\_dilep\_mttytt} & 16 &  1.63  &  1.01  & 1.12      \\
%-----------------------------------------------------------------------------------------
\href{https://arxiv.org/abs/1610.04191}{\tt CMS\_tt\_13TeV\_ljets\_2015\_mtt} & 8 &  0.94  &  0.72  &  0.97     \\
%-----------------------------------------------------------------------------------------
\href{https://arxiv.org/abs/1708.07638}{\tt CMS\_tt\_13TeV\_dilep\_2015\_mtt} & 6 &  1.30   &  1.42   & 1.52      \\
%-----------------------------------------------------------------------------------------
\href{https://arxiv.org/abs/1803.08856}{\tt CMS\_tt\_13TeV\_ljets\_2016\_mtt}  {{\bf (*)}} & 10 &  1.99  &  1.70  & 2.22      \\
%-----------------------------------------------------------------------------------------
\href{https://arxiv.org/abs/1811.06625}{\tt CMS\_tt\_13TeV\_dilep\_2016\_mtt}  {{\bf (*)}} & 7 & 2.28  & 1.96   & 2.52      \\
%-----------------------------------------------------------------------------------------
\href{https://arxiv.org/abs/1908.07305}{\tt ATLAS\_tt\_13TeV\_ljets\_2016\_mtt} & 7 & 0.99   & 1.81   & 1.02       \\
%-----------------------------------------------------------------------------------------
\href{https://arxiv.org/abs/1709.05327}{\tt ATLAS\_CMS\_tt\_AC\_8TeV} & 6 &  0.86  &  0.70  & 0.86       \\
%-----------------------------------------------------------------------------------------
\href{https://inspirehep.net/literature/1743677}{\tt ATLAS\_tt\_AC\_13TeV} & 5 &  0.03  & 0.32   &  0.26     \\
%-----------------------------------------------------------------------------------------
\href{https://arxiv.org/abs/1612.02577}{\tt ATLAS\_WhelF\_8TeV}  {{\bf (*)}} & 3 &  1.97  & 1.30   & 1.38       \\
%-----------------------------------------------------------------------------------------
\href{https://arxiv.org/abs/1605.09047}{\tt CMS\_WhelF\_8TeV} & 3 &  0.30  & 0.64   &  0.58    \\
%-----------------------------------------------------------------------------------------
\midrule
\href{https://arxiv.org/abs/1509.05276}{\tt ATLAS\_ttZ\_8TeV} & 1 & 1.31   &  0.76     &  1.24      \\
%-------------------------------------------------------------------------------
\href{https://arxiv.org/abs/1609.01599}{\tt ATLAS\_ttZ\_13TeV} & 1 &  0.01    & 0.12   &  0.05     \\
%-------------------------------------------------------------------------------
\href{https://arxiv.org/abs/1901.03584}{\tt ATLAS\_ttZ\_13TeV\_2016} & 1 & 0.001   &  0.35   &  0.10     \\
%-------------------------------------------------------------------------------
\href{https://arxiv.org/abs/1510.01131}{\tt CMS\_ttZ\_8TeV} & 1 &  0.04    &  0.19   &  0.05      \\
%-------------------------------------------------------------------------------
\href{https://arxiv.org/abs/1711.02547}{\tt CMS\_ttZ\_13TeV} & 1 & 0.90    & 0.17   &  0.41      \\
%-------------------------------------------------------------------------------
\href{https://arxiv.org/abs/1907.11270}{\tt CMS\_ttZ\_13TeV\_pTZ} & 4 &  0.73  &  0.69   & 0.91      \\
%-------------------------------------------------------------------------------
\href{https://arxiv.org/abs/1509.05276}{\tt ATLAS\_ttW\_8TeV} & 1 & 1.33   & 0.47    & 1.22      \\
%-------------------------------------------------------------------------------
\href{https://arxiv.org/abs/1609.01599}{\tt ATLAS\_ttW\_13TeV} & 1 &  0.83   &  0.56   & 0.81      \\
%-------------------------------------------------------------------------------
\href{https://arxiv.org/abs/1901.03584}{\tt ATLAS\_ttW\_13TeV\_2016} & 1 &   0.23   & 0.14    &  0.00      \\
%-------------------------------------------------------------------------------
\href{https://arxiv.org/abs/1510.01131}{\tt CMS\_ttW\_8TeV} & 1 &  1.54    &  0.68     &  1.43     \\
%-------------------------------------------------------------------------------
\href{https://arxiv.org/abs/1711.02547}{\tt CMS\_ttW\_13TeV} & 1 &  0.03   &   0.57    &  0.14     \\
%-------------------------------------------------------------------------------
\midrule
\href{https://arxiv.org/abs/1705.10141}{\tt CMS\_ttbb\_13TeV}  {{\bf (*)}} & 1 &  4.96    &  2.65     &  6.66     \\
%-------------------------------------------------------------------------------
\href{https://arxiv.org/abs/1909.05306}{\tt CMS\_ttbb\_13TeV\_2016} & 1 &  1.75    &  0.35     &  3.09     \\
%-------------------------------------------------------------------------------
\href{https://arxiv.org/abs/1811.12113}{\tt ATLAS\_ttbb\_13TeV\_2016} & 1 & 0.91     &  1.68     &  0.55     \\
%-------------------------------------------------------------------------------
\href{https://arxiv.org/abs/1710.10614}{\tt CMS\_tttt\_13TeV} & 1 &  0.05   & 0.02      &  0.08     \\
%-------------------------------------------------------------------------------
\href{https://arxiv.org/abs/1908.06463}{\tt CMS\_tttt\_13TeV\_run2} & 1 &  0.05    & 1.15      &  2.04     \\
%-------------------------------------------------------------------------------
\href{https://arxiv.org/abs/2007.14858}{\tt ATLAS\_tttt\_13TeV\_run2}  {{\bf (*)}} & 1 &  2.35   &  0.70     &  0.30     \\
%-------------------------------------------------------------------------------
\bottomrule
\end{tabular}
\caption{\small The values of the $\chi^2$ per data point corresponding to
  the baseline settings of our analysis.
  %
  We indicate the results for the $t\bar{t}$ datasets, both in inclusive
  production and in association with vector bosons or heavy quarks.
%
  We display the SM values and then the best-fit SMEFT results obtained
  in analyses based on theory predictions at either $\mathcal{O}\lp \Lambda^{-2}\rp$ or 
$\mathcal{O}\lp \Lambda^{-4}\rp$ accuracy.
%
Each dataset has a hyperlink pointing to the original publication.
%
For those datasets for which more than one differential distribution is available,
we indicate the specific
ones used in the fit.
%
    {Datasets indicated with {\bf (*)} are excluded from the ``conservative'' EFT
    fit to be discussed in Sect.~\ref{sec:dataset_dependence}.}
\label{eq:chi2-baseline}
}
\end{table}
%%%%%%%%%%%%%%%%%%%%%%%%%%%%%%%%%%%%%%%%%%%%%%%%%%%%%%%%%%%%%%%%%%%%%%%%%%%%%%%%


%%%%%%%%%%%%%%%%%%%%%%%%%%%%%%%%%%%%%%%%%%%%%%%%%%%%%%%%%%%%%%%%%%%%%%%%%%%%%%%%%%%%%%%
\begin{table}[htbp]
  \centering
  \footnotesize
  \renewcommand{\arraystretch}{1.50}
  \begin{tabular}{l|C{1.1cm}|C{2.1cm}|C{2.5cm}|C{2.5cm}}
       \multirow{2}{*}{ Dataset}   & \multirow{2}{*}{$ n_{\rm dat}$} & \multirow{2}{*}{ $\chi^2_{\rm SM}/n_{\rm dat}$} &  $\chi^2_{\rm EFT}/n_{\rm dat}$   & $\chi^2_{\rm EFT}/n_{\rm dat}$     \\
      &   &   & $\mathcal{O}\lp \Lambda^{-2}\rp$ &  $\mathcal{O}\lp \Lambda^{-4}\rp$  \\
       \toprule
%-----------------------------------------------------------------------------------------
 \href{https://arxiv.org/abs/1403.7366}{\tt CMS\_t\_tch\_8TeV\_inc} & 2 &  0.29    &  0.17    & 0.21     \\
 %-----------------------------------------------------------------------------------------
 \href{https://arxiv.org/abs/1702.02859}{\tt ATLAS\_t\_tch\_8TeV} & 4 &  0.89     & 0.71     & 0.66     \\
 %-----------------------------------------------------------------------------------------
 \href{https://cds.cern.ch/record/1956681}{\tt CMS\_t\_tch\_8TeV\_diff\_Yt} & 6 &  0.20    &  0.11    &  0.16    \\
 %-----------------------------------------------------------------------------------------
 \href{https://arxiv.org/abs/1603.02555}{\tt CMS\_t\_sch\_8TeV} & 1 &  1.26     &  0.94     &  1.16    \\
 %-----------------------------------------------------------------------------------------
 \href{https://arxiv.org/abs/1511.05980}{\tt ATLAS\_t\_sch\_8TeV} & 1 &  0.08     &  0.90    &  0.25    \\
 %-----------------------------------------------------------------------------------------
 \href{https://arxiv.org/abs/1609.03920}{\tt ATLAS\_t\_tch\_13TeV} & 2 &  0.01     &  0.06    &  0.02    \\
 %-----------------------------------------------------------------------------------------
 \href{https://arxiv.org/abs/1610.00678}{\tt CMS\_t\_tch\_13TeV\_inc} & 2 &  0.35    & 0.24     & 0.35     \\
 %-----------------------------------------------------------------------------------------
 \href{https://cds.cern.ch/record/2151074}{\tt CMS\_t\_tch\_13TeV\_diff\_Yt} & 4 &   0.52   & 0.47     &  0.47    \\
 %-----------------------------------------------------------------------------------------
 \href{https://arxiv.org/abs/1907.08330}{\tt CMS\_t\_tch\_13TeV\_2016\_diff\_Yt} & 5 &  0.60  & 0.59  & 0.59      \\
 %-----------------------------------------------------------------------------------------
\midrule
 %-----------------------------------------------------------------------------------------
 \href{https://arxiv.org/abs/1712.02825}{\tt ATLAS\_tZ\_13TeV\_inc} & 1 &  0.00     & 0.04     &  0.00    \\
 %-----------------------------------------------------------------------------------------
 \href{https://arxiv.org/abs/2002.07546}{\tt ATLAS\_tZ\_13TeV\_run2\_inc} & 1 &  0.05    &  0.07    &  0.01    \\
 %-----------------------------------------------------------------------------------------
 \href{https://arxiv.org/abs/1712.02825}{\tt CMS\_tZ\_13TeV\_inc} & 1 &  0.66    & 0.36     & 0.64     \\
 %-----------------------------------------------------------------------------------------
 \href{https://arxiv.org/abs/1812.05900}{\tt CMS\_tZ\_13TeV\_2016\_inc} & 1 &   1.23    &  0.33    &  1.16    \\
 %-----------------------------------------------------------------------------------------
 \href{https://arxiv.org/abs/1510.03752}{\tt ATLAS\_tW\_8TeV\_inc} & 1 &  0.02     &  0.01    &  0.05    \\
 %-----------------------------------------------------------------------------------------
 \href{https://arxiv.org/abs/2007.01554}{\tt ATLAS\_tW\_slep\_8TeV\_inc} & 1 &   0.13    &  0.15    & 0.11    \\
 %-----------------------------------------------------------------------------------------
 \href{https://arxiv.org/abs/1401.2942}{\tt CMS\_tW\_8TeV\_inc} & 1 &  0.00     & 0.00     & 0.00     \\
 %-----------------------------------------------------------------------------------------
 \href{https://arxiv.org/abs/1612.07231}{\tt ATLAS\_tW\_13TeV\_inc} & 1 &  0.52     &  0.55    &  0.47    \\
 %-----------------------------------------------------------------------------------------
 \href{https://arxiv.org/abs/1805.07399}{\tt CMS\_tW\_13TeV\_inc}  {{\bf (*)}} & 1 &  3.79     & 3.49     &  4.33    \\
 %-----------------------------------------------------------------------------------------
 \midrule
 %-----------------------------------------------------------------------------------------
 \href{https://arxiv.org/abs/1606.02266}{\tt ATLAS\_CMS\_SSinc\_RunI} & 22 &   0.86    & 0.86     & 0.89     \\
 %-----------------------------------------------------------------------------------------
 \href{https://arxiv.org/abs/1909.02845}{\tt ATLAS\_SSinc\_RunII} & 16 &   0.54   &  0.55    &  0.54    \\
 %-----------------------------------------------------------------------------------------
 \href{https://arxiv.org/abs/1809.10733}{\tt CMS\_SSinc\_RunII} & 24 &  0.77     &  0.70    &  0.68    \\
 %-----------------------------------------------------------------------------------------
 \href{https://arxiv.org/abs/1909.02845}{\tt ATLAS\_ggF\_ZZ\_13TeV} & 6 &  0.96     &   0.84   &  0.81    \\
 %-----------------------------------------------------------------------------------------
 \href{https://inspirehep.net/literature/1725274}{\tt CMS\_ggF\_aa\_13TeV} & 6 & 1.05   & 1.04   & 1.05     \\
 %-----------------------------------------------------------------------------------------
 \href{http://cdsweb.cern.ch/record/2682844/files/ATLAS-CONF-2019-032.pdf}{\tt ATLAS\_H\_13TeV\_2015\_pTH} & 9 & 1.11      & 1.10     & 1.08     \\ 
 %-----------------------------------------------------------------------------------------
 \href{https://arxiv.org/abs/1812.06504}{\tt CMS\_H\_13TeV\_2015\_pTH} & 9 &  0.80     &  0.78    & 0.78     \\
 %-----------------------------------------------------------------------------------------
 \href{https://arxiv.org/abs/1903.04618}{\tt ATLAS\_WH\_Hbb\_13TeV} & 2 &   0.10    &  0.07    &  0.15    \\
 %-----------------------------------------------------------------------------------------
 \href{https://arxiv.org/abs/1903.04618}{\tt ATLAS\_ZH\_Hbb\_13TeV} & 3 &   0.50    &  0.41    &  0.30    \\
 %-----------------------------------------------------------------------------------------
 \midrule
 %-----------------------------------------------------------------------------------------
 \href{https://arxiv.org/abs/1905.04242}{\tt ATLAS\_WW\_13TeV\_2016\_memu} & 13 & 1.64   &  1.64    &  1.67    \\
 %-----------------------------------------------------------------------------------------
 \href{https://arxiv.org/abs/1902.05759}{\tt ATLAS\_WZ\_13TeV\_2016\_mTWZ} & 6 &  0.81   &  0.81  & 0.80     \\
 %-----------------------------------------------------------------------------------------
 \href{https://arxiv.org/abs/1901.03428}{\tt CMS\_WZ\_13TeV\_2016\_pTZ} & 11 &  1.46   &  1.44    &  1.39    \\
 %-----------------------------------------------------------------------------------------
 \href{https://arxiv.org/abs/1302.3415}{\tt LEP\_eeWW\_182GeV} & 10 &  1.38  &  1.38   & 1.38     \\
 %-----------------------------------------------------------------------------------------
 \href{https://arxiv.org/abs/1302.3415}{\tt LEP\_eeWW\_189GeV} & 10 & 0.88    & 0.88   &  0.89    \\
 %-----------------------------------------------------------------------------------------
 \href{https://arxiv.org/abs/1302.3415}{\tt LEP\_eeWW\_198GeV} & 10 &  1.61   &  1.61    &  1.61    \\
 %-----------------------------------------------------------------------------------------
 \href{https://arxiv.org/abs/1302.3415}{\tt LEP\_eeWW\_206GeV} & 10 &  1.09   &  1.08   &  1.08    \\
 %----------------------------------------------------------------------------------------- 
\bottomrule
\end{tabular}
  \caption{\small Same as Table~\ref{eq:chi2-baseline} now for the single top
    datasets (inclusive and in association with gauge bosons), the Higgs production and decay measurements (signal streghts and differential
    distributions), and the LEP and LHC
    diboson cross-sections.
\label{eq:chi2-baseline2}
}
\end{table}
%%%%%%%%%%%%%%%%%%%%%%%%%%%%%%%%%%%%%%%%%%%%%%%%%%%%%%%%%%%%%%%%%%%%%%%%%%%%%%%%


%%%%%%%%%%%%%%%%%%%%%%%%%%%%%%%%%%%%%%%%%%%%%%%%%%%%%%%%%%%%%%%

%%%%%%%%%%%%%%%%%%%%%%%%%%%%%%%%%%%%%%%%%%%%%%%%%%%%%%%%%%%%%%
%%%%%%%%%%%%%%%%%%%%%%%%%%%%%%%%%%%%%%%%%%%%%%%%%%%%%%%%%%%%%%%%%%%%%%%%%%%%%%%%%%%%%%%
\begin{table}[htbp]
  \centering
  \small
   \renewcommand{\arraystretch}{1.70}
   \begin{tabular}{l|C{1.0cm}|C{1.8cm}|C{2.7cm}|C{2.7cm}}
        \multirow{2}{*}{ Dataset}   & \multirow{2}{*}{$ n_{\rm dat}$} & \multirow{2}{*}{ $\chi^2_{\rm SM}$} &  $\chi^2_{\rm EFT}$   & $\chi^2_{\rm EFT}$     \\
      &   &   & $\mathcal{O}\lp \Lambda^{-2}\rp$ &  $\mathcal{O}\lp \Lambda^{-4}\rp$  \\
        \toprule
%-----------------------------------------------------------------------------------------
 $t\bar{t}$ inclusive  & 83 &  1.46    &  1.32    &  1.42       \\
 $t\bar{t}$ charge asymmetry  & 11 &  0.60    &  0.39    &  0.59       \\
 $t\bar{t}+V$  & 14  &  0.65     &  0.48     &  0.65       \\
 single-top inclusive  &  27  &   0.43    &  0.44     &  0.41       \\
 single-top $+V$ & 9  &  0.71     &  0.55     &  0.75       \\
 $t\bar{t}b\bar{b}$ \& $t\bar{t}t\bar{t}$   & 6   &  1.68     &  1.09     &  2.12       \\
 Higgs signal strenghts (Run I)  &  22  &   0.86     &  0.85     &  0.90       \\
 Higgs signal strenghts (Run II)  &  40 &   0.67    &  0.64     &  0.63       \\
 Higgs differential \& STXS  &  35  & 0.88     &  0.85     &   0.83      \\
 Diboson (LEP+LHC)  & 70  &  1.31     &  1.31     &  1.30       \\
 \midrule
 {\bf Total}  & {\bf 317}  &  {\bf 1.05 }   & {\bf 0.98 }  & {\bf 1.04 } \\
%-------------------------------------------------------------------------------
\bottomrule
\end{tabular}
\caption{\small Summary of the $\chi^2$ results listed in Tables~\ref{eq:chi2-baseline}
  and~\ref{eq:chi2-baseline2}.
  %
  We indicate the total values for each group of processes
  as well as for the global dataset.
\label{eq:chi2-baseline-grouped}
}
\end{table}
%%%%%%%%%%%%%%%%%%%%%%%%%%%%%%%%%%%%%%%%%%%%%%%%%%%%%%%%%%%%%%%%%%%%%%%%%%%%%%%%


%%%%%%%%%%%%%%%%%%%%%%%%%%%%%%%%%%%%%%%%%%%%%%%%%%%%%%%%%%%%%%%

%%%%%%%%%%%%%%%%%%%%%%%%%%%%%%%%%%%%%%%%%%%%%%%%%%%%%%%%%%%%%%%%%%%%%
\begin{figure}[t]
  \begin{center}
    \includegraphics[width=0.99\linewidth]{plots_v2/Chi2_Bar_Baseline.pdf}
    \caption{\small Graphical representation of the results of Tables~\ref{eq:chi2-baseline}
      and~\ref{eq:chi2-baseline2},
      displaying the values of the $\chi^2$ per data point, Eq.~(\ref{eq:chi2definition3})
      for the all datasets used as input in the fit.
      %
     The $\chi^2$ values are shown  for the SM and for the two global SMEFT
      baseline fits, based on theory calculations at either the linear $\mathcal{O}\lp \Lambda^{-2}\rp$ or
       quadratic $\mathcal{O}\lp \Lambda^{-4}\rp$ order in the EFT expansion.
     \label{fig:chi2_barplot} }
  \end{center}
\end{figure}
%%%%%%%%%%%%%%%%%%%%%%%%%%%%%%%%%%%%%%%%%%%%%%%%%%%%%%%%%%%%%%%%%%%%%%

Let us discuss first the $\chi^2$ results evaluated in terms of the groups
of processes, listed in Table~\ref{eq:chi2-baseline-grouped}.
%
One can observe that the global $\chi^2$ per data point decreases from 1.05 when using
SM theory to 0.98 (linear) and 1.04 (quadratic) once SMEFT corrections
are accounted for.
%
Considering the fit quality to the various groups of processes, we find
that the description of the inclusive top-quark pair
cross-sections (without the $A_C$ data),
composed by $n_{\rm dat}=83$ data points, is improved once EFT corrections
are accounted for with $\chi^2=1.46$ in the SM decreasing
to 1.32 and 1.42 in the linear and quadratic EFT fits
respectively.
%
For the rest of the datasets, and in particular for the Higgs and diboson
measurements, the overall EFT fit quality is similar to that
obtained using SM calculations.

Inspection of the $\chi^2$ values associated to individual datasets
reported in Tables~\ref{eq:chi2-baseline}
and~\ref{eq:chi2-baseline2}, as well
as their graphical representation from Fig.~\ref{fig:chi2_barplot},
reveals that in some cases the agreement
between the prior SM theoretical calculations and the data is poor.
%
This is the case, in particular, for some of the inclusive $t\bar{t}$ datasets such as
{\tt ATLAS\_tt\_8TeV\_ljets\_mtt} and
{\tt CMS\_tt\_13TeV\_dilep\_2016\_mtt}, binned in terms of the top-quark
pair invariant mass distribution $m_{t\bar{t}}$, with $\chi^2=2.95$ and 2.28
for $n_{\rm dat}=7$ points each.
%
Such relatively high values of the $\chi^2$ do not necessarily imply the need for
some New Physics effects,
but could also be explained by issues with the modelling of the experimental
systematic correlations in differential distributions, as discussed for the ATLAS 8 TeV
lepton+jets data in~\cite{Czakon:2016olj,Amoroso:2020lgh,ATL-PHYS-PUB-2018-017}.
%
Nevertheless, when all the inclusive $t\bar{t}$ datasets are considered collectively,
a value of $\chi^2_{\rm SM} = 1.46$ for  the $n_{\rm dat}=83$ data points in the fit is obtained.
%
In Sect.~\ref{sec:dataset_dependence} we will assess  the stability of the global
fit results by presenting fit variants with the individual datasets
that lead to a poor $\chi^2$ removed.
%
We will also study variants where distributions sensitive
to the high energy behaviour of the EFT, such as $m_{t\bar{t}}$ in top quark pair
production, have their bins with $m_{t\bar{t}}\gsim 1$ TeV removed from the fit.

Beyond the inclusive $t\bar{t}$ datasets, there are some other instances of a sub-optimal agreement
between SM theory and data.
%
In all cases, there exist comparable measurements of the same process, either
from the same experiment at a different center-of-mass energy
$\sqrt{s}$ or from a different experiment
at the same value of $\sqrt{s}$, for which the $\chi^2$ reveals good consistency with the SM.
%
These include {\tt CMS\_ttbb\_13TeV}, when comparing with the same measurement based
on the full 2016 luminosity, and
{\tt CMS\_tW\_13TeV\_inc}, where again the ATLAS measurement at the same $\sqrt{s}$
exhibit as good $\chi^2$.
%
All in all, one finds a reasonable description of the global input dataset
when using SM cross-sections which is further improved in the EFT fit.

For the rest of this section, when presenting the results of fits corresponding
to variations of the baseline settings, such as fits based on
reduced datasets, we will only indicate the $\chi^2$ values
associated to groups of processes using the same format
as Table~\ref{eq:chi2-baseline-grouped}, rather than to individual
datasets, and comment when required on the results for the latter.



\subsection{Constraints on the EFT parameter space}

Following this assessment of the fit quality,
we move to present the constraints on the SMEFT parameter space
that can be derived from the present global fit.
%
We will present results for the $n_{\rm op}=50$ Wilson
coefficients listed in Table~\ref{tab:operatorbasis},
with the understanding that only 36 of them are linearly
independent.\footnote{We note that the EWPO constraints of Eq.~(\ref{eq:2independents}) set
  the four-lepton operator to zero, $c_{\ell\ell}=0$, and hence we exclude this coefficient
from the plots and tables of this section.}
%
Specifically, we provide the 95\% confidence level intervals for each
EFT coefficients,
study their posterior probability distributions, evaluate the pattern
of their correlations, and compare the marginalised bounds with those
obtained in individual fits where only one coefficient is varied at a time.
%
We will also assess the overall consistency of the fit results with respect
to the Standard Model hypothesis.
%
The results discussed here correspond to the global dataset with the baseline theory
settings for both $\mathcal{O}\lp \Lambda^{-2}\rp$ and $\mathcal{O}\lp \Lambda^{-4}\rp$
theory calculations.
%
Fits based on either reduced datasets or alternative theory settings
are then discussed in Sects.~\ref{sec:dataset_dependence}
and~\ref{subsec:loqcd}  respectively.

%%%%%%%%%%%%%%%%%%%%%%%%%%%%%%%%%%%%%%%%%%%%%%%%%%%%%%%%%%%%%%%%%%%%%
\begin{figure}[t]
  \begin{center}
     \includegraphics[width=0.99\linewidth]{plots_v2/Coeffs_Hist_Baseline.pdf}
    \caption{\small The normalised posterior probability distributions associated
      to each of the $n_{\rm op}=50$ fit coefficients considered in the present
      analysis, for both the linear and quadratic EFT fits.
      %
      Note that the $x$-axis range is different in each case.
      %
      From top to bottom and from left to right, we display
      the four-heavy, two-light-two-heavy,
      two-fermion, and purely bosonic coefficients.
      %
      Only 36 of these coefficients are independent
      as indicated in Table~\ref{tab:operatorbasis}.
     \label{fig:posterior_coeffs} }
  \end{center}
\end{figure}
%%%%%%%%%%%%%%%%%%%%%%%%%%%%%%%%%%%%%%%%%%%%%%%%%%%%%%%%%%%%%%%%%%%%%%

\paragraph{Posterior distributions.}
%
Fig.~\ref{fig:posterior_coeffs} displays
the normalised posterior probability distributions associated
to each of the $n_{\rm op}=50$ fit coefficients considered in the present
analysis, for the linear (blue) and quadratic (orange) EFT fits.
%
As discussed in Sect.~\ref{sec:nestedsampling},
the NS prior sampling volumes have been optimised to ensure
that the posterior distribution associated to each coefficient
is fully contained within them.
%
One can observe how in general
the $\mathcal{O}\lp \Lambda^{-4}\rp$ corrections modify significantly
 the distributions that is obtained from the linear fits, for instance
 by shifting its median or by decreasing its variance.
 %
 For several coefficients, the posterior distributions would be poorly
 described in the Gaussian approximation,
 and in some cases one finds multi-modal distributions
 such as for the Yukawa operators $c_{\varphi c}$, $c_{\varphi b}$, and
 $c_{\varphi \tau}$.
 %
 Such double-humped distributions can be traced back to the (quasi)-degenerate
 minima in the individual
 $\chi^2$ profiles reported in
 Figs.~\ref{fig:quartic-individual-fits} and~\ref{fig:quartic-individual-fits-2}.
%
 We can also observe how the four-heavy coefficients can only be meaningfully
 constrained in the quadratic fit.
 %
 All in all, inclusion of the quadratic EFT corrections modifies
 in a significant manner the posterior distributions associated
 to most of the fit coefficients as compared to the linear approximation.

\paragraph{Confidence level intervals.}
%
From the posterior probability distributions displayed in Fig.~\ref{fig:posterior_coeffs},
one can derive the marginalised 95\% CL intervals on the 
EFT coefficients both for the linear and quadratic fits.
%
These results are collected in Table~\ref{tab:coeff-bounds-baseline}
(for $\Lambda=1$ TeV)
and represented graphically in
Fig.~\ref{fig:globalfit-baseline-coeffsabs-lin-vs-quad}.
%
In addition,  Table~\ref{tab:coeff-bounds-baseline} also includes
the corresponding obtained in  individual NS fits, where only one
operator is varied at a time and the rest are set
to their SM values (recall the $\chi^2$ profiles from
Figs.~\ref{fig:quartic-individual-fits} and~\ref{fig:quartic-individual-fits-2}).
%
We will further discuss the outcome of these individual fits below.

%%%%%%%%%%%%%%%%%%%%%%%%%%%%%%%%%%%%%%%%%%%
%%%%%%%%%%%%%%%%%%%%%%%%%%%%%%%%%%%%%%%%%%%%%%%%%%%%%%%%%%%%%%%%%%%%%%%%%%%%%%%%%%%%%%%
\begin{table}[htbp]
  \centering
  \scriptsize
   \renewcommand{\arraystretch}{1.24}
   \begin{tabular}{l|C{0.8cm}|C{2.3cm}|C{2.3cm}|C{4.0cm}|C{4.0cm}}
     \multirow{2}{*}{Class}   &  \multirow{2}{*}{DoF}
     &  \multicolumn{2}{c|}{ 95\% CL bounds, $\mathcal{O}\lp \Lambda^{-2}\rp$} &
     \multicolumn{2}{c}{95\% CL bounds, $\mathcal{O}\lp \Lambda^{-4}\rp$,} \\ 
 &  & Individual & Marginalised &  Individual & Marginalised  \\ \toprule
 \multirow{5}{*}{4H}
 &{\tt cQQ1}& [-6.132,23.281] & [-190,189] & [-2.229,2.019] & [-2.995,3.706] \\ \cline{2-6}
 & {\tt cQQ8}  & [-26.471,57.778] & [-190,170] & [-6.812,5.834] & [-11.177,8.170] \\ \cline{2-6}
 & {\tt cQt1}& [-195,159] & [-190,189] & [-1.830,1.862] & [-1.391,1.251] \\ \cline{2-6}
 & {\tt cQt8}& [-5.722,20.105] & [-190,162] & [-4.213,3.346] & [-3.040,2.202] \\ \cline{2-6}
 & {\tt ctt1}& [-2.782,12.114] & [-115,153] & [-1.151,1.025] & [-0.791,0.714] \\ \hline
\multirow{14}{*}{2L2H}
 & {\tt c81qq}& [-0.273,0.509] & [-2.258,4.822] & [-0.373,0.309] & [-0.555,0.236] \\ \cline{2-6}
 & {\tt c11qq}& [-3.603,0.307] & [-8.047,9.400] & [-0.303,0.225] & [-0.354,0.249] \\ \cline{2-6}
 & {\tt c83qq}& [-1.813,0.625] & [-3.014,7.365] & [-0.470,0.439] & [-0.462,0.497] \\ \cline{2-6}
 & {\tt c13qq}& [-0.099,0.155] & [-0.163,0.296] & [-0.088,0.166] & [-0.167,0.197] \\ \cline{2-6}
 & {\tt c8qt}& [-0.396,0.612] & [-4.035,4.394] & [-0.483,0.393] & [-0.687,0.186] \\ \cline{2-6}
 & {\tt c1qt}& [-0.784,2.771] & [-12.382,6.626] & [-0.205,0.271] & [-0.222,0.226] \\ \cline{2-6}
 & {\tt c8ut}& [-0.774,0.607] & [-16.952,0.368] & [-0.911,0.347] & [-1.118,0.260] \\ \cline{2-6}
 & {\tt c1ut}& [-6.046,0.424] & [-15.565,15.379] & [-0.380,0.293] & [-0.383,0.331] \\ \cline{2-6}
 & {\tt c8qu}& [-1.508,1.022] & [-12.745,13.758] & [-1.007,0.521] & [-1.002,0.312] \\ \cline{2-6}
 & {\tt c1qu}& [-0.938,2.462] & [-16.996,1.072] & [-0.281,0.371] & [-0.207,0.339] \\ \cline{2-6}
 & {\tt c8dt}& [-1.458,1.365] & [-5.494,25.358] & [-1.308,0.638] & [-1.329,0.643] \\ \cline{2-6}
 & {\tt c1dt}& [-9.504,-0.086] & [-27.673,11.356] & [-0.449,0.371] & [-0.474,0.347] \\ \cline{2-6}
 & {\tt c8qd}& [-2.393,2.042] & [-24.479,11.233] & [-1.615,0.888] & [-1.256,0.715] \\ \cline{2-6}
 & {\tt c1qd}& [-0.889,6.459] & [-3.239,34.632] & [-0.332,0.436] & [-0.370,0.384] \\ \hline
\multirow{23}{*}{2FB}
 & {\tt ctp}& [-1.331,0.355] & [-5.739,3.435] & [-1.286,0.348] & [-2.319,2.797] \\ \cline{2-6}
 & {\tt ctG}& [0.007,0.111] & [-0.127,0.403] & [0.006,0.107] & [0.062,0.243] \\ \cline{2-6}
 & {\tt cbp}& [-0.006,0.040] & [-0.033,0.105]& [-0.007,0.035]$\cup$ [-0.403,-0.360] & [-0.035,0.047]$\cup$ [-0.430,-0.338] \\ \cline{2-6}
 & {\tt ccp}& [-0.025,0.117] & [-0.316,0.134] & [-0.004,0.370] & [-0.096,0.484] \\ \cline{2-6}
 & {\tt ctap}& [-0.026,0.035] & [-0.027,0.044] & [-0.027,0.040]$\cup$ [0.395,0.462] & [-0.019,0.037]$\cup$ [0.389,0.480] \\ \cline{2-6}
 & {\tt ctW}& [-0.093,0.026] & [-0.313,0.123] & [-0.084,0.029] & [-0.241,0.086] \\ \cline{2-6}
 & {\tt ctZ}& [-0.039,0.099] & [-15.869,5.636] & [-0.044,0.094] & [-1.129,0.856] \\ \cline{2-6}
 & {\tt cpl1}& [-0.664,1.016] & [-0.244,0.375] & [-0.281,0.343] & [-0.106,0.129] \\ \cline{2-6}
 & {\tt c3pl1}& [-0.472,0.080] & [-0.098,0.120] & [-0.432,0.062] & [-0.209,0.046] \\ \cline{2-6}
 & {\tt cpl2}& [-0.664,1.016] & [-0.244,0.375] & [-0.281,0.343] & [-0.106,0.129] \\ \cline{2-6}
 & {\tt c3pl2}& [-0.472,0.080] & [-0.098,0.120] & [-0.432,0.062] & [-0.209,0.046] \\ \cline{2-6}
 & {\tt cpl3}& [-0.664,1.016] & [-0.244,0.375] & [-0.281,0.343] & [-0.106,0.129] \\ \cline{2-6}
 & {\tt c3pl3}& [-0.472,0.080] & [-0.098,0.120] & [-0.432,0.062] & [-0.209,0.046] \\ \cline{2-6}
 & {\tt cpe}& [-1.329,2.033] & [-0.487,0.749] & [-0.562,0.687] & [-0.213,0.258] \\ \cline{2-6}
 & {\tt cpmu}& [-1.329,2.033] & [-0.487,0.749] & [-0.562,0.687] & [-0.213,0.258] \\ \cline{2-6}
 & {\tt cpta}& [-1.329,2.033] & [-0.487,0.749] & [-0.562,0.687] & [-0.213,0.258] \\ \cline{2-6}
 & {\tt c3pq}& [-0.472,0.080] & [-0.098,0.120] & [-0.432,0.062] & [-0.209,0.046] \\ \cline{2-6}
 & {\tt c3pQ3}& [-0.350,0.353] & [-1.145,0.740] & [-0.375,0.344] & [-0.615,0.481] \\ \cline{2-6}
 & {\tt cpqMi}& [-2.905,0.490] & [-0.171,0.106] & [-2.659,0.381] & [-0.060,0.216] \\ \cline{2-6}
 & {\tt cpQM}& [-0.998,1.441] & [-1.690,11.569] & [-1.147,1.585] & [-2.250,2.855] \\ \cline{2-6}
 & {\tt cpui}& [-1.355,0.886] & [-0.499,0.325] & [-0.458,0.375] & [-0.172,0.142] \\ \cline{2-6}
 & {\tt cpdi}& [-0.443,0.678] & [-0.162,0.250] & [-0.187,0.229] & [-0.071,0.086] \\ \cline{2-6}
 & {\tt cpt}& [-2.087,2.463] & [-3.270,18.267] & [-3.028,2.195] & [-13.260,3.955] \\ \hline
\multirow{7}{*}{B}
 & {\tt cpG}& [-0.002,0.005] & [-0.043,0.012] & [-0.002,0.005] & [-0.019,0.003] \\ \cline{2-6}
 & {\tt cpB}& [-0.005,0.002] & [-0.739,0.289] & [-0.005,0.002]$\cup$ [0.085,0.092] & [-0.114,0.108] \\ \cline{2-6}
 & {\tt cpW}& [-0.018,0.007] & [-0.592,0.677] & [-0.016,0.007]$\cup$ [0.281,0.305] & [-0.145,0.303] \\ \cline{2-6}
 & {\tt cpWB}& [-2.905,0.490] & [-0.462,0.694] & [-2.659,0.381] & [-0.170,0.273] \\ \cline{2-6}
 & {\tt cpd}& [-0.428,1.214] & [-2.002,3.693] & [-0.404,1.199]$\cup$ [-34.04,-32.61] & [-1.523,1.482] \\ \cline{2-6}
 & {\tt cpD}& [-4.066,2.657] & [-1.498,0.974] & [-1.374,1.124] & [-0.516,0.425] \\ \cline{2-6}
 & {\tt cWWW}& [-1.057,1.318] & [-1.049,1.459] & [-0.208,0.236] & [-0.182,0.222] \\ \bottomrule
\end{tabular}
   \caption{\small The 95\% CL bounds for all the
     EFT coefficients
     considered in this analysis, for  both individual and global (marginalised) fits
     obtained using either linear or quadratic EFT calculations.
\label{tab:coeff-bounds-baseline}
}
\end{table}
%%%%%%%%%%%%%%%%%%%%%%%%%%%%%%%%%%%%%%%%%%%%%%%%%%%%%%%%%%%%%%%%%%%%%%%%%%%%%%%%

%%%%%%%%%%%%%%%%%%%%%%%%%%%%%%%%%%%%%%%%%%

%%%%%%%%%%%%%%%%%%%%%%%%%%%%%%%%%%%%%%%%%%%%%%%%%%%%%%%%%%%%%%%%%%%%%
\begin{figure}[t]
  \begin{center}
    \includegraphics[width=0.99\linewidth]{plots_v2/Coeffs_Central_Baseline.pdf}
    \vspace{-0.1cm}
    \caption{\small The best-fit (median) value of the EFT coefficients $c_i/\Lambda^2$
      and their associated 95\% CL intervals for the  global fits
      based on either linear or quadratic EFT calculations,
      whose posterior distributions are represented in
      Fig.~\ref{fig:posterior_coeffs}.
      %
      The dashed horizontal line indicates the SM expectation.
     \label{fig:globalfit-baseline-coeffsabs-lin-vs-quad} }
  \end{center}
\end{figure}
%%%%%%%%%%%%%%%%%%%%%%%%%%%%%%%%%%%%%%%%%%%%%%%%%%%%%%%%%%%%%%%%%%%%%%

From the marginalised bounds displayed in Fig.~\ref{fig:globalfit-baseline-coeffsabs-lin-vs-quad},
one can observe how the uncertainties associated to the fit coefficients are in all cases
reduced in the quadratic fit in comparison to the linear one.
%
The 95\% CL interval is disjoint for the Yukawa coefficients $c_{b\varphi}$
and $c_{\tau\varphi}$ in the quadratic fit,
with both a SM-like solution and a second one far from the SM.
%
For the linear fit, we find that all EFT coefficients agree with the SM expectation
at the 95\% CL level.
%
For the quadratic fit instead, this is not the case only for
the chromo-magnetic operator $c_{tG}$.
%
We will trace back below the origin of this discrepancy,
here we only point out that at the level of individual fits
$c_{tG}$ exhibits the same trend but there
agrees with the SM at the  95\% CL
as indicated in Fig.~\ref{fig:quartic-individual-fits-2}.
%
{  We note that for unconstrained operators, such as the four-heavy operators
  in the linear fit, the best-fit value (median) should be ignored since
the underlying posterior is essentially flat.}

The global fit results of Fig.~\ref{fig:globalfit-baseline-coeffsabs-lin-vs-quad}
are further scrutinized in Fig.~\ref{fig:globalfit-baseline-bounds-lin-vs-quad},
which displays both the magnitude of the 95\% CL intervals 
   and the 68\% CL residuals compared to the SM hypothesis
   associated to the linear and quadratic EFT fits.
   %
   In the upper panel,
      the horizontal line indicates the boundaries of the sampling volume
      used for the poorly-constrained coefficients
as explained in Sect.~\ref{sec:nestedsampling}.
%
From these comparisons, one can observe how the inclusion of quadratic corrections
leads to markedly more stringent bounds for most of the fit coefficients,
a trend which is specially significant for the four-heavy (unconstrained
in the linear fit) and two-light-two-heavy operators which modify the properties
of the top quark.
%
The only exception is the charm Yukawa coefficient $c_{\varphi c}$, since there the quadratic
corrections introduce a second degenerate solution thus enlarging the magnitude
of the CL interval.

%%%%%%%%%%%%%%%%%%%%%%%%%%%%%%%%%%%%%%%%%%%%%%%%%%%%%%%%%%%%%%%%%%%%%
\begin{figure}[t]
  \begin{center}
 \includegraphics[width=0.85\linewidth]{plots_v2/Coeffs_Bar_Baseline.pdf}
 \includegraphics[width=0.85\linewidth]{plots_v2/Coeffs_Residuals_Baseline.pdf}
 \vspace{-0.3cm}
 \caption{\small The magnitude of the 95\% CL intervals (top)
   and the value of the 68\% CL residuals compared to the SM hypothesis (bottom panel)
   corresponding to the global fit results
   displayed in Fig.~\ref{fig:globalfit-baseline-coeffsabs-lin-vs-quad}.
   %
   In the upper plot, the dashed horizontal line indicates the maximum
   prior volume used for the sampling of unconstrained coefficients.
         \label{fig:globalfit-baseline-bounds-lin-vs-quad} }
  \end{center}
\end{figure}
%%%%%%%%%%%%%%%%%%%%%%%%%%%%%%%%%%%%%%%%%%%%%%%%%%%%%%%%%%%%%%%%%%%%%%

The 68\% CL residuals displayed in the bottom panel of
Fig.~\ref{fig:globalfit-baseline-bounds-lin-vs-quad} are defined by
\be
\label{eq:fit_residual}
R(c_i)\equiv \frac{\lp c_i|_{\rm EFT} -c_i|_{\rm SM} \rp }{\delta c_i} \, ,\qquad
i=1,\ldots,n_{\rm op} \, ,
\ee
with $c_i|_{\rm EFT}$ being the median of the posterior distribution
from the EFT fit, $c_i|_{\rm SM}=0$, and $\delta c_i$ is the total fit uncertainty
for this parameter.
%
We can observe that $|R(c_i)|\lsim 1$ for most of the fit coefficients,
both for the linear and quadratic cases.
%
The only exception is $c_{tG}$, where a residual of
$R(c_{tG})\simeq 3.5$ is found in the quadratic fit.
%
Nevertheless, for a large enough number of EFT coefficients
one would expect a fraction of these residuals to be larger than
unity, even if the SM is the
underlying theory.
%
Fig.~\ref{fig:residuals-histo} then displays
the normalised distribution of
these fit residuals.
%
While these coefficients are correlated among them (see the
following discussion) and thus
cannot be treated as independent variables,
the shapes of these distributions are reasonably close
to a Gaussian, specially for the linear fit, highlighting
again the overall consistency of the fit results
with the SM expectations.

%%%%%%%%%%%%%%%%%%%%%%%%%%%%%%%%%%%%%%%%%%%%%%%%%%%%%%%%%%%%%%%%%%%%%
\begin{figure}[t]
  \begin{center}
\includegraphics[width=0.70\linewidth]{plots_v2/Coeffs_Residuals_Hist_Baseline.pdf}
    \vspace{-0.2cm}
    \caption{\small The (normalised) distribution of
      the fit residuals shown in the bottom panel of
      Fig.~\ref{fig:globalfit-baseline-bounds-lin-vs-quad}.
     \label{fig:residuals-histo} }
  \end{center}
\end{figure}
%%%%%%%%%%%%%%%%%%%%%%%%%%%%%%%%%%%%%%%%%%%%%%%%%%%%%%%%%%%%%%%%%%%%%%

\paragraph{Correlations.}
%
The correlation coefficient between any two fit coefficients
$c_i$ and $c_j$ can be evaluated as follows,
\be
\label{eq:correlationL2CT}
\rho\lp c_i,c_j\rp=\frac{\lp \frac{1}{N_{\rm spl}}\sum_{k=1}^{N_{\rm spl}}
c_i^{(k)} c_j^{(k)}\rp -\la c_i\ra \la c_j\ra
}{\delta c_i \delta c_j} \, ,\qquad i,j=1,\ldots,n_{\rm op} \, ,
\ee
where $N_{\rm spl}$ denotes the number of samples produced by NS, $\la c_i\ra$ indicates
the mean value of this coefficient, and, as in Fig.~\ref{eq:fit_residual},
$\delta c_i$ is the corresponding uncertainty (standard deviation).
%
The values of Eq.~(\ref{eq:correlationL2CT})
are displayed in Fig.~\ref{fig:globalfit-correlations} 
separately for the linear  and quadratic fits.
%
We display only the numerical values for the pair-wise
coefficient combinations for which the correlation coefficient
is numerically significant,
$|\rho(c_i,c_j)|\ge 0.5$.
%
The pairs $(c_i,c_j)$ that do not appear in Fig.~\ref{fig:globalfit-correlations}  have a correlation
coefficient below this threshold.

%%%%%%%%%%%%%%%%%%%%%%%%%%%%%%%%%%%%%%%%%%%%%%%%%%%%%%%%%%%%%%%%%%%%%
\begin{figure}[htbp]
  \begin{center}
  \includegraphics[width=0.49\linewidth]{plots_v2/Coeffs_Corr_Baseline_lin.pdf}
  \includegraphics[width=0.49\linewidth]{plots_v2/Coeffs_Corr_Baseline_quad.pdf}
  \vspace{-0.3cm}
  \caption{\small The correlation coefficients $\rho\lp c_i,c_j\rp$
    between the EFT coefficients
      in the linear (left) and quadratic (right panel) fits.
      %
      We only display the  entries with significant (anti)-correlation,
      $|\rho|\ge 0.5$.
      %
      Pairs of coefficients $(c_i,c_j)$ that do not displayed here have a correlation
coefficient below this threshold.
     \label{fig:globalfit-correlations} }
  \end{center}
\end{figure}
%%%%%%%%%%%%%%%%%%%%%%%%%%%%%%%%%%%%%%%%%%%%%%%%%%%%%%%%%%%%%%%%%%%%%%

We observe how the  majority of
the fit coefficients are  loosely correlated among them, that is,
their correlations being $|\rho| \le 0.5$.
%
One  also finds that while several of the two-light-two-heavy coefficients turn out to be
strongly correlated
at the linear EFT level, this pattern disappears once   the quadratic  corrections
are accounted for.
%
Concerning the two-fermion operators,
the correlation patterns present at the linear level
are also often reduced  in the quadratic fits.
%
For instance, $c_{tZ}$ displays a strong correlation with $c_{\varphi B}$ at the linear level
which is then washed out  by the quadratic effects.
%
The purely bosonic operators exhibit
in general more stable correlations, for example $c_{\varphi W B}$ 
is strongly anti-correlated  with
$c_{\varphi D}$ in a manner which is similar in the linear and the quadratic fits.
%
Furthermore,
we do not find any pair of fit coefficients where the quadratic corrections
flip the sign of their correlation.

In general, from Fig.~\ref{fig:globalfit-correlations} one can
conclude that only a moderate subset of Wilson coefficients end
up being strongly (anti-)correlated
among them after the fit, specially so once quadratic EFT corrections are taken into account.
%
This finding is partially explained by our wide input dataset,
which makes possible constraining independently most if not all
the EFT degrees of freedom.

{  For completeness, App.~\ref{sec:fullcovmat}
  provides the
  correlation matrices for the complete set of operators
considered in this analysis.}

%%%%%%%%%%%%%%%%%%%%%%%%%%%%%%%%%%%%%%%%%%%%%%%%%%%%%%%%%%%%%%%%%%%%%%%%
%%%%%%%%%%%%%%%%%%%%%%%%%%%%%%%%%%%%%%%%%%%%%%%%%%%%%%%%%%%%%%%%%%%%%%%%

\paragraph{Individual fits.}
%
As motivated in Sect.~\ref{sec:quarticfits}, individual
(one-parameter) fits have several useful applications.
%
These include representing a benchmark reference for the global fit
results, where the obtained bounds can only loosen as compared
to one-parameter fits.
%
The 95\% CL bounds associated to the one-parameter linear and quadratic EFT
fits were reported in Table~\ref{tab:coeff-bounds-baseline},
and the corresponding graphical comparison with the marginalised
global fit results is displayed in
Fig.~\ref{fig:globalfit-baseline-bounds-lin-NS-ind-vs-marg}.
%
{  While the expectation is that individual bounds
are comparable or more stringent than the marginalised ones, }
this property does not necessarily hold for the coefficients
constrained by the EWPOs, for which an individual fit is not meaningful.
%
{  Indeed, one-parameter fits are ill-defined in
  the case of the coefficients constrained by EWPOs since
  these coefficients cannot be determined independently from each other.
  %
  Hence the comparison between marginalised and individual bounds
  is only meaningul for the 34 independent coefficients listed
  in Table~2.5.
}

%%%%%%%%%%%%%%%%%%%%%%%%%%%%%%%%%%%%%%%%%%%%%%%%%%%%%%%%%%%%%%%%%%%%%
\begin{figure}[t]
  \begin{center}
    \includegraphics[width=0.91\linewidth]{plots_v2/Coeffs_Bar_individ_lin.pdf}
    \includegraphics[width=0.91\linewidth]{plots_v2/Coeffs_Bar_individ_quad.pdf}
    \vspace{-0.4cm}
    \caption{\small Comparison of the magnitude of 95\% CL intervals in the global
      (marginalised) and individual fits at the linear (top) and quadratic
      (bottom) level, see also Table~\ref{tab:coeff-bounds-baseline}.
     \label{fig:globalfit-baseline-bounds-lin-NS-ind-vs-marg} }
  \end{center}
\end{figure}
%%%%%%%%%%%%%%%%%%%%%%%%%%%%%%%%%%%%%%%%%%%%%%%%%%%%%%%%%%%%%%%%%%%%%%

Considering first the results of the linear analysis,
one can observe how for the fitted degrees of freedom the individual bounds are tighter
(or at most comparable) than the marginalised
ones by a large amount, around a factor ten or more in most cases.
%
These differences are particularly striking for some of the two-fermion operators,
in particular for $c_{tZ}$, as well as for bosonic operators such as
$c_{\varphi B}$ and $c_{\varphi W}$,
for which the differences between the individual and marginalised results can be as large
as two orders of magnitude.
%
Specifically, in the cases of $c_{tZ}$ and $c_{\varphi B}$, the 95\% CL intervals
found in the linear EFT analysis are
increased as follows when going from the individual to the marginalised fits:
\bea
\nonumber
c_{tZ}:\qquad [-0.04,0.10]~~{\rm (individual)}  \quad&{\rm vs}&         \quad [-17,5.6]~~{\rm (marginalised)} \, ,\\
c_{\varphi B}:~~ \quad [-0.005,0.002] ~~{\rm (individual)}  \quad&{\rm vs}&   \quad  [-0.7,0.3]~~{\rm (marginalised)}\, .
\nonumber
\eea
%
This effect clearly emphasizes the importance of adopting a fitting basis as wide as possible,
in order to avoid obtaining artificially stringent bounds simply because one is being
blind to other relevant directions of the parameter space.
%
One important exception of this rule would be those cases where one is guided by 
specific UV-complete models, which motivate the reduction in the parameter space
to a subset of operators.
%
We also note that the triple gauge operator $c_W$ is one of the few coefficients whose individual
and marginalised bounds are identical: this can be traced back
to the fact that this operator is very weakly correlated with
other coefficients (see also  Fig.~\ref{fig:globalfit-correlations}), being
constrained exclusively by the diboson data.

Inspection of the corresponding results
from the quadratic fits, bottom panel of Fig.~\ref{fig:globalfit-baseline-bounds-lin-NS-ind-vs-marg}, 
 reveals that the differences between individual and marginalised bounds are in general
smaller as compared to the linear case.
%
This effect is particularly visible
for the two-light-two-heavy and the four-heavy operators, for which one finds
that the individual fits underestimate the magnitude of the 95\% CL interval by around
a factor two on average, rather than by a factor 10 as in the linear case.
%
The situation is instead similar to the linear fits for the two-fermion and the purely
bosonic operators, and for example now also for  $c_{tZ}$, $c_{\varphi B}$ and $c_{\varphi W}$ one finds
large differences between marginalised and individual fits.
%
One should point out, however,
that even on those cases where the magnitude of the bound
does not vary much, the central best-fit values can still shift in a non-negligible manner.

%%%%%%%%%%%%%%%%%%%%%%%%%%%%%%%%%%%%%%%%%%%%%%%%%%%%%%%%%%%%%%%%%%%%%%%%
%%%%%%%%%%%%%%%%%%%%%%%%%%%%%%%%%%%%%%%%%%%%%%%%%%%%%%%%%%%%%%%%%%%%%%%%

\paragraph{Two-parameter fits.}
%
To complement the insights provided by individual fits, it can also be instructive
to carry out two-parameter fits, specially to investigate the relative
interplay between specific pairs of EFT coefficients.
%
In such fits, two coefficients are allowed to vary simultaneously
while the rest are set to zero.
%
To illustrate the information that can be provided
by such two-parameter fits, Fig.~\ref{fig:2Dfits} displays
representative results for fits performed at the linear order.
%
We display the 95\% CL ellipses obtained when different subsets of data are used as input,
as well as for the complete dataset, labelled as ``All Data (2D)''.
%
For reference, we also show here the marginalised bounds obtained
from the global fit.

%%%%%%%%%%%%%%%%%%%%%%%%%%%%%%%%%%%%%%%%%%%%%%%%%%%%%%%%%%%%%%%%%%%%%
\begin{figure}[t]
  \begin{center}
    \includegraphics[width=0.32\linewidth]{plots_v2/Ellipse_OtG_Otp_NLO_NHO.pdf}
    \includegraphics[width=0.32\linewidth]{plots_v2/Ellipse_OtG_OpG_NLO_NHO.pdf}
    \includegraphics[width=0.32\linewidth]{plots_v2/Ellipse_OpG_Otp_NLO_NHO.pdf}
    \includegraphics[width=0.32\linewidth]{plots_v2/Ellipse_O81qq_O83qq_NLO_NHO.pdf}
    \includegraphics[width=0.32\linewidth]{plots_v2/Ellipse_O8dt_O8ut_NLO_NHO.pdf}
    \includegraphics[width=0.32\linewidth]{plots_v2/Ellipse_O8dt_O8qd_NLO_NHO.pdf}
    \vspace{-0.2cm}
    \caption{\small Representative results for two-parameter fits carried out
      at linear order in the EFT.
      %
      We display the 95\% CL ellipses obtained for different data subsets
      and for the complete dataset, labelled as ``All Data (2D)''.
      %
      For reference, we also show the marginalised bounds obtained
      in the global fit.
      %
      The black square in the center of the plot indicates the SM value.
     \label{fig:2Dfits} }
  \end{center}
\end{figure}
%%%%%%%%%%%%%%%%%%%%%%%%%%%%%%%%%%%%%%%%%%%%%%%%%%%%%%%%%%%%%%%%%%%%%%

To begin with, the upper panels of Fig.~\ref{fig:2Dfits} display two-parameter fits
for the three possible pair-wise combinations of the $c_{t \varphi}$, $c_{tG}$, and $c_{\varphi G}$
coefficients,
which connect Higgs production in gluon fusion with top quark pair production, see also
the Fisher information table of  Fig.~\ref{fig:FisherMatrix}.
%
These comparisons illustrate the relative impact of the various dataset in constraining
each coefficient.
%
For example, from the $\lp c_{t\varphi},c_{tG}\rp$ fit we see that the sensitivity of
$c_{tG}$ is driven by $t\bar{t}$ data, while the Higgs differential
measurements have a flat direction
resulting in a elongated ellipse.
%
The overlap between $t\bar{t}$ data and  Higgs differential
measurements results in similar constraints as compared to those
provided by the Higgs signal strengths alone.
%
Note that, as in the case of the individual fits reported in
Fig.~\ref{fig:globalfit-baseline-bounds-lin-NS-ind-vs-marg}, also for two-parameter fits
the obtained bounds are more stringent as compared to the global marginalised results.
%
Similar considerations apply to the $\lp c_{\varphi G},c_{tG}\rp$ fit, while
from the $\lp c_{\varphi G},c_{t\varphi}\rp$ one learns that the sensitivity
is still dominated by the Higgs signal strengths rather than by the differential cross-section
measurements.

Then the bottom panels of Fig.~\ref{fig:2Dfits} display two-parameter fits involving the
two-light-two-heavy coefficients $c_{Qq}^{1,8}$, $c_{Qq}^{3,8}$, $c_{tu}^8$, $c_{td}^8$, and $c_{tq}^8$,
all of which are constrained mostly from top quark pair differential distributions as indicated
by the Fisher information matrix.
%
Here the scope is to illustrate the relative sensitivity provided by some of the $t\bar{t}$
datasets that enter the fit: single-inclusive $m_{t\bar{t}}$ distributions, the double-differential
$(m_{t\bar{t}},y_{t\bar{t}})$ distributions, and $t\bar{t}V$ measurements.
%
The results confirm both that the $m_{t\bar{t}}$ distributions completely dominate
the fit of these coefficients, and that the marginalised CL ellipses are rather broader
than for the two-dimensional fits.
%
The latter is again in agreement with the results of the individual
linear fits, reported in the upper panel of
Fig.~\ref{fig:globalfit-baseline-bounds-lin-NS-ind-vs-marg}.

%%%%%%%%%%%%%%%%%%%%%%%%%%%%%%%%%%%%%%%%%%%%%%%%%%%%%%%%%%%%%%%%%%%%%%%%%%%%%%%%%%%%%%%%
%%%%%%%%%%%%%%%%%%%%%%%%%%%%%%%%%%%%%%%%%%%%%%%%%%%%%%%%%%%%%%%%%%%%%%%%%%%%%%%%%%%%%%%%



\subsection{Dataset dependence}
\label{sec:dataset_dependence}

The discussion so far has focused on the output of the global fits
obtained for the baseline dataset summarised in
Tables~\ref{eq:input_datasets} to~\ref{eq:input_datasets4}.
%
Here we aim to assess the dependence of these results with respect to the choice
of input dataset.
%
With this purpose, we consider here fits for the following variations:
\begin{itemize}

\item A fit which includes only top quark measurements.
  %
  This fit makes possible quantifying the interplay
  between the top and the Higgs data in the global fit.

\item A fit which includes only Higgs boson production and decay data,
  which provides complementary information as compared to
  the top-only fit.

\item  A fit which includes only top quark measurements, but now restricted to
  the same dataset as in our original study from~\cite{Hartland:2019bjb}.
  %
  This comparison allows one to assess the impact
  in the top-only EFT fit of the new LHC top quark measurements that have
  become available in the last two years.

\item A fit where the diboson data is removed, to determine how much weight 
  the diboson cross-sections carry in the global fit results.

\item A fit where all high-energy bins, defined as those bins
  probing the region $E\gsim 1 $ TeV, are removed.
  %
  The motivation for such a fit is to study how important are the constraints
  provided by the high-energy region in the global fit results,
  which in turn is an important input to  establish the validity
of the EFT approximation.

\item A fit where those datasets displaying poor agreement with the SM cross-sections
  are removed.
  %
  Specifically, here one removes the datasets whose $\chi^2$ differs by more
than $3\sigma$ from their statistical expectation assuming the SM hypothesis.
%
While such disagreements between data and SM theory
could very well indicate hints of BSM physics, they can also be explained
by for example issues with the  experimental correlation models.
%
Hence, this fit  allows us to verify to which extent the baseline results
are driven from the datasets that disagree the most with the SM predictions.
%
{  The datasets indicated with {\bf (*)} in Tables~\ref{eq:chi2-baseline} and~\ref{eq:chi2-baseline2} are those
  excluded from this ``conservative'' EFT fit.}

  
\end{itemize}
Note that, as explained in Sect.~\ref{sec:settings_expdata}, for the purposes
of categorisation into datasets
the $t\bar{t}h$ cross-sections are considered part of the Higgs measurements.
%
Furthermore, we note that all these fits are based on  quadratic EFT calculations
and that the constraints provided by the EWPOs on
the EFT parameter space are always accounted for.

%%%%%%%%%%%%%%%%%%%%%%%%%%%%%%%%%%%%%%%%%%%%%%%%%%%
%%%%%%%%%%%%%%%%%%%%%%%%%%%%%%%%%%%%%%%%%%%%%%%%%%%%%%%%%%%%%%%%%%%%%%%%%%%%%%%%%%%%%%%
\begin{table}[t]
  \centering
  \tiny
   \renewcommand{\arraystretch}{1.90}
  \begin{tabular}{l|C{0.8cm}|C{0.7cm}|C{1.0cm}|C{1.0cm}|C{1.0cm}|C{1.5cm}|C{0.9cm}|C{1.2cm}|C{1.2cm}}
    \multirow{3}{*}{Dataset}   & \multirow{3}{*}{$ n_{\rm dat}$} & \multirow{3}{*}{$\chi^2_{\rm SM}$} &  \multicolumn{6}{c}{$\chi^2_{\rm eft}$}   \\\cmidrule(lr{0.7em}){4-10}
    &   &  &  baseline  &  top-only  & top-only & Higgs-only &   diboson  & high-$E$   &  poor $\chi^2_{\rm sm}$  \\
       &   &  &    &  (2021)  & (2018) & &  excluded  & excluded  &  excluded  \\
 \toprule
%-------------------------------------------------------------------------------
$t\bar{t}$ incl.        &  83   & 1.46  & 1.42  &  1.44 & 1.52~(63) & --- & 1.42   &  1.40~(67)   &  0.95~(67)   \\
%-------------------------------------------------------------------------------
$t\bar{t}$ charge asym.           &  11   & 0.60  & 0.59  &  0.58 & ---       & --- & 0.60   &  0.58   &  0.56   \\
%-------------------------------------------------------------------------------
$t\bar{t}V$             &  14   &  0.65 & 0.65  & 0.69  & 0.64~(8)  & --- & 0.65   & 0.72    &   0.68     \\
%-------------------------------------------------------------------------------
single-$t$ incl.        &  27   & 0.43  & 0.41  &  0.40 & 0.36~(22) & --- & 0.41   & 0.41    &  0.46  \\
%--------------------------------------------------------------------------------------------
$tV$                    &  9    & 0.71  & 0.75  &  0.65 & 0.76~(6)  & --- & 0.75   & 0.80    &  0.31~(8)   \\
%--------------------------------------------------------------------------------------------
 $t\bar{t}Q\bar{Q}$     &  6    & 1.68  & 2.12  &  2.29 &  4.73~(2) & --- & 2.12   & 2.40    &  1.54~(4)   \\
%-------------------------------------------------------------------------------
{\bf Top total}         & 150   & 1.10  & 1.09  &  1.10 & 1.22~(101)& --- & 1.09   & 1.06~(134) &  0.82~(123)   \\
\midrule
%-------------------------------------------------------------------------------
Higgs $\mu_i^f$ (RI)    &  22   & 0.86  & 0.90  &  ---  &  ---      &   0.90  &  0.89   & 0.89  &  0.89     \\
%-------------------------------------------------------------------------------
Higgs $\mu_i^f$ (RII)   &  40   & 0.67  & 0.63  &  ---  &  ---      &   0.63  &  0.62   & 0.63  &  0.62    \\
%-------------------------------------------------------------------------------
Higgs  STXS             &  35   & 0.88  & 0.83  &  ---  &  ---      &   0.82  &  0.83   & 0.83  &  0.83   \\
%-------------------------------------------------------------------------------
{\bf Higgs  total}      &  97   & 0.78  & 0.76  &  ---  &  ---      &   0.76   & 0.76   & 0.76  &  0.76   \\
\midrule
{\bf Diboson}           &  70   & 1.31  & 1.30  &  ---  &  ---      &  ---  &   ---  &  1.31   &  1.30    \\
%-------------------------------------------------------------------------------    
\bottomrule
    {\bf Total $n_{\rm dat}$}    & 317    &  317   &  317   &  150  & 101  & 97   &247  & 301  & 287    \\
    {\bf Total $\chi^2$}    & ---    &  {\bf 1.05}    &  {\bf 1.04}  &  {\bf 1.10}  &  {\bf 1.22}  &
    {\bf 0.75}  &  {\bf 0.96} &  {\bf 1.02}   & {\bf 0.89}    \\ 
%-------------------------------------------------------------------------------
\bottomrule
\end{tabular}
  \caption{\small Same as Table~\ref{eq:chi2-baseline-grouped}  for EFT fits obtained from
    variations of the baseline dataset.
    %
    We list the results of the following fits: including only  top quark measurements (either for the 2018 or the current
    dataset); a Higgs-only dataset; without the diboson cross-sections; with the high-energy bins excluded;
    and with the datasets with a poor $\chi^2_{\rm sm}$ excluded.
    %
    In all cases, the  quadratic EFT corrections are accounted for.
    %
    The numbers in parentheses indicate the number of data points, in the case that these are different
    from those of the baseline settings (listed in the second column).
\label{eq:chi2-datasetvariations}
}
\end{table}
%%%%%%%%%%%%%%%%%%%%%%%%%%%%%%%%%%%%%%%%%%%%%%%%%%%%%%%%%%%%%%%%%%%%%%%%%%%%%%%%


%%%%%%%%%%%%%%%%%%%%%%%%%%%%%%%%%%%%%%%%%%%%%%%%%%%

To begin with,  Table~\ref{eq:chi2-datasetvariations}
collects the values of the $\chi^2$ per data points for EFT fits obtained from
variations of the input dataset.
%
We list the results of the various fits described above:
including only  top quark measurements (either from the current or the 2018
dataset); with a Higgs-only dataset; without the diboson cross-sections; with the high-energy bins excluded;
and with the datasets with a poor $\chi^2_{\rm sm}$ excluded.
%
The numbers in parentheses indicate the number of data points, in the case that these are different
from those of the baseline settings listed in the second column.
%
We observe how the description of the Higgs cross-sections is essentially
unaffected in these fits with reduced datasets.
%
Concerning the total $\chi^2$ for the top data, we see that it is stable in the fit
where the high-energy bins are removed, but that is markedly improved (from 1.10 to 0.82)
in the fit where the datasets with poor $\chi^2_{\rm sm}$ are excluded
and the number of top-quark points in the fit decreases from $n_{\rm dat}=150$
to 123.

Then in Fig.~\ref{fig:global_vs_toponly} we compare the magnitude of the 95\% CL bounds,
same as in the upper panel of Fig.~\ref{fig:globalfit-baseline-bounds-lin-vs-quad},
between the global fit results with those obtained in the top-only 
and Higgs-only fits.
%
As mentioned above, these fits
allow us to  assess the interplay
between the top and the Higgs data in the global analysis, in other words,
to identify what are the main benefits of the simultaneous mapping of the EFT parameter space
as compared to carrying out separate fits to each group of processes.
%
First of all, we note that the global fit bounds are more stringent for
all the EFT coefficients than in either the top-only or Higgs-only fit, highlighting the overall
consistency of the two datasets.
%
Secondly, the cross-talk of the top and Higgs data is found to be most
relevant for the two-fermion coefficients $c_{\varphi t}$
and $c_{\varphi Q}^{(-)}$, whose bounds are improved
by around a factor 2 in the global fit as compared to the top-only fit.
%
Another operator that benefits from the global fit is
$c_{\varphi G}$, which is unconstrained in the top-only fit but
whose bound in the global fit is clearly improved as compared to the Higgs-only fit.
%
These comparisons show how by breaking degeneracies one gains information in the global fit as compared to the partial ones, sometimes in unexpected directions in the parameter space such as for $c_{\varphi G}$  in this case.
%
The bottom panel of Fig.~\ref{fig:global_vs_toponly} also indicates that in a Higgs-only fit
a large number of EFT coefficients are poorly constrained, in particular
those involving fermion bilinears.

%%%%%%%%%%%%%%%%%%%%%%%%%%%%%%%%%%%%%%%%%%%%%%%%%%%%%%%%%%%%%%%%%%%%%
\begin{figure}[t]
  \begin{center}
    \includegraphics[width=0.90\linewidth]{plots_v2/Coeffs_Bar_Top_only.pdf}
    \includegraphics[width=0.90\linewidth]{plots_v2/Coeffs_Bar_Higgs_only.pdf}
    \caption{\label{fig:global_vs_toponly} \small
      Same as upper panel of Fig.~\ref{fig:globalfit-baseline-bounds-lin-vs-quad}
      now comparing the global fit results with those obtained in a top-only (upper)
    and Higgs-only (lower panel) fits.}
  \end{center}
\end{figure}
%%%%%%%%%%%%%%%%%%%%%%%%%%%%%%%%%%%%%%%%%%%%%%%%%%%%%%%%%%%%%%%%%%%%%%

%%%%%%%%%%%%%%%%%%%%%%%%%%%%%%%%%%%%%%%%%%%%%%%%%%%%%%%%%%%%%%%%%%%%%
\begin{figure}[t]
  \begin{center}
    \includegraphics[width=0.90\linewidth]{plots_v2/Coeffs_Bar_noVV.pdf}
   \includegraphics[width=0.90\linewidth]{plots_v2/Coeffs_Bar_top19_top21.pdf}
    \caption{\label{fig:global_vs_top2019} \small
      Same as Fig.~\ref{fig:global_vs_toponly},
      now comparing the global fit with a no-diboson fit (upper)
    and the two top-only fits with different datasets (lower panel).}
  \end{center}
\end{figure}
%%%%%%%%%%%%%%%%%%%%%%%%%%%%%%%%%%%%%%%%%%%%%%%%%%%%%%%%%%%%%%%%%%%%%%

Next, Fig.~\ref{fig:global_vs_top2019} displays a
similar comparison as in Fig.~\ref{fig:global_vs_toponly}
now comparing first the outcome of the global fit with that of a fit
where the diboson cross-sections have been removed,
and second comparing two top-only fits, namely the fit displayed
in the upper panel of Fig.~\ref{fig:global_vs_toponly} with  a
fit based on the same dataset as our previous study from~\cite{Hartland:2019bjb}.
%
The fit without diboson data demonstrates that the constraints provided by the diboson
cross-sections are negligible in comparison with those provided by the Higgs data
(and the EWPOs) for all coefficients considered in the fit, except for the triple
gauge operator $c_{W}$.
%
This result is consistent with the Fisher information
analysis of Fig.~\ref{fig:FisherMatrix}, and indicates that, apart from $c_W$, the diboson
data does not provide competitive information on the EFT parameter space in the context
of a global fit.

The comparison of the two top-only fits in the bottom
panel of Fig.~\ref{fig:global_vs_top2019} illustrates how for all coefficients
the bounds  are improved thanks to the more recent LHC measurements.\footnote{Recall that now we consider
  the top Yukawa coefficient $c_{t\varphi}$ as part of the Higgs dataset.}
%
The improvement is consistent across the board, quantifies the additional information
brought in by the new top-quark cross-section measurements
(see Table~\ref{eq:chi2-datasetvariations}),
and confirms that the broader and more diverse the input dataset is,
 the more stringent the resulting constraints
on the EFT parameter space that will be obtained.

%%%%%%%%%%%%%%%%%%%%%%%%%%%%%%%%%%%%%%%%%%%%%%%%%%%%%%%%%%%%%%%%%%%%%
\begin{figure}[t]
  \begin{center}
    \includegraphics[width=0.80\linewidth]{plots_v2/Coeffs_Central_goodSM.pdf}
   \includegraphics[width=0.80\linewidth]{plots_v2/Coeffs_Central_nohighE.pdf}
   \caption{\label{fig:global_vs_gooddata} \small
     Same as  Fig.~\ref{fig:globalfit-baseline-coeffsabs-lin-vs-quad}
     comparing the global fit results with those of the fit excluding datasets with poor $\chi^2_{\rm sm}$
   (upper) and with the fit where the bins with $E\gsim 1$ TeV are removed (bottom panel).}
  \end{center}
\end{figure}
%%%%%%%%%%%%%%%%%%%%%%%%%%%%%%%%%%%%%%%%%%%%%%%%%%%%%%%%%%%%%%%%%%%%%%

To continue with this discussion of the dataset dependence of our results, we consider now
the outcome of
two more fits: first, one where the datasets exhibiting poor agreement with the SM predictions
are excluded, and second, another where all bins sensitive to the high-energy region,
defined as $E\gsim 1 $ TeV, are removed.
%
The best-fit values and 95\% CL intervals of these two fits are compared
with the baseline results in Fig.~\ref{fig:global_vs_gooddata}.
%
As indicated in Table~\ref{eq:chi2-datasetvariations}, in the fit where those
datasets with poor $\chi^2_{\rm sm}$
have been removed, one is essentially cutting away 27 points from top quark production, mostly
from the inclusive $t\bar{t}$ category.
%
The only coefficients that are affected by this reduction in the dataset are some of the
two-light-two-heavy operators, whose bounds are mildly enlarged consistently
with the loss of experimental information.
%
This comparison highlights the stability of the global fit results, whose
outcome is unchanged when potentially problematic datasets with high $\chi^2_{\rm sm}$
are excluded from the fit.
%
Concerning the outcome of the fit without the high-energy bins, as expected the only differences
are observed again for the two-light-two-heavy coefficients, with a similar
outcome as in the previous fit.
%
From this analysis, one can conclude that the global fit is not dominated by the high-energy regions
where the EFT validity could be questioned, and hence that results are stable
upon removal of these high-energy bins.

Finally, we show in Fig.~\ref{fig:wo_CMS2Dttbar} a comparison
of the outcome of quadratic EFT fits with and without
the CMS top-quark pair double-differential $(m_{t\bar{t}},y_{t\bar{t}})$ distributions.
%
We have identified this dataset as the one being responsible for
driving upwards the fit value of the chromo-magnetic
operator $c_{tG}$.
%
Indeed, 
one can observe how once this dataset is removed
then $c_{tG}$ agrees with the SM at the 95\% CL.
%
Given that both in the global linear and the individual quadratic fits
$c_{tG}$ also agrees with the SM (even in fits where {\tt CMS\_tt2D\_8TeV\_dilep\_mttytt}
is included), the pull found in the global quadratic case must arise from a non-trivial
interplay between different EFT degrees of freedom.
%
Further studies are required to elucidate why this specific dataset has such as strong
pull on $c_{tG}$ in the quadratic fits.

%%%%%%%%%%%%%%%%%%%%%%%%%%%%%%%%%%%%%%%%%%%%%%%%%%%%%%%%%%%%%%%%%%%%%
\begin{figure}[t]
  \begin{center}
    \includegraphics[width=0.80\linewidth]{plots_v2/Coeffs_Central_Fit_no_CMS2Dtt.pdf}
   \caption{\label{fig:wo_CMS2Dttbar} \small
    The outcome of quadratic EFT fits with and without
   the CMS top-quark pair double-differential $(m_{t\bar{t}},y_{t\bar{t}})$ distributions.}
  \end{center}
\end{figure}
%%%%%%%%%%%%%%%%%%%%%%%%%%%%%%%%%%%%%%%%%%%%%%%%%%%%%%%%%%%%%%%%%%%%%%

%%%%%%%%%%%%%%%%%%%%%%%%%%%%%%%%%%%%%%%%%%%%%%%%%%%%%%%%%%%%%%%%
%%%%%%%%%%%%%%%%%%%%%%%%%%%%%%%%%%%%%%%%%%%%%%%%%%%%%%%%%%%%%%%%




\clearpage
\subsection{Impact of NLO QCD corrections in the EFT cross-sections}
\label{subsec:loqcd}

In addition to the choice of input dataset, another important
factor that determines the outcome of a global analysis
such as the present one is the  accuracy
of the EFT theoretical calculations.
%
Here we assess the role played at the level of the fit results
by the inclusion of NLO QCD corrections to the EFT cross-sections,
both in the linear and in the quadratic fits.
%
As indicated in Table~\ref{eq:table-processes-theory},
our baseline fit includes these NLO corrections
to the EFT calculations whenever available, so now we switch them off deliberately
to quantify how much they affect the fit outcome.\footnote{This study
  is also motivated by the fact that many EFT fits
rely on LO QCD for the EFT cross-sections.}
%
In the following, the theoretical predictions
for the SM cross-sections, based on the state-of-the-art calculations,
remain unchanged, and only the EFT ones are modified as compared to the baseline
settings.
%
First of all, Table~\ref{eq:chi2-theoryvariations}
compares the values of the $\chi^2$ for the various groups of processes
in quadratic fits with and without NLO QCD corrections to  the  EFT cross-sections,
as well as for the associated SM results.
%
One can observe how the overall fit quality
is similar whether or not NLO QCD effects are not accounted for.
%
Nevertheless, as will be discussed next, this does
not imply that the fit posterior distributions
are likewise unchanged.

%%%%%%%%%%%%%%%%%%%%%%%%%%%%%%%%%%%%%%%%%%%%%%%%%%%
%%%%%%%%%%%%%%%%%%%%%%%%%%%%%%%%%%%%%%%%%%%%%%%%%%%%%%%%%%%%%%%%%%%%%%%%%%%%%%%%%%%%%%%
\begin{table}[t]
%\begin{table}[t]
  \centering
  \footnotesize
   \renewcommand{\arraystretch}{1.40}
  \begin{tabular}{l|C{1.0cm}|C{1.2cm}|C{2.0cm}|C{2.7cm}|C{2.2cm}}
    \multirow{2}{*}{Dataset}   & \multirow{2}{*}{$ n_{\rm dat}$} & $\chi^2_{\rm SM}$ &  $\chi^2_{\rm EFT}$   & $\chi^2_{\rm EFT}$    & $\chi^2_{\rm EFT}$    \\
      &   &  &  (baseline)  &  (LO QCD in EFT)  & (top-philic)   \\
 \toprule
%-------------------------------------------------------------------------------
$t\bar{t}$ inclusive        &  83    &  1.46  &  1.42      &   1.39     &   1.41      \\
%-------------------------------------------------------------------------------
$t\bar{t}$ AC                & 11    &  0.60  &  0.59      &   0.57     &   0.60      \\
%-------------------------------------------------------------------------------
$t\bar{t}V$                 &  14    &  0.65  &   0.65     &   0.54     &  0.68     \\
%-------------------------------------------------------------------------------
single top inclusive        &  27    &  0.43  &   0.41     &   0.42     &  0.41      \\
%--------------------------------------------------------------------------------------------
$tV$                        &  9     &  0.71  &   0.75     &   0.68     &  0.78      \\
%--------------------------------------------------------------------------------------------
 $t\bar{t}Q\bar{Q}$         &  6     & 1.68  &  2.12       &   2.24     &  2.16       \\
%-------------------------------------------------------------------------------
{\bf Top quark total}       &  150   & 1.10  &  1.09       &   1.06     &  1.09      \\
\midrule
%-------------------------------------------------------------------------------
Higgs $\mu_i^f$  (Run I)    &  22    & 0.86  &  0.90       &  0.95     &  0.90       \\
%-------------------------------------------------------------------------------
Higgs $\mu_i^f$  (Run II)   &   40   & 0.67  &  0.63       &  0.67     &  0.63       \\
%-------------------------------------------------------------------------------
Higgs differential \& STXS  &  35    & 0.88  &  0.83       &  0.78     &  0.83       \\
%-------------------------------------------------------------------------------
{\bf Higgs  total}          &   97   & 0.78  &  0.76       &  0.77     &  0.76     \\
\midrule
{\bf Diboson}               &  70    & 1.31  &  1.30       &   1.32    &   1.30      \\
%-------------------------------------------------------------------------------    
\midrule
{\bf Global dataset}        & {\bf 317}   & {\bf 1.05}  &  {\bf 1.04}   &   {\bf 1.03}     &   {\bf 1.04}      \\
%-------------------------------------------------------------------------------
\bottomrule
\end{tabular}
  \caption{\small Same as Table~\ref{eq:chi2-baseline-grouped} now for fits based on
    variations of the theory settings as compared to the baseline ones.
    %
    Specifically, we provide the results of a fit where the EFT cross-sections are evaluated at LO in the QCD
    expansion, as well as those of the top-philic scenario where the  parameter space
    has been restricted as described in Sect.~\ref{sec:topphilic}.
    %
    In both cases, quadratic EFT corrections are being included.
    %
    Note that the SM cross-sections are always evaluated using state-of-the-art theory
    calculations.
\label{eq:chi2-theoryvariations}
}
\end{table}
%%%%%%%%%%%%%%%%%%%%%%%%%%%%%%%%%%%%%%%%%%%%%%%%%%%%%%%%%%%%%%%%%%%%%%%%%%%%%%%%


%%%%%%%%%%%%%%%%%%%%%%%%%%%%%%%%%%%%%%%%%%%%%%%%%%%s

%%%%%%%%%%%%%%%%%%%%%%%%%%%%%%%%%%%%%%%%%%%%%%%%%%%%%%%%%%%%%%%%%%%%%
\begin{figure}[htbp]
  \begin{center}
    \includegraphics[width=0.99\linewidth]{plots_v2/Coeffs_Hist_LOvsNLO_lin.pdf}
    \includegraphics[width=0.99\linewidth]{plots_v2/Coeffs_Central_LOvsNLO_lin.pdf}
    \caption{\small Top: comparison of the posterior probability
      distributions of the Wilson coefficients
      between linear fits with and without NLO QCD corrections to the EFT cross-sections.
      %
      Bottom: the corresponding 95\% CL intervals, compared to the SM expectation.
     \label{fig:posterior-distributions-NLO-vs-LO-linear} }
  \end{center}
\end{figure}
%%%%%%%%%%%%%%%%%%%%%%%%%%%%%%%%%%%%%%%%%%%%%%%%%%%%%%%%%%%%%%%%%%%%%%xs

%%%%%%%%%%%%%%%%%%%%%%%%%%%%%%%%%%%%%%%%%%%%%%%%%%%%%%%%%%%%%%%%%%%%%
\begin{figure}[htbp]
  \begin{center}
    \includegraphics[width=0.99\linewidth]{plots_v2/Coeffs_Hist_LOvsNLO_quad.pdf}
    \includegraphics[width=0.99\linewidth]{plots_v2/Coeffs_Central_LOvsNLO_quad.pdf}
    \caption{\small
      Same as Fig.~\ref{fig:posterior-distributions-NLO-vs-LO-linear}
      for the quadratic EFT fits.
      \label{fig:posterior-distributions-NLO-vs-LO-quad} }
  \end{center}
\end{figure}
%%%%%%%%%%%%%%%%%%%%%%%%%%%%%%%%%%%%%%%%%%%%%%%%%%%%%%%%%%%%%%%%%%%%%%xs

Figs.~\ref{fig:posterior-distributions-NLO-vs-LO-linear}
and~\ref{fig:posterior-distributions-NLO-vs-LO-quad} then display the
posterior probability distributions 
and the corresponding 95\% CL intervals for the Wilson coefficients,
comparing the results of linear and quadratic fits respectively
with and without NLO corrections to the EFT cross-sections.
%
Scrutinizing first the linear fit results collected in
Fig.~\ref{fig:posterior-distributions-NLO-vs-LO-linear},
one can observe that these posterior distributions can be
severely distorted when LO EFT calculations are used as compared
to the baseline,
for instance in terms of a shift in the best-fit values and/or due
to an increase in the width of the Gaussian distributions.
%
Also in the LO linear fit, all considered coefficients
agree with the SM expectation at the 95\% CL.
%
Note that the two-light-two-heavy singlet operators do not
interfere with the SM at LO,
and hence the corresponding coefficients turn out to be unconstrained
in the linear LO fit.
%
Remarkably, for several fit coefficients
such as $c_{tZ}$, $c_{\varphi B}$, and $c_{W}$, one finds that a marked
improvement in the obtained bounds is achieved upon the inclusion
of the NLO QCD corrections to the EFT cross-sections.
%
One would conclude that, at least in the global linear EFT fit,
the inclusion of NLO QCD corrections
is of clear importance to obtain both more accurate and more precise results
for the Wilson coefficients.
%
Alternatively, one could account for the missing
higher-order uncertainties (MHOUs) in the EFT cross-sections, which
are usually neglected, using for instance the approach advocated
in~\cite{AbdulKhalek:2019bux,AbdulKhalek:2019ihb}.
%
Implementing MHOUs systematically is expected to further improve
the overall compatibility of EFT fits performed with and
without NLO QCD corrections.

Moving to the associated comparisons in the case of the quadratic fits summarised in
Fig.~\ref{fig:posterior-distributions-NLO-vs-LO-quad},
also here we find that the parameter distributions can be modified in a marked
way depending on whether or not NLO QCD calculations are adopted.
%
As an illustration, the operator that modifies the charm Yukawa interaction,
$c_{c\varphi}$, exhibits
a bimodal distribution once NLO effects are accounted for,
while the dominant solution for the $c_{\varphi t}$  coefficient
is far from the SM in the LO fit but SM-like in the NLO case
(though the 95\% CL interval itself remains stable).
%
As opposed to the case of the linear fits,
in the quadratic case one finds that the addition of NLO corrections
does not in general reduce the uncertainties
on the fit coefficients, but rather distorts
the posterior distributions and shifts the central values.
%
As an illustration,
if NLO QCD corrections are removed, the posterior
distribution for the two-light-two-heavy coefficient
$c_{Qq}^{3,8}$ is shifted such that it does not agree anymore
with the SM at the 95\% CL.


In the specific case of the $c_{\varphi t}$  coefficient, one can verify that the corresponding individual $\chi^2$ profile
(analog of Fig.~\ref{fig:quartic-individual-fits-2} for LO fits) does not
exhibit this second solution, and hence it must be induced by the cross-talk
with other coefficients in the fit.
%
To validate this hypothesis, Fig.~\ref{fig:Ellipse_Opt_OtZ_LO_HO} displays the outcome of 
 two-parameter quadratic fits for
     $(c_{\varphi t},c_{tZ})$  and $(c_{\varphi t},c_{\varphi W})$ 
 comparing the results of the LO EFT fit  with its NLO counterpart.
 %
 In both cases, the LO two-parameter fits based on the full dataset
 favour the solution far from the SM, while the NLO ones
 instead favour the SM-like one.
 %
 The explanation for this behaviour can be traced back
 to the fact that the non-SM solution is disfavored
 by the NLO EFT corrections to $hZ$ associated production,
 in particular those related to gluon-induced contributions.

%%%%%%%%%%%%%%%%%%%%%%%%%%%%%%%%%%%%%%%%%%%%%%%%%%%%%%%%%%%%%%%%%%%%%
\begin{figure}[t]
  \begin{center}
    \includegraphics[width=0.41\linewidth]{plots_v2/Ellipse_Opt_OtZ_LO_HO.pdf}
    \includegraphics[width=0.41\linewidth]{plots_v2/Ellipse_Opt_OtZ_NLO_HO.pdf}
    \includegraphics[width=0.41\linewidth]{plots_v2/Ellipse_Opt_OpW_LO_HO.pdf}
    \includegraphics[width=0.41\linewidth]{plots_v2/Ellipse_Opt_OpW_NLO_HO.pdf}
    \caption{\small
      Same as Fig.~\ref{fig:2Dfits} for the two-parameter quadratic fits
      of $(c_{\varphi t},c_{tZ})$ (upper) and $(c_{\varphi t},c_{\varphi W})$ (lower panels)
      comparing the results of the LO EFT fit (left) with its NLO counterpart (right panels)
      \label{fig:Ellipse_Opt_OtZ_LO_HO} }
  \end{center}
\end{figure}
%%%%%%%%%%%%%%%%%%%%%%%%%%%%%%%%%%%%%%%%%%%%%%%%%%%%%%%%%%%%%%%%%%%%%%xs

Another remarkable effect of the  NLO QCD corrections
to the EFT cross-sections can be observed in the modified
correlation patterns.
%
Fig.~\ref{fig:Coeffs_Corr_260121-NS_GLOBAL_LO} displays
the same correlations maps as in Fig.~\ref{fig:globalfit-correlations} now
for global fits based
on LO EFT calculations at  the linear and quadratic level.
%
Specially for the linear fits, we observe that correlations
become more sizable in general
for the two-fermion and purely bosonic operators,
while these are
reduced once NLO corrections are accounted for.
%
This feature demonstrates how NLO QCD effects may reduce parameter correlations
by introducing additional sensitivity to the fit coefficients for the same input dataset.

%%%%%%%%%%%%%%%%%%%%%%%%%%%%%%%%%%%%%%%%%%%%%%%%%%%%%%%%%%%%%%%%%%%%%
\begin{figure}[t]
  \begin{center}
    \includegraphics[width=0.49\linewidth]{plots_v2/Coeffs_Corr_NS_GLOBAL_LO_NHO.pdf}
    \includegraphics[width=0.49\linewidth]{plots_v2/Coeffs_Corr_NS_GLOBAL_LO_HO.pdf}
     \caption{\small
       Same as Fig.~\ref{fig:globalfit-correlations} for
       LO EFT calculations in linear (left) and quadratic (right) fits.
      \label{fig:Coeffs_Corr_260121-NS_GLOBAL_LO} }
  \end{center}
\end{figure}
%%%%%%%%%%%%%%%%%%%%%%%%%%%%%%%%%%%%%%%%%%%%%%%%%%%%%%%%%%%%%%%%%%%%%%xs


\subsection{The top-philic scenario}

To conclude this section, we present results for a global EFT fit
carried out in the top-philic scenario defined in Sect.~\ref{sec:topphilic}.
%
In this scenario, we have the 9 equations of Eq.~(\ref{eq:topphilic}) that 
relate a subset of the 14
two-heavy-two-light coefficients listed in Table~\ref{tab:operatorbasis}
among them,  leaving 5 independent parameters to be constrained
in the fit.
%
Given the more constraining assumptions associated
to the top-philic scenario, one expects to find an improvement
in the bounds of the two-light-two-heavy EFT operators due to the fact
that the parameter space is being restricted by theoretical
considerations, rather than by data in this case.

The values of the $\chi^2$ for each group of datasets in the top-philic scenario
were reported in Table~\ref{eq:chi2-theoryvariations},
where we see that the fit quality is very similar to the fit with the baseline settings.
%
Fig.~\ref{fig:Coeffs_Bar_TopPhilic} then displays the 95\% CL
intervals for the EFT coefficients comparing the
global fit results with those of the top-philic scenario.
%
The only operators that are affected in a significant manner
turn out to be the two-light-two-heavy
operators, with the bounds in several of them such as
$c_{td}^1$, $c_{Qq}^{1,1}$, and $c_{tq}^1$  improving by almost an order of magnitude.
%
The fact that only the bounds on the two-light-two-heavy operators are modified
is consistent with the top-philic scenario, given that only
this specific group of EFT coefficients is being constrained by its model assumptions.

%%%%%%%%%%%%%%%%%%%%%%%%%%%%%%%%%%%%%%%%%%%%%%%%%%%%%%%%%%%%%%%%%%%%%
\begin{figure}[t]
  \begin{center}
     \includegraphics[width=0.99\linewidth]{plots_v2/Coeffs_Bar_topphilic.pdf}
     \caption{\small Same as Fig.~\ref{fig:posterior_coeffs}
       comparing the global fit results with the same fit in the top-philic
       scenario defined by the relations in Eq.~(\ref{eq:topphilic}).
     \label{fig:Coeffs_Bar_TopPhilic} }
  \end{center}
\end{figure}
%%%%%%%%%%%%%%%%%%%%%%%%%%%%%%%%%%%%%%%%%%%%%%%%%%%%%%%%%%%%%%%%%%%%%%

It is worth emphasizing at this point that,
from the technical point of view, carrying out global EFT fits with specific restrictions
in the parameter space motivated by UV-completions, such as those arising
in the top-philic scenario and leading to Fig.~\ref{fig:Coeffs_Bar_TopPhilic},
is relatively straightforward.
%
Indeed, the most efficient fitting strategy would be to start from the broadest possible
parameter space, and once the corresponding fit has been performed,
introduce model assumptions relating EFT coefficients  in a systematic manner.
%
This way one can connect with specific models for  UV-completions of the SM,
which typically result in a rather smaller number of EFT coefficients to be constrained
from data.


