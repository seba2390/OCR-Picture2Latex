\subsection{Fit quality}

To begin with, we investigate the quality of the fit in terms
of the $\chi^2$ values for the individual
datasets as well as for the global one.
%
The values that will be provided here correspond to
a modified version of Eq.~(\ref{eq:chi2definition2}),
specifically
\begin{equation}
  \chi^2 \equiv \frac{1}{n_{\rm dat}}\sum_{i,j=1}^{n_{\rm dat}}\lp 
  \la \sigma^{(\rm th)}_i\lp {\boldsymbol c}^{(k)} \rp \ra
  -\sigma^{(\rm exp)}_i\rp ({\rm cov}^{-1})_{ij}
\lp 
 \la \sigma^{(\rm th)}_j\lp {\boldsymbol c}^{(k)} \rp \ra
  -\sigma^{(\rm exp)}_j\rp
 \label{eq:chi2definition3}
    \; ,
\end{equation}
where the average over the theory predictions is evaluated over the the $N_{\rm spl}$ samples
provided by NS,
and the covariance matrix is evaluated with the  experimental definition~\cite{Ball:2012wy}.
%
Note that in general the average over theory predictions does not correspond to the theory
prediction evaluated using the average value of the Wilson coefficients,
\be
\la \sigma^{(\rm th)}_i\lp {\boldsymbol c}^{(k)} \rp \ra \ne
\sigma^{(\rm th)}_i\lp \la  {\boldsymbol c}^{(k)}  \ra \rp \, , 
\ee
due to the presence of the quadratic corrections to the EFT cross-sections.

With the figure of merit defined in Eq.~(\ref{eq:chi2definition3}),
we collect in Tables~\ref{eq:chi2-baseline} and~\ref{eq:chi2-baseline2}
the values of the $\chi^2$ per data point corresponding to
the baseline settings of our analysis.
%
We display both the values based on the SM theory predictions
as well as the best-fit SMEFT results obtained with $\mathcal{O}\lp \Lambda^{-2}\rp$ and
$\mathcal{O}\lp \Lambda^{-4}\rp$ calculations.
%
Note that, for ease of reference, in these tables
each dataset has associated a hyperlink pointing to the original publication.
%
For those datasets for which more than one differential distribution is available, we indicate the specific one used in the fit.
%
Then Table~\ref{eq:chi2-baseline-grouped} presents the
summary of these $\chi^2$ values now indicating the total values for each group of processes
as well as for the global dataset.
%
Furthermore, the results of Tables~\ref{eq:chi2-baseline}
and~\ref{eq:chi2-baseline2} are graphically represented in Fig.~\ref{fig:chi2_barplot}.

%%%%%%%%%%%%%%%%%%%%%%%%%%%%%%%%%%%%%%%%%%%%%%%%%%%%%%%%%%%%%%
%%%%%%%%%%%%%%%%%%%%%%%%%%%%%%%%%%%%%%%%%%%%%%%%%%%%%%%%%%%%%%%%%%%%%%%%%%%%%%%%%%%%%%%
\begin{table}[htbp]
  \centering
  \footnotesize
   \renewcommand{\arraystretch}{1.55}
  \begin{tabular}{l|C{1.1cm}|C{2.1cm}|C{2.5cm}|C{2.5cm}}
   \multirow{2}{*}{ Dataset}   & \multirow{2}{*}{$ n_{\rm dat}$} & \multirow{2}{*}{ $\chi^2_{\rm SM}$} &  $\chi^2_{\rm EFT}$   & $\chi^2_{\rm EFT}$     \\
      &   &   & $\mathcal{O}\lp \Lambda^{-2}\rp$ &  $\mathcal{O}\lp \Lambda^{-4}\rp$  \\
 \toprule
%-----------------------------------------------------------------------------------------
 \href{https://arxiv.org/abs/1511.04716}{\tt ATLAS\_tt\_8TeV\_ljets\_mtt} {{\bf (*)}}
 & 7 &  2.95   &  2.46  &  2.71     \\
%-----------------------------------------------------------------------------------------
 \href{https://arxiv.org/abs/1607.07281}{\tt ATLAS\_tt\_8TeV\_dilep\_mtt} & 6 &  0.09   &  0.12  &  0.12   \\
%-----------------------------------------------------------------------------------------
\href{https://arxiv.org/abs/1505.04480}{\tt CMS\_tt\_8TeV\_ljets\_ytt} & 10 &  0.91   &  1.19   &  1.05     \\
%-----------------------------------------------------------------------------------------
\href{https://arxiv.org/abs/1703.01630}{\tt CMS\_tt2D\_8TeV\_dilep\_mttytt} & 16 &  1.63  &  1.01  & 1.12      \\
%-----------------------------------------------------------------------------------------
\href{https://arxiv.org/abs/1610.04191}{\tt CMS\_tt\_13TeV\_ljets\_2015\_mtt} & 8 &  0.94  &  0.72  &  0.97     \\
%-----------------------------------------------------------------------------------------
\href{https://arxiv.org/abs/1708.07638}{\tt CMS\_tt\_13TeV\_dilep\_2015\_mtt} & 6 &  1.30   &  1.42   & 1.52      \\
%-----------------------------------------------------------------------------------------
\href{https://arxiv.org/abs/1803.08856}{\tt CMS\_tt\_13TeV\_ljets\_2016\_mtt}  {{\bf (*)}} & 10 &  1.99  &  1.70  & 2.22      \\
%-----------------------------------------------------------------------------------------
\href{https://arxiv.org/abs/1811.06625}{\tt CMS\_tt\_13TeV\_dilep\_2016\_mtt}  {{\bf (*)}} & 7 & 2.28  & 1.96   & 2.52      \\
%-----------------------------------------------------------------------------------------
\href{https://arxiv.org/abs/1908.07305}{\tt ATLAS\_tt\_13TeV\_ljets\_2016\_mtt} & 7 & 0.99   & 1.81   & 1.02       \\
%-----------------------------------------------------------------------------------------
\href{https://arxiv.org/abs/1709.05327}{\tt ATLAS\_CMS\_tt\_AC\_8TeV} & 6 &  0.86  &  0.70  & 0.86       \\
%-----------------------------------------------------------------------------------------
\href{https://inspirehep.net/literature/1743677}{\tt ATLAS\_tt\_AC\_13TeV} & 5 &  0.03  & 0.32   &  0.26     \\
%-----------------------------------------------------------------------------------------
\href{https://arxiv.org/abs/1612.02577}{\tt ATLAS\_WhelF\_8TeV}  {{\bf (*)}} & 3 &  1.97  & 1.30   & 1.38       \\
%-----------------------------------------------------------------------------------------
\href{https://arxiv.org/abs/1605.09047}{\tt CMS\_WhelF\_8TeV} & 3 &  0.30  & 0.64   &  0.58    \\
%-----------------------------------------------------------------------------------------
\midrule
\href{https://arxiv.org/abs/1509.05276}{\tt ATLAS\_ttZ\_8TeV} & 1 & 1.31   &  0.76     &  1.24      \\
%-------------------------------------------------------------------------------
\href{https://arxiv.org/abs/1609.01599}{\tt ATLAS\_ttZ\_13TeV} & 1 &  0.01    & 0.12   &  0.05     \\
%-------------------------------------------------------------------------------
\href{https://arxiv.org/abs/1901.03584}{\tt ATLAS\_ttZ\_13TeV\_2016} & 1 & 0.001   &  0.35   &  0.10     \\
%-------------------------------------------------------------------------------
\href{https://arxiv.org/abs/1510.01131}{\tt CMS\_ttZ\_8TeV} & 1 &  0.04    &  0.19   &  0.05      \\
%-------------------------------------------------------------------------------
\href{https://arxiv.org/abs/1711.02547}{\tt CMS\_ttZ\_13TeV} & 1 & 0.90    & 0.17   &  0.41      \\
%-------------------------------------------------------------------------------
\href{https://arxiv.org/abs/1907.11270}{\tt CMS\_ttZ\_13TeV\_pTZ} & 4 &  0.73  &  0.69   & 0.91      \\
%-------------------------------------------------------------------------------
\href{https://arxiv.org/abs/1509.05276}{\tt ATLAS\_ttW\_8TeV} & 1 & 1.33   & 0.47    & 1.22      \\
%-------------------------------------------------------------------------------
\href{https://arxiv.org/abs/1609.01599}{\tt ATLAS\_ttW\_13TeV} & 1 &  0.83   &  0.56   & 0.81      \\
%-------------------------------------------------------------------------------
\href{https://arxiv.org/abs/1901.03584}{\tt ATLAS\_ttW\_13TeV\_2016} & 1 &   0.23   & 0.14    &  0.00      \\
%-------------------------------------------------------------------------------
\href{https://arxiv.org/abs/1510.01131}{\tt CMS\_ttW\_8TeV} & 1 &  1.54    &  0.68     &  1.43     \\
%-------------------------------------------------------------------------------
\href{https://arxiv.org/abs/1711.02547}{\tt CMS\_ttW\_13TeV} & 1 &  0.03   &   0.57    &  0.14     \\
%-------------------------------------------------------------------------------
\midrule
\href{https://arxiv.org/abs/1705.10141}{\tt CMS\_ttbb\_13TeV}  {{\bf (*)}} & 1 &  4.96    &  2.65     &  6.66     \\
%-------------------------------------------------------------------------------
\href{https://arxiv.org/abs/1909.05306}{\tt CMS\_ttbb\_13TeV\_2016} & 1 &  1.75    &  0.35     &  3.09     \\
%-------------------------------------------------------------------------------
\href{https://arxiv.org/abs/1811.12113}{\tt ATLAS\_ttbb\_13TeV\_2016} & 1 & 0.91     &  1.68     &  0.55     \\
%-------------------------------------------------------------------------------
\href{https://arxiv.org/abs/1710.10614}{\tt CMS\_tttt\_13TeV} & 1 &  0.05   & 0.02      &  0.08     \\
%-------------------------------------------------------------------------------
\href{https://arxiv.org/abs/1908.06463}{\tt CMS\_tttt\_13TeV\_run2} & 1 &  0.05    & 1.15      &  2.04     \\
%-------------------------------------------------------------------------------
\href{https://arxiv.org/abs/2007.14858}{\tt ATLAS\_tttt\_13TeV\_run2}  {{\bf (*)}} & 1 &  2.35   &  0.70     &  0.30     \\
%-------------------------------------------------------------------------------
\bottomrule
\end{tabular}
\caption{\small The values of the $\chi^2$ per data point corresponding to
  the baseline settings of our analysis.
  %
  We indicate the results for the $t\bar{t}$ datasets, both in inclusive
  production and in association with vector bosons or heavy quarks.
%
  We display the SM values and then the best-fit SMEFT results obtained
  in analyses based on theory predictions at either $\mathcal{O}\lp \Lambda^{-2}\rp$ or 
$\mathcal{O}\lp \Lambda^{-4}\rp$ accuracy.
%
Each dataset has a hyperlink pointing to the original publication.
%
For those datasets for which more than one differential distribution is available,
we indicate the specific
ones used in the fit.
%
    {Datasets indicated with {\bf (*)} are excluded from the ``conservative'' EFT
    fit to be discussed in Sect.~\ref{sec:dataset_dependence}.}
\label{eq:chi2-baseline}
}
\end{table}
%%%%%%%%%%%%%%%%%%%%%%%%%%%%%%%%%%%%%%%%%%%%%%%%%%%%%%%%%%%%%%%%%%%%%%%%%%%%%%%%


%%%%%%%%%%%%%%%%%%%%%%%%%%%%%%%%%%%%%%%%%%%%%%%%%%%%%%%%%%%%%%%%%%%%%%%%%%%%%%%%%%%%%%%
\begin{table}[htbp]
  \centering
  \footnotesize
  \renewcommand{\arraystretch}{1.50}
  \begin{tabular}{l|C{1.1cm}|C{2.1cm}|C{2.5cm}|C{2.5cm}}
       \multirow{2}{*}{ Dataset}   & \multirow{2}{*}{$ n_{\rm dat}$} & \multirow{2}{*}{ $\chi^2_{\rm SM}/n_{\rm dat}$} &  $\chi^2_{\rm EFT}/n_{\rm dat}$   & $\chi^2_{\rm EFT}/n_{\rm dat}$     \\
      &   &   & $\mathcal{O}\lp \Lambda^{-2}\rp$ &  $\mathcal{O}\lp \Lambda^{-4}\rp$  \\
       \toprule
%-----------------------------------------------------------------------------------------
 \href{https://arxiv.org/abs/1403.7366}{\tt CMS\_t\_tch\_8TeV\_inc} & 2 &  0.29    &  0.17    & 0.21     \\
 %-----------------------------------------------------------------------------------------
 \href{https://arxiv.org/abs/1702.02859}{\tt ATLAS\_t\_tch\_8TeV} & 4 &  0.89     & 0.71     & 0.66     \\
 %-----------------------------------------------------------------------------------------
 \href{https://cds.cern.ch/record/1956681}{\tt CMS\_t\_tch\_8TeV\_diff\_Yt} & 6 &  0.20    &  0.11    &  0.16    \\
 %-----------------------------------------------------------------------------------------
 \href{https://arxiv.org/abs/1603.02555}{\tt CMS\_t\_sch\_8TeV} & 1 &  1.26     &  0.94     &  1.16    \\
 %-----------------------------------------------------------------------------------------
 \href{https://arxiv.org/abs/1511.05980}{\tt ATLAS\_t\_sch\_8TeV} & 1 &  0.08     &  0.90    &  0.25    \\
 %-----------------------------------------------------------------------------------------
 \href{https://arxiv.org/abs/1609.03920}{\tt ATLAS\_t\_tch\_13TeV} & 2 &  0.01     &  0.06    &  0.02    \\
 %-----------------------------------------------------------------------------------------
 \href{https://arxiv.org/abs/1610.00678}{\tt CMS\_t\_tch\_13TeV\_inc} & 2 &  0.35    & 0.24     & 0.35     \\
 %-----------------------------------------------------------------------------------------
 \href{https://cds.cern.ch/record/2151074}{\tt CMS\_t\_tch\_13TeV\_diff\_Yt} & 4 &   0.52   & 0.47     &  0.47    \\
 %-----------------------------------------------------------------------------------------
 \href{https://arxiv.org/abs/1907.08330}{\tt CMS\_t\_tch\_13TeV\_2016\_diff\_Yt} & 5 &  0.60  & 0.59  & 0.59      \\
 %-----------------------------------------------------------------------------------------
\midrule
 %-----------------------------------------------------------------------------------------
 \href{https://arxiv.org/abs/1712.02825}{\tt ATLAS\_tZ\_13TeV\_inc} & 1 &  0.00     & 0.04     &  0.00    \\
 %-----------------------------------------------------------------------------------------
 \href{https://arxiv.org/abs/2002.07546}{\tt ATLAS\_tZ\_13TeV\_run2\_inc} & 1 &  0.05    &  0.07    &  0.01    \\
 %-----------------------------------------------------------------------------------------
 \href{https://arxiv.org/abs/1712.02825}{\tt CMS\_tZ\_13TeV\_inc} & 1 &  0.66    & 0.36     & 0.64     \\
 %-----------------------------------------------------------------------------------------
 \href{https://arxiv.org/abs/1812.05900}{\tt CMS\_tZ\_13TeV\_2016\_inc} & 1 &   1.23    &  0.33    &  1.16    \\
 %-----------------------------------------------------------------------------------------
 \href{https://arxiv.org/abs/1510.03752}{\tt ATLAS\_tW\_8TeV\_inc} & 1 &  0.02     &  0.01    &  0.05    \\
 %-----------------------------------------------------------------------------------------
 \href{https://arxiv.org/abs/2007.01554}{\tt ATLAS\_tW\_slep\_8TeV\_inc} & 1 &   0.13    &  0.15    & 0.11    \\
 %-----------------------------------------------------------------------------------------
 \href{https://arxiv.org/abs/1401.2942}{\tt CMS\_tW\_8TeV\_inc} & 1 &  0.00     & 0.00     & 0.00     \\
 %-----------------------------------------------------------------------------------------
 \href{https://arxiv.org/abs/1612.07231}{\tt ATLAS\_tW\_13TeV\_inc} & 1 &  0.52     &  0.55    &  0.47    \\
 %-----------------------------------------------------------------------------------------
 \href{https://arxiv.org/abs/1805.07399}{\tt CMS\_tW\_13TeV\_inc}  {{\bf (*)}} & 1 &  3.79     & 3.49     &  4.33    \\
 %-----------------------------------------------------------------------------------------
 \midrule
 %-----------------------------------------------------------------------------------------
 \href{https://arxiv.org/abs/1606.02266}{\tt ATLAS\_CMS\_SSinc\_RunI} & 22 &   0.86    & 0.86     & 0.89     \\
 %-----------------------------------------------------------------------------------------
 \href{https://arxiv.org/abs/1909.02845}{\tt ATLAS\_SSinc\_RunII} & 16 &   0.54   &  0.55    &  0.54    \\
 %-----------------------------------------------------------------------------------------
 \href{https://arxiv.org/abs/1809.10733}{\tt CMS\_SSinc\_RunII} & 24 &  0.77     &  0.70    &  0.68    \\
 %-----------------------------------------------------------------------------------------
 \href{https://arxiv.org/abs/1909.02845}{\tt ATLAS\_ggF\_ZZ\_13TeV} & 6 &  0.96     &   0.84   &  0.81    \\
 %-----------------------------------------------------------------------------------------
 \href{https://inspirehep.net/literature/1725274}{\tt CMS\_ggF\_aa\_13TeV} & 6 & 1.05   & 1.04   & 1.05     \\
 %-----------------------------------------------------------------------------------------
 \href{http://cdsweb.cern.ch/record/2682844/files/ATLAS-CONF-2019-032.pdf}{\tt ATLAS\_H\_13TeV\_2015\_pTH} & 9 & 1.11      & 1.10     & 1.08     \\ 
 %-----------------------------------------------------------------------------------------
 \href{https://arxiv.org/abs/1812.06504}{\tt CMS\_H\_13TeV\_2015\_pTH} & 9 &  0.80     &  0.78    & 0.78     \\
 %-----------------------------------------------------------------------------------------
 \href{https://arxiv.org/abs/1903.04618}{\tt ATLAS\_WH\_Hbb\_13TeV} & 2 &   0.10    &  0.07    &  0.15    \\
 %-----------------------------------------------------------------------------------------
 \href{https://arxiv.org/abs/1903.04618}{\tt ATLAS\_ZH\_Hbb\_13TeV} & 3 &   0.50    &  0.41    &  0.30    \\
 %-----------------------------------------------------------------------------------------
 \midrule
 %-----------------------------------------------------------------------------------------
 \href{https://arxiv.org/abs/1905.04242}{\tt ATLAS\_WW\_13TeV\_2016\_memu} & 13 & 1.64   &  1.64    &  1.67    \\
 %-----------------------------------------------------------------------------------------
 \href{https://arxiv.org/abs/1902.05759}{\tt ATLAS\_WZ\_13TeV\_2016\_mTWZ} & 6 &  0.81   &  0.81  & 0.80     \\
 %-----------------------------------------------------------------------------------------
 \href{https://arxiv.org/abs/1901.03428}{\tt CMS\_WZ\_13TeV\_2016\_pTZ} & 11 &  1.46   &  1.44    &  1.39    \\
 %-----------------------------------------------------------------------------------------
 \href{https://arxiv.org/abs/1302.3415}{\tt LEP\_eeWW\_182GeV} & 10 &  1.38  &  1.38   & 1.38     \\
 %-----------------------------------------------------------------------------------------
 \href{https://arxiv.org/abs/1302.3415}{\tt LEP\_eeWW\_189GeV} & 10 & 0.88    & 0.88   &  0.89    \\
 %-----------------------------------------------------------------------------------------
 \href{https://arxiv.org/abs/1302.3415}{\tt LEP\_eeWW\_198GeV} & 10 &  1.61   &  1.61    &  1.61    \\
 %-----------------------------------------------------------------------------------------
 \href{https://arxiv.org/abs/1302.3415}{\tt LEP\_eeWW\_206GeV} & 10 &  1.09   &  1.08   &  1.08    \\
 %----------------------------------------------------------------------------------------- 
\bottomrule
\end{tabular}
  \caption{\small Same as Table~\ref{eq:chi2-baseline} now for the single top
    datasets (inclusive and in association with gauge bosons), the Higgs production and decay measurements (signal streghts and differential
    distributions), and the LEP and LHC
    diboson cross-sections.
\label{eq:chi2-baseline2}
}
\end{table}
%%%%%%%%%%%%%%%%%%%%%%%%%%%%%%%%%%%%%%%%%%%%%%%%%%%%%%%%%%%%%%%%%%%%%%%%%%%%%%%%


%%%%%%%%%%%%%%%%%%%%%%%%%%%%%%%%%%%%%%%%%%%%%%%%%%%%%%%%%%%%%%%

%%%%%%%%%%%%%%%%%%%%%%%%%%%%%%%%%%%%%%%%%%%%%%%%%%%%%%%%%%%%%%
%%%%%%%%%%%%%%%%%%%%%%%%%%%%%%%%%%%%%%%%%%%%%%%%%%%%%%%%%%%%%%%%%%%%%%%%%%%%%%%%%%%%%%%
\begin{table}[htbp]
  \centering
  \small
   \renewcommand{\arraystretch}{1.70}
   \begin{tabular}{l|C{1.0cm}|C{1.8cm}|C{2.7cm}|C{2.7cm}}
        \multirow{2}{*}{ Dataset}   & \multirow{2}{*}{$ n_{\rm dat}$} & \multirow{2}{*}{ $\chi^2_{\rm SM}$} &  $\chi^2_{\rm EFT}$   & $\chi^2_{\rm EFT}$     \\
      &   &   & $\mathcal{O}\lp \Lambda^{-2}\rp$ &  $\mathcal{O}\lp \Lambda^{-4}\rp$  \\
        \toprule
%-----------------------------------------------------------------------------------------
 $t\bar{t}$ inclusive  & 83 &  1.46    &  1.32    &  1.42       \\
 $t\bar{t}$ charge asymmetry  & 11 &  0.60    &  0.39    &  0.59       \\
 $t\bar{t}+V$  & 14  &  0.65     &  0.48     &  0.65       \\
 single-top inclusive  &  27  &   0.43    &  0.44     &  0.41       \\
 single-top $+V$ & 9  &  0.71     &  0.55     &  0.75       \\
 $t\bar{t}b\bar{b}$ \& $t\bar{t}t\bar{t}$   & 6   &  1.68     &  1.09     &  2.12       \\
 Higgs signal strenghts (Run I)  &  22  &   0.86     &  0.85     &  0.90       \\
 Higgs signal strenghts (Run II)  &  40 &   0.67    &  0.64     &  0.63       \\
 Higgs differential \& STXS  &  35  & 0.88     &  0.85     &   0.83      \\
 Diboson (LEP+LHC)  & 70  &  1.31     &  1.31     &  1.30       \\
 \midrule
 {\bf Total}  & {\bf 317}  &  {\bf 1.05 }   & {\bf 0.98 }  & {\bf 1.04 } \\
%-------------------------------------------------------------------------------
\bottomrule
\end{tabular}
\caption{\small Summary of the $\chi^2$ results listed in Tables~\ref{eq:chi2-baseline}
  and~\ref{eq:chi2-baseline2}.
  %
  We indicate the total values for each group of processes
  as well as for the global dataset.
\label{eq:chi2-baseline-grouped}
}
\end{table}
%%%%%%%%%%%%%%%%%%%%%%%%%%%%%%%%%%%%%%%%%%%%%%%%%%%%%%%%%%%%%%%%%%%%%%%%%%%%%%%%


%%%%%%%%%%%%%%%%%%%%%%%%%%%%%%%%%%%%%%%%%%%%%%%%%%%%%%%%%%%%%%%

%%%%%%%%%%%%%%%%%%%%%%%%%%%%%%%%%%%%%%%%%%%%%%%%%%%%%%%%%%%%%%%%%%%%%
\begin{figure}[t]
  \begin{center}
    \includegraphics[width=0.99\linewidth]{plots_v2/Chi2_Bar_Baseline.pdf}
    \caption{\small Graphical representation of the results of Tables~\ref{eq:chi2-baseline}
      and~\ref{eq:chi2-baseline2},
      displaying the values of the $\chi^2$ per data point, Eq.~(\ref{eq:chi2definition3})
      for the all datasets used as input in the fit.
      %
     The $\chi^2$ values are shown  for the SM and for the two global SMEFT
      baseline fits, based on theory calculations at either the linear $\mathcal{O}\lp \Lambda^{-2}\rp$ or
       quadratic $\mathcal{O}\lp \Lambda^{-4}\rp$ order in the EFT expansion.
     \label{fig:chi2_barplot} }
  \end{center}
\end{figure}
%%%%%%%%%%%%%%%%%%%%%%%%%%%%%%%%%%%%%%%%%%%%%%%%%%%%%%%%%%%%%%%%%%%%%%

Let us discuss first the $\chi^2$ results evaluated in terms of the groups
of processes, listed in Table~\ref{eq:chi2-baseline-grouped}.
%
One can observe that the global $\chi^2$ per data point decreases from 1.05 when using
SM theory to 0.98 (linear) and 1.04 (quadratic) once SMEFT corrections
are accounted for.
%
Considering the fit quality to the various groups of processes, we find
that the description of the inclusive top-quark pair
cross-sections (without the $A_C$ data),
composed by $n_{\rm dat}=83$ data points, is improved once EFT corrections
are accounted for with $\chi^2=1.46$ in the SM decreasing
to 1.32 and 1.42 in the linear and quadratic EFT fits
respectively.
%
For the rest of the datasets, and in particular for the Higgs and diboson
measurements, the overall EFT fit quality is similar to that
obtained using SM calculations.

Inspection of the $\chi^2$ values associated to individual datasets
reported in Tables~\ref{eq:chi2-baseline}
and~\ref{eq:chi2-baseline2}, as well
as their graphical representation from Fig.~\ref{fig:chi2_barplot},
reveals that in some cases the agreement
between the prior SM theoretical calculations and the data is poor.
%
This is the case, in particular, for some of the inclusive $t\bar{t}$ datasets such as
{\tt ATLAS\_tt\_8TeV\_ljets\_mtt} and
{\tt CMS\_tt\_13TeV\_dilep\_2016\_mtt}, binned in terms of the top-quark
pair invariant mass distribution $m_{t\bar{t}}$, with $\chi^2=2.95$ and 2.28
for $n_{\rm dat}=7$ points each.
%
Such relatively high values of the $\chi^2$ do not necessarily imply the need for
some New Physics effects,
but could also be explained by issues with the modelling of the experimental
systematic correlations in differential distributions, as discussed for the ATLAS 8 TeV
lepton+jets data in~\cite{Czakon:2016olj,Amoroso:2020lgh,ATL-PHYS-PUB-2018-017}.
%
Nevertheless, when all the inclusive $t\bar{t}$ datasets are considered collectively,
a value of $\chi^2_{\rm SM} = 1.46$ for  the $n_{\rm dat}=83$ data points in the fit is obtained.
%
In Sect.~\ref{sec:dataset_dependence} we will assess  the stability of the global
fit results by presenting fit variants with the individual datasets
that lead to a poor $\chi^2$ removed.
%
We will also study variants where distributions sensitive
to the high energy behaviour of the EFT, such as $m_{t\bar{t}}$ in top quark pair
production, have their bins with $m_{t\bar{t}}\gsim 1$ TeV removed from the fit.

Beyond the inclusive $t\bar{t}$ datasets, there are some other instances of a sub-optimal agreement
between SM theory and data.
%
In all cases, there exist comparable measurements of the same process, either
from the same experiment at a different center-of-mass energy
$\sqrt{s}$ or from a different experiment
at the same value of $\sqrt{s}$, for which the $\chi^2$ reveals good consistency with the SM.
%
These include {\tt CMS\_ttbb\_13TeV}, when comparing with the same measurement based
on the full 2016 luminosity, and
{\tt CMS\_tW\_13TeV\_inc}, where again the ATLAS measurement at the same $\sqrt{s}$
exhibit as good $\chi^2$.
%
All in all, one finds a reasonable description of the global input dataset
when using SM cross-sections which is further improved in the EFT fit.

For the rest of this section, when presenting the results of fits corresponding
to variations of the baseline settings, such as fits based on
reduced datasets, we will only indicate the $\chi^2$ values
associated to groups of processes using the same format
as Table~\ref{eq:chi2-baseline-grouped}, rather than to individual
datasets, and comment when required on the results for the latter.

