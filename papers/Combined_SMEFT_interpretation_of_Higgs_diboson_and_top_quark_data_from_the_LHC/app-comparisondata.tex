\section{Comparison with experimental data}
\label{sect:app_comparison_data}

In this appendix, we present a systematic comparison between the
experimental data used as input to the fit with  the corresponding theoretical cross-sections
based both on the SM and on the best-fit SMEFT results, either
 at  linear $\mathcal{O}\lp \Lambda^{-2}\rp$ or at quadratic $\mathcal{O}\lp \Lambda^{-4}\rp$ accuracy.
%
In these comparisons, the experimental measurements will be presented both
in terms of unshifted central values, where the error band represents the total uncertainty, and once
the best-fit systematic shifts have been subtracted, such that the error band contains only the statistical component.
%
Note that the evaluation of the shifted data is only possible whenever the full breakup of the experimental
systematic uncertainties is made available by in {\tt HepData}.
%
If this is not the case, for example when only the full experimental correlation matrix is provided
or no information on correlations is released, we will display only the unshifted data.

%%%%%%%%%%%%%%%%%%%%%%%%%%%%%%%%%%%%%%%%%%%%%%%%%%%%%%%%%%%%%%%%%%%%%
\begin{figure}[htbp]
  \begin{center}
    \includegraphics[width=0.325\linewidth]{plots_v2/DvT01_ATLAS_tt_8TeV_ljets_Mtt.pdf}
    \includegraphics[width=0.325\linewidth]{plots_v2/DvT02_CMS_tt_8TeV_ljets_Ytt.pdf}
    \includegraphics[width=0.325\linewidth]{plots_v2/DvT03_CMS_tt2D_8TeV_dilep_MttYtt.pdf}
    \includegraphics[width=0.325\linewidth]{plots_v2/DvT04_CMS_tt_13TeV_ljets_2015_Mtt.pdf}
    \includegraphics[width=0.325\linewidth]{plots_v2/DvT05_CMS_tt_13TeV_dilep_2015_Mtt.pdf}
    \includegraphics[width=0.325\linewidth]{plots_v2/DvT06_CMS_tt_13TeV_ljets_2016_Mtt.pdf}
    \includegraphics[width=0.325\linewidth]{plots_v2/DvT07_CMS_tt_13TeV_dilep_2016_Mtt.pdf}
    \includegraphics[width=0.325\linewidth]{plots_v2/DvT08_ATLAS_tt_13TeV_ljets_2016_Mtt.pdf}
    \includegraphics[width=0.325\linewidth]{plots_v2/DvT09_ATLAS_t_tch_8TeV.pdf}
    \includegraphics[width=0.325\linewidth]{plots_v2/DvT10_CMS_t_tch_8TeV_diff_Yt.pdf}
    \includegraphics[width=0.325\linewidth]{plots_v2/DvT11_CMS_t_tch_13TeV_diff_Yt.pdf}
    \includegraphics[width=0.325\linewidth]{plots_v2/DvT12_SingleTop.pdf}
    \includegraphics[width=0.325\linewidth]{plots_v2/DvT30_ATLAS_CMS_tt_AC_8TeV.pdf}
    \includegraphics[width=0.325\linewidth]{plots_v2/DvT31_ATLAS_tt_AC_13TeV.pdf}
    \caption{\small Comparison between experimental data and best-fit EFT theory
      predictions (both for the linear and the quadratic fits)
      for representative differential top quark pair
      and single top quark production datasets.
      %
      Both the data and the EFT fit results are normalised to the central value of the SM cross-section.
      %
      The data is presented both with unshifted central values
      (where the band represents the total experimental error) and once
      the best-fit systematic shifts have been subtracted (so that the error band
      contains only the statistical component).
      %
      The error band in the EFT prediction indicates the 95\% CL interval evaluated
      over the $N_{\rm spl}$ samples produced by the NS method.
     \label{fig:data_vs_theory_top_diff} }
  \end{center}
\end{figure}
%%%%%%%%%%%%%%%%%%%%%%%%%%%%%%%%%%%%%%%%%%%%%%%%%%%%%%%%%%%%%%%%%%%%%%

To begin with, Fig.~\ref{fig:data_vs_theory_top_diff} displays
the comparison between experimental data and the best-fit EFT theory
predictions (for linear and quadratic fits)
in the case of representative differential top quark pair
and single top quark production datasets.
%
Both the data and the EFT fit results are normalised to the central value of the SM theory prediction.
%
This implies that the more the fit results deviate from unity, the larger the best-fit EFT effects
are for this specific observable.
%
Furthermore, the error band in the EFT prediction indicates the associated 95\% CL interval evaluated
over the $N_{\rm spl}$ samples produced by the NS method, {  that is, the 95\% interval
of the corresponding marginalised posterior distributions shown in Fig.~\ref{fig:posterior_coeffs}.}

From these comparisons, one can observe how for some datasets the best-fit EFT results
move in the direction of the experimental data, for instance in the case of the $m_{t\bar{t}}$
distributions at large invariant masses for inclusive $t\bar{t}$ production.
%
This is an important kinematic region in the fit, since energy-growing effects increase the EFT
sensitivity.
%
Interestingly, in the highest $m_{t\bar{t}}$ bins for some of the 13 TeV top datasets
the 95\% CL interval associated to the EFT prediction does not include the SM expectation.
%
In the case of the single-top $t$-channel differential cross-sections, the EFT fit results
are very close to the SM predictions, indicating that EFT effects are well constrained
for this process at the scale of the present experimental uncertainties.
%
We also note that the uncertainty band associated to the EFT prediction
can turn out to be rather different in the $\mathcal{O}\lp \Lambda^{-2}\rp$ fits
as compared to the $\mathcal{O}\lp \Lambda^{-4}\rp$ ones, with the latter in general
being more precise than the former for the processes considered here.
%
{  We note that the CMS $t\bar{t}$ double differential distributions at 8 TeV
  (upper right plot in Fig.~\ref{fig:data_vs_theory_top_diff}) are provided in bins
  of both $m_{t\bar{t}}$ and $y_{t\bar{t}}$, and hence there is more than one data point
for each $m_{t\bar{t}}$ bin.}

%%%%%%%%%%%%%%%%%%%%%%%%%%%%%%%%%%%%%%%%%%%%%%%%%%%%%%%%%%%%%%%%%%%%%
\begin{figure}[t]
  \begin{center}
    \includegraphics[width=0.325\linewidth]{plots_v2/DvT14_Whelicity.pdf}
  \includegraphics[width=0.325\linewidth]{plots_v2/DvT12_SingleTop.pdf}
  \includegraphics[width=0.325\linewidth]{plots_v2/DvT15_Four_Heavy.pdf}
  \includegraphics[width=0.325\linewidth]{plots_v2/DvT23_ATLAS_WW_13TeV_2016_memu.pdf}
  \includegraphics[width=0.325\linewidth]{plots_v2/DvT24_ATLAS_WZ_13TeV_2016_mTWZ.pdf}
  \includegraphics[width=0.325\linewidth]{plots_v2/DvT25_CMS_WZ_13TeV_2016_pTZ.pdf}
  \includegraphics[width=0.325\linewidth]{plots_v2/DvT26_LEP_eeWW_182GeV.pdf}
  \includegraphics[width=0.325\linewidth]{plots_v2/DvT28_LEP_eeWW_198GeV.pdf}
  \includegraphics[width=0.325\linewidth]{plots_v2/DvT29_LEP_eeWW_206GeV.pdf}
  \caption{\small Same as Fig.~\ref{fig:data_vs_theory_top_diff} now for the $W$ helicity fractions,
    the single-top $s$-channel and $tV$ total cross-sections, the four-heavy quark fiducial
    measurements, the LHC diboson differential distributions at 13 TeV, and the LEP diboson cross-sections
    at different center of mass energies.
 \label{fig:data_vs_theory_v2} }
 \end{center}
\end{figure}
%%%%%%%%%%%%%%%%%%%%%%%%%%%%%%%%%%%%%%%%%%%%%%%%%%%%%%%%%%%%%%%%%%%%%%

Then Fig.~\ref{fig:data_vs_theory_v2} displays the same comparison between data and the SM and EFT
predictions as in Fig.~\ref{fig:data_vs_theory_top_diff} now for the $W$ helicity fractions,
the single-top $s$-channel and $tV$ total cross-sections, the four-heavy-quark fiducial
measurements, the LHC diboson differential distributions at 13 TeV, and the LEP diboson cross-sections
at different center-of-mass energies.
%
Note that contrary to the rest of the datasets, the comparison for the $W$ helicity fraction is carried
out at the absolute rather than at the normalised level.
%
Concerning the single top measurements, the best-fit EFT predictions tend
to move towards the experimental data, which in most cases is somewhat
higher than the SM prediction.
%
For some datasets, such as single-top $s$-channel cross-section at 8 TeV and the $tW$ cross-sections
at 13 TeV, the agreement between ATLAS and CMS is at best marginal and thus
the EFT fit interpolates between the two measurements.
%
A similar behaviour is observed for the $t\bar{t}t\bar{t}$ cross-sections at 13 TeV.
%
Furthermore, as was the case for the processes
considered in Fig.~\ref{fig:data_vs_theory_top_diff}, the EFT fit uncertainties
appear to be reduced in the quadratic case.

%%%%%%%%%%%%%%%%%%%%%%%%%%%%%%%%%%%%%%%%%%%%%%%%%%%%%%%%%%%%%%%%%%%%%
\begin{figure}[t]
  \begin{center}
    \includegraphics[width=0.49\linewidth]{plots_v2/DvT19_ATLAS_H_13TeV_2015_pTH.pdf}
    \includegraphics[width=0.49\linewidth]{plots_v2/DvT20_CMS_H_13TeV_2015_pTH.pdf}
  \includegraphics[width=0.49\linewidth]{plots_v2/DvT21_ATLAS_ggF_ZZ_13TeV.pdf}
  \includegraphics[width=0.49\linewidth]{plots_v2/DvT22_CMS_ggF_aa_13TeV.pdf}
  \vspace{-0.4cm}
  \caption{\small Same as Fig.~\ref{fig:data_vs_theory_top_diff} for
   representative Higgs  measurements from ATLAS and CMS at $\sqrt{s}=13$ TeV,
    namely the $p_{T,H}$ distributions summing over all production modes and final states (upper panels),
    and the Simplified Template Cross-Section measurements (bottom panels)
    corresponding
    to the $ZZ$ (left) and the $\gamma\gamma$ final states (right panel).
 \label{fig:data_vs_theory_v4} }
 \end{center}
\end{figure}
%%%%%%%%%%%%%%%%%%%%%%%%%%%%%%%%%%%%%%%%%%%%%%%%%%%%%%%%%%%%%%%%%%%%%%

%%%%%%%%%%%%%%%%%%%%%%%%%%%%%%%%%%%%%%%%%%%%%%%%%%%%%%%%%%%%%%%%%%%%%
\begin{figure}[htbp]
  \begin{center}
  \includegraphics[width=0.86\linewidth]{plots_v2/DvT16_ATLASCMS_RunI.pdf}
  \includegraphics[width=0.86\linewidth]{plots_v2/DvT17_ATLAS_RunII.pdf}
  \includegraphics[width=0.86\linewidth]{plots_v2/DvT18_CMS_RunII.pdf}
  \caption{\small Same as Fig.~\ref{fig:data_vs_theory_top_diff} for the Higgs boson
    signal strengths corresponding to different production mechanisms and decay channels.
    %
    From top to bottom we show the ATLAS+CMS Run I combination and the ATLAS and CMS Run II measurements
    at 13 TeV.
    %
    Note that by the definition of the signal strengths the SM predictions correspond to $\mu_i^{(f)}=1$
    in all cases, see also App.~\ref{sec:signalstrenghts}.
 \label{fig:data_vs_theory_v3} }
 \end{center}
\end{figure}
%%%%%%%%%%%%%%%%%%%%%%%%%%%%%%%%%%%%%%%%%%%%%%%%%%%%%%%%%%%%%%%%%%%%%%

%%%%%%%%%%%%%%%%%%%%%%%%%%%%%%%%%%%%%%%%%%%%%%%%%%%%%%%%%%%%%%%%%%%%%
\begin{figure}[htbp]
  \begin{center}
  \includegraphics[width=0.49\linewidth]{plots_v2/DvT32_ATLAS_WH_Hbb_13TeV.pdf}
  \includegraphics[width=0.49\linewidth]{plots_v2/DvT33_ATLAS_ZH_Hbb_13TeV.pdf}
  \caption{\small Same as Fig.~\ref{fig:data_vs_theory_top_diff} for the Higgs boson
    associated production STXS from the ATLAS $VH$ measurement at 13 TeV.
    %
 \label{fig:data_vs_theory_v5} }
 \end{center}
\end{figure}
%%%%%%%%%%%%%%%%%%%%%%%%%%%%%%%%%%%%%%%%%%%%%%%%%%%%%%%%%%%%%%%%%%%%%%

Moving to the LEP and LHC diboson datasets, one finds that for electron-positron collisions
the EFT fit result is very close to the SM cross-section with a vanishing uncertainty.
%
This result is likely to be related to the constraints imposed by the EWPOs as well as by the
LHC diboson data.
%
Nevertheless, the SM predictions are in good agreement with the LEP data for all
four center-of-mass energies considered to begin with.
%
In the case of the LHC measurements, for the ATLAS $m_{e\mu}$ and $m_T^{WZ}$
distributions in the $W^+W^-$ and $WZ$ final states, respectively, the data is in good
agreement with the SM and the net effect of the EFT corrections is small, except
perhaps for the highest-energy bin of the $m_{e\mu}$ and $m_T^{WZ}$ distribution.
%
Similar considerations apply for the CMS 13 TeV $WZ$ dataset, where we observe good
agreement between theory and data also for the high $p_{T}^Z$ region.

Concerning the comparison between experimental data and theory calculations for the Higgs production
and decay measurements,  Fig.~\ref{fig:data_vs_theory_v4} displays
representative Higgs  measurements from ATLAS and CMS at $\sqrt{s}=13$ TeV,
namely the $p_{T,H}$ distributions inclusive over all production modes and final states,
and the Simplified Template Cross-Section measurements 
corresponding
to the $ZZ$  and the $\gamma\gamma$ final states for ATLAS and CMS
respectively.
%
Then Fig.~\ref{fig:data_vs_theory_v3} summarizes
the results corresponding to Higgs boson signal strengths for
different production mechanisms and decay channels.
%
Specifically, we show  the ATLAS+CMS Run I combination and the ATLAS and CMS Run II measurements at 13 TeV.
%
Note that, by construction, in the signal strengths the SM predictions correspond to $\mu_i^{(f)}=1$,
see the discussion of App.~\ref{sec:signalstrenghts} for more details.

In the case of the differential Higgs distributions, we can observe the good agreement
both the SM and the EFT predictions within the relatively large
experimental uncertainties.
%
For these distributions, the EFT effects can reach a magnitude of up to a few percent
in the global fit.
%
For instance, for the CMS $p_{T,H}$ distribution in the quadratic fit,
the best-fit results are $\simeq 15\%$ higher than the SM for the $p_{T,H}=1$ TeV bin.
%
For the case of the signal strengths, also a fair agreement is found, though for some combinations
of production channel and decay mode the experimental uncertainties still remain
rather large.

%%%%%%%%%%%%%%%%%%%%%%%%%%%%%%%%%%%%%%%%%%%%%%%%%%%%%%%%%%%%%%%%%%%%%%%%
%%%%%%%%%%%%%%%%%%%%%%%%%%%%%%%%%%%%%%%%%%%%%%%%%%%%%%%%%%%%%%%%%%%%%%%%5
