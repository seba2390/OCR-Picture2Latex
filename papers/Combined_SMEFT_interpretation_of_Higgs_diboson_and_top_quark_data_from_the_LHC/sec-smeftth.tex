\section{EFT description of the top, Higgs, and electroweak sectors}
\label{sec:smefttheory}

In this section we collect the definitions and conventions that will
be used to construct the dimension-six operators and the associated degrees of freedom (DoFs)
relevant for the theoretical description of the processes considered
in this analysis.
%
These are operators that modify the production and decay of Higgs bosons
and top quarks at hadron colliders, precision electroweak measurements from LEP/SLC,
and gauge-boson pair production cross-sections both at LEP2 and at the LHC.

First of all,
we provide explicit definitions for the operators
and for the physical EFT coefficients adopted in this work,
as well as the corresponding notational conventions.
%
Following the recommendation of the LHC Top Quark Working Group
\cite{AguilarSaavedra:2018nen} as well as the strategy of our previous work
\cite{Hartland:2019bjb}, in the top-quark sector we fit specific degrees
of freedom closely related to the experimental measurements, instead of directly using the
Warsaw-basis operator coefficients.
%
Our degrees of freedom are therefore linear combinations of the
Warsaw-basis operator coefficients, which appear in the interference with SM
amplitudes, and represent interactions of physical fields after 
electroweak symmetry breaking.
%
These combinations are then aligned with
physically relevant directions of the parameter space, and thus have
a more transparent physical interpretation. They also represent the
maximal information that can be extracted from measuring a certain process.

We will then discuss how the constraints provided by the electroweak precision
observables (EWPOs) from LEP/SLC can be approximately accounted for by means of a series of
restrictions on the EFT parameter space.
%
We also discuss theoretical constraints on the operator coefficients following
a more restrictive assumption about the UV-complete theory, namely the
so-called top-philic scenario.
%
Finally, we discuss several theoretical relations that must be satisfied
by the EFT cross-sections following the requirement that physical cross-sections are positive-definite
quantities.


\subsection{Operator basis and degrees of freedom}
\label{sec:operatorbasis}

\paragraph{Conventions.}
%
Let us start by summarizing the notation and conventions that are adopted
in this work concerning the relevant dimension-six SMEFT operators.
%
Here we follow the notation of the Warsaw basis presented in~\cite{Grzadkowski:2010es}.
%
In this notation, flavour indices are labelled by $i,j,k$ and $l$; left-handed
quark and lepton fermion SU(2)$_L$ doublets are denoted by $q_i$, $\ell_i$;
the right-handed quark singlets by $u_i$, $d_i$, while
the right-handed lepton singlets are denoted by $e$, $\mu$, $\tau$ without using
flavor index. Given the special role of the top-quark in this work, we 
use $Q$ and $t$ to denote the left-handed top-bottom doublet and the right-handed
top singlet, instead of using $q_3$ and $u_3$.
%
The Higgs doublet is denoted by $\varphi$;
the antisymmetric SU(2) tensor by $\varepsilon\equiv i\tau^2$;
$\tilde{\varphi}=\varepsilon\varphi^*$; and we define
\be
\FDF\equiv \varphi^\dagger(iD_\mu \varphi) - (iD_\mu\varphi^\dagger) \varphi \,, \qquad 
\FDFI\equiv \varphi^\dagger\tau^I(iD_\mu \varphi) - (iD_\mu\varphi^\dagger) \tau^I\varphi \, ,
\ee where $\tau^I$ are the Pauli matrices.
%
In the following,
$G^A_{\mu\nu}$, $W^I_{\mu\nu}$, and $B_{\mu\nu}$ stand for the SU(3) strong and SU(2)$_L$ and U(1)$_Y$ weak gauge
field strengths respectively, and the covariant derivatives include all the relevant
interaction terms.
%
For instance, the gluon field strength tensor is given by
\be
G_{\mu\nu}^A = \partial_{\mu} G_{\nu}^A - \partial_{\nu} G_{\mu}^A + g_s f^{ABC}G_{\mu}^B G_{\nu}^C \, ,
\ee
where $G_{\mu}^A$ is the gluon field, $A, B, C$ are color indices in the adjoint
representation, $g_s$ is the strong coupling and $f^{ABC}$ are the structure
constants of SU(3).
%
Similar definitions hold for the electroweak $W_I^{\mu\nu}$ and $B^{\mu\nu}$
field strength tensors, for instance one has
\be
W_{\mu\nu}^I = \partial_{\mu} W_{\nu}^I - \partial_{\nu}W_{\mu}^I + g_w \epsilon_{JK}^I W_{\mu}^J W_{\mu}^K \, ,
\ee
where $g_w$ is the SU(2)$_L$ coupling constant.

\paragraph{Flavour assumptions.}
%
The number of independent dimension-six operators can be unfeasibly large, if all three
generations of the SM fermions are taken into account: there are 2499 in total
\cite{Alonso:2013hga}, with 572 four-fermion operators that are in principle relevant for top-quark
physics~\cite{AguilarSaavedra:2010zi}.
%
In this analysis, we follow closely the strategy which we adopted in our previous top-quark
sector study~\cite{Hartland:2019bjb} and that has been documented in the LHC Top
Quark Working Group note~\cite{AguilarSaavedra:2018nen}: we implement the
Minimal Flavour Violation (MFV) hypothesis~\cite{DAmbrosio:2002vsn} in the
quark sector as the baseline scenario.
%
A slight difference is that instead of a $U(2)_q\times U(2)_u \times U(2)_d$
flavour symmetry among the first two generations,
we now impose the $U(2)_q\times U(2)_u \times U(3)_d$ symmetry,
under the assumption that the Yukawa couplings are nonzero only for the top quark.
%
This flavour assumption is consistent with the {\tt SMEFT@NLO} model~\cite{Degrande:2020evl},
the implementation of automated one-loop calculation in the SMEFT which we will use to the provide
theoretical inputs for our global fit, as discussed in the next section.

As a result of the different flavour assumption,
the EFT parameter space is further
reduced compared to~\cite{Hartland:2019bjb}.
%
In particular, the coefficients of operators with right-handed bottom quarks
are either set to zero or set equal to the corresponding down-quark ones.
%
Furthermore, we then slightly relax our assumptions by keeping the bottom and
charm quark Yukawa operators in our fit, to account for the current LHC
sensitivity to these parameters.
%
All other light quark Yukawa operators are set to zero,
since we do not expect to have any sensitivity on their coefficients.

Concerning the leptonic sector, the adopted flavour symmetry is $(U(1)_\ell \times
U(1)_e)^3$, also following~\cite{AguilarSaavedra:2018nen}.
%
This assumption sets all the lepton masses as well as their Yukawa couplings to
zero in the SM, while leaving independent parameters for each lepton-antilepton
pair of a given generation.
This is then relaxed by including the $\tau$ Yukawa operator, to account 
for the expected LHC sensitivity arising from dedicated measurements.
%
In practice, the lepton flavor assumptions do not have implications for the EFT fit
given the constraints from $Z$-pole measurements at LEP and SLC, see the discussion below.

\paragraph{Purely bosonic operators.}
%
Table~\ref{tab:oper_bos} reports the purely bosonic dimension-six operators that
modify the production and decay of Higgs bosons as well as the interactions
of the electroweak gauge bosons.
%
For each operator, we indicate its definition in terms of the SM fields
and the notation that we will use both
for the operators and for the Wilson coefficients.
%
These operators modify several important Higgs boson production and decay processes
that are (or will become) accessible at the LHC, as well as the production
of gauge boson pairs both in electron-positron and in proton-proton collisions.

%%%%%%%%%%%%%%%%%%%%%%%%%%%%%%%%%%%%%%
%%%%%%%%%%%%%%%%%%%%%%%%%%%%%%%%%%%%%%%%%%%%%%%%%%%%%%%5
\begin{table}[t] 
  \begin{center}
    \renewcommand{\arraystretch}{1.90}
        \begin{tabular}{lll}
          \toprule
          Operator $\qquad$ & Coefficient $\qquad\qquad\qquad$ & Definition \\
        \midrule
        $\Op{\varphi G}$ & $c_{\varphi G}$  & $\left(\pdp\right)G^{\mu\nu}_{\sss A}\,
        G_{\mu\nu}^{\sss A}$  \\ \hline
        %%%%%%%%%%%%%%%%%%%%%%%%%%%%%%%%%%%%%%%%%%%%%%%%%%%%%%%%%%%%%%%
        $\Op{\varphi B}$ & $c_{\varphi B}$ & $\left(\pdp\right)B^{\mu\nu}\,B_{\mu\nu}$\\ \hline
        %%%%%%%%%%%%%%%%%%%%%%%%%%%%%%%%%%%%%%%%%%%%%%%%%%%%%%%%%%%%%%%
        $\Op{\varphi W}$ &$c_{\varphi W}$ & $\left(\pdp\right)W^{\mu\nu}_{\sss I}\,
        W_{\mu\nu}^{\sss I}$ \\ \hline
        %%%%%%%%%%%%%%%%%%%%%%%%%%%%%%%%%%%%%%%%%%%%%%%%%%%%%%%%%%%%%%%
        $\Op{\varphi W B}$ &$c_{\varphi W B}$ & $(\varphi^\dagger \tau_{\sss I}\varphi)\,B^{\mu\nu}W_{\mu\nu}^{\sss I}\,$ \\ \hline
        %%%%%%%%%%%%%%%%%%%%%%%%%%%%%%%%%%%%%%%%%%%%%%%%%%%%%%%%%%%%%%%
        $\Op{\varphi d}$ & $c_{\varphi d}$ & $\partial_\mu(\pdp)\partial^\mu(\pdp)$ \\ \hline
        %%%%%%%%%%%%%%%%%%%%%%%%%%%%%%%%%%%%%%%%%%%%%%%%%%%%%%%%%%%%%%%
        $\Op{\varphi D}$ & $c_{\varphi D}$ & $(\varphi^\dagger D^\mu\varphi)^\dagger(\varphi^\dagger D_\mu\varphi)$ \\ \hline
         $\mathcal{O}_{W}$&   $c_{WWW}$ & $\epsilon_{IJK}W_{\mu\nu}^I W^{J,\nu\rho} W^{K,\mu}_\rho$ \\
       \bottomrule
        \end{tabular}
        \caption{Purely bosonic dimension-six operators that
          modify the production and decay of Higgs bosons and
          the interactions of the electroweak gauge bosons.
          %
          For each operator, we indicate its definition in terms of the SM
          fields,
          and the notational conventions that will be used
          both for the operator and for the Wilson coefficient.
	  %
          The operators $O_{\varphi WB}$ and $O_{\varphi D}$
          are severely
          constrained by the EWPOs together with several of
          the two-fermion operators from Table~\ref{tab:oper_ferm_bos}.
           \label{tab:oper_bos}}
\end{center}
\end{table}
%%%%%%%%%%%%%%%%%%%%%%%%%%%%%%%%%%%%%%%%%%%%%%%%%%%%%%%%%%

%%%%%%%%%%%%%%%%%%%%%%%%%%%%%%%%%%%%%%

One can comment on some interesting features of the operators defined
in Table~\ref{tab:oper_bos}.
%
To begin with, the operators $O_{\varphi WB}$ and $O_{\varphi D}$ are the ones
often identified as the $S$ and $T$ oblique parameters, though this
identification is basis-dependent and is not strictly correct in the Warsaw
basis.
%
Together with several of the two-fermion operators listed in Table~\ref{tab:oper_ferm_bos}, they are severely
constrained by the $Z$-pole and $W$-pole measurements available from LEP and
SLC, but with 2 linear combinations left unconstrained. These two combinations
in turn modify the electroweak triple gauge boson (TGC) couplings and the Higgs-electroweak
interactions.  They are thus constrained mainly by the diboson measurements
at the LEP2 and the LHC, as well as the Higgs measurements at the LHC. We will
discuss this property in more detail in the following section.
%
The operator $O_{W}$ generates a TGC coupling modification which is purely transversal
and is hence constrained only by diboson data.

The rest of the bosonic operators listed in Table~\ref{tab:oper_bos} modify only the Higgs boson couplings, and represent 
degrees of freedom that are accessible only with Higgs data.
First, the operators $\mathcal{O}_{\varphi W}$ and $\mathcal{O}_{\varphi B}$
modify the interaction between Higgs bosons and electroweak gauge bosons. 
%
At the LHC, they can be probed for example by means of the Higgs decays into
weak vector bosons, $h\to ZZ^*$ and $h \to W^+W^-$, as well as in the
vector-boson-fusion (VBF) process and in associated production with vector bosons, $hW$
and $hZ$.
%
In addition, the $\mathcal{O}_{\varphi G}$ operator is similar but 
introduces a direct coupling between the Higgs boson and gluons.
It therefore enters the Higgs total width and branching ratios,
the production cross section in gluon fusion channel, 
as well as the associated production channel $t\bar{t}h$.
Finally, the $O_{\varphi d}$ operator generates a wavefunction correction to the
Higgs boson, which rescales all the Higgs boson couplings in a universal manner.

{  In principle, one could also include in
  Table~\ref{tab:oper_bos} the triple-gluon
  operator $\mathcal{O}_{G}$, which contributes to $t\bar{t}(V)$ and Higgs+jets production.
  %
  However this operator is already tightly constrained by
  multi-jet production measurements at the LHC~\cite{Hirschi:2018etq},
  as discussed also in~\cite{Hartland:2019bjb}.
  %
  It is found that the bounds on the coefficient $c_G$ obtained
  from multijet data are very stringent and
  beyond the sensitivity achievable via either top quark or Higgs production measurements
  at the LHC.
  %
  For this reason, $\mathcal{O}_{G}$ is not considered in the present analysis.
}
  
\paragraph{Two-fermion operators.}
%       
%%%%%%%%%%%%%%%%%%%%%%%%%%%%%%%%%%%%%%%%
%%%%%%%%%%%%%%%%%%%%%%%%%%%%%%%%%%%%%%%%%%%%%%%%%%%%%%%%%%
\begin{table}[htbp]
  \begin{center}
    \renewcommand{\arraystretch}{1.45}
    \begin{tabular}{lll}
      \toprule
      Operator $\qquad$ & Coefficient & Definition \\
                \midrule \midrule
		&3rd generation quarks&\\
                \midrule \midrule
%%%%%%%%%%%%%%%%%%%%%%%%%%%%%%%%%%%%%%%%%%%%%%%%%%%%%%%%%%%%%%%%%%%%%%%%%%%
    $\Op{\varphi Q}^{(1)}$ & $c_{\varphi Q}^{(1)}$~(*) & $i\big(\varphi^\dagger\lra{D}_\mu\,\varphi\big)
 \big(\bar{Q}\,\gamma^\mu\,Q\big)$ \\\hline
%%%%%%%%%%%%%%%%%%%%%%%%%%%%%%%%%%%%%%%%%%%%%%%%%%%%%%%%%%%%%%%%%%%%%%%%%%%
    $\Op{\varphi Q}^{(3)}$ & $c_{\varphi Q}^{(3)}$  & $i\big(\varphi^\dagger\lra{D}_\mu\,\tau_{\sss I}\varphi\big)
 \big(\bar{Q}\,\gamma^\mu\,\tau^{\sss I}Q\big)$ \\ \hline
 %%%%%%%%%%%%%%%%%%%%%%%%%%%%%%%%%%%%%%%%%%%%%%%%%%%%%%%%%%%%%%%%%%%%%%%%%%%
    $\Op{\varphi t}$ & $c_{\varphi t}$& $i\big(\varphi^\dagger\,\lra{D}_\mu\,\,\varphi\big)
 \big(\bar{t}\,\gamma^\mu\,t\big)$ \\ \hline
%%%%%%%%%%%%%%%%%%%%%%%%%%%%%%%%%%%%%%%%%%%%%%%%%%%%%%%%%%%%%%%%%%%%%%%%%%%
      $\Op{tW}$ & $c_{tW}$ & $i\big(\bar{Q}\tau^{\mu\nu}\,\tau_{\sss I}\,t\big)\,
 \tilde{\varphi}\,W^I_{\mu\nu}
 + \text{h.c.}$ \\  \hline
%%%%%%%%%%%%%%%%%%%%%%%%%%%%%%%%%%%%%%%%%%%%%%%%%%%%%%%%%%%%%%%%%%%%%%%%%%%
 $\Op{tB}$ & $c_{tB}$~(*) &
 $i\big(\bar{Q}\tau^{\mu\nu}\,t\big)
 \,\tilde{\varphi}\,B_{\mu\nu}
 + \text{h.c.}$ \\\hline
%%%%%%%%%%%%%%%%%%%%%%%%%%%%%%%%%%%%%%%%%%%%%%%%%%%%%%%%%%%%%%%%%%%%%%%%%%%
    $\Op{t G}$ & $c_{tG}$ & $ig{\sss S}\,\big(\bar{Q}\tau^{\mu\nu}\,T_{\sss A}\,t\big)\,
 \tilde{\varphi}\,G^A_{\mu\nu}
 + \text{h.c.}$ \\  \hline
%%%%%%%%%%%%%%%%%%%%%%%%%%%%%%%%%%%%%%%%%%%%%%%%%%%%%%%%%%%%%%%%%%%%%%%%%%%
    $\Op{t \varphi}$ & $c_{t\varphi}$ & $\left(\pdp\right)
 \bar{Q}\,t\,\tilde{\varphi} + \text{h.c.}$  \\\hline
%%%%%%%%%%%%%%%%%%%%%%%%%%%%%%%%%%%%%%%%%%%%%%%%%%%%%%%%%%%%%%%%%%%%%%%%%%%
    $\Op{b \varphi}$ & $c_{b\varphi}$ & $\left(\pdp\right)
 \bar{Q}\,b\,\varphi + \text{h.c.}$ \\
%%%%%%%%%%%%%%%%%%%%%%%%%%%%%%%%%%%%%%%%%%%%%%%%%%%%%%%%%%%%%%%%%%%%%%%%%%%
                \midrule \midrule
		&1st, 2nd generation quarks&\\
                \midrule \midrule
%%%%%%%%%%%%%%%%%%%%%%%%%%%%%%%%%%%%%%%%%%%%%%%%%%%%%%%%%%%%%%%%%%%%%%%%%%%
    $\Op{\varphi q}^{(1)}$ & $c_{\varphi q}^{(1)}$~(*) & $\sum\limits_{\sss i=1,2} i\big(\varphi^\dagger\lra{D}_\mu\,\varphi\big)
 \big(\bar{q}_i\,\gamma^\mu\,q_i\big)$ \\\hline
%%%%%%%%%%%%%%%%%%%%%%%%%%%%%%%%%%%%%%%%%%%%%%%%%%%%%%%%%%%%%%%%%%%%%%%%%%%
    $\Op{\varphi q}^{(3)}$ & $c_{\varphi q}^{(3)}$ & $\sum\limits_{\sss i=1,2} i\big(\varphi^\dagger\lra{D}_\mu\,\tau_{\sss I}\varphi\big)
 \big(\bar{q}_i\,\gamma^\mu\,\tau^{\sss I}q_i\big)$ \\  \hline
%%%%%%%%%%%%%%%%%%%%%%%%%%%%%%%%%%%%%%%%%%%%%%%%%%%%%%%%%%%%%%%%%%%%%%%%%%%
  ${\Op{\varphi u i}}$ &
      ${{c_{\varphi u i}}}$ & $\sum\limits_{\sss i=1,2,3} i\big(\varphi^\dagger\,\lra{D}_\mu\,\,\varphi\big)
 \big(\bar{u}_i\,\gamma^\mu\,u_i\big)$\\ \hline
%%%%%%%%%%%%%%%%%%%%%%%%%%%%%%%%%%%%%%%%%%%%%%%%%%%%%%%%%%%%%%%%%%%%%%%%%%%
 ${\Op{\varphi d i}}$ &
      ${{c_{\varphi d i}}}$ & $\sum\limits_{\sss i=1,2,3} i\big(\varphi^\dagger\,\lra{D}_\mu\,\,\varphi\big)
 \big(\bar{d}_i\,\gamma^\mu\,d_i\big)$\\ \hline
%%%%%%%%%%%%%%%%%%%%%%%%%%%%%%%%%%%%%%%%%%%%%%%%%%%%%%%%%%%%%%%%%%%%%%%%%%%
    $\Op{c \varphi}$ & $c_{c \varphi}$ & $\left(\pdp\right)
 \bar{q}_2\,c\,\tilde\varphi + \text{h.c.}$ \\
 %%%%%%%%%%%%%%%%%%%%%%%%%%%%%%%%%%%%%%%%%%%%%%%%%%%%%%%%%%%%%%%%%%%%%%%%%%%
                \midrule \midrule
		&two-leptons&\\
                \midrule \midrule
%%%%%%%%%%%%%%%%%%%%%%%%%%%%%%%%%%%%%%%%%%%%%%%%%%%%%%%%%%%%%%%%%%%%%%%%%%%
    $\Op{\varphi \ell_i}^{(1)}$ & $c_{\varphi \ell_i}^{(1)}$ & $ i\big(\varphi^\dagger\lra{D}_\mu\,\varphi\big)
   \big(\bar{\ell}_i\,\gamma^\mu\,\ell_i\big)$ \\\hline 
%%%%%%%%%%%%%%%%%%%%%%%%%%%%%%%%%%%%%%%%%%%%%%%%%%%%%%%%%%%%%%%%%%%%%%%%%%%
    $\Op{\varphi \ell_i}^{(3)}$ & $c_{\varphi \ell_i}^{(3)}$ & $ i\big(\varphi^\dagger\lra{D}_\mu\,\tau_{\sss I}\varphi\big)
 \big(\bar{\ell}_i\,\gamma^\mu\,\tau^{\sss I}\ell_i\big)$ \\  \hline
%%%%%%%%%%%%%%%%%%%%%%%%%%%%%%%%%%%%%%%%%%%%%%%%%%%%%%%%%%%%%%%%%%%%%%%%%%%
    $\Op{\varphi e}$ & $c_{\varphi e}$ & $ i\big(\varphi^\dagger\lra{D}_\mu\,\varphi\big)
 \big(\bar{e}\,\gamma^\mu\,e\big)$ \\\hline
%%%%%%%%%%%%%%%%%%%%%%%%%%%%%%%%%%%%%%%%%%%%%%%%%%%%%%%%%%%%%%%%%%%%%%%%%%%
    $\Op{\varphi \mu}$ & $c_{\varphi \mu}$ & $ i\big(\varphi^\dagger\lra{D}_\mu\,\varphi\big)
 \big(\bar{\mu}\,\gamma^\mu\,\mu\big)$ \\  \hline
%%%%%%%%%%%%%%%%%%%%%%%%%%%%%%%%%%%%%%%%%%%%%%%%%%%%%%%%%%%%%%%%%%%%%%%%%%%
    $\Op{\varphi \tau}$ & $c_{\varphi \tau}$ & $i\big(\varphi^\dagger\lra{D}_\mu\,\varphi\big)
 \big(\bar{\tau}\,\gamma^\mu\,\tau\big)$ \\  \hline
%%%%%%%%%%%%%%%%%%%%%%%%%%%%%%%%%%%%%%%%%%%%%%%%%%%%%%%%%%%%%%%%%%%%%%%%%%%
    $\Op{\tau \varphi}$ & $c_{\tau \varphi}$ & $\left(\pdp\right)
 \bar{\ell_3}\,\tau\,{\varphi} + \text{h.c.}$ \\
%%%%%%%%%%%%%%%%%%%%%%%%%%%%%%%%%%%%%%%%%%%%%%%%%%%%%%%%%%%%%%%%%%%%%%%%%%%
                \midrule \midrule
		&four-lepton &\\
                \midrule \midrule
 $\Op{\ell\ell}$ & $c_{\ell\ell}$ & $\left(\bar \ell_1\gamma_\mu \ell_2\right) \left(\bar \ell_2\gamma^\mu \ell_1\right)$ \\
 \hline
  \bottomrule
\end{tabular}
\caption{Same as Table~\ref{tab:oper_bos}
  for the operators containing two fermion fields, either
  quarks or leptons, as well as the four-lepton operator $\Op{\ell\ell}$.
  %
  The flavor index $i$ runs from 1 to 3.
  %
  The coefficients indicated with (*) in the second column do not correspond to physical degrees of freedom
  in the fit, but are rather replaced by  $c_{\varphi q_i}^{(-)}$, $c_{\varphi Q_i}^{(-)}$, and
  $c_{tZ}$ defined in Table~\ref{tab:oper_ferm_bos2}.
\label{tab:oper_ferm_bos}}
\end{center}
\end{table}
%%%%%%%%%%%%%%%%%%%%%%%%%%%%%%%%%%%%%%%%%%%%%%%%%%%%%%%5

%%%%%%%%%%%%%%%%%%%%%%%%%%%%%%%%%%%%%%
%
Table~\ref{tab:oper_ferm_bos} collects, using the same format
as in Table~\ref{tab:oper_bos}, the relevant Warsaw-basis operators
that contain two fermion fields, either quarks or leptons,
plus a single four-lepton operator.
%
From top to bottom, we list the two-fermion operators involving 3rd generation quarks,
those involving 1st and 2nd generation quarks, and
operators containing two leptonic fields (of any generation).
%
We also include in this list the four-lepton operator $\mathcal{O}_{\ell\ell}$.

The operators that involve a top-quark field, either $Q$ (left-handed doublet) or $t$
(right-handed singlet),
are crucial for the interpretation of LHC top-quark measurements.
%
Interestingly, all of them involve at least one Higgs-boson field, which
introduces an interplay between the top and Higgs sectors of the SMEFT.
%
For example, the chromo-magnetic dipole operator $O_{tG}$ and the dimension-six Yukawa
operator $O_{t\varphi}$ are constrained by both top quark measurements, such
as $t\bar t{h}$ associated production, as well as Higgs measurements, such
as Higgs production through gluon fusion.
%
Furthermore, the electroweak-dipole operators, $O_{tW}$ and $O_{tB}$, as well as the
current operators, 
$O_{\varphi Q}^{(3)}$ and $O_{\varphi t}$, can be constrained by the associated production
of single top-quarks and Higgs bosons,
as well as by the loop-induced Higgs decays into a $Z\gamma$ final state.

In Table~\ref{tab:oper_ferm_bos} we also list operators that contain light quark
(1st and 2nd generation)
and leptonic fields (of any generation).
%
The light quark operators enter the Higgs production through the $Vh$ and
VBF channels, as well as the diboson processes.
%
These operators also modify the Higgs boson width and branching ratios.
%
For example, the Higgs decay width to $q\bar{q}\ell^+\ell^-$ becomes modified by operators that
induce an effective $Zhq\bar{q}$ vertex, such as $\mathcal{O}_{\varphi u}$.
%
The leptonic operators are relevant for the same reason,
once we account for the leptonic decays of the Higgs and gauge bosons.
%
In addition, indirect contributions arise from  the $\mathcal{O}_{\varphi
\ell_1}^{(3)}$, $\mathcal{O}_{\varphi \ell_2}^{(3)}$, and $\mathcal{O}_{\ell\ell}$ operators, which
 modify the measurement of the Fermi constant, $G_F$,
and this affects the extracted SM parameters. They therefore introduce a universal
contribution to all electroweak interactions, and are relevant for $Vh$, VBF, and for the
diboson channels.

%%%%%%%%%%%%%%%%%%%%%%%%%%%%%%%%%%%%%%%%%%%%%%%%%%%%%%%%%%%%%%%%%%%%%%%%%%%
\begin{table}[htbp]
  \begin{center}
    \renewcommand{\arraystretch}{1.49}
    \begin{tabular}{ll}
      \toprule
      DoF $\qquad$  & Definition \\
                \midrule
$c_{\varphi Q}^{(-)}$ &  $c_{\varphi Q}^{(1)} - c_{\varphi Q}^{(3)}$ \\
\midrule
$c_{tZ}$ &   $-\sin\theta_W c_{tB} + \cos\theta_W c_{tW} $\\
\midrule
$c_{\varphi q}^{(-)}$ & $ c_{\varphi q}^{(1)} - c_{\varphi q}^{(3)}$ \\
  \bottomrule
\end{tabular}
    \caption{Additional degrees of freedom defined from linear combinations of
      the two-fermion operators listed in Table~\ref{tab:oper_ferm_bos}.
The first two DoFs modify the $t\bar{t}Z$ couplings,
while the third combination is introduced for consistency with the first one.
%
These are the DoFs that enter at the fit level, replacing those
marked with  (*) in Table~\ref{tab:oper_ferm_bos}.
\label{tab:oper_ferm_bos2}}
\end{center}
\end{table}
%%%%%%%%%%%%%%%%%%%%%%%%%%%%%%%%%%%%%%%%%%%%%%%%%%%%%%%%%%%%%%%%%%%%%%%%%%%

We point out that
most of the operator coefficients defined in Table~\ref{tab:oper_ferm_bos} correspond
directly to degrees of freedom used in
the fit, except for three of them, which are
indicated with a (*) in the second column.
%
Instead, following Ref.~\cite{AguilarSaavedra:2018nen},
three additional degrees of freedom are defined from the linear
combinations indicated in Table~\ref{tab:oper_ferm_bos2}.
%
These are the DoFs that enter at the fit level, replacing those
marked with  a (*) in Table~\ref{tab:oper_ferm_bos}.

Finally, we note that, as mentioned above, here flavour universality in the
leptonic sector is not imposed, and thus the coefficients of the operators
involving bilinears in the electron, muon, and tau lepton fields are in
principle independent.  In total we have 23 independent fit parameters, defined
from two-fermion operators, plus in addition the four-lepton
operator $c_{\ell\ell}$.
%
However, in practice, this flexibility will not be relevant for the present
fit due to the constraints from the EWPOs, to be discussed next.

\paragraph{The role of electroweak precision observables.}
%
At this point, one should note that a subset
of the dimension-six operators defined in
Tables~\ref{tab:oper_bos} and~\ref{tab:oper_ferm_bos} are already well
constrained by the electroweak precision observables (EWPO)~\cite{Han:2004az}
measured at the $Z$-pole~\cite{ALEPH:2005ab} and  the $W$-pole at the LEP and
SLC electron-position colliders.
%
Given in particular the high accuracy of these LEP measurements, these constraints are known
to dominate in many cases when compared to those provided by the LHC
cross-sections.  
%
Specifically, the operators sensitive to the EWPO are the following (with definitions
presented in Tables~\ref{tab:oper_bos} and \ref{tab:oper_ferm_bos})
\be
\mathcal{O}_{\varphi WB},
\mathcal{O}_{\varphi D},
\mathcal{O}_{\varphi q}^{\sss(1)},
\mathcal{O}_{\varphi q}^{\sss(3)}, \mathcal{O}_{\varphi u i}, \mathcal{O}_{\varphi d i},
\mathcal{O}_{\varphi \ell_i}^{\sss(3)}, \mathcal{O}_{\varphi \ell_i}^{\sss(1)},
\mathcal{O}_{\varphi e/\mu/\tau}, \mathcal{O}_{\ell\ell} \, . \label{eq:LEPconstrainedDoFs}
\ee
Note that, with $i=1,2,3$, these add up to 16 operators, rather than the 10 which would
correspond to the flavour universal configuration in the leptonic sector.

Fourteen linear combinations of the coefficients associated to these 16 operators  
are constrained by the LEP EWPOs~\cite{Falkowski:2014tna},
leaving therefore only two linear combinations unconstrained. 
%
These two remaining unconstrained directions  can be
determined  from the information contained in
diboson production cross-sections~\cite{Grojean:2006nn,Alonso:2013hga,Brivio:2017bnu}
as well as by the Higgs production and decay measurements.
{ 
For completeness, the 14 linear combinations of bosonic and two-fermion Wilson coefficients 
which are constrained by the EWPOs measured at LEP are the following~\cite{Brivio:2017bnu}:
% \begin{align} \nonumber 
% &\frac{1}{4} g_1^2 \left(- 2c_{\varphi \ell_1}^{(3)}-2 c_{\varphi \ell_2}^{(3)}+c_{\ell\ell}\right)-\frac{c_{\varphi D}g_w^2}{4}-g_1 g_w c_{\varphi W B}, \\ \nonumber 
% &c_{\varphi \ell_i}^{(3)}-f\left(-\frac{1}{2},-1\right)+f\left(\frac{1}{2},0\right)\,, \quad  i=1,2,3, \\ \nonumber
% &f\left(-\frac{1}{2},-1\right)-\frac{c_{\varphi \ell_i}^{(3)}}{2}-\frac{c_{\varphi \ell_i}^{(1)}}{2} \,, \quad  i=1,2,3, \nonumber \\
% &f(0,-1)-\frac{c_{\varphi e}}{2}, \,\, \quad f(0,-1)-\frac{c_{\varphi \mu}}{2}, \,\, \quad f(0,-1)-\frac{c_{\varphi \tau}}{2}, \label{LEPconstraints}\\
% %&f\left(\frac{1}{2},\frac{2}{3}\right)-\frac{c_{\varphi q_i}^{(-)}}{2}, \,\,\quad f\left(-\frac{1}{2},-\frac{1}{3}\right)-\frac{c_{\varphi q_i}^{(-)}}{2}-c_{\varphi q_i}^{(3)} \nonumber\\
% &f\left(\frac{1}{2},\frac{2}{3}\right)-\frac{c_{\varphi q}^{(-)}}{2}, \,\,\quad f\left(-\frac{1}{2},-\frac{1}{3}\right)-\frac{c_{\varphi q}^{(-)}}{2}-c_{\varphi q}^{(3)} \nonumber\\
% &f\left(0,\frac{2}{3}\right)-\frac{c_{\varphi u i}}{2}, \,\, \quad f\left(0,-\frac{1}{3}\right)-\frac{c_{\varphi d i}}{2}, \nonumber
% \end{align} where the function $f$ is defined by:
% \begin{equation}
% f(T_3,Q)=\left(-\frac{c_{\varphi \ell_1}^{(3)}}{2}-\frac{c_{\varphi \ell_2}^{(3)}}{2}+\frac{c_{\ell\ell}}{4}-\frac{c_{\varphi D}}{4}\right) \left(\frac{g_1^2 Q}{g_w^2-g_1^2}+T_3\right)-c_{\varphi W B}\frac{ Q g_1 g_w}{g_w^2-g_1^2} \, ,
% \end{equation}
\begin{align}
\nonumber
\delta g_{V}^{l_i}&=\delta \bar{g}_{Z} \bar{g}_{V}^{l_i}+Q^{l_i} \delta s_{\theta}^{2}+\Delta_{V}^{l_i} = 0\, , \quad i=1,2,3\, , \\\nonumber
\delta g_{A}^{l_i}&=\delta \bar{g}_{Z} \bar{g}_{A}^{l_i}+\Delta_{A}^{l_i} = 0\, , \quad i=1,2,3\, , \\\nonumber
\delta g_{V}^{u}&=\delta \bar{g}_{Z} \bar{g}_{V}^{u}+Q^{u} \delta s_{\theta}^{2}+\Delta_{V}^{u} = 0\, , \\\nonumber
\delta g_{A}^{u}&=\delta \bar{g}_{Z} \bar{g}_{A}^{u}+\Delta_{A}^{u} = 0\, ,\label{LEPconstraints} \\
\delta g_{V}^{d}&=\delta \bar{g}_{Z} \bar{g}_{V}^{d}+Q^{d} \delta s_{\theta}^{2}+\Delta_{V}^{d} = 0\, , \\\nonumber
\delta g_{A}^{d}&=\delta \bar{g}_{Z} \bar{g}_{A}^{d}+\Delta_{A}^{d} = 0\, , \\\nonumber
\delta g_V^{W, l_i}&= \frac{c_{ll} + 2 c_{\varphi \ell_i}^{(3)} - c_{\varphi \ell_1}^{(3)} - c_{\varphi \ell_2}^{(3)}}{4 \sqrt{2} G_F} = 0 \, , \quad i=1,2,3\, , \\\nonumber
\delta g_V^{W, q}&= \frac{ c_{ll} + c_{\varphi q}^{(3)} - c_{\varphi \ell_1}^{(3)} - c_{\varphi \ell_2}^{(3)}}{4 \sqrt{2} G_F} = 0 \, ,
\end{align}
where $g_1$ and $g_w$ are the corresponding electroweak couplings, $\bar{g}_{V}^{f}=T_{3} / 2-Q^{f} \bar{s}_{\theta}^{2}$, $\bar{g}_{A}^{f}=T_{3} / 2$ and
\begin{align*}
\Delta_{V}^{\ell_i} &=-\frac{1}{4 \sqrt{2} \hat{G}_{F}}\left(c_{\varphi \ell_i}^{(1)}+c_{\varphi \ell_i}^{(3)}+c_{\varphi e_i}\right) & \Delta_{A}^{\ell_i}&=-\frac{1}{4 \sqrt{2} \hat{G}_{F}}\left(c_{\varphi \ell_i}^{(1)}+c_{\varphi \ell_i}^{(3)}-c_{\varphi e_i}\right) \\
\Delta_{V}^{u} &=-\frac{1}{4 \sqrt{2} \hat{G}_{F}}\left(c_{\varphi q}^{(1)}-c_{\varphi q}^{(3)}+c_{\varphi u i}\right) & \Delta_{A}^{u} &=-\frac{1}{4 \sqrt{2} \hat{G}_{F}}\left(c_{\varphi q}^{(1)}-c_{\varphi q}^{(3)}-c_{\varphi u i}\right) \\
\Delta_{V}^{d} &=-\frac{1}{4 \sqrt{2} \hat{G}_{F}}\left(c_{\varphi q}^{(1)}+c_{\varphi q}^{(3)}+c_{\varphi d i}\right) & \Delta_{A}^{d} &=-\frac{1}{4 \sqrt{2} \hat{G}_{F}}\left(c_{\varphi q}^{(1)}+c_{\varphi q}^{(3)}-c_{\varphi d i}\right) \\
\delta g_Z &=-\frac{1}{4 \sqrt{2} \hat{G}_{F}}\left(c_{\varphi D}+2 c_{\varphi \ell_1}^{(3)} + 2 c_{\varphi \ell_2}^{(3)}-2 c_{l l}\right) & \delta s_{\theta}^{2}&= \frac{\hat{m}_{W}^{2}}{2 \sqrt{2} \hat{G}_{F} \hat{m}_{Z}^{2}}\left(c_{\varphi D} + \sqrt{\frac{\hat{m}_{Z}^{2}}{\hat{m}_{W}^{2}} - 1} \, c_{\varphi W B}  \right) \, ,\\ 
\end{align*}
where we have used the notation of Ref.~\cite{Brivio:2017bnu}. We note that the modifications of the
$W$ and $Z$ couplings in Eq.~(\ref{LEPconstraints}) are given in the $\lp m_W,m_Z, G_F\rp$ scheme.
%
Similar 
expressions in the $\lp a_{\rm ew},m_Z, G_F\rp$ scheme can be found in 
 App.~A of~\cite{Falkowski:2001958}.
%
}
 In a flavour universal scenario, the 
 $Z$- and $W$-pole observables used to 
 constrain these couplings
 are those listed in Table~1 of~\cite{Falkowski:2014tna}. These constrain 8 out of 10 linear combinations of 
 the Warsaw operators which are present in the flavour universal scenario.
%
 Our
 assumption in this work is stronger, as we assume that there are enough observables to constrain all but 2 degrees of
 freedom of Eq.~(\ref{eq:LEPconstrainedDoFs}). In order to achieve this, one
 would need to go beyond the standard EWPO observables e.g.
beyond those of Table 1 of~\cite{Falkowski:2014tna},
in particular by  including more data which will allow one to constrain more
degrees of freedom which appear in our non-flavour universal scenario.
%
For example, we distinguish
between leptons of different generations, and therefore a setup like the one
of~\cite{Efrati:2015eaa} would be more appropriate.
%
Our assumption is that these observables are precise enough to constrain all but two linear combinations. This is supported for example by Ref.~\cite{Efrati:2015eaa} which suggests that even with less
restrictive flavour assumptions,
the constraints on these Wilson coefficients remain relatively stringent.


While in this work we do not explicitly include any EWPO data in the present
fit, we still need to account for the information that they provide on the
SMEFT parameter space.
%
{  As motivated above,}
this is achieved by assuming that the EWPOs are precise enough to allow us to
set the 14 linear combinations of Eq.~\eqref{LEPconstraints} to zero in our
fit.
%
{  In other words, one can derive
  constraints in the SMEFT parameter space, Eq.~(\ref{eq:2independents}) below,
  that emulate the information contained in the EWPOs
  by setting all quantities in Eq.~(\ref{LEPconstraints}) to zero.}
%
The remaining two degrees of freedom can be parametrized by, say,
$ c_{\varphi WB}$ and $c_{\varphi D}$, if the following replacements are made
\begin{flalign}
	\left(
\begin{array}{c}
c_{\varphi \ell_i}^{\sss(3)} \\
 c_{\varphi \ell_i}^{\sss(1)} \\
 c_{\varphi e/\mu/\tau} \\
 c_{\varphi q}^{(-)} \\
 c_{\varphi q}^{(3)} \\
 c_{\varphi u} \\
 c_{\varphi d} \\
 c_{\ell\ell} \\
\end{array}
\right)
= 
\left(
\begin{array}{cc}
 -\frac{1}{t_W} & -\frac{1}{4 t_W^2} \\
 0 & -\frac{1}{4} \\
 0 & -\frac{1}{2} \\
 \frac{1}{t_W} & \frac{1}{4 s_W^2}-\frac{1}{6} \\
 -\frac{1}{t_W} & -\frac{1}{4 t_W^2} \\
 0 & \frac{1}{3} \\
 0 & -\frac{1}{6} \\
 0 & 0 \\
\end{array}
\right)
\left(
\begin{array}{c}
	c_{\varphi WB}\\ c_{\varphi D}
\end{array}
\right) \, .
\label{eq:2independents}
\end{flalign}
These relations will emulate the impact of LEP EWPOs in the fit, and allow us to
produce a consistent fit without explicitly including the EWPOs.

{ 
  We note that there is one additional combination of Wilson coefficients that
  could be constrained by the EWPOs, namely
  \be
  \label{eq:ZbbLEP}
  c_{\varphi Q}^{(1)} +  c_{\varphi Q}^{(3)} =  c_{\varphi Q}^{(-)}
  + 2c_{\varphi Q}^{(3)} \, ,
  \ee
  which modifies the left-handed coupling of the $Z$ boson
  to bottom quarks.
  %
  However, in this work we prefer to constrain this
  combination directly from top quark production
  measurements rather than from the EWPOs.
  Therefore we have kept both $c_{\varphi Q}^{(-)}$
   and $c_{\varphi Q}^{(3)}$ in our fit to be able to assess 
   how well top measurements can constrain these two degrees of freedom. 
  %
 As a cross-check,
  we have verified that the global fit results for other operators are essentially
  unchanged if Eq.~(\ref{eq:ZbbLEP}) is assumed to be constrained
  by LEP rather than by top quark data in the fit, whilst bounds on $c_{\varphi Q}^{(-)}$
  would improve by about a factor of two had we applied this constraint in the fit. 
}

Thanks to these 14 constraints, the 7 and
24 operators listed in Tables~\ref{tab:oper_bos} and~\ref{tab:oper_ferm_bos} respectively
are then reduced to 17 independent degrees of freedom to be constrained by the
LHC experimental data and the LEP diboson cross-sections.
%
This allows us to set bounds on all operator coefficients
listed in Tables~\ref{tab:oper_bos} and~\ref{tab:oper_ferm_bos}.
%
Of course, the bounds on the 16
operators of Eq.~(\ref{eq:LEPconstrainedDoFs})
will be highly correlated as indicated by Eq.~(\ref{eq:2independents}).
%
When presenting results for the independent DoFs, for example when evaluating
the Fisher Information matrix or the principal components, we will select
$c_{\varphi W B}$ and $c_{\varphi D}$, with the understanding that the
replacements of Eq.~(\ref{eq:2independents}) have been made.
%
Note that it has been argued that the diboson channels at the LHC can in
principle compete with EWPO \cite{Zhang:2016zsp,Grojean:2018dqj}, which
indicates that in an accurate fit one should always include the full set of
EWPO constraints explicitly, as has been done, for example, in the combined
Higgs/electroweak fits of~\cite{Ellis:2018gqa,Ellis:2020unq}.
%
We  however leave this option to future work.

\paragraph{Four-fermion top quark operators.}
%
We finally discuss the four-quark operators which involve the top quark 
fields and thus modify the production of top quarks at hadron colliders.
%
The dimension-six four-fermion operators sensitive to top quarks can be classified into
two categories: operators composed by four heavy quark fields (top and/or bottom quarks) and 
operators composed by two light and two heavy quark fields.
The physical degrees of freedom corresponding to four-heavy and
two-light-two-heavy interactions that we use in the present analysis are
constructed in terms of suitable linear combinations of the four fermion
coefficients in the Warsaw basis, whose corresponding operators are defined as
\begin{align}
	\qq{1}{qq}{ijkl}
	&= (\bar q_i \gamma^\mu q_j)(\bar q_k\gamma_\mu q_l)
	 \nonumber
	,\\
	\qq{3}{qq}{ijkl}
	&= (\bar q_i \gamma^\mu \tau^I q_j)(\bar q_k\gamma_\mu \tau^I q_l)
 \nonumber
	,\\
	\qq{1}{qu}{ijkl}
	&= (\bar q_i \gamma^\mu q_j)(\bar u_k\gamma_\mu u_l)
         \nonumber
	,\\
	\qq{8}{qu}{ijkl}
	&= (\bar q_i \gamma^\mu T^A q_j)(\bar u_k\gamma_\mu T^A u_l)
         \nonumber
	,\\
	\qq{1}{qd}{ijkl}
	&= (\bar q_i \gamma^\mu q_j)(\bar d_k\gamma_\mu d_l)
         \nonumber
	,\\
	\qq{8}{qd}{ijkl}
	&= (\bar q_i \gamma^\mu T^A q_j)(\bar d_k\gamma_\mu T^A d_l)
        \label{eq:FourQuarkOp} %%%%%%%%%%%%%%%%%%%%%%%%
	,\\
	\qq{}{uu}{ijkl}
	&=(\bar u_i\gamma^\mu u_j)(\bar u_k\gamma_\mu u_l)
         \nonumber
	,\\
	\qq{1}{ud}{ijkl}
	&=(\bar u_i\gamma^\mu u_j)(\bar d_k\gamma_\mu d_l)
         \nonumber
	,\\
	\qq{8}{ud}{ijkl}
	&=(\bar u_i\gamma^\mu T^A u_j)(\bar d_k\gamma_\mu T^A d_l)
         \nonumber \, ,
\end{align}
where recall that $i,j,k,l$ are fermion generation indices.
%
In Table~\ref{eq:summaryOperatorsTop} we provide the definition of all degrees of freedom
that enter the fit
in terms of the coefficients of Warsaw basis operators of Eq.~(\ref{eq:FourQuarkOp}).
%
Within our flavour assumptions, the coefficients associated to different values of
the generation indices $i$ ($i=1,2$) or $j$ ($j=1,2,3$) will be the same.

%%%%%%%%%%%%%%%%%%%%%%%%%%%%%%%%%%%%%%%%%%%%%%%%
%%%%%%%%%%%%%%%%%%%%%%%%%%%%%%%%%%%%%%%%%%%%%%%%%%%%%%%5
\begin{table}[htbp] 
  \begin{center}
    \renewcommand{\arraystretch}{1.53}
        \begin{tabular}{ll}
          \toprule
          DoF $\qquad$ &  Definition (in  Warsaw basis notation) \\
          \midrule
      $c_{QQ}^1$    &   $2\ccc{1}{qq}{3333}-\frac{2}{3}\ccc{3}{qq}{3333}$ \\ \hline
%%%%%%%%%%%%%%%%%%%%%%%%%%%%%%%%%%5%%%%%%%%%%%%%%%%%%%%%%%%%%%%%%%%%%%%%%%%%%%%%%%%%%%%%%%%%%%%%%%%%%%%%%
    $c_{QQ}^8$       &         $8\ccc{3}{qq}{3333}$\\  \hline
          %%%%%%%%%%%%%%%%%%%%%%%%%%%%%%%%%%5%%%%%%%%%%%%%%%%%%%%%%%%%%%%%%%%%%%%%%%%%%%%%%%%%%%%%%%%%%%%%%%%%%%%%%
 $c_{Qt}^1$         &         $\ccc{1}{qu}{3333}$\\   \hline
          %%%%%%%%%%%%%%%%%%%%%%%%%%%%%%%%%%5%%%%%%%%%%%%%%%%%%%%%%%%%%%%%%%%%%%%%%%%%%%%%%%%%%%%%%%%%%%%%%%%%%%%%%
 $c_{Qt}^8$         &         $\ccc{8}{qu}{3333}$\\   \hline
          %%%%%%%%%%%%%%%%%%%%%%%%%%%%%%%%%%5%%%%%%%%%%%%%%%%%%%%%%%%%%%%%%%%%%%%%%%%%%%%%%%%%%%%%%%%%%%%%%%%%%%%%%
%  $\mathcal{O}_{Qb}^1$         &    {\tt OQb1}~~(\verb|cQb1|,~~$c_{Qb}^1$)         &   $\qq{1}{qd}{3333}$\\    \hline
          %%%%%%%%%%%%%%%%%%%%%%%%%%%%%%%%%%5%%%%%%%%%%%%%%%%%%%%%%%%%%%%%%%%%%%%%%%%%%%%%%%%%%%%%%%%%%%%%%%%%%%%%%
 % $\mathcal{O}_{Qb}^8$            &    {\tt OQb8}~~(\verb|cQb1|,~~$c_{Qb}^8$) &    $\qq{8}{qd}{3333}$\\   \hline
          %%%%%%%%%%%%%%%%%%%%%%%%%%%%%%%%%%5%%%%%%%%%%%%%%%%%%%%%%%%%%%%%%%%%%%%%%%%%%%%%%%%%%%%%%%%%%%%%%%%%%%%%%
  $c_{tt}^1$         &     $\ccc{}{uu}{3333}$  \\    \hline
          %%%%%%%%%%%%%%%%%%%%%%%%%%%%%%%%%%5%%%%%%%%%%%%%%%%%%%%%%%%%%%%%%%%%%%%%%%%%%%%%%%%%%%%%%%%%%%%%%%%%%%%%%
%     $\mathcal{O}_{tb}^1$        &    {\tt Otb1}~~(\verb|ctb1|,~~$c_{tb}^1$)  &     $\qq{1}{ud}{3333}$\\   \hline
          %%%%%%%%%%%%%%%%%%%%%%%%%%%%%%%%%%5%%%%%%%%%%%%%%%%%%%%%%%%%%%%%%%%%%%%%%%%%%%%%%%%%%%%%%%%%%%%%%%%%%%%%%
 %  $\mathcal{O}_{tb}^8$          &    {\tt Otb8}~~(\verb|ctb8|,~~$c_{tb}^8$)  &       $\qq{8}{ud}{3333}$ \\    \hline
          %%%%%%%%%%%%%%%%%%%%%%%%%%%%%%%%%%5%%%%%%%%%%%%%%%%%%%%%%%%%%%%%%%%%%%%%%%%%%%%%%%%%%%%%%%%%%%%%%%%%%%%%%
%  $\mathcal{O}_{QtQb}^1$           &    {\tt OQtQb1}~~(\verb|cQtQb1|,~~$c_{QtQb}^1$)  &       $\qq{1}{quqd}{3333}$ \\    \hline
          %%%%%%%%%%%%%%%%%%%%%%%%%%%%%%%%%%5%%%%%%%%%%%%%%%%%%%%%%%%%%%%%%%%%%%%%%%%%%%%%%%%%%%%%%%%%%%%%%%%%%%%%%
 %   $\mathcal{O}_{QtQb}^8$         &    {\tt OQtQb8}~~(\verb|cQtQb8|,~~$c_{QtQb}^8$) &     $\qq{8}{quqd}{3333}$ \\
          %%%%%%%%%%%%%%%%%%%%%%%%%%%%%%%%%%5%%%%%%%%%%%%%%%%%%%%%%%%%%%%%%%%%%%%%%%%%%%%%%%%%%%%%%%%%%%%%%%%%%%%%%
            \midrule      
  $c_{Qq}^{1,8}$       &  	 $\ccc{1}{qq}{i33i}+3\ccc{3}{qq}{i33i}$     \\   \hline
        %%%%%%%%%%%%%%%%%%%%%%%%%%%%%%%%%%5%%%%%%%%%%%%%%%%%%%%%%%%%%%%%%%%%%%%%%%%%%%%%%%%%%%%%%%%%%%%%%%%%%%%%%
  $c_{Qq}^{1,1}$         &   $\ccc{1}{qq}{ii33}+\frac{1}{6}\ccc{1}{qq}{i33i}+\frac{1}{2}\ccc{3}{qq}{i33i} $   \\    \hline
        %%%%%%%%%%%%%%%%%%%%%%%%%%%%%%%%%%5%%%%%%%%%%%%%%%%%%%%%%%%%%%%%%%%%%%%%%%%%%%%%%%%%%%%%%%%%%%%%%%%%%%%%%
   $c_{Qq}^{3,8}$         &   $\ccc{1}{qq}{i33i}-\ccc{3}{qq}{i33i} $   \\   \hline
        %%%%%%%%%%%%%%%%%%%%%%%%%%%%%%%%%%5%%%%%%%%%%%%%%%%%%%%%%%%%%%%%%%%%%%%%%%%%%%%%%%%%%%%%%%%%%%%%%%%%%%%%%
  $c_{Qq}^{3,1}$          & 	$\ccc{3}{qq}{ii33}+\frac{1}{6}(\ccc{1}{qq}{i33i}-\ccc{3}{qq}{i33i}) $   \\     \hline
        %%%%%%%%%%%%%%%%%%%%%%%%%%%%%%%%%%5%%%%%%%%%%%%%%%%%%%%%%%%%%%%%%%%%%%%%%%%%%%%%%%%%%%%%%%%%%%%%%%%%%%%%%
   $c_{tq}^{8}$         &  $ \ccc{8}{qu}{ii33}   $ \\    \hline
        %%%%%%%%%%%%%%%%%%%%%%%%%%%%%%%%%%5%%%%%%%%%%%%%%%%%%%%%%%%%%%%%%%%%%%%%%%%%%%%%%%%%%%%%%%%%%%%%%%%%%%%%%
   $c_{tq}^{1}$       &   $  \ccc{1}{qu}{ii33} $\\    \hline
        %%%%%%%%%%%%%%%%%%%%%%%%%%%%%%%%%%5%%%%%%%%%%%%%%%%%%%%%%%%%%%%%%%%%%%%%%%%%%%%%%%%%%%%%%%%%%%%%%%%%%%%%%
   $c_{tu}^{8}$      &   $2\ccc{}{uu}{i33i}$  \\     \hline
        %%%%%%%%%%%%%%%%%%%%%%%%%%%%%%%%%%5%%%%%%%%%%%%%%%%%%%%%%%%%%%%%%%%%%%%%%%%%%%%%%%%%%%%%%%%%%%%%%%%%%%%%%
    $c_{tu}^{1}$        &   $ \ccc{}{uu}{ii33} +\frac{1}{3} \ccc{}{uu}{i33i} $ \\   \hline
        %%%%%%%%%%%%%%%%%%%%%%%%%%%%%%%%%%5%%%%%%%%%%%%%%%%%%%%%%%%%%%%%%%%%%%%%%%%%%%%%%%%%%%%%%%%%%%%%%%%%%%%%%
    $c_{Qu}^{8}$         &  $  \ccc{8}{qu}{33ii}$\\     \hline
        %%%%%%%%%%%%%%%%%%%%%%%%%%%%%%%%%%5%%%%%%%%%%%%%%%%%%%%%%%%%%%%%%%%%%%%%%%%%%%%%%%%%%%%%%%%%%%%%%%%%%%%%%
    $c_{Qu}^{1}$     &  $  \ccc{1}{qu}{33ii}$  \\     \hline
        %%%%%%%%%%%%%%%%%%%%%%%%%%%%%%%%%%5%%%%%%%%%%%%%%%%%%%%%%%%%%%%%%%%%%%%%%%%%%%%%%%%%%%%%%%%%%%%%%%%%%%%%%
    $c_{td}^{8}$        &   $\ccc{8}{ud}{33jj}$ \\    \hline
        %%%%%%%%%%%%%%%%%%%%%%%%%%%%%%%%%%5%%%%%%%%%%%%%%%%%%%%%%%%%%%%%%%%%%%%%%%%%%%%%%%%%%%%%%%%%%%%%%%%%%%%%%
    $c_{td}^{1}$          &  $ \ccc{1}{ud}{33jj}$ \\     \hline
        %%%%%%%%%%%%%%%%%%%%%%%%%%%%%%%%%%5%%%%%%%%%%%%%%%%%%%%%%%%%%%%%%%%%%%%%%%%%%%%%%%%%%%%%%%%%%%%%%%%%%%%%%
    $c_{Qd}^{8}$        &   $ \ccc{8}{qd}{33jj}$ \\     \hline
        %%%%%%%%%%%%%%%%%%%%%%%%%%%%%%%%%%5%%%%%%%%%%%%%%%%%%%%%%%%%%%%%%%%%%%%%%%%%%%%%%%%%%%%%%%%%%%%%%%%%%%%%%
    $c_{Qd}^{1}$         &   $ \ccc{1}{qd}{33jj}$\\
        %%%%%%%%%%%%%%%%%%%%%%%%%%%%%%%%%%5%%%%%%%%%%%%%%%%%%%%%%%%%%%%%%%%%%%%%%%%%%%%%%%%%%%%%%%%%%%%%%%%%%%%%%
         \bottomrule
  \end{tabular}
  \caption{\small Definition of the four-fermion degrees of freedom that enter into
    the fit in terms of the coefficients of Warsaw basis operators of Eq.~(\ref{eq:FourQuarkOp}).
    %
    These DoFs are classified into four-heavy (upper) and two-light-two-heavy
    (bottom part) operators. The flavor index $i$ is either 1 or 2, 
    and $j$ is either 1, 2 or 3: with our flavor assumptions,  these coefficients will be the same
    regardless of the specific values that $i$ and $j$ take.
\label{eq:summaryOperatorsTop}}
  \end{center}
\end{table}
%%%%%%%%%%%%%%%%%%%%%%%%%%%%%%%%%%%%%%%%%%%%%%%%%%%%%%%%%%%%%%%%%%%%%%%%%%%%%%%



%%%%%%%%%%%%%%%%%%%%%%%%%%%%%%%%%%%%%%%%%%%%%%%%

Comparing with our previous EFT analysis of the top quark sector,
in this work due to the different flavor assumptions
several degrees of freedom that were used there as  independent fit parameters are now absent.
%
{  The reason
  is that here we assume  $U (2)_q \times U (2)_u \times U (3)_d $
  as compared to  $U (2)_q \times U (2)_u \times U (2)_d$ in~\cite{Hartland:2019bjb}.
%
  The difference is that right-handed bottom  quarks are now treated on the same footing as the right-handed down-type quarks of the first two generations.
  %
  Furthermore, this
  flavour assumption forbids quark
  bilinears such as the chirality-flipping $\bar{Q}b$ and the right-handed charged current $\bar{t}b$.
  %
  These modified flavor assumptions have two main consequences.
}
%
First of all, the coefficients $c_{QtQb}^1$ and $c_{QtQb}^8$ are set to zero.
%
In addition, four-heavy operators that involve
right-handed bottom quarks are not free parameters anymore.
%
The correspondence between these four-heavy degrees of freedom
from~\cite{Hartland:2019bjb} and those of the present work is
\be
c_{Qb}^1=c_{Qd}^1 \,,
\quad c_{Qb}^8=c_{Qd}^8\,,
\quad c_{tb}^1=c_{td}^1 \,,
\quad c_{tb}^8=c_{td}^8 \, .
\ee
Furthermore, we do not have $c_{Qb,tb}^{1,8}$ in the present fit anymore.
%
These considerations explain why the 11 four-heavy operators of our previous study
are now reduced to the 5 listed in Table~\ref{eq:summaryOperatorsTop}.

All in all, in total we end up with 5 degrees of
freedom involving four heavy quark fields and 14 involving two light and two
heavy quark fields, for a total of 19 independent parameters at the fit level
associated to four-quark operators.
%
The more stringent flavour assumptions restricting the four-heavy operators
imply that the constraints that we will obtain in the present fit for the four-fermion
operators will be superior, thanks to these new constraints as well as the
addition of the latest top production measurements from Run II of the LHC.

{  One should also mention that the flavour assumptions
  adopted in this work allow in principle for the presence of
  additional four-fermion operators that do not involve
  the top quark.
  %
  One example would be operators containing four bottom quarks.
  %
  These operators are however not directly constrained by any of the measurements
  that we consider in this work, and hence we do not take them into account.
  %
  Future work with an extended dataset e.g. with LHC dijet and multijet measurements~\cite{Domenech:2012ai,Biekotter:2018rhp}, Drell-Yan production~\cite{Dawson:2018dxp},
  and low-energy measurements~\cite{Falkowski:2017pss} will allow directly constraining such
  light four-fermion operators.
  }

\paragraph{Overview of the degrees of freedom.}
%
We summarise in Table~\ref{tab:operatorbasis}
the degrees of freedom  considered in the present work.
%
These are associated either to the Wilson coefficients of Warsaw-basis operators
or to linear combinations of those.
%
We categorize the DoFs into five disjoint classes, from top to bottom: four-quark (two-light-two-heavy), four-quark (four-heavy), four-lepton, two-fermion, and purely bosonic DoFs.
%
We end up with 50 EFT coefficients that enter the theory predictions associated to the processes
input to the fit, of which 36 are independent.
%
The 16 DoFs displayed in the last columns are subject to the 14 constraints from the EWPOs
listed in Eq.~(\ref{eq:2independents}),
leaving only 2 independent combinations to be constrained by the fit.
%
When presenting results for the independent DoFs, for example when evaluating the
Fisher Information matrix, we will select $c_{\varphi W B}$ and $c_{\varphi D}$,
for illustration purposes.
%
Then in Table~\ref{tab:notation_coeffs} we indicate the
notation that will be used to indicate the EFT coefficients
listed in  Table~\ref{tab:operatorbasis} in the
subsequent sections, as well as in the released output files
with the results of the global analysis,
where again
only two of the 16 EFT coefficients labelled in blue are independent fit parameters.

%%%%%%%%%%%%%%%%%%%%%%%%%%%%%%%%%%%%%%%%%%%%%%%%
%%%%%%%%%%%%%%%%%%%%%%%%%%%%%%%%%%%%%%%%%%%%%%%%%%%%%%%%%%%%%%%%%%%%%%%%%%%%%%%
%%%%%%%%%%%%%%%%%%%%%%%%%%%%%%%%%%%%%%%%%%%%%%%%%%%%%%%%%%%%%%%%%%%%%%%%%%%%%%%
\begin{table}[htbp] 
  \begin{center}
    \renewcommand{\arraystretch}{1.70}
        \begin{tabular}{cccc}
          \toprule
  Class  &  $N_{\rm dof}$ &  Independent DOFs  & DoF in EWPOs\\
          \midrule
          \multirow{4}{*}{four-quark}    &  \multirow{5}{*}{14}
          & $c_{Qq}^{1,8}$, $c_{Qq}^{1,1}$, $c_{Qq}^{3,8}$,   \\
          \multirow{4}{*}{(two-light-two-heavy)}   &
          &  $c_{Qq}^{3,1}$,  $c_{tq}^{8}$,  $c_{tq}^{1}$,  \\
             &    & $c_{tu}^{8}$, $c_{tu}^{1}$, $c_{Qu}^{8}$,\\
            &    & $c_{Qu}^{1}$, $c_{td}^{8}$, $c_{td}^{1}$,   \\
          &    &  $c_{Qd}^{8}$, $c_{Qd}^{1}$   \\
%------------------------------------------------------------------------
          \midrule
                    \multirow{1}{*}{four-quark}      &  \multirow{2}{*}{5}
                    & $c_{QQ}^1$, $c_{QQ}^8$, $c_{Qt}^1$, &   \\
          \multirow{1}{*}{(four-heavy)}      &   & $c_{Qt}^8$, $c_{tt}^1$ &  \\
%-------------------------------------------------------------------------------
\midrule
                    \multirow{1}{*}{four-lepton}      &  \multirow{1}{*}{1}
                    &   &  $c_{\ell\ell}$ \\
	  \midrule
%------------------------------------------------------------------------
      \multirow{4}{*}{two-fermion}     &  \multirow{5}{*}{23} &  $c_{t\varphi}$, $c_{tG}$,   $c_{b\varphi}$,    &  
                     $c_{\varphi \ell_1}^{(1)}$, $c_{\varphi \ell_1}^{(3)}$, $c_{\varphi \ell_2}^{(1)}$ \\
       \multirow{4}{*}{(+ bosonic fields)}      &    & $c_{c\varphi}$, $c_{\tau\varphi}$,   $c_{tW}$,   &
       $c_{\varphi \ell_2}^{(3)}$, $c_{\varphi \ell_3}^{(1)}$, $c_{\varphi \ell_3}^{(3)}$,\\
            &    &  $c_{tZ}$,  $c_{\varphi Q}^{(3)}$, $c_{\varphi Q}^{(-)}$,     &
            $c_{\varphi e}$, $c_{\varphi \mu}$, $c_{\varphi \tau}$,  \\
       &    &  $c_{\varphi t}$     & $c_{\varphi q}^{(3)}$, $c_{\varphi q}^{(-)}$, \\
       &    &       &   ${{c_{\varphi u i}}}$,
        ${{c_{\varphi d i}}}$ \\
       \midrule
%------------------------------------------------------------------------
      \multirow{2}{*}{Purely bosonic}     &  \multirow{2}{*}{7} &
      $c_{\varphi G}$, $c_{\varphi B}$, $c_{\varphi W}$,   &  $c_{\varphi W B}$, $c_{\varphi D}$   \\
               &   &  $c_{\varphi d}$,  $c_{WWW}$   &  \\
            \midrule
          Total  & 50 (36 independent)   & 34   & 16 (2 independent)   \\
         \bottomrule
  \end{tabular}
  \caption{\small \label{tab:operatorbasis} Summary of the degrees of freedom
  considered in the present work.
    %
   We categorize these DoFs into five disjoint classes: four-quark (two-light-two-heavy),
   four-quark (four-heavy), four-lepton, two-fermion, and purely bosonic DoFs.
   %
   The 16 DoFs displayed in the last columns are subject to 14 constraints from the EWPOs,
   leaving only 2 independent combinations to be constrained by the fit.
 }
  \end{center}
\end{table}
%%%%%%%%%%%%%%%%%%%%%%%%%%%%%%%%%%%%%%%%%%%%%%%%%%%%%%%%%%%%%%%%%%%%%%%%%%%%%%%
%%%%%%%%%%%%%%%%%%%%%%%%%%%%%%%%%%%%%%%%%%%%%%%%%%%%%%%%%%%%%%%%%%%%%%%%%%%%%%%

%%%%%%%%%%%%%%%%%%%%%%%%%%%%%%%%%%%%%%%%%%%%%%%%

%%%%%%%%%%%%%%%%%%%%%%%%%%%%%%%%%%%%%
%%%%%%%%%%%%%%%%%%%%%%%%%%%%%%%%%%%%%%%%%%%%%%%%%%%%%%%%%%%%%%%%%%%%%%%%%%%%%%%
%%%%%%%%%%%%%%%%%%%%%%%%%%%%%%%%%%%%%%%%%%%%%%%%%%%%%%%%%%%%%%%%%%%%%%%%%%%%%%%
\begin{table}[htbp] 
  \begin{center}
    \renewcommand{\arraystretch}{1.80}
        \begin{tabular}{ccc}
          \toprule
  Class  &   $\qquad\qquad$ DoF$\qquad\qquad$   & $\qquad \qquad$ Notation $\qquad\qquad$ \\
          \midrule
          \multirow{4}{*}{four-quark}     & $c_{Qq}^{1,8}$, $c_{Qq}^{1,1}$, $c_{Qq}^{3,8}$,
                                            & {\tt c81qq}, {\tt c11qq}, {\tt c83qq},   \\
          \multirow{4}{*}{(two-light-two-heavy)}     & $c_{Qq}^{3,1}$,  $c_{tq}^{8}$,  $c_{tq}^{1}$,
                                                     &  {\tt c13qq}, {\tt c8qt}, {\tt c1qt},  \\
          &  $c_{tu}^{8}$, $c_{tu}^{1}$, $c_{Qu}^{8}$,
          &{\tt c8ut}, {\tt c1ut}, {\tt c8qu},\\
          &  $c_{Qu}^{1}$, $c_{td}^{8}$, $c_{td}^{1}$,
          &{\tt c1qu}, {\tt c8dt}, {\tt c1dt},   \\
              & $c_{Qd}^{8}$, $c_{Qd}^{1}$   & {\tt c8qd}, {\tt c1qd}   \\
%------------------------------------------------------------------------
          \midrule
          \multirow{1}{*}{four-quark}      & $c_{QQ}^1$, $c_{QQ}^8$, $c_{Qt}^1$,
                                              & {\tt cQQ1}, {\tt  cQQ8}, {\tt cQt1},   \\
          \multirow{1}{*}{(four-heavy)}      &  $c_{Qt}^8$, $c_{tt}^1$
                                            & {\tt cQt8}, {\tt ctt1} \\
%------------------------------------------------------------------------
          \midrule
	  four-lepton  & $c_{\ell\ell}$ & {\tt \color{blue} cll}\\
	  \midrule
%------------------------------------------------------------------------
          \multirow{7}{*}{two-fermion}      & $c_{t\varphi}$, $c_{tG}$,   $c_{b\varphi}$,
                                            &  {\tt ctp}, {\tt ctG}, {\tt cbp}, \\
          \multirow{7}{*}{(+ bosonic fields)}&       $c_{c\varphi}$, $c_{\tau\varphi}$,   $c_{tW}$,
                                             &        {\tt ccp}, {\tt ctap}, {\tt  ctW},       \\
          &    $c_{tZ}$,  $c_{\varphi Q}^{(3)}$, $c_{\varphi Q}^{(-)}$,
          &    {\tt ctZ}, {\tt c3pQ3}, {\tt cpQM},            \\
          &     $c_{\varphi t}$, $c_{\varphi \ell_1}^{(1)}$, $c_{\varphi \ell_1}^{(3)}$,
          &     {\tt cpt}, {\tt \color{blue} cpl1},  {\tt \color{blue} c3pl1},          \\
          &   $c_{\varphi \ell_2}^{(1)}$, $c_{\varphi \ell_2}^{(3)}$, $c_{\varphi \ell_3}^{(1)}$,
          &  {\tt \color{blue} cpl2}, {\tt \color{blue} c3pl2},  {\tt \color{blue} cpl3},         \\
          &   $c_{\varphi \ell_3}^{(3)}$, $c_{\varphi e}$, $c_{\varphi \mu}$,
          &  {\tt \color{blue} c3pl3}, {\tt \color{blue} cpe},   {\tt \color{blue} cpmu},    \\
          &  $c_{\varphi \tau}$, $c_{\varphi q}^{(3)}$, $c_{\varphi q}^{(-)}$, 
          &     {\tt \color{blue}cpta},{\tt \color{blue} c3pq}, {\tt \color{blue}cpqMi},          \\
          &     ${ {c_{\varphi u i}}}$,
        ${ {c_{\varphi d i}}}$
          &     {\tt \color{blue}cpui}, {\tt \color{blue} cpdi}           \\
       \midrule
%------------------------------------------------------------------------
       \multirow{3}{*}{Purely bosonic}      & $c_{\varphi G}$, $c_{\varphi B}$, $c_{\varphi W}$,
                                            &   {\tt cpG}, {\tt cpB}, {\tt cpW}, \\
       &   $c_{\varphi d}$, $c_{\varphi W B}$, $c_{\varphi D}$,
       &   {\tt cpd}, {\tt \color{blue} cpWB}, {\tt \color{blue} cpD}, \\    
       &  $c_{WWW}$
       &    {\tt   cWWW}  \\
         \bottomrule
  \end{tabular}
        \caption{The notation that will be used to indicate the EFT coefficients
          listed in  Table~\ref{tab:operatorbasis} in the
          subsequent sections, as well as in the released output files
          with the results of the global analysis.
          %
Only two of the 16 EFT coefficients labelled in blue are independent fit parameters.
\label{tab:notation_coeffs}}
\end{center}
\end{table}
%%%%%%%%%%%%%%%%%%%%%%%%%%%%%%%%%%%%%%%%%%%%%%%%%%%%%%%5

%%%%%%%%%%%%%%%%%%%%%%%%%%%%%%%%%%%%%

\subsection{The top-philic scenario}
\label{sec:topphilic}

The four-fermion operators defined in the previous section and listed in
Table~\ref{eq:summaryOperatorsTop}
correspond to a specific set of assumptions concerning
the flavour structure of the UV-completion of the Standard Model.
%
However, there exist well-motivated BSM scenarios that suggest further
restrictions in the SMEFT parameter space spanned
by these four-fermion operators.
%
Therefore, phenomenological explorations of the SMEFT would benefit from comparing
results obtained in different scenarios concerning the possible UV completion,
from more restrictive to more general.

With this motivation, we have implemented a new
feature in the {\tt SMEFiT} analysis framework
which allows one to implement arbitrary restrictions
in the EFT parameter space, for example those motivated by specific BSM scenarios 
or existing constraints such as those from EWPO, as discussed in the previous section.
%
As a proof of concept, here we will present results
for the top-philic scenario introduced in~\cite{AguilarSaavedra:2018nen}.
%
This  scenario is not constructed by imposing a specific flavour symmetry, but
rather by assuming that new physics couples predominantly to the third-generation left-handed doublet,
the third-generation right-handed up-type quark singlet, the gauge bosons, and the Higgs boson.
%
In other words, that new physics interacts mostly with the top and bottom quarks as well as with
the bosonic sector.
%
The top-philic scenario satisfies the flavour assumptions that we are imposing in this work,
but is based on a more restrictive theoretical assumption.

The restrictions in the EFT parameter space introduced by the top-philic assumption
lead to a number of relations between the DoFs listed in
Table~\ref{tab:operatorbasis}.
%
These relations are the following:
\bea
c_{QDW} &=& c_{Qq}^{3,1}  \, , \nonumber \\
c_{QDB} &=& 6c_{Qq}^{1,1} = \frac{3}{2}c_{Qu}^1 = -3c_{Qd}^1 \, , \nonumber  \\
c_{tDB} &=& 6c_{tq}^1 = \frac{3}{2}c_{tu}^1 = -3c_{td}^1 \, ,  \label{eq:topphilic}\\
c_{QDG} &=& c_{Qq}^{1,8} = c_{Qu}^{8} = c_{Qd}^8 \, , \nonumber \\
c_{tDG} &=& c_{tq}^8 = c_{tu}^8 = c_{td}^8 \, ,\nonumber \\
c_{Qq}^{3,8} &=& 0 \,,\nonumber
\eea
which can be implemented as an additional restriction at the fitting level.
%
Therefore, we now have 9 equations that relate a subset of the 14
two-heavy-two-light degrees of freedom listed in Table~\ref{tab:operatorbasis} among them,
which leave 5 independent two-heavy-two-light  degrees of freedom.
%
The number of operators coupling the top quark with gauge bosons,
as well as that of the four-heavy operators, is not modified.
%
By comparing with Table~\ref{tab:operatorbasis}, we see that in the top-philic
scenario the EFT fit will constrain 41 DoFs, of which 27 are independent.

In principle, the top-philic assumption also implies non-trivial correlations between 
the light-fermion couplings to the gauge and Higgs bosons. However, following our
strategy to include the EWPO, most of them are already set to zero, while the two remaining
degrees of freedom are not affected.
%
The same assumptions also imply that the light fermion
Yukawa operator coefficients are proportional to the SM Yukawa couplings. 
%
As will be shown in Sect.~\ref{sec:results}, imposing the additional relations of the
top-philic scenario leads to more stringent bounds on all the
relevant Wilson coefficients,
due to the fact that the same amount of experimental information is now
used to constrain a significantly more limited parameter space.

\subsection{Cross-section positivity}

The constraint that physical cross-sections are (semi-)positive definite
quantities can also be accounted for in global SMEFT analyses. 
This positivity requirement has different implications depending on whether
the EFT expansion is considered up to either the linear or quadratic level. 

The expansion up to linear terms, $\mathcal{O}(\Lambda^{-2})$, does not automatically
lead to positive-definite cross sections, as in this case the new physics terms are
generated by interference with the SM amplitudes, and their sign and size directly
depend on the Wilson coefficients $c_i$. Imposing the positivity of the cross sections
will therefore set (possibly one-sided) bounds on the Wilson coefficients. 
This can be easily implemented in the fitting procedure if helpful.
%
In fact, we do not find the need to do so, since the fitted experimental data
already leads to positive-definite cross-sections.

The expansion up to quadratic terms, $\mathcal{O}(\Lambda^{-4})$,
i.e. specifically those coming from squaring the 
linearly expanded amplitudes, obviously automatically leads to positive definite cross sections. No constraints
on the Wilson coefficients can therefore be obtained or need to be imposed.
%
On the other hand, verifying
the positivity of the cross section at the quadratic level provides a sanity check that the theoretical calculation  of the various contributions is correctly performed,
also taking into account the MC generation uncertainties.
%
The conditions that have to be  met are simple to obtain. 
%
Consider the  SMEFT Lagrangian
\begin{equation}
\mathcal{L} =   \mathcal{L}_{\rm SM}  + \sum^{ n_{\rm op}}_{i=1} \frac{c_i}{\Lambda^2}\mathcal{O}_i \,,
\end{equation}
where $\mathcal{O}_i$ stand for dimension-six operators and $c_i$ are the
corresponding Wilson coefficients, which we assume to be real. 
%
Any observable calculated using this Lagrangian can be written as a quadratic form
\begin{eqnarray}
\Sigma&=&c_0^2 \Sigma_{00} \nonumber \\
 &+&c_0 c_1 \Sigma_{01} + c_1 c_0 \Sigma_{10}  +  c_0 c_2 \Sigma_{02} +\dots \nonumber \\
 &+&c_1^2 \Sigma_{11} + c_1 c_2 \Sigma_{12} +  c_1 c_3 \Sigma_{13} + \ldots \nonumber \\
&=& {\boldsymbol c^T} \cdot  \boldsymbol{\Sigma} \cdot {\boldsymbol c}. 
\end{eqnarray}
The first line corresponds to the SM contribution, where $c_0$ is an auxiliary coefficient that can be set to unity at the end.  The second line corresponds to the linear ${\cal O}(\Lambda^{-2})$ EFT contributions, while the third line to the ${\cal O}(\Lambda^{-4})$ contributions.
%
$\boldsymbol{\Sigma}$ is by construction a symmetric matrix.\footnote{Note that with respect to
  the convention where $\sigma= \sigma_{\rm SM} + \sum^{n_{\rm op}}_{i=1} c_i \sigma_i  + \sum^{n_{\rm op}}_{i<j} c_i c_j \sigma_{ij}$, one has to account for factors of 2, {\it e.g.}, $\sigma_i=2 \Sigma_{i0}$ and $\sigma_{ij}=2 \Sigma_{ij}$ for $i\neq  j$.}

Given that a physical cross-section must be either positive or null,
the matrix $\boldsymbol{\Sigma}$ must be semi-positive-definite.
%
We can therefore use the Sylvester criterion that states that a symmetric matrix is semi-positive-definite if and only if all principal minors are greater or equal to zero. As a simple example, the constraints coming from the $2\times 2$ minors are:
\begin{equation}
  \label{eq:sylvester}
  \lp \Sigma_{ii}\Sigma_{jj} - \Sigma_{ij}^2 \rp   \ge 0 \, , \qquad i,j=0,\ldots\, n_{\rm op}\,.
\end{equation}
We have verified that the Sylvester criterion, and Eq.~(\ref{eq:sylvester}) in particular,
are satisfied by the EFT calculations used as input to the present analysis.


