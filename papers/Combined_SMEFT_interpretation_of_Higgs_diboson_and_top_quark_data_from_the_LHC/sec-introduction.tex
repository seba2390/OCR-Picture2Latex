\section{Introduction}

A powerful, model-independent framework to constrain, identify,  and parametrise potential
deviations with respect to the predictions of the Standard
Model (SM) is
provided by the Standard Model Effective Field
Theory (SMEFT)~\cite{Weinberg:1979sa,Buchmuller:1985jz,Grzadkowski:2010es}, see also~\cite{Brivio:2017vri}
for a review.
%
A particularly attractive feature of the SMEFT is its capability to systematically correlate
deviations from the SM between different processes, for example between  Higgs 
and top quark cross-sections, or between high-$p_T$ and flavor observables.

A direct consequence of this model independence  is the high dimensionality
of the parameter space spanned by the relevant higher-dimensional EFT operators.
%
Indeed, the number of Wilson coefficients
constrained in  typical SMEFT analyses can vary between
just a few up to the several tens or even hundreds, depending
on the specific  assumptions adopted concerning
the flavour, family (non-)universality of the couplings, and CP-symmetry
structure (among others) of the UV-complete theory.
%
For this reason, the full exploitation of the SMEFT potential
for indirect New Physics searches from precision measurements requires
combining the information provided by the broadest possible dataset.

The phenomenology of the SMEFT has attracted significant
 attention, with most analyses focusing on specific sectors
of the parameter space and groups of processes.
%
Some of these recent studies have targeted the top quark
properties~\cite{Buckley:2016cfg,Buckley:2015lku,Hartland:2019bjb,Brivio:2019ius},
the Higgs and electroweak gauge sector~\cite{Biekotter:2018rhp,Ellis:2018gqa,Almeida:2018cld},
single and double gauge boson
production~\cite{Baglio:2020ibv,Alioli:2018ljm,Ethier:2021ydt,Greljo:2017vvb},
vector-boson scattering~\cite{Gomez-Ambrosio:2018pnl,Ethier:2021ydt,Dedes:2020xmo},
and flavour and low-energy
observables~\cite{Aebischer:2018iyb,Falkowski:2019xoe,Falkowski:2017pss}, among
several others.
%
Furthermore, analyses that combine the constraints of different
groups of processes in the EFT parameter space, such as the Higgs and electroweak sector with
the top quark one~\cite{Ellis:2020unq} or 
top quark data with $B$-meson observables~\cite{Bissmann:2020mfi,Bruggisser:2021duo},
have also been presented.
%
These and related studies demonstrate that a
global interpretation of the SMEFT is unavoidable and makes possible
benefiting from hitherto unexpected connections, such as the  correlation
of the LHCb flavour anomalies~\cite{Pich:2019pzg,Aaij:2021vac}
at the $B$-meson scale with the high-$p_T$ tails
at the LHC~\cite{Greljo:2017vvb,Fuentes-Martin:2020lea}.

With the ultimate motivation of performing a truly global EFT interpretation
of particle physics data, the {\tt SMEFiT} fitting framework was developed in~\cite{Hartland:2019bjb}
and applied to the analysis of the top quark properties at the LHC as a proof-of-concept.
%
This novel EFT fitting methodology, inspired by techniques deployed by the NNPDF
Collaboration to
determine the proton's parton distribution
functions (PDFs)~\cite{Ball:2008by,Ball:2010de,Ball:2014uwa,Rojo:2018qdd,Forte:2020yip},
made possible constraining the Wilson coefficients
associated to 34 independent dimension-six
operators that modify the production cross-sections of top
quarks.
%
Our results improved over
existing bounds~\cite{AguilarSaavedra:2018nen} for the wide majority of
directions in the SMEFT parameter space and in several
cases the associated Wilson coefficients were constrained for the first time.
%
Subsequently, {\tt SMEFiT} was extended with the Bayesian reweighting method~\cite{vanBeek:2019evb}
developed for PDFs~\cite{Ball:2011gg,Ball:2010gb} which  allows one
constraining the EFT parameter space {\it a posteriori}
with novel measurements without requiring a dedicated fit.
%
 {\tt SMEFiT} has also been recently applied for the first SMEFT interpretation of vector boson
 scattering data~\cite{Ethier:2021ydt} from the full Run II dataset.

In this work, we complement and
extend the {\tt SMEFiT} analysis framework of~\cite{Hartland:2019bjb} in several directions.
%
First and foremost, we extend the dimension-six EFT operator basis
in order to simultaneously describe top-quark
measurements together with Higgs boson production and decay cross-sections,
as well as with weak gauge boson pair production from LEP and the LHC.
%
Specifically, we consider Higgs signal strengths, differential distributions,
and simplified template cross-section (STXS) measurements from 
ATLAS and CMS taken at Runs I and II.
%
Furthermore, we account for the most recent top-quark observables
from the Run II dataset, such as updated measurements of four-top,
top quark pair in association with a $Z$ boson, and differential
single-top and top quark pair production.
%
We also include the differential distributions
in gauge boson pair production from LEP
and the LHC, which constrain complementary directions in the EFT space.
 %
In addition, we account in an indirect manner for the information
provided by electroweak precision observables (EWPO) from
LEP~\cite{ALEPH:2005ab}
by means of imposing restrictions on specific combinations
of the EFT coefficients.

A second improvement as compared to~\cite{Hartland:2019bjb}
concerns the fitting methodology.
%
On the one hand, the  Monte Carlo replica fitting method has been
upgraded by means of more efficient optimizers and the imposition of
post-fit quality selection criteria for the replicas.
%
On the other hand, we have implemented a novel, independent
approach to constrain the  parameter space based on
Nested Sampling (NS) by means of the MultiNest algorithm~\cite{Feroz:2013hea}.
%
As opposed to the replica fitting method, which is an
optimisation problem,
NS aims to reconstruct the posterior probability distribution given
the model and the data by means of Bayesian inference.
%
We have cross-validated the performance
of  the two methods and demonstrated that they
lead to equivalent results.
%
The availability of two orthogonal fitting strategies
strengthens the robustness of {\tt SMEFiT} and facilitates the
combined interpretation of data from different processes.

From the combination of the  improved fitting framework
and the extensive input dataset, we 
derive individual, two-dimensional, and global (marginalised) bounds for  36
independent directions (and 14 dependent ones) in the EFT parameter space.
%
The EFT cross-sections used in this analysis account for
either only the linear
or for both linear and quadratic effects, $\mathcal{O}\lp \Lambda^{-2}\rp$ and
$\mathcal{O}\lp \Lambda^{-4}\rp$ respectively, and
include NLO QCD corrections whenever available.
%
We demonstrate in detail how the inclusion of NLO QCD
and $\mathcal{O}\lp \Lambda^{-4}\rp$ corrections
in the EFT calculations
is instrumental in order to accurately pin down the  posterior distributions associated
to the fitted Wilson coefficients.

By means of information geometry and principal component analysis techniques,
we quantify the sensitivity of each
of the input datasets to the various Wilson coefficients.
%
We validate these statistical diagnosis tools by means of a series of  fits restricted
to subsets of processes, such as Higgs-only and top-only EFT analyses.
%
Specifically, we quantify the interplay between the top-quark and Higgs measurements
in the determination of EFT degrees of freedom sensitive to both processes,
such as the modifications of the top Yukawa coupling.
%
Furthermore, we explore how the EFT fit results are modified when additional, UV-inspired theory restrictions
are imposed in the parameter space, and present results
for the case of a top-philic model.

The paper is organised as follows.
%
First of all, Sect.~\ref{sec:smefttheory} discusses the operator basis,
flavour assumptions, the fitted degrees of freedom,
and the top-philic scenario.
%
Then Sect.~\ref{sec:settings_expdata} describes the top-quark, Higgs,
and diboson datasets that are used as input to the analysis together
with the corresponding SM and EFT calculations.
%
The methodological improvements in {\tt SMEFiT}, together with the description of  the fit settings,
are presented in Sect.~\ref{sec:fitsettings}.
%
The main results of this work, namely the combined SMEFT
interpretation of top-quark, Higgs, and diboson measurements at the LHC,
are presented and discussed in Sect.~\ref{sec:results}.
%
Finally, in Sect.~\ref{sec:summary} we summarise and discuss future steps
in this project.

Supplementary information is provided in three appendices.
%
In App.~\ref{sect:app_comparison_data} we present
the comparison between the SM and SMEFT theory predictions
with the experimental datasets used as input to
the fit;
in App.~\ref{sec:signalstrenghts} we describe the implementation
of  the Higgs signal strength measurements;
{  in App.~\ref{sec:fullcovmat} we present the 
  correlation matrices for the complete set of operators
considered in the analysis; and}
%
then
in App.~\ref{sec:delivery} we discuss how
the results of this work are rendered publicly available
and provide usage instruction.
