%%%%%%%%%%%%%%%%%%%%%%%%%%%%%%%%%%%%%%%%%%%%%%%%%%%%%%%%%%%%%%%%%%%%%%%%%%%%%%%%%%%%%%%
\begin{table}[t]
  \centering
  \small
   \renewcommand{\arraystretch}{1.30}
  \begin{tabular}{C{3.9cm}|C{6.6cm}|C{2cm}}
 Category   & Processes    &  $n_{\rm dat}$     \\
    \toprule
    \multirow{6}{*}{Top quark production}   &  $t\bar{t}$ (inclusive)   &  94  \\
    &  $t\bar{t}Z$, $t\bar{t}W$    & 14 \\
    &   single top (inclusive)   & 27 \\
    &  $tZ, tW$   &  9\\
    &  $t\bar{t}t\bar{t}$, $t\bar{t}b\bar{b}$    & 6 \\
    &  {\bf Total}    & {\bf 150 }  \\
    \midrule
    \multirow{3.3}{*}{Higgs production} & Run I signal strengths  &22   \\
    \multirow{3.1}{*}{and decay} & Run II  signal strengths  & 40  \\
    & Run II, differential distributions \& STXS  & 35  \\
    &  {\bf Total}    & {\bf 97}  \\
    \midrule
    \multirow{3}{*}{Diboson production} & LEP-2 &40   \\
     & LHC & 30  \\
    &  {\bf Total}    & {\bf 70}  \\
    \bottomrule
   Baseline dataset     & {\bf Total}      & {\bf 317}  \\
\bottomrule
  \end{tabular}
  \caption{\small The number of data points $n_{\rm dat}$ in our baseline dataset
    for each of the categories of processes considered here.
 \label{eq:table_dataset_overview}
}
\end{table}
%%%%%%%%%%%%%%%%%%%%%%%%%%%%%%%%%%%%%%%%%%%%%%%%%%%%%%%%%%%%%%%%%%%%%%%%%%%%%%%%


