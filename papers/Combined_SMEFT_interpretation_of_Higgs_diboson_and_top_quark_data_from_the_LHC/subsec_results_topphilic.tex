\subsection{The top-philic scenario}

To conclude this section, we present results for a global EFT fit
carried out in the top-philic scenario defined in Sect.~\ref{sec:topphilic}.
%
In this scenario, we have the 9 equations of Eq.~(\ref{eq:topphilic}) that 
relate a subset of the 14
two-heavy-two-light coefficients listed in Table~\ref{tab:operatorbasis}
among them,  leaving 5 independent parameters to be constrained
in the fit.
%
Given the more constraining assumptions associated
to the top-philic scenario, one expects to find an improvement
in the bounds of the two-light-two-heavy EFT operators due to the fact
that the parameter space is being restricted by theoretical
considerations, rather than by data in this case.

The values of the $\chi^2$ for each group of datasets in the top-philic scenario
were reported in Table~\ref{eq:chi2-theoryvariations},
where we see that the fit quality is very similar to the fit with the baseline settings.
%
Fig.~\ref{fig:Coeffs_Bar_TopPhilic} then displays the 95\% CL
intervals for the EFT coefficients comparing the
global fit results with those of the top-philic scenario.
%
The only operators that are affected in a significant manner
turn out to be the two-light-two-heavy
operators, with the bounds in several of them such as
$c_{td}^1$, $c_{Qq}^{1,1}$, and $c_{tq}^1$  improving by almost an order of magnitude.
%
The fact that only the bounds on the two-light-two-heavy operators are modified
is consistent with the top-philic scenario, given that only
this specific group of EFT coefficients is being constrained by its model assumptions.

%%%%%%%%%%%%%%%%%%%%%%%%%%%%%%%%%%%%%%%%%%%%%%%%%%%%%%%%%%%%%%%%%%%%%
\begin{figure}[t]
  \begin{center}
     \includegraphics[width=0.99\linewidth]{plots_v2/Coeffs_Bar_topphilic.pdf}
     \caption{\small Same as Fig.~\ref{fig:posterior_coeffs}
       comparing the global fit results with the same fit in the top-philic
       scenario defined by the relations in Eq.~(\ref{eq:topphilic}).
     \label{fig:Coeffs_Bar_TopPhilic} }
  \end{center}
\end{figure}
%%%%%%%%%%%%%%%%%%%%%%%%%%%%%%%%%%%%%%%%%%%%%%%%%%%%%%%%%%%%%%%%%%%%%%

It is worth emphasizing at this point that,
from the technical point of view, carrying out global EFT fits with specific restrictions
in the parameter space motivated by UV-completions, such as those arising
in the top-philic scenario and leading to Fig.~\ref{fig:Coeffs_Bar_TopPhilic},
is relatively straightforward.
%
Indeed, the most efficient fitting strategy would be to start from the broadest possible
parameter space, and once the corresponding fit has been performed,
introduce model assumptions relating EFT coefficients  in a systematic manner.
%
This way one can connect with specific models for  UV-completions of the SM,
which typically result in a rather smaller number of EFT coefficients to be constrained
from data.

