\clearpage
\subsection{Impact of NLO QCD corrections in the EFT cross-sections}
\label{subsec:loqcd}

In addition to the choice of input dataset, another important
factor that determines the outcome of a global analysis
such as the present one is the  accuracy
of the EFT theoretical calculations.
%
Here we assess the role played at the level of the fit results
by the inclusion of NLO QCD corrections to the EFT cross-sections,
both in the linear and in the quadratic fits.
%
As indicated in Table~\ref{eq:table-processes-theory},
our baseline fit includes these NLO corrections
to the EFT calculations whenever available, so now we switch them off deliberately
to quantify how much they affect the fit outcome.\footnote{This study
  is also motivated by the fact that many EFT fits
rely on LO QCD for the EFT cross-sections.}
%
In the following, the theoretical predictions
for the SM cross-sections, based on the state-of-the-art calculations,
remain unchanged, and only the EFT ones are modified as compared to the baseline
settings.
%
First of all, Table~\ref{eq:chi2-theoryvariations}
compares the values of the $\chi^2$ for the various groups of processes
in quadratic fits with and without NLO QCD corrections to  the  EFT cross-sections,
as well as for the associated SM results.
%
One can observe how the overall fit quality
is similar whether or not NLO QCD effects are not accounted for.
%
Nevertheless, as will be discussed next, this does
not imply that the fit posterior distributions
are likewise unchanged.

%%%%%%%%%%%%%%%%%%%%%%%%%%%%%%%%%%%%%%%%%%%%%%%%%%%
%%%%%%%%%%%%%%%%%%%%%%%%%%%%%%%%%%%%%%%%%%%%%%%%%%%%%%%%%%%%%%%%%%%%%%%%%%%%%%%%%%%%%%%
\begin{table}[t]
%\begin{table}[t]
  \centering
  \footnotesize
   \renewcommand{\arraystretch}{1.40}
  \begin{tabular}{l|C{1.0cm}|C{1.2cm}|C{2.0cm}|C{2.7cm}|C{2.2cm}}
    \multirow{2}{*}{Dataset}   & \multirow{2}{*}{$ n_{\rm dat}$} & $\chi^2_{\rm SM}$ &  $\chi^2_{\rm EFT}$   & $\chi^2_{\rm EFT}$    & $\chi^2_{\rm EFT}$    \\
      &   &  &  (baseline)  &  (LO QCD in EFT)  & (top-philic)   \\
 \toprule
%-------------------------------------------------------------------------------
$t\bar{t}$ inclusive        &  83    &  1.46  &  1.42      &   1.39     &   1.41      \\
%-------------------------------------------------------------------------------
$t\bar{t}$ AC                & 11    &  0.60  &  0.59      &   0.57     &   0.60      \\
%-------------------------------------------------------------------------------
$t\bar{t}V$                 &  14    &  0.65  &   0.65     &   0.54     &  0.68     \\
%-------------------------------------------------------------------------------
single top inclusive        &  27    &  0.43  &   0.41     &   0.42     &  0.41      \\
%--------------------------------------------------------------------------------------------
$tV$                        &  9     &  0.71  &   0.75     &   0.68     &  0.78      \\
%--------------------------------------------------------------------------------------------
 $t\bar{t}Q\bar{Q}$         &  6     & 1.68  &  2.12       &   2.24     &  2.16       \\
%-------------------------------------------------------------------------------
{\bf Top quark total}       &  150   & 1.10  &  1.09       &   1.06     &  1.09      \\
\midrule
%-------------------------------------------------------------------------------
Higgs $\mu_i^f$  (Run I)    &  22    & 0.86  &  0.90       &  0.95     &  0.90       \\
%-------------------------------------------------------------------------------
Higgs $\mu_i^f$  (Run II)   &   40   & 0.67  &  0.63       &  0.67     &  0.63       \\
%-------------------------------------------------------------------------------
Higgs differential \& STXS  &  35    & 0.88  &  0.83       &  0.78     &  0.83       \\
%-------------------------------------------------------------------------------
{\bf Higgs  total}          &   97   & 0.78  &  0.76       &  0.77     &  0.76     \\
\midrule
{\bf Diboson}               &  70    & 1.31  &  1.30       &   1.32    &   1.30      \\
%-------------------------------------------------------------------------------    
\midrule
{\bf Global dataset}        & {\bf 317}   & {\bf 1.05}  &  {\bf 1.04}   &   {\bf 1.03}     &   {\bf 1.04}      \\
%-------------------------------------------------------------------------------
\bottomrule
\end{tabular}
  \caption{\small Same as Table~\ref{eq:chi2-baseline-grouped} now for fits based on
    variations of the theory settings as compared to the baseline ones.
    %
    Specifically, we provide the results of a fit where the EFT cross-sections are evaluated at LO in the QCD
    expansion, as well as those of the top-philic scenario where the  parameter space
    has been restricted as described in Sect.~\ref{sec:topphilic}.
    %
    In both cases, quadratic EFT corrections are being included.
    %
    Note that the SM cross-sections are always evaluated using state-of-the-art theory
    calculations.
\label{eq:chi2-theoryvariations}
}
\end{table}
%%%%%%%%%%%%%%%%%%%%%%%%%%%%%%%%%%%%%%%%%%%%%%%%%%%%%%%%%%%%%%%%%%%%%%%%%%%%%%%%


%%%%%%%%%%%%%%%%%%%%%%%%%%%%%%%%%%%%%%%%%%%%%%%%%%%s

%%%%%%%%%%%%%%%%%%%%%%%%%%%%%%%%%%%%%%%%%%%%%%%%%%%%%%%%%%%%%%%%%%%%%
\begin{figure}[htbp]
  \begin{center}
    \includegraphics[width=0.99\linewidth]{plots_v2/Coeffs_Hist_LOvsNLO_lin.pdf}
    \includegraphics[width=0.99\linewidth]{plots_v2/Coeffs_Central_LOvsNLO_lin.pdf}
    \caption{\small Top: comparison of the posterior probability
      distributions of the Wilson coefficients
      between linear fits with and without NLO QCD corrections to the EFT cross-sections.
      %
      Bottom: the corresponding 95\% CL intervals, compared to the SM expectation.
     \label{fig:posterior-distributions-NLO-vs-LO-linear} }
  \end{center}
\end{figure}
%%%%%%%%%%%%%%%%%%%%%%%%%%%%%%%%%%%%%%%%%%%%%%%%%%%%%%%%%%%%%%%%%%%%%%xs

%%%%%%%%%%%%%%%%%%%%%%%%%%%%%%%%%%%%%%%%%%%%%%%%%%%%%%%%%%%%%%%%%%%%%
\begin{figure}[htbp]
  \begin{center}
    \includegraphics[width=0.99\linewidth]{plots_v2/Coeffs_Hist_LOvsNLO_quad.pdf}
    \includegraphics[width=0.99\linewidth]{plots_v2/Coeffs_Central_LOvsNLO_quad.pdf}
    \caption{\small
      Same as Fig.~\ref{fig:posterior-distributions-NLO-vs-LO-linear}
      for the quadratic EFT fits.
      \label{fig:posterior-distributions-NLO-vs-LO-quad} }
  \end{center}
\end{figure}
%%%%%%%%%%%%%%%%%%%%%%%%%%%%%%%%%%%%%%%%%%%%%%%%%%%%%%%%%%%%%%%%%%%%%%xs

Figs.~\ref{fig:posterior-distributions-NLO-vs-LO-linear}
and~\ref{fig:posterior-distributions-NLO-vs-LO-quad} then display the
posterior probability distributions 
and the corresponding 95\% CL intervals for the Wilson coefficients,
comparing the results of linear and quadratic fits respectively
with and without NLO corrections to the EFT cross-sections.
%
Scrutinizing first the linear fit results collected in
Fig.~\ref{fig:posterior-distributions-NLO-vs-LO-linear},
one can observe that these posterior distributions can be
severely distorted when LO EFT calculations are used as compared
to the baseline,
for instance in terms of a shift in the best-fit values and/or due
to an increase in the width of the Gaussian distributions.
%
Also in the LO linear fit, all considered coefficients
agree with the SM expectation at the 95\% CL.
%
Note that the two-light-two-heavy singlet operators do not
interfere with the SM at LO,
and hence the corresponding coefficients turn out to be unconstrained
in the linear LO fit.
%
Remarkably, for several fit coefficients
such as $c_{tZ}$, $c_{\varphi B}$, and $c_{W}$, one finds that a marked
improvement in the obtained bounds is achieved upon the inclusion
of the NLO QCD corrections to the EFT cross-sections.
%
One would conclude that, at least in the global linear EFT fit,
the inclusion of NLO QCD corrections
is of clear importance to obtain both more accurate and more precise results
for the Wilson coefficients.
%
Alternatively, one could account for the missing
higher-order uncertainties (MHOUs) in the EFT cross-sections, which
are usually neglected, using for instance the approach advocated
in~\cite{AbdulKhalek:2019bux,AbdulKhalek:2019ihb}.
%
Implementing MHOUs systematically is expected to further improve
the overall compatibility of EFT fits performed with and
without NLO QCD corrections.

Moving to the associated comparisons in the case of the quadratic fits summarised in
Fig.~\ref{fig:posterior-distributions-NLO-vs-LO-quad},
also here we find that the parameter distributions can be modified in a marked
way depending on whether or not NLO QCD calculations are adopted.
%
As an illustration, the operator that modifies the charm Yukawa interaction,
$c_{c\varphi}$, exhibits
a bimodal distribution once NLO effects are accounted for,
while the dominant solution for the $c_{\varphi t}$  coefficient
is far from the SM in the LO fit but SM-like in the NLO case
(though the 95\% CL interval itself remains stable).
%
As opposed to the case of the linear fits,
in the quadratic case one finds that the addition of NLO corrections
does not in general reduce the uncertainties
on the fit coefficients, but rather distorts
the posterior distributions and shifts the central values.
%
As an illustration,
if NLO QCD corrections are removed, the posterior
distribution for the two-light-two-heavy coefficient
$c_{Qq}^{3,8}$ is shifted such that it does not agree anymore
with the SM at the 95\% CL.


In the specific case of the $c_{\varphi t}$  coefficient, one can verify that the corresponding individual $\chi^2$ profile
(analog of Fig.~\ref{fig:quartic-individual-fits-2} for LO fits) does not
exhibit this second solution, and hence it must be induced by the cross-talk
with other coefficients in the fit.
%
To validate this hypothesis, Fig.~\ref{fig:Ellipse_Opt_OtZ_LO_HO} displays the outcome of 
 two-parameter quadratic fits for
     $(c_{\varphi t},c_{tZ})$  and $(c_{\varphi t},c_{\varphi W})$ 
 comparing the results of the LO EFT fit  with its NLO counterpart.
 %
 In both cases, the LO two-parameter fits based on the full dataset
 favour the solution far from the SM, while the NLO ones
 instead favour the SM-like one.
 %
 The explanation for this behaviour can be traced back
 to the fact that the non-SM solution is disfavored
 by the NLO EFT corrections to $hZ$ associated production,
 in particular those related to gluon-induced contributions.

%%%%%%%%%%%%%%%%%%%%%%%%%%%%%%%%%%%%%%%%%%%%%%%%%%%%%%%%%%%%%%%%%%%%%
\begin{figure}[t]
  \begin{center}
    \includegraphics[width=0.41\linewidth]{plots_v2/Ellipse_Opt_OtZ_LO_HO.pdf}
    \includegraphics[width=0.41\linewidth]{plots_v2/Ellipse_Opt_OtZ_NLO_HO.pdf}
    \includegraphics[width=0.41\linewidth]{plots_v2/Ellipse_Opt_OpW_LO_HO.pdf}
    \includegraphics[width=0.41\linewidth]{plots_v2/Ellipse_Opt_OpW_NLO_HO.pdf}
    \caption{\small
      Same as Fig.~\ref{fig:2Dfits} for the two-parameter quadratic fits
      of $(c_{\varphi t},c_{tZ})$ (upper) and $(c_{\varphi t},c_{\varphi W})$ (lower panels)
      comparing the results of the LO EFT fit (left) with its NLO counterpart (right panels)
      \label{fig:Ellipse_Opt_OtZ_LO_HO} }
  \end{center}
\end{figure}
%%%%%%%%%%%%%%%%%%%%%%%%%%%%%%%%%%%%%%%%%%%%%%%%%%%%%%%%%%%%%%%%%%%%%%xs

Another remarkable effect of the  NLO QCD corrections
to the EFT cross-sections can be observed in the modified
correlation patterns.
%
Fig.~\ref{fig:Coeffs_Corr_260121-NS_GLOBAL_LO} displays
the same correlations maps as in Fig.~\ref{fig:globalfit-correlations} now
for global fits based
on LO EFT calculations at  the linear and quadratic level.
%
Specially for the linear fits, we observe that correlations
become more sizable in general
for the two-fermion and purely bosonic operators,
while these are
reduced once NLO corrections are accounted for.
%
This feature demonstrates how NLO QCD effects may reduce parameter correlations
by introducing additional sensitivity to the fit coefficients for the same input dataset.

%%%%%%%%%%%%%%%%%%%%%%%%%%%%%%%%%%%%%%%%%%%%%%%%%%%%%%%%%%%%%%%%%%%%%
\begin{figure}[t]
  \begin{center}
    \includegraphics[width=0.49\linewidth]{plots_v2/Coeffs_Corr_NS_GLOBAL_LO_NHO.pdf}
    \includegraphics[width=0.49\linewidth]{plots_v2/Coeffs_Corr_NS_GLOBAL_LO_HO.pdf}
     \caption{\small
       Same as Fig.~\ref{fig:globalfit-correlations} for
       LO EFT calculations in linear (left) and quadratic (right) fits.
      \label{fig:Coeffs_Corr_260121-NS_GLOBAL_LO} }
  \end{center}
\end{figure}
%%%%%%%%%%%%%%%%%%%%%%%%%%%%%%%%%%%%%%%%%%%%%%%%%%%%%%%%%%%%%%%%%%%%%%xs
