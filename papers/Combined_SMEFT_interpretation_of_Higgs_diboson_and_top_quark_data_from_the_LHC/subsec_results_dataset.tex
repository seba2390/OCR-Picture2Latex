\subsection{Dataset dependence}
\label{sec:dataset_dependence}

The discussion so far has focused on the output of the global fits
obtained for the baseline dataset summarised in
Tables~\ref{eq:input_datasets} to~\ref{eq:input_datasets4}.
%
Here we aim to assess the dependence of these results with respect to the choice
of input dataset.
%
With this purpose, we consider here fits for the following variations:
\begin{itemize}

\item A fit which includes only top quark measurements.
  %
  This fit makes possible quantifying the interplay
  between the top and the Higgs data in the global fit.

\item A fit which includes only Higgs boson production and decay data,
  which provides complementary information as compared to
  the top-only fit.

\item  A fit which includes only top quark measurements, but now restricted to
  the same dataset as in our original study from~\cite{Hartland:2019bjb}.
  %
  This comparison allows one to assess the impact
  in the top-only EFT fit of the new LHC top quark measurements that have
  become available in the last two years.

\item A fit where the diboson data is removed, to determine how much weight 
  the diboson cross-sections carry in the global fit results.

\item A fit where all high-energy bins, defined as those bins
  probing the region $E\gsim 1 $ TeV, are removed.
  %
  The motivation for such a fit is to study how important are the constraints
  provided by the high-energy region in the global fit results,
  which in turn is an important input to  establish the validity
of the EFT approximation.

\item A fit where those datasets displaying poor agreement with the SM cross-sections
  are removed.
  %
  Specifically, here one removes the datasets whose $\chi^2$ differs by more
than $3\sigma$ from their statistical expectation assuming the SM hypothesis.
%
While such disagreements between data and SM theory
could very well indicate hints of BSM physics, they can also be explained
by for example issues with the  experimental correlation models.
%
Hence, this fit  allows us to verify to which extent the baseline results
are driven from the datasets that disagree the most with the SM predictions.
%
{  The datasets indicated with {\bf (*)} in Tables~\ref{eq:chi2-baseline} and~\ref{eq:chi2-baseline2} are those
  excluded from this ``conservative'' EFT fit.}

  
\end{itemize}
Note that, as explained in Sect.~\ref{sec:settings_expdata}, for the purposes
of categorisation into datasets
the $t\bar{t}h$ cross-sections are considered part of the Higgs measurements.
%
Furthermore, we note that all these fits are based on  quadratic EFT calculations
and that the constraints provided by the EWPOs on
the EFT parameter space are always accounted for.

%%%%%%%%%%%%%%%%%%%%%%%%%%%%%%%%%%%%%%%%%%%%%%%%%%%
%%%%%%%%%%%%%%%%%%%%%%%%%%%%%%%%%%%%%%%%%%%%%%%%%%%%%%%%%%%%%%%%%%%%%%%%%%%%%%%%%%%%%%%
\begin{table}[t]
  \centering
  \tiny
   \renewcommand{\arraystretch}{1.90}
  \begin{tabular}{l|C{0.8cm}|C{0.7cm}|C{1.0cm}|C{1.0cm}|C{1.0cm}|C{1.5cm}|C{0.9cm}|C{1.2cm}|C{1.2cm}}
    \multirow{3}{*}{Dataset}   & \multirow{3}{*}{$ n_{\rm dat}$} & \multirow{3}{*}{$\chi^2_{\rm SM}$} &  \multicolumn{6}{c}{$\chi^2_{\rm eft}$}   \\\cmidrule(lr{0.7em}){4-10}
    &   &  &  baseline  &  top-only  & top-only & Higgs-only &   diboson  & high-$E$   &  poor $\chi^2_{\rm sm}$  \\
       &   &  &    &  (2021)  & (2018) & &  excluded  & excluded  &  excluded  \\
 \toprule
%-------------------------------------------------------------------------------
$t\bar{t}$ incl.        &  83   & 1.46  & 1.42  &  1.44 & 1.52~(63) & --- & 1.42   &  1.40~(67)   &  0.95~(67)   \\
%-------------------------------------------------------------------------------
$t\bar{t}$ charge asym.           &  11   & 0.60  & 0.59  &  0.58 & ---       & --- & 0.60   &  0.58   &  0.56   \\
%-------------------------------------------------------------------------------
$t\bar{t}V$             &  14   &  0.65 & 0.65  & 0.69  & 0.64~(8)  & --- & 0.65   & 0.72    &   0.68     \\
%-------------------------------------------------------------------------------
single-$t$ incl.        &  27   & 0.43  & 0.41  &  0.40 & 0.36~(22) & --- & 0.41   & 0.41    &  0.46  \\
%--------------------------------------------------------------------------------------------
$tV$                    &  9    & 0.71  & 0.75  &  0.65 & 0.76~(6)  & --- & 0.75   & 0.80    &  0.31~(8)   \\
%--------------------------------------------------------------------------------------------
 $t\bar{t}Q\bar{Q}$     &  6    & 1.68  & 2.12  &  2.29 &  4.73~(2) & --- & 2.12   & 2.40    &  1.54~(4)   \\
%-------------------------------------------------------------------------------
{\bf Top total}         & 150   & 1.10  & 1.09  &  1.10 & 1.22~(101)& --- & 1.09   & 1.06~(134) &  0.82~(123)   \\
\midrule
%-------------------------------------------------------------------------------
Higgs $\mu_i^f$ (RI)    &  22   & 0.86  & 0.90  &  ---  &  ---      &   0.90  &  0.89   & 0.89  &  0.89     \\
%-------------------------------------------------------------------------------
Higgs $\mu_i^f$ (RII)   &  40   & 0.67  & 0.63  &  ---  &  ---      &   0.63  &  0.62   & 0.63  &  0.62    \\
%-------------------------------------------------------------------------------
Higgs  STXS             &  35   & 0.88  & 0.83  &  ---  &  ---      &   0.82  &  0.83   & 0.83  &  0.83   \\
%-------------------------------------------------------------------------------
{\bf Higgs  total}      &  97   & 0.78  & 0.76  &  ---  &  ---      &   0.76   & 0.76   & 0.76  &  0.76   \\
\midrule
{\bf Diboson}           &  70   & 1.31  & 1.30  &  ---  &  ---      &  ---  &   ---  &  1.31   &  1.30    \\
%-------------------------------------------------------------------------------    
\bottomrule
    {\bf Total $n_{\rm dat}$}    & 317    &  317   &  317   &  150  & 101  & 97   &247  & 301  & 287    \\
    {\bf Total $\chi^2$}    & ---    &  {\bf 1.05}    &  {\bf 1.04}  &  {\bf 1.10}  &  {\bf 1.22}  &
    {\bf 0.75}  &  {\bf 0.96} &  {\bf 1.02}   & {\bf 0.89}    \\ 
%-------------------------------------------------------------------------------
\bottomrule
\end{tabular}
  \caption{\small Same as Table~\ref{eq:chi2-baseline-grouped}  for EFT fits obtained from
    variations of the baseline dataset.
    %
    We list the results of the following fits: including only  top quark measurements (either for the 2018 or the current
    dataset); a Higgs-only dataset; without the diboson cross-sections; with the high-energy bins excluded;
    and with the datasets with a poor $\chi^2_{\rm sm}$ excluded.
    %
    In all cases, the  quadratic EFT corrections are accounted for.
    %
    The numbers in parentheses indicate the number of data points, in the case that these are different
    from those of the baseline settings (listed in the second column).
\label{eq:chi2-datasetvariations}
}
\end{table}
%%%%%%%%%%%%%%%%%%%%%%%%%%%%%%%%%%%%%%%%%%%%%%%%%%%%%%%%%%%%%%%%%%%%%%%%%%%%%%%%


%%%%%%%%%%%%%%%%%%%%%%%%%%%%%%%%%%%%%%%%%%%%%%%%%%%

To begin with,  Table~\ref{eq:chi2-datasetvariations}
collects the values of the $\chi^2$ per data points for EFT fits obtained from
variations of the input dataset.
%
We list the results of the various fits described above:
including only  top quark measurements (either from the current or the 2018
dataset); with a Higgs-only dataset; without the diboson cross-sections; with the high-energy bins excluded;
and with the datasets with a poor $\chi^2_{\rm sm}$ excluded.
%
The numbers in parentheses indicate the number of data points, in the case that these are different
from those of the baseline settings listed in the second column.
%
We observe how the description of the Higgs cross-sections is essentially
unaffected in these fits with reduced datasets.
%
Concerning the total $\chi^2$ for the top data, we see that it is stable in the fit
where the high-energy bins are removed, but that is markedly improved (from 1.10 to 0.82)
in the fit where the datasets with poor $\chi^2_{\rm sm}$ are excluded
and the number of top-quark points in the fit decreases from $n_{\rm dat}=150$
to 123.

Then in Fig.~\ref{fig:global_vs_toponly} we compare the magnitude of the 95\% CL bounds,
same as in the upper panel of Fig.~\ref{fig:globalfit-baseline-bounds-lin-vs-quad},
between the global fit results with those obtained in the top-only 
and Higgs-only fits.
%
As mentioned above, these fits
allow us to  assess the interplay
between the top and the Higgs data in the global analysis, in other words,
to identify what are the main benefits of the simultaneous mapping of the EFT parameter space
as compared to carrying out separate fits to each group of processes.
%
First of all, we note that the global fit bounds are more stringent for
all the EFT coefficients than in either the top-only or Higgs-only fit, highlighting the overall
consistency of the two datasets.
%
Secondly, the cross-talk of the top and Higgs data is found to be most
relevant for the two-fermion coefficients $c_{\varphi t}$
and $c_{\varphi Q}^{(-)}$, whose bounds are improved
by around a factor 2 in the global fit as compared to the top-only fit.
%
Another operator that benefits from the global fit is
$c_{\varphi G}$, which is unconstrained in the top-only fit but
whose bound in the global fit is clearly improved as compared to the Higgs-only fit.
%
These comparisons show how by breaking degeneracies one gains information in the global fit as compared to the partial ones, sometimes in unexpected directions in the parameter space such as for $c_{\varphi G}$  in this case.
%
The bottom panel of Fig.~\ref{fig:global_vs_toponly} also indicates that in a Higgs-only fit
a large number of EFT coefficients are poorly constrained, in particular
those involving fermion bilinears.

%%%%%%%%%%%%%%%%%%%%%%%%%%%%%%%%%%%%%%%%%%%%%%%%%%%%%%%%%%%%%%%%%%%%%
\begin{figure}[t]
  \begin{center}
    \includegraphics[width=0.90\linewidth]{plots_v2/Coeffs_Bar_Top_only.pdf}
    \includegraphics[width=0.90\linewidth]{plots_v2/Coeffs_Bar_Higgs_only.pdf}
    \caption{\label{fig:global_vs_toponly} \small
      Same as upper panel of Fig.~\ref{fig:globalfit-baseline-bounds-lin-vs-quad}
      now comparing the global fit results with those obtained in a top-only (upper)
    and Higgs-only (lower panel) fits.}
  \end{center}
\end{figure}
%%%%%%%%%%%%%%%%%%%%%%%%%%%%%%%%%%%%%%%%%%%%%%%%%%%%%%%%%%%%%%%%%%%%%%

%%%%%%%%%%%%%%%%%%%%%%%%%%%%%%%%%%%%%%%%%%%%%%%%%%%%%%%%%%%%%%%%%%%%%
\begin{figure}[t]
  \begin{center}
    \includegraphics[width=0.90\linewidth]{plots_v2/Coeffs_Bar_noVV.pdf}
   \includegraphics[width=0.90\linewidth]{plots_v2/Coeffs_Bar_top19_top21.pdf}
    \caption{\label{fig:global_vs_top2019} \small
      Same as Fig.~\ref{fig:global_vs_toponly},
      now comparing the global fit with a no-diboson fit (upper)
    and the two top-only fits with different datasets (lower panel).}
  \end{center}
\end{figure}
%%%%%%%%%%%%%%%%%%%%%%%%%%%%%%%%%%%%%%%%%%%%%%%%%%%%%%%%%%%%%%%%%%%%%%

Next, Fig.~\ref{fig:global_vs_top2019} displays a
similar comparison as in Fig.~\ref{fig:global_vs_toponly}
now comparing first the outcome of the global fit with that of a fit
where the diboson cross-sections have been removed,
and second comparing two top-only fits, namely the fit displayed
in the upper panel of Fig.~\ref{fig:global_vs_toponly} with  a
fit based on the same dataset as our previous study from~\cite{Hartland:2019bjb}.
%
The fit without diboson data demonstrates that the constraints provided by the diboson
cross-sections are negligible in comparison with those provided by the Higgs data
(and the EWPOs) for all coefficients considered in the fit, except for the triple
gauge operator $c_{W}$.
%
This result is consistent with the Fisher information
analysis of Fig.~\ref{fig:FisherMatrix}, and indicates that, apart from $c_W$, the diboson
data does not provide competitive information on the EFT parameter space in the context
of a global fit.

The comparison of the two top-only fits in the bottom
panel of Fig.~\ref{fig:global_vs_top2019} illustrates how for all coefficients
the bounds  are improved thanks to the more recent LHC measurements.\footnote{Recall that now we consider
  the top Yukawa coefficient $c_{t\varphi}$ as part of the Higgs dataset.}
%
The improvement is consistent across the board, quantifies the additional information
brought in by the new top-quark cross-section measurements
(see Table~\ref{eq:chi2-datasetvariations}),
and confirms that the broader and more diverse the input dataset is,
 the more stringent the resulting constraints
on the EFT parameter space that will be obtained.

%%%%%%%%%%%%%%%%%%%%%%%%%%%%%%%%%%%%%%%%%%%%%%%%%%%%%%%%%%%%%%%%%%%%%
\begin{figure}[t]
  \begin{center}
    \includegraphics[width=0.80\linewidth]{plots_v2/Coeffs_Central_goodSM.pdf}
   \includegraphics[width=0.80\linewidth]{plots_v2/Coeffs_Central_nohighE.pdf}
   \caption{\label{fig:global_vs_gooddata} \small
     Same as  Fig.~\ref{fig:globalfit-baseline-coeffsabs-lin-vs-quad}
     comparing the global fit results with those of the fit excluding datasets with poor $\chi^2_{\rm sm}$
   (upper) and with the fit where the bins with $E\gsim 1$ TeV are removed (bottom panel).}
  \end{center}
\end{figure}
%%%%%%%%%%%%%%%%%%%%%%%%%%%%%%%%%%%%%%%%%%%%%%%%%%%%%%%%%%%%%%%%%%%%%%

To continue with this discussion of the dataset dependence of our results, we consider now
the outcome of
two more fits: first, one where the datasets exhibiting poor agreement with the SM predictions
are excluded, and second, another where all bins sensitive to the high-energy region,
defined as $E\gsim 1 $ TeV, are removed.
%
The best-fit values and 95\% CL intervals of these two fits are compared
with the baseline results in Fig.~\ref{fig:global_vs_gooddata}.
%
As indicated in Table~\ref{eq:chi2-datasetvariations}, in the fit where those
datasets with poor $\chi^2_{\rm sm}$
have been removed, one is essentially cutting away 27 points from top quark production, mostly
from the inclusive $t\bar{t}$ category.
%
The only coefficients that are affected by this reduction in the dataset are some of the
two-light-two-heavy operators, whose bounds are mildly enlarged consistently
with the loss of experimental information.
%
This comparison highlights the stability of the global fit results, whose
outcome is unchanged when potentially problematic datasets with high $\chi^2_{\rm sm}$
are excluded from the fit.
%
Concerning the outcome of the fit without the high-energy bins, as expected the only differences
are observed again for the two-light-two-heavy coefficients, with a similar
outcome as in the previous fit.
%
From this analysis, one can conclude that the global fit is not dominated by the high-energy regions
where the EFT validity could be questioned, and hence that results are stable
upon removal of these high-energy bins.

Finally, we show in Fig.~\ref{fig:wo_CMS2Dttbar} a comparison
of the outcome of quadratic EFT fits with and without
the CMS top-quark pair double-differential $(m_{t\bar{t}},y_{t\bar{t}})$ distributions.
%
We have identified this dataset as the one being responsible for
driving upwards the fit value of the chromo-magnetic
operator $c_{tG}$.
%
Indeed, 
one can observe how once this dataset is removed
then $c_{tG}$ agrees with the SM at the 95\% CL.
%
Given that both in the global linear and the individual quadratic fits
$c_{tG}$ also agrees with the SM (even in fits where {\tt CMS\_tt2D\_8TeV\_dilep\_mttytt}
is included), the pull found in the global quadratic case must arise from a non-trivial
interplay between different EFT degrees of freedom.
%
Further studies are required to elucidate why this specific dataset has such as strong
pull on $c_{tG}$ in the quadratic fits.

%%%%%%%%%%%%%%%%%%%%%%%%%%%%%%%%%%%%%%%%%%%%%%%%%%%%%%%%%%%%%%%%%%%%%
\begin{figure}[t]
  \begin{center}
    \includegraphics[width=0.80\linewidth]{plots_v2/Coeffs_Central_Fit_no_CMS2Dtt.pdf}
   \caption{\label{fig:wo_CMS2Dttbar} \small
    The outcome of quadratic EFT fits with and without
   the CMS top-quark pair double-differential $(m_{t\bar{t}},y_{t\bar{t}})$ distributions.}
  \end{center}
\end{figure}
%%%%%%%%%%%%%%%%%%%%%%%%%%%%%%%%%%%%%%%%%%%%%%%%%%%%%%%%%%%%%%%%%%%%%%

%%%%%%%%%%%%%%%%%%%%%%%%%%%%%%%%%%%%%%%%%%%%%%%%%%%%%%%%%%%%%%%%
%%%%%%%%%%%%%%%%%%%%%%%%%%%%%%%%%%%%%%%%%%%%%%%%%%%%%%%%%%%%%%%%


