\subsection{Constraints on the EFT parameter space}

Following this assessment of the fit quality,
we move to present the constraints on the SMEFT parameter space
that can be derived from the present global fit.
%
We will present results for the $n_{\rm op}=50$ Wilson
coefficients listed in Table~\ref{tab:operatorbasis},
with the understanding that only 36 of them are linearly
independent.\footnote{We note that the EWPO constraints of Eq.~(\ref{eq:2independents}) set
  the four-lepton operator to zero, $c_{\ell\ell}=0$, and hence we exclude this coefficient
from the plots and tables of this section.}
%
Specifically, we provide the 95\% confidence level intervals for each
EFT coefficients,
study their posterior probability distributions, evaluate the pattern
of their correlations, and compare the marginalised bounds with those
obtained in individual fits where only one coefficient is varied at a time.
%
We will also assess the overall consistency of the fit results with respect
to the Standard Model hypothesis.
%
The results discussed here correspond to the global dataset with the baseline theory
settings for both $\mathcal{O}\lp \Lambda^{-2}\rp$ and $\mathcal{O}\lp \Lambda^{-4}\rp$
theory calculations.
%
Fits based on either reduced datasets or alternative theory settings
are then discussed in Sects.~\ref{sec:dataset_dependence}
and~\ref{subsec:loqcd}  respectively.

%%%%%%%%%%%%%%%%%%%%%%%%%%%%%%%%%%%%%%%%%%%%%%%%%%%%%%%%%%%%%%%%%%%%%
\begin{figure}[t]
  \begin{center}
     \includegraphics[width=0.99\linewidth]{plots_v2/Coeffs_Hist_Baseline.pdf}
    \caption{\small The normalised posterior probability distributions associated
      to each of the $n_{\rm op}=50$ fit coefficients considered in the present
      analysis, for both the linear and quadratic EFT fits.
      %
      Note that the $x$-axis range is different in each case.
      %
      From top to bottom and from left to right, we display
      the four-heavy, two-light-two-heavy,
      two-fermion, and purely bosonic coefficients.
      %
      Only 36 of these coefficients are independent
      as indicated in Table~\ref{tab:operatorbasis}.
     \label{fig:posterior_coeffs} }
  \end{center}
\end{figure}
%%%%%%%%%%%%%%%%%%%%%%%%%%%%%%%%%%%%%%%%%%%%%%%%%%%%%%%%%%%%%%%%%%%%%%

\paragraph{Posterior distributions.}
%
Fig.~\ref{fig:posterior_coeffs} displays
the normalised posterior probability distributions associated
to each of the $n_{\rm op}=50$ fit coefficients considered in the present
analysis, for the linear (blue) and quadratic (orange) EFT fits.
%
As discussed in Sect.~\ref{sec:nestedsampling},
the NS prior sampling volumes have been optimised to ensure
that the posterior distribution associated to each coefficient
is fully contained within them.
%
One can observe how in general
the $\mathcal{O}\lp \Lambda^{-4}\rp$ corrections modify significantly
 the distributions that is obtained from the linear fits, for instance
 by shifting its median or by decreasing its variance.
 %
 For several coefficients, the posterior distributions would be poorly
 described in the Gaussian approximation,
 and in some cases one finds multi-modal distributions
 such as for the Yukawa operators $c_{\varphi c}$, $c_{\varphi b}$, and
 $c_{\varphi \tau}$.
 %
 Such double-humped distributions can be traced back to the (quasi)-degenerate
 minima in the individual
 $\chi^2$ profiles reported in
 Figs.~\ref{fig:quartic-individual-fits} and~\ref{fig:quartic-individual-fits-2}.
%
 We can also observe how the four-heavy coefficients can only be meaningfully
 constrained in the quadratic fit.
 %
 All in all, inclusion of the quadratic EFT corrections modifies
 in a significant manner the posterior distributions associated
 to most of the fit coefficients as compared to the linear approximation.

\paragraph{Confidence level intervals.}
%
From the posterior probability distributions displayed in Fig.~\ref{fig:posterior_coeffs},
one can derive the marginalised 95\% CL intervals on the 
EFT coefficients both for the linear and quadratic fits.
%
These results are collected in Table~\ref{tab:coeff-bounds-baseline}
(for $\Lambda=1$ TeV)
and represented graphically in
Fig.~\ref{fig:globalfit-baseline-coeffsabs-lin-vs-quad}.
%
In addition,  Table~\ref{tab:coeff-bounds-baseline} also includes
the corresponding obtained in  individual NS fits, where only one
operator is varied at a time and the rest are set
to their SM values (recall the $\chi^2$ profiles from
Figs.~\ref{fig:quartic-individual-fits} and~\ref{fig:quartic-individual-fits-2}).
%
We will further discuss the outcome of these individual fits below.

%%%%%%%%%%%%%%%%%%%%%%%%%%%%%%%%%%%%%%%%%%%
%%%%%%%%%%%%%%%%%%%%%%%%%%%%%%%%%%%%%%%%%%%%%%%%%%%%%%%%%%%%%%%%%%%%%%%%%%%%%%%%%%%%%%%
\begin{table}[htbp]
  \centering
  \scriptsize
   \renewcommand{\arraystretch}{1.24}
   \begin{tabular}{l|C{0.8cm}|C{2.3cm}|C{2.3cm}|C{4.0cm}|C{4.0cm}}
     \multirow{2}{*}{Class}   &  \multirow{2}{*}{DoF}
     &  \multicolumn{2}{c|}{ 95\% CL bounds, $\mathcal{O}\lp \Lambda^{-2}\rp$} &
     \multicolumn{2}{c}{95\% CL bounds, $\mathcal{O}\lp \Lambda^{-4}\rp$,} \\ 
 &  & Individual & Marginalised &  Individual & Marginalised  \\ \toprule
 \multirow{5}{*}{4H}
 &{\tt cQQ1}& [-6.132,23.281] & [-190,189] & [-2.229,2.019] & [-2.995,3.706] \\ \cline{2-6}
 & {\tt cQQ8}  & [-26.471,57.778] & [-190,170] & [-6.812,5.834] & [-11.177,8.170] \\ \cline{2-6}
 & {\tt cQt1}& [-195,159] & [-190,189] & [-1.830,1.862] & [-1.391,1.251] \\ \cline{2-6}
 & {\tt cQt8}& [-5.722,20.105] & [-190,162] & [-4.213,3.346] & [-3.040,2.202] \\ \cline{2-6}
 & {\tt ctt1}& [-2.782,12.114] & [-115,153] & [-1.151,1.025] & [-0.791,0.714] \\ \hline
\multirow{14}{*}{2L2H}
 & {\tt c81qq}& [-0.273,0.509] & [-2.258,4.822] & [-0.373,0.309] & [-0.555,0.236] \\ \cline{2-6}
 & {\tt c11qq}& [-3.603,0.307] & [-8.047,9.400] & [-0.303,0.225] & [-0.354,0.249] \\ \cline{2-6}
 & {\tt c83qq}& [-1.813,0.625] & [-3.014,7.365] & [-0.470,0.439] & [-0.462,0.497] \\ \cline{2-6}
 & {\tt c13qq}& [-0.099,0.155] & [-0.163,0.296] & [-0.088,0.166] & [-0.167,0.197] \\ \cline{2-6}
 & {\tt c8qt}& [-0.396,0.612] & [-4.035,4.394] & [-0.483,0.393] & [-0.687,0.186] \\ \cline{2-6}
 & {\tt c1qt}& [-0.784,2.771] & [-12.382,6.626] & [-0.205,0.271] & [-0.222,0.226] \\ \cline{2-6}
 & {\tt c8ut}& [-0.774,0.607] & [-16.952,0.368] & [-0.911,0.347] & [-1.118,0.260] \\ \cline{2-6}
 & {\tt c1ut}& [-6.046,0.424] & [-15.565,15.379] & [-0.380,0.293] & [-0.383,0.331] \\ \cline{2-6}
 & {\tt c8qu}& [-1.508,1.022] & [-12.745,13.758] & [-1.007,0.521] & [-1.002,0.312] \\ \cline{2-6}
 & {\tt c1qu}& [-0.938,2.462] & [-16.996,1.072] & [-0.281,0.371] & [-0.207,0.339] \\ \cline{2-6}
 & {\tt c8dt}& [-1.458,1.365] & [-5.494,25.358] & [-1.308,0.638] & [-1.329,0.643] \\ \cline{2-6}
 & {\tt c1dt}& [-9.504,-0.086] & [-27.673,11.356] & [-0.449,0.371] & [-0.474,0.347] \\ \cline{2-6}
 & {\tt c8qd}& [-2.393,2.042] & [-24.479,11.233] & [-1.615,0.888] & [-1.256,0.715] \\ \cline{2-6}
 & {\tt c1qd}& [-0.889,6.459] & [-3.239,34.632] & [-0.332,0.436] & [-0.370,0.384] \\ \hline
\multirow{23}{*}{2FB}
 & {\tt ctp}& [-1.331,0.355] & [-5.739,3.435] & [-1.286,0.348] & [-2.319,2.797] \\ \cline{2-6}
 & {\tt ctG}& [0.007,0.111] & [-0.127,0.403] & [0.006,0.107] & [0.062,0.243] \\ \cline{2-6}
 & {\tt cbp}& [-0.006,0.040] & [-0.033,0.105]& [-0.007,0.035]$\cup$ [-0.403,-0.360] & [-0.035,0.047]$\cup$ [-0.430,-0.338] \\ \cline{2-6}
 & {\tt ccp}& [-0.025,0.117] & [-0.316,0.134] & [-0.004,0.370] & [-0.096,0.484] \\ \cline{2-6}
 & {\tt ctap}& [-0.026,0.035] & [-0.027,0.044] & [-0.027,0.040]$\cup$ [0.395,0.462] & [-0.019,0.037]$\cup$ [0.389,0.480] \\ \cline{2-6}
 & {\tt ctW}& [-0.093,0.026] & [-0.313,0.123] & [-0.084,0.029] & [-0.241,0.086] \\ \cline{2-6}
 & {\tt ctZ}& [-0.039,0.099] & [-15.869,5.636] & [-0.044,0.094] & [-1.129,0.856] \\ \cline{2-6}
 & {\tt cpl1}& [-0.664,1.016] & [-0.244,0.375] & [-0.281,0.343] & [-0.106,0.129] \\ \cline{2-6}
 & {\tt c3pl1}& [-0.472,0.080] & [-0.098,0.120] & [-0.432,0.062] & [-0.209,0.046] \\ \cline{2-6}
 & {\tt cpl2}& [-0.664,1.016] & [-0.244,0.375] & [-0.281,0.343] & [-0.106,0.129] \\ \cline{2-6}
 & {\tt c3pl2}& [-0.472,0.080] & [-0.098,0.120] & [-0.432,0.062] & [-0.209,0.046] \\ \cline{2-6}
 & {\tt cpl3}& [-0.664,1.016] & [-0.244,0.375] & [-0.281,0.343] & [-0.106,0.129] \\ \cline{2-6}
 & {\tt c3pl3}& [-0.472,0.080] & [-0.098,0.120] & [-0.432,0.062] & [-0.209,0.046] \\ \cline{2-6}
 & {\tt cpe}& [-1.329,2.033] & [-0.487,0.749] & [-0.562,0.687] & [-0.213,0.258] \\ \cline{2-6}
 & {\tt cpmu}& [-1.329,2.033] & [-0.487,0.749] & [-0.562,0.687] & [-0.213,0.258] \\ \cline{2-6}
 & {\tt cpta}& [-1.329,2.033] & [-0.487,0.749] & [-0.562,0.687] & [-0.213,0.258] \\ \cline{2-6}
 & {\tt c3pq}& [-0.472,0.080] & [-0.098,0.120] & [-0.432,0.062] & [-0.209,0.046] \\ \cline{2-6}
 & {\tt c3pQ3}& [-0.350,0.353] & [-1.145,0.740] & [-0.375,0.344] & [-0.615,0.481] \\ \cline{2-6}
 & {\tt cpqMi}& [-2.905,0.490] & [-0.171,0.106] & [-2.659,0.381] & [-0.060,0.216] \\ \cline{2-6}
 & {\tt cpQM}& [-0.998,1.441] & [-1.690,11.569] & [-1.147,1.585] & [-2.250,2.855] \\ \cline{2-6}
 & {\tt cpui}& [-1.355,0.886] & [-0.499,0.325] & [-0.458,0.375] & [-0.172,0.142] \\ \cline{2-6}
 & {\tt cpdi}& [-0.443,0.678] & [-0.162,0.250] & [-0.187,0.229] & [-0.071,0.086] \\ \cline{2-6}
 & {\tt cpt}& [-2.087,2.463] & [-3.270,18.267] & [-3.028,2.195] & [-13.260,3.955] \\ \hline
\multirow{7}{*}{B}
 & {\tt cpG}& [-0.002,0.005] & [-0.043,0.012] & [-0.002,0.005] & [-0.019,0.003] \\ \cline{2-6}
 & {\tt cpB}& [-0.005,0.002] & [-0.739,0.289] & [-0.005,0.002]$\cup$ [0.085,0.092] & [-0.114,0.108] \\ \cline{2-6}
 & {\tt cpW}& [-0.018,0.007] & [-0.592,0.677] & [-0.016,0.007]$\cup$ [0.281,0.305] & [-0.145,0.303] \\ \cline{2-6}
 & {\tt cpWB}& [-2.905,0.490] & [-0.462,0.694] & [-2.659,0.381] & [-0.170,0.273] \\ \cline{2-6}
 & {\tt cpd}& [-0.428,1.214] & [-2.002,3.693] & [-0.404,1.199]$\cup$ [-34.04,-32.61] & [-1.523,1.482] \\ \cline{2-6}
 & {\tt cpD}& [-4.066,2.657] & [-1.498,0.974] & [-1.374,1.124] & [-0.516,0.425] \\ \cline{2-6}
 & {\tt cWWW}& [-1.057,1.318] & [-1.049,1.459] & [-0.208,0.236] & [-0.182,0.222] \\ \bottomrule
\end{tabular}
   \caption{\small The 95\% CL bounds for all the
     EFT coefficients
     considered in this analysis, for  both individual and global (marginalised) fits
     obtained using either linear or quadratic EFT calculations.
\label{tab:coeff-bounds-baseline}
}
\end{table}
%%%%%%%%%%%%%%%%%%%%%%%%%%%%%%%%%%%%%%%%%%%%%%%%%%%%%%%%%%%%%%%%%%%%%%%%%%%%%%%%

%%%%%%%%%%%%%%%%%%%%%%%%%%%%%%%%%%%%%%%%%%

%%%%%%%%%%%%%%%%%%%%%%%%%%%%%%%%%%%%%%%%%%%%%%%%%%%%%%%%%%%%%%%%%%%%%
\begin{figure}[t]
  \begin{center}
    \includegraphics[width=0.99\linewidth]{plots_v2/Coeffs_Central_Baseline.pdf}
    \vspace{-0.1cm}
    \caption{\small The best-fit (median) value of the EFT coefficients $c_i/\Lambda^2$
      and their associated 95\% CL intervals for the  global fits
      based on either linear or quadratic EFT calculations,
      whose posterior distributions are represented in
      Fig.~\ref{fig:posterior_coeffs}.
      %
      The dashed horizontal line indicates the SM expectation.
     \label{fig:globalfit-baseline-coeffsabs-lin-vs-quad} }
  \end{center}
\end{figure}
%%%%%%%%%%%%%%%%%%%%%%%%%%%%%%%%%%%%%%%%%%%%%%%%%%%%%%%%%%%%%%%%%%%%%%

From the marginalised bounds displayed in Fig.~\ref{fig:globalfit-baseline-coeffsabs-lin-vs-quad},
one can observe how the uncertainties associated to the fit coefficients are in all cases
reduced in the quadratic fit in comparison to the linear one.
%
The 95\% CL interval is disjoint for the Yukawa coefficients $c_{b\varphi}$
and $c_{\tau\varphi}$ in the quadratic fit,
with both a SM-like solution and a second one far from the SM.
%
For the linear fit, we find that all EFT coefficients agree with the SM expectation
at the 95\% CL level.
%
For the quadratic fit instead, this is not the case only for
the chromo-magnetic operator $c_{tG}$.
%
We will trace back below the origin of this discrepancy,
here we only point out that at the level of individual fits
$c_{tG}$ exhibits the same trend but there
agrees with the SM at the  95\% CL
as indicated in Fig.~\ref{fig:quartic-individual-fits-2}.
%
{  We note that for unconstrained operators, such as the four-heavy operators
  in the linear fit, the best-fit value (median) should be ignored since
the underlying posterior is essentially flat.}

The global fit results of Fig.~\ref{fig:globalfit-baseline-coeffsabs-lin-vs-quad}
are further scrutinized in Fig.~\ref{fig:globalfit-baseline-bounds-lin-vs-quad},
which displays both the magnitude of the 95\% CL intervals 
   and the 68\% CL residuals compared to the SM hypothesis
   associated to the linear and quadratic EFT fits.
   %
   In the upper panel,
      the horizontal line indicates the boundaries of the sampling volume
      used for the poorly-constrained coefficients
as explained in Sect.~\ref{sec:nestedsampling}.
%
From these comparisons, one can observe how the inclusion of quadratic corrections
leads to markedly more stringent bounds for most of the fit coefficients,
a trend which is specially significant for the four-heavy (unconstrained
in the linear fit) and two-light-two-heavy operators which modify the properties
of the top quark.
%
The only exception is the charm Yukawa coefficient $c_{\varphi c}$, since there the quadratic
corrections introduce a second degenerate solution thus enlarging the magnitude
of the CL interval.

%%%%%%%%%%%%%%%%%%%%%%%%%%%%%%%%%%%%%%%%%%%%%%%%%%%%%%%%%%%%%%%%%%%%%
\begin{figure}[t]
  \begin{center}
 \includegraphics[width=0.85\linewidth]{plots_v2/Coeffs_Bar_Baseline.pdf}
 \includegraphics[width=0.85\linewidth]{plots_v2/Coeffs_Residuals_Baseline.pdf}
 \vspace{-0.3cm}
 \caption{\small The magnitude of the 95\% CL intervals (top)
   and the value of the 68\% CL residuals compared to the SM hypothesis (bottom panel)
   corresponding to the global fit results
   displayed in Fig.~\ref{fig:globalfit-baseline-coeffsabs-lin-vs-quad}.
   %
   In the upper plot, the dashed horizontal line indicates the maximum
   prior volume used for the sampling of unconstrained coefficients.
         \label{fig:globalfit-baseline-bounds-lin-vs-quad} }
  \end{center}
\end{figure}
%%%%%%%%%%%%%%%%%%%%%%%%%%%%%%%%%%%%%%%%%%%%%%%%%%%%%%%%%%%%%%%%%%%%%%

The 68\% CL residuals displayed in the bottom panel of
Fig.~\ref{fig:globalfit-baseline-bounds-lin-vs-quad} are defined by
\be
\label{eq:fit_residual}
R(c_i)\equiv \frac{\lp c_i|_{\rm EFT} -c_i|_{\rm SM} \rp }{\delta c_i} \, ,\qquad
i=1,\ldots,n_{\rm op} \, ,
\ee
with $c_i|_{\rm EFT}$ being the median of the posterior distribution
from the EFT fit, $c_i|_{\rm SM}=0$, and $\delta c_i$ is the total fit uncertainty
for this parameter.
%
We can observe that $|R(c_i)|\lsim 1$ for most of the fit coefficients,
both for the linear and quadratic cases.
%
The only exception is $c_{tG}$, where a residual of
$R(c_{tG})\simeq 3.5$ is found in the quadratic fit.
%
Nevertheless, for a large enough number of EFT coefficients
one would expect a fraction of these residuals to be larger than
unity, even if the SM is the
underlying theory.
%
Fig.~\ref{fig:residuals-histo} then displays
the normalised distribution of
these fit residuals.
%
While these coefficients are correlated among them (see the
following discussion) and thus
cannot be treated as independent variables,
the shapes of these distributions are reasonably close
to a Gaussian, specially for the linear fit, highlighting
again the overall consistency of the fit results
with the SM expectations.

%%%%%%%%%%%%%%%%%%%%%%%%%%%%%%%%%%%%%%%%%%%%%%%%%%%%%%%%%%%%%%%%%%%%%
\begin{figure}[t]
  \begin{center}
\includegraphics[width=0.70\linewidth]{plots_v2/Coeffs_Residuals_Hist_Baseline.pdf}
    \vspace{-0.2cm}
    \caption{\small The (normalised) distribution of
      the fit residuals shown in the bottom panel of
      Fig.~\ref{fig:globalfit-baseline-bounds-lin-vs-quad}.
     \label{fig:residuals-histo} }
  \end{center}
\end{figure}
%%%%%%%%%%%%%%%%%%%%%%%%%%%%%%%%%%%%%%%%%%%%%%%%%%%%%%%%%%%%%%%%%%%%%%

\paragraph{Correlations.}
%
The correlation coefficient between any two fit coefficients
$c_i$ and $c_j$ can be evaluated as follows,
\be
\label{eq:correlationL2CT}
\rho\lp c_i,c_j\rp=\frac{\lp \frac{1}{N_{\rm spl}}\sum_{k=1}^{N_{\rm spl}}
c_i^{(k)} c_j^{(k)}\rp -\la c_i\ra \la c_j\ra
}{\delta c_i \delta c_j} \, ,\qquad i,j=1,\ldots,n_{\rm op} \, ,
\ee
where $N_{\rm spl}$ denotes the number of samples produced by NS, $\la c_i\ra$ indicates
the mean value of this coefficient, and, as in Fig.~\ref{eq:fit_residual},
$\delta c_i$ is the corresponding uncertainty (standard deviation).
%
The values of Eq.~(\ref{eq:correlationL2CT})
are displayed in Fig.~\ref{fig:globalfit-correlations} 
separately for the linear  and quadratic fits.
%
We display only the numerical values for the pair-wise
coefficient combinations for which the correlation coefficient
is numerically significant,
$|\rho(c_i,c_j)|\ge 0.5$.
%
The pairs $(c_i,c_j)$ that do not appear in Fig.~\ref{fig:globalfit-correlations}  have a correlation
coefficient below this threshold.

%%%%%%%%%%%%%%%%%%%%%%%%%%%%%%%%%%%%%%%%%%%%%%%%%%%%%%%%%%%%%%%%%%%%%
\begin{figure}[htbp]
  \begin{center}
  \includegraphics[width=0.49\linewidth]{plots_v2/Coeffs_Corr_Baseline_lin.pdf}
  \includegraphics[width=0.49\linewidth]{plots_v2/Coeffs_Corr_Baseline_quad.pdf}
  \vspace{-0.3cm}
  \caption{\small The correlation coefficients $\rho\lp c_i,c_j\rp$
    between the EFT coefficients
      in the linear (left) and quadratic (right panel) fits.
      %
      We only display the  entries with significant (anti)-correlation,
      $|\rho|\ge 0.5$.
      %
      Pairs of coefficients $(c_i,c_j)$ that do not displayed here have a correlation
coefficient below this threshold.
     \label{fig:globalfit-correlations} }
  \end{center}
\end{figure}
%%%%%%%%%%%%%%%%%%%%%%%%%%%%%%%%%%%%%%%%%%%%%%%%%%%%%%%%%%%%%%%%%%%%%%

We observe how the  majority of
the fit coefficients are  loosely correlated among them, that is,
their correlations being $|\rho| \le 0.5$.
%
One  also finds that while several of the two-light-two-heavy coefficients turn out to be
strongly correlated
at the linear EFT level, this pattern disappears once   the quadratic  corrections
are accounted for.
%
Concerning the two-fermion operators,
the correlation patterns present at the linear level
are also often reduced  in the quadratic fits.
%
For instance, $c_{tZ}$ displays a strong correlation with $c_{\varphi B}$ at the linear level
which is then washed out  by the quadratic effects.
%
The purely bosonic operators exhibit
in general more stable correlations, for example $c_{\varphi W B}$ 
is strongly anti-correlated  with
$c_{\varphi D}$ in a manner which is similar in the linear and the quadratic fits.
%
Furthermore,
we do not find any pair of fit coefficients where the quadratic corrections
flip the sign of their correlation.

In general, from Fig.~\ref{fig:globalfit-correlations} one can
conclude that only a moderate subset of Wilson coefficients end
up being strongly (anti-)correlated
among them after the fit, specially so once quadratic EFT corrections are taken into account.
%
This finding is partially explained by our wide input dataset,
which makes possible constraining independently most if not all
the EFT degrees of freedom.

{  For completeness, App.~\ref{sec:fullcovmat}
  provides the
  correlation matrices for the complete set of operators
considered in this analysis.}

%%%%%%%%%%%%%%%%%%%%%%%%%%%%%%%%%%%%%%%%%%%%%%%%%%%%%%%%%%%%%%%%%%%%%%%%
%%%%%%%%%%%%%%%%%%%%%%%%%%%%%%%%%%%%%%%%%%%%%%%%%%%%%%%%%%%%%%%%%%%%%%%%

\paragraph{Individual fits.}
%
As motivated in Sect.~\ref{sec:quarticfits}, individual
(one-parameter) fits have several useful applications.
%
These include representing a benchmark reference for the global fit
results, where the obtained bounds can only loosen as compared
to one-parameter fits.
%
The 95\% CL bounds associated to the one-parameter linear and quadratic EFT
fits were reported in Table~\ref{tab:coeff-bounds-baseline},
and the corresponding graphical comparison with the marginalised
global fit results is displayed in
Fig.~\ref{fig:globalfit-baseline-bounds-lin-NS-ind-vs-marg}.
%
{  While the expectation is that individual bounds
are comparable or more stringent than the marginalised ones, }
this property does not necessarily hold for the coefficients
constrained by the EWPOs, for which an individual fit is not meaningful.
%
{  Indeed, one-parameter fits are ill-defined in
  the case of the coefficients constrained by EWPOs since
  these coefficients cannot be determined independently from each other.
  %
  Hence the comparison between marginalised and individual bounds
  is only meaningul for the 34 independent coefficients listed
  in Table~2.5.
}

%%%%%%%%%%%%%%%%%%%%%%%%%%%%%%%%%%%%%%%%%%%%%%%%%%%%%%%%%%%%%%%%%%%%%
\begin{figure}[t]
  \begin{center}
    \includegraphics[width=0.91\linewidth]{plots_v2/Coeffs_Bar_individ_lin.pdf}
    \includegraphics[width=0.91\linewidth]{plots_v2/Coeffs_Bar_individ_quad.pdf}
    \vspace{-0.4cm}
    \caption{\small Comparison of the magnitude of 95\% CL intervals in the global
      (marginalised) and individual fits at the linear (top) and quadratic
      (bottom) level, see also Table~\ref{tab:coeff-bounds-baseline}.
     \label{fig:globalfit-baseline-bounds-lin-NS-ind-vs-marg} }
  \end{center}
\end{figure}
%%%%%%%%%%%%%%%%%%%%%%%%%%%%%%%%%%%%%%%%%%%%%%%%%%%%%%%%%%%%%%%%%%%%%%

Considering first the results of the linear analysis,
one can observe how for the fitted degrees of freedom the individual bounds are tighter
(or at most comparable) than the marginalised
ones by a large amount, around a factor ten or more in most cases.
%
These differences are particularly striking for some of the two-fermion operators,
in particular for $c_{tZ}$, as well as for bosonic operators such as
$c_{\varphi B}$ and $c_{\varphi W}$,
for which the differences between the individual and marginalised results can be as large
as two orders of magnitude.
%
Specifically, in the cases of $c_{tZ}$ and $c_{\varphi B}$, the 95\% CL intervals
found in the linear EFT analysis are
increased as follows when going from the individual to the marginalised fits:
\bea
\nonumber
c_{tZ}:\qquad [-0.04,0.10]~~{\rm (individual)}  \quad&{\rm vs}&         \quad [-17,5.6]~~{\rm (marginalised)} \, ,\\
c_{\varphi B}:~~ \quad [-0.005,0.002] ~~{\rm (individual)}  \quad&{\rm vs}&   \quad  [-0.7,0.3]~~{\rm (marginalised)}\, .
\nonumber
\eea
%
This effect clearly emphasizes the importance of adopting a fitting basis as wide as possible,
in order to avoid obtaining artificially stringent bounds simply because one is being
blind to other relevant directions of the parameter space.
%
One important exception of this rule would be those cases where one is guided by 
specific UV-complete models, which motivate the reduction in the parameter space
to a subset of operators.
%
We also note that the triple gauge operator $c_W$ is one of the few coefficients whose individual
and marginalised bounds are identical: this can be traced back
to the fact that this operator is very weakly correlated with
other coefficients (see also  Fig.~\ref{fig:globalfit-correlations}), being
constrained exclusively by the diboson data.

Inspection of the corresponding results
from the quadratic fits, bottom panel of Fig.~\ref{fig:globalfit-baseline-bounds-lin-NS-ind-vs-marg}, 
 reveals that the differences between individual and marginalised bounds are in general
smaller as compared to the linear case.
%
This effect is particularly visible
for the two-light-two-heavy and the four-heavy operators, for which one finds
that the individual fits underestimate the magnitude of the 95\% CL interval by around
a factor two on average, rather than by a factor 10 as in the linear case.
%
The situation is instead similar to the linear fits for the two-fermion and the purely
bosonic operators, and for example now also for  $c_{tZ}$, $c_{\varphi B}$ and $c_{\varphi W}$ one finds
large differences between marginalised and individual fits.
%
One should point out, however,
that even on those cases where the magnitude of the bound
does not vary much, the central best-fit values can still shift in a non-negligible manner.

%%%%%%%%%%%%%%%%%%%%%%%%%%%%%%%%%%%%%%%%%%%%%%%%%%%%%%%%%%%%%%%%%%%%%%%%
%%%%%%%%%%%%%%%%%%%%%%%%%%%%%%%%%%%%%%%%%%%%%%%%%%%%%%%%%%%%%%%%%%%%%%%%

\paragraph{Two-parameter fits.}
%
To complement the insights provided by individual fits, it can also be instructive
to carry out two-parameter fits, specially to investigate the relative
interplay between specific pairs of EFT coefficients.
%
In such fits, two coefficients are allowed to vary simultaneously
while the rest are set to zero.
%
To illustrate the information that can be provided
by such two-parameter fits, Fig.~\ref{fig:2Dfits} displays
representative results for fits performed at the linear order.
%
We display the 95\% CL ellipses obtained when different subsets of data are used as input,
as well as for the complete dataset, labelled as ``All Data (2D)''.
%
For reference, we also show here the marginalised bounds obtained
from the global fit.

%%%%%%%%%%%%%%%%%%%%%%%%%%%%%%%%%%%%%%%%%%%%%%%%%%%%%%%%%%%%%%%%%%%%%
\begin{figure}[t]
  \begin{center}
    \includegraphics[width=0.32\linewidth]{plots_v2/Ellipse_OtG_Otp_NLO_NHO.pdf}
    \includegraphics[width=0.32\linewidth]{plots_v2/Ellipse_OtG_OpG_NLO_NHO.pdf}
    \includegraphics[width=0.32\linewidth]{plots_v2/Ellipse_OpG_Otp_NLO_NHO.pdf}
    \includegraphics[width=0.32\linewidth]{plots_v2/Ellipse_O81qq_O83qq_NLO_NHO.pdf}
    \includegraphics[width=0.32\linewidth]{plots_v2/Ellipse_O8dt_O8ut_NLO_NHO.pdf}
    \includegraphics[width=0.32\linewidth]{plots_v2/Ellipse_O8dt_O8qd_NLO_NHO.pdf}
    \vspace{-0.2cm}
    \caption{\small Representative results for two-parameter fits carried out
      at linear order in the EFT.
      %
      We display the 95\% CL ellipses obtained for different data subsets
      and for the complete dataset, labelled as ``All Data (2D)''.
      %
      For reference, we also show the marginalised bounds obtained
      in the global fit.
      %
      The black square in the center of the plot indicates the SM value.
     \label{fig:2Dfits} }
  \end{center}
\end{figure}
%%%%%%%%%%%%%%%%%%%%%%%%%%%%%%%%%%%%%%%%%%%%%%%%%%%%%%%%%%%%%%%%%%%%%%

To begin with, the upper panels of Fig.~\ref{fig:2Dfits} display two-parameter fits
for the three possible pair-wise combinations of the $c_{t \varphi}$, $c_{tG}$, and $c_{\varphi G}$
coefficients,
which connect Higgs production in gluon fusion with top quark pair production, see also
the Fisher information table of  Fig.~\ref{fig:FisherMatrix}.
%
These comparisons illustrate the relative impact of the various dataset in constraining
each coefficient.
%
For example, from the $\lp c_{t\varphi},c_{tG}\rp$ fit we see that the sensitivity of
$c_{tG}$ is driven by $t\bar{t}$ data, while the Higgs differential
measurements have a flat direction
resulting in a elongated ellipse.
%
The overlap between $t\bar{t}$ data and  Higgs differential
measurements results in similar constraints as compared to those
provided by the Higgs signal strengths alone.
%
Note that, as in the case of the individual fits reported in
Fig.~\ref{fig:globalfit-baseline-bounds-lin-NS-ind-vs-marg}, also for two-parameter fits
the obtained bounds are more stringent as compared to the global marginalised results.
%
Similar considerations apply to the $\lp c_{\varphi G},c_{tG}\rp$ fit, while
from the $\lp c_{\varphi G},c_{t\varphi}\rp$ one learns that the sensitivity
is still dominated by the Higgs signal strengths rather than by the differential cross-section
measurements.

Then the bottom panels of Fig.~\ref{fig:2Dfits} display two-parameter fits involving the
two-light-two-heavy coefficients $c_{Qq}^{1,8}$, $c_{Qq}^{3,8}$, $c_{tu}^8$, $c_{td}^8$, and $c_{tq}^8$,
all of which are constrained mostly from top quark pair differential distributions as indicated
by the Fisher information matrix.
%
Here the scope is to illustrate the relative sensitivity provided by some of the $t\bar{t}$
datasets that enter the fit: single-inclusive $m_{t\bar{t}}$ distributions, the double-differential
$(m_{t\bar{t}},y_{t\bar{t}})$ distributions, and $t\bar{t}V$ measurements.
%
The results confirm both that the $m_{t\bar{t}}$ distributions completely dominate
the fit of these coefficients, and that the marginalised CL ellipses are rather broader
than for the two-dimensional fits.
%
The latter is again in agreement with the results of the individual
linear fits, reported in the upper panel of
Fig.~\ref{fig:globalfit-baseline-bounds-lin-NS-ind-vs-marg}.

%%%%%%%%%%%%%%%%%%%%%%%%%%%%%%%%%%%%%%%%%%%%%%%%%%%%%%%%%%%%%%%%%%%%%%%%%%%%%%%%%%%%%%%%
%%%%%%%%%%%%%%%%%%%%%%%%%%%%%%%%%%%%%%%%%%%%%%%%%%%%%%%%%%%%%%%%%%%%%%%%%%%%%%%%%%%%%%%%

