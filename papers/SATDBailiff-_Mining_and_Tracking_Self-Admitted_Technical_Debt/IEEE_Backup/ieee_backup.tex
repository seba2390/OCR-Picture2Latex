\documentclass[conference]{IEEEtran}
\IEEEoverridecommandlockouts

%% The '5p' and 'times' class options of elsarticle are used for Elsevier CRC
%\documentclass[ 5p,times,authoryear]{elsarticle}

%\documentclass[3p]{elsarticle}
%\pdfoutput=1

%\documentclass[number,preprint,review,1p,times]{elsarticle}

% The preceding line is only needed to identify funding in the first footnote. If that is unneeded, please comment it out.
\usepackage{cite}
\usepackage{amsmath,amssymb,amsfonts}
\usepackage{algorithmic}
\usepackage{graphicx}
\usepackage{textcomp}
\usepackage{xcolor}
\usepackage{tikz}
\usetikzlibrary{backgrounds}
\usepackage{verbatim}
\usepackage{ulem}
\def\BibTeX{{\rm B\kern-.05em{\sc i\kern-.025em b}\kern-.08em
    T\kern-.1667em\lower.7ex\hbox{E}\kern-.125emX}}
\begin{document}

\usetikzlibrary{calc,trees,positioning,arrows,chains,shapes.geometric,%
    decorations.pathreplacing,decorations.pathmorphing,shapes,%
    matrix,shapes.symbols}

\tikzset{
>=stealth',
  punktchain/.style={
    rectangle, 
    rounded corners, 
    % fill=black!10,
    draw=black, very thick,
    text width=10em, 
    minimum height=3em, 
    text centered, 
    on chain},
  line/.style={draw, thick, <-},
  element/.style={
    tape,
    top color=white,
    bottom color=blue!50!black!60!,
    minimum width=8em,
    draw=blue!40!black!90, very thick,
    text width=10em, 
    minimum height=3.5em, 
    text centered, 
    on chain},
  every join/.style={->, thick},
}

\title{SATDBailiff - A Self-Admitted Technical Debt Empirical Aggregation Tool\\}

\author{
  \IEEEauthorblockN{Ben Christians, Mohamed Wiem Mkaouer}
  \IEEEauthorblockA{\textit{Software Engineering Dept.} \\
    \textit{Rochester Institute of Technology}\\
    NY, USA \\
    \{bbc7909,mwmvse\}@rit.edu
  }
}

\maketitle

\begin{abstract}
% What is SATD?
Self-Admitted Technical Debt (SATD) is a metaphorical concept to describe the self-documented addition of technical debt to a software project in the form of source-code comments.
% How does SATD specifically impact the field of SE?
SATD can linger in projects and degrade source-code quality, but can also be more visible than unintentionally added or undocumented technical debt.
% Why is this a problem/call to action?
Understanding the implications of adding SATD to a software project is important for developers to understand the quality trade-offs they are intentionally making.
% What is currently being done about the problem?
However, little has been done to establish a quality and large scale empirical history for SATD in software projects.
% What is my solution to the problem?
SATDBailiff is a first attempt at an empirical aggregation tool for collecting SATD instances from Git repositories.
% What does my solution do?
The tool mines empirical histories of SATD comments by looking at all versions of source code, and identifying SATD using a previously published Natural Language Processing tool designed specifically for SATD detection.
\end{abstract}

\begin{IEEEkeywords}
Self-Admitted Technical Debt, Mining Software Repositories
\end{IEEEkeywords}

\section{Introduction}
% What is Technical Debt?
Technical debt is a metaphor that describes taking shortcuts in software development that will require additional time to fix (or pay back) in the future \cite{Kruchten}. 
% How is SATD different than normal Technical debt?
Self-Admitted Technical Debt (SATD) is a candid form of technical debt in which the contributor of the debt self-documents the location of the debt.
% What are the benefits of looking at SATD?
Developers believe that including a self-admission will make it easier to pay back, or reduce the likelihood of technical debt being forgotten $<$Citation?$>$. Understanding the implications of this belief is vital to assure high-quality performance in software development teams.

% What is the current state of SATD research?
The topic of SATD has been investigated since 2014 \cite{Potdar}. Since then, little work has been done to develop an empirical understanding of SATD in Java projects \cite{Maldonado, Bavota}.
% What are the limitations with it?
These studies included the publication of their datasets, which contained 2 and 5 different projects, respectively. The quality of the dataset of \cite{Maldonado} has been brought into question by studies using it \cite{Zampetti}, and not much attention has been drawn to \cite{Bavota}.
% What are the opportunities for improvement?
Since these publications, new SATD classification models have been made available \cite{Huang}.
% What can I do about these opportunities?
Taking advantage of this improved detection tool, as well as fixing some of the data quality issues noted in \cite{Maldonado} has an opportunity to provide further research efforts with a highly accurate, large scale empirical history of Java projects previously unavailable.

% What exactly will this study/tool aim to do at a high level?
This study will aim to leverage this new model, in conjunction with other automation logic to provide a highly accurate mining tool. Packaging this logic as a tool will allow further efforts to expand past these 7 previously available software projects. In addition to publication of this tool, the empirical history of 796 open source software projects will be made available as produced by SATDBailiff $<$How do I format this name correctly?$>$.
% Graphic of process?

\section{Overview of SATDBailiff}
\label{sec:overview}

% What exactly is the tool?
SATDBailiff is a Java tool designed to mine empirical histories of SATD instances from Java project Git repositories at a large scale.
% What is it trying to do?
This is done with the goal of tracking the addition, removal, and changes of SATD Instances during the process of software development.
% Why will the tool be useful to other researchers?
The datasets provided by this tool can be used to understand what development actions are taken (and by whom) during the addition and removal of SATD. The tool was designed with modularity in mind, to allow for the implementation of new classification models and output formats.

% What steps does it take to achieve its goal?
To collect these empirical histories, the Java tool leverages the Eclipse JGit library \footnote{https://github.com/eclipse/jgit}, JavaParser \footnote{https://github.com/javaparser/javaparser}, and the SATD Detector tool presented by Huang et al. \cite{Huang}. These tools are used to gather Git commit metadata, Java Comment metadata, and for classifying Java comments as SATD respectively as shown in Figure~\ref{fig:flow}.

\begin{figure}
    \centering
    \begin{tikzpicture} 
        \useasboundingbox (-4,-6.3) rectangle (4,1);
        \scope[node distance=.8cm, start chain=going below, transform canvas={scale=0.8}]
                \node[punktchain, join] (repo) {Github Repository};
                \node[punktchain, join] (jgit) {JGit};
                  \node [inner sep=0,minimum size=0] (inv1) [below left=6em and 1em of jgit] {Commit Metadata};
                \node[punktchain, join] (jpar) {JavaParser};
                  \node [inner sep=0,minimum size=0] (inv2) [below right=4em and 1em of jpar] {Comment Metadata};
                \node[punktchain, join] (satd) {SATDDetector};
                \node[punktchain, join] (emph) {Empirical History};
                
                \draw[-, thick] (jgit.west) -| (inv1.north);
                \draw[->, thick] (inv1.south) |- (emph.west);
                \draw[-, thick] (jpar.east) -| (inv2.north);
                \draw[->, thick] (inv2.south) |- (emph.east);
        \endscope
    \end{tikzpicture}
    \caption{SATDBailiff flow}
    \label{fig:flow}
\end{figure}

The tool reviews each commit that is part of the master branch of the repository ranging back to the oldest commit in the repository. Each change to a file between each commit are searched for additions, changes, and removals to SATD instances. Commits with multiple parents (i.e. Merge commits) are ignored, as no changes to SATD should be made to the project during those commits $<$Citation to specific SE principle?$>$. The result of this process is a complete empirical history of all operations to SATD instances between two points in a project's lifetime.

% What data does it output?
The implicit implementation of the tool outputs to MySQL, but a the tool supports easy extensibility for other output formats. To showcase the data mined with the tool, a simplified sample from the Apache Tomcat project is included in Figure~\ref{fig:sample_data}.
% Why does the data look the way it does?
The data includes some important features:
\begin{enumerate}
    \item \textbf{SATD Id and SATD Instance Id}. Each entry has two Ids. The SATD Id is a unique identifier for a specific operation to an SATD Instance. An SATD Instance ID is an over-arching Identifier used to group SATD operations. The set of these operations details a unique SATD Instance. In Figure~\ref{fig:sample_data} the "instance\_id" field represents the shared instance between the three SATD operations. Each entry in this sample would have a different and unique SATD Id.
    \item \textbf{Resolution}. Each SATD operation has a single operation that impacts the SATD. These operations include: SATD\_ADDED, SATD\_REMOVED, SATD\_CHANGED, FILE\_REMOVED, FILE\_PATH\_CHANGED, and CLASS\_OR\_METHOD\_CHANGED. These resolutions are meant to describe the location of the SATD in the project.
    \item \textbf{Comment Metadata}. Each SATD operation records the comment metadata at the time of the operation. This includes data such as the comment type (Line, Block, or JavaDoc as recorded by JavaParser), start and end line, containing class and method, the file name, and the comment itself.
    \item \textbf{Commit Metadata}. Each operation records the commit metadata of both the operation commit and its parent commit. This includes a timestamp, author, committer, and SHA1 commit hash.
\end{enumerate}

\begin{figure}
    \centering
    \begin{tabular}{ |p{4.4em}|p{7.6em}|p{3.5em}|p{4.8em}|   }
     \hline
     \multicolumn{4}{|c|}{SATD Instances} \\
     \hline
     instance\_id & resolution & commit & comment\\
     \hline
     789 & ADDED         & 09b640e & TODO: 404\\
     789 & PATH\_CHANGED & decfe2a & TODO: 404\\ 
     789 & FILE\_REMOVED & a457153 & None\\
    
     \hline
    \end{tabular}
    \caption{Simplified Sample Data from the Apache Tomcat project}
    \label{fig:sample_data}
\end{figure}

% What issues have other attempts faced that this tool overcomes?
Previously, Maldonado established an empirical history that contained many entries that recorded SATD Instance changes as removals and additions. In addition to mistaking file renames, this would count the following entry as having both resolved the original SATD instance and added the new version to the project:

\vspace{1em}

// perhaps it would be better if we did not \sout{impement} \uwave{implement} this.

\vspace{1em}

% How does the tool overcome these issues?
SATDBailiff resolves this issue by handling SATD comment and file name changes as operations in-between additions and removals of SATD. File renames are easily handled as Git already provides algorithms for detecting these. Changes in the SATD comment are more difficult to determine because changes can possibly remove the SATD comment entirely, replacing it with a new non-SATD or irrelevant comment. In this case, only SATD comments which preserve the original intent of the SATD comment should be recorded as a SATD\_CHANGED operation. To determine whether a comment change preserves the original intent of the SATD instance, a normalized Levenshtein distance \cite{Yujian} is used to determine the extent of the modifications. If this normalized distance is less than an arbitrarily chosen threshold of 0.5, then we can determine the SATD instances are related after an update.

To verify the accuracy of SATDBailiff, a manual analysis was performed on a stratified random sample of 200 entries mined from 5 large open source Java projects. The number of samples taken from each project correlates with the number of SATD instances mined from the projects. These projects were selected to verify the accuracy of the tool because they were the same 5 projects used by Maldonado to establish a prior empirical history. An entry contains all operations performed an SATD Instance during the lifetime of the project. An simplified example of this can be seen in Figure~\ref{fig:sample_data}. The results of this analysis can be seen in Figure~\ref{fig:manual_analysis}. 

\begin{figure}
    \centering
    \begin{tabular}{ |p{4.4em}|p{4.6em}|p{4.5em}|p{4.8em}|   }
     \hline
     \textbf{Project} & \textbf{\# Entries} & \textbf{\# Correct} & \textbf{Accuracy}\\
     \hline
     log4j          & 5   & 5   & 1.000\\
     camel          & 60  & 54  & 0.900\\ 
     gerrit         & 14  & 13  & 0.929\\
     hadoop         & 61  & 56  & 0.918\\ 
     tomcat         & 59  & 46  & 0.780\\
     \hline
     \textbf{Total} & 200 & 174 & \textbf{0.870}\\ 
    
     \hline
    \end{tabular}
    \caption{SATDBailiff Manual Analysis Results}
    \label{fig:manual_analysis}
\end{figure}

During the manual analysis, a "correct" entry was identified as an entry in which every operation made to the SATD instance could be located using the Github interface. Any additional, missing, or inaccurate operations reported would result in the entire entry being incorrect. For transparency of this analysis, a Github link to the exact source modification was recorded.

% What are the issues faced by my tool?
The results of the manual analysis (Figure~\ref{fig:manual_analysis}) find the tool to have an accuracy of 0.870. A review of the incorrect entries show that 10 of the 26 entries that were incorrect resulted from a logical error that existed in the version of the tool that was tested. Other incorrect entries resulted from co-located SATD operations, such as an SATD containing file being renamed and the SATD instance being removed. A few other errors arose from small and complex logical issues that could not be fully tracked through the system. Overall, many of the issues of the tool are likely solvable, however time considerations have lead to them remaining in the project.

% Why are these not easily solvable?
Difficulties in solving these issues come from the imperfect nature of working with Edit Scripts produced by Git differencing tools. Edit scripts are used to show changes in files inside of a Git repository, and do not always reflect the true intentions of the developer \cite{Frick}. Git (and thus, JGit) uses the Myers \cite{Myers} and the Histogram diff algorithm to produce edit scripts. The tool provides the ability to change between these two algorithms. Both of these algorithms maintain a manually validated accuracy less than 0.9 \cite{Frick}. This serves as a significant limitation in the upper bound of accuracy achievable by this tool.

% What steps does the tool take to assure quality data?
To assure that these errors are not able to silently pollute the dataset, the tool reports any known errors that are encountered during the mining process.
% Why are these steps taken?
This workaround was taken as an optimistic precaution for an issue that may not have a perfect alternative solution.
% Example of this process in action
For example, SATD that is added during a merge commit which was not present in either of the merge branches is not detected with an SATD\_ADDED entry. If that SATD is modified or removed later, then the entry would be added to the project before the SATD\_ADDED entry was found. Because the search occurs chronologically starting with the oldest commit in the project, the system can detect this as an issue and will output an error to the terminal during runtime.

\section{Using SATDBailiff}

This section describes the usage of SATDBailiff and its features.
\subsection{Installation}
The most up-to-date precompiled binaries can be found on the project's GitHub ( $<$FINAL GITHUB URL$>$) or at the tool's website ($<$ TOOL WEBSITE URL$>$). The project can be run using a java version 1.8 and is otherwise OS independent.
\subsection{Usage}
SATDBailiff can be used either through its CLI or API, and can be easily modified to support different output types and classification models.
\subsubsection{CLI}
The SATDBailiff CLI has two main inputs: A file containing a list of Github repositories and a MySQL configuration file. The CLI has many other optional inputs to optimally configure the tool.

The file containing repository information details which repositories will be mined by the tool. This file can be used to specify the terminal commit for each file. This is used to add an terminal point in time to which the tool can mine, with the main goal of assuring reproducibility between datasets. The terminal commit value will default to the most recently available commit in the repository. The output format of the data also allows for a manual filtering of data by date. However, it should be noted that un-merged branches of the repository at certain timestamps are likely to cause difference between a pre- and post-execution commit filtering. Git credentials for private repositories can be added as a separate program argument.

By default, SATDBailiff outputs to a MySQL database. This means that a MySQL database must be set up to receive the system's output. Additionally, configuration for this database must be supplied to the program to connect to the database. A description for the required fields, as well as the required schema for the database can be found in the Github repository.

Other program arguments include: 
\begin{itemize}
    \item File differencing algorithms available through JGit - currently Histogram and Myers
    \item The Normalized Levenshtein distance threshold described in Section~\ref{sec:overview}.
    \item A toggle for error output
    \item A help menu display 
\end{itemize}

When run, SATDBailiff will output the following data:

\vspace{4em}

$<$ Include sample of CLI here $>$

\vspace{4em}

This output includes a detailed description of the runtime duration of the tool, the number of commits differences, and a description of each entry that caused any sort of detectable error in the system. For usability, a loading bar is included for quick reference of the tool's progress on the current project as well as an reference to the commit that is currently under review.

\subsubsection{API}

The tool is also available in the form of a Java API. All functionality that is achievable through the CLI is available as part of the API.

$<$ Should I include a link to API Docs? I don't have anything set up for this yet. $>$

\subsection{Interpreting Output}
$<$ I was planning on including the MySQL schema here with a description of each part of the output, but I ended up describing most of it earlier in the paper. Is it worth going deeper into it here? $>$

\section{Conclusion}

% In this paper, ...
In this paper, a preview of SATDBailiff was given and its benefits were discussed. The tool aims to offer an unmatched ability to extract SATD instances and the development operations upon them from Github repositories.
% What was discussed in this paper?
This paper discussed the benefits that a high quality empirical history would have for the further study of SATD, as well as the ways in which prior studies had opportunity to be improved upon.

\section{Future Work}

% What issues with my tool's accuracy can be addressed?
An acceptable level of accuracy was achieved with SATDBailiff, however there is still opportunity for improvement. In their nature, source code differencing tools may not always record modifications that reflect the true nature of a developer's intentions. Because of these inaccuracies, several edge cases appears as recurring issues. These cases are noted on the tool's website. Some attempts to fix these edge-cases were made, however their large numbers and unpredictability made it a futile task to complete at a large scale. As Edit Scripts are more reliably instantiated from these differences, a more accurate detection of SATD-impacting operations will be possible.

% What other approaches can be taken to make a similar tool that might produce better accuracy or speed?
Edit scripts were generated for this tool using the Histogram and Myers algorithms made available through JGit. Using a differencing tool like GumTree \cite{Falleri} or a hybrid approach \cite{Matsumoto}. Modification of this algorithm may have positive impacts on the project's runtime which may currently be excessive for larger project.

% What additional data would be useful to include?


\bibliography{paper}
\bibliographystyle{plain}

\end{document}
