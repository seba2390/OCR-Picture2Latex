% Template for ICASSP-2021 paper; to be used with:
%          spconf.sty  - ICASSP/ICIP LaTeX style file, and
%          IEEEbib.bst - IEEE bibliography style file.
% --------------------------------------------------------------------------
%
%
%
\documentclass{article}
\usepackage{spconf,amsmath,graphicx}
\usepackage{xcolor}
\usepackage{booktabs}
\usepackage{hyperref}
\usepackage{caption}
\usepackage{tikz}
\usepackage{pifont}% http://ctan.org/pkg/pifont
\newcommand{\cmark}{\ding{51}}%
\newcommand{\xmark}{\ding{55}}%
\newcommand{\trillsson}{{TRILLsson}}
\newcommand{\wavvec}{{Wav2Vec$2.0$}}
\captionsetup{font=footnotesize}

% Example definitions.
% --------------------
\def\x{{\mathbf x}}
\def\L{{\cal L}}

% Title.
% ------
%\title{TRILLsson: Publicly Available and On-Device Universal Paralinguistic Speech Representations via Cross-Architecture Knowledge Distillation}
%\title{TRILLsson: Paralinguistic Speech Representations via Knowledge Distillation}
\title{TRILLsson: Distilled Universal Paralinguistic Speech Representations}
% alternate names DISTRILL

%
% Single address.
% ---------------
\name{Joel Shor$^1$, Subhashini Venugopalan$^2$}
\address{Verily Life Sciences$^1$, Google Research$^2$ \\
joelshor@verily.com}

\begin{document}
%\ninept
%
\maketitle
%
\begin{abstract}
Recent advances in self-supervision have dramatically improved the quality of speech representations.
However, deployment of state-of-the-art embedding models on devices has been restricted due to their limited public availability and large resource footprint.
Our work addresses these issues by \textbf{publicly releasing a collection of paralinguistic speech models}\footnote[1]{\scriptsize\url{https://tfhub.dev/s?q=trillsson}} that are small and near state-of-the-art performance. Our approach is based on knowledge distillation, and our models are distilled on public data only. We explore different architectures 
and thoroughly evaluate our models on the Non-Semantic Speech (NOSS) benchmark.
Our largest distilled model is \textbf{less than 15\% the size} of the original model (314MB vs 2.2GB), achieves \textbf{over 96\% the accuracy on 6 of 7 tasks}, and is trained on 6.5\% the data. The smallest model is \textbf{1\% in size} (22MB) and achieves over \textbf{90\% the accuracy on 6 of 7 tasks}. Our models outperform the open source Wav2Vec 2.0 model on 6 of 7 tasks, and our smallest model outperforms the open source Wav2Vec 2.0 on both emotion recognition tasks despite being 7\% the size.
%However there are still constraints that have prevented them from being more widely deployed on devices particularly for paralinguistic tasks. For one, the models are often too large to be run in resource constrained environments. Second, general-purpose speech representations from large self-supervised techniques are rarely publicly available. 
%Our work addresses these. We train a collection of near state-of-the-art paralinguistic speech representations on only public data and \textbf{publicly release them}\footnote[1]{\url{https://tfhub.dev/s?q=nonsemantic-speech-benchmark}}.
%Our work addresses these by training a collection of near state-of-the-art paralinguistic speech representations on public data and \textbf{publicly releasing them}\footnote[1]{\url{https://tfhub.dev/s?q=nonsemantic-speech-benchmark}}.
%We achieve this dramatic reduction in model size and latency via a collection of techniques such as limited context windows and generating target vectors at the right timescale.
\end{abstract}
%
\begin{keywords}
speech, representations, on-device, paralinguistic speech
\end{keywords}
%
\section{Introduction}
\label{sec:intro}

\begin{figure}[t]
\begin{center}
   \includegraphics[width=1.0\linewidth]{figures/nas_comp_v3}
\end{center}
   \vspace{-4mm}
   \caption{The comparison between NetAdaptV2 and related works. The number above a marker is the corresponding total search time measured on NVIDIA V100 GPUs.}
\label{fig:nas_comparison}
\end{figure}

\section{Introduction}
\label{sec:introduction}

Neural architecture search (NAS) applies machine learning to automatically discover deep neural networks (DNNs) with better performance (e.g., better accuracy-latency trade-offs) by sampling the search space, which is the union of all discoverable DNNs. The search time is one key metric for NAS algorithms, which accounts for three steps: 1) training a \emph{super-network}, whose weights are shared by all the DNNs in the search space and trained by minimizing the loss across them, 2) training and evaluating sampled DNNs (referred to as \emph{samples}), and 3) training the discovered DNN. Another important metric for NAS is whether it supports non-differentiable search metrics such as hardware metrics (e.g., latency and energy). Incorporating hardware metrics into NAS is the key to improving the performance of the discovered DNNs~\cite{eccv2018-netadapt, Tan2018MnasNetPN, cai2018proxylessnas, Chen2020MnasFPNLL, chamnet}.


There is usually a trade-off between the time spent for the three steps and the support of non-differentiable search metrics. For example, early reinforcement-learning-based NAS methods~\cite{zoph2017nasreinforcement, zoph2018nasnet, Tan2018MnasNetPN} suffer from the long time for training and evaluating samples. Using a super-network~\cite{yu2018slimmable, Yu_2019_ICCV, autoslim_arxiv, cai2020once, yu2020bignas, Bender2018UnderstandingAS, enas, tunas, Guo2020SPOS} solves this problem, but super-network training is typically time-consuming and becomes the new time bottleneck. The gradient-based methods~\cite{gordon2018morphnet, liu2018darts, wu2018fbnet, fbnetv2, cai2018proxylessnas, stamoulis2019singlepath, stamoulis2019singlepathautoml, Mei2020AtomNAS, Xu2020PC-DARTS} reduce the time for training a super-network and training and evaluating samples at the cost of sacrificing the support of non-differentiable search metrics. In summary, many existing works either have an unbalanced reduction in the time spent per step (i.e., optimizing some steps at the cost of a significant increase in the time for other steps), which still leads to a long \emph{total} search time, or are unable to support non-differentiable search metrics, which limits the performance of the discovered DNNs.

In this paper, we propose an efficient NAS algorithm, NetAdaptV2, to significantly reduce the \emph{total} search time by introducing three innovations to \emph{better balance} the reduction in the time spent per step while supporting non-differentiable search metrics:

\textbf{Channel-level bypass connections (mainly reduce the time for training and evaluating samples, Sec.~\ref{subsec:channel_level_bypass_connections})}: Early NAS works only search for DNNs with different numbers of filters (referred to as \emph{layer widths}). To improve the performance of the discovered DNN, more recent works search for DNNs with different numbers of layers (referred to as \emph{network depths}) in addition to different layer widths at the cost of training and evaluating more samples because network depths and layer widths are usually considered independently. In NetAdaptV2, we propose \emph{channel-level bypass connections} to merge network depth and layer width into a single search dimension, which requires only searching for layer width and hence reduces the number of samples.

\textbf{Ordered dropout (mainly reduces the time for training a super-network, Sec.~\ref{subsec:ordered_droput})}: We adopt the idea of super-network to reduce the time for training and evaluating samples. In previous works, \emph{each} DNN in the search space requires one forward-backward pass to train. As a result, training multiple DNNs in the search space requires multiple forward-backward passes, which results in a long training time. To address the problem, we propose \emph{ordered dropout} to jointly train multiple DNNs in a \emph{single} forward-backward pass, which decreases the required number of forward-backward passes for a given number of DNNs and hence the time for training a super-network.

\textbf{Multi-layer coordinate descent optimizer (mainly reduces the time for training and evaluating samples and supports non-differentiable search metrics, Sec.~\ref{subsec:optimizer}):} NetAdaptV1~\cite{eccv2018-netadapt} and MobileNetV3~\cite{Howard_2019_ICCV}, which utilizes NetAdaptV1, have demonstrated the effectiveness of the single-layer coordinate descent (SCD) optimizer~\cite{book2020sze} in discovering high-performance DNN architectures. The SCD optimizer supports both differentiable and non-differentiable search metrics and has only a few interpretable hyper-parameters that need to be tuned, such as the per-iteration resource reduction. However, there are two shortcomings of the SCD optimizer. First, it only considers one layer per optimization iteration. Failing to consider the joint effect of multiple layers may lead to a worse decision and hence sub-optimal performance. Second, the per-iteration resource reduction (e.g., latency reduction) is limited by the layer with the smallest resource consumption (e.g., latency). It may take a large number of iterations to search for a very deep network because the per-iteration resource reduction is relatively small compared with the network resource consumption. To address these shortcomings,  we propose the \emph{multi-layer coordinate descent (MCD) optimizer} that considers multiple layers per optimization iteration to improve performance while reducing search time and preserving the support of non-differentiable search metrics.

Fig.~\ref{fig:nas_comparison} (and Table~\ref{tab:nas_result}) compares NetAdaptV2 with related works. NetAdaptV2 can reduce the search time by up to $5.8\times$ and $2.4\times$ on ImageNet~\cite{imagenet_cvpr09} and NYU Depth V2~\cite{nyudepth} respectively and discover DNNs with better performance than state-of-the-art NAS works. Moreover, compared to NAS-discovered MobileNetV3~\cite{Howard_2019_ICCV}, the discovered DNN has $1.8\%$ higher accuracy with the same latency.



\section{Related Work}
Our work draws from, and improves upon, several research threads.

\textbf{Sustainability.}~\citet{srba2016stack} conducted a case study on why StackOverflow, the largest and oldest of the sites in \CQA{StackExchange} network, is failing. They shed some insights into knowledge market failure such as novice and negligent users generating low quality content perpetuating the decline of the market. However, they do not provide a systematic way to understand and prevent failures in these markets.~\citet{wu2016} introduced a framework for understanding the user strategies in a knowledge market---revealing the importance of diverse user strategies for sustainable markets. In this paper, we present an alternative model that provides many interesting insights including knowledge market sustainability.

\textbf{Activity Dynamics.}~\citet{walk2016} modeled user-level activity dynamics in \CQA{StackExchange} using two factors: intrinsic activity decay, and positive peer influence. However, the model proposed there does not reveal the collective platform dynamics, and the eventual success or failure of a platform.~\citet{abufouda2017} developed two models for predicting the interaction decay of community members in online social communities. Similar to~\citet{walk2016}, these models accommodate user-level dynamics, whereas we concentrate on the collective platform dynamics.~\citet{wu2011} proposed a discrete generalized beta distribution (DGBD) model that reveals several insights into the collective platform dynamics, notably the concept of a size-dependent distribution. In this paper, we improve upon the concept of a size-dependent distribution.  

\textbf{Economic Perspective.} \citet{Kumar2010} proposed an economic view of CQA platforms, where they concentrated on the growth of two types of users in a market setting: users who provide questions, and users who provide answers. In this paper, we concentrate on a subsequent problem---the ``relation'' between user growth and content generation in a knowledge market.~\citet{butler2001} proposed a resource-based theory of sustainable social structures. While they treat members as resources, like we do, our model differs in that it concentrates on a market setting, instead of a network setting, and takes the complex content dependency of the platform into consideration. Furthermore, our model provides a systematic way to understand successes and failures of knowledge markets, which none of these models provide.  

\textbf{Scale Study.}~\citet{lin2017} examined Reddit communities to characterize the effect of user growth in voting patterns, linguistic patterns, and community network patterns. Their study reveals that these patterns do not change much after a massive growth in the size of the user community.~\citet{tausczik2017} investigated the effects of crowd size on solution quality in StackExchange communities. Their study uncovers three distinct levels of group size in the crowd that affect solution quality: topic audience size, question audience size, and number of contributors. In this paper, we examine the consequence of scale on knowledge markets from a different perspective by using a set of health metrics.

\textbf{Stability.} Successes and failures of platforms have been studied from the perspective of user retention and stability~\cite{patil2013, garcia2013, kapoor2014, ellis2016}. Notably,~\citet{patil2013} studied the dynamics of group stability based on the average increase or decrease in member growth. Our paper examines stability in a different manner---namely, by considering the relative exchangeability of users as a function of scale.

\textbf{User Growth.} Successes and failures of user communities have also been widely studied from the perspective of user growth~\cite{Kumar2006, Backstrom2006, kairam2012, Ribeiro2014, zang2016}.~\citet{kairam2012} examined diffusion and non-diffusion growth to design models that predict the longevity of social groups.~\citet{Ribeiro2014} proposed a daily active user prediction model which classifies membership based websites as sustainable and unsustainable. While this perspective is important, we argue that studying the successes and failures of communities based on content production can perhaps be more meaningful~\cite{kraut2014, zhu2014, zhu2014niche}.

\textbf{Modeling CQA Websites.} There is a rich body of work that extensively analyzed CQA websites~\cite{Adamic2008, chen2010, anderson2012, wang2013, srba2016}, along with user behavior~\cite{zhang2007, liu2011, pal2012, hanrahan2012, upadhyay2017}, roles~\cite{furtado2013, kumar2016}, and content generation~\cite{baezaYates2015, Yang2015, ferrara2017}. Notably,~\citet{Yang2015} noted the \emph{scalability problem} of CQA---namely, the volume of questions eventually subsumes the capacity of the answerers within the community. Understanding and modeling this phenomenon is one of the goals of this paper.


\section{Approach}
\label{sec:experiments}

%%%%%%%%%%%%%%%%%%%%%%%%%%%

\begin{table}[!t]
\captionsetup{size=footnotesize}
\centering
\caption{Training datasets. We only use AudioSet and Libri-light for distillation. YT-U was used to train CAP12~\cite{cap12}. }\label{tab:train_ds}
\footnotesize
\begin{tabular}{c c c c c c } \toprule
  Dataset & Total (Hours) & Samples &  Avg len (s)  \\
  \midrule
 Speech AudioSet & 4.9K & 1.8M & 9.9 \\
 Libri-light & 53K & 6.0M & 31.8 \\
 \midrule
 YT-U & 900K & 295M & 11.0 \\
 \bottomrule
\end{tabular}
\end{table}
%\vspace{-5em}

%%%%%%%%%%%%%%%%%%%%%%%%%%%
\begin{table}[!t]
\captionsetup{size=footnotesize}
\centering
\caption{Downstream evaluation datasets.
$^*$Results in our study used a subset of Voxceleb filtered according to YouTube's privacy guidelines.}\label{tab:eval}
\footnotesize
\begin{tabular}{c c c c c c } \toprule
  Dataset        & Target & Classes     & Samples     &  \begin{tabular}{@{}c@{}}Avg \\ length (s)\end{tabular}  \\
  \midrule
 VoxCeleb$^*$~\cite{voxceleb}  
 %&  Speaker ID   & 1,251    & 148,642   & 8.2 \\
 &  Speaker ID   & 1,251    & 12,052     & 8.4 \\
 VoxForge~\cite{voxforge}   
 & Language ID   & 6        & 176,438   & 5.8 \\
 \begin{tabular}{@{}c@{}}Speech \\ Commands\cite{speechcommands}\end{tabular} 
 & Command       & 12       & 100,503   & 1.0  \\
 %\begin{tabular}{@{}c@{}}Masked \\ Speech~\cite{compare2020}\end{tabular}   
 %& Mask wearing  & 2        & 36,554    & 1.0  \\
 ASVSpoof~\cite{asvspoof}   
 & \begin{tabular}{@{}c@{}}Synthetic \\ or not\end{tabular}
 & 2  & 121,461  & 3.2  \\
 Euphonia~\cite{euphonia}
 & Dysarthria & 5 & 15,224 & 6.4 \\
 CREMA-D~\cite{cremad}
 & Emotion       & 6        & 7,438     & 2.5 \\
 IEMOCAP~\cite{iemocap} 
 & Emotion       & 4        & 5,531         & 4.5    \\
 \bottomrule
\end{tabular}
%\vspace{-14mm}
\end{table}

\begin{figure}[t]
{\centering
  \includegraphics[width=0.9\columnwidth]{images/eval_flow.png}\hspace{0.2cm}
  \caption{Depiction of the evaluation process.}\label{fig:eval_flow}}
  \vspace{-3mm}
\end{figure}

%%%%%%%%%%%%%%%%%%%%%%%%%%%

%\begin{figure}[t]
%{\centering
%  \includegraphics[width=0.7\columnwidth]{images/training.png}\hspace{0.2cm}
%    \caption{Two paradigms for generating target embeddings from a teacher. In both cases, fixed context-window students are trained to match the target. Targets can be generated by either computing teacher embeddings over the entire audio clip (global matching) or over just the fixed-length window that the student sees (local matching). In this work, we use local matching.}\label{fig:training}}
  %\caption{Diagram of two different approaches for generating targets for fixed context window student models from transformer teacher models.}\label{fig:training}}
%  \vspace{-3mm}
%\end{figure}
%%%%%%%%%%%%%%%%%%%%%%%%%%%

Shor et. al. \cite{cap12} demonstrated that the 1024 dimension representation of the 12th layer of the CAP Conformer model (referred to as ``CAP12'') achieved at or near state-of-the-art performance across all tasks in the paralinguistic Non-Semantic Speech Benchmark (NOSS)~\cite{trill}. CAP12 is a 606M parameter (2.2GB) Conformer model trained via a modified \wavvec\  self-supervised training loss on a 900M+ hour speech dataset derived from YouTube (YT-U~\cite{bigssl}, Tab.~\ref{tab:train_ds}). %\cite{cap12} showed that the 1024 dimension representation of the 12th layer of the model (referred to as ``CAP12'') has impressive performance on a wide range of paralinguistic tasks.  
In this work, we use CAP12 as the teacher and distill this model to several ``lite'' architectures based on the teacher-student distillation approach~\cite{distillation}.

\subsection{Student architectures.}
%\noindent\textbf{Student architectures.}
\label{sec:kd}
We explore 3 different student architectures of varying sizes.
\begin{enumerate}\itemsep0em
    \item \textbf{Audio Spectrogram Transformer (AST)}~\cite{ast} is a Transformer-based model for audio classification. We train student models with different depths and widths.
    
    \item \textbf{EfficientNetv2}~\cite{efficientnetv2} was designed by neural architecture search on image classification. The architecture is mobile friendly. Different versions of this architecture vary in terms of depths and filters.
    
    \item \textbf{Resnetish}~\cite{hershey2017cnn} are modified ResNet-50 architectures designed to take audio spectral features as input. Different versions of this architecture include different depths and different number of filters per layer.
\end{enumerate}

%%%%%%%%%%%%%%%%%%%%%%%%%%%
% Floating fig.
\begin{centering}
\begin{table*}[t]
\scriptsize
\vspace{-0.2cm}
\caption{Test performance on the NOSS Benchmark and extended tasks. 
``Prev. SoTA" are usually domain specific, but all other rows are linear models on time-averaged input. TRILLsson model sizes are shown without frontends.
$\uparrow$ indicates higher values are better, and $\downarrow$ indicates lower is better.
%``Prev SoTA" are arbitrarily complicated models, but \textbf{all other rows are linear models on time-averaged input}.
$^\dagger$We use a filtered subset of Voxceleb1 according to YouTube’s privacy guidelines. We omit previous SoTA results on this dataset, since they used the entire dataset.
$^{**}$ASVSpoof uses equal error rate~\cite{asvspoof}. We report the best single-model performance (as compared to model ensembles).
$^{\#}$Euphonia is the only non-public dataset. We use a larger dataset than was reported on in~\cite{cap12}.
$^*$We use the public Wav2Vec 2.0 model from \href{https://huggingface.co/docs/transformers/model_doc/wav2vec2}{Hugging Face}~\cite{transformers_lib}
}\label{tab:results}

\centering
%\begin{tabular}{@{} llcc|cccccccc @{}}
\begin{tabular}{@{} llcc|ccccccc @{}}
\toprule[2pt]
\begin{tabular}{@{}c@{}}Model\\(input size)\end{tabular} &
%Arch. &
\begin{tabular}{@{}c@{}}Params\\ (M)\end{tabular} &
\begin{tabular}{@{}c@{}}Size\\ (MB)\end{tabular} &
Public &
%\begin{tabular}{@{}c@{}}Avg D-\\Prime $\uparrow$\end{tabular}   & 
Voxceleb1$^\dagger$ $\uparrow$ & 
Voxforge $\uparrow$\ & 
\begin{tabular}{@{}c@{}}Speech $\uparrow$\\ Commands\end{tabular}   &  
\begin{tabular}{@{}c@{}}ASVSpoof\\ 2019$^{**}$ $\downarrow$ \end{tabular} & 
Euphonia$^{\#}$ $\uparrow$ &
CREMA-D $\uparrow$ &
IEMOCAP $\uparrow$
\\
\toprule[2pt]
\textbf{Prev. SoTA}
& -
& -
& -
%& -
%& -
& -
& 99.8~\cite{cap12} 
& 97.9~\cite{speech_commandssota} 
& 2.5~\cite{cap12} 
& -
& 88.2~\cite{cap12} 
& 79.2~\cite{cap12}
\\
\midrule
\textbf{CAP12} \\
% Architecture avg test voxceleb voxforge speechcommands asvspoof euphonia cremad iemocap params size 
\quad (full)  
& 606 & 2,200 & \xmark
%& 3.31
& 51.0  & 99.7      & 97.1 & 2.5 & 46.9  & 88.2 & 75.5  \\
\quad (3s) 
& 606 & 2,200 & \xmark
%& 3.25
& 47.9  & 99.4      & 97.1 & 3.8 & 46.9  & 88.1 & 74.3    \\
\quad (2s) 
& 606 & 2,200 & \xmark
%& 3.10
& 48.1  & 99.4      & 97.0 & 6.9 & 46.9  & 85.3 & 72.7   \\
\midrule
\midrule
\textbf{Baselines} \\
%\quad \begin{tabular}{@{}c@{}}Wav2Vec2*\\(full)\end{tabular} & Transformer
%\quad Wav2Vec2 Lg L6$^*$
%& \tiny{Transformer}
%& ? & ? & \cmark
%& ?
%& ? & ? & ? & ? & ? & ? & ? \\
%\quad Wav2Vec2 Lg. final
%& \tiny{Transformer}
%& ? & ? & \cmark
%& ?
%& ? & ? & ? & ? & ? & ? & ? \\
\quad Wav2Vec2 Sm. L6$^*$
& 93.4 & 360 & \cmark
%& 2.87
& 17.9 & 98.5 & \textbf{95.0} & 6.7 & 48.2 & 77.4 & 65.8 \\
\quad Wav2Vec2 Sm.$^*$
& 93.4 & 360 & \cmark
%& 2.31
& 1.7 & 95.9 & 89.3 & 11.2 & 50.0 & 58.0 & 52.4 \\
\quad TRILL 
%& \tiny{Resnetish}
& 24.5 & 87 & \cmark
%& 2.06
& 13.8 & 84.5 & 77.6 & 6.3 & 47.0 & 65.7 & 55.4 \\
%\quad FRILL (1s)     & MobileNet
%& 10 & 36 & \cmark
%& 1.97 & 13.8 & 78.8 & 74.3 & 7.8 & 46.2 & 71.3 & 59.9 \\
\quad YAMNet
%& \tiny{Resnetish}
& 3.7 & 17 & \cmark
%& 2.00
& 9.6 & 79.8 & 78.5 & 6.7 & 43.8 & 66.4 & 57.5 \\
\midrule
\textbf{TRILLsson} \\
% voxceleb voxforge speechcommands asvspoof euphonia cremad iemocap 

% This is what the new code generates.
% ast_sec2_l24_m2048,	678.0	0.447263	0.995385	0.936810	0.071780	0.482906	0.856684	0.712329	
%\quad 5 (AST) 
%& \tiny{AST}
%& 176.9 & 678 & \cmark
%& 3.31
%& \textbf{44.7} & \textbf{99.5} & 93.7 & 7.2 & 48.3 & 85.7 & \textbf{71.2}  \\

% ????
% This is the old model.
%vox-filtered-acc-test	voxforge-acc-test	speech_commands-acc-test	asvspoof2019-eer-test	euphonia_5cls-acc-test	crema_d-acc-test	iemocap-acc-test
%trillsson5_public	0.461949	0.996656	0.939468	0.054253	0.480963	0.860540	0.726833
\quad 5 (AST) 
%& \tiny{AST}
& 88.6 & 314 & \cmark
%& ?
& \textbf{46.2} & \textbf{99.7} & 93.9 & \textbf{5.4} & 48.1 & 86.1 & 72.7  \\

% ast_sec2_l24_m512,	390.0	0.429907	0.995418	0.932106	0.090685	0.459984	0.857969	0.709106
%\quad 4 (AST) 
%& 101.3 & 390 & \cmark
%& 3.21
%& 43.0 & \textbf{99.5} & 93.2 & 9.1 & 46.0 & \textbf{85.8} & 70.9    \\

%vox-filtered-acc-test	voxforge-acc-test	speech_commands-acc-test	asvspoof2019-eer-test	euphonia_5cls-acc-test	crema_d-acc-test	iemocap-acc-test
%trillsson4_public	0.431242	0.996020	0.944785	0.070835	0.506605	0.862468	0.722804
% ???
\quad 4 (AST) 
& 63.4 & 224 & \cmark
%& ?
& 43.1 & 99.6 & 94.5 & 7.1 & \textbf{50.7} & \textbf{86.2} & \textbf{73.2}    \\

% This is the new model, which is too large.
% efficientnetv2bM,	231.0	0.409880	0.991940	0.937219	0.071112	0.480963	0.837404	0.705077
%\quad 3 (EffNetv2) 
%& 54.1 & 231 & \cmark
%& 3.20
%& 41.0 & 99.2 & 93.7 & 7.1 & 48.1 & 83.7 & 70.5    \\

% efficientnetv2bS,	99.0	0.400534	0.991806	0.932106	0.068359	0.474359	0.832262	0.703465
\quad 3 (EffNetv2) 
& 21.5 & 99 & \cmark
%& ?
& 40.1 & 99.2 & 93.2 & 6.8 & 47.4 & 83.2 & 70.3    \\

% efficientnetv2b1,	42.0	0.375167	0.991873	0.920654	0.065784	0.445610	0.825835	0.697824
\quad 2 (EffNetv2) 
& 8.1 & 42 & \cmark
%& 3.06
& 37.5 & 99.2 & 92.1 & 6.6 & 44.6 & 82.6 & 69.8    \\

% resnetish_sec2_2352_0.5,	22.0	0.299065	0.978563	0.890798	0.130392	0.458819	0.795630	0.670427
\quad 1 (ResNet) 
& 5.0 & 22 & \cmark
%& 2.79
& 36.6 & 98.6 & 91.2 & 7.5 & 43.3 & 81.3 & 68.5    \\
\bottomrule
\end{tabular}
\vspace{-4mm}
\end{table*}
\end{centering}

%\begin{figure}[t]
%{\centering
%  \includegraphics[width=0.9\columnwidth]{images/final_model_performance.png}\hspace{0.2cm}
%  \caption{Test set accuracies for final TRILLsson models.}\label{fig:final_model_performance}}
%  \vspace{-3mm}
%\end{figure}

%%%%%%%%%%%%%%%%%%%%%%%%%%%

Based on the Figure 1 lower in \cite{cap12} and the fact that some benchmark datasets have audio that's mostly less than 3 seconds (Tab.~\ref{tab:eval})  we focus on student architectures that process 2 seconds of audio at a time. When operating on audio less than 2 seconds, we symmetrically pad the audio around the end.

For the frontend, we use window length of 25ms and a hop length of 10ms. We use log-magnitude mel-frequency spectrograms with 80 bins ranging from 125 Hz to 7500 Hz. The frame width for all models is 2 seconds of audio, and we treat the frame advance between successive model patches as a hyperparameter. Since the model sees audio of exactly 2 seconds during training (see Sec~\ref{subsec:matching}), the frame advance hyperparameter can be explored after the model is fully trained.

\subsection{Distillation targets: global vs local matching}
%\noindent\textbf{Distillation paradigms.}
\label{subsec:matching}
There are two paradigms for generating targets from a teacher that borrow ideas from distilling large vision models~\cite{caron2021emerging}. In both cases the audio is first chunked into context windows which the student processes. Then the student is trained to match a target embedding generated using the teacher model. In ``global matching", the target is the average teacher embedding from the entire un-chunked audio clip. In the ``local matching" paradigm, the target is generated by averaging the teacher's output on the same context window that the student sees.  In this work, we focus on ``local matching."

\subsection{Training datasets}
We perform distillation on two open source datasets which together contain about 58K hours of speech data (Tab. \ref{tab:train_ds}). \textbf{Audioset}~\cite{audioset} clips are collection from YouTube, so it represents a variety of settings and acoustic environments. We use the speech subset of this data, which yields approximately 5K hours. \textbf{Libri-light}~\cite{librilight} contains 60k hours of audio derived from open-source audio books in the LibriVox project. It is the largest publicly available, unlabeled semi-supervised dataset to date. We note that the CAP12 teacher model~\cite{cap12} is trained on \textbf{YT-U}~\cite{bigssl}, which is a 900K hour dataset derived from YouTube.

%To evaluate the quality of our student networks, we use a modified version of NOSS datasets and the evaluation procedure. We describe the datasets and evaluation protocol in Section \ref{subsec:eval}. Our main aggregate comparison metric is ``Equivalent D-Prime," which we describe and motivate in Section~\ref{subsubsec:dprime}. We then describe our baseline comparisons in Section~\ref{subsubsec:baselines}. We describe our knowledge distillation approach in Sec.~\ref{sec:kd}. For student networks, we select 3 types of ``lite'' architectures, and train many variants of each. We describe these hyperparameters in the remaining sections.
%, and summarize them in Tab.~\ref{tab:train_hp}.

%%%%%%%%%%%%%%%%%%%%%%%%%%%%%%%%%
\begin{figure*}[t]
{\centering
  \includegraphics[width=2.0\columnwidth]{images/all_performance.pdf}\hspace{0.2cm}
  \caption{Comparison of ``average $d'$" vs. ``model size" for various student model architectures and sizes. Performance in this figure is across test sets, although only dev-set performance is used to select "best" models.}\label{fig:performance}}
%  \caption{`aggregate embedding performance" vs ``model size" for various student model architectures and sizes}\label{fig:performance}}
\end{figure*}
%%%%%%%%%%%%%%%%%%%%%%%%%%%%%%%%%%
\vspace{-3mm}
\subsection{Representation Evaluation}
\label{subsec:eval}

We compare and evaluate the distilled representations on several tasks including detection of speaker, language, command, synthetic speech, dysarthria, and emotion. Table~\ref{tab:eval} provides an overview of the 7 datasets these tasks come from ~\cite{trill, cap12, asremb}.
%Table~\ref{tab:eval} describes the 7 datasets from ~\cite{trill, cap12, asremb} that we use to compare and evaluate the distilled representations. The tasks include detection of speaker, language, command, synthetic speech, dysarthria, and emotion. %5 are from the NOSS~\cite{trill} benchmark and 2 additional tasks described in \cite{cap12} and \cite{asremb}
%In order to evaluate distilled representations, we compare their performances on 7 tasks, 5 in the ``Non-Semantic Speech Benchmark (NOSS)" and 2 additional tasks described in \cite{cap12} and \cite{asremb}. %We follow the evaluation criteria described in \cite{cap12}, and summarize here for completeness (see Fig~\ref{fig:eval_flow} and Tab~\ref{tab:eval} for an overview).
As depicted in Fig~\ref{fig:eval_flow}, for each (model, eval dataset) pair, we first generate the candidate embeddings for the train, dev, and test splits. We then train three types of linear models on the train set embeddings. We take the model that performs best on the dev set, and report the model's performance on the test set as the score for that (model, eval dataset) pair. %When a single overall score is required (e.g. to compare embeddings), we aggregate by averaging d-prime scores (Sec.~\ref{subsubsec:dprime}) across tasks.

% Floating fig.
%\begin{table}[!t]
\captionsetup{size=footnotesize}
\centering
\caption{Model hyparameters. Most but not all combinations are valid, so ``total models" isn't a strict carteisian product of the hparams.}\label{tab:train_hp}
\footnotesize
\begin{tabular}{r r l } \toprule
  Architecture & Hyperparameter & Values \\
  \midrule
  \textbf{AST} & Depths  & 4, 6, 12, 24 \\
               & Filter width & 512, 1024, 2048 \\
               & Chunk size (s)  & 2, 3 \\
               & Target matching  & local, global \\
               & \textbf{Total models} & 56 \\
  
  \midrule
  \textbf{EfficientNet} & Version & 1, 2 \\
               & Size    & 0-6/7 \\
               & Chunk size (s)  & 2, 3 \\
               & Target matching  & local, global \\
               & \textbf{Total models} & 60 \\
               \midrule
  \textbf{Resnetish} & Depths  & 8-12\\
        & Filter scale  & 1.0, .75, 0.5, .25 \\
        & Chunk size (s)  & 2, 3 \\
        & Target matching  & local, global \\
  & \textbf{Total models} & 20 \\
 \bottomrule
\end{tabular}
\end{table}
%\vspace{-5em}
\vspace{-2mm}
\subsubsection{Aggregation metric: Equivalent D-Prime ($d'$)}
\label{subsubsec:dprime}
To fairly compare with previous results, we report on the typical performance metric for each dataset (accuracy or equal error rate). However, to aggregate performance into a single scalar, for the purpose of comparing embedding quality, we average the ``equivalent d-prime" ($d'$) metric, defined as follows: 
\begin{equation}
    d' = \sqrt{2}Z(AUC)
\end{equation}
where ``AUC" is the ``Area Under the Receiver Operating Characteristic Curve" (ROC AUC), and ``$Z()$" is the inverse Cumulative Distribution Function for the standard normal distribution.

$d'$ is better suited to aggregate performance on multiple datasets for two reasons. First, ROC AUC and $d'$, unlike accuracy, take into account performance at various levels along the accuracy/recall curve. This makes it more reflective of the overall performance of a particular representation. Second, unlike ROC AUC, $d'$ doesn't saturate in the highly performant regime. Thus, 1 unit of $d'$ is in some sense more equivalent, so averaging $d'$ values across datasets is more natural than averaging AUC values. We use the average $d'$ scores to sort and select models.

\subsection{Models for comparison}
\label{subsubsec:baselines}
To contextualize performance on the NOSS benchmark, we compare our models to 1) previous state-of-the-art (SoTA) results, 2) publicly available speech representation models, and 3) CAP12 with different input sizes. Previous SoTA models are been mostly domain-specific. The public models we compare to are:

\begin{enumerate}\itemsep0em
    \item \textbf{Hugging Face's Wav2Vec \text{2.0}}~\cite{baevski2020wav2vec,transformers_lib}: This model was trained on the approximately 1K hour Librispeech~\cite{librispeech}. We use the TensorFlow model from the Hugging Face library. We compute the performances for each layer, and find that \textbf{layer 6 of 11 performs the best overall}.
    \item \textbf{TRILL}~\cite{trill}: A Resnet triplet-loss model trained on AudioSet. We access this model from TensorFlow Hub.
    \item \textbf{YAMNet}~\cite{hershey2017cnn}: This is a supervised audio classification network, also trained on AudioSet. We use layer 19 as in \cite{trill}. We access this model through TensorFlow Hub.
\end{enumerate}

%We make three comparisons to the teacher CAP12 model. As in \cite{bigssl}, we report NOSS numbers on CAP12 when it is provided the entire audio clip. We also report results when embeddings are computed on 2 and 3 second chunks, and then averaged over the clip. These last two comparisons explore the importance of limited context window on generating high-quality paralinguistic representations.

%\subsection{Knowledge Distillation}
%\label{sec:kd}

%To create small TRILLsson models, we perform  student training by matching CAP12~\cite{cap12} features on a collection of open source datasets (Tab. \ref{tab:train_ds}).
%We experiment with different context window sizes and two methods for generating targets (Sec.~\ref{subsec:matching}), and we explore a variety of student architectures (Sec.~\ref{subsubsec:students}).
%The list of hyperameters can be found in Tab.~\ref{tab:train_hp}


%\subsubsection{Distillation teacher: CAP12}

%The CAP12 model serves as the teacher. It is a 606M parameter conformer model. CAP's training objective is a modified Wav2Vec 2.0 self-supervised training loss on a 900M+ hour speech dataset derived from YouTube (YT-U~\cite{bigssl}, Tab.~\ref{tab:train_ds}). One intermediate layer, layer 12, shows particularly impressive performance on a wide range of paralinguistic tasks~\cite{cap12}. We refer to this 1024 dimension representation of speech as ``CAP12."

%\subsubsection{Distillation targets: global vs local matching}
%\label{subsec:matching}

%We explore two training paradigms that borrow ideas from distilling large vision models~\cite{caron2021emerging}. In the first method, which we call \textbf{``local matching"}, the student is tasked with matching the teacher on small chunks of audio e.g. 2 seconds. The training audio is chunking into the minimum size that the student processes as once, and the target is generated by averaging the teacher's output on that clip. In \textbf{``global matching"}, the student takes minimally chunked audio in the same way, but the target is the average teacher embedding from the entire, pre-chunking clip. This process is compared in Fig~\ref{fig:training}.

%We note that ``local matching" is a harder problem, in the sense that perfectly solving it implies that a network has also learned to do global matching. However, we also note that ``local matching" is harder than necessary; perfectly solving the ``global matching" problem would allow a student to reproduce a teacher's performance on our downstream evaluation tasks.

%\subsubsection{Distillation student architectures}
%\label{subsubsec:students}

%e experiment with three types of student architectures, and different hyperparameters for each. For all models, we experiment with 2 sizes of input (2 seconds and 3 seconds) and two types of targets (global matching and local matching).

%\begin{enumerate}
%    \item \textbf{AST}~\cite{ast}: The Audio Spectrogram Transformer (AST) is a convolution-free, transformer-based model for audio classification. We train student models with different depths and widths.
    
%    \item \textbf{EfficientNet/EfficientNetv2}: EfficientNetV2~\cite{efficientnet,efficientnetv2} is the result of neural architecture search on image classification. The architecture is designed to be high performing and mobile friendly. Different versions of this architecture involve different model sizes, which vary in terms of depths and filters.
    
%    \item \textbf{Resnetish}~\cite{hershey2017cnn}: This architecture is a Resnet architecture designed to take audio spectral features as input. Different versions of this architecture include different depths and a different number of filters per layer.
    
%\end{enumerate}

%\subsubsection{Training datasets}
%To create the distilled TRILLsson models, we perform  student training by matching teacher features on a collection of open source datasets (Tab. \ref{tab:train_ds}).

%\begin{enumerate}

%\item \textbf{Audioset}~\cite{audioset}: An audio event dataset. These clips are collected from YouTube, so it represents a variety of settings and acoustic environments. We use the speech subset of this data, which yields approximately 5K hours.

%\item \textbf{Libri-Light}~\cite{librilight}: Contains 60k hours of audio derived from open-source audio books in the LibriVox project. It is the largest publicly available, unlabeled semi-supervised dataset to date.

%\item \textbf{YT-U}~\cite{bigssl} (teacher network only): A 900k hour dataset derived from YouTube. YT-U is built by first randomly collecting 3 million
%hours of audio from ``speech-heavy" YouTube videos. The results are then segmented, and the non-speech segments are removed to yield approximately 900k hours of unlabeled audio data.

%\end{enumerate}

\section{Results}
\label{sec:results}

\section{Experimental Evaluation}

Our experiments focus on online meta-learning of image classification tasks. In these settings, an effective algorithm should adapt to changing tasks as quickly as possible, rapidly identify when the task has changed, and adjust its parameters accordingly. Furthermore, a successful algorithm should make use of past task shifts to meta-learn effectively, thus accelerating the speed with which it adapts to future task changes. In all cases, the algorithm receives one data point at a time, and the task changes periodically. In order to allow for a comparison with prior methods, which generally assume known task boundaries, the data stream is partitioned into discrete tasks, but our algorithm is not aware of which datapoint belongs to which tasks or where the boundaries occur. The prior methods, in contrast, \emph{are} provided with this information, thus giving them an advantage. We first describe the specific task formulations, and then the prior methods that we compare to.

Online meta-learning algorithms should adapt to each task as quickly as possible, and also use data from past tasks to accelerate acquisition of future tasks. Therefore, we report our results as a learning curve, with one axis corresponding to the number of seen tasks, and the other axis corresponding to the cumulative error rate on that task. This error rate is computed using a held-out validation data for each task after the adaptation that task. 

We evaluate prior online meta-learning methods and baselines on three different datasets (Rainbow-MNIST, CIFAR100 and CELEBA). TOE (train on everything), TFS (train from scratch), FTL (follow the leader), FTML (follow the meta-leader)~\citep{finn19a}, LwF~\citep{li2017learning}, iCaRL~\citep{rebuffi2016icarl} and MOCA~\citep{harrison2020continuous} are the baseline methods we compare against our method FOML. See Section 4 for more detailed description of these methods.

\noindent \textbf{Datasets:} We compare TOE, TFS, FTL, FTML, LwF, iCaLR and FOML on three different datasets. Rainbow-MNIST~\citep{finn19a} was created by changing the background color, scale and rotation of the MNIST dataset. It includes 7 different background colors, 2 scales (full and half) and 4 different rotations. This leads to a total of 56 number of tasks. Each individual task is to classify the images into 10 classes. We use the same partition with 900 samples per each task, as in prior work~\citep{finn19a}. However, this task contains relatively simple images, and only 56 tasks. To create a much longer task sequence with significantly more realistic images, we modified the CIFAR-100 and CELEBA datasets to create an online meta-learning benchmark, which we call online-CIFAR100 and online-CELEBA, respectively. Every task is a randomly sampled set of classes, and the goal is to classify whether two images in this set belongs to same class or not. Specifically, each task corresponds to 5 classes, and every datapoint consists of a \emph{pair} of images, each corresponding to one of the 5 classes for that task. The goal is to predict whether the two images belong to the same class or not. Note that different tasks are not mutually exclusive, which \emph{in principle} should favor a TOE-style method, since meta-learning is known to underperform with non-mutually-exclusive tasks~\citep{yin2019meta}. To make sure the data distribution changes smoothly over tasks, we only change a subset of the classes between consecutive tasks. This allow us to create a very large number of tasks (1200 for online-CIFAR100), and evaluate our method and the baselines on much longer and more realistic task sequences.

\begin{figure*}[!t]
\centering
    \begin{subfigure}{0.32\textwidth}
        \centering
        \includegraphics[width=0.95\textwidth]{figs/exp4x.pdf}
    \end{subfigure} 
    \begin{subfigure}{0.32\textwidth}
        \centering
        \includegraphics[width=0.95\textwidth]{figs/exp_cifar_1_main.pdf}
    \end{subfigure}
    \begin{subfigure}{0.32\textwidth}
        \centering
        \includegraphics[width=0.95\textwidth]{figs/exp_celeba_1_main.pdf}
    \end{subfigure}
    \caption{\footnotesize{ \emph{Comparison between online algorithms:} We compare our method with baselines and prior approaches, including TFS (Train from Scratch), TOE (Train on Everything), FTL (Follow the Leader) and FTML (Follow the Meta Leader). \textbf{a:} Performance relative to the number of tasks seen over the course of online training on the Rainbow-MNIST dataset. As the number of task increases, FOML achieves lower error rates compared to other methods. We also compare our method with continual learning baselines: LwF~\citep{li2017learning}, iCaRL~\citep{rebuffi2016icarl} and MOCA~\citep{harrison2020continuous}. MOCA~\cite{harrison2020continuous} archive similar performance to ours at the end of the learning, but FOML makes significantly faster progress. \textbf{b:} Error rates on the Online-CIFAR100 dataset. Note that FOML not only achieves lower error rates on average, but also reaches the lowest error (of around 17\%) more quickly than the other methods. \textbf{c:} Performance of FOML on the CELEBA dataset. This dataset contains more than 1000 classes, and we follow the same protocol as in Online-CIFAR100 experiments. Our method, FOML, learns more quickly on this task as well.}}
    \label{fig:main_plots}

\end{figure*}





\noindent \textbf{Implementation Details:} We use a simple 4 layer convolutional neural network with 8,16,32,64 filters at each layer for both experiments. However, for the CIFAR-100 experiments, a Siamese version of the same network is used. All the methods were trained via a cross-entropy loss, with their best performing hyper-parameters, on a single NVIDA-2080 GPU machine. Please see Appendix~\ref{sec:A1} for more details on hyper-parameters and network architecture.


\noindent \textbf{Results on Rainbow-MNIST:} As shown in Fig~\ref{fig:main_plots}, FOML attains the lowest error rate on most tasks in Rainbow-MNIST, except a small segment in the very beginning. The performance of TFS is similar performance across all the tasks, and does not improve over time. This is because it resets its weights every time it encounters a new task, and therefore cannot not gain any advantage from past data. TOE has larger error rates at the start, but as we add more data into the buffer, TOE is able to improve. On the other hand, both FTL and FTML start with similar performance, but FTML achieve much lower error rates at the end of the sequence compared to FTL, consistently with prior work~\citep{finn19a}. The final error rates of FOML are around 10\%, and it reaches this performance significantly faster than FTML, after less than 20 tasks. Note that FTML also has access to task boundaries, while FOML does not.

\noindent \textbf{Results on Online-CIFAR100 and Online-CELEBA:} We use a Siamese network for this experiment, where each image is fed into a 7-layer convolutional network, and each branch outputs a 128 dimensional embedding vector. A difference of these vectors are fed into a fully connected layer for the final classification. Each task contains data from 5 classes, and each new task introduces three new classes, and retains two of the classes from the previous task, providing a degree of temporal coherence while still requiring each algorithm to handle persistent shift in the task. Fig~\ref{fig:main_plots} shows the error rates of various online learning methods, where each method is trained over a sequence of 1200 tasks for online-CIFAR100. All the methods start with initial error rates of 50\%. The tasks are not mutually exclusive, so in principle TOE can attain good performance, but it makes the slowest progress among all the methods, suggesting that simple pretraining is not sufficient to \emph{accelerate} learning. FTL uses a similar pre-training strategy as TOE. However it has an adaptation stage where the model is fine-tuned on the new task. This allows it to make slower progress.
As expected from prior work~\citep{finn19a}, the meta-learning procedure used by FTML allows it to make faster progress than FTL. However, FOML makes faster progress on average, and achieves the lowest final error rate ($\sim15\%$) after sequence of 1200 tasks. 

\vspace{-0.2cm}
\subsection{Ablation Studies}
\vspace{-0.2cm}
We perform various ablations by varying the number of online parameters used for the meta-update $K$, importance of meta-model to analysis the properties of our method.

\noindent \textbf{Number of online parameters used for the meta-update:} Our method periodically updates the online weights and meta weights. The meta-updates involves taking $K$ recent online parameters and updating the meta model via MAML gradient. Therefore, meta-updates depend on the trajectory of the online parameters. In this experiment, we investigate how the performance of FOML changes as we vary the number of parameters used for the meta-update ($K$ in Algorithm~\ref{alg:OML}). Fig~\ref{fig:ablations} shows the performance of our method with various values of $K$: $K=[1,2,3,5,10]$. We can see that the performance improves when we update the meta parameters over longer trajectory of online parameters (larger $K$). We speculate that this is due to the longer sequences providing a clearer signal for how the meta-parameters influence online updates over many steps.

\noindent \textbf{Importance of meta update:} FOML keeps track of separate online parameters and meta-parameters, and each of them is updated via corresponding updates. However, only the online parameters $\phi$ are used for evaluation, while the meta-parameters $\theta$ only influence them via the regularizer, and have no direct effect on evaluation performance. This might raise the question: how important is the contribution of the meta-parameters to the performance of the algorithm during online training? We train a model with and without meta-updates, and the performance is shown in Fig~\ref{fig:ablations}. None that, the model without meta-updates is identical to our method, except that the meta-updates themselves are not performed. We can clearly see that the model trained with meta-updates preforms much better than a model trained without meta-updates. The model trained without meta-updates generally does not improve significantly as more tasks are seen, while the model trained with meta-updates improves with each task, and reaches significantly lower final error. This shows that, even though $\theta$ and $\phi$ are decoupled and only connected via a regularization, the meta-learning component of our method really is critical for its good performance.

\begin{figure}[!t]
    \centering
    \begin{minipage}[]{0.45\textwidth}
        \includegraphics[width=0.95\columnwidth]{figs/test5.pdf}
    \end{minipage}
    \begin{minipage}[]{0.45\textwidth}
        \includegraphics[width=0.95\columnwidth]{figs/test7.pdf}
    \end{minipage}
    \caption{\footnotesize{\emph{Ablation experiments:} \textbf{a)} We vary the number of online updates $K$ used before the meta-update, to see how it affects the performance of our method. The performance of FOML improves as the number of online updates is increased. \textbf{b)} This experiment shows how FOML performs with and without meta updates, to confirm that the meta-training is indeed an essential component of our method. With meta-updates, FOML learns more quickly, and performance improves with more tasks.}}
    \label{fig:ablations}
\end{figure}



\vspace{-0.4cm}
\section{Conclusion}
\vspace{-0.4cm}
We presented FOML, a MAML-based algorithm for online meta-learning that does not require ground truth knowledge of task boundaries, and does not require resetting the parameter vector back to the meta-learned parameters for every task. FOML is conceptually simple, maintaining just two parameter vectors over the entire online adaptation process: a vector of online parameters $\phi$, which are updated continually on each new batch of datapoints, and a vector of meta-parameters $\theta$, which are updated correspondingly with meta-updates to accelerate the online adaptation process, and influence the online updates via a regularizer. We find that even a relatively simple task sampling scheme that selects datapoints at random from a buffer of all seen data enables effective meta-training that accelerates the speed with which FOML can adapt to each new task, and we find that FOML reaches a final performance that is comparable to or better than baselines and prior methods, while learning to adapt quickly to new tasks significantly faster. While our work focuses on supervised classification problems, a particularly exciting direction for future work is to extend such online meta-learning methods to other types of online supervision that may be more readily available, including self-supervision and prediction, so that models equipped with online meta-learning can continually improve as they see more of the world.


\vspace{-3mm}

\section{Conclusion}
\label{sec:conclusion}
\section{Conclusion}
In this work, we present a novel strategy for addressing few-shot open-set recognition. We frame the few-shot open-set classification task as a meta-learning problem similar to \cite{peeler}, but unlike their strategy, we do not solely rely on thresholding softmax scores to indicate the openness of a sample. We argue that existing thresholding type FSOSR methods \cite{peeler,snatcher} rely heavily on the choice of a carefully tuned threshold to achieve good performance. Additionally, the proclivity of softmax to overfit to unseen classes makes it an unreliable choice as an open-set indicator, especially when there is a dearth of samples. Instead, we propose to use a reconstruction of exemplar images as a key signal to detect out-of-distribution samples. 
The learned embedding which is used to classify the sample is further modulated to ensure a proficient gap between the seen and unseen class clusters in the feature space. Finally, the modulated embedding, the softmax score, and the quality reconstructed exemplar are jointly utilized to cognize if the sample is in-distribution or out-of-distribution. 
The enhanced performance of our framework is verified empirically over a wide variety of few-shot tasks and the results establish it as the new state-of-the-art. In the future, we would like to extend this approach to more cross-domain few-shot tasks, including videos.
\vspace{-2em}
\section{Acknowledgement}
This work was partially supported by US National Science Foundation grant 2008020 and US Office of Naval Research grants N00014-19-1-2264 and N00014-18-1-2252.
\vspace{-1em}

\vspace{-3mm}
\section{Acknowledgements}
We want to thank Aren Jansen, Channing Moore, Wei Han, Daniel Park, Yu Zhang, and the entire author list of \cite{bigssl}, without which this work wouldn't have been possible.


\bibliographystyle{IEEE}
\bibliography{biblio}
%\bibliography{strings,refs}

%\section{Appendix}
%% !TEX root = ../supp.tex
% !TEX spellcheck = en-US

\section{Physical Rendering of SwissCube}
\label{sec:appendix}

Although the European Space Agency has organized a satellite pose estimation challenge and released the SPEED satellite dataset, the unavailability of the target 3D model makes the pose accuracy not depending on the pose estimation method alone. Furthermore, the limited varieties of lighting also make it soon saturated and less discriminative, as discussed in Section~\ref{sec:related}.

To fully demonstrate the effectiveness of our method in space, we introduce the Swisscube satellite dataset. Swisscube is a Cubesat-type satellite which was designed at EPFL and launched in 2009. Given the accurate CAD files and material properties of each component of it, we synthesize photorealistic images using physically based rendering~\cite{xx}.

 
% !TEX root = ../top.tex
% !TEX spellcheck = en-US

\begin{table}
    \centering
    \begin{small}
    % \rowcolors{2}{white}{gray!10}
    \begin{tabular}{lcc}
        \toprule
        &	SPEED & {\bf CubeSat}\\
        \midrule
        % Synthetic  &12k & 40k \\
        % Real       & 5 & 300 \\
        % \midrule
        Size & 12k & 40k \\
        Accurate 3D model   &\xmark  & \cmark  \\
        Complex lighting    &\xmark  & \cmark \\
        Physical modeling    &\xmark  & \cmark \\
        Colors        &\xmark  & \cmark \\
        Sequences         &\xmark  & \cmark  \\
        Depth distribution & non-uniform & uniform \\
        \bottomrule
    \end{tabular}
    \end{small}
    % \vspace{-3mm}
    \caption{{\bf CubeSat dataset.} bla bla bla bla bla bla bla bla bla bla bla bla bla bla bla bla bla bla bla bla bla bla bla bla bla bla bla bla bla bla bla bla bla bla bla bla bla bla bla bla bla bla bla bla bla bla bla bla bla bla bla bla bla bla bla bla bla bla bla bla bla bla bla bla bla bla bla bla bla bla bla bla bla bla bla bla bla bla bla bla }
    \label{tab:swisscube_vs_speed}
\end{table} 

% -----------------------------------
% Physically-based spectral rendering

\subsubsection{Physically-based spectral rendering}

In this section we provide a high-level description of the Swisscube satellite dataset which collects 40'000  physically-based synthetic images. While the SPEED satellite dataset images were produced using an OpenGL-based RGB rendering framework, we opted for a physically-based approach, where every element of the rendering pipeline were carefully modeled to mimic reality.

While the RGB model is often used to render color images, using tristimulus RGB colors in the rendering simulation generally yields non-physical results. For instance, surface reflectance properties of an object can be highly dependent on the wavelengths, which won't be accurately reproduced with RGB values. In a spectral renderer, colors are represented as spectral power distributions, resulting in improved accuracy especially when measured spectral data is available. For the Swisscube dataset, using a spectral renderer was a necessity as it was a requirement to correctly model the spectral responses of the solar irradiance emission, material reflectance properties and Earth surface radiance. Although the rendering simulation uses spectral colors, the resulting images will be converted to RGB images.

Relying on a physically-based rendering pipeline also gives us more control on the dynamic range of the output images. Thus we were able to accurately reproduce highlights orders of magnitude brighter than darker region of the images. An appropriate gamma curve could then be applied to produce images that can be viewed on regular displays.

To achieve all of this, we build our pipeline around the Mitsuba 2 renderer [???] which is a highly modular open-source framework that supports spectral rendering.

% -------------------
% The 3D / CAD model

\subsubsection{Accurate 3D model from CAD data}

For this dataset, we modelled every mechanical parts of the SwissCube, such as solar panels, antennas, and screws based directly on the raw CAD files. We carefully assign material reflection properties to each part. Given the physically-based nature of the pipeline, it would be possible to use efficient material acquisition technique such as [???] in the future for better results. Due to confidential reasons, we only release the mesh geometry data of the combined SwissCube without separable pieces to the public, which is enough for perfect registration.

% Citation for material acquistion technique: https://rgl.epfl.ch/publications/Dupuy2018Adaptive

% --------------------
% Physically-based Sun

\subsubsection{Modeling a physically-based Sun emitter}

In order to correctly model the illumination from the Sun, we leveraged the vast literature in astrophysics. As the target object will be placed above the Earth atmosphere, it is not enough to use specular solar irradiance measurements made at ground surface [???] as those will be be affected by the highly variable and absorbing constituents of the Earth atmosphere. Instead, we rely on the air mass zero reference spectrum [???], also known as extraterrestrial solar irradiance, mainly based on data from satellites and space shuttle missions. Figure \ref{fig:swisscube_sun_spectrum} show its spectral power distribution. We then use a point light source to represent the Sun, placed at the correct distance to the Earth. Note that is was necessary to scale the Sun irradiance to account for its surface area.

% Groud surface citation: https://www.osapublishing.org/ao/abstract.cfm?uri=ao-21-3-554
% ASTM Standard Extraterrestrial Spectrum Reference: https://www.nrel.gov/grid/solar-resource/spectra-astm-e490.html

% !TEX root = ../top.tex
% !TEX spellcheck = en-US

\begin{figure}[t]
    \begin{center}
    \includegraphics[width=1.0\linewidth]{./fig/swisscube_sun_spectrum/sun_spectrum_plot.jpeg}
    % \fbox{\rule{0pt}{2in} \rule{0.25\linewidth}{0pt}}
    \end{center}
    \vspace{-6mm}
    \caption{{\bf Air mass zero solar spectral power distribution.} 
    }
    \label{fig:swisscube_sun_spectrum}
\end{figure}


% ---------------------------
% Modeling the stars / galaxy

\subsubsection{Modeling the stars and galaxies}

We also added galaxies and other astronomical objects to our pipeline as we believe those could distract the learning algorithm. Based on the HYG database star catalogue [???], we could generate a high-resolution environment map that we later used as an second emitter. The HYG database contains around 220 thousands astronomical objects, mostly galaxies but also star clusters and Nebulae along with information regarding their position and brightness. Figure \ref{fig:swisscube_stars_envmap} shows the astronomical object projected on a spherical coordinate 2D map with their respective physically-based brightness.

Compared to the sun illumination, the irradiance coming from the stars is orders of magnitude lower. On the other hand, to maximize the diversities of the generated data, we decided to increase the actual brightness of each star in the galaxies to make them more apparent in rendering, which we think is a beneficial perturbation for a dataset.

% !TEX root = ../top.tex
% !TEX spellcheck = en-US

\begin{figure}[t]
    \begin{center}
    \includegraphics[width=1.0\linewidth]{./fig/swisscube_stars_envmap/stars_envmap.jpeg}
    % \fbox{\rule{0pt}{2in} \rule{0.25\linewidth}{0pt}}
    \end{center}
    \vspace{-6mm}
    \caption{{\bf Astronomical objects environment map based on the HYG dataset.} 
    }
    \label{fig:swisscube_stars_envmap}
\end{figure}


% HYG database link: http://www.astronexus.com/hyg

% ---------------------------------------------------
% Modeling the Earth radiance using the VIIRS dataset

\subsubsection{Spectral Earth radiance using the VIIRS dataset}

Properly modeling the Earth is very important here as it often occupies a large portion of the images. Moreover, the Sun light reflecting off the Earth is drastically affecting the illumination of the target object. In our pipeline, the Earth is a represented as a very large sphere, reflecting light coming from the Sun emitter onto the target object or directly towards the camera. Based on the NASA Visible Infrared Imaging Radiometer Suite (VIIRS) Level-1B data products [????], we generated a spectral radiance texture to model the reflectance of the Earth and its atmosphere. The VIIRS data products are produced by whiskbroom scanning radiometers on satellites orbiting around the Earth at a nominal altitude of 829 km, providing a full daily coverage of the Earth. These data products include 6 bands in the visible spectrum with high spatial resolution which we could use to generate a spectral reflectance texture. Figure \ref{fig:swisscube_earth_renders} shows the result of this process compared to the use of a simple Earth albedo texture available from the NASA website [????].

% !TEX root = ../supp.tex
% !TEX spellcheck = en-US

\begin{figure}[t]
    \centering
    \begin{tabular}{ccc}
    \includegraphics[height=2.4cm]{fig/swisscube_earth_renders_L1/earth_L1_albedo_texture.png}&
    \includegraphics[height=2.4cm]{fig/swisscube_earth_renders_L1/earth_L1_spectral_texture.png}&
    \includegraphics[height=2.4cm]{fig/swisscube_earth_renders_L1/earth_L1_DSCOVR.png}\\
    (a)&(b)&(c)\\
    \end{tabular}
    \vspace{-3mm}
    \caption{\small {\bf Rendered Earth compared to DSCOVR photograph.} (a) Ground-level albedo texture doesn't account for the scattering effects introduced by the atmosphere, resulting in over saturated colors when viewed from space. (b) Rendering of the Earth using our spectral texture based on the VIIRS data products. (c) Ground-truth real photograph taken by the DSCOVR satellite at the L1 Lagrange point.}
    \label{fig:swisscube_earth_renders}
\end{figure}

% VIIRS website: https://earthdata.nasa.gov/earth-observation-data/near-real-time/download-nrt-data/viirs-nrt#ed-corrected-reflectance
% Earth albedo texture: https://visibleearth.nasa.gov/images/57735/the-blue-marble-land-surface-ocean-color-sea-ice-and-clouds/57737l
% Real image DSCOVR website: https://www.nesdis.noaa.gov/content/dscovr-deep-space-climate-observatory

\subsubsection{Bring everything together at real scale}

We placed all those elements in a virtual scene at the real scale to ensure high fidelity images and produce a more comprehensive dataset. We could then generate sequences of images, simulating various docking procedures by varying camera and target vehicle poses and respective speed. Each sequence contains 100 consecutive images.



To achieve highest reflection of the real world, we place the SwissCube in the actual-working orbits about 700 km above the Earth's surface during rendering and model most of the space-borne items, such as the Sun, the Earth, and galaxies, physically. 

Depth range.

\subsubsection{Dataset generation and specs}

We generate 400 sequences in total, each sequence contains 100 consecutive images with random angular speed between 0.xx to 0.xx.

Fig.~\ref{fig:swisscube_vs_speed} shows some examples of the generated SwissCube dataset.

bla bla bla bla bla bla bla bla bla bla bla bla bla bla bla bla bla bla bla bla bla bla bla bla bla bla bla bla bla bla bla bla bla bla bla bla bla bla bla bla bla bla 

As discussed Section~\ref{sec:related}. Table~\ref{tab:swisscube_vs_speed} shows the comparison between SwissCube and SPEED dataset. Note that, in SPEED dataset, there are 2998 more synthetic images and also 300 more real images in the test set. However, Their ground truth labels are not accessible, so here we do not take them into account.

\YH{TODO}

From the total 400 sequences of images in the SwissCube dataset, we take all the images from the first 300 sequences as our training set and that from the last 100 as the test. In this subsection, we will evaluate the methods' performance in different depth ranges, that is, the whole depth range will be divided to three regions denoted as {\it Near}, {\it Medium} and {\it Far}, which correspond to depth range [1d-4d], [4d-7d] and [7d-10d], respectively.


\end{document}
