\section{Uniqueness approaches}\label{sec:uniqueness}
Choose, for simplicity the excitations $u_{0i}=0$, $u_{1i}=0$, $u_{2i}=0$, $f_i\neq0$ and assume that 
for two models given by 
\begin{equation}\label{eqn:model}
\begin{aligned}
%&N, \ c, \ \mathcal{A}, \ c_{k\ell}, \ \alpha_k, \ \beta_{k\ell}, \ B_i,\\ 
%&\tilde N, \ \tilde c, \ \tilde{\mathcal{A}}, \ \tilde{c}_{k\ell}, \ \tilde{\alpha}_k, \ \tilde \beta_{k\ell}, \ \tilde B_i,\\ 
&N, \ J, \ c, \ \mathcal{A}, \ b_{k}, \ \alpha_k, \ \beta_{k}, \ d_j, \ \gamma_j,\ B_i,\\ 
&\tilde N, \ \tilde J, \ \tilde c, \ \tilde{\mathcal{A}}, \ \tilde{b}_{k}, \ \tilde{\alpha}_k, \ \tilde \beta_{k}, \ \tilde d_j, \ \tilde \gamma_j,\ \tilde B_i,
\end{aligned}
\end{equation}
we have equality of the observations
\[
B_i u(t) = \tilde B_i \tilde u(t)\, \quad t\in(0,T)
\]
from two possibly different space parts of the excitation $f_i$, $\tilde{f}_i$, and possibly using two different test specimen characterised by $\mathcal{A}$, $\tilde{\mathcal{A}}$,
while the temporal part $\sigma$ of the excitation is assumed to be the same in both experiments, and its Laplace transform to vanish nowhere $\hat{\sigma}(s)\neq0$ for all $s$.
By analyticity (due to the multinomial Mittag-Leffler functions reresentation cf. \cite[Theorem 4.1]{LuchkoGorenflo:1999})
we get equality for all $t>0$, thus equality of the Laplace transforms
\begin{equation}\label{eqn:BaBb}
\begin{aligned}
&\sum_{n=1}^\infty \frac{\langle f_{i},\varphi_n\rangle B_i\varphi_n}{
s^2+c^2\lambda_n+\sum_{j=1}^J d_j s^{\gamma_j+2}+\sum_{k=1}^N s^{\alpha_k} b_{k}(\lambda_n)^{\beta_{k}}}\\
&\qquad=
\sum_{n=1}^\infty \frac{\langle {\tilde f}_{i},{\tilde \varphi}_n\rangle {\tilde B}_i {\tilde \varphi}_n}{
s^2+{\tilde c}^2 {\tilde \lambda}_n+\sum_{j=1}^{\tilde{J}} \tilde{d}_j s^{\tilde{\gamma}_j+2}+\sum_{k=1}^{\tilde N} s^{{\tilde \alpha}_k} {\tilde b}_{k}({\tilde \lambda}_n)^{{\tilde \beta}_{k}}}, \quad i=1,\ldots I\,, \quad s\in\mathbb{C}_\theta
\end{aligned}
\end{equation}
in some sector $\mathbb{C}_\theta$ of the complex plane,
due to our assumption that $\hat{\sigma}(s)=\hat{\tilde{\sigma}}(s)$ and $\hat{\sigma}(s)\neq0$ for all $s$.
We have to disentangle the sum over $n$ with the asymptotics for $s\to\infty$ in order to recover the unknown quantities in \eqref{eqn:model} or at least some of them.

\subsection{Smooth excitations}

We make use of the Weyl estimate $\lambda_n \approx C_d\, n^{2/d}$ in
$\mathbb{R}^d$ as well as additional regularity of the excitation to approximate
the sum over $n$ \eqref{eqn:BaBb} of fractions by a single fraction of the form
\[
\sum_{n=1}^\infty \frac{\langle f_{i},\varphi_n\rangle B_i\varphi_n}{\omega(s,\lambda_n)}
\approx\frac{\sum_{n=1}^\infty\langle f_{i},\varphi_n\rangle B_i\varphi_n}{\omega(s,\lambda_1)}
\mbox{ with } \omega(s,\lambda) = s^2+c^2\lambda+\sum_{j=1}^J d_j s^{\gamma_j+2}+\sum_{k=1}^N s^{\alpha_k} b_{k}\lambda^{\beta_{k}}
\] 
for large real positive $s$.
We focus exposition on the source excitation case $u_\ell=0$, $\ell=0,1,2$, $f\neq0$ from \eqref{eqn:BaBb}, the other cases follow similar. Also, since we only need one excitation here $I=1$, we just skip the subscript $i$.
Indeed, for $s\in\mathbb{R}$, $s\geq1$ we can estimate the difference
\[
\begin{aligned}
D:=&\left|
\frac{\sum_{n=1}^\infty\langle f,\varphi_n\rangle B\varphi_n}{\omega(s,\lambda_1)}-
\sum_{n=1}^\infty \frac{\langle f,\varphi_n\rangle B\varphi_n}{\omega(s,\lambda_n)}
\right|\\
&=\left|\sum_{n=1}^\infty
\frac{\langle f,\varphi_n\rangle B\varphi_n\,(c^2(\lambda_n-\lambda_1)+\sum_{k=1}^N s^{\alpha_k} b_{k}((\lambda_n)^{\beta_{k}}-(\lambda_1)^{\beta_{k}}))}{\omega(s,\lambda_n)\, \omega(s,\lambda_1)}\right|\\
&\leq \sup_{n\in\mathbb{N}} \{\lambda_n^{-\nu}|B\varphi_n|\}\, 
\|f\|_{\dot{H}^{2\mu+2\nu}(\Omega)}
\left(\sum_{n=1}^\infty\left(
\frac{c^2\lambda_n+\sum_{k=1}^N s^{\alpha_k} b_{k}(\lambda_n)^{\beta_{k}}}{
\lambda_n^\mu s^2(s^2+c^2\lambda_n)}\right)^2\right)^{\!1/2}\\ 
&\leq \sup_{n\in\mathbb{N}} \{\lambda_n^{-\nu}|B\varphi_n|\}\, 
\|f\|_{\dot{H}^{2\mu+2\nu}(\Omega)}
%s^{-(2-\alpha_N+\bar{\alpha})} \frac{c^2+\sum_{k=1}^N b_{k}(\lambda_1)^{\beta_{k}-1}}{
%(\tfrac{2}{\bar{\alpha}})^{\frac{\bar{\alpha}}{2}} 
%(\tfrac{2c^2}{2-\bar{\alpha}})^{\frac{2-\bar{\alpha}}{2}}}
%\left(\sum_{n=1}^\infty\lambda_n^{-2(\mu-1+\frac{2-\bar{\alpha}}{2})}\right)^{\!1/2} 
s^{-(2-\alpha_N+2/p)} \frac{c^2+\sum_{k=1}^N b_{k}(\lambda_1)^{\beta_{k}-1}}{
p^{1/p}(\tfrac{p}{p-1}c^2)^{\frac{p-1}{p}}}
\left(\sum_{n=1}^\infty\lambda_n^{-2(\mu-1/p)}\right)^{\!1/2} 
\end{aligned}
\]
where we have used Young's inequality to estimate
\[
s^2+c^2\lambda_n \geq 
%s^{\bar{\alpha}}(\tfrac{2}{\bar{\alpha}})^{\frac{\bar{\alpha}}{2}} 
%(\tfrac{2c^2}{2-\bar{\alpha}}\lambda_n)^{\frac{2-\bar{\alpha}}{2}}
s^{2/p}p^{1/p}(\tfrac{p}{p-1}c^2\lambda_n)^{\frac{p-1}{p}}.
\]
Since $\lambda_n \sim n^{2/d}$, the sum converges for $2(\mu-\frac{1}{p})>\frac{d}{2}$. 
On the other hand, to get an $O(s^{-(2+\gamma_J+\epsilon)}) $ estimate for $D$ with $\epsilon>0$, we need $2-\alpha_N+2/p>2+\gamma_J$ and therefore choose
\begin{equation}\label{eqn:mu}
\mu>\frac{d}{4}+\frac{\max\{\alpha_N+\gamma_J,\tilde{\alpha}_{\tilde{N}}+\tilde{\gamma}_{\tilde{J}}\}}{2}\,, \quad p:= \frac{2}{\max\{\alpha_N+\gamma_J,\tilde{\alpha}_{\tilde{N}}+\tilde{\gamma}_{\tilde{J}}\}}
\end{equation} 
(taking into account the tilde version of the above estimate as well).
This yields
\[
\begin{aligned}
&\frac{\sum_{n=1}^\infty\langle f_{i}\varphi_n\rangle B_i\varphi_n}{
s^2+c^2\lambda_1+\sum_{j=1}^J d_j s^{\gamma_j+2}+\sum_{k=1}^{N} s^{\alpha_k} b_{k}(\lambda_1)^{\beta_{k}}}\\
&\qquad=
\frac{\sum_{n=1}^\infty\langle \tilde{f}_{i}\tilde{\varphi}_n\rangle \tilde{B}_i\tilde{\varphi}_n}{
s^2+\tilde{c}^2\tilde{\lambda}_1+\sum_{j=1}^{\tilde{J}} \tilde{d}_j s^{\tilde{\gamma}_j+2}+\sum_{k=1}^{\tilde{N}} s^{\tilde{\alpha}_k} \tilde{b}_{k}(\tilde{\lambda}_1)^{\tilde{\beta}_{k}}}
+O(s^{-(2+\max\{\gamma_J,\tilde{\gamma}_{\tilde{J}}\}+\epsilon)}), \quad s\in[1,\infty) 
\end{aligned}
\]
with $\epsilon = \mu-\frac{d}{4}-\frac{\max\{\alpha_N+\gamma_J,\tilde{\alpha}_{\tilde{N}}+\tilde{\gamma}_{\tilde{J}}\}}{2}>0$. 
Thus, our assumptions on $\mu$, $\nu$ are 
\begin{equation}\label{eqn:munu}
\mu>\frac{d}{4}+\frac{\max\{\alpha_N+\gamma_J,\tilde{\alpha}_{\tilde{N}}+\tilde{\gamma}_{\tilde{J}}\}}{2},
\quad
\sup_{n\in\mathbb{N}} \{\lambda_n^{-\nu}|B\varphi_n|\}, \ \sup_{n\in\mathbb{N}} \{\tilde{\lambda}_n^{-\nu}|B\tilde{\varphi}_n|\} <\infty.
\end{equation}

In case of rational powers 
\begin{equation}\label{eqn:rationalpowers}
\begin{aligned}
&\alpha_k=\frac{p_k}{q_k}, \ \tilde{\alpha}_k=\frac{\tilde{p}_k}{\tilde{q}_k}, \ \gamma_j=\frac{n_j}{m_j}, \  \tilde{\gamma}_j=\frac{\tilde{n}_j}{\tilde{m}_j}, \\
&\bar{q}=\mbox{lcm}(q_1,\ldots,q_{N},\tilde{q}_1,\ldots,\tilde{q}_{\tilde{N}},m_1,\ldots,m_{J},\tilde{m}_1,\ldots,\tilde{m}_{\tilde{J}}),
\end{aligned}
\end{equation} 
setting $z=s^{1/\bar{q}}$ we read \eqref{eqn:BaBb_singlemodes} as an equality between two rational functions of $z$, whose coefficients and powers therefore need to coincide (provided $\{s^{1/\bar{q}}\, : \, s\in\mathbb{C}_\theta\}\supseteq [\underline{z},\infty)$ for some $\underline{z}\geq0$).
In the general case of real powers one can use the induction proof from \cite[Theorem 1.1]{JinKian:2021} to obtain the same uniqueness.
This yields
\[
\begin{aligned}
&(I)\qquad&&N=\tilde{N},\qquad J=\tilde{J}\,,\\
&(II)&&c^2\lambda_1=\tilde{c}^2\tilde{\lambda}_1\,, \\
&(III)&&\alpha_k=\tilde{\alpha}_k\,,\ \gamma_j=\tilde{\gamma}_j\,, &&k=1,\ldots N\,, \ j=1,\ldots J\\
&(IV)&&b_k(\lambda_1)^{\beta_{k}}
=\tilde{b}_k(\tilde{\lambda}_1)^{\tilde{\beta}_{k}}\,, \ d_j=\tilde{d}_j\,, \quad 
&& \ k=1,\ldots N\,, \ j=1,\ldots J
\end{aligned}
\]
From this it becomes clear that we get uniqueness of 
\[
N, \ J, \ c, \ b_{k}, \ \alpha_k, \ d_j, \ \gamma_j,
\]
provided all $\beta_k=\tilde{\beta}_k$ are known and $\lambda_1=\tilde{\lambda}_1$ (but still possibly unknown) since we can then divide by $\lambda_1$ and $\lambda_1^{\beta_k}$ in (II) and (IV), respectively.
However, there seems to be no chance to simultaneously obtain $b_k$ and $\beta_k$ from (IV). Therefore, we will consider a setting with two different excitations in Section~\ref{sec:singlemode} for this purpose.

The same derivation can be made for the variable $c=c(x)$ model \eqref{eqn:general_c} in place of \eqref{eqn:general}, with 
$c$, $\mathcal{A}$, $L^2(\Omega)$, $\|\varphi\|_{L^2}=1$ replaced by 
$1$, $\mathcal{A}_c$, $L^2_{1/c^2}(\Omega)$, $\|\varphi\|_{L^2_{1/c^2}}=1$. 
From (II) with $c$, $\tilde{c}$ replaced by unity, we get $\lambda_1=\tilde{\lambda}_1$ so that we do not need to assume this. 

\begin{theorem}\label{thm:uniqueness_smoothdata}
Let $f,\tilde{f}\in \dot{H}^{2\mu+2\nu}(\Omega)$ for $\mu,\nu$ sufficiently large so that \eqref{eqn:munu} holds.
 
Then
\[
B u(t) = \tilde B \tilde u(t)\, \quad t\in(0,T)
\]
\begin{enumerate}
\item[(i)] 
for the solutions $u$, $\tilde{u}$ of \eqref{eqn:general}
with vanishing initial data and known equal $\beta_k=\tilde{\beta}_k$, possibly unknown but equal $\lambda_1=\tilde{\lambda}_1$, possibly unknown (rest of) $\mathcal{A}$, $\tilde{\mathcal{A}}$, and possibly unknown $f,\tilde{f}$
implies
\[
N=\tilde{N}, \ J=\tilde{J}, \ c=\tilde{c}, \ b_k=\tilde{b}_k, \ \alpha_k=\tilde{\alpha}_k, \ d_j=\tilde{d}_j, \ \gamma_j=\tilde{\gamma}_j,\quad k\in\{1,\ldots,N\}, j\in\{1,\ldots,J\}\,. 
\]
\item[(ii)]
for the solutions $u$, $\tilde{u}$ of \eqref{eqn:general_c}
with vanishing initial data and known equal $\beta_k=\tilde{\beta}_k$, and possibly unknown $\mathcal{A}_c$, $\tilde{\mathcal{A}}_{\tilde{c}}$
implies
\[
N=\tilde{N}, \ J=\tilde{J}, \ b_k=\tilde{b}_k, \ \alpha_k=\tilde{\alpha}_k, \ d_j=\tilde{d}_j, \ \gamma_j=\tilde{\gamma}_j,\quad k\in\{1,\ldots,N\}, j\in\{1,\ldots,J\}\,. 
\]
\end{enumerate}
\end{theorem}

\begin{remark}\label{rem:uniqueness_u0smooth}
This uniqueness approach also extends to different excitation combinations such as 
$u_0\not=0$, $u_j=0$, $j=1,2$, $f=0$; 
$u_1\not=0$, $u_j=0$, $j=0,2$, $f=0$; 
$u_2\not=0$, $u_j=0$, $j=0,1$, $f=0$; (the latter only in case $\gamma_J>0$) 

Consider for example the case $u_0\not=0$, $u_j=0$, $j=1,2$, $f=0$ that will be relevant for additionally recovering $u_0$, cf. Remark~\ref{rem:uniqueness_smoothdata_and_u1_u0} and which is the one that differs most from the setting above in view of the fact that the numerator of the resolvent solution depends on $s$ and $\lambda_n$.
The crucial estimate on the error that is made by replacing $\lambda_n$ by $\lambda_1$ then becomes, for $s\in\mathbb{R}$, $s\geq1$, 
\revision{and using the decomposition $\omega(s,\lambda)=s^2+\omega_d(s)+\omega_b(s,\lambda)$ with $\omega_d(s)=\sum_{j=1}^J d_j s^{\gamma_j+2}$, $\omega_b(s,\lambda)=\sum_{k=1}^N s^{\alpha_k} b_{k}\lambda^{\beta_{k}}$
\[
\begin{aligned}
D:=&\left|
\sum_{n=1}^\infty
\frac{\langle u_0,\varphi_n\rangle B\varphi_n(\omega(s,\lambda_1)-c^2\lambda_1)}{s\omega(s,\lambda_1)}-
\sum_{n=1}^\infty\frac{\langle u_0,\varphi_n\rangle B\varphi_n(\omega(s,\lambda_n)-c^2\lambda_n)}{s\omega(s,\lambda_n)}
\right|\\
&=\left|\sum_{n=1}^\infty
\frac{\langle u_0,\varphi_n\rangle B\varphi_n\,c^2\Bigl(\lambda_n(s^2+\omega_d(s)+\omega_b(s,\lambda_1))-\lambda_1(s^2+\omega_d(s)+\omega_b(s,\lambda_n))\Bigr)}{s\,\omega(s,\lambda_n)\, \omega(s,\lambda_1)}\right|\\
\\
&\leq\sum_{n=1}^\infty
\frac{|\langle u_0,\varphi_n\rangle B\varphi_n|\,c^2\Bigl((\lambda_n-\lambda_1)(s^2+\omega_d(s))+\lambda_n\omega_b(s,\lambda_1)-\lambda_1\omega_b(s,\lambda_n)\Bigr)}{s\,\omega(s,\lambda_n)\, \omega(s,\lambda_1)}\\
&\leq \sup_{n\in\mathbb{N}} \{\lambda_n^{-\nu}|B\varphi_n|\}\, 
\|u_0\|_{\dot{H}^{2\mu+2\nu}(\Omega)}
\left(\sum_{n=1}^\infty\left(
\frac{(s^2+\bar{d} s^{2+\gamma_J} +\bar{b}s^{\alpha_N})c^2\lambda_n}{
\lambda_n^\mu s^3(s^2+c^2\lambda_n)}\right)^2\right)^{1/2}
\end{aligned}
\]
for $\bar{d}=\sum_{j=1}^J d_j$, $\bar{b}=\sum_{k=1}^N b_{k}\max\{1,\lambda_1\}$,
where we have used $(\lambda_n-\lambda_1)(s^2+\omega_d(s))\geq0$ and $\lambda_n\omega_b(s,\lambda_1)-\lambda_1\omega_b(s,\lambda_n)
=\sum_{k=1}^N s^{\alpha_k} b_{k}\bigl(\lambda_n^{\beta_{k}}\lambda_1^{\beta_{k}}(\lambda_n^{1-\beta_{k}}-\lambda_1^{1-\beta_{k}})\bigr)\geq0$.
}
\\
To achieve an $O(s^{-(2+\max\{\gamma_J,\tilde{\gamma}_{\tilde{J}}\}+\epsilon)})$ estimate, instead of \eqref{eqn:munu} we thus assume
\begin{equation}\label{eqn:munu_u0}
\mu>\frac{d}{4}+1,\quad \max\{\gamma_J,\tilde{\gamma}_{\tilde{J}}\}<\frac12\, \quad
\quad
\sup_{n\in\mathbb{N}} \{\lambda_n^{-\nu}|B\varphi_n|\}, \ \sup_{n\in\mathbb{N}} \{\tilde{\lambda}_n^{-\nu}|B\tilde{\varphi}_n|\} <\infty.
\end{equation}
to recover Theorem~\ref{thm:uniqueness_smoothdata} with $f$, $\tilde{f}$ replaced by $u_0$, $\tilde{u}_0$.
\end{remark}

\begin{remark}
Note that in view of the fact that $\mathcal{A}$, $\mathcal{A}_c$ do not need to be known (up to the fact that a single eigenvalue is supposed to coincide), we are able to identify all relevant information on the damping model in an unknown medium.
\end{remark}


\subsection{Single mode excitation}\label{sec:singlemode}
Assume that we know at least some of the eigenfunctions and can use them as excitations $f_{i}=\varphi_{n_i}$, $\tilde{f}_{i}=\tilde{\varphi}_{\tilde{n}_i}$, $i=1,\ldots, I$, so that \eqref{eqn:BaBb} becomes
\begin{equation}\label{eqn:BaBb_singlemodes}
\begin{aligned}
&\frac{\langle f_{i},\varphi_{n_i}\rangle B_i\varphi_{n_i}}{
s^2+c^2\lambda_{n_i}+\sum_{j=1}^J d_j s^{\gamma_j+2}+\sum_{k=1}^N s^{\alpha_k} b_{k}(\lambda_{n_i})^{\beta_{k}}}\\
&\qquad=
\frac{\langle {\tilde f}_{i},{\tilde \varphi}_{\tilde{n}_i}\rangle {\tilde B}_i {\tilde \varphi}_{\tilde{n}_i}}{
s^2+{\tilde c}^2 {\tilde \lambda}_{\tilde{n}_i}+\sum_{j=1}^{\tilde{J}} \tilde{d}_j s^{\tilde{\gamma}_j+2}+\sum_{k=1}^{\tilde N} s^{{\tilde \alpha}_k} {\tilde b}_{k}({\tilde \lambda}_{\tilde{n}_i})^{{\tilde \beta}_{k}}}, \quad i=1,\ldots I\,, \quad s\in\mathbb{C}_\theta\,. \end{aligned}
\end{equation}

We can use the rational function or induction argument as above to compare powers of $s$ and conclude the following:
\[
\begin{aligned}
&(I)\qquad&&N=\tilde{N},\qquad J=\tilde{J}\\
&(II)&&c^2\lambda_{n_i}=\tilde{c}^2\tilde{\lambda}_{\tilde{n}_i}\,, \quad&& i=1,\ldots,I\\
&(III)&&\alpha_k=\tilde{\alpha}_k\,,\ \gamma_j=\tilde{\gamma}_j\,, &&k=1,\ldots N\,, \ j=1,\ldots J\\
&(IV)&&b_k(\lambda_{n_i})^{\beta_{k}}
=\tilde{b}_k(\tilde{\lambda}_{\tilde{n}_i})^{\tilde{\beta}_{k}}\,, \ d_j=\tilde{d}_j\,, \quad 
&&i=1,\ldots,I\,, \ k=1,\ldots N\,, \ j=1,\ldots J
\end{aligned}
\]
To extract $b_k$ and $\beta_k$, which can be done separately for each $k$, (skipping the subscript $k$) we use $I=2$ excitations, assume that the corresponding eigenvalues of $\mathcal{A}$ and $\tilde{\mathcal{A}}$ are known and equal and define 
\[
F(b,\beta)=
(b(\lambda_{n_1})^\beta,b(\lambda_{n_2})^\beta)^T\,.
\]
One easily sees that the $2\times2$ matrix $F'(b,\beta)$ is regular for $\lambda_{n_1}\neq\lambda_{n_2}$ and $b\not=0$ and thus by the Inverse Function Theorem $F$ is injective. Thus from (I)-(IV) we obtain all the constants in \eqref{eqn:model}.

\begin{theorem}\label{thm:uniqueness_singlemode}
Let $f_1=\varphi_{n_1}$, $f_2=\varphi_{n_2}$, $\tilde{f}_1=\tilde{\varphi}_{\tilde{n}_1}$, $\tilde{f}_2=\tilde{\varphi}_{\tilde{n}_2}$ for some $n_1\not=n_2$, $\tilde{n}_1\not=\tilde{n}_2$ $\in\mathbb{N}$.
Then 
\[
B_i u_i(t) = \tilde B_i \tilde u_i(t), \quad t\in(0,T),\quad i=1,2
\]
for the solutions $u_i$, $\tilde{u}_i$ of \eqref{eqn:general}
with vanishing initial data, known and equal $\lambda_i=\tilde{\lambda}_i$, $i=1,2$ and possibly unknown (rest of) $\mathcal{A}$, $\tilde{\mathcal{A}}$
implies
\[
N=\tilde{N}, \ J=\tilde{J}, \ c=\tilde{c}, \ b_k=\tilde{b}_k, \ \alpha_k=\tilde{\alpha}_k, \ \beta_{k}=\tilde{\beta}_k, \ d_j=\tilde{d}_j, \ \gamma_j=\tilde{\gamma}_j,\quad k\in\{1,\ldots,N\}, j\in\{1,\ldots,J\}\,. 
\]
\end{theorem}

\begin{remark}
Scaling with unity in the excitation has been chosen here just for simplicity of exposition. Clearly also multiples of eigenfunctions can be used.

Again, uniqueness can be shown analogously with eigenfunction excitation by initial data rather than by sources.
\end{remark}

We briefly consider the additional recovery of a spatially varying wave speed $c$ and for this purpose look at  the {\sc ch} model \eqref{eqn:CH_c} as an example.
With $\mathcal{A}_c=-c(x)^2\triangle$, (I)--(IV) (with $N=1$, $c=1$, $\beta=1$, $b_{k}=b$) yields
\[
\alpha=\tilde{\alpha}\,, \quad 
\lambda_{n_i}=\tilde{\lambda}_{\tilde{n}_i}\,, \quad 
b\,\lambda_{n_i}=\tilde{b}\,\tilde{\lambda}_{\tilde{n}_i}\,, \quad i=1,\ldots,I\,.
\]
Using inverse Sturm-Liouville theory, one can recover the spatially varying coefficient $c(x)$ in one space dimension from knowledge of all eigenvalues (actually just two sets of eigenvalues corresponding to different boundary conditions),
see, \cite{RundellSacks:1992b,CCPR:1997}.
However, in order to do so via this single mode exciation approach, we would need infinitely many measurements
$\{n_i\, : i=1,\ldots I\}=\mathbb{N}$.

The goal of obtaining knowledge on the eigenvalues can be much better achieved by going back to \eqref{eqn:BaBb} and considering equality of poles (and residues), which according to 
\cite[Section 4]{KaltenbacherRundell:2021b}
%Section 11.1 in fde-lect 
gives us $\lambda_n$ and $b\lambda_n^\beta$ (note that we have set $c=1$ since $c(x)$ is contained in $\mathcal{A}_c$) from a single excitation $u_1$ with nonvanishing Fourier coefficients. We will now consider this approach for the more general setting \eqref{eqn:general_2ndorder} in Section~\ref{sec:poles}.

\subsection{Poles and residues}\label{sec:poles}
In this section we focus on the second order in time case
\eqref{eqn:general_2ndorder} and achieve separation of summands in the
eigenfunction expansion by using poles (and residues) of $\hat{h}(s)$.
For this purpose we have to prove that each summand $\lambda_n$ gives rise to at
least one pole (in case of {\sc ch} we know there are two complex conjugates) and
that these poles differ for different $\lambda_n$.
This is also known for {\sc ch} and {\sc fz}, see 
\cite[Section 4]{KaltenbacherRundell:2021b}.
%Section 11.1 in fde-lect.
Indeed the results from 
\cite[Lemma 4.1, Remark 4.1]{KaltenbacherRundell:2021b}
%Lemma 11.4 in fde-lect 
on {\sc ch} carry over to the model \eqref{eqn:general_2ndorder} in the setting
\begin{equation}\label{eqn:samebeta}
\beta_1=\cdots=\beta_k=\beta, \quad b_1,\cdots,b_k\geq0 
%\mbox{ and }b_k>0\mbox{ for at least one} k
\end{equation}
that is, \eqref{eqn:general_2ndorder} with a single $\beta$ and dissipative behaviour (as is natural to demand from a damping model). 
Existence of at least one pole for each $\lambda_n$ can even be shown for the most
general model \eqref{eqn:mostgeneral_2ndorder}. 
\begin{lemma}\label{lem:what}
For each $\lambda_n$ there exists at least one root of 
$\omega^{\text{mg}}_n(s)=s^2+c^2\lambda_n+\sum_{k=1}^N s^{\alpha_k}\sum_{\ell=1}^{M_k} c_{k\ell}\lambda_n^{\beta_{k\ell}}$, that is, a pole of the Laplace transformed relaxation solution $\hat{w}^{\text{mg}}_{fn}$, $\hat{w}^{\text{mg}}_{0n}$, or $\hat{w}^{\text{mg}}_{1n}$.
Moreover, the poles are single.
\\
In case of \eqref{eqn:samebeta}, the poles of $\hat{w}^{\text{g}}_{fn}$, $\hat{w}^{\text{g}}_{0n}$, $\hat{w}^{\text{g}}_{1n}$, which are the roots of $\omega^{\text{g}}_n(s)=s^2+c^2\lambda_n+\sum_{k=1}^N b_k s^{\alpha_k} \lambda_n^\beta$, lie in the left half plane and differ for different $\lambda_n$.  
\end{lemma}
\begin{proof}
Let $f_n(z) = z^2 + c^2\lambda_n$, $g_n(z) = \sum_{k=1}^N z^{\alpha_k}\sum_{\ell=1}^{M_k} c_{k\ell}\lambda_n^{\beta_{k\ell}}$.
Let $C_R$ be the circle radius $R$, centre at the origin.
Then, due to $\alpha_k\leq1$, for a sufficiently large $R>c^2\lambda_n$, the estimate $|g_n(z)| < |f_n(z)|$ holds on $C_R$ and so Rouch\'{e}'s theorem shows that
$f_n(z)$ and $\omega^{\text{mg}}_n(z)=(f_n+g_n)(z)$ have the same number of roots, counted with multiplicity, within $C_R$.
For $f_n$ these are only at $z=\pm i\sqrt{\lambda_n}c$
and so $\omega^{\text{mg}}_n$ has 
precisely one single root in the third and in the fourth quadrant, respectively.
\\
The fact that the poles of $\hat{w}^{\text{g}}_{fn}$, $\hat{w}^{\text{g}}_{0n}$, $\hat{w}^{\text{g}}_{1n}$ lie in the left half plane in case of \eqref{eqn:samebeta} follows by an energy argument similar to the one in the proof of Lemma 11.3 fde-lect.
Suppose now that $\omega^{\text{g}}$ has a root at $p=r e^{i\theta}$,
where $\pi/2\leq\theta<\pi$, for both $\lambda_n$ and $\lambda_m$. Then for $p=r e^{i\theta}$, subtracting the equations $\omega^{\text{g}}_n(p)=0$ and $\omega^{\text{g}}_m(p)=0$ we obtain
\begin{equation}\label{eqn:lambdan-lambdam}
\frac{c^2(\lambda_n - \lambda_m)}{\lambda_n^\beta-\lambda_m^\beta} = - \sum_{k=1}^N b_k p^{\alpha_k} = \frac{p^2+c^2\lambda_n}{\lambda_n^\beta}
\end{equation}
thus
$$
\frac{p^2}{c^2}
=\frac{\lambda_n^\beta(\lambda_n - \lambda_m)}{\lambda_n^\beta-\lambda_m^\beta}-\lambda_n
=\frac{\lambda_n^\beta\lambda_m^\beta(\lambda_n^{1-\beta}-\lambda_m^{1-\beta})}{\lambda_n^\beta-\lambda_m^\beta}
$$
Now if $\lambda_n \not=\lambda_m$ then the right hand side is positive and real
and so $2\theta =\pi$.
This means that $\sum_{k=1}^N b_k s^{\alpha_k}= \sum_{k=1}^N b_k r^{\alpha_k\pi/2}$ (where all $b_k$ have the same sign) has nonvanishing imaginary part, a contradiction to the fact that the left hand side of \eqref{eqn:lambdan-lambdam} is real.
\end{proof}
Since we will separate the summands by taking residues on both sides of the identity \eqref{eqn:BaBb}, the following identity is also crucial.
\begin{lemma}\label{lem:residues}
For each $\lambda_n$, $n\in\mathbb{N}$, the residues of $\hat{w}^{\text{g}}_{fn}$, $\hat{w}^{\text{g}}_{0n}$, $\hat{w}^{\text{g}}_{1n}$ do not vanish.
\end{lemma}
\begin{proof}
By l'Hospital's rule and due to the fact that the poles $p_n$ are single, we get for $\omega(s)=\omega^{\text{mg}}_{fn}(s)=\omega^{\text{mg}}_{1n}(s)$
\[
\begin{aligned}
\mbox{Res}(\hat{w}^{\text{mg}}_n;p_n)&=\lim_{s\to p_n}\frac{(s-p_n)}{\omega(s)}
=\lim_{s\to p_n}\frac{1}{\omega'(s)}\not=0\,.
\end{aligned}
\]
Moreover, with $\omega(s)=\omega^{\text{mg}}_{0n}(s)$, $d(s)=\frac{\omega(s)-c\lambda^2}{s}$,
\[
\begin{aligned}
\mbox{Res}(\hat{w}^{\text{mg}}_n;p_n)&=\lim_{s\to p_n}\frac{(s-p_n)d(s)}{\omega(s)}
=\lim_{s\to p_n}\frac{d(s)+(s-p_n)d'(s)}{\omega'(s)}=\frac{d(p_n)}{\omega'(p_n)}\\
&=\frac{\omega(p_n)-c\lambda_n^2}{p_n\omega'(p_n)}=-\frac{c\lambda_n^2}{p_n\omega'(p_n)}\not=0\,.
\end{aligned}
\]
\end{proof}

As a consequence of Lemma~\ref{lem:what}, \ref{lem:residues}, by taking residues at the poles $p_n$ of the Laplace transformed time trace data $\hat{h}(s)$, 
%in case of \eqref{eqn:samebeta} 
we obtain from \eqref{eqn:BaBb} 
%\begin{equation}\label{eqn:BaBb_residues}
%\begin{aligned}
%&\frac{\langle u_{1,i},\varphi_{n_i}\rangle B_i\varphi_{n_i}}{
%2p_n+\sum_{k=1}^N \alpha_k p_n^{\alpha_k-1}\sum_{\ell=1}^{M_k} c_{k\ell}(\lambda_{n_i})^{\beta_{k\ell}}}\\
%&\qquad=
%\frac{\langle {\tilde u}_{1,i},{\tilde \varphi}_{\tilde{n}_i}\rangle {\tilde B}_i {\tilde \varphi}_{\tilde{n}_i}}{
%2p_n+\sum_{k=1}^{\tilde N} {\tilde \alpha}_k p_n^{{\tilde \alpha}_k-1}\sum_{\ell=1}^{{\tilde M}_k} {\tilde c}_{k\ell}({\tilde \lambda}_{\tilde{n}_i})^{{\tilde \beta}_{k\ell}}}, \quad i=1,\ldots I\,, \quad n\in\mathbb{N}
%\end{aligned}
%\end{equation}
\begin{equation}\label{eqn:BaBb_residues}
\begin{aligned}
\mbox{Res}(\hat{h};p_n)=&\frac{1}{
2p_n+\sum_{k=1}^N \alpha_k b_k p_n^{\alpha_k-1}\, (\lambda_{n})^{\beta_k}}
\sum_{k\in K^{\lambda_n}}\langle f,\varphi_{n,k}\rangle B\varphi_{n,k}
\\&\qquad
=\frac{1}{
2p_n+\sum_{k=1}^{\tilde N} {\tilde \alpha}_k \tilde{b}_k p_n^{{\tilde \alpha}_k-1}({\tilde \lambda}_{n})^{\tilde{\beta}_k}}
\sum_{k\in K^{\tilde{\lambda}_n}} \langle {\tilde f},{\tilde \varphi}_{n,k}\rangle {\tilde B} {\tilde \varphi}_{n,k}
, \\ 
&i=1,\ldots I\,, \quad n\in\mathbb{N}\,. 
\end{aligned}
\end{equation}
\revision{where the cardinality of the index set $K^{\lambda_n}$ is equal to the multiplicity of the eigenvalue $\lambda_n$.}

In order to achieve nonzero numerators in \eqref{eqn:BaBb}, \eqref{eqn:BaBb_singlemodes} and \eqref{eqn:BaBb_residues}, we assume
\begin{equation}\label{eqn:Bnonzero}
B\varphi_{n}\not=0, \quad \tilde{B}\tilde{\varphi}_{n} \not=0\quad\mbox{ for all }n\in\mathbb{N},
\end{equation}
and
\begin{equation}\label{eqn:fmodesnonzero}
\langle f,\varphi_n\rangle\not=0, \quad \langle \tilde{f},\tilde{\varphi}_n\rangle\not=0\quad \mbox{ for all }n\in\mathbb{N},
\end{equation} 
%we can e.g. use $f=\delta$. 
\revision{see remark~\ref{rem:nonzerocoeff} below.}
\Margin{Ref 2 (iii)}

Assume that, 
\begin{equation}\label{eqn:fBequal}
\sum_{k\in K^{\lambda_n}}\langle f,\varphi_{n,k}\rangle B\varphi_{n,k}
=\sum_{k\in K^{\tilde{\lambda}_n}} \langle {\tilde f},{\tilde \varphi}_{n,k}\rangle {\tilde B} {\tilde \varphi}_{n,k}\not=0\,, \quad 
\mbox{ for all }n\in\mathbb{N}.
\end{equation}
(e.g. by assuming $f=\tilde{f}$, $B=\tilde{B}$, $\mathcal{A}=\tilde{\mathcal{A}}$). 
Then from the reciprocals of \eqref{eqn:BaBb_residues} we get, after division by 
$\sum_{k\in K^{\lambda_n}}\langle f,\varphi_{n,k}\rangle B\varphi_{n,k}$ and subtracting $2p_n$ on both sides
\begin{equation}\label{eqn:BaBb_residues_recip}
\sum_{k=1}^N \alpha_k b_k p_n^{\alpha_k-1}\, (\lambda_{n})^{\beta_k}
=\sum_{k=1}^{\tilde N} {\tilde \alpha}_k \tilde{b}_k p_n^{{\tilde \alpha}_k-1}(\tilde{\lambda_{n}})^{\tilde{\beta}_k}.
\end{equation}
In case $\beta=\beta_k=\tilde{\beta}_k=\tilde{\beta}$, $\lambda_n=\tilde{\lambda}_n$, this can be divided by $(\lambda_{n})^{\beta}$ and simplifies to 
\[
\sum_{k=1}^N \alpha_k b_k p_n^{\alpha_k-1}
=\sum_{k=1}^{\tilde N} {\tilde \alpha}_k \tilde{b}_k p_n^{{\tilde \alpha}_k-1}.
\]
that is, in case of rational powers \eqref{eqn:rationalpowers}
an equality between two polynomials of $z=s^{1/\bar{q}}$ at infinitely many different (according to the second part of Lemma~\ref{lem:what}) points $p_n^{1/\bar{q}}$. 
%\footnote{In case of general real powers we can proceed inductively, using asymptotics and $|p_n|\to\infty$(?)Assume we can show (by Rouche's Theorem?) that $p_n\sim\lambda_n^\gamma$ for some $\gamma>0$. Then the limiting argument could also work for $\beta_k\not=1$}
This yields
\[
N=\tilde{N}\,,\quad \alpha_k=\tilde{\alpha}_k\,, \quad b_k=\tilde{b}_k\,,\quad  k=1,\ldots N.
\]
Thus we recover the results of Theorem~\ref{thm:uniqueness_smoothdata} under essentially more restrictive assumptions 
-- however we can drop the smoothness assumption on the excitation $f$, so that $f=\delta$ is admissible.
Thus the use of poles and residues has no clear advantage over the smooth excitation approach as long as we only aim at recovering the damping term constants. However, it allows to gain additional information on spatially vaying coefficients $c=c(x)$ or $f=f(x)$ or $u_0=u_0(x)$.

\subsubsection{Additional recovery of $c(x)$}

First of all, just using knowledge of the pole locations of the data,
Lemma~\ref{lem:what} allows to obtain the following corollary of
Theorem~\ref{thm:uniqueness_smoothdata} (ii) on uniqueness of the numbers $N$, $J$,
coefficients $b_k$, orders $\alpha_k$, as well as the function $c(x)$ in one space
dimension. Note that the assumption $f=\delta$ is not compatible with the
smoothness assumptions from Theorem \ref{thm:uniqueness_smoothdata}, thus we cannot
apply the strategy from \cite{KaltenbacherRundell:2021b} of obtaining, besides the
eigenvalues, also the values $\varphi(x_0)$ and therewith the norming constants of
the eigenfunctions.
We thus apply a different result from Sturm-Liouville theory,
according to which a pair of eigenfunction sequences corresponding to two different
boundary impedances  suffices to recover the potential $q$ of the differential
operator $-\triangle + q$, see \cite{RundellSacks:1992a,CCPR:1997}.
%Theorem 9.14 in fde-lect
Using the transformation of variables 
\[
\xi(x)=\int_0^xc(s)^{-1}\, ds, \quad \eta(\xi)=c^{-1/2}(x) y(x), \quad q(\xi)= \frac14(c'(x)^2-2c(x)c''(x)),
\]
so that
\[
-c^2(x)y_{xx}(x)=\lambda y(x), \ x\in (0,L) \ \Leftrightarrow \
-\eta_{\xi\xi}(\xi)+q(\xi)\eta(\xi)=\lambda\eta(\xi), \ \xi\in (0,\int_0^Lc(s)^{-1}\, ds) 
\]
this transfers to the recovery of the sound speed $c(x)$.

Since these results hold in one space dimension only, in this subsection we consider 
\begin{equation}\label{eqn:general_c_oned}
\begin{aligned}
&u_{tt}-c(x)\triangle u
%+\sum_{j=1}^J d_j\partial_t^{2+\gamma_j}u+
\sum_{k=1}^N b_k \partial_t^{\alpha_k} (-c(x)\triangle u)^{\beta_k}u=\sigma(t)f\quad && x\in (0,L), \ t\in(0,T)\\
&u'(0,t)-H_0u(0,t)=0\,,\quad u'(L,t)+H_Lu(L,t)=0\quad && t\in(0,T)\\
&u(x,0)=u_0(x)\,, \quad u_t(x,0)=u_1(x)\, \quad u_{tt}(x,0)=u_2(x)\quad &&x\in(0,L)
\end{aligned}
\end{equation}
for some $L>0$ and some nonnegative impedance constants $H_0$, $H_L$.
For two different right hand boundary impedance values $H_{L,i}$, $i=1,2$ we denote by $u^i$ the solution of \eqref{eqn:general_c_oned} with $H_L=H_{L,i}$ and by 
$(\varphi_{n,i}$, $\lambda_{n,i})_{n\in\mathbb{N}}$ the eigenpairs of $-c(x)\triangle$ with boundary conditions $\varphi_{n,i}'(0)-H_0\varphi_{n,i}(0)=0$, $\varphi_{n,i}'(L)+H_{L,i}\varphi_{n,i}(L)=0$, 
and weigthed $L^2$ normalization $\int_0^L \frac{1}{c(x)^2}\varphi_{n,i}(x)^2\, dx = 1$, $i=1,2$.

For constructing sufficiently smooth exciations with nonvanishing coefficients with respect to the eigenfunction basis, 
we point to Remark~\ref{rem:nonzerocoeff} below.
%Lemmas~\ref{lem:nonzerocoeff}, \ref{lem:nonzerocoeff_simple} in the appendix.

Moroever, we assume that the observations satisfy 
\begin{equation}\label{eqn:Binonzero}
B_i\varphi_{n,i}\not=0, \quad \tilde{B}_i\tilde{\varphi}_{n,i} \not=0\mbox{ for all }n\in\mathbb{N}, \quad i=1,2, 
\end{equation}
\revision{which is, e.g., the case with the choice   
$B_i v:= v(x_0)$, $x_0\in \{0,L\}$,
%assuming $H_{x_0,i}>0$
see remark~\ref{rem:nonzerocoeff} below.}
\Margin{Ref 2 (iii)}

Analogous notation is used for \eqref{eqn:general_c_oned} with $c$, 
%$d_j$, 
$b_k$, $\alpha_k$, $\beta_k$, $f$, $H_0$, $H_L$ replaced by their corresponding tilde counterparts.

From Theorem~\ref{thm:uniqueness_smoothdata} and uniquenss of the eigenvalues from the poles we obtain the following result on combined identification of the coefficients in the damping model and the space-dependent sound speed.
\begin{corollary}\label{cor:uniqueness_smoothdata_and_c}
Let $f,\tilde{f}\in \dot{H}^{2\mu+2\nu}(0,L)$ for $\mu,\nu$ satisfying \eqref{eqn:munu} and let
\eqref{eqn:fmodesnonzero}, \eqref{eqn:Binonzero} hold.

Then
\[
B_i u_i(t) = \tilde B_i \tilde u_i(t)\, \quad t\in(0,T), \quad i=1,2
\]
for the solutions $u_i$, $\tilde{u}_i$ of \eqref{eqn:general_c_oned}
with vanishing initial data, $H_L=H_{L,i}$, $\tilde{H}_L=\tilde{H}_{L,i}$, known equal $\beta_k=\tilde{\beta}_k$, and unknown $c$, $\tilde{c}$ $\in C^2(0,L)$,
implies
\[
\begin{aligned}
&N=\tilde{N}, \ 
%J=\tilde{J}, \ 
b_k=\tilde{b}_k, \ \alpha_k=\tilde{\alpha}_k, \ 
%d_j=\tilde{d}_j, \ \gamma_j=\tilde{\gamma}_j,
\quad k\in\{1,\ldots,N\}, 
%j\in\{1,\ldots,J\},
\\ 
&c(x)=\tilde{c}(x), \quad x\in(0,L)\,. 
\end{aligned}
\]
\end{corollary}
%In order to employ Lemma~\ref{lem:nonzerocoeff}, we would need  $\mu+\nu<\frac14$.
The same result can be obtained without smoothness assumption on $f$ 
%beyond $f\in L^2(\Omega)$, 
provided \eqref{eqn:fBequal}, and $\beta=\beta_k=\tilde{\beta}_k=\tilde{\beta}$ by a poles/residual argument, cf. \eqref{eqn:BaBb_residues_recip}.

\revision{
\begin{remark}\label{rem:nonzerocoeff}
Some comments on the verification of conditions \eqref{eqn:fmodesnonzero}, \eqref{eqn:Binonzero}, \eqref{eqn:fBequal} are in order, in particular in view of the fact that $c(x)$, $\tilde{c}(x)$ and therefore also the eigenfunctions are unknown.
\\
First of all, note that choosing $f,\tilde{f}=\delta_{x_0}$ formally satisfies this condition provided we can make sure that none of the eigenfunctions vanish at the point $x_0$. This is, e.g., the case if $x_0\in\{0,L\}$ is a boundary point for then in case $x_0=L$ the impedance conditions 
%with $H_{L,i}>0$ 
would imply 
$\varphi_n'(0)-H_0\varphi_n(0)=0$, $\varphi_n'(L)=0$, $\varphi_n(L)=0$, which by uniqueness of solutions to the second order ODE $-c(x)\varphi_n''(x)-\lambda_n\varphi_n(x)=0$ with these boundary conditions would imply $\varphi_n\equiv0$, a contradiction to $\varphi_n$ being an eigenfunction.
Likewise for $x_0=0$.
\\
Note that setting $B_i v:= v(x_0)$, $x_0\in \{0,L\}$, the same argument yields \eqref{eqn:Binonzero} as well as \eqref{eqn:fBequal}.
\\
However, clearly $f=\delta_{x_0}$ does not satisfy the regularity requirements from the first part of the corollary, thus we also indicate a possibility of constructing a smooth $f$ that satisfies \eqref{eqn:fmodesnonzero}.
For any $s>0$, $\epsilon>0$, the function 
$f(x)=c(x)^2\sum_{n=1}^\infty \lambda_n^{-s} n^{-(\frac12+\epsilon)} \varphi_n(x)$ lies in $\dot{H}^{s}$ and satisfies 
$\langle f,\varphi_k\rangle_{L^2_{1/c^2}}  \geq \lambda_n^{-s} n^{-(\frac12+\epsilon)} $ for all $k\in\mathbb{N}$.
Unfortunately this is only a result on existence of $f$, since the reconstruction requires knowledge of $c$.
\end{remark}
}
\Margin{Ref 2 (iii)}

\subsubsection{Additional recovery of $f(x)$ (or $u_0(x)$ or $u_1(x)$)}

In order to reconstruct the possibly unknown excitation $f$ (analogously for $u_1$, $u_0$, the latter relevant in PAT) we use measurements not only at finitely many points but on a surface $\Sigma$  
and make the linear independence assumption
\begin{equation}\label{eqn:ass_inj_Sigma_rem}
\left(\sum_{k\in K^{\lambda_n}} m_k \varphi_k(x) = 0 \ \mbox{ for all }x\in\Sigma\right)
\ \Longrightarrow \ \left(m_k=0 \mbox{ for all }k\in K^{\lambda_n}\right)
\quad n\in\mathbb{N}\,.
\end{equation}
Moreover, for the purpose of recovering $f$ (or $u_1$, or $u_0$) we assume $c(x)$ to be known and therefore $\varphi_{n}=\tilde{\varphi}_{n}$, $\lambda_{n}=\tilde{\lambda}_{n}$.
Thus in place of \eqref{eqn:BaBb_residues} we have
\begin{equation}\label{eqn:BaBb_residues_Sigma}
\begin{aligned}
&\frac{1}{
2p_n+\sum_{k=1}^N \alpha_k b_k p_n^{\alpha_k-1}\, (\lambda_{n})^\beta}
\sum_{k\in K^{\lambda_n}}\langle f,\varphi_{n,k}\rangle \varphi_{n,k}(x)
\\&\qquad
=\frac{1}{
2p_n+\sum_{k=1}^{\tilde N} {\tilde \alpha}_k \tilde{b}_k p_n^{{\tilde \alpha}_k-1}(\lambda_{n})^{\tilde{\beta}}}
\sum_{k\in K^{\tilde{\lambda}_n}}\langle \tilde{f},\varphi_{n,k}\rangle \varphi_{n,k}(x), 
\quad x\in\Sigma\,, \quad n\in\mathbb{N}\,. 
\end{aligned}
\end{equation}

Again, we can obtain a combined identification result from Theorem~\ref{thm:uniqueness_smoothdata}, uniquenss of the eigenvalues from the poles, and \eqref{eqn:BaBb_residues_Sigma}.

\begin{corollary}\label{cor:uniqueness_smoothdata_and_f}
Let $f,\tilde{f}\in \dot{H}^{2\mu+2\nu}(\Omega)$ for $\mu,\nu$ saftisfying \eqref{eqn:munu} and let  
\eqref{eqn:BaBb_residues_Sigma} hold.

Then
\[
u(x,t) = \tilde{u}(x,t)\, \quad x\in\Sigma, \ t\in(0,T), 
\]
for the solutions $u$, $\tilde{u}$ of \eqref{eqn:general_c}
with vanishing initial data known equal $\beta_k=\tilde{\beta}_k$, and unknown $f$, $\tilde{f}$ $\in L^2(\Omega)$,
implies
\[
\begin{aligned}
&N=\tilde{N}, \ 
%J=\tilde{J}, \ 
b_k=\tilde{b}_k, \ \alpha_k=\tilde{\alpha}_k, \ 
%d_j=\tilde{d}_j, \ \gamma_j=\tilde{\gamma}_j,
\quad k\in\{1,\ldots,N\}, 
%j\in\{1,\ldots,J\},
\\ 
&f(x)=\tilde{f}(x), \quad x\in\Omega\,. 
\end{aligned}
\]
\end{corollary}

\begin{remark}
The same result can be obtained without smoothness assumption on $f$ 
%beyond $f\in L^2(\Omega)$, 
provided \eqref{eqn:fBequal}, and $\beta=\beta_k=\tilde{\beta}_k=\tilde{\beta}$ by a poles/residual argument, cf. \eqref{eqn:BaBb_residues_recip}.
\end{remark}

\begin{remark}\label{rem:uniqueness_smoothdata_and_u1_u0}
The result carries over to the setting $f=0$, $u_0=0$, where $u_1=u_1(x)\not=0$ is supposed to be recovered almost without changes.

In the setting $f=0$, $u_1=0$, of recovering $u_0=u_0(x)\not=0$ (as relevant in PAT), we can rely on Remark~\ref{rem:uniqueness_u0smooth} to conclude that Corollay~\ref{cor:uniqueness_smoothdata_and_f} remains valid with $f$, $\tilde{f}$, \eqref{eqn:munu} replaced by $u_0$, $\tilde{u}_0$, \eqref{eqn:munu_u0}.
\end{remark}

