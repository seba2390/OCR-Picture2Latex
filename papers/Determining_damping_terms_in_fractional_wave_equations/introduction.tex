\section{Introduction}

Wave phenomena like seismic waves or ultrasound propagation are known to exhibit power law frequency-dependent damping,
which can be modelled by time fractional derivative terms in the corresponding {\sc pde}s.
As a matter of fact, the adequate representation of attenuation with power law 
frequency dependence was perhaps the first application of fractional time 
derivative concepts and among the driving forces for their development, 
see Caputo \cite{Caputo:1967} and Caputo and Mainardi~\cite{CaputoMainardi:1971a}.
 
Fractionally damped wave equations arise by combining classical balance laws
with fractional constitutive relations.
More precisely, consider the equation of motion (resulting from a balance of forces) and the representation of the mechanical strain as the symmetric gradient of the mechanical displacement 
\begin{equation}\label{eqn:motion-strain}
\rho\vec{u}_{tt}=\mbox{div}\sigma+\vec{f}\,, \quad \epsilon=\frac12(\nabla\vec{u}+(\nabla\vec{u})^T)
\end{equation}
where $\rho$ is the mass density, $\vec{u}(x,t)$ the vector of displacements and $\sigma(x,t)$ the stress tensor. The crucial constitutive equation is now the material law that relates stress $\sigma$ and strain $\epsilon$.
Some often encountered instances are 
the fractional Newton (also called Scott-Blair model), Voigt, Maxwell, and Zener models
\begin{equation*}
\sigma = b_1 \partial_t^{\alpha} \epsilon\,, \quad
\sigma = b_0 \epsilon + b_1 \partial_t^{\alpha}\epsilon\,, \quad 
\sigma + d_1\partial_t^{\gamma}\sigma = b_0 \epsilon\,, \quad
\sigma + d_1\partial_t^{\gamma}\sigma = b_0 \epsilon + b_1 \partial_t^{\alpha}\epsilon\,.
\end{equation*}
For a survey on fractionally damped acoustic wave equations we refer to \cite{CaiChenFangHolm_survey2018}.
Driven by applications that are not covered by these models, a general model class has been defined by 
\begin{equation}\label{eqn:sumfrac}
\sum_{j=0}^J d_j \partial_t^{\gamma_j} \sigma = \sum_{k=0}^K b_k \partial_t^{\alpha_k} \epsilon
\end{equation}
with the normalisation $d_0=1$, where $0\leq\gamma_0<\gamma_1\cdots\leq \gamma_J$ and $0\leq\alpha_0<\alpha_1\cdots\leq \alpha_K$
cf., e.g., \cite[section 3.1.1]{Atanackovic_etal:2014}, \cite{bagley1986fractional,schmidt2002finite}.
Combination of \eqref{eqn:sumfrac} with \eqref{eqn:motion-strain} leads to a general wave equation, which  
for simplicity of exposition in the scalar case of one-dimensional mechanics or in acoustics reads as 
\[
\sum_{j=0}^J d_j \partial_t^{2+\gamma_j} u - \sum_{k=0}^K b_k \partial_t^{\alpha_k} \triangle u = r\,.
\]
In this more complicated situation, direct reconstruction of the damping terms from experimental data is not possible any more. It is the aim of this paper to investigate reconstruction of the orders $\alpha_k$, $\gamma_j$ and the coefficients $b_k$, $d_j$ from time trace measurements of $u$, which plays the role of a displacement in mechanics or of a pressure in acoustics.
In the latter application case, on which we mainly focus here, certain imaging modalities additionally require  reconstruction of an unknown sound speed $c(x)$ in $b_0=c^2$ (ultrasound tomography), and/or an initial condition $u_0(x)$ (equivalently, of the spatially varying part $f$ of the source $r(x,t)=\sigma(t)f(x)$; photoacoustic tomography).
Besides investigating these tasks, inspired by recent work by Jin and Kian \cite{JinKian:2021} in the context of diffusion models, we will also study the question of identifiability of the damping terms in an unknown medium, that is, e.g., with unknown $c(x)$, without aiming at its reconstruction. 
While to the best of our knowledge, this is the first work on recovering
multiple fractional damping in wave type equations, much more literature is
available for anomalous diffusion {\sc pde}s.
Besides the reference \cite{JinKian:2021},
we also refer to, e.g.,
\cite{HatanoNakagawaWangYamamoto:2013,LiYamamoto:2015,LiZhangJiaYamamoto:2013,Yamamoto:2021,SunLiZhang:2021}.

We will present results on uniqueness and on reconstruction algorithms in
each of three paradigms depending on the availability of measured data.
These are:  a full time trace $u(x_0,t)$, $t >0$ for some fixed point $x_0$;
extreme large time measurements; and extremely small time measurements.
The main tool used in the latter two cases is a Tauberian theorem that
relates the asymptotic behaviour of the solution in time to the structure
of the representation of the solution in the Laplace transformed formulation
that contains specific algebraic information on the fractional operator.


The remainder of this paper is organised as follows.
In subsections~\ref{sec:models}, \ref{sec:inverse} we introduce the models under consideration and state the inverse problem.
Section~\ref{sec:forward} is devoted to the forward problem: We provide an analysis of the general midterm fractional wave equation and describe the numerical forward solver used in our reconstructions.
In section~\ref{sec:uniqueness} we prove uniqueness results based on different paradigms: Using the Weyl estimate on decay of eigenvalues together with smoothness of the excitation, applying single mode excitations and exploiting information on poles and residues in the Laplace domain.
Section~\ref{sec:recon_meth} develops reconstruction methods in the three scenarios indicated above: full time (that is, Laplace domain), large time and small time observations. 
Correspondingly, in section~\ref{sec:reconstructions} we provide numerical reconstructions along with a discussion of the approaches.

\input models
\input inverse
%\input preliminaries
