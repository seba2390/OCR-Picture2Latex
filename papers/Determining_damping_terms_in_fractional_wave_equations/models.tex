\subsection{Models} \label{sec:models}
We consider the second order in time damped wave equation
\begin{equation}\label{eqn:general_2ndorder}
%\begin{aligned}&
u_{tt}+c^2\mathcal{A}u+\sum_{k=1}^N b_k \partial_t^{\alpha_k} \mathcal{A}^{\beta_k}u=r
%\sigma(t)f\quad &&t\in(0,T)\\
%&u(x,0)=u_0(x)\,, \quad u_t(x,0)=u_1(x)\quad &&x\in\Omega
%\end{aligned}
\end{equation}
where $\mathcal{A}=-\mathbb{L}$ on $\Omega\subseteq\mathbb{R}^d$, equipped with homogeneous Dirichlet/ Neumann/ impedance boundary conditions, for an elliptic differential operator $\mathbb{L}$ and 
%$u_1,f\in L^2(\Omega)$, $u_0\in \mathcal{D}(\mathcal{A}^{1/2})$, 
$\alpha_k\in(0,1]$, $\beta_{k}\in(\frac12,1]$.

Throughout this paper, we denote by $\partial_t^\alpha$ the (partial) Caputo-Djrbashian fractional time derivative of order $\alpha\in(n-1,n)$ with $n\in\mathbb{N}$ by
\[
\partial_t^\alpha u  = \frac{1}{\Gamma(n-\alpha)}\int_{0}^{t}\frac{\partial^n_t u(s)}{(t-s)^{\alpha+1-n}}  \,ds\;,
\]
where $\partial^n_t$ denotes the $n$-th integer order partial time derivative and for
$\gamma\in(0,1)$, and $I_t^\gamma$ is the Abel integral operator defined by 
$
I_t^\gamma[v](t) = \frac{1}{\Gamma(\gamma)}\int_{0}^{t}\frac{v(s)}{(t-s)^{1-\gamma}}ds\,.
$
For details on fractional differentiation and subdiffusion equations,
we refer to, e.g.,
\cite{Dzjbashian:1966,Djrbashian:1993,
MainardiGorenflo:2000,SakamotoYamamoto:2011a,SamkoKilbasMarichev:1993}.
See also the tutorial paper on inverse problems for anomalous diffusion processes
\cite{JinRundell:2015}.

We assume that the differentiation orders with respect to time are distinct, that is
\begin{equation}\label{eqn:alphas_distinct}
0<\alpha_1<\alpha_2<\cdots<\alpha_N\leq1,\quad
0<\gamma_1<\gamma_2<\cdots<\gamma_J\leq1
\end{equation}
a property that is crucial for distinguishing the different asymptotic terms in the solution $u$ and its Laplace transform.

More generally than \eqref{eqn:general_2ndorder},
in most of this paper we also take higher than second order time derivatives into account
\begin{equation}\label{eqn:general}
%\begin{aligned}&
u_{tt}+c^2\mathcal{A}u+\sum_{j=1}^J d_j\partial_t^{2+\gamma_j}u+\sum_{k=1}^N b_k \partial_t^{\alpha_k} \mathcal{A}^{\beta_k}u=r
%\sigma(t)f\quad &&t\in(0,T)\\
%&u(x,0)=u_0(x)\,, \quad u_t(x,0)=u_1(x)\,,\quad u_{tt}(x,0)=u_2(x)\quad &&x\in\Omega
%\end{aligned}
\end{equation}
%which includes the fractional Zener model.

Note that the usual models from \cite{KaltenbacherRundell:2021b} are contained in \eqref{eqn:general}, which is actually an extension of the general model from, e.g., the book \cite{Atanackovic_etal:2014}
%(see also (2.50) in fde-lect) 
in the sense that it also allows for fractional powers of the operator $\mathcal{A}$.

Typically, in acoustics we just have $\mathbb{L}=\triangle$, and then the operator $\mathcal{A}$ is known. To take into account a (possibly unknown) spatially varying speed of sound $c(x)$, one can instead consider 
\begin{equation}\label{eqn:general_c}
%\begin{aligned}&
u_{tt}+\mathcal{A}_c u+\sum_{j=1}^J d_j\partial_t^{2+\gamma_j}u+\sum_{k=1}^N b_k \partial_t^{\alpha_k} \mathcal{A}_c^{\beta_k}u=r
%\sigma(t)f\quad &&t\in(0,T)\\
%&u(x,0)=u_0(x)\,, \quad u_t(x,0)=u_1(x)\, \quad u_{tt}(x,0)=u_2(x)\quad &&x\in\Omega
%\end{aligned}
\end{equation}
with $\mathcal{A}_c=-c(x)^2\triangle$; in order to get a selfadjoint operator $\mathcal{A}_c$ we then use the weighted $L^2$ inner product with weight function $\frac{1}{c^2}$.

\medskip

Note that in order to assume \eqref{eqn:alphas_distinct} without loss of generality, so allowing for all possible combinations of $\alpha_k$, $\beta_k$, we would have to consider
\begin{equation}\label{eqn:mostgeneral_2ndorder}
%\begin{aligned}&
u_{tt}+c^2\mathcal{A}u+\sum_{k=1}^N\partial_t^{\alpha_k}\sum_{\ell=1}^{M_k} c_{k\ell}\mathcal{A}^{\beta_{k\ell}} u=r
%\sigma(t)f\quad &&t\in(0,T)\\
%&u(x,0)=u_0(x)\,, \quad u_t(x,0)=u_1(x)\quad &&x\in\Omega\,,
%\end{aligned}
\end{equation}
or, including higher than second time derivatives such as in \eqref{eqn:general}
\begin{equation}\label{eqn:mostgeneral}
%\begin{aligned}&
u_{tt}+c^2\mathcal{A}u+\sum_{j=1}^J d_j\partial_t^{2+\gamma_j}u+\sum_{k=1}^N\partial_t^{\alpha_k}\sum_{\ell=1}^{M_k} c_{k\ell}\mathcal{A}^{\beta_{k\ell}}u=r.
%\sigma(t)f\quad &&t\in(0,T)\\
%&u(x,0)=u_0(x)\,, \quad u_t(x,0)=u_1(x)\,,\quad u_{tt}(x,0)=u_2(x)\quad &&x\in\Omega
%\end{aligned}
\end{equation}
%whose version with spatially varying sound speed is
%\begin{equation}\label{eqn:mostgeneral_2ndorder_c}
%\begin{aligned}
%&u_{tt}+\mathcal{A}_c u+\sum_{k=1}^N\partial_t^{\alpha_k}\sum_{\ell=1}^{M_k} c_{k\ell}\mathcal{A}_c^{\beta_{k\ell}}u =\sigma(t)f\quad &&t\in(0,T)\\
%&u(x,0)=u_0(x)\,, \quad u_t(x,0)=u_1(x)\quad &&x\in\Omega
%\end{aligned}
%\end{equation}
However, \eqref{eqn:mostgeneral_2ndorder}, \eqref{eqn:mostgeneral} might be
viewed as over-parameterised models and they actually lead to difficulties
in proving uniqueness of solutions.

%At the most general end of the spectrum, we also mention the case of space dependent  fractional orders $\alpha_k=\alpha_k(x)$, which we will not touch here, though.

\medskip

Important special cases of \eqref{eqn:general_c} are the Caputo-Wismer-Kelvin-Chen-Holm %{\sc ch} 
\begin{equation}\label{eqn:CH_c}
%\begin{aligned}&
u_{tt}+\mathcal{A}_c u+ b \partial_t^{\alpha} \mathcal{A}_c^\beta u=r
%\sigma(t)f\quad &&t\in(0,T)\\
%&u(x,0)=u_0(x)\,, \quad u_t(x,0)=u_1(x)\quad &&x\in\Omega
%\end{aligned}
\end{equation}
and the fractional Zener {\sc fz} model
\begin{equation}\label{eqn:FZ_c}
%\begin{aligned}&
u_{tt}+\mathcal{A}_c u+ d \partial_t^{2+\gamma}  u + b \partial_t^{\alpha} \mathcal{A}_c u=r .
%\sigma(t)f\quad &&t\in(0,T)\\
%&u(x,0)=u_0(x)\,, \quad u_t(x,0)=u_1(x)\quad &&x\in\Omega
%\end{aligned}
\end{equation}

These {\sc pde} models will be considered on a time interval $t\in(0,T)$ and driven by initial conditions and/or a separable source term
\begin{equation}\label{eqn:init_source}
\begin{aligned}
&r(x,t)=\sigma(t)f(x)\,, \quad x\in\Omega, \quad t\in(0,T)\\
&u(x,0)=u_0(x)\,, \quad u_t(x,0)=u_1(x)\,, \quad (u_{tt}(x,0)=u_2(x)\mbox{ if }\gamma_J>0)\quad x\in\Omega. \end{aligned}
\end{equation}
Note that boundary conditions are already incorporated into the operator $\mathcal{A}$ or $\mathcal{A}_c$, respectively. 
