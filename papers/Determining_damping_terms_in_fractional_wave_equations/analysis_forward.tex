%\gamma->\theta
%\alpha->\gamma
%\beta-> \alpha
%b_m -> b_k
%a_n -> d_j
%\sum_{n=0}^N -> \sum_{j=0}^J
%sum_{m=0}^M -> \sum_{k=0}^K
%m_* -> k_*
%M -> K
%N -> J
%\underline{a} -> d_J
%\underline{b} -> b_K
%C_b -> remove terms
\def\calAtil{\tilde{\mathcal{A}}}

\subsection{Analysis of the forward problem}

In this section we prove well-posedness of the forward problem \eqref{eqn:general_c}.
Just as in the inverse problem which is the main topic of this paper,
we focus on the case of constant coefficients $b_k$, $d_j$, while the sound speed $c=c(x)$ contained in the operator $\calAtil$ may vary in space. We refer to Chapter 7 of the upcoming book \cite{BBB} for an analysis in case all coefficients are space dependent.

Our analysis will be based on energy estimates obtained by multiplying the {\sc pde} (or actually its Galerkin semidiscretization with respect to space) with appropriate multipliers. For this purpose, we will make use of certain mapping properties of fractional derivative operators that can be quantified via upper and lower bounds.

We will use the following results that hold for $\theta\in[0,1)$.
%\begin{itemize}
%	\item
%	From \cite[Lemma 1]{Alikhanov:11}: For any absolutely continuous function $w$,
%	\begin{equation}\label{eqn:Alikhanov_1}
%	\partial_t^{\theta}w(t) w(t)\geq \tfrac12(\partial_t^{\theta} w^2)(t).
%	\end{equation}
%\end{itemize}
\begin{itemize}	
	\item
	From \cite[Lemma 2.3]{Eggermont:1987}; see also \cite[Theorem 1]{VoegeliNedaiaslSauter:2016}: For any $w\in H^{-\theta/2}(0,t)$,
	\begin{equation}\label{eqn:coercivityI}
	\int_0^t \langle {_0I_t}^{\theta} w(s),  w(s) \rangle ds \geq \cos ( \pi\theta/2 ) \| w \|_{H^{-\theta/2}(0,t)}^2, 
	\end{equation}	 
	\item 
	From \cite[Theorems 2.1, 2.2, 2.4 and Proposition 2.1]{KubicaRyszewskaYamamoto:2020}:
	There exist constants $0<\underline{C}(\theta)<\overline{C}(\theta)$ such that for any $w\in H^\theta(0,t)$ (with $w(0)=0$ if $\theta\geq\frac12$), the equivalence estimates 
	\begin{equation}\label{eqn:equivalence_pos}
\underline{C}(\theta) \|w\|_{H^\theta(0,t)} \leq \|\partial_t^\theta w\|_{L^2(0,t)}\leq
\overline{C}(\theta) \|w\|_{H^\theta(0,t)}
	\end{equation}
hold.
Likewise, for any $w\in L^2(0,t)$ we have $\partial_t^\theta w\in H^{-\theta}(0,t)$ and 
	\begin{equation}\label{eqn:equivalence_neg}
\underline{C}(\theta) \|w\|_{L^2(0,T)} \leq \|\partial_t^\theta w\|_{H^{-\theta}(0,t)}\leq
\overline{C}(\theta) \|w\|_{L^2(0,t)}.
	\end{equation}
Moreover, (by a duality argument) there exists a constant $C(2\theta)>0$ such that for any $w\in L^2(0,t)$
	\begin{equation}\label{eqn:boundednessI}
	\int_0^t ({_0I_t}^{2\theta} w(s))  w(s) ds \leq C(2\theta)\| w \|_{H^{-\theta}(0,t)}^2, 
	\end{equation}	 
\end{itemize}
\rerevision{Note that the latter two facts \eqref{eqn:equivalence_neg}, \eqref{eqn:boundednessI} hold true for $\theta=1$ as well.}


We are now prepared to state and prove our main theorem on well-posedness of an initial value problem 
\begin{equation}\label{eqn:sumfrac_wave_acou_ibvp}
  \left\{\begin{aligned}
\sum_{j=0}^J d_j &\partial_t^{2+\gamma_j} u + \sum_{k=0}^K b_k \partial_t^{\alpha_k} 
\calAtil u = r\quad \mbox{ in }\Omega\times(0,T),\\
    &u(x,0)  = u_0(x)\,, \ u_t(x,0) = u_1(x) \quad x\in\Omega,\\
&(\mbox{if }\gamma_J>0:\ u_{tt}(x,0) = u_2(x)\quad x\in\Omega,)
  \end{aligned}\right.
\end{equation}
which clearly covers the models from Section~\ref{sec:models} in case of $\beta_1=\cdots=\beta_K$.
Here $\calAtil$ is a selfadjoint (with respect to some possibly weighted space $L^2_w(\Omega)$ with inner product $\langle\cdot,\cdot\rangle$) operator with
eigenvalues and eigenfunctions $(\lambda_i,\varphi_i)_{i\in\mathbb{N}}$,
and $\dot{H}^1(\Omega)=\{v\in L^2_w(\Omega)\, : \, \|\calAtil^{1/2}v\|_{L^2_w(\Omega)}<\infty\}$.
This includes the cases $\calAtil=-\triangle$, $\calAtil=-c(x)^2\triangle$, $\calAtil=-c(x)^2\nabla\cdot(\tfrac{1}{\rho(x)}\nabla)$ relevant for acoustics, where in the latter two cases we use the inner product $\langle v_1,v_2\rangle=\int_\Omega\frac{1}{c^2(x)}\, v_1(x)v_2(x)\ dx$ in which $\calAtil$ is selfadjoint.

The coefficients and orders will be assumed to satisfy 
%\revision{(more generally than \eqref{eqn:alphas_distinct}, since we do not assume boundedness from above)}
\begin{equation}\label{eqn:gammaalpha}
0\leq\gamma_0<\gamma_1<\cdots< \gamma_J
 \leq1
\,, \qquad 0\leq\alpha_0<\alpha_1<\cdots< \alpha_K
 \leq1
\,, 
\qquad \gamma_J\leq\alpha_K\,,
\end{equation}
\Margin{Ref 2 (ii)}
and 
\begin{equation}\label{eqn:dJbK}
d_j\geq0\,, \ j\in\{1,\ldots,J\}\,, \quad b_k\geq 0\,, \ k\in\{\rerevision{k_*},\ldots,K\}\,, \quad d_J>0 \,, \quad b_K>0 \,.
\end{equation} 
with $k_*$ to be specified below.
\rerevision{The condition $\gamma_J\leq\alpha_K$ has been shown to be necessary for thermodynamic consistency, see \cite[Lemma 3.1]{Atanackovic_etal:2014}.\\
Note that the proof of Theorem~\ref{thm:sumfrac} also goes through without the restrictions $\gamma_j \leq1$, $\alpha_k \leq1$, but orders larger than one would require introducing further initial conditions, which would make the exposition less transparent. Moreover, the focus in this paper is actually on wave type equations where this restriction of the orders is physically relevant.  
}

We first of all consider the following time integrated and therefore weaker form with $\tilde{r}(x,t)=\int_0^t r(x,s)\, ds$
\begin{equation}\label{eqn:sumfrac_wave_acou_ibvp_timeint}
  \left\{\begin{aligned}
&\sum_{j=0}^J d_j
\Bigl(\partial_t^{1+\gamma_j} u-\frac{t^{1-\gamma_j}}{\Gamma(2-\gamma_j)}u_2(x)\Bigr)\\
&+ \sum_{k=0}^K b_k \Bigl({_0I_t}^{1-\alpha_k} \calAtil u - \frac{t^{1-\alpha_k}}{\Gamma(2-\alpha_k)}\calAtil u_0\Bigr)
= \tilde{r} \quad \mbox{ in }\Omega\times(0,T),\\
    &\hspace*{1cm}u(x,0)  = u_0(x)\,, \ u_t(x,0) = u_1(x) \quad x\in\Omega
  \end{aligned}\right.
\end{equation}

The results are obtained by testing the {\sc pde} in %\eqref{eqn:sumfrac_wave_acou_ibvp_timeint}
\eqref{eqn:sumfrac_wave_acou_ibvp} 
with 
\rerevision{
$\partial_t^{\tau} u$ for some $\tau>0$
%$\in [1+\gamma_J,1+\alpha_K]$ 
that needs to be well chosen in order to exploit the estimates \eqref{eqn:coercivityI}-\eqref{eqn:boundednessI}.
Generalizing the common choice $\tau=1$ in case of the classical second order wave equation $\gamma_0=\cdots=\gamma_J=\alpha_0=\cdots=\alpha_K=0$ and in view of the stability condition $\gamma_J\leq\alpha_K$ we choose $\tau\in [1+\gamma_J,1+\alpha_K]$, actually, 
$\tau=1+\alpha_{k_*}$ with $k_*$ according to 
\begin{equation}\label{eqn:mast}
k_*\in\{1,\ldots,K\}\mbox{ such that }\alpha_{k_*-1}<\gamma_J\leq\alpha_{k_*},
\end{equation}
if $\alpha_0<\gamma_J$ and $k_*=0$ otherwise.
More precisely, we will assume that 
\begin{equation}\label{eqn:i-or-ii}
%\begin{aligned}
(i) \   \alpha_0<\gamma_J, \ k_*\mbox{ as in \eqref{eqn:mast} }, 
\ \mbox{ or } \
(ii)\  
\alpha_0\geq\gamma_J, \ k_*:=0
%\end{aligned}
\end{equation}
holds.
%
Additionally, we assume
\begin{equation}\label{alphakstar}
\alpha_K-1\leq\alpha_{k_*}\leq 1+\min\{\gamma_0,\alpha_0\}
\ \mbox{ and either } \ \gamma_J<\alpha_K \mbox{ or }\ \gamma_J=\alpha_K=\alpha_0.
\end{equation}
The upper and lower bounds on $\alpha_{k_*}$ in \eqref{alphakstar} are needed to guarantee $\theta\in[0,1)$ in several instances of $\theta$ used in \eqref{eqn:coercivityI}-\eqref{eqn:boundednessI} in the proof below.
\\
Note that we are considering the case $\gamma_J=\alpha_K$ only in the specific setting of a single $\calAtil$ term.
Indeed, when admitting several $\calAtil$ terms, well-posedness depends on specific coefficient constellations. As an example of this, let us mention the fractional Moore-Gibson-Thompson equations arising in acoustics
\begin{equation}\label{fMGT}
u_{tt}+\mathfrak{T}\partial_t^{2+\alpha} u-\mathfrak{c}^2\Delta u-\mathfrak{b}\partial_t^{\alpha} \Delta u = f
\end{equation}
with positive parameters $\mathfrak{T},\mathfrak{b},\mathfrak{c},\alpha$.
It is well-posed only if $\mathfrak{b}\geq\mathfrak{c}^2\mathfrak{T}$, according to \cite[Proposition 7.1]{fracJMGT}.
% see also \cite[Remark 7.1]{fracJMGT}.
As the analysis of this example also shows, even lower order time derivatives of $\calAtil u$ (such as the $-\mathfrak{c}^2\Delta u$ term in \eqref{fMGT}) may prohibit global in time boundedness of solutions. 
It will turn out in the proof that we therefore have to impose a bound on the magnitude of the lower order term coefficients in terms of the highest order one
\begin{equation}\label{smallnessbk}
\begin{aligned}
&\sum_{k=0}^{k_*-1} C(1+\alpha_{k_*}-\alpha_k)\overline{C}((1+\alpha_{k_*}-\alpha_k)/2)\, |b_k|\, \|_0I_t^{(\alpha_K-\alpha_k)/2}\|_{L^2(0,T)\to L^2(0,T)}\\
&\leq
\underline{C}((1+\alpha_{k_*}-\alpha_K)/2)\, b_K\, \cos ( \pi(1+\alpha_{k_*}-\alpha_K)/2 ).
\end{aligned}
\end{equation}
that due to $T$ dependence of $\|_0I_t^{(\alpha_K-\alpha_k)/2}\|_{L^2(0,T)\to L^2(0,T)}$ clearly restricts the maximal time $T$ unless all $b_k$ with index lower than $k_*$ vanish, cf. Corollary~\ref{cor:global} below.
}

\rerevision{By applying the test functions as mentioned above}, we will derive estimates for the energy functionals (which are not supposed to represent physical energies but are just used for mathematical purposes)
\begin{equation}\label{eqn:Ea}
\begin{aligned}
\mathcal{E}_\gamma[u](t)=&
\sum_{j=0}^{J-1} \cos ( \pi(1+\gamma_j-\alpha_{k_*})/2 ) \| \sqrt{d_j}\partial_t^{(3+\gamma_j+\alpha_{k_*})/2} u \|_{L^2(0,t;L^2_w(\Omega))}^2 + \underline{\mathcal{E}}_\gamma[u](t)
\end{aligned}
\end{equation}
where 
\begin{equation}\label{eqn:ulEa}
\underline{\mathcal{E}}_\gamma[u](t)=
\begin{cases}\tilde{d} \|\partial_t^{(3+\gamma_J+\alpha_{k_*})/2} u \|_{L^2(0,t;L^2_w(\Omega))}^2
&\mbox{ if }\gamma_J<\alpha_{k_*}\\
\frac{d_J}{4}\|\partial_t^{1+\gamma_J} u \|_{L^\infty(0,t;L^2_w(\Omega))}^2 &\mbox{ if }\gamma_J=\alpha_{k_*}
\end{cases}
\end{equation}
where 
$\tilde{d}=d_J\,\cos ( \pi(1+\gamma_J-\alpha_{k_*})/2 )  \underline{C}((1+\gamma_J-\alpha_{k_*})/2)^2$ with $\underline{C}$ as in \eqref{eqn:equivalence_neg}, 
\rerevision{$\theta=(1+\gamma_J-\alpha_{k_*})/2\in[0,1)$}
and 
\begin{equation}\label{eqn:Eb}
\begin{aligned}
\mathcal{E}_\alpha[u](t)=&
\frac14\|\sqrt{b_{k_*}}\partial_t^{\alpha_{k_*}}\calAtil^{1/2}u\|_{L^\infty(0,t);L^2_w(\Omega)}^2\\
&+\sum_{k=k_*+1}^K
\cos ( \pi(1+\alpha_{k_*}-\alpha_k)/2 ) \| \sqrt{b_k}\partial_t^{(1+\alpha_{k_*}+\alpha_k)/2} \calAtil^{1/2}u \|_{L^2(0,t;L^2_w(\Omega))}^2
\end{aligned}
\end{equation}
where the sum in \eqref{eqn:Eb} is void in case $k_*=K$.

The solution space induced by these energies is $U_\gamma\cap U_\alpha$ where
\begin{equation}\label{eqn:defU}
\begin{aligned}
U_\gamma&=\begin{cases}
H^{(3+\gamma_J+\alpha_{k_*})/2}(0,T;L^2_w(\Omega))&\mbox{ if }\gamma_J<\alpha_{k_*}\\
\{v\in L^2(0,T;L^2_w(\Omega))\, : \, \partial^{1+\gamma_J} v\in L^\infty(0,T;L^2_w(\Omega))&\mbox{ if }\gamma_J=\alpha_{k_*}
\end{cases}
\\
U_\alpha&=\begin{cases}
H^{(1+\alpha_K+\alpha_{k_*})/2}(0,T;\dot{H}(\Omega))&\mbox{ if }K>k_*\\
\{v\in \rerevision{H^{(1-\epsilon)/2+\alpha_K}(0,T;\dot{H}(\Omega))}\, : \, \partial^{\alpha_K} v\in L^\infty(0,T;\dot{H}(\Omega))&\mbox{ if }K=k_*
\end{cases}
\end{aligned}
\end{equation}
\rerevision{for any $\epsilon\in(0,\alpha_K-\max\{\alpha_{K-1},\gamma_J\})$.}

\begin{theorem}\label{thm:sumfrac}
Assume that the coefficients $\gamma_j$, $\alpha_k$, $d_j$, $b_k$ satisfy \eqref{eqn:gammaalpha}, \eqref{eqn:dJbK}, \eqref{eqn:i-or-ii}, \eqref{alphakstar}, 
\rerevision{and that $T$ and/or the coefficients $b_k$ for $k\leq k_*$ are small enough so that \eqref{smallnessbk} holds.}


Then for any $u_0,u_1\in \dot{H}^1(\Omega)$, ($u_2\in L^2_w(\Omega)$ if $\gamma_J>0$), $r\in H^{\alpha_{k_*}-\gamma_J}(0,T;L^2_w(\Omega))$, the time integrated initial boundary value problem
\eqref{eqn:sumfrac_wave_acou_ibvp_timeint} (to be understood in a weak $\dot{H}^{-1}(\Omega)$ sense with respect to space and an $L^2(0,T)$ sense with respect to time) has a unique solution $u\in U_\gamma\cap U_\alpha$ cf. \eqref{eqn:defU}.
This solution satisfies the energy estimate
\begin{equation}\label{eqn:enest_sumfrac}
\begin{aligned}
\mathcal{E}_\gamma[u](t)+\mathcal{E}_\alpha[u](t)
\leq 
C(t) \bigl(\|u_0\|_{\dot{H}^1(\Omega)}^2 + C_0+\|r\|_{H^{\alpha_{k_*}-\gamma_J}(0,t,L^2_w(\Omega))}^2\bigr)\,, \quad t\in(0,T)
\end{aligned}
\end{equation}
for some constant $C(t)$ depending on time, \rerevision{which is bounded as $t\to0$ but} in general grows exponentially as
$t\to\infty$, $\mathcal{E}_\gamma$, $\mathcal{E}_\alpha$ defined as in \eqref{eqn:Ea}, \eqref{eqn:Eb}, and 
\begin{equation}\label{eqn:C0}
C_0 = \begin{cases} 
\|u_1\|_{\dot{H}^1(\Omega)}^2 + \|u_2\|_{L^2_w(\Omega)}^2 &\mbox{ if }\alpha_{k_*}>0\\ 
\|u_1\|_{L^2_w(\Omega)}^2 &\mbox{ if }\alpha_{k_*}=0.
\end{cases} 
\end{equation}
\end{theorem}
%Note that condition \eqref{eqn:gap} is void if $\gamma_J\leq\alpha_k$ holds for all $k\in \{0,\ldots,K\}$.

\begin{proof}
We exclude the case $\alpha_K=0$ since this via \eqref{eqn:gammaalpha} implies that also $\gamma_J=0$ and we would therefore deal with the conventional second order wave equation whose analysis can be found in textbooks, cf. e.g., \cite{Evans:2010}. Thus for the remainder of this proof we assume $\alpha_K>0$ to hold.

\noindent
{\bf Step 1.} Galerkin discretisation.\\
In order to prove existence and uniqueness of solutions to \eqref{eqn:sumfrac_wave_acou_ibvp},
we apply the usual Faedo-Galerkin approach of discretisation in space with eigenfunctions $\varphi_i$ corresponding to eigenvalues $\lambda_i$ of $\calAtil$, 
$u(x,t)\approx u^L(x,t)=\sum_{i=1}^L u^L_i(t)\varphi_i(x)$ and testing with $\varphi_j$, that is,
\begin{equation}\label{eqn:Galerkin}
\begin{aligned}
&\begin{cases}
\langle \sum_{j=0}^J d_j \partial_t^{2+\gamma_j} u^L(t) + \sum_{k=0}^K b_k \partial_t^{\alpha_k} 
\calAtil u^L(t) - r(t),v\rangle_{L^2_w(\Omega)} = 0 \quad t\in(0,T)\\
\langle u^L(0)-u_0,v\rangle = \langle u^L_t(0)-u_1,v\rangle = 0\,, \quad
(\mbox{if }\gamma_J>0:\ \langle u_{tt} - u_2,v\rangle = 0)
\end{cases}\\
&\mbox{ for all }v\in\mbox{span}(\varphi_1,\ldots, \varphi_L)\,.
\end{aligned}
\end{equation}
\begin{comment}
Due to the fact that the eigenfunctions $\varphi_i$ form an orthonormal basis of $L^2_w(\Omega)$, we can write the second line of \eqref{eqn:Galerkin} as $u^L(0)=u_0^L$, $u^L_t(0)=u_1^L$, ($u^L_{tt}(0)=u_2^L$,) with $u_k^L(x)= \sum_{i=1}^L\langle u_k,\varphi_i\rangle \varphi_i(x)$.
The projected {\sc pde} initial value problem \eqref{eqn:Galerkin} leads to the {\sc ode} system 
\begin{equation}\label{eqn:Galerkin_ode}
\sum_{j=0}^J M_{d,j}^L \partial_t^{2+\gamma_j} \underline{u}^L(t) 
+ \sum_{k=0}^K K_{b,k}^L \partial_t^{\alpha_k} \underline{u}^L(t)
= \underline{r}^L(t) \,, \quad t\in(0,T)
\end{equation}
with initial conditions
\begin{equation}\label{eqn:Galerkin_init}
\underline{u}^L(0)=\underline{u}_0^L\,, \quad
\underline{u}_t^L(0)=\underline{u}_1^L\,, \quad
(\mbox{if }\gamma_J>0:\ \underline{u}_{tt}^L(0)=\underline{u}_2^L)\,.
\end{equation}
Here the matrices and vectors are defined by 
\begin{equation*}
\begin{aligned}
&\underline{u}^L (t)= (u^L_i(t))_{i=1,\ldots L}\,, \quad 
\underline{u}^L_k  = (\langle u_k,\varphi_i\rangle)_{i=1,\ldots L}\,, \ k\in\{0,1,2\}\,,\\
&\underline{r}^L (t) = (\langle r(t),\varphi_i\rangle)_{i=1,\ldots L}\,, \quad
\Lambda^L = \mbox{diag}(\lambda_1,\ldots,\lambda_L)\,,\\
&M_{d,j}^L=(\langle d_j\varphi_i,\varphi_j\rangle)_{i,j=1,\ldots L}\,, \quad 
M_{b,k}^L=(\langle b_k\varphi_i,\varphi_j\rangle)_{i,j=1,\ldots L}\,, \\ 
&K_{b,k}^L%=(\langle b_k \calAtil\varphi_i,\varphi_j\rangle)_{i,j=1,\ldots L}
=(\lambda_i\langle b_k \varphi_i,\varphi_j\rangle)_{i,j=1,\ldots L}
=\Lambda^L M_{b,k}^L(t)\,.
\end{aligned}
\end{equation*}
Note that due to \eqref{eqn:dJbK} and since we assume the eigenfunctions to be normalised in $L^2_w(\Omega)$, the matrix $M_{d,J}^L$ is positive definite with eigenvalues bounded away from zero by $d_J$.
\end{comment}
To prove existence of a solution $\underline{u}^L\in H^{2+\gamma_J}(0,T;\mathbb{R}^L)$
to \eqref{eqn:Galerkin}, 
%\eqref{eqn:Galerkin_ode}, \eqref{eqn:Galerkin_init}, 
we rewrite it as a system of Volterra integral equations for 
\[
\underline{\xi}^L:= \partial_t^{2+\gamma_J}\underline{u}^L = 
\begin{cases}
{_0I_t}^{1-\gamma_J}\underline{u}_{ttt}^L &\mbox{ if }\gamma_J>0\\
\underline{u}_{tt}^L &\mbox{ if }\gamma_J=0\,.
\end{cases}
\]
\begin{comment}
For this purpose we use the identities %(cf. \eqref{eqn:RL-int-power})
\[
\begin{aligned}
\partial_t^{2+\gamma_j}\underline{u}^L
&={_0I_t}^{\gamma_J-\gamma_j}\underline{\xi}^L \mbox{ for } \gamma_j>0\,, \quad
\partial_t^{2+\gamma_0}\underline{u}^L={_0I_t}^{\gamma_J}\underline{\xi}^L+\underline{u}_2^L \mbox{ if } \gamma_0=0\,, \\
\underline{u}_t^L(t) 
&= \begin{cases}
{_0I_t}^2 \underline{u}_{ttt}^L(t) + t \underline{u}_2^L + \underline{u}_1^L &\mbox{ if }\gamma_J>0\\
{_0I_t}^1 \underline{u}_{tt}^L + \underline{u}_1^L &\mbox{ if }\gamma_J=0
\end{cases}
\\
\partial_t^{\alpha_k}\underline{u}^L(t)  
&={_0I_t}^{1-\alpha_k}\underline{u}_t^L(t) \\
&={_0I_t}^{2+\gamma_J-\alpha_k} \underline{\xi}^L (t) 
+ \frac{t^{2-\alpha_k}\underline{u}_2^L}{\Gamma(3-\alpha_k)}
+ \frac{t^{1-\alpha_k}\underline{u}_1^L}{\Gamma(2-\alpha_k)}\,,
\end{aligned}
\]
where the term containing $\underline{u}_2^L$ can be skipped in case $\gamma_J=0$.
This results in the following Volterra integral equation for $\xi$
\[
\begin{aligned}
\underline{\xi}^L(t) + (M_{a,J}^L)^{-1}\Bigl\{
\sum_{j=0}^J M_{a,n}^L {_0I_t}^{\gamma_J-\gamma_j} \underline{\xi}^L(t) 
+ \sum_{k=0}^K K_{b,k}^L {_0I_t}^{2+\gamma_J-\alpha_k} \underline{\xi}^L(t)\Bigr\}
= \underline{\tilde{r}}^L(t)\,,\\ 
\quad t\in(0,T),
\end{aligned}
\]
where $\underline{\tilde{r}}^L(t)=(M_{a,J}^L)^{-1}\left\{ \underline{r}^L(t)
- \sum_{k=0}^K K_{b,k}^L \left(\frac{t^{2-\alpha_k}\underline{u}_2^L}{\Gamma(3-\alpha_k)}
+ \frac{t^{1-\alpha_k}\underline{u}_1^L}{\Gamma(2-\alpha_k)}
\right)\right\}$,
and again the terms containing $\underline{u}_2^L$ can be skipped in case $\gamma_J=0$.
\end{comment}
Unique solvability of the resulting system in $L^2(0,T^L)$ follows from~\cite[Theorem 4.2, p. 241 in \S 9]{GLS90}. Then from	
\[ \left \{
	\begin{aligned}
	\partial_t^{\gamma_J}\underline{u}_{tt}^L =\underline{\xi} \in L^2(0,T^L), \quad
	\underline{u}^L_{tt}(0)=\underline{u}_{2}^L
	\end{aligned} \right.
	\]
in case $\gamma_J>0$ we have a unique $\underline{u}_{tt}^L \in H^{\gamma_J}(0,T^L)$ (cf.~\cite[\S 3.3]{KubicaRyszewskaYamamoto:2020}); the same trivially holds true in case $\gamma_J=0$. 

From the energy estimates below, this solution actually exists for all times $t\in[0,T)$, so the maximal time horizon of existence is actually $T^L=T$, independently of the discretisation level $L$.

\noindent
{\bf Step 2.} energy estimates.\\
With $k_*$ defined by \eqref{eqn:mast} in case (i) and set to $k_*=0$ in case (ii), we use $v=\partial_t^{1+\alpha_{k_*}}u^L(s)$ as a test function in \eqref{eqn:Galerkin} (with $t$ replaced by $s$) and integrate for $s$ from $0$ to $t$. 
The resulting terms can be estimated as follows.

Inequality \eqref{eqn:coercivityI} 
\rerevision{with $\theta=1+\gamma_j-\alpha_{k_*}\in[0,1)$}
yields
\[
\begin{aligned}
&\int_0^t\langle d_j \partial_t^{2+\gamma_j} u^L(s), \partial_t^{1+\alpha_{k_*}}u^L(s)\rangle\, ds
%\\&
= \int_0^t\langle d_j \partial_t^{2+\gamma_j} u^L(s), {_0I_t}^{1+\gamma_j-\alpha_{k_*}}\partial_t^{2+\gamma_j}u^L(s) \rangle\, ds\\
&\geq \cos ( \pi(1+\gamma_j-\alpha_{k_*})/2 ) \| \sqrt{d_j}\partial_t^{2+\gamma_j} u^L \|_{H^{-(1+\gamma_j-\alpha_{k_*})/2}(0,t;L^2_w(\Omega))}^2
%\\&
\quad\mbox{ for }\gamma_j<\alpha_{k_*}
\end{aligned} 
\]
and \rerevision{with $\theta=1+\alpha_{k_*}-\alpha_k\in[0,1)$}
\[
\begin{aligned}
&\int_0^t\langle b_k \partial_t^{\alpha_k} \calAtil u^L(s), \partial_t^{1+\alpha_{k_*}}u^L(s)\rangle\, ds
=\int_0^t\langle b_k \partial_t^{\alpha_k} \calAtil^{1/2}u^L(s), \partial_t^{1+\alpha_{k_*}}\calAtil^{1/2}u^L(s)\rangle\, ds\\
&= \int_0^t\langle b_k {_0I_t}^{1+\alpha_{k_*}-\alpha_k}\partial_t^{1+\alpha_{k_*}} \calAtil^{1/2}u^L(s), \partial_t^{1+\alpha_{k_*}}\calAtil^{1/2}u^L(s) \rangle\, ds\\
&\geq \cos ( \pi(1+\alpha_{k_*}-\alpha_k)/2 ) \| \sqrt{b_k}\partial_t^{1+\alpha_{k_*}} \calAtil^{1/2}u^L \|_{H^{-(1+\alpha_{k_*}-\alpha_k)/2}(0,t;L^2_w(\Omega))}^2
%\\&
\ \mbox{ for }k>{k_*}\,.
\end{aligned} 
\]
In particular for $k=K>{k_*}$ (which implies $1+\alpha_{k_*}-\alpha_K<1$; the case $k_*=K$ will be considered below), by \eqref{eqn:equivalence_neg} 
\rerevision{with $\theta=(1+\alpha_{k_*}-\alpha_K)/2\in[0,1)$}, noting that 
$\partial_t^{(1+\alpha_{k_*}+\alpha_K)/2} \calAtil^{1/2}u^L$ vanishes at $t=0$ and that $\partial_t^{(1+\alpha_{k_*}-\alpha_k)/2} \partial_t^{(1+\alpha_{k_*}+\alpha_K)/2}\calAtil^{1/2}u^L
=\partial_t^{1+\alpha_{k_*}} \calAtil^{1/2}u^L$,  
\begin{equation}\label{eqn:estbMupper}
\begin{aligned}
&\int_0^t\langle b_K \partial_t^{\alpha_K} \calAtil^{1/2}u^L(s), \partial_t^{1+\alpha_{k_*}}\calAtil^{1/2}u^L(s)\rangle\, ds\geq \rerevision{\underline{C}_{K,\alpha_{k_*}}} \| \partial_t^{(1+\alpha_{k_*}+\alpha_K)/2}\calAtil^{1/2}u^L \|_{L^2(0,t;L^2_w(\Omega))}^2
\end{aligned} 
\end{equation}
with 
\[
\rerevision{\underline{C}_{K,\alpha_{k_*}}}=\underline{C}((1+\alpha_{k_*}-\alpha_K)/2)\, b_K\, \cos ( \pi(1+\alpha_{k_*}-\alpha_K)/2 ). 
\] 

From 
\rerevision{the identity $\int_0^t\langle\partial_t v(s),v(s)\rangle ds =\frac12\|v(t)\|^2-\frac12\|v(0)\|^2$}
and Young's inequality in the case $\gamma_J=\alpha_{k_*}$ we get
\[
\begin{aligned}
&\int_0^t\langle d_J \partial_t^{2+\gamma_J} u^L(s), \partial_t^{1+\alpha_{k_*}}u^L(s)\rangle\, ds
%\\&
=\int_0^t\langle d_J \Bigl(\partial_t\partial_t^{1+\alpha_{k_*}} u^L(s)
-\frac{s^{-\alpha_{k_*}}}{\Gamma(1-\alpha_{k_*})} u_2^L\Bigr), \partial_t^{1+\alpha_{k_*}}u^L(s)\rangle\, ds\\
&\geq\frac12\|\sqrt{d_J}\partial_t^{1+\alpha_{k_*}} u^L(t)\|_{L^2_w(\Omega)}^2
-\frac{t^{2(1-\alpha_{k_*})}\,\|\sqrt{d_J}u_2^L\|_{L^2_w(\Omega)}^2}{\Gamma(2-\alpha_{k_*})^2}
%\\&\qquad
-\frac14 \|\sqrt{d_J}\partial_t^{1+\alpha_{k_*}}u^L\|_{L^\infty(0,t);L^2_w(\Omega)}^2\\
&\mbox{ if }\gamma_J=\alpha_{k_*}\in(0,1)
\end{aligned} 
\]
%(where the terms containing $u_2^L$ are skipped in case $\gamma_J=0$ or $\alpha_{k_*}=1$)
\[
\begin{aligned}
&\int_0^t\langle d_J \partial_t^{2+\gamma_J} u^L(s), \partial_t^{1+\alpha_{k_*}}u^L(s)\rangle\, ds
%\\&
=\frac12\|\sqrt{d_J}\partial_t^{1+\alpha_{k_*}} u^L)(t)\|_{L^2_w(\Omega)}^2
-\frac12\|\sqrt{d_J} u^L_{1+\alpha_{k_*}}\|_{L^2_w(\Omega)}^2\\
&\mbox{ if }\gamma_J=\alpha_{k_*}\in\{0,1\}
\end{aligned} 
\]
and 
\[
\begin{aligned}
&\int_0^t\langle b_{k_*} \partial_t^{\alpha_{k_*}} \calAtil^{1/2}u^L(s), \partial_t^{1+\alpha_{k_*}}\calAtil^{1/2}u^L(s)\rangle\, ds\\
&=\int_0^t\langle b_{k_*} \partial_t\partial_t^{\alpha_{k_*}} \calAtil^{1/2}u^L(s)
-\frac{s^{-\alpha_{k_*}}}{\Gamma(1-\alpha_{k_*})} \calAtil^{1/2}u_1^L, \partial_t^{\alpha_{k_*}}\calAtil^{1/2}u^L(s)\rangle\, ds\\
&\geq\frac12\|\sqrt{b_{k_*}}\partial_t^{\alpha_{k_*}} \calAtil^{1/2}u^L)(t)\|_{L^2_w(\Omega)}^2
-\frac{t^{2(1-\alpha_{k_*})}\,\|\sqrt{b_{k_*}}\calAtil^{1/2}u_1^L\|_{L^2_w(\Omega)}^2}{\Gamma(2-\alpha_{k_*})^2} 
\\&\qquad
-\frac14 \|\sqrt{b_{k_*}}\partial_t^{\alpha_{k_*}}\calAtil^{1/2}u^L\|_{L^\infty(0,t);L^2_w(\Omega)}^2
\ \mbox{ if }\alpha_{k_*}\in(0,1)
\,,
\end{aligned} 
\]
\[
\begin{aligned}
&\int_0^t\langle b_{k_*} \partial_t^{\alpha_{k_*}} \calAtil^{1/2}u^L(s), \partial_t^{1+\alpha_{k_*}}\calAtil^{1/2}u^L(s)\rangle\, ds\\
&=\frac12\|\sqrt{b_{k_*}}\partial_t^{\alpha_{k_*}} \calAtil^{1/2}u^L)(t)\|_{L^2_w(\Omega)}^2
-\frac12\|\sqrt{b_{k_*}} \calAtil^{1/2}u^L_{\alpha_{k_*}}\|_{L^2_w(\Omega)}^2
\ \mbox{ if }\alpha_{k_*}\in\{0,1\}
\,.
\end{aligned} 
\]

Hence in the case $k_*=K$, where due to our assumption that $\alpha_K>0$ at the beginning of the proof, $\partial_t^{\alpha_K} \calAtil^{1/2}u^L(0)=0$,
\[
\begin{aligned}
&\sup_{t'\in(0,t)}\int_0^{t'}\langle b_K \partial_t^{\alpha_K} \calAtil^{1/2}u^L(s), \partial_t^{1+\alpha_{k_*}}\calAtil^{1/2}u^L(s)\rangle\, ds\\
&\geq \frac{b_K}{4}\|\partial_t^{\alpha_K}\calAtil^{1/2}u^L\|_{L^\infty(0,t);L^2_w(\Omega)}^2
-\frac{t^{2(1-\alpha_{K})}\,\|\sqrt{b_{K}}\calAtil^{1/2}u_1^L\|_{L^2_w(\Omega)}^2}{\Gamma(2-\alpha_{K})^2}.
\end{aligned} 
\]

Since the difference $1+\alpha_{k_*}-\alpha_k$ is larger than one for $k< k_*$, 
we have to employ \eqref{eqn:boundednessI} in that case \rerevision{(for this purpose, note that still, according to our assumptions, $\theta=(1+\alpha_{k_*}-\alpha_k)/2\in[0,1]$)}.
From this and \eqref{eqn:equivalence_neg}, 
\rerevision{again with $\theta=(1+\alpha_{k_*}-\alpha_k)/2\in[0,1]$,}
we obtain
\begin{equation}\label{eqn:estbmlower}
\begin{aligned}
&\int_0^t\langle b_k \partial_t^{\alpha_k} u^L(s), \partial_t^{1+\alpha_{k_*}}u^L(s)\rangle\, ds\\
&= \int_0^t\langle b_k \, {_0I_t}^{1+\alpha_{k_*}-\alpha_k}\partial_t^{1+\alpha_{k_*}} \calAtil^{1/2}u^L(s), \partial_t^{1+\alpha_{k_*}}\calAtil^{1/2}u^L(s) \rangle\, ds\\
&\geq -C(1+\alpha_{k_*}-\alpha_k) \|\sqrt{b_k}\partial_t^{1+\alpha_{k_*}} \calAtil^{1/2}u^L\|_{H^{-(1+\alpha_{k_*}-\alpha_k)/2}(0,t;L^2_w(\Omega))}^2\\
&\geq -C(1+\alpha_{k_*}-\alpha_k)\overline{C}((1+\alpha_{k_*}-\alpha_k)/2)\|\sqrt{b_k}\partial_t^{(1+\alpha_{k_*}+\alpha_k)/2}\calAtil^{1/2}u^L\|_{L^2(0,t;L^2_w(\Omega))}^2\\
&\mbox{ for }k<{k_*}\,.
\end{aligned} 
\end{equation}

\rerevision{
Note that in case $k_*<K$, the potentially negative contributions arising from the $k<k_*$ terms \eqref{eqn:estbmlower} are dominated by means of the leading energy term from \eqref{eqn:Eb}, that is, $\| \partial_t^{(1+\alpha_{k_*}+\alpha_K)/2}\calAtil^{1/2}u^L \|_{L^2(0,t;L^2_w(\Omega))}^2$ from \eqref{eqn:estbMupper}, due to the fact that $(1+\alpha_{k_*}+\alpha_k)/2< (1+\alpha_{k_*}+\alpha_K)/2$ for $k<{k_*}$.
}
%[To dominate the potentially negative contributions arising from the $k<k_*$ terms \eqref{eqn:estbmlower} by means of the leading energy term from \eqref{eqn:Eb}, we need to impose the condition $1+\alpha_{K-1}\leq\alpha_K$ in case $k_*=K$, cf. \eqref{eqn:i-or-ii}.]

\rerevision{To achieve this also in case $k_*=K$ with $\alpha_{K-1}<\gamma_J<\alpha_K$, we additionally test with $v=\partial_t^{1+\alpha_K-\epsilon}u^L(s)$ in \eqref{eqn:Galerkin} where $0<\epsilon<\alpha_K-\max\{\alpha_{K-1},\gamma_J\}$.
Repeating the above estimates with $\alpha_{k_*}$ replaced by $\alpha_K-\epsilon$ one sees that 
the leading energy term then is $\| \partial_t^{\alpha_K+(1-\epsilon)/2}\calAtil^{1/2}u^L \|_{L^2(0,t;L^2_w(\Omega))}^2$ from \eqref{eqn:estbMupper}. This still dominates the potentially negative contributions arising from the $k<K$ terms according to \eqref{eqn:estbmlower} (with $\alpha_K-\epsilon$ in place of $\alpha_{k_*}$), since $(1+\alpha_K-\epsilon+\alpha_k)/2< \alpha_K+(1-\epsilon)/2$ for $k<K$.\\
Note that in the case $\gamma_J=\alpha_K=\alpha_0$, no potentially negative terms \eqref{eqn:estbmlower} arise.}


Finally, the right hand side term can be estimated by means of \eqref{eqn:equivalence_neg} 
\rerevision{with $\theta=\alpha_{k_*}-\gamma_J\in[0,1)$}
and Young's inequality  
\[
\begin{aligned}
&\int_0^t \langle r(s), \partial_t^{1+\alpha_{k_*}}u^L(s)\rangle\, ds
\leq\|r\|_{H^{\alpha_{k_*}-\gamma_J}(0,T,L^2_w(\Omega))}
\|\partial_t^{1+\alpha_{k_*}}u^L(s)\|_{H^{-(\alpha_{k_*}-\gamma_J)}(0,t,L^2_w(\Omega))}\\
&\leq \overline{d} \|\partial_t^{1+\gamma_J}u^L(s)\|_{L^2(0,t,L^2_w(\Omega))}^2
+\frac{1}{4\overline{d}\overline{C}(\alpha_{k_*}-\gamma_J)^2} \|r\|_{H^{\alpha_{k_*}-\gamma_J}(0,T,L^2_w(\Omega))}^2\\
&\rerevision{\leq \int_0^t \underline{\mathcal{E}}_\gamma[u^L](s)\, ds + C \|r\|_{H^{\alpha_{k_*}-\gamma_J}(0,T,L^2_w(\Omega))}^2}
\end{aligned} 
\]
\rerevision{with some constants $\overline{d}, C>0$.}

Altogether we arrive at the following estimate 
\begin{equation}\label{eqn:enest0}
\mathcal{E}_\gamma[u^L](t)+\mathcal{E}_\alpha[u^L](t)\leq 
\rerevision{\int_0^t \underline{\mathcal{E}}_\gamma[u^L](s)\, ds} 
+\mbox{rhs}[u^L](t) + 
\mbox{rhs}^L_{0r}(t),
\end{equation}
where $\mathcal{E}_\gamma$, \rerevision{$\underline{\mathcal{E}}_\gamma$}, $\mathcal{E}_\alpha$ are defined as in \eqref{eqn:Ea}, \rerevision{\eqref{eqn:ulEa}}, \eqref{eqn:Eb},
\[
\begin{aligned}
&\mbox{rhs}[u](t)=
\sum_{k=0}^{k_*-1}C(1+\alpha_{k_*}-\alpha_k)\overline{C}((1+\alpha_{k_*}-\alpha_k)/2)\|\sqrt{b_k}\partial_t^{(1+\alpha_{k_*}+\alpha_k)/2}\calAtil^{1/2}u\|_{L^2(0,t;L^2_w(\Omega))}^2\\
%(note that due to the normalisation $\alpha_K=1$ the first sum ends at $K-1$)
&\mbox{rhs}^L_{0,r}(t)=
1_{\gamma_J=\alpha_{k_*}\in(0,1)} \frac{t^{2(1-\alpha_{k_*})}\,\|\sqrt{d_J}u_2^L\|_{L^2_w(\Omega)}^2}{\Gamma(2-\alpha_{k_*})^2}
%\\&\hspace*{2cm}
+1_{\alpha_{k_*}\in(0,1)} \frac{t^{2(1-\alpha_{k_*})}\,\|\sqrt{b_{k_*}}\calAtil^{1/2}u_1^L\|_{L^2_w(\Omega)}^2}{\Gamma(2-\alpha_{k_*})^2}\\
&\hspace*{2cm}+\frac{1}{2\tilde{d}\overline{C}(\alpha_{k_*}-\gamma_J)^2} \|r\|_{H^{\alpha_{k_*}-\gamma_J}(0,T,L^2_w(\Omega))}^2
\end{aligned}
\]
where the boolean variable $1_B$ is equal to one if $B$ is true and vanishes otherwise.

We now substitute $\mathcal{E}_\gamma[u](t)$, $\mathcal{E}_\alpha[u](t)$ by lower bounds containing only an estimate from below of their leading order terms  $\underline{\mathcal{E}}_\gamma[u](t)$ as defined in \eqref{eqn:ulEa} and 
\begin{equation}\label{eqn:ulEb}
\underline{\mathcal{E}}_\alpha[u](t) :=
\begin{cases}
\rerevision{\underline{C}_{K,\alpha_{k_*}}} \| \partial_t^{(1+\alpha_{k_*}+\alpha_K)/2}\calAtil^{1/2}u \|_{L^2(0,t;L^2_w(\Omega))}^2
&\mbox{ if }k_*<K\\
%\frac{b_K}{4}\|\partial_t^{\alpha_K}\calAtil^{1/2}u\|_{L^\infty(0,t;L^2_w(\Omega)}^2
\rerevision{\underline{C}_{K,\alpha_K-\epsilon} \| \partial_t^{\alpha_K+(1-\epsilon)/2}\calAtil^{1/2}u \|_{L^2(0,t;L^2_w(\Omega))}^2}
&\mbox{ if }k_*=K.
\end{cases}
\end{equation}
We then obtain from \eqref{eqn:enest0}
\begin{equation}\label{eqn:enest1}
\underline{\mathcal{E}}_\gamma[u^L](t)+\underline{\mathcal{E}}_\alpha[u^L](t)\leq 
\rerevision{\int_0^t \underline{\mathcal{E}}_\gamma[u^L](s)\, ds} 
+\mbox{rhs}[u^L](t) + \mbox{rhs}_{0,r}^L(t)
+ \widetilde{\mbox{rhs}}_0^L(t)
\end{equation}
where 
\[
\widetilde{\mbox{rhs}}_0^L(t)=
1_{k_*=K,\,\alpha_K\in(0,1)}\frac{t^{2(1-\alpha_{K})}\,\|\sqrt{b_{K}}\calAtil^{1/2}u_1^L\|_{L^2_w(\Omega)}^2}{\Gamma(2-\alpha_{K})^2}\,.
\]
\rerevision{
It is readily checked that 
$
\mbox{rhs}[u](t)\leq c_{b,\alpha}(t)\underline{\mathcal{E}}_\alpha[u](t)
$
for 
\[
c_{b,\alpha}(t)= \frac{1}{\underline{C}_{K,\alpha_{k_*}}}\sum_{k=0}^{k_*-1} C(1+\alpha_{k_*}-\alpha_k)\overline{C}((1+\alpha_{k_*}-\alpha_k)/2) |b_k| \|_0I_t^{(\alpha_K-\alpha_k)/2}\|_{L^2(0,t)\to L^2(0,t)} 
.\]
Therefore, under the smallness assumption \eqref{smallnessbk}, that is, $c_{b,\alpha}(T)<1$ 
estimate \eqref{eqn:enest1} yields
\[
\begin{aligned}
&\underline{\mathcal{E}_\gamma}[u^L](t)+(1-c_{b,\alpha}(T))\underline{\mathcal{E}}_\alpha[u^L](t)
%\\&
\leq \tilde{C} \bigl(\rerevision{\int_0^t \underline{\mathcal{E}}_\gamma[u^L](s)\, ds} 
+\|u_0\|_{\dot{H}^1(\Omega)}^2 + C_0
+\|r\|_{H^{\alpha_{k_*}-\gamma_J}(0,T,L^2_w(\Omega))}^2\bigr)
\end{aligned}
\]
for some $\tilde{C}$ independent of $t$, $L$, with $C_0$ as in \eqref{eqn:C0}.
Applying Gronwall's inequality and }
re-inserting into \eqref{eqn:enest0} we end up with 
\begin{equation}\label{eqn:enest_sumfrac_L}
\begin{aligned}
&\mathcal{E}_\gamma[u^L](t)+\mathcal{E}_\alpha[u^L](t)
%\\&
\leq C(T) \bigl(\|u_0\|_{\dot{H}^1(\Omega)}^2 + C_0 +\|r\|_{H^{\alpha_{k_*}-\gamma_J}(0,T,L^2_w(\Omega))}^2\bigr).
\end{aligned}
\end{equation}


\noindent
{\bf Step 3.} weak limits.\\
As a consequence of \eqref{eqn:enest_sumfrac_L}, the Galerkin solutions $u^L$ are uniformly bounded in $U_\gamma\cap U_\alpha$. Therefore the sequence $(u^L)_{L\in\mathbb{N}}$ has a weakly(*) convergent subsequence $(u^{L_k})_{k\in\mathbb{N}}$ with limit $u\in U_\gamma\cap U_\alpha$.
To see that $u$ satisfies \eqref{eqn:sumfrac_wave_acou_ibvp_timeint}, we revisit \eqref{eqn:Galerkin}, integrate it with respect to time and take the $L^2(0,T)$ product with arbitrary smooth test functions $\psi$ to conclude that  
%\[\begin{aligned}
%&\int_0^T\psi(t)\langle \sum_{j=0}^J d_j \partial_t^{2+\gamma_j} u(t) + \sum_{k=0}^K b_k \partial_t^{\alpha_k} \calAtil u(t) - r(t),v\rangle_{L^2_w(\Omega)}\, dt \\
%&= \int_0^T\psi(t)\langle \sum_{j=0}^J d_j \partial_t^{2+\gamma_j} (u-u^{L_k})(t) + \sum_{k=0}^K b_k \partial_t^{\alpha_k} \calAtil (u-u^{L_k})(t),v\rangle_{L^2_w(\Omega)}\, dt  
%\end{aligned}\]
\[
\begin{aligned}
&\int_0^T\psi(t)\Bigl\{\langle \sum_{j=0}^J d_j \Bigl(\partial_t^{1+\gamma_j} u(t)-\frac{t^{1-\gamma_j}}{\Gamma(2-\gamma_j)}u_2\Bigr),v\rangle_{L^2_w(\Omega)} \\
&\qquad\qquad+ \langle\sum_{k=0}^K b_k \Bigl({_0I_t}^{1-\alpha_k} \calAtil u(t) - \frac{t^{1-\alpha_k}}{\Gamma(2-\alpha_k)}\calAtil u_0\Bigr) - r(t),v\rangle_{\dot{H}(\Omega)^*,\dot{H}(\Omega)}\Bigr\}\, dt \\
&= \int_0^T\psi(t)\Bigl\{\langle \sum_{j=0}^J d_j \partial_t^{1+\gamma_j} (u-u^{L_k})(t),v\rangle_{L^2_w(\Omega)}\\ 
&\qquad\qquad + \langle\sum_{k=0}^K b_k {_0I_t}^{1-\alpha_k} \calAtil (u-u^{L_k})(t),v\rangle_{\dot{H}(\Omega)^*,\dot{H}(\Omega)}\Bigr\}\, dt  
\end{aligned}
\]
for all $v\in \mbox{span}(\varphi_1,\ldots, \varphi_K)$, $K\leq L_k$ and any $\psi\in C_c^\infty(0,T)$. Taking the limit $k\to\infty$ by using the previously mentioned weak(*) limit of $u^{L_k}$ 
%and the infinite smoothness of $\psi$ that allows appropriate integration by parts with respect to time to reduce the time derivatives on the $u-u^{L_k}$ to the level covered by $U_\gamma\cap U_\alpha$, 
we conclude 
\[
%\int_0^T\psi(t)\langle \sum_{j=0}^J d_j \partial_t^{2+\gamma_j} u(t) + \sum_{k=0}^K b_k \partial_t^{\alpha_k} \calAtil u(t) - r(t),v\rangle_{L^2_w(\Omega)}\, dt = 0
\begin{aligned}
&\int_0^T\psi(t)\Bigl\{\langle \sum_{j=0}^J d_j \Bigl(\partial_t^{1+\gamma_j} u(t)-\frac{t^{1-\gamma_j}}{\Gamma(2-\gamma_j)}u_2\Bigr),v\rangle_{L^2_w(\Omega)} \\
&\qquad\qquad+ \langle\sum_{k=0}^K b_k \Bigl({_0I_t}^{1-\alpha_k} \calAtil u(t) - \frac{t^{1-\alpha_k}}{\Gamma(2-\alpha_k)}\calAtil u_0\Bigr) - r(t),v\rangle_{\dot{H}(\Omega)^*,\dot{H}(\Omega)}\Bigr\}\, dt 
=0
\end{aligned}
\]
for all $v\in \mbox{span}(\varphi_1,\ldots, \varphi_K)$ with arbitrary $K\in\mathbb{N}$, and any $\psi\in C_c^\infty(0,T)$. 
Therefore the weak limit $u$ indeed satisfies the time integrated {\sc pde} and we have proven the existence part of the theorem. 

To verify the initial conditions, we use the fact that due to our assumption $\alpha_K>0$, $U_\alpha$ continuously embeds into $C([0,T];\dot{H}(\Omega))$ and therefore $u(0)=u_0$ is attained in an $\dot{H}(\Omega)$ sense. 
Also, $U_\gamma$ continuously embeds into $C^1([0,T];L^2_w(\Omega))$ in case 
$\gamma_J<\alpha_{k_*}$ or $\gamma_J>0$. 
In the remaining case $\gamma_J=\alpha_{k_*}=0$, attainment of $u_t(0)=u_1$ in an $L^2_w(\Omega)$ sense can be shown analogously to the conventional second order wave equation, cf., e.g., \cite[Theorem 3 in Section 7.2]{Evans:2010}. 
\begin{comment}
In case of  $K=k_*$ and $\alpha_K=0$ we still have $U_\gamma\cap U_\alpha\subseteq C^1([0,T];L^2_w(\Omega))\cap C_w([0,T];\dot{H}(\Omega))$ with the subscript $w$ denoting weak continuity, by \cite[Lemma 3.3]{temam:2012}.
In the case $\gamma_J=\alpha_*=0$, we can divide by $d_J$ and make use of the proof of Corollary~\ref{cor:reg_sumfrac} to obtain uniform $H^2(0,T;\dot{H}(\Omega)^*)$ boundedness of the Galerkin approximations as well as the solution, so that \cite[Lemma 3.3]{temam:2012} applies and yields uniform boundedness in $C_w^1([0,T];L^2_w(\Omega))\cap C_{w}([0,T];\dot{H}(\Omega))$.
We can therefore take limits $k\to\infty$ along the weakly convergent subsequence $u^{L_k}$ to verify attainment of the initial conditions $(u(0),u_t(0))=(u_0,u_1)$ in an $\dot{H}(\Omega)\times L^2_w(\Omega)$ sense.
\end{comment}

Moreover, taking weak limits in the energy estimate \eqref{eqn:enest_sumfrac_L}
together with weak lower semicontinuity of the norms contained in the definitions
of $\mathcal{E}_\gamma$, $\mathcal{E}_\alpha$, implies \eqref{eqn:enest_sumfrac}.

\noindent
{\bf Step 4.} uniqueness.\\
The manipulations carried out in Step 2. of the proof are also feasible with the Galerkin approximation $u^L$ replaced by a solution $u$ of the {\sc pde} itself, and lead to the energy estimate \eqref{eqn:enest_sumfrac} independently of the Galerkin approximation procedure. 
From this we conclude that the  {\sc pde} with vanishing right hand side and initial data only has the zero solution, which due to linearity of the {\sc pde} implies uniqueness.
\end{proof}

\rerevision{
\begin{corollary}\label{cor:global}
Under the assumptions of Theorem~\ref{thm:sumfrac} 
 with $b_k=0$, $k=0,\ldots,k_*-1$, the assertion extends to hold globally in time, that is, for all $t\in(0,\infty)$. If additionally, $r=0$, then the energy is globally bounded 
\[
\mathcal{E}_\gamma[u](t)+\mathcal{E}_\alpha[u](t) \leq 
C \bigl(\|u_0\|_{\dot{H}^1(\Omega)}^2 + C_0\bigr)\,, \quad t\in(0,\infty)
\]
with some constant $C>0$ independent of time. 
\end{corollary}
}
\begin{remark}
Using the so-called multinomial (or multivariate) Mittag-Leffler functions and separation of variables in principle enables a solution representation by separation of variables, cf. \cite[Theorem 4.1]{LuchkoGorenflo:1999}.
However, proving convergence of the infinite sums in this representation would require extensive estimates of these Mittag-Leffler functions. Thus, in order to keep the exposition transparent without having to introduce too much additional machinery, we chose to remain with the energy argument in the proof above. 
\end{remark}

\begin{comment}
An inspection of the proof of Theorem~\ref{thm:sumfrac} shows that in particular coefficient settings, certain time dependent terms vanish and therefore a zero right hand side leads to a decrease of energy.
\begin{corollary}\label{cor:reg_sumfrac}
Under the conditions of Theorem~\ref{thm:sumfrac}, in case (ii) with 
%constant coefficients $b_k$ and 
$r=0$, $\alpha_0\in\{0,1\}$, the energy $\mathcal{E}_\gamma[u](t)+\mathcal{E}_\alpha[u](t)$ is a monotonically decreasing function of time.
\end{corollary}
\end{comment}

In order to establish sufficient regularity of the highest order term $\partial_t^{2+\gamma_J} u^L$ so that also the original equation \eqref{eqn:sumfrac_wave_acou_ibvp} holds, we make use of the {\sc pde} itself. 
\begin{corollary}
Under the conditions of Theorem~\ref{thm:sumfrac} 
the solution $u$ of \eqref{eqn:sumfrac_wave_acou_ibvp_timeint} 
satisfies $\partial_t^{2+\gamma_J} u\in L^2(0,T;\dot{H}(\Omega)^*)$ and is therefore also a solution to the original {\sc pde} \eqref{eqn:sumfrac_wave_acou_ibvp} in an $L^2(0,T;\dot{H}(\Omega)^*)$ sense.
\end{corollary}
\begin{proof}
From the Galerkin approximation \eqref{eqn:Galerkin} of \eqref{eqn:sumfrac_wave_acou_ibvp}, due to invariance of the space $\mbox{span}(\varphi_1,\ldots, \varphi_L)$ we can substitute $v$ by $\calAtil^{-1/2}v$ there and move $\calAtil^{-1/2}$ to the left hand side of the inner product by taking the adjoint
to arrive at 
\begin{equation}\label{eqn:Galerkin_lower}
\begin{aligned}
&\langle \sum_{j=0}^J \partial_t^{2+\gamma_j}\calAtil^{-1/2}[d_j  u^L(t)] + \sum_{k=0}^K \partial_t^{\alpha_k} \calAtil^{-1/2}[b_k  
\calAtil u^L(t)] - \calAtil^{-1/2}r(t),v\rangle_{L^2_w(\Omega)} = 0 \\
&\quad t\in(0,T) \quad \mbox{ for all }v\in\mbox{span}(\varphi_1,\ldots, \varphi_L)\,.
\end{aligned}
\end{equation}
From Theorem~\ref{thm:sumfrac} we know that 
$\tilde{r}=-\sum_{k=0}^K \partial_t^{\alpha_k} \calAtil^{-1/2}[b_k  \calAtil u(t)] + \calAtil^{-1/2}r(t)$ is contained in $L^2(0,T;L^2_w(\Omega))$. 
Thus \eqref{eqn:Galerkin_lower} can be viewed as a Galerkin discretisation of a Volterra integral equation for $\zeta =  \partial_t^{2+\gamma_J}\calAtil^{-1/2} u(t)$ with $\tilde{r}$ as right hand side. The corresponding coefficient vector $\underline{\zeta}^L$ of the solution $\zeta^L$ is therefore bounded by 
\[
\|\underline{\zeta}^L\|_{\ell^2(\mathbb{R}^L)}
\leq C(\|\tilde{r}\|_{L^2(0,T;L^2_w(\Omega))}+\|\zeta(0)\|_{L^2_w(\Omega)})
\]
with $C$ independent of $L$. From this we deduce 
\[ \|\zeta^L\|_{L^2(0,T;L^2_w(\Omega))}
\leq C(\|\tilde{r}\|_{L^2(0,T;L^2_w(\Omega))}+\|\zeta(0)\|_{L^2_w(\Omega)})
\]
and thus, by taking the limit $L\to\infty$ the same bound for $\|\zeta^L\|_{L^2(0,T;L^2_w(\Omega))}$ itself.
\end{proof}

