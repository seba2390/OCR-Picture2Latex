\subsection{Inverse problems} \label{sec:inverse}
\rerevision{
As in \cite{JinKian:2021}, we assume that data has been obtained from a quite general experimental setup, in which the {\sc pde} \eqref{eqn:general} with its unknown coefficients remains the same, while both excitation \eqref{eqn:init_source} and observation location may vary between the experiments. We number the experiments with an index $i$ to denote the corresponding excitation by $(u_{0,i},u_{1,i},f_i)$, and the corresponding observation operator by $B_i$. Given $I$ experiments, we thus 
assume to have the following $I$ time trace observations for different driving by initial and/or source data:
\begin{equation}\label{eqn:obs_titr}
\begin{aligned}
&h_i(t)=(B_i u_i)(t), \ t\in(0,T) \quad\mbox{ where $u_i$ solves \eqref{eqn:general}, \eqref{eqn:init_source} with } (u_0,u_1,f)=(u_{0,i},u_{1,i},f_i)\,,\\
&i=1,\ldots I.
\end{aligned}
\end{equation}
Examples of observation operators are 
\[
(a) \ \; B_i v= v(x_i)\quad \mbox{ or }\quad 
(b) \ \; B_i v= \int_{\Sigma_i}\eta_i(x) v(x)\, dx
\]
for some points $x_i\in\overline{\Omega}$ and some weight functions
$\eta_i\in L^\infty(\omega)$, $\Sigma_i\subseteq\partial\Omega$, provided that these evaluations are well-defined, that is, $u(t)\in C(\overline{\Omega})$ in case (a), or $u(t)\in L^1(\omega)$ in case (b), which according to Theorem~\ref{thm:sumfrac} is the case if $\Omega\subseteq\mathbb{R}^d$ with $d=1$ for case (a) or $d\in\{1,2,3\}$ for case (b). 
}
%\revision{Note again that the index $i$ corresponds to the $i$th experiment (consisting of excitation and measurements) and that between the different experiments, both exctation $(u_{0,i},u_{1,i},f_i)$ and measurement $B_i$ may change.}
\Margin{Ref 2 (i)}

The coefficients in \eqref{eqn:general} that we aim to in recover are 
%first of all the constants
%\[
%N, \ M_k, \ c, \ d_j, \ c_{k\ell}, \ \alpha_k, \ \beta_{k\ell}, \ \gamma_j, \quad
%j=1,\ldots,J, \ k=1,\ldots,N, \ell=1,\ldots,M_k
%\]
%in \eqref{eqn:mostgeneral}, which reduces to the set
\[
N, \ J, \ c, \ b_k, \ d_j, \ \alpha_k, \ \beta_k, \ \gamma_j, \quad
j=1,\ldots,J, \ k=1,\ldots,N\;.
\]

Moreover, we consider the problem of identifying, besides the differentiation orders, some space dependent quantities. 
In practice the most relevant ones are either the speed of sound $c=c(x)$ in ultrasound tonography, or the initial data $u_0=u_0(x)$ while $u_1=0$ (equivalently the source term $f=f(x)$, while $u_0=0$, $u_1=0$) in photoacoustic tomography {\sc pat}.
For the latter purpose, we will consider observations not only at single points (a) or patches (b), but over a surface $\Sigma$.
