%\documentclass[prb,twocolumn,amsmath,amssymb]{revtex4}
\documentclass[prb,preprint,amsmath,amssymb]{revtex4}
%\DeclareOption{floatfix}{abooleantrye\force@deferlist@sw}%
%\documentstyle[12pt]{article}
%\def\ssp{\def\baselinestretch{1.0}\large\normalsize}
%\def\dsp{\def\baselinestretch{1.2}\large\normalsize}

\usepackage{color}
\textheight 9.2in
%\linewidth 6.5in
\headsep 0.6in
\begin{document}
\author{Kevin Leung$^{1,*}$ and Katherine L.~Jungjohann$^2$}
\affiliation{$^1$Sandia National Laboratories, 
Albuquerque, NM 87185, United States\\
$^2$Center for Integrated Nanotechnologies, Sandia National Laboratories,
Albuquerque, NM 87185, United States\\
\tt kleung@sandia.gov}
\date{\today}
\title{Spatial Heterogeneities and Onset of Passivation 
Breakdown at Lithium Anode Interfaces}
%\setlength{\unitlength}{1 mm}
                                                                                
\input epsf
%\ssp
\renewcommand{\thetable}{\arabic{table}}

 
\begin{abstract}

Effective passivation of lithium metal surfaces, and prevention of
battery-shorting lithium dendrite growth, are critical for implementing
lithium-metal-anodes for batteries with increased power densities.  Nanoscale
surface heterogeneities can be ``hot spots'' where anode passivation breaks
down.  Motivated by the observation of lithium dendrites in pores and grain
boundaries in all-solid batteries, we examine lithium metal surfaces covered
with Li$_2$O and/or LiF thin films with grain boundaries in them.  Electronic
structure calculations show that, at $>$0.25~V computed equilibrium
overpotential, Li$_2$O grain boundaries with sufficiently large pores can
accommodate Li$^{(0)}$ atoms which aid $e^-$ leakage and passivation
breakdown.  Strain often accompanies Li-insertion; applying a $\sim$1.7\%
strain already lowers the computed overpotential to 0.1~V.  Lithium metal
nanostructures as thin as 12~\AA\, are thermodynamically favored inside
cracks in Li$_2$O films, becoming ``incipient lithium filaments.''  LiF films
are more resistant to lithium metal growth.  The models used herein should in
turn inform passivating strategies in all-solid-state batteries.

\end{abstract}

\maketitle

\section*{INTRODUCTION}
 
Lithium metal is the most gravimetrically efficient anode candidate material
for next-generation batteries.\cite{li1}  Replacing graphite with Li anode
would yield a 3$\times$ increase in anode capacity.  As Li(s) is extremely
electronegative, and reacts with almost all electrolytes in liquid
electrolyte-based lithium ion batteries (henceforth LELIB), it requires surface
passivation films to block electron tunneling to, and direct chemical contact
with the electrolyte.\cite{pnnl,cui,archer,jungjohann,dudney1} The innermost
layers of such protective films tend to be inorganic in nature.  They are
either formed naturally from electrolyte decomposition products (``solid
electrolyte interphase'' or ``SEI''), are artificial coatings, or are formed
with solid electrolytes.\cite{pearse,dudney}

Even when using protection schemes such as coating lithium metal with
passivation films, lithium dendrites may still grow
from lithium metal anodes under adverse (e.g., overpotential) conditions.
These dendrites can penetrate the separator containing the liquid electrolyte,
reach the cathode,\cite{naturetem,garcia} and cause a short circuit and possibly
a fire.  Dendrites are therefore significant battery reliability and safety
concerns.  Historically, many studies of dendrite formation in liquid
electrolyte-based lithium ion batteries (henceforth LELIB) with lithium metal
anodes have focused on homogeneous
films.\cite{newman,santosh1,holzwarth,yue1,yue}  One popular viewpoint is
adopted from electroplating of metal in water, where a solid blocking film is
absent, and dendrites are assume to arise from spontaneous, local fluctuations
of electric fields.  While this viewpoint leads to useful mitigating
strategies,\cite{pnnl} it ignores the fact that lithium nucleation and
dissolution occurs at
particular locations on the SEI film in LELIB;\cite{jungjohann} it cannot
describe the entire passivation breakdown mechanism. The influence of SEI
spatial heterogeneity has been addressed in transition electron
microscopy studies.\cite{jungjohann,naturetem}  However, most lithium dendrites
cannot be imaged in battery settings until they are at least 100~nm in diameter,
by which time their growth is rapid and unmanageable.  Atomic-scale
inhomogeneities, from which dendrites may originate, are difficult to image due
to the small length scale, low-scattering elemental composition, and the
buried nature of the interface.  Similarly, modeling
of dendrite growth dynamics has focused on the phase-field method,
which deals with meso-, not atomic, lengthscales.\cite{garcia}

Spatial inhomogeneities or ``hot spots'' on the surfaces of graphite anodes
and LiCoO$_2$ cathodes in LELIB are known to exist, leading to battery
failure.\cite{harris}  While solid electrolyte-based all-solid lithium ion
batteries (henceforth SELIB) are inherently safer than LELIB, lithium metal
also reacts with many Li$^+$-conducting solid electrolytes.  In particular,
dendrites are found to grow inside pores and grain boundaries in some solid
electrolyte materials.\cite{pore,pore1} These findings emphasize the
importance of studying spatial inhomogeneities at solid-solid interfaces.  
We hypothesize that atomic lengthscale hot spots at solid-solid interfaces
between the SEI and lithium metal anode surfaces are also the locations where
passivation starts to break down in LELIB.\cite{jungjohann}    While we focus
on the SEI in LELIB, we draw on concepts from SELIB and from electronic
materials.\cite{shluger,liair,fisher,islam,twin}  The models used herein can in
turn be applied to interfaces in SELIB studies.

\begin{figure}
\centerline{\hbox{ (a) \epsfxsize=2.20in \epsfbox{fig1a.ps} 
		   \epsfxsize=2.20in \epsfbox{fig1b.ps} (b)}}
\centerline{\hbox{ (c) \epsfxsize=2.20in \epsfbox{fig1c.ps} 
		   \epsfxsize=2.20in \epsfbox{fig1d.ps} (d)}}
\centerline{\hbox{ (e) \epsfxsize=2.20in \epsfbox{fig1e.ps} 
		   \epsfxsize=2.20in \epsfbox{fig1f.ps} (f)}}
\caption[]
{\label{fig1} \noindent
Representative systems studied in this work.  (a) $\Sigma_5$ grain boundaries
(GB) in LiF; (b) ``16$^o$'' GB in LIF; (c) 16$^o$ GB in Li$_2$O, and (d) GB in
a mixed LiF/Li$_2$O film; (e) Li$_2$O with 16$^o$ grain boundary on Li(s)
surface; (f) Li metal slab inside a $\sim$12~\AA\, crack in Li$_2$O.  
The orange lines indicate the lattice mismatch directions.
Red and pink spheres depict O and F atoms.  The colors of Li atoms depend on
their origins: silver: from Li metal anode; blue: Li$_2$O; cyan: LiF; green:
Li manually added to system and/or Li that has a localized excess electron
according to Bader charge analysis (i.e., Li$^{(0)}$).  The yellow transparent
shapes are contours of excess $e^-$ locations (see text).
}
\end{figure}

While this work is motivated by the science of dendrite formation, it
focuses on lithium metal-induced passivation breakdown on smaller lengthscales
that may cause continuous electrolyte decomposition and may be one of the
root causes of dendrites under adverse conditions.  We apply electronic
structure (Density Functional Theory or DFT) calculations to explore
electron-blockage breakdown and ``incipient lithium filament''
formation in the defect regions of the passivation films.  Understanding the
initial stages of passivation film failure and lithium growth will inform 
early diagnosis, and will potentially lead to self-healing mechanisms that
prevent catastrophic anode failure.  DFT can capture
bond-breaking events and reveal detrimental through-SEI $e^-$ conduction
pathways at sub-nanometer lengthscale; it is complementary to TEM and
phase-field studies.\cite{naturetem,garcia}  
We focus on two SEI components, Li$_2$O and LiF.
Unlike Li$_2$CO$_3$ and other organic SEI components in LELIB,\cite{batt}
Li$_2$O cannot be readily electrochemically reduced by Li(s).  A thin layer
of Li$_2$O is predicted to exist on Li surfaces as the innermost inorganic
SEI layer,\cite{batt} unless LiF, likewise stable, has been deposited first.
This innermost inorganic layer is arguably the most important SEI
component for blocking $e^-$ transport into the liquid electrolyte.\cite{yue}

Our goal is two-fold: to demonstrate that grain boundaries can aid electron
leakage through passivating films, and that cracks initated there can lead to
(sub)-nanoscale Li metal (i.e., incipient filament) growth.  Regarding $e^-$
leakage, we show that Li$^{(0)}$ atoms can reside and diffuse in Li$_2$O
grain boundaries with sufficiently large pore sizes, at $<0.25$~V computed
overpotential (see definition below) vs.~Li$^+$/Li(s) reference.  Li$^{(0)}$
has also been proposed to be $e^-$ carriers in Li$_2$CO$_3$
crystals.\cite{yueli2co3}   This is one possible mechanism responsible for
electron transfer through the
SEI.\cite{tang}  The specific models examined include grain boundaries in
crystalline LiF, Li$_2$O, the heterogeneous boundaries between them
(Fig.~\ref{fig1}a-d),
and thin films containing these grain
boundaries deposited on Li metal anode surfaces with well defined
electronic voltages\cite{solid} (Fig.~\ref{fig1}e).  Regarding Li metal
growth, back-of-the-envelope calculations suggest that Li metal nucleation
can already occur inside a 12~\AA\, pre-existing crack within a Li$_2$O film
at modest computed overpotentials (see the supporting information document,
S.I., Sec.~S1).  Fig.~\ref{fig1}f illustrates a thin Li metal growth inside
a nanometer-wide linear crack.  Other works have focused on the beneficial
effects of heterogeneous interfaces in the SEI.\cite{qi16} 

We stress that grain boundaries in crystals are non-equilibrium structures
and reflect kinetic constraints associated with crystal-growth conditions.
SEI film formation, which occurs in liquid at room temperature, is already
severely kinetically constrained, with many components being thermodynamically
metastable.\cite{batt}  The defects in SEI components are even harder to
characterize experimentally than undefected SEI regions, and models of such
are difficult construct in a systematic way to take proper account of the
kinetic formation constraints.  Cracks that develop in materials are also
clearly kinetically drive.\cite{crack} Nevertheless, it is critical to study
such defects, largely neglected in the literature.  In this work, we adopt
plausible grain boundary models from crystals in the literature\cite{shluger}
and use our own construction (Fig.~S1 in the S.I.).  We show that annealing
these models at high temperature, as has been done in some modeling
publications,\cite{dawson} actually yield ambiguous results.  High temperature
growth are crystal-growth, not SEI-formation, conditions.  To some extent, our
Li$_2$O grain boundary models can be taken as amorphous regions, which we
have postulated to result from electrochemical reduction of other SEI
products.\cite{batt} 

\section*{METHOD}

\subsection*{Construction of the Models}

Our static DFT calculations apply periodically replicated simulation cells,
the Vienna Atomic Simulation Package (VASP) version
5.3,\cite{vasp1,vasp1a,vasp2,vasp3} and the Perdew-Burke-Ernzerhof (PBE)
functional.\cite{pbe}  All simulation cells considered are overall
charge-neutral.  A 400~eV planewave energy cutoff is imposed, except
that a 500~eV cutoff is used when optimizing simulation cell sizes.
Representative simulation cell dimensions, stoichiometries, and Brillouin
zone sampling settings are listed in Table~\ref{table1}.  Other calculations
involve variations on these cells.  Many calculations
involve slab-like simulation cells with a 10-12~\AA\, vacuum region.  In
these cases, the dipole moment correction is applied.\cite{dipole}
In calculations of Li monolayer binding energies, spin-polarized DFT is
applied if there is an odd number of Li atoms in the simulation cell.
Some of these calculations apply the generally more accurate
DFT/HSE06 and DFT/PBE0 functionals.\cite{hse06a,hse06b,hse06c,pbe0}

The two main grain boundaries of interest are the (310)/[100] (henceforth
called ``$\Sigma_5$'') in LiF (001) films, and the one formed by
counter-rotating two Li$_2$O (111) slabs by 16.1$^o$ (simply referred to as
``16$^o$'').  They are chosen because of the stability of LiF (001) and Li$_2$O
(111) surfaces.  Mixed LiF/Li$_2$O boundaries are created by joining the two
surfaces of a LiF (310) slab on to Li$_2$O ($\bar{1}$10) facets
(Fig.~\ref{fig1}d).  In all cases, the $x$-direction is perpendicular to
the grain boundaries.  

The simulation cell containing two $\Sigma_5$ LiF grain boundaries
(Fig.~\ref{fig1}a) is created as follows.  First a LiF crystal at optimal
lattice constants is rotated 18.4$^o$ and cleaved to expose (310) surfaces in
the $z$-direction.  A second, mirror-image  slab is created by reflecting the
first about the $x$-$y$ plane.  The two are pasted together in the periodically
replicated simulation cell to create the two boundaries.   The $x$-dimension
of the cell is varied while using a higher (500~eV) energy cutoff to obtain
the optimal cell length.  There are multiple ways to align these simple cubic
lattice LiF slabs.  The ``coincident site lattice'' (CSL) approach,\cite{fisher}
which posits that the mirror or junction plane is a (310) plane of atoms
common to both slabs, is found to be less energetically favorable than placing
the boundary half a lattice constant from both surfaces, with the slabs shifted
so that a Li always coordinates to a~F (Fig.~\ref{fig1}a).  This configuration
is in fact adopted from Ref.~\onlinecite{shluger}. 
%Attempts to insert LiF dimers into the grain boundaries yield unfavorable
%energies after accounting for the LiF chemical potential, assumed to be equal
%to the LiF crystal cohesive energy.  

There are limited electronic structure studies of grain
boundaries in fluorite lattice structures of AB$_2$ stoichiometry relevant
to Li$_2$O.\cite{fisher,islam,shluger,brutzel} A simple $\Sigma_5$ grain
boundary is created for Li$_2$O by joining (310) facets.  Since the Li$_2$O
lattice structure is different from that of LiF, the model used in
Fig.~\ref{fig1}a is inapplicable, and the CSL approach to is
applied instead (Fig.~S4 in the S.I.).

For Li$_2$O, the problem with this $\Sigma_5$ grain boundary is that one of
its orthogonal surfaces is (001).  This is the most stable surface for LiF, but
is a high energy surface for Li$_2$O.  The lowest energy facet of Li$_2$O
is (111).\cite{li2o} $\Sigma_5$ is not compatible with a (111) film coating
the Li metal surface.  Instead, taking the (111) direction as the $z$-axis,
we rotate two Li$_2$O slabs in the $x$-$y$ plane by 16.1$^o$ in opposite
directions, join them together in a way to maximize Li-O contacts, and optimize
the $x$ lattice constant as described in the previous paragraph.  This angle
is chosen to give a modest system size with best lattice matching with the
metal surface supercell.  See Fig.~\ref{fig1}c and Table~\ref{table1} for more
details.  In this ``16$^o$'' model, manual insertion of two (but not more)
3-atom Li$_2$O formula units into the grain boundary regions is enegetically
favorable.  As will be discussed, this grain boundary contains sufficient void
space, even after insertion of the two Li$_2$O units, to effectively
accommodate Li$^{(0)}$.  Applying simulated annealing for 12~ps at 500~K and
reoptimizing the structure change the total energy of the simulation
cell by only 0.2~eV.  

\begin{table}\centering
\begin{tabular}{c|r|r|l|r} \hline
system & dimensions & stoichiometry & $k$-sampling & Figure \\ \hline
LiF $\Sigma_5$ GB &  28.53$\times$8.14$\times$12.88 & Li$_{160}$F$_{160}$ &
          2$\times$1$\times$2  & Fig.~\ref{fig1}a \\
LiF 16$^o$ GB &  23.25$\times$10.38$\times$7.05 & Li$_{96}$F$_{96}$ &
          1$\times$2$\times$2  & Fig.~\ref{fig1}b \\
Li$_2$O $\Sigma_5$ GB & 10.37$\times$4.64$\times$22.92 & Li$_{88}$O$_{44}$ &
          2$\times$4$\times$1  & Fig.~S4, S.I. \\
Li$_2$O 16$^o$ GB$^{**}$& 29.50$\times$11.82$\times$8.06 & Li$_{216}$O$_{108}$
        & 1$\times$2$\times$2  & Fig.~\ref{fig1}c \\
LiF/Li$_2$O GB & 27.03$\times$13.12$\times$8.06 & Li$_{180}$O$_{48}$F$_{84}$ &
          1$\times$1$\times$2  & Fig.~\ref{fig1}d \\
LiF GB on Li(s) &  28.53$\times$30.00$\times$19.32 & Li$_{528}$F$_{240}$ &
          1$\times$2$\times$2  & Fig.~\ref{fig3}a-b \\
Li$_2$O GB on Li(s) & 29.50$\times$23.65$\times$32.00 &
        Li$_{804}$O$_{216}$ & 1$\times$1$\times$1 & Fig.~\ref{fig3}c-d \\
LiF/Li$_2$O GB on Li(s) & 27.03$\times$13.12$\times$36.00 &
        Li$_{506}$O$_{80}$F$_{154}$ & 1$\times$1$\times$1 & Fig.~\ref{fig5}a-b\\
LiF$^*$ & 4.07$\times$4.07$\times$24.00 & Li$_{12}$F$_{12}$ &
          4$\times$4$\times$1  & Fig.~\ref{fig2}a \\
Li$_2$O$^*$ & 24.00$\times$3.28$\times$8.06 & Li$_{24}$O$_{12}$ &
          1$\times$4$\times$2  & Fig.~\ref{fig2}b \\
LiF crack on Li(s) & 27.15$\times$20.36$\times$36.00 & Li$_{463}$F$_{140}$ &
          1$\times$1$\times$1  & Fig.~\ref{fig7}a \\
Li$_2$O crack on Li(s) & 23.65$\times$29.25$\times$32.00& Li$_{566}$O$_{106}$ &
          1$\times$1$\times$1  & Fig.~\ref{fig1}f \\
\hline
\end{tabular}
\caption[]
{\label{table1} \noindent
Computational details of representative simulation cells.  ``GB'' refers to
the existence of two matching grain boundaries in the cell.  If unlabelled,
Li$_2$O GB is of the 16$^o$ variety while LiF GB is $\Sigma_5$.  The dimensions
are in \AA$^3$.  $^*$For these systems, we have found that doubling the
density of the $k$-point grid in both lateral dimensions changes the
Li monolayer binding energies by less than 0.05~eV per added Li atom.
$^{**}$Doubling the lateral $k$-point grid changes the total energy by
less than 0.001~eV/atom.
}
\end{table}

For comparison purposes, a similar 16$^o$ grain boundary model for the
LiF surface is also created by rotating LiF (111) slabs by 16.1$^o$
(Fig.~\ref{fig1}b, Table~\ref{table1}).

Finally, mixed LiF/Li$_2$O boundaries are created by joining the two surfaces
of a LiF (310) slab on to Li$_2$O ($\bar{1}$10) facets (Fig.~\ref{fig1}d)
The good lattice matching of these two surfaces
allow cations on one material surface to be coordinated to anions on the
other.  The cell size is optimized as before.  DFT-based molecular dynamics
simulations are further conducted at T=500~K for 7~ps, followed by simulated
annealing to T=100~K in a 3.5~ps trajectory.  This procedure lowers the total
energy, but does not lead to passivation of undercoordinated O$^{2-}$ at
the grain boundary.  Preliminary investigation shows that Li$^{(0)}$ readily
bind to these O$^{2-}$.  To improve passivation,
four LiF dimer units are inserted into voids between the two components so
that all O$^{2-}$ at the interfaces are coordinated to LiF.  Geometry
optimization is re-initiated.  Adding LiF units in this way is found to be
energetically favorable after subtracting the relevant LiF chemical potential,
and appears more fruitful in passivating the grain boundary region with
respect to Li$^{(0)}$ leakage than DFT-based simulated annealing.  
Since SEI formation occurs at room temperature and is kinetically controlled,
it cannot be ruled out that undercoordinated O$^{2-}$ actually exists
at these mixed grain boundaries. Our intention is to construct configurations
that are least hospitable to Li$^{(0)}$ insertion.

The systems described above are periodically replicated; the simulation
cells have no vacuum regions.  To determine the effect of Li metal in their
vicinity, we cut out $\sim 10$~\AA\, films of these materials.
Li ``interlayers'' are added to the bottom of the oxide and/or fluoride films,
such that a Li atom 2~\AA\, exists below each O$^{2-}$ and/or F$^-$ anion
(see Sec. S4 below for rationale).  These films are then placed on Li (001)
or Li (011) surfaces, and the resulting slabs are optimized.  Li (001) and
Li (011) terminations are used interchangeably because they are of similar
surface energies.\cite{holzwarth} We choose the Li(s) facet that gives the
best lattice matching with the inorganic thin film in each case.  The lateral
lattice constants of the soft Li metal are strained to match to the oxide
and/or fluoride.  Afterwards, we also apply strain to the entire systems in
the direction perpendicular to the grain boundaries by various amounts to
mimic curvatures that can develop during lithium plating through the SEI film.
If the model contains a Li$_2$O or LiF film on a Li metal surface, the $z$
direction is perpendicular to the metal surface.  Note that the unstrained
Li$_2$O-on-Li(s) system (Table~\ref{table1}) is in fact first strained by
8.4\% in three successive steps and recompressed to its original cell
dimensions.  This procedure is found to lower the energy of the unstrained
system by 0.6~eV.

For simulation cells containing interfaces with metallic Li electrode slabs,
the true instantaneous electronic voltage ($V_e$) can be computed.
At equilibrium, the Li chemical potential should be consistent with the
electronic voltage.\cite{solid}  We have not applied simulated
annealing to simulation cells with lithium metal anode present.  The
experimental lithium melting point is 180.5$^o$C, or 453.5~K.  Li surfaces
melt at even lower temperatures, making simulated annealing impractical.

\subsection*{Properties and Analysis}

When attempting to insert Li$^{(0)}$ atoms into grain boundaries,
we focus on O$^{2-}$ anions which are coordinated to 6 Li$^+$ or less,
manually add a Li near one of these O$^{2-}$, and optimize the configuration.
The reported binding energies represent the lowest energy obtained in several
insertion attempts.  In perfect crystals of Li$_2$O, each O$^{2-}$ is
coordinated to 8~Li$^+$ and each Li$^+$ to 4~O$^{2-}$.  
We have not considered inserting a Li$^+$ into any of the grain boundaries,
which would create charged simulation cells.  While the excess Li$^+$ may
enhance Li$^+$ concentration and conductivity, it does not address the
failure of passivating film with regard to blocking $e^-$ transport.

The per atom binding energy ($E_{\rm b}$) of Li$^{(0)}$ inside
grain boundaries, or of Li metal films inside cracks, is used to define the
computed overpotential ${\cal V}$.  Thus
${\cal V}$=$-[E_{\rm b}-n E_{\rm Li(s)}]/n|e|$, where $n$ is the number of Li
inserted and $|e|$ is the electronic charge.  This definition of overpotential
broadly corresponds to that of Ref.~\onlinecite{norskov}.  Experimentally,
overpotentials arise from kinetic constraints which are not specified in
this work.  Our definition of ${\cal V}$ merely reflects a convenient way to
describe the insertion energy.

Bader charge analysis\cite{bader} is used to identify localized Li$^{(0)}$
in the insulating Li$_2$O and LiF regions.  Like all charge decomposition
schemes, such analysis is approximate.  It is augmented by examining
the spatial distribution of the excess electron
$\Delta \rho_e({\bf r})$.  Here we first compute the total charge density of a
configuration.  Then the total charge density of the same configuration with
one or more Bader-identified Li$^{(0)}$ removed is computed, keeping the
simulation cell charge-neutral and all other atoms frozen.  Finally, the second
charge density is subtracted from the first.  In the figures, the yellow
transparent shapes represent $\Delta \rho_e({\bf r})$ with density values of
$\sim$0.06~$|e|$/\AA$^3$ or more.  We also plot $\Delta \rho_e(z)$, derived
from integrating $\Delta \rho_e({\bf r})$ over the $x$ and $y$ dimensions.

\section*{RESULTS}
\subsection*{Bulk-like systems with grain boundaries}

First we insert a charge-neutral Li$^{(0)}$ atom into one of the two grain
boundaries in simulation cells mimicking bulk-like SEI materials.
Fig.~\ref{fig1}a depicts an optimized $\Sigma_5$ grain boundary in LiF with
a Li$^{(0)}$ atom coordinated to two F$^-$ anions.  The binding energy is
$-1.41$~eV relative to Li metal cohesive energy (unfavorable).  In other words,
inserting this Li$^{(0)}$ requires a computed overpotential of 1.41~V relative
to the Li$^+$/Li(s) reference.  Bader charge analysis indicates that the
excess $e^-$ is located on the newly added Li. This is confirmed by plotting
the spatial distribution of the excess electron, $\Delta \rho_e({\bf r})$,
in Fig.~\ref{fig1}a.  We have also added a chain of 4~Li atoms in the grain
boundary the periodically replicated cell, which become in effect an infinite
1-D line of Li.  In this case, the computed overpotential needed is lower,
but remains an unfavorable $0.84$~V. Hence thermally-activated insertion of
Li atom into this LiF grain boundary is expected to be rare.  The 16$^o$
grain boundary for LiF (Fig.~\ref{fig1}b) yields a 1.05~V computed
overpotential for inserting one Li$^{(0)}$.  This LiF film would expose a
high energy (111) surface if it were placed on lithium metal surfaces.  Hence
it is not the focus of our studies.

Li~atom insertion into the 16.1$^o$ Li$_2$O boundary (Fig.~\ref{fig1}c) is more
favorable, only requiring an computed overpotential of 0.22~V.  Bader charge
analysis suggests that the excess $e^-$ is centered not at the added Li,
but around another Li$^+$ on the interior grain-boundary surface which is
coordinated to 3 O$^{2-}$ ions in the oxide.  In Fig.~S1 in the S.I., we show
that the SEI film is not a metallic conductor.  The excess $e^-$ resides
inside the band gap.  However, the Li$^{(0)}$ can move by multi-atom hopping.
The diffusion along the 16$^o$ grain boundary is associated with a modest,
0.79~eV barrier.  (See the S.I., Sec. S2.)  Therefore these excess $e^-$ are
reasonably mobile at room temperatures.  We also examine a $\Sigma_5$-like
grain boundary in a Li$_2$O slab.  The computed overpotential for inserting
a Li$^{(0)}$ into the void space there is $1.36$~eV, similar to that associated
with the $\Sigma_5$ boundary in LiF.  The difference between the two Li$_2$O
grain boundaries appears to have a structural origin.
In the optimized configurations after Li$^{(0)}$
insertion, the extra Li$^{(0)}$ in the 16$^o$ grain boundary simulation
cell is 1.85, 1.86, and 1.90~\AA\, from the 3 nearest O$^{2-}$, while the
distances are 1.88, 1.96, and 2.56~\AA\, in the $\Sigma_5$ simulation cell. 
The nearest distances between the added Li and existing Li$^+$ in the lattice
are 2.17 and 2.14~\AA\, in the two cases.  These distances favor Li insertion
into the 16$^o$ grain boundary, which is henceforth our focus.

\begin{figure}
\centerline{\hbox{  \epsfxsize=1.20in \epsfbox{fig2.ps}
                   \hspace*{0.05in} \epsfxsize=0.78in \epsfbox{fig2d.ps}
                    \epsfxsize=0.9in \epsfbox{fig2e.ps}
                    \epsfxsize=0.84in \epsfbox{fig2f.ps} }}
\caption[]
{\label{fig2} \noindent
Excess charge density as a function of the $z$-coordinate, $\Delta \rho_e(z)$,
for Li monolayer adsorption on to (a) LiF (110); (b) Li$_2$O (001);
and (c) Li$_2$O (110) with LiF dimer coating.  The circles indicate the
projection of atoms along the $z$-direction.  Red and pink denote O~and~F;
Li are blue and cyan, and the adsorbed Li monolayer is green.  The three
ball-and-stick images, from second left to right, illustrates systems
(a)-(c) along with contours of excess electron densities.
}
\end{figure}

\subsection*{Surface Energetics}

Given the lack of experimental characterization of defect structures
in the SEI, a qualitative understanding of the above results is needed
to establish they are generally viable, even in disordered regions.
It is possible that defect regions in SEI films may be more accuractely
described as disordered than as defects in crystalline regions.  We propose
that one qualitative difference between LiF and Li$_2$O is that LiF is
a material with a negative electron affinity,\cite{shluger} while Li$_2$O
supports surface states that can accommodate excess $e^-$.  Electron affinity
is relevant because the interior surfaces inside a grain boundary confine a
vacuum region.  We quantify this effect by comparing an adsorbed monolayer
of Li metal on LiF (001) and Li$_2$O ($\bar{1}$10) (Fig.~\ref{fig2},
Table~\ref{table2}).  Here the model Li monolayer, as opposed
to a well-isolated Li~atom on the surface, allows the use of smaller 
simulation cells, and therefore more costly hybrid DFT functionals which more
accurately describe electron localization effects. 

The computed overpotentials for adding the Li monolayer are 0.45~V and 0.23~V,
respectively.  Even Li$_2$O (111), the most stable Li$_2$O facet, exhibits a
smaller computed overpotential towards Li monolayer adsorption than LiF (001)
(Table~\ref{table2}).  Two other Li$_2$O facets actually favor Li monolayer
adsorption (Table~\ref{table2}).  There are only two (111) surfaces in Li$_2$O
crystals, and mulitple facets must be exposed at its grain boundaries.

\begin{table}\centering
\begin{tabular}{l|r|r|r|r} \hline
facet & (111) & (310) & ($\bar{1}$10) & ``16$^o$'' \\ \hline
sur.~energy & 0.54 & 1.11 &  0.94  & 1.06 \\
Li monolayer & $-0.346$ & 0.028 & $-0.228$ & 0.036 \\
\hline
\end{tabular}
\caption[]
{\label{table2} \noindent
Surface energies (J/m)$^2$ and binding energies of Li monolayer relative
to Li bulk chemical potential (eV/Li) for selected Li$_2$O facets.
Positive Li monolayer adsorption energies mean favorable adsorption.  No
attempt is made to remove strain in the Li adatom films.\cite{holzwarth}
}
\end{table}

Fig.~\ref{fig2}a-b compare the integrated differential charge densities
($\Delta \rho_e(z)$) after adding an Li monolayer to LiF (001) and
Li$_2$O ($\bar{1}$10).  On LiF (001), $\Delta \rho_e(z)$ is almost entirely
localized on the Li adatoms.  On Li$_2$O ($\bar{1}$10), a more substantial
part of $\Delta \rho_e(z)$ has leaked on to the top layer O$^{2-}$ ions.
The three-dimensional contour plots in Fig.~\ref{fig2} further confirm this
difference.  Using the HSE06 and PBE0 functionals yield energy differences
that are within 50~meV of those computed using the PBE functional, and
$\Delta \rho_e(z)$ profiles that are indistinguishable from PBE predictions.

We have also coated the Li$_2$O surface with a LiF monolayer
(Fig.~\ref{fig2}c). The computed overpotential associated with adding Li to
this surface, 0.64~V, is even less favorable than that on bare LiF (001).
LiF can originate from decomposition of PF$_6^-$ counter-ions found in organic
solvent-based electrolytes, but is more rapidly released when fluoroethylene
carbonate (FEC) additive molecules are present.\cite{fec1,fec2,fec3}
LiF appears to play a special role in electrode surface passivation.

\begin{figure}
\centerline{\hbox{ (a) \epsfxsize=2.20in \epsfbox{fig3a.ps} 
		   \epsfxsize=2.20in \epsfbox{fig3b.ps} (b)}}
\centerline{\hbox{ (c) \epsfxsize=2.20in \epsfbox{fig3c.ps} 
		   \epsfxsize=2.20in \epsfbox{fig3d.ps} (d)}}
\centerline{\hbox{ (e) \epsfxsize=2.20in \epsfbox{fig3e.ps} 
		   \epsfxsize=2.20in \epsfbox{fig3f.ps} (f)}}
\caption[]
{\label{fig3} \noindent
Side and top views of films with grain boundaries on Li metal surfaces.
(a)-(b): LiF; (c)-(d): Li$_2$O; (e)-(f): Li$_2$O strained by $\sim$12\%.
For color key, see Fig.~\ref{fig1}.  In the top view panels, the
Li metal underneath are depicted as transparent.
}
\end{figure}

\subsection*{Grain boundaries in films on lithium metal}

Whether Li$^{(0)}$ can reside inside grain boundaries, however, ultimately
depends on the interface-modified Fermi level of the anode in contact with
the surface film.  Next we attempt to insert Li$^{(0)}$ in thin films
with grain boundaries deposited on Li metal.  We avoid adding Li near the Li
metal surface, where it will simply be absorbed into the metal electrode, or
on the outer film surface.  Fig.~\ref{fig3}a-b show two perspectives of
a $\sim$10~\AA\, thick LiF film, cut from Fig.~\ref{fig1}a deposited on Li
metal.  A Li$^{(0)}$ is added approximately 6.1~\AA\, from the center of mass
of the top Li layer in the anode (Fig.~\ref{fig3}a).  The computed
overpotential is 1.49~V, similar to the LiF model without Li metal
(Fig.~\ref{fig1}a).  Bader analysis indeed reveals that the added Li has
an excess $e^-$.

The Li$_2$O film on Li(s) (Fig.~\ref{fig3}c-d) already contains two Li$^{(0)}$
atoms in one of its grain boundaries after geometry optimization; manually
inserting Li$^{(0)}$ is not needed.  One Li$^{(0)}$ is on the outer surface
coordinated only to one O$^{2-}$.  Of more interest is the other Li,
coordinated to two O$^{2-}$, halfway through the Li$_2$O film (5.6~\AA\, from
the Li metal surface).  Removing the latter Li$^{(0)}$ reveals that it has an
computed overpotential of only 0.25~V.  This overpotential is strongly
strain-dependent, and falls to a mere 0.1~V upon applying a 1.7\% strain.  The
existence of a Li$^{(0)}$ inside the Li$_2$O film makes this film
non-passivating.

Next, a 3.5~\AA, or about 12~\%, tensile strain is applied to the Li$_2$O
cell with 16$^o$ grain boundaries in the $x$-direction, in 3 successive
increments each followed by geometry optimization (Fig.~\ref{fig3}e-f).  This
mimics possible surface curvatures arising from Li plating during charging.
Silicon anodes are known to expand volumetrically by up to 400\% during
charging, while graphite can expand its c-axis spacing by $\sim$10\%.  There is
less documented data about local strain on SEI-covered Li surfaces.  We choose
a $12\%$ expansion as the outer limit.  Substantial bulging of Li metal anode
surface, and buckling of the film above it, are observed upon applying the
strain (Fig.~\ref{fig3}e).  Multiple Li-O ionic bond cleavage events occur
at one of the two grain boundaries, leaving a sub-nanometer-sized crack.
Inserting 9 Li atoms into the crack is found to require no computed
overpotential, yielding a small Li particle on the outside surface of the
Li$_2$O film (Fig.~\ref{fig1}e).  This suggests that incipient lithium metal
dendrites can nucleate on sub-nanometer defect features inside cracks in 
Li$_2$O.  Note that we do not claim to separate the effects
of strain and broken bonds.  As mentioned above, our grain-boundary
models can be thought of as amorphous regions in the SEI.

\begin{figure}
\centerline{\hbox{ \epsfxsize=2.20in \epsfbox{fig4ab.ps} \hspace*{0.1in}
	 \epsfxsize=2.40in \epsfbox{fig4c.ps} (c) }}
\caption[]
{\label{fig4} \noindent
(a)-(b) Orbital energies along the $x$-direction for the 3.5-\AA\,-stretched
Li$_2$O-covered Li surface, with and without 4 PF$_6^-$ anions
(Fig.~\ref{fig3}e and panel (c)), respectively.  Black and red circles
represent $z$$<$10~\AA\, and $z$$>$10~\AA\, contributions, while green
circles denote orbitals associated with the Li$^{(0)}$ inside the Li$_2$O
film identified using Bader charge analysis.  (c) Similar to Fig.~\ref{fig3}e,
but with 4 PF$_6^-$ added.  P and F are in green and pink.
}
\end{figure}

We also address the instantaneous electronic voltage ($V_e$) of the systems
we have studied.  The work function of the Li metal electrode, modified by
the thin film, is the absolute Fermi level of the electrode referenced to
vacuum.  Dividing the work function by $|e|$ and subtracting 1.37~V yields
$V_e$ referenced to Li$^+$/Li(s).\cite{solid} At equilibrium, $V_e$ should
be equal to the Li=(Li$^+$+$e^-$) chemical potential-derived ``equilibrium
voltage,'' or ''computed overpotential,'' discussed earlier.

The LiF- and Li$_2$O-coated Li metal surfaces, respectively Fig.~\ref{fig3}a-b
and Fig.~\ref{fig3}c-d, exhibit $V_e$=0.07~V and 0.16~V.  These computed
voltages are very close to the Li-plating potential of 0~V vs.~Li$^+$/Li(s).
Fig.~\ref{fig4}a depicts the the energies of Kohn-Sham orbitals along the
$z$-direction in the Li$_2$O film strained by 3.5~\AA\, (Fig.~\ref{fig3}e-f).
Green circles represent orbitals associated with the Li identified by Bader
charge analysis to be a Li$^{(0)}$.  The relevant, occupied localized orbital
on the Li$^{(0)}$ is about 0.4~eV below the lithium metal Fermi level.

In our model, the Li(s)/SEI-film and SEI/vacuum interfaces
separate Li(s) from the vacuum region.  In more realistic electrochemical
systems, the outer surfaces of the inorganic films are in contact with
organic SEI components and/or liquid electrolytes.  To maintain a 0.0~V
applied voltage in those more realistic systems, the charge distribution in
the SEI film, electrolyte, and their interfaces may be slightly different.
Hence we need to show that varying $V_e$ has only small effects on the 
existence of Li$^{(0)}$ inside the SEI.  

Fig.~\ref{fig4}c depicts four PF$_6^-$ adsorbed on the outer oxide surface.
PF$_6^-$ retains all its negative charge in vacuum, without the need for
solvation.  In the charge-neutral simulation cell, PF$_6^-$ induce
compensating positive charges on the Li(s) surface.  The dipole surface
density created exerts a large electric field and raises $V_e$ by
2.66~V.\cite{solid}  But this electric field is found to raise the energy
of the Li$^{(0)}$ orbital by only 0.3~eV (Fig.~\ref{fig4}b), even though
$V_e$ is raised by several times that much.  The Li$^{(0)}$ orbital remains
occupied (Fig.~\ref{fig4}b).  The reason is that the $e^-$ is localized
away from the thin film-vacuum interface, and does not experience
the entire voltage drop through the inorganic film.  Note that if the model
does not contain an interface, but is a bulk simulation cell (vanishing
electric field gradient approximation), DFT would erroneously predict that
the voltage has no effect on a charge-neutral defect like Li$^{(0)}$.

\subsection*{Grain Boundary in Mixed Li$_2$O/LiF Films}

Fig.~\ref{fig5} depicts mixed Li$_2$O/LiF films with grain boundaries on
Li metal surfaces.  When a 3~\AA\, strain is applied to this sytem (panels
(c)-(d)), cleavage of Li-F ionic bonds, rather than cleavage of Li$_2$O from
LiF, is observed.  Comparing the unstrained and strained configurations,
Fig.~\ref{fig5}a vs.~Fig.~\ref{fig5}c and Fig.~\ref{fig5}b
vs.~Fig.~\ref{fig5}d, reveals that the Li$_2$O region remains ordered
while LiF near the interface exhibits a disordered lattice structure.
Note that, by construction, the two LiF/Li$_2$O interfaces in the simulation
cell are different and respond to strain differently.  The grain boundary
atomic environment qualitatively resemble that in LiF, which does not
favor Li$^{(0)}$ insertion and has been discussed previously.  We have not
systematically examined inserting Li$^{(0)}$ here.

\begin{figure}
\centerline{\hbox{ (a) \epsfxsize=2.20in \epsfbox{fig5a.ps}
                   \epsfxsize=2.20in \epsfbox{fig5b.ps} (b)}}
\centerline{\hbox{ (c) \epsfxsize=2.20in \epsfbox{fig5c.ps}
                   \epsfxsize=2.20in \epsfbox{fig5d.ps} (d)}}
\caption[]
{\label{fig5} \noindent
Side and top views of mixed Li$_2$O/LiF films with grain boundaries on
Li metal surfaces.  (a)-(b): unstrained; (c)-(d): strained by 3~\AA.
Red and pink are O and F atoms.  Li is blue, cyan, and silver depending
on whether it starts out in Li$_2$O, LiF, or Li metal.  The Li metal
atoms underneath the film in the top view panels are depicted as transparent.
The circle indicates a disordered LiF region.
}
\end{figure}

Since Li(s) is metallic, the thin-film-coated anodes studied in this work
each exhibits a single work function or $E_{\rm F}$ at all spatial points
inside the metallic region.  The unique work function for each model electrode
is obtained by averaging over the electrostatic potential $\phi(x,y,z)$ in the
$x$-$y$ plane at a $z$ position $z_o$ sufficiently deep into the vacuum
region, and using that as the zero energy reference.  However, an effective
local voltage $V(x)$ can be defined by averaging $\phi(x,y,z_o)$ over the
$y$-direction.  This function is useful for computing $e^-$ tunneling
probability at different $x$-positions.  Fig.~\ref{fig6}a depicts this
``local voltage'' for the unstrained mixed surface film (Fig.~\ref{fig5}a-b)
along the $x$-direction perpendicular to the grain boundaries.  $V(x)$ is
found to be inhomogeneous, with variations exceeding 0.25~V and its lowest
value at one of the LiF-Li$_2$O grain boundaries.

To illustrate the implications of $V(x)$ spatial inhomogeneity, we add
a fluoroethylene carbonate (FEC) molecule, a popular electrolyte
additive,\cite{fec1,fec2,fec3} at two $x$ positions.  The location at the
left-most grain boundary is at a higher voltage.  The FEC there has not
been electrochemically reduced after a 5~ps DFT-based molecular dynamics
trajectory at T=350~K.  At the end of the trajectory (Fig.~\ref{fig6}b),
this FEC has diffused to the almost middle of the Li$_2$O region, but
remains intact.  The rightmost grain boundary is at a substantially more
negative local potential, and is a ``hot spot'' for passivation breakdown.
Placing a FEC there and initiating molecular dynamics leads to FEC reductive
decomposition within 0.5~ps (Fig.~\ref{fig6}c).  Although anecdotal, this
evidence underscores the importance of spatial inhomogeneities to electrolyte
decomposition -- even in the absence of cracks or Li$^{(0)}$ in the grain
boundaries.

\begin{figure}
\centerline{\hbox{ \epsfxsize=3.00in \epsfbox{fig6a.ps} (a)}}
\centerline{\hbox{ \epsfxsize=3.00in \epsfbox{fig6b.ps} (b)}}
\centerline{\hbox{ \epsfxsize=3.00in \epsfbox{fig6c.ps} (c)}}
\caption[]
{\label{fig6} \noindent
(a) Mean ``local potential'' along the $x$-direction for
mixed LiF/Li$_2$O grain boundaries on Li(s) without FEC molecule;
(b) FEC remains intact at high potential grain boundary region;
(c) FEC decompose at low potential grain boundary region.
The Li anode in panels (b) and (c) are out of the frame.
}
\end{figure}

\subsection*{Films with Cracks}

Finally, in view of the atomic lengthscale crack developing in strained Li$_2$O
films, we also examine surface films with wider gaps or cracks to determine
whether Li nanosheets can grow there.  Fig.~\ref{fig1}f and~Fig.~\ref{fig7}a
depict $\sim$10~\AA\, thick Li$_2$O and LiF films, both with
$\sim$12~\AA\,-wide gaps, on Li(s) surfaces.  The computed overpotentials
needed for inserting body-center-cubic Li metal into these gaps (126 and 118
Li atoms) are +0.05 and $-0.10$~V, respectively.  It is therefore energetically
favorable to insert a Li(s) nanosheet into the Li$_2$O gap but not LiF.
Fig.~\ref{fig7}b depicts the Li$_2$O film with a gap decorated with LiF dimers
at a surface density of $\sim$2.9~nm$^{-2}$.  The configuration is optimized
before introducing 118~Li inside the gap in the cell. The computed overpotential
needed to insert that Li metal sheet is $-0.07$~V, suggesting that Li insertion
may be thermodynamically favorable at 0~V vs.~Li$^+$/Li(s).  While LiF
dimer-coated Li$_2$O yields an unfavorable monolayer Li adsorption energy
(Fig.~\ref{fig2}c), the F$^-$ anions at the surface can be readily absorbed
into the added Li metal nanosheet, negating that lithium-phobic condition.  A
thicker layer of LiF is apparently needed to impede Li intrusion.  These
predictions may have impact on future attempts at passivating boundaries or
cracks in SEI.

\begin{figure}
\centerline{\hbox{ \epsfxsize=2.20in \epsfbox{fig7a.ps} (a)  
		   \epsfxsize=2.20in \epsfbox{fig7b.ps} (b)}}
\caption[]
{\label{fig7} \noindent
(a)-(b) Side views of $\sim$12~\AA\, thick Li sheets (incipient ``lithium
filament'') inside a crack in a LiF film, and a Li$_2$O film decorated by
LiF dimers, respectively.  For color key, see Fig.~\ref{fig1}.
The ovas indicate LiF decoration on the oxide surfaces.
}
\end{figure}

\section*{CONCLUSIONS}

In conclusion, this computational work illustrates the effects of atomic
lengthscale inhomogeneities in passivation films covering lithium anode
surfaces.  The lengthscale considered is too small to be conclusively imaged
using current experimental methods, but our predictions will help
motivate future experimental work.  Both LiF and Li$_2$O on Li metal
surfaces exhibit wide band gaps and block electrons if they are defect
free.\cite{yue}  But mobile Li$^{(0)}$ is found to reside in and diffuse
along Li$_2$O grain boundaries at $<0.25$~V computed overpotential if there is 
sufficient void space there.  Strain often accompanies Li$^+$ insertion
into anodes. Applying a 1.7\% strain lowers the computed overpotential needed
for Li$^{(0)}$ formation in the grain boundary to 0.1~V.  Furthermore,
there can be significant spatial fluctuations in the local potential.
As such, Li$^{(0)}$ species may be responsible
for through-SEI $e^-$ transport, initial passivation breakdown of surface
films, and slow increase of impedance at the interface as the battery ages.  
Upon further application of strain, subnanometer-sized particles of Li metal
can grow in atomic lengthscale gaps that develop at Li$_2$O grain boundaries,
forming ``incipient lithium filaments'' that may cause subsequent growth of
dendrites under adverse conditions.  These findings appear qualitatively
consistent with the fact that applying pressure, which reduces void spaces,
improves the performance of Li metal anodes.\cite{pressure} The negative
electron affinity material LiF is much more resistant to Li$^{(0)}$ insertion.
Our grain boundary models are meant to represent defected/amorphous
regions of SEI inorganic layers, the growth of which are kinetically
controlled.  Therefore we also illustrate the fundamental material and
surface difference between Li$_2$O and LiF by considering the energetics
of monolayer Li metal films on these surfaces.  The results suggest that
the difference between these materials in the SEI is likely to exist
independent of the specific model used.  This simple test can potentially be
used to examine the $e^-$-blocking ability of novel, artificial coating layers
and their defects.  In general, we postulate that additives
and new strategies need to mitigate passivation failures of Li anode
protection films at inhomogeneities, not just for defect-free films.

\section*{ACKNOWLEGEMENTS}
 
This work was performed, in part, at the Center for Integrated
Nanotechnologies, an Office of Science User Facility operated for the U.S.
Department of Energy (DOE) Office of Science. Sandia National Laboratories
is a multi-mission laboratory managed and operated by National Technology
and Engineering Solutions of Sandia, LLC., a wholly owned subsidiary of
Honeywell International, Inc., for the U.S. Department of Energy's National
Nuclear Security Administration under contract DE-NA-0003525.
KL, who performed the calculations, was supported by the Assistant Secretary
for Energy Efficiency and Renewable Energy, Office of Vehicle Technologies of
the U.S.~Department of Energy under Contract No. DE-AC02-05CH11231,
Subcontract No. 7060634 under the Advanced Batteries Materials Research (BMR)
Program.  KLJ, who provided the experimental motivation, was supported by
Nanostructures for Electrical Energy Storage (NEES), an Energy Frontier
Research Center funded by the U.S.~Department of Energy, Office of Science,
Office of Basic Energy Sciences under Award Number DESC0001160.

\section*{Supporting Information for Publication}
Supporting information is available free of charge on the ACS Publications
website at DOI: 

Details of models used; Li$^{(0)}$ diffusion along grain boundaries;
Li adsorption on planar surfaces; interfaces between crystalline
LiF/Li$_2$O and Li metal; Li$_2$O $\Sigma_5$ grain boundaries; model
with a thicker Li$_2$O surface film.

%
\bibliographystyle{./IEEEtran}
\bibliography{./IEEEabrv,./IEEEexample}

@article{kousha2,
	title={Robust Privacy-Utility Tradeoffs Under Differential Privacy and Hamming Distortion},
	author={Kousha Kalantari and Lalitha Sankar and Anand D. Sarwate},
	journal={IEEE Transactions on Information Forensics and Security},
	volume={13},
	number={11},
	pages={2816--2830},
	year={2018},
	publisher={IEEE}
}



@article{DBLP:ITjournal,
  author    = {Naanin Takbiri and
               Amir Houmansadr and
               Dennis Goeckel and
               Hossein Pishro-Nik},
  title     = {Matching Anonymized and Obfuscated Time Series to Users' Profiles
},
  journal   = {CoRR},
  volume    = {abs/1710.00197},
  year      = {2017},
  url       = {https://arxiv.org/abs/1710.00197},
  archivePrefix = {arXiv},
  eprint    = {1710.00197},
  timestamp = {Sat, 30 Sep 2017 13:03:19 GMT},
}


@inproceedings{KeConferance,
   title={Bayesian Time Series Matching and Privacy},
    author={Ke Le and 	Hossein Pishro-Nik and Dennis Goeckel},
	booktitle={51th Asilomar Conference on Signals, Systems and Computers},
   year={2017},
  	address={Pacific Grove, CA}
   }	

@article{DBLP:journals/corr/LiaoSCT17,
  author    = {Jiachun Liao and
               Lalitha Sankar and
               Fl{\'{a}}vio du Pin Calmon and
               Vincent Yan Fu Tan},
  title     = {Hypothesis Testing under Maximal Leakage Privacy Constraints},
  journal   = {CoRR},
  volume    = {abs/1701.07099},
  year      = {2017},
  url       = {http://arxiv.org/abs/1701.07099},
  timestamp = {Wed, 07 Jun 2017 14:40:47 +0200},
  biburl    = {http://dblp.uni-trier.de/rec/bib/journals/corr/LiaoSCT17},
  bibsource = {dblp computer science bibliography, http://dblp.org}
}


@inproceedings{sankar,
    title={On information-theoretic privacy with general distortion cost functions},
    author={Kalantari, Kousha and Sankar, Lalitha and  Kosut, Oliver},
    booktitle={2017 IEEE International Symposium on Information Theory (ISIT)},
    year={2017},
      	address={Aachen, Germany},

    organization={IEEE}
    
  }	


@article{hyposankar,
  author    = {Jiachun Liao and
               Lalitha Sankar and
               Vincent Y. F. Tan and
               Fl{\'{a}}vio du Pin Calmon},
  title     = {Hypothesis Testing in the High Privacy Limit},
  journal   = {CoRR},
  volume    = {abs/1607.00533},
  year      = {2016},
  url       = {http://arxiv.org/abs/1607.00533},
  timestamp = {Wed, 07 Jun 2017 14:41:19 +0200},
  biburl    = {http://dblp.uni-trier.de/rec/bib/journals/corr/LiaoSTC16},
  bibsource = {dblp computer science bibliography, http://dblp.org}
}

@article{battery15,
  author    = {Simon Li and
               Ashish Khisti and
               Aditya Mahajan},
  title     = {Privacy-Optimal Strategies for Smart Metering Systems with a Rechargeable
               Battery},
  journal   = {CoRR},
  volume    = {abs/1510.07170},
  year      = {2015},
  url       = {http://arxiv.org/abs/1510.07170},
  timestamp = {Wed, 07 Jun 2017 14:40:53 +0200},
  biburl    = {http://dblp.uni-trier.de/rec/bib/journals/corr/LiKM15},
  bibsource = {dblp computer science bibliography, http://dblp.org}
}



@inproceedings{geo2013,
   title={Geo-indistinguishability: differential privacy for location-based systems},
    author={Miguel E. Andres and Nicolas E. Bordenabe and Konstantinos Chatzikokolakis and Catuscia Palamidessi},
	booktitle={Proceedings of the 2013 ACM SIGSAC conference on Computer and communications security},

   year={2013},
   pages={901--914},
  	address={New York, NY}
   }	

@article{Yeb17,
  author    = {Min Ye and
               Alexander Barg},
  title     = {Optimal Schemes for Discrete Distribution Estimation under Locally
               Differential Privacy},
  journal   = {CoRR},
  volume    = {abs/1702.00610},
  year      = {2017},
  url       = {http://arxiv.org/abs/1702.00610},
  timestamp = {Wed, 07 Jun 2017 14:41:08 +0200},
  biburl    = {http://dblp.uni-trier.de/rec/bib/journals/corr/YeB17},
  bibsource = {dblp computer science bibliography, http://dblp.org}
}

@inproceedings{info2012,
    title={Information-Theoretic Foundations of Differential Privacy},
    author={Mir J. Darakhshan},
    booktitle={International Symposium on Foundations and Practice of Security},
    year={2012},
    organization={Springer}
  }	

@inproceedings{diff2017,
    title={Dynamic Differential Location Privacy with Personalized Error Bounds},
    author={Lei Yu and Ling Liu and Calton Pu},
    booktitle={The Network and Distributed System Security Symposium},
    year={2017}
  }	

@inproceedings{ciss2017,
    title={Fundamental Limits of Location Privacy using Anonymization},
    author={N. Takbiri and A. Houmansadr and D.L. Goeckel and H. Pishro-Nik},
    booktitle={Annual Conference on Information Science and Systems (CISS)},
    year={2017},
    organization={IEEE}
  }	

@inproceedings{sit2017,
    title={Limits of Location Privacy under Anonymization and Obfuscation},
    author={Nazanin Takbiri and Amir Houmansadr and Dennis L. Goeckel and Hossein Pishro-Nik},
    booktitle={International Symposium on Information Theory (ISIT)},
    year={2017},
    organization={IEEE}
  }	



@article{tifs2016,
	Author = {Z. Montazeri and A. Houmansadr and H. Pishro-Nik},
	Journal = {IEEE Transaction on Information Forensics and Security, to appear},
	Publisher = {IEEE},
	Title = {{Achieving Perfect Location Privacy in Wireless Devices Using Anonymization}},
	Year = {2017}}
	
	@article{matching,
  title={Asymptotically Optimal Matching of Multiple Sequences to Source Distributions and Training Sequences},
  author={Jayakrishnan Unnikrishnan},
  journal={IEEE Transactions on Information Theory},
  volume={61},
  number={1},
  pages={452-468},
  year={2015},
  publisher={IEEE}
}

@article{Naini2016,
	Author = {F. Naini and J. Unnikrishnan and P. Thiran and M. Vetterli},
	Journal = {IEEE Transactions on Information Forensics and Security},
	Publisher = {IEEE},
	Title = {Where You Are Is Who You Are: User Identification by Matching Statistics},
	 volume={11},
    number={2},
     pages={358--372},
    Year = {2016}
}



@inproceedings{montazeri2016defining,
    title={Defining perfect location privacy using anonymization},
    author={Montazeri, Zarrin and Houmansadr, Amir and Pishro-Nik, Hossein},
    booktitle={2016 Annual Conference on Information Science and Systems (CISS)},
    pages={204--209},
    year={2016},
    organization={IEEE}
  }	
	
	
@inproceedings{Mont1610Achieving,
  title={Achieving Perfect Location Privacy in Markov Models Using Anonymization},
  author={Montazeri, Zarrin and Houmansadr, Amir and H.Pishro-Nik},
  booktitle="2016 International Symposium on Information Theory and its Applications
  (ISITA2016)",
  address="Monterey, USA",
  days=30,
  month="oct",
  year=2016,
}

@misc{uber-stats,
title = {{By The Numbers 24 Amazing Uber Statistics}},
author = {Craig Smith},
note = {\url{http://expandedramblings.com/index.php/uber-statistics/}},
year=2015,
month= "September"
}


@misc{GMaps-users,
title = {{55\% of U.S. iOS users with Google Maps use it weekly}},
author = {Mike Dano},
note = {\url{http://www.fiercemobileit.com/story/55-us-ios-users-google-maps-use-it-weekly/2013-08-27}},
year=2013
}

@misc{Google-stats,
title = {{Statistics and facts about Google}},
note = {\url{http://www.statista.com/topics/1001/google/}},
}

@misc{yelp-stats,
title = {{By The Numbers: 45 Amazing Yelp Statistics}},
author = {Craig Smith},
note = {\url{http://expandedramblings.com/index.php/yelp-statistics/}},
year=2015,
month= "May"
}


@misc{Uber-hacker,
title = {{Is Uber's rider database a sitting duck for hackers?}},
author = {Craig Timberg},
note = {\url{https://www.washingtonpost.com/news/the-switch/wp/2014/12/01/is-ubers-rider-database-a-sitting-duck-for-hackers/}},
year=2014,
month= "December"
}

@misc{Uber-godview,
title = {{``God View'': Uber Investigates Its Top New York Executive For Privacy Violations}},
note = {\url{https://www.washingtonpost.com/news/the-switch/wp/2014/12/01/is-ubers-rider-database-a-sitting-duck-for-hackers/}},
year=2014,
month= "November"
}


@misc{Uber-breach-statement,
title = {{Uber Statement}},
note = {\url{http://newsroom.uber.com/2015/02/uber-statement/}},
year=2015,
month= "February"
}



@inproceedings{zhou2007privacy,
  title={Privacy-preserving detection of sybil attacks in vehicular ad hoc networks},
  author={Zhou, Tong and Choudhury, Romit Roy and Ning, Peng and Chakrabarty, Krishnendu},
  booktitle={Mobile and Ubiquitous Systems: Networking \& Services, 2007. MobiQuitous 2007. Fourth Annual International Conference on},
  pages={1--8},
  year={2007},
  organization={IEEE}
}

@article{chang2012footprint,
  title={Footprint: Detecting sybil attacks in urban vehicular networks},
  author={Chang, Shan and Qi, Yong and Zhu, Hongzi and Zhao, Jizhong and Shen, Xuemin Sherman},
  journal={Parallel and Distributed Systems, IEEE Transactions on},
  volume={23},
  number={6},
  pages={1103--1114},
  year={2012},
  publisher={IEEE}
}

@inproceedings{shokri2012protecting,
	Author = {Shokri, Reza and Theodorakopoulos, George and Troncoso, Carmela and Hubaux, Jean-Pierre and Le Boudec, Jean-Yves},
	Booktitle = {Proceedings of the 2012 ACM conference on Computer and communications security},
	Organization = {ACM},
	Pages = {617--627},
	Title = {Protecting location privacy: optimal strategy against localization attacks},
	Year = {2012}}


@inproceedings{gruteser2003anonymous,
	Author = {Gruteser, Marco and Grunwald, Dirk},
	Booktitle = {Proceedings of the 1st international conference on Mobile systems, applications and services},
	Organization = {ACM},
	Pages = {31--42},
	Title = {Anonymous usage of location-based services through spatial and temporal cloaking},
	Year = {2003}}



@inproceedings{hoh2007preserving,
	Author = {Hoh, Baik and Gruteser, Marco and Xiong, Hui and Alrabady, Ansaf},
	Booktitle = {Proceedings of the 14th ACM conference on Computer and communications security},
	Organization = {ACM},
	Pages = {161--171},
	Title = {Preserving privacy in gps traces via uncertainty-aware path cloaking},
	Year = {2007}}


@inproceedings{shokri2011quantifying,
	Author = {Shokri, Reza and Theodorakopoulos, George and Le Boudec, Jean-Yves and Hubaux, Jean-Pierre},
	Booktitle = {Security and Privacy (SP), 2011 IEEE Symposium on},
	Organization = {IEEE},
	Pages = {247--262},
	Title = {Quantifying location privacy},
	Year = {2011}}


@inproceedings{bordenabe2014optimal,
	Author = {Bordenabe, Nicol{\'a}s E and Chatzikokolakis, Konstantinos and Palamidessi, Catuscia},
	Booktitle = {Proceedings of the 2014 ACM SIGSAC Conference on Computer and Communications Security},
	Organization = {ACM},
	Pages = {251--262},
	Title = {Optimal geo-indistinguishable mechanisms for location privacy},
	Year = {2014}}




@inproceedings{freudiger2009non,
	Author = {Freudiger, Julien and Manshaei, Mohammad Hossein and Hubaux, Jean-Pierre and Parkes, David C},
	Booktitle = {Proceedings of the 16th ACM conference on Computer and communications security},
	Organization = {ACM},
	Pages = {324--337},
	Title = {On non-cooperative location privacy: a game-theoretic analysis},
	Year = {2009}}

@inproceedings{ma2009location,
	Author = {Ma, Zhendong and Kargl, Frank and Weber, Michael},
	Booktitle = {Sarnoff Symposium, 2009. SARNOFF'09. IEEE},
	Organization = {IEEE},
	Pages = {1--6},
	Title = {A location privacy metric for v2x communication systems},
	Year = {2009}}

@article{1corser2016evaluating,
  title={Evaluating Location Privacy in Vehicular Communications and Applications},
  author={Corser, George P and Fu, Huirong and Banihani, Abdelnasser},
  journal={IEEE Transactions on Intelligent Transportation Systems},
  volume={17},
  number={9},
  pages={2658-2667},
  year={2016},
  publisher={IEEE}
}
@article{2zhang2016designing,
  title={On Designing Satisfaction-Ratio-Aware Truthful Incentive Mechanisms for k-Anonymity Location Privacy},
  author={Zhang, Yuan and Tong, Wei and Zhong, Sheng},
  journal={IEEE Transactions on Information Forensics and Security},
  volume={11},
  number={11},
  pages={2528--2541},
  year={2016},
  publisher={IEEE}
}

@article{11dewri2014exploiting,
  title={Exploiting service similarity for privacy in location-based search queries},
  author={Dewri, Rinku and Thurimella, Ramakrisha},
  journal={IEEE Transactions on Parallel and Distributed Systems},
  volume={25},
  number={2},
  pages={374--383},
  year={2014},
  publisher={IEEE}
}

@inproceedings{gedik2005location,
	Author = {Gedik, Bu{\u{g}}ra and Liu, Ling},
	Booktitle = {Distributed Computing Systems, 2005. ICDCS 2005. Proceedings. 25th IEEE International Conference on},
	Organization = {IEEE},
	Pages = {620--629},
	Title = {Location privacy in mobile systems: A personalized anonymization model},
	Year = {2005}}

@inproceedings{zhong2009distributed,
	Author = {Zhong, Ge and Hengartner, Urs},
	Booktitle = {Pervasive Computing and Communications, 2009. PerCom 2009. IEEE International Conference on},
	Organization = {IEEE},
	Pages = {1--10},
	Title = {A distributed k-anonymity protocol for location privacy},
	Year = {2009}}

@inproceedings{mokbel2006new,
	Author = {Mokbel, Mohamed F and Chow, Chi-Yin and Aref, Walid G},
	Booktitle = {Proceedings of the 32nd international conference on Very large data bases},
	Organization = {VLDB Endowment},
	Pages = {763--774},
	Title = {The new Casper: query processing for location services without compromising privacy},
	Year = {2006}}

@article{kalnis2007preventing,
	Author = {Kalnis, Panos and Ghinita, Gabriel and Mouratidis, Kyriakos and Papadias, Dimitris},
	Journal = {Knowledge and Data Engineering, IEEE Transactions on},
	Number = {12},
	Pages = {1719--1733},
	Publisher = {IEEE},
	Title = {Preventing location-based identity inference in anonymous spatial queries},
	Volume = {19},
	Year = {2007}}


@article{sweeney2002k,
	Author = {Sweeney, Latanya},
	Journal = {International Journal of Uncertainty, Fuzziness and Knowledge-Based Systems},
	Number = {05},
	Pages = {557--570},
	Publisher = {World Scientific},
	Title = {k-anonymity: A model for protecting privacy},
	Volume = {10},
	Year = {2002}}

@article{sweeney2002achieving,
	Author = {Sweeney, Latanya},
	Journal = {International Journal of Uncertainty, Fuzziness and Knowledge-Based Systems},
	Number = {05},
	Pages = {571-588},
	Publisher = {World Scientific},
	Title = {Achieving k-anonymity privacy protection using generalization and suppression},
	Volume = {10},
	Year = {2002}}

@inproceedings{liu2013game,
	Author = {Liu, Xinxin and Liu, Kaikai and Guo, Linke and Li, Xiaolin and Fang, Yuguang},
	Booktitle = {INFOCOM, 2013 Proceedings IEEE},
	Organization = {IEEE},
	Pages = {2985--2993},
	Title = {A game-theoretic approach for achieving k-anonymity in location based services},
	Year = {2013}}

@inproceedings{hoh2005protecting,
	Author = {Hoh, Baik and Gruteser, Marco},
	Booktitle = {Security and Privacy for Emerging Areas in Communications Networks, 2005. SecureComm 2005. First International Conference on},
	Organization = {IEEE},
	Pages = {194--205},
	Title = {Protecting location privacy through path confusion},
	Year = {2005}}

@article{beresford2003location,
	Author = {Beresford, Alastair R and Stajano, Frank},
	Journal = {IEEE Pervasive computing},
	Number = {1},
	Pages = {46--55},
	Publisher = {IEEE},
	Title = {Location privacy in pervasive computing},
	Year = {2003}}


@inproceedings{palanisamy2011mobimix,
	Author = {Palanisamy, Balaji and Liu, Ling},
	Booktitle = {Data Engineering (ICDE), 2011 IEEE 27th International Conference on},
	Organization = {IEEE},
	Pages = {494--505},
	Title = {Mobimix: Protecting location privacy with mix-zones over road networks},
	Year = {2011}}
@inproceedings{freudiger2009optimal,
	Author = {Freudiger, Julien and Shokri, Reza and Hubaux, Jean-Pierre},
	Booktitle = {Privacy enhancing technologies},
	Organization = {Springer},
	Pages = {216--234},
	Title = {On the optimal placement of mix zones},
	Year = {2009}}

@article{manshaei2013game,
	Author = {Manshaei, Mohammad Hossein and Zhu, Quanyan and Alpcan, Tansu and Bac{\c{s}}ar, Tamer and Hubaux, Jean-Pierre},
	Journal = {ACM Computing Surveys (CSUR)},
	Number = {3},
	Pages = {25},
	Publisher = {ACM},
	Title = {Game theory meets network security and privacy},
	Volume = {45},
	Year = {2013}}

@article{19freudiger2013non,
  title={Non-cooperative location privacy},
  author={Freudiger, Julien and Manshaei, Mohammad Hossein and Hubaux, Jean-Pierre and Parkes, David C},
  journal={IEEE Transactions on Dependable and Secure Computing},
  volume={10},
  number={2},
  pages={84--98},
  year={2013},
  publisher={IEEE}
}


@article{paulet2014privacy,
	Author = {Paulet, Russell and Kaosar, Md Golam and Yi, Xun and Bertino, Elisa},
	Journal = {Knowledge and Data Engineering, IEEE Transactions on},
	Number = {5},
	Pages = {1200--1210},
	Publisher = {IEEE},
	Title = {Privacy-preserving and content-protecting location based queries},
	Volume = {26},
	Year = {2014}}

@article{khoshgozaran2011location,
	Author = {Khoshgozaran, Ali and Shahabi, Cyrus and Shirani-Mehr, Houtan},
	Journal = {Knowledge and Information Systems},
	Number = {3},
	Pages = {435--465},
	Publisher = {Springer},
	Title = {Location privacy: going beyond K-anonymity, cloaking and anonymizers},
	Volume = {26},
	Year = {2011}}

@article{18shokri2014hiding,
  title={Hiding in the mobile crowd: Locationprivacy through collaboration},
  author={Shokri, Reza and Theodorakopoulos, George and Papadimitratos, Panos and Kazemi, Ehsan and Hubaux, Jean-Pierre},
  journal={IEEE transactions on dependable and secure computing},
  volume={11},
  number={3},
  pages={266--279},
  year={2014},
  publisher={IEEE}
}

@article{8zurbaran2015near,
  title={Near-Rand: Noise-based Location Obfuscation Based on Random Neighboring Points},
  author={Zurbaran, Mayra Alejandra and Avila, Karen and Wightman, Pedro and Fernandez, Michael},
  journal={IEEE Latin America Transactions},
  volume={13},
  number={11},
  pages={3661--3667},
  year={2015},
  publisher={IEEE}
}

@inproceedings{hoh2007preserving,
	Author = {Hoh, Baik and Gruteser, Marco and Xiong, Hui and Alrabady, Ansaf},
	Booktitle = {Proceedings of the 14th ACM conference on Computer and communications security},
	Organization = {ACM},
	Pages = {161--171},
	Title = {Preserving privacy in gps traces via uncertainty-aware path cloaking},
	Year = {2007}}

@inproceedings{ho2011differential,
	Author = {Ho, Shen-Shyang and Ruan, Shuhua},
	Booktitle = {Proceedings of the 4th ACM SIGSPATIAL International Workshop on Security and Privacy in GIS and LBS},
	Organization = {ACM},
	Pages = {17--24},
	Title = {Differential privacy for location pattern mining},
	Year = {2011}}


@article{12hwang2014novel,
  title={A novel time-obfuscated algorithm for trajectory privacy protection},
  author={Hwang, Ren-Hung and Hsueh, Yu-Ling and Chung, Hao-Wei},
  journal={IEEE Transactions on Services Computing},
  volume={7},
  number={2},
  pages={126--139},
  year={2014},
  publisher={IEEE}
  }

  @article{16haghnegahdar2014privacy,
  title={Privacy Risks in Publishing Mobile Device Trajectories},
  author={Haghnegahdar, Alireza and Khabbazian, Majid and Bhargava, Vijay K},
  journal={IEEE Wireless Communications Letters},
  volume={3},
  number={3},
  pages={241--244},
  year={2014},
  publisher={IEEE}
}

@article{20gao2013trpf,
  title={TrPF: A trajectory privacy-preserving framework for participatory sensing},
  author={Gao, Sheng and Ma, Jianfeng and Shi, Weisong and Zhan, Guoxing and Sun, Cong},
  journal={IEEE Transactions on Information Forensics and Security},
  volume={8},
  number={6},
  pages={874--887},
  year={2013},
  publisher={IEEE}
}

@article{21ma2013privacy,
  title={Privacy vulnerability of published anonymous mobility traces},
  author={Ma, Chris YT and Yau, David KY and Yip, Nung Kwan and Rao, Nageswara SV},
  journal={IEEE/ACM Transactions on Networking},
  volume={21},
  number={3},
  pages={720--733},
  year={2013},
  publisher={IEEE}
}

@article{6li2016privacy,
  title={Privacy Leakage of Location Sharing in Mobile Social Networks: Attacks and Defense},
  author={Li, Huaxin and Zhu, Haojin and Du, Suguo and Liang, Xiaohui and Shen, Xuemin},
  journal={IEEE Transactions on Dependable and Secure Computing},
  year={2016},
  volume={PP},
  number={99},
  publisher={IEEE}
}


@article{14zhang2014privacy,
  title={Privacy quantification model based on the Bayes conditional risk in Location-Based Services},
  author={Zhang, Xuejun and Gui, Xiaolin and Tian, Feng and Yu, Si and An, Jian},
  journal={Tsinghua Science and Technology},
  volume={19},
  number={5},
  pages={452--462},
  year={2014},
  publisher={TUP}
}

@article{4olteanu2016quantifying,
  title={Quantifying Interdependent Privacy Risks with Location Data},
  author={Olteanu, Alexandra-Mihaela and Huguenin, K{\'e}vin and Shokri, Reza and Humbert, Mathias and Hubaux, Jean-Pierre},
  journal={IEEE Transactions on Mobile Computing},
  year={2016},
  volume={PP},
  number={99},
  pages={1-1},
  publisher={IEEE}
}















@misc{Leberknight2010,
	Author = {Leberknight, C. and Chiang, M. and Poor, H. and Wong, F.},
	Howpublished = {\url{http://www.princeton.edu/~chiangm/anticensorship.pdf}},
	Title = {{A Taxonomy of Internet Censorship and Anti-censorship}},
	Year = {2010}}

@techreport{ultrasurf-analysis,
	Author = {Appelbaum, Jacob},
	Institution = {The Tor Project},
	Title = {{Technical analysis of the Ultrasurf proxying software}},
	Url = {http://scholar.google.com/scholar?hl=en\&btnG=Search\&q=intitle:Technical+analysis+of+the+Ultrasurf+proxying+software\#0},
	Year = {2012},
	Bdsk-Url-1 = {http://scholar.google.com/scholar?hl=en%5C&btnG=Search%5C&q=intitle:Technical+analysis+of+the+Ultrasurf+proxying+software%5C#0}}

@misc{gifc:07,
	Howpublished = {\url{http://www.internetfreedom.org/archive/Defeat\_Internet\_Censorship\_White\_Paper.pdf}},
	Key = {defeatcensorship},
	Publisher = {Global Internet Freedom Consortium (GIFC)},
	Title = {{Defeat Internet Censorship: Overview of Advanced Technologies and Products}},
	Type = {White Paper},
	Year = {2007}}

@article{pan2011survey,
	Author = {Pan, J. and Paul, S. and Jain, R.},
	Journal = {Communications Magazine, IEEE},
	Number = {7},
	Pages = {26--36},
	Publisher = {IEEE},
	Title = {{A Survey of the Research on Future Internet Architectures}},
	Volume = {49},
	Year = {2011}}

@misc{nsf-fia,
	Howpublished = {\url{http://www.nets-fia.net/}},
	Key = {FIA},
	Title = {{NSF Future Internet Architecture Project}}}

@misc{NDN,
	Howpublished = {\url{http://www.named- data.net}},
	Key = {NDN},
	Title = {{Named Data Networking Project}}}

@inproceedings{MobilityFirst,
	Author = {Seskar, I. and Nagaraja, K. and Nelson, S. and Raychaudhuri, D.},
	Booktitle = {Asian Internet Engineering Conference},
	Title = {{Mobilityfirst Future internet Architecture Project}},
	Year = {2011}}

@incollection{NEBULA,
	Author = {Anderson, T. and Birman, K. and Broberg, R. and Caesar, M. and Comer, D. and Cotton, C. and Freedman, M.~J. and Haeberlen, A. and Ives, Z.~G. and Krishnamurthy, A. and others},
	Booktitle = {The Future Internet},
	Pages = {16--26},
	Publisher = {Springer},
	Title = {{The NEBULA Future Internet Architecture}},
	Year = {2013}}

@inproceedings{XIA,
	Author = {Anand, A. and Dogar, F. and Han, D. and Li, B. and Lim, H. and Machado, M. and Wu, W. and Akella, A. and Andersen, D.~G. and Byers, J.~W. and others},
	Booktitle = {ACM Workshop on Hot Topics in Networks},
	Pages = {2},
	Title = {{XIA: An Architecture for an Evolvable and Trustworthy Internet}},
	Year = {2011}}

@inproceedings{ChoiceNet,
	Author = {Rouskas, G.~N. and Baldine, I. and Calvert, K.~L. and Dutta, R. and Griffioen, J. and Nagurney, A. and Wolf, T.},
	Booktitle = {ONDM},
	Title = {{ChoiceNet: Network Innovation Through Choice}},
	Year = {2013}}

@misc{nsf-find,
	Howpublished = {http://www.nets-find.net/},
	Title = {{NSF NeTS FIND Initiative}}}

@article{traid,
	Author = {Cheriton, D.~R. and Gritter, M.},
	Title = {{TRIAD: A New Next-Generation Internet Architecture}},
	Year = {2000}}

@inproceedings{dona,
	Author = {Koponen, T. and Chawla, M. and Chun, B-G. and Ermolinskiy, A. and Kim, K.~H. and Shenker, S. and Stoica, I.},
	Booktitle = {ACM SIGCOMM Computer Communication Review},
	Number = {4},
	Organization = {ACM},
	Pages = {181--192},
	Title = {{A Data-Oriented (and Beyond) Network Architecture}},
	Volume = {37},
	Year = {2007}}

@misc{ultrasurf,
	Howpublished = {\url{http://www.ultrareach.com}},
	Key = {ultrasurf},
	Title = {{Ultrasurf}}}

@misc{tor-bridge,
	Author = {Dingledine, R. and Mathewson, N.},
	Howpublished = {\url{https://svn.torproject.org/svn/projects/design-paper/blocking.html}},
	Title = {{Design of a Blocking-Resistant Anonymity System}}}

@inproceedings{McLachlanH09,
	Author = {J. McLachlan and N. Hopper},
	Booktitle = {WPES},
	Title = {{On the Risks of Serving Whenever You Surf: Vulnerabilities in Tor's Blocking Resistance Design}},
	Year = {2009}}

@inproceedings{mahdian2010,
	Author = {Mahdian, M.},
	Booktitle = {{Fun with Algorithms}},
	Title = {{Fighting Censorship with Algorithms}},
	Year = {2010}}

@inproceedings{McCoy2011,
	Author = {McCoy, D. and Morales, J.~A. and Levchenko, K.},
	Booktitle = {FC},
	Title = {{Proximax: A Measurement Based System for Proxies Dissemination}},
	Year = {2011}}

@inproceedings{Sovran2008,
	Author = {Sovran, Y. and Libonati, A. and Li, J.},
	Booktitle = {IPTPS},
	Title = {{Pass it on: Social Networks Stymie Censors}},
	Year = {2008}}

@inproceedings{rbridge,
	Author = {Wang, Q. and Lin, Zi and Borisov, N. and Hopper, N.},
	Booktitle = {{NDSS}},
	Title = {{rBridge: User Reputation based Tor Bridge Distribution with Privacy Preservation}},
	Year = {2013}}

@inproceedings{telex,
	Author = {Wustrow, E. and Wolchok, S. and Goldberg, I. and Halderman, J.},
	Booktitle = {{USENIX Security}},
	Title = {{Telex: Anticensorship in the Network Infrastructure}},
	Year = {2011}}

@inproceedings{cirripede,
	Author = {Houmansadr, A. and Nguyen, G. and Caesar, M. and Borisov, N.},
	Booktitle = {CCS},
	Title = {{Cirripede: Circumvention Infrastructure Using Router Redirection with Plausible Deniability}},
	Year = {2011}}

@inproceedings{decoyrouting,
	Author = {Karlin, J. and Ellard, D. and Jackson, A. and Jones, C. and Lauer, G. and Mankins, D. and Strayer, W.},
	Booktitle = {{FOCI}},
	Title = {{Decoy Routing: Toward Unblockable Internet Communication}},
	Year = {2011}}

@inproceedings{routing-around-decoys,
	Author = {M.~Schuchard and J.~Geddes and C.~Thompson and N.~Hopper},
	Booktitle = {{CCS}},
	Title = {{Routing Around Decoys}},
	Year = {2012}}

@inproceedings{parrot,
	Author = {A. Houmansadr and C. Brubaker and V. Shmatikov},
	Booktitle = {IEEE S\&P},
	Title = {{The Parrot is Dead: Observing Unobservable Network Communications}},
	Year = {2013}}

@misc{knock,
	Author = {T. Wilde},
	Howpublished = {\url{https://blog.torproject.org/blog/knock-knock-knockin-bridges-doors}},
	Title = {{Knock Knock Knockin' on Bridges' Doors}},
	Year = {2012}}

@inproceedings{china-tor,
	Author = {Winter, P. and Lindskog, S.},
	Booktitle = {{FOCI}},
	Title = {{How the Great Firewall of China Is Blocking Tor}},
	Year = {2012}}

@misc{discover-bridge,
	Howpublished = {\url{https://blog.torproject.org/blog/research-problems-ten-ways-discover-tor-bridges}},
	Key = {tenways},
	Title = {{Ten Ways to Discover Tor Bridges}}}

@inproceedings{freewave,
	Author = {A.~Houmansadr and T.~Riedl and N.~Borisov and A.~Singer},
	Booktitle = {{NDSS}},
	Title = {{I Want My Voice to Be Heard: IP over Voice-over-IP for Unobservable Censorship Circumvention}},
	Year = 2013}

@inproceedings{censorspoofer,
	Author = {Q. Wang and X. Gong and G. Nguyen and A. Houmansadr and N. Borisov},
	Booktitle = {CCS},
	Title = {{CensorSpoofer: Asymmetric Communication Using IP Spoofing for Censorship-Resistant Web Browsing}},
	Year = {2012}}

@inproceedings{skypemorph,
	Author = {H. Moghaddam and B. Li and M. Derakhshani and I. Goldberg},
	Booktitle = {CCS},
	Title = {{SkypeMorph: Protocol Obfuscation for Tor Bridges}},
	Year = {2012}}

@inproceedings{stegotorus,
	Author = {Weinberg, Z. and Wang, J. and Yegneswaran, V. and Briesemeister, L. and Cheung, S. and Wang, F. and Boneh, D.},
	Booktitle = {CCS},
	Title = {{StegoTorus: A Camouflage Proxy for the Tor Anonymity System}},
	Year = {2012}}

@techreport{dust,
	Author = {{B.~Wiley}},
	Howpublished = {\url{http://blanu.net/ Dust.pdf}},
	Institution = {School of Information, University of Texas at Austin},
	Title = {{Dust: A Blocking-Resistant Internet Transport Protocol}},
	Year = {2011}}

@inproceedings{FTE,
	Author = {K.~Dyer and S.~Coull and T.~Ristenpart and T.~Shrimpton},
	Booktitle = {CCS},
	Title = {{Protocol Misidentification Made Easy with Format-Transforming Encryption}},
	Year = {2013}}

@inproceedings{fp,
	Author = {Fifield, D. and Hardison, N. and Ellithrope, J. and Stark, E. and Dingledine, R. and Boneh, D. and Porras, P.},
	Booktitle = {PETS},
	Title = {{Evading Censorship with Browser-Based Proxies}},
	Year = {2012}}

@misc{obfsproxy,
	Howpublished = {\url{https://www.torproject.org/projects/obfsproxy.html.en}},
	Key = {obfsproxy},
	Publisher = {The Tor Project},
	Title = {{A Simple Obfuscating Proxy}}}

@inproceedings{Tor-instead-of-IP,
	Author = {Liu, V. and Han, S. and Krishnamurthy, A. and Anderson, T.},
	Booktitle = {HotNets},
	Title = {{Tor instead of IP}},
	Year = {2011}}

@misc{roger-slides,
	Howpublished = {\url{https://svn.torproject.org/svn/projects/presentations/slides-28c3.pdf}},
	Key = {torblocking},
	Title = {{How Governments Have Tried to Block Tor}}}

@inproceedings{infranet,
	Author = {Feamster, N. and Balazinska, M. and Harfst, G. and Balakrishnan, H. and Karger, D.},
	Booktitle = {USENIX Security},
	Title = {{Infranet: Circumventing Web Censorship and Surveillance}},
	Year = {2002}}

@inproceedings{collage,
	Author = {S.~Burnett and N.~Feamster and S.~Vempala},
	Booktitle = {USENIX Security},
	Title = {{Chipping Away at Censorship Firewalls with User-Generated Content}},
	Year = {2010}}

@article{anonymizer,
	Author = {Boyan, J.},
	Journal = {Computer-Mediated Communication Magazine},
	Month = sep,
	Number = {9},
	Title = {{The Anonymizer: Protecting User Privacy on the Web}},
	Volume = {4},
	Year = {1997}}

@article{schulze2009internet,
	Author = {Schulze, H. and Mochalski, K.},
	Journal = {IPOQUE Report},
	Pages = {351--362},
	Title = {Internet Study 2008/2009},
	Volume = {37},
	Year = {2009}}

@inproceedings{cya-ccs13,
	Author = {J.~Geddes and M.~Schuchard and N.~Hopper},
	Booktitle = {{CCS}},
	Title = {{Cover Your ACKs: Pitfalls of Covert Channel Censorship Circumvention}},
	Year = {2013}}

@inproceedings{andana,
	Author = {DiBenedetto, S. and Gasti, P. and Tsudik, G. and Uzun, E.},
	Booktitle = {{NDSS}},
	Title = {{ANDaNA: Anonymous Named Data Networking Application}},
	Year = {2012}}

@inproceedings{darkly,
	Author = {Jana, S. and Narayanan, A. and Shmatikov, V.},
	Booktitle = {IEEE S\&P},
	Title = {{A Scanner Darkly: Protecting User Privacy From Perceptual Applications}},
	Year = {2013}}

@inproceedings{NS08,
	Author = {A.~Narayanan and V.~Shmatikov},
	Booktitle = {IEEE S\&P},
	Title = {Robust de-anonymization of large sparse datasets},
	Year = {2008}}

@inproceedings{NS09,
	Author = {A.~Narayanan and V.~Shmatikov},
	Booktitle = {IEEE S\&P},
	Title = {De-anonymizing social networks},
	Year = {2009}}

@inproceedings{memento,
	Author = {Jana, S. and Shmatikov, V.},
	Booktitle = {IEEE S\&P},
	Title = {{Memento: Learning secrets from process footprints}},
	Year = {2012}}

@misc{plugtor,
	Howpublished = {\url{https://www.torproject.org/docs/pluggable-transports.html.en}},
	Key = {PluggableTransports},
	Publisher = {The Tor Project},
	Title = {{Tor: Pluggable transports}}}

@misc{psiphon,
	Author = {J.~Jia and P.~Smith},
	Howpublished = {\url{http://www.cdf.toronto.edu/~csc494h/reports/2004-fall/psiphon_ae.html}},
	Title = {{Psiphon: Analysis and Estimation}},
	Year = 2004}

@misc{china-github,
	Howpublished = {\url{http://mobile.informationweek.com/80269/show/72e30386728f45f56b343ddfd0fdb119/}},
	Key = {github},
	Title = {{China's GitHub Censorship Dilemma}}}

@inproceedings{txbox,
	Author = {Jana, S. and Porter, D. and Shmatikov, V.},
	Booktitle = {IEEE S\&P},
	Title = {{TxBox: Building Secure, Efficient Sandboxes with System Transactions}},
	Year = {2011}}

@inproceedings{airavat,
	Author = {I. Roy and S. Setty and A. Kilzer and V. Shmatikov and E. Witchel},
	Booktitle = {NSDI},
	Title = {{Airavat: Security and Privacy for MapReduce}},
	Year = {2010}}

@inproceedings{osdi12,
	Author = {A. Dunn and M. Lee and S. Jana and S. Kim and M. Silberstein and Y. Xu and V. Shmatikov and E. Witchel},
	Booktitle = {OSDI},
	Title = {{Eternal Sunshine of the Spotless Machine: Protecting Privacy with Ephemeral Channels}},
	Year = {2012}}

@inproceedings{ymal,
	Author = {J. Calandrino and A. Kilzer and A. Narayanan and E. Felten and V. Shmatikov},
	Booktitle = {IEEE S\&P},
	Title = {{``You Might Also Like:'' Privacy Risks of Collaborative Filtering}},
	Year = {2011}}

@inproceedings{srivastava11,
	Author = {V. Srivastava and M. Bond and K. McKinley and V. Shmatikov},
	Booktitle = {PLDI},
	Title = {{A Security Policy Oracle: Detecting Security Holes Using Multiple API Implementations}},
	Year = {2011}}

@inproceedings{chen-oakland10,
	Author = {Chen, S. and Wang, R. and Wang, X. and Zhang, K.},
	Booktitle = {IEEE S\&P},
	Title = {{Side-Channel Leaks in Web Applications: A Reality Today, a Challenge Tomorrow}},
	Year = {2010}}

@book{kerck,
	Author = {Kerckhoffs, A.},
	Publisher = {University Microfilms},
	Title = {{La cryptographie militaire}},
	Year = {1978}}

@inproceedings{foci11,
	Author = {J. Karlin and D. Ellard and A.~Jackson and C.~ Jones and G. Lauer and D. Mankins and W.~T.~Strayer},
	Booktitle = {FOCI},
	Title = {{Decoy Routing: Toward Unblockable Internet Communication}},
	Year = 2011}

@inproceedings{sun02,
	Author = {Sun, Q. and Simon, D.~R. and Wang, Y. and Russell, W. and Padmanabhan, V. and Qiu, L.},
	Booktitle = {IEEE S\&P},
	Title = {{Statistical Identification of Encrypted Web Browsing Traffic}},
	Year = {2002}}

@inproceedings{danezis,
	Author = {Murdoch, S.~J. and Danezis, G.},
	Booktitle = {IEEE S\&P},
	Title = {{Low-Cost Traffic Analysis of Tor}},
	Year = {2005}}

@inproceedings{pakicensorship,
	Author = {Z.~Nabi},
	Booktitle = {FOCI},
	Title = {The Anatomy of {Web} Censorship in {Pakistan}},
	Year = {2013}}

@inproceedings{irancensorship,
	Author = {S.~Aryan and H.~Aryan and A.~Halderman},
	Booktitle = {FOCI},
	Title = {Internet Censorship in {Iran}: {A} First Look},
	Year = {2013}}

@inproceedings{ford10efficient,
	Author = {Amittai Aviram and Shu-Chun Weng and Sen Hu and Bryan Ford},
	Booktitle = {\bibconf[9th]{OSDI}{USENIX Symposium on Operating Systems Design and Implementation}},
	Location = {Vancouver, BC, Canada},
	Month = oct,
	Title = {Efficient System-Enforced Deterministic Parallelism},
	Year = 2010}

@inproceedings{ford10determinating,
	Author = {Amittai Aviram and Sen Hu and Bryan Ford and Ramakrishna Gummadi},
	Booktitle = {\bibconf{CCSW}{ACM Cloud Computing Security Workshop}},
	Location = {Chicago, IL},
	Month = oct,
	Title = {Determinating Timing Channels in Compute Clouds},
	Year = 2010}

@inproceedings{ford12plugging,
	Author = {Bryan Ford},
	Booktitle = {\bibconf[4th]{HotCloud}{USENIX Workshop on Hot Topics in Cloud Computing}},
	Location = {Boston, MA},
	Month = jun,
	Title = {Plugging Side-Channel Leaks with Timing Information Flow Control},
	Year = 2012}

@inproceedings{ford12icebergs,
	Author = {Bryan Ford},
	Booktitle = {\bibconf[4th]{HotCloud}{USENIX Workshop on Hot Topics in Cloud Computing}},
	Location = {Boston, MA},
	Month = jun,
	Title = {Icebergs in the Clouds: the {\em Other} Risks of Cloud Computing},
	Year = 2012}

@misc{mullenize,
	Author = {Washington Post},
	Howpublished = {\url{http://apps.washingtonpost.com/g/page/world/gchq-report-on-mullenize-program-to-stain-anonymous-electronic-traffic/502/}},
	Month = {oct},
	Title = {{GCHQ} report on {`MULLENIZE'} program to `stain' anonymous electronic traffic},
	Year = {2013}}

@inproceedings{shue13street,
	Author = {Craig A. Shue and Nathanael Paul and Curtis R. Taylor},
	Booktitle = {\bibbrev[7th]{WOOT}{USENIX Workshop on Offensive Technologies}},
	Month = aug,
	Title = {From an {IP} Address to a Street Address: Using Wireless Signals to Locate a Target},
	Year = 2013}

@inproceedings{knockel11three,
	Author = {Jeffrey Knockel and Jedidiah R. Crandall and Jared Saia},
	Booktitle = {\bibbrev{FOCI}{USENIX Workshop on Free and Open Communications on the Internet}},
	Location = {San Francisco, CA},
	Month = aug,
	Year = 2011}

@misc{rfc4960,
	Author = {R. {Stewart, ed.}},
	Month = sep,
	Note = {RFC 4960},
	Title = {Stream Control Transmission Protocol},
	Year = 2007}

@inproceedings{ford07structured,
	Author = {Bryan Ford},
	Booktitle = {\bibbrev{SIGCOMM}{ACM SIGCOMM}},
	Location = {Kyoto, Japan},
	Month = aug,
	Title = {Structured Streams: a New Transport Abstraction},
	Year = {2007}}

@misc{spdy,
	Author = {Google, Inc.},
	Note = {\url{http://www.chromium.org/spdy/spdy-whitepaper}},
	Title = {{SPDY}: An Experimental Protocol For a Faster {Web}}}

@misc{quic,
	Author = {Jim Roskind},
	Month = jun,
	Note = {\url{http://blog.chromium.org/2013/06/experimenting-with-quic.html}},
	Title = {Experimenting with {QUIC}},
	Year = 2013}

@misc{podjarny12not,
	Author = {G.~Podjarny},
	Month = jun,
	Note = {\url{http://www.guypo.com/technical/not-as-spdy-as-you-thought/}},
	Title = {{Not as SPDY as You Thought}},
	Year = 2012}

@inproceedings{cor,
	Author = {Jones, N.~A. and Arye, M. and Cesareo, J. and Freedman, M.~J.},
	Booktitle = {FOCI},
	Title = {{Hiding Amongst the Clouds: A Proposal for Cloud-based Onion Routing}},
	Year = {2011}}

@misc{torcloud,
	Howpublished = {\url{https://cloud.torproject.org/}},
	Key = {tor cloud},
	Title = {{The Tor Cloud Project}}}

@inproceedings{scramblesuit,
	Author = {Philipp Winter and Tobias Pulls and Juergen Fuss},
	Booktitle = {WPES},
	Title = {{ScrambleSuit: A Polymorphic Network Protocol to Circumvent Censorship}},
	Year = 2013}

@article{savage2000practical,
	Author = {Savage, S. and Wetherall, D. and Karlin, A. and Anderson, T.},
	Journal = {ACM SIGCOMM Computer Communication Review},
	Number = {4},
	Pages = {295--306},
	Publisher = {ACM},
	Title = {Practical network support for IP traceback},
	Volume = {30},
	Year = {2000}}

@inproceedings{ooni,
	Author = {Filast, A. and Appelbaum, J.},
	Booktitle = {{FOCI}},
	Title = {{OONI : Open Observatory of Network Interference}},
	Year = {2012}}

@misc{caida-rank,
	Howpublished = {\url{http://as-rank.caida.org/}},
	Key = {caida rank},
	Title = {{AS Rank: AS Ranking}}}

@inproceedings{usersrouted-ccs13,
	Author = {A.~Johnson and C.~Wacek and R.~Jansen and M.~Sherr and P.~Syverson},
	Booktitle = {CCS},
	Title = {{Users Get Routed: Traffic Correlation on Tor by Realistic Adversaries}},
	Year = {2013}}

@inproceedings{edman2009awareness,
	Author = {Edman, M. and Syverson, P.},
	Booktitle = {{CCS}},
	Title = {{AS-awareness in Tor path selection}},
	Year = {2009}}

@inproceedings{DecoyCosts,
	Author = {A.~Houmansadr and E.~L.~Wong and V.~Shmatikov},
	Booktitle = {NDSS},
	Title = {{No Direction Home: The True Cost of Routing Around Decoys}},
	Year = {2014}}

@article{cordon,
	Author = {Elahi, T. and Goldberg, I.},
	Journal = {University of Waterloo CACR},
	Title = {{CORDON--A Taxonomy of Internet Censorship Resistance Strategies}},
	Volume = {33},
	Year = {2012}}

@inproceedings{privex,
	Author = {T.~Elahi and G.~Danezis and I.~Goldberg	},
	Booktitle = {{CCS}},
	Title = {{AS-awareness in Tor path selection}},
	Year = {2014}}

@inproceedings{changeGuards,
	Author = {T.~Elahi and K.~Bauer and M.~AlSabah and R.~Dingledine and I.~Goldberg},
	Booktitle = {{WPES}},
	Title = {{ Changing of the Guards: Framework for Understanding and Improving Entry Guard Selection in Tor}},
	Year = {2012}}

@article{RAINBOW:Journal,
	Author = {A.~Houmansadr and N.~Kiyavash and N.~Borisov},
	Journal = {IEEE/ACM Transactions on Networking},
	Title = {{Non-Blind Watermarking of Network Flows}},
	Year = 2014}

@inproceedings{info-tod,
	Author = {A.~Houmansadr and S.~Gorantla and T.~Coleman and N.~Kiyavash and and N.~Borisov},
	Booktitle = {{CCS (poster session)}},
	Title = {{On the Channel Capacity of Network Flow Watermarking}},
	Year = {2009}}

@inproceedings{johnson2014game,
	Author = {Johnson, B. and Laszka, A. and Grossklags, J. and Vasek, M. and Moore, T.},
	Booktitle = {Workshop on Bitcoin Research},
	Title = {{Game-theoretic Analysis of DDoS Attacks Against Bitcoin Mining Pools}},
	Year = {2014}}

@incollection{laszka2013mitigation,
	Author = {Laszka, A. and Johnson, B. and Grossklags, J.},
	Booktitle = {Decision and Game Theory for Security},
	Pages = {175--191},
	Publisher = {Springer},
	Title = {{Mitigation of Targeted and Non-targeted Covert Attacks as a Timing Game}},
	Year = {2013}}

@inproceedings{schottle2013game,
	Author = {Schottle, P. and Laszka, A. and Johnson, B. and Grossklags, J. and Bohme, R.},
	Booktitle = {EUSIPCO},
	Title = {{A Game-theoretic Analysis of Content-adaptive Steganography with Independent Embedding}},
	Year = {2013}}

@inproceedings{CloudTransport,
	Author = {C.~Brubaker and A.~Houmansadr and V.~Shmatikov},
	Booktitle = {PETS},
	Title = {{CloudTransport: Using Cloud Storage for Censorship-Resistant Networking}},
	Year = {2014}}

@inproceedings{sweet,
	Author = {W.~Zhou and A.~Houmansadr and M.~Caesar and N.~Borisov},
	Booktitle = {HotPETs},
	Title = {{SWEET: Serving the Web by Exploiting Email Tunnels}},
	Year = {2013}}

@inproceedings{ahsan2002practical,
	Author = {Ahsan, K. and Kundur, D.},
	Booktitle = {Workshop on Multimedia Security},
	Title = {{Practical data hiding in TCP/IP}},
	Year = {2002}}

@incollection{danezis2011covert,
	Author = {Danezis, G.},
	Booktitle = {Security Protocols XVI},
	Pages = {198--214},
	Publisher = {Springer},
	Title = {{Covert Communications Despite Traffic Data Retention}},
	Year = {2011}}

@inproceedings{liu2009hide,
	Author = {Liu, Y. and Ghosal, D. and Armknecht, F. and Sadeghi, A.-R. and Schulz, S. and Katzenbeisser, S.},
	Booktitle = {ESORICS},
	Title = {{Hide and Seek in Time---Robust Covert Timing Channels}},
	Year = {2009}}

@misc{image-watermark-fing,
	Author = {Jonathan Bailey},
	Howpublished = {\url{https://www.plagiarismtoday.com/2009/12/02/image-detection-watermarking-vs-fingerprinting/}},
	Title = {{Image Detection: Watermarking vs. Fingerprinting}},
	Year = {2009}}

@inproceedings{Servetto98,
	Author = {S. D. Servetto and C. I. Podilchuk and K. Ramchandran},
	Booktitle = {Int. Conf. Image Processing},
	Title = {Capacity issues in digital image watermarking},
	Year = {1998}}

@inproceedings{Chen01,
	Author = {B. Chen and G.W.Wornell},
	Booktitle = {IEEE Trans. Inform. Theory},
	Pages = {1423--1443},
	Title = {Quantization index modulation: A class of provably good methods for digital watermarking and information embedding},
	Year = {2001}}

@inproceedings{Karakos00,
	Author = {D. Karakos and A. Papamarcou},
	Booktitle = {IEEE Int. Symp. Information Theory},
	Pages = {47},
	Title = {Relationship between quantization and distribution rates of digitally watermarked data},
	Year = {2000}}

@inproceedings{Sullivan98,
	Author = {J. A. OSullivan and P. Moulin and J. M. Ettinger},
	Booktitle = {IEEE Int. Symp. Information Theory},
	Pages = {297},
	Title = {Information theoretic analysis of steganography},
	Year = {1998}}

@inproceedings{Merhav00,
	Author = {N. Merhav},
	Booktitle = {IEEE Trans. Inform. Theory},
	Pages = {420--430},
	Title = {On random coding error exponents of watermarking systems},
	Year = {2000}}

@inproceedings{Somekh01,
	Author = {A. Somekh-Baruch and N. Merhav},
	Booktitle = {IEEE Int. Symp. Information Theory},
	Pages = {7},
	Title = {On the error exponent and capacity games of private watermarking systems},
	Year = {2001}}

@inproceedings{Steinberg01,
	Author = {Y. Steinberg and N. Merhav},
	Booktitle = {IEEE Trans. Inform. Theory},
	Pages = {1410--1422},
	Title = {Identification in the presence of side information with application to watermarking},
	Year = {2001}}

@article{Moulin03,
	Author = {P. Moulin and J.A. O'Sullivan},
	Journal = {IEEE Trans. Info. Theory},
	Number = {3},
	Title = {Information-theoretic analysis of information hiding},
	Volume = 49,
	Year = 2003}

@article{Gelfand80,
	Author = {S.I.~Gelfand and M.S.~Pinsker},
	Journal = {Problems of Control and Information Theory},
	Number = {1},
	Pages = {19-31},
	Title = {{Coding for channel with random parameters}},
	Url = {citeseer.ist.psu.edu/anantharam96bits.html},
	Volume = {9},
	Year = {1980},
	Bdsk-Url-1 = {citeseer.ist.psu.edu/anantharam96bits.html}}

@book{Wolfowitz78,
	Author = {J. Wolfowitz},
	Edition = {3rd},
	Location = {New York},
	Publisher = {Springer-Verlag},
	Title = {Coding Theorems of Information Theory},
	Year = 1978}

@article{caire99,
	Author = {G. Caire and S. Shamai},
	Journal = {IEEE Transactions on Information Theory},
	Number = {6},
	Pages = {2007--2019},
	Title = {On the Capacity of Some Channels with Channel State Information},
	Volume = {45},
	Year = {1999}}

@inproceedings{wright2007language,
	Author = {Wright, Charles V and Ballard, Lucas and Monrose, Fabian and Masson, Gerald M},
	Booktitle = {USENIX Security},
	Title = {{Language identification of encrypted VoIP traffic: Alejandra y Roberto or Alice and Bob?}},
	Year = {2007}}

@inproceedings{backes2010speaker,
	Author = {Backes, Michael and Doychev, Goran and D{\"u}rmuth, Markus and K{\"o}pf, Boris},
	Booktitle = {{European Symposium on Research in Computer Security (ESORICS)}},
	Pages = {508--523},
	Publisher = {Springer},
	Title = {{Speaker Recognition in Encrypted Voice Streams}},
	Year = {2010}}

@phdthesis{lu2009traffic,
	Author = {Lu, Yuanchao},
	School = {Cleveland State University},
	Title = {{On Traffic Analysis Attacks to Encrypted VoIP Calls}},
	Year = {2009}}

@inproceedings{wright2008spot,
	Author = {Wright, Charles V and Ballard, Lucas and Coull, Scott E and Monrose, Fabian and Masson, Gerald M},
	Booktitle = {IEEE Symposium on Security and Privacy},
	Pages = {35--49},
	Title = {Spot me if you can: Uncovering spoken phrases in encrypted VoIP conversations},
	Year = {2008}}

@inproceedings{white2011phonotactic,
	Author = {White, Andrew M and Matthews, Austin R and Snow, Kevin Z and Monrose, Fabian},
	Booktitle = {IEEE Symposium on Security and Privacy},
	Pages = {3--18},
	Title = {Phonotactic reconstruction of encrypted VoIP conversations: Hookt on fon-iks},
	Year = {2011}}

@inproceedings{fancy,
	Author = {Houmansadr, Amir and Borisov, Nikita},
	Booktitle = {Privacy Enhancing Technologies},
	Organization = {Springer},
	Pages = {205--224},
	Title = {The Need for Flow Fingerprints to Link Correlated Network Flows},
	Year = {2013}}

@article{botmosaic,
	Author = {Amir Houmansadr and Nikita Borisov},
	Doi = {10.1016/j.jss.2012.11.005},
	Issn = {0164-1212},
	Journal = {Journal of Systems and Software},
	Keywords = {Network security},
	Number = {3},
	Pages = {707 - 715},
	Title = {BotMosaic: Collaborative network watermark for the detection of IRC-based botnets},
	Url = {http://www.sciencedirect.com/science/article/pii/S0164121212003068},
	Volume = {86},
	Year = {2013},
	Bdsk-Url-1 = {http://www.sciencedirect.com/science/article/pii/S0164121212003068},
	Bdsk-Url-2 = {http://dx.doi.org/10.1016/j.jss.2012.11.005}}

@inproceedings{ramsbrock2008first,
	Author = {Ramsbrock, Daniel and Wang, Xinyuan and Jiang, Xuxian},
	Booktitle = {Recent Advances in Intrusion Detection},
	Organization = {Springer},
	Pages = {59--77},
	Title = {A first step towards live botmaster traceback},
	Year = {2008}}

@inproceedings{potdar2005survey,
	Author = {Potdar, Vidyasagar M and Han, Song and Chang, Elizabeth},
	Booktitle = {Industrial Informatics, 2005. INDIN'05. 2005 3rd IEEE International Conference on},
	Organization = {IEEE},
	Pages = {709--716},
	Title = {A survey of digital image watermarking techniques},
	Year = {2005}}

@book{cole2003hiding,
	Author = {Cole, Eric and Krutz, Ronald D},
	Publisher = {John Wiley \& Sons, Inc.},
	Title = {Hiding in plain sight: Steganography and the art of covert communication},
	Year = {2003}}

@incollection{akaike1998information,
	Author = {Akaike, Hirotogu},
	Booktitle = {Selected Papers of Hirotugu Akaike},
	Pages = {199--213},
	Publisher = {Springer},
	Title = {Information theory and an extension of the maximum likelihood principle},
	Year = {1998}}

@misc{central-command-hack,
	Author = {Everett Rosenfeld},
	Howpublished = {\url{http://www.cnbc.com/id/102330338}},
	Title = {{FBI investigating Central Command Twitter hack}},
	Year = {2015}}

@misc{sony-psp-ddos,
	Howpublished = {\url{http://n4g.com/news/1644853/sony-and-microsoft-cant-do-much-ddos-attacks-explained}},
	Key = {sony},
	Month = {December},
	Title = {{Sony and Microsoft cant do much -- DDoS attacks explained}},
	Year = {2014}}

@misc{sony-hack,
	Author = {David Bloom},
	Howpublished = {\url{http://goo.gl/MwR4A7}},
	Title = {{Online Game Networks Hacked, Sony Unit President Threatened}},
	Year = {2014}}

@misc{home-depot,
	Author = {Dune Lawrence},
	Howpublished = {\url{http://www.businessweek.com/articles/2014-09-02/home-depots-credit-card-breach-looks-just-like-the-target-hack}},
	Title = {{Home Depot's Suspected Breach Looks Just Like the Target Hack}},
	Year = {2014}}

@misc{target,
	Author = {Julio Ojeda-Zapata},
	Howpublished = {\url{http://www.mercurynews.com/business/ci_24765398/how-did-hackers-pull-off-target-data-heist}},
	Title = {{Target hack: How did they do it?}},
	Year = {2014}}


@article{probabilitycourse,
	Author = {H. Pishro-Nik},
	note = {\url{http://www.probabilitycourse.com}},
	Title = {Introduction to probability, statistics, and random processes},
    Year = {2014}}

@article{our-isit-location,
	Author = {Z. Montazeri and A. Houmansadr and H. Pishro-Nik},
	Journal = {To be submitted to IEEE ISIT},
	Title = {Location Privacy for Multi-State Networks},
	Year = {2016}}






@article{kafsi2013entropy,
	Author = {Kafsi, Mohamed and Grossglauser, Matthias and Thiran, Patrick},
	Journal = {Information Theory, IEEE Transactions on},
	Number = {9},
	Pages = {5577--5583},
	Publisher = {IEEE},
	Title = {The entropy of conditional Markov trajectories},
	Volume = {59},
	Year = {2013}}

@inproceedings{gruteser2003anonymous,
	Author = {Gruteser, Marco and Grunwald, Dirk},
	Booktitle = {Proceedings of the 1st international conference on Mobile systems, applications and services},
	Organization = {ACM},
	Pages = {31--42},
	Title = {Anonymous usage of location-based services through spatial and temporal cloaking},
	Year = {2003}}

@inproceedings{husted2010mobile,
	Author = {Husted, Nathaniel and Myers, Steven},
	Booktitle = {Proceedings of the 17th ACM conference on Computer and communications security},
	Organization = {ACM},
	Pages = {85--96},
	Title = {Mobile location tracking in metro areas: malnets and others},
	Year = {2010}}

@inproceedings{li2009tradeoff,
	Author = {Li, Tiancheng and Li, Ninghui},
	Booktitle = {Proceedings of the 15th ACM SIGKDD international conference on Knowledge discovery and data mining},
	Organization = {ACM},
	Pages = {517--526},
	Title = {On the tradeoff between privacy and utility in data publishing},
	Year = {2009}}





@incollection{humbert2010tracking,
	Author = {Humbert, Mathias and Manshaei, Mohammad Hossein and Freudiger, Julien and Hubaux, Jean-Pierre},
	Booktitle = {Decision and Game Theory for Security},
	Pages = {38--57},
	Publisher = {Springer},
	Title = {Tracking games in mobile networks},
	Year = {2010}}


@article{palamidessi2006probabilistic,
	Author = {Palamidessi, Catuscia},
	Journal = {Electronic Notes in Theoretical Computer Science},
	Pages = {33--42},
	Publisher = {Elsevier},
	Title = {Probabilistic and nondeterministic aspects of anonymity},
	Volume = {155},
	Year = {2006}}



@inproceedings{freudiger2007mix,
	Author = {Freudiger, Julien and Raya, Maxim and F{\'e}legyh{\'a}zi, M{\'a}rk and Papadimitratos, Panos and Hubaux, Jean-Pierre},
	Booktitle = {CM Workshop on Wireless Networking for Intelligent Transportation Systems (WiN-ITS)},
	Title = {Mix-zones for location privacy in vehicular networks},
	Year = {2007}}



@inproceedings{niu2014achieving,
	Author = {Niu, Ben and Li, Qinghua and Zhu, Xiaoyan and Cao, Guohong and Li, Hui},
	Booktitle = {INFOCOM, 2014 Proceedings IEEE},
	Organization = {IEEE},
	Pages = {754--762},
	Title = {Achieving k-anonymity in privacy-aware location-based services},
	Year = {2014}}



@inproceedings{kido2005protection,
	Author = {Kido, Hidetoshi and Yanagisawa, Yutaka and Satoh, Tetsuji},
	Booktitle = {Data Engineering Workshops, 2005. 21st International Conference on},
	Organization = {IEEE},
	Pages = {1248--1248},
	Title = {Protection of location privacy using dummies for location-based services},
	Year = {2005}}

@inproceedings{gedik2005location,
	Author = {Gedik, Bu{\u{g}}ra and Liu, Ling},
	Booktitle = {Distributed Computing Systems, 2005. ICDCS 2005. Proceedings. 25th IEEE International Conference on},
	Organization = {IEEE},
	Pages = {620--629},
	Title = {Location privacy in mobile systems: A personalized anonymization model},
	Year = {2005}}


@incollection{duckham2005formal,
	Author = {Duckham, Matt and Kulik, Lars},
	Booktitle = {Pervasive computing},
	Pages = {152--170},
	Publisher = {Springer},
	Title = {A formal model of obfuscation and negotiation for location privacy},
	Year = {2005}}

@inproceedings{kido2005anonymous,
	Author = {Kido, Hidetoshi and Yanagisawa, Yutaka and Satoh, Tetsuji},
	Booktitle = {Pervasive Services, 2005. ICPS'05. Proceedings. International Conference on},
	Organization = {IEEE},
	Pages = {88--97},
	Title = {An anonymous communication technique using dummies for location-based services},
	Year = {2005}}

@incollection{duckham2006spatiotemporal,
	Author = {Duckham, Matt and Kulik, Lars and Birtley, Athol},
	Booktitle = {Geographic Information Science},
	Pages = {47--64},
	Publisher = {Springer},
	Title = {A spatiotemporal model of strategies and counter strategies for location privacy protection},
	Year = {2006}}

@inproceedings{shankar2009privately,
	Author = {Shankar, Pravin and Ganapathy, Vinod and Iftode, Liviu},
	Booktitle = {Proceedings of the 11th international conference on Ubiquitous computing},
	Organization = {ACM},
	Pages = {31--40},
	Title = {Privately querying location-based services with SybilQuery},
	Year = {2009}}

@inproceedings{chow2009faking,
	Author = {Chow, Richard and Golle, Philippe},
	Booktitle = {Proceedings of the 8th ACM workshop on Privacy in the electronic society},
	Organization = {ACM},
	Pages = {105--108},
	Title = {Faking contextual data for fun, profit, and privacy},
	Year = {2009}}

@incollection{xue2009location,
	Author = {Xue, Mingqiang and Kalnis, Panos and Pung, Hung Keng},
	Booktitle = {Location and Context Awareness},
	Pages = {70--87},
	Publisher = {Springer},
	Title = {Location diversity: Enhanced privacy protection in location based services},
	Year = {2009}}

@article{wernke2014classification,
	Author = {Wernke, Marius and Skvortsov, Pavel and D{\"u}rr, Frank and Rothermel, Kurt},
	Journal = {Personal and Ubiquitous Computing},
	Number = {1},
	Pages = {163--175},
	Publisher = {Springer-Verlag},
	Title = {A classification of location privacy attacks and approaches},
	Volume = {18},
	Year = {2014}}

@misc{cai2015cloaking,
	Author = {Cai, Y. and Xu, G.},
	Month = jan # {~1},
	Note = {US Patent App. 14/472,462},
	Publisher = {Google Patents},
	Title = {Cloaking with footprints to provide location privacy protection in location-based services},
	Url = {https://www.google.com/patents/US20150007341},
	Year = {2015},
	Bdsk-Url-1 = {https://www.google.com/patents/US20150007341}}

@article{gedik2008protecting,
	Author = {Gedik, Bu{\u{g}}ra and Liu, Ling},
	Journal = {Mobile Computing, IEEE Transactions on},
	Number = {1},
	Pages = {1--18},
	Publisher = {IEEE},
	Title = {Protecting location privacy with personalized k-anonymity: Architecture and algorithms},
	Volume = {7},
	Year = {2008}}

@article{kalnis2006preserving,
	Author = {Kalnis, Panos and Ghinita, Gabriel and Mouratidis, Kyriakos and Papadias, Dimitris},
	Publisher = {TRB6/06},
	Title = {Preserving anonymity in location based services},
	Year = {2006}}



@article{terrovitis2011privacy,
	Author = {Terrovitis, Manolis},
	Journal = {ACM SIGKDD Explorations Newsletter},
	Number = {1},
	Pages = {6--18},
	Publisher = {ACM},
	Title = {Privacy preservation in the dissemination of location data},
	Volume = {13},
	Year = {2011}}

@article{shin2012privacy,
	Author = {Shin, Kang G and Ju, Xiaoen and Chen, Zhigang and Hu, Xin},
	Journal = {Wireless Communications, IEEE},
	Number = {1},
	Pages = {30--39},
	Publisher = {IEEE},
	Title = {Privacy protection for users of location-based services},
	Volume = {19},
	Year = {2012}}



@incollection{chatzikokolakis2015geo,
	Author = {Chatzikokolakis, Konstantinos and Palamidessi, Catuscia and Stronati, Marco},
	Booktitle = {Distributed Computing and Internet Technology},
	Pages = {49--72},
	Publisher = {Springer},
	Title = {Geo-indistinguishability: A Principled Approach to Location Privacy},
	Year = {2015}}

@inproceedings{ngo2015location,
	Author = {Ngo, Hoa and Kim, Jong},
	Booktitle = {Computer Security Foundations Symposium (CSF), 2015 IEEE 28th},
	Organization = {IEEE},
	Pages = {63--74},
	Title = {Location Privacy via Differential Private Perturbation of Cloaking Area},
	Year = {2015}}


@inproceedings{um2010advanced,
	Author = {Um, Jung-Ho and Kim, Hee-Dae and Chang, Jae-Woo},
	Booktitle = {Social Computing (SocialCom), 2010 IEEE Second International Conference on},
	Organization = {IEEE},
	Pages = {1093--1098},
	Title = {An advanced cloaking algorithm using Hilbert curves for anonymous location based service},
	Year = {2010}}

@inproceedings{bamba2008supporting,
	Author = {Bamba, Bhuvan and Liu, Ling and Pesti, Peter and Wang, Ting},
	Booktitle = {Proceedings of the 17th international conference on World Wide Web},
	Organization = {ACM},
	Pages = {237--246},
	Title = {Supporting anonymous location queries in mobile environments with privacygrid},
	Year = {2008}}

@inproceedings{zhangwei2010distributed,
	Author = {Zhangwei, Huang and Mingjun, Xin},
	Booktitle = {Networks Security Wireless Communications and Trusted Computing (NSWCTC), 2010 Second International Conference on},
	Organization = {IEEE},
	Pages = {468--471},
	Title = {A distributed spatial cloaking protocol for location privacy},
	Volume = {2},
	Year = {2010}}

@article{chow2011spatial,
	Author = {Chow, Chi-Yin and Mokbel, Mohamed F and Liu, Xuan},
	Journal = {GeoInformatica},
	Number = {2},
	Pages = {351--380},
	Publisher = {Springer},
	Title = {Spatial cloaking for anonymous location-based services in mobile peer-to-peer environments},
	Volume = {15},
	Year = {2011}}

@inproceedings{lu2008pad,
	Author = {Lu, Hua and Jensen, Christian S and Yiu, Man Lung},
	Booktitle = {Proceedings of the Seventh ACM International Workshop on Data Engineering for Wireless and Mobile Access},
	Organization = {ACM},
	Pages = {16--23},
	Title = {Pad: privacy-area aware, dummy-based location privacy in mobile services},
	Year = {2008}}

@incollection{khoshgozaran2007blind,
	Author = {Khoshgozaran, Ali and Shahabi, Cyrus},
	Booktitle = {Advances in Spatial and Temporal Databases},
	Pages = {239--257},
	Publisher = {Springer},
	Title = {Blind evaluation of nearest neighbor queries using space transformation to preserve location privacy},
	Year = {2007}}

@inproceedings{ghinita2008private,
	Author = {Ghinita, Gabriel and Kalnis, Panos and Khoshgozaran, Ali and Shahabi, Cyrus and Tan, Kian-Lee},
	Booktitle = {Proceedings of the 2008 ACM SIGMOD international conference on Management of data},
	Organization = {ACM},
	Pages = {121--132},
	Title = {Private queries in location based services: anonymizers are not necessary},
	Year = {2008}}



@article{nguyen2013differential,
	Author = {Nguyen, Hiep H and Kim, Jong and Kim, Yoonho},
	Journal = {Journal of Computing Science and Engineering},
	Number = {3},
	Pages = {177--186},
	Title = {Differential privacy in practice},
	Volume = {7},
	Year = {2013}}

@inproceedings{lee2012differential,
	Author = {Lee, Jaewoo and Clifton, Chris},
	Booktitle = {Proceedings of the 18th ACM SIGKDD international conference on Knowledge discovery and data mining},
	Organization = {ACM},
	Pages = {1041--1049},
	Title = {Differential identifiability},
	Year = {2012}}

@inproceedings{andres2013geo,
	Author = {Andr{\'e}s, Miguel E and Bordenabe, Nicol{\'a}s E and Chatzikokolakis, Konstantinos and Palamidessi, Catuscia},
	Booktitle = {Proceedings of the 2013 ACM SIGSAC conference on Computer \& communications security},
	Organization = {ACM},
	Pages = {901--914},
	Title = {Geo-indistinguishability: Differential privacy for location-based systems},
	Year = {2013}}

@inproceedings{machanavajjhala2008privacy,
	Author = {Machanavajjhala, Ashwin and Kifer, Daniel and Abowd, John and Gehrke, Johannes and Vilhuber, Lars},
	Booktitle = {Data Engineering, 2008. ICDE 2008. IEEE 24th International Conference on},
	Organization = {IEEE},
	Pages = {277--286},
	Title = {Privacy: Theory meets practice on the map},
	Year = {2008}}

@article{dewri2013local,
	Author = {Dewri, Rinku},
	Journal = {Mobile Computing, IEEE Transactions on},
	Number = {12},
	Pages = {2360--2372},
	Publisher = {IEEE},
	Title = {Local differential perturbations: Location privacy under approximate knowledge attackers},
	Volume = {12},
	Year = {2013}}

@inproceedings{chatzikokolakis2013broadening,
	Author = {Chatzikokolakis, Konstantinos and Andr{\'e}s, Miguel E and Bordenabe, Nicol{\'a}s Emilio and Palamidessi, Catuscia},
	Booktitle = {Privacy Enhancing Technologies},
	Organization = {Springer},
	Pages = {82--102},
	Title = {Broadening the Scope of Differential Privacy Using Metrics.},
	Year = {2013}}



@inproceedings{cheng2006preserving,
	Author = {Cheng, Reynold and Zhang, Yu and Bertino, Elisa and Prabhakar, Sunil},
	Booktitle = {Privacy Enhancing Technologies},
	Organization = {Springer},
	Pages = {393--412},
	Title = {Preserving user location privacy in mobile data management infrastructures},
	Year = {2006}}




@article{krumm2009survey,
	Author = {Krumm, John},
	Journal = {Personal and Ubiquitous Computing},
	Number = {6},
	Pages = {391--399},
	Publisher = {Springer},
	Title = {A survey of computational location privacy},
	Volume = {13},
	Year = {2009}}

@article{Rakhshan2015letter,
	Author = {Rakhshan, Ali and Pishro-Nik, Hossein},
	Journal = {IEEE Wireless Communications Letter},
	Publisher = {IEEE},
	Title = {A Stochastic Geometry Model for Customized Vehicular Communication},
	Year = {2015, submitted}}

@article{Rakhshan2015Journal,
	Author = {Rakhshan, Ali and Pishro-Nik, Hossein},
	Journal = {IEEE Transactions on Wireless Communications},
	Publisher = {IEEE},
	Title = {Improving Safety on Highways by Customizing Vehicular Ad Hoc Networks},
	Year = {2015, submitted}}

@inproceedings{Rakhshan2015Cogsima,
	Author = {Rakhshan, Ali and Pishro-Nik, Hossein},
	Booktitle = {IEEE International Multi-Disciplinary Conference on Cognitive Methods in Situation Awareness and Decision Support},
	Organization = {IEEE},
	Title = {A New Approach to Customization of Accident Warning Systems to Individual Drivers},
	Year = {2015}}

@inproceedings{Rakhshan2015CISS,
	Author = {Rakhshan, Ali and Pishro-Nik, Hossein and Nekoui, Mohammad},
	Booktitle = {Conference on Information Sciences and Systems},
	Organization = {IEEE},
	Pages = {1--6},
	Title = {Driver-based adaptation of Vehicular Ad Hoc Networks for design of active safety systems},
	Year = {2015}}

@inproceedings{Rakhshan2014IV,
	Author = {Rakhshan, Ali and Pishro-Nik, Hossein and Ray, Evan},
	Booktitle = {Intelligent Vehicles Symposium},
	Organization = {IEEE},
	Pages = {1181--1186},
	Title = {Real-time estimation of the distribution of brake response times for an individual driver using Vehicular Ad Hoc Network.},
	Year = {2014}}

@inproceedings{Rakhshan2013Globecom,
	Author = {Rakhshan, Ali and Pishro-Nik, Hossein and Fisher, Donald and Nekoui, Mohammad},
	Booktitle = {IEEE Global Communications Conference},
	Organization = {IEEE},
	Pages = {1333--1337},
	Title = {Tuning collision warning algorithms to individual drivers for design of active safety systems.},
	Year = {2013}}

@article{Nekoui2012Journal,
	Author = {Nekoui, Mohammad and Pishro-Nik, Hossein},
	Journal = {IEEE Transactions on Wireless Communications},
	Number = {8},
	Pages = {2895--2905},
	Publisher = {IEEE},
	Title = {Throughput Scaling laws for Vehicular Ad Hoc Networks},
	Volume = {11},
	Year = {2012}}

@article{Nekoui2013Journal,
	Author = {Nekoui, Mohammad and Pishro-Nik, Hossein},
	Journal = {Journal on Selected Areas in Communications, Special Issue on Emerging Technologies in Communications},
	Number = {9},
	Pages = {491--503},
	Publisher = {IEEE},
	Title = {Analytic Design of Active Safety Systems for Vehicular Ad hoc Networks},
	Volume = {31},
	Year = {2013}}

@article{Nekoui2011Journal,
	Author = {Nekoui, Mohammad and Pishro-Nik, Hossein and Ni, Daiheng},
	Journal = {International Journal of Vehicular Technology},
	Pages = {1--11},
	Publisher = {Hindawi Publishing Corporation},
	Title = {Analytic Design of Active Safety Systems for Vehicular Ad hoc Networks},
	Volume = {2011},
	Year = {2011}}

@inproceedings{Nekoui2011MOBICOM,
	Author = {Nekoui, Mohammad and Pishro-Nik, Hossein},
	Booktitle = {MOBICOM workshop on VehiculAr InterNETworking},
	Organization = {ACM},
	Title = {Analytic Design of Active Vehicular Safety Systems in Sparse Traffic},
	Year = {2011}}

@inproceedings{Nekoui2011VTC,
	Author = {Nekoui, Mohammad and Pishro-Nik, Hossein},
	Booktitle = {VTC-Fall},
	Organization = {IEEE},
	Title = {Analytical Design of Inter-vehicular Communications for Collision Avoidance},
	Year = {2011}}

@inproceedings{Bovee2011VTC,
	Author = {Bovee, Ben Louis and Nekoui, Mohammad and Pishro-Nik, Hossein},
	Booktitle = {VTC-Fall},
	Organization = {IEEE},
	Title = {Evaluation of the Universal Geocast Scheme For VANETs},
	Year = {2011}}

@inproceedings{Nekoui2010MOBICOM,
	Author = {Nekoui, Mohammad and Pishro-Nik, Hossein},
	Booktitle = {MOBICOM},
	Organization = {ACM},
	Title = {Fundamental Tradeoffs in Vehicular Ad Hoc Networks},
	Year = {2010}}

@inproceedings{Nekoui2010IVCS,
	Author = {Nekoui, Mohammad and Pishro-Nik, Hossein},
	Booktitle = {IVCS},
	Organization = {IEEE},
	Title = {A Universal Geocast Scheme for Vehicular Ad Hoc Networks},
	Year = {2010}}

@inproceedings{Nekoui2009ITW,
	Author = {Nekoui, Mohammad and Pishro-Nik, Hossein},
	Booktitle = {IEEE Communications Society Conference on Sensor, Mesh and Ad Hoc Communications and Networks Workshops},
	Organization = {IEEE},
	Pages = {1--3},
	Title = {A Geometrical Analysis of Obstructed Wireless Networks},
	Year = {2009}}

@article{Eslami2013Journal,
	Author = {Eslami, Ali and Nekoui, Mohammad and Pishro-Nik, Hossein and Fekri, Faramarz},
	Journal = {ACM Transactions on Sensor Networks},
	Number = {4},
	Pages = {51},
	Publisher = {ACM},
	Title = {Results on finite wireless sensor networks: Connectivity and coverage},
	Volume = {9},
	Year = {2013}}

@article{shokri2014optimal,
	  title={Optimal user-centric data obfuscation},
 	 author={Shokri, Reza},
 	 journal={arXiv preprint arXiv:1402.3426},
 	 year={2014}
	}
@article{chatzikokolakis2015location,
  title={Location privacy via geo-indistinguishability},
  author={Chatzikokolakis, Konstantinos and Palamidessi, Catuscia and Stronati, Marco},
  journal={ACM SIGLOG News},
  volume={2},
  number={3},
  pages={46--69},
  year={2015},
  publisher={ACM}

}
@inproceedings{shokri2011quantifying2,
  title={Quantifying location privacy: the case of sporadic location exposure},
  author={Shokri, Reza and Theodorakopoulos, George and Danezis, George and Hubaux, Jean-Pierre and Le Boudec, Jean-Yves},
  booktitle={Privacy Enhancing Technologies},
  pages={57--76},
  year={2011},
  organization={Springer}
}

@inproceedings{calmon2015fundamental,
  title={Fundamental limits of perfect privacy},
  author={Calmon, Flavio P and Makhdoumi, Ali and M{\'e}dard, Muriel},
  booktitle={Information Theory (ISIT), 2015 IEEE International Symposium on},
  pages={1796--1800},
  year={2015},
  organization={IEEE}
}

@inproceedings{salamatian2013hide,
  title={How to hide the elephant-or the donkey-in the room: Practical privacy against statistical inference for large data.},
  author={Salamatian, Salman and Zhang, Amy and du Pin Calmon, Flavio and Bhamidipati, Sandilya and Fawaz, Nadia and Kveton, Branislav and Oliveira, Pedro and Taft, Nina},
  booktitle={GlobalSIP},
  pages={269--272},
  year={2013}
}

@article{sankar2013utility,
  title={Utility-privacy tradeoffs in databases: An information-theoretic approach},
  author={Sankar, Lalitha and Rajagopalan, S Raj and Poor, H Vincent},
  journal={Information Forensics and Security, IEEE Transactions on},
  volume={8},
  number={6},
  pages={838--852},
  year={2013},
  publisher={IEEE}
}
@inproceedings{ghinita2007prive,
  title={PRIVE: anonymous location-based queries in distributed mobile systems},
  author={Ghinita, Gabriel and Kalnis, Panos and Skiadopoulos, Spiros},
  booktitle={Proceedings of the 16th international conference on World Wide Web},
  pages={371--380},
  year={2007},
  organization={ACM}
}

@article{beresford2004mix,
  title={Mix zones: User privacy in location-aware services},
  author={Beresford, Alastair R and Stajano, Frank},
  year={2004},
  publisher={IEEE}
}


@article{csiszar1996almost,
  title={Almost independence and secrecy capacity},
  author={Csisz{\'a}r, Imre},
  journal={Problemy Peredachi Informatsii},
  volume={32},
  number={1},
  pages={48--57},
  year={1996},
  publisher={Russian Academy of Sciences, Branch of Informatics, Computer Equipment and Automatization}
}

@article{yamamoto1983source,
  title={A source coding problem for sources with additional outputs to keep secret from the receiver or wiretappers (corresp.)},
  author={Yamamoto, Hirosuke},
  journal={IEEE Transactions on Information Theory},
  volume={29},
  number={6},
  pages={918--923},
  year={1983},
  publisher={IEEE}
}



  @inproceedings{golle2009anonymity,
  title={On the anonymity of home/work location pairs},
  author={Golle, Philippe and Partridge, Kurt},
  booktitle={International Conference on Pervasive Computing},
  pages={390--397},
  year={2009},
  organization={Springer}
}

@inproceedings{zang2011anonymization,
  title={Anonymization of location data does not work: A large-scale measurement study},
  author={Zang, Hui and Bolot, Jean},
  booktitle={Proceedings of the 17th annual international conference on Mobile computing and networking},
  pages={145--156},
  year={2011},
  organization={ACM}
}
@article{wang2015privacy,
  title={Privacy-preserving collaborative spectrum sensing with multiple service providers},
  author={Wang, Wei and Zhang, Qian},
  journal={IEEE Transactions on Wireless Communications},
  volume={14},
  number={2},
  pages={1011--1019},
  year={2015},
  publisher={IEEE}
}

@article{wang2015toward,
  title={Toward long-term quality of protection in mobile networks: a context-aware perspective},
  author={Wang, Wei and Zhang, Qian},
  journal={IEEE Wireless Communications},
  volume={22},
  number={4},
  pages={34--40},
  year={2015},
  publisher={IEEE}
}

@inproceedings{niu2015enhancing,
  title={Enhancing privacy through caching in location-based services},
  author={Niu, Ben and Li, Qinghua and Zhu, Xiaoyan and Cao, Guohong and Li, Hui},
  booktitle={2015 IEEE Conference on Computer Communications (INFOCOM)},
  pages={1017--1025},
  year={2015},
  organization={IEEE}
}

%% This BibTeX bibliography file was created using BibDesk.
%% http://bibdesk.sourceforge.net/

%% Created for Zarrin Montazeri at 2015-11-09 18:45:31 -0500


%% Saved with string encoding Unicode (UTF-8)



%%%%%%%%%%%%%%IOT%%%%%%%%%%%%%%%%%%%%%%%%%%%%%%%%%%%%%%%%%%%%%%%%%%%


@article{osma2015,
	title={Impact of Time-to-Collision Information on Driving Behavior in Connected Vehicle Environments Using A Driving Simulator Test Bed},
	journal{Journal of Traffic and Logistics Engineering},
	author={Osama A. Osman, Julius Codjoe, and Sherif Ishak},
	volume={3},
	number={1},
	pages={18--24},
	year={2015}
}


@article{charisma2010,
	title={Dynamic Latent Plan Models},
	author={Charisma F. Choudhurya, Moshe Ben-Akivab and Maya Abou-Zeid},
	journal={Journal of Choice Modelling},
	volume={3},
	number={2},
	pages={50--70},
	year={2010},
	publisher={Elsvier}
}


@misc{noble2014,
	author = {A. M. Noble, Shane B. McLaughlin, Zachary R. Doerzaph and Thomas A. Dingus},
	title = {Crowd-sourced Connected-vehicle Warning Algorithm using Naturalistic Driving Data},
	howpublished = {Downloaded from \url{http://hdl.handle.net/10919/53978}},
	
	month = August,
	year = 2014
}


@phdthesis{charisma2007,
	title    = {Modeling Driving Decisions with Latent Plans},
	school   = {Massachusetts Institute of Technology },
	author   = {Charisma Farheen Choudhury},
	year     = {2007}, %other attributes omitted
}


@article{chrysler2015,
	title={Cost of Warning of Unseen Threats:Unintended Consequences of Connected Vehicle Alerts},
	author={S. T. Chrysler, J. M. Cooper and D. C. Marshall},
	journal={Transportation Research Record: Journal of the Transportation Research Board},
	volume={2518},
	pages={79--85},
	year={2015},
}





@article{FTC2015,
	title={Internet of Things: Privacy and Security in a Connected World},
	author={FTC Staff Report},
	year={2015}
}



%% Saved with string encoding Unicode (UTF-8)
@inproceedings{1zhou2014security,
	title={Security/privacy of wearable fitness tracking {I}o{T} devices},
	author={Zhou, Wei and Piramuthu, Selwyn},
	booktitle={Information Systems and Technologies (CISTI), 2014 9th Iberian Conference on},
	pages={1--5},
	year={2014},
	organization={IEEE}
}


@inproceedings{3ukil2014iot,
	title={{I}o{T}-privacy: To be private or not to be private},
	author={Arijit Ukil and Soma Bandyopadhyay and Arpan Pal},
	booktitle={Computer Communications Workshops (INFOCOM WKSHPS), IEEE Conference on},
	pages={123--124},
	year={2014},
	organization={IEEE}
}

@inproceedings{4Hosseinzadeh2014,
	title={Security in the Internet of Things through obfuscation and diversification},
	author={Hosseinzadeh, Shohreh and Rauti, Sampsa and Hyrynsalmi, Sami and Leppanen, Ville},
	booktitle={Computing, Communication and Security (ICCCS), IEEE Conference on},
	pages={123--124},
	year={2015},
	organization={IEEE}
}
@article{4arias2015privacy,
	title={Privacy and security in internet of things and wearable devices},
	author={Arias, Orlando and Wurm, Jacob and Hoang, Khoa and Jin, Yier},
	journal={IEEE Transactions on Multi-Scale Computing Systems},
	volume={1},
	number={2},
	pages={99--109},
	year={2015},
	publisher={IEEE}
}
@inproceedings{5ullah2016novel,
	title={A novel model for preserving Location Privacy in Internet of Things},
	author={Ullah, Ikram and Shah, Munam Ali},
	booktitle={Automation and Computing (ICAC), 2016 22nd International Conference on},
	pages={542--547},
	year={2016},
	organization={IEEE}
}
@inproceedings{6sathishkumar2016enhanced,
	title={Enhanced location privacy algorithm for wireless sensor network in Internet of Things},
	author={Sathishkumar, J and Patel, Dhiren R},
	booktitle={Internet of Things and Applications (IOTA), International Conference on},
	pages={208--212},
	year={2016},
	organization={IEEE}
}
@inproceedings{7zhou2012preserving,
	title={Preserving sensor location privacy in internet of things},
	author={Zhou, Liming and Wen, Qiaoyan and Zhang, Hua},
	booktitle={Computational and Information Sciences (ICCIS), 2012 Fourth International Conference on},
	pages={856--859},
	year={2012},
	organization={IEEE}
}

@inproceedings{8ukil2015privacy,
	title={Privacy for {I}o{T}: Involuntary privacy enablement for smart energy systems},
	author={Ukil, Arijit and Bandyopadhyay, Soma and Pal, Arpan},
	booktitle={Communications (ICC), 2015 IEEE International Conference on},
	pages={536--541},
	year={2015},
	organization={IEEE}
}

@inproceedings{9dalipi2016security,
	title={Security and Privacy Considerations for {I}o{T} Application on Smart Grids: Survey and Research Challenges},
	author={Dalipi, Fisnik and Yayilgan, Sule Yildirim},
	booktitle={Future Internet of Things and Cloud Workshops (FiCloudW), IEEE International Conference on},
	pages={63--68},
	year={2016},
	organization={IEEE}
}
@inproceedings{10harris2016security,
	title={Security and Privacy in Public {I}o{T} Spaces},
	author={Harris, Albert F and Sundaram, Hari and Kravets, Robin},
	booktitle={Computer Communication and Networks (ICCCN), 2016 25th International Conference on},
	pages={1--8},
	year={2016},
	organization={IEEE}
}

@inproceedings{11al2015security,
	title={Security and privacy framework for ubiquitous healthcare {I}o{T} devices},
	author={Al Alkeem, Ebrahim and Yeun, Chan Yeob and Zemerly, M Jamal},
	booktitle={Internet Technology and Secured Transactions (ICITST), 2015 10th International Conference for},
	pages={70--75},
	year={2015},
	organization={IEEE}
}
@inproceedings{12sivaraman2015network,
	title={Network-level security and privacy control for smart-home {I}o{T} devices},
	author={Sivaraman, Vijay and Gharakheili, Hassan Habibi and Vishwanath, Arun and Boreli, Roksana and Mehani, Olivier},
	booktitle={Wireless and Mobile Computing, Networking and Communications (WiMob), 2015 IEEE 11th International Conference on},
	pages={163--167},
	year={2015},
	organization={IEEE}
}

@inproceedings{13srinivasan2016privacy,
	title={Privacy conscious architecture for improving emergency response in smart cities},
	author={Srinivasan, Ramya and Mohan, Apurva and Srinivasan, Priyanka},
	booktitle={Smart City Security and Privacy Workshop (SCSP-W), 2016},
	pages={1--5},
	year={2016},
	organization={IEEE}
}
@inproceedings{14sadeghi2015security,
	title={Security and privacy challenges in industrial internet of things},
	author={Sadeghi, Ahmad-Reza and Wachsmann, Christian and Waidner, Michael},
	booktitle={Design Automation Conference (DAC), 2015 52nd ACM/EDAC/IEEE},
	pages={1--6},
	year={2015},
	organization={IEEE}
}
@inproceedings{15otgonbayar2016toward,
	title={Toward Anonymizing {I}o{T} Data Streams via Partitioning},
	author={Otgonbayar, Ankhbayar and Pervez, Zeeshan and Dahal, Keshav},
	booktitle={Mobile Ad Hoc and Sensor Systems (MASS), 2016 IEEE 13th International Conference on},
	pages={331--336},
	year={2016},
	organization={IEEE}
}
@inproceedings{16rutledge2016privacy,
	title={Privacy Impacts of {I}o{T} Devices: A SmartTV Case Study},
	author={Rutledge, Richard L and Massey, Aaron K and Ant{\'o}n, Annie I},
	booktitle={Requirements Engineering Conference Workshops (REW), IEEE International},
	pages={261--270},
	year={2016},
	organization={IEEE}
}

@inproceedings{17andrea2015internet,
	title={Internet of Things: Security vulnerabilities and challenges},
	author={Andrea, Ioannis and Chrysostomou, Chrysostomos and Hadjichristofi, George},
	booktitle={Computers and Communication (ISCC), 2015 IEEE Symposium on},
	pages={180--187},
	year={2015},
	organization={IEEE}
}






























%%%%%%%%%%%%%%%%%%%%%%%%%%%%%%%%%%%%%%%%%%%%%%%%%%%%%%%%%%%


@misc{epfl-mobility-20090224,
	author = {Michal Piorkowski and Natasa Sarafijanovic-Djukic and Matthias Grossglauser},
	title = {{CRAWDAD} dataset epfl/mobility (v. 2009-02-24)},
	howpublished = {Downloaded from \url{http://crawdad.org/epfl/mobility/20090224}},
	doi = {10.15783/C7J010},
	month = feb,
	year = 2009
}

@misc{roma-taxi-20140717,
	author = {Lorenzo Bracciale and Marco Bonola and Pierpaolo Loreti and Giuseppe Bianchi and Raul Amici and Antonello Rabuffi},
	title = {{CRAWDAD} dataset roma/taxi (v. 2014-07-17)},
	howpublished = {Downloaded from \url{http://crawdad.org/roma/taxi/20140717}},
	doi = {10.15783/C7QC7M},
	month = jul,
	year = 2014
}

@misc{rice-ad_hoc_city-20030911,
	author = {Jorjeta G. Jetcheva and Yih-Chun Hu and Santashil PalChaudhuri and Amit Kumar Saha and David B. Johnson},
	title = {{CRAWDAD} dataset rice/ad\_hoc\_city (v. 2003-09-11)},
	howpublished = {Downloaded from \url{http://crawdad.org/rice/ad_hoc_city/20030911}},
	doi = {10.15783/C73K5B},
	month = sep,
	year = 2003
}

@misc{china:2012,
	author = {Microsoft Research Asia},
	title = {GeoLife GPS Trajectories},
	year = {2012},
	howpublished= {\url{https://www.microsoft.com/en-us/download/details.aspx?id=52367}},
}


@misc{china:2011,
	ALTauthor = {Microsoft Research Asia)},
	ALTeditor = {},
	title = {GeoLife GPS Trajectories,
	year = {2012},
	url = {https://www.microsoft.com/en-us/download/details.aspx?id=52367},
	}
	
	
	@misc{longversion,
	author = {N. Takbiri and A. Houmansadr and D.L. Goeckel and H. Pishro-Nik},
	title = {{Limits of Location Privacy under Anonymization and Obfuscation}},
	howpublished = "\url{https://dl.dropboxusercontent.com/u/49263048/ISIT_2017-2.pdf}",
	year = 2017,
	month= "January",
	note = "[Online; accessed 22-Jan-2017]"
	}
	
	
	
	@article{matching,
	title={Asymptotically Optimal Matching of Multiple Sequences to Source Distributions and Training Sequences},
	author={Jayakrishnan Unnikrishnan},
	journal={ IEEE Transactions on Information Theory},
	volume={61},
	number={1},
	pages={452-468},
	year={2015},
	publisher={IEEE}
	}
	
	
	@article{Naini2016,
	Author = {F. Naini and J. Unnikrishnan and P. Thiran and M. Vetterli},
	Journal = {IEEE Transactions on Information Forensics and Security},
	Publisher = {IEEE},
	Title = {Where You Are Is Who You Are: User Identification by Matching Statistics},
	volume={11},
	number={2},
	pages={358--372},
	Year = {2016}
	}
	
	
	
	@inproceedings{holowczak2015cachebrowser,
	title={{CacheBrowser: Bypassing Chinese Censorship without Proxies Using Cached Content}},
	author={Holowczak, John and Houmansadr, Amir},
	booktitle={Proceedings of the 22nd ACM SIGSAC Conference on Computer and Communications Security},
	pages={70--83},
	year={2015},
	organization={ACM}
	}
	@misc{cb-website,
	Howpublished = {\url{https://cachebrowser.net/#/}},
	Title = {{CacheBrowser}},
	key={cachebrowser}
	}
	
	@inproceedings{GameOfDecoys,
	title={{GAME OF DECOYS: Optimal Decoy Routing Through Game Theory}},
	author={Milad Nasr and Amir Houmansadr},
	booktitle={The $23^{rd}$ ACM Conference on Computer and Communications Security (CCS)},
	year={2016}
	}
	
	@inproceedings{CDNReaper,
	title={{Practical Censorship Evasion Leveraging Content Delivery Networks}},
	author={Hadi Zolfaghari and Amir Houmansadr},
	booktitle={The $23^{rd}$ ACM Conference on Computer and Communications Security (CCS)},
	year={2016}
	}
	
	@misc{Leberknight2010,
	Author = {Leberknight, C. and Chiang, M. and Poor, H. and Wong, F.},
	Howpublished = {\url{http://www.princeton.edu/~chiangm/anticensorship.pdf}},
	Title = {{A Taxonomy of Internet Censorship and Anti-censorship}},
	Year = {2010}}
	
	@techreport{ultrasurf-analysis,
	Author = {Appelbaum, Jacob},
	Institution = {The Tor Project},
	Title = {{Technical analysis of the Ultrasurf proxying software}},
	Url = {http://scholar.google.com/scholar?hl=en\&btnG=Search\&q=intitle:Technical+analysis+of+the+Ultrasurf+proxying+software\#0},
	Year = {2012},
	Bdsk-Url-1 = {http://scholar.google.com/scholar?hl=en%5C&btnG=Search%5C&q=intitle:Technical+analysis+of+the+Ultrasurf+proxying+software%5C#0}}
	
	@misc{gifc:07,
	Howpublished = {\url{http://www.internetfreedom.org/archive/Defeat\_Internet\_Censorship\_White\_Paper.pdf}},
	Key = {defeatcensorship},
	Publisher = {Global Internet Freedom Consortium (GIFC)},
	Title = {{Defeat Internet Censorship: Overview of Advanced Technologies and Products}},
	Type = {White Paper},
	Year = {2007}}
	
	@article{pan2011survey,
	Author = {Pan, J. and Paul, S. and Jain, R.},
	Journal = {Communications Magazine, IEEE},
	Number = {7},
	Pages = {26--36},
	Publisher = {IEEE},
	Title = {{A Survey of the Research on Future Internet Architectures}},
	Volume = {49},
	Year = {2011}}
	
	@misc{nsf-fia,
	Howpublished = {\url{http://www.nets-fia.net/}},
	Key = {FIA},
	Title = {{NSF Future Internet Architecture Project}}}
	
	@misc{NDN,
	Howpublished = {\url{http://www.named- data.net}},
	Key = {NDN},
	Title = {{Named Data Networking Project}}}
	
	@inproceedings{MobilityFirst,
	Author = {Seskar, I. and Nagaraja, K. and Nelson, S. and Raychaudhuri, D.},
	Booktitle = {Asian Internet Engineering Conference},
	Title = {{Mobilityfirst Future internet Architecture Project}},
	Year = {2011}}
	
	@incollection{NEBULA,
	Author = {Anderson, T. and Birman, K. and Broberg, R. and Caesar, M. and Comer, D. and Cotton, C. and Freedman, M.~J. and Haeberlen, A. and Ives, Z.~G. and Krishnamurthy, A. and others},
	Booktitle = {The Future Internet},
	Pages = {16--26},
	Publisher = {Springer},
	Title = {{The NEBULA Future Internet Architecture}},
	Year = {2013}}
	
	@inproceedings{XIA,
	Author = {Anand, A. and Dogar, F. and Han, D. and Li, B. and Lim, H. and Machado, M. and Wu, W. and Akella, A. and Andersen, D.~G. and Byers, J.~W. and others},
	Booktitle = {ACM Workshop on Hot Topics in Networks},
	Pages = {2},
	Title = {{XIA: An Architecture for an Evolvable and Trustworthy Internet}},
	Year = {2011}}
	
	@inproceedings{ChoiceNet,
	Author = {Rouskas, G.~N. and Baldine, I. and Calvert, K.~L. and Dutta, R. and Griffioen, J. and Nagurney, A. and Wolf, T.},
	Booktitle = {ONDM},
	Title = {{ChoiceNet: Network Innovation Through Choice}},
	Year = {2013}}
	
	@misc{nsf-find,
	Howpublished = {http://www.nets-find.net/},
	Title = {{NSF NeTS FIND Initiative}}}
	
	@article{traid,
	Author = {Cheriton, D.~R. and Gritter, M.},
	Title = {{TRIAD: A New Next-Generation Internet Architecture}},
	Year = {2000}}
	
	@inproceedings{dona,
	Author = {Koponen, T. and Chawla, M. and Chun, B-G. and Ermolinskiy, A. and Kim, K.~H. and Shenker, S. and Stoica, I.},
	Booktitle = {ACM SIGCOMM Computer Communication Review},
	Number = {4},
	Organization = {ACM},
	Pages = {181--192},
	Title = {{A Data-Oriented (and Beyond) Network Architecture}},
	Volume = {37},
	Year = {2007}}
	
	@misc{ultrasurf,
	Howpublished = {\url{http://www.ultrareach.com}},
	Key = {ultrasurf},
	Title = {{Ultrasurf}}}
	
	@misc{tor-bridge,
	Author = {Dingledine, R. and Mathewson, N.},
	Howpublished = {\url{https://svn.torproject.org/svn/projects/design-paper/blocking.html}},
	Title = {{Design of a Blocking-Resistant Anonymity System}}}
	
	@inproceedings{McLachlanH09,
	Author = {J. McLachlan and N. Hopper},
	Booktitle = {WPES},
	Title = {{On the Risks of Serving Whenever You Surf: Vulnerabilities in Tor's Blocking Resistance Design}},
	Year = {2009}}
	
	@inproceedings{mahdian2010,
	Author = {Mahdian, M.},
	Booktitle = {{Fun with Algorithms}},
	Title = {{Fighting Censorship with Algorithms}},
	Year = {2010}}
	
	@inproceedings{McCoy2011,
	Author = {McCoy, D. and Morales, J.~A. and Levchenko, K.},
	Booktitle = {FC},
	Title = {{Proximax: A Measurement Based System for Proxies Dissemination}},
	Year = {2011}}
	
	@inproceedings{Sovran2008,
	Author = {Sovran, Y. and Libonati, A. and Li, J.},
	Booktitle = {IPTPS},
	Title = {{Pass it on: Social Networks Stymie Censors}},
	Year = {2008}}
	
	@inproceedings{rbridge,
	Author = {Wang, Q. and Lin, Zi and Borisov, N. and Hopper, N.},
	Booktitle = {{NDSS}},
	Title = {{rBridge: User Reputation based Tor Bridge Distribution with Privacy Preservation}},
	Year = {2013}}
	
	@inproceedings{telex,
	Author = {Wustrow, E. and Wolchok, S. and Goldberg, I. and Halderman, J.},
	Booktitle = {{USENIX Security}},
	Title = {{Telex: Anticensorship in the Network Infrastructure}},
	Year = {2011}}
	
	@inproceedings{cirripede,
	Author = {Houmansadr, A. and Nguyen, G. and Caesar, M. and Borisov, N.},
	Booktitle = {CCS},
	Title = {{Cirripede: Circumvention Infrastructure Using Router Redirection with Plausible Deniability}},
	Year = {2011}}
	
	@inproceedings{decoyrouting,
	Author = {Karlin, J. and Ellard, D. and Jackson, A. and Jones, C. and Lauer, G. and Mankins, D. and Strayer, W.},
	Booktitle = {{FOCI}},
	Title = {{Decoy Routing: Toward Unblockable Internet Communication}},
	Year = {2011}}
	
	@inproceedings{routing-around-decoys,
	Author = {M.~Schuchard and J.~Geddes and C.~Thompson and N.~Hopper},
	Booktitle = {{CCS}},
	Title = {{Routing Around Decoys}},
	Year = {2012}}
	
	@inproceedings{parrot,
	Author = {A. Houmansadr and C. Brubaker and V. Shmatikov},
	Booktitle = {IEEE S\&P},
	Title = {{The Parrot is Dead: Observing Unobservable Network Communications}},
	Year = {2013}}
	
	@misc{knock,
	Author = {T. Wilde},
	Howpublished = {\url{https://blog.torproject.org/blog/knock-knock-knockin-bridges-doors}},
	Title = {{Knock Knock Knockin' on Bridges' Doors}},
	Year = {2012}}
	
	@inproceedings{china-tor,
	Author = {Winter, P. and Lindskog, S.},
	Booktitle = {{FOCI}},
	Title = {{How the Great Firewall of China Is Blocking Tor}},
	Year = {2012}}
	
	@misc{discover-bridge,
	Howpublished = {\url{https://blog.torproject.org/blog/research-problems-ten-ways-discover-tor-bridges}},
	Key = {tenways},
	Title = {{Ten Ways to Discover Tor Bridges}}}
	
	@inproceedings{freewave,
	Author = {A.~Houmansadr and T.~Riedl and N.~Borisov and A.~Singer},
	Booktitle = {{NDSS}},
	Title = {{I Want My Voice to Be Heard: IP over Voice-over-IP for Unobservable Censorship Circumvention}},
	Year = 2013}
	
	@inproceedings{censorspoofer,
	Author = {Q. Wang and X. Gong and G. Nguyen and A. Houmansadr and N. Borisov},
	Booktitle = {CCS},
	Title = {{CensorSpoofer: Asymmetric Communication Using IP Spoofing for Censorship-Resistant Web Browsing}},
	Year = {2012}}
	
	@inproceedings{skypemorph,
	Author = {H. Moghaddam and B. Li and M. Derakhshani and I. Goldberg},
	Booktitle = {CCS},
	Title = {{SkypeMorph: Protocol Obfuscation for Tor Bridges}},
	Year = {2012}}
	
	@inproceedings{stegotorus,
	Author = {Weinberg, Z. and Wang, J. and Yegneswaran, V. and Briesemeister, L. and Cheung, S. and Wang, F. and Boneh, D.},
	Booktitle = {CCS},
	Title = {{StegoTorus: A Camouflage Proxy for the Tor Anonymity System}},
	Year = {2012}}
	
	@techreport{dust,
	Author = {{B.~Wiley}},
	Howpublished = {\url{http://blanu.net/ Dust.pdf}},
	Institution = {School of Information, University of Texas at Austin},
	Title = {{Dust: A Blocking-Resistant Internet Transport Protocol}},
	Year = {2011}}
	
	@inproceedings{FTE,
	Author = {K.~Dyer and S.~Coull and T.~Ristenpart and T.~Shrimpton},
	Booktitle = {CCS},
	Title = {{Protocol Misidentification Made Easy with Format-Transforming Encryption}},
	Year = {2013}}
	
	@inproceedings{fp,
	Author = {Fifield, D. and Hardison, N. and Ellithrope, J. and Stark, E. and Dingledine, R. and Boneh, D. and Porras, P.},
	Booktitle = {PETS},
	Title = {{Evading Censorship with Browser-Based Proxies}},
	Year = {2012}}
	
	@misc{obfsproxy,
	Howpublished = {\url{https://www.torproject.org/projects/obfsproxy.html.en}},
	Key = {obfsproxy},
	Publisher = {The Tor Project},
	Title = {{A Simple Obfuscating Proxy}}}
	
	@inproceedings{Tor-instead-of-IP,
	Author = {Liu, V. and Han, S. and Krishnamurthy, A. and Anderson, T.},
	Booktitle = {HotNets},
	Title = {{Tor instead of IP}},
	Year = {2011}}
	
	@misc{roger-slides,
	Howpublished = {\url{https://svn.torproject.org/svn/projects/presentations/slides-28c3.pdf}},
	Key = {torblocking},
	Title = {{How Governments Have Tried to Block Tor}}}
	
	@inproceedings{infranet,
	Author = {Feamster, N. and Balazinska, M. and Harfst, G. and Balakrishnan, H. and Karger, D.},
	Booktitle = {USENIX Security},
	Title = {{Infranet: Circumventing Web Censorship and Surveillance}},
	Year = {2002}}
	
	@inproceedings{collage,
	Author = {S.~Burnett and N.~Feamster and S.~Vempala},
	Booktitle = {USENIX Security},
	Title = {{Chipping Away at Censorship Firewalls with User-Generated Content}},
	Year = {2010}}
	
	@article{anonymizer,
	Author = {Boyan, J.},
	Journal = {Computer-Mediated Communication Magazine},
	Month = sep,
	Number = {9},
	Title = {{The Anonymizer: Protecting User Privacy on the Web}},
	Volume = {4},
	Year = {1997}}
	
	@article{schulze2009internet,
	Author = {Schulze, H. and Mochalski, K.},
	Journal = {IPOQUE Report},
	Pages = {351--362},
	Title = {Internet Study 2008/2009},
	Volume = {37},
	Year = {2009}}
	
	@inproceedings{cya-ccs13,
	Author = {J.~Geddes and M.~Schuchard and N.~Hopper},
	Booktitle = {{CCS}},
	Title = {{Cover Your ACKs: Pitfalls of Covert Channel Censorship Circumvention}},
	Year = {2013}}
	
	@inproceedings{andana,
	Author = {DiBenedetto, S. and Gasti, P. and Tsudik, G. and Uzun, E.},
	Booktitle = {{NDSS}},
	Title = {{ANDaNA: Anonymous Named Data Networking Application}},
	Year = {2012}}
	
	@inproceedings{darkly,
	Author = {Jana, S. and Narayanan, A. and Shmatikov, V.},
	Booktitle = {IEEE S\&P},
	Title = {{A Scanner Darkly: Protecting User Privacy From Perceptual Applications}},
	Year = {2013}}
	
	@inproceedings{NS08,
	Author = {A.~Narayanan and V.~Shmatikov},
	Booktitle = {IEEE S\&P},
	Title = {Robust de-anonymization of large sparse datasets},
	Year = {2008}}
	
	@inproceedings{NS09,
	Author = {A.~Narayanan and V.~Shmatikov},
	Booktitle = {IEEE S\&P},
	Title = {De-anonymizing social networks},
	Year = {2009}}
	
	@inproceedings{memento,
	Author = {Jana, S. and Shmatikov, V.},
	Booktitle = {IEEE S\&P},
	Title = {{Memento: Learning secrets from process footprints}},
	Year = {2012}}
	
	@misc{plugtor,
	Howpublished = {\url{https://www.torproject.org/docs/pluggable-transports.html.en}},
	Key = {PluggableTransports},
	Publisher = {The Tor Project},
	Title = {{Tor: Pluggable transports}}}
	
	@misc{psiphon,
	Author = {J.~Jia and P.~Smith},
	Howpublished = {\url{http://www.cdf.toronto.edu/~csc494h/reports/2004-fall/psiphon_ae.html}},
	Title = {{Psiphon: Analysis and Estimation}},
	Year = 2004}
	
	@misc{china-github,
	Howpublished = {\url{http://mobile.informationweek.com/80269/show/72e30386728f45f56b343ddfd0fdb119/}},
	Key = {github},
	Title = {{China's GitHub Censorship Dilemma}}}
	
	@inproceedings{txbox,
	Author = {Jana, S. and Porter, D. and Shmatikov, V.},
	Booktitle = {IEEE S\&P},
	Title = {{TxBox: Building Secure, Efficient Sandboxes with System Transactions}},
	Year = {2011}}
	
	@inproceedings{airavat,
	Author = {I. Roy and S. Setty and A. Kilzer and V. Shmatikov and E. Witchel},
	Booktitle = {NSDI},
	Title = {{Airavat: Security and Privacy for MapReduce}},
	Year = {2010}}
	
	@inproceedings{osdi12,
	Author = {A. Dunn and M. Lee and S. Jana and S. Kim and M. Silberstein and Y. Xu and V. Shmatikov and E. Witchel},
	Booktitle = {OSDI},
	Title = {{Eternal Sunshine of the Spotless Machine: Protecting Privacy with Ephemeral Channels}},
	Year = {2012}}
	
	@inproceedings{ymal,
	Author = {J. Calandrino and A. Kilzer and A. Narayanan and E. Felten and V. Shmatikov},
	Booktitle = {IEEE S\&P},
	Title = {{``You Might Also Like:'' Privacy Risks of Collaborative Filtering}},
	Year = {2011}}
	
	@inproceedings{srivastava11,
	Author = {V. Srivastava and M. Bond and K. McKinley and V. Shmatikov},
	Booktitle = {PLDI},
	Title = {{A Security Policy Oracle: Detecting Security Holes Using Multiple API Implementations}},
	Year = {2011}}
	
	@inproceedings{chen-oakland10,
	Author = {Chen, S. and Wang, R. and Wang, X. and Zhang, K.},
	Booktitle = {IEEE S\&P},
	Title = {{Side-Channel Leaks in Web Applications: A Reality Today, a Challenge Tomorrow}},
	Year = {2010}}
	
	@book{kerck,
	Author = {Kerckhoffs, A.},
	Publisher = {University Microfilms},
	Title = {{La cryptographie militaire}},
	Year = {1978}}
	
	@inproceedings{foci11,
	Author = {J. Karlin and D. Ellard and A.~Jackson and C.~ Jones and G. Lauer and D. Mankins and W.~T.~Strayer},
	Booktitle = {FOCI},
	Title = {{Decoy Routing: Toward Unblockable Internet Communication}},
	Year = 2011}
	
	@inproceedings{sun02,
	Author = {Sun, Q. and Simon, D.~R. and Wang, Y. and Russell, W. and Padmanabhan, V. and Qiu, L.},
	Booktitle = {IEEE S\&P},
	Title = {{Statistical Identification of Encrypted Web Browsing Traffic}},
	Year = {2002}}
	
	@inproceedings{danezis,
	Author = {Murdoch, S.~J. and Danezis, G.},
	Booktitle = {IEEE S\&P},
	Title = {{Low-Cost Traffic Analysis of Tor}},
	Year = {2005}}
	
	@inproceedings{pakicensorship,
	Author = {Z.~Nabi},
	Booktitle = {FOCI},
	Title = {The Anatomy of {Web} Censorship in {Pakistan}},
	Year = {2013}}
	
	@inproceedings{irancensorship,
	Author = {S.~Aryan and H.~Aryan and A.~Halderman},
	Booktitle = {FOCI},
	Title = {Internet Censorship in {Iran}: {A} First Look},
	Year = {2013}}
	
	@inproceedings{ford10efficient,
	Author = {Amittai Aviram and Shu-Chun Weng and Sen Hu and Bryan Ford},
	Booktitle = {\bibconf[9th]{OSDI}{USENIX Symposium on Operating Systems Design and Implementation}},
	Location = {Vancouver, BC, Canada},
	Month = oct,
	Title = {Efficient System-Enforced Deterministic Parallelism},
	Year = 2010}
	
	@inproceedings{ford10determinating,
	Author = {Amittai Aviram and Sen Hu and Bryan Ford and Ramakrishna Gummadi},
	Booktitle = {\bibconf{CCSW}{ACM Cloud Computing Security Workshop}},
	Location = {Chicago, IL},
	Month = oct,
	Title = {Determinating Timing Channels in Compute Clouds},
	Year = 2010}
	
	@inproceedings{ford12plugging,
	Author = {Bryan Ford},
	Booktitle = {\bibconf[4th]{HotCloud}{USENIX Workshop on Hot Topics in Cloud Computing}},
	Location = {Boston, MA},
	Month = jun,
	Title = {Plugging Side-Channel Leaks with Timing Information Flow Control},
	Year = 2012}
	
	@inproceedings{ford12icebergs,
	Author = {Bryan Ford},
	Booktitle = {\bibconf[4th]{HotCloud}{USENIX Workshop on Hot Topics in Cloud Computing}},
	Location = {Boston, MA},
	Month = jun,
	Title = {Icebergs in the Clouds: the {\em Other} Risks of Cloud Computing},
	Year = 2012}
	
	@misc{mullenize,
	Author = {Washington Post},
	Howpublished = {\url{http://apps.washingtonpost.com/g/page/world/gchq-report-on-mullenize-program-to-stain-anonymous-electronic-traffic/502/}},
	Month = {oct},
	Title = {{GCHQ} report on {`MULLENIZE'} program to `stain' anonymous electronic traffic},
	Year = {2013}}
	
	@inproceedings{shue13street,
	Author = {Craig A. Shue and Nathanael Paul and Curtis R. Taylor},
	Booktitle = {\bibbrev[7th]{WOOT}{USENIX Workshop on Offensive Technologies}},
	Month = aug,
	Title = {From an {IP} Address to a Street Address: Using Wireless Signals to Locate a Target},
	Year = 2013}
	
	@inproceedings{knockel11three,
	Author = {Jeffrey Knockel and Jedidiah R. Crandall and Jared Saia},
	Booktitle = {\bibbrev{FOCI}{USENIX Workshop on Free and Open Communications on the Internet}},
	Location = {San Francisco, CA},
	Month = aug,
	Year = 2011}
	
	@misc{rfc4960,
	Author = {R. {Stewart, ed.}},
	Month = sep,
	Note = {RFC 4960},
	Title = {Stream Control Transmission Protocol},
	Year = 2007}
	
	@inproceedings{ford07structured,
	Author = {Bryan Ford},
	Booktitle = {\bibbrev{SIGCOMM}{ACM SIGCOMM}},
	Location = {Kyoto, Japan},
	Month = aug,
	Title = {Structured Streams: a New Transport Abstraction},
	Year = {2007}}
	
	@misc{spdy,
	Author = {Google, Inc.},
	Note = {\url{http://www.chromium.org/spdy/spdy-whitepaper}},
	Title = {{SPDY}: An Experimental Protocol For a Faster {Web}}}
	
	@misc{quic,
	Author = {Jim Roskind},
	Month = jun,
	Note = {\url{http://blog.chromium.org/2013/06/experimenting-with-quic.html}},
	Title = {Experimenting with {QUIC}},
	Year = 2013}
	
	@misc{podjarny12not,
	Author = {G.~Podjarny},
	Month = jun,
	Note = {\url{http://www.guypo.com/technical/not-as-spdy-as-you-thought/}},
	Title = {{Not as SPDY as You Thought}},
	Year = 2012}
	
	@inproceedings{cor,
	Author = {Jones, N.~A. and Arye, M. and Cesareo, J. and Freedman, M.~J.},
	Booktitle = {FOCI},
	Title = {{Hiding Amongst the Clouds: A Proposal for Cloud-based Onion Routing}},
	Year = {2011}}
	
	@misc{torcloud,
	Howpublished = {\url{https://cloud.torproject.org/}},
	Key = {tor cloud},
	Title = {{The Tor Cloud Project}}}
	
	@inproceedings{scramblesuit,
	Author = {Philipp Winter and Tobias Pulls and Juergen Fuss},
	Booktitle = {WPES},
	Title = {{ScrambleSuit: A Polymorphic Network Protocol to Circumvent Censorship}},
	Year = 2013}
	
	@article{savage2000practical,
	Author = {Savage, S. and Wetherall, D. and Karlin, A. and Anderson, T.},
	Journal = {ACM SIGCOMM Computer Communication Review},
	Number = {4},
	Pages = {295--306},
	Publisher = {ACM},
	Title = {Practical network support for IP traceback},
	Volume = {30},
	Year = {2000}}
	
	@inproceedings{ooni,
	Author = {Filast, A. and Appelbaum, J.},
	Booktitle = {{FOCI}},
	Title = {{OONI : Open Observatory of Network Interference}},
	Year = {2012}}
	
	@misc{caida-rank,
	Howpublished = {\url{http://as-rank.caida.org/}},
	Key = {caida rank},
	Title = {{AS Rank: AS Ranking}}}
	
	@inproceedings{usersrouted-ccs13,
	Author = {A.~Johnson and C.~Wacek and R.~Jansen and M.~Sherr and P.~Syverson},
	Booktitle = {CCS},
	Title = {{Users Get Routed: Traffic Correlation on Tor by Realistic Adversaries}},
	Year = {2013}}
	
	@inproceedings{edman2009awareness,
	Author = {Edman, M. and Syverson, P.},
	Booktitle = {{CCS}},
	Title = {{AS-awareness in Tor path selection}},
	Year = {2009}}
	
	@inproceedings{DecoyCosts,
	Author = {A.~Houmansadr and E.~L.~Wong and V.~Shmatikov},
	Booktitle = {NDSS},
	Title = {{No Direction Home: The True Cost of Routing Around Decoys}},
	Year = {2014}}
	
	@article{cordon,
	Author = {Elahi, T. and Goldberg, I.},
	Journal = {University of Waterloo CACR},
	Title = {{CORDON--A Taxonomy of Internet Censorship Resistance Strategies}},
	Volume = {33},
	Year = {2012}}
	
	@inproceedings{privex,
	Author = {T.~Elahi and G.~Danezis and I.~Goldberg	},
	Booktitle = {{CCS}},
	Title = {{AS-awareness in Tor path selection}},
	Year = {2014}}
	
	@inproceedings{changeGuards,
	Author = {T.~Elahi and K.~Bauer and M.~AlSabah and R.~Dingledine and I.~Goldberg},
	Booktitle = {{WPES}},
	Title = {{ Changing of the Guards: Framework for Understanding and Improving Entry Guard Selection in Tor}},
	Year = {2012}}
	
	@article{RAINBOW:Journal,
	Author = {A.~Houmansadr and N.~Kiyavash and N.~Borisov},
	Journal = {IEEE/ACM Transactions on Networking},
	Title = {{Non-Blind Watermarking of Network Flows}},
	Year = 2014}
	
	@inproceedings{info-tod,
	Author = {A.~Houmansadr and S.~Gorantla and T.~Coleman and N.~Kiyavash and and N.~Borisov},
	Booktitle = {{CCS (poster session)}},
	Title = {{On the Channel Capacity of Network Flow Watermarking}},
	Year = {2009}}
	
	@inproceedings{johnson2014game,
	Author = {Johnson, B. and Laszka, A. and Grossklags, J. and Vasek, M. and Moore, T.},
	Booktitle = {Workshop on Bitcoin Research},
	Title = {{Game-theoretic Analysis of DDoS Attacks Against Bitcoin Mining Pools}},
	Year = {2014}}
	
	@incollection{laszka2013mitigation,
	Author = {Laszka, A. and Johnson, B. and Grossklags, J.},
	Booktitle = {Decision and Game Theory for Security},
	Pages = {175--191},
	Publisher = {Springer},
	Title = {{Mitigation of Targeted and Non-targeted Covert Attacks as a Timing Game}},
	Year = {2013}}
	
	@inproceedings{schottle2013game,
	Author = {Schottle, P. and Laszka, A. and Johnson, B. and Grossklags, J. and Bohme, R.},
	Booktitle = {EUSIPCO},
	Title = {{A Game-theoretic Analysis of Content-adaptive Steganography with Independent Embedding}},
	Year = {2013}}
	
	@inproceedings{CloudTransport,
	Author = {C.~Brubaker and A.~Houmansadr and V.~Shmatikov},
	Booktitle = {PETS},
	Title = {{CloudTransport: Using Cloud Storage for Censorship-Resistant Networking}},
	Year = {2014}}
	
	@inproceedings{sweet,
	Author = {W.~Zhou and A.~Houmansadr and M.~Caesar and N.~Borisov},
	Booktitle = {HotPETs},
	Title = {{SWEET: Serving the Web by Exploiting Email Tunnels}},
	Year = {2013}}
	
	@inproceedings{ahsan2002practical,
	Author = {Ahsan, K. and Kundur, D.},
	Booktitle = {Workshop on Multimedia Security},
	Title = {{Practical data hiding in TCP/IP}},
	Year = {2002}}
	
	@incollection{danezis2011covert,
	Author = {Danezis, G.},
	Booktitle = {Security Protocols XVI},
	Pages = {198--214},
	Publisher = {Springer},
	Title = {{Covert Communications Despite Traffic Data Retention}},
	Year = {2011}}
	
	@inproceedings{liu2009hide,
	Author = {Liu, Y. and Ghosal, D. and Armknecht, F. and Sadeghi, A.-R. and Schulz, S. and Katzenbeisser, S.},
	Booktitle = {ESORICS},
	Title = {{Hide and Seek in Time---Robust Covert Timing Channels}},
	Year = {2009}}
	
	@misc{image-watermark-fing,
	Author = {Jonathan Bailey},
	Howpublished = {\url{https://www.plagiarismtoday.com/2009/12/02/image-detection-watermarking-vs-fingerprinting/}},
	Title = {{Image Detection: Watermarking vs. Fingerprinting}},
	Year = {2009}}
	
	@inproceedings{Servetto98,
	Author = {S. D. Servetto and C. I. Podilchuk and K. Ramchandran},
	Booktitle = {Int. Conf. Image Processing},
	Title = {Capacity issues in digital image watermarking},
	Year = {1998}}
	
	@inproceedings{Chen01,
	Author = {B. Chen and G.W.Wornell},
	Booktitle = {IEEE Trans. Inform. Theory},
	Pages = {1423--1443},
	Title = {Quantization index modulation: A class of provably good methods for digital watermarking and information embedding},
	Year = {2001}}
	
	@inproceedings{Karakos00,
	Author = {D. Karakos and A. Papamarcou},
	Booktitle = {IEEE Int. Symp. Information Theory},
	Pages = {47},
	Title = {Relationship between quantization and distribution rates of digitally watermarked data},
	Year = {2000}}
	
	@inproceedings{Sullivan98,
	Author = {J. A. OSullivan and P. Moulin and J. M. Ettinger},
	Booktitle = {IEEE Int. Symp. Information Theory},
	Pages = {297},
	Title = {Information theoretic analysis of steganography},
	Year = {1998}}
	
	@inproceedings{Merhav00,
	Author = {N. Merhav},
	Booktitle = {IEEE Trans. Inform. Theory},
	Pages = {420--430},
	Title = {On random coding error exponents of watermarking systems},
	Year = {2000}}
	
	@inproceedings{Somekh01,
	Author = {A. Somekh-Baruch and N. Merhav},
	Booktitle = {IEEE Int. Symp. Information Theory},
	Pages = {7},
	Title = {On the error exponent and capacity games of private watermarking systems},
	Year = {2001}}
	
	@inproceedings{Steinberg01,
	Author = {Y. Steinberg and N. Merhav},
	Booktitle = {IEEE Trans. Inform. Theory},
	Pages = {1410--1422},
	Title = {Identification in the presence of side information with application to watermarking},
	Year = {2001}}
	
	@article{Moulin03,
	Author = {P. Moulin and J.A. O'Sullivan},
	Journal = {IEEE Trans. Info. Theory},
	Number = {3},
	Title = {Information-theoretic analysis of information hiding},
	Volume = 49,
	Year = 2003}
	
	@article{Gelfand80,
	Author = {S.I.~Gelfand and M.S.~Pinsker},
	Journal = {Problems of Control and Information Theory},
	Number = {1},
	Pages = {19-31},
	Title = {{Coding for channel with random parameters}},
	Url = {citeseer.ist.psu.edu/anantharam96bits.html},
	Volume = {9},
	Year = {1980},
	Bdsk-Url-1 = {citeseer.ist.psu.edu/anantharam96bits.html}}
	
	@book{Wolfowitz78,
	Author = {J. Wolfowitz},
	Edition = {3rd},
	Location = {New York},
	Publisher = {Springer-Verlag},
	Title = {Coding Theorems of Information Theory},
	Year = 1978}
	
	@article{caire99,
	Author = {G. Caire and S. Shamai},
	Journal = {IEEE Transactions on Information Theory},
	Number = {6},
	Pages = {2007--2019},
	Title = {On the Capacity of Some Channels with Channel State Information},
	Volume = {45},
	Year = {1999}}
	
	@inproceedings{wright2007language,
	Author = {Wright, Charles V and Ballard, Lucas and Monrose, Fabian and Masson, Gerald M},
	Booktitle = {USENIX Security},
	Title = {{Language identification of encrypted VoIP traffic: Alejandra y Roberto or Alice and Bob?}},
	Year = {2007}}
	
	@inproceedings{backes2010speaker,
	Author = {Backes, Michael and Doychev, Goran and D{\"u}rmuth, Markus and K{\"o}pf, Boris},
	Booktitle = {{European Symposium on Research in Computer Security (ESORICS)}},
	Pages = {508--523},
	Publisher = {Springer},
	Title = {{Speaker Recognition in Encrypted Voice Streams}},
	Year = {2010}}
	
	@phdthesis{lu2009traffic,
	Author = {Lu, Yuanchao},
	School = {Cleveland State University},
	Title = {{On Traffic Analysis Attacks to Encrypted VoIP Calls}},
	Year = {2009}}
	
	@inproceedings{wright2008spot,
	Author = {Wright, Charles V and Ballard, Lucas and Coull, Scott E and Monrose, Fabian and Masson, Gerald M},
	Booktitle = {IEEE Symposium on Security and Privacy},
	Pages = {35--49},
	Title = {Spot me if you can: Uncovering spoken phrases in encrypted VoIP conversations},
	Year = {2008}}
	
	@inproceedings{white2011phonotactic,
	Author = {White, Andrew M and Matthews, Austin R and Snow, Kevin Z and Monrose, Fabian},
	Booktitle = {IEEE Symposium on Security and Privacy},
	Pages = {3--18},
	Title = {Phonotactic reconstruction of encrypted VoIP conversations: Hookt on fon-iks},
	Year = {2011}}
	
	@inproceedings{fancy,
	Author = {Houmansadr, Amir and Borisov, Nikita},
	Booktitle = {Privacy Enhancing Technologies},
	Organization = {Springer},
	Pages = {205--224},
	Title = {The Need for Flow Fingerprints to Link Correlated Network Flows},
	Year = {2013}}
	
	@article{botmosaic,
	Author = {Amir Houmansadr and Nikita Borisov},
	Doi = {10.1016/j.jss.2012.11.005},
	Issn = {0164-1212},
	Journal = {Journal of Systems and Software},
	Keywords = {Network security},
	Number = {3},
	Pages = {707 - 715},
	Title = {BotMosaic: Collaborative network watermark for the detection of IRC-based botnets},
	Url = {http://www.sciencedirect.com/science/article/pii/S0164121212003068},
	Volume = {86},
	Year = {2013},
	Bdsk-Url-1 = {http://www.sciencedirect.com/science/article/pii/S0164121212003068},
	Bdsk-Url-2 = {http://dx.doi.org/10.1016/j.jss.2012.11.005}}
	
	@inproceedings{ramsbrock2008first,
	Author = {Ramsbrock, Daniel and Wang, Xinyuan and Jiang, Xuxian},
	Booktitle = {Recent Advances in Intrusion Detection},
	Organization = {Springer},
	Pages = {59--77},
	Title = {A first step towards live botmaster traceback},
	Year = {2008}}
	
	@inproceedings{potdar2005survey,
	Author = {Potdar, Vidyasagar M and Han, Song and Chang, Elizabeth},
	Booktitle = {Industrial Informatics, 2005. INDIN'05. 2005 3rd IEEE International Conference on},
	Organization = {IEEE},
	Pages = {709--716},
	Title = {A survey of digital image watermarking techniques},
	Year = {2005}}
	
	@book{cole2003hiding,
	Author = {Cole, Eric and Krutz, Ronald D},
	Publisher = {John Wiley \& Sons, Inc.},
	Title = {Hiding in plain sight: Steganography and the art of covert communication},
	Year = {2003}}
	
	@incollection{akaike1998information,
	Author = {Akaike, Hirotogu},
	Booktitle = {Selected Papers of Hirotugu Akaike},
	Pages = {199--213},
	Publisher = {Springer},
	Title = {Information theory and an extension of the maximum likelihood principle},
	Year = {1998}}
	
	@misc{central-command-hack,
	Author = {Everett Rosenfeld},
	Howpublished = {\url{http://www.cnbc.com/id/102330338}},
	Title = {{FBI investigating Central Command Twitter hack}},
	Year = {2015}}
	
	@misc{sony-psp-ddos,
	Howpublished = {\url{http://n4g.com/news/1644853/sony-and-microsoft-cant-do-much-ddos-attacks-explained}},
	Key = {sony},
	Month = {December},
	Title = {{Sony and Microsoft cant do much -- DDoS attacks explained}},
	Year = {2014}}
	
	@misc{sony-hack,
	Author = {David Bloom},
	Howpublished = {\url{http://goo.gl/MwR4A7}},
	Title = {{Online Game Networks Hacked, Sony Unit President Threatened}},
	Year = {2014}}
	
	@misc{home-depot,
	Author = {Dune Lawrence},
	Howpublished = {\url{http://www.businessweek.com/articles/2014-09-02/home-depots-credit-card-breach-looks-just-like-the-target-hack}},
	Title = {{Home Depot's Suspected Breach Looks Just Like the Target Hack}},
	Year = {2014}}
	
	@misc{target,
	Author = {Julio Ojeda-Zapata},
	Howpublished = {\url{http://www.mercurynews.com/business/ci_24765398/how-did-hackers-pull-off-target-data-heist}},
	Title = {{Target hack: How did they do it?}},
	Year = {2014}}
	
	
	@article{probabilitycourse,
	Author = {H. Pishro-Nik},
	note = {\url{http://www.probabilitycourse.com}},
	Title = {Introduction to probability, statistics, and random processes},
	Year = {2014}}
	
	
	
	@inproceedings{shokri2011quantifying,
	Author = {Shokri, Reza and Theodorakopoulos, George and Le Boudec, Jean-Yves and Hubaux, Jean-Pierre},
	Booktitle = {Security and Privacy (SP), 2011 IEEE Symposium on},
	Organization = {IEEE},
	Pages = {247--262},
	Title = {Quantifying location privacy},
	Year = {2011}}
	
	@inproceedings{hoh2007preserving,
	Author = {Hoh, Baik and Gruteser, Marco and Xiong, Hui and Alrabady, Ansaf},
	Booktitle = {Proceedings of the 14th ACM conference on Computer and communications security},
	Organization = {ACM},
	Pages = {161--171},
	Title = {Preserving privacy in gps traces via uncertainty-aware path cloaking},
	Year = {2007}}
	
	
	
	@article{kafsi2013entropy,
	Author = {Kafsi, Mohamed and Grossglauser, Matthias and Thiran, Patrick},
	Journal = {Information Theory, IEEE Transactions on},
	Number = {9},
	Pages = {5577--5583},
	Publisher = {IEEE},
	Title = {The entropy of conditional Markov trajectories},
	Volume = {59},
	Year = {2013}}
	
	@inproceedings{gruteser2003anonymous,
	Author = {Gruteser, Marco and Grunwald, Dirk},
	Booktitle = {Proceedings of the 1st international conference on Mobile systems, applications and services},
	Organization = {ACM},
	Pages = {31--42},
	Title = {Anonymous usage of location-based services through spatial and temporal cloaking},
	Year = {2003}}
	
	@inproceedings{husted2010mobile,
	Author = {Husted, Nathaniel and Myers, Steven},
	Booktitle = {Proceedings of the 17th ACM conference on Computer and communications security},
	Organization = {ACM},
	Pages = {85--96},
	Title = {Mobile location tracking in metro areas: malnets and others},
	Year = {2010}}
	
	@inproceedings{li2009tradeoff,
	Author = {Li, Tiancheng and Li, Ninghui},
	Booktitle = {Proceedings of the 15th ACM SIGKDD international conference on Knowledge discovery and data mining},
	Organization = {ACM},
	Pages = {517--526},
	Title = {On the tradeoff between privacy and utility in data publishing},
	Year = {2009}}
	
	@inproceedings{ma2009location,
	Author = {Ma, Zhendong and Kargl, Frank and Weber, Michael},
	Booktitle = {Sarnoff Symposium, 2009. SARNOFF'09. IEEE},
	Organization = {IEEE},
	Pages = {1--6},
	Title = {A location privacy metric for v2x communication systems},
	Year = {2009}}
	
	@inproceedings{shokri2012protecting,
	Author = {Shokri, Reza and Theodorakopoulos, George and Troncoso, Carmela and Hubaux, Jean-Pierre and Le Boudec, Jean-Yves},
	Booktitle = {Proceedings of the 2012 ACM conference on Computer and communications security},
	Organization = {ACM},
	Pages = {617--627},
	Title = {Protecting location privacy: optimal strategy against localization attacks},
	Year = {2012}}
	
	@inproceedings{freudiger2009non,
	Author = {Freudiger, Julien and Manshaei, Mohammad Hossein and Hubaux, Jean-Pierre and Parkes, David C},
	Booktitle = {Proceedings of the 16th ACM conference on Computer and communications security},
	Organization = {ACM},
	Pages = {324--337},
	Title = {On non-cooperative location privacy: a game-theoretic analysis},
	Year = {2009}}
	
	@incollection{humbert2010tracking,
	Author = {Humbert, Mathias and Manshaei, Mohammad Hossein and Freudiger, Julien and Hubaux, Jean-Pierre},
	Booktitle = {Decision and Game Theory for Security},
	Pages = {38--57},
	Publisher = {Springer},
	Title = {Tracking games in mobile networks},
	Year = {2010}}
	
	@article{manshaei2013game,
	Author = {Manshaei, Mohammad Hossein and Zhu, Quanyan and Alpcan, Tansu and Bac{\c{s}}ar, Tamer and Hubaux, Jean-Pierre},
	Journal = {ACM Computing Surveys (CSUR)},
	Number = {3},
	Pages = {25},
	Publisher = {ACM},
	Title = {Game theory meets network security and privacy},
	Volume = {45},
	Year = {2013}}
	
	@article{palamidessi2006probabilistic,
	Author = {Palamidessi, Catuscia},
	Journal = {Electronic Notes in Theoretical Computer Science},
	Pages = {33--42},
	Publisher = {Elsevier},
	Title = {Probabilistic and nondeterministic aspects of anonymity},
	Volume = {155},
	Year = {2006}}
	
	@inproceedings{mokbel2006new,
	Author = {Mokbel, Mohamed F and Chow, Chi-Yin and Aref, Walid G},
	Booktitle = {Proceedings of the 32nd international conference on Very large data bases},
	Organization = {VLDB Endowment},
	Pages = {763--774},
	Title = {The new Casper: query processing for location services without compromising privacy},
	Year = {2006}}
	
	@article{kalnis2007preventing,
	Author = {Kalnis, Panos and Ghinita, Gabriel and Mouratidis, Kyriakos and Papadias, Dimitris},
	Journal = {Knowledge and Data Engineering, IEEE Transactions on},
	Number = {12},
	Pages = {1719--1733},
	Publisher = {IEEE},
	Title = {Preventing location-based identity inference in anonymous spatial queries},
	Volume = {19},
	Year = {2007}}
	
	@article{freudiger2007mix,
	title={Mix-zones for location privacy in vehicular networks},
	author={Freudiger, Julien and Raya, Maxim and F{\'e}legyh{\'a}zi, M{\'a}rk and Papadimitratos, Panos and Hubaux, Jean-Pierre},
	year={2007}
	}
	@article{sweeney2002k,
	Author = {Sweeney, Latanya},
	Journal = {International Journal of Uncertainty, Fuzziness and Knowledge-Based Systems},
	Number = {05},
	Pages = {557--570},
	Publisher = {World Scientific},
	Title = {k-anonymity: A model for protecting privacy},
	Volume = {10},
	Year = {2002}}
	
	@article{sweeney2002achieving,
	Author = {Sweeney, Latanya},
	Journal = {International Journal of Uncertainty, Fuzziness and Knowledge-Based Systems},
	Number = {05},
	Pages = {571--588},
	Publisher = {World Scientific},
	Title = {Achieving k-anonymity privacy protection using generalization and suppression},
	Volume = {10},
	Year = {2002}}
	
	@inproceedings{niu2014achieving,
	Author = {Niu, Ben and Li, Qinghua and Zhu, Xiaoyan and Cao, Guohong and Li, Hui},
	Booktitle = {INFOCOM, 2014 Proceedings IEEE},
	Organization = {IEEE},
	Pages = {754--762},
	Title = {Achieving k-anonymity in privacy-aware location-based services},
	Year = {2014}}
	
	@inproceedings{liu2013game,
	Author = {Liu, Xinxin and Liu, Kaikai and Guo, Linke and Li, Xiaolin and Fang, Yuguang},
	Booktitle = {INFOCOM, 2013 Proceedings IEEE},
	Organization = {IEEE},
	Pages = {2985--2993},
	Title = {A game-theoretic approach for achieving k-anonymity in location based services},
	Year = {2013}}
	
	@inproceedings{kido2005protection,
	Author = {Kido, Hidetoshi and Yanagisawa, Yutaka and Satoh, Tetsuji},
	Booktitle = {Data Engineering Workshops, 2005. 21st International Conference on},
	Organization = {IEEE},
	Pages = {1248--1248},
	Title = {Protection of location privacy using dummies for location-based services},
	Year = {2005}}
	
	@inproceedings{gedik2005location,
	Author = {Gedik, Bu{\u{g}}ra and Liu, Ling},
	Booktitle = {Distributed Computing Systems, 2005. ICDCS 2005. Proceedings. 25th IEEE International Conference on},
	Organization = {IEEE},
	Pages = {620--629},
	Title = {Location privacy in mobile systems: A personalized anonymization model},
	Year = {2005}}
	
	@inproceedings{bordenabe2014optimal,
	Author = {Bordenabe, Nicol{\'a}s E and Chatzikokolakis, Konstantinos and Palamidessi, Catuscia},
	Booktitle = {Proceedings of the 2014 ACM SIGSAC Conference on Computer and Communications Security},
	Organization = {ACM},
	Pages = {251--262},
	Title = {Optimal geo-indistinguishable mechanisms for location privacy},
	Year = {2014}}
	
	@incollection{duckham2005formal,
	Author = {Duckham, Matt and Kulik, Lars},
	Booktitle = {Pervasive computing},
	Pages = {152--170},
	Publisher = {Springer},
	Title = {A formal model of obfuscation and negotiation for location privacy},
	Year = {2005}}
	
	@inproceedings{kido2005anonymous,
	Author = {Kido, Hidetoshi and Yanagisawa, Yutaka and Satoh, Tetsuji},
	Booktitle = {Pervasive Services, 2005. ICPS'05. Proceedings. International Conference on},
	Organization = {IEEE},
	Pages = {88--97},
	Title = {An anonymous communication technique using dummies for location-based services},
	Year = {2005}}
	
	@incollection{duckham2006spatiotemporal,
	Author = {Duckham, Matt and Kulik, Lars and Birtley, Athol},
	Booktitle = {Geographic Information Science},
	Pages = {47--64},
	Publisher = {Springer},
	Title = {A spatiotemporal model of strategies and counter strategies for location privacy protection},
	Year = {2006}}
	
	@inproceedings{shankar2009privately,
	Author = {Shankar, Pravin and Ganapathy, Vinod and Iftode, Liviu},
	Booktitle = {Proceedings of the 11th international conference on Ubiquitous computing},
	Organization = {ACM},
	Pages = {31--40},
	Title = {Privately querying location-based services with SybilQuery},
	Year = {2009}}
	
	@inproceedings{chow2009faking,
	Author = {Chow, Richard and Golle, Philippe},
	Booktitle = {Proceedings of the 8th ACM workshop on Privacy in the electronic society},
	Organization = {ACM},
	Pages = {105--108},
	Title = {Faking contextual data for fun, profit, and privacy},
	Year = {2009}}
	
	@incollection{xue2009location,
	Author = {Xue, Mingqiang and Kalnis, Panos and Pung, Hung Keng},
	Booktitle = {Location and Context Awareness},
	Pages = {70--87},
	Publisher = {Springer},
	Title = {Location diversity: Enhanced privacy protection in location based services},
	Year = {2009}}
	
	@article{wernke2014classification,
	Author = {Wernke, Marius and Skvortsov, Pavel and D{\"u}rr, Frank and Rothermel, Kurt},
	Journal = {Personal and Ubiquitous Computing},
	Number = {1},
	Pages = {163--175},
	Publisher = {Springer-Verlag},
	Title = {A classification of location privacy attacks and approaches},
	Volume = {18},
	Year = {2014}}
	
	@misc{cai2015cloaking,
	Author = {Cai, Y. and Xu, G.},
	Month = jan # {~1},
	Note = {US Patent App. 14/472,462},
	Publisher = {Google Patents},
	Title = {Cloaking with footprints to provide location privacy protection in location-based services},
	Url = {https://www.google.com/patents/US20150007341},
	Year = {2015},
	Bdsk-Url-1 = {https://www.google.com/patents/US20150007341}}
	
	@article{gedik2008protecting,
	Author = {Gedik, Bu{\u{g}}ra and Liu, Ling},
	Journal = {Mobile Computing, IEEE Transactions on},
	Number = {1},
	Pages = {1--18},
	Publisher = {IEEE},
	Title = {Protecting location privacy with personalized k-anonymity: Architecture and algorithms},
	Volume = {7},
	Year = {2008}}
	
	@article{kalnis2006preserving,
	Author = {Kalnis, Panos and Ghinita, Gabriel and Mouratidis, Kyriakos and Papadias, Dimitris},
	Publisher = {TRB6/06},
	Title = {Preserving anonymity in location based services},
	Year = {2006}}
	
	@inproceedings{hoh2005protecting,
	Author = {Hoh, Baik and Gruteser, Marco},
	Booktitle = {Security and Privacy for Emerging Areas in Communications Networks, 2005. SecureComm 2005. First International Conference on},
	Organization = {IEEE},
	Pages = {194--205},
	Title = {Protecting location privacy through path confusion},
	Year = {2005}}
	
	@article{terrovitis2011privacy,
	Author = {Terrovitis, Manolis},
	Journal = {ACM SIGKDD Explorations Newsletter},
	Number = {1},
	Pages = {6--18},
	Publisher = {ACM},
	Title = {Privacy preservation in the dissemination of location data},
	Volume = {13},
	Year = {2011}}
	
	@article{shin2012privacy,
	Author = {Shin, Kang G and Ju, Xiaoen and Chen, Zhigang and Hu, Xin},
	Journal = {Wireless Communications, IEEE},
	Number = {1},
	Pages = {30--39},
	Publisher = {IEEE},
	Title = {Privacy protection for users of location-based services},
	Volume = {19},
	Year = {2012}}
	
	@article{khoshgozaran2011location,
	Author = {Khoshgozaran, Ali and Shahabi, Cyrus and Shirani-Mehr, Houtan},
	Journal = {Knowledge and Information Systems},
	Number = {3},
	Pages = {435--465},
	Publisher = {Springer},
	Title = {Location privacy: going beyond K-anonymity, cloaking and anonymizers},
	Volume = {26},
	Year = {2011}}
	
	@incollection{chatzikokolakis2015geo,
	Author = {Chatzikokolakis, Konstantinos and Palamidessi, Catuscia and Stronati, Marco},
	Booktitle = {Distributed Computing and Internet Technology},
	Pages = {49--72},
	Publisher = {Springer},
	Title = {Geo-indistinguishability: A Principled Approach to Location Privacy},
	Year = {2015}}
	
	@inproceedings{ngo2015location,
	Author = {Ngo, Hoa and Kim, Jong},
	Booktitle = {Computer Security Foundations Symposium (CSF), 2015 IEEE 28th},
	Organization = {IEEE},
	Pages = {63--74},
	Title = {Location Privacy via Differential Private Perturbation of Cloaking Area},
	Year = {2015}}
	
	@inproceedings{palanisamy2011mobimix,
	Author = {Palanisamy, Balaji and Liu, Ling},
	Booktitle = {Data Engineering (ICDE), 2011 IEEE 27th International Conference on},
	Organization = {IEEE},
	Pages = {494--505},
	Title = {Mobimix: Protecting location privacy with mix-zones over road networks},
	Year = {2011}}
	
	@inproceedings{um2010advanced,
	Author = {Um, Jung-Ho and Kim, Hee-Dae and Chang, Jae-Woo},
	Booktitle = {Social Computing (SocialCom), 2010 IEEE Second International Conference on},
	Organization = {IEEE},
	Pages = {1093--1098},
	Title = {An advanced cloaking algorithm using Hilbert curves for anonymous location based service},
	Year = {2010}}
	
	@inproceedings{bamba2008supporting,
	Author = {Bamba, Bhuvan and Liu, Ling and Pesti, Peter and Wang, Ting},
	Booktitle = {Proceedings of the 17th international conference on World Wide Web},
	Organization = {ACM},
	Pages = {237--246},
	Title = {Supporting anonymous location queries in mobile environments with privacygrid},
	Year = {2008}}
	
	@inproceedings{zhangwei2010distributed,
	Author = {Zhangwei, Huang and Mingjun, Xin},
	Booktitle = {Networks Security Wireless Communications and Trusted Computing (NSWCTC), 2010 Second International Conference on},
	Organization = {IEEE},
	Pages = {468--471},
	Title = {A distributed spatial cloaking protocol for location privacy},
	Volume = {2},
	Year = {2010}}
	
	@article{chow2011spatial,
	Author = {Chow, Chi-Yin and Mokbel, Mohamed F and Liu, Xuan},
	Journal = {GeoInformatica},
	Number = {2},
	Pages = {351--380},
	Publisher = {Springer},
	Title = {Spatial cloaking for anonymous location-based services in mobile peer-to-peer environments},
	Volume = {15},
	Year = {2011}}
	
	@inproceedings{lu2008pad,
	Author = {Lu, Hua and Jensen, Christian S and Yiu, Man Lung},
	Booktitle = {Proceedings of the Seventh ACM International Workshop on Data Engineering for Wireless and Mobile Access},
	Organization = {ACM},
	Pages = {16--23},
	Title = {Pad: privacy-area aware, dummy-based location privacy in mobile services},
	Year = {2008}}
	
	@incollection{khoshgozaran2007blind,
	Author = {Khoshgozaran, Ali and Shahabi, Cyrus},
	Booktitle = {Advances in Spatial and Temporal Databases},
	Pages = {239--257},
	Publisher = {Springer},
	Title = {Blind evaluation of nearest neighbor queries using space transformation to preserve location privacy},
	Year = {2007}}
	
	@inproceedings{ghinita2008private,
	Author = {Ghinita, Gabriel and Kalnis, Panos and Khoshgozaran, Ali and Shahabi, Cyrus and Tan, Kian-Lee},
	Booktitle = {Proceedings of the 2008 ACM SIGMOD international conference on Management of data},
	Organization = {ACM},
	Pages = {121--132},
	Title = {Private queries in location based services: anonymizers are not necessary},
	Year = {2008}}
	
	@article{paulet2014privacy,
	Author = {Paulet, Russell and Kaosar, Md Golam and Yi, Xun and Bertino, Elisa},
	Journal = {Knowledge and Data Engineering, IEEE Transactions on},
	Number = {5},
	Pages = {1200--1210},
	Publisher = {IEEE},
	Title = {Privacy-preserving and content-protecting location based queries},
	Volume = {26},
	Year = {2014}}
	
	@article{nguyen2013differential,
	Author = {Nguyen, Hiep H and Kim, Jong and Kim, Yoonho},
	Journal = {Journal of Computing Science and Engineering},
	Number = {3},
	Pages = {177--186},
	Title = {Differential privacy in practice},
	Volume = {7},
	Year = {2013}}
	
	@inproceedings{lee2012differential,
	Author = {Lee, Jaewoo and Clifton, Chris},
	Booktitle = {Proceedings of the 18th ACM SIGKDD international conference on Knowledge discovery and data mining},
	Organization = {ACM},
	Pages = {1041--1049},
	Title = {Differential identifiability},
	Year = {2012}}
	
	@inproceedings{andres2013geo,
	Author = {Andr{\'e}s, Miguel E and Bordenabe, Nicol{\'a}s E and Chatzikokolakis, Konstantinos and Palamidessi, Catuscia},
	Booktitle = {Proceedings of the 2013 ACM SIGSAC conference on Computer \& communications security},
	Organization = {ACM},
	Pages = {901--914},
	Title = {Geo-indistinguishability: Differential privacy for location-based systems},
	Year = {2013}}
	
	@inproceedings{machanavajjhala2008privacy,
	Author = {Machanavajjhala, Ashwin and Kifer, Daniel and Abowd, John and Gehrke, Johannes and Vilhuber, Lars},
	Booktitle = {Data Engineering, 2008. ICDE 2008. IEEE 24th International Conference on},
	Organization = {IEEE},
	Pages = {277--286},
	Title = {Privacy: Theory meets practice on the map},
	Year = {2008}}
	
	@article{dewri2013local,
	Author = {Dewri, Rinku},
	Journal = {Mobile Computing, IEEE Transactions on},
	Number = {12},
	Pages = {2360--2372},
	Publisher = {IEEE},
	Title = {Local differential perturbations: Location privacy under approximate knowledge attackers},
	Volume = {12},
	Year = {2013}}
	
	@inproceedings{chatzikokolakis2013broadening,
	Author = {Chatzikokolakis, Konstantinos and Andr{\'e}s, Miguel E and Bordenabe, Nicol{\'a}s Emilio and Palamidessi, Catuscia},
	Booktitle = {Privacy Enhancing Technologies},
	Organization = {Springer},
	Pages = {82--102},
	Title = {Broadening the Scope of Differential Privacy Using Metrics.},
	Year = {2013}}
	
	@inproceedings{zhong2009distributed,
	Author = {Zhong, Ge and Hengartner, Urs},
	Booktitle = {Pervasive Computing and Communications, 2009. PerCom 2009. IEEE International Conference on},
	Organization = {IEEE},
	Pages = {1--10},
	Title = {A distributed k-anonymity protocol for location privacy},
	Year = {2009}}
	
	@inproceedings{ho2011differential,
	Author = {Ho, Shen-Shyang and Ruan, Shuhua},
	Booktitle = {Proceedings of the 4th ACM SIGSPATIAL International Workshop on Security and Privacy in GIS and LBS},
	Organization = {ACM},
	Pages = {17--24},
	Title = {Differential privacy for location pattern mining},
	Year = {2011}}
	
	@inproceedings{cheng2006preserving,
	Author = {Cheng, Reynold and Zhang, Yu and Bertino, Elisa and Prabhakar, Sunil},
	Booktitle = {Privacy Enhancing Technologies},
	Organization = {Springer},
	Pages = {393--412},
	Title = {Preserving user location privacy in mobile data management infrastructures},
	Year = {2006}}
	
	@article{beresford2003location,
	Author = {Beresford, Alastair R and Stajano, Frank},
	Journal = {IEEE Pervasive computing},
	Number = {1},
	Pages = {46--55},
	Publisher = {IEEE},
	Title = {Location privacy in pervasive computing},
	Year = {2003}}
	
	@inproceedings{freudiger2009optimal,
	Author = {Freudiger, Julien and Shokri, Reza and Hubaux, Jean-Pierre},
	Booktitle = {Privacy enhancing technologies},
	Organization = {Springer},
	Pages = {216--234},
	Title = {On the optimal placement of mix zones},
	Year = {2009}}
	
	@article{krumm2009survey,
	Author = {Krumm, John},
	Journal = {Personal and Ubiquitous Computing},
	Number = {6},
	Pages = {391--399},
	Publisher = {Springer},
	Title = {A survey of computational location privacy},
	Volume = {13},
	Year = {2009}}
	
	@article{Rakhshan2016letter,
	Author = {Rakhshan, Ali and Pishro-Nik, Hossein},
	Journal = {IEEE Wireless Communications Letter},
	Publisher = {IEEE},
	Title = {Interference Models for Vehicular Ad Hoc Networks},
	Year = {2016, submitted}}
	
	@article{Rakhshan2015Journal,
	Author = {Rakhshan, Ali and Pishro-Nik, Hossein},
	Journal = {IEEE Transactions on Wireless Communications},
	Publisher = {IEEE},
	Title = {Improving Safety on Highways by Customizing Vehicular Ad Hoc Networks},
	Year = {to appear, 2017}}
	
	@inproceedings{Rakhshan2015Cogsima,
	Author = {Rakhshan, Ali and Pishro-Nik, Hossein},
	Booktitle = {IEEE International Multi-Disciplinary Conference on Cognitive Methods in Situation Awareness and Decision Support},
	Organization = {IEEE},
	Title = {A New Approach to Customization of Accident Warning Systems to Individual Drivers},
	Year = {2015}}
	
	@inproceedings{Rakhshan2015CISS,
	Author = {Rakhshan, Ali and Pishro-Nik, Hossein and Nekoui, Mohammad},
	Booktitle = {Conference on Information Sciences and Systems},
	Organization = {IEEE},
	Pages = {1--6},
	Title = {Driver-based adaptation of Vehicular Ad Hoc Networks for design of active safety systems},
	Year = {2015}}
	
	@inproceedings{Rakhshan2014IV,
	Author = {Rakhshan, Ali and Pishro-Nik, Hossein and Ray, Evan},
	Booktitle = {Intelligent Vehicles Symposium},
	Organization = {IEEE},
	Pages = {1181--1186},
	Title = {Real-time estimation of the distribution of brake response times for an individual driver using Vehicular Ad Hoc Network.},
	Year = {2014}}
	
	@inproceedings{Rakhshan2013Globecom,
	Author = {Rakhshan, Ali and Pishro-Nik, Hossein and Fisher, Donald and Nekoui, Mohammad},
	Booktitle = {IEEE Global Communications Conference},
	Organization = {IEEE},
	Pages = {1333--1337},
	Title = {Tuning collision warning algorithms to individual drivers for design of active safety systems.},
	Year = {2013}}
	
	@article{Nekoui2012Journal,
	Author = {Nekoui, Mohammad and Pishro-Nik, Hossein},
	Journal = {IEEE Transactions on Wireless Communications},
	Number = {8},
	Pages = {2895--2905},
	Publisher = {IEEE},
	Title = {Throughput Scaling laws for Vehicular Ad Hoc Networks},
	Volume = {11},
	Year = {2012}}
	
	
	
	
	
	
	
	
	
	@article{Nekoui2011Journal,
	Author = {Nekoui, Mohammad and Pishro-Nik, Hossein and Ni, Daiheng},
	Journal = {International Journal of Vehicular Technology},
	Pages = {1--11},
	Publisher = {Hindawi Publishing Corporation},
	Title = {Analytic Design of Active Safety Systems for Vehicular Ad hoc Networks},
	Volume = {2011},
	Year = {2011}}
	
	
	
	
	
	
	@article{shokri2014optimal,
	title={Optimal user-centric data obfuscation},
	author={Shokri, Reza},
	journal={arXiv preprint arXiv:1402.3426},
	year={2014}
	}
	@article{chatzikokolakis2015location,
	title={Location privacy via geo-indistinguishability},
	author={Chatzikokolakis, Konstantinos and Palamidessi, Catuscia and Stronati, Marco},
	journal={ACM SIGLOG News},
	volume={2},
	number={3},
	pages={46--69},
	year={2015},
	publisher={ACM}
	
	}
	@inproceedings{shokri2011quantifying2,
	title={Quantifying location privacy: the case of sporadic location exposure},
	author={Shokri, Reza and Theodorakopoulos, George and Danezis, George and Hubaux, Jean-Pierre and Le Boudec, Jean-Yves},
	booktitle={Privacy Enhancing Technologies},
	pages={57--76},
	year={2011},
	organization={Springer}
	}
	
	
	@inproceedings{Mont1603:Defining,
	AUTHOR="Zarrin Montazeri and Amir Houmansadr and Hossein Pishro-Nik",
	TITLE="Defining Perfect Location Privacy Using Anonymization",
	BOOKTITLE="2016 Annual Conference on Information Science and Systems (CISS) (CISS
	2016)",
	ADDRESS="Princeton, USA",
	DAYS=16,
	MONTH=mar,
	YEAR=2016,
	KEYWORDS="Information Theoretic Privacy; location-based services; Location Privacy;
	Information Theory",
	ABSTRACT="The popularity of mobile devices and location-based services (LBS) have
	created great concerns regarding the location privacy of users of such
	devices and services. Anonymization is a common technique that is often
	being used to protect the location privacy of LBS users. In this paper, we
	provide a general information theoretic definition for location privacy. In
	particular, we define perfect location privacy. We show that under certain
	conditions, perfect privacy is achieved if the pseudonyms of users is
	changed after o(N^(2/r?1)) observations by the adversary, where N is the
	number of users and r is the number of sub-regions or locations.
	"
	}
	@article{our-isita-location,
	Author = {Zarrin Montazeri and Amir Houmansadr and Hossein Pishro-Nik},
	Journal = {IEEE International Symposium on Information Theory and Its Applications (ISITA)},
	Title = {Achieving Perfect Location Privacy in Markov Models Using Anonymization},
	Year = {2016}
	}
	@article{our-TIFS,
	Author = {Zarrin Montazeri and Hossein Pishro-Nik and Amir Houmansadr},
	Journal = {IEEE Transactions on Information Forensics and Security, under revison},
	Title = {Perfect Location Privacy Using Anonymization in Mobile Networks},
	Year = {2016},
	note={Available on arxiv.org}
	}
	
	
	
	@techreport{sampigethaya2005caravan,
	title={CARAVAN: Providing location privacy for VANET},
	author={Sampigethaya, Krishna and Huang, Leping and Li, Mingyan and Poovendran, Radha and Matsuura, Kanta and Sezaki, Kaoru},
	year={2005},
	institution={DTIC Document}
	}
	@incollection{buttyan2007effectiveness,
	title={On the effectiveness of changing pseudonyms to provide location privacy in VANETs},
	author={Butty{\'a}n, Levente and Holczer, Tam{\'a}s and Vajda, Istv{\'a}n},
	booktitle={Security and Privacy in Ad-hoc and Sensor Networks},
	pages={129--141},
	year={2007},
	publisher={Springer}
	}
	@article{sampigethaya2007amoeba,
	title={AMOEBA: Robust location privacy scheme for VANET},
	author={Sampigethaya, Krishna and Li, Mingyan and Huang, Leping and Poovendran, Radha},
	journal={Selected Areas in communications, IEEE Journal on},
	volume={25},
	number={8},
	pages={1569--1589},
	year={2007},
	publisher={IEEE}
	}
	
	@article{lu2012pseudonym,
	title={Pseudonym changing at social spots: An effective strategy for location privacy in vanets},
	author={Lu, Rongxing and Li, Xiaodong and Luan, Tom H and Liang, Xiaohui and Shen, Xuemin},
	journal={Vehicular Technology, IEEE Transactions on},
	volume={61},
	number={1},
	pages={86--96},
	year={2012},
	publisher={IEEE}
	}
	@inproceedings{lu2010sacrificing,
	title={Sacrificing the plum tree for the peach tree: A socialspot tactic for protecting receiver-location privacy in VANET},
	author={Lu, Rongxing and Lin, Xiaodong and Liang, Xiaohui and Shen, Xuemin},
	booktitle={Global Telecommunications Conference (GLOBECOM 2010), 2010 IEEE},
	pages={1--5},
	year={2010},
	organization={IEEE}
	}
	@inproceedings{lin2011stap,
	title={STAP: A social-tier-assisted packet forwarding protocol for achieving receiver-location privacy preservation in VANETs},
	author={Lin, Xiaodong and Lu, Rongxing and Liang, Xiaohui and Shen, Xuemin Sherman},
	booktitle={INFOCOM, 2011 Proceedings IEEE},
	pages={2147--2155},
	year={2011},
	organization={IEEE}
	}
	@inproceedings{gerlach2007privacy,
	title={Privacy in VANETs using changing pseudonyms-ideal and real},
	author={Gerlach, Matthias and Guttler, Felix},
	booktitle={Vehicular Technology Conference, 2007. VTC2007-Spring. IEEE 65th},
	pages={2521--2525},
	year={2007},
	organization={IEEE}
	}
	@inproceedings{el2002security,
	title={Security issues in a future vehicular network},
	author={El Zarki, Magda and Mehrotra, Sharad and Tsudik, Gene and Venkatasubramanian, Nalini},
	booktitle={European Wireless},
	volume={2},
	year={2002}
	}
	
	@article{hubaux2004security,
	title={The security and privacy of smart vehicles},
	author={Hubaux, Jean-Pierre and Capkun, Srdjan and Luo, Jun},
	journal={IEEE Security \& Privacy Magazine},
	volume={2},
	number={LCA-ARTICLE-2004-007},
	pages={49--55},
	year={2004}
	}
	
	
	
	@inproceedings{duri2002framework,
	title={Framework for security and privacy in automotive telematics},
	author={Duri, Sastry and Gruteser, Marco and Liu, Xuan and Moskowitz, Paul and Perez, Ronald and Singh, Moninder and Tang, Jung-Mu},
	booktitle={Proceedings of the 2nd international workshop on Mobile commerce},
	pages={25--32},
	year={2002},
	organization={ACM}
	}
	@misc{NS-3,
	Howpublished = {\url{https://www.nsnam.org/}}},
}
@misc{testbed,
	Howpublished = {\url{http://www.its.dot.gov/testbed/PDF/SE-MI-Resource-Guide-9-3-1.pdf}}},
@misc{NGSIM,
	Howpublished = {\url{http://ops.fhwa.dot.gov/trafficanalysistools/ngsim.htm}},
}

@misc{National-a2013,
	Author = {National Highway Traffic Safety Administration},
	Howpublished = {\url{http://ops.fhwa.dot.gov/trafficanalysistools/ngsim.htm}},
	Title = {2013 Motor Vehicle Crashes: Overview. Traffic Safety Factors},
	Year = {2013}
}

@inproceedings{karnadi2007rapid,
	title={Rapid generation of realistic mobility models for VANET},
	author={Karnadi, Feliz Kristianto and Mo, Zhi Hai and Lan, Kun-chan},
	booktitle={Wireless Communications and Networking Conference, 2007. WCNC 2007. IEEE},
	pages={2506--2511},
	year={2007},
	organization={IEEE}
}
@inproceedings{saha2004modeling,
	title={Modeling mobility for vehicular ad-hoc networks},
	author={Saha, Amit Kumar and Johnson, David B},
	booktitle={Proceedings of the 1st ACM international workshop on Vehicular ad hoc networks},
	pages={91--92},
	year={2004},
	organization={ACM}
}
@inproceedings{lee2006modeling,
	title={Modeling steady-state and transient behaviors of user mobility: formulation, analysis, and application},
	author={Lee, Jong-Kwon and Hou, Jennifer C},
	booktitle={Proceedings of the 7th ACM international symposium on Mobile ad hoc networking and computing},
	pages={85--96},
	year={2006},
	organization={ACM}
}
@inproceedings{yoon2006building,
	title={Building realistic mobility models from coarse-grained traces},
	author={Yoon, Jungkeun and Noble, Brian D and Liu, Mingyan and Kim, Minkyong},
	booktitle={Proceedings of the 4th international conference on Mobile systems, applications and services},
	pages={177--190},
	year={2006},
	organization={ACM}
}

@inproceedings{choffnes2005integrated,
	title={An integrated mobility and traffic model for vehicular wireless networks},
	author={Choffnes, David R and Bustamante, Fabi{\'a}n E},
	booktitle={Proceedings of the 2nd ACM international workshop on Vehicular ad hoc networks},
	pages={69--78},
	year={2005},
	organization={ACM}
}

@inproceedings{Qian2008Globecom,
	title={CA Secure VANET MAC Protocol for DSRC Applications},
	author={Yi, Q. and Lu, K. and Moyeri, N.{\'a}n E},
	booktitle={Proceedings of IEEE GLOBECOM 2008},
	pages={1--5},
	year={2008},
	organization={IEEE}
}





@inproceedings{naumov2006evaluation,
	title={An evaluation of inter-vehicle ad hoc networks based on realistic vehicular traces},
	author={Naumov, Valery and Baumann, Rainer and Gross, Thomas},
	booktitle={Proceedings of the 7th ACM international symposium on Mobile ad hoc networking and computing},
	pages={108--119},
	year={2006},
	organization={ACM}
}
@article{sommer2008progressing,
	title={Progressing toward realistic mobility models in VANET simulations},
	author={Sommer, Christoph and Dressler, Falko},
	journal={Communications Magazine, IEEE},
	volume={46},
	number={11},
	pages={132--137},
	year={2008},
	publisher={IEEE}
}




@inproceedings{mahajan2006urban,
	title={Urban mobility models for vanets},
	author={Mahajan, Atulya and Potnis, Niranjan and Gopalan, Kartik and Wang, Andy},
	booktitle={2nd IEEE International Workshop on Next Generation Wireless Networks},
	volume={33},
	year={2006}
}

@inproceedings{Rakhshan2016packet,
	title={Packet success probability derivation in a vehicular ad hoc network for a highway scenario},
	author={Rakhshan, Ali and Pishro-Nik, Hossein},
	booktitle={2016 Annual Conference on Information Science and Systems (CISS)},
	pages={210--215},
	year={2016},
	organization={IEEE}
}

@inproceedings{Rakhshan2016CISS,
	Author = {Rakhshan, Ali and Pishro-Nik, Hossein},
	Booktitle = {Conference on Information Sciences and Systems},
	Organization = {IEEE},
	Pages = {210--215},
	Title = {Packet Success Probability Derivation in a Vehicular Ad Hoc Network for a Highway Scenario},
	Year = {2016}}

@article{Nekoui2013Journal,
	Author = {Nekoui, Mohammad and Pishro-Nik, Hossein},
	Journal = {Journal on Selected Areas in Communications, Special Issue on Emerging Technologies in Communications},
	Number = {9},
	Pages = {491--503},
	Publisher = {IEEE},
	Title = {Analytic Design of Active Safety Systems for Vehicular Ad hoc Networks},
	Volume = {31},
	Year = {2013}}


@inproceedings{Nekoui2011MOBICOM,
	Author = {Nekoui, Mohammad and Pishro-Nik, Hossein},
	Booktitle = {MOBICOM workshop on VehiculAr InterNETworking},
	Organization = {ACM},
	Title = {Analytic Design of Active Vehicular Safety Systems in Sparse Traffic},
	Year = {2011}}

@inproceedings{Nekoui2011VTC,
	Author = {Nekoui, Mohammad and Pishro-Nik, Hossein},
	Booktitle = {VTC-Fall},
	Organization = {IEEE},
	Title = {Analytical Design of Inter-vehicular Communications for Collision Avoidance},
	Year = {2011}}

@inproceedings{Bovee2011VTC,
	Author = {Bovee, Ben Louis and Nekoui, Mohammad and Pishro-Nik, Hossein},
	Booktitle = {VTC-Fall},
	Organization = {IEEE},
	Title = {Evaluation of the Universal Geocast Scheme For VANETs},
	Year = {2011}}

@inproceedings{Nekoui2010MOBICOM,
	Author = {Nekoui, Mohammad and Pishro-Nik, Hossein},
	Booktitle = {MOBICOM},
	Organization = {ACM},
	Title = {Fundamental Tradeoffs in Vehicular Ad Hoc Networks},
	Year = {2010}}

@inproceedings{Nekoui2010IVCS,
	Author = {Nekoui, Mohammad and Pishro-Nik, Hossein},
	Booktitle = {IVCS},
	Organization = {IEEE},
	Title = {A Universal Geocast Scheme for Vehicular Ad Hoc Networks},
	Year = {2010}}

@inproceedings{Nekoui2009ITW,
	Author = {Nekoui, Mohammad and Pishro-Nik, Hossein},
	Booktitle = {IEEE Communications Society Conference on Sensor, Mesh and Ad Hoc Communications and Networks Workshops},
	Organization = {IEEE},
	Pages = {1--3},
	Title = {A Geometrical Analysis of Obstructed Wireless Networks},
	Year = {2009}}

@article{Eslami2013Journal,
	Author = {Eslami, Ali and Nekoui, Mohammad and Pishro-Nik, Hossein and Fekri, Faramarz},
	Journal = {ACM Transactions on Sensor Networks},
	Number = {4},
	Pages = {51},
	Publisher = {ACM},
	Title = {Results on finite wireless sensor networks: Connectivity and coverage},
	Volume = {9},
	Year = {2013}}


@article{Jiafu2014Journal,
	Author = {Jiafu, W. and Zhang, D. and Zhao, S. and Yang, L. and Lloret, J.},
	Journal = {Communications Magazine},
	Number = {8},
	Pages = {106-113},
	Publisher = {IEEE},
	Title = {Context-aware vehicular cyber-physical systems with cloud support: architecture, challenges, and solutions},
	Volume = {52},
	Year = {2014}}

@inproceedings{Haas2010ACM,
	Author = {Haas, J.J. and Hu, Y.},
	Booktitle = {international workshop on VehiculAr InterNETworking},
	Organization = {ACM},
	Title = {Communication requirements for crash avoidance.},
	Year = {2010}}

@inproceedings{Yi2008GLOBECOM,
	Author = {Yi, Q. and Lu, K. and Moayeri, N.},
	Booktitle = {GLOBECOM},
	Organization = {IEEE},
	Title = {CA Secure VANET MAC Protocol for DSRC Applications.},
	Year = {2008}}

@inproceedings{Mughal2010ITSim,
	Author = {Mughal, B.M. and Wagan, A. and Hasbullah, H.},
	Booktitle = {International Symposium on Information Technology (ITSim)},
	Organization = {IEEE},
	Title = {Efficient congestion control in VANET for safety messaging.},
	Year = {2010}}

@article{Chang2011Journal,
	Author = {Chang, Y. and Lee, C. and Copeland, J.},
	Journal = {Selected Areas in Communications},
	Pages = {236 –249},
	Publisher = {IEEE},
	Title = {Goodput enhancement of VANETs in noisy CSMA/CA channels},
	Volume = {29},
	Year = {2011}}

@article{Garcia-Costa2011Journal,
	Author = {Garcia-Costa, C. and Egea-Lopez, E. and Tomas-Gabarron, J. B. and Garcia-Haro, J. and Haas, Z. J.},
	Journal = {Transactions on Intelligent Transportation Systems},
	Pages = {1 –16},
	Publisher = {IEEE},
	Title = {A stochastic model for chain collisions of vehicles equipped with vehicular communications},
	Volume = {99},
	Year = {2011}}

@article{Carbaugh2011Journal,
	Author = {Carbaugh, J. and Godbole,  D. N. and Sengupta, R. and Garcia-Haro, J. and Haas, Z. J.},
	Publisher = {Transportation Research Part C (Emerging Technologies)},
	Title = {Safety and capacity analysis of automated and manual highway systems},
	Year = {1997}}

@article{Goh2004Journal,
	Author = {Goh, P. and Wong, Y.},
	Publisher = {Appl Health Econ Health Policy},
	Title = {Driver perception response time during the signal change interval},
	Year = {2004}}

@article{Chang1985Journal,
	Author = {Chang, M.S. and Santiago, A.J.},
	Pages = {20-30},
	Publisher = {Transportation Research Record},
	Title = {Timing traffic signal changes based on driver behavior},
	Volume = {1027},
	Year = {1985}}

@article{Green2000Journal,
	Author = {Green, M.},
	Pages = {195-216},
	Publisher = {Transportation Human Factors},
	Title = {How long does it take to stop? Methodological analysis of driver perception-brake times.},
	Year = {2000}}

@article{Koppa2005,
	Author = {Koppa, R.J.},
	Pages = {195-216},
	Publisher = {http://www.fhwa.dot.gov/publications/},
	Title = {Human Factors},
	Year = {2005}}

@inproceedings{Maxwell2010ETC,
	Author = {Maxwell, A. and Wood, K.},
	Booktitle = {Europian Transport Conference},
	Organization = {http://www.etcproceedings.org/paper/review-of-traffic-signals-on-high-speed-roads},
	Title = {Review of Traffic Signals on High Speed Road},
	Year = {2010}}

@article{Wortman1983,
	Author = {Wortman, R.H. and Matthias, J.S.},
	Publisher = {Arizona Department of Transportation},
	Title = {An Evaluation of Driver Behavior at Signalized Intersections},
	Year = {1983}}
@inproceedings{Zhang2007IASTED,
	Author = {Zhang, X. and Bham, G.H.},
	Booktitle = {18th IASTED International Conference: modeling and simulation},
	Title = {Estimation of driver reaction time from detailed vehicle trajectory data.},
	Year = {2007}}


@inproceedings{bai2003important,
	title={IMPORTANT: A framework to systematically analyze the Impact of Mobility on Performance of RouTing protocols for Adhoc NeTworks},
	author={Bai, Fan and Sadagopan, Narayanan and Helmy, Ahmed},
	booktitle={INFOCOM 2003. Twenty-second annual joint conference of the IEEE computer and communications. IEEE societies},
	volume={2},
	pages={825--835},
	year={2003},
	organization={IEEE}
}


@inproceedings{abedi2008enhancing,
	title={Enhancing AODV routing protocol using mobility parameters in VANET},
	author={Abedi, Omid and Fathy, Mahmood and Taghiloo, Jamshid},
	booktitle={Computer Systems and Applications, 2008. AICCSA 2008. IEEE/ACS International Conference on},
	pages={229--235},
	year={2008},
	organization={IEEE}
}


@article{AlSultan2013Journal,
	Author = {Al-Sultan, Saif and Al-Bayatti, Ali H. and Zedan, Hussien},
	Journal = {IEEE Transactions on Vehicular Technology},
	Number = {9},
	Pages = {4264-4275},
	Publisher = {IEEE},
	Title = {Context Aware Driver Behaviour Detection System in Intelligent Transportation Systems},
	Volume = {62},
	Year = {2013}}






@article{Leow2008ITS,
	Author = {Leow, Woei Ling and Ni, Daiheng and Pishro-Nik, Hossein},
	Journal = {IEEE Transactions on Intelligent Transportation Systems},
	Number = {2},
	Pages = {369--374},
	Publisher = {IEEE},
	Title = {A Sampling Theorem Approach to Traffic Sensor Optimization},
	Volume = {9},
	Year = {2008}}



@article{REU2007,
	Author = {D. Ni and H. Pishro-Nik and R. Prasad and M. R. Kanjee and H. Zhu and T. Nguyen},
	Journal = {in 14th World Congress on Intelligent Transport Systems},
	Title = {Development of a prototype intersection collision avoidance system under VII},
	Year = {2007}}




@inproceedings{salamatian2013hide,
	title={How to hide the elephant-or the donkey-in the room: Practical privacy against statistical inference for large data.},
	author={Salamatian, Salman and Zhang, Amy and du Pin Calmon, Flavio and Bhamidipati, Sandilya and Fawaz, Nadia and Kveton, Branislav and Oliveira, Pedro and Taft, Nina},
	booktitle={GlobalSIP},
	pages={269--272},
	year={2013}
}

@article{sankar2013utility,
	title={Utility-privacy tradeoffs in databases: An information-theoretic approach},
	author={Sankar, Lalitha and Rajagopalan, S Raj and Poor, H Vincent},
	journal={Information Forensics and Security, IEEE Transactions on},
	volume={8},
	number={6},
	pages={838--852},
	year={2013},
	publisher={IEEE}
}
@inproceedings{ghinita2007prive,
	title={PRIVE: anonymous location-based queries in distributed mobile systems},
	author={Ghinita, Gabriel and Kalnis, Panos and Skiadopoulos, Spiros},
	booktitle={Proceedings of the 16th international conference on World Wide Web},
	pages={371--380},
	year={2007},
	organization={ACM}
}

@article{beresford2004mix,
	title={Mix zones: User privacy in location-aware services},
	author={Beresford, Alastair R and Stajano, Frank},
	year={2004},
	publisher={IEEE}
}

%@inproceedings{Mont1610Achieving,
	%  title={Achieving Perfect Location Privacy in Markov Models Using Anonymization},
	%  author={Montazeri, Zarrin and Houmansadr, Amir and H.Pishro-Nik},
	%  booktitle="2016 International Symposium on Information Theory and its Applications
	%  (ISITA2016)",
	%  address="Monterey, USA",
	%  days=30,
	%  month=oct,
	%  year=2016,
	%}

@article{csiszar1996almost,
	title={Almost independence and secrecy capacity},
	author={Csisz{\'a}r, Imre},
	journal={Problemy Peredachi Informatsii},
	volume={32},
	number={1},
	pages={48--57},
	year={1996},
	publisher={Russian Academy of Sciences, Branch of Informatics, Computer Equipment and Automatization}
}

@article{yamamoto1983source,
	title={A source coding problem for sources with additional outputs to keep secret from the receiver or wiretappers (corresp.)},
	author={Yamamoto, Hirosuke},
	journal={IEEE Transactions on Information Theory},
	volume={29},
	number={6},
	pages={918--923},
	year={1983},
	publisher={IEEE}
}


@inproceedings{calmon2015fundamental,
	title={Fundamental limits of perfect privacy},
	author={Calmon, Flavio P and Makhdoumi, Ali and M{\'e}dard, Muriel},
	booktitle={Information Theory (ISIT), 2015 IEEE International Symposium on},
	pages={1796--1800},
	year={2015},
	organization={IEEE}
}



@inproceedings{Lehman1999Large-Sample-Theory,
	title={Elements of Large Sample Theory},
	author={E. L. Lehman},
	organization={Springer},
	year={1999}
}


@inproceedings{Ferguson1999Large-Sample-Theory,
	title={A Course in Large Sample Theory},
	author={Thomas S. Ferguson},
	organization={CRC Press},
	year={1996}
}



@inproceedings{Dembo1999Large-Deviations,
	title={Large Deviation Techniques and Applications, Second Edition},
	author={A. Dembo and O. Zeitouni},
	organization={Springer},
	year={1998}
}


%%%%%%%%%%%%%%%%%%%%%%%%%%%%%%%%%%%%%%%%%%%%%%%%
Hossein's Coding Journals
%%%%%%%%%%%%%%%%%%%%%%

@ARTICLE{myoptics,
	AUTHOR =       "H. Pishro-Nik and N. Rahnavard and J. Ha and F. Fekri and A. Adibi ",
	TITLE =        "Low-density parity-check codes for volume holographic memory systems",
	JOURNAL =      " Appl. Opt.",
	YEAR =         "2003",
	volume =       "42",
	pages =        "861-870  "
}






@ARTICLE{myit,
	AUTHOR =       "H. Pishro-Nik and F. Fekri  ",
	TITLE =        "On Decoding of Low-Density Parity-Check Codes on the Binary Erasure Channel",
	JOURNAL =      "IEEE Trans. Inform. Theory",
	YEAR =         "2004",
	volume =       "50",
	pages =        "439--454"
}




@ARTICLE{myitpuncture,
	AUTHOR =       "H. Pishro-Nik and F. Fekri  ",
	TITLE =        "Results on Punctured Low-Density Parity-Check Codes and Improved Iterative Decoding Techniques",
	JOURNAL =      "IEEE Trans. on Inform. Theory",
	YEAR =         "2007",
	volume =       "53",
	number=        "2",
	pages =        "599--614",
	month= "February"
}




@ARTICLE{myitlinmimdist,
	AUTHOR =       "H. Pishro-Nik and F. Fekri",
	TITLE =        "Performance of Low-Density Parity-Check Codes With Linear Minimum Distance",
	JOURNAL =         "IEEE Trans. Inform. Theory ",
	YEAR =         "2006",
	volume =       "52",
	number="1",
	pages =        "292 --300"
}






@ARTICLE{myitnonuni,
	AUTHOR =       "H. Pishro-Nik and N. Rahnavard and F. Fekri  ",
	TITLE =        "Non-uniform Error Correction Using Low-Density Parity-Check Codes",
	JOURNAL =      "IEEE Trans. Inform. Theory",
	YEAR =         "2005",
	volume =       "51",
	number=  "7",
	pages =        "2702--2714"
}





@article{eslamitcomhybrid10,
	author = {A. Eslami and S. Vangala and H. Pishro-Nik},
	title = {Hybrid channel codes for highly efficient FSO/RF communication systems},
	journal = {IEEE Transactions on Communications},
	volume = {58},
	number = {10},
	year = {2010},
	pages = {2926--2938},
}


@article{eslamitcompolar13,
	author = {A. Eslami and H. Pishro-Nik},
	title = {On Finite-Length Performance of Polar Codes: Stopping Sets, Error Floor, and Concatenated Design},
	journal = {IEEE Transactions on Communications},
	volume = {61},
	number = {13},
	year = {2013},
	pages = {919--929},
}



@article{saeeditcom11,
	author = {H. Saeedi and H. Pishro-Nik and  A. H. Banihashemi},
	title = {Successive maximization for the systematic design of universally capacity approaching rate-compatible
	sequences of LDPC code ensembles over binary-input output-symmetric memoryless channels},
	journal = {IEEE Transactions on Communications},
	year = {2011},
	volume={59},
	number = {7}
}


@article{rahnavard07,
	author = {Rahnavard, N. and Pishro-Nik, H. and Fekri, F.},
	title = {Unequal Error Protection Using Partially Regular LDPC Codes},
	journal = {IEEE Transactions on Communications},
	year = {2007},
	volume = {55},
	number = {3},
	pages = {387 -- 391}
}


@article{hosseinira04,
	author = {H. Pishro-Nik and F. Fekri},
	title = {Irregular repeat-accumulate codes for volume holographic memory systems},
	journal = {Journal of Applied Optics},
	year = {2004},
	volume = {43},
	number = {27},
	pages = {5222--5227},
}


@article{azadeh2015Ephemeralkey,
	author = {A. Sheikholeslami and D. Goeckel and H. Pishro-Nik},
	title = {Jamming Based on an Ephemeral Key to Obtain Everlasting Security in Wireless Environments},
	journal = {IEEE Transactions on Wireless Communications},
	year = {2015},
	volume = {14},
	number = {11},
	pages = {6072--6081},
}


@article{azadeh2014Everlasting,
	author = {A. Sheikholeslami and D. Goeckel and H. Pishro-Nik},
	title = {Everlasting secrecy in disadvantaged wireless environments against sophisticated eavesdroppers},
	journal = {48th Asilomar Conference on Signals, Systems and Computers},
	year = {2014},
	pages = {1994--1998},
}


@article{azadeh2013ISIT,
	author = {A. Sheikholeslami and D. Goeckel and H. Pishro-Nik},
	title = {Artificial intersymbol interference (ISI) to exploit receiver imperfections for secrecy},
	journal = {IEEE International Symposium on Information Theory (ISIT)},
	year = {2013},
}


@article{azadeh2013Jsac,
	author = {A. Sheikholeslami and D. Goeckel and H. Pishro-Nik},
	title = {Jamming Based on an Ephemeral Key to Obtain Everlasting Security in Wireless Environments},
	journal = {IEEE Journal on Selected Areas in Communications},
	year = {2013},
	volume = {31},
	number = {9},
	pages = {1828--1839},
}


@article{azadeh2012Allerton,
	author = {A. Sheikholeslami and D. Goeckel and H. Pishro-Nik},
	title = {Exploiting the non-commutativity of nonlinear operators for information-theoretic security in disadvantaged wireless environments},
	journal = {50th Annual Allerton Conference on Communication, Control, and Computing},
	year = {2012},
	pages = {233--240},
}


@article{azadeh2012Infocom,
	author = {A. Sheikholeslami and D. Goeckel and H. Pishro-Nik},
	title = {Jamming Based on an Ephemeral Key to Obtain Everlasting Security in Wireless Environments},
	journal = {IEEE INFOCOM},
	year = {2012},
	pages = {1179--1187},
}

@article{1corser2016evaluating,
	title={Evaluating Location Privacy in Vehicular Communications and Applications},
	author={Corser, George P and Fu, Huirong and Banihani, Abdelnasser},
	journal={IEEE Transactions on Intelligent Transportation Systems},
	volume={17},
	number={9},
	pages={2658-2667},
	year={2016},
	publisher={IEEE}
}
@article{2zhang2016designing,
	title={On Designing Satisfaction-Ratio-Aware Truthful Incentive Mechanisms for k-Anonymity Location Privacy},
	author={Zhang, Yuan and Tong, Wei and Zhong, Sheng},
	journal={IEEE Transactions on Information Forensics and Security},
	volume={11},
	number={11},
	pages={2528--2541},
	year={2016},
	publisher={IEEE}
}
@article{3li2016privacy,
	title={Privacy-preserving Location Proof for Securing Large-scale Database-driven Cognitive Radio Networks},
	author={Li, Yi and Zhou, Lu and Zhu, Haojin and Sun, Limin},
	journal={IEEE Internet of Things Journal},
	volume={3},
	number={4},
	pages={563-571},
	year={2016},
	publisher={IEEE}
}
@article{4olteanu2016quantifying,
	title={Quantifying Interdependent Privacy Risks with Location Data},
	author={Olteanu, Alexandra-Mihaela and Huguenin, K{\'e}vin and Shokri, Reza and Humbert, Mathias and Hubaux, Jean-Pierre},
	journal={IEEE Transactions on Mobile Computing},
	year={2016},
	volume={PP},
	number={99},
	pages={1-1},
	publisher={IEEE}
}
@article{5yi2016practical,
	title={Practical Approximate k Nearest Neighbor Queries with Location and Query Privacy},
	author={Yi, Xun and Paulet, Russell and Bertino, Elisa and Varadharajan, Vijay},
	journal={IEEE Transactions on Knowledge and Data Engineering},
	volume={28},
	number={6},
	pages={1546--1559},
	year={2016},
	publisher={IEEE}
}
@article{6li2016privacy,
	title={Privacy Leakage of Location Sharing in Mobile Social Networks: Attacks and Defense},
	author={Li, Huaxin and Zhu, Haojin and Du, Suguo and Liang, Xiaohui and Shen, Xuemin},
	journal={IEEE Transactions on Dependable and Secure Computing},
	year={2016},
	volume={PP},
	number={99},
	publisher={IEEE}
}

@article{7murakami2016localization,
	title={Localization Attacks Using Matrix and Tensor Factorization},
	author={Murakami, Takao and Watanabe, Hajime},
	journal={IEEE Transactions on Information Forensics and Security},
	volume={11},
	number={8},
	pages={1647--1660},
	year={2016},
	publisher={IEEE}
}
@article{8zurbaran2015near,
	title={Near-Rand: Noise-based Location Obfuscation Based on Random Neighboring Points},
	author={Zurbaran, Mayra Alejandra and Avila, Karen and Wightman, Pedro and Fernandez, Michael},
	journal={IEEE Latin America Transactions},
	volume={13},
	number={11},
	pages={3661--3667},
	year={2015},
	publisher={IEEE}
}

@article{9tan2014anti,
	title={An anti-tracking source-location privacy protection protocol in wsns based on path extension},
	author={Tan, Wei and Xu, Ke and Wang, Dan},
	journal={IEEE Internet of Things Journal},
	volume={1},
	number={5},
	pages={461--471},
	year={2014},
	publisher={IEEE}
}

@article{10peng2014enhanced,
	title={Enhanced Location Privacy Preserving Scheme in Location-Based Services},
	author={Peng, Tao and Liu, Qin and Wang, Guojun},
	journal={IEEE Systems Journal},
	year={2014},
	volume={PP},
	number={99},
	pages={1-12},
	publisher={IEEE}
}
@article{11dewri2014exploiting,
	title={Exploiting service similarity for privacy in location-based search queries},
	author={Dewri, Rinku and Thurimella, Ramakrisha},
	journal={IEEE Transactions on Parallel and Distributed Systems},
	volume={25},
	number={2},
	pages={374--383},
	year={2014},
	publisher={IEEE}
}

@article{12hwang2014novel,
	title={A novel time-obfuscated algorithm for trajectory privacy protection},
	author={Hwang, Ren-Hung and Hsueh, Yu-Ling and Chung, Hao-Wei},
	journal={IEEE Transactions on Services Computing},
	volume={7},
	number={2},
	pages={126--139},
	year={2014},
	publisher={IEEE}
}
@article{13puttaswamy2014preserving,
	title={Preserving location privacy in geosocial applications},
	author={Puttaswamy, Krishna PN and Wang, Shiyuan and Steinbauer, Troy and Agrawal, Divyakant and El Abbadi, Amr and Kruegel, Christopher and Zhao, Ben Y},
	journal={IEEE Transactions on Mobile Computing},
	volume={13},
	number={1},
	pages={159--173},
	year={2014},
	publisher={IEEE}
}

@article{14zhang2014privacy,
	title={Privacy quantification model based on the Bayes conditional risk in Location-Based Services},
	author={Zhang, Xuejun and Gui, Xiaolin and Tian, Feng and Yu, Si and An, Jian},
	journal={Tsinghua Science and Technology},
	volume={19},
	number={5},
	pages={452--462},
	year={2014},
	publisher={TUP}
}

@article{15bilogrevic2014privacy,
	title={Privacy-preserving optimal meeting location determination on mobile devices},
	author={Bilogrevic, Igor and Jadliwala, Murtuza and Joneja, Vishal and Kalkan, K{\"u}bra and Hubaux, Jean-Pierre and Aad, Imad},
	journal={IEEE transactions on information forensics and security},
	volume={9},
	number={7},
	pages={1141--1156},
	year={2014},
	publisher={IEEE}
}
@article{16haghnegahdar2014privacy,
	title={Privacy Risks in Publishing Mobile Device Trajectories},
	author={Haghnegahdar, Alireza and Khabbazian, Majid and Bhargava, Vijay K},
	journal={IEEE Wireless Communications Letters},
	volume={3},
	number={3},
	pages={241--244},
	year={2014},
	publisher={IEEE}
}
@article{17malandrino2014verification,
	title={Verification and inference of positions in vehicular networks through anonymous beaconing},
	author={Malandrino, Francesco and Borgiattino, Carlo and Casetti, Claudio and Chiasserini, Carla-Fabiana and Fiore, Marco and Sadao, Roberto},
	journal={IEEE Transactions on Mobile Computing},
	volume={13},
	number={10},
	pages={2415--2428},
	year={2014},
	publisher={IEEE}
}
@article{18shokri2014hiding,
	title={Hiding in the mobile crowd: Locationprivacy through collaboration},
	author={Shokri, Reza and Theodorakopoulos, George and Papadimitratos, Panos and Kazemi, Ehsan and Hubaux, Jean-Pierre},
	journal={IEEE transactions on dependable and secure computing},
	volume={11},
	number={3},
	pages={266--279},
	year={2014},
	publisher={IEEE}
}
@article{19freudiger2013non,
	title={Non-cooperative location privacy},
	author={Freudiger, Julien and Manshaei, Mohammad Hossein and Hubaux, Jean-Pierre and Parkes, David C},
	journal={IEEE Transactions on Dependable and Secure Computing},
	volume={10},
	number={2},
	pages={84--98},
	year={2013},
	publisher={IEEE}
}
@article{20gao2013trpf,
	title={TrPF: A trajectory privacy-preserving framework for participatory sensing},
	author={Gao, Sheng and Ma, Jianfeng and Shi, Weisong and Zhan, Guoxing and Sun, Cong},
	journal={IEEE Transactions on Information Forensics and Security},
	volume={8},
	number={6},
	pages={874--887},
	year={2013},
	publisher={IEEE}
}
@article{21ma2013privacy,
	title={Privacy vulnerability of published anonymous mobility traces},
	author={Ma, Chris YT and Yau, David KY and Yip, Nung Kwan and Rao, Nageswara SV},
	journal={IEEE/ACM Transactions on Networking},
	volume={21},
	number={3},
	pages={720--733},
	year={2013},
	publisher={IEEE}
}
@article{22niu2013pseudo,
	title={Pseudo-Location Updating System for privacy-preserving location-based services},
	author={Niu, Ben and Zhu, Xiaoyan and Chi, Haotian and Li, Hui},
	journal={China Communications},
	volume={10},
	number={9},
	pages={1--12},
	year={2013},
	publisher={IEEE}
}
@article{23dewri2013local,
	title={Local differential perturbations: Location privacy under approximate knowledge attackers},
	author={Dewri, Rinku},
	journal={IEEE Transactions on Mobile Computing},
	volume={12},
	number={12},
	pages={2360--2372},
	year={2013},
	publisher={IEEE}
}
@inproceedings{24kanoria2012tractable,
	title={Tractable bayesian social learning on trees},
	author={Kanoria, Yashodhan and Tamuz, Omer},
	booktitle={Information Theory Proceedings (ISIT), 2012 IEEE International Symposium on},
	pages={2721--2725},
	year={2012},
	organization={IEEE}
}
@inproceedings{25farias2005universal,
	title={A universal scheme for learning},
	author={Farias, Vivek F and Moallemi, Ciamac C and Van Roy, Benjamin and Weissman, Tsachy},
	booktitle={Proceedings. International Symposium on Information Theory, 2005. ISIT 2005.},
	pages={1158--1162},
	year={2005},
	organization={IEEE}
}
@inproceedings{26misra2013unsupervised,
	title={Unsupervised learning and universal communication},
	author={Misra, Vinith and Weissman, Tsachy},
	booktitle={Information Theory Proceedings (ISIT), 2013 IEEE International Symposium on},
	pages={261--265},
	year={2013},
	organization={IEEE}
}
@inproceedings{27ryabko2013time,
	title={Time-series information and learning},
	author={Ryabko, Daniil},
	booktitle={Information Theory Proceedings (ISIT), 2013 IEEE International Symposium on},
	pages={1392--1395},
	year={2013},
	organization={IEEE}
}
@inproceedings{28krzakala2013phase,
	title={Phase diagram and approximate message passing for blind calibration and dictionary learning},
	author={Krzakala, Florent and M{\'e}zard, Marc and Zdeborov{\'a}, Lenka},
	booktitle={Information Theory Proceedings (ISIT), 2013 IEEE International Symposium on},
	pages={659--663},
	year={2013},
	organization={IEEE}
}
@inproceedings{29sakata2013sample,
	title={Sample complexity of Bayesian optimal dictionary learning},
	author={Sakata, Ayaka and Kabashima, Yoshiyuki},
	booktitle={Information Theory Proceedings (ISIT), 2013 IEEE International Symposium on},
	pages={669--673},
	year={2013},
	organization={IEEE}
}
@inproceedings{30predd2004consistency,
	title={Consistency in a model for distributed learning with specialists},
	author={Predd, Joel B and Kulkarni, Sanjeev R and Poor, H Vincent},
	booktitle={IEEE International Symposium on Information Theory},
	year={2004},
	organization={IEEE}
}
@inproceedings{31nokleby2016rate,
	title={Rate-Distortion Bounds on Bayes Risk in Supervised Learning},
	author={Nokleby, Matthew and Beirami, Ahmad and Calderbank, Robert},
	booktitle={2016 IEEE International Symposium on Information Theory (ISIT)},
	pages={2099-2103},
	year={2016},
	organization={IEEE}
}

@inproceedings{32le2016imperfect,
	title={Are imperfect reviews helpful in social learning?},
	author={Le, Tho Ngoc and Subramanian, Vijay G and Berry, Randall A},
	booktitle={Information Theory (ISIT), 2016 IEEE International Symposium on},
	pages={2089--2093},
	year={2016},
	organization={IEEE}
}
@inproceedings{33gadde2016active,
	title={Active Learning for Community Detection in Stochastic Block Models},
	author={Gadde, Akshay and Gad, Eyal En and Avestimehr, Salman and Ortega, Antonio},
	booktitle={2016 IEEE International Symposium on Information Theory (ISIT)},
	pages={1889-1893},
	year={2016}
}
@inproceedings{34shakeri2016minimax,
	title={Minimax Lower Bounds for Kronecker-Structured Dictionary Learning},
	author={Shakeri, Zahra and Bajwa, Waheed U and Sarwate, Anand D},
	booktitle={2016 IEEE International Symposium on Information Theory (ISIT)},
	pages={1148-1152},
	year={2016}
}
@article{35lee2015speeding,
	title={Speeding up distributed machine learning using codes},
	author={Lee, Kangwook and Lam, Maximilian and Pedarsani, Ramtin and Papailiopoulos, Dimitris and Ramchandran, Kannan},
	booktitle={2016 IEEE International Symposium on Information Theory (ISIT)},
	pages={1143-1147},
	year={2016}
}
@article{36oneto2016statistical,
	title={Statistical Learning Theory and ELM for Big Social Data Analysis},
	author={Oneto, Luca and Bisio, Federica and Cambria, Erik and Anguita, Davide},
	journal={ieee CompUTATionAl inTelliGenCe mAGAzine},
	volume={11},
	number={3},
	pages={45--55},
	year={2016},
	publisher={IEEE}
}
@article{37lin2015probabilistic,
	title={Probabilistic approach to modeling and parameter learning of indirect drive robots from incomplete data},
	author={Lin, Chung-Yen and Tomizuka, Masayoshi},
	journal={IEEE/ASME Transactions on Mechatronics},
	volume={20},
	number={3},
	pages={1036--1045},
	year={2015},
	publisher={IEEE}
}
@article{38wang2016towards,
	title={Towards Bayesian Deep Learning: A Framework and Some Existing Methods},
	author={Wang, Hao and Yeung, Dit-Yan},
	journal={IEEE Transactions on Knowledge and Data Engineering},
	volume={PP},
	number={99},
	year={2016},
	publisher={IEEE}
}


%%%%%Informationtheoreticsecurity%%%%%%%%%%%%%%%%%%%%%%%




@inproceedings{Bloch2011PhysicalSecBook,
	title={Physical-Layer Security},
	author={M. Bloch and J. Barros},
	organization={Cambridge University Press},
	year={2011}
}



@inproceedings{Liang2009InfoSecBook,
	title={Information Theoretic Security},
	author={Y. Liang and H. V. Poor and S. Shamai (Shitz)},
	organization={Now Publishers Inc.},
	year={2009}
}


@inproceedings{Zhou2013PhysicalSecBook,
	title={Physical Layer Security in Wireless Communications},
	author={ X. Zhou and L. Song and Y. Zhang},
	organization={CRC Press},
	year={2013}
}

@article{Ni2012IEA,
	Author = {D. Ni and H. Liu and W. Ding and  Y. Xie and H. Wang and H. Pishro-Nik and Q. Yu},
	Journal = {IEA/AIE},
	Title = {Cyber-Physical Integration to Connect Vehicles for Transformed Transportation Safety and Efficiency},
	Year = {2012}}



@inproceedings{Ni2012Inproceedings,
	Author = {D. Ni, H. Liu, Y. Xie, W. Ding, H. Wang, H. Pishro-Nik, Q. Yu and M. Ferreira},
	Booktitle = {Spring Simulation Multiconference},
	Date-Added = {2016-09-04 14:18:42 +0000},
	Date-Modified = {2016-09-06 16:22:14 +0000},
	Title = {Virtual Lab of Connected Vehicle Technology},
	Year = {2012}}

@inproceedings{Ni2012Inproceedings,
	Author = {D. Ni, H. Liu, W. Ding, Y. Xie, H. Wang, H. Pishro-Nik and Q. Yu,},
	Booktitle = {IEA/AIE},
	Date-Added = {2016-09-04 09:11:02 +0000},
	Date-Modified = {2016-09-06 14:46:53 +0000},
	Title = {Cyber-Physical Integration to Connect Vehicles for Transformed Transportation Safety and Efficiency},
	Year = {2012}}


@article{Nekoui_IJIPT_2009,
	Author = {M. Nekoui and D. Ni and H. Pishro-Nik and R. Prasad and M. Kanjee and H. Zhu and T. Nguyen},
	Journal = {International Journal of Internet Protocol Technology (IJIPT)},
	Number = {3},
	Pages = {},
	Publisher = {},
	Title = {Development of a VII-Enabled Prototype Intersection Collision Warning System},
	Volume = {4},
	Year = {2009}}


@inproceedings{Pishro_Ganz_Ni,
	Author = {H. Pishro-Nik, A. Ganz, and Daiheng Ni},
	Booktitle = {Forty-Fifth Annual Allerton Conference on Communication, Control, and Computing. Allerton House, Monticello, IL},
	Date-Added = {},
	Date-Modified = {},
	Number = {},
	Pages = {},
	Title = {The capacity of vehicular ad hoc networks},
	Volume = {},
	Year = {September 26-28, 2007}}

@inproceedings{Leow_Pishro_Ni_1,
	Author = {W. L. Leow, H. Pishro-Nik and Daiheng Ni},
	Booktitle = {IEEE Global Telecommunications Conference, Washington, D.C.},
	Date-Added = {},
	Date-Modified = {},
	Number = {},
	Pages = {},
	Title = {Delay and Energy Tradeoff in Multi-state Wireless Sensor Networks},
	Volume = {},
	Year = {November 26-30, 2007}}


@misc{UMass-Trans,
	title = {{UMass Transportation Center}},
	note = {\url{http://www.umasstransportationcenter.org/}},
}


@inproceedings{Haenggi2013book,
	title={Stochastic geometry for wireless networks},
	author={M. Haenggi},
	organization={Cambridge Uinversity Press},
	year={2013}
}


%%%%%%%%%%%%%%%%personalization%%%%%%%%%%%%%%%%%%%%%%%%%%%%%%%%%%

@article{osma2015,
	title={Impact of Time-to-Collision Information on Driving Behavior in Connected Vehicle Environments Using A Driving Simulator Test Bed},
	journal{Journal of Traffic and Logistics Engineering},
	author={Osama A. Osman, Julius Codjoe, and Sherif Ishak},
	volume={3},
	number={1},
	pages={18--24},
	year={2015}
}


@article{charisma2010,
	title={Dynamic Latent Plan Models},
	author={Charisma F. Choudhurya, Moshe Ben-Akivab and Maya Abou-Zeid},
	journal={Journal of Choice Modelling},
	volume={3},
	number={2},
	pages={50--70},
	year={2010},
	publisher={Elsvier}
}


@misc{noble2014,
	author = {A. M. Noble, Shane B. McLaughlin, Zachary R. Doerzaph and Thomas A. Dingus},
	title = {Crowd-sourced Connected-vehicle Warning Algorithm using Naturalistic Driving Data},
	howpublished = {Downloaded from \url{http://hdl.handle.net/10919/53978}},
	
	month = August,
	year = 2014
}


@phdthesis{charisma2007,
	title    = {Modeling Driving Decisions with Latent Plans},
	school   = {Massachusetts Institute of Technology },
	author   = {Charisma Farheen Choudhury},
	year     = {2007}, %other attributes omitted
}


@article{chrysler2015,
	title={Cost of Warning of Unseen Threats:Unintended Consequences of Connected Vehicle Alerts},
	author={S. T. Chrysler, J. M. Cooper and D. C. Marshall},
	journal={Transportation Research Record: Journal of the Transportation Research Board},
	volume={2518},
	pages={79--85},
	year={2015},
}

@misc{nsf_cps,
	title = {Cyber-Physical Systems (CPS) PROGRAM SOLICITATION NSF 17-529},
	howpublished = {Downloaded from \url{https://www.nsf.gov/publications/pub_summ.jsp?WT.z_pims_id=503286&ods_key=nsf17529}},
}



%%%%%%%%%%%%%%IOT%%%%%%%%%%%%%%%%%%%%%%%%%%%%%%%%%%%%%%%%%%%%%%%%%%%




@article{FTC2015,
	title={Internet of Things: Privacy and Security in a Connected World},
	author={FTC Staff Report},
	year={2015}
}



%% Saved with string encoding Unicode (UTF-8)
@inproceedings{1zhou2014security,
	title={Security/privacy of wearable fitness tracking {I}o{T} devices},
	author={Zhou, Wei and Piramuthu, Selwyn},
	booktitle={Information Systems and Technologies (CISTI), 2014 9th Iberian Conference on},
	pages={1--5},
	year={2014},
	organization={IEEE}
}

@article{2nia2016comprehensive,
	title={A comprehensive study of security of internet-of-things},
	author={Mohsenia, Arsalan and Jha, Niraj K},
	journal={IEEE Transactions on Emerging Topics in Computing},
	volum={5},
	number={4},
	pages={586--602},
	year={2017},
	publisher={IEEE}
}

@inproceedings{3ukil2014iot,
	title={{I}o{T}-privacy: To be private or not to be private},
	author={Ukil, Arijit and Bandyopadhyay, Soma and Pal, Arpan},
	booktitle={Computer Communications Workshops (INFOCOM WKSHPS), IEEE Conference on},
	pages={123--124},
	year={2014},
	organization={IEEE}
}

@article{4arias2015privacy,
	title={Privacy and security in internet of things and wearable devices},
	author={Arias, Orlando and Wurm, Jacob and Hoang, Khoa and Jin, Yier},
	journal={IEEE Transactions on Multi-Scale Computing Systems},
	volume={1},
	number={2},
	pages={99--109},
	year={2015},
	publisher={IEEE}
}
@inproceedings{5ullah2016novel,
	title={A novel model for preserving Location Privacy in Internet of Things},
	author={Ullah, Ikram and Shah, Munam Ali},
	booktitle={Automation and Computing (ICAC), 2016 22nd International Conference on},
	pages={542--547},
	year={2016},
	organization={IEEE}
}
@inproceedings{6sathishkumar2016enhanced,
	title={Enhanced location privacy algorithm for wireless sensor network in Internet of Things},
	author={Sathishkumar, J and Patel, Dhiren R},
	booktitle={Internet of Things and Applications (IOTA), International Conference on},
	pages={208--212},
	year={2016},
	organization={IEEE}
}
@inproceedings{7zhou2012preserving,
	title={Preserving sensor location privacy in internet of things},
	author={Zhou, Liming and Wen, Qiaoyan and Zhang, Hua},
	booktitle={Computational and Information Sciences (ICCIS), 2012 Fourth International Conference on},
	pages={856--859},
	year={2012},
	organization={IEEE}
}

@inproceedings{8ukil2015privacy,
	title={Privacy for {I}o{T}: Involuntary privacy enablement for smart energy systems},
	author={Ukil, Arijit and Bandyopadhyay, Soma and Pal, Arpan},
	booktitle={Communications (ICC), 2015 IEEE International Conference on},
	pages={536--541},
	year={2015},
	organization={IEEE}
}

@inproceedings{9dalipi2016security,
	title={Security and Privacy Considerations for {I}o{T} Application on Smart Grids: Survey and Research Challenges},
	author={Dalipi, Fisnik and Yayilgan, Sule Yildirim},
	booktitle={Future Internet of Things and Cloud Workshops (FiCloudW), IEEE International Conference on},
	pages={63--68},
	year={2016},
	organization={IEEE}
}
@inproceedings{10harris2016security,
	title={Security and Privacy in Public {I}o{T} Spaces},
	author={Harris, Albert F and Sundaram, Hari and Kravets, Robin},
	booktitle={Computer Communication and Networks (ICCCN), 2016 25th International Conference on},
	pages={1--8},
	year={2016},
	organization={IEEE}
}

@inproceedings{11al2015security,
	title={Security and privacy framework for ubiquitous healthcare {I}o{T} devices},
	author={Al Alkeem, Ebrahim and Yeun, Chan Yeob and Zemerly, M Jamal},
	booktitle={Internet Technology and Secured Transactions (ICITST), 2015 10th International Conference for},
	pages={70--75},
	year={2015},
	organization={IEEE}
}
@inproceedings{12sivaraman2015network,
	title={Network-level security and privacy control for smart-home {I}o{T} devices},
	author={Sivaraman, Vijay and Gharakheili, Hassan Habibi and Vishwanath, Arun and Boreli, Roksana and Mehani, Olivier},
	booktitle={Wireless and Mobile Computing, Networking and Communications (WiMob), 2015 IEEE 11th International Conference on},
	pages={163--167},
	year={2015},
	organization={IEEE}
}

@inproceedings{13srinivasan2016privacy,
	title={Privacy conscious architecture for improving emergency response in smart cities},
	author={Srinivasan, Ramya and Mohan, Apurva and Srinivasan, Priyanka},
	booktitle={Smart City Security and Privacy Workshop (SCSP-W), 2016},
	pages={1--5},
	year={2016},
	organization={IEEE}
}
@inproceedings{14sadeghi2015security,
	title={Security and privacy challenges in industrial internet of things},
	author={Sadeghi, Ahmad-Reza and Wachsmann, Christian and Waidner, Michael},
	booktitle={Design Automation Conference (DAC), 2015 52nd ACM/EDAC/IEEE},
	pages={1--6},
	year={2015},
	organization={IEEE}
}
@inproceedings{15otgonbayar2016toward,
	title={Toward Anonymizing {I}o{T} Data Streams via Partitioning},
	author={Otgonbayar, Ankhbayar and Pervez, Zeeshan and Dahal, Keshav},
	booktitle={Mobile Ad Hoc and Sensor Systems (MASS), 2016 IEEE 13th International Conference on},
	pages={331--336},
	year={2016},
	organization={IEEE}
}
@inproceedings{16rutledge2016privacy,
	title={Privacy Impacts of {I}o{T} Devices: A SmartTV Case Study},
	author={Rutledge, Richard L and Massey, Aaron K and Ant{\'o}n, Annie I},
	booktitle={Requirements Engineering Conference Workshops (REW), IEEE International},
	pages={261--270},
	year={2016},
	organization={IEEE}
}

@inproceedings{17andrea2015internet,
	title={Internet of Things: Security vulnerabilities and challenges},
	author={Andrea, Ioannis and Chrysostomou, Chrysostomos and Hadjichristofi, George},
	booktitle={Computers and Communication (ISCC), 2015 IEEE Symposium on},
	pages={180--187},
	year={2015},
	organization={IEEE}
}






























%%%%%%%%%%%%%%%%%%%%%%%%%%%%%%%%%%%%%%%%%%%%%%%%%%%%%%%%%%%


@misc{epfl-mobility-20090224,
	author = {Michal Piorkowski and Natasa Sarafijanovic-Djukic and Matthias Grossglauser},
	title = {{CRAWDAD} dataset epfl/mobility (v. 2009-02-24)},
	howpublished = {Downloaded from \url{http://crawdad.org/epfl/mobility/20090224}},
	doi = {10.15783/C7J010},
	month = feb,
	year = 2009
}

@misc{roma-taxi-20140717,
	author = {Lorenzo Bracciale and Marco Bonola and Pierpaolo Loreti and Giuseppe Bianchi and Raul Amici and Antonello Rabuffi},
	title = {{CRAWDAD} dataset roma/taxi (v. 2014-07-17)},
	howpublished = {Downloaded from \url{http://crawdad.org/roma/taxi/20140717}},
	doi = {10.15783/C7QC7M},
	month = jul,
	year = 2014
}

@misc{rice-ad_hoc_city-20030911,
	author = {Jorjeta G. Jetcheva and Yih-Chun Hu and Santashil PalChaudhuri and Amit Kumar Saha and David B. Johnson},
	title = {{CRAWDAD} dataset rice/ad\_hoc\_city (v. 2003-09-11)},
	howpublished = {Downloaded from \url{http://crawdad.org/rice/ad_hoc_city/20030911}},
	doi = {10.15783/C73K5B},
	month = sep,
	year = 2003
}

@misc{china:2012,
	author = {Microsoft Research Asia},
	title = {GeoLife GPS Trajectories},
	year = {2012},
	howpublished= {\url{https://www.microsoft.com/en-us/download/details.aspx?id=52367}},
}


@misc{china:2011,
	ALTauthor = {Microsoft Research Asia)},
	ALTeditor = {},
	title = {GeoLife GPS Trajectories,
	year = {2012},
	url = {https://www.microsoft.com/en-us/download/details.aspx?id=52367},
	}
	
	
	@misc{longversion,
	author = {N. Takbiri and A. Houmansadr and D.L. Goeckel and H. Pishro-Nik},
	title = {{Limits of Location Privacy under Anonymization and Obfuscation}},
	howpublished = "\url{http://www.ecs.umass.edu/ece/pishro/Papers/ISIT_2017-2.pdf}",
	year = 2017,
	month= "January",
	note = "Summarized version submitted to IEEE ISIT 2017"
	}
	
	@misc{isit_ke,
	author = {K. Li and D. Goeckel and H. Pishro-Nik},
	title = {{Bayesian Time Series Matching and Privacy}},
	note = "submitted to IEEE ISIT 2017"
	}
	
	@article{matching,
	title={Asymptotically Optimal Matching of Multiple Sequences to Source Distributions and Training Sequences},
	author={Jayakrishnan Unnikrishnan},
	journal={ IEEE Transactions on Information Theory},
	volume={61},
	number={1},
	pages={452-468},
	year={2015},
	publisher={IEEE}
	}
	
	
	@article{Naini2016,
	Author = {F. Naini and J. Unnikrishnan and P. Thiran and M. Vetterli},
	Journal = {IEEE Transactions on Information Forensics and Security},
	Publisher = {IEEE},
	Title = {Where You Are Is Who You Are: User Identification by Matching Statistics},
	volume={11},
	number={2},
	pages={358--372},
	Year = {2016}
	}
	
	
	
	@inproceedings{holowczak2015cachebrowser,
	title={{CacheBrowser: Bypassing Chinese Censorship without Proxies Using Cached Content}},
	author={Holowczak, John and Houmansadr, Amir},
	booktitle={Proceedings of the 22nd ACM SIGSAC Conference on Computer and Communications Security},
	pages={70--83},
	year={2015},
	organization={ACM}
	}
	@misc{cb-website,
	Howpublished = {\url{https://cachebrowser.net/#/}},
	Title = {{CacheBrowser}},
	key={cachebrowser}
	}
	
	@inproceedings{GameOfDecoys,
	title={{GAME OF DECOYS: Optimal Decoy Routing Through Game Theory}},
	author={Milad Nasr and Amir Houmansadr},
	booktitle={The $23^{rd}$ ACM Conference on Computer and Communications Security (CCS)},
	year={2016}
	}
	
	@inproceedings{CDNReaper,
	title={{Practical Censorship Evasion Leveraging Content Delivery Networks}},
	author={Hadi Zolfaghari and Amir Houmansadr},
	booktitle={The $23^{rd}$ ACM Conference on Computer and Communications Security (CCS)},
	year={2016}
	}
	
	@misc{Leberknight2010,
	Author = {Leberknight, C. and Chiang, M. and Poor, H. and Wong, F.},
	Howpublished = {\url{http://www.princeton.edu/~chiangm/anticensorship.pdf}},
	Title = {{A Taxonomy of Internet Censorship and Anti-censorship}},
	Year = {2010}}
	
	@techreport{ultrasurf-analysis,
	Author = {Appelbaum, Jacob},
	Institution = {The Tor Project},
	Title = {{Technical analysis of the Ultrasurf proxying software}},
	Url = {http://scholar.google.com/scholar?hl=en\&btnG=Search\&q=intitle:Technical+analysis+of+the+Ultrasurf+proxying+software\#0},
	Year = {2012},
	Bdsk-Url-1 = {http://scholar.google.com/scholar?hl=en%5C&btnG=Search%5C&q=intitle:Technical+analysis+of+the+Ultrasurf+proxying+software%5C#0}}
	
	@misc{gifc:07,
	Howpublished = {\url{http://www.internetfreedom.org/archive/Defeat\_Internet\_Censorship\_White\_Paper.pdf}},
	Key = {defeatcensorship},
	Publisher = {Global Internet Freedom Consortium (GIFC)},
	Title = {{Defeat Internet Censorship: Overview of Advanced Technologies and Products}},
	Type = {White Paper},
	Year = {2007}}
	
	@article{pan2011survey,
	Author = {Pan, J. and Paul, S. and Jain, R.},
	Journal = {Communications Magazine, IEEE},
	Number = {7},
	Pages = {26--36},
	Publisher = {IEEE},
	Title = {{A Survey of the Research on Future Internet Architectures}},
	Volume = {49},
	Year = {2011}}
	
	@misc{nsf-fia,
	Howpublished = {\url{http://www.nets-fia.net/}},
	Key = {FIA},
	Title = {{NSF Future Internet Architecture Project}}}
	
	@misc{NDN,
	Howpublished = {\url{http://www.named- data.net}},
	Key = {NDN},
	Title = {{Named Data Networking Project}}}
	
	@inproceedings{MobilityFirst,
	Author = {Seskar, I. and Nagaraja, K. and Nelson, S. and Raychaudhuri, D.},
	Booktitle = {Asian Internet Engineering Conference},
	Title = {{Mobilityfirst Future internet Architecture Project}},
	Year = {2011}}
	
	@incollection{NEBULA,
	Author = {Anderson, T. and Birman, K. and Broberg, R. and Caesar, M. and Comer, D. and Cotton, C. and Freedman, M.~J. and Haeberlen, A. and Ives, Z.~G. and Krishnamurthy, A. and others},
	Booktitle = {The Future Internet},
	Pages = {16--26},
	Publisher = {Springer},
	Title = {{The NEBULA Future Internet Architecture}},
	Year = {2013}}
	
	@inproceedings{XIA,
	Author = {Anand, A. and Dogar, F. and Han, D. and Li, B. and Lim, H. and Machado, M. and Wu, W. and Akella, A. and Andersen, D.~G. and Byers, J.~W. and others},
	Booktitle = {ACM Workshop on Hot Topics in Networks},
	Pages = {2},
	Title = {{XIA: An Architecture for an Evolvable and Trustworthy Internet}},
	Year = {2011}}
	
	@inproceedings{ChoiceNet,
	Author = {Rouskas, G.~N. and Baldine, I. and Calvert, K.~L. and Dutta, R. and Griffioen, J. and Nagurney, A. and Wolf, T.},
	Booktitle = {ONDM},
	Title = {{ChoiceNet: Network Innovation Through Choice}},
	Year = {2013}}
	
	@misc{nsf-find,
	Howpublished = {http://www.nets-find.net/},
	Title = {{NSF NeTS FIND Initiative}}}
	
	@article{traid,
	Author = {Cheriton, D.~R. and Gritter, M.},
	Title = {{TRIAD: A New Next-Generation Internet Architecture}},
	Year = {2000}}
	
	@inproceedings{dona,
	Author = {Koponen, T. and Chawla, M. and Chun, B-G. and Ermolinskiy, A. and Kim, K.~H. and Shenker, S. and Stoica, I.},
	Booktitle = {ACM SIGCOMM Computer Communication Review},
	Number = {4},
	Organization = {ACM},
	Pages = {181--192},
	Title = {{A Data-Oriented (and Beyond) Network Architecture}},
	Volume = {37},
	Year = {2007}}
	
	@misc{ultrasurf,
	Howpublished = {\url{http://www.ultrareach.com}},
	Key = {ultrasurf},
	Title = {{Ultrasurf}}}
	
	@misc{tor-bridge,
	Author = {Dingledine, R. and Mathewson, N.},
	Howpublished = {\url{https://svn.torproject.org/svn/projects/design-paper/blocking.html}},
	Title = {{Design of a Blocking-Resistant Anonymity System}}}
	
	@inproceedings{McLachlanH09,
	Author = {J. McLachlan and N. Hopper},
	Booktitle = {WPES},
	Title = {{On the Risks of Serving Whenever You Surf: Vulnerabilities in Tor's Blocking Resistance Design}},
	Year = {2009}}
	
	@inproceedings{mahdian2010,
	Author = {Mahdian, M.},
	Booktitle = {{Fun with Algorithms}},
	Title = {{Fighting Censorship with Algorithms}},
	Year = {2010}}
	
	@inproceedings{McCoy2011,
	Author = {McCoy, D. and Morales, J.~A. and Levchenko, K.},
	Booktitle = {FC},
	Title = {{Proximax: A Measurement Based System for Proxies Dissemination}},
	Year = {2011}}
	
	@inproceedings{Sovran2008,
	Author = {Sovran, Y. and Libonati, A. and Li, J.},
	Booktitle = {IPTPS},
	Title = {{Pass it on: Social Networks Stymie Censors}},
	Year = {2008}}
	
	@inproceedings{rbridge,
	Author = {Wang, Q. and Lin, Zi and Borisov, N. and Hopper, N.},
	Booktitle = {{NDSS}},
	Title = {{rBridge: User Reputation based Tor Bridge Distribution with Privacy Preservation}},
	Year = {2013}}
	
	@inproceedings{telex,
	Author = {Wustrow, E. and Wolchok, S. and Goldberg, I. and Halderman, J.},
	Booktitle = {{USENIX Security}},
	Title = {{Telex: Anticensorship in the Network Infrastructure}},
	Year = {2011}}
	
	@inproceedings{cirripede,
	Author = {Houmansadr, A. and Nguyen, G. and Caesar, M. and Borisov, N.},
	Booktitle = {CCS},
	Title = {{Cirripede: Circumvention Infrastructure Using Router Redirection with Plausible Deniability}},
	Year = {2011}}
	
	@inproceedings{decoyrouting,
	Author = {Karlin, J. and Ellard, D. and Jackson, A. and Jones, C. and Lauer, G. and Mankins, D. and Strayer, W.},
	Booktitle = {{FOCI}},
	Title = {{Decoy Routing: Toward Unblockable Internet Communication}},
	Year = {2011}}
	
	@inproceedings{routing-around-decoys,
	Author = {M.~Schuchard and J.~Geddes and C.~Thompson and N.~Hopper},
	Booktitle = {{CCS}},
	Title = {{Routing Around Decoys}},
	Year = {2012}}
	
	@inproceedings{parrot,
	Author = {A. Houmansadr and C. Brubaker and V. Shmatikov},
	Booktitle = {IEEE S\&P},
	Title = {{The Parrot is Dead: Observing Unobservable Network Communications}},
	Year = {2013}}
	
	@misc{knock,
	Author = {T. Wilde},
	Howpublished = {\url{https://blog.torproject.org/blog/knock-knock-knockin-bridges-doors}},
	Title = {{Knock Knock Knockin' on Bridges' Doors}},
	Year = {2012}}
	
	@inproceedings{china-tor,
	Author = {Winter, P. and Lindskog, S.},
	Booktitle = {{FOCI}},
	Title = {{How the Great Firewall of China Is Blocking Tor}},
	Year = {2012}}
	
	@misc{discover-bridge,
	Howpublished = {\url{https://blog.torproject.org/blog/research-problems-ten-ways-discover-tor-bridges}},
	Key = {tenways},
	Title = {{Ten Ways to Discover Tor Bridges}}}
	
	@inproceedings{freewave,
	Author = {A.~Houmansadr and T.~Riedl and N.~Borisov and A.~Singer},
	Booktitle = {{NDSS}},
	Title = {{I Want My Voice to Be Heard: IP over Voice-over-IP for Unobservable Censorship Circumvention}},
	Year = 2013}
	
	@inproceedings{censorspoofer,
	Author = {Q. Wang and X. Gong and G. Nguyen and A. Houmansadr and N. Borisov},
	Booktitle = {CCS},
	Title = {{CensorSpoofer: Asymmetric Communication Using IP Spoofing for Censorship-Resistant Web Browsing}},
	Year = {2012}}
	
	@inproceedings{skypemorph,
	Author = {H. Moghaddam and B. Li and M. Derakhshani and I. Goldberg},
	Booktitle = {CCS},
	Title = {{SkypeMorph: Protocol Obfuscation for Tor Bridges}},
	Year = {2012}}
	
	@inproceedings{stegotorus,
	Author = {Weinberg, Z. and Wang, J. and Yegneswaran, V. and Briesemeister, L. and Cheung, S. and Wang, F. and Boneh, D.},
	Booktitle = {CCS},
	Title = {{StegoTorus: A Camouflage Proxy for the Tor Anonymity System}},
	Year = {2012}}
	
	@techreport{dust,
	Author = {{B.~Wiley}},
	Howpublished = {\url{http://blanu.net/ Dust.pdf}},
	Institution = {School of Information, University of Texas at Austin},
	Title = {{Dust: A Blocking-Resistant Internet Transport Protocol}},
	Year = {2011}}
	
	@inproceedings{FTE,
	Author = {K.~Dyer and S.~Coull and T.~Ristenpart and T.~Shrimpton},
	Booktitle = {CCS},
	Title = {{Protocol Misidentification Made Easy with Format-Transforming Encryption}},
	Year = {2013}}
	
	@inproceedings{fp,
	Author = {Fifield, D. and Hardison, N. and Ellithrope, J. and Stark, E. and Dingledine, R. and Boneh, D. and Porras, P.},
	Booktitle = {PETS},
	Title = {{Evading Censorship with Browser-Based Proxies}},
	Year = {2012}}
	
	@misc{obfsproxy,
	Howpublished = {\url{https://www.torproject.org/projects/obfsproxy.html.en}},
	Key = {obfsproxy},
	Publisher = {The Tor Project},
	Title = {{A Simple Obfuscating Proxy}}}
	
	@inproceedings{Tor-instead-of-IP,
	Author = {Liu, V. and Han, S. and Krishnamurthy, A. and Anderson, T.},
	Booktitle = {HotNets},
	Title = {{Tor instead of IP}},
	Year = {2011}}
	
	@misc{roger-slides,
	Howpublished = {\url{https://svn.torproject.org/svn/projects/presentations/slides-28c3.pdf}},
	Key = {torblocking},
	Title = {{How Governments Have Tried to Block Tor}}}
	
	@inproceedings{infranet,
	Author = {Feamster, N. and Balazinska, M. and Harfst, G. and Balakrishnan, H. and Karger, D.},
	Booktitle = {USENIX Security},
	Title = {{Infranet: Circumventing Web Censorship and Surveillance}},
	Year = {2002}}
	
	@inproceedings{collage,
	Author = {S.~Burnett and N.~Feamster and S.~Vempala},
	Booktitle = {USENIX Security},
	Title = {{Chipping Away at Censorship Firewalls with User-Generated Content}},
	Year = {2010}}
	
	@article{anonymizer,
	Author = {Boyan, J.},
	Journal = {Computer-Mediated Communication Magazine},
	Month = sep,
	Number = {9},
	Title = {{The Anonymizer: Protecting User Privacy on the Web}},
	Volume = {4},
	Year = {1997}}
	
	@article{schulze2009internet,
	Author = {Schulze, H. and Mochalski, K.},
	Journal = {IPOQUE Report},
	Pages = {351--362},
	Title = {Internet Study 2008/2009},
	Volume = {37},
	Year = {2009}}
	
	@inproceedings{cya-ccs13,
	Author = {J.~Geddes and M.~Schuchard and N.~Hopper},
	Booktitle = {{CCS}},
	Title = {{Cover Your ACKs: Pitfalls of Covert Channel Censorship Circumvention}},
	Year = {2013}}
	
	@inproceedings{andana,
	Author = {DiBenedetto, S. and Gasti, P. and Tsudik, G. and Uzun, E.},
	Booktitle = {{NDSS}},
	Title = {{ANDaNA: Anonymous Named Data Networking Application}},
	Year = {2012}}
	
	@inproceedings{darkly,
	Author = {Jana, S. and Narayanan, A. and Shmatikov, V.},
	Booktitle = {IEEE S\&P},
	Title = {{A Scanner Darkly: Protecting User Privacy From Perceptual Applications}},
	Year = {2013}}
	
	@inproceedings{NS08,
	Author = {A.~Narayanan and V.~Shmatikov},
	Booktitle = {IEEE S\&P},
	Title = {Robust de-anonymization of large sparse datasets},
	Year = {2008}}
	
	@inproceedings{NS09,
	Author = {Arvind Narayanan and Vitaly Shmatikov},
	Booktitle = {IEEE S\&P},
	Title = {De-anonymizing Social Networks},
	Year = {2009}}
	
	@inproceedings{memento,
	Author = {Jana, S. and Shmatikov, V.},
	Booktitle = {IEEE S\&P},
	Title = {{Memento: Learning secrets from process footprints}},
	Year = {2012}}
	
	@misc{plugtor,
	Howpublished = {\url{https://www.torproject.org/docs/pluggable-transports.html.en}},
	Key = {PluggableTransports},
	Publisher = {The Tor Project},
	Title = {{Tor: Pluggable transports}}}
	
	@misc{psiphon,
	Author = {J.~Jia and P.~Smith},
	Howpublished = {\url{http://www.cdf.toronto.edu/~csc494h/reports/2004-fall/psiphon_ae.html}},
	Title = {{Psiphon: Analysis and Estimation}},
	Year = 2004}
	
	@misc{china-github,
	Howpublished = {\url{http://mobile.informationweek.com/80269/show/72e30386728f45f56b343ddfd0fdb119/}},
	Key = {github},
	Title = {{China's GitHub Censorship Dilemma}}}
	
	@inproceedings{txbox,
	Author = {Jana, S. and Porter, D. and Shmatikov, V.},
	Booktitle = {IEEE S\&P},
	Title = {{TxBox: Building Secure, Efficient Sandboxes with System Transactions}},
	Year = {2011}}
	
	@inproceedings{airavat,
	Author = {I. Roy and S. Setty and A. Kilzer and V. Shmatikov and E. Witchel},
	Booktitle = {NSDI},
	Title = {{Airavat: Security and Privacy for MapReduce}},
	Year = {2010}}
	
	@inproceedings{osdi12,
	Author = {A. Dunn and M. Lee and S. Jana and S. Kim and M. Silberstein and Y. Xu and V. Shmatikov and E. Witchel},
	Booktitle = {OSDI},
	Title = {{Eternal Sunshine of the Spotless Machine: Protecting Privacy with Ephemeral Channels}},
	Year = {2012}}
	
	@inproceedings{ymal,
	Author = {J. Calandrino and A. Kilzer and A. Narayanan and E. Felten and V. Shmatikov},
	Booktitle = {IEEE S\&P},
	Title = {{``You Might Also Like:'' Privacy Risks of Collaborative Filtering}},
	Year = {2011}}
	
	@inproceedings{srivastava11,
	Author = {V. Srivastava and M. Bond and K. McKinley and V. Shmatikov},
	Booktitle = {PLDI},
	Title = {{A Security Policy Oracle: Detecting Security Holes Using Multiple API Implementations}},
	Year = {2011}}
	
	@inproceedings{chen-oakland10,
	Author = {Chen, S. and Wang, R. and Wang, X. and Zhang, K.},
	Booktitle = {IEEE S\&P},
	Title = {{Side-Channel Leaks in Web Applications: A Reality Today, a Challenge Tomorrow}},
	Year = {2010}}
	
	@book{kerck,
	Author = {Kerckhoffs, A.},
	Publisher = {University Microfilms},
	Title = {{La cryptographie militaire}},
	Year = {1978}}
	
	@inproceedings{foci11,
	Author = {J. Karlin and D. Ellard and A.~Jackson and C.~ Jones and G. Lauer and D. Mankins and W.~T.~Strayer},
	Booktitle = {FOCI},
	Title = {{Decoy Routing: Toward Unblockable Internet Communication}},
	Year = 2011}
	
	@inproceedings{sun02,
	Author = {Sun, Q. and Simon, D.~R. and Wang, Y. and Russell, W. and Padmanabhan, V. and Qiu, L.},
	Booktitle = {IEEE S\&P},
	Title = {{Statistical Identification of Encrypted Web Browsing Traffic}},
	Year = {2002}}
	
	@inproceedings{danezis,
	Author = {Murdoch, S.~J. and Danezis, G.},
	Booktitle = {IEEE S\&P},
	Title = {{Low-Cost Traffic Analysis of Tor}},
	Year = {2005}}
	
	@inproceedings{pakicensorship,
	Author = {Z.~Nabi},
	Booktitle = {FOCI},
	Title = {The Anatomy of {Web} Censorship in {Pakistan}},
	Year = {2013}}
	
	@inproceedings{irancensorship,
	Author = {S.~Aryan and H.~Aryan and A.~Halderman},
	Booktitle = {FOCI},
	Title = {Internet Censorship in {Iran}: {A} First Look},
	Year = {2013}}
	
	@inproceedings{ford10efficient,
	Author = {Amittai Aviram and Shu-Chun Weng and Sen Hu and Bryan Ford},
	Booktitle = {\bibconf[9th]{OSDI}{USENIX Symposium on Operating Systems Design and Implementation}},
	Location = {Vancouver, BC, Canada},
	Month = oct,
	Title = {Efficient System-Enforced Deterministic Parallelism},
	Year = 2010}
	
	@inproceedings{ford10determinating,
	Author = {Amittai Aviram and Sen Hu and Bryan Ford and Ramakrishna Gummadi},
	Booktitle = {\bibconf{CCSW}{ACM Cloud Computing Security Workshop}},
	Location = {Chicago, IL},
	Month = oct,
	Title = {Determinating Timing Channels in Compute Clouds},
	Year = 2010}
	
	@inproceedings{ford12plugging,
	Author = {Bryan Ford},
	Booktitle = {\bibconf[4th]{HotCloud}{USENIX Workshop on Hot Topics in Cloud Computing}},
	Location = {Boston, MA},
	Month = jun,
	Title = {Plugging Side-Channel Leaks with Timing Information Flow Control},
	Year = 2012}
	
	@inproceedings{ford12icebergs,
	Author = {Bryan Ford},
	Booktitle = {\bibconf[4th]{HotCloud}{USENIX Workshop on Hot Topics in Cloud Computing}},
	Location = {Boston, MA},
	Month = jun,
	Title = {Icebergs in the Clouds: the {\em Other} Risks of Cloud Computing},
	Year = 2012}
	
	@misc{mullenize,
	Author = {Washington Post},
	Howpublished = {\url{http://apps.washingtonpost.com/g/page/world/gchq-report-on-mullenize-program-to-stain-anonymous-electronic-traffic/502/}},
	Month = {oct},
	Title = {{GCHQ} report on {`MULLENIZE'} program to `stain' anonymous electronic traffic},
	Year = {2013}}
	
	@inproceedings{shue13street,
	Author = {Craig A. Shue and Nathanael Paul and Curtis R. Taylor},
	Booktitle = {\bibbrev[7th]{WOOT}{USENIX Workshop on Offensive Technologies}},
	Month = aug,
	Title = {From an {IP} Address to a Street Address: Using Wireless Signals to Locate a Target},
	Year = 2013}
	
	@inproceedings{knockel11three,
	Author = {Jeffrey Knockel and Jedidiah R. Crandall and Jared Saia},
	Booktitle = {\bibbrev{FOCI}{USENIX Workshop on Free and Open Communications on the Internet}},
	Location = {San Francisco, CA},
	Month = aug,
	Year = 2011}
	
	@misc{rfc4960,
	Author = {R. {Stewart, ed.}},
	Month = sep,
	Note = {RFC 4960},
	Title = {Stream Control Transmission Protocol},
	Year = 2007}
	
	@inproceedings{ford07structured,
	Author = {Bryan Ford},
	Booktitle = {\bibbrev{SIGCOMM}{ACM SIGCOMM}},
	Location = {Kyoto, Japan},
	Month = aug,
	Title = {Structured Streams: a New Transport Abstraction},
	Year = {2007}}
	
	@misc{spdy,
	Author = {Google, Inc.},
	Note = {\url{http://www.chromium.org/spdy/spdy-whitepaper}},
	Title = {{SPDY}: An Experimental Protocol For a Faster {Web}}}
	
	@misc{quic,
	Author = {Jim Roskind},
	Month = jun,
	Note = {\url{http://blog.chromium.org/2013/06/experimenting-with-quic.html}},
	Title = {Experimenting with {QUIC}},
	Year = 2013}
	
	@misc{podjarny12not,
	Author = {G.~Podjarny},
	Month = jun,
	Note = {\url{http://www.guypo.com/technical/not-as-spdy-as-you-thought/}},
	Title = {{Not as SPDY as You Thought}},
	Year = 2012}
	
	@inproceedings{cor,
	Author = {Jones, N.~A. and Arye, M. and Cesareo, J. and Freedman, M.~J.},
	Booktitle = {FOCI},
	Title = {{Hiding Amongst the Clouds: A Proposal for Cloud-based Onion Routing}},
	Year = {2011}}
	
	@misc{torcloud,
	Howpublished = {\url{https://cloud.torproject.org/}},
	Key = {tor cloud},
	Title = {{The Tor Cloud Project}}}
	
	@inproceedings{scramblesuit,
	Author = {Philipp Winter and Tobias Pulls and Juergen Fuss},
	Booktitle = {WPES},
	Title = {{ScrambleSuit: A Polymorphic Network Protocol to Circumvent Censorship}},
	Year = 2013}
	
	@article{savage2000practical,
	Author = {Savage, S. and Wetherall, D. and Karlin, A. and Anderson, T.},
	Journal = {ACM SIGCOMM Computer Communication Review},
	Number = {4},
	Pages = {295--306},
	Publisher = {ACM},
	Title = {Practical network support for IP traceback},
	Volume = {30},
	Year = {2000}}
	
	@inproceedings{ooni,
	Author = {Filast, A. and Appelbaum, J.},
	Booktitle = {{FOCI}},
	Title = {{OONI : Open Observatory of Network Interference}},
	Year = {2012}}
	
	@misc{caida-rank,
	Howpublished = {\url{http://as-rank.caida.org/}},
	Key = {caida rank},
	Title = {{AS Rank: AS Ranking}}}
	
	@inproceedings{usersrouted-ccs13,
	Author = {A.~Johnson and C.~Wacek and R.~Jansen and M.~Sherr and P.~Syverson},
	Booktitle = {CCS},
	Title = {{Users Get Routed: Traffic Correlation on Tor by Realistic Adversaries}},
	Year = {2013}}
	
	@inproceedings{edman2009awareness,
	Author = {Edman, M. and Syverson, P.},
	Booktitle = {{CCS}},
	Title = {{AS-awareness in Tor path selection}},
	Year = {2009}}
	
	@inproceedings{DecoyCosts,
	Author = {A.~Houmansadr and E.~L.~Wong and V.~Shmatikov},
	Booktitle = {NDSS},
	Title = {{No Direction Home: The True Cost of Routing Around Decoys}},
	Year = {2014}}
	
	@article{cordon,
	Author = {Elahi, T. and Goldberg, I.},
	Journal = {University of Waterloo CACR},
	Title = {{CORDON--A Taxonomy of Internet Censorship Resistance Strategies}},
	Volume = {33},
	Year = {2012}}
	
	@inproceedings{privex,
	Author = {T.~Elahi and G.~Danezis and I.~Goldberg	},
	Booktitle = {{CCS}},
	Title = {{AS-awareness in Tor path selection}},
	Year = {2014}}
	
	@inproceedings{changeGuards,
	Author = {T.~Elahi and K.~Bauer and M.~AlSabah and R.~Dingledine and I.~Goldberg},
	Booktitle = {{WPES}},
	Title = {{ Changing of the Guards: Framework for Understanding and Improving Entry Guard Selection in Tor}},
	Year = {2012}}
	
	@article{RAINBOW:Journal,
	Author = {A.~Houmansadr and N.~Kiyavash and N.~Borisov},
	Journal = {IEEE/ACM Transactions on Networking},
	Title = {{Non-Blind Watermarking of Network Flows}},
	Year = 2014}
	
	@inproceedings{info-tod,
	Author = {A.~Houmansadr and S.~Gorantla and T.~Coleman and N.~Kiyavash and and N.~Borisov},
	Booktitle = {{CCS (poster session)}},
	Title = {{On the Channel Capacity of Network Flow Watermarking}},
	Year = {2009}}
	
	@inproceedings{johnson2014game,
	Author = {Johnson, B. and Laszka, A. and Grossklags, J. and Vasek, M. and Moore, T.},
	Booktitle = {Workshop on Bitcoin Research},
	Title = {{Game-theoretic Analysis of DDoS Attacks Against Bitcoin Mining Pools}},
	Year = {2014}}
	
	@incollection{laszka2013mitigation,
	Author = {Laszka, A. and Johnson, B. and Grossklags, J.},
	Booktitle = {Decision and Game Theory for Security},
	Pages = {175--191},
	Publisher = {Springer},
	Title = {{Mitigation of Targeted and Non-targeted Covert Attacks as a Timing Game}},
	Year = {2013}}
	
	@inproceedings{schottle2013game,
	Author = {Schottle, P. and Laszka, A. and Johnson, B. and Grossklags, J. and Bohme, R.},
	Booktitle = {EUSIPCO},
	Title = {{A Game-theoretic Analysis of Content-adaptive Steganography with Independent Embedding}},
	Year = {2013}}
	
	@inproceedings{CloudTransport,
	Author = {C.~Brubaker and A.~Houmansadr and V.~Shmatikov},
	Booktitle = {PETS},
	Title = {{CloudTransport: Using Cloud Storage for Censorship-Resistant Networking}},
	Year = {2014}}
	
	@inproceedings{sweet,
	Author = {W.~Zhou and A.~Houmansadr and M.~Caesar and N.~Borisov},
	Booktitle = {HotPETs},
	Title = {{SWEET: Serving the Web by Exploiting Email Tunnels}},
	Year = {2013}}
	
	@inproceedings{ahsan2002practical,
	Author = {Ahsan, K. and Kundur, D.},
	Booktitle = {Workshop on Multimedia Security},
	Title = {{Practical data hiding in TCP/IP}},
	Year = {2002}}
	
	@incollection{danezis2011covert,
	Author = {Danezis, G.},
	Booktitle = {Security Protocols XVI},
	Pages = {198--214},
	Publisher = {Springer},
	Title = {{Covert Communications Despite Traffic Data Retention}},
	Year = {2011}}
	
	@inproceedings{liu2009hide,
	Author = {Liu, Y. and Ghosal, D. and Armknecht, F. and Sadeghi, A.-R. and Schulz, S. and Katzenbeisser, S.},
	Booktitle = {ESORICS},
	Title = {{Hide and Seek in Time---Robust Covert Timing Channels}},
	Year = {2009}}
	
	@misc{image-watermark-fing,
	Author = {Jonathan Bailey},
	Howpublished = {\url{https://www.plagiarismtoday.com/2009/12/02/image-detection-watermarking-vs-fingerprinting/}},
	Title = {{Image Detection: Watermarking vs. Fingerprinting}},
	Year = {2009}}
	
	@inproceedings{Servetto98,
	Author = {S. D. Servetto and C. I. Podilchuk and K. Ramchandran},
	Booktitle = {Int. Conf. Image Processing},
	Title = {Capacity issues in digital image watermarking},
	Year = {1998}}
	
	@inproceedings{Chen01,
	Author = {B. Chen and G.W.Wornell},
	Booktitle = {IEEE Trans. Inform. Theory},
	Pages = {1423--1443},
	Title = {Quantization index modulation: A class of provably good methods for digital watermarking and information embedding},
	Year = {2001}}
	
	@inproceedings{Karakos00,
	Author = {D. Karakos and A. Papamarcou},
	Booktitle = {IEEE Int. Symp. Information Theory},
	Pages = {47},
	Title = {Relationship between quantization and distribution rates of digitally watermarked data},
	Year = {2000}}
	
	@inproceedings{Sullivan98,
	Author = {J. A. OSullivan and P. Moulin and J. M. Ettinger},
	Booktitle = {IEEE Int. Symp. Information Theory},
	Pages = {297},
	Title = {Information theoretic analysis of steganography},
	Year = {1998}}
	
	@inproceedings{Merhav00,
	Author = {N. Merhav},
	Booktitle = {IEEE Trans. Inform. Theory},
	Pages = {420--430},
	Title = {On random coding error exponents of watermarking systems},
	Year = {2000}}
	
	@inproceedings{Somekh01,
	Author = {A. Somekh-Baruch and N. Merhav},
	Booktitle = {IEEE Int. Symp. Information Theory},
	Pages = {7},
	Title = {On the error exponent and capacity games of private watermarking systems},
	Year = {2001}}
	
	@inproceedings{Steinberg01,
	Author = {Y. Steinberg and N. Merhav},
	Booktitle = {IEEE Trans. Inform. Theory},
	Pages = {1410--1422},
	Title = {Identification in the presence of side information with application to watermarking},
	Year = {2001}}
	
	@article{Moulin03,
	Author = {P. Moulin and J.A. O'Sullivan},
	Journal = {IEEE Trans. Info. Theory},
	Number = {3},
	Title = {Information-theoretic analysis of information hiding},
	Volume = 49,
	Year = 2003}
	
	@article{Gelfand80,
	Author = {S.I.~Gelfand and M.S.~Pinsker},
	Journal = {Problems of Control and Information Theory},
	Number = {1},
	Pages = {19-31},
	Title = {{Coding for channel with random parameters}},
	Url = {citeseer.ist.psu.edu/anantharam96bits.html},
	Volume = {9},
	Year = {1980},
	Bdsk-Url-1 = {citeseer.ist.psu.edu/anantharam96bits.html}}
	
	@book{Wolfowitz78,
	Author = {J. Wolfowitz},
	Edition = {3rd},
	Location = {New York},
	Publisher = {Springer-Verlag},
	Title = {Coding Theorems of Information Theory},
	Year = 1978}
	
	@article{caire99,
	Author = {G. Caire and S. Shamai},
	Journal = {IEEE Transactions on Information Theory},
	Number = {6},
	Pages = {2007--2019},
	Title = {On the Capacity of Some Channels with Channel State Information},
	Volume = {45},
	Year = {1999}}
	
	@inproceedings{wright2007language,
	Author = {Wright, Charles V and Ballard, Lucas and Monrose, Fabian and Masson, Gerald M},
	Booktitle = {USENIX Security},
	Title = {{Language identification of encrypted VoIP traffic: Alejandra y Roberto or Alice and Bob?}},
	Year = {2007}}
	
	@inproceedings{backes2010speaker,
	Author = {Backes, Michael and Doychev, Goran and D{\"u}rmuth, Markus and K{\"o}pf, Boris},
	Booktitle = {{European Symposium on Research in Computer Security (ESORICS)}},
	Pages = {508--523},
	Publisher = {Springer},
	Title = {{Speaker Recognition in Encrypted Voice Streams}},
	Year = {2010}}
	
	@phdthesis{lu2009traffic,
	Author = {Lu, Yuanchao},
	School = {Cleveland State University},
	Title = {{On Traffic Analysis Attacks to Encrypted VoIP Calls}},
	Year = {2009}}
	
	@inproceedings{wright2008spot,
	Author = {Wright, Charles V and Ballard, Lucas and Coull, Scott E and Monrose, Fabian and Masson, Gerald M},
	Booktitle = {IEEE Symposium on Security and Privacy},
	Pages = {35--49},
	Title = {Spot me if you can: Uncovering spoken phrases in encrypted VoIP conversations},
	Year = {2008}}
	
	@inproceedings{white2011phonotactic,
	Author = {White, Andrew M and Matthews, Austin R and Snow, Kevin Z and Monrose, Fabian},
	Booktitle = {IEEE Symposium on Security and Privacy},
	Pages = {3--18},
	Title = {Phonotactic reconstruction of encrypted VoIP conversations: Hookt on fon-iks},
	Year = {2011}}
	
	@inproceedings{fancy,
	Author = {Houmansadr, Amir and Borisov, Nikita},
	Booktitle = {Privacy Enhancing Technologies},
	Organization = {Springer},
	Pages = {205--224},
	Title = {The Need for Flow Fingerprints to Link Correlated Network Flows},
	Year = {2013}}
	
	@article{botmosaic,
	Author = {Amir Houmansadr and Nikita Borisov},
	Doi = {10.1016/j.jss.2012.11.005},
	Issn = {0164-1212},
	Journal = {Journal of Systems and Software},
	Keywords = {Network security},
	Number = {3},
	Pages = {707 - 715},
	Title = {BotMosaic: Collaborative network watermark for the detection of IRC-based botnets},
	Url = {http://www.sciencedirect.com/science/article/pii/S0164121212003068},
	Volume = {86},
	Year = {2013},
	Bdsk-Url-1 = {http://www.sciencedirect.com/science/article/pii/S0164121212003068},
	Bdsk-Url-2 = {http://dx.doi.org/10.1016/j.jss.2012.11.005}}
	
	@inproceedings{ramsbrock2008first,
	Author = {Ramsbrock, Daniel and Wang, Xinyuan and Jiang, Xuxian},
	Booktitle = {Recent Advances in Intrusion Detection},
	Organization = {Springer},
	Pages = {59--77},
	Title = {A first step towards live botmaster traceback},
	Year = {2008}}
	
	@inproceedings{potdar2005survey,
	Author = {Potdar, Vidyasagar M and Han, Song and Chang, Elizabeth},
	Booktitle = {Industrial Informatics, 2005. INDIN'05. 2005 3rd IEEE International Conference on},
	Organization = {IEEE},
	Pages = {709--716},
	Title = {A survey of digital image watermarking techniques},
	Year = {2005}}
	
	@book{cole2003hiding,
	Author = {Cole, Eric and Krutz, Ronald D},
	Publisher = {John Wiley \& Sons, Inc.},
	Title = {Hiding in plain sight: Steganography and the art of covert communication},
	Year = {2003}}
	
	@incollection{akaike1998information,
	Author = {Akaike, Hirotogu},
	Booktitle = {Selected Papers of Hirotugu Akaike},
	Pages = {199--213},
	Publisher = {Springer},
	Title = {Information theory and an extension of the maximum likelihood principle},
	Year = {1998}}
	
	@misc{central-command-hack,
	Author = {Everett Rosenfeld},
	Howpublished = {\url{http://www.cnbc.com/id/102330338}},
	Title = {{FBI investigating Central Command Twitter hack}},
	Year = {2015}}
	
	@misc{sony-psp-ddos,
	Howpublished = {\url{http://n4g.com/news/1644853/sony-and-microsoft-cant-do-much-ddos-attacks-explained}},
	Key = {sony},
	Month = {December},
	Title = {{Sony and Microsoft cant do much -- DDoS attacks explained}},
	Year = {2014}}
	
	@misc{sony-hack,
	Author = {David Bloom},
	Howpublished = {\url{http://goo.gl/MwR4A7}},
	Title = {{Online Game Networks Hacked, Sony Unit President Threatened}},
	Year = {2014}}
	
	@misc{home-depot,
	Author = {Dune Lawrence},
	Howpublished = {\url{http://www.businessweek.com/articles/2014-09-02/home-depots-credit-card-breach-looks-just-like-the-target-hack}},
	Title = {{Home Depot's Suspected Breach Looks Just Like the Target Hack}},
	Year = {2014}}
	
	@misc{target,
	Author = {Julio Ojeda-Zapata},
	Howpublished = {\url{http://www.mercurynews.com/business/ci_24765398/how-did-hackers-pull-off-target-data-heist}},
	Title = {{Target hack: How did they do it?}},
	Year = {2014}}
	
	
	@article{probabilitycourse,
	Author = {H. Pishro-Nik},
	note = {\url{http://www.probabilitycourse.com}},
	Title = {Introduction to probability, statistics, and random processes},
	Year = {2014}}
	
	
	
	@inproceedings{shokri2011quantifying,
	Author = {Shokri, Reza and Theodorakopoulos, George and Le Boudec, Jean-Yves and Hubaux, Jean-Pierre},
	Booktitle = {Security and Privacy (SP), 2011 IEEE Symposium on},
	Organization = {IEEE},
	Pages = {247--262},
	Title = {Quantifying location privacy},
	Year = {2011}}
	
	@inproceedings{hoh2007preserving,
	Author = {Hoh, Baik and Gruteser, Marco and Xiong, Hui and Alrabady, Ansaf},
	Booktitle = {Proceedings of the 14th ACM conference on Computer and communications security},
	Organization = {ACM},
	Pages = {161--171},
	Title = {Preserving privacy in gps traces via uncertainty-aware path cloaking},
	Year = {2007}}
	
	
	
	@article{kafsi2013entropy,
	Author = {Kafsi, Mohamed and Grossglauser, Matthias and Thiran, Patrick},
	Journal = {Information Theory, IEEE Transactions on},
	Number = {9},
	Pages = {5577--5583},
	Publisher = {IEEE},
	Title = {The entropy of conditional Markov trajectories},
	Volume = {59},
	Year = {2013}}
	
	@inproceedings{gruteser2003anonymous,
	Author = {Gruteser, Marco and Grunwald, Dirk},
	Booktitle = {Proceedings of the 1st international conference on Mobile systems, applications and services},
	Organization = {ACM},
	Pages = {31--42},
	Title = {Anonymous usage of location-based services through spatial and temporal cloaking},
	Year = {2003}}
	
	@inproceedings{husted2010mobile,
	Author = {Husted, Nathaniel and Myers, Steven},
	Booktitle = {Proceedings of the 17th ACM conference on Computer and communications security},
	Organization = {ACM},
	Pages = {85--96},
	Title = {Mobile location tracking in metro areas: malnets and others},
	Year = {2010}}
	
	@inproceedings{li2009tradeoff,
	Author = {Li, Tiancheng and Li, Ninghui},
	Booktitle = {Proceedings of the 15th ACM SIGKDD international conference on Knowledge discovery and data mining},
	Organization = {ACM},
	Pages = {517--526},
	Title = {On the tradeoff between privacy and utility in data publishing},
	Year = {2009}}
	
	@inproceedings{ma2009location,
	Author = {Ma, Zhendong and Kargl, Frank and Weber, Michael},
	Booktitle = {Sarnoff Symposium, 2009. SARNOFF'09. IEEE},
	Organization = {IEEE},
	Pages = {1--6},
	Title = {A location privacy metric for v2x communication systems},
	Year = {2009}}
	
	@inproceedings{shokri2012protecting,
	Author = {Shokri, Reza and Theodorakopoulos, George and Troncoso, Carmela and Hubaux, Jean-Pierre and Le Boudec, Jean-Yves},
	Booktitle = {Proceedings of the 2012 ACM conference on Computer and communications security},
	Organization = {ACM},
	Pages = {617--627},
	Title = {Protecting location privacy: optimal strategy against localization attacks},
	Year = {2012}}
	
	@inproceedings{freudiger2009non,
	Author = {Freudiger, Julien and Manshaei, Mohammad Hossein and Hubaux, Jean-Pierre and Parkes, David C},
	Booktitle = {Proceedings of the 16th ACM conference on Computer and communications security},
	Organization = {ACM},
	Pages = {324--337},
	Title = {On non-cooperative location privacy: a game-theoretic analysis},
	Year = {2009}}
	
	@incollection{humbert2010tracking,
	Author = {Humbert, Mathias and Manshaei, Mohammad Hossein and Freudiger, Julien and Hubaux, Jean-Pierre},
	Booktitle = {Decision and Game Theory for Security},
	Pages = {38--57},
	Publisher = {Springer},
	Title = {Tracking games in mobile networks},
	Year = {2010}}
	
	@article{manshaei2013game,
	Author = {Manshaei, Mohammad Hossein and Zhu, Quanyan and Alpcan, Tansu and Bac{\c{s}}ar, Tamer and Hubaux, Jean-Pierre},
	Journal = {ACM Computing Surveys (CSUR)},
	Number = {3},
	Pages = {25},
	Publisher = {ACM},
	Title = {Game theory meets network security and privacy},
	Volume = {45},
	Year = {2013}}
	
	@article{palamidessi2006probabilistic,
	Author = {Palamidessi, Catuscia},
	Journal = {Electronic Notes in Theoretical Computer Science},
	Pages = {33--42},
	Publisher = {Elsevier},
	Title = {Probabilistic and nondeterministic aspects of anonymity},
	Volume = {155},
	Year = {2006}}
	
	@inproceedings{mokbel2006new,
	Author = {Mokbel, Mohamed F and Chow, Chi-Yin and Aref, Walid G},
	Booktitle = {Proceedings of the 32nd international conference on Very large data bases},
	Organization = {VLDB Endowment},
	Pages = {763--774},
	Title = {The new Casper: query processing for location services without compromising privacy},
	Year = {2006}}
	
	@article{kalnis2007preventing,
	Author = {Kalnis, Panos and Ghinita, Gabriel and Mouratidis, Kyriakos and Papadias, Dimitris},
	Journal = {Knowledge and Data Engineering, IEEE Transactions on},
	Number = {12},
	Pages = {1719--1733},
	Publisher = {IEEE},
	Title = {Preventing location-based identity inference in anonymous spatial queries},
	Volume = {19},
	Year = {2007}}
	
	@article{freudiger2007mix,
	title={Mix-zones for location privacy in vehicular networks},
	author={Freudiger, Julien and Raya, Maxim and F{\'e}legyh{\'a}zi, M{\'a}rk and Papadimitratos, Panos and Hubaux, Jean-Pierre},
	year={2007}
	}
	@article{sweeney2002k,
	Author = {Sweeney, Latanya},
	Journal = {International Journal of Uncertainty, Fuzziness and Knowledge-Based Systems},
	Number = {05},
	Pages = {557--570},
	Publisher = {World Scientific},
	Title = {k-anonymity: A model for protecting privacy},
	Volume = {10},
	Year = {2002}}
	
	@article{sweeney2002achieving,
	Author = {Sweeney, Latanya},
	Journal = {International Journal of Uncertainty, Fuzziness and Knowledge-Based Systems},
	Number = {05},
	Pages = {571--588},
	Publisher = {World Scientific},
	Title = {Achieving k-anonymity privacy protection using generalization and suppression},
	Volume = {10},
	Year = {2002}}
	
	@inproceedings{niu2014achieving,
	Author = {Niu, Ben and Li, Qinghua and Zhu, Xiaoyan and Cao, Guohong and Li, Hui},
	Booktitle = {INFOCOM, 2014 Proceedings IEEE},
	Organization = {IEEE},
	Pages = {754--762},
	Title = {Achieving k-anonymity in privacy-aware location-based services},
	Year = {2014}}
	
	@inproceedings{liu2013game,
	Author = {Liu, Xinxin and Liu, Kaikai and Guo, Linke and Li, Xiaolin and Fang, Yuguang},
	Booktitle = {INFOCOM, 2013 Proceedings IEEE},
	Organization = {IEEE},
	Pages = {2985--2993},
	Title = {A game-theoretic approach for achieving k-anonymity in location based services},
	Year = {2013}}
	
	@inproceedings{kido2005protection,
	Author = {Kido, Hidetoshi and Yanagisawa, Yutaka and Satoh, Tetsuji},
	Booktitle = {Data Engineering Workshops, 2005. 21st International Conference on},
	Organization = {IEEE},
	Pages = {1248--1248},
	Title = {Protection of location privacy using dummies for location-based services},
	Year = {2005}}
	
	@inproceedings{gedik2005location,
	Author = {Gedik, Bu{\u{g}}ra and Liu, Ling},
	Booktitle = {Distributed Computing Systems, 2005. ICDCS 2005. Proceedings. 25th IEEE International Conference on},
	Organization = {IEEE},
	Pages = {620--629},
	Title = {Location privacy in mobile systems: A personalized anonymization model},
	Year = {2005}}
	
	@inproceedings{bordenabe2014optimal,
	Author = {Bordenabe, Nicol{\'a}s E and Chatzikokolakis, Konstantinos and Palamidessi, Catuscia},
	Booktitle = {Proceedings of the 2014 ACM SIGSAC Conference on Computer and Communications Security},
	Organization = {ACM},
	Pages = {251--262},
	Title = {Optimal geo-indistinguishable mechanisms for location privacy},
	Year = {2014}}
	
	@incollection{duckham2005formal,
	Author = {Duckham, Matt and Kulik, Lars},
	Booktitle = {Pervasive computing},
	Pages = {152--170},
	Publisher = {Springer},
	Title = {A formal model of obfuscation and negotiation for location privacy},
	Year = {2005}}
	
	@inproceedings{kido2005anonymous,
	Author = {Kido, Hidetoshi and Yanagisawa, Yutaka and Satoh, Tetsuji},
	Booktitle = {Pervasive Services, 2005. ICPS'05. Proceedings. International Conference on},
	Organization = {IEEE},
	Pages = {88--97},
	Title = {An anonymous communication technique using dummies for location-based services},
	Year = {2005}}
	
	@incollection{duckham2006spatiotemporal,
	Author = {Duckham, Matt and Kulik, Lars and Birtley, Athol},
	Booktitle = {Geographic Information Science},
	Pages = {47--64},
	Publisher = {Springer},
	Title = {A spatiotemporal model of strategies and counter strategies for location privacy protection},
	Year = {2006}}
	
	@inproceedings{shankar2009privately,
	Author = {Shankar, Pravin and Ganapathy, Vinod and Iftode, Liviu},
	Booktitle = {Proceedings of the 11th international conference on Ubiquitous computing},
	Organization = {ACM},
	Pages = {31--40},
	Title = {Privately querying location-based services with SybilQuery},
	Year = {2009}}
	
	@inproceedings{chow2009faking,
	Author = {Chow, Richard and Golle, Philippe},
	Booktitle = {Proceedings of the 8th ACM workshop on Privacy in the electronic society},
	Organization = {ACM},
	Pages = {105--108},
	Title = {Faking contextual data for fun, profit, and privacy},
	Year = {2009}}
	
	@incollection{xue2009location,
	Author = {Xue, Mingqiang and Kalnis, Panos and Pung, Hung Keng},
	Booktitle = {Location and Context Awareness},
	Pages = {70--87},
	Publisher = {Springer},
	Title = {Location diversity: Enhanced privacy protection in location based services},
	Year = {2009}}
	
	@article{wernke2014classification,
	Author = {Wernke, Marius and Skvortsov, Pavel and D{\"u}rr, Frank and Rothermel, Kurt},
	Journal = {Personal and Ubiquitous Computing},
	Number = {1},
	Pages = {163--175},
	Publisher = {Springer-Verlag},
	Title = {A classification of location privacy attacks and approaches},
	Volume = {18},
	Year = {2014}}
	
	@misc{cai2015cloaking,
	Author = {Cai, Y. and Xu, G.},
	Month = jan # {~1},
	Note = {US Patent App. 14/472,462},
	Publisher = {Google Patents},
	Title = {Cloaking with footprints to provide location privacy protection in location-based services},
	Url = {https://www.google.com/patents/US20150007341},
	Year = {2015},
	Bdsk-Url-1 = {https://www.google.com/patents/US20150007341}}
	
	@article{gedik2008protecting,
	Author = {Gedik, Bu{\u{g}}ra and Liu, Ling},
	Journal = {Mobile Computing, IEEE Transactions on},
	Number = {1},
	Pages = {1--18},
	Publisher = {IEEE},
	Title = {Protecting location privacy with personalized k-anonymity: Architecture and algorithms},
	Volume = {7},
	Year = {2008}}
	
	@article{kalnis2006preserving,
	Author = {Kalnis, Panos and Ghinita, Gabriel and Mouratidis, Kyriakos and Papadias, Dimitris},
	Publisher = {TRB6/06},
	Title = {Preserving anonymity in location based services},
	Year = {2006}}
	
	@inproceedings{hoh2005protecting,
	Author = {Hoh, Baik and Gruteser, Marco},
	Booktitle = {Security and Privacy for Emerging Areas in Communications Networks, 2005. SecureComm 2005. First International Conference on},
	Organization = {IEEE},
	Pages = {194--205},
	Title = {Protecting location privacy through path confusion},
	Year = {2005}}
	
	@article{terrovitis2011privacy,
	Author = {Terrovitis, Manolis},
	Journal = {ACM SIGKDD Explorations Newsletter},
	Number = {1},
	Pages = {6--18},
	Publisher = {ACM},
	Title = {Privacy preservation in the dissemination of location data},
	Volume = {13},
	Year = {2011}}
	
	@article{shin2012privacy,
	Author = {Shin, Kang G and Ju, Xiaoen and Chen, Zhigang and Hu, Xin},
	Journal = {Wireless Communications, IEEE},
	Number = {1},
	Pages = {30--39},
	Publisher = {IEEE},
	Title = {Privacy protection for users of location-based services},
	Volume = {19},
	Year = {2012}}
	
	@article{khoshgozaran2011location,
	Author = {Khoshgozaran, Ali and Shahabi, Cyrus and Shirani-Mehr, Houtan},
	Journal = {Knowledge and Information Systems},
	Number = {3},
	Pages = {435--465},
	Publisher = {Springer},
	Title = {Location privacy: going beyond K-anonymity, cloaking and anonymizers},
	Volume = {26},
	Year = {2011}}
	
	@incollection{chatzikokolakis2015geo,
	Author = {Chatzikokolakis, Konstantinos and Palamidessi, Catuscia and Stronati, Marco},
	Booktitle = {Distributed Computing and Internet Technology},
	Pages = {49--72},
	Publisher = {Springer},
	Title = {Geo-indistinguishability: A Principled Approach to Location Privacy},
	Year = {2015}}
	
	@inproceedings{ngo2015location,
	Author = {Ngo, Hoa and Kim, Jong},
	Booktitle = {Computer Security Foundations Symposium (CSF), 2015 IEEE 28th},
	Organization = {IEEE},
	Pages = {63--74},
	Title = {Location Privacy via Differential Private Perturbation of Cloaking Area},
	Year = {2015}}
	
	@inproceedings{palanisamy2011mobimix,
	Author = {Palanisamy, Balaji and Liu, Ling},
	Booktitle = {Data Engineering (ICDE), 2011 IEEE 27th International Conference on},
	Organization = {IEEE},
	Pages = {494--505},
	Title = {Mobimix: Protecting location privacy with mix-zones over road networks},
	Year = {2011}}
	
	@inproceedings{um2010advanced,
	Author = {Um, Jung-Ho and Kim, Hee-Dae and Chang, Jae-Woo},
	Booktitle = {Social Computing (SocialCom), 2010 IEEE Second International Conference on},
	Organization = {IEEE},
	Pages = {1093--1098},
	Title = {An advanced cloaking algorithm using Hilbert curves for anonymous location based service},
	Year = {2010}}
	
	@inproceedings{bamba2008supporting,
	Author = {Bamba, Bhuvan and Liu, Ling and Pesti, Peter and Wang, Ting},
	Booktitle = {Proceedings of the 17th international conference on World Wide Web},
	Organization = {ACM},
	Pages = {237--246},
	Title = {Supporting anonymous location queries in mobile environments with privacygrid},
	Year = {2008}}
	
	@inproceedings{zhangwei2010distributed,
	Author = {Zhangwei, Huang and Mingjun, Xin},
	Booktitle = {Networks Security Wireless Communications and Trusted Computing (NSWCTC), 2010 Second International Conference on},
	Organization = {IEEE},
	Pages = {468--471},
	Title = {A distributed spatial cloaking protocol for location privacy},
	Volume = {2},
	Year = {2010}}
	
	@article{chow2011spatial,
	Author = {Chow, Chi-Yin and Mokbel, Mohamed F and Liu, Xuan},
	Journal = {GeoInformatica},
	Number = {2},
	Pages = {351--380},
	Publisher = {Springer},
	Title = {Spatial cloaking for anonymous location-based services in mobile peer-to-peer environments},
	Volume = {15},
	Year = {2011}}
	
	@inproceedings{lu2008pad,
	Author = {Lu, Hua and Jensen, Christian S and Yiu, Man Lung},
	Booktitle = {Proceedings of the Seventh ACM International Workshop on Data Engineering for Wireless and Mobile Access},
	Organization = {ACM},
	Pages = {16--23},
	Title = {Pad: privacy-area aware, dummy-based location privacy in mobile services},
	Year = {2008}}
	
	@incollection{khoshgozaran2007blind,
	Author = {Khoshgozaran, Ali and Shahabi, Cyrus},
	Booktitle = {Advances in Spatial and Temporal Databases},
	Pages = {239--257},
	Publisher = {Springer},
	Title = {Blind evaluation of nearest neighbor queries using space transformation to preserve location privacy},
	Year = {2007}}
	
	@inproceedings{ghinita2008private,
	Author = {Ghinita, Gabriel and Kalnis, Panos and Khoshgozaran, Ali and Shahabi, Cyrus and Tan, Kian-Lee},
	Booktitle = {Proceedings of the 2008 ACM SIGMOD international conference on Management of data},
	Organization = {ACM},
	Pages = {121--132},
	Title = {Private queries in location based services: anonymizers are not necessary},
	Year = {2008}}
	
	@article{paulet2014privacy,
	Author = {Paulet, Russell and Kaosar, Md Golam and Yi, Xun and Bertino, Elisa},
	Journal = {Knowledge and Data Engineering, IEEE Transactions on},
	Number = {5},
	Pages = {1200--1210},
	Publisher = {IEEE},
	Title = {Privacy-preserving and content-protecting location based queries},
	Volume = {26},
	Year = {2014}}
	
	@article{nguyen2013differential,
	Author = {Nguyen, Hiep H and Kim, Jong and Kim, Yoonho},
	Journal = {Journal of Computing Science and Engineering},
	Number = {3},
	Pages = {177--186},
	Title = {Differential privacy in practice},
	Volume = {7},
	Year = {2013}}
	
	@inproceedings{lee2012differential,
	Author = {Lee, Jaewoo and Clifton, Chris},
	Booktitle = {Proceedings of the 18th ACM SIGKDD international conference on Knowledge discovery and data mining},
	Organization = {ACM},
	Pages = {1041--1049},
	Title = {Differential identifiability},
	Year = {2012}}
	
	@inproceedings{andres2013geo,
	Author = {Andr{\'e}s, Miguel E and Bordenabe, Nicol{\'a}s E and Chatzikokolakis, Konstantinos and Palamidessi, Catuscia},
	Booktitle = {Proceedings of the 2013 ACM SIGSAC conference on Computer \& communications security},
	Organization = {ACM},
	Pages = {901--914},
	Title = {Geo-indistinguishability: Differential privacy for location-based systems},
	Year = {2013}}
	
	@inproceedings{machanavajjhala2008privacy,
	Author = {Machanavajjhala, Ashwin and Kifer, Daniel and Abowd, John and Gehrke, Johannes and Vilhuber, Lars},
	Booktitle = {Data Engineering, 2008. ICDE 2008. IEEE 24th International Conference on},
	Organization = {IEEE},
	Pages = {277--286},
	Title = {Privacy: Theory meets practice on the map},
	Year = {2008}}
	
	@article{dewri2013local,
	Author = {Dewri, Rinku},
	Journal = {Mobile Computing, IEEE Transactions on},
	Number = {12},
	Pages = {2360--2372},
	Publisher = {IEEE},
	Title = {Local differential perturbations: Location privacy under approximate knowledge attackers},
	Volume = {12},
	Year = {2013}}
	
	@inproceedings{chatzikokolakis2013broadening,
	Author = {Chatzikokolakis, Konstantinos and Andr{\'e}s, Miguel E and Bordenabe, Nicol{\'a}s Emilio and Palamidessi, Catuscia},
	Booktitle = {Privacy Enhancing Technologies},
	Organization = {Springer},
	Pages = {82--102},
	Title = {Broadening the Scope of Differential Privacy Using Metrics.},
	Year = {2013}}
	
	@inproceedings{zhong2009distributed,
	Author = {Zhong, Ge and Hengartner, Urs},
	Booktitle = {Pervasive Computing and Communications, 2009. PerCom 2009. IEEE International Conference on},
	Organization = {IEEE},
	Pages = {1--10},
	Title = {A distributed k-anonymity protocol for location privacy},
	Year = {2009}}
	
	@inproceedings{ho2011differential,
	Author = {Ho, Shen-Shyang and Ruan, Shuhua},
	Booktitle = {Proceedings of the 4th ACM SIGSPATIAL International Workshop on Security and Privacy in GIS and LBS},
	Organization = {ACM},
	Pages = {17--24},
	Title = {Differential privacy for location pattern mining},
	Year = {2011}}
	
	@inproceedings{cheng2006preserving,
	Author = {Cheng, Reynold and Zhang, Yu and Bertino, Elisa and Prabhakar, Sunil},
	Booktitle = {Privacy Enhancing Technologies},
	Organization = {Springer},
	Pages = {393--412},
	Title = {Preserving user location privacy in mobile data management infrastructures},
	Year = {2006}}
	
	@article{beresford2003location,
	Author = {Beresford, Alastair R and Stajano, Frank},
	Journal = {IEEE Pervasive computing},
	Number = {1},
	Pages = {46--55},
	Publisher = {IEEE},
	Title = {Location privacy in pervasive computing},
	Year = {2003}}
	
	@inproceedings{freudiger2009optimal,
	Author = {Freudiger, Julien and Shokri, Reza and Hubaux, Jean-Pierre},
	Booktitle = {Privacy enhancing technologies},
	Organization = {Springer},
	Pages = {216--234},
	Title = {On the optimal placement of mix zones},
	Year = {2009}}
	
	@article{krumm2009survey,
	Author = {Krumm, John},
	Journal = {Personal and Ubiquitous Computing},
	Number = {6},
	Pages = {391--399},
	Publisher = {Springer},
	Title = {A survey of computational location privacy},
	Volume = {13},
	Year = {2009}}
	
	@article{Rakhshan2016letter,
	Author = {Rakhshan, Ali and Pishro-Nik, Hossein},
	Journal = {IEEE Wireless Communications Letter},
	Publisher = {IEEE},
	Title = {Interference Models for Vehicular Ad Hoc Networks},
	Year = {2016, submitted}}
	
	@article{Rakhshan2015Journal,
	Author = {Rakhshan, Ali and Pishro-Nik, Hossein},
	Journal = {IEEE Transactions on Wireless Communications},
	Publisher = {IEEE},
	Title = {Improving Safety on Highways by Customizing Vehicular Ad Hoc Networks},
	Year = {to appear, 2017}}
	
	@inproceedings{Rakhshan2015Cogsima,
	Author = {Rakhshan, Ali and Pishro-Nik, Hossein},
	Booktitle = {IEEE International Multi-Disciplinary Conference on Cognitive Methods in Situation Awareness and Decision Support},
	Organization = {IEEE},
	Title = {A New Approach to Customization of Accident Warning Systems to Individual Drivers},
	Year = {2015}}
	
	@inproceedings{Rakhshan2015CISS,
	Author = {Rakhshan, Ali and Pishro-Nik, Hossein and Nekoui, Mohammad},
	Booktitle = {Conference on Information Sciences and Systems},
	Organization = {IEEE},
	Pages = {1--6},
	Title = {Driver-based adaptation of Vehicular Ad Hoc Networks for design of active safety systems},
	Year = {2015}}
	
	@inproceedings{Rakhshan2014IV,
	Author = {Rakhshan, Ali and Pishro-Nik, Hossein and Ray, Evan},
	Booktitle = {Intelligent Vehicles Symposium},
	Organization = {IEEE},
	Pages = {1181--1186},
	Title = {Real-time estimation of the distribution of brake response times for an individual driver using Vehicular Ad Hoc Network.},
	Year = {2014}}
	
	@inproceedings{Rakhshan2013Globecom,
	Author = {Rakhshan, Ali and Pishro-Nik, Hossein and Fisher, Donald and Nekoui, Mohammad},
	Booktitle = {IEEE Global Communications Conference},
	Organization = {IEEE},
	Pages = {1333--1337},
	Title = {Tuning collision warning algorithms to individual drivers for design of active safety systems.},
	Year = {2013}}
	
	@article{Nekoui2012Journal,
	Author = {Nekoui, Mohammad and Pishro-Nik, Hossein},
	Journal = {IEEE Transactions on Wireless Communications},
	Number = {8},
	Pages = {2895--2905},
	Publisher = {IEEE},
	Title = {Throughput Scaling laws for Vehicular Ad Hoc Networks},
	Volume = {11},
	Year = {2012}}
	
	
	
	
	
	
	
	
	
	@article{Nekoui2011Journal,
	Author = {Nekoui, Mohammad and Pishro-Nik, Hossein and Ni, Daiheng},
	Journal = {International Journal of Vehicular Technology},
	Pages = {1--11},
	Publisher = {Hindawi Publishing Corporation},
	Title = {Analytic Design of Active Safety Systems for Vehicular Ad hoc Networks},
	Volume = {2011},
	Year = {2011}}
	
	
	
	
	
	
	@article{shokri2014optimal,
	title={Optimal user-centric data obfuscation},
	author={Shokri, Reza},
	journal={arXiv preprint arXiv:1402.3426},
	year={2014}
	}
	@article{chatzikokolakis2015location,
	title={Location privacy via geo-indistinguishability},
	author={Chatzikokolakis, Konstantinos and Palamidessi, Catuscia and Stronati, Marco},
	journal={ACM SIGLOG News},
	volume={2},
	number={3},
	pages={46--69},
	year={2015},
	publisher={ACM}
	
	}
	@inproceedings{shokri2011quantifying2,
	title={Quantifying location privacy: the case of sporadic location exposure},
	author={Shokri, Reza and Theodorakopoulos, George and Danezis, George and Hubaux, Jean-Pierre and Le Boudec, Jean-Yves},
	booktitle={Privacy Enhancing Technologies},
	pages={57--76},
	year={2011},
	organization={Springer}
	}
	
	
	@inproceedings{Mont1603:Defining,
	AUTHOR="Zarrin Montazeri and Amir Houmansadr and Hossein Pishro-Nik",
	TITLE="Defining Perfect Location Privacy Using Anonymization",
	BOOKTITLE="2016 Annual Conference on Information Science and Systems (CISS) (CISS
	2016)",
	ADDRESS="Princeton, USA",
	DAYS=16,
	MONTH=mar,
	YEAR=2016,
	KEYWORDS="Information Theoretic Privacy; location-based services; Location Privacy;
	Information Theory",
	ABSTRACT="The popularity of mobile devices and location-based services (LBS) have
	created great concerns regarding the location privacy of users of such
	devices and services. Anonymization is a common technique that is often
	being used to protect the location privacy of LBS users. In this paper, we
	provide a general information theoretic definition for location privacy. In
	particular, we define perfect location privacy. We show that under certain
	conditions, perfect privacy is achieved if the pseudonyms of users is
	changed after o(N^(2/r?1)) observations by the adversary, where N is the
	number of users and r is the number of sub-regions or locations.
	"
	}
	@article{our-isita-location,
	Author = {Zarrin Montazeri and Amir Houmansadr and Hossein Pishro-Nik},
	Journal = {IEEE International Symposium on Information Theory and Its Applications (ISITA)},
	Title = {Achieving Perfect Location Privacy in Markov Models Using Anonymization},
	Year = {2016}
	}
	@article{our-TIFS,
	Author = {Zarrin Montazeri and Hossein Pishro-Nik and Amir Houmansadr},
	Journal = {IEEE Transactions on Information Forensics and Security, accepted with mandatory minor revisions},
	Title = {Perfect Location Privacy Using Anonymization in Mobile Networks},
	Year = {2017},
	note={Available on arxiv.org}
	}
	
	
	
	@techreport{sampigethaya2005caravan,
	title={CARAVAN: Providing location privacy for VANET},
	author={Sampigethaya, Krishna and Huang, Leping and Li, Mingyan and Poovendran, Radha and Matsuura, Kanta and Sezaki, Kaoru},
	year={2005},
	institution={DTIC Document}
	}
	@incollection{buttyan2007effectiveness,
	title={On the effectiveness of changing pseudonyms to provide location privacy in VANETs},
	author={Butty{\'a}n, Levente and Holczer, Tam{\'a}s and Vajda, Istv{\'a}n},
	booktitle={Security and Privacy in Ad-hoc and Sensor Networks},
	pages={129--141},
	year={2007},
	publisher={Springer}
	}
	@article{sampigethaya2007amoeba,
	title={AMOEBA: Robust location privacy scheme for VANET},
	author={Sampigethaya, Krishna and Li, Mingyan and Huang, Leping and Poovendran, Radha},
	journal={Selected Areas in communications, IEEE Journal on},
	volume={25},
	number={8},
	pages={1569--1589},
	year={2007},
	publisher={IEEE}
	}
	
	@article{lu2012pseudonym,
	title={Pseudonym changing at social spots: An effective strategy for location privacy in vanets},
	author={Lu, Rongxing and Li, Xiaodong and Luan, Tom H and Liang, Xiaohui and Shen, Xuemin},
	journal={Vehicular Technology, IEEE Transactions on},
	volume={61},
	number={1},
	pages={86--96},
	year={2012},
	publisher={IEEE}
	}
	@inproceedings{lu2010sacrificing,
	title={Sacrificing the plum tree for the peach tree: A socialspot tactic for protecting receiver-location privacy in VANET},
	author={Lu, Rongxing and Lin, Xiaodong and Liang, Xiaohui and Shen, Xuemin},
	booktitle={Global Telecommunications Conference (GLOBECOM 2010), 2010 IEEE},
	pages={1--5},
	year={2010},
	organization={IEEE}
	}
	@inproceedings{lin2011stap,
	title={STAP: A social-tier-assisted packet forwarding protocol for achieving receiver-location privacy preservation in VANETs},
	author={Lin, Xiaodong and Lu, Rongxing and Liang, Xiaohui and Shen, Xuemin Sherman},
	booktitle={INFOCOM, 2011 Proceedings IEEE},
	pages={2147--2155},
	year={2011},
	organization={IEEE}
	}
	@inproceedings{gerlach2007privacy,
	title={Privacy in VANETs using changing pseudonyms-ideal and real},
	author={Gerlach, Matthias and Guttler, Felix},
	booktitle={Vehicular Technology Conference, 2007. VTC2007-Spring. IEEE 65th},
	pages={2521--2525},
	year={2007},
	organization={IEEE}
	}
	@inproceedings{el2002security,
	title={Security issues in a future vehicular network},
	author={El Zarki, Magda and Mehrotra, Sharad and Tsudik, Gene and Venkatasubramanian, Nalini},
	booktitle={European Wireless},
	volume={2},
	year={2002}
	}
	
	@article{hubaux2004security,
	title={The security and privacy of smart vehicles},
	author={Hubaux, Jean-Pierre and Capkun, Srdjan and Luo, Jun},
	journal={IEEE Security \& Privacy Magazine},
	volume={2},
	number={LCA-ARTICLE-2004-007},
	pages={49--55},
	year={2004}
	}
	
	
	
	@inproceedings{duri2002framework,
	title={Framework for security and privacy in automotive telematics},
	author={Duri, Sastry and Gruteser, Marco and Liu, Xuan and Moskowitz, Paul and Perez, Ronald and Singh, Moninder and Tang, Jung-Mu},
	booktitle={Proceedings of the 2nd international workshop on Mobile commerce},
	pages={25--32},
	year={2002},
	organization={ACM}
	}
	@misc{NS-3,
	Howpublished = {\url{https://www.nsnam.org/}}},
}
@misc{testbed,
	Howpublished = {\url{http://www.its.dot.gov/testbed/PDF/SE-MI-Resource-Guide-9-3-1.pdf}}},
@misc{NGSIM,
	Howpublished = {\url{http://ops.fhwa.dot.gov/trafficanalysistools/ngsim.htm}},
}

@misc{National-a2013,
	Author = {National Highway Traffic Safety Administration},
	Howpublished = {\url{http://ops.fhwa.dot.gov/trafficanalysistools/ngsim.htm}},
	Title = {2013 Motor Vehicle Crashes: Overview. Traffic Safety Factors},
	Year = {2013}
}

@inproceedings{karnadi2007rapid,
	title={Rapid generation of realistic mobility models for VANET},
	author={Karnadi, Feliz Kristianto and Mo, Zhi Hai and Lan, Kun-chan},
	booktitle={Wireless Communications and Networking Conference, 2007. WCNC 2007. IEEE},
	pages={2506--2511},
	year={2007},
	organization={IEEE}
}
@inproceedings{saha2004modeling,
	title={Modeling mobility for vehicular ad-hoc networks},
	author={Saha, Amit Kumar and Johnson, David B},
	booktitle={Proceedings of the 1st ACM international workshop on Vehicular ad hoc networks},
	pages={91--92},
	year={2004},
	organization={ACM}
}
@inproceedings{lee2006modeling,
	title={Modeling steady-state and transient behaviors of user mobility: formulation, analysis, and application},
	author={Lee, Jong-Kwon and Hou, Jennifer C},
	booktitle={Proceedings of the 7th ACM international symposium on Mobile ad hoc networking and computing},
	pages={85--96},
	year={2006},
	organization={ACM}
}
@inproceedings{yoon2006building,
	title={Building realistic mobility models from coarse-grained traces},
	author={Yoon, Jungkeun and Noble, Brian D and Liu, Mingyan and Kim, Minkyong},
	booktitle={Proceedings of the 4th international conference on Mobile systems, applications and services},
	pages={177--190},
	year={2006},
	organization={ACM}
}

@inproceedings{choffnes2005integrated,
	title={An integrated mobility and traffic model for vehicular wireless networks},
	author={Choffnes, David R and Bustamante, Fabi{\'a}n E},
	booktitle={Proceedings of the 2nd ACM international workshop on Vehicular ad hoc networks},
	pages={69--78},
	year={2005},
	organization={ACM}
}

@inproceedings{Qian2008Globecom,
	title={CA Secure VANET MAC Protocol for DSRC Applications},
	author={Yi, Q. and Lu, K. and Moyeri, N.{\'a}n E},
	booktitle={Proceedings of IEEE GLOBECOM 2008},
	pages={1--5},
	year={2008},
	organization={IEEE}
}





@inproceedings{naumov2006evaluation,
	title={An evaluation of inter-vehicle ad hoc networks based on realistic vehicular traces},
	author={Naumov, Valery and Baumann, Rainer and Gross, Thomas},
	booktitle={Proceedings of the 7th ACM international symposium on Mobile ad hoc networking and computing},
	pages={108--119},
	year={2006},
	organization={ACM}
}
@article{sommer2008progressing,
	title={Progressing toward realistic mobility models in VANET simulations},
	author={Sommer, Christoph and Dressler, Falko},
	journal={Communications Magazine, IEEE},
	volume={46},
	number={11},
	pages={132--137},
	year={2008},
	publisher={IEEE}
}




@inproceedings{mahajan2006urban,
	title={Urban mobility models for vanets},
	author={Mahajan, Atulya and Potnis, Niranjan and Gopalan, Kartik and Wang, Andy},
	booktitle={2nd IEEE International Workshop on Next Generation Wireless Networks},
	volume={33},
	year={2006}
}

@inproceedings{Rakhshan2016packet,
	title={Packet success probability derivation in a vehicular ad hoc network for a highway scenario},
	author={Rakhshan, Ali and Pishro-Nik, Hossein},
	booktitle={2016 Annual Conference on Information Science and Systems (CISS)},
	pages={210--215},
	year={2016},
	organization={IEEE}
}

@inproceedings{Rakhshan2016CISS,
	Author = {Rakhshan, Ali and Pishro-Nik, Hossein},
	Booktitle = {Conference on Information Sciences and Systems},
	Organization = {IEEE},
	Pages = {210--215},
	Title = {Packet Success Probability Derivation in a Vehicular Ad Hoc Network for a Highway Scenario},
	Year = {2016}}

@article{Nekoui2013Journal,
	Author = {Nekoui, Mohammad and Pishro-Nik, Hossein},
	Journal = {Journal on Selected Areas in Communications, Special Issue on Emerging Technologies in Communications},
	Number = {9},
	Pages = {491--503},
	Publisher = {IEEE},
	Title = {Analytic Design of Active Safety Systems for Vehicular Ad hoc Networks},
	Volume = {31},
	Year = {2013}}


@inproceedings{Nekoui2011MOBICOM,
	Author = {Nekoui, Mohammad and Pishro-Nik, Hossein},
	Booktitle = {MOBICOM workshop on VehiculAr InterNETworking},
	Organization = {ACM},
	Title = {Analytic Design of Active Vehicular Safety Systems in Sparse Traffic},
	Year = {2011}}

@inproceedings{Nekoui2011VTC,
	Author = {Nekoui, Mohammad and Pishro-Nik, Hossein},
	Booktitle = {VTC-Fall},
	Organization = {IEEE},
	Title = {Analytical Design of Inter-vehicular Communications for Collision Avoidance},
	Year = {2011}}

@inproceedings{Bovee2011VTC,
	Author = {Bovee, Ben Louis and Nekoui, Mohammad and Pishro-Nik, Hossein},
	Booktitle = {VTC-Fall},
	Organization = {IEEE},
	Title = {Evaluation of the Universal Geocast Scheme For VANETs},
	Year = {2011}}

@inproceedings{Nekoui2010MOBICOM,
	Author = {Nekoui, Mohammad and Pishro-Nik, Hossein},
	Booktitle = {MOBICOM},
	Organization = {ACM},
	Title = {Fundamental Tradeoffs in Vehicular Ad Hoc Networks},
	Year = {2010}}

@inproceedings{Nekoui2010IVCS,
	Author = {Nekoui, Mohammad and Pishro-Nik, Hossein},
	Booktitle = {IVCS},
	Organization = {IEEE},
	Title = {A Universal Geocast Scheme for Vehicular Ad Hoc Networks},
	Year = {2010}}

@inproceedings{Nekoui2009ITW,
	Author = {Nekoui, Mohammad and Pishro-Nik, Hossein},
	Booktitle = {IEEE Communications Society Conference on Sensor, Mesh and Ad Hoc Communications and Networks Workshops},
	Organization = {IEEE},
	Pages = {1--3},
	Title = {A Geometrical Analysis of Obstructed Wireless Networks},
	Year = {2009}}

@article{Eslami2013Journal,
	Author = {Eslami, Ali and Nekoui, Mohammad and Pishro-Nik, Hossein and Fekri, Faramarz},
	Journal = {ACM Transactions on Sensor Networks},
	Number = {4},
	Pages = {51},
	Publisher = {ACM},
	Title = {Results on finite wireless sensor networks: Connectivity and coverage},
	Volume = {9},
	Year = {2013}}


@article{Jiafu2014Journal,
	Author = {Jiafu, W. and Zhang, D. and Zhao, S. and Yang, L. and Lloret, J.},
	Journal = {Communications Magazine},
	Number = {8},
	Pages = {106-113},
	Publisher = {IEEE},
	Title = {Context-aware vehicular cyber-physical systems with cloud support: architecture, challenges, and solutions},
	Volume = {52},
	Year = {2014}}

@inproceedings{Haas2010ACM,
	Author = {Haas, J.J. and Hu, Y.},
	Booktitle = {international workshop on VehiculAr InterNETworking},
	Organization = {ACM},
	Title = {Communication requirements for crash avoidance.},
	Year = {2010}}

@inproceedings{Yi2008GLOBECOM,
	Author = {Yi, Q. and Lu, K. and Moayeri, N.},
	Booktitle = {GLOBECOM},
	Organization = {IEEE},
	Title = {CA Secure VANET MAC Protocol for DSRC Applications.},
	Year = {2008}}

@inproceedings{Mughal2010ITSim,
	Author = {Mughal, B.M. and Wagan, A. and Hasbullah, H.},
	Booktitle = {International Symposium on Information Technology (ITSim)},
	Organization = {IEEE},
	Title = {Efficient congestion control in VANET for safety messaging.},
	Year = {2010}}

@article{Chang2011Journal,
	Author = {Chang, Y. and Lee, C. and Copeland, J.},
	Journal = {Selected Areas in Communications},
	Pages = {236 –249},
	Publisher = {IEEE},
	Title = {Goodput enhancement of VANETs in noisy CSMA/CA channels},
	Volume = {29},
	Year = {2011}}

@article{Garcia-Costa2011Journal,
	Author = {Garcia-Costa, C. and Egea-Lopez, E. and Tomas-Gabarron, J. B. and Garcia-Haro, J. and Haas, Z. J.},
	Journal = {Transactions on Intelligent Transportation Systems},
	Pages = {1 –16},
	Publisher = {IEEE},
	Title = {A stochastic model for chain collisions of vehicles equipped with vehicular communications},
	Volume = {99},
	Year = {2011}}

@article{Carbaugh2011Journal,
	Author = {Carbaugh, J. and Godbole,  D. N. and Sengupta, R. and Garcia-Haro, J. and Haas, Z. J.},
	Publisher = {Transportation Research Part C (Emerging Technologies)},
	Title = {Safety and capacity analysis of automated and manual highway systems},
	Year = {1997}}

@article{Goh2004Journal,
	Author = {Goh, P. and Wong, Y.},
	Publisher = {Appl Health Econ Health Policy},
	Title = {Driver perception response time during the signal change interval},
	Year = {2004}}

@article{Chang1985Journal,
	Author = {Chang, M.S. and Santiago, A.J.},
	Pages = {20-30},
	Publisher = {Transportation Research Record},
	Title = {Timing traffic signal changes based on driver behavior},
	Volume = {1027},
	Year = {1985}}

@article{Green2000Journal,
	Author = {Green, M.},
	Pages = {195-216},
	Publisher = {Transportation Human Factors},
	Title = {How long does it take to stop? Methodological analysis of driver perception-brake times.},
	Year = {2000}}

@article{Koppa2005,
	Author = {Koppa, R.J.},
	Pages = {195-216},
	Publisher = {http://www.fhwa.dot.gov/publications/},
	Title = {Human Factors},
	Year = {2005}}

@inproceedings{Maxwell2010ETC,
	Author = {Maxwell, A. and Wood, K.},
	Booktitle = {Europian Transport Conference},
	Organization = {http://www.etcproceedings.org/paper/review-of-traffic-signals-on-high-speed-roads},
	Title = {Review of Traffic Signals on High Speed Road},
	Year = {2010}}

@article{Wortman1983,
	Author = {Wortman, R.H. and Matthias, J.S.},
	Publisher = {Arizona Department of Transportation},
	Title = {An Evaluation of Driver Behavior at Signalized Intersections},
	Year = {1983}}
@inproceedings{Zhang2007IASTED,
	Author = {Zhang, X. and Bham, G.H.},
	Booktitle = {18th IASTED International Conference: modeling and simulation},
	Title = {Estimation of driver reaction time from detailed vehicle trajectory data.},
	Year = {2007}}


@inproceedings{bai2003important,
	title={IMPORTANT: A framework to systematically analyze the Impact of Mobility on Performance of RouTing protocols for Adhoc NeTworks},
	author={Bai, Fan and Sadagopan, Narayanan and Helmy, Ahmed},
	booktitle={INFOCOM 2003. Twenty-second annual joint conference of the IEEE computer and communications. IEEE societies},
	volume={2},
	pages={825--835},
	year={2003},
	organization={IEEE}
}


@inproceedings{abedi2008enhancing,
	title={Enhancing AODV routing protocol using mobility parameters in VANET},
	author={Abedi, Omid and Fathy, Mahmood and Taghiloo, Jamshid},
	booktitle={Computer Systems and Applications, 2008. AICCSA 2008. IEEE/ACS International Conference on},
	pages={229--235},
	year={2008},
	organization={IEEE}
}


@article{AlSultan2013Journal,
	Author = {Al-Sultan, Saif and Al-Bayatti, Ali H. and Zedan, Hussien},
	Journal = {IEEE Transactions on Vehicular Technology},
	Number = {9},
	Pages = {4264-4275},
	Publisher = {IEEE},
	Title = {Context Aware Driver Behaviour Detection System in Intelligent Transportation Systems},
	Volume = {62},
	Year = {2013}}






@article{Leow2008ITS,
	Author = {Leow, Woei Ling and Ni, Daiheng and Pishro-Nik, Hossein},
	Journal = {IEEE Transactions on Intelligent Transportation Systems},
	Number = {2},
	Pages = {369--374},
	Publisher = {IEEE},
	Title = {A Sampling Theorem Approach to Traffic Sensor Optimization},
	Volume = {9},
	Year = {2008}}



@article{REU2007,
	Author = {D. Ni and H. Pishro-Nik and R. Prasad and M. R. Kanjee and H. Zhu and T. Nguyen},
	Journal = {in 14th World Congress on Intelligent Transport Systems},
	Title = {Development of a prototype intersection collision avoidance system under VII},
	Year = {2007}}




@inproceedings{salamatian2013hide,
	title={How to hide the elephant-or the donkey-in the room: Practical privacy against statistical inference for large data.},
	author={Salamatian, Salman and Zhang, Amy and du Pin Calmon, Flavio and Bhamidipati, Sandilya and Fawaz, Nadia and Kveton, Branislav and Oliveira, Pedro and Taft, Nina},
	booktitle={GlobalSIP},
	pages={269--272},
	year={2013}
}

@article{sankar2013utility,
	title={Utility-privacy tradeoffs in databases: An information-theoretic approach},
	author={Sankar, Lalitha and Rajagopalan, S Raj and Poor, H Vincent},
	journal={Information Forensics and Security, IEEE Transactions on},
	volume={8},
	number={6},
	pages={838--852},
	year={2013},
	publisher={IEEE}
}
@inproceedings{ghinita2007prive,
	title={PRIVE: anonymous location-based queries in distributed mobile systems},
	author={Ghinita, Gabriel and Kalnis, Panos and Skiadopoulos, Spiros},
	booktitle={Proceedings of the 16th international conference on World Wide Web},
	pages={371--380},
	year={2007},
	organization={ACM}
}

@article{beresford2004mix,
	title={Mix zones: User privacy in location-aware services},
	author={Beresford, Alastair R and Stajano, Frank},
	year={2004},
	publisher={IEEE}
}

%@inproceedings{Mont1610Achieving,
	%  title={Achieving Perfect Location Privacy in Markov Models Using Anonymization},
	%  author={Montazeri, Zarrin and Houmansadr, Amir and H.Pishro-Nik},
	%  booktitle="2016 International Symposium on Information Theory and its Applications
	%  (ISITA2016)",
	%  address="Monterey, USA",
	%  days=30,
	%  month=oct,
	%  year=2016,
	%}

@article{csiszar1996almost,
	title={Almost independence and secrecy capacity},
	author={Csisz{\'a}r, Imre},
	journal={Problemy Peredachi Informatsii},
	volume={32},
	number={1},
	pages={48--57},
	year={1996},
	publisher={Russian Academy of Sciences, Branch of Informatics, Computer Equipment and Automatization}
}

@article{yamamoto1983source,
	title={A source coding problem for sources with additional outputs to keep secret from the receiver or wiretappers (corresp.)},
	author={Yamamoto, Hirosuke},
	journal={IEEE Transactions on Information Theory},
	volume={29},
	number={6},
	pages={918--923},
	year={1983},
	publisher={IEEE}
}


@inproceedings{calmon2015fundamental,
	title={Fundamental limits of perfect privacy},
	author={Calmon, Flavio P and Makhdoumi, Ali and M{\'e}dard, Muriel},
	booktitle={Information Theory (ISIT), 2015 IEEE International Symposium on},
	pages={1796--1800},
	year={2015},
	organization={IEEE}
}



@inproceedings{Lehman1999Large-Sample-Theory,
	title={Elements of Large Sample Theory},
	author={E. L. Lehman},
	organization={Springer},
	year={1999}
}


@inproceedings{Ferguson1999Large-Sample-Theory,
	title={A Course in Large Sample Theory},
	author={Thomas S. Ferguson},
	organization={CRC Press},
	year={1996}
}



@inproceedings{Dembo1999Large-Deviations,
	title={Large Deviation Techniques and Applications, Second Edition},
	author={A. Dembo and O. Zeitouni},
	organization={Springer},
	year={1998}
}


%%%%%%%%%%%%%%%%%%%%%%%%%%%%%%%%%%%%%%%%%%%%%%%%
Hossein's Coding Journals
%%%%%%%%%%%%%%%%%%%%%%

@ARTICLE{myoptics,
	AUTHOR =       "H. Pishro-Nik and N. Rahnavard and J. Ha and F. Fekri and A. Adibi ",
	TITLE =        "Low-density parity-check codes for volume holographic memory systems",
	JOURNAL =      " Appl. Opt.",
	YEAR =         "2003",
	volume =       "42",
	pages =        "861-870  "
}






@ARTICLE{myit,
	AUTHOR =       "H. Pishro-Nik and F. Fekri  ",
	TITLE =        "On Decoding of Low-Density Parity-Check Codes on the Binary Erasure Channel",
	JOURNAL =      "IEEE Trans. Inform. Theory",
	YEAR =         "2004",
	volume =       "50",
	pages =        "439--454"
}




@ARTICLE{myitpuncture,
	AUTHOR =       "H. Pishro-Nik and F. Fekri  ",
	TITLE =        "Results on Punctured Low-Density Parity-Check Codes and Improved Iterative Decoding Techniques",
	JOURNAL =      "IEEE Trans. on Inform. Theory",
	YEAR =         "2007",
	volume =       "53",
	number=        "2",
	pages =        "599--614",
	month= "February"
}




@ARTICLE{myitlinmimdist,
	AUTHOR =       "H. Pishro-Nik and F. Fekri",
	TITLE =        "Performance of Low-Density Parity-Check Codes With Linear Minimum Distance",
	JOURNAL =         "IEEE Trans. Inform. Theory ",
	YEAR =         "2006",
	volume =       "52",
	number="1",
	pages =        "292 --300"
}






@ARTICLE{myitnonuni,
	AUTHOR =       "H. Pishro-Nik and N. Rahnavard and F. Fekri  ",
	TITLE =        "Non-uniform Error Correction Using Low-Density Parity-Check Codes",
	JOURNAL =      "IEEE Trans. Inform. Theory",
	YEAR =         "2005",
	volume =       "51",
	number=  "7",
	pages =        "2702--2714"
}





@article{eslamitcomhybrid10,
	author = {A. Eslami and S. Vangala and H. Pishro-Nik},
	title = {Hybrid channel codes for highly efficient FSO/RF communication systems},
	journal = {IEEE Transactions on Communications},
	volume = {58},
	number = {10},
	year = {2010},
	pages = {2926--2938},
}


@article{eslamitcompolar13,
	author = {A. Eslami and H. Pishro-Nik},
	title = {On Finite-Length Performance of Polar Codes: Stopping Sets, Error Floor, and Concatenated Design},
	journal = {IEEE Transactions on Communications},
	volume = {61},
	number = {13},
	year = {2013},
	pages = {919--929},
}



@article{saeeditcom11,
	author = {H. Saeedi and H. Pishro-Nik and  A. H. Banihashemi},
	title = {Successive maximization for the systematic design of universally capacity approaching rate-compatible
	sequences of LDPC code ensembles over binary-input output-symmetric memoryless channels},
	journal = {IEEE Transactions on Communications},
	year = {2011},
	volume={59},
	number = {7}
}


@article{rahnavard07,
	author = {Rahnavard, N. and Pishro-Nik, H. and Fekri, F.},
	title = {Unequal Error Protection Using Partially Regular LDPC Codes},
	journal = {IEEE Transactions on Communications},
	year = {2007},
	volume = {55},
	number = {3},
	pages = {387 -- 391}
}


@article{hosseinira04,
	author = {H. Pishro-Nik and F. Fekri},
	title = {Irregular repeat-accumulate codes for volume holographic memory systems},
	journal = {Journal of Applied Optics},
	year = {2004},
	volume = {43},
	number = {27},
	pages = {5222--5227},
}


@article{azadeh2015Ephemeralkey,
	author = {A. Sheikholeslami and D. Goeckel and H. Pishro-Nik},
	title = {Jamming Based on an Ephemeral Key to Obtain Everlasting Security in Wireless Environments},
	journal = {IEEE Transactions on Wireless Communications},
	year = {2015},
	volume = {14},
	number = {11},
	pages = {6072--6081},
}


@article{azadeh2014Everlasting,
	author = {A. Sheikholeslami and D. Goeckel and H. Pishro-Nik},
	title = {Everlasting secrecy in disadvantaged wireless environments against sophisticated eavesdroppers},
	journal = {48th Asilomar Conference on Signals, Systems and Computers},
	year = {2014},
	pages = {1994--1998},
}


@article{azadeh2013ISIT,
	author = {A. Sheikholeslami and D. Goeckel and H. Pishro-Nik},
	title = {Artificial intersymbol interference (ISI) to exploit receiver imperfections for secrecy},
	journal = {IEEE International Symposium on Information Theory (ISIT)},
	year = {2013},
}


@article{azadeh2013Jsac,
	author = {A. Sheikholeslami and D. Goeckel and H. Pishro-Nik},
	title = {Jamming Based on an Ephemeral Key to Obtain Everlasting Security in Wireless Environments},
	journal = {IEEE Journal on Selected Areas in Communications},
	year = {2013},
	volume = {31},
	number = {9},
	pages = {1828--1839},
}


@article{azadeh2012Allerton,
	author = {A. Sheikholeslami and D. Goeckel and H. Pishro-Nik},
	title = {Exploiting the non-commutativity of nonlinear operators for information-theoretic security in disadvantaged wireless environments},
	journal = {50th Annual Allerton Conference on Communication, Control, and Computing},
	year = {2012},
	pages = {233--240},
}


@article{azadeh2012Infocom,
	author = {A. Sheikholeslami and D. Goeckel and H. Pishro-Nik},
	title = {Jamming Based on an Ephemeral Key to Obtain Everlasting Security in Wireless Environments},
	journal = {IEEE INFOCOM},
	year = {2012},
	pages = {1179--1187},
}

@article{1corser2016evaluating,
	title={Evaluating Location Privacy in Vehicular Communications and Applications},
	author={Corser, George P and Fu, Huirong and Banihani, Abdelnasser},
	journal={IEEE Transactions on Intelligent Transportation Systems},
	volume={17},
	number={9},
	pages={2658-2667},
	year={2016},
	publisher={IEEE}
}
@article{2zhang2016designing,
	title={On Designing Satisfaction-Ratio-Aware Truthful Incentive Mechanisms for k-Anonymity Location Privacy},
	author={Zhang, Yuan and Tong, Wei and Zhong, Sheng},
	journal={IEEE Transactions on Information Forensics and Security},
	volume={11},
	number={11},
	pages={2528--2541},
	year={2016},
	publisher={IEEE}
}
@article{3li2016privacy,
	title={Privacy-preserving Location Proof for Securing Large-scale Database-driven Cognitive Radio Networks},
	author={Li, Yi and Zhou, Lu and Zhu, Haojin and Sun, Limin},
	journal={IEEE Internet of Things Journal},
	volume={3},
	number={4},
	pages={563-571},
	year={2016},
	publisher={IEEE}
}
@article{4olteanu2016quantifying,
	title={Quantifying Interdependent Privacy Risks with Location Data},
	author={Olteanu, Alexandra-Mihaela and Huguenin, K{\'e}vin and Shokri, Reza and Humbert, Mathias and Hubaux, Jean-Pierre},
	journal={IEEE Transactions on Mobile Computing},
	year={2016},
	volume={PP},
	number={99},
	pages={1-1},
	publisher={IEEE}
}
@article{5yi2016practical,
	title={Practical Approximate k Nearest Neighbor Queries with Location and Query Privacy},
	author={Yi, Xun and Paulet, Russell and Bertino, Elisa and Varadharajan, Vijay},
	journal={IEEE Transactions on Knowledge and Data Engineering},
	volume={28},
	number={6},
	pages={1546--1559},
	year={2016},
	publisher={IEEE}
}
@article{6li2016privacy,
	title={Privacy Leakage of Location Sharing in Mobile Social Networks: Attacks and Defense},
	author={Li, Huaxin and Zhu, Haojin and Du, Suguo and Liang, Xiaohui and Shen, Xuemin},
	journal={IEEE Transactions on Dependable and Secure Computing},
	year={2016},
	volume={PP},
	number={99},
	publisher={IEEE}
}

@article{7murakami2016localization,
	title={Localization Attacks Using Matrix and Tensor Factorization},
	author={Murakami, Takao and Watanabe, Hajime},
	journal={IEEE Transactions on Information Forensics and Security},
	volume={11},
	number={8},
	pages={1647--1660},
	year={2016},
	publisher={IEEE}
}
@article{8zurbaran2015near,
	title={Near-Rand: Noise-based Location Obfuscation Based on Random Neighboring Points},
	author={Zurbaran, Mayra Alejandra and Avila, Karen and Wightman, Pedro and Fernandez, Michael},
	journal={IEEE Latin America Transactions},
	volume={13},
	number={11},
	pages={3661--3667},
	year={2015},
	publisher={IEEE}
}

@article{9tan2014anti,
	title={An anti-tracking source-location privacy protection protocol in wsns based on path extension},
	author={Tan, Wei and Xu, Ke and Wang, Dan},
	journal={IEEE Internet of Things Journal},
	volume={1},
	number={5},
	pages={461--471},
	year={2014},
	publisher={IEEE}
}

@article{10peng2014enhanced,
	title={Enhanced Location Privacy Preserving Scheme in Location-Based Services},
	author={Peng, Tao and Liu, Qin and Wang, Guojun},
	journal={IEEE Systems Journal},
	year={2014},
	volume={PP},
	number={99},
	pages={1-12},
	publisher={IEEE}
}
@article{11dewri2014exploiting,
	title={Exploiting service similarity for privacy in location-based search queries},
	author={Dewri, Rinku and Thurimella, Ramakrisha},
	journal={IEEE Transactions on Parallel and Distributed Systems},
	volume={25},
	number={2},
	pages={374--383},
	year={2014},
	publisher={IEEE}
}

@article{12hwang2014novel,
	title={A novel time-obfuscated algorithm for trajectory privacy protection},
	author={Hwang, Ren-Hung and Hsueh, Yu-Ling and Chung, Hao-Wei},
	journal={IEEE Transactions on Services Computing},
	volume={7},
	number={2},
	pages={126--139},
	year={2014},
	publisher={IEEE}
}
@article{13puttaswamy2014preserving,
	title={Preserving location privacy in geosocial applications},
	author={Puttaswamy, Krishna PN and Wang, Shiyuan and Steinbauer, Troy and Agrawal, Divyakant and El Abbadi, Amr and Kruegel, Christopher and Zhao, Ben Y},
	journal={IEEE Transactions on Mobile Computing},
	volume={13},
	number={1},
	pages={159--173},
	year={2014},
	publisher={IEEE}
}

@article{14zhang2014privacy,
	title={Privacy quantification model based on the Bayes conditional risk in Location-Based Services},
	author={Zhang, Xuejun and Gui, Xiaolin and Tian, Feng and Yu, Si and An, Jian},
	journal={Tsinghua Science and Technology},
	volume={19},
	number={5},
	pages={452--462},
	year={2014},
	publisher={TUP}
}

@article{15bilogrevic2014privacy,
	title={Privacy-preserving optimal meeting location determination on mobile devices},
	author={Bilogrevic, Igor and Jadliwala, Murtuza and Joneja, Vishal and Kalkan, K{\"u}bra and Hubaux, Jean-Pierre and Aad, Imad},
	journal={IEEE transactions on information forensics and security},
	volume={9},
	number={7},
	pages={1141--1156},
	year={2014},
	publisher={IEEE}
}
@article{16haghnegahdar2014privacy,
	title={Privacy Risks in Publishing Mobile Device Trajectories},
	author={Haghnegahdar, Alireza and Khabbazian, Majid and Bhargava, Vijay K},
	journal={IEEE Wireless Communications Letters},
	volume={3},
	number={3},
	pages={241--244},
	year={2014},
	publisher={IEEE}
}
@article{17malandrino2014verification,
	title={Verification and inference of positions in vehicular networks through anonymous beaconing},
	author={Malandrino, Francesco and Borgiattino, Carlo and Casetti, Claudio and Chiasserini, Carla-Fabiana and Fiore, Marco and Sadao, Roberto},
	journal={IEEE Transactions on Mobile Computing},
	volume={13},
	number={10},
	pages={2415--2428},
	year={2014},
	publisher={IEEE}
}
@article{18shokri2014hiding,
	title={Hiding in the mobile crowd: Locationprivacy through collaboration},
	author={Shokri, Reza and Theodorakopoulos, George and Papadimitratos, Panos and Kazemi, Ehsan and Hubaux, Jean-Pierre},
	journal={IEEE transactions on dependable and secure computing},
	volume={11},
	number={3},
	pages={266--279},
	year={2014},
	publisher={IEEE}
}
@article{19freudiger2013non,
	title={Non-cooperative location privacy},
	author={Freudiger, Julien and Manshaei, Mohammad Hossein and Hubaux, Jean-Pierre and Parkes, David C},
	journal={IEEE Transactions on Dependable and Secure Computing},
	volume={10},
	number={2},
	pages={84--98},
	year={2013},
	publisher={IEEE}
}
@article{20gao2013trpf,
	title={TrPF: A trajectory privacy-preserving framework for participatory sensing},
	author={Gao, Sheng and Ma, Jianfeng and Shi, Weisong and Zhan, Guoxing and Sun, Cong},
	journal={IEEE Transactions on Information Forensics and Security},
	volume={8},
	number={6},
	pages={874--887},
	year={2013},
	publisher={IEEE}
}
@article{21ma2013privacy,
	title={Privacy vulnerability of published anonymous mobility traces},
	author={Ma, Chris YT and Yau, David KY and Yip, Nung Kwan and Rao, Nageswara SV},
	journal={IEEE/ACM Transactions on Networking},
	volume={21},
	number={3},
	pages={720--733},
	year={2013},
	publisher={IEEE}
}
@article{22niu2013pseudo,
	title={Pseudo-Location Updating System for privacy-preserving location-based services},
	author={Niu, Ben and Zhu, Xiaoyan and Chi, Haotian and Li, Hui},
	journal={China Communications},
	volume={10},
	number={9},
	pages={1--12},
	year={2013},
	publisher={IEEE}
}
@article{23dewri2013local,
	title={Local differential perturbations: Location privacy under approximate knowledge attackers},
	author={Dewri, Rinku},
	journal={IEEE Transactions on Mobile Computing},
	volume={12},
	number={12},
	pages={2360--2372},
	year={2013},
	publisher={IEEE}
}
@inproceedings{24kanoria2012tractable,
	title={Tractable bayesian social learning on trees},
	author={Kanoria, Yashodhan and Tamuz, Omer},
	booktitle={Information Theory Proceedings (ISIT), 2012 IEEE International Symposium on},
	pages={2721--2725},
	year={2012},
	organization={IEEE}
}
@inproceedings{25farias2005universal,
	title={A universal scheme for learning},
	author={Farias, Vivek F and Moallemi, Ciamac C and Van Roy, Benjamin and Weissman, Tsachy},
	booktitle={Proceedings. International Symposium on Information Theory, 2005. ISIT 2005.},
	pages={1158--1162},
	year={2005},
	organization={IEEE}
}
@inproceedings{26misra2013unsupervised,
	title={Unsupervised learning and universal communication},
	author={Misra, Vinith and Weissman, Tsachy},
	booktitle={Information Theory Proceedings (ISIT), 2013 IEEE International Symposium on},
	pages={261--265},
	year={2013},
	organization={IEEE}
}
@inproceedings{27ryabko2013time,
	title={Time-series information and learning},
	author={Ryabko, Daniil},
	booktitle={Information Theory Proceedings (ISIT), 2013 IEEE International Symposium on},
	pages={1392--1395},
	year={2013},
	organization={IEEE}
}
@inproceedings{28krzakala2013phase,
	title={Phase diagram and approximate message passing for blind calibration and dictionary learning},
	author={Krzakala, Florent and M{\'e}zard, Marc and Zdeborov{\'a}, Lenka},
	booktitle={Information Theory Proceedings (ISIT), 2013 IEEE International Symposium on},
	pages={659--663},
	year={2013},
	organization={IEEE}
}
@inproceedings{29sakata2013sample,
	title={Sample complexity of Bayesian optimal dictionary learning},
	author={Sakata, Ayaka and Kabashima, Yoshiyuki},
	booktitle={Information Theory Proceedings (ISIT), 2013 IEEE International Symposium on},
	pages={669--673},
	year={2013},
	organization={IEEE}
}
@inproceedings{30predd2004consistency,
	title={Consistency in a model for distributed learning with specialists},
	author={Predd, Joel B and Kulkarni, Sanjeev R and Poor, H Vincent},
	booktitle={IEEE International Symposium on Information Theory},
	year={2004},
	organization={IEEE}
}
@inproceedings{31nokleby2016rate,
	title={Rate-Distortion Bounds on Bayes Risk in Supervised Learning},
	author={Nokleby, Matthew and Beirami, Ahmad and Calderbank, Robert},
	booktitle={2016 IEEE International Symposium on Information Theory (ISIT)},
	pages={2099-2103},
	year={2016},
	organization={IEEE}
}

@inproceedings{32le2016imperfect,
	title={Are imperfect reviews helpful in social learning?},
	author={Le, Tho Ngoc and Subramanian, Vijay G and Berry, Randall A},
	booktitle={Information Theory (ISIT), 2016 IEEE International Symposium on},
	pages={2089--2093},
	year={2016},
	organization={IEEE}
}
@inproceedings{33gadde2016active,
	title={Active Learning for Community Detection in Stochastic Block Models},
	author={Gadde, Akshay and Gad, Eyal En and Avestimehr, Salman and Ortega, Antonio},
	booktitle={2016 IEEE International Symposium on Information Theory (ISIT)},
	pages={1889-1893},
	year={2016}
}
@inproceedings{34shakeri2016minimax,
	title={Minimax Lower Bounds for Kronecker-Structured Dictionary Learning},
	author={Shakeri, Zahra and Bajwa, Waheed U and Sarwate, Anand D},
	booktitle={2016 IEEE International Symposium on Information Theory (ISIT)},
	pages={1148-1152},
	year={2016}
}
@article{35lee2015speeding,
	title={Speeding up distributed machine learning using codes},
	author={Lee, Kangwook and Lam, Maximilian and Pedarsani, Ramtin and Papailiopoulos, Dimitris and Ramchandran, Kannan},
	booktitle={2016 IEEE International Symposium on Information Theory (ISIT)},
	pages={1143-1147},
	year={2016}
}
@article{36oneto2016statistical,
	title={Statistical Learning Theory and ELM for Big Social Data Analysis},
	author={Oneto, Luca and Bisio, Federica and Cambria, Erik and Anguita, Davide},
	journal={ieee CompUTATionAl inTelliGenCe mAGAzine},
	volume={11},
	number={3},
	pages={45--55},
	year={2016},
	publisher={IEEE}
}
@article{37lin2015probabilistic,
	title={Probabilistic approach to modeling and parameter learning of indirect drive robots from incomplete data},
	author={Lin, Chung-Yen and Tomizuka, Masayoshi},
	journal={IEEE/ASME Transactions on Mechatronics},
	volume={20},
	number={3},
	pages={1036--1045},
	year={2015},
	publisher={IEEE}
}
@article{38wang2016towards,
	title={Towards Bayesian Deep Learning: A Framework and Some Existing Methods},
	author={Wang, Hao and Yeung, Dit-Yan},
	journal={IEEE Transactions on Knowledge and Data Engineering},
	volume={PP},
	number={99},
	year={2016},
	publisher={IEEE}
}


%%%%%Informationtheoreticsecurity%%%%%%%%%%%%%%%%%%%%%%%




@inproceedings{Bloch2011PhysicalSecBook,
	title={Physical-Layer Security},
	author={M. Bloch and J. Barros},
	organization={Cambridge University Press},
	year={2011}
}



@inproceedings{Liang2009InfoSecBook,
	title={Information Theoretic Security},
	author={Y. Liang and H. V. Poor and S. Shamai (Shitz)},
	organization={Now Publishers Inc.},
	year={2009}
}


@inproceedings{Zhou2013PhysicalSecBook,
	title={Physical Layer Security in Wireless Communications},
	author={ X. Zhou and L. Song and Y. Zhang},
	organization={CRC Press},
	year={2013}
}

@article{Ni2012IEA,
	Author = {D. Ni and H. Liu and W. Ding and  Y. Xie and H. Wang and H. Pishro-Nik and Q. Yu},
	Journal = {IEA/AIE},
	Title = {Cyber-Physical Integration to Connect Vehicles for Transformed Transportation Safety and Efficiency},
	Year = {2012}}



@inproceedings{Ni2012Inproceedings,
	Author = {D. Ni, H. Liu, Y. Xie, W. Ding, H. Wang, H. Pishro-Nik, Q. Yu and M. Ferreira},
	Booktitle = {Spring Simulation Multiconference},
	Date-Added = {2016-09-04 14:18:42 +0000},
	Date-Modified = {2016-09-06 16:22:14 +0000},
	Title = {Virtual Lab of Connected Vehicle Technology},
	Year = {2012}}

@inproceedings{Ni2012Inproceedings,
	Author = {D. Ni, H. Liu, W. Ding, Y. Xie, H. Wang, H. Pishro-Nik and Q. Yu,},
	Booktitle = {IEA/AIE},
	Date-Added = {2016-09-04 09:11:02 +0000},
	Date-Modified = {2016-09-06 14:46:53 +0000},
	Title = {Cyber-Physical Integration to Connect Vehicles for Transformed Transportation Safety and Efficiency},
	Year = {2012}}


@article{Nekoui_IJIPT_2009,
	Author = {M. Nekoui and D. Ni and H. Pishro-Nik and R. Prasad and M. Kanjee and H. Zhu and T. Nguyen},
	Journal = {International Journal of Internet Protocol Technology (IJIPT)},
	Number = {3},
	Pages = {},
	Publisher = {},
	Title = {Development of a VII-Enabled Prototype Intersection Collision Warning System},
	Volume = {4},
	Year = {2009}}


@inproceedings{Pishro_Ganz_Ni,
	Author = {H. Pishro-Nik, A. Ganz, and Daiheng Ni},
	Booktitle = {Forty-Fifth Annual Allerton Conference on Communication, Control, and Computing. Allerton House, Monticello, IL},
	Date-Added = {},
	Date-Modified = {},
	Number = {},
	Pages = {},
	Title = {The capacity of vehicular ad hoc networks},
	Volume = {},
	Year = {September 26-28, 2007}}

@inproceedings{Leow_Pishro_Ni_1,
	Author = {W. L. Leow, H. Pishro-Nik and Daiheng Ni},
	Booktitle = {IEEE Global Telecommunications Conference, Washington, D.C.},
	Date-Added = {},
	Date-Modified = {},
	Number = {},
	Pages = {},
	Title = {Delay and Energy Tradeoff in Multi-state Wireless Sensor Networks},
	Volume = {},
	Year = {November 26-30, 2007}}


@misc{UMass-Trans,
	title = {{UMass Transportation Center}},
	note = {\url{http://www.umasstransportationcenter.org/}},
}


@inproceedings{Haenggi2013book,
	title={Stochastic geometry for wireless networks},
	author={M. Haenggi},
	organization={Cambridge Uinversity Press},
	year={2013}
}


%% This BibTeX bibliography file was created using BibDesk.
%% http://bibdesk.sourceforge.net/

%% Created for Zarrin Montazeri at 2015-11-09 18:45:31 -0500


%% Saved with string encoding Unicode (UTF-8)


%%%%%%%%%%%%%%%%personalization%%%%%%%%%%%%%%%%%%%%%%%%%%%%%%%%%%

@article{osma2015,
	title={Impact of Time-to-Collision Information on Driving Behavior in Connected Vehicle Environments Using A Driving Simulator Test Bed},
	journal{Journal of Traffic and Logistics Engineering},
	author={Osama A. Osman, Julius Codjoe, and Sherif Ishak},
	volume={3},
	number={1},
	 pages={18--24},
	year={2015}
}


@article{charisma2010,
	title={Dynamic Latent Plan Models},
	author={Charisma F. Choudhurya, Moshe Ben-Akivab and Maya Abou-Zeid},
	journal={Journal of Choice Modelling},
	volume={3},
	number={2},
	pages={50--70},
	year={2010},
	publisher={Elsvier}
}


@misc{noble2014,
	author = {A. M. Noble, Shane B. McLaughlin, Zachary R. Doerzaph and Thomas A. Dingus},
	title = {Crowd-sourced Connected-vehicle Warning Algorithm using Naturalistic Driving Data},
	howpublished = {Downloaded from \url{http://hdl.handle.net/10919/53978}},

	month = August,
	year = 2014
}


@phdthesis{charisma2007,
	title    = {Modeling Driving Decisions with Latent Plans},
	school   = {Massachusetts Institute of Technology },
	author   = {Charisma Farheen Choudhury},
	year     = {2007}, %other attributes omitted
}


@article{chrysler2015,
	title={Cost of Warning of Unseen Threats:Unintended Consequences of Connected Vehicle Alerts},
	author={S. T. Chrysler, J. M. Cooper and D. C. Marshall},
	journal={Transportation Research Record: Journal of the Transportation Research Board},
	volume={2518},
	pages={79--85},
	year={2015},
}

@misc{nsf_cps,
        title = {Cyber-Physical Systems (CPS) PROGRAM SOLICITATION NSF 17-529},
        howpublished = {Downloaded from \url{https://www.nsf.gov/publications/pub_summ.jsp?WT.z_pims_id=503286&ods_key=nsf17529}},
}



%%%%%%%%%%%%%%IOT%%%%%%%%%%%%%%%%%%%%%%%%%%%%%%%%%%%%%%%%%%%%%%%%%%%




@article{FTC2015,
  title={Internet of Things: Privacy and Security in a Connected World},
  author={FTC Staff Report},
  year={2015}
}



@article{0Quest2016,
  title={The Quest for Privacy in the Internet of Things},
  author={ Pawani Porambag and Mika Ylianttila and Corinna Schmitt and  Pardeep Kumar and  Andrei Gurtov and  Athanasios V. Vasilakos},
  journal={IEEE Cloud Computing},
  volum={3},
  number={2},
  year={2016},
  publisher={IEEE}
}

%% Saved with string encoding Unicode (UTF-8)
@inproceedings{1zhou2014security,
  title={Security/privacy of wearable fitness tracking {I}o{T} devices},
  author={Zhou, Wei and Piramuthu, Selwyn},
  booktitle={Information Systems and Technologies (CISTI), 2014 9th Iberian Conference on},
  pages={1--5},
  year={2014},
  organization={IEEE}
}


@inproceedings{3ukil2014iot,
  title={{I}o{T}-privacy: To be private or not to be private},
  author={Ukil, Arijit and Bandyopadhyay, Soma and Pal, Arpan},
  booktitle={Computer Communications Workshops (INFOCOM WKSHPS), IEEE Conference on},
  pages={123--124},
  year={2014},
  organization={IEEE}
}


@article{4arias2015privacy,
  title={Privacy and security in internet of things and wearable devices},
  author={Arias, Orlando and Wurm, Jacob and Hoang, Khoa and Jin, Yier},
  journal={IEEE Transactions on Multi-Scale Computing Systems},
  volume={1},
  number={2},
  pages={99--109},
  year={2015},
  publisher={IEEE}
}
@inproceedings{5ullah2016novel,
  title={A novel model for preserving Location Privacy in Internet of Things},
  author={Ullah, Ikram and Shah, Munam Ali},
  booktitle={Automation and Computing (ICAC), 2016 22nd International Conference on},
  pages={542--547},
  year={2016},
  organization={IEEE}
}
@inproceedings{6sathishkumar2016enhanced,
  title={Enhanced location privacy algorithm for wireless sensor network in Internet of Things},
  author={Sathishkumar, J and Patel, Dhiren R},
  booktitle={Internet of Things and Applications (IOTA), International Conference on},
  pages={208--212},
  year={2016},
  organization={IEEE}
}
@inproceedings{7zhou2012preserving,
  title={Preserving sensor location privacy in internet of things},
  author={Zhou, Liming and Wen, Qiaoyan and Zhang, Hua},
  booktitle={Computational and Information Sciences (ICCIS), 2012 Fourth International Conference on},
  pages={856--859},
  year={2012},
  organization={IEEE}
}

@inproceedings{8ukil2015privacy,
  title={Privacy for {I}o{T}: Involuntary privacy enablement for smart energy systems},
  author={Ukil, Arijit and Bandyopadhyay, Soma and Pal, Arpan},
  booktitle={Communications (ICC), 2015 IEEE International Conference on},
  pages={536--541},
  year={2015},
  organization={IEEE}
}

@inproceedings{9dalipi2016security,
  title={Security and Privacy Considerations for {I}o{T} Application on Smart Grids: Survey and Research Challenges},
  author={Dalipi, Fisnik and Yayilgan, Sule Yildirim},
  booktitle={Future Internet of Things and Cloud Workshops (FiCloudW), IEEE International Conference on},
  pages={63--68},
  year={2016},
  organization={IEEE}
}
@inproceedings{10harris2016security,
  title={Security and Privacy in Public {I}o{T} Spaces},
  author={Harris, Albert F and Sundaram, Hari and Kravets, Robin},
  booktitle={Computer Communication and Networks (ICCCN), 2016 25th International Conference on},
  pages={1--8},
  year={2016},
  organization={IEEE}
}

@inproceedings{11al2015security,
  title={Security and privacy framework for ubiquitous healthcare {I}o{T} devices},
  author={Al Alkeem, Ebrahim and Yeun, Chan Yeob and Zemerly, M Jamal},
  booktitle={Internet Technology and Secured Transactions (ICITST), 2015 10th International Conference for},
  pages={70--75},
  year={2015},
  organization={IEEE}
}
@inproceedings{12sivaraman2015network,
  title={Network-level security and privacy control for smart-home {I}o{T} devices},
  author={Sivaraman, Vijay and Gharakheili, Hassan Habibi and Vishwanath, Arun and Boreli, Roksana and Mehani, Olivier},
  booktitle={Wireless and Mobile Computing, Networking and Communications (WiMob), 2015 IEEE 11th International Conference on},
  pages={163--167},
  year={2015},
  organization={IEEE}
}

@inproceedings{13srinivasan2016privacy,
  title={Privacy conscious architecture for improving emergency response in smart cities},
  author={Srinivasan, Ramya and Mohan, Apurva and Srinivasan, Priyanka},
  booktitle={Smart City Security and Privacy Workshop (SCSP-W), 2016},
  pages={1--5},
  year={2016},
  organization={IEEE}
}
@inproceedings{14sadeghi2015security,
  title={Security and privacy challenges in industrial internet of things},
  author={Sadeghi, Ahmad-Reza and Wachsmann, Christian and Waidner, Michael},
  booktitle={Design Automation Conference (DAC), 2015 52nd ACM/EDAC/IEEE},
  pages={1--6},
  year={2015},
  organization={IEEE}
}
@inproceedings{15otgonbayar2016toward,
  title={Toward Anonymizing {I}o{T} Data Streams via Partitioning},
  author={Otgonbayar, Ankhbayar and Pervez, Zeeshan and Dahal, Keshav},
  booktitle={Mobile Ad Hoc and Sensor Systems (MASS), 2016 IEEE 13th International Conference on},
  pages={331--336},
  year={2016},
  organization={IEEE}
}
@inproceedings{16rutledge2016privacy,
  title={Privacy Impacts of {I}o{T} Devices: A SmartTV Case Study},
  author={Rutledge, Richard L and Massey, Aaron K and Ant{\'o}n, Annie I},
  booktitle={Requirements Engineering Conference Workshops (REW), IEEE International},
  pages={261--270},
  year={2016},
  organization={IEEE}
}

@inproceedings{17andrea2015internet,
  title={Internet of Things: Security vulnerabilities and challenges},
  author={Andrea, Ioannis and Chrysostomou, Chrysostomos and Hadjichristofi, George},
  booktitle={Computers and Communication (ISCC), 2015 IEEE Symposium on},
  pages={180--187},
  year={2015},
  organization={IEEE}
}






























%%%%%%%%%%%%%%%%%%%%%%%%%%%%%%%%%%%%%%%%%%%%%%%%%%%%%%%%%%%


@misc{epfl-mobility-20090224,
    author = {Michal Piorkowski and Natasa Sarafijanovic-Djukic and Matthias Grossglauser},
    title = {{CRAWDAD} dataset epfl/mobility (v. 2009-02-24)},
    howpublished = {Downloaded from \url{http://crawdad.org/epfl/mobility/20090224}},
    doi = {10.15783/C7J010},
    month = feb,
    year = 2009
}

@misc{roma-taxi-20140717,
    author = {Lorenzo Bracciale and Marco Bonola and Pierpaolo Loreti and Giuseppe Bianchi and Raul Amici and Antonello Rabuffi},
    title = {{CRAWDAD} dataset roma/taxi (v. 2014-07-17)},
    howpublished = {Downloaded from \url{http://crawdad.org/roma/taxi/20140717}},
    doi = {10.15783/C7QC7M},
    month = jul,
    year = 2014
}

@misc{rice-ad_hoc_city-20030911,
    author = {Jorjeta G. Jetcheva and Yih-Chun Hu and Santashil PalChaudhuri and Amit Kumar Saha and David B. Johnson},
    title = {{CRAWDAD} dataset rice/ad\_hoc\_city (v. 2003-09-11)},
    howpublished = {Downloaded from \url{http://crawdad.org/rice/ad_hoc_city/20030911}},
    doi = {10.15783/C73K5B},
    month = sep,
    year = 2003
}

@misc{china:2012,
author = {Microsoft Research Asia},
title = {GeoLife GPS Trajectories},
year = {2012},
howpublished= {\url{https://www.microsoft.com/en-us/download/details.aspx?id=52367}},
}


@misc{china:2011,
ALTauthor = {Microsoft Research Asia)},
ALTeditor = {},
title = {GeoLife GPS Trajectories,
year = {2012},
url = {https://www.microsoft.com/en-us/download/details.aspx?id=52367},
}


@misc{longversion,
  author = {N. Takbiri and A. Houmansadr and D.L. Goeckel and H. Pishro-Nik},
  title = {{Limits of Location Privacy under Anonymization and Obfuscation}},
  howpublished = "\url{http://www.ecs.umass.edu/ece/pishro/Papers/ISIT_2017-2.pdf}",
  year = 2017,
month= "January",
  note = "Summarized version submitted to IEEE ISIT 2017"
}

@misc{isit_ke,
  author = {K. Li and D. Goeckel and H. Pishro-Nik},
  title = {{Bayesian Time Series Matching and Privacy}},
  note = "submitted to IEEE ISIT 2017"
}

@article{matching,
  title={Asymptotically Optimal Matching of Multiple Sequences to Source Distributions and Training Sequences},
  author={Jayakrishnan Unnikrishnan},
  journal={ IEEE Transactions on Information Theory},
  volume={61},
  number={1},
  pages={452-468},
  year={2015},
  publisher={IEEE}
}


@article{Naini2016,
	Author = {F. Naini and J. Unnikrishnan and P. Thiran and M. Vetterli},
	Journal = {IEEE Transactions on Information Forensics and Security},
	Publisher = {IEEE},
	Title = {Where You Are Is Who You Are: User Identification by Matching Statistics},
	 volume={11},
    number={2},
     pages={358--372},
    Year = {2016}
}



@inproceedings{holowczak2015cachebrowser,
  title={{CacheBrowser: Bypassing Chinese Censorship without Proxies Using Cached Content}},
  author={Holowczak, John and Houmansadr, Amir},
  booktitle={Proceedings of the 22nd ACM SIGSAC Conference on Computer and Communications Security},
  pages={70--83},
  year={2015},
  organization={ACM}
}
@misc{cb-website,
	Howpublished = {\url{https://cachebrowser.net/#/}},
	Title = {{CacheBrowser}},
	key={cachebrowser}
}

@inproceedings{GameOfDecoys,
 title={{GAME OF DECOYS: Optimal Decoy Routing Through Game Theory}},
 author={Milad Nasr and Amir Houmansadr},
 booktitle={The $23^{rd}$ ACM Conference on Computer and Communications Security (CCS)},
 year={2016}
}

@inproceedings{CDNReaper,
 title={{Practical Censorship Evasion Leveraging Content Delivery Networks}},
 author={Hadi Zolfaghari and Amir Houmansadr},
 booktitle={The $23^{rd}$ ACM Conference on Computer and Communications Security (CCS)},
 year={2016}
}

@misc{Leberknight2010,
	Author = {Leberknight, C. and Chiang, M. and Poor, H. and Wong, F.},
	Howpublished = {\url{http://www.princeton.edu/~chiangm/anticensorship.pdf}},
	Title = {{A Taxonomy of Internet Censorship and Anti-censorship}},
	Year = {2010}}

@techreport{ultrasurf-analysis,
	Author = {Appelbaum, Jacob},
	Institution = {The Tor Project},
	Title = {{Technical analysis of the Ultrasurf proxying software}},
	Url = {http://scholar.google.com/scholar?hl=en\&btnG=Search\&q=intitle:Technical+analysis+of+the+Ultrasurf+proxying+software\#0},
	Year = {2012},
	Bdsk-Url-1 = {http://scholar.google.com/scholar?hl=en%5C&btnG=Search%5C&q=intitle:Technical+analysis+of+the+Ultrasurf+proxying+software%5C#0}}

@misc{gifc:07,
	Howpublished = {\url{http://www.internetfreedom.org/archive/Defeat\_Internet\_Censorship\_White\_Paper.pdf}},
	Key = {defeatcensorship},
	Publisher = {Global Internet Freedom Consortium (GIFC)},
	Title = {{Defeat Internet Censorship: Overview of Advanced Technologies and Products}},
	Type = {White Paper},
	Year = {2007}}

@article{pan2011survey,
	Author = {Pan, J. and Paul, S. and Jain, R.},
	Journal = {Communications Magazine, IEEE},
	Number = {7},
	Pages = {26--36},
	Publisher = {IEEE},
	Title = {{A Survey of the Research on Future Internet Architectures}},
	Volume = {49},
	Year = {2011}}

@misc{nsf-fia,
	Howpublished = {\url{http://www.nets-fia.net/}},
	Key = {FIA},
	Title = {{NSF Future Internet Architecture Project}}}

@misc{NDN,
	Howpublished = {\url{http://www.named- data.net}},
	Key = {NDN},
	Title = {{Named Data Networking Project}}}

@inproceedings{MobilityFirst,
	Author = {Seskar, I. and Nagaraja, K. and Nelson, S. and Raychaudhuri, D.},
	Booktitle = {Asian Internet Engineering Conference},
	Title = {{Mobilityfirst Future internet Architecture Project}},
	Year = {2011}}

@incollection{NEBULA,
	Author = {Anderson, T. and Birman, K. and Broberg, R. and Caesar, M. and Comer, D. and Cotton, C. and Freedman, M.~J. and Haeberlen, A. and Ives, Z.~G. and Krishnamurthy, A. and others},
	Booktitle = {The Future Internet},
	Pages = {16--26},
	Publisher = {Springer},
	Title = {{The NEBULA Future Internet Architecture}},
	Year = {2013}}

@inproceedings{XIA,
	Author = {Anand, A. and Dogar, F. and Han, D. and Li, B. and Lim, H. and Machado, M. and Wu, W. and Akella, A. and Andersen, D.~G. and Byers, J.~W. and others},
	Booktitle = {ACM Workshop on Hot Topics in Networks},
	Pages = {2},
	Title = {{XIA: An Architecture for an Evolvable and Trustworthy Internet}},
	Year = {2011}}

@inproceedings{ChoiceNet,
	Author = {Rouskas, G.~N. and Baldine, I. and Calvert, K.~L. and Dutta, R. and Griffioen, J. and Nagurney, A. and Wolf, T.},
	Booktitle = {ONDM},
	Title = {{ChoiceNet: Network Innovation Through Choice}},
	Year = {2013}}

@misc{nsf-find,
	Howpublished = {http://www.nets-find.net/},
	Title = {{NSF NeTS FIND Initiative}}}

@article{traid,
	Author = {Cheriton, D.~R. and Gritter, M.},
	Title = {{TRIAD: A New Next-Generation Internet Architecture}},
	Year = {2000}}

@inproceedings{dona,
	Author = {Koponen, T. and Chawla, M. and Chun, B-G. and Ermolinskiy, A. and Kim, K.~H. and Shenker, S. and Stoica, I.},
	Booktitle = {ACM SIGCOMM Computer Communication Review},
	Number = {4},
	Organization = {ACM},
	Pages = {181--192},
	Title = {{A Data-Oriented (and Beyond) Network Architecture}},
	Volume = {37},
	Year = {2007}}

@misc{ultrasurf,
	Howpublished = {\url{http://www.ultrareach.com}},
	Key = {ultrasurf},
	Title = {{Ultrasurf}}}

@misc{tor-bridge,
	Author = {Dingledine, R. and Mathewson, N.},
	Howpublished = {\url{https://svn.torproject.org/svn/projects/design-paper/blocking.html}},
	Title = {{Design of a Blocking-Resistant Anonymity System}}}

@inproceedings{McLachlanH09,
	Author = {J. McLachlan and N. Hopper},
	Booktitle = {WPES},
	Title = {{On the Risks of Serving Whenever You Surf: Vulnerabilities in Tor's Blocking Resistance Design}},
	Year = {2009}}

@inproceedings{mahdian2010,
	Author = {Mahdian, M.},
	Booktitle = {{Fun with Algorithms}},
	Title = {{Fighting Censorship with Algorithms}},
	Year = {2010}}

@inproceedings{McCoy2011,
	Author = {McCoy, D. and Morales, J.~A. and Levchenko, K.},
	Booktitle = {FC},
	Title = {{Proximax: A Measurement Based System for Proxies Dissemination}},
	Year = {2011}}

@inproceedings{Sovran2008,
	Author = {Sovran, Y. and Libonati, A. and Li, J.},
	Booktitle = {IPTPS},
	Title = {{Pass it on: Social Networks Stymie Censors}},
	Year = {2008}}

@inproceedings{rbridge,
	Author = {Wang, Q. and Lin, Zi and Borisov, N. and Hopper, N.},
	Booktitle = {{NDSS}},
	Title = {{rBridge: User Reputation based Tor Bridge Distribution with Privacy Preservation}},
	Year = {2013}}

@inproceedings{telex,
	Author = {Wustrow, E. and Wolchok, S. and Goldberg, I. and Halderman, J.},
	Booktitle = {{USENIX Security}},
	Title = {{Telex: Anticensorship in the Network Infrastructure}},
	Year = {2011}}

@inproceedings{cirripede,
	Author = {Houmansadr, A. and Nguyen, G. and Caesar, M. and Borisov, N.},
	Booktitle = {CCS},
	Title = {{Cirripede: Circumvention Infrastructure Using Router Redirection with Plausible Deniability}},
	Year = {2011}}

@inproceedings{decoyrouting,
	Author = {Karlin, J. and Ellard, D. and Jackson, A. and Jones, C. and Lauer, G. and Mankins, D. and Strayer, W.},
	Booktitle = {{FOCI}},
	Title = {{Decoy Routing: Toward Unblockable Internet Communication}},
	Year = {2011}}

@inproceedings{routing-around-decoys,
	Author = {M.~Schuchard and J.~Geddes and C.~Thompson and N.~Hopper},
	Booktitle = {{CCS}},
	Title = {{Routing Around Decoys}},
	Year = {2012}}

@inproceedings{parrot,
	Author = {A. Houmansadr and C. Brubaker and V. Shmatikov},
	Booktitle = {IEEE S\&P},
	Title = {{The Parrot is Dead: Observing Unobservable Network Communications}},
	Year = {2013}}

@misc{knock,
	Author = {T. Wilde},
	Howpublished = {\url{https://blog.torproject.org/blog/knock-knock-knockin-bridges-doors}},
	Title = {{Knock Knock Knockin' on Bridges' Doors}},
	Year = {2012}}

@inproceedings{china-tor,
	Author = {Winter, P. and Lindskog, S.},
	Booktitle = {{FOCI}},
	Title = {{How the Great Firewall of China Is Blocking Tor}},
	Year = {2012}}

@misc{discover-bridge,
	Howpublished = {\url{https://blog.torproject.org/blog/research-problems-ten-ways-discover-tor-bridges}},
	Key = {tenways},
	Title = {{Ten Ways to Discover Tor Bridges}}}

@inproceedings{freewave,
	Author = {A.~Houmansadr and T.~Riedl and N.~Borisov and A.~Singer},
	Booktitle = {{NDSS}},
	Title = {{I Want My Voice to Be Heard: IP over Voice-over-IP for Unobservable Censorship Circumvention}},
	Year = 2013}

@inproceedings{censorspoofer,
	Author = {Q. Wang and X. Gong and G. Nguyen and A. Houmansadr and N. Borisov},
	Booktitle = {CCS},
	Title = {{CensorSpoofer: Asymmetric Communication Using IP Spoofing for Censorship-Resistant Web Browsing}},
	Year = {2012}}

@inproceedings{skypemorph,
	Author = {H. Moghaddam and B. Li and M. Derakhshani and I. Goldberg},
	Booktitle = {CCS},
	Title = {{SkypeMorph: Protocol Obfuscation for Tor Bridges}},
	Year = {2012}}

@inproceedings{stegotorus,
	Author = {Weinberg, Z. and Wang, J. and Yegneswaran, V. and Briesemeister, L. and Cheung, S. and Wang, F. and Boneh, D.},
	Booktitle = {CCS},
	Title = {{StegoTorus: A Camouflage Proxy for the Tor Anonymity System}},
	Year = {2012}}

@techreport{dust,
	Author = {{B.~Wiley}},
	Howpublished = {\url{http://blanu.net/ Dust.pdf}},
	Institution = {School of Information, University of Texas at Austin},
	Title = {{Dust: A Blocking-Resistant Internet Transport Protocol}},
	Year = {2011}}

@inproceedings{FTE,
	Author = {K.~Dyer and S.~Coull and T.~Ristenpart and T.~Shrimpton},
	Booktitle = {CCS},
	Title = {{Protocol Misidentification Made Easy with Format-Transforming Encryption}},
	Year = {2013}}

@inproceedings{fp,
	Author = {Fifield, D. and Hardison, N. and Ellithrope, J. and Stark, E. and Dingledine, R. and Boneh, D. and Porras, P.},
	Booktitle = {PETS},
	Title = {{Evading Censorship with Browser-Based Proxies}},
	Year = {2012}}

@misc{obfsproxy,
	Howpublished = {\url{https://www.torproject.org/projects/obfsproxy.html.en}},
	Key = {obfsproxy},
	Publisher = {The Tor Project},
	Title = {{A Simple Obfuscating Proxy}}}

@inproceedings{Tor-instead-of-IP,
	Author = {Liu, V. and Han, S. and Krishnamurthy, A. and Anderson, T.},
	Booktitle = {HotNets},
	Title = {{Tor instead of IP}},
	Year = {2011}}

@misc{roger-slides,
	Howpublished = {\url{https://svn.torproject.org/svn/projects/presentations/slides-28c3.pdf}},
	Key = {torblocking},
	Title = {{How Governments Have Tried to Block Tor}}}

@inproceedings{infranet,
	Author = {Feamster, N. and Balazinska, M. and Harfst, G. and Balakrishnan, H. and Karger, D.},
	Booktitle = {USENIX Security},
	Title = {{Infranet: Circumventing Web Censorship and Surveillance}},
	Year = {2002}}

@inproceedings{collage,
	Author = {S.~Burnett and N.~Feamster and S.~Vempala},
	Booktitle = {USENIX Security},
	Title = {{Chipping Away at Censorship Firewalls with User-Generated Content}},
	Year = {2010}}

@article{anonymizer,
	Author = {Boyan, J.},
	Journal = {Computer-Mediated Communication Magazine},
	Month = sep,
	Number = {9},
	Title = {{The Anonymizer: Protecting User Privacy on the Web}},
	Volume = {4},
	Year = {1997}}

@article{schulze2009internet,
	Author = {Schulze, H. and Mochalski, K.},
	Journal = {IPOQUE Report},
	Pages = {351--362},
	Title = {Internet Study 2008/2009},
	Volume = {37},
	Year = {2009}}

@inproceedings{cya-ccs13,
	Author = {J.~Geddes and M.~Schuchard and N.~Hopper},
	Booktitle = {{CCS}},
	Title = {{Cover Your ACKs: Pitfalls of Covert Channel Censorship Circumvention}},
	Year = {2013}}

@inproceedings{andana,
	Author = {DiBenedetto, S. and Gasti, P. and Tsudik, G. and Uzun, E.},
	Booktitle = {{NDSS}},
	Title = {{ANDaNA: Anonymous Named Data Networking Application}},
	Year = {2012}}

@inproceedings{darkly,
	Author = {Jana, S. and Narayanan, A. and Shmatikov, V.},
	Booktitle = {IEEE S\&P},
	Title = {{A Scanner Darkly: Protecting User Privacy From Perceptual Applications}},
	Year = {2013}}

@inproceedings{NS08,
	Author = {A.~Narayanan and V.~Shmatikov},
	Booktitle = {IEEE S\&P},
	Title = {Robust de-anonymization of large sparse datasets},
	Year = {2008}}

@inproceedings{NS09,
	Author = {A.~Narayanan and V.~Shmatikov},
	Booktitle = {IEEE S\&P},
	Title = {De-anonymizing social networks},
	Year = {2009}}

@inproceedings{memento,
	Author = {Jana, S. and Shmatikov, V.},
	Booktitle = {IEEE S\&P},
	Title = {{Memento: Learning secrets from process footprints}},
	Year = {2012}}

@misc{plugtor,
	Howpublished = {\url{https://www.torproject.org/docs/pluggable-transports.html.en}},
	Key = {PluggableTransports},
	Publisher = {The Tor Project},
	Title = {{Tor: Pluggable transports}}}

@misc{psiphon,
	Author = {J.~Jia and P.~Smith},
	Howpublished = {\url{http://www.cdf.toronto.edu/~csc494h/reports/2004-fall/psiphon_ae.html}},
	Title = {{Psiphon: Analysis and Estimation}},
	Year = 2004}

@misc{china-github,
	Howpublished = {\url{http://mobile.informationweek.com/80269/show/72e30386728f45f56b343ddfd0fdb119/}},
	Key = {github},
	Title = {{China's GitHub Censorship Dilemma}}}

@inproceedings{txbox,
	Author = {Jana, S. and Porter, D. and Shmatikov, V.},
	Booktitle = {IEEE S\&P},
	Title = {{TxBox: Building Secure, Efficient Sandboxes with System Transactions}},
	Year = {2011}}

@inproceedings{airavat,
	Author = {I. Roy and S. Setty and A. Kilzer and V. Shmatikov and E. Witchel},
	Booktitle = {NSDI},
	Title = {{Airavat: Security and Privacy for MapReduce}},
	Year = {2010}}

@inproceedings{osdi12,
	Author = {A. Dunn and M. Lee and S. Jana and S. Kim and M. Silberstein and Y. Xu and V. Shmatikov and E. Witchel},
	Booktitle = {OSDI},
	Title = {{Eternal Sunshine of the Spotless Machine: Protecting Privacy with Ephemeral Channels}},
	Year = {2012}}

@inproceedings{ymal,
	Author = {J. Calandrino and A. Kilzer and A. Narayanan and E. Felten and V. Shmatikov},
	Booktitle = {IEEE S\&P},
	Title = {{``You Might Also Like:'' Privacy Risks of Collaborative Filtering}},
	Year = {2011}}

@inproceedings{srivastava11,
	Author = {V. Srivastava and M. Bond and K. McKinley and V. Shmatikov},
	Booktitle = {PLDI},
	Title = {{A Security Policy Oracle: Detecting Security Holes Using Multiple API Implementations}},
	Year = {2011}}

@inproceedings{chen-oakland10,
	Author = {Chen, S. and Wang, R. and Wang, X. and Zhang, K.},
	Booktitle = {IEEE S\&P},
	Title = {{Side-Channel Leaks in Web Applications: A Reality Today, a Challenge Tomorrow}},
	Year = {2010}}

@book{kerck,
	Author = {Kerckhoffs, A.},
	Publisher = {University Microfilms},
	Title = {{La cryptographie militaire}},
	Year = {1978}}

@inproceedings{foci11,
	Author = {J. Karlin and D. Ellard and A.~Jackson and C.~ Jones and G. Lauer and D. Mankins and W.~T.~Strayer},
	Booktitle = {FOCI},
	Title = {{Decoy Routing: Toward Unblockable Internet Communication}},
	Year = 2011}

@inproceedings{sun02,
	Author = {Sun, Q. and Simon, D.~R. and Wang, Y. and Russell, W. and Padmanabhan, V. and Qiu, L.},
	Booktitle = {IEEE S\&P},
	Title = {{Statistical Identification of Encrypted Web Browsing Traffic}},
	Year = {2002}}

@inproceedings{danezis,
	Author = {Murdoch, S.~J. and Danezis, G.},
	Booktitle = {IEEE S\&P},
	Title = {{Low-Cost Traffic Analysis of Tor}},
	Year = {2005}}

@inproceedings{pakicensorship,
	Author = {Z.~Nabi},
	Booktitle = {FOCI},
	Title = {The Anatomy of {Web} Censorship in {Pakistan}},
	Year = {2013}}

@inproceedings{irancensorship,
	Author = {S.~Aryan and H.~Aryan and A.~Halderman},
	Booktitle = {FOCI},
	Title = {Internet Censorship in {Iran}: {A} First Look},
	Year = {2013}}

@inproceedings{ford10efficient,
	Author = {Amittai Aviram and Shu-Chun Weng and Sen Hu and Bryan Ford},
	Booktitle = {\bibconf[9th]{OSDI}{USENIX Symposium on Operating Systems Design and Implementation}},
	Location = {Vancouver, BC, Canada},
	Month = oct,
	Title = {Efficient System-Enforced Deterministic Parallelism},
	Year = 2010}

@inproceedings{ford10determinating,
	Author = {Amittai Aviram and Sen Hu and Bryan Ford and Ramakrishna Gummadi},
	Booktitle = {\bibconf{CCSW}{ACM Cloud Computing Security Workshop}},
	Location = {Chicago, IL},
	Month = oct,
	Title = {Determinating Timing Channels in Compute Clouds},
	Year = 2010}

@inproceedings{ford12plugging,
	Author = {Bryan Ford},
	Booktitle = {\bibconf[4th]{HotCloud}{USENIX Workshop on Hot Topics in Cloud Computing}},
	Location = {Boston, MA},
	Month = jun,
	Title = {Plugging Side-Channel Leaks with Timing Information Flow Control},
	Year = 2012}

@inproceedings{ford12icebergs,
	Author = {Bryan Ford},
	Booktitle = {\bibconf[4th]{HotCloud}{USENIX Workshop on Hot Topics in Cloud Computing}},
	Location = {Boston, MA},
	Month = jun,
	Title = {Icebergs in the Clouds: the {\em Other} Risks of Cloud Computing},
	Year = 2012}

@misc{mullenize,
	Author = {Washington Post},
	Howpublished = {\url{http://apps.washingtonpost.com/g/page/world/gchq-report-on-mullenize-program-to-stain-anonymous-electronic-traffic/502/}},
	Month = {oct},
	Title = {{GCHQ} report on {`MULLENIZE'} program to `stain' anonymous electronic traffic},
	Year = {2013}}

@inproceedings{shue13street,
	Author = {Craig A. Shue and Nathanael Paul and Curtis R. Taylor},
	Booktitle = {\bibbrev[7th]{WOOT}{USENIX Workshop on Offensive Technologies}},
	Month = aug,
	Title = {From an {IP} Address to a Street Address: Using Wireless Signals to Locate a Target},
	Year = 2013}

@inproceedings{knockel11three,
	Author = {Jeffrey Knockel and Jedidiah R. Crandall and Jared Saia},
	Booktitle = {\bibbrev{FOCI}{USENIX Workshop on Free and Open Communications on the Internet}},
	Location = {San Francisco, CA},
	Month = aug,
	Year = 2011}

@misc{rfc4960,
	Author = {R. {Stewart, ed.}},
	Month = sep,
	Note = {RFC 4960},
	Title = {Stream Control Transmission Protocol},
	Year = 2007}

@inproceedings{ford07structured,
	Author = {Bryan Ford},
	Booktitle = {\bibbrev{SIGCOMM}{ACM SIGCOMM}},
	Location = {Kyoto, Japan},
	Month = aug,
	Title = {Structured Streams: a New Transport Abstraction},
	Year = {2007}}

@misc{spdy,
	Author = {Google, Inc.},
	Note = {\url{http://www.chromium.org/spdy/spdy-whitepaper}},
	Title = {{SPDY}: An Experimental Protocol For a Faster {Web}}}

@misc{quic,
	Author = {Jim Roskind},
	Month = jun,
	Note = {\url{http://blog.chromium.org/2013/06/experimenting-with-quic.html}},
	Title = {Experimenting with {QUIC}},
	Year = 2013}

@misc{podjarny12not,
	Author = {G.~Podjarny},
	Month = jun,
	Note = {\url{http://www.guypo.com/technical/not-as-spdy-as-you-thought/}},
	Title = {{Not as SPDY as You Thought}},
	Year = 2012}

@inproceedings{cor,
	Author = {Jones, N.~A. and Arye, M. and Cesareo, J. and Freedman, M.~J.},
	Booktitle = {FOCI},
	Title = {{Hiding Amongst the Clouds: A Proposal for Cloud-based Onion Routing}},
	Year = {2011}}

@misc{torcloud,
	Howpublished = {\url{https://cloud.torproject.org/}},
	Key = {tor cloud},
	Title = {{The Tor Cloud Project}}}

@inproceedings{scramblesuit,
	Author = {Philipp Winter and Tobias Pulls and Juergen Fuss},
	Booktitle = {WPES},
	Title = {{ScrambleSuit: A Polymorphic Network Protocol to Circumvent Censorship}},
	Year = 2013}

@article{savage2000practical,
	Author = {Savage, S. and Wetherall, D. and Karlin, A. and Anderson, T.},
	Journal = {ACM SIGCOMM Computer Communication Review},
	Number = {4},
	Pages = {295--306},
	Publisher = {ACM},
	Title = {Practical network support for IP traceback},
	Volume = {30},
	Year = {2000}}

@inproceedings{ooni,
	Author = {Filast, A. and Appelbaum, J.},
	Booktitle = {{FOCI}},
	Title = {{OONI : Open Observatory of Network Interference}},
	Year = {2012}}

@misc{caida-rank,
	Howpublished = {\url{http://as-rank.caida.org/}},
	Key = {caida rank},
	Title = {{AS Rank: AS Ranking}}}

@inproceedings{usersrouted-ccs13,
	Author = {A.~Johnson and C.~Wacek and R.~Jansen and M.~Sherr and P.~Syverson},
	Booktitle = {CCS},
	Title = {{Users Get Routed: Traffic Correlation on Tor by Realistic Adversaries}},
	Year = {2013}}

@inproceedings{edman2009awareness,
	Author = {Edman, M. and Syverson, P.},
	Booktitle = {{CCS}},
	Title = {{AS-awareness in Tor path selection}},
	Year = {2009}}

@inproceedings{DecoyCosts,
	Author = {A.~Houmansadr and E.~L.~Wong and V.~Shmatikov},
	Booktitle = {NDSS},
	Title = {{No Direction Home: The True Cost of Routing Around Decoys}},
	Year = {2014}}

@article{cordon,
	Author = {Elahi, T. and Goldberg, I.},
	Journal = {University of Waterloo CACR},
	Title = {{CORDON--A Taxonomy of Internet Censorship Resistance Strategies}},
	Volume = {33},
	Year = {2012}}

@inproceedings{privex,
	Author = {T.~Elahi and G.~Danezis and I.~Goldberg	},
	Booktitle = {{CCS}},
	Title = {{AS-awareness in Tor path selection}},
	Year = {2014}}

@inproceedings{changeGuards,
	Author = {T.~Elahi and K.~Bauer and M.~AlSabah and R.~Dingledine and I.~Goldberg},
	Booktitle = {{WPES}},
	Title = {{ Changing of the Guards: Framework for Understanding and Improving Entry Guard Selection in Tor}},
	Year = {2012}}

@article{RAINBOW:Journal,
	Author = {A.~Houmansadr and N.~Kiyavash and N.~Borisov},
	Journal = {IEEE/ACM Transactions on Networking},
	Title = {{Non-Blind Watermarking of Network Flows}},
	Year = 2014}

@inproceedings{info-tod,
	Author = {A.~Houmansadr and S.~Gorantla and T.~Coleman and N.~Kiyavash and and N.~Borisov},
	Booktitle = {{CCS (poster session)}},
	Title = {{On the Channel Capacity of Network Flow Watermarking}},
	Year = {2009}}

@inproceedings{johnson2014game,
	Author = {Johnson, B. and Laszka, A. and Grossklags, J. and Vasek, M. and Moore, T.},
	Booktitle = {Workshop on Bitcoin Research},
	Title = {{Game-theoretic Analysis of DDoS Attacks Against Bitcoin Mining Pools}},
	Year = {2014}}

@incollection{laszka2013mitigation,
	Author = {Laszka, A. and Johnson, B. and Grossklags, J.},
	Booktitle = {Decision and Game Theory for Security},
	Pages = {175--191},
	Publisher = {Springer},
	Title = {{Mitigation of Targeted and Non-targeted Covert Attacks as a Timing Game}},
	Year = {2013}}

@inproceedings{schottle2013game,
	Author = {Schottle, P. and Laszka, A. and Johnson, B. and Grossklags, J. and Bohme, R.},
	Booktitle = {EUSIPCO},
	Title = {{A Game-theoretic Analysis of Content-adaptive Steganography with Independent Embedding}},
	Year = {2013}}

@inproceedings{CloudTransport,
	Author = {C.~Brubaker and A.~Houmansadr and V.~Shmatikov},
	Booktitle = {PETS},
	Title = {{CloudTransport: Using Cloud Storage for Censorship-Resistant Networking}},
	Year = {2014}}

@inproceedings{sweet,
	Author = {W.~Zhou and A.~Houmansadr and M.~Caesar and N.~Borisov},
	Booktitle = {HotPETs},
	Title = {{SWEET: Serving the Web by Exploiting Email Tunnels}},
	Year = {2013}}

@inproceedings{ahsan2002practical,
	Author = {Ahsan, K. and Kundur, D.},
	Booktitle = {Workshop on Multimedia Security},
	Title = {{Practical data hiding in TCP/IP}},
	Year = {2002}}

@incollection{danezis2011covert,
	Author = {Danezis, G.},
	Booktitle = {Security Protocols XVI},
	Pages = {198--214},
	Publisher = {Springer},
	Title = {{Covert Communications Despite Traffic Data Retention}},
	Year = {2011}}

@inproceedings{liu2009hide,
	Author = {Liu, Y. and Ghosal, D. and Armknecht, F. and Sadeghi, A.-R. and Schulz, S. and Katzenbeisser, S.},
	Booktitle = {ESORICS},
	Title = {{Hide and Seek in Time---Robust Covert Timing Channels}},
	Year = {2009}}

@misc{image-watermark-fing,
	Author = {Jonathan Bailey},
	Howpublished = {\url{https://www.plagiarismtoday.com/2009/12/02/image-detection-watermarking-vs-fingerprinting/}},
	Title = {{Image Detection: Watermarking vs. Fingerprinting}},
	Year = {2009}}

@inproceedings{Servetto98,
	Author = {S. D. Servetto and C. I. Podilchuk and K. Ramchandran},
	Booktitle = {Int. Conf. Image Processing},
	Title = {Capacity issues in digital image watermarking},
	Year = {1998}}

@inproceedings{Chen01,
	Author = {B. Chen and G.W.Wornell},
	Booktitle = {IEEE Trans. Inform. Theory},
	Pages = {1423--1443},
	Title = {Quantization index modulation: A class of provably good methods for digital watermarking and information embedding},
	Year = {2001}}

@inproceedings{Karakos00,
	Author = {D. Karakos and A. Papamarcou},
	Booktitle = {IEEE Int. Symp. Information Theory},
	Pages = {47},
	Title = {Relationship between quantization and distribution rates of digitally watermarked data},
	Year = {2000}}

@inproceedings{Sullivan98,
	Author = {J. A. OSullivan and P. Moulin and J. M. Ettinger},
	Booktitle = {IEEE Int. Symp. Information Theory},
	Pages = {297},
	Title = {Information theoretic analysis of steganography},
	Year = {1998}}

@inproceedings{Merhav00,
	Author = {N. Merhav},
	Booktitle = {IEEE Trans. Inform. Theory},
	Pages = {420--430},
	Title = {On random coding error exponents of watermarking systems},
	Year = {2000}}

@inproceedings{Somekh01,
	Author = {A. Somekh-Baruch and N. Merhav},
	Booktitle = {IEEE Int. Symp. Information Theory},
	Pages = {7},
	Title = {On the error exponent and capacity games of private watermarking systems},
	Year = {2001}}

@inproceedings{Steinberg01,
	Author = {Y. Steinberg and N. Merhav},
	Booktitle = {IEEE Trans. Inform. Theory},
	Pages = {1410--1422},
	Title = {Identification in the presence of side information with application to watermarking},
	Year = {2001}}

@article{Moulin03,
	Author = {P. Moulin and J.A. O'Sullivan},
	Journal = {IEEE Trans. Info. Theory},
	Number = {3},
	Title = {Information-theoretic analysis of information hiding},
	Volume = 49,
	Year = 2003}

@article{Gelfand80,
	Author = {S.I.~Gelfand and M.S.~Pinsker},
	Journal = {Problems of Control and Information Theory},
	Number = {1},
	Pages = {19-31},
	Title = {{Coding for channel with random parameters}},
	Url = {citeseer.ist.psu.edu/anantharam96bits.html},
	Volume = {9},
	Year = {1980},
	Bdsk-Url-1 = {citeseer.ist.psu.edu/anantharam96bits.html}}

@book{Wolfowitz78,
	Author = {J. Wolfowitz},
	Edition = {3rd},
	Location = {New York},
	Publisher = {Springer-Verlag},
	Title = {Coding Theorems of Information Theory},
	Year = 1978}

@article{caire99,
	Author = {G. Caire and S. Shamai},
	Journal = {IEEE Transactions on Information Theory},
	Number = {6},
	Pages = {2007--2019},
	Title = {On the Capacity of Some Channels with Channel State Information},
	Volume = {45},
	Year = {1999}}

@inproceedings{wright2007language,
	Author = {Wright, Charles V and Ballard, Lucas and Monrose, Fabian and Masson, Gerald M},
	Booktitle = {USENIX Security},
	Title = {{Language identification of encrypted VoIP traffic: Alejandra y Roberto or Alice and Bob?}},
	Year = {2007}}

@inproceedings{backes2010speaker,
	Author = {Backes, Michael and Doychev, Goran and D{\"u}rmuth, Markus and K{\"o}pf, Boris},
	Booktitle = {{European Symposium on Research in Computer Security (ESORICS)}},
	Pages = {508--523},
	Publisher = {Springer},
	Title = {{Speaker Recognition in Encrypted Voice Streams}},
	Year = {2010}}

@phdthesis{lu2009traffic,
	Author = {Lu, Yuanchao},
	School = {Cleveland State University},
	Title = {{On Traffic Analysis Attacks to Encrypted VoIP Calls}},
	Year = {2009}}

@inproceedings{wright2008spot,
	Author = {Wright, Charles V and Ballard, Lucas and Coull, Scott E and Monrose, Fabian and Masson, Gerald M},
	Booktitle = {IEEE Symposium on Security and Privacy},
	Pages = {35--49},
	Title = {Spot me if you can: Uncovering spoken phrases in encrypted VoIP conversations},
	Year = {2008}}

@inproceedings{white2011phonotactic,
	Author = {White, Andrew M and Matthews, Austin R and Snow, Kevin Z and Monrose, Fabian},
	Booktitle = {IEEE Symposium on Security and Privacy},
	Pages = {3--18},
	Title = {Phonotactic reconstruction of encrypted VoIP conversations: Hookt on fon-iks},
	Year = {2011}}

@inproceedings{fancy,
	Author = {Houmansadr, Amir and Borisov, Nikita},
	Booktitle = {Privacy Enhancing Technologies},
	Organization = {Springer},
	Pages = {205--224},
	Title = {The Need for Flow Fingerprints to Link Correlated Network Flows},
	Year = {2013}}

@article{botmosaic,
	Author = {Amir Houmansadr and Nikita Borisov},
	Doi = {10.1016/j.jss.2012.11.005},
	Issn = {0164-1212},
	Journal = {Journal of Systems and Software},
	Keywords = {Network security},
	Number = {3},
	Pages = {707 - 715},
	Title = {BotMosaic: Collaborative network watermark for the detection of IRC-based botnets},
	Url = {http://www.sciencedirect.com/science/article/pii/S0164121212003068},
	Volume = {86},
	Year = {2013},
	Bdsk-Url-1 = {http://www.sciencedirect.com/science/article/pii/S0164121212003068},
	Bdsk-Url-2 = {http://dx.doi.org/10.1016/j.jss.2012.11.005}}

@inproceedings{ramsbrock2008first,
	Author = {Ramsbrock, Daniel and Wang, Xinyuan and Jiang, Xuxian},
	Booktitle = {Recent Advances in Intrusion Detection},
	Organization = {Springer},
	Pages = {59--77},
	Title = {A first step towards live botmaster traceback},
	Year = {2008}}

@inproceedings{potdar2005survey,
	Author = {Potdar, Vidyasagar M and Han, Song and Chang, Elizabeth},
	Booktitle = {Industrial Informatics, 2005. INDIN'05. 2005 3rd IEEE International Conference on},
	Organization = {IEEE},
	Pages = {709--716},
	Title = {A survey of digital image watermarking techniques},
	Year = {2005}}

@book{cole2003hiding,
	Author = {Cole, Eric and Krutz, Ronald D},
	Publisher = {John Wiley \& Sons, Inc.},
	Title = {Hiding in plain sight: Steganography and the art of covert communication},
	Year = {2003}}

@incollection{akaike1998information,
	Author = {Akaike, Hirotogu},
	Booktitle = {Selected Papers of Hirotugu Akaike},
	Pages = {199--213},
	Publisher = {Springer},
	Title = {Information theory and an extension of the maximum likelihood principle},
	Year = {1998}}

@misc{central-command-hack,
	Author = {Everett Rosenfeld},
	Howpublished = {\url{http://www.cnbc.com/id/102330338}},
	Title = {{FBI investigating Central Command Twitter hack}},
	Year = {2015}}

@misc{sony-psp-ddos,
	Howpublished = {\url{http://n4g.com/news/1644853/sony-and-microsoft-cant-do-much-ddos-attacks-explained}},
	Key = {sony},
	Month = {December},
	Title = {{Sony and Microsoft cant do much -- DDoS attacks explained}},
	Year = {2014}}

@misc{sony-hack,
	Author = {David Bloom},
	Howpublished = {\url{http://goo.gl/MwR4A7}},
	Title = {{Online Game Networks Hacked, Sony Unit President Threatened}},
	Year = {2014}}

@misc{home-depot,
	Author = {Dune Lawrence},
	Howpublished = {\url{http://www.businessweek.com/articles/2014-09-02/home-depots-credit-card-breach-looks-just-like-the-target-hack}},
	Title = {{Home Depot's Suspected Breach Looks Just Like the Target Hack}},
	Year = {2014}}

@misc{target,
	Author = {Julio Ojeda-Zapata},
	Howpublished = {\url{http://www.mercurynews.com/business/ci_24765398/how-did-hackers-pull-off-target-data-heist}},
	Title = {{Target hack: How did they do it?}},
	Year = {2014}}


@article{probabilitycourse,
	Author = {H. Pishro-Nik},
	note = {\url{http://www.probabilitycourse.com}},
	Title = {Introduction to probability, statistics, and random processes},
    Year = {2014}}



@inproceedings{shokri2011quantifying,
	Author = {Shokri, Reza and Theodorakopoulos, George and Le Boudec, Jean-Yves and Hubaux, Jean-Pierre},
	Booktitle = {Security and Privacy (SP), 2011 IEEE Symposium on},
	Organization = {IEEE},
	Pages = {247--262},
	Title = {Quantifying location privacy},
	Year = {2011}}

@inproceedings{hoh2007preserving,
	Author = {Hoh, Baik and Gruteser, Marco and Xiong, Hui and Alrabady, Ansaf},
	Booktitle = {Proceedings of the 14th ACM conference on Computer and communications security},
	Organization = {ACM},
	Pages = {161--171},
	Title = {Preserving privacy in gps traces via uncertainty-aware path cloaking},
	Year = {2007}}



@article{kafsi2013entropy,
	Author = {Kafsi, Mohamed and Grossglauser, Matthias and Thiran, Patrick},
	Journal = {Information Theory, IEEE Transactions on},
	Number = {9},
	Pages = {5577--5583},
	Publisher = {IEEE},
	Title = {The entropy of conditional Markov trajectories},
	Volume = {59},
	Year = {2013}}

@inproceedings{gruteser2003anonymous,
	Author = {Gruteser, Marco and Grunwald, Dirk},
	Booktitle = {Proceedings of the 1st international conference on Mobile systems, applications and services},
	Organization = {ACM},
	Pages = {31--42},
	Title = {Anonymous usage of location-based services through spatial and temporal cloaking},
	Year = {2003}}

@inproceedings{husted2010mobile,
	Author = {Husted, Nathaniel and Myers, Steven},
	Booktitle = {Proceedings of the 17th ACM conference on Computer and communications security},
	Organization = {ACM},
	Pages = {85--96},
	Title = {Mobile location tracking in metro areas: malnets and others},
	Year = {2010}}

@inproceedings{li2009tradeoff,
	Author = {Li, Tiancheng and Li, Ninghui},
	Booktitle = {Proceedings of the 15th ACM SIGKDD international conference on Knowledge discovery and data mining},
	Organization = {ACM},
	Pages = {517--526},
	Title = {On the tradeoff between privacy and utility in data publishing},
	Year = {2009}}

@inproceedings{ma2009location,
	Author = {Ma, Zhendong and Kargl, Frank and Weber, Michael},
	Booktitle = {Sarnoff Symposium, 2009. SARNOFF'09. IEEE},
	Organization = {IEEE},
	Pages = {1--6},
	Title = {A location privacy metric for v2x communication systems},
	Year = {2009}}

@inproceedings{shokri2012protecting,
	Author = {Shokri, Reza and Theodorakopoulos, George and Troncoso, Carmela and Hubaux, Jean-Pierre and Le Boudec, Jean-Yves},
	Booktitle = {Proceedings of the 2012 ACM conference on Computer and communications security},
	Organization = {ACM},
	Pages = {617--627},
	Title = {Protecting location privacy: optimal strategy against localization attacks},
	Year = {2012}}

@inproceedings{freudiger2009non,
	Author = {Freudiger, Julien and Manshaei, Mohammad Hossein and Hubaux, Jean-Pierre and Parkes, David C},
	Booktitle = {Proceedings of the 16th ACM conference on Computer and communications security},
	Organization = {ACM},
	Pages = {324--337},
	Title = {On non-cooperative location privacy: a game-theoretic analysis},
	Year = {2009}}

@incollection{humbert2010tracking,
	Author = {Humbert, Mathias and Manshaei, Mohammad Hossein and Freudiger, Julien and Hubaux, Jean-Pierre},
	Booktitle = {Decision and Game Theory for Security},
	Pages = {38--57},
	Publisher = {Springer},
	Title = {Tracking games in mobile networks},
	Year = {2010}}

@article{manshaei2013game,
	Author = {Manshaei, Mohammad Hossein and Zhu, Quanyan and Alpcan, Tansu and Bac{\c{s}}ar, Tamer and Hubaux, Jean-Pierre},
	Journal = {ACM Computing Surveys (CSUR)},
	Number = {3},
	Pages = {25},
	Publisher = {ACM},
	Title = {Game theory meets network security and privacy},
	Volume = {45},
	Year = {2013}}

@article{palamidessi2006probabilistic,
	Author = {Palamidessi, Catuscia},
	Journal = {Electronic Notes in Theoretical Computer Science},
	Pages = {33--42},
	Publisher = {Elsevier},
	Title = {Probabilistic and nondeterministic aspects of anonymity},
	Volume = {155},
	Year = {2006}}

@inproceedings{mokbel2006new,
	Author = {Mokbel, Mohamed F and Chow, Chi-Yin and Aref, Walid G},
	Booktitle = {Proceedings of the 32nd international conference on Very large data bases},
	Organization = {VLDB Endowment},
	Pages = {763--774},
	Title = {The new Casper: query processing for location services without compromising privacy},
	Year = {2006}}

@article{kalnis2007preventing,
	Author = {Kalnis, Panos and Ghinita, Gabriel and Mouratidis, Kyriakos and Papadias, Dimitris},
	Journal = {Knowledge and Data Engineering, IEEE Transactions on},
	Number = {12},
	Pages = {1719--1733},
	Publisher = {IEEE},
	Title = {Preventing location-based identity inference in anonymous spatial queries},
	Volume = {19},
	Year = {2007}}
	
@article{freudiger2007mix,
  title={Mix-zones for location privacy in vehicular networks},
  author={Freudiger, Julien and Raya, Maxim and F{\'e}legyh{\'a}zi, M{\'a}rk and Papadimitratos, Panos and Hubaux, Jean-Pierre},
  year={2007}
}
@article{sweeney2002k,
	Author = {Sweeney, Latanya},
	Journal = {International Journal of Uncertainty, Fuzziness and Knowledge-Based Systems},
	Number = {05},
	Pages = {557--570},
	Publisher = {World Scientific},
	Title = {k-anonymity: A model for protecting privacy},
	Volume = {10},
	Year = {2002}}

@article{sweeney2002achieving,
	Author = {Sweeney, Latanya},
	Journal = {International Journal of Uncertainty, Fuzziness and Knowledge-Based Systems},
	Number = {05},
	Pages = {571--588},
	Publisher = {World Scientific},
	Title = {Achieving k-anonymity privacy protection using generalization and suppression},
	Volume = {10},
	Year = {2002}}

@inproceedings{niu2014achieving,
	Author = {Niu, Ben and Li, Qinghua and Zhu, Xiaoyan and Cao, Guohong and Li, Hui},
	Booktitle = {INFOCOM, 2014 Proceedings IEEE},
	Organization = {IEEE},
	Pages = {754--762},
	Title = {Achieving k-anonymity in privacy-aware location-based services},
	Year = {2014}}

@inproceedings{liu2013game,
	Author = {Liu, Xinxin and Liu, Kaikai and Guo, Linke and Li, Xiaolin and Fang, Yuguang},
	Booktitle = {INFOCOM, 2013 Proceedings IEEE},
	Organization = {IEEE},
	Pages = {2985--2993},
	Title = {A game-theoretic approach for achieving k-anonymity in location based services},
	Year = {2013}}

@inproceedings{kido2005protection,
	Author = {Kido, Hidetoshi and Yanagisawa, Yutaka and Satoh, Tetsuji},
	Booktitle = {Data Engineering Workshops, 2005. 21st International Conference on},
	Organization = {IEEE},
	Pages = {1248--1248},
	Title = {Protection of location privacy using dummies for location-based services},
	Year = {2005}}

@inproceedings{gedik2005location,
	Author = {Gedik, Bu{\u{g}}ra and Liu, Ling},
	Booktitle = {Distributed Computing Systems, 2005. ICDCS 2005. Proceedings. 25th IEEE International Conference on},
	Organization = {IEEE},
	Pages = {620--629},
	Title = {Location privacy in mobile systems: A personalized anonymization model},
	Year = {2005}}

@inproceedings{bordenabe2014optimal,
	Author = {Bordenabe, Nicol{\'a}s E and Chatzikokolakis, Konstantinos and Palamidessi, Catuscia},
	Booktitle = {Proceedings of the 2014 ACM SIGSAC Conference on Computer and Communications Security},
	Organization = {ACM},
	Pages = {251--262},
	Title = {Optimal geo-indistinguishable mechanisms for location privacy},
	Year = {2014}}

@incollection{duckham2005formal,
	Author = {Duckham, Matt and Kulik, Lars},
	Booktitle = {Pervasive computing},
	Pages = {152--170},
	Publisher = {Springer},
	Title = {A formal model of obfuscation and negotiation for location privacy},
	Year = {2005}}

@inproceedings{kido2005anonymous,
	Author = {Kido, Hidetoshi and Yanagisawa, Yutaka and Satoh, Tetsuji},
	Booktitle = {Pervasive Services, 2005. ICPS'05. Proceedings. International Conference on},
	Organization = {IEEE},
	Pages = {88--97},
	Title = {An anonymous communication technique using dummies for location-based services},
	Year = {2005}}

@incollection{duckham2006spatiotemporal,
	Author = {Duckham, Matt and Kulik, Lars and Birtley, Athol},
	Booktitle = {Geographic Information Science},
	Pages = {47--64},
	Publisher = {Springer},
	Title = {A spatiotemporal model of strategies and counter strategies for location privacy protection},
	Year = {2006}}

@inproceedings{shankar2009privately,
	Author = {Shankar, Pravin and Ganapathy, Vinod and Iftode, Liviu},
	Booktitle = {Proceedings of the 11th international conference on Ubiquitous computing},
	Organization = {ACM},
	Pages = {31--40},
	Title = {Privately querying location-based services with SybilQuery},
	Year = {2009}}

@inproceedings{chow2009faking,
	Author = {Chow, Richard and Golle, Philippe},
	Booktitle = {Proceedings of the 8th ACM workshop on Privacy in the electronic society},
	Organization = {ACM},
	Pages = {105--108},
	Title = {Faking contextual data for fun, profit, and privacy},
	Year = {2009}}

@incollection{xue2009location,
	Author = {Xue, Mingqiang and Kalnis, Panos and Pung, Hung Keng},
	Booktitle = {Location and Context Awareness},
	Pages = {70--87},
	Publisher = {Springer},
	Title = {Location diversity: Enhanced privacy protection in location based services},
	Year = {2009}}

@article{wernke2014classification,
	Author = {Wernke, Marius and Skvortsov, Pavel and D{\"u}rr, Frank and Rothermel, Kurt},
	Journal = {Personal and Ubiquitous Computing},
	Number = {1},
	Pages = {163--175},
	Publisher = {Springer-Verlag},
	Title = {A classification of location privacy attacks and approaches},
	Volume = {18},
	Year = {2014}}

@misc{cai2015cloaking,
	Author = {Cai, Y. and Xu, G.},
	Month = jan # {~1},
	Note = {US Patent App. 14/472,462},
	Publisher = {Google Patents},
	Title = {Cloaking with footprints to provide location privacy protection in location-based services},
	Url = {https://www.google.com/patents/US20150007341},
	Year = {2015},
	Bdsk-Url-1 = {https://www.google.com/patents/US20150007341}}

@article{gedik2008protecting,
	Author = {Gedik, Bu{\u{g}}ra and Liu, Ling},
	Journal = {Mobile Computing, IEEE Transactions on},
	Number = {1},
	Pages = {1--18},
	Publisher = {IEEE},
	Title = {Protecting location privacy with personalized k-anonymity: Architecture and algorithms},
	Volume = {7},
	Year = {2008}}

@article{kalnis2006preserving,
	Author = {Kalnis, Panos and Ghinita, Gabriel and Mouratidis, Kyriakos and Papadias, Dimitris},
	Publisher = {TRB6/06},
	Title = {Preserving anonymity in location based services},
	Year = {2006}}

@inproceedings{hoh2005protecting,
	Author = {Hoh, Baik and Gruteser, Marco},
	Booktitle = {Security and Privacy for Emerging Areas in Communications Networks, 2005. SecureComm 2005. First International Conference on},
	Organization = {IEEE},
	Pages = {194--205},
	Title = {Protecting location privacy through path confusion},
	Year = {2005}}

@article{terrovitis2011privacy,
	Author = {Terrovitis, Manolis},
	Journal = {ACM SIGKDD Explorations Newsletter},
	Number = {1},
	Pages = {6--18},
	Publisher = {ACM},
	Title = {Privacy preservation in the dissemination of location data},
	Volume = {13},
	Year = {2011}}

@article{shin2012privacy,
	Author = {Shin, Kang G and Ju, Xiaoen and Chen, Zhigang and Hu, Xin},
	Journal = {Wireless Communications, IEEE},
	Number = {1},
	Pages = {30--39},
	Publisher = {IEEE},
	Title = {Privacy protection for users of location-based services},
	Volume = {19},
	Year = {2012}}

@article{khoshgozaran2011location,
	Author = {Khoshgozaran, Ali and Shahabi, Cyrus and Shirani-Mehr, Houtan},
	Journal = {Knowledge and Information Systems},
	Number = {3},
	Pages = {435--465},
	Publisher = {Springer},
	Title = {Location privacy: going beyond K-anonymity, cloaking and anonymizers},
	Volume = {26},
	Year = {2011}}

@incollection{chatzikokolakis2015geo,
	Author = {Chatzikokolakis, Konstantinos and Palamidessi, Catuscia and Stronati, Marco},
	Booktitle = {Distributed Computing and Internet Technology},
	Pages = {49--72},
	Publisher = {Springer},
	Title = {Geo-indistinguishability: A Principled Approach to Location Privacy},
	Year = {2015}}

@inproceedings{ngo2015location,
	Author = {Ngo, Hoa and Kim, Jong},
	Booktitle = {Computer Security Foundations Symposium (CSF), 2015 IEEE 28th},
	Organization = {IEEE},
	Pages = {63--74},
	Title = {Location Privacy via Differential Private Perturbation of Cloaking Area},
	Year = {2015}}

@inproceedings{palanisamy2011mobimix,
	Author = {Palanisamy, Balaji and Liu, Ling},
	Booktitle = {Data Engineering (ICDE), 2011 IEEE 27th International Conference on},
	Organization = {IEEE},
	Pages = {494--505},
	Title = {Mobimix: Protecting location privacy with mix-zones over road networks},
	Year = {2011}}

@inproceedings{um2010advanced,
	Author = {Um, Jung-Ho and Kim, Hee-Dae and Chang, Jae-Woo},
	Booktitle = {Social Computing (SocialCom), 2010 IEEE Second International Conference on},
	Organization = {IEEE},
	Pages = {1093--1098},
	Title = {An advanced cloaking algorithm using Hilbert curves for anonymous location based service},
	Year = {2010}}

@inproceedings{bamba2008supporting,
	Author = {Bamba, Bhuvan and Liu, Ling and Pesti, Peter and Wang, Ting},
	Booktitle = {Proceedings of the 17th international conference on World Wide Web},
	Organization = {ACM},
	Pages = {237--246},
	Title = {Supporting anonymous location queries in mobile environments with privacygrid},
	Year = {2008}}

@inproceedings{zhangwei2010distributed,
	Author = {Zhangwei, Huang and Mingjun, Xin},
	Booktitle = {Networks Security Wireless Communications and Trusted Computing (NSWCTC), 2010 Second International Conference on},
	Organization = {IEEE},
	Pages = {468--471},
	Title = {A distributed spatial cloaking protocol for location privacy},
	Volume = {2},
	Year = {2010}}

@article{chow2011spatial,
	Author = {Chow, Chi-Yin and Mokbel, Mohamed F and Liu, Xuan},
	Journal = {GeoInformatica},
	Number = {2},
	Pages = {351--380},
	Publisher = {Springer},
	Title = {Spatial cloaking for anonymous location-based services in mobile peer-to-peer environments},
	Volume = {15},
	Year = {2011}}

@inproceedings{lu2008pad,
	Author = {Lu, Hua and Jensen, Christian S and Yiu, Man Lung},
	Booktitle = {Proceedings of the Seventh ACM International Workshop on Data Engineering for Wireless and Mobile Access},
	Organization = {ACM},
	Pages = {16--23},
	Title = {Pad: privacy-area aware, dummy-based location privacy in mobile services},
	Year = {2008}}

@incollection{khoshgozaran2007blind,
	Author = {Khoshgozaran, Ali and Shahabi, Cyrus},
	Booktitle = {Advances in Spatial and Temporal Databases},
	Pages = {239--257},
	Publisher = {Springer},
	Title = {Blind evaluation of nearest neighbor queries using space transformation to preserve location privacy},
	Year = {2007}}

@inproceedings{ghinita2008private,
	Author = {Ghinita, Gabriel and Kalnis, Panos and Khoshgozaran, Ali and Shahabi, Cyrus and Tan, Kian-Lee},
	Booktitle = {Proceedings of the 2008 ACM SIGMOD international conference on Management of data},
	Organization = {ACM},
	Pages = {121--132},
	Title = {Private queries in location based services: anonymizers are not necessary},
	Year = {2008}}

@article{paulet2014privacy,
	Author = {Paulet, Russell and Kaosar, Md Golam and Yi, Xun and Bertino, Elisa},
	Journal = {Knowledge and Data Engineering, IEEE Transactions on},
	Number = {5},
	Pages = {1200--1210},
	Publisher = {IEEE},
	Title = {Privacy-preserving and content-protecting location based queries},
	Volume = {26},
	Year = {2014}}

@article{nguyen2013differential,
	Author = {Nguyen, Hiep H and Kim, Jong and Kim, Yoonho},
	Journal = {Journal of Computing Science and Engineering},
	Number = {3},
	Pages = {177--186},
	Title = {Differential privacy in practice},
	Volume = {7},
	Year = {2013}}

@inproceedings{lee2012differential,
	Author = {Lee, Jaewoo and Clifton, Chris},
	Booktitle = {Proceedings of the 18th ACM SIGKDD international conference on Knowledge discovery and data mining},
	Organization = {ACM},
	Pages = {1041--1049},
	Title = {Differential identifiability},
	Year = {2012}}

@inproceedings{andres2013geo,
	Author = {Andr{\'e}s, Miguel E and Bordenabe, Nicol{\'a}s E and Chatzikokolakis, Konstantinos and Palamidessi, Catuscia},
	Booktitle = {Proceedings of the 2013 ACM SIGSAC conference on Computer \& communications security},
	Organization = {ACM},
	Pages = {901--914},
	Title = {Geo-indistinguishability: Differential privacy for location-based systems},
	Year = {2013}}

@inproceedings{machanavajjhala2008privacy,
	Author = {Machanavajjhala, Ashwin and Kifer, Daniel and Abowd, John and Gehrke, Johannes and Vilhuber, Lars},
	Booktitle = {Data Engineering, 2008. ICDE 2008. IEEE 24th International Conference on},
	Organization = {IEEE},
	Pages = {277--286},
	Title = {Privacy: Theory meets practice on the map},
	Year = {2008}}

@article{dewri2013local,
	Author = {Dewri, Rinku},
	Journal = {Mobile Computing, IEEE Transactions on},
	Number = {12},
	Pages = {2360--2372},
	Publisher = {IEEE},
	Title = {Local differential perturbations: Location privacy under approximate knowledge attackers},
	Volume = {12},
	Year = {2013}}

@inproceedings{chatzikokolakis2013broadening,
	Author = {Chatzikokolakis, Konstantinos and Andr{\'e}s, Miguel E and Bordenabe, Nicol{\'a}s Emilio and Palamidessi, Catuscia},
	Booktitle = {Privacy Enhancing Technologies},
	Organization = {Springer},
	Pages = {82--102},
	Title = {Broadening the Scope of Differential Privacy Using Metrics.},
	Year = {2013}}

@inproceedings{zhong2009distributed,
	Author = {Zhong, Ge and Hengartner, Urs},
	Booktitle = {Pervasive Computing and Communications, 2009. PerCom 2009. IEEE International Conference on},
	Organization = {IEEE},
	Pages = {1--10},
	Title = {A distributed k-anonymity protocol for location privacy},
	Year = {2009}}

@inproceedings{ho2011differential,
	Author = {Ho, Shen-Shyang and Ruan, Shuhua},
	Booktitle = {Proceedings of the 4th ACM SIGSPATIAL International Workshop on Security and Privacy in GIS and LBS},
	Organization = {ACM},
	Pages = {17--24},
	Title = {Differential privacy for location pattern mining},
	Year = {2011}}

@inproceedings{cheng2006preserving,
	Author = {Cheng, Reynold and Zhang, Yu and Bertino, Elisa and Prabhakar, Sunil},
	Booktitle = {Privacy Enhancing Technologies},
	Organization = {Springer},
	Pages = {393--412},
	Title = {Preserving user location privacy in mobile data management infrastructures},
	Year = {2006}}

@article{beresford2003location,
	Author = {Beresford, Alastair R and Stajano, Frank},
	Journal = {IEEE Pervasive computing},
	Number = {1},
	Pages = {46--55},
	Publisher = {IEEE},
	Title = {Location privacy in pervasive computing},
	Year = {2003}}

@inproceedings{freudiger2009optimal,
	Author = {Freudiger, Julien and Shokri, Reza and Hubaux, Jean-Pierre},
	Booktitle = {Privacy enhancing technologies},
	Organization = {Springer},
	Pages = {216--234},
	Title = {On the optimal placement of mix zones},
	Year = {2009}}

@article{krumm2009survey,
	Author = {Krumm, John},
	Journal = {Personal and Ubiquitous Computing},
	Number = {6},
	Pages = {391--399},
	Publisher = {Springer},
	Title = {A survey of computational location privacy},
	Volume = {13},
	Year = {2009}}

@article{Rakhshan2016letter,
	Author = {Rakhshan, Ali and Pishro-Nik, Hossein},
	Journal = {IEEE Wireless Communications Letter},
	Publisher = {IEEE},
	Title = {Interference Models for Vehicular Ad Hoc Networks},
	Year = {2016, submitted}}

@article{Rakhshan2015Journal,
	Author = {Rakhshan, Ali and Pishro-Nik, Hossein},
	Journal = {IEEE Transactions on Wireless Communications},
	Publisher = {IEEE},
	Title = {Improving Safety on Highways by Customizing Vehicular Ad Hoc Networks},
	Year = {to appear, 2017}}

@inproceedings{Rakhshan2015Cogsima,
	Author = {Rakhshan, Ali and Pishro-Nik, Hossein},
	Booktitle = {IEEE International Multi-Disciplinary Conference on Cognitive Methods in Situation Awareness and Decision Support},
	Organization = {IEEE},
	Title = {A New Approach to Customization of Accident Warning Systems to Individual Drivers},
	Year = {2015}}

@inproceedings{Rakhshan2015CISS,
	Author = {Rakhshan, Ali and Pishro-Nik, Hossein and Nekoui, Mohammad},
	Booktitle = {Conference on Information Sciences and Systems},
	Organization = {IEEE},
	Pages = {1--6},
	Title = {Driver-based adaptation of Vehicular Ad Hoc Networks for design of active safety systems},
	Year = {2015}}

@inproceedings{Rakhshan2014IV,
	Author = {Rakhshan, Ali and Pishro-Nik, Hossein and Ray, Evan},
	Booktitle = {Intelligent Vehicles Symposium},
	Organization = {IEEE},
	Pages = {1181--1186},
	Title = {Real-time estimation of the distribution of brake response times for an individual driver using Vehicular Ad Hoc Network.},
	Year = {2014}}

@inproceedings{Rakhshan2013Globecom,
	Author = {Rakhshan, Ali and Pishro-Nik, Hossein and Fisher, Donald and Nekoui, Mohammad},
	Booktitle = {IEEE Global Communications Conference},
	Organization = {IEEE},
	Pages = {1333--1337},
	Title = {Tuning collision warning algorithms to individual drivers for design of active safety systems.},
	Year = {2013}}

@article{Nekoui2012Journal,
	Author = {Nekoui, Mohammad and Pishro-Nik, Hossein},
	Journal = {IEEE Transactions on Wireless Communications},
	Number = {8},
	Pages = {2895--2905},
	Publisher = {IEEE},
	Title = {Throughput Scaling laws for Vehicular Ad Hoc Networks},
	Volume = {11},
	Year = {2012}}









@article{Nekoui2011Journal,
	Author = {Nekoui, Mohammad and Pishro-Nik, Hossein and Ni, Daiheng},
	Journal = {International Journal of Vehicular Technology},
	Pages = {1--11},
	Publisher = {Hindawi Publishing Corporation},
	Title = {Analytic Design of Active Safety Systems for Vehicular Ad hoc Networks},
	Volume = {2011},
	Year = {2011}}





	
@article{shokri2014optimal,
	  title={Optimal user-centric data obfuscation},
 	 author={Shokri, Reza},
 	 journal={arXiv preprint arXiv:1402.3426},
 	 year={2014}
	}
@article{chatzikokolakis2015location,
  title={Location privacy via geo-indistinguishability},
  author={Chatzikokolakis, Konstantinos and Palamidessi, Catuscia and Stronati, Marco},
  journal={ACM SIGLOG News},
  volume={2},
  number={3},
  pages={46--69},
  year={2015},
  publisher={ACM}

}
@inproceedings{shokri2011quantifying2,
  title={Quantifying location privacy: the case of sporadic location exposure},
  author={Shokri, Reza and Theodorakopoulos, George and Danezis, George and Hubaux, Jean-Pierre and Le Boudec, Jean-Yves},
  booktitle={Privacy Enhancing Technologies},
  pages={57--76},
  year={2011},
  organization={Springer}
}


@inproceedings{Mont1603:Defining,
AUTHOR="Zarrin Montazeri and Amir Houmansadr and Hossein Pishro-Nik",
TITLE="Defining Perfect Location Privacy Using Anonymization",
BOOKTITLE="2016 Annual Conference on Information Science and Systems (CISS) (CISS
2016)",
ADDRESS="Princeton, USA",
DAYS=16,
MONTH=mar,
YEAR=2016,
KEYWORDS="Information Theoretic Privacy; location-based services; Location Privacy;
Information Theory",
ABSTRACT="The popularity of mobile devices and location-based services (LBS) have
created great concerns regarding the location privacy of users of such
devices and services. Anonymization is a common technique that is often
being used to protect the location privacy of LBS users. In this paper, we
provide a general information theoretic definition for location privacy. In
particular, we define perfect location privacy. We show that under certain
conditions, perfect privacy is achieved if the pseudonyms of users is
changed after o(N^(2/r?1)) observations by the adversary, where N is the
number of users and r is the number of sub-regions or locations.
"
}
@article{our-isita-location,
	Author = {Zarrin Montazeri and Amir Houmansadr and Hossein Pishro-Nik},
	Journal = {IEEE International Symposium on Information Theory and Its Applications (ISITA)},
	Title = {Achieving Perfect Location Privacy in Markov Models Using Anonymization},
	Year = {2016}
	}
@article{our-TIFS,
	Author = {Zarrin Montazeri and Hossein Pishro-Nik and Amir Houmansadr},
	Journal = {IEEE Transactions on Information Forensics and Security, accepted with mandatory minor revisions},
	Title = {Perfect Location Privacy Using Anonymization in Mobile Networks},
	Year = {2017},
    note={Available on arxiv.org}
	}



@techreport{sampigethaya2005caravan,
  title={CARAVAN: Providing location privacy for VANET},
  author={Sampigethaya, Krishna and Huang, Leping and Li, Mingyan and Poovendran, Radha and Matsuura, Kanta and Sezaki, Kaoru},
  year={2005},
  institution={DTIC Document}
}
@incollection{buttyan2007effectiveness,
  title={On the effectiveness of changing pseudonyms to provide location privacy in VANETs},
  author={Butty{\'a}n, Levente and Holczer, Tam{\'a}s and Vajda, Istv{\'a}n},
  booktitle={Security and Privacy in Ad-hoc and Sensor Networks},
  pages={129--141},
  year={2007},
  publisher={Springer}
}
@article{sampigethaya2007amoeba,
  title={AMOEBA: Robust location privacy scheme for VANET},
  author={Sampigethaya, Krishna and Li, Mingyan and Huang, Leping and Poovendran, Radha},
  journal={Selected Areas in communications, IEEE Journal on},
  volume={25},
  number={8},
  pages={1569--1589},
  year={2007},
  publisher={IEEE}
}

@article{lu2012pseudonym,
  title={Pseudonym changing at social spots: An effective strategy for location privacy in vanets},
  author={Lu, Rongxing and Li, Xiaodong and Luan, Tom H and Liang, Xiaohui and Shen, Xuemin},
  journal={Vehicular Technology, IEEE Transactions on},
  volume={61},
  number={1},
  pages={86--96},
  year={2012},
  publisher={IEEE}
}
@inproceedings{lu2010sacrificing,
  title={Sacrificing the plum tree for the peach tree: A socialspot tactic for protecting receiver-location privacy in VANET},
  author={Lu, Rongxing and Lin, Xiaodong and Liang, Xiaohui and Shen, Xuemin},
  booktitle={Global Telecommunications Conference (GLOBECOM 2010), 2010 IEEE},
  pages={1--5},
  year={2010},
  organization={IEEE}
}
@inproceedings{lin2011stap,
  title={STAP: A social-tier-assisted packet forwarding protocol for achieving receiver-location privacy preservation in VANETs},
  author={Lin, Xiaodong and Lu, Rongxing and Liang, Xiaohui and Shen, Xuemin Sherman},
  booktitle={INFOCOM, 2011 Proceedings IEEE},
  pages={2147--2155},
  year={2011},
  organization={IEEE}
}
@inproceedings{gerlach2007privacy,
  title={Privacy in VANETs using changing pseudonyms-ideal and real},
  author={Gerlach, Matthias and Guttler, Felix},
  booktitle={Vehicular Technology Conference, 2007. VTC2007-Spring. IEEE 65th},
  pages={2521--2525},
  year={2007},
  organization={IEEE}
}
@inproceedings{el2002security,
  title={Security issues in a future vehicular network},
  author={El Zarki, Magda and Mehrotra, Sharad and Tsudik, Gene and Venkatasubramanian, Nalini},
  booktitle={European Wireless},
  volume={2},
  year={2002}
}

@article{hubaux2004security,
  title={The security and privacy of smart vehicles},
  author={Hubaux, Jean-Pierre and Capkun, Srdjan and Luo, Jun},
  journal={IEEE Security \& Privacy Magazine},
  volume={2},
  number={LCA-ARTICLE-2004-007},
  pages={49--55},
  year={2004}
}



@inproceedings{duri2002framework,
  title={Framework for security and privacy in automotive telematics},
  author={Duri, Sastry and Gruteser, Marco and Liu, Xuan and Moskowitz, Paul and Perez, Ronald and Singh, Moninder and Tang, Jung-Mu},
  booktitle={Proceedings of the 2nd international workshop on Mobile commerce},
  pages={25--32},
  year={2002},
  organization={ACM}
}
@misc{NS-3,
	Howpublished = {\url{https://www.nsnam.org/}}},
}
@misc{testbed,
	Howpublished = {\url{http://www.its.dot.gov/testbed/PDF/SE-MI-Resource-Guide-9-3-1.pdf}}},
@misc{NGSIM,
	Howpublished = {\url{http://ops.fhwa.dot.gov/trafficanalysistools/ngsim.htm}},
	}

@misc{National-a2013,
	Author = {National Highway Traffic Safety Administration},
	Howpublished = {\url{http://ops.fhwa.dot.gov/trafficanalysistools/ngsim.htm}},
	Title = {2013 Motor Vehicle Crashes: Overview. Traffic Safety Factors},
	Year = {2013}
	}

	@inproceedings{karnadi2007rapid,
	  title={Rapid generation of realistic mobility models for VANET},
	  author={Karnadi, Feliz Kristianto and Mo, Zhi Hai and Lan, Kun-chan},
	  booktitle={Wireless Communications and Networking Conference, 2007. WCNC 2007. IEEE},
	  pages={2506--2511},
	  year={2007},
	  organization={IEEE}
	}
	@inproceedings{saha2004modeling,
  title={Modeling mobility for vehicular ad-hoc networks},
  author={Saha, Amit Kumar and Johnson, David B},
  booktitle={Proceedings of the 1st ACM international workshop on Vehicular ad hoc networks},
  pages={91--92},
  year={2004},
  organization={ACM}
}
@inproceedings{lee2006modeling,
  title={Modeling steady-state and transient behaviors of user mobility: formulation, analysis, and application},
  author={Lee, Jong-Kwon and Hou, Jennifer C},
  booktitle={Proceedings of the 7th ACM international symposium on Mobile ad hoc networking and computing},
  pages={85--96},
  year={2006},
  organization={ACM}
}
@inproceedings{yoon2006building,
  title={Building realistic mobility models from coarse-grained traces},
  author={Yoon, Jungkeun and Noble, Brian D and Liu, Mingyan and Kim, Minkyong},
  booktitle={Proceedings of the 4th international conference on Mobile systems, applications and services},
  pages={177--190},
  year={2006},
  organization={ACM}
}

@inproceedings{choffnes2005integrated,
	title={An integrated mobility and traffic model for vehicular wireless networks},
	author={Choffnes, David R and Bustamante, Fabi{\'a}n E},
	booktitle={Proceedings of the 2nd ACM international workshop on Vehicular ad hoc networks},
	pages={69--78},
	year={2005},
	organization={ACM}
}

@inproceedings{Qian2008Globecom,
	title={CA Secure VANET MAC Protocol for DSRC Applications},
	author={Yi, Q. and Lu, K. and Moyeri, N.{\'a}n E},
	booktitle={Proceedings of IEEE GLOBECOM 2008},
	pages={1--5},
	year={2008},
	organization={IEEE}
}





	@inproceedings{naumov2006evaluation,
  title={An evaluation of inter-vehicle ad hoc networks based on realistic vehicular traces},
  author={Naumov, Valery and Baumann, Rainer and Gross, Thomas},
  booktitle={Proceedings of the 7th ACM international symposium on Mobile ad hoc networking and computing},
  pages={108--119},
  year={2006},
  organization={ACM}
}
	@article{sommer2008progressing,
  title={Progressing toward realistic mobility models in VANET simulations},
  author={Sommer, Christoph and Dressler, Falko},
  journal={Communications Magazine, IEEE},
  volume={46},
  number={11},
  pages={132--137},
  year={2008},
  publisher={IEEE}
}




	@inproceedings{mahajan2006urban,
  title={Urban mobility models for vanets},
  author={Mahajan, Atulya and Potnis, Niranjan and Gopalan, Kartik and Wang, Andy},
  booktitle={2nd IEEE International Workshop on Next Generation Wireless Networks},
  volume={33},
  year={2006}
}

@inproceedings{Rakhshan2016packet,
  title={Packet success probability derivation in a vehicular ad hoc network for a highway scenario},
  author={Rakhshan, Ali and Pishro-Nik, Hossein},
  booktitle={2016 Annual Conference on Information Science and Systems (CISS)},
  pages={210--215},
  year={2016},
  organization={IEEE}
}

@inproceedings{Rakhshan2016CISS,
	Author = {Rakhshan, Ali and Pishro-Nik, Hossein},
	Booktitle = {Conference on Information Sciences and Systems},
	Organization = {IEEE},
	Pages = {210--215},
	Title = {Packet Success Probability Derivation in a Vehicular Ad Hoc Network for a Highway Scenario},
	Year = {2016}}

@article{Nekoui2013Journal,
	Author = {Nekoui, Mohammad and Pishro-Nik, Hossein},
	Journal = {Journal on Selected Areas in Communications, Special Issue on Emerging Technologies in Communications},
	Number = {9},
	Pages = {491--503},
	Publisher = {IEEE},
	Title = {Analytic Design of Active Safety Systems for Vehicular Ad hoc Networks},
	Volume = {31},
	Year = {2013}}


@inproceedings{Nekoui2011MOBICOM,
	Author = {Nekoui, Mohammad and Pishro-Nik, Hossein},
	Booktitle = {MOBICOM workshop on VehiculAr InterNETworking},
	Organization = {ACM},
	Title = {Analytic Design of Active Vehicular Safety Systems in Sparse Traffic},
	Year = {2011}}

@inproceedings{Nekoui2011VTC,
	Author = {Nekoui, Mohammad and Pishro-Nik, Hossein},
	Booktitle = {VTC-Fall},
	Organization = {IEEE},
	Title = {Analytical Design of Inter-vehicular Communications for Collision Avoidance},
	Year = {2011}}

@inproceedings{Bovee2011VTC,
	Author = {Bovee, Ben Louis and Nekoui, Mohammad and Pishro-Nik, Hossein},
	Booktitle = {VTC-Fall},
	Organization = {IEEE},
	Title = {Evaluation of the Universal Geocast Scheme For VANETs},
	Year = {2011}}

@inproceedings{Nekoui2010MOBICOM,
	Author = {Nekoui, Mohammad and Pishro-Nik, Hossein},
	Booktitle = {MOBICOM},
	Organization = {ACM},
	Title = {Fundamental Tradeoffs in Vehicular Ad Hoc Networks},
	Year = {2010}}

@inproceedings{Nekoui2010IVCS,
	Author = {Nekoui, Mohammad and Pishro-Nik, Hossein},
	Booktitle = {IVCS},
	Organization = {IEEE},
	Title = {A Universal Geocast Scheme for Vehicular Ad Hoc Networks},
	Year = {2010}}

@inproceedings{Nekoui2009ITW,
	Author = {Nekoui, Mohammad and Pishro-Nik, Hossein},
	Booktitle = {IEEE Communications Society Conference on Sensor, Mesh and Ad Hoc Communications and Networks Workshops},
	Organization = {IEEE},
	Pages = {1--3},
	Title = {A Geometrical Analysis of Obstructed Wireless Networks},
	Year = {2009}}

@article{Eslami2013Journal,
	Author = {Eslami, Ali and Nekoui, Mohammad and Pishro-Nik, Hossein and Fekri, Faramarz},
	Journal = {ACM Transactions on Sensor Networks},
	Number = {4},
	Pages = {51},
	Publisher = {ACM},
	Title = {Results on finite wireless sensor networks: Connectivity and coverage},
	Volume = {9},
	Year = {2013}}


@article{Jiafu2014Journal,
	Author = {Jiafu, W. and Zhang, D. and Zhao, S. and Yang, L. and Lloret, J.},
	Journal = {Communications Magazine},
	Number = {8},
	Pages = {106-113},
	Publisher = {IEEE},
	Title = {Context-aware vehicular cyber-physical systems with cloud support: architecture, challenges, and solutions},
	Volume = {52},
	Year = {2014}}

@inproceedings{Haas2010ACM,
	Author = {Haas, J.J. and Hu, Y.},
	Booktitle = {international workshop on VehiculAr InterNETworking},
	Organization = {ACM},
	Title = {Communication requirements for crash avoidance.},
	Year = {2010}}

@inproceedings{Yi2008GLOBECOM,
	Author = {Yi, Q. and Lu, K. and Moayeri, N.},
	Booktitle = {GLOBECOM},
	Organization = {IEEE},
	Title = {CA Secure VANET MAC Protocol for DSRC Applications.},
	Year = {2008}}

@inproceedings{Mughal2010ITSim,
	Author = {Mughal, B.M. and Wagan, A. and Hasbullah, H.},
	Booktitle = {International Symposium on Information Technology (ITSim)},
	Organization = {IEEE},
	Title = {Efficient congestion control in VANET for safety messaging.},
	Year = {2010}}

@article{Chang2011Journal,
	Author = {Chang, Y. and Lee, C. and Copeland, J.},
	Journal = {Selected Areas in Communications},
	Pages = {236 –249},
	Publisher = {IEEE},
	Title = {Goodput enhancement of VANETs in noisy CSMA/CA channels},
	Volume = {29},
	Year = {2011}}

@article{Garcia-Costa2011Journal,
	Author = {Garcia-Costa, C. and Egea-Lopez, E. and Tomas-Gabarron, J. B. and Garcia-Haro, J. and Haas, Z. J.},
	Journal = {Transactions on Intelligent Transportation Systems},
	Pages = {1 –16},
	Publisher = {IEEE},
	Title = {A stochastic model for chain collisions of vehicles equipped with vehicular communications},
	Volume = {99},
	Year = {2011}}

@article{Carbaugh2011Journal,
	Author = {Carbaugh, J. and Godbole,  D. N. and Sengupta, R. and Garcia-Haro, J. and Haas, Z. J.},
	Publisher = {Transportation Research Part C (Emerging Technologies)},
	Title = {Safety and capacity analysis of automated and manual highway systems},
	Year = {1997}}

@article{Goh2004Journal,
	Author = {Goh, P. and Wong, Y.},
	Publisher = {Appl Health Econ Health Policy},
	Title = {Driver perception response time during the signal change interval},
	Year = {2004}}

@article{Chang1985Journal,
	Author = {Chang, M.S. and Santiago, A.J.},
	Pages = {20-30},
	Publisher = {Transportation Research Record},
	Title = {Timing traffic signal changes based on driver behavior},
	Volume = {1027},
	Year = {1985}}

@article{Green2000Journal,
	Author = {Green, M.},
	Pages = {195-216},
	Publisher = {Transportation Human Factors},
	Title = {How long does it take to stop? Methodological analysis of driver perception-brake times.},
	Year = {2000}}

@article{Koppa2005,
	Author = {Koppa, R.J.},
	Pages = {195-216},
	Publisher = {http://www.fhwa.dot.gov/publications/},
	Title = {Human Factors},
	Year = {2005}}

@inproceedings{Maxwell2010ETC,
	Author = {Maxwell, A. and Wood, K.},
	Booktitle = {Europian Transport Conference},
	Organization = {http://www.etcproceedings.org/paper/review-of-traffic-signals-on-high-speed-roads},
	Title = {Review of Traffic Signals on High Speed Road},
	Year = {2010}}

@article{Wortman1983,
	Author = {Wortman, R.H. and Matthias, J.S.},
	Publisher = {Arizona Department of Transportation},
	Title = {An Evaluation of Driver Behavior at Signalized Intersections},
	Year = {1983}}
@inproceedings{Zhang2007IASTED,
	Author = {Zhang, X. and Bham, G.H.},
	Booktitle = {18th IASTED International Conference: modeling and simulation},
	Title = {Estimation of driver reaction time from detailed vehicle trajectory data.},
	Year = {2007}}


@inproceedings{bai2003important,
  title={IMPORTANT: A framework to systematically analyze the Impact of Mobility on Performance of RouTing protocols for Adhoc NeTworks},
  author={Bai, Fan and Sadagopan, Narayanan and Helmy, Ahmed},
  booktitle={INFOCOM 2003. Twenty-second annual joint conference of the IEEE computer and communications. IEEE societies},
  volume={2},
  pages={825--835},
  year={2003},
  organization={IEEE}
}


@inproceedings{abedi2008enhancing,
	  title={Enhancing AODV routing protocol using mobility parameters in VANET},
	  author={Abedi, Omid and Fathy, Mahmood and Taghiloo, Jamshid},
	  booktitle={Computer Systems and Applications, 2008. AICCSA 2008. IEEE/ACS International Conference on},
	  pages={229--235},
	  year={2008},
	  organization={IEEE}
	}


@article{AlSultan2013Journal,
	Author = {Al-Sultan, Saif and Al-Bayatti, Ali H. and Zedan, Hussien},
	Journal = {IEEE Transactions on Vehicular Technology},
	Number = {9},
	Pages = {4264-4275},
	Publisher = {IEEE},
	Title = {Context Aware Driver Behaviour Detection System in Intelligent Transportation Systems},
	Volume = {62},
	Year = {2013}}






@article{Leow2008ITS,
	Author = {Leow, Woei Ling and Ni, Daiheng and Pishro-Nik, Hossein},
	Journal = {IEEE Transactions on Intelligent Transportation Systems},
	Number = {2},
	Pages = {369--374},
	Publisher = {IEEE},
	Title = {A Sampling Theorem Approach to Traffic Sensor Optimization},
	Volume = {9},
	Year = {2008}}



@article{REU2007,
	Author = {D. Ni and H. Pishro-Nik and R. Prasad and M. R. Kanjee and H. Zhu and T. Nguyen},
	Journal = {in 14th World Congress on Intelligent Transport Systems},
	Title = {Development of a prototype intersection collision avoidance system under VII},
	Year = {2007}}




@inproceedings{salamatian2013hide,
  title={How to hide the elephant-or the donkey-in the room: Practical privacy against statistical inference for large data.},
  author={Salamatian, Salman and Zhang, Amy and du Pin Calmon, Flavio and Bhamidipati, Sandilya and Fawaz, Nadia and Kveton, Branislav and Oliveira, Pedro and Taft, Nina},
  booktitle={GlobalSIP},
  pages={269--272},
  year={2013}
}

@article{sankar2013utility,
  title={Utility-privacy tradeoffs in databases: An information-theoretic approach},
  author={Sankar, Lalitha and Rajagopalan, S Raj and Poor, H Vincent},
  journal={Information Forensics and Security, IEEE Transactions on},
  volume={8},
  number={6},
  pages={838--852},
  year={2013},
  publisher={IEEE}
}
@inproceedings{ghinita2007prive,
  title={PRIVE: anonymous location-based queries in distributed mobile systems},
  author={Ghinita, Gabriel and Kalnis, Panos and Skiadopoulos, Spiros},
  booktitle={Proceedings of the 16th international conference on World Wide Web},
  pages={371--380},
  year={2007},
  organization={ACM}
}

@article{beresford2004mix,
  title={Mix zones: User privacy in location-aware services},
  author={Beresford, Alastair R and Stajano, Frank},
  year={2004},
  publisher={IEEE}
}

%@inproceedings{Mont1610Achieving,
%  title={Achieving Perfect Location Privacy in Markov Models Using Anonymization},
%  author={Montazeri, Zarrin and Houmansadr, Amir and H.Pishro-Nik},
%  booktitle="2016 International Symposium on Information Theory and its Applications
%  (ISITA2016)",
%  address="Monterey, USA",
%  days=30,
%  month=oct,
%  year=2016,
%}

@article{csiszar1996almost,
  title={Almost independence and secrecy capacity},
  author={Csisz{\'a}r, Imre},
  journal={Problemy Peredachi Informatsii},
  volume={32},
  number={1},
  pages={48--57},
  year={1996},
  publisher={Russian Academy of Sciences, Branch of Informatics, Computer Equipment and Automatization}
}

@article{yamamoto1983source,
  title={A source coding problem for sources with additional outputs to keep secret from the receiver or wiretappers (corresp.)},
  author={Yamamoto, Hirosuke},
  journal={IEEE Transactions on Information Theory},
  volume={29},
  number={6},
  pages={918--923},
  year={1983},
  publisher={IEEE}
}


@inproceedings{calmon2015fundamental,
  title={Fundamental limits of perfect privacy},
  author={Calmon, Flavio P and Makhdoumi, Ali and M{\'e}dard, Muriel},
  booktitle={Information Theory (ISIT), 2015 IEEE International Symposium on},
  pages={1796--1800},
  year={2015},
  organization={IEEE}
}



@inproceedings{Lehman1999Large-Sample-Theory,
	title={Elements of Large Sample Theory},
	author={E. L. Lehman},
	organization={Springer},
	year={1999}
}


@inproceedings{Ferguson1999Large-Sample-Theory,
	title={A Course in Large Sample Theory},
	author={Thomas S. Ferguson},
	organization={CRC Press},
	year={1996}
}



@inproceedings{Dembo1999Large-Deviations,
	title={Large Deviation Techniques and Applications, Second Edition},
	author={A. Dembo and O. Zeitouni},
	organization={Springer},
	year={1998}
}


%%%%%%%%%%%%%%%%%%%%%%%%%%%%%%%%%%%%%%%%%%%%%%%%
Hossein's Coding Journals
%%%%%%%%%%%%%%%%%%%%%%

@ARTICLE{myoptics,
  AUTHOR =       "H. Pishro-Nik and N. Rahnavard and J. Ha and F. Fekri and A. Adibi ",
  TITLE =        "Low-density parity-check codes for volume holographic memory systems",
  JOURNAL =      " Appl. Opt.",
  YEAR =         "2003",
  volume =       "42",
  pages =        "861-870  "
 }






@ARTICLE{myit,
  AUTHOR =       "H. Pishro-Nik and F. Fekri  ",
  TITLE =        "On Decoding of Low-Density Parity-Check Codes on the Binary Erasure Channel",
  JOURNAL =      "IEEE Trans. Inform. Theory",
  YEAR =         "2004",
  volume =       "50",
  pages =        "439--454"
  }




@ARTICLE{myitpuncture,
  AUTHOR =       "H. Pishro-Nik and F. Fekri  ",
  TITLE =        "Results on Punctured Low-Density Parity-Check Codes and Improved Iterative Decoding Techniques",
  JOURNAL =      "IEEE Trans. on Inform. Theory",
  YEAR =         "2007",
  volume =       "53",
  number=        "2",
  pages =        "599--614",
  month= "February"
  }




@ARTICLE{myitlinmimdist,
  AUTHOR =       "H. Pishro-Nik and F. Fekri",
  TITLE =        "Performance of Low-Density Parity-Check Codes With Linear Minimum Distance",
  JOURNAL =         "IEEE Trans. Inform. Theory ",
  YEAR =         "2006",
  volume =       "52",
  number="1",
  pages =        "292 --300"
  }






@ARTICLE{myitnonuni,
  AUTHOR =       "H. Pishro-Nik and N. Rahnavard and F. Fekri  ",
  TITLE =        "Non-uniform Error Correction Using Low-Density Parity-Check Codes",
  JOURNAL =      "IEEE Trans. Inform. Theory",
  YEAR =         "2005",
  volume =       "51",
  number=  "7",
  pages =        "2702--2714"
 }





@article{eslamitcomhybrid10,
 author = {A. Eslami and S. Vangala and H. Pishro-Nik},
 title = {Hybrid channel codes for highly efficient FSO/RF communication systems},
 journal = {IEEE Transactions on Communications},
 volume = {58},
 number = {10},
 year = {2010},
 pages = {2926--2938},
 }


@article{eslamitcompolar13,
 author = {A. Eslami and H. Pishro-Nik},
 title = {On Finite-Length Performance of Polar Codes: Stopping Sets, Error Floor, and Concatenated Design},
 journal = {IEEE Transactions on Communications},
 volume = {61},
 number = {13},
 year = {2013},
 pages = {919--929},
 }



 @article{saeeditcom11,
 author = {H. Saeedi and H. Pishro-Nik and  A. H. Banihashemi},
 title = {Successive maximization for the systematic design of universally capacity approaching rate-compatible
 sequences of LDPC code ensembles over binary-input output-symmetric memoryless channels},
 journal = {IEEE Transactions on Communications},
 year = {2011},
 volume={59},
 number = {7}
 }


@article{rahnavard07,
 author = {Rahnavard, N. and Pishro-Nik, H. and Fekri, F.},
 title = {Unequal Error Protection Using Partially Regular LDPC Codes},
 journal = {IEEE Transactions on Communications},
 year = {2007},
 volume = {55},
 number = {3},
 pages = {387 -- 391}
 }


 @article{hosseinira04,
 author = {H. Pishro-Nik and F. Fekri},
 title = {Irregular repeat-accumulate codes for volume holographic memory systems},
 journal = {Journal of Applied Optics},
 year = {2004},
 volume = {43},
 number = {27},
 pages = {5222--5227},
 }


@article{azadeh2015Ephemeralkey,
 author = {A. Sheikholeslami and D. Goeckel and H. Pishro-Nik},
 title = {Jamming Based on an Ephemeral Key to Obtain Everlasting Security in Wireless Environments},
 journal = {IEEE Transactions on Wireless Communications},
 year = {2015},
 volume = {14},
 number = {11},
 pages = {6072--6081},
}


@article{azadeh2014Everlasting,
 author = {A. Sheikholeslami and D. Goeckel and H. Pishro-Nik},
 title = {Everlasting secrecy in disadvantaged wireless environments against sophisticated eavesdroppers},
 journal = {48th Asilomar Conference on Signals, Systems and Computers},
 year = {2014},
 pages = {1994--1998},
}


@article{azadeh2013ISIT,
 author = {A. Sheikholeslami and D. Goeckel and H. Pishro-Nik},
 title = {Artificial intersymbol interference (ISI) to exploit receiver imperfections for secrecy},
 journal = {IEEE International Symposium on Information Theory (ISIT)},
 year = {2013},
}


@article{azadeh2013Jsac,
 author = {A. Sheikholeslami and D. Goeckel and H. Pishro-Nik},
 title = {Jamming Based on an Ephemeral Key to Obtain Everlasting Security in Wireless Environments},
 journal = {IEEE Journal on Selected Areas in Communications},
 year = {2013},
 volume = {31},
 number = {9},
 pages = {1828--1839},
}


@article{azadeh2012Allerton,
 author = {A. Sheikholeslami and D. Goeckel and H. Pishro-Nik},
 title = {Exploiting the non-commutativity of nonlinear operators for information-theoretic security in disadvantaged wireless environments},
 journal = {50th Annual Allerton Conference on Communication, Control, and Computing},
 year = {2012},
 pages = {233--240},
}


@article{azadeh2012Infocom,
 author = {A. Sheikholeslami and D. Goeckel and H. Pishro-Nik},
 title = {Jamming Based on an Ephemeral Key to Obtain Everlasting Security in Wireless Environments},
 journal = {IEEE INFOCOM},
 year = {2012},
 pages = {1179--1187},
}

@article{1corser2016evaluating,
  title={Evaluating Location Privacy in Vehicular Communications and Applications},
  author={Corser, George P and Fu, Huirong and Banihani, Abdelnasser},
  journal={IEEE Transactions on Intelligent Transportation Systems},
  volume={17},
  number={9},
  pages={2658-2667},
  year={2016},
  publisher={IEEE}
}
@article{2zhang2016designing,
  title={On Designing Satisfaction-Ratio-Aware Truthful Incentive Mechanisms for k-Anonymity Location Privacy},
  author={Zhang, Yuan and Tong, Wei and Zhong, Sheng},
  journal={IEEE Transactions on Information Forensics and Security},
  volume={11},
  number={11},
  pages={2528--2541},
  year={2016},
  publisher={IEEE}
}
@article{3li2016privacy,
  title={Privacy-preserving Location Proof for Securing Large-scale Database-driven Cognitive Radio Networks},
  author={Li, Yi and Zhou, Lu and Zhu, Haojin and Sun, Limin},
  journal={IEEE Internet of Things Journal},
  volume={3},
  number={4},
  pages={563-571},
  year={2016},
  publisher={IEEE}
}
@article{4olteanu2016quantifying,
  title={Quantifying Interdependent Privacy Risks with Location Data},
  author={Olteanu, Alexandra-Mihaela and Huguenin, K{\'e}vin and Shokri, Reza and Humbert, Mathias and Hubaux, Jean-Pierre},
  journal={IEEE Transactions on Mobile Computing},
  year={2016},
  volume={PP},
  number={99},
  pages={1-1},
  publisher={IEEE}
}
@article{5yi2016practical,
  title={Practical Approximate k Nearest Neighbor Queries with Location and Query Privacy},
  author={Yi, Xun and Paulet, Russell and Bertino, Elisa and Varadharajan, Vijay},
  journal={IEEE Transactions on Knowledge and Data Engineering},
  volume={28},
  number={6},
  pages={1546--1559},
  year={2016},
  publisher={IEEE}
}
@article{6li2016privacy,
  title={Privacy Leakage of Location Sharing in Mobile Social Networks: Attacks and Defense},
  author={Li, Huaxin and Zhu, Haojin and Du, Suguo and Liang, Xiaohui and Shen, Xuemin},
  journal={IEEE Transactions on Dependable and Secure Computing},
  year={2016},
  volume={PP},
  number={99},
  publisher={IEEE}
}

@article{7murakami2016localization,
  title={Localization Attacks Using Matrix and Tensor Factorization},
  author={Murakami, Takao and Watanabe, Hajime},
  journal={IEEE Transactions on Information Forensics and Security},
  volume={11},
  number={8},
  pages={1647--1660},
  year={2016},
  publisher={IEEE}
}
@article{8zurbaran2015near,
  title={Near-Rand: Noise-based Location Obfuscation Based on Random Neighboring Points},
  author={Zurbaran, Mayra Alejandra and Avila, Karen and Wightman, Pedro and Fernandez, Michael},
  journal={IEEE Latin America Transactions},
  volume={13},
  number={11},
  pages={3661--3667},
  year={2015},
  publisher={IEEE}
}

@article{9tan2014anti,
  title={An anti-tracking source-location privacy protection protocol in wsns based on path extension},
  author={Tan, Wei and Xu, Ke and Wang, Dan},
  journal={IEEE Internet of Things Journal},
  volume={1},
  number={5},
  pages={461--471},
  year={2014},
  publisher={IEEE}
}

@article{10peng2014enhanced,
  title={Enhanced Location Privacy Preserving Scheme in Location-Based Services},
  author={Peng, Tao and Liu, Qin and Wang, Guojun},
  journal={IEEE Systems Journal},
  year={2014},
  volume={PP},
  number={99},
  pages={1-12},
  publisher={IEEE}
}
@article{11dewri2014exploiting,
  title={Exploiting service similarity for privacy in location-based search queries},
  author={Dewri, Rinku and Thurimella, Ramakrisha},
  journal={IEEE Transactions on Parallel and Distributed Systems},
  volume={25},
  number={2},
  pages={374--383},
  year={2014},
  publisher={IEEE}
}

@article{12hwang2014novel,
  title={A novel time-obfuscated algorithm for trajectory privacy protection},
  author={Hwang, Ren-Hung and Hsueh, Yu-Ling and Chung, Hao-Wei},
  journal={IEEE Transactions on Services Computing},
  volume={7},
  number={2},
  pages={126--139},
  year={2014},
  publisher={IEEE}
}
@article{13puttaswamy2014preserving,
  title={Preserving location privacy in geosocial applications},
  author={Puttaswamy, Krishna PN and Wang, Shiyuan and Steinbauer, Troy and Agrawal, Divyakant and El Abbadi, Amr and Kruegel, Christopher and Zhao, Ben Y},
  journal={IEEE Transactions on Mobile Computing},
  volume={13},
  number={1},
  pages={159--173},
  year={2014},
  publisher={IEEE}
}

@article{14zhang2014privacy,
  title={Privacy quantification model based on the Bayes conditional risk in Location-Based Services},
  author={Zhang, Xuejun and Gui, Xiaolin and Tian, Feng and Yu, Si and An, Jian},
  journal={Tsinghua Science and Technology},
  volume={19},
  number={5},
  pages={452--462},
  year={2014},
  publisher={TUP}
}

@article{15bilogrevic2014privacy,
  title={Privacy-preserving optimal meeting location determination on mobile devices},
  author={Bilogrevic, Igor and Jadliwala, Murtuza and Joneja, Vishal and Kalkan, K{\"u}bra and Hubaux, Jean-Pierre and Aad, Imad},
  journal={IEEE transactions on information forensics and security},
  volume={9},
  number={7},
  pages={1141--1156},
  year={2014},
  publisher={IEEE}
}
@article{16haghnegahdar2014privacy,
  title={Privacy Risks in Publishing Mobile Device Trajectories},
  author={Haghnegahdar, Alireza and Khabbazian, Majid and Bhargava, Vijay K},
  journal={IEEE Wireless Communications Letters},
  volume={3},
  number={3},
  pages={241--244},
  year={2014},
  publisher={IEEE}
}
@article{17malandrino2014verification,
  title={Verification and inference of positions in vehicular networks through anonymous beaconing},
  author={Malandrino, Francesco and Borgiattino, Carlo and Casetti, Claudio and Chiasserini, Carla-Fabiana and Fiore, Marco and Sadao, Roberto},
  journal={IEEE Transactions on Mobile Computing},
  volume={13},
  number={10},
  pages={2415--2428},
  year={2014},
  publisher={IEEE}
}
@article{18shokri2014hiding,
  title={Hiding in the mobile crowd: Locationprivacy through collaboration},
  author={Shokri, Reza and Theodorakopoulos, George and Papadimitratos, Panos and Kazemi, Ehsan and Hubaux, Jean-Pierre},
  journal={IEEE transactions on dependable and secure computing},
  volume={11},
  number={3},
  pages={266--279},
  year={2014},
  publisher={IEEE}
}
@article{19freudiger2013non,
  title={Non-cooperative location privacy},
  author={Freudiger, Julien and Manshaei, Mohammad Hossein and Hubaux, Jean-Pierre and Parkes, David C},
  journal={IEEE Transactions on Dependable and Secure Computing},
  volume={10},
  number={2},
  pages={84--98},
  year={2013},
  publisher={IEEE}
}
@article{20gao2013trpf,
  title={TrPF: A trajectory privacy-preserving framework for participatory sensing},
  author={Gao, Sheng and Ma, Jianfeng and Shi, Weisong and Zhan, Guoxing and Sun, Cong},
  journal={IEEE Transactions on Information Forensics and Security},
  volume={8},
  number={6},
  pages={874--887},
  year={2013},
  publisher={IEEE}
}
@article{21ma2013privacy,
  title={Privacy vulnerability of published anonymous mobility traces},
  author={Ma, Chris YT and Yau, David KY and Yip, Nung Kwan and Rao, Nageswara SV},
  journal={IEEE/ACM Transactions on Networking},
  volume={21},
  number={3},
  pages={720--733},
  year={2013},
  publisher={IEEE}
}
@article{22niu2013pseudo,
  title={Pseudo-Location Updating System for privacy-preserving location-based services},
  author={Niu, Ben and Zhu, Xiaoyan and Chi, Haotian and Li, Hui},
  journal={China Communications},
  volume={10},
  number={9},
  pages={1--12},
  year={2013},
  publisher={IEEE}
}
@article{23dewri2013local,
  title={Local differential perturbations: Location privacy under approximate knowledge attackers},
  author={Dewri, Rinku},
  journal={IEEE Transactions on Mobile Computing},
  volume={12},
  number={12},
  pages={2360--2372},
  year={2013},
  publisher={IEEE}
}
@inproceedings{24kanoria2012tractable,
  title={Tractable bayesian social learning on trees},
  author={Kanoria, Yashodhan and Tamuz, Omer},
  booktitle={Information Theory Proceedings (ISIT), 2012 IEEE International Symposium on},
  pages={2721--2725},
  year={2012},
  organization={IEEE}
}
@inproceedings{25farias2005universal,
  title={A universal scheme for learning},
  author={Farias, Vivek F and Moallemi, Ciamac C and Van Roy, Benjamin and Weissman, Tsachy},
  booktitle={Proceedings. International Symposium on Information Theory, 2005. ISIT 2005.},
  pages={1158--1162},
  year={2005},
  organization={IEEE}
}
@inproceedings{26misra2013unsupervised,
  title={Unsupervised learning and universal communication},
  author={Misra, Vinith and Weissman, Tsachy},
  booktitle={Information Theory Proceedings (ISIT), 2013 IEEE International Symposium on},
  pages={261--265},
  year={2013},
  organization={IEEE}
}
@inproceedings{27ryabko2013time,
  title={Time-series information and learning},
  author={Ryabko, Daniil},
  booktitle={Information Theory Proceedings (ISIT), 2013 IEEE International Symposium on},
  pages={1392--1395},
  year={2013},
  organization={IEEE}
}
@inproceedings{28krzakala2013phase,
  title={Phase diagram and approximate message passing for blind calibration and dictionary learning},
  author={Krzakala, Florent and M{\'e}zard, Marc and Zdeborov{\'a}, Lenka},
  booktitle={Information Theory Proceedings (ISIT), 2013 IEEE International Symposium on},
  pages={659--663},
  year={2013},
  organization={IEEE}
}
@inproceedings{29sakata2013sample,
  title={Sample complexity of Bayesian optimal dictionary learning},
  author={Sakata, Ayaka and Kabashima, Yoshiyuki},
  booktitle={Information Theory Proceedings (ISIT), 2013 IEEE International Symposium on},
  pages={669--673},
  year={2013},
  organization={IEEE}
}
@inproceedings{30predd2004consistency,
  title={Consistency in a model for distributed learning with specialists},
  author={Predd, Joel B and Kulkarni, Sanjeev R and Poor, H Vincent},
  booktitle={IEEE International Symposium on Information Theory},
  year={2004},
organization={IEEE}
}
@inproceedings{31nokleby2016rate,
  title={Rate-Distortion Bounds on Bayes Risk in Supervised Learning},
  author={Nokleby, Matthew and Beirami, Ahmad and Calderbank, Robert},
  booktitle={2016 IEEE International Symposium on Information Theory (ISIT)},
pages={2099-2103},
  year={2016},
organization={IEEE}
}

@inproceedings{32le2016imperfect,
  title={Are imperfect reviews helpful in social learning?},
  author={Le, Tho Ngoc and Subramanian, Vijay G and Berry, Randall A},
  booktitle={Information Theory (ISIT), 2016 IEEE International Symposium on},
  pages={2089--2093},
  year={2016},
  organization={IEEE}
}
@inproceedings{33gadde2016active,
  title={Active Learning for Community Detection in Stochastic Block Models},
  author={Gadde, Akshay and Gad, Eyal En and Avestimehr, Salman and Ortega, Antonio},
  booktitle={2016 IEEE International Symposium on Information Theory (ISIT)},
  pages={1889-1893},
  year={2016}
}
@inproceedings{34shakeri2016minimax,
  title={Minimax Lower Bounds for Kronecker-Structured Dictionary Learning},
  author={Shakeri, Zahra and Bajwa, Waheed U and Sarwate, Anand D},
  booktitle={2016 IEEE International Symposium on Information Theory (ISIT)},
  pages={1148-1152},
  year={2016}
}
@article{35lee2015speeding,
  title={Speeding up distributed machine learning using codes},
  author={Lee, Kangwook and Lam, Maximilian and Pedarsani, Ramtin and Papailiopoulos, Dimitris and Ramchandran, Kannan},
  booktitle={2016 IEEE International Symposium on Information Theory (ISIT)},
  pages={1143-1147},
  year={2016}
}
@article{36oneto2016statistical,
  title={Statistical Learning Theory and ELM for Big Social Data Analysis},
  author={Oneto, Luca and Bisio, Federica and Cambria, Erik and Anguita, Davide},
  journal={ieee CompUTATionAl inTelliGenCe mAGAzine},
  volume={11},
  number={3},
  pages={45--55},
  year={2016},
  publisher={IEEE}
}
@article{37lin2015probabilistic,
  title={Probabilistic approach to modeling and parameter learning of indirect drive robots from incomplete data},
  author={Lin, Chung-Yen and Tomizuka, Masayoshi},
  journal={IEEE/ASME Transactions on Mechatronics},
  volume={20},
  number={3},
  pages={1036--1045},
  year={2015},
  publisher={IEEE}
}
@article{38wang2016towards,
  title={Towards Bayesian Deep Learning: A Framework and Some Existing Methods},
  author={Wang, Hao and Yeung, Dit-Yan},
  journal={IEEE Transactions on Knowledge and Data Engineering},
  volume={PP},
  number={99},
  year={2016},
  publisher={IEEE}
}


%%%%%Informationtheoreticsecurity%%%%%%%%%%%%%%%%%%%%%%%




@inproceedings{Bloch2011PhysicalSecBook,
	title={Physical-Layer Security},
	author={M. Bloch and J. Barros},
	organization={Cambridge University Press},
	year={2011}
}



@inproceedings{Liang2009InfoSecBook,
	title={Information Theoretic Security},
	author={Y. Liang and H. V. Poor and S. Shamai (Shitz)},
	organization={Now Publishers Inc.},
	year={2009}
}


@inproceedings{Zhou2013PhysicalSecBook,
	title={Physical Layer Security in Wireless Communications},
	author={ X. Zhou and L. Song and Y. Zhang},
	organization={CRC Press},
	year={2013}
}

@article{Ni2012IEA,
	Author = {D. Ni and H. Liu and W. Ding and  Y. Xie and H. Wang and H. Pishro-Nik and Q. Yu},
	Journal = {IEA/AIE},
	Title = {Cyber-Physical Integration to Connect Vehicles for Transformed Transportation Safety and Efficiency},
	Year = {2012}}



@inproceedings{Ni2012Inproceedings,
	Author = {D. Ni, H. Liu, Y. Xie, W. Ding, H. Wang, H. Pishro-Nik, Q. Yu and M. Ferreira},
	Booktitle = {Spring Simulation Multiconference},
	Date-Added = {2016-09-04 14:18:42 +0000},
	Date-Modified = {2016-09-06 16:22:14 +0000},
	Title = {Virtual Lab of Connected Vehicle Technology},
	Year = {2012}}

@inproceedings{Ni2012Inproceedings,
	Author = {D. Ni, H. Liu, W. Ding, Y. Xie, H. Wang, H. Pishro-Nik and Q. Yu,},
	Booktitle = {IEA/AIE},
	Date-Added = {2016-09-04 09:11:02 +0000},
	Date-Modified = {2016-09-06 14:46:53 +0000},
	Title = {Cyber-Physical Integration to Connect Vehicles for Transformed Transportation Safety and Efficiency},
	Year = {2012}}


@article{Nekoui_IJIPT_2009,
	Author = {M. Nekoui and D. Ni and H. Pishro-Nik and R. Prasad and M. Kanjee and H. Zhu and T. Nguyen},
	Journal = {International Journal of Internet Protocol Technology (IJIPT)},
	Number = {3},
	Pages = {},
	Publisher = {},
	Title = {Development of a VII-Enabled Prototype Intersection Collision Warning System},
	Volume = {4},
	Year = {2009}}


@inproceedings{Pishro_Ganz_Ni,
	Author = {H. Pishro-Nik, A. Ganz, and Daiheng Ni},
	Booktitle = {Forty-Fifth Annual Allerton Conference on Communication, Control, and Computing. Allerton House, Monticello, IL},
	Date-Added = {},
	Date-Modified = {},
	Number = {},
	Pages = {},
	Title = {The capacity of vehicular ad hoc networks},
	Volume = {},
	Year = {September 26-28, 2007}}

@inproceedings{Leow_Pishro_Ni_1,
	Author = {W. L. Leow, H. Pishro-Nik and Daiheng Ni},
	Booktitle = {IEEE Global Telecommunications Conference, Washington, D.C.},
	Date-Added = {},
	Date-Modified = {},
	Number = {},
	Pages = {},
	Title = {Delay and Energy Tradeoff in Multi-state Wireless Sensor Networks},
	Volume = {},
	Year = {November 26-30, 2007}}


@misc{UMass-Trans,
title = {{UMass Transportation Center}},
note = {\url{http://www.umasstransportationcenter.org/}},
}


@inproceedings{Haenggi2013book,
	title={Stochastic geometry for wireless networks},
	author={M. Haenggi},
	organization={Cambridge Uinversity Press},
	year={2013}
}


\begin{references}

\bibitem{li1}
Xu, W.; Wang, J.; Ding, F.; Chen, X.; Nasybulin, E.; Zhang, Y.; Zhang, J.-G.
Lithium Metal Anodes for Rechargeable Batteries.
{\it Energy Envir.~Sci.} {\bf 2014}, {\it 7}, 513-537.

\bibitem{pnnl}
Ding, F.; Xu, W.; Graff, G.L.; Zhang, J.; Sushko, M.L.; Chen, X.; Shao, Y.;
Engelhard, M.H.; Nie, Z.; Xiao, J.; Liu, X.; Sushko, P.V.; Liu, J., 
Zhang, J.-G.  Dendrite-free Lithium Deposition via Self-Healing Electrostatic
Shield Mechanism.  {\it J.~Am.~Chem.~Soc.} {\bf 2013}, {\it 135}, 4450-4456.

\bibitem{cui}
Liu, W.; Lin, D.; Pei, A.; Cui, Y.
Stabilizing Lithium Metal Anodes by Uniform Li-Ion Flux Distribution
in Nanochannel Confinement.
{\it J.~Am.~Chem.~Soc.} {\bf 2016}, {\it 138}, 15443-15450.

\bibitem{archer}
Choudhury, S.; Mangal, R.; Agrawal, A.; Archer, L.A.
A Highly Reversible Room-Temperature Lithium Metal Battery Based on
Crosslinked Hairy Nanoparticles.
{\it Nat.~Commun.} {\bf 2016}, {\it 6}, 10101.

\bibitem{jungjohann}
Leenheer, A.J.; Jungjohann, K.L.; Zavadil, K.R.; Sullivan, J.P.; Harris,
C.T.  Lithium Electrodeposition Dynamics in Aprotic Electrolyte Observed
{\it in Situ via} Transmission Electron Microscopy.
{\it ACS Nano} {\bf 2015}, {\it 9}, 4379-4389.

\bibitem{dudney1}
Sacci, R.L.; Black, J.M.; Balke, N.; Dudney, N.J.; Moore, K.L.; Unocic, R.R.  
Nanoscale Imaging of Fundamental Li Battery Chemistry: Solid-Electrolyte
Interphase Formation and Preferential Growth of Lithium Metal Nanoclusters.
{\it Nano Lett.} {\bf 2015}, {\it 15}, 2011-2018.

\bibitem{pearse}
Kozen, A.C.; Lin, C.F.; Pearse, A.J.; Schroeder, M.A.; Han, X.G.; 
Hu, L.B.; Lee, S.B.; Rubloff, G.W.; Noked, M. 
Next-generation Lithium Metal Anode Engineering via Atomic Layer Deposition.
{\it ACS Nano} {\bf 2015}, {\it 9}, 5884-5892.

\bibitem{dudney}
Ma, C.; Cheng, Y.; Yin, K.; Luo, J.; Sharafi, A.; Sakamoto, J.; Li, J.; 
More, K.L.; Dudney, N.J.; Chi, M.  
Interfacial Stability of Li Metal-Solid Electrolyte Elucidated via {\it in Situ}
Electron Microscopy.  {\it Nano Lett.} {\bf 2016}, {\it 16}, 7030-7036.

\bibitem{naturetem}
Harry, K.J.; Hallinan, D.T.; Parkinson, D.Y.; MacDowell, A.A.; Balsara, N.P.
Detection of Subsurface Structures Underneath Dendrites Formed on Cycled
Lithium Metal Electrodes.  {\it Nature Materials} {\bf 2014}, {\it 13}, 69-73.

\bibitem{garcia}
Jana, A.; Ely, D.R.; Garcia, R.E.  
Dendrite-separator Interactions in Lithium-based Batteries.
{\it J.~Power~Sources} {\bf 2015}, {\it 275}, 912-921.

\bibitem{newman}
Ferrese, A.; Newman, J.
Mechanical Deformation of a Lithium-Metal Anode due to a Very Stiff
Separator, {\it J.~Electrochem.~Soc.} {\bf 2014}, {\it 161}, A1350-A1359,
and references therein.

\bibitem{santosh1}
Santosh, K.C.; Xiong, K.; Longo, R.C.; Cho, K.  
Interface Phenomena Between Li Anode and Lithium Phosphate Electrolyte for
Li-Ion Battery.  {\it J.~Power~Sources} {\bf 2014}, {\it 244}, 136-142.

\bibitem{holzwarth}
Lepley, N.D.; Holzwarth, N.A.W.  
Modeling Interfaces Between Solids: Application to Li Battery Materials.
{\it Phys.~Rev.~B} {\bf 2015}, {\it 92}, 214201.

\bibitem{yue1}
Liu, Z.; Qi, Y.; Lin, Y.X.; Chen, L.; Lu, P.; Chen, L.Q.
Interfacial Study on Solid Electrolyte Interphase at Li Metal Anode:
Implication for Li Dendrite Growth.
{\it J.~Electrochem.~Soc.} {\bf 2016}, {\it 163}, A592-A598.

\bibitem{yue}
Lin, Y.-X.; Liu, Z.; Leung, K.; Chen, L.Q.; Lu, P.; Qi, Y.
Connecting the Irreversible Capacity Loss in Li-ion Batteries with the
Electronic Insulating Properties of Solid Electrolyte Interphase (SEI)
Components.  {\it J.~Power Sources} {\bf 2016}, {\it 309}, 221-230.

\bibitem{harris}
Harris, S.J.; Lu, P.  Effects of Inhomogeneities -- Nanoscale to
Mesoscale -- on the Durability of Li-Ion Batteries.  {\it J.~Phys.~Chem.~C}
{\bf 2013}, {\it 117}, 6481-6492.

\bibitem{pore}
Ren, Y.; Shen, Y.; Lin, Y., Nan, C.-W.
Direct Observation of Lithium Dendrites inside Garnet-type Lithium-ion
Solid Electrolyte.
{\it Electrochem.~Commun.} {\bf 2015}, {\it 57}, 27-30.

\bibitem{pore1}
Cheng, E.J.; Sharafi, A.; Sakamoto, J.
Intergranular Li Metal Propagation Through Polycrystalline
Li$_{6.25}$Al$_{0.25}$La$_3$Zr$_2$O$_{12}$ Ceramic Electrolyte.
{\it Electrochim.~Acta} {\bf 2017}, {\it 223}, 85-91.

\bibitem{shluger}
McKenna, K.P.; Shluger, A.L.  
Electron-trapping Polycrystalline Materials with Negative Electron Affinity.
{\it Nature Materials} {\bf 2008}, {\it 7}, 859-862.

\bibitem{liair}
Geng, W.T.; He, B.L.; Ohno, T.  Grain Boundary Induced Conductivity in 
Li$_2$O$_2$.  {\it J.~Phys.~Chem.~C} {\bf 2013}, {\it 117}, 25222-25228.

\bibitem{fisher}
Fisher, C.A.J.; Matsubara, H.  Molecular Dynamics Investigations of
Grain Boundary Phenomena in Cubic Zirconia.  {\it Comput.~Mater.~Sci.}
{\bf 1999}, {\it 14}, 177-184.

\bibitem{islam}
Olson, C.L.; Nelson, J.; Islam, M.S.  
Defect Chemistry, Surface Structures, and Lithium Insertion in Anatase TiO$_2$.
{\it J.~Phys.~Chem.} {\bf 2006}, {\it 110}, 9995-10001.

\bibitem{twin}
Nie, A.; Gan, L.-Y.; Cheng, Y.; Li, Q.; Yuan, Y.; Mashayek, F.; 
Wang, H.; Klie, R.; Schwingenschlogl, U.; Shahbasizn-Yassar, R.
Twin Boundary-Assisted Lithium Ion Transport.  {\it Nano Lett.}
{\bf 2015}, {\it 15}, 610-615.

\bibitem{batt}
Leung, K.; Soto, F.; Hankins, K.; Balbuena, P.B.; Harrison, K.L.
Stability of Solid Electrolyte Interphase Components on Lithium
Metal and Reactive Anode Material Surfaces.
{\it J.~Phys.~Chem.~C} {\bf 2016}, {\it 120}, 6302-6313.

\bibitem{yueli2co3}
Shi, Q.; Lu, P.; Liu, Z.; Qi, Y.; Hector Jr., L.G.; Li, H.; Harris, S.J.
Direct Calculation of Li-ion Transport in the Solid Electrolyte Interphase.
{\it J.~Am.~Chem.~Soc.} {\bf 2012}, {\it 134}, 15476-15487.

\bibitem{tang}
Tang, M.; Newman, J. 
Why is the Solid-Electrolyte-Interphase Selective?  Through-Film
Ferrocenium Reduction on Highly Oriented Pyrolytic Graphite.
{\it J.~Electrochem.~Soc.} {\bf 2012}, {\it 159}, A1922-A1927.

\bibitem{qi16}
Zhang, Q.; Pan, J.; Lu, P.; Liu, Z.; Verbrugge, M.W.; Sheldon, B.W.;
Cheng, Y.-T.; Qi, Y.; Xiao, X.
Synergetic Effects of Inorganic Components in Solid Electrolyte Interphase
on High Cycle Efficiency of Lithium Ion Batteries.
{\it Nano Lett.} {\bf 2016}, {\it 16}, 2011-2016.

\bibitem{solid}
Leung,~K.; Leenheer, A.
How Voltage Drops are Manifested by Lithium Ion Configurations at Interfaces
and in Thin Films on Battery Electrodes.
{\it J.~Phys.~Chem.~C} {\bf 2015}, {\it 119}, 10234-10246.

\bibitem{crack}
Holland, D.; Marder, M.
Ideal Brittle Fracture of Silicon Studied with Molecular Dynamics.
{\it Phys.~Rev.~Lett.} {\bf 1998}, {\it 80}, 746-749.

\bibitem{dawson}
Dawson, J.A.; Chen, H.; Tanaka, I.  First-Principles Calculations of
Oxygen Vacancy Formation and Metallic Behavior at a $\beta$-MnO$_2$
Grain Boundary. {\it ACS Appl.~Mater.~Interface} {\bf 2015}, {\it 7},
1726-1734.

\bibitem{vasp1}
Kresse, G.; Furthm\"{u}ller,~J. 
Efficient Iterative Schemes for Ab Initio Total-Energy Calculations
Using a Plane-wave Basis Set.  {\it Phys.~Rev.~B} {\bf 1996}, {\it 54},
11169-11186. 

\bibitem{vasp1a}
Kresse, G.; Furthm\"{u}ller,~J. 
Efficiency of {\it Ab-initio} Total Energy Calculations for Metals and
Semiconductors using a Plane-Wave Basis Set.
{\it Comput.~Mater.~Sci.} {\bf 1996}, {\it 6}, 15-50.

\bibitem{vasp2}
Kresse~G.; Joubert,~D. 
From Ultrasoft Pseudopotentials to the Projector Augmented-Wave Method.
{\it Phys.~Rev.~B} {\bf 1999}, {\it 59}, 1758-1775.

\bibitem{vasp3}
Paier,~J.; Marsman,~M.; Kresse,~G. 
Why Does the B3LYP Hybrid Functional Fail for Metals?
{\it J. Chem. Phys.} {\bf 2007}, {\it 127}, 024103.

\bibitem{pbe}
Perdew,~J.P., Burke,~K.;  Ernzerhof,~.M.
Generalized Gradient Approximation Made Simple.
{\it Phys. Rev. Lett.} {\bf 1996}, {\it 77}, 3865-3868.

\bibitem{dipole}
Neugebauer, J.; Scheffler, M.  Adsorbate-Substrate and Adsorbate-Adsorbate
Interactions of Na and K adlayers on Al(111).
{\it Phys.~Rev.~B} {\bf 1992}, {\it 46}, 16067-16080.

\bibitem{hse06a}
Heyd, J.; Scuseria, G.E.; Ernzerhof, M.
Hybrid Functionals based on a Screened Coulomb Potential.
{\it J.~Chem.~Phys.} {\bf 2003}, {\it 118}, 8207-8215.

\bibitem{hse06b}
Heyd, J.; Scuseria, G.E.; Ernzerhof, M.
Hybrid Functionals Based on a Screened Coulomb Potential.
{\it J.~Chem.~Phys.} {\bf 2006}, {\it 124}, 219906.

\bibitem{hse06c}
Vydrov, O.A.; Heyd, J.; Krukau, A.V.; Scuseria, G.E.
Importance of Short-Range versus Long-Range Hartree Fock
Exchange for the Performance of Hybrid Density Functionals.
{\it J.~Chem.~Phys.}, {\bf 2006}, {\it 125}, 074106.

\bibitem{pbe0}
Adamo, C.; Barone, V.  Towards Reliable Density Functional Methods without
Adjustable Parameters: the PBE0 Model.
{\it J.~Chem.~Phys.} {\bf 1999}, {\it 110}, 6158-6170.

\bibitem{brutzel}
Van Brutzel, L.; Vincent-Aublant E.  Grain Boundary Influence on Displacement
Cascades in UO$_2$: A Molecular Dynamics Study.
{\it J.~Nuc.~Mater.} {\bf 2008}, {\it 377}, 522-527.

\bibitem{li2o}
Hayoun, M.; Meyer, M.  Surface Effects on Atomic Diffusion in a Superionic
Conductor: A Molecular dynamics Study of Lithium Oxide.  {\it Sur.~Sci.}
{\bf 2013}, {\it 607}, 118-123.

\bibitem{norskov}
Hummelshoj, J.S.; Luntz, A.C.; Norskov, J.K.
Theoretical Evidence for Low Kinetic Overpotential in Li-O$_2$ Electrochemistry.
{\it J.~Chem.~Phys.} {\bf 2013}, {\it 138}, 034703.

\bibitem{bader}
Henkelman, G.; Arnaldsson, A.; J\'{o}nsson, H.
A Fast and Robust Algorithm for Bader Decomposition of Charge Density.
{\it Comput.~Mater.~Sci.} {\bf 2006}, {\it 36}, 354-360.

\bibitem{fec1}
Leung,~K.; Rempe,~S.B.; Foster,~M.E.; Ma,~Y.; Martinez de la Hoz,~J.M.;
Sai,~N.; Balbuena,~P.B.
Modeling Electrochemical Decomposition of Fluoroethylene Carbonate
on Silicon Anode Surfaces in Lithium Ion Batteries
{\it J. Electrochem. Soc.}, {\bf 2014}, {\it 161}, A213-A221,
and references therein.

\bibitem{fec2}
Okuno, Y.; Ushirogata, K.; Sodeyama, K.; Tateyama, Y. 
Decomposition of the Fluoroethylene Carbonate Additive and the Glue Effect
of Lithium Fluoride Products for the Solid Electrolyte Interphase: an Ab
Initio Study.  {\it Phys.~Chem.~Chem.~Phys.} {\bf 2016}, {\it 18}, 8643-8653.

\bibitem{fec3}
Jung, R.; Metzger, M.; Haering, D.; Solchenbach S.; Marino, C.;
Tsiouvaras, N.; Stinner, C.; Gasteiger, H.A.
Consumption of Fluoroethylene Carbonate (FEC) on Si-C
Composite Electrodes for Li-Ion Batteries.
{\it J.~Electrochem.~Soc.} {\bf 2016}, {\it 163}, A1705-A1716.

\bibitem{pressure}
Mikhaylik, Y.; Kovalev, I.; Schock, R.; Kumaresan, K.; Xu, J.; Affinito, J.
High Energy Rechargeabe Li-S Cells for EV Application.  Status, Remaining
Problems and Solutions.  {\it ECS Trans.} {\bf 2010}, {\it 25}, 23-34.

\end{references}

\end{document}
