\section{Task}

% In this section, we first formally formulate the pinyin input method as a text generation task.
% Then we introduce how the pinyin input method is evaluated.
% In this section, we describe the task definition and the evaluation metric.

% \paragraph{Definition} 
The input of pinyin input method includes a sequence of Chinese characters $C=\{w_1,\dots,w_n\}$ as the context and a sequence of pinyin $P=\{p_{n+1},\dots,p_{n+k}\}$, where $w_i\in\mathcal{V}_w$, $p_{n+j}\in\mathcal{V}_p$, and $\mathcal{V}_w$ and $\mathcal{V}_p$ are the vocabularies of words and pinyin, respectively.
% of pinyin syllables. and is the vocabulary of 
The output is a sequence of Chinese characters $O=\{w_{n+1},\dots,w_{n+k}\}$, where  $w_{n+i}\in\mathcal{V}_w$.
The number of output characters is the same as the number of pinyin (i.e., $k$) and each character should be pronounced with the corresponding pinyin. The output sequence is desired to follow the context of Chinese characters to form a coherent sentence. 
% whose sequence length is $n$
% whose sequence length is $k$
% Given a context of characters $C$ and a sequence of pinyin $P$, the task of pinyin input method is to convert the pinyin sequence to a sequence of Chinese characters $O$.
% % The context $C$ can be tokenized as a sequence of tokens denoted as $\{w_1, \dots,w_l\}$, the pinyin sequence is a list of pinyin syllables whose index starts with $l+1$, denoted as $\{p_{l+1}, p_{l+2},\dots,p_{l+k}\}$.
% %  $\{w_{l+1}, w_{l+2},\dots,w_{l+k}\}$, subjecting to the constraint that, for each time step $t>l$, $w_t\in\mathcal{W}(p_t)$.
% We denote the word context as $C=\{w_1,\dots,w_l\}$, where $w_i\in\mathcal{V}_w$ and $\mathcal{V}_w$ is the vocabulary of 
% % used for the pinyin input method, which includes 
% Chinese characters.
% % and other symbols~(digits, alphabets, punctuations and so on).
% We denote the pinyin sequence as $P=\{p_{l+1},\dots,p_{l+k}\}$, where $p_{l+j}\in\mathcal{V}_p$ and $\mathcal{V}_p$ is the vocabulary of pinyin syllables.
% % and all the pinyin initials are also included in the vocabulary.
% The target is denoted as $O=\{w_{l+1},\dots,w_{l+k}\}$, where each token is a Chinese character $w_{l+i}\in\mathcal{V}_w$ and the number of target characters is same as the number of pinyin syllables in the input sequence. The target sequence is desired to follow the context of characters to form a fluent sentence. 
% $\mathcal{V}_w$ is the vocabulary used for the pinyin input method, which includes Chinese characters and other symbols~(digits, alphabets, punctuations and so on). 
% $\mathcal{V}_p$ is the pinyin vocabulary of all the pinyin syllables and all the pinyin initials are also included in the vocabulary.
% We denote the set of legitimate candidate characters sharing the same pinyin $p$ as $\mathcal{W}(p)$.
% The context of words can be 
As mentioned earlier in the introduction section, the input pinyin might be perfect (e.g., ``\texttt{wo men}'') or abbreviated (e.g., ``\texttt{w m}''). Examples of the task are given in Table~\ref{tab:task-definition-example}.\footnote{People may also input pinyin like ``\texttt{l b y you dian shi}'', we leave this as a future work.}
In our definition, one situation is that the context of characters is empty, which corresponds to the scenario that people are entering pinyin at the beginning of a sentence.
% This is the most studied scenario in literature \cite{jia-zhao-2014-joint,zhang-etal-arxiv-tracing,zhang-etal-2019-open} and we make comparison on an existing dataset.
The other situation is that the context includes real words, which stands for the scenario that people are entering pinyin in the middle of a written sentence. 
% We create a dataset for this scenario.

In this paper, we assume that the oracle pinyin segmentation results are provided.
% (i.e., the transformation from a pinyin sequence to a list of pinyin syllables).
Sometimes, a raw pinyin sequence can be mapped to different segmentation results.
For example, the raw pinyin input ``\texttt{jianshi}'' can be segmented as ``\texttt{ji an shi}'' (``集安市'', a city in the southwestern part of Jilin province, China) or ``\texttt{jian shi}'' (``见识'', which is translated as ``experience'' in English). 
% pinyin input has no errors and is already segmented into syllables. 
Pinyin segmentation is a subtask \cite{zhao-etal-2006-improved,p2c-seg} of pinyin input method, which is well solved with the accuracy of 98\% \cite{zhang-etal-arxiv-tracing}.
We leave the integration of pinyin segmentation as future work.

% \mhcomment{add the paper that pinyin segmentation is not hard, ``The segmentation of pinyin syllables already has very high accuracy (over 98\%) by using a trigram language model for pinyin syllables with improved Kneser-Ney smoothing (Kneser and Ney, 1995)."~\cite{zhang-etal-arxiv-tracing} }

% means that the first pinyin syllable is the first character of a sentence.
% Existing works typically study in this setting, 
% This setting has been addressed in existing works~\cite{jia-zhao-2014-joint,zhang-etal-arxiv-tracing,zhang-etal-2019-open}.

% refer readers interested in pinyin segmentation to these works .

% the definition is more general which does not exclude the case that a piece of context is given as its prefix.
% In this work, we conduct experiments on a traditional benchmark to do comparison with them.
% In line with them, we also require the target sequence consists of continuous Chinese characters.

