\documentclass[letterpaper, 10 pt, conference]{ieeeconf}  % Comment this line out if you need a4paper

%\documentclass[a4paper, 10pt, conference]{ieeeconf}% Use this line for a4 paper

\IEEEoverridecommandlockouts% This command is only needed if 
% you want to use the \thanks command

\overrideIEEEmargins  % Needed to meet printer requirements.




%In case you encounter the following error:
%Error 1010 The PDF file may be corrupt (unable to open PDF file) OR
%Error 1000 An error occurred while parsing a contents stream. Unable to analyze the PDF file.
%This is a known problem with pdfLaTeX conversion filter. The file cannot be opened with acrobat reader
%Please use one of the alternatives below to circumvent this error by uncommenting one or the other
%\pdfobjcompresslevel=0
%\pdfminorversion=4

% See the \addtolength command later in the file to balance the column lengths
% on the last page of the document

% numbers option provides compact numerical references in the text. 
% \usepackage{natbib}
\usepackage{times}
\usepackage{multicol}
\usepackage[bookmarks=true]{hyperref}
\usepackage{xcolor}
\usepackage{hyperref}
\usepackage{amssymb}
\usepackage{amsmath, amsfonts}
\usepackage{graphicx}
\usepackage{siunitx}
\usepackage{standalone}
\usepackage{booktabs}
%\usepackage{algorithm,algorithmicx,algpseudocode}
\usepackage[ruled,vlined]{algorithm2e}
\usepackage{mdframed}
\usepackage{fancyvrb}
\usepackage{soul}
\usepackage{dsfont,mathabx}
\usepackage[font=footnotesize, skip=5pt]{caption}

\newtheorem{theorem}{Theorem}
\newtheorem{problem}{Problem}

% GENERAL
\newcommand{\todo}[1]{\textcolor{red}{TODO: #1}}
\newcommand{\improve}[1]{\textcolor{cyan}{#1}}
\newcommand{\note}[1]{\textcolor{blue}{NOTE: #1}}
\newcommand{\question}[1]{\textcolor{orange}{Q: #1}}


%%%%%%%%%%%%%%%%%%%%%%%%%%%%%%%%%%%%%%%%%%%%%%%%%%%%%%%
%%%%


\title{\LARGE \bf
Interactive Human-in-the-loop Coordination of Manipulation Skills\\ Learned from Demonstration
}


\author{Meng Guo$^{1}$ and Mathias B\"urger$^{2}$% <-this % stops a space
\thanks{$^{1}$College of Engineering, Peking University, China. $^{2}$Bosch Center for Artificial Intelligence (BCAI), Germany. Corresponding author: Meng Guo. \texttt{meng.guo@pku.edu.cn}.}}% <-this % stops a space


\begin{document}
  
\maketitle



%%========================================
\maketitle

%%========================================
\begin{abstract}
\label{sec:abstract}

%% 1. what is the problem 
Scientific applications that run on leadership computing facilities often face the challenge 
of being unable to fit leading science cases onto accelerator devices due to memory constraints 
(memory-bound applications).
%
% 2. what is your solution 
In this work, the authors studied one such US Department of Energy mission-critical condensed matter 
physics application, Dynamical Cluster Approximation (DCA++), and this paper discusses how device memory-bound challenges were successfully reduced  by proposing an effective 
``all-to-all'' communication method---a ring communication algorithm. 
%
This implementation takes advantage of acceleration on GPUs and remote direct memory access (RDMA) for fast data exchange between GPUs. 
%
\\Additionally, the ring algorithm was optimized with sub-ring communicators
and multi-threaded support to further reduce communication overhead and 
expose more concurrency, respectively.
%
% 3. What's the cherry-picked evaluation result you want to mention
The computation and communication were also analyzed 
by using the Autonomic Performance Environment for Exascale 
(APEX) profiling tool,  and this paper further discusses the 
performance trade-off for the ring algorithm implementation. 
%
The memory analysis on the ring algorithm shows that the allocation size for the authors' most 
memory-intensive data structure per GPU is now reduced to $1/p$ of the original size, where $p$ is the number of GPUs in the ring communicator.
%
The communication analysis suggests that 
the distributed Quantum Monte Carlo execution time grows linearly as sub-ring size increases, and the cost of messages passing through the network interface connector could be a limiting factor.


%
% \todoRed{Ronnie: Next sentence needs rewrite, too much information about Green's function that no one knows in the abstract; recommend generalizing.} \emph {However, DCA++ is currently facing memory-bound challenge as 
% a larger device array $G_t$ is limited by device memory size, where
% $G_t$ is a two-particle Green's function that allows condensed matter
% scientists to explore larger and more complex (higher fidelity)
% physics cases.}

\end{abstract}

\keywords{DCA++, Quantum Monte Carlo, GPU Remote Direct Memory Access, memory-bound issue, exascale machines}

%%========================================


%%========================================
\section{Introduction}  \label{sec:introduction}

\newcommand\inexpIntro[3]{#1?(#2,#3).}
\newcommand\rinexpIntro[3]{*#1?(#2,#3).}
\newcommand\outexpIntro[3]{#1!(#2,#3).}
\newcommand\outatomIntro[3]{#1!(#2,#3)}

We propose a fully automated method for proving termination of \(\pi\)-calculus processes.
Although there have been a lot of studies on termination analysis for the \(\pi\)-calculus
and related calculi~\cite{Deng06IC,Demangeon07,SangiorgiTermination,KobayashiHybrid,Yoshida04IC,DBLP:journals/jlp/DemangeonHS10,Venet98SAS}, most of them have been rather theoretical,
and there have been surprisingly little efforts in developing  fully automated termination
verification methods and tools based on them. To our knowledge,
Kobayashi's \typical{}~\cite{TyPiCal,KobayashiHybrid} is the only exception that
can prove termination of \(\pi\)-calculus processes (extended with natural numbers)
fully automatically, but its termination analysis is quite limited (see Section~\ref{sec:relatedwork}).

Our method is based on a reduction to termination analysis for sequential programs:
we translate a \(\pi\)-calculus process \(P\) to a sequential program \(S_P\), so that
if \(S_P\) is terminating, so is \(P\). The reduction allows us to use
powerful, mature methods and tools
for termination analysis of sequential programs~\cite{heizmann2016ultimate,freqterm,DBLP:conf/lics/PodelskiR04,Kuwahara2014Termination,DBLP:journals/cacm/CookPR11}.

The idea of the translation is to convert a chain of communications on replicated input
channels to a chain of recursive function calls of the target sequential program.
Let us consider the following Fibonacci process:
\begin{align*}
    & \rinexpIntro{\fib}{n}{r}
        \ifexp{n<2}{ \soutatom{r}{1} \\ &\quad}
                   { \nuexp{s_1} \nuexp{s_2} (\outatomIntro{\fib}{n-1}{s_1} \PAR \outatomIntro{\fib}{n-2}{s_2} \PAR \sinexp{s_1}{x}\sinexp{s_2}{y}\soutatom{r}{x+y}) \\}
    & \PAR \outatomIntro{\fib}{m}{r}
\end{align*}
Here, the process
$\rinexpIntro{\fib}{n}{r} \ldots$ is a function server that computes the \(n\)-th Fibonacci number
in parallel and returns the result to \(r\),
and $\outatom{\fib}{m}{r}$ sends a request for computing the \(m\)-th Fibonacci number;
those who are not familiar with the syntax of the \(\pi\)-calculus may wish to consult
Section~\ref{sec:targetlanguage} first.
To prove that the process above is terminating for any integer \(m\),
it suffices to show that there is no infinite chain of communications on $\fib$:
\[
    \fib(m,r) \to \fib(m_1,r_1) \to \fib(m_2,r_2) \to \cdots.
\]
We convert the process above to the following program:\footnote{The actual translation
  given later is a little more complex.}
\begin{verbatim}
 let rec fib(n) = if n<2 then () else (fib(n-1) [] fib(n-2)) in
 fib(m)
\end{verbatim}
Here, \texttt{[]} represents the non-deterministic choice.
Note that, although the calculation of Fibonacci numbers is not preserved,
for each chain of communications on \texttt{fib}, there is a corresponding
sequence of recursive calls:
\[
\mathtt{fib}(m) \to \mathtt{fib}(m_1) \to \mathtt{fib}(m_2) \to \cdots.
\]
Thus, the termination of the sequential program above implies the termination of
the original process.
As shown in the example above, (i) each communication on a replicated input channel
is converted to a function call, (ii) each communication on a non-replicated input
channel is just removed (or, in the actual translation, replaced by a call of
a trivial function defined by \(f(\seq{x})=(\,)\)), and (iii) parallel composition
is replaced by a non-deterministic choice.
We formalize the translation outlined above and prove its correctness.

The basic translation sketched above sometimes loses too much information.
For example, consider the following process:
\begin{align*}
    & \rinexpIntro{\pre}{n}{r} \soutatom{r}{n-1} \\
    & \PAR \rinexpIntro{f}{n}{r} \ifexp{n<0}{ \soutatom{r}{1} }
                                       { \nuexp{s} (\outatomIntro{\pre}{n}{s} \PAR \sinexp{s}{x}\outatomIntro{f}{x}{r}) } \\
    & \PAR \outatomIntro{f}{m}{r}
\end{align*}
The translation sketched above would yield:
\begin{verbatim}
  let pred(n) = n-1 in
  let rec f(n) = if n<0 then () else (pred(n) [] f(*)) in
  f(m)
\end{verbatim}
Here, \texttt{*} represents a non-deterministic integer: since we have removed
the input $\sinatom{s}{x}$, we do not have information about the value of \( x \).
As a result, the sequential program above is non-terminating, although the original
process is terminating.
To remedy this problem, we also refine the basic translation above by using a refinement
type system for the \(\pi\)-calculus. Using the refinement type system,
we can infer that the value of \(x\) in the original process is less than \(n\),
so that we can refine the definition of \texttt{f} to:
\begin{verbatim}
 let rec f(n) = ... else (pred(n) [] let x=* in assume(x<n);f(x))
\end{verbatim}
The target program is now terminating, from which
we can deduce that the original process is also terminating.
We have implemented an automated tool based on the refined translation above.

The contributions of this paper are summarized as follows.
\begin{itemize}
\item The formalization of the basic translation from the \(\pi\)-calculus
  (extended with integers) to sequential programs, and a proof of its correctness.
\item The formalization of a refined translation based on a refinement type system.
\item An implementation of the refined translation, including automated refinement type
  inference based on CHC solving, and experiments to evaluate the effectiveness of
  our method.
\end{itemize}

The rest of this paper is structured as follows.
Section~\ref{sec:targetlanguage} introduces the source and target languages
of our translation.
Section~\ref{sec:approach} 
formalizes the basic translation, and proves its correctness.
Section~\ref{sec:refinement} refines the basic translation by using a refinement type system.
Section~\ref{sec:implementation} reports an implementation and experiments.
Section~\ref{sec:relatedwork} discusses related work,
and Section~\ref{sec:conclusion} concludes the paper.

%%========================================
\textbf{Related work}:
% Object detection related datasets/algo in non-medical domain
% Locally labeled CXR dataset
A few CXR datasets have localized abnormality annotations \cite{shih2019augmenting,filice2020crowdsourcing,jaeger2014two} that are curated manually. These are high quality gold standard ground truth datasets but tend to be smaller in scale (< 30,000 images) and have a narrow coverage, with typically only 1-2 labels. In addition, since most labeling efforts only have abnormality semantics attached, no direct relationships with the affected anatomical locations are available. 

%MEHDI: repeated concepts from above. I am removing the following: 

%The lack of anatomic semantics in the annotation is a limitation for complex multi-modal clinical reasoning work, e.g., differential diagnosis, since clinicians often integrate information along anatomical lines, and for downstream report generation tasks, which often requires describing not only the abnormality but also correctly communicate the location of the abnormalities (and medical devices) to the receiving clinicians. 

Two recent CXR datasets have labels for anatomies described in the reports. In \cite{datta2020dataset}, a small manually annotated dataset (2000 reports) included 10 abnormalities that are individually associated with 29 unique spatial locations (anatomies) at the report level. Another CXR dataset has automatically extracted abnormality and anatomy labels as disconnected concepts that are only correlated at the study level from  160,000 reports using a supervised NLP algorithm \cite{bustos2020padchest}. This was trained on a smaller set of manually annotated data. Neither datasets contain localized annotations for the associated CXR images, nor any comparison relation annotations between sequential exams, both of which are available in the Chest ImaGenome dataset. In Table \ref{tab:related}, we present a comparison of our Chest ImagGenome dataset with other datasets available in the literature.

% Table -- Kashyap

% MEdical imaging datasets to go here: Discussed that we will only focus on cxr datasets that are available for this paper. 
% \caption{\color{red} Kashyap, feel free to continue with the table. We should remove the questionmarks and add a line for our dataset (since all others are not graph). For longer text, using abbreviations and explaining them in the caption often works better. If fill in the values is not possible, it is better to remove the table altogether.}


\begin{table}[t!]
\caption{Summary of existing chest X-ray datasets}
\resizebox{\textwidth}{!}{%
\begin{tabular}{@{}lllllllll@{}}
\toprule
\textbf{Dataset} & \textbf{Annotation Level} & \textbf{Annotation Method} & \textbf{Num Labels} & \textbf{Anatomy Labeled} & \textbf{Graph} & \textbf{Dataset Size} & \textbf{Temporal Labels} & \textbf{Reports} \\ \midrule
SIIM-ACR Pneumothorax Segmentation \cite{filice2020crowdsourcing} & Segmentation & Manual + augmented & 1 & No & No & 12,047 & No & No \\
RSNA Pneumonia Detection Challenge   \cite{shih2019augmenting} & Bounding Boxes & Manual & 1 & No & No & 30,000 & No & No \\
Indiana University Chest X-ray collection \cite{demner2016preparing} & Global & Automated & 10 & No & No & 3,813 & No & Yes \\
NIH CXR dataset \cite{wang2017chestx} & Global & Automated & 14 & No & No & 112,120 & No & No \\
PLCO \cite{team2000prostate} & Global & Automated & 24 & Yes & No & 236,000 & Yes & No \\
Stanford CheXpert \cite{irvin2019chexpert} & Global & Automated & 14 & No & No & 224,316 & No & No \\
MIMIC-CXR \cite{johnson2019mimic} & Global & Automated & 14 & No & No & 377,110 & No & Yes \\
Dutta \cite{datta2020dataset} & Global & Manual & 10 & Yes & Yes & 2,000 & No & Yes \\
PadChest \cite{bustos2020padchest} & Global & Manual + automated & 297 & Yes & No & 160,868 & No & Yes \\
Montgomery County Chest X-ray   \cite{jaeger2014two} & Segmentation & Manual & 1 & Yes & No & 138 & No & No \\
Shenzen Hospital Chest X-ray   \cite{jaeger2014two} & Segmentation & Manual & 1 & Yes & No & 662 & No & No \\  \hline \hline
\textbf{Chest ImaGenome} & Bounding Boxes & Automated & 131 & Yes & Yes & 242,072 & Yes & Yes \\
\bottomrule
\end{tabular}%
}
\label{tab:related}
\vspace{-0.4cm}
\end{table}
% removed (Derived from MIMIC-CXR \cite{johnson2019mimic}) % makes table really small

%%========================================
\section{Preliminaries}

\subsection{Notation}

Let $\mX \subset \R^{I_1 \times \cdots \times I_K}$ be the space of
order-$K$ tensors, where $I_k$ denotes the dimensionality of the $k$-th
mode for $k=1,\dots,K$.  For brevity, we define
$I_{<k} := \prod_{k'<k}I_{k'}$; similarly, $I_{\leq k}, I_{k<}$ and
$I_{k \leq}$ are defined.  For a vector $Y \in \R^d$, $[Y]_i$ denotes
the $i$-th element of $Y$.  Similarly, $[X]_{i_1,\ldots,i_K}$ denotes
the $(i_1,\ldots,i_K)$ elements of a tensor $X\in\mX$. Let
$[X]_{i_1,\ldots,i_{k-1},:,i_{k+1},\ldots,i_K}$ denote an
$I_k$-dimensional vector
$(X_{i_1,\ldots,i_{k-1},j,i_{k+1},\ldots,i_K})_{j=1}^{I_k}$ called the
mode-$k$ fiber.  For a vector $Y \in \R^d$, $\|Y\| = (Y^T Y)^{1/2}$
denotes the $\ell_2$-norm and $\|Y\|_{\infty} = \max_i|[Y]_i|$ denotes
the max norm.  For tensors $X,X' \in \mX$, an inner product is defined
as
$\langle X,X' \rangle := \sum_{i_1,\ldots,i_K =1}^{I_1 \dots I_K}
X(i_1,\ldots,i_K)X'(i_1,\ldots,i_K)$
and $\|X\|_{F} = \langle X,X \rangle^{1/2}$ denotes the Frobenius
norm.  For a matrix $Z$, $\|Z\|_s := \sum_{j} \sigma_{j}(Z)$ denotes
the Schatten-1 norm, where $\sigma_j(\cdot)$ is a $j$-th singular value
of $Z$.

\subsection{Tensor Train Decomposition}

%\textit{Tensor train (TT) decomposition} is a tensor factorization
%method with a matrix product representation
%\cite{oseledets2010tt,oseledets2011tensor}.  
Let us define a tuple of positive integers $(R_1, \ldots, R_{K-1})$
and an order-$3$ tensor $G_k \in \R^{I_k \times R_{k-1} \times R_k}$
for each $k = 1,\ldots,K$.  Here, we set $R_0 = R_K = 1$.  Then, TT
decomposition represents each element of $X$ as follows:
\begin{align}
	X_{i_1,\ldots,i_K} = [G_1]_{i_1,:,:} [G_2]_{i_2,:,:} \cdots [G_K]_{i_K,:,:}. \label{eq:tt}
\end{align}
Note that $[G_k]_{i_k,:,:}$ is an $R_{k-1} \times R_k$ matrix.  We
define $\mG := \{G_k\}_{k=1}^K$ as a set of the tensors, and let $X(\mG)$
be a tensor whose elements are represented by $\mG$ as
\eqref{eq:tt}.  The tuple $(R_1, \ldots, R_{K-1})$ controls
the complexity of TT decomposition, and it is called a \textit{Tensor
  Train (TT) rank}.  Note that TT decomposition is universal, i.e.,
any tensor can be represented by TT decomposition with sufficiently
large TT rank~\cite{oseledets2010tt}.


When we evaluate the computational complexity, we assume the shape of
$\mG$ is roughly symmetric. That is, we assume there exist
$I,R\in\mathbb{N}$ such that $I_k=O(I)$ for $k=1,\dots,K$ and
$R_k=O(R)$ for $k=1,\dots,K-1$.


\subsection{Tensor Completion Problem}

Suppose there exists a true tensor $X^* \in \mX$ that is unknown, and
a part of the elements of $X^*$ is observed with some noise.  Let
$S \subset \{(j_1,j_2,
\ldots,j_K)\}_{j_1,\ldots,j_K=1}^{I_1,\ldots,I_K}$
be a set of indexes of the observed elements and
$n := |S| \leq \prod_{k=1}^K I_k$ be the number of observations.  Let
$j(i)$ be an $i$-th element of $S$ for $i=1,\ldots,n$, and $y_i$
denote $i$-th observation from $X^*$ with noise.  We consider the
following observation model:
\begin{align}
	y_i = [X^*]_{j(i)} + \epsilon_i, \label{model:obs}
\end{align}
where $\epsilon_i$ is i.i.d. noise with zero mean and variance
$\sigma^2$.  For simplicity, we introduce  observation vector
$Y := (y_1, \ldots, y_n)$, noise vector
$\mE := (\epsilon_1, \ldots , \epsilon_n)$, and rearranging operator
$\mathfrak{X} : \mX \to \mathbb{R}^n$ that randomly picks the elements of $X$.
%  $[\mathfrak{X}(X)]_i = [X]_{j(i)}$.
Then, the model \eqref{model:obs} is rewritten as follows:
\begin{align*}
	Y = \mathfrak{X}(X^*) + \mE.
\end{align*}

%%%
The goal of tensor completion is to estimate the true tensor $X^*$
from the observation vector $Y$.  Because the estimation problem is
ill-posed, we need to restrict the degree of freedom of $X^*$, such as
rank. Because the direct optimization of rank is difficult, its convex
surrogation is alternatively
used~\cite{candes2012exact,candes2010matrix, krishnamurthy2013low,
  zhang2016exact, phien2016efficient}.  For tensor
completion, the convex surrogation yields the following optimization
problem
\cite{gandy2011tensor,liu2013tensor,signoretto2011tensor,tomioka2010estimation}:
\begin{align}
	\min_{X \in \Theta} \left[ \frac{1}{2n} \|Y - \mathfrak{X}(X)\|^2 + \lambda_n \|X\|_{s^*} \right], \label{opt:general}
\end{align}
where $\Theta \subset \mX$ is a convex subset of $\mX$, 
%and
%$\Omega : \Theta \to \R_+$ is a regularization for tensors, 
$\lambda_n\geq 0$ is a regularization coefficient, and
$ \|\cdot\|_{s^*}$ is the overlapped Schatten norm defined as
$ \|X\|_{s^*} := \frac{1}{K} \sum_{k=1}^K \|\tilde{X}_{(k)}\|_s$.
Here, $\tilde{X}_{(k)}$ is the $k$-unfolding matrix defined by
concatenating the mode-$k$ fibers of $X$.  The overlapped Schatten
norm regularizes the rank of $X$ in terms of Tucker
decomposition~\cite{negahban2011estimation, tomioka2011statistical}.
Although the Tucker rank of $X^*$ is unknown in general, the convex
optimization adjusts the rank depending on $\lambda_n$.

To solve the convex problem~\eqref{opt:general}, the ADMM algorithm is often
employed~\cite{boyd2011distributed,tomioka2010estimation,
  tomioka2011statistical}.  Since the overlapped Schatten norm is not
differentiable, the ADMM algorithm avoids the differentiation of the
regularization term by alternatively minimizing the augmented
Lagrangian function iteratively.


%%% Local Variables:
%%% mode: latex
%%% TeX-master: "TTcomp_NIPS2017.tex"
%%% End:

%%========================================
%!TEX root = main.tex
\section{Problem Definition and Notations}
\label{sec:problem}







% In this section, we will first describe key concepts and notations used in this paper, and formally define our problem. Then we will use a case study to make our idea of story tree more concrete.

% \subsection{Problem Definition and Notations}
% \label{subsec:problem-define}

We first present some definitions of key concepts in the top-down hierarchy: \textit{topic} $\rightarrow$ \textit{story} $\rightarrow$ \textit{event} to be used in this paper.

\begin{definition}
  \textit{Event}: an event $\mathcal{E}$ is a set of one or several documents that contain highly similar information.
\end{definition}

\begin{definition}
  \textit{Story}: a story $\mathcal{S}$ is a tree of events that revolve around a group of specific persons and happen at certain places during specific times. A directed edge from event $\mathcal{E}_1$ to $\mathcal{E}_2$ indicates a temporal evolution or a logical connection from $\mathcal{E}_1$ to $\mathcal{E}_2$.
\end{definition}

\begin{definition}
  \textit{Topic}: a topic consists of a set of stories that are highly correlated or similar to each other.
  \vspace{-1mm}
\end{definition}


Each topic may contain multiple story trees, and each story tree consists of multiple logically connected events.
In our work, events (instead of news documents) are the smallest atomic units. Each event is also assumed to belong to a single story and contains partial information about that story.
For instance, considering the topic \textit{American presidential election}, \textit{2016 U.S. presidential election} is a story within this topic, and  \textit{Trump and Hilary's first television debate} is an event within this story.


We now introduce some notations and describe our problem formally. Given a news document stream $D = \{ \mathcal{D}_1, \mathcal{D}_2, \ldots, \mathcal{D}_t,\ldots \}$, where $\mathcal{D}_t$ is the set of news documents collected on time period $t$, our objective is to: a) cluster all news documents $D$ into a set of events $E = \{ \mathcal{E}_1, \ldots, \mathcal{E}_{|E|} \}$, and b) connect the extracted events to form a set of stories $S = \{ \mathcal{S}_1, ..., \mathcal{S}_{|S|} \}$. Each story $\mathcal{S} = (E, L)$ contains a set of events $E$ and a set of links $L$, where $L_{i,j} := <\mathcal{E}_i, \mathcal{E}_j>$ denotes a directed link from event $\mathcal{E}_i$ to $\mathcal{E}_j$, which indicates a temporal evolution or logical connection relationship.

%We now illustrate our problem with an example. (A example Fig) Fig... shows ...
Furthermore, we require the events and story trees to be extracted in an online or incremental manner. That is, we extract events from each $\mathcal D_t$ individually when the news corpus $\mathcal D_t$ arrives in time period $t$, and \emph{merge} the discovered events into the existing story trees that were found at time $t-1$. This is a unique strength of our scheme as compared to prior work, since we do not need to repeatedly process older documents and can deliver  a set of evolving yet logically consistent story trees to users.  

% \subsection{Case Study}
% \label{subsec:case-study}

\begin{figure}
\includegraphics[width=3.4in]{figure/StoryStructures}
\caption{Different structures to characterize a story.}
\vspace{-2mm}
\label{fig:storyStructures}
\vspace{-2mm}
\end{figure}

For example, Fig.~\ref{fig:CaseStudy} illustrates the story tree of ``2016 U.S. presidential election''. The story contains $20$ nodes, where each node indicates an event in 2016 U.S. election, and each link indicates a temporal evolution or a logical connection between two events. %For example, event $19$ says America votes to elect new president, and event $20$ says Donald Trump is elected president. 
The index number on each node represents the event sequence over the timeline. There are $6$ paths within this story tree, where the path $1 \rightarrow 20$ indicates the whole presidential election process, branch $3 \rightarrow 6$ is about Hilary's health conditions, branch $7 \rightarrow 13$ talks about television debates, $14 \rightarrow 18$ depicts the investigation into Hilary's ``mail door'', etc. As we can see, by modeling the evolutionary and logical structure of a story into a story tree, users can easily grasp the logic of news stories and learn the main information quickly. 


Let us represent each story by an empty root node $s$ from which the story is originated, and denote each event by an event node $e$. The events in a story can be organized in one of the following four structures shown in Fig. \ref{fig:storyStructures}: a) a flat structure that does not include dependencies between events; b) a timeline structure that organizes events by their timestamps; c) a graph structure that checks the connection between all pairs of events and maintains a subset of most strong connections; d) a tree structure, which represents a story's evolving structure by a tree.  

Compared with a tree structure, sorting events by timestamps omits the logical connection between events, while using directed acyclic graphs to model event dependencies without considering the evolving consistency of the whole story can leads to unnecessary connections between events.
Through extensive user experience studies in Sec.~\ref{sec:eval}, we show that tree structures are the most effective way to represent breaking news stories as compared to other structures, including the more complex graph structures. 

%%========================================
%%========================================
\begin{figure}[t!]
    \centering
    \includegraphics[width=0.6\linewidth]{figures/single_demo.png}
    %--------------------
    \caption{Illustration of the generalization problem in Sec.~\ref{subsec:limit-few}. 
Given the single demonstration (in black), the retrieved trajectory (in blue) is significantly displaced w.r.t. the excepted trajectory (in green).}
    \label{fig:single_demo}
    %--------------------
    \vspace{-0.15cm}
\end{figure}
%========================================
%==============================
\section{Current Limitations}\label{sec:limitation}
In this section, we discuss three limitations of the current skill coordination methods that are built upon the TP-HSMM model in Sec.~\ref{subsec:tp-hsmm}. 
These limitations serve as the main motivation for the proposed solution in the subsequent section. 


%==============================
\subsection{Generalization from Very Few Demos}\label{subsec:limit-few}
As described in Sec.~\ref{subsec:tp-hsmm}, the TP-HSMM of a skill is computed as a parameterized average of \emph{all} demos. 
However, when only one or very few demos are provided (compared with the number of frames), 
the resulting model can have over-confident GMMs at different stages of the skill. 
One direct consequence is that the generalization to new scenarios is heavily biased towards this one demo.
As shown in Fig.~\ref{fig:single_demo} for a picking skill, the retrieved trajectory ends up between the new object pose and the demonstrated pose.
Even though this can be mitigated by adding new demos, 
it is still desirable to learn a reasonable model with very few demos,
especially for simple skills such as picking and dropping within free spaces.

%========================================
\begin{figure}[t!]
    \centering
    \includegraphics[width=0.45\linewidth]{figures/branch_gaussian_2p_box.png}
    \includegraphics[width=0.45\linewidth]{figures/branch_svm_2p_box.png}
    %--------------------
    \caption{
Comparison between branch selection based on the Gaussian preconditions (left) 
and the proposed selector (right), 
    for the bin-picking skill with 5 branches (four sides and the center). 
    Note that different colors indicate the predicted branches at that sample point (in dots),
 while the \emph{projected} training data are indicated as diamonds (left).}
    \label{fig:branch_compare}
    %--------------------
    \vspace{-0.15cm}
\end{figure}
%========================================

%==============================
\subsection{Multiple Branches for One Skill}\label{subsec:limit-branch}
Often there are multiple ways of executing the same skill under different scenarios (called \emph{branches}).
For instance, as shown in the experiment, there are five different ways of picking objects from a bin, i.e., approaching with different angles depending on the distances to each boundary.  
Our earlier work~\cite{rozo2020learning} proposed a learning algorithm for TP-HSMM with multiple branches, 
and moreover a precondition model that chooses the best branch based on the first GMMs of all branches.
However, this model requires a large number of demos to cover the area of interest 
and does not generalizes well to new scenarios.
This is mainly due to the usage of Gaussian clustering over few samples in high dimensions. 
Fig.~\ref{fig:branch_compare} illustrates one example of the bin-picking skill described earlier with 5 branches.
It can be seen that the precondition model from~\cite{rozo2020learning} can yield bad choices even close to the demonstrated scenes. 

%==============================
\subsection{Manual Conditioning in Task Graph}\label{subsec:manual-specification}
Lastly, as mentioned in Sec.~\ref{sec:introduction}, complex manipulation tasks often contain various sequences of skills to account for different scenarios. 
A high-level abstraction of such relations is often referred as task networks~\cite{hayes2016autonomously}.
In such networks, a valid plan evolves by transition from one skill to another until the task is solved. 
As shown in Fig.~\ref{fig:framework}, different sequences can share some common skills and one skill may appear several times in the same sequence. 
Even though the graph structure that encapsulates these sequences can be sketched easily, 
the \emph{conditions} on these transitions are particularly difficult and tedious to specify manually. 
Often it is required to modify these conditions whenever the workspace or the goal specification is changed. 

%%========================================
%!TEX root = main.tex

\parindent 20pt

\section{Appendix---Geometric Chung-Lu Directed Graph Model is Well-defined} \label{apdx:model}

Let us state the problem in a slightly more general setting. Fix $n=2m$, where $m$ is some natural number. Suppose that for each pair $i, j \in [n]$ we have $a_{ij}=a_{ji} \in \R_+$ if $i\leq m$, $j>m$, $j\neq i+m$, and we have $a_{ij}=0$ otherwise, that is, if $i,j\in [m]$ or $i,j \in [2m] \setminus [m]$ or $j=i+m$. The $a_{ij}$ elements taken together form a matrix $\textbf{A}$, which is symmetric. Finally, for each $i \in [n]$ we have $b_i \in \R_+ \cup \{0\}$, and 
\begin{equation}\label{eq:assumption}
\sum_{i=1}^{m}b_i=\sum_{i=m+1}^{n}b_i>0.
\end{equation}

\medskip

In our application, $m=|V|$ is the number of nodes in a non-empty directed graph $G=(V,E)$, and elements of vector $\mathbf{b} = (b_i)_{i\in[n]}$ correspond to the degree distribution of the graph: for $i\in [m]$, $b_i$ is the out-degree of $v_i$ (that is, $b_i=w_i^{out}$), and for $i\in [2m] \setminus [m]$, $b_i$ is the in-degree of $v_{i-m}$ (that is, $b_i=w_{i-m}^{in}$). The assumption that $\sum_{i=1}^{m}b_i=\sum_{i=m+1}^{n}b_i>0$ is satisfied as the total in-degree is equal to the total out-degree, and the graph is not empty. Positive elements of matrix $\textbf{A}$ satisfy $a_{i,j+m}=a_{j+m,i} = g(d_{i,j}) \in (0,1]$ for $i,j\in[m], i\neq j$, and correspond to the distances between embeddings of the corresponding nodes $v_i$ and $v_j$. The case $i=j$ is excluded as indices $i$ and $i+m$ correspond to the same node in the original directed graph and so $a_{i,i+m}=a_{i+m,i} = g(d_{i,i}) = 0$. Finally, since isolated nodes may be ignored, we may assume that $w_i^{out}+w_i^{in} > 0$, that is,
\begin{equation}\label{eq:non-degenerate}
b_i + b_{i+m} > 0 \qquad \text{ for all } i \in [m].
\end{equation}

Our goal is to investigate if there is a solution, $x_i \in \R_+\cup\{0\}$ for $i \in [n]$, of the following system of equations:
\begin{equation}\label{eq:system}
b_i = x_i \sum_{j=1}^na_{ij}x_j \qquad \text{ for all } i\in[n].
\end{equation}
If there is one, then is this solution unique? The solution to~(\ref{eq:system}) will yield the solution to our problem: for $i \in [m]$, $x_i^{out} = x_i$ and $x_i^{in}=x_{i+m}$.

\medskip

The $m=2$ case is a degenerate case that exhibits a different behaviour but it is easy to investigate. In this case, by assumption~(\ref{eq:assumption}), $b_1 = b_4 = x_1 x_4 a_{12}$ and $b_2 = b_3 = x_2 x_3 a_{21}$. There are infinite number of solutions, each of them being of the form $(x_1, x_2, x_3, x_4) = (s, t, b_2/(a_{12}t), b_1/(a_{12}s))$ for some $t, s \in \R_+$. Having said that, in our application, all of these solutions yield the same random graph with the following distribution: $p_{12} = x_1 x_4 a_{12} = b_1 = w^{out}_1$ and $p_{21} = x_2 x_3 a_{21} = b_2 = w^{out}_2$.

\medskip

Suppose now that $m \ge 3$. We will show that the desired solution of~(\ref{eq:system}) exists if
\begin{equation}\label{eq:condition1}
\sum_{i=m+1}^nb_i > b_j + b_{j+m} \quad \text{for} \quad j\in[m],
\end{equation}
and
\begin{equation}\label{eq:condition2}
\sum_{i=1}^mb_i > b_j + b_{j-m} \quad \text{for} \quad j\in[2m] \setminus [m]
\end{equation}
(also recall that by assumption~(\ref{eq:assumption}), $\sum_{i=1}^{m}b_i=\sum_{i=m+1}^{n}b_i>0$, and by assumption~(\ref{eq:non-degenerate}), $b_i+b_{i+m}>0$ for all $i \in [m]$).
In other words, the condition is that the in-degree of any node $v_i$ is smaller than the sum of out-degrees of nodes other than $v_i$, and the out-degree of $v_i$ is smaller than the sum of in-degrees of nodes other than $v_i$. This is a very mild condition that holds in our application. Indeed, properties~(\ref{eq:condition1})--(\ref{eq:condition2}) with non-strict inequalities are satisfied for \emph{all} directed graphs $G$, and strict inequalities are satisfied \emph{unless} $G$ has an independent set of size $n-1$, that is, $G$ is a star with one node being part of \emph{every} edge. 

These degenerate cases can be ignored in further analysis, as such configurations of $\mathbf{b}$ allow to reconstruct the graph deterministically. The implementation of the framework identifies such cases and returns the corresponding 0/1 values of $p_{ij}$. For degenerate cases, the corresponding system of equations~(\ref{eq:system}) might or might not have a solution depending on the parameters. For example when $m=3$ and $\mathbf{b}=(2,0,0,0,1,1)$ the system has infinitely many solutions of the form $(t,0,0,s,1/(a_{51}t), 1/(a_{61}t)))$ for any $t,s>0$, all of them yielding the same deterministic graph: $p_{12}=p_{13}=1$, $p_{21}=p_{23}=p_{31}=p_{32}=0$. However, for $m=3$, $\mathbf{b}=(2,1,1,2,1,1)$ and non zero elements $a_{ij}$ equal to $1$ the system has no solutions. We do not try to classify cases when the system has the solution in the proof as they lead to deterministic graphs and we handle them in the implementation of the framework separately anyway.

\medskip

Let us make the following observations that will be useful later on. Provided that properties~(\ref{eq:condition1})--(\ref{eq:condition2}) are satisfied, the following properties hold:
\begin{itemize}
    \item $x_i=0$ if and only if $b_i=0$. Indeed, if $x_i = 0$, then trivially $b_i =0$. Suppose then that $b_i=0$ and by symmetry we may assume that $i\in[m]$. By properties~(\ref{eq:condition1})--(\ref{eq:condition2}), $b_j > 0$ for at least one value of $j \in [2m]\setminus[m]$ and $j\neq i+m$. It follows that $x_j > 0$, $a_{ij} > 0$, and so $b_i \ge x_i a_{ij} x_j$. As a result, $x_i$ has to be equal to zero in order for $b_i$ to be zero. 
    \item we may assume that $x_1=1$. Indeed, one can reorder the nodes so that $b_1>0$. Then, one can multiply $x_i$ for all $i\in[m]$ by any positive constant $\alpha \in \R_+$ and divide $x_i$ for all $i \in [2m] \setminus [m]$ by $\alpha$ and the solution will not change.
\end{itemize}

The last observation means that we need to introduce the constraint $x_1=1$ if we ever hope to prove the uniqueness of the solution. If we do not do that, then there will be an infinite number of solutions but all of them will yield the same edge distribution for the random graph as the particular solution we are searching for.

\medskip

We will start with proving the uniqueness. After that, we will show that~(\ref{eq:condition1})--(\ref{eq:condition2}) are sufficient conditions.

\subsection{Uniqueness}

Let us assume that $m \ge 3$. For a contradiction, suppose that we have two different solutions: $\mathbf{x} = (x_i)_{i\in[n]}$ ($x_i \in \R_+$, $i \in [n]$) with $x_1=1$ and $\mathbf{y} = (y_i)_{i \in [n]}$ ($y_i \in \R_+$, $i \in [n]$) with $y_1=1$. It follows that for all $i \in [n]$ we have
$$
b_i = f_i(\mathbf{x}) = f_i(\mathbf{y}), \quad \text{ where } f_i(\mathbf{x}) = x_i \sum_{j=1}^n a_{ij} x_j.
$$
Let us analyze what happens at point $\mathbf{z} = t\mathbf{x} + (1-t)\mathbf{y}$ for some $t \in [0,1]$ (that is, $z_i = tx_i+(1-t)y_i$, $i \in [n]$). For each $i \in [n]$ we get
\begin{eqnarray*}
f_i(\mathbf{z}) &=& (tx_i+(1-t)y_i) \sum_{j=1}^n a_{ij} (tx_j+(1-t)y_j) \\
&=& \sum_{j=1}^n a_{ij} \left( t^2 x_i x_j+ t(1-t)(x_i y_j + x_j y_i) + (1-t)^2 y_i y_j \right) \\
&=& f_i(\textbf{x}) t^2 + \frac {t(1-t)}{x_i y_i} (f_i(\textbf{y}) x_i^2 + f_i(\textbf{x}) y_i^2) + f_i(\textbf{y}) (1-t)^2 \\
&=& b_i \left( t^2 + \frac {t(1-t)}{x_i y_i} (x_i^2 + y_i^2) + (1-t)^2 \right) \\
&=& b_i \left( 1 - 2t(1-t) + \frac {t(1-t)}{x_i y_i} (x_i^2 + y_i^2) \right) \\
&=& b_i \left( 1 + \frac {t(1-t)}{x_i y_i} (x_i^2 - 2x_iy_i + y_i^2) \right) \\
&=& b_i \left( 1+t(1-t)\frac{(x_i-y_i)^2}{x_iy_i} \right) =: g_i(t).
\end{eqnarray*}
Note that $g_i'(1/2)=0$ for all $i$ (as either $x_i-y_i$ vanishes and so $g_i(t)$ is a constant function or it does not vanish but then $g_i(t)$ is a parabola with a maximum at $t=1/2$). For convenience, let $\mathbf{v} = (\mathbf{x} + \mathbf{y})/2$ and $\mathbf{s} = (\mathbf{x} - \mathbf{y})/2$ (that is, $v_i=(x_i+y_i)/2$ and $s_i=(x_i-y_i)/2$ for all $i \in [n]$). It follows that 
$$
\frac{d g_i}{dh} \Big( \mathbf{v}+h\mathbf{s} ~|~ h=0 \Big)=0.
$$
On the other hand,
$$
g_i(\mathbf{v}+h\mathbf{s}) = \sum_{j=1}^na_{ij}(v_i+hs_i)(v_j+hs_j)
$$
and so
$$
\frac{d g_i}{dh}(\mathbf{v}+h\mathbf{s}) = \sum_{j=1}^na_{ij}(s_i(v_j+hs_j)+s_j(v_i+hs_i)).
$$
Combining the two observations, we get that
\begin{equation}\label{eq:condition_for_s}
0=\frac{d g_i}{dh} \Big( \mathbf{v}+h\mathbf{s} ~|~ h=0 \Big) = \sum_{j=1}^na_{ij}(s_iv_j+s_jv_i).
\end{equation}

Now, for $i \ge 1$, let $u_i = s_i / v_i$, provided that $v_i \neq 0$.
Recall that in particular $s_1=(x_1-y_1)/2=0$ and $v_1=(x_1+y_1)/2=1$, as we assumed that $x_1=y_1=1$, and so $u_1=0$.
Additionally, if $v_i=0$ (that is, the corresponding node has in-degree 0 or out-degree 0), then we may take $u_i=0$, as it will cancel out anyway. Denote the set of indices $i$ when $v_i=0$ by $Z$. Substituting it to~(\ref{eq:condition_for_s}) we get:
$$
\sum_{j=1}^na_{ij}v_iv_j(u_i+u_j) = 0.
$$

Notice that we can rescale all $u_i$'s by the same multiplicative factor so that $u_{\ell}=1$ for some $\ell \in [n]$ and for all other indices $|u_i|\leq1$ (potentially with negative rescaling factor). For index $\ell$ we have:
$$
\sum_{j=1}^n a_{\ell j} v_{\ell}v_j(u_{\ell}+u_j) = \sum_{j=1}^n a_{\ell j} v_{\ell}v_j(1+u_j) = 0.
$$
But this possible only if $\ell\in[m]$, as otherwise the left hand side of the above equation is at least its first term, namely, $a_{\ell 1}v_{\ell} v_1(1+u_1) =a_{\ell 1}v_{\ell} > 0$. If $\ell \in[m]$, then we get
$$
\sum_{j=m+1}^n a_{\ell j}v_{\ell} v_j (u_{\ell}+u_j) = \sum_{j=m+1}^n a_{\ell j}v_{\ell} v_j (1+u_j) = 0
$$
as $a_{\ell j}=0$ for $j\in[m]$. But this means that $u_j=-1$ for $j\in[n]\setminus([m]\cup Z)$.

Let us concentrate on any index $\ell'\in [n]\setminus([m]\cup Z)$ (note that this set is non-empty).
For this index, we have the following condition:
$$
\sum_{j=1}^m a_{\ell' j}v_{\ell'}v_j(u_{\ell'}+u_j) = \sum_{j=1}^m a_{\ell' j}v_{\ell'}v_j(-1+u_j) = 0.
$$
However, since $|u_j|\leq 1$ for all $j \in[m]$, all entries of the sum are non-negative and so they would all have to be equal to $0$ for the sum to be $0$. This is not possible as $u_1=0$ and so $a_{\ell'1}v_{\ell'}v_1(-1+u_1) =-a_{\ell'1}v_{\ell'}<0$. The desired contradiction shows that the solution is unique.

% \subsection{Necessity}

% Suppose that $\mathbf{x} = (x_i)_{i\in[n]}$ ($x_i \in \R_+ \cup \{0\}$, $i \in [n]$) is a solution to~(\ref{eq:system}). We will show that the condition~(\ref{eq:condition1}) has to be satisfied. By symmetry, the same argument proves that the condition~(\ref{eq:condition2}) is satisfied too.

% Consider any $j\in[m]$. Let us first observe that
% \begin{eqnarray*}
% - b_{j+m} + \sum_{i=m+1}^nb_i &=& \sum_{i\in[n]\setminus([m]\cup\{j+m\})} b_i =  \sum_{i\in[n]\setminus([m]\cup\{j+m\})}\left( x_i \sum_{k=1}^na_{ik}x_k \right) \\
% &\geq& \sum_{i\in[n]\setminus([m]\cup\{j+m\})}\left( x_i a_{ij}x_j \right).
% \end{eqnarray*}

% Finally, we use the fact that $a_{ij}=0$ if $i\in[m] \cup \{j+m\}$ to get that 
% $$
% - b_{j+m} + \sum_{i=m+1}^nb_i \geq \sum_{i\in[n]\setminus([m]\cup\{j+m\})}\left( x_i a_{ij}x_j \right) = x_j\sum_{i=1}^{n}x_i a_{ij}=b_j.
% $$
% It follows that~(\ref{eq:condition1}) is a necessary condition for the existence of a solution to~(\ref{eq:system}).

\subsection{Sufficiency}

We will continue assuming that $m \geq 3$. For a contradiction, suppose that there exists a vector $\mathbf{b} = (b_i)_{i \in [n]}$, that satisfies~(\ref{eq:condition1})--(\ref{eq:condition2}), and $\sum_{i=1}^{m}b_i=\sum_{i=m+1}^{n}b_i>0$ (assumption~(\ref{eq:assumption})) but for which there is no solution to the system~(\ref{eq:system}). Without loss of generality, since one can reorder nodes and relabel in- and out-degrees if needed, we may assume that $b_1$ is a largest value in vector $\mathbf{b}$. We will call such vectors \emph{infeasible}. On the other hand, vectors that yield a solution $\mathbf{x} = (x_i)_{i \in [n]}$, with $x_i\geq0$ for all $i$, will be called \emph{feasible}. As proved earlier, if the solution exists, then it must be unique (remember that we assume that $x_1=1$). We will introduce more vectors $\textbf{b}$ (both feasible and infeasible) below but we assume that matrix $\textbf{A}$ is fixed.

Let us now construct another vector $\mathbf{b}' = (b'_i)_{i \in [n]}$ for which there exists a solution to~(\ref{eq:system}) (that is, $\mathbf{b}'$ is feasible) but also $b'_1=b_1$ is a largest element in $\mathbf{b}'$. Indeed, it can be done easily by, for example, taking $x'_1=1$, $x'_i = s$ for $i\in[m]\setminus\{1\}$ ($s$ is a fixed but sufficiently small positive constant for the inequalities below to hold), and $x'_i = b_1 / (\sum_{j \in [n]} a_{1j}) = b_1 / (\sum_{j \in [n] \setminus [m]} a_{1j})$ for $i\in[n]\setminus[m]$. Vector $\mathbf{b}'$ is now defined by the system~(\ref{eq:system}). We immediately get that
$$
b'_1 = x'_1 \sum_{j \in [n]} a_{1j} x'_j = \sum_{j \in [n] \setminus [m]} a_{1j} x'_j = b_1.
$$
Also, $s$ can be made arbitrarily small so that $b'_i<b_1$ for $i\in[m]\setminus\{1\}$.
Finally, for $i \in[n]\setminus[m]$ we have
$$
b'_i = x'_i \sum_{j \in [m]} a_{ij} x'_j = x'_i a_{i1} + x'_i \sum_{j=2}^m a_{ij} s = b_1 \frac{a_{i1}} {\sum_{j \in [n]} a_{1j}} + x'_i \sum_{j=2}^m a_{ij} s.
$$
Since $m \ge 3$, the first term is smaller than $b_1$. Hence, since $s$ can be made arbitrarily small, we can ensure that $b'_i < b_1$. The desired properties hold.

We will consider points along the line segment between $\mathbf{b}'$ and $\mathbf{b}$, namely,
$$
\mathbf{b}(t) = (b_i(t))_{i \in [n]} = (1-t)\mathbf{b}'+t\mathbf{b}, \qquad \text{ for } t\in[0,1].
$$
Since $\mathbf{b}'$ is feasible and we already proved that~(\ref{eq:condition1})--(\ref{eq:condition2}) are necessary conditions, we know that $\mathbf{b}'$ satisfies~(\ref{eq:condition1})--(\ref{eq:condition2}). But, as a result, not only $\mathbf{b}$ and $\mathbf{b}'$ satisfy these properties but also $\mathbf{b}(t)$ satisfies them for any $t \in [0,1]$. In particular, it follows that there exists a universal constant $\eps > 0$ such that for any $t \in [0,1]$ we have 
\begin{equation}\label{eq:separation}
(1-\eps)\sum_{i=m+2}^nb_i(t) > b_1(t).
\end{equation}

Fix $t \in [0,1)$ and suppose that $\mathbf{b}(t)$ is feasible. Let $\mathbf{x}(t) = (x_i(t))_{i\in [n]}$ be the (unique) solution for $\mathbf{b}(t)$. From the analysis performed in the proof of uniqueness of the solution it follows that our transformation is a local diffeomorphism, that is, the differential of the transformation is bijective for the admissible values of $x_i$ and $b_i$. (Note that this also covers the case $t=0$. This case is on the boundary of the considered range of $t$ but it is an interior point of the domain of the mapping.) In the following considerations, we assume that point $x_1$ and $b_1$ are removed from the analysis (as they are fixed) and also that the indices from the set $Z$ (that is, as defined above, the set of indices $i$ for which $v_i=0$) are excluded as they are fixed. As a result we may move to a manifold of a dimension $n'<n$ by dropping the dimensions that are fixed. In the considered manifold, any open set in $\R^{n'}$ containing (part of) $\mathbf{x}(t)$ is mapped to an open set in $\R^{n'}$ containing (part of) $\mathbf{b}(t)$. In particular, there exists $\delta > 0$ such that $\mathbf{b}(s)$ is feasible for any $t - \delta \le s \le t + \delta$. Combining this observation with the fact that $\mathbf{b}'=\mathbf{b}(0)$ is feasible, $\mathbf{b}=\mathbf{b}(1)$ is \emph{not} feasible we get that there exists $T \in (0,1]$ such that $\mathbf{b}(T)$ is not feasible but $\mathbf{b}(t)$ is feasible for any $t \in [0,T)$. Indeed, if no such $T$ exists (that is, there is no minimal infeasible $t \in (0,1]$), then there would exist a decreasing sequence of infeasible values of $t$ that converges to a feasible $t$. This is not possible as in some neighbourhood of a feasible point $t$, points are also feasible.

Consider any sequence $(t_i)_{i \in \N}$ of real numbers $t_i \in [0,T)$ such that $t_i\to T$ as $i \to \infty$; for example, $t_i = T(1-1/i)$. All limits from now on will be for $i \to \infty$. Recall that $\mathbf{b}(t_i)$ is feasible and so $\mathbf{x}(t_i)$ is well-defined.

Before we move forward, let us show that there exists a sufficiently large but universal constant $\Delta$ such that for all $t \in [0,T)$ and all $i$ (except possibly $i=m+1$), we have $x_i(t) \le \Delta$. Indeed, by our assumption on the solution, $x_1(t)=1\le \Delta$. By the equation~(\ref{eq:system}) for $b_1(t) = b_1$, we have for $i\in[n]\setminus[m+1]$
$$
x_i(t) = \frac {1}{a_{1i}} \cdot x_1(t) a_{1i} x_i(t) < \frac {1}{a_{1i}} \cdot x_1(t) \sum_{j=1}^n a_{1j} x_j(t) = \frac {b_1(t)}{a_{1i}} = \frac {b_1}{a_{1i}} \le \Delta.
$$
But this immediately means that $x_i(t)$ are also bounded for $i\in[m]$ by considering any equation for $b_i(t) \le b_1(t) =b_1$ where $i\in[n]\setminus([m]\cup Z)$. This implies that only $x_{m+1}(t)$ can potentially be unbounded.

If $x_{m+1}(t)$ is bounded for all $t \in [0,T)$, then by the Bolzano-Weierstrass theorem the sequence $t_i$ has a subsequence $(\mathbf{x}(t_{s_i}))_{i \in [n]}$ such that $\Vert \mathbf{x}(t_{s_i}) \Vert \to c$ for some $c \in \R$. However, if this is the case then, by continuity of our transformation, the limiting value $\mathbf{b}(T)$ would be feasible, giving us the desired contradiction. It remains to consider the case when $x_{m+1}(t_{s_i}) \to \infty$ for some subsequence $s_i$.
However, this implies that $x_{j}(t_{s_i})\to 0$ for all $j\in[m]\setminus\{1\}$. This means that in the limit we have $x_i(T)=b_i(T)/a_{i1}$ for $i\in[n]\setminus[m+1]$. Substituting it into the first equation we get 
$$
b_1(T) = x_1(T)\sum_{i=m+2}^na_{1i}x_i(T) = 1\cdot\sum_{i=m+2}^na_{1i}\cdot b_i(T)/a_{i1} = \sum_{i=m+2}^nb_i(T).
$$
This contradicts (\ref{eq:separation}), which concludes the proof.

\bigskip

As a final note, let us observe that the proof implies that if $b_1$ gets close to $\sum_{i=m+2}^nb_i$ (from below) and $b_{m+1}>0$, then indeed $x_{m+1}$ will grow to be a large number. This consideration has a numerical impact as it might affect the convergence of numeric algorithms finding $x_i$ due to floating point computation precision issues. The proof also shows that the case when the conditions~(\ref{eq:condition1})--(\ref{eq:condition2}) are not satisfied (that is, we would have an equality instead of inequality) will have a solution if $b_{m+1}=0$ and otherwise will not have a solution (this corresponds to the two examples we have given earlier).

\subsection{Model with Loops}

In order to accommodate loops that are present in the graph, we relax the assumption that $a_{i,i+m}=0$ for $i\in[m]$ and now assume that $a_{i,i+m} \ge 0$ for $i\in[m]$. However, using the notation from the previous section we will additionally assume that $a_{1, m+1}>0$, that is, for the largest element of $\mathbf{b}$, we assume that the corresponding node has a loop. This auxiliary assumption is satisfied in our application as, in fact, all landmarks have loops.

The $m=2$ case continues to be a degenerate case that has to be delt with independently. Consider the following set of equations:
$$
\left[\begin{matrix}
b_1 \\ b_2 \\ b_3 \\ b_4
\end{matrix}\right] =
\left[\begin{matrix}
a_{13} & a_{14} & 0 & 0 \\
0 & 0 & a_{23} & a_{24} \\
a_{13} & 0 & a_{23} & 0 \\
0 & a_{14} & 0 & a_{24}
\end{matrix}\right]
\left[\begin{matrix}
x_1x_3 \\ x_1x_4 \\ x_2x_3 \\ x_2x_4
\end{matrix}\right].
$$
As before, we assume that $x_1=1$ and, since $b_1+b_2=b_3+b_4$ the system reduces to: 
$$
\left[\begin{matrix}
b_1 \\ b_2 \\ b_3
\end{matrix}\right] =
\left[\begin{matrix}
a_{13} & a_{14} & 0 & 0 \\
0 & 0 & a_{23} & a_{24} \\
a_{13} & 0 & a_{23} & 0
\end{matrix}\right]
\left[\begin{matrix}
x_3 \\ x_4 \\ x_2x_3 \\ x_2x_4
\end{matrix}\right] =
\left[\begin{matrix}
a_{14} & 0 & a_{13} & 0 \\
0 & a_{24} & 0 & a_{23} \\
0 & 0 & a_{13} & a_{23} 
\end{matrix}\right]
\left[\begin{matrix}
x_4  \\ x_2x_4 \\ x_3 \\ x_2x_3
\end{matrix}\right]
.
$$
Equivalently, since $b_1$ is the largest element of $\mathbf{b}$, for some non-negative $p_i$ and positive $q=a_{23}$: 
$$
\left[\begin{matrix}
p_1 \\ p_2 \\ p_3
\end{matrix}\right] =
\left[\begin{matrix}
1 & 0 & 0 & -q \\
0 & 1 & 0 & q \\
0 & 0 & 1 & q
\end{matrix}\right]
\left[\begin{matrix}
x_4  \\ x_2x_4 \\ x_3 \\ x_2x_3
\end{matrix}\right].
$$
If $x_2=0$, we get a unique positive solution for $x_3$ and $x_4$ and it exists only if $p_2=0$ (which happens for $b_2=0$).
On the other hand, if $x_2>0$ (recall that, by assumption, $x_i \ge 0$), then in the above system of equations one can reduce $x_3$ and $x_4$ leaving only $x_2$ as a variable in the quadratic equation:
$$
Q(x_2)=q(p_1+p_3)x_2^2+(-p_2q+p_3q+p_1)x_2-p_2 = 0.
$$
Since $x_2 > 0$, we get that $p_2 =x_2(x_4+qx_3) > 0$. (Note that if $x_3=x_4=0$, then $b_3=b_4=0$ and, as a consequence, $b_1=b_2=0$ which gives as a contradiction as $b_1>0$.) As the term $(p_1+p_3)q$ is positive and $-p_2$ is negative we get that there exists exactly one positive solution $x_2$ of this equation. Indeed, since $Q(0) = -p_2 <0$ and the parabola $Q(x_2)$ has the coefficient $(p_1+p_3)q>0$ associated with the quadratic term, there is exactly one positive solution $x_2 > 0$. Now, assuming that $x_2>0$, from the third equation we see that $x_3>0$ and from the first equation we get that $x_4>0$. 

In summary, for $m=2$, subject to the constraint $x_1=1$, the solution of the system always exists and is unique.
If $m\geq3$ we also show that the solution exists always. The part of the proof of uniqueness remains unchanged. The part for sufficiency also remains unchanged until we reach the case where we consider $x_{m+1}(t_{s_i}) \to\infty$. However as $a_{1,m+1}>0$ this is not possible since $x_1=1$ and $b_1$ is fixed. So we are left with the cases that $\Vert \mathbf{x}(t_{s_i}) \Vert \to c$ for some $c \in \R$ which means that $\mathbf{b}(t)$ always converges to a feasible solution as $t\to1$ (even if in conditions~(\ref{eq:condition1})--(\ref{eq:condition2}) we have an equality).




\section{Appendix---Scalable Implementation with Landmarks}\label{apdx:landmarks}

Recall that in \textbf{Step~1} of the algorithm, we obtain a partition $\textbf{C}$ of the set of nodes $V$ into $\ell$ communities: $C_1, \ldots, C_\ell$. The partition is then carefully refined by repeatedly splitting some parts of it with the goal to reach precisely $n'=4 \sqrt{n}$ parts; $n'$ might be adjusted by more experienced user, if needed. (The number of communities is typically relatively small. However, if $\ell \ge 4 \sqrt{n}$, then of course there is no need to do the refinement. However, in such rare cases each part in the initial partition is forced to be split into $s$ parts anyway. We fixed $s=4$ as a default value.) The heuristic algorithm is quite involved as it needs to find a good compromise between the quality of the approximation and the speed. The reader is directed to~\cite{Embedding_Complex_Networks_Scalable} for more details on how the refinement is obtained. 

Once the partition is refined, each part $C_i$ is replaced by its landmark $u_i$. The procedure depends on whether we deal with undirected or directed graphs. Let us start with undirected graphs. The position of landmark $u_i$ in the embedded space $\R^k$ is assigned as follows:
\begin{equation}\label{eq:position_landmark}
\emb(u_i) = \frac {\sum_{j \in C_i} w_j \ \emb(v_j)}{\sum_{j \in C_i} w_j}.
\end{equation}
In order to measure a variation within a cluster, we also compute the weighted sum of squared errors:
\begin{equation}\label{eq:error_landmark}
e_i = \sum_{j \in C_i} w_j \ \dist \big( \emb(u_i), \emb(v_j) \big)^2.
\end{equation}
The expected degree of landmark $u_i$ (that we denote as $w'_i$ in order to distinguish it from $w_i$, the expected degree of node $w_i$) is the sum of the expected degrees of the associated nodes in the original model, that is, $w'_i := \sum_{j \in C_i} w_j$.

The approximated algorithm uses the auxiliary \emph{Geometric Chung-Lu} (GCL) model on the set of landmarks $V = \{ u_1, \ldots, u_{n'} \}$ in which each pair of landmarks $u_i, u_j$, independently of other pairs, forms an edge with probability $p'_{i,j}$, where
\begin{equation*}
p'_{i,j} = x_i' x'_j g(d_{i,j}) 
\end{equation*}
for some carefully tuned weights $x'_i \in \R_+$. Additionally, for $i\in[n']$, the probability of creating a self loop around landmark $u_i$ is equal to 
$$
p'_{i,i} = (x'_i)^2 g(d_{i,i}), \qquad \text{ where } \qquad d_{i,i} = \sqrt{ \frac{e_i}{\sum_{j \in C_i} w_j} }.
$$
Note that the ``distance'' $d_{i,i}$ from landmark $u_i$ to itself is an approximation of the unobserved weighted average distance $d_{a,b}$ over all pairs of nodes $a$ and $b$ associated with $u_i$.
The weights are selected such that the expected degree of landmark $u_i$ is $w'_i$; that is, for all $i \in [n']$
$$
w'_i = \sum_{j \in [n']} p'_{i,j} =  x'_i \sum_{j \in [n']} x'_j g(d_{i,j}).
$$
The relationship between the weights in the auxiliary model and the original one is expected to be as follows: for any node $v_k \in C_i$ associated with landmark $u_i$ we have
$$
x_k \approx x'_i \ \frac {w_k}{\sum_{j \in C_i} w_j}.
$$

The adjustment for directed graph is straightforward. We use the same algorithm and positions for landmarks, that is, we still use~(\ref{eq:position_landmark}) and~(\ref{eq:error_landmark}) but with $w_j$ being the total degree of landmark $u_j$, namely, $w_j = w_j^{in} + w_j^{out}$. The probability for an ordered pair of landmarks $u_i, u_j$ to form a directed edge in the auxiliary \emph{Geometric Chung-Lu Directed Graph Model} is equal to 
$$
p'_{i,j} = {x'}_i^{out} {x'}_j^{in} g(d_{i,j}) 
$$
and
$$
p'_{i,i} = {x'}_i^{out} {x'}_i^{in} g(d_{i,i}), \qquad \text{ where } \qquad d_{i,i} = \sqrt{ \frac{e_i}{\sum_{j \in C_i} w_j} }.
$$
As before, any node $v_k \in C_i$ associated with landmark $u_i$ inherits a fraction of its weights, that is, we expect that the original weights are well approximated by the following:
$$
x_k^{out} \approx {x'}_i^{out} \ \frac {w_k^{out}}{\sum_{j \in C_i} w_j^{out}} \qquad  \text{and} \qquad x_k^{in} \approx {x'}_i^{in} \ \frac {w_k^{in}}{\sum_{j \in C_i} w_j^{in}}.
$$

%%========================================
\newcommand{\twomoons}{{\tt Twomoons}}
\newcommand{\gauss}{{\tt Gauss}}
\newcommand{\sculpture}{{\tt Sculpture}}
\newcommand{\baseline}{{\tt Baseline}}
\newcommand{\MM}{{\tt MsgPassing}}
\newcommand{\blackboard}{{\tt Blackboard}}
\newcommand{\ncut}{\text{ncut}}
\newcommand{\chensays}[2][]{\textcolor{blue} {\textsc{Jiecao #1:} \emph{#2}}}

\section{Experiments}
In this section we present experimental results for  graph clustering in the message passing and blackboard models. We will compare the following three algorithms. (1) \baseline: each site sends all the data to the coordinator directly; (2) \MM: our algorithm in the message passing model (Section~\ref{sec:gcmessage}); (3) 
\blackboard: our algorithm in  the blackboard model (Section~\ref{sec:bb}).


%Since both of our algorithms are crucially based on the use of spectral scarification, our main focus in the experiments is to investigate to what extend the quality of the spectral clustering algorithms will be affected by using spectral sparsification, the saving of communication costs by using spectral sparsificaion, ...
%
%
%The goal of this experiment is not to demonstrate the effectiveness of the spectral clustering algorithm. We mainly want to investigate the following, 
%\begin{itemize}
%\item to what extend the quality of clustered results will be affected by using spectral sparsification.
%\item saving of communication costs by using spectral sparsifier.
%\item the affect of constants in algorithms of the message passing/blackboard model.
%\end{itemize}
%
%
%\subsection{The Setup}
%\paragraph{Reference Algorithms}
%We compare different algorithms in our experiment.

%Note that we can also run \MM~ in the blackboard model.

Besides giving the visualized results of these algorithms on various datasets, we also measure the qualities of the results via the {\em normalized cut}, defined as 
\[
\ncut(A_1, \ldots, A_{k}) = \frac{1}{2}\sum_{i\in[k]}\frac{w(A_i, V\backslash A_i)}{\vol(A_i)},
\]
 which is a standard objective function to be minimized for spectral clustering algorithms. 
%We will compare the communication costs of these algorithms in different settings.

%We also compare the total communication costs of different algorithms/models. As the unit does not matter in our case, we normalize all communication costs by the cost of \baseline.  Whenever possible, we will visualize the clustered results.

We implemented the algorithms using multiple languages, including Matlab, Python and C++. Our experiments were conducted on an IBM NeXtScale nx360 M4 server, which is equipped with 2 Intel Xeon E5-2652 v2 8-core processors, 32GB RAM and 250GB local storage.


\subsection{Datasets.}
We test the algorithms in the following real and synthetic datasets, which is visualized in \figref{visualization}.


\begin{figure}[h]
     \centering
     \subfigure[\twomoons]{\includegraphics[width=0.23\textwidth]{twomoons-14000-original.png}\label{fig:twomoons}}
     ~~
     \subfigure[\gauss]{\includegraphics[width=0.23\textwidth]{gauss-10000-original.png}\label{fig:gauss}}
     ~~
     \subfigure[\sculpture]{\includegraphics[width=0.13\textwidth,height=0.16\textwidth]{sculpture-11680-original.jpg}\label{fig:sculpture}}
     \caption{Visualization of the datasets for our experiments.}
     \label{fig:visualization}
\end{figure}



\vspace{-1mm}
\begin{itemize}
\item \twomoons : this dataset contains $n=14,000$ coordinates in $\mathbb{R}^2$. We consider each point to be a vertex. For any two vertices $u, v$, we add an edge with weight $w(u,v) = \exp\{-\|u-v\|_2^2/\sigma^2\}$ with $\sigma = 0.1$ when one vertex is among the $7000$-nearest points of the other.  This construction results in a graph with about $110,000,000$ edges.

\item  \gauss : this dataset contains $n = 10,000$ points in $\mathbb{R}^2$. There are $4$ clusters in this dataset, each generated using a Gaussian distribution. We construct a complete graph as the similarity graph.  For any two vertices $u, v$, we define the weight $w(u,v) = \exp\{-\|u-v\|_2^2/\sigma^2\}$ with $\sigma = 1$. The resulting graph has about $100,000,000$ edges.

\item \sculpture : a photo of \textit{The Greek Slave}~\footnote{Available in e.g., \url{http://artgallery.yale.edu/collections/objects/14794}}. We use an $80\times 150$ version of this photo where each pixel is viewed as a vertex. To construct a similarity graph, we map each pixel to a point in $\mathbb{R}^5$, i.e., $(x, y, r, g, b)$, where the latter three coordinates are the RGB values. For any two vertices $u, v$, we  put an edge between $u, v$ with weight $w(u,v) = \exp\{-\|u-v\|_2^2/\sigma^2\}$ with $\sigma = 0.5$ if one of $u, v$ is among the $5000$-nearest points of the other. This results in a graph with about $70,000,000$ edges.
\end{itemize}
\vspace{-1mm}
In the distributed model edges are randomly partitioned across $s$ sites. 

%\vspace{-1.5mm}



\subsection{Results on clustering quality}
%{\em Quality.} \
\begin{figure*}[ht]
     \centering
     \subfigure[\baseline]{\includegraphics[width=0.2\textwidth]{twomoons-14000-original-clustered.png}\label{fig:twomoons-clustered-original}}
     \subfigure[\MM]{\includegraphics[width=0.2\textwidth]{twomoons-14000-sparsify-clustered-15.png}\label{fig:twomoons-clustered-sparsify}}
     \subfigure[\blackboard]{\includegraphics[width=0.2\textwidth]{twomoons-14000-chain-clustered.png}\label{fig:twomoons-clustered-chain}}
     \caption*{\twomoons, $k = 2$;}

\subfigure[\baseline]{\includegraphics[width=0.2\textwidth]{gauss-10000-original-clustered.png}\label{fig:gauss-clustered-original}}
     \subfigure[\MM]{\includegraphics[width=0.2\textwidth]{gauss-10000-sparsify-clustered-15.png}\label{fig:gauss-clustered-sparsify}}
     \subfigure[\blackboard]{\includegraphics[width=0.2\textwidth]{gauss-10000-chain-clustered.png}\label{fig:gauss-clustered-chain}}
     \caption*{\gauss, $k = 4$}


     \subfigure[\baseline]{\includegraphics[width=0.2\textwidth,height=0.2\textwidth]{sculpture-11680-original-clustered.png}\label{fig:sculpture-clustered-original}}  
     \subfigure[\MM]{\includegraphics[width=0.2\textwidth,height=0.2\textwidth]{sculpture-11680-sparsify-clustered-15.png}\label{fig:sculpture-clustered-sparsify}}
     \subfigure[\blackboard]{\includegraphics[width=0.2\textwidth,height=0.2\textwidth]{sculpture-11680-chain-clustered.png}\label{fig:sculpture-clustered-chain}}
     \caption*{\sculpture, $k = 3$. }


     
     \caption{Visualization of the results on \twomoons, \gauss\ and \sculpture. In the message passing model each site samples $5 n$ edges; in the blackboard model all sites jointly sample $10n$ edges (in \twomoons~ and \gauss) or $20n$ edges (in \sculpture) and the chain has length $18$. $s = 15$.}
     \label{fig:quality-1}
\end{figure*}

We visualize the clustered results for 
the \twomoons, \gauss\ and \sculpture\ in Figure~\ref{fig:quality-1}.
% and visualize the clustered results for \gauss\ and \sculpture in Figure~\ref{fig:quality-2}.
It can be seen that \baseline, \MM\ and \blackboard\ give results of very similar qualities.  For simplicity, here we only present the visualization for $s=15$. Similar results were observed when we varied the values of $s$.  
%\he{To Qin: Do you plan to have two titles (Results \& Quality)?}


% \begin{figure*}[h]
%      \centering
% \subfigure[\baseline]{\includegraphics[width=0.3\textwidth]{gauss-10000-original-clustered.png}\label{fig:gauss-clustered-original}}
%      \subfigure[\MM]{\includegraphics[width=0.3\textwidth]{gauss-10000-sparsify-clustered-15.png}\label{fig:gauss-clustered-sparsify}}
%      \subfigure[\blackboard]{\includegraphics[width=0.3\textwidth]{gauss-10000-chain-clustered.png}\label{fig:gauss-clustered-chain}}
%      \caption*{\gauss, $k = 4$}


%      \subfigure[\baseline]{\includegraphics[width=0.2\textwidth]{sculpture-11680-original-clustered.png}\label{fig:sculpture-clustered-original}}  
%      \subfigure[\MM]{\includegraphics[width=0.2\textwidth]{sculpture-11680-sparsify-clustered-15.png}\label{fig:sculpture-clustered-sparsify}}
%      \subfigure[\blackboard]{\includegraphics[width=0.2\textwidth]{sculpture-11680-chain-clustered.png}\label{fig:sculpture-clustered-chain}}
%      \caption*{\sculpture, $k = 3$. }

%      \caption{Visualization of results on \gauss\ and \sculpture; in the message passing model each site samples $5 n$ edges; in the blackboard model all sites jointly sample $10n$ (in \gauss) or $20n$ (in \sculpture) edges and the chain has length $18$.}
%      \label{fig:quality-2}
% \end{figure*}


We also compare the normalized cut (ncut) values of the clustering results of different algorithms.  The results are presented in Figure \ref{fig:quality}. In all datasets, the ncut values of different algorithms are very close. The ncut value of \MM\ slightly decreases when we increase the value of $s$, while the ncut value of \blackboard\ is independent of $s$.
%We comment that in general, it is difficult to compare \MM\ and \blackboard\ directly because they are affected by different parameters.


\begin{figure*}[!ht]
  \centering
  \subfigure[\twomoons]{\includegraphics[width=0.33\textwidth]{twomoons-14000-ncut.png}\label{fig:twomoons-quality}}\hspace*{-1.1em}
  \subfigure[\gauss]{\includegraphics[width=0.31\textwidth]{gauss-10000-ncut.png}\label{fig:gauss-quality}}\hspace*{-1.1em}
  \subfigure[\sculpture]{\includegraphics[width=0.31\textwidth]{sculpture-11680-ncut.png}\label{fig:sculpture-quality}}\hspace*{-1.1em}
  \subfigure{\includegraphics[width=0.14\textwidth]{legend.png}}
     \caption{Comparisons on normalized cuts. In the message passing model, each site samples $5n$ edges; in each round of the algorithm in the blackboard model, all sites jointly sample $10n$ edges (in \twomoons~and \gauss) or $20n$ edges (in \sculpture) edges and the chain has length $18$.}
     \label{fig:quality}
\end{figure*}

%\textcolor{red}{To Jiecao: Can you put the color lines indicating baseline, message passing, and blackboard within one row in Pic 2? Withthis we can save some space.}

%\vspace{-1.5mm}

\subsection{Results on communication costs} 
\begin{figure*}[!ht]
     \centering
     \subfigure[\twomoons]{\includegraphics[width=0.3\textwidth]{twomoons-14000-communication.png}\label{fig:twomoons-communication}}
     \subfigure[\gauss]{\includegraphics[width=0.3\textwidth]{gauss-10000-communication.png}\label{fig:gauss-communication}}
     \subfigure[\sculpture]{\includegraphics[width=0.3\textwidth]{sculpture-11680-communication.png}\label{fig:sculpture-communication}}


     \subfigure[\twomoons]{\includegraphics[width=0.32\textwidth]{twomoons-14000-communication-2.png}\label{fig:twomoons-communication-2}}
     \subfigure[\gauss]{\includegraphics[width=0.32\textwidth]{gauss-10000-communication-2.png}\label{fig:gauss-communication-2}}
     \subfigure[\sculpture]{\includegraphics[width=0.32\textwidth]{sculpture-11680-communication-2.png}\label{fig:sculpture-communication-2}}
     \caption{Comparisons on communication costs. In the message passing model, each site samples $5n$ edges; in each round of the algorithm in the blackboard model, all sites jointly sample $10n$ (in \twomoons~and \gauss) or $20n$ (in \sculpture) edges and the chain has length $18$. }
     \label{fig:communication}
\end{figure*}

We compare the communication costs of different algorithms in Figure \ref{fig:communication}. We observe that while achieving similar clustering qualities as \baseline, both \MM\ and \blackboard\ are significantly more communication-efficient (by one or two orders of magnitudes in our experiments). We also notice that the value of $s$ does not affect the communication cost of \blackboard, while the communication cost of \MM\ grows almost linearly with $s$; when $s$ is large, \MM\ uses significantly more communication than \blackboard. These confirm our theory.  %In Figure~\ref{fig:mm-const} and Figure~\ref{fig:blackboard-const}   in Appendix~\ref{sec:parameters} we present how the performance of \MM\ and \blackboard\ are affected by their parameters.

%
%
%\vspace{-1.5mm}
%\paragraph{Summary.}  From our experimental results we conclude that \MM\ and \blackboard\ achieve similar clustering quality as the native algorithm \baseline, while significantly reduce the communication cost.  When the number of sites is large, \blackboard\ is more communication efficient than \MM, as predicted by our theory.



\subsection{Parameters in \MM\ and \blackboard}
\label{sec:parameters}

Figure \ref{fig:mm-const} shows in \MM how the value of ncut is affected by the number of sites and the number of edges sampled in each site. 
Here, each site samples $cn$ edges. 
When $c=3$ and $s=1$, the ncut value diverges in all datasets. This is because with such a small $c$, the algorithm does not generate a valid sparsifier. In general, increasing $c$ or $s$ will slightly decrease the ncut value. But once they are above some thresholds, the ncut values of \MM\ and \baseline\ become very close.

Figure \ref{fig:blackboard-const} shows in \blackboard  how the ncut value is affected by the number of iterations and the number of edges sampled. When the number of iterations is set to be $5$, ncut values diverge in all datasets. This is because we cannot expect to generate a valid sparsifier by using such few iterations. It can be seen from \ref{fig:bb-gauss-constant} that for a fixed $c$, performing more iterations will help to reduce ncut values. From the same figure, one can also conclude that for fixed iterations, increasing $c$ also helps to reduce the ncut values.



\begin{figure*}[h!t]
     \centering
     \subfigure[\twomoons]{\includegraphics[width=0.3\textwidth]{twomoons-c.png}\label{fig:mm-twomoons-constant}}
     \subfigure[\gauss~dataset]{\includegraphics[width=0.3\textwidth]{gauss-c.png}\label{fig:mm-gauss-constant}}
     \subfigure[\sculpture]{\includegraphics[width=0.3\textwidth]{sculpture-c.png}\label{fig:mm-sculpture-constant}}
     \caption{The pictures above show the $\ncut$ values with respect to the values of $c$ and $s$ for the \MM\ algorithm. Here  
 each site samples $c n$ edges.}
     \label{fig:mm-const}
\end{figure*}


\begin{figure*}[h!t]
     \centering
     \subfigure[\twomoons]{\includegraphics[width=0.3\textwidth]{twomoons-iter.png}\label{fig:bb-twomoons-constant}}
     \subfigure[\gauss]{\includegraphics[width=0.3\textwidth]{gauss-iter.png}\label{fig:bb-gauss-constant}}
     \subfigure[\sculpture]{\includegraphics[width=0.3\textwidth]{sculpture-iter.png}\label{fig:bb-sculpture-constant}}
     \caption{The pictures above show how the $\ncut$ values are affected by the number of iterations and the value of $c$ for the \blackboard\ algorithm. Here 
all sites jointly sample $c n$ edges. }
     \label{fig:blackboard-const}
\end{figure*}






%%========================================

\begin{comment}
\begin{figure}
\includegraphics[width=\linewidth]{figs/beyond_tss_lesion.pdf}
\caption[]{End-to-End runtime lesion study of the entire MNIST dataset and the FMA featurized music dataset. Each of DROP's contributions provides a runtime improvement.}
\label{fig:beyond_lesion}
\end{figure}
\end{comment}



\section{Conclusion}
\label{sec:conclusion}

Advanced data analytics techniques must scale to rising data volumes. 
DR techniques offer a powerful toolkit when processing these datasets, with PCA frequently outperforming popular techniques in exchange for high computational cost. 
In response, we propose DROP, a new dimensionality reduction optimizer. 
DROP combines progressive sampling, progress estimation, and online aggregation to identify high quality low dimensional bases via PCA without processing the entire dataset by balancing the runtime of downstream tasks and achieved dimensionality. 
Thus, DROP provides a first step in bridging the gap between quality and efficiency in end-to-end DR for downstream \red{analytics}. 

%We revisit canonical operators for time series dimensionality reduction and the measurement study of~\cite{keogh-study}, and show that PCA is more effective than popular alternatives in the data mining literature often by a margin of over $2\times$ on average on gold-standard time series benchmark data sets with respect to output data dimension. More surprisingly, we empirically demonstrate that a small number of samples are sufficient to accurately characterize directions of maximum variance and obtain a high-quality low-dimensional transformation.



%%========================================


%% \section*{Acknowledgment}
%% The authors would like to thank Muhamed Begovic from University of Stuttgart for his help on the hardware setup, during his internship at BCAI.



%========================================
\bibliographystyle{IEEEtran}
\bibliography{contents/references}

\end{document}
