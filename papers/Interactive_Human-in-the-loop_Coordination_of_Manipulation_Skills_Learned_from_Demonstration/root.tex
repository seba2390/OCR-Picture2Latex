\documentclass[letterpaper, 10 pt, conference]{ieeeconf}  % Comment this line out if you need a4paper

%\documentclass[a4paper, 10pt, conference]{ieeeconf}% Use this line for a4 paper

\IEEEoverridecommandlockouts% This command is only needed if 
% you want to use the \thanks command

\overrideIEEEmargins  % Needed to meet printer requirements.




%In case you encounter the following error:
%Error 1010 The PDF file may be corrupt (unable to open PDF file) OR
%Error 1000 An error occurred while parsing a contents stream. Unable to analyze the PDF file.
%This is a known problem with pdfLaTeX conversion filter. The file cannot be opened with acrobat reader
%Please use one of the alternatives below to circumvent this error by uncommenting one or the other
%\pdfobjcompresslevel=0
%\pdfminorversion=4

% See the \addtolength command later in the file to balance the column lengths
% on the last page of the document

% numbers option provides compact numerical references in the text. 
% \usepackage{natbib}
\usepackage{times}
\usepackage{multicol}
\usepackage[bookmarks=true]{hyperref}
\usepackage{xcolor}
\usepackage{hyperref}
\usepackage{amssymb}
\usepackage{amsmath, amsfonts}
\usepackage{graphicx}
\usepackage{siunitx}
\usepackage{standalone}
\usepackage{booktabs}
%\usepackage{algorithm,algorithmicx,algpseudocode}
\usepackage[ruled,vlined]{algorithm2e}
\usepackage{mdframed}
\usepackage{fancyvrb}
\usepackage{soul}
\usepackage{dsfont,mathabx}
\usepackage[font=footnotesize, skip=5pt]{caption}

\newtheorem{theorem}{Theorem}
\newtheorem{problem}{Problem}

% GENERAL
\newcommand{\todo}[1]{\textcolor{red}{TODO: #1}}
\newcommand{\improve}[1]{\textcolor{cyan}{#1}}
\newcommand{\note}[1]{\textcolor{blue}{NOTE: #1}}
\newcommand{\question}[1]{\textcolor{orange}{Q: #1}}


%%%%%%%%%%%%%%%%%%%%%%%%%%%%%%%%%%%%%%%%%%%%%%%%%%%%%%%
%%%%


\title{\LARGE \bf
Interactive Human-in-the-loop Coordination of Manipulation Skills\\ Learned from Demonstration
}


\author{Meng Guo$^{1}$ and Mathias B\"urger$^{2}$% <-this % stops a space
\thanks{$^{1}$College of Engineering, Peking University, China. $^{2}$Bosch Center for Artificial Intelligence (BCAI), Germany. Corresponding author: Meng Guo. \texttt{meng.guo@pku.edu.cn}.}}% <-this % stops a space


\begin{document}
  
\maketitle



%%========================================
\maketitle

%%========================================
  In this paper, we explore the connection between secret key agreement and secure omniscience within the setting of the multiterminal source model with a wiretapper who has side information. While the secret key agreement problem considers the generation of a maximum-rate secret key through public discussion, the secure omniscience problem is concerned with communication protocols for omniscience that minimize the rate of information leakage to the wiretapper. The starting point of our work is a lower bound on the minimum leakage rate for omniscience, $\rl$, in terms of the wiretap secret key capacity, $\wskc$. Our interest is in identifying broad classes of sources for which this lower bound is met with equality, in which case we say that there is a duality between secure omniscience and secret key agreement. We show that this duality holds in the case of certain finite linear source (FLS) models, such as two-terminal FLS models and pairwise independent network models on trees with a linear wiretapper. Duality also holds for any FLS model in which $\wskc$ is achieved by a perfect linear secret key agreement scheme. We conjecture that the duality in fact holds unconditionally for any FLS model. On the negative side, we give an example of a (non-FLS) source model for which duality does not hold if we limit ourselves to communication-for-omniscience protocols with at most two (interactive) communications.  We also address the secure function computation problem and explore the connection between the minimum leakage rate for computing a function and the wiretap secret key capacity.
  
%   Finally, we demonstrate the usefulness of our lower bound on $\rl$ by using it to derive equivalent conditions for the positivity of $\wskc$ in the multiterminal model. This extends a recent result of Gohari, G\"{u}nl\"{u} and Kramer (2020) obtained for the two-user setting.
  
   
%   In this paper, we study the problem of secret key generation through an omniscience achieving communication that minimizes the 
%   leakage rate to a wiretapper who has side information in the setting of multiterminal source model.  We explore this problem by deriving a lower bound on the wiretap secret key capacity $\wskc$ in terms of the minimum leakage rate for omniscience, $\rl$. 
%   %The former quantity is defined to be the maximum secret key rate achievable, and the latter one is defined as the minimum possible leakage rate about the source through an omniscience scheme to a wiretapper. 
%   The main focus of our work is the characterization of the sources for which the lower bound holds with equality \textemdash it is referred to as a duality between secure omniscience and wiretap secret key agreement. For general source models, we show that duality need not hold if we limit to the communication protocols with at most two (interactive) communications. In the case when there is no restriction on the number of communications, whether the duality holds or not is still unknown. However, we resolve this question affirmatively for two-user finite linear sources (FLS) and pairwise independent networks (PIN) defined on trees, a subclass of FLS. Moreover, for these sources, we give a single-letter expression for $\wskc$. Furthermore, in the direction of proving the conjecture that duality holds for all FLS, we show that if $\wskc$ is achieved by a \emph{perfect} secret key agreement scheme for FLS then the duality must hold. All these results mount up the evidence in favor of the conjecture on FLS. Moreover, we demonstrate the usefulness of our lower bound on $\wskc$ in terms of $\rl$ by deriving some equivalent conditions on the positivity of secret key capacity for multiterminal source model. Our result indeed extends the work of Gohari, G\"{u}nl\"{u} and Kramer in two-user case.
%%========================================


%%========================================
% \leavevmode
% \\
% \\
% \\
% \\
% \\
\section{Introduction}
\label{introduction}

AutoML is the process by which machine learning models are built automatically for a new dataset. Given a dataset, AutoML systems perform a search over valid data transformations and learners, along with hyper-parameter optimization for each learner~\cite{VolcanoML}. Choosing the transformations and learners over which to search is our focus.
A significant number of systems mine from prior runs of pipelines over a set of datasets to choose transformers and learners that are effective with different types of datasets (e.g. \cite{NEURIPS2018_b59a51a3}, \cite{10.14778/3415478.3415542}, \cite{autosklearn}). Thus, they build a database by actually running different pipelines with a diverse set of datasets to estimate the accuracy of potential pipelines. Hence, they can be used to effectively reduce the search space. A new dataset, based on a set of features (meta-features) is then matched to this database to find the most plausible candidates for both learner selection and hyper-parameter tuning. This process of choosing starting points in the search space is called meta-learning for the cold start problem.  

Other meta-learning approaches include mining existing data science code and their associated datasets to learn from human expertise. The AL~\cite{al} system mined existing Kaggle notebooks using dynamic analysis, i.e., actually running the scripts, and showed that such a system has promise.  However, this meta-learning approach does not scale because it is onerous to execute a large number of pipeline scripts on datasets, preprocessing datasets is never trivial, and older scripts cease to run at all as software evolves. It is not surprising that AL therefore performed dynamic analysis on just nine datasets.

Our system, {\sysname}, provides a scalable meta-learning approach to leverage human expertise, using static analysis to mine pipelines from large repositories of scripts. Static analysis has the advantage of scaling to thousands or millions of scripts \cite{graph4code} easily, but lacks the performance data gathered by dynamic analysis. The {\sysname} meta-learning approach guides the learning process by a scalable dataset similarity search, based on dataset embeddings, to find the most similar datasets and the semantics of ML pipelines applied on them.  Many existing systems, such as Auto-Sklearn \cite{autosklearn} and AL \cite{al}, compute a set of meta-features for each dataset. We developed a deep neural network model to generate embeddings at the granularity of a dataset, e.g., a table or CSV file, to capture similarity at the level of an entire dataset rather than relying on a set of meta-features.
 
Because we use static analysis to capture the semantics of the meta-learning process, we have no mechanism to choose the \textbf{best} pipeline from many seen pipelines, unlike the dynamic execution case where one can rely on runtime to choose the best performing pipeline.  Observing that pipelines are basically workflow graphs, we use graph generator neural models to succinctly capture the statically-observed pipelines for a single dataset. In {\sysname}, we formulate learner selection as a graph generation problem to predict optimized pipelines based on pipelines seen in actual notebooks.

%. This formulation enables {\sysname} for effective pruning of the AutoML search space to predict optimized pipelines based on pipelines seen in actual notebooks.}
%We note that increasingly, state-of-the-art performance in AutoML systems is being generated by more complex pipelines such as Directed Acyclic Graphs (DAGs) \cite{piper} rather than the linear pipelines used in earlier systems.  
 
{\sysname} does learner and transformation selection, and hence is a component of an AutoML systems. To evaluate this component, we integrated it into two existing AutoML systems, FLAML \cite{flaml} and Auto-Sklearn \cite{autosklearn}.  
% We evaluate each system with and without {\sysname}.  
We chose FLAML because it does not yet have any meta-learning component for the cold start problem and instead allows user selection of learners and transformers. The authors of FLAML explicitly pointed to the fact that FLAML might benefit from a meta-learning component and pointed to it as a possibility for future work. For FLAML, if mining historical pipelines provides an advantage, we should improve its performance. We also picked Auto-Sklearn as it does have a learner selection component based on meta-features, as described earlier~\cite{autosklearn2}. For Auto-Sklearn, we should at least match performance if our static mining of pipelines can match their extensive database. For context, we also compared {\sysname} with the recent VolcanoML~\cite{VolcanoML}, which provides an efficient decomposition and execution strategy for the AutoML search space. In contrast, {\sysname} prunes the search space using our meta-learning model to perform hyperparameter optimization only for the most promising candidates. 

The contributions of this paper are the following:
\begin{itemize}
    \item Section ~\ref{sec:mining} defines a scalable meta-learning approach based on representation learning of mined ML pipeline semantics and datasets for over 100 datasets and ~11K Python scripts.  
    \newline
    \item Sections~\ref{sec:kgpipGen} formulates AutoML pipeline generation as a graph generation problem. {\sysname} predicts efficiently an optimized ML pipeline for an unseen dataset based on our meta-learning model.  To the best of our knowledge, {\sysname} is the first approach to formulate  AutoML pipeline generation in such a way.
    \newline
    \item Section~\ref{sec:eval} presents a comprehensive evaluation using a large collection of 121 datasets from major AutoML benchmarks and Kaggle. Our experimental results show that {\sysname} outperforms all existing AutoML systems and achieves state-of-the-art results on the majority of these datasets. {\sysname} significantly improves the performance of both FLAML and Auto-Sklearn in classification and regression tasks. We also outperformed AL in 75 out of 77 datasets and VolcanoML in 75  out of 121 datasets, including 44 datasets used only by VolcanoML~\cite{VolcanoML}.  On average, {\sysname} achieves scores that are statistically better than the means of all other systems. 
\end{itemize}


%This approach does not need to apply cleaning or transformation methods to handle different variances among datasets. Moreover, we do not need to deal with complex analysis, such as dynamic code analysis. Thus, our approach proved to be scalable, as discussed in Sections~\ref{sec:mining}.
%%========================================
\section{Related Work}\label{sec:related}
 
The authors in \cite{humphreys2007noncontact} showed that it is possible to extract the PPG signal from the video using a complementary metal-oxide semiconductor camera by illuminating a region of tissue using through external light-emitting diodes at dual-wavelength (760nm and 880nm).  Further, the authors of  \cite{verkruysse2008remote} demonstrated that the PPG signal can be estimated by just using ambient light as a source of illumination along with a simple digital camera.  Further in \cite{poh2011advancements}, the PPG waveform was estimated from the videos recorded using a low-cost webcam. The red, green, and blue channels of the images were decomposed into independent sources using independent component analysis. One of the independent sources was selected to estimate PPG and further calculate HR, and HRV. All these works showed the possibility of extracting PPG signals from the videos and proved the similarity of this signal with the one obtained using a contact device. Further, the authors in \cite{10.1109/CVPR.2013.440} showed that heart rate can be extracted from features from the head as well by capturing the subtle head movements that happen due to blood flow.

%
The authors of \cite{kumar2015distanceppg} proposed a methodology that overcomes a challenge in extracting PPG for people with darker skin tones. The challenge due to slight movement and low lighting conditions during recording a video was also addressed. They implemented the method where PPG signal is extracted from different regions of the face and signal from each region is combined using their weighted average making weights different for different people depending on their skin color. 
%

There are other attempts where authors of \cite{6523142,6909939, 7410772, 7412627} have introduced different methodologies to make algorithms for estimating pulse rate robust to illumination variation and motion of the subjects. The paper \cite{6523142} introduces a chrominance-based method to reduce the effect of motion in estimating pulse rate. The authors of \cite{6909939} used a technique in which face tracking and normalized least square adaptive filtering is used to counter the effects of variations due to illumination and subject movement. 
The paper \cite{7410772} resolves the issue of subject movement by choosing the rectangular ROI's on the face relative to the facial landmarks and facial landmarks are tracked in the video using pose-free facial landmark fitting tracker discussed in \cite{yu2016face} followed by the removal of noise due to illumination to extract noise-free PPG signal for estimating pulse rate. 

Recently, the use of machine learning in the prediction of health parameters have gained attention. The paper \cite{osman2015supervised} used a supervised learning methodology to predict the pulse rate from the videos taken from any off-the-shelf camera. Their model showed the possibility of using machine learning methods to estimate the pulse rate. However, our method outperforms their results when the root mean squared error of the predicted pulse rate is compared. The authors in \cite{hsu2017deep} proposed a deep learning methodology to predict the pulse rate from the facial videos. The researchers trained a convolutional neural network (CNN) on the images generated using Short-Time Fourier Transform (STFT) applied on the R, G, \& B channels from the facial region of interests.
The authors of \cite{osman2015supervised, hsu2017deep} only predicted pulse rate, and we extended our work in predicting variance in the pulse rate measurements as well.

All the related work discussed above utilizes filtering and digital signal processing to extract PPG signals from the video which is further used to estimate the PR and PRV.  %
The method proposed in \cite{kumar2015distanceppg} is person dependent since the weights will be different for people with different skin tone. In contrast, we propose a deep learning model to predict the PR which is independent of the person who is being trained. Thus, the model would work even if there is no prior training model built for that individual and hence, making our model robust. 

%
%%========================================
%\vspace{-1em}
\section{Preliminaries}
%\vspace{-1em}
\label{sec:preliminaries}
Datasets $X, X' \in \mathcal{X}$ are neighbors if they differ by no more than one datapoint -- i.e., $X \simeq X'$ if $d(X, X') \leq 1$. We will define $d(\cdot)$ to be the number of coordinates that differ between two datasets of the same size $n$: $d(X, Y) = \#\{i \in [n]: X_i \neq Y_i  \}$.

We use $||\cdot||$ to denote the radius of the smallest Euclidean ball that contains the input set, e.g. $||\mathcal{X}|| = \sup_{x \in \mathcal{X}} ||x||$.

The parameter $\phi$ denotes the privacy parameters associated with a mechanism (e.g. noise level, regularization). $\mathcal{M}_{\phi}$ is a mechanism parameterized by $\phi$.
For mechanisms with continuous output space, we will take $\text{Pr}[\mathcal{M}(X) = y]$ to be the probability density function of $\mathcal{M}(X)$ at $y$.



    \begin{definition}[Differential privacy \citep{dwork2006calibrating}] \label{def:dp}
        Fix $\epsilon, \delta \geq 0$. 
A randomized algorithm $\mathcal{M}: \mathcal{X} \rightarrow \mathcal{S}$ satisfies $(\epsilon, \delta)$-DP if for all neighboring datasets $X \simeq X'$ and for all measurable sets $S \subset \mathcal{S}$, 
            \[\text{Pr}\big[\mathcal{M}(X) \in S\big] \leq e^{\epsilon}\text{Pr}\big[\mathcal{M}(X') \in S\big] + \delta.\]
    \end{definition}
%\vspace{-3mm}
% We now define \emph{data-dependent} differential privacy that conditions on an input dataset $X$. 

% \begin{definition}[Data-dependent privacy\cite{papernot2018scalable}]\vspace{-1mm}
% \label{def:data_dep_dp}
% Suppose we have $\delta > 0$ and a function $\epsilon: \mathcal{X} \rightarrow \mathbb{R}$. We say that mechanism $\mathcal{M}$ satisfies ($\epsilon(X), \delta$) data-dependent DP for dataset $X$ if for all possible output sets $S$ and neighboring datasets $X'$,
% \begin{align*}
%     \text{Pr}\big[\mathcal{M}(X) \in S\big] &\leq e^{\epsilon(X)}\text{Pr}\big[\mathcal{M}(X') \in S\big] + \delta, \\
%       \text{Pr}\big[\mathcal{M}(X') \in S\big] &\leq e^{\epsilon(X)}\text{Pr}\big[\mathcal{M}(X) \in S\big] + \delta.
% \end{align*}
% \end{definition}

%\subsection{Additive Noise Mechanisms}
Suppose we wish to privately release the output of a real-valued function $f: \mathcal{X} \rightarrow \mathcal{R}$. We can do so by calculating the \emph{global sensitivity} $\Delta_{GS}$, calibrating the noise scale to the global sensitivity and then adding sampled noise to the output.



\begin{definition}[Local / Global sensitivity]
The local $\ell_*$-sensitivity of a function $f$ is defined as $\Delta_{LS}(X) = \max\limits_{X \simeq X'} || f(X) - f(X') ||_* $ and the global sensitivity of $f$ is $\Delta_{GS} = \sup_X \Delta_{LS}(X)$.
% The local $\ell_*$-sensitivity of a function $f: \cX \to \mathbb{R}^d$ is defined as $\Delta_{LS}(X) = \max\limits_{X \simeq X'} || f(X) - f(X') ||_* $ and the global sensitivity of $f$ is $\Delta_{GS} = \sup_X \Delta_{LS}(X)$.
\end{definition}
%\vspace{-2mm}
% The choice of $\ell_*$ depends on which kind of noise we use, e.g., $\ell_2$-norm is used for Gaussian noise.

\begin{comment}

\begin{definition}[Global sensitivity]
The global $\ell_*$-sensitivity of a function $f: \mathcal{X} \rightarrow \mathcal{R}^d$ is defined as
\begin{align*}
    \Delta_{GS} &= \max\limits_{X, X' \in \mathcal{X}:X \simeq X'} || f(X) - f(X') ||_*. 
\end{align*}
\end{definition}
\begin{definition}[Laplace mechanism]
The Laplace mechanism $\mathcal{M}: \mathcal{X} \rightarrow \mathbb{R}$ applied to a function $f$ is given as
\begin{align*}
    \mathcal{M}(X) &= f(X) + \text{Lap}\left(b \right).
\end{align*}
\end{definition}
\begin{theorem}
Suppose the function $f: \mathcal{X} \rightarrow \mathbb{R}$ has global $\ell_1$-sensitivity $\Delta_f$. Then the Laplace mechanism satisfies $\epsilon$-differential privacy with noise parameter $b = \Delta_f/\epsilon$.	 \vspace{-2mm}
\end{theorem}

\begin{definition}[Gaussian mechanism]
The Gaussian mechanism $\mathcal{M}: \mathcal{X} \rightarrow \mathbb{R}$ applied to a function $f$ is given as
\begin{align*}
    \mathcal{M}(X) &= f(X) + \mathcal{N}(0, \sigma^2).
\end{align*}
\end{definition}
\begin{theorem}
Suppose the function $f: \mathcal{X} \rightarrow \mathbb{R}$ has global $\ell_2$-sensitivity $\Delta_f$. Then the Gaussian mechanism satisfies $(\epsilon, \delta)$-differential privacy with noise parameter $\sigma = \Delta_f\sqrt{2 \log(1.25/\delta)}/\epsilon$.
\end{theorem}
Both the Laplace and Gaussian mechanisms generalize easily to releasing the output of a $d$-dimensional function $f$ by adding i.i.d. noise to each coordinate.
\begin{definition}[Local sensitivity]
The local $\ell_*$-sensitivity of a function $f: \mathcal{X} \rightarrow \mathbb{R}^d$ is defined as
\begin{align*}
    \Delta_{LS}(X) &= \max\limits_{X \simeq X'} || f(X) - f(X') ||_*. 
\end{align*}
\end{definition}
\end{comment}
% \todo{Define Global sensitivity}
%define global/local sensitivity
%Laplace, Gaussian mech
% explain how related to PTR and its generalization

%\vspace{-0.em}
\subsection{Propose-Test-Release}
%\vspace{-0.5em}
Calibrating the noise level to the local sensitivity $\Delta_{LS}(X)$ of a function would allow us to add less noise and therefore achieve higher utility for releasing private queries. However, the local sensitivity is a data-dependent function and na\"ively calibrating the noise level to $\Delta_{LS}(X)$ will not satisfy DP.

PTR resolves this issue in a three-step procedure: \textbf{propose} a bound on the local sensitivity, privately \textbf{test} that the bound is valid (with high probability), and if so calibrate noise according to the bound and \textbf{release} the output.

% \begin{figure}[t]
% \vspace{-1em}
% \centering
% \resizebox{0.95\columnwidth}{!}{%
% \begin{minipage}{0.50\textwidth}
% \begin{algorithm}[H]
% \caption{Propose-Test-Release \cite{dwork2009differential}}
% \label{alg:classic_ptr}
% \begin{algorithmic}[1]
% \STATE{\textbf{Input}: Dataset $X$; privacy parameters $\epsilon,\delta$; proposed bound $\beta$ on $\Delta_{LS}(X)$; query function $f: \mathcal{X} \rightarrow \mathbb{R}$.}
% \STATE{\textbf{Output}: $f^P(X)$ or $\perp$.}
% \STATE{Compute the distance $\gamma(X)$ to the nearest dataset $X''$ such that $ \Delta_{LS}(X'')> \beta$:
% $\gamma(X) = \min\limits_{X''} \{ \text{dist}(X, X''): \Delta_{LS}(X'')> \beta \}$.}
% \STATE{Privately release $\gamma^P(X) = \gamma(X) + \text{Lap}\left(\frac{1}{\epsilon}\right)$.}
% \IF{$\gamma^P(X) > \dfrac{\log(1/\delta)}{\epsilon}$}\vspace{-1pt}
% \STATE{Release $f^P(X) = f(X) + \text{Lap}\left(\frac{\beta}{\epsilon}\right)$.}\vspace{-2pt}
% \ELSE
% \STATE{Output $\perp$.}\vspace{-1pt}
% \ENDIF
% \end{algorithmic}
% \end{algorithm}
% \end{minipage}
% \quad
% \begin{minipage}{0.46\textwidth}
% \begin{algorithm}[H]
% \caption{Generalized PTR}
% \label{alg:gen_ptr}
% \begin{algorithmic}[1]
% \STATE{{Input}: Proposed~parameter~$\phi$;~privacy~parameters~$\epsilon,  \hat{\epsilon}, \hat{\delta}$; dataset $X$;\blue{an $(\epsilon,\delta)$~DP test $\cT$; ~data-dependent~DP~function~$\epsilon_{\phi}(\cdot, \hat{\delta})$;~mechanism~$\mathcal{M}_{\phi}$.}}
% \STATE{\textbf{Output}: 
% $\mathcal{M}_{\phi}(X)$ or $\perp$.}
% \STATE{Let $\cT$ privately test if $\epsilon_\phi(X,\hat{\delta}) \leq \hat{\epsilon}$.}% with privacy limit $(\hat{\epsilon}, \hat{\delta})$ }.
% \IF{the test $\cT$ passes}
% \vspace{1pt}
% \STATE{Run $\theta = \mathcal{M}_{\phi}(X)$ and output $\theta$.}\vspace{2pt}

% \ELSE \vspace{2pt}

% \STATE{Output $\perp$.}\vspace{2pt}

% \ENDIF
% \end{algorithmic}
% \end{algorithm}
% \end{minipage}
% }
% \vspace{-1em}
% \end{figure}


% \begin{theorem} 
% Algorithm~\ref{alg:classic_ptr} satisfies ($2 \epsilon, \delta$)-DP.
% \cite{dwork2009differential}
% \end{theorem}

PTR privately computes the distance $\cD_{\beta}(X)$ between the input dataset $X$ and the nearest dataset $X''$ whose local sensitivity exceeds the proposed bound $\beta$:
\begin{align*}
    \cD_{\beta}(X) = \min\limits_{X''} \{ d(X, X''): \Delta_{LS}(X'')> \beta \}.
\end{align*}
%\vspace{-.8em}
% The $\epsilon$-DP "test" fails (with probability $\delta$) if PTR decides to release $f^P(X)$ when $\gamma(X) = 0$, i.e. when dataset $X$ has local sensitivity greater than $\beta$.

\begin{figure}[H]
\vspace{-1.4em}
\centering
% \resizebox{0.95\columnwidth}{!}{%
% \begin{minipage}{0.54\textwidth}
\begin{algorithm}[H]
\caption{Propose-Test-Release \citep{dwork2009differential}}
\label{alg:classic_ptr}
\begin{algorithmic}[1]
\STATE{\textbf{Input}: Dataset $X$; privacy parameters $\epsilon,\delta$; proposed bound $\beta$ on $\Delta_{LS}(X)$; query function $f: \mathcal{X} \rightarrow \mathbb{R}$.}
% \STATE{\textbf{Output}: $f^P(X)$ or $\perp$.}
% \STATE{Compute the distance $\gamma(X)$ to the nearest dataset $X''$ such that $ \Delta_{LS}(X'')> \beta$.}
% \STATE{Privately release $\gamma^P = \gamma(X; \beta) + \text{Lap}\left(\frac{1}{\epsilon}\right)$.}

\STATE{\textbf{if} $\cD_{\beta}(X) + \text{Lap}\left(\frac{1}{\epsilon}\right) \leq \frac{\log(1/\delta)}{\epsilon}$ \textbf{then} output $\perp$,}
\STATE{\textbf{else} release $f(X) + \text{Lap}\left(\frac{\beta}{\epsilon}\right)$.}
\end{algorithmic}
\end{algorithm}
\end{figure}
%\vspace{-.8em}
\begin{theorem} 
Algorithm~\ref{alg:classic_ptr} satisfies ($2 \epsilon, \delta$)-DP.
\citep{dwork2009differential}
\end{theorem}
%\vspace{-1em}
Rather than proposing an arbitrary threshold $\beta$, one can also privately release an upper bound of the local sensitivity and calibrate noise according to this upper bound. This was used for node DP in graph statistics \citep{kasiviswanathan2013analyzing}, and for fitting topic models using spectral methods \citep{decarolis2020end}.

%This gives a more efficient alternative and avoids the need to propose $\beta$. This variant is  the local sensitivity itself has a global sensitivity.

%there exist other variants of PTR --- e.g., compute a differentially private upper bound of the local sensitivity and calibrate noise according to this upper bound. This type of PTR requires a global sensitivity of the local sensitivity. We refer readers to the excellent summary of PTR in  section 3 of \citet{vadhan2017complexity}.

% We may mention other types of PTR: propose and release
% There are other variants of PTR... 
% \vspace{-1mm}
% \subsection{Motivation}
% \vspace{-1mm}
% Why do we want to generalize PTR beyond noise-adding mechanisms? For other mechanisms, the local sensitivity either does not exist or is only defined for specific data-dependent quantities (e.g., the sensitivity of the score function in the exponential mechanism) rather than the mechanism's output. We give a concrete example below. 

%In this section, we give a concrete example to demonstrate this limitation and motivate our generalization.
%This section discusses a few limitations of PTR approaches that motivated our work.
%Let us first ask, ``is PTR a general framework applicable to any mechanism with a data-dependent analysis?'' If so, we could explore other less costly approaches to privately test the local sensitivity.

%However, the answer is unfortunately ``no''.  The reasons are twofold. First, the framework above applies only to ``noise-adding'' mechanisms --- where we have a well-defined local sensitivity (of the output), and the noise scale is calibrated according to that. For other non-noise-adding mechanisms, the local sensitivity either does not exist or is only defined for specific data-dependent quantities (e.g., the sensitivity of the score function in the exponential mechanism) rather than the mechanism's output. Consider the difficulties of applying PTR to the following example.


% \begin{example}[Private posterior sampling]\label{exp: posterior}
% Let $\cM: \cX\times \cY \to \Theta $ be a private posterior sampling   mechanism~\citep{minami2016differential,wang2015privacy,gopi2022private} for approximately minimizing $F_{X}(\theta)$. % for linear regression problem, i.e., $\min_{\theta} \frac{1}{2}||y-X\theta||^2 + \lambda ||\theta||^2$. 
% $\cM$ samples $\theta \sim P(\theta)\propto e^{-\gamma(F_X(\theta)+ 0.5\lambda ||\theta||^2)}$ with parameters $\gamma, \lambda$. $\gamma,\lambda$ cannot be appropriately chosen for this mechanism to satisfy DP without going through a sensitivity calculation of $\arg\min F_X(\theta)$. In fact, the global and local sensitivity of the minimizer is unbounded even in linear regression problems, i.e., when $F_X(\theta) = \frac{1}{2}||y-X\theta||^2.$ 
% %The local sensitivity $\Delta:=||P_{X,y}(\theta)-P_{X', y'}(\theta)||$ is not well-defined  for the sampling algorithm, thus the standard PTR is not applicable.  
% \end{example}
% Output perturbation algorithms do work for the above problem when we regularize, but they are known to be suboptimal in theory and in practice \cite{chaudhuri2011differentially}.% do not achieve the level of utility in theory and in practice when comparing to posterior sampling. 
 

% Moreover, even in the cases of noise-adding mechanisms where PTR seems to be applicable, it does not lead to a tight privacy guarantee. Specifically, by an example of privacy amplification by post-processing (Example~\ref{exp: binary_vote} in the appendix), we demonstrate that the local sensitivity does not capture all sufficient statistics for data-dependent privacy analysis and thus is loose.

% Instead of identifying sufficient statistics of each mechanism, we develop a unified framework --- generalized PTR, offering the flexibility for any mechanism to exploit data-dependent quantities.

% \textbf{On data-dependent DP losses.} In addition to the above, there has been an increasing list of empirical DP work that fix the parameters of a randomized algorithm while reporting the resulting data-dependent DP losses $\epsilon(\text{Data})$ after running on a specific dataset \citep{ligett2017accuracy,papernot2018scalable,zhu2020private, feldman2021individual}. The data-dependent DP losses are often smaller than the worst-case DP losses, but technically speaking, these algorithms are not formally DP with DP guarantees any smaller than that of the worst-case. In addition, the data-dependent DP losses themselves are sensitive, and thus cannot be reported. A typical solution is to privately release $\epsilon(\text{Data})$, but it still does not satisfy DP as this would require a prescribed $(\epsilon,\delta)$-DP parameter to be satisfied for all input datasets. Part of our contribution is to resolve this conundrum by showing that a simple post-processing step of the privately released upper bound of $\epsilon(\text{Data})$ gives a formal DP algorithm.
%\yq{Shall we combine this part with the related work section?}

%Instead,  exploits data-dependent quantities by first privately choosing $\gamma, \lambda$ adapted to the dataset and then applying posterior sampling with the sanitized parameters.  

%%========================================
We briefly recall the framework of statistical inference via empirical risk minimization.
Let $(\bbZ, \calZ)$ be a measurable space.
Let $Z \in \bbZ$ be a random element following some unknown distribution $\Prob$.
Consider a parametric family of distributions $\calP_\Theta := \{P_\theta: \theta \in \Theta \subset \reals^d\}$ which may or may not contain $\Prob$.
We are interested in finding the parameter $\theta_\star$ so that the model $P_{\theta_\star}$ best approximates the underlying distribution $\Prob$.
For this purpose, we choose a \emph{loss function} $\score$ and minimize the \emph{population risk} $\risk(\theta) := \Expect_{Z \sim \Prob}[\score(\theta; Z)]$.
Throughout this paper, we assume that
\begin{align*}
     \theta_\star = \argmin_{\theta \in \Theta} L(\theta)
\end{align*}
uniquely exists and satisfies $\theta_\star \in \text{int}(\Theta)$, $\nabla_\theta L(\theta_\star) = 0$, and $\nabla_\theta^2 L(\theta_\star) \succ 0$.

\myparagraph{Consistent loss function}
We focus on loss functions that are consistent in the following sense.

\begin{customasmp}{0}\label{asmp:proper_loss}
    When the model is \emph{well-specified}, i.e., there exists $\theta_0 \in \Theta$ such that $\Prob = P_{\theta_0}$, it holds that $\theta_0 = \theta_\star$.
    We say such a loss function is \emph{consistent}.
\end{customasmp}

In the statistics literature, such loss functions are known as proper scoring rules \citep{dawid2016scoring}.
We give below two popular choices of consistent loss functions.

\begin{example}[Maximum likelihood estimation]
    A widely used loss function in statistical machine learning is the negative log-likelihood $\score(\theta; z) := -\log{p_\theta(z)}$ where $p_\theta$ is the probability mass/density function for the discrete/continuous case.
    When $\Prob = P_{\theta_0}$ for some $\theta_0 \in \Theta$,
    we have $L(\theta) = \Expect[-\log{p_\theta(Z)}] = \kl(p_{\theta_0} \Vert p_\theta) - \Expect[\log{p_{\theta_0}(Z)}]$ where $\kl$ is the Kullback-Leibler divergence.
    As a result, $\theta_0 \in \argmin_{\theta \in \Theta} \kl(p_{\theta_0} \Vert p_\theta) = \argmin_{\theta \in \Theta} L(\theta)$.
    Moreover, if there is no $\theta$ such that $p_\theta \txtover{a.s.}{=} p_{\theta_0}$, then $\theta_0$ is the unique minimizer of $L$.
    We give in \Cref{tab:glms} a few examples from the class of generalized linear models (GLMs) proposed by \citet{nelder1972generalized}.
\end{example}

\begin{example}[Score matching estimation]
    Another important example appears in \emph{score matching} \citep{hyvarinen2005estimation}.
    Let $\bbZ = \reals^\tau$.
    Assume that $\Prob$ and $P_\theta$ have densities $p$ and $p_\theta$ w.r.t the Lebesgue measure, respectively.
    Let $p_\theta(z) = q_\theta(z) / \Lambda(\theta)$ where $\Lambda(\theta)$ is an unknown normalizing constant. We can choose the loss
    \begin{align*}
        \score(\theta; z) := \Delta_z \log{q_\theta(z)} + \frac12 \norm{\nabla_z \log{q_\theta(z)}}^2 + \text{const}.
    \end{align*}
    Here $\Delta_z := \sum_{k=1}^p \partial^2/\partial z_k^2$ is the Laplace operator.
    Since \cite[Thm.~1]{hyvarinen2005estimation}
    \begin{align*}
        L(\theta) = \frac12 \Expect\left[ \norm{\nabla_z q_\theta(z) - \nabla_z p(z)}^2 \right],
    \end{align*}
    we have, when $p = p_{\theta_0}$, that $\theta_0 \in \argmin_{\theta \in \Theta} L(\theta)$.
    In fact, when $q_\theta > 0$ and there is no $\theta$ such that $p_\theta \txtover{a.s.}{=} p_{\theta_0}$, the true parameter $\theta_0$ is the unique minimizer of $L$ \cite[Thm.~2]{hyvarinen2005estimation}.
\end{example}

\myparagraph{Empirical risk minimization}
Assume now that we have an i.i.d.~sample $\{Z_i\}_{i=1}^n$ from $\Prob$.
To learn the parameter $\theta_\star$ from the data, we minimize the empirical risk to obtain the \emph{empirical risk minimizer}
\begin{align*}
    \theta_n \in \argmin_{\theta \in \Theta} \left[ L_n(\theta) := \frac1n \sum_{i=1}^n \score(\theta; Z_i) \right].
\end{align*}
This applies to both maximum likelihood estimation and score matching estimation. 
In \Cref{sec:main_results}, we will prove that, with high probability, the estimator $\theta_n$ exists and is unique under a generalized self-concordance assumption.

\begin{figure}
    \centering
    \includegraphics[width=0.45\textwidth]{graphs/logistic-dikin} %0.4
    \caption{Dikin ellipsoid and Euclidean ball.}
    \label{fig:logistic_dikin}
\end{figure}

\myparagraph{Confidence set}
In statistical inference, it is of great interest to quantify the uncertainty in the estimator $\theta_n$.
In classical asymptotic theory, this is achieved by constructing an asymptotic confidence set.
We review here two commonly used ones, assuming the model is well-specified.
We start with the \emph{Wald confidence set}.
It holds that $n(\theta_n - \theta_\star)^\top H_n(\theta_n) (\theta_n - \theta_\star) \rightarrow_d \chi_d^2$, where $H_n(\theta) := \nabla^2 L_n(\theta)$.
Hence, one may consider a confidence set $\{\theta: n(\theta_n - \theta)^\top H_n(\theta_n) (\theta_n - \theta) \le q_{\chi_d^2}(\delta) \}$ where $q_{\chi_d^2}(\delta)$ is the upper $\delta$-quantile of $\chi_d^2$.
The other is the \emph{likelihood-ratio (LR) confidence set} constructed from the limit $2n [L_n(\theta_\star) - L_n(\theta_n)] \rightarrow_d \chi_d^2$, which is known as the Wilks' theorem \citep{wilks1938large}.
These confidence sets enjoy two merits: 1) their shapes are an ellipsoid (known as the \emph{Dikin ellipsoid}) which is adapted to the optimization landscape induced by the population risk; 2) they are asymptotically valid, i.e., their coverages are exactly $1 - \delta$ as $n \rightarrow \infty$.
However, due to their asymptotic nature, it is unclear how large $n$ should be in order for it to be valid.

Non-asymptotic theory usually focuses on developing finite-sample bounds for the \emph{excess risk}, i.e., $\Prob(L(\theta_n) - L(\theta_\star) \le C_n(\delta)) \ge 1 - \delta$.
To obtain a confidence set, one may assume that the population risk is twice continuously differentiable and $\lambda$-strongly convex.
Consequently, we have $\lambda \norm{\theta_n - \theta_\star}_2^2 / 2 \le L(\theta_n) - L(\theta_\star)$ and thus we can consider the confidence set $\calC_{\text{finite}, n}(\delta) := \{\theta: \norm{\theta_n - \theta}_2^2 \le 2C_n(\delta)/\lambda\}$.
Since it originates from a finite-sample bound, it is valid for fixed $n$, i.e., $\Prob(\theta_\star \in \calC_{\text{finite}, n}(\delta)) \ge 1 - \delta$ for all $n$; however, it is usually conservative, meaning that the coverage is strictly larger than $1 - \delta$.
Another drawback is that its shape is a Euclidean ball which remains the same no matter which loss function is chosen.
We illustrate this phenomenon in \Cref{fig:logistic_dikin}.
Note that a similar observation has also been made in the bandit literature \citep{faury2020improved}.

We are interested in developing finite-sample confidence sets.
However, instead of using excess risk bounds and strong convexity, we construct in \Cref{sec:main_results} the Wald and LR confidence sets in a non-asymptotic fashion, under a generalized self-concordance condition.
These confidence sets have the same shape as their asymptotic counterparts while maintaining validity for fixed $n$.
These new results are achieved by characterizing the critical sample size enough to enter the asymptotic regime.

%%========================================
%%========================================
\begin{figure}[t!]
    \centering
    \includegraphics[width=0.6\linewidth]{figures/single_demo.png}
    %--------------------
    \caption{Illustration of the generalization problem in Sec.~\ref{subsec:limit-few}. 
Given the single demonstration (in black), the retrieved trajectory (in blue) is significantly displaced w.r.t. the excepted trajectory (in green).}
    \label{fig:single_demo}
    %--------------------
    \vspace{-0.15cm}
\end{figure}
%========================================
%==============================
\section{Current Limitations}\label{sec:limitation}
In this section, we discuss three limitations of the current skill coordination methods that are built upon the TP-HSMM model in Sec.~\ref{subsec:tp-hsmm}. 
These limitations serve as the main motivation for the proposed solution in the subsequent section. 


%==============================
\subsection{Generalization from Very Few Demos}\label{subsec:limit-few}
As described in Sec.~\ref{subsec:tp-hsmm}, the TP-HSMM of a skill is computed as a parameterized average of \emph{all} demos. 
However, when only one or very few demos are provided (compared with the number of frames), 
the resulting model can have over-confident GMMs at different stages of the skill. 
One direct consequence is that the generalization to new scenarios is heavily biased towards this one demo.
As shown in Fig.~\ref{fig:single_demo} for a picking skill, the retrieved trajectory ends up between the new object pose and the demonstrated pose.
Even though this can be mitigated by adding new demos, 
it is still desirable to learn a reasonable model with very few demos,
especially for simple skills such as picking and dropping within free spaces.

%========================================
\begin{figure}[t!]
    \centering
    \includegraphics[width=0.45\linewidth]{figures/branch_gaussian_2p_box.png}
    \includegraphics[width=0.45\linewidth]{figures/branch_svm_2p_box.png}
    %--------------------
    \caption{
Comparison between branch selection based on the Gaussian preconditions (left) 
and the proposed selector (right), 
    for the bin-picking skill with 5 branches (four sides and the center). 
    Note that different colors indicate the predicted branches at that sample point (in dots),
 while the \emph{projected} training data are indicated as diamonds (left).}
    \label{fig:branch_compare}
    %--------------------
    \vspace{-0.15cm}
\end{figure}
%========================================

%==============================
\subsection{Multiple Branches for One Skill}\label{subsec:limit-branch}
Often there are multiple ways of executing the same skill under different scenarios (called \emph{branches}).
For instance, as shown in the experiment, there are five different ways of picking objects from a bin, i.e., approaching with different angles depending on the distances to each boundary.  
Our earlier work~\cite{rozo2020learning} proposed a learning algorithm for TP-HSMM with multiple branches, 
and moreover a precondition model that chooses the best branch based on the first GMMs of all branches.
However, this model requires a large number of demos to cover the area of interest 
and does not generalizes well to new scenarios.
This is mainly due to the usage of Gaussian clustering over few samples in high dimensions. 
Fig.~\ref{fig:branch_compare} illustrates one example of the bin-picking skill described earlier with 5 branches.
It can be seen that the precondition model from~\cite{rozo2020learning} can yield bad choices even close to the demonstrated scenes. 

%==============================
\subsection{Manual Conditioning in Task Graph}\label{subsec:manual-specification}
Lastly, as mentioned in Sec.~\ref{sec:introduction}, complex manipulation tasks often contain various sequences of skills to account for different scenarios. 
A high-level abstraction of such relations is often referred as task networks~\cite{hayes2016autonomously}.
In such networks, a valid plan evolves by transition from one skill to another until the task is solved. 
As shown in Fig.~\ref{fig:framework}, different sequences can share some common skills and one skill may appear several times in the same sequence. 
Even though the graph structure that encapsulates these sequences can be sketched easily, 
the \emph{conditions} on these transitions are particularly difficult and tedious to specify manually. 
Often it is required to modify these conditions whenever the workspace or the goal specification is changed. 

%%========================================
%!TEX root = main.tex

\parindent 20pt

\section{Appendix---Geometric Chung-Lu Directed Graph Model is Well-defined} \label{apdx:model}

Let us state the problem in a slightly more general setting. Fix $n=2m$, where $m$ is some natural number. Suppose that for each pair $i, j \in [n]$ we have $a_{ij}=a_{ji} \in \R_+$ if $i\leq m$, $j>m$, $j\neq i+m$, and we have $a_{ij}=0$ otherwise, that is, if $i,j\in [m]$ or $i,j \in [2m] \setminus [m]$ or $j=i+m$. The $a_{ij}$ elements taken together form a matrix $\textbf{A}$, which is symmetric. Finally, for each $i \in [n]$ we have $b_i \in \R_+ \cup \{0\}$, and 
\begin{equation}\label{eq:assumption}
\sum_{i=1}^{m}b_i=\sum_{i=m+1}^{n}b_i>0.
\end{equation}

\medskip

In our application, $m=|V|$ is the number of nodes in a non-empty directed graph $G=(V,E)$, and elements of vector $\mathbf{b} = (b_i)_{i\in[n]}$ correspond to the degree distribution of the graph: for $i\in [m]$, $b_i$ is the out-degree of $v_i$ (that is, $b_i=w_i^{out}$), and for $i\in [2m] \setminus [m]$, $b_i$ is the in-degree of $v_{i-m}$ (that is, $b_i=w_{i-m}^{in}$). The assumption that $\sum_{i=1}^{m}b_i=\sum_{i=m+1}^{n}b_i>0$ is satisfied as the total in-degree is equal to the total out-degree, and the graph is not empty. Positive elements of matrix $\textbf{A}$ satisfy $a_{i,j+m}=a_{j+m,i} = g(d_{i,j}) \in (0,1]$ for $i,j\in[m], i\neq j$, and correspond to the distances between embeddings of the corresponding nodes $v_i$ and $v_j$. The case $i=j$ is excluded as indices $i$ and $i+m$ correspond to the same node in the original directed graph and so $a_{i,i+m}=a_{i+m,i} = g(d_{i,i}) = 0$. Finally, since isolated nodes may be ignored, we may assume that $w_i^{out}+w_i^{in} > 0$, that is,
\begin{equation}\label{eq:non-degenerate}
b_i + b_{i+m} > 0 \qquad \text{ for all } i \in [m].
\end{equation}

Our goal is to investigate if there is a solution, $x_i \in \R_+\cup\{0\}$ for $i \in [n]$, of the following system of equations:
\begin{equation}\label{eq:system}
b_i = x_i \sum_{j=1}^na_{ij}x_j \qquad \text{ for all } i\in[n].
\end{equation}
If there is one, then is this solution unique? The solution to~(\ref{eq:system}) will yield the solution to our problem: for $i \in [m]$, $x_i^{out} = x_i$ and $x_i^{in}=x_{i+m}$.

\medskip

The $m=2$ case is a degenerate case that exhibits a different behaviour but it is easy to investigate. In this case, by assumption~(\ref{eq:assumption}), $b_1 = b_4 = x_1 x_4 a_{12}$ and $b_2 = b_3 = x_2 x_3 a_{21}$. There are infinite number of solutions, each of them being of the form $(x_1, x_2, x_3, x_4) = (s, t, b_2/(a_{12}t), b_1/(a_{12}s))$ for some $t, s \in \R_+$. Having said that, in our application, all of these solutions yield the same random graph with the following distribution: $p_{12} = x_1 x_4 a_{12} = b_1 = w^{out}_1$ and $p_{21} = x_2 x_3 a_{21} = b_2 = w^{out}_2$.

\medskip

Suppose now that $m \ge 3$. We will show that the desired solution of~(\ref{eq:system}) exists if
\begin{equation}\label{eq:condition1}
\sum_{i=m+1}^nb_i > b_j + b_{j+m} \quad \text{for} \quad j\in[m],
\end{equation}
and
\begin{equation}\label{eq:condition2}
\sum_{i=1}^mb_i > b_j + b_{j-m} \quad \text{for} \quad j\in[2m] \setminus [m]
\end{equation}
(also recall that by assumption~(\ref{eq:assumption}), $\sum_{i=1}^{m}b_i=\sum_{i=m+1}^{n}b_i>0$, and by assumption~(\ref{eq:non-degenerate}), $b_i+b_{i+m}>0$ for all $i \in [m]$).
In other words, the condition is that the in-degree of any node $v_i$ is smaller than the sum of out-degrees of nodes other than $v_i$, and the out-degree of $v_i$ is smaller than the sum of in-degrees of nodes other than $v_i$. This is a very mild condition that holds in our application. Indeed, properties~(\ref{eq:condition1})--(\ref{eq:condition2}) with non-strict inequalities are satisfied for \emph{all} directed graphs $G$, and strict inequalities are satisfied \emph{unless} $G$ has an independent set of size $n-1$, that is, $G$ is a star with one node being part of \emph{every} edge. 

These degenerate cases can be ignored in further analysis, as such configurations of $\mathbf{b}$ allow to reconstruct the graph deterministically. The implementation of the framework identifies such cases and returns the corresponding 0/1 values of $p_{ij}$. For degenerate cases, the corresponding system of equations~(\ref{eq:system}) might or might not have a solution depending on the parameters. For example when $m=3$ and $\mathbf{b}=(2,0,0,0,1,1)$ the system has infinitely many solutions of the form $(t,0,0,s,1/(a_{51}t), 1/(a_{61}t)))$ for any $t,s>0$, all of them yielding the same deterministic graph: $p_{12}=p_{13}=1$, $p_{21}=p_{23}=p_{31}=p_{32}=0$. However, for $m=3$, $\mathbf{b}=(2,1,1,2,1,1)$ and non zero elements $a_{ij}$ equal to $1$ the system has no solutions. We do not try to classify cases when the system has the solution in the proof as they lead to deterministic graphs and we handle them in the implementation of the framework separately anyway.

\medskip

Let us make the following observations that will be useful later on. Provided that properties~(\ref{eq:condition1})--(\ref{eq:condition2}) are satisfied, the following properties hold:
\begin{itemize}
    \item $x_i=0$ if and only if $b_i=0$. Indeed, if $x_i = 0$, then trivially $b_i =0$. Suppose then that $b_i=0$ and by symmetry we may assume that $i\in[m]$. By properties~(\ref{eq:condition1})--(\ref{eq:condition2}), $b_j > 0$ for at least one value of $j \in [2m]\setminus[m]$ and $j\neq i+m$. It follows that $x_j > 0$, $a_{ij} > 0$, and so $b_i \ge x_i a_{ij} x_j$. As a result, $x_i$ has to be equal to zero in order for $b_i$ to be zero. 
    \item we may assume that $x_1=1$. Indeed, one can reorder the nodes so that $b_1>0$. Then, one can multiply $x_i$ for all $i\in[m]$ by any positive constant $\alpha \in \R_+$ and divide $x_i$ for all $i \in [2m] \setminus [m]$ by $\alpha$ and the solution will not change.
\end{itemize}

The last observation means that we need to introduce the constraint $x_1=1$ if we ever hope to prove the uniqueness of the solution. If we do not do that, then there will be an infinite number of solutions but all of them will yield the same edge distribution for the random graph as the particular solution we are searching for.

\medskip

We will start with proving the uniqueness. After that, we will show that~(\ref{eq:condition1})--(\ref{eq:condition2}) are sufficient conditions.

\subsection{Uniqueness}

Let us assume that $m \ge 3$. For a contradiction, suppose that we have two different solutions: $\mathbf{x} = (x_i)_{i\in[n]}$ ($x_i \in \R_+$, $i \in [n]$) with $x_1=1$ and $\mathbf{y} = (y_i)_{i \in [n]}$ ($y_i \in \R_+$, $i \in [n]$) with $y_1=1$. It follows that for all $i \in [n]$ we have
$$
b_i = f_i(\mathbf{x}) = f_i(\mathbf{y}), \quad \text{ where } f_i(\mathbf{x}) = x_i \sum_{j=1}^n a_{ij} x_j.
$$
Let us analyze what happens at point $\mathbf{z} = t\mathbf{x} + (1-t)\mathbf{y}$ for some $t \in [0,1]$ (that is, $z_i = tx_i+(1-t)y_i$, $i \in [n]$). For each $i \in [n]$ we get
\begin{eqnarray*}
f_i(\mathbf{z}) &=& (tx_i+(1-t)y_i) \sum_{j=1}^n a_{ij} (tx_j+(1-t)y_j) \\
&=& \sum_{j=1}^n a_{ij} \left( t^2 x_i x_j+ t(1-t)(x_i y_j + x_j y_i) + (1-t)^2 y_i y_j \right) \\
&=& f_i(\textbf{x}) t^2 + \frac {t(1-t)}{x_i y_i} (f_i(\textbf{y}) x_i^2 + f_i(\textbf{x}) y_i^2) + f_i(\textbf{y}) (1-t)^2 \\
&=& b_i \left( t^2 + \frac {t(1-t)}{x_i y_i} (x_i^2 + y_i^2) + (1-t)^2 \right) \\
&=& b_i \left( 1 - 2t(1-t) + \frac {t(1-t)}{x_i y_i} (x_i^2 + y_i^2) \right) \\
&=& b_i \left( 1 + \frac {t(1-t)}{x_i y_i} (x_i^2 - 2x_iy_i + y_i^2) \right) \\
&=& b_i \left( 1+t(1-t)\frac{(x_i-y_i)^2}{x_iy_i} \right) =: g_i(t).
\end{eqnarray*}
Note that $g_i'(1/2)=0$ for all $i$ (as either $x_i-y_i$ vanishes and so $g_i(t)$ is a constant function or it does not vanish but then $g_i(t)$ is a parabola with a maximum at $t=1/2$). For convenience, let $\mathbf{v} = (\mathbf{x} + \mathbf{y})/2$ and $\mathbf{s} = (\mathbf{x} - \mathbf{y})/2$ (that is, $v_i=(x_i+y_i)/2$ and $s_i=(x_i-y_i)/2$ for all $i \in [n]$). It follows that 
$$
\frac{d g_i}{dh} \Big( \mathbf{v}+h\mathbf{s} ~|~ h=0 \Big)=0.
$$
On the other hand,
$$
g_i(\mathbf{v}+h\mathbf{s}) = \sum_{j=1}^na_{ij}(v_i+hs_i)(v_j+hs_j)
$$
and so
$$
\frac{d g_i}{dh}(\mathbf{v}+h\mathbf{s}) = \sum_{j=1}^na_{ij}(s_i(v_j+hs_j)+s_j(v_i+hs_i)).
$$
Combining the two observations, we get that
\begin{equation}\label{eq:condition_for_s}
0=\frac{d g_i}{dh} \Big( \mathbf{v}+h\mathbf{s} ~|~ h=0 \Big) = \sum_{j=1}^na_{ij}(s_iv_j+s_jv_i).
\end{equation}

Now, for $i \ge 1$, let $u_i = s_i / v_i$, provided that $v_i \neq 0$.
Recall that in particular $s_1=(x_1-y_1)/2=0$ and $v_1=(x_1+y_1)/2=1$, as we assumed that $x_1=y_1=1$, and so $u_1=0$.
Additionally, if $v_i=0$ (that is, the corresponding node has in-degree 0 or out-degree 0), then we may take $u_i=0$, as it will cancel out anyway. Denote the set of indices $i$ when $v_i=0$ by $Z$. Substituting it to~(\ref{eq:condition_for_s}) we get:
$$
\sum_{j=1}^na_{ij}v_iv_j(u_i+u_j) = 0.
$$

Notice that we can rescale all $u_i$'s by the same multiplicative factor so that $u_{\ell}=1$ for some $\ell \in [n]$ and for all other indices $|u_i|\leq1$ (potentially with negative rescaling factor). For index $\ell$ we have:
$$
\sum_{j=1}^n a_{\ell j} v_{\ell}v_j(u_{\ell}+u_j) = \sum_{j=1}^n a_{\ell j} v_{\ell}v_j(1+u_j) = 0.
$$
But this possible only if $\ell\in[m]$, as otherwise the left hand side of the above equation is at least its first term, namely, $a_{\ell 1}v_{\ell} v_1(1+u_1) =a_{\ell 1}v_{\ell} > 0$. If $\ell \in[m]$, then we get
$$
\sum_{j=m+1}^n a_{\ell j}v_{\ell} v_j (u_{\ell}+u_j) = \sum_{j=m+1}^n a_{\ell j}v_{\ell} v_j (1+u_j) = 0
$$
as $a_{\ell j}=0$ for $j\in[m]$. But this means that $u_j=-1$ for $j\in[n]\setminus([m]\cup Z)$.

Let us concentrate on any index $\ell'\in [n]\setminus([m]\cup Z)$ (note that this set is non-empty).
For this index, we have the following condition:
$$
\sum_{j=1}^m a_{\ell' j}v_{\ell'}v_j(u_{\ell'}+u_j) = \sum_{j=1}^m a_{\ell' j}v_{\ell'}v_j(-1+u_j) = 0.
$$
However, since $|u_j|\leq 1$ for all $j \in[m]$, all entries of the sum are non-negative and so they would all have to be equal to $0$ for the sum to be $0$. This is not possible as $u_1=0$ and so $a_{\ell'1}v_{\ell'}v_1(-1+u_1) =-a_{\ell'1}v_{\ell'}<0$. The desired contradiction shows that the solution is unique.

% \subsection{Necessity}

% Suppose that $\mathbf{x} = (x_i)_{i\in[n]}$ ($x_i \in \R_+ \cup \{0\}$, $i \in [n]$) is a solution to~(\ref{eq:system}). We will show that the condition~(\ref{eq:condition1}) has to be satisfied. By symmetry, the same argument proves that the condition~(\ref{eq:condition2}) is satisfied too.

% Consider any $j\in[m]$. Let us first observe that
% \begin{eqnarray*}
% - b_{j+m} + \sum_{i=m+1}^nb_i &=& \sum_{i\in[n]\setminus([m]\cup\{j+m\})} b_i =  \sum_{i\in[n]\setminus([m]\cup\{j+m\})}\left( x_i \sum_{k=1}^na_{ik}x_k \right) \\
% &\geq& \sum_{i\in[n]\setminus([m]\cup\{j+m\})}\left( x_i a_{ij}x_j \right).
% \end{eqnarray*}

% Finally, we use the fact that $a_{ij}=0$ if $i\in[m] \cup \{j+m\}$ to get that 
% $$
% - b_{j+m} + \sum_{i=m+1}^nb_i \geq \sum_{i\in[n]\setminus([m]\cup\{j+m\})}\left( x_i a_{ij}x_j \right) = x_j\sum_{i=1}^{n}x_i a_{ij}=b_j.
% $$
% It follows that~(\ref{eq:condition1}) is a necessary condition for the existence of a solution to~(\ref{eq:system}).

\subsection{Sufficiency}

We will continue assuming that $m \geq 3$. For a contradiction, suppose that there exists a vector $\mathbf{b} = (b_i)_{i \in [n]}$, that satisfies~(\ref{eq:condition1})--(\ref{eq:condition2}), and $\sum_{i=1}^{m}b_i=\sum_{i=m+1}^{n}b_i>0$ (assumption~(\ref{eq:assumption})) but for which there is no solution to the system~(\ref{eq:system}). Without loss of generality, since one can reorder nodes and relabel in- and out-degrees if needed, we may assume that $b_1$ is a largest value in vector $\mathbf{b}$. We will call such vectors \emph{infeasible}. On the other hand, vectors that yield a solution $\mathbf{x} = (x_i)_{i \in [n]}$, with $x_i\geq0$ for all $i$, will be called \emph{feasible}. As proved earlier, if the solution exists, then it must be unique (remember that we assume that $x_1=1$). We will introduce more vectors $\textbf{b}$ (both feasible and infeasible) below but we assume that matrix $\textbf{A}$ is fixed.

Let us now construct another vector $\mathbf{b}' = (b'_i)_{i \in [n]}$ for which there exists a solution to~(\ref{eq:system}) (that is, $\mathbf{b}'$ is feasible) but also $b'_1=b_1$ is a largest element in $\mathbf{b}'$. Indeed, it can be done easily by, for example, taking $x'_1=1$, $x'_i = s$ for $i\in[m]\setminus\{1\}$ ($s$ is a fixed but sufficiently small positive constant for the inequalities below to hold), and $x'_i = b_1 / (\sum_{j \in [n]} a_{1j}) = b_1 / (\sum_{j \in [n] \setminus [m]} a_{1j})$ for $i\in[n]\setminus[m]$. Vector $\mathbf{b}'$ is now defined by the system~(\ref{eq:system}). We immediately get that
$$
b'_1 = x'_1 \sum_{j \in [n]} a_{1j} x'_j = \sum_{j \in [n] \setminus [m]} a_{1j} x'_j = b_1.
$$
Also, $s$ can be made arbitrarily small so that $b'_i<b_1$ for $i\in[m]\setminus\{1\}$.
Finally, for $i \in[n]\setminus[m]$ we have
$$
b'_i = x'_i \sum_{j \in [m]} a_{ij} x'_j = x'_i a_{i1} + x'_i \sum_{j=2}^m a_{ij} s = b_1 \frac{a_{i1}} {\sum_{j \in [n]} a_{1j}} + x'_i \sum_{j=2}^m a_{ij} s.
$$
Since $m \ge 3$, the first term is smaller than $b_1$. Hence, since $s$ can be made arbitrarily small, we can ensure that $b'_i < b_1$. The desired properties hold.

We will consider points along the line segment between $\mathbf{b}'$ and $\mathbf{b}$, namely,
$$
\mathbf{b}(t) = (b_i(t))_{i \in [n]} = (1-t)\mathbf{b}'+t\mathbf{b}, \qquad \text{ for } t\in[0,1].
$$
Since $\mathbf{b}'$ is feasible and we already proved that~(\ref{eq:condition1})--(\ref{eq:condition2}) are necessary conditions, we know that $\mathbf{b}'$ satisfies~(\ref{eq:condition1})--(\ref{eq:condition2}). But, as a result, not only $\mathbf{b}$ and $\mathbf{b}'$ satisfy these properties but also $\mathbf{b}(t)$ satisfies them for any $t \in [0,1]$. In particular, it follows that there exists a universal constant $\eps > 0$ such that for any $t \in [0,1]$ we have 
\begin{equation}\label{eq:separation}
(1-\eps)\sum_{i=m+2}^nb_i(t) > b_1(t).
\end{equation}

Fix $t \in [0,1)$ and suppose that $\mathbf{b}(t)$ is feasible. Let $\mathbf{x}(t) = (x_i(t))_{i\in [n]}$ be the (unique) solution for $\mathbf{b}(t)$. From the analysis performed in the proof of uniqueness of the solution it follows that our transformation is a local diffeomorphism, that is, the differential of the transformation is bijective for the admissible values of $x_i$ and $b_i$. (Note that this also covers the case $t=0$. This case is on the boundary of the considered range of $t$ but it is an interior point of the domain of the mapping.) In the following considerations, we assume that point $x_1$ and $b_1$ are removed from the analysis (as they are fixed) and also that the indices from the set $Z$ (that is, as defined above, the set of indices $i$ for which $v_i=0$) are excluded as they are fixed. As a result we may move to a manifold of a dimension $n'<n$ by dropping the dimensions that are fixed. In the considered manifold, any open set in $\R^{n'}$ containing (part of) $\mathbf{x}(t)$ is mapped to an open set in $\R^{n'}$ containing (part of) $\mathbf{b}(t)$. In particular, there exists $\delta > 0$ such that $\mathbf{b}(s)$ is feasible for any $t - \delta \le s \le t + \delta$. Combining this observation with the fact that $\mathbf{b}'=\mathbf{b}(0)$ is feasible, $\mathbf{b}=\mathbf{b}(1)$ is \emph{not} feasible we get that there exists $T \in (0,1]$ such that $\mathbf{b}(T)$ is not feasible but $\mathbf{b}(t)$ is feasible for any $t \in [0,T)$. Indeed, if no such $T$ exists (that is, there is no minimal infeasible $t \in (0,1]$), then there would exist a decreasing sequence of infeasible values of $t$ that converges to a feasible $t$. This is not possible as in some neighbourhood of a feasible point $t$, points are also feasible.

Consider any sequence $(t_i)_{i \in \N}$ of real numbers $t_i \in [0,T)$ such that $t_i\to T$ as $i \to \infty$; for example, $t_i = T(1-1/i)$. All limits from now on will be for $i \to \infty$. Recall that $\mathbf{b}(t_i)$ is feasible and so $\mathbf{x}(t_i)$ is well-defined.

Before we move forward, let us show that there exists a sufficiently large but universal constant $\Delta$ such that for all $t \in [0,T)$ and all $i$ (except possibly $i=m+1$), we have $x_i(t) \le \Delta$. Indeed, by our assumption on the solution, $x_1(t)=1\le \Delta$. By the equation~(\ref{eq:system}) for $b_1(t) = b_1$, we have for $i\in[n]\setminus[m+1]$
$$
x_i(t) = \frac {1}{a_{1i}} \cdot x_1(t) a_{1i} x_i(t) < \frac {1}{a_{1i}} \cdot x_1(t) \sum_{j=1}^n a_{1j} x_j(t) = \frac {b_1(t)}{a_{1i}} = \frac {b_1}{a_{1i}} \le \Delta.
$$
But this immediately means that $x_i(t)$ are also bounded for $i\in[m]$ by considering any equation for $b_i(t) \le b_1(t) =b_1$ where $i\in[n]\setminus([m]\cup Z)$. This implies that only $x_{m+1}(t)$ can potentially be unbounded.

If $x_{m+1}(t)$ is bounded for all $t \in [0,T)$, then by the Bolzano-Weierstrass theorem the sequence $t_i$ has a subsequence $(\mathbf{x}(t_{s_i}))_{i \in [n]}$ such that $\Vert \mathbf{x}(t_{s_i}) \Vert \to c$ for some $c \in \R$. However, if this is the case then, by continuity of our transformation, the limiting value $\mathbf{b}(T)$ would be feasible, giving us the desired contradiction. It remains to consider the case when $x_{m+1}(t_{s_i}) \to \infty$ for some subsequence $s_i$.
However, this implies that $x_{j}(t_{s_i})\to 0$ for all $j\in[m]\setminus\{1\}$. This means that in the limit we have $x_i(T)=b_i(T)/a_{i1}$ for $i\in[n]\setminus[m+1]$. Substituting it into the first equation we get 
$$
b_1(T) = x_1(T)\sum_{i=m+2}^na_{1i}x_i(T) = 1\cdot\sum_{i=m+2}^na_{1i}\cdot b_i(T)/a_{i1} = \sum_{i=m+2}^nb_i(T).
$$
This contradicts (\ref{eq:separation}), which concludes the proof.

\bigskip

As a final note, let us observe that the proof implies that if $b_1$ gets close to $\sum_{i=m+2}^nb_i$ (from below) and $b_{m+1}>0$, then indeed $x_{m+1}$ will grow to be a large number. This consideration has a numerical impact as it might affect the convergence of numeric algorithms finding $x_i$ due to floating point computation precision issues. The proof also shows that the case when the conditions~(\ref{eq:condition1})--(\ref{eq:condition2}) are not satisfied (that is, we would have an equality instead of inequality) will have a solution if $b_{m+1}=0$ and otherwise will not have a solution (this corresponds to the two examples we have given earlier).

\subsection{Model with Loops}

In order to accommodate loops that are present in the graph, we relax the assumption that $a_{i,i+m}=0$ for $i\in[m]$ and now assume that $a_{i,i+m} \ge 0$ for $i\in[m]$. However, using the notation from the previous section we will additionally assume that $a_{1, m+1}>0$, that is, for the largest element of $\mathbf{b}$, we assume that the corresponding node has a loop. This auxiliary assumption is satisfied in our application as, in fact, all landmarks have loops.

The $m=2$ case continues to be a degenerate case that has to be delt with independently. Consider the following set of equations:
$$
\left[\begin{matrix}
b_1 \\ b_2 \\ b_3 \\ b_4
\end{matrix}\right] =
\left[\begin{matrix}
a_{13} & a_{14} & 0 & 0 \\
0 & 0 & a_{23} & a_{24} \\
a_{13} & 0 & a_{23} & 0 \\
0 & a_{14} & 0 & a_{24}
\end{matrix}\right]
\left[\begin{matrix}
x_1x_3 \\ x_1x_4 \\ x_2x_3 \\ x_2x_4
\end{matrix}\right].
$$
As before, we assume that $x_1=1$ and, since $b_1+b_2=b_3+b_4$ the system reduces to: 
$$
\left[\begin{matrix}
b_1 \\ b_2 \\ b_3
\end{matrix}\right] =
\left[\begin{matrix}
a_{13} & a_{14} & 0 & 0 \\
0 & 0 & a_{23} & a_{24} \\
a_{13} & 0 & a_{23} & 0
\end{matrix}\right]
\left[\begin{matrix}
x_3 \\ x_4 \\ x_2x_3 \\ x_2x_4
\end{matrix}\right] =
\left[\begin{matrix}
a_{14} & 0 & a_{13} & 0 \\
0 & a_{24} & 0 & a_{23} \\
0 & 0 & a_{13} & a_{23} 
\end{matrix}\right]
\left[\begin{matrix}
x_4  \\ x_2x_4 \\ x_3 \\ x_2x_3
\end{matrix}\right]
.
$$
Equivalently, since $b_1$ is the largest element of $\mathbf{b}$, for some non-negative $p_i$ and positive $q=a_{23}$: 
$$
\left[\begin{matrix}
p_1 \\ p_2 \\ p_3
\end{matrix}\right] =
\left[\begin{matrix}
1 & 0 & 0 & -q \\
0 & 1 & 0 & q \\
0 & 0 & 1 & q
\end{matrix}\right]
\left[\begin{matrix}
x_4  \\ x_2x_4 \\ x_3 \\ x_2x_3
\end{matrix}\right].
$$
If $x_2=0$, we get a unique positive solution for $x_3$ and $x_4$ and it exists only if $p_2=0$ (which happens for $b_2=0$).
On the other hand, if $x_2>0$ (recall that, by assumption, $x_i \ge 0$), then in the above system of equations one can reduce $x_3$ and $x_4$ leaving only $x_2$ as a variable in the quadratic equation:
$$
Q(x_2)=q(p_1+p_3)x_2^2+(-p_2q+p_3q+p_1)x_2-p_2 = 0.
$$
Since $x_2 > 0$, we get that $p_2 =x_2(x_4+qx_3) > 0$. (Note that if $x_3=x_4=0$, then $b_3=b_4=0$ and, as a consequence, $b_1=b_2=0$ which gives as a contradiction as $b_1>0$.) As the term $(p_1+p_3)q$ is positive and $-p_2$ is negative we get that there exists exactly one positive solution $x_2$ of this equation. Indeed, since $Q(0) = -p_2 <0$ and the parabola $Q(x_2)$ has the coefficient $(p_1+p_3)q>0$ associated with the quadratic term, there is exactly one positive solution $x_2 > 0$. Now, assuming that $x_2>0$, from the third equation we see that $x_3>0$ and from the first equation we get that $x_4>0$. 

In summary, for $m=2$, subject to the constraint $x_1=1$, the solution of the system always exists and is unique.
If $m\geq3$ we also show that the solution exists always. The part of the proof of uniqueness remains unchanged. The part for sufficiency also remains unchanged until we reach the case where we consider $x_{m+1}(t_{s_i}) \to\infty$. However as $a_{1,m+1}>0$ this is not possible since $x_1=1$ and $b_1$ is fixed. So we are left with the cases that $\Vert \mathbf{x}(t_{s_i}) \Vert \to c$ for some $c \in \R$ which means that $\mathbf{b}(t)$ always converges to a feasible solution as $t\to1$ (even if in conditions~(\ref{eq:condition1})--(\ref{eq:condition2}) we have an equality).




\section{Appendix---Scalable Implementation with Landmarks}\label{apdx:landmarks}

Recall that in \textbf{Step~1} of the algorithm, we obtain a partition $\textbf{C}$ of the set of nodes $V$ into $\ell$ communities: $C_1, \ldots, C_\ell$. The partition is then carefully refined by repeatedly splitting some parts of it with the goal to reach precisely $n'=4 \sqrt{n}$ parts; $n'$ might be adjusted by more experienced user, if needed. (The number of communities is typically relatively small. However, if $\ell \ge 4 \sqrt{n}$, then of course there is no need to do the refinement. However, in such rare cases each part in the initial partition is forced to be split into $s$ parts anyway. We fixed $s=4$ as a default value.) The heuristic algorithm is quite involved as it needs to find a good compromise between the quality of the approximation and the speed. The reader is directed to~\cite{Embedding_Complex_Networks_Scalable} for more details on how the refinement is obtained. 

Once the partition is refined, each part $C_i$ is replaced by its landmark $u_i$. The procedure depends on whether we deal with undirected or directed graphs. Let us start with undirected graphs. The position of landmark $u_i$ in the embedded space $\R^k$ is assigned as follows:
\begin{equation}\label{eq:position_landmark}
\emb(u_i) = \frac {\sum_{j \in C_i} w_j \ \emb(v_j)}{\sum_{j \in C_i} w_j}.
\end{equation}
In order to measure a variation within a cluster, we also compute the weighted sum of squared errors:
\begin{equation}\label{eq:error_landmark}
e_i = \sum_{j \in C_i} w_j \ \dist \big( \emb(u_i), \emb(v_j) \big)^2.
\end{equation}
The expected degree of landmark $u_i$ (that we denote as $w'_i$ in order to distinguish it from $w_i$, the expected degree of node $w_i$) is the sum of the expected degrees of the associated nodes in the original model, that is, $w'_i := \sum_{j \in C_i} w_j$.

The approximated algorithm uses the auxiliary \emph{Geometric Chung-Lu} (GCL) model on the set of landmarks $V = \{ u_1, \ldots, u_{n'} \}$ in which each pair of landmarks $u_i, u_j$, independently of other pairs, forms an edge with probability $p'_{i,j}$, where
\begin{equation*}
p'_{i,j} = x_i' x'_j g(d_{i,j}) 
\end{equation*}
for some carefully tuned weights $x'_i \in \R_+$. Additionally, for $i\in[n']$, the probability of creating a self loop around landmark $u_i$ is equal to 
$$
p'_{i,i} = (x'_i)^2 g(d_{i,i}), \qquad \text{ where } \qquad d_{i,i} = \sqrt{ \frac{e_i}{\sum_{j \in C_i} w_j} }.
$$
Note that the ``distance'' $d_{i,i}$ from landmark $u_i$ to itself is an approximation of the unobserved weighted average distance $d_{a,b}$ over all pairs of nodes $a$ and $b$ associated with $u_i$.
The weights are selected such that the expected degree of landmark $u_i$ is $w'_i$; that is, for all $i \in [n']$
$$
w'_i = \sum_{j \in [n']} p'_{i,j} =  x'_i \sum_{j \in [n']} x'_j g(d_{i,j}).
$$
The relationship between the weights in the auxiliary model and the original one is expected to be as follows: for any node $v_k \in C_i$ associated with landmark $u_i$ we have
$$
x_k \approx x'_i \ \frac {w_k}{\sum_{j \in C_i} w_j}.
$$

The adjustment for directed graph is straightforward. We use the same algorithm and positions for landmarks, that is, we still use~(\ref{eq:position_landmark}) and~(\ref{eq:error_landmark}) but with $w_j$ being the total degree of landmark $u_j$, namely, $w_j = w_j^{in} + w_j^{out}$. The probability for an ordered pair of landmarks $u_i, u_j$ to form a directed edge in the auxiliary \emph{Geometric Chung-Lu Directed Graph Model} is equal to 
$$
p'_{i,j} = {x'}_i^{out} {x'}_j^{in} g(d_{i,j}) 
$$
and
$$
p'_{i,i} = {x'}_i^{out} {x'}_i^{in} g(d_{i,i}), \qquad \text{ where } \qquad d_{i,i} = \sqrt{ \frac{e_i}{\sum_{j \in C_i} w_j} }.
$$
As before, any node $v_k \in C_i$ associated with landmark $u_i$ inherits a fraction of its weights, that is, we expect that the original weights are well approximated by the following:
$$
x_k^{out} \approx {x'}_i^{out} \ \frac {w_k^{out}}{\sum_{j \in C_i} w_j^{out}} \qquad  \text{and} \qquad x_k^{in} \approx {x'}_i^{in} \ \frac {w_k^{in}}{\sum_{j \in C_i} w_j^{in}}.
$$

%%========================================
\section{Experimental Evaluation}
\label{sec:experiment}
To demonstrate the viability of our modeling methodology, we show experimentally how through the deliberate combination and configuration of parallel FREEs, full control over 2DOF spacial forces can be achieved by using only the minimum combination of three FREEs.
To this end, we carefully chose the fiber angle $\Gamma$ of each of these actuators to achieve a well-balanced force zonotope (Fig.~\ref{fig:rigDiagram}).
We combined a contracting and counterclockwise twisting FREE with a fiber angle of $\Gamma = 48^\circ$, a contracting and clockwise twisting FREE with $\Gamma = -48^\circ$, and an extending FREE with $\Gamma = -85^\circ$.
All three FREEs were designed with a nominal radius of $R$ = \unit[5]{mm} and a length of $L$ = \unit[100]{mm}.
%
\begin{figure}
    \centering
    \includegraphics[width=0.75\linewidth]{figures/rigDiagram_wlabels10.pdf}
    \caption{In the experimental evaluation, we employed a parallel combination of three FREEs (top) to yield forces along and moments about the $z$-axis of an end effector.
    The FREEs were carefully selected to yield a well-balanced force zonotope (bottom) to gain full control authority over $F^{\hat{z}_e}$ and $M^{\hat{z}_e}$.
    To this end, we used one extending FREE, and two contracting FREEs which generate antagonistic moments about the end effector $z$-axis.}
    \label{fig:rigDiagram}
\end{figure}


\subsection{Experimental Setup}
To measure the forces generated by this actuator combination under a varying state $\vec{x}$ and pressure input $\vec{p}$, we developed a custom built test platform (Fig.~\ref{fig:rig}). 
%
\begin{figure}
    \centering
    \includegraphics[width=0.9\linewidth]{figures/photos/rig_labeled.pdf}
    \caption{\revcomment{1.3}{This experimental platform is used to generate a targeted displacement (extension and twist) of the end effector and to measure the forces and torques created by a parallel combination of three FREEs. A linear actuator and servomotor impose an extension and a twist, respectively, while the net force and moment generated by the FREEs is measured with a force load cell and moment load cell mounted in series.}}
    \label{fig:rig}
\end{figure}
%
In the test platform, a linear actuator (ServoCity HDA 6-50) and a rotational servomotor (Hitec HS-645mg) were used to impose a 2-dimensional displacement on the end effector. 
A force load cell (LoadStar  RAS1-25lb) and a moment load cell (LoadStar RST1-6Nm) measured the end-effector forces $F^{\hat{z_e}}$ and moments $M^{\hat{z_e}}$, respectively.
During the experiments, the pressures inside the FREEs were varied using pneumatic pressure regulators (Enfield TR-010-g10-s). 

The FREE attachment points (measured from the end effector origin) were measured to be:
\begin{align}
    \vec{d}_1 &= \bmx 0.013 & 0 & 0 \emx^T  \text{m}\\
    \vec{d}_2 &= \bmx -0.006 & 0.011 & 0 \emx^T  \text{m}\\
    \vec{d}_3 &= \bmx -0.006 & -0.011 & 0 \emx^T \text{m}
%    \vec{d}_i &= \bmx 0 & 0 & 0 \emx^T , && \text{for } i = 1,2,3
\end{align}
All three FREEs were oriented parallel to the end effector $z$-axis:
\begin{align}
    \hat{a}_i &= \bmx 0 & 0 & 1 \emx^T, \hspace{20pt} \text{for } i = 1,2,3
\end{align}
Based on this geometry, the transformation matrices $\bar{\mathcal{D}}_i$ were given by:
\begin{align}
    \bar{\mathcal{D}}_1 &= \bmx 0 & 0 & 1 & 0 & -0.013 & 0 \\ 0 & 0 & 0 & 0 & 0 & 1 \emx^T  \\
    \bar{\mathcal{D}}_2 &= \bmx 0 & 0 & 1 & 0.011 & 0.006 & 0 \\ 0 & 0 & 0 & 0 & 0 & 1 \emx^T  \\
    \bar{\mathcal{D}}_3 &= \bmx 0 & 0 & 1 & -0.011 & 0.006 & 0 \\ 0 & 0 & 0 & 0 & 0 & 1 \emx^T 
%    \bar{\mathcal{D}}_i &= \bmx 0 & 0 & 1 & 0 & 0 & 0 \\ 0 & 0 & 0 & 0 & 0 & 1 \emx^T , && \text{for } i = 1,2,3
\end{align}
These matrices were used in equation \eqref{eq:zeta} to yield the state-dependent fluid Jacobian $\bar{J}_x$ and to compute the resulting force zontopes.
%while using measured values of $\vec{\zeta}^{\,\text{meas}} (\vec{q}, \vec{P})$ and $\vec{\zeta}^{\,\text{meas}} (\vec{q}, 0)$ in \eqref{eq:fiberIso} yields the empirical measurements of the active force.



\subsection{Isolating the Active Force}
To compare our model force predictions (which focus only on the active forces induced by the fibers)
to those measured empirically on a physical system, we had to remove the elastic force components attributed to the elastomer. 
Under the assumption that the elastomer force is merely a function of the displacement $\vec{x}$ and independent of pressure $\vec{p}$ \cite{bruder2017model}, this force component can be approximated by the measured force at a pressure of $\vec{p}=0$. 
That is: 
\begin{align}
    \vec{f}_{\text{elast}} (\vec{x}) = \vec{f}_{\text{\,meas}} (\vec{x}, 0)
\end{align}
With this, the active generalized forces were measured indirectly by subtracting off the force generated at zero pressure:
\begin{align}
    \vec{f} (\vec{x}, \vec{p})  &= \vec{f}_{\text{meas}} (\vec{x}, \vec{p}) - \vec{f}_{\text{meas}} (\vec{x}, 0)     \label{eq:fiberIso}
\end{align}


%To validate our parallel force model, we compare its force predictions, $\vec{\zeta}_{\text{pred}}$, to those measured empirically on a physical system, $\vec{\zeta}_\text{meas}$. 
%From \eqref{eq:Z} and \eqref{eq:zeta}, the force at the end effector is given by:
%\begin{align}
%    \vec{\zeta}(\vec{q}, \vec{P}) &= \sum_{i=1}^n \bar{\mathcal{D}}_i \left( {\bar{J}_V}_i^T(\vec{q_i}) P_i + \vec{Z}_i^{\text{elast}} (\vec{q_i}) \right) \\
%    &= \underbrace{\sum_{i=1}^n \bar{\mathcal{D}}_i {\bar{J}_V}_i^T(\vec{q_i}) P_i}_{\vec{\zeta}^{\,\text{fiber}} (\vec{q}, \vec{P})} + \underbrace{\sum_{i=1}^n \bar{\mathcal{D}}_i \vec{Z}_i^{\text{elast}} (\vec{q_i})}_{\vec{\zeta}^{\text{elast}} (\vec{q})}   \label{eq:zetaSplit}
%     &= \vec{\zeta}^{\,\text{fiber}} (\vec{q}, \vec{P}) + \vec{\zeta}^{\text{elast}} (\vec{q})
%\end{align}
%\Dan{These will need to reflect changes made to previous section.}
%The model presented in this paper does not specify the elastomer forces, $\vec{\zeta}^{\text{elast}}$, therefore we only validate its predictions %of the fiber forces, $\vec{\zeta}^{\,\text{fiber}}$. 
%We isolate the fiber forces by noting that $\vec{\zeta}^{\text{elast}} (\vec{q}) = \vec{\zeta}(\vec{q}, 0)$ and rearranging \eqref{eq:zetaSplit}
%\begin{align}
%    \vec{\zeta}^{\,\text{fiber}} (\vec{q}, \vec{P})  &= \vec{\zeta}(\vec{q}, \vec{P}) - \vec{\zeta}(\vec{q}, 0)     \label{eq:fiberIso}
%%    \vec{\zeta}^{\,\text{fiber}}_{\text{emp}} (\vec{q}, \vec{P})  &= \vec{\zeta}_{\text{emp}}(\vec{q}, \vec{P}) - %\vec{\zeta}_{\text{emp}}(\vec{q}, 0)
%\end{align}
%Thus we measure the fiber forces indirectly by subtracting off the forces generated at zero pressure.  


\subsection{Experimental Protocol}
The force and moment generated by the parallel combination of FREEs about the end effector $z$-axis  was measured in four different geometric configurations: neutral, extended, twisted, and simultaneously extended and twisted (see Table \ref{table:RMSE} for the exact deformation amounts). 
At each of these configurations, the forces were measured at all pressure combinations in the set
\begin{align}
    \mathcal{P} &= \left\{ \bmx \alpha_1 & \alpha_2 & \alpha_3 \emx^T p^{\text{max}} \, : \, \alpha_i = \left\{ 0, \frac{1}{4}, \frac{1}{2}, \frac{3}{4}, 1 \right\} \right\}
\end{align}
with $p^{\text{max}}$ = \unit[103.4]{kPa}. 
\revcomment{3.2}{The experiment was performed twice using two different sets of FREEs to observe how fabrication variability might affect performance. The results from Trial 1 are displayed in Fig.~\ref{fig:results} and the error for both trials is given in Table \ref{table:RMSE}.}



\subsection{Results}

\begin{figure*}[ht]
\centering

\def\picScale{0.08}    % define variable for scaling all pictures evenly
\def\plotScale{0.2}    % define variable for scaling all plots evenly
\def\colWidth{0.22\linewidth}

\begin{tikzpicture} %[every node/.style={draw=black}]
% \draw[help lines] (0,0) grid (4,2);
\matrix [row sep=0cm, column sep=0cm, style={align=center}] (my matrix) at (0,0) %(2,1)
{
& \node (q1) {(a) $\Delta l = 0, \Delta \phi = 0$}; & \node (q2) {(b) $\Delta l = 5\text{mm}, \Delta \phi = 0$}; & \node (q3) {(c) $\Delta l = 0, \Delta \phi = 20^\circ$}; & \node (q4) {(d) $\Delta l = 5\text{mm}, \Delta \phi = 20^\circ$};

\\

&
\node[style={anchor=center}] {\includegraphics[width=\colWidth]{figures/photos/s0w0pic_colored.pdf}}; %\fill[blue] (0,0) circle (2pt);
&
\node[style={anchor=center}] {\includegraphics[width=\colWidth]{figures/photos/s5w0pic_colored.pdf}}; %\fill[blue] (0,0) circle (2pt);
&
\node[style={anchor=center}] {\includegraphics[width=\colWidth]{figures/photos/s0w20pic_colored.pdf}}; %\fill[blue] (0,0) circle (2pt);
&
\node[style={anchor=center}] {\includegraphics[width=\colWidth]{figures/photos/s5w20pic_colored.pdf}}; %\fill[blue] (0,0) circle (2pt);

\\

\node[rotate=90] (ylabel) {Moment, $M^{\hat{z}_e}$ (N-m)};
&
\node[style={anchor=center}] {\includegraphics[width=\colWidth]{figures/plots3/s0w0.pdf}}; %\fill[blue] (0,0) circle (2pt);
&
\node[style={anchor=center}] {\includegraphics[width=\colWidth]{figures/plots3/s5w0.pdf}}; %\fill[blue] (0,0) circle (2pt);
&
\node[style={anchor=center}] {\includegraphics[width=\colWidth]{figures/plots3/s0w20.pdf}}; %\fill[blue] (0,0) circle (2pt);
&
\node[style={anchor=center}] {\includegraphics[width=\colWidth]{figures/plots3/s5w20.pdf}}; %\fill[blue] (0,0) circle (2pt);

\\

& \node (xlabel1) {Force, $F^{\hat{z}_e}$ (N)}; & \node (xlabel2) {Force, $F^{\hat{z}_e}$ (N)}; & \node (xlabel3) {Force, $F^{\hat{z}_e}$ (N)}; & \node (xlabel4) {Force, $F^{\hat{z}_e}$ (N)};

\\
};
\end{tikzpicture}

\caption{For four different deformed configurations (top row), we compare the predicted and the measured forces for the parallel combination of three FREEs (bottom row). 
\revcomment{2.6}{Data points and predictions corresponding to the same input pressures are connected by a thin line, and the convex hull of the measured data points is outlined in black.}
The Trial 1 data is overlaid on top of the theoretical force zonotopes (grey areas) for each of the four configurations.
Identical colors indicate correspondence between a FREE and its resulting force/torque direction.}
\label{fig:results}
\end{figure*}






% & \node (a) {(a)}; & \node (b) {(b)}; & \node (c) {(c)}; & \node (d) {(d)};


For comparison, the measured forces are superimposed over the force zonotope generated by our model in Fig.~\ref{fig:results}a-~\ref{fig:results}d.
To quantify the accuracy of the model, we defined the error at the $j^{th}$ evaluation point as the difference between the modeled and measured forces
\begin{align}
%    \vec{e}_j &= \left( {\vec{\zeta}_{\,\text{mod}}} - {\vec{\zeta}_{\,\text{emp}}} \right)_j
%    e_j &= \left( F/M_{\,\text{mod}} - F/M_{\,\text{emp}} \right)_j
    e^F_j &= \left( F^{\hat{z}_e}_{\text{pred}, j} - F^{\hat{z}_e}_{\text{meas}, j} \right) \\
    e^M_j &= \left( M^{\hat{z}_e}_{\text{pred}, j} - M^{\hat{z}_e}_{\text{meas}, j} \right)
\end{align}
and evaluated the error across all $N = 125$ trials of a given end effector configuration.
% using the following metrics:
% \begin{align}
%     \text{RMSE} &= \sqrt{ \frac{\sum_{j=1}^{N} e_j^2}{N} } \\
%     \text{Max Error} &= \max \{ \left| e_j \right| : j = 1, ... , N \}
% \end{align}
As shown in Table \ref{table:RMSE}, the root-mean-square error (RMSE) is less than \unit[1.5]{N} (\unit[${8 \times 10^{-3}}$]{Nm}), and the maximum error is less than \unit[3]{N}  (\unit[${19 \times 10^{-3}}$]{Nm}) across all trials and configurations.

\begin{table}[H]
\centering
\caption{Root-mean-square error and maximum error}
\begin{tabular}{| c | c || c | c | c | c|}
    \hline
     & \rule{0pt}{2ex} \textbf{Disp.} & \multicolumn{2}{c |}{\textbf{RMSE}} & \multicolumn{2}{c |}{\textbf{Max Error}} \\ 
     \cline{2-6}
     & \rule{0pt}{2ex} (mm, $^\circ$) & F (N) & M (Nm) & F (N) & M (Nm) \\
     \hline
     \multirow{4}{*}{\rotatebox[origin=c]{90}{\textbf{Trial 1}}}
     & 0, 0 & 1.13 & $3.8 \times 10^{-3}$ & 2.96 & $7.8 \times 10^{-3}$ \\
     & 5, 0 & 0.74 & $3.2 \times 10^{-3}$ & 2.31 & $7.4 \times 10^{-3}$ \\
     & 0, 20 & 1.47 & $6.3 \times 10^{-3}$ & 2.52 & $15.6 \times 10^{-3}$\\
     & 5, 20 & 1.18 & $4.6 \times 10^{-3}$ & 2.85 & $12.4 \times 10^{-3}$ \\  
     \hline
     \multirow{4}{*}{\rotatebox[origin=c]{90}{\textbf{Trial 2}}}
     & 0, 0 & 0.93 & $6.0 \times 10^{-3}$ & 1.90 & $13.3 \times 10^{-3}$ \\
     & 5, 0 & 1.00 & $7.7 \times 10^{-3}$ & 2.97 & $19.0 \times 10^{-3}$ \\
     & 0, 20 & 0.77 & $6.9 \times 10^{-3}$ & 2.89 & $15.7 \times 10^{-3}$\\
     & 5, 20 & 0.95 & $5.3 \times 10^{-3}$ & 2.22 & $13.3 \times 10^{-3}$ \\  
     \hline
\end{tabular}
\label{table:RMSE}
\end{table}

\begin{figure}
    \centering
    \includegraphics[width=\linewidth]{figures/photos/buckling.pdf}
    \caption{At high fluid pressure the FREE with fiber angle of $-85^\circ$ started to buckle.  This effect was less pronounced when the system was extended along the $z$-axis.}
    \label{fig:buckling}
\end{figure}

%Experimental precision was limited by unmodeled material defects in the FREEs, as well as sensor inaccuracy. While the commercial force and moment sensors used have a quoted accuracy of 0.02\% for the force sensor and 0.2\% for the moment sensor (LoadStar Sensors, 2015), a drifting of up to 0.5 N away from zero was noticed on the force sensor during testing.

It should be noted, that throughout the experiments, the FREE with a fiber angle of $-85^\circ$ exhibited noticeable buckling behavior at pressures above $\approx$ \unit[50]{kPa} (Fig.~\ref{fig:buckling}). 
This behavior was more pronounced during testing in the non-extended configurations (Fig.~\ref{fig:results}a~and~\ref{fig:results}c). 
The buckling might explain the noticeable leftward offset of many of the points in Fig.~\ref{fig:results}a and Fig.~\ref{fig:results}c, since it is reasonable to assume that buckling reduces the efficacy of of the FREE to exert force in the direction normal to the force sensor. 

\begin{figure}
    \centering
    \includegraphics[width=\linewidth]{figures/zntp_vs_x4.pdf}
    \caption{A visualization of how the \emph{force zonotope} of the parallel combination of three FREEs (see Fig.~\ref{fig:rig}) changes as a function of the end effector state $x$. One can observe that the change in the zonotope ultimately limits the work-space of such a system.  In particular the zonotope will collapse for compressions of more than \unit[-10]{mm}.  For \revcomment{2.5}{scale and comparison, the convex hulls of the measured points from Fig.~\ref{fig:results}} are superimposed over their corresponding zonotope at the configurations that were evaluated experimentally.}
    % \marginnote{\#2.5}
    \label{fig:zntp_vs_x}
\end{figure}
%%========================================
% \vspace{-0.5em}
\section{Conclusion}
% \vspace{-0.5em}
Recent advances in multimodal single-cell technology have enabled the simultaneous profiling of the transcriptome alongside other cellular modalities, leading to an increase in the availability of multimodal single-cell data. In this paper, we present \method{}, a multimodal transformer model for single-cell surface protein abundance from gene expression measurements. We combined the data with prior biological interaction knowledge from the STRING database into a richly connected heterogeneous graph and leveraged the transformer architectures to learn an accurate mapping between gene expression and surface protein abundance. Remarkably, \method{} achieves superior and more stable performance than other baselines on both 2021 and 2022 NeurIPS single-cell datasets.

\noindent\textbf{Future Work.}
% Our work is an extension of the model we implemented in the NeurIPS 2022 competition. 
Our framework of multimodal transformers with the cross-modality heterogeneous graph goes far beyond the specific downstream task of modality prediction, and there are lots of potentials to be further explored. Our graph contains three types of nodes. While the cell embeddings are used for predictions, the remaining protein embeddings and gene embeddings may be further interpreted for other tasks. The similarities between proteins may show data-specific protein-protein relationships, while the attention matrix of the gene transformer may help to identify marker genes of each cell type. Additionally, we may achieve gene interaction prediction using the attention mechanism.
% under adequate regulations. 
% We expect \method{} to be capable of much more than just modality prediction. Note that currently, we fuse information from different transformers with message-passing GNNs. 
To extend more on transformers, a potential next step is implementing cross-attention cross-modalities. Ideally, all three types of nodes, namely genes, proteins, and cells, would be jointly modeled using a large transformer that includes specific regulations for each modality. 

% insight of protein and gene embedding (diff task)

% all in one transformer

% \noindent\textbf{Limitations and future work}
% Despite the noticeable performance improvement by utilizing transformers with the cross-modality heterogeneous graph, there are still bottlenecks in the current settings. To begin with, we noticed that the performance variations of all methods are consistently higher in the ``CITE'' dataset compared to the ``GEX2ADT'' dataset. We hypothesized that the increased variability in ``CITE'' was due to both less number of training samples (43k vs. 66k cells) and a significantly more number of testing samples used (28k vs. 1k cells). One straightforward solution to alleviate the high variation issue is to include more training samples, which is not always possible given the training data availability. Nevertheless, publicly available single-cell datasets have been accumulated over the past decades and are still being collected on an ever-increasing scale. Taking advantage of these large-scale atlases is the key to a more stable and well-performing model, as some of the intra-cell variations could be common across different datasets. For example, reference-based methods are commonly used to identify the cell identity of a single cell, or cell-type compositions of a mixture of cells. (other examples for pretrained, e.g., scbert)


%\noindent\textbf{Future work.}
% Our work is an extension of the model we implemented in the NeurIPS 2022 competition. Now our framework of multimodal transformers with the cross-modality heterogeneous graph goes far beyond the specific downstream task of modality prediction, and there are lots of potentials to be further explored. Our graph contains three types of nodes. while the cell embeddings are used for predictions, the remaining protein embeddings and gene embeddings may be further interpreted for other tasks. The similarities between proteins may show data-specific protein-protein relationships, while the attention matrix of the gene transformer may help to identify marker genes of each cell type. Additionally, we may achieve gene interaction prediction using the attention mechanism under adequate regulations. We expect \method{} to be capable of much more than just modality prediction. Note that currently, we fuse information from different transformers with message-passing GNNs. To extend more on transformers, a potential next step is implementing cross-attention cross-modalities. Ideally, all three types of nodes, namely genes, proteins, and cells, would be jointly modeled using a large transformer that includes specific regulations for each modality. The self-attention within each modality would reconstruct the prior interaction network, while the cross-attention between modalities would be supervised by the data observations. Then, The attention matrix will provide insights into all the internal interactions and cross-relationships. With the linearized transformer, this idea would be both practical and versatile.

% \begin{acks}
% This research is supported by the National Science Foundation (NSF) and Johnson \& Johnson.
% \end{acks}
%%========================================


%% \section*{Acknowledgment}
%% The authors would like to thank Muhamed Begovic from University of Stuttgart for his help on the hardware setup, during his internship at BCAI.



%========================================
\bibliographystyle{IEEEtran}
\bibliography{contents/references}

\end{document}
