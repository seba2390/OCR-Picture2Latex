% ****** Start of file apssamp.tex ******
%
%   This file is part of the APS files in the REVTeX 4.2 distribution.
%   Version 4.2a of REVTeX, December 2014
%
%   Copyright (c) 2014 The American Physical Society.
%
%   See the REVTeX 4 README file for restrictions and more information.
%
% TeX'ing this file requires that you have AMS-LaTeX 2.0 installed
% as well as the rest of the prerequisites for REVTeX 4.2
%
% See the REVTeX 4 README file
% It also requires running BibTeX. The commands are as follows:
%
%  1)  latex apssamp.tex
%  2)  bibtex apssamp
%  3)  latex apssamp.tex
%  4)  latex apssamp.tex
%
\documentclass[%
 reprint,
%superscriptaddress,
%groupedaddress,
%unsortedaddress,
%runinaddress,
%frontmatterverbose, 
%preprint,
%preprintnumbers,
%nofootinbib,
%nobibnotes,
%bibnotes,
 amsmath,amssymb,
 aps,
pra,
%prl,
%rmp,
%prstab,
%prstper,
%floatfix,
]{revtex4-2}

\usepackage{graphicx}% Include figure files
\usepackage{dcolumn}% Align table columns on decimal point
\usepackage{bm}% bold math
\usepackage{color}
\usepackage{amsthm, amssymb, amsfonts}
\usepackage{ulem}
\usepackage{setspace}
\usepackage{cancel}
\newcommand{\da}{^\dagger}
\newcommand{\ra}{\rangle}
\newcommand{\la}{\langle}
\usepackage{lipsum, babel}

\usepackage{lineno}
%\linenumbers
%\usepackage{amsmath}
%\usepackage[makeroom]{cancel}
%\usepackage{hyperref}% add hypertext capabilities
%\usepackage[mathlines]{lineno}% Enable numbering of text and display math
%\linenumbers\relax % Commence numbering lines

%\usepackage[showframe,%Uncomment any one of the following lines to test 
%%scale=0.7, marginratio={1:1, 2:3}, ignoreall,% default settings
%%text={7in,10in},centering,
%%margin=1.5in,
%%total={6.5in,8.75in}, top=1.2in, left=0.9in, includefoot,
%%height=10in,a5paper,hmargin={3cm,0.8in},
%]{geometry}

\begin{document}

\preprint{APS/123-QED}

\title{Fast tunable coupling scheme of \textcolor{black}{Kerr parametric oscillators} based on shortcuts to adiabaticity}
%\\
%2. Fast tunable coupling scheme of Kerr-nonlinear parametric oscillators with fixed amplitude of pump field and always-on coupling\\
%3. Fast tunable coupling scheme of Kerr-nonlinear parametric oscillators based on transitionless rotation in phase space}% Force line breaks with \\
%\thanks{A footnote to the article title}%
%}

\author{S. Masuda$^{1,2}$}
\email{shumpei.masuda@aist.go.jp}
\author{T. Kanao$^{3}$}
\author{H. Goto$^{3}$}
\author{Y. Matsuzaki$^{1,2}$}
\author{T. Ishikawa$^{1,2}$}
\author{S. Kawabata$^{1,2}$}
\affiliation{%
$^1$Research Center for Emerging Computing Technologies (RCECT), National Institute of Advanced Industrial Science and Technology (AIST), 1-1-1, Umezono, Tsukuba, Ibaraki 305-8568, Japan
}%
\affiliation{%
$^2$NEC-AIST Quantum Technology Cooperative Research Laboratory, National Institute of Advanced Industrial Science and Technology (AIST), Tsukuba, Ibaraki 305-8568, Japan
}%
\affiliation{%
$^3$Frontier Research Laboratory, Corporate Research \& Development Center, Toshiba Corporation, 1, Komukai-Toshiba-cho, Saiwai-ku, Kawasaki 212-8582, Japan
}%
% \altaffiliation[Also at ]{Physics Department, XYZ University.}%Lines break automatically or can be forced with \\
%\author{Second Author}%
% \email{Second.Author@institution.edu}
%\affiliation{%
% Authors' institution and/or address\\
% This line break forced with \textbackslash\textbackslash
%}%

%\collaboration{MUSO Collaboration}%\noaffiliation

%\author{Charlie Author}
% \homepage{http://www.Second.institution.edu/~Charlie.Author}
%\affiliation{
% Second institution and/or address\\
% This line break forced% with \\
%}%
%\affiliation{
% Third institution, the second for Charlie Author
%}%
%\author{Delta Author}
%\affiliation{%
% Authors' institution and/or address\\
% This line break forced with \textbackslash\textbackslash
%}%

%\collaboration{CLEO Collaboration}%\noaffiliation

\date{\today}% It is always \today, today,
             %  but any date may be explicitly specified

\begin{abstract}
\textcolor{black}{Kerr parametric oscillators} (KPOs), which can be implemented with superconducting parametrons possessing large Kerr nonlinearity,  have been attracting much attention in terms of their applications to quantum annealing, universal quantum computation and studies of quantum many-body systems.
It is of practical importance for these studies to realize fast and accurate tunable coupling between KPOs in a simple manner.
We develop a simple scheme of fast tunable coupling of KPOs with high tunability in speed and amplitude using the fast transitionless rotation of a KPO in the phase space based on the shortcuts to adiabaticity.
Our scheme enables rapid switching of the effective coupling between KPOs,
and can be implemented with always-on linear coupling between KPOs, by controlling the phase \textcolor{black}{of the pump field} and \textcolor{black}{the resonance frequency of the KPO} without controlling the amplitude of the pump field nor using additional drive fields \textcolor{black}{and couplers}.
%\textcolor{black}{The effective coupling between KPOs can be controlled rapidly, while the amplitude of the pump field and the linear coupling between resonators constituting the KPOs  are kept constant.}
%Our scheme is based on a fast transitionless rotation of \textcolor{black}{a KPO in the phase space}, which can be implemented by controlling the phase \textcolor{black}{and frequency} of the pump field \sout{and the detuning of the KPO}.
%\textcolor{black}{The transitionless rotation of a KPO enables one to switch quickly the effective coupling between KPOs even with a always-on linear coupling between resonators constituting the KPOs.}
We apply the coupling scheme to a two-qubit gate, and show that our scheme realizes high gate fidelity compared to a purely adiabatic one, by mitigating undesired nonadiabatic transitions.
%\textcolor{black}{The developed simple scheme of coupling with high tunability in terms of speed and amplitude will extend the degrees of freedom of controls of the KPO systems, and is expected to lead to advances in the field of quantum computing.}
\end{abstract}

%\keywords{Suggested keywords}%Use showkeys class option if keyword
                              %display desired
\maketitle

%\tableofcontents
\section{Introduction}
In the mid-twentieth century, classical parametric phase-locked oscillators \cite{Onyshkevych1959,Goto1959}, called parametrons were utilized as classical bits of digital computers.
Recently, \textcolor{black}{Kerr parametric oscillators} (KPOs) \textcolor{black}{sometimes called Kerr-cat qubits}~\cite{Milburn1991,Wielinga1993,Goto2016}, which are parametrons in the single-photon Kerr regime~\cite{Wang2019,Grimm2020} where the nonlinearity is larger than the decay rate, attracted increasing attention in terms of their applications to quantum information processing~\cite{Goto2019} and studies of quantum many-body systems~\cite{Dykman2018,Rota2019}.


%studied theoretically~\cite{Kinsler1991,Wustmann2013,Zhang2017} and experimentally \cite{Wang2019}. 

In the circuit-QED architecture, which is a promising platform of quantum information processing \cite{You2005,Gambetta2017,Wendin2017,Krantz2019,Gu2019,Blais2020},
KPOs can be implemented~\cite{Meaney2014,Wang2019,Goto2019,Grimm2020} by a superconducting resonator with Kerr-nonlinearity realized by the Josephson junctions, driven by an oscillating pump field.
Two coherent states with opposite phases can exist stably in a KPO, and are used as qubit states.
%  in contrast to transmons which utilize a single-photon and a zero-photon states as qubit states.
Bit-flip error of a KPO is suppressed because of the stability of the coherent states against photon loss,
and thus phase-flip error dominates bit-flip error in a KPO.
It is expected that quantum error correction for KPOs can be performed with less overhead owing to such biased errors compared to conventional qubits with unbiased errors~\cite{Tuckett2019,Ataides2021}.

Quantum annealing \cite{Goto2016,Nigg2017,Puri2017,Zhao2018,Onodera2020,Goto2020a,Kanao2021}
and universal quantum computation \cite{Cochrane1999,Goto2016b,Puri2017b} using KPOs have been studied theoretically.
Single-qubit operations were experimentally demonstrated~\cite{Grimm2020}.
Two-qubit gate operations, which preserve the biased feature of errors and allow one to use its advantage, were studied theoretically~\cite{Puri2020}, and 
% and the exponential increase of the bit-flip time with the increase of the cat size \cite{Grimm2020} 
high error-correction performance by concatenating the XZZX surface code~\cite{Ataides2021} with KPOs were numerically demonstrated~\cite{Darmawan2021}.
Fast and accurate controls~\cite{Kanao2021b,Xu2021,Kang2021}, spectroscopy~\cite{Yamaji2021,Masuda2021b}, controls and dynamics not confined in qubit space~\cite{Zhang2017,Wang2019}, Boltzmann sampling~\cite{Goto2018}, effect of strong pump field~\cite{Masuda2020}\textcolor{black}{, effect of decay and dephasing \cite{Puri2017b}}, quantum phase transitions~\cite{Dykman2018,Rota2019} and quantum chaos~\cite{Milburn1991,Hovsepyan2016,Goto2021b} have been the subject of investigations of KPO systems.
Many of the above studies use multi-KPO systems where the inter-KPO coupling plays a major role determining the property of the system.
Simple coupling scheme of KPOs with high tunability in terms of speed and amplitude will extend the degrees of freedom of controls, and is highly desirable for significant advances in the fields.

Many of the relevant control schemes of KPOs resort to quantum adiabatic dynamics~\cite{Goto2019}.
However, in practice, there are unwanted excitations due to the violation of the quantum adiabatic theorem in the controls, when performed in a short time.
Fast and accurate manipulations of KPOs have been studied using the shortcuts to adiabaticity (STA)~\cite{Rice2003,Torrontegui2013,Masuda2015,Masuda2016,Palmero2016,Campo2019,Guery-Odelin2019,Lizuain2019}, a group of protocols which mitigate or eliminate completely such unwanted excitations realizing the desired final state.
Fast creation of a cat state (a superposition of two coherent states with opposite phases) \cite{Puri2017} and traveling cat states \cite{Goto2019b} and
geometric quantum computation with cat qubits were proposed~\cite{Kang2021} based on the STA.

%using the invariant-based reverse engineering protocol~\cite{Muga2009,Chen2009}, which is also categorized to STA.

%In the circuit QED architecture, KPOs were applied to the qubit readout~\cite{Yamamoto2014,Yamamoto2016}.
In this paper, we develop a scheme of fast tunable $ZZ$ coupling of KPOs using the counter-diabatic (CD) protocol~\cite{Rice2003,Masuda2016}, which is categorized to the STA.
The coupling scheme is based on a fast transitionless rotation of a KPO in the phase space, 
%which can be implemented by controlling the phase of the pump field and the detuning of the KPO.
and importantly can be implemented with the fixed amplitude of the pump field and with always-on coupling between resonators constituting the KPOs in contrast to other schemes~\cite{Goto2016b,Puri2017b,Puri2020},
and moreover does not require additional driving fields \textcolor{black}{ in contrast to the schemes in Ref.~\cite{Darmawan2021,Chono2022}}.
In our scheme, the coupled KPOs can be identical because the controlled relative phase of the pump fields can eliminate undesired energy transfers between KPOs. Thus, the scheme will mitigate hardware requirements, complexity of sample design and frequency crowding, which are critical and ubiquitous problem of current quantum computing technologies.
%\textcolor{black}{The transitionless rotation of a KPO enables one to switch quickly the effective coupling between KPOs even with a always-on linear coupling between resonators constituting the KPOs.}
We apply this scheme to $ZZ$ rotation ($R_{zz}$ gate), and show that our scheme realizes high fidelity compared to a purely adiabatic scheme,  mitigating undesired nonadiabatic transitions.
%\sout{To our knowledge, previous controls using the STA for realistic systems utilize auxiliary external fields or modulation of the amplitude of the external field}~\cite{Torrontegui2013,Masuda2016,Campo2019,Guery-Odelin2019}). \sout{We believe that this is the first proposal of the control using the STA which can be implemented with fixed amplitude of the oscillating external field.} 
%\sout{Furthermore, this is the first proposal of transitionless rotation of a quantum system in phase space, although rotation of quantum systems in real space have been studied in different systems}%systems~\cite{Masuda2015,Palmero2016,Lizuain2019}.}
%\textcolor{red}{Cite in different places ~\cite{Masuda2015,Palmero2016,Lizuain2019}.}
%because the state of the system is disturbed by undesired nonadiabatic transitions

%In this paper, we first develop a method of fast rotation of the Wigner function of a KPO based on the CD protocol.
%By using the fast rotation of a KPO, we develop a scheme of a fast tunable ZZ coupling between KPOs, which gives rise to ZZ gate.
%In the method, time dependent detuning is used to eliminate undesired nonadiabatic transitions completely.
%Next, we introduce a scheme of a tunable ZZ coupling between KPOs based on the rotation of a KPO, which gives rise to ZZ gate.
%Then, we apply the method of the fast rotation to switch rapidly the coupling between the KPOs.


The $R_{zz}$ gate using the ideal tunable coupling, $g(t)(a_1a_2^\dagger + a_1^\dagger a_2)$ called beam-splitter type, was studied in Ref.~[\citenum{Goto2016b,Puri2017b,Puri2020}], however without the crucial analysis on the implementation of the coupling. 
($g$ and $a_l$ are the coupling amplitude and annihilation operator of resonator $l$, respectively.) 
\textcolor{black}{The tunable coupling approximating the beam-splitter type was implemented
between superconducting resonators using a transmon as a coupler~\cite{Gao2018}, and
between a KPO and a readout cavity using an additional microwave drive~\cite{Grimm2020}.}
%\sout{The tunable coupling approximating the beam-splitter type was implemented, e.g., for superconducting resonators, 
%and the operation fidelity of approximately 0.98 was presented (the fidelity can be increased to ~0.99 when post-selections are performed for a coupler)~\cite{Gao2018}.
%The implementation of the coupling requires a transmon, which couples the two resonators. The infidelity of the operation is attributed to the decoherence due to unwanted excitations of the transmon.}
\textcolor{black}{$R_{zz}$ gates using an additional drive to either of KPO were also proposed~\cite{Darmawan2021,Chono2022}, which require qubits with different frequencies. }
\textcolor{black}{On the other hand,
our scheme does not require couplers (KPOs can be directly coupled) nor difference between qubit  frequencies. 
}
%Moreover, the proposed adiabatic method can be implemented by the time dependent phase of the pump field, which can be accurately tailored with hardware devices out of the cryostat such as arbitrary waveform generator~\cite{Wang2019}. 

\section{Transitionless rotation of a KPO}
Before introducing the coupling scheme, we first develop the method of the fast transitionless rotation of a KPO used for the coupling scheme.
We consider a KPO of which Hamiltonian is written in a rotating frame as~\cite{Wielinga1993,Cochrane1999,Goto2019}
\begin{eqnarray}
\frac{H(\theta)}{\hbar} = \frac{K}{2}a^{\dagger 2} a^2 - \frac{p}{2} (a^{\dagger 2}e^{2i\theta} + a^2e^{-2i\theta}), 
\label{H_1_24_22}
\end{eqnarray}
where $K$ is the nonlinearity parameter, $p$ and $2\theta$ are the amplitude and phase of the pump field (see also Appendix~\ref{Hamiltonian of a parametron2}).
Hereafter, we assume that $K$ and $p$ are positive for simplicity, although they are negative for realized KPO reported e.g. in Ref.~[\citenum{Wang2019}], because the overall sign of the Hamiltonian is not of physical importance~\cite{Goto2019}.
\textcolor{black}{The nonlinearity of the system was implemented by Josephson junctions, e.g., in Ref.~[\citenum{Meaney2014,Wang2019}]. (The intrinsic non-linearity of disordered superconductors such as granular aluminum~\cite{Winkel2020} is also expected to be used as a source of the nonlinearity.)}
This system has two degenerate ground states represented as $(|\alpha e^{i\theta} \rangle \pm |-\alpha e^{i\theta} \rangle)/\sqrt{2}$ with $\alpha=\sqrt{p/K}$ when $p\gg K$, which are called the even and odd cat states, respectively. 

The phase of the pump field determines the orientation of the Wigner function of energy eigenstates of the KPO~\cite{Puri2020} (see Appendix~\ref{Rotation and disturbances due to nonadiabatic transitions} for definition of the Wigner function).
Figure~\ref{Wig0_11_25_21} shows the Wigner function of the even cat state for $\theta=0$ and $\pi/4$.
%Not only the ground state, but also the orientation of the Wigner function of other energy eigenstates are determined by $\theta$.
As explained later, this phase dependence of KPOs can be used to tune the effective coupling between KPOs.
In order to intuitively understand the phase dependence of KPOs, we consider the effective potential defined by $V(\alpha)=\langle \alpha | H | \alpha \rangle$~\cite{Zhang2017} with a complex variable $\alpha$.
$V(\alpha)$ is represented as 
$V(\alpha)=|\alpha|^2 \Big{(} \frac{K}{2}|\alpha|^2 - p\cos(2(\theta_\alpha - \theta) ) \Big{)}$, 
where $\theta_\alpha={\rm arg}[\alpha]$.
The effective potential is oriented with the increase of $\theta$ as illustrated in 
Fig.~\ref{Wig0_11_25_21}(c) and \ref{Wig0_11_25_21}(d), and
the orientation of the effective potential coincides with the one of the Wigner function.
\begin{figure}[h!]
\begin{center}
\includegraphics[width=8.5cm]{figure1.eps}
\end{center}
\caption{
The Wigner function \textcolor{black}{$W(\xi)$} of the even cat state for $\theta=0$ (a) and $\pi/4$~(b)\textcolor{black}{, where $\xi=x+iy$}.
We set $p/K=7$.
Illustration of effective potential $V(\alpha)$ for $\theta=0$~(c) and $\pi/4$~(d).
}
\label{Wig0_11_25_21}
\end{figure}




The Wigner function can be rotated by changing $\theta$ gradually.
When the rate of change of $\theta$ is sufficiently small, an adiabatic dynamics leads to a simple rotation of the Wigner function.
On the other hand, when the rate of change of $\theta$ is large, the Wigner function is disturbed due to unwanted nonadiabatic transitions as shown in Appendix~\ref{Rotation and disturbances due to nonadiabatic transitions}.

\textcolor{black}{Unwanted nonadiabatic transitions can be eliminated by adding the detuning of which time dependence is designed by the CD protocol~\cite{Rice2003}.
The modified Hamiltonian is represented as
\begin{eqnarray}
H'(t) = H(\theta(t)) - \hbar \dot\theta(t) a^\dagger a,  
%\label{H_mod_11_25_21}
\end{eqnarray}
where dot denotes the time derivative.
The modified Hamiltonian $H'$ is composed of $H(\theta)$ in Eq.~(\ref{H_1_24_22}) and $- \hbar \dot\theta(t) a^\dagger a$, which we call CD term (see Appendix~\ref{Theory of rotation} for the derivation of the CD term).
The additional detuning to eliminate nonadiabatic transitions during the rotation was obtained for constant rate of the rotation for a similar system~\cite{Guillaud2019}. And the same detuning term appears when the Hamiltonian for the transitionless CX gate when projected to a particular state of the control qubit~\cite{Puri2020}. }



%Note that $H_{\theta(t)}$ is not the Hamiltonian which realizes the rotated state.


The CD term can be implemented by the detuning $\Delta$ in KPOs, which is the difference between the resonance frequency of the KPO and half of the pump frequency~\cite{Goto2016,Goto2019} and appears as a term, $\hbar\Delta a^\dagger a$, in the KPO Hamiltonian~\cite{Goto2019}.
%The detuning should be set as $\Delta(t)=-\dot\theta(t)$ because the detuning appears as a term, $\hbar\Delta a^\dagger a$, in Hamiltonian of the KPO~\cite{Goto2019}.
%of which term in Hamiltonian is represented as $\hbar\Delta a^\dagger a$.
The detuning can be tuned by controlling the resonance frequency of the KPO via the magnetic flux~\cite{Wang2019}.
\textcolor{black}{(It is known that the resonance frequency of a superconducting resonator can be modulated half a gigahertz in 1~ns~\cite{ZLWang2013}.)}
\textcolor{black}{The resonance frequency of the KPO is modified as $\omega(t) = - \Delta(t)  + K + \omega_p/2$, where $\omega_p$ is the angular frequency of the pump field (see Appendix~\ref{Hamiltonian of a parametron2} for details).} 
Therefore, controlling the phase of the pump field \textcolor{black}{and the resonance frequency of the KPO} can rotate a KPO without any disturbance.
The performances of the controls with and without the CD term are compared by numerical simulations in Appendix~\ref{Rotation and disturbances due to nonadiabatic transitions}.

%%%%%%%%%%
\section{Fast tunable coupling scheme}
%The effective coupling between KPOs can be controlled via the relative phase of the pumps.
We introduce a scheme of fast tunable coupling for KPOs based on the above transitionless rotation of a single KPO. 
We consider two linearly coupled KPOs with the same resonance frequencies with Hamiltonian 
\begin{eqnarray}
\frac{H_{\rm tot}(t)}{\hbar} &=&  \sum_{l=1}^{2} \Big{[} \frac{K_l}{2}a_l^{\dagger 2} a_l^2 - \frac{p_l}{2} (a_l^{\dagger 2}e^{{2i\theta_l(t)}} + a_l^2e^{{-2i{\theta_l(t)}}}) \Big{]} \nonumber\\
&& + J (a_1a_2^\dagger + a_1^\dagger a_2)
-  \dot\theta_1(t) a_1^\dagger a_1,
\label{H_2KPO_11_25_21}
\end{eqnarray}
where $K_l$ is the nonlinearity parameter, $p_l$ and $2\theta_l$ are the amplitude and phase of the pump field of KPO $l$. Here, $J$ is the fixed coupling coefficient between the KPOs.
We emphasize that the effective coupling between KPOs can be turned off even with fixed $J$ as shown below.
The last term in Eq.~(\ref{H_2KPO_11_25_21}) is the CD term for transitionless rotation of KPO 1.
We, hereafter, assume that $p_l=p$, $K_l=K$ and $\theta_1(t)=\theta(t)$ and $\theta_2=0$ for simplicity.
Note that only the phase of KPO 1 is modulated, while that of KPO 2 is fixed.
\textcolor{black}{We assume that the phase of the pump fields can be precisely controlled in this study.
There might be slow changes of the phase due to phase drifts in actual experiments. Such phase drift can be monitored during the measurement, and the phase can be adjusted before each measurement~\cite{Wang2019}.}

In the parameter regime where $p\gg J$ and $K$,
four states represented by $|\alpha e^{i\theta},\alpha \rangle$, $|\alpha e^{i\theta},-\alpha \rangle$, $|-\alpha e^{i\theta},\alpha \rangle$, $|-\alpha e^{i\theta},-\alpha \rangle$ with $\alpha=\sqrt{p/K}$ are stable due to the exponential suppression of bit-flip rate caused with the increase of $\alpha$ \cite{Puri2019}.
Hereafter, these states are denoted by $|\bar{0},\bar{0}\rangle$, $|\bar{0},\bar{1}\rangle$, $|\bar{1},\bar{0}\rangle$ and $|\bar{1},\bar{1}\rangle$, respectively.
The interaction terms in the Hamiltonian shift the energy of the states because 
\begin{eqnarray}
\langle \bar{0},\bar{0} (\bar{1},\bar{1})| (a_1 a_2^\dagger + a_1^\dagger a_2) |  \bar{0},\bar{0} (\bar{1},\bar{1}) \rangle
&=& 2|\alpha|^2 \cos\theta, \nonumber\\
\langle \bar{0},\bar{1} (\bar{1},\bar{0}) | (a_1 a_2^\dagger + a_1^\dagger a_2) |  \bar{0},\bar{1} (\bar{1},\bar{0}) \rangle
&=& - 2|\alpha|^2 \cos\theta 
%\langle \pm \alpha e^{i\theta},  \pm \alpha  | a_1 a_2^\dagger + a_1^\dagger a_2 |  \pm\alpha e^{i\theta},  \pm\alpha \rangle
%&=& 2|\alpha|^2 \cos\theta, \nonumber\\
%\langle \pm \alpha e^{i\theta},  \mp \alpha  | a_1 a_2^\dagger + a_1^\dagger a_2 |  \pm\alpha e^{i\theta},  \mp\alpha \rangle
%&=& - 2|\alpha|^2 \cos\theta \nonumber\\
\label{Int_11_29_21}
\end{eqnarray}
while off-diagonal elements, such as 
%$\langle \alpha e^{i\theta},  \alpha  | a_1 a_2^\dagger + a_1^\dagger a_2 |  - \alpha e^{i\theta}, \alpha \rangle$
$\langle \bar{0},  \bar{0}  | a_1 a_2^\dagger + a_1^\dagger a_2 |  \bar{0},  \bar{1} \rangle$, are negligible (note that $\langle -\alpha| \alpha \rangle\simeq 0$).
Importantly, the shift of the energies can be controlled via $\theta$ as seen in Eq.~(\ref{Int_11_29_21}), and the shift of the energies becomes zero when $\theta = \pi/2 + \pi n$, where $n$ is an integer.
Thus, the effective coupling between KPOs can be tuned and even turned off. 
\textcolor{black}{(This controllability of the effective coupling is not lost even in systems with asymmetry between KPOs, as explained in Appendix~\ref{Asymmetry}.)}
A pulsed effective coupling can be used to perform a $R_{zz}$ gate as shown bellow. 
%\textcolor{black}{sign of interaction can be changed. Maybe dynamical decoupling type can be used.}



We consider the phase of the pump field
%We numerically evaluate the performance of the coupling scheme with $\theta$
controlled for $0 \le t \le T$ as $\theta(t) = \frac{\pi}{2} - \theta_{\rm amp} \pi [1-\cos(2\pi t/T)]$,
%\begin{eqnarray}
%\theta(t) = \frac{\pi}{2} - \theta_{\rm amp} \pi [1-\cos(2\pi t/T)],
%\label{theta_11_29_21}
%\end{eqnarray}
where $\theta_{\rm amp}$ is a constant parameter, which determines maximum strength of the effective coupling during the control.
$\theta$ is chosen to be $\pi/2$ at $t=0$ and $T$ so that the effective coupling is off at the initial and final times of the control. 
We assume that the initial state is one of the states, $|\bar{0},\bar{0}\rangle$, $|\bar{0},\bar{1}\rangle$, $|\bar{1},\bar{0}\rangle$ and $|\bar{1},\bar{1}\rangle$.
%$|\alpha e^{i\pi/2}, \alpha\rangle$, $|\alpha e^{i\pi/2},-\alpha \rangle$, $|-\alpha e^{i\pi/2},\alpha \rangle$, $|-\alpha e^{i\pi/2},-\alpha \rangle$.
%Hereafter, we denote these states as $|R,R\rangle$, $|R,L\rangle$, $|L,R\rangle$ and $|L,L\rangle$, respectively.
For sufficiently large $T$, the state of the system evolves adiabatically from $|i,j\rangle$ to $e^{i\varphi_{ij}}|i,j\rangle$, where $i,j=\bar{0},\bar{1}$.
Here, $\varphi_{ij}$ is the dynamical phase at $t=T$ due to the energy shift, and is written 
%The energy shift caused by the coupling terms gives rise to a dynamical phase $\theta_{ij}$ at $t=T$, and the energy eigenstates change from $|i,j\rangle$ to $e^{i\theta_{ij}}|i,j\rangle$, where $i,j=R,L$ and
\begin{eqnarray}
\varphi_{ij} = \left\{
\begin{array}{cl}
2J|\alpha|^2\int_0^T  \cos\theta(t) dt &  {\rm for} \ i=j, \\
-2J|\alpha|^2\int_0^T \cos\theta(t) dt & {\rm for} \ i\ne j
\end{array}
\right.
\label{theta_1_6_22}
\end{eqnarray}
\textcolor{black}{(see Appendix~\ref{Asymmetry} for the case that there is asymmetry between the KPOs)}.
Thus, we can perform $R_{zz}$ gates simply by controlling the phase of the pump field.
%In equation (\ref{theta_1_6_22}), we assumed that the energy of $|i,j\rangle$ is zero if $J=0$ for simplicity.
\textcolor{black}{The dynamical phase increases with the rate of $2J|\alpha|^2\cos\theta(t)$.
The maximum value of $|\varphi_{ij}|$ is $2J|\alpha|^2T$.
In other word, the minimum duration for $\varphi_{ij}$ to be realized is given by $T=|\varphi_{ij}|/2J|\alpha|^2$.}
When $T$ is sufficiently large, the CD term is not necessary because it is
proportional to $\dot{\theta}$, and therefore 
 much smaller than the other parameters.
However, for the small $T$ regime where $\dot\theta$ is comparable to or greater than the other parameters, the final state considerably deviates from $e^{i\varphi_{ij}}|i,j\rangle$ due to nonadiabatic transitions unless the CD term is used.
%there are undesired nonadiabatic transitions in the control without CD term.
%Importantly, the nonadiabatic transitions are mitigated by the CD term as shown below.

In order to compare the performance of the controls with and without the CD term,
we numerically simulate the dynamics with the initial state of $|\Psi(0)\rangle=|i,j\rangle$ and $|\Psi_s\rangle\equiv\sum_{i,j}|i,j\rangle/\sqrt{4}$, and obtain the fidelity  defined by \textcolor{black}{$F=$}$|\langle \Psi_{\rm ideal} | \Psi(T)\rangle|^2$, where $|\Psi_{\rm ideal}\rangle=U_{\rm ideal}|\Psi(0)\rangle$, and $U_{\rm ideal}$ denotes the operator representing the ideal gate operation, e.g., $U_{\rm ideal}|i,j\rangle =  e^{i\varphi_{ij}}|i,j\rangle$.
Figure~\ref{fid_T_1_12_22} shows the dependence of the \textcolor{black}{infidelity} on $T$ for the controls with and without the CD term for $\theta_{\rm amp}=0.1$ \textcolor{black}{for the initial state of $|\Psi_s\rangle$}. 
\textcolor{black}{ The results for the initial state of $|i,j\rangle$ are approximately the same as the ones for the initial state of $|\Psi_s\rangle$.}
The fidelity of the control without the CD term (purely adiabatic scheme) is degraded by the nonadiabatic transitions as $T$ decreases. 
On the other hand, the fidelity of the control with the CD term is approximately unity in such small $T$ regime.
For example, the fidelity of the control with the CD term averaged over the initial states is approximately 0.9995, while the averaged fidelity of the control without the CD term is less than 0.89 for $T=K^{-1}$.
The fidelity of the control with the CD term slightly decreases from unity as $T$ decreases. 
We attribute this to the fact that the CD term is designed for transitionless rotation of an individual KPO ($J=0$), and therefore there is finite nonadiabatic transitions for $J\ne 0$.
However, it is noteworthy that the CD term can work well also for the case with $J\ne 0$.
%The fidelity of the control without the CD term tends to increase with $T$ when $T$ is sufficiently large (see $T\ge 1.3K^{-1}$)  due to the decrease of  the nonadiabatic transitions.}
%This is because the nonadiabatic transitions disturb the state in the control without the CD term in such small $T$ regime.
\textcolor{black}{
The infidelity for both the controls decreases in the short-$T$ regime. 
We attribute this to the followings: the control duration is so short that the deviation of the final state of the KPO from its initial state is moderate; the phase in Eq.~(\ref{theta_1_6_22}) to be imprinted on the state of the KPO at $t=T$ is small for short $T$, and therefore the target state $|\Psi_{\rm ideal}\rangle$ is approximately the same as the initial state.}

\begin{figure}[h!]
\begin{center}
\includegraphics[width=7cm]{figure2.eps}
\end{center}
\caption{
\textcolor{black}{Dependence of the infidelity on $T$ for the controls with and without the CD term for the initial state of $|\Psi_s\rangle$.
The used parameters are $p/K=7$, $J/K=0.2$ and $\theta_{\rm amp}=0.1$.
The dashed lines are guides to eyes.
}
%The solid and dashed lines are guides to eyes.
}
\label{fid_T_1_12_22}
\end{figure}

Figure~\ref{fid_sft1_11_30_21}(a) shows the dependence of the \textcolor{black}{infidelity} on $\theta_{\rm amp}$ for the controls with $T=K^{-1}$ \textcolor{black}{for the initial state of $|\Psi_s\rangle$}. 
\textcolor{black}{ The results for the initial state of $|i,j\rangle$ are approximately the same as the ones for the initial state of $|\Psi_s\rangle$.}
It is seen that the fidelity of the control with the CD term is much higher than that of the control without the CD term for large $\theta_{\rm amp}$.
\textcolor{black}{The fidelity of the control with the CD term decreases, as well as the control without the CD term, with the increase of $\theta_{\rm amp}$. This is attributed to the fact that the CD term is exact only for the case of $J=0$, and that nonadiabatic transitions increase with $\theta_{\rm amp}$.}
%For example, the fidelity of the control with the CD term averaged over the initial states is approximately 0.999, while the averaged fidelity of the control without the CD term is less than 0.89 for $\theta_{\rm amp}=0.1$.
%The fidelity of the control with the CD term slightly decreases from unity as $\theta_{\rm amp}$ increases. 
%We attribute this to the fact that the CD term is designed for transitionless rotation of an individual KPO ($J=0$), and thus there is finite nonadiabatic transitions for $J\ne 0$.
%However, it is noteworthy that the CD term designed for an single KPO can work well also for the case with $J\ne 0$.
We numerically obtain phase $\varphi_{ij}$, which the system acquires during the control with the CD term, and compare it with the analytic one in Eq.~(\ref{theta_1_6_22}).
We define the phase at $t=T$ as $\varphi_{ij} = \arg [ \langle ij | \Psi(T)\rangle ]$
for the simulation with the initial state, $| \Psi(0)\rangle = |i,j\rangle$. 
Figure~\ref{fid_sft1_11_30_21}(b) shows the dependence of $\varphi_{ij}$ on $\theta_{\rm amp}$ for $T=K^{-1}$.
The relative phases, $\varphi_{\bar{1}\bar{0}}-\varphi_{\bar{0}\bar{0}}$ and $\varphi_{\bar{0}\bar{1}}-\varphi_{\bar{0}\bar{0}}$, monotonically decrease with $\theta_{\rm amp}$ in the used range of $\theta_{\rm amp}$, while $\varphi_{\bar{1}\bar{1}}-\varphi_{\bar{0}\bar{0}}$ is approximately zero.
It is seen that numerical results agree well with Eq.~(\ref{theta_1_6_22}).
%Small discrepancy comes from that the deviation of the state from ideal one.
\begin{figure}[h!]
\begin{center}
\includegraphics[width=7cm]{figure3.eps}
\end{center}
\caption{
(a) Dependence of the \textcolor{black}{infidelity} on $\theta_{\rm amp}$ for the controls with and without the CD term \textcolor{black}{for the initial state of $|\Psi_s\rangle$}. 
%The labels, $ij$ and $\Psi_s$, on the figure indicate the results of the control without the CD term with initial state $|ij\rangle$ and $|\Psi_s\rangle$, respectively. 
%The purple squares are for the control with CD term, and the data points for all initial states are almost overlapping. 
The dashed lines are guides to eyes.
(b) Dependence of $\varphi_{ij}$ on $\theta_{\rm amp}$ of the control with the CD term.
The solid curve represents $\varphi_{\bar{0}\bar{1}(\bar{1}\bar{0})}$ in Eq.~{(\ref{theta_1_6_22})}.
The used parameters are $p/K=7$, $J/K=0.2$ and $T=K^{-1}$.
}
\label{fid_sft1_11_30_21}
\end{figure}

%\textcolor{black}{
%In Fig.~\ref{fid_sft1_11_30_21}(a), the fidelity slightly decreases with the increase of the $\varphi_{ij}-\varphi_{00}$. This is also attributed to the fact that the CD term is exact only for the case of $J=0$, and that nonadiabatic transitions increases due to large $\dot\theta$.
%}

%Figure~\ref{fid_T_1_12_22} shows the dependence of the fidelity on $T$ for the controls with and without the CD term. 
%The fidelity of the control with the CD term is much higher than the control without the CD term for $T<1.4K^{-1}$.
%The fidelity of the control without the CD term tends to increase with $T$ when $T$ is sufficiently large (see $T\ge 1.3K^{-1}$)  due to the decrease of  the nonadiabatic transitions.
%This is because the nonadiabatic transitions disturb the state in the control without the CD term in such small $T$ regime.

\textcolor{black}{Figure~\ref{fid_fin_com_6_18_22_2} shows the $T$-dependence of the infidelity 
of $R_{zz}$ gate with $(\varphi_{\bar{1}\bar{0}} - \varphi_{\bar{0}\bar{0}})/\pi=-0.5$ for various amplitude of the pump field, $p$.
The value of  $\theta_{\rm amp}$ was chosen so that $(\varphi_{\bar{1}\bar{0}} - \varphi_{\bar{0}\bar{0}})/\pi=-0.5$.
It is seen that the CD term considerably increases the fidelity in the parameter range studied.
The fidelity also increases with respect to $p$ because the nonadiabatic transitions are mitigated for larger pump amplitude due to the increase of the gap between energy levels~\cite{Masuda2020}.}
\begin{figure}[h!]
\begin{center}
\includegraphics[width=7cm]{figure4.eps}
\end{center}
\caption{
\textcolor{black}{
Dependence of the infidelity on $T$ for the controls with and without the CD term 
for various values of $p$.
The initial state is $|\Psi_s\rangle$ and $J/K=0.2$.
The value of  $\theta_{\rm amp}$ was chosen so that  $(\varphi_{\bar{1}\bar{0}} - \varphi_{\bar{0}\bar{0}})/\pi=-0.5$.
The solid and dashed lines are guides to eyes.
}
}
\label{fid_fin_com_6_18_22_2}
\end{figure}
\textcolor{black}{The performance of our coupling scheme is compared with the one based on the ideal tunable coupling with the form of $g(t)(a_1a_2^\dagger + a_1^\dagger a_2)$ in Appendix~\ref{Beam splitter type}.}


\textcolor{black}{
As exemplified in Fig.~\ref{fid_fin_com_6_18_22_2}, the gate fidelity can be increased by increasing control duration $T$.
However, the gate fidelity is decreased  for larger $T$ when there is the decoherence.
We examine the performance of the controls under the effect of the decoherence using the master equation:
\begin{eqnarray}
\frac{d\rho(t)}{dt} &=& -\frac{i}{\hbar}[H_{\rm tot}(t),\rho(t)] + \mathcal{L}[\rho(t)],\nonumber\\
\mathcal{L}[\rho] &=& \sum_l \frac{\kappa_l}{2} ([a_l\rho,a_l^\dagger] + [a_l,\rho a_l^\dagger])\nonumber\\
&&+\gamma_p^{(l)} ([a_l^\dagger a_l\rho,a_l^\dagger a_l] + [a_l^\dagger a_l,\rho a_l^\dagger a_l]), 
\label{ME_6_28_22}
\end{eqnarray}
where $\kappa_l$ and $\gamma_p^{(l)}$ are the decay and dephasing rates of KPO $l$.
We assume that $\gamma_p^{(l)},\kappa_l=\kappa$ in numerical simulations for simplicity (the effect of  the pure decay and pure dephasing are examined in Appendix~\ref{Robustness}).
Figure~\ref{fid_fin_com_kappa_6_18_22} shows the dependence of the infidelity on $T$ for $\kappa=10^{-3}K$ and $10^{-4}K$.
It is seen that the fidelity of the control with the CD term is greatly improved in the short-$T$ regime compared to that of the control without CD term, while the efficiency of both the controls are degraded for large $T$.
This result shows that the control with the CD term allows fast $R_{zz}$ gates with high fidelity avoiding unwanted effects of the decoherence and nonadiabatic transitions.
}
\textcolor{black}{
At the minimums and a plateau of the infidelity seen in Fig.~\ref{fid_fin_com_kappa_6_18_22}, the mitigation of nonadibatic transitions 
accompanied with the increase of $T$ is balancing with the effect of decoherence.
}

\begin{figure}[h!]
\begin{center}
\includegraphics[width=7cm]{figure5.eps}
\end{center}
\caption{
\textcolor{black}{
Dependence of the infidelity on $T$ for the controls with and without the CD term 
for $J/K=0.2$, $\kappa=10^{-3}K$ and $10^{-4}K$ for $p/K=5$ (a) and  $p/K=7$ (b).
The initial state is $|\Psi_s\rangle$.
The value of  $\theta_{\rm amp}$ was chosen so that  $(\varphi_{\bar{1}\bar{0}} - \varphi_{\bar{0}\bar{0}})/\pi=-0.5$.
The solid and dashed lines are guides to eyes.
}
}
\label{fid_fin_com_kappa_6_18_22}
\end{figure}

\textcolor{black}{
Because the detuning of a KPO opens the energy gap between energy levels~\cite{Goto2016b}, imperfection of the control of the detuning can disturb the state of the KPO.  
The robustness of our scheme against the error in the CD term is examined in Appendix~\ref{Robustness}.}


\section{Conclusions and discussions}
A fast tunable $ZZ$ coupling scheme of KPOs has been developed using the transitionless rotation of a KPO in the phase space based on the CD protocol.
The effective coupling between KPOs can be turned off even with always-on linear coupling between the resonators constituting the KPOs. 
We have examined the performance of our scheme applying it to $R_{zz}$ gate, and compared with the results of a purely adiabatic scheme, which utilizes only a controlled phase of the pump field.
It has been shown that our scheme greatly enhances the fidelity of $R_{zz}$ gate compared to the adiabatic scheme by eliminating undesired nonadiabatic transitions, when applied in a short time. 


\textcolor{black}{
%The adiabatic scheme can be implemented simply by the time dependent phase of the pump field, which can be accurately tailored with hardwares out of the cryostat such as arbitrary waveform generator, and does not require additional microwave drives to KPOs nor controlling the amplitudes of the linear coupling between KPOs. 
The CD protocol can be realized by the time dependent detuning implemented by controlling the resonance frequency of the KPO. 
Because time dependent detuning can be used also for the $R_x$ gate~\cite{Goto2016b}, our scheme is compatible with the $R_x$ gate in the sense that the use of the time dependent detuning will not add extra experimental equipment and tasks, such as calibration, to complete the set of universal gates. }

\textcolor{black}{A KPO can be loaded to its ground state, which is a cat state, from the vacuum state by adiabatically ramping the pump amplitude~\cite{Cochrane1999,Goto2016}.
We examine the efficiency of the adiabatic loading scheme when the effective coupling between the KPOs is off in Appendix~\ref{Loading into ground state}.}

While we are preparing our manuscript, we came to know that other group independently studied the tunability of the effective coupling solely by the phase of the pump field~\cite{NEC_paper}. 
However, this method has recourse to an adiabatic evolution of the system and, therefore, is not suitable for fast tuning of the coupling. Our scheme resolves the shortcoming of the adiabatic scheme. 


\textcolor{black}{A comment on the readout is in order. 
We assume that each KPO is read using the output field from the KPO through a different readout transmission line. When the coupling between the KPOs are off, the state of one KPO is a superposition of $|\alpha\rangle$ and $|-\alpha\rangle$ while that of the other is a superposition of $|i\alpha\rangle$ and $|-i\alpha\rangle$. Even if there is a small leakage from KPO~1 to the readout transmission line for KPO~2, they can be distinguished because of the phase difference of $\pi/2$.}
\textcolor{black}{Reading out a KPO by coupling it to a transmission line may shorten the lifetime of the cat states. Such unwanted effect will be mitigated by using a readout cavity attached to the KPO~\cite{Grimm2020}.
}

Before closing, we point out that our coupling scheme will find wider applications in quantum technologies, although we particularly applied to $R_{zz}$ gate in this paper to demonstrate the effectiveness of the scheme.
For example, the coupling scheme can be useful for quantum annealing and quantum simulation in which time dependent qubit-qubit couplings are utilized.
Our scheme is used to decrease undesired population transfers out of the qubit space caused by the rotation of a KPO (not nonadiabatic transitions of the whole system which may be caused by time dependent effective coupling), and therefore has a different motivation from other studies based on the STA which consider ideal spin models and aim decreasing nonadiabatic transitions of the model systems~\cite{delCampo2012,Damski2014,Okuyama2016,Sels2017,Setiawan2019}.
Our scheme can be implemented by the simple manner and even independent of energy-level structure of the system. Performance of our scheme in quantum annealing and quantum simulation deserves further quantitative investigation. 


\begin{acknowledgments}
It is a pleasure to acknowledge discussions with T. Yamamoto.
This paper is partly based on results obtained from a project, JPNP16007, commissioned by the New Energy and Industrial Technology Development Organization (NEDO), Japan. 
S.M. acknowledges the support from JSPS KAKENHI (grant number 18K03486). 
Y. M. was supported by MEXT's Leading Initiative for Excellent Young Researchers and JST PRESTO (Grant No. JPMJPR1919), Japan.
\end{acknowledgments}

\appendix
%\section{Hamiltonian and implementation of time dependent detuning}
%\label{Realization of time dependent detuning}
%\textcolor{black}{This section is deleted.}
%We consider the case that the frequency of the pump field is time dependent.
%Hamiltonian of a KPO in a lab frame is represented as~(see e.g.,  \cite{Wang2019} for the case with fixed frequency of the pump field)
%\begin{eqnarray}
%\frac{H}{\hbar} &=& \omega_0 a^\dagger a + \frac{K}{12}(a^\dagger + a)^4 \nonumber\\
%&& - p(a^\dagger+a)^2\cos\Big{(}\int_0^t\omega_p(t')dt' - 2\theta(t) \Big{)},
%\end{eqnarray}
%where $p$ and $2\theta$ are the amplitude and phase of the pump field, $\omega_0$ and $K$ are the resonance angular frequency and nonlinear parameter of the resonator constituting the KPO.
%We move on a rotating frame defined by an unitary operator
%$U(t) = \exp\Big{[} \frac{i\int_0^t\omega_p(t')dt'}{2}a^\dagger a\Big{]}$,
%and omit non-resonant rapidly oscillating terms (the rotating wave approximation) to obtain 
%\begin{eqnarray}
%\frac{H}{\hbar} &=& \Delta(t) a^\dagger a +
%\frac{K}{2}a^{\dagger 2} a^2 \nonumber\\
%&&- \frac{p}{2} (a^{\dagger 2}e^{2i\theta(t)} + a^2e^{-2i\theta(t)}),
%\label{H_2_7_22}
%\end{eqnarray}
%where we assumed that $\dot{\theta}(t)$ is much smaller than $\omega_p(t)$ for any $t$.
%The detuning $\Delta$ is written as
%\begin{eqnarray}
%\Delta(t) = \omega_0 - K - \omega_p(t)/2.
%\label{Delta_2_17_22}
%\end{eqnarray}
%Hamiltonian in Eq.~(\ref{H_2_7_22}) is the same as Eq.~(\ref{H_1_24_22}) when $\Delta(t)=0$.
%As seen in Eq.~(\ref{Delta_2_17_22}), the time dependent detuning can be implemented by the time dependent frequency of the pump field.

\textcolor{black}{
\section{Hamiltonian of a KPO}
\label{Hamiltonian of a parametron2}}
\textcolor{black}{
Although a derivation of an effective Hamiltonian for a KPO was shown in Ref.~\cite{Wang2019},
we present it to make this paper self-contained.
We consider a KPO composed of a SQUID-array resonator with $N$ SQUIDs illustrated in  Fig.~\ref{KPO_system_6_1_20}.
The effective Hamiltonian of the system is represented as
\begin{eqnarray}
H= 4E_C n^2 - NE_J[\Phi(t)] \cos\frac{\phi}{N},
\label{H_KPO_4_16_20}
\end{eqnarray}
where $\phi$, $n$, $E_J$ and $E_C$ are the overall phase across the junction array, its conjugate variable and the Josephson energy of a SQUID, respectively. 
$E_C$ is the charging energy of the resonator, including the contributions of the junction capacitances $C_J$ and the shunt capacitance $C$.
We assume that all the Josephson junctions are identical.
The effective Hamiltonian (\ref{H_KPO_4_16_20}) with a single degree of freedom, $\phi$, is valid provided that $E_J$ is much larger than the charging energy of a single junction~\cite{Frattini2017,Noguchi2020}.
%$E_C$ is the charging energy of the resonator, including the contributions of the junction capacitances $C_J$ and the shunt capacitance $C$, and can be experimentally extracted or calculated by finite-element capacitance simulation~\cite{Wang2019}. 
The Josephson energy can be modulated as $E_J(t)=E_J+\delta E_J \cos\omega_p t$ by the time-dependent external magnetic flux, $\Phi(t)$, threading the SQUIDs.
For simplicity, we set the phase of the pump field to be zero, $\theta=0$.
}

\begin{figure}
\begin{center}
\includegraphics[width=4.5cm]{figure6.eps}
\end{center}
\caption{
\textcolor{black}{
Circuit model of a KPO consisting of $N$~SQUIDs and a shunt capacitor $C$. 
$\phi$ is the overall phase across the junction array;  $\Phi(t)$ is the external magnetic flux threading the SQUIDs; $E_J$ and $C_J$ are the Josephson energy of a single SQUID and the capacitance of a single Josephson junction, respectively.  
}
}
\label{KPO_system_6_1_20}
\end{figure}



\textcolor{black}{
We can obtain an approximate Hamiltonian by taking into account up to the fourth order of $\phi/N$ in equation~(\ref{H_KPO_4_16_20}) as
\begin{eqnarray}
\frac{H}{\hbar} &=& \omega \Big{(} a^\dagger a + \frac{1}{2} \Big{)}
- \frac{K}{12} (a + a^\dagger)^4
\nonumber\\
 && + \Big{[} - \frac{N\delta E_J}{\hbar}  +
 p (a + a^\dagger)^2  - \frac{K p}{3\omega} (a + a^\dagger)^4 \Big{]}\nonumber\\
 && \times \cos\omega_p t,
\label{H_9_1_20}
\end{eqnarray}
where $\omega = \frac{1}{\hbar}\sqrt{8E_CE_J/N}$, $K=E_C/\hbar N^2$ and $p = 2\omega \delta E_J/ 8E_J$. 
%Here, $p$ is called amplitude of the pump field in the main text.
$n$ and $\phi$ are related to the annihilation operator $a$ as
$n = -in_0(a-a^\dagger)$ and $\phi = \phi_0 (a+a^\dagger)$
with $n_0^2=\sqrt{E_J/32 N E_C}$ and $\phi_0^2 = \sqrt{2NE_C/E_J}$.
Above, we considered the parameter regime, where $\phi_0/N = 2\sqrt{K/\omega}$ is sufficiently smaller than unity so that the expansion of $\cos(\phi/N)$ is valid,
and took into account up to the fourth order of $\phi/N$ to see the effect of the Kerr nonlinearity.
In equation~(\ref{H_9_1_20}), we neglect the last term assuming that $Kp\ll  \omega$, and drop c-valued terms to obtain the following Hamiltonian
\begin{eqnarray}
\frac{H}{\hbar} = \omega a^\dagger a 
- \frac{K}{12} (a + a^\dagger)^4
+ p (a + a^\dagger)^2
\cos\omega_p t.
\end{eqnarray}
Moving to the rotating frame at the frequency of $\omega_p/2$ and using the rotating wave approximation, we obtain 
\begin{eqnarray}
\frac{H}{\hbar} = -\Delta a^\dagger a - \frac{K}{2}a^{\dagger 2} a^2 + \frac{p}{2}(a^{\dagger 2} + a^2),
\end{eqnarray}
where $\Delta=-\omega + K + \omega_p/2$.
Puting $\Delta=0$ and changing the sign of the Hamiltonian we can obtain Eq.~(\ref{H_1_24_22}) for $\theta=0$.
}
\textcolor{black}{When we take into account higher order terms with respect to $\phi$, we have higher order terms with respect to $a$ such as the term proportional to  $\frac{pK}{\omega}a^\dagger a^3$.
The higher order terms can be neglected when $p$ and $K$ are much smaller than $\omega$. We consider such parameter regimes throughout this paper.
We also refer readers to Ref.~[\citenum{Puri2017b}], which studied the effect of higher-order terms.}


\section{Performance of rotation schemes for single KPO}
\label{Rotation and disturbances due to nonadiabatic transitions}
We compare the performance of the rotation schemes with and without the CD term.
As an example, we consider the case that $\theta$ is increased from 0 to $\pi/2$ for $0\le t\le T$ as
\begin{eqnarray}
\theta(t) = \frac{\pi}{4}\Big{[} 1- \cos\Big{(} \frac{\pi t}{T} \Big{)} \Big{]}.
\label{theta0_11_29_21}
\end{eqnarray}
The initial state is a ground state well approximated by $(|\alpha\rangle + |-\alpha\rangle)/\sqrt{2}$, where $\alpha=\sqrt{p/K}$. 
The Wigner function of the initial state is presented in Fig.~\ref{Wig0_11_25_21}(a).
The Wigner function is defined by $W(\xi)=\frac{2}{\pi}{\rm Tr}[D(-\xi)\rho D(\xi)P]$, with $\xi=x+iy$, density operator $\rho$, displacement operator $D(\xi)=\exp(\xi a^\dagger - \xi^\ast a)$ and parity operator $P={\rm exp}(i\pi a^\dagger a)$~\cite{Leonhardt1997,Deleglise2008,Goto2016}.
We fix $p$ and $K$, while $\theta$ is changed during the control. %We neglect any effects of decay assuming that $1/T$ is sufficiently shorter than the decay rate.

%When $T$ is sufficiently large, an adiabatic dynamics leads a simple rotation of the Wigner function, and the disturbance in the Wigner function is negligible.
%On the other hand, when $T$ is small, the Wigner function is disturbed due to unwanted nonadiabatic transitions.
Figures~\ref{Wig_11_25_21}(a) and \ref{Wig_11_25_21}(b) show the Wigner function at $t=T$ for $T=0.6K^{-1}$ and $1.5K^{-1}$, respectively, for the control without the CD term.
The Wigner function at $t=T$ is disturbed due to nonadiabatic transitions for $T=0.6K^{-1}$, while
the Wigner function is almost ideally rotated for $1.5K^{-1}$ because the system evolves almost adiabatically.
\begin{figure}[h!]
\begin{center}
\includegraphics[width=8cm]{figure7.eps}
\end{center}
\caption{
The Wigner function of the final state of the control without the CD term
for $T=0.6K^{-1}$ (b) and $T=1.5K^{-1}$.
The used parameters are the same as Fig.~\ref{Wig0_11_25_21}.
}
\label{Wig_11_25_21}
\end{figure}

We define the fidelity of the control as $|\langle\Psi_{\theta(T)}|\Psi(T)\rangle|^2$, where $|\Psi(T)\rangle$ and $|\Psi_{\theta(T)}\rangle$ are the final state of the control and the state ideally rotated by angle $\theta(T)$, respectively.
Figure~\ref{fid_com_11_28_21} shows the $T$-dependence of the fidelity for both the controls.
In the control without the CD term, the fidelity is degraded due to nonadiabatic transitions for small $T$, while the fidelity becomes close to unity for sufficiently large $T$.
\begin{figure}[h!]
\begin{center}
\includegraphics[width=7cm]{figure8.eps}
\end{center}
\caption{
$T$-dependence of the fidelity of a rotation of a KPO.
The black circles and red crosses correspond to the controls with and without the CD term.
The inset shows the time dependence of the detuning $\Delta(t) = -\dot\theta(t)$ for $T=0.6K^{-1}$ in the control with the CD term.
}
\label{fid_com_11_28_21}
\end{figure}
On the other hand, the fidelity of the control with the CD term is unity.
The inset of Fig.~\ref{fid_com_11_28_21} shows the time dependence of the detuning $\Delta(t) = -\dot\theta(t)$ for $T=0.6K^{-1}$.


\textcolor{black}{
\section{Derivation of CD term for rotation of a KPO in phase space}
\label{Theory of rotation}
}
%The Wigner function of any energy eigenstates rotates with the phase of the pump field $\theta$.
The rotation of a KPO is characterized by operator $U$ defined by
\begin{eqnarray}
U(\theta) = e^{i\theta a^\dagger a}.
\label{U_11_29_21_2}
\end{eqnarray}
$U(\theta)$ rotates a state of a KPO in the $\alpha$ space.
This fact is easily confirmed by letting $U$ act on coherent state $|\alpha\rangle$ to obtain
\begin{eqnarray}
a U(\theta) |\alpha\rangle = \alpha e^{i\theta} U(\theta) |\alpha\rangle,
\end{eqnarray}
where we used $U^\dagger(\theta) a U(\theta)=a e^{i\theta}$.



Suppose that $|\phi_m\rangle$ is $m$th eigenstate of $H(0)$ with eigenenergy $E_m$.
The time independent Schr\"{o}dinger equation is written as
\begin{eqnarray}
H(0) |\phi_m\rangle = E_m|\phi_m\rangle.
\end{eqnarray}
Then, we can obtain
\begin{eqnarray}
H(\theta) U(\theta) |\phi_m\rangle = E_m U(\theta) |\phi_m\rangle,
\end{eqnarray}
where we used 
\begin{eqnarray}
H(\theta) = U(\theta)  H({0}) U^\dagger(\theta).
\label{Htheta_11_29_21}
\end{eqnarray}
%Here, $H_0$ denotes $H_{\theta=0}$.
The above discussion shows that if $|\phi_m\rangle$ is an eigenstate of $H(0)$,
$U(\theta)|\phi_m\rangle$, which is a rotated state by $\theta$, is an eigenstate of $H(\theta)$.
This fact is independent of energy eigenstates.
Therefore, we can rotate an arbitrary state of a KPO by adiabatically changing $\theta$.

\textcolor{black}{Now we derive a modified Hamiltonian, which realizes an ideal rotation without nonadiabatic transitions, using the CD protocol~\cite{Rice2003}. 
We consider a dynamics in the system with $\theta=0$ as a reference.
Suppose that $|\Psi(t)\rangle$ is a solution of the Schr\"{o}dinger equation
\begin{eqnarray}
i\hbar \frac{d}{dt} |{\Psi}(t)\rangle = H(0)|\Psi(t)\rangle,
\label{SE_11_29_21}
\end{eqnarray}
where $H(0)$ denotes $H(\theta=0)$.
The state rotated by $\theta(t)$ is represented as $U(\theta(t))|\Psi(t)\rangle$.
We can straightforwardly obtain the relation
\begin{eqnarray}
i\hbar\frac{d}{dt} \big\{ U(\theta(t))|\Psi(t)\rangle \big\} = H' U(\theta(t)) |\Psi(t)\rangle,
\end{eqnarray}
with
\begin{eqnarray}
H' = H(\theta(t)) - \hbar \dot\theta(t) a^\dagger a,   
%\label{H_mod_11_25_21}
\end{eqnarray}
where we have used Eqs. (\ref{U_11_29_21_2}), (\ref{SE_11_29_21}), (\ref{Htheta_11_29_21}) and $U^\dagger (\theta(t)) U (\theta(t)) =1$.
The rotated state $U(\theta(t))|\Psi(t)\rangle$ is a solution of the Schr\"{o}dinger equation corresponding to Hamiltonian $H'$ composed of $H(\theta)$ in Eq.~(\ref{H_1_24_22}) and $- \hbar \dot\theta(t) a^\dagger a$, which we call CD term.
}

\textcolor{black}{
\section{Asymmetries in system}
\label{Asymmetry}
}
\textcolor{black}{We consider the case that the two KPOs are not identical.
Although the amplitude and phase of the pump fields can be externally tuned, it is difficult to exactly set the value of $K_l$ due to imperfections of the fabrication.
In order to examine the effect of the asymmetry in $K_l$, we set $K_1=K$, $K_2=rK$, $p_1=p_2=p$, where
$r$ is the constant parameter characterizing the asymmetry in $K_l$.}

\textcolor{black}{
The four states, represented by $|\alpha_1 e^{i\theta},\alpha_2 \rangle$, $|\alpha_1 e^{i\theta},-\alpha_2 \rangle$, $|-\alpha_1 e^{i\theta},\alpha_2 \rangle$, $|-\alpha_1 e^{i\theta},-\alpha_2 \rangle$ with $\alpha_1=\sqrt{p/K}$ and $\alpha_2=\sqrt{p/rK}$, are stable due to the exponential suppression of bit-flip rate when $\alpha_l$ is sufficiently large.
These states are denoted by $|\bar{0},\bar{0}\rangle$, $|\bar{0},\bar{1}\rangle$, $|\bar{1},\bar{0}\rangle$ and $|\bar{1},\bar{1}\rangle$, respectively.
Then, the counterpart of Eq.~(\ref{Int_11_29_21}) is represented as 
\begin{eqnarray}
\langle \bar{0},\bar{0} (\bar{1},\bar{1})| (a_1 a_2^\dagger + a_1^\dagger a_2) |  \bar{0},\bar{0} (\bar{1},\bar{1}) \rangle
&=& \frac{2p}{\sqrt{r}K} \cos\theta, \nonumber\\
\langle \bar{0},\bar{1} (\bar{1},\bar{0}) | (a_1 a_2^\dagger + a_1^\dagger a_2) |  \bar{0},\bar{1} (\bar{1},\bar{0}) \rangle
&=& - \frac{2p}{\sqrt{r}K} \cos\theta,
\nonumber\\
\end{eqnarray}
and off-diagonal elements, such as 
$\langle \bar{0},  \bar{0}  | a_1 a_2^\dagger + a_1^\dagger a_2 |  \bar{0},  \bar{1} \rangle$, are negligible when $\langle -\alpha| \alpha \rangle\simeq 0$.
Therefore, the coupling can be tuned via $\theta$ as in the case of identical KPOs.
The dynamical phase imprinted on these states at $t=T$, is written as
\begin{eqnarray}
\varphi_{ij} = \left\{
\begin{array}{cl}
\frac{2Jp}{\sqrt{r}K}\int_0^T  \cos\theta(t) dt &  {\rm for} \ i=j, \\
-\frac{2Jp}{\sqrt{r}K}\int_0^T \cos\theta(t) dt & {\rm for} \ i\ne j.
\end{array}
\right.
\label{theta_5_23_22}
\end{eqnarray}
}



\textcolor{black}{\section{Ideal tunable coupling}
\label{Beam splitter type}}
\textcolor{black}{
We consider the ideal tunable coupling with the form of $g(t)(a_1a_2^\dagger + a_1^\dagger a_2)$, which is called beam-splitter type~\cite{Gao2018}.
We compare the performance of our $R_{zz}$ gate with that of the $R_{zz}$ gate based on the ideal beam-splitter-type coupling.
For the control based on the ideal beam-splitter-type coupling, we set $g(t)=J\cos\theta(t)$ with $\theta(t) = \frac{\pi}{2} - \theta_{\rm amp} \pi [1-\cos(2\pi t/T)]$ for $0\le t \le T$, and fix $\theta_{1,2}$ to zero.
Figure~\ref{fid_BS_com_7_3_22} shows the infidelity of the $R_{zz}$ gates as a function of $T$.
The fidelity for the ideal beam-splitter-type coupling is higher than that of our scheme, although the difference is modest when the pump amplitude is small or $T$ is short.}
\begin{figure}[h!]
\begin{center}
\includegraphics[width=7cm]{figure9.eps}
\end{center}
\caption{
\textcolor{black}{$T$-dependence of the infidelity of the $R_{zz}$ gates based on the ideal beam-splitter-type coupling and the control with the CD term for $p/K=4$ and 6.
The initial state is $|\Psi_s\rangle$.
The value of $\theta_{\rm amp}$ was chosen so that  $(\varphi_{\bar{1}\bar{0}} - \varphi_{\bar{0}\bar{0}})/\pi=-0.5$.
The used parameters is $J/K=0.2$.
The dashed lines are guides to eyes.}
}
\label{fid_BS_com_7_3_22}
\end{figure}


\textcolor{black}{
\section{Robustness}
\label{Robustness}}
\textcolor{black}{We examine the robustness of our scheme against the decoherence and errors in the CD term and in the resonance frequency of KPOs.}

\textcolor{black}{
\subsection{Decoherence}
\label{Decoherence}
}

\textcolor{black}{
We consider the $R_{zz}$ gate with the CD term in the case that there is pure decay or pure dephasing.
Figure~\ref{fid_fin_com_purekappa_6_25_22} shows the dependence of the infidelity of the $R_{zz}$ gate on $T$.
It is seen that the $T$-dependence of the infidelity is quantitatively the same in the both cases.
The infidelity has a minimum with respect to $T$, where the suppression of nonadiabatic transitions balances with unwanted transitions due to the decoherence.
}
\begin{figure}[h!]
\begin{center}
\includegraphics[width=7cm]{figure10.eps}
\end{center}
\caption{
\textcolor{black}{Dependence of the infidelity on $T$ for the controls with the CD term 
for the cases of the pure decay with $\kappa_l=\kappa$ and pure dephasing with $\gamma_p^{(l)}=\gamma$.
The initial state is $|\Psi_s\rangle$.
The value of  $\theta_{\rm amp}$ was chosen so that  $(\varphi_{\bar{1}\bar{0}} - \varphi_{\bar{0}\bar{0}})/\pi=-0.5$
The used parameters are $p/K=7$ and $J/K=0.2$.
The dashed lines are guides to eyes.}
}
\label{fid_fin_com_purekappa_6_25_22}
\end{figure}

\textcolor{black}{
\subsection{Imperfection of CD term}
\label{Imperfection of CD term}}
\textcolor{black}{The CD term may depart from the ideal one, $-\dot{\theta}_1(t)a_1^\dagger a_1$, due to imperfection of the control of the detuning.
In order to examine the robustness of our scheme against the error in the CD term, we assume that the imperfect CD term is represented as $-\xi\dot{\theta}_1(t)a_1^\dagger a_1$, where $\xi$ is the constant parameter characterizing the degree of the error.}
\textcolor{black}{Figure~\ref{fid_error_5_22_22} shows the dependence of the infidelity of the control with the CD term on $\xi$. 
We confirmed that the fidelity is significantly higher than that of the control without the CD term in the range of $\xi$ used.
Therefore, our scheme is robust against the error in the CD term.}
\textcolor{black}{As shown in Fig.~\ref{fid_error_5_22_22}(b), the infidelity monotonically decreases as $\xi$ increases for $0\le \xi\le 1$, where $\xi=0$ corresponds to the control without the CD term.
This result shows that our method can improve the control fidelity even with limited tunability of the detuning.}
\begin{figure}[]
\begin{center}
\includegraphics[width=7cm]{figure11.eps}
\end{center}
\caption{
\textcolor{black}{$\xi$-dependence of the infidelity of the control with the CD term \textcolor{black}{for $0.8\le \xi\le 1.2$ (a) and for $0\le \xi\le 1$ (b)}.
The initial state is $|\Psi_s\rangle$.
The value of  $\theta_{\rm amp}$ was chosen so that  $(\varphi_{\bar{1}\bar{0}} - \varphi_{\bar{0}\bar{0}})/\pi=-0.5$.
The used parameters are $p/K=7$, $T/K=1$ and $J/K=0.2$.
The infidelity, ${\rm log}_{10}(1-F)$, of the control without the CD field is approximately $-1$.
}
}
\label{fid_error_5_22_22}
\end{figure}


\textcolor{black}{
In our scheme, the resonance frequencies of the KPOs are tuned to $\omega =  K + \omega_p/2$ when the coupling is off, so that the detuning is zero.
However, there might be an error in $\omega$.
We examine the robustness of our scheme against the error in $\omega$.
In order to describe the error of $\omega$, we introduce an additional constant detuning, $\Delta'$, of KPO~2. 
Figure~\ref{fid_Delta_KPO1_6_30_22} shows the dependence of the infidelity on $\Delta'$.
It is seen that the fidelity of the control is not sensitive to the discrepancy of the resonance frequency of the KPO 2 in the parameter range studied.
}
\begin{figure}[]
\begin{center}
\includegraphics[width=7cm]{figure12.eps}
\end{center}
\caption{
\textcolor{black}{Dependence of the infidelity on $\Delta'$ for the control with the CD term.
The initial state is $|\Psi_s\rangle$.
The value of  $\theta_{\rm amp}$ was chosen so that  $(\varphi_{\bar{1}\bar{0}} - \varphi_{\bar{0}\bar{0}})/\pi=-0.5$.
The used parameters are $p/K=7$, $T/K=1$ and $J/K=0.2$.
The dashed lines are guides to eyes.}
}
\label{fid_Delta_KPO1_6_30_22}
\end{figure}


\textcolor{black}{
\section{Loading into ground state}
\label{Loading into ground state}
}
\textcolor{black}{
A KPO can be loaded to its ground state, which is the cat state, from the vacuum state by adiabatically ramping the pump amplitude~\cite{Cochrane1999,Goto2016}.
The control fidelity is degraded due to nonadiabatic transitions, and the degradation is enhanced when the duration of the control becomes short.
Time dependent detuning can suppress nonadiabatic transitions, thus it can be used to shorten the duration of the loading process~\cite{Goto_patent,Masuda2020}.
We examine the efficiency of these adiabatic loading protocols in our two-KPO system.
We assume that the coupling is set to be off, that is, $\theta=\pi/2$ throughout the initialization, and the decoherence is negligible (the effect of the decoherence is studied for a KPO e.g. in Ref.~[\citenum{Masuda2020}]).
The pump amplitude is monotonically increased from zero to $p_{\rm max}$ for $0\le t \le T$ as
$p_l(t)=p_{\rm max}[1 - \cos(\pi t/T) ] / 2$ for $l=1,2$. We set $p_l/K=4$ and $J/K=0.2$.
}

\textcolor{black}{
We first examine the loading protocol with the detuning fixed to zero.
The control fidelity is defined by the squared amplitude of the overlap between the state at $t=T$ and the  ground state of $H_{\rm tot}(T)$ in Eq.~(\ref{H_2KPO_11_25_21}) with $\dot\theta=0$.
%For comparison, we also calculated the fidelity defined with the ground state of the Hamiltonian in Eq.~(\ref{H_2KPO_11_25_21}) with $J=0$. 
For comparison, the fidelity of the control without the always-on linear coupling ($J=0$) is also presented, where the fidelity is defined with the ground state for $J=0$.
The squared amplitude of the overlap between the two ground states corresponding to $J/K=0$ and $0.2$ is 0.994. 
As seen in Fig.~\ref{fid_p4_com_6_25_22}, the fidelities of the controls for $J/K=0$ and $0.2$ are approximately the same for the parameters used. 
}


\textcolor{black}{
Next, we consider the loading protocol with the time dependent detuning.
The role of the detuning is to open the gap between energy levels of each KPO to mitigate unwanted nonadiabatic transitions, and thus the role is different from that of the detuning used in the main text. 
The time dependence of the detuning is given by $\Delta_l(t)=\Delta_{\rm max}[1 + \cos(\pi t/T) ] / 2$.
The detuning monotonically decreases from $\Delta_{\rm max}$ to zero for $0\le t\le T$. 
We set $\Delta_{\rm max}/K=3$ in this paper.
Figure~\ref{fid_p4_com_6_25_22} shows that fidelity of the control for $J/K=0.2$ is greatly improved
as well as that for  $J=0$ compared to the fidelity of the control without detuning.
}
\begin{figure}
\begin{center}
\includegraphics[width=7.5cm]{figure13.eps}
\end{center}
\caption{
\textcolor{black}{
Dependence of the infidelity of the adiabatic loading protocols on the loading time $T$ for $p/K=4$. 
The light blue asterisks and green crosses are for the control without detuning,
while the others are for the control with the time dependent detuning.
The light blue asterisks and the red filled circles are for $J/K=0.2$, while the others are for $J=0$.
The dashed lines are guides to eyes.
%The control fidelity for the light blue asterisks and red filled circles are defined with the ground state of the Hamiltonian in Eq.~(\ref{H_2KPO_11_25_21}) with $J/K=0.2$, while the fidelity for the others are defined with the ground state of the Hamiltonian with $J=0$.
}
}
\label{fid_p4_com_6_25_22}
\end{figure}

\begin{thebibliography}{99}
\bibitem{Onyshkevych1959} L. S. Onyshkevych, W. F. Kosonocky and  A. W. Lo, 
Parametric phase-locked oscillator-characteristics and applications to digital systems,
{Trans. Inst. Radio Engrs.} {\bf EC-8,} 277--286 (1959).

\bibitem{Goto1959}  E. Goto, 
The parametron, a digital computing element which utilizes parametric oscillation,
{Proc. Inst. Radio Engrs.} {\bf 47,} 1304--1316 (1959).
\bibitem{Milburn1991} G. J. Milburn and C. A. Holmes, 
Quantum coherence and classical chaos in a pulsed parametric oscillator with a Kerr nonlinearity,
Phys. Rev. A {\bf 44,} 4704 (1991).
\bibitem{Wielinga1993} B. Wielinga and G. J. Milburn, 
Quantum tunneling in a Kerr medium with parametric pumping,
Phys. Rev. A {\bf 48,} 2494 (1993).
\bibitem{Goto2016} H. Goto, 
Bifurcation-based adiabatic quantum computation with a nonlinear oscillator network,
{Sci. Rep.} {\bf 6,} 21686 (2016).
\bibitem{Wang2019} Z. Wang, M. Pechal, E. A. Wollack, P. Arrangoiz-Arriola, M. Gao, N. R. Lee and A. H. Safavi-Naeini, 
Quantum dynamics of a few-photon parametric oscillator,
{Phys. Rev.} X  {\bf 9,} 021049 (2019).
\bibitem{Grimm2020} A. Grimm, N. E. Frattini, S. Puri, S. O. Mundhada, S. Touzard, M. Mirrahimi, S. M. Girvin, S. Shankar and M. H. Devoret, 
Stabilization and operation of a Kerr-cat qubit,
{Nature} {\bf 584,} 205 (2020).
%\bibitem{Kirchmair2013}  G. Kirchmair,  B. Vlastakis,  Z. Leghtas, S. E. Nigg, H. Paik, E. Ginossar, M. Mirrahimi, L. Frunzio, S. M. Girvin and R. J. Schoelkopf,  {Nature} {\bf 495,} 205--209 (2013).
\bibitem{Goto2019} H. Goto, 
Quantum computation based on quantum adiabatic bifurcations of Kerr-nonlinear parametric oscillators,
J. Phys. Soc. Jpn. {\bf 88}, 061015 (2019).
\bibitem{Dykman2018} M. I. Dykman, C. Bruder, N. L\"{o}rch and Y. Zhang, 
Interaction-induced time-symmetry breaking in driven quantum oscillators,
Phys. Rev. B {\bf 98,} 195444 (2018).
\bibitem{Rota2019} R. Rota, F. Minganti, C. Ciuti and V. Savona, 
Quantum critical regime in a quadratically driven nonlinear photonic lattice,
Phys. Rev. Lett. {\bf 122,} 110405 (2019).

\bibitem{You2005} J. Q. You and F. Nori, 
Superconducting circuits and quantum information,
{Physics Today} {\bf 58,} 42--47 (2005).
\bibitem{Gambetta2017} J. M. Gambetta, J. M. Chow and M. Steffen, 
Building logical qubits in a superconducting quantum computing system,
{npj Quantum Information} {\bf 3,} 2 (2017).
\bibitem{Wendin2017} G. Wendin, 
Quantum information processing with superconducting circuits: a review,
{Reports on Progress in Physics} {\bf 80,} 106001 (2017).
\bibitem{Krantz2019}  P. Krantz, M. Kjaergaard, F. Yan, T. P. Orlando, S. Gustavsson and W. D. Oliver,
A quantum engineer's guide to superconducting qubits, 
{Appl. Phys. Rev.} {\bf 6,} 021318 (2019).
\bibitem{Gu2019} X. Gu, A. F. Kockum, A. Miranowicz, Y-xi. Liu, and F. Nori, 
Microwave photonics with superconducting quantum circuits,
{Physics Reports} {\bf 718-719,} 1--102 (2019). 
\bibitem{Blais2020} A. Blais, A. L. Grimsmo, S. M. Girvin and A. Wallraff, 
Circuit quantum electrodynamics,
{Rev. Mod. Phys.} {\bf 93,} 25005 (2021).

\bibitem{Meaney2014} C. H. Meaney, H. Nha, T. Duty and  G. J. Milburn, 
Quantum and classical nonlinear dynamics in a microwave cavity,
EPJ Quantum Technol. {\bf 1,} 7 (2014). 




\bibitem{Tuckett2019} 
D. K. Tuckett, A. S. Darmawan, C. T. Chubb, S. Bravyi, S. D. Bartlett and S. T. Flammia, 
Tailoring surface codes for highly biased noise,
Phys. Rev. X {\bf 9,} 041031 (2019).

\bibitem{Ataides2021} J. P. B. Ataides, D. K. Tuckett, S. D. Bartlett, S. T. Flammia and B. J. Brown, 
The XZZX surface code,
{Nat. Commun.} {\bf 12,} 2172 (2021).



\bibitem{Nigg2017} S. E. Nigg, N. L\"{o}rch and R. P. Tiwari, 
Robust quantum optimizer with full connectivity,
{Sci. Adv.} {\bf 3,} e1602273 (2017).
\bibitem{Puri2017} S. Puri, C. K. Andersen, A. L. Grimsmo and A. Blais, 
Quantum annealing with all-to-all connected nonlinear oscillators,
{Nat. Commun.} {\bf 8,} 15785 (2017).
\bibitem{Zhao2018} P. Zhao, Z. Jin, P. Xu, X. Tan, H. Yu, and Y. Yu, 
Two-photon driven Kerr resonator for quantum annealing with three-dimensional circuit QED,
Phys. Rev. Applied {\bf 10,} 024019 (2018).
\bibitem{Onodera2020} T. Onodera, E. Ng and P. L. McMahon, 
A quantum annealer with fully programmable all-to-all coupling via Floquet engineering,
npj Quantum Inf. {\bf 6,} 48 (2020).
\bibitem{Goto2020a} H. Goto and T. Kanao, 
Quantum annealing using vacuum states as effective excited states of driven systems,
Commun. Phys. {\bf 3, } 235 (2020). 
\bibitem{Kanao2021} T. Kanao and H. Goto, 
High-accuracy Ising machine using Kerr-nonlinear parametric oscillators with local four-body interactions,
{npj Quantum Inf.} {\bf 7,} 18 (2021).
\bibitem{Cochrane1999} P. T. Cochrane, G. J. Milburn and W. J. Munro, 
Macroscopically distinct quantum-superposition states as a bosonic code for amplitude damping,
Phys. Rev. A {\bf 59,} 2631 (1999).
\bibitem{Goto2016b} H. Goto, 
Universal quantum computation with a nonlinear oscillator network,
{Phys. Rev.} A {\bf 93,} 050301(R) (2016).
\bibitem{Puri2017b} S. Puri, S. Boutin and A. Blais, 
Engineering the quantum states of light in a Kerr-nonlinear resonator by two-photon driving,
{npj Quantum Inf.} {\bf 3,} 18 (2017).
\bibitem{Puri2020} S. Puri, L. St-Jean, J. A. Gross, A. Grimm, N. E. Frattini, P. S. Iyer,
A. Krishna, S. Touzard, L. Jiang, A. Blais et al., 
Bias-preserving gates with stabilized cat qubits,
{Sci. Adv.} {\bf 6,} eaay5901 (2020).
%\bibitem{Lescanne2020} R. Lescanne, M. Villiers, T. Peronnin, A. Sarlette, M. Delbecq, B. Huard, T. Kontos, M. Mirrahimi and Z. Leghtas, {Nat. Phys.} {\bf 16,} 509 (2020).


\bibitem{Darmawan2021} A. S. Darmawan, B. J. Brown, A. L. Grimsmo, D. K. Tuckett and S. Puri, 
Practical quantum error correction with the XZZX code and Kerr-cat qubits,
{PRX Quantum} {\bf 2,} 030345 (2021).

\bibitem{Kanao2021b} T. Kanao, S. Masuda, S. Kawabata and H. Goto, 
Quantum gate for a Kerr nonlinear parametric oscillator using effective excited states,
Phys. Rev. Applied {\bf 18,} 014019 (2022).
\bibitem{Xu2021} Q. Xu, J. K. Iverson, F. G.S.L. Brand\~{a}o, and L. Jiang, 
Engineering fast bias-preserving gates on stabilized cat qubits,
{Phys. Rev. Research} {\bf 4,} 013082 (2022).
\bibitem{Kang2021}
Y. H. Kang, Y. H. Chen, X. Wang, J. Song, Y. Xia, A. Miranowicz, S. B. Zheng and F. Nori, 
Nonadiabatic geometric quantum computation with cat-state qubits via invariant-based reverse engineering,
Phys. Rev. Research {\bf 4,} 013233 (2022).
\bibitem{Yamaji2021} T. Yamaji, S. Kagami, A. Yamaguchi, T. Satoh, K. Koshino, H. Goto, Z. R. Lin, Y. Nakamura and  T. Yamamoto, 
Spectroscopic observation of the crossover from a classical Duffing oscillator to a Kerr parametric oscillator,
Phys. Rev. A {\bf 105,} 023519 (2022).
\bibitem{Masuda2021b} S. Masuda, A. Yamaguchi, T. Yamaji, T. Yamamoto, T. Ishikawa, Y. Matsuzaki and S. Kawabata, 
Theoretical study of reflection spectroscopy for superconducting quantum parametrons,
New J. Phys. {\bf 23,} 093023 (2021).
\bibitem{Zhang2017} Y. Zhang and M. I. Dykman, 
Preparing quasienergy states on demand: A parametric oscillator,
{Phys. Rev.} A {\bf 95,} 053841 (2017).
\bibitem{Goto2018} H. Goto, Z. Lin, and Y. Nakamura, 
Boltzmann sampling from the Ising model using quantum heating of coupled nonlinear oscillators,
Sci. Rep. {\bf 8,} 7154 (2018).
\bibitem{Masuda2020} S. Masuda, T. Ishikawa, Y. Matsuzaki and S. Kawabata, 
Controls of a superconducting quantum parametron under a strong pump field,
{Sci. Rep.} {\bf 11,} 11459 (2021).
\bibitem{Hovsepyan2016} G. H. Hovsepyan, A. R. Shahinyan, L. Y. Chew, and G. Yu. Kryuchkyan,
Phase locking and quantum statistics in a parametrically driven nonlinear resonator,
Phys. Rev. A {\bf 93,} 043856 (2016).
\bibitem{Goto2021b} H. Goto and T. Kanao, 
Chaos in coupled Kerr-nonlinear parametric oscillators,
Phys. Rev. Research {\bf 3,} 043196 (2021).
\bibitem{Rice2003}  M. Demirplak and S. A. Rice, 
Adiabatic population transfer with control fields,
J. Phys. Chem. {\bf 107,} 9937 (2003).
\bibitem{Torrontegui2013} E. Torrontegui, S. Ib\'{a}\~{n}ez, S. Mart\'{\i}nez-Garaot, M. Modugno, A. del Campo, D. Gu\'ery-Odelin, A. Ruschhaupt, Xi Chen and J. G. Muga, 
Shortcuts to adiabaticity,
{Adv. At. Mol. Opt. Phys.} {\bf 62,} 117 (2013).
\bibitem{Masuda2015} S. Masuda and S. A. Rice, 
Rotation of the orientation of the wave function distribution of a charged particle and its utilization,
J. Phys. Chem. B {\bf 119,} 11079 (2015).
\bibitem{Masuda2016} S. Masuda and S. A. Rice, {\it Controlling quantum dynamics with assisted adiabatic processes,} {Advances in Chemical Physics} {\bf 159,} 51--136 (New York: Wiley 2016).
\bibitem{Campo2019} A. del Campo and K. Kim, 
Focus on shortcuts to adiabaticity,
New J. Phys. {\bf 21,} 050201 (2019).
\bibitem{Guery-Odelin2019} D. Gu\'ery-Odelin, A. Ruschhaupt,  A. Kiely,  E. Torrontegui, S. Mart\'{\i}nez-Garaot and J. G. Muga, 
Shortcuts to adiabaticity: Concepts, methods, and applications,
{Rev. Mod. Phys.} {\bf 91} 045001 (2019).
\bibitem{Palmero2016} M. Palmero, S. Wang, D. Gu\'ery-Odelin, J.-S. Li, and J. G.
Muga, 
Shortcuts to adiabaticity for an ion in a rotating radially-tight trap,
New J. Phys. {\bf 18,} 043014 (2016).
\bibitem{Lizuain2019} I. Lizuain, A. Tobalina, A. Rodriguez-Prieto, and J G Muga, 
Fast state and trap rotation of a particle in an anisotropic potential,
J. Phys. A: Math. Theor. {\bf 52,} 465301 (2019).
\bibitem{Goto2019b} H. Goto, Z. Lin, T. Yamamoto and Y. Nakamura, 
On-demand generation of traveling cat states using a parametric oscillator,
Phys. Rev. A {\bf 99,} 023838 (2019).
\bibitem{Chono2022} H. Chono, T. Kanao, H. Goto, 
Two-qubit gate using conditional driving for highly detuned Kerr-nonlinear parametric oscillators,
arXiv:2204.03347 (2022).
\bibitem{Gao2018} Y. Y. Gao, B. J. Lester, Y. Zhang, C. Wang,
S. Rosenblum, L. Frunzio, L. Jiang, S. M. Girvin and R. J. Schoelkopf, 
Programmable interference between two microwave quantum memories,
Phys. Rev. X {\bf 8,} 021073 (2018).
%\bibitem{Meaney2014} C. H. Meaney, H. Nha, T. Duty, G. J. Milburn, EPJ Quantum Technol. {\bf 1,} 7 (2014).
\bibitem{Winkel2020} 
P. Winkel, K. Borisov, L. Grunhaupt, D. Rieger, M. Spiecker, F. Valenti, A. V. Ustinov, W. Wernsdorfer and I. M. Pop, 
Implementation of a transmon qubit using superconducting granular aluminum,
Phys. Rev. X {\bf 10,} 031032 (2020).


\bibitem{Guillaud2019} J. Guillaud and M. Mirrahimi,  
Repetition cat qubits for fault-tolerant quantum computation,
Phys. Rev. X {\bf 9,} 041053 (2019).
\bibitem{ZLWang2013} Z. L. Wang, Y. P. Zhong, L. J. He, H. Wang, J. M. Martinis, A. N. Cleland, and Q. W. Xie, 
Quantum state characterization of a fast tunable superconducting resonator,
Appl. Phys. Lett. {\bf 102,} 163503 (2013).


\bibitem{Puri2019} S. Puri, A. Grimm, P. Campagne-Ibarcq, A. Eickbusch, K. Noh, G. Roberts, L. Jiang, M. Mirrahimi, M. H. Devoret and S. M. Girvin,
Stabilized cat in a driven nonlinear cavity: A fault-tolerant error syndrome detector,
Phys. Rev. X {\bf 9,} 041009  (2019).

\bibitem{NEC_paper} T. Yamaji and A. Yamaguchi (Not published).
\bibitem{delCampo2012} A. del Campo, M. M. Rams, and W. H. Zurek,
Assisted finite-rate adiabatic passage across a quantum critical point: exact solution for the quantum Ising model,
Phys. Rev. Lett. {\bf 109,} 115703  (2012).
\bibitem{Damski2014} B. Damski, 
Counterdiabatic driving of the quantum Ising model,
J. Stat. Mech. P12019 (2014).
\bibitem{Okuyama2016} M. Okuyama and K. Takahashi, 
From classical nonlinear integrable systems to quantum shortcuts to adiabaticity,
Phys. Rev. Lett. {\bf 117,} 070401 (2016).
\bibitem{Sels2017} N. Sels and  A. Polkovnikov, 
Minimizing irreversible losses in quantum systems by local counterdiabatic driving, 
PNAS {\bf 114}, E3909--E3916 (2017).
\bibitem{Setiawan2019} I. Setiawan, B. E. Gunara, S. Avazbaev and K. Nakamura, 
Fast-forward approach to adiabatic quantum dynamics of regular spin clusters: Nature of geometry-dependent driving interactions,
Phys. Rev. A {\bf 99}, 062116 (2019).
\bibitem{Frattini2017} N. E. Frattini, U. Vool, S. Shankar, A. Narla, K. M. Sliwa and M. H.  Devoret, 
3-wave mixing Josephson dipole element,
{Appl. Phys. Lett.} {\bf 110,} 222603 (2017).
\bibitem{Noguchi2020} A. Noguchi, A. Osada, S.  Masuda, S. Kono,  K. Heya, S. P.  Wolski, H. Takahashi,  T. Sugiyama,  D. Lachance-Quirion and Y. Nakamura, 
Fast parametric two-qubit gates with suppressed residual interaction using the second-order nonlinearity of a cubic transmon,
{Phys. Rev.} A {\bf 102,} 062408 (2020).
\bibitem{Leonhardt1997} U. Leonhardt, {\it Measuring the Quantum State of Light} (Cambridge
University Press, Cambridge, U.K., 1997).
\bibitem{Deleglise2008} S. Del\'{e}glise, I. Dotsenko, C. Sayrin, J. Bernu, M. Brune, J.-M. Raimond, and S. Haroche, 
Reconstruction of non-classical cavity field states with snapshots of their decoherence,
Nature {\bf 455,} 510 (2008).
\bibitem{Goto_patent} H. Goto, Inventor; Kabushiki Kaisha Toshiba assignee. Quantum computation apparatus and quantum computation method, 
United States patent US 10,250,271 B2. 2019 Apr 2.




%\bibitem{Kinsler1991} Kinsler P and Drummond P D 1991 {\it Phys. Rev.} A {\bf 43} 6194
%\bibitem{Wustmann2013} Wustmann W and Shumeiko V  2013 {\it Phys. Rev.} B {\bf 87} 184501 
%\bibitem{Ding2017} S. Ding, G. Maslennikov, R. Habl\"{u}tzel, H. Loh, and D. Matsukevich, Phys. Rev. Lett. {\bf 119,} 150404 (2017).
%\bibitem{Yamamoto2014} Z. R. Lin, K. Inomata, K. Koshino, W. D. Oliver, Y. Nakamura, J. S. Tsai and T. Yamamoto, 
%{Nat. Commun.} {\bf 5,} 4480 (2014).
%\bibitem{Yamamoto2016} T. Yamamoto, K. Koshino and Y. Nakamura,  {Lecture Notes in Physics} {\bf 911,} 495--513 (2016).
%\bibitem{Nakamura1999} Y. Nakamura, Yu. A. Pashkin and J. S. Tsai, Nature {\bf398,} 786--788 (1999).


%\bibitem{You2007} J. Q. You, X. Hu, S. Ashhab and F. Nori, {Physical Review} B {\bf 75,} 140515(R) (2007).
%\bibitem{You2011} J. Q. You and F. Nori, {Nature} {\bf 474,}  589--597 (2011).




%\bibitem{Puri2018} S. Puri, S. Boutin, and A. Blais, 2018 {\it npj Quantum Inf.} {\bf 3} 18.






%\bibitem{Koshino2013} K. Koshino, K. Inomata, T. Yamamoto and Y. Nakamura, {New J. Phys.} {\bf 15,} 115010 (2013).
%\bibitem{Banacloche2013} J. Gea-Banacloche, {Phys. Rev.} A {\bf 87,} 023832 (2013).


%\bibitem{Wiseman2020} Wiseman H M and Milburn G J 2009 {\it Quantum Measurement and Control}  (Cambridge, Cambridge University Press)


%\bibitem{Walls2008} D. F. Walls and G. J. Milburn, {\it Quantum Optics 2nd Edition} (Berlin: Springer 2008).

%\bibitem{Frattini2017} N. E. Frattini, U. Vool, S. Shankar, A. Narla, K. M. Sliwa and M. H. Devoret, {Appl. Phys. Lett.} {\bf 110,} 222603 (2017).
%\bibitem{Noguchi2020} A. Noguchi, A. Osada, S. Masuda, S. Kono, K. Heya, S. P. Wolski, H. Takahashi, T. Sugiyama,  D. Lachance-Quirion and Y. Nakamura, {Phys. Rev.} A {\bf 102,} 062408 (2020).
%\bibitem{Zhang2017}  Y. Zhang and M. I. Dykman, Phys. Rev. A {\bf 95,} 053841 (2017).





%\bibitem{Muga2009} J. G. Muga, X. Chen, A. Ruschhaupt, and D. Gu\'{e}ry-Odelin,
%J. Phys. B {\bf 42,} 241001 (2009).
%\bibitem{Chen2009} X. Chen, I. Lizuain, A. Ruschhaupt, D. Gu\'{e}ry-Odelin and J. G. Muga,
%Phys. Rev. Lett. {\bf 105,} 123003 (2010).






\end{thebibliography}

%\end{minipage}
%\end{center}

\end{document}
%
% ****** End of file apssamp.tex ******
