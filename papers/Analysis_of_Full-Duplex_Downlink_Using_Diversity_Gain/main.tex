%% bare_jrnl_comsoc.tex
%% V1.4b
%% 2015/08/26
%% by Michael Shell
%% see http://www.michaelshell.org/
%% for current contact information.
%%
%% This is a skeleton file demonstrating the use of IEEEtran.cls
%% (requires IEEEtran.cls version 1.8b or later) with an IEEE
%% Communications Society journal paper.
%%
%% Support sites:
%% http://www.michaelshell.org/tex/ieeetran/
%% http://www.ctan.org/pkg/ieeetran
%% and
%% http://www.ieee.org/

%%*************************************************************************
%% Legal Notice:
%% This code is offered as-is without any warranty either expressed or
%% implied; without even the implied warranty of MERCHANTABILITY or
%% FITNESS FOR A PARTICULAR PURPOSE! 
%% User assumes all risk.
%% In no event shall the IEEE or any contributor to this code be liable for
%% any damages or losses, including, but not limited to, incidental,
%% consequential, or any other damages, resulting from the use or misuse
%% of any information contained here.
%%
%% All comments are the opinions of their respective authors and are not
%% necessarily endorsed by the IEEE.
%%
%% This work is distributed under the LaTeX Project Public License (LPPL)
%% ( http://www.latex-project.org/ ) version 1.3, and may be freely used,
%% distributed and modified. A copy of the LPPL, version 1.3, is included
%% in the base LaTeX documentation of all distributions of LaTeX released
%% 2003/12/01 or later.
%% Retain all contribution notices and credits.
%% ** Modified files should be clearly indicated as such, including  **
%% ** renaming them and changing author support contact information. **
%%*************************************************************************


% *** Authors should verify (and, if needed, correct) their LaTeX system  ***
% *** with the testflow diagnostic prior to trusting their LaTeX platform ***
% *** with production work. The IEEE's font choices and paper sizes can   ***
% *** trigger bugs that do not appear when using other class files.       ***                          ***
% The testflow support page is at:
% http://www.michaelshell.org/tex/testflow/



%\documentclass[journal,comsoc]{IEEEtran}
\documentclass[conference]{IEEEtran}

%
% If IEEEtran.cls has not been installed into the LaTeX system files,
% manually specify the path to it like:
% \documentclass[journal,comsoc]{../sty/IEEEtran}


\usepackage[T1]{fontenc}% optional T1 font encoding


% Some very useful LaTeX packages include:
% (uncomment the ones you want to load)


% *** MISC UTILITY PACKAGES ***
%
%\usepackage{ifpdf}
% Heiko Oberdiek's ifpdf.sty is very useful if you need conditional
% compilation based on whether the output is pdf or dvi.
% usage:
% \ifpdf
%   % pdf code
% \else
%   % dvi code
% \fi
% The latest version of ifpdf.sty can be obtained from:
% http://www.ctan.org/pkg/ifpdf
% Also, note that IEEEtran.cls V1.7 and later provides a builtin
% \ifCLASSINFOpdf conditional that works the same way.
% When switching from latex to pdflatex and vice-versa, the compiler may
% have to be run twice to clear warning/error messages.






% *** CITATION PACKAGES ***
%
%\usepackage{cite}
% cite.sty was written by Donald Arseneau
% V1.6 and later of IEEEtran pre-defines the format of the cite.sty package
% \cite{} output to follow that of the IEEE. Loading the cite package will
% result in citation numbers being automatically sorted and properly
% "compressed/ranged". e.g., [1], [9], [2], [7], [5], [6] without using
% cite.sty will become [1], [2], [5]--[7], [9] using cite.sty. cite.sty's
% \cite will automatically add leading space, if needed. Use cite.sty's
% noadjust option (cite.sty V3.8 and later) if you want to turn this off
% such as if a citation ever needs to be enclosed in parenthesis.
% cite.sty is already installed on most LaTeX systems. Be sure and use
% version 5.0 (2009-03-20) and later if using hyperref.sty.
% The latest version can be obtained at:
% http://www.ctan.org/pkg/cite
% The documentation is contained in the cite.sty file itself.






% *** GRAPHICS RELATED PACKAGES ***
%
\ifCLASSINFOpdf
  % \usepackage[pdftex]{graphicx}
  % declare the path(s) where your graphic files are
  % \graphicspath{{../pdf/}{../jpeg/}}
  % and their extensions so you won't have to specify these with
  % every instance of \includegraphics
  % \DeclareGraphicsExtensions{.pdf,.jpeg,.png}
\else
  % or other class option (dvipsone, dvipdf, if not using dvips). graphicx
  % will default to the driver specified in the system graphics.cfg if no
  % driver is specified.
  % \usepackage[dvips]{graphicx}
  % declare the path(s) where your graphic files are
  % \graphicspath{{../eps/}}
  % and their extensions so you won't have to specify these with
  % every instance of \includegraphics
  % \DeclareGraphicsExtensions{.eps}
\fi
% graphicx was written by David Carlisle and Sebastian Rahtz. It is
% required if you want graphics, photos, etc. graphicx.sty is already
% installed on most LaTeX systems. The latest version and documentation
% can be obtained at: 
% http://www.ctan.org/pkg/graphicx
% Another good source of documentation is "Using Imported Graphics in
% LaTeX2e" by Keith Reckdahl which can be found at:
% http://www.ctan.org/pkg/epslatex
%
% latex, and pdflatex in dvi mode, support graphics in encapsulated
% postscript (.eps) format. pdflatex in pdf mode supports graphics
% in .pdf, .jpeg, .png and .mps (metapost) formats. Users should ensure
% that all non-photo figures use a vector format (.eps, .pdf, .mps) and
% not a bitmapped formats (.jpeg, .png). The IEEE frowns on bitmapped formats
% which can result in "jaggedy"/blurry rendering of lines and letters as
% well as large increases in file sizes.
%
% You can find documentation about the pdfTeX application at:
% http://www.tug.org/applications/pdftex

 \usepackage{graphicx}
          \usepackage{float}



% *** MATH PACKAGES ***
%
\usepackage{amsmath}
% A popular package from the American Mathematical Society that provides
% many useful and powerful commands for dealing with mathematics.
% Do NOT use the amsbsy package under comsoc mode as that feature is
% already built into the Times Math font (newtxmath, mathtime, etc.).
% 
% Also, note that the amsmath package sets \interdisplaylinepenalty to 10000
% thus preventing page breaks from occurring within multiline equations. Use:
\interdisplaylinepenalty=2500
% after loading amsmath to restore such page breaks as IEEEtran.cls normally
% does. amsmath.sty is already installed on most LaTeX systems. The latest
% version and documentation can be obtained at:
% http://www.ctan.org/pkg/amsmath


% Select a Times math font under comsoc mode or else one will automatically
% be selected for you at the document start. This is required as Communications
% Society journals use a Times, not Computer Modern, math font.
\usepackage[cmintegrals]{newtxmath}
% The freely available newtxmath package was written by Michael Sharpe and
% provides a feature rich Times math font. The cmintegrals option, which is
% the default under IEEEtran, is needed to get the correct style integral
% symbols used in Communications Society journals. Version 1.451, July 28,
% 2015 or later is recommended. Also, do *not* load the newtxtext.sty package
% as doing so would alter the main text font.
% http://www.ctan.org/pkg/newtx
%
% Alternatively, you can use the MathTime commercial fonts if you have them
% installed on your system:
%\usepackage{mtpro2}
%\usepackage{mt11p}
%\usepackage{mathtime}


%\usepackage{bm}
% The bm.sty package was written by David Carlisle and Frank Mittelbach.
% This package provides a \bm{} to produce bold math symbols.
% http://www.ctan.org/pkg/bm





% *** SPECIALIZED LIST PACKAGES ***
%
%\usepackage{algorithmic}
% algorithmic.sty was written by Peter Williams and Rogerio Brito.
% This package provides an algorithmic environment fo describing algorithms.
% You can use the algorithmic environment in-text or within a figure
% environment to provide for a floating algorithm. Do NOT use the algorithm
% floating environment provided by algorithm.sty (by the same authors) or
% algorithm2e.sty (by Christophe Fiorio) as the IEEE does not use dedicated
% algorithm float types and packages that provide these will not provide
% correct IEEE style captions. The latest version and documentation of
% algorithmic.sty can be obtained at:
% http://www.ctan.org/pkg/algorithms
% Also of interest may be the (relatively newer and more customizable)
% algorithmicx.sty package by Szasz Janos:
% http://www.ctan.org/pkg/algorithmicx




% *** ALIGNMENT PACKAGES ***
%
%\usepackage{array}
% Frank Mittelbach's and David Carlisle's array.sty patches and improves
% the standard LaTeX2e array and tabular environments to provide better
% appearance and additional user controls. As the default LaTeX2e table
% generation code is lacking to the point of almost being broken with
% respect to the quality of the end results, all users are strongly
% advised to use an enhanced (at the very least that provided by array.sty)
% set of table tools. array.sty is already installed on most systems. The
% latest version and documentation can be obtained at:
% http://www.ctan.org/pkg/array


% IEEEtran contains the IEEEeqnarray family of commands that can be used to
% generate multiline equations as well as matrices, tables, etc., of high
% quality.




% *** SUBFIGURE PACKAGES ***
%\ifCLASSOPTIONcompsoc
%  \usepackage[caption=false,font=normalsize,labelfont=sf,textfont=sf]{subfig}
%\else
%  \usepackage[caption=false,font=footnotesize]{subfig}
%\fi
% subfig.sty, written by Steven Douglas Cochran, is the modern replacement
% for subfigure.sty, the latter of which is no longer maintained and is
% incompatible with some LaTeX packages including fixltx2e. However,
% subfig.sty requires and automatically loads Axel Sommerfeldt's caption.sty
% which will override IEEEtran.cls' handling of captions and this will result
% in non-IEEE style figure/table captions. To prevent this problem, be sure
% and invoke subfig.sty's "caption=false" package option (available since
% subfig.sty version 1.3, 2005/06/28) as this is will preserve IEEEtran.cls
% handling of captions.
% Note that the Computer Society format requires a larger sans serif font
% than the serif footnote size font used in traditional IEEE formatting
% and thus the need to invoke different subfig.sty package options depending
% on whether compsoc mode has been enabled.
%
% The latest version and documentation of subfig.sty can be obtained at:
% http://www.ctan.org/pkg/subfig




% *** FLOAT PACKAGES ***
%
%\usepackage{fixltx2e}
% fixltx2e, the successor to the earlier fix2col.sty, was written by
% Frank Mittelbach and David Carlisle. This package corrects a few problems
% in the LaTeX2e kernel, the most notable of which is that in current
% LaTeX2e releases, the ordering of single and double column floats is not
% guaranteed to be preserved. Thus, an unpatched LaTeX2e can allow a
% single column figure to be placed prior to an earlier double column
% figure.
% Be aware that LaTeX2e kernels dated 2015 and later have fixltx2e.sty's
% corrections already built into the system in which case a warning will
% be issued if an attempt is made to load fixltx2e.sty as it is no longer
% needed.
% The latest version and documentation can be found at:
% http://www.ctan.org/pkg/fixltx2e


%\usepackage{stfloats}
% stfloats.sty was written by Sigitas Tolusis. This package gives LaTeX2e
% the ability to do double column floats at the bottom of the page as well
% as the top. (e.g., "\begin{figure*}[!b]" is not normally possible in
% LaTeX2e). It also provides a command:
%\fnbelowfloat
% to enable the placement of footnotes below bottom floats (the standard
% LaTeX2e kernel puts them above bottom floats). This is an invasive package
% which rewrites many portions of the LaTeX2e float routines. It may not work
% with other packages that modify the LaTeX2e float routines. The latest
% version and documentation can be obtained at:
% http://www.ctan.org/pkg/stfloats
% Do not use the stfloats baselinefloat ability as the IEEE does not allow
% \baselineskip to stretch. Authors submitting work to the IEEE should note
% that the IEEE rarely uses double column equations and that authors should try
% to avoid such use. Do not be tempted to use the cuted.sty or midfloat.sty
% packages (also by Sigitas Tolusis) as the IEEE does not format its papers in
% such ways.
% Do not attempt to use stfloats with fixltx2e as they are incompatible.
% Instead, use Morten Hogholm'a dblfloatfix which combines the features
% of both fixltx2e and stfloats:
%
% \usepackage{dblfloatfix}
% The latest version can be found at:
% http://www.ctan.org/pkg/dblfloatfix




%\ifCLASSOPTIONcaptionsoff
%  \usepackage[nomarkers]{endfloat}
% \let\MYoriglatexcaption\caption
% \renewcommand{\caption}[2][\relax]{\MYoriglatexcaption[#2]{#2}}
%\fi
% endfloat.sty was written by James Darrell McCauley, Jeff Goldberg and 
% Axel Sommerfeldt. This package may be useful when used in conjunction with 
% IEEEtran.cls'  captionsoff option. Some IEEE journals/societies require that
% submissions have lists of figures/tables at the end of the paper and that
% figures/tables without any captions are placed on a page by themselves at
% the end of the document. If needed, the draftcls IEEEtran class option or
% \CLASSINPUTbaselinestretch interface can be used to increase the line
% spacing as well. Be sure and use the nomarkers option of endfloat to
% prevent endfloat from "marking" where the figures would have been placed
% in the text. The two hack lines of code above are a slight modification of
% that suggested by in the endfloat docs (section 8.4.1) to ensure that
% the full captions always appear in the list of figures/tables - even if
% the user used the short optional argument of \caption[]{}.
% IEEE papers do not typically make use of \caption[]'s optional argument,
% so this should not be an issue. A similar trick can be used to disable
% captions of packages such as subfig.sty that lack options to turn off
% the subcaptions:
% For subfig.sty:
% \let\MYorigsubfloat\subfloat
% \renewcommand{\subfloat}[2][\relax]{\MYorigsubfloat[]{#2}}
% However, the above trick will not work if both optional arguments of
% the \subfloat command are used. Furthermore, there needs to be a
% description of each subfigure *somewhere* and endfloat does not add
% subfigure captions to its list of figures. Thus, the best approach is to
% avoid the use of subfigure captions (many IEEE journals avoid them anyway)
% and instead reference/explain all the subfigures within the main caption.
% The latest version of endfloat.sty and its documentation can obtained at:
% http://www.ctan.org/pkg/endfloat
%
% The IEEEtran \ifCLASSOPTIONcaptionsoff conditional can also be used
% later in the document, say, to conditionally put the References on a 
% page by themselves.




% *** PDF, URL AND HYPERLINK PACKAGES ***
%
%\usepackage{url}
% url.sty was written by Donald Arseneau. It provides better support for
% handling and breaking URLs. url.sty is already installed on most LaTeX
% systems. The latest version and documentation can be obtained at:
% http://www.ctan.org/pkg/url
% Basically, \url{my_url_here}.




% *** Do not adjust lengths that control margins, column widths, etc. ***
% *** Do not use packages that alter fonts (such as pslatex).         ***
% There should be no need to do such things with IEEEtran.cls V1.6 and later.
% (Unless specifically asked to do so by the journal or conference you plan
% to submit to, of course. )


% correct bad hyphenation here
\hyphenation{op-tical net-works semi-conduc-tor}


\begin{document}
%
% paper title
% Titles are generally capitalized except for words such as a, an, and, as,
% at, but, by, for, in, nor, of, on, or, the, to and up, which are usually
% not capitalized unless they are the first or last word of the title.
% Linebreaks \\ can be used within to get better formatting as desired.
% Do not put math or special symbols in the title.
\title{Analysis of Full-Duplex Downlink Using Diversity Gain}
%
%
% author names and IEEE memberships
% note positions of commas and nonbreaking spaces ( ~ ) LaTeX will not break
% a structure at a ~ so this keeps an author's name from being broken across
% two lines.
% use \thanks{} to gain access to the first footnote area
% a separate \thanks must be used for each paragraph as LaTeX2e's \thanks
% was not built to handle multiple paragraphs
%

\author{\IEEEauthorblockN{Chandan Pradhan and Garimella Rama Murthy}
\IEEEauthorblockA{Signal Processing and Communication Research Center\\
IIIT Hyderabad, India\\
Email: chandan.pradhan@research.iiit.ac.in, rammurthy.iiit.ac.in}}
   
   

% note the % following the last \IEEEmembership and also \thanks - 
% these prevent an unwanted space from occurring between the last author name
% and the end of the author line. i.e., if you had this:
% 
% \author{....lastname \thanks{...} \thanks{...} }
%                     ^------------^------------^----Do not want these spaces!
%
% a space would be appended to the last name and could cause every name on that
% line to be shifted left slightly. This is one of those "LaTeX things". For
% instance, "\textbf{A} \textbf{B}" will typeset as "A B" not "AB". To get
% "AB" then you have to do: "\textbf{A}\textbf{B}"
% \thanks is no different in this regard, so shield the last } of each \thanks
% that ends a line with a % and do not let a space in before the next \thanks.
% Spaces after \IEEEmembership other than the last one are OK (and needed) as
% you are supposed to have spaces between the names. For what it is worth,
% this is a minor point as most people would not even notice if the said evil
% space somehow managed to creep in.



% The paper headers
%\markboth{Journal of \LaTeX\ Class Files,~Vol.~14, No.~8, August~2015}%
%{Shell \MakeLowercase{\textit{et al.}}: Bare Demo of IEEEtran.cls for IEEE Communications Society Journals}

\markboth{IEEE SIGNAL PROCESSING LETTERS}%
{Shell \MakeLowercase{\textit{et al.}}: Bare Demo of IEEEtran.cls for IEEE Communications Society Journals}

% The only time the second header will appear is for the odd numbered pages
% after the title page when using the twoside option.
% 
% *** Note that you probably will NOT want to include the author's ***
% *** name in the headers of peer review papers.                   ***
% You can use \ifCLASSOPTIONpeerreview for conditional compilation here if
% you desire.




% If you want to put a publisher's ID mark on the page you can do it like
% this:
%\IEEEpubid{0000--0000/00\$00.00~\copyright~2015 IEEE}
% Remember, if you use this you must call \IEEEpubidadjcol in the second
% column for its text to clear the IEEEpubid mark.



% use for special paper notices
%\IEEEspecialpapernotice{(Invited Paper)}




% make the title area
\maketitle

% As a general rule, do not put math, special symbols or citations
% in the abstract or keywords.
\begin{abstract}
The paper carries out performance analysis of a multiuser full-duplex (FD) communication system. Multiple FD UEs share the same spectrum resources, simultaneously, at both the uplink and downlink.  This results in co-channel interference (CCI) at the downlink of a UE from uplink signals of other UEs. This work proposes the use of diversity gain at the receiver to mitigate the effects of the CCI. For this an architecture for the FD eNB and FD UE is proposed and corresponding downlink operation is described. Finally, the performance of the system is studied in terms of downlink capacity of a UE. It is shown that through the  deployment of sufficient number of transmit and receive antennas at the eNB and UEs, respectively, significant improvement in  performance can be achieved in the presence of CCI.  
\end{abstract}

% Note that keywords are not normally used for peerreview papers.
\begin{IEEEkeywords}
Full-duplex, Multiuser Communication, Co-channel interference, Diversity gain, Downlink Capacity.
\end{IEEEkeywords}






% For peer review papers, you can put extra information on the cover
% page as needed:
% \ifCLASSOPTIONpeerreview
% \begin{center} \bfseries EDICS Category: 3-BBND \end{center}
% \fi
%
% For peerreview papers, this IEEEtran command inserts a page break and
% creates the second title. It will be ignored for other modes.
\IEEEpeerreviewmaketitle


\section{Introduction}

\IEEEPARstart{T}{he} inevitable high bandwidth requirement in the future cellular network has made researchers to come up with revolutionary ideas in recent times. One such idea is the introduction of full-duplex (FD) communication. A FD systems make the simultaneous in-band transceiving feasible, i.e, simultaneous uplink and downlink operation using the same spectrum resources. In recent years, extensive work has been done in the area of self-interference cancellation (SIC) design, including for compact devices like laptops and smart phones, enabling FD communication for both single and multiple antenna transceiver units \cite{full_duplex, compact}. The designs aim in optimal cancellation of interference from the receiver chains introduced by the transmitter chains of the transceiver unit. While this is far from true today for cellular networks, sufficient progress is being made in this direction to start considering the FD model and its implications \cite{fd_small}. \par

     In the conventional LTE system, for uplink and downlink, single carrier frequency division multiple access (SC-FDMA) and orthogonal frequency division multiple access (OFDMA) is used for multiple access respectively. For FD operation, same subcarriers can be allocated to a UE for uplink and downlink. Hence, the use of SC-FDMA for both uplink and downlink is proposed \cite{ants}, due to its advantages over OFDMA in terms of bit error rate (BER) performance and energy efficiency \cite{sc_fdma}. \par
     
    In this work, multiple user equipments (UEs) are considered operating in FD mode on the same set of subcarriers simultaneously. However, the use of the same subcarriers for both uplink and downlink results in co-channel interference (CCI) in downlink of a UE from uplink signals of other UEs operating in the same subcarriers. The prior work \cite{ants}, considers deploying the smart antenna technique at the UEs with highly spatially correlated multiple antennas.  In this work, a converse scenario is considered with a highly scattered environment which prevents the use of the method proposed in \cite{ants} to mitigate the CCI. Hence, here the use of diversity gain at the UEs is analyzed to mitigate the effect of CCI and allowing the UEs to share the same spectrum resources. Also, the proposed architecture for the UE has less computational complexity than the architecture deploying smart antenna technology proposed in the previous work. \par
    
    The rest of the paper is organized as follows. In section 2, the system model for the proposed method is discussed. Section 3 presents the downlink operation and diversity gain methodology to tackle the CCI at the downlink of a UE. The system performance is analyzed through simulations in section 4. Finally, we conclude in section 5. \par
    
    Notation: $[.]^T$,$(.)^H$ denote transpose and Hermitian respectively. $||.||$  denotes the Euclidean norm. $(.)^d$ and $(.)^u$ denote downlink and uplink components, respectively. \par
    
\section{System Model}

In this work, for facilitating FD communication, both the eNB and UEs operate in FD mode. An FD eNB with $N_e$ antennas and $K$ FD UEs with $N_r$ antennas is considered each such that $N_e \geq KN_r$. Assuming highly scattered environment, the number of data streams per UE is given by $Q=N_r$. For the proposed transceiver architecture shown in  fig.1 and fig.2 for eNB and UE respectively, the Analog and Digital SIC unit at the RF front end,  includes the SIC circuitry  \cite{full_duplex,compact} enabling the FD communication.  The details of SIC design are not discussed here. \par    

 \begin{figure}
\centering
\includegraphics[width=3.25in ,height=2.0in]{fig1.png}
  % where an .eps filename suffix will be assumed under latex, 
% and a .pdf suffix will be assumed for pdflatex; or what has been declared
% via \DeclareGraphicsExtensions.
\vspace{-0.5em}
   \caption{Transceiver structure for the proposed eNB architecture}
   \label{fone}
\end{figure}

\begin{figure}
\centering
\includegraphics[width=3.25in ,height=1.75in]{fig2.png}
  % where an .eps filename suffix will be assumed under latex, 
% and a .pdf suffix will be assumed for pdflatex; or what has been declared
% via \DeclareGraphicsExtensions.
\vspace{-0.5em}
   \caption{Transceiver structure for the proposed UE architecture}
   \label{ftwo}
   \vspace{-1.5em}
\end{figure}

  Let each subcarrier allocated be shared by $\acute{K}$ UEs simultaneously, where $\acute{K}$ is given by \cite{sc_fdma}:
  
  \begin{equation}
      \label{one}
     \acute{K} = min\Big( \left \lfloor{\frac{N_e}{N_r}}\right \rfloor ,K \Big) 
\end{equation}. 

Each UE is allocated M (=$\left \lfloor{\frac{N\acute{K}}{K}}\right \rfloor$ ) subcarriers, where $N$ is the total number of subcarriers available.  Keeping this in mind, a case of $\acute{K}=K$, i.e., all the $K$ UEs are allocated all the $N$ subcarriers, is considered. However, the appropriate number of co-existing UEs depends on the CCI experienced by the UEs in their downlink and hence can be $\leq \acute{K}$. The channel between each eNB antenna and UEs antenna is assumed to be frequency selective with $L$ taps. The FD operation allows the channel reciprocity between downlink and uplink, $h^{u}_{j,i,k} (b) = h^{d}_{j,i,k} (b)$, where  $ h^{u}_{j,i,k} (b)$ and $h^{d}_{j,i,k} (b) $  denotes $b^{th}$ time domain  uplink and downlink channel coefficient between $j^{th}$ antenna at the eNB and $k^{th}$  antenna of the $i^{th}$ UE, respectively, $b = 0, 1, 2, ... , L-1$, $j = 1 , 2, ..., N_e$, $k = 1, 2,...,N_r $ and $i = 1,2, ...,K$. \par

     In downlink, the channel reciprocity property of FD communication enables the transmitter (eNB) to acquire CSI with ease. The CSI can be used for precoding the UE data at eNB to perform the SVD based beamforming. In the uplink, successive interference cancellation with optimal ordering (SSIC-OO) algorithm \cite{ants} is used at the eNB to segregate signals of UEs sharing the same subcarriers. In \cite{ants}, a smart antenna approach is deployed to avoid the CCI at downlink of UEs. Here, the paper concentrates on the \textit{downlink} operation and to avoid CCI at downlink of a UE, diversity combining technique is used to minimize interference from the uplink signals of other co-existing UEs. \par

%\begin{figure}
%\centering
%\includegraphics[width=2.5in ,height=1.5in]{fig3.png}
  % where an .eps filename suffix will be assumed under latex, 
% and a .pdf suffix will be assumed for pdflatex; or what has been declared
% via \DeclareGraphicsExtensions.
%\vspace{-0.5em}
 %  \caption{Structure for the Beamforming unit for the proposed UE architecture and the formation of directed beam toward eNB}
  % \label{fthree}
   %\vspace{-1.0em}
%\end{figure}


  
\section{Full-Duplex Multiuser Downlink Operation}
    In the downlink, let $\mathbf{x}^d_i$  denote the $i^{th}$ UE information data block of length $M$, shown in fig.\ref{fone}. The output of the M-point DFT block is given by:

\begin{equation}
\label{two}
\mathbf{\bar{x}}_i^d = \mathbf{F}_M \mathbf{x}_i^d
       \end{equation} 
    
                                                                 
where $i=1,2,...,K$, $\mathbf{F}_M$  is the M-point DFT matrix and $\mathbf{x}_i^d = [x_i^d(1),x_i^d(2),...,x_i^d(m),...,x_i^d(M)]^T $. $x_i^d(m)$ is the $m^{th}$ data symbol of $i^{th}$ user and unlike in \cite{sc_fdma} same $M$ symbols are transmitted in all the $Q$ data streams. This output is then passed through the pre-processing / subcarrier allocation block. Let  $\mathbf{P}^i_m$ denotes $N_e X Q$  precoding vector for the $i^{th}$ UE on the $m^{th}$ subcarrier. The precoded output vector of size $N_e X 1$ for the $i^{th}$ UE on the $m^{th}$ subcarrier is given by:

\begin{equation}
\label{three}
\mathbf{z}^d_i(m) = \mathbf{P}^i_m \mathbf{\tilde{x}}_i^d (m)
       \end{equation}   
       
where $m=1,2,...,M$ and $\mathbf{z}^d_i(m) = [z^d_{i,1}(m), z^d_{i,2}(m),...,z^d_{i,N_e}(m)]^T$. $\mathbf{\tilde{x}}_i^d (m)= (\bar{x}_i^d(m) \mathbf{J}_{1,Q})^T$ where $\bar{x}_i^d(m)$ is the $m^{th}$ element of $\mathbf{\bar{x}}_i^d$ and $\mathbf{J}_{1,Q}$ is the $1 X Q$ unit matrix. Let $\mathbf{A}^i$, represents the $N X M$ subcarrier allocation matrix for the $i^{th}$ UE \cite{sc_fdma}. The transmit signal vector, at the $j^{th}$ antenna of the eNB, after subcarrier allocation and IDFT operation is given by:  

\begin{equation}
\label{four}
\mathbf{s}^d_j = \mathbf{\bar{F}}_N \mathbf{e}^d_j
\vspace{-0.10em}
       \end{equation} 
       
where $j=1,2,...,N_e$, $\mathbf{\bar{F}}_N$ denotes the N-point IDFT matrix, $\mathbf{e}^d_j = \sum_{i=1}^{K} \mathbf{A}^i \mathbf{z}^d_{i,j}$ and $\mathbf{z}^d_{i,j}=[z^d_{i,j}(1), z^d_{i,j}(2),...,z^d_{i,j}(M)]^T$. This signal is then transmitted after the addition of the cyclic prefix (CP). \par

In this work, the UE transceiver unit (Fig.\ref{ftwo}) consists of a uniformly spaced linear antenna array of $N_r$ elements with an inter element distance of $h$. The angle with respect to the array normal at which the plane wave impinges upon the array is represented as $\psi$. Each $k^{th}$ antenna introduces some phase delay $\alpha^{k-1}_x = \exp(-j2 \pi (k-1) \frac{h}{\lambda} \sin(\psi_x))$ to the received signal, where $\lambda$ is the wavelength. The index $x=e$ for direction of arrival (DoA) of the eNB $(\psi_e)$ w.r.t $l^{th}$ UE or $x=q$ for DoA of the $q^{th}$ UE $(q \neq l)$ w.r.t $l^{th}$ UE, whose uplink signal results in CCI at downlink of the $l^{th}$ UE. Algorithms like Root-MUSIC (due less computational complexity) can be used for estimating the DoAs. \textit{However, the proposed method is independent of availability of information about DoA of interfering UEs.}\par

     At the $l^{th}$ UE, the downlink received signal at $k^{th}$ antenna is obtained after multiplying the factor $\bigtriangleup_k$ followed by SIC cancellation in receive chain, where $\bigtriangleup_k = (\alpha^{k-1}_e)^*$. This is required for \textit{co-phasing} of the downlink signal from the eNB by removing the phase $(\alpha^{k-1}_e)$ introduced by the $k^{th}$ antenna \cite{mrc}. Hence, the received signal at $k^{th}$ antenna, is given by:   \vspace{-1.00em}
     
     \begin{multline}
       \label{five}
 \mathbf{y}^d_{l,k}= \sum_{j=1}^{N_e} \mathbf{h}^d_{j,l,k} \otimes \mathbf{s}^d_j + \mathbf{n}_{l,k}  \\
+ \Big [ \sum_{\substack{ q=l \\ q \neq l}}^{K} \sum_{\acute{k}=1}^{N_r} \mathbf{h}^u_{q,\acute{k},l,k} \otimes \mathbf{s}^u_{q,\acute{k}}] \alpha^{k-1}_{q}  \bigtriangleup_k 
       \end{multline} 
       
       
where $ l=1,2,...,K$, $k=1,2,...,N_r$. $\otimes$  denotes the circular convolution operation. $\mathbf{y}^d_{l,k}$ is $N_{cp}X1$ vector where $N_{cp}$ is the receive symbol size with CP. $\mathbf{h}^{d}_{j,l,k}=[h^d_{j,l,k}(0),h^d_{j,l,k}(1),...,h^d_{j,l,k}(L-1),(N_{cp}-L)zeros]^T$ is the complex i.i.d Rayleigh fading channel coefficient between  $j^{th}$ antenna of the eNB and $k^{th}$ antenna of the $l^{th}$ UE in the downlink.  $\mathbf{n}_{l,k} \in N(0,\sigma_n^2 \mathbf{I}_{N_{cp}})$ is the channel noise at $k^{th}$ antenna of the $l^{th}$ UE. $\mathbf{h}^u_{q,\acute{k},l,k}=[h^u_{q,\acute{k},l,k}(0),h^u_{q,\acute{k},l,k}(1),...,h^u_{q,\acute{k},l,k}(L-1)),(N_{cp}-L)zeros]^T$ is the complex i.i.d Rayleigh fading channel coefficient between $\acute{k}^{th}$ antenna of the $q^{th}$ UE and $k^{th}$ antenna of the $l^{th}$ UE. $\mathbf{s}^u_{q,\acute{k}}$ is the uplink signal from $\acute{k}^{th}$ antenna of the $q^{th}$ UE. $\alpha^{k-1}_{q}$ is the spatial response (or the phase delay) of $k^{th}$ antenna of the $l^{th}$ UE in the DoA of $\psi_{q}$. This signal can be represented by:

 \begin{equation}
       \label{six}
 \mathbf{y}^d_{l,k}= \sum_{j=1}^{N_e} \mathbf{h}^d_{j,l,k} \otimes \mathbf{s}^d_j + \mathbf{I}_{l,k} 
       \end{equation} 
       
where $\mathbf{I}_{l,k} = \mathbf{n}_{l,k} + \Big [ \sum_{\substack{ q=l \\ q \neq l}}^{K} \sum_{\acute{k}=1}^{N_r} \mathbf{h}^u_{q,\acute{k},l,k} \otimes \mathbf{s}^u_{q,\acute{k}} \Big ] \alpha^{k-1}_{q}  \bigtriangleup_k  $ is the noise and interference suffered by the $l^{th}$ UE at the $k^{th}$ antenna. After the removal of CP, the signal is converted to the frequency domain: 

                                                                   
       \begin{equation}
       \label{seven}
      \mathbf{\tilde{y}}^d_{l,k}= \mathbf{F}_N \mathbf{y}^d_{l,k}  
      =\sum_{j=1}^{N_e} \mathbf{{H}}^d_{j,l,k} \mathbf{e}^d_j + \mathbf{\tilde{I}}_{l,k}  
       \end{equation}
       
where $\mathbf{F}_N$ is the N-point DFT matrix. ${\mathbf{H}}^d_{j,l,k} = diag(\mathbf{F}_N \mathbf{h}^d_{j,l,k})$ is the $N X N$ diagonal matrix whose diagonal elements are frequency domain coefficients between $j^{th}$ transmit antenna of the eNB and $k^{th}$ antenna of the $l^{th}$ UE. 

The signal is then subjected to the subcarrier deallocation and after simplifications \cite{ants,sc_fdma}, we get:

   \begin{equation}
       \label{eight}
     \mathbf{\bar{y}}^d_l (m) =  \mathbf{H}^d_l (m) \sum_{i=1}^{K} \mathbf{P}^i_m \mathbf{\tilde{x}}^d_i (m) + \mathbf{\bar{I}}_l (m)
       \end{equation}

where $\mathbf{\bar{y}}_l (m) = [\bar{y}_{l,1}(m), \bar{y}_{l,2}(m),...,\bar{y}_{l,N_r}(m)]^T$, $\bar{y}_{l,k}(m)$ is the $m^{th}$ element of $\mathbf{\bar{y}}_{l,k} = \mathbf{\bar{A}}^i \mathbf{\tilde{y}}^d_{l,k}$. $\mathbf{\bar{A}}^i$ is the $M X N$ deallocation matrix given by $\mathbf{\bar{A}}^i = (\mathbf{A}^i)^T$. $\mathbf{H}^d_l (m)$  is the $N_r X N_e$ frequency domain channel coefficient matrix of the $l^{th}$ UE on the $m^{th}$ subcarrier. $(k,j)^{th}$ entry in the $\mathbf{H}^d_l (m)$ is the $m^{th}$ diagonal element of matrix $\mathbf{{H}}^d_{j,l,k}$. $\mathbf{\bar{I}}_l (m)$  is channel noise and interference for the $l^{th}$ UE on the $m^{th}$ subcarrier. 

The SVD decomposition of channel matrix  is given by $\mathbf{H}^d_l (m)= \mathbf{U}^d_{m,l} \mathbf{E}_{m,l}^d (\mathbf{V}^d_{m,l})^H$. $\mathbf{U}^d_{m,l}$ is a $N_r X Q$ unitary matrix containing the eigenvectors corresponding to non-zero eigenvalues of $\mathbf{H}^d_l(m) (\mathbf{H}^d_l(m))^H$, $\mathbf{E}^d_{m,l}$  is a $Q X Q$ diagonal matrix containing the non-zero eigenvalues $(\lambda^d_{m,l})$ of $\mathbf{H}^d_l(m) (\mathbf{H}^d_l(m))^H$ such that $\mathbf{E}^d_{m,l}=diag((\lambda^d_{m,l,1})^{1/2},(\lambda^d_{m,l,2})^{1/2}, ...,(\lambda^d_{m,l,Q})^{1/2})$ and $\mathbf{V}^d_{m,l}$ is a $N_e X Q$  unitary matrix containing the eigenvectors corresponding to non-zero eigenvalues of $(\mathbf{H}^d_l(m))^H \mathbf{H}^d_l(m)$. The received signal vector on the $m^{th}$ subcarrier due to all UEs sharing the subcarriers is hence can be given by: 

                          \begin{equation}
       \label{nine}
      \mathbf{\bar{y}}^d (m) = \mathbf{U}^d_m \mathbf{E}^d_m (\mathbf{V}^d_m)^H \mathbf{P}_m \mathbf{\tilde{x}}^d(m) + \mathbf{\bar{I}} (m)
       \end{equation} 
       
where $\mathbf{\bar{y}}^d (m) =  [(\mathbf{\bar{y}}^d_l(m))^T,(\mathbf{\bar{y}}^d_2(m))^T,...,(\mathbf{\bar{y}}^d_K(m))^T]^T$, $\mathbf{U}^d_m= diag(\mathbf{U}^d_{m,1},\mathbf{U}^d_{m,2},...,\mathbf{U}^d_{m,K})$, $\mathbf{V}^d_m = [\mathbf{V}^d_{m,1},\mathbf{V}^d_{m,2},...,\mathbf{V}^d_{m,K}]$, $\mathbf{E}^d_m=diag(\mathbf{E}^d_{m,1}, \mathbf{E}^d_{m,2},...,\mathbf{E}^d_{m,K})$, $\mathbf{P}_m=[\mathbf{P}^1_m, \mathbf{P}^2_m,...,\mathbf{P}^K_m]$, $\mathbf{\tilde{x}}^d(m)=[(\mathbf{\tilde{x}}_1^d(m))^T, (\mathbf{\tilde{x}}_2^d(m))^T,...,(\mathbf{\tilde{x}}_K^d(m))^T]^T$ and $\mathbf{\bar{I}}(m)=[(\mathbf{\bar{I}}_1(m))^T,(\mathbf{\bar{I}}_2(m))^T,...,(\mathbf{\bar{I}}_K(m))^T]^T$. \par

The interference from the downlink of other UEs on the $l^{th}$ UE can be completely eliminated by choosing the precoding matrix as $\mathbf{P}_m=[(\mathbf{V}^d_m)^H]^+ $ which is the pseudo inverse of $\mathbf{V}^d_m$. At the $l^{th}$ UE, the received signal on the $m^{th}$ subcarrier after post-processing, .i.e, multiplying with $(\mathbf{U}^d_{m,l})^H$ is given by \cite{sc_fdma}: \vspace{-0.50em}
      
 \begin{equation}
       \label{ten}
             \mathbf{\tilde{y}}^d_{l} (m) = \mathbf{{E}}^d_{m,l} \mathbf{\tilde{x}}^d_{l} (m) + \mathbf{\tilde{I}}_{l}(m)
        \end{equation} 
       
Now, defining $\mathbf{\bar{E}}^d_{m,l} = [E^d_{m,l}(1,1),E^d_{m,l}(2,2),...,E^d_{m,l}(j,j),...,E^d_{m,l}(Q,Q)]^T$ where $E^d_{m,l}(j,j)$ is the $j^{th}$ diagonal element of $\mathbf{E}^d_{m,l}$. As $\mathbf{\tilde{x}}^d_l(m) = (\bar{x}^d_{l}(m) \mathbf{J}_{1,Q})^T$, above equation can be rearranged to: \vspace{-0.50em}

\begin{equation}
       \label{eleven}
\mathbf{\tilde{y}}^d_{l} (m) = \mathbf{\bar{E}}^d_{m,l} \bar{x}^d_{l} (m) + \mathbf{\tilde{I}}_{l}(m) 
\end{equation}

Now, assuming rich scattering environment and equal channel conditions between the $l^{th}$ UE and $K-1$ interfering UEs (especially in a small cell scenario), the received signal for the $l^{th}$ user on the $m^{th}$ subcarrier can be obtained by MRC \footnote{MMSE equalizer requires knowledge of covariance of the interference at the receiver. As this information is not available at the UE in downlink, MRC is deployed which requires only the downlink channel estimate.} \cite{mrc} as:

   
   \begin{equation}
       \label{twelve}
      \hat{\bar{x}}^d_l(m)=((\mathbf{\bar{E}}^d_l(m))^H (\mathbf{\bar{E}}^d_l(m))^{-1} (\mathbf{\bar{E}}^d_l(m))^H \mathbf{\tilde{y}}^d_l(m)
       \end{equation} 
       
This signal for the $l^{th}$ UE is then converted to time domain by an M-point IDFT operation given by:

\begin{equation}
       \label{thirteen}
     \mathbf{\hat{x}}^d_l = \mathbf{\bar{F}}_M \mathbf{\hat{\bar{x}}}^d_l
       \end{equation} 
       
where $l=1,2,...,K$ and $\mathbf{\bar{F}}_M$  is M-point inverse IDFT matrix and $\mathbf{\hat{\bar{x}}}^d_l =[\hat{\bar{x}}^d_l (1), \hat{\bar{x}}^d_l (2), ...,\hat{\bar{x}}^d_l (M)]^T$ . This is used for decoding of signal for the $l^{th}$  UE. For the analysis of system performance, the effective SINR $(\gamma_{eff^m_l})$ \footnote{In this work, equal power allocation across all the $Q$ streams and $N_r$ streams is considered at the downlink and uplink, respectively.} is considered for the $l^{th}$ UE on the $m^{th}$ subcarrier from the equation (\ref{twelve}):

\begin{equation}
       \label{fourteen}
      \gamma_{eff^m_l} = [\sigma_{\bar{n}_l}^2 + \sum_{\substack{ q=l \\ q \neq l}}^{K} \sigma_q^2 \beta^q_m ]^{-1}||\mathbf{\bar{E}}^d_{m,l}||^2 (\beta^l_m Q^{-1})
       \end{equation}

where $\sigma_{\bar{n}_l}^2 = E[|\bar{n}_{l,k}(m)|^2], \forall m, l, k$ such that $\bar{n}_{l,k}(m)$ is the frequency domain i.i.d noise variable for $k^{th}$ antenna of the $l^{th}$ UE on the $m^{th}$ subcarrier. $\sum_{\substack{ q=l \\ q \neq l}}^{K} \sigma_q^2 \beta_m^q$ is the total interference power, such that $\sigma_q^2 = |\mathbf{H}^u_{q,\acute{k},l,k}(m)|^2, \forall m,\acute{k}, k, l, k$ where $\mathbf{H}^u_{q,\acute{k},l,k}(m)$ is the frequency domain i.i.d channel coefficient between $\acute{k}^{th}$ antenna of the $q^{th}$ UE and $k^{th}$ antenna of the $l^{th}$ UE on the $m^{th}$ subcarrier and $\beta_m^q = |\tilde{x}^u_q(m)|^2, \forall m$ is the total uplink power allocated to the signal of the $q^{th}$ UE on the $m^{th}$ subcarrier.  $\beta_m^l = |\tilde{x}^d_l(m)|^2, \forall m, l$ is the total power allocated to downlink signal of the $l^{th}$ UE on the $m^{th}$ subcarrier. Now, considering $\beta_m^l \geq \beta_m^q, \forall m,q$, to simplify the equation (\ref{fourteen}), we take $\beta_m^q = \beta_m^u, \forall m,q$, $\beta_m^l= Q\beta_m^u $ and $\sigma_q^2 = \sigma_u^2, \forall q$. Hence, the equation (\ref{fourteen}) can be expressed as: \vspace{-0.50em}

\begin{equation}
       \label{fifteen}
      \gamma_{eff^m_l} = [\sigma_{\bar{n}_l}^2 + (K-1) \sigma_u^2 \beta^u_m ]^{-1}||\mathbf{\bar{E}}^d_{m,l}||^2 \beta_m^u
       \end{equation}

The derivation of the expression is included in the appendix. It is important to observe that the effective SINR can be controlled by the gain factor, .i.e, $||\mathbf{\bar{E}}^d_{m,l}||^2$. This is discussed through simulation results in the next section. 

\section{Simulation Results}
To validate the inclusion of SIC design in the FD system in the proposed architecture, MATLAB simulations are carried out for BER performance in downlink and uplink in \cite{ants}. Here, an FD eNB and three spatially uncorrelated FD UEs (say UE1, UE2 and UE3) sharing the same spectrum resources at both the downlink and uplink, are considered. The capacity \footnote{The capacity here is defined as number of correct bits received per second per Hertz} $(bits/s/Hz)$ at the downlink of UE1 is analyzed with two other UEs acting as the interference source. This work assumes perfect cancellation of self-interference for the FD operation. The channel between each antenna of eNB and UEs' is taken as frequency selective with $L=10$. The modulation scheme used is 16-QAM (no coding). The bandwidth allocated to the UEs is taken to be 3 MHz which is split into 256 subcarriers, out of which 180 subcarriers are occupied by the UEs. A cyclic prefix of duration $4.69 \mu s$ is used. The transmit powers of the UEs are normalized to unity, .i.e, $\beta_m^u = 1, \forall m$ \par

Fig.\ref{fthree} and fig.\ref{ffour} show the downlink capacity of UE1 vs. the channel SNR for the increasing number of the receive antennas $(N_r)$ and transmit antennas $(N_e)$ at the UEs and the eNB, respectively. In case of the increasing $N_r$ at the UEs, the number of transmit antennas at the eNB is kept constant at $N_e = 20$. It can be seen from fig.\ref{fthree} that with the increase in $N_r$, there is an improvement in the downlink capacity of the UE. This improvement is due to the increase in the diversity order which results in higher magnitude of $||\mathbf{\bar{E}}^d_{m,l}||^2, \forall m, l$ improving the  $\gamma_{eff^m_l}, \forall m$. Similarly, for the increasing $(N_e)$ at the eNB, the number of receive antennas at the UE is kept constant at $N_r = 4$. An improvement in the downlink capacity of the UE can be observed (fig.\ref{ffour}) with the increase in $N_e$. This is due to increase in magnitude of each eigenvalues, .i.e, $E^d_{m,l}(j,j), j=1,2,...,Q, \forall m, l$ resulting in the higher $||\mathbf{\bar{E}}^d_{m,l}||^2$ and hence improving the $\gamma_{eff^m_l}$. Also, from fig.3 and fig.4, it can be observed that at the higher channel SNR region, the downlink system performance is only interference limited. \par 

In all the above simulations, with the increase in $N_r$ and $N_e$, the downlink capacity approaches the ideal value where there is no CCI, .i.e, $K = 1$. The results are also compared with the conventional scenario of no diversity gain at the UE, .i.e, $N_r =1$. It can be observed that there is a significant improvement in the downlink capacity with the diversity gain. Hence, increasing both $N_r$ and $N_e$ improves the system performance at the downlink in the presence of CCI. However, increasing $N_r$ at the UEs, increases the computational complexity at the UEs including the power consumption. With the power consumption not a constraint at the eNB, comparatively, and profound research on massive MIMO (deploying large $N_e$) in recent years \cite{massive}, the prospect of using large antennas arrays at eNB to tackle the CCI in case of the FD downlink seems so be encouraging. Moreover, the channel reciprocity property of the FD communication will aid the CSI actualization at the eNB.   \par 

\begin{figure}
\centering
\includegraphics[width=2.75in ,height=1.65in]{fig3.png}
  % where an .eps filename suffix will be assumed under latex, 
% and a .pdf suffix will be assumed for pdflatex; or what has been declared
% via \DeclareGraphicsExtensions.
\vspace{-0.5em}
   \caption{Downlink capacity vs channel SNR of UE1 for different $N_r$}
  % \vspace{-0.55em}
   \label{fthree}
   \vspace{-1.0em}
\end{figure}

\begin{figure}
\centering
\includegraphics[width=2.75in ,height=1.65in]{fig4.png}
  % where an .eps filename suffix will be assumed under latex, 
% and a .pdf suffix will be assumed for pdflatex; or what has been declared
% via \DeclareGraphicsExtensions.
\vspace{-0.5em}
   \caption{Downlink capacity vs channel SNR of UE1 for different $N_e$}
   \vspace{-0.45em}
   \label{ffour}
      \vspace{-1.25em}
\end{figure}


\section{Conclusion}
 In this paper, an analysis is carried out to show the use of diversity gain in making the multiuser full-duplex communication feasible. An architecture for the FD eNB and FD UE enabling the corresponding downlink operation is described and performance of the system is studied in terms of the downlink capacity for a UE. From the simulation, it can be seen that the downlink capacity of a UE is significantly improved in the presence of CCI by increasing the number of receive and transmit antennas. However, the use of large number of antennas at eNB is suggested to be a favorable option keeping in mind the power constraint at the UEs and encouraging progress made in the field of massive MIMO. 
 

\section*{Appendix}

The noise and interference in the equation (\ref{six}) is given by:

\begin{equation}
\label{sixteen}
 \mathbf{I}_{l,k} = \mathbf{n}_{l,k} + \Big [ \sum_{\substack{ q=l \\ q \neq l}}^{K} \sum_{\acute{k}=1}^{N_r} \mathbf{h}^u_{q,\acute{k},l,k} \otimes \mathbf{s}^u_{q,\acute{k}} \Big ] \alpha^{k-1}_{q}  \bigtriangleup_k
\end{equation}
       
 After removing the CP and converting it to the frequency domain, we have:
 
 \begin{equation}
\label{seventeen}
\begin{split}
  \mathbf{\tilde{I}}_{l,k} &= \mathbf{F}_N \mathbf{I}_{l,k} \\
   &= \mathbf{\tilde{n}}_{l,k} + \Big [ \sum_{\substack{ q=l \\ q \neq l}}^{K} \sum_{\acute{k}=1}^{N_r} \mathbf{H}^u_{q,\acute{k},l,k} \mathbf{x}^u_{q,\acute{k}} \Big ] \alpha^{k-1}_{q}  \bigtriangleup_k
   \end{split}
\end{equation}

where $\mathbf{H}^u_{q,\acute{k},l,k} = diag(\mathbf{F}_N \mathbf{h}^u_{q,\acute{k},l,k})$ is the $N X N$ diagonal matrix whose diagonal elements are frequency domain coefficients between $\acute{k}^{th}$ transmit antenna of the $q^{th}$ UE and $k^{th}$ receive antenna of the $l^{th}$ UE. $\mathbf{x}^u_{q,\acute{k}}$ is the transmit signal from $\acute{k}$ antenna of the $q^{th}$ UE. This signal is then subjected to the subcarrier deallocation and after simplifications, similar to equation (\ref{eight}), we get:

   \begin{equation}
       \label{eighteen}
     \mathbf{\bar{I}}_l (m) = \mathbf{\bar{n}}_l(m) + \sum_{\substack{ q=l \\ q \neq l}}^{K} \mathbf{\alpha}_q \mathbf{\bigtriangleup} \mathbf{H}^u_{q,l}(m)  \mathbf{\bar{x}}^u_q (m)
       \end{equation}

where $\mathbf{\bar{I}}_l (m) = [\bar{I}_{l,1}(m), \bar{I}_{l,2}(m),... ,\bar{I}_{l,N_r}(m)]^T$, $ \mathbf{\alpha}_q =  diag(\alpha_q^{0}, \alpha_q^{1}, ...., \alpha_q^{N_r-1})$ and $\mathbf{\bigtriangleup} = diag(\bigtriangleup_1, \bigtriangleup_2, ..., \bigtriangleup_{N_r})$. $\mathbf{H}^u_{q,l} (m)$  is the $N_r X N_r$ frequency domain channel coefficient vector between the $l^{th}$ U and the $q^{th}$ UE on the $m^{th}$ subcarrier. This signal for $k^{th}$ antenna of the $l^{th}$ UE is given by:

   \begin{equation}
       \label{nineteen}
     \bar{I}_{l,k}(m) = \bar{n}_{l,k}(m) + \Big [ \sum_{\substack{ q=l \\ q \neq l}}^{K} \sum_{\acute{k}=1}^{N_r} H^u_{q,\acute{k},l,k}(m) \bar{x}^u_{q,\acute{k}}(m) \Big ] \alpha^{k-1}_{q}  \bigtriangleup_k
    \end{equation}
    
Now, considering the fact that the $\bar{n}_{l,k}(m)$ and $H^u_{q,\acute{k},l,k}(m)$ are i.i.d, the power in the above signal is given by:

\begin{equation}
       \label{twenty}
       \begin{split}
     E[|\bar{I}_{l,k}(m)|^2] &= \sigma^2_{\bar{n}_{l}(m)} +  \sum_{\substack{ q=l \\ q \neq l}}^{K} \sum_{\acute{k}=1}^{N_r} \sigma_q^2 (\beta^q_m N_r^{-1}) \\
     & = \sigma^2_{\bar{n}_{l}(m)} + \sum_{\substack{ q=l \\ q \neq l}}^{K} \sigma_q^2 \beta^q_m, \forall l, k
     \end{split}
     \end{equation}
     
The signal is now subjected to post-processing from equation (\ref{ten}), .i.e, multiplying with $(\mathbf{U}^d_{m,l})^H$:

\begin{equation}
\label{twenty_one}
  \mathbf{\tilde{I}}_l(m) = (\mathbf{U}^d_{m,l})^H \mathbf{\bar{I}}_l(m)
\end{equation}
   
where $\mathbf{\tilde{I}}_l(m)=[{\tilde{I}}_{l,1}(m), {\tilde{I}}_{l,2}(m), ...,{\tilde{I}}_{l,Q}(m)]^T$. As $Q=N_r$ and $\mathbf{U}^d_{m,l}$ is an unitary matrix, we have $|\mathbf{\tilde{I}}_l(m)|^2 = |\mathbf{\bar{I}}_l(m)|^2$. The power of the signal at the $j^{th}$ data stream of the $l^{th}$ UE is given by:

\begin{equation}
\label{twenty_two}
\begin{split}
E[|\tilde{I}_{l,j}(m)|^2] &= E[|\bar{I}_{l,k}(m)|^2] \\ 
&= \sigma^2_{\bar{n}_{l}(m)} + \sum_{\substack{ q=l \\ q \neq l}}^{K} \sigma_q^2 \beta^q_m, \forall j
\end{split}
     \end{equation}
     
The received signal for the $l^{th}$ user at the $m^{th}$ subcarrier obtained by MRC is given in equation (\ref{twelve}) as:
   
   \begin{equation}
       \label{twenty_three}
       \begin{split}
      \hat{\bar{x}}^d_l(m) &=((\mathbf{\bar{E}}^d_l(m))^H (\mathbf{\bar{E}}^d_l(m))^{-1} (\mathbf{\bar{E}}^d_l(m))^H \mathbf{\tilde{y}}^d_l(m) \\
        &= \bar{x}_l^d(m) + ((\mathbf{\bar{E}}^d_l(m))^H (\mathbf{\bar{E}}^d_l(m))^{-1} (\mathbf{\bar{E}}^d_l(m))^H \mathbf{\tilde{I}}_l(m)
        \end{split}
       \end{equation} 
       
From this signal, the interference and noise term is represented by:

\begin{equation}
\label{twenty_four}
  I_n = ((\mathbf{\bar{E}}^d_l(m))^H (\mathbf{\bar{E}}^d_l(m))^{-1} (\mathbf{\bar{E}}^d_l(m))^H \mathbf{\tilde{I}}_l(m)
\end{equation}

As the elements of $\mathbf{\tilde{I}}_l(m)$ are i.i.d, the power in the above signal is given by:

\begin{equation}
\label{twenty_five}
 \begin{split}
 E[|I_n|^2] &= [||\mathbf{E}_{m,l}^d||^2]^{-1} |\tilde{I}_{l,k}(m)|^2 \\ 
   & = [||\mathbf{E}_{m,l}^d||^2]^{-1} (\sigma^2_{\bar{n}_{l}(m)} + \sum_{\substack{ q=l \\ q \neq l}}^{K} \sigma_q^2 \beta^q_m) 
   \end{split}
\end{equation}

Hence, the effective SINR $(\gamma_{eff^m_l})$ for the $l^{th}$ UE on the $m^{th}$ subcarrier can now be given by:
\begin{equation}
       \label{twenty_six}
       \begin{split}
      \gamma_{eff^m_l} &= [\sigma_{\bar{n}_l}^2 + \sum_{\substack{ q=l \\ q \neq l}}^{K} \sigma_q^2 \beta^q_m ]^{-1}||\mathbf{\bar{E}}^d_{m,l}||^2 |\bar{x}_l^d (m)|^2 \\
            &= [\sigma_{\bar{n}_l}^2 + \sum_{\substack{ q=l \\ q \neq l}}^{K} \sigma_q^2 \beta^q_m ]^{-1}||\mathbf{\bar{E}}^d_{m,l}||^2 (\beta^l_m Q^{-1})
      \end{split}
       \end{equation}


% conference papers do not normally have an appendix


% use section* for acknowledgement

% trigger a \newpage just before the given reference
% number - used to balance the columns on the last page
% adjust value as needed - may need to be readjusted if
% the document is modified later
%\IEEEtriggeratref{8}
% The "triggered" command can be changed if desired:
%\IEEEtriggercmd{\enlargethispage{-5in}}

% references section

% can use a bibliography generated by BibTeX as a .bbl file
% BibTeX documentation can be easily obtained at:
% http://www.ctan.org/tex-archive/biblio/bibtex/contrib/doc/
% The IEEEtran BibTeX style support page is at:
% http://www.michaelshell.org/tex/ieeetran/bibtex/
%\bibliographystyle{IEEEtran}
% argument is your BibTeX string definitions and bibliography database(s)
%\bibliography{IEEEabrv,../bib/paper}
%
% <OR> manually copy in the resultant .bbl file
% set second argument of \begin to the number of references
% (used to reserve space for the reference number labels box)
\begin{thebibliography}{1}

\bibitem{full_duplex}
Sabharwal, Ashutosh, Philip Schniter, Dongning Guo, Daniel W. Bliss, Sampath Rangarajan, and Risto Wichman. "In-band full-duplex wireless: Challenges and opportunities." Selected Areas in Communications, IEEE Journal on 32, no. 9 (2014): 1637-1652.

\bibitem{compact}
Debaillie, Bjorn, Dirk-Jan van den Broek, Cristina Lavin, Barend van Liempd, Eric Klumperink, Carmen Palacios, Jan Craninckx, Bram Nauta, and Aarno Parssinen. "Analog/RF solutions enabling compact full-duplex radios." Selected Areas in Communications, IEEE Journal on 32, no. 9 (2014): 1662-1673.

\bibitem{fd_small}
Goyal, Shri, Pei Liu, Shivendra Panwar, Robert DiFazio, Rui Yang, Jialing Li, and Erdem Bala. "Improving small cell capacity with common-carrier full duplex radios." In Communications (ICC), 2014 IEEE International Conference on, pp. 4987-4993. IEEE, 2014.

\bibitem{ants}
Pradhan, Chandan and Garimella Rama Murthy. "Full-Duplex Transceiver for Future Cellular Network: A Smart Antenna Approach.": Avalaible: http://arxiv.org/abs/1509.03000

\bibitem{sc_fdma}
Eshwaraiah, Harsha S., and A. Chockalingam. "SC-FDMA for multiuser communication on the downlink." In Communication Systems and Networks (COMSNETS), 2013 Fifth International Conference on, pp. 1-7. IEEE, 2013.

   
\bibitem{mrc}
Goldsmith, Andrea. Wireless communications. Cambridge university press, 2005.

\bibitem{massive}
Zheng, Kan, Long Zhao, Jie Mei, Bin Shao, Wei Xiang, and Lajos Hanzo. "Survey of Large-Scale MIMO Systems."

%\bibitem{spectrum_radio}
%Cabric, Danijela, Ian D. O'Donnell, MS-W. Chen, and Robert W. Brodersen. "Spectrum sharing %radios." Circuits and Systems Magazine, IEEE 6, no. 2 (2006): 30-45.



%\bibitem{SVD_mimo}
%Liu, Wei, L-L. Yang, and Lajos Hanzo. "SVD Assisted Joint Transmitter and Receiver Design for the %Downlink of MIMO Systems." In Vehicular Technology Conference, 2008. VTC 2008-Fall. IEEE 68th, pp. 1-5. %IEEE, 2008

%\vspace{-2.5em}

\end{thebibliography}



%\bibliography{bibfile}

%\bibliography{bibfile}{}
%\bibliographystyle{plain}


% that's all folks

% An example of a floating figure using the graphicx package.
% Note that \label must occur AFTER (or within) \caption.
% For figures, \caption should occur after the \includegraphics.
% Note that IEEEtran v1.7 and later has special internal code that
% is designed to preserve the operation of \label within \caption
% even when the captionsoff option is in effect. However, because
% of issues like this, it may be the safest practice to put all your
% \label just after \caption rather than within \caption{}.
%
% Reminder: the "draftcls" or "draftclsnofoot", not "draft", class
% option should be used if it is desired that the figures are to be
% displayed while in draft mode.
%
%\begin{figure}[!t]
%\centering
%\includegraphics[width=2.5in]{myfigure}
% where an .eps filename suffix will be assumed under latex, 
% and a .pdf suffix will be assumed for pdflatex; or what has been declared
% via \DeclareGraphicsExtensions.
%\caption{Simulation results for the network.}
%\label{fig_sim}
%\end{figure}

% Note that the IEEE typically puts floats only at the top, even when this
% results in a large percentage of a column being occupied by floats.


% An example of a double column floating figure using two subfigures.
% (The subfig.sty package must be loaded for this to work.)
% The subfigure \label commands are set within each subfloat command,
% and the \label for the overall figure must come after \caption.
% \hfil is used as a separator to get equal spacing.
% Watch out that the combined width of all the subfigures on a 
% line do not exceed the text width or a line break will occur.
%
%\begin{figure*}[!t]
%\centering
%\subfloat[Case I]{\includegraphics[width=2.5in]{box}%
%\label{fig_first_case}}
%\hfil
%\subfloat[Case II]{\includegraphics[width=2.5in]{box}%
%\label{fig_second_case}}
%\caption{Simulation results for the network.}
%\label{fig_sim}
%\end{figure*}
%
% Note that often IEEE papers with subfigures do not employ subfigure
% captions (using the optional argument to \subfloat[]), but instead will
% reference/describe all of them (a), (b), etc., within the main caption.
% Be aware that for subfig.sty to generate the (a), (b), etc., subfigure
% labels, the optional argument to \subfloat must be present. If a
% subcaption is not desired, just leave its contents blank,
% e.g., \subfloat[].


% An example of a floating table. Note that, for IEEE style tables, the
% \caption command should come BEFORE the table and, given that table
% captions serve much like titles, are usually capitalized except for words
% such as a, an, and, as, at, but, by, for, in, nor, of, on, or, the, to
% and up, which are usually not capitalized unless they are the first or
% last word of the caption. Table text will default to \footnotesize as
% the IEEE normally uses this smaller font for tables.
% The \label must come after \caption as always.
%
%\begin{table}[!t]
%% increase table row spacing, adjust to taste
%\renewcommand{\arraystretch}{1.3}
% if using array.sty, it might be a good idea to tweak the value of
% \extrarowheight as needed to properly center the text within the cells
%\caption{An Example of a Table}
%\label{table_example}
%\centering
%% Some packages, such as MDW tools, offer better commands for making tables
%% than the plain LaTeX2e tabular which is used here.
%\begin{tabular}{|c||c|}
%\hline
%One & Two\\
%\hline
%Three & Four\\
%\hline
%\end{tabular}
%\end{table}





% if have a single appendix:
%\appendix[Proof of the Zonklar Equations]
% or
%\appendix  % for no appendix heading
% do not use \section anymore after \appendix, only \section*
% is possibly needed

% use appendices with more than one appendix
% then use \section to start each appendix
% you must declare a \section before using any
% \subsection or using \label (\appendices by itself
% starts a section numbered zero.)
%





% Can use something like this to put references on a page
% by themselves when using endfloat and the captionsoff option.
\ifCLASSOPTIONcaptionsoff
  \newpage
\fi



% trigger a \newpage just before the given reference
% number - used to balance the columns on the last page
% adjust value as needed - may need to be readjusted if
% the document is modified later
%\IEEEtriggeratref{8}
% The "triggered" command can be changed if desired:
%\IEEEtriggercmd{\enlargethispage{-5in}}

% references section

% can use a bibliography generated by BibTeX as a .bbl file
% BibTeX documentation can be easily obtained at:
% http://mirror.ctan.org/biblio/bibtex/contrib/doc/
% The IEEEtran BibTeX style support page is at:
% http://www.michaelshell.org/tex/ieeetran/bibtex/
%\bibliographystyle{IEEEtran}
% argument is your BibTeX string definitions and bibliography database(s)
%\bibliography{IEEEabrv,../bib/paper}
%
% <OR> manually copy in the resultant .bbl file
% set second argument of \begin to the number of references
% (used to reserve space for the reference number labels box)


% biography section
% 
% If you have an EPS/PDF photo (graphicx package needed) extra braces are
% needed around the contents of the optional argument to biography to prevent
% the LaTeX parser from getting confused when it sees the complicated
% \includegraphics command within an optional argument. (You could create
% your own custom macro containing the \includegraphics command to make things
% simpler here.)
%\begin{IEEEbiography}[{\includegraphics[width=1in,height=1.25in,clip,keepaspectratio]{mshell}}]{Michael Shell}
% or if you just want to reserve a space for a photo:

% You can push biographies down or up by placing
% a \vfill before or after them. The appropriate
% use of \vfill depends on what kind of text is
% on the last page and whether or not the columns
% are being equalized.

%\vfill

% Can be used to pull up biographies so that the bottom of the last one
% is flush with the other column.
%\enlargethispage{-5in}


% that's all folks
\end{document}

