\documentclass[lettersize,journal]{IEEEtran}
%\documentclass[journal]{IEEEtran}
%\IEEEoverridecommandlockouts
\usepackage{booktabs} % For formal tables
\usepackage{caption}
\usepackage{adjustbox}
%\usepackage{subfig}
\usepackage{subcaption}
\usepackage{float}
\usepackage{url}
\usepackage{epstopdf}
\usepackage{graphics,graphicx}
\usepackage{cite}
\usepackage{balance}
\usepackage{array}
\usepackage{pifont}
\usepackage{multirow}
%\usepackage{fixltx2e}
%\usepackage[algo2e]{algorithm2e}
\usepackage{algpseudocode}
\usepackage{algorithm, algpseudocode}
\usepackage{amssymb}
\usepackage{amsmath,amsfonts}
%\usepackage{footnote}
%\usepackage[ruled]{algorithm2e}
%\usepackage{amssymb}
%\usepackage{algorithmic}
\usepackage{algorithm, algpseudocode}
\usepackage{xcolor}
\usepackage{verbatim}
\usepackage{paralist, tabularx, enumitem}
\hyphenation{op-tical net-works semi-conduc-tor IEEE-Xplore}

\newcommand{\new}[1]{\textcolor{blue}{#1}}
%\newcommand{\new}[1]{\textcolor{black}{#1}}
\newcommand{\doubt}[1]{\textcolor{red}{DOUBT: #1}}
\newcommand{\todo}[1]{\textcolor{red}{TODO: #1}}

\algrenewcommand\algorithmicrequire{\textbf{Input:}}
\algrenewcommand\algorithmicensure{\textbf{Output:}}

% removes running headers
%\pagestyle{plain}

%%%%%%%%%%%%%%%%%%%%%%%%%%%
\begin{document}
	% paper title
	\title{FaiRIR: Mitigating Exposure Bias from Related Item Recommendations in Two-Sided Platforms}

	
	 
	\author{\IEEEauthorblockN{Abhisek Dash, Abhijnan Chakraborty, Saptarshi Ghosh, Animesh Mukherjee, and Krishna P. Gummadi}
	\thanks{A. Dash, S. Ghosh and A, Mukherjee are with Indian Institute of Technology Kharagpur, Kharagpur 721302, India 
	(e-mail: dash.abhi93@iitkgp.ac.in; saptarshi@cse.iitkgp.ac.in; animeshm@cse.iitkgp.ac.in).}
	\thanks{A. Chakraborty is with Indian Institute of Technology Delhi, Delhi 110016, India. (e-mail: abhijnan@iitd.ac.in)
	}
	\thanks{K. P. Gummadi is with the Max Planck Institute for Software Systems, 66123 Saarbrucken, Germany. (e-mail: gummadi@mpi-sws.org )}
	}
	
	\maketitle	
	\begin{abstract}
		Related Item Recommendations (RIRs) are ubiquitous in most online platforms today, including e-commerce and content streaming sites. These recommendations not only help users compare items related to a given item, but also play a major role in bringing traffic to individual items, thus deciding the exposure that different items receive. 
		With a growing number of people depending on such platforms to earn their livelihood, it is important to understand whether different items are receiving their desired exposure.  
		To this end, our experiments on multiple real-world RIR datasets reveal that the existing RIR algorithms often result in very skewed exposure distribution of items, and the quality of items is not a plausible explanation for such skew in exposure.
		To mitigate this exposure bias, we introduce multiple flexible interventions (\textit{FaiRIR}) in the RIR pipeline. We instantiate these mechanisms with two well-known algorithms for constructing related item recommendations -- rating-SVD and item2vec -- and show on real-world data that our mechanisms allow for a fine-grained control on the exposure distribution, often at a small or no cost in terms of recommendation quality, measured in terms of relatedness and user satisfaction.\footnote{\textcolor{red}{This work has been accepted as a regular paper in IEEE Transactions on Computational Social Systems 2022 (IEEE TCSS'22).}}
	\end{abstract}
	\begin{IEEEkeywords}
		Related Item Recommendation, Two sided platforms, Exposure Bias, FaiRIR, 
	\end{IEEEkeywords}
	% !TEX root = ../arxiv.tex

Unsupervised domain adaptation (UDA) is a variant of semi-supervised learning \cite{blum1998combining}, where the available unlabelled data comes from a different distribution than the annotated dataset \cite{Ben-DavidBCP06}.
A case in point is to exploit synthetic data, where annotation is more accessible compared to the costly labelling of real-world images \cite{RichterVRK16,RosSMVL16}.
Along with some success in addressing UDA for semantic segmentation \cite{TsaiHSS0C18,VuJBCP19,0001S20,ZouYKW18}, the developed methods are growing increasingly sophisticated and often combine style transfer networks, adversarial training or network ensembles \cite{KimB20a,LiYV19,TsaiSSC19,Yang_2020_ECCV}.
This increase in model complexity impedes reproducibility, potentially slowing further progress.

In this work, we propose a UDA framework reaching state-of-the-art segmentation accuracy (measured by the Intersection-over-Union, IoU) without incurring substantial training efforts.
Toward this goal, we adopt a simple semi-supervised approach, \emph{self-training} \cite{ChenWB11,lee2013pseudo,ZouYKW18}, used in recent works only in conjunction with adversarial training or network ensembles \cite{ChoiKK19,KimB20a,Mei_2020_ECCV,Wang_2020_ECCV,0001S20,Zheng_2020_IJCV,ZhengY20}.
By contrast, we use self-training \emph{standalone}.
Compared to previous self-training methods \cite{ChenLCCCZAS20,Li_2020_ECCV,subhani2020learning,ZouYKW18,ZouYLKW19}, our approach also sidesteps the inconvenience of multiple training rounds, as they often require expert intervention between consecutive rounds.
We train our model using co-evolving pseudo labels end-to-end without such need.

\begin{figure}[t]%
    \centering
    \def\svgwidth{\linewidth}
    \input{figures/preview/bars.pdf_tex}
    \caption{\textbf{Results preview.} Unlike much recent work that combines multiple training paradigms, such as adversarial training and style transfer, our approach retains the modest single-round training complexity of self-training, yet improves the state of the art for adapting semantic segmentation by a significant margin.}
    \label{fig:preview}
\end{figure}

Our method leverages the ubiquitous \emph{data augmentation} techniques from fully supervised learning \cite{deeplabv3plus2018,ZhaoSQWJ17}: photometric jitter, flipping and multi-scale cropping.
We enforce \emph{consistency} of the semantic maps produced by the model across these image perturbations.
The following assumption formalises the key premise:

\myparagraph{Assumption 1.}
Let $f: \mathcal{I} \rightarrow \mathcal{M}$ represent a pixelwise mapping from images $\mathcal{I}$ to semantic output $\mathcal{M}$.
Denote $\rho_{\bm{\epsilon}}: \mathcal{I} \rightarrow \mathcal{I}$ a photometric image transform and, similarly, $\tau_{\bm{\epsilon}'}: \mathcal{I} \rightarrow \mathcal{I}$ a spatial similarity transformation, where $\bm{\epsilon},\bm{\epsilon}'\sim p(\cdot)$ are control variables following some pre-defined density (\eg, $p \equiv \mathcal{N}(0, 1)$).
Then, for any image $I \in \mathcal{I}$, $f$ is \emph{invariant} under $\rho_{\bm{\epsilon}}$ and \emph{equivariant} under $\tau_{\bm{\epsilon}'}$, \ie~$f(\rho_{\bm{\epsilon}}(I)) = f(I)$ and $f(\tau_{\bm{\epsilon}'}(I)) = \tau_{\bm{\epsilon}'}(f(I))$.

\smallskip
\noindent Next, we introduce a training framework using a \emph{momentum network} -- a slowly advancing copy of the original model.
The momentum network provides stable, yet recent targets for model updates, as opposed to the fixed supervision in model distillation \cite{Chen0G18,Zheng_2020_IJCV,ZhengY20}.
We also re-visit the problem of long-tail recognition in the context of generating pseudo labels for self-supervision.
In particular, we maintain an \emph{exponentially moving class prior} used to discount the confidence thresholds for those classes with few samples and increase their relative contribution to the training loss.
Our framework is simple to train, adds moderate computational overhead compared to a fully supervised setup, yet sets a new state of the art on established benchmarks (\cf \cref{fig:preview}).

	\section{Related Work}\label{sec:related}
 
The authors in \cite{humphreys2007noncontact} showed that it is possible to extract the PPG signal from the video using a complementary metal-oxide semiconductor camera by illuminating a region of tissue using through external light-emitting diodes at dual-wavelength (760nm and 880nm).  Further, the authors of  \cite{verkruysse2008remote} demonstrated that the PPG signal can be estimated by just using ambient light as a source of illumination along with a simple digital camera.  Further in \cite{poh2011advancements}, the PPG waveform was estimated from the videos recorded using a low-cost webcam. The red, green, and blue channels of the images were decomposed into independent sources using independent component analysis. One of the independent sources was selected to estimate PPG and further calculate HR, and HRV. All these works showed the possibility of extracting PPG signals from the videos and proved the similarity of this signal with the one obtained using a contact device. Further, the authors in \cite{10.1109/CVPR.2013.440} showed that heart rate can be extracted from features from the head as well by capturing the subtle head movements that happen due to blood flow.

%
The authors of \cite{kumar2015distanceppg} proposed a methodology that overcomes a challenge in extracting PPG for people with darker skin tones. The challenge due to slight movement and low lighting conditions during recording a video was also addressed. They implemented the method where PPG signal is extracted from different regions of the face and signal from each region is combined using their weighted average making weights different for different people depending on their skin color. 
%

There are other attempts where authors of \cite{6523142,6909939, 7410772, 7412627} have introduced different methodologies to make algorithms for estimating pulse rate robust to illumination variation and motion of the subjects. The paper \cite{6523142} introduces a chrominance-based method to reduce the effect of motion in estimating pulse rate. The authors of \cite{6909939} used a technique in which face tracking and normalized least square adaptive filtering is used to counter the effects of variations due to illumination and subject movement. 
The paper \cite{7410772} resolves the issue of subject movement by choosing the rectangular ROI's on the face relative to the facial landmarks and facial landmarks are tracked in the video using pose-free facial landmark fitting tracker discussed in \cite{yu2016face} followed by the removal of noise due to illumination to extract noise-free PPG signal for estimating pulse rate. 

Recently, the use of machine learning in the prediction of health parameters have gained attention. The paper \cite{osman2015supervised} used a supervised learning methodology to predict the pulse rate from the videos taken from any off-the-shelf camera. Their model showed the possibility of using machine learning methods to estimate the pulse rate. However, our method outperforms their results when the root mean squared error of the predicted pulse rate is compared. The authors in \cite{hsu2017deep} proposed a deep learning methodology to predict the pulse rate from the facial videos. The researchers trained a convolutional neural network (CNN) on the images generated using Short-Time Fourier Transform (STFT) applied on the R, G, \& B channels from the facial region of interests.
The authors of \cite{osman2015supervised, hsu2017deep} only predicted pulse rate, and we extended our work in predicting variance in the pulse rate measurements as well.

All the related work discussed above utilizes filtering and digital signal processing to extract PPG signals from the video which is further used to estimate the PR and PRV.  %
The method proposed in \cite{kumar2015distanceppg} is person dependent since the weights will be different for people with different skin tone. In contrast, we propose a deep learning model to predict the PR which is independent of the person who is being trained. Thus, the model would work even if there is no prior training model built for that individual and hence, making our model robust. 

%
	\vspace{-1mm}
\section{RIR Systems and exposure thereof} \label{sec:fairness}

In this section, we present the notion of relatedness between items and how we instantiate an item model capturing it. We also demonstrate the operationalization of exposure induced by an RIR algorithm with respect to the discussed instantiation. %of the item model.

\vspace{-1mm}
\subsection{Relatedness of recommended items}
The primary goal of related item recommendations is to maximize the {\it relatedness of recommended items} to the source item
%The quality of recommendations depends heavily on what consumers perceive as `relatedness' of the recommended items to the source item 
that the consumer has viewed/purchased/liked. 
Though there is no sacrosanct definition of relatedness, two items can be thought of as related over multiple dimensions: \\
(i)~{\bf Content based relatedness}, e.g., movies of the same genre, items from the same producer or brand, etc., \\
(ii)~{\bf Compatibility}: two items can be related if they are either the substitute or complement of one another, e.g., items that are frequently purchased together: a smartphone and its cover, \\ %For example, users who have consumed an item have also consumed another item  or users consumed a pair of items together.
(iii)~{\bf External feedback on recommendation platforms}: user-actions such as likes and ratings, also define relatedness. For example, items being rated similarly, liked or disliked by a number of common users can be considered as related. 

\noindent
Relatedness, therefore, is subjective, and RIRs are judged 
%on how consumers perceive their quality to be; in other words, recommendations are relevant when people 
based on whether the consumers find the source and the recommended items to be related. Additionally, the metric to measure relatedness between items is often domain-dependent.
%Note that the notion of relatedness is a {\it local} property of an item, with respect to a given %seed or 
%source item.
{\it The concept of `relatedness' is analogous to `accuracy' or 'relevance' in the context of a related item recommendation system} -- just like classifiers are traditionally designed to optimize for accuracy, RIR systems are traditionally designed to optimize (maximize) relatedness.

\begin{figure}[tb] 
	\begin{tabular}{|c|c|}
		\hline
		Items & Related items \\
		\hline
		$I_1$  & $I_2$, $I_3$, $I_4$ \\
		\hline
		$I_2$  & $I_3$, $I_5$, $I_6$\\
		\hline
		$I_3$  & $I_4$, $I_5$, $I_6$ \\
		\hline
	\end{tabular}
	\adjustimage{width=0.75cm,valign=c}{figures/arrow.png}\quad
	\adjustimage{width=3.0cm,valign=c}{figures/Network3BigData.pdf}\quad
	\caption{\bf A sample item model and its corresponding RIN.}
	\label{Fig:RINCreation}
	\vspace{-7mm}
\end{figure}


\subsection{Instantiating Item Model by Related Item Network}

As shown in Figure~\ref{fig:recopipeline}, 
both Related Item Recommendations and Personalized Recommendation systems utilize an {\it Item Model} that captures the relatedness among items.
%Usually the relatedness among items is decided by using some Related Item Recommendation algorithm, some of which are discussed in Section~\ref{sec: RIRAlgos}. 
We now discuss an intuitive way to instantiate the Item Model, which was developed in our prior works~\cite{dash2021umpire, dash2019network}.

We utilize an instantiation of the Item Model of a recommendation system as a {\it Related Item Network} (RIN). 
A RIN is a directed network, with each node being analogous to an item in the universe,  
and a directed edge between two nodes implies that the corresponding source and destination items are related (based on some underlying notion of relatedness).
For instance, let us consider an item model as shown in the table in Figure~\ref{Fig:RINCreation}, and its corresponding RIN. 
Since item `$I_2$' is related to item `$I_1$', the corresponding nodes in the RIN are connected via a directed edge (from `$I_1$' to `$I_2$').%\todo{Is this picture already used in some earlier paper? Then we should redraw a different example. --AM}
%there exists a directed edge from `A' to `B'. 

Once this instantiation of item model is constructed, a simple way to generate the Related Item Recommendations is as follows. For a particular source item, one can recommend those items to which it links in the RIN. For instance, in Figure~\ref{Fig:RINCreation}, the recommendations for source item `$I_1$' are items `$I_2$', `$I_3$', and `$I_4$'.


\subsection{Estimating observed exposure}\label{sec: EstOexp}

We define the observed exposure $E_o(i)$ of an item $i$ as the exposure it actually gets after the deployment of a RIR algorithm.
Ideally, the observed exposure of items should be quantified by click-through rates or other user interaction signals.
However, the availability of such comprehensive user-item interactions is seldom possible for third-party researchers due to the sensitive nature of the information.
Counting the number of recommendations received by items (analogous to in-degree of items in RIN) may be a possible work-around in such situations. 
However, the importance of all recommendations is {\it not} the same -- %. The importance of a recommendation 
it varies with the source item, e.g., a recommendation from a popular source item is expected to yield more visibility (for the destination item) than a recommendation from a non-popular item.
 
Taking such observations into consideration, we use the `Random Surfer model'~\cite{random-surfer-model} to estimate the observed exposure. 
In general, users tend to visit the page of an item and then they start exploring different items recommended on the page. Alternatively, they can also randomly consume any other item thereafter. 
By simulating such user exploration for a large number of iterations, we take the {\it steady state visit frequency of a node} \textit{i} as its observed exposure $E_o(i)$ (more details can be found in our prior work~\cite{dash2021umpire}). Note that the notion of observed exposure of an item is very similar to PageRank of the corresponding node, in this formulation.


\vspace{1mm} \noindent
\textbf{Considerations during Random Surfer simulation: }
While simulating user browsing behavior, we note that different users can have different propensity to follow recommendations. 
The `teleportation probability' $\alpha \in [0, 1]$ of the Random Surfer model captures such considerations. 
The surfer chooses to traverse the recommended items with probability $(1 - \alpha)$, and teleport to a random item with probability $\alpha$. Throughout the paper, we report results for ($\alpha = 0.15$) which is the most prevalent value of teleportation in the literature~\cite{dash2021umpire, Brin98theanatomy}. Finally. we normalize the observed 
exposure scores of all items %in such a way 
such that $\sum_{\forall i \in \mathbf{I}}^{}{E_o(i)} = 1$.
	%auto-ignore
\begin{figure}[t!]
\centering
\includegraphics[width=1.0\linewidth]{figures/wireishard.pdf}
% \includegraphics[width=1.0\linewidth]{figures/wiresarehard2.pdf}
\caption{\textbf{Challenges of wire segmentation.} Wires have a diverse set of appearances. Challenges include but are not limited to (a) structural complexity, (b) visibility and thickness, (c) partial occlusion by other objects, (d) camera aberration artifacts, and variations in (e) object attachment, (f) color, (g) width and (h) shape.
% \zwei{this needs to be correspondent to the attributes you mentioned}
}
\vspace{-5.5mm}
\label{fig:motivation}
\end{figure}
	\section{Desired Exposure and Exposure bias}
%In this section, we %elaborate on
Next, we discuss how exposure can be % justly 
{\it fairly distributed} among a set of items, by motivating it through the lens of distributive justice~\cite{yaari1984dividing}. We %end this section by defining 
then define `Exposure bias', given the desired and observed distributions of exposure. 

\vspace{-2 mm}
\subsection{Desired exposure of items}
\label{sec:desired}
%There can be multiple notions of what is desired
Exposure in an online platform is a beneficial commodity, %it is convenient to think that individuals (items) 
hence the producers of items would prefer having more of it (than having less). In such a scenario, an intuitive notion of  fairness would be \textbf{equality of exposure}, i.e., the exposure should be uniformly distributed among all the items (by recommended them uniformly). 
%This strategy advocates for  (or attention) for all items such that they all are recommended to users uniformly. %Put differently, the desired objective is equality of opportunity.
However, the characteristics of the %individuals (
items %among whom the exposure is to be distributed are to be understood, 
should also be taken into account, since these characteristics may provide prima facie grounds for a departure from equality~\cite{yaari1984dividing}. 
For instance, all items are probably {\it not} of similar merit or intrinsic quality. This difference in `merit' or `quality' can be a justified reason for departure from equality. Thereby, the `desired exposure' of an individual (item) can be determined by its `deservingness' (merit)\footnote{Desiredness should not be confused with deservingness, i.e., desiredness $\ne$ deservingness in general. Deservingness, in contrast, is an extreme case of desiredness.}. This departure from equality is well established through the notion of \textbf{meritocratic fairness} and the related literature on meritocracy~\cite{joseph2018meritocratic,joseph2016fairness}. For instance, a high-quality item is considered more deserving of %popularity / 
user attention than a low-quality item. 

%Again, not all items have equal contribution to the society. In such cases, 
Alternatively, the desired exposures of various items can also be driven by a broader idea of societal welfare. %the welfare of the society. 
For instance, YouTube `Up next' related video recommendation has recently been criticised for leading users to far-right echo chamber and extremist content~\cite{youtubeRadical}, potentially influencing elections (e.g., the Brazilian presidential election~\cite{youtubeBrazil}). In response, YouTube tweaked its `Up next' algorithm, and started recommending Fox News videos from far-right conspiracy theory videos, instead of other videos from the same channels~\cite{youtubeFoxnews}. Clearly, YouTube deemed some videos {\it unworthy} of the exposure they were getting earlier and decided to nudge users to follow other videos. 
In some scenarios, %since there is a need for exposure of the items and their producers (providers) alike, it can be preferred 
it might be legally required to provide each item with some minimum amount of exposure, regardless of the item attributes~\cite{patro2020incremental,patro2020fairrec}. 


\vspace{1 mm}
\noindent
\textbf{Desiredness as a control knob for fairness: }Note that, we do {\it not} argue for any particular notion of desired exposure distribution; rather, the formulation and algorithms given in the subsequent sections are \textit{\textbf{agnostic}} to any measure of desiredness. 
Rather than advocating for any specific desired exposure, we perceive desiredness as a necessary {\it controllable knob} in our framework to ensure fairness in the final outcomes. 
Hence, if some legislation or a particular platform has a sacrosanct quantification of the desiredness of each item, the same can be easily plugged into our proposed fairness interventions.

\vspace{-2mm}
\subsection{Estimating desired exposure}
We denote the desired exposure of item $i$ as $E_d(i)$, and the desired exposure distribution over all items as $E_d$.
%For the purpose of 
In this work, as a proof of concept, we consider a generic formulation to accommodate multiple types of desired exposure distributions. We consider, a fraction $\beta \in [0, 1]$ of the total exposure is equally distributed among all items. This fraction of the exposure takes care of the minimum exposure of all items (and their producers). It is meant to provide all items with some minimum exposure to satisfy the basic needs of the items and their producers (as argued in~\cite{patro2020incremental}). The remaining ($1-\beta$) fraction of the total exposure is distributed proportional to the quality or merit of individual items, thus advocating \textit{meritocratic fairness}~\cite{joseph2018meritocratic,joseph2016fairness}. Notice, the above formulation of $E_d$ reduces to purely meritocratic distribution of exposure for $\beta = 0$, and to uniform distribution of exposure for $\beta = 1$. The exposure distributions are normalized so that the total exposure of all items in the item-set sum up to 1, i.e., $\sum_{i \in \mathbf{I}}^{}{E_d(i)} = 1$. 


As mentioned earlier, in this work, we assume the {\it average user-rating} of an item as the quantification of its merit / quality. The importance that we attach to an item's merit to obtain its desired exposure is controlled by the parameter $\beta$.


\vspace{1mm}
\noindent
{\bf A potential limitation of user-ratings:} One potential concern about using average user-ratings as a quality metric, might be that the number of ratings an item gets is partly driven by the %recommendations produced by 
existing recommendation algorithms. 
However, we believe that, although a user may have been led to an item via some recommendation, her rating would reflect the inherent quality of the item as perceived by her. 
Further, we also considered a slightly different quality measure -- average user-rating of an item, weighted by its number of ratings. The qualitative results of the analyses remained similar
in this setting too. %as for the simple user-rating measure. 
Hence, for simplicity and completeness we 
consider the average user-rating score to be the indicator of quality throughout this paper. 
%The extent of importance that is attached to this quality is regulated by the parameter $\beta$ to generate various desired exposure distributions.

\vspace{-1mm}
\subsection{Defining exposure bias}  \label{sub:exp-bias}
According to our formulation, a RIR system would be fair (unbiased), if it gives every item an observed exposure that is proportional to its desired exposure. 
Since $E_o(i)$ and $E_d(i)$ denotes the {\it observed and desired exposures} of item $i$, mathematically, a RIR system is fair if $\frac{E_o(i)}{E_d(i)} = \frac{E_o(j)}{E_d(j)} \hspace{2mm}  \forall {i, j} \in \mathbf{I}$.
%\begin{equation}
%\label{eq:merit-fairness-2}
%\frac{E_o(i)}{E_d(i)} = \frac{E_o(j)}{E_d(j)} \hspace{2mm}  %\forall {i, j} \in \mathbf{I}
%\end{equation}
%The above equations are agnostic of any desired exposure distribution.
As discussed in the previous section, a RIR system $R$ may lead to items getting different observed exposures than what is desired. 
{\it Exposure Bias} ($ExpBias$) is the %as a measure of 
deviation caused due to $R$ between the desired and observed exposure of items.
Following the set up in our prior work~\cite{dash2021umpire}, we measure $ExpBias$ by KL divergence~\cite{cha2007comprehensive} between the observed exposure distribution $E_o = \{E_o(i) \; \forall i \in \mathbf{I}\}$ and the desired exposure distribution $E_d = \{E_d(i) \; \forall i \in \mathbf{I}\}$: 
%\vspace{1mm}
\setlength{\belowdisplayskip}{0pt} 
\setlength{\abovedisplayskip}{0pt} 
%\scriptsize
\small
\begin{align}
%\nonumber
ExpBias(R) = D_{KL} (E_o||E_d)= \sum_{i }{E_o(i) \hspace{1mm} log \hspace{1mm} \Big(\frac{E_o(i)}{E_d(i)}\Big)}
\end{align}\normalsize


\vspace{1mm}
\noindent
\textbf{Categorization of items}:
Based on the observed and desired exposure of items, we categorize items in three different classes based on how closely the observed exposure replicates their desired exposure.\\
(a) {\bf Under-exposed:} item $i$ is \textit{under-exposed} if $1-\epsilon \leq \frac{E_o(i)}{E_d(i)}$, \\
(b) {\bf Over-exposed:} item $i$ is \textit{over-exposed} if $\frac{E_o(i)}{E_d(i)} \geq 1+\epsilon$, \\
(c) {\bf Adequately-exposed:} item $i$ is \textit{Adequately-exposed} if $1-\epsilon \leq \frac{E_o(i)}{E_d(i)} \leq 1+\epsilon$, \\
While this threshold ($\epsilon$) can be chosen based on prier context and established regulations, in this paper, we use $\epsilon=0.2$. Note that similar thresholds have been used in multiple prior works too~\cite{chakraborty2017who, dash2018beyond, dash2021umpire}. 

	\section{Evaluation}
\label{sec:eval}

This section evaluates the performance, area and power of the \ZF architecture demonstrating how it improves over the state-of-the-art  DaDianNao accelerator~\cite{DaDiannao}. 
Section~\ref{sec:eval:method} details the experimental methodology. 
Section~\ref{sec:eval:performance} evaluates the performance of \ZF. 
Sections~\ref{sec:eval:area} and \ref{sec:eval:power} evaluate the area and power of \ZF, and Section~\ref{sec:add-ineffectual} considers the removal of non-zero neurons.

%
%

%

\subsection{Methodology}
\label{sec:eval:method}


%
%
\begin{table}[t!]
\centering
\begin{tabular}{|l|l|l|}
\hline
\textbf{Network} & \pbox{5cm}{\textbf{Conv.} \\ \textbf{Layers}} & \textbf{Source} \\ \hline \hline
alex     	 & 5 & Caffe: bvlc\_reference\_caffenet \\ \hline
google 	 & 59 & Caffe: bvlc\_googlenet \\ \hline
nin 		 & 12 & Model Zoo: NIN-imagenet \\ \hline
vgg19  	 & 16 & Model Zoo: VGG 19-layer \\ \hline
cnnM  & 5 & Model Zoo: VGG\_CNN\_M\_2048 \\ \hline
cnnS  & 5 & Model Zoo: VGG\_CNN\_S \\ \hline
\end{tabular}
\caption{Networks used}
\label{table:networks}
\end{table}

The evaluation uses the set of popular~\cite{AlexNIPS2012}, and state-of-the-art convolutional neural networks~\cite{ILSVRC15}\cite{nin}\cite{vgg}\cite{vgg19} shown in Table \ref{table:networks}. 
These networks perform image classification on the ILSVRC12 dataset~\cite{ILSVRC15}, which contains $256\times256$ images across 1000 classes. 
The experiments use a randomly selected set of 1000 images, one from each class. The networks are available, pre-trained for Caffe, either as part of the distribution or at the Caffe Model Zoo~\cite{model-zoo}.

%
%
We created a cycle accurate simulator of the baseline accelerator and \ZF. 
The simulator integrates with the Caffe framework~\cite{caffe} to enable on-the-fly validation of the layer ouput neurons. 
%
%
%
%
The area and power characteristics of \ZF and \BASE are measured 
with synthesized implementations. The two designs are implemented 
in Verilog and synthesized via the Synopsis Design 
Compiler~\cite{synopsys_site} with the TSMC 65nm library. 
The NBin, NBout, and \ZF offset SRAM buffers were modeled using 
the Artisan single-ported register file memory compiler~\cite{artisan} 
using double-pumping to allow a read and write per cycle. The eDRAM 
area and energy was modeled with \textit{Destiny}~\cite{destiny}.
%



%
%
%
%
%

%
%

%
%
%
%

%
%
%
%
%
%
%
%
%
%
%
%
%

	\section{Evaluation}
\label{sec:evaluation}
\begin{table*}[!t]
\begin{center}
%\small
\caption {Benchmarks and applications for the study of the application-level resilience}
\vspace{-5pt}
\label{tab:benchmark}
\tiny
\begin{tabular}{|p{1.7cm}|p{7.5cm}|p{4cm}|p{2.5cm}|}
\hline
\textbf{Name} 	& \textbf{Benchmark description} 		& \textbf{Execution phase for evaluation}  			& \textbf{Target data objects}             \\ \hline \hline
CG (NPB)             & Conjugate Gradient, irregular memory access (input class S)   & The routine conj\_grad in the main computation loop  & The arrays $r$ and $colidx$     \\\hline
MG (NPB)    	       & Multi-Grid on a sequence of meshes (input class S)             & The routine mg3P in the main computation loop & The arrays $u$ and $r$ 	\\ \hline
FT (NPB)             & Discrete 3D fast Fourier Transform (input class S)            & The routine fftXYZ in the main computation loop  & The arrays $plane$ and $exp1$    \\ \hline
BT (NPB)             & Block Tri-diagonal solver (input class S)         		& The routine x\_solve in the main computation loop & The arrays $grid\_points$ and $u$	\\ \hline
SP (NPB)             & Scalar Penta-diagonal solver (input class S)         		& The routine x\_solve in the main computation loop & The arrays $rhoi$ and $grid\_points$  \\ \hline
LU (NPB)            & Lower-Upper Gauss-Seidel solver (input class S)        	& The routine ssor 	& The arrays $u$ and $rsd$ \\ \hline \hline
LULESH~\cite{IPDPS13:LULESH} & Unstructured Lagrangian explicit shock hydrodynamics (input 5x5x5) & 
The routine CalcMonotonicQRegionForElems 
& The arrays $m\_elemBC$ and $m\_delv\_zeta$ \\ \hline
AMG2013~\cite{anm02:amg} & An algebraic multigrid solver for linear systems arising from problems on unstructured grids (we use  GMRES(10) with AMG preconditioner). We use a compact version from LLNL with input matrix $aniso$. & The routine hypre\_GMRESSolve & The arrays $ipiv$ and $A$   \\ \hline
%$hierarchy.levels[0].R.V$ \\ \hline
\end{tabular}
\end{center}
\vspace{-5pt}
\end{table*}

%We evaluate the effectiveness of ARAT, and 
%We use ARAT to study the application-level resilience.
%The goal is to demonstrate 
%that aDVF can be a very useful metric to quantify the resilience of data objects
%at the application level. 
We study 12 data objects from six benchmarks of the NAS parallel benchmark (NPB) suite (we use SNU\_NPB-1.0.3) and 4 data objects from two scientific applications. 
%which is a c version of NPB 3.3, but ARAT can work for Fortran.
Those data objects are chosen to be representative: they have various data access patterns and participate in various execution phases.  
%For the benchmarks, we use CLASS S as the input problems and use the default compiler options of NPB.
For those benchmarks and applications, we use their default compiler options, and use gcc 4.7.3 and LLVM 3.4.2 for trace generation.
To count the algorithm-level fault masking, we use the default convergence thresholds (or the fault tolerance levels) for those benchmarks.
Table~\ref{tab:benchmark} gives 
%for->on by anzheng
detailed information on the benchmarks and applications.
The maximum fault propagation path for aDVF analysis is set to 10 by default.
%the value shadowing threshold is set as 0.01 (except for BT, we use $1 \times 10^{-6}$).
%These value shadowing thresholds are chosen such that any error corruption
%that results in the operand's value variance less than 1\% (for the threshold 0.01) or 0.0001\% (for the threshold $1 \times 10^{-6}$) during the 
%trace analysis does not impact the outcome correctness of six benchmarks.
%LU: check the newton-iteration residuals against the tolerance levels
%SP: check the newton-iteration residuals against the tolerance levels
%BT: check the newton-iteration residuals against the tolerance levels

\subsection{Resilience Modeling Results}
%We use ARAT to calculate aDVF values of 16 data objects. 
Figure~\ref{fig:aDVF_3tiers_profiling}
shows the aDVF results and breaks them down into the three levels 
(i.e., the operation-level, fault propagation level, and algorithm-level).
Figure~\ref{fig:aDVF_3classes_profiling} shows the 
%for->of by anzheng
results for the analyses at the levels of the operation and fault propagation,
and further breaks down the results into 
the three classes (i.e., the value overwriting, logical and comparison operations,
and value shadowing). %based on the reasons of the fault masking.
We have multiple interesting findings from the results.

\begin{figure*}
	\centering
        \includegraphics[width=0.8\textwidth]{three_tiers_gray.pdf}
% * <azguolu@gmail.com> 2017-03-23T03:20:28.808Z:
%
% ^.
        \vspace{-5pt}
        \caption{The breakdown of aDVF results based on the three level analysis. The $x$ axis is the data object name.}
        \vspace{-8pt}
        \label{fig:aDVF_3tiers_profiling}
\end{figure*}


\begin{figure*}
	\centering
	\includegraphics[width=0.8\textwidth]{three_types_gray.pdf}
	\vspace{-5pt}
	\caption{The breakdown of aDVF results based on the three classes of fault masking. The $x$ axis is the data object name. \textit{zeta} and \textit{elemBC} in LULESH are \textit{m\_delv\_zeta} and \textit{m\_elemBC} respectively.} % Anzheng
	\vspace{-5pt}
	\label{fig:aDVF_3classes_profiling}
    %\vspace{-5pt}
\end{figure*}

(1) Fault masking is common across benchmarks and applications.
Several data objects (e.g., $r$ in CG, and $exp1$ and $plane$ in FT)
have aDVF values close to 1 in Figure~\ref{fig:aDVF_3tiers_profiling}, 
which indicates that most of operations working on these data objects
have fault masking.
However, a couple of data objects have much less intensive fault masking.
For example, the aDVF value of $colidx$ in CG is 0.28 (Figure~\ref{fig:aDVF_3tiers_profiling}). 
Further study reveals that $colidx$ is an array to store column indexes of sparse matrices, and there is few operation-level or fault propagation-level fault masking  (Figure~\ref{fig:aDVF_3classes_profiling}).
The corruption of it can easily cause segmentation fault caught by the
algorithm-level analysis. 
$grid\_points$ in SP and BT also have a relatively small aDVF value (0.14 and 0.38 for SP and BT respectively in Figure~\ref{fig:aDVF_3tiers_profiling}).
Further study reveals that $grid\_points$ defines input problems for SP and BT. 
A small corruption of $grid\_points$ 
%change->changes by anzheng
can easily cause major changes in computation
caught by the fault propagation analysis. 

The data object $u$ in BT also has a relatively small aDVF value (0.82 in Figure~\ref{fig:aDVF_3tiers_profiling}).
Further study reveals that $u$ is read-only in our target code region
for matrix factorization and Jacobian, neither of which is friendly
for fault masking.
Furthermore, the major fault masking for $u$ comes from value shadowing,
and value shadowing only happens in a couple of the least significant bits 
of the operands that reference $u$, which further reduces the value of aDVF.
%also reduces fault masking.

(2) The data type is strongly correlated with fault masking.
Figure~\ref{fig:aDVF_3tiers_profiling} reveals that the integer data objects ($colidx$ in CG, $grid\_points$ in BT and SP, $m\_elemBC$ in LULESH) appear to be 
more sensitive to faults than the floating point data objects 
($u$ and $r$ in MG, $exp1$ and $plane$ in FT, $u$ and $rsd$ in LU, $m\_delv\_zeta$ in LULESH, and $rhoi$ in SP).
In HPC applications, the integer data objects are commonly employed to
define input problems and bound computation boundaries (e.g., $colidx$ in CG and $grid\_points$ in BT), 
or track computation status (e.g., $m\_elemBC$ in LULESH). Their corruption 
%these integer data objects
is very detrimental to the application correctness. 

(3) Operation-level fault masking is very common.
For many data objects, the operation-level fault masking contributes 
more than 70\% of the aDVF values. For $r$ in CG, $exp1$ in FT, and $rhoi$ in SP,
the contribution of the operation-level fault masking is close to 99\% (Figure~\ref{fig:aDVF_3tiers_profiling}).

Furthermore, the value shadowing is a very common operation level fault masking,
especially for floating point data objects (e.g., $u$ and $r$ in BT, $m\_delv\_zeta$ in LULESH, and $rhoi$ in SP in Figure~\ref{fig:aDVF_3classes_profiling}).
This finding has a very important indication for studying the application resilience.
In particular, the values of a data object can be different across different input problems. If the values of the data object are different, 
then the number of fault masking events due to the value shadowing will be different. 
Hence, we deduce that the application resilience
can be correlated with the input problems,
because of the correlation between the value shadowing and input problems. 
We must consider the input problems when studying the application resilience.
This conclusion is consistent with a very recent work~\cite{sc16:guo}.

(4) The contribution of the algorithm-level fault masking to the application resilience can be nontrivial.
For example, the algorithm-level fault masking contributes 19\% of the aDVF value for $u$ in MG and 27\% for $plane$ in FT (Figure~\ref{fig:aDVF_3tiers_profiling}).
The large contribution of algorithm-level fault masking in MG is consistent with
the results of existing work~\cite{mg_ics12}. 
For FT (particularly 3D FFT), the large contribution of algorithm-level fault masking in $plane$ (Figure~\ref{fig:aDVF_3tiers_profiling})
comes from frequent transpose and 1D FFT computations that average out 
or overwrite the data corruption.
CG, as an iterative solver, is known to have the algorithm-level fault masking
because of the iterative nature~\cite{2-shantharam2011characterizing}.
Interestingly, the algorithm-level fault masking in CG contributes most to the resilience of $colidx$ which is a vulnerable integer data object (Figure~\ref{fig:aDVF_3tiers_profiling}).

%Our study reveals the algorithm-level fault masking of CG from
%two perspectives. First, $a$ in CG, which is an array for intermediate results,
%has few algorithm-level fault masking (0.008\%);
%Second, $x$ in CG, which is a result vector, has 5.4\% of the aDVF value coming from the algorithm-level fault masking.
%This result indicates that the effects of the algorithm-level fault masking
%are not uniform across data objects. 

(5) Fault masking at the fault propagation level is small.
For all data objects, the contribution of the fault masking at the level of fault propagation is less than 5\% (Figure~\ref{fig:aDVF_3tiers_profiling}).
For 6 data objects ($r$ and $colidx$ in CG, $grid\_points$ and $u$ in BT, and 
$grid\_points$ and $rhoi$ in SP),  there is no fault masking at the level of fault propagation.
In combination with the finding 4, we conclude that once the fault
is propagated, it is difficult to mask it because of the contamination of
more data objects after fault propagation, and only the algorithm semantics can tolerate  propagated faults well. 
%This finding is consistent with our sensitivity analysis. 

(6) Fault masking by logical and comparison operations is small,
%For all data objects, the fault masking contributions due to logical and comparison operations are very small, 
comparing with the contributions of value shadowing and overwriting (Figure~\ref{fig:aDVF_3classes_profiling}). 
Among all data objects, 
the logical and comparison operations in $grid\_points$ in BT contribute the most (25\% contribution in Figure~\ref{fig:aDVF_fine_profiling}), 
because of intensive ICmp operations (integer comparison). %logical OR and SHL (left shifting).


(7) The resilience varies across data objects. %within the same application.
This fact is especially pronounced in two data objects $colidx$ and $r$ in CG (Figure~\ref{fig:aDVF_3tiers_profiling}).
 $colidx$ has aDVF much smaller than $r$, which means $colidx$ is much less resilient than $r$ (see finding 1 for a detailed analysis on $colidx$). 
Furthermore, $colidx$ and $r$ have different algorithm-level
fault masking (see finding 4 for a detailed analysis).

\begin{comment}
\textbf{Finding 7: The resilience of the same data objects varies across different applications.}
This fact is especially pronounced in BT and SP.
BT and SP address the same numerical problem but with different algorithms.
BT and SP have the same data objects, $qs$ and $rhoi$, but
$qs$ manifests different resilience in BT and SP.
This result is interesting, because it indicates that by using
different algorithms, we have opportunities to
improve the resilience of data objects.
\end{comment}

To further investigate the reasons for fault masking, 
we break down the aDVF results at the granularity of LLVM instructions,
based on the analyses at the levels of operation and fault propagation.
The results are shown in Figure~\ref{fig:aDVF_fine_profiling}.
%Because of the space limitation, 
%we only show one data object per benchmark, but each selected data object has the most diverse fault masking events within the corresponding benchmark.
%Based on Figure~\ref{fig:aDVF_fine_profiling}, we have another interesting finding.

(8) Arithmetic operations make a lot of contributions to fault masking.
%For $r$ in CG, $r$ in MG, $exp1$ in FT, $u$ in BT, $qs$ in SP, and $u$ in LU,
%the arithmetic operations, FMul (100\%), Add (16\%), FMul (85\%), 
%FMul (94\%), FMul (28\%), and FAdd (50\%)
For $r$ in CG, $u$ in BT, $plane$ and $exp1$ in FT, $m\_elemBC$ in LULESH, 
arithmetic operations (addition, multiplication, and division) contribute to almost 100\% of the fault masking (Figure~\ref{fig:aDVF_fine_profiling}).  
%(at the operation level and the fault propagation level).
%For $qs$ in SP and $u$ in LU, the store operation also makes
%important contributions as the arithmetic operations because of value overwriting.

\begin{figure*}
	\centering
	\includegraphics[width=0.77\textheight, height=0.23\textheight]{pie_chart.pdf}
	\vspace{-10pt}
	\caption{Breakdown of the aDVF results based on the analyses at the levels of operation and fault propagation}
    \vspace{-10pt}
	\label{fig:aDVF_fine_profiling}
\end{figure*}


\subsection{Sensitivity Study}
\label{sec:eval_sen}
%\textbf{change the fault propagation threshold and study the sensitivity of analysis to the threshold}
ARAT uses 10 as the default fault propagation analysis threshold. 
The fault propagation analysis will not go beyond 10 operations. Instead,
we will use deterministic fault injection after 10 operations. 
In this section, we study the impact of this threshold on the modeling accuracy. We use a range of threshold values and examine how the aDVF value varies and whether
the identification of fault masking varies. 
Figure~\ref{fig:sensitivity_error_propagation} shows the results for 
%add , after BT by anzheng
multiple data objects in CG, BT, and SP.
We perform the sensitivity study for all 16 data objects.
%in six benchmarks and two applications.
Due to the page space limitation, we only show the results for three data objects,
but we summarize the sensitivity study results for all data objects in this section.
%but other data objects in all benchmarks have the same trend.

Our results reveal that the identification of fault masking by tracking fault propagation is not significantly 
affected by the fault propagation analysis threshold. Even if we use a rather large threshold (50), 
the variation of aDVF values is 4.48\% on average among all data objects,
and the variation at each of the three levels of analysis (the operation level, fault propagation level,  and algorithm level) is less than 5.2\% on average. 
In fact, using a threshold value of 5 is sufficiently accurate in most of the cases (14 out of 16 data objects).
This result is consistent with our finding 5 (i.e., fault masking at the fault propagation level is small). %in most benchmarks).
However, we do find a data object ($m\_elementBC$ in LULESH) %and $exp1$ in FT) 
showing relatively high-sensitive (up to 15\% variation) to the threshold. For this uncommon data object, using 50 as the fault propagation path is sufficient. 

%In other words, even though using a larger threshold value can identify more error masking by tracking error 
%propagation, the implicit error masking induced by the error propagation is very limited.

\begin{figure}
		\begin{center}
		\includegraphics[width=0.48\textwidth,height=0.11\textheight]{sensi_study_gray.pdf}
		\vspace{-15pt}
		\caption{Sensitivity study for fault propagation threshold}
		\label{fig:sensitivity_error_propagation}
		\end{center}
\vspace{-15pt}
\end{figure}


\begin{comment}
\subsection{Comparison with the Traditional Random Fault Injection}
%\textbf{compare with the traditional fault injection to verify accuracy}
To show the effectiveness of our resilience modeling, we compare traditional random fault injection
and our analytical modeling. Figure~\ref{fig:comparison_fi} and Table~\ref{tab:comparison} show the results.
The figure shows the success rate of all random fault injection. The ``success'' means the application
outcome is verified successfully by the benchmarks and the execution does not have any segfault. The success rate is used as a metric
to evaluate the application resilience.

We use a data-oriented approach to perform random fault injection.
In particular, given a data object, for each fault injection test we trigger a bit flip
in an operand of a random instruction, and this operand must be a reference to the
target data object. We develop a tool based on PIN~\cite{pintool} to implement the above fault injection functionality.
For each data object, we conduct five sets of random fault injection tests, 
and each set has 200 tests (in total 1000 tests per data object). 
We show the results for CG and FT in this section, but we find that
the conclusions we draw from CG and FT are also valid for the other four benchmarks.


%\begin{table*}
%\label{tab:success_rate}
%\begin{centering}
%\renewcommand\arraystretch{1.1}
%\begin{tabular}{|c|c|c|c|c|c|c|}
%\hline 
%Success Rate (Difference) & Test set 1 & Test set 2 & Test set 3 & Test set 4 & Test set 5 & Average\tabularnewline
%\hline 
%\hline 
%CG-a & 66.1\% (11.7\%) & 68.5\% (15.7\%) & 56.7\% (4.21\%) & 61.3\% (3.57\%) & 43.3\% (26.8\%) & 59.2\%\tabularnewline
%\hline 
%CG-x & 99.2\% (2.2\%) & 98.6\% (1.5\%) & 96.5\% (0.63\%) & 97.8\% (0.64\%) & 93.6\% (3.7\%) & 97.1\%\tabularnewline
%\hline 
%CG-colidx & 36.8\% (12.7\%) & 49.6\% (17.8\%) & 40.2\% (4.6\%) & 52.6\% (24.9\%) & 31.4\% (25.4\%) & 42.1\%\tabularnewline
%\hline 
%FT-exp1 & 52.7\% (1.4\%) & 22.6\% (56.5\%) & 78.5\% (51.0\%) & 60.7\% (16.7\%) & 45.4\% (12.7\%) & 51.9\%\tabularnewline
%\hline 
%FT-plane & 82.1\% (2.5\%) & 79.3\% (5.6\%) & 99.5\% (18.2\%) & 93.2\% (10.7\%) & 66.8\% (20.6\%) & 84.2\%\tabularnewline
%\hline 
%\end{tabular}
%\par\end{centering}
%\caption{XXXXX}
%\end{table*}


\begin{table*}
\begin{centering}
\caption{\small The results for random fault injection. The numbers in parentheses for each set of tests (200 tests per set) are the success rate difference from the average success rate of 1000 fault injection tests.}
\label{tab:comparison}
\renewcommand\arraystretch{1.1}
\begin{tabular}{|c|p{2.2cm}|p{2.2cm}|p{2.2cm}|p{2.2cm}|p{2.2cm}|p{1.8cm}|}
\hline 
       %& Test set 1 & Test set 2 & Test set 3 & Test set 4 & Test set 5 & Average\tabularnewline
       & \hspace{13pt} Test set 1 \hspace{1pt}/  & \hspace{13pt} Test set 2 \hspace{1pt}/ & \hspace{13pt} Test set 3 \hspace{1pt}/ & \hspace{13pt} Test set 4 \hspace{1pt}/ & \hspace{13pt} Test set 5 \hspace{1pt}/ & Ave. of all test / \\
       & success rate (diff.) & success rate (diff.) & success rate (diff.) & success rate (diff.) & success rate (diff.) & \hspace{5pt} success rate \\
\hline 
\hline 
CG-a & 66.1\% (6.9\%) & 68.5\% (9.3\%) & 56.7\% (-2.5\%) & 61.3\% (2.1\%) & 43.3\% (-15.9\%) & 59.2\%\tabularnewline
\hline 
CG-x & 99.2\% (2.1\%) & 98.6\% (1.5\%) & 96.5\% (-0.6\%) & 97.8\% (0.7\%) & 93.6\% (-3.5\%) & 97.1\%\tabularnewline
\hline 
CG-colidx & 36.8\% (-5.3\%) & 49.6\% (7.5\%) & 40.2\% (-2.0\%) & 52.6\% (10.5\%) & 31.4\% (-10.7\%) & 42.1\%\tabularnewline
\hline 
FT-exp1 & 52.7\% (0.8\%) & 22.6\% (-29.3\%) & 78.5\% (26.6\%) & 60.7\% (8.8\%) & 45.4\% (-6.5\%) & 51.9\%\tabularnewline
\hline 
FT-plane & 82.1\% (-2.1\%) & 79.3\% (-4.9\%) & 99.5\% (15.3\%) & 93.2\% (9.0\%) & 66.8\% (-17.4\%) & 84.2\%\tabularnewline
\hline 
\end{tabular}
\par\end{centering}
\vspace{-0.4cm}
\end{table*}

\begin{figure}
	\begin{center}
		\includegraphics[width=0.48\textwidth,keepaspectratio]{verifi-study.png}
		\caption{The traditional random fault injection vs. ARAT}
		\label{fig:comparison_fi}
	\end{center}
\vspace{-0.7cm}
\end{figure}


We first notice from Table~\ref{tab:comparison} that 
%across 5 sets of random fault injection tests, there are big variances (up to 55.9\% in $exp1$ of FT) in terms of the success rate. 
the results of 5 test sets can be quite different from each other and from 1000 random fault inject tests (up to 29.3\%).
1000 fault injection tests provide better statistical significance than 200 fault injection tests.
We expect 1000 fault injection tests potentially provide higher accuracy to quantify the application resilience.
The above result difference is clearly an indication to the randomness of fault injection, and there
is no guarantee on the random fault injection accuracy.

%In Figure~\ref{fig:comparison_fi}, 
We compare the success rate of 1000 fault inject tests with the aDVF value (Fig.~\ref{fig:comparison_fi}). 
We find that the order of the success rate of the three data objects in CG (i.e., $colidx < a < x$) and the two data objects in FT 
(i.e., $exp1 < plane$) is the same as the order of the aDVF values of these data objects. 
%In fact, 1000 fault injection tests
%account for \textcolor{blue}{\textbf{xxx\%}} of total memory references to the data object,
%and provide better resilience quantification than 200 fault injection tests.
The same order (or the same resilience trend)
%between our approach and the random fault injection based on a large number of tests 
is a demonstration of the effectiveness of our approach.
Note that the values of the aDVF and success rate %for a data object
cannot be exactly the same (even if we have sufficiently large numbers of random fault injection), 
because aDVF and random fault injection quantify
the resilience based on different metrics.
Also, the random fault injection can miss some fault masking events that can be captured by our approach.

\end{comment}
	\section{Conclusion}
This paper introduces SysNoise, a harmful noise that frequently happens when the source training system switches to a disparate target system in deployments. We first identify and classify SysNoise based on the inference stage, and thereafter build a holistic benchmark and framework to quantitatively measure the impact of SysNoise on image classification, object detection, segmentation, and natural language processing tasks. Our large-scale experiments revealed that SysNoise is highly-influential and will cause model performance degeneration; additionally, common mitigations like data augmentation and adversarial training show limited effects on SysNoise.  

In the future, we will evaluate SysNoise on the real-world systems, and will continuously develop the benchmark to include more tasks. Our findings open a new research topic and we hope it will raise research attention to the performance and robustness of deep learning deployment systems. 

%%%%%%%%%%%%%%%%%%%%%%%%%%%%%%%%%%%%%%%%%%%%%%%%%%%%%%%%%%%%
	 
	\section*{Acknowledgment}
	The authors thank the anonymous reviewers and the Associate Editor whose comments helped to improve the paper. This research is supported in part by (1)~a grant from the Max Planck Society through a Max Planck Partner Group at IIT Kharagpur, and (2)~a European Research Council (ERC) Advanced Grant for the project ``Foundations for Fair Social Computing", funded under the European Union's Horizon 2020 Framework Programme (grant agreement no. 789373). Additionally, A. Dash is supported by a fellowship from Tata Consultancy Services.
	
	\balance
	\bibliographystyle{IEEEtran}
	\bibliography{Main}

\end{document}