\documentclass[lettersize,journal]{IEEEtran}
%\documentclass[journal]{IEEEtran}
%\IEEEoverridecommandlockouts
\usepackage{booktabs} % For formal tables
\usepackage{caption}
\usepackage{adjustbox}
%\usepackage{subfig}
\usepackage{subcaption}
\usepackage{float}
\usepackage{url}
\usepackage{epstopdf}
\usepackage{graphics,graphicx}
\usepackage{cite}
\usepackage{balance}
\usepackage{array}
\usepackage{pifont}
\usepackage{multirow}
%\usepackage{fixltx2e}
%\usepackage[algo2e]{algorithm2e}
\usepackage{algpseudocode}
\usepackage{algorithm, algpseudocode}
\usepackage{amssymb}
\usepackage{amsmath,amsfonts}
%\usepackage{footnote}
%\usepackage[ruled]{algorithm2e}
%\usepackage{amssymb}
%\usepackage{algorithmic}
\usepackage{algorithm, algpseudocode}
\usepackage{xcolor}
\usepackage{verbatim}
\usepackage{paralist, tabularx, enumitem}
\hyphenation{op-tical net-works semi-conduc-tor IEEE-Xplore}

\newcommand{\new}[1]{\textcolor{blue}{#1}}
%\newcommand{\new}[1]{\textcolor{black}{#1}}
\newcommand{\doubt}[1]{\textcolor{red}{DOUBT: #1}}
\newcommand{\todo}[1]{\textcolor{red}{TODO: #1}}

\algrenewcommand\algorithmicrequire{\textbf{Input:}}
\algrenewcommand\algorithmicensure{\textbf{Output:}}

% removes running headers
%\pagestyle{plain}

%%%%%%%%%%%%%%%%%%%%%%%%%%%
\begin{document}
	% paper title
	\title{FaiRIR: Mitigating Exposure Bias from Related Item Recommendations in Two-Sided Platforms}

	
	 
	\author{\IEEEauthorblockN{Abhisek Dash, Abhijnan Chakraborty, Saptarshi Ghosh, Animesh Mukherjee, and Krishna P. Gummadi}
	\thanks{A. Dash, S. Ghosh and A, Mukherjee are with Indian Institute of Technology Kharagpur, Kharagpur 721302, India 
	(e-mail: dash.abhi93@iitkgp.ac.in; saptarshi@cse.iitkgp.ac.in; animeshm@cse.iitkgp.ac.in).}
	\thanks{A. Chakraborty is with Indian Institute of Technology Delhi, Delhi 110016, India. (e-mail: abhijnan@iitd.ac.in)
	}
	\thanks{K. P. Gummadi is with the Max Planck Institute for Software Systems, 66123 Saarbrucken, Germany. (e-mail: gummadi@mpi-sws.org )}
	}
	
	\maketitle	
	\begin{abstract}
		Related Item Recommendations (RIRs) are ubiquitous in most online platforms today, including e-commerce and content streaming sites. These recommendations not only help users compare items related to a given item, but also play a major role in bringing traffic to individual items, thus deciding the exposure that different items receive. 
		With a growing number of people depending on such platforms to earn their livelihood, it is important to understand whether different items are receiving their desired exposure.  
		To this end, our experiments on multiple real-world RIR datasets reveal that the existing RIR algorithms often result in very skewed exposure distribution of items, and the quality of items is not a plausible explanation for such skew in exposure.
		To mitigate this exposure bias, we introduce multiple flexible interventions (\textit{FaiRIR}) in the RIR pipeline. We instantiate these mechanisms with two well-known algorithms for constructing related item recommendations -- rating-SVD and item2vec -- and show on real-world data that our mechanisms allow for a fine-grained control on the exposure distribution, often at a small or no cost in terms of recommendation quality, measured in terms of relatedness and user satisfaction.\footnote{\textcolor{red}{This work has been accepted as a regular paper in IEEE Transactions on Computational Social Systems 2022 (IEEE TCSS'22).}}
	\end{abstract}
	\begin{IEEEkeywords}
		Related Item Recommendation, Two sided platforms, Exposure Bias, FaiRIR, 
	\end{IEEEkeywords}
	Reinforcement learning has achieved great success in areas such as Game-playing \citep{silver2018general,vinyals2019grandmaster}, robotics \cite{kober2013reinforcement}, large language models \citep{ouyang2022training}, etc.
However, due to safety concerns or physical limitations, in some real-world reinforcement learning problems, we must consider additional constraints that may influence the optimal policy and the learning process \citep{garcia2015comprehensive}.
% For example, a robotic arm must not take actions that may cause harm to itself or the environments.
A standard framework to handle such cases is the constrained Markov Decision Process (CMDP) \citep{altman1999constrained}.
Within the CMDP framework, the agent has to maximize
the expected cumulative reward while
obeying a finite number of constraints, which are usually in the form of expected cumulative cost criteria.

However, we are sometimes concerned with the problem with a continuum of constraints.
For example,
the constraints we meet might be time-evolving or subject to uncertain parameters, which
cannot be formulated as an ordinary CMDP
(see Examples \ref{Example_Time_Evolving} and  \ref{Example_Uncertain}).
In this paper we would study a generalized CMDP  
to address the above problem.  Because the constraints are not only infinite-number but also lie
in a continuous set,
the generalization is not trivial. Fortunately, we find that we can borrow the idea behind semi-infinite programming (SIP) \citep{remez1934determination, hettich1993semi} to deal with the semi-infinite constraints.
Accordingly, we propose \emph{semi-infinitely constrained Markov decision processes} (SICMDPs)
as a novel complement to the ordinary CMDP framework.
%More specifically,  an SICMDP model %, we consider 
%contains a continuum of constraints whereas an ordinary CMDP contains a finite number of constraints. 

%This generalization is natural but not trivial. However, we can brows the idea  
%The idea is quite natural and can be backtracked
%to the practice of extending linear programming to linear semi-infinite programming (LSIP) %\cite{remez1934determination, GobernaLSIO1998}.
%In addition, 
%As a complementary approach to the ordinary CMDP framework, 
%SICMDP can be used to model these problems  which cannot be described by a finite number of constraints
%that are not covered by .
%For example,
%the restrictions we consider can be time-evolving or subject to uncertain parameters
%, thus
%cannot be described by a finite number of constraints but a continuum of constraints 
%(see Examples \ref{Example_Time_Evolving} and  \ref{Example_Uncertain}).

We also present two reinforcement learning algorithms to solve SICMDPs called SI-CRL and SI-CPO, respectively.
SI-CRL is a model-based reinforcement learning algorithm designed for tabular cases, and SI-CPO is a policy optimization algorithm for non-tabular cases.
% and analyze its performance both theoretically and empirically.
The main challenge is that we need to deal with a continuum of constraints, thus reinforcement learning algorithms for ordinary CMDPs do not work anymore.
In SI-CRL, we tackle this difficulty by first transforming the reinforcement learning problem to an equivalent LSIP problem, which can then be solved using methods in the LSIP literature like the dual exchange methods \citep{Hu1990,reemtsen1998numerical}.
In SI-CPO, we resort to the idea of cooperative stochastic approximation developed in \cite{lan2020algorithms, wei2020comirror}.
As far as we know, we are the first to introduce tools from semi-infinitely programming (SIP) into the reinforcement learning community for solving constrained reinforcement learning problems.

% To the best of our knowledge, we are the first to apply tools from semi-infinitely programming (SIP) to solve reinforcement learning problems.
Furthermore, we give theoretical analysis for both SI-CRL and SI-CPO.
We decompose the error of SI-CRL into two parts: the statistical error from approximating the true SICMDP with an offline dataset and the optimization error due to the fact that the solution of the LSIP problem obtained by the dual exchange method is inexact.
On the optimization side, we show that the iteration complexity of SI-CRL is $O\left(\left\{\mathrm{diam}(Y)L\sqrt{|\gS|^2|\gA|m}/\left[(1-\gamma)\epsilon\right]\right\}^m\right)$.
On the statistical side, we show that the sample complexity of SI-CRL is $\widetilde O\left(\frac{|S|^2|A|^2}{\epsilon^2(1-\gamma)^3}\right)$ if the offline dataset is generated by a generative model, and $\widetilde O\left(\frac{|S||A|}{\nu_{\min} \epsilon^2(1-\gamma)^3}\right)$ if the dataset is generated by a probability measure $\nu$ as considered in \cite{chen2019information}.
Here $\widetilde O$ means that all logarithm terms are discarded.
For SI-CPO, things become a little more complicated because other than the statistical error and the optimization error, we also need to consider the function approximation error, which comes from imperfect policy parametrizations.
It is shown if the function approximation error can be controlled to $O(\epsilon)$ order, the iteration complexity of SI-CPO is $\widetilde{O}\left(\frac{1}{\epsilon^2(1-\gamma)^6}\right)$ and the sample complexity of SI-CPO is $\widetilde{O}(\frac{1}{\epsilon^4(1-\gamma)^{10}})$.
Here our iteration complexity bound is equivalent to a typical $\widetilde O(1/\sqrt{T})$ global convergence rate.

We perform a set of numerical experiments to illustrate the SICMDP model and validate our proposed algorithms.
Specifically, we examine two numerical examples, namely the discharge of sewage and ship route planning.
Through the discharge of sewage example, we show the advantage of the SICMDP framework over the CMDP baseline obtained by naive discretization in modeling realistic sequential decision-making problems.
Moreover, we demonstrate the effectiveness of the SI-CRL and SI-CPO algorithms in such tabular environments. 
In the ship route planning example, we illustrate the benefits of the SICMDP framework and the ability of the SI-CPO algorithm to address complex continuous control tasks involving continuous state spaces with modern deep reinforcement learning techniques.

% In summary, our contributions are listed as follows.
% First, we present the SICMDP model, which can be viewed as a generalization of the ordinary CMDP model.
% Second, we propose an algorithm to perform reinforcement learning for SICMDPs, which is called SI-CRL, and we believe that we are the first to apply tools from SIP
% to solve reinforcement learning problems.
% Third, we give a theoretical analysis of SI-CRL and identify both its sample complexity and iteration complexity.
% In addition, we perform numerical experiments to illustrate the SICMDP model and validate the SI-CRL algorithm.
% \{This paragraph can be removed!!! \}





	\textbf{Related work}:
% Object detection related datasets/algo in non-medical domain
% Locally labeled CXR dataset
A few CXR datasets have localized abnormality annotations \cite{shih2019augmenting,filice2020crowdsourcing,jaeger2014two} that are curated manually. These are high quality gold standard ground truth datasets but tend to be smaller in scale (< 30,000 images) and have a narrow coverage, with typically only 1-2 labels. In addition, since most labeling efforts only have abnormality semantics attached, no direct relationships with the affected anatomical locations are available. 

%MEHDI: repeated concepts from above. I am removing the following: 

%The lack of anatomic semantics in the annotation is a limitation for complex multi-modal clinical reasoning work, e.g., differential diagnosis, since clinicians often integrate information along anatomical lines, and for downstream report generation tasks, which often requires describing not only the abnormality but also correctly communicate the location of the abnormalities (and medical devices) to the receiving clinicians. 

Two recent CXR datasets have labels for anatomies described in the reports. In \cite{datta2020dataset}, a small manually annotated dataset (2000 reports) included 10 abnormalities that are individually associated with 29 unique spatial locations (anatomies) at the report level. Another CXR dataset has automatically extracted abnormality and anatomy labels as disconnected concepts that are only correlated at the study level from  160,000 reports using a supervised NLP algorithm \cite{bustos2020padchest}. This was trained on a smaller set of manually annotated data. Neither datasets contain localized annotations for the associated CXR images, nor any comparison relation annotations between sequential exams, both of which are available in the Chest ImaGenome dataset. In Table \ref{tab:related}, we present a comparison of our Chest ImagGenome dataset with other datasets available in the literature.

% Table -- Kashyap

% MEdical imaging datasets to go here: Discussed that we will only focus on cxr datasets that are available for this paper. 
% \caption{\color{red} Kashyap, feel free to continue with the table. We should remove the questionmarks and add a line for our dataset (since all others are not graph). For longer text, using abbreviations and explaining them in the caption often works better. If fill in the values is not possible, it is better to remove the table altogether.}


\begin{table}[t!]
\caption{Summary of existing chest X-ray datasets}
\resizebox{\textwidth}{!}{%
\begin{tabular}{@{}lllllllll@{}}
\toprule
\textbf{Dataset} & \textbf{Annotation Level} & \textbf{Annotation Method} & \textbf{Num Labels} & \textbf{Anatomy Labeled} & \textbf{Graph} & \textbf{Dataset Size} & \textbf{Temporal Labels} & \textbf{Reports} \\ \midrule
SIIM-ACR Pneumothorax Segmentation \cite{filice2020crowdsourcing} & Segmentation & Manual + augmented & 1 & No & No & 12,047 & No & No \\
RSNA Pneumonia Detection Challenge   \cite{shih2019augmenting} & Bounding Boxes & Manual & 1 & No & No & 30,000 & No & No \\
Indiana University Chest X-ray collection \cite{demner2016preparing} & Global & Automated & 10 & No & No & 3,813 & No & Yes \\
NIH CXR dataset \cite{wang2017chestx} & Global & Automated & 14 & No & No & 112,120 & No & No \\
PLCO \cite{team2000prostate} & Global & Automated & 24 & Yes & No & 236,000 & Yes & No \\
Stanford CheXpert \cite{irvin2019chexpert} & Global & Automated & 14 & No & No & 224,316 & No & No \\
MIMIC-CXR \cite{johnson2019mimic} & Global & Automated & 14 & No & No & 377,110 & No & Yes \\
Dutta \cite{datta2020dataset} & Global & Manual & 10 & Yes & Yes & 2,000 & No & Yes \\
PadChest \cite{bustos2020padchest} & Global & Manual + automated & 297 & Yes & No & 160,868 & No & Yes \\
Montgomery County Chest X-ray   \cite{jaeger2014two} & Segmentation & Manual & 1 & Yes & No & 138 & No & No \\
Shenzen Hospital Chest X-ray   \cite{jaeger2014two} & Segmentation & Manual & 1 & Yes & No & 662 & No & No \\  \hline \hline
\textbf{Chest ImaGenome} & Bounding Boxes & Automated & 131 & Yes & Yes & 242,072 & Yes & Yes \\
\bottomrule
\end{tabular}%
}
\label{tab:related}
\vspace{-0.4cm}
\end{table}
% removed (Derived from MIMIC-CXR \cite{johnson2019mimic}) % makes table really small

	\vspace{-1mm}
\section{RIR Systems and exposure thereof} \label{sec:fairness}

In this section, we present the notion of relatedness between items and how we instantiate an item model capturing it. We also demonstrate the operationalization of exposure induced by an RIR algorithm with respect to the discussed instantiation. %of the item model.

\vspace{-1mm}
\subsection{Relatedness of recommended items}
The primary goal of related item recommendations is to maximize the {\it relatedness of recommended items} to the source item
%The quality of recommendations depends heavily on what consumers perceive as `relatedness' of the recommended items to the source item 
that the consumer has viewed/purchased/liked. 
Though there is no sacrosanct definition of relatedness, two items can be thought of as related over multiple dimensions: \\
(i)~{\bf Content based relatedness}, e.g., movies of the same genre, items from the same producer or brand, etc., \\
(ii)~{\bf Compatibility}: two items can be related if they are either the substitute or complement of one another, e.g., items that are frequently purchased together: a smartphone and its cover, \\ %For example, users who have consumed an item have also consumed another item  or users consumed a pair of items together.
(iii)~{\bf External feedback on recommendation platforms}: user-actions such as likes and ratings, also define relatedness. For example, items being rated similarly, liked or disliked by a number of common users can be considered as related. 

\noindent
Relatedness, therefore, is subjective, and RIRs are judged 
%on how consumers perceive their quality to be; in other words, recommendations are relevant when people 
based on whether the consumers find the source and the recommended items to be related. Additionally, the metric to measure relatedness between items is often domain-dependent.
%Note that the notion of relatedness is a {\it local} property of an item, with respect to a given %seed or 
%source item.
{\it The concept of `relatedness' is analogous to `accuracy' or 'relevance' in the context of a related item recommendation system} -- just like classifiers are traditionally designed to optimize for accuracy, RIR systems are traditionally designed to optimize (maximize) relatedness.

\begin{figure}[tb] 
	\begin{tabular}{|c|c|}
		\hline
		Items & Related items \\
		\hline
		$I_1$  & $I_2$, $I_3$, $I_4$ \\
		\hline
		$I_2$  & $I_3$, $I_5$, $I_6$\\
		\hline
		$I_3$  & $I_4$, $I_5$, $I_6$ \\
		\hline
	\end{tabular}
	\adjustimage{width=0.75cm,valign=c}{figures/arrow.png}\quad
	\adjustimage{width=3.0cm,valign=c}{figures/Network3BigData.pdf}\quad
	\caption{\bf A sample item model and its corresponding RIN.}
	\label{Fig:RINCreation}
	\vspace{-7mm}
\end{figure}


\subsection{Instantiating Item Model by Related Item Network}

As shown in Figure~\ref{fig:recopipeline}, 
both Related Item Recommendations and Personalized Recommendation systems utilize an {\it Item Model} that captures the relatedness among items.
%Usually the relatedness among items is decided by using some Related Item Recommendation algorithm, some of which are discussed in Section~\ref{sec: RIRAlgos}. 
We now discuss an intuitive way to instantiate the Item Model, which was developed in our prior works~\cite{dash2021umpire, dash2019network}.

We utilize an instantiation of the Item Model of a recommendation system as a {\it Related Item Network} (RIN). 
A RIN is a directed network, with each node being analogous to an item in the universe,  
and a directed edge between two nodes implies that the corresponding source and destination items are related (based on some underlying notion of relatedness).
For instance, let us consider an item model as shown in the table in Figure~\ref{Fig:RINCreation}, and its corresponding RIN. 
Since item `$I_2$' is related to item `$I_1$', the corresponding nodes in the RIN are connected via a directed edge (from `$I_1$' to `$I_2$').%\todo{Is this picture already used in some earlier paper? Then we should redraw a different example. --AM}
%there exists a directed edge from `A' to `B'. 

Once this instantiation of item model is constructed, a simple way to generate the Related Item Recommendations is as follows. For a particular source item, one can recommend those items to which it links in the RIN. For instance, in Figure~\ref{Fig:RINCreation}, the recommendations for source item `$I_1$' are items `$I_2$', `$I_3$', and `$I_4$'.


\subsection{Estimating observed exposure}\label{sec: EstOexp}

We define the observed exposure $E_o(i)$ of an item $i$ as the exposure it actually gets after the deployment of a RIR algorithm.
Ideally, the observed exposure of items should be quantified by click-through rates or other user interaction signals.
However, the availability of such comprehensive user-item interactions is seldom possible for third-party researchers due to the sensitive nature of the information.
Counting the number of recommendations received by items (analogous to in-degree of items in RIN) may be a possible work-around in such situations. 
However, the importance of all recommendations is {\it not} the same -- %. The importance of a recommendation 
it varies with the source item, e.g., a recommendation from a popular source item is expected to yield more visibility (for the destination item) than a recommendation from a non-popular item.
 
Taking such observations into consideration, we use the `Random Surfer model'~\cite{random-surfer-model} to estimate the observed exposure. 
In general, users tend to visit the page of an item and then they start exploring different items recommended on the page. Alternatively, they can also randomly consume any other item thereafter. 
By simulating such user exploration for a large number of iterations, we take the {\it steady state visit frequency of a node} \textit{i} as its observed exposure $E_o(i)$ (more details can be found in our prior work~\cite{dash2021umpire}). Note that the notion of observed exposure of an item is very similar to PageRank of the corresponding node, in this formulation.


\vspace{1mm} \noindent
\textbf{Considerations during Random Surfer simulation: }
While simulating user browsing behavior, we note that different users can have different propensity to follow recommendations. 
The `teleportation probability' $\alpha \in [0, 1]$ of the Random Surfer model captures such considerations. 
The surfer chooses to traverse the recommended items with probability $(1 - \alpha)$, and teleport to a random item with probability $\alpha$. Throughout the paper, we report results for ($\alpha = 0.15$) which is the most prevalent value of teleportation in the literature~\cite{dash2021umpire, Brin98theanatomy}. Finally. we normalize the observed 
exposure scores of all items %in such a way 
such that $\sum_{\forall i \in \mathbf{I}}^{}{E_o(i)} = 1$.
	\section{Motivations for Empirical Study}
\label{sec:motivations}
The key question that we try to answer is when and why we should use standard
iteration space tiling over cache oblivious tiling.  The two approaches
perform similar partitioning of the iteration space, but the schedules given
to the partitions are different.  Theoretically, cache oblivious code seems to
have advantages over iteration space tiling.  However, many factors complicate
the actual performance, which made our initial experiments difficult to
interpret.  In this section, we describe the obstacles between the theory and
practice we have identified.

We use Single-Level Tiling (SLT) for iteration space tiling, and Cache
Oblivious Tiling (COT) for cache oblivious techniques in this
paper, which are further described in Section~\ref{sec:background}.

\paragraph{Recursion Overhead} This is a well-known overhead of
COT~\cite{yotov2007experimental}.  The recursion introduces overheads, such as
function call overhead, and increased register pressure.  Furthemore, the
functions force inter-procedural analysis/optimization, known to be more
difficult for compilers well.  Thus, the leaf tiles must be ``sufficiently
large'' to avoid excessive overhead due to the recursion.

 \paragraph{Recursive Split Constraints the Tile Sizes} In typical cache
 oblivious algorithms, the problem is recursively split into halves in each
 dimension. This is in fact a rather coarse-grained exploration of the
 hierarchical partitioning of the iteration space. For instance, if the
 current problem size is $B^3$, then the next sub-problem would be
 $(\frac{B}{2})^3$.  If the best problem size for utilizing a level of cache
 is $(B-x)^3$ where $x\ll \frac{B}{2}$ then the subproblems due to
 divide-and-conquer will not match the best.  This is another factor that
 necessitates fine tuning of leaf tile sizes even for COT, since the utilization
 rate of L1 cache has strong impact on performance.  

%\paragraph{COT Leads to Imbalanced Tiles} Current COT tools recursively split
%the problem into halves in each dimension.  If the original bounds are not
%powers of two, every power-of-two leaf will be paired with a non-power-of-two
%leaf.  Since leaf tile sizes are often carefully tuned, thismeans that half
%the leaves will be suboptimal.  Our code generator incorporates a simple
%optimization that ensures that such suboptimal leaf nodes only occur at the
%boundaries of the iteration space.

\paragraph{COT has more Conflict Misses} The divide-and-conquer execution
order may negatively affect cache interference, especially with high
dimensional data.  This happens when the memory is allocated such that the
accesses are contiguous along some direction in the iteration space (typically
along innermost canonical axis).  With lexicographic order of execution, this
contiguity is largely preserved in the tiled execution.  However,
divide-and-conquer executes neighboring tiles in all dimensions, and many of
those tiles access some distant location in memory.  In contrast to accessing
contiguous regions of memory, accessing various segments of the memory
increases the chances of conflicts.

\paragraph{Hardware Prefetching}  Modern architectures are equipped with
hardware prefetchers that can bring data to the L1 cache. When
having sufficient locality at L2 or LLC makes the program compute-bound, then
the latency to L2/LLC can be hidden by the prefetcher. For such programs, it is
unnecessary to tile for the fastest cache, and larger tiles targeting slower
caches improve performance by maximizing prefetcher
effectiveness~\cite{mehta2016turbotiling}. When the primary objective is speed,
the leaf tiles for COT should also be large, which negates the benefit of
divide-and-conquer, as the leafs are already targeting slower caches.
Prefetching have little impact on parallel executions, since prefetching is
bandwidth limited. When multiple cores try to prefetch at the same time,
the bandwidth limit is quickly reached, and the latency hiding effect is
lost. Furthermore, smaller tile sizes are better for parallel execution for
load balancing  reasons.


These factors limit the effectiveness of COT in various ways and are also
closely tied to the characteristics of the computation. Our empirical study
illustrate the impact of these factors on polyhedral computations.

% Local Variables: ***
% TeX-master: "TACO2017.tex" ***
% fill-column: 78 ***
% End: ***

	\section{Desired Exposure and Exposure bias}
%In this section, we %elaborate on
Next, we discuss how exposure can be % justly 
{\it fairly distributed} among a set of items, by motivating it through the lens of distributive justice~\cite{yaari1984dividing}. We %end this section by defining 
then define `Exposure bias', given the desired and observed distributions of exposure. 

\vspace{-2 mm}
\subsection{Desired exposure of items}
\label{sec:desired}
%There can be multiple notions of what is desired
Exposure in an online platform is a beneficial commodity, %it is convenient to think that individuals (items) 
hence the producers of items would prefer having more of it (than having less). In such a scenario, an intuitive notion of  fairness would be \textbf{equality of exposure}, i.e., the exposure should be uniformly distributed among all the items (by recommended them uniformly). 
%This strategy advocates for  (or attention) for all items such that they all are recommended to users uniformly. %Put differently, the desired objective is equality of opportunity.
However, the characteristics of the %individuals (
items %among whom the exposure is to be distributed are to be understood, 
should also be taken into account, since these characteristics may provide prima facie grounds for a departure from equality~\cite{yaari1984dividing}. 
For instance, all items are probably {\it not} of similar merit or intrinsic quality. This difference in `merit' or `quality' can be a justified reason for departure from equality. Thereby, the `desired exposure' of an individual (item) can be determined by its `deservingness' (merit)\footnote{Desiredness should not be confused with deservingness, i.e., desiredness $\ne$ deservingness in general. Deservingness, in contrast, is an extreme case of desiredness.}. This departure from equality is well established through the notion of \textbf{meritocratic fairness} and the related literature on meritocracy~\cite{joseph2018meritocratic,joseph2016fairness}. For instance, a high-quality item is considered more deserving of %popularity / 
user attention than a low-quality item. 

%Again, not all items have equal contribution to the society. In such cases, 
Alternatively, the desired exposures of various items can also be driven by a broader idea of societal welfare. %the welfare of the society. 
For instance, YouTube `Up next' related video recommendation has recently been criticised for leading users to far-right echo chamber and extremist content~\cite{youtubeRadical}, potentially influencing elections (e.g., the Brazilian presidential election~\cite{youtubeBrazil}). In response, YouTube tweaked its `Up next' algorithm, and started recommending Fox News videos from far-right conspiracy theory videos, instead of other videos from the same channels~\cite{youtubeFoxnews}. Clearly, YouTube deemed some videos {\it unworthy} of the exposure they were getting earlier and decided to nudge users to follow other videos. 
In some scenarios, %since there is a need for exposure of the items and their producers (providers) alike, it can be preferred 
it might be legally required to provide each item with some minimum amount of exposure, regardless of the item attributes~\cite{patro2020incremental,patro2020fairrec}. 


\vspace{1 mm}
\noindent
\textbf{Desiredness as a control knob for fairness: }Note that, we do {\it not} argue for any particular notion of desired exposure distribution; rather, the formulation and algorithms given in the subsequent sections are \textit{\textbf{agnostic}} to any measure of desiredness. 
Rather than advocating for any specific desired exposure, we perceive desiredness as a necessary {\it controllable knob} in our framework to ensure fairness in the final outcomes. 
Hence, if some legislation or a particular platform has a sacrosanct quantification of the desiredness of each item, the same can be easily plugged into our proposed fairness interventions.

\vspace{-2mm}
\subsection{Estimating desired exposure}
We denote the desired exposure of item $i$ as $E_d(i)$, and the desired exposure distribution over all items as $E_d$.
%For the purpose of 
In this work, as a proof of concept, we consider a generic formulation to accommodate multiple types of desired exposure distributions. We consider, a fraction $\beta \in [0, 1]$ of the total exposure is equally distributed among all items. This fraction of the exposure takes care of the minimum exposure of all items (and their producers). It is meant to provide all items with some minimum exposure to satisfy the basic needs of the items and their producers (as argued in~\cite{patro2020incremental}). The remaining ($1-\beta$) fraction of the total exposure is distributed proportional to the quality or merit of individual items, thus advocating \textit{meritocratic fairness}~\cite{joseph2018meritocratic,joseph2016fairness}. Notice, the above formulation of $E_d$ reduces to purely meritocratic distribution of exposure for $\beta = 0$, and to uniform distribution of exposure for $\beta = 1$. The exposure distributions are normalized so that the total exposure of all items in the item-set sum up to 1, i.e., $\sum_{i \in \mathbf{I}}^{}{E_d(i)} = 1$. 


As mentioned earlier, in this work, we assume the {\it average user-rating} of an item as the quantification of its merit / quality. The importance that we attach to an item's merit to obtain its desired exposure is controlled by the parameter $\beta$.


\vspace{1mm}
\noindent
{\bf A potential limitation of user-ratings:} One potential concern about using average user-ratings as a quality metric, might be that the number of ratings an item gets is partly driven by the %recommendations produced by 
existing recommendation algorithms. 
However, we believe that, although a user may have been led to an item via some recommendation, her rating would reflect the inherent quality of the item as perceived by her. 
Further, we also considered a slightly different quality measure -- average user-rating of an item, weighted by its number of ratings. The qualitative results of the analyses remained similar
in this setting too. %as for the simple user-rating measure. 
Hence, for simplicity and completeness we 
consider the average user-rating score to be the indicator of quality throughout this paper. 
%The extent of importance that is attached to this quality is regulated by the parameter $\beta$ to generate various desired exposure distributions.

\vspace{-1mm}
\subsection{Defining exposure bias}  \label{sub:exp-bias}
According to our formulation, a RIR system would be fair (unbiased), if it gives every item an observed exposure that is proportional to its desired exposure. 
Since $E_o(i)$ and $E_d(i)$ denotes the {\it observed and desired exposures} of item $i$, mathematically, a RIR system is fair if $\frac{E_o(i)}{E_d(i)} = \frac{E_o(j)}{E_d(j)} \hspace{2mm}  \forall {i, j} \in \mathbf{I}$.
%\begin{equation}
%\label{eq:merit-fairness-2}
%\frac{E_o(i)}{E_d(i)} = \frac{E_o(j)}{E_d(j)} \hspace{2mm}  %\forall {i, j} \in \mathbf{I}
%\end{equation}
%The above equations are agnostic of any desired exposure distribution.
As discussed in the previous section, a RIR system $R$ may lead to items getting different observed exposures than what is desired. 
{\it Exposure Bias} ($ExpBias$) is the %as a measure of 
deviation caused due to $R$ between the desired and observed exposure of items.
Following the set up in our prior work~\cite{dash2021umpire}, we measure $ExpBias$ by KL divergence~\cite{cha2007comprehensive} between the observed exposure distribution $E_o = \{E_o(i) \; \forall i \in \mathbf{I}\}$ and the desired exposure distribution $E_d = \{E_d(i) \; \forall i \in \mathbf{I}\}$: 
%\vspace{1mm}
\setlength{\belowdisplayskip}{0pt} 
\setlength{\abovedisplayskip}{0pt} 
%\scriptsize
\small
\begin{align}
%\nonumber
ExpBias(R) = D_{KL} (E_o||E_d)= \sum_{i }{E_o(i) \hspace{1mm} log \hspace{1mm} \Big(\frac{E_o(i)}{E_d(i)}\Big)}
\end{align}\normalsize


\vspace{1mm}
\noindent
\textbf{Categorization of items}:
Based on the observed and desired exposure of items, we categorize items in three different classes based on how closely the observed exposure replicates their desired exposure.\\
(a) {\bf Under-exposed:} item $i$ is \textit{under-exposed} if $1-\epsilon \leq \frac{E_o(i)}{E_d(i)}$, \\
(b) {\bf Over-exposed:} item $i$ is \textit{over-exposed} if $\frac{E_o(i)}{E_d(i)} \geq 1+\epsilon$, \\
(c) {\bf Adequately-exposed:} item $i$ is \textit{Adequately-exposed} if $1-\epsilon \leq \frac{E_o(i)}{E_d(i)} \leq 1+\epsilon$, \\
While this threshold ($\epsilon$) can be chosen based on prier context and established regulations, in this paper, we use $\epsilon=0.2$. Note that similar thresholds have been used in multiple prior works too~\cite{chakraborty2017who, dash2018beyond, dash2021umpire}. 

	
\section{The Proposed Method}
\label{sec:Method}
% 开头阐述模型图
% 如图所示,模型的训练分为两阶段,大数据集预训练以及小数据集finetune。在预训练阶段,posterior encoder 以 linear spectrogram作为输入,输出隐变量Z,隐变量Z送入Phoneme predictor。Phoneme predictor 输出Phoneme probability,与Phoneme Look Up Table 相乘得到Phoneme embedding。该embeddding作为歌声合成输入的一部分。
% 2.1 阐述预训练框架
%     标注精良的单歌手歌声数据集例如Opencpop,往往很难有较大的规模,因为标注十分耗费人力。并且由于单歌手的音域固定,所训练出的歌声合成模型很难拥有较广的音域,从而丧失了和真人演唱相比,可能具有的音域优势。为了能够利用大的歌声合成数据集以提高歌声合成系统的音域表现,受xxx文章的启发,我们基于proposed方法的框架,采用了melody-unsupervision在大的歌声合成数据集上做预训练,但又与该文章有着不同。由于我们预训练阶段使用的歌声合成数据集没有详细的标注,仅仅只有text、phonemes,但并没有wav在时间上的详细标注,因此我们希望能够借助ASR的训练方式,通过Phoneme Predictor预测出每一个frame的音素注意力向量 。该音素注意力向量与Phonemes lookup table相乘,最终得到了frame-level的Phonemes embedding。此外,我们从wav中提取出连续的音高,并且将其量化为乐谱音高,经过embedding layer后得到frame-level的pitch embedding。同时由于Opensinger是一个多歌手的数据集,我们使用了基于ECAPA-TDNN的speaker encoder去建模不同歌手的音色信息。
% 2.1.1 Phonemes predictor
% 由于数据集没有Phonemes 的time alignments信息,因此我们采用ASR的训练方法,在pronunciation 层面上使用CTC Loss对Phonemes predictor进行训练。具体而言,Phonemes predictor包含两层FFT Blocks ,一层线性层。线性层将hidden channels 映射到Phonemes 总数对应的类别数。对于线性层的输出,我们取softmax后得到每一个phonemes的概率 p,作为注意力向量与phonemes lookup table相乘得到phonemes embedding。同时我们对p取log,与ground truth标签计算CTC Loss。值得注意的是,与xxx工作不同,由于预训练时没有duration信息,我们的phonemes predictor输出的注意力向量是frame-level的。
% 2.1.2 阐述fine-tune架构
%     finetune架构的搭建整体基于proposed 方法,在proposed 方法的基础上,提出了 d-durator 和bi-flow,提升了我们模型的性能。
% 具体来说,在finetune阶段,模型读取预训练模型的checkpoints,并且在OpenCpop上进行finetune。由于事先在大规模数据集进行了预训练,模型的可合成音域得到了提升。
% 2.2 阐述可微分的上采样模块
% 以前的歌声合成模型,大多采用简单的复制操作将phoneme-level信息转变为frame-level信息,导致模型存在韵律问题。受xxx工作启发,我们提出了可学习的时长预测器,包含一个时长预测器以及可学习的上采样层。时长预测器输出每一个phonemes占发声总时长的比例,该比例与note duration相乘,送入可微分的上采样层。该上采样层 takes a phoneme hidden sequence as input, and outputs a sequence of prior distribution at the frame level。 Compared to simply repeating each phoneme hidden sequence with the predicted duration in a hard way, the differentiable upsampling layer enables more flexible duration adjustment for each phoneme. Also, the differentiable upsampling layer makes the phoneme to frame expansion differentiable, and thus can be jointly optimized with other modules in the TTS system.
% 2.3 阐述双向flow层
% 原先的工作中,flow层在训练阶段将复杂的后验分布映射为简单的先验分布,并且在推理阶段将简单的先验分布转换为复杂的后验分布。但这里存在着training和inference时候的mismatch,也就是 train in backward direction but infer in forward direction。因此我们在训练的时候提出了一个双向的flow模型,对posterior encoder预测的后验分布以及prior encoder预测的先验分布进行转换,并计算loss。值得一提的是,我们发现将flow model两边的kl loss
The training stage of the proposed model consists of two steps: the multi-singer pre-training step and the single-singer fine-tuning step. The architecture of the proposed model is illustrated in Fig.\ref{fig: architecture}, which consists of a prior encoder, a posterior encoder, and a decoder together with a discriminator.
The proposed model is designed from our previous work \cite{zhou22f_interspeech} with the following modifications.
The posterior encoder utilizes a phoneme predictor to predict frame-level phoneme probabilities in the pre-training step. 
The prior encoder adds a speaker encoder to model the timbre variations, replaces the length regulator with a differentiable duration regulator to improve the rhythm naturalness, and upgrades the flow module to be bi-directional to improve the sound quality.

% As illustrated in Fig.\ref{fig: architecture}, the training of the proposed model consists of two stages: the pre-training stage and the fine-tuning stage. 

% In the pre-training stage, the posterior encoder takes the linear spectrogram as input and predicts the latent representation $z$. 
% The phoneme predictor estimates the frame-level phoneme probability $p$ given the latent representation $z$. 
% We multiply $p$ with the phoneme lookup table to get the phoneme embedding. The pitch is extracted from the waveform and we quantified it into note pitch. 

% we reload the checkpoint of the pre-train model and resume training in the OpenCpop\cite{wang2022opencpop} datasets. 


\subsection{The Melody-Unsupervised Multi-Singer Pre-Training Step}
Since the multi-singer training data has no phonemic timing information, in the pre-training step, this work utilizes the automatic speech recognition (ASR) training strategy to train a phoneme predictor in the posterior encoder and predict the frame-level phoneme probabilities $p$.
The probability vectors are multiplied with the phoneme look-up table to obtain the frame-level phoneme embeddings.
In addition, the continuous pitch $f_{0}$ is estimated from the audio and quantized into the note pitch.
The note pitch is passed through the embedding layer to obtain frame-level pitch embeddings.
Moreover, we apply a speaker encoder to extract frame-level speaker embeddings to model the timbre variations of different singers.
Since the pre-training step directly deals with estimated pitch values and focuses on enhancing the vocal range, the pitch predictor, the energy predictor and the duration-related modules are dropped during the pre-training step.


\subsubsection{Phoneme predictor}
We train the phoneme predictor using the connectionist temporal classification (CTC) \cite{graves2006connectionist} loss. 
It contains two layers of FFT blocks and one linear layer.
The linear layer maps the hidden channels to the number of phoneme categories. 
We obtain the probability vector $p$ after taking the softmax operation on the linear layer's output, then multiply it with the phoneme look-up table to get frame-level phoneme embeddings.
Meanwhile, we take the log function of $p$ to compute the CTC loss with the ground truth phoneme sequences. 

\subsubsection{Speaker encoder}
This work adopts one of the state-of-the-art speaker recognition models, i.e. ECAPA-TDNN \cite{desplanques2020ecapa}, as the speaker encoder. Its advanced network architecture and attentive statistics pooling layer have shown great effectiveness in both speaker recognition \cite{desplanques2020ecapa} and voice conversion \cite{guo2022improving,li2022hierarchical}. The speaker encoder is configured as the one with 512 channels in Table 1 of \cite{desplanques2020ecapa}, and it extracts 192-dimensional frame-level speaker embeddings from the audio's Mel-Spectrograms. These embeddings are given as the speaker condition in the multi-singer pre-training step. 

\subsection{The Single-Singer Fine-Tuning Step}
In the fine-tuning step, it loads the pre-trained model parameters, then utilizes the single-singer Opencpop dataset to fine-tune model parameters.
It uses the phoneme and note-pitch annotations provided by the dataset to derive the phoneme and pitch embeddings, instead of using the phoneme probability vectors and quantized f0 values in the pre-training step.
Note that the phoneme and note-pitch annotations are at the phoneme level, rather than the frame level, such that a duration regulator after the note encoder is necessary to up-sample the embedding vectors into the frame level.
As for the speaker embedding, we use the pre-trained speaker encoder to extract an averaged speaker embedding over the Opencpop dataset, then utilize it as a fixed speaker condition during the fine-tuning step.
Moreover, the energy predictor and the pitch predictor join the fine-tuning process to enhance the expressiveness and pitch accurateness of the synthesized samples, following our previous work\cite{zhou22f_interspeech}.
% The fine-tuning architecture is built based on the proposed method \cite{zhou22f_interspeech}, and on top of the proposed method, a differentiable duration predictor and bi-directional flow model are proposed to improve the performance of the synthesized singing voice.
% Specifically, in the fine-tuning phase, the model reloads the checkpoints of the pre-trained model and performs inference on OpenCpop.
The synthesizable vocal range of the model is enhanced due to the multi-singer pre-training on a large-scale dataset.

\subsection{Differentiable Duration Regulator}
Most previous SVS systems simply replicate each phoneme hidden representation with the predicted duration in a hard way, which may degrade the rhythm naturalness.
Inspired by \cite{tan2022naturalspeech}, we leverage a differentiable duration regulator, which contains a duration predictor and a differentiable up-sampling layer. 
The duration predictor outputs the ratio of each phoneme to the corresponding note duration, then the ratio is multiplied by the note duration and fed to the differentiable up-sampling layer.
The differentiable up-sampling layer leverages the predicted duration to learn a projection matrix to extend the phoneme hidden sequence from the phoneme level to the frame level.
It makes the phoneme-to-frame expansion differentiable and thus can be jointly optimized with other modules in the system.

\subsection{Bi-directional Flow}
In the previous work \cite{zhou22f_interspeech}, the flow model maps the complex posterior distribution to the simple prior distribution in the training stage while operating reversely in the inference stage.
This process suffers from the mismatch problem between the training and inference stages.
Therefore, we leverage a bi-directional flow module\cite{tan2022naturalspeech} during training, which bridges the complex posterior distribution and the simple prior distribution bi-directionally to alleviate the mismatch issue in the inference stage.
% not only maps the complex posterior distribution to the simple prior distribution during training, but also maps the simple prior distribution to the complex posterior distribution, in order to improve the quality of the synthesized singing voice.
It is worth noting that we observed that the system can easily fail to train and encounter gradient explosion when the KL losses on both sides of the flow contribute equally, so we define the reverse KL loss weight as 0.5.





	%!TEX root = main.tex
\section{Evaluation}
\label{sec:eval}

In this section, we evaluate the performance of our unsupervised Ordered Word Mover's Distance metric and supervised Multi-scale Sentence Matching model with factorized sentences as input. We apply our algorithms to semantic textual similarity estimation tasks and sentence pair paraphrase identification tasks, based on four datasets: STSbenchmark, SICK, MSRP and MSRvid. 

\subsection{Experimental Setup}
\label{subsec:setup}


\begin{table}[tb]
  \caption{Description of evaluation datasets.}
  \label{tab:datasets}
  \begin{tabular}{lllll}
    \toprule
    Dataset & Task & Train & Dev & Test\\
    \midrule
    STSbenchmark & Similarity scoring & $5748$ & $1500$ & $1378$ \\
    SICK & Similarity scoring & $4500$ & $500$ & $4927$ \\
    MSRP & Paraphrase identification & $4076$ & - & $1725$ \\
    MSRvid & Similarity scoring & $750$ & - & $750$ \\
    \bottomrule
  \end{tabular}
  \vspace{-2mm}
\end{table}

We will start with a brief description for each dataset:
\begin{itemize}
\item \textbf{STSbenchmark}\cite{cer2017semeval}: it is a dataset for semantic textual similarity (STS) estimation. The task is to assign a similarity score to each sentence pair on a scale of 0.0 to 5.0, with 5.0 being the most similar.

\item \textbf{SICK}\cite{marelli2014sick}: it is another STS dataset from the SemEval 2014 task 1. It has the same scoring mechanism as STSbenchmark, where 0.0 represents the least amount of relatedness and 5.0 represents the most.

\item \textbf{MSRvid}: the Microsoft Research Video Description Corpus contains 1500 sentences that are concise summaries on the content of a short video. Each pair of sentences is also assigned a semantic similarity score between 0.0 and 5.0. 

\item \textbf{MSRP}\cite{quirk2004monolingual}: the Microsoft Research Paraphrase Corpus is a set of 5800 sentence pairs collected from news articles on the Internet. Each sentence pair is labeled 0 or 1, with 1 indicating that the two sentences are paraphrases of each other.
\end{itemize}

Table \ref{tab:datasets} shows a detailed breakdown of the datasets used in evaluation.
For STSbenchmark dataset we use the provided train/dev/test split.
The SICK dataset does not provide development set out of the box, so we extracted 500 instances from the training set as the development set.
For MSRP and MSRvid, since their sizes are relatively small to begin with, we did not create any development set for them.

One metric we used to evaluate the performance of our proposed models on the task of semantic textual similarity estimation is the Pearson Correlation coefficient, commonly denoted by $r$. Pearson Correlation is defined as:
\begin{equation}
\label{eq:pearson}
 r = cov(X,Y) /( \sigma_X \sigma_Y),
\end{equation}
where $cov(X,Y)$ is the co-variance between distributions X and Y, and $\sigma_X$, $\sigma_Y$ are the standard deviations of X and Y.
The Pearson Correlation coefficient can be thought as a measure of how well two distributions fit on a straight line. Its value has range [-1, 1], where a value of 1 indicates that data points from two distribution lie on the same line with a positive slope.
% Due to this unique property, we believe the Pearson Correlation coefficient is a strong indicator of the performance of our metric. 

Another metric we utilized is the Spearman's Rank Correlation coefficient. Commonly denoted by $r_s$, the Spearman's Rank Correlation coefficient shares a similar mathematical expression with the Pearson Correlation coefficient, but it is applied to ranked variables.
Formally it is defined as \cite{wiki:spearman}:
\begin{equation}
\label{eq:spearman}
 \rho = cov(rg_X, rg_Y) / (\sigma_{rg_X} \sigma_{rg_Y}),
\end{equation}
where $rg_X$, $rg_Y$ denotes the ranked variables derived from $X$ and $Y$. $cov(rg_X,rg_Y)$, $\sigma_{rg_X}$, $\sigma_{rg_Y}$ corresponds to the co-variance and standard deviations of the rank variables. The term ranked simply means that each instance in X is ranked higher or lower against every other instances in X and the same for Y. We then compare the rank values of X and Y with \ref{eq:spearman}. Like the Pearson Correlation coefficient, the Spearman's Rank Correlation coefficient has an output range of [-1, 1], and it measures the monotonic relationship between X and Y. A Spearman's Rank Correlation value of 1 implies that as X increases, Y is guaranteed to increase as well.
The Spearman's Rank Correlation is also less sensitive to noise created by outliers compared to the Pearson Correlation.

For the task of paraphrase identification, the classification accuracy of label $1$ and the F1 score are used as metrics. 

In the supervised learning portion, we conduct the experiments on the aforementioned four datasets. We use training sets to train the models, development set to tune the hyper-parameters and each test set is only used once in the final evaluation. For datasets without any development set, we will use cross-validation in the training process to prevent overfitting, that is, use $10\%$ of the training data for validation and the rest is used in training. For each model, we carry out training for 10 epochs. We then choose the model with the best validation performance to be evaluated on the test set.  


\subsection{Unsupervised Matching with OWMD}
\label{subsec:eval-owmd}

To evaluate the effectiveness of our Ordered Word Mover's Distance metric, we first take an unsupervised approach towards the similarity estimation task on the STSbenchmark, SICK and MSRvid datasets. Using the distance metrics listed in Table \ref{tab:compare-pearson} and \ref{tab:compare-spearman}, we first computed the distance between two sentences, then calculated the Pearson Correlation coefficients and the Spearman's Rank Correlation coefficients between all pair's distances and their labeled scores. We did not use the MSRP dataset since it is a binary classification problem.


In our proposed Ordered Word Mover's Distance metric, distance between two sentences is calculated using the order preserving Word Mover's Distance algorithm. For all three datasets, we performed hyper-parameter tuning using the training set and calculated the Pearson Correlation coefficients on the test and development set. We found that for the STSbenchmark dataset, setting $\lambda_1=10$, $\lambda_2=0.03$ produces the most optimal result. For the SICK dataset, a combination of $\lambda_1=3.5$, $\lambda_2=0.015$ works best. And for the MSRvid dataset, the highest Pearson Correlation is attained when $\lambda_1=0.01$, $\lambda_2=0.02$.
We maintain a max iteration of 20 since in our experiments we found that it is sufficient for the correlation result to converge.
During hyper-parameter tuning we discovered that using the Euclidean metric along with $\sigma=10$ produces better results, so all OWMD results summarized in Table \ref{tab:compare-pearson} and \ref{tab:compare-spearman} are acquired under these parameter settings. Finally, it is worth mentioning that our OWMD metric calculates the distances using factorized versions of sentences, while all other metrics use the original sentences. Sentence factorization is a necessary preprocessing step for the OWMD metric.


We compared the performance of Ordered Word Mover's Distance metric with the following methods:

\begin{itemize}
\item \textbf{Bag-of-Words (BoW)}: in the Bag-of-Words metric, distance between two sentences is computed as the cosine similarity between the word counts of the sentences.

\item \textbf{LexVec}~\cite{salle2016enhancing}: calculate the cosine similarity between the  averaged 300-dimensional LexVec word embedding of the two sentences. 

\item \textbf{GloVe}~\cite{pennington2014glove}: calculate the cosine similarity between the averaged 300-dimensional GloVe 6B word embedding of the two sentences. 

\item \textbf{Fastext}~\cite{joulin2016bag}: calculate the cosine similarity between the averaged 300-dimensional Fastext word embedding of the two sentences. 

\item \textbf{Word2vec}~\cite{mikolov2013efficient}: calculate the cosine similarity between the averaged 300-dimensional Word2vec word embedding of the two sentences.

\item \textbf{Word Mover's Distance (WMD)}~\cite{kusner2015word}: estimating the semantic distance between two sentences by WMD introduced in Sec.~\ref{sec:owmd}.
\end{itemize} 


\begin{table}[tb]
  \caption{Pearson Correlation results on different distance metrics.}
  \label{tab:compare-pearson}
  \begin{tabular}{c|cc|cc|c}
    \toprule
    \multirow{2}{*}{Algorithm} & \multicolumn{2}{c}{STSbenchmark} & \multicolumn{2}{c}{SICK} & MSRvid\\ 
     & Test & Dev & Test & Dev & Test\\
    \midrule
    BoW & $0.5705$ & $0.6561$ & $0.6114$ & $0.6087$ & $0.5044$ \\
    LexVec & $0.5759$ & $0.6852$ & $0.6948$ & $\mathbf{0.6811}$ & $0.7318$\\
    GloVe & $0.4064$ & $0.5207$ & $0.6297$ & $0.5892$  & $0.5481$ \\
    Fastext & $0.5079$ & $0.6247$ & $0.6517$ & $0.6421$  & $0.5517$  \\
    Word2vec & $0.5550$ & $0.6911$ & $\mathbf{0.7021}$ & $0.6730$  & $0.7209$  \\
    WMD & $0.4241$ & $0.5679$ & $0.5962$ & $0.5953$  & $0.3430$  \\
    OWMD & $\mathbf{0.6144}$ & $\mathbf{0.7240}$ & $0.6797$ & $0.6772$  & $\mathbf{0.7519}$  \\
    \bottomrule
  \end{tabular}
  \vspace{-4mm}
\end{table}

\begin{table}[tb]
  \caption{Spearman's Rank Correlation results on different distance metrics.}
  \label{tab:compare-spearman}
  \begin{tabular}{c|cc|cc|c}
    \toprule
    \multirow{2}{*}{Algorithm} & \multicolumn{2}{c}{STSbenchmark} & \multicolumn{2}{c}{SICK} & MSRvid\\ 
     & Test & Dev & Test & Dev & Test\\
    \midrule
    BoW & $0.5592$ & $0.6572$ & $0.5727$ & $0.5894$ & $0.5233$ \\
    LexVec & $0.5472$ & $0.7032$ & $0.5872$ & $0.5879$ & $0.7311$\\
    GloVe & $0.4268$ & $0.5862$ & $0.5505$ & $0.5490$  & $0.5828$ \\
    Fastext & $0.4874$ & $0.6424$ & $0.5739$ & $0.5941$  & $0.5634$  \\
    Word2vec & $0.5184$ & $0.7021$ & $0.6082$ & $0.6056$  & $0.7175$  \\
    WMD & $0.4270$ & $0.5781$ & $0.5488$ & $0.5612$  & $0.3699$  \\
    OWMD & $\mathbf{0.5855}$ & $\mathbf{0.7253}$ & $\mathbf{0.6133}$ & $\mathbf{0.6188}$  & $\mathbf{0.7543}$  \\
    \bottomrule
  \end{tabular}
  \vspace{-2mm}
\end{table}


Table \ref{tab:compare-pearson} and Table \ref{tab:compare-spearman} compare the performance of different metrics in terms of the Pearson Correlation coefficients and the Spearman's Rank Correlation coefficients.
We can see that the result of our OWMD metric achieves the best performance on all the datasets in terms of the Spearman's Rank Correlation coefficients.
It also produced the best Pearson Correlation results on the STSbenchmark and the MSRvid dataset, while the performance on SICK datasets are close to the best.
This can be attributed to the two characteristics of OWMD. First, the input sentence is re-organized into a predicate-argument structure using the sentence factorization tree. Therefore, corresponding semantic units in the two sentences will be aligned roughly in order. Second, our OWMD metric takes word positions into consideration and penalizes disordered matches. Therefore, it will produce less mismatches compared with the WMD metric.

% On the SICK dataset, although the result of our metric falls slightly behind Word2vec, LexVec on the test set and Word2vec on the development set, we still believe that it is a superior metric because it produced competitive results across multiple datasets. 

% Table \ref{tab:compare-spearman} presents the Spearman's Rank Correlation coefficients acquired with the same distance metrics. We can observe that our OWMD metric achieves the highest correlation scores on all three datasets. Which proves once again that OWMD is a better distance metric for the task of semantic similarity detection.

\subsection{Supervised Multi-scale Semantic Matching}
\label{subsec:eval-multilayer}

\begin{table*}[tb]
  \caption{A comparison among different supervised learning models in terms of accuracy, F1 score, Pearson's $r$ and Spearman's $\rho$ on various test sets.}
  \label{tab:sts}
  \begin{tabular}{c|cc|cc|cc|cc}
    \toprule
    \multirow{2}{*}{Model} & \multicolumn{2}{c}{MSRP} & \multicolumn{2}{c}{SICK} & \multicolumn{2}{c}{MSRvid} & \multicolumn{2}{c}{STSbenchmark}\\ 
     & Acc.(\%) & F1(\%) & $r$ & $\rho$ & $r$ & $\rho$ & $r$ & $\rho$ \\
    \midrule
    MaLSTM & $66.95$ & $73.95$ & $0.7824$ & $0.71843$ & $0.7325$ & $0.7193$ & $0.5739$ & $0.5558$\\
    Multi-scale MaLSTM & $\mathbf{74.09}$ & $\mathbf{82.18}$ & $\mathbf{0.8168}$ & $\mathbf{0.74226}$ & $\mathbf{0.8236}$ & $\mathbf{0.8188}$ & $\mathbf{0.6839}$ & $\mathbf{0.6575}$\\
    \midrule
    HCTI & $73.80$ & $80.85$ & $0.8408$ & $0.7698$ & $\mathbf{0.8848}$ & $\mathbf{0.8763}$  & $\mathbf{0.7697}$ & $\mathbf{0.7549}$ \\
    Multi-scale HCTI & $\mathbf{74.03}$ & $\mathbf{81.76}$ & $\mathbf{0.8437}$ & $\mathbf{0.7729}$ & $0.8763$ & $0.8686$  & $0.7269$ & $0.7033$  \\
    \bottomrule
  \end{tabular}
  \vspace{-2mm}
\end{table*}

The use of sentence factorization can improve both existing unsupervised metrics and existing supervised models. 
% We extend the normal Siamese model to Fig. \ref{fig:network} to take advantage of different level of information in the factorized sentence. 
To evaluate how the performance of existing Siamese neural networks can be improved by our sentence factorization technique and the multi-scale Siamese architecture, we implemented two types of Siamese sentence matching models, HCTI \cite{mueller2016siamese} and MaLSTM \cite{shao2017hcti}. HCTI is a Convolutional Neural Network (CNN) based Siamese model, which achieves the best Pearson Correlation coefficient on STSbenchmark dataset in SemEval2017 competition (compared with all the other neural network models). MaLSTM is a Siamese adaptation of the Long Short-Term Memory (LSTM) network for learning sentence similarity. As the source code of HCTI is not released in public, we implemented it according to \cite{shao2017hcti} by Keras \cite{chollet2015keras}. With the same parameter settings listed in paper \cite{shao2017hcti} and tried our best to optimize the model, we got a Pearson correlation of 0.7697 (0.7833 in paper \cite{shao2017hcti}) in STSbencmark test dataset.

We extended HCTI and MaLSTM to our proposed Siamese architecture in Fig. \ref{fig:network}, namely the Multi-scale MaLSTM and the Multi-scale HCTI. To evaluate the performance of our models, the experiment is conducted on two tasks: 1) semantic textual similarity estimation based on the STSbenchmark, MSRvid, and SICK2014 datasets; 2) paraphrase identification based on the MSRP dataset.

Table \ref{tab:sts} shows the results of HCTI, MaLSTM and our multi-scale models on different datasets. Compared with the original models, our models with multi-scale semantic units of the input sentences as network inputs significantly improved the performance on most datasets. 
Furthermore, the improvements on different tasks and datasets also proved the general applicability of our proposed architecture.

Compared with MaLSTM, our multi-scaled Siamese models with factorized sentences as input perform much better on each dataset. For MSRvid and STSbenmark dataset, both Pearson's $r$ and Spearman's $\rho$ increase about $10\%$ with Multi-scale MaLSTM. Moreover, the Multi-scale MaLSTM achieves the highest accuracy and F1 score on the MSRP dataset compared with other models listed in Table \ref{tab:sts}.

There are two reasons why our Multi-scale MaLSTM significantly outperforms MaLSTM model. First, for an input sentence pair, 
we explicitly model their semantic units with the factorization algorithm.
%we explicitly model the different scales of semantics of them with the semantic units produced by our sentence factorization algorithm. 
Second, our multi-scaled network architecture is 
specifically designed
%specially adapted to 
for multi-scaled sentences representations. Therefore, it is able to explicitly match a pair of sentences at different granularities.

We also report the results of HCTI and Multi-scale HCTI in Table \ref{tab:sts}. For the paraphrase identification task, our model shows better accuracy and F1 score on MSRP dataset. For the semantic textual similarity estimation task, the performance varies across datasets. On the SICK dataset, the performance of Multi-scale HCTI is close to HCTI with slightly better Pearson' $r$ and Spearman's $\rho$. However, the Multi-scale HCTI is not able to outperform HCTI on MSRvid and STSbenchmark. HCTI is still the best neural network model on the STSbenchmark dataset, and the MSRvid dataset is a subset of STSbenchmark.
Although HCTI has strong performance on these two datasets, it performs worse than our model on other datasets.
% Overall, the experimental results demonstrated the superior applicability and generalizability of our proposed models.
Overall, the experimental results demonstrated the general applicability of our proposed model architecture, which performs well on various semantic matching tasks.

% \begin{table}[tb]
%   \caption{Results of Accuracy and F1 score on MSRP test dataset.}
%   \label{tab:MSRP result}
%   \begin{tabular}{lllll}
%     \toprule
%     Model & Acc.(\%) & F1(\%)  \\
%     \midrule
%     MaLSTM & $66.95$ & $73.95$ \\
%     Factorized MaLSTM & $\mathbf{74.09}$ & $\mathbf{82.18}$ \\
%     HCTI & $73.80$ & $80.85$ \\
%     Factorized HCTI & $\mathbf{74.03}$ & $\mathbf{81.76}$ \\
%     \bottomrule
%   \end{tabular}
%   \vspace{0mm}
% \end{table}


% \begin{table}[tb]
%   \caption{Results of Pearson's $r$ and Spearman's $\rho$ on SICK test dataset.}
%   \label{tab:SICK result}
%   \begin{tabular}{lllll}
%     \toprule
%     Model & r & $\rho$ \\
%     \midrule
%     MaLSTM & $0.7824$ & $0.71843$ \\
%     Factorized MaLSTM & $\mathbf{0.8168}$ & $\mathbf{0.74226}$ \\
%     HCTI & $0.8408$ & $\mathbf{0.7698}$ \\
%     Factorized HCTI & $\mathbf{0.8429}$ & $0.7676$ \\
%     \bottomrule
%   \end{tabular}
%   \vspace{0mm}
% \end{table}


% \begin{table}[tb]
%   \caption{Results of Pearson's $r$ and Spearman's $\rho$ on MSRvid test dataset.}
%   \label{tab:MSRvid result}
%   \begin{tabular}{lll}
%     \toprule
%     Model & r & $\rho$  \\
%     \midrule
%     MaLSTM & $0.7325$ & $0.7193$ \\
%     Factorized MaLSTM & $\mathbf{0.8236}$ & $\mathbf{0.8188}$ \\
%     HCTI & $\mathbf{0.8848}$ & $\mathbf{0.8763}$ \\
%     Factorized HCTI & $0.8763$ & $0.8686$ \\
%     \bottomrule
%   \end{tabular}
%   \vspace{0mm}
% \end{table}



% \begin{table}[tb]
%   \caption{Results of Pearson's $r$ and Spearman's $\rho$ on STSbenchmark test dataset.}
%   \label{tab:STSbenchmark result}
%   \begin{tabular}{lllll}
%     \toprule
%     Model & r & $\rho$ \\
%     \midrule
%     MaLSTM & $0.5739$ & $0.5558$ \\
%     Factorized MaLSTM & $\mathbf{0.6839}$ & $\mathbf{0.6575}$ \\
%     HCTI & $\mathbf{0.7697}$ & $\mathbf{0.7549}$ \\
%     Factorized HCTI & $0.7269$ & $0.7033$ \\
%     \bottomrule
%   \end{tabular}
%   \vspace{0mm}
% \end{table}




	\section{Conclusion}

\if 0
In this paper, we considered related item recommendations (RIRs) in movie and e-commerce domains%the movie domain (IMDb and Movielens)
, and  
%a ubiquitous tool in many online platforms which along with helping consumers to compare related items, effectively decide the exposure different items get. 
%After estimating the deserved and observed exposure of different movies in IMDb and MovieLens, we 
studied the impact of the RIRs on the exposure of items. We observed that RIRs, by trying to make two related items closer, induce {\it exposure bias} by not considering the desired exposure to different items. 
Although there can be alternate ways to estimate the observed exposure of different items, we strongly believe that the qualitative result will remain unchanged. We further proposed three  interventions in the recommendation pipeline, which can reduce the exposure bias, while maintaining the effectiveness of recommendation, as evident from the user surveys. 
Note that, though the experiments in this paper are conducted on two datasets, the proposed algorithms are applicable to any other domain, including job recommendation sites, and so on.

%Note that, though the experiments in this paper are conducted on movie recommendations, the algorithms proposed in this paper are applicable to any other domain, including ecommerce sites, job recommendation sites, and so on.

Apart from the RIRs that we considered in this work, another popular form of recommendations are user-specific or 
personalized recommendations (see Figure~\ref{fig:recopipeline}); in future, we would like to develop a similar framework for mitigating exposure bias due to personalized recommendations.


%In this work, we considered the RIRs that only utilize the relatedness with respect to one source item. In many platforms, personalization factors are imposed on top of this underlying mechanism, where for a particular user, instead of one item, multiple items viewed by the user are considered. Capturing the effects of personalization on exposure bias is a broad and complex area of research that we plan to pursue in future.

%This work opens multiple directions for future work. An immediate next step is to capture the effects of personalization on exposure bias of related item recommendations. % is another broad and complex area of research that we want to pursue in future.
\fi 

In this paper, we considered the impact of related item recommendations (RIRs)
on the exposure of items. We show that existing RIRs induce exposure bias by not considering any notion of the desired exposure of items. 
Although there can be alternate ways to estimate the exposure of different items, we believe that the qualitative results would remain unchanged. 
We further proposed a novel suit of algorithms (\textbf{FaiRIR}), which can reduce the exposure bias, while maintaining the effectiveness of recommendations. 
Note that, though the experiments in this paper are conducted on two datasets (from movie and e-commerce domains)
for proof of concept, the proposed algorithms are applicable to any other domain, including job recommendation sites, and others.

In this work, we considered RIRs that only utilize the relatedness with respect to one source item. 
Many platforms provide personalized recommendations to users, where multiple items viewed by a target user are considered. We plan to study the effects of personalization on exposure bias in future.

%In this work, we considered RIRs that only utilize the relatedness with respect to one source item. 
%In many platforms, personalization factors are imposed over this underlying mechanism, where for a particular user, instead of one item, multiple items viewed by a user are considered. We plan to study the effects of personalization on exposure bias in future.

	 
	\section*{Acknowledgment}
	The authors thank the anonymous reviewers and the Associate Editor whose comments helped to improve the paper. This research is supported in part by (1)~a grant from the Max Planck Society through a Max Planck Partner Group at IIT Kharagpur, and (2)~a European Research Council (ERC) Advanced Grant for the project ``Foundations for Fair Social Computing", funded under the European Union's Horizon 2020 Framework Programme (grant agreement no. 789373). Additionally, A. Dash is supported by a fellowship from Tata Consultancy Services.
	
	\balance
	\bibliographystyle{IEEEtran}
	\bibliography{Main}

\end{document}