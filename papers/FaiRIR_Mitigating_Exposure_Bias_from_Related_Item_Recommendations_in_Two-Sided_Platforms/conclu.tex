\section{Conclusion}

\if 0
In this paper, we considered related item recommendations (RIRs) in movie and e-commerce domains%the movie domain (IMDb and Movielens)
, and  
%a ubiquitous tool in many online platforms which along with helping consumers to compare related items, effectively decide the exposure different items get. 
%After estimating the deserved and observed exposure of different movies in IMDb and MovieLens, we 
studied the impact of the RIRs on the exposure of items. We observed that RIRs, by trying to make two related items closer, induce {\it exposure bias} by not considering the desired exposure to different items. 
Although there can be alternate ways to estimate the observed exposure of different items, we strongly believe that the qualitative result will remain unchanged. We further proposed three  interventions in the recommendation pipeline, which can reduce the exposure bias, while maintaining the effectiveness of recommendation, as evident from the user surveys. 
Note that, though the experiments in this paper are conducted on two datasets, the proposed algorithms are applicable to any other domain, including job recommendation sites, and so on.

%Note that, though the experiments in this paper are conducted on movie recommendations, the algorithms proposed in this paper are applicable to any other domain, including ecommerce sites, job recommendation sites, and so on.

Apart from the RIRs that we considered in this work, another popular form of recommendations are user-specific or 
personalized recommendations (see Figure~\ref{fig:recopipeline}); in future, we would like to develop a similar framework for mitigating exposure bias due to personalized recommendations.


%In this work, we considered the RIRs that only utilize the relatedness with respect to one source item. In many platforms, personalization factors are imposed on top of this underlying mechanism, where for a particular user, instead of one item, multiple items viewed by the user are considered. Capturing the effects of personalization on exposure bias is a broad and complex area of research that we plan to pursue in future.

%This work opens multiple directions for future work. An immediate next step is to capture the effects of personalization on exposure bias of related item recommendations. % is another broad and complex area of research that we want to pursue in future.
\fi 

In this paper, we considered the impact of related item recommendations (RIRs)
on the exposure of items. We show that existing RIRs induce exposure bias by not considering any notion of the desired exposure of items. 
Although there can be alternate ways to estimate the exposure of different items, we believe that the qualitative results would remain unchanged. 
We further proposed a novel suit of algorithms (\textbf{FaiRIR}), which can reduce the exposure bias, while maintaining the effectiveness of recommendations. 
Note that, though the experiments in this paper are conducted on two datasets (from movie and e-commerce domains)
for proof of concept, the proposed algorithms are applicable to any other domain, including job recommendation sites, and others.

In this work, we considered RIRs that only utilize the relatedness with respect to one source item. 
Many platforms provide personalized recommendations to users, where multiple items viewed by a target user are considered. We plan to study the effects of personalization on exposure bias in future.

%In this work, we considered RIRs that only utilize the relatedness with respect to one source item. 
%In many platforms, personalization factors are imposed over this underlying mechanism, where for a particular user, instead of one item, multiple items viewed by a user are considered. We plan to study the effects of personalization on exposure bias in future.
