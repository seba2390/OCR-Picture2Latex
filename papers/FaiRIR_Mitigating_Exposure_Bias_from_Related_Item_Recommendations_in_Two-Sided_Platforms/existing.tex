\vspace{-1mm}
\section{RIR Systems and exposure thereof} \label{sec:fairness}

In this section, we present the notion of relatedness between items and how we instantiate an item model capturing it. We also demonstrate the operationalization of exposure induced by an RIR algorithm with respect to the discussed instantiation. %of the item model.

\vspace{-1mm}
\subsection{Relatedness of recommended items}
The primary goal of related item recommendations is to maximize the {\it relatedness of recommended items} to the source item
%The quality of recommendations depends heavily on what consumers perceive as `relatedness' of the recommended items to the source item 
that the consumer has viewed/purchased/liked. 
Though there is no sacrosanct definition of relatedness, two items can be thought of as related over multiple dimensions: \\
(i)~{\bf Content based relatedness}, e.g., movies of the same genre, items from the same producer or brand, etc., \\
(ii)~{\bf Compatibility}: two items can be related if they are either the substitute or complement of one another, e.g., items that are frequently purchased together: a smartphone and its cover, \\ %For example, users who have consumed an item have also consumed another item  or users consumed a pair of items together.
(iii)~{\bf External feedback on recommendation platforms}: user-actions such as likes and ratings, also define relatedness. For example, items being rated similarly, liked or disliked by a number of common users can be considered as related. 

\noindent
Relatedness, therefore, is subjective, and RIRs are judged 
%on how consumers perceive their quality to be; in other words, recommendations are relevant when people 
based on whether the consumers find the source and the recommended items to be related. Additionally, the metric to measure relatedness between items is often domain-dependent.
%Note that the notion of relatedness is a {\it local} property of an item, with respect to a given %seed or 
%source item.
{\it The concept of `relatedness' is analogous to `accuracy' or 'relevance' in the context of a related item recommendation system} -- just like classifiers are traditionally designed to optimize for accuracy, RIR systems are traditionally designed to optimize (maximize) relatedness.

\begin{figure}[tb] 
	\begin{tabular}{|c|c|}
		\hline
		Items & Related items \\
		\hline
		$I_1$  & $I_2$, $I_3$, $I_4$ \\
		\hline
		$I_2$  & $I_3$, $I_5$, $I_6$\\
		\hline
		$I_3$  & $I_4$, $I_5$, $I_6$ \\
		\hline
	\end{tabular}
	\adjustimage{width=0.75cm,valign=c}{figures/arrow.png}\quad
	\adjustimage{width=3.0cm,valign=c}{figures/Network3BigData.pdf}\quad
	\caption{\bf A sample item model and its corresponding RIN.}
	\label{Fig:RINCreation}
	\vspace{-7mm}
\end{figure}


\subsection{Instantiating Item Model by Related Item Network}

As shown in Figure~\ref{fig:recopipeline}, 
both Related Item Recommendations and Personalized Recommendation systems utilize an {\it Item Model} that captures the relatedness among items.
%Usually the relatedness among items is decided by using some Related Item Recommendation algorithm, some of which are discussed in Section~\ref{sec: RIRAlgos}. 
We now discuss an intuitive way to instantiate the Item Model, which was developed in our prior works~\cite{dash2021umpire, dash2019network}.

We utilize an instantiation of the Item Model of a recommendation system as a {\it Related Item Network} (RIN). 
A RIN is a directed network, with each node being analogous to an item in the universe,  
and a directed edge between two nodes implies that the corresponding source and destination items are related (based on some underlying notion of relatedness).
For instance, let us consider an item model as shown in the table in Figure~\ref{Fig:RINCreation}, and its corresponding RIN. 
Since item `$I_2$' is related to item `$I_1$', the corresponding nodes in the RIN are connected via a directed edge (from `$I_1$' to `$I_2$').%\todo{Is this picture already used in some earlier paper? Then we should redraw a different example. --AM}
%there exists a directed edge from `A' to `B'. 

Once this instantiation of item model is constructed, a simple way to generate the Related Item Recommendations is as follows. For a particular source item, one can recommend those items to which it links in the RIN. For instance, in Figure~\ref{Fig:RINCreation}, the recommendations for source item `$I_1$' are items `$I_2$', `$I_3$', and `$I_4$'.


\subsection{Estimating observed exposure}\label{sec: EstOexp}

We define the observed exposure $E_o(i)$ of an item $i$ as the exposure it actually gets after the deployment of a RIR algorithm.
Ideally, the observed exposure of items should be quantified by click-through rates or other user interaction signals.
However, the availability of such comprehensive user-item interactions is seldom possible for third-party researchers due to the sensitive nature of the information.
Counting the number of recommendations received by items (analogous to in-degree of items in RIN) may be a possible work-around in such situations. 
However, the importance of all recommendations is {\it not} the same -- %. The importance of a recommendation 
it varies with the source item, e.g., a recommendation from a popular source item is expected to yield more visibility (for the destination item) than a recommendation from a non-popular item.
 
Taking such observations into consideration, we use the `Random Surfer model'~\cite{random-surfer-model} to estimate the observed exposure. 
In general, users tend to visit the page of an item and then they start exploring different items recommended on the page. Alternatively, they can also randomly consume any other item thereafter. 
By simulating such user exploration for a large number of iterations, we take the {\it steady state visit frequency of a node} \textit{i} as its observed exposure $E_o(i)$ (more details can be found in our prior work~\cite{dash2021umpire}). Note that the notion of observed exposure of an item is very similar to PageRank of the corresponding node, in this formulation.


\vspace{1mm} \noindent
\textbf{Considerations during Random Surfer simulation: }
While simulating user browsing behavior, we note that different users can have different propensity to follow recommendations. 
The `teleportation probability' $\alpha \in [0, 1]$ of the Random Surfer model captures such considerations. 
The surfer chooses to traverse the recommended items with probability $(1 - \alpha)$, and teleport to a random item with probability $\alpha$. Throughout the paper, we report results for ($\alpha = 0.15$) which is the most prevalent value of teleportation in the literature~\cite{dash2021umpire, Brin98theanatomy}. Finally. we normalize the observed 
exposure scores of all items %in such a way 
such that $\sum_{\forall i \in \mathbf{I}}^{}{E_o(i)} = 1$.