\vspace{-5pt}
\section{Conclusion}
Previous work has argued that sparse and dense retrievers encode different relevance characteristics of a document~\cite{lin2020pretrained,wang2021bert}. Because of this, methods for the combination of these two signals have emerged; the simplest method being the interpolation of the scores originating from a sparse and a dense retriever~\cite{wang2021bert}.

In this paper, we conducted an extensive investigation to study the effect of interpolation between sparse and dense retrievers in the context of Pseudo Relevance Feedback for dense retrievers, and in particular for the scalable and effective Vector-PRF method~\cite{li2021pseudo}. 
%how to combine vector-PRF and interpolation with sparse retrievers for improving both early precision and recall of dense retrievers. 
With this respect, we studied applying BM25 and uniCOIL as sparse retrievers, along with three dense retrievers: ANCE, TCTV2 and DistillBERT. In terms of when to interpolate sparse and dense signals, we considered doing this before PRF, after PRF, and both before and after PRF.

%\todo{With the trained models included, one factor to consider is the non-deterministic results due to dense representation learning. If dense retrievers trained from scratch instead of using the original checkpoint, because of the non-deterministic behaviors such as random initialization, random negative samples, might lead to different effectiveness. However, in our previous paper~\cite{li2022improving}, we show that the difference between trained from scratch and using the original checkpoint is very minor and not statistically significant.}

The empirical results show that interpolation often can boost retrieval effectiveness, regardless of the choice of sparse and dense retrievers. Among the choices of when to interpolate, we found that interpolating both before and after the PRF process is the condition that most often lead to substantial gains.

%interpolating with dense retrievers with Pre-PRF, Post-PRF and Both-PRF settings. From the results, we find that when uniCOIL is interpolated with dense PRF models, Post-PRF or Both-PRF interpoletion method should always used to achieve the highest effectiveness. However, this is different when BM25 is used as sparse representation......\todo{What was our finding on bm25}.
%When comparing between unsupervised or learned sparse models, we also find that using unsupervised sparse model to conduct PRF interpolation can always achieve a higher recall, while using learned sparse model can ensure a higher early precision measures.