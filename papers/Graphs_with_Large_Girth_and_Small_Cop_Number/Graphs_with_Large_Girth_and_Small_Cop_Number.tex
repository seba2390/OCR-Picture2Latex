\documentclass{amsart}
% \pdfoutput=1
%%%%%%%%%%%%%%%Long/short version
\usepackage{etoolbox}
%%%%%%%%%%%%%%%%%END Long/short version macros
\usepackage[utf8]{inputenc}
\usepackage[T1]{fontenc}
\usepackage{lmodern}
\usepackage[foot]{amsaddr}
\usepackage{fullpage}
\usepackage{microtype}
\usepackage{amsfonts}
%\usepackage{epsfig}
\usepackage{graphicx}
\usepackage{tabularx}
\usepackage{array}
\usepackage[usenames,dvipsnames]{color}
\usepackage{amsmath}
\usepackage{amsthm}
\usepackage{amssymb}
\usepackage{xcolor}
\usepackage{tikz}
\usetikzlibrary{backgrounds}
\usetikzlibrary{shapes.geometric}
\usetikzlibrary{calc}
\usepackage{hyperref}
\usepackage{algorithm}
\usepackage{dsfont}
%\newtheorem*{conjecture}{Conjecture}

\tikzstyle{legend_general}=[rectangle, rounded corners, thin,
                          top color= white,bottom color=lavander!25,
                          minimum width=2.5cm, minimum height=0.8cm,
                          violet]

\def\AC{\color{blue}}



\DeclareMathOperator{\dist}{dist}
\DeclareMathOperator{\exponential}{exp}


\newtheorem{theorem}{Theorem}[section]
\newtheorem{proposition}[theorem]{Proposition}
\newtheorem{conjecture}[theorem]{Conjecture}
\newtheorem{question}[theorem]{Question}
\newtheorem{claim}[theorem]{Claim}
\newtheorem{lemma}[theorem]{Lemma}
\newtheorem{corollary}[theorem]{Corollary}
\theoremstyle{definition}
\newtheorem{definition}[theorem]{Definition}
\newtheorem{example}[theorem]{Example}
\newtheorem{remark}[theorem]{Remark}
\newtheorem{observation}[theorem]{Observation}
\DeclareMathOperator\df{:=}

\def\epsilon{\varepsilon}

\title{Graphs with Large Girth and Small Cop Number}
\author{Alexander Clow\\
\footnotesize Department of Mathematics\\
\footnotesize Simon Fraser University\\
\footnotesize Burnaby, British Columbia, Canada\\
\footnotesize\tt alexander\_clow@sfu.ca\\}
\date{\today}


\begin{document}

\maketitle

\begin{abstract}
In this paper we consider the cop number of graphs with no, or few, short cycles. We show that when the girth of $G$ is at least $7$ and the minimum degree is sufficiently large, $\delta \geq (\ln{n})^{\frac{1}{1-\alpha}}$ where $\alpha \in (0,1)$, then $c(G) = o(n \delta^{1-\lfloor \frac{g}{4} \rfloor})$ as $n \rightarrow \infty$. This extends work of Frankl and implies that if $G$ is large and dense in the sense that $\delta \geq n^{\frac{2 + o(1)}{g-1}}$ while also having girth $g \geq 7$, then $G$ satisfies Meyniel's conjecture, that is $c(G) = O(\sqrt{n})$. Moreover, it implies that if $G$ is large and dense in the sense that there $\delta \geq n^{\epsilon}$ for some $\epsilon >0$, while also having girth $g \geq 7$, then there exists an $\alpha>0$ such that $c(G) = O(n^{1-\alpha})$, thereby satisfying the weak Meyniel's conjecture. Of course, this implies similar results for dense graphs with small, that is $O(n^{1-\alpha})$, numbers of short cycles, as each cycle can be broken by adding a single cop. We also, show that there are graphs $G$ with girth $g$ and minimum degree $\delta$ such that the cop number is at most $o(g (\delta-1)^{(1+o(1))\frac{g}{4}})$. This resolves a recent conjecture by Bradshaw, Hosseini, Mohar, and Stacho, by showing that the constant $\frac{1}{4}$ cannot be improved in the exponent of a lower bound $c(G) \geq \frac{1}{g} (\delta - 1)^{\lfloor \frac{g-1}{4}\rfloor}$.
\end{abstract}



\section{Introduction}

In this paper we consider the game of \emph{cops and robbers} on graphs. Cops and robbers is a 2-player game played on a graph $G = (V,E)$. To being the game, the cop player places $k$ cops onto vertices of the graph, then the robber player chooses a vertex of the graph to place the robber at the beginning of the game. After this players take turns moving their pieces from the current vertex they occupy, say $v$, to a vertex in $N[v]$. Note this allows a player to not move a given cop or robber, should they desire to do so. The cops win if after finitely many moves a cop is able to move onto the vertex currently occupied by the robber, which we call capturing the robber, while the robber wins if there exists a strategy which indefinitely evades capture. The least number of cops required to win the game no matter which vertex the robber begins on is known as the cop number of a graph, denoted $c(G)$. Suppose all graphs we consider are connected.

Cops and robbers was introduced independently by Quilliot \cite{quilliot1983problemes}, and Nowakowski and Winkler \cite{nowakowski1983vertex}; while the cop number of a graph, was first introduced by Aigner and Fromme \cite{aigner1984game}. In recent decades cops and robbers has seen a remarkable amount of attention with many general results as well as results for specific graph classes appearing in the literature. For instance, the cop number of planar graphs \cite{aigner1984game}, graphs of higher genus \cite{bonato2020topological, bowler2019bounding, quilliot1985short, schroder2001copnumber}, Cayley graphs \cite{bradshaw2020proof,frankl1987cops}, graph products \cite{neufeld1998game}, and random graphs \cite{bollobas2013cops,luczak2010chasing, pralat2016meyniel} have all been extensively studied. Significant attention has also been paid to computational questions involving cop number, with MacGillivray and Clarke \cite{clarke2012characterizations} showing that deciding if a graph is $k$-cop win is fixed parameter polynomial time in the order of the graph and $k$, while Kinnersley \cite{kinnersley2015cops} proved that determining if $c(G) \leq k$ is EXPTIME-complete when $k$ is not fixed.

The most famous problem in cops and robbers is undoubtedly \emph{Meyniel's conjecture}, which states that for all graphs $G$, $c(G) = O(\sqrt{n})$. If correct, then this bound could not be improved as there are known graph families with $c(G) = \Omega(\sqrt{n})$ \cite{bonato2013cops, pralat2010does}. Despite Meyniel's conjecture being resolved for a number of graph classes such as Abelian Cayley graphs \cite{bradshaw2020proof} and random graphs \cite{bollobas2013cops, pralat2016meyniel}, the larger conjecture remains widely open despite significant effort \cite{chiniforooshan2008better,frieze2012variations, lu2012meyniel, scott2011bound}. In fact, it remains to be shown that there exists a $\alpha>0$ such that $c(G) = O(n^{1-\alpha})$, with this problem sometimes begin dubbed the \emph{weak Meyniel's conjecture}. For more on Meyniel's conjecture see \cite{baird2013meyniel}.



In this paper we focus on graphs with no, or few short cycles. The \emph{girth} of a graph $G$ is the length of the shortest cycle in $G$.  Considering graphs of large girth allows our analysis to avoid complicated local structures appearing in may graphs. Cops and robbers on graphs with large girth was first considered by Frankel in \cite{frankl1987cops} who showed that that $c(G) > (\delta-1)^{(1-o(1))\frac{g}{8}}$ and if $G$ is a $d$-regular graph of girth $g$, then $c(G) \leq n (1+(\frac{g}{4} +o(1)) \ln{d}) d^{-(1-o(1))\frac{g}{4}}$. This lower bound was recently improved by Bradshaw, Hosseini, Mohar, and Stacho in \cite{bradshaw2023cop} who showed that $c(G) \geq \frac{1}{g} (\delta - 1)^{\lfloor \frac{g-1}{4}\rfloor}$ and conjecture that the exponential coefficient of $\frac{1}{4}$ cannot be improved. This conjecture is quite reasonable as it is implied by combining Meyniel's conjecture and a conjecture that exists in folklore stating that there exist $d$-regular graphs of girth $g$ and order $d^{(1+o(1))\frac{g}{2}}$. We prove in Theorem~\ref{Thm: Large Girth Small Cop Number} that this conjecture from \cite{bradshaw2023cop} is true. Additionally, we improve Frankl's upper bound for the cop number of a graph in terms of girth when the minimum degree of a graph is sufficiently large as a function of $n$ in Theorem~\ref{Thm: Cop Number Upper Bound}.


\begin{theorem}\label{Thm: Cop Number Upper Bound}
If $G = (V,E)$ be a graph of order $n$ which has girth $g \geq 7$ and $\delta = \delta(G) \geq (\ln{n})^{\frac{1}{1-\alpha}}$ such that $\alpha \in (0,1)$, then for all $\epsilon>0$, 
\[
c(G) \leq (\epsilon + o(1))n \delta^{1 - \lfloor \frac{g}{4} \rfloor}.
\]
\end{theorem}


\begin{corollary}\label{Corollary: Meyniel's Conjecutre Implication}
If $G = (V,E)$ has girth $g \geq 7$ and $\delta = \delta(G) \geq n^{x}$ where $x = \frac{1-y}{\lfloor \frac{g}{4} \rfloor - 1}$ is a constant, then $c(G) = o(n^{y})$. 
\end{corollary}


Thus, we have shown that all large graphs with girth $g \geq 7$ and $\delta \geq n^{\frac{1}{2(\lfloor \frac{g}{4} \rfloor - 1)}}$ must have $c = o(\sqrt{n})$. Therefore, all such graphs trivially satisfy Meyniel's conjecture. Furthermore, Corollary~\ref{Corollary: Meyniel's Conjecutre Implication} implies that if $G$ is a sufficiently large graph with girth at least $7$ and $\delta \geq n^{\epsilon}$ where $\epsilon>0$, then there exists an $\alpha>0$ such that $c(G) = o(n^{1-\alpha})$ thereby showing that all such graphs satisfy the weak Meyniel's conjecture. 

Furthermore, we note that this implies that if $G$ is a graph with at most $O(n^{1-\alpha})$ cycles of length $g-1$ or less, then we may conclude that if $\delta \geq n^{\frac{\alpha}{ \lfloor \frac{g}{4} \rfloor-o(1)}}$, then $c(G) = O(n^{1-\alpha})$. This is because choosing one vertex per cycle to form a set of vertices $S$, then placing a cop at each vertex in $S$, means that 
\[
c(G) \leq |S|+c(G-S) = O(n^{1-\alpha})
\]
given the cops on $S$ can remain on $S$ for the full game, thereby confining the robber to $G-S$ where $c(G-S)$ cops will be able to capture the robber.


A $d$-regular graph with girth $g$ of minimal order is called a $(d,g)$-cage and the order of a $(d,g)$-cage is denoted $n(d,g)$. The problem of constructing $(d,g)$-cages and determining $n(d,g)$ has been extensively studied by a number of authors. For a survey on such problems see \cite{exoo2012dynamic}. Of particular interest for us is the following bound by Lazebnik, Ustimenko, and Woldar \cite{lazebnik1997upper}; let $d \geq 2$ and $g \geq 5$ be integers, and let $q$ denote an odd prime power for which $d \leq q$, then 
\[
n(d,g) \leq 2dq^{\frac{3}{4}g-a}
\]
where $a = 4,11/\! 4, 7/\! 2, 13/\! 4$, for $g \equiv 0,1,2,3 \mod{4}$ respectively.


\begin{corollary}\label{Corollary: Cop Number of Specific Cages}
If $G = (V,E)$ is a $(p^k,4t+3)$-cage where $t\geq 1$ is fixed and $p$ is an odd prime, then $c(G) = o(p^{2tk}) = o(\delta^{(1-o(1))\frac{g}{2}})$ as $k \rightarrow \infty$.
\end{corollary}




\begin{theorem}\label{Thm: Large Girth Small Cop Number}
There exists graphs $G$ with $\delta(G) = \delta$ and girth $g$ where 
\[
c(G) = o(g (\delta-1)^{(1+o(1))\frac{g}{4}}).
\]
\end{theorem}







\vspace{0.5cm}
\section{Proof of Theorem~\ref{Thm: Cop Number Upper Bound}}

As the title suggests in this section we will prove Theorem~\ref{Thm: Cop Number Upper Bound}. Our proof is in part probabilistic and proceeds similarly to arguments that appear in \cite{lu2012meyniel, scott2011bound, bradshaw2023cop}. For readers unfamiliar with the probabilistic method  we recommend \cite{alon2016probabilistic} as a reference. Additionally, we adopt notation specific to cops and robbers from \cite{bonato2011game}. Observe that for a graph $G=(V,E)$ and a vertex $v\in V$ we define $B_r(v) := \{u \in V: \dist(u,v) \leq r\}$.


\begin{lemma}[\cite{aigner1984game}]\label{Lemma: Geodesic Path}
Let $G = (V,E)$ be graph. If $p = v_0,\dots, v_k$ is a shortest path from $v_0$ to $v_k$, then $p$ is $1$-guardable.
\end{lemma}


Note that Lemma~\ref{Lemma: Geodesic Path} is a standard tool used to upper bound cop number. For instance Lemma~\ref{Lemma: Geodesic Path} is used by Aigner and  Fromme in \cite{aigner1984game} to show that every planar graphs has cop number at most $3$, which is tight, while also being employed in recent research, such as that of Scott and Sudakov \cite{scott2011bound} who showed $c(G) \leq n 2^{-(1+o(1))\sqrt{\log{n}}}$. The following lemma is less standard but is easily verified. Surprisingly, these two lemmas are sufficient prerequisites to prove Theorem~\ref{Thm: Cop Number Upper Bound}. 



\begin{lemma}\label{Lemma: Given u_{t}}
Let $G = (V,E)$ be a graph with girth $g = 4t+3 + r$ where $0 \leq r \leq 3$ and $t \geq 1$. Then, for all $u,v\in V$, if $\dist(v,u) = t$, then letting $U = B_{t+1}(u) \setminus (\cup_{w\neq u \in V;\\ \dist(v,w)=t}B_{t+1}(w))$,
\[
 |U|  \geq (\delta-1)^{t+1}.
\]
\end{lemma}

\begin{proof}
Let $G = (V,E)$ be a graph with girth $g = 4t+3+ r$ where $0 \leq r \leq 3$ and $t \geq 1$.  Then, for all $v \in V$, $T_v = G[B_{2t+1}(v)]$ is a tree, or the only cycles in $T_v$ contain $v$ and an edge joining two vertices with distance $2t+1$ from $v$. Let $u$ be a fixed but arbitrary vertex of distance $t$ from $v$ and let $U = B_{t+1}(u) \setminus (\cup_{w\neq u \in V;\\ \dist(v,w)=t}B_{t+1}(w))$.

It is clear that there are at least $(\delta-1)^{t+1}$ vertices $z$ in $T_v$ with $\dist(z,v) = 2t+1$ with a shortest path $v$ in $T_v$ containing $u$, hence such leaves are within distance $t+1$ of $u$. Furthermore such a shortest path connecting $z$ and $v$ must be unique as girth of $G$ is $g \geq 4t +3$. Then it is clear that for all $w \neq u$ such that $\dist(w,v) = t$, $w$ has no path to any such vertex $z$ of length less than or equal to $t+1$ as this would contradict the uniqueness of a shortest path connecting $z$ and $v$. The result follows immediately.
\end{proof}



\begin{proof}[Proof of Theorem~\ref{Thm: Cop Number Upper Bound}]
Let $G = (V,E)$ be a graph of order $n$ which has girth $g \geq 7$ such that $\delta = \delta(G) \geq (\ln{n})^{\frac{1}{1-\alpha}}$ for $0 < \alpha < 1$. Let $C$ be a random subset of $V$ such that for all $v \in V$, $\mathbb{P}(v \in C) = p$ where $p \in (0,1)$ is to be chosen later. 

Let $g = 4t+3 + r$ where $0 \leq r \leq 3$. Suppose $v \in V$ be fixed but arbitrary and let $E_v$ be the event that there exits a vertex $u \in V$ where  $\dist(u,v) = t$ and there is no cop within distance $2t+1$ of $v$ whose shortest path to $v$ contains $u$. Then, applying Lemma~\ref{Lemma: Given u_{t}}, the union bound, and the fact that $G[B_{2t+1}(v)]$ is a tree, or a graph whose only cycles contain $v$ and an edge joining two vertices with distance $2t+1$ from $v$, we can see that,
\begin{align*} 
\mathbb{P}(E_v) \leq \Delta(\Delta-1)^{t-1} (1-p)^{(\delta-1)^{t+1}} \leq \Delta(\Delta-1)^{t-1} \exponential(-p(\delta-1)^{t+1}).
\end{align*}
Note that $\delta \geq (\ln{n})^{\frac{1}{1-\alpha}}$ implies $n \leq e^{\delta^{1-\alpha}}$. Hence, letting $p = \epsilon \delta^{-t}$ where $\epsilon>0$ and $\beta >1- \alpha$, and applying the union bound,
\begin{align*} 
\mathbb{P}(\bigcup_{v\in V}E_v) \leq n \mathbb{P}(E_v) \\
 \leq n\Delta(\Delta-1)^{t-1} \exponential(-p(\delta-1)^{t+1}) \\
\leq n^{t+1} \exponential(-p(\delta-1)^{t+1}) \\
\leq n^{t+1} \exponential(-(\epsilon-o(1))\delta^{t+1  - t})\\
\leq  n^{t+1} \exponential(-(\epsilon-o(1)) \delta) \\
\leq \exponential((t+1)\delta^{1-\alpha}-(\epsilon-o(1)) \delta) \rightarrow 0
\end{align*}
as $\delta \rightarrow \infty$ with $n \rightarrow \infty$, given our assumption that $1- \alpha< 1$. So asymptotically almost surly for every vertex $v$ and $u \in V$ where $\dist(u,v) = t$, there is a cop with distance at most $2t+1$ from $v$ whose shortest path to  $v$ contains $u$. 

Suppose at the beginning of the game we place $1$ cop on each vertex of $C$ and suppose without loss of generality that the robber begins the game on vertex $v$. Then asymptotically almost surly the cops will win in at most $2t+1$ turns by following a strategy where each cop takes a geodesic path towards $v$. This is because given $G[B_t(v)]$ is a tree by our assumption that $G$ has girth $g \geq 4t+3$ and with high probability the robber will not be able to escape $B_t(v)$ before there is a cop on the unique geodesic path connecting  each $u \in B_t(v)$ to $v$; where this cop will either be occupying the same vertex as the robber or a vertex nearer to $u$ than the the robber. From here the result follows by Lemma~\ref{Lemma: Geodesic Path}.

Now we will proceed to bound $|C|$ using Chernoff's Bound. Note that $|C|$ is a binomial random variable, hence $\mathbb{E}(|C|) = np = \epsilon n \delta^{ - t}$. Now applying Chernoff's Bound,
\begin{align*} 
\mathbb{P}(|C| \geq \mathbb{E}(|C|)+k) \leq \exponential(- \frac{k^2}{2(\mathbb{E}(|C|)+ \frac{k}{3})})
\end{align*}
so letting $k = \mathbb{E}(|C|)^{c}$ where $\frac{1}{2}<c<1$ it is clear that 
\begin{align*} 
\mathbb{P}(|C| \geq \mathbb{E}(|C|)+k) \rightarrow 0
\end{align*}
as $n \rightarrow \infty$. So asymptotically almost surly, $|C| \leq  \mathbb{E}(|C|) + \mathbb{E}(|C|)^{c} = (\epsilon+o(1))n \delta^{-t}$. Noting that $t \geq \lfloor \frac{g}{4} \rfloor -1$ this completes the proof.
\end{proof}



Note that our proof also implies that the capture time for $(\epsilon + o(1))n \delta^{1 - \lfloor \frac{g}{4} \rfloor}$ cops is approximately $\frac{g}{2}$. This is somewhat surprising, at least to the author, as this implies the capture time does not necessarily depend on $n$ for such graphs. Due to this we believe that Theorem~\ref{Thm: Cop Number Upper Bound} is unlikely to be a tight. However, it is notable that discovering an improvement with a large capture time may be difficult as it will likely involve a more complicated strategy by the cops.




\vspace{0.5cm}
\section{Proof of Theorem~\ref{Thm: Large Girth Small Cop Number}}


In this section we will show that bipartite Ramanujan graphs introduced by Lubotsky, Phillips, and Sarnak in \cite{lubotzky1988ramanujan} have cop number at most $(\epsilon + o(1))g\delta^{(1+o(1))\frac{g}{4}}$ for every $\epsilon>0$. Note that these are the same graphs considered in \cite{bradshaw2023cop} to show that there are graphs who's cop number is at most $(\delta-1)^{(1+o(1))\frac{3g}{8}}$.  Our argument employs similar ideas to that of \cite{bradshaw2023cop}, hence we will use two lemmas lemmas which appear in \cite{bradshaw2023cop}. We begin by defining the \emph{isoperimetric function}, $\Phi (G,k)$, 
\[
\Phi (G,k) := \min_{S\subset V(G); |S| \leq k} \frac{|N(S)\setminus S|}{|S|} .
\]



\begin{lemma}[\cite{bradshaw2023cop}]\label{Lemma: Neighbourhood Size given Normalized Isoperimetric Function}
Let $G = (V,E)$ be a graph of order $n$. If $\Phi (G,n^{1-\alpha}) \geq \beta > 0$, then for all integers $r$, and any nonempty set $S \subset V$, 
\[
|B_r(S)| \geq \min\{ n^{1-\alpha}, |S|(1+\beta)^r\}.
\]
\end{lemma}

Hence, for graphs whose isoperimetric function is large we can lower bound the size of the $r$-neighbourhood of any small set of vertices $S$. This will be critical to ensuring that there are sufficiently many cops in a position to help capturing the robber in the proof of Theorem~\ref{Thm: Large Girth Small Cop Number}.


The next step is to ensure that the graphs we consider have sufficiently large $\Phi (G,n^{1-\alpha})$. Again we borrow a Lemma from \cite{bradshaw2023cop} to do this. As this lemma relies on ideas from spectral graph theory which we do not introduce here, we refer readers who are curious about or unfamiliarly with spectral graph theory to  \cite{chung1997spectral}. Note that the proof of Lemma~\ref{Lemma: Isomtermteric Lower Bound for X^{p,q}} in \cite{bradshaw2023cop} relies on a result of Tanner \cite{tanner1984explicit}. 

\begin{lemma}[\cite{bradshaw2023cop}]\label{Lemma: Isomtermteric Lower Bound for X^{p,q}}
Let $0 < \alpha < 1$ be fixed. Let $G$ be a connected and $d$-regular bipartite graph. Suppose that $G$ has eigenvalues $d = \lambda_1(G) \geq \lambda_2(G) \geq \dots \geq \lambda_n(G) = -d$. Then 
\[
\Phi (G,n^{1-\alpha}) \geq (\frac{d}{\lambda_2(G)})^2 - 1 - o(1)
\]
as $n \rightarrow \infty$.
\end{lemma}

With both of these results in hand we are ready to prove Theorem~\ref{Thm: Large Girth Small Cop Number}.



\begin{proof}[Proof of Theorem~\ref{Thm: Large Girth Small Cop Number}]
Let $G = (V,E)$ be the Ramanujan graph $X^{p,q}$ discovered in \cite{lubotzky1988ramanujan}, where $d = p+1$, and $p,q \equiv 1 \mod{4}$ are prime, and $q \geq \sqrt{p}$ such that the Legendre symbol of $q$ and $p$ satisfies $(\frac{q}{p}) = -1$. Then $G$ is a $d$-regular Ramanujan graph on $n = q(q^2-1)$ vertices with girth $g \geq \frac{4 \log{q}}{\log{p}} -1$ \cite{lubotzky1988ramanujan}. From the definition of Ramanujan graphs the second eigenvalue of $G$, $\lambda_2$, has $\lambda_2^2 \leq 4(d-1)$.

Applying Lemma~\ref{Lemma: Isomtermteric Lower Bound for X^{p,q}}, this implies that as $n \rightarrow \infty$, that is $q \rightarrow \infty$, 
\[
\Phi (G,n^{1-\alpha})  \geq \frac{d^2}{4(d-1)} - 1 - o(1) \geq (1-o(1))\frac{d}{4}
\]
for any $0 < \alpha < 1$. Let $t = \lfloor \frac{g-2}{2} \rfloor$ and observe that for $G[B_t(v)]$ is a tree. For sufficiently large $q$ there exists an $0< \alpha < 1$ such that
\begin{align*} 
|B_t(v) \setminus B_{t-1}(v)| = d(d-1)^{t-1} = (1-o(1))d^t \ll  n^{1-\alpha}.
\end{align*}
Let such an $\alpha$ be fixed. Then, for sufficiently large $q$, Lemma~\ref{Lemma: Neighbourhood Size given Normalized Isoperimetric Function} implies that for any $S \subset V$ with $|S| \ll n^{1-\alpha}$, 
\begin{align*} 
|B_{t+1}(S)| \geq |S|(1+(1-o(1))\frac{d}{4})^{t+1} = |S|(\frac{d}{4+o(1)})^{t+1}.
\end{align*}

% Look above

Now let $C$ be a random subset of $V$ such that for all $u \in V$, $\mathbb{P}(u \in C) = 4\epsilon t (\frac{d}{4+o(1)})^{1-t}$ where $\epsilon \geq \frac{4+o(1)}{d}$. For a fixed $v \in V$ and let $H_v$ be the random bipartite graph formed from $G$ and $C$ as follows; take $V(H_v) = (B_t(v) \setminus B_{t-1}(v)) \cup (C \cap B_{t+1}(B_t(v) \setminus B_{t-1}(v))$ where $B_t(v) \setminus B_{t-1}(v)$ and $C \cap B_{t+1}(B_t(v) \setminus B_{t-1}(v)$  is the bipartition, then let there be an edge from $u \in B_t(v) \setminus B_{t-1}(v)$ to $w\in C \cap B_{t+1}(B_t(v) \setminus B_{t-1}(v)$ iff $w \in B_{t+1}(u)$. Let $E_v$ be the event that there is no $(B_t(v) \setminus B_{t-1}(v))$-saturating matching in $H_v$.

For fixed $S\subset V$ such that $|S| \ll n^{1-\alpha}$ let $A_S$ be the random variable that counts $|C \cap B_{t+1}(S)|$. Then $A_S$ is a binomial random variable, hence, $\mathbb{E}(A_S) \geq (1-o(1)) |S| 4\epsilon t \frac{d^2}{16+o(1)} \geq (4-o(1)) |S| t \frac{d}{4+o(1)}$, and applying Chernoff's Bound, 
\begin{align*} 
\mathbb{P}(A_S \leq \mathbb{E}(A_S) + k) \leq \exponential (- \frac{k^2}{2\mathbb{E}(A_S)}).
\end{align*}
Observe that applying Hall's Marriage Theorem and the union bound, and letting $k =  (2-o(1))|S| t \frac{d}{4+o(1)}$ and $|S| =s$
\begin{align*}
\mathbb{P}(E_v) \leq \sum_{m=1}^{d^t} \dbinom{d^t}{m} \mathbb{P}(A_S < s\mid s = m) \\
< \sum_{m=1}^{d^t} \dbinom{d^t}{m} \mathbb{P}(A_S \leq  \frac{1}{2}\mathbb{E}(A_S)) \\
\leq \sum_{m=1}^{d^t} \dbinom{d^t}{m} \mathbb{P}(A_S \leq  (2-o(1))|S| t \frac{d}{4+o(1)})\\
\leq \sum_{m=1}^{d^t} d^{tm} \exponential (- \frac{k^2}{2\mathbb{E}(A_S)}) \\
\leq \sum_{m=1}^{d^t} \exponential (\ln(d) m t - (1-o(1)) m t \frac{d}{4+o(1)}) < R
\end{align*}
 as $q,p \rightarrow \infty$  and therefore $t,d \rightarrow \infty$ where $R< 1$ is a constant, by the convergence of geometric series. Assume $\sqrt{p}\leq q \leq e^p$. Finally, consider that when $q \leq e^{p+1}$,
\begin{align*}
\mathbb{P}(\bigcup_{v \in V} E_v) \leq n \mathbb{P}(E_v) \\
\leq n \sum_{m=1}^{d^t} \exponential (\ln(d) m t - (1-o(1)) m t \frac{d}{4+o(1)}) \\
\leq (1+o(1))e^{3p+3} \sum_{m=1}^{d^t} \exponential (\ln(d) m t - (1-o(1)) m t \frac{d}{4+o(1)})\\
= \sum_{m=1}^{d^t} \exponential (3d + \ln(d) m t - (1-o(1)) m t \frac{d}{4+o(1)}) < N
\end{align*}
 as $q,p \rightarrow \infty$  and therefore $t,d \rightarrow \infty$ where $N< 1$ is a constant, again given the convergence of geometric series.

So if $q \leq e^{p+1} \rightarrow \infty$, for every $v \in V$, $H_v$ has a $(B_t(v) \setminus B_{t-1}(v))$-saturating matching with probability bounded away from $0$. Of course the existence of such a matching implies that if we place cops on $C$ at the beginning of the game, then cops win with probability bounded away from $0$. This is because if the robber begins at $v$, then the cops win if for all $u \in B_t(v) \setminus B_{t-1}(v)$ and $(u,w)$ in a $(B_t(v) \setminus B_{t-1}(v))$-saturating matching of $H_v$ the cop at $w$ takes a shortest path to $u$, then there will be a cop on every vertex of $B_t(v) \setminus B_{t-1}(v)$ within $t+1$ cop turns. Hence, by the same argument as in Theorem~\ref{Thm: Cop Number Upper Bound} the robber will be captured in $2t+1$ turns. Thus, cops win starting on $C$ with probability bounded away from $0$ as $p,q \rightarrow \infty$.

As in Theorem~\ref{Thm: Cop Number Upper Bound} we proceed to bound the size of $C$ using Chernoff's Bound. Observe that, 
\begin{align*}
\mathbb{E}(|C|) = n 4\epsilon t (\frac{d}{4+o(1)})^{1-t} \\
\leq  (2\epsilon + o(1)) g (d-1)^{(\frac{3}{4}+o(1))g}  (\frac{d}{4+o(1)})^{1-t} 
\end{align*}
given $n = (1-o(1))q^3$ and $g \geq 4 \log_{p}(q)$ implies $n \leq (d-1)^{(1+o(1))\frac{3}{4}g}$. From this, 
\begin{align*}
\mathbb{E}(|C|) \leq (2\epsilon+o(1)) g (d-1)^{(3 + \log_{p}(4+o(1)) - 2+ o(1)) \frac{g}{4}} \\
= (2\epsilon+o(1)) g (d-1)^{(1+o(1))\frac{g}{4}}
= o(g (d-1)^{(1+o(1))\frac{g}{4}})
\end{align*}
as $p,q \rightarrow \infty$ given we can pick $\epsilon \rightarrow 0$ due to our only assumption regarding $\epsilon$ be that it be great then or equal to $\frac{4+o(1)}{d} \rightarrow 0$. From here it is easy to see that using the same Chernoff Bound argument as in the proof of Theorem~\ref{Thm: Cop Number Upper Bound} that
\begin{align*}
\mathbb{P}(|C| \leq (1+o(1))\mathbb{E}(|C|)) \rightarrow 1
\end{align*}
as $p,q \rightarrow \infty$. This concludes the proof.
\end{proof}









\vspace{0.5cm}
\section{Conclusion}

In this paper we have improved an upper bound for the cop number given by Frankl appearing in \cite{frankl1987cops} for of graphs with girth at least $7$ and large minimum degree. In doing so we have shown that such graphs have cop number $o(n \delta^{(1-o(1))\frac{g}{4}})$ where the asymptotics here are in $n$. This result implies that for all $\alpha \in (0,1)$ then there exists a sufficiently large $0< \epsilon< 1$ such that if $G$ has $\delta \geq n^{\epsilon}$ and has at most $O(n^{1-\alpha})$ cycles of length $6$ or less, then $c(G) = O(n^{1-\alpha})$. We conjecture that this upper bound is not best possible, as the the capture time implied by our proof is independent of the size of our graphs, which leads us to believe that there may be more complicated strategy which proves more advantageous for the cops. We also conjecture that our bound holds for graphs with girth less than $7$ and graphs with minimum degree $\delta = O(\ln{n})$ and less. Additionally, we have proven a conjecture of Bradshaw, Hosseini, Mohar, and Stacho from \cite{bradshaw2023cop}, and in so doing shown that the lower bound $c(G) \geq \frac{1}{g} (\delta - 1)^{\lfloor \frac{g-1}{4}\rfloor}$ is within a sub-exponential factor of being tight.


\vspace{0.5cm}
\section*{Acknowledgements}
We would also like to acknowledge the support of the Natural Sciences and Engineering Research Council of Canada (NSERC) for support through the Canadian Graduate Scholarship -- Master's program.


%{\AC Thanks to Laco and Peter feedback. Trent for introducing me to the isometric function.}

\bibliographystyle{plain}
\bibliography{CopsRobbers}


\end{document}