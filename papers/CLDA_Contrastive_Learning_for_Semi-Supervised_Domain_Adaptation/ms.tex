\documentclass{article}
\pdfoutput=1
% if you need to pass options to natbib, use, e.g.:
    \PassOptionsToPackage{numbers, compress}{natbib}
% before loading neurips_2021

% ready for submission
%   \usepackage{neurips_2021}

% to compile a preprint version, e.g., for submission to arXiv, add add the
% [preprint] option:
%  \usepackage[preprint,nonatbib]{neurips_2021}

% to compile a camera-ready version, add the [final] option, e.g.:
      \usepackage[final,nonatbib]{neurips_2021}

% to avoid loading the natbib package, add option nonatbib:
% \usepackage[nonatbib]{neurips_2021}

\usepackage[utf8]{inputenc} % allow utf-8 input
\usepackage[T1]{fontenc}    % use 8-bit T1 fonts
\usepackage{url}            % simple URL typesetting
\usepackage{booktabs}       % professional-quality tables
\usepackage{amsfonts}       % blackboard math symbols
\usepackage{nicefrac}       % compact symbols for 1/2, etc.
\usepackage{microtype}      % microtypography
\usepackage{xcolor} 
\usepackage{graphicx}
\usepackage{array,multirow} 
\usepackage{booktabs} % for professional tables
\usepackage[lofdepth,lotdepth]{subfig}
\usepackage[ruled,vlined,linesnumbered]{algorithm2e}
\SetKwInput{KwInput}{Input}
\SetKwInput{KwOutput}{Output}
\usepackage{float}
\usepackage{mathtools} % for splitfrac
% \usepackage[table]{xcolor}
% \floatstyle{plaintop}
\usepackage{bbold}
\restylefloat{table}
\newcommand{\ankit}[1]{{\color{blue}{ [Ankit: #1]}}}
\usepackage{booktabs} % colors
\usepackage[
backend=biber,
maxcitenames=50,
maxnames=50,
style=numeric,
citestyle=numeric
]{biblatex}
\addbibresource{ms.bib}
\usepackage{hyperref}       % hyperlinks
\title{CLDA: Contrastive Learning for Semi-Supervised Domain Adaptation}


\author{%
  Ankit Singh \\
  Department of Computer Science\\
  Indian Institute of Technology, Madras \\
  \texttt{singh.ankit@cse.iitm.ac.in} \\
}

\begin{document}

\maketitle

\begin{abstract}
Unsupervised Domain Adaptation (UDA) aims to align the labeled source distribution with the unlabeled target distribution to obtain domain invariant predictive models. However, the application of well-known UDA approaches does not generalize well in Semi-Supervised Domain Adaptation (SSDA) scenarios where few labeled samples from the target domain are available.
This paper proposes a simple \textbf{C}ontrastive \textbf{L}earning framework for semi-supervised \textbf{D}omain \textbf{A}daptation (\textbf{CLDA}) that attempts to bridge the intra-domain gap between the labeled and unlabeled target distributions and the inter-domain gap between source and unlabeled target distribution in SSDA. We suggest employing class-wise contrastive learning to reduce the inter-domain gap and instance-level contrastive alignment between the original(input image) and strongly augmented unlabeled target images to minimize the intra-domain discrepancy. We have empirically shown that both of these modules complement each other to achieve superior performance. Experiments on three well-known domain adaptation benchmark datasets, namely DomainNet, Office-Home, and Office31, demonstrate the effectiveness of our approach. CLDA achieves state-of-the-art results on all the above datasets.
\end{abstract}

\IEEEraisesectionheading{\section{Introduction}}

\IEEEPARstart{V}{ision} system is studied in orthogonal disciplines spanning from neurophysiology and psychophysics to computer science all with uniform objective: understand the vision system and develop it into an integrated theory of vision. In general, vision or visual perception is the ability of information acquisition from environment, and it's interpretation. According to Gestalt theory, visual elements are perceived as patterns of wholes rather than the sum of constituent parts~\cite{koffka2013principles}. The Gestalt theory through \textit{emergence}, \textit{invariance}, \textit{multistability}, and \textit{reification} properties (aka Gestalt principles), describes how vision recognizes an object as a \textit{whole} from constituent parts. There is an increasing interested to model the cognitive aptitude of visual perception; however, the process is challenging. In the following, a challenge (as an example) per object and motion perception is discussed. 



\subsection{Why do things look as they do?}
In addition to Gestalt principles, an object is characterized with its spatial parameters and material properties. Despite of the novel approaches proposed for material recognition (e.g.,~\cite{sharan2013recognizing}), objects tend to get the attention. Leveraging on an object's spatial properties, material, illumination, and background; the mapping from real world 3D patterns (distal stimulus) to 2D patterns onto retina (proximal stimulus) is many-to-one non-uniquely-invertible mapping~\cite{dicarlo2007untangling,horn1986robot}. There have been novel biology-driven studies for constructing computational models to emulate anatomy and physiology of the brain for real world object recognition (e.g.,~\cite{lowe2004distinctive,serre2007robust,zhang2006svm}), and some studies lead to impressive accuracy. For instance, testing such computational models on gold standard controlled shape sets such as Caltech101 and Caltech256, some methods resulted $<$60\% true-positives~\cite{zhang2006svm,lazebnik2006beyond,mutch2006multiclass,wang2006using}. However, Pinto et al.~\cite{pinto2008real} raised a caution against the pervasiveness of such shape sets by highlighting the unsystematic variations in objects features such as spatial aspects, both between and within object categories. For instance, using a V1-like model (a neuroscientist's null model) with two categories of systematically variant objects, a rapid derogate of performance to 50\% (chance level) is observed~\cite{zhang2006svm}. This observation accentuates the challenges that the infinite number of 2D shapes casted on retina from 3D objects introduces to object recognition. 

Material recognition of an object requires in-depth features to be determined. A mineralogist may describe the luster (i.e., optical quality of the surface) with a vocabulary like greasy, pearly, vitreous, resinous or submetallic; he may describe rocks and minerals with their typical forms such as acicular, dendritic, porous, nodular, or oolitic. We perceive materials from early age even though many of us lack such a rich visual vocabulary as formalized as the mineralogists~\cite{adelson2001seeing}. However, methodizing material perception can be far from trivial. For instance, consider a chrome sphere with every pixel having a correspondence in the environment; hence, the material of the sphere is hidden and shall be inferred implicitly~\cite{shafer2000color,adelson2001seeing}. Therefore, considering object material, object recognition requires surface reflectance, various light sources, and observer's point-of-view to be taken into consideration.


\subsection{What went where?}
Motion is an important aspect in interpreting the interaction with subjects, making the visual perception of movement a critical cognitive ability that helps us with complex tasks such as discriminating moving objects from background, or depth perception by motion parallax. Cognitive susceptibility enables the inference of 2D/3D motion from a sequence of 2D shapes (e.g., movies~\cite{niyogi1994analyzing,little1998recognizing,hayfron2003automatic}), or from a single image frame (e.g., the pose of an athlete runner~\cite{wang2013learning,ramanan2006learning}). However, its challenging to model the susceptibility because of many-to-one relation between distal and proximal stimulus, which makes the local measurements of proximal stimulus inadequate to reason the proper global interpretation. One of the various challenges is called \textit{motion correspondence problem}~\cite{attneave1974apparent,ullman1979interpretation,ramachandran1986perception,dawson1991and}, which refers to recognition of any individual component of proximal stimulus in frame-1 and another component in frame-2 as constituting different glimpses of the same moving component. If one-to-one mapping is intended, $n!$ correspondence matches between $n$ components of two frames exist, which is increased to $2^n$  for one-to-any mappings. To address the challenge, Ullman~\cite{ullman1979interpretation} proposed a method based on nearest neighbor principle, and Dawson~\cite{dawson1991and} introduced an auto associative network model. Dawson's network model~\cite{dawson1991and} iteratively modifies the activation pattern of local measurements to achieve a stable global interpretation. In general, his model applies three constraints as it follows
\begin{inlinelist}
	\item \textit{nearest neighbor principle} (shorter motion correspondence matches are assigned lower costs)
	\item \textit{relative velocity principle} (differences between two motion correspondence matches)
	\item \textit{element integrity principle} (physical coherence of surfaces)
\end{inlinelist}.
According to experimental evaluations (e.g.,~\cite{ullman1979interpretation,ramachandran1986perception,cutting1982minimum}), these three constraints are the aspects of how human visual system solves the motion correspondence problem. Eom et al.~\cite{eom2012heuristic} tackled the motion correspondence problem by considering the relative velocity and the element integrity principles. They studied one-to-any mapping between elements of corresponding fuzzy clusters of two consecutive frames. They have obtained a ranked list of all possible mappings by performing a state-space search. 



\subsection{How a stimuli is recognized in the environment?}

Human subjects are often able to recognize a 3D object from its 2D projections in different orientations~\cite{bartoshuk1960mental}. A common hypothesis for this \textit{spatial ability} is that, an object is represented in memory in its canonical orientation, and a \textit{mental rotation} transformation is applied on the input image, and the transformed image is compared with the object in its canonical orientation~\cite{bartoshuk1960mental}. The time to determine whether two projections portray the same 3D object
\begin{inlinelist}
	\item increase linearly with respect to the angular disparity~\cite{bartoshuk1960mental,cooperau1973time,cooper1976demonstration}
	\item is independent from the complexity of the 3D object~\cite{cooper1973chronometric}
\end{inlinelist}.
Shepard and Metzler~\cite{shepard1971mental} interpreted this finding as it follows: \textit{human subjects mentally rotate one portray at a constant speed until it is aligned with the other portray.}



\subsection{State of the Art}

The linear mapping transformation determination between two objects is generalized as determining optimal linear transformation matrix for a set of observed vectors, which is first proposed by Grace Wahba in 1965~\cite{wahba1965least} as it follows. 
\textit{Given two sets of $n$ points $\{v_1, v_2, \dots v_n\}$, and $\{v_1^*, v_2^* \dots v_n^*\}$, where $n \geq 2$, find the rotation matrix $M$ (i.e., the orthogonal matrix with determinant +1) which brings the first set into the best least squares coincidence with the second. That is, find $M$ matrix which minimizes}
\begin{equation}
	\sum_{j=1}^{n} \vert v_j^* - Mv_j \vert^2
\end{equation}

Multiple solutions for the \textit{Wahba's problem} have been published, such as Paul Davenport's q-method. Some notable algorithms after Davenport's q-method were published; of that QUaternion ESTimator (QU\-EST)~\cite{shuster2012three}, Fast Optimal Attitude Matrix \-(FOAM)~\cite{markley1993attitude} and Slower Optimal Matrix Algorithm (SOMA)~\cite{markley1993attitude}, and singular value decomposition (SVD) based algorithms, such as Markley’s SVD-based method~\cite{markley1988attitude}. 

In statistical shape analysis, the linear mapping transformation determination challenge is studied as Procrustes problem. Procrustes analysis finds a transformation matrix that maps two input shapes closest possible on each other. Solutions for Procrustes problem are reviewed in~\cite{gower2004procrustes,viklands2006algorithms}. For orthogonal Procrustes problem, Wolfgang Kabsch proposed a SVD-based method~\cite{kabsch1976solution} by minimizing the root mean squared deviation of two input sets when the determinant of rotation matrix is $1$. In addition to Kabsch’s partial Procrustes superimposition (covers translation and rotation), other full Procrustes superimpositions (covers translation, uniform scaling, rotation/reflection) have been proposed~\cite{gower2004procrustes,viklands2006algorithms}. The determination of optimal linear mapping transformation matrix using different approaches of Procrustes analysis has wide range of applications, spanning from forging human hand mimics in anthropomorphic robotic hand~\cite{xu2012design}, to the assessment of two-dimensional perimeter spread models such as fire~\cite{duff2012procrustes}, and the analysis of MRI scans in brain morphology studies~\cite{martin2013correlation}.

\subsection{Our Contribution}

The present study methodizes the aforementioned mentioned cognitive susceptibilities into a cognitive-driven linear mapping transformation determination algorithm. The method leverages on mental rotation cognitive stages~\cite{johnson1990speed} which are defined as it follows
\begin{inlinelist}
	\item a mental image of the object is created
	\item object is mentally rotated until a comparison is made
	\item objects are assessed whether they are the same
	\item the decision is reported
\end{inlinelist}.
Accordingly, the proposed method creates hierarchical abstractions of shapes~\cite{greene2009briefest} with increasing level of details~\cite{konkle2010scene}. The abstractions are presented in a vector space. A graph of linear transformations is created by circular-shift permutations (i.e., rotation superimposition) of vectors. The graph is then hierarchically traversed for closest mapping linear transformation determination. 

Despite of numerous novel algorithms to calculate linear mapping transformation, such as those proposed for Procrustes analysis, the novelty of the presented method is being a cognitive-driven approach. This method augments promising discoveries on motion/object perception into a linear mapping transformation determination algorithm.





\section{Related Work}

  \begin{figure*}[t]
    \centering
    \includegraphics[width=\linewidth]{figures/pipeline}
    % \vspace{-12pt}
    \vspace{-20pt}
    \caption{The overview of our approach. Given the six RGB stream inputs surrounding the performer and objects, our approach generates high-quality human-object meshes and free-view rendering results. ``DR'' indicates differentiable rendering.
    }
    \vspace{-10pt}
    \label{fig:pipeline}
  \end{figure*}
    
  
\noindent{\textbf{Human Performance Capture.}}
Markerless human performance capture techniques have been widely investigated to achieve human free-viewpoint video or reconstruct the geometry. 
%
The high-end solutions~\cite{motion2fusion,TotalCapture,collet2015high,chen2019tightcap} adopt studio-setup with dense cameras to produce high-quality reconstruction and surface motion, but the synchronized and calibrated multi-camera systems are both difficult to deploy and expensive.
%
The recent low-end approaches~\cite{Xiang_2019_CVPR,LiveCap2019tog,chen2021sportscap, he2021challencap} enable light-weight performance capture under the single-view setup or even hand-held capture setup or drone-based capture setup~\cite{xu2017flycap}.
%
However, these methods require a naked human model or pre-scanned template. 
Volumetric fusion based methods~\cite{newcombe2015CVPR,DoubleFusion,BodyFusion,HybridFusion} enables free-form dynamic reconstruction. But they still suffer from careful and orchestrated motions, especially for a self-scanning process where the performer turns around carefully to obtain complete reconstruction. 
%
\cite{robustfusion} breaks self-scanning constraint by introducing implicit occupancy method.
%
 All these methods suffer from the limited mesh resolution leading to uncanny texturing output. Recent method~\cite{mustafa2020temporally} leverages unsupervised temporally coherent human reconstruction to generate free-viewpoint rendering. It is still hard for this method to get photo-realistic rendering results.
%
Comparably, our approach enables the high-fidelity capture of human-object interactions and eliminates the additional motion constraint under the sparse view RGB camera settings.


\noindent{\textbf{Neural Rendering.}}
The recent progress of differentiable neural rendering brings huge potential for 3D scene modeling and photo-realistic novel view synthesis. Researchers explore various data representations to pursue better performance and characteristics, such as point-clouds~\cite{Wu_2020_CVPR,aliev2019neural,suo2020neural3d}, voxels~\cite{lombardi2019neural}, texture meshes~\cite{thies2019deferred,liu2019neural} or implicit functions~\cite{park2019deepsdf,nerf,meng2021gnerf,chen2021mvsnerf,wang2021mirrornerf,luo2021convolutional}. 
%
However, these methods require inevitable pre-scene training to a new scene.
%
For neural modeling and rendering of dynamic scenes, NHR~\cite{Wu_2020_CVPR} embeds spatial features into sparse dynamic point-clouds, Neural Volumes~\cite{NeuralVolumes} transforms input images into a 3D volume representation by a VAE network.
% 
More recently, \cite{park2020deformable,pumarola2020d,li2020neural,xian2020space,tretschk2020non,peng2021neural,zhang2021editable} extend neural radiance field (NeRF)~\cite{nerf} into the dynamic setting. 
%
They learn a spatial mapping from the canonical scene to the current scene at each time step and regress the canonical radiance field. 
% 
However, for all the dynamic approaches above, dense spatial views or full temporal frames are required in training for high fidelity novel view rendering, leading to deployment difficulty and unacceptable training time overhead. Recent approaches~\cite{peng2021neural} and ~\cite{NeuralHumanFVV2021CVPR} adopt a sparse set of camera views to synthesize photo-realistic novel views of a performer. However, in the scenario of human-object interaction, these methods fail to generate both realistic performers and realistic objects.
Comparably, our approach explores the sparse capture setup and fast generates photo-realistic texture of challenging human-object interaction in novel views.

% \myparagraph{\textbf{Human-object capture}}
\noindent{\textbf{Human-object capture.}}
%
Early high-end work~\cite{collet2015high} captures both human and objects by reconstruction and rendering with dense cameras. 
%
Recently, several works explore the relation between human and scene to estimate 3D human pose and locate human position~\cite{hassan2019resolving,HPS,liu20204d}, naturally place human~\cite{PSI2019,PLACE:3DV:2020,hassan2021populating} or predict human motion~\cite{cao2020long}. 
%
Another related direction~\cite{GRAB:2020,hampali2021handsformer,liu2021semi} models the relationship between hand and objects for generation or capture.
%
PHOSA~\cite{2020phosa_Arrangements} runs human-object capture without any scene- or object-level 3D supervision using constraints to resolve ambiguity. 
However, they only recover the naked human bodies and produce a visually reasonable spatial arrangement.
%
A concurrent close work is RobustFusion(journal)~\cite{su2021robustfusion}. They capture human and objects by volumetric fusion respectively, and track object by Iterative Closest Point (ICP). 
However, their texturing quality is limited by mesh resolution and color representation, and the occluded region is ambiguous in 3D space.
%
Comparably, our approach enables photo-realistic novel view synthesis and accurate human object arrangement in 3D world space
under the human-object interaction for the light-weight sparse RGB settings.

\section{Method}\label{sec:method}
%

\subsection{Interaction-aware Human-Object Capture}\label{sec:human_capture}
Classical multi-view stereo reconstruction approaches \citep{Furukawa2013,Strecha2008,Newcombe2011,collet2015high} and recent neural rendering approaches \citep{Wu_2020_CVPR,NeuralVolumes,nerf} rely on multi-view dome based setup to achieve high-fidelity reconstruction and rendering results.
%
However, they suffer from both sparse-view inputs and occlusion of objects.
%
To this end, we propose a novel implicit human-object capture scheme to model the mutual influence between human and object from only sparse-view RGB inputs.

\noindent{\textbf{(a) Occlusion-aware Implicit Human Reconstruction.}}
For the human reconstruction, we perform a neural implicit geometry generation to jointly utilize both the pixel-aligned image features and global human motion priors with the aid of an occlusion-aware training data augmentation.
% 

Without dense RGB cameras and depth cameras, traditional multi-view stereo approaches \citep{collet2015high,motion2fusion} and depth-fusion approaches \citep{KinectFusion,UnstructureLan,robustfusion} can hardly reconstruct high-quality human meshes.
%
With implicit function approaches \citep{PIFU_2019ICCV,PIFuHD}, we can generate fine-detailed human meshes with sparse-view RGB inputs.
%
However, the occlusion from human-object interaction can still cause severe artifacts.
%
To end this, we thus utilize the pixel-aligned image features and global human motion priors.

% 
Specifically, we adopt the off-the-shelf instance segmentation approach \citep{Bolya_2019_ICCV} to obtain human and object masks, thus distinguishing the human and object separately from the sparse-view RGB input streams.
%
Meanwhile, we apply the parametric model estimation to provide human motion priors for our implicit human reconstruction.
%
We voxelized the mesh of this estimated human model to represent it with a volume field.

We give both the pixel-aligned image features and global human motion priors in volume representation to two different encoders of our implicit function, as shown in Fig. \ref{fig:pipeline} (a).
%
Different from \cite{2020phosa_Arrangements} with only a single RGB input, we use pixel-aligned image features from the multi-view inputs and concatenate them with our encoded voxel-aligned features.
%
We finally decode the pixel-aligned and voxel-aligned feature to occupancy values with a multilayer perceptron (MLP).

For each query 3D point $P$ on the volume grid, we follow PIFu \citep{saito2019pifu} to formulate the implicit function $f$ as:
\begin{align}
	f( \Phi(P),\Psi(P),Z(P)) & = \sigma : \sigma \in [0.0, 1.0],             \\
	\Phi(P)                  & = \frac{1}{n} \sum_{i}^{n}F_{I_{i}}(\pi_{i}(P)), \\
	\Psi(P)                  & =  G(F_{V},P),
\end{align}
where $p = \pi_{i}(P)$ denotes the projection of 3D point to camera view $i$, $F_{I_{i}}(x)= g(I_{i}(p))$ is the image feature at $p$.
%
$\Psi(P) = G(F_{V},P)$ denotes the voxel aligned features at $P$, $F_{V}$ is the voxel feature.
%
To better deal with occlusion, we introduce an occlusion-aware reconstruction loss to enhance the prediction at the occluded part of human.
% 
It is formulated as:
	\begin{align}
		 & \mathcal{L}_{\sigma} = \lambda_{occ}\sum_{t=1}^T \left\| \sigma_{occ}^{gt} - \sigma_{occ}^{pred} \right\|_2^2 + \lambda_{vis}\sum_{t=1}^T \left\| \sigma_{vis}^{gt} - \sigma_{vis}^{pred} \right\|_2^2.
	\end{align}

% 
Here, $\lambda_{occ}$ and $\lambda_{vis}$ represent the weight of occlusion points and visible points, respectively.
%
$\sigma_{occ}$ and $\sigma_{vis}$ are the training sampling points at the occlusion area and visible area.


\begin{figure}[t]
    \centering
    \includegraphics[width=\linewidth]{figures/data_augmentation}
    \vspace{-10pt}
    \caption{Illustration of our synthetic 3D data with both human and objects.}
    % \vspace{-1mm}
    \vspace{-15pt}
    \label{fig:DataAugmentation}
\end{figure}

\begin{figure*}[t]
	\centering
	\includegraphics[width=\linewidth]{figures/pipeline_net}
	\caption{Illustration of our layered human-object rendering approach, which not only includes a direction-aware neural texture blending scheme to encode the occlusion information explicitly but also adopts a spatial-temporal texture completion for the occluded regions based on the human motion priors.}
	\vspace{-10pt}
	\label{fig:pipeline_net}
\end{figure*}

For the detail of the parametric model estimation, we fit the parametric human model, SMPL \citep{SMPL2015}, to capture occluded human with the predicted 2D keypoints.
%
Specifically, we use Openpose \citep{Openpose} as our joint detector to estimate 2D human keypoints from sparse-view RGB inputs.
%
To estimate the pose/shape parameters of SMPL as our human prior for occluded human, we formulate the energy function $\boldsymbol{E}_{\mathrm{prior}}$ of this optimization as:
\begin{align} \label{eq:opt}
	\boldsymbol{E}_{\mathrm{prior}}(\boldsymbol{\theta}_t, \boldsymbol{\beta}) = \boldsymbol{E}_{\mathrm{2D}} + \lambda_{\mathrm{T}}\boldsymbol{E}_{\mathrm{T}}
\end{align}
% 
Here, $\boldsymbol{E}_{\mathrm{2D}}$ represents the re-projection constraint on 2D keypoints detected from sparse-view RGB inputs, while $\boldsymbol{E}_{\mathrm{T}}$ enforces the final pose and shape to be temporally smooth.
%
$\boldsymbol{\theta}_t$ is the pose parameters of frame $t$, while $\boldsymbol{\beta}$ is the shape parameters.
%
Note that this temporal smoothing enables globally consistent capture during the whole sequence, and benefits the parametric model estimation when some part of the body is gradually occluded.
%
We follow \cite{he2021challencap} to formulate the 2D term $\boldsymbol{E}_{\mathrm{2D}}$ and the temporal term $\boldsymbol{E}_{\mathrm{T}}$ under the sparse-view setting.
%

Moreover, we apply an occlusion-aware data augmentation to reduce the domain gap between our training set and the challenging human-object interaction testing set.
%
Specially, we randomly sample some objects from ShapeNet dataset~\cite{chang2015shapenet}.
%
We then randomly rotate and place them around human before training, as shown in Fig. \ref{fig:DataAugmentation}.
%
By simulating the occlusion of human-object interaction, our network is more robust to occluded human features.

With both the pixel-aligned image features and the statistical human motion priors under this occlusion-aware data augmentation training, our implicit function generates high-quality human meshes with only spare RGB inputs and occlusions from human-object interaction.

\noindent{\textbf{(b) Human-aware Object Tracking.}}
%
For the objects around the human, people recover them from depth maps~\cite{new2011kinect}, implicit fields~\cite{mescheder2019occupancy}, or semantic parts~\cite{chen2018autosweep}. We perform a template-based object alignment for the first frame and human-aware tracking to maintain temporal consistency and prevent the segmentation uncertainty caused by interaction. With the inspiration of PHOSA \cite{2020phosa_Arrangements}, we consider each object as a rigid body mesh.


To faithfully and robustly capture object in 3D space as time going, we introduce a human-aware tracking method.
%
Expressly, we assume objects are rigid bodies and transforming rigidly in the human-object interaction activities.
%
So the object mesh $O_{t}$ at frame $t$ can be represented as: $O_{t} = R_{t}O_{t-1}+T_{t}$.
% 
Based on the soft rasterization rendering~\cite{ravi2020accelerating}, the rotation $R_{t}$ and the translation $T_{t}$ can be naively optimized by comparing $\mathcal{L}_{2}$ norm between the rendered silhouette $S_{t}^{i}$ and object mask $\mathcal{M}o_{t}^{i}$.
%

Human is also an important cue to locate the object position.
%
From the 2D perspective, when objects are occluded by the human at a camera view, the  $\mathcal{L}_{2}$ loss between rendered silhouette and occluded mask will lead to the wrong object location due to the wrong guidelines at the occluded area.
%
So we remove the occluded area affected by human mask $\mathcal{M}h_{t}^{i}$ when computing the $\mathcal{L}_{2}$ loss.
%
From the 3D perspective, human can not interpenetrate an rigid object, so we also add an interpenetration loss $\mathcal{L}_{P}$~\cite{jiang2020mpshape} to regularize optimization. Our total object tracking loss is:
\begin{align}
	\mathcal{L}_{track} = \lambda_{1}\sum_{i=0}^{n}\| \mathcal{B}(\mathcal{M}h_{t}^{i}==0) \odot  S_{t}^{i} - \mathcal{M}o_{t}^{i}  \|  + \lambda_{2}\mathcal{L}_{P},
\end{align}
where n denotes view numbers, $\lambda_{1}$ denotes weight of silhouette loss, $\lambda_{2}$ denotes weight of interpenetration loss, $\mathcal{B}$ represents an binary operation, it returns 0 when the condition is true, else 1.
% 	}

Our implicit human-object capture utilizes both the pixel-aligned image features and global human motion priors with the aid of an occlusion-aware training data augmentation, and captures objects with the template-based alignment and the human-aware tracking to maintain temporal consistency and prevent the segmentation uncertainty caused by interaction. Thus, our approaches can generate high-quality human-object geometry with sparse inputs and occlusions.

\subsection{Layered Human-Object Rendering}\label{sec:rendering}
We introduce a neural human-object rendering pipeline to encode local fine-detailed human geometry and texture features from adjacent input views, so as to produce photo-realistic layered output in the target view, as illustrated in Fig. \ref{fig:pipeline_net}.

\begin{figure*}[t]
	\centering
	\includegraphics[width=\linewidth]{figures/gallery}
	\vspace{-20pt}
	\caption{The geometry and texture results of our proposed approach, which generates photo-realistic rendering results and high fidelity geometry on a various of sequences, such as rolling a box, playing with balls.}
    \vspace{-10pt}
	\label{fig:gallery}
\end{figure*}

\noindent{\textbf{(c) Direction-aware Neural Texture Blending.}} \label{sec:neuralBlending}
%
While traditional image-based rendering approaches always show the artifacts with the sparse-view texture blending, we follow \cite{NeuralHumanFVV2021CVPR} to propose a direction-aware neural texture blending approach to render photo-realistic human in the novel view.
% %
For a novel view image $I_{n}$, the linear combination of two source view $I_{1}$ and $I_{2}$ with blending weight map $W$ is formulate as:
\begin{align}
	I_{n} = W \cdot I_{1} + (1 - W) \cdot I_{2}.
\end{align}
However, in the sparse-view setting, the neural blending approach \citep{NeuralHumanFVV2021CVPR} can still generate unsmooth results. As the reason of these artifacts, the imbalance of angles between two source views with a novel view will lead to the imbalance wrapped image quality.
%

Different from \citet{NeuralHumanFVV2021CVPR}, we thus propose a direction-aware neural texture blending to eliminate such artifacts, as shown in Fig. \ref{fig:pipeline_net}.
%
The direction and angle between the two source views and target view will be an important cue for neural rendering quality, especially under occluded scenarios. 
%
Given novel view depth $D_{n}$ and source view depth $D_{1}$, $D_{2}$, we wrap them to the novel view $D_{1}^{n}$ and $D_{2}^{n}$, then compute the occlusion map $O_{i} = D_{n}- D_{i}^{n} (i=1,2)$.
%
Then, we unproject $D_{i}$ to point-clouds.
%
For each point $P$, we compute the cosine value between $\overrightarrow{c_{i}P}$ and $\overrightarrow{c_{n}P}$ to get angle map $A_{i}$, where $c_{i}$ denotes the optical center of source camera $i$, $c_{n}$ denotes the optical center of novel view camera.
%
Thus, we introduce a direction-aware blending network $\Theta_{DAN}$ to utilize global feature from image and local feature from human geometry to generate the blending weight map $W$, which can be formulated as:
\begin{align}
	W = \Theta_{DAN}(I_{1},O_{1},A_{1},I_{2},O_{2},A_{2}),
\end{align}
% 

\noindent{\textbf{(d) Spatial-temporal Texture Completion.}}
%
While human-object interaction activities consistently lead to occlusion, the missing texture on human, therefore, leads to severe artifacts for free-viewpoint rendering.
%
To end this, we propose a spatial-temporal texture completion method to generate a texture-completed proxy in the canonical human space.
%
We use the temporal and spatial information to complete the missing texture at view $i$ and time $t$ from different times and different views.

Specifically, we first use the non-rigid deformation to register an up-sampled SMPL model (41330 vertices) with the captured human meshes.
%	
Then, for each point on the proxy, we find the nearest visible points along with all views and all frames, then assign an interpolation color to this point.
% 
We thus generate a canonical human space with the fused texture.
%
For the occluded part of human in novel view, we render the texture-completed image and blend it with the neural rendering results in Sec. \ref{sec:neuralBlending} (c).

We utilize a layered human-object rendering strategy to render human-object together with the reconstruction and tracking of object.
%
For each frame, we render human with our novel neural texture blending while rendering objects through a classical graphics pipeline with color correction matrix (CCM).
%
To combine human and object rendering results, we utilize the depth buffer from the geometry of our human-object capture.
%

\noindent{\textbf{Training Strategy.}} To enable our sparse-view neural human performance rendering under human-object interaction, we need to train the direction-aware blending network $\Theta_{DAN}$ properly.
% 

We follow \citet{NeuralHumanFVV2021CVPR} to utilize 1457 scans from the Twindom dataset \cite{Twindom} to train our DAN $\Theta_{DAN}$ properly.
%
Differently, we randomly place the performers inside the virtual camera views and augment this dataset by randomly placing some objects from ShapeNet dataset~\cite{chang2015shapenet}.
%
By simulating the occlusion of human-object interaction, we make our network more robust to occluded human.
%
Our training dataset contains RGB images, depth maps and angle maps for all the views and models.

For the training of our direction-aware blending network $\Theta_{DAN}$, we set out to apply the following learning scheme to enable more robust blending weight learning.
%
The appearance loss function with the perceptual term ~\cite{Johnson2016Perceptual} is to make the blended texture as close as possible to the ground truth, which is formulated as:
\begin{align}
	\left.\mathcal{L}_{r g b}=\frac{1}{n} \sum_{j}^{n}
	\left(
	\left\|I_{r}^{j}-I_{g t}^{j}\right\|_{2}^{2}
% 	\right)
	+\left\|\varphi
	\left(
	I_{r}^{j}
	\right)
	-\varphi
	\left(
	I_{g t}^{j}
	\right)
	\right\|_{2}^{2}
	\right) \right.
\end{align}
where $I_{g t}$ is the ground truth RGB images; $\varphi(\cdot)$ denotes the output features of the third-layer of pre-trained VGG-19.

With the aid of such occlusion analysis, our texturing scheme maps the input adjacent images into a photo-realistic texture output of human-object activities in the target view through efficient blending weight learning, without requiring further per-scene training.


\section{Experiments}
\label{sec:experiments}
\subsection{Experimental Setup}
We evaluate the effectiveness of our approach on three different domain adaptation datasets: DomainNet~\cite{peng2019moment}, Office-Home~\cite{Venkateswara2017DeepHN} and Office31~\cite{Saenko2010AdaptingVC}. DomainNet ~\cite{peng2019moment} is a large-scale domain adaptation dataset with 345 classes across 6 domains. Following MME ~\cite{Saito2019SemiSupervisedDA}, we use a subset of the dataset containing 126 categories across four domains: Real(R), Clipart(C), Sketch(S), and Painting(P). The performance on DomainNet is evaluated using 7 different combinations out of possible 12 combinations. Office-Home~\cite{Venkateswara2017DeepHN} is another widely used domain adaptation benchmark dataset with 65 classes across four domains: Art(Ar), Product(Pr), Clipart(Cl), and Real (Rl). We perform experiments on all possible combinations of 4 domains. Office31~\cite{Saenko2010AdaptingVC} is a relatively smaller dataset containing just 31 categories of data across three domains- Amazon(A), Dslr(D), Webcam(W). Following prior work ~\cite{Saito2019SemiSupervisedDA, Kim2020AttractPA}, we evaluate our approach on two combinations for the office31 dataset. 

For the fair comparison, we use the data-splits (train, validation, and test splits) released by ~\cite{Saito2019SemiSupervisedDA} on Github \footnote{\url{https://github.com/VisionLearningGroup/SSDA_MME}}. We use the same settings for the benchmark datasets as in the prior work ~\cite{Saito2019SemiSupervisedDA, Kim2020AttractPA}, including the number of labeled samples in the target domain, which are consistent across all experiments.

\subsection{Implementation Details}
Similar to the previous works on SSDA ~\cite{Saito2019SemiSupervisedDA, Kim2020AttractPA, Li2020OnlineMF}, we use Resnet34 and Alexnet as the backbone networks in our paper. We only used VGG for Office31 due to its higher memory requirements. The feature generator model is initialized with ImageNet weights, and the classifier is randomly initialized and has the same architecture as in ~\cite{Saito2019SemiSupervisedDA, Kim2020AttractPA, Li2020OnlineMF}. All our experiments are performed using Pytorch ~\cite{Paszke2019PyTorchAI}.We use an identical set of hyperparameters ($\alpha= 4$, $\beta= 1$ ) across all our experiments other than minibatch size. All the hyperparameters values are decided using validation performance on Product to Art experiments on the Office-Home dataset. We have set $\tau=5$ in our experiments. Each minibatch of size $B$  contains an equal number of source and labeled target examples, while the number of unlabeled target samples is $\mu \times B$. We study the effect of $\mu$ in section \ref{ablation}. Resnet34 experiments are performed with minibatch size, $B= 32$ and Alexnet models are trained with $B= 24$. We use $\mu=4$ for all our experiments. We use SGD optimizer with a momentum of $0.9$ and an initial learning rate of $0.01$ with cosine learning rate decay for all our experiments. Weight decay is set to $0.0005$ for all our models. Other details of the experiments are included in the Appendix.
\subsection{Baselines}
We compare our CLDA framework with previous state-of-the-art SSDA approaches : \textbf{MME} ~\cite{Saito2019SemiSupervisedDA}, \textbf{APE} ~\cite{Kim2020AttractPA}, \textbf{BiAT} ~\cite{Jiang2020BidirectionalAT} , \textbf{UODA} ~\cite{Qin2020Contradictory}, ~\textbf{Meta-MME} ~\cite{Li2020OnlineMF} and ~\textbf{ENT} ~\cite{Grandvalet2004SemisupervisedLB} using the performance reported by these papers. papers. We also included the results from adversarial based baseline methods:
% DANN [12] ,ADR
% 212 [43] and CDANWe also include the results from UDA (Unsupervised Domain Adaptation) based baseline methods using the adversarial approach for the Semi-Supervised Domain Adaptation task: 
\textbf{DANN} ~\cite{Ganin2016DomainAdversarialTO}, \textbf{ADR} ~\cite{Saito2018AdversarialDR} and \textbf{CDAN} ~\cite{Long2018ConditionalAD} as reported in \cite{Saito2019SemiSupervisedDA}. We also provide the \textbf{S+T} results where the model is trained using all the labeled samples across domains.
\begin{table}[!t]
\caption{ \textbf{Performance Comparison in Office-Home.} Numbers show top-1 accuracy values for different domain adaptation scenarios under 3-shot setting using Alexnet and Resnet34 as backbone networks. We have highlighted the best method for each transfer task. CLDA surpasses all the baseline methods in most adaptation scenarios. Our Proposed framework achieves the best average performance among all compared methods.
}
\renewcommand{\arraystretch}{1.2}
\vspace{2mm}
\centering
\label{base_office_home_table}
\resizebox{\columnwidth}{!}{
\begin{tabular}{c|c|cccccccccccc|c}
\specialrule{.1em}{.05em}{.05em}
Net & Method & Rl$\rightarrow$Cl & Rl$\rightarrow$Pr & Rl$\rightarrow$Ar & Pr$\rightarrow$Rl & Pr$\rightarrow$Cl & Pr$\rightarrow$Ar & Ar$\rightarrow$Pl & Ar$\rightarrow$Cl & Ar$\rightarrow$Rl & Cl$\rightarrow$Rl & Cl$\rightarrow$Ar & Cl$\rightarrow$Pr & Mean \\
\hline
\multirow{8}{*}{Alexnet} & S+T & 44.6 & 66.7 & 47.7 & 57.8 & 44.4 & 36.1 & 57.6 & 38.8 & 57.0 & 54.3 & 37.5 & 57.9 & 50.0 \\
 & DANN & 47.2 & 66.7 & 46.6 & 58.1 & 44.4 & 36.1 & 57.2 & 39.8 & 56.6 & 54.3 & 38.6 & 57.9 & 50.3 \\
 & ADR  & 37.8 & 63.5 & 45.4 & 53.5 & 32.5 & 32.2 & 49.5 & 31.8 & 53.4 & 49.7 & 34.2 & 50.4 & 44.5 \\
& CDAN & 36.1 & 62.3 & 42.2 & 52.7 & 28.0 & 27.8 & 48.7 & 28.0 & 51.3 & 41.0 & 26.8 & 49.9 & 41.2 \\
 & ENT & 44.9 & 70.4 & 47.1 & 60.3 & 41.2 & 34.6 & 60.7 & 37.8 & 60.5 & 58.0 & 31.8 & 63.4 & 50.9 \\
 & MME & 51.2 & 73.0 & 50.3 & 61.6 & 47.2 & 40.7 & 63.9 & 43.8 & 61.4 & 59.9 & 44.7 & 64.7 & 55.2 \\
 & Meta-MME & 50.3 & - & - & - & 48.3 & 40.3 & - & 44.5 & - & - & 44.5 & - & - \\
 & BiAT & - & - & - & - & - & - & - & - & - & - & - & - & 56.4 \\
 & APE & \textbf{51.9} & \textbf{74.6} & 51.2 & 61.6 & 47.9 & 42.1 & 65.5 & 44.5 & 60.9 & 58.1 & 44.3 & 64.8 & 55.6 \\
 & \textbf{CLDA}(ours) & 51.5 & 74.1 & \textbf{54.3} & \textbf{67.0} & \textbf{47.9} & \textbf{47.0} & \textbf{65.8} & \textbf{47.4} & \textbf{66.6} & \textbf{64.1} & \textbf{46.8} & \textbf{67.5} & \textbf{58.3} \\
\hline
\hline
\multirow{7}{*}{Resnet34} & S+T & 55.7 & 80.8 & 67.8 & 73.1 & 53.8 & 63.5 & 73.1 & 54.0 & 74.2 & 68.3 & 57.6 & 72.3 & 66.2 \\
 & DANN & 57.3 & 75.5 & 65.2 & 69.2 & 51.8 & 56.6 & 68.3 & 54.7 & 73.8 & 67.1 & 55.1 & 67.5 & 63.5 \\
 & ENT & 62.6 & 85.7 & 70.2 & 79.9 & 60.5 & 63.9 & 79.5 & 61.3 & 79.1 & 76.4 & 64.7 & 79.1 & 71.9 \\
 & MME & 64.6 & 85.5 & 71.3 & 80.1 & 64.6 & 65.5 & 79.0 & 63.6 & 79.7 & 76.6 & 67.2 & 79.3 & 73.1 \\
 & Meta-MME & 65.2 & - & - & - & 64.5 & 66.7 & - & 63.3 & - & - & 67.5 & - & - \\
 & APE & \textbf{66.4} & 86.2 & 73.4 & 82.0 & \textbf{65.2} & 66.1 & 81.1 & \textbf{63.9} & 80.2 & 76.8 & 66.6 & 79.9 & 74.0 \\
 & \textbf{CLDA} (ours) & 66.0 & \textbf{87.6} & \textbf{76.7} & \textbf{82.2} & 63.9 & \textbf{72.4} & \textbf{81.4} & 63.4 & \textbf{81.3} & \textbf{80.3} & \textbf{70.5} & \textbf{80.9} & \textbf{75.5} \\
\specialrule{.1em}{.05em}{.05em}
\end{tabular}}

\vspace{2mm}
\end{table}
% %------------------------------------------------------------------------- 
% %-------------------------------------------------------------------------
\subsection{Results}
Table  ~\ref{base_office_home_table}- ~\ref{base_office_table} show  top-1 accuracies  and mean accuracies for different combination of domain adaptation scenarios for all three datasets in comparison with baseline SSDA methods.

\noindent\textbf{Office-Home.} Table ~\ref{base_office_home_table} contains the results of the Office-Home dataset for 3-shot setting with Alexnet and Resnet34 as backbone networks. Results for the $1$-shot adaptation scenarios are included in the Appendix ~\ref{office_home_1_shot}. 
Our method consistently performs better than the baseline approaches and achieves $58.3\%$  and $75.5\%$ mean accuracy with Alexnet and Resnet34, respectively. Our approach surpasses the state-of-the-art SSDA approaches in most of the adaptation tasks. In some domain adaptation cases, such as Cl $\rightarrow$ Rl, Rl $\rightarrow$ Ar and Pr $\rightarrow$ Ar, we exceeded APE by more than $3\%$.

\noindent\textbf{DomainNet}: Our CLDA approach surpasses the performance of existing SSDA baselines as shown in Table ~\ref{base_domainNet_table}. Using Alexnet backbone, our method improves over BiAT by $5.2\%$ and $4.9\%$ in 1-shot and 3-shot settings, respectively. We obtain similarly improved performance when we switch the neural backbone from Alexnet to Resnet34. With Resnet34 as the backbone, we gain $4.3\%$ and $3.6\%$ over APE in 1-shot and 3-shot settings, respectively. Similar to the Office-Home, our approach surpasses the well-known domain adaptation benchmarks methods in most domain adaptation tasks of the DomainNet dataset. Such consistent improved performance shows that our approach reduces both inter and intra domain discrepancy prevalent in SSDA. 

\noindent\textbf{Office31}: Similar to other datasets, our proposed method with Alexnet and VGG as neural backbone achieves the best performance in both domain adaption scenarios for office31 as shown in Table ~\ref{base_office_table}. Using Alexnet backbone, we beat the APE ~\cite{Kim2020AttractPA} by $3.2\%$ in 3-shot and BiAT by $7.3\%$ in 1-shot settings. We observe similar gains over all the baselines methods with VGG as the neural network backbone. This shows the efficacy of our proposed approach irrespective of the used backbone.







\begin{table}[!t]

\caption{ \textbf{Performance Comparison in DomainNet.} Numbers show Top-1 accuracy values for different domain adaptation scenarios under 1-shot and 3-shot settings using Alexnet and Resnet34 as backbone networks. CLDA achieves better performance than all the baseline methods in most of the domain adaptation tasks. We have highlighted the best approach for each domain adaptation task. Our Proposed framework achieves the best average performance among all compared methods.
}
\renewcommand{\arraystretch}{1.2}
\vspace{2mm}
\label{base_domainNet_table}
\begin{center}{
\resizebox{\columnwidth}{!}{
\begin{tabular}{c|c|cccccccccccccc|cc}
\specialrule{.1em}{.05em}{.05em}
\multirow{2}{*}{Net} & \multirow{2}{*}{Method} & \multicolumn{2}{c}{R$\rightarrow$C} & \multicolumn{2}{c}{R$\rightarrow$P} & \multicolumn{2}{c}{P$\rightarrow$C} & \multicolumn{2}{c}{C$\rightarrow$S} & \multicolumn{2}{c}{S$\rightarrow$P} & \multicolumn{2}{c}{R$\rightarrow$S} & \multicolumn{2}{c|}{P$\rightarrow$R} & \multicolumn{2}{c}{Mean} \\
 & & 1-shot & 3-shot & 1-shot & 3-shot & 1-shot & 3-shot & 1-shot & 3-shot & 1-shot & 3-shot & 1-shot & 3-shot & 1-shot & 3-shot & 1-shot & 3-shot \\ \hline
\multirow{8}{*}{Alexnet} & S+T & 43.3 & 47.1 & 42.4 & 45.0 & 40.1 & 44.9 & 33.6 & 36.4 & 35.7 & 38.4 & 29.1 & 33.3 & 55.8 & 58.7 & 40.0 & 43.4 \\
 & DANN & 43.3 & 46.1 & 41.6 & 43.8 & 39.1 & 41.0 & 35.9 & 36.5 & 36.9 & 38.9 & 32.5 & 33.4 & 53.5 & 57.3 & 40.4 & 42.4 \\
 & ADR      & 43.1 & 46.2 & 41.4 & 44.4 & 39.3 & 43.6 & 32.8 & 36.4 & 33.1 & 38.9 & 29.1 & 32.4 & 55.9 & 57.3 & 39.2 & 42.7 \\
 & CDAN     & 46.3 & 46.8 & 45.7 & 45.0 & 38.3 & 42.3 & 27.5 & 29.5 & 30.2 & 33.7 & 28.8 & 31.3 & 56.7 & 58.7 & 39.1 & 41.0 \\
 & ENT & 37.0 & 45.5 & 35.6 & 42.6 & 26.8 & 40.4 & 18.9 & 31.1 & 15.1 & 29.6 & 18.0 & 29.6 & 52.2 & 60.0 & 29.1 & 39.8 \\
 & MME & 48.9 & 55.6 & 48.0 & 49.0 & 46.7 & 51.7 & 36.3 & 39.4 & 39.4 & 43.0 & 33.3 & 37.9 & 56.8 & 60.7 & 44.2 & 48.2 \\
 & Meta-MME & - & 56.4 & - & 50.2 & & 51.9 & - & 39.6 & - & 43.7 & - & 38.7 & - & 60.7 & - & 48.8 \\
 & BiAT & 54.2 & 58.6 & 49.2 & 50.6 & 44.0 & 52.0 & 37.7 & 41.9 & 39.6 & 42.1 & 37.2 & 42.0 & 56.9 & 58.8 & 45.5 & 49.4 \\
 & APE & 47.7 & 54.6 & 49.0 & 50.5 & 46.9 & 52.1 & 38.5 & 42.6 & 38.5 & 42.2 & 33.8 & 38.7 & 57.5 & 61.4 & 44.6 & 48.9 \\
 & \textbf{CLDA} (ours) & \textbf{56.3} & \textbf{59.9} & \textbf{56.0} & \textbf{57.2} & \textbf{50.8} & \textbf{54.6} & \textbf{42.5} & \textbf{47.3} & \textbf{46.8} & \textbf{51.4} & \textbf{38.0} & \textbf{42.7} & \textbf{64.4} & \textbf{67.0} & \textbf{50.7} & \textbf{54.3} \\ 
\hline
\hline
\multirow{9}{*}{Resnet34} & S+T & 55.6 & 60.0 & 60.6 & 62.2 & 56.8 & 59.4 & 50.8 & 55.0 & 56.0 & 59.5 & 46.3 & 50.1 & 71.8 & 73.9 & 56.9 & 60.0 \\
 & DANN & 58.2 & 59.8 & 61.4 & 62.8 & 56.3 & 59.6 & 52.8 & 55.4 & 57.4 & 59.9 & 52.2 & 54.9 & 70.3 & 72.2 & 58.4 & 60.7 \\
  & ADR      & 57.1 & 60.7 & 61.3 & 61.9 & 57.0 & 60.7 & 51.0 & 54.4 & 56.0 & 59.9 & 49.0 & 51.1 & 72.0 & 74.2 & 57.6 & 60.4 \\
 & CDAN     & 65.0 & 69.0 & 64.9 & 67.3 & 63.7 & 68.4 & 53.1 & 57.8 & 63.4 & 65.3 & 54.5 & 59.0 & 73.2 & 78.5 & 62.5 & 66.5 \\
 & ENT & 65.2 & 71.0 & 65.9 & 69.2 & 65.4 & 71.1 & 54.6 & 60.0 & 59.7 & 62.1 & 52.1 & 61.1 & 75.0 & 78.6 & 62.6 & 67.6 \\
 & MME & 70.0 & 72.2 & 67.7 & 69.7 & 69.0 & 71.7 & 56.3 & 61.8 & 64.8 & 66.8 & 61.0 & 61.9 & 76.1 & 78.5 & 66.4 & 68.9 \\
 & UODA & 72.7 & 75.4 & 70.3 & 71.5 & 69.8 & 73.2 & 60.5 & 64.1 & 66.4 & 69.4 & 62.7 & 64.2 & 77.3 & 80.8 & 68.5 & 71.2 \\
 & Meta-MME & - & 73.5 & - & 70.3 & - & 72.8 & - & 62.8 & - & 68.0 & - & 63.8 & - & 79.2 & - & 70.1 \\
 & BiAT & 73.0 & 74.9 & 68.0 & 68.8 & 71.6 & 74.6 & 57.9 & 61.5 & 63.9 & 67.5 & 58.5 & 62.1 & 77.0 & 78.6 & 67.1 & 69.7 \\
 & APE & 70.4 & 76.6 & 70.8 & 72.1 & \textbf{72.9} & \textbf{76.7} & 56.7 & 63.1 & 64.5 & 66.1 & 63.0 & 67.8 & 76.6 & 79.4 & 67.6 & 71.7 \\
 & \textbf{CLDA} (ours) & \textbf{76.1} & \textbf{77.7} & \textbf{75.1} & \textbf{75.7} & 71.0 & 76.4 & \textbf{63.7} & \textbf{69.7} & \textbf{70.2} & \textbf{73.7} & \textbf{67.1} & \textbf{71.1} & \textbf{80.1} & \textbf{82.9} & \textbf{71.9} & \textbf{75.3} \\ 
 \specialrule{.1em}{.05em}{.05em}
 
\end{tabular}}}
\end{center}
\end{table}
%------------------------------------------------------------------------- 

% %-------------------------------------------------------------------------

\begin{table}[t]
\centering
\caption{ \textbf{Performance Comparison in Office31.} Numbers show Top-1 accuracy values for different domain adaptation scenarios under 1-shot and 3-shot settings using Alexnet and VGG as backbone networks. CLDA outperforms all the baseline approaches in both scenarios. We have highlighted the superior method on each domain adaptation task. Our Proposed framework achieves the best mean accuracy among all baseline methods.
}
\renewcommand{\arraystretch}{1.2}
\label{base_office_table}
\begin{center}{
\resizebox{\columnwidth}{!}{
\begin{tabular}{c|cc|cc|cc||cc|cc|cc}
\specialrule{.1em}{.05em}{.05em}
\multicolumn{7}{c}{Alexnet} & \multicolumn{6}{c}{VGG} \\
\hline
 \multirow{3}{*}{Method} & \multicolumn{2}{c}{W$\rightarrow$A} & \multicolumn{2}{c|}{D$\rightarrow$A} & \multicolumn{2}{c}{Mean} & \multicolumn{2}{c}{W$\rightarrow$A} & \multicolumn{2}{c|}{D$\rightarrow$A} & \multicolumn{2}{c}{Mean}\\
 & 1-shot & 3-shot & 1-shot & 3-shot & 1-shot & 3-shot & 1-shot & 3-shot & 1-shot & 3-shot & 1-shot & 3-shot \\ \hline
 S+T & 50.4 & 61.2 & 50.0 & 62.4 & 50.2 & 61.8 &169.2 &73.2 &68.2 &73.3 &68.7 &73.25 \\
 DANN & 57.0 & 64.4 & 54.5 & 65.2 & 55.8 & 64.8 &69.3 &75.4 &70.4 &74.6 &69.85 &75.0\\
 ADR & 50.2 & 61.2 & 50.9 & 61.4 & 50.6 & 61.3 &69.7 &73.3 &69.2 &74.1 &69.45 &73.7\\
 CDAN & 50.4 & 60.3 & 48.5 & 61.4 & 49.5 & 60.8 &65.9 &74.4 &64.4 &71.4 &65.15 &72.9\\
 ENT & 50.7 & 64.0 & 50.0 & 66.2 & 50.4 & 65.1 &69.1 &75.4 &72.1 &75.1 &70.6 &75.25\\
 MME & 57.2 & 67.3 & 55.8 & 67.8 & 56.5 & 67.6 &73.1 &76.3 &73.6 &\textbf{77.6} &73.35 &76.95\\
BiAT & 57.9 & 68.2 & 54.6 & 68.5 & 56.3 & 68.4 &- &- &- &- &- &- \\
 APE & - & 67.6 & - & 69.0 & - & 68.3 &- &- &- &- &- &-\\
 CLDA & \textbf{64.6} & \textbf{70.5} & \textbf{62.7} & \textbf{72.5} & \textbf{63.6} & \textbf{71.5} &\textbf{76.2} &\textbf{78.6} &\textbf{75.1} &76.7 &\textbf{75.6} &\textbf{77.6} \\
\specialrule{.1em}{.05em}{.05em}
\end{tabular}}}
% \vspace{-2.0mm}
\end{center}
\end{table}


\begin{table}[t]
\small
\begin{center}
\begin{tabular}{ccccc}
\shline
\multirow{2}{*}{{Method}} & {CIFAR-10}&{CIFAR-100} \\
& Acc $\uparrow$(Forget $\downarrow$) & Acc $\uparrow$(Forget $\downarrow$) \\ 
\midrule
baseline & 46.4\std{$\pm$1.2}(36.0\std{$\pm$}2.1) & 18.8\std{$\pm$0.8}(18.5\std{$\pm$}0.7) \\
w/o \methodname & 53.1\std{$\pm$1.4}(24.7\std{$\pm$2.0}) & 19.3\std{$\pm$0.7}(15.9\std{$\pm$0.9}) \\
w/o \dataaugname & 52.0\std{$\pm$1.5}(34.6\std{$\pm$2.4}) & 21.5\std{$\pm$0.5}(16.3\std{$\pm$0.8}) \\ 
\hline
w/o $\mathcal{L}^{\mathrm{new}}_{\mathrm{pro}}$ & 54.8\std{$\pm$1.2}(\textbf{22.1}\std{$\pm$3.0}) & 19.6\std{$\pm$0.8}(19.9\std{$\pm$0.7}) \\
w/o $\mathcal{L}^{\mathrm{seen}}_{\mathrm{pro}}$ & 55.7\std{$\pm$1.4}(25.5\std{$\pm$1.5}) & 20.1\std{$\pm$0.4}(16.2\std{$\pm$0.6}) \\ 
$\mathcal{L}^{\mathrm{seen}}_{\mathrm{pro}}$ w/o $\mathcal{C}^\mathrm{new}$ & 56.2\std{$\pm$1.2}(26.4\std{$\pm$2.3}) & 20.8\std{$\pm$0.6}(17.9\std{$\pm$0.7}) \\ 
\hline
{\frameworkName} (\textbf{ours}) & \textbf{57.8}\std{$\pm$1.1}(23.2\std{$\pm$1.3}) & \textbf{22.7}\std{$\pm$0.7}(\textbf{15.0}\std{$\pm$0.8}) \\ 
\shline 
\end{tabular}
\end{center}
\caption{Ablation studies on CIFAR-10 ($M=0.1k$) and CIFAR-100 ($M=0.5k$). 
``baseline'' means $\mathcal{L}_\mathrm{INS}+\mathcal{L}_\mathrm{CE}$.
``$\mathcal{L}^{\mathrm{seen}}_{\mathrm{pro}}$ w/o $\mathcal{C}^\mathrm{new}$'' means $\mathcal{L}^{\mathrm{seen}}_{\mathrm{pro}}$ do not consider new classes in current task.
}
\label{tab:ablation}
\end{table}

\section{Conclusion}
In this work, we present a novel single-stage contrastive learning framework for semi-supervised domain adaptation. The framework consists of Inter-Domain Contrastive Alignment and Instance-Contrastive Alignment, where the former maximizes the similarity between centroids of the same class from both domains and later maximizes the similarity between augmented views of the unlabeled target images. We show that both of the components of the framework are necessary for improved performance. We demonstrate the effectiveness of our approach on three standard domain adaptation benchmark datasets, outperforming the well-known SSDA methods.

\section{Acknowledgments and Disclosure of Funding}
The work is supported by Half-Time Research Assistantship (HTRA) grants from the Ministry of Education, India. We would also like to thank Saurav Chakraborty and  Athira Nambiar for their valuable suggestions and feedback to improve the work.
% \newpage 
\printbibliography
\nocite{*}

\newpage
\appendix




Here we prove Lemma \ref{lem:cotapower} that gives a large deviation result for sums of independent heavy tailed random variables. The lemma is a more precise version of the results in \cite{Omelchenko2019} (although here we treat only with positive random variables which are bounded away from zero), and the proof follows closely what was done there.

\begin{lemma}
Let $S_m=\sum_{i=1}^m Z_i$ where $\{Z_i\}_{i\in\NN}$ is a sequence of i.i.d. absolutely continuous random variables taking values in $[1,\infty)$ such that there are $V,\gamma>0$ with
\[1-F_Z(x)=\PP(Z_i\geq x)\leq Vx^{-\gamma}\] 
for all $x>0$. Then there are $\auxc,\auxl>0$ depending on $V$, $\gamma$ and $\EE(Z_1)$ (if it exists) alone such that:
\begin{itemize}
    \item If $\gamma<1$, then for all $L>\auxl$,
    $\displaystyle\PP(S_m\geq Lm^{\frac{1}{\gamma}})\leq \auxc L^{-\gamma}$.
    \item If $\gamma=1$, then for all $L>\auxl$, $\displaystyle\PP(S_m\geq Lm\log(m))\leq \left(\frac{\auxc}{L\log(m)}\right)^{1-\frac{\auxl}{L}}$.
    \item If $\gamma>1$, then for all $L>\auxl$,
    $\displaystyle\PP(S_m\geq Lm)\,\leq\,\auxc L^{-\gamma}m^{-((\gamma-1)\wedge\frac{\gamma}{2})}$.
\end{itemize}
\end{lemma}
\begin{remark}
We remark that tighter results can be obtained for $\gamma \ge 1$ below, but the current result is sufficient for our purposes.
\end{remark}
\begin{proof}
We proceed as in \cite{Omelchenko2019} by assuming first that $\gamma\leq 1$. Fix $x>0$ and define the event $B=\{\forall 1\leq i< m,\,Z_i\leq x\}$ so that 
\[
    \PP(S_m\geq x)\;\leq\;\PP(\overline{B})+\PP(S_m\geq x\,|\,B)\PP(B)\;\leq\;mVx^{-\gamma}+\PP(S^{(x)}_m\geq x)\PP(B)
\]
where $S^{(x)}$ is the sum of $m$ i.i.d. random variables with c.d.f. $\frac{F_Z(y)}{F_Z(x)}$ for any $y \in [0,x]$. We can thus use a Chernoff bound to deduce
\[
    \PP(S_m\geq x)\;\leq\;mVx^{-\gamma}+e^{-\lambda x}\EE\big(e^{\lambda S^{(x)}_m}\big)\PP(B)\;=\;mVx^{-\gamma}+e^{-\lambda x}\left(\int_1^xe^{\lambda y}dF_Z(y)\right)^m,
\]
where we used independence of the random variables to conclude that $\PP(B)=(F_Z(x))^m$. Define now $M=\frac{2\gamma}{\lambda}$ for some $\lambda$ to be chosen below so that $M<x$ holds. Hence,
\[R(\lambda,x)\,:=\,\int_1^xe^{\lambda y}dF_Z(y)\,=\,\int_1^Me^{\lambda y}dF_Z(y)\,+\,\int_M^xe^{\lambda y}dF_Z(y),\]
which we bound separately. First notice that for some constant $C_1=C_1(\gamma, V)$ we have
\begin{align*}
    \int_1^Me^{\lambda y}dF_Z(y)&\leq\,e^{\lambda M}F_Z(M)-\lambda\int_1^Me^{\lambda y}F_Z(y)dy\\[3pt]
    &\leq\,e^{\lambda M}F_Z(M)-e^{\lambda M}+1+\lambda\int_1^Me^{\lambda y}(1-F_Z(y))dy\\[3pt]
    &\leq\,e^{\lambda M}F_Z(M)-e^{\lambda M}+1+\lambda Ve^{\lambda M}\int_1^My^{-\gamma}dy\;\leq\,1+C_1Q(\lambda),
\end{align*}
where $Q(\lambda)=\lambda^{\gamma}$ if $\gamma<1$ and $Q(\lambda)=-\lambda\log(\frac{\lambda}{2\gamma})$ if $\gamma=1$. For the integral between $M$ and $x$ observe that
\begin{align*}
    \int_M^xe^{\lambda y}dF_Z(y)&\leq\,e^{\lambda M}(1-F_Z(M))+\lambda\int_M^xe^{\lambda y}(1-F_Z(y))dy\\[3pt]
    &\leq\,Ve^{\lambda M}M^{-\gamma}+\lambda V\int_M^xe^{\lambda y}y^{-\gamma}dy\\[3pt]
    &=\,Ve^{2\gamma}\left(\frac{\lambda}{2\gamma}\right)^{\gamma}+ Ve^{\lambda x}x^{-\gamma}\int_0^{\lambda(x-M)}e^{-w}\left(1-\frac{w}{\lambda x}\right)^{-\gamma}dw,
\end{align*}
where in the last line we used the change of variables $w=\lambda(x-y)$. Now, since $M = 2\gamma/\lambda$, the function $f(w)=e^{\frac{w}{2}}(1-\frac{w}{\lambda x})^{\gamma}$ has a positive derivative for $w \in [0, \lambda(x-M)]$. Hence for $w\in[0,\lambda(x-M)]$ we have  $(1-\frac{w}{\lambda x})^{-\gamma}\leq e^{w/2}$  and hence the last integral is therefore smaller than $\int_0^\infty e^{-w/2}dw=2$, giving
\[\int_M^xe^{\lambda y}dF_Z(y)\,\leq\,Ve^{2\gamma}\left(\frac{\lambda}{2\gamma}\right)^{\gamma}+ 2Ve^{\lambda x}x^{-\gamma}\,=\,C_2\lambda^{\gamma}+C_3e^{\lambda x}x^{-\gamma}\]
for some constants $C_2, C_3$ depending only on $V$ and $\gamma$. Putting together both bounds for $R(\lambda,x)$ we arrive at
\begin{align*}\PP(S_m\geq x)&\leq\;mVx^{-\gamma}+e^{-\lambda x}\left(1+C_1Q(\lambda)+C_2\lambda^{\gamma}+ C_3e^{\lambda x}x^{-\gamma}\right)^m\\[3pt]
&\leq\;mVx^{-\gamma}+\exp\left(-\lambda x+mC_1Q(\lambda)+mC_2\lambda^{\gamma}+ mC_3e^{\lambda x}x^{-\gamma}\right).
\end{align*}
Our aim at this point to choose $\lambda$ such that the term on the right is small, which is achieved when taking
\[\lambda=\frac{1}{x}\log\left(\frac{x^{\gamma}}{m}\right),\]
so that $me^{\lambda x}x^{-\gamma}=1$. Assume first that $\gamma<1$ so $x=Lm^{\frac{1}{\gamma}}$ for $L$ large, for which $\lambda=\frac{\gamma\log(L)}{Lm^{\frac{1}{\gamma}}}$ is small, while $\lambda x=\gamma\log(L)$ is large so the assumption $M<x$ is justified. Now, since $\gamma<1$, $Q(\lambda)=\lambda^{\gamma}$ and hence we have
\[-\lambda x+mC_1Q(\lambda)+mC_2\lambda^{\gamma}+ mC_3e^{\lambda x}x^{-\gamma}\,=\,-\gamma\log(L)+(C_1+C_2)\left(\frac{\gamma\log(L)}{L}\right)^{\gamma}+C_3,\]
and since we are assuming $L$ large, we have $(\frac{\gamma\log(L)}{L})^{\gamma}\leq 1$ which finally gives
\begin{align*}\PP(S_m\geq x)&\leq\;mVx^{-\gamma}+\exp\left(-\lambda x+mC_1Q(\lambda)+mC_2\lambda^{\gamma}+ mC_3e^{\lambda x}x^{-\gamma}\right)\\[3pt]
&\leq\;VL^{-\gamma}+\exp\left(-\gamma\log(L)+C_1+C_2+C_3\right)\;=\;\auxc L^{-\gamma},
\end{align*}
which proves the first point of the theorem. Suppose now that $\gamma=1$ so that $x=Lm\log(m)$ for $L\geq \auxl$ for some $\auxl$ large, and hence $\lambda=\frac{1}{Lm\log(m)}\log(L\log(m))$ is small, while $\lambda x=\log(L\log(m))$ is large, so again the assumption $M<x$ is justified. For this choice of $\gamma$ we have $mQ(\lambda)=-m\lambda\log(\frac{\lambda}{2})$ which we can bound as 
\begin{align*}
-m\lambda\log(\tfrac{\lambda}{2})&=\frac{\log(L\log(m))}{L\log(m)}\log\left(\frac{2Lm\log(m)}{\log(L\log(m))}\right)\\[3pt]&\leq\frac{(\log(2L\log(m))^2}{L\log(m)}+\frac{\log(L\log(m))}{L}\leq C_4+\frac{\log(L\log(m))}{L}
\end{align*}
for some constant $C_4$, and hence we arrive at
\begin{align*}\PP(S_m\geq x)&\leq\;mVx^{-1}+\exp\left(-\lambda x+mC_1Q(\lambda)+mC_2\lambda+ mC_3e^{\lambda x}x^{-1}\right)\\[3pt]
&\leq\;\frac{V}{L\log(m)}+C_5\exp\Big({-}\log(L\log(m))+\frac{C_1}{L}\log(L\log(m))\Big)\;\leq\;\Big(\frac{\auxc}{L\log(m)}\Big)^{1-\frac{\auxl}{L}}
\end{align*}
for some constant $C_5$, and where the last inequality holds by choosing $\auxl$ larger than $C_1$ and also by choosing $\auxc$ large enough.

Suppose now that $\gamma>1$ so that $E_0:=\EE(Z_1)$ exists. In this case we can perform a similar computation to the one before to deduce that 
\[
    \PP(S_m-mE_0\geq x)\;\leq\;mVx^{-\gamma}+e^{-\lambda x}\Big(e^{-\lambda E_0}\int_1^xe^{\lambda y}dF_Z(y)\Big)^m,
\]
and we can divide the integral $\int_1^xe^{\lambda y}dF_Z(y)$ as before so that
\[\int_1^xe^{\lambda y}dF_Z(y)\,=\,\int_1^Me^{\lambda y}dF_Z(y)+\int_M^xe^{\lambda y}dF_Z(y)\]
where again $M=\frac{2\gamma}{\lambda}$ (and for our choice of small $\lambda$ below again we have $M < x$). Now, the main difference in this case is the treatment of the first term, for which we have
\begin{align*}
    \int_1^Me^{\lambda y}dF_Z(y)&=\,\int_1^MdF_Z(y)+\lambda\int_1^MydF_Z(y)+\int_1^M\left(e^{\lambda y}-1-\lambda y\right)dF_Z(y)\\[3pt]&\leq\,1+\lambda E_0-\left(e^{\lambda y}-1-\lambda y\right)(1-F_Z(y))\bigg|^M_1+\lambda\int_1^M\left(e^{\lambda y}-1\right)(1-F_Z(y))dy\\[3pt]&\leq 1+\lambda E_0+\left(e^{\lambda }-1-\lambda\right)+\lambda V\int_1^M\left(e^{\lambda y}-1\right)y^{-\gamma}dy\\[3pt]&\leq 1+\lambda E_0+\left(e^{\lambda }-1-\lambda\right)+\frac{\lambda V}{\gamma-1}\left(e^{\lambda}-1\right)+\frac{\lambda^2 V}{\gamma-1}e^{\lambda M}\int_1^My^{1-\gamma}dy\\[3pt]&\leq 1+\lambda E_0+\lambda^2+\frac{2\lambda^2 V}{\gamma-1}+C_1 W(\lambda),
\end{align*}
for some value $C_1$ depending on $\gamma$ alone, where we used that $\gamma>1$, that $\lambda$ is small, but also $\lambda M=2\gamma$, and where 
\[W(\lambda)=\left\{\begin{array}{cl}\lambda^{\gamma}&\text{ if }\gamma<2\\[3pt]-\lambda^2\log(\lambda)&\text{ if }\gamma=2\\[3pt]\lambda^2&\text{ if }\gamma>2\end{array}\right.\]
Since $\lambda$ is small we conclude that the $W(\lambda)$ is at least of the same order as the terms containing $\lambda^2$ and hence
\[\int_1^Me^{\lambda y}dF_Z(y)\;\leq\;1+\lambda E_0+3W(\lambda).\]
Treating the integral $\int_M^xe^{\lambda y}dF_Z(y)$ as in the case $\gamma\leq 1$ we finally obtain 
\begin{align*}\PP(S_m-mE_0\geq x)&\leq\;mVx^{-\gamma}+e^{-\lambda x-\lambda mE_0}\left(1+\lambda E_0+3W(\lambda)+C_2\lambda^{\gamma}+ C_3e^{\lambda x}x^{-\gamma}\right)^m\\[4pt]
&\leq\;mVx^{-\gamma}+\exp\left(-\lambda x+mC_4W(\lambda)+ mC_3e^{\lambda x}x^{-\gamma}\right).
\end{align*}
Now, since we are interested in the probability $\PP(S_m\geq Lm)$ for $L$ larger than some $\auxl$ which we can take larger than $2E_0$ we have
\[\PP(S_m\geq Lm)\,\leq\,\PP(S_m-mE_0\geq Lm/2),\]
and hence we can take $x=Lm/2$. For $\gamma<2$ we choose $\lambda=\frac{1}{x}\log(\frac{x^\gamma}{m})$ as before (which is small) for which $mC_3e^{\lambda x}x^{-\gamma}=C_3$ and hence
\[\PP(S_m-mE_0\geq x)\;\leq\;2^\gamma VL^{-\gamma}m^{1-\gamma}+\exp\left(-\log(L^\gamma m^{\gamma-1}/2^{\gamma})+3C_5m\lambda^\gamma+ C_3\right),  \]
but $m\lambda^\gamma=\frac{2^\gamma\log^\gamma(L^\gamma m^{\gamma-1}2^{-\gamma})}{L^\gamma m^{\gamma-1}}\leq 1$ for $L^\gamma m^{\gamma-1}$ large enough, and hence
\[\PP(S_m-mE_0\geq x)\;\leq\;\auxc L^{-\gamma}m^{1-\gamma}.\]
Suppose now that $\gamma\geq 2$ and choose $\lambda=\frac{\gamma}{x}\log(\frac{x}{\sqrt{m}})$ for which we have $mC_3e^{\lambda x}x^{-\gamma}=C_3m^{1-\frac{\gamma}{2}}\leq C_3$, giving 
\[\PP(S_m-mE_0\geq x)\;\leq\;2^\gamma VL^{-\gamma}m^{1-\gamma}+\exp\left(-\log(L^\gamma m^{\frac{\gamma}{2}}/2^{\gamma})+C_6mW(\lambda)+ C_3\right).\]
Now, if $\gamma=2$, then $W(\lambda)=\lambda^2\log(1/\lambda)$ so for some constant $C_7$
\[mW(\lambda)=\frac{16}{L^2m}\log^2(\tfrac{L \sqrt{m}}{2})\log\left(\frac{Lm}{4\log(L\sqrt{m}/2)}\right)\leq\frac{C_1}{L^2m}\log^3(L^2m)\leq 1\]
for large $L^2m$, while if $\gamma>2$, $W(\lambda)=\lambda^2$, and so 
\[mW(\lambda)=\frac{2\gamma^2}{L^2m}\log^2(\tfrac{L \sqrt{m}}{2})\leq 1\]
for large $L^2m$. In any case scenario, we obtain
\[\PP(S_m-mE_0\geq x)\;\leq\;2^\gamma VL^{-\gamma}m^{1-\gamma}+\auxc L^{-\gamma}m^{-\frac{\gamma}{2}},\]
but for $\gamma\geq 2$ we have $\frac{\gamma}{2}\leq\gamma-1$ and hence the second term dominates the first, giving the result.
\end{proof}

\end{document}
