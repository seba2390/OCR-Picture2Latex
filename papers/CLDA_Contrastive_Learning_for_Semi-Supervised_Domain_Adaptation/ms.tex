\documentclass{article}
\pdfoutput=1
% if you need to pass options to natbib, use, e.g.:
    \PassOptionsToPackage{numbers, compress}{natbib}
% before loading neurips_2021

% ready for submission
%   \usepackage{neurips_2021}

% to compile a preprint version, e.g., for submission to arXiv, add add the
% [preprint] option:
%  \usepackage[preprint,nonatbib]{neurips_2021}

% to compile a camera-ready version, add the [final] option, e.g.:
      \usepackage[final,nonatbib]{neurips_2021}

% to avoid loading the natbib package, add option nonatbib:
% \usepackage[nonatbib]{neurips_2021}

\usepackage[utf8]{inputenc} % allow utf-8 input
\usepackage[T1]{fontenc}    % use 8-bit T1 fonts
\usepackage{url}            % simple URL typesetting
\usepackage{booktabs}       % professional-quality tables
\usepackage{amsfonts}       % blackboard math symbols
\usepackage{nicefrac}       % compact symbols for 1/2, etc.
\usepackage{microtype}      % microtypography
\usepackage{xcolor} 
\usepackage{graphicx}
\usepackage{array,multirow} 
\usepackage{booktabs} % for professional tables
\usepackage[lofdepth,lotdepth]{subfig}
\usepackage[ruled,vlined,linesnumbered]{algorithm2e}
\SetKwInput{KwInput}{Input}
\SetKwInput{KwOutput}{Output}
\usepackage{float}
\usepackage{mathtools} % for splitfrac
% \usepackage[table]{xcolor}
% \floatstyle{plaintop}
\usepackage{bbold}
\restylefloat{table}
\newcommand{\ankit}[1]{{\color{blue}{ [Ankit: #1]}}}
\usepackage{booktabs} % colors
\usepackage[
backend=biber,
maxcitenames=50,
maxnames=50,
style=numeric,
citestyle=numeric
]{biblatex}
\addbibresource{ms.bib}
\usepackage{hyperref}       % hyperlinks
\title{CLDA: Contrastive Learning for Semi-Supervised Domain Adaptation}


\author{%
  Ankit Singh \\
  Department of Computer Science\\
  Indian Institute of Technology, Madras \\
  \texttt{singh.ankit@cse.iitm.ac.in} \\
}

\begin{document}

\maketitle

\begin{abstract}
Unsupervised Domain Adaptation (UDA) aims to align the labeled source distribution with the unlabeled target distribution to obtain domain invariant predictive models. However, the application of well-known UDA approaches does not generalize well in Semi-Supervised Domain Adaptation (SSDA) scenarios where few labeled samples from the target domain are available.
This paper proposes a simple \textbf{C}ontrastive \textbf{L}earning framework for semi-supervised \textbf{D}omain \textbf{A}daptation (\textbf{CLDA}) that attempts to bridge the intra-domain gap between the labeled and unlabeled target distributions and the inter-domain gap between source and unlabeled target distribution in SSDA. We suggest employing class-wise contrastive learning to reduce the inter-domain gap and instance-level contrastive alignment between the original(input image) and strongly augmented unlabeled target images to minimize the intra-domain discrepancy. We have empirically shown that both of these modules complement each other to achieve superior performance. Experiments on three well-known domain adaptation benchmark datasets, namely DomainNet, Office-Home, and Office31, demonstrate the effectiveness of our approach. CLDA achieves state-of-the-art results on all the above datasets.
\end{abstract}

\section{Introduction}
\label{sec:Introduction}


The goal in top-$\size$ recommendation is to recommend to each
consumer a small set of $\size$ items from a large collection of
items~\cite{cremonesi2010performance}.  For example, Netflix may want
to recommend $\size$ appealing movies to each consumer.  Collaborative
Filtering (CF)~\cite{herlocker2002empirical,lee2012comparative} is a
common top-$\size$ recommendation method.  CF infers user interests by
analyzing partially observed user-item interaction data, such as user
ratings on movies or historical purchase
logs~\cite{kanagal2012supercharging}. The main assumption in CF is that
users with similar interaction patterns have similar interests.


Standard CF methods for top-$\size$ recommendation focus on making  suggestions  that accurately reflect the user's preference history. However, as  observed in previous work,  CF recommendations are generally biased toward  popular items, leading to a rich get richer effect~\cite{vargas2014improving,steck2011item}.  The major reasons for this are \textit{popularity bias} and \textit{sparsity} of CF interaction data (detailed in Section~\ref{sec:related-work}). In a nutshell, to maintain  accuracy, recommendations are generated from the dense regions of the data,  where the popular items lie.  

However,  accurately suggesting popular items, may not be satisfactory for the consumers. For example, in Netflix, an accuracy-focused movie recommender may recommend ``Star Wars: The Force Awakens'' to users who have seen ``Star Wars: Rogue One''.  But, those users are probably already aware of ``The Force Awakens''. Considering additional factors, such as novelty of recommendations,  can lead to more effective suggestions~\cite{cremonesi2010performance,Castells2015,zhang2008avoiding,ziegler2005improving,zhang2012auralist}. 
%Second, accuracy-focused models typically achieve a   overall item-space coverage across their recommendations,  whereas high item-space coverage helps providers of the items increase revenue
%, users satisfaction since they are  likely already aware of or can find these items on their own.  

Focusing on popular items also adversely affects the satisfaction of  the providers of the items. This is because  accuracy-focused models typically achieve a  low overall item space coverage across their recommendations, whereas   high item space coverage helps providers of the items increase their revenue~\cite{vargas2014improving,Castells2015,adomavicius2011maximizing,anderson2006thelongtail, yin2012challenging,adomavicius2012improving}.
%accuracy-focused models typically achieve a

In contrast to the relatively small number of popular items, there are copious  {\it long-tail\/} items that have fewer observations (e.g., ratings) available. More precisely,  using the Pareto  principle (i.e.,~the $80/20$ rule),  long-tail items can be defined as items that generate the lower $20\%$ of observations~\cite{yin2012challenging}. Experimentally we found that these items correspond to almost $85\%$ of the items in several datasets (Sections~\ref{sec:Notation} and \ref{sec:Experiments}). %Table~\ref{tab:DatasetStatsticsSmall})


As previously shown, one way to improve the novelty of top-$\size$ sets is to recommend interesting long-tail items~\cite{cremonesi2010performance,ge2010beyond}.  The intuition  is that since they have fewer observations available,  they are more likely to be unseen~\cite{Kaminskas:2016:DSN:3028254.2926720}.  
 %For example, in online commerce,  newly added items are long-tail items that are yet to be discovered.  
Moreover, long-tail item promotion also results in higher overall coverage of the item space%, which increases profits for providers of the items
~\cite{vargas2014improving,Castells2015,zhang2008avoiding,zhang2012auralist,adomavicius2011maximizing,anderson2006thelongtail,yin2012challenging,jambor2010optimizing}. Because long-tail promotion reduces accuracy~\cite{steck2011item}, there are trade-offs to be explored.


%original submitted to ICDE
%This work studies three aspects of top-$\size$ recommendation: accuracy, novelty, and item-space coverage, and examines their trade-offs. In most previous work, predictions of a base recommendation system are re-ranked to handle their trade-offs~\cite{adomavicius2012improving,jambor2010optimizing,zhang2013personalize,wang2009portfolio}. Due to performance considerations, however, these techniques are not customized per user. For example,  parameters that balance the trade-off between novelty and accuracy are cross-validated at a global level.  This can be detrimental since users have varying preferences for  objectives such as long-tail novelty. We explore how to  automatically infer  user  preference for long-tail novelty, and how to leverage  it to correct  the popularity bias in standard recommender models. Our work does not rely on any additional contextual data, although such data, if available, can help promote newly-added long-tail items~\cite{agarwal2009regression,Saveski:2014:ICR:2645710.2645751}.

This work studies three aspects of top-$\size$ recommendation: accuracy, novelty, and item space coverage, and examines their trade-offs. In most previous work, predictions of a base recommendation algorithm are \textit{re-ranked} to handle these trade-offs~\cite{adomavicius2012improving,jambor2010optimizing,zhang2013personalize,wang2009portfolio}. The re-ranking models are computationally efficient but suffer from two drawbacks. First, due to performance considerations,  parameters that balance the trade-off between novelty and accuracy  are not customized per user. Instead they are cross-validated at a global level.  This can be detrimental since users have varying preferences for  objectives such as long-tail novelty. Second,  the re-ranking methods are often limited to a specific base recommender  that may be sensitive to dataset density. 
As a result, the datasets are pruned and the problem is studied in dense settings~\cite{adomavicius2012improving,ho2014likes}; but real world  scenarios are often sparse~\cite{kanagal2012supercharging,liu2017experimental}.   
% Because  dataset density can impact the performance of most base recommenders (like R-SVD), which in turn affects the performance of the re-ranking model, 

\iffalse
We address these limitations by directly inferring  user  preference for long-tail novelty  from interaction data.  This  allows us to customize the re-ranking  per user, and design a \textit{generic} framework, which resolves the second problem. In particular, since the long-tail novelty preferences are estimated independently of any base  recommender model, we can  plug-in an appropriate base recommender w.r.t. the dataset sparsity.% including ones that are more suitable for sparse settings.  

Modelling  user  preference for  long-tail novelty using only item popularity statistics, e.g., the average popularity of rated items as in~\cite{jugovac2017efficient}, disregards additional information like whether the user found the item interesting and the long-tail preferences of other users  of the items. \iffalse To incorporate them, we introduce the notion of  \emph{item long-tail importance}. Both  user long-tail preferences and item long-tail importance are dependent:  a user has high preference for discovering long-tail items if she is interested in important long-tail items, and an item that is associated with many of these kinds of users is likely to be more important.  We propose a joint optimization framework to directly learn,  from interaction data, both the users' long-tail preferences and the  items' long-tail importance. \fi
We propose an optimization approach that  incorporates  this information and  directly learns,  from interaction data, the users' long-tail novelty preferences.

Next, we use these learned preferences  to design a  top-$\size$ recommendation framework thats is generic, and provides customized balance between accuracy, novelty, and coverage. We refer to it as framework as GANC.  Using GANC, we design a novel algorithm, {\it Ordered Sampling-based Locally Greedy (OSLG)\/}, that relies on the learned long-tail novelty preferences  to scalably correct for popularity bias. Our work does not rely on any additional contextual data, although such data, if available, can help promote newly-added long-tail items~\cite{agarwal2009regression,Saveski:2014:ICR:2645710.2645751}. In summary:
\fi

We address the first limitation by directly inferring  user  preference for long-tail novelty  from interaction data.   Estimating these  preferences  using only item popularity statistics, e.g., the average popularity of rated items as in~\cite{jugovac2017efficient}, disregards additional information, like whether the user found the item interesting or the long-tail preferences of other users  of the items. We propose an approach that  incorporates  this information and  learns the users' long-tail novelty preferences from interaction data.

This approach allows us to customize the re-ranking  per user, and  design a \textit{generic} re-ranking framework, which resolves the second limitation of prior work. In particular, since the long-tail novelty preferences are estimated independently of any base recommender, we can  plug-in an appropriate one w.r.t. different factors, such as the dataset sparsity.

Our top-$\size$ recommendation framework, \textbf{GANC}, is \textbf{G}eneric, and provides customized balance between \textbf{A}ccuracy, \textbf{N}ovelty, and \textbf{C}overage. % Moreover, based on the learned long-tail novelty preferences, we also design a novel algorithm, {\it Ordered Sampling-based Locally Greedy (OSLG)\/}, that relies on the learned long-tail novelty preferences  to scalably correct for popularity bias. 
Our work does not rely on any additional contextual data, although such data, if available, can help promote newly-added long-tail items~\cite{agarwal2009regression,Saveski:2014:ICR:2645710.2645751}. In summary:

%Consider  the following toy example:
\vspace{-0.2cm}
\begin{table}[htb]
\centering
\scriptsize
%\small
\begin{tabular}{ccccccc} 
%\toprule
%&\multirow{2}{*}{}&\multicolumn{7}{c}{Ratings}\\
& & \cellcolor{blue!35}$w_1$ &\cellcolor{blue!18} $w_2$ & $\dots$ &\cellcolor{blue!8} $w_{89}$  &\cellcolor{blue!8} $w_{99}$   
\\
&   &$i_1$&$i_2$&$\dots$&$i_{89}$&$i_{90}$\\ 
\cmidrule(r){3-7} 	 
%\midrule
\cellcolor{red!35}$\theta_1$  &$u_1 $   &5 &   & $\dots$ &  &   \\
\cellcolor{red!28}$\theta_2$  &$u_2$     &5 &    & $\dots$ &  &  \\
 $\theta_3=?$  &$\bf u_3$  &5 &  &   $\dots$ &  &  \\
\cellcolor{red!10}$\theta_4$ & $u_4$  &  &5   & $\dots$ & &\\ 
\cellcolor{red!10}$\theta_5$ & $u_5$  &  & 5  & $\dots$ & &\\ 
$\theta_6=?$  & $\bf u_6$ & &5  &      $\dots$& &  \\ 
 & & $\hdots$  &$\hdots$   &$\hdots$   &$\hdots$   &$\hdots$  \\
%\midrule 
\cmidrule(r){3-7} 	 
\multicolumn{2}{c}{item pop.}  & 3  & 3  & $\dots$ &50&60\\  
%\bottomrule
%$ f_i$    &3  &3  &1  &3  &1  &2  \\  \hline
\end{tabular}
%#.
\caption{Simplified user-item interaction data. The user long-tail novelty preference ($\theta_u$), item long-tail importance weight ($w_i$) are highlighted. Darker colors indicate larger values. } \label{tab:example}
\end{table} 
\vspace{-0.2cm}
\begin{example}  
In Table~\ref{tab:example}, we are interested in estimating $\theta_3$ and $\theta_6$,  the long-tail preference of users $u_3$ and $u_6$ who have each rated a single movie. Additional ratings for other users  are not included here.  Considering only rating information, we observe $i_1$ and $i_2$ are  equally popular $|\mathcal{U}_{i_1}^{\trainset}| = |\mathcal{U}_{i_2}^{\trainset}|=3$, and $r_{31}=5$ and $r_{62}=5$. Using Eq.~\ref{eq:tfidf-risk}  we have $\theta_3 = \theta_6$. However, if we were given the long-tail preferences of the each item's user set, specifically that $u_1$ and $u_2$ have high long-tail preference (darker red), while $u_4$ and $u_5$ have lower long-tail preference (lighter red), we could conclude $i_1$ is a more important long-tail item compared to $i_2$ (indicated by a darker blue shade for $w_1$), and we expect  $\theta_3 \geq \theta_6$.

% On the other hand, if we knew that $u_4$ and $u_5$ have lower long-tail preference, we could conclude $i_2$ is a  less significant long-tail item. Therefore, However, if we  consider the long-tail preferences of other users, we may reason differently.    We need another variable $w_i$ which captures this information. 
%we would conclude that $u_3$ has higher long-tail preference compared to $u_6$, since the users $i_1$ is a more prominent long-tail item. 

% Relying only  on item popularity information, we would  conclude   $u_3$ and $u_6$ have equal long-tail preference, since $i_1$ and $i_2$ are  equally popular. However, considering  the second column,  long-tail preference of users,  long-tail importance for each item,  which captures the long-tail preference of its users. Since  that  both users of $i_1$ have high long-tail preference while  the users of $i_2$ have lower preference,  we may conclude $i_1$ is a more important long-tail item compared to $i_2$. Therefore, $u_3$'s long-tail preference should be at least as large as $u_6$'s preference. Specifically, consider two  items $i_1$ and $i_2$, with the following rating data: $i_1=\{u_1:5, u_2:5, u_3:5 \}$, $i_2=\{u_4:5, u_5:5, u_6:5\}$.  

%Table~\ref{tab:example} shows  simplified rating data. We want an estimate of the long-tail preference of $u_3$ and $u_6$, who have each  rated a single movie.  Relying only  on movie popularity information, we would  conclude   $u_3$ and $u_6$ have similar long-tail preference, since $m_1$ and $m_2$ are  equally popular. However, considering the long-tail preferences of other users of those movies, we may reason differently: since $u_1$ and $u_2$ have high long-tail preference, and $u_4$ and $u_5$ have low long-tail preference, $m_1$ is a more prominent long-tail item compared to $m_2$. Therefore, it is likely that $u_3$ has higher long-tail preference compared to $u_6$.considering the long-tail preferences of other users of those movies, we may reason differently.  For example, 
\label{ex:running}
\end{example}



%------------------------------

\iffalse
\begin{example}
Table~\ref{tab:example} shows rating data for a simplified system. %Note the user-item interaction matrix is sparse.
For this example, we define popular movies as those that have received  three or more ratings; $\{m_1, m_2, m_4\}$ are popular and  $\{m_3, m_5, m_6\}$ are niche movies. We observe $u_1$ and $u_3$  have rated relatively popular movies (risk-averse) while $u_2$ and $u_4$ have rated niche movies (risk-loving). 
\label{ex:running}
\end{example}

\begin{table}[htb]
\centering
\scriptsize
\begin{tabular}{ccccccc} 
\toprule
			&$m_1$ &$m_2$   &$m_3$    &$m_4$   &$m_5$ &$m_6$  \\ \hline 
$u_1 $ &5  &4  & - &-  &-  &-   \\
$u_2$  &-  &-  &-  &-  &5  &5   \\
$u_3$  &-  &4  &-  &5  &-  &-   \\
$u_4$  &-  &-  &3  &-  &-  &4   \\ 
$u_5$  &5  &-  &-  &3  &-  &-   \\ 
$u_6$  &4  &2  &-  &4  &-  &-   \\ 
\bottomrule
%$ f_i$    &3  &3  &1  &3  &1  &2  \\  \hline
\end{tabular}
\caption{User-Movie rating data} \label{tab:example}
\end{table}

It is essential to consider consumer characteristics in designing recommender systems so that they promote long-tail items to the right group of users and spread demand evenly between hit and niche items.  

\fi





%------------------------------
\iffalse
\begin{table}[htb]
\centering
\scriptsize
\begin{tabular}{ccccccc} 
\toprule
			&$m_1$ &$m_2$   &$m_3$    &$m_4$   &$m_5$ &$m_6$  \\ \hline 
$u_1 $ &\textbf{5}  & \textbf{4}  &\textcolor{gray}{ 1.2} &-  &-  &-   \\
$u_2$  &-  &-  &-  &-  & \textbf{5}  &\textbf{5}   \\
$u_3$  &-  &\textbf{4}  &-  &\textbf{5}  &-  &-   \\
$u_4$  &-  &-  &\textbf{3}  &-  &-  &\textbf{4}   \\ 
$u_5$  &\textbf{5}  &-  &-  &\textbf{3}  &-  &-   \\ 
$u_6$  &\textbf{4}  &\textbf{2}  &-  &\textbf{4}  &-  &-   \\ 
\bottomrule
%$ f_i$    &3  &3  &1  &3  &1  &2  \\  \hline
\end{tabular}
\caption{User-Movie rating data} \label{tab:example}
\end{table}
% $\mathcal{P}^1= \{ \mathcal{P}_1^1 \{i_1,i_2,i_3\}, \mathcal{P}_2^1:\{i_2,i_3,i_5\}  \}$
 %$\mathcal{P}^2= \{ \mathcal{P}_1^2: \{i_1,i_2,i_3\}, \mathcal{P}_2^2:\{i_2,i_5,i_6\}  \}$
 %$\mathcal{P}^3= \{ \mathcal{P}_1^3: \{i_7,i_8,i_9\}, \mathcal{P}_2^3:\{i_{10},i_{11},i_{12}\}  \}$
\begin{table}[htb]
\centering
\tiny
\begin{tabular}{ccc} 
\toprule
		&$u_1$&$u_2$  \\ \hline 
$\mathcal{P}^1 $ & $\{i_1,i_2,i_3\}$ & $\{i_2,i_3,i_5\} $ \\
$\mathcal{P}^2$ & $\{i_1,i_2,i_3\}$ & $\{i_2,i_5,i_6\} $ \\
$\mathcal{P}^3$ & $\{i_7,i_8,i_9\}$ & $\{i_{10},i_{11},i_{12} \}$ \\
\bottomrule
%$ f_i$    &3  &3  &1  &3  &1  &2  \\  \hline
\end{tabular}
\caption{Top-$\size$ allocations to users.} \label{tab:paretoExamples}
\end{table}
\fi


\iffalse
When considering long-tail items, it is important to consider consumers' willingness  to explore niche or unpopular items and their propensity towards similar items. In particular, they can be characterized by their  {\it risk degree\/} and {\it focusing degree\/}, respectively.  We compute these estimates  based on historical rating information. The following example further describes these notions in the context of movie rating data. 

\begin{example}  
Table~\ref{tab:example} shows rating data for a simplified system with $6$ users, $6$ movies, and $3$ genres. $m_i^{j}$ implies that movie $m_i$ belongs to genre $j$. Note the user-item interaction matrix is sparse. 
  For this setting, we define popular movies as those that have received  three or more ratings; $\{m_1, m_2, m_4\}$ are popular and  $\{m_3, m_5, m_6\}$ are niche movies. We now profile the users according to their risk and focusing degree. E.g., $u_1$ has rated relatively popular movies belonging to the same genre (risk-averse, high focusing degree); $u_2$ has rated niches movies in the same genre (risk-loving, high focusing degree); $u_3$ has rated popular movies in two different genres (risk-averse, low focusing degree), and $u_4$ has rated niches movies in two different genres (risk-loving, low focusing degree). 
\label{ex:running}
\end{example}
\begin{table}[htb]
\centering
\tiny
\begin{tabular}{ccccccc} 
\toprule
			&$m_1^{1}$ &$m_2^{1}$   &$m_3^{2}$    &$m_4^{3}$   &$m_5^{3}$ &$m_6^{3}$  \\ \hline 
$u_1 $ &5  &4  &-  &-  &-  &-   \\
$u_2$  &-  &-  &-  &-  &5  &5   \\
$u_3$  &-  &4  &-  &5  &-  &-   \\
$u_4$  &-  &-  &3  &-  &-  &4   \\ 
$u_5$  &5  &-  &-  &3  &-  &-   \\ 
$u_6$  &4  &2  &-  &4  &-  &-   \\ 
\bottomrule
%$ f_i$    &3  &3  &1  &3  &1  &2  \\  \hline
\end{tabular}
\caption{User-Movie rating data} \label{tab:example}
\end{table}
It is essential to consider these consumer characteristics in designing recommender systems so that they promote long-tail items to the right group of users and spread demand evenly between the hit and niche items.  
\fi
\iffalse
\begin{center}
\begin{figure*}[tp]
%\scalebox{0.5}{%
\resizebox{1\textwidth}{!}{%
%\small%\addtolength{\tabcolsep}{5pt}% below sums to 8
\begin{tabularx}{1.5\textwidth}{>{\hsize=2.5\hsize}X>{\hsize=2.5\hsize}X>{\hsize=0.5\hsize}X>{\hsize=0.5\hsize}X>{\hsize=0.5\hsize}X>{\hsize=0.5\hsize}X>{\hsize=0.5\hsize}X>{\hsize=0.5\hsize}X}
    \multirow{12}{*}{\includegraphics[scale=0.3]{codeForExample/popularity-movie.png}} & \multirow{12}{*}{\includegraphics[scale=0.3]{codeForExample/scatterplot.png}} & & & & & & \\
%   & &               &       &       &       &       &       \\
    & &\multicolumn{1}{l|}{}               &$m_1^{g1}$   	&$m_2^{g1}$    	&$m_3^{g2}$    &$m_4^{g2}$      &$m_5^{g3}$    \\ \cline{3-8}%\hline
    & &\multicolumn{1}{l|}{u1}          &5  &5  &-  &-   &-  \\
    & &\multicolumn{1}{l|}{u2}    		&-  &-  &4  &4  &5  \\
    & &\multicolumn{1}{l|}{u3}   			&1  &2  &1  &-  &-   \\
    & &\multicolumn{1}{l|}{u4}     		&1  &-  &-  &-  &-  \\
    & &               &       &       &       &       &       \\
    & &               &       &       &       &       &       \\
    & &               &       &       &       &       &       \\
    & &               &       &       &       &       &	\\
    \\
\end{tabularx}}
\caption{User-Movie interaction data a) Popularity-Movie histogram b)Movie genres/clusters c) User-Movie rating data} \label{fig:example}
\end{figure*}
\end{center}
\fi



%We propose a novel approach that allows us to  promote long-tail items in a targeted manner, thereby improving the novelty of top-$\size$ sets, the overall item-space coverage across recommendations, while maintaining reasonable levels of accuracy.

%Next, we integrate these learned preferences  in a generic  top-$\size$ recommendation framework to provide customized balance between accuracy and coverage.

%sequentially make recommendations, while adjusting its parameters with regard to the set of top-$\size$ recommendations made so far. However, since  sequential parameter updates  cause  scalability issues, we propose a sampling based algorithm. This variant of our framework, called {\it Ordered Sampling-based Locally Greedy (OSLG)\/},  allows us to  correct for the popularity bias in recommendations with regard to individual user long-tail preferences. 

%ICDE submission
%Our framework differs with  prior work in the following aspects:  unlike~\cite{adomavicius2011maximizing,adomavicius2012improving,zhang2013personalize,ho2014likes},  the long-tail preference personalization in our framework is learned rather than optimized using cross-validation or parameter tuning. In other words, our personalization method is independent of the underlying base  recommendation models.  Moreover, our framework is  generic. This enables us to  plug-in several base recommenders, and evaluate their  effectiveness without requiring  extensive tuning for the accuracy and coverage trade-off. 


%\vspace{-2.8pt}
\begin{itemize}

\item  We examine various measures for estimating user long-tail novelty preference in Section~\ref{sec:lt-pref} and formulate an optimization problem  to directly learn users' preferences for long-tail  items from interaction data in Section~\ref{sec:learning-lt-pref}. %In addition, we introduce several heuristics for measuring the user preference for less common items from historical rating data.% 

\item  We integrate the user preference estimates into GANC %, a generic re-ranking framework that provides customized balance between accuracy, novelty, and coverage 
(Section~\ref{sec:RiskbasedReranking}), and  introduce {\it Ordered Sampling-based Locally Greedy (OSLG)\/}, a scalable algorithm that relies  on user long-tail preferences to correct the popularity bias (Section~\ref{sec:optimizationAlgorithm}).
%We introduce OSLG, a scalable algorithm that relies  on user long-tail preferences to  maximize item space coverage \textcolor{red}{while maintaining acceptable levels of accuracy} (Section~\ref{sec:optimizationAlgorithm}).

\item   We conduct an extensive empirical study and evaluate performance from  accuracy, novelty, and coverage perspectives (Section~\ref{sec:Experiments}).  We use five  datasets with varying density and difficulty levels. %:  Netflix, MovieTweetings, and MovieLens (100K, 1M, 10M). 
  In contrast to most related work,  our evaluation considers realistic settings that include a large number of infrequent  items and users. %This enables us to study the impact of  data density on the performance trade-offs of several  state of the art top-$\size$ recommendation algorithms. %   %,  and use the all-items ranking protocol~\cite{steck2013evaluation,vargas2014improving}, where performance is measured using all items with train data. to evaluate the performance of several  state of the art top-$\size$ recommendation algorithms 
 
\item Our empirical results confirm that the performance of re-ranking models is impacted by the underlying   base recommender and the dataset density. Our generic approach enables us to easily incorporate a suitable base recommender to devise an effective solution for both dense and sparse settings. In dense settings, we use the same base recommender as existing re-ranking approaches, and we outperform them in accuracy and coverage metrics. For sparse settings, we plug-in a more suitable base recommender, and devise an effective solution that is competitive with existing top-$\size$ recommendation methods in accuracy and novelty. 

%Directly estimating the long-tail novelty preferences allows us to customize re-ranking per user, and  devise a generic framework.   
 
\end{itemize}

Section~\ref{sec:related-work} describes related work. Section~\ref{sec:conclusion} concludes.


\section{Related Work}
The loopholes we present in this paper are explored using packet injection, in which an attacker sends fake WiFi packets to devices in a secured WiFi network.
Packet injection has been used in the past to perform various types of attacks against WiFi networks
such as denial of service attacks for a particular client device or total disruption of the network~\cite{vanhoef2020protecting, dos, rogue-ap, deauth}. These attacks use different approaches such as beacon stuffing to send false information to WiFi devices~\cite{beacon-stuffing-1, beacon-stuffing-2}, or Traffic Indication Map (TIM) forgery to prevent clients from receiving data ~\cite{bellardo2003802, tim-forgery}. However, all of these attacks focus on spoofing 802.11 MAC-layer management frames to interrupt the normal operation of WiFi networks. 
To provide a countermeasure for some of these attacks, the 802.11w standard~\cite{ieee802.11w} 
introduces a protected management frame that prevents attackers from spoofing 802.11 management frames. 
Instead of spoofing 802.11 MAC frames, we exploit properties of the 802.11 physical layer to force a device to stay awake and respond when it should not. 
These loopholes open the door to multiple research avenues including new security and privacy threats. 


%\textcolor{blue}{Our recent preliminary work has shown that all WiFi devices respond with ACKs to packets received from outside of their network~\cite{abedi2020wifi}. However, this workshop paper does not show how to keep WiFi devices awake and avoid going to sleep mode. Moreover, it does not explore turning WiFi devices into sensitive motion sensors and monitoring people's breathing rates. In contrast,  \name\ shows the first attack which forces a WiFi device to be awake and pushes it to continuously transmit. We show how our technique enables an attacker to monitor the breathing rate of people by analyzing their WiFi signals.}

% The related work can be divided into three categories. The first category is WiFi sensing systems that use WiFi signals to infer some information, such as gesture detection, to enable a useful application for the user (the good). The second category is WiFi attacks which show how an attacker can interfere with the normal operation of WiFi networks (the bad). Finally, some recent studies show new privacy attacks against users by analyzing their WiFi signals (the ugly).


%Over the past decade, there has been a significant amount of research on WiFi sensing where WiFi signals are used to detect human activities~\cite{wifi-sensing-survey} to enable useful applications. These systems target different applications such as tracking and localization\cite{adib2013see, activity-recognition-1}, human detection \cite{gong2016adaptive, gong2015wifi}, gesture recognition \cite{abdelnasser2015wigest, gesture-recognition-1, gesture-recognition-2, gesture-recognition-3} and vital measurement such as respiration rate~\cite{abdelnasser2015ubibreathe, adib2015smart, breathing-rate-1, breathing-rate-2}. However, these systems target applications with social benefits and cannot be easily used by an attacker to create privacy and security threats. This is because either these techniques require cooperation from the target WiFi device or the attacker needs to be very close to the target to use these systems.


%WiFi Sensing is a technique that uses ambient WiFi signals to detect events or human activities~\cite{wifi-sensing-survey}. The motivation behind WiFi sensing is that we can obtain certain information without dedicated sensors. In particular, WiFi sensing techniques analyze changes in WiFi signals to infer different types of information. As mentioned earlier in Section~\ref{sec:csi}, CSI has been shown to be well suited for sensing techniques. For instance, there are applications that can identify the number of people in a closed room and their relative locations\cite{adib2013see, activity-recognition-1}. Applications that  develop wireless device-free human detection \cite{gong2016adaptive, gong2015wifi} are also implemented to determine the presence of human activities. Other than this, subtle movements like gesture recognition \cite{abdelnasser2015wigest, gesture-recognition-1, gesture-recognition-2, gesture-recognition-3} and vital measurement such as respiration rate~\cite{abdelnasser2015ubibreathe, adib2015smart, breathing-rate-1, breathing-rate-2}, can also be measured using wireless sensing. These applications are great tools that bring convenience to people's lives.However, for these techniques to work WiFi devices should cooperate to enable WiFi sensing. Therefore, an attacker cannot use these techniques to perform WiFi sensing since he/she has no access to the target building and the devices inside it.

% \subsection{The Bad}
% \label{sec:stealth}


%\emph{Beacon Injection:} 
%This attack is performed by forging 802.11 beacons and broadcasting them to all devices in a WiFi network~\cite{beacon-stuffing-1, beacon-stuffing-2}. The attacking device pretends to be the actual access point and injects false information in the forged beacons to enable a variety of attacks such as the ``evil twin access point'' attack~\cite{evil-twin}. 
%Since beacons are broadcasted to all devices, this will attack all the devices at the same time. The forged beacon frames can also be sent (i.e., unicast) to a particular device to attack individual devices rather than the entire network.

%\emph{TIM Forgery:}
%Traffic Indication Map (TIM), as mentioned in Section \ref{sec:beacon}, is used in 802.11 beacon frames. It contains information about whether sleeping devices have buffered packets at Access Point (AP) or not. It is suggested that an adversary can manipulate the Time Indication Map (TIM) inside beacons to change the behavior of WiFi devices~\cite{bellardo2003802, tim-forgery}. \name\ builds on these attacks. In particular, \name\ forges the TIM to make a device believe that it has buffered data to receive, so it cannot enter the sleep mode.
%For instance, the adversary can forge the TIM to make a device believe that it has buffered data to receive, so it cannot enter the sleep mode. In contrast, it can also prevent a device from receiving any data by telling it that the AP  has no buffered data for it. 


\textbf{WiFi sensing attack:} Over the past decade, there has been a significant amount of research on WiFi sensing where WiFi signals are used to detect human activities~\cite{iot-wifi-localization, rf-sensing, wifi-sensing-survey,adib2015smart, breathing-rate-1, breathing-rate-2,gesture-recognition-1, gesture-recognition-2, gesture-recognition-3,pu2013whole}. However, these systems target applications with social benefits and cannot be easily used by an attacker to create privacy and security threats. This is because either these techniques require cooperation from the target WiFi device or the attacker needs to be very close to the target to use these systems. A recent study shows that by capturing WiFi signals coming out of a private building, it is possible for an adversary to track user movements inside that building~\cite{zhu2018tu}. However, this attack has a bootstrapping stage which requires the attacker to walk around the target building for a long time to find the location of the WiFi devices. Furthermore, since this work relies on only the normal intermittent WiFi activities, it cannot capture continuous data such as breathing rate.  

\textbf{Battery draining attack:} 
Battery draining attacks date back to 1999 \cite{stajano1999resurrecting} and there have been many studies on such attacks and potential defense mechanisms since then~\cite{caviglione2012energy}.
%Battery draining attack was first introduced by Stajano, F. and Anderson, R. in 1999 \cite{stajano1999resurrecting}, where they prophesied that the battery of a mobile device can be exhausted by a malicious user even via legitimate usage. Extensive studies have been conducted to investigate the attack and defense manners. 
%One possible approach is by constantly launching canonical attacks and forcing defense systems to attempt to defeat them. Running defending software consumes CPU resources which leads to quick drainage of battery energy \cite{caviglione2012energy}. 
Battery discharge models and energy vulnerability due to operating systems have been investigated \cite{zhang2010accurate,jindal2013hypnos}. A more recent study plays multimedia files implicitly to increase power consumption during web browsing \cite{fiore2014multimedia, fiore2017exploiting}. In terms of defending, a monitoring agent that searches for abnormal current draw is discussed in \cite{buennemeyer2008mobile}. In contrast, our attack exploits the loopholes in the 802.11 physical layer protocol and the power-hungry WiFi transmission to quickly drain a target device's battery. We will discuss in Section~\ref{sec:cannot-be-fixed} that stopping our proposed attack is nearly impossible on today's WiFi devices.


This paper is an extension of our previous workshop publication ~\cite{polite-wifi}. The workshop paper shows preliminary results for our finding that WiFi devices respond with ACKs to packets received from outside of their network, and provides a brief discussion on potential privacy and security concerns of this behavior without studying them. We have also explored how the WiFi power saving mechanism can be exploited to keep a target device awake in a localization attack~\cite{wi-peep}. 
In this paper, we provide an in-depth study of these previously discovered loopholes. We also design and perform two privacy and security attacks, based on these loopholes. Finally, we implement these attacks on off-the-shelve WiFi devices and present detailed performance evaluations.


% To the best of our knowledge, \name is the first attack that forces WiFi devices to continuously transmit, enabling an attacker, who does not have access to the network, to estimate the breathing rate of a person from outside of the building using a low-cost WiFi module.

%\name\ combines techniques from WiFi sensing, attacks against WiFi networks to enable a new privacy attack to show for the first time that an attacker can estimate the respiration rate of a person from outside of the building.


%These attacks relies on the fact that wireless signals pass through walls; therefore, an attacker who is outside a building can receive signals coming from inside that building. The attacker can analyze the distortions in WiFi signals, caused by the body of a target person, to infer various types of information from a target building.

%To make the problem worse, it has been shown that WiFi devices send acknowledgments (ACK) when they receive fake packets coming from outside the their network~\cite{abedi2020wifi}. This behavior, called \emph{Polite WiFi}, enlightens the possibility of turning any WiFi device into a secret sensor.  This is because the CSI information can be extract from the ACKs send by a victim device to perform WiFi sensing and potentially obtain some sensitive information.
%\name\ combines techniques from WiFi sensing, attacks against WiFi networks to enable a new privacy attack to show for the first time that an attacker can estimate the respiration rate of a person from outside of the building.
%We also propose potential methodologies to protect against this attack.


% Another privacy attack which uses WiFi to extract the International Mobile Subscriber Identity (IMSI) from mobile phones \cite{o2017mobile}. The leakage of IMSI may results in potential tracking of the devices through link to the target user's other hardware addresses such as WiFi MAC addresses. As a result, the mobile device can be turned into a secret tracker which works for the attacker.


% some features of the techniques mentioned above like \emph{Polite WiFi} and TIM forgery, and it can detect the target people's breathing rate without their notice. The leak of breathing rate may seem to be harmless, but attacker can use this information to further infer people's presence in the house.






\begin{figure*}[!h]
 \centering
 \centerline{\includegraphics[width=17cm]{framework}}
 \caption{The proposed multi-source remote sensing image registration framework of MS-HLMO, including Harris feature point detection, Histogram of Local Main Orientation feature extraction, and multi-scale registration strategy.}
 \label{fig:framework}
\end{figure*}

The framework of the proposed MS-HLMO registration algorithm is shown in Fig.\ref{fig:framework}. The input multi-source image pair to be registered is preprocessed, which includes data normalization and basic denoising. Then, the preprocessed single-band images are used for feature points detection and feature extraction. Harris corner point detection, which contains detail treatments for multi-source images, is adopted to generate feature points between the image pair for matching. The key process of the proposed HLMO feature extraction is carried out in a multi-scale strategy, in which Gaussian pyramids are built to create a scale-space of the images. The HLMO feature descriptors of each Harris corner point are extracted on the PMOM of the images. The feature points between the image pair are then matched according to the descriptors, and Fast Sample Consensus (FSC) is carried out to remove the outliers. The matching results in the scale-space are combined through a multi-scale matching strategy. Finally, the spatial transformation between the original image pair is determined by the coordinate relationship between matched feature points according to a selected transformation model.



\subsection{Harris Feature Point Detection}
\label{ssec:subhead}
Harris corner \cite{harris1988combined} is one of the most stable feature points, which is slightly affected by intensity and scale difference and has high computational efficiency \cite{gao2021multi,2021Multi}. It has the advantage in multi-source remote sensing images with multi-modal properties and large data size. Here, the similar strategy \cite{2021Multi} is used for feature points detection. The Harris corner response of each pixel is calculated by:
\begin{equation}
cornerness = \frac{{{\rm{det}}(\textbf{\emph{M}})}}{{{\rm{tr}}(\textbf{\emph{M}})}}
\end{equation}
\begin{equation}
\textbf{\emph{M}} = \left[ {\begin{array}{*{20}{l}}
{\sum\limits_{{\bf{W}_\sigma}} {{{\bf{G}}_x}^2} }&{\sum\limits_{{\bf{W}_\sigma}} {{{\bf{G}}_x} {{\bf{G}}_y}}}\\
{\sum\limits_{{\bf{W}_\sigma}} {{{\bf{G}}_x} {{\bf{G}}_y}} }&{\sum\limits_{{\bf{W}_\sigma}} {{{\bf{G}}_y}^2}}
\end{array}} \right]
\end{equation}
where $\rm{det}(\textbf{\emph{M}})$ and $\rm{tr}(\textbf{\emph{M}})$ are the determinant and trace of $\textbf{\emph{M}}$, respectively, ${{\bf{G}}_x}$ and ${{\bf{G}}_y}$ are the image's gradient along $x$ and $y$ directions, respectively, and $\bf{W}_\sigma$ is a Gaussian window with variance $\sigma$. Pixels with strong response are considered to be feature points with distinct structure and stability between multi-source images.

An important issue in practical multi-source remote sensing image registration is that the data size and scale relations between images to be registered are diverse. For example, an image with high resolution covers a smaller spatial area. Many of the existing algorithms only deal with the ideal case that the image pair has the same scale and size. This paper focuses on solving several key problems at the same time, that is, two uncertain factors of image scale and size should be considered simultaneously. The proposed MS-HLMO adopts local non-maximum suppression (LNMS) to solve this problem. Since the size and scale difference of the image pairs are uncertain, in Harris corner detection, it is expected that the feature points in the image pair are distributed as uniformly as possible with the use of LNSM. Then, the ratio of the window size in LNMS is set depending on the ratio of the data size of the image pair:
\begin{equation}
ratio = \sqrt {\frac{{M \times N}}{{m \times {\rm{n}}}}}
\end{equation}
where $M,N$ and $m,n$ are the length and width of the two images, respectively.

\begin{figure}[!h]
    \centering
        \subfloat[]{\includegraphics[width=2.5in]{keypoints_1.pdf}
        \label{fig:harris:a}}
    \hfil
        \subfloat[]{\includegraphics[width=2.5in]{keypoints_2.pdf}
        \label{fig:harris:b}}
    \caption{Examples of feature points detection results with LNMS. (a) Detection results of the image pair with different scale. (b) Detection results of the image pair with different scale and size.}
    \label{fig:harris}
\end{figure}

Consequently, the distribution of feature points is consistent with the images' scale proportion when there are scale differences, as shown in Fig.\ref{fig:harris:a}. When there is a size difference, the feature points are far more uniformly distributed, and the repeatability is higher, as shown in Fig.\ref{fig:harris:b}.



\subsection{Histogram of Local Main Orientation}
\label{ssec:subhead}

%\subsubsection{Robust Feature of Gradient Orientation}
%
%Most feature-point-based registration algorithms use a local descriptor to extract the neighborhood information of the keypoints, and generate their feature vectors for similarity matching.  For example, SIFT, SURF, HOG and PIIFD use local gradient information for statistics. However, the performance of these algorithms is greatly reduced when processing multi-source , especially multi-sensor images. Through practice and analysis on , it is believed that multi-modal images will have unpredictable and serious differences in the magnitude of gradient, but the orientation of gradient is a more stable feature.
%
%\begin{figure}[h!]
% \begin{center}
%  \includegraphics[width=3.5in]{Robust_Gradient_Orientation.pdf}
%  \caption{Robust gradient orientation.}
%  \label{fig:res}
% \end{center}
%\end{figure}
%
%To intuitively show the characteristics of this orientation, we briefly summarize the common intensity distortion in multi-modal images, as shown in Fig.0. The four images can be taken as the edge of an object or the interface of two substances in the image. Assume that the center point is the detected feature point, and the image block represents the intensity information of the neighborhood of it. In Fig.0 (a), the intensity amplitude of the left part is lower than that of the right part. Obviously, the gradient orientation is horizontal to the right, which is 0°; Take Fig.0 (a) as the original or reference image, then Fig.0 (b), Fig.0 (c), and Fig.0 (d) is considered as intensity distortions ${{\cal F}_{{\rm{Ra}}}}( \bullet )$ of (a). In (b), the magnitude relationship of the two parts remains unchanged, but the difference between them changes. So the gradient amplitude changes, but the orientation remains 0°; In (c), due to large intensity distortion, the magnitude relationship changes. At this time, the gradient orientation is reversed compared with (a), but notice that it is still on the same line with the orientation in (a). If the gradient orientation is limited to [0°, 180°), the orientation in (c) is still 0°. (d) represents a general situation, which is considered as non-linear distortion of intensity, or some degradation of (a), such as down-sampling or blurring. At this time, the amplitude is hard to determine, but the orientation is still 0° to the right horizontally. A typical multi-source data is shown in Fig.0, which is an optical-infrared image pair. It basically contains the above intensity distortion problems, which is discussed in detail in the next subsection.
%
%In multi-source images, due to the differences in sensors, tempors, environments, etc., various intensity distortions may be caused, resulting in the multi-modal attributes of the image. The morphology of the detailed part of the image is basically in the four cases in Fig.0. In summary, the magnitude of the local gradient varies, but the orientation is basically stable, which defines the feature information that should be focused.


\subsubsection{Partial main orientation map}
Feature-point-based registration algorithms often use a local descriptor to extract the neighborhood information of the keypoints, and generate their feature vectors for similarity matching. For example, SIFT \cite{lowe2004distinctive}, HOG \cite{dalal2005histograms}, SURF \cite{bay2006surf} and PIIFD \cite{chen2010partial} employ gradient information as the basic feature. However, the performance of these algorithms is greatly degraded when processing multi-source, especially multi-sensor images. So, it is critical to extract invariant feature that is robust to ${{\cal F}_{\rm{Int}}}( \bullet )$, ${{\cal F}_{\rm{Rot}}}( \bullet )$, and ${{\cal F}_{\rm{Chg}}}( \bullet )$ for feature points description. A partial main orientation map (PMOM) is designed as the feature map in MS-HLMO, in which the Average Squared Gradient (ASG)\cite{kass1987analyzing} is adopted.

The ASG is a gradient weighting method. The elementary gradient of the image along $x$ and $y$ directions, i.e., ${\bf{G}}_x$ and ${\bf{G}}_y$ are calculated as
\begin{equation} \label{eqPMOM1}
\left[ \begin{array}{l}
{{\bf{G}}_x}(x,y)\\
{{\bf{G}}_y}(x,y)
\end{array} \right] = \left[ \begin{array}{l}
\frac{\partial }{{\partial x}}{\bf{I}}(x,y)\\
\frac{\partial }{{\partial y}}{\bf{I}}(x,y)
\end{array} \right]
\end{equation}
where ${\bf{I}}(x,y)$ represents the single-layer gray-scale image. The magnitude and orientation of its gradient, i.e., ${\bf{G}}_\rho$ and ${\bf{G}}_\varphi$ are
\begin{equation} \label{eqPMOM2}
\left[ \begin{array}{l}
{{\bf{G}}_\rho}\\
{{\bf{G}}_\varphi}
\end{array} \right] = \left[ \begin{array}{l}
\sqrt {{{\bf{G}}_x}^2 + {{\bf{G}}_y}^2} \\
\arctan \frac{{\bf{G}}_y}{{\bf{G}}_x}
\end{array} \right]\end
{equation}

In the ASG, a locally weighted squared gradient \cite{kass1987analyzing} along $x$ and $y$ directions, ${\bf{G}}_{{{\bf{W}}_\sigma},s,x}$ and ${\bf{G}}_{{\bf{W}_\sigma},s,y}$ are obtained as
\begin{equation} \label{eqPMOM3}
\left[ \begin{array}{l}
{{\bf{G}}_{{{\bf{W}}_\sigma},s,x}}\\
{{\bf{G}}_{{{\bf{W}}_\sigma},s,y}}
\end{array} \right] = \left[ \begin{array}{l}
\sum\limits_{{\bf{W}}_\sigma} {{{\bf{G}}_x}^2 - {{\bf{G}}_y}^2} \\
\sum\limits_{{\bf{W}}_\sigma} {2{{\bf{G}}_x}{{\bf{G}}_y}}
\end{array} \right]
\end{equation}
where ${\bf{W}}_\sigma$ is a Gaussian window with variance $\sigma$. Accordingly, the orientation of this gradient is
\begin{equation} \label{eqPMOM4}
{{\bf{G}}_{{{\bf{W}}_\sigma },s,\varphi }} = \angle ({{\bf{G}}_{{{\bf{W}}_\sigma },s,x}},{{\bf{G}}_{{{\bf{W}}_\sigma },s,y}})
\end{equation}
where $\angle (X,Y)$ is defined as
\begin{equation} \label{eqPMOM5}
\angle (X,Y) = \left\{ \begin{array}{l}
\arctan (\frac{Y}{X}), X \ge 0\\
\arctan (\frac{Y}{X}) + \pi , X < 0,Y \ge 0\\
\arctan (\frac{Y}{X}) - \pi , X < 0,Y < 0
\end{array} \right.
\end{equation}
making ${\bf{G}}_{{{\bf{W}}_\sigma },s,\varphi }$ within $(-\pi,\pi)$. According to \cite{kass1987analyzing}, this gradient is obtained by doubling the angle of the original gradient, so the orientation of the ASG is
\begin{equation} \label{eqPMOM6}
{{\bf{G}}_{{{\bf{W}}_\sigma},\varphi }} = \frac{1}{2}{{\bf{G}}_{{{\bf{W}}_\sigma},s,\varphi }}
\end{equation}

Compared with the classical gradient operator, the ASG orientation ${{\bf{G}}_{{{\bf{W}}_\sigma},\varphi}}\in(-\frac{\pi}{2},\frac{\pi}{2})$ reflects the weighted gradient orientation of a local region ${\bf{W}}_\sigma$, which is more robust and stable. In addition, the $x$ direction gradient is constant, and this characteristic meets the requirement that not affected by the reversal of gradient in intensity difference. Note that when $\sigma$ increases, the scale of ASG increases, which makes the local orientation more invariant to intensity difference and noise, but the uniqueness of local features decreases. From this, the following function is defined:
\begin{equation} \label{eqPMOM7}
{{\bf{G}}_{PMOM}} = \frac{1}{2}\angle (\sum\limits_\sigma  {\sum\limits_{{\bf{W}}_\sigma } {{{\bf{G}}_x}^2 - {{\bf{G}}_y}^2}} ,\sum\limits_\sigma  {\sum\limits_{{\bf{W}}_\sigma } {2{{\bf{G}}_x}{{\bf{G}}_y}} } )
\end{equation}
where a series of scale $\sigma$ are preset, the weighted responses ${\bf{G}}_{{{\bf{W}}_\sigma},s,x}$ and ${\bf{G}}_{{{\bf{W}}_\sigma},s,y}$ at each scale are added, and the ASG orientation is obtained. By filtering the image with Eq.(\ref{eqPMOM7}), the PMOM is obtained, where its value reflects the overall orientation of multiple scales in each partial area of the image.

\begin{figure}[h!]
 \begin{center}
  \includegraphics[width=3.5in]{Comparion_of_Feature_Maps.pdf}
  \caption{Comparison of feature maps of a selected scene, including a visible and an infrared image.}
  \label{fig:maps}
 \end{center}
\end{figure}

A visualized comparison of PMOM with other feature maps of typical multi-source data is shown in Fig.\ref{fig:maps}. The original data is a visible-infrared image pair, which contains obvious intensity difference. For comparison purposes, the images have been registered manually, basically eliminating the spatial differences of scale, rotation, and size. The magnitude and orientation of images' gradient are shown in Fig.\ref{fig:maps}, which are obtained using Eqs.(\ref{eqPMOM1})-(\ref{eqPMOM2}). These are the basic feature information in most algorithms \cite{lowe2004distinctive,dalal2005histograms,bay2006surf,chen2010partial,2021Multi}. It is observed that, due to the multi-modal properties of the original image pair, these two feature maps have large differences and instability, which is the main reason for the failure of most algorithms. The MIM \cite{li2019rift} of RIFT shown in Fig.\ref{fig:maps} also focuses on the local orientation of the image, where the maximum index is the main orientation among several ones. Compared with the directional gradient, Gabor transformation has a more stable response, which leads to RIFT robust to intensity difference. However, its value will also mutate due to local changes in images, and the rotation invariance is slightly poor. The image pair's PMOMs are shown at the bottom of Fig.\ref{fig:maps}. Compared with MIM, the proposed PMOM is not only more robust and stable between multi-modal images, but also continuous in value, which is conducive to achieving effective rotation invariance. In HLMO, PMOM is used as the unique feature information to extract local features of keypoints that are invariant to multi-modal properties.


\subsubsection{Descriptor extraction}

After determining the feature points and the specific feature for discrimination, the next step is to make use of the local feature information around each point and generate descriptors. Gradient Location and Orientation Histogram (GLOH) has shown excellent ability through experiments \cite{mikolajczyk2005performance}, and has been applied in multi-source remote sensing image registration \cite{dellinger2014sar,ma2016remote}. The original GLOH descriptor is a circular region divided by three circles, similar to that shown in Fig.\ref{fig:des180}, in which the two outer circular regions are divided into 4 parts, and the radius of the circular region divided are 5, 9, and 11. The partition size and the number are then improved \cite{mikolajczyk2005performance,dellinger2014sar,ma2016remote}. However, different parameters have various effects when treating multifarious types of images. In addition, if the number of outer ring regions is too small, the character of feature points is not significant, which makes it difficult to match accurately. If it is too large, the features are unstable, and the dimension of the descriptor is too high, which increases the burden of redundant calculation. To deal with this, a generalized GLOH-like (GGLOH) descriptor is proposed, and its structure of which is shown in Fig.\ref{fig:des:a}.

\begin{figure}[!h]
    \centering
        \subfloat[]{\includegraphics[width=2.2in]{Descriptor_Structure_a.png}%
        \label{fig:des:a}}
    \hfil
        \subfloat[]{\includegraphics[width=1.1in]{Descriptor_Structure_b.png}%
        \label{fig:des:b}}
    \caption{Descriptor structure of the proposed GGLOH. (a) Subregion partition of the local neighborhood of a feature point. (b) Angle quantification within each subregion.}
    \label{fig:des}
\end{figure}

Let ${{\rm{A}}^0}$ denote the central circular region, and ${\rm{A}}_j^i,(i=1,2, j=1,...,{N_{\rm{A}}})$ represent the sector subregion $j$ in the outer ring region $i$. Let ${N_{\rm{A}}}$ be the number of the subregions in each out ring region, which is even, ${\theta _{\rm{0}}}$ be the main orientation of the feature point, and $R_0$, $R_1$, $R_2$ be the radii of the central and outer regions, respectively. Note that the orientations of pixels’ gradient in each region are counted as feature information, therefore, fair use of information in each region is expected. The number of pixels in each region should be roughly the same, and the weight of the outer regions should not change due to the change of ${N_{\rm{A}}}$. So the area of each region should be the same, that is
\begin{equation} \label{eqggloh}
{N_{\rm{A}}} \cdot \pi {\rm{R}}_{\rm{0}}^2 = \pi ({\rm{R}}_{\rm{1}}^2 - {\rm{R}}_{\rm{0}}^2) = \pi ({\rm{R}}_{\rm{2}}^2 - {\rm{R}}_{\rm{1}}^2)
\end{equation}
which also fixes the relationship between $R_0$, $R_1$, $R_2$ and ${N_{\rm{A}}}$. When ${N_{\rm{A}}}$ is given different values, the stability and importance of each region’s feature remains the same. In HLMO, the GGLOH is used to extract local features on the PMOM, where the orientation values within $(-\frac{\pi}{2},\frac{\pi}{2})$ are uniformly quantified to ${N_{\rm{O}}}$ values, as shown in Fig.\ref{fig:des:b}, where ${\phi_k}(k=1,2,...,{N_{\rm{O}}})$ is the angle after quantization. A histogram with ${N_{\rm{O}}}$ bins is obtained in each region.

It is simple to achieve rotation invariance of HLMO. For each keypoint, the PMOM value at its location is the main orientation, that is, the reference orientation ${\theta _0}$ of the GGLOH. Then, all of the PMOM values within the local area of GGLOH also take ${\theta _0}$ as the reference (0°), that is, all angle values minus ${\theta _0}$, and those beyond $(-\frac{\pi}{2},\frac{\pi}{2})$ are flipped to their opposite angles.

\begin{figure}[h!]
 \begin{center}
  \includegraphics[width=3.3in]{Problem_of_180.pdf}
  \caption{The problem caused by the jump of main orientation near $-\frac{\pi}{2}$ or $\frac{\pi}{2}$.}
  \label{fig:des180}
 \end{center}
\end{figure}

Another key problem is that the rotation and nonlinear intensity difference may cause the jump of the main orientations of some feature points near $-\frac{\pi}{2}$ and $\frac{\pi}{2}$. An example is shown in Fig.\ref{fig:des180}. In PIIFD \cite{chen2010partial}, a similar problem has been discovered and improvement has been made for SIFT. Then a similar strategy is adopted to process GLOH-like descriptors,

\begin{equation}
{{\bf{D}}_1} = \left[ {\begin{array}{*{20}{c}}
{\begin{array}{*{20}{c}}
{{\bf{H}}_1^1}&{{\bf{H}}_2^1}& \cdots &{{\bf{H}}_{{{{N_{\rm{A}}}} \mathord{\left/
 {\vphantom {{{N_{\rm{A}}}} 2}} \right.
 \kern-\nulldelimiterspace} 2}}^1}
\end{array}}\\
{\begin{array}{*{20}{c}}
{{\bf{H}}_1^2}&{{\bf{H}}_2^2}& \cdots &{{\bf{H}}_{{{{N_{\rm{A}}}} \mathord{\left/
 {\vphantom {{{N_{\rm{A}}}} 2}} \right.
 \kern-\nulldelimiterspace} 2}}^2}
\end{array}}
\end{array}} \right]
\end{equation}

\begin{equation}
{{\bf{D}}_2} = \left[ {\begin{array}{*{20}{c}}
{\begin{array}{*{20}{c}}
{{\bf{H}}_{{{{N_{\rm{A}}}} \mathord{\left/
 {\vphantom {{{N_{\rm{A}}}} 2}} \right.
 \kern-\nulldelimiterspace} 2} + 1}^1}&{{\bf{H}}_{{{{N_{\rm{A}}}} \mathord{\left/
 {\vphantom {{{N_{\rm{A}}}} 2}} \right.
 \kern-\nulldelimiterspace} 2} + 2}^1}& \cdots &{{\bf{H}}_{{N_{\rm{A}}}}^1}
\end{array}}\\
{\begin{array}{*{20}{c}}
{{\bf{H}}_{{{{N_{\rm{A}}}} \mathord{\left/
 {\vphantom {{{N_{\rm{A}}}} 2}} \right.
 \kern-\nulldelimiterspace} 2} + 1}^2}&{{\bf{H}}_{{{{N_{\rm{A}}}} \mathord{\left/
 {\vphantom {{{N_{\rm{A}}}} 2}} \right.
 \kern-\nulldelimiterspace} 2} + 2}^2}& \cdots &{{\bf{H}}_{{N_{\rm{A}}}}^2}
\end{array}}
\end{array}} \right]
\end{equation}

\begin{equation} \label{eqdes}
{\bf{D}} = \left[ {\begin{array}{*{20}{c}}
{{{\bf{D}}_1} + {{\bf{D}}_2}}\\
{c\left| {{{\bf{D}}_1} - {{\bf{D}}_2}} \right|}
\end{array}} \right]
\end{equation}
where ${\bf{H}}_j^i$ is the histogram vector of gradient orientation of region ${\rm{A}}_j^i$. In this way, no matter whether the main orientation of feature points is reversed 180° or not, descriptor ${\bf{D}}$ is composed of the addition and subtraction of the upper and lower parts of GGLOH according to the main orientation axis, without changing the regions' order. Finally, a descriptor vector ${\bf{D}}_P$ is generated for the feature point $P$, whose dimension is $(2 \cdot {N_{\rm{A}}}+1) \cdot {N_{\rm{O}}}$.

%\subsection{Advanced Outlier Removal}
%\label{ssec:subhead}


\subsection{Multi-scale Registration Strategy}
\label{ssec:subhead}
Scale difference ${{\cal F}_{{\rm{Sc}}}}( \bullet )$ of multi-source images has a great influence on local features. Some algorithms have quantitative scale judging methods, such as SIFT \cite{lowe2004distinctive} and LHOPC. However, it is found that these methods are invalid in images with large modal differences. The reason is that when images do not belong to the same degradation model, it is not credible to judge the scale quantitatively through local image feature information. Multi-source images often have scale differences, and sometimes the scale proportion is unknown. In order to deal with this key problem and realize scale robustness, a multi-scale feature extraction and matching strategy is designed in MS-HLMO.
%由于传感器成像能力或成像条件的差异,图像显示出了尺度的差异。这种尺度差异往往体现在两个方面,一个是分辨率,一个是模糊程度。其中分辨率是指图像中一个像素对应实际空间的尺寸,另外一个是指

\begin{figure}[h!]
 \begin{center}
  \includegraphics[width=2.5in]{Pyramid.pdf}
  \caption{Structure of the Gaussian pyramid used in MS-HLMO.}
  \label{fig:pyramid}
 \end{center}
\end{figure}

Local information of feature points is extracted in the scale-space of the images. Based on the scale-space theory \cite{lindeberg1994scale}, the method of building image's Gaussian pyramids is adopted. The schematic diagram of establishing Gaussian pyramid of the image in the proposed algorithm is shown in Fig.\ref{fig:pyramid}. The original image is first sampled down step by step to obtain a series of images with different resolutions, that is, the first layer in each octave. Then in each octave, a series of Gaussian blurs are performed:
\begin{equation}\label{eqGauss1}
{\bf{L}} = {\bf{G}} * {\bf{I}}
\end{equation}
\begin{equation}\label{eqGauss2}
{\bf{G}} = \frac{1}{{\sqrt {2\pi {\sigma ^2}} }}{e^{\frac{{-({x^2} + {y^2})}}{{2{\sigma ^2}}}}}
\end{equation}
where ${\bf{I}}$ is the original image, ${\bf{G}}$ is a Gaussian kernel with a standard deviation of $\sigma$, and ${\bf{L}}$ is the Gaussian blur image.
%高斯尺度空间模拟出了图像尺度变化的效果,为获取图像多个尺度层面的特征提供了重要帮助。

\begin{algorithm}[htp]
\caption{\label{multi1} \footnotesize Proposed MS-HLMO Feature Extraction}
\begin{algorithmic}
\footnotesize
\STATE {\bf Input:} single-band image $\mathbf{I}$, feature point set $P_{\mathbf{I}}$, total number of octaves ${N_{\rm{GO}}}$ and layers ${N_{\rm{GL}}}$ in Gaussian pyramid, subregion and angle partition parameters ${N_{\rm{A}}}$, ${N_{\rm{O}}}$ in GGLOH , patch size $S$ of HLMO.
\STATE Through down-sampling and Eq.(\ref{eqGauss1})(\ref{eqGauss2}), the Gaussian pyramid ${\bf{G}}_{\bf{I}}(O,L)$ of image $\mathbf{I}$ is built with ${N_{\rm{GO}}}$ octaves and ${N_{\rm{GL}}}$ layers in each octave.
\STATE In each layer of ${\bf{G}}_{\bf{I}}(O,L)$:
\STATE \hspace*{0.1in}Calculate the PMOM of this layer to get ${\bf{F}}_{\bf{I}}(O,L)$ according to Eq.(\ref{eqPMOM1})(\ref{eqPMOM7})
\STATE \hspace*{0.1in}For each feature point $P$ in $P_{\bf{I}}$:
\STATE \hspace*{0.2in}Calculate the corresponding position
\STATE \hspace*{0.2in}Take the PMOM value at the position as the main orientation ${\theta_0}$
\STATE \hspace*{0.2in}Taking the main orientation as the reference direction ($0^{\circ}$), establish a GGLOH window with size (diameter) of S
\STATE \hspace*{0.2in}Statistics the PMOM value within each region of GGLOH to obtain the basic feature descriptor $D_{1}(P,O,L)$ and $D_{2}(P,O,L)$
\STATE \hspace*{0.2in}Obtain the descriptor $D(P,O,L)$ of $P$ with Eq.\ref{eqdes}.
\STATE {\bf Output:} feature descriptor set $D_{P_{\bf{I}}}(O,L)$
\end{algorithmic}
\end{algorithm}

After the scale-space of the images is established, for each Harris corner point, the HLMO descriptor is calculated by obtaining the local information at the corresponding location of each feature point in the scale-space. The proposed multi-scale HLMO feature extraction method is provided in Algorithm 1, where $O$ is the octave number in the Gaussian pyramid, and $L$ is the layer number. The algorithm outputs the feature point descriptor set $D_{P_{\bf{I}}}(O,L)$, which contains $(2 \cdot {N_{\rm{A}}}+1) \cdot {N_{\rm{O}}}$-dimensional vectors for each feature point at each scale.

The next step is to match the feature point sets of the image pair according to the descriptor sets. The process of the multi-scale feature matching is provided in Algorithm 2. In the process, each scale is matched in turn. Then the matching results are merged step by step while the outlier removal is carried out to realize the optimization of matching points. The final matching results are used to determine the spatial transformation between images. The most critical is to combine all the matching results of feature points and remove outliers, so as to maximize the correct matches of all scales. Fig.\ref{fig:pyramids} shows this process visually. Obviously, this is a general approach to handle all kinds of unknown scale differences. When the scale proportion of images is known or can be estimated, then the above process can be greatly simplified. In this case, the proposed multi-scale strategy still has the advantages of enhancing feature matching and maximizing the number of matching points.

\begin{algorithm}[htp]
\caption{\label{multi2} \footnotesize Proposed MS-HLMO Feature Matching}
\begin{algorithmic}
\footnotesize
\STATE {\bf Input:} feature point set of the image pair $P_{{\bf{I}}1}$, $P_{{\bf{I}}2}$, feature descriptor set of the image pair $D_{P_{{\bf{I}}1}}(O_{1},L_{1})$, $D_{P_{{\bf{I}}2}}(O_{2},L_{2})$
\STATE Take each layer of $D_{P_{{\bf{I}}1}}(O_{1},L_{1})$:
\STATE \hspace*{0.1in}Take each layer of $D_{P_{{\bf{I}}2}}(O_{2},L_{2})$:
\STATE \hspace*{0.2in}Match $P_{{\bf{I}}1}$ and $P_{{\bf{I}}2}$ using Euclidean distance of the descriptorss
\STATE \hspace*{0.2in}Remove outliers, producing the matching result of a single scale $M(O_{1},O_{2},L_{1},L_{2})$
\STATE The matching results of all layers in each octave of the scale-space are union and then optimized with outlier removal, producing the matching result $M_{L}(O_{1},O_{2})$
\STATE The matching results of all octaves in $M_{L}(O_{1},O_{2})$ are union and then optimized with outlier removal, producing the final matching result $M_{OL}(P_{{\bf{I}}1},P_{{\bf{I}}2})$
\STATE {\bf Output:} feature points matching set $M_{OL}(P_{{\bf{I}}1},P_{{\bf{I}}2})$
\end{algorithmic}
\end{algorithm}

\begin{figure}[h!]
 \begin{center}
  \includegraphics[width=3.5in]{MS_Matching.pdf}
  \caption{Multi-scale keypoints matching strategy in MS-HLMO.}
  \label{fig:pyramids}
 \end{center}
\end{figure}



\subsection{Experimental Settings}
We evaluated the proposed models on reduced SUN RGB-D and Places365 dataset. To be noticed, aim to investigate the generalizability of our model, we evaluate our model pretrained on Places365-7 dataset on the SUN RGB-D dataset without retraining it. We'll introduce the implementation details and training procedure and different experiment settings.

\subsubsection{Implementation Details}
For the PlacesCNN model, ResNet18 or ResNet50 architecture is adopted in our experiment for ablation study. The optimizer used is the Stochastic Gradient Descent (SGD) with an initial learning rate of 0.01, the momentum of 0.9, and the weight decay of 0.0001. We decrease the learning rate 10 times every 10 epoch, and every time when updating the learning rate, we reload the parameters which have the best accuracy before this timestamp. The total number of epoch during training is 40. To be noticed, we use the training sets of Places365-7 and Places365-14 dataset for learning the BORM statistically, while testing on the SUN RGB-D dataset, where the BORM is the same as the Places365-7 dataset. 


\subsubsection{Dataset Settings}

\begin{table}[]
\centering
	\caption{Dataset split setting, where the number of training set and number of test set are listed below.}
	\label{tab:dataset_split}
	\begin{tabular}{l|l|l}
		\hline
		Dataset      & Training & Test  \\ \hline
		Places365-7  &  35000   & 701     \\ 
		Places365-14 &  75000   & 1500    \\ 
		SUN RGB-D    &  35000(from Places365-7)        & 2077    \\ \hline
	\end{tabular}
\end{table}
\ \\
\textbf{Places365 Dataset:}
In this paper, we use the reduced Places365 \cite{zhou2017places}  dataset to test our methods, since it is the most largest and challenging scene classification dataset yet, and it contains broad categories in the indoor environment. In the experiment, we only consider the indoor scene recognition. There are two different settings on the reduced Places365 dataset. The one is Places365 with 7 classes includes Corridor, Dinning Room, Kitchen, Living Room, Bedroom, Office, and Bathroom, denoted as Places365-7.  The test set setting is the same as the official dataset and described in \cite{pal2019deduce}.
In addition, we use the reduced Places365 with 14 indoor scenes in Home environment includes Wet bar, Home theater, Balcony, Closet, Kitchen, Bedroom, Playroom, Laundromat, Bathroom, Living Room, Home office, Dining room, Staircase, and Garage denoted as Places365-14. The dataset splitting follows the same setting as described in \cite{chen2019scene}. The dataset splition can be seen from Table. \ref{tab:dataset_split}.

\textbf{SUN RGB-D Dataset:}
SUN RGB-D dataset \cite{song2015sun} is a challenging dataset for scene understanding that contains not only RGB images but also depth information of each image. It contains 3784 images collected by Kinect V2 and 1159 collected by Intel RealSense. Moreover, it incorporates 1449 images from the NYUDepth V2 \cite{silberman2012indoor}, and 554 images from the Berkeley B3DO Dataset \cite{janoch2013category}, both captured by Kinect V1. Finally, it takes 3389 manually selected distinguished frames without significant motion blur from the SUN3D videos \cite{xiao2013sun3d} captured by Asus Xtion.

In our experiment, we mainly consider the indoor environment understanding. Therefore, we use the reduced SUN RGB-D dataset includes Office, Kitchen, Bedroom, Corridor, Bathroom, Living room, and Dining room, where the test set split is the same as the official dataset. There are 3741 RGB images in total for testing. And we test our model pretrained on the Place365-7 on SUN RGB-D dataset without retraining.



\subsection{Experimental Results}
\subsubsection{Effect of Object Knowledge}
As illustrated in Fig. \ref{fig:om_ablation}, a group of ablation studies have been conducted for evaluating the effect of object knowledge to indoor scene recognition on the Places365-14, Places365-7, and SUN RGB-D dataset, respectively. The x-axis represent the number of object information IOM have about the indoor scene and is added by 20 from 90 to 150 sequentially selected from the vector. Plus, IOM-80 is the baseline accuracy. Obviously, as the number of object information increases, the much better scene recognition accuracy is achieved, e.g., on the Places365-14 dataset, the IOM-150 reaches 74.1\% accuracy, which is \textbf{10.0\%} higher than IOM-80. This improvement shows the number of object information is proportional to scene recognition accuracy. Moreover, the comparison experiments between the IOM and OM on three datasets, shows an average of \textbf{20.5\%} improvements can be achieved. After analyzing the object categories of OM pretrained on the MS COCO, we observed there are only half of the object categories are related to indoor scenes. In contrast, the other half is related to outdoor scenes. Therefore, the relevance of object categories of object model with the scenes is essential for scene recognition, e.g., the information of elephant and giraffe in OM will not be valuable for indoor scene recognition. Similarly, the bus and train are not beneficial to indoor scene recognition.

\subsubsection{{Analysis of BORM}}

\begin{figure}[tbp]
	\centering
	\includegraphics[width=0.4\textwidth]{Fig//om_abl.jpg}
	\caption{Ablation study of improved object model (IOM) with different number of object knowledge ranging from 80 to 150 categories as shown in horizontal axis. The vertical axis shows the accuracy on percentage.}
	\label{fig:om_ablation}
\end{figure}

%\begin{table*}[]
%	\centering
%	\scriptsize
%	\caption{Ablation study of accuracy on percentage of improved object model (IOM) with different number of object knowledge ranging from 80 to 150 categories.}
%	\label{tab:om_ablation}
%	\begin{tabular}{l|c|c|cccccccc}
%		\hline
%		\textbf{} & {$\Phi_{obj}$ \cite{pal2019deduce}} & OM-80 & {IOM-80} & {IOM-90} & {IOM-100} & {IOM-110} & {IOM-120} & {IOM-130} & {IOM-140} & {IOM-150} \\ \hline 
%		{Places365-14} & - & 47.0 & 64.1 & 64.4 & 66.1 & 71.5 & 71.9 & 72.5 & 73.7 & \textbf{74.1} \\ 
%		{Places365-7} & 62.6 & 73.0 & 80.9 & 81.6 & 81.6 & 81.9 & 81.7 & 82.0 & 82.1 & \textbf{82.6} \\ 
%		{SUN RGBD} & 53.6 & 59.2 & 65.9 & 66.8 & 66.7 & 67.0 & 66.8 & 67.3 & 67.8 & \textbf{68.1} \\ \hline
%	\end{tabular}
%\end{table*}


%\begin{table*}[htbp]
%\centering
%\scriptsize
%\caption{Scene recognition accuracy in percentage on the reduced Places365-7 dataset}
%\label{tab:place365_7}
%\begin{tabular}{cccccccccc}
%\hline
%            & ResNet18 & ResNet50 & $\Phi_{obj}$ \cite{pal2019deduce} & OM  & IOM  & BORM  & CBORM \\ \hline
%Bathroom    & 87 & 94 & 65 & 71  & 87      &  88     &  95   \\
%Bedroom     & 82 & 83 & 74 & 84  & 92      &  92     &  81   \\
%Corridor    & 96 & 93 & 90 & 92  & 91      &  89     &  95   \\
%Dining room & 81 & 71 & 94 & 74  & 85      &  80     &  93   \\
%Kitchen     & 83 & 84 & 62 & 65  & 73      &  83     &  94   \\
%Living room & 55 & 66 & 25 & 66  & 73      &  72     &  92   \\
%Office      & 79 & 88 & 29 & 58  & 76      &  78     &  81   \\ \hline
%Avg         & 80.4 & 82.7 & 62.6 & 72.9   &  82.4   & \textbf{83.1}  & \textbf{90.1}    \\ \hline
%\end{tabular}
%\end{table*}



We conduct experiments for BORM and IOM in the reduced Places365-7 dataset, and results are displayed in Table \ref{tab:place365_7_sota}. The experiment results show the BORM and IOM model has an advantage over the OM and yields an average accuracy of 83.1\% and 82.4\%, surpassing the OM model about \textbf{20\%} accuracy. The experiment result proves that with more object knowledge about the surrounding environment, the greater scene recognition accuracy can be reached.  Then, we test the model pretrained on the Places365-7 dataset on the SUN RGB-D test set, and similar conclusion can be drawn. Moreover, the BORM outperforms the IOM with 0.7\% and 1.1\% accuracy on the Places365-7 and SUN RGB-D dataset, respectively, which validates the knowledge of the co-occurrences between object pairs and their probabilistic relation forms an important indoor scene representation.


Similarly, in the Table \ref{tab:place365_14_sota}, we have conducted experiments on the reduced Places365-14 dataset, and experiment results show the IOM and BORM tremendously improves the performance over OM with \textbf{27\%} accuracy. The results suggest the effectiveness of BORM and IOM over the OM especially when the number of scene classes of dataset is large.



\subsubsection{Performance Comparison}


\begin{table}[htbp]
	\centering
	\scriptsize
	\begin{minipage}[t]{.44\textwidth}
		\centering
		\caption{Comparison with the state-of-the-art methods on the reduced Places365-7 Dataset and SUN dataset of scene recognition accuracy}
		\label{tab:place365_7_sota}
		\begin{tabular}{c|ccc}
			\hline
			Method     & Config   & Acc(Places365-7) & Acc(SUN) \\ \hline
			\multirow{2}{*}{PlacesCNN \cite{zhou2017places}}     & ResNet18 & 80.4  & 63.3  \\
			& ResNet50 & 82.7 & 67.2   \\ \hline
			\multirow{3}{*}{Deduce \cite{pal2019deduce}}     &  $\Phi_{obj}$ (OM)      & 62.6 & 53.6      \\
			& $\Phi_{scene}$    & 87.3 & 66.8    \\
			& $\Phi_{comb.}$ & 88.1 & 70.1    \\ \hline
			%\multirow{1}{*} {Baseline}
			%& OM   & 72.9 & 59.2   \\   \hline 
			\multirow{3}{*}{Ours} 	
			& IOM  & 82.4 & 68.1   \\
			& BORM & 83.1 & 69.2   \\ \cdashline{2-4}
%			& CIOM & 90.1 & 71.8   \\ 
			& CBORM& \textbf{90.1} & \textbf{72.1\%}   \\	\hline
		\end{tabular}
	\end{minipage}

	\hspace{1cm}

	\begin{minipage}[t]{.44\textwidth}
		\centering
		\caption{Comparison with the state-of-the-art methods on the reduced Places365-14 Dataset of scene recognition accuracy, the * indicates the re-implement of the method. }
		\label{tab:place365_14_sota}
		\begin{tabular}{c|cc}
			\hline
			Method                    & Config           & Acc \\ \hline
			\multirow{2}{*}{PlacesCNN \cite{zhou2017places}}     & ResNet18 & 76.0  \\
			& ResNet50 & 80.0   \\ \hline
			\multirow{1}{*}{Word2Vec \cite{chen2019scene}} 
			%& ResNet50         & 83.5    \\
			& ResNet50+Word2Vec         & 83.7    \\ \hline
%			\multirow{1}{*}{Deduce \cite{pal2019deduce}}     & Obj(re-implement)      & 47.0     \\ \hline
            \multirow{1}{*}{{*}Deduce \cite{pal2019deduce}} 
            
            & $\Phi_{obj}$ (OM) 		   & 47.0	 \\ \hline
			\multirow{3}{*}{Ours}  
			& IOM          & 74.1    \\
			& BORM         & 74.9    \\ \cdashline{2-3}
%			& CIOM & 85.5    		\\
			& CBORM & \textbf{85.8}				\\	\hline
		\end{tabular}
	\end{minipage}
\end{table}



As shown in Table \ref{tab:place365_7_sota}, we conduct the ablation study of using only the BORM model and the CBORM model. We found the CBORM yield an average accuracy of \textbf{90.1\%}, which greatly outperforms the ResNet18 and ResNet50 of PlacesCNN baselines about \textbf{10\%} and \textbf{8\%} respectively. Meanwhile, CBORM outperforms the BORM with 7\% accuracy and 3\% on the Places365-7 and SUN RGB-D dataset, respectively.

%Also, it surpass the Scene (ResNet18 model that pretrained on Places365 and finetuned on the Places365-7 dataset). 

In comparison with the state of the art, Table \ref{tab:place365_7_sota} shows that the CBORM improves the scene recognition by \textbf{2.0\%} in the reduce Places365-7 dataset.  Also, as shown in Table \ref{tab:place365_14_sota}, the CBORM improve the recognition accuracy by \textbf{2.1\%} in the reduced Places365-14 dataset. Both results show the combined model achieves comparable results to some recent approaches that use the word-embedding method to extract the semantic meaning of the environment, or use the combination of scene and object representations for better scene understanding. Moreover, Table \ref{tab:place365_7_sota} shows the performance of our method over the method in \cite{pal2019deduce} with \textbf{2\%}, showing the excellent generalization ability of CBORM over other methods on the Reduced SUN RGB-D dataset.

These results demonstrate that CBORM is successful in recognizing the scene images with a competitive accuracy. This improved effectiveness of CBORM over the state-of-the-art justifies our reasonable assumption that relation of object pairs is an essential complementary information for indoor scene recognition.





\section{Conclusion}
In this work, we present a novel single-stage contrastive learning framework for semi-supervised domain adaptation. The framework consists of Inter-Domain Contrastive Alignment and Instance-Contrastive Alignment, where the former maximizes the similarity between centroids of the same class from both domains and later maximizes the similarity between augmented views of the unlabeled target images. We show that both of the components of the framework are necessary for improved performance. We demonstrate the effectiveness of our approach on three standard domain adaptation benchmark datasets, outperforming the well-known SSDA methods.

\section{Acknowledgments and Disclosure of Funding}
The work is supported by Half-Time Research Assistantship (HTRA) grants from the Ministry of Education, India. We would also like to thank Saurav Chakraborty and  Athira Nambiar for their valuable suggestions and feedback to improve the work.
% \newpage 
\printbibliography
\nocite{*}

\newpage
\appendix




Here we prove Lemma \ref{lem:cotapower} that gives a large deviation result for sums of independent heavy tailed random variables. The lemma is a more precise version of the results in \cite{Omelchenko2019} (although here we treat only with positive random variables which are bounded away from zero), and the proof follows closely what was done there.

\begin{lemma}
Let $S_m=\sum_{i=1}^m Z_i$ where $\{Z_i\}_{i\in\NN}$ is a sequence of i.i.d. absolutely continuous random variables taking values in $[1,\infty)$ such that there are $V,\gamma>0$ with
\[1-F_Z(x)=\PP(Z_i\geq x)\leq Vx^{-\gamma}\] 
for all $x>0$. Then there are $\auxc,\auxl>0$ depending on $V$, $\gamma$ and $\EE(Z_1)$ (if it exists) alone such that:
\begin{itemize}
    \item If $\gamma<1$, then for all $L>\auxl$,
    $\displaystyle\PP(S_m\geq Lm^{\frac{1}{\gamma}})\leq \auxc L^{-\gamma}$.
    \item If $\gamma=1$, then for all $L>\auxl$, $\displaystyle\PP(S_m\geq Lm\log(m))\leq \left(\frac{\auxc}{L\log(m)}\right)^{1-\frac{\auxl}{L}}$.
    \item If $\gamma>1$, then for all $L>\auxl$,
    $\displaystyle\PP(S_m\geq Lm)\,\leq\,\auxc L^{-\gamma}m^{-((\gamma-1)\wedge\frac{\gamma}{2})}$.
\end{itemize}
\end{lemma}
\begin{remark}
We remark that tighter results can be obtained for $\gamma \ge 1$ below, but the current result is sufficient for our purposes.
\end{remark}
\begin{proof}
We proceed as in \cite{Omelchenko2019} by assuming first that $\gamma\leq 1$. Fix $x>0$ and define the event $B=\{\forall 1\leq i< m,\,Z_i\leq x\}$ so that 
\[
    \PP(S_m\geq x)\;\leq\;\PP(\overline{B})+\PP(S_m\geq x\,|\,B)\PP(B)\;\leq\;mVx^{-\gamma}+\PP(S^{(x)}_m\geq x)\PP(B)
\]
where $S^{(x)}$ is the sum of $m$ i.i.d. random variables with c.d.f. $\frac{F_Z(y)}{F_Z(x)}$ for any $y \in [0,x]$. We can thus use a Chernoff bound to deduce
\[
    \PP(S_m\geq x)\;\leq\;mVx^{-\gamma}+e^{-\lambda x}\EE\big(e^{\lambda S^{(x)}_m}\big)\PP(B)\;=\;mVx^{-\gamma}+e^{-\lambda x}\left(\int_1^xe^{\lambda y}dF_Z(y)\right)^m,
\]
where we used independence of the random variables to conclude that $\PP(B)=(F_Z(x))^m$. Define now $M=\frac{2\gamma}{\lambda}$ for some $\lambda$ to be chosen below so that $M<x$ holds. Hence,
\[R(\lambda,x)\,:=\,\int_1^xe^{\lambda y}dF_Z(y)\,=\,\int_1^Me^{\lambda y}dF_Z(y)\,+\,\int_M^xe^{\lambda y}dF_Z(y),\]
which we bound separately. First notice that for some constant $C_1=C_1(\gamma, V)$ we have
\begin{align*}
    \int_1^Me^{\lambda y}dF_Z(y)&\leq\,e^{\lambda M}F_Z(M)-\lambda\int_1^Me^{\lambda y}F_Z(y)dy\\[3pt]
    &\leq\,e^{\lambda M}F_Z(M)-e^{\lambda M}+1+\lambda\int_1^Me^{\lambda y}(1-F_Z(y))dy\\[3pt]
    &\leq\,e^{\lambda M}F_Z(M)-e^{\lambda M}+1+\lambda Ve^{\lambda M}\int_1^My^{-\gamma}dy\;\leq\,1+C_1Q(\lambda),
\end{align*}
where $Q(\lambda)=\lambda^{\gamma}$ if $\gamma<1$ and $Q(\lambda)=-\lambda\log(\frac{\lambda}{2\gamma})$ if $\gamma=1$. For the integral between $M$ and $x$ observe that
\begin{align*}
    \int_M^xe^{\lambda y}dF_Z(y)&\leq\,e^{\lambda M}(1-F_Z(M))+\lambda\int_M^xe^{\lambda y}(1-F_Z(y))dy\\[3pt]
    &\leq\,Ve^{\lambda M}M^{-\gamma}+\lambda V\int_M^xe^{\lambda y}y^{-\gamma}dy\\[3pt]
    &=\,Ve^{2\gamma}\left(\frac{\lambda}{2\gamma}\right)^{\gamma}+ Ve^{\lambda x}x^{-\gamma}\int_0^{\lambda(x-M)}e^{-w}\left(1-\frac{w}{\lambda x}\right)^{-\gamma}dw,
\end{align*}
where in the last line we used the change of variables $w=\lambda(x-y)$. Now, since $M = 2\gamma/\lambda$, the function $f(w)=e^{\frac{w}{2}}(1-\frac{w}{\lambda x})^{\gamma}$ has a positive derivative for $w \in [0, \lambda(x-M)]$. Hence for $w\in[0,\lambda(x-M)]$ we have  $(1-\frac{w}{\lambda x})^{-\gamma}\leq e^{w/2}$  and hence the last integral is therefore smaller than $\int_0^\infty e^{-w/2}dw=2$, giving
\[\int_M^xe^{\lambda y}dF_Z(y)\,\leq\,Ve^{2\gamma}\left(\frac{\lambda}{2\gamma}\right)^{\gamma}+ 2Ve^{\lambda x}x^{-\gamma}\,=\,C_2\lambda^{\gamma}+C_3e^{\lambda x}x^{-\gamma}\]
for some constants $C_2, C_3$ depending only on $V$ and $\gamma$. Putting together both bounds for $R(\lambda,x)$ we arrive at
\begin{align*}\PP(S_m\geq x)&\leq\;mVx^{-\gamma}+e^{-\lambda x}\left(1+C_1Q(\lambda)+C_2\lambda^{\gamma}+ C_3e^{\lambda x}x^{-\gamma}\right)^m\\[3pt]
&\leq\;mVx^{-\gamma}+\exp\left(-\lambda x+mC_1Q(\lambda)+mC_2\lambda^{\gamma}+ mC_3e^{\lambda x}x^{-\gamma}\right).
\end{align*}
Our aim at this point to choose $\lambda$ such that the term on the right is small, which is achieved when taking
\[\lambda=\frac{1}{x}\log\left(\frac{x^{\gamma}}{m}\right),\]
so that $me^{\lambda x}x^{-\gamma}=1$. Assume first that $\gamma<1$ so $x=Lm^{\frac{1}{\gamma}}$ for $L$ large, for which $\lambda=\frac{\gamma\log(L)}{Lm^{\frac{1}{\gamma}}}$ is small, while $\lambda x=\gamma\log(L)$ is large so the assumption $M<x$ is justified. Now, since $\gamma<1$, $Q(\lambda)=\lambda^{\gamma}$ and hence we have
\[-\lambda x+mC_1Q(\lambda)+mC_2\lambda^{\gamma}+ mC_3e^{\lambda x}x^{-\gamma}\,=\,-\gamma\log(L)+(C_1+C_2)\left(\frac{\gamma\log(L)}{L}\right)^{\gamma}+C_3,\]
and since we are assuming $L$ large, we have $(\frac{\gamma\log(L)}{L})^{\gamma}\leq 1$ which finally gives
\begin{align*}\PP(S_m\geq x)&\leq\;mVx^{-\gamma}+\exp\left(-\lambda x+mC_1Q(\lambda)+mC_2\lambda^{\gamma}+ mC_3e^{\lambda x}x^{-\gamma}\right)\\[3pt]
&\leq\;VL^{-\gamma}+\exp\left(-\gamma\log(L)+C_1+C_2+C_3\right)\;=\;\auxc L^{-\gamma},
\end{align*}
which proves the first point of the theorem. Suppose now that $\gamma=1$ so that $x=Lm\log(m)$ for $L\geq \auxl$ for some $\auxl$ large, and hence $\lambda=\frac{1}{Lm\log(m)}\log(L\log(m))$ is small, while $\lambda x=\log(L\log(m))$ is large, so again the assumption $M<x$ is justified. For this choice of $\gamma$ we have $mQ(\lambda)=-m\lambda\log(\frac{\lambda}{2})$ which we can bound as 
\begin{align*}
-m\lambda\log(\tfrac{\lambda}{2})&=\frac{\log(L\log(m))}{L\log(m)}\log\left(\frac{2Lm\log(m)}{\log(L\log(m))}\right)\\[3pt]&\leq\frac{(\log(2L\log(m))^2}{L\log(m)}+\frac{\log(L\log(m))}{L}\leq C_4+\frac{\log(L\log(m))}{L}
\end{align*}
for some constant $C_4$, and hence we arrive at
\begin{align*}\PP(S_m\geq x)&\leq\;mVx^{-1}+\exp\left(-\lambda x+mC_1Q(\lambda)+mC_2\lambda+ mC_3e^{\lambda x}x^{-1}\right)\\[3pt]
&\leq\;\frac{V}{L\log(m)}+C_5\exp\Big({-}\log(L\log(m))+\frac{C_1}{L}\log(L\log(m))\Big)\;\leq\;\Big(\frac{\auxc}{L\log(m)}\Big)^{1-\frac{\auxl}{L}}
\end{align*}
for some constant $C_5$, and where the last inequality holds by choosing $\auxl$ larger than $C_1$ and also by choosing $\auxc$ large enough.

Suppose now that $\gamma>1$ so that $E_0:=\EE(Z_1)$ exists. In this case we can perform a similar computation to the one before to deduce that 
\[
    \PP(S_m-mE_0\geq x)\;\leq\;mVx^{-\gamma}+e^{-\lambda x}\Big(e^{-\lambda E_0}\int_1^xe^{\lambda y}dF_Z(y)\Big)^m,
\]
and we can divide the integral $\int_1^xe^{\lambda y}dF_Z(y)$ as before so that
\[\int_1^xe^{\lambda y}dF_Z(y)\,=\,\int_1^Me^{\lambda y}dF_Z(y)+\int_M^xe^{\lambda y}dF_Z(y)\]
where again $M=\frac{2\gamma}{\lambda}$ (and for our choice of small $\lambda$ below again we have $M < x$). Now, the main difference in this case is the treatment of the first term, for which we have
\begin{align*}
    \int_1^Me^{\lambda y}dF_Z(y)&=\,\int_1^MdF_Z(y)+\lambda\int_1^MydF_Z(y)+\int_1^M\left(e^{\lambda y}-1-\lambda y\right)dF_Z(y)\\[3pt]&\leq\,1+\lambda E_0-\left(e^{\lambda y}-1-\lambda y\right)(1-F_Z(y))\bigg|^M_1+\lambda\int_1^M\left(e^{\lambda y}-1\right)(1-F_Z(y))dy\\[3pt]&\leq 1+\lambda E_0+\left(e^{\lambda }-1-\lambda\right)+\lambda V\int_1^M\left(e^{\lambda y}-1\right)y^{-\gamma}dy\\[3pt]&\leq 1+\lambda E_0+\left(e^{\lambda }-1-\lambda\right)+\frac{\lambda V}{\gamma-1}\left(e^{\lambda}-1\right)+\frac{\lambda^2 V}{\gamma-1}e^{\lambda M}\int_1^My^{1-\gamma}dy\\[3pt]&\leq 1+\lambda E_0+\lambda^2+\frac{2\lambda^2 V}{\gamma-1}+C_1 W(\lambda),
\end{align*}
for some value $C_1$ depending on $\gamma$ alone, where we used that $\gamma>1$, that $\lambda$ is small, but also $\lambda M=2\gamma$, and where 
\[W(\lambda)=\left\{\begin{array}{cl}\lambda^{\gamma}&\text{ if }\gamma<2\\[3pt]-\lambda^2\log(\lambda)&\text{ if }\gamma=2\\[3pt]\lambda^2&\text{ if }\gamma>2\end{array}\right.\]
Since $\lambda$ is small we conclude that the $W(\lambda)$ is at least of the same order as the terms containing $\lambda^2$ and hence
\[\int_1^Me^{\lambda y}dF_Z(y)\;\leq\;1+\lambda E_0+3W(\lambda).\]
Treating the integral $\int_M^xe^{\lambda y}dF_Z(y)$ as in the case $\gamma\leq 1$ we finally obtain 
\begin{align*}\PP(S_m-mE_0\geq x)&\leq\;mVx^{-\gamma}+e^{-\lambda x-\lambda mE_0}\left(1+\lambda E_0+3W(\lambda)+C_2\lambda^{\gamma}+ C_3e^{\lambda x}x^{-\gamma}\right)^m\\[4pt]
&\leq\;mVx^{-\gamma}+\exp\left(-\lambda x+mC_4W(\lambda)+ mC_3e^{\lambda x}x^{-\gamma}\right).
\end{align*}
Now, since we are interested in the probability $\PP(S_m\geq Lm)$ for $L$ larger than some $\auxl$ which we can take larger than $2E_0$ we have
\[\PP(S_m\geq Lm)\,\leq\,\PP(S_m-mE_0\geq Lm/2),\]
and hence we can take $x=Lm/2$. For $\gamma<2$ we choose $\lambda=\frac{1}{x}\log(\frac{x^\gamma}{m})$ as before (which is small) for which $mC_3e^{\lambda x}x^{-\gamma}=C_3$ and hence
\[\PP(S_m-mE_0\geq x)\;\leq\;2^\gamma VL^{-\gamma}m^{1-\gamma}+\exp\left(-\log(L^\gamma m^{\gamma-1}/2^{\gamma})+3C_5m\lambda^\gamma+ C_3\right),  \]
but $m\lambda^\gamma=\frac{2^\gamma\log^\gamma(L^\gamma m^{\gamma-1}2^{-\gamma})}{L^\gamma m^{\gamma-1}}\leq 1$ for $L^\gamma m^{\gamma-1}$ large enough, and hence
\[\PP(S_m-mE_0\geq x)\;\leq\;\auxc L^{-\gamma}m^{1-\gamma}.\]
Suppose now that $\gamma\geq 2$ and choose $\lambda=\frac{\gamma}{x}\log(\frac{x}{\sqrt{m}})$ for which we have $mC_3e^{\lambda x}x^{-\gamma}=C_3m^{1-\frac{\gamma}{2}}\leq C_3$, giving 
\[\PP(S_m-mE_0\geq x)\;\leq\;2^\gamma VL^{-\gamma}m^{1-\gamma}+\exp\left(-\log(L^\gamma m^{\frac{\gamma}{2}}/2^{\gamma})+C_6mW(\lambda)+ C_3\right).\]
Now, if $\gamma=2$, then $W(\lambda)=\lambda^2\log(1/\lambda)$ so for some constant $C_7$
\[mW(\lambda)=\frac{16}{L^2m}\log^2(\tfrac{L \sqrt{m}}{2})\log\left(\frac{Lm}{4\log(L\sqrt{m}/2)}\right)\leq\frac{C_1}{L^2m}\log^3(L^2m)\leq 1\]
for large $L^2m$, while if $\gamma>2$, $W(\lambda)=\lambda^2$, and so 
\[mW(\lambda)=\frac{2\gamma^2}{L^2m}\log^2(\tfrac{L \sqrt{m}}{2})\leq 1\]
for large $L^2m$. In any case scenario, we obtain
\[\PP(S_m-mE_0\geq x)\;\leq\;2^\gamma VL^{-\gamma}m^{1-\gamma}+\auxc L^{-\gamma}m^{-\frac{\gamma}{2}},\]
but for $\gamma\geq 2$ we have $\frac{\gamma}{2}\leq\gamma-1$ and hence the second term dominates the first, giving the result.
\end{proof}

\end{document}
