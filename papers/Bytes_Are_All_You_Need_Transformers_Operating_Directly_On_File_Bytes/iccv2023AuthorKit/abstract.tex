Modern deep learning approaches usually transform inputs into a modality-specific form. For example, the most common deep learning approach to image classification involves decoding image file bytes into an RGB tensor which is passed into a neural network. Instead, we investigate performing classification directly on file bytes, without the need for decoding files at inference time. Using file bytes as model inputs enables the development of models which can operate on multiple input modalities. Our model, \emph{ByteFormer}, achieves an ImageNet Top-1 classification accuracy of $77.33\%$ when training and testing directly on TIFF file bytes using a transformer backbone with configuration similar to DeiT-Ti ($72.2\%$ accuracy when operating on RGB images). Without modifications or hyperparameter tuning, ByteFormer achieves $95.42\%$ classification accuracy when operating on WAV files from the Speech Commands v2 dataset (compared to state-of-the-art accuracy of $98.7\%$). Additionally, we demonstrate that ByteFormer has applications in privacy-preserving inference. ByteFormer is capable of performing inference on particular obfuscated input representations with no loss of accuracy. We also demonstrate ByteFormer's ability to perform inference with a hypothetical privacy-preserving camera which avoids forming full images by consistently masking $90\%$ of pixel channels, while still achieving $71.35\%$ accuracy on ImageNet. Our code will be made available at \url{https://github.com/apple/ml-cvnets/tree/main/examples/byteformer}.