\section{Introduction}

The chore division problem is the problem of fairly allocating a divisible undesirable object among $n$ players. Imagine that we want to divide up a job among some players. There are many ways to assign this job to them, especially when players have a different evaluation of cost for each part of the job. For example, one might prefer doing certain tasks and other may not be good at them. This gives rise to an important question: How can one fairly assign a task to others?

The problem of fairly allocating an undesirable object, called chore, was first introduced by \citeboth {gardner1978aha} in the 1970s. Many definitions of fairness have been proposed for the chore division. The most important ones are \textit{proportionality} and \textit{envy-freeness}. An allocation of the chore is \textit{proportional} if everyone receives at most $1/n$ of the chore in his perspective. The other definition of fairness is envy-freeness. An allocation is \textit{envy-free} if each player receives a part that he thinks is the smallest. 
%everyone thinks what allocated to him costs less than others'.

The chore division is the dual problem of the well-studied cake cutting problem in which we want to fairly allocate a divisible good among players. This problem was introduced in the 1940s, and popularized by \citeboth{robertson1998cake}. The same criteria of fairness can also be defined for this problem. For the case of two players, the simple \textit{cut-and-choose} protocol provides both proportionality and envy-freeness. In this protocol, one player cuts the cake into two equally preferred pieces, and the other chooses the best piece.

Despite the simple algorithm for the two-player proportional allocation, finding a fair allocation for more players is more challenging. In 1948, \citeboth{steinhaus1948} proposed a proportional protocol for three players. Later, Banach, Knaster, and Steinhaus proposed an $O(n^2)$ protocol inspired by the cut-and-choose protocol for proportional allocation among $n$ players. \citeboth{even1984note} improved this result by providing an $O(n \log n)$ divide and conquer protocol. Also, they showed that no protocol can proportionally allocate the cake using less than $n$ cuts; however this lower bound was not tight. The main difficulty of obtaining any lower bound for the cake cutting problem was the lack of any formal way to represent protocols. Finally, \citeboth{robertson1998cake} gave a formal model for cake cutting protocols. Their model covers almost all discrete cake cutting protocols. Later, \citeboth{edmonds2006cake} provided an $\Omega( n \log n)$ lower bound for the proportional cake cutting. Their result shows that the proportional protocol by Even and Paz is asymptotically tight.

However, finding an envy-free allocation seems to be much harder. For a long time, the only known discrete and bounded envy-free protocols were for $n \le 3$. Every other protocol required an unbounded number of queries \cite{brams1995envy,pikhurko2000envy}. \citeboth{alijani2017envy} gave a bounded protocol for the envy-free allocation under different assumptions on the valuation functions. In a recent work, \citeboth{procaccia2009thou} proved an $\Omega(n^2)$ lower bound for envy-free allocation which shows that finding an envy-free allocation is truly harder than proportional allocation. In a recent breakthrough, Aziz and Mackenzie \cite{aziz4,azizn} provided the first discrete and bounded protocol for envy-free allocation. Their protocol requires $n^{n^{n^{n^{n^n}}}}$ queries in the worst case.

Despite all the studies in the cake cutting, the results known for the chore division are very limited. The same divide and conquer algorithm by Even and Paz, finds a proportional allocation using $O( n \log n)$ queries. However, no lower bound was known for this problem. For the envy-free chore division, \citeboth{peterson2009n} gave an envy-free protocol for $n$ players, although their protocol is unbounded. Another protocol by \citeboth{peterson2002four} finds an envy-free allocation for 4 players using \textit{moving-knife} procedure. However, the moving-knife procedure is not discrete and could not be captured using any discrete protocol. Only recently, a protocol by \citeboth{dehghani2018envy} provides the first discrete and bounded protocol for the envy-free chore division. 
% As of today, no discrete and bounded protocol is known for envy-free allocation of a chore among $n > 3$ players. Also, no lower bound was known for this problem.

In this paper, we give an $\Omega( n \log n)$  lower bound for the proportional chore division. Our method shows a close relation between chore division and cake cutting. We introduce a subproblem similar to \textit{thin-rich game} introduced in \cite{edmonds2006cake}, and show that solving both proportional cake cutting and proportional chore division requires solving this problem, and solving this problem is hard. Our method can also be seen roughly as a reduction from proportional chore division to proportional cake cutting. We introduce the notion of \textit{dual} of a valuation function, and we show that how we can use dual functions to reduce some problems in chore division to similar problems in cake cutting. Since envy-freeness implies proportionality, our result also shows that any envy-free chore division protocol requires at least $\Omega( n \log n)$ queries.

%We also study the problem when players have unequal entitlements. Formally, entitlement describes the proportion of the chore that each player expects to receive. Getting back to our example about dividing a job, suppose that we want to divide up a job among a full-time employee and a part-time employee. In this case, the full-time employee expects to receive a bigger proportion of the job. Therefore, the entitlement of this player is greater than the other player's. Some studies in fair allocation with entitlements are \cite{robertson1998cake,pikhurko2000envy,farhadi2017fair}.

%An interesting result by \citeboth{stromquist1985sets} provides an existential proof that there is an envy-free allocation of the chore such that it requires cutting the chore at most $O( n \log n)$ times even if players have unequal entitlements. Although, it does not provide any protocol for this problem. Despite this result, in this paper, we show that there is no discrete and bounded protocol for proportional chore division when players have different entitlements. Also, we represent an unbounded but finite protocol for this problem.