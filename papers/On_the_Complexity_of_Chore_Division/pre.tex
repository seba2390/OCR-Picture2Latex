\section{Preliminaries}
In chore division (resp. cake cutting) problem, we are asked to partition a divisible undesirable (resp. desirable) object, usually modeled by the interval $[0,1]$, among $n$ players. Let $N = \{1,2,\ldots,n\}$ be the set of players. Each player $i$ has a valuation function $v_i$ that indicates, given a subinterval $I\subseteq [0,1]$, the cost (resp. profit) of that interval for the player $i$.
%In chore division and cake cutting problems, we are asked to partition a divisible undesirable object among $n$ players. A chore is usually modeled by the interval $[0,1]$. Let $N=\{1,2,\cdots,n\}$ be the set of players, each player $i$ has a valuation function $v_i$ that indicates his cost for each subinterval in $[0,1]$. In the cake cutting problem, this function represents how good an interval is for that player.
For an interval $[x,y]$, we use $v_i(x,y)$ to denote the player $i$'s valuation for this interval. We assume that valuation functions are \textit{non-negative}, \textit{additive}, \textit{divisible} and \textit{normalized}, in other words, for each player $i$, his valuation function $v_i$ satisfies the following properties:
\begin{itemize}
\item \textit{Non-negative}: $v_i(I) \ge 0$ for every subinterval $I$ in $[0,1]$.
\item \textit{Additive}: $v_i(I_1 \cup I_2) = v_i(I_1) + v_i(I_2)$ for all disjoint intervals $I_1$ and $I_2$.
\item \textit{Divisible}: for every interval $I$ and $0 \le \lambda \le 1$, there exists an interval $I' \subseteq I$ such that $v_i(I')= \lambda v_i(I)$.
\item \textit{Normalized}: $v_i(0,1)=1$.
\end{itemize}

For an interval $I=[x,y]$, we denote $Left(I)=x$ and $Right(I)=y$. Also, we use $|I|=y-x$ to denote the width of $I$. We say that an interval $I$ is non-empty if $|I|>0$.

 We say that $P$ is a \textit{piece} of the chore if it is union of finite disjoint intervals, i.e., $P=\cup_{i=1}^{k} I_i$. For a piece $P$, we use $|P|$ to denote its width which is
$$
|P| = \sum_{i=1}^{k} |I_i| = \sum_{i=1}^{k} Right(I_i)-Left(I_i) \,.
$$
Similarly, we use $v(P)$ to denote the value of the function $v$ for $P$. It follows from additivity of valuation functions that
$$
v(P) = \sum_{i=1}^{k} v(I_i) \,.
$$
Also, we use $D_v(P)=v(P)/|P|$ to denote the density of $P$. 

The complexity of a protocol is the number of queries it makes. We use the standard Robertson and Webb query model which allows two types of queries on a valuation function $v$.
\begin{itemize}
\item ${\ev}_{v}(x,y) : $ returns $v(x,y)$.
\item ${\ct}_{v} (x,r) : $ returns $y \in [0,1]$ such that $v(x,y)=r$ or declares that answer does not exist.
\end{itemize}

An \textit{allocation} $X=(X_1,X_2,\cdots,X_n)$ is a partitioning of chore into $n$ parts  $X_1,X_2,\cdots,X_n$ such that each player $i$ receives $X_i$. We say that an allocation $X=(X_1,X_2,\cdots,X_n)$ is proportional if $v_i(X_i) \le 1/n$ for every player $i$.
%This definition can be generalized to the case players have unequal entitlements. Formally, we call an allocation $X$ proportional if $v_i(X_i) \le e_i$ for each player $i$ where $e_i > 0$ is his entitlement. Entitlements always add up to 1, i.e., $\sum_{i=1}^{n} e_i =1$. 