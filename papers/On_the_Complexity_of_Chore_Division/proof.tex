\section{Lower Bound on Finding a Heavy Piece}
In this section we prove Theorem \ref{findheavy} and give an $\Omega( \log n)$ lower bound for finding a heavy piece in positive $(0,2)$-dense valuation functions. We introduce special types of valuation functions, called \textit{balanced-value-trees}, and present an adversarial strategy that gives an $\Omega( \log n)$ lower bound for finding a heavy piece on balanced-value-trees. Balanced-value-trees are very similar to \textit{value trees} valuation functions introduced in \cite{edmonds2006cake}, but value trees cannot be used for our problem since these valuation functions are not necessarily $(0,2)$-dense.

Assume that $n \ge 3^{11}$ is a power of three. A balanced-value-tree is a ternary balanced tree with $n$ leaves and depth $d=\log_3 n$. Each non-leaf node $v$ in the tree has three children, we use $l(u)$, $m(u)$ and $r(u)$ to denote its left, middle, and right child. Each node in the tree corresponds to an interval in $[0,1]$. For each node $u$, we use $I_u$ to denote the interval that corresponds to $u$, and we use $V(u)$ and $D(u)$ to denote the value and the density of the interval $I_u$. Let $r$ be the root of the tree, then $|I_r|=1$ and $V(r)=1$. For every non-leaf node $u$, its children partition the interval corresponding to $u$ into three disjoint intervals with the equal width, i.e., $I_{u} = I_{l(u)} \cup I_{m(u)} \cup I_{r(u)}$ and $|I_{l(u)}| = |I_{m(u)}| = |I_{r(u)}|= |I_u|/3$. It follows that width of every leaf in the tree is $1/n$. We call a node $u$ \textit{critical} if $D(u) \times \beta > 2$ where $\beta= 2^{6/ \ln(n)}$.

We label each edge in the tree such that the value of every node $u$ be the product of the label of edges along the path from the root to $u$. For a non-leaf critical node $u$, the label of edges between this node and its children are $1/3$ . Every other non-leaf node has an edge with label $\beta/3$ called a \textit{heavy edge}, and the label of its two other edges which we call them \textit{light} are $1/2-\beta/6$. Since $n \ge 3^{11}$, we have $1/3 \le \beta/3 < 1/2$, i.e., the value of every heavy edge is between $1/3$ and $1/2$. Also, the value of every light edge is between $1/4$ and $1/3$.  We assume that the valuation of every leaf in the tree is uniform, this means that for every leaf $u$, and an interval $I \subseteq I_u$, we have $V(I)=\dfrac{V(u) |I|}{|I_u|}$. Therefore we can find the valuation of an arbitrary interval using the tree. Note that, it follows from the definition of critical nodes and our labeling that all children of a critical node are also critical. 

\begin{lemma}
In a balanced-value-tree, the value of every non-leaf node is the sum of the values of its children.
\end{lemma}
\begin{proof}
Consider a non-leaf node $u$, if node $u$ is critical then all its edges to its children are labeled $1/3$, therefore $V(l(u))+ V(m(u))+ V(r(u)) = (V(u)/3 + V(u)/3 + V(u)/3)= V(u)$. The lemma holds in this case.

Otherwise, we suppose that a node $u$ is not critical, therefore $V(l(u))+ V(m(u))+ V(r(u)) = V(u) (\beta/3 + 1/2-\beta/6 + 1/2-\beta/6) = V(u)$. This completes the proof.
\end{proof}

For a node $u$, we use $h(u)$, $q(u)$ and $z(u)$ to denote respectively the number of heavy, light, and other edges along the path from the root to $u$. In the following lemma we show that how we can compute the density of node $u$ from $h(u)$, $q(u)$ and $z(u)$.
\begin{lemma}
\label{dentree}
In a balanced-value-tree we have $D(u)=\beta ^ {h(u)} (3/2-\beta/2) ^ {q(u)}$ for every node $u$ in the tree.
\end{lemma}
\begin{proof}
By the definition of balanced-value-trees we have
\begin{align*}
D(u) & = \dfrac{V(u)}{|I_u|} \\
& = \dfrac{(\beta/3)^{h(u)} (1/2-\beta/6)^{q(u)} (1/3)^{z(u)} }{(1/3)^{h(u)+q(u)+z(u)}} \\
& = \beta ^ {h(u)} (3/2-\beta/2) ^ {q(u)} \,.
\end{align*}
\end{proof}

Now we are ready to show that balanced-value-trees are positive and $(0,2)$-dense.
\begin{lemma}
Consider any balanced-value-tree, the valuation that this tree represents is positive and $(0,2)$-dense.
\end{lemma} 
\begin{proof}
Our goal is to show that density of every non-empty interval $I$ is at most $2$ and greater than $0$. By the definition of the balanced-value-tree, the valuation for every interval $I$ is greater than zero, so as its density.

Now it remains to show that density of every interval is at most $2$. Assume for the sake of contradiction that there is an interval with density greater than $2$, it follows that there is at least one leaf in the tree with the density greater than $2$. Let $u$ be this leaf. Consider the case that $u$ is a critical leaf, let $u_1,u_2,\cdots,u_k$ be the path from the root to $u$ where $u_1=r$ and $u_k=u$. Let $u_t$ be the first node in this sequence that is critical. $t$ is larger than one, since the root is not critical. It follows from the definition of the balanced-value-trees that $D(u)= D(u_t) \le \beta D(u_{t-1})$. Since $u_{t-1}$ is not critical, it implies that $ \beta D(u_{t-1}) \le 2$, therefore $D(u) \le 2$ which is a contradiction.

Otherwise, suppose that $u$ is not critical, therefore $D(u) < \beta D(u) \le 2$, thus we have the contradiction for both cases.  
\end{proof}

We say that a non-critical leaf $u$ in the tree is \textit{rich} if $D(u) \ge 1/2$. The following lemma shows that we can use any protocol that finds a heavy piece to find either a rich or a critical leaf in a balanced-value-tree.

\begin{lemma}
\label{reduc}
If a protocol finds a heavy piece in positive $(0,2)$-dense valuation functions with at most $T(n)$ queries, then using $O(T(n))$ queries we can find either a rich or a critical leaf in a balanced-value-tree. 
\end{lemma}
\begin{proof}
Consider the valuation function derived from the balanced-value-tree. Let piece $P$ be the output of the protocol for this valuation function. Since $P$ is heavy, we have $|P| \le 1/n$ and $V(P) \ge 1/2n$. Therefore, $D(P) \ge 1/2$. Since $P$ is the union of at most $O(T(n))$ intervals, there is an interval $I$ with the density at least $1/2$. We can find this interval with $O(T(n))$ queries. Since $|I| \le |P| \le 1/n$, the interval $I$ overlaps with at most two leaves, and one of these leaves has a density at least $1/2$ which can be found with $O(1)$ queries. The density of this leaf is at least $1/2$, thus it is either critical or rich.
\end{proof}

For the rest of the section, our goal is to show that any protocol for finding a leaf which is either rich or critical must make $\Omega( \log n)$ queries in the worst case. To this end, we first show that $h(u) = \Omega(\log n)$ for critical and rich leaves.
\begin{lemma}
\label{tooheavies}
Let $u$ be a leaf which is either rich or critical, then $h(u) > (\ln n)/6-1$.
\end{lemma} 
\begin{proof}
First, we assume that $u$ is a critical leaf, therefore $\beta D(u) >2 \Rightarrow D(u) > 2/\beta$. It implies from lemma \ref{dentree} that
$$2/\beta < D(u) = \beta ^ {h(u)} (3/2-\beta/2) ^ {q(u)} \le \beta ^ {h(u)} \,.$$
Therefore,
$$ \beta^{h(u)+1} > 2 \Rightarrow h(u) > (\log_{\beta} 2) -1 = (\ln n)/6-1 \,.
$$
Otherwise, suppose that $u$ is a rich leaf, therefore $D(u) \ge 1/2$. Since $u$ is a leaf and is not critical, we have $h(u)+q(u)=d=\log_3 n$. We complete the proof by showing that $h(u)$ must be greater than $(\ln n)/6$. For the sake of contradiction suppose that $h(u) \le (\ln n)/6$, therefore the maximum density that $u$ can have is
$$
\beta ^ { (\ln n)/6} (3/2-\beta/2)^{\log_3 n - (\ln n)/6} \,.
$$
Let denote $f(n)$ to be the function above.

It is easy to verify that function $f$ is an increasing function in $n$. Therefore,
$$
f(n) \le \lim_{x \to \infty} f(x) = 2^{3/2 - 3/\ln 3} \approx 0.426 < \dfrac{1}{2} \,.$$
which is a contradiction. Thus, for this case $h(u) > (\ln n)/6$.
\end{proof}

We now show that any protocol that finds either a critical or a rich leaf must make $\Omega( \log n)$ quries in the worst case. We give an adversarial strategy very similar to the strategy represented in \cite{edmonds2006cake} which prevents any protocol to find a critical or a rich leaf with less than $\Omega( \log n)$ queries. Consider a balanced-value-tree. At the beginning, the label of each edge is unknown to the protocol. However, after each query instead of answering the query, we reveal the label of some edges in the tree such that the answer of the query can uniquely be determined from the revealed edges. Let $u$ be a node in the tree and $u_1, u_2, \cdots, u_k$ be the path from the root to $u$ where $u_1 = r$ and $u_k = u$. We say that node $u$ is \textit{revealed} if for every $1 \le i \le k$, all labels of node $u_i$ to its children are revealed. The following lemma shows the information that a player can get from the revealed labels.  
\begin{lemma} 
(Lemma 2.2 in \cite{edmonds2006cake})
\label{information}
\begin{itemize}
\item For any revealed node $u$, the value of $V(u)$ can be determined.
\item For any revealed node $u$, values $V(0,Left(I_u))$ and $V(0,Right(I_u))$ can be computed.
\item Let $u,v$ be two revealed leaves, and $x \in I_u$ and $y \in I_v$, then values $V(0,x)$ and $V(x,y)$ can be computed.
\item Let $u$ be a revealed leaf, $x \in I_u$, $\alpha$ be a value and $v$ be the leaf that contains a point $y$ such that $V(x,y)= \alpha$, then the least common ancestor of $u$ and $v$ can be computed. 
\end{itemize}
\end{lemma}
We are now ready to provide an adversarial strategy against finding a critical or a rich leaf.
\begin{lemma}
\label{strategy}
The query complexity of any protocol that finds either a critical or a rich leaf in a balanced-value-tree is $\Omega( \log n)$.
\end{lemma}
\begin{proof}
We follow the following strategy for the first $\lfloor ((\ln n)/6 -1)/2 \rfloor$ queries. The strategy is very similar to the one in \cite{edmonds2006cake} with some minor changes, and we just highlight the main ideas in this strategy and the details of it can be found in the original paper. The strategy reveals the label of some edges after each query such that the answer of the query can be computed and keeps the following invariants. First, for each node in the tree either none or all of its edges to its children are revealed. Second, after $m$ queries, in any path from the root to a leaf at most $2m$ heavy edges are revealed. Three, all the revealed nodes form a connected component in the tree. Now we show that how we reveal the label of edges for each query.
\begin{itemize}
\item For ${\ev}(x,y)$ query, let $u_k$ be the leaf containing $x$, $u_1, u_2, \cdots, u_k$ be the path from the root to $u$, and $u_l$ be the first unrevealed node. For each $u_i$, $l \le i <k$, we reveal the label of $u_i$ to its children such that the edge between $u_i$ and $u_{i+1}$ be light, and $u_i$ has exactly two light edges and one heavy edge. Similarly, we reveal the edges in the same way for the point $y$.  Lemma \ref{information} shows that the answer of this query can be computed using these edges.
\item For ${\ct}(x,\alpha)$ query, we reveal the edges in the same way as the last case for the point $x$. Let $y$ be a point such that $V(x,y) = \alpha$. Using Lemma \ref{information}, we can find the least common ancestor of leaves containing $x$ and $y$. Let $u$ be this node. Let $u'$ be the first unrevealed node from $u$ towards the leaf containing $y$. Let $\gamma$ be the value that we must find in the subtree with the root $u'$. Recall that label of heavy edges are $1/3 \le \beta/3 < 1/2$. If $\gamma > \beta/3$, then we reveal the label of the first children to be heavy, and two others to be light. Otherwise, we reveal the first two edges to be light and the last one to be heavy. Since $\beta/3 < 1/2$, the edge between $u'$ and its child that contains $y$ is always light. We reveal the edges in the same way for the child which contains $y$ and do the same thing until we reveal the leaf containing $y$. 
\end{itemize}
We follow this strategy for the first $\lfloor ((\ln n)/6 -1)/2 \rfloor$ queries. Since $n \ge 3^{11}$, we have $((\ln n)/6 -1)/2 >0$. After these queries, at most $(\ln n)/6-1$ heavy edges are revealed in any path from the root to a leaf. By lemma \ref{tooheavies}, any critical or rich leaf must have more than $(\ln n)/6-1$ heavy edges in its path to the root.
Therefore, the protocol could not be sure about any critical or rich leaf. After these queries, the density of no interval is greater than $2$, since every critical node should have more than $(\ln n)/6-1$ heavy edges in its path to the root and no node is revealed to be critical. Therefore, this is a valid labeling of a balanced-value-tree, and no protocol can find a critical or a rich leaf against this strategy with less than $\lfloor ((\ln n)/6 -1)/2 \rfloor$ queries; hence, the query complexity of any protocol is $\Omega( \log n)$. 
\end{proof}

Now we can prove Theorem \ref{findheavy}.
\findheavy*
\begin{proof}
Let $T(n)$ be the query complexity of a protocol that finds a heavy piece. By Lemma \ref{reduc}, we can find either a critical or a rich leaf using $O(T(n))$ queries. Since query complexity of finding a critical or a rich leaf is $\Omega( \log n)$, we have $T(n) = \Omega( \log n)$.
\end{proof}