\section{Lower Bound on Chore Division}
In this section we provide an $\Omega(n \log n)$ lower bound for the proportional chore division. In the cake cutting, \citeboth{edmonds2006cake} presents an $\Omega(n \log n)$ lower bound by showing that finding a dense part for an arbitrary valuation function is hard and a protocol must use at least $\Omega( \log n)$ queries. Later, they show that any proportional protocol for cake cutting finds a dense part for at least $\Omega(n)$ of $n$ arbitrary valuation functions.

%For the chore division, we consider special type of valuation functions which density of each part is at least $1/2$, and we represent a mapping from these valuation functions to low-density valuation functions. After that, we show that using any proportional protocol for chore division, one can find dense pieces for at least $\Omega(n)$ of $n$ low-density valuation functions, and we complete the proof by showing that finding this piece requires at least $\Omega( \log n)$ queries in the worst case.

For the chore division problem, we consider a special type of valuation functions in which density of each piece of the chore is at least $1/2$. Then, we represent a mapping from these valuation functions to low-density valuation functions, and show that finding any dense piece in low-density valuation functions requires $\Omega( \log n)$ number of queries. Finally, we provide a lower-bound for the proportional chore division by showing that using any protocol for this problem, we can find a dense piece for at least $\Omega(n)$ of $n$ arbitrary low-density valuation functions.

\begin{definition}
Given values $\alpha$ and $\beta$ such that $0 \le \alpha \le 1 \le \beta$, we say that a valuation function $v$ is \textit{$(\alpha,\beta)$-dense} if $\alpha \le D_v(I) \le \beta$ for every non-empty subinterval $I$ in $[0,1]$.

Moreover, a valuation function $v$ is positive if $D_v(I) >0$ for every non-empty subinterval $I$ in $[0,1]$.
\end{definition}
An example of $(\alpha,\beta)$-dense valuation functions is uniform functions. Since density of every interval in a uniform valuation function is $1$, these valuation functions are $(1,1)$-dense. 
 
For an arbitrary positive valuation function $v$, we define its dual function and show that every query on the dual function can be answered using $O(1)$ queries on the $v$. Later, we show that the dual function is an appropriate mapping from high-density functions to low-density functions. 
\begin{definition}
For a positive valuation function $v$, we use $v^{*}$ to denote its dual function and define it as follows.
$$
v^{*}(x,y) = {\ct}_{v}(0,y) - {\ct}_{v}(0,x)
$$
for every subinterval $[x,y]$ in $[0,1]$
\end{definition}
Note that in a positive valuation function, every ${\ct}_v(x,y)$ query has a unique answer, therefore the function above is well-defined for positive functions.
\begin{lemma}
\label{O1}
For a positive valuation function $v$ and its dual function $v^{*}$ the following holds.
\begin{itemize}
\item ${\ev}_{v^{*}}(x,y) = {\ct}_{v}(0,y) - {\ct}_{v}(0,x)$
\item ${\ct}_{v^{*}}(x,r) = {\ev}_{v}(0,{\ct}_{v}(0,x)+r)$
\end{itemize}
So we can answer each query on $v^{*}$ using $O(1)$ queries on $v$.
\end{lemma}
\begin{proof}
Based on definition of the dual function, we have:
$$
{\ev}_{v^{*}}(x,y)= v^{*}(x,y) = {\ct}_{v}(0,y) - {\ct}_{v}(0,x) \,.
$$
For the cut query, it should return a $y$ such that $v^*(x,y)=r$. We complete the proof by showing that $v^*(x, {\ev}_{v}(0,{\ct}_{v}(0,x)+r))=r$. By the definition of dual function, we have
%For proving the second part we must show that $v^{-1}(x,{\ct}_{v^{-1}}(x,r)) = r$. 
\begin{align*}
&v^{*}(x,{\ev}_{v}(0,{\ct}_{v}(0,x)+r)) \\
&= {\ct}_{v}(0,{\ev}_{v}(0,{\ct}_{v}(0,x)+r)) - {\ct}_{v}(0,x) \,.
\end{align*}
Note that for any positive valuation function $v$, and $0 \le x \le 1$, we have ${\ct}_v(0,{\ev}_v(0,x))=x$. Therefore,
\begin{align*}
&v^{*}(x,{\ev}_{v}(0,{\ct}_{v}(0,x)+r)) \\
&= {\ct}_{v}(0,{\ev}_{v}(0,{\ct}_{v}(0,x)+r)) - {\ct}_{v}(0,x) \\
&= {\ct}_{v}(0,x)+r - {\ct}_{v}(0,x) = r \,.
\end{align*}
% &= {\ct}_{v}(0,x)+r - {\ct}_{v}(0,x) = r \,.
\end{proof}\\
In the following observation, we show that for a valuation function $v$, dual function of ${v^{*}}$ is $v$.
\begin{observation2}
\label{dualdual}
Let $v$ be a positive valuation function and $v^*$ be its dual function, then dual of $v^*$ is $v$. 
\end{observation2}
\begin{proof}
Let function $u$ be the dual of $v^*$, then the valuation of $u$ for an interval $[x,y]$ is
$$
u(x,y) = {\ct}_{v^*}(0,y) - {\ct}_{v^*}(0,x) \,.
$$
Since $v^*$ is the dual of $v$, by Lemma \ref{O1} we have,
\begin{align*}
&u(x,y) = {\ct}_{v^*}(0,y) - {\ct}_{v^*}(0,x) \\
&= {\ev}_v(0,{\ct}_v(0,0)+y) - {\ev}_v(0,{\ct}_v(0,0)+x) \\
&= {\ev}_v(0,y) - {\ev}_v(0,x) = v(x,y) \,. 
\end{align*}
Therefore, valuation of $u$ for any interval is the same as valuation of $v$; hence, $u=v$.
\end{proof}

% we first another problem similar to thin-rich game introduced in \cite{???}, and show that solving this problem requires at least $\Omega(\log n)$ queries. We define heavy and light pieces as below.
We introduce high-density and low-density pieces. \citeboth{edmonds2006cake} showed that a protocol must use at least $\Omega ( \log n)$ queries in order to find a high-density piece for an arbitrary valuation function. We expand their result by showing that finding a high-density piece for a positive $(0,2)$-dense valuation function is still hard.
\begin{definition}
A piece $X$ is heavy with respect to valuation function $v$ if its width is at most $1/n$ and the valuation of $v$ on this piece be at least $1/2n$, i.e., $|X| \le 1/n$ and $v(X) \ge 1/2n$.

Similarly, a piece is light with respect to $v$ if $|X| \ge 1/2n$ and $v(X) \le 1/n$.
\end{definition}
Note that heavy and light pieces are not exclusive, and a piece can be both heavy and light.

\begin{restatable}{theorem}{findheavy}
\label{findheavy}
Any protocol that finds a heavy piece for an arbitrary positive $(0,2)$-dense valuation function requires $\Omega(\log n)$ queries in the worst case.
\end{restatable}
This theorem is our main tool to obtain a lower-bound for proportional chore division. First we show how we can use this theorem to prove the $\Omega( n \log n)$ lower bound for chore division, and then in the next section we provide a proof for this theorem.
%Now we are ready to prove that complexity of any proportional chore division protocol is at least $\Omega( n \log n)$
%\begin{theorem}
%Any protocol for the proportional chore division makes at least $\Omega ( n \log n)$ queries in the worst case.
%\end{theorem}
%\begin{proof}

We show that any protocol for this chore division problem requires $\Omega ( n \log n)$ queries even if all the players' valuation functions are $(1/2,\infty)$-dense. Specifically, we show that given $n$ arbitrary positive $(0,2)$-dense valuation functions, one can use their dual functions and any proportional chore division protocol to find a heavy piece for at least $\Omega(n)$ of them. First we show that if a valuation function $v$ is positive and $(0,2)$-dense, then its dual is $(1/2,\infty)$-dense.
%Consider $n$ positive $(0,2)$-dense valuation functions $v_1, v_2, \cdots, v_n$, and solve proportional chore division problem for their dual functions $v^{*}_1, v^{*}_2, \cdots, v^{*}_n$. Let $X_1, X_2, \cdots, X_n$ be the allocation returned by the protocol in which $X_i$ is the piece allocated to player $i$. Since the protocol is proportional, the value of the piece $X_i$ for the player $i$ is at most $1/n$, i.e., $v_i^{-1}(X_i) \le 1/n$.

%First we show that all the valuation functions $v^{*}_1, v^{*}_2, \cdots, v^{*}_n$ are $(1/2,\infty)$-dense and later we use this observation to show that for at least $n/3$ of valuation functions $v_1, v_2, \cdots, v_n$, we can find a heavy piece using any protocol for chore division.
\begin{lemma}
\label{dual}
The dual of a positive $(0,2)$-dense valuation function is positive and $(1/2,\infty)$-dense.
\end{lemma}
\begin{proof}
Suppose that $v$ is a positive $(0,2)$-dense valuation function. Let $[x,y]$ be a non-empty subinterval in $[0,1]$. We show that the density of the interval $[x,y]$ is at least $1/2$ with respect to $v^{*}$. For an interval $[x,y]$ we have $v^*(x,y)= {\ct}_v(0,y)-{\ct}_v(0,x)$. Therefore, we have
$$
D_{v^*} (x,y) = \dfrac{{\ct}_v(0,y)-{\ct}_v(0,x)}{y-x} \,.
$$
Note that $x={\ev}_v(0, {\ct}_v(0,x))$, and $y={\ev}_v(0, {\ct}_v(0,y))$. Therefore, by setting $l= {\ct}_v(0,x)$, and $r = {\ct}_v(0,y)$, we have $x={\ev}_v(0,l)$ and $y={\ev}_v(0,r)$. Therefore,
\begin{align*}
D_{v^*} (x,y) &= \dfrac{r-l}{{\ev}_v(0,r)-{\ev}_v(0,l)}\\
&= \dfrac{r-l}{{\ev}_v(l,r)} \\
&= \dfrac{1}{D_v(l,r)} \,.
\end{align*}
Since $v$ is positive $(0,2)$-dense, we have $0 < D_v(l,r) \le 2$, therefore,
$$
D_{v^*} (x,y) = \dfrac{1}{D_v(l,r)} \ge \dfrac{1}{2} \,.
$$
\end{proof}

%Therefore all inverse valuation functions are $(1/2,\infty)$-dense, now we show that in any proportional allocation of the chore at least $n/3$ of the players get a piece with width at least $1/2n$, so their pieces are light in their perspective.
Now, we show that if $n$ players all have a $(1/2,\infty)$-dense valuation functions, then in any proportional allocation of chore to these players, at least $n/3$ of allocated pieces are light.
\begin{lemma}
\label{light}
Given $n$ players with $(1/2,\infty)$-dense valuation functions $u_1,\cdots,u_n$, let $X_1,X_2,\cdots,X_n$ be any proportional allocation of the chore to the players such that $X_i$ is allocated to the player $i$, then at least $n/3$ of the allocated pieces are light for their owners.
\end{lemma}
\begin{proof}
For each player $i$, we use $w_i$ to denote the width of the piece allocated to player $i$, i.e., $w_i = |X_i|$. Therefore, $\sum_{i=1}^{n} w_i =1$. Since the allocation is proportional we have $u_i(X_i) \le 1/n$ for every $u_i$. Also, Since the valuation functions are $(1/2,\infty)$-dense, we have $u_i(X_i) \ge w_i/2$ for every player $i$, therefore $w_i \le 2 u_i(X_i) \le 2/n$. Now assume that $t$ is the number of pieces with the width less than $1/2n$. Since the width of every other piece is at most $2/n$, the following holds.
$$
\dfrac{t}{2n}+ \dfrac{2(n-t)} {n} \ge 1 \Rightarrow t \le \dfrac{2n}{3} \,.
$$
Therefore at least $n-t \ge n/3$ of the $w_i$ are at least $1/2n$, and the width of at least $n/3$ of the $X_i$ are at least $1/2n$. Note that because of the proportionality, the value of each of these pieces is at most $1/n$ for its owner. Therefore these pieces are light.
\end{proof}

%Let $X'$ be the set of light pieces returned by the protocol. According to lemma \ref{light}, $|X'| \ge n/3$. for each light piece $x' \in X'$ we define its inverse as follows.
Now we present a mapping from light pieces to heavy pieces in the dual of the valuation function.
\begin{definition}
For an interval $I=[a,b]$ and a positive valuation function $v$, we define the dual of this interval with respect to $v$ as $I_v^{*}=[\ev_v(0,a),\ev_v(0,b)]$.

For a part $P$ which is union of finite disjoint intervals $I_1,\cdots, I_k$, we define the dual of $P$ as $P_v^{*}= \cup_{i=1}^{k} {I_i}_v^{*}$.
\end{definition}
%Let $x' \in X'$ be a light piece, suppose that this piece is allocated to the player $i$. We complete the proof by showing that ${x'}^{-1}_{v^{-1}_i}$ is a heavy piece in respect to ${v^{-1}_i}^{-1}=v_i$. Since we have shown that finding a heavy piece in a positive $(0,2)$-dense valuation function requires at least $\Omega(\log n)$ queries and any proportional protocol finds a heavy piece for at least $n/3$ of the $(0,2)$-dense valuation functions, it follows that complexity of the any proportional protocol is at least $\Omega( n \log n)$.
\begin{lemma}
\label{heavy}
Let $P$ be a light piece with respect to $v$ where $v$ is a positive valuation function, then $P_{v}^{*}$ is heavy with respect to $v^{*}$.
\end{lemma}
\begin{proof}
Suppose that $P= \cup_{i=1}^{k} I_i$, then $P_v^{*}= \cup_{i=1}^{k} {I_i}_v^{*}$. It follows that
\begin{align*}
|P_v^{*}|&=\sum_{i=1}^{k} |{I_i}_v^{*}|\\
&=\sum_{i=1}^{k} {\ev}_v(0,Right(I_i))- {\ev}_v(0,Left(I_i))\\&
 = \sum_{i=1}^{k} v(I_i) = v(P) \le \dfrac{1}{n} \,.
\end{align*}
Also, for an interval $I=[a,b]$, we have,
\begin{align*}
v^*(I_v^*) &= v^*(\ev_v(0,a),\ev_v(0,b)) \\
&= {\ct}_v(0,\ev_v(0,b)) - {\ct}_v(0,\ev_v(0,a)) 
\\&= b-a = |I| \,.
\end{align*}
Therefore,
$$
v^{*}(P_v^{*})=\sum_{i=1}^{k} v^{*}({I_i}_v^{*}) =  \sum_{i=1}^{k} |I_i| = |P| \ge \dfrac{1}{2n} \,.
$$
This completes the proof.
\end{proof}

Now we are ready to prove that complexity of any proportional chore division protocol is at least $\Omega( n \log n)$
\begin{theorem}
Any protocol for the proportional chore division makes at least $\Omega ( n \log n)$ queries in the worst case.
\end{theorem}
\begin{proof}
%We show that any protocol for this chore division problem requires $\Omega ( n \log n)$ queries even if all the players' valuation functions are $(1/2,\infty)$-dense. Specifically, we show than given $n$ arbitrary positive $(0,2)$-dense valuation functions, one can use their dual functions and any proportional chore division protocol to find a heavy piece for at least $\Omega(n)$ of them. First we show that if a valuation function $v$ is positive and $(0,2)$-dense, then its dual is $(1/2,\infty)$-dense.
Suppose that the query complexity of a chore division protocol is $T(n)$.
Consider $n$ arbitrary positive $(0,2)$-dense valuation functions $v_1, v_2, \cdots, v_n$, and solve the proportional chore division problem for their dual functions $v^{*}_1, v^{*}_2, \cdots, v^{*}_n$. Let $X_1, X_2, \cdots, X_n$ be the pieces allocated to the players respectively. For every $v^*_i$, by Lemma \ref{O1}, we can answer each query on this function by making $O(1)$ queries on $v_i$. Therefore, we can find the proportional chore division for the dual valuation functions $v^{*}_1, v^{*}_2, \cdots, v^{*}_n$ using $O(T(n))$ queries.

According to the Lemma \ref{dual}, valuation functions  $v^{*}_1, v^{*}_2, \cdots, v^{*}_n$ are $(1/2,\infty)$-dense. Therefore, by Lemma \ref{light}, at least $n/3$ of the $X_1, X_2, \cdots, X_n$ are light with respect to the dual valuation functions. Let $Y_1, Y_2, \cdots, Y_n$ be the dual of pieces, where $Y_i$ is the dual of $X_i$ with respect to $v^*_i$, i.e., $Y_i = {X^*_i}_{v^*_i}$. Since the dual of $v^{*}_i$ is $v_i$, applying Lemma \ref{heavy} implies that at least $n/3$ of the dual pieces $Y_1, Y_2, \cdots, Y_n$ are heavy for the valuation functions $v_1, v_2, \cdots, v_n$. Since the protocol makes at most $O(T(n))$ queries, the pieces returned by this protocol are at most union of $O(T(n))$ intervals, so we can calculate the dual of all these pieces using $O(T(n))$ queries. Therefore we can find heavy piece for at least $n/3$ of the valuation functions using $O(T(n))$ queries. This along with Lemma \ref{findheavy} implies that $T(n)= \Omega( n \log n)$. Note that pieces $Y_1, Y_2, \cdots, Y_n$ are not necessarily disjoint. However, we only want to find a heavy piece for $\Omega(n)$ of valuation functions, and since this is not an allocation, $Y_1, Y_2, \cdots, Y_n$ can intersect.


%Since the protocol is proportional, the value of the piece $X_i$ for the player $i$ is at most $1/n$, i.e., $v_i^{-1}(X_i) \le 1/n$.

%First we show that all the valuation functions $v^{*}_1, v^{*}_2, \cdots, v^{*}_n$ are $(1/2,\infty)$-dense and later we use this observation to show that for at least $n/3$ of valuation functions $v_1, v_2, \cdots, v_n$, we can find a heavy piece using any protocol for chore division.

%If we give this protocol inverse of $n$ positive $(0,2)$-dense valuation functions, according to lemma \ref{O1}, we can still answer each query in $O(1)$, so the protocol finds a proportional allocation for the inverse functions using $O(T(n))$ queries. By lemma \ref{light}, in the allocation returned by this protocol there are at least $n/3$ light pieces in respect to inverse functions. Since the protocol makes at most $O(T(n))$ queries, the pieces returned by this protocol are at most union of $O(T(n))$ intervals, so one can calculate the inverse of all these pieces using $O(T(n))$ queries. Lemma \ref{heavy} shows these are at least $n/3$ heavy pieces among these new pieces. Therefore the protocol finds a heavy piece for at least $n/3$ of valuation functions using $O(T(n))$ queries. This along with lemma \ref{findheavy} implies that $T(n)= \Omega( n \log n)$. 
\end{proof} 
