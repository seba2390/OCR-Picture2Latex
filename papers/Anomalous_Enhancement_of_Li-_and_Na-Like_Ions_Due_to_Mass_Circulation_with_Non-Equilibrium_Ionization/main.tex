% mnras_template.tex 
%
% LaTeX template for creating an MNRAS paper
%
% v3.0 released 14 May 2015
% (version numbers match those of mnras.cls)
%
% Copyright (C) Royal Astronomical Society 2015
% Authors:
% Keith T. Smith (Royal Astronomical Society)

% Change log
%
% v3.0 May 2015
%    Renamed to match the new package name
%    Version number matches mnras.cls
%    A few minor tweaks to wording
% v1.0 September 2013
%    Beta testing only - never publicly released
%    First version: a simple (ish) template for creating an MNRAS paper

%%%%%%%%%%%%%%%%%%%%%%%%%%%%%%%%%%%%%%%%%%%%%%%%%%
% Basic setup. Most papers should leave these options alone.
\documentclass[fleqn,usenatbib]{mnras}

% MNRAS is set in Times font. If you don't have this installed (most LaTeX
% installations will be fine) or prefer the old Computer Modern fonts, comment
% out the following line
\usepackage{newtxtext,newtxmath}
% Depending on your LaTeX fonts installation, you might get better results with one of these:
%\usepackage{mathptmx}
%\usepackage{txfonts}

% add comment package by T.Matsumoto
\usepackage{comment}
\usepackage{array} % for table

% Use vector fonts, so it zooms properly in on-screen viewing software
% Don't change these lines unless you know what you are doing
\usepackage[T1]{fontenc}

% Allow "Thomas van Noord" and "Simon de Laguarde" and alike to be sorted by "N" and "L" etc. in the bibliography.
% Write the name in the bibliography as "\VAN{Noord}{Van}{van} Noord, Thomas"
\DeclareRobustCommand{\VAN}[3]{#2}
\let\VANthebibliography\thebibliography
\def\thebibliography{\DeclareRobustCommand{\VAN}[3]{##3}\VANthebibliography}


%%%%% AUTHORS - PLACE YOUR OWN PACKAGES HERE %%%%%

% Only include extra packages if you really need them. Common packages are:
\usepackage{graphicx}	% Including figure files
\usepackage{amsmath}	% Advanced maths commands
% \usepackage{amssymb}	% Extra maths symbols

%%%%%%%%%%%%%%%%%%%%%%%%%%%%%%%%%%%%%%%%%%%%%%%%%%

%%%%% AUTHORS - PLACE YOUR OWN COMMANDS HERE %%%%%

% Please keep new commands to a minimum, and use \newcommand not \def to avoid
% overwriting existing commands. Example:
%\newcommand{\pcm}{\,cm$^{-2}$}	% per cm-squared

%%%%%%%%%%%%%%%%%%%%%%%%%%%%%%%%%%%%%%%%%%%%%%%%%%

%%%%%%%%%%%%%%%%%%% TITLE PAGE %%%%%%%%%%%%%%%%%%%

% Title of the paper, and the short title which is used in the headers.
% Keep the title short and informative.
%\title[Short title, max. 45 characters]{The Effect of Non-equilibrium Ionization in Shock-Heated Loops}
\title[Anomalous Ion Enhancement]{Anomalous Enhancement of Li- and Na-Like Ions Due to Mass Circulation with Non-Equilibrium Ionization}

% The list of authors, and the short list which is used in the headers.
% If you need two or more lines of authors, add an extra line using \newauthor
\author[T.Matsumoto et al.]{
T. Matsumoto,$^{1,2}$\thanks{E-mail: takuma.matsumoto@isee.nagoya-u.ac.jp}
\\
% List of institutions
$^{1}$Centre for Integrated Data Science, Institute for Space-Earth Environmental Research, Nagoya University, Furocho, Chikusa-ku, Nagoya, Aichi 464-8601, Japan\\
$^{2}$National Astronomical Observatory of Japan, 2-21-1 Osawa, Mitaka, Tokyo 181-8588, Japan
}

% These dates will be filled out by the publisher
\date{Accepted XXX. Received YYY; in original form ZZZ}

% Enter the current year, for the copyright statements etc.
\pubyear{2015}

% Don't change these lines
\begin{document}
\label{firstpage}
\pagerange{\pageref{firstpage}--\pageref{lastpage}}
\maketitle

% Abstract of the paper (no more than 250/200 words for main/letter) 
\begin{abstract}
The solar ultraviolet intensities of spectral lines originating from Li- and Na-like ions have been observed to surpass the expectations derived from plasmas with coronal approximation. 
The violation of the coronal approximation can be partially attributed to non-equilibrium ionization (NEI) due to dynamic processes occurring in the vicinity of the transition region.
However, the quantitative analysis of these dynamic effects has not yet been conducted.
To investigate the impact of these dynamics, a set of equations governing NEI for multiple ion species was solved numerically in conjunction with 1.5-dimensional magnetohydrodynamic equations describing an Alfv\'{e}n-wave-heated coronal loop. 
Following the injection of Alfv\'{e}n waves from the photosphere, the system undergoes a time evolution characterized by phases of evaporation, condensation, and quasi-steady states.
During the evaporation phase, the ionization fractions of Li- and Na-like ions were observed to increase, with a maximum enhancement of 1.6 when compared to the fractions in ionization equilibrium. 
This over-fractionation of Li- and Na-like ions was found to be induced by the evaporation process, while collisions between shocks and the transition region did not exhibit deviations from ionization equilibrium. Consequently, the intensities calculated using the coronal approximation underestimated the intensities of Li- and Na-like ions by up to 60\%.
Conversely, under-fractions of at least 0.9 was observed during the condensation phase and the quasi-steady state.
Given that the degree of over/under-fraction exhibits a dependency on mass motions, our study has a possibility to impose limitations on both the mass circulation in coronal heating and mass loss processes.


%This is a simple template for authors to write new MNRAS papers.
%The abstract should briefly describe the aims, methods, and main results of the paper.
%It should be a single paragraph not more than 250 words (200 words for Letters).
%No references should appear in the abstract.
\end{abstract}

% Select between one and six entries from the list of approved keywords.
% Don't make up new ones.
\begin{keywords}
Sun: corona -- Sun: chromosphere -- Sun: transition region -- Sun: UV radiation 
\end{keywords}

%%%%%%%%%%%%%%%%%%%%%%%%%%%%%%%%%%%%%%%%%%%%%%%%%%

%%%%%%%%%%%%%%%%% BODY OF PAPER %%%%%%%%%%%%%%%%%%

\section{Introduction}
Ultraviolet (UV) emissions have posed a long-standing challenge due to inconsistencies between observations and theoretical predictions. To address these disparities, researchers have explored the concept of non-equilibrium ionization (NEI) \citep{1964spre.conf..730N,1978ApJ...226..698M,1979ApJ...229L.101D}. NEI becomes particularly relevant when the time scales associated with ionization and recombination processes exceed those of dynamic phenomena, such as the transit time of mass motions within the thin transition region, the dynamical heating time scale, and the time scales of acoustic fluctuations.

The anomalous behavior of Li- and Na-like ions, stemming from the violation of the coronal approximations, has been a subject of investigation since the early days of UV observations \citep{1971RSPTA.270...81B,1972ApJ...178..527D}. Subsequent in-depth analyses have revealed that the intensities of spectral lines originating from these ions can exceed theoretical expectations based on the coronal approximation by factors of up to approximately 5 \citep{1995ApJ...455L..85J}. While advanced atomic models have partially mitigated this discrepancy \citep{2023MNRAS.521.4696D}, challenges remain.

To address these gaps, the consideration of NEI is a crucial avenue. Given the intrinsic time dependence associated with NEI, it necessitates the solution of either time-steady or dynamic equations to assess its impact. While various studies have demonstrated deviations from ionization equilibrium within the context of phenomena such as siphon flows \citep{1989ApJ...338.1131N,1990ApJ...362..370S}, solar flares \citep{1984SoPh...90..357M,2011ApJ...742...70I,2015PhPl...22j1206I}, and nanoflares \citep{1982ApJ...255..783M, 1993ApJ...402..741H, 2009A&A...502..409B}, to the best of our knowledge, there have been no quantitative and systematic investigations into the emission anomalies of Li- and Na-like ions within the framework of NEI.

In this study, we adopted a 1.5-dimensional magnetohydrodynamics (MHD) model with Alfv\'{e}n-wave-heated coronal loops \citep{2004ApJ...601L.107M, 2004ESASP.575...80M} to predict the emission from ions in NEI. The hot corona is maintained via shocks produced through nonlinear mode conversions from Alfv\'{e}n waves \citep{1982SoPh...75...35H, 1999ApJ...514..493K}, and its behaviors in quasi-steady states \citep{2010ApJ...712..494A} and during long-term evolution \citep{2019ApJ...885..164W} have been extensively investigated. By solving a set of equations for NEI, we can quantitatively explore the behaviors of Li- and Na-like ions in the coronal loop with complex dynamics.

Because the transition region, where most Li- and Na-like ions form, is a very thin layer, even small mass flows associated with evaporation, condensation, and shock propagation can occur in time scales shorter than the ionization and recombination time scale. This phenomenon may contribute to deviations in ionization fractions from ionization equilibrium, although detailed numerical simulations are required to quantitatively estimate the degree of departure in ionization fraction and the associated emissions.

The primary objective of this study is to investigate the differences in emissions between plasmas in NEI and plasmas in ionization equilibrium. This paper is organized as follows: Section 2 presents models and assumptions, Section 3 details our simulations and analysis with discussions, and Section 4 provides the conclusions of the study.


\section{Models and assumptions}
In this study, we investigated the impact of NEI on emergent intensity by employing a coronal loop model heated by Alfvén waves. We concurrently solved the dynamic heating process governed by MHD equations and the evolution of the ion fraction. Subsequently, we reconstructed UV radiations based on the obtained ion fractions, electron density, and temperature.

We employed a 1.5-dimensional numerical model for the coronal loop, building upon the pioneering works of \cite{2004ApJ...601L.107M} and \cite{2004ESASP.575...80M}. This model naturally reproduces warm loops as a consequence of Alfv\'{e}n wave injection from the photosphere. The hot corona is achieved through MHD shocks generated by nonlinear mode conversion from Alfv\'{e}n waves \citep{1982SoPh...75...35H, 1999ApJ...514..493K}. 
The model's properties have been extensively investigated, including parameter dependencies \citep{2010ApJ...712..494A, 2010ApJ...710.1857M}, and differences from the nanoflare model \citep{2008ApJ...688..669A}.

The fundamental equations solved are as follows:
The equation of mass conservation:
\begin{equation}
\frac{\upartial \rho A}{\upartial t} + \frac{\upartial \rho v_{\rm s} A}{\upartial s} = 0, \label{eq_eom}
\end{equation}
the equation of momentum conservation along the field line:
\begin{equation}
    \frac{\upartial \rho v_s A}{\upartial t} + \frac{\upartial }{\upartial s} \left( \left[ \rho v_s^2 + P_{\rm g} + \frac{B_\phi^2}{2} \right] A\right) = \left( P_{\rm g} + \frac{\rho v_\phi^2}{2}\right) \frac{d A}{d s} - \rho g_{\rm s} A, \label{eq_eoms}
\end{equation}
the equation of angular momentum conservation:
\begin{equation}
\frac{\upartial \rho v_\phi A^{3/2}}{\upartial t} + \frac{\upartial}{\upartial s} \left( \left[ \rho v_\phi v_{\rm s} - B_\phi B_{\rm s}\right] A^{3/2} \right) = \rho L_{\rm trq} A,\label{eq_eomp}
\end{equation}
the induction equation:
\begin{equation}
\frac{\upartial B_\phi A^{1/2}}{\upartial t} + \frac{\upartial}{\upartial s} \left( \left[ B_\phi v_{\rm s} - B_{\rm s} v_{\phi} \right] A^{1/2} \right) = 0, \label{eq_ie}
\end{equation}
and the equation of total energy conservation:
\begin{align}
&\frac{\upartial {\cal E} A}{\upartial t} + \frac{\upartial}{\upartial s}  \left( \left[ \left\{ {\cal E} + P_{\rm g} + \frac{B_\phi^2}{2} \right\} v_s - B_\phi B_{\rm s} v_\phi \right] A \right) \nonumber \\ 
&= - L_{\rm rad} A + Q_{\rm cnd} A - \rho v_{\rm s} g_{\rm s} A + \rho v_\phi L_{\rm trq} A^{1/2}. \label{eq_eot}
\end{align}
In these equations, $\rho$ and $P_{\rm g}$ represent mass density and gas pressure, respectively; $v_{\rm s}$ is the velocity along the field line; $v_\phi$ is the toroidal velocity; $B_\phi$ is the toroidal magnetic field strength normalized by $\sqrt{4\pi}$. ${\cal E}$ is the total energy density given by
\begin{equation}
    {\cal E} = \frac{1}{2} \rho \left( v_{\rm s}^2 + v_\phi^2 \right) + \frac{P_{\rm g}}{\gamma -1} + \frac{B_\phi^2}{2},
\end{equation}
where $\gamma$ is the ratio of specific heats and was taken to be $5/3$.

The variables $g_{\rm s}$ and $A$ represent gravitational acceleration along the field line and the cross-section of the flux tube, respectively. These variables depend on the shape of the flux tube, and we adopted the same shape as used in \cite{2004ApJ...601L.107M}. The field line below the chromosphere is significantly inclined from the vertical direction, leading to p-mode leakage.

Equation \ref{eq_eot} includes source terms accounting for radiation, thermal conduction, gravity, and torque. For radiation, we employed a composite model that considers both optically thick and thin radiative loss mechanisms \citep{2021A&A...656A.111S}. The radiative loss functions from optically thin plasma were estimated via CHIANTI, assuming photospheric element abundances. We did not account for the feedback of NEI in the radiative loss function, although it has been suggested that the impact of NEI on the radiative loss function is projected to be at most 5\% in dense atmospheres after solar flares \citep{1984SoPh...90..357M} or at most a factor of 2 to 4 in coronal loops \citep{1982ApJ...255..783M, 1993ApJ...402..741H}.

We selected the Spitzer-type thermal conduction. To reduce numerical diffusion near the lower transition region, we implemented a broadening technique for temperatures below $0.15$ MK, denoted as $T_{\rm c}$ \citep{2009ApJ...690..902L}. This technique involves adjustments to both the thermal conduction coefficient and radiative cooling. 
By implementing this technique, the width of the transition region below $T<T_{\rm c}$ will be broadened by a factor of $\sim (T_{\rm c}/T)^{5/2}$.
Consequently, we achieved a reduction in the relative temperature difference between adjacent grid points, $\Delta \ln T$, to less than 5\% for the current grid size, thereby ensuring the allocation of more than 20 grid points for the transition region. We will discuss the limitations of these modifications in subsequent sections.

The amplitude of the torque at the footpoints is denoted as $L_{\rm trq}$. This torque was enforced only at the footpoints, and its amplitude was determined such that the root mean square of the toroidal velocity amplitude reached $\sim$ 1 km s$^{-1}$. This velocity amplitude is consistent with the horizontal velocities in the granular cells \citep{1998ApJ...509..435V, 2010ApJ...716L..19M, 2012ApJ...752...48C, 2017ApJ...849....7O}.


We adopted an approximated equation of state in our model to include the effect of a change in molecular weight \citep{2014MNRAS.440..971M}. Although this slightly modified the atmospheric structure below the chromosphere ($T<10^4$ K) compared to constant molecular weight, the effects on the transition region and the corona should be subtle.

We investigated the impact of NEI for the elements C, N, O, Si, and S. In this regard, we solved a series of NEI equations for these elements, expressed as follows:
\begin{eqnarray}
    \frac{\upartial N_{\rm i}}{\upartial t} + \nabla \cdot \left(N_{\rm i} \mathbf{v} \right) = N_{\rm e} \left( S_{\rm i-1} N_{\rm i-1} + \alpha_{\rm i+1} N_{\rm i+1} - \left( S_{\rm i} + \alpha_{\rm i} \right) N_{\rm i} \label{eq_nei} \right), 
\end{eqnarray}
where $N_{\rm i}$ denotes the number density of a specific element in the ith ionization stage while $N_{\rm e}$ indicates the electron number density. The coefficients on the right-hand side, $S_{\rm i}$ and $\alpha_{\rm i}$, indicate temperature-dependent ionization and recombination rate coefficients obtained from CHIANTI atomic database 10.0.2 \citep{2021ApJ...909...38D}. To focus solely on the effects of NEI, we assumed a constant electron density of $5\times10^9$ cm$^{-3}$ when calculating the coefficients, although the density dependence removes some discrepancies between theory and observations on Li- and Na-like ions \citep{2023MNRAS.521.4696D}. Note that we did not include the feedback effects on MHD variables through the radiative cooling function.

We developed an original MHD code capable of conducting precise and stable simulations, even in the low beta region surrounding the transition region. To calculate numerical flux, we employed the HLL-approximated Riemann solver \citep{1991JCoPh..92..273E}. Furthermore, conservative variables were reconstructed in each cell using a 3rd order weighted essentially non-oscillatory (WENO) scheme and subsequently integrated in time through the 3rd order Arbitrary Derivative Riemann Problem (ADER) scheme \citep{2009JCoPh.228.2480B}. Considering that the time scale of thermal conduction is generally much shorter than that of dynamics, we adopted an operator split method, implicitly solving thermal conduction using the super time-stepping method \citep{2012MNRAS.422.2102M}. Because the shortest time scale in equations (\ref{eq_nei}) is often smaller than dynamical time scales, we implicitly solved the right-hand side of equations (\ref{eq_nei}).

We applied the same boundary and initial conditions as outlined in \cite{2010ApJ...712..494A}. In brief, the initial conditions maintained hydrostatic equilibrium up to a height of 2 Mm, above which an artificially denser atmosphere was assumed to avoid severe CFL condition. The numerical domain spaned 103 Mm, including 2 Mm of subphotospheric layers at both ends to mitigate artificial oscillations arising from the boundaries. Grid spacing started at 10 km for the initial 16 Mm from both boundaries and gradually increased to 100 km in the central region. We initially computed the entire evolution of coronal loops using this coarse grid spacing. Subsequently, we concentrated on three specific 20-minute time intervals and conducted additional calculations with finer grid sizes, reaching down to 1.25 km near the transition regions. This approach enabled us to reduce computational costs while effectively resolving the narrow transition region.

\section{Results and Discussions}
In this study, we performed 1.5-dimensional MHD simulations of a coronal loop model subjected to Alfv\'{e}n wave heating. Simultaneously, we solved the NEI equations for selected elements. Subsequently, by tracking the dynamic evolution of temperature, electron density, and ionization fractions, we calculated the emergent intensity employing the CHIANTI database. Our results unveil that specific Li- and Na-like ions display higher intensities when contrasted with the predictions based on the coronal approximation.

\subsection{Properties of coronal model}

\begin{figure}
	\includegraphics[width=\columnwidth]{figure_028.eps}
    \caption{Temporal evolution of (a) the maximum temperature and (b) the coronal mass normalized by the cross sectional area at the photosphere.}
    \label{fig:time_evolution}
\end{figure}

The system underwent both the evaporation ($t<200$ min) and condensation ($200<t<500$ min) phases before attaining a quasi-steady state ($t>500$ min) (Fig. \ref{fig:time_evolution}). 
The coronal mass in Fig. \ref{fig:time_evolution}(b) was normalized by the cross-sectional area at the base, denoted as $A_0$, and defined as $\int \rho A/A_0 ds$, where integration was performed over the region where the temperature exceeded 10$^5$ K.
In the quasi-steady state, the model reproduced a warm loop with an apex temperature of approximately 1.1 MK, featuring sharp transition regions between the chromosphere and the corona (the minimum temperature scale height of $\sim$ 70 km on average).
The average electron density at the apex was 6.2 $\times$ 10$^8$ cm$^{-3}$, while the average length of the coronal loop was 79 Mm. 
These characteristics aligned with those of warm loops that have been extensively studied since the work of \cite{1999ApJ...515..842A}. 

\begin{figure}
	\includegraphics[width=\columnwidth]{figure_024.eps}
    \caption{Time distance diagram of (a) temperature and (b) velocity along the field line.
    The black solid line in panel (b) indicates the contour at $T = 10^5$ K that represent the transition region.}
    \label{fig:td_diagram}
\end{figure}

Two significant dynamics pertaining to the transition region are noteworthy: the collision of shocks and evaporation/condensation processes.
Firstly, the transition region exhibited vertical motion in response to the interaction with MHD shocks emanating from the photosphere (see Fig. \ref{fig:td_diagram}). 
The typical altitude of the transition region measured approximately 9 Mm, with temporal variations of up to 3 Mm. 
These motions have previously been interpreted as spicule motion \citep{1982SoPh...75...35H, 1999ApJ...514..493K}.
This spicular dynamics did not change significantly through the whole calculation after the formation of the hot corona. 
Secondly, the shock-induced heating coincided with evaporation and condensation processes (see Fig. \ref{fig:time_evolution}b), resulting in the circulation of materials between the corona and the chromosphere. 
These phenomena contributed significantly to the dynamic ionization and recombination processes occurring within the region, rendering them pivotal factors for consideration in the examination of emission originating from the transition region. 
Subsequent subsections of this study will delve into these aspects in greater detail.

\subsection{Ionization fractions in the evaporation phase}

\begin{figure}
	\includegraphics[width=\columnwidth]{figure_004_C.eps}
    \caption{(a) A snapshot of ionization fractions of C ions as a function of $s$ for the entire loop.
    (b) Zoomed-in view of (a) focused on the transition region.
    Solid lines depict the NEI fractions, while dashed lines represent the ionization fraction in equilibrium state.
    }
    \label{fig:ion_frac_s}
\end{figure}

Significant departures from ionization equilibrium were observed in the middle of corona as well as near the transition region. 
Figure \ref{fig:ion_frac_s} presented a snapshot of the ionization fractions of C ions during the evaporation phase. 
The spatial distribution of ionization fractions exhibited two notable features.
First, in the middle corona, the spatial profile of ionization fractions appeared considerably smoother than that in the equilibrium state (Fig. \ref{fig:ion_frac_s}a). 
This phenomenon can be attributed to the longer ionization time scale, which prevents the ionization fraction from promptly responding to temperature fluctuations induced by shocks and acoustic waves. 
Previous simulations with time-dependent heating rates have demonstrated that NEI effects can arise due to temperature fluctuations, even in the absence of mass motion \citep{1982ApJ...255..783M}. 
Observations have also indicated that the variability time scales are often constrained by ionization processes, regardless of the underlying atmospheric dynamics \citep{1989SoPh..122..245G}.
Second, the peak of the ionization fraction for Li- and Na-like ions (e.g., \ion{C}{IV} in Fig. \ref{fig:ion_frac_s}b) was frequently displaced to higher altitudes. 
Consequently, the distributions of \ion{C}{III} and \ion{C}{V} were also shifted to higher altitudes.

The aforementioned deviations in the transition region were also evident in the probability distribution function (PDF) of the ionization fraction in temperature space (Fig. \ref{fig:ion_frac_te}).
Due to the effects of NEI, the ionization fraction did not solely depend on the local temperature, leading to a certain distribution with finite width at a specific temperature.
We discretized the temperature space into bins of $\Delta \log T = 0.05$ and computed PDFs, denoted as $F_{\rm i}$, for each temperature bin as follows:
\begin{equation}
F_{\rm i}(T;x) = P(T; n_{\rm i} \le x),
\end{equation}
where $i$ serves as the index for ion species;
$P(n_{\rm i} \le x)$ represents the probability that a particular ionization fraction, $n_{\rm i}$, is smaller than $x$ at temperature $T$.
To estimate PDF in the evaporation phase, we used the data between $t=$ 135 to $t=$ 145 min.
The solid lines in Fig. \ref{fig:ion_frac_te} depicted the ranges of 1-sigma levels for PDF ($F_{\rm i} \in [0.17,0.83]$), whereas the dashed lines illustrate the ionization fraction at the equilibrium state.
The over-fraction and shift toward higher temperature was found for \ion{C}{IV} while other C ions almost followed the ionization fraction in equilibrium.

\begin{figure}
	\includegraphics[width=\columnwidth]{figure_020_C.eps}
    \caption{Ionization fractions of C ions as a function of temperature.
    Solid lines depict 1 sigma intervals of PDF of the NEI fractions.
    Dashed lines represent the fractions in ionization equilibrium.
    }
    \label{fig:ion_frac_te}
\end{figure}

The reason for the enhancement of \ion{C}{IV} fraction can be primarily attributed to the evaporation from the chromosphere to the corona.
In figure \ref{fig:ion_frac_trace}, we depicted evolution of a certain plasma parcel co-moving with fluid to trace its temperature, electron density, position, and ionization fractions.
From $t=$ 137.5 min, the ionization fraction of \ion{C}{IV} increased as the plasma temperature increased to its formation temperature.
As time went on, the plasma parcel evaporated to coronal temperature from $t=$ 142.5 min in 40 sec, and then, the ionization fraction started to decrease.
Because the evaporation time scale of $\sim$ 40 sec was comparable to the recombination time scale, the ionization fraction stayed larger than that in equilibrium during the evaporation process.
The similar over-fractionation can be found in the steady solution of siphon flow \citep{1989ApJ...338.1131N, 1990ApJ...362..370S}.
The over-fraction of \ion{C}{IV} is also shown in the evaporation phase in nanoflare-heated loops \cite{1993ApJ...402..741H}.
Throughout the evaporation process, this plasma parcel encountered shocks at $t\sim$ 140 min and 140.8 min. Upon interaction with the shocks, the \ion{C}{IV} ionization fraction increased spontaneously. Despite these collisions occurring within a 10-second timeframe, which is shorter than the ionization time scales, the \ion{C}{IV} ionization fraction closely tracked the equilibrium fraction.
Considering that \ion{C}{III} and \ion{C}{V} displayed over- and under-fraction, respectively, the \ion{C}{IV} ionization fraction was likely maintained by an enhancement of ionization and a reduction in recombination during the shock passage.
Nano-flare-generated waves entering to the transition region from the corona could also modify the ionization fraction \citep{1993ApJ...402..741H}, although we did not observe this phenomena probably because our model lacks nanoflares. 

\begin{figure*}
	\includegraphics[width=\linewidth]{figure_029e_rise.eps}
    \caption{ (a) Temperature (dashed line), electron density (dotted line), position (solid line), and (b) ionization fractions of a traced particle in the evaporation phase.
    The solid and the dotted lines in panel (b) indicate the ionization fraction in non-equilibrium and equilibrium, respectively.
    }
    \label{fig:ion_frac_trace}
\end{figure*}

\subsection{Ionization fractions in the condensation and quasi-steady phase}

\begin{figure*}
	\includegraphics[width=\linewidth]{figure_029e_cond.eps}
    \caption{ The same format as in the figure \ref{fig:ion_frac_trace} in condensation phase.
    }
    \label{fig:ion_frac_trace_cond}
\end{figure*}

During the condensation phase, the \ion{C}{IV} ions exhibited, on average, an under-fraction.
Figure \ref{fig:ion_frac_trace_cond} displayed the pertinent physical characteristics of the condensing plasma parcel. 
Due to the finite duration required for the recombination process from \ion{C}{V} to \ion{C}{IV}, which exceeded the condensation timescale, the ionization fraction of \ion{C}{IV} remained below that of the equilibrium state. 
This under-fraction of \ion{C}{IV} is qualitatively consistent with the simulation by \cite{1993ApJ...402..741H} during the condensation phase in a nanoflare-heated loop.
Moreover, it was observed that the plasma parcel experienced a collision with a shock at $t\sim$ 311 min, during which the \ion{C}{IV} ionization fraction once again closely followed the equilibrium fraction, as was noted during the evaporation phase. Despite the plasma undergoing repeated evaporation (e.g., at $t\sim$ 307.8 min) and condensation episodes during the whole condensation phase, the average ionization fraction of \ion{C}{IV} was consistently smaller than that in the equilibrium state.

While there were fluctuations in the deviations from equilibrium, the average ionization fraction of \ion{C}{IV} remained nearly consistent with that of the equilibrium state during the quasi-steady phase.

\subsection{Synthetic UV intensity}

The synthetic UV intensities obtained using the coronal approximation were generally smaller than those derived with NEI by a maximum of 40\% during the evaporation phase (Table \ref{tab:ratio}). The UV intensity computation under the coronal approximation is governed by the following formula:
\begin{eqnarray}
    I(\lambda) = \int Ab(X) C(T, \lambda, N_{\rm e}) N_{\rm e} N_{\rm H} ds. \label{eq:intensity}
\end{eqnarray}
Here, $Ab(X)$ represents the abundance of element $X$, assuming photospheric abundance. The function $C(T, \lambda, N_{\rm e})$ corresponds to the contribution functions for each spectral line, calculated using the CHIANTI software.
For NEI plasma intensities, we modified the integrand in eq. (\ref{eq:intensity}) by multiplying it with the ionization fraction ratio, $N_{\rm i;NEI}/N_{\rm i;EI}$. Among several spectral lines, we specifically highlighted lines with a ratio $I_{\rm EI}/I_{\rm NEI}$ less than 0.8 and an intensity greater than 10 erg cm$^{-2}$ s$^{-1}$ sr$^{-1}$.
While the effect of NEI significantly enhanced the intensity through the over-fraction during the evaporation phase, our model still exhibited discrepancies between $R$ (defined as $I_{\rm EI}/I_{\rm NEI}$) and $I_{\rm th}/I_{\rm obs}$, as observed in \cite{1995ApJ...455L..85J} and \cite{2023MNRAS.521.4696D}. The under-fraction observed during the condensation and quasi-steady phases resulted in the ratio $R<1.1$ at most.

\begin{table}
\centering
\caption{Comparison of intensities from ionization fractions in equilibrium and non-equilibrium during the evaporation phase.}
\label{tab:ratio}
\begin{tabular}{lrrrrrp{8cm}<{\raggedright\arraybackslash}}
\hline
Ion (Seq)  & $\lambda$ [\AA] & $I_{\rm EI}$ & $I_{\rm NEI}$ & $R$& $I_{\rm th}/I_{\rm obs}$\\
\hline
\ion{C}{IV} (Li)&1548.19&3452.0&4328.2&0.8&0.197$^{\rm a}$, 0.31-1.28$^{\rm b}$\\
\ion{C}{IV} (Li)&1550.77&1725.0&2162.2&0.8&0.34-1.01$^{\rm b}$\\
\ion{C}{IV} (Li)&384.17&27.0&43.0&0.6&\\
\ion{C}{IV} (Li)&312.42&19.4&30.8&0.6&\\
\ion{C}{IV} (Li)&419.71&25.1&39.2&0.6&\\
\ion{C}{IV} (Li)&384.03&15.0&23.9&0.6&\\
\ion{N}{V} (Li)&1238.82&417.2&599.7&0.7&0.198$^{\rm a}$, 0.28$^{\rm b}$\\
\ion{N}{V} (Li)&1242.80&208.6&299.7&0.7&0.17-0.40$^{\rm b}$\\
\ion{N}{V} (Li)&247.71&10.5&15.6&0.7&\\
\ion{N}{V} (Li)&209.27&7.2&10.7&0.7&\\
\ion{N}{V} (Li)&266.38&7.5&11.4&0.7&\\
\ion{O}{VI} (Li)&1031.91&3586.7&4675.5&0.8&0.524$^{\rm a}$,0.5-0.71$^{\rm b}$\\
\ion{O}{VI} (Li)&1037.61&1785.8&2326.5&0.8&0.41-0.76$^{\rm b}$\\
\ion{O}{VI} (Li)&173.08&143.9&185.6&0.8&\\
\ion{O}{VI} (Li)&184.12&95.9&125.5&0.8&\\
\ion{O}{VI} (Li)&172.94&80.1&103.4&0.8&\\
\ion{O}{VI} (Li)&183.94&48.4&63.3&0.8&\\
\ion{Si}{IV} (Na)&1128.34&33.5&41.7&0.8&0.16-0.19$^{\rm b}$\\
\ion{Si}{IV} (Na)&1122.48&18.8&23.4&0.8&\\
\ion{S}{VI} (Na)&933.38&120.4&163.6&0.7&0.31-0.59$^{\rm b}$\\
\ion{S}{VI} (Na)&944.52&60.0&81.5&0.7&0.58-1.15$^{\rm b}$\\
\ion{S}{VI} (Na)&712.67&8.2&11.2&0.7&0.4$^{\rm b}$\\
\hline
\multicolumn{6}{p{8cm}}{
Note. Ion - emitting ion; Seq - ion isoelectronic sequence; $\lambda$ - wavelength; $I_{\rm EI}$ and $I_{\rm NEI}$ - predicted intensity in EI and NEI [ergs cm$^{-2}$ s$^{-1}$ sr$^{-1}$]; $R$ - ratio of $I_{\rm EI}$ to $I_{\rm NEI}$; $I_{\rm th}/I_{\rm obs}$ - ratio of theoretical predictions of intensity to observed intensity ($^{\rm a}$ \cite{1995ApJ...455L..85J} and $^{\rm b}$ \cite{2023MNRAS.521.4696D}).}\\
\end{tabular}
\end{table}

\subsection{The effect of broadening technique of the transition region}

In our study, we implemented a numerical broadening technique for the transition region to mitigate numerical diffusion effects near the lower transition region, which was identified as a factor reducing the impacts NEI. 
To investigate the impact of this technique, we conducted simulations with a broader transition region, specifically setting $T_{\rm c}=0.2$ MK. 
The results of this simulation revealed a increase in the ratio of $R$, such as an increase from 0.6 to 0.8 for \ion{C}{IV} ions 384.17 \AA, and similar increases were observed for other Li- and Na-like ions. 
This increase in the intensity ratio can be attributed to the broadening of the lower transition region, which results in a longer dynamical time scale for traversing the layer compared to realistic conditions, thus potentially reducing the influence of NEI.
Consequently, we consider the intensity ratio obtained under our parameter settings to represent an upper limit in this context.

\subsection{Mass circulation}

The magnitude of over-fraction evolved through the mass circulation process occurring between the corona and the chromosphere, as our model has unveiled distinct ionization fractions across various phases.
Our model disclosed that evaporation occurred prior to $t<$ 200 min, followed by the onset of condensation until approximately $t\sim$ 500 min. 
The duration of these mass cycles could be partly affected by NEI effect on the radiative cooling function that is suggested to increase the cooling time scale of the coronal loop \citep{2003A&A...401..699B} if we include the feedback effects.
Subsequently, the system maintained a quasi-steady state, at least until $t=$ 800 min, though it is worth noting that the continuity of this quasi-steady state beyond that point is not guaranteed.
Similar models with significantly extended computational durations have elucidated the cyclic evolution of coronal loops \citep{2019ApJ...885..164W}, a phenomenon that could be attributed to the thermal instability inherent in coronal loops \citep{1982A&A...108L...1K}.
While providing a quantitative estimate for the amplitude of over-fraction remains challenging, it is reasonable to anticipate that the ratio $R$ presented during the evaporation phase (as detailed in Table \ref{tab:ratio}) would undergo increase when taking the condensation phase into consideration.


\section{Conclusions}
In this study we have conducted 1.5-dimensional MHD simulations for Alfv\'{e}n-wave-heated coronal loops simultaneously solving a series of NEI equations for several ion species. 
After introducing the Alfv\'{e}nic fluctuations at the foot point of the loop, the system experiences the evaporation and condensation phases before it reaches the quasi-steady state.
During the evaporation phase, the over-fractionation of Li- and Na-like ions results in the intensity ratios, $R$, as low as 0.6. Conversely, during the condensation and quasi-steady phases, the under-fractionation leads to $R$ values as high as 1.1. These pronounced fluctuations in ionization fractions are primarily attributed to the processes of evaporation and condensation between the corona and the chromosphere. Interestingly, the collision with shocks do not significantly deviate the \ion{C}{IV} ion fraction from the ionization equilibrium.

While our model has successfully demonstrated a reduction in the intensity ratios $R$ during the evaporation phase, a noticeable gap between the model and observational data still persists. Bridging this gap necessitates further investigations into atomic physics aspects, such as density effects, photoionization, and charge transfer, which collectively have the potential to align the theoretically predicted UV intensities of transition region spectral lines with observations \citep{2023MNRAS.521.4696D}.
Furthermore, the consideration of first ionization potential effects may play a crucial role in narrowing the discrepancy between our model and observations, particularly in the case of \ion{Si}{IV}. This hypothesis has been discussed in prior studies and has the potential to explain anomalous behaviors in the line ratio between \ion{Si}{IV} and \ion{O}{IV} \citep{2015ApJ...802....5O,2016ApJ...817...46M}.

The observed over-fractionation of Li- and Na-like ions holds significant scientific interest, as it may serve as an indicator of mass motions closely linked to coronal heating mechanisms and mass loss processes. This phenomenon gains relevance in the context of the wealth of supporting evidence for impulsive heating events that drive the cycle of evaporation and condensation in the solar atmosphere \citep{2006SoPh..234...41K}. Consequently, the extent of over-fractionation could potentially offer valuable constraints on the amplitude and frequency of such impulsive heating events \citep{2011ApJS..194...26B}.
Furthermore, it's worth noting that the solar wind are known to expand the UV-observed solar atmosphere via the effects of NEI \citep{1964spre.conf..730N,1978ApJ...226..698M,1979ApJ...229L.101D}. Consequently, the degree of over-fraction could be considered as a metric for mass loss processes in the Sun. Importantly, anomalies in the ionization states of Li- and Na-like ions have also been identified in stellar atmospheres \citep{2002A&A...385..968D}, suggesting that the observation of these ions may provide valuable insights into the heating and mass-loss mechanisms in other stars.

Multi-dimensional simulations that incorporate NEI effects are crucial for conducting a quantitative comparison between theoretical models and complex observational data \citep{2013ApJ...767...43O, 2015ApJ...802....5O, 2016ApJ...817...46M}. However, such simulations remain a formidable challenge, even with the current computational resources at our disposal. This challenge arises from the necessity to accurately resolve the narrow transition region when solving the advection equations for Li- and Na-like ions.
The width of the transition region where those ions form, determined by the Field length or $\sqrt{\kappa T/|\rho {\cal L}|}$ \citep{1990ApJ...358..375B}, was sometimes going down to $\sim$ 20 km at $T\sim10^5$ K in our simulation. Notably, this width is broadened by a factor of $(T_c/10^5~{\rm K})^{5/2} \sim 2.8$ with our broadening technique, effectively reducing the width to approximately 7 km in the realistic situation. To achieve a high-resolution representation of this thin layer, typically requiring at least 10 grid points to mitigate numerical diffusion, we find it necessary to employ the adaptive mesh refinement schemes, even in one-dimensional simulations \citep{2003A&A...401..699B}.

The extent of over-fractionation is likely contingent on the mass flux or the rate of mass exchange between the corona and the chromosphere. However, our current model does not comprehensively explore this particular parameter. To conduct a thorough investigation of these parameters, it would be advantageous to employ time-steady solutions \citep{1989ApJ...338.1131N,1990ApJ...362..370S,2020ApJ...901..150G}.
By deriving the over- and under-fractionation behaviors of Li- and Na-like ions as functions of mass flux, we could potentially establish valuable constraints on coronal heating mechanisms and mass loss rates. These constraints would be particularly informative when compared with UV observations in future.

\section*{Acknowledgements}

This work was supported by JSPS KAKENHI Grant Number JP23K03456.
This study was carried by using the computational resource of the Center for Integrated Data Science, Institute for Space-Earth Environmental Research, Nagoya University through the joint research program.
CHIANTI is a collaborative project involving George Mason University, the University of Michigan (USA), University of Cambridge (UK) and NASA Goddard Space Flight Center (USA).

%%%%%%%%%%%%%%%%%%%%%%%%%%%%%%%%%%%%%%%%%%%%%%%%%%
\section*{Data Availability}
The data underlying this article will be shared upon reasonable request by the corresponding author.




%%%%%%%%%%%%%%%%%%%% REFERENCES %%%%%%%%%%%%%%%%%%

% The best way to enter references is to use BibTeX:

\bibliographystyle{mnras}
\bibliography{myref} % if your bibtex file is called example.bib


% Alternatively you could enter them by hand, like this:
% This method is tedious and prone to error if you have lots of references
%\begin{thebibliography}{99}
%\bibitem[\protect\citeauthoryear{Author}{2012}]{Author2012}
%Author A.~N., 2013, Journal of Improbable Astronomy, 1, 1
%\bibitem[\protect\citeauthoryear{Others}{2013}]{Others2013}
%Others S., 2012, Journal of Interesting Stuff, 17, 198
%\end{thebibliography}

%%%%%%%%%%%%%%%%%%%%%%%%%%%%%%%%%%%%%%%%%%%%%%%%%%


% Don't change these lines
\bsp	% typesetting comment
\label{lastpage}
\end{document}

% End of mnras_template.tex
