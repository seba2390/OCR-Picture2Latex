\providecommand{\bibDirName}{include}  % *Modification: define file location
\providecommand{\parentDir}{.}  % *Modification: define file location
\providecommand{\mainDir}{.}  % *Modification: define file location
\providecommand{\includeDirName}{include}  % *Modification: define file location

\newcommand{\tableDir}{\mainDir/tables/}
\newcommand{\figDir}{\mainDir/figs/}

\newcommand{\biblio}{\bibliographystyle{include/splncs04}\bibliography{\mainDir/\bibDirName/akazachk}}  % *Modification: added 
%\newcommand{\biblio}{\printbibliography} % for biblatex

%== Other  =================================================

\newcommand{\rlaptext}[1]{\rlap{\text{#1}}}
\DeclareMathAlphabet{\mathpzc}{OT1}{pzc}{m}{it}

%== Tikz  ==================================================

\newcommand{\xangle}{-20}
\newcommand{\yangle}{45}
\newcommand{\zangle}{90}

\newcommand{\xlength}{1}
\newcommand{\ylength}{1}
\newcommand{\zlength}{1}

\pgfmathsetmacro{\xx}{\xlength*cos(\xangle)}
\pgfmathsetmacro{\xy}{\xlength*sin(\xangle)}
\pgfmathsetmacro{\yx}{\ylength*cos(\yangle)}
\pgfmathsetmacro{\yy}{\ylength*sin(\yangle)}
\pgfmathsetmacro{\zx}{\zlength*cos(\zangle)}
\pgfmathsetmacro{\zy}{\zlength*sin(\zangle)}

\newcommand{\xX}{\xx}
\newcommand{\xY}{\xy}
\newcommand{\yX}{\yx}
\newcommand{\yY}{\yy}
\newcommand{\zX}{\zx}
\newcommand{\zY}{\zy}

%== Algorithms  ============================================

\RequirePackage{algorithmicx}
%\makeatletter
%\newcounter{algorithmicH}% New algorithmic-like hyperref counter
%\let\oldalgorithmic\algorithmic
%\renewcommand{\algorithmic}{%
%  \stepcounter{algorithmicH}% Step counter
%  \oldalgorithmic}% Do what was always done with algorithmic environment
%\renewcommand{\theHALG@line}{ALG@line.\thealgorithmicH.\arabic{ALG@line}}
%\makeatother
\algnewcommand\algorithmicinput{\textbf{Input:}}
\algnewcommand\Input{\item[\algorithmicinput]}
\algnewcommand\algorithmicinit{\textbf{Initialize:}}
\algnewcommand\Initialize{\item[\algorithmicinit]}
\algnewcommand\algorithmicoutput{\textbf{Output:}}
\algnewcommand\Output{\item[\algorithmicoutput]}
\algnewcommand\algorithmicpreprocessing{\textbf{Preprocessing:}}
\algnewcommand\Preprocessing{\item[\algorithmicpreprocessing]}
\newenvironment{varalgorithm}[1]
  {\algorithm\renewcommand{\thealgorithm}{#1}}
  {\endalgorithm}

\makeatletter
\let\OldStatex\Statex
\renewcommand{\Statex}[1][3]{%
  \setlength\@tempdima{\algorithmicindent}%
  {{\parfillskip0pt\par}}\OldStatex\hskip\dimexpr#1\@tempdima\relax}
\makeatother

%== Text mode  =============================================

% complexity class
\newcommand{\complexity}[1]{\ensuremath{\mathcal{#1}}}

%% problem such as Max-Cut (conflicts with informs style)
%\newcommand{\problem}[1]{\textsc{#1}}

%== Custom math commands  ==================================

% "such that" such as used in the definition of a set
%\newcommand{\suchthat}{\mid}
\newcommand{\suchthat}{: }
%\newcommand{\suchthat}{\ \vert\ }

% "given" such as used in conditional probabilities and expectations
\newcommand{\given}{\nonscript\;\middle|\nonscript\;}

% indicator function (1 if true; 0 if false)
\newcommand{\indicator}[1]{\mathbf{1}\left[ #1\right]}

% definition of variable
%\newcommand{\defas}{\vcentcolon=} % vcentcolon needs mathtools
\newcommand{\defas}{:=}
\newcommand{\defeq}{\defas}
\newcommand{\eqdef}{=:}

\DeclareMathOperator{\bigO}{\mathcal{O}} % big O(micron)
\DeclareMathOperator*{\argmax}{arg\,max}
\DeclareMathOperator*{\argmin}{arg\,min}
\DeclareMathOperator*{\sgn}{sgn}
\DeclareMathOperator*{\conv}{conv}
\DeclareMathOperator*{\cone}{cone}
\DeclareMathOperator*{\rec}{rec}
\DeclareMathOperator*{\interior}{int}
\DeclareMathOperator*{\relint}{relint}
\DeclareMathOperator*{\relbd}{relbd}
\DeclareMathOperator*{\bd}{bd}
\DeclareMathOperator*{\cl}{cl}
\DeclareMathOperator*{\proj}{proj}
\DeclareMathOperator*{\err}{err}
\DeclareMathOperator*{\score}{sc}
\DeclareMathOperator*{\poly}{poly}
\newcommand{\jhat}{\hat\jmath}

% Linear algebra stuff
\DeclareMathOperator*{\cspan}{span}
\DeclareMathOperator*{\rspan}{rowspan}
\DeclareMathOperator*{\rank}{rank}
% \DeclareMathOperator*{\ker}{ker}
\DeclareMathOperator{\tr}{tr}
\DeclareMathOperator{\ind}{\mathbbm{1}}
\renewcommand\vec{\mathbf}

%\DeclarePairedDelimiter{\abs}{\lvert}{\rvert} % abs, if mathtools is used
\newcommand{\abs}[1]{\lvert#1\rvert}
\newcommand{\bigabs}[1]{\left\lvert#1\right\rvert}
\newcommand{\card}[1]{\lvert #1 \rvert}
\newcommand{\cardinality}[1]{\lvert #1 \rvert}
\newcommand{\norm}[1]{\lVert #1 \rVert}
\newcommand{\bignorm}[1]{\left\lVert #1 \right\rVert}
\newcommand{\innerprod}[1]{\langle #1 \rangle}
\newcommand{\bigmid}{\,\middle|\,}

\newcommand{\ceil}[1]{\left\lceil #1 \right\rceil}
\newcommand{\floor}[1]{\left\lfloor #1 \right\rfloor}

\newcommand{\polar}{^{\circ}}
\newcommand{\reversepolar}{^{-}}

% scientific notation, 1\e{9} will print as 1x10^9
\providecommand{\e}[1]{\ensuremath{\times 10^{#1}}}

% number systems
\newcommand{\field}[1]{\mathbbm{#1}}
\newcommand{\reals}{\field{R}}
\newcommand{\integers}{\field{Z}}
\newcommand{\rationals}{\field{Q}}
\newcommand{\posreals}{\reals_{>0}}
\newcommand{\nonnegreals}{\reals_{\ge 0}}
\newcommand{\posintegers}{\integers_{>0}}
\newcommand{\nonnegintegers}{\integers_{\ge 0}}
\newcommand{\posrationals}{\rationals_{>0}}
\newcommand{\nonnegrationals}{\rationals_{\ge 0}}
\newcommand{\R}{\reals}
\newcommand{\Rplus}{\nonnegreals}
\newcommand{\Rplusplus}{\posreals}
\newcommand{\C}{\field{C}}
\newcommand{\N}{\field{N}}
\newcommand{\Z}{\integers}
\newcommand{\Zplus}{\nonnegintegers}
\newcommand{\Zplusplus}{\posintegers}
\newcommand{\F}{\field{F}}
\newcommand{\Q}{\rationals}
%\newcommand{\E}{\mathbbm{E}}

%\newcommand{\Var}{\operatorname{Var}}
%\newcommand{\Cov}[1]{\text{Cov}\left[ #1 \right]}
%\newcommand{\E}{\mathbbm{E}}
%\newcommand{\rv}[1]{\boldsymbol{#1}}

\renewcommand{\Pr}{\mathop{\bf Pr\/}}
\newcommand{\E}{\mathop{\bf E\/}}
\newcommand{\Ex}{\mathop{\bf E\/}}
\newcommand{\Var}{\mathop{\bf Var\/}}
\newcommand{\Cov}{\mathop{\bf Cov\/}}
\newcommand{\stddev}{\mathop{\bf stddev\/}}

% short forms
\newcommand{\eps}{\epsilon}
\newcommand{\lam}{\lambda}
\newcommand{\vphi}{\varphi}
\newcommand{\la}{\langle}
\newcommand{\ra}{\rangle}
\newcommand{\wt}[1]{\widetilde{#1}}
\newcommand{\wh}[1]{\widehat{#1}}
%\newcommand{\ul}[1]{\underline{#1}}
\newcommand{\ol}[1]{\overline{#1}}

% calligraphic letters
\newcommand{\calA}{\mathcal{A}}
\newcommand{\calB}{\mathcal{B}}
\newcommand{\calC}{\mathcal{C}}
\newcommand{\calD}{\mathcal{D}}
\newcommand{\calE}{\mathcal{E}}
\newcommand{\calF}{\mathcal{F}}
\newcommand{\calG}{\mathcal{G}}
\newcommand{\calH}{\mathcal{H}}
\newcommand{\calI}{\mathcal{I}}
\newcommand{\calJ}{\mathcal{J}}
\newcommand{\calK}{\mathcal{K}}
\newcommand{\calL}{\mathcal{L}}
\newcommand{\calM}{\mathcal{M}}
\newcommand{\calN}{\mathcal{N}}
\newcommand{\calO}{\mathcal{O}}
\newcommand{\calP}{\mathcal{P}}
\newcommand{\calQ}{\mathcal{Q}}
\newcommand{\calR}{\mathcal{R}}
\newcommand{\calS}{\mathcal{S}}
\newcommand{\calT}{\mathcal{T}}
\newcommand{\calU}{\mathcal{U}}
\newcommand{\calV}{\mathcal{V}}
\newcommand{\calW}{\mathcal{W}}
\newcommand{\calX}{\mathcal{X}}
\newcommand{\calY}{\mathcal{Y}}
\newcommand{\calZ}{\mathcal{Z}}

\newcommand\T{{\mathpalette\raiseT\intercal}} % custom transpose operator
\newcommand\raiseT[2]{\raisebox{0.25ex}{$#1#2$}}
\newcommand{\phT}{{\vphantom{\T}}} % vphantom to adjust subscript due to transpose operator

%% C plus plus
\usepackage{relsize}
%c from texinfo.tex
\def\ifmonospace{\ifdim\fontdimen3\font=0pt }

\newcommand{\Iff}{if\textcompwordmark f\xspace} % if and only iff

\def\Cpp{%
\ifmonospace%
    C++%
\else%
    %C\kern-.1667em\raise.30ex\hbox{\smaller{++}}%
    C\nolinebreak[4]\raisebox{0.5ex}{\tiny\textbf{++}}%
\fi%
\spacefactor1000 }

%== Cross-project definitions  =============================

\newcommand{\Clp}{\texttt{Clp}}
\newcommand{\Cplex}{\texttt{CPLEX}}
\newcommand{\Gurobi}{\texttt{Gurobi}}

\newcommand\basicvars{\mathcal{B}}
\newcommand\NB{\mathcal{N}}
\DeclareMathOperator{\vertices}{vert}
\newcommand{\pointset}{\mathcal{P}}
\newcommand{\rayset}{\mathcal{R}}
\newcommand{\CutLP}{CutLP}

\newcommand{\opt}[1]{\bar{#1}}
\newcommand{\lpopt}{\opt x}

\newcommand{\facetindex}[1]{^{#1}}
\newcommand{\initfacetindex}[1]{\init\facetindex{#1}}
\newcommand{\splitindex}[1]{_{#1}}

\newcommand{\Pindex}[1]{P\splitindex{#1}}
\newcommand{\Sindex}[1]{S\splitindex{#1}}
\newcommand{\Pk}{\Pindex{k}}
\newcommand{\Sk}{\Sindex{k}}

\newcommand{\onfloor}{\facetindex{1}}
\newcommand{\onceil}{\facetindex{2}}
\newcommand{\floorpointset}{\pointset\onfloor}
\newcommand{\ceilpointset}{\pointset\onceil}
\newcommand{\floorrayset}{\rayset\onfloor}
\newcommand{\ceilrayset}{\rayset\onceil}

%\newcommand{\intvars}{I}
\newcommand{\intvars}{\mathcal{I}}
%\newcommand{\PI}{P_{\intvars}}
\newcommand{\PI}{P_{I}}
\newcommand{\OPTCuts}{z'}
\newcommand{\OPTLP}{z_{\mathrm{LP}}}
\newcommand{\OPTIP}{z_{\mathrm{IP}}}
\DeclareMathOperator{\GAP}{\%\ gap\ closed}
\newcommand\LB{\mathit{LB}}
\newcommand\UB{\mathit{UB}}

%== Theorems  ==============================================
\ifspringer
%\newtheorem{theorem}{Theorem}
%\newtheorem{corollary}[theorem]{Corollary}
%\newtheorem{lemma}[theorem]{Lemma}
%\newtheorem{proposition}[theorem]{Proposition}
%\newtheorem{claim}[theorem]{Claim}
%\newtheorem{example}[theorem]{Example}
%\newtheorem{remark}[theorem]{Remark}
%\newtheorem{observation}[theorem]{Observation}
%\newtheorem{definition}[theorem]{Definition}
%\newtheorem{conjecture}[theorem]{Conjecture}
%
%\newtheorem*{theorem*}{Theorem}
%\newtheorem*{corollary*}{Corollary}
%\newtheorem*{lemma*}{Lemma}
%\newtheorem*{proposition*}{Proposition}
%\newtheorem*{claim*}{Claim}
%\newtheorem*{example*}{Example}
%\newtheorem*{remark*}{Remark}
%\newtheorem*{observation*}{Observation}
%\newtheorem*{definition*}{Definition}
%\newtheorem*{conjecture*}{Conjecture}
\else
\newtheorem{theorem}{Theorem}

%% Everything below will be numbered sequentially
\newtheorem{corollary}[theorem]{Corollary}
\newtheorem{lemma}[theorem]{Lemma}
\newtheorem{proposition}[theorem]{Proposition}
\newtheorem{claim}[theorem]{Claim}
\newtheorem{example}[theorem]{Example}
\newtheorem{remark}[theorem]{Remark}
\newtheorem{observation}[theorem]{Observation}
\newtheorem{definition}[theorem]{Definition}
\newtheorem{conjecture}[theorem]{Conjecture}

\newtheorem{thm}[theorem]{Theorem}
\newtheorem{cor}[theorem]{Corollary}
\newtheorem{lem}[theorem]{Lemma}
\newtheorem{prop}[theorem]{Proposition}
\newtheorem{rem}[theorem]{Remark}
\newtheorem{obs}[theorem]{Observation}
\newtheorem{defn}[theorem]{Definition}
\newtheorem{conj}[theorem]{Conjecture}
\fi