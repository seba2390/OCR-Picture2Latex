%%%%%%%%%%%%%%%%%%%%%%%%%%%%%%%%%%%%%%%%%%%%%%%%%%%%%%%%%%%%%%%%%%%%%%%%%%%%%%%%
%2345678901234567890123456789012345678901234567890123456789012345678901234567890
%        1         2         3         4         5         6         7         8
\documentclass[letterpaper, 10 pt, conference]{ieeeconf}  % Comment this line out if you need a4paper

%\documentclass[a4paper, 10pt, conference]{ieeeconf}      % Use this line for a4 paper

\IEEEoverridecommandlockouts                              % This command is only needed if 
% you want to use the \thanks command

\overrideIEEEmargins                                      % Needed to meet printer requirements.

%In case you encounter the following error:
%Error 1010 The PDF file may be corrupt (unable to open PDF file) OR
%Error 1000 An error occurred while parsing a contents stream. Unable to analyze the PDF file.
%This is a known problem with pdfLaTeX conversion filter. The file cannot be opened with acrobat reader
%Please use one of the alternatives below to circumvent this error by uncommenting one or the other
%\pdfobjcompresslevel=0
%\pdfminorversion=4

% See the \addtolength command later in the file to balance the column lengths
% on the last page of the document

% The following packages can be found on http:\\www.ctan.org
%\usepackage{graphics} % for pdf, bitmapped graphics files
%\usepackage{epsfig} % for postscript graphics files
%\usepackage{mathptmx} % assumes new font selection scheme installed
%\usepackage{times} % assumes new font selection scheme installed
\usepackage{amsmath} % assumes amsmath package installed
\usepackage{amssymb}  % assumes amsmath package installed

%%% MY PACKAGES
\usepackage{amssymb}
\usepackage{amsmath}
\usepackage{color}
\usepackage{graphicx}
%\usepackage{amsthm}
\usepackage{dsfont}
\usepackage{mathrsfs}
\usepackage{etoolbox}
\usepackage{float}
%\usepackage{subfig}
\usepackage{subfigure}
%\usepackage{subfigure}
\usepackage{amsbsy}
\usepackage{mathabx}
\usepackage{comment}
\usepackage{hyperref}
\usepackage{algorithm}
\usepackage{algpseudocode}
\usepackage{caption}
\captionsetup{font=small}
\usepackage{relsize}
\usepackage{siunitx}
\usepackage{soul}
\sisetup{output-complex-root = \mathbf{i}}
\usepackage[noadjust]{cite}
\usepackage{textcomp}
\usepackage{multirow}
\usepackage[normalem]{ulem}

% OUR COLORS
\definecolor{aoenglish}{rgb}{0.0, 0.5, 0.0}
\definecolor{darkblue}{rgb}{0.0, 0.0, 0.55}
\definecolor{darkmagenta}{rgb}{0.55, 0.0, 0.55}
\definecolor{electricviolet}{rgb}{0.56, 0.0, 1.0}
\definecolor{electricyellow}{rgb}{1.0, 1.0, 0.0}
\definecolor{forestgreen}{rgb}{0.13, 0.55, 0.13}
\definecolor{fuchsia}{rgb}{1.0, 0.0, 1.0}
\definecolor{gamboge}{rgb}{0.89, 0.61, 0.06}
\definecolor{goldenpoppy}{rgb}{0.99, 0.76, 0.0}
\definecolor{indigo}{rgb}{0.29, 0.0, 0.51}
\definecolor{internationalorange}{rgb}{1.0, 0.31, 0.0}
\definecolor{lava}{rgb}{0.81, 0.06, 0.13}
\definecolor{selectiveyellow}{rgb}{1.0, 0.73, 0.0}
\definecolor{turquoiseblue}{rgb}{0.0, 1.0, 0.94}
\definecolor{turquoise}{rgb}{0.19, 0.84, 0.78}

%%% MY COMMANDS
\definecolor{modcol}{RGB}{255,102,178}
\newtheorem{SERMprob}{\textbf{Problem}}

\newcommand{\VBr}[1]{\textcolor{magenta}{#1}}
\newcommand{\MFa}[1]{\textcolor{blue}{#1}}
\newcommand{\SFo}[1]{\textcolor{black}{#1}}
\newcommand{\ACh}[1]{\textcolor{black}{#1}}

\newcommand{\ToDo}[1]{\textcolor{blue}{#1}}
\newcommand{\Mod}[1]{\textcolor{blue}{#1}}
\newcommand{\Mods}[1]{\textcolor{modcol}{\sout{#1}}}
\definecolor{MFabg}{RGB}{76, 191, 229}
\newcommand{\MFatodo}[1]{\todo[color=MFabg,inline]{\textbf{Marco}: #1}}
\newtheorem{prb}{Problem}
\newtheorem{prop}{Proposition}
\newtheorem{cor}{Corollary}
\newtheorem{ex}{Example}
\newtheorem{thm}{Theorem}
\newtheorem{lem}{Lemma}
\newtheorem{dfn}{Definition}
\newtheorem{rmk}{Remark}
\def \nI {n_{I}}
\def \nM {n_{M}}
\def \VI {\Vmc_{I}}
\def \VM {\Vmc_{M}}

\newcommand{\colvec}[2][.9]{%
	\scalebox{#1}{%
		\renewcommand{\arraystretch}{.9}%
		$\begin{bmatrix}#2\end{bmatrix}$%
	}
}


%% INPUT FILES
%%% NOTATION

%% Set theory
\renewcommand{\emptyset}{\varnothing} % empty set
\newcommand{\Rset}[1]{\mathbb{R}^{#1}} % R^n
\newcommand{\Cset}[1]{\mathbb{C}^{#1}} % C^n
\newcommand{\Zset}[1]{\mathbb{Z}^{#1}} % Z^n
\newcommand{\Nset}[1]{\mathbb{N}^{#1}} % N^n

%% Bold symbols
\def \a {\mathbf{a}} 
\def \b {\mathbf{b}} 
\def \c {\mathbf{c}} 
\def \d {\mathbf{d}} 
\def \e {\mathbf{e}} 
\def \f {\mathbf{f}} 
\def \g {\mathbf{g}} 
\def \h {\mathbf{h}} 
%\def \i {\mathbf{i}} 
\def \j {\mathbf{j}}
\def \k {\mathbf{k}}
\def \l {\mathbf{l}} 
\def \m {\mathbf{m}} 
\def \n {\mathbf{n}} 
\def \o {\mathbf{o}} 
\def \p {\mathbf{p}} 
\def \q {\mathbf{q}} 
\def \r {\mathbf{r}} 
\def \s {\mathbf{s}} 
\def \t {\mathbf{t}} 
\def \u {\mathbf{u}} 
\def \v {\mathbf{v}} 
\def \w {\mathbf{w}} 
\def \x {\mathbf{x}} 
\def \y {\mathbf{y}} 
\def \z {\mathbf{z}} 

\def \A {\mathbf{A}} 
\def \B {\mathbf{B}} 
\def \C {\mathbf{C}} 
\def \D {\mathbf{D}} 
\def \E {\mathbf{E}} 
\def \F {\mathbf{F}} 
\def \G {\mathbf{G}} 
\def \H {\mathbf{H}} 
\def \I {\mathbf{I}} 
\def \J {\mathbf{J}} 
\def \K {\mathbf{K}} 
\def \L {\mathbf{L}} 
\def \M {\mathbf{M}} 
\def \N {\mathbf{N}} 
\def \O {\mathbf{O}} 
\def \P {\mathbf{P}} 
\def \Q {\mathbf{Q}} 
\def \R {\mathbf{R}} 
\def \S {\mathbf{S}} 
\def \T {\mathbf{T}} 
\def \U {\mathbf{U}} 
\def \V {\mathbf{V}} 
\def \W {\mathbf{W}} 
\def \X {\mathbf{X}} 
\def \Y {\mathbf{Y}} 
\def \Z {\mathbf{Z}} 


%% Capital letters in mathcal
\def\Amc{\mathcal{A}}
\def\Bmc{\mathcal{B}}
\def\Cmc{\mathcal{C}}
\def\Dmc{\mathcal{D}}
\def\Emc{\mathcal{E}}
\def\Fmc{\mathcal{F}}
\def\Gmc{\mathcal{G}}
\def\Hmc{\mathcal{H}}
\def\Imc{\mathcal{I}}
\def\Jmc{\mathcal{J}}
\def\Kmc{\mathcal{K}}
\def\Lmc{\mathcal{L}}
\def\Mmc{\mathcal{M}}
\def\Nmc{\mathcal{N}}
\def\Omc{\mathcal{O}}
\def\Pmc{\mathcal{P}}
\def\Qmc{\mathcal{Q}}
\def\Rmc{\mathcal{R}}
\def\Smc{\mathcal{S}}
\def\Tmc{\mathcal{T}}
\def\Umc{\mathcal{U}}
\def\Vmc{\mathcal{V}}
\def\Wmc{\mathcal{W}}
\def\Xmc{\mathcal{X}}
\def\Ymc{\mathcal{Y}}
\def\Zmc{\mathcal{Z}}

% bold Greek letters
\newcommand{\Balpha}{\boldsymbol{\alpha}}
\newcommand{\Bbeta}{\boldsymbol{\beta}}
\newcommand{\Bgamma}{\boldsymbol{\gamma}}
\newcommand{\Bdelta}{\boldsymbol{\delta}}
\newcommand{\Bepsilon}{\boldsymbol{\epsilon}}
\newcommand{\Bvarepsilon}{\boldsymbol{\varepsilon}}
\newcommand{\Bzeta}{\boldsymbol{\zeta}}
\newcommand{\Beta}{\boldsymbol{\eta}}
\newcommand{\Btheta}{\boldsymbol{\theta}}
\newcommand{\Bvartheta}{\boldsymbol{\vartheta}}
\newcommand{\Biota}{\boldsymbol{\iota}}
\newcommand{\Bkappa}{\boldsymbol{\kappa}}
\newcommand{\Blambda}{\boldsymbol{\lambda}}
\newcommand{\Bmu}{\boldsymbol{\mu}}
\newcommand{\Bnu}{\boldsymbol{\nu}}
\newcommand{\Bxi}{\boldsymbol{\xi}}
\newcommand{\Bomicron}{\boldsymbol{\omicron}}
\newcommand{\Bpi}{\boldsymbol{\pi}}
\newcommand{\Brho}{\boldsymbol{\rho}}
\newcommand{\Bsigma}{\boldsymbol{\sigma}}
\newcommand{\Bvarsigma}{\boldsymbol{\varsigma}}
\newcommand{\Btau}{\boldsymbol{\tau}}
\newcommand{\Bupsilon}{\boldsymbol{\upsilon}}
\newcommand{\Bphi}{\boldsymbol{\phi}}
\newcommand{\Bvarphi}{\boldsymbol{\varphi}}
\newcommand{\Bchi}{\boldsymbol{\chi}}
\newcommand{\Bpsi}{\boldsymbol{\psi}}
\newcommand{\Bomega}{\boldsymbol{\omega}}

\newcommand{\BAlpha}{\boldsymbol{\Alpha}}
\newcommand{\BBeta}{\boldsymbol{\Beta}}
\newcommand{\BGamma}{\boldsymbol{\Gamma}}
\newcommand{\BDelta}{\boldsymbol{\Delta}}
\newcommand{\BEpsilon}{\boldsymbol{\Epsilon}}
%\newcommand{\Bvarepsilon}{\boldsymbol{\varepsilon}}
\newcommand{\BZeta}{\boldsymbol{\Zeta}}
\newcommand{\BEta}{\boldsymbol{\Eta}}
\newcommand{\BTheta}{\boldsymbol{\Theta}}
%\newcommand{\Bvartheta}{\boldsymbol{\vartheta}}
\newcommand{\BIota}{\boldsymbol{\Iota}}
\newcommand{\BKappa}{\boldsymbol{\Kappa}}
\newcommand{\BLambda}{\boldsymbol{\Lambda}}
\newcommand{\BMu}{\boldsymbol{\Mu}}
\newcommand{\BNu}{\boldsymbol{\Nu}}
\newcommand{\BXi}{\boldsymbol{\Xi}}
\newcommand{\BOmicron}{\boldsymbol{\Omicron}}
\newcommand{\BPi}{\boldsymbol{\Pi}}
\newcommand{\BRho}{\boldsymbol{\Rho}}
\newcommand{\BSigma}{\boldsymbol{\Sigma}}
%\newcommand{\Bvarsigma}{\boldsymbol{\varsigma}}
\newcommand{\BTau}{\boldsymbol{\Tau}}
\newcommand{\BUpsilon}{\boldsymbol{\Upsilon}}
\newcommand{\BPhi}{\boldsymbol{\Phi}}
%\newcommand{\Bvarphi}{\boldsymbol{\varphi}}
\newcommand{\BChi}{\boldsymbol{\Chi}}
\newcommand{\BPsi}{\boldsymbol{\Psi}}
\newcommand{\BOmega}{\boldsymbol{\Omega}}

% scr fonts
\def \Ascr {\mathscr{A}}
\def \Bscr {\mathscr{B}}
\def \Cscr {\mathscr{C}}
\def \Dscr {\mathscr{D}}
\def \Escr {\mathscr{E}}
\def \Fscr {\mathscr{F}}
\def \Gscr {\mathscr{G}}
\def \Hscr {\mathscr{H}}
\def \Iscr {\mathscr{I}}
\def \Jscr {\mathscr{J}}
\def \Kscr {\mathscr{K}}
\def \Lscr {\mathscr{L}}
\def \Mscr {\mathscr{M}}
\def \Nscr {\mathscr{N}}
\def \Oscr {\mathscr{O}}
\def \Pscr {\mathscr{P}}
\def \Qscr {\mathscr{Q}}
\def \Rscr {\mathscr{R}}
\def \Sscr {\mathscr{S}}
\def \Tscr {\mathscr{T}}
\def \Uscr {\mathscr{U}}
\def \Vscr {\mathscr{V}}
\def \Wscr {\mathscr{W}}
\def \Xscr {\mathscr{X}}
\def \Yscr {\mathscr{Y}}
\def \Zscr {\mathscr{Z}}

% rm fonts
\def \arm {\mathrm{a}}
\def \brm {\mathrm{b}}
\def \crm {\mathrm{c}}
\def \drm {\mathrm{d}}
\def \erm {\mathrm{e}}
\def \frm {\mathrm{f}}
\def \grm {\mathrm{g}}
\def \hrm {\mathrm{h}}
\def \irm {\mathrm{i}}
\def \jrm {\mathrm{j}}
\def \krm {\mathrm{k}}
\def \lrm {\mathrm{l}}
\def \mrm {\mathrm{m}}
\def \nrm {\mathrm{n}}
\def \orm {\mathrm{o}}
\def \prm {\mathrm{p}}
\def \qrm {\mathrm{q}}
\def \srm {\mathrm{s}}
\def \trm {\mathrm{t}}
\def \urm {\mathrm{u}}
\def \vrm {\mathrm{v}}
\def \wrm {\mathrm{w}}
\def \xrm {\mathrm{x}}
\def \yrm {\mathrm{y}}
\def \zrm {\mathrm{z}}

\def \Arm {\mathrm{A}}
\def \Brm {\mathrm{B}}
\def \Crm {\mathrm{C}}
\def \Drm {\mathrm{D}}
\def \Erm {\mathrm{E}}
\def \Frm {\mathrm{F}}
\def \Grm {\mathrm{G}}
\def \Hrm {\mathrm{H}}
\def \Irm {\mathrm{I}}
\def \Jrm {\mathrm{J}}
\def \Krm {\mathrm{K}}
\def \Lrm {\mathrm{L}}
\def \Mrm {\mathrm{M}}
\def \Nrm {\mathrm{N}}
\def \Orm {\mathrm{O}}
\def \Prm {\mathrm{P}}
\def \Qrm {\mathrm{Q}}
\def \Rrm {\mathrm{R}}
\def \Srm {\mathrm{S}}
\def \Trm {\mathrm{T}}
\def \Urm {\mathrm{U}}
\def \Vrm {\mathrm{V}}
\def \Wrm {\mathrm{W}}
\def \Xrm {\mathrm{X}}
\def \Yrm {\mathrm{Y}}
\def \Zrm {\mathrm{Z}}


%%%%%%%%%%%%%%%%%%%%%%%

%\newtheorem{prb}{Poblem}
%\newtheorem{prop}{Proposition}
%\newtheorem{cor}{Corollary}
%\newtheorem{ex}{Example}
%\newtheorem{fact}{Fact}
%\newtheorem{thm}{Theorem}
%\newtheorem{lem}{Lemma}
%\newtheorem{dfn}{Definition}
%\newtheorem{asm}{Assumption}
%\theoremstyle{remark}
%\newtheorem{rmk}{Remark}
%\newtheorem*{prob}{\textbf{ Problem}}

%\theoremstyle{definition}
\newtheorem{defn}{Definition}[section]
\newtheorem{asm}{Assumption}[section]
%
%\theoremstyle{plain}
%\newtheorem{thm}{Theorem}[section]
%\newtheorem{lem}{Lemma}[section]
%\newtheorem{prop}{Proposition}[section]




%-------------------------------------


% --------------------

%% Derivatives
\newcommand{\der}[2]{\dfrac{\drm{#1}}{\drm{#2}}} % total derivative
\newcommand{\derk}[3]{\dfrac{\drm^{#3}{#1}}{\drm{#2}^{#3}}} % total derivative of the k-th order
\newcommand{\pder}[2]{\dfrac{\partial{#1}}{\partial{#2}}} % partial derivative
\newcommand{\pderk}[3]{\dfrac{\partial^{#3}{#1}}{\partial{#2}^{#3}}} % partial derivative of the k-th order

% continuous space
\newcommand{\contspace}[1]{\Cscr^{#1}} % smooth spaces

% distance
\newcommand{\dist}[1] 	{\mathrm{dist}({#1})}
\newcommand{\normeu}[1] {\left\|{#1}\right\|_{2}} % Euclidean norm
\newcommand{\normp}[2] {\left\|{#1}\right\|_{#2}}
\newcommand{\angb}[1] {\left<{#1}\right>}

\newcommand{\Exp}[1]{\expectation[#1]} % expectation

\def \ones			{{\mathds{1}}} % ones vector
\def \zeros			{{\mathbf{0}}} % ones vector
\newcommand{\onesvec}[1] 	{\ones_{#1}} % ones vector
\newcommand{\zerovec}[1] 	{\mathbf{0}_{#1}} % zero vector


\DeclareMathOperator{\diag}{diag} % diagonal 
%\DeclareMathOperator{\Diag}{Diag} % Diagonal
%\DeclareMathOperator{\tr}{tr} % trace
\DeclareMathOperator{\rk}{rk} % rank
\DeclareMathOperator{\coft}{adj} % adjugate
\DeclareMathOperator{\im}{im} % image

\newcommand{\eye}[1]{\mathbf{I}_{#1}} % identity matrix
\renewcommand{\vec}[3]{\mathrm{vec}_{#1}^{#2}(#3)} % vectorization operator


\def \hadamard {\odot} % Hadamard product
\def \kronecker {\otimes} % Kronecker product
\DeclareMathOperator{\Imag}{\Im} % Image of a linear application

% -------------------------------



%% Graph Theory
\def \graph		{\mathcal{G}} % graph
\def \subgraph	{\graph_{S}} % subgraph
\def \Csubgraph	{\graph_{\overline{S}}} % subgraph complement
\def \grapho	{\graph^{o}} % oriented graph
\def \graphd	{\graph^{d}} % directed graph
\def \graphw	{\graph^{w}} % weighted graph
\def \graphv	{\graph^{v}} % weighted graph (vertices)
\def \graphe	{\graph^{e}} % weighted graph (edges)
\def \vertsym 	{v} % vertex symbol
\newcommand		{\vertex}[1]{\vertsym_{#1}} % a vertex
\newcommand		{\weightv}[1]{w_{\vertex{#1}}} % a vertex weight
\def \nodes		{\mathcal{V}} % set of vertices
\def \Snodes	{\nodes_{S}} % subset of vertices
\def \CSnodes	{\nodes_{\overline{S}}} % complement subset of vertices
\def \edges		{\mathcal{E}} % set of edges
\def \Sedges	{\edges_{S}} % subset of edges
\def \CSedges	{\edges_{\overline{S}}} % complement subset of edges
\def \cutS 		{\partial\Sedges} % cut of a graph
\def \cutCS 	{\partial\CSedges} % cut of a graph
\def \weights	{\mathcal{W}} % set of weights
\def \weightsv	{\weights_{v}} % set of vertex weights
\def \weightse	{\weights_{e}} % set of edge weights
\def \edgesd	{\edges^{d}} % set of edges (directed)
\def \edgeso	{\edges^{o}} % set of edges (oriented)
\newcommand		{\neighbors}[1]{\mathcal{N}_{#1}}	% neighborhood of a node
\newcommand		{\neighborsi}[1]{\overline{\mathcal{N}}_{#1}}
\newcommand		{\edge}[1]{e_{#1}} % an edge
\newcommand		{\weighte}[1]{w_{\edge{#1}}} % an edge weight
\def \nonodes 	{n} % number of nodes
\def \Snonodes 	{\nonodes_{S}} % number of nodes in a subgraph
\def \CSnonodes {\nonodes_{\overline{S}}} % number of nodes in a subgraph complement
\def \noedges 	{E} % number of edges
%\newcommand{\vol}[1]	{\mathrm{vol}({#1})} % volume of a graph
\renewcommand{\deg}[1]	{\mathrm{deg}({#1})} % degree of a vertex
\newcommand{\degv}[1]	{\mathrm{deg}_{v}({#1})} % vertex-degree
\newcommand{\dege}[1]	{\mathrm{deg}_{e}({#1})} % edge-degree
\newcommand{\degw}[1]	{\mathrm{deg}_{w}({#1})} % weighted-degree
\newcommand{\volv}[1]	{\mathrm{vol}_{v}({#1})} % weighted volume of a graph (vertices)
\newcommand{\vole}[1]	{\mathrm{vol}_{e}({#1})} % weighted volume of a graph (edges)
\newcommand{\volw}[1]	{\mathrm{vol}_{w}({#1})} % weighted volume of a graph (general)
\def \mindeg 			{\mathrm{deg}_{m}} % minimum degree
\def \maxdeg 			{\mathrm{deg}_{M}} % maximum degree
\newcommand{\degm}[1]	{\mindeg({#1})} % minimum degree of a graph
\newcommand{\degM}[1]	{\maxdeg({#1})} % maximum degree of a graph
\newcommand{\conductance}[1]		{h^{Con}_{\graph}({#1})} % cut ratio
\newcommand{\cheegcutratio}[1]	{h^{Ch}_{\graph}({#1})} % Cheeger cut ratio
\def \cutbipart		 	{h_{\graph}} % cut bipartition
\newcommand{\cheeger}{\mathfrakk{h}_{\graph}^{Ch}} % Cheeger isoperimetric constant
\newcommand{\cheegergen}[1]{\mathfrakk{h}_{\graph}^{#1}} % generalized Cheeger isoperimetric constant
\newcommand{\conductconst}{\cheegergen{Con}} % conductance of a graph
\newcommand{\bipartconst}{\cheegergen{}} % bipartition constant of a graph
\def \lap		 		{\mathbf{L}} % Laplacian matrix of a graph
\def \geneig {\lambda} % generic eigenvalue
\newcommand{\eig}[1]{\geneig_{#1}} % indexed eigenvalue
\newcommand{\lapeig}[1] {\eig{#1}^{\lap}} % an eigenvalue of the Laplacian matrix
\newcommand{\lapeigvec}[1] {\eigvec{#1}^{\lap}} % an eigenvector of the Laplacian matrix
\newcommand{\lapeiglvec}[1] {\eiglvec{#1}^{\lap}} % an left eigenvector of the Laplacian matrix
\newcommand{\conseiglvec}[1] {\eiglvec{#1}^{\cons}} % an left eigenvector of the consensus matrix	
\def \adj 				{\mathbf{A}} % adjacency matrix of a graph	
\def \dgr 				{\mathbf{D}} % degree matrix of a graph	
\def \inc 				{\mathbf{E}} % incidence matrix of a graph	
\newcommand{\pathnodes}[1] 	{\pi_{#1}} % a path in a graph
\newcommand{\walknodes}[1] 	{\bar{\pi}_{#1}} % a walk in a graph
\newcommand{\shpathnodes}[1]{\pi^{\star}_{#1}} % a shortest path in a graph
\def \diamgraphsym {\phi} % diameter of a graph
\newcommand{\diamgraph}[1] 	{\diamgraphsym({#1})} % diameter of a graph
\def \radiusgraphsym {\mathfrakk{r}} % radius of a graph
\newcommand{\radiusgraph}[1] 	{\radiusgraphsym({#1})} % radius of a graph
\def \nospantreesym 	{\tau} % number of spanning tree (symbol)
\def \nospantree 		{\tau(\graph)} % number of spanning tree
\def \normlap		 	{\boldsymbol{\mathcal{L}}} % normalized Laplacian matrix of a graph
\newcommand{\normlapeig}[1] {\eig{#1}^{\normlap}} % an eigenvalue of the normalized Laplacian matrix
%\newcommand{\randic}{\pmb{\mathcalligra{R}}\hspace{2.5pt}} % Randic matrix
%\newcommand{\ring}[2]{C_{#1}(1,{#2})} % ring graphs
\newcommand{\complete}[1]{K_{#1}} % complete graphs
\newcommand{\pathgraph}[1]{P_{#1}} % path graph
\newcommand{\graphdens}[1]{\mathrm{dens}({#1})} % density of a graph
\def \densthr {\varepsilon_{D}} % density threshold for graph desity definition
\def \commondeg {\mathrm{d}} % common degree of a regular graph
\def \girth {\mathrm{g}} % girth of a graph
\def \epsmax {2/\specrad{\lap}}
\newcommand{\rate}[1]		{\mathrm{r}_{#1}} % rate of convergence





\title{\LARGE \bf On the impact of regularization in data-driven predictive control}

\author{Valentina~Breschi~\IEEEmembership{Member,~IEEE,} Alessandro~Chiuso~\IEEEmembership{Fellow,~IEEE,}\\ Marco~Fabris and~Simone~Formentin~\IEEEmembership{Member,~IEEE,}
\thanks{This project was partially supported by the Italian Ministry of University and Research under the PRIN'17 project \textquotedblleft Data-driven learning of constrained control systems \textquotedblright, contract no. 2017J89ARP.}
	\thanks{A. Chiuso and M. Fabris are with the Dept. of Information Engineering, University of Padova, Padua, Italy.
	V. Breschi is with Department of Electrical Engineering, Eindhoven University of Technology, Eindhoven, Netherlands.
	S. Formentin is with Dipartimento di Elettronica, Informatica e Bioingegneria (DEIB), Politecnico di Milano, Via G. Ponzio 34/5, 20133 Milano, Italy.
}
	%\thanks{\mbox{Corresp. author: M. Fabris, {\tt \scriptsize \href{mailto:marco.fabris.1@unipd.it}{marco.fabris.1@unipd.it}}}}
}


\begin{document}
	
\maketitle
\thispagestyle{empty}
\pagestyle{empty}


\begin{abstract}
Model predictive control (MPC) is a control strategy widely used in industrial applications. However, its implementation typically requires a mathematical model of the system being controlled, which can be a time-consuming and expensive task. Data-driven predictive control (DDPC) methods offer an alternative approach that does not require an explicit mathematical model, but instead optimize the control policy directly from data. In this paper, we study the impact of two different regularization penalties on the closed-loop performance of a recently introduced data-driven method called $\gamma$-DDPC. Moreover, we discuss the tuning of the related coefficients in different data and noise scenarios, to provide some guidelines for the end user.
	
\end{abstract}

\begin{keywords}
  %predictive control, data-driven control, regularization
  Data driven control, Predictive control for linear systems, Uncertain systems
\end{keywords}
	
	
%%%%%%%%%%%%%%%%%%%%%%%%%%%%%%%%
%%%%%%%%%%%%%%%%%%%%%%%%%%%%%%%%
%%%%%%%%%%%%%%%%%%%%%%%%%%%%%%%%
%%%%%%%%%%%%%%%%%%%%%%%%%%%%%%%%

%%%%%%%%%%%%%%%%%%%%%%%%%%%%%%%%%%%%%%%%%%%%%%%%%%%%%%%%%%
\section{Introduction} 
\label{sec:intro}
Model predictive control (MPC) is a popular control strategy that has been successfully applied in a wide range of applications \cite{borrelli2017predictive}. However, a major limitation of MPC is that it requires a mathematical model of the system being controlled, which can be a costly and time-consuming task. This requirement has led to the development of data-driven predictive control (DDPC) methods, which aim to learn the control policy directly from data without the need for a mathematical model of the plant \cite{berberich2020data,breschi2022design,sassella2022data}.

Nonetheless, the data-based predictor used in DDPC is not exempt from shortcomings, due to the %coming from the 
presence of noise on the measured data. Therefore, different techniques have been proposed to make the closed-loop performance less sensitive to such a noise, e.g., robust design in case hard power bounds are given \cite{berberich2020data}, dynamic mode decomposition \cite{sassella2022noise} and regularization \cite{dorfler2022bridging}. The latter in particular can be used to prevent the data-based predictor to overfit the historical data, by tuning a few penalty coefficients. In the pioneering work \cite{dorfler2022bridging}, the design of such terms is discussed for different kinds of regularization, and the authors highlight %ed 
the significant efforts required in terms of trial-and-error tuning, especially as far as some specific parameters are concerned. In \cite{breschi2022roleArXiv}, we showed that regularization may be avoided in case the data set is large enough and the DDPC problem is reformulated thanks to subspace identification tools, so as to shrink the number of decision variables, into the so-called $\gamma$-DDPC method. Finally, in \cite{breschi2022uncertainty}, we have %discussed the 
focused on %case of 
finite size %of the 
data sets and used %where 
asymptotic arguments to show that regularization might instead be useful to counteract the prediction error variance, due to the use of noisy data in the predictor. Two different regularization options have been introduced, and an on-line tuning of %their coefficients 
the associated penalizations has been proposed, based on the prior knowledge of %about 
the variance expression.

This paper %can be seen as 
is a follow-up of \cite{breschi2022uncertainty}, since our goal here is to analyze the joint tuning of the two regularization terms of $\gamma$-DDPC and analyze their impact on the closed-loop performance. In particular we shall discuss the role of the driving input color (spectra), and some qualitative guidelines about regularization design will be drawn by means of extensive simulations on a benchmark linear system as well as on a challenging nonlinear problem, namely, wheel slip control in braking maneuvering. Finally, offline and on-line regularization tuning will be compared.

The remainder of the paper is as follows. In Section \ref{sec:setting}, the predictive control problem setting is described, and the regularization tuning issue is mathematically formulated. Section \ref{sec:reg} illustrates the considered regularization techniques for $\gamma$-DDPC and discusses the role of each term, also by means of two numerical case studies. The paper is ended by some concluding remarks.

\textit{Notation.} Given a signal (say $u(t) \in \mathbb{R}^m$, the associated (block) Hankel matrix $U_{[t_0,t_1],N} \in \mathbb{R}^{m(t_1-t_0+1) \times N}$ is defined as:
\begin{equation}\label{eq:Hankel}
	U_{[t,s],N}\!:=\!\!\frac{1}{\sqrt{N}}\!\begin{bmatrix}
		u(t) & u(t\!+\!1) & \cdots & u(t\!+\!N\!-\!1)\\
		u(t\!+\!1) & u(t\!+\!2) & \cdots & u(0\!+\!N)\\
		\vdots & \vdots & \ddots & \vdots\\
		u(s) &u(s\!+\!1) & \dots & u(s\!+\!N\!-\!1)  
	\end{bmatrix}\!\!,
\end{equation}
while we use the shorthand $U_{t}:= U_{[t,t],N}$ to denote a single (block) row Hankel, namely:
\begin{equation}\label{eq:Hankel:onerow}
	U_{t}:= \frac{1}{\sqrt{N}}\begin{bmatrix}
		u(t) & u(t\!+\!1) & \cdots & u(t\!+\!N\!-\!1)
	\end{bmatrix}. 
\end{equation}
%%%%%%%%%%%%%%%%%%%%%%%%%%%%%%%%%%%%%%%%%%%%%%%%%%%%%%%%%%

\section{Problem setting}\label{sec:setting}
Our goal is to design a controller for an 
\emph{unknown} plant that can be modeled by \emph{linear time-invariant} (LTI) discrete-time linear (stochastic) system $\mathcal{S}$.  Without loss of generality, we consider its state space description  in 
%
%
%Consider a discrete-time, \emph{linear time-invariant} (LTI) %stochastic plant $\mathcal{S}$ described in minimal (i.e., reachable %and observable) 
\emph{innovation form}
\begin{equation}\label{eq:stoc_sys}
	\begin{cases}
		x(t+1)=Ax(t)+Bu(t)+Ke(t)\\
		y(t)=Cx(t)+Du(t)+e(t), 	\end{cases} \quad  t \in \mathbb{Z},
\end{equation}
where $x(t)\in \mathbb{R}^{n}$, $u(t) \in \mathbb{R}^{m}$ and $e(t) \in \mathbb{R}^{p}$ are the state, input and  innovation process  respectively, while $y(t) \in \mathbb{R}^{p}$ is the corresponding output signal.

Under the \emph{unrealistic} assumption that the system matrices $(A,B,C,D,K)$ are known, the predictive constrained tracking control problem of interest for this paper (for a given reference $y_{r}(t)$ and a prediction horizon $T$) can be formulated as follows 
\begin{subequations}\label{eq:RHPC_prob}
	\begin{align}
		&\underset{\{u(k)\}_{t}^{t+T-1}}{\mbox{minimize}}~\frac{1}{2}\sum_{k=t}^{t+T-1} \ell(u(k),\hat{y}(k),y_{r}(k)) \label{eq:cost}\\
		& \mbox{s.t. } \hat x(k\!+\!1)\!=\!\!A\hat x(k)\!+\!Bu(k),~k \!\in\! [t,t\!+\!T),\\
		& \qquad  \hat y(k)\!=\!C\hat x(k)+Du(k),~k \in [t,t+T),\\ 		
		&\qquad \hat x(t) = \hat x_{init},\\ 
		&\qquad u(k) \in \mathcal{U},~\hat y(k) \in \mathcal{Y},~k \in [t,t+T),
	\end{align}
	where {{$k \in \mathbb{Z}$}}, $\hat x_{init}$ is the state-estimate at time $t$, which can be obtained by running a conventional Kalman filter given the input-output measurements available up to time $t$, and  $y_r$ is the reference signal. Instead, $\ell(\cdot)$ is a convex loss function, penalizing both the tracking performance and the control effort, \emph{e.g.,} 
	\begin{equation}\label{eq:loss}
		\ell(u(k),\hat{y}(k),y_{r}(k))=\|\hat{y}(k)\!-\! y_{r}(k)
		\|_{Q}^{2}\!+\!\|u(k)\|_{R}^{2},
	\end{equation}
\end{subequations}
where the penalties $Q \in \mathbb{R}^{p \times p}$ and $R \in \mathbb{R}^{m \times m}$, with $Q \succeq 0$ and $R \succ 0$, are selected to trade-off between tracking performance and control effort. 

A standard assumption in \emph{data-driven} control is that the system matrices $(A,B,C,D,K)$ are \emph{not known}, and only a  finite sequence of input/output data $\mathcal{D}_{N_{data}}=\{u(j),y(j)\}_{j=1}^{N_{data}}$. We would like to stress that in our  framework measured data are by assumption noisy, in the sense that there is no LTI system that, with the given input $u(t)$, produces exactly the measured output. 

In this paper, we follow that data-driven predictive control problem formulation provided in \cite{breschi2022role,breschi2022uncertainty}, and we refer to those papers for a connection with the recent related literature such as \cite{dorfler2022bridging,berberich2020data}.

To this purpose, we need to introduce the Hankel matrices, including past and future values of inputs and outputs, with respect to time $t$. In particular, with obvious use of the subscripts $P$ and $F$, we define:
\begin{align}\label{eq:future}
	U_F\!:=&U_{[\rho,\rho+T-1],N},~Y_F\!:=\! Y_{[\rho,\rho+T\!-\!1],N},
\end{align}
where $N:=N_{data}-T-\rho$ and $\rho$ is the \textquotedblleft past horizon\textquotedblright. 

Based on \eqref{eq:stoc_sys} the Hankel $Y_F$ can be written as
\begin{subequations}
	\begin{equation}
		{Y}_{F}=\Gamma X_{\rho} +\mathcal{H}_{d}U_{F}+ \mathcal{H}_{s}E_F, 
	\end{equation}
	where $E_F$ is the Hankel of future innovations,
	\begin{equation}\label{eq:observability}
		\Gamma=\begin{bmatrix} C \\ CA \\ CA^{2}\\ \vdots \\ CA^{T-1} \end{bmatrix},
	\end{equation}   
	and $\mathcal{H}_{d} \in \mathbb{R}^{pT \times mT}$ and $\mathcal{H}_s \in \mathbb{R}^{pT \times pT}$ are the Toeplitz matrices formed with the Markov parameters of the system, namely
	\begin{align}
		& 	\mathcal{H}_{d} = \begin{bmatrix} 
			D & 0  & 0 & \dots & 0  \\
			CB & D  &  0 &\dots & 0 \\
			CAB & CB & D  &  \dots & 0 \\
			\vdots & \vdots  & \vdots &  \ddots & \vdots & \\
			CA^{T-2}B & CA^{T-3}B  & CA^{T-4}B & \ldots &D 
		\end{bmatrix},\\
		&\mathcal{H}_s = \begin{bmatrix} 
			I & 0  & 0 & \dots & 0 \\
			CK & I  &  0 &\dots & 0 \\
			CAK & CK & I  &  \dots & 0 \\
			\vdots & \vdots  & \vdots &  \ddots & \vdots  \\
			CA^{T-2}K & CA^{T-3}K  & CA^{T-4}K & \ldots &I 
		\end{bmatrix}. 
	\end{align}
\end{subequations}

Let us now define $z(t)$ as the joint input/output process
\begin{equation}\label{eq:z}
	z(t):=\begin{bmatrix} u(t)\\
		y(t)
	\end{bmatrix},
\end{equation}
with the associated Hankel matrix being $Z_P\!\!:=\!Z_{[0,\rho-1],N}$. The orthogonal projection of $Y_F$ onto the row space of $Z_P$ and $U_F$ turns out to be given by
\begin{align}
	\hat{Y}_{F}&=\Gamma \hat X_{\rho} +\mathcal{H}_{d}U_{F}+ \underbrace{\mathcal{H}_{s}\Pi_{Z_{P},U_{F}}(E_F)}_{O_P(1/\sqrt{N})} \label{eq:Proj:F}
\end{align}
where the last term vanishes\footnote{For a more formal statement on this, we refer the reader to standard literature on subspace identification.} (in probability) as $1/\sqrt{N}$.
This means that, when the matrices $(A,B,C,D,K)$ are \emph{unknown}, future outputs can still be predicted directly from data. In fact, given any (past) joint input and output trajectory and future control inputs 
\begin{equation}\label{eq:zinit}
	\begin{matrix}
		z_{init}:=\begin{bmatrix}
			z(t-\rho)\\
			\vdots\\
			z(t-2)\\
			z(t-1)
		\end{bmatrix}, \quad & u_{f}:=\begin{bmatrix}
			u(t)\\
			u(t+1)\\
			\vdots\\
			u(t+T-1)
		\end{bmatrix},
	\end{matrix}
\end{equation}
the prediction $\hat y_f$ of future outputs $y_f$ 
\begin{equation}
	y_{f}:=\begin{bmatrix}
		y(t)\\
		y(t+1)\\
		\vdots\\
		y(t+T-1)
	\end{bmatrix},
\end{equation}
based on past inputs $z_{init}$ and future inputs $u_f$ 
can be obtained from\footnote{Conditions on $
	\rho$ for this to hold are provided in \cite{breschi2022role}.}
\begin{equation}\label{eq:DDPC_standard}
	\begin{bmatrix}
		z_{init}\\u_{f}\\ \hat y_{f}			
	\end{bmatrix}=\begin{bmatrix}
		Z_{P}\\U_{F}\\ \hat Y_{F}
	\end{bmatrix}\alpha + O_P(1/
	\sqrt{N}), 
\end{equation}
with $\alpha \in \mathbb{R}^{N}$ to be optimized as in, e.g., \cite{breschi2022role}, \cite{berberich2020data}, \cite{coulson2019data}. 

Following subspace identification \cite{Vanov-book} ideas,  the orthogonal projection \eqref{eq:Proj:F} can be written  exploiting the LQ decomposition of the data matrices. In particular, let us define 
\begin{equation}\label{eq:LQ}
	\begin{bmatrix}
		Z_{P}\\U_{F}\\ Y_{F}
	\end{bmatrix}=
	\begin{bmatrix}
		L_{11} & 0 & 0  \\
		L_{21} & L_{22} &  0\\
		L_{31} & L_{32} & L_{33} 
	\end{bmatrix}
	\begin{bmatrix}
		Q_{1}\\
		Q_{2}\\
		Q_{3}
	\end{bmatrix}.
\end{equation}
where the matrices $\{L_{ii}\}_{i=1}^{3}$ are all non-singular and $Q_{i}$ have orthonormal rows, i.e., $Q_{i}Q_{i}^{\top}=I$, for $i=1,\ldots,3$, $Q_i Q_j^\top = 0$, $i\neq j$. The orthogonal projection 
\eqref{eq:Proj:F}  can be written in the form:
$$
\hat Y_F = L_{31} Q_1 + L_{32}Q_2
$$
With this notation, following the same rationale of \cite{breschi2022role,breschi2022uncertainty}, we can further reformulate \eqref{eq:DDPC_standard} as:
\begin{equation}\label{eq:LQ_prediction}
	\begin{bmatrix}z_{init}\\u_{f}\\ \hat y_{f}		
	\end{bmatrix}=\begin{bmatrix}
		Z_{P}\\U_{F}\\ \hat Y_{F}
	\end{bmatrix}\alpha=
	\begin{bmatrix}
		L_{11} & 0   \\
		L_{21} & L_{22} \\
		L_{31} & L_{32} 
	\end{bmatrix}
	\underbrace{\begin{bmatrix}
			Q_{1}\\
			Q_{2}
		\end{bmatrix}\alpha}_{\gamma} + O_P(1/
	\sqrt{N}).
\end{equation}

and the parameters
\begin{equation}\label{eq:gamma}
	\gamma=\begin{bmatrix}
		\gamma_{1}\\
		\gamma_{2}
	\end{bmatrix},
\end{equation}
become the new decision variables. In addition, in \cite{breschi2022uncertainty} it was suggested to add a (slack) optimization variable $\gamma_3$ to model the projection error in \eqref{eq:Proj:F} and avoid overfitting. In particular, the prediction  (with slack) can be written as:
$$
\bar y_f = \underbrace{\begin{bmatrix}
		L_{31} & L_{32} 
	\end{bmatrix} \begin{bmatrix}
		\gamma_{1}\\\gamma_{2}
\end{bmatrix}}_{=\hat y_f} + L_{33}\gamma_3
$$

We refer the reader to \cite{breschi2022uncertainty} for a sound statistical motivation of this particular expression of the slack $L_{33}\gamma_3$. In particular, since $L_{33}$ is generically of full rank, constraints/regularization should be imposed on the slack optimization variable $\gamma_3$.

%and the sets $ \mathcal{U}$, $ \mathcal{Y}$ denote inputs and output constraints. The tunable symmetric weights $Q \in \mathbb{R}^{p \times p}$ and $R \in \mathbb{R}^{m \times m}$, with $Q \succeq 0$ and $R \succ 0$, are selected to trade-off between tracking performance and control effort. %For simplicity, without loss of generality, from now on we will consider a constant reference along the prediction horizon, i.e., \[y_{r}(k)=y_{r}(t),~k \in [t,t+T).\]

A data-driven predictive controller with the same objectives and constraints of \eqref{eq:RHPC_prob} can be formulated as follows \cite{breschi2022role}
\begin{subequations}\label{eq:RHPC_prob_dd_gamma}
	\begin{align}
		&\underset{\gamma_2,\gamma_3}{\mbox{min}}~\frac{1}{2}\sum_{k=t}^{t+T-1} \ell(u(k),\bar y(k),y_{r}(k)) + \Psi(\gamma_1,\gamma_2,\gamma_3) \label{eq:cost_gammaDDPC}\\
		&~~\mbox{s.t.}~~\begin{bmatrix}
			u_{f}\\
			\bar y_{f}
		\end{bmatrix}=\begin{bmatrix}
			L_{21} & L_{22} & 0 \\
			L_{31} & L_{32} & L_{33}
		\end{bmatrix}\begin{bmatrix}
			\gamma_{1}^\star\\\gamma_{2} \\ \gamma_3
		\end{bmatrix} \label{eq:prediction_model3},\\
		&~~~~~~~~~u(k) \in \mathcal{U},~\bar y(k) \in \mathcal{Y},~k \in [t,t+T), \label{eq:constraints2}
	\end{align}
\end{subequations}
with 
\begin{equation}\label{eq:loss}
	\ell(u(k),\bar y(k),y_{r}(k))=\|\bar y(k)\!-\! y_{r}(k)
	\|_{Q}^{2}\!+\!\|u(k)\|_{R}^{2},
\end{equation}
and
\begin{equation}\label{eq:init_terms}
	\gamma_{1}^\star=
	L_{11}^{-1}z_{init},
\end{equation}
where $z_{init}$ is defined as in \eqref{eq:zinit} and the choice of $\gamma_{1}$ straightforwardly follows from the initial conditions (showing the advantages of using $\gamma$ instead of $\alpha$ as the decision vector). 

%Note that this formulation matches most of the regularized DDPC schemes proposed in \cite{dorfler2022bridging}, each based on a specific choice of $\Psi(\gamma_2,\gamma_3)$.  

%The advantage of this formulation with respect to the one commonly used in DDPC lays in  by decoupling $\alpha$ in the three components $\gamma_1, \gamma_2$ and  $\gamma_3$ The advantage of this formulation is mainly that the contribution of $\alpha$ in the optimization problem \eqref{eq:RHPC_prob} is decoupled in the three components $\gamma_1, \gamma_2$ and  $\gamma_3$. This decoupling turns out to be very 

The purpose of this paper is \textit{to study the design and impact  of the regularization term $\Psi(\gamma_1,\gamma_2,\gamma_3)$ within a noisy stochastic environment, and provide the end user with useful hints on how to tune such a penalty term}.

\section{The role of regularization}\label{sec:reg}

In \cite{breschi2022uncertainty}, it has been argued that the average variance of the error on the future output predictions $\hat y_f$ due to the finite data projection errors in \eqref{eq:Proj:F}, is proportional to $\|\gamma_1\|^2  + \|\gamma_2\|^2$. Since, in the optimization problem \eqref{eq:RHPC_prob_dd_gamma}, $\gamma_1$ is determined by the initial conditions, it only remains to regularize $\gamma_2$ so as to avoid an (unnecessarily) high variance on the predictor and, therefore, poor control performance. In this paper, we consider also an alternative regularization term that penalizes directly the control input effort (in addition to the control penalty already embedded in the control cost), and discuss its relation with regularization on $\gamma_2$. Differently from  \cite{breschi2022uncertainty}, we consider this jointly with presence of a slack variable $\gamma_3$ and thus a related regularization. These considerations lead to the following two forms  of the regularization term $\Psi(\gamma_1,\gamma_2,\gamma_3)$ in 
\eqref{eq:RHPC_prob_dd_gamma}:

\begin{enumerate}
	\item[(a)] {\bf Regularization on $\gamma_2$ and slack $\gamma_3$} 
	\begin{equation}\label{eq:reg1}
		\Psi_{\gamma_2}(\gamma_1,\gamma_2,\gamma_3) := \beta_2\|\gamma_2\|^2 + \beta_3 \|\gamma_3\|^2
	\end{equation}
	\item[(b)] {\bf Regularization on input $u_f$ and slack $\gamma_3$} 
	\begin{equation}\label{eq:reg2}
		\begin{array}{rcl}
			\Psi_u(\gamma_1,\gamma_2,\gamma_3) &:=& \beta_2\|u_f\|^2 + \beta_3 \|\gamma_3\|^2 \\
			& = & \beta_2\|L_{21} \gamma_1 + L_{22} \gamma_2\|^2 + \beta_3 \|\gamma_3\|^2 \\
		\end{array}
	\end{equation}
	
\end{enumerate}

\subsection{Theoretical analysis}
We first state a Theorem the establishes the connection between \eqref{eq:reg1} and \eqref{eq:reg2}.

\begin{thm}\label{thm:reg}
	If the training input sequence $u(t)$ in the Hankel matrices $U_F$ and $U_P$ is (zero mean) white with variance $\sigma^2 I$, the regularization terms $\Psi_{\gamma_2}$ in \eqref{eq:reg1} and $\Psi_{u}$ in \eqref{eq:reg2} are asymptotically (in $N$) equivalent up to a rescaling of the weight $\beta_2$.
\end{thm}
\begin{proof}
	Under the assumption that $u(t)$ is white noise, then the future inputs are uncorrelated with past input and output data, so that the projection $\hat U_F:=\Pi_{Z_{P}}(U_F)$ of $U_F$ on the joint past $Z_P$ tends to zero as the number of data $N$ goes to infinity, more precisely
	\begin{equation}\label{eq:L21}
		\hat U_F: = L_{21}Q_1 + O_P(1/\sqrt{N}).
	\end{equation}
	Since $Q_1Q_1^\top = I$, it follows that $L_{21} = O_P(1/\sqrt{N})$. In addition, since $u$ is white, its sample covariance matrix $U_FU_F^\top$ converges to $\sigma^2 I$, i.e. 
	\begin{equation}\label{eq:L22}
		U_FU_F^\top = \underbrace{L_{21}L_{21}^\top}_{O_P(1/{N})} + L_{22}L_{22}^\top \mathop{\longrightarrow}^{N \rightarrow\infty} \sigma^2  I
	\end{equation}
	Equations \eqref{eq:L21} and \eqref{eq:L22} imply that, asymptotically in $N$, $L_{21}\simeq 0$ and $L_{22}\simeq \sigma I$. Therefore we have:
	\begin{equation}\label{eq:reg2eqreg1}
		\begin{array}{rcl}
			\Psi_u(\gamma_1,\gamma_2,\gamma_3) &:=&  \beta_2\|L_{21} \gamma_1 + L_{22} \gamma_2\|^2 + \beta_3 \|\gamma_3\|^2\\
			& \simeq & \beta_2 \sigma^2\|\gamma_2\|^2 + \beta_3 \|\gamma_3\|^2
		\end{array}
	\end{equation}
	showing that, up to the rescaling of the weight $\beta_2$, this is equivalent to $\Psi_{\gamma_2}(\gamma_1,\gamma_2,\gamma_3) $
\end{proof}

This result has two important implications:
\begin{itemize}
	\item  When the (training) input is white, regularization on $\gamma_2$ is equivalent to a penalty on the future input energy, which is typically present in the control cost. As such, we can  argue that, in this case, the control cost has an \textit{indirect but important} effect in counteracting the effect of the noise variance in the predictor.
	\item When the training input is not white, the control energy cost \emph{is not} equivalent to penalizing the norm of $\gamma_2$, which on the other hand should be penalized to limit the effect of noise variance. The simulation results in the next section indeed confirm that, when noise input is not white, regularization on $\gamma_2$ (i.e. $\Psi_{\gamma_2}$) has to be preferred.
\end{itemize}
\subsection{Experimental analysis}\label{sec:examples}
In this section we shall illustrate, exploiting two numerical examples (one linear and one non-linear), the role of different  regularization terms in the optimal control problem \eqref{eq:RHPC_prob_dd_gamma}. In particular,  following the rationale proposed in \cite{dorfler2022bridging}, we evaluate the  closed-loop performance over $T_v$ feedback steps as measured by the performance index:
\begin{equation}\label{eq:general_perf_index}
	J(u,y)\!=\!  \dfrac{1}{T_{v}}\!\sum\limits_{t=0}^{T_{v}-1}\!\!\left( \left\|u(t)\!-\!u_r(t) \right\|^{2}_{R} \!+\! \left\|y(t)\!-\!y_{r}(t) \right\|^{2}_{Q}\right) .
\end{equation}
%%FIG
%trim={<left> <lower> <right> <upper>}
\begin{figure}[h!]
	\centering
	\subfigure[Performance indexes]{\includegraphics[height=0.26\textwidth, trim={1.5cm 0cm 3cm 1cm},clip]{./images_CDC23/Markovski_costs}}
	\subfigure[Corresponding minimizers]{\includegraphics[height=0.29\textwidth,trim={0cm 0cm 2cm 1cm},clip]{./images_CDC23/Markovski_scatter}\label{fig:Markovski_scatter}}
	\subfigure[Optimal performance under constraint $\beta_2=0$]
	{\includegraphics[height=0.26\textwidth,trim={0cm 0cm 2cm 0cm},clip]{./images_CDC23/performances_b2=0}\label{fig:performances_b2=0}}
	\caption{(a): comparison between the Kalman-filter-based oracle performance $J_{OR}$ and the minimum cost realizations $\widehat{J}^{N_{data}}_{n_s,rg}$ for $\Sigma_{L}$ over $100$ Monte Carlo runs; (b): distribution of the corresponding minimizers $(\beta_2^\star,\beta_3^\star)$; (c) Optimal performance under the constraint $\beta_2=0$.}
	\label{fig:Markovski_costs}
	%\vspace{6mm}
\end{figure}
%%%%








\subsubsection{Benchmark LTI system}
Consider the SISO, %The aim of this experiment is to discuss the performances of the benchmark single-input, single-output, 
$5$-th order %, linear time-invariant 
system in \cite{LandauReyKarimi1995} ($\Sigma_{L}$ in the sequel) with a prediction horizon $T=20$. %, as one adopts different types of data sets and regularizations. 
%In particular
To assess the impact of the \emph{training} data  on closed-loop performance, we consider four data sets of two different lengths $N_{data}$ (either $250$ or $1000$), obtained either with white noise input (denoted with $n_s=w$) or with a low-pass   (obtained filtering white noise with  a discrete-time low-pass filter with cut-off angular frequency $1.8~rad/s$) input sequence (denoted with $n_s=c$).
White noise is added to the output to guarantee  a signal-to-noise ratio of $15$~dB.


The data-driven optimal control problem \eqref{eq:RHPC_prob_dd_gamma} is solved  
fixing $Q = 10^{3}$ and $R = 10^{-2}$,
setting the output reference $y_{r}(t) = \sin(5\pi t/(T+T_{v}-1))$ and the input reference $u_r(t) = 0$, with $T_{v} = 50$. The two different regularization strategies discussed in Section \ref{sec:reg} are denoted with the shorthands $rg=2$ when using \eqref{eq:reg1} and  $rg=u$ when using \eqref{eq:reg2}.

The ``oracle" value of $(\beta_2,\beta_3)$ leading to \textbf{the minimum cost \eqref{eq:general_perf_index}}, here denoted as $J_{n_s,rg}^{N_{data}}= J(u^{N_{data}}_{n_s,rg},y^{N_{data}}_{n_s,rg})$ to account for the different data set lengths, input and regularization strategies, are searched over a rectangular logarithmic-spaced grid $G_{23}$ with $7$ points per decade, % for the values of $(\beta_2,\beta_3)$ is selected, 
so that $G_{23} \subseteq \left\lbrace 0 \right\rbrace \cup [10^{-4}, 10^{0}] \cup \left\lbrace +\infty \right\rbrace \times \left\lbrace 0 \right\rbrace \cup [10^{-3},10^{0}] \cup \left\lbrace +\infty \right\rbrace$.

We perform $100$ Monte Carlo experiments (i.e. $100$ different training data sets with the output of  $\Sigma_L$ corrupted by white noise) to tune $\beta_2$ and $\beta_3$
for all the four considered training scenarios and possible regularization. For each Monte Carlo run, and for each set of possible parameters in the grid $G_{23}$,  the closed loop performance index  \eqref{eq:general_perf_index} is computed by averaging over $100$ closed loop experiments  (all with the same control law but different closed measurements errors) the corresponding performance index $J^{N_{data}}_{n_s,rg}(i)$, i.e.  
\begin{equation}\label{eq:avg_index}
	{\widebar{J}^{N_{data}}_{n_s,rg}} = \frac{1}{100}\sum\nolimits_{i=1}^{100}	J^{N_{data}}_{n_s,rg}(i),
\end{equation}
The  optimal values of $\beta_2$ and $\beta_3$  over the grid $G_{23}$ is obtained by finding the minimum $\widehat{J}^{N_{data}}_{n_s,rg} {=\min_{(\beta_2,\beta_3) \in G_{23}} \{\widebar{J}^{N_{data}}_{n_s,rg}\}}$. 



The results are reported  in Fig. \ref{fig:Markovski_costs}, based on which  we can make the following general considerations:
\begin{itemize}
	\item As expected based on Theorem \ref{thm:reg}, when the input is white noise, the two types of regularization provide the same performance (minor differences for $N_{data}=250$ are due to  sample variability).
	\item The ``optimal'' (oracle) closed loop performance obtained with the two different regularization strategies differ when the input is not white. In particular, the penalty \eqref{eq:reg1} that acts directly on $\gamma_2$ and thus controls the predictor variance provides the best performance, particularly so for small data sets where the effect of noise has more impact.  
	\item The location of the minimum points (see Fig. \ref{fig:Markovski_costs}) is different depending on the employed training signal, especially in the low data regime; indeed, preference is given to exploiting the regularization term on $\gamma_2$ (and in fact $\gamma_3 = \infty$ in most cases).
	\item Based on the comparison between Fig.\ref{fig:Markovski_costs}(a) and Fig. \ref{fig:Markovski_costs}(c) (in which $\beta_2$ is constrained to zero), we can observe that the impact of $\beta_2$ is significant in the low-data regime (equivalent to large noise in the predictor), whereas for larger data sets its impact can be neglected and the optimal performances exploiting only $\beta_3$ match those obtained optimizing jointly $\beta_2$ and $\beta_3$.
\end{itemize}



In light of the above observations, the general validity of \eqref{eq:RHPC_prob_dd_gamma} constrained to either $\beta_2=0$ or $\beta_3=\infty$ devised in \cite{breschi2022uncertainty} is strengthened, as different types of data set are given. The effectiveness of these $\gamma$-DDPC schemes is also reinforced since it is evident that, in most cases, the operative tuning of either the sole parameter $\beta_2$ or the sole parameter $\beta_3$ is worth to be carried out in practice.

%%FIG BRAKING SYSTEM PERFORMANCES, ORACLE, data set
%trim={<left> <lower> <right> <upper>}
\begin{figure}[t!]
	\centering
  \subfigure[Training data set of size $10^4$]{\includegraphics[height=0.25\textwidth, trim={0.3cm 0cm 1cm 0cm},clip]{./images_CDC23/training}\label{fig:training}}
  \\
	\subfigure[Performance indexes]{\includegraphics[width=0.3\textwidth, trim={1.8cm 0cm 3cm 0.3cm},clip]{./images_CDC23/BS_perf_idx_exactz}\label{fig:BS_perf_idx_exact}}
	%\hspace{7mm}
 \\
	\subfigure[MPC-based oracle]{\includegraphics[height=0.25\textwidth, trim={0.3cm 0cm 1cm 0cm},clip]{./images_CDC23/BS_oracle_inout_exact}\label{fig:BS_oracle_inout_exact}}
 

	\caption{(a): training data set employed for all the $\gamma$-DDPC experiments on the wheel slip control problem. (b): comparison of the performance indexes obtained with different $\gamma$-DDPC strategies (bar and hat notation indicating offline and online approaches respectively) and a model-based oracle; (c): input/output tracking obtained from an MPC-based oracle. Mean (line) and $1.95$ times the standard deviation (shaded area) of the closed-loop input/output trajectories; the reference input and output are indicated with black dashed lines. 
 }
	\label{fig:BS_sims_tuned}
	%\vspace{6mm}
\end{figure}
%%%%



%%FIG gamma-DDPC RESULTS BRAKING SYSTEM
%trim={<left> <lower> <right> <upper>}
\begin{figure*}[t!]
	\centering
	\subfigure[$\bar J_0$: $(\beta_2,\beta_3) = (0,+\infty)$]{\includegraphics[height=0.25 \textwidth, trim={0.3cm 0cm 1cm 0.2cm},clip]{./images_CDC23/BS_noreg_inout_exactz}\label{fig:BS_noreg_inout_exact}}
	\hspace{5mm}
	\subfigure[$\bar J_2$: $(\beta_2,\beta_3)=(\bar{\beta}_2,+\infty)$]{\includegraphics[height=0.25\textwidth, trim={0.3cm 0cm 1cm 0cm},clip]{./images_CDC23/BS_b2bar_inout_exactz}\label{fig:BS_b2bar_inout_exact}}
	\hspace{5mm}
	\subfigure[$\bar J_3$: $(\beta_2,\beta_3) = (0,\bar{\beta}_3$)]{\includegraphics[height=0.25\textwidth, trim={0.3cm 0cm 1cm 0cm},clip]{./images_CDC23/BS_b3bar_inout_exactz}\label{fig:BS_b3bar_inout_exact}}
	\\
	\subfigure[$\bar J_{23}$: $(\beta_2,\beta_3) = (\beta^{\star}_2,\beta^{\star}_3)$]{\includegraphics[height=0.25\textwidth, trim={0.3cm 0cm 1cm 0.2cm},clip]{./images_CDC23/BS_b23bar_inout_exactz}\label{fig:BS_b23bar_inout_exact}}
	\hspace{5mm}
	\subfigure[$\hat J_2$: $(\beta_2,\beta_3) = (\hat \beta_{2},+\infty)$]{\includegraphics[height=0.25\textwidth, trim={0.3cm 0cm 1cm 0cm},clip]{./images_CDC23/BS_b2tuned_inout_exactz}\label{fig:BS_b2tuned_inout_exact}}
	\hspace{6mm}
	\subfigure[$\hat J_3$: $(\beta_2,\beta_3) = (0,\hat \beta_{3})$]{\includegraphics[height=0.25\textwidth, trim={0.3cm 0cm 1cm 0cm},clip]{./images_CDC23/BS_b3tuned_inout_exactz}\label{fig:BS_b3tuned_inout_exact}}
	\caption{For all diagrams: mean (line) and $1.95$ times the standard deviation (shaded area) of the closed-loop input/output trajectories; the reference input and output are indicated with black dashed lines.
        (a): $\gamma$-DDPC with no regularization;
	(b)-(c): offline regularization strategies employing $\bar{\beta}_2$ and $\bar{\beta}_3$ separately;
 (d): offline regularization strategies employing $\bar{\beta}_2$ and $\bar{\beta}_3$ jointly;
 (e)-(f): online regularization strategies employing $\hat{\beta}_2$ and $\hat{\beta}_3$ separately.}
 \label{fig:BS_sims_new}
	%\vspace{6mm}
\end{figure*}
%%%%



\subsubsection{Wheel slip control problem}
We now consider the problem of designing a wheel slip controller, steering the vehicle slip $\lambda(t) \in [0,1]$ to a constant target value $\lambda_{r}$. The design is carried out by focusing on quasi-stationary operating condition (the parameters of the vehicle, its velocity and the road profile are assumed to be constant). In both data collection and closed-loop testing, the behavior of the braking system (from now on indicated as $\Sigma_{NL}$) is simulated based on the following \emph{nonlinear} differential equation \cite{Formentin2015}:
\begin{subequations}\label{eq:slip_dynamics}
	\begin{equation}
		\dot{\lambda}(t)=-\frac{1}{v}\left(\frac{1-\lambda(t)}{m}+\frac{r^2}{J}\right)mg\mu(\lambda(t))+\frac{r}{Jv}T_{b}(t),
	\end{equation}
	where the road friction coefficient $\mu(\lambda(t))$ is assumed to be dictated by the Burckhardt model, \emph{i.e.,} 
	\begin{equation}\label{eq:Burckhardt_model}
		\mu(\lambda(t))=\alpha_{1}\left(1-e^{-\alpha_2\lambda(t)}\right)-\alpha_3 \lambda(t),
	\end{equation}
\end{subequations}
$T_b(t)$ [Nm] is the \emph{controllable} braking torque and the remaining parameters are set to the same values used in \cite{sassella2022}. Although this dynamics is clearly non-linear, it is possible to identify two main operating regions of the system\footnote{In this case, these two regions are limited by the slip $\bar{\lambda}=0.17$, for slip values lower than $0.17$; while it becomes unstable for higher slips.}, where the behavior of the slip can be approximated as linear. To comply with our framework (see \eqref{eq:stoc_sys}), we thus consider both data collection and simulation tests where the vehicle generally operates in a low-slip regime. Accordingly, data are gathered by performing closed-loop experiments with the benchmark controller introduced in \cite{savaresi2010active}, selecting a slip reference uniformly chosen at random in the interval $[0,0.15]$ and collecting data at a sampling rate of $100$~[Hz].
In particular, the output $y_{trn}(t)$ of the employed training data set shown in Fig. \ref{fig:training} is generated by exploiting a closed-loop experiment wherein the output is corrupted by a zero-mean white noise process with variance $\sigma^2_n = 10^{-6}$ and, also, zero-mean white noise with variance $10^8 \sigma^2_n$ is added to the input $u_{trn}(t)$ provided by the controller.



Meanwhile, the reference slip for the closed-loop tests is $\lambda_{r}=0.1$, corresponding to a reference braking torque $T_{b,r}=768.9$ [Nm]. To improve the tracking performance in closed-loop, apart from the terms weighting the tracking error and the difference between the predicted and reference torque, respectively weighted by $Q = 10^{3}$ and $R = 10^{-7}$, the cost of the $\gamma$-DDPC problem \eqref{eq:cost_gammaDDPC} is augmented with a term penalizing abrupt variations of the input (weighted as $10^{-4}$), a term penalizing the integral of the tracking error (weighted as $10^5$), and two terms further penalizing the difference between the slip and torque references and their actual value over the last step of the prediction horizon (weighted as $10^3$ and $5 \cdot 10^{-6}$, respectively). The following constraint is also added at each feedback step $t\geq 0$ for $s = 0,\ldots,T-2$:
\begin{equation}
	\begin{cases}
		q(t+s) = q(t+s-1) + y_r(t+s-1)-\hat{y} (t+s-1);\\
		q(t-1) = y_r(t-1)-y(t-1);
	\end{cases}
\end{equation}
to account for the known dynamics of the integrator. Nonetheless, performances are still assessed via the index in \eqref{eq:general_perf_index} over a closed-loop test of $T_{v}=50$ steps. %In the next lines, we propose a numerical test for the regularized $\gamma$-DDPC strategies proposed in \cite{breschi2022uncertainty}, which are applied to the system described by \eqref{eq:slip_dynamics} and hereafter denoted with $\Sigma_{NL}$. The quantities $(T_{b},\lambda)$ are considered the input/output pair $(u,y)$ for the system $\Sigma_{NL}$; whereas, the pair $(u_{r}(t),y_{r}(t)) = (T_{b,r},\lambda_{r})$ is set as input/output reference $\forall t \geq 0$. In order to obtain proper tracking, a differential input term, an integral output error term and input/output terminal terms were added to the cost functions in \MFa{funzionali di costo dei vari DDPC} while running each $\gamma$-DDPC scheme. Furthermore, the closed-loop performance over $T_{v}=80$ steps is validated by means of the index \eqref{eq:general_perf_index}, where $R = 10^{-5}$ and $Q = 10^{4}$ are set.
A Monte Carlo campaign with $100$ iterations is run on the above setup, corrupting the %while the 
output of $\Sigma_{NL}$ %is corrupted at each time step by 
with a white noise having signal-to-noise ratio $40$ dB. For each of the $100$ tests, the regularization parameters $\beta_2$ and $\beta_3$ are both selected from a grid over $[10^{-4},10^4]$ comprising of $15$ logarithmic-spaced points. For the joint optimization, the squared grid $\{\bar{\beta}_2,\bar{\beta}_2/10,\bar{\beta}_2 /100, 10^8\} \times \{ 0,\bar{\beta}_3,10\bar{\beta}_3,100\bar{\beta}_3\}$ composed by the optimal values $(\bar{\beta}_2,\bar{\beta}_3)$ obtained via offline $\gamma$-DDPC is instead taken into account.
Fig. \ref{fig:BS_perf_idx_exact} depicts the distributions of the %realizations of the 
performance index in \eqref{eq:general_perf_index} as the selected regularization strategy varies considering $(\beta_2,\beta_3)$ tuned either offline or online and comparing $\gamma$-DDPC with a MPC-based oracle (see also Fig. \ref{fig:BS_oracle_inout_exact}). In particular, the input-output trajectories of all $\gamma$-DDPC strategies can be summarized in Fig. \ref{fig:BS_sims_new}. Although the MPC-based oracle displays evident preeminence, it is worthwhile to appreciate that all these trajectories are characterized by solid performances (rise time of at most $5$ steps, settling time of about $25$ steps, maximum overshoot of $40\%$ or less with no cross into the unstable region), with such traits indicating that $\gamma$-DDPC schemes remain competitive even in nonlinear scenarios. Noticeably, the online strategies (implementable on real applications) shown in Fig. \ref{fig:BS_b2tuned_inout_exact}-\ref{fig:BS_b3tuned_inout_exact} share similar performances with the corresponding offline strategies in Fig. \ref{fig:BS_b2bar_inout_exact} - \ref{fig:BS_b3bar_inout_exact}, especially that relying on the online tuning of parameter $\beta_3$. Moreover, within the setup of this numerical example, one observes that the performance of the offline strategy based on $\beta_2$ strictly matches with that of the scheme lacking of regularization (Fig. \ref{fig:BS_noreg_inout_exact}); whereas, the performance of the offline strategy based on $\beta_3$ strictly matches that of the scheme in which a joint optimization of both $\beta_2$ and $\beta_3$ (Fig. \ref{fig:BS_b23bar_inout_exact}) is carried out. Hence, under this setup and with the data collected in this numerical example, it emerges once again that the optimal tuning based on the sole penalty parameter $\beta_3$ (i.e., setting $\beta_2 = 0$) can be considered \textit{in practice} for high-data regimes. This, in turn, may lead to significantly diminish the computational burden associated to the tuning of the penalty parameters whenever a real implementation based on the proposed regularized scheme \eqref{eq:RHPC_prob_dd_gamma} is considered and a big training data set is available.



The above comparison further highlights that regularized DDPC approaches can be competitive w.r.t. traditional model-based controllers and that $\gamma$-DDPC solution with the online tuning proposed in \cite{breschi2022uncertainty} can be robustly effective also when dealing with nonlinear systems.

%To conclude, in these last numerical results we have remarked that regularized DDPC approaches attain equivalent performances to those of the related well-known model-based versions. Moreover, the online $\gamma$-DDPC solution proposed in \cite{breschi2022uncertainty} proves to be robustly effective even for scenarios where system nonlinearities represent a central aspect to be taken into account.


\section{Concluding remarks and future directions}\label{sec:conclusions}
In this paper, we discuss different regularization strategies for the recently introduced $\gamma$-DDPC approach and assess their tuning and impact on the closed-loop performance. In particular, we theoretically prove that when the input is white, regularizing $\gamma_2$ is asymptotically equivalent to an additional weighting of the control effort. This fact has an evident implication for the design of the control cost: when the input is colored, a regularization of $\gamma_2$ can limit the impact of the noise variance. Numerical examples further illustrate that the tuning of the penalty parameters in the $\gamma$-DDPC can be decoupled without dramatically impacting the performance corresponding to a (more costly) joint regularization wherein both $\beta_2$ and $\beta_3$ are accounted for. Finally, simulations have shown how a suitable choice of $\beta_3$ alone could be sufficient to achieve near-optimal regularized performance when the size of the dataset is large.

Future work will be devoted to the extensive experimental assessment of the considered regularization strategies, as well as to a deeper theoretical analysis of the optimization of the sole $\beta_3$. 

	
\bibliographystyle{IEEEtran}
\bibliography{biblio}
	

	
	
	
\end{document}


