
\section{Experimental Evaluation}
\label{sec:evaluation}
%------------------------------------------------------------------------
\paragraph*{Datasets.}
We evaluate our method using four different datasets: Watertight dataset~\cite{giorgi2007shrec}, SHREC10 correspondence benchmark~\cite{bronstein2010shrecCorr}, SHREC11 robustness benchmark~\cite{bronstein2010shrec}, and a custom database TN-SCAPE which we constructed  by adding topological noise to five meshes from SCAPE dataset~\cite{anguelov2005scape}.

Original SCAPE dataset contains mesh models for a human body in different poses where there is a one-to-one correspondence between the models. TN-SCAPE is composed of five mesh models  from SCAPE and their five noisy versions which we constructed. 
The second data set, SHREC10 correspondence benchmark, includes three objects. Each object comes with a base shape model (null shape)  and five additional forms obtained from an isometric deformation of the null shape by adding topological noise of increasing strength. In our experiments, we match each null shape to the five models with topological noise.
The third data set, the retrieval training set of SHREC11 robustness benchmark,  contains twelve different shape models. For each model, there are one null shape, one isometric shape and five shapes with increasing degree of topological noise. We use a subset of these shapes for which the ground truth correspondence is available. 
%
For  the three out of four datasets (SHREC10, SHREC11, and TN-SCAPE), ground truth correspondence information is available.

\paragraph*{Evaluation Metric.}
For  three datasets for which the ground truth correspondence information is available,  we measure the performance of our correspondence method in terms of the deviation from the ground truth correspondence. Let $\S$ be the set of correspondence pairs between source and target shape. The average ground truth correspondence error is defined as
\begin{equation}
D_{\mathrm{grd}} = \frac{1}{|\S|} \sum_{(s_i, \, t_j) \in \S} {g(t_i, \, t_j)}
\label{eqn:Dgrd}
\end{equation}
where $(s_i, \, t_i)$ represents the ground truth correspondence and $g(t_i, \, t_j)$ is the geodesic distance between the vertices $t_i$ and $t_j$ on the target shape.

For the first data set, Watertight, ground truth correspondence information is not available; therefore, in evaluating the matching results for the shape pairs from this dataset, we present only visual matching results for the topologically different human and teapot shape pairs.

Since our focus is on the sparse correspondence between sample vertices of the input shapes, for better interpretation of errors, we consider the normalized average ground truth error ${\widetilde{D}_{\mathrm{grd}}}$ formulated as the average ground truth error $D_{\mathrm{grd}}$ divided by the sampling radius $r$. The sampling algorithm cannot guarantee the same sampling on the two shapes so ${\widetilde{D}_{\mathrm{grd}}} \le 1$ holds for the optimal mapping.

%-------------------------------------------------------------------------

%-------------------------------------------------------------------------
\section{Results and Discussion}
\label{sec:results}
%Our correspondence method handles topology noise by incorporating topologically similar interpretations of two shapes. 
\noindent We first demonstrate our approach using visual examples. Then, we present three groups of numerical evaluation tests using the three datasets for which the ground truth correspondences are available.  In all of our experiments, the dimension of the spectral domain is $6$. {In the experiments with SHREC11 robustness benchmark and TN-SCAPE dataset, we used $80$ sample vertices where the number of mesh vertices is $1.5$K and $12.5$K, respectively. In the experiments with SHREC10 correspondence benchmark, we used $280$ sample vertices where the number of mesh vertices ranges from $20$K to $50$K.}

\begin{figure*}[htb]
  \centering
  \begin{tabular}{ccc}
  \includegraphics[width=.23\linewidth]{watertight_16_3_orig.png} &
  \includegraphics[width=.23\linewidth]{watertight_16_3_biharm.png} &
  \includegraphics[width=.46\linewidth]{watertight_16_3_prop_all.png}\\
  (a) & (b) & (c)
  \end{tabular}
  \caption{\label{fig:watertight_16_3}
  One-to-one mappings for a pair of human shapes from Watertight with different topology where both hands of the sitting woman are attached to the legs. (a) Correspondence between topologically different shapes using geodesic distance leads to errors where the right arm of the first shape is unmatched and the left arm is matched to the head of the sitting woman (mappings in red color). (b) Correspondence using biharmonic distance results in mapping the head of each shape to the arms of the other shape (mappings in red color). (c) Correspondence between inner iso-surfaces at level $0.4$ leads to a better result (left). Note that deflated forms of the shapes have the same topology as the arms of the sitting woman are separated from the legs. The mapping between the iso-surfaces is transferred to the input shapes (right).}
\end{figure*}
\begin{figure*}[htb]
  \centering
  \begin{tabular}{ccc}
  \includegraphics[width=.22\linewidth]{watertight_16_3_orig_embed3.png} &
  \includegraphics[width=.22\linewidth]{watertight_16_3_biharm_embed3.png} &
  \includegraphics[width=.22\linewidth]{watertight_16_3_prop_embed3.png} \\
  (a) & (b) & (c)
  \end{tabular}
  \caption{\label{fig:watertight_16_3_embed}
  Spectral embeddings and alignments that lead to the one-to-one mappings shown in Figure~\ref{fig:watertight_16_3}. Only the first $3$ dimensions are plotted. Red and blue points correspond to the standing and sitting woman, respectively. Sample vertices on the input shapes are embedded (a) using their pairwise geodesic distance (b) using their pairwise biharmonic distance. (c) Sample vertices on the inner iso-surfaces are embedded using their pairwise geodesic distance.}
\end{figure*}

Our approach employs the matching algorithm \cite{ys2012EM} on deflated or inflated forms of the pair of input models and then transfers the resulting mapping to the input models. Therefore, in order to evaluate  our contribution, we compare our results with two other mappings, one using geodesic distance and the other using  biharmonic distance~\cite{Lipman10}, both  obtained by applying the same correspondence algorithm (that we use) directly on the input models.

The biharmonic distance is insensitive to small topology changes~\cite{Lipman10}; thus, it is expected to work well. Note that replacing the topology-sensitive geodesic distance with a topologically robust one is a common approach for handling the topological noise~\cite{Bronstein10, Sharma10, Sharma11}. We also give an illustrative result for comparing our method with the one~\cite{Mateus08} that performs the best in topology noise category of SHREC10 correspondence benchmark.

In Figure~\ref{fig:watertight_16_3}, the resulting one-to-one mappings for a pair of human shapes from Watertight dataset are presented. The input models have different topologies as the arms of the sitting woman are merged with the legs. As shown in Figure~\ref{fig:watertight_16_3_embed}~(a)~and~(b), for both geodesic and biharmonic mapping, the connection between the arms and the legs of the sitting woman is reflected in the spectral domain (see the blue points representing the embedded sample vertices of the sitting woman). This connection affects both normalization and alignment of the embedded vertices. The initial alignment in Figure~\ref{fig:watertight_16_3_embed}~(a) leads to the local optimum in which the legs are correctly mapped but one arm of the first shape is unmatched and the other arm is matched to the head of the second shape, which in turn degrades the quality of the geodesic-based correspondence algorithm (see Figure~\ref{fig:watertight_16_3}~(a)). Similarly, the erroneous initial alignment in Figure~\ref{fig:watertight_16_3_embed}~(b) based on biharmonic distances yields the unsatisfactory mapping in Figure~\ref{fig:watertight_16_3}~(b) where the head of each shape is matched to the arms of the other shape. Note that the topological difference implies distortion of the shape isometries which leads to bad initialization and erroneous convergence of the correspondence algorithm. Our approach considers topologically similar representations of the input shapes and finds a mapping between the deflated forms of the input models computed as the inner iso-surfaces at level $0.4$. The topological similarity between the iso-surfaces enables a better initialization as in Figure~\ref{fig:watertight_16_3_embed}~(c) and yields the correct mapping in Figure~\ref{fig:watertight_16_3}~(c).


\begin{figure}[t!]
  \centering
  \begin{tabular}{cc}
  \includegraphics[width=.3\linewidth]{watertight_367_379_orig.png} &
  \includegraphics[width=.6\linewidth]{watertight_367_379_prop_all.png} \\
  (a) & (b)\\
  \end{tabular}
  \caption{\label{fig:watertight_367_379}
  One-to-one mappings for a pair of teapot shapes from Watertight with different topology. The upper joint of the handle is cut in the farther shape. (a) Correspondence between topologically different shapes using geodesic distance leads to errors where the spouts are matched with the handles. (b) Correspondence between outer iso-surfaces at level $0.43$ leads to a better result as the handles and spouts are correctly matched (left). Note that inflated forms of the shapes have the same topology as open end of the farther shape handle joins the main body. The mapping between the iso-surfaces is transferred to the input shapes (right).
  }
\end{figure}

%%% Asli, bu iki figuru minipage icine koydum (sen bir bak uygun mu), siralari bozulmasin diye. \FloatBarrier calismiyor nedense ...
%\begin{figure*}[htb]
%\centering
%\begin{minipage}[b]{0.45\linewidth}
%  \centering
%  \begin{tabular}{cc}
%  \includegraphics[width=.32\linewidth]{figures/fig_watertight_367_379/watertight_367_379_orig.png} &
%  \includegraphics[width=.64\linewidth]{figures/fig_watertight_367_379/watertight_367_379_prop_all.png} \\
%  (a) & (b)\\
%  \end{tabular}
%  \caption{\label{fig:watertight_367_379}
%  One-to-one mappings for a pair of teapot shapes from Watertight with different topology. The upper joint of the handle is cut in the farther shape. (a) Correspondence between topologically different shapes using geodesic distance leads to errors where the spouts are matched with the handles. (b) Correspondence between outer iso-surfaces at level $0.43$ leads to a better result as the handles and spouts are correctly matched (left). Note that inflated forms of the shapes have the same topology as open end of the farther shape handle joins the main body. The mapping between the iso-surfaces is transferred to the input shapes (right).
%  }
%  \end{minipage}
%  \hskip 1cm
%  \begin{minipage}[b]{0.4\linewidth}
%    \includegraphics[width=1\linewidth]{figures/fig_shrec10_ex/shrec10_ex_topo5.png}
%  \caption{\label{fig:shrec10ex}
%  One-to-one mapping obtained by our method for two shapes from SHREC10 correspondence benchmark where the sitting man has topological noise of degree five.
%  }
%    \end{minipage}
%\end{figure*}

Our approach can also handle shapes with holes or breaks by considering their inflated forms. In Figure~\ref{fig:watertight_367_379}, we present the geodesic mapping and our proposed mapping for a pair of teapot shapes from Watertight dataset. The teapot models are topologically different as the upper joint of the handle is cut in the farther shape. The geodesic mapping shown in Figure~\ref{fig:watertight_367_379}~(a) erroneously matches the spouts with the handles. Our approach considers the inflated forms of the input shapes computed as the outer iso-surfaces at level $0.43$. Note that the inflated models have the same topology as the open end of the broken handle joins the main body and we obtain the correct mapping given in Figure~\ref{fig:watertight_367_379}~(b).

In the first group of evaluation tests, we present the performance of our method in comparison with the geodesic-based and biharmonic-based mappings using SHREC10 correspondence benchmark. In the experiments, the null shape is mapped to each of the five shapes in the topological noise category. We examine how each method performs while the noise strength increases. In Table~\ref{table:shrec10res}, we present average of the normalized ground truth error ${\widetilde{D}_{\mathrm{grd}}}$ over the obtained results where the highest topology noise strength is different at each row. The geodesic-based mapping performs the best for the smallest noise degree but it immediately gets worse when the noise strength becomes greater than one. The biharmonic-based mapping diverges from being optimal when the noise strength is greater than three so it is more robust compared to the geodesic-based one. Our method is robust to the topological noise as all of the mappings are very close to the optimal and it performs the best for all of the experiments except the one with the smallest noise degree. We also expect that our method performs better than the related work~\cite{Bronstein10} that handles topology noise by employing the diffusion-based distance in the correspondence algorithm of~\cite{bronstein2006generalized}. This expectation is because of the fact that our method performs better than the correspondence algorithm~\cite{ys2012EM} running with biharmonic distances (see Table~\ref{table:shrec10res} biharmonic vs. proposed), and yet \cite{Bronstein10} employs a reportedly worse correspondence algorithm \cite{bronstein2006generalized} running with the same biharmonic distances. In Figure~\ref{fig:shrec10ex}, we show our mapping result for two human shapes from SHREC10 correspondence benchmark where the sitting man has topological noise of degree five. {The symmetric flip in Figure~\ref{fig:shrec10ex}, as well as in Figures \ref{fig:shrec11ex}, \ref{fig:scape2ex} and \ref{fig:mateus}, does not interfere with the topological noise robustness feature of our algorithm as one can always alleviate the symmetric flips by employing a denser sampling~\cite{Sahillioglu12b}.}
\begin{figure}[t]
  \centering
  \includegraphics[width=.75\linewidth]{shrec10_ex_topo5.png}
  \caption{\label{fig:shrec10ex}
  One-to-one mapping obtained by our method for two shapes from SHREC10 correspondence benchmark where the sitting man has topological noise of degree five.
  }
\end{figure}
\begin{table}[t]
\centering
\begin{tabular}{|c|c|c|c|}
 \hline
 Noise strength & geodesic & biharmonic & proposed \\
 \hline
 $= 1$ & $\mathbf{0.86}$ & $1.67$ & $1.12$ \\
 \hline
 $\le 2$ & $2.15$ & $1.69$ & $\mathbf{1.13}$\\
 \hline
 $\le 3$ & $2.59$ & $1.70$ & $\mathbf{1.12}$\\
 \hline
 $\le 4$ & $3.58$ & $2.22$ & $\mathbf{1.12}$\\
 \hline
 $\le 5$ & $3.59$ & $2.53$ & $\mathbf{1.11}$\\
 \hline
\end{tabular}
\caption{\label{table:shrec10res}
Performance of our method in comparison with the geodesic-based and biharmonic-based mappings using the topology noise category of SHREC10 correspondence benchmark. The results represent average of ${\widetilde{D}_{\mathrm{grd}}}$ over the mappings. The highest topology noise strength is different at each row.
}
\end{table}

\begin{figure*}[htb]
  \centering
  \includegraphics[width=.75\linewidth]{use_shrec11_cmds_one2one_v3_039.png}
  \caption{\label{fig:shrec11plot}
  Normalized average ground truth error ${\widetilde{D}_{\mathrm{grd}}}$ for one-to-one mappings between topologically different pairs of shapes from SHREC11 robustness benchmark. The errors in green and blue color are obtained using geodesic-based and biharmonic-based mapping, respectively. The errors in red color shows the performance of our approach. Some correspondence errors are high due to the symmetric flip (SF) problem indicated by triangles. Observe that our approach successfully handles topology noise by using topologically comparable forms of the input shapes.
  }
\end{figure*}
\begin{table}[t]
\centering
\begin{tabular}{|c|c|c|c|}
\hline
 & geodesic & biharmonic & proposed \\
 \hline
 {$\#$ of ${\widetilde{D}_{\mathrm{grd}}} \le 1$} & $69$ & $48$ & $88$ \\
 \hline
 $avg({\widetilde{D}_{\mathrm{grd}}})$ & $1.83$ & $1.37$ & $0.30$\\
 \hline
\end{tabular}
\caption{\label{table:shrec11res}
Summary of the results presented in Figure~\ref{fig:shrec11plot}. Performance of our proposed method in comparison with the geodesic and biharmonic mappings using SHREC11 robustness benchmark. In the first row, the number of optimal results for which ${\widetilde{D}_{\mathrm{grd}}}\le 1$ (over all of $99$ mappings) is given. In the second row, the average of ${\widetilde{D}_{\mathrm{grd}}}$ over all the mappings (excluding the results with symmetric flip) is given.}
\end{table}
\begin{figure}[h!]
  \centering
  \includegraphics[width=.39\linewidth]{shrec11_ex2.png}
  \includegraphics[width=.59\linewidth]{shrec11_ex1.png}\\
  \includegraphics[width=.62\linewidth]{shrec11_ex3.png}
  \caption{\label{fig:shrec11ex}
  One-to-one mappings obtained by our method for three pairs of shapes from SHREC11 robustness benchmark.
  }
\end{figure}
In the next group of evaluation tests, we demonstrate the performance of our approach using the SHREC11 robustness benchmark. We use the shape models 0002, 0004, 0005, 0007, 0008, 0012 and 0014 for which the ground truth correspondence is available. For each model, we use the isometric shape and five shapes with topology noise. We also use the null shape from the models 0002 and 0007. In Figure~\ref{fig:shrec11plot}, we present the average ground truth error for the one-to-one mappings obtained using our proposed method, the geodesic-based and biharmonic-based mappings. The input pairs are the shapes from each model where at least one of them has topological distortion. As shown in Figure~\ref{fig:shrec11plot} and summarized in Table~\ref{table:shrec11res}, our approach successfully handles the topology noise as almost all of our mappings are optimal (${\widetilde{D}_{\mathrm{grd}}} \le 1$). Note that normalized average ground truth error is large for some results due to the symmetric flip problem. Excluding the mappings with symmetric flip, the average of the normalized average ground truth errors over all results, $avg({\widetilde{D}_{\mathrm{grd}}})$, is very small for our proposed method compared to the geodesic-based and biharmonic-based mappings (see Table~\ref{table:shrec11res}). Overall, the geodesic-based method either solves the correspondence problem or yields a result with an extremely high error. The biharmonic-based method has less number of optimal mappings but it generally solves some part of the correspondence problem and therefore decreases $avg({\widetilde{D}_{\mathrm{grd}}})$. Our method is robust to topological noise as it gives an optimal result for almost all of the mappings. In Figure~\ref{fig:shrec11ex}, we present one-to-one mappings obtained by our method for three pairs of shapes from SHREC11 robustness benchmark.

%\begin{figure}[htb]
%  \centering
%  \includegraphics[width=.9\linewidth]{figures/fig_shrec10_ex/shrec10_ex_topo5.png}
%  \caption{\label{fig:shrec10ex}
%  One-to-one mapping obtained by our method for two shapes from SHREC10 correspondence benchmark where the sitting man has topological noise of degree $5$.
%  }
%\end{figure}
%

\begin{figure}[t!]
  \centering
  \includegraphics[width=.17\linewidth]{mesh004.png}
  \includegraphics[width=.21\linewidth]{mesh017.png}
  \includegraphics[width=.20\linewidth]{mesh019.png}
  \includegraphics[width=.17\linewidth]{mesh039.png}
  \includegraphics[width=.20\linewidth]{mesh060.png}\\
  \includegraphics[width=.17\linewidth]{mesh004topo.png}
  \includegraphics[width=.21\linewidth]{mesh017topo.png}
  \includegraphics[width=.20\linewidth]{mesh019topo.png}
  \includegraphics[width=.17\linewidth]{mesh039topo.png}
  \includegraphics[width=.20\linewidth]{mesh060topo.png}
  \caption{\label{fig:scapetopomodels}
  TN-SCAPE dataset that we construct by adding topology noise to five meshes from SCAPE. In the first row, the original shapes are shown and their counterparts with added topological noise are given in the second row.
  }
\end{figure}
\begin{figure}[t!]
  \centering
  \includegraphics[width=.72\linewidth]{scape2_cmds_one2one2_biharmonic1SF.png}
  \caption{\label{fig:scapetopoplot}
  Normalized average ground truth error ${\widetilde{D}_{\mathrm{grd}}}$ for one-to-one mappings between topologically different pairs of shapes from TN-SCAPE dataset. The notations are the same as in Figure~\ref{fig:shrec11plot}. Observe that our approach successfully handles topology noise by using topologically comparable forms of the input shapes.
  }
\end{figure}
\begin{figure}[t]
  \centering
  \includegraphics[width=.84\linewidth]{scape2_ex.png}
  \caption{\label{fig:scape2ex}
  One-to-one mapping obtained by our method for a pair of shapes from TN-SCAPE dataset where both shapes have topology noise.
  }
\end{figure}
\begin{table}[t]
\centering
\begin{tabular}{|c|c|c|c|}
\hline
 & geodesic & biharmonic & proposed \\
 \hline
 {$\#$ of ${\widetilde{D}_{\mathrm{grd}}} \le 1$} & $14$ & $2$ & $21$ \\
 \hline
 $avg({\widetilde{D}_{\mathrm{grd}}})$ & $2.44$ & $2.70$ & $0.70$\\
 \hline
\end{tabular}
\caption{\label{table:scapetopores}
Summary of the results presented in Figure~\ref{fig:scapetopoplot}. Performance of our proposed method in comparison with the geodesic and biharmonic mapping using TN-SCAPE dataset. In the first row, the number of optimal results for which ${\widetilde{D}_{\mathrm{grd}}}\le 1$ (over all of $35$ mappings) is given. In the second row, the average of ${\widetilde{D}_{\mathrm{grd}}}$ over all the mappings (excluding the results with symmetric flip) is given.
}
\end{table}

In the final group of evaluation tests, we use TN-SCAPE dataset shown in Figure~\ref{fig:scapetopomodels} for further evaluation of our performance. The experimental settings are similar to the ones used in SHREC11 robustness benchmark. Again, we compare our mapping with the geodesic- and biharmonic-based mappings and we ensure that at least one shape in each input pair has topology noise. The experimental results are presented in Figure~\ref{fig:scapetopoplot}. In accordance with the previous results, our method successfully handles topology noise as most of the mappings are optimal (${\widetilde{D}_{\mathrm{grd}}} \le 1$) and few of them are very close to the optimal (${\widetilde{D}_{\mathrm{grd}}}$ is around $1$). Also, the average error measure $avg({\widetilde{D}_{\mathrm{grd}}})$ over all results, excluding the symmetric flips, is again very small for our proposed mapping compared to the geodesic- and biharmonic-based mappings (see Table~\ref{table:scapetopores}). In Figure~\ref{fig:scape2ex}, we show our proposed mapping for a pair of shapes from TN-SCAPE dataset where both shapes have topology noise.

\begin{figure*}[t!]
  \centering
  \begin{tabular}{cc}
  \includegraphics[width=.39\linewidth]{prop_0007null_topology1.png} &
  \includegraphics[width=.39\linewidth]{mateus_0007null_topology1.png}\\
  %(a) & (b)\\
  \end{tabular}
  \caption{\label{fig:mateus}
  A visual comparison of our approach with the method~\cite{Mateus08} that performs the best in the topology noise category of SHREC10 correspondence benchmark (Left) Our mapping result (Right) Dense mapping result obtained by the method~\cite{Mateus08}.
  }
\end{figure*}

Finally, we present a visual comparison of our approach with the method~\cite{Mateus08} that performs the best in the topology noise category of SHREC10 correspondence benchmark. We run the method~\cite{Mateus08} on a pair of horse shapes from SHREC11 robustness benchmark using its code available on the web. One of the horse shapes has the topological noise as its back legs are linked to each other. Figure~\ref{fig:mateus} shows that our approach successfully handles the topology noise whereas the method~\cite{Mateus08} fails to solve the correspondence problem under the given topology noise.

{We conduct our experiments on a $64$-bit workstation equipped with quad-core i7 processor (with clock frequency adjusted to 4.2GHz) and 32GB of RAM. We voxelize the mesh model of each shape so that the number of shape voxels is approximately $500$K. Computing the entire field (containing all deflations or inflations) over this voxel set takes, on the average,  $49$ seconds. Extracting a single level surface of the  field (a graded deflation or inflation) takes, on the average, $0.18$ seconds. }