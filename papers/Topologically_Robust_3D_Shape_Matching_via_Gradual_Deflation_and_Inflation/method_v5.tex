%-------------------------------------------------------------------------
%\section{The Method}
%\label{sec:method}


A straightforward means to erode (or dilate) shapes in curvature dependent manner   is to move each point on the shape boundary in the direction of surface normal. If the speed is chosen as proportional to a local feature such as curvature, the erosion (or dilation)  becomes curvature dependent.
Such curvature dependent shape boundary motion can be numerically implemented using the level set method~\cite{osher2006level}.
In this work, instead of directly implementing curvature dependent boundary motion, we use an approximate model from \cite{TSP96} which provides a simpler and more efficient computational model. Furthermore, using the approximate model, we obtain all of the deflated or inflated forms (of varying rate) in a single shot.

The used approximate model yields an approximately curvature-dependent distance field $v$ such that
\begin{equation}
v(\mathbf{x}) \approx \rho \biggl( 1+ \frac{\rho}{2} \; curv(\mathbf{x}) \biggr) \frac{\partial v}{\partial n}(\mathbf{x}) + O\left({\rho^3} \right),
\label{eq:approx_v}
\end{equation}
where $curv(\mathbf{x})$ is the curvature of the level surface of $v$ passing through the point $\mathbf{x}$, $n$ is the direction of the inward (or outward) normal and $\rho$ is a parameter determining the smoothness of the level surfaces.

The field $v$ is the minimizer of
\begin{equation}
\Lambda_{\rho}(v) = \frac{1}{2} \int {\left\{ \rho \, {||\nabla v||}^2 + \frac{\left(1-v\right)^2}{\rho} \right\} d\mathbf{x}},
\label{eq:energy_field}
\end{equation}
subject to the boundary condition $v = 0$. Hence, it can be  obtained by solving the  Euler Lagrange equation of \eqref{eq:energy_field}, which is a linear elliptic PDE. We discretize the PDE on a standard grid via finite-difference method.
%Homogeneous Dirichlet condition is applied on the shape boundary.
In the case of field computation outside the shape volume, \emph{i.e.}, inflation, we impose homogeneous Neumann condition on the grid boundary. We determine the parameter $\rho$ as the maximal radius of the shape volume computed using Euclidean distance transform. The discretization yields a linear system of equations with a sparse and symmetric positive definite system matrix. We solve the resulting system using the Cholesky decomposition based direct solver, specifically the MATLAB built-in CHOLMOD implementation.

We first voxelize the mesh model of a given shape. Once the curvature-sensitive distance field $v$ is computed inside (or outside) the shape volume, we generate a collection of level surfaces giving deflated or inflated  forms of the shape boundary at a variety of rates. We obtain the level surfaces using the MATLAB implementation for extracting iso-surface data from volume data. Note that the level surfaces are closed and connected meshes since the field is a smooth and continuous function. If there are more than one connected component in a level surface, we consider the largest one. The set of field values for which the collection of iso-surfaces are extracted are determined for the levels $t$ from $0$ to $1$ sampled at an interval such as $0.01$ using the formula $v_{max}\,\frac{e^{4t}-1}{e^4-1}$ where $v_{max}$ is the maximum value of the field.

\begin{figure}[htb]
  \centering
  \begin{tabular}{c}
  \includegraphics[width=.26\linewidth]{0002topology3fc.png}
    \includegraphics[width=.26\linewidth]{0002topology3f_v_in_rho_17_level0p1c.png}
    \includegraphics[width=.26\linewidth]{0002topology3f_v_in_rho_17_level0p3c.png}\\
  %(a) \\
  \includegraphics[width=.24\linewidth]{16c.png}
    \includegraphics[width=.24\linewidth]{16copy_vi_r85_lev0p350_c1_g1fc.png}
    \includegraphics[width=.24\linewidth]{16copy_vi_r85_lev0p400_c1_g0fc.png}\\
  %(b)\\
  \includegraphics[width=.24\linewidth]{379c.png}
  \includegraphics[width=.24\linewidth]{379_vo_r54_lev0p380_c1_g0c.png}
  \includegraphics[width=.24\linewidth]{379_vo_r54_lev0p430_c1_g1c.png}\\
  %(c)\\
  \end{tabular}
  \caption{\label{fig:fieldisosurfaces}
  (Top) A shape with topological noise from SHREC11 robustness benchmark and the inner iso-surfaces of the curvature-sensitive distance field at level $0.1$ and $0.3$. (Middle) A shape from Watertight dataset where the hands are connected to the legs and the inner iso-surfaces of the distance field at level $0.35$ and $0.40$. (Bottom) A teapot shape from Watertight dataset where the handle is cut on the upper joint and the outer iso-surfaces of the distance field at level $0.38$  and $0.43$.}
\end{figure}

In Figure~\ref{fig:fieldisosurfaces}, three illustrative examples are given.  The shape in the top row has topological noise where it is from SHREC11 robustness benchmark~\cite{bronstein2010shrec}. The curvature-sensitive distance field is computed inside the shape volume and the iso-surfaces at level $0.1$ and $0.3$ are the boundaries of the {deflated} forms.  At level $0.3$, the links induced by the topological noise disappear while the main shape structures are preserved. In the middle row, the shape from Watertight dataset~\cite{giorgi2007shrec} has genus two as both hands are connected to the legs. The wrists are thinner than the other main shape structures so the arms are separated from the legs for the inner iso-surface at level $0.4$. In the bottom row, considering the teapot shape from Watertight dataset, we compute the curvature-sensitive distance field outside the shape volume. The level surfaces of the field are the boundaries of the {inflated} forms. The iso-surface at level $0.43$ has genus one since the upper joint of the handle is connected to the body.

%We solve for the distance field once for each shape to obtain the level surfaces simultaneously. We give the details of the field computation and extraction of the level surfaces in \S\ref{ssec:field}.

Once we have a collection of level surfaces for each shape, we determine the levels at which the iso-surfaces from both shapes are manifold meshes with the same genus number.  We compute the genus number of surfaces using the Euler formula after checking their manifoldness. Note that the level surfaces are always closed and connected meshes but they may include some non-manifold vertices or edges. Among the pairs of level surfaces, we consider the one with the smallest genus number at the smallest level for applying the 3D shape correspondence algorithm. The mapping computed between the selected pair of level surfaces  is transferred to the input shapes by computing their vertices closest to the matched points of the level surfaces.


{To produce the mapping between the pair of level surfaces extracted from our $v$ field \eqref{eq:energy_field}, we employ the isometric shape correspondence algorithm in \cite{ys2012EM} whose block diagram is given in Figure~\ref{fig:em}. Extracted  level surfaces have trustworthy geodesic distances. This is important for the isometric correspondence algorithm to perform well. The algorithm starts by defining an isometric distortion function that measures, for a given map, deviation from isometry, or equivalently, that quantifies the quality of a given map. The basic idea is then to search the space of all possible maps to minimize this isometric distortion. This minimization problem is cast as maximization of the likelihood function in a probabilistic setting that is solved via Expectation-Maximization (EM) algorithm.}

\begin{figure} [t!]
\begin{center}
\includegraphics[width=\linewidth]{em.png}
\end{center}
   \caption{Block diagram of the EM algorithm \cite{ys2012EM}.}
\label{fig:em}
\end{figure}

Given source and target meshes equipped with geodesic distances, the EM algorithm alternates between i) recomputing the expected value of a matrix that encodes the probability of source vertex $s_i$ being in correspondence with target vertex $t_j$ (E-step), and ii) estimating a mapping that maximizes the log-likelihood by using first bipartite perfect matching, and then a greedy optimization algorithm (M-step). The probability matrix, hence the EM algorithm, is initialized based on the Euclidean distances between the vertices embedded into spectral domain through classical Multi-Dimensional Scaling (MDS). It is then filled based on the current mapping produced by the M-step. As the M-step produces more reliable maps, the probability matrix, \emph{i.e.}, the E-step, becomes more accurate, which in turn leads to an even better M-step. This alternating optimization between E-step and M-step yields the optimal one-to-one mapping in the minimum-distortion sense.

The isometric distortion function that guides the optimization gives the difference between the pairwise geodesic distances between sampled points on the source shape, and the distance of their images under the map, aggregated over all pairs of sampled points using a standard norm. Specifically, 
\begin{equation}
D_{\mathrm{iso}}(\S) = \frac{1}{|\S|} \sum_{(s_i, t_j) \in \S} d_{\mathrm{iso}}(s_i, t_j)\label{eq:iso}
\end{equation}
where $\S$ denotes the set of correspondence pairs between $S$ and $T$, and
\begin{equation}
d_{\mathrm{iso}}(s_i, t_j) = \frac{1}{|\S|-1} \sum_{\substack{(s_l, t_m) \in \S\\(s_l, t_m)\neq(s_i, t_j)}} |g(s_i, s_l) - g(t_j, t_m)|
\label{eq:diso}
\end{equation}
where $g(.,.)$ is the geodesic distance between two base vertices, or more generally, between two points on a given surface. Hence, $d_{\mathrm{iso}}(s_i, t_j)$ is the contribution of the individual correspondence $(s_i, t_j)$ to the overall isometric distortion. Both $d_{\mathrm{iso}}$ and $D_{\mathrm{iso}}$ take values in the interval $[0,1]$ since the function $g$ is normalized with respect to the maximum geodesic distance over the surface. Note that the entries in the probability matrix are defined in terms of isometric distortion $d_{\mathrm{iso}}(s_i, t_j)$; namely the value $e^{-d_{\mathrm{iso}}(s_i,t_j)}$ is used as the probability of matching $s_i$ to $t_j$.

{Although the EM algorithm rests on the basic assumption that the shapes to be matched are perfectly isometric, the experiments conducted in \cite{ys2012EM} show that it performs well also on nearly isometric shapes, e.g., a male matching with a female. In the case of severe deviations from isometry, e.g., a cat and a giraffe, however,  the initially selected samples can be in very different configurations on the two surfaces so that unintuitive matchings can be generated as the output of the algorithm. We finally note that the method can handle input meshes with arbitrary genus.}
