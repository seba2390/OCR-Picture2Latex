%%
%% This is file `sample-manuscript.tex',
%% generated with the docstrip utility.
%%
%% The original source files were:
%%
%% samples.dtx  (with options: `manuscript')
%% 
%% IMPORTANT NOTICE:
%% 
%% For the copyright see the source file.
%% 
%% Any modified versions of this file must be renamed
%% with new filenames distinct from sample-manuscript.tex.
%% 
%% For distribution of the original source see the terms
%% for copying and modification in the file samples.dtx.
%% 
%% This generated file may be distributed as long as the
%% original source files, as listed above, are part of the
%% same distribution. (The sources need not necessarily be
%% in the same archive or directory.)
%%
%%
%% Commands for TeXCount
%TC:macro \cite [option:text,text]
%TC:macro \citep [option:text,text]
%TC:macro \citet [option:text,text]
%TC:envir table 0 1
%TC:envir table* 0 1
%TC:envir tabular [ignore] word
%TC:envir displaymath 0 word
%TC:envir math 0 word
%TC:envir comment 0 0
%%
%%
%% The first command in your LaTeX source must be the \documentclass
%% command.
%%
%% For submission and review of your manuscript please change the
%% command to \documentclass[manuscript, screen, review]{acmart}.
%%
%% When submitting camera ready or to TAPS, please change the command
%% to \documentclass[sigconf]{acmart} or whichever template is required
%% for your publication.
%%
%%
\documentclass[manuscript,screen]{acmart}
%%
%% \BibTeX command to typeset BibTeX logo in the docs
\AtBeginDocument{%
  \providecommand\BibTeX{{%
    Bib\TeX}}}

%% Rights management information.  This information is sent to you
%% when you complete the rights form.  These commands have SAMPLE
%% values in them; it is your responsibility as an author to replace
%% the commands and values with those provided to you when you
%% complete the rights form.
\setcopyright{acmcopyright}
\copyrightyear{2018}
\acmYear{2023}
\acmDOI{XXXXXXX.XXXXXXX}

\usepackage{tabularx}
\usepackage{bbding}
\usepackage{wasysym}
\usepackage{threeparttable}
\usepackage{stfloats}
\usepackage{multirow}
\usepackage{makecell}
\usepackage{color}
\usepackage{enumerate}
\usepackage{listings}
\usepackage{xcolor}

\acmJournal{CSUR}


% \acmConference[Conference acronym 'XX]{Make sure to enter the correct
%   conference title from your rights confirmation emai}{June 03--05,
%   2018}{Woodstock, NY}
%%
%%  Uncomment \acmBooktitle if the title of the proceedings is different
%%  from ``Proceedings of ...''!
%%
%%\acmBooktitle{Woodstock '18: ACM Symposium on Neural Gaze Detection,
%%  June 03--05, 2018, Woodstock, NY}
\acmPrice{15.00}
\acmISBN{978-1-4503-XXXX-X/18/06}


\lstdefinelanguage{Solidity}{
	keywords=[1]{anonymous, assembly, assert, balance, break, call, callcode, case, catch, class, constant, continue, constructor, contract, debugger, default, delegatecall, delete, do, else, emit, event, experimental, export, external, false, finally, for, function, gas, if, implements, import, in, indexed, instanceof, interface, internal, is, length, library, log0, log1, log2, log3, log4, memory, modifier, new, payable, pragma, private, protected, public, pure, push, require, return, returns, revert, selfdestruct, send, solidity, storage, struct, suicide, super, switch, then, this, throw, transfer, true, try, typeof, using, value, view, while, with, addmod, ecrecover, keccak256, mulmod, ripemd160, sha256, sha3}, % generic keywords including crypto operations
	keywordstyle=[1]\color{blue}\bfseries,
	keywords=[2]{address, bool, byte, bytes, bytes1, bytes2, bytes3, bytes4, bytes5, bytes6, bytes7, bytes8, bytes9, bytes10, bytes11, bytes12, bytes13, bytes14, bytes15, bytes16, bytes17, bytes18, bytes19, bytes20, bytes21, bytes22, bytes23, bytes24, bytes25, bytes26, bytes27, bytes28, bytes29, bytes30, bytes31, bytes32, enum, int, int8, int16, int24, int32, int40, int48, int56, int64, int72, int80, int88, int96, int104, int112, int120, int128, int136, int144, int152, int160, int168, int176, int184, int192, int200, int208, int216, int224, int232, int240, int248, int256, mapping, string, uint, uint8, uint16, uint24, uint32, uint40, uint48, uint56, uint64, uint72, uint80, uint88, uint96, uint104, uint112, uint120, uint128, uint136, uint144, uint152, uint160, uint168, uint176, uint184, uint192, uint200, uint208, uint216, uint224, uint232, uint240, uint248, uint256, var, void, ether, finney, szabo, wei, days, hours, minutes, seconds, weeks, years},	% types; money and time units
	keywordstyle=[2]\color{teal}\bfseries,
	keywords=[3]{block, blockhash, coinbase, difficulty, gaslimit, number, timestamp, msg, data, gas, sender, sig, value, now, tx, gasprice, origin},	% environment variables
	keywordstyle=[3]\color{violet}\bfseries,
	identifierstyle=\color{black},
	sensitive=true,
	comment=[l]{//},
	morecomment=[s]{/*}{*/},
	commentstyle=\color{gray}\ttfamily,
	stringstyle=\color{red}\ttfamily,
	morestring=[b]',
	morestring=[b]"
}

\lstset{ %
  language=Solidity,                % the language of the code
  basicstyle=\footnotesize,           % the size of the fonts that are used for the code
  numbers=left,                   % where to put the line-numbers
  numberstyle=\tiny\color{gray},  % the style that is used for the line-numbers
  stepnumber=1,                   % the step between two line-numbers. If it's 1, each line 
                                  % will be numbered
  numbersep=5pt,                  % how far the line-numbers are from the code
  backgroundcolor=\color{white},      % choose the background color. You must add \usepackage{color}
  showspaces=false,               % show spaces adding particular underscores
  showstringspaces=false,         % underline spaces within strings
  showtabs=false,                 % show tabs within strings adding particular underscores
  frame=single,                   % adds a frame around the code
  rulecolor=\color{black},        % if not set, the frame-color may be changed on line-breaks within not-black text (e.g. commens (green here))
  tabsize=2,                      % sets default tabsize to 2 spaces
  captionpos=b,                   % sets the caption-position to bottom
  breaklines=true,                % sets automatic line breaking
  breakatwhitespace=false,        % sets if automatic breaks should only happen at whitespace
  title=\lstname,                 % show the filename of files included with \lstinputlisting;
                                  % also try caption instead of title
  keywordstyle=\color{blue},          % keyword style
  commentstyle=\color{dkgreen},       % comment style
  stringstyle=\color{mauve},         % string literal style
  escapeinside={\%*}{*)},            % if you want to add LaTeX within your code
  morekeywords={*,...},               % if you want to add more keywords to the set
}

%%
%% Submission ID.
%% Use this when submitting an article to a sponsored event. You'll
%% receive a unique submission ID from the organizers
%% of the event, and this ID should be used as the parameter to this command.
%%\acmSubmissionID{123-A56-BU3}

%%
%% For managing citations, it is recommended to use bibliography
%% files in BibTeX format.
%%
%% You can then either use BibTeX with the ACM-Reference-Format style,
%% or BibLaTeX with the acmnumeric or acmauthoryear sytles, that include
%% support for advanced citation of software artefact from the
%% biblatex-software package, also separately available on CTAN.
%%
%% Look at the sample-*-biblatex.tex files for templates showcasing
%% the biblatex styles.
%%

%%
%% The majority of ACM publications use numbered citations and
%% references.  The command \citestyle{authoryear} switches to the
%% "author year" style.
%%
%% If you are preparing content for an event
%% sponsored by ACM SIGGRAPH, you must use the "author year" style of
%% citations and references.
%% Uncommenting
%% the next command will enable that style.
%%\citestyle{acmauthoryear}


%%
%% end of the preamble, start of the body of the document source.
\begin{document}

%%
%% The "title" command has an optional parameter,
%% allowing the author to define a "short title" to be used in page headers.
\title{Survey on Quality Assurance of Smart Contracts}

%%
%% The "author" command and its associated commands are used to define
%% the authors and their affiliations.
%% Of note is the shared affiliation of the first two authors, and the
%% "authornote" and "authornotemark" commands
%% used to denote shared contribution to the research.
% \author{Ben Trovato}
% \authornote{Both authors contributed equally to this research.}
% \email{trovato@corporation.com}
% \orcid{1234-5678-9012}
% \author{G.K.M. Tobin}
% \authornotemark[1]
% \email{webmaster@marysville-ohio.com}
% \affiliation{%
%   \institution{Institute for Clarity in Documentation}
%   \streetaddress{P.O. Box 1212}
%   \city{Dublin}
%   \state{Ohio}
%   \country{USA}
%   \postcode{43017-6221}
% }

\author{Zhiyuan Wei}
\affiliation{%
  \institution{Beijing Institute of Technology}
  \city{Beijing}
  \country{China}
}
\email{weizhiyuan@bit.edu.cn}

\author{Jing Sun}
\affiliation{%
  \institution{University of Auckland}
  \city{Auckland}
  \country{New Zealand}
}
\email{jing.sun@auckland.ac.nz}

\author{Zijian Zhang}
\authornotemark[1]
\email{zhangzijian@bit.edu.cn}
\author{Xianhao Zhang}
\email{1120191274@bit.edu.cn}
\author{Xiaoxuan Yang}
\email{yangxiaoxuan979@gmail.com}
\author{Liehuang Zhu}
\email{liehuangz@bit.edu.cn}
\affiliation{%
  \institution{Beijing Institute of Technology}
  \city{Beijing}
  \country{China}
}

%%
%% By default, the full list of authors will be used in the page
%% headers. Often, this list is too long, and will overlap
%% other information printed in the page headers. This command allows
%% the author to define a more concise list
%% of authors' names for this purpose.
\renewcommand{\shortauthors}{Wei et al.}

%%
%% The abstract is a short summary of the work to be presented in the
%% article.
\begin{abstract}
  With the increasing adoption of smart contracts, ensuring their security has become a critical concern. Numerous vulnerabilities and attacks have been identified and exploited, resulting in significant financial losses. In response, researchers have developed various tools and techniques to identify and prevent vulnerabilities in smart contracts. In this survey, we present a systematic overview of the quality assurance of smart contracts, covering vulnerabilities, attacks, defenses, and tool support.
  By classifying vulnerabilities based on known attacks, we can identify patterns and common weaknesses that need to be addressed. Moreover, in order to effectively protect smart contracts, we have created a labeled dataset to evaluate various vulnerability detection tools and compare their effectiveness.
\end{abstract}

%%
%% The code below is generated by the tool at http://dl.acm.org/ccs.cfm.
%% Please copy and paste the code instead of the example below.
%%
\begin{CCSXML}
<ccs2012>
 <concept>
  <concept_id>10010520.10010553.10010562</concept_id>
  <concept_desc>Computer systems organization~Surveys and overviews</concept_desc>
  <concept_significance>500</concept_significance>
 </concept>
 <concept>
  <concept_id>10010520.10010575.10010755</concept_id>
  <concept_desc>Security and privacy~Distributed systems security</concept_desc>
  <concept_significance>300</concept_significance>
 </concept>
%  <concept>
%   <concept_id>10010520.10010553.10010554</concept_id>
%   <concept_desc>Computer systems organization~Robotics</concept_desc>
%   <concept_significance>100</concept_significance>
%  </concept>
%  <concept>
%   <concept_id>10003033.10003083.10003095</concept_id>
%   <concept_desc>Networks~Network reliability</concept_desc>
%   <concept_significance>100</concept_significance>
%  </concept>
</ccs2012>
\end{CCSXML}

\ccsdesc[500]{General and reference~Surveys and overviews}
\ccsdesc[300]{Security and privacy~Distributed systems security}

%%
%% Keywords. The author(s) should pick words that accurately describe
%% the work being presented. Separate the keywords with commas.
\keywords{smart contract, security, vulnerabilities, attacks, defenses}

\received{1 June 2023}
\received[revised]{1 June 2023}
\received[accepted]{1 June 2023}

%%
%% This command processes the author and affiliation and title
%% information and builds the first part of the formatted document.
\maketitle

\section{Introduction}
With the development of the internet, blockchain technology has gained popularity among academia and industries as a means to explore diverse applications where security and privacy are of utmost importance \cite{wood2014ethereum, hughes2019blockchain}.
As a distributed ledger, blockchain technology is replicated and shared among a network of peer-to-peer nodes. It eliminates the need for intermediaries, thereby providing decentralization, transparency, immutability, security, and reliability. By maintaining a chronologically growing and immutable data record, the blockchain system becomes an ideal infrastructure for future business networks. This can lead to a significant reduction in operational costs for the business system \cite{casino2019systematic}.

Within the realm of blockchain technology, smart contracts have gained significant prominence. Smart contracts are self-executing contracts with terms agreed upon by buyers and sellers, implemented through lines of code \cite{LuuCOSH16, AtzeiBC16}. They ensure the transparency and immutability of contract rules, which are enforced by all participants of the blockchain network. Smart contracts allows decentralized applications (DApps) to be built on the blockchain network. These DApps can be applied to a wide range of domains, such as financial services \cite{ZhangWLSL22}, healthcare services \cite{QahtanSZOZA23}, Internet of Things \cite{GuptaSK23}, crowdfunding management \cite{SaadatRNZ19} and supply chain \cite{SinghMGM23, SultanaTK22}. In most cases, smart contracts are associated with the native cryptocurrency of the blockchain, which is used to facilitate payment for their execution on the network. This incentivizes network participants to validate and execute smart contracts, ensuring that the blockchain remains secure and decentralized. Since smart contracts are a relatively new technology that has the potential to revolutionize the way business transactions are conducted, researchers are drawn to the novelty and potential impact of smart contracts and are interested in exploring their potential applications. Despite the growing interest, there are still many open research questions that need to be addressed.

Compared to traditional software programs, smart contracts indeed have characteristics that make them more vulnerable to software attacks. First, smart contracts are immutable, meaning that once deployed on the blockchain, their code cannot be modified. While immutability ensures trust and transparency, it also means that any vulnerabilities or errors in the code cannot be easily rectified without deploying a new version of the contract. Second, smart contracts often handle valuable digital assets, such as cryptocurrencies or digital tokens. The presence of these valuable assets attracts attackers who are more actively exploiting vulnerabilities in the smart contract code. Third, public blockchains, such as Ethereum, are permissionless, meaning that anyone can access and interact with smart contracts deployed on the blockchain. This open access reduces the barrier to entry for potential attackers, making it easier for them to identify and exploit vulnerabilities in smart contracts. Moreover, many smart contract developers choose to publish the code of their contracts openly on websites like Etherscan \cite{etherscan}. While this promotes transparency and collaboration within the community, it also makes it easier for attackers to analyze the code, identify weaknesses, and devise attack strategies. Due to these factors, the exploitation of smart contracts has resulted in significant financial losses in the past. A notorious malicious attack occurred in 2016 when attackers successfully diverted over 3.6 million ether into a ``child DAO" (Decentralized Autonomous Organization). This exploit resulted in a staggering loss of approximately 70 million USD, causing a significant drop in the price of each ether from 20 USD to 13 USD \cite{Falkon2017}. Another significant incident involved the theft of a substantial amount of ether in the Parity Multisig Wallet version 1.5+. This exploit led to a loss of approximately 30 million USD \cite{Palladino2017}.

Due to the novelty and potential impact of smart contracts, there are some notable surveys involving vulnerable smart contracts from various perspectives. Atzeri et al. \cite{AtzeiBC16} were pioneers in providing a survey on smart contract security. They classified 12 vulnerabilities into three categories: Solidity, EVM bytecode, and blockchain. This classification has been widely adopted by subsequent researchers.
Zheng et al. \cite{zheng2020overview} compared different smart contract platforms and provided a categorization of applications. Their survey focused on evaluating the features and characteristics of various platforms.
Chen et al. \cite{ChenPNX20} went beyond vulnerability analysis and considered defense mechanisms for blockchain security. Their survey discussed different defense techniques and strategies employed to enhance the security of smart contracts.
Angelo and Salzer \cite{AngeloS19} conducted a survey on vulnerability detection tools specifically designed for Ethereum smart contracts. Their study encompassed tools from both academia and industry, providing insights into the available tools for identifying vulnerabilities. 
Durieux et al. \cite{DurieuxFAC20} conducted an evaluation review of nine smart contract detection tools. They assessed these tools using a labeled dataset and a large number of real-world smart contracts, providing a comprehensive analysis of their effectiveness.
Tolmach et al. \cite{TolmachLLLL22} focused their study on the formal verification of smart contracts across various applications. Their survey delved into the application of formal verification techniques to verify the correctness and security of smart contracts. 

While previous surveys offer valuable insights into specific aspects of smart contract security, they may not provide a comprehensive analysis of vulnerable smart contracts from all perspectives. It is crucial to develop a thorough understanding of vulnerabilities, attacks, defenses, and tool evaluation to gain a holistic view of the challenges and potential solutions related to smart contract security. To bridge this gap and offer a more complete analysis, our paper aims to incorporate multiple perspectives. By examining vulnerabilities, attacks, defenses, and tools, we seek to provide a comprehensive understanding of the challenges posed by vulnerable smart contracts and explore potential solutions. The primary objective of our paper is to provide a comprehensive and up-to-date analysis that contributes to the existing body of research on smart contract security. To achieve this, we outline the following key contributions:

\begin{itemize}
  \item{\textbf{Novel Vulnerability Classification}:} We propose a novel vulnerability classification that enhances the understanding of the underlying causes of vulnerabilities in smart contracts. This classification will aid researchers in categorizing and analyzing vulnerabilities more effectively, providing a solid foundation for targeted security measures and improved vulnerability management.
  
  \item{\textbf{In-depth Analysis of Real-World Attacks}:} We conduct an extensive analysis of real-world attacks on smart contracts to gain valuable insights into the methods employed by attackers and the potential consequences of these attacks. By examining and dissecting these attacks, we aim to provide a clearer understanding of the exploitation of vulnerabilities in practice, enabling developers and auditors to proactively address potential threats.
  
  \item{\textbf{Exploration of Defense Mechanisms}:} We thoroughly explore existing defense mechanisms employed to mitigate smart contract attacks. Through this exploration, we identify areas for improvement and potential new approaches to enhance the security of smart contracts. By analyzing the strengths and weaknesses of current defense mechanisms, we aim to contribute to the development of more robust and effective security practices.
  
  \item{\textbf{Evaluation of Vulnerability-Detecting Tools}:} We conduct a comprehensive evaluation of 12 representative vulnerability-detecting tools used in smart contract analysis. This evaluation encompasses the accuracy, performance, and effectiveness of each tool. By providing insights into the strengths and weaknesses of these tools, we assist researchers and practitioners in selecting the most suitable tools for identifying vulnerabilities in smart contracts.

  \item{\textbf{Benchmark Dataset for Tool Evaluation}:} To facilitate the evaluation of vulnerability-detecting tools, we create a benchmark dataset comprising 100 vulnerable contract cases and 10 correct contract cases. This benchmark dataset serves as a standardized reference for evaluating the effectiveness of these tools. It enables fair and objective comparisons, allowing researchers and practitioners to assess the capabilities of different tools in a consistent manner.
\end{itemize}

Through these key contributions, our paper aims to serve as a comprehensive resource for researchers, developers, and auditors in the field of smart contract security. We aspire to advance secure smart contract development practices and promote the widespread adoption of secure smart contracts in real-world applications.

The rest of this paper is constructed as follows. Section \ref{Overview} provides a brief overview of smart contract platforms and discusses the methodology employed in this survey to ensure a comprehensive analysis. Section \ref{vulnerability} presents a detailed analysis of 40 smart contract vulnerabilities, examining their root causes and shedding light on the underlying factors that contribute to their existence. Section \ref{attacks} explores 8 representative attacks and demonstrates how these vulnerabilities can be exploited. By illustrating real-world attack scenarios, we aim to enhance our understanding of the potential consequences of these vulnerabilities. Section \ref{defense} examines various defense methodologies and repair techniques available for smart contracts. We discuss the effectiveness of these defensive measures in mitigating vulnerabilities and reducing the likelihood of successful attacks. Section \ref{evaluation} focuses on the evaluation of 12 commonly used tools for detecting vulnerabilities in smart contracts. We assess the accuracy and performance of these tools by subjecting them to a rigorous evaluation against a carefully curated benchmarking dataset. Section \ref{discussion} concludes the paper by summarizing the key findings and contributions. We also discuss future directions and potential research areas to further advance the field of smart contract security. The Appendix includes information on detection tools specifically designed for smart contracts.



\section{Overview of Smart Contracts and Survey Methodology}
\label{Overview}
\subsection{Smart Contracts}

The basic idea of smart contracts can be attributed to Nick Szabo, who introduced the concept in 1996, predating the creation of blockchain technology. Szabo envisioned smart contracts as computer protocols that enable parties to engage in digitally verifiable and self-executing agreements. These contracts are written in code format and facilitate secure and efficient transactions, reducing costs and expediting execution compared to traditional contracts \cite{Szabo2018SmartC}. However, the development of smart contracts faced challenges in trustless systems until the emergence of Ethereum in 2015 \cite{KhanLGBB21}. Ethereum's introduction of a blockchain-based platform specifically designed for executing smart contracts revolutionized the field. Ethereum's success paved the way for the proliferation of other blockchain platforms that also support smart contract development. These platforms include Hyperledger \cite{Saad2020}, EOSIO \cite{HeZ00L0YJ21}, Tezos \cite{BernardoCCJPT20}, NEO \cite{NguyenDT19}, and even Bitcoin, which implemented its own version of smart contracts \cite{BartolettiZ19}. We present a comprehensive comparison of four prominent blockchain platforms, highlighting their similarities and differences in terms of smart contract implementation and security. 

% \begin{table}[t]
%   \renewcommand\arraystretch{2}
%   \tiny
%   \centering
%   \caption{Comparative                   
% analysis of smart contracts on different blockchain platforms}
%   \begin{tabular}{|p{1.5cm}<{\raggedright}|p{2.6cm}<{\raggedright}|p{2.6cm}<{\raggedright}|p{2.6cm}<{\raggedright}|p{2.6cm}<{\raggedright}|}\bottomrule
%   \textbf{Platform}  & \textbf{Bitcoin} & \textbf{Ethereum} & \textbf{Hyperledger Fabric} & \textbf{EOSIO}  \\ \hline \hline
%   Program Language  & Script & Solidity & Go, JavaScript, and Java  & C++, Python, and JavaScript \\ \hline
%   Consensus Mechanism  & POW & POW/POS & PBFT/Raft & DPOS+BFT \\ \hline
%   Operating Environments & Script-based &EVM & Docker & EOS VM  \\\hline
%   Cryptocurrency  & Bitcoin (BTC) & Ether (ETH) & None & EOS \\ \hline
%   Scalability & High &High & Low & High \\\hline
%   Speed & 27 TPS \cite{bitcointps} & 30 TPS \cite{ethereumtps} & -- & 4,000 TPS \\\hline
%   Confirmation Time & >1000s &<100s & <10s & <10s \\\hline
%   Application Scenarios &  Send/receive messages, verify signatures, put/search transactions & Financial transaction, supply chain, IoT and medical application & Enterprise level licensing blockchain, financial transaction &  Financial transaction, gambling \\\hline
%   Existing Security Issues & Informal narration of cryptographic protocols; the use of poorly documented features & Errors in source code or EVM; attacks from blockchain & Errors in source code, attacks from blockchain & Errors in source code; lack of verification in transaction  \\\hline
%   The Main Solutions & The design of secure cryptographic protocols & Vulnerability detecting tools, contracts monitoring  & Vulnerability detecting tools, upgradation & Vulnerability detecting tools, upgradation \\\hline
%   Related survey  & \cite{BartolettiZ18a,BartolettiLMZ22,AtzeiBCLZ18} & \cite{AtzeiBC17,AngeloS19,ChenPNX20, zhou2020ever} & \cite{Saad2020,VaccaSVC21} & \cite{abs-2207-09227} \\ \hline
% \end{tabular}
% \label{smart_contracts}
% \end{table} 

\begin{itemize}
  \item \textbf{Bitcoin} primarily aims to facilitate secure peer-to-peer currency transactions without the need for intermediaries. While Bitcoin was not originally designed to support extensive smart contract functionalities such as those of Ethereum, it does provide limited capabilities for executing cryptographic protocols. Bitcoin smart contracts are primarily written in a programming language named Script, which is a simple, stack-based programming language that is specifically designed for transactions. These smart contracts on Bitcoin are often referred to as script-based contracts. Researchers have focused on studying the construction and capabilities of Bitcoin's scripting language, such as Ivy \cite{ivy}, BALZAC \cite{AtzeiBLZ18}, SIMPLICIT~\cite{OConnor17}, and BITML \cite{AtzeiBLYZ19}. 
  
  \item \textbf{Ehtereum} is the pioneering blockchain platform that introduced the deployment of smart contracts. It allows developers to create their own smart contracts using Solidity, a Turing-complete programming language. These smart contracts are executed on the Ethereum Virtual Machine (EVM), which provides a runtime environment for their execution. The introduction of smart contracts on Ethereum has significantly expanded the range of applications and use cases for blockchain technology. This development has given rise to a new paradigm known as Decentralized Applications (DApps), which leverage the capabilities of smart contracts to enable decentralized and trustless interactions. Solidity is designed specifically for Ethereum and allows developers to write code that complies with EVM bytecode, which is the low-level representation of smart contracts.
  
  \item \textbf{Hyperledger Fabric (HF)}  is a blockchain platform designed primarily for enterprise. Unlike public blockchains like Bitcoin and Ethereum, HF operates on a permissioned network where a defined set of trusted organizations participate through a membership service provider \cite{AndroulakiBBCCC18}. In Hyperledger Fabric, smart contracts are referred to as \textit{chaincode}. One of the notable advantages of HF is its flexibility in terms of programming language support for writing chain codes. Developers can choose from various programming languages such as Golang, Java, and JavaScript to implement their smart contracts, allowing them to leverage their existing expertise and use the most suitable language for their application \cite{KhanLGBB21}. Privacy is another key feature of HF smart contracts.
  
  \item \textbf{EOSIO} introduces several innovative features to the field of blockchain technology, e.g., delegated proof of stake consensus (DPOS) and updatable smart contracts \cite{HeZ00L0YJ21}. EOSIO supports a range of programming languages, including C++, Python, and JavaScript, giving developers the flexibility to choose the language they are most comfortable with and that best suits their application requirements. This multi-language support enables developers to leverage their existing skills and expertise, making it easier to create and deploy smart contracts on the EOSIO platform. One key advantage of EOSIO smart contracts is their ability to handle high transaction throughput with almost 4,000 transactions per second (TPS) \cite{EOSIONetwork}.
\end{itemize}

As the use of smart contracts becomes more widespread, the importance of their security cannot be overstated. In the past few years, smart contracts have suffered from an increasing number of security issues, leading to financial losses and reputational damage. Therefore, it is crucial for academics and industries to pay attention to the security of smart contracts and develop techniques and tools to ensure their safety. Figure \ref{fig_statistics} illustrates an increasing trend in both academic papers and software tools from 2016 to 2020, indicating a growing interest and activity in smart contract research and development. However, there has been a slight linear descent in the number of papers and tools in the subsequent years of 2021 and 2022, which may suggest a possible stabilization or saturation of the field. This further indicates a perfect timing for conducting a substantial literature review of the field. Nevertheless, it is important to note that the numbers alone do not provide insights into the quality or impact of the papers and tools.

\begin{figure}[ht]
	\centering
	\includegraphics[width=3.5in]{figure/paper_year.png}
	\caption{Papers published in the previous study from 2016 to 2022}
	\label{fig_statistics}
\end{figure}


\subsection{Survey Methodology}
\label{survey_methodology}
The objective of our survey is to provide a comprehensive analysis of different approaches to dealing with vulnerable smart contracts. To achieve this, we have formulated a set of \textit{Research Questions (RQ)} that outline the scope of our review:

\begin{enumerate}[RQ 1:]
  \item {[Vulnerabilities]} What are the common vulnerabilities that exist in smart contracts, and how can we classify them based on their characteristics?
  \item {[Attacks]} How do attackers exploit these vulnerabilities, and what are the consequences of these attacks on smart contracts?
  \item {[Defenses]} What defense methodologies are available to protect smart contracts against attacks, and how do these defenses mitigate the risks associated with vulnerabilities?
  \item {[Effectiveness]} How effective are the existing vulnerability analysis tools in identifying and detecting vulnerabilities in Solidity smart contracts, and what are the strengths and limitations of these tools in terms of accuracy, performance, and coverage?
  \item {[Tools and Benchmark]} How do we select representative vulnerability analysis tools for evaluation, and how can we create a standardized benchmark of vulnerable and correct smart contract cases to evaluate the effectiveness of these tools?
\end{enumerate}

By addressing these research questions, our survey aims to provide comprehensive insights into the vulnerabilities, attacks, defense mechanisms, effectiveness of analysis tools, and the creation of a benchmark for smart contracts. To achieve this, we conducted extensive searches across various databases and search engines, including Google Scholar, IEEE Xplore, ACM Digital Library, Springer Link, Scopus, and preprint repositories. We employed the following keywords and search terms: \textit{``blockchain" AND ``smart contracts" AND ``vulnerabilities" OR ``security" OR ``attacks"}.
Next, we anticipate a substantial number of search results that require screening to identify relevant papers. The screening process generally consists of two stages: a title and abstract screening stage, followed by a full-text screening stage \cite{polanin2019best}. Once we have identified the relevant papers, we assess their quality based on predefined inclusion criteria. 
This generally involves assessing the relevance, methodology, validity, reliability, credibility, clarity, and impact. The selected papers should address the topic of smart contracts in a meaningful way and provide valuable insights for our investigation. The methodology employed should be suitable for addressing our research questions, transparent, and capable of being replicated. The results should be accurate and consistent, and the conclusions drawn should be well-supported by the data. The authors should present their ideas clearly and concisely, with the paper being free of errors and inconsistencies. Furthermore, the paper should have made a significant contribution to the field of smart contracts.

After evaluating these quality attributes, we have collected 174 related papers from 72 conference proceedings (including CCS, IEEE S\&P, NDSS, USENIX, Euro S\&P, Financial Crypto, ASIA CCS, ICSE, ASE) and 11 journals (including IEEE TSE, IEEE COMST), as well as preprints published between 2015 and 2022. Once we have assessed the quality of the selected papers, we proceed to extract the relevant data from them. This process involves gathering information about the research questions, methodology, findings, and conclusions presented in each paper. We have described the collected articles in an online repository available at: \url{https://sites.google.com/view/weifar-sc-survey-publication}.

\section{Vulnerability in Smart Contracts}
\label{vulnerability}

\begin{figure}[!ht]
    \centering
    \includegraphics[width=4.5in]{figure/re_revised_location_vul_tax.pdf}
    \caption{A classification of smart contracts vulnerabilities and their causes, where $\CIRCLE$  means the vulnerability has already been solved,  $\Circle$ means the vulnerability is widely discussed (not solved), and $\LEFTcircle$ means the vulnerability has been fixed by specific approaches or tools, \textit{VE} means vulnerabilities of Ethereum, \textit{VH} means vulnerabilities of HF, \textit{VS} means vulnerabilities of ESOIO.}
    \label{fig_classification}
\end{figure}

This section presents a robust methodology for effectively identifying and categorizing vulnerabilities in blockchain smart contracts. To ensure comprehensive coverage of various vulnerability types, our investigation is conducted from multiple perspectives. For Ethereum smart contracts, we primarily rely on two key projects: the Decentralized Application Security Project (DASP) \footnote{https://dasp.co/} and the Smart Contract Weakness Classification (SWC) Registry \footnote{https://swcregistry.io/}. DASP provides a ranking of the top 10 smart contract vulnerabilities. However, it should be noted that this list has not been updated since its initial release in 2018. On the other hand, the SWC Registry provides an implementation of weakness classification and currently lists 37 vulnerabilities (as of the time of writing). Although it lacks a concrete classification and ranking, it offers valuable insights into specific vulnerability types.

In addition to these projects, we also draw insights from several related research papers, including \cite{AtzeiBCLZ18, LvWWZ21, HeZ00L0YJ21, perez2021smart, hu2021comprehensive, rameder2022review}. By diligently collecting and examining all identified smart contract vulnerabilities and their respective causes, our aim is to establish a comprehensive methodology for categorizing the root causes of vulnerabilities in blockchain smart contracts. This methodology is built upon the well-established Common Weakness Enumeration (CWE) rules and effectively identifies four primary root causes of vulnerabilities: coding standards, data authenticity, access control, and control flow management. To illustrate our classification framework, Figure \ref{fig_classification} provides a visual representation of smart contract vulnerabilities based on their root causes and corresponding secondary causes. Our classification framework encompasses 14 distinct secondary causes, which are associated with 40 specific vulnerabilities found in Ethereum, Hyperledger Fabric (HF), and EOSIO. Furthermore, the figure also indicates the status of each vulnerability, indicating whether it has been eliminated, can be mitigated by specific methods, or remains unsolved. 
Recognizing the lack of consistency in the naming and definition of vulnerabilities across various studies, we have taken measures to ensure clarity and uniformity in our analysis. As part of our research, we have standardized the names of these vulnerabilities. Furthermore, we offer detailed explanations for each vulnerability to promote better comprehension and facilitate future research and analysis.

\subsection{Improper Adherence to Coding Standards}
This type of weakness happens when a smart contract is not developed in accordance with established coding rules and best practices. This issue often arises due to the relative novelty of programming languages used for smart contracts, leading to a shortage of experienced developers in the domain. Furthermore, some developers may lack a comprehensive understanding of the specific coding standards for these programming languages, resulting in mistakes and vulnerabilities within the smart contract code. Improper adherence to coding standards can manifest in various ways:

\subsubsection{Syntax Errors (VE1, VE2)}
These errors occur when the code violates the syntax rules of the programming language, such as spelling and punctuation flaws. Two specific examples of syntax errors are \textit{typographical error (VE1)} and \textit{Right-To-Left-Override control character (VE2)}. \textit{VE1} refers to a typographical error in the code where an incorrect operator is used, and \textit{VE2} involves the misuse of the U+202E unicode character. Both of these vulnerabilities can be mitigated by following best practices and employing preventive measures. 
%Pre-condition checks and proper code review processes can help identify typographical errors and incorrect operators, ensuring the code functions as intended.
%  Additionally, using reliable libraries such as SafeMath \cite{SafeMath} can provide a robust foundation for performing secure mathematical operations, reducing the risk of errors and vulnerabilities.

\subsubsection{Version Issues (VE3, VE4, VE5, VE6, VE7)} 
Version issues in smart contracts can arise due to the rapid progress and updates in smart contract technology, including changes in compiler versions. When developers write code using outdated or deprecated functions, operators, or coding standards in a new compiler version, it can result in unexpected behaviors and potentially exploitable states. This category of vulnerabilities includes five specific flaws: \textit{outdated compiler version (VE3)}, \textit{floating pragma (VE4)}, \textit{use of deprecated Solidity functions (VE5)}, \textit{incorrect constructor name (VE6)}, and \textit{uninitialized storage pointer (VE7)}. To avoid these vulnerabilities, it is essential to stay updated on the latest version of the compiler and adhere to the recommended coding standards and best practices. 
% Using a recent version of the compiler ensures compatibility with the latest language features, bug fixes, and security enhancements. 
% Additionally, developers should review and update their codebase to use the recommended functions, constructors, and storage initialization techniques specified in the updated compiler version.

\subsubsection{Irrelevant Code (VE8, VE9, VE10)}
Irrelevant code in smart contracts refers to code that is not essential for the execution or functionality of the contract. While this code may not directly impact the correctness of the contract, it can introduce security vulnerabilities or make them harder to detect. It is not uncommon for programming code to contain unused or shadowing parts. This category of vulnerabilities includes three specific flaws: \textit{presence of unused variables (VE8)}, \textit{code with no effects (VE9)}, and \textit{shadowing state variables (VE10)}. To mitigate these vulnerabilities, it is important for contract writers to thoroughly test the functionality and behavior of the code before deployment. 
% This includes testing each intended behavior of the contract to ensure that it works as intended and does not contain irrelevant or unused code. 
% By conducting comprehensive testing and code reviews, contract writers can reduce the risk of vulnerabilities introduced by irrelevant code in smart contracts.


\subsubsection{Visibility (VE11, VE12)}
Solidity provides visibility labels for functions and variables, namely \textit{public}, \textit{external}, \textit{private}, or \textit{internal}. Each visibility label determines who can access or call specific functions or variables. The default visibility setting in Solidity is \textit{public}, which means that if the contract writer does not explicitly specify the visibility, functions and variables will be treated as \textit{public} by default. Forgetting to set the appropriate visibility for a function or variable can lead to two vulnerabilities: \textit{function default visibility (VE11)} and \textit{state variable default visibility (VE12)}. To mitigate these risks, contract writers should carefully consider the suitable visibility for each function and variable. 
% Implementing pre-condition checks is crucial to ensuring that only authorized parties have access to critical functions or sensitive state variables.


\subsection{Insufficient Verification of Data Authenticity}
This type of weakness occurs when systems fail to properly verify the origin or authenticity of data, which can allow attackers to manipulate or access sensitive information. This can lead to a wide range of security issues, including \textit{cryptographic signatures} and \textit{cryptographic data}.

\subsubsection{Cryptographic Signatures (VE13, VE14, VE15)}
Cryptographic signatures play a crucial role in validating the authenticity and integrity of data within blockchain systems. In Ethereum (and Bitcoin), the Elliptic Curve Digital Signature Algorithm (ECDSA) is commonly used for cryptographic signature generation and verification. However, there are certain vulnerabilities related to cryptographic signatures that can be exploited by attackers: \textit{signature malleability (VE13)}, \textit{missing protection against signature replay attacks (VE14)}, and \textit{lack of proper signature verification (VE15)}. To mitigate these vulnerabilities, it is of utmost importance to implement robust signature verification mechanisms. 
% In the context of Ethereum, the built-in function \textit{ecrecover()} serves as a valuable tool for verifying ECDSA signatures. However, it is crucial to exercise diligence and precision when utilizing this function to ensure the thorough validation of signature integrity and authenticity. By doing so, we can effectively prevent issues such as signature malleability, replay attacks, and inadequate verification.


\subsubsection{Cryptographic Data (VE16, VS1, VS2)}
This type of vulnerability in smart contracts refers to situations where sensitive data, despite being marked as \textit{private}, can still be accessed by unauthorized parties. This vulnerability arises due to the inherent transparency of blockchain transactions, which allows the content of transactions to be readable by anyone.  Attackers can easily access and acquire the data stored in the contract, leading to significant financial losses for the contract creator and participants. The vulnerabilities associated with cryptographic data include \textit{unencrypted private data on-chain (VE16)}, \textit{fake EOS (VS1)}, and \textit{forged notification, fake receipt (VS2)}. To address these vulnerabilities, it is crucial to prioritize the proper encryption of sensitive data before storing it on-chain. 
% This ensures that the data remains secure and protected. Furthermore, developers should exercise caution when dealing with contracts that involve private data, taking extra care to thoroughly test their contracts. By doing so, they can effectively minimize the risk of falling victim to honeypots or other malicious schemes.


\subsection{Improper Access Control}
This type of weakness arises when unauthorized users gain access to a contract and can perform actions that they should not be allowed to. Such vulnerabilities can have significant repercussions for the smart contract ecosystem, including financial losses and other adverse outcomes. Improper access control vulnerabilities manifest in two primary forms: \textit{unprotected low-level function} and \textit{coding issues}. Addressing these vulnerabilities is crucial to maintaining the security and integrity of smart contracts.

\subsubsection{Unprotected Low-level Function (VE17, VE18, VE19)}
Users can utilize Solidity's low-level functions \textit{SELFDESTRUCT}, \textit{tx.origin}, and \textit{DELEGATECALL} to control contracts. These low-level functions provide powerful capabilities but can be easily abused by malicious users if not used with caution. The following are the vulnerabilities associated with unprotected low-level functions:

\begin{itemize}
  \item \textit{Unsafe suicide (VE17)} The \textit{SELFDESTRUCT} function allows a contract to be removed from the blockchain, returning any remaining Ether to a designated target address. While this can be useful in certain scenarios, it carries risks. If Ether is sent to a contract that has self-destructed, the funds will be permanently lost and cannot be recovered. Consequently, it is imperative to exercise caution when utilizing the \textit{SELFDESTRUCT} function. Developers should carefully consider the variables and conditions involved before employing this function. 
%   It is crucial to avoid referencing contract addresses or funds that could be manipulated in a way that leads to unintended loss of Ether. By adhering to these precautions, developers can mitigate the risk of irreversible Ether loss and ensure the integrity of their smart contracts.
  
  \item \textit{Authorization through tx.origin (VE18)} The \textit{tx.origin} variable represents the address that initiated a transaction, while \textit{msg.sender} represents the immediate invoker of a function. This vulnerability happens when one contract calls another contract, \textit{tx.origin} does not represent the calling address but rather the original initiator of the transaction. This can result in funds being transferred to the wrong address. To mitigate this vulnerability, it is recommended to use \textit{msg.sender} instead of \textit{tx.origin} for authorization checks.
  
  \item \textit{Unsafe delegatecall (VE19)} This vulnerability arises from the \textit{DELEGATECALL} instruction, which allows third-party code to be executed within the context of a current contract. This vulnerability, also known as \textit{delegatecall to untrusted callee}, can be exploited by attackers to take control of another contract, especially in proxy contracts where code can be dynamically loaded from different addresses at runtime. If an attacker can manipulate the address used in \textit{DELEGATECALL}, they may modify storage or execute malicious code, leading to unauthorized actions such as fund theft or contract destruction.
\end{itemize}

\subsubsection{Coding Issues (VE20, VE21)}
Due to unintentionally exposing some functions, malicious parties can withdraw some or all Ether from the contract account. This type of flaw leads to \textit{unprotected Ether withdrawal (VE20)} and \textit{write to arbitrary storage location (VE21)}. This type of flaw can be avoided by carefully designing the code or structure.

\subsection{Insufficient Control Flow Management}
This type of weakness occurs when attackers can exploit the openness of the public blockchain to gain control over the program's execution in unexpected ways. This weakness can manifest in various forms as follows:

\subsubsection{Improper Input (VE22, VE23, VE24)} Due to error handling in EVM, improper input can cause \textit{assert violation (VE22)}, \textit{requirement violation (VE23)} and \textit{wrong address (VE24)}. The Solidity assert(), require(), as guard functions are introduced to improve the readability of contract code. However, assert() and require() require strong logical conditions, and improper input will cause errors. Moreover, the length of a contract address should be 40 hexadecimal characters. If the address length is incorrect, the contract can still be deployed without any warning from the compiler. Ethereum automatically registers a new address that is owned by nobody, and any Ether sent to this address becomes inaccessible. To mitigate these vulnerabilities, it is important to carefully handle input validation. 

\subsubsection{Incorrect Calculation (VE25, VH1, VE26, VS3)} This type of weakness happens when contracts perform a calculation that generates incorrect results and that may lead to a larger security issue such as arbitrary code execution. \textit{Arithmetic overflow/underflow (VE25, VH1)} is the most common error in software, and it both happens in Ethereum and HF. \textit{Call-stack overflow (VE26)} occurs due to EVM imposing a limit on the depth of the call-stack, allowing a maximum of 1024 nested function calls. If an attacker successfully reaches this limit by repeatedly invoking functions, it can result in a call-stack overflow vulnerability. Once the call-stack reaches its maximum depth, subsequent instructions, such as the \textit{send} instruction, will fail. \textit{Asset overflow (VS3)} specifically pertains to the EOSIO blockchain. It occurs when there is an overflow in the asset type, which represents token balances and other asset values on the EOSIO platform.

\begin{itemize}
 \item \textit{Arithmetic overflow/underflow (VE25)} This vulnerability, commonly known in software programming, is not specific to smart contracts. It occurs when an arithmetic operation produces a value that exceeds the maximum or minimum range of integer representation. In Ethereum contracts, this vulnerability arises due to the behavior of the EVM's integer arithmetic and the lack of automatic checks for arithmetical correctness. For instance, if the result of an addition operation surpasses the maximum value representable by a specific integer type, it wraps around to a lower-than-expected value without raising an error or warning. Research by Torres et al. \cite{TorresSS18} has identified over 42,000 contracts, particularly ERC-20 Token contracts, vulnerable to arithmetic overflow/underflow. To mitigate this issue, it is advisable to employ libraries such as SafeMath \cite{SafeMath}. 
 
\end{itemize}


\subsubsection{Denial of Service (VE27, VE28, VE29)} Denial of Service vulnerabilities can affect smart contracts and result in exceptions that may lead to undesirable consequences such as contract lock-ups or freezing of funds. There are multiple ways in which DoS attacks can be carried out, with two primary incentives: failed calls and gas consumption. One type of DoS vulnerability is \textit{DoS with failed call (VE27)}, where an external call, whether accidental or deliberate, fails. This vulnerability is particularly relevant in payment scenarios where multiple calls are executed within a single transaction. The failure of an external call can disrupt the intended flow of the contract, potentially leading to undesired consequences. Another DoS problem can be concluded as gas-related vulnerabilities. 
Although gas is a mechanism designed to prevent resource abuse, attackers can exploit this mechanism to trigger the other two vulnerabilities: \textit{insufficient gas griefing (VE28)} and \textit{DoS with block gas limit (VE29)}.
Mitigating these DoS vulnerabilities requires careful design and consideration of gas usage. 
 

\begin{itemize}
  \item \textit{DoS with block gas limit (VE29)} To safeguard the network, each block has a predetermined maximum amount of gas that can be consumed, known as the \textit{Block Gas Limit}. The gas consumption of a transaction must be less than or equal to the Block Gas Limit; otherwise, the transaction will fail to execute and any changes made during its execution will be rolled back. This ensures that no single transaction monopolizes excessive resources within a block, promoting fair usage and preventing DoS attacks that could overwhelm the network. It is important to note that different EVM instructions have varying gas costs. Some operations, such as ADD, AND, and POP, have relatively low gas costs, while others, like SSTORE, incur higher gas costs. This differentiation encourages efficient and responsible use of gas resources. 
\end{itemize}

\subsubsection{Use of Low-level Function (VE30, VE31, VE32, VE33)} 
Solidity's low-level functions, such as \textit{call}, \textit{transfer}, \textit{send},\textit{mstore} and \textit{abi.encodePacked}, provide users with control and flexibility when interacting with smart contracts.  However, improper use of these low-level functions can introduce unexpected behavior and vulnerabilities into the contract's program logic. These vulnerabilities include \textit{unchecked send (VE30)}, \textit{arbitrary jump with function type variable (VE31)}, \textit{hash collisions (VE32)}, and \textit{message call with hardcoded gas amount (VE33)}. To mitigate these vulnerabilities, it is crucial to carefully review and validate the usage of low-level functions, handle exceptions appropriately, account for potential changes in gas costs, and implement robust input validation and verification mechanisms.

\begin{itemize}
  \item \textit{Unchecked send (VE30)} This vulnerability is also described as \textit{unchecked low-level call} \textit{unhandled exceptions}, or \textit{exception disorder}. This vulnerability happens when the call fails accidentally or an attacker forces the call to fail.
  In some cases, developers may include code to check the success of the call, but they neglect to handle the exceptions properly. As a result, funds intended for transfer may not reach the intended recipient. This vulnerability stems from the inconsistent exception-handling behavior in Solidity, which can lead to unexpected outcomes if not handled correctly. 
\end{itemize}


\subsubsection{Improper Behavioral Workflow (VE34, VE35, VE36, VE37)} 
It refers to vulnerabilities that arise when the expected order or sequence of operations within a smart contract is manipulated by malicious users, leading to unexpected states or undesired behavior. These vulnerabilities include \textit{reentrancy (VE34)}, \textit{unexpected Ether balance (VE35)}, \textit{incorrect inheritance order (VE36)}, and \textit{infinite loop (VE37)}.
\textit{Unexpected Ether balance (VE35)} occurs when malicious users intentionally send funds to a contract in a specific manner to disrupt its intended behavior or cause a denial-of-service condition. By manipulating the contract's ether balance, attackers can affect the contract's functionality and, in extreme cases, render it unusable. \textit{Incorrect inheritance order (VE36)} is a vulnerability that arises from the improper ordering of contract inheritance. Malicious users can manipulate the inheritance order to achieve unexpected outcomes and potentially exploit vulnerabilities. \textit{Infinite loop (VE37)} refers to a vulnerability where a contract falls into an infinite loop, leading to non-termination of contract execution. 

\begin{itemize}
  \item \textit{Reentrancy (VE34)} This vulnerability occurs when a contract invokes a function from an external contract, and the called contract has sufficient gas to invoke a callback into the calling contract. This creates a loop where the called contract re-enters the calling contract before the initial invocation is completed. Malicious attackers can exploit this vulnerability to manipulate the execution flow and potentially exploit vulnerabilities present in the contract. It is crucial to carefully review and secure contract interactions to prevent re-entrance attacks and ensure the integrity and security of the smart contract system. 
\end{itemize}


\subsubsection{Consensus Issues (VE38, VE39, VE40, VS4)} In the blockchain, the synchronization of subsequent blocks with the majority of the network relies on following a consensus protocol, such as Proof of Work (PoW) or Proof of Stake (PoS). This consensus protocol allows network participants sufficient time to reach an agreement on which transactions should be included in the blocks. However, the synchronization process itself introduces vulnerabilities that can be exploited by attackers. These vulnerabilities include \textit{transaction order dependency (VE38)} (TOD), \textit{time manipulation (VE39)}, and \textit{bad randomness (VE40, VS4)}.

\begin{itemize}
  \item \textit{TOD (VE38)} This vulnerability, also known as Frontrunning, occurs due to the prioritization mechanism for transactions in blockchain blocks. Miners have the ability to choose which transactions to include in a block and the order in which they are arranged. Since transactions are often prioritized based on gas price, a malicious miner who can see and react to transactions before they are mined may manipulate the transaction order to their advantage. By varying the order of transactions and manipulating the output of the contract, they can manage undesirable outcomes or financial losses for users. 
  \item \textit{Time manipulation (VE39)}  This vulnerability arises when smart contracts rely on the timestamp information from blocks to perform certain functions. In Solidity, the current timestamp can be obtained using \textit{block.stamp} or \textit{now}. However, this timestamp value can be manipulated by miners. If a contract's functionality is dependent on the timestamp, miners can profit by choosing a suitable timestamp to manipulate the contract's behavior. This vulnerability is also referred to as \textit{block values as a proxy for time}.
  \item \textit{Bad randomness (VE40)} This vulnerability refers to vulnerabilities in the generation of random numbers within smart contracts. Random numbers are often used to make decisions or determine outcomes. If the random number generation process is flawed, malicious actors may be able to predict the outcome of the contract and exploit it. One example of bad randomness is the use of a predictable seed value for the random number generator. If an attacker can guess or determine the seed value, they can predict the generated random numbers and manipulate the contract accordingly to their advantage.
\end{itemize}


\subsection{Common Vulnerability Ranking}
\begin{figure}[!ht]
	\centering
	\includegraphics[width=4in]{figure/vulner_num.png}
	\caption{Vulnerabilities Frequency Statistics}
	\label{vulner_num}
\end{figure}

Since the last update of the DASP (Decentralized Application Security Project) in 2018, we have created a new list of the top 10 smart contract vulnerabilities that pose significant risks to the security and functionality of contracts. This list is based on the frequency of occurrence among various analysis tools available at \url{https://sites.google.com/view/sc-analysis-toollist}. Figure \ref{vulner_num} provides statistics on the occurrence of 22 vulnerabilities. The top 10 vulnerabilities are as follows:
    \textit{reentrancy (VE34)}, 
    \textit{arithmetic overflow/underflow (VE25)}, 
    \textit{DoS with block gas limit (VE29)}, 
    \textit{unsafe suicidal (VE17)}, 
    \textit{unsafe delegatecall (VE19)}, 
    \textit{unchecked send (VE30)}, 
    \textit{TOD (VE38)}, 
    \textit{time manipulation (VE39)}, 
    \textit{authorization through tx.origin (VE18)}, 
    and various other vulnerabilities. 
These vulnerabilities have been identified as the most common and high-risk issues that developers should prioritize when assessing the security of their smart contracts.

\textit{Reentrancy} attracts significant attention from researchers due to its difficulty in detection and mitigation. The complex and decentralized nature of smart contracts makes it challenging to ensure the atomic execution of functions and proper handling of reentrant calls. 
\textit{Arithmetic overflow/underflow} is a common issue in software programs, particularly those written in low-level languages. 
Gas usage plays a crucial role in the EVM and the smart contract ecosystem, and accurately estimating the required gas for an operation can be difficult. Given that smart contract applications often handle sensitive data and substantial amounts of value, \textit{DoS with block gas limit} can lead to denial-of-service attacks, data corruption, and financial losses for users. 
\textit{Unsafe suicidal}, \textit{unsafe delegatecall}, and \textit{authorization through tx.origin} vulnerabilities occur when the access control of a smart contract is flawed. Access control determines which entities can interact with the contract and what actions they can perform. These vulnerabilities can result in security risks and potential financial losses when access control is compromised. 
\textit{Unchecked send} occurs when a contract fails to handle exceptions from failed calls appropriately. This vulnerability causes the smart contract to behave unexpectedly and compromises its secure operation. 
\textit{TOD}, \textit{time manipulation}, and \textit{bad randomness} vulnerabilities are related to consensus issues influenced by the blockchain network. Smart contracts are executed on the blockchain and must follow the same transaction order as the underlying blockchain. This means that even if a smart contract is designed to be resistant to \textit{TOD} and \textit{time manipulation}, it can still be vulnerable if the blockchain is not resistant to these issues.

In this section, we performed a comprehensive analysis of the root causes of vulnerabilities in the smart contract domain and introduced a novel classification system to effectively categorize them. Furthermore, we conducted a statistical ranking of the most frequently encountered vulnerabilities based on existing research. These findings offer conclusive and precise answers that effectively address our research question, \textbf{RQ1}, as outlined in Section \ref{survey_methodology}. 

By gaining a good understanding of these vulnerabilities in smart contracts and their ranking, developers can effectively allocate their time and resources, prioritizing the resolution of the most critical security concerns. Additionally, to fully comprehend the potential damage caused by these vulnerabilities, it is crucial to explore the common types of attacks that can exploit them. By carefully examining the relationship between vulnerabilities and attacks, developers can identify potential attack vectors and proactively implement robust measures to mitigate these risks.

\section{Attacks on Smart Contracts}
\label{attacks}
Over the past six years, smart contracts have experienced a substantial increase in daily transaction volume, rising from an average of 23 transactions per day in 2016 to an average of 4 million transactions per day in 2021 \cite{etherdailytx}. Unfortunately, this growth has also led to an increase in the number of attackers exploiting vulnerabilities in smart contracts. Consequently, several high-profile attacks have occurred, resulting in substantial financial losses \cite{Charles}. 

\begin{figure}[!ht]
  \centering
  \includegraphics[width=5in]{figure/attack_timeline.png}
  \caption{Several high-profile attacks from 2016 to 2021}
  \label{fig_attack_timeline}
\end{figure}

One of the most notable attacks on smart contracts was the DAO attack in 2016, which led to millions of dollars worth of Ether being drained from the organization. Since then, smart contracts has experienced several high-profile attacks from 2016 to 2021, as shown in Figure \ref{fig_attack_timeline}. 
To gain a better understanding of these attacks, we have analyzed and identified eight representative attack patterns. Figure \ref{fig_attack} illustrates the major application domains of smart contracts and the corresponding targeted attack patterns. Additionally, we conduct an examination of the vulnerabilities that contribute to these attacks. Through the classification of vulnerabilities based on known attack patterns, we can identify common weakness that require attention.
Following this analysis, we provide a comprehensive breakdown of each attack, shedding light on the specific vulnerabilities that contribute to these attacks. 

\begin{figure}[!ht]
    \centering
    \includegraphics[width=4.5in]{figure/revised_vul_attack_1.pdf}
    \caption{The relationships between attacks and vulnerabilities}
    \label{fig_attack}
\end{figure}

\subsection{The DAO Attack (A1)}
The DAO (Decentralized Autonomous Organization) was a groundbreaking project launched on the Ethereum blockchain in 2016. It aimed to create a decentralized venture capital fund where participants could invest in projects and vote on funding decisions through the use of smart contracts.  The DAO project gained significant attention and raised over \$150 million USD in funding from investors. However, the DAO's success was short-lived as it fell victim to a critical security vulnerability. In June 2016, an attacker exploited a flaw in the DAO's smart contract code, allowing them to drain approximately \$60 million USD worth of ether from the organization \cite{TheDAOevent}. 
This event, known as the DAO attack, was a significant setback for the Ethereum community and led to a contentious debate about the immutability of blockchain transactions and the need for a hard fork to recover the stolen funds. Ultimately, a hard fork was implemented to create a new version of the Ethereum blockchain that reversed the effects of the hack and returned the stolen funds to their rightful owners. Following the occurrence of The DAO hack, other smart contracts, such as Spankchain \cite{spankchain} and Lendf.me \cite{siliconangle}, also experienced losses due to comparable security vulnerabilities. Atzei et al. \cite{AtzeiBC17} provided two attack contract examples, called Mallory and Mallory2. In Mallory, attacker manipulates the control flow and exploits \textit{reentrancy (VE34)}. In Mallory2, attacker efficiently exploits \textit{arithmetic overflow/underflow (VE25)} and \textit{unchecked send (VE30)} vulnerabilities by using only two calls.


\subsection{Parity Wallet Attack (A2)}
The Parity Wallet attack encompasses two distinct incidents that occurred in 2017. During these incidents, the Parity Multisig Wallet suffered breaches, leading to substantial financial losses. The Parity Wallet is comprised of two components: a library contract and wallet contracts. The library contract contains the essential functions of a wallet, while the wallet contracts act as proxies that delegate calls to the library contract through the delegatecall mechanism. In both incidents, the vulnerabilities enabled the attackers to obtain unauthorized control over the wallets. The first incident is caused by exploiting \textit{unsafe delegatecal (VE19)} and more than 30M USD worth of ether is drained \cite{Palladino2017}. In this incident, the attacker initiated two transactions to manipulate each contract involved. The first transaction was aimed at gaining ownership of the victim's wallet contract.
The second incident on the Parity Multisig Wallet exploited the vulnerability known as \textit{suicide (VE17)} and resulted in the locking of more than \$280 million USD worth of funds \cite{Sergey}. In this incident, the attacker leveraged the delegatecall mechanism to initialize themselves as the owner of the wallet contract, similar to the first incident.


\subsection{Fomo3D Attacks (A3)}
The Fomo3D contract was an Ethereum game where participants could purchase keys using Ether and refer others to the game to earn more Ether. The objective of the game was to be the last participant to purchase a key before the timer expired, thereby winning the entire pot of Ether. The attacker purchased a ticket and then sent multiple transactions with high gas prices in rapid succession, effectively consuming a significant portion of the block's gas limit. This action caused other transactions related to Fomo3D, including the key purchases made by other participants, to be delayed or stuck in a pending state until the timer ran out.
The attackers gain an advantage by exploiting two vulnerabilities: \textit{bad randomness (VE40)} and \textit{DoS with block gas limit (V29)}.


\subsection{KotET Attack (A4)}
The King of the Ether Throne (KotET) contract was also a game contract where participants competed to win the throne and receive all the Ether held in the contract. This game contract was implemented as a contract account on the Ethereum blockchain.
The KotET attack, which occurred in February 2016 \cite{KotET}, exploited two vulnerabilities: \textit{unchecked send (VE30)} and \textit{insufficient gas griefing (VE28)}. In the attack, when the KotET contract attempted to transfer funds to another wallet contract, both contracts required sufficient gas to successfully process the transaction. However, if the wallet contract had insufficient gas, it would fail to complete the payment, resulting in the funds being returned to the KotET contract. Importantly, the KotET contract was not aware of the payment failure, and the latest player would be crowned as the King, while the compensation payment intended for the previous player would not be sent.


\subsection{BatchOverflow Attack (A5)}
The BatchOverflow attack, which occurred in April 2018, targeted the Beauty Ecosystem Coin (BEC) token. The attack exploited \textit{arithmetic overflow/underflow (VE25)} vulnerability to achieve an unauthorized increase in digital assets. This vulnerability resulted in the theft of BEC tokens and a temporary shutdown of the exchange platform \cite{BECevent}. According to the blockchain security firm, PeckShield \cite{peckshield}, the BatchOverflow attack was not limited to the BEC token. According to the blockchain security firm PeckShield, they discovered similar integer overflow vulnerabilities in around 12 other token smart contracts. Some examples include SMT (proxyOverflow), UET (transferFlow), SCA (multiOverflow), HXG (burnOverflow), and others.


\subsection{Frontrunning Attack (A6,A7)}
Both the DEXs attack and the Bancor attack can be categorized as Frontrunning attacks. The concept of Frontrunning is not exclusive to blockchain and is commonly observed in traditional financial markets. Frontrunning involves manipulating financial markets by gaining undisclosed information about transactions beforehand \cite{TorresCS21}. This practice is generally deemed illegal in most countries. 
In the context of blockchain, every transaction is publicly visible in the pending pool before being included in a block. Miners, who have the authority to choose which transactions to include, often prioritize those with higher gas prices. This creates an opportunity for attackers to manipulate the transaction order and maximize their own profits. 
Frontrunning attacks exploit the vulnerability known as \textit{TOD (VE38)}, which is also referred to as the Frontrunning vulnerability. Eskandari et al. \cite{EskandariMC19} categorize this vulnerability into three types of attacks: displacement, insertion, and suppression attacks.


\subsubsection{DEXs Attack (A6)} 
The decentralized exchange (DEX) is an exchange platform built on smart contracts where users can exchange their ERC-20 tokens for ether or other tokens. Frontrunning attacks have become a significant concern in DEXs. These attacks exploit \textit{TOD} vulnerability to manipulate the execution of transactions and gain an unfair advantage in trading. When a searcher detects a transaction that can yield a profit, it attempts to front-run that transaction by submitting its own transaction with higher gas fees to ensure its execution before the victim's transaction. 
These front-running attacks are performed by highly competitive bots that aim to maximize their profits by extracting Miner's Extractable Value (MEV). According to Flashbots \cite{flashbots}, searchers extracted approximately 691 million USD worth of value from Ethereum in January 2023 alone.

\subsubsection{Bancor Event (A7)}
The Bancor ICO is a decentralized exchange platform that allows users to create and trade their own tokens. During the audit of its exchange smart contract, two vulnerabilities were discovered: \textit{TOD (VE38)} and \textit{function default visibility (VE11)}. The attacker exploited the \textit{TOD} vulnerability, which enabled them to execute transactions ahead of others, resulting in a profit of 135,229 USD worth of ether \cite{Bancor}. Fortunately, no real-world attack occurred, but the vulnerability itself was identified and addressed.

\subsection{Honeypots (A8)}
Honeypots in the context of smart contracts are a type of fraudulent scheme that capitalizes on security vulnerabilities while employing deceptive tactics. The fundamental concept of a honeypot involves intentionally creating a smart contract that appears to possess an obvious flaw or vulnerability, enticing potential victims to exploit it for financial gain. However, the contract is deliberately designed in a way that when someone attempts to take advantage of the apparent vulnerability, they end up losing their funds instead. Honeypots manipulate human greed and the desire for quick profits, luring unsuspecting users into falling for the trap. The aim is to present an enticing opportunity for financial gain, while in reality, it is a trap orchestrated by malicious actors. 
According to Torres et al. \cite{TorresSS19}, honeypots typically take advantage of specific vulnerabilities to carry out their deceptive schemes such as \textit{incorrect constructor name (VE6)}, \textit{shadowing state variables (VE10)}, and \textit{incorrect inheritance order (VE36)}.


In this section, we have examined eight common attack patterns and their associated lists of vulnerabilities. This analysis has allowed us to address the second research question, \textbf{RQ2}, as outlined in Section \ref{survey_methodology}, which focuses on how these common attacks exploit vulnerabilities and the consequences of such attacks. 
By studying past attacks, we gain valuable insights into common vulnerabilities and attack patterns. This knowledge can be utilized by smart contract developers to enhance the security of their contracts. While it may be challenging to completely eliminate all attacks on vulnerable smart contracts, there are steps that can be taken to minimize risks and improve contract security. By implementing best practices and employing appropriate security measures, developers can reduce the likelihood of their contracts being exploited by attackers. Building upon our understanding of attack patterns, we can now explore various defense methodologies in the next section.


\section{Defense Methodologies}
\label{defense}
Defense technologies evolve alongside the advancement of attacks, and there is a substantial body of work on security measures for smart contracts \cite{zhou2020ever, ChenPNX20, TorresCS21}. 
Figure \ref{fig_defense} provides an overview of a comprehensive set of research solutions for smart contract defense. Generally speaking, defense strategies for smart contracts can be categorized into two main groups: proactive and reactive. Proactive defense strategies involve taking preventive measures to mitigate attacks before they occur, while reactive defense strategies focus on responding to attacks after they have happened. Proactive defense strategies aim to address known attacks that have been previously identified and studied, with defense mechanisms developed specifically for them. These attacks can be prevented or mitigated using contract analyzers or security enhancements. Reactive defense strategies, on the other hand, pertain to attacks that have not been previously identified or studied, and for which defense mechanisms may not exist. Monitoring contracts play a crucial role in addressing such attacks. However, it is important to note that identifying vulnerabilities and deploying monitoring contracts alone are insufficient for effective smart contract defenses. The repair of vulnerable smart contracts is also a key technology in the defense process \cite{RodlerLKD21, TorresJS22}. While defene encompasses a broader range of strategies and measures, repair serves as a specific action to fix or address vulnerabilities. 
The following section provides a detailed description of the defense methodologies involved.

\begin{figure}[!ht]
	\centering
	\includegraphics[width=5in]{figure/revised_defense_methods_1.pdf}
	\caption{Research ideas for the defense of smart contracts}
	\label{fig_defense}
\end{figure}

\subsection{Contract Analyzer}
Contract analyzers play a crucial role in reducing the risk of vulnerabilities in smart contracts before their deployment. Researchers employ various methodologies to analyze smart contracts, many of which are publicly available under open-source licenses. There are five common methodologies for smart contract analysis, such as formal verification, symbolic execution, fuzzing, intermediate representation, and machine learning.

\subsubsection{Formal Verification}
It is a mathematical-based technique to build and check formal proofs that satisfy a particular property. Formal verification is applied to ensure that software behaves and performs as expected in its specifications and requirements based on large reachable state spaces. Smart contracts are often written in programming languages that are amenable to formal verification \cite{harz2018towards, jiao2020semantic}. For example, the Solidity programming language used for Ethereum smart contracts has a well-defined syntax and semantics that can be precisely modeled and verified. For smart contracts, we distinguish mainly two families of formal verification methods, namely model checking and theorem proving.

\textbf{Model checking} It lists all possible states and checks them individually to confirm whether the contract has the corresponding characteristic. Compared to other programs, smart contracts have some unique characteristics that make model checking particularly well-suited for vulnerability detection \cite{garfatta2021survey, AbdellatifB18, AhrendtBEPPRS19, crincoli2022}. First, model checking is most effective on systems with a finite number of states, and smart contracts are often small enough to be modeled in their entirety. Second, smart contracts are often designed to be deterministic, which makes it easier to construct formal models to accurately capture their behavior. This technique is relevant to check partial specifications early in the design process \cite{NehaiPD18}. Model checkers, such as SIPN \cite{BaiCDH18}, FDR \cite{QuHCWML18}, and NuSMV \cite{NehaiPD18}, are utilized to successfully verify the correctness and necessary properties of smart contracts \cite{QuHCWML18}.  Model checking is typically used to detect specific types of vulnerabilities, such as buffer overflows or integer overflows.

\textbf{Theorem proving} It is a technique that describes the desired properties of a system using mathematical logic and uses a theorem prover to generate proofs that verify these properties based on evidence rules. In the context of smart contracts, theorem proving is used to ensure that a contract satisfies a specific set of properties, such as correctness, safety, or liveness. Since smart contracts are typically deterministic, theorem proving can construct mathematical proofs to demonstrate the satisfaction of these properties. 
Unlike model checking, which is limited to finite systems, theorem proving can handle the verification of infinite systems. This makes it well-suited for analyzing smart contracts that may involve complex interactions and potential infinite behaviors. Several theorem provers, such as Coq and Isabelle/HOL, have been developed to provide formal semantics and support the theorem-proving process for smart contracts. 
For instance, Amani et al. extended the existing definition of the Ethereum Virtual Machine (EVM) into Isabelle/HOL with the consideration of gas, allowing for formal verification of EVM-based smart contracts. However, it's worth noting that formal verification through theorem proving is a semi-automated process that often requires manual interaction. It is commonly used to detect broader classes of vulnerabilities, including logic errors and design flaws, rather than specific instances of vulnerabilities.

\subsubsection{Symbolic Execution}
It systematically explores more possible execution paths simultaneously to trigger deep program errors. This approach does not require a specified input value but abstracts the input values into symbols. From the vulnerability detection perspective, symbolic execution offers developers specific input to the triggered vulnerability, which can be used to confirm or debug. Symbolic execution has the advantage of achieving high test coverage with as few test cases as possible, thereby digging out deep program errors. 
Moreover, symbolic execution is often combined with constraint solving to reason whether an execution path is reachable. However, when the program is large and complex, symbolic execution will generate too many paths which may lead to path explosion. A smart contract can be a maximum of 24KB or it will run out of gas \cite{Ehtereum}. Therefore, symbolic execution is perhaps the most popular approach for smart contracts. Moreover, Z3 SMT solvers are used to check which paths are flexible. 
Oyente \cite{LuuCOSH16} was the first attempt for smart contract verification, using a Control Flow Graph (CFG) representation of the bytecode to identify vulnerabilities like \textit{transaction-ordering dependence}, \textit{timestamp dependence}, \textit{mishandled exceptions} and \textit{reentrancy}. Other tools, like Manticore and Maian, have extended the capabilities of symbolic execution to detect additional vulnerabilities such as \textit{arithmetic overflow/underflow}, \textit{unsafe suicide},  and \textit{unchecked send}. 
However, scalability is a concern with symbolic execution, especially in complex application scenarios. The exploration of deep program paths can be slow and resource-intensive. SMARTEST uses symbolic execution to  identify vulnerable transaction sequences in smart contracts \cite{SoHO21}. By combining symbolic execution with a language model, SMARTEST can effectively prioritize program paths that are likely to reveal vulnerabilities.


\subsubsection{Fuzzing}
It is a software testing technique that involves executing target programs with a large number of abnormal or random test cases to detect vulnerabilities. It has gained significant attention in both industry and academia \cite{LiJCLLCLWBCL021, HeBATV19} due to its simplicity and practical effectiveness in identifying software vulnerabilities. Major software vendors like Google  \cite{oss-fuzz} and Microsoft \cite{onefuzz} employ fuzzing techniques to uncover vulnerabilities in their products. 
In the context of smart contracts, fuzzing has been utilized as a means of vulnerability detection, although there are relatively fewer works specifically focused on smart contract fuzzing in recent years. ContractFuzzer \cite{0001LC18}, for example, uses the Application Binary Interface (ABI) specification of contracts as input for fuzzing to detect vulnerabilities. It relies on user-provided input seeds. Echidna \cite{echidna}, on the other hand, uses falsified user-defined predicates or Solidity assertions as input seeds, which are then subjected to grammar-based fuzzing to detect vulnerabilities. Harvey \cite{WustholzC20} incorporates a prediction component to generate new input seeds for gray box fuzzing. sFUZZ \cite{NguyenP0L020} adopts an adaptive strategy to select input seeds, which are then fed into the prominent fuzzer AFL (American Fuzzy Lop) \cite{AFL}.

However, these methods are more effective in finding shallow bugs and less effective in identifying bugs that lie deep in the execution flow. This limitation is due to the heavy reliance on input seeds in fuzzing. An alternative approach that has shown promising results in traditional programs is hybrid fuzzing, which combines fuzzing with symbolic execution. 
ILF (Imitation Learning Fuzzer)  \cite{HeBATV19} addresses this limitation by using a symbolic execution expert to generate a large number of training sequences, which are then fed into imitation learning prior to fuzzing. ILF achieves improved coverage and performs well on both large and small contracts. However, ILF is limited to the contracts used for imitation learning in its training phase. 
ConFuzzius \cite{TorresIGS21}, on the other hand, leverages lightweight symbolic execution to analyze execution traces and employs a constraint solver to obtain input seeds for the fuzzer. This approach enhances the effectiveness of fuzzing by incorporating symbolic execution-based analysis. 
By combining fuzzing with symbolic execution or other complementary techniques, researchers aim to enhance the effectiveness of vulnerability detection in smart contracts, addressing both shallow and deep vulnerabilities.


\subsubsection{Intermediate Representation}
To accurately analyze smart contracts, some researchers also explore converting contracts into an intermediate representation (IR) with highly semantic information, which is more suitable for the analysis and detection of common security issues. Different from formal verification, IR relies on semantic-based transformation. The analysis process can be divided into four stages: lexical analysis, syntax analysis, semantic analysis, and transformation.
The lexical analysis uses Scanner to check whether the input code is a combination of several legitimate words; the syntax analyzer checks whether the combination of these legitimate words meets grammatical rules; semantic analysis checks whether semantics are reasonable; the transformer converts source code or bytecode into machine code, such as XML. Then, the analyzer detects vulnerabilities through specific methods. 

Slither \cite{FeistGG19} transfer contracts to its internal representation language (SlithIR) which uses Static Single Assignment (SSA) form to facilitate the computation of code analyses. 
EthIR \cite{AlbertGLRS18} based on Oyente translates CFGs to a rule-based representation (RBR) of the bytecode.
Smartcheck \cite{TikhomirovVITMA18} directly translates source code into an XML-based IR and then checks it against XPath patterns.  
MadMax \cite{GrechKJBSS18} based on the Vandal \cite{abs-1809-03981} decompiler translates EVM bytecode to a structured IR to check gas-related Vulnerabilities.
NeuCheck \cite{LuWZSE21} employs the Solidity parser ANTLR to complete the transformation from source code to an IR (XML parse tree). VRust \cite{CuiZGT022} successfully translates Rust source code of Solana smart contracts into Mid-level IR.


However, there are two challenges in IR analysis: (1) Because of semantic heterogeneity, it is unavoidable to produce semantic missing during the security analysis; (2) IR takes more processing time.


\subsubsection{Machine Learning}
It has achieved encouraging results in the field of program security \cite{ZhengGWLXLC20, ChakrabortyKDR22, hanif2021rise}. Compared with the above methods, machine learning combines static analysis and dynamic detection, so as to solve the problems of the high false negative rate of static analysis and low code coverage of dynamic analysis. Moreover, machine learning has good scalability and adaptability for novel vulnerabilities type.

There exist a few prior works that use machine learning methods to analyze smart contracts. Tann et al. \cite{abs-1811-06632} introduced a long short-term memory (LSTM) \cite{ZhengGWLXLC20} model to handle the semantic representations of smart contract opcode to detect contract security threats. Their model can achieve higher detection accuracy than symbolic execution analyzer Maian \cite{NikolicKSSH18}, where both are based on the same vulnerabilities taxonomy. Qian et al. \cite{QianLHZW20} applied the bi-directional long short-term memory with the attention (BLSTM-ATT) mechanism in their sequential model to \textit{reentrancy} detection. This framework transforms source code into contract snippets and feeds the sequential model with feature vector representations parsed from these snippets.  
Zhuang et al. \cite{ZhuangLQLWH20} proposed a degree-free graph convolutional neural network (DR-GCN) and a novel temporal message propagation network (TMP) detect vulnerabilities. The source code of a contract is converted to a contracted graph and then to the normalized graphs by a node elimination process. The normalized graphs are fed to DR-GCN and TMP for vulnerability modeling and detection.
ContractWard \cite{WangSXLWS21} trains its machine learning model on bigram features extracted from the opcodes of the compiled smart contract.
SmartMixModel \cite{ShakyaMHMC22} extracts high-level syntactic features from source code as well as low-level bytecode features from the smart contract. And then these features are trained on a machine learning model.

Some researchers also proposed that machine learning will be more efficient if it is in conjunction with fuzzing. 
SoliAudit \cite{LiaoTHT19} employs machine learning to detect known vulnerabilities without expert knowledge or predefined patterns and fuzzing to identify potential weaknesses. However, there is no correlation between the two methods, and fuzzing is more of a complement to machine learning in terms of the detection of vulnerabilities.
ILF \cite{HeBATV19} uses an imitation learning model to learn a fuzzer from training sequences.  


\subsection{Security Enhancement}
Before deployment, some measures can be done to enforce the security of smart contracts. Gas and data are the most important factors for smart contracts. Thus, if gas cost and data privacy have been carefully checked before being deployed, the security of smart contracts will be enhanced. There are some methods for gas cost and data privacy.

\subsubsection{Gas Estimation and Optimization}
Gas is one of the most important mechanisms in EVM to assign a cost to the execution of an instruction. This mechanism can effectively prevent resource abuse (especially DoS attack) and avoid ``infinite" loops \cite{ChenLWCLLAZ17}. When issuing a transaction, the sender needs to specify a gas limit and a gas price before submitting it to the network. Gas represents much more than just the cost of processing transactions on the Ethereum network. A smart contract is capable of running various applications, allowing it to form a decentralized web.
As such, while gas could technically be described as ``transaction fees", it should be used with caution. Referring to Section \ref{vulnerability}, there are several gas-related vulnerabilities. 
The EVM throws an exception when a transaction run out of gas during its execution, and then EVM will immediately return to the former state.

To prevent this, many Ethereum wallets, such as Metamask \cite{METAMASK}, can statically estimate the cost of a transaction before it is executed.
However, there are many operations that are difficult to estimate during executions. In order to address this problem, some researchers proposed some methods that estimate gas and optimize smart contracts \cite{AlbertGRS19,ashraf2020gasfuzzer, danielius2020}. 
Albert et al. proposed an automatic gas analyzer Gastap that infers gas upper bounds for all its public functions \cite{AlbertGRS19}. Gastap requires complex transformation and analysis of the source code that includes several key techniques: Oyente \cite{LuuCOSH16}, EthIR \cite{AlbertGLRS18}, Saco \cite{AlbertAFGGMPR14} and Pubs \cite{AlbertAGP08}.
Gasol \cite{AlbertCGRR20}, as an extension of Gastap, introduces an automatic optimization of the selected function that reduces inefficient gas consumption. Li et al. \cite{ChenFLZLLXCZ21} developed a GasChecker tool that identifies the gas-inefficient code of smart contracts. They summarized ten gas-inefficient programming patterns that assisted users in better tailoring contracts to avoid gas waste.
Gas Gauge \cite{abs-2112-14771} can automatically estimate the gas cost for the target function and the loop-bound threshold. Besides, Gas Gauge can find all the loops and furnish the gas-related instances to help developers with suggestions.
Li et al. \cite{LiNCYH20} estimated the gas for new transactions by learning the relationship between the historical transaction traces and their gas costs.

\subsubsection{Data Privacy-Preserving}
In addition to gas concerns, another major concern for the smart contract is private data \cite{KalodnerGCWF18}. Since the openness and transparency of public blockchains, the privacy overlay feature on the chain is absent. This not only leads to some security issues but also prevents their wider adoption. 
It is problematic for applications that handle sensitive data such as voting schemes \cite{McCorrySH17}, electronic medical records\cite{LiuLYZDG18}, or crowdsensing \cite{PerezZ22}. 

A promising approach to address private data is to design new blockchain infrastructures supporting private data \cite{KosbaMSWP16}. Several blockchain infrastructures have been proposed to support private data by adding trust third-party assumptions, such as Hawk \cite{KosbaMSWP16}, Arbitrum \cite{KalodnerGCWF18}, Ekiden \cite{ChengZKHHJJ0S19}, and Town Crier \cite{zhang2016town}.
Hawk \cite{KosbaMSWP16} helps programmers write a private smart contract without any specialists and generates an efficient cryptographic protocol for interacting with other parties. Arbitrum \cite{KalodnerGCWF18} helps programmers create a private smart contract by following the specific behavior of a virtual machine (VM) and allows honest parties to advance the VM state on-chain. Both Hawk and Arbitrum rely on trusted managers, which may be instantiated with trusted computing hardware or a multi-party computation among the users themselves.
Ekiden \cite{ChengZKHHJJ0S19} processes a smart contract over private data off-chain in trusted execution environments (TEEs) and provides safe interaction among smart contracts on-chain.

An alternative approach is hiding it with cryptographic primitives, such as zero-knowledge proofs. Hawk \cite{KosbaMSWP16} relies on zero-knowledge proofs to enforce the correctness of contract execution and money conservation on the chain.
Steffen et al. \cite{SteffenBGMTV19} proposed zkay contracts containing private data and utilized non-interactive zero-knowledge (NIZK) proof to protect private data. To be more executable on blockchains, zkay contracts are transferred into contracts equivalent in terms of privacy and functionality.
Zapper \cite{SteffenBV22} also leverages NIZK proof to correct state updates without revealing private information and an oblivious Merkle tree construction to hide the sender and receiver.


\subsection{Runtime Monitoring}
The capability to deploy smart contracts is one of the most important features of blockchains. Once deployed on a blockchain, smart contracts become immutable, including any vulnerabilities they may have. While tools are available for scanning smart contracts prior to deployment, they often have limited scopes and may not detect all vulnerabilities, leaving room for unknown runtime attacks. As a solution to enhance the security of post-deployment smart contracts, runtime monitoring techniques analyze and monitor the runtime behavior of smart contracts, providing higher coverage and precision in detecting attacks compared to pre-deployment analysis alone \cite{ellul2018runtime}. Runtime monitoring can be categorized into transaction monitoring and state monitoring. 

\subsubsection{Transactions Monitoring}
As Ethereum can be seen as a transaction-based state machine, a transaction contains much information aiming to change the blockchain state.  This information also can be used to detect and prevent attacks.
ECFChecker \cite{GrossmanAGMRSZ18} examines transactions to determine if they exhibit the characteristics of a reentrancy attack, where a contract can be called recursively before previous invocations have been completed.
Sereum \cite{RodlerLKD19} aims to prevent reentrancy attacks by employing taint tracking techniques. It tracks the flow of data from storage variables to control-flow decisions, helping identify potential vulnerabilities \cite{hu2021transaction, 0002CLLGZLZCHTL20}. Both ECFChecker and Sereum rely on modified versions of Ethereum Virtual Machine (EVM) and primarily focus on detecting reentrancy attacks.
In contrast, ÆGIS \cite{TorresBNJ19, TorresBNPJM20} takes a broader approach by providing an extensible framework for detecting new vulnerabilities in smart contracts. It maintains attack patterns and reverts transactions that match these patterns, thereby enhancing security. The framework allows for the storage and voting of new attack patterns through a smart contract, enabling the community to actively contribute to the detection and prevention of novel attacks. SODA \cite{0002CLLGZLZCHTL20} based on a modified EVM-based client provided a platform for developing various applications to detect malicious transactions in real time. SODA has been integrated into popular blockchains that support the EVM, increasing its accessibility and usability.
Horus \cite{ZhouQTLG21} leverages transaction graph-based analysis to identify the flow of stolen assets. By examining the transaction graph, Horus traces the movement of assets and detect potential theft or unauthorized transfers.


\subsubsection{State Monitoring}
In addition to transactions, blockchain state variables can provide valuable information about the real-time status of a smart contract. Monitoring and analyzing the state variables can be an effective approach to detecting and preventing attacks.
Solythesis \cite{LiCL20} allows users to instrument user-specified invariants into smart contract code. These invariants represent specific conditions that should always hold true during the execution of the contract. By tracking the transactions that violate these invariants, Solythesis can efficiently enforce powerful monitoring of the contract's state. This approach helps identify abnormal or unexpected changes in the state variables and allows for proactive detection of potential vulnerabilities.
ContractGuard \cite{WangHXZC20} takes a similar approach but introduces the concept of an intrusion detection system (IDS) to monitor the behavior of a deployed smart contract. The IDS continuously monitors the state variables and looks for abnormal or suspicious behaviors that deviate from expected patterns. 
% If an abnormal state is detected, ContractGuard can roll back all the changes to the contract state and notify the relevant users about the potential intrusion.


\subsection{Post-deployment Repair}
Ideally, smart contracts should be deployed with the highest possible level of security. However, this presents a challenge not only for smart contracts but also for all software programs. Furthermore, the underlying blockchain technology ensures immutability, meaning that the past cannot be altered. Consequently, updating the code of a deployed contract or addressing vulnerabilities becomes unfeasible. Apart from fixing vulnerabilities, there are numerous other reasons to modify contract code, including adapting business logic or enhancing functionality.

\subsubsection{Contract Migration}
In many cases, an available approach is to deploy a new instance of the contract and migrate old contract data to it. This approach is called contract migration and has been successfully implemented in various token migration events, such as Augur \cite{Augur}, VeChainThor\cite{vechain}, and TRON \cite{TRON}. A contract migration process typically involves two steps: data recovery and data writing \cite{josselinfeist}. 
During data recovery, the data from the old contract is extracted, including both public and private variables. While retrieving public variables is relatively straightforward, handling private variables and mappings can be more complex. Special measures need to be taken to ensure the integrity and consistency of the migrated data. To address the challenges associated with data migration, some developers employ separate contracts to store and manage the data \cite{ProxyPatterns}. This can simplify the migration process by separating the logic from the data storage. By migrating the logic to a new contract and keeping the data in a separate contract, the migration becomes more manageable.
It's important to note that contract migration procedures can be costly, especially for token-based contracts with a large number of accounts \cite{josselinfeist}. Migrating data for a large number of accounts requires careful planning and execution to avoid potential issues and ensure the accuracy and security of the migrated data.


\subsubsection{Upgradable Contracts}
Another promising approach to modify smart contracts code is upgradable contract \cite{josselinfeist2, ZhengGHG21, ProxyPatterns}. This approach involves incorporating an upgradability mechanism into smart contracts, allowing for seamless upgrades without the need for data migration or external reference updates. An upgradeable contract can be implemented by setting up a proxy pattern architecture \cite{ZhengGHG21}.
In this pattern, one contract is split into two separate contracts, one for logic execution and another for data storage. A logic contract stores no state but implements all the business logic; a proxy contract holds all funds and all internal states, but does not implement any business logic. 
Instead of replacing the entire contract, only the logic contract needs to be modified or replaced. The proxy contract remains intact, and all existing data and external references are maintained. This allows for seamless upgrades without disrupting the contract's functionality or requiring data migration. Several approaches have been proposed to realize upgradable contracts.
Zeppelin \cite{ProxyPatterns} proposed three proxy pattern architectures: inherited storage, eternal storage, and unstructured storage. All three patterns rely on the DELEGATECALL instruction, thus these patterns are also called delegate-proxy patterns. 
For example, EVMPatch \cite{RodlerLKD21} uses a proxy pattern to enable the quick and cost-effective upgrade of smart contracts. 
EVMPatch\cite{RodlerLKD21} utilizes the delegate-proxy pattern to enable quick and cost-effective upgrades of smart contracts. It provides a framework for patching vulnerabilities or adding new features to deployed contracts without requiring data migration or disrupting the contract's functionality.


In this section, we have thoroughly examined and categorized various defense methodologies aimed at preventing or mitigating the damage caused by smart contract attacks. The findings provide a clear and comprehensive response to the third research question, \textbf{RQ3}, as outlined in Section \ref{survey_methodology}. However, it is important to acknowledge the existing dearth of empirical evidence that would enable an evaluation of the effectiveness of these methodologies and their associated tools. Despite their claims of superiority, it is imperative for researchers and developers to undertake empirical studies to ascertain which methods are truly practical and beneficial in real-world scenarios.


\section{Evaluation}
\label{evaluation}
Our objective is to provide a comprehensive overview of the state-of-the-art automated analysis tools available for smart contracts. Given that the field of smart contract analysis is relatively new and rapidly evolving, it can be challenging to stay updated with the latest developments and understand the strengths and limitations of existing tools. Therefore, we have conducted an extensive review of the literature and websites to compile a list of the most promising analysis tools for smart contracts. Additionally, to address RQ4 and RQ5, which focus on evaluating the performance of existing tools, we propose a systematic evaluation approach that involves selecting appropriate tools, utilizing datasets of smart contracts, defining criteria for the assessment, and conducting experiments.

\subsection{Experimental Setup}

\subsubsection {Tool Selection} 
Vulnerability-detecting tools are the most widely used method to assist developers in discovering security vulnerabilities in smart contracts. A wide range of analysis tools has been developed for this purpose, and we have compiled a list of 82 such tools based on academic literature and internet sources. Table in \url{https://sites.google.com/view/sc-analysis-toollist} presents the list of tools, along with key properties such as venue, methodology, input object, open-source link, and vulnerability ID. However, despite the abundance of vulnerability-finding tools, only a few papers have proposed a systematic approach to evaluate their effectiveness \cite{GhalebP20,DurieuxFAC20}. To address this gap and answer RQ5, we adopted the four selection criteria below proposed by Durieux et al. \cite{DurieuxFAC20}. 

\begin{enumerate}[C1:]
  \item {[Available]} The tool must be publicly available and accessible for download or installation with a command line interface (CLI).
  \item {[Funcctionality]} The tool must be designed especially for smart contracts and must have the ability and detect vulnerabilities. This excludes constructing artifacts like control flow graphs.
  \item {[Compatibility]} The tool must be the source code of the smart contract. That excludes tools that only consider EVM bytecode.
  \item {[Documentation]} The tool must provide comprehensive documentation and user guides.
\end{enumerate}

These criteria were utilized to filter our list of tools, resulting in the identification of 12 tools that meet the requirements: ConFuzzius \cite{ConFuzzius}, Conkas \cite{conkas}, Maian \cite{MAIAN}, Manticore \cite{manticore}, Mythril \cite{mythril}, Osiris \cite{Osiris}, Oyente \cite{oyente}, Securify \cite{securify2}, sfuzz \cite{sfuzz}, Slither \cite{slither}, Smartcheck \cite{smartcheck}, and solhint \cite{solhint}.

\subsubsection{Dataset Construction}
One important issue when evaluating analysis tools is how to obtain a sufficient number of vulnerable smart contracts. Although there are many open-source analysis tools available, comparing and reproducing them can be challenging due to the lack of publicly available datasets. Most analysis tools primarily check or detect only some of the well-known Ethereum smart contract vulnerabilities. To evaluate the effectiveness of any analysis tool, it is crucial to establish a standard benchmark. While several researchers have published their datasets (Durieux et al., 2020; Ghaleb et al., 2020; ConFuzzius), these datasets have limitations such as small sample sizes or an uneven distribution of vulnerable contracts. 
To address this limitation and partially answer RQ5, we created an annotated dataset consisting of 110 contract test cases. These cases are divided into 11 sub-datasets, with 10 sub-datasets containing known vulnerabilities corresponding to the top 10 categories mentioned in Section \ref{vulnerability}, and one sub-dataset representing correct contracts. These contracts vary in code sizes and encompass a wide range of applications. We have made our benchmark publicly available on GitHub: \url{https://github.com/bit-smartcontract-analysis/smartcontract-benchmark}.

\subsubsection{Hardware Configuration}
We obtained the most recent versions of the selected analysis tools from their respective public GitHub repositories, except for version 0.3.4 of the Manticore tool. The tools were executed on a 64-bit Ubuntu 18.04.5 LTS machine with 32 GB of memory and an Intel(R) Core(TM) i5-13400 CPU (6 cores clocked at 4.4 GHz). To address RQ4, we evaluated the effectiveness of each tool in terms of accuracy, performance, solidity version, and category coverage.


\subsection{Experimental Results}

\subsubsection{Accuarcy}
We first measure the accuracy of the selected tools in finding vulnerabilities.  There is currently no widely accepted standard or systematic method to evaluate the accuracy of analysis tools for smart contracts in finding security vulnerabilities. 
We test the selected tools on our benchmark, and the results are summarized in Table \ref{test-TP}. This table presents an overview of the strengths and weaknesses of the selected tools across the Top 10 categories in Section \ref{vulnerability}.  Each tool is represented as a column, while the different vulnerability categories are listed as rows. The numbers in each cell show
the number of true positives identified by each tool for each vulnerability category. The format used to report the results is ``x/y", where x is the number of vulnerabilities correctly detected and y is the total number of cases tested for that category.
It is notable that none of the tested tools was able to identify all categories of vulnerabilities. Mythril and ConFuzzius outperformed the other tools by detecting 8 categories of vulnerabilities among all the tools. 

Table \ref{test-TP} shows that different tools perform better at identifying certain vulnerability categories than others. For example, Conkas and Slither perform well in detecting \textit{reentrancy} vulnerabilities, while Smartcheck excels at identifying \textit{time manipulation} issues. The table provides valuable information for developers and security analysts in selecting the most effective tools for detecting specific types of vulnerabilities in smart contracts.

The last row of Table \ref{test-TP} shows the total number of true positives detected by each tool out of a total of 100 test cases. This data gives an insight into the accuracy of each tool's performance in identifying Top 10 categories.
Mythril has the highest total number of 54 true positives, while Conkas has a high number of 44 true positives and ConFuzzius has 42 true positives. The interesting thing is that all the top 3 tools combine different analysis methods. For example, Mythril uses symbolic execution, SMT calculation, and taint analysis. The combination of different analysis methods can lead to a more effective analysis, as it can leverage the strengths of each method to overcome their respective weaknesses. 


\begin{table}[ht]
  \renewcommand\arraystretch{2}
  \scriptsize
  \centering
  \caption{True-positives identified per category by each tool}
  \begin{tabular}{|p{1.95cm}<{\raggedright}|p{0.6cm}<{\raggedright}|p{0.6cm}<{\raggedright}|p{0.5cm}<{\raggedright}|p{0.5cm}<{\raggedright}|p{0.6cm}<{\raggedright}|p{0.5cm}<{\raggedright}|p{0.5cm}<{\raggedright}|p{0.5cm}<{\raggedright}|p{0.5cm}<{\raggedright}|p{0.5cm}<{\raggedright}|p{0.5cm}<{\raggedright}|p{0.5cm}<{\raggedright}|}\bottomrule
  \textbf{Category} & \rotatebox{90}{\thead{ConFuzzius}} & \rotatebox{90}{\thead{Conkas}}  & \rotatebox{90}{\thead{Maian}} & \rotatebox{90}{\thead{Manticore}} & \rotatebox{90}{\thead{Mythril}}  & \rotatebox{90}{\thead{Osiris}}  & \rotatebox{90}{\thead{Oyente}}& \rotatebox{90}{\thead{Securify}} & \rotatebox{90}{\thead{sFuzz}}  & \rotatebox{90}{\thead{Slither}} & \rotatebox{90}{\thead{Smartcheck}}  & \rotatebox{90}{\thead{solhint}}   \\ \hline 
  Reentrancy & 9/10 & 10/10  & 0/10 & 8/10 & 9/10 & 7/10 & 7/10 & 8/10 & 6/10 & 9/10 & 8/10 & 0/10 \\ \hline
  Arithmetic & 7/10 & 9/10  & 0/10 & 7/10 & 7/10 & 9/10 & 9/10 & 0/10 & 6/10 & 0/10 & 1/10 & 0/10 \\ \hline
  GaslessSend & 0/10 & 0/10  & 0/10 & 0/10 &0/10 & 0/10 & 0/10 & 0/10 & 3/10 & 0/10 & 7/10 & 0/10 \\ \hline
  UnsafeSuicidal & 4/10 & 0/10  & 4/10 & 0/10 &3/10 & 0/10 & 0/10 & 0/10 & 0/10 & 6/10 & 0/10 & 4/10 \\ \hline
  UnsafeDelegatecall & 1/10 & 0/10 & 0/10 & 4/10 &6/10 & 0/10 & 0/10 & 0/10 & 5/10 & 7/10 & 0/10 & 0/10 \\ \hline
  UncheckedSend & 9/10 & 10/10  & 0/10 & 2/10 & 9/10 & 0/10 & 0/10 & 0/10 & 0/10 & 0/10 & 8/10 & 6/10 \\ \hline
  TOD & 2/10 & 7/10  & 0/10 & 0/10 &0/10 & 0/10 & 2/10 & 2/10 & 0/10 & 0/10 & 9/10 & 0/10 \\ \hline
  TimeManipulation & 8/10 & 8/10  & 0/10 & 7/10 &6/10 & 2/10 & 0/10 & 0/10 & 1/10 & 8/10 & 0/10 & 10/10 \\ \hline
  tx.origin & 0/10 & 0/10  & 0/10 & 3/10 &6/10 & 0/10 & 0/10 & 0/10 & 0/10 & 0/10 & 0/10 & 9/10 \\ \hline
  BadRandomness & 2/10 & 0/10  & 0/10 & 0/10 &8/10 & 0/10 & 0/10 & 0/10 & 0/10 & 0/10 & 0/10 & 5/10 \\ \hline
  Total & \textbf{42/100} & \textbf{44/100} & 4/100 & 31/100 & \textbf{54/100} & 18/100 & 18/100 & 10/100 & 21/100 & 30/100 & 33/100 & 34/100 \\ \hline
\end{tabular}
\label{test-TP}
\end{table}

Table \ref{test-accuracy} displays the accuracy of each tool by calculating the number of true positives and false negatives in each vulnerability category. The first row lists the selected tools, and each subsequent row shows the true positives and false negatives for each tool in that category. We obtained the false negatives from 10 correct test cases. The last row presents the accuracy of each tool, which is calculated as the ratio of the total number of true positives to the sum of true positives and false positives. For example, ConFuzzius has 42 true positives out of 100 test cases, and it also has 7 false negatives out of 10 test cases. Therefore, its accuracy is calculated as (42+7)/(100+10)=45\%. It is evident from the table that Mythril outperforms the other tools in terms of correctly identifying vulnerabilities in the tested smart contracts with an accuracy of 58\%.

\begin{table}[ht]
  \renewcommand\arraystretch{2}
  \scriptsize
  \centering
  \caption{Accuarcy of each tool}
  \begin{tabular}{|p{1.95cm}<{\raggedright}|p{0.5cm}<{\raggedright}|p{0.5cm}<{\raggedright}|p{0.5cm}<{\raggedright}|p{0.5cm}<{\raggedright}|p{0.5cm}<{\raggedright}|p{0.5cm}<{\raggedright}|p{0.5cm}<{\raggedright}|p{0.5cm}<{\raggedright}|p{0.5cm}<{\raggedright}|p{0.5cm}<{\raggedright}|p{0.5cm}<{\raggedright}|p{0.5cm}<{\raggedright}|}\bottomrule
  & \rotatebox{90}{\thead{ConFuzzius}} & \rotatebox{90}{\thead{Conkas}} & \rotatebox{90}{\thead{Maian}} & \rotatebox{90}{\thead{Manticore}} & \rotatebox{90}{\thead{Mythril}}  & \rotatebox{90}{\thead{Osiris}}  & \rotatebox{90}{\thead{Oyente}}& \rotatebox{90}{\thead{Securify}} & \rotatebox{90}{\thead{sFuzz}}  & \rotatebox{90}{\thead{Slither}} & \rotatebox{90}{\thead{Smartcheck}}  & \rotatebox{90}{\thead{solhint}}   \\ \hline 
  True-positives & 42/100 & 44/100  & 4/100 & 31/100 & 54/100 & 18/100 & 18/100 & 10/100 & 21/100 & 30/100 & 33/100 & 34/100 \\ \hline
  False-negatives & 7/10 & 6/10  & 10/10 & 10/10 & 10/10 & 8/10 & 8/10 & 10/10 & 9/10 & 10/10 & 10/10 & 10/10 \\ \hline
  Accuarcy & 45\% & 46\% & 13\%  & 37\% & \textbf{58\%} & 24\% & 24\% & 18\% & 27\% & 36\% & 39\% & 40\% \\ \hline
\end{tabular}
\label{test-accuracy}
\end{table} 


\subsubsection{Performance}
Execution time is another crucial factor to consider when evaluating the effectiveness of a tool because the longer it takes for a tool to run, the less efficient it is in terms of time. To assess performance, we calculated the execution time for each tool, including the average time and total time, as indicated in Table \ref{test-performance}. The total time represents the cumulative execution time for all the test processes, while the average time is obtained by dividing the total time by the number of successful test cases. It is important to note that unsuccessful test cases have been excluded from the calculation of the average time. Slither and solhint demonstrate the shortest average execution time, completing a test case in just 1 second, while ConFuzzius and sFuzz exhibit the longest average execution time, requiring approximately 18 minutes per test case. The total time taken by each tool also varies significantly, with ConFuzzius consuming the longest duration of over 26 hours, while Slither proves to be the most time-efficient, requiring only 2 minutes.  It is worth mentioning that the longer average execution times of ConFuzzius and sFuzz could be attributed to their utilization of fuzzing methods, which inherently entail lengthier execution times. 
In summary, Table \ref{test-performance} offers valuable insights into the performance of different analysis tools, aiding software developers in selecting the most suitable tool for their testing requirements.

\begin{table}[ht]
  \renewcommand\arraystretch{2}
  \scriptsize
  \centering
  \caption{Execution time for each tool}
  \begin{tabular}{|p{1.4cm}<{\raggedright}|p{0.6cm}<{\raggedright}|p{0.55cm}<{\raggedright}|p{0.55cm}<{\raggedright}|p{0.55cm}<{\raggedright}|p{0.55cm}<{\raggedright}|p{0.55cm}<{\raggedright}|p{0.55cm}<{\raggedright}|p{0.55cm}<{\raggedright}|p{0.55cm}<{\raggedright}|p{0.55cm}<{\raggedright}|p{0.55cm}<{\raggedright}|p{0.55cm}<{\raggedright}|}\bottomrule
  & \rotatebox{90}{\thead{ConFuzzius}} & \rotatebox{90}{\thead{Conkas}}  & \rotatebox{90}{\thead{Maian}} & \rotatebox{90}{\thead{Manticore}} & \rotatebox{90}{\thead{Mythril}}  & \rotatebox{90}{\thead{Osiris}}  & \rotatebox{90}{\thead{Oyente}}& \rotatebox{90}{\thead{Securify}} & \rotatebox{90}{\thead{sFuzz}}  & \rotatebox{90}{\thead{Slither}} & \rotatebox{90}{\thead{Smartcheck}}  & \rotatebox{90}{\thead{solhint}}   \\ \hline 
  Average Time & {0:18:9} & 0:0:48 & 0:0:57 & 0:11:2 & 0:4:17 & 0:1:32 & 0:0:4  & 0:0:27 & 0:18:0 & \textbf{0:0:1} & 0:0:3 & \textbf{0:0:1}  \\ \hline
  Total Time & {26:01:4} & 1:13:38 & 1:35:50 & 9:33:40  & 6:29:40 & 1:47:31 & 0:5:38 & 0:45:55 & 18:0:19 & \textbf{0:2:15} & 0:4:42 & 0:2:40\\ \hline
\end{tabular}
\label{test-performance}
\end{table} 

\subsubsection{Effectiveness}
During our experiments, we discovered that aside from accuracy and execution time, there are two additional crucial factors to consider when assessing the overall quality of analysis tools. These factors are version compatibility and category coverage. Version compatibility refers to the ability of a tool to support different versions of the programming language or framework used in smart contracts. It is essential because smart contract development frameworks often undergo updates and introduce new features or changes. An effective analysis tool should be compatible with a wide range of versions to ensure its usefulness and relevance in different development environments. Category coverage refers to the extent to which an analysis tool can detect vulnerabilities across various categories or types. Smart contract vulnerabilities can vary in nature, and different tools may have varying levels of effectiveness in detecting them. A comprehensive analysis tool should have a broad category coverage, capable of identifying vulnerabilities across multiple categories, ensuring a more thorough and comprehensive evaluation of smart contracts. By considering these factors alongside accuracy and execution time, we can obtain a more comprehensive assessment of the overall quality and effectiveness of analysis tools for smart contracts.


\begin{itemize}
  \item{Version compatibility}: The Solidity language is constantly evolving, and new versions may introduce new features, syntax, or changes that may affect the behavior of smart contracts. Tools that cannot analyze contracts written in newer versions of Solidity may be limited in their ability to detect vulnerabilities or may produce inaccurate results, which could lead to false positives. Until January 2023, the latest version of Solidity was 0.8.19. We have assigned a rating value to different Solidity versions based on their compatibility with the tool. The score starts from 1 for version 0.4.x and increases to 5 for version 0.8.x. We use $S_{v}$ to represent this score.
  \item{Category coverage}: The category coverage of each tool is also important in evaluating their effectiveness. This allows the developers to determine if a tool is suitable for detecting specific categories of vulnerabilities in smart contracts. A tool that has a wide category coverage and can detect multiple categories of vulnerabilities may be more effective than a tool that can only detect a limited number of vulnerabilities. We assign a score to each tool based on the categories of vulnerabilities it can detect. For example, if a tool can detect 5 categories of vulnerabilities, you can assign it a score of 5 points. We use $S_{c}$ to represent this score. 
\end{itemize}

Based on the scoring criteria mentioned above, we have determined the score values of $S_{v}$ (version compatibility) and $S_{c}$ (category coverage) as presented in Table \ref{version-category}. The highest compatibility version for each tool is indicated in the second row, which was used to calculate the rating value for each tool.

\begin{table}[ht]
  \renewcommand\arraystretch{2}
  \scriptsize
  \centering
  \caption{The scores of compatibility version ($S_v$) and category coverage ($S_c$) for each tool}
  \begin{tabular}{|p{1.4cm}<{\raggedright}|p{0.55cm}<{\raggedright}|p{0.55cm}<{\raggedright}|p{0.55cm}<{\raggedright}|p{0.55cm}<{\raggedright}|p{0.55cm}<{\raggedright}|p{0.55cm}<{\raggedright}|p{0.55cm}<{\raggedright}|p{0.55cm}<{\raggedright}|p{0.55cm}<{\raggedright}|p{0.55cm}<{\raggedright}|p{0.55cm}<{\raggedright}|p{0.55cm}<{\raggedright}|}\bottomrule
  & \rotatebox{90}{\thead{ConFuzzius}} & \rotatebox{90}{\thead{Conkas}}  & \rotatebox{90}{\thead{Maian}} & \rotatebox{90}{\thead{Manticore}} & \rotatebox{90}{\thead{Mythril}}  & \rotatebox{90}{\thead{Osiris}}  & \rotatebox{90}{\thead{Oyente}}& \rotatebox{90}{\thead{Securify}} & \rotatebox{90}{\thead{sFuzz}}  & \rotatebox{90}{\thead{Slither}} & \rotatebox{90}{\thead{Smartcheck}}  & \rotatebox{90}{\thead{solhint}}   \\ \hline 
  Comp.version & 0.8.x & 0.5.x  & 0.8.x & 0.8.x & 0.8.x & 0.4.21 & 0.4.19  & 0.5.11 & 0.4.24 & 0.6.x & 0.8.x & 0.8.x\\ \hline
  $S_v$ & 5 & 2  & 5 & 5 & 5 & 1 & 1  & 2 & 1 & 3 & 5 & 5 \\ \hline
  $S_c$ & 8 & 5 & 1 & 6 & 7 & 3 & 3  & 2 & 5 & 4 & 5 & 5 \\ \hline
\end{tabular}
\label{version-category}
\end{table} 

It is crucial to obtain a balance between accuracy, average execution time, compatibility version, and category coverage when evaluating the effectiveness of a tool. One common approach to achieving this balance is by using a weighted sum method, where the four factors are assigned different weights based on their relative importance. In this approach, we assign a weight of $\alpha$ to accuracy, $\beta$ to average execution time, $\gamma$ to compatibility version, and $(1- \alpha - \beta - \gamma)$ to category coverage. We then calculate the overall score ($Score$) using the following formula:


\begin{align*}
  Score = \alpha \times A *100  + \beta \times (1/AEX)*100 + \gamma \times S_v + (1- \alpha - \beta - \gamma) \times S_c
\end{align*}
where  $A$ is the accuracy value and $AEX$ is the average execution time.

The scores for each tool can be found in Table \ref{test-score}. The table consists of three sets of scores, each with different weightings for the four measuring attributes: accuracy, execution time, version compatibility, and category coverage. The highest-scoring tool in each row is highlighted in bold. 
In the first row, where the weightings for $A$ (accuracy) and $AEX$ (execution time) are higher, Slither and solhint receive the highest scores. Similarly, in the second and third rows, where the weightings for $S_v$ (version compatibility) and $S_c$ (category coverage) are higher, respectively, Slither and solhint still achieve the highest scores. 
It is important to note that each tool has its own strengths and weaknesses, and the choice of tool should depend on the specific needs and goals of the user. For example, Slither excels in semantic-level analysis, while solhint is better suited for canonical code. This explains why solhint performs well in detecting vulnerabilities related to low-level opcodes such as \textit{call} and \textit{tx.origin}, but may not be able to detect vulnerabilities like \textit{reentrancy} and \textit{unsafe delegatecall} as shown in Table \ref{test-TP}. Maian, on the other hand, differs from other tools as it focuses on vulnerabilities involving a long sequence of invocations of a contract. 

\begin{table}[ht]
  \renewcommand\arraystretch{2}
  \scriptsize
  \centering
  \caption{The overall scores for each tool}
  \begin{tabular}{|p{0.25cm}<{\raggedright}|p{0.25cm}<{\raggedright}|p{0.25cm}<{\raggedright}|p{0.55cm}<{\raggedright}|p{0.55cm}<{\raggedright}|p{0.55cm}<{\raggedright}|p{0.55cm}<{\raggedright}|p{0.55cm}<{\raggedright}|p{0.55cm}<{\raggedright}|p{0.55cm}<{\raggedright}|p{0.55cm}<{\raggedright}|p{0.55cm}<{\raggedright}|p{0.55cm}<{\raggedright}|p{0.55cm}<{\raggedright}|p{0.55cm}<{\raggedright}|}\bottomrule
    $\alpha$ &  $\beta$ &  $\gamma $ & \rotatebox{90}{\thead{ConFuzzius}} & \rotatebox{90}{\thead{Conkas}}   & \rotatebox{90}{\thead{Maian}} & \rotatebox{90}{\thead{Manticore}} & \rotatebox{90}{\thead{Mythril}}  & \rotatebox{90}{\thead{Osiris}}  & \rotatebox{90}{\thead{Oyente}}& \rotatebox{90}{\thead{Securify}} & \rotatebox{90}{\thead{sFuzz}}  & \rotatebox{90}{\thead{Slither}} & \rotatebox{90}{\thead{Smartcheck}}  & \rotatebox{90}{\thead{solhint}}   \\ \hline 
  0.4 &0.4 &0.1 & 19.3 & 19.9  & 6.5 & 16.0 & 24.7 & 10.4 & 18.0  & 9.1 & 11.4 & \textbf{55.1} & 29.9 & \textbf{57.0}  \\ \hline
  0.25 & 0.25 &0.25 & 14.5 & 13.7  & 5.2 & 12.0 & 17.8 & 7.3 & 12.0  & 6.4 & 8.3 & \textbf{35.8} & 20.6 & \textbf{37.5}  \\ \hline
  0.1 &0.1 & 0.4 & 9.7 & 7.6  & 3.9 & 8.1 & 11.0 & 4.1 & 6.0  & 3.8 & 5.1 & \textbf{16.4} & 11.2 & \textbf{18.0}  \\ \hline
\end{tabular}
\label{test-score}
\end{table} 

Overall, the scores presented in Table \ref{test-score} offer valuable insights into the performance of each tool when different weighting schemes are applied. By adjusting the weights assigned to the measuring factors, users can customize the evaluation criteria based on their specific requirements and priorities. This flexibility allows users to emphasize the aspects that are most critical to the evaluation process. By doing so, they can make more targeted decisions when selecting a tool. 
These findings provide a clear response to the fourth research question, \textbf{RQ4}, as outlined in Section \ref{survey_methodology}. The evaluation of the selected analysis tools based on various factors such as accuracy, execution time, version compatibility, and category coverage has shed light on their overall effectiveness. By considering these factors, developers and researchers can make informed decisions about the most suitable tools for their smart contract analysis needs.

In addition, it is worth mentioning that using a combination of analysis tools could be an effective approach to improve the overall quality of code and detect potential vulnerabilities. For instance, solhint can be used to perform grammatical checks and ensure code adherence to standards, while Mythril can identify known vulnerabilities and help prevent attacks. Slither, on the other hand, can provide a deeper analysis of the code and detect some semantic-level issues. By using these tools in combination, developers can ensure that their code is well-written, adheres to standards, and is free from vulnerabilities to the extent possible.

Finally, in order to address the fifth research question, \textbf{RQ5}, as outlined in Section \ref{survey_methodology}, we have employed established criteria for selecting the different analysis tools and developed an annotated dataset comprising 110 smart contracts. The dataset is divided into 11 sub-datasets, with 10 sub-datasets containing the faulty contracts with the top 10 known vulnerabilities discussed in Section \ref{vulnerability}, and one sub-dataset consisting of correct contracts. This dataset serves as a valuable resource for researchers and practitioners to conduct comprehensive evaluations of the capabilities and limitations of various tools. It enables them to make fair judgments when selecting the most appropriate tools for their specific requirements. By providing a standardized and annotated set of smart contracts, the dataset facilitates objective comparisons and assessments, promoting transparency and reliability in smart contract tool evaluation.

\subsection{Threat to validity}
Threats to validity are factors that have the potential to impact the results of an experiment and the validity of its findings. In our research, we have identified two specific aspects that could pose threats to the validity of our study: the categorization of smart contract vulnerabilities and the generality of the evaluation datasets.

One potential threat to the validity of our evaluation is the subjectivity and variation among researchers in evaluating and categorizing vulnerabilities and their associated smart contracts. Different researchers may have diverse perspectives, criteria, and interpretations when assessing the severity and classification of vulnerabilities. This subjectivity can introduce bias and affect the validity of the comparisons presented in the evaluation. 
To mitigate this threat, we have adopted a systematic approach based on industry standards and best practices in Section \ref{vulnerability}. We have thoroughly reviewed and discussed each vulnerability category to ensure a consistent and objective classification. This involved extensive research, consultation with experts, and careful consideration of existing literature. We have also provided clear definitions, criteria, and explanations for each vulnerability category considered in our analysis. 
By providing this transparency and documentation of our evaluation process, we aim to minimize ambiguity and facilitate a more consistent understanding of the vulnerabilities across different researchers and readers.

The generality of the evaluation datasets represents another potential threat to the validity of our research. This threat refers to the extent to which the datasets used for evaluation accurately reflect real-world scenarios and the usage patterns of smart contracts. If the evaluation datasets are limited in scope or fail to encompass the diversity of smart contract applications, the findings and conclusions may lack generalizability. 
To mitigate this threat, we have made significant efforts to address dataset limitations. We have conducted an extensive collection of contract tests from various sources, including publicly available datasets and our own developed test cases. Our dataset consists of 110 contract test cases, which have been carefully selected to cover a wide range of applications and different code sizes. 
By incorporating diverse contract test cases, we aim to provide a more representative evaluation of smart contract vulnerabilities and increase the generalizability of our findings.

While we have taken measures to address these threats, it is important to acknowledge that limitations may still exist. To further enhance the validity of future studies, researchers can focus on refining vulnerability categorization criteria and collecting larger, more diverse datasets that better capture real-world scenarios.


\section{Conclusions}
\label{discussion}

The adoption of smart contract technology is rapidly increasing, leading to significant research efforts focused on enhancing smart contract security. In this survey, we have conducted a comprehensive study on smart contract security, encompassing vulnerabilities, attacks, defenses, and tool support. Our analysis has contributed to novel classifications of common vulnerability types and attack patterns, with a specific focus on the connection between them. Additionally, we have investigated defense methodologies aimed at mitigating the risks associated with these vulnerabilities. 
Moreover, we have conducted experiments using 12 open-source vulnerability-detecting tools and applied weighting-based assessment criteria to evaluate their accuracy, performance, and overall effectiveness. This evaluation provides valuable insights into the capabilities and limitations of the representative tools in the field, helping researchers and practitioners in selecting the most suitable options. 
Additionally, we have created an annotated dataset comprising 110 smart contracts, serving as a standardized benchmark for conducting thorough evaluations of smart contract analysis tools. This dataset facilitates comparative studies and enables researchers to assess the performance of different tools in detecting vulnerabilities and improving overall security. 
These findings emphasize the critical importance of continuous research and development in the field of smart contract security.

However, the smart contract landscape is evolving rapidly, with new functionalities and protocols leading to the emergence of new security vulnerabilities. To make smart contract languages more robust, it is crucial to continue investing in research and development. For instance, there has been a growing interest in using programming languages other than Solidity for smart contract development. Languages like Go and Rust have gained attention due to their stronger syntax and logical soundness, offering potential solutions to address some of the security issues associated with Solidity. 
Furthermore, there is a need for more powerful analysis tools capable of identifying dynamic or logic errors within smart contracts. Existing tools primarily focus on known vulnerabilities and attacks, while effective methodologies for dealing with unknown attacks are still limited. Thus, protecting smart contracts from unknown attacks poses a significant challenge for future research. Additionally, developing automated approaches for repairing vulnerable smart contracts after deployment could prove to be a fruitful direction. 
In conclusion, the security of smart contracts remains an ongoing concern that demands continuous attention and innovation to address evolving threats.


%%
%% The acknowledgments section is defined using the "acks" environment
%% (and NOT an unnumbered section). This ensures the proper
%% identification of the section in the article metadata, and the
%% consistent spelling of the heading.
\begin{acks}
    This work is supported by National Key Research and Development Program of China under the grant No.2021YFB2701202, and National Natural Science Foundation of China (NSFC) under the grant No.62172040,  and Anhui Provincial Natural Science Foundation under the grant No.2008085MF196.
\end{acks}

%%
%% The next two lines define the bibliography style to be used, and
%% the bibliography file.
\bibliographystyle{ACM-Reference-Format}
\bibliography{example}


%%
%% If your work has an appendix, this is the place to put it.


\end{document}
\endinput
%%
%% End of file `sample-manuscript.tex'.
