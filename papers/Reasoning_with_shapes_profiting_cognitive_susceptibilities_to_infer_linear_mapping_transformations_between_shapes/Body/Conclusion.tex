\section{Conclusion and Discussion}
Understanding human brain mechanism and adapting it to different disciplines has always been of interest in various disciplines, such as in deep learning~\cite{hinton2007learning,hinton2006fast}. Ballard defined deictic computation~\cite{ballard1997deictic} that benefits computational methods to study the connection between body and real world movements, to cognitive tasks. Roger B. Nelsen’s ``Proofs without Words''~\cite{mackenzie1993proofs} and Martin Gardner’s ``aha! Solutions''~\cite{gardner1978aha} encouraged the present research to focus on a cognitive-oriented approach for linear mapping transformation between two shapes. The proposed method iterates over different abstractions of image frames, from the most abstracted to the most detailed, while tuning the top transformations of previous iteration to obtain best mapping linear transformations.

The proposed method is implemented and tested over variety of inputs. The method is assessed for its accuracy on determining linear mapping transformations. The experiments showed that the output of the method is reliable even under challenging conditions such as deformed and noise image frames. Additionally, the size of abstraction matrix ($\Gamma$) is independent from the size of the input images frames, and accordingly the computational cost of the proposed method is independent from the resolution of input image frames.