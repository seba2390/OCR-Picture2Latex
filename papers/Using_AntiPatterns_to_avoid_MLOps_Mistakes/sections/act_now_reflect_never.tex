\subsection{`Act Now, Reflect Never' AntiPattern}\label{sec:meta_modeling}
Once models are placed in production, we have seen that predictions are sometimes
used as-is without any filtering, updating, reflection, or even periodic manual inspection. This is an issue especially in situations where we see 1) concept drift (discussed in section~\ref{sec:concept_drift}), 2) irrelevant or easily recognisable erroneous predictions, and 3) adversarial attacks.

It is important to have systems in place that can monitor, track, and debug deployed models. For instance, under such situations it can be productive to have a meta-model that evaluates every model prediction and deems if it is trustworthy (or of required quality) to be delivered. For example, Ramakrishnan et. al.~\cite{beatingthenews} describe a meta-model called the fusion and suppression system that is responsible for the generation of final set of alerts from an underlying alert-stream originating from multiple ML models. The fusion and suppression system is responsible for performing duplicate detection, filling in missing values, and is also used to fine-tune precision / recall by suppressing alerts deemed to be of low quality.  A second solution could be to inspect model decisions further by employing explanation frameworks like LIME~\cite{ribeiro2016should}. Fig.~\ref{fig:meta-modeling} characterizes modeling decisions using meta-modeling frameworks.

\begin{figure*}[!ht]
    \begin{minipage}{0.45\textwidth}
            \includegraphics[scale=0.2]{figures/model_decision_explanation_figures/lime_feature_importances.png}
            \subcaption{Feature Importance Characterization}
    \end{minipage}
    \hfill
    \begin{minipage}{0.45\textwidth}
            \includegraphics[scale=0.376]{figures/model_decision_explanation_figures/lime_explanation_single_instance.png}
            \subcaption{Single Instance Explanation using LIME framework}
    \end{minipage}
    \caption{Inspecting Model Decisions using explanation
    frameworks. Demonstrated on churn (i.e., Attrition) detection application using the LIME~\cite{ribeiro2016should} framework.}
    \label{fig:meta-modeling}
\end{figure*}