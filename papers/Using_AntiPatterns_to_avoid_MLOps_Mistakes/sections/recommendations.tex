How do we make use of these lessons learnt and operationalize them in a production financial ML setting?
Specific recommendations we have include:
\begin{enumerate}
    \item Use AntiPatterns presented here to document a model management process to avoid costly but routine mistakes in model development, deployment, and approval.
    \item Use assertions to track data quality across the enterprise. This is crucial since ML models can be so dependent on faulty or noisy data and suitable checks and balances can ensure a safe operating environment for ML algorithms.
    \item Document data lineage along with transformations to support creation of `audit trails' so models can be situated back in time and in specific data slices for re-training or re-tuning.
    \item Use ensembles to maintain a palette of models including remedial and compensatory pipelines in the event of errors. Track model histories through the lifecycle of an application.
    \item Ensure human-in-the-loop operational capability at multiple levels. Use our model presented for rethinking ML deployment from Section~\ref{sec:rethinkingmodeldeployment} as a basis to support interventions and communication opportunities.
\end{enumerate}

Overall, the model development and management pipeline in our organization supports four classes of stakeholders: (i) the data steward (who holds custody of datasets and sets performance standards), (ii) the model developer (an ML person who designs algorithms), (iii) the model engineer (who places models in production and tracks performance), and (iv) the model certification authority (group of professionals who ensure compliance with standards and risk levels).
In particular, as ML models continue to make their way into more financial decision making systems, the model certification authority within the organization is crucial to ensuring regulatory compliance, from performance, safety, and auditability perspectives.
Bringing such multiple stakeholder groups together ensures a structured process where benefits and risks of ML models are well documented and understood at all stages of development and deployment.

{\small
\subsection*{Disclaimer:} BNY Mellon is the corporate brand of The Bank of New York Mellon Corporation and may be used to reference the corporation as a whole and/or its various subsidiaries generally.  This material does not constitute a recommendation by BNY Mellon of any kind.  The information herein is not intended to provide tax, legal, investment, accounting, financial or other professional advice on any matter, and should not be used or relied upon as such.  The views expressed within this material are those of the contributors and not necessarily those of BNY Mellon.  BNY Mellon has not independently verified the information contained in this material and makes no representation as to the accuracy, completeness, timeliness, merchantability or fitness for a specific purpose of the information provided in this material.  BNY Mellon assumes no direct or consequential liability for any errors in or reliance upon this material.}

