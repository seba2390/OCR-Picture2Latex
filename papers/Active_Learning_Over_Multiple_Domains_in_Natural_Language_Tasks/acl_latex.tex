% This must be in the first 5 lines to tell arXiv to use pdfLaTeX, which is strongly recommended.
\pdfoutput=1
% In particular, the hyperref package requires pdfLaTeX in order to break URLs across lines.

\documentclass[11pt]{article}

% Remove the "review" option to generate the final version.
\usepackage{acl}

% Standard package includes
\usepackage{times}
\usepackage{latexsym}

% For proper rendering and hyphenation of words containing Latin characters (including in bib files)
\usepackage[T1]{fontenc}
% For Vietnamese characters
% \usepackage[T5]{fontenc}
% See https://www.latex-project.org/help/documentation/encguide.pdf for other character sets

% This assumes your files are encoded as UTF8
\usepackage[utf8]{inputenc}

% This is not strictly necessary, and may be commented out,
% but it will improve the layout of the manuscript,
% and will typically save some space.
\usepackage{microtype}

% Added by us:
\usepackage{times}
\usepackage{latexsym}
\renewcommand{\UrlFont}{\ttfamily\small}
\usepackage{footmisc}
\usepackage{float}
\usepackage{bbm}
\usepackage{adjustbox}
\usepackage{amssymb}
\usepackage{pifont}
\usepackage{microtype}
\usepackage{pdfpages}
% \usepackage[dvipsnames]{xcolor}
\usepackage[section]{placeins}
\usepackage{subcaption,amsfonts,dcolumn}
\usepackage{svg}

\usepackage{todonotes}
% \usepackage[disable]{todonotes}
\presetkeys{todonotes}{inline}{}
\usepackage{booktabs}
\usepackage{multirow}

\usepackage{graphicx}
\usepackage{amsmath}
\usepackage{makecell}
\usepackage{algorithmic}
\usepackage{algorithm}


% \newcommand\BibTeX{B\textsc{ib}\TeX}

\newcommand{\dal}{\textsc{DAL}}
\newcommand{\dals}{\textsc{DAL}$*$}
\newcommand{\dalt}{\textsc{DAL-T}}
\newcommand{\dale}{\textsc{DAL-E}}
\newcommand{\dalts}{\textsc{DAL-T}$*$}
\newcommand{\dales}{\textsc{DAL-E}$*$}
\newcommand{\conf}{\textsc{CONF}}
\newcommand{\confa}{\textsc{CONF}$\uparrow$}
\newcommand{\confd}{\textsc{CONF}$\downarrow$}
\newcommand{\entr}{\textsc{ENTR}}
\newcommand{\entra}{\textsc{ENTR}$\uparrow$}
\newcommand{\entrd}{\textsc{ENTR}$\downarrow$}
\newcommand{\eod}{\textsc{ENG}}
\newcommand{\eoda}{\textsc{ENG}$\uparrow$}
\newcommand{\eodd}{\textsc{ENG}$\downarrow$}
\newcommand{\bald}{\textsc{BALD}}
\newcommand{\balda}{\textsc{BALD}$\uparrow$}
\newcommand{\baldd}{\textsc{BALD}$\downarrow$}
\newcommand{\rca}{\textsc{RCA}}
\newcommand{\rcas}{\textsc{$\widetilde{RCA}$}}
\newcommand{\knn}{\textsc{kNN}}
\newcommand{\knnq}{\textsc{kNN-Q}}
\newcommand{\knnc}{\textsc{kNN-C}}
\newcommand{\knnqc}{\textsc{kNN-QC}}
\newcommand{\knns}{\textsc{kNN}$*$}
\newcommand{\knnqs}{\textsc{kNN}$*$\textsc{-Q}}
\newcommand{\knncs}{\textsc{kNN}$*$\textsc{-C}}
\newcommand{\knnqcs}{\textsc{kNN}$*$\textsc{-QC}}
\newcommand{\target}{\textbf{target}}
\newcommand{\source}{\textbf{source}}

\newcommand*\samethanks[1][\value{footnote}]{\footnotemark[#1]}
\newcommand{\specificthanks}[1]{\textdagger}% Inserts a specific \thanks symbol

\newcommand{\abr}[1]{\textsc{#1}}
\newcommand{\respace}{\vspace*{-.2cm}}
\newcommand{\cmark}{\ding{51}}
\newcommand{\xmark}{\ding{55}}
\newcommand{\argmax}{\mathop{\mathrm{argmax}}}

\newcommand{\julia}[1]{\textcolor{cyan}{\bf [Julia: #1]}}
\newcommand{\shayne}[1]{\textcolor{orange}{\bf [Shayne: #1]}}
\newcommand{\edward}[1]{\textcolor{red}{\bf [Edward: #1]}}
\newcommand{\yi}[1]{\textcolor{blue}{\bf [Yi: #1]}}

\definecolor{myred}{RGB}{219, 67, 37}
\definecolor{mygreen}{RGB}{0, 97, 100}

% If the title and author information does not fit in the area allocated, uncomment the following
%
%\setlength\titlebox{<dim>}
%
% and set <dim> to something 5cm or larger.

\title{Active Learning Over Multiple Domains in Natural Language Tasks}

% Author information can be set in various styles:
% For several authors from the same institution:
% \author{Author 1 \and ... \and Author n \\
%         Address line \\ ... \\ Address line}
% if the names do not fit well on one line use
%         Author 1 \\ {\bf Author 2} \\ ... \\ {\bf Author n} \\
% For authors from different institutions:
% \author{Author 1 \\ Address line \\  ... \\ Address line
%         \And  ... \And
%         Author n \\ Address line \\ ... \\ Address line}
% To start a seperate ``row'' of authors use \AND, as in
% \author{Author 1 \\ Address line \\  ... \\ Address line
%         \AND
%         Author 2 \\ Address line \\ ... \\ Address line \And
%         Author 3 \\ Address line \\ ... \\ Address line}

\author{
  Shayne Longpre\thanks{\;\;Equal contribution.}\, \thanks{\;\;Work performed while at Apple.} \\
  \href{mailto:slongpre@mit.edu}{\tt slongpre@mit.edu}
  \And
  Julia Reisler\footnotemark[1] \\
  \href{mailto:jreisler@apple.com}{\tt jreisler@apple.com}
  \AND
  Edward Greg Huang\footnotemark[1]\:\:\footnotemark[2] \\
  \href{mailto:eghuang@berkeley.edu}{\tt eghuang@berkeley.edu}
  \And 
  Yi Lu\footnotemark[1]\:\:\footnotemark[2] \\
  \href{mailto:yilu331@gmail.com}{\tt yilu331@gmail.com}
  \AND
  Andrew Frank 
  \And
  Nikhil Ramesh \And
  Chris DuBois \AND
  \normalfont{Apple}
 \tt
%   \href{mailto:slongpre@mit.edu}{\tt slongpre@mit.edu}\tt{, \{}\href{mailto:jreisler@apple.com}{jreisler}\tt , 
%  \href{mailto:eghuang@apple.com}{\tt eghuang}\tt , 
%  \href{mailto:ylu7@apple.com}{\tt yilu331}\tt \}{\tt @apple.com}
  }

\begin{document}
\maketitle


\begin{abstract} %  


% Revised version from Noah's version
Representation learning for text via pretraining a language model on a large corpus has become a standard starting point for building NLP systems.  This approach stands in contrast to autoencoders, also trained on raw text, but with the objective of learning to encode each input as a vector that allows full reconstruction.  Autoencoders are attractive because of their latent space structure and generative properties. We therefore explore the construction of a sentence-level autoencoder from a pretrained, frozen transformer language model. We adapt the masked language modeling objective as a generative, denoising one, while only training a sentence bottleneck and a single-layer modified transformer decoder.
We demonstrate that the sentence representations discovered by our model achieve better quality than previous methods that extract representations from pretrained transformers on text similarity tasks, style transfer (an example of controlled generation), and single-sentence classification tasks in the GLUE benchmark, while using fewer parameters than large pretrained models.\footnote{Our code is available at: \url{https://github.com/ivanmontero/autobot}} 



% % Noah's 5/10 version with comments
% Representation learning for text via pretraining a language model on a large corpus has become a standard starting point for building NLP systems.  This approach stands in contrast to autoencoders, also trained on raw text, but with the objective of learning to encode each inputs as a vector that allows full reconstruction.  Autoencoders are attractive because of their generative latent space properties. We therefore explore the construction of a sentence-level autoencoder from a pretrained, frozen transformer language model. We adapt the masked language modeling objective as a generative, denosing one, We demonstrate that the sentence representations discovered by our method are a \nascomment{nail down claims about effiency, it's not clear to me right now} and perform competitively with the state of the art \nascomment{check that claim!} on text similarity tasks, style transfer (an example of controlled generation), and sentence classification tasks in the GLUE benchmark.

% \nascomment{fill this in}. (to keep computational cost low) 

% \nascomment{one thing not clear from the paper and possibly worth raising even here:  what data gets used to learn the autoencoder?  also, we need to nail down terminology.  in some places you say that the autoencoder is ``pretrained'' but I think that's confusing since we start from a pretrained model.  do we want to say that we finetune the autoencoder?} 


% old version commented out 5/10 by Noah
%Methods for obtaining unsupervised language representations have gained much attention recently for their performance on downstream tasks, and have largely relied on partial reconstruction of the input though masked language modeling or next token prediction through language modeling.
%Autoencoders, trained on full reconstruction of the input through a single fixed length, bottlenecked representation, have not been extensively studied for pretraining sentence-level representations, especially with transformers.
%To this end, we propose \textsc{Autobots}, a new class of text autoencoders with a sentence bottleneck derived from pretrained transformers on unlabeled data.
%Our training objective is to reconstruct sentences fully from a learned bottleneck representation while keeping the underlying pretrained model fixed.
%We demonstrate that the resulting sentence representations outperform previous methods on text similarity tasks while being parameter efficient, and can also be used for controlled generation tasks such as style transfer.
%Notably our method maintains the performance of the pretrained language model on other supervised  downstream tasks.







% Pretrained language models based on transformers have been shown to produce sub-optimal sentence representations for tasks requiring text similarity \cite{Reimers2019SentenceBERT}.
% Previous methods that ameliorate this issue have introduced auxiliary sentence similarity objectives that rely on annotated data and compose sentence representations with non-parametric pooling that exhibits limited desirable properties and is specific to text similarity tasks.
% To this end, we propose \textsc{Autobots}, a new class of text autoencoders with a sentence bottleneck derived from pretrained transformers on unlabeled data.
% Our training objective is to to reconstruct sentences fully from a learned bottleneck representation as opposed to partially from variable-length representations in existing masked language modeling objectives.
% We demonstrate that our resulting sentence representations outperform previous methods on text similarity tasks while being parameter efficient, and can also be used for controlled generation tasks such as style transfer.
% Notably our method maintains the performance of the pretrained language model on other supervised  downstream tasks.

% \nikos{new take on the abstract, please check. I tried to emphasize on the sentence BERT method to have a clear message about our goal of the paper. There are some nice points in the previous abstract that should probably be incorporated in the introduction but it would be crucial not to make big claims about pretraining in general since it would be hard to compete with big LM papers. }


% ===== nikos =====
% Pretrained language models based on transformers have been shown to produce sub-optimal sentence representations for tasks requiring text similarity \cite{Reimers2019SentenceBERT}.
% Previous methods that ameliorate this issue have introduced auxiliary sentence similarity objectives that rely on annotated data and compose sentence representations with non-parametric pooling that exhibits limited desirable properties and is specific to text similarity tasks.
% To this end, we propose \textsc{Autobots}, a new class of text autoencoders with a sentence bottleneck derived from pretrained transformers on unlabeled data.
% Our training objective is to to reconstruct sentences fully from a learned bottleneck representation as opposed to partially from variable-length representations in existing masked language modeling objectives.
% We demonstrate that our resulting sentence representations outperform previous methods on text similarity tasks while being parameter efficient, and can also be used for controlled generation tasks such as style transfer.
% Notably our method maintains the performance of the pretrained language model on other supervised  downstream tasks.
% =================





% Methods for obtaining unsupervised language representations have gained much attention recently for their performance on downstream tasks, and have largely relied on partial reconstruction of the input though masked language modeling or next token prediction through language modeling. Autoencoders, trained on full reconstruction of the input through a single fixed length, bottlenecked representation, have not been extensively studied for pretraining sentence-level representations, especially with transformers.
%In this paper, we propose the \textit{Transformer Autoencoder} by converting the architecture of \citet{vaswani17} to an autoencoder that uses attention to bottleneck the input to a single, fixed vector from which the input can be fully reconstructed. Leveraging the encoder advancements of \citet{devlin-etal-2019-bert}, we introduce \textbf{BARNEY}, which stands for \textbf{B}ERT \textbf{A}ggregated \textbf{R}epresentations by Reduci\textbf{n}g and R\textbf{e}constructing Full\textbf{y}, and demonstrate the significant performance of this novel pretraining objective in downstream classification, sentence representation, reconstruction, and controlled generation tasks.

\end{abstract}

% Autoencoders have 

% Unsupervised pretraining has gained much attention recently 
% We propose the transformer

% Methods for obtaining unsupervised language representations have gained much attention recently for their performance on downstream tasks, and have largely relied on partial reconstruction of the input though masked language modeling or next token prediction through language modelling.

% Text autoencoders obtain an unsupervised sentence-level representation by bottlenecking with reconstruction abilities, but largely rely on recurreence in the decoder.

% In this paper, we propose the \textit{Transformer Autoencoder} model by converting the architecture of \citet{vaswani17} to an autoencoder that uses attention to bottleneck the input to a single, fixed vector from which the input can be fully reconstructed. Leveraging the encoder advancements of \citet{devlin-etal-2019-bert}, we introduce textbf{BARNEY}, which stands for \textbf{B}ERT \textbf{A}ggregated \textbf{R}epresentations by Reduci\textbf{n}g and R\textbf{e}constructing Full\textbf{y}, to demonstrate the performance of this novel pretraining objective in downstream classification, sentence representation, reconstruction, and controlled genration tasks.

% by leveraging the encoder advancements of \citet{devlin-etal-2019-bert} in a transformer auto


% effectively enriching the resulting sentence representations with reconstructive properties. Concretely, we introduce a context attention bottleneck after the original encoder, and a modify the decoder to perform full reconstruction of the input conditioned on the bottleneck representation.
 
% Currently langauge representat
% 
 

% VERSION 1:

% % to learn 
% % \textit{BERT isn't an autoencoder, but BARNEY is!}

% Autoencoders have gained much attention for their unsupervised ability to reduce variable length sequences to a single latent space with reconstruction capabilities, yet current architectures rely heavily on recurrence. Similar to autoencoders, current state-of-the-art pretrained transformer models produce variable-length token represenations by learning to partially reconstruct their input, yet rely on a special token to produce sentence representations for most downstream natural language understanding tasks requiring a single vector.


% % Current state-of-the-art pretrained language models produce variable-length token representations by learning to partially reconstruct the input, and rely on a special token to produce sentence representations for most downstream natural language understanding tasks requiring a single vector. 
% %\nikos{we may want to mention that it's not possible to produce the full sentence based on cls alone} 
% %\nikos{the token level representations are used for word-level tasks though, so we have to make sure we don't overclaim here.}

% In this paper, we propose the \textit{Transformer Autoencoder} model that results from  converting the architecture of \citet{vaswani17} to an autoencoder that projects the input to a single, fixed vector from which the input can be fully reconstructed, effectively enriching the resulting sentence representations with reconstructive properties. 
% Concretely, we introduce a context attention bottleneck after the original encoder, and a modify the decoder to perform full reconstruction of the input %(instead of partial) 
% conditioned on the bottleneck representation.
% %to the "encoder-decoder" layer in the decoder stacks.
% %
% %We preserve the parallelism in the original transformer by analyzing dot-product attention's decomposition in the many-to-one and one-to-many settings. \nikos{Not sure I fully get this point. }
% %
% %\textbf{TAE} can be pre-trained on unlabeled corpora to produce a single sequence-level representation. \nikos{this point could be skipped}

% % Lastly, we develop an example framework called \textbf{BARNEY} based on transformer autoencoders for pretraining akin to BERT \citet{devlin-etal-2019-bert}
% Lastly, we introduce \textbf{BARNEY}, which stands for \textbf{B}ERT \textbf{A}ggregated \textbf{R}epresentations by Reduci\textbf{n}g and R\textbf{e}constructing Full\textbf{y}, which both leverages advancements of \citet{devlin-etal-2019-bert} and introduces a novel pretraining objective for transformers. We evaluate its sentence representations on downstream classification, reconstruction and controlled generation tasks.\footnote{Our code will be made available at [link]} 


% %, which stands for \textbf{B}idirectional \textbf{A}ggregated Transformer \textbf{R}epresentations by Reduci\textbf{n}g and R\textbf{e}constructing Full\textbf{y}. \nikos{is it ok if we skip the abbreviations until we decide the proper framing?}
% % and  we evaluate its sentence representations on downstream classification, reconstruction and controlled generation tasks.\footnote{Our code will be made available at [link]} 

Neural networks are powerful models that excel at a wide range of tasks.
However, they are notoriously difficult to interpret and extracting explanations 
    for their predictions is an open research problem. Linear models, in contrast, are generally considered interpretable, because
    the \emph{contribution} 
    (`the weighted input') of every dimension to the output is explicitly given.
Interestingly, many modern neural networks implicitly model the output as a linear transformation of the input;
    a ReLU-based~\cite{nair2010rectified} neural network, e.g.,
    is piece-wise linear and the output thus a linear transformation of the input, cf.~\cite{montufar2014number}.
    However, due to the highly non-linear manner in which these linear transformations are `chosen', the corresponding contributions per input dimension do not seem to represent the learnt model parameters well, cf.~\cite{adebayo2018sanity}, and a lot of research is being conducted to find better explanations for the decisions of such neural networks, cf.~\cite{simonyan2013deep,springenberg2014striving,zhou2016CAM,selvaraju2017grad,shrikumar2017deeplift,sundararajan2017axiomatic,srinivas2019full,bach2015pixel}.
    
In this work, we introduce a novel network architecture, the \textbf{Convolutional Dynamic Alignment Networks (CoDA-Nets)}, {for which the model-inherent contribution maps are faithful projections of the internal computations and thus good `explanations' of the model prediction.} 
There are two main components to the interpretability of the CoDA-Nets. 
    First, the CoDA-Nets are \textbf{dynamic linear}, i.e., they compute their outputs through a series of input-dependent linear transforms, which are based on our novel \mbox{\textbf{Dynamic Alignment Units (DAUs)}}. 
        As in linear models, the output can thus be decomposed into individual input contributions, see Fig.~\ref{fig:teaser}.
    Second, the DAUs are structurally biased to compute weight vectors that \textbf{align with \mbox{relevant} patterns} in their inputs. 
In combination, the CoDA-Nets thus inherently  
produce contribution maps that are `optimised for interpretability': 
since each linear transformation matrix and thus their combination is optimised to align with discriminative features, the contribution maps reflect the most discriminative features \emph{as used by the model}.

With this work, we present a new direction for building inherently more interpretable neural network architectures with high modelling capacity.
In detail, we would like to highlight the following contributions:
\begin{enumerate}[wide, label={\textbf{(\arabic*)}}, itemsep=-.5em, topsep=0em, labelwidth=0em, labelindent=0pt]
    \item We introduce the Dynamic Alignment Units (DAUs), which 
    improve the interpretability of neural networks and have two key properties:
    they are 
    \emph{dynamic linear} 
    and align their weights with discriminative input patterns.
    \item Further, we show that networks of DAUs \emph{inherit} these two properties. In particular, we introduce Convolutional Dynamic Alignment Networks (CoDA-Nets), which are built out of multiple layers of DAUs. As a result, the \emph{model-inherent contribution maps} of CoDA-Nets highlight discriminative patterns in the input.
    \item We further show that the alignment of the DAUs can be promoted 
    by applying a `temperature scaling' to the final output of the CoDA-Nets. 
    \item We show that the resulting contribution maps 
    perform well under commonly employed \emph{quantitative} criteria for attribution methods. Moreover, under \emph{qualitative} inspection, we note that they exhibit a high degree of detail.
    \item Beyond interpretability, 
    CoDA-Nets are performant classifiers and yield competitive classification accuracies on the CIFAR-10 and TinyImagenet datasets.
\end{enumerate}
% \section{Engines}

% To produce a PDF file, pdf\LaTeX{} is strongly recommended (over original \LaTeX{} plus dvips+ps2pdf or dvipdf). Xe\LaTeX{} also produces PDF files, and is especially suitable for text in non-Latin scripts.

\section{Related Work}
\respace
\label{sec:relatedwork}

\paragraph{Active Learning in NLP}
\citet{lowell2019practical} shows how inconsistent active learning methods are in NLP, even under regular conditions. 
However, \citet{dor2020active, siddhant-lipton-2018-deep} survey active learning methods in NLP and find notable gains over random baselines.
\citet{kouw2019review} survey domain adaptation without target labels, similar to our setting, but for non-language tasks.
We reference more active learning techniques in Section~\ref{sec:methods}.

\respace
\paragraph{Domain Shift Detection}
\citet{elsahar-galle-2019-annotate} attempt to predict accuracy drops due to domain shifts and \citet{rabanser2018failing} surveys different domain shift detection methods. 
\citet{arora2021types} examine calibration and density estimation for textual OOD detection.
% Both evaluate domain discriminators for detecting domain shift using task agnostic and specific embeddings, which motivates our \dal{} methods.

\respace
\paragraph{Active Learning under Distribution Shift}
A few previous works investigated active learning under distribution shifts, though mainly in image classification, with single source and target domains.
\citet{kirsch2021active} finds that \bald{}, which is often considered the state of the art for unshifted domain settings, can get stuck on irrelevant source domain or junk data. 
\citet{pmlr-v130-zhao21b} investigates \emph{label shift}, proposing a combination of predicted class balanced subsampling and importance weighting. 
\citet{10.1007/978-3-642-23808-6_7}, whose approach corrects joint distribution shift, relies on the \emph{covariate shift assumption}. 
However, in practical settings, there may be general distributional shifts where neither the \emph{covariate shift} nor \emph{label shift} assumptions hold.

\respace
\paragraph{Transfer Learning from Multiple Domains}
Attempts to better understand how to handle shifted domains for better generalization or target performance has motivated work in question answering \citep{talmor2019multiqa, fisch2019mrqa, longpre2019exploration, kamath2020selective} and classification tasks \citep{ruder2018strong, sheoran2020recommendation}.
\citet{ruder2017learning} show the benefits of both data similarity and diversity in transfer learning.
\citet{ruckle2020multicqa} find that sampling from a wide-variety of source domains (data scale) outperforms sampling similar domains in question answering.
\citet{he2021multi} investigate a version of multi-domain active learning where models are trained and evaluated on examples from all domains, focusing on robustness across domains.
% \section{Preamble}

% The first line of the file must be
% \begin{quote}
% \begin{verbatim}
% \documentclass[11pt]{article}
% \end{verbatim}
% \end{quote}

% To load the style file in the review version:
% \begin{quote}
% \begin{verbatim}
% \usepackage[review]{acl}
% \end{verbatim}
% \end{quote}
% For the final version, omit the \verb|review| option:
% \begin{quote}
% \begin{verbatim}
% \usepackage{acl}
% \end{verbatim}
% \end{quote}

% To use Times Roman, put the following in the preamble:
% \begin{quote}
% \begin{verbatim}
% \usepackage{times}
% \end{verbatim}
% \end{quote}
% (Alternatives like txfonts or newtx are also acceptable.)

% Please see the \LaTeX{} source of this document for comments on other packages that may be useful.

% Set the title and author using \verb|\title| and \verb|\author|. Within the author list, format multiple authors using \verb|\and| and \verb|\And| and \verb|\AND|; please see the \LaTeX{} source for examples.

% By default, the box containing the title and author names is set to the minimum of 5 cm. If you need more space, include the following in the preamble:
% \begin{quote}
% \begin{verbatim}
% \setlength\titlebox{<dim>}
% \end{verbatim}
% \end{quote}
% where \verb|<dim>| is replaced with a length. Do not set this length smaller than 5 cm.

\respace
\section{Multi-Domain Active Learning}
\respace
\label{sec:task}

% Suppose we have $K$ domains $D_1, D_2, ..., D_k \in D$.\footnote{We define a domain as a dataset collected independently of the others.}
%Suppose we have multiple domains $D = \bigcup\limits_{i=1} D_{i}$.\footnote{We define a domain as a dataset collected independently of the others.}
Suppose we have multiple domains $D_1, D_2, ..., D_k$.\footnote{We define a domain as a dataset collected independently of the others.}
Let one of the $k$ domains be the \target{} set $D_T$, and let the other $k-1$ domains comprise the \source{} set $D_S=\bigcup\limits_{i\neq T} D_{i}$.
% Each domain comprises a training set $D_i^{train}$, development set $D_i^{dev}$, and test set $D_i^{test}$, all sets of $(x, y)$ input-label pairs.
% One of these domains is designated the \target{} set $D_T$, while the other $\{D_i^{train} \mid i\ne T\}$ domains collectively become the \source{} set $D_S$.

\respace
\respace
\paragraph{Given:}
\begin{itemize}\itemsep0em
    \item \textbf{Target:} Small samples of \textit{labeled} data points $(x, y)$ from the \target{} domain. \\
    $D_T^{train},D_T^{dev}, D_T^{test} \sim D_T$.\footnote{$|D_T^{train}|=2000$ to simulate a small but reasonable quantity of labeled, in-domain training data for active learning scenarios.}
    \item  \textbf{Source:} A large sample of \textit{unlabeled} points $(x)$ from the \source{} domains. \\
    $D_S=\bigcup\limits_{i\neq T} D_{i}$
\end{itemize}
\respace
\paragraph{Task:}
\begin{enumerate}\itemsep0em
    \item \textbf{Choose} $n$ samples from $D_S$ to label. \\
    $D_{S}^{chosen} \subset D_S$, $|D_{S}^{chosen}|=n$, selected by $ \argmax_{x \in D_S} A_{f}(x) $ where $ A_{f} $ is an acquisition function: a policy to select unlabeled examples from $D_S$ for labeling.
    \item \textbf{Train} a model $M$ on $D^{final-train}$, validating on $D_{T}^{dev}$. \\
    $D^{final-train} = D_T^{train} \cup D_S^{chosen}$
    \item \textbf{Evaluate} $M$ on $D_T^{test}$, giving score $s$.
\end{enumerate}

For Step 1, the practitioner chooses $n$ samples with the highest scores according to their acquisition function $A_f$. 
$M$ is fine-tuned on these $n$ samples, then evaluated on $D_T^{test}$ to demonstrate $A_f$'s ability to choose relevant out-of-distribution training examples.

% Many functions train an acquisition model $M_A$, on $D_T^{train}$.
% Both $M_A$ and $M$ use $D_T^{dev}$ as their development set.
% Section \ref{sec:methods} describes the choice of $M_A$ and how it is used for each acquisition method $A_f$.
% \section{Document Body}

% \subsection{Footnotes}

% Footnotes are inserted with the \verb|\footnote| command.\footnote{This is a footnote.}

% \subsection{Tables and figures}

% See Table~\ref{tab:accents} for an example of a table and its caption.
% \textbf{Do not override the default caption sizes.}

% \begin{table}
% \centering
% \begin{tabular}{lc}
% \hline
% \textbf{Command} & \textbf{Output}\\
% \hline
% \verb|{\"a}| & {\"a} \\
% \verb|{\^e}| & {\^e} \\
% \verb|{\`i}| & {\`i} \\ 
% \verb|{\.I}| & {\.I} \\ 
% \verb|{\o}| & {\o} \\
% \verb|{\'u}| & {\'u}  \\ 
% \verb|{\aa}| & {\aa}  \\\hline
% \end{tabular}
% \begin{tabular}{lc}
% \hline
% \textbf{Command} & \textbf{Output}\\
% \hline
% \verb|{\c c}| & {\c c} \\ 
% \verb|{\u g}| & {\u g} \\ 
% \verb|{\l}| & {\l} \\ 
% \verb|{\~n}| & {\~n} \\ 
% \verb|{\H o}| & {\H o} \\ 
% \verb|{\v r}| & {\v r} \\ 
% \verb|{\ss}| & {\ss} \\
% \hline
% \end{tabular}
% \caption{Example commands for accented characters, to be used in, \emph{e.g.}, Bib\TeX{} entries.}
% \label{tab:accents}
% \end{table}

% \subsection{Hyperlinks}

% Users of older versions of \LaTeX{} may encounter the following error during compilation: 
% \begin{quote}
% \tt\verb|\pdfendlink| ended up in different nesting level than \verb|\pdfstartlink|.
% \end{quote}
% This happens when pdf\LaTeX{} is used and a citation splits across a page boundary. The best way to fix this is to upgrade \LaTeX{} to 2018-12-01 or later.

% \subsection{Citations}

% \begin{table*}
% \centering
% \begin{tabular}{lll}
% \hline
% \textbf{Output} & \textbf{natbib command} & \textbf{Old ACL-style command}\\
% \hline
% \citep{Gusfield:97} & \verb|\citep| & \verb|\cite| \\
% \citealp{Gusfield:97} & \verb|\citealp| & no equivalent \\
% \citet{Gusfield:97} & \verb|\citet| & \verb|\newcite| \\
% \citeyearpar{Gusfield:97} & \verb|\citeyearpar| & \verb|\shortcite| \\
% \hline
% \end{tabular}
% \caption{\label{citation-guide}
% Citation commands supported by the style file.
% The style is based on the natbib package and supports all natbib citation commands.
% It also supports commands defined in previous ACL style files for compatibility.
% }
% \end{table*}

% Table~\ref{citation-guide} shows the syntax supported by the style files.
% We encourage you to use the natbib styles.
% You can use the command \verb|\citet| (cite in text) to get ``author (year)'' citations, like this citation to a paper by \citet{Gusfield:97}.
% You can use the command \verb|\citep| (cite in parentheses) to get ``(author, year)'' citations \citep{Gusfield:97}.
% You can use the command \verb|\citealp| (alternative cite without parentheses) to get ``author, year'' citations, which is useful for using citations within parentheses (e.g. \citealp{Gusfield:97}).

% \subsection{References}

% \nocite{Ando2005,borschinger-johnson-2011-particle,andrew2007scalable,rasooli-tetrault-2015,goodman-etal-2016-noise,harper-2014-learning}

% The \LaTeX{} and Bib\TeX{} style files provided roughly follow the American Psychological Association format.
% If your own bib file is named \texttt{custom.bib}, then placing the following before any appendices in your \LaTeX{} file will generate the references section for you:
% \begin{quote}
% \begin{verbatim}
% \bibliographystyle{acl_natbib}
% \bibliography{custom}
% \end{verbatim}
% \end{quote}

% You can obtain the complete ACL Anthology as a Bib\TeX{} file from \url{https://aclweb.org/anthology/anthology.bib.gz}.
% To include both the Anthology and your own .bib file, use the following instead of the above.
% \begin{quote}
% \begin{verbatim}
% \bibliographystyle{acl_natbib}
% \bibliography{anthology,custom}
% \end{verbatim}
% \end{quote}

% Please see Section~\ref{sec:bibtex} for information on preparing Bib\TeX{} files.

% \subsection{Appendices}

% Use \verb|\appendix| before any appendix section to switch the section numbering over to letters. See Appendix~\ref{sec:appendix} for an example.

\section{Methods}
\label{sec:methods}

We identify four families of methods relevant to active learning over multiple shifted domains.
\textbf{Uncertainty methods} are common in standard active learning for measuring example uncertainty or familiarity to a model; \textbf{$\mathcal{H}$-Divergence} techniques train classifiers for domain shift detection; \textbf{Semantic Similarity Detection} finds data points similar to points in the target domain; and \textbf{Reverse Classification Accuracy} approximates the benefit of training on a dataset. 
A limitation of our work is we do not cover all method families, such as domain adaptation, just those we consider most applicable.
%Not all active learning and domain shift detection methods are represented here, such as query-by-committee or density weighted sampling \citep{settles2009active}, but we derive \~18 of the most prevalent variants from prior work, including novel extensions/variants of existing paradigms for the multi-domain active learning setting (see \knn{}, \rcas{} and \dale{}).
We derive $\sim$18 active learning variants, comprising the most prevalent and effective from prior work, and novel extensions/variants of existing paradigms for the multi-domain active learning setting (see \knn{}, \rcas{} and \dale{}).

Furthermore, we split the families into two acquisition strategies: \textbf{Single Pool Strategy} and \textbf{Domain Budget Allocation}. 
\textbf{Single Pool Strategy}, comprising the first three families of methods, treats all examples as coming from one single unlabeled pool. \textbf{Domain Budget Allocation}, consisting of \textbf{Reverse Classification Accuracy} methods, simply allocate an example budget for each domain.

We enumerate acquisition methods $A_f$ below. 
Each method produces a full ranking of examples in the source set $D_S$.
To rank examples, most acquisition methods train an acquisition model, $M_A$, using the same model architecture as $M$. 
$M_A$ is trained on all samples from $D_T^{train}$, except for \dal{} and \knn{}, which split $D_T^{train}$ into two equal segments, one for training $M_A$ and one for an internal model.
%For methods where $M_A$ requires supervision, $M_A$ is trained on $D_T^{train}$, or for variants of \dal{} and \knn{} which train another supervised model, they are each trained on separate 1000 examples from $D_T^{train}$.
Some methods have both ascending and descending orders of these rankings (denoted by $\uparrow$ and $\downarrow$ respectively, in the method abbreviations), to test whether similar or distant examples are preferred in a multi-domain setting.

% For Uncertainty methods (Section~\ref{sec:uncertainty-methods}) and Reverse Classification Accuracy (Section~\ref{sec:rca-methods}) we use BERT-Base \citep{devlin2019bert}.
% For $\mathcal{H}$-Divergence Methods (Section~\ref{sec:h-divergence-methods}) and Semantic Similarity Detection (Section~\ref{sec:kNN-method}) we experiment with both task-agnostic and task-specific embeddings.
Certain methods use vector representations of candidate examples.
We benchmark with both task-agnostic and task-specific encoders.
The task-agnostic embeddings are taken from the last layer's CLS token in \citet{reimers-2019-sentence-bert}'s sentence encoder (Appendix for details).
%\footnote{A RoBERTa-Large model fine-tuned on SNLI \citep{bowman2015large} and STSb \citep{cer2017semeval}: \url{https://github.com/UKPLab/sentence-transformers}}
The task-specific embeddings are taken from the last layer's CLS token in the trained model $M_A$.

The motivation of the task-specific variant is that each example's representation will capture task-relevant differences between examples while ignoring irrelevant differences.\footnote{For instance, consider in one domain every example is prefixed with ``Text:'' while the other is not --- telling the difference is trivial, but the examples could be near-identical with respect to the task.}
The versions of \dal{} and \knn{} methods that use task-specific vectors are denoted with ``$*$'' in their abbreviation. 
Otherwise, they use task-agnostic vectors.

    \subsection{Uncertainty Methods}
    \label{sec:uncertainty-methods}
    \respace
    These methods measure the uncertainty of a trained model on a new example.
    Uncertainty can reflect either \textit{aleatoric} uncertainty, due to ambiguity inherent in the example, or \textit{epistemic} uncertainty, due to limitations of the model \citep{kendall2017uncertainties}.
    For the following methods, let $Y$ be the set of all possible labels produced from the model $M(x)$ and $l_y$ be the logit value for $y \in Y$.

        \paragraph{Confidence (\conf)}
        A model's confidence $P(y|x)$ in its prediction $y$ estimates the difficulty or unfamiliarity of an example \citep{guo2017calibration, elsahar-galle-2019-annotate}.
        \respace
        $$ A_{\text{CONF}}(x, M_A) = -\max(P(y|x))$$ 
        \respace
        \paragraph{Entropy (\entr)}
        Entropy applies Shannon entropy \citep{shannon1948mathematical} to the full distribution of class probabilities for each example, formalized as $A_{\text{ENTR}}$.
        \respace
        \respace
         $$ A_{\text{ENTR}}(x, M_A) = -\sum_{i = 1}^{|Y|} { P(y_i|x) \cdot \log{P(y_i|x)}  }$$ 
        
        % $$ A_{\text{ENTR}}(x, M_A) = -\sum_{i = 1}^{|Y|} { P(y_i|x) \cdot \log{P(y_i|x)}  }$$

        \respace
        \respace
        \paragraph{Energy-based Out-of-Distribution Detection (\eod)}
        \citet{liu2020energy} use an \textit{energy-based score} to distinguish between in- and out-distribution examples.
        They demonstrate this method is less susceptible to overconfidence issues of softmax approaches.
        \respace
        % We chose $T=1$ to maximize the distinction between in- and out-domain predictions.
        $$ A_{ENG}(x, M_A) = - \log \sum_{y \in Y} e^{l_y} $$ 
        \respace
        \respace
        \paragraph{Bayesian Active Learning by Disagreement (\bald)}
        \citet{gal2016} introduces estimating uncertainty by measuring prediction disagreement over multiple inference passes, each with a distinct dropout mask.
        \bald{} isolates \textit{epistemic} uncertainty, as the model would theoretically produce stable predictions over inference passes given sufficient capacity.
        We conduct $T=20$ forward passes on $x$.
        $\hat{y}_t = \text{argmax}_{i}P(y_i|x)_t$, representing the predicted class on the $t$-th model pass on $x$.
        Following \citep{lowell2019practical}, ties are broken by taking the mean label entropy over all $T$ runs. 
        \respace
        \respace
        $$ A_{\text{BALD}}(x, M_{A}) = 1 - \frac{\text{count}(\text{mode}_{t \in T} (\hat{y}_t))}{T} $$
        
        % $$ A_{ENG}(x, M_A) = -T \cdot \log \sum_{y \in Y} e^{l_y / T} \;\;\;\;\;\; A_{\text{BALD}}(x, M_{A}) = 1 - \frac{\text{count}(\text{mode}_{t \in T} (\hat{y}_t))}{T} $$
        
    \respace
    \respace
    \subsection{$\mathcal{H}$-Divergence Methods}
    \label{sec:h-divergence-methods}
    \citet{ben2006analysis, ben2010theory} formalize the divergence between two domains as the $\mathcal{H}$-Divergence, which they approximate as the difficulty for a discriminator to differentiate between the two.\footnote{The approximation is also referred to as Proxy $\mathcal{A}$-Distance (PAD) from \citep{elsahar-galle-2019-annotate}}
    Discriminative Active Learning (\dal) applies this concept to the active learning setting \citep{gissin2019discriminative}.
    
    We explore variants of \dal{}, using an XGBoost decision tree \citep{Chen:2016:XST:2939672.2939785} as the discriminator model $g$.\footnote{Hyperparameter choices and training procedures are detailed in the Appendix.}
    For the following methods, let $D_T^{train-B}$ be the 1k examples from $D_T^{train}$ that were \emph{not} used to train $M_A$. 
    Let $E$ be an encoder function, which can be task-specific or agnostic as described above. 
    We use samples both from $D_T^{train-B}$ and $D_S$ to train the discriminator. 
    We assign samples origin labels $l$, which depend on the \dal{} variant. 
    Samples from $D_S$ with discriminator predictions closest to 1 are selected for labeling.
    The acquisition scoring function for each \dal{} method and training set definition, respectively, are:
    \respace
    \respace
    $$A_{\text{DAL}}(x, g, E) = g(E(x)) $$
    $$ \{(E(x), \: l) \; | \; x \in D_T^{train-B} \cup D_{S} \}$$

    %This is inspired by deep NLP models producing intermediate representations that exclude features irrelevant to the NLP task. Motivated by this work and by \citep{gissin2019discriminative}, which uses a discriminator for active learning, we explore $TAL_E$ and $TAL_T$.
    %Both $TAL_E$ and $TAL_T$ perform discriminative active learning. 
    % Both $TAL_E$ and $TAL_T$ perform discriminatory active learning where
    % %the input to the discriminator $E(x)$ is the penultimate layer from model $M_1$.
    % the input $g(\xi)$ to the discriminator $C$ is the penultimate layer from model $M_1$.
    % Their acquisition functions can be written as 
    % %$$x* = argmax_{x\in D_{source}}C(g(x)).$$
    % $$ a_{TAL}(x, M_1) = C(g(\xi), x) $$

    % For both methods, $C$ is an XGBoost gradient boosting decision tree \citep{Chen:2016:XST:2939672.2939785}. Hyperparameter choices are discussed in the appendix. \shayne{Do you know how to set up an appendix?}
    % \julia{Define C}
        \respace
        \respace
        \paragraph{Discriminative Active Learning --- Target (\dalt)}
        \dalt{} trains a discriminator $g$ to distinguish between target examples in $D_T^{train-B}$ and out-of-distribution examples from $D_{S}$. 
        For \dalt{}, $l=\mathbbm{1}_{D_T^{train-B}}(x)$.
        % \footnote{Multiple models, each trained on a different segment of the data, are used to fairly train and select from $D_{S}$. Full details are in the appendix.}
        % To do this, it assigns pseudo-labels to samples in $D_{val}$ and $D_{S}$ such that $g$'s training set consists of samples 
        % %$\{(g(x), l) \; | \; l=\mathbbm{1}_{D_{val}}(x), x\in \{D_{val}, D_{source}\}\}$.
        % $\{(g(\xi), \: l) \; | \; l=\mathbbm{1}_{D_{val}}(\xi), \; \xi \in D_{val} \cup D_{S} \}$.
        
        \respace
        \paragraph{Discriminative Active Learning --- Error (\dale)}
        \dale{} is a novel variant of \dal.
        \dale's approach is to find examples that are similar to those in the target domain that $M_A$ misclassified.
        %These ``erroneous'' examples represent those the model finds most challenging from the target domain.
        %We define a non-erroneous sample as one where the prediction from $M_1$ is an exact match of its label. 
        % We define an erroneous sample as one where the predicted pseudo-label is incorrect. 
        We partition $D_T^{train-B}$ further into erroneous samples $D_T^{err}$ and correct samples $D_T^{corr}$, where $D_T^{train-B} = D_T^{err} \cup D_T^{corr}$. For \dale{}, $l=\mathbbm{1}_{D^{err}_T}(x)$.

        % For both methods, the discriminator is an XGBoost gradient boosting decision tree \citep{Chen:2016:XST:2939672.2939785}. 
        % The model had $10$ for $n\_estimators$, $2$ for $max\_depth$, hist for $tree\_method$, $5$ for $lambda$, $binary:logistic$ for $objective$, $0.1$ for $learning\_rate$, and $5$ for $gamma$. The rest of the hyperparameters were set to the default XGBoost parameters.

    \respace
    \subsection{Reverse Classification Accuracy}
    \label{sec:rca-methods}

        \respace
        \paragraph{\rca}

        Reverse Classification Accuracy (\rca) estimates how effective source set $D_{i, i \in S}$ is as a training data for target test set $D_T$ \citep{fan2006reverse, elsahar-galle-2019-annotate}.
        Without gold labels for $D_{i}$ we compute soft labels instead, using the BERT-Base $M_A$ trained on the small labeled set $D_T^{train}$.
        We then train a child model $M_i$ on $D_{i}$ using these soft labels, and evaluate the child model on $D_T^{dev}$.
        \rca{} chooses examples randomly from whichever domain $i$ produced the highest score $s_i$.
        \respace
        \respace
        
        $$A_{\text{RCA}} = \mathbbm{1}_{D_{(\argmax_{i\in S}s_{i})}}(x) $$ 
        $$ \widetilde{RCA}:\;\;\; \tau_i = \dfrac{s_i}{s_T - s_i}, \; |D^{chosen}_i| = \dfrac{\tau_i}{\sum\limits_{j} s_j}$$

        \respace
        \respace
        \respace
        \paragraph{RCA-Smoothed (\rcas)}
        Standard \rca{} only selects examples from one domain $D_{i}$.
        We develop a novel variant which samples from multiple domains, proportional to their relative performance on the target domain $D_T^{dev}$.
        RCA-smoothed (\rcas) selects $|D^{chosen}_i|$ examples from source domain $i$, based on the relative difference between the performance $s_i$ (of child model $M_i$ trained on domain $i$ with pseudo-labels from $M_A$) on the target domain, and the performance $s_T$ of a model trained directly on the target domain $D_T^{dev}$. 
        Since these strategies directly estimates model performance on the target domain resulting from training on each source domain, \rca{} and \rcas{} are strong \textbf{Domain Budget Allocation} candidates.
        
    \respace
    \subsection{Nearest Neighbour / Semantic Similarity Detection (\knn)}
    \label{sec:kNN-method}
    \respace
    
    Nearest neighbour methods (\knn) are used to find examples that are semantically similar.
    Using sentence encoders we can search the source set $D_S$ to select the top $k$ nearest examples by cosine similarity to the target set.
    We represent the target set as the mean embedding of $D_T^{train}$. 
    For question answering, where an example contains two sentences (the query and context), we refer to \knnq{} where we only encode the query text, \knnc{} where we only encode the context text, or \knnqc{} where we encode both concatenated together.
    The acquisition scoring function per example, uses either a task-specific or task-agnostic encoder $E$:
    \respace
    $$A_{\text{KNN}}(x, E) = \text{CosSim}(E(x), \text{Mean}( E(D_{T}^{train} ))$$

    
    %For task-specific vector representations, we use the common convention of the final hidden layer of the $0$th token of the model $M_A$ trained on $D_{T}^{train}$.
    %The task-specific variants of the \knn{} approach are suffixed with $*$, such as \knns.
    
    



% \section{Bib\TeX{} Files}
% \label{sec:bibtex}

% Unicode cannot be used in Bib\TeX{} entries, and some ways of typing special characters can disrupt Bib\TeX's alphabetization. The recommended way of typing special characters is shown in Table~\ref{tab:accents}.

% Please ensure that Bib\TeX{} records contain DOIs or URLs when possible, and for all the ACL materials that you reference.
% Use the \verb|doi| field for DOIs and the \verb|url| field for URLs.
% If a Bib\TeX{} entry has a URL or DOI field, the paper title in the references section will appear as a hyperlink to the paper, using the hyperref \LaTeX{} package.
\respace
\respace
\section{Experiments}
\respace
\label{sec:experiments}

Experiments are conducted on two common NLP tasks: question answering (QA) and sentiment analysis (SA), each with several available domains.

\respace
\paragraph{Question Answering} 
We employ 6 diverse QA datasets from the MRQA 2019 workshop \citep{fisch2019mrqa}, shown in Table~\ref{datasets}.\footnote{The workshop pre-processed all datasets into a similar format, for fully answerable, span-extraction QA: \url{https://github.com/mrqa/MRQA-Shared-Task-2019}.}
We sample 60k examples from each dataset for training, 5k for validation, and 5k for testing. 
Questions and contexts are collected with varying procedures and sources, representing a wide diversity of datasets.

\section{Low-Voltage Load Forecasting Datasets} 
\label{secdatasets}

A number of interesting features were discovered about the data in the reviewed 221 papers. Firstly, only 52 use at least one openly available datasets to illustrate the results, i.e. less than 24\% of the journals presented results that could be potentially replicated by the wider research community. Of these 52 papers using open data, 22 (or $42\%$)  of them used the Irish CER Smart Metering Project data~\cite{Commission2012csm}, four used data from UK Low Carbon London project~\cite{UK2014ulc}, four from Ausgrid\footnote{\url{https://www.ausgrid.com.au/Industry/Our-Research/Data-to-share/Solar-home-electricity-data}} and three used the UMass dataset. In other words, out of the papers using open data, $56\%$, presented results that used data from only four open data sets. 

The overuse of a particular dataset can result in biases (both conscious and unconscious) where methods are developed and tested but the features of the data may be well known or familiar from overuse. In these cases a scientifically rigorous experiment is impossible. Further, reliance on a single dataset (especially those which are no more than 2 years in length like the Irish CER dataset~\cite{Commission2012csm}) risks the development of models which may be based on spurious features and patterns and may not be representative of the wider energy system. This can be alleviated somewhat by including multiple open data sets, as it was done in some of the papers reviewed (\cite{abera2020mla, Laurinec2019due, Wang2018aef}). To support the LV forecast research community the authors are going to share a modifiable list of open data sets at the LV level. A list and some of the properties of the data are shown in Table \ref{tab:datasets}.  

With the rapid change in low carbon technologies being connected to the grid and home, new energy efficiency interventions, and adjustments in demand usage behaviours, demand data can quickly become irrelevant or unrepresentative. Further, such data sets are based on trials where participants are subject to incentives or other interventions. For example, different tariffs were considered for some households in the Irish CER dataset~\cite{Commission2012csm}. This means that their demand may not represent `normal', every-day behaviour.  

There are now some initiatives that are attempting to solve some of the issues of sparse and intermittent open data produced by limited innovation projects. An example of this is the Smart Energy Research Lab from UCL in the  UK\footnote{\url{https://www.ucl.ac.uk/bartlett/energy/research/energy-and-buildings/smart-energy-research-group-serg}} which is attempting to make smart meter data (as well as other useful data sets associated with the same homes) available on an ongoing basis for research from a wider range of participants as well as a large control group of households which will not participate in any initiatives or trials. 

\subsection{Data resolution}
Resolution is another important aspect of the data. Half-hourly data is the standard resolution of smart meter data although it can be as low as 10 or 15 minutes. The smallest stated resolution of the data within the papers had 48 with hourly data, 51 half-hourly, 23 as 15 minutes and 8 as 10 minutes. There was also a large number of papers where the resolution was not clear or no real data was presented (62 papers). Data with resolutions of between ten minutes and an hour are probably sufficient for demand control applications and are likely representative of what data is available in practice. However, this isn't sufficient for more high-resolution applications such as voltage control. In fact, only 11 papers considered data of resolution of 1 minute or less. This could pose a difficulty for validating the common voltage and Var control application in this review (see Section~\ref{sec:LVLF-applications}). A large number of papers where the resolution is not clear should also be a concern, as this prevents recreation of the results. 

\subsection{Forecast horizon}
Another crucial aspect of this review is the forecast horizon. Different horizons are useful for different applications. Short term (day to the week ahead) are typical for operational time scales, whereas long-term forecasts (over a year) are more useful for planning. The majority of the papers reviewed were at a short term time scale with 80 of the papers considering day ahead forecasts, another common horizon was an hour ahead. Very few papers went at shorter horizons than an hour (twelve). There were slightly more papers that forecast beyond a day (16 were between 2 days and a week ahead) and only 13 papers were at horizons of a month or more. Once again there was a large number of papers, 80 in total, where the horizon length was not identifiable. 

\subsection{Overview of LV datasets}
As we have already mentioned, the current choice of open datasets that can be used for a benchmark is very limited, relying mostly on the CER Irish smart meter data, which is now a decade old and has some selection bias limitations (most of the houses have 3-4 bedrooms etc.). In order to continually expand research in this area, a strategy is required to regularly open more diverse datasets, converging towards common formats and standards and clarifying licences and terms of usage. Clear license information is especially relevant for industry-based research. We have established a list of open datasets (see Table~\ref{tab:datasets} and \url{https://low-voltage-loadforecasting.github.io/}) with the hope that it will continue to grow, and that new methods for privacy and safety protection (anonymisation, aggregation, synthetic 'look alike' datasets etc.) will enable more availability of datasets in the future. Finally it is vital to provide proper and thorough documentation with the data sets. The quality of the datasets is not clear and in many cases any preprocessing or data-cleaning techniques are not provided with the data.

Most datasets are at the residential level collected from smart meters. More diverse datasets of non-residential customers and different grid levels (substations and transformers) are needed for better LV forecasting research. 

Some datasets that have been cited in the literature like PLAID~\cite{Medico2020ava} OCTES, BLUED~\cite{Anderson2012baf},  and DRED~\cite{UttamaNambi2015lle} were offline at the time of writing. The Pecanstreet Dataport~\cite{Pecan2018d} database was once publicly available for research but then closed, and now only a subset is still accessible. Other datasets are available for download but are hard to trace, as identifiers like a DOI or a paper to cite are not available. All of these obstruct the reproducibility of the research. Therefore, new datasets should be published on archiving platforms like IEEE data port~\footnote{\url{https://ieee-dataport.org/}}, Zenodo~\footnote{\url{https://zenodo.org/}}, Figshare~\footnote{\url{https://figshare.com/}}, or  arXiv~\footnote{\url{https://arxiv.org/help/submit\#datasets}}. A recent contribution to more reproducible time series research is the Monash Time Series Forecasting Archive~\cite{godahewa2021mts}. It contains the dataset of the UK Low Carbon London trial~\cite{UK2014ulc} and the UCI datasets~\cite{candanedo2017ddp, Hebrail2012ihe}.


\begin{sidewaystable*}
	\tiny
	\caption{Overview of Low-voltage Load Datasets (see online version for embedded hyperlink to the data set by clicking on the name).} \label{tab:datasets}
	\resizebox{!}{0.9\height}{
		\begin{tabular}{p{0.16\linewidth}p{0.05\linewidth}p{0.04\linewidth}p{0.05\linewidth}p{0.04\linewidth}p{0.04\linewidth}p{0.02\linewidth}p{0.02\linewidth}p{0.12\linewidth}p{0.14\linewidth}p{0.14\linewidth}}
			\\
			\toprule
			Name & Type  & No. Customers & Resolution & Duration & Intervention & Sub-metering & Weather avail. & Location & Other data provided & Access/Licence \\
			\midrule 
			\href{https://www.ea.tuwien.ac.at//projects/adres_concept/EN/}{ADRES}~\cite{einfalt2011kfa}
			& Households & 30    & 1 s   & 2 weeks & None  & No    & No    & Austria (Upper Austria) & Voltage & Free for Research (E-Mail) \\
			\href{https://www.ausgrid.com.au/Industry/Our-Research/Data-to-share/Solar-home-electricity-data}{Ausgrid Solar Home} &  Households &  300 & 30 min &   3 years & None &  No &  No &  Australia (NSW) &  &   No Licence \\
			\href{https://www.ausgrid.com.au/Industry/Our-Research/Data-to-share/Distribution-zone-substation-data}{Ausgrid substation data} &   Substation & 225 & 15 min &  20 years & None &  No &  No &  Australia (NSW) &  & No Licence \\
			\href{https://sourceforge.net/projects/greend/}{GREEND Electrical Energy Dataset (GREEND)}~\cite{monacchi2014gae} & Households & 8     & 1 s   & 3-6 months & None  & Yes   & No    & Austria, Italy & Occupancy, Building type & Free (Access Form) \\
			\href{https://archive.ics.uci.edu/ml/datasets/Appliances+energy+prediction}{UCI Appliances}~\cite{candanedo2017ddp} & Households & 1     & 10 min & 4.5 months & None  & No    & Yes   & Belgium (Mons) & Lights, Building information & Free (No Licence) \\
			\href{https://ieee-dataport.org/open-access/industrial-machines-dataset-electrical-load-disaggregation}{INDUSTRIAL MACHINES}\cite{Bandeira2018imd}
			& Industrial & 1 & 1 Hz  & 1 month & None  & Yes   & No    & Brasil (Minas Gerais) &       & CC BY \\
			\href{https://dataverse.harvard.edu/dataset.xhtml?persistentId=doi:10.7910/DVN/ZJW4LC}{Rainforest Automation Energy} \cite{Makonin2018rtr} & Households & 2     & 1 Hz  & 2 months & None  & Yes   & Yes   & Canada & Environmental, Heat Pump,  & CC BY \\
			\href{https://dataverse.harvard.edu/dataset.xhtml?persistentId=doi\%3A10.7910/DVN/FIE0S4\%20}{AMPds2} \cite{Makonin2016ata, Makonin2016ewa}  & Households & 1     & 1min  & 2 years & None  & Yes   & Yes   & Canada (Alberta) & Gas, Water, Building Type and Plan & CC BY \\
			\href{https://carleton.ca/sbes/publications/electric-demand-profiles-downloadable/}{Sustainable Building Energy Systems 2017} \cite{Johnson2017edf} & Households & 23    & 1 min & 1 year & None  & Yes   & No    & Canada (Ottawa) & Sociodemographic (Occupants, Age, Size) & Free (Attribution, E-Mail) \\
			\href{https://carleton.ca/sbes/publications/electric-demand-profiles-downloadable/}{Sustainable Building Energy Systems 2013} \cite{Saldanha2012mee} & Households & 12    & 1 min & 1 year & None  & Yes   & No    & Canada (Ottawa) & Sociodemographic (Occupants, Age, Size) & Free (Attribution, E-Mail) \\
			\href{https://data.lab.fiware.org/organization/9569f9bd-42bd-414f-b8d9-112553ea9dfb?tags=FINESCE}{FINESCE Horsens}
			& Households & 20    & 1 h   & several days & None  & Yes   & Yes   & Denmark (Horsens) & EV, PV, Heat Pump, Heating, Smart Home,  & CC BY-SA \\
			
			\href{https://archive.ics.uci.edu/ml/datasets/Individual+household+electric+power+consumption}{UCI Individual household electric power cons.}~\cite{Hebrail2012ihe} & Households & 1     & 1min  & 4 years & None  & Yes   & No    & France (Sceaux) & Reactive Power, Voltage & CC BY \\
			\href{https://mediatum.ub.tum.de/1375836}{BLOND-50}~\cite{Kriechbaumer2017bbo} & Commerical & 1     & 50 kHz & 213 days & None  & Yes   & No    & Germany &       & CC BY \\
			\href{https://mediatum.ub.tum.de/1375836}{BLOND-250}~\cite{Kriechbaumer2017bbo} & Commerical & 1     & 250 kHz & 50 days & None  & Yes   & No    & Germany &       & CC BY \\
			\href{https://zenodo.org/record/3855575\#.YKQgGKgzaUk}{Fresh Energy}~\cite{Beyertt2020fzb} & Households & 200   & 15 min & 1 year & Behaviorial & Yes   & No    & Germany & Agegroup, Gender of main customer & CC BY \\
			\href{https://data.lab.fiware.org/organization/9569f9bd-42bd-414f-b8d9-112553ea9dfb?tags=FINESCE}{FINESCE Factory} 
			& Industrial & 1     & 1 min & 2 days & None  & Yes   & No    & Germany (Aachen) & Machines & CC BY-SA \\
			\href{https://pvspeicher.htw-berlin.de/wp-content/uploads/MFH-Lastprofil_2014_17274_kWh.csv}{HTW Lichte Weiten}~\cite{htw2019ldb}
			& Households & 1 building & 15 minute & 1 year & None  & No    & No    & Germany (Berlin) &       & Free (No Licence) \\
			\href{https://pvspeicher.htw-berlin.de/veroeffentlichungen/daten/lastprofile/}{HTW Synthetic}~\cite{Tjaden2015rel} & Households & 74    & 1 s   & 1 year & None  & No    & No    & Germany (Representative) & Synthetic dataset merging & CC BY-NC \\
			\href{https://data.open-power-system-data.org/household_data/}{CoSSMic} \cite{Open2020dph} & Households, SME & 11    & 1min, 15min, 1H & 1-3 years & None  & Yes   & No    & Germany (South) & PV, EV, Type (Residential/SME) & CC BY \\
			\href{https://im.iism.kit.edu/sciber.php}{SciBER}~\cite{Staudt2018san} & Municipal & 107   & 15min & 3 years & None  & No    & No    & Germany (South) & Type (Office, Gym, ...) & CC BY \\
			\href{https://iawe.github.io/}{iAWE}~\cite{batra2013idi}
			& Households & 1     & 1 Hz  & 2 months & None  & Yes   & No    & India (New Delhi) & Water & Free (No Licence) \\
			\href{https://combed.github.io/}{COMBED}~\cite{Batra2014aco} & Commerical & 1     & 30 s  & 1 month & None  & Yes   & No    & India (New Delhi) &       & Free (No Licence) \\
			\href{http://www.ucd.ie/issda/data/commissionforenergyregulationcer/}{Irish CER Smart Metering Project data}~\cite{Commission2012csm} & Households, SME, Other & 3835  & 30min & 1.5 years & Tariff & No    & No    & Ireland & Type (Residential/SME/Other) & Free (Signed Access Form) \\
			
			\href{https://github.com/Nikasa1889/ShortTermLoadForecasting}{Hvaler Substation Level data}~\cite{DangHa2017lst} & Substation & 20    & 1 h   & 2 years & None  & No    & No    & Norway (Hvaler) &       & Free (No Licence) \\
			\href{http://web.lums.edu.pk/~eig/CXyzsMgyXGpW1sBo}{Energy Informatics Group Pakistan}~\cite{Pereira2014sap} & Households & 42    & 1 min & 1 year & None  & Yes   & No    & Pakistan & Sociodemographic (building properties, no of people, devices) & Free (No Licence) \\
			\href{https://archive.ics.uci.edu/ml/datasets/ElectricityLoadDiagrams20112014}{UCI Electricity Load Diagrams}~\cite{Godahewa2021ehd}
			& Different & 370   & 15 min & 2 years & None  & No    & No    & Portugal &       & Free (No Licence) \\
			
			\href{http://www.vs.inf.ethz.ch/res/show.html?what=eco-data}{Electricity Consumption and Occupancy (ECO)}~\cite{Christian2014ted, Wilhelm2015hom} & Households & 6     & 1 Hz  & 8 months & None  & Yes   & No    & Switzerland & Occupancy & CC BY \\
			\href{https://www.gov.uk/government/publications/household-electricity-survey--2}{Household Electricity Survey (HES)}~\cite{Zimmermann2012hes} & Households & ~{}250 & 2 min & 1 month (255) to 1 year (26) & None  & Yes   & No    & UK    & Consumer Archetype & Request \\
			\href{https://beta.ukdataservice.ac.uk/datacatalogue/studies/study?id=8634}{METER}~\cite{Grunewald2019muh} & Households & 529   & 1 min & 28 hours & None  & No    & No    & UK    & Activity data, Sociodemographic & Free for Research (Access Form) \\
			\href{https://datashare.ed.ac.uk/handle/10283/3647}{IDEAL Household Energy Dataset}~\cite{goddard2020ihe}
			& Households & 255   & 1 s     & 3 years & None  & Yes   & No    & UK    & Smart Home, Sociodemographic, energy awareness survey, room temperature and humidity, building characteristics & CC BY \\
			\href{http://www.networkrevolution.co.uk/resources/project-data/}{Customer-Led Network Revolution project data}~\cite{sidebotham2015cln} & Households, SMEs & ~{}12000 & 30 min & > 1 year & Time of Use & No    & No    & UK    & EV, PV, Heatpump, Tariff,  & CC BY-SA \\
			\href{https://jack-kelly.com/data/}{UK Domestic Appliance-Level Electricity (UK-DALE)}~\cite{Jack2015tud} & Households & 5     & 16 kHz, 1s & months, one house > 4 years & None  & Yes   & No    & UK (London area) &       & CC BY \\
			\href{https://data.london.gov.uk/dataset/smartmeter-energy-use-data-in-london-households}{UK Low Carbon London}~\cite{UK2014ulc,Godahewa2021lsm} & Households & 5567  & 30min & 2 years & Time of Use & No    &  No   & UK (London) & CACI Acorn group & Free (No Licence) \\
			\href{https://www.refitsmarthomes.org/datasets/}{REFIT}~\cite{Murray2016rel, Murray2017ael} & Households & 20    & 8 s   & 2 years & None  & Yes   & Yes   & UK (Loughborough) & PV, Gas, Water, Sociodemographic (Occupancy, Dwelling Age, Dwelling Type, No. Bedrooms) & CC BY \\
			\href{https://www.spenergynetworks.co.uk/pages/flexible_network_data_share.aspx}{Flexible Networks for a Low Carbon Future} & Substations & Several Secondary  & 30 min & 1 year & None  & No    & No    & UK (St Andrews, Whitchurch, Ruabon) &       & Free (Access Form) \\
			\href{https://ukerc.rl.ac.uk/DC/cgi-bin/edc_search.pl?GoButton=Detail\&WantComp=146\&\&RELATED=1}{NTVV Substations} & Substation & 316   & 5 s   & > 4 years & None  & No    & No    & UK (Thames Valley) &       & Open Access (Any purpose) \\
			\href{https://ukerc.rl.ac.uk/DC/cgi-bin/edc_search.pl?GoButton=Detail\&WantComp=147\&\&RELATED=1}{NTVV Smart Meter} & Buildings & 316   & 30 min & > 4 years & None  & No    & No    & UK (Thames Valley) &       & Open Access (Any purpose) \\
			
			\href{https://site.ieee.org/pes-iss/data-sets/}{IEEE PES Open Data Sets} 
			& Households, Commercial & 15    & 1 min, 5 min, 15 min & 2 weeks & None  & No    & No    & USA   & Connection limit & Free (No Licence) \\
			\href{http://redd.csail.mit.edu/}{Reference Energy Disaggregation Data Set (REDD)}~\cite{Kolter2011rap} & Households &  ~{}10 & 1 kHz & 3-19 days & None  & Yes   & No    & USA (Boston) & Voltage & Free (Attribution, E-Mail) \\
			\href{http://wzy.ece.iastate.edu/Testsystem.html}{Iowa Distribution Test Systems}~\cite{Bu2019atd} & Substation & 240 nodes & 1 H   & 1 year & None  & Yes   & No    & USA (Iowa) & Grid data & Free (Attribution) \\
			\href{https://www.pecanstreet.org/dataport/}{Pecanstreet Dataport (Academic)}~\cite{Pecan2018d} & Households & 30    & 1min, 15min, 1H & 2-3 years & None  & Yes   & Yes   & USA (mostly Austin and Boulder) & PV, EV, Water, Gas, Sociodemographic & Free for Research (Access Form) \\
			
			\href{https://neea.org/resources/rbsa-ii-combined-database}{Residential Building Stock Assessment}~\cite{Larson2014jua} & Households & 101   & 15 min & 27 months & None  & Yes   & No    & USA (North West Region) & Building Type (Single Family, Manufactured, Multifamily) & Free (Access Form) \\
			
			\href{http://lass.cs.umass.edu/projects/smart/}{SMART* Home 2017}~\cite{Barker2012sao} & Households & 7     & 1 Hz  & > 2 years & None  & Yes   & Yes   & USA (Western Massachussets) &       & Free (No Licence) \\
			\href{http://lass.cs.umass.edu/projects/smart/}{SMART* Apartment}~\cite{Barker2012sao} & Households & 114   & 1 min & 2 years & None  & No    & Yes   & USA (Western Massachussets) &       & Free (No Licence) \\
			\href{http://lass.cs.umass.edu/projects/smart/}{SMART* Occupancy}~\cite{Barker2012sao} & Households & 2     & 1 min & 3 weeks & None  & No    & No    & USA (Western Massachussets) & Occupancy & Free (No Licence) \\
			\href{http://lass.cs.umass.edu/projects/smart/}{SMART* Microgrid}~\cite{Barker2012sao} & Households & 443   & 1 min & 1 day & None  & No    & No    & USA (Western Massachussets) &       & Free (No Licence) \\
			\href{http://lass.cs.umass.edu/projects/smart/}{SMART* Home 2013}~\cite{Barker2012sao} & Households & 3     & 1 Hz  & 3 months & None  & Yes   & No    & USA (Western Massachussets) & Solar, Wind, Environmental, Smart Home, Voltage,  & Free (No Licence) \\
			\bottomrule
		\end{tabular}
	}
\end{sidewaystable*}


\paragraph{Sentiment Analysis}
For the sentiment analysis classification task, we follow \citep{blitzer-etal-2007-biographies} and \citep{ruder2018strong} by randomly selecting 6 Amazon multi-domain review datasets, as well as Yelp reviews \citep{asghar2016yelp} and IMDB movie reviews datasets \citep{maas-EtAl:2011:ACL-HLT2011}.~\footnote{\url{https://jmcauley.ucsd.edu/data/amazon/}, \url{https://www.yelp.com/dataset}, \url{https://ai.stanford.edu/~amaas/data/sentiment/}.}
Altogether, these datasets exhibit wide diversity based on review length and topic (see Table~\ref{datasets}).
We normalize all datasets to have 5 sentiment classes: very negative, negative, neutral, positive, and very positive. 
We sample 50k examples for training, 5k for validation, and 5k for testing.

\respace
\paragraph{Experimental Setup}
\label{sec:ex-setup}
To evaluate methods for the multi-domain active learning task, we conduct the experiment described in Section~\ref{sec:task} for each acquisition method, rotating each domain as the target set.
Model $M$, a BERT-Base model ~\citep{devlin2019bert}, is chosen via hyperparameter grid search over learning rate, number of epochs, and gradient accumulation.
The large volume of experiments entailed by this search space limits our capacity to benchmark performance variability due to isolated factors (the acquisition method, the target domain, or fine-tuning final models).
However, our hyper-parameter search closely mimics the process of an ML practitioner looking to select a best method and model, so we believe our experiment design captures a fair comparison among methods.
See Algorithm~\ref{alg:experiment} in Appendix Section \ref{sec:appendix-expdesign} for full details.




% !TEX root = 0-qqQQmain.tex

\section{Results} \label{results}

After the ultraviolet and infrared pole subtraction described in the previous section, we arrive at the main result of this paper, the fully analytical expressions for the finite remainders of the helicity amplitudes for process \eqref{s_channel}.
As previously mentioned,  the helicity amplitudes for other $2\to2$ quark processes with different initial states, including the equal-flavour case $q=Q$, can be obtained by a combination of analytical continuation and momenta renaming from the ones for our main process \eqref{s_channel}.  We  discuss this in more detail in Section \ref{extra_results}.
We provide all finite remainders  in electronic format as ancillary files attached
to the \texttt{arXiv} submission of this manuscript.

\subsection{Checks}
We have performed various checks on our results. First of all, we have verified that the IR poles of our scattering
amplitudes  follow the pattern predicted in refs~\cite{Becher:2009qa,Becher:2009cu,Almelid:2015jia} up to three loops.
We have then checked the finite part of our one loop amplitudes for all different partonic channels against the automated one-loop generator \texttt{OpenLoops}~\cite{Cascioli:2011va,Buccioni:2019sur}.
Finally, we have checked our one- and two-loop amplitudes through to order $\epsilon^4$ and $\epsilon^2$, respectively,
 against the results presented in refs~\cite{Glover:2004si,Ahmed:2019qtg}.

In order to successfully perform this check, one has to pay particular attention when comparing the amplitudes before IR subtraction, 
due to a subtlety in the  dimensional regularisation scheme 
used in refs~\cite{Glover:2004si,Ahmed:2019qtg}. This is due to the fact that the tensor structures
used to decompose the scattering amplitude in those references contain an explicit dependence on the dimensional-regulation
parameter $\epsilon$, even if the external states are taken to be in four dimensions, as in the 't Hooft-Veltman prescription.
While ignoring this dependence does not change the finite remainder of the scattering amplitudes after UV and IR poles have been subtracted,
it does change the bare results.
We illustrate this point explicitly for the one loop case. 
In refs \cite{Glover:2004si,Ahmed:2019qtg},  the following four tensors are used to decompose the amplitude at one loop
in CDR\footnote{Note that here even $\widetilde T_1$  and  $\widetilde T_2$ do not coincide with our definitions.}

\begin{align} 
\label{tensors_henn_D1} \widetilde T_1 &= {\bar u} (p_2)\: \slashed p_3 \: { u} (p_1) \times   {\bar u} (p_4) \: \slashed p_1 \:{ u}(p_3) \; , \\
\label{tensors_henn_D2} \widetilde T_2 &= {\bar u} (p_2)\: \gamma^\alpha \: { u} (p_1) \times {\bar u} (p_4) \: \gamma_\alpha \:{ u}(p_3) \; , \\
\label{tensors_henn_D3} \widetilde T_3 &= {\bar u} (p_2)\: \slashed p_3   \gamma^\mu   \gamma^\nu \: { u} (p_1) \times   {\bar u} (p_4) \:
 \slashed p_1 \gamma_\mu  \gamma_\nu  \:{ u}(p_3) \; , \\
\label{tensors_henn_D4} \widetilde T_4 &= {\bar u} (p_2)\: \gamma^\alpha \gamma^\mu  \gamma^\nu \: { u} (p_1) \times {\bar u} (p_4) \: \gamma_\alpha  \gamma_\mu   \gamma_\nu \:{ u}(p_3) \; , 
\end{align} 
where we stress that this decomposition is loop dependent and is only sufficient up to one-loop order, $L \leq 1$
\begin{equation}\label{decomp_tensors_henn}
\bar{\mathbfcal{A}}^{L \leq 1}  =   \widetilde{\mathbfcal{F}}_1 \; \widetilde{T}_1 + \widetilde{\mathbfcal{F}}_2 \; \widetilde{T}_2 + \widetilde{\mathbfcal{F}}_3 \; \widetilde{T}_3 + \widetilde{\mathbfcal{F}}_4 \; \widetilde{T}_4.
\end{equation}

Importantly, in the 't Hooft-Veltman scheme the vector indices of the $\gamma$ matrices in eqs.~\eqref{tensors_henn_D1}---\eqref{tensors_henn_D4} that are not 
explicitly contracted with four-dimensional vector fields 
are in general to be taken in 
$d$ dimensions. While this makes no difference for the 
first two tensors, the second two~(\ref{tensors_henn_D3},\ref{tensors_henn_D4})  
depend explicitly on $d$ and are responsible for an ambiguity 
in the way the helicity amplitudes are defined.
Let us consider for example the fourth tensors, eq.~\eqref{tensors_henn_D4}. 
If we consider the $\gamma$ matrices to carry $d$ dimensional vector indices we can
split the four dimensional part from the
$\epsilon$-dependent one by writing 
$\gamma^\mu_d = \gamma^\mu_4 + \gamma^{\mu}_{-2 \epsilon}$ and then use the equation 
\begin{equation}
    {\rm Tr}[\gamma_4^{\mu_1} \dots \gamma_4^{\mu_n} \gamma_{-2\epsilon}^{\nu_1} \dots \gamma_{-2\epsilon}^{\nu_m} ] = \frac{1}{4}     {\rm Tr}[\gamma_4^{\mu_1} \dots \gamma_4^{\mu_n} ]     {\rm Tr}[ \gamma_{-2\epsilon}^{\nu_1} \dots \gamma_{-2\epsilon}^{\nu_m} ] 
\end{equation}
as done in ref.~\cite{Cullen:2010jv} to extract the $(-2\epsilon)$-dimensional dependence of the $\gamma$-strings. All traces can then be evaluated as usual using the Clifford algebra relation $\{\gamma_\mu,\gamma_\nu\} = 2 g_{\mu \nu}$ for $d$-dimensional indices $\mu,\nu$. In this case, the dimensional splitting procedure simply amounts to $\epsilon$ dependent coefficients of the 4-dimensional $\gamma$-strings. For instance, taking the first string of $\widetilde{T}_4$ we find  
\begin{align}
    {\bar u} (p_2)\: \gamma^\alpha \gamma^\mu  \gamma^\nu \: { u} (p_1)& = g_{-2\epsilon}^{\mu \nu}  {\bar u} (p_2)\: \gamma_4^\alpha \: { u} (p_1) \nonumber \\
    &+ g_{-2\epsilon}^{\mu \alpha}  {\bar u} (p_2)\: \gamma_4^\nu \: { u} (p_1) \nonumber \\ 
    &-  g_{-2\epsilon}^{\nu \alpha}  {\bar u} (p_2)\: \gamma_4^\mu \: { u} (p_1),
\end{align}
where ${\rm Tr}[\gamma_{-2\epsilon}^\mu \gamma_{-2\epsilon}^\nu] = g_{-2\epsilon}^{\mu \nu} $ is the $(-2\epsilon)$-dimensional part of the metric tensor. Repeating the exercise for the other fermion string and then fixing the external helicities to $(+,-,+,-)$ 
we find
\begin{equation}
\widetilde T_4|_{(+,-,+,-)} = (32 - 12 \epsilon) \; [24]\langle 3 1 \rangle\,.
\end{equation}
Similarly, by repeating the same exercise for both helicities as we did 
in eqs.~\eqref{H_ij},  
but this time starting from the tensor decomposition in eq.~\eqref{decomp_tensors_henn},
we find
\begin{align}
\mathbfcal{H}_1 &=  -tu\: \left[ \widetilde{\mathbfcal{F}}_1 + 8 \widetilde{\mathbfcal{F}}_3 \right]
+2 t\left[ \widetilde{\mathbfcal{F}}_2 + 16 \widetilde{\mathbfcal{F}}_4 \right]  
+ 2 \epsilon \left[ t u\: \widetilde{\mathbfcal{F}}_3 - 6t \widetilde{\mathbfcal{F}}_4 \right] \: ,  \label{matrix_H1}\\
\mathbfcal{H}_2 &=  -tu \: \left[ \widetilde{\mathbfcal{F}}_1 + 4 \widetilde{\mathbfcal{F}}_3\right] 
- 2 u\left[ \widetilde{\mathbfcal{F}}_2 + 4 \widetilde{\mathbfcal{F}}_4 \right] 
+ 2\epsilon \left[ t u \:\widetilde{\mathbfcal{F}}_3 + 6 u\:\widetilde{\mathbfcal{F}}_4 \right] \; .\label{matrix_H2}
\end{align}
It is instructive to compare these formulas to the corresponding ones 
obtained in our approach, see eq.~\eqref{H1H2def}.
Our expressions for the helicity amplitudes, despite not displaying 
any explicit dependence on the parameter
$\epsilon$,  are exact in the 't Hooft-Veltman scheme. 
Notice, in particular, that the two tensors onto which we decompose the
amplitude,
${T}_1$ and ${T}_2$ in eq.~\eqref{tensors}, 
have been chosen such that it makes no practical difference
whether the $\gamma$ algebra to fix the helicity amplitudes
is performed $4$ or in $d=4 - 2 \epsilon$ dimensions. 
Upon substituting the form factors provided in refs~\cite{Glover:2004si,Ahmed:2019qtg} in eqs.~\eqref{matrix_H1} and \eqref{matrix_H2} (and in the corresponding
generalisations for the two-loop corrections),
we find perfect agreement up to weight six with 
our results for the \emph{bare helicity amplitudes}.

We stress, nevertheless, that the results for the bare helicity amplitudes as provided in refs~\cite{Glover:2004si,Ahmed:2019qtg} are obtained 
by setting $\epsilon=0$ in the coefficients of the form factors 
of \eqref{matrix_H1} and \eqref{matrix_H2} before substituting
the results for the form factors. This amounts to having assumed that the $\gamma$ matrices in eq.~\eqref{decomp_tensors_henn} are purely four-dimensional.  
This produces a difference for the bare amplitudes with respect to ours 
of order $\mathcal{O}(\epsilon)$ 
at one loop and $\mathcal{O}(1/\epsilon)$ at 
two loops.\footnote{Note that in this approach, one needs two more tensors 
at two loops, which depend quadratically on $\epsilon$.}  
However, it is easy to see that, as long as this choice is made 
consistently to all orders,  one obtains the same results for the 
finite remainders in $d=4$. 
In fact, one can imagine to first subtract UV and IR poles at the level of 
the individual form factors $\widetilde{\mathbfcal{F}}_j$ and, only afterwards,
substitute the finite form factors in eqs.~\eqref{matrix_H1} and \eqref{matrix_H2}, and fix $\epsilon = 0$. 
It is then obvious that the finite remainder cannot depend on 
the $\epsilon$-suppressed contributions in eqs.~\eqref{matrix_H1} 
and \eqref{matrix_H2}.
 We have verified the last statement directly, finding perfect agreement with refs~\cite{Glover:2004si,Ahmed:2019qtg} at the level
 of the finite remainders. 

\subsection{Numerical Evaluation}

\begin{figure}
\center
\includegraphics[width=1\textwidth]{fig/allH_1.pdf}
\caption{Real (left) and imaginary (right) parts of the form factors $\mathbfcal{H}_{1,\text{fin}}^{[i],(L)}$ relevant for helicities $(+,-,+,-)$.   Colour components $[i]$ and number of loops $(L)$ are specified in the legends.}
\label{allH_1}
\end{figure} 

\begin{figure}
\center
\includegraphics[width=1\textwidth]{fig/allH_2.pdf}
\caption{Real (left) and imaginary (right) parts of the form factors $\mathbfcal{H}_{2,\text{fin}}^{[i],(L)}$ relevant for helicities $(+,-,-,+)$.   Colour components $[i]$ and number of loops $(L)$ are specified in the legends. }
\label{allH_2}
\end{figure} 
We present numerical results for the finite form factors defined in  \eqref{one}, \eqref{two} and \eqref{three} calculated in the physical region $0<x<1$ for the process in eq.~\eqref{s_channel}, $q\bar{q}\to Q\bar{Q}$. To evaluate our results numerically we made use of the \texttt{Mathematica} package \texttt{PolyLogTools} \cite{Duhr:2019tlz},
which in turn uses the \texttt{Ginac} library~\cite{Bauer:2000cp,cln,Vollinga:2004sn}.
For the various parameters we use the following values:
\begin{equation}
N_c = 3, \quad n_f = 5, \quad \as = 0.118, \quad  \mu^2 = s.
\end{equation}
We show results for $\mathbfcal{H}_{1,\text{fin}}$ corresponding to helicities $(+,-,+,-)$ in figure~\ref{allH_1} and for $\mathbfcal{H}_{2,\text{fin}}$ corresponding to helicities $(+,-,-,+)$ in figure~\ref{allH_2}.
In the figures, we present the two colour components of the form factors individually, where we recall that our colour decomposition reads: 
\begin{equation}
{\mathbfcal{H}}_{i,\text{fin}}^{(L)}  
= 
\begin{pmatrix}
{\mathcal{H}}_{i,\text{fin}}^{[1],(L)}  \\
{\mathcal{H}}_{i,\text{fin}}^{[2],(L)} 
\end{pmatrix} \quad i=1,2 \; .
\end{equation}
Here, the colour index $[1]$ is related to the colour structure  ${\delta}_{ i_1 i_4} {\delta}_{i_2 i_3}$ while the index $[2]$ refers to the coefficient of ${\delta}_{ i_1 i_2} {\delta}_{ i_3 i_4}$.
Lastly, the index $(L)$ refers to the number of loops of the corresponding amplitude.  



\respace
\respace
\section{Conclusion}
\respace
\label{sec:conclusion}

We examine a challenging variant of active learning where target data is scarce, and multiple shifted domains operate as the source set of unlabeled data.
For practitioners facing multi-domain active learning, we benchmark 18 acquisition functions, demonstrating the $\mathcal{H}$-Divergence family of methods and our proposed variant \dale{} achieve the best results.
Our analysis shows the importance of example selection in existing methods, and also the surprising potential of domain budget allocation strategies.
Combining families of methods, or trying domain adaptation techniques on top of selected example sets, offer promising directions for future work.
\respace
\section*{Acknowledgements}
\respace
\label{sec:acknowledgments}

The authors would like to thank Moises Goldszmidt, Ehsan Mousavi, and Stephen Pulman for helpful discussions and feedback. 
\clearpage
% Entries for the entire Anthology, followed by custom entries
\bibliography{anthology,custom}
\bibliographystyle{acl_natbib}

\appendix

\appendix
\section{Appendix}
\subsection*{Acknowledgements}
Special thanks to Jeff Meredith for assisting with the website. 
The research is based upon work supported by the Department of Defense (DOD), Naval Information Warfare Systems Command (NAVWAR), via the Department of Energy (DOE) under contract  DE-AC05-00OR22725. The views and conclusions contained herein are those of the authors and should not be interpreted as representing the official policies or endorsements, either expressed or implied, of the DOD, NAVWAR, or the U.S. Government. The U.S. Government is authorized to reproduce and distribute reprints for Governmental purposes notwithstanding any copyright annotation thereon. 
\\
\begin{table}
%\vspace{-20pt}
\centering
\setlength\extrarowheight{3pt}
\begin{tabular}{m{2.95cm} m{2.75cm} m{2.75cm} m{2.75cm}}
\toprule
% >{\columncolor[HTML]{FFFFFF}}c |
% >{\columncolor[HTML]{FFFFFF}}c |
% >{\columncolor[HTML]{FFFFFF}}c |
% >{\columncolor[HTML]{FFFFFF}}c c}
\textbf{Raw Score} $[1-5]$ & \textbf{PR} $[0-1]$ & \textbf{ML} $[1-5]$ & \textbf{ML + PR} $[0-1]$\\
%   {\color[HTML]{000000} \textbf{\begin{tabular}[c]{@{}l@{}} PR $[0-1]$ \end{tabular}}}&
%   {\color[HTML]{000000} \textbf{\begin{tabular}[c]{@{}l@{}}ML $[1-5]$ \end{tabular}}} &
%   {\color[HTML]{000000} \textbf{\begin{tabular}[c]{@{}l@{}}ML + PR $[0-1]$ \end{tabular}}}
\hline
Tool 6 (4.363)  & Tool 0 (.341) & Tool 6 (4.23)  & Tool 6 (.207) \\ 
Tool 9 (4.071) & Tool 1 (.114)    & Tool 9 (4.06) & Tool 9 (.192) \\ 
Tool 0 (4.071)  & Tool 6 (.097)  & Tool 0 (4.04) & Tool 0 (.146)  \\ 
Tool 1 (3.813) & Tool 9 (.085) & Tool 5 (3.77)   & Tool 1 (.070)  \\ 
Tool 5 (3.750) & Tool 3 (.057)  & Tool 4 (3.75)  & Tool 5 (.061)   \\ 
Tool 4 (3.636)  & Tool 2 (.052) & Tool 1 (3.75)  & Tool 3 (.055)  \\ 
Tool 3 (3.571)  & Tool 4 (.048)  & Tool 3 (3.66) & Tool 4 (.054)  \\ 
Tool 2 (3.438)  & Tool 5 (.043)  & Tool 2 (3.47) & Tool 2 (.046)  \\
Tool 7 (3.273)  & Tool 7 (.042) & Tool 7 (3.44) & Tool 7 (.046)  \\ 
\bottomrule
\end{tabular}
\captionsetup{font = scriptsize}
\caption{Four methods used to report or derive overall ratings of the tools. Raw score: average of user-defined ratings; PR: average of overall ratings derived from PageRank algorithm on the raw data; ML: average of overall ratings derived from machine learning predictions on populated data; ML + PR: average of overall ratings derived from machine learning and PageRank algorithm on populated data.}
\label{tab:Overall Scores}
\end{table}


%\subsection{Qualtrics Survey}

\begin{center}
\begin{table}
\setlength\extrarowheight{4pt}
\begin{tabular}{  m{1.85cm}  m{2.5cm} m{8cm}  } 
 \toprule
  \textbf{Question} \# & \textbf{Question type} & \textbf{Question} \\
  \hline
  1 & Pre-Survey & How familiar are you with SOAR tools?\\ 
  2 &  Pre-Survey &  Which of these best fits your role?   \\ 
  3 & Pre-Survey & How many years have you been in that role? \\ 
  4 & Pre-Survey & Please rank the following capabilities in order of importance, with 1 being the most important and 7 being the least important in your SOC. \\ 
  \midrule
  1 & Familiarity & How familiar are you with this tool? \\
  2 & Quality & What do you think of the quality of these videos? \\ 
  3 & Overall Score & What is you overall impression of this tool? \\ 
  4 & Ranking & Does the tool present and prioritize data in a way that is beneficial? \\ 
  5 & Ingestion & Do you think this tool could effectively ingest the data in your SOC? \\ 
  6 & Playbooks & Does the tool provide steps (playbook, workflow) that guide tier 1 or junior analysts through common tasks? \\ 
  7 & Ticketing & Does the tool automate tasks in a way that would increase efficiency? \\ 
  8 & Collaboration & Does the tool enable multiple analysts to effectively collaborate (simultaneously)? \\ 
  9 & Automation & Does the tool enable a hand off of investigations (for example, between two shifts or across SOCs)? \\
  10 & Free response & Is there anything else about this tool that you would like to share? \\
  N/A & Overall Ranking & Please rank the tools that you reviewed by order of preference, where 1 indicates the tool that you would most like to see used in your SOC. You can drag and drop the tool names to reorder them. \\
  \bottomrule
  \end{tabular}
  \captionsetup{font = scriptsize}
  \caption{\label{tab:Survey}Survey questionnaire given to the SOC operators. The survey was delivered electronically and included 4 pre-survey questions, 10 questions about the specific tools (including their aspects), and 1 question about the overall ranking. All ratings questions were scored 1-5, with 1 being the worst and 5 being the best. On the two ranking questions, the operators ranked their most preferred aspect/tool as 1 and their least preferred as the highest value.}
  \end{table}
\end{center}



%Title,Survey,QuestionType,QuestionSortOrder,ResponseFieldName,SurveyStep
%"How familiar are you with SOAR tools?","Tool Survey","Familiar","1","Q1Answer","Demographics"
%"Which of these best fits your role?","Tool Survey","Roles","2","Q2Answer","Demographics"
%"How many years have you been in that role?","Tool Survey","Role Years","3","Q3Answer","Demographics"
%"Please rank the following capabilities in order of importance, with 1 being the most important and 7 being the least important in your SOC.","Tool Survey","Rank Capabilities","4","Q4Answer","Demographics"
%"How familiar are you with this tool?","Tool Survey","Familiar With","1","Q1Answer","Tool Survey"
%"What do you think of the quality of these videos?","Tool Survey","Quality","2","Q2Answer","Tool Survey"
%"What is you overall impression of this tool?","Tool Survey","Rating","3","Q3Answer","Tool Survey"
%"Does the tool present and prioritize data in a way that is beneficial?","Tool Survey","Rating","4","Q4Answer","Tool Survey"
%"Do you think this tool could effectively ingest the data in your SOC?","Tool Survey","Rating","5","Q5Answer","Tool Survey"
%"Does the tool provide steps (playbook, workflow) that guide tier 1 or junior analysts through common tasks?","Tool Survey","Rating","6","Q6Answer","Tool Survey"
%"Does the tool automate tasks in a way that would increase efficiency?","Tool Survey","Rating","7","Q7Answer","Tool Survey"
%"Does the tool enable multiple analysts to effectively collaborate (simultaneously)?","Tool Survey","Rating","8","Q8Answer","Tool Survey"
%"Does the tool enable a hand off of investigations (for example, between two shifts or across SOCs)?","Tool Survey","Rating","9","Q9Answer","Tool Survey"
%"Is there anything else about this tool that you would like to share?","Tool Survey","Text Area","10","Q10Answer","Tool Survey"


% \section{Qualtrics Survey}
% \label{appx:survey}

% \newcommand*{\clarify}[3]{
% 	\noindent \texttt{Q#1 shown if answer to Q#2 is #3.}
	
% 	\vspace{.5em}
% }

% \newcommand*{\questionS}[2]{
% 	\noindent
% 	\begin{minipage}{\columnwidth}
% 		\textbf{Q#1.} #2\\
% 	\end{minipage}
% }

% \newcommand*{\squestion}[3]{
% 	\noindent
% 	\begin{minipage}{\columnwidth}
% 		\textbf{Q#1.} #2\\
% 		\begin{itemize*}[before=\itshape,label={$\circ$}]
% 			#3
% 		\end{itemize*}
% 		\vspace{\baselineskip}	
% 	\end{minipage}
% }


% %Should the preamble be included?

% \questionS{1}{Please enter your Prolific Academic ID.}

% \squestion{2}{How frequently do you use instant messaging tools for group chat?}{
% 	\item Daily
% 	\item 4-6 times a week
% 	\item 2-3 times a week
% 	\item Once a week
% 	\item Rarely
% 	\item Never
% }

% \subsection{Tools}
% \squestion{3}{Please mark which of the following tools, if any, you have used. \textit{(select all that apply)}}{
% 	\item Blackberry Messenger
% 	\item Discord
% 	\item Facebook Messenger
% 	\item iMessage
% 	\item IMO
% 	\item Instagram Direct
% 	\item Kakaotalk
% 	\item Kik
% 	\item Line
% 	\item Marco Polo
% 	\item Signal
% 	\item Skype
% 	\item Slack
% 	\item Snapchat
% 	\item Telegram
% 	\item Viber
% 	\item WeChat
% 	\item WhatsApp
% 	\item N/A
% }

% \questionS{4}{For what purposes do you use instant messaging tools for group communication?}
% \questionS{5}{What, if anything, do you like about using instant messaging tools for group communication?}
% \questionS{6}{What, if anything, do you dislike about using instant messaging tools for group communication?}
% \questionS{7}{How do you choose which instant messaging tools to use for group communication?}

% \subsection{Group Dynamics: Participation}

% \squestion{8}{When using instant messaging tools for group communication, who do you talk to? \textit{(select all that apply)}}{
% 	\item Immediate/nuclear family members
% 	\item Extended family members
% 	\item Friends
% 	\item Work colleagues
% 	\item Other
% 	\item I prefer not to answer
% }

% \squestion{9}{On average, how large are your instant messaging groups?}{
% 	\item 3-5 people
% 	\item 6-10 people
% 	\item 11-20 people
% 	\item 20+ people
% 	\item My groups vary largely in size
% 	\item Unsure / I prefer not to answer
% }

% \clarify{10}{9}{\\``My groups vary largely in size''}
% \questionS{10}{Why do your instant messaging groups vary largely in size?}

\end{document}
