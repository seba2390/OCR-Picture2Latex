% This must be in the first 5 lines to tell arXiv to use pdfLaTeX, which is strongly recommended.
\pdfoutput=1
% In particular, the hyperref package requires pdfLaTeX in order to break URLs across lines.

\documentclass[11pt]{article}

% Remove the "review" option to generate the final version.
\usepackage{acl}

% Standard package includes
\usepackage{times}
\usepackage{latexsym}

% For proper rendering and hyphenation of words containing Latin characters (including in bib files)
\usepackage[T1]{fontenc}
% For Vietnamese characters
% \usepackage[T5]{fontenc}
% See https://www.latex-project.org/help/documentation/encguide.pdf for other character sets

% This assumes your files are encoded as UTF8
\usepackage[utf8]{inputenc}

% This is not strictly necessary, and may be commented out,
% but it will improve the layout of the manuscript,
% and will typically save some space.
\usepackage{microtype}

% Added by us:
\usepackage{times}
\usepackage{latexsym}
\renewcommand{\UrlFont}{\ttfamily\small}
\usepackage{footmisc}
\usepackage{float}
\usepackage{bbm}
\usepackage{adjustbox}
\usepackage{amssymb}
\usepackage{pifont}
\usepackage{microtype}
\usepackage{pdfpages}
% \usepackage[dvipsnames]{xcolor}
\usepackage[section]{placeins}
\usepackage{subcaption,amsfonts,dcolumn}
\usepackage{svg}

\usepackage{todonotes}
% \usepackage[disable]{todonotes}
\presetkeys{todonotes}{inline}{}
\usepackage{booktabs}
\usepackage{multirow}

\usepackage{graphicx}
\usepackage{amsmath}
\usepackage{makecell}
\usepackage{algorithmic}
\usepackage{algorithm}


% \newcommand\BibTeX{B\textsc{ib}\TeX}

\newcommand{\dal}{\textsc{DAL}}
\newcommand{\dals}{\textsc{DAL}$*$}
\newcommand{\dalt}{\textsc{DAL-T}}
\newcommand{\dale}{\textsc{DAL-E}}
\newcommand{\dalts}{\textsc{DAL-T}$*$}
\newcommand{\dales}{\textsc{DAL-E}$*$}
\newcommand{\conf}{\textsc{CONF}}
\newcommand{\confa}{\textsc{CONF}$\uparrow$}
\newcommand{\confd}{\textsc{CONF}$\downarrow$}
\newcommand{\entr}{\textsc{ENTR}}
\newcommand{\entra}{\textsc{ENTR}$\uparrow$}
\newcommand{\entrd}{\textsc{ENTR}$\downarrow$}
\newcommand{\eod}{\textsc{ENG}}
\newcommand{\eoda}{\textsc{ENG}$\uparrow$}
\newcommand{\eodd}{\textsc{ENG}$\downarrow$}
\newcommand{\bald}{\textsc{BALD}}
\newcommand{\balda}{\textsc{BALD}$\uparrow$}
\newcommand{\baldd}{\textsc{BALD}$\downarrow$}
\newcommand{\rca}{\textsc{RCA}}
\newcommand{\rcas}{\textsc{$\widetilde{RCA}$}}
\newcommand{\knn}{\textsc{kNN}}
\newcommand{\knnq}{\textsc{kNN-Q}}
\newcommand{\knnc}{\textsc{kNN-C}}
\newcommand{\knnqc}{\textsc{kNN-QC}}
\newcommand{\knns}{\textsc{kNN}$*$}
\newcommand{\knnqs}{\textsc{kNN}$*$\textsc{-Q}}
\newcommand{\knncs}{\textsc{kNN}$*$\textsc{-C}}
\newcommand{\knnqcs}{\textsc{kNN}$*$\textsc{-QC}}
\newcommand{\target}{\textbf{target}}
\newcommand{\source}{\textbf{source}}

\newcommand*\samethanks[1][\value{footnote}]{\footnotemark[#1]}
\newcommand{\specificthanks}[1]{\textdagger}% Inserts a specific \thanks symbol

\newcommand{\abr}[1]{\textsc{#1}}
\newcommand{\respace}{\vspace*{-.2cm}}
\newcommand{\cmark}{\ding{51}}
\newcommand{\xmark}{\ding{55}}
\newcommand{\argmax}{\mathop{\mathrm{argmax}}}

\newcommand{\julia}[1]{\textcolor{cyan}{\bf [Julia: #1]}}
\newcommand{\shayne}[1]{\textcolor{orange}{\bf [Shayne: #1]}}
\newcommand{\edward}[1]{\textcolor{red}{\bf [Edward: #1]}}
\newcommand{\yi}[1]{\textcolor{blue}{\bf [Yi: #1]}}

\definecolor{myred}{RGB}{219, 67, 37}
\definecolor{mygreen}{RGB}{0, 97, 100}

% If the title and author information does not fit in the area allocated, uncomment the following
%
%\setlength\titlebox{<dim>}
%
% and set <dim> to something 5cm or larger.

\title{Active Learning Over Multiple Domains in Natural Language Tasks}

% Author information can be set in various styles:
% For several authors from the same institution:
% \author{Author 1 \and ... \and Author n \\
%         Address line \\ ... \\ Address line}
% if the names do not fit well on one line use
%         Author 1 \\ {\bf Author 2} \\ ... \\ {\bf Author n} \\
% For authors from different institutions:
% \author{Author 1 \\ Address line \\  ... \\ Address line
%         \And  ... \And
%         Author n \\ Address line \\ ... \\ Address line}
% To start a seperate ``row'' of authors use \AND, as in
% \author{Author 1 \\ Address line \\  ... \\ Address line
%         \AND
%         Author 2 \\ Address line \\ ... \\ Address line \And
%         Author 3 \\ Address line \\ ... \\ Address line}

\author{
  Shayne Longpre\thanks{\;\;Equal contribution.}\, \thanks{\;\;Work performed while at Apple.} \\
  \href{mailto:slongpre@mit.edu}{\tt slongpre@mit.edu}
  \And
  Julia Reisler\footnotemark[1] \\
  \href{mailto:jreisler@apple.com}{\tt jreisler@apple.com}
  \AND
  Edward Greg Huang\footnotemark[1]\:\:\footnotemark[2] \\
  \href{mailto:eghuang@berkeley.edu}{\tt eghuang@berkeley.edu}
  \And 
  Yi Lu\footnotemark[1]\:\:\footnotemark[2] \\
  \href{mailto:yilu331@gmail.com}{\tt yilu331@gmail.com}
  \AND
  Andrew Frank 
  \And
  Nikhil Ramesh \And
  Chris DuBois \AND
  \normalfont{Apple}
 \tt
%   \href{mailto:slongpre@mit.edu}{\tt slongpre@mit.edu}\tt{, \{}\href{mailto:jreisler@apple.com}{jreisler}\tt , 
%  \href{mailto:eghuang@apple.com}{\tt eghuang}\tt , 
%  \href{mailto:ylu7@apple.com}{\tt yilu331}\tt \}{\tt @apple.com}
  }

\begin{document}
\maketitle

\abstract{
\gls*{alpr} systems have shown remarkable performance on \glspl*{lp} from multiple regions due to advances in deep learning and the increasing availability of datasets.
The evaluation of deep \gls*{alpr} systems is usually done within each dataset; therefore, it is questionable if such results are a reliable indicator of generalization ability.
In this paper, we propose a traditional-split \textit{versus} leave-one-dataset-out experimental setup to empirically assess the cross-dataset generalization of $12$ \gls*{ocr} models applied to \gls*{lp} recognition on nine publicly available datasets with a great variety in several aspects (e.g., acquisition settings, image resolution, and \gls*{lp} layouts).
We also introduce a public dataset for end-to-end \gls*{alpr} that is the first to contain images of vehicles with Mercosur \glspl*{lp} and the one with the highest number of motorcycle images.
The experimental results shed light on the limitations of the traditional-split protocol for evaluating approaches in the \gls*{alpr} context, as there are significant drops in performance for most datasets when training and testing the models in a leave-one-dataset-out~fashion.
\ifscitepress
\else
\vspace{3mm}
\fi
}
% !TEX root = 0-20SPAWC_SAND.tex
% DO NOT REMOVE THE ABOVE COMMENT!
\section{Introduction}
%
Millimeter-wave (mmWave) and massive multi-user (MU) multiple-input multiple-output (MIMO) will be  core technologies for future wireless systems~\cite{larsson14a, rappaport15a}.
%
The combination of these technologies enables simultaneous communication to multiple user equipments (UEs) at unprecedentedly high data rates. 
%
These advantages come at the cost of significantly increased power consumption, implementation complexity, and system costs. A viable solution to address these challenges is the use of low-resolution data converters combined with sophisticated but efficient baseband processing algorithms in all-digital basestations (BS) architectures~\cite{dutta2019case,jacobsson17b,li17b,mo16b,panagiotis20}. 

\subsection{Channel Estimation with Low-Resolution Data Converters}
%
Coarse quantization of the received baseband samples, due to the use of low-resolution analog-to-digital converters (ADCs) at the BS, together with the high path loss at mmWave or terahertz (THz) frequencies~\cite{rappaport15b, gao16}, renders the acquisition of accurate channel estimates a challenging task.
%
Fortunately,  wave propagation at mmWave or THz frequencies is directional~\cite{akdeniz14a} and channels typically consist only of a few dominant propagation paths~\cite{rappaport13a,rappaport15a}. Both of these properties cause the channel vectors to be sparse in the beamspace domain, which can be exploited to perform denoising that improves reliability of data transmission~\cite{alkhateeb14a,mo14b,tang13,brady13,ghods19a}. 

Practical sparsity-exploiting channel denoising methods for mmWave massive MU-MIMO systems must exhibit low computational complexity due to the large number of BS antenna elements and the potentially large number of UEs that commmunicate simultaneously.
%
A low-complexity mmWave channel denoising algorithm called BEACHES (short for beamspace channel estimation) has been proposed recently in~\cite{ghods19a}. This method has orders-of-magnitude lower complexity than state-of-the-art denoising methods, such as atomic norm minimization (ANM)~\cite{bhaskar13} and Newtonized orthogonal matching pursuit (NOMP) \cite{mamandipoor16}. 
%
However, all of these existing denoising methods perform poorly when denoising channel vectors that were acquired through low-resolution data converters. 
%
Channel estimation with 1-bit ADCs has been analyzed in~\cite{li17b,li16a,jacobsson17b,mollen16c,studer16a}. 
%
Beamspace sparsity of mmWave channels has been exploited to denoise channel vectors from 1-bit measurements in \cite{mo16b, huang19,kaushik18}.
%
However, all of these denoising methods exhibit high complexity, ignore beamspace sparsity, and/or require a number of parameters that must be adapted to the instantaneous propagation conditions, such as the number of dominant propagation paths.  


\subsection{Contributions}
%
We propose low-complexity channel estimation algorithms for mmWave massive MU-MIMO systems that operate with 1-bit data converters.
%
By using a Bussgang-like decomposition~\cite{bussgang52a} of the 1-bit measurement process, our methods adapt the optimal denoising parameters to the channel's instantaneous sparsity via Stein's unbiased risk estimate (SURE).
%
We propose two methods that build upon BEACHES put forward in~\cite{ghods19a} and a novel method, referred to as Sparsity-Adaptive oNe-bit Denoiser (SAND), which automatically tunes two algorithm parameters to minimize the channel estimation mean-square error (MSE). 
%
To demonstrate the efficacy of our channel estimation algorithms, we perform  MSE and bit error rate (BER) simulations with  line-of-sight (LoS) and non-LoS mmWave channels in a massive MU-MIMO system. 

\subsection{Notation}
%
Lowercase and uppercase boldface letters denote column vectors and matrices, respectively. 
%
The $k$th entry of the vector~$\bma$ is~$a_k$; 
the real and imaginary parts are $\realindex{\bma} = \realpart{\bma}$ and $\imagindex{\bma} = \imagpart{\bma}$, respectively. 
For a matrix $\bA$, 
its transpose and Hermitian transpose are $\bA^\Tran$ and $\bA^\Herm$, respectively. 
A complex Gaussian vector $\bma$ with mean $\bmm$ and covariance $\bK$ is written as $\bma\sim\CN(\bmm, \bK)$.
%
Expectation is denoted by $\smolE{\cdot}$. 
%
% \section{Engines}

% To produce a PDF file, pdf\LaTeX{} is strongly recommended (over original \LaTeX{} plus dvips+ps2pdf or dvipdf). Xe\LaTeX{} also produces PDF files, and is especially suitable for text in non-Latin scripts.

\section{Related Work}
\respace
\label{sec:relatedwork}

\paragraph{Active Learning in NLP}
\citet{lowell2019practical} shows how inconsistent active learning methods are in NLP, even under regular conditions. 
However, \citet{dor2020active, siddhant-lipton-2018-deep} survey active learning methods in NLP and find notable gains over random baselines.
\citet{kouw2019review} survey domain adaptation without target labels, similar to our setting, but for non-language tasks.
We reference more active learning techniques in Section~\ref{sec:methods}.

\respace
\paragraph{Domain Shift Detection}
\citet{elsahar-galle-2019-annotate} attempt to predict accuracy drops due to domain shifts and \citet{rabanser2018failing} surveys different domain shift detection methods. 
\citet{arora2021types} examine calibration and density estimation for textual OOD detection.
% Both evaluate domain discriminators for detecting domain shift using task agnostic and specific embeddings, which motivates our \dal{} methods.

\respace
\paragraph{Active Learning under Distribution Shift}
A few previous works investigated active learning under distribution shifts, though mainly in image classification, with single source and target domains.
\citet{kirsch2021active} finds that \bald{}, which is often considered the state of the art for unshifted domain settings, can get stuck on irrelevant source domain or junk data. 
\citet{pmlr-v130-zhao21b} investigates \emph{label shift}, proposing a combination of predicted class balanced subsampling and importance weighting. 
\citet{10.1007/978-3-642-23808-6_7}, whose approach corrects joint distribution shift, relies on the \emph{covariate shift assumption}. 
However, in practical settings, there may be general distributional shifts where neither the \emph{covariate shift} nor \emph{label shift} assumptions hold.

\respace
\paragraph{Transfer Learning from Multiple Domains}
Attempts to better understand how to handle shifted domains for better generalization or target performance has motivated work in question answering \citep{talmor2019multiqa, fisch2019mrqa, longpre2019exploration, kamath2020selective} and classification tasks \citep{ruder2018strong, sheoran2020recommendation}.
\citet{ruder2017learning} show the benefits of both data similarity and diversity in transfer learning.
\citet{ruckle2020multicqa} find that sampling from a wide-variety of source domains (data scale) outperforms sampling similar domains in question answering.
\citet{he2021multi} investigate a version of multi-domain active learning where models are trained and evaluated on examples from all domains, focusing on robustness across domains.
% \section{Preamble}

% The first line of the file must be
% \begin{quote}
% \begin{verbatim}
% \documentclass[11pt]{article}
% \end{verbatim}
% \end{quote}

% To load the style file in the review version:
% \begin{quote}
% \begin{verbatim}
% \usepackage[review]{acl}
% \end{verbatim}
% \end{quote}
% For the final version, omit the \verb|review| option:
% \begin{quote}
% \begin{verbatim}
% \usepackage{acl}
% \end{verbatim}
% \end{quote}

% To use Times Roman, put the following in the preamble:
% \begin{quote}
% \begin{verbatim}
% \usepackage{times}
% \end{verbatim}
% \end{quote}
% (Alternatives like txfonts or newtx are also acceptable.)

% Please see the \LaTeX{} source of this document for comments on other packages that may be useful.

% Set the title and author using \verb|\title| and \verb|\author|. Within the author list, format multiple authors using \verb|\and| and \verb|\And| and \verb|\AND|; please see the \LaTeX{} source for examples.

% By default, the box containing the title and author names is set to the minimum of 5 cm. If you need more space, include the following in the preamble:
% \begin{quote}
% \begin{verbatim}
% \setlength\titlebox{<dim>}
% \end{verbatim}
% \end{quote}
% where \verb|<dim>| is replaced with a length. Do not set this length smaller than 5 cm.

\respace
\section{Multi-Domain Active Learning}
\respace
\label{sec:task}

% Suppose we have $K$ domains $D_1, D_2, ..., D_k \in D$.\footnote{We define a domain as a dataset collected independently of the others.}
%Suppose we have multiple domains $D = \bigcup\limits_{i=1} D_{i}$.\footnote{We define a domain as a dataset collected independently of the others.}
Suppose we have multiple domains $D_1, D_2, ..., D_k$.\footnote{We define a domain as a dataset collected independently of the others.}
Let one of the $k$ domains be the \target{} set $D_T$, and let the other $k-1$ domains comprise the \source{} set $D_S=\bigcup\limits_{i\neq T} D_{i}$.
% Each domain comprises a training set $D_i^{train}$, development set $D_i^{dev}$, and test set $D_i^{test}$, all sets of $(x, y)$ input-label pairs.
% One of these domains is designated the \target{} set $D_T$, while the other $\{D_i^{train} \mid i\ne T\}$ domains collectively become the \source{} set $D_S$.

\respace
\respace
\paragraph{Given:}
\begin{itemize}\itemsep0em
    \item \textbf{Target:} Small samples of \textit{labeled} data points $(x, y)$ from the \target{} domain. \\
    $D_T^{train},D_T^{dev}, D_T^{test} \sim D_T$.\footnote{$|D_T^{train}|=2000$ to simulate a small but reasonable quantity of labeled, in-domain training data for active learning scenarios.}
    \item  \textbf{Source:} A large sample of \textit{unlabeled} points $(x)$ from the \source{} domains. \\
    $D_S=\bigcup\limits_{i\neq T} D_{i}$
\end{itemize}
\respace
\paragraph{Task:}
\begin{enumerate}\itemsep0em
    \item \textbf{Choose} $n$ samples from $D_S$ to label. \\
    $D_{S}^{chosen} \subset D_S$, $|D_{S}^{chosen}|=n$, selected by $ \argmax_{x \in D_S} A_{f}(x) $ where $ A_{f} $ is an acquisition function: a policy to select unlabeled examples from $D_S$ for labeling.
    \item \textbf{Train} a model $M$ on $D^{final-train}$, validating on $D_{T}^{dev}$. \\
    $D^{final-train} = D_T^{train} \cup D_S^{chosen}$
    \item \textbf{Evaluate} $M$ on $D_T^{test}$, giving score $s$.
\end{enumerate}

For Step 1, the practitioner chooses $n$ samples with the highest scores according to their acquisition function $A_f$. 
$M$ is fine-tuned on these $n$ samples, then evaluated on $D_T^{test}$ to demonstrate $A_f$'s ability to choose relevant out-of-distribution training examples.

% Many functions train an acquisition model $M_A$, on $D_T^{train}$.
% Both $M_A$ and $M$ use $D_T^{dev}$ as their development set.
% Section \ref{sec:methods} describes the choice of $M_A$ and how it is used for each acquisition method $A_f$.
% \section{Document Body}

% \subsection{Footnotes}

% Footnotes are inserted with the \verb|\footnote| command.\footnote{This is a footnote.}

% \subsection{Tables and figures}

% See Table~\ref{tab:accents} for an example of a table and its caption.
% \textbf{Do not override the default caption sizes.}

% \begin{table}
% \centering
% \begin{tabular}{lc}
% \hline
% \textbf{Command} & \textbf{Output}\\
% \hline
% \verb|{\"a}| & {\"a} \\
% \verb|{\^e}| & {\^e} \\
% \verb|{\`i}| & {\`i} \\ 
% \verb|{\.I}| & {\.I} \\ 
% \verb|{\o}| & {\o} \\
% \verb|{\'u}| & {\'u}  \\ 
% \verb|{\aa}| & {\aa}  \\\hline
% \end{tabular}
% \begin{tabular}{lc}
% \hline
% \textbf{Command} & \textbf{Output}\\
% \hline
% \verb|{\c c}| & {\c c} \\ 
% \verb|{\u g}| & {\u g} \\ 
% \verb|{\l}| & {\l} \\ 
% \verb|{\~n}| & {\~n} \\ 
% \verb|{\H o}| & {\H o} \\ 
% \verb|{\v r}| & {\v r} \\ 
% \verb|{\ss}| & {\ss} \\
% \hline
% \end{tabular}
% \caption{Example commands for accented characters, to be used in, \emph{e.g.}, Bib\TeX{} entries.}
% \label{tab:accents}
% \end{table}

% \subsection{Hyperlinks}

% Users of older versions of \LaTeX{} may encounter the following error during compilation: 
% \begin{quote}
% \tt\verb|\pdfendlink| ended up in different nesting level than \verb|\pdfstartlink|.
% \end{quote}
% This happens when pdf\LaTeX{} is used and a citation splits across a page boundary. The best way to fix this is to upgrade \LaTeX{} to 2018-12-01 or later.

% \subsection{Citations}

% \begin{table*}
% \centering
% \begin{tabular}{lll}
% \hline
% \textbf{Output} & \textbf{natbib command} & \textbf{Old ACL-style command}\\
% \hline
% \citep{Gusfield:97} & \verb|\citep| & \verb|\cite| \\
% \citealp{Gusfield:97} & \verb|\citealp| & no equivalent \\
% \citet{Gusfield:97} & \verb|\citet| & \verb|\newcite| \\
% \citeyearpar{Gusfield:97} & \verb|\citeyearpar| & \verb|\shortcite| \\
% \hline
% \end{tabular}
% \caption{\label{citation-guide}
% Citation commands supported by the style file.
% The style is based on the natbib package and supports all natbib citation commands.
% It also supports commands defined in previous ACL style files for compatibility.
% }
% \end{table*}

% Table~\ref{citation-guide} shows the syntax supported by the style files.
% We encourage you to use the natbib styles.
% You can use the command \verb|\citet| (cite in text) to get ``author (year)'' citations, like this citation to a paper by \citet{Gusfield:97}.
% You can use the command \verb|\citep| (cite in parentheses) to get ``(author, year)'' citations \citep{Gusfield:97}.
% You can use the command \verb|\citealp| (alternative cite without parentheses) to get ``author, year'' citations, which is useful for using citations within parentheses (e.g. \citealp{Gusfield:97}).

% \subsection{References}

% \nocite{Ando2005,borschinger-johnson-2011-particle,andrew2007scalable,rasooli-tetrault-2015,goodman-etal-2016-noise,harper-2014-learning}

% The \LaTeX{} and Bib\TeX{} style files provided roughly follow the American Psychological Association format.
% If your own bib file is named \texttt{custom.bib}, then placing the following before any appendices in your \LaTeX{} file will generate the references section for you:
% \begin{quote}
% \begin{verbatim}
% \bibliographystyle{acl_natbib}
% \bibliography{custom}
% \end{verbatim}
% \end{quote}

% You can obtain the complete ACL Anthology as a Bib\TeX{} file from \url{https://aclweb.org/anthology/anthology.bib.gz}.
% To include both the Anthology and your own .bib file, use the following instead of the above.
% \begin{quote}
% \begin{verbatim}
% \bibliographystyle{acl_natbib}
% \bibliography{anthology,custom}
% \end{verbatim}
% \end{quote}

% Please see Section~\ref{sec:bibtex} for information on preparing Bib\TeX{} files.

% \subsection{Appendices}

% Use \verb|\appendix| before any appendix section to switch the section numbering over to letters. See Appendix~\ref{sec:appendix} for an example.

\section{Methods}
\label{sec:methods}

We identify four families of methods relevant to active learning over multiple shifted domains.
\textbf{Uncertainty methods} are common in standard active learning for measuring example uncertainty or familiarity to a model; \textbf{$\mathcal{H}$-Divergence} techniques train classifiers for domain shift detection; \textbf{Semantic Similarity Detection} finds data points similar to points in the target domain; and \textbf{Reverse Classification Accuracy} approximates the benefit of training on a dataset. 
A limitation of our work is we do not cover all method families, such as domain adaptation, just those we consider most applicable.
%Not all active learning and domain shift detection methods are represented here, such as query-by-committee or density weighted sampling \citep{settles2009active}, but we derive \~18 of the most prevalent variants from prior work, including novel extensions/variants of existing paradigms for the multi-domain active learning setting (see \knn{}, \rcas{} and \dale{}).
We derive $\sim$18 active learning variants, comprising the most prevalent and effective from prior work, and novel extensions/variants of existing paradigms for the multi-domain active learning setting (see \knn{}, \rcas{} and \dale{}).

Furthermore, we split the families into two acquisition strategies: \textbf{Single Pool Strategy} and \textbf{Domain Budget Allocation}. 
\textbf{Single Pool Strategy}, comprising the first three families of methods, treats all examples as coming from one single unlabeled pool. \textbf{Domain Budget Allocation}, consisting of \textbf{Reverse Classification Accuracy} methods, simply allocate an example budget for each domain.

We enumerate acquisition methods $A_f$ below. 
Each method produces a full ranking of examples in the source set $D_S$.
To rank examples, most acquisition methods train an acquisition model, $M_A$, using the same model architecture as $M$. 
$M_A$ is trained on all samples from $D_T^{train}$, except for \dal{} and \knn{}, which split $D_T^{train}$ into two equal segments, one for training $M_A$ and one for an internal model.
%For methods where $M_A$ requires supervision, $M_A$ is trained on $D_T^{train}$, or for variants of \dal{} and \knn{} which train another supervised model, they are each trained on separate 1000 examples from $D_T^{train}$.
Some methods have both ascending and descending orders of these rankings (denoted by $\uparrow$ and $\downarrow$ respectively, in the method abbreviations), to test whether similar or distant examples are preferred in a multi-domain setting.

% For Uncertainty methods (Section~\ref{sec:uncertainty-methods}) and Reverse Classification Accuracy (Section~\ref{sec:rca-methods}) we use BERT-Base \citep{devlin2019bert}.
% For $\mathcal{H}$-Divergence Methods (Section~\ref{sec:h-divergence-methods}) and Semantic Similarity Detection (Section~\ref{sec:kNN-method}) we experiment with both task-agnostic and task-specific embeddings.
Certain methods use vector representations of candidate examples.
We benchmark with both task-agnostic and task-specific encoders.
The task-agnostic embeddings are taken from the last layer's CLS token in \citet{reimers-2019-sentence-bert}'s sentence encoder (Appendix for details).
%\footnote{A RoBERTa-Large model fine-tuned on SNLI \citep{bowman2015large} and STSb \citep{cer2017semeval}: \url{https://github.com/UKPLab/sentence-transformers}}
The task-specific embeddings are taken from the last layer's CLS token in the trained model $M_A$.

The motivation of the task-specific variant is that each example's representation will capture task-relevant differences between examples while ignoring irrelevant differences.\footnote{For instance, consider in one domain every example is prefixed with ``Text:'' while the other is not --- telling the difference is trivial, but the examples could be near-identical with respect to the task.}
The versions of \dal{} and \knn{} methods that use task-specific vectors are denoted with ``$*$'' in their abbreviation. 
Otherwise, they use task-agnostic vectors.

    \subsection{Uncertainty Methods}
    \label{sec:uncertainty-methods}
    \respace
    These methods measure the uncertainty of a trained model on a new example.
    Uncertainty can reflect either \textit{aleatoric} uncertainty, due to ambiguity inherent in the example, or \textit{epistemic} uncertainty, due to limitations of the model \citep{kendall2017uncertainties}.
    For the following methods, let $Y$ be the set of all possible labels produced from the model $M(x)$ and $l_y$ be the logit value for $y \in Y$.

        \paragraph{Confidence (\conf)}
        A model's confidence $P(y|x)$ in its prediction $y$ estimates the difficulty or unfamiliarity of an example \citep{guo2017calibration, elsahar-galle-2019-annotate}.
        \respace
        $$ A_{\text{CONF}}(x, M_A) = -\max(P(y|x))$$ 
        \respace
        \paragraph{Entropy (\entr)}
        Entropy applies Shannon entropy \citep{shannon1948mathematical} to the full distribution of class probabilities for each example, formalized as $A_{\text{ENTR}}$.
        \respace
        \respace
         $$ A_{\text{ENTR}}(x, M_A) = -\sum_{i = 1}^{|Y|} { P(y_i|x) \cdot \log{P(y_i|x)}  }$$ 
        
        % $$ A_{\text{ENTR}}(x, M_A) = -\sum_{i = 1}^{|Y|} { P(y_i|x) \cdot \log{P(y_i|x)}  }$$

        \respace
        \respace
        \paragraph{Energy-based Out-of-Distribution Detection (\eod)}
        \citet{liu2020energy} use an \textit{energy-based score} to distinguish between in- and out-distribution examples.
        They demonstrate this method is less susceptible to overconfidence issues of softmax approaches.
        \respace
        % We chose $T=1$ to maximize the distinction between in- and out-domain predictions.
        $$ A_{ENG}(x, M_A) = - \log \sum_{y \in Y} e^{l_y} $$ 
        \respace
        \respace
        \paragraph{Bayesian Active Learning by Disagreement (\bald)}
        \citet{gal2016} introduces estimating uncertainty by measuring prediction disagreement over multiple inference passes, each with a distinct dropout mask.
        \bald{} isolates \textit{epistemic} uncertainty, as the model would theoretically produce stable predictions over inference passes given sufficient capacity.
        We conduct $T=20$ forward passes on $x$.
        $\hat{y}_t = \text{argmax}_{i}P(y_i|x)_t$, representing the predicted class on the $t$-th model pass on $x$.
        Following \citep{lowell2019practical}, ties are broken by taking the mean label entropy over all $T$ runs. 
        \respace
        \respace
        $$ A_{\text{BALD}}(x, M_{A}) = 1 - \frac{\text{count}(\text{mode}_{t \in T} (\hat{y}_t))}{T} $$
        
        % $$ A_{ENG}(x, M_A) = -T \cdot \log \sum_{y \in Y} e^{l_y / T} \;\;\;\;\;\; A_{\text{BALD}}(x, M_{A}) = 1 - \frac{\text{count}(\text{mode}_{t \in T} (\hat{y}_t))}{T} $$
        
    \respace
    \respace
    \subsection{$\mathcal{H}$-Divergence Methods}
    \label{sec:h-divergence-methods}
    \citet{ben2006analysis, ben2010theory} formalize the divergence between two domains as the $\mathcal{H}$-Divergence, which they approximate as the difficulty for a discriminator to differentiate between the two.\footnote{The approximation is also referred to as Proxy $\mathcal{A}$-Distance (PAD) from \citep{elsahar-galle-2019-annotate}}
    Discriminative Active Learning (\dal) applies this concept to the active learning setting \citep{gissin2019discriminative}.
    
    We explore variants of \dal{}, using an XGBoost decision tree \citep{Chen:2016:XST:2939672.2939785} as the discriminator model $g$.\footnote{Hyperparameter choices and training procedures are detailed in the Appendix.}
    For the following methods, let $D_T^{train-B}$ be the 1k examples from $D_T^{train}$ that were \emph{not} used to train $M_A$. 
    Let $E$ be an encoder function, which can be task-specific or agnostic as described above. 
    We use samples both from $D_T^{train-B}$ and $D_S$ to train the discriminator. 
    We assign samples origin labels $l$, which depend on the \dal{} variant. 
    Samples from $D_S$ with discriminator predictions closest to 1 are selected for labeling.
    The acquisition scoring function for each \dal{} method and training set definition, respectively, are:
    \respace
    \respace
    $$A_{\text{DAL}}(x, g, E) = g(E(x)) $$
    $$ \{(E(x), \: l) \; | \; x \in D_T^{train-B} \cup D_{S} \}$$

    %This is inspired by deep NLP models producing intermediate representations that exclude features irrelevant to the NLP task. Motivated by this work and by \citep{gissin2019discriminative}, which uses a discriminator for active learning, we explore $TAL_E$ and $TAL_T$.
    %Both $TAL_E$ and $TAL_T$ perform discriminative active learning. 
    % Both $TAL_E$ and $TAL_T$ perform discriminatory active learning where
    % %the input to the discriminator $E(x)$ is the penultimate layer from model $M_1$.
    % the input $g(\xi)$ to the discriminator $C$ is the penultimate layer from model $M_1$.
    % Their acquisition functions can be written as 
    % %$$x* = argmax_{x\in D_{source}}C(g(x)).$$
    % $$ a_{TAL}(x, M_1) = C(g(\xi), x) $$

    % For both methods, $C$ is an XGBoost gradient boosting decision tree \citep{Chen:2016:XST:2939672.2939785}. Hyperparameter choices are discussed in the appendix. \shayne{Do you know how to set up an appendix?}
    % \julia{Define C}
        \respace
        \respace
        \paragraph{Discriminative Active Learning --- Target (\dalt)}
        \dalt{} trains a discriminator $g$ to distinguish between target examples in $D_T^{train-B}$ and out-of-distribution examples from $D_{S}$. 
        For \dalt{}, $l=\mathbbm{1}_{D_T^{train-B}}(x)$.
        % \footnote{Multiple models, each trained on a different segment of the data, are used to fairly train and select from $D_{S}$. Full details are in the appendix.}
        % To do this, it assigns pseudo-labels to samples in $D_{val}$ and $D_{S}$ such that $g$'s training set consists of samples 
        % %$\{(g(x), l) \; | \; l=\mathbbm{1}_{D_{val}}(x), x\in \{D_{val}, D_{source}\}\}$.
        % $\{(g(\xi), \: l) \; | \; l=\mathbbm{1}_{D_{val}}(\xi), \; \xi \in D_{val} \cup D_{S} \}$.
        
        \respace
        \paragraph{Discriminative Active Learning --- Error (\dale)}
        \dale{} is a novel variant of \dal.
        \dale's approach is to find examples that are similar to those in the target domain that $M_A$ misclassified.
        %These ``erroneous'' examples represent those the model finds most challenging from the target domain.
        %We define a non-erroneous sample as one where the prediction from $M_1$ is an exact match of its label. 
        % We define an erroneous sample as one where the predicted pseudo-label is incorrect. 
        We partition $D_T^{train-B}$ further into erroneous samples $D_T^{err}$ and correct samples $D_T^{corr}$, where $D_T^{train-B} = D_T^{err} \cup D_T^{corr}$. For \dale{}, $l=\mathbbm{1}_{D^{err}_T}(x)$.

        % For both methods, the discriminator is an XGBoost gradient boosting decision tree \citep{Chen:2016:XST:2939672.2939785}. 
        % The model had $10$ for $n\_estimators$, $2$ for $max\_depth$, hist for $tree\_method$, $5$ for $lambda$, $binary:logistic$ for $objective$, $0.1$ for $learning\_rate$, and $5$ for $gamma$. The rest of the hyperparameters were set to the default XGBoost parameters.

    \respace
    \subsection{Reverse Classification Accuracy}
    \label{sec:rca-methods}

        \respace
        \paragraph{\rca}

        Reverse Classification Accuracy (\rca) estimates how effective source set $D_{i, i \in S}$ is as a training data for target test set $D_T$ \citep{fan2006reverse, elsahar-galle-2019-annotate}.
        Without gold labels for $D_{i}$ we compute soft labels instead, using the BERT-Base $M_A$ trained on the small labeled set $D_T^{train}$.
        We then train a child model $M_i$ on $D_{i}$ using these soft labels, and evaluate the child model on $D_T^{dev}$.
        \rca{} chooses examples randomly from whichever domain $i$ produced the highest score $s_i$.
        \respace
        \respace
        
        $$A_{\text{RCA}} = \mathbbm{1}_{D_{(\argmax_{i\in S}s_{i})}}(x) $$ 
        $$ \widetilde{RCA}:\;\;\; \tau_i = \dfrac{s_i}{s_T - s_i}, \; |D^{chosen}_i| = \dfrac{\tau_i}{\sum\limits_{j} s_j}$$

        \respace
        \respace
        \respace
        \paragraph{RCA-Smoothed (\rcas)}
        Standard \rca{} only selects examples from one domain $D_{i}$.
        We develop a novel variant which samples from multiple domains, proportional to their relative performance on the target domain $D_T^{dev}$.
        RCA-smoothed (\rcas) selects $|D^{chosen}_i|$ examples from source domain $i$, based on the relative difference between the performance $s_i$ (of child model $M_i$ trained on domain $i$ with pseudo-labels from $M_A$) on the target domain, and the performance $s_T$ of a model trained directly on the target domain $D_T^{dev}$. 
        Since these strategies directly estimates model performance on the target domain resulting from training on each source domain, \rca{} and \rcas{} are strong \textbf{Domain Budget Allocation} candidates.
        
    \respace
    \subsection{Nearest Neighbour / Semantic Similarity Detection (\knn)}
    \label{sec:kNN-method}
    \respace
    
    Nearest neighbour methods (\knn) are used to find examples that are semantically similar.
    Using sentence encoders we can search the source set $D_S$ to select the top $k$ nearest examples by cosine similarity to the target set.
    We represent the target set as the mean embedding of $D_T^{train}$. 
    For question answering, where an example contains two sentences (the query and context), we refer to \knnq{} where we only encode the query text, \knnc{} where we only encode the context text, or \knnqc{} where we encode both concatenated together.
    The acquisition scoring function per example, uses either a task-specific or task-agnostic encoder $E$:
    \respace
    $$A_{\text{KNN}}(x, E) = \text{CosSim}(E(x), \text{Mean}( E(D_{T}^{train} ))$$

    
    %For task-specific vector representations, we use the common convention of the final hidden layer of the $0$th token of the model $M_A$ trained on $D_{T}^{train}$.
    %The task-specific variants of the \knn{} approach are suffixed with $*$, such as \knns.
    
    




\begin{table}[t]
  \centering
  \footnotesize
\begin{minipage}[t]{\textwidth}
  \caption{Comparison with state-of-the-art methods in zero-shot segmentation. mIoU$^s$ and mIoU$^u$ denote the mIoU(\%) of seen classes and unseen classes. }
\resizebox{1.0\textwidth}{!}{

    \renewcommand\arraystretch{1.2} % 1.95
    
    % \begin{tabular}{l|ccc|ccc|ccc}
    \begin{tabular}{l|lll|lll|lll}
      \Xhline{0.8pt}
      % \toprule
      \multirow{2}{*}{\textbf{Method}} &\multicolumn{3}{c|}{COCO-Stuff} & \multicolumn{3}{c|}{Pascal-VOC} & \multicolumn{3}{c}{ADE20K}\\ 
      & \textbf{mIoU$^s$} & \textbf{mIoU$^u$} & \textbf{hIoU} & \textbf{mIoU$^s$} & \textbf{mIoU$^u$} & \textbf{hIoU} & \textbf{mIoU$^s$} & \textbf{mIoU$^u$} & \textbf{hIoU} \\ 
      \hline
      SPNet\cite{spnet} & 34.6 & 26.9 & 30.3 & 77.8 & 25.8 & 38.8 & - & - & -\\
      ZS5\cite{zs5} & 34.9 & 10.6 & 16.2 & 78.0 & 21.2 & 33.3 & - & - & - \\
      CaGNet\cite{cagnet} & 35.6 & 13.4 & 19.5 & 78.6 & 30.3 & 43.7 & - & - & -\\
      STRICT\cite{STRICT} & 35.3 & 30.3 & 32.6 & 82.7 & 35.6 & 73.3 & - & - & -\\
      \cdashline{1-10}[0.8pt/2pt]
      ZegFormer\cite{zegformer} & 36.7 & 36.2 & 36.4 & 90.1 & 70.6 & 79.2 & 17.4 & 5.1 & 7.9\\
      ZegFormer +MAFT & 36.4 $_{\textcolor{blue}{-0.3}}$ & 40.1 $_{\textcolor{red}{+3.9}}$ & 38.1 $_{\textcolor{red}{+1.7}}$ & 91.5 $_{\textcolor{red}{+1.4}}$ & 80.7 $_{\textcolor{red}{+10.1}}$ & 85.7 $_{\textcolor{red}{+6.5}}$ & 16.6 $_{\textcolor{blue}{-0.8}}$ & 7.0 $_{\textcolor{red}{+1.9}}$ & 9.8 $_{\textcolor{red}{+1.9}}$\\
      \cdashline{1-10}[0.8pt/2pt]
      ZSSeg\cite{zsseg} & 40.4 & 36.5 & 38.3 & 86.6 & 59.7 & 69.4 & 18.0 & 4.5 & 7.2 \\
      ZSSeg +MAFT & 40.6 $_{\textcolor{red}{+0.2}}$ & 40.1 $_{\textcolor{red}{+3.6}}$ & 40.3 $_{\textcolor{red}{+2.0}}$ & 88.4 $_{\textcolor{red}{+1.8}}$ & 66.2 $_{\textcolor{red}{+6.5}}$ & 75.7 $_{\textcolor{red}{+6.3}}$ & 18.9 $_{\textcolor{red}{+0.9}}$ & 6.7 $_{\textcolor{red}{+2.2}}$ & 9.9 $_{\textcolor{red}{+2.7}}$ \\
      \cdashline{1-10}[0.8pt/2pt]
      FreeSeg\cite{freeseg} & 42.4 & 42.2 & 42.3 & 91.9 & 78.6 & 84.7 & 22.3 & 4.4 & 7.3 \\
      FreeSeg +MAFT & 43.3 $_{\textcolor{red}{+0.9}}$ & 50.4 $_{\textcolor{red}{+8.2}}$ & 46.5 $_{\textcolor{red}{+4.2}}$ & 91.4 $_{\textcolor{blue}{-0.5}}$ & 81.8 $_{\textcolor{red}{+3.2}}$ & 86.3 $_{\textcolor{red}{+1.6}}$ & 21.4 $_{\textcolor{blue}{-0.9}}$ & 8.7 $_{\textcolor{red}{+4.3}}$  & 12.4 $_{\textcolor{red}{+5.1}}$ \\
      % \rowcolor{gray!10}{} Ours & \textbf{42.2} & \textbf{49.1} & \textbf{45.3} & \textbf{91.8} & \textbf{82.6} & \textbf{86.9} & \textbf{44.2} & \textbf{28.6} & \textbf{39.8} \\
      \Xhline{0.8pt}
      % \bottomrule
      \end{tabular}
      }
      % \end{threeparttable}
      \label{tab:zss}
\end{minipage}

\begin{minipage}[t]{\textwidth}
\caption{Results on representative methods \cite{zegformer, zsseg, freeseg} with/without MAFT. Here we remove the \textit{ensemble} operation, and only maintain CLIP classifier results.}
 \resizebox{1.0\textwidth}{!}{

    \renewcommand\arraystretch{1.2} % 1.95
    
    % \begin{tabular}{l|ccc|ccc|ccc}
    \begin{tabular}{l|lll|lll|lll}
      \Xhline{0.8pt}
      % \toprule
      \multirow{2}{*}{\textbf{Method}} &\multicolumn{3}{c|}{COCO-Stuff} & \multicolumn{3}{c|}{Pascal-VOC} & \multicolumn{3}{c}{ADE20K}\\ 
      %\cline{2-10}
      & \textbf{mIoU$^s$} & \textbf{mIoU$^u$} & \textbf{hIoU} & \textbf{mIoU$^s$} & \textbf{mIoU$^u$} & \textbf{hIoU} & \textbf{mIoU$^s$} & \textbf{mIoU$^u$} & \textbf{hIoU} \\ 
      \hline
      ZegFormer\cite{zegformer} & 18.5 & 23.0 & 20.5 & 81.4 & 76.8 & 79.0 & 5.1 & 2.6 & 3.5\\
      ZegFormer +MAFT & 35.1 $_{\textcolor{red}{+16.6}}$ & 31.6 $_{\textcolor{red}{+7.6}}$ & 33.3 $_{\textcolor{red}{+12.7}}$ & 87.6 $_{\textcolor{red}{+6.2}}$ & 79.9 $_{\textcolor{red}{+3.1}}$ & 83.5 $_{\textcolor{red}{+4.5}}$ & 15.8 $_{\textcolor{red}{+10.8}}$ & 7.0 $_{\textcolor{red}{+4.4}}$ & 9.8 $_{\textcolor{red}{+6.3}}$\\
      \cdashline{1-10}[0.8pt/2pt]
      ZSSeg\cite{zsseg} & 20.6 & 27.4 & 23.6 & 82.0 & 71.2 & 76.2 & 5.9 & 2.8 & 3.9 \\
      ZSSeg +MAFT & 36.1 $_{\textcolor{red}{+15.5}}$ & 35.9 $_{\textcolor{red}{+8.3}}$ & 36.0 $_{\textcolor{red}{+12.4}}$ & 87.1 $_{\textcolor{red}{+5.1}}$ & 76.1 $_{\textcolor{red}{+4.9}}$ & 81.2 $_{\textcolor{red}{+5.0}}$ & 17.2 $_{\textcolor{red}{+11.3}}$ & 7.2 $_{\textcolor{red}{+4.4}}$ & 10.2 $_{\textcolor{red}{+6.3}}$ \\
      \cdashline{1-10}[0.8pt/2pt]
      FreeSeg\cite{freeseg} & 22.3 & 29.3 & 25.3 & 87.4 & 74.7 & 80.5 & 6.5 & 2.8 & 3.9 \\
      FreeSeg +MAFT & 40.1 $_{\textcolor{red}{+17.8}}$ & 49.7 $_{\textcolor{red}{+20.4}}$ & 44.4 $_{\textcolor{red}{+19.1}}$ & 90.4 $_{\textcolor{red}{+3.0}}$ & 84.7 $_{\textcolor{red}{+10.0}}$ & 87.5 $_{\textcolor{red}{+7.0}}$ & 21.3 $_{\textcolor{red}{+14.8}}$ & 8.7 $_{\textcolor{red}{+5.9}}$  & 12.2 $_{\textcolor{red}{+8.3}}$ \\
      % \rowcolor{gray!10}{} Ours & \textbf{42.2} & \textbf{49.1} & \textbf{45.3} & \textbf{91.8} & \textbf{82.6} & \textbf{86.9} & \textbf{44.2} & \textbf{28.6} & \textbf{39.8} \\
      \Xhline{0.8pt}
      % \bottomrule
      \end{tabular}
      }
      % \end{threeparttable}
      \label{tab:zss-woensem}
\end{minipage}

  \vspace{-2mm}
 \end{table}


% \begin{table}[h]
%   \centering
%   \footnotesize
% \begin{minipage}[b]{80mm}



    
%     \begin{tabular}{l|ccc|ccc|ccc}

%       \end{tabular}
      
%       % \end{threeparttable}
%       \label{tab:zss}
% \end{minipage}

% \begin{minipage}[b]{80mm}


    
%     \begin{tabular}{l|ccc|ccc|ccc}

%       \end{tabular}


% \end{minipage}

%  \end{table}
\subsection{Setting}
\noindent \textbf{Dataset.}
We first follow \cite{zs5, gu2020context, pastore2021closer, zegformer, zsseg} to conduct experiments on three popular zero-shot segmentation benchmarks, Pascal-VOC, COCO-Stuff and ADE20K, to evaluate our method. Then, we evaluate MAFT on the \textit{open-vocabulary} setting \cite{ovseg, zsseg}, \textit{i.e.}, training on COCO-Stuff and testing on ADE20K (A-847, A-150), Pascal-Context (PC-459, PC-59), and Pascal-VOC (PAS-20). More details of the dataset settings are provided in the Appendix.

\noindent \textbf{Evaluation Metrics.}
To quantitatively evaluate the performance, we follow standard practice \cite{zs5, spnet, cagnet, STRICT, zegformer, zsseg, freeseg}, adopt mean Intersection over Union (mIoU) to respectively evaluate the performance for seen classes (IoU$^s$) and unseen classes (IoU$^u$). We also employ the harmonic mean IoU (hIoU) among the seen and unseen classes to measure comprehensive performance.

\noindent \textbf{Methods.}
Three representative methods are used to verify the generality of MAFT. We unify the three methods into the same framework, with all methods using ResNet101 as the backbone of Proposal Generator and ViT-B/16 CLIP model for a fair comparison.

\begin{itemize}[itemsep=2pt,topsep=0pt,parsep=0pt]
\item \textbf{ZegFormer} (CVPR 2022) \cite{zegformer} is an early adopter of the "frozen CLIP" paradigm. It uses MaskFormer as Proposal Generator and employs an \textit{ensemble} operation to improve the confidence of the results.
\item \textbf{ZSSeg} (ECCV 2022) \cite{zsseg} uses MaskFormer as Proposal Generator and introduces learnable prompts to improve classification accuracy, which significantly affects the subsequent methods. ZSSeg also adopts a self-training strategy, this strategy is excluded from all methods for a fair comparison.
\item \textbf{FreeSeg} (CVPR 2023) \cite{freeseg} represents the state-of-the-art method, unifies semantic, instance, and panoptic segmentation tasks and uses annotations from all three tasks for fusion training. We retrain FreeSeg with only the semantic annotations to ensure fairness.
\end{itemize}

\noindent \textbf{Implementation details.}
We employ ResNet101 as backbone of the Proposal Generator and ViT-B/16 CLIP model. The training process  consists of two stages.
For the \textbf{first} stage, we follow the official code of ZegFormer, ZSSeg and FreeSeg for model training. 
For the \textbf{second} stage, we fine-tune IP-CLIP Encoder with MAFT. We take the batch size
of 16 and set CLIP input image size to 480$\times$480. The optimizer is AdamW with a learning rate of 0.00001 and weight decay of 0.00001. The number of training iterations is set to 100 for Pascal-VOC, 1000 for COCO-Stuff and 5000 for ADE20K.

\subsection{Comparisons with State-of-the-art Methods}
\begin{table}
  \centering
  \footnotesize
  % \vspace{-10pt}
  \caption{Comparison with state-of-the-art methods on the \textit{open-vocabulary} setting. mIoU is used to evaluate the performance. * denotes additional training data is used.}
   % \vspace{-10pt}
  % \begin{threeparttable}
  % \resizebox{0.9\textwidth}{
    \renewcommand\arraystretch{1.05} % 1.95
    % \begin{tabular}{l|ccccc}
    \begin{tabular}{l|lllll}
      % \toprule
      \Xhline{0.7pt}

      & \textbf{A-847} & \textbf{A-150} & \textbf{PC-459} & \textbf{PC-59} & \textbf{PAS-20}\\ 
      \hline
      SPNet\cite{spnet} & ~~~- & ~~~- & ~~~- & ~~~24.3 & ~~~18.3\\
      ZSSeg\cite{zs5}  & ~~~- & ~~~- & ~~~- & ~~~19.4 & ~~~38.3\\
      LSeg+\cite{ghiasi2021open} & ~~~2.5 & ~~~13.0 & ~~~5.2 & ~~~36.0 & ~~~59.0\\
      OVSeg\cite{ovseg} & ~~~7.1 & ~~~24.8 & ~~~11.0 & ~~~53.3 & ~~~92.6\\
      OpenSeg* \cite{ghiasi2022scaling} &  ~~~8.8 & ~~~28.6 &  ~~~12.2  &  ~~~48.2 & ~~~72.2 \\     
      \cdashline{1-6}[0.8pt/2pt]
      FreeSeg\cite{freeseg} & ~~~7.1 & ~~~17.9 & ~~~6.4 & ~~~34.4 & ~~~85.6 \\
      FreeSeg +MAFT & ~~~10.1 $_{\textcolor{red}{+3.0}}$ & ~~~29.1 $_{\textcolor{red}{+11.2}}$ & ~~~12.8 $_{\textcolor{red}{+6.4}}$ & ~~~53.5 $_{\textcolor{red}{+19.1}}$ & ~~~90.0 $_{\textcolor{red}{+4.4}}$ \\
      \Xhline{0.7pt}
      % \bottomrule
      \end{tabular}
      % }
      % \end{threeparttable}
      \label{tab:ovs}
  \vspace{-2mm}
  \end{table}  




In this section, three representative methods are used  \cite{zegformer, zsseg, freeseg} to evaluate the effectiveness of MAFT. We compare three representative methods with MAFT and frozen CLIP. Additionally, we compare the results with previous state-of-the-art methods  \cite{spnet, zs5, cagnet, STRICT}.

\noindent \textbf{Comparisons in the \textit{zero-shot} setting.}
In Tab. \ref{tab:zss}, MAFT remarkably improves the performance. MAFT promotes the state-of-the-art performance by + 8.2\% on COCO, + 3.2\% on Pascal, and +4.3\% on ADE20K in terms of mIoU for unseen classes. It is important to note that the results for seen classes are mainly based on $A^p$ rather than $A^c$ due to the \textit{ensemble} operation in \cite{zegformer, zsseg, freeseg} (Details in Sec. \ref{sec:prelimiary}). Therefore, the effect of MAFT on the seen classes is relatively insignificant. 

\noindent \textbf{Comparisons without ensemble strategy.}
To better showcase the performance gains from MAFT, we removed the \textit{ensemble} operation in \cite{zegformer, zsseg, freeseg} and presented the results in Tab. \ref{tab:zss-woensem}.  It can be seen that the performance of different methods is significantly improved after applying MAFT. In particular, the state-of-the-art method FreeSeg achieves hIoU improvements of 19.1\%, 7.0\%, and 8.3\% on COCO, VOC2012 and ADE20K datasets. 

\noindent \textbf{Comparisons in the \textit{open-vocabulary} setting.}
We further evaluated the transferability of MAFT in the \textit{open-vocabulary} setting \cite{ovseg, zsseg}, using FreeSeg as a baseline for comparison. Results are shown in Tab. \ref{tab:ovs}.
Compared with OVSeg \cite{ovseg} and OpenSeg \cite{ghiasi2022scaling}, FreeSeg achieves suboptimal performance. However, the proposed MAFT enhances the performance of A-847, A-150, PC-459, PC-59 and PAS-20 by 3.0\%,11.2\%, 6.4\%, 19.1\% and 4.4\%, and outperforms OpenSeg on all five datasets.

\subsection{Ablation Study}
\begin{table*}
\vspace{-2mm}
\caption{\textbf{Ablations on COCO dataset.} GFLOPs in (a) is used to measure the computation of CLIP Image Encoder. The best results are highlighted with \textcolor{red}{red}, and the default settings are highlighted with \textcolor{gray}{gray} background.}
\label{tab:ablations}
% \vspace{-1mm}
\resizebox{1.0\textwidth}{!}{
\centering
% a - Ablation on Statics, Hand-Craft and Learnable
    \subfloat[Ablation on components of \textbf{MAFT}. $ft$ denotes the mask-aware fine-tining
    \label{tab:ab_1}]
    { 
    % \!\!\!\!\!
    % \renewcommand\tabcolsep{11.5pt}
    \renewcommand\arraystretch{1.1} % 1.95
    % \footnotesize
    \small
    \begin{tabular}
    {l|lll|c}
    % {p{27.8mm}|p{15mm}p{15mm}}
    \Xhline{0.7px}
    % \Xhline{0.7px}
    \centering
     & \textbf{mIoU$^s$} & \textbf{mIoU$^u$} & \textbf{hIoU} & GFLOPs \\
    \hline
    \multicolumn{1}{c|}{FreeSeg} & 22.3 &  29.3 &  25.3 &  1127.0\\
    \multicolumn{1}{c|}{+ IP-CLIP} & 29.4 $_{+7.1}$ &  36.2 $_{+6.9}$ &  32.4 $_{+7.1}$ &  53.4\\
    \multicolumn{1}{c|}{+ $ft$ ($\mathcal{L}_{ma}$)} & 39.9 $_{+17.6}$ &  47.1 $_{+17.8}$ &  43.1 $_{+17.8}$ &  53.4\\
    \rowcolor{gray!10}\multicolumn{1}{c|}{+ $ft$ ($\mathcal{L}_{ma}$ + $\mathcal{L}_{dis}$)} & 
    40.1$_{\textcolor{red}{+17.8}}$ &  
    49.7$_{\textcolor{red}{+20.4}}$ &  
    44.4$_{\textcolor{red}{+19.0}}$ &  
    53.4\\
    \Xhline{0.7px}
    % \Xhline{0.7px}
    \end{tabular}
    }
    \hspace{2mm}

% b - Ablation on SI module
    \subfloat[Ablation on \textbf{mask-aware loss $\mathcal{L}_{ma}$}. $\mathcal{L}_{dis}$ is removed.
    \label{tab:ab_2}]
    { 
    % \!\!\!\!\!
    % \renewcommand\tabcolsep{11.5pt}
    \renewcommand\arraystretch{1.1} % 1.95
    % \footnotesize
    \small
    \begin{tabular}
    {l|ccc}
    % {p{27.8mm}|p{15mm}p{15mm}}
    \Xhline{0.7px}
    \centering
     & \textbf{mIoU$^s$} & \textbf{mIoU$^u$} & \textbf{hIoU} \\
    \hline 
    % \multicolumn{1}{c|}{FreeSeg} & 22.3 &  29.3 &  25.3\\
    \multicolumn{1}{c|}{$L_1$} & 38.6 $_{+16.3}$ &  45.8 $_{+16.5}$ &  41.8 $_{+16.5}$ \\
    \multicolumn{1}{c|}{$L_2$} & 40.0 $_{+17.7}$ &  45.8 $_{+16.5}$ &  42.7 $_{+17.4}$\\
    \rowcolor{gray!10}\multicolumn{1}{c|}{$SmoothL_1$} 
    & 39.9 $_{+17.6}$
    & 47.1 $_{\textcolor{red}{+17.8}}$ 
    & 43.1 $_{\textcolor{red}{+17.8}}$\\
    \multicolumn{1}{c|}{$KL$} & 40.9 $_{\textcolor{red}{+18.6}}$ &  41.8 $_{+12.5}$ &  41.3 $_{+16.0}$\\
    \Xhline{0.7px}
    \end{tabular}
    }
}


\resizebox{1.0\textwidth}{!}{

\centering

    % \hspace{4mm}
    \subfloat[Ablation of the \textbf{training iterations}
    \label{tab:ab-iter}]
    { 
    \centering%
    \footnotesize %scriptsize
    \renewcommand\tabcolsep{10pt}
    \renewcommand\arraystretch{1.2} % 1.95

        \begin{tabular}{l|cc}
        \Xhline{0.7px}
        \centering
         & \textbf{mIoU$^s$} & \textbf{mIoU$^u$}\\
        \hline
         500 iters & 37.7 &  47.0 \\
      \rowcolor{gray!10}  1k iters & 40.0 &  \textcolor{red}{49.7} \\ 
        2k iters & 41.1 &  47.6  \\
        3k iters & 41.4 &  46.5 \\
        4k iters & 41.5 &  46.1 \\
        5k iters & \textcolor{red}{42.0} &  45.7\\
        
        \Xhline{0.7px}
        \end{tabular}
    }
    \hspace{4mm}
    
% d - Ablation on SI module
    \subfloat[Ablation of the \textbf{frozen units in CLIP}
    \label{tab:ab-units}]
    { 
    \centering%
    \footnotesize
    \renewcommand\tabcolsep{6pt}
    \renewcommand\arraystretch{1.2} % 1.95

        \begin{tabular}{l|ccc}
        \Xhline{0.7px}
        \centering
         & \textbf{mIoU$^s$} & \textbf{mIoU$^u$} & \textbf{hIoU} \\
        \hline
         None & 40.6 &  44.7 &  42.5 \\
        + $cls.$ & \textcolor{red}{40.7} &  44.7 &  42.7 \\ 
        + $pos.$ & 40.6 &  44.9 &  42.8 \\
        + $mlp$ & 40.3 &  48.7 &  44.1 \\
 \rowcolor{gray!10}  + $conv.$ & 40.0 & \textcolor{red}{49.7} &  \textcolor{red}{44.3} \\
        + $proj.$ & 40.2 &  49.1 &  44.2 \\
        
        \Xhline{0.7px}
        \end{tabular}
    }
    \hspace{4mm}
    
% e - Ablation on SI module
    \subfloat[Ablation of the \textbf{start mask attention layer $L$}
    \label{tab:ab-layer}]
    { 
    \centering%
    \footnotesize
    \renewcommand\tabcolsep{11pt}
    \renewcommand\arraystretch{1.2} % 1.95

        \begin{tabular}{l|ccc}
        \Xhline{0.7px}
        \centering
         & \textbf{mIoU$^s$} & \textbf{mIoU$^u$} & \textbf{hIoU} \\
        \hline
         0 & 39.3 &  46.4 &  42.6 \\
        2 & 39.2 &  46.4 &  42.5 \\ 
        4 & 39.5 &  46.6 &  42.6 \\
        6 & \textcolor{red}{40.0} &  47.8 &  43.6 \\
      \rowcolor{gray!10}  8 & \textcolor{red}{40.0} &  \textcolor{red}{49.7} &  \textcolor{red}{44.3} \\
        10 & 39.9 &  45.7 &  42.6 \\
        
        \Xhline{0.7px}
        \end{tabular}
    }
}

% main caption
\vspace{-3mm}
\end{table*}




We conduct ablation studies on various choices of designs of our MAFT to show their contribution to the final results in Tab. \ref{tab:ablations}. FreeSeg is used as the baseline model and \textit{ensemble} operation is removed.

\noindent \textbf{Component-wise ablations.} To understand the effect of each component in the MAFT, including the IP-CLIP Encoder and the fine-tuning strategy ($\mathcal{L}_{ma}$, $\mathcal{L}_{dis}$), we start with standard FreeSeg and progressively add each design. (Tab. \ref{tab:ab_1}). 
FreeSeg uses frozen CLIP and yields inferior performance due to CLIP's mask-unaware property ($1^{st}$ row). Then, IP-CLIP Encoder obtains rich context information and greatly reduces the omputational costs, resulting in an improvement of 7.1\% on seen classes and 6.9\% on unseen classes. However, mask-aware is not accomplished at this point.
Using only $\mathcal{L}_{ma}$ for fine-tuning CLIP produces decent performance  (the $3^{rd}$ result). The introduction of $\mathcal{L}_{dis}$ (the $4^{th}$ result) maintains transferability while learning mask-aware representations, which further enhances the performance on unseen classes by 2.6\%.

\noindent \textbf{Effect of different $\mathcal{L}_{ma}$.} 
\textit{Mask-aware} Loss $\mathcal{L}_{ma}$ is an essential component of MAFT. In Tab. \ref{tab:ab_2}, we investigate how different loss functions ($L1$, $L2$, $SmoothL1$ and $KL$ Loss) impact performance, here we remove $\mathcal{L}_{dis}$ for analysis. Results show $SmoothL1$ Loss boosts performance on $C_{unseen}$ to 47.1\% (+17.8\%), $KL$ Loss provides +12.5\% improvement on $C_{seen}$, but only +11.8\% on $C_{unseen}$, manifesting $KL$ Loss compromises the model of transferability comparing with $SmoothL1$ Loss.

\noindent \textbf{Training iterations.} 
Tab. \ref{tab:ab-iter} examines the impact of training iterations. Increasing the number of iterations leads to gradual improvement of IoU$^s$, but it also results in significant overfitting on unseen classes. Therefore, we choose to fine-tune 1k iterations to maximize the zero-shot ability.

\noindent \textbf{Frozen units in CLIP.} 
We also explore the impact of fine-tuning units within IP-CLIP Encoder. As illustrated in Fig. \ref{fig:finetune}, IP-CLIP Encoder comprises convolution layers (dubbed as $conv.$), class embedding ($cls.$), Transformer layers, final projection ($proj.$) and positional embedding ($pos.$, not shown in Fig. \ref{fig:finetune}). We start with fine-tuning the entire IP-CLIP Encoder, and then freezing each unit sequentially, as specified in Tab. \ref{tab:ab-units}. We only freeze $MLP$ in the Transformer layers (dubbed as $mlp$). Compared with fine-tuning the entire IP-CLIP Encoder, the performance of mIoU$^u$ is improved by 5.0\% when freezing $conv.$, $cls.$, $pos.$ and $mlp$.

\noindent \textbf{Start mask attention layer}.
Tab. \ref{tab:ab-layer} presents the results of the start mask attention layer ($L$). 
We observe a significant improvement in the performance of unseen classes by +3.4\% when the value of $L$ increases from 0 to 8. This could be attributed to the fact that starting masked Multihead Attention later enables $F^{i*}_{cls}$ to gain more context information. However, the performance significantly drops when $L=10$ (from 49.7\% to 45.7\%), which may be due to the loss of mask-aware property.

\subsection{Extending MAFT with SAM}
% \begin{table*}
% \vspace{-2mm}
% \caption{Comparison with SAM.}
% \label{tab:sam}
% % \vspace{-1mm}
% \resizebox{1.0\textwidth}{!}{
% \centering
% % a - Ablation on Statics, Hand-Craft and Learnable
%     \subfloat[Results in zero-shot segmentation. Pascal-VOC.
%     \label{tab:sam-1}]
%     { 
%     % \!\!\!\!\!
%     % \renewcommand\tabcolsep{11.5pt}
%     \renewcommand\arraystretch{1.1} % 1.95
%     % \footnotesize
%     \small
%     \begin{tabular}
%     {l|llll}
%     \Xhline{0.7px}
%     \centering
%      & \textbf{mIoU$^s$} & \textbf{mIoU$^u$} & \textbf{hIoU} & \textbf{mIoU} \\
%     \hline
%     \multicolumn{1}{c|}{SAM} & 85.1 &  86.7 &  85.9 &  85.5\\
%     \multicolumn{1}{c|}{SAM + MAFT} & 91.0 _\textcolor{red}{+5.9} &  88.6 _\textcolor{red}{+1.9} &  89.8 _\textcolor{red}{+3.9} &  90.4 _\textcolor{red}{+4.9} \\
%     \Xhline{0.7px}
%     \end{tabular}
%     }
%     % \hspace{2mm}

% % b - Ablation on SI module
%     \subfloat[Results in zero-shot segmentation. COCO-Stuff.
%     \label{tab:sam-2}]
%     { 
%     % \!\!\!\!\!
%     % \renewcommand\tabcolsep{11.5pt}
%     \renewcommand\arraystretch{1.1} % 1.95
%     % \footnotesize
%     \small
%     \begin{tabular}
%     {l|llll}
%     % {p{27.8mm}|p{15mm}p{15mm}}
%     \Xhline{0.7px}
%     \centering
%      & \textbf{mIoU$^s$} & \textbf{mIoU$^u$} & \textbf{hIoU} & \textbf{mIoU} \\
%     \hline 
%     \multicolumn{1}{c|}{SAM} & 43.1 &  43.3 &  43.2 &  42.1\\
%     \multicolumn{1}{c|}{SAM + MAFT} & 43.4 _\textcolor{red}{+0.3} &  51.5 _\textcolor{red}{+8.2} &  47.1 _\textcolor{red}{+3.9} & 44.1 _\textcolor{red}{+2.0}\\
%     \Xhline{0.7px}
%     \end{tabular}
%     }
% }
% \resizebox{0.9\textwidth}{!}{
% \centering
%     \hspace{8mm}
%     \subfloat[Results in open-vocabulary segmentation.
%     \label{tab:sam-3}]
%     { 
%     \centering%
%     \footnotesize %scriptsize
%     \renewcommand\tabcolsep{10pt}
%     % \renewcommand\arraystretch{1.2} % 1.95

%     \begin{tabular}{c|lllll}
%       % \toprule
%       \Xhline{0.7pt}

%       & \textbf{A-847} & \textbf{A-150} & \textbf{PC-459} & \textbf{PC-59} & \textbf{PAS-20}\\ 
%       \hline
%        SAM & ~~~7.1 & ~~~17.9 & ~~~6.4 & ~~~34.4 & ~~~85.6 \\
%       SAM + MAFT & ~~~10.1 $_{\textcolor{red}{+3.0}}$ & ~~~29.1 $_{\textcolor{red}{+11.2}}$ & ~~~12.8 $_{\textcolor{red}{+6.4}}$ & ~~~53.5 $_{\textcolor{red}{+19.1}}$ & ~~~90.0 $_{\textcolor{red}{+4.4}}$ \\
%       \Xhline{0.7pt}
%       % \bottomrule
%       \end{tabular}
%     }    
% }
%   \vspace{-5mm}
% \end{table*}



\begin{table*}
\vspace{-2mm}
\caption{Comparison with SAM. We use SAM-H as the proposal generator.}
\label{tab:sam}
% \vspace{-1mm}
\resizebox{1.0\textwidth}{!}{
\centering
% a - Ablation on Statics, Hand-Craft and Learnable
    % \hspace{8mm}
    \subfloat[Results in zero-shot segmentation.
    \label{tab:sam-1}]
    { 
    % \!\!\!\!\!
    % \renewcommand\tabcolsep{11.5pt}
    \renewcommand\arraystretch{1.1} % 1.95
    \footnotesize
    % \small
    \begin{tabular}
    {l|llll|llll}
    \Xhline{0.7px}
    \centering
    & \multicolumn{4}{c|}{Pascal-VOC} & \multicolumn{4}{c}{COCO-Stuff} \\
     & \textbf{mIoU$^s$} & \textbf{mIoU$^u$} & \textbf{hIoU} & \textbf{mIoU} & \textbf{mIoU$^s$} & \textbf{mIoU$^u$} & \textbf{hIoU} & \textbf{mIoU} \\
    \hline
    \multicolumn{1}{l|}{SAM} & 85.1 &  86.7 &  85.9 &  85.5 & 43.1 &  43.3 &  43.2 &  42.1\\
    \multicolumn{1}{l|}{SAM + MAFT} & 91.0 $_{\textcolor{red}{+5.9}}$ &  88.6 $_{\textcolor{red}{+1.9}}$ &  89.8 $_{\textcolor{red}{+3.9}}$ &  90.4 $_{\textcolor{red}{+4.9}}$ & 43.4 $_{\textcolor{red}{+0.3}}$ &  51.5 $_{\textcolor{red}{+8.2}}$ &  47.1 $_{\textcolor{red}{+3.9}}$ & 44.1 $_{\textcolor{red}{+2.0}}$\\
    \Xhline{0.7px}
    \end{tabular}
    }
    % \hspace{2mm}
}
\resizebox{1.0\textwidth}{!}{
\centering
    % \hspace{8mm}
    \subfloat[Results in open-vocabulary segmentation.
    \label{tab:sam-3}]
    { 
    \centering%
    \footnotesize %scriptsize
    \renewcommand\tabcolsep{10pt}
    \renewcommand\arraystretch{1.1} % 1.95

    \begin{tabular}{l|lllll}
      % \toprule
      \Xhline{0.7pt}

      & \textbf{A-847} & \textbf{A-150} & \textbf{PC-459} & \textbf{PC-59} & \textbf{PAS-20}\\ 
      \hline
       SAM & ~~~7.1 & ~~~17.9 & ~~~6.4 & ~~~34.4 & ~~~85.6 \\
      SAM + MAFT & ~~~10.1 $_{\textcolor{red}{+3.0}}$ & ~~~29.1 $_{\textcolor{red}{+11.2}}$ & ~~~12.8 $_{\textcolor{red}{+6.4}}$ & ~~~53.5 $_{\textcolor{red}{+19.1}}$ & ~~~90.0 $_{\textcolor{red}{+4.4}}$ \\
      \Xhline{0.7pt}
      % \bottomrule
      \end{tabular}
    }    
}
  \vspace{-5mm}
\end{table*}

We explore using the Segment Anything Model \cite{kirillov2023segment} (SAM) as the proposal generator. We evaluate the performance with SAM-H using an original CLIP (dubbed $\mathrm{SAM}$) or a mask-aware fine-tuned CLIP (dubbed $\mathrm{SAM+MAFT}$). In fact, SAM can be seamlessly integrated into our framework as the proposal generator. The results are shown in Tab. \ref{tab:sam}. Experiments are conducted under both \textit{zero-shot} setting and \textit{open-vocabulary} setting.

It can be observed that $\mathrm{SAM+MAFT}$ obtains significant improvement over $\mathrm{SAM}$ under both settings. Besides, $\mathrm{SAM+MAFT}$ also surpasses $\mathrm{FreeSeg+MAFT}$ on all benchmarks. Particularly, in the zero-shot setting (Pascal-VOC), $\mathrm{SAM+MAFT}$ outperforms $\mathrm{FreeSeg+MAFT}$ by 6.8\% in terms of mIoU$^u$. This enhancement can be attributed to the stronger generalization capabilities of SAM for unseen classes. 

\subsection{Extending MAFT with more Vision-Language Models}
\begin{table}[h]
  \centering
  \footnotesize
 \vspace{-6mm}
  \caption{Comparison with more Vision-Language Models.}
\resizebox{1.0\textwidth}{!}{

    \renewcommand\arraystretch{1.1} % 1.95
    
    \begin{tabular}{l|c|lllll}
      \Xhline{0.7pt}

      & \textbf{backbone} & \textbf{~A-847} & \textbf{~A-150} & \textbf{PC-459} & \textbf{PC-59} & \textbf{PAS-20}\\ 
      \hline
       OVSeg \cite{ovseg} & \multirow{3}{*}{{ViT-L}} & ~~~9.0 & ~~~29.6 & ~~~12.4 & ~~~55.7 & ~~~94.5 \\
       FreeSeg \cite{freeseg} & & ~~~8.5 & ~~~21.0 & ~~~7.6 & ~~~33.8 & ~~~86.4 \\
      FreeSeg + MAFT & & ~~~12.1 $_{\textcolor{red}{+3.6}}$ & ~~~32.0 $_{\textcolor{red}{+11.0}}$ & ~~~15.7 $_{\textcolor{red}{+8.1}}$ & ~~~58.5 $_{\textcolor{red}{+24.7}}$ & ~~~92.1 $_{\textcolor{red}{+5.7}}$ \\
      \cdashline{1-7}[0.8pt/2pt]
      FreeSeg \cite{freeseg} &\multirow{2}{*}{{Res50}} &  ~~~5.3 & ~~~15.5 & ~~~5.4 & ~~~28.2 & ~~~87.1 \\
      FreeSeg + MAFT & & ~~~8.4 $_{\textcolor{red}{+3.1}}$ & ~~~27.0 $_{\textcolor{red}{+11.5}}$ & ~~~9.9 $_{\textcolor{red}{+4.5}}$ & ~~~50.8 $_{\textcolor{red}{+22.6}}$ & ~~~89.0 $_{\textcolor{red}{+1.9}}$ \\
      \Xhline{0.7pt}
      \end{tabular}
      }
      \label{tab:backbone}

 \end{table}

In order to demonstrate the efficacy and robustness of MAFT, we conduct experiments using stronger (CLIP-ViT-L) and ResNet-based (CLIP-Res50) Vision-Language Models. The open-vocabulary results are shown in Tab. \ref{tab:backbone}, we also include the results of OVSeg with CLIP-ViT-L for comparison.

\noindent \textbf{CLIP-ViT-L.}
According to Tab. \ref{tab:backbone}, FreeSeg with a standard CLIP-ViT-L model (dubbed $\mathrm{FreeSeg}$) still can not achieve satisfactory results. However, by integrating our MAFT (dubbed $\mathrm{FreeSeg+MAFT}$), the segmentation results are remarkably enhanced, thus establishing new state-of-the-art benchmarks.

\noindent \textbf{CLIP-Res50.}
Our MAFT can easily adapted into ResNet-based models. Specifically, we modified the $\mathrm{AttentionPool2d}$ unit within CLIP-R50 Image Encoder. The mask proposals are introduced as attention bias ($B$) in Multihead Attention, with $F_{cls}$ being repeated N times. Notably in CLIP-R50, $F_{cls}$ is obtained via $\mathrm{GlobalAveragePooling}$ performing on $F_{feat}$. The results are presented in Tab. \ref{tab:backbone}. The performance on all 5 datasets is improved by a large margin. $\mathrm{FreeSeg+MAFT}$ with CLIP-R50 achieves competitive results with some CLIP-ViT-B-based methods according to Tab. \ref{tab:ovs}.

\subsection{Qualitative Study}

\noindent \textbf{Visualizations of typical proposals.}
Fig. \ref{fig:vis-proposal} shows frozen CLIP and mask-aware CLIP classifications of typical proposals, 
including high-quality proposals of foreground ($p_1$, $p_4$), high-quality proposals of background ($p_3$, $p_6$), a proposal with background noise ($p_2$), and a proposal containing part of the foreground ($p_5$). The proposal regions are highlighted in green or yellow. \\
Several observations can be obtained: (1) The frozen CLIP provides good predictions for $p_1$ and $p_4$. (2) The frozen CLIP assigns $p_2$ as $cat$ and $p_5$ as $horse$, with scores even higher than $p_1$, $p_4$, indicating the frozen CLIP cannot distinguish proposals containing information on the same objects. (3) The frozen CLIP fails to give correct predictions for $p_3$ and $p_6$, which may be due to the lack of context information. (4) Our mask-aware CLIP gives good predictions for high-quality proposals ($p_1$, $p_3$, $p_4$, $p_6$) and provides suitable predictions for $p_2$ and $p_5$.


\noindent \textbf{Qualitative analysis.}
We show some visual examples in Fig. \ref{fig:vis-final}. Some segmentation results of FreeSeg contain background noise (\textit{e.g.} the $1^{st}$ \& $2^{nd}$ row, $3^{rd}$ column) or contain only part of the objects ($3^{rd}$ row, $3^{rd}$ column). In ADE20K-847 dataset, too many classes may lead to the anticipated results (last row, $3^{rd}$ column) with the frozen CLIP.
Using a mask-aware CLIP to learn mask-aware representations can significantly improve these segmentation results, as evident from the last column.

More visual samples are shown in the Appendix.
% !TEX root = 0-qqQQmain.tex

\section{Results} \label{results}

After the ultraviolet and infrared pole subtraction described in the previous section, we arrive at the main result of this paper, the fully analytical expressions for the finite remainders of the helicity amplitudes for process \eqref{s_channel}.
As previously mentioned,  the helicity amplitudes for other $2\to2$ quark processes with different initial states, including the equal-flavour case $q=Q$, can be obtained by a combination of analytical continuation and momenta renaming from the ones for our main process \eqref{s_channel}.  We  discuss this in more detail in Section \ref{extra_results}.
We provide all finite remainders  in electronic format as ancillary files attached
to the \texttt{arXiv} submission of this manuscript.

\subsection{Checks}
We have performed various checks on our results. First of all, we have verified that the IR poles of our scattering
amplitudes  follow the pattern predicted in refs~\cite{Becher:2009qa,Becher:2009cu,Almelid:2015jia} up to three loops.
We have then checked the finite part of our one loop amplitudes for all different partonic channels against the automated one-loop generator \texttt{OpenLoops}~\cite{Cascioli:2011va,Buccioni:2019sur}.
Finally, we have checked our one- and two-loop amplitudes through to order $\epsilon^4$ and $\epsilon^2$, respectively,
 against the results presented in refs~\cite{Glover:2004si,Ahmed:2019qtg}.

In order to successfully perform this check, one has to pay particular attention when comparing the amplitudes before IR subtraction, 
due to a subtlety in the  dimensional regularisation scheme 
used in refs~\cite{Glover:2004si,Ahmed:2019qtg}. This is due to the fact that the tensor structures
used to decompose the scattering amplitude in those references contain an explicit dependence on the dimensional-regulation
parameter $\epsilon$, even if the external states are taken to be in four dimensions, as in the 't Hooft-Veltman prescription.
While ignoring this dependence does not change the finite remainder of the scattering amplitudes after UV and IR poles have been subtracted,
it does change the bare results.
We illustrate this point explicitly for the one loop case. 
In refs \cite{Glover:2004si,Ahmed:2019qtg},  the following four tensors are used to decompose the amplitude at one loop
in CDR\footnote{Note that here even $\widetilde T_1$  and  $\widetilde T_2$ do not coincide with our definitions.}

\begin{align} 
\label{tensors_henn_D1} \widetilde T_1 &= {\bar u} (p_2)\: \slashed p_3 \: { u} (p_1) \times   {\bar u} (p_4) \: \slashed p_1 \:{ u}(p_3) \; , \\
\label{tensors_henn_D2} \widetilde T_2 &= {\bar u} (p_2)\: \gamma^\alpha \: { u} (p_1) \times {\bar u} (p_4) \: \gamma_\alpha \:{ u}(p_3) \; , \\
\label{tensors_henn_D3} \widetilde T_3 &= {\bar u} (p_2)\: \slashed p_3   \gamma^\mu   \gamma^\nu \: { u} (p_1) \times   {\bar u} (p_4) \:
 \slashed p_1 \gamma_\mu  \gamma_\nu  \:{ u}(p_3) \; , \\
\label{tensors_henn_D4} \widetilde T_4 &= {\bar u} (p_2)\: \gamma^\alpha \gamma^\mu  \gamma^\nu \: { u} (p_1) \times {\bar u} (p_4) \: \gamma_\alpha  \gamma_\mu   \gamma_\nu \:{ u}(p_3) \; , 
\end{align} 
where we stress that this decomposition is loop dependent and is only sufficient up to one-loop order, $L \leq 1$
\begin{equation}\label{decomp_tensors_henn}
\bar{\mathbfcal{A}}^{L \leq 1}  =   \widetilde{\mathbfcal{F}}_1 \; \widetilde{T}_1 + \widetilde{\mathbfcal{F}}_2 \; \widetilde{T}_2 + \widetilde{\mathbfcal{F}}_3 \; \widetilde{T}_3 + \widetilde{\mathbfcal{F}}_4 \; \widetilde{T}_4.
\end{equation}

Importantly, in the 't Hooft-Veltman scheme the vector indices of the $\gamma$ matrices in eqs.~\eqref{tensors_henn_D1}---\eqref{tensors_henn_D4} that are not 
explicitly contracted with four-dimensional vector fields 
are in general to be taken in 
$d$ dimensions. While this makes no difference for the 
first two tensors, the second two~(\ref{tensors_henn_D3},\ref{tensors_henn_D4})  
depend explicitly on $d$ and are responsible for an ambiguity 
in the way the helicity amplitudes are defined.
Let us consider for example the fourth tensors, eq.~\eqref{tensors_henn_D4}. 
If we consider the $\gamma$ matrices to carry $d$ dimensional vector indices we can
split the four dimensional part from the
$\epsilon$-dependent one by writing 
$\gamma^\mu_d = \gamma^\mu_4 + \gamma^{\mu}_{-2 \epsilon}$ and then use the equation 
\begin{equation}
    {\rm Tr}[\gamma_4^{\mu_1} \dots \gamma_4^{\mu_n} \gamma_{-2\epsilon}^{\nu_1} \dots \gamma_{-2\epsilon}^{\nu_m} ] = \frac{1}{4}     {\rm Tr}[\gamma_4^{\mu_1} \dots \gamma_4^{\mu_n} ]     {\rm Tr}[ \gamma_{-2\epsilon}^{\nu_1} \dots \gamma_{-2\epsilon}^{\nu_m} ] 
\end{equation}
as done in ref.~\cite{Cullen:2010jv} to extract the $(-2\epsilon)$-dimensional dependence of the $\gamma$-strings. All traces can then be evaluated as usual using the Clifford algebra relation $\{\gamma_\mu,\gamma_\nu\} = 2 g_{\mu \nu}$ for $d$-dimensional indices $\mu,\nu$. In this case, the dimensional splitting procedure simply amounts to $\epsilon$ dependent coefficients of the 4-dimensional $\gamma$-strings. For instance, taking the first string of $\widetilde{T}_4$ we find  
\begin{align}
    {\bar u} (p_2)\: \gamma^\alpha \gamma^\mu  \gamma^\nu \: { u} (p_1)& = g_{-2\epsilon}^{\mu \nu}  {\bar u} (p_2)\: \gamma_4^\alpha \: { u} (p_1) \nonumber \\
    &+ g_{-2\epsilon}^{\mu \alpha}  {\bar u} (p_2)\: \gamma_4^\nu \: { u} (p_1) \nonumber \\ 
    &-  g_{-2\epsilon}^{\nu \alpha}  {\bar u} (p_2)\: \gamma_4^\mu \: { u} (p_1),
\end{align}
where ${\rm Tr}[\gamma_{-2\epsilon}^\mu \gamma_{-2\epsilon}^\nu] = g_{-2\epsilon}^{\mu \nu} $ is the $(-2\epsilon)$-dimensional part of the metric tensor. Repeating the exercise for the other fermion string and then fixing the external helicities to $(+,-,+,-)$ 
we find
\begin{equation}
\widetilde T_4|_{(+,-,+,-)} = (32 - 12 \epsilon) \; [24]\langle 3 1 \rangle\,.
\end{equation}
Similarly, by repeating the same exercise for both helicities as we did 
in eqs.~\eqref{H_ij},  
but this time starting from the tensor decomposition in eq.~\eqref{decomp_tensors_henn},
we find
\begin{align}
\mathbfcal{H}_1 &=  -tu\: \left[ \widetilde{\mathbfcal{F}}_1 + 8 \widetilde{\mathbfcal{F}}_3 \right]
+2 t\left[ \widetilde{\mathbfcal{F}}_2 + 16 \widetilde{\mathbfcal{F}}_4 \right]  
+ 2 \epsilon \left[ t u\: \widetilde{\mathbfcal{F}}_3 - 6t \widetilde{\mathbfcal{F}}_4 \right] \: ,  \label{matrix_H1}\\
\mathbfcal{H}_2 &=  -tu \: \left[ \widetilde{\mathbfcal{F}}_1 + 4 \widetilde{\mathbfcal{F}}_3\right] 
- 2 u\left[ \widetilde{\mathbfcal{F}}_2 + 4 \widetilde{\mathbfcal{F}}_4 \right] 
+ 2\epsilon \left[ t u \:\widetilde{\mathbfcal{F}}_3 + 6 u\:\widetilde{\mathbfcal{F}}_4 \right] \; .\label{matrix_H2}
\end{align}
It is instructive to compare these formulas to the corresponding ones 
obtained in our approach, see eq.~\eqref{H1H2def}.
Our expressions for the helicity amplitudes, despite not displaying 
any explicit dependence on the parameter
$\epsilon$,  are exact in the 't Hooft-Veltman scheme. 
Notice, in particular, that the two tensors onto which we decompose the
amplitude,
${T}_1$ and ${T}_2$ in eq.~\eqref{tensors}, 
have been chosen such that it makes no practical difference
whether the $\gamma$ algebra to fix the helicity amplitudes
is performed $4$ or in $d=4 - 2 \epsilon$ dimensions. 
Upon substituting the form factors provided in refs~\cite{Glover:2004si,Ahmed:2019qtg} in eqs.~\eqref{matrix_H1} and \eqref{matrix_H2} (and in the corresponding
generalisations for the two-loop corrections),
we find perfect agreement up to weight six with 
our results for the \emph{bare helicity amplitudes}.

We stress, nevertheless, that the results for the bare helicity amplitudes as provided in refs~\cite{Glover:2004si,Ahmed:2019qtg} are obtained 
by setting $\epsilon=0$ in the coefficients of the form factors 
of \eqref{matrix_H1} and \eqref{matrix_H2} before substituting
the results for the form factors. This amounts to having assumed that the $\gamma$ matrices in eq.~\eqref{decomp_tensors_henn} are purely four-dimensional.  
This produces a difference for the bare amplitudes with respect to ours 
of order $\mathcal{O}(\epsilon)$ 
at one loop and $\mathcal{O}(1/\epsilon)$ at 
two loops.\footnote{Note that in this approach, one needs two more tensors 
at two loops, which depend quadratically on $\epsilon$.}  
However, it is easy to see that, as long as this choice is made 
consistently to all orders,  one obtains the same results for the 
finite remainders in $d=4$. 
In fact, one can imagine to first subtract UV and IR poles at the level of 
the individual form factors $\widetilde{\mathbfcal{F}}_j$ and, only afterwards,
substitute the finite form factors in eqs.~\eqref{matrix_H1} and \eqref{matrix_H2}, and fix $\epsilon = 0$. 
It is then obvious that the finite remainder cannot depend on 
the $\epsilon$-suppressed contributions in eqs.~\eqref{matrix_H1} 
and \eqref{matrix_H2}.
 We have verified the last statement directly, finding perfect agreement with refs~\cite{Glover:2004si,Ahmed:2019qtg} at the level
 of the finite remainders. 

\subsection{Numerical Evaluation}

\begin{figure}
\center
\includegraphics[width=1\textwidth]{fig/allH_1.pdf}
\caption{Real (left) and imaginary (right) parts of the form factors $\mathbfcal{H}_{1,\text{fin}}^{[i],(L)}$ relevant for helicities $(+,-,+,-)$.   Colour components $[i]$ and number of loops $(L)$ are specified in the legends.}
\label{allH_1}
\end{figure} 

\begin{figure}
\center
\includegraphics[width=1\textwidth]{fig/allH_2.pdf}
\caption{Real (left) and imaginary (right) parts of the form factors $\mathbfcal{H}_{2,\text{fin}}^{[i],(L)}$ relevant for helicities $(+,-,-,+)$.   Colour components $[i]$ and number of loops $(L)$ are specified in the legends. }
\label{allH_2}
\end{figure} 
We present numerical results for the finite form factors defined in  \eqref{one}, \eqref{two} and \eqref{three} calculated in the physical region $0<x<1$ for the process in eq.~\eqref{s_channel}, $q\bar{q}\to Q\bar{Q}$. To evaluate our results numerically we made use of the \texttt{Mathematica} package \texttt{PolyLogTools} \cite{Duhr:2019tlz},
which in turn uses the \texttt{Ginac} library~\cite{Bauer:2000cp,cln,Vollinga:2004sn}.
For the various parameters we use the following values:
\begin{equation}
N_c = 3, \quad n_f = 5, \quad \as = 0.118, \quad  \mu^2 = s.
\end{equation}
We show results for $\mathbfcal{H}_{1,\text{fin}}$ corresponding to helicities $(+,-,+,-)$ in figure~\ref{allH_1} and for $\mathbfcal{H}_{2,\text{fin}}$ corresponding to helicities $(+,-,-,+)$ in figure~\ref{allH_2}.
In the figures, we present the two colour components of the form factors individually, where we recall that our colour decomposition reads: 
\begin{equation}
{\mathbfcal{H}}_{i,\text{fin}}^{(L)}  
= 
\begin{pmatrix}
{\mathcal{H}}_{i,\text{fin}}^{[1],(L)}  \\
{\mathcal{H}}_{i,\text{fin}}^{[2],(L)} 
\end{pmatrix} \quad i=1,2 \; .
\end{equation}
Here, the colour index $[1]$ is related to the colour structure  ${\delta}_{ i_1 i_4} {\delta}_{i_2 i_3}$ while the index $[2]$ refers to the coefficient of ${\delta}_{ i_1 i_2} {\delta}_{ i_3 i_4}$.
Lastly, the index $(L)$ refers to the number of loops of the corresponding amplitude.  



\respace
\respace
\section{Conclusion}
\respace
\label{sec:conclusion}

We examine a challenging variant of active learning where target data is scarce, and multiple shifted domains operate as the source set of unlabeled data.
For practitioners facing multi-domain active learning, we benchmark 18 acquisition functions, demonstrating the $\mathcal{H}$-Divergence family of methods and our proposed variant \dale{} achieve the best results.
Our analysis shows the importance of example selection in existing methods, and also the surprising potential of domain budget allocation strategies.
Combining families of methods, or trying domain adaptation techniques on top of selected example sets, offer promising directions for future work.
\respace
\section*{Acknowledgements}
\respace
\label{sec:acknowledgments}

The authors would like to thank Moises Goldszmidt, Ehsan Mousavi, and Stephen Pulman for helpful discussions and feedback. 
\clearpage
% Entries for the entire Anthology, followed by custom entries
\bibliography{anthology,custom}
\bibliographystyle{acl_natbib}

\appendix

% \section{Example Appendix}
% \label{sec:appendix}

% This is an appendix.

\clearpage
\appendix

\section{Multi-Domain Active Learning Task}
\label{sec:appendix-task}

In this section, we would enumerate real-world settings in which a practitioner would be interested in multi-domain active learning methods.
We expect this active learning variant to be applicable to cold starts, rare classes, personalization, and settings where the modelers are constrained by privacy considerations, or a lack of labelers with domain expertise.

\respace
\begin{itemize}\itemsep0em
    \item In the \textbf{cold start} scenario, for a new NLP problem, there is often little to no target data available yet (labeled or unlabelled), but there are related sources of unlabelled data to try. 
    Perhaps an engineer has collected small amounts of training data from an internal population. Because the data size is small, the engineer is considering out-of-domain samples, collected from user studies, repurposed from other projects, scraped from the web, etc..
    \item In the \textbf{rare class} scenario, take an example of a new platform/forum/social media company classifying hate speech against a certain minority group. 
    Perhaps the prevalence of positive, in-domain samples on the social media platform is small, so an engineer uses out-domain samples from books, other social media platforms, or from combing the internet.
    \item In a \textbf{personalization} setting, like spam filtering or auto-completion on a keyboard, each user may only have a couple hundred of their own samples, but out-domain samples from other users may be available in greater quantities.
    \item In the \textbf{privacy constrained} setting, a company may collect data from internal users, user studies, and beta testers; however, a commitment to user privacy may incentivize the company to keep the amount of labeled data from the target user population low.
    \item Lastly, labeling in-domain data may require certain \textbf{domain knowledge}, which would lead to increased expenses and difficulty in finding annotators. 
    As an example, take a text classification problem in a rare language. 
    It may be easy to produce out-domain samples by labeling English text and machine translating it to the rare language, whereas generating in-domain labeled data would require annotators who are fluent in the rare language.
\end{itemize}
    
In each of these settings, target distribution data may not be amply available, but semi-similar unlabelled domains often are. 
This rules out many domain adaptation methods that rely heavily on unlabelled target data.

We were able to simulate the base conditions of this problem with sentiment analysis and question answering datasets, since they are rich in domain diversity. 
We believe these datasets are reasonable proxies to represent the base problem, and yield general-enough insights for a practitioner starting on this problem.

% Our evaluation setup is also designed for realistic measurements of cold-start active learning scenarios.
% In these settings, practitioners are likely to conduct one large active learning experiment to bootstrap the redo not continuously sampling 50-200 examples for labelling
% , where target-domain labelled data is limited in a cold start scenario.
% Whereas traditional active learning is sometimes evaluated by , over several iterations, we assume there is one big labelling batch
% for this setting, where target unlabelled data is not abundantly available.

\section{Reproducibility}
\label{sec:appendix-expdesign}

\subsection{Datasets and Model Training}

We choose question answering and sentiment analysis tasks as they are core NLP tasks, somewhat representative of many classification and information-seeking problems.
Multi-domain active learning is not limited to any subset of NLP tasks, so we believe these datasets are a reasonable proxie for the problem.

For question answering, the MRQA shared task \citep{fisch2019mrqa} includes SQuAD \citep{rajpurkar2016squad}, NewsQA \citep{trischler2016newsqa}, TriviaQA \citep{joshi2017triviaqa}, SearchQA \citep{dunn2017searchqa}, HotpotQA \citep{yang2018hotpotqa}, and Natural Questions \citep{kwiatkowski2019natural}.

For the sentiment analysis classification task, we use Amazon datasets following \citep{blitzer-etal-2007-biographies} and \citep{ruder2018strong}, as well as Yelp reviews \citep{asghar2016yelp} and IMDB movie reviews datasets \citep{maas-EtAl:2011:ACL-HLT2011}.~\footnote{\url{https://jmcauley.ucsd.edu/data/amazon/}, \url{https://www.yelp.com/dataset}, \url{https://ai.stanford.edu/~amaas/data/sentiment/}.}
Both question answering and sentiment analysis datasets are described in Table~\ref{datasets}.

For reproducibility, we share our hyper-parameter selection in Table~\ref{model-hyperparams}. 
Hyper-parameters are taken from \citet{longpre2019exploration} for training all Question Answering (QA) models since their parameters are tuned for the same datasets in the MRQA Shared Task. 
We found these choices to provide stable and strong results across all datasets.
For sentiment analysis, we initially experimented on a small portion of the datasets to arrive at a strong set of base hyper-parameters to tune from.

Our BERT question answering modules build upon the standard PyTorch \citep{NEURIPS2019_9015} implementations from HuggingFace, and are trained on one NVIDIA Tesla V100 GPU.\footnote{\url{https://github.com/huggingface/transformers}}.

\vspace{.5cm}
\begin{table}[htb]
	\small
	\centering
	\begin{tabular}{ll}
		\toprule
		\textbf{Model Parameter} & \textbf{Value} \\
		\midrule
		Base Pre-trained Model & BERT-base \\
		Model Size (\# params) & $108.3M$ \\
		
		Learning Rate & $5e-5$ \\
		Optimizer & Adam \\
		Gradient Accumulation & 1 \\
		Dropout & $0.1$ \\
		Lower Case & False \\
		\midrule
		\textbf{Question Answering model} & {} \\
		Avg. Train Time & $2h\ 20m$ \\
		Batch Size & 25 \\
		Num Epochs & 2 \\
		Max Query Length & 64 \\
		Max Sequence Length & 512 \\
		\midrule
		\textbf{Sentiment Classifcation model} & {} \\
		Avg. Train Time & $43m$ \\
		Batch Size & 20 \\
		Num Epochs & 3 \\
		Max Sequence Length & 128 \\
		\bottomrule
	\end{tabular}
	\caption{\label{model-hyperparams}
		Hyperparameter selection for task models.
	}
\end{table}

\subsection{Experimental Design}

For more detail regarding the experimental design we include Algorith~\ref{alg:exp-design}, using notation described in the multi-domain active learning task definition.

\begin{algorithm}[htb]
    \caption{\label{alg:exp-design}\textsc{Experimental Design}}
  \label{alg:experiment}
\begin{algorithmic}[1]
    % \SetCommentSty{itshape}
% \SetKwComment{Comment}{$\triangleright$ }{}
  \footnotesize
  \FOR{each Acquisition Function $A_f$}
  \FOR{each Target set $D_T \sim D$}
  \STATE $D_{T}^{train}, D_{T}^{dev}, D_{T}^{test} \sim D_T$ \\
  \STATE $D_{S} := \{x \in D \; | \; x \notin D_T \}$
  \STATE $M_A \leftarrow \textsc{Train}(D_{T}^{train}, D_{T}^{dev})$\\
  \STATE $D^{chosen} \leftarrow [\text{Rank}_{x \in D_{S}} A_{f}(x, M_A)][:n]$\\
  \STATE $D^{final-train} = D_T^{train} \cup D^{chosen}$
  \STATE $M \leftarrow \textsc{GridSearch} (D^{final-train}, D_{T}^{dev})$\\
  \STATE $(A_f, D_T) = s_T^{A_f} \leftarrow  M(D_{T}^{test}) $ \\
  \ENDFOR
  \ENDFOR

  \RETURN Scores Dictionary $(A_f, D_T) \rightarrow s_T^{A_f}$ \\
\end{algorithmic}
\end{algorithm}


% \begin{algorithm}[t]
%     \caption{\textsc{Experimental Design}}
%   \label{alg:experiment}
% \begin{algorithmic}[1]
%     % \SetCommentSty{itshape}
% % \SetKwComment{Comment}{$\triangleright$ }{}
%   \footnotesize
%   \REQUIRE{Model $M_{1}$, Acquisition Method $A$, Target sets ($D^{T}_{train}$, $D^{T}_{val}$, $D^{T}_{test}$), Source sets $D^{j}_{source}\sim Q_j$ Sample Size $n$}
%   \ENSURE{Accuracy $s$ of each acquisition method on each target test set \{($A$, $D^T$) $ \rightarrow s^T_A$\}} \\
%   \STATE $|D^{T}_{train}|$ $\gets$ $2,000$
% %   \STATE $D_{train}$ $\gets$ {sample(shuffle($D$), $N$)}
  
%   \FOR{each Acquisition Method $A$}
%   \FOR{each Target set $T$}
%   \STATE // \textit{Train $M_1$ on $D^T_{train}$, with $D^T_{val}$ validation}\\
%   \STATE $M_1 \leftarrow $ \textsc{Train}($M_1$, $D^T_{train}$, $D^T_{val}$)\\
  
% %   \STATE $M_{\alpha} \leftarrow $ $M.$\textsc{Use}($H_{\alpha}$) \\ \label{alg:apply-hyp}
  
%   \STATE // \textit{Use $A$ to select $n$ unlabelled examples $E$ from $D_{source}$.}
%   \STATE $E \leftarrow $ \textsc{A}($M_1$, $D_{source}$)\\
  
%   \STATE // \textit{Select best new model $M_2$ tuned on $E$}
%   \STATE $M_2 \leftarrow $ \textsc{GridSearch}($(D^T_{train},E)$, $D^T_{val}$)\\
%   \STATE $($A$, $D^T$):=s^T_A \leftarrow $ $M_2(D^T_{test})$ \\
%   \ENDFOR
%   \ENDFOR

%   \RETURN Scores Dict $($A$, $D^T$) \rightarrow s^T_A$ \\
% \end{algorithmic}
% \end{algorithm}

\section{Acquisition functions}

\subsection{Task Agnostic Embeddings}

To compute the semantic similarity between two examples, we computed the example embeddings using the pre-trained model from a sentence-transformer \citep{reimers-2019-sentence-bert}. 
We used the RoBERTa large model, which has 24 layers, 1024 hidden layers, 16 heads, 355M parameters, and fine tuning on the SNLI \citep{snli:emnlp2015}, MultiNLI \citep{N18-1101}, and STSBenchmark \citep{cer2017semeval} datasets. 
Its training procedure is documented in \url{https://www.sbert.net/examples/training/sts/README.html}. 

\subsection{Bayesian active learning by disagreement (BALD)}
We note that Siddant and Lipton's presentation of BALD is more closely related to the Variation Ratios acquisition function described in \citet{gal2017deep} than the description of dropout as a Bayesian approximation given in \citet{gal2016}. 
In particular, \citet{gal2017deep} found that Variation Ratios performed on par or better than Houlsby's BALD on MNIST but was less suitable for ISIC2016.

\subsection{Discriminative Active Learning Model (DAL) Training}
\dal{}'s training set is created using the methods detailed in Section~\ref{sec:h-divergence-methods}. 
The training set is then partitioned into five equally sized folds. 
In order to predict on data that is not used to train the discriminator, we use 5-fold cross validation. 
The model is trained on four folds, balancing the positive and negative classes using sample weights. 
The classifier then predicts on the single held-out fold. 
This process is repeated five times so that each example is in the held out fold exactly once. 
Custom model parameters are shown in Table~\ref{tbl:dal-hyperparams}; model parameters not shown in the table are the default XGBClassifier parameters in xgboost 1.0.2.
The motivations for choice in model and architecture are the small amount of target domain examples requiring a simple model to prevent overfitting and the ability of decision trees to capture collective interactions between features. 

\section{Full Method Performances}
We provide a full breakdown of final method performances in Tables~\ref{tab:MRQA-performance} and ~\ref{tab:sent-performance}.

\vspace{.5cm}
\begin{table*}[t]
	\small
	\centering
	\begin{tabular}{ll}
		\toprule
		\textbf{Model Parameter} & \textbf{Value} \\
		\textbf{DAL Discriminator} & {} \\
		\midrule
		Model Type & XGBoost \\
		Model Size (\# trees) &  10\\
		Model Size (maximum depth) &  2\\
		\midrule
		Learning Rate &  0.1\\
		Objective &  binary:logistic\\
		Booster & gbtree\\
		Tree Method & gpu\_hist\\
		Gamma & 5 \\
		Min Child Weight & 5 \\
		Max Delta Step & 0 \\
		Subsample & 1 \\
		Colsample Bytree & 1 \\
		Colsample Bynode & 1 \\
		Reg Alpha & 0 \\
		Reg Lambda & 5 \\
		Scale Pos Weight & 1 \\
		\bottomrule
	\end{tabular}
	\caption{Hyperparameter selection for \dal{} discriminators.}
	\label{tbl:dal-hyperparams}
\end{table*}

% \toprule
\begin{table*}[t]
\centering
\begin{adjustbox}{width=1\textwidth}
\small
\begin{tabular}{ | l | l | ccccccccccccccccccc |}
  \midrule  \textbf{Train Size} & \textbf{Target} & \textbf{random} & \textbf{\confa} & \textbf{\confd} & \textbf{\entra} & \textbf{\entrd} & \textbf{\eoda} & \textbf{\eodd} & \textbf{\balda} & \textbf{\baldd} & \textbf{\dales} & \textbf{\dalts} & \textbf{\dale} & \textbf{\dalt} & \textbf{\rca} & \textbf{\rcas} & \textbf{\knns} & \textbf{\knnc} & \textbf{\knnq} & \textbf{\knnqc} \\ \midrule
\multirow{6}{*}{10000} & \abr{HotpotQA} & 65.76 & 64.15 & 64.38 & 64.59 & \underline{\textbf{66.03}} & 65.39 & 62.39 & 63.13 & 61.45 & 65.42 & 65.33 & 63.58 & 63.18 & 65.19 & 65.33 & 62.25 & 64.28 & 63.98 & 63.51\\ 
 
 & \abr{NaturalQ} & 63.05 & 61.59 & 62.14 & \underline{\textbf{64.61}} & 63.56 & 61.52 & 62.44 & 58.35 & 61.56 & 63.0 & 62.79 & 62.54 & 62.7 & 58.72 & 62.73 & 59.75 & 61.94 & 63.28 & 61.84\\ 
 
 & \abr{NewsQA} & 53.51 & 47.72 & 51.82 & 54.14 & 52.85 & 52.54 & 48.06 & 55.61 & 52.76 & 51.36 & 50.93 & 54.41 & 54.69 & \underline{\textbf{55.93}} & 54.31 & 50.77 & 53.13 & 55.52 & 52.91\\ 
 
 & \abr{SearchQA} & 62.83 & 58.46 & 63.84 & 63.12 & 64.25 & 62.18 & 63.22 & 63.26 & \underline{\textbf{65.12}} & 62.6 & 62.59 & 63.28 & 63.31 & 62.32 & 62.03 & 61.84 & 63.84 & 63.27 & 62.39\\ 
 
 & \abr{SQuAD} & 75.97 & 73.23 & 75.33 & 76.28 & 73.41 & 76.22 & 73.07 & 75.65 & 73.13 & 76.61 & 76.75 & \underline{\textbf{77.0}} & 76.88 & \underline{\textbf{77.0}} & 76.25 & 76.74 & 74.24 & 75.08 & 74.94\\ 
 
 & \abr{TriviaQA} & 61.44 & 58.19 & 60.17 & 59.75 & 57.57 & 59.64 & 59.4 & 60.02 & 58.32 & 61.89 & 61.24 & \underline{\textbf{61.94}} & 61.06 & 58.88 & 60.81 & 60.45 & 59.98 & 60.82 & 60.37\\ 
\midrule 
\multirow{6}{*}{20000} & \abr{HotpotQA} & 66.29 & 64.12 & 64.3 & 65.15 & \underline{\textbf{67.53}} & 65.86 & 63.86 & 67.51 & 63.76 & 67.05 & 67.13 & 64.48 & 64.23 & 65.81 & 66.78 & 61.96 & 64.14 & 64.68 & 64.13\\ 
 
 & \abr{NaturalQ} & 63.62 & 63.65 & 62.12 & \underline{\textbf{64.87}} & 64.11 & 63.3 & 60.32 & 64.86 & 63.63 & 63.99 & 64.14 & 63.98 & 63.38 & 59.21 & 63.76 & 60.34 & 61.54 & 63.81 & 62.43\\ 
 
 & \abr{NewsQA} & 54.71 & 48.32 & 52.68 & 55.44 & 54.78 & 53.01 & 47.56 & \underline{\textbf{57.69}} & 55.2 & 52.15 & 52.1 & 55.62 & 56.29 & 57.33 & 57.47 & 50.16 & 53.55 & 55.5 & 54.94\\ 
 
 & \abr{SearchQA} & 62.53 & 61.93 & 64.08 & 63.51 & 62.56 & 61.46 & 63.21 & 64.27 & \underline{\textbf{67.22}} & 62.92 & 63.14 & 63.65 & 63.3 & 62.13 & 63.01 & 62.24 & 64.88 & 63.32 & 63.84\\ 
 
 & \abr{SQuAD} & 76.32 & 75.33 & 75.53 & 77.61 & 72.17 & 76.54 & 72.79 & 78.02 & 74.15 & 77.51 & 77.72 & 77.7 & 77.59 & \underline{\textbf{78.57}} & 78.0 & 77.79 & 75.93 & 76.56 & 76.27\\ 
 
 & \abr{TriviaQA} & 62.45 & 61.37 & 61.97 & 61.64 & 61.21 & 62.38 & 60.2 & 61.74 & 60.54 & \underline{\textbf{63.38}} & 62.56 & 62.7 & 61.99 & 59.76 & 61.65 & 62.54 & 61.83 & 62.08 & 62.84\\ 
\midrule 
\multirow{6}{*}{30000} & \abr{HotpotQA} & 65.98 & 64.79 & 66.33 & 64.43 & 68.3 & 65.76 & 63.39 & \underline{\textbf{69.17}} & 63.44 & 67.09 & 67.51 & 64.91 & 65.34 & 65.92 & 67.79 & 62.32 & 64.09 & 65.85 & 64.86\\ 
 
 & \abr{NaturalQ} & 63.61 & 63.49 & 63.18 & 64.51 & 64.65 & 63.87 & 62.68 & \underline{\textbf{66.4}} & 63.62 & 64.66 & 65.12 & 64.84 & 64.24 & 59.18 & 63.64 & 61.63 & 62.32 & 64.24 & 62.66\\ 
 
 & \abr{NewsQA} & 55.18 & 47.73 & 54.26 & 56.79 & 54.48 & 54.62 & 48.38 & \underline{\textbf{58.4}} & 56.7 & 53.48 & 53.48 & 55.63 & 56.17 & 57.7 & 56.84 & 49.19 & 54.89 & 56.24 & 54.54\\ 
 
 & \abr{SearchQA} & 62.28 & 61.9 & 62.86 & 63.73 & 63.5 & 62.17 & 63.85 & 66.67 & \underline{\textbf{68.61}} & 62.97 & 63.61 & 63.52 & 63.3 & 61.89 & 62.99 & 62.4 & 63.7 & 63.37 & 63.76\\ 
 
 & \abr{SQuAD} & 77.75 & 74.1 & 76.78 & 76.98 & 75.08 & 76.76 & 73.08 & 78.7 & 77.04 & 79.21 & 78.71 & 78.08 & 79.24 & \underline{\textbf{80.18}} & 78.38 & 78.76 & 75.77 & 77.88 & 77.13\\ 
 
 & \abr{TriviaQA} & 63.2 & 62.34 & 61.98 & 62.01 & 61.87 & 62.98 & 60.13 & 61.85 & 62.49 & \underline{\textbf{64.36}} & 64.35 & 63.21 & 62.97 & 61.36 & 62.81 & 63.22 & 62.94 & 62.91 & 63.89\\ 
\midrule
\end{tabular}
\end{adjustbox}
\caption{\label{tab:MRQA-performance}MQRA F1 scores from each active learning method over every training set size and target domain. The best performances are bolded and underlined.}
\end{table*} 
% \toprule
\begin{table*}[!ht]
\centering
\begin{adjustbox}{width=1\textwidth}
\small
\begin{tabular}{ | l | l | ccccccccccccccccc |}
  \midrule  \textbf{Train Size} & \textbf{Target} & \textbf{random} & \textbf{\confa} & \textbf{\confd} & \textbf{\entra} & \textbf{\entrd} & \textbf{\eoda} & \textbf{\eodd} & \textbf{\balda} & \textbf{\baldd} & \textbf{\dales} & \textbf{\dalts} & \textbf{\dale} & \textbf{\dalt} & \textbf{\rca} & \textbf{\rcas} & \textbf{\knns} & \textbf{\knn} \\ \midrule
\multirow{8}{*}{10000} & \abr{Amzn-B} & 65.04 & \underline{\textbf{68.66}} & 65.36 & 65.32 & 68.08 & 66.38 & 68.62 & 64.46 & 68.2 & 67.16 & 66.98 & 68.28 & 67.68 & 67.24 & 68.3 & 65.66 & 67.06\\ 
 
 & \abr{Amzn-H} & 66.36 & 68.98 & 66.32 & 67.36 & 68.84 & 65.98 & 69.3 & 66.64 & 69.36 & \underline{\textbf{70.04}} & 68.4 & 69.32 & 69.1 & 69.6 & 69.14 & 68.52 & 68.84\\ 
 
 & \abr{Amzn-M} & 68.38 & 70.2 & 67.3 & 68.1 & 69.66 & 67.4 & 70.4 & 67.42 & 69.74 & \underline{\textbf{70.42}} & 69.4 & 70.16 & 70.06 & 69.88 & 70.08 & 69.44 & 69.56\\ 
 
 & \abr{Amzn-So} & 61.06 & 63.94 & 61.92 & 61.38 & 64.32 & 62.42 & 64.3 & 61.24 & 64.3 & 63.46 & 63.04 & 64.2 & 64.22 & 62.4 & 64.12 & 63.72 & \underline{\textbf{64.42}}\\ 
 
 & \abr{Amzn-Sp} & 64.92 & 67.12 & 64.22 & 64.68 & 66.5 & 64.68 & 66.1 & 64.06 & 67.58 & 67.12 & 66.66 & 68.04 & 67.62 & \underline{\textbf{68.14}} & 66.98 & 66.16 & 66.94\\ 
 
 & \abr{Amzn-T} & 65.4 & 67.88 & 65.94 & 65.54 & 67.26 & 65.56 & 68.02 & 64.64 & 67.8 & \underline{\textbf{68.44}} & 65.68 & 67.86 & 68.24 & 66.66 & 67.06 & 65.36 & 67.64\\ 
 
 & \abr{Imdb} & 58.05 & 59.32 & 59.48 & 58.76 & 58.78 & 58.88 & 58.54 & 58.02 & 59.9 & 59.68 & 60.46 & 60.4 & \underline{\textbf{60.52}} & 59.94 & 59.52 & 58.96 & 60.1\\ 
 
 & \abr{Yelp} & 66.75 & 64.94 & 63.82 & 64.36 & 65.58 & 63.46 & 65.88 & 64.42 & 66.38 & 66.06 & 66.4 & 65.84 & 66.98 & 66.0 & \underline{\textbf{67.04}} & 66.24 & 65.46\\ 
\midrule 
\multirow{8}{*}{20000} & \abr{Amzn-B} & 64.68 & \underline{\textbf{69.12}} & 65.92 & 65.18 & 68.46 & 67.08 & 69.04 & 66.5 & 68.88 & 68.18 & 65.64 & 68.16 & 68.68 & 67.9 & 67.88 & 66.26 & 67.64\\ 
 
 & \abr{Amzn-H} & 67.16 & 69.46 & 65.04 & 64.94 & 69.54 & 65.94 & \underline{\textbf{70.32}} & 65.32 & 69.84 & \underline{\textbf{70.32}} & 68.08 & 69.94 & 70.04 & 70.16 & 70.28 & 67.66 & 69.18\\ 
 
 & \abr{Amzn-M} & 68.76 & 70.86 & 66.2 & 67.84 & 69.98 & 66.18 & 70.82 & 66.7 & 70.52 & \underline{\textbf{71.48}} & 69.56 & 70.84 & 70.54 & 71.32 & 69.86 & 68.78 & 70.28\\ 
 
 & \abr{Amzn-So} & 61.5 & 64.98 & 62.1 & 62.28 & \underline{\textbf{65.56}} & 61.66 & 65.2 & 61.82 & 65.34 & 64.7 & 64.74 & 64.88 & 64.56 & 63.18 & 64.22 & 64.26 & 65.3\\ 
 
 & \abr{Amzn-Sp} & 65.68 & 67.18 & 65.04 & 63.78 & 67.14 & 65.66 & 66.34 & 63.52 & 67.36 & 68.22 & \underline{\textbf{68.78}} & 68.36 & 68.72 & 68.42 & 67.54 & 65.32 & 67.84\\ 
 
 & \abr{Amzn-T} & 65.92 & 68.1 & 66.26 & 65.04 & 68.44 & 65.52 & 68.18 & 64.76 & 69.02 & 69.62 & 65.76 & \underline{\textbf{69.72}} & 69.1 & 67.12 & 68.38 & 66.6 & 68.58\\ 
 
 & \abr{Imdb} & 58.58 & 59.56 & 58.88 & 58.38 & 58.74 & 59.74 & 58.76 & 58.84 & 58.96 & 60.2 & \underline{\textbf{60.76}} & 60.1 & 60.06 & 59.94 & 60.54 & 59.38 & 59.98\\ 
 
 & \abr{Yelp} & 66.39 & 66.52 & 62.92 & 64.34 & 65.74 & 63.34 & 66.18 & 64.06 & 66.62 & \underline{\textbf{67.6}} & 66.9 & 67.2 & 66.16 & 65.98 & 66.86 & 65.46 & 67.4\\ 
\midrule 
\multirow{8}{*}{30000} & \abr{Amzn-B} & 65.18 & 68.96 & 65.02 & 63.42 & 68.9 & 65.96 & 69.12 & 63.72 & \underline{\textbf{69.42}} & 68.86 & 67.16 & 69.22 & 68.08 & 68.2 & 69.06 & 66.32 & 68.42\\ 
 
 & \abr{Amzn-H} & 67.0 & \underline{\textbf{71.1}} & 64.82 & 64.62 & 69.92 & 64.78 & 70.54 & 63.56 & 70.38 & 70.86 & 67.96 & 70.38 & 70.6 & 70.32 & 70.2 & 68.46 & 70.74\\ 
 
 & \abr{Amzn-M} & 69.48 & 70.96 & 67.16 & 66.34 & 70.5 & 68.06 & 71.14 & 66.48 & 71.14 & \underline{\textbf{71.56}} & 70.28 & 70.38 & 71.0 & 71.28 & 70.98 & 68.92 & 70.64\\ 
 
 & \abr{Amzn-So} & 62.94 & 66.06 & 62.0 & 61.56 & 66.06 & 61.76 & 65.98 & 60.52 & \underline{\textbf{66.36}} & 65.88 & 66.0 & 65.58 & 65.8 & 63.44 & 65.98 & 64.58 & 66.22\\ 
 
 & \abr{Amzn-Sp} & 67.06 & 67.82 & 63.44 & 63.08 & 68.0 & 64.14 & 67.82 & 63.6 & 69.16 & 68.7 & 67.86 & \underline{\textbf{69.38}} & 68.56 & 68.42 & 68.3 & 66.24 & 67.96\\ 
 
 & \abr{Amzn-T} & 66.1 & 69.04 & 65.8 & 66.14 & 68.2 & 66.96 & 69.0 & 63.4 & 69.62 & \underline{\textbf{70.22}} & 67.08 & 70.0 & 69.62 & 68.1 & 68.8 & 67.2 & 69.72\\ 
 
 & \abr{Imdb} & 59.67 & 58.9 & 59.84 & 57.9 & 59.1 & 59.66 & 59.3 & 58.58 & 59.76 & 59.78 & \underline{\textbf{61.58}} & 60.7 & 60.8 & 60.6 & 60.32 & 60.4 & 60.64\\ 
 
 & \abr{Yelp} & 66.93 & 66.28 & 63.68 & 64.28 & 66.7 & 63.38 & 67.34 & 63.62 & 67.16 & 66.78 & 67.46 & \underline{\textbf{68.2}} & 66.94 & 66.22 & 67.18 & 65.54 & 67.82\\ 
\midrule 
\end{tabular}
\end{adjustbox}
\caption{\label{tab:sent-performance}Sentiment accuracy scores from each active learning method over every training set size and target domain. The best performances are bolded and underlined.}
\end{table*} 

% \section{Comparing Example Rankings}
% \begin{figure*}
%   \includegraphics[width=\textwidth]{figures/KendallsTau_combined.pdf}
%   \caption{\label{fig:kendall-tau-heatmap} Kendall's Tau Coefficients for MRQA (above diagonal) and Sentiment (below diagonal). 
%   Kendall's Tau coefficient is computed between the example rankings of each pair of methods. 
%   The heatmap contains these coefficients averaged over each target dataset. 
%   1 indicates a perfect relationship between the rankings, 0 means no relationship, and -1 means an anti-relationship.}
% \end{figure*}


\section{Kendall's Tau}
\subsection{Definition}
Kendall's Tau is a statistic that measures the rank correlation between two quantities. 
Let $X$ and $Y$ be random variables with $(x_1, y_1), (x_2, y_2), ..., (x_n, y_n)$ as observations drawn from the joint distribution. 
Given a pair $(x_i, y_i)$ and $(x_j, y_j)$, where $i\neq j$, we have:

$\frac{y_j-y_i}{x_j-x_i}>0:$ pair is concordant

$\frac{y_j-y_i}{x_j-x_i}<0:$ pair is discordant

$\frac{y_j-y_i}{x_j-x_i}=0:$ pair is a tie

Let $n_c$ be the number of concordant pairs and $n_d$ the number of discordant pairs. Let ties add 0.5 to the concordant and discordant pair counts each. Then, Kendall's Tau is computed as:\footnote{\url{https://www.itl.nist.gov/div898/software/dataplot/refman2/auxillar/kendell.htm}}

$\tau = \frac{n_c-n_d}{n_c+n_d}$

\begin{figure}[h]
\centering
\begin{subfigure}{\linewidth}
  \centering
  \includegraphics[width=\linewidth]{figures/KendallsTauMRQANormalized.pdf}
  \caption{MRQA}
  \label{fig:sub1}
\end{subfigure}
\begin{subfigure}{\linewidth}
  \centering
  \includegraphics[width=\linewidth]{figures/KendallsTauSentNormalized.pdf}
  \caption{Sentiment}
  \label{fig:sub2}
\end{subfigure}
\caption{Kendall Tau scores normalized by intra-family scores according to the family of the method on the y-axis (with uncertainty-ascending and uncertainty-descending as distinct families). If the cell's corresponding Kendall Tau score  is within the intra-family range, it's value will be in $[0, 1]$. Below the range is negative, and above the range is greater than 1.} 
\label{fig:kt-normalized}
\end{figure}

\subsection{Inter-Family Comparison}
Here, we extend on our comparison of example rankings by presenting plots of Kendall Tau scores normalized by intra-family scores in \ref{fig:kt-normalized}. 
For the sentiment setting, the ranges of intra-family Kendall Tau coefficients are smaller than the MRQA setting. 
Methods in the uncertainty family have especially strong correlations with each other and much weaker with methods outside of the family. 
For H-divergence based methods, intra-family correlations are not’t as strong as for the uncertainty family; in fact, the Kendall Taus between \dale{}/\knn{} and \dalt{}/\knn{} appear to be slightly within the H-divergence intra-family range. 

Furthermore, intra-family ranges are quite large for all families in the MRQA setting. 
For each method, there is at least one other method from a different family with which it had a higher Kendall Tau coefficient than the least similar methods of its own family.

\section{Relating Domain Distances to Performance}
\label{appendix:domain-distances}

% From results in Figures \ref{fig:mrqa-performances} and \ref{fig:sent-performances}, we observe the best method for a given target set is often difficult to predict.
% In an attempt to understand why certain methods work well, we hoped to derive some relationship between the distribution of examples selected (close or far from the target), and the final performance of this selection.
    
% We estimated the \textit{distance} between two domains by computing the wasserstein distance between their constitutent example embeddings (we tried with both task-agnostic and task-specific encoders).
% Unfortunately, we found no discernable relationship between these distances and the performances of single domain baselines.
% We believe this is either because our estimated distances were simply not reliable measures of domain relevance, since each target clearly benefited more from some domains than others during training, but the distances had no correlation.
% We hypothesize that a single scalar value, for domain distance, is not sufficient to represent the relevance, informativeness, and diversity of potential training sources.
    
We investigated why certain methods work better than others. 
One hypothesis is that there exists a relationship between between target-source domain distances and method performance. 
We estimated the distance between two domains by computing the Wasserstein distance between random samples of 3k example embeddings from each domain. 
We experimented with two kinds of example embeddings: 1. A task agnostic embedding computed by the sentence transformer used in the \knn{} method, and 2. A task specific embedding computed by a model trained with the source domain used in the \dals{} method. 
Given that there are $k - 1$ source domains for each target domain, we tried aggregating domain distances over its mean, minimum, maximum, and variance to see if Wasserstein domain distances could be indicative of relative performance across all methods.

Figure \ref{fig:avg_distance_perf}, Figure \ref{fig:min_distance_perf}, Figure \ref{fig:max_distance_perf}, and Figure \ref{fig:var_distance_perf} each show, for a subset of methods, the relationship between each domain distance aggregation and the final performance gap between the best performing method. 
Unfortunately, we found no consistent relationship for both MRQA and the sentiment classification tasks. 
We believe that this result arose either because our estimated domain distances were not reliable measures of domain relevance, or because the aggregated domain distances are not independently sufficient to discern relative performance differences across methods.

\begin{figure*}[ht]
  \centering
     \caption*{Figures 4-7: The average domain distance is calculated by finding the distance between 3k examples from $D_T$ and the combined set made from choosing 3k examples from each domain in $D_S$. 
     Since the Wasserstein metric is symmetric, this yields $k$ points for comparison.}
\includegraphics[width=\textwidth]{figures/avg_distance_perf_flatten.pdf}
    \caption{\label{fig:avg_distance_perf} Average Wasserstein domain distance vs performance.}
\end{figure*}
\respace
\begin{figure*}[ht]
  \centering
    \includegraphics[width=\textwidth]{figures/min_distance_perf_v2.pdf}
    \caption{\label{fig:min_distance_perf} Minimum Wasserstein domain distance vs method performance.}
\end{figure*}
\respace
\begin{figure*}[ht]
  \centering
    \includegraphics[width=\textwidth]{figures/max_distance_perf_flatten.pdf}
    \caption{\label{fig:max_distance_perf} Maximum Wasserstein domain distance vs method performance.}
\end{figure*}
\clearpage
\begin{figure*}[ht]
  \centering
    \includegraphics[width=\textwidth]{figures/var_distance_perf_flatten.pdf}
    \caption{\label{fig:var_distance_perf} Wasserstein Domain distance variance vs performance.}
\end{figure*}

\end{document}
