% \section{Bib\TeX{} Files}
% \label{sec:bibtex}

% Unicode cannot be used in Bib\TeX{} entries, and some ways of typing special characters can disrupt Bib\TeX's alphabetization. The recommended way of typing special characters is shown in Table~\ref{tab:accents}.

% Please ensure that Bib\TeX{} records contain DOIs or URLs when possible, and for all the ACL materials that you reference.
% Use the \verb|doi| field for DOIs and the \verb|url| field for URLs.
% If a Bib\TeX{} entry has a URL or DOI field, the paper title in the references section will appear as a hyperlink to the paper, using the hyperref \LaTeX{} package.
\respace
\respace
\section{Experiments}
\respace
\label{sec:experiments}

Experiments are conducted on two common NLP tasks: question answering (QA) and sentiment analysis (SA), each with several available domains.

\respace
\paragraph{Question Answering} 
We employ 6 diverse QA datasets from the MRQA 2019 workshop \citep{fisch2019mrqa}, shown in Table~\ref{datasets}.\footnote{The workshop pre-processed all datasets into a similar format, for fully answerable, span-extraction QA: \url{https://github.com/mrqa/MRQA-Shared-Task-2019}.}
We sample 60k examples from each dataset for training, 5k for validation, and 5k for testing. 
Questions and contexts are collected with varying procedures and sources, representing a wide diversity of datasets.

\section{Low-Voltage Load Forecasting Datasets} 
\label{secdatasets}

A number of interesting features were discovered about the data in the reviewed 221 papers. Firstly, only 52 use at least one openly available datasets to illustrate the results, i.e. less than 24\% of the journals presented results that could be potentially replicated by the wider research community. Of these 52 papers using open data, 22 (or $42\%$)  of them used the Irish CER Smart Metering Project data~\cite{Commission2012csm}, four used data from UK Low Carbon London project~\cite{UK2014ulc}, four from Ausgrid\footnote{\url{https://www.ausgrid.com.au/Industry/Our-Research/Data-to-share/Solar-home-electricity-data}} and three used the UMass dataset. In other words, out of the papers using open data, $56\%$, presented results that used data from only four open data sets. 

The overuse of a particular dataset can result in biases (both conscious and unconscious) where methods are developed and tested but the features of the data may be well known or familiar from overuse. In these cases a scientifically rigorous experiment is impossible. Further, reliance on a single dataset (especially those which are no more than 2 years in length like the Irish CER dataset~\cite{Commission2012csm}) risks the development of models which may be based on spurious features and patterns and may not be representative of the wider energy system. This can be alleviated somewhat by including multiple open data sets, as it was done in some of the papers reviewed (\cite{abera2020mla, Laurinec2019due, Wang2018aef}). To support the LV forecast research community the authors are going to share a modifiable list of open data sets at the LV level. A list and some of the properties of the data are shown in Table \ref{tab:datasets}.  

With the rapid change in low carbon technologies being connected to the grid and home, new energy efficiency interventions, and adjustments in demand usage behaviours, demand data can quickly become irrelevant or unrepresentative. Further, such data sets are based on trials where participants are subject to incentives or other interventions. For example, different tariffs were considered for some households in the Irish CER dataset~\cite{Commission2012csm}. This means that their demand may not represent `normal', every-day behaviour.  

There are now some initiatives that are attempting to solve some of the issues of sparse and intermittent open data produced by limited innovation projects. An example of this is the Smart Energy Research Lab from UCL in the  UK\footnote{\url{https://www.ucl.ac.uk/bartlett/energy/research/energy-and-buildings/smart-energy-research-group-serg}} which is attempting to make smart meter data (as well as other useful data sets associated with the same homes) available on an ongoing basis for research from a wider range of participants as well as a large control group of households which will not participate in any initiatives or trials. 

\subsection{Data resolution}
Resolution is another important aspect of the data. Half-hourly data is the standard resolution of smart meter data although it can be as low as 10 or 15 minutes. The smallest stated resolution of the data within the papers had 48 with hourly data, 51 half-hourly, 23 as 15 minutes and 8 as 10 minutes. There was also a large number of papers where the resolution was not clear or no real data was presented (62 papers). Data with resolutions of between ten minutes and an hour are probably sufficient for demand control applications and are likely representative of what data is available in practice. However, this isn't sufficient for more high-resolution applications such as voltage control. In fact, only 11 papers considered data of resolution of 1 minute or less. This could pose a difficulty for validating the common voltage and Var control application in this review (see Section~\ref{sec:LVLF-applications}). A large number of papers where the resolution is not clear should also be a concern, as this prevents recreation of the results. 

\subsection{Forecast horizon}
Another crucial aspect of this review is the forecast horizon. Different horizons are useful for different applications. Short term (day to the week ahead) are typical for operational time scales, whereas long-term forecasts (over a year) are more useful for planning. The majority of the papers reviewed were at a short term time scale with 80 of the papers considering day ahead forecasts, another common horizon was an hour ahead. Very few papers went at shorter horizons than an hour (twelve). There were slightly more papers that forecast beyond a day (16 were between 2 days and a week ahead) and only 13 papers were at horizons of a month or more. Once again there was a large number of papers, 80 in total, where the horizon length was not identifiable. 

\subsection{Overview of LV datasets}
As we have already mentioned, the current choice of open datasets that can be used for a benchmark is very limited, relying mostly on the CER Irish smart meter data, which is now a decade old and has some selection bias limitations (most of the houses have 3-4 bedrooms etc.). In order to continually expand research in this area, a strategy is required to regularly open more diverse datasets, converging towards common formats and standards and clarifying licences and terms of usage. Clear license information is especially relevant for industry-based research. We have established a list of open datasets (see Table~\ref{tab:datasets} and \url{https://low-voltage-loadforecasting.github.io/}) with the hope that it will continue to grow, and that new methods for privacy and safety protection (anonymisation, aggregation, synthetic 'look alike' datasets etc.) will enable more availability of datasets in the future. Finally it is vital to provide proper and thorough documentation with the data sets. The quality of the datasets is not clear and in many cases any preprocessing or data-cleaning techniques are not provided with the data.

Most datasets are at the residential level collected from smart meters. More diverse datasets of non-residential customers and different grid levels (substations and transformers) are needed for better LV forecasting research. 

Some datasets that have been cited in the literature like PLAID~\cite{Medico2020ava} OCTES, BLUED~\cite{Anderson2012baf},  and DRED~\cite{UttamaNambi2015lle} were offline at the time of writing. The Pecanstreet Dataport~\cite{Pecan2018d} database was once publicly available for research but then closed, and now only a subset is still accessible. Other datasets are available for download but are hard to trace, as identifiers like a DOI or a paper to cite are not available. All of these obstruct the reproducibility of the research. Therefore, new datasets should be published on archiving platforms like IEEE data port~\footnote{\url{https://ieee-dataport.org/}}, Zenodo~\footnote{\url{https://zenodo.org/}}, Figshare~\footnote{\url{https://figshare.com/}}, or  arXiv~\footnote{\url{https://arxiv.org/help/submit\#datasets}}. A recent contribution to more reproducible time series research is the Monash Time Series Forecasting Archive~\cite{godahewa2021mts}. It contains the dataset of the UK Low Carbon London trial~\cite{UK2014ulc} and the UCI datasets~\cite{candanedo2017ddp, Hebrail2012ihe}.


\begin{sidewaystable*}
	\tiny
	\caption{Overview of Low-voltage Load Datasets (see online version for embedded hyperlink to the data set by clicking on the name).} \label{tab:datasets}
	\resizebox{!}{0.9\height}{
		\begin{tabular}{p{0.16\linewidth}p{0.05\linewidth}p{0.04\linewidth}p{0.05\linewidth}p{0.04\linewidth}p{0.04\linewidth}p{0.02\linewidth}p{0.02\linewidth}p{0.12\linewidth}p{0.14\linewidth}p{0.14\linewidth}}
			\\
			\toprule
			Name & Type  & No. Customers & Resolution & Duration & Intervention & Sub-metering & Weather avail. & Location & Other data provided & Access/Licence \\
			\midrule 
			\href{https://www.ea.tuwien.ac.at//projects/adres_concept/EN/}{ADRES}~\cite{einfalt2011kfa}
			& Households & 30    & 1 s   & 2 weeks & None  & No    & No    & Austria (Upper Austria) & Voltage & Free for Research (E-Mail) \\
			\href{https://www.ausgrid.com.au/Industry/Our-Research/Data-to-share/Solar-home-electricity-data}{Ausgrid Solar Home} &  Households &  300 & 30 min &   3 years & None &  No &  No &  Australia (NSW) &  &   No Licence \\
			\href{https://www.ausgrid.com.au/Industry/Our-Research/Data-to-share/Distribution-zone-substation-data}{Ausgrid substation data} &   Substation & 225 & 15 min &  20 years & None &  No &  No &  Australia (NSW) &  & No Licence \\
			\href{https://sourceforge.net/projects/greend/}{GREEND Electrical Energy Dataset (GREEND)}~\cite{monacchi2014gae} & Households & 8     & 1 s   & 3-6 months & None  & Yes   & No    & Austria, Italy & Occupancy, Building type & Free (Access Form) \\
			\href{https://archive.ics.uci.edu/ml/datasets/Appliances+energy+prediction}{UCI Appliances}~\cite{candanedo2017ddp} & Households & 1     & 10 min & 4.5 months & None  & No    & Yes   & Belgium (Mons) & Lights, Building information & Free (No Licence) \\
			\href{https://ieee-dataport.org/open-access/industrial-machines-dataset-electrical-load-disaggregation}{INDUSTRIAL MACHINES}\cite{Bandeira2018imd}
			& Industrial & 1 & 1 Hz  & 1 month & None  & Yes   & No    & Brasil (Minas Gerais) &       & CC BY \\
			\href{https://dataverse.harvard.edu/dataset.xhtml?persistentId=doi:10.7910/DVN/ZJW4LC}{Rainforest Automation Energy} \cite{Makonin2018rtr} & Households & 2     & 1 Hz  & 2 months & None  & Yes   & Yes   & Canada & Environmental, Heat Pump,  & CC BY \\
			\href{https://dataverse.harvard.edu/dataset.xhtml?persistentId=doi\%3A10.7910/DVN/FIE0S4\%20}{AMPds2} \cite{Makonin2016ata, Makonin2016ewa}  & Households & 1     & 1min  & 2 years & None  & Yes   & Yes   & Canada (Alberta) & Gas, Water, Building Type and Plan & CC BY \\
			\href{https://carleton.ca/sbes/publications/electric-demand-profiles-downloadable/}{Sustainable Building Energy Systems 2017} \cite{Johnson2017edf} & Households & 23    & 1 min & 1 year & None  & Yes   & No    & Canada (Ottawa) & Sociodemographic (Occupants, Age, Size) & Free (Attribution, E-Mail) \\
			\href{https://carleton.ca/sbes/publications/electric-demand-profiles-downloadable/}{Sustainable Building Energy Systems 2013} \cite{Saldanha2012mee} & Households & 12    & 1 min & 1 year & None  & Yes   & No    & Canada (Ottawa) & Sociodemographic (Occupants, Age, Size) & Free (Attribution, E-Mail) \\
			\href{https://data.lab.fiware.org/organization/9569f9bd-42bd-414f-b8d9-112553ea9dfb?tags=FINESCE}{FINESCE Horsens}
			& Households & 20    & 1 h   & several days & None  & Yes   & Yes   & Denmark (Horsens) & EV, PV, Heat Pump, Heating, Smart Home,  & CC BY-SA \\
			
			\href{https://archive.ics.uci.edu/ml/datasets/Individual+household+electric+power+consumption}{UCI Individual household electric power cons.}~\cite{Hebrail2012ihe} & Households & 1     & 1min  & 4 years & None  & Yes   & No    & France (Sceaux) & Reactive Power, Voltage & CC BY \\
			\href{https://mediatum.ub.tum.de/1375836}{BLOND-50}~\cite{Kriechbaumer2017bbo} & Commerical & 1     & 50 kHz & 213 days & None  & Yes   & No    & Germany &       & CC BY \\
			\href{https://mediatum.ub.tum.de/1375836}{BLOND-250}~\cite{Kriechbaumer2017bbo} & Commerical & 1     & 250 kHz & 50 days & None  & Yes   & No    & Germany &       & CC BY \\
			\href{https://zenodo.org/record/3855575\#.YKQgGKgzaUk}{Fresh Energy}~\cite{Beyertt2020fzb} & Households & 200   & 15 min & 1 year & Behaviorial & Yes   & No    & Germany & Agegroup, Gender of main customer & CC BY \\
			\href{https://data.lab.fiware.org/organization/9569f9bd-42bd-414f-b8d9-112553ea9dfb?tags=FINESCE}{FINESCE Factory} 
			& Industrial & 1     & 1 min & 2 days & None  & Yes   & No    & Germany (Aachen) & Machines & CC BY-SA \\
			\href{https://pvspeicher.htw-berlin.de/wp-content/uploads/MFH-Lastprofil_2014_17274_kWh.csv}{HTW Lichte Weiten}~\cite{htw2019ldb}
			& Households & 1 building & 15 minute & 1 year & None  & No    & No    & Germany (Berlin) &       & Free (No Licence) \\
			\href{https://pvspeicher.htw-berlin.de/veroeffentlichungen/daten/lastprofile/}{HTW Synthetic}~\cite{Tjaden2015rel} & Households & 74    & 1 s   & 1 year & None  & No    & No    & Germany (Representative) & Synthetic dataset merging & CC BY-NC \\
			\href{https://data.open-power-system-data.org/household_data/}{CoSSMic} \cite{Open2020dph} & Households, SME & 11    & 1min, 15min, 1H & 1-3 years & None  & Yes   & No    & Germany (South) & PV, EV, Type (Residential/SME) & CC BY \\
			\href{https://im.iism.kit.edu/sciber.php}{SciBER}~\cite{Staudt2018san} & Municipal & 107   & 15min & 3 years & None  & No    & No    & Germany (South) & Type (Office, Gym, ...) & CC BY \\
			\href{https://iawe.github.io/}{iAWE}~\cite{batra2013idi}
			& Households & 1     & 1 Hz  & 2 months & None  & Yes   & No    & India (New Delhi) & Water & Free (No Licence) \\
			\href{https://combed.github.io/}{COMBED}~\cite{Batra2014aco} & Commerical & 1     & 30 s  & 1 month & None  & Yes   & No    & India (New Delhi) &       & Free (No Licence) \\
			\href{http://www.ucd.ie/issda/data/commissionforenergyregulationcer/}{Irish CER Smart Metering Project data}~\cite{Commission2012csm} & Households, SME, Other & 3835  & 30min & 1.5 years & Tariff & No    & No    & Ireland & Type (Residential/SME/Other) & Free (Signed Access Form) \\
			
			\href{https://github.com/Nikasa1889/ShortTermLoadForecasting}{Hvaler Substation Level data}~\cite{DangHa2017lst} & Substation & 20    & 1 h   & 2 years & None  & No    & No    & Norway (Hvaler) &       & Free (No Licence) \\
			\href{http://web.lums.edu.pk/~eig/CXyzsMgyXGpW1sBo}{Energy Informatics Group Pakistan}~\cite{Pereira2014sap} & Households & 42    & 1 min & 1 year & None  & Yes   & No    & Pakistan & Sociodemographic (building properties, no of people, devices) & Free (No Licence) \\
			\href{https://archive.ics.uci.edu/ml/datasets/ElectricityLoadDiagrams20112014}{UCI Electricity Load Diagrams}~\cite{Godahewa2021ehd}
			& Different & 370   & 15 min & 2 years & None  & No    & No    & Portugal &       & Free (No Licence) \\
			
			\href{http://www.vs.inf.ethz.ch/res/show.html?what=eco-data}{Electricity Consumption and Occupancy (ECO)}~\cite{Christian2014ted, Wilhelm2015hom} & Households & 6     & 1 Hz  & 8 months & None  & Yes   & No    & Switzerland & Occupancy & CC BY \\
			\href{https://www.gov.uk/government/publications/household-electricity-survey--2}{Household Electricity Survey (HES)}~\cite{Zimmermann2012hes} & Households & ~{}250 & 2 min & 1 month (255) to 1 year (26) & None  & Yes   & No    & UK    & Consumer Archetype & Request \\
			\href{https://beta.ukdataservice.ac.uk/datacatalogue/studies/study?id=8634}{METER}~\cite{Grunewald2019muh} & Households & 529   & 1 min & 28 hours & None  & No    & No    & UK    & Activity data, Sociodemographic & Free for Research (Access Form) \\
			\href{https://datashare.ed.ac.uk/handle/10283/3647}{IDEAL Household Energy Dataset}~\cite{goddard2020ihe}
			& Households & 255   & 1 s     & 3 years & None  & Yes   & No    & UK    & Smart Home, Sociodemographic, energy awareness survey, room temperature and humidity, building characteristics & CC BY \\
			\href{http://www.networkrevolution.co.uk/resources/project-data/}{Customer-Led Network Revolution project data}~\cite{sidebotham2015cln} & Households, SMEs & ~{}12000 & 30 min & > 1 year & Time of Use & No    & No    & UK    & EV, PV, Heatpump, Tariff,  & CC BY-SA \\
			\href{https://jack-kelly.com/data/}{UK Domestic Appliance-Level Electricity (UK-DALE)}~\cite{Jack2015tud} & Households & 5     & 16 kHz, 1s & months, one house > 4 years & None  & Yes   & No    & UK (London area) &       & CC BY \\
			\href{https://data.london.gov.uk/dataset/smartmeter-energy-use-data-in-london-households}{UK Low Carbon London}~\cite{UK2014ulc,Godahewa2021lsm} & Households & 5567  & 30min & 2 years & Time of Use & No    &  No   & UK (London) & CACI Acorn group & Free (No Licence) \\
			\href{https://www.refitsmarthomes.org/datasets/}{REFIT}~\cite{Murray2016rel, Murray2017ael} & Households & 20    & 8 s   & 2 years & None  & Yes   & Yes   & UK (Loughborough) & PV, Gas, Water, Sociodemographic (Occupancy, Dwelling Age, Dwelling Type, No. Bedrooms) & CC BY \\
			\href{https://www.spenergynetworks.co.uk/pages/flexible_network_data_share.aspx}{Flexible Networks for a Low Carbon Future} & Substations & Several Secondary  & 30 min & 1 year & None  & No    & No    & UK (St Andrews, Whitchurch, Ruabon) &       & Free (Access Form) \\
			\href{https://ukerc.rl.ac.uk/DC/cgi-bin/edc_search.pl?GoButton=Detail\&WantComp=146\&\&RELATED=1}{NTVV Substations} & Substation & 316   & 5 s   & > 4 years & None  & No    & No    & UK (Thames Valley) &       & Open Access (Any purpose) \\
			\href{https://ukerc.rl.ac.uk/DC/cgi-bin/edc_search.pl?GoButton=Detail\&WantComp=147\&\&RELATED=1}{NTVV Smart Meter} & Buildings & 316   & 30 min & > 4 years & None  & No    & No    & UK (Thames Valley) &       & Open Access (Any purpose) \\
			
			\href{https://site.ieee.org/pes-iss/data-sets/}{IEEE PES Open Data Sets} 
			& Households, Commercial & 15    & 1 min, 5 min, 15 min & 2 weeks & None  & No    & No    & USA   & Connection limit & Free (No Licence) \\
			\href{http://redd.csail.mit.edu/}{Reference Energy Disaggregation Data Set (REDD)}~\cite{Kolter2011rap} & Households &  ~{}10 & 1 kHz & 3-19 days & None  & Yes   & No    & USA (Boston) & Voltage & Free (Attribution, E-Mail) \\
			\href{http://wzy.ece.iastate.edu/Testsystem.html}{Iowa Distribution Test Systems}~\cite{Bu2019atd} & Substation & 240 nodes & 1 H   & 1 year & None  & Yes   & No    & USA (Iowa) & Grid data & Free (Attribution) \\
			\href{https://www.pecanstreet.org/dataport/}{Pecanstreet Dataport (Academic)}~\cite{Pecan2018d} & Households & 30    & 1min, 15min, 1H & 2-3 years & None  & Yes   & Yes   & USA (mostly Austin and Boulder) & PV, EV, Water, Gas, Sociodemographic & Free for Research (Access Form) \\
			
			\href{https://neea.org/resources/rbsa-ii-combined-database}{Residential Building Stock Assessment}~\cite{Larson2014jua} & Households & 101   & 15 min & 27 months & None  & Yes   & No    & USA (North West Region) & Building Type (Single Family, Manufactured, Multifamily) & Free (Access Form) \\
			
			\href{http://lass.cs.umass.edu/projects/smart/}{SMART* Home 2017}~\cite{Barker2012sao} & Households & 7     & 1 Hz  & > 2 years & None  & Yes   & Yes   & USA (Western Massachussets) &       & Free (No Licence) \\
			\href{http://lass.cs.umass.edu/projects/smart/}{SMART* Apartment}~\cite{Barker2012sao} & Households & 114   & 1 min & 2 years & None  & No    & Yes   & USA (Western Massachussets) &       & Free (No Licence) \\
			\href{http://lass.cs.umass.edu/projects/smart/}{SMART* Occupancy}~\cite{Barker2012sao} & Households & 2     & 1 min & 3 weeks & None  & No    & No    & USA (Western Massachussets) & Occupancy & Free (No Licence) \\
			\href{http://lass.cs.umass.edu/projects/smart/}{SMART* Microgrid}~\cite{Barker2012sao} & Households & 443   & 1 min & 1 day & None  & No    & No    & USA (Western Massachussets) &       & Free (No Licence) \\
			\href{http://lass.cs.umass.edu/projects/smart/}{SMART* Home 2013}~\cite{Barker2012sao} & Households & 3     & 1 Hz  & 3 months & None  & Yes   & No    & USA (Western Massachussets) & Solar, Wind, Environmental, Smart Home, Voltage,  & Free (No Licence) \\
			\bottomrule
		\end{tabular}
	}
\end{sidewaystable*}


\paragraph{Sentiment Analysis}
For the sentiment analysis classification task, we follow \citep{blitzer-etal-2007-biographies} and \citep{ruder2018strong} by randomly selecting 6 Amazon multi-domain review datasets, as well as Yelp reviews \citep{asghar2016yelp} and IMDB movie reviews datasets \citep{maas-EtAl:2011:ACL-HLT2011}.~\footnote{\url{https://jmcauley.ucsd.edu/data/amazon/}, \url{https://www.yelp.com/dataset}, \url{https://ai.stanford.edu/~amaas/data/sentiment/}.}
Altogether, these datasets exhibit wide diversity based on review length and topic (see Table~\ref{datasets}).
We normalize all datasets to have 5 sentiment classes: very negative, negative, neutral, positive, and very positive. 
We sample 50k examples for training, 5k for validation, and 5k for testing.

\respace
\paragraph{Experimental Setup}
\label{sec:ex-setup}
To evaluate methods for the multi-domain active learning task, we conduct the experiment described in Section~\ref{sec:task} for each acquisition method, rotating each domain as the target set.
Model $M$, a BERT-Base model ~\citep{devlin2019bert}, is chosen via hyperparameter grid search over learning rate, number of epochs, and gradient accumulation.
The large volume of experiments entailed by this search space limits our capacity to benchmark performance variability due to isolated factors (the acquisition method, the target domain, or fine-tuning final models).
However, our hyper-parameter search closely mimics the process of an ML practitioner looking to select a best method and model, so we believe our experiment design captures a fair comparison among methods.
See Algorithm~\ref{alg:experiment} in Appendix Section \ref{sec:appendix-expdesign} for full details.



