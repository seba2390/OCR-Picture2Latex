
In this section we introduce our semantic and memory model. Our goal is to
formalize a general purpose SMC compiler, showing correctness with respect to standard C semantics
and non-interference to guarantee no leakage of private data. 
We therefore introduce two models, C (referred to as \vanillaC) as well as the semantics for the SMC compiler (referred to as \piccoC). 
We do not abstract away memory, instead we introduce a byte-level memory model, 
inspired by the memory model used by CompCert~\cite{leroy2012compcert}, a formally verified C compiler. 
Specifically, we build from their approach of byte-level representation of data and permissions. 

\begin{figure}
\begin{minipage}{0.48\textwidth}\footnotesize
$\begin{array}{l c l}
	\Type &::=& \RT{\llabel\ \btype} \mid \RT{\llabel\ \btype*} \mid \btype \mid {\btype*} \mid \Tlist \to \Type 
\\	\btype &::=& \Int \mid \Float \mid \Void 
\\	\llabel &::=& \rPriv \mid \rPub
\\	\Tlist &::=&  {[\ ]} \mid {\Type::\Tlist} 
\\ \\	\stmt &::=& {\var = \Expr} \mid {*\x = \Expr} \mid {\stmt_1; \stmt_2} \mid {\If (\Expr)\ \stmt_1\ \Else\ \stmt_2}

\\	  && \mid {\Type\ \x (\plist)\ \{ \stmt\}} \mid {\While (\Expr)\ \stmt} 
 \mid {\{ \stmt \}} \mid \decl \mid \Expr
\\	\Expr &::=& \Expr\ \binop\ \Expr \mid {\preop\ \x} \mid \var
		\mid {\x(\Elist)} \mid \builtin 
	\\	  && \mid {(\Type)\ \Expr}  \mid {( \Expr )}  \mid \val
\\	\decl &::=&  \Type\ \var \mid {\Type\ \x ( \plist )}
\\	\var &::=& \x \mid {\x[\Expr]} 
\\	\val &::=& n \mid (\loc, \offset) \mid [{\val_0},\ {...},\ {\val_n}] \mid \Null \mid \Skip
\\ \\ 	\builtin &::=& {\Malloc(\Expr)} \mid \RT{\PMalloc(\Expr,\ \Type)} \mid {\sizeof(\Type)}
	\\	  && \mid {\free(\Expr)} \mid \RT{\pfree(\Expr)} 
	\\ && \mid \RT{\smcinput(\var, e)} \mid \RT{\smcoutput(\var, e)}
\\ 	\binop &::=&{-} \mid {+} \mid \div \mid \cdot \mid {==}  \mid {!=} \mid {<}
\\	\preop &::=& \& \mid {*} \mid ++
\\	\Elist &::=& {\Elist,\ \Expr} \mid \Expr \mid \Void
\\	\plist &::=&  \plist,\ \Type\ \var \mid {\Type\ \var} \mid \Void
\end{array}$
\captionof{figure}{Combined \vanillaC /\piccoC\ Grammar. The color \red{red} denotes terms specific to programs written in \piccoC. 
} 	
\label{Fig: \piccoC grammar}
\end{minipage}
\qquad
\begin{tabular}{l}
\begin{minipage}{0.44\textwidth}\footnotesize
$\begin{array}{r l}	
% Configuration
\Config ::= & \epsilon \mid (\pid, \gamma, \sigma, \DMap, \Acc, \stmt) \Mid \Config \\
\\
% Environment
	\gamma ::=& [ \-\ ]\ \mid\ \gamma[\x\ \to\ (\loc,\ \Type)]							\\
% Memory
	\sigma ::=& [ \-\ ]\ \mid\ \sigma [\loc\ \to\ (\byte,\ \Type,\ n,\ \PermL)  	\\
% Permissions 
	\PermL ::=& [ \-\ ]\ \mid\ [(0,\ \llabel_0,\ \perm_0), ..., (\bytelen, \llabel_\bytelen, \perm_\bytelen)] \\
	\perm ::=& \PermF \mid \PermN		 									\\
% Permission Tuples - "Size"
	\bytelen ::= & \tau(\Type)\cdot n - 1		\\
% Change Map & locals map
	\DMap ::= & [ \-\ ] \mid \dmap::\DMap \\
	\dmap ::= & [ \-\ ] \mid ((\loc, \offset)\to(\val_1, \val_2, \tagb, \Type))::\dmap \\
\\ 
% Location List
	\locLL ::= & \epsilon \mid (\pid, \locL) \Mid \locLL \\
	\locL ::= & [ \-\ ] \mid (\loc, \offset)::\locL \\
	\codeLL ::= & \epsilon \mid (\pid, \codeL) \Mid \codeLL \\
	\codeL ::= & [ \-\ ] \mid \code::\codeL 
	\end{array}
$
\captionof{figure}{Configuration: party identifier $\pid$, environment $\gamma$, memory $\sigma$, \changeMap\ $\DMap$, accumulator $\Acc$, and statement $\stmt$.}
\label{Fig: mem model}
\end{minipage}
\end{tabular}
\end{figure}


Figure~\ref{Fig: \piccoC grammar} gives the combined \vanillaC\ and \piccoC\ grammar, which is a subset of the ANSI C grammar. 
We include one dimensional arrays, branches, loops, dynamic memory allocation, and pointers. 
Arrays are zero-indexed, and it is possible to overshoot their bounds. 
We chose not to include structs or multi-dimensional arrays, as they are an extension of this core subset.
The interested reader can find the full semantics within the scope of the grammar given in Appendix~\ref{app: semantics}.
We use the color \red{red} to denote terms in the \piccoC\ grammar that are not present in \vanillaC, including annotated types ($\llabel\ \btype$, $\llabel\ \btype *$), privacy labels ($\Pub$, $\Priv$), and primitive functions ($\PMalloc$, $\pfree$) for allocation and deallocation of 
memory for private pointers.

We denote types as $\Type$, basic types as $\btype$ ($\btype *$ as a pointer type), privacy annotations as $\llabel$, and function types as $\Tlist \to \Type$ 
(where $\Tlist$ is a type list, $[\ ]$ an empty list, and $::$ list concatenation). 
Values $\val$ include numbers $n$, locations ($\loc$, $\offset$) consisting of a memory block identifier and an offset, lists of values, $\Null$, and $\Skip$ (to show a statement being reduced to completion).
Declarations include variable and function declarations, where $\plist$ is the function parameter list.
For unary operations, we include: $\&$, to obtain the address of a variable; $*$, to allow dereferencing pointers; and $++$, to allow pre-incrementing and to model a basic pointer arithmetic. 


%%%%%%%%%%%%%%%%%%%%%%%
% 
%      	 Memory model input
% 
%%%%%%%%%%%%%%%%%%%%%%%

\subsection{Memory Model} \label{subsec: mem model}
We assume the existance $n$ communicating parties, each with a separate memory.
Our memory model encodes each memory as a contiguous region of {\em blocks}, which are sequences of bytes and metadata.
We introduce an execution environment $\gamma$ and memory $\sigma$, shown in Figure \ref{Fig: mem model}.
Each block is assigned an identifier $\loc$ (to be discussed more later in this section). 
Blocks are never recycled nor cleared when they are freed. We chose this view of memory to preserve all allocated data, which, in conjunction with data-oblivious execution, represents the worst case for maintaining privacy. 
Direct memory access through pointers or manipulation of array indices allows programs to access any block for which the memory address is computable (e.g., as an offset or direct pointer access). 
To obtain the byte representations of data we leverage functions similar to CompCert, using $\Encode$ for values, $\EncodePtr$ for pointers, and $\EncodeFun$ for functions. Likewise, to obtain the human-readable data back, we use respective decode functions such as $\Decode$. We use a specialized version ($\DecodeArr$) for obtaining a specific index within an array data block.
We introduce the particulars in the following subsections.


% Fig: mem model moved to semantics.tex to be placed next to the grammar

\subsubsection{Environment} 
The environment, $\gamma$, maintains a mapping of each live variable $\x$ to its memory block identifier (where the data $\x$ is stored) and its type. 
At the start of a program, the environment is empty, i.e., $\gamma = [\-\ ]$. 
Variables that are no longer live are removed from the environment, based on scoping.  
We use the environment to facilitate the lookup of variables (i.e., for reads, writes, function calls) in memory $\sigma$. 



\subsubsection{Memory Blocks and Identifiers} \label{subsec: mem block}
The memory, $\sigma$ (shown in Figure~\ref{Fig: mem model}), is a mapping of each identifier $\loc$ to its memory block, which contains 
the byte representation $\byte$ of data stored there and metadata about the block.
Metadata consists of  
a type $\Type$ associated with the block, 
the number of elements $n$ of that type stored in $\byte$, 
and a list of byte-wise permissions tuples $[(0,\ \llabel_0,\ \perm_0), ..., (\bytelen,\ \llabel_\bytelen,\ \perm_\bytelen)]$, where $\bytelen = \tau(\Type)\cdot n - 1$ and function $\tau$ provides the size of the given type in bytes. 
A new memory block identifier is obtained from function $\phi$. 
These identifiers are monotonically increasing with each allocation. 
Every block is added to memory $\sigma$ on allocation, and is never cleared of data nor removed from $\sigma$ upon deallocation. 
Metadata cannot be accessed or modified directly by the program (the semantic rules control modification).
A memory block can be of an arbitrary size, which is constant and determined at allocation (with the exception of private pointers, to be discussed later in subsection~\ref{subsec: picco ptr eval}). 
We represent a memory location as a two-tuple of a memory block identifier and an offset. This allows us to use pointers to refer to any arbitrary memory location, as in C.





\subsubsection{Permissions} 
\label{subsec: permissions}
A permission $\perm$ can either be $\PermF$ (i.e., can be written to, read from, etc.) or 
$\PermN$ (i.e., already freed). 
These byte-wise permissions are modeled after a subset of those used by CompCert, and we extend their permission model by including a privacy label.
Each memory block has a list of permission tuples, one for each byte of data 
stored in that block. 
A permission tuple consists of the position of the byte that it corresponds to, and the privacy label $\llabel$ and permission $\perm$ for that byte of data. 
These permissions are important in reading and writing data to memory, especially when it comes to overshooting arrays and other out-of-bound memory accesses possible through the use of pointers. 
In particular, permissions allow us to keep track of deallocated memory (e.g. a block freed - note that the memory stored in the block itself is not overwritten or cleared and the block can still be accessed indirectly through direct
memory manipulations). 
All permission tuples corresponding to a memory block of a function type will have public privacy labels, as the instructions for a function are accessible from the program itself. 
Those for a normal variable or an array will have privacy labels corresponding to their type (i.e., public for public types, private for private types); those for pointers are more complex and will be discussed later in subsection~\ref{subsec: picco ptr eval}.  




\subsubsection{Malloc and Free}
Allocation of dynamic memory in C is provided by \TT{malloc}, which takes a number of bytes as its argument. 
When \Code{malloc} is called, 
a new memory block with identifier $\loc$ is obtained, initialized as a void type of the given size, and returned. 
This block then needs to be cast to the desired type. 
However, when dealing with private data, the programmer is unlikely to know the internal representation and the size of the private data types. 
For that reason, when allocating memory for private data, we adopt PICCO's \TT{pmalloc} functionality which takes two arguments: the type and the number of elements of that type to be allocated. 
The semantics then handle sizing the new memory block for the given private type. 
When \Code{free} or \Code{pfree} is called, if the argument is a variable of a pointer type, the permissions for all bytes of the location the pointer refers to will be set to $\PermN$, but the data stored there will not be erased. 
When \Code{pfree} is called with a pointer with a single location, it behaves identically to \Code{free}. 
The use of \TT{pfree} with multiple locations is a bit more involved, to be discussed later in Subsection~\ref{subsec: picco mem alloc/dealloc}. 
It is important to note that memory allocation and deallocation are public side effects, and therefore are {\em not allowed} within private-conditioned branches. 



\subsubsection{Public vs. Private Blocks}
To distinguish between public and private blocks, we assume that private blocks will be encrypted 
and we will use basic private primitives implementing specific operations to manage them. 
For modeling purposes, 
these primitives can decrypt the required blocks, perform 
the operations they are meant to implement, and encrypt the result. 
In our model, a program can also access private blocks by means of standard non-private operations or through of pointers. In this case, the operation will just interpret the encrypted value as a public value. This approach gives us a conceptual distinction between a \emph{concrete memory} and its corresponding \emph{logical content}, i.e. public values and values of the private data prior to encryption. Our model as described in the next section will work on concrete memories, but in Section~\ref{sec: noninterference}, for the proof of noninterference, it will be convenient to refer to the logical content of a memory. We will use the notation $\sigma\ell$ to denote the logical content of $\sigma$.  




%%%%%%%%%%%%%%%%%%%%%%%
%
%   Begin Vanilla C semantics description 
%
%%%%%%%%%%%%%%%%%%%%%%%


\begin{figure*}
\footnotesize
\begin{tabular}{l}
% BASIC  EVAL
Multiparty Binary Operation  \\
  	\inferrule{\begin{array}{l l}
		\begin{array}{l}
		\qquad((\pidA, \hgamma, \hsigma,\ \bsq, \bsq, \hExpr_1)\ \Mid ...\Mid
		(\pidZ, \hgamma, \hsigma,\ \bsq, \bsq, \hExpr_1))
			\crcr\Veval_{\codeVLL_1} 
			((\pidA, \hgamma, \hsigma_1, \bsq, \bsq, \hn_1) \Mid ...\Mid
			(\pidZ, \hgamma, \hsigma_1, \bsq, \bsq, \hn_1))
		\crcr \qquad ((\pidA, \hgamma, \hsigma_1, \bsq, \bsq, \hExpr_2)\ \Mid ...\Mid
			(\pidZ, \hgamma, \hsigma_1, \bsq, \bsq, \hExpr_2))
			\crcr\Veval_{\codeVLL_2} 
			((\pidA, \hgamma, \hsigma_2, \bsq, \bsq, \hn_2)\Mid ...\Mid
			(\pidZ, \hgamma, \hsigma_2, \bsq, \bsq, \hn_2))
		\end{array}
	& \begin{array}{l}
		 \hn_1 \binop\ \hn_2 = \hn_3
		\crcr \binop\in\{\cdot, +, -, \div\}
	\end{array}\end{array}}
	{\begin{array}{l}
	((\pidA, \hgamma, \hsigma,\ \bsq, \bsq, \hExpr_1 \binop\ \hExpr_2)\Mid ...\Mid
	(\pidZ, \hgamma, \hsigma,\ \bsq, \bsq, \hExpr_1 \binop\ \hExpr_2)) 
		\Veval_{\codeVLL_1\addC\codeVLL_2\addC\codeVM{mpb}} 
		\crcr((\pidA, \hgamma, \hsigma_2, \bsq, \bsq, \hn_3) 
			\qquad\ \Mid ...\Mid 
		(\pidZ, \hgamma, \hsigma_2, \bsq, \bsq, \hn_3))
		\end{array}}
\\ \\
 Write \\
\inferrule{\begin{array}{l}
		((\pid, \hgamma, \hsigma, \bsq, \bsq, \hExpr)\Mid \hConfig)  \Veval_{\codeVLL} 
			((\pid, \hgamma, \hsigma_1, \bsq, \bsq, \hn)\Mid \hConfig_1)
		\crcr \hgamma(\hx) = (\hloc, \hbtype)  
		\qq \Update(\hsigma_1, \hloc, \hn, \hbtype) = \hsigma_2
	\end{array}}
	{((\pid, \hgamma, \hsigma, \bsq, \bsq, \hx = \hExpr) \Mid \hConfig)
		\Veval_{\codeVLL\addC[\codeVS{w}]} 
		((\pid, \hgamma, \hsigma_2, \bsq, \bsq, \Skip)\Mid \hConfig_1)}
\\ \\
Malloc 	\\ 
\inferrule{\begin{array}{l}
		((\pid, \hgamma, \hsigma, \bsq, \bsq, \hExpr) \Mid \hConfig) \Veval_{\codeVLL} 
			((\pid, \hgamma, \hsigma_1, \bsq, \bsq, \hat{n})\Mid \hConfig_1)
		\qq \hloc = \phi()
		\crcr \hsigma_2 = \hsigma_1 \big[\hloc \to \big(\Null, \Void*, \hat{n}, 
			\PermL(\PermF, \Void*, \Pub, \hn)\big) \big] 
	\end{array} }
	{((\pid, \hgamma, \hsigma, \bsq, \bsq, \Malloc (\hExpr))\Mid \hConfig) 
		\Veval_{\codeVLL\addC[\codeVS{mal}]} 
		((\pid, \hgamma, \hsigma_2, \bsq, \bsq, (\hloc, 0))\Mid \hConfig_1)}
\\ \\
Multiparty Free \\
	\inferrule{\begin{array}{l}
		\hgamma(\hx) = (\hloc, \hbtype*) 
		\qq \hsigma(\hloc) = (\hbyte, {\hbtype*}, 1, \PtrPermL(\PermF, \hbtype*, \Pub, 1))
		\crcr \DecodePtr({ \hbtype*}, 1, \hbyte) = [1, [({\hloc_1}, 0)], [1], 1]
		\crcr \SelectFreeable(\hgamma, [({\hloc_1}, 0)], [1], \hsigma) = 1
		\qq \Free(\hsigma, \hloc_1) = \hsigma_1 
	\end{array}}
	{\begin{array}{l}
	((\pidA, \hgamma, \hsigma, \bsq, \bsq, \free (\x)) \Mid ... \Mid 
	(\pidZ, \hgamma, \hsigma, \bsq, \bsq, \free (\x)))
		\Veval_{\codeVM{fre}} \crcr
		((\pidA, \hgamma, \hsigma_1, \bsq, \bsq, \Skip)\Mid ... \Mid 
		(\pidZ, \hgamma, \hsigma_1, \bsq, \bsq, \Skip))
		\end{array}}
\\ \\
% ALLOC / DEALLOC
Pointer Declaration 	\\
\inferrule{\begin{array}{l l}
	\begin{array}{l}
		(\hType = \hbtype *)
		\crcr \hloc = \phi() 
		\crcr \hgamma_1 = \hgamma[\hx \to (\hloc, \hType)]
		\end{array}
		&\begin{array}{l}
		\getIndirection(*) = \hindir \crcr
		\hbyte = \EncodePtr(\hType*, [1, [(\hlocDefault, 0)], [1], \hindir])
		\crcr \hsigma_1 = \hsigma[\hloc \to (\hbyte, \hType, 0, \PtrPermL(\PermF, \hType, \Pub, 0))]
	\end{array}\end{array}}					
	{((\pid, \hgamma, \hsigma, \bsq, \bsq, \hType \hx) \Mid \hConfig)
		\Veval_{[\codeVS{dp}]} 
		((\pid, \hgamma_1, \hsigma_1, \bsq, \bsq, \Skip)\Mid \hConfig)}
\\ \\ 
% PTR  EVAL
% ARR  EVAL
Multiparty Array Read \\ %\\ 
\inferrule{\begin{array}{l}
		\hgamma(\hx) = (\hloc, \Const\ \hbtype*) \crcr
		((\pidA, \hgamma, \hsigma, \bsq, \bsq, \hExpr) \Mid ... \Mid 
		(\pidZ, \hgamma, \hsigma, \bsq, \bsq, \hExpr)) \Veval_{\codeVLL} 
			((\pidA, \hgamma, \hsigma_1, \bsq, \bsq, \hind)\Mid ... \Mid 
			(\pidZ, \hgamma, \hsigma_1, \bsq, \bsq, \hind)) 
		\crcr \hsigma_1(\hloc) = (\hbyte, \Const\ \hbtype*, 1, \PtrPermL(\PermF, \Const\ \hbtype*, \Pub, 1)) 
		\crcr \DecodePtr(\Const\ \hbtype*, 1, \hbyte) = [1, [(\hloc_1, 0)], [1], 1] 
		\crcr \hsigma_1(\hloc_1) = (\hbyte_1, \hbtype, \hnl, \ArrPermL(\PermF, \hbtype, \Pub, \hnl))  
		\qq 0 \leq \hind \leq \hnl - 1 
		\crcr \DecodeArr({\hbtype},\ \hind,\ {\hbyte_1}) = \hn_{\hind} 
	\end{array}}
	{\begin{array}{l}
	((\pidA, \hgamma, \hsigma,\ \bsq, \bsq, \hx[\hExpr]) \Mid ... \Mid 
	(\pidZ, \hgamma, \hsigma,\ \bsq, \bsq, \hx[\hExpr]))
		\Veval_{\codeVLL\addC\codeVM{mpra}} \crcr
		((\pidA, \hgamma, \hsigma_1, \bsq, \bsq, \hn_{\hind})\ \ \Mid ... \Mid 
		(\pidZ, \hgamma, \hsigma_1, \bsq, \bsq, \hn_{\hind}))
		\end{array}}
\end{tabular}
\caption{\vanillaC\ semantic rules.}
\label{Fig: \vanillaC sem rules}
\end{figure*}















\subsection{\vanillaC\ Semantics} \label{van C descr}


%%%%%%%%%%%%%%%%%%%%%%%
% 
%       Codes For Evaluations
% 
%%%%%%%%%%%%%%%%%%%%%%%
In order to facilitate the correspondence between the \vanillaC\ and \piccoC\ semantics, we model our semantics using big-step evaluation judgements and define our C semantics with respect
to multiple {\em non interacting} parties that evaluate the same program. 
These judgements are defined over a six-tuple configuration $\Config = ((\pid, \gamma, \sigma, \DMap, \Acc, \stmt) \Mid \Config_1)$, where each rule is a reduction from one configuration to a subsequent. 
In the semantics, we denote the party identifier $\pid$;
the environment as $\gamma$; 
memory as $\sigma$;
a mapping structure for location-based tracking of changes at each level of nesting of private-conditioned branches $\DMap$;
the level of nesting of private-conditioned branches as $\Acc$; and
a big-step evaluation of a statement $\stmt$ to a value $\val$ using $\eval$.  
We use a $\hat{\ }$ and $\Veval$ to distinguish the \vanillaC\ semantics from those we use in the next section for \piccoC, as well as $\bsq$ in \vanillaC\ as a placeholder for $\Acc$ and $\DMap$ to maintain the same shape of configurations as that of \piccoC\ used in the next section. 
We annotate each evaluation with party-wise lists of the evaluation codes of all rules that were used during the execution of the rule (i.e., $\Veval_{\codeVLL}$) to facilitate reasoning over evaluation trees. 
We extend the list concatenation operator $::$ to also work over party-wise lists such as $\codeVLL$, defining its behavior as concatenating the lists within $\codeVLL$ by party (i.e., $[(1, \codeVL^1_1), (2, \codeVL^2_1)]::[(1,\codeVL^1_2)] = [(1, \codeVL^1_1::\codeVL^1_2), (2, \codeVL^2_1)]$). 
The assertions in each semantic rule can be read in sequential order, from left to right and top to bottom. 

We present a subset of  the \vanillaC\ semantics  rules in Figure~\ref{Fig: \vanillaC
  sem rules}, focusing on the most interesting rules. Specifically, we present rules for arrays and pointers
which we will compare with the rules for the
\piccoC\ semantics in Section~\ref{smc C descr}. Because the semantics
rules are mostly standard, we will describe one rule to familiarize 
the reader with our notation. 

Rule Multiparty Binary Operation is an example of a rule for the evaluation of a binary operation (comparison operations are handled separately). In \vanillaC, multiparty rules occur at the same time in all parties, but without any communication between the parties. 
We have the starting state $((\pidA,$ $\hgamma,$ $\hsigma,$ $\bsq,$ $\bsq,$ $\hExpr_1\ \binop\ \hExpr_2)$ $\Mid ...\Mid $ $(\pidZ,$ $\hgamma,$ $\hsigma,$ $\bsq,$ $\bsq,$ $\hExpr_1\ \binop\ \hExpr_2))$, 
with all parties at current environment $\hgamma$, current memory $\hsigma$, and the starting statement $\hExpr_1\ \binop\ \hExpr_2$.
First, all parties evaluate expression $\hExpr_1$, using the
current environment and memory states, 
resulting in environment $\hgamma$, memory
$\hsigma_1$, and number $\hn_1$.  We repeat this for
$\hExpr_2$. 
We then evaluate $\hn_1\ \binop\ \hn_2 = \hn_3$ here, and we will return $\hn_3$.  
The end state is then
$((\pidA, \hgamma, \hsigma_2, \bsq, \bsq, \hn_3)$ $\Mid ...\Mid$ $(\pidZ, \hgamma, \hsigma_2, \bsq, \bsq, \hn_3))$.




































%%%%%%%%%%%%%%%%%%%%%%%
%
%   Begin Picco C semantics description 
%
%%%%%%%%%%%%%%%%%%%%%%%

\begin{figure*}\footnotesize
\begin{tabular}{l}
% BASIC  EVAL
Multiparty Binary Operation	\\ 
\inferrule{\begin{array}{l l}
	\begin{array}{l}
		\qquad((\pidA, \gamma^{\pidA}, \sigma^{\pidA}, \DMap^{\pidA}, \Acc, \Expr_{1}) \Mid ...\Mid (\pidZ, \gamma^{\pidZ}, \sigma^{\pidZ}, \DMap^{\pidZ}, \Acc, \Expr_{1})) 
			\crcr\Deval{\locLL_1}{\codeLL_1} 
			((\pidA, \gamma^{\pidA}, \sigma^{\pidA}_{1}, \DMap^{\pidA}_{1}, \Acc, \n^{\pidA}_{1}) \Mid ...\Mid (\pidZ, \gamma^{\pidZ}, \sigma^{\pidZ}_{1}, \DMap^{\pidZ}_{1}, \Acc, \n^{\pidZ}_{1}))
		\crcr
		\qquad((\pidA, \gamma^{\pidA}, \sigma^{\pidA}_{1}, \DMap^{\pidA}_{1}, \Acc, \Expr_{2}) \Mid ...\Mid (\pidZ, \gamma^{\pidZ}, \sigma^{\pidZ}_{1}, \DMap^{\pidZ}_{1}, \Acc, \Expr_{2})) 
			\crcr\Deval{\locLL_2}{\codeLL_2} 
			((\pidA, \gamma^{\pidA}, \sigma^{\pidA}_{2}, \DMap^{\pidA}_{2}, \Acc, \n^{\pidA}_{2})\Mid ...\Mid (\pidZ, \gamma^{\pidZ}, \sigma^{\pidZ}_{2}, \DMap^{\pidZ}_{2}, \Acc, \n^{\pidZ}_{2}))
		\crcr
		\MPC{b}(\binop, [\n^\pidA_1, ..., \n^\pidZ_1], [\n^\pidA_2, ..., \n^\pidZ_2]) = (\n^{\pidA}_{3}, ..., \n^{\pidZ}_{3})
	\end{array}
	&\begin{array}{l}
		\{(\Expr_{1}, \Expr_{2}) \isPriv \gamma^\pid\}^{\pidZ}_{\pid = \pidA}
		\crcr \binop\in\{\cdot, +, -, \div\}
	\end{array}\end{array}}
	{\begin{array}{l}
	((\pidA, \gamma^{\pidA}, \sigma^{\pidA}, \DMap^{\pidA}, \Acc, {\Expr_{1}\ \binop\ \Expr_{2}})\Mid ...\Mid (\pidZ, \gamma^{\pidZ}, \sigma^{\pidZ}, \DMap^{\pidZ},\Acc, {\Expr_{1}\ \binop\ \Expr_{2}})) 
		\Deval{\locLL_1 \addL \locLL_2}{\codeLL_1 \addC \codeLL_2 \addC \codeMP{mpb}} 
		\crcr((\pidA, \gamma^{\pidA}, \sigma^{\pidA}_{2}, \DMap^{\pidA}_{2}, \Acc, \n^{\pidA}_{3}) 
			\qquad\-\ \-\ \-\ \Mid ...\Mid 
		(\pidZ, \gamma^{\pidZ}, \sigma^{\pidZ}, \DMap^{\pidZ}_{2}, \Acc, \n^{\pidZ}_{3})) \end{array}}
\\ \\
Write Private Variable Public Value		\\
\inferrule{\begin{array}{l l}
		\begin{array}{l}
		(\Expr) \isPub \gamma \crcr
		\crcr \gamma(\x) = (\loc, {\Priv\ \btype})  
		\end{array}
		& \begin{array}{l}
		((\pid, \gamma, \sigma, {\DMap}, \Acc, \Expr) \Mid  \Config)  
			\Deval{\locLL_1}{\codeLL_1}  ((\pid, \gamma, \sigma{_1}, {\DMap_1}, \Acc, \n) \Mid  \Config_1) 
		\crcr 
		\Update(\sigma{_1}, \loc, \Encrypt(\n), \Acc, \Priv\ \btype) = \sigma{_2}
	\end{array}\end{array}}
	{((\pid, \gamma,\ \sigma,\ {\DMap},\ \Acc,\ {\x = \Expr}) \Mid  \Config)\ 
		\Deval{\locLL_1 \addL (\pid, [(\loc, 0)])}{\codeLL_1 \addC \codeSP{w2}}  
		((\pid, \gamma,\ \sigma{_2},\ {\DMap_1},\ \Acc,\ \Skip) \Mid  \Config_1)}
\\ \\
Private Malloc \\
\inferrule{\begin{array}{l}
		(\Expr) \isPub \gamma \qq
		\Acc = \AccZ \qq
		(\Type = {\Priv\ \btype*}) \lor (\Type = {\Priv\ \btype}) \crcr
		\loc = \phi()	\qq
		((\pid, \gamma,\ \sigma,\ {\DMap},\ \Acc,\ \Expr) \Mid  \Config)\ 
			\Deval{\locLL_1}{\codeLL_1}  ((\pid, \gamma,\ \sigma{_1},\ {\DMap},\ \Acc,\ {n}) \Mid  \Config_1) 
		\crcr \sigma_2 = \sigma{_1} \big[\loc \to \big(\Null,\ \Void*,\ {n\cdot\tau(\Type)},\ \PermL(\PermF, \Void*, \Priv, \n\cdot\tau(\Type))\big)\big]	
	\end{array}}
	{\begin{array}{l}
	((\pid, \gamma,\ \sigma,\ {\DMap},\ \Acc,\ {\PMalloc (\Expr,\ \Type)}) \Mid  \Config) 
		\Deval{\locLL_1 \addL (\pid, [(\loc, 0)])}{\codeLL_1 \addC \codeSP{malp}} 
		((\pid, \gamma,\ \sigma{_2},\ {\DMap},\ \Acc,\ (\loc, 0)) \Mid  \Config_1)
	\end{array} }
\\ \\
% ALLOC / DEALLOC
Multiparty Private Free	\\
\inferrule{ \begin{array}{l}
		\{\gamma^\pid(\x) = (\loc^\pid,\ {\Priv\ \btype*})\}^{\pidZ}_{\pid = \pidA}  
		\qq \Acc = \AccZ  \qq
		(\btype = \Int) \lor (\btype = \Float) \crcr
		\{\sigma^\pid(\loc^\pid) = (\byte^\pid, \Priv\ \btype*, \nl, \PtrPermL(\PermF, \Priv\ \btype*, \Priv, \nl))\}^{\pidZ}_{\pid = \pidA} \crcr
		\{[\nl,\ \locL^\pid\ \tagbL^\pid,\ \indir] = \DecodePtr({\Priv\ \btype*},\ \nl,\ \byte^\pid)\}^{\pidZ}_{\pid = \pidA}
		\qq \{\nl > 1\}^{\pidZ}_{\pid = \pidA}  \crcr
		\If(\indir>1) \{\Type = \Priv\ \btype* \}\ \Else\ \{ \Type = \Priv\ \btype \}
		\crcr
		\{\SelectFreeable(\gamma^\pid, \locL^\pid, \tagbL^\pid, \sigma^\pid) = 1\}^{\pidZ}_{\pid = \pidA} 
		\crcr
		\{\forall (\loc^\pid_m, 0) \in \locL.\quad \sigma^\pid(\loc^\pid_m) = (\byte^\pid_m, \Type, n, \PermL(\PermF, \Type, \Priv, n))\}^{\pidZ}_{\pid = \pidA}
		\crcr
		\PFree([[\byte^\pidA_0, ..., \byte^\pidA_{\nl-1}], ..., [\byte^\pidZ_0, ..., \byte^\pidZ_{\nl-1}]], [\tagbL^\pidA, ...\tagbL^\pidZ]) 
			\crcr\qq= ([[\byte'^\pidA_0, ..., \byte'^\pidA_{\nl-1}], ..., [\byte'^\pidZ_0, ..., \byte'^\pidZ_{\nl-1}]], [\tagbL'^\pidA, ..., \tagbL'^\pidZ])
		\crcr
		\{\UpdateBytesFree(\sigma^\pid, \locL^\pid, [\byte'^\pid_0, ..., \byte'^\pid_{\nl-1}]) = \sigma^\pid_1\}^{\pidZ}_{\pid = \pidA}
		\crcr
		\{\sigma^\pid_2 = \UpdatePtrLocs(\sigma^\pid_1, \locL^\pid[1:\nl-1], \tagbL^\pid[1:\nl-1], \locL^\pid[0], \tagbL^\pid[0])\}^{\pidZ}_{\pid = \pidA}
	\end{array}}
	{\begin{array}{l}((\pidA, \gamma^{\pidA}, \sigma^{\pidA}, \DMap^\pidA, \Acc, {\pfree (\x)})\Mid ...\Mid (\pidZ, \gamma^{\pidZ}, \sigma^{\pidZ}, \DMap^\pidZ, \Acc, {\pfree (\x)})) 
		\Deval{(\pidA, \locL^\pidA) \Mid ... \Mid (\pidZ, \locL^\pidZ)}{\codeMP{mpfre}} 
		\crcr((\pidA, \gamma^{\pidA}, \sigma^{\pidA}_2, \DMap^\pidA, \Acc, \Skip)\quad\-\ \ \Mid ...\Mid 
		(\pidZ, \gamma^{\pidZ}, \sigma^{\pidZ}_2, \DMap^\pidZ, \Acc, \Skip)) \end{array}}
\\ \\ 
% PTR  EVAL
Private Pointer Declaration \\ 
\inferrule{\begin{array}{l}\begin{array}{l l}
		\getIndirection(*) = \indir\qq
		&((\Type = {\btype *}) \lor (\Type = {\Priv\ \btype*})) \land ((\btype = \Int) \lor (\btype = \Float))
		\end{array}\crcr \begin{array}{l l l} 
		\loc = \phi() \qq
		& \gamma{_1} = \gamma[\x \to (\loc, \Priv\ \btype*)] 	
		\end{array}\crcr\begin{array}{l}
		\byte = \EncodePtr(\Priv\ \btype*, [1, [(\locDefault, 0)], [1], \indir])	\crcr
		\sigma{_1} = \sigma[\loc \to (\byte, \Priv\ \btype*, 1, \PtrPermL(\PermF, \Priv\ \btype*, \Priv, 1))]
	\end{array}\end{array}}					
	{((\pid, \gamma,\ \sigma,\ {\DMap},\ \Acc,\ {\Type\ \x}) \Mid  \Config)\ 
		\Deval{(\pid, [(\loc, 0)])}{\codeSP{dp1}}  ((\pid, \gamma{_1},\ \sigma{_1},\ {\DMap},\ \Acc,\ \Skip) \Mid  \Config)}
% ARR  EVAL
\\ \\
Multiparty Array Read Private Index \\
	\inferrule{\begin{array}{l}
		\qquad((\pidA, \gamma^{\pidA}, \sigma^{\pidA}, \DMap^\pidA, \Acc, \Expr)\ \Mid ...\Mid (\pidZ, \gamma^{\pidZ}, \sigma^{\pidZ}, \DMap^\pidZ, \Acc, \Expr)) 
		\qquad\ \
		\{(\Expr) \isPriv \gamma^\pid\}^{\pidZ}_{\pid = \pidA} 
		\crcr		
			\crcr\Deval{\locLL_1}{\codeLL_1} ((\pidA, \gamma^{\pidA}_{}, \sigma^{\pidA}_{1}, \DMap^{\pidA}_{1}, \Acc, \ind^{\pidA})\Mid ...\Mid (\pidZ, \gamma^{\pidZ}_{}, \sigma^{\pidZ}_{1}, \DMap^{\pidZ}_{1}, \Acc, \ind^{\pidZ}))
		\qquad\ \{\gamma^\pid(\x) = (\loc^\pid, \Const\ \llabel\ \btype*)\}^{\pidZ}_{\pid = \pidA}
		\crcr
		\{\sigma^\pid_1(\loc^\pid) = (\byte^\pid,\ {\llabel\ \Const\ \btype*}, 1, 
			\PtrPermL(\PermF, {\llabel\ \Const\ \btype*}, \llabel, 1))\}^{\pidZ}_{\pid = \pidA} 
		\crcr \{\DecodePtr({\llabel\ \Const\ \btype*},\ 1,\ \byte^\pid) 
			= [1,\ [({\loc^\pid_1}, 0)],\ [1],\ 1]\}^{\pidZ}_{\pid = \pidA}  
		\crcr \{\sigma^\pid_1({\loc^\pid_1}) = ({\byte^\pid_1}, {\llabel\ \btype}, {\nl}, 
									\ArrPermL(\PermF, \llabel\ \btype, \llabel, {\nl}))\}^{\pidZ}_{\pid = \pidA}
		\crcr 
			\{\forall \ind \in \{0...\nl-1\} \quad \DecodeArr({\llabel\ \btype}, \ind, {\byte^\pid_1}) =  \n^\pid_\ind\}^{\pidZ}_{\pid = \pidA}
		\crcr
		\MPC{ar}((\ind^\pidA, [\n^{\pidA}_{0}, ..., \n^{\pidA}_{\nl-1}]), ..., (\ind^\pidZ, [\n^{\pidZ}_{0}, ..., \n^{\pidZ}_{\nl-1}])) = (\n^{\pidA}, ..., \n^{\pidZ})
		\qq \{(\n^\pid) \isPriv \gamma^\pid\}^{\pidZ}_{\pid = \pidA} 
		\crcr \locLL_2 = (\pidA, [(\loc^\pidA, 0), (\loc^\pidA_1, 0), ..., (\loc^\pidA_1, \nl-1)])\Mid ... \Mid (\pidZ, [(\loc^\pidZ, 0), (\loc^\pidZ_1, 0), ..., (\loc^\pidZ_1, \nl-1)])
	\end{array}}
	{\begin{array}{l}
		((\pidA, \gamma^{\pidA}, \sigma^{\pidA}, \DMap^\pidA, \Acc, \x[\Expr])\Mid ...\Mid (\pidZ, \gamma^{\pidZ}, \sigma^{\pidZ}, \DMap^\pidZ, \Acc, \x[\Expr])) 
			\Deval{\locLL_1 \addL \locLL_2}{\codeLL_1\addC \codeMP{mpra}} 
			\crcr((\pidA, \gamma^{\pidA}_{}, \sigma^{\pidA}_{1}, \DMap^{\pidA}_{1}, \Acc, \n^{\pidA})\-\ \Mid ...\Mid (\pidZ, \gamma^{\pidZ}_{}, \sigma^{\pidZ}_{1}, \DMap^{\pidZ}_{1}, \Acc, \n^{\pidZ})) \end{array}}
\end{tabular}
\caption{\piccoC\ semantic rules.}
\label{Fig: \piccoC sem rules}
\end{figure*}


























\subsection{\piccoC\ Semantics} \label{smc C descr}


The \piccoC\ semantics are defined over multiple {\em interacting} parties.
The \piccoC\ semantics used to define the behavior of 
parties are mostly standard, with \emph{non-interactive} semantic rules identical to those of \vanillaC\ semantics aside from additional assertions over the privacy labels of data and properly managing the private data. 
A few notable exceptions are \emph{interactive} SMC operations (and in general operations over private values)
and the private-conditioned \Code{if else} statement, discussed in later in this section.    
To prevent leakage from within private-conditioned branches, we restrict all public side effects (i.e., the use of functions with public side effects, allocation and deallocation of memory, and any modifications to public variables). 
Additionally, in the case of pointer dereference write and array write statements, we have an additional check for when this occurs within a private-conditioned branch, as we need to perform additional analysis to ensure the location being written to is tracked properly due to the potential for the pointer's location being modified or an out-of-bounds array write. 
To enforce these restrictions, we use the assertion $\Acc = \AccZ$ within each restricted rule -- as the accumulator $\Acc$ is incremented at each level of nesting of a private-conditioned branch, this will result in a runtime failure. 
We annotate each evaluation with party-wise lists of the evaluation codes $\codeLL$ of all rules that were used during the execution of the rule (i.e., $\Deval{\locLL}{\codeLL}$) in order to keep an accurate evaluation tree, and party-wise lists of locations accessed $\locLL$ in order to show data-obliviousness (i.e., that given the same program and public data, we will always access the same set of locations). 


%%%%%%%%%%%%%%%%%%%%%%
%
% 		Basic Evaluations
%
%%%%%%%%%%%%%%%%%%%%%%
\subsubsection{Basic Evaluation} \label{subsec: picco basic eval}
% Less Than True
To better illustrate the correspondence between \piccoC\  and \vanillaC\  let us consider the Multiparty Binary Operation rule.
\piccoC\ rule Multiparty Binary Operation asserts that one of the given binary operators ($\cdot, +,-,\div$) is used and additionally that either expression contains private data with relation to the environment. 
We use the notation $(\Expr_{1}, \Expr_{2}) \isPriv \gamma$ to show this relation, and  
notation $\{...\}^{\pidZ}_{\pid = \pidA}$ to show that all parties will ensure that property holds locally.
We then use the multiparty protocol $\MPC{b}$, passing the given binary operator and the current values of $\n^\pid_1$ and $\n^\pid_2$ for each party $\pid$. 
This protocol will dictate how communication occurs and what data is exchanged between parties. 
We receive $\n^\pid_3$ as the result for each party, which we then return appropriately.   We assume that the protocol is implemented correctly (i.e. provided by the underlying SMC cryptographic library) and define this assumption formally, its impact on our noninterferences proof, and how to reason if a library adheres to our assumption later in Section~\ref{sec: noninterference}. 
Within the multiparty rules, each party maintains control of their own data, only sharing it with other parties in the ways dictated by the multiparty protocols. We choose to show the execution of the entire computational process here in order to emphasize what data is involved, and that each of the parties will take part in this computation. 

% Write Priv x = Pub
In rule Write Private Variable Public Value, we assert that $\Expr$ contains only public data,  
then evaluate $\Expr$ to $n$. We look up $\x$ in $\gamma$, asserting that it is a $\Priv\ \btype$ at location $\loc$. 
When we call $\Update$ to store this value to memory, we must pass $\Encrypt(n)$ as $\x$ is private and we are assigning it the public value $n$. 


%%%%%%%%%%%%%%%%%%%%%%
%
% 		Pointer Evaluations
%
%%%%%%%%%%%%%%%%%%%%%%
\subsubsection{Pointer Evaluations} \label{subsec: picco ptr eval}
In order to maintain data-oblivious execution, we need to allow storing multiple locations for pointers when they are modified within a private-conditioned branching statement. To achieve this, 
% Pointer structure
the structure of the data stored by pointers is as follows: the number $\nl$ of locations being pointed to; a list of $\nl$ locations being pointed to; a list of $\nl$ tags; and the level of indirection of the pointer. 
% 
The privacy labels of the byte-wise permissions corresponding to the number $\nl$ of locations to which the pointer refers, the list of $\nl$ locations, and the level of indirection will always be public, as it is visible to an observer of memory the number and which locations are touched by a pointer, and the level of indirection is visible in the source program.
%
The privacy labels of the permissions corresponding to the tags of public pointers will always be public; for private pointers, they will be private, as these protect an observer of memory from being able to tell which location is the true location. 
%
%%
% Priv Ptr Decl
In rule Private Pointer Declaration, we assert that the type of the variable being declared is an int or float pointer that is either declared as private or missing a privacy label (i.e., $\btype*$) and therefore is assumed to be private. 
As with the \vanillaC\ pointer declaration, we check the level of indirection and add the appropriate mappings to the environment and memory. 


%%%%%%%%%%%%%%%%%%%%%%
%
% 		Mem Alloc / Dealloc
%
%%%%%%%%%%%%%%%%%%%%%%
\subsubsection{Memory Allocation and Deallocation} \label{subsec: picco mem alloc/dealloc}
% PMalloc
When allocating private memory, we provide the \TT{pmalloc} builtin function to internally handle obtaining the size of the private type; the programmer to only needs to know how many elements of the given type they desire to allocate.
In rule Private Malloc, we assert that the given type is either private int or private float, as this function only handles those types, and that the accumulator $\Acc$ is $\AccZ$ (i.e., we are not inside an \Code{if else} statement branching on private data, as this function causes public side effects). 
Then we evaluate $\Expr$ to $n$ and obtain the next open memory location $\loc$ from $\phi$. We add to $\sigma{_1}$ the new mapping from $\loc$ to the tuple of a $\Null$ set of bytes; the type $\Type$; the size $n$; and a list of $\Priv$, $\PermF$ permissions. 
As with public \Code{malloc}, we return the new location, $(\loc, 0)$. 

% PFree
When deallocating private memory, we provide the \TT{pfree} builtin function to handle private pointers potentially having multiple locations. 
In the case of a single location, it behaves identically to Public Free; however, with multiple locations, we need to deterministically free a single location (which may or may not be the true location that was intended to be freed) to maintain data-obliviousness. We describe this case in more detail here.   
In rule Multiparty Private Free, we assert that $\x$ is a private pointer of type int or float, we are not inside a private-conditioned branch ($\Acc$ is $\AccZ$, as this rule causes public side effects), and that the number of locations the pointer refers to ($\nl$) is greater than 1 for all parties. 
We then assert that \emph{all} locations referred to by $\x$ are freeable (i.e., they are all memory blocks that were allocated via malloc) and proceed to retrieve the data that is stored for each of these locations.
This data and the tag lists are then passed to $\PFree$, as this is what we will need in order to privately free a location without revealing if it was the true location. 

To accomplish this, we must free one location based on publicly available information, regardless of the true location of the pointer. For that reason, and without loss of generality, we free the first location, $\loc_0$. 
Since $\loc_0$ may not be the true location and may be in use by other pointers, we need to do additional computation to maintain correctness without disclosing whether or not this was the true location. 
In particular, if $\loc_0$ is not the true location, we preserve the content of $\loc_0$ by obliviously copying it to the pointer's true location prior to freeing. 
This behavior is defined in function $\PFree$, and follows the strategy suggested in~\cite{Zhang18}.
$\PFree$ returns the modified bytes and tag lists. $\UpdateBytesFree$ then updates these in their corresponding locations in memory and marks the permissions of $\loc_0$ as $\PermN$ (i.e., this block has been freed). 
The remaining step is to update other pointers that stored $\loc_0$ on their lists to point to the updated location instead of $\loc_0$, which is accomplished by $\UpdatePtrLocs$. 



%%%%%%%%%%%%%%%%%%%%%%
%
% 		Array Evaluations
%
%%%%%%%%%%%%%%%%%%%%%%
\subsubsection{Array Evaluations} \label{subsec: picco arr eval}
With array evaluations, all evaluations that use a public index behave nearly identically to those over public data. 
The difference is that we have an additional check within array writes at a public index to see if we are in a private-conditioned branch; if so, we must ensure we properly track the modification made (this is because a public index that is not hard-coded could have lead to an out-of-bounds array write). 
We will discuss this further in the following section. 
When we have a private index, it is necessary to hide which location we a reading from or writing to to maintain data-obliviousness. One such semantic rule is Multiparty Array Read Private Index.
% 
% PUB 1D ARR READ PRIV INDEX
Here, we assert that the privacy label of $\Expr$ is private as this rule handles private indexing into a public array. 
We assert that $\x$ is a public array of type int or float, as we must return a private value when using a private index. 
Then we evaluate $\Expr$ to the private index $\ind$, and perform lookups to obtain the data of the array. 
Now, because we have a private index, we must obtain the value without revealing which location we are taking the value from. 
To do this, we use multiparty protocol $\MPC{ar}$, which returns a private number containing the value from the desired location.  
It is important to note here that even if the private index is beyond the bounds of the array, we do not access beyond the elements within the array, as that would reveal information about the private index. 
An example of how this protocol can be implemented is to iterate over all values stored in the array; at each value, we encrypt the current index number $m$, privately compare it to $\ind$, and perform a bitwise \Code{and} operation over this and the encrypted value $\n_m$ stored at index $m$. We perform a bitwise \Code{or} operation over each such value obtained from the array to attain our final encrypted value $\n$, which is returned. 
%









































%%%%%%%%%%%%%%%%%%%%%%%
%
%   picco C If Else description 
%
%%%%%%%%%%%%%%%%%%%%%%%

\subsubsection{\piccoC\ If Else} \label{subsec: \piccoC if else}

The public \Code{if else} rules 
are nearly identical to the \vanillaC\ rules, (e.g., \vanillaC\ rule If Else True shown in Figure~\ref{Fig: if else \vanillaC true}), 
with the added assertion that the guard of the conditional is public (i.e., does not contain private data): $(\Expr) \isPub \gamma$.
%
The private \TT{if else} rules, shown in Figures~\ref{Fig: iep vt} and~\ref{Fig: iep lt}, are more interesting. Our strategy for dealing with private-conditioned branches involves executing both branches as a sequence of statements (with some additional helper algorithms to aid in storing changes, restoration between branches, and resolution of true values). 
We chose to use big-step semantics to facilitate the comparison of the \piccoC\ semantics with the \vanillaC\ semantics, and for its proof of correctness that we will discuss in the next Section. 
% general description
We give also an example of \piccoC\ code in Figure~\ref{Fig: if else \piccoC code} and~\ref{Fig: if else \piccoC code dp}, and the corresponding execution in Figure~\ref{Fig: if else \piccoC expanded} and~\ref{Fig: if else piccoC expanded dp}. We use coloring throughout Figure~\ref{Fig: if else color} and~\ref{Fig: if else color dp} to highlight the corresponding sections of code and rule execution. 

% SPECIFICS  --  IF ELSE
The starting and ending states of the \piccoC\ Private If Else rules are essentially the same as the starting and ending states of the corresponding \vanillaC\ If Else rule; however, there are several additional assertions that guarantee that both of the private-conditioned branches are executed. 
The assertions of these semantic rules are listed sequentially, from top to bottom.
We have two different styles of tracking modifications within conditional code blocks that are used within these rules: variable tracking and location tracking. 
Variable tracking is used when there are only single-level changes within the private-conditioned branches, whereas location tracking is used when we have multi-level changes (i.e., a branch contains a pointer dereference write) or potential out-of-bounds changes (i.e., array write at a public index). 

The main idea of both styles is to first store the original value of each variable that is modified within either branch; execute the \TT{then} branch; save the resulting values from the \TT{then} branch and restore all modified variables to their original values; execute the \TT{else} branch; and finally, to securely resolve which values should be kept -- those from the \TT{then} branch or those from the \TT{else} branch. 
In the variable style of tracking, we utilize temporary variables to keep track of all modifications made during either branch -- initializing the \TT{else} temporary with the original value, storing the result of the \TT{then} branch in the \TT{then} temporary and using the \TT{else} temporary to restore the original value, and finally using the result of the private-conditional and what is stored in each variable at the end of the \TT{else} branch as well as it's corresponding \TT{then} temporary to securely resolve what values to continue evaluating the program with. 

This style of tracking is robust enough for many uses, however, there are two notable exceptions where we run into issues, both involving the potential of the location we track not being the location that is actually modified. 
The first exception involves pointer dereference writes -- these alone are not an issue, but when location the pointer refers to is modified and we also perform pointer dereference writes, it becomes clear that variable tracking cannot easily find and handle these cases. 
The second exception involves array writes at public indices -- these become problematic due to the potential for writing out-of-bounds. As most array indices are not hard-coded, it isn't obvious that the write will be within bounds until execution, and to ensure we catch all of these cases we must use a more robust style of tracking to catch out-of-bounds writes. We stress here that array writes at private indices do not fall within this exception, as this operation will securely update the array within its bounds (as updating beyond the bounds of the array would leak that this private value is larger than the size of the array), and as such we can simply track the entire array properly using variable tracking.  
It is possible to ensure that we find all of the locations that are modified in both of these cases by dynamically adding these types of modifications as they are evaluated, which is the goal of the location tracking. 
In the location style of tracking, we still follow a similar evaluation pattern as with variable tracking, storing the original values for locations we know will be modified first, then restoring between branches, and resolving at the end. As we evaluate each branch and come upon one of these special cases, we will check to see if we have already marked that location for tracking, and if not we add that location and its original value before the modification occurs. 
It is worthwhile to stress again the role of the accumulator here with respect to other statements. We increment it when we evaluate the \TT{then} and \TT{else} statements, so that if we attempt to evaluate a (sub)statement with public side effects or restricted operations, we have an (oblivious) runtime failure. It also facilitates scoping of temporary variables within nested private-conditioned \TT{if else} statements.
We proceed to further describe the different assertions and specifics of both styles next. 


% insert IF ELSE figure


\begin{figure*} \footnotesize
\begin{tabular}{l}
\hspace{0.3cm}
\begin{tabular}{l l}
\begin{subfigure}{.36\textwidth}
\begin{lstlisting}
private int a=3,b=7,c=0;		
if ($\Code{\ExprC{a<b}}$) $\Code{\sC{c=a;}}$
else $\Code{\ssC{c=b;}}$
\end{lstlisting}
	\caption{\piccoC\ code.}
	\label{Fig: if else \piccoC code}	
\end{subfigure}		
&
\begin{subfigure}{.57\textwidth}
\begin{lstlisting}
private int a=3,b=7,c=0,$\Code{\initC{res=}\ExprC{a<b},\initC{c\_t=c,c\_e=c};}$
$\Code{\sC{c=a;}}$ $\Code{\restC{c\_t=c; c=c\_e;}}$
$\Code{\ssC{c=b;}}$ $\resoC{\Code{c=(res}\cdot\Code{c\_t)+((1-res)}\cdot\Code{c);}}$ 	
\end{lstlisting}
	\caption{Variable-tracking execution.}	
	\label{Fig: if else \piccoC expanded}
\end{subfigure} 
\end{tabular}
\\ \\
\begin{subfigure}{\textwidth}
	\inferrule{\begin{array}{l} 
		\qquad
			\ExprC{((\pidA, \gamma^\pidA, \sigma^\pidA, \DMap^\pidA, \Acc, \Expr) \ \ \Mid ... \Mid
					 (\pidZ, \gamma^\pidZ, \sigma^\pidZ, \DMap^\pidZ, \Acc, \Expr))} 
			\crcr\ExprC{\Deval{\locLL_1}{\codeLL_1} 
			((\pidA, \gamma^\pidA, \sigma^\pidA_1, \DMap^\pidA_1, \Acc, n^\pidA) \Mid ... \Mid
			 (\pidZ, \gamma^\pidZ, \sigma^\pidZ_1, \DMap^\pidZ_1, \Acc, n^\pidZ))}
		\qq
			\{(\ExprC\Expr) \isPriv \gamma^\pid\}^\pidZ_{\pid = \pidA}
		\crcr 
			\extC{\{\DynExtract(}\sC{\stmt_1}\extC{,}\ \ssC{\stmt_2} \extC{, \gamma^\pid) = (\x_\vl, 0)\}^\pidZ_{\pid = \pidA}}
		\crcr
			\initC{\{\Initialize(}\extC{\x_\vl}\initC{,\ \gamma^\pid, \sigma^\pid_1, \ExprC{\n^\pid},\ \AccPP) = (\gamma^\pid_1, \sigma^\pid_2, \locL^\pid_2)\}^\pidZ_{\pid = \pidA}}
		\crcr\qquad
			\sC{((\pidA, \gamma^\pidA_1, \sigma^\pidA_2, \DMap^\pidA_1, \AccPP, \stmt_1) \ \ \ \Mid ... \Mid
			 	  (\pidZ, \gamma^\pidZ_1, \sigma^\pidZ_2, \DMap^\pidZ_1, \AccPP, \stmt_1))}
			\crcr\sC{\Deval{\locLL_3}{\codeLL_2} 
				((\pidA, \gamma^\pidA_2, \sigma^\pidA_3, \DMap^\pidA_2, \AccPP, \Skip) \Mid ... \Mid
				 (\pidZ, \gamma^\pidZ_2, \sigma^\pidZ_3, \DMap^\pidZ_2, \AccPP, \Skip))}
		\crcr
			\restC{\{\Restore(}\extC{\x_\vl}\restC{,\ \gamma^\pid_1,\ \sigma^\pid_3,\ \AccPP) = (\sigma^\pid_4, \locL^\pid_4)\}^\pidZ_{\pid = \pidA}}
		\crcr\qquad
			\ssC{((\pidA, \gamma^\pidA_1, \sigma^\pidA_4, \DMap^\pidA_2, \AccPP, \stmt_2) \ \ \ \Mid ... \Mid
				   (\pidZ, \gamma^\pidZ_1, \sigma^\pidZ_4, \DMap^\pidZ_2, \AccPP, \stmt_2))}
				\crcr\ssC{\Deval{\locLL_5}{\codeLL_3} 
				((\pidA, \gamma^\pidA_3, \sigma^\pidA_5, \DMap^\pidA_3, \AccPP, \Skip) \Mid  ... \Mid
				 (\pidZ, \gamma^\pidZ_3, \sigma^\pidZ_5, \DMap^\pidZ_3, \AccPP, \Skip))}
		\crcr
			\resoC{\{\Resolve\_\mathrm{Retrieve}(\extC{\x_\vl},\ \AccPP, \gamma^\pid_1, \sigma^\pid_5) 
				= ([(\val^\pid_{t1}, \val^\pid_{e1}), ..., (\val^\pid_{tm}, \val^\pid_{em})], 
					\ExprC{\n^\pid}, \locL^\pid_6)\}^\pidZ_{\pid = \pidA}}
		\crcr
			\resoC{\MPC{resolve}([\ExprC{\n^\pidA}, ..., \ExprC{\n^\pidZ}], 
				[[(\val^\pidA_{t1}, \val^\pidA_{e1}), ..., (\val^\pidA_{tm}, \val^\pidA_{em})]], ..., 
				 [(\val^\pidZ_{t1}, \val^\pidZ_{e1}), ..., (\val^\pidZ_{tm}, \val^\pidZ_{em})]])} 
				\crcr\qquad\resoC{= [[\val^\pidA_1, ..., \val^\pidA_m], ... [\val^\pidZ_1, ..., \val^\pidZ_m]]}
		\crcr
			\resoC{\{\Resolve\_\mathrm{Store}(\extC{\x_\vl},\ \AccPP, \gamma^\pid_1, \sigma^\pid_5, 
				[\val^\pid_1, ..., \val^\pid_m]) = (\sigma^\pid_6, \locL^\pid_7)\}^\pidZ_{\pid = \pidA}}
		\crcr
			\locLL_6 = \locLL_1 \addL (\pidA, \locL^\pidA_2) \Mid ... \Mid (\pidZ, \locL^\pidZ_2) \addL \locLL_3 
						\addL (\pidA, \locL^\pidA_4) \Mid ... \Mid (\pidZ, \locL^\pidZ_4) \addL \locLL_5
						\addL (\pidA, \locL^\pidA_6) \Mid ... \Mid (\pidZ, \locL^\pidZ_6) 
		\crcr 
			\locLL_7 = \locLL_6 \addL (\pidA, \locL^\pidA_7) \Mid ... \Mid (\pidZ, \locL^\pidZ_7)
		\qq 
			\codeLL_4 = \codeLL_1 \addC \codeLL_2 \addC \codeLL_3
	\end{array} }
	{\begin{array}{l} 
	((\pidA, \gamma^\pidA, \sigma^\pidA, \DMap^\pidA, \Acc, \If\ (\ExprC\Expr)\ \sC{\stmt_1}\ \Else\ \ssC{\stmt_2}) 
	 \Mid ... \Mid 
	 (\pidZ, \gamma^\pidZ, \sigma^\pidZ, \DMap^\pidZ, \Acc, \If\ (\ExprC\Expr)\ \sC{\stmt_1}\ \Else\ \ssC{\stmt_2})) 
		\Deval{\locLL_7}{\codeLL_4 \addC \codeSP{iep}} 
		\crcr ((\pidA, \gamma^\pidA, \resoC{\sigma^\pidA_{6}}, \resoC{\DMap^\pidA_{3}}, \Acc, \Skip) 
		\qq \-\ \-\ \Mid ... \Mid 
		 (\pidZ, \gamma^\pidZ, \resoC{\sigma^\pidZ_{6}}, \resoC{\DMap^\pidZ_{3}}, \Acc, \Skip) )
		\end{array}}
	\caption{\piccoC\ rule Private If Else - Variable Tracking.}
	\label{Fig: iep vt}
\end{subfigure}
\\ \\
\begin{subfigure}{\textwidth}
	\inferrule{\begin{array}{l}
		\ExprC{((\pid, \hgamma, \hsigma,\ \bsq, \bsq, \hExpr)\ \ \Mid \hConfig)\ \ }
		\ExprC{\Veval_{\codeVLL_1}\ }
			\ExprC{((\pid, \hgamma,\ \hsigma_1, \bsq, \bsq, \hat{n})\ \ \ \Mid \hConfig_1)}
		\qq \ExprC{\hat{n}} \neq \ExprC{0}
		\crcr\sC{((\pid, \hgamma, \hsigma_1, \bsq, \bsq, \hstmt_1) \Mid \hConfig_1)\ }  
			\sC{\Veval_{\codeVLL_2}((\pid, \hgamma_1, \hsigma_2, \bsq, \bsq, \Skip)\Mid \hConfig_2)}
	\end{array}}
	{\begin{array}{l}
	((\pid, \hgamma, \hsigma, \bsq, \bsq, \If (\ExprC{\hExpr})\ \sC{\hstmt_1}\ \Else\ \ssC{\hstmt_2}) \Mid \hConfig)
		\Veval_{\codeVLL_1\addC\codeVLL_2\addC[\codeVS{iet}]} 
		((\pid, \hgamma, \hsigma_2, \bsq, \bsq, \Skip) \Mid \hConfig_2)\end{array}}
	\caption{\vanillaC\ rule If Else True.}	
	\label{Fig: if else \vanillaC true}
\end{subfigure}	
\end{tabular}
\caption{\Code{if else} branching on private data example (\ref{Fig: if else \piccoC code}, \ref{Fig: if else \piccoC expanded}) matching to the \piccoC\ variable-tracking (\ref{Fig: iep vt}) and \vanillaC\ (\ref{Fig: if else \vanillaC true}) semantic rules. Coloring in the rules highlight the corresponding code and rule execution.}
\label{Fig: if else color}
\end{figure*}







































%


\paragraph{Conditional Code Block Variable Tracking}
For this style of tracking, we first evaluate expression $\Expr$ over environment $\gamma$, memory $\sigma$ and accumulator $\Acc$ 
to obtain some number $n$; the same environment, 
and a potentially updated memory (e.g. in the case $\Expr = x++$). 
% green / extract _vars
We then extract the non-local variables that are modified within either branch, and check whether multi-level modifications or array writes at a public index occur. This is achieved with Algorithm $\DynExtract$ by iterating through both statement $\stmt_1$ and $\stmt_2$ and storing the variable names in list $\vl_{\Acc+1}$, as well as updating and returning a tag to indicate whether we have found multi-level modifications (0 for false, 1 for true).
%
% red / initialize_vars
Next we call Algorithm $\Initialize$, which 
stores $n$ as the value of a temporary variable $\res_{\Acc+1}$, using $\Acc +1$ to denote the current level of nesting in the upcoming \Code{then} and \Code{else} statements. The variable $\res_{\Acc+1}$  is later used in the resolution phase, to select the result according to the branching condition. 
It then iterates through the list of variables, creating two temporary versions of each variable, named $\x\_{then\_\Acc}$ and $\x\_{else\_\Acc}$, and storing each in memory with the initial value of what $\x$ has in the memory $\sigma_1$. 
%
% light blue / stmt 1
Next is the evaluation of the \TT{then} statement, and afterwards 
%
%
% yellow / restore_vars
we must restore the original memory. %, such that $\sigma_1\subset\sigma_5$. 
To do this, we call $\Restore$, which iterates through each of the variables $\x$ contained within $\vl_{\Acc+1}$, storing their current value into their \Code{then} temporary (i.e., $\x\_\mathit{then}_{\Acc+1} = \x$) and restoring their original value from their \Code{else} temporary (i.e., $\x = \x\_\mathit{else}_{\Acc+1}$). 
Once we have completed this, 
%
% light purple / stmt 2
 the evaluation of the \TT{else} statement can occur. 
 
 
% orange / resolve_vars
Finally, we need to perform the resolution of all changes made to variables in either branch. 
To enable this, we call Algorithm $\ResolveR$ to iterate through each of the variables within $\vl_{\Acc+1}$ and grab their values accordingly, as well as retrieving the result of the private condition (whose value we stored in $\res_{\Acc+1}$). 
We then use multiparty protocol $\MPC{resolve}$ to facilitate the resolution of the true values, as these computations require communication between parties. 
For variables that are not array or pointer variables (e.g., those in~\ref{Fig: if else \piccoC code}), we perform a series of binary operations over the byte values of the private variables as shown in~\ref{Fig: if else \piccoC expanded} (e.g., \Code{c=(res$\cdot$c\_t)+((1-res)$\cdot$c\_e)}). 
The process is similar for arrays, with some addition bookkeeping due to their structure as a const pointer referring to the location with the array data. 
For pointers, we must handle the different locations referred to from each branch, merging the two location lists and finding what the true location is. 
The resolved values are then returned, and Algorithm $\ResolveS$ stores all each back into memory for its respective variable. 
Notice that, in the conclusion, we revert to the original environment $\gamma$. In this way, all the temporary variables we used become out of scope and are discarded - in particular, this prevents reusing the same temporary variable mapping if we have multiple (not nested) private if else statements.





\begin{figure*} \footnotesize
\begin{tabular}{l}
\begin{tabular}{l l}
\hspace{0.3cm}
\begin{subfigure}{.27\textwidth}
\begin{lstlisting}
private int a=3,
		b=7,c=5,*p=&a;
if ($\Code{\ExprC{a>b}}$)
	$\Code{\sC{*p=c;}}$ 
else 
	$\Code{\ssC{p=\&b;}}$
\end{lstlisting}
	\caption{\piccoC\ code.}
	\label{Fig: if else \piccoC code dp}	
\end{subfigure}		
&
\begin{subfigure}{0.69\textwidth}
\begin{lstlisting}[emph={[2]res, resolve}, emphstyle={[2]\color{blue}}]
private int a=3,b=7,c=5,*p=&a,$\Code{\initC{res=}\ExprC{a>b}}$;
$\initC{\DMap[\Acc][\loc_\TT{p}]=([1,[(\loc_\TT{a}, 0)],[1],1],[],0, \Code{private int*});}$  
$\sC{\DMap[\Acc][\loc_\TT{a}]=(3,0,0,\Code{private int});}$   $\sC{\Code{*p=c;}}$ 
$\restC{\DMap[\Acc][\loc_\TT{p}]=([1,[(\loc_\TT{a}, 0)],[1],1],[1,[(\loc_\TT{a}, 0)],[1],1],1, \Code{private int*});}$
$\restC{\DMap[\Acc][\loc_\TT{a}]=(3,5,1, \Code{private int});}$ $\restC{\loc_\TT{p}=\DMap[\Acc][\loc_\TT{p}][0];}$  $\restC{\loc_\TT{a}=\DMap[\Acc][\loc_\TT{a}][0];}$
$\Code{\ssC{p=\&b;}}$
$\resoC{\loc_\TT{p}=\Code{resolve(res,}\DMap[\Acc][\loc_\TT{p}],\loc_\TT{p});}$   $\resoC{\loc_\TT{a}=\Code{resolve(res,}\DMap[\Acc][\loc_\TT{a}],\loc_\TT{a});}$
\end{lstlisting}
	\caption{Location-tracking execution.}	
	\label{Fig: if else piccoC expanded dp}
\end{subfigure} 
\end{tabular}
\\ \\
\begin{subfigure}{\textwidth}
\inferrule{\begin{array}{l} 
		\qquad
			\ExprC{((\pidA, \gamma^\pidA, \sigma^\pidA, \DMap^\pidA, \Acc, \Expr) \ \ \Mid ... \Mid
				 	 (\pidZ, \gamma^\pidZ, \sigma^\pidZ, \DMap^\pidZ, \Acc, \Expr))} 
			\crcr\ExprC{\Deval{\locLL_1}{\codeLL_1} 
			((\pidA, \gamma^\pidA, \sigma^\pidA_1, \DMap^\pidA_1, \Acc, n^\pidA) \Mid ... \Mid
			  (\pidZ, \gamma^\pidZ, \sigma^\pidZ_1, \DMap^\pidZ_1, \Acc, n^\pidZ))}
		\qq 
			\{(\ExprC\Expr) \isPriv \gamma^\pid\}^\pidZ_{\pid = \pidA}
		\crcr 
			\extC{\{\DynExtract(}\sC{\stmt_1}\extC{,}\ \ssC{\stmt_2} \extC{, \gamma^\pid) = (\x_\vl, 1)\}^\pidZ_{\pid = \pidA}}
		\crcr
			\initC{\{\DynInit(\DMap^\pid_1,}\ \extC{\x_\vl}\initC{,\ \gamma^\pid,\ \sigma^\pid_1,\ \ExprC{\n^\pid}, \AccPP) = (\gamma^\pid_1, \sigma^\pid_2, \DMap^\pid_2, \locL^\pid_2)\}^\pidZ_{\pid = \pidA}}
		\crcr\qquad 
			\sC{((\pidA, \gamma^\pidA_1, \sigma^\pidA_2, \DMap^\pidA_2, \AccPP, \stmt_1) \ \ \ \Mid  ... \Mid
				  (\pidZ, \gamma^\pidZ_1, \sigma^\pidZ_2, \DMap^\pidZ_2, \AccPP, \stmt_1))}
				\crcr\sC{\Deval{\locLL_3}{\codeLL_2} 
				((\pidA, \gamma^\pidA_2, \sigma^\pidA_3, \DMap^\pidA_3, \AccPP, \Skip) \Mid  ... \Mid
				 (\pidZ, \gamma^\pidZ_2, \sigma^\pidZ_3, \DMap^\pidZ_3, \AccPP, \Skip))}
		\crcr
			\restC{\{\DynRestore(\sigma^\pid_3, \DMap^\pid_3, \AccPP) = (\sigma^\pid_4, \DMap^\pid_4, \locL^\pid_4)\}^\pidZ_{\pid = \pidA}}
		\crcr\qquad
			\ssC{((\pidA, \gamma^\pidA_1, \sigma^\pidA_4, \DMap^\pidA_4, \AccPP, \stmt_2) \ \ \ \Mid  ... \Mid
				    (\pidZ, \gamma^\pidZ_1, \sigma^\pidZ_4, \DMap^\pidZ_4, \AccPP, \stmt_2))}
				\crcr\ssC{\Deval{\locLL_5}{\codeLL_3} 
				((\pidA, \gamma^\pidA_3, \sigma^\pidA_5, \DMap^\pidA_5, \AccPP, \Skip) \Mid  ... \Mid
				  (\pidZ, \gamma^\pidZ_3, \sigma^\pidZ_5, \DMap^\pidZ_5, \AccPP, \Skip))}
		\crcr
			\resoC{\{\DynResolve\_\mathrm{Retrieve}(\gamma^\pid_1, \sigma^\pid_5, \DMap^\pid_5, \AccPP) 
				= ([(\val^\pid_{t1}, \val^\pid_{e1}), ..., (\val^\pid_{tm}, \val^\pid_{em})], 
					\ExprC{\n^\pid}, \locL^\pid_6)\}^\pidZ_{\pid = \pidA}}
		\crcr
			\resoC{\MPC{resolve}([\ExprC{\n^\pidA}, ..., \ExprC{\n^\pidZ}], 
				[[(\val^\pidA_{t1}, \val^\pidA_{e1}), ..., (\val^\pidA_{tm}, \val^\pidA_{em})]], ..., 
				 [(\val^\pidZ_{t1}, \val^\pidZ_{e1}), ..., (\val^\pidZ_{tm}, \val^\pidZ_{em})]])} 
				\crcr\qquad\resoC{= [[\val^\pidA_1, ..., \val^\pidA_m], ... [\val^\pidZ_1, ..., \val^\pidZ_m]]}
		\crcr
			\resoC{\{\DynResolve\_\mathrm{Store}(\DMap^\pid_5, \sigma^\pid_5, \AccPP, [\val^\pid_1, ..., \val^\pid_m]) 
				= (\sigma^\pid_6, \DMap^\pid_6, \locL^\pid_7)\}^\pidZ_{\pid = \pidA}}
		\crcr
			\locLL_6 = \locLL_1 \addL (\pidA, \locL^\pidA_2) \Mid ... \Mid (\pidZ, \locL^\pidZ_2) \addL \locLL_3 
						\addL (\pidA, \locL^\pidA_4) \Mid ... \Mid (\pidZ, \locL^\pidZ_4) \addL \locLL_5
						\addL (\pidA, \locL^\pidA_6) \Mid ... \Mid (\pidZ, \locL^\pidZ_6) 
		\crcr 
			\locLL_7 = \locLL_6 \addL (\pidA, \locL^\pidA_7) \Mid ... \Mid (\pidZ, \locL^\pidZ_7)
		\qq 
			\codeLL_4 = \codeLL_1 \addC \codeLL_2 \addC \codeLL_3
	\end{array} }
	{\begin{array}{l} 
	((\pidA, \gamma^\pidA, \sigma^\pidA, \DMap^\pidA, \Acc, \If\ (\ExprC\Expr)\ \sC{\stmt_1}\ \Else\ \ssC{\stmt_2}) 
	 \Mid ... \Mid 
	 (\pidZ, \gamma^\pidZ, \sigma^\pidZ, \DMap^\pidZ, \Acc, \If\ (\ExprC\Expr)\ \sC{\stmt_1}\ \Else\ \ssC{\stmt_2})) 
		 \Deval{\locLL_7}{\codeLL_4 \addC \codeMP{iepd}} 
		 \crcr ((\pidA, \gamma^\pidA, \resoC{\sigma^\pidA_{6}}, \resoC{\DMap^\pidA_{6}}, \Acc, \Skip) 
		 	 \qq \-\ \-\ \Mid ... \Mid
		  	(\pidZ, \gamma^\pidZ, \resoC{\sigma^\pidZ_{6}}, \resoC{\DMap^\pidZ_{6}}, \Acc, \Skip))
		\end{array}}
	\caption{\piccoC\ rule Private If Else - Location Tracking.}
	\label{Fig: iep lt}
\end{subfigure}
\\ \\
\begin{subfigure}{\textwidth}
Multiparty If Else False  		\\
\inferrule{\begin{array}{l}
	\qquad 
		\ExprC{((\pidA, \hgamma,\ \hsigma,\ \bsq, \bsq, \hExpr)\ \ \ \Mid ... \Mid
		  (\pidZ, \hgamma,\ \hsigma,\ \bsq, \bsq, \hExpr))} 
	 \crcr \ExprC{\Veval_{\codeVLL_1} 
			((\pidA, \hgamma,\ \hsigma_1, \bsq, \bsq, \hat{n})\ \ \ \Mid ... \Mid
			  (\pidZ, \hgamma,\ \hsigma_1, \bsq, \bsq, \hat{n}))}
	\qq \ExprC\hn = 0
	\crcr\qquad \sC{((\pidA, \hgamma,\ \hsigma_1, \bsq, \bsq, \hstmt_1)\ \ \Mid ... \Mid
			   (\pidZ, \hgamma,\ \hsigma_1, \bsq, \bsq, \hstmt_1))} 
	\crcr \sC{\Veval_{\codeVLL_2} 
			((\pidA, \hgamma_1, \hsigma_2, \bsq, \bsq, \Skip) \Mid ... \Mid
			 (\pidZ, \hgamma_1, \hsigma_2, \bsq, \bsq, \Skip))}
	\crcr\qquad \ssC{((\pidA, \hgamma,\ \hsigma_1, \bsq, \bsq, \hstmt_2)\ \ \Mid ... \Mid
			   (\pidZ, \hgamma,\ \hsigma_1, \bsq, \bsq, \hstmt_2))} 
	\crcr \ssC{\Veval_{\codeVLL_3} 
			((\pidA, \hgamma_2, \hsigma_3, \bsq, \bsq, \Skip) \Mid ... \Mid
			 (\pidZ, \hgamma_2, \hsigma_3, \bsq, \bsq, \Skip))}
	\end{array}}
	{\begin{array}{l}
	((\pidA, \hgamma, \hsigma,\ \bsq, \bsq, \If (\hExpr)\ \hstmt_1\ \Else\ \hstmt_2)\Mid ... \Mid
	  (\pidZ, \hgamma, \hsigma,\ \bsq, \bsq, \If (\hExpr)\ \hstmt_1\ \Else\ \hstmt_2)) 
		\Veval_{\codeVLL_1\addC\codeVLL_2\addC\codeVLL_3\addC[\codeVS{mpief}]} 
		\crcr
		((\pidA, \hgamma, \ssC{\hsigma_3}, \bsq, \bsq, \Skip)
		\qq\ \Mid ... \Mid
		 (\pidZ, \hgamma, \ssC{\hsigma_3}, \bsq, \bsq, \Skip))
		 \end{array}}
	\caption{\vanillaC\ rule Multiparty If Else False.}
	\label{Fig: van mpief}
\end{subfigure}
\end{tabular}
\caption{\Code{if else} branching on private data example (\ref{Fig: if else \piccoC code dp}, \ref{Fig: if else piccoC expanded dp}), \piccoC\ location-tracking (\ref{Fig: iep lt}), and \vanillaC\ Multiparty If Else False (\ref{Fig: van mpief}) rules. Coloring in the rules highlight the corresponding code and rule execution.}
%
\label{Fig: if else color dp}
\end{figure*}

































\paragraph{Conditional Code Block Location Tracking}
\label{sec: priv if lt desc}
Here we track modifications during private-conditioned branches at the level of memory blocks and offsets, which ensures that we do not update any data in memory inaccurately, as is shown in Figure~\ref{Fig: simple pointer challenge ex} using variable tracking SMC techniques.  
To facilitate this, we use the mapping structure $\DMap$ to track changes to each location at each level of nesting. This structure maps locations to a four-tuple of the original value, the \TT{then} branch value, a tag to notate whether the \TT{then} branch value was updated during the restoration phase, and the type of value stored (i.e., $(\loc, \offset) \to$ $(\val_1, \val_2$, $\tagb$, $\Type$)). The tag is used to allow us to add to $\DMap$ as we encounter pointer dereference writes and array writes at public indices without needing to track which branch we are in. It is always initialized as 0, and updated to 1 when we enter the restoration phase and store a value into the \TT{then} position. This way, if a location was added in the \TT{else} branch (i.e., was not modified in the \TT{then} branch), we know to use the original value as the \TT{then} value when we resolve the true value of that location at the end. 

The overall structure of the location tracking rule is similar to the variable tracking rule. 
We first evaluate $\Expr$ to $\n$, then call $\DynExtract$ to find variables that are modified during the execution of either branch and that there are multi-level modifications within at least one branch. 
We then call $\DynInit$, which stores the result of the private conditional and uses the variables we found to create the initial mapping $\DMap$.  
Next, we proceed to evaluate the \TT{then} branch, and call $\DynRestore$ to update $\DMap$ with the ending \TT{then} values for all locations that are tracked and restore the original values back into memory. 
After, we evaluate the \TT{else} branch and, once complete, call $\DynResolveR$ to retrieve the result of the conditional and the \TT{then} and \TT{else} values for each location. 
As with variable tracking, we use multiparty protocol $\MPC{resolve}$ to obtain the true values, and then store them back into their respective locations using Algorithm $\DynResolveS$. 
%
It is important to note that when we evaluate a pointer dereference write or array write at a public index inside a branch, we check to see if the given location is in $\DMap[\Acc]$.  If it is not, we add a mapping to store the original data (i.e., $(\loc, \offset) \to$ (\TT{orig}, $\Null$, $0$, $\Type$)). Notice that the data can either be a regular value (i.e., for a memory block storing a private int) or a pointer data structure representing a private pointer (i.e., for a memory block storing a private int*). 

In Figure~\ref{Fig: if else piccoC expanded dp}, we show an approximation of the execution of the pointer challenge example shown in Figure~\ref{Fig: simple pointer challenge ex}. 
When we reach the private-conditioned branching statement, we first store the result of the condition \TT{a < b}. As we execute the \TT{then} branch, we add the entry for $\loc_\TT{a}$ to $\DMap$, as \TT{p} refers to \TT{a}. We restore between branches by resetting $\loc_\TT{a}$ to its original value stored in $\DMap[\loc_\TT{a}][0]$. As we execute the \TT{else} branch, we add the entry for $\loc_\TT{p}$ to $\DMap$, as we are modifying which location \TT{p} points to. Finally, we resolve the true values for each modified location in $\DMap$. This approach eliminates the issues shown in Figure~\ref{Fig: simple pointer challenge ex}, as we do not rely on the pointer's current location to appropriately resolve the true values. 










\subsection{Overshooting Memory Bounds} \label{Sec: Overshooting}
%
It is possible to overshoot memory bounds in both \vanillaC\ and \piccoC. 
When overshooting occurs, we read the bytes of data as the type we expected it to be (i.e., bytes containing private data accessed by a public variable would be decoded as though they are public - no encryption or decryption occurs, but computations using the variable beyond that point will be garbage). 
This ensures that no information about private data can be leaked when overshooting. 
This is discussed further in Appendix~\ref{app: array oob} for the interested reader. 
We can only prove correctness over well-aligned accesses (i.e., those that iterate only over aligned elements of the same type, as with one array spilling into a subsequent array), as these would still produce readable data that is not garbage. 
When proving noninterference, we must prove that these cases (particularly those involving private data) cannot leak any information about the private data that is affected. We discuss this in more detail in the following section.








