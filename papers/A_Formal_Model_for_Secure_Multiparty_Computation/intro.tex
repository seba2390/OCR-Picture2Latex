

% 1st para - Overview SMC
Secure Multiparty Computation (SMC) allows multiple parties to jointly compute over private data, revealing only the outcomes of the computation to designated recipients. Secure computation is needed in many domains, particularly the medical, military, and financial sectors. 
SMC is commonly implemented using low-level techniques like secret sharing~\cite{Shamir79}, garbled circuits~\cite{Yao86}, or homomorphic encryption~\cite{ElGamal85,Paillier99}. 
These low-level techniques are designed to enable computations among parties which are secure and efficient. While these low-level techniques provide efficiency, their use makes programming SMC applications challenging and error prone. 
To address this concern, several works have proposed high-level languages, DSLs, or language extensions providing abstractions, which can then be compiled down to low-level SMC primitives, to support programmers in writing SMC applications. 
As a result, there now exists a plethora of languages providing different expressivity, offering different features and performance trade-offs, using different threat models, and suitable for different domains.
 Similarly, there exist a number of SMC DSLs.  Although DSLs can make it easier to write SMC programs, there remains a disconnect between the DSL and its integration with the language of the underlying implementation of the compiled protocol. 
 %It is left to the programmer to sufficiently learn the nuances of the new DSL, and to the compiler developers to manage differences in semantics between the DSL and the programming language used to implement and compose the underlying SMC techniques (usually written in C). 

In the effort to unify our knowledge in this space, a recent work~\cite{HastingsHNZ19} compared several compilers and tools in terms of their expressivity and usability. We highlight two items among the lessons learned and recommendations to the community. First, there were numerous correctness issues and undocumented limitations present in the works surveyed. This finding is also echoed in~\cite{Mood16}, which found correctness issues in several two-party compilers. Second, the authors of~\cite{HastingsHNZ19} recommend that the community take a more principled approach to language design and verification, e.g., by defining and implementing type rules. This would help with ensuring correctness as well as reduce security-related corner cases overlooked by the compiler designers. 

To help achieves these goals, in this paper, we present a formalism of a general-purpose SMC system designed for C. We choose C because it provides the low level language framework targeted by most DSLs and there exist numerous direct language extensions for which multiple SMC compilers have been developed. This allows programmers to write secure multiparty programs in C, which the compiler will translate into secure computation protocols, avoiding managing the interactions between different languages. Given the maturity of SMC compilers today, modern implementations provide support for all C features, allowing private-conditioned branches (i.e., \texttt{if-else} statements whose conditional expressions use private data), use of private arrays, private indexing into arrays, and working with pointers to private data; all while ensuring that no private data is leaked over the execution of any given program. However, formally modeling semantics and translation of these features as done by SMC compilers presents non-trivial challenges not attempted in any prior work. Furthermore, what is interesting about C (and not present in the available well-typed DSLs) is that features such as pointer manipulation allow one to write programs that erroneously access unintended regions in memory. However, even in those circumstances, it is possible to show that the compiled protocol will not reveal any unintended information about private data that it handles.


Our contributions in this paper are:
\begin{enumerate}%[leftmargin=4mm,topsep=1mm,itemsep=.5mm]
\item a formal model for a general-purpose secure multiparty
  computation compiler, formalizing state of the art SMC techniques in 
  C. Our formal model supports distributed multi-party computation in the
  presence of private-conditioned branches, pointers to private data, 
  pointer arithmetic and general pointer operations in  private-conditioned branches. 
\item a formal proof based on this model that common SMC approaches
  guarantee correctness and a strong form of non-interference over execution
  traces consisting of multiple computing parties.  This shows that pointer
  operations can be safely managed with no restrictions on the program. 
 \item an implementation of our formal model in the PICCO SMC compiler and evaluation over micro-benchmarks and SMC programs.
\end{enumerate}
%In this, we have found there are still some aspects (e.g. indirectly modifying data inside branches based on private data) that are valid C code, but not accepted by PICCO currently. We aim to increase compatibility of PICCO with the entire C language.

We present an overview of related work in the next section. Section~\ref{sec:background}  provides an overview of SMC, specifically highlighting features present in C extensions that support SMC, it presents SMC compiler background, and a motivating examples.
Section~\ref{Sec: Semantics} defines our semantics, introducing a model that ensures correctness and non-interference for SMC programs with full support for pointers.  Section~\ref{Sec: Metatheory} formalizes our metatheory and provides proof sketches.  Section~\ref{Sec:implementation} details our implementation of our model in the PICCO SMC compiler and  discusses a bug in the private-conditioned branch implementation that our
implementation solved.  Section~\ref{Sec:evaluation} provides evaluation results of our implementation.  We conclude and provide remarks on future work in Section~\ref{sec:conclusions}. The appendices provide additional information including definitions, algorithms, theorems and proofs. Additional details, results, and proofs are available in~\cite{amys-dissertation}.




