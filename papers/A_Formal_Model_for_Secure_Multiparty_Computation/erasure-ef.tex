



Here, we show the full erasure function in Figure~\ref{Fig: app erasure}. 
This function is intended to take a \piccoC\ program or configuration and remove all private privacy labels, decrypt any private data, and clear any additional tracking features that are specific to \piccoC; this process will result in a \vanillaC\ program or configuration. 




\begin{figure*}
\footnotesize
\begin{minipage}{0.5\textwidth}
% STMT
\begin{subfigure}{\textwidth}
$\begin{array}{l}
\bm{\erasure(\rstmt)} =  \\ 
	\mid\ \RT{\x[\Expr]}\ =>\ \bm{\x[\erasure(\rExpr)]}  \\ 
	\mid\ \RT{\x(\Elist)}\ =>\ \bm{\x(\erasure(\RT{\Elist}))} \\ 
	\mid\ \RT{\Expr_1\ \binop\ \Expr_2}\ =>\ \bm{\erasure(\RT{\Expr_1})\ \binop\ \erasure(\RT{\Expr_2})}  \\ 
	\mid\ \RT{\unop\ \x}\ =>\ \bm{\unop\ \x}  \\ 
	\mid\ \RT{( \Expr )}\ =>\ \bm{(\erasure(\rExpr))}  \\ 
	\mid\ \RT{(\Type)\ \Expr}\ =>\ \bm{\erasure(\rType))\ \erasure(\rExpr)} \\ 
	\mid\ \RT{\var = \Expr}\ =>\ \bm{\erasure(\RT\var) = \erasure(\rExpr)}  \\ 
	\mid\ \RT{*\x = \Expr}\ =>\ \bm{*\x = \erasure(\rExpr)}  \\ 
	\mid\ \RT{\stmt_1;\ \stmt_2}\ =>\ \bm{\erasure(\RT{\stmt_1});\ \erasure(\RT{\stmt_2})}  \\ 
	\mid\ \RT{\{ \stmt \}}\ =>\ \bm{\{ \erasure(\rstmt) \}}  \\ 
	\mid\ \RT{\free(\Expr)}\ =>\ \bm{\free(\erasure(\rExpr))}  \\ 
	\mid\ \RT{\pfree(\Expr)}\ =>\ \bm{\free(\erasure(\rExpr))}  \\ 
	\mid\ \RT{\sizeof(\Type)}\ =>\ \bm{\sizeof(\erasure(\rType))}  \\ 
	\mid\ \RT{\Malloc(\Expr)}\ =>\ \bm{\Malloc(\erasure(\rExpr))}  \\ 
	\mid\ \RT{\PMalloc(\Expr,\ \Type)}\ =>\ \\ \ \quad \bm{\Malloc(\sizeof(\erasure(\rType)) \cdot \erasure(\rExpr))}  \\ 
	\mid\ \RT{\smcinput(\Elist)}\ =>\ \\ \ \quad \bm{\inputFun(\erasure(\RT\Elist))}  \\ 
	\mid\ \RT{\smcoutput(\Elist)}\ =>\ \\ \ \quad \bm{\outputFun(\erasure(\RT\Elist))} \\ 
	\mid\ \RT{[\val_0, ..., \val_n]} =>\ \\ \ \quad  \bm{[\erasure(\RT{\val_0}),\ \erasure(\RT{...}),\ \erasure(\RT{\val_n})]} \\ 
	\mid\ \RT{\Type\ \var}\ =>\ \\ \ \quad  \bm{\erasure(\rType)\ \erasure(\RT{\var})}  \\ 
	\mid\ \RT{\Type\ \var = \Expr}\ =>\ \\ \ \quad  \bm{\erasure(\rType)\ \erasure(\RT{\var}) = \erasure(\rExpr)}  \\ 
	\mid\ \RT{\Type\ \x(\plist)}\ =>\ \\ \ \quad  \bm{\erasure(\rType)\ \x(\erasure(\RT{\plist}))}  \\ 
	\mid\ \RT{\Type\ \x (\plist)\ \{ \stmt\}}\ =>\ \\ \ \quad  \bm{\erasure(\RT{\Type\ \x (\plist)})\ \{ \erasure(\rstmt) \} }  \\ 
	\mid\ \RT{\If (\Expr)\ \stmt_1\ \Else\ \stmt_2}\ =>\ \\ \ \quad \bm{\If (\erasure(\rExpr))\ \erasure(\RT{\stmt_1})\ \Else\ \erasure(\RT{\stmt_2})}  \\ 
	\mid\ \RT{\While\ (\Expr)\ \stmt}\ \\ \ \ \ =>\ \bm{\While\ (\erasure(\rExpr))\ \erasure(\rstmt)}  \\ 
	\mid\ \RT\_\ =>\ \bm{\stmt}
\end{array}$
\caption{Erasure function over statements} 	\label{Fig: erasure stmt}
\end{subfigure}
\end{minipage}
%
\begin{minipage}{0.4\textwidth}
% CONFIG
\begin{subfigure}{\textwidth}
$\begin{array}{l}
\bm{\erasure(\RT\Config)} = 
	\\ \mid \RT{\Config_1} \Mid \RT{\Config_2}\ =>\ \bm{\erasure(\RT{\Config_1})} \Mid \bm{\erasure(\RT{\Config_2})}
	\\ \mid (\RT\pid, \rrgamma,\ \rrsigma,\ \RT{\DMap}, \rAcc,\ \rstmt) \ =>\
		\\ \ \ \ \bm{(\pid, \erasure(\rrgamma, \rrsigma, [\ ], [\ ]), \bsq, \bsq, \erasure(\rstmt))}
\end{array}$
\caption{Erasure function over configurations} 	\label{Fig: erasure config}
\end{subfigure}
\\ \\ \\
% TYPES  /  LISTS
\begin{subfigure}{\textwidth}
$\begin{array}{l}
\bm{\erasure(\rType)} =  \\  	
	\mid\ \rlabel\ \rbtype\ =>\ \bm{\btype}  \\ 
	\mid\ \RT{\rlabel\ \rbtype\ *}\ =>\ \bm{\btype*}  \\ 
	\mid\ \RT{\Tlist \to \rType}\ =>\ \\ \ \quad \bm{\erasure(\RT{\Tlist}) \to \erasure(\rType))}  \\ 
	\mid\ \RT\_\ => \bm{\Type} \\ \\
\bm{\erasure(\rTlist)} =  \\ 
	\mid\ \RT{[\ ]}\ =>\ \bm{[\ ]}  \\ 
	\mid\ \RT{\Type::\Tlist}\ =>\ \bm{\erasure(\RT{\Type})::\erasure(\RT{\Tlist})} 
\end{array}$
\caption{Erasure function over types and type lists} 	\label{Fig: erasure ty}
\end{subfigure}
\\ \\ \\
% LISTS
\begin{subfigure}{\textwidth} 
$\begin{array}{l}
\bm{\erasure(\rElist)} = \\ 
	\mid\ \RT{\Elist,\ \Expr}\ =>\ \bm{\erasure(\RT{\Elist}),\ \erasure(\rExpr)}  \\ 
	\mid\ \rExpr\ =>\ \bm{\erasure(\rExpr)}  \\ 
	\mid\ \rVoid\ =>\ \bm{\Void} \\  \\
\bm{\erasure(\rplist)} =  \\ 
	\mid\ \RT{\plist,\ \Type\ \var}\ =>\ \\ \ \quad \bm{\erasure(\RT{\plist}),\ \erasure(\RT{\Type\ \var})}  \\ 
	\mid\ \RT{\Type\ \var}\ =>\ \bm{\erasure(\rType)\ \erasure(\RT\var)}  \\ 
	\mid\ \rVoid\ =>\ \bm{\Void}
\end{array}$
\caption{Erasure function over lists} 	\label{Fig: erasure list}
\end{subfigure}
\end{minipage}
\\ \\ \\
% BYTES
\centering
\begin{subfigure}{0.8\textwidth}
$\begin{array}{l}
\bm{\erasure(\rbyte,\ \rType,\ \rn)} = \\ 
% public variable / array => no modifications necessary
	\mid\ (\rbyte,\ \rPub\ \rbtype,\ \rn) =>\ \bm{\byte} 	\\ 
% private variable => need to decrypt stored values
	\mid\ (\rbyte,\ \rPriv\ \rbtype,\ \RT1) =>\  \\ \-\ \quad
		\RT{\val_1 = \Decode(\Type,\ 1,\ \byte)};\ 
		\val_2 = \RT{\Decrypt(\val_1)};\ 
		\byte_1 = \Encode(\btype,\ \val_2);\ 
		\bm{\byte_1}	\\ 
% private array	=> need to decrypt stored values
	\mid\ (\rbyte,\ \rPriv\ \rbtype,\ \rn) =>\  
		\RT{\val_1 = \Decode(\Type,\ n,\ \byte)};\ \\ \ \quad
		[\val_1', ..., \val_n'] = \RT{[\Decrypt(\val_1), \Decrypt(...), \Decrypt(\val_n)]};\ 
		\byte_1 = \Encode(\btype,\ [\val_1',\ ...,\ \val_n']);\ 
		\bm{\byte_1} \\ 	
% public pointer, single loc => need to remove labels
	\mid\ (\rbyte,\ \RT{\Pub\ \btype\ *},\ \RT1) =>\ 
		[\RT1,\ [(\rloc, \RT\offset)],\ [\RT1],\ \RT\indir] = \RT{\DecodePtr(\Pub\ \btype\ *,\ 1,\ \byte)};\ \\ \-\ \quad
		\byte_1 = \EncodePtr(\btype\ *,\ [1,\ [(\loc, \offset)],\ [1],\ \erasure(\RT{\Type'}),\ \indir]);\ 
		\bm{\byte_1} \\ 
% private pointer, single loc => need to remove labels, reduce size of offset
	\mid\ (\rbyte,\ \RT{\Priv\ \btype\ *},\ \RT1) =>\ 
		[\RT1,\ [(\rloc, \RT\offset)],\ [\RT1],\ \RT\indir] = \RT{\DecodePtr(\Priv\ \btype\ *,\ 1,\ \byte)};\ \\ \-\ \quad
		\If (\RT\indir = 1) \Then \{ \RT{\Type_1} = \rPub\ \rbtype; \RT{\Type_2} = \rPriv\ \rbtype \} \Else \{ \RT{\Type_1} = \rPub\ \rbtype\RT*; \RT{\Type_2} = \rPriv\ \rbtype\RT* \}; \ \\ \-\ \quad 
		\offset_1 = \frac{\RT\offset \cdot \tau(\RT{\Type_1})}{\tau(\RT{\Type_2})};\  
		\byte_1 = \EncodePtr(\btype\ *,\ [1,\ [(\loc, \offset_1)],\ [1],\ \erasure(\RT{\Type'}),\ \indir]);\ 
		\bm{\byte_1} \\ 
% private pointer, multiple locs => need to remove labels, find true loc if multiple, reduce size of offset
	\mid\ (\rbyte,\ \RT{\Priv\ \btype\ *},\ \rn) =>\ 
		[\RT\nl,\ \RT\locL,\ \RT\tagbL,\ \RT\indir] = \RT{\DecodePtr(\Priv\ \btype\ *,\ n,\ \byte)};\  \\ \-\ \quad
		(\loc, \RT\offset) = \rrDeclassifyPtr([\RT\nl,\ \RT\locL,\ \RT\tagbL,\ \RT\indir],\ \RT{\rPriv\ \btype*});\ \\ \-\ \quad
		\If (\RT\indir = 1) \Then \{ \RT{\Type_1} = \rPub\ \rbtype; \RT{\Type_2} = \rPriv\ \rbtype \} \Else \{ \RT{\Type_1} = \rPub\ \rbtype\RT*; \RT{\Type_2} = \rPriv\ \rbtype\RT* \}; \ \\ \-\ \quad
		\offset_1 = \frac{\RT\offset \cdot \tau(\RT{\Type_1})}{\tau(\RT{\Type_2})};\  
		\byte_1 = \EncodePtr(\btype\ *,\ [1,\ [(\loc, \offset_1)],\ [1],\ \ \indir]);\ 
		\bm{\byte_1} \\ 
% function  => need to call erasure on stored param types and statement
	\mid\ (\rbyte,\ \RT{\Tlist \to \Type},\ \RT1) =>\  \\ \-\ \quad
		\RT{(\rstmt,\ \rn,\ \rplist) = \DecodeFun(\byte)};\ % \\ \-\ \qq
		\byte_1 = \EncodeFun(\erasure(\rstmt),\ \bsq,\ \erasure(\rplist));\ 
		\bm{\byte_1}
\end{array}$
\caption{Erasure function over bytes} 	\label{Fig: erasure bytes}
\end{subfigure}
\caption{The Erasure function, broken down into various functionalities. } 	\label{Fig: app erasure}
\end{figure*}




Figure~\ref{Fig: erasure config} shows erasure over an entire configuration, calling $\erasure$ on the four-tuple of the environment, memory, and two empty maps needed as the base for the \vanillaC\ environment and memory; removing the accumulator (i.e., replacing it with $\Box$); and calling $\erasure$ on the statement. 
%
%
Figure~\ref{Fig: erasure ty} shows erasure over types and type lists (i.e., for function types).
Here, we remove any privacy labels given to the types, with unlabeled types being returned as is. 
For function types, we must iterate over the entire list of types as well as the return type. 
%
%
Figure~\ref{Fig: erasure list} shows erasure over expression lists (i.e., from function calls) and parameter lists (i.e., from function definitions). 

Figure~\ref{Fig: erasure stmt} shows erasure over statements. 
For statements, we case over the various possible statements. 
When we reach a private value (i.e., $\Encrypt(n)$), we decrypt and then return the decrypted value. 
For function $\PMalloc$, we replace the function name with $\Malloc$, modifying the argument to appropriately evaluate the expected size of the type. 
For functions $\pfree$, $\smcinput$, and $\smcoutput$, we simply replace the function name with its \vanillaC\ equivalent. 
All other cases recursively call the erasure function as needed, with the last case $(\_)$ handling all cases that are already identical to the \vanillaC equivalent (i.e., $\Null$, locations). 

Figure~\ref{Fig: erasure bytes} shows erasure over bytes stored in memory, which is used from within the erasure on the environment and memory. 
This function takes the byte-wise data representation, the type that it should be interpreted as, and the size expected for the data. 
For regular public types, we do not need to modify the byte-wise data. 
For regular private types (i.e., single values and array data), we get back the value(s) from the representation, decrypt, and obtain the byte-wise data for the decrypted value(s). 
For pointers with a single location, we must get back the pointer data structure, then simply remove the privacy label from the type stored there. 
For private pointers with multiple locations, we must declassify the pointer, retrieving it's true location and returning the byte-wise data for the pointer data structure with only that location. 
For functions, we get back the function data, then call $\erasure$ on the function body, remove the tag for whether the function has public side effects (i.e., replace with $\Box$), and call $\erasure$ on the function parameter list. 





% ENV  /  MEM
\begin{figure*}\footnotesize
$\begin{array}{l}
\bm{\erasure(\rrgamma,\ \rrsigma, \hgamma, \hsigma)} = \\ 
	\mathrm{match}\ (\rrgamma, \rrsigma)\ \mathrm{with} \\ 
% empty, empty
	\mid\ (\RT{[\ ]}, \RT{[\ ]})\ =>\ \bm{(\hgamma,\ \hsigma)} \\ 
% empty, allocated pub
	\mid\ (\RT{[\ ]}, \RT{\sigma_1[\loc \to (\Null,\ \Void*,\ n,\ \PermL(\PermF,\ \Void*,\ \Pub,\ n))])}\ =>\ \\ \-\ \quad
		 \bm{(\erasure(\RT{[\ ]},\ \rsigma{_1},\ \hgamma,\ \hsigma[\loc \to (\Null,\ \Void*,\ \hn,} \
			\bm{\PermL(\perm,\ \Void*,\ \Pub,\ \hn))]))} 
	\\ 
% empty, allocated priv
	\mid\ (\RT{[\ ]}, \RT{\sigma_1[\loc \to (\Null,\ \RT{\Void*},\ n,\ \PermL(\PermF,\ \RT{\Type},\ \Priv,\ n))])}\ =>\ \\ \-\ \quad
		\hn = \Big(\frac{\rn}{\tau(\rType)} \Big) \cdot \tau(\erasure(\rType)) \\ \-\ \quad
		\bm{(\erasure(\RT{[\ ]},\ \rsigma{_1},\ \hgamma,\ \hsigma[\loc \to (\Null,\ \Void*,\ \hn,} \
			\bm{\PermL(\perm,\ \Void*,\ \Pub,\ \hn))]))} 
	\\ 
% empty, var/array
	\mid\ (\RT{[\ ]}, \RT{\sigma_1[\loc \to (\byte,\ \Type,\ n,\ \PermL(\perm,\ \Type,\ \llabel,\ n))])}\ =>\ \\ \-\ \quad
		\bm{(\erasure(\RT{[\ ]}, \rsigma{_1}, \hgamma, \hsigma[\loc \to (\erasure(\rbyte, \rType, \rn), \erasure(\rType), n,}
			\bm{\PermL(\perm, \erasure(\rType), \Pub, n))]))} 
	\\ 
% empty, ptr
	\mid\ (\RT{[\ ]}, \RT{\sigma_1[\loc \to (\byte,\ \Type,\ n,\ \PtrPermL(\perm,\ \Type,\ \llabel,\ n))])}\ =>\ \\ \ \ \ 
		\bm{(\erasure(\RT{[\ ]}, \rsigma{_1}, \hgamma, \hsigma[\loc \to (\erasure(\rbyte, \rType, \rn), \erasure(\rType), n,}
			\bm{\PtrPermL(\perm, \erasure(\rType), \Pub, n))]))}
	\\ 
% empty, func
	\mid\ (\RT{[\ ]},\RT{\sigma_1[\loc \to (\byte,\ \Type,\ 1,\ \FunPermL(\Pub))])}\ =>\ \\ \-\ \quad
		\bm{(\erasure(\RT{[\ ]},\ \rsigma{_1},\ \hgamma,\ \hsigma[\loc \to (\erasure(\rbyte, \rType, \RT1),\ \erasure(\rType),\ 1,}\
			\bm{\FunPermL(\Pub))]))}
	\\ 
% var, var
	\mid\ (\RT{\gamma_1[\x \to (\loc,\ \llabel\ \btype)]},\ \rsigma{_1}\RT{[\loc \to (\byte,\ \llabel\ \btype,\ 1,\ \PermL(\perm,\ \llabel\ \btype,\ \llabel,\ 1))]})\ =>\ \\ \-\ \quad
	 	\bm{(\erasure(\rgamma{_1}, \rsigma{_1}, \hgamma[\x \to (\loc, \btype)], \hsigma[\loc \to}
			\bm{(\erasure(\rbyte, \rlabel\ \rbtype, \RT1), \btype, 1,} % \\ \-\ \qq\qquad
			\bm{\PermL(\perm, \btype, \Pub, 1))]))}
	\\ 
% RES var, var
	\mid\ (\RT{\gamma_1[\res\_n \to (\loc,\ \Priv\ \btype)]},\ \rsigma{_1}\RT{[\loc \to (\byte,\ \Priv\ \btype,\ 1,}\ \RT{\PermL(\perm,\ \Priv\ \btype,\ \Priv,\ 1))]})\ =>\ 
		\\ \-\ \quad
		\bm{(\erasure(\rgamma{_1},\ \rsigma{_1},\ \hgamma,\ \hsigma))}
	\\ 
% THEN var, var
	\mid\ (\RT{\gamma_1[\x\_then\_n \to (\loc, \llabel\ \btype)]}, \rsigma{_1}\RT{[\loc \to (\byte, \llabel\ \btype, 1,} \RT{\PermL(\perm, \llabel\ \btype, \llabel, 1))]})\ =>\ \\ \-\ \quad
		%\\ \-\ \qquad
		\bm{(\erasure(\rgamma{_1}, \rsigma{_1}, \hgamma, \hsigma))}
	\\ 
% ELSE var, var
	\mid\ (\RT{\gamma_1[\x\_else\_n \to (\loc, \llabel\ \btype)]},\ \rsigma{_1}\RT{[\loc \to (\byte, \llabel\ \btype, 1,} \RT{\PermL(\perm, \llabel\ \btype, \llabel, 1))]})\ =>\ \\ \-\ \quad
	%\\ \-\ \qquad
		\bm{(\erasure(\rgamma{_1}, \rsigma{_1}, \hgamma, \hsigma))}
	\\ 
% array, array
	\mid\ (\RT{\gamma_1[\x \to (\loc,\ \llabel\ \Const\ \btype*)]},\ \rsigma{_1}\RT{[\loc \to (\byte,\ \llabel\ \Const\ \btype*,\ 1,}\ \RT{\PermL(\perm,\ \llabel\ \Const\ \btype*,\ \llabel,\ 1))]})\ =>\ 
	\\ \-\ \quad
		\RT{\DecodePtr(\llabel\ \Const\ \btype*, 1, \byte)} = \RT{[1,\ [(\loc_1, 0)],\ [1],\ 1]}; \\ \-\ \quad
			\rsigma{_1} = \rsigma{_2}\RT{[\loc_1 \to (\byte_1,\ \llabel\ \btype,\ n,\ \PermL(\perm,\ \llabel\ \btype,\ \llabel,\ n))]}; \\ \-\ \quad
		\bm{(\erasure(\rgamma{_1},\ \rsigma{_2},\ \hgamma[\x \to (\loc,\ \erasure(\llabel\ \Const\ \btype*))],}\ \\ \-\ \quad
			\bm{\hsigma[\loc \to (\erasure(\rbyte, \RT{\llabel\ \Const\ \btype*}, \RT1), \Const\ \btype*), 1,} 
				\bm{\PtrPermL(\perm,\ \Const\ \btype*,\ \Pub,\ 1))]} \
			\\ \-\ \quad
			\bm{[\loc_1 \to (\erasure(\RT{\byte_1}, \rlabel\ \rbtype, \rn), \btype, n,} %\\ \-\ \qq
				\bm{\PermL(\perm,\ \btype,\ \Pub,\ n))]))}
	\\ 
% THEN array, array
	\mid\ (\RT{\gamma_1[\x\_then\_n \to (\loc,\ \llabel\ \Const\ \btype*)]}, \rsigma{_1}\RT{[\loc \to (\byte, \llabel\ \Const\ \btype*, 1,} 
		\RT{\PtrPermL(\perm, \llabel\ \Const\ \btype*, \llabel, 1))]}) =>\ \\ \-\ \quad
		\RT{\DecodePtr(\llabel\ \Const\ \btype*, 1, \byte)} = \RT{[1,\ [(\loc_1, 0)],\ [1],\ 1]}; \\ \-\ \quad
			\rsigma{_1} = \rsigma{_2}\RT{[\loc_1 \to (\byte_1,\ \llabel\ \btype,\ n,\ \PermL(\perm,\ \llabel\ \btype,\ \llabel,\ n))]}; %\\ \-\ \qq
		\bm{(\erasure(\rgamma{_1},\ \rsigma{_2},\ \hgamma,\ \hsigma))}
	\\ 
% ELSE array, array
	\mid\ (\RT{\gamma_1[\x\_else\_n \to (\loc,\ \llabel\ \Const\ \btype*)]},\ \rsigma{_1}\RT{[\loc \to (\byte,\ \llabel\ \Const\ \btype*,\ 1,}\ 
		\RT{\PermL(\perm,\ \llabel\ \Const\ \btype*,\ \llabel,\ 1))]}) =>\ \\ \-\ \quad
		\RT{\DecodePtr(\llabel\ \Const\ \btype*, 1, \byte)} = \RT{[1,\ [(\loc_1, 0)],\ [1],\ 1]}; \\ \-\ \quad
			\rsigma{_1} = \rsigma{_2}\RT{[\loc_1 \to (\byte_1,\ \llabel\ \btype,\ n,\ \PermL(\perm,\ \llabel\ \btype,\ \llabel,\ n))]}; %\\ \-\ \qq
		\bm{(\erasure(\rgamma{_1},\ \rsigma{_2},\ \hgamma,\ \hsigma))}
	\\ 
% pointer, pointer
	\mid\ (\RT{\gamma_1[\x \to (\loc,\ \llabel\ \btype*)]},\ \rsigma{_1}\RT{[\loc \to (\byte,\ \llabel\ \btype*,\ n,}\ 
			\RT{\PtrPermL(\perm,\ \llabel\ \btype*,\ \llabel,\ n))]})\ =>\ \\ \-\ \quad
		 \bm{(\erasure(\rgamma{_1},\ \rsigma{_1},\ \hgamma[\x \to (\loc,\ \erasure(\llabel\ \btype*))],\ } \\ \-\ \quad
			\bm{\hsigma[\loc \to (\erasure(\rbyte, \rType, \rn),\ \erasure(\rType),\ n,}\ 
				\bm{\PtrPermL(\perm,\ \erasure(\rType),\ \Pub,\ n))]))}
	\\ 
% THEN pointer, pointer
	\mid\ (\RT{\gamma_1[\x\_then\_n \to (\loc,\ \llabel\ \btype*)]},\ \rsigma{_1}\RT{[\loc \to (\byte,\ \llabel\ \btype*,\ n,}\ 
			\RT{\PtrPermL(\perm,\ \llabel\ \btype*,\ \llabel,\ n))]})\ %
			=>\ \\ \-\ \quad
		\bm{(\erasure(\rgamma{_1},\ \rsigma{_1},\ \hgamma,\ \hsigma))}
	\\ 
% ELSE pointer, pointer
	\mid\ (\RT{\gamma_1[\x\_else\_n \to (\loc,\ \llabel\ \btype*)]},\ \rsigma{_1}\RT{[\loc \to (\byte,\ \llabel\ \btype*,\ n,}\ 
			\RT{\PtrPermL(\perm,\ \llabel\ \btype*,\ \llabel,\ n))]})\ %
			=>\ \\ \-\ \quad
		 \bm{(\erasure(\rgamma{_1},\ \rsigma{_1},\ \hgamma,\ \hsigma))}
	\\ 
% function, function
	\mid\ (\RT{\gamma_1[\x \to (\loc,\ \Tlist \to \Type)]}, \RT{\sigma_1[\loc \to (\byte,\ \Tlist \to \Type,\ 1,\ \FunPermL(\Pub))]}\ =>\ \\ \-\ \quad
		\bm{(\erasure(\rgamma{_1},\ \rsigma{_1},\ \hgamma[\x \to (\loc,\ \erasure(\RT{\Tlist \to \Type}))],}\ \\ \-\ \quad
			\bm{\hsigma[\loc \to (\erasure(\rbyte, \RT{\Tlist \to \Type}, \RT1),\ \erasure(\rTlist \to \rType),\ 1,\ \FunPermL(\Pub))]))}
\end{array}$
\caption{Erasure function over the environment and memory} 	\label{Fig: erasure env mem}
\end{figure*}







Figure~\ref{Fig: erasure env mem} shows erasure over the environment and memory. 
In order to properly handle all types of variables and data stored, we must iterate over both the \piccoC\ environment and memory maps, and pass along the \vanillaC\ environment and memory maps as we remove elements from the \piccoC\ maps and either add to them to the \vanillaC\ maps or discard them. 
The first case is the base case, when the \piccoC\ environment and memory are both empty, and we return the \vanillaC\ environment and memory. 
Next, we have three cases which continue to iterate through the \piccoC\ memory after the environment has been emptied. These cases are possible due to the fact that in \piccoC\ we remove mappings from the environment once they are out of scope, but we never remove mappings from memory. 

Then we have three cases to handle regular variables. The first adds mappings to the \vanillaC\ environment and memory without the privacy annotations on the types, and calls $\erasure$ on the byte-wise data stored at that location (the behavior of this is shown in Figure~\ref{Fig: erasure bytes} and described later in this section). The other two remove temporary variables (an their corresponding data) inserted by an \TT{if else} statement branching on private data. 
The cases for arrays, pointers, and functions behave similarly; however, when we have an array we handle the array pointer as well as the array data within those cases. 












