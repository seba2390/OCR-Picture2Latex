
The public \Code{if else} rules 
are nearly identical to the \vanillaC\ rules, (e.g., \vanillaC\ rule If Else True shown in Figure~\ref{Fig: if else \vanillaC true}), 
with the added assertion that the guard of the conditional is public (i.e., does not contain private data): $(\Expr) \isPub \gamma$.
%
The private \TT{if else} rules, shown in Figures~\ref{Fig: iep vt} and~\ref{Fig: iep lt}, are more interesting. Our strategy for dealing with private-conditioned branches involves executing both branches as a sequence of statements (with some additional helper algorithms to aid in storing changes, restoration between branches, and resolution of true values). 
We chose to use big-step semantics to facilitate the comparison of the \piccoC\ semantics with the \vanillaC\ semantics, and for its proof of correctness that we will discuss in the next Section. 
% general description
We give also an example of \piccoC\ code in Figure~\ref{Fig: if else \piccoC code} and~\ref{Fig: if else \piccoC code dp}, and the corresponding execution in Figure~\ref{Fig: if else \piccoC expanded} and~\ref{Fig: if else piccoC expanded dp}. We use coloring throughout Figure~\ref{Fig: if else color} and~\ref{Fig: if else color dp} to highlight the corresponding sections of code and rule execution. 

% SPECIFICS  --  IF ELSE
The starting and ending states of the \piccoC\ Private If Else rules are essentially the same as the starting and ending states of the corresponding \vanillaC\ If Else rule; however, there are several additional assertions that guarantee that both of the private-conditioned branches are executed. 
The assertions of these semantic rules are listed sequentially, from top to bottom.
We have two different styles of tracking modifications within conditional code blocks that are used within these rules: variable tracking and location tracking. 
Variable tracking is used when there are only single-level changes within the private-conditioned branches, whereas location tracking is used when we have multi-level changes (i.e., a branch contains a pointer dereference write) or potential out-of-bounds changes (i.e., array write at a public index). 

The main idea of both styles is to first store the original value of each variable that is modified within either branch; execute the \TT{then} branch; save the resulting values from the \TT{then} branch and restore all modified variables to their original values; execute the \TT{else} branch; and finally, to securely resolve which values should be kept -- those from the \TT{then} branch or those from the \TT{else} branch. 
In the variable style of tracking, we utilize temporary variables to keep track of all modifications made during either branch -- initializing the \TT{else} temporary with the original value, storing the result of the \TT{then} branch in the \TT{then} temporary and using the \TT{else} temporary to restore the original value, and finally using the result of the private-conditional and what is stored in each variable at the end of the \TT{else} branch as well as it's corresponding \TT{then} temporary to securely resolve what values to continue evaluating the program with. 

This style of tracking is robust enough for many uses, however, there are two notable exceptions where we run into issues, both involving the potential of the location we track not being the location that is actually modified. 
The first exception involves pointer dereference writes -- these alone are not an issue, but when location the pointer refers to is modified and we also perform pointer dereference writes, it becomes clear that variable tracking cannot easily find and handle these cases. 
The second exception involves array writes at public indices -- these become problematic due to the potential for writing out-of-bounds. As most array indices are not hard-coded, it isn't obvious that the write will be within bounds until execution, and to ensure we catch all of these cases we must use a more robust style of tracking to catch out-of-bounds writes. We stress here that array writes at private indices do not fall within this exception, as this operation will securely update the array within its bounds (as updating beyond the bounds of the array would leak that this private value is larger than the size of the array), and as such we can simply track the entire array properly using variable tracking.  
It is possible to ensure that we find all of the locations that are modified in both of these cases by dynamically adding these types of modifications as they are evaluated, which is the goal of the location tracking. 
In the location style of tracking, we still follow a similar evaluation pattern as with variable tracking, storing the original values for locations we know will be modified first, then restoring between branches, and resolving at the end. As we evaluate each branch and come upon one of these special cases, we will check to see if we have already marked that location for tracking, and if not we add that location and its original value before the modification occurs. 
It is worthwhile to stress again the role of the accumulator here with respect to other statements. We increment it when we evaluate the \TT{then} and \TT{else} statements, so that if we attempt to evaluate a (sub)statement with public side effects or restricted operations, we have an (oblivious) runtime failure. It also facilitates scoping of temporary variables within nested private-conditioned \TT{if else} statements.
We proceed to further describe the different assertions and specifics of both styles next. 


% insert IF ELSE figure


\begin{figure*} \footnotesize
\begin{tabular}{l}
\hspace{0.3cm}
\begin{tabular}{l l}
\begin{subfigure}{.36\textwidth}
\begin{lstlisting}
private int a=3,b=7,c=0;		
if ($\Code{\ExprC{a<b}}$) $\Code{\sC{c=a;}}$
else $\Code{\ssC{c=b;}}$
\end{lstlisting}
	\caption{\piccoC\ code.}
	\label{Fig: if else \piccoC code}	
\end{subfigure}		
&
\begin{subfigure}{.57\textwidth}
\begin{lstlisting}
private int a=3,b=7,c=0,$\Code{\initC{res=}\ExprC{a<b},\initC{c\_t=c,c\_e=c};}$
$\Code{\sC{c=a;}}$ $\Code{\restC{c\_t=c; c=c\_e;}}$
$\Code{\ssC{c=b;}}$ $\resoC{\Code{c=(res}\cdot\Code{c\_t)+((1-res)}\cdot\Code{c);}}$ 	
\end{lstlisting}
	\caption{Variable-tracking execution.}	
	\label{Fig: if else \piccoC expanded}
\end{subfigure} 
\end{tabular}
\\ \\
\begin{subfigure}{\textwidth}
	\inferrule{\begin{array}{l} 
		\qquad
			\ExprC{((\pidA, \gamma^\pidA, \sigma^\pidA, \DMap^\pidA, \Acc, \Expr) \ \ \Mid ... \Mid
					 (\pidZ, \gamma^\pidZ, \sigma^\pidZ, \DMap^\pidZ, \Acc, \Expr))} 
			\crcr\ExprC{\Deval{\locLL_1}{\codeLL_1} 
			((\pidA, \gamma^\pidA, \sigma^\pidA_1, \DMap^\pidA_1, \Acc, n^\pidA) \Mid ... \Mid
			 (\pidZ, \gamma^\pidZ, \sigma^\pidZ_1, \DMap^\pidZ_1, \Acc, n^\pidZ))}
		\qq
			\{(\ExprC\Expr) \isPriv \gamma^\pid\}^\pidZ_{\pid = \pidA}
		\crcr 
			\extC{\{\DynExtract(}\sC{\stmt_1}\extC{,}\ \ssC{\stmt_2} \extC{, \gamma^\pid) = (\x_\vl, 0)\}^\pidZ_{\pid = \pidA}}
		\crcr
			\initC{\{\Initialize(}\extC{\x_\vl}\initC{,\ \gamma^\pid, \sigma^\pid_1, \ExprC{\n^\pid},\ \AccPP) = (\gamma^\pid_1, \sigma^\pid_2, \locL^\pid_2)\}^\pidZ_{\pid = \pidA}}
		\crcr\qquad
			\sC{((\pidA, \gamma^\pidA_1, \sigma^\pidA_2, \DMap^\pidA_1, \AccPP, \stmt_1) \ \ \ \Mid ... \Mid
			 	  (\pidZ, \gamma^\pidZ_1, \sigma^\pidZ_2, \DMap^\pidZ_1, \AccPP, \stmt_1))}
			\crcr\sC{\Deval{\locLL_3}{\codeLL_2} 
				((\pidA, \gamma^\pidA_2, \sigma^\pidA_3, \DMap^\pidA_2, \AccPP, \Skip) \Mid ... \Mid
				 (\pidZ, \gamma^\pidZ_2, \sigma^\pidZ_3, \DMap^\pidZ_2, \AccPP, \Skip))}
		\crcr
			\restC{\{\Restore(}\extC{\x_\vl}\restC{,\ \gamma^\pid_1,\ \sigma^\pid_3,\ \AccPP) = (\sigma^\pid_4, \locL^\pid_4)\}^\pidZ_{\pid = \pidA}}
		\crcr\qquad
			\ssC{((\pidA, \gamma^\pidA_1, \sigma^\pidA_4, \DMap^\pidA_2, \AccPP, \stmt_2) \ \ \ \Mid ... \Mid
				   (\pidZ, \gamma^\pidZ_1, \sigma^\pidZ_4, \DMap^\pidZ_2, \AccPP, \stmt_2))}
				\crcr\ssC{\Deval{\locLL_5}{\codeLL_3} 
				((\pidA, \gamma^\pidA_3, \sigma^\pidA_5, \DMap^\pidA_3, \AccPP, \Skip) \Mid  ... \Mid
				 (\pidZ, \gamma^\pidZ_3, \sigma^\pidZ_5, \DMap^\pidZ_3, \AccPP, \Skip))}
		\crcr
			\resoC{\{\Resolve\_\mathrm{Retrieve}(\extC{\x_\vl},\ \AccPP, \gamma^\pid_1, \sigma^\pid_5) 
				= ([(\val^\pid_{t1}, \val^\pid_{e1}), ..., (\val^\pid_{tm}, \val^\pid_{em})], 
					\ExprC{\n^\pid}, \locL^\pid_6)\}^\pidZ_{\pid = \pidA}}
		\crcr
			\resoC{\MPC{resolve}([\ExprC{\n^\pidA}, ..., \ExprC{\n^\pidZ}], 
				[[(\val^\pidA_{t1}, \val^\pidA_{e1}), ..., (\val^\pidA_{tm}, \val^\pidA_{em})]], ..., 
				 [(\val^\pidZ_{t1}, \val^\pidZ_{e1}), ..., (\val^\pidZ_{tm}, \val^\pidZ_{em})]])} 
				\crcr\qquad\resoC{= [[\val^\pidA_1, ..., \val^\pidA_m], ... [\val^\pidZ_1, ..., \val^\pidZ_m]]}
		\crcr
			\resoC{\{\Resolve\_\mathrm{Store}(\extC{\x_\vl},\ \AccPP, \gamma^\pid_1, \sigma^\pid_5, 
				[\val^\pid_1, ..., \val^\pid_m]) = (\sigma^\pid_6, \locL^\pid_7)\}^\pidZ_{\pid = \pidA}}
		\crcr
			\locLL_6 = \locLL_1 \addL (\pidA, \locL^\pidA_2) \Mid ... \Mid (\pidZ, \locL^\pidZ_2) \addL \locLL_3 
						\addL (\pidA, \locL^\pidA_4) \Mid ... \Mid (\pidZ, \locL^\pidZ_4) \addL \locLL_5
						\addL (\pidA, \locL^\pidA_6) \Mid ... \Mid (\pidZ, \locL^\pidZ_6) 
		\crcr 
			\locLL_7 = \locLL_6 \addL (\pidA, \locL^\pidA_7) \Mid ... \Mid (\pidZ, \locL^\pidZ_7)
		\qq 
			\codeLL_4 = \codeLL_1 \addC \codeLL_2 \addC \codeLL_3
	\end{array} }
	{\begin{array}{l} 
	((\pidA, \gamma^\pidA, \sigma^\pidA, \DMap^\pidA, \Acc, \If\ (\ExprC\Expr)\ \sC{\stmt_1}\ \Else\ \ssC{\stmt_2}) 
	 \Mid ... \Mid 
	 (\pidZ, \gamma^\pidZ, \sigma^\pidZ, \DMap^\pidZ, \Acc, \If\ (\ExprC\Expr)\ \sC{\stmt_1}\ \Else\ \ssC{\stmt_2})) 
		\Deval{\locLL_7}{\codeLL_4 \addC \codeSP{iep}} 
		\crcr ((\pidA, \gamma^\pidA, \resoC{\sigma^\pidA_{6}}, \resoC{\DMap^\pidA_{3}}, \Acc, \Skip) 
		\qq \-\ \-\ \Mid ... \Mid 
		 (\pidZ, \gamma^\pidZ, \resoC{\sigma^\pidZ_{6}}, \resoC{\DMap^\pidZ_{3}}, \Acc, \Skip) )
		\end{array}}
	\caption{\piccoC\ rule Private If Else - Variable Tracking.}
	\label{Fig: iep vt}
\end{subfigure}
\\ \\
\begin{subfigure}{\textwidth}
	\inferrule{\begin{array}{l}
		\ExprC{((\pid, \hgamma, \hsigma,\ \bsq, \bsq, \hExpr)\ \ \Mid \hConfig)\ \ }
		\ExprC{\Veval_{\codeVLL_1}\ }
			\ExprC{((\pid, \hgamma,\ \hsigma_1, \bsq, \bsq, \hat{n})\ \ \ \Mid \hConfig_1)}
		\qq \ExprC{\hat{n}} \neq \ExprC{0}
		\crcr\sC{((\pid, \hgamma, \hsigma_1, \bsq, \bsq, \hstmt_1) \Mid \hConfig_1)\ }  
			\sC{\Veval_{\codeVLL_2}((\pid, \hgamma_1, \hsigma_2, \bsq, \bsq, \Skip)\Mid \hConfig_2)}
	\end{array}}
	{\begin{array}{l}
	((\pid, \hgamma, \hsigma, \bsq, \bsq, \If (\ExprC{\hExpr})\ \sC{\hstmt_1}\ \Else\ \ssC{\hstmt_2}) \Mid \hConfig)
		\Veval_{\codeVLL_1\addC\codeVLL_2\addC[\codeVS{iet}]} 
		((\pid, \hgamma, \hsigma_2, \bsq, \bsq, \Skip) \Mid \hConfig_2)\end{array}}
	\caption{\vanillaC\ rule If Else True.}	
	\label{Fig: if else \vanillaC true}
\end{subfigure}	
\end{tabular}
\caption{\Code{if else} branching on private data example (\ref{Fig: if else \piccoC code}, \ref{Fig: if else \piccoC expanded}) matching to the \piccoC\ variable-tracking (\ref{Fig: iep vt}) and \vanillaC\ (\ref{Fig: if else \vanillaC true}) semantic rules. Coloring in the rules highlight the corresponding code and rule execution.}
\label{Fig: if else color}
\end{figure*}







































%


\paragraph{Conditional Code Block Variable Tracking}
For this style of tracking, we first evaluate expression $\Expr$ over environment $\gamma$, memory $\sigma$ and accumulator $\Acc$ 
to obtain some number $n$; the same environment, 
and a potentially updated memory (e.g. in the case $\Expr = x++$). 
% green / extract _vars
We then extract the non-local variables that are modified within either branch, and check whether multi-level modifications or array writes at a public index occur. This is achieved with Algorithm $\DynExtract$ by iterating through both statement $\stmt_1$ and $\stmt_2$ and storing the variable names in list $\vl_{\Acc+1}$, as well as updating and returning a tag to indicate whether we have found multi-level modifications (0 for false, 1 for true).
%
% red / initialize_vars
Next we call Algorithm $\Initialize$, which 
stores $n$ as the value of a temporary variable $\res_{\Acc+1}$, using $\Acc +1$ to denote the current level of nesting in the upcoming \Code{then} and \Code{else} statements. The variable $\res_{\Acc+1}$  is later used in the resolution phase, to select the result according to the branching condition. 
It then iterates through the list of variables, creating two temporary versions of each variable, named $\x\_{then\_\Acc}$ and $\x\_{else\_\Acc}$, and storing each in memory with the initial value of what $\x$ has in the memory $\sigma_1$. 
%
% light blue / stmt 1
Next is the evaluation of the \TT{then} statement, and afterwards 
%
%
% yellow / restore_vars
we must restore the original memory. %, such that $\sigma_1\subset\sigma_5$. 
To do this, we call $\Restore$, which iterates through each of the variables $\x$ contained within $\vl_{\Acc+1}$, storing their current value into their \Code{then} temporary (i.e., $\x\_\mathit{then}_{\Acc+1} = \x$) and restoring their original value from their \Code{else} temporary (i.e., $\x = \x\_\mathit{else}_{\Acc+1}$). 
Once we have completed this, 
%
% light purple / stmt 2
 the evaluation of the \TT{else} statement can occur. 
 
 
% orange / resolve_vars
Finally, we need to perform the resolution of all changes made to variables in either branch. 
To enable this, we call Algorithm $\ResolveR$ to iterate through each of the variables within $\vl_{\Acc+1}$ and grab their values accordingly, as well as retrieving the result of the private condition (whose value we stored in $\res_{\Acc+1}$). 
We then use multiparty protocol $\MPC{resolve}$ to facilitate the resolution of the true values, as these computations require communication between parties. 
For variables that are not array or pointer variables (e.g., those in~\ref{Fig: if else \piccoC code}), we perform a series of binary operations over the byte values of the private variables as shown in~\ref{Fig: if else \piccoC expanded} (e.g., \Code{c=(res$\cdot$c\_t)+((1-res)$\cdot$c\_e)}). 
The process is similar for arrays, with some addition bookkeeping due to their structure as a const pointer referring to the location with the array data. 
For pointers, we must handle the different locations referred to from each branch, merging the two location lists and finding what the true location is. 
The resolved values are then returned, and Algorithm $\ResolveS$ stores all each back into memory for its respective variable. 
Notice that, in the conclusion, we revert to the original environment $\gamma$. In this way, all the temporary variables we used become out of scope and are discarded - in particular, this prevents reusing the same temporary variable mapping if we have multiple (not nested) private if else statements.





\begin{figure*} \footnotesize
\begin{tabular}{l}
\begin{tabular}{l l}
\hspace{0.3cm}
\begin{subfigure}{.27\textwidth}
\begin{lstlisting}
private int a=3,
		b=7,c=5,*p=&a;
if ($\Code{\ExprC{a>b}}$)
	$\Code{\sC{*p=c;}}$ 
else 
	$\Code{\ssC{p=\&b;}}$
\end{lstlisting}
	\caption{\piccoC\ code.}
	\label{Fig: if else \piccoC code dp}	
\end{subfigure}		
&
\begin{subfigure}{0.69\textwidth}
\begin{lstlisting}[emph={[2]res, resolve}, emphstyle={[2]\color{blue}}]
private int a=3,b=7,c=5,*p=&a,$\Code{\initC{res=}\ExprC{a>b}}$;
$\initC{\DMap[\Acc][\loc_\TT{p}]=([1,[(\loc_\TT{a}, 0)],[1],1],[],0, \Code{private int*});}$  
$\sC{\DMap[\Acc][\loc_\TT{a}]=(3,0,0,\Code{private int});}$   $\sC{\Code{*p=c;}}$ 
$\restC{\DMap[\Acc][\loc_\TT{p}]=([1,[(\loc_\TT{a}, 0)],[1],1],[1,[(\loc_\TT{a}, 0)],[1],1],1, \Code{private int*});}$
$\restC{\DMap[\Acc][\loc_\TT{a}]=(3,5,1, \Code{private int});}$ $\restC{\loc_\TT{p}=\DMap[\Acc][\loc_\TT{p}][0];}$  $\restC{\loc_\TT{a}=\DMap[\Acc][\loc_\TT{a}][0];}$
$\Code{\ssC{p=\&b;}}$
$\resoC{\loc_\TT{p}=\Code{resolve(res,}\DMap[\Acc][\loc_\TT{p}],\loc_\TT{p});}$   $\resoC{\loc_\TT{a}=\Code{resolve(res,}\DMap[\Acc][\loc_\TT{a}],\loc_\TT{a});}$
\end{lstlisting}
	\caption{Location-tracking execution.}	
	\label{Fig: if else piccoC expanded dp}
\end{subfigure} 
\end{tabular}
\\ \\
\begin{subfigure}{\textwidth}
\inferrule{\begin{array}{l} 
		\qquad
			\ExprC{((\pidA, \gamma^\pidA, \sigma^\pidA, \DMap^\pidA, \Acc, \Expr) \ \ \Mid ... \Mid
				 	 (\pidZ, \gamma^\pidZ, \sigma^\pidZ, \DMap^\pidZ, \Acc, \Expr))} 
			\crcr\ExprC{\Deval{\locLL_1}{\codeLL_1} 
			((\pidA, \gamma^\pidA, \sigma^\pidA_1, \DMap^\pidA_1, \Acc, n^\pidA) \Mid ... \Mid
			  (\pidZ, \gamma^\pidZ, \sigma^\pidZ_1, \DMap^\pidZ_1, \Acc, n^\pidZ))}
		\qq 
			\{(\ExprC\Expr) \isPriv \gamma^\pid\}^\pidZ_{\pid = \pidA}
		\crcr 
			\extC{\{\DynExtract(}\sC{\stmt_1}\extC{,}\ \ssC{\stmt_2} \extC{, \gamma^\pid) = (\x_\vl, 1)\}^\pidZ_{\pid = \pidA}}
		\crcr
			\initC{\{\DynInit(\DMap^\pid_1,}\ \extC{\x_\vl}\initC{,\ \gamma^\pid,\ \sigma^\pid_1,\ \ExprC{\n^\pid}, \AccPP) = (\gamma^\pid_1, \sigma^\pid_2, \DMap^\pid_2, \locL^\pid_2)\}^\pidZ_{\pid = \pidA}}
		\crcr\qquad 
			\sC{((\pidA, \gamma^\pidA_1, \sigma^\pidA_2, \DMap^\pidA_2, \AccPP, \stmt_1) \ \ \ \Mid  ... \Mid
				  (\pidZ, \gamma^\pidZ_1, \sigma^\pidZ_2, \DMap^\pidZ_2, \AccPP, \stmt_1))}
				\crcr\sC{\Deval{\locLL_3}{\codeLL_2} 
				((\pidA, \gamma^\pidA_2, \sigma^\pidA_3, \DMap^\pidA_3, \AccPP, \Skip) \Mid  ... \Mid
				 (\pidZ, \gamma^\pidZ_2, \sigma^\pidZ_3, \DMap^\pidZ_3, \AccPP, \Skip))}
		\crcr
			\restC{\{\DynRestore(\sigma^\pid_3, \DMap^\pid_3, \AccPP) = (\sigma^\pid_4, \DMap^\pid_4, \locL^\pid_4)\}^\pidZ_{\pid = \pidA}}
		\crcr\qquad
			\ssC{((\pidA, \gamma^\pidA_1, \sigma^\pidA_4, \DMap^\pidA_4, \AccPP, \stmt_2) \ \ \ \Mid  ... \Mid
				    (\pidZ, \gamma^\pidZ_1, \sigma^\pidZ_4, \DMap^\pidZ_4, \AccPP, \stmt_2))}
				\crcr\ssC{\Deval{\locLL_5}{\codeLL_3} 
				((\pidA, \gamma^\pidA_3, \sigma^\pidA_5, \DMap^\pidA_5, \AccPP, \Skip) \Mid  ... \Mid
				  (\pidZ, \gamma^\pidZ_3, \sigma^\pidZ_5, \DMap^\pidZ_5, \AccPP, \Skip))}
		\crcr
			\resoC{\{\DynResolve\_\mathrm{Retrieve}(\gamma^\pid_1, \sigma^\pid_5, \DMap^\pid_5, \AccPP) 
				= ([(\val^\pid_{t1}, \val^\pid_{e1}), ..., (\val^\pid_{tm}, \val^\pid_{em})], 
					\ExprC{\n^\pid}, \locL^\pid_6)\}^\pidZ_{\pid = \pidA}}
		\crcr
			\resoC{\MPC{resolve}([\ExprC{\n^\pidA}, ..., \ExprC{\n^\pidZ}], 
				[[(\val^\pidA_{t1}, \val^\pidA_{e1}), ..., (\val^\pidA_{tm}, \val^\pidA_{em})]], ..., 
				 [(\val^\pidZ_{t1}, \val^\pidZ_{e1}), ..., (\val^\pidZ_{tm}, \val^\pidZ_{em})]])} 
				\crcr\qquad\resoC{= [[\val^\pidA_1, ..., \val^\pidA_m], ... [\val^\pidZ_1, ..., \val^\pidZ_m]]}
		\crcr
			\resoC{\{\DynResolve\_\mathrm{Store}(\DMap^\pid_5, \sigma^\pid_5, \AccPP, [\val^\pid_1, ..., \val^\pid_m]) 
				= (\sigma^\pid_6, \DMap^\pid_6, \locL^\pid_7)\}^\pidZ_{\pid = \pidA}}
		\crcr
			\locLL_6 = \locLL_1 \addL (\pidA, \locL^\pidA_2) \Mid ... \Mid (\pidZ, \locL^\pidZ_2) \addL \locLL_3 
						\addL (\pidA, \locL^\pidA_4) \Mid ... \Mid (\pidZ, \locL^\pidZ_4) \addL \locLL_5
						\addL (\pidA, \locL^\pidA_6) \Mid ... \Mid (\pidZ, \locL^\pidZ_6) 
		\crcr 
			\locLL_7 = \locLL_6 \addL (\pidA, \locL^\pidA_7) \Mid ... \Mid (\pidZ, \locL^\pidZ_7)
		\qq 
			\codeLL_4 = \codeLL_1 \addC \codeLL_2 \addC \codeLL_3
	\end{array} }
	{\begin{array}{l} 
	((\pidA, \gamma^\pidA, \sigma^\pidA, \DMap^\pidA, \Acc, \If\ (\ExprC\Expr)\ \sC{\stmt_1}\ \Else\ \ssC{\stmt_2}) 
	 \Mid ... \Mid 
	 (\pidZ, \gamma^\pidZ, \sigma^\pidZ, \DMap^\pidZ, \Acc, \If\ (\ExprC\Expr)\ \sC{\stmt_1}\ \Else\ \ssC{\stmt_2})) 
		 \Deval{\locLL_7}{\codeLL_4 \addC \codeMP{iepd}} 
		 \crcr ((\pidA, \gamma^\pidA, \resoC{\sigma^\pidA_{6}}, \resoC{\DMap^\pidA_{6}}, \Acc, \Skip) 
		 	 \qq \-\ \-\ \Mid ... \Mid
		  	(\pidZ, \gamma^\pidZ, \resoC{\sigma^\pidZ_{6}}, \resoC{\DMap^\pidZ_{6}}, \Acc, \Skip))
		\end{array}}
	\caption{\piccoC\ rule Private If Else - Location Tracking.}
	\label{Fig: iep lt}
\end{subfigure}
\\ \\
\begin{subfigure}{\textwidth}
Multiparty If Else False  		\\
\inferrule{\begin{array}{l}
	\qquad 
		\ExprC{((\pidA, \hgamma,\ \hsigma,\ \bsq, \bsq, \hExpr)\ \ \ \Mid ... \Mid
		  (\pidZ, \hgamma,\ \hsigma,\ \bsq, \bsq, \hExpr))} 
	 \crcr \ExprC{\Veval_{\codeVLL_1} 
			((\pidA, \hgamma,\ \hsigma_1, \bsq, \bsq, \hat{n})\ \ \ \Mid ... \Mid
			  (\pidZ, \hgamma,\ \hsigma_1, \bsq, \bsq, \hat{n}))}
	\qq \ExprC\hn = 0
	\crcr\qquad \sC{((\pidA, \hgamma,\ \hsigma_1, \bsq, \bsq, \hstmt_1)\ \ \Mid ... \Mid
			   (\pidZ, \hgamma,\ \hsigma_1, \bsq, \bsq, \hstmt_1))} 
	\crcr \sC{\Veval_{\codeVLL_2} 
			((\pidA, \hgamma_1, \hsigma_2, \bsq, \bsq, \Skip) \Mid ... \Mid
			 (\pidZ, \hgamma_1, \hsigma_2, \bsq, \bsq, \Skip))}
	\crcr\qquad \ssC{((\pidA, \hgamma,\ \hsigma_1, \bsq, \bsq, \hstmt_2)\ \ \Mid ... \Mid
			   (\pidZ, \hgamma,\ \hsigma_1, \bsq, \bsq, \hstmt_2))} 
	\crcr \ssC{\Veval_{\codeVLL_3} 
			((\pidA, \hgamma_2, \hsigma_3, \bsq, \bsq, \Skip) \Mid ... \Mid
			 (\pidZ, \hgamma_2, \hsigma_3, \bsq, \bsq, \Skip))}
	\end{array}}
	{\begin{array}{l}
	((\pidA, \hgamma, \hsigma,\ \bsq, \bsq, \If (\hExpr)\ \hstmt_1\ \Else\ \hstmt_2)\Mid ... \Mid
	  (\pidZ, \hgamma, \hsigma,\ \bsq, \bsq, \If (\hExpr)\ \hstmt_1\ \Else\ \hstmt_2)) 
		\Veval_{\codeVLL_1\addC\codeVLL_2\addC\codeVLL_3\addC[\codeVS{mpief}]} 
		\crcr
		((\pidA, \hgamma, \ssC{\hsigma_3}, \bsq, \bsq, \Skip)
		\qq\ \Mid ... \Mid
		 (\pidZ, \hgamma, \ssC{\hsigma_3}, \bsq, \bsq, \Skip))
		 \end{array}}
	\caption{\vanillaC\ rule Multiparty If Else False.}
	\label{Fig: van mpief}
\end{subfigure}
\end{tabular}
\caption{\Code{if else} branching on private data example (\ref{Fig: if else \piccoC code dp}, \ref{Fig: if else piccoC expanded dp}), \piccoC\ location-tracking (\ref{Fig: iep lt}), and \vanillaC\ Multiparty If Else False (\ref{Fig: van mpief}) rules. Coloring in the rules highlight the corresponding code and rule execution.}
%
\label{Fig: if else color dp}
\end{figure*}

































\paragraph{Conditional Code Block Location Tracking}
\label{sec: priv if lt desc}
Here we track modifications during private-conditioned branches at the level of memory blocks and offsets, which ensures that we do not update any data in memory inaccurately, as is shown in Figure~\ref{Fig: simple pointer challenge ex} using variable tracking SMC techniques.  
To facilitate this, we use the mapping structure $\DMap$ to track changes to each location at each level of nesting. This structure maps locations to a four-tuple of the original value, the \TT{then} branch value, a tag to notate whether the \TT{then} branch value was updated during the restoration phase, and the type of value stored (i.e., $(\loc, \offset) \to$ $(\val_1, \val_2$, $\tagb$, $\Type$)). The tag is used to allow us to add to $\DMap$ as we encounter pointer dereference writes and array writes at public indices without needing to track which branch we are in. It is always initialized as 0, and updated to 1 when we enter the restoration phase and store a value into the \TT{then} position. This way, if a location was added in the \TT{else} branch (i.e., was not modified in the \TT{then} branch), we know to use the original value as the \TT{then} value when we resolve the true value of that location at the end. 

The overall structure of the location tracking rule is similar to the variable tracking rule. 
We first evaluate $\Expr$ to $\n$, then call $\DynExtract$ to find variables that are modified during the execution of either branch and that there are multi-level modifications within at least one branch. 
We then call $\DynInit$, which stores the result of the private conditional and uses the variables we found to create the initial mapping $\DMap$.  
Next, we proceed to evaluate the \TT{then} branch, and call $\DynRestore$ to update $\DMap$ with the ending \TT{then} values for all locations that are tracked and restore the original values back into memory. 
After, we evaluate the \TT{else} branch and, once complete, call $\DynResolveR$ to retrieve the result of the conditional and the \TT{then} and \TT{else} values for each location. 
As with variable tracking, we use multiparty protocol $\MPC{resolve}$ to obtain the true values, and then store them back into their respective locations using Algorithm $\DynResolveS$. 
%
It is important to note that when we evaluate a pointer dereference write or array write at a public index inside a branch, we check to see if the given location is in $\DMap[\Acc]$.  If it is not, we add a mapping to store the original data (i.e., $(\loc, \offset) \to$ (\TT{orig}, $\Null$, $0$, $\Type$)). Notice that the data can either be a regular value (i.e., for a memory block storing a private int) or a pointer data structure representing a private pointer (i.e., for a memory block storing a private int*). 

In Figure~\ref{Fig: if else piccoC expanded dp}, we show an approximation of the execution of the pointer challenge example shown in Figure~\ref{Fig: simple pointer challenge ex}. 
When we reach the private-conditioned branching statement, we first store the result of the condition \TT{a < b}. As we execute the \TT{then} branch, we add the entry for $\loc_\TT{a}$ to $\DMap$, as \TT{p} refers to \TT{a}. We restore between branches by resetting $\loc_\TT{a}$ to its original value stored in $\DMap[\loc_\TT{a}][0]$. As we execute the \TT{else} branch, we add the entry for $\loc_\TT{p}$ to $\DMap$, as we are modifying which location \TT{p} points to. Finally, we resolve the true values for each modified location in $\DMap$. This approach eliminates the issues shown in Figure~\ref{Fig: simple pointer challenge ex}, as we do not rely on the pointer's current location to appropriately resolve the true values. 







