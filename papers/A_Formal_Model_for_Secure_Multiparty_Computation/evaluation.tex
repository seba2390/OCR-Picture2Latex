To highlight the feasibility of our approach we provide preliminary performances numbers over both microbenchmarks and real-world SMC programs.  All experiments were run in a local and distributed manner.  We leverage local runs, where all participants in the SMC program execute on the same machine, to analyze overheads and benefits of our approach. We also  provide distributed deployment of the same benchmarks to illustrate real-world performance. In the distributed configuration, each participant in the SMC program is executed on a separate machine.
We ran our experiments using single-threaded execution on three 2.10GHz machines running CentOS-7. The machines were connected via 1Gbps Ethernet.  


% short description of programs run
Benchmark pay-gap is the program shown in Figure~\ref{Fig: salary vs gender}, where the average female salary and average male salary is computed.  
Benchmark LR-parser is a program that will parse and execute a mathematical function over private data. This program is given as input the function as a list of tokens and the private data to execute the function over.
Benchmark h\_analysis is a program that takes two sets of input data and calculates the percent difference between each element of the two data sets.
All of the "pb" benchmarks are programs designed to emphasize the difference in executing using single-statement resolution, as PICCO does, and when using the conditional code block tracking, as in \piccoC. These programs are run through a large number of iterations of branches where variables are modified several times in each branch, as in the example in Figure~\ref{Fig: resolution cost ex}.

When executing locally, we ran all three computational parties on a single machine. 
When executing distributed, one computational party is run on each machine. 
All programs were run for 50 times both distributed and locally. 
LR-parser has the shortest execution time of our benchmarks (seconds), and its timing is more influenced by small differences in communication overhead during the computation, resulting in a greater standard deviation. 
Program pb-reuse has the longest runtime, executing in about 5 and 3.5 minutes locally and 15 and 8 minutes distributed for PICCO and \piccoC, respectively. 
In our computations, we first took the average time of each of the three computational parties, then the total average and standard deviation of all runs. We calculated the percent speedup in Figure~\ref{Fig: percent diff} using PICCO's timings as a baseline.   We can see from our micro-benchmarks which stress private-conditioned branches that our approach provides significant runtime improvement.  This is not surprising as our resolution mechanism requires less
communication between parties.  However, this benefit is reduced in real-world SMC programs and is proportional to the number of private-conditioned branches as well as their complexity (i.e. the number of private variables they use and modify). Similarly we observe that performance improvement or reduction is dilated when communication overheads increase (e.g. we execute in a distributed setting).  This is also not surprising as the communication cost as a percentage of total runtime increases in a distributed execution.


\begin{figure*}
\begin{subfigure}{0.48\textwidth}\footnotesize
\centering{\includegraphics[width=\textwidth]{LocalPercentDiff.png}}
\caption{Local}
\label{Fig: local percent diff}
\end{subfigure}
\quad
\begin{subfigure}{0.48\textwidth}\footnotesize
\centering{\includegraphics[width=\textwidth]{DistributedPercentDiff.png}}
\caption{Distributed}
\label{Fig:dist percent diff}
\end{subfigure}
\caption{Percentage Runtime Differences}
\label{Fig: percent diff}
\vspace{-0.3cm}
\end{figure*}





