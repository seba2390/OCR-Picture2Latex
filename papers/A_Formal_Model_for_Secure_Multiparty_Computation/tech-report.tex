% !TEX TS-program = pdflatex

\documentclass[11pt]{article}
\usepackage[margin=1in]{geometry}
\usepackage{authblk}

\bibliographystyle{plain}

\title{A Formal Model for Secure Multiparty Computations} 
\date{}

\author[1]{Amy Rathore}
\author[1]{Marina Blanton}
\author[2]{Marco Gaboardi}
\author[1]{Lukasz Ziarek}
\affil[1]{Department of Computer Science and Engineering, University at Buffalo}
\affil[2]{Department of Computer Science, Boston University}

%\keywords{Secure Multiparty Computation, Formal Model}



\usepackage{pgfplotstable}
\pgfplotstableset{precision=5}
\usepackage{booktabs}

%%%%%%%%%%%%%%%%%%%%%%%%%%%%%%%%%%%%%%%%
\usepackage{enumitem}
\usepackage{subcaption} 
\usepackage{mathpartir}
\usepackage{xcolor}

\usepackage{bbding}
\usepackage{amsmath,amssymb,amsfonts}
\usepackage{amsthm}
\usepackage{algorithmic}
\usepackage{algorithm}
\usepackage{bm}
\usepackage{multirow}
\usepackage{color}
\usepackage{textcomp}
\usepackage{graphicx}
\usepackage{stmaryrd}
\usepackage{float}
\usepackage{hyperref}

%\usepackage{comment}


%%%%%%%%%%%%%%%%%%%%%%%%%%%%%%%%%%%%%%%%
% Listings setup
\usepackage{listings}
\usepackage[T1]{fontenc}


% full circle, half circle, and empty circle                                 
\newcommand{\fc}{\CircleSolid}
\newcommand{\hc}{\HalfCircleRight}
\newcommand{\ec}{\CircleShadow}

\newcommand{\bsq}{\Box}
\newcommand{\bsqSet}{\bsq}

%%%%%%%%%%%%%%%%%%%%%%%%%%%%%%%%%%%%%%%%
% Colors
\colorlet{darkpink}{red!60!purple!50}
\colorlet{lightpurple}{purple!60!blue!60}
\colorlet{purple}{purple!70!blue!100}
\colorlet{teal}{green!70!blue!90!black!100}
\colorlet{darkgreen}{green!50!black!100}
\colorlet{orange}{orange!70!red!80}
\colorlet{yellow}{yellow!100!orange!70!red!80!black!80}

\newcommand{\pink}[1]{\textcolor{darkpink}{#1}}
\newcommand{\purple}[1]{\textcolor{purple}{#1}}
\newcommand{\teal}[1]{\textcolor{teal}{#1}}
\newcommand{\green}[1]{\textcolor{green}{#1}}
\newcommand{\orange}[1]{\textcolor{orange}{#1}}
\newcommand{\cyan}[1]{\textcolor{cyan}{#1}}
\newcommand{\red}[1]{\textcolor{red}{#1}}
\newcommand{\blue}[1]{\textcolor{blue}{#1}}

\newcommand{\TT}[1]{\texttt{#1}}

\newcommand{\textgr}[1]{\textcolor{teal}{#1}}
\newcommand{\GT}[1]{\textgr{\ensuremath{#1}}}
\newcommand{\retStep}[1]{\GT{#1}}



\newcommand{\ExprC}[1]{\blue{#1}}
\newcommand{\TExprC}{dark\ blue}

\newcommand{\resC}[1]{\purple{#1}}
\newcommand{\TresC}{dark\ purple}

\newcommand{\extC}[1]{\teal{#1}}
\newcommand{\TextC}{green}

\newcommand{\initC}[1]{\red{#1}}
\newcommand{\TinitC}{red}

\newcommand{\sC}[1]{\cyan{#1}}
\newcommand{\TsC}{light\ blue}

\newcommand{\ssC}[1]{\textcolor{lightpurple}{#1}}
%\newcommand{\ssC}[1]{\pink{#1}}
\newcommand{\TssC}{light\ purple}

\newcommand{\restC}[1]{\textcolor{yellow}{#1}}
\newcommand{\TrestC}{yellow}

\newcommand{\resoC}[1]{\orange{#1}}
\newcommand{\TresoC}{orange}


% --------------------------------------------------------------------
\newcommand{\THESYSTEM}{\textsf{FormalPICCO}\xspace}
% --------------------------------------------------------------------

\theoremstyle{plain}
\newtheorem{lemma}{Lemma}%[section]
\newtheorem{theorem}{Theorem}%[section]
\newtheorem{axiom}{Axiom}%[section]
\newtheorem{corollary}{Corollary}%[section]


\theoremstyle{definition}
\newtheorem{definition}{Definition}%[section]




\def\p#1{\mathrel{\ooalign{\hfil$\mapstochar\mkern 5mu$\hfil\cr$#1$}}}


%%%%%%%%%%%%%%%%%%%%%%%%%%
%                       Amy's macros
%%%%%%%%%%%%%%%%%%%%%%%%%%

%Grammar
\newcommand{\eval}{\Downarrow}
\newcommand{\Expr}{e}
\newcommand{\stmt}{s}
\newcommand{\val}{v}
\newcommand{\valL}{V}
\newcommand{\var}{\mathit{var}}
\newcommand{\cchar}{c}
\newcommand{\str}{str}
\newcommand{\x}{x}
\newcommand{\vl}{\mathit{list}}
\newcommand{\res}{\mathit{res}}
\newcommand{\binop}{\mathit{bop}}
\newcommand{\unop}{\mathit{uop}}
\newcommand{\preop}{\mathit{uop}}
\newcommand{\postop}{\mathit{pop}}
\newcommand{\decl}{\mathit{decl}}
\newcommand{\Type}{\mathit{ty}}
\newcommand{\func}{f}
\newcommand{\btype}{\mathit{bty}}
\newcommand{\llabel}{a}
\newcommand{\builtin}{\mathit{prim}}
\newcommand{\Elist}{E}
\newcommand{\Tlist}{\mathit{tyL}}
\newcommand{\plist}{P}
\newcommand{\IOlist}{\mathit{args}}
\newcommand{\optype}{\llabel}
\newcommand{\bytelabel}{k}
\newcommand{\bytelen}{\kappa}
\newcommand{\n}{n}

\newcommand{\byte}{\omega}
\newcommand{\byteL}{I}
\newcommand{\tagb}{j}
\newcommand{\tagbL}{J}
\newcommand{\perm}{p}
\newcommand{\permT}{z}

\newcommand{\temp}{temp}
\newcommand{\tempcounter}{\mathit{ctr}}
\newcommand{\tempVar}{\ensuremath{\mathit{temp}_\mathit{\tempcounter}}}

\newcommand{\Const}{\mathrm{const}}
\newcommand{\Skip}{\mathrm{skip}}

%types
\newcommand{\Void}{\mathrm{void}}
\newcommand{\Char}{\mathrm{char}}
\newcommand{\Int}{\mathrm{int}}
\newcommand{\Float}{\mathrm{float}}
\newcommand{\pint}{\mathrm{priv\_int}}
\newcommand{\pfloat}{\mathrm{priv\_float}}
\newcommand{\Priv}{\mathrm{private}}
\newcommand{\Pub}{\mathrm{public}}
\newcommand{\Und}{\mathrm{undecided}}
\newcommand{\Freed}{\mathrm{freed}}

\newcommand{\OpEnc}{\mathrm{enc}}
\newcommand{\ti}{\mathrm{u}}
\newcommand{\If}{\mathrm{if}}
\newcommand{\Then}{\mathrm{then}}
\newcommand{\Else}{\mathrm{else}}
\newcommand{\For}{\mathrm{for}}
\newcommand{\free}{\mathrm{free}}
\newcommand{\pfree}{\mathrm{pfree}}
\newcommand{\Malloc}{\mathrm{malloc}}
\newcommand{\PMalloc}{\mathrm{pmalloc}}
\newcommand{\Null}{\mathrm{NULL}}
\newcommand{\While}{\mathrm{while}}
\newcommand{\Return}{\mathrm{return}}
\newcommand{\smcinput}{\mathrm{smcinput}}
\newcommand{\smcoutput}{\mathrm{smcoutput}}
\newcommand{\smcopen}{\mathrm{smcopen}}
\newcommand{\Cont}{\mathrm{continue}}
\newcommand{\Break}{\mathrm{break}}

\newcommand{\Llist}{\mathrm{List}}
\newcommand{\Ptr}{\mathrm{Pointer}}

%Permissions
\newcommand{\PermF}{\mathrm{Freeable}}
\newcommand{\PermW}{\mathrm{Writable}}
\newcommand{\PermR}{\mathrm{Readable}}
\newcommand{\PermNE}{\mathrm{Nonempty}}
\newcommand{\PermN}{\mathrm{None}}

%Functions
\newcommand{\Decrypt}{\mathrm{decrypt}}
\newcommand{\rrDecrypt}{\RT\Decrypt}
\newcommand{\bDecrypt}{\BT\Decrypt}

\newcommand{\Encrypt}{\mathrm{encrypt}}
\newcommand{\rrEncrypt}{\RT\Encrypt}
\newcommand{\bEncrypt}{\BT\Encrypt}

\newcommand{\bitand}{\mathrm{bitwise\_and}}
\newcommand{\rrbitand}{\RT\bitand}
\newcommand{\bbitand}{\BT\bitand}

\newcommand{\bitor}{\mathrm{bitwise\_or}}
\newcommand{\rrbitor}{\RT\bitor}
\newcommand{\bbitor}{\BT\bitor}

\newcommand{\GetLoc}{\mathrm{GetLocation}}
\newcommand{\rrGetLoc}{\RT\GetLoc}
\newcommand{\bGetLoc}{\BT\GetLoc}

\newcommand{\IncLocList}{\mathrm{IncrementList}}
\newcommand{\rrIncLocList}{\RT\IncLocList}
\newcommand{\bIncLocList}{\BT{\IncLocList}}

\newcommand{\DecLocList}{\mathrm{DecrementList}}
\newcommand{\rrDecLocList}{\RT\DecLocList}
\newcommand{\bDecLocList}{\BT{\DecLocList}}

\newcommand{\OOBArrR}{\mathrm{ArrayReadOutOfBounds}}
\newcommand{\GetVal}{\mathrm{GetValue}}
\newcommand{\Extract}{\mathrm{ExtractVariables}}
\newcommand{\Initialize}{\mathrm{InitializeVariables}}
\newcommand{\Resolve}{\mathrm{ResolveVariables}}
\newcommand{\ResolveR}{\mathrm{ResolveVariables\_Retrieve}}
\newcommand{\ResolveS}{\mathrm{ResolveVariables\_Store}}
\newcommand{\ResolveVal}{\mathrm{ResolveValue}}
\newcommand{\ResolvePtr}{\mathrm{ResolvePointer}}
\newcommand{\ResolveArr}{\mathrm{ResolveArray}}
\newcommand{\Restore}{\mathrm{RestoreVariables}}
\newcommand{\resolve}{\mathrm{resolve}}

\newcommand{\sizeof}{\mathrm{sizeof}}

\newcommand{\Free}{\mathrm{Free}}
\newcommand{\rrFree}{\RT\Free}
\newcommand{\bFree}{\BT{\mathrm{T}\Free}}

\newcommand{\MPC}[1]{\mathrm{MPC}_\mathit{{#1}}}
\newcommand{\PFree}{\MPC{free}}
\newcommand{\rrPFree}{\RT\PFree}
\newcommand{\bPFree}{\BT{\mathrm{T}\PFree}}

\newcommand{\Retrieve}{\mathrm{Retrieve\_vals}}
\newcommand{\rrRetrieve}{\RT\Retrieve}
\newcommand{\bRetrieve}{\BT{\mathrm{T}\Retrieve}}

\newcommand{\DerefPtrPub}{\mathrm{DerefPtr}}
\newcommand{\DerefPtrPubHLI}{\mathrm{DerefPtrHLI}}

\newcommand{\DerefPtr}{\mathrm{DerefPrivPtr}}
\newcommand{\rrDerefPtr}{\RT\DerefPtr}
\newcommand{\bDerefPtr}{\BT{\mathrm{T}\DerefPtr}}

\newcommand{\ExtractVar}{\mathrm{Extract\_variables}}
\newcommand{\rrExtractVar}{\RT\ExtractVar}
\newcommand{\bExtractVar}{\BT{\mathrm{T}\ExtractVar}}

\newcommand{\CondAssign}{\mathrm{CondAssign}}
\newcommand{\rrCondAssign}{\RT\CondAssign}
\newcommand{\bCondAssign}{\BT{\mathrm{T}\CondAssign}}

\newcommand{\ArrRead}{\mathrm{ArrayRead}}

\newcommand{\ArrR}{\mathrm{1DArrayRead}}
\newcommand{\rrArrR}{\RT\ArrR}
\newcommand{\bArrR}{\BT{\mathrm{T}\ArrR}}

\newcommand{\AArrR}{\mathrm{2DArrayRead}}
\newcommand{\rrAArrR}{\RT\AArrR}
\newcommand{\bAArrR}{\BT{\mathrm{T}\AArrR}}

\newcommand{\Cast}{\mathrm{Cast}}
\newcommand{\rrCast}{\RT\Cast}
\newcommand{\bCast}{\BT{\mathrm{T}\Cast}}

\newcommand{\Declassify}{\mathrm{Declassify}}
\newcommand{\rrDeclassify}{\RT\Declassify}
\newcommand{\bDeclassify}{\BT{\mathrm{T}\Declassify}}

\newcommand{\DeclassifyPtr}{\mathrm{DeclassifyPtr}}
\newcommand{\rrDeclassifyPtr}{\RT\DeclassifyPtr}
\newcommand{\bDeclassifyPtr}{\BT{\mathrm{T}\DeclassifyPtr}}

\newcommand{\Update}{\mathrm{UpdateVal}}
\newcommand{\TUpdate}{\mathrm{T\_\Update}}
\newcommand{\rrUpdate}{\RT\Update}
\newcommand{\bUpdate}{\BT{\mathrm{T}\Update}}

\newcommand{\UpdateDerefVals}{\mathrm{UpdateDerefVals}}

\newcommand{\UpdateArr}{\mathrm{UpdateArr}}

\newcommand{\UpdateP}{\mathrm{UpdatePriv}}
\newcommand{\TUpdateP}{\mathrm{T\_\UpdateP}}
\newcommand{\rrUpdateP}{\RT\UpdateP}
\newcommand{\bUpdateP}{\BT{\mathrm{T}\UpdateP}}

\newcommand{\UpdatePtr}{\mathrm{UpdatePtr}}
\newcommand{\TUpdatePtr}{\mathrm{T\_\UpdatePtr}}
\newcommand{\rrUpdatePtr}{\RT\UpdatePtr}
\newcommand{\bUpdatePtr}{\BT{\mathrm{T}\UpdatePtr}}

\newcommand{\UpdatePPtr}{\mathrm{UpdatePrivPtr}}
\newcommand{\TUpdatePPtr}{\mathrm{T\_\UpdatePPtr}}
\newcommand{\rrUpdatePPtr}{\RT\UpdatePPtr}
\newcommand{\bUpdatePPtr}{\BT{\mathrm{T}\UpdatePPtr}}

\newcommand{\InputVal}{\mathrm{InputValue}}
\newcommand{\rrInputVal}{\RT\InputVal}
\newcommand{\bInputVal}{\BT{\mathrm{T}\InputVal}}

\newcommand{\InputArr}{\mathrm{InputArray}}
\newcommand{\rrInputArr}{\RT\InputArr}
\newcommand{\bInputArr}{\BT{\mathrm{T}\InputArr}}

\newcommand{\OutputVal}{\mathrm{OutputValue}}
\newcommand{\rrOutputVal}{\RT\OutputVal}
\newcommand{\bOutputVal}{\BT{\mathrm{T}\OutputVal}}

\newcommand{\OutputArr}{\mathrm{OutputArray}}
\newcommand{\rrOutputArr}{\RT\OutputArr}
\newcommand{\bOutputArr}{\BT{\mathrm{T}\OutputArr}}

\newcommand{\CheckPub}{\mathrm{CheckPublicEffects}}
\newcommand{\rrCheckPub}{\RT\CheckPub}
\newcommand{\bCheckPub}{\BT{\mathrm{T}\CheckPub}}

\newcommand{\DecodeArr}{\mathrm{DecodeArr}}
\newcommand{\rrDecodeArr}{\RT\Decode}

\newcommand{\Decode}{\mathrm{DecodeVal}}
\newcommand{\rrDecode}{\RT\Decode}
\newcommand{\bDecode}{\BT{\mathrm{T}\Decode}}

\newcommand{\DecodePtr}{\mathrm{DecodePtr}}
\newcommand{\rrDecodePtr}{\RT\DecodePtr}
\newcommand{\bDecodePtr}{\BT{\mathrm{T}\DecodePtr}}

\newcommand{\DecodeFun}{\mathrm{DecodeFun}}
\newcommand{\rrDecodeFun}{\RT\DecodeFun}
\newcommand{\bDecodeFun}{\BT{\mathrm{T}\DecodeFun}}

\newcommand{\Encode}{\mathrm{EncodeVal}}
\newcommand{\rrEncode}{\RT\Encode}
\newcommand{\bEncode}{\BT{\mathrm{T}\Encode}}

\newcommand{\EncodeArr}{\mathrm{EncodeArr}}

\newcommand{\EncodePtr}{\mathrm{EncodePtr}}
\newcommand{\rrEncodePtr}{\RT\EncodePtr}
\newcommand{\bEncodePtr}{\BT{\mathrm{T}\EncodePtr}}

\newcommand{\EncodeFun}{\mathrm{EncodeFun}}
\newcommand{\rrEncodeFun}{\RT\EncodeFun}
\newcommand{\bEncodeFun}{\BT{\mathrm{T}\EncodeFun}}

\newcommand{\ReadOOB}{\mathrm{ReadOOB}}
\newcommand{\rrReadOOB}{\RT\ReadOOB}
\newcommand{\bReadOOB}{\BT{\mathrm{T}\ReadOOB}}

\newcommand{\ReadOOBAA}{\mathrm{ReadOOB2D}}
\newcommand{\rrReadOOBAA}{\RT\ReadOOBAA}
\newcommand{\bReadOOBAA}{\BT{\mathrm{T}\ReadOOBAA}}

\newcommand{\ReadOOBAAPubPriv}{\mathrm{ReadOOB2DPubPriv}}
\newcommand{\rrReadOOBAAPubPriv}{\RT\ReadOOBAAPubPriv}
\newcommand{\bReadOOBAAPubPriv}{\BT{\mathrm{T}\ReadOOBAAPubPriv}}

\newcommand{\ReadOOBAAPrivPub}{\mathrm{ReadOOB2DPrivPub}}
\newcommand{\rrReadOOBAAPrivPub}{\RT\ReadOOBAAPrivPub}
\newcommand{\bReadOOBAAPrivPub}{\BT{\mathrm{T}\ReadOOBAAPrivPub}}

\newcommand{\WriteOOB}{\mathrm{WriteOOB}}
\newcommand{\rrWriteOOB}{\RT\WriteOOB}
\newcommand{\bWriteOOB}{\BT{\mathrm{T}\WriteOOB}}

\newcommand{\WriteOOBAA}{\mathrm{WriteOOB2D}}
\newcommand{\rrWriteOOBAA}{\RT\WriteOOBAA}
\newcommand{\bWriteOOBAA}{\BT{\mathrm{T}\WriteOOBAA}}

\newcommand{\WriteOOBAAPubPriv}{\mathrm{WriteOOB2DPubPriv}}
\newcommand{\rrWriteOOBAAPubPriv}{\RT\WriteOOBAAPubPriv}
\newcommand{\bWriteOOBAAPubPriv}{\BT{\mathrm{T}\WriteOOBAAPubPriv}}

\newcommand{\WriteOOBAAPrivPub}{\mathrm{WriteOOB2DPrivPub}}
\newcommand{\rrWriteOOBAAPrivPub}{\RT\WriteOOBAAPrivPub}
\newcommand{\bWriteOOBAAPrivPub}{\BT{\mathrm{T}\WriteOOBAAPrivPub}}

\newcommand{\TY}{\mathrm{Label}}
\newcommand{\rrTY}{\RT\TY}
\newcommand{\bTY}{\BT{\mathrm{T}\TY}}

\newcommand{\Bytes}{\mathrm{Bytes}}
\newcommand{\Byte}{\mathrm{Byte}}
\newcommand{\Bit}{\mathrm{Bit}}
\newcommand{\Env}{\mathrm{Env}}
\newcommand{\EnvT}{\mathrm{T\ Env}}
\newcommand{\Label}{\mathrm{Label}}
\newcommand{\Loc}{\mathrm{Loc}}
\newcommand{\Mem}{\mathrm{Mem}}
\newcommand{\Perm}{\mathrm{Perm}}
\newcommand{\Sz}{\mathrm{Size}}
\newcommand{\Ty}{\mathrm{Type}}
\newcommand{\Val}{\mathrm{Val}}
\newcommand{\Var}{\mathrm{Var}}
\newcommand{\llist}{\mathrm{list}}


\newcommand{\pluseq}{\mathrel{+}=}
\newcommand{\minuseq}{\mathrel{-}=}
\newcommand{\plpl}{\mathrel{+}\mathrel{+}}
\newcommand{\subsub}{\mathrel{-}\mathrel{-}}
\newcommand{\peq}{\cong_{pub}}


% Accumulator Macros
\newcommand{\Acc}{\mathrm{acc}}
\newcommand{\AccZ}{0}
\newcommand{\AccP}{1}
\newcommand{\AccPP}{\Acc+1}
\newcommand{\rAccPP}{\RT{\AccPP}}
\newcommand{\rAccZ}{\RT{\AccZ}}
\newcommand{\rAccP}{\RT{\AccP}}

%Spacing
\newcommand{\nq}{\\ \-\ \quad}
\newcommand{\q}{\\ \-\ \qquad}
\newcommand{\nqq}{\\ \-\ \qquad\qquad}
\newcommand{\nqqq}{\\ \-\ \qquad\qquad\qquad}
\newcommand{\qq}{\qquad\qquad}
\newcommand{\ssp}{~\\~\\~\\}			% semantics space
\newcommand{\rsp}{~\\~\\}
\newcommand{\tsp}{~\\~\\}			%translation space
\newcommand{\xsp}{~\\~\\~\\~\\~\\~\\~\\~\\~\\~\\~\\~\\}

%Spacing with bold 
%Used in translation section
\newcommand{\qb}[1]{\\ \-\ \qquad\bm{#1}}
\newcommand{\nqb}[1]{\\ \-\ \qquad\bm{#1}}
\newcommand{\qqb}[1]{\\ \-\ \qquad\qquad\bm{#1}}
\newcommand{\qqqb}[1]{\\ \-\ \qquad\qquad\qquad\bm{#1}}


\newcommand{\num}[1]{\underline{#1}}

% Red (UNtranslated) 
\newcommand{\textred}[1]{\textcolor{red}{#1}}
\newcommand{\RT}[1]{\textred{\ensuremath{#1}}}
\newcommand{\rAcc}{\RT\Acc}
\newcommand{\rExpr}{\RT\Expr}
\newcommand{\rstmt}{\RT\stmt}
\newcommand{\rx}{\RT\x}
\newcommand{\rn}{\RT{n}}
\newcommand{\rrgamma}{\RT\gamma}
\newcommand{\rrsigma}{\RT\sigma}
\newcommand{\rgamma}[1]{\RT{\gamma{#1}}}
\newcommand{\rsigma}[1]{\RT{\sigma{#1}}}
\newcommand{\rPermF}{\RT\PermF}
\newcommand{\rVoid}{\RT\Void}
\newcommand{\rSkip}{\RT\Skip}
\newcommand{\rNull}{\RT\Null}
\newcommand{\rChar}{\RT\Char}
\newcommand{\rInt}{\RT\Int}
\newcommand{\rFloat}{\RT\Float}
\newcommand{\rpint}{\RT\pint}
\newcommand{\rpfloat}{\RT\pfloat}
\newcommand{\rlabel}{\RT\llabel}
\newcommand{\rPriv}{\RT\Priv}
\newcommand{\rPub}{\RT\Pub}
\newcommand{\rUnd}{\RT\Und}
\newcommand{\rloc}{\RT\loc}
\newcommand{\rval}{\RT\val}
\newcommand{\rType}{\RT\Type}
\newcommand{\rbtype}{\RT\btype}
\newcommand{\rbyte}{\RT\byte}
\newcommand{\rBreak}{\RT\Break}
\newcommand{\rCont}{\RT\Cont}
\newcommand{\rReturn}{\RT\Return}
\newcommand{\rElist}{\RT\Elist}
\newcommand{\rTlist}{\RT\Tlist}
\newcommand{\rplist}{\RT\plist}
\newcommand{\rIOlist}{\RT\IOlist}
\newcommand{\rtau}{\RT\tau}
\newcommand{\rphi}{\RT\phi}
\newcommand{\rtagb}{\RT\tagb}
\newcommand{\rtagbL}{\RT\tagbL}
\newcommand{\rlocL}{\RT\locL}


% Blue (translated)
\newcommand{\textbl}[1]{\textcolor{blue}{#1}}
\newcommand{\BT}[1]{\textbl{\ensuremath{#1}}}
\newcommand{\BTT}[1]{\BT{\tilde{#1}}}
\newcommand{\bExpr}{\BTT\Expr}
\newcommand{\bx}{\BT\x}
\newcommand{\bn}{\BT{n}}
\newcommand{\bbtgamma}{\BTT\gamma}
\newcommand{\bbtsigma}{\BTT\sigma}
\newcommand{\btgamma}[1]{\BT{\tilde{\gamma}{#1}}}
\newcommand{\btsigma}[1]{\BT{\tilde{\sigma}{#1}}}
\newcommand{\bAcc}{\BT\Acc}
\newcommand{\bstmt}{\BTT\stmt}
\newcommand{\bPermF}{\BT\PermF}
\newcommand{\bVoid}{\BT\Void}
\newcommand{\bSkip}{\BT\Skip}
\newcommand{\bNull}{\BT\Null}
\newcommand{\bChar}{\BT\Char}
\newcommand{\bInt}{\BT\Int}
\newcommand{\bFloat}{\BT\Float}
\newcommand{\bpint}{\BT\pint}
\newcommand{\bpfloat}{\BT\pfloat}
\newcommand{\blabel}{\BT\llabel}
\newcommand{\bPriv}{\BT\Priv}
\newcommand{\bPub}{\BT\Pub}
\newcommand{\bUnd}{\BT\Und}
\newcommand{\bloc}{\BT\loc}
\newcommand{\bval}{\BT\val}
\newcommand{\bType}{\BTT\Type}
\newcommand{\bbtype}{\BTT\btype}
\newcommand{\bbyte}{\BT\byte}
\newcommand{\bBreak}{\BT\Break}
\newcommand{\bCont}{\BT\Cont}
\newcommand{\bReturn}{\BT\Return}
\newcommand{\bElist}{\BTT\Elist}
\newcommand{\bTlist}{\BTT\Tlist}
\newcommand{\bplist}{\BTT\plist}
\newcommand{\bIOlist}{\BTT\IOlist}
\newcommand{\btau}{\BTT\tau}
\newcommand{\bphi}{\BT\phi}

% Vanilla C eval
\newcommand{\Veval}{\eval'}

% Picco C eval
\newcommand{\Peval}{\eval}

% Transactional C eval
\newcommand{\Teval}{\eval^{t}}
\newcommand{\TSeval}{\eval^{t*}}

\newcommand{\Oeval}{\eval_{orig}}
\newcommand{\evalsub}{\Rightarrow}

\newcommand{\N}{\mathbb{N}}

% input/output file names (for proof)
\newcommand{\inputFile}{input}
\newcommand{\outputFile}{output}


% Van C input/output func names
\newcommand{\inputFun}{\mathrm{mcinput}}
\newcommand{\outputFun}{\mathrm{mcoutput}}


% Permission Lists
\newcommand{\PermL}{\mathrm{PermL}}
\newcommand{\VarPermL}{\PermL}
\newcommand{\FunPermL}{\PermL\_\mathrm{Fun}}
\newcommand{\ArrPermL}{\PermL}
\newcommand{\AArrPermL}{\PermL\_\mathrm{2DArr}}
\newcommand{\PtrPermL}{\PermL\_\mathrm{Ptr}}


% POINTERS
% Indirection of pointers Macros
\newcommand{\getIndirection}{\mathrm{GetIndirection}}
\newcommand{\indir}{i}
\newcommand{\hindir}{\hat{\indir}}

% offset for pointers
\newcommand{\offset}{\ensuremath{\mu}}
\newcommand{\Hoffset}{\hat{\offset}}
% number of locations for pointers
\newcommand{\nl}{\alpha}
\newcommand{\hnl}{\hat{\nl}}


% Index for array
\newcommand{\ind}{i}
\newcommand{\hind}{\hat{\ind}}


% HATS for Van C Macros
\newcommand{\hgamma}{\hat{\gamma}}
\newcommand{\hsigma}{\hat{\sigma}}
\newcommand{\hx}{\hat{\x}}
\newcommand{\hAcc}{\bsq}
\newcommand{\hstmt}{\hat{\stmt}}
\newcommand{\hExpr}{\hat{\Expr}}
\newcommand{\hType}{\hat{\Type}}
\newcommand{\hbtype}{\hat{\btype}}
\newcommand{\hval}{\hat{\val}}
\newcommand{\hvalL}{\hat{\valL}}
\newcommand{\hbyte}{\hat{\byte}}
\newcommand{\hElist}{\hat{\Elist}}
\newcommand{\hTlist}{\hat{\Tlist}}
\newcommand{\hplist}{\hat{\plist}}
\newcommand{\hn}{\hat{n}}
\newcommand{\Hm}{\hat{m}}
\newcommand{\hloc}{\hat{\loc}}
\newcommand{\hlocL}{\hat{\locL}}
\newcommand{\htagbL}{\hat{\tagbL}}




% Language Macros
\newcommand{\piccoC}{SMC$^2$}
\newcommand{\Transactional}{Location-tracking}
\newcommand{\transactional}{location-tracking}
\newcommand{\TpiccoC}{\Transactional\ SMC C}
\newcommand{\DynamicPicco}{\piccoC}
\newcommand{\vanillaC}{Vanilla C}

% Coded Evaluation Macros

\newcommand{\codeV}{\hat{\code}}
\newcommand{\codeVL}{\hat{\codeL}}
\newcommand{\codeVLL}{\hat{\codeLL}}
\newcommand{\codeVV}[1]{\mathit{\hat{#1}}}
\newcommand{\codeVS}[1]{(\pid, [\codeVV{#1}])}
\newcommand{\codeVM}[1]{(\mathrm{ALL}, [\codeVV{#1}])}
\newcommand{\codeVP}[1]{\codeVLL\addC\codeVS{#1}}



\newcommand{\piccoCodes}{SmcC}
\newcommand{\piccoCodesT}{SmcCT}
\newcommand{\piccoCodesX}{SmcCX}
\newcommand{\vanillaCodes}{VanC}
\newcommand{\vanillaCodesX}{VanCX}


\newcommand{\erasure}{\mathrm{Erase}}
\newcommand{\Terasure}{\mathrm{T}\erasure}


\newcommand{\loweq}{\simeq_L}
\newcommand{\deriv}{\triangleright}
\newcommand{\step}{\ensuremath{\mathrm{Step}}}
\newcommand{\nop}{\mathit{no-op}}


\newcommand{\R}[1]{\blue{({#1})}}
\newcommand{\B}[1]{\blue{({#1})}}


\newcommand{\checkOOB}{\mathrm{CheckOvershooting}}
\newcommand{\checkOOBM}{\checkOOB_M}


%\newcommand{\mit}[1]{\mathit{#1}}

% Macro for Program Code
\newcommand{\Code}[1]{\texttt{#1}}


% Macro for error in derivation ex in appendix
\newcommand{\Err}{\RT{\varepsilon}}


% Macro for getting function type list
\newcommand{\GetFunTyL}{\ensuremath{\mathrm{GetFunTypeList}}}

%Macro for getting the assignment statements for parameters of function calls (parameter = argument)
\newcommand{\GetFunParamAssign}{\ensuremath{\mathrm{GetFunParamAssign}}}


% Transactional Delta Map
\newcommand{\changeMap}{location map}
\newcommand{\DMap}{\Delta}
\newcommand{\dmap}{\delta}
\newcommand{\Vset}{\beta}
% Pointer List
\newcommand{\ptr}{\mathit{ptr}}

% Transactional Local Var Maps
\newcommand{\localMap}{local variable location tracker}
\newcommand{\locals}{\mathrm{locals}}
\newcommand{\Locals}{\mathrm{\chi}}

% Transactional branch trackers
\newcommand{\branchID}{branch identifier}
\newcommand{\Branch}{\mathrm{branch}}
\newcommand{\BranchN}{\mathit{none}}
\newcommand{\BranchT}{\mathit{then}}
\newcommand{\BranchE}{\mathit{else}}

% Transactional Priv If Helper Func Names
\newcommand{\ResolveL}{\mathrm{T\_resolve}}
\newcommand{\RestoreL}{\mathrm{T\_restore}}

% Transactional Congruence
\newcommand{\Tcong}{\cong^t}




% Logical XOR
\newcommand{\lxor}{\oplus}


\newcommand{\mg}[1]{\textcolor{blue}{MG: #1}}
\newcommand{\ap}[1]{\teal{AP: #1}}


% Alignment indicator
\newcommand{\alignInd}{\eta}
\newcommand{\halignInd}{\hat{\alignInd}}
\newcommand{\ralignInd}{\RT{\alignInd}}

\newcommand{\checkAlign}{\mathrm{CheckAlignment}}
\newcommand{\UpdateOffset}{\mathrm{UpdateOffset}}
\newcommand{\UpdateBytes}{\mathrm{UpdateBytes}}
\newcommand{\UpdateOOB}{\mathrm{UpdateOvershooting}}
\newcommand{\UpdateOOBP}{\mathrm{UpdateOvershootingPriv}}
\newcommand{\UpdateBytesFree}{\mathrm{UpdateBytesFree}}

% Transactional Versions of Algos
\newcommand{\TUpdateOffset}{\mathrm{T\_\UpdateOffset}}
\newcommand{\TWriteOOB}{\mathrm{T\_\WriteOOB}}




% Default location for private pointers
\newcommand{\locDefault}{\loc_\mathit{default}}
\newcommand{\hlocDefault}{\hloc_\mathit{default}}
\newcommand{\rlocDefault}{\RT{\locDefault}}


% Assist FREE / PFREE
\newcommand{\SelectFreeable}{\mathrm{CheckFreeable}}
\newcommand{\UpdatePtrLocs}{\mathrm{UpdatePointerLocations}}

\newcommand{\CheckCode}{\mathrm{CheckCodeCongruence}}
\newcommand{\CheckLocCong}{\mathrm{CheckIDCongruence}}
\newcommand{\GetFinalSwap}{\mathrm{GetFinalSwap}}

\newcommand{\LocMap}{map}
\newcommand{\Pcong}{\ensuremath{\cong_{\psi}}}
\newcommand{\PCong}[1]{\ensuremath{\cong_{\psi_{#1}}}}
\newcommand{\Swap}{\mathrm{SwapMemory}}
%\newcommand{\SwapMemBlock}{\mathrm{SwapMemoryBlock}}
\newcommand{\GetLocSwap}{\mathrm{GetLocationSwap}}



\newcommand{\TODO}{\ap{TODO}}




%PICCO evaluations
\newcommand{\PCeval}{\eval^{p}}
\newcommand{\PCAcc}{\mathrm{ind}}
\newcommand{\PCAccL}{I}

% PICCO-specific algo names
\newcommand{\SmcUpdatePtr}{\mathrm{smc\_update\_int\_ptr}}
\newcommand{\SMCUpdatePtr}{\mathrm{update\_int\_ptr}}
\newcommand{\SmcDerefPtrWrite}{\mathrm{smc\_dereference\_write\_ptr}}
\newcommand{\SMCDerefPtrWrite}{\mathrm{dereference\_write\_ptr}}
\newcommand{\SmcShrinkPtr}{\mathrm{smc\_shrink\_ptr}}
\newcommand{\SMCShrinkPtr}{\mathrm{shrink\_ptr}}
\newcommand{\GetTBranchStmt}{\mathrm{GetThenBranchStatements}}
\newcommand{\GetEBranchStmt}{\mathrm{GetElseBranchStatements}}



% Dynamic Picco
\newcommand{\Deval}[2]{\eval^{#1}_\mathit{#2}}
\newcommand{\DPeval}[2]{\eval^{p:#1}_\mathit{#2}}
\newcommand{\dyn}{opt}
%\newcommand{\BranchTD}{\BranchT\_\dyn}
%\newcommand{\BranchED}{\BranchE\_\dyn}

\newcommand{\Dyn}{}
\newcommand{\DynExtract}{\mathrm{\Dyn Extract}}
\newcommand{\DynInit}{\mathrm{\Dyn Initialize}}
\newcommand{\DynUpdate}{\mathrm{DynamicUpdate}}
\newcommand{\DynRestore}{\mathrm{\Dyn Restore}}
\newcommand{\DynResolve}{\mathrm{\Dyn Resolve}}
\newcommand{\DynResolveR}{\mathrm{\Dyn Resolve\_Retrieve}}
\newcommand{\DynResolveS}{\mathrm{\Dyn Resolve\_Store}}


% Label (Pub/Priv) Turnstile
\newcommand{\isPriv}{\vdash}
\newcommand{\isPub}{\nvdash}
\newcommand{\Config}{C}
\newcommand{\hConfig}{\hat{C}}
\newcommand{\pidstyle}[1]{\mathrm{{#1}}}
\newcommand{\pid}{\pidstyle{p}}
\newcommand{\pidA}{\pidstyle{1}}
\newcommand{\pidZ}{\pidstyle{q}}

\newcommand{\loc}{l}
\newcommand{\locL}{L}
\newcommand{\locLL}{\mathcal{L}}
\newcommand{\addL}{::}		% add party-wise location-touch tracking together

\newcommand{\code}{d}
\newcommand{\codeL}{D}
\newcommand{\rcodeL}{\RT\codeL}
\newcommand{\codeLL}{\mathcal{D}}
\newcommand{\rcodeLL}{\RT\codeLL}
\newcommand{\addC}{::}		% add party-wise code tracing together
\newcommand{\addMPC}[1]{\mathrm{AddMPC}([\pidA, ..., \pidZ], \mathit{#1})}
\newcommand{\codeMP}[1]{(\mathrm{ALL}, [\mathit{#1}])}
\newcommand{\rcodeMP}[1]{(\RT{\mathrm{ALL}}, [\RT{\mathit{#1}}])}
\newcommand{\codeSP}[1]{(\pid, [\mathit{#1}])}
\newcommand{\rcodeSP}[1]{(\RT\pid, [\RT{{#1}}])}

\newcommand{\GetBytes}{\mathrm{GetBytes}}
\newcommand{\SetBytes}{\mathrm{SetBytes}}

\newcommand{\Mid}{\-\ \Vert\ }
\newcommand{\A}[3]{{#1}^{#2}_{#3}}











% 	Translation Names

\newcommand{\TVar}{T\ Variable}
\newcommand{\TA}{T\ 1D\ Array\ Variable}
\newcommand{\TAA}{T\ 2D\ Array\ Variable}

\newcommand{\TNull}{T\ NULL}
\newcommand{\TN}{T\ Number\ Constant}
\newcommand{\TChar}{T\ Char\ Constant}
\newcommand{\TStr}{T\ String\ Constant}
\newcommand{\TLoc}{T\ Location}
\newcommand{\TList}{T\ List}
\newcommand{\TPtrVal}{T\ Pointer\ Value}
\newcommand{\TSkip}{T\ Skip}

\newcommand{\TMalloc}{T\ Public\ Malloc}
\newcommand{\TPmalloc}{T\ Private\ Malloc}
\newcommand{\TFree}{T\ Public\ Free}
\newcommand{\TPfree}{T\ Private\ Free}
\newcommand{\TSmcopen}{T\ Declassify}
\newcommand{\TSmcin}{T\ Input\ Variable}
\newcommand{\TSmcinA}{T\ Input\ 1D\ Array}
\newcommand{\TSmcinAA}{T\ Input\ 2D\ Array}
\newcommand{\TSmcout}{T\ Output\ Variable}
\newcommand{\TSmcoutA}{T\ Output\ 1D Array}
\newcommand{\TSmcoutAA}{T\ Output\ 2D Array}

\newcommand{\TBinOp}{T\ Binary\ Operation}
\newcommand{\TUnOp}{T\ Unary\ Operation}
\newcommand{\TPostOp}{T\ Postfix\ Operation}
\newcommand{\TPreOp}{T\ Prefix\ Operation}

\newcommand{\TImpDecl}{T\ Implicit\ Type\ Variable\ Declaration}
\newcommand{\TImpPtrDecl}{T\ Implicit\ Type\ Pointer\ Declaration}
\newcommand{\TDecl}{T\ Variable\ Declaration}
\newcommand{\TPtrDecl}{T\ Pointer\ Declaration}

\newcommand{\TType}{T\ Type}
\newcommand{\TCast}{T\ Cast}
\newcommand{\TSizeof}{T\ Size\ Of\ Type}
\newcommand{\TStmtBlock}{T\ Statement\ Block}
\newcommand{\TStmtSeq}{T\ Statement\ Sequencing}
\newcommand{\TParen}{T\ Parentheses}

\newcommand{\TAssign}{T\ Assignment}
\newcommand{\TPtrAssign}{T\ Pointer\ Assignment}
\newcommand{\TIf}{T\ If}
\newcommand{\TIfElse}{T\ If\ Else}
\newcommand{\TWhile}{T\ While}
\newcommand{\TFor}{T\ For}

\newcommand{\TFunDef}{T\ Function\ Definition}
\newcommand{\TFunDecl}{T\ Function\ Declaration}
\newcommand{\TFunCall}{T\ Function\ Call}
\newcommand{\TElist}{T\ Expression\ List}
\newcommand{\TEps}{T\ Epsilon}
\newcommand{\TPlist}{T\ Parameter\ List}
\newcommand{\TVoid}{T\ Void}
\newcommand{\TReturn}{T\ Return}

\newcommand{\TEnv}{T\ Environment}
\newcommand{\THeap}{T\ Memory}
\newcommand{\TVarBytes}{T\ Variable\ Bytes}
\newcommand{\TPtrBytes}{T\ Pointer\ Bytes}
\newcommand{\TPerm}{T\ Permission}
\newcommand{\TLabel}{T\ Label}
\newcommand{\TTlistE}{T\ Empty\ Type\ List}
\newcommand{\TTlist}{T\ Non-empty\ Type\ List}








%  Untranslated Semantics Names

\newcommand{\uruleSseq}{Statement Sequencing}
\newcommand{\uruleSB}{Statement Block}
\newcommand{\uruleParens}{Parentheses}

\newcommand{\uruleCastPLoc}{Cast Private Location}
\newcommand{\uruleCastLoc}{Cast Public Location}
\newcommand{\uruleCastVal}{Cast Public Value}
\newcommand{\uruleCastPVal}{Cast Private Value}

\newcommand{\uruleWrite}{Write Variable}
\newcommand{\uruleConvWrite}{Write Private Variable Public Value}
\newcommand{\uruleRead}{Read Variable}

\newcommand{\uruleDecl}{Public Declaration}
\newcommand{\urulePDecl}{Private Declaration}
\newcommand{\uruleADecl}{Public 1 Dimension Array Declaration}
\newcommand{\urulePADecl}{Private 1 Dimension Array Declaration}
\newcommand{\uruleAADecl}{Public 2 Dimension Array Declaration}
\newcommand{\urulePAADecl}{Private 2 Dimension Array Declaration}

\newcommand{\uruleForInit}{For Initial}
\newcommand{\uruleForCont}{For Continue}
\newcommand{\uruleForEnd}{For End}

\newcommand{\uruleWhileEnd}{While End}
\newcommand{\uruleWhileCont}{While Continue}

\newcommand{\uruleAdd}{Public Addition}
\newcommand{\urulePAdd}{Private Addition}
\newcommand{\urulePubPrivAdd}{Public - Private Addition}
\newcommand{\urulePrivPubAdd}{Private - Public Addition}

\newcommand{\uruleSub}{Public Subtraction}
\newcommand{\urulePSub}{Private Subtraction}
\newcommand{\urulePrivPubSub}{Private - Public Subtraction}
\newcommand{\urulePubPrivSub}{Public - Private Subtraction}

\newcommand{\uruleMult}{Public Multiplication}
\newcommand{\urulePMult}{Private Multiplication}
\newcommand{\urulePrivPubMult}{Private - Public Multiplication}
\newcommand{\urulePubPrivMult}{Public - Private Multiplication}

\newcommand{\uruleDiv}{Public Division}
\newcommand{\urulePDiv}{Private Division}
\newcommand{\urulePrivPubDiv}{Private - Public Division}
\newcommand{\urulePubPrivDiv}{Public - Private Division}

\newcommand{\uruleLT}{Public Less Than True}
\newcommand{\urulePLT}{Private Less Than True}
\newcommand{\urulePubPrivLT}{Public - Private Less Than True}
\newcommand{\urulePrivPubLT}{Private - Public Less Than True}
\newcommand{\uruleLTf}{Public Less Than False}
\newcommand{\urulePLTf}{Private Less Than False}
\newcommand{\urulePubPrivLTf}{Public - Private Less Than False}
\newcommand{\urulePrivPubLTf}{Private - Public Less Than False}

\newcommand{\uruleGT}{Public Greater Than True}
\newcommand{\urulePGT}{Private Greater Than True}
\newcommand{\urulePubPrivGT}{Public - Private Greater Than True}
\newcommand{\urulePrivPubGT}{Private - Public Greater Than True}
\newcommand{\uruleGTf}{Public Greater Than False}
\newcommand{\urulePGTf}{Private Greater Than False}
\newcommand{\urulePubPrivGTf}{Public - Private Greater Than False}
\newcommand{\urulePrivPubGTf}{Private - Public Greater Than False}

\newcommand{\uruleLTEQ}{Public Less Than Equal To True}
\newcommand{\urulePLTEQ}{Private Less Than Equal To True}
\newcommand{\urulePubPrivLTEQ}{Public - Private Less Than Equal To True}
\newcommand{\urulePrivPubLTEQ}{Private - Public Less Than Equal To True}
\newcommand{\uruleLTEQf}{Public Less Than Equal To False}
\newcommand{\urulePLTEQf}{Private Less Than Equal To False}
\newcommand{\urulePubPrivLTEQf}{Public - Private Less Than Equal To False}
\newcommand{\urulePrivPubLTEQf}{Private - Public Less Than Equal To False}

\newcommand{\uruleGTEQ}{Public Greater Than Equal To True}
\newcommand{\urulePGTEQ}{Private Greater Than Equal To True}
\newcommand{\urulePubPrivGTEQ}{Public - Private Greater Than Equal To True}
\newcommand{\urulePrivPubGTEQ}{Private - Public Greater Than Equal To True}
\newcommand{\uruleGTEQf}{Public Greater Than Equal To False}
\newcommand{\urulePGTEQf}{Private Greater Than Equal To False}
\newcommand{\urulePubPrivGTEQf}{Public - Private Greater Than Equal To False}
\newcommand{\urulePrivPubGTEQf}{Private - Public Greater Than Equal To False}

\newcommand{\uruleEQ}{Public Equal To True}
\newcommand{\urulePEQ}{Private Equal To True}
\newcommand{\urulePubPrivEQ}{Public - Private Equal To True}
\newcommand{\urulePrivPubEQ}{Private - Public Equal To True}
\newcommand{\uruleEQf}{Public Equal To False}
\newcommand{\urulePEQf}{Private Equal To False}
\newcommand{\urulePubPrivEQf}{Public - Private Equal To False}
\newcommand{\urulePrivPubEQf}{Private - Public Equal To False}

\newcommand{\uruleNEQ}{Public Not Equal To True}
\newcommand{\urulePNEQ}{Private Not Equal To True}
\newcommand{\urulePubPrivNEQ}{Public - Private Not Equal To True}
\newcommand{\urulePrivPubNEQ}{Private - Public Not Equal To True}
\newcommand{\uruleNEQf}{Public Not Equal To False}
\newcommand{\urulePNEQf}{Private Not Equal To False}
\newcommand{\urulePubPrivNEQf}{Public - Private Not Equal To False}
\newcommand{\urulePrivPubNEQf}{Private - Public Not Equal To False}

\newcommand{\uruleAnd}{Public And True}
\newcommand{\urulePAnd}{Private And True}
\newcommand{\urulePubPrivAnd}{Public - Private And True}
\newcommand{\urulePrivPubAnd}{Private - Public And True}
\newcommand{\uruleAndf}{Public And False}
\newcommand{\urulePAndf}{Private And False}
\newcommand{\urulePubPrivAndf}{Public - Private And False}
\newcommand{\urulePrivPubAndf}{Private - Public And False}

\newcommand{\uruleOr}{Public Or True}
\newcommand{\urulePOr}{Private Or True}
\newcommand{\urulePubPrivOr}{Public - Private Or True}
\newcommand{\urulePrivPubOr}{Private - Public Or True}
\newcommand{\uruleOrf}{Public Or False}
\newcommand{\urulePOrf}{Private Or False}
\newcommand{\urulePubPrivOrf}{Public - Private Or False}
\newcommand{\urulePrivPubOrf}{Private - Public Or False}

\newcommand{\uruleBitAnd}{Bitwise And}
\newcommand{\urulePubPrivBitAnd}{Public - Private Bitwise And}
\newcommand{\urulePrivPubBitAnd}{Private - Public Bitwise And}
\newcommand{\uruleBitOr}{Bitwise Or}
\newcommand{\urulePubPrivBitOr}{Public - Private Bitwise Or}
\newcommand{\urulePrivPubBitOr}{Private - Public Bitwise Or}

\newcommand{\uruleDeclassify}{Declassification}
\newcommand{\uruleDeclassifyPtr}{Pointer Declassification}

\newcommand{\uruleMalloc}{Malloc}
\newcommand{\urulePMalloc}{Private Malloc}
\newcommand{\uruleFree}{Free}
\newcommand{\urulePFree}{Private Free}

\newcommand{\urulePtrWriteLocHH}{Pointer Assignment Statement}
\newcommand{\urulePtrWriteLocH}{Pointer Write Single Location Higher Level Indirection}

\newcommand{\urulePtrWriteLoc}{Pointer Write Single Location Single Level Indirection}
\newcommand{\urulePPtrWriteLocH}{Private Pointer Write Single Location Higher Level Indirection}
\newcommand{\urulePPtrWL}{Private Pointer Write Single Location Single Level Indirection}
\newcommand{\urulePPtrWML}{Private Pointer Write Multiple Locations Single Level Indirection}
\newcommand{\urulePPtrWMLH}{Private Pointer Write Multiple Locations Higher  Level Indirection}

\newcommand{\urulePPtrWV}{Private Pointer Dereference Write Public Value}
\newcommand{\urulePPtrWPV}{Private Pointer Dereference Write Private Value}
\newcommand{\urulePtrWrite}{Public Pointer Dereference Write Public Value}
\newcommand{\urulePtrWriteH}{Public Pointer Dereference Write Higher Level Indirection}
\newcommand{\urulePPtrWriteH}{Private Pointer Dereference Write Value Higher Level Indirection}
\newcommand{\urulePtrReadLoc}{Pointer Read Single Location}
\newcommand{\urulePPtrReadM}{Private Pointer Read Multiple Locations}

\newcommand{\uruleArrR}{1D Array Read Public Index}
\newcommand{\urulePArrR}{Private 1D Array Read Private Index}
\newcommand{\uruleArrPR}{Public 1D Array Read Private Index}

\newcommand{\uruleArrW}{1D Array Write Public Index}
\newcommand{\urulePArrW}{Private 1D Array Write Public Value Public Index}
\newcommand{\urulePArrWP}{Private 1D Array Write Public Value Private Index}
\newcommand{\urulePArrWPP}{Private 1D Array Write Private Value Private Index}

\newcommand{\uruleOOBArrR}{1D Array Read Out of Bounds Public Index}
\newcommand{\uruleOOBArrW}{1D Array Write Out of Bounds Public Index}
\newcommand{\uruleOOBPArrW}{Private 1D Array Write Public Value Out of Bounds Public Index}

\newcommand{\uruleAArrR}{2D Array Read Public Index}
\newcommand{\urulePAArrRPubPriv}{Private 2D Array Read Public-Private Index}
\newcommand{\uruleAArrRPubPriv}{Public 2D Array Read Public-Private Index}
\newcommand{\urulePAArrRPrivPub}{Private 2D Array Read Private-Public Index}
\newcommand{\uruleAArrRPrivPub}{Public 2D Array Read Private-Public Index}
\newcommand{\urulePAArrPR}{Private 2D Array Read Private Index}
\newcommand{\uruleAArrPR}{Public 2D Array Read Private Index}
\newcommand{\uruleAArrROOB}{Public 2D Array Read Public Index Out Of Bounds}
\newcommand{\urulePAArrROOB}{Private 2D Array Read Public Index Out Of Bounds}
\newcommand{\uruleAArrRPubPrivOOB}{Public 2D Array Read Public - Private Index Out Of Bounds}
\newcommand{\urulePAArrRPubPrivOOB}{Private 2D Array Read Public - Private Index Out Of Bounds}
\newcommand{\uruleAArrRPrivPubOOB}{Public 2D Array Read Private - Public Index Out Of Bounds}
\newcommand{\urulePAArrRPrivPubOOB}{Private 2D Array Read Private - Public Index Out Of Bounds}

\newcommand{\uruleAArrW}{Public 2D Array Write Public Value Public Index}
\newcommand{\urulePAArrW}{Private 2D Array Write Public Value Public Index}
\newcommand{\urulePAArrWP}{Private 2D Array Write Private Value Public Index}
\newcommand{\urulePAArrWPubPriv}{Private 2D Array Write Public Value Public - Private Index}
\newcommand{\urulePAArrWPPubPriv}{Private 2D Array Write Private Value Public - Private Index}
\newcommand{\urulePAArrWPrivPub}{Private 2D Array Write Public Value Private - Public Index}
\newcommand{\urulePAArrWPPrivPub}{Private 2D Array Write Private Value Private - Public Index}
\newcommand{\urulePAArrWPriv}{Private 2D Array Write Public Value Private Index}
\newcommand{\urulePAArrWPP}{Private 2D Array Write Private Value Private Index}
\newcommand{\uruleAArrWOOB}{Public 2D Array Write Public Value Out of Bounds Public Index}
\newcommand{\urulePAArrWOOB}{Private 2D Array Write Public Value Out of Bounds Public Index}
\newcommand{\urulePAArrWPOOB}{Private 2D Array Write Private Value Out of Bounds Public Index}
\newcommand{\urulePAArrPubPrivWOOB}{Private 2D Array Write Public Value Out of Bounds Public-Private Index}
\newcommand{\urulePAArrPubPrivWPOOB}{Private 2D Array Write Private Value Out of Bounds Public-Private Index}
\newcommand{\urulePAArrPrivPubWOOB}{Private 2D Array Write Public Value Out of Bounds Private - Public Index}
\newcommand{\urulePAArrPrivPubWPOOB}{Private 2D Array Write Private Value Out of Bounds Private - Public Index}

\newcommand{\uruleNeg}{Public Negation}
\newcommand{\urulePNeg}{Private Negation}
\newcommand{\uruleNot}{Public Not True}
\newcommand{\urulePNot}{Private Not True}
\newcommand{\uruleNotF}{Public Not False}
\newcommand{\urulePNotF}{Private Not False}
\newcommand{\uruleLoc}{Address Of}

\newcommand{\urulePtrReadVal}{Pointer Single Location Dereference Single Level Indirection}
\newcommand{\urulePtrDeref}{Pointer Single Location Dereference Higher Level Indirection}
\newcommand{\urulePPtrReadV}{Private Pointer Dereference Single Level Indirection}
\newcommand{\urulePPtrReadVH}{Private Pointer Dereference Higher Level Indirection}

\newcommand{\urulePreIncVar}{Pre-Increment Public Variable}
\newcommand{\urulePreIncPVar}{Pre-Increment Private Variable}
\newcommand{\urulePreDecVar}{Pre-Decrement Public Variable}
\newcommand{\urulePreDecPVar}{Pre-Decrement Private Variable}
\newcommand{\urulePreIncPtrS}{Pre-Increment Pointer Single Level Indirection Single Location}
\newcommand{\urulePreIncPtrSH}{Pre-Increment Pointer Higher Level Indirection Single Location}
\newcommand{\urulePreIncPtr}{Pre-Increment Pointer Single Level Indirection Multiple Locations}
\newcommand{\urulePreIncPtrH}{Pre-Increment Pointer Higher Level Indirection Multiple Locations}
\newcommand{\urulePreDecPtrS}{Pre-Decrement Pointer Single Level Indirection Single Location}
\newcommand{\urulePreDecPtrSH}{Pre-Decrement Pointer Higher Level Indirection Single Location}
\newcommand{\urulePreDecPtr}{Pre-Decrement Pointer Single Level Indirection Multiple Locations}
\newcommand{\urulePreDecPtrH}{Pre-Decrement Pointer Higher Level Indirection Multiple Locations}

\newcommand{\urulePostIncVar}{Post-Increment Public Variable}
\newcommand{\urulePostIncPVar}{Post-Increment Private Variable}
\newcommand{\urulePostDecVar}{Post-Decrement Public Variable}
\newcommand{\urulePostDecPVar}{Post-Decrement Private Variable}
\newcommand{\urulePostIncPtrS}{Post-Increment Pointer Single Level Indirection Single Location}
\newcommand{\urulePostIncPtrSH}{Post-Increment Pointer Higher Level Indirection Single Location}
\newcommand{\urulePostIncPtr}{Post-Increment Pointer Single Level Indirection Multiple Locations}
\newcommand{\urulePostIncPtrH}{Post-Increment Pointer Higher Level Indirection Multiple Locations}
\newcommand{\urulePostDecPtrS}{Post-Decrement Pointer Single Level Indirection Single Location}
\newcommand{\urulePostDecPtrSH}{Post-Decrement Pointer Higher Level Indirection Single Location}
\newcommand{\urulePostDecPtr}{Post-Decrement Pointer Single Level Indirection Multiple Locations}
\newcommand{\urulePostDecPtrH}{Post-Decrement Pointer Higher Level Indirection Multiple Locations}

\newcommand{\uruleIfT}{Public If True Statement}
\newcommand{\uruleIfF}{Public If False Statement}
\newcommand{\uruleIfElseT}{Public If Else True Statement}
\newcommand{\uruleIfElseF}{Public If Else False Statement}
\newcommand{\uruleIfP}{Private If Statement}
\newcommand{\urulePrivIfElse}{Private If Else Statement}

\newcommand{\uruleResVal}{Resolve Variables - Value}
\newcommand{\uruleResPtr}{Resolve Variables - Pointer}
\newcommand{\uruleResArr}{Resolve Variables - Array}
\newcommand{\uruleResVar}{Resolve Variables - Empty}
\newcommand{\uruleRestore}{Restore Variables}
\newcommand{\uruleRestoreE}{Restore Variables - Empty}
\newcommand{\uruleInitialize}{Initialize Variables}
\newcommand{\uruleInitializeE}{Initialize Variables - Empty}
\newcommand{\uruleExtract}{Extract Variables}

\newcommand{\uruleSizeofTy}{Size of type}
\newcommand{\uruleEncrypt}{Encrypt}

\newcommand{\uruleSmcinput}{SMC Input Value}
\newcommand{\uruleSmcinputArr}{SMC Input 1D Array}
\newcommand{\uruleSmcinputAArr}{SMC Input 2D Array}
\newcommand{\uruleSmcoutput}{SMC Output Value}
\newcommand{\uruleSmcoutputArr}{SMC Output 1D Array}
\newcommand{\uruleSmcoutputAArr}{SMC Output 2D Array}

\newcommand{\uruleReturnV}{Return}
\newcommand{\uruleReturnSB}{Return Statement Block}
\newcommand{\uruleReturnSS}{Return Statement Sequencing}

\newcommand{\uruleFunctionDecl}{Function Declaration}
\newcommand{\uruleFunctionDef}{Function Definition}
\newcommand{\uruleFunctionPreDef}{Pre-Declared Function Definition}

\newcommand{\uruleFunctionCall}{Function Call Without Public Side Effects}
\newcommand{\uruleFunctionCallPub}{Function Call With Public Side Effects}
\newcommand{\uruleFunctionCallNR}{Function Call No Return Without Public Side Effects}
\newcommand{\uruleFunctionCallNRPub}{Function Call No Return With Public Side Effects}

\newcommand{\uruleFunctionArg}{Function Argument Assignment}
\newcommand{\uruleFunctionSArg}{Function Single Argument Assignment}
\newcommand{\uruleFunctionEArg}{Function Empty Argument Assignment}

\newcommand{\uruleParamTy}{Parameter List Get Type}
\newcommand{\uruleParamTyS}{Single Parameter Get Type}
\newcommand{\uruleParamTyE}{Empty Parameter List Get Type}

%%%%%%%%%%%%%%%%%%
%
%     Translated Rule Names
%
%%%%%%%%%%%%%%%%%%

\newcommand{\ruleSseq}{Translated Statement Sequencing}
\newcommand{\ruleSB}{Translated Statement Block}
\newcommand{\ruleParens}{Translated Parentheses}

\newcommand{\ruleCastPLoc}{Translated Cast Private Location}
\newcommand{\ruleCastLoc}{Translated Cast Public Location}
\newcommand{\ruleCastVal}{Translated Cast Public Value}
\newcommand{\ruleCastPVal}{Translated Cast Private Value}

\newcommand{\ruleWrite}{Translated Write Variable}
\newcommand{\ruleRead}{Translated Read Variable}

\newcommand{\ruleDeclAssign}{Declaration Assignment}
\newcommand{\ruleDecl}{Translated Public Declaration}
\newcommand{\rulePDecl}{Translated Private Declaration}
\newcommand{\ruleADecl}{Translated Public 1 Dimension Array Declaration}
\newcommand{\rulePADecl}{Translated Private 1 Dimension Array Declaration}
\newcommand{\ruleAADecl}{Translated Public 2 Dimension Array Declaration}
\newcommand{\rulePAADecl}{Translated Private 2 Dimension Array Declaration}

\newcommand{\ruleForInit}{Translated For Initial}
\newcommand{\ruleForCont}{Translated For Continue}
\newcommand{\ruleForEnd}{Translated For End}
\newcommand{\ruleWhileEnd}{Translated While End}
\newcommand{\ruleWhileCont}{Translated While Continue}

\newcommand{\ruleAdd}{Translated Public Addition}
\newcommand{\rulePAdd}{Translated Private Addition}
\newcommand{\ruleSub}{Translated Subtraction}
\newcommand{\rulePSub}{Translated Private Subtraction}
\newcommand{\ruleMult}{Translated Multiplication}
\newcommand{\rulePMult}{Translated Private Multiplication}
\newcommand{\ruleDiv}{Translated Division}
\newcommand{\rulePDiv}{Translated Private Division}
\newcommand{\ruleLTf}{Translated Less Than False}
\newcommand{\rulePLTf}{Translated Private Less Than False}
\newcommand{\ruleLT}{Translated Public Less Than True}
\newcommand{\rulePLT}{Translated Private Less Than True}
\newcommand{\ruleGTf}{Translated Greater Than False}
\newcommand{\rulePGTf}{Translated Private Greater Than False}
\newcommand{\ruleGT}{Translated Greater Than True}
\newcommand{\rulePGT}{Translated Private Greater Than True}
\newcommand{\ruleLTEQf}{Translated Less Than Equal To False}
\newcommand{\rulePLTEQf}{Translated Private Less Than Equal To False}
\newcommand{\ruleLTEQ}{Translated Less Than Equal To True}
\newcommand{\rulePLTEQ}{Translated Private Less Than Equal To True}
\newcommand{\ruleGTEQf}{Translated Greater Than Equal To False}
\newcommand{\rulePGTEQf}{Translated Private Greater Than Equal To False}
\newcommand{\ruleGTEQ}{Translated Greater Than Equal To True}
\newcommand{\rulePGTEQ}{Translated Private Greater Than Equal To True}
\newcommand{\ruleEQf}{Translated Equal To False}
\newcommand{\rulePEQf}{Translated Private Equal To False}
\newcommand{\ruleEQ}{Translated Equal To True}
\newcommand{\rulePEQ}{Translated Private Equal To True}
\newcommand{\ruleNEQf}{Translated Not Equal To False}
\newcommand{\rulePNEQf}{Translated Private Not Equal To False}
\newcommand{\ruleNEQ}{Translated Not Equal To True}
\newcommand{\rulePNEQ}{Translated Private Not Equal To True}
\newcommand{\ruleAndf}{Translated And False}
\newcommand{\rulePAndf}{Translated Private And False}
\newcommand{\ruleAnd}{Translated And True}
\newcommand{\rulePAnd}{Translated Private And True}
\newcommand{\ruleOrf}{Translated Or False}
\newcommand{\rulePOrf}{Translated Private Or False}
\newcommand{\ruleOr}{Translated Or True}
\newcommand{\rulePOr}{Translated Private Or True}
\newcommand{\ruleBitAnd}{Translated Bitwise And}
\newcommand{\rulePBitAnd}{Translated Private Bitwise And}
\newcommand{\ruleBitOr}{Translated Bitwise Or}
\newcommand{\rulePBitOr}{Translated Private Bitwise Or}

\newcommand{\ruleDeclassify}{Translated Declassification}
\newcommand{\ruleDeclassifyPtr}{Translated Pointer Declassification}

\newcommand{\ruleMalloc}{Translated Malloc}
\newcommand{\rulePMalloc}{Translated Private Malloc}
\newcommand{\ruleFree}{Translated Free}
\newcommand{\rulePFree}{Translated Private Free}

\newcommand{\rulePtrWriteLoc}{Translated Pointer Single Location Write Location Single Level Indirection}
\newcommand{\rulePtrWriteLocH}{Translated Pointer Write Single Location Higher Level Indirection}
\newcommand{\rulePPtrWriteLocH}{Translated Private Pointer Write Single Location Higher Level Indirection}
\newcommand{\rulePPtrWL}{Translated Private Pointer Write Single Location Single Level Indirection}
\newcommand{\rulePPtrWML}{Translated Private Pointer Write Multiple Locations Single Level Indirection}
\newcommand{\rulePPtrWMLH}{Translated Private Pointer Write Multiple Locations Higher Level Indirection}

\newcommand{\rulePPtrWPV}{Translated Private Pointer Dereference Write Private Value}
\newcommand{\rulePtrWrite}{Translated Public Pointer Dereference Write Public Value}
\newcommand{\rulePtrWriteH}{Translated Public Pointer Dereference Write Value Higher Level Indirection}
\newcommand{\rulePPtrWriteH}{Translated Private Pointer Dereference Write Value Higher Level Indirection}

\newcommand{\rulePtrReadLoc}{Translated Pointer Read Single Location}
\newcommand{\rulePPtrReadM}{Translated Private Pointer Read Multiple Locations}
\newcommand{\rulePtrReadVal}{Translated Pointer Single Location Dereference Single Level Indirection}
\newcommand{\rulePtrDeref}{Translated Pointer Single Location Dereference Higher Level Indirection}
\newcommand{\rulePPtrReadV}{Translated Private Pointer Dereference Single Level Indirection}
\newcommand{\rulePPtrReadVH}{Translated Private Pointer Dereference Higher Level Indirection}

\newcommand{\ruleArrR}{Translated 1D Array Read Public Index}
\newcommand{\rulePArrR}{Translated Private 1D Array Read Private Index}
\newcommand{\ruleArrPR}{Translated Public 1D Array Read Private Index}

\newcommand{\ruleArrW}{Translated 1D Array Write Public Index}
\newcommand{\rulePArrW}{Translated Private 1D Array Write Public Index}
\newcommand{\rulePArrWPP}{Translated Private 1D Array Write Private Value Private Index}

\newcommand{\ruleOOBArrR}{Translated 1D Array Read Out of Bounds Public Index}
\newcommand{\ruleOOBArrW}{Translated 1D Array Write Out of Bounds Public Index}
\newcommand{\ruleOOBPArrW}{Translated Private 1D Array Write Out of Bounds Public Index}

\newcommand{\ruleAArrR}{Translated 2D Array Read Public Index}
\newcommand{\rulePAArrRPubPriv}{Translated Private 2D Array Read Public - Private Index}
\newcommand{\ruleAArrRPubPriv}{Translated Public 2D Array Read Public - Private Index}
\newcommand{\rulePAArrRPrivPub}{Translated Private 2D Array Read Private - Public Index}
\newcommand{\ruleAArrRPrivPub}{Translated Public 2D Array Read Private - Public Index}
\newcommand{\rulePAArrPR}{Translated Private 2D Array Read Private Index}
\newcommand{\ruleAArrPR}{Translated Public 2D Array Read Private Index}

\newcommand{\ruleAArrROOB}{Public 2D Array Read Public Index Out Of Bounds}
\newcommand{\rulePAArrROOB}{Private 2D Array Read Public Index Out Of Bounds}
\newcommand{\ruleAArrRPubPrivOOB}{Public 2D Array Read Public - Private Index Out Of Bounds}
\newcommand{\rulePAArrRPubPrivOOB}{Private 2D Array Read Public - Private Index Out Of Bounds}
\newcommand{\ruleAArrRPrivPubOOB}{Public 2D Array Read Private - Public Index Out Of Bounds}
\newcommand{\rulePAArrRPrivPubOOB}{Private 2D Array Read Private - Public Index Out Of Bounds}

\newcommand{\ruleAArrW}{Public 2D Array Write Public Value Public Index}
\newcommand{\rulePAArrWP}{Private 2D Array Write Private Value Public Index}
\newcommand{\rulePAArrWPPubPriv}{Private 2D Array Write Private Value Public - Private Index}
\newcommand{\rulePAArrWPPrivPub}{Private 2D Array Write Private Value Private - Public Index}
\newcommand{\rulePAArrWPP}{Private 2D Array Write Private Value Private Index}

\newcommand{\ruleAArrWOOB}{Translated Public 2D Array Write Public Value Out of Bounds Public Index}
\newcommand{\rulePAArrWPOOB}{Translated Private 2D Array Write Private Value Out of Bounds Public Index}
\newcommand{\rulePAArrPubPrivWPOOB}{Translated Private 2D Array Write Private Value Out of Bounds Public-Private Index}
\newcommand{\rulePAArrPrivPubWPOOB}{Translated Private 2D Array Write Private Value Out of Bounds Private - Public Index}

\newcommand{\ruleNeg}{Translated Public Negation}
\newcommand{\rulePNeg}{Translated Private Negation}
\newcommand{\ruleNot}{Translated Public Not True}
\newcommand{\rulePNot}{Translated Private Not True}
\newcommand{\ruleNotF}{Translated Public Not False}
\newcommand{\rulePNotF}{Translated Private Not False}

\newcommand{\ruleLoc}{Translated Address Of}

\newcommand{\rulePreIncVar}{Translated Pre-Increment Public Variable}
\newcommand{\rulePreIncPVar}{Translated Pre-Increment Private Variable}
\newcommand{\rulePreDecVar}{Translated Pre-Decrement Variable}
\newcommand{\rulePreDecPVar}{Translated Pre-Decrement Private Variable}
\newcommand{\rulePreIncPtrS}{Translated Pre-Increment Pointer Single Level Indirection Single Location}
\newcommand{\rulePreIncPtrSH}{Translated Pre-Increment Pointer Higher Level Indirection Single Location}
\newcommand{\rulePreIncPtr}{Translated Pre-Increment Pointer Single Level Indirection Multiple Locations}
\newcommand{\rulePreIncPtrH}{Translated Pre-Increment Pointer Higher Level Indirection Multiple Locations}
\newcommand{\rulePreDecPtr}{Translated Pre-Decrement Pointer Single Level Indirection}
\newcommand{\rulePreDecPtrH}{Translated Pre-Decrement Pointer Higher Level Indirection}

\newcommand{\rulePostIncVar}{Translated Post-Increment Variable}
\newcommand{\rulePostDecVar}{Translated Post-Decrement Variable}
\newcommand{\rulePostIncPtr}{Translated Post-Increment Pointer Single Level Indirection}
\newcommand{\rulePostIncPtrH}{Translated Post-Increment Pointer Higher Level Indirection}
\newcommand{\rulePostDecPtr}{Translated Post-Decrement Pointer Single Level Indirection}
\newcommand{\rulePostDecPtrH}{Translated Post-Decrement Pointer Higher Level Indirection}

\newcommand{\ruleIfT}{Translated Public If True}
\newcommand{\ruleIfF}{Translated Public If False}
\newcommand{\ruleIfElseT}{Translated Public If Else True}
\newcommand{\ruleIfElseF}{Translated Public If Else False}
\newcommand{\ruleTPrivIf}{Translated Translated Private If}
\newcommand{\ruleA}{A}
\newcommand{\ruleB}{B}
\newcommand{\ruleTPrivIfElse}{Translated Private If Else}

\newcommand{\ruleResVal}{Translated Resolve Variables - Value}
\newcommand{\ruleResPtr}{Translated Resolve Variables - Pointer}
\newcommand{\ruleResArr}{Translated Resolve Variables - Array}
\newcommand{\ruleResVar}{Translated Resolve Variables}
\newcommand{\ruleRestore}{Translated Restore Variables}
\newcommand{\ruleRestoreE}{Translated Restore Variables - Empty}
\newcommand{\ruleInitialize}{Translated Initialize Variables}
\newcommand{\ruleInitializeE}{Translated Initialize Variables - Empty}
\newcommand{\ruleExtract}{Translated Extract Variables}

\newcommand{\ruleSizeofTy}{Translated Size of type}
\newcommand{\ruleSizeofVar}{Translated Size of variable}
\newcommand{\ruleEncrypt}{Translated Encrypt}

\newcommand{\ruleSmcinput}{Translated SMC Input Value}
\newcommand{\ruleSmcinputArr}{Translated SMC Input 1D Array}
\newcommand{\ruleSmcinputAArr}{Translated SMC Input 2D Array}
\newcommand{\ruleSmcoutput}{Translated SMC Output Value}
\newcommand{\ruleSmcoutputArr}{Translated SMC Output 1D Array}
\newcommand{\ruleSmcoutputAArr}{Translated SMC Output 2D Array}

\newcommand{\ruleReturnV}{Translated Return}
\newcommand{\ruleReturnSB}{Translated Return Statement Block}
\newcommand{\ruleReturnSS}{Translated Return Statement Sequencing}

\newcommand{\ruleFunctionDecl}{Translated Function Declaration}
\newcommand{\ruleFunctionDef}{Translated Function Definition}
\newcommand{\ruleFunctionPreDef}{Translated Pre-Declared Function Definition}

\newcommand{\ruleFunctionCall}{Translated Function Call Without Public Side Effects}
\newcommand{\ruleFunctionCallPub}{Translated Function Call With Public Side Effects}
\newcommand{\ruleFunctionCallNR}{Translated Function Call No Return Without Public Side Effects}
\newcommand{\ruleFunctionCallNRPub}{Translated Function Call No Return With Public Side Effects}

\newcommand{\ruleFunctionArg}{Translated Function Argument Assignment}
\newcommand{\ruleFunctionSArg}{Translated Function Single Argument Assignment}
\newcommand{\ruleFunctionEArg}{Translated Function Empty Argument Assignment}

\newcommand{\ruleParamTy}{Translated Parameter List Get Type}
\newcommand{\ruleParamTyS}{Translated Single Parameter Get Type}
\newcommand{\ruleParamTyE}{Translated Empty Parameter List Get Type}











\begin{document}

\lstset{language=C, basicstyle=\footnotesize\ttfamily, numbers=left, numbersep=5pt, tabsize=2,mathescape=true}
\lstset{emph={[1]public,private,smcinput,smcoutput,smcopen,pfree,pmalloc},emphstyle={[1]\color{red}}}

\maketitle

\begin{abstract}
Although Secure Multiparty Computation (SMC) has seen considerable development in recent years, its use is challenging, resulting in complex code which obscures whether the security properties or correctness guarantees hold in practice. 
For this reason, several works have investigated the use of formal methods  to provide guarantees for
SMC systems. However, these approaches have been 
applied mostly to domain specific languages (DSL),
neglecting general-purpose approaches. 
In this paper, we consider a formal
model for an SMC system for  annotated C programs. 
We choose C due to its popularity
in the cryptographic community and being the only general-purpose language
for which SMC compilers exist.
Our formalization supports all
key features of C -- including private-conditioned branching statements, mutable arrays
(including out of bound array access), pointers to private data, etc. 
We use this formalization to  characterize correctness and security properties of annotated C, with the
latter being a form of non-interference on execution traces.  We realize our formalism as an implementation in the PICCO SMC compiler and provide
evaluation results on SMC programs written in C.
\end{abstract}

\section{Introduction} \label{Sec: Introduction}
% !TEX root = ../arxiv.tex

Unsupervised domain adaptation (UDA) is a variant of semi-supervised learning \cite{blum1998combining}, where the available unlabelled data comes from a different distribution than the annotated dataset \cite{Ben-DavidBCP06}.
A case in point is to exploit synthetic data, where annotation is more accessible compared to the costly labelling of real-world images \cite{RichterVRK16,RosSMVL16}.
Along with some success in addressing UDA for semantic segmentation \cite{TsaiHSS0C18,VuJBCP19,0001S20,ZouYKW18}, the developed methods are growing increasingly sophisticated and often combine style transfer networks, adversarial training or network ensembles \cite{KimB20a,LiYV19,TsaiSSC19,Yang_2020_ECCV}.
This increase in model complexity impedes reproducibility, potentially slowing further progress.

In this work, we propose a UDA framework reaching state-of-the-art segmentation accuracy (measured by the Intersection-over-Union, IoU) without incurring substantial training efforts.
Toward this goal, we adopt a simple semi-supervised approach, \emph{self-training} \cite{ChenWB11,lee2013pseudo,ZouYKW18}, used in recent works only in conjunction with adversarial training or network ensembles \cite{ChoiKK19,KimB20a,Mei_2020_ECCV,Wang_2020_ECCV,0001S20,Zheng_2020_IJCV,ZhengY20}.
By contrast, we use self-training \emph{standalone}.
Compared to previous self-training methods \cite{ChenLCCCZAS20,Li_2020_ECCV,subhani2020learning,ZouYKW18,ZouYLKW19}, our approach also sidesteps the inconvenience of multiple training rounds, as they often require expert intervention between consecutive rounds.
We train our model using co-evolving pseudo labels end-to-end without such need.

\begin{figure}[t]%
    \centering
    \def\svgwidth{\linewidth}
    \input{figures/preview/bars.pdf_tex}
    \caption{\textbf{Results preview.} Unlike much recent work that combines multiple training paradigms, such as adversarial training and style transfer, our approach retains the modest single-round training complexity of self-training, yet improves the state of the art for adapting semantic segmentation by a significant margin.}
    \label{fig:preview}
\end{figure}

Our method leverages the ubiquitous \emph{data augmentation} techniques from fully supervised learning \cite{deeplabv3plus2018,ZhaoSQWJ17}: photometric jitter, flipping and multi-scale cropping.
We enforce \emph{consistency} of the semantic maps produced by the model across these image perturbations.
The following assumption formalises the key premise:

\myparagraph{Assumption 1.}
Let $f: \mathcal{I} \rightarrow \mathcal{M}$ represent a pixelwise mapping from images $\mathcal{I}$ to semantic output $\mathcal{M}$.
Denote $\rho_{\bm{\epsilon}}: \mathcal{I} \rightarrow \mathcal{I}$ a photometric image transform and, similarly, $\tau_{\bm{\epsilon}'}: \mathcal{I} \rightarrow \mathcal{I}$ a spatial similarity transformation, where $\bm{\epsilon},\bm{\epsilon}'\sim p(\cdot)$ are control variables following some pre-defined density (\eg, $p \equiv \mathcal{N}(0, 1)$).
Then, for any image $I \in \mathcal{I}$, $f$ is \emph{invariant} under $\rho_{\bm{\epsilon}}$ and \emph{equivariant} under $\tau_{\bm{\epsilon}'}$, \ie~$f(\rho_{\bm{\epsilon}}(I)) = f(I)$ and $f(\tau_{\bm{\epsilon}'}(I)) = \tau_{\bm{\epsilon}'}(f(I))$.

\smallskip
\noindent Next, we introduce a training framework using a \emph{momentum network} -- a slowly advancing copy of the original model.
The momentum network provides stable, yet recent targets for model updates, as opposed to the fixed supervision in model distillation \cite{Chen0G18,Zheng_2020_IJCV,ZhengY20}.
We also re-visit the problem of long-tail recognition in the context of generating pseudo labels for self-supervision.
In particular, we maintain an \emph{exponentially moving class prior} used to discount the confidence thresholds for those classes with few samples and increase their relative contribution to the training loss.
Our framework is simple to train, adds moderate computational overhead compared to a fully supervised setup, yet sets a new state of the art on established benchmarks (\cf \cref{fig:preview}).


\section{Related Work}
\paragraph*{SMC compilers}
Work on SMC compilers was initiated in 2004 and a significant body of work has been developed. Notable examples include two-party computation compilers and tools Fairplay~\cite{Malkhi04}, TASTY~\cite{Henecka10}, ABY~\cite{Demmler15a}, PCF~\cite{Kreuter13}, TinyGarble~\cite{Songhori15}, Frigate~\cite{Mood16}, SCVM~\cite{Liu14}, and ObliVM~\cite{Liu15}; three-party Sharemind~\cite{Bogdanov08}; and multi-party FairplayMP~\cite{BenDavid08},  VIFF~\cite{DamgardGKN09}, and more recently SCALE-MAMBA, which evolved from \cite{BendlinDOZ11,DamgardPSZ12,NielsenNOB12}.
These compilers use custom DSLs to represent user programs, and notable exceptions are CBMC-GC~\cite{Holzer12} (intended to support general-purpose ANSI-C programs in the two-party setting, but not all features were realized at the time) and PICCO~\cite{Zhang13,Zhang18} (takes programs written in an extension of C, supports all C features, and produces multi-party protocols).
%
The above compilers did not come with a formalism of their type
system~\footnote{The ObliVM publication~\cite{Liu15} suggests that
there is a type system behind the ObliVM language, but no further
information could be found.}, while this was later developed for
Sharemind~\cite{sokk16}. There are also SMC DSLs with formal models,
such as Wysteria~\cite{RastogiHH14} with a formal model based on an
operational semantics and Wys*~\cite{RastogiSH17} which provides
support for SMC by means of an embedded DSL hosted in F*, a
dependently typed language supporting full verification. A different
approach is given in~\cite{PettaiL15} with an automated technique to
prove SMC protocols secure.

We provide a summary of significant features supported
in recent compilers in Table~\ref{tab:smc-compilers} (Wys*~\cite{RastogiSH17} inherits its expressivity from Wysteria and is
omitted).
\begin{table} \small \centering \setlength{\tabcolsep}{1ex} \footnotesize
\begin{tabular}{|c|c|c|c|c|c|c|} \hline
  \multirow{3}{*}{Compiler} & \multicolumn{6}{|c|}{Supported features} \\\cline{2-7}
  & \multirow{2}{*}{loops} & priv.  & mixed & float. & dyn. & seman. \\
  & & cond. & mode & point & mem. & formal. \\ \hline
  Fairplay~\cite{Malkhi04} & \hc & \fc & \ec & \ec & \ec & \ec \\ \hline
  Sharemind~\cite{Bogdanov08,jagomagis2010secrec} & \fc & \hc & \fc & \fc & \ec & \fc \\ \hline
  CBMC-GC~\cite{Holzer12} & \hc & \fc & \ec & \hc & \ec & \ec \\ \hline 
  PICCO~\cite{Zhang13,Zhang18} & \fc & \fc & \fc & \fc & \fc & \ec \\ \hline
  SCALE-MAMBA & \fc & \fc & \fc & \fc & \ec & \ec \\ \hline
  Wysteria~\cite{RastogiHH14} & \fc & \fc & \fc & \ec & \ec & \fc \\ \hline
  Frigate~\cite{Mood16} & \fc & \fc & \ec & \ec & \ec & \ec \\ \hline
  ABY~\cite{Demmler15a} & \hc & \hc & \fc & \fc & \ec & \ec \\ \hline
  ObliVM~\cite{Liu15} & \fc & \fc & \fc & \fc & \fc & \ec \\ \hline
  SCVM~\cite{Liu14} & \fc & \fc & \fc & \ec & \ec  & \fc \\ \hline
\end{tabular}
\caption{Language features supported in SMC compilers.} \label{tab:smc-compilers}
\end{table}
They are supporting loops, private-conditioned branches, supporting both private and public values (mixed mode), floating point arithmetic on private values, dynamically allocated memory, and having semantic formalism. 
Note that compilers that translate computation into Boolean circuits such as CBMC-GC need to unroll loops and thus can only support a bounded number of loop iterations, denoted as \hc\ in the table. 
ABY also appears to have this limitation and for that reason expects input sizes at compile time. 
Recent compilers that work with a circuit representation (e.g., Wysteria, ObliVM) store compiled programs using intermediate representation and perform loop unrolling and circuit generation at runtime. 
To the best of our knowledge, Sharemind permits updating only a single variable in a private-conditioned branch (i.e., \texttt{if (cond) a = b; else a = c;}). 
Similarly, in ABY the programmer has to encode all logic associated with conditional statements using multiplexers. 
CBMC-GC did not support floating point arithmetic based on open-source software at the time of publication. 

Dynamic memory management is often not discussed in prior work. CBMC-GC is said to support dynamic memory allocation, as long as this can be encoded as a bounded program, but the use of dynamic arrays and memory deallocation is not mentioned. PICCO explicitly supports C-style memory allocation and deallocation as well as dynamic arrays. ObliVM does not explicitly discuss dynamically allocated arrays, but we believe they are supported.
Similarly, out-of-bounds array access in user programs is also not typically discussed in the SMC literature. Therefore, it is difficult to tell what the behavior might be, i.e., whether the compiler checks for this and, if not, whether the behavior of the corresponding compiled program is undefined. Wysteria and PICCO are two notable exceptions: Wysteria has a strongly typed language and will prevent such programs from compiling (recall that it supports only static sizes). PICCO will compile programs with out-of-bounds memory accesses. While the behavior of such programs is undefined in C (and no correctness guarantees can be provided), its analysis demonstrates that no privacy violations take place. We formalize this behavior in this work.

\paragraph*{Non-interference}
Non-interference is a standard information flow security property guaranteeing that information about private data does not directly affect publicly observable data. We will show non-interference over executions of programs using the formal
model and its extension developed in this paper to prove security when SMC techniques and C language primitives are composed. Non-interference and its several variants have been extensively studied by means of language-based techniques, including type systems~\cite{VolpanoS97,AbadiBHR99}, runtime monitor~\cite{AustinF09,SabelfeldR09}, and multi-execution~\cite{DevrieseP10}, to cite a few. One of the challenges in guaranteeing non-interference when attackers can inspect the state of the computation is to guarantee that private information is not implicitly leaked by means of the control flow path, i.e., that the computation is data-oblivious. Several language-based methods have been designed to guarantee that systems are secure against leakage from branching statements, including timing analysis~\cite{Ford12} and multi-path execution~\cite{PlanulM13,Mitchell0SZ12,LaudP16}.
In particular, \cite{Mitchell0SZ12} considered an approach similar to the one we use here. However, these approaches do not prevent private data leakage from explicit memory management. Building on these early works,  several recent works~\cite{PatrignaniG17,AbateBCD0HPTT20} have shown that in the context of secure compilation the natural notion that one needs to consider is a form of non-interference extended to traces. Inspired by this work, this is the notion we use in this paper when reasoning about non-interference. 






\section{Overview}
\label{sec:background}
\section{Background and Motivation}

\subsection{IBM Streams}

IBM Streams is a general-purpose, distributed stream processing system. It
allows users to develop, deploy and manage long-running streaming applications
which require high-throughput and low-latency online processing.

The IBM Streams platform grew out of the research work on the Stream Processing
Core~\cite{spc-2006}.  While the platform has changed significantly since then,
that work established the general architecture that Streams still follows today:
job, resource and graph topology management in centralized services; processing
elements (PEs) which contain user code, distributed across all hosts,
communicating over typed input and output ports; brokers publish-subscribe
communication between jobs; and host controllers on each host which
launch PEs on behalf of the platform.

The modern Streams platform approaches general-purpose cluster management, as
shown in Figure~\ref{fig:streams_v4_v6}. The responsibilities of the platform
services include all job and PE life cycle management; domain name resolution
between the PEs; all metrics collection and reporting; host and resource
management; authentication and authorization; and all log collection. The
platform relies on ZooKeeper~\cite{zookeeper} for consistent, durable metadata
storage which it uses for fault tolerance.

Developers write Streams applications in SPL~\cite{spl-2017} which is a
programming language that presents streams, operators and tuples as
abstractions. Operators continuously consume and produce tuples over streams.
SPL allows programmers to write custom logic in their operators, and to invoke
operators from existing toolkits. Compiled SPL applications become archives that
contain: shared libraries for the operators; graph topology metadata which tells
both the platform and the SPL runtime how to connect those operators; and
external dependencies. At runtime, PEs contain one or more operators. Operators
inside of the same PE communicate through function calls or queues. Operators
that run in different PEs communicate over TCP connections that the PEs
establish at startup. PEs learn what operators they contain, and how to connect
to operators in other PEs, at startup from the graph topology metadata provided
by the platform.

We use ``legacy Streams'' to refer to the IBM Streams version 4 family. The
version 5 family is for Kubernetes, but is not cloud native. It uses the
lift-and-shift approach and creates a platform-within-a-platform: it deploys a
containerized version of the legacy Streams platform within Kubernetes.

\subsection{Kubernetes}

Borg~\cite{borg-2015} is a cluster management platform used internally at Google
to schedule, maintain and monitor the applications their internal infrastructure
and external applications depend on. Kubernetes~\cite{kube} is the open-source
successor to Borg that is an industry standard cloud orchestration platform.

From a user's perspective, Kubernetes abstracts running a distributed
application on a cluster of machines. Users package their applications into
containers and deploy those containers to Kubernetes, which runs those
containers in \emph{pods}. Kubernetes handles all life cycle management of pods,
including scheduling, restarting and migration in case of failures.

Internally, Kubernetes tracks all entities as \emph{objects}~\cite{kubeobjects}.
All objects have a name and a specification that describes its desired state.
Kubernetes stores objects in etcd~\cite{etcd}, making them persistent,
highly-available and reliably accessible across the cluster. Objects are exposed
to users through \emph{resources}. All resources can have
\emph{controllers}~\cite{kubecontrollers}, which react to changes in resources.
For example, when a user changes the number of replicas in a
\code{ReplicaSet}, it is the \code{ReplicaSet} controller which makes sure the
desired number of pods are running. Users can extend Kubernetes through
\emph{custom resource definitions} (CRDs)~\cite{kubecrd}. CRDs can contain
arbitrary content, and controllers for a CRD can take any kind of action.

Architecturally, a Kubernetes cluster consists of nodes. Each node runs a
\emph{kubelet} which receives pod creation requests and makes sure that the
requisite containers are running on that node. Nodes also run a
\emph{kube-proxy} which maintains the network rules for that node on behalf of
the pods. The \emph{kube-api-server} is the central point of contact: it
receives API requests, stores objects in etcd, asks the scheduler to schedule
pods, and talks to the kubelets and kube-proxies on each node. Finally,
\emph{namespaces} logically partition the cluster. Objects which should not know
about each other live in separate namespaces, which allows them to share the
same physical infrastructure without interference.

\subsection{Motivation}
\label{sec:motivation}

Systems like Kubernetes are commonly called ``container orchestration''
platforms. We find that characterization reductive to the point of being
misleading; no one would describe operating systems as ``binary executable
orchestration.'' We adopt the idea from Verma et al.~\cite{borg-2015} that
systems like Kubernetes are ``the kernel of a distributed system.'' Through CRDs
and their controllers, Kubernetes provides state-as-a-service in a distributed
system. Architectures like the one we propose are the result of taking that view 
seriously.

The Streams legacy platform has obvious parallels to the Kubernetes
architecture, and that is not a coincidence: they solve similar problems.
Both are designed to abstract running arbitrary user-code across a distributed
system.  We suspect that Streams is not unique, and that there are many
non-trivial platforms which have to provide similar levels of cluster
management.  The benefits to being cloud native and offloading the platform
to an existing cloud management system are: 
\begin{itemize}
    \item Significantly less platform code.
    \item Better scheduling and resource management, as all services on the cluster are 
        scheduled by one platform.
    \item Easier service integration.
    \item Standardized management, logging and metrics.
\end{itemize}
The rest of this paper presents the design of replacing the legacy Streams 
platform with Kubernetes itself.



\section{Semantics} \label{Sec: Semantics}

In this section we introduce our semantic and memory model. Our goal is to
formalize a general purpose SMC compiler, showing correctness with respect to standard C semantics
and non-interference to guarantee no leakage of private data. 
We therefore introduce two models, C (referred to as \vanillaC) as well as the semantics for the SMC compiler (referred to as \piccoC). 
We do not abstract away memory, instead we introduce a byte-level memory model, 
inspired by the memory model used by CompCert~\cite{leroy2012compcert}, a formally verified C compiler. 
Specifically, we build from their approach of byte-level representation of data and permissions. 

\begin{figure}
\begin{minipage}{0.48\textwidth}\footnotesize
$\begin{array}{l c l}
	\Type &::=& \RT{\llabel\ \btype} \mid \RT{\llabel\ \btype*} \mid \btype \mid {\btype*} \mid \Tlist \to \Type 
\\	\btype &::=& \Int \mid \Float \mid \Void 
\\	\llabel &::=& \rPriv \mid \rPub
\\	\Tlist &::=&  {[\ ]} \mid {\Type::\Tlist} 
\\ \\	\stmt &::=& {\var = \Expr} \mid {*\x = \Expr} \mid {\stmt_1; \stmt_2} \mid {\If (\Expr)\ \stmt_1\ \Else\ \stmt_2}

\\	  && \mid {\Type\ \x (\plist)\ \{ \stmt\}} \mid {\While (\Expr)\ \stmt} 
 \mid {\{ \stmt \}} \mid \decl \mid \Expr
\\	\Expr &::=& \Expr\ \binop\ \Expr \mid {\preop\ \x} \mid \var
		\mid {\x(\Elist)} \mid \builtin 
	\\	  && \mid {(\Type)\ \Expr}  \mid {( \Expr )}  \mid \val
\\	\decl &::=&  \Type\ \var \mid {\Type\ \x ( \plist )}
\\	\var &::=& \x \mid {\x[\Expr]} 
\\	\val &::=& n \mid (\loc, \offset) \mid [{\val_0},\ {...},\ {\val_n}] \mid \Null \mid \Skip
\\ \\ 	\builtin &::=& {\Malloc(\Expr)} \mid \RT{\PMalloc(\Expr,\ \Type)} \mid {\sizeof(\Type)}
	\\	  && \mid {\free(\Expr)} \mid \RT{\pfree(\Expr)} 
	\\ && \mid \RT{\smcinput(\var, e)} \mid \RT{\smcoutput(\var, e)}
\\ 	\binop &::=&{-} \mid {+} \mid \div \mid \cdot \mid {==}  \mid {!=} \mid {<}
\\	\preop &::=& \& \mid {*} \mid ++
\\	\Elist &::=& {\Elist,\ \Expr} \mid \Expr \mid \Void
\\	\plist &::=&  \plist,\ \Type\ \var \mid {\Type\ \var} \mid \Void
\end{array}$
\captionof{figure}{Combined \vanillaC /\piccoC\ Grammar. The color \red{red} denotes terms specific to programs written in \piccoC. 
} 	
\label{Fig: \piccoC grammar}
\end{minipage}
\qquad
\begin{tabular}{l}
\begin{minipage}{0.44\textwidth}\footnotesize
$\begin{array}{r l}	
% Configuration
\Config ::= & \epsilon \mid (\pid, \gamma, \sigma, \DMap, \Acc, \stmt) \Mid \Config \\
\\
% Environment
	\gamma ::=& [ \-\ ]\ \mid\ \gamma[\x\ \to\ (\loc,\ \Type)]							\\
% Memory
	\sigma ::=& [ \-\ ]\ \mid\ \sigma [\loc\ \to\ (\byte,\ \Type,\ n,\ \PermL)  	\\
% Permissions 
	\PermL ::=& [ \-\ ]\ \mid\ [(0,\ \llabel_0,\ \perm_0), ..., (\bytelen, \llabel_\bytelen, \perm_\bytelen)] \\
	\perm ::=& \PermF \mid \PermN		 									\\
% Permission Tuples - "Size"
	\bytelen ::= & \tau(\Type)\cdot n - 1		\\
% Change Map & locals map
	\DMap ::= & [ \-\ ] \mid \dmap::\DMap \\
	\dmap ::= & [ \-\ ] \mid ((\loc, \offset)\to(\val_1, \val_2, \tagb, \Type))::\dmap \\
\\ 
% Location List
	\locLL ::= & \epsilon \mid (\pid, \locL) \Mid \locLL \\
	\locL ::= & [ \-\ ] \mid (\loc, \offset)::\locL \\
	\codeLL ::= & \epsilon \mid (\pid, \codeL) \Mid \codeLL \\
	\codeL ::= & [ \-\ ] \mid \code::\codeL 
	\end{array}
$
\captionof{figure}{Configuration: party identifier $\pid$, environment $\gamma$, memory $\sigma$, \changeMap\ $\DMap$, accumulator $\Acc$, and statement $\stmt$.}
\label{Fig: mem model}
\end{minipage}
\end{tabular}
\end{figure}


Figure~\ref{Fig: \piccoC grammar} gives the combined \vanillaC\ and \piccoC\ grammar, which is a subset of the ANSI C grammar. 
We include one dimensional arrays, branches, loops, dynamic memory allocation, and pointers. 
Arrays are zero-indexed, and it is possible to overshoot their bounds. 
We chose not to include structs or multi-dimensional arrays, as they are an extension of this core subset.
The interested reader can find the full semantics within the scope of the grammar given in Appendix~\ref{app: semantics}.
We use the color \red{red} to denote terms in the \piccoC\ grammar that are not present in \vanillaC, including annotated types ($\llabel\ \btype$, $\llabel\ \btype *$), privacy labels ($\Pub$, $\Priv$), and primitive functions ($\PMalloc$, $\pfree$) for allocation and deallocation of 
memory for private pointers.

We denote types as $\Type$, basic types as $\btype$ ($\btype *$ as a pointer type), privacy annotations as $\llabel$, and function types as $\Tlist \to \Type$ 
(where $\Tlist$ is a type list, $[\ ]$ an empty list, and $::$ list concatenation). 
Values $\val$ include numbers $n$, locations ($\loc$, $\offset$) consisting of a memory block identifier and an offset, lists of values, $\Null$, and $\Skip$ (to show a statement being reduced to completion).
Declarations include variable and function declarations, where $\plist$ is the function parameter list.
For unary operations, we include: $\&$, to obtain the address of a variable; $*$, to allow dereferencing pointers; and $++$, to allow pre-incrementing and to model a basic pointer arithmetic. 


%%%%%%%%%%%%%%%%%%%%%%%
% 
%      	 Memory model input
% 
%%%%%%%%%%%%%%%%%%%%%%%

\subsection{Memory Model} \label{subsec: mem model}
We assume the existance $n$ communicating parties, each with a separate memory.
Our memory model encodes each memory as a contiguous region of {\em blocks}, which are sequences of bytes and metadata.
We introduce an execution environment $\gamma$ and memory $\sigma$, shown in Figure \ref{Fig: mem model}.
Each block is assigned an identifier $\loc$ (to be discussed more later in this section). 
Blocks are never recycled nor cleared when they are freed. We chose this view of memory to preserve all allocated data, which, in conjunction with data-oblivious execution, represents the worst case for maintaining privacy. 
Direct memory access through pointers or manipulation of array indices allows programs to access any block for which the memory address is computable (e.g., as an offset or direct pointer access). 
To obtain the byte representations of data we leverage functions similar to CompCert, using $\Encode$ for values, $\EncodePtr$ for pointers, and $\EncodeFun$ for functions. Likewise, to obtain the human-readable data back, we use respective decode functions such as $\Decode$. We use a specialized version ($\DecodeArr$) for obtaining a specific index within an array data block.
We introduce the particulars in the following subsections.


% Fig: mem model moved to semantics.tex to be placed next to the grammar

\subsubsection{Environment} 
The environment, $\gamma$, maintains a mapping of each live variable $\x$ to its memory block identifier (where the data $\x$ is stored) and its type. 
At the start of a program, the environment is empty, i.e., $\gamma = [\-\ ]$. 
Variables that are no longer live are removed from the environment, based on scoping.  
We use the environment to facilitate the lookup of variables (i.e., for reads, writes, function calls) in memory $\sigma$. 



\subsubsection{Memory Blocks and Identifiers} \label{subsec: mem block}
The memory, $\sigma$ (shown in Figure~\ref{Fig: mem model}), is a mapping of each identifier $\loc$ to its memory block, which contains 
the byte representation $\byte$ of data stored there and metadata about the block.
Metadata consists of  
a type $\Type$ associated with the block, 
the number of elements $n$ of that type stored in $\byte$, 
and a list of byte-wise permissions tuples $[(0,\ \llabel_0,\ \perm_0), ..., (\bytelen,\ \llabel_\bytelen,\ \perm_\bytelen)]$, where $\bytelen = \tau(\Type)\cdot n - 1$ and function $\tau$ provides the size of the given type in bytes. 
A new memory block identifier is obtained from function $\phi$. 
These identifiers are monotonically increasing with each allocation. 
Every block is added to memory $\sigma$ on allocation, and is never cleared of data nor removed from $\sigma$ upon deallocation. 
Metadata cannot be accessed or modified directly by the program (the semantic rules control modification).
A memory block can be of an arbitrary size, which is constant and determined at allocation (with the exception of private pointers, to be discussed later in subsection~\ref{subsec: picco ptr eval}). 
We represent a memory location as a two-tuple of a memory block identifier and an offset. This allows us to use pointers to refer to any arbitrary memory location, as in C.





\subsubsection{Permissions} 
\label{subsec: permissions}
A permission $\perm$ can either be $\PermF$ (i.e., can be written to, read from, etc.) or 
$\PermN$ (i.e., already freed). 
These byte-wise permissions are modeled after a subset of those used by CompCert, and we extend their permission model by including a privacy label.
Each memory block has a list of permission tuples, one for each byte of data 
stored in that block. 
A permission tuple consists of the position of the byte that it corresponds to, and the privacy label $\llabel$ and permission $\perm$ for that byte of data. 
These permissions are important in reading and writing data to memory, especially when it comes to overshooting arrays and other out-of-bound memory accesses possible through the use of pointers. 
In particular, permissions allow us to keep track of deallocated memory (e.g. a block freed - note that the memory stored in the block itself is not overwritten or cleared and the block can still be accessed indirectly through direct
memory manipulations). 
All permission tuples corresponding to a memory block of a function type will have public privacy labels, as the instructions for a function are accessible from the program itself. 
Those for a normal variable or an array will have privacy labels corresponding to their type (i.e., public for public types, private for private types); those for pointers are more complex and will be discussed later in subsection~\ref{subsec: picco ptr eval}.  




\subsubsection{Malloc and Free}
Allocation of dynamic memory in C is provided by \TT{malloc}, which takes a number of bytes as its argument. 
When \Code{malloc} is called, 
a new memory block with identifier $\loc$ is obtained, initialized as a void type of the given size, and returned. 
This block then needs to be cast to the desired type. 
However, when dealing with private data, the programmer is unlikely to know the internal representation and the size of the private data types. 
For that reason, when allocating memory for private data, we adopt PICCO's \TT{pmalloc} functionality which takes two arguments: the type and the number of elements of that type to be allocated. 
The semantics then handle sizing the new memory block for the given private type. 
When \Code{free} or \Code{pfree} is called, if the argument is a variable of a pointer type, the permissions for all bytes of the location the pointer refers to will be set to $\PermN$, but the data stored there will not be erased. 
When \Code{pfree} is called with a pointer with a single location, it behaves identically to \Code{free}. 
The use of \TT{pfree} with multiple locations is a bit more involved, to be discussed later in Subsection~\ref{subsec: picco mem alloc/dealloc}. 
It is important to note that memory allocation and deallocation are public side effects, and therefore are {\em not allowed} within private-conditioned branches. 



\subsubsection{Public vs. Private Blocks}
To distinguish between public and private blocks, we assume that private blocks will be encrypted 
and we will use basic private primitives implementing specific operations to manage them. 
For modeling purposes, 
these primitives can decrypt the required blocks, perform 
the operations they are meant to implement, and encrypt the result. 
In our model, a program can also access private blocks by means of standard non-private operations or through of pointers. In this case, the operation will just interpret the encrypted value as a public value. This approach gives us a conceptual distinction between a \emph{concrete memory} and its corresponding \emph{logical content}, i.e. public values and values of the private data prior to encryption. Our model as described in the next section will work on concrete memories, but in Section~\ref{sec: noninterference}, for the proof of noninterference, it will be convenient to refer to the logical content of a memory. We will use the notation $\sigma\ell$ to denote the logical content of $\sigma$.  




%%%%%%%%%%%%%%%%%%%%%%%
%
%   Begin Vanilla C semantics description 
%
%%%%%%%%%%%%%%%%%%%%%%%


\begin{figure*}
\footnotesize
\begin{tabular}{l}
% BASIC  EVAL
Multiparty Binary Operation  \\
  	\inferrule{\begin{array}{l l}
		\begin{array}{l}
		\qquad((\pidA, \hgamma, \hsigma,\ \bsq, \bsq, \hExpr_1)\ \Mid ...\Mid
		(\pidZ, \hgamma, \hsigma,\ \bsq, \bsq, \hExpr_1))
			\crcr\Veval_{\codeVLL_1} 
			((\pidA, \hgamma, \hsigma_1, \bsq, \bsq, \hn_1) \Mid ...\Mid
			(\pidZ, \hgamma, \hsigma_1, \bsq, \bsq, \hn_1))
		\crcr \qquad ((\pidA, \hgamma, \hsigma_1, \bsq, \bsq, \hExpr_2)\ \Mid ...\Mid
			(\pidZ, \hgamma, \hsigma_1, \bsq, \bsq, \hExpr_2))
			\crcr\Veval_{\codeVLL_2} 
			((\pidA, \hgamma, \hsigma_2, \bsq, \bsq, \hn_2)\Mid ...\Mid
			(\pidZ, \hgamma, \hsigma_2, \bsq, \bsq, \hn_2))
		\end{array}
	& \begin{array}{l}
		 \hn_1 \binop\ \hn_2 = \hn_3
		\crcr \binop\in\{\cdot, +, -, \div\}
	\end{array}\end{array}}
	{\begin{array}{l}
	((\pidA, \hgamma, \hsigma,\ \bsq, \bsq, \hExpr_1 \binop\ \hExpr_2)\Mid ...\Mid
	(\pidZ, \hgamma, \hsigma,\ \bsq, \bsq, \hExpr_1 \binop\ \hExpr_2)) 
		\Veval_{\codeVLL_1\addC\codeVLL_2\addC\codeVM{mpb}} 
		\crcr((\pidA, \hgamma, \hsigma_2, \bsq, \bsq, \hn_3) 
			\qquad\ \Mid ...\Mid 
		(\pidZ, \hgamma, \hsigma_2, \bsq, \bsq, \hn_3))
		\end{array}}
\\ \\
 Write \\
\inferrule{\begin{array}{l}
		((\pid, \hgamma, \hsigma, \bsq, \bsq, \hExpr)\Mid \hConfig)  \Veval_{\codeVLL} 
			((\pid, \hgamma, \hsigma_1, \bsq, \bsq, \hn)\Mid \hConfig_1)
		\crcr \hgamma(\hx) = (\hloc, \hbtype)  
		\qq \Update(\hsigma_1, \hloc, \hn, \hbtype) = \hsigma_2
	\end{array}}
	{((\pid, \hgamma, \hsigma, \bsq, \bsq, \hx = \hExpr) \Mid \hConfig)
		\Veval_{\codeVLL\addC[\codeVS{w}]} 
		((\pid, \hgamma, \hsigma_2, \bsq, \bsq, \Skip)\Mid \hConfig_1)}
\\ \\
Malloc 	\\ 
\inferrule{\begin{array}{l}
		((\pid, \hgamma, \hsigma, \bsq, \bsq, \hExpr) \Mid \hConfig) \Veval_{\codeVLL} 
			((\pid, \hgamma, \hsigma_1, \bsq, \bsq, \hat{n})\Mid \hConfig_1)
		\qq \hloc = \phi()
		\crcr \hsigma_2 = \hsigma_1 \big[\hloc \to \big(\Null, \Void*, \hat{n}, 
			\PermL(\PermF, \Void*, \Pub, \hn)\big) \big] 
	\end{array} }
	{((\pid, \hgamma, \hsigma, \bsq, \bsq, \Malloc (\hExpr))\Mid \hConfig) 
		\Veval_{\codeVLL\addC[\codeVS{mal}]} 
		((\pid, \hgamma, \hsigma_2, \bsq, \bsq, (\hloc, 0))\Mid \hConfig_1)}
\\ \\
Multiparty Free \\
	\inferrule{\begin{array}{l}
		\hgamma(\hx) = (\hloc, \hbtype*) 
		\qq \hsigma(\hloc) = (\hbyte, {\hbtype*}, 1, \PtrPermL(\PermF, \hbtype*, \Pub, 1))
		\crcr \DecodePtr({ \hbtype*}, 1, \hbyte) = [1, [({\hloc_1}, 0)], [1], 1]
		\crcr \SelectFreeable(\hgamma, [({\hloc_1}, 0)], [1], \hsigma) = 1
		\qq \Free(\hsigma, \hloc_1) = \hsigma_1 
	\end{array}}
	{\begin{array}{l}
	((\pidA, \hgamma, \hsigma, \bsq, \bsq, \free (\x)) \Mid ... \Mid 
	(\pidZ, \hgamma, \hsigma, \bsq, \bsq, \free (\x)))
		\Veval_{\codeVM{fre}} \crcr
		((\pidA, \hgamma, \hsigma_1, \bsq, \bsq, \Skip)\Mid ... \Mid 
		(\pidZ, \hgamma, \hsigma_1, \bsq, \bsq, \Skip))
		\end{array}}
\\ \\
% ALLOC / DEALLOC
Pointer Declaration 	\\
\inferrule{\begin{array}{l l}
	\begin{array}{l}
		(\hType = \hbtype *)
		\crcr \hloc = \phi() 
		\crcr \hgamma_1 = \hgamma[\hx \to (\hloc, \hType)]
		\end{array}
		&\begin{array}{l}
		\getIndirection(*) = \hindir \crcr
		\hbyte = \EncodePtr(\hType*, [1, [(\hlocDefault, 0)], [1], \hindir])
		\crcr \hsigma_1 = \hsigma[\hloc \to (\hbyte, \hType, 0, \PtrPermL(\PermF, \hType, \Pub, 0))]
	\end{array}\end{array}}					
	{((\pid, \hgamma, \hsigma, \bsq, \bsq, \hType \hx) \Mid \hConfig)
		\Veval_{[\codeVS{dp}]} 
		((\pid, \hgamma_1, \hsigma_1, \bsq, \bsq, \Skip)\Mid \hConfig)}
\\ \\ 
% PTR  EVAL
% ARR  EVAL
Multiparty Array Read \\ %\\ 
\inferrule{\begin{array}{l}
		\hgamma(\hx) = (\hloc, \Const\ \hbtype*) \crcr
		((\pidA, \hgamma, \hsigma, \bsq, \bsq, \hExpr) \Mid ... \Mid 
		(\pidZ, \hgamma, \hsigma, \bsq, \bsq, \hExpr)) \Veval_{\codeVLL} 
			((\pidA, \hgamma, \hsigma_1, \bsq, \bsq, \hind)\Mid ... \Mid 
			(\pidZ, \hgamma, \hsigma_1, \bsq, \bsq, \hind)) 
		\crcr \hsigma_1(\hloc) = (\hbyte, \Const\ \hbtype*, 1, \PtrPermL(\PermF, \Const\ \hbtype*, \Pub, 1)) 
		\crcr \DecodePtr(\Const\ \hbtype*, 1, \hbyte) = [1, [(\hloc_1, 0)], [1], 1] 
		\crcr \hsigma_1(\hloc_1) = (\hbyte_1, \hbtype, \hnl, \ArrPermL(\PermF, \hbtype, \Pub, \hnl))  
		\qq 0 \leq \hind \leq \hnl - 1 
		\crcr \DecodeArr({\hbtype},\ \hind,\ {\hbyte_1}) = \hn_{\hind} 
	\end{array}}
	{\begin{array}{l}
	((\pidA, \hgamma, \hsigma,\ \bsq, \bsq, \hx[\hExpr]) \Mid ... \Mid 
	(\pidZ, \hgamma, \hsigma,\ \bsq, \bsq, \hx[\hExpr]))
		\Veval_{\codeVLL\addC\codeVM{mpra}} \crcr
		((\pidA, \hgamma, \hsigma_1, \bsq, \bsq, \hn_{\hind})\ \ \Mid ... \Mid 
		(\pidZ, \hgamma, \hsigma_1, \bsq, \bsq, \hn_{\hind}))
		\end{array}}
\end{tabular}
\caption{\vanillaC\ semantic rules.}
\label{Fig: \vanillaC sem rules}
\end{figure*}















\subsection{\vanillaC\ Semantics} \label{van C descr}


%%%%%%%%%%%%%%%%%%%%%%%
% 
%       Codes For Evaluations
% 
%%%%%%%%%%%%%%%%%%%%%%%
In order to facilitate the correspondence between the \vanillaC\ and \piccoC\ semantics, we model our semantics using big-step evaluation judgements and define our C semantics with respect
to multiple {\em non interacting} parties that evaluate the same program. 
These judgements are defined over a six-tuple configuration $\Config = ((\pid, \gamma, \sigma, \DMap, \Acc, \stmt) \Mid \Config_1)$, where each rule is a reduction from one configuration to a subsequent. 
In the semantics, we denote the party identifier $\pid$;
the environment as $\gamma$; 
memory as $\sigma$;
a mapping structure for location-based tracking of changes at each level of nesting of private-conditioned branches $\DMap$;
the level of nesting of private-conditioned branches as $\Acc$; and
a big-step evaluation of a statement $\stmt$ to a value $\val$ using $\eval$.  
We use a $\hat{\ }$ and $\Veval$ to distinguish the \vanillaC\ semantics from those we use in the next section for \piccoC, as well as $\bsq$ in \vanillaC\ as a placeholder for $\Acc$ and $\DMap$ to maintain the same shape of configurations as that of \piccoC\ used in the next section. 
We annotate each evaluation with party-wise lists of the evaluation codes of all rules that were used during the execution of the rule (i.e., $\Veval_{\codeVLL}$) to facilitate reasoning over evaluation trees. 
We extend the list concatenation operator $::$ to also work over party-wise lists such as $\codeVLL$, defining its behavior as concatenating the lists within $\codeVLL$ by party (i.e., $[(1, \codeVL^1_1), (2, \codeVL^2_1)]::[(1,\codeVL^1_2)] = [(1, \codeVL^1_1::\codeVL^1_2), (2, \codeVL^2_1)]$). 
The assertions in each semantic rule can be read in sequential order, from left to right and top to bottom. 

We present a subset of  the \vanillaC\ semantics  rules in Figure~\ref{Fig: \vanillaC
  sem rules}, focusing on the most interesting rules. Specifically, we present rules for arrays and pointers
which we will compare with the rules for the
\piccoC\ semantics in Section~\ref{smc C descr}. Because the semantics
rules are mostly standard, we will describe one rule to familiarize 
the reader with our notation. 

Rule Multiparty Binary Operation is an example of a rule for the evaluation of a binary operation (comparison operations are handled separately). In \vanillaC, multiparty rules occur at the same time in all parties, but without any communication between the parties. 
We have the starting state $((\pidA,$ $\hgamma,$ $\hsigma,$ $\bsq,$ $\bsq,$ $\hExpr_1\ \binop\ \hExpr_2)$ $\Mid ...\Mid $ $(\pidZ,$ $\hgamma,$ $\hsigma,$ $\bsq,$ $\bsq,$ $\hExpr_1\ \binop\ \hExpr_2))$, 
with all parties at current environment $\hgamma$, current memory $\hsigma$, and the starting statement $\hExpr_1\ \binop\ \hExpr_2$.
First, all parties evaluate expression $\hExpr_1$, using the
current environment and memory states, 
resulting in environment $\hgamma$, memory
$\hsigma_1$, and number $\hn_1$.  We repeat this for
$\hExpr_2$. 
We then evaluate $\hn_1\ \binop\ \hn_2 = \hn_3$ here, and we will return $\hn_3$.  
The end state is then
$((\pidA, \hgamma, \hsigma_2, \bsq, \bsq, \hn_3)$ $\Mid ...\Mid$ $(\pidZ, \hgamma, \hsigma_2, \bsq, \bsq, \hn_3))$.




































%%%%%%%%%%%%%%%%%%%%%%%
%
%   Begin Picco C semantics description 
%
%%%%%%%%%%%%%%%%%%%%%%%

\begin{figure*}\footnotesize
\begin{tabular}{l}
% BASIC  EVAL
Multiparty Binary Operation	\\ 
\inferrule{\begin{array}{l l}
	\begin{array}{l}
		\qquad((\pidA, \gamma^{\pidA}, \sigma^{\pidA}, \DMap^{\pidA}, \Acc, \Expr_{1}) \Mid ...\Mid (\pidZ, \gamma^{\pidZ}, \sigma^{\pidZ}, \DMap^{\pidZ}, \Acc, \Expr_{1})) 
			\crcr\Deval{\locLL_1}{\codeLL_1} 
			((\pidA, \gamma^{\pidA}, \sigma^{\pidA}_{1}, \DMap^{\pidA}_{1}, \Acc, \n^{\pidA}_{1}) \Mid ...\Mid (\pidZ, \gamma^{\pidZ}, \sigma^{\pidZ}_{1}, \DMap^{\pidZ}_{1}, \Acc, \n^{\pidZ}_{1}))
		\crcr
		\qquad((\pidA, \gamma^{\pidA}, \sigma^{\pidA}_{1}, \DMap^{\pidA}_{1}, \Acc, \Expr_{2}) \Mid ...\Mid (\pidZ, \gamma^{\pidZ}, \sigma^{\pidZ}_{1}, \DMap^{\pidZ}_{1}, \Acc, \Expr_{2})) 
			\crcr\Deval{\locLL_2}{\codeLL_2} 
			((\pidA, \gamma^{\pidA}, \sigma^{\pidA}_{2}, \DMap^{\pidA}_{2}, \Acc, \n^{\pidA}_{2})\Mid ...\Mid (\pidZ, \gamma^{\pidZ}, \sigma^{\pidZ}_{2}, \DMap^{\pidZ}_{2}, \Acc, \n^{\pidZ}_{2}))
		\crcr
		\MPC{b}(\binop, [\n^\pidA_1, ..., \n^\pidZ_1], [\n^\pidA_2, ..., \n^\pidZ_2]) = (\n^{\pidA}_{3}, ..., \n^{\pidZ}_{3})
	\end{array}
	&\begin{array}{l}
		\{(\Expr_{1}, \Expr_{2}) \isPriv \gamma^\pid\}^{\pidZ}_{\pid = \pidA}
		\crcr \binop\in\{\cdot, +, -, \div\}
	\end{array}\end{array}}
	{\begin{array}{l}
	((\pidA, \gamma^{\pidA}, \sigma^{\pidA}, \DMap^{\pidA}, \Acc, {\Expr_{1}\ \binop\ \Expr_{2}})\Mid ...\Mid (\pidZ, \gamma^{\pidZ}, \sigma^{\pidZ}, \DMap^{\pidZ},\Acc, {\Expr_{1}\ \binop\ \Expr_{2}})) 
		\Deval{\locLL_1 \addL \locLL_2}{\codeLL_1 \addC \codeLL_2 \addC \codeMP{mpb}} 
		\crcr((\pidA, \gamma^{\pidA}, \sigma^{\pidA}_{2}, \DMap^{\pidA}_{2}, \Acc, \n^{\pidA}_{3}) 
			\qquad\-\ \-\ \-\ \Mid ...\Mid 
		(\pidZ, \gamma^{\pidZ}, \sigma^{\pidZ}, \DMap^{\pidZ}_{2}, \Acc, \n^{\pidZ}_{3})) \end{array}}
\\ \\
Write Private Variable Public Value		\\
\inferrule{\begin{array}{l l}
		\begin{array}{l}
		(\Expr) \isPub \gamma \crcr
		\crcr \gamma(\x) = (\loc, {\Priv\ \btype})  
		\end{array}
		& \begin{array}{l}
		((\pid, \gamma, \sigma, {\DMap}, \Acc, \Expr) \Mid  \Config)  
			\Deval{\locLL_1}{\codeLL_1}  ((\pid, \gamma, \sigma{_1}, {\DMap_1}, \Acc, \n) \Mid  \Config_1) 
		\crcr 
		\Update(\sigma{_1}, \loc, \Encrypt(\n), \Acc, \Priv\ \btype) = \sigma{_2}
	\end{array}\end{array}}
	{((\pid, \gamma,\ \sigma,\ {\DMap},\ \Acc,\ {\x = \Expr}) \Mid  \Config)\ 
		\Deval{\locLL_1 \addL (\pid, [(\loc, 0)])}{\codeLL_1 \addC \codeSP{w2}}  
		((\pid, \gamma,\ \sigma{_2},\ {\DMap_1},\ \Acc,\ \Skip) \Mid  \Config_1)}
\\ \\
Private Malloc \\
\inferrule{\begin{array}{l}
		(\Expr) \isPub \gamma \qq
		\Acc = \AccZ \qq
		(\Type = {\Priv\ \btype*}) \lor (\Type = {\Priv\ \btype}) \crcr
		\loc = \phi()	\qq
		((\pid, \gamma,\ \sigma,\ {\DMap},\ \Acc,\ \Expr) \Mid  \Config)\ 
			\Deval{\locLL_1}{\codeLL_1}  ((\pid, \gamma,\ \sigma{_1},\ {\DMap},\ \Acc,\ {n}) \Mid  \Config_1) 
		\crcr \sigma_2 = \sigma{_1} \big[\loc \to \big(\Null,\ \Void*,\ {n\cdot\tau(\Type)},\ \PermL(\PermF, \Void*, \Priv, \n\cdot\tau(\Type))\big)\big]	
	\end{array}}
	{\begin{array}{l}
	((\pid, \gamma,\ \sigma,\ {\DMap},\ \Acc,\ {\PMalloc (\Expr,\ \Type)}) \Mid  \Config) 
		\Deval{\locLL_1 \addL (\pid, [(\loc, 0)])}{\codeLL_1 \addC \codeSP{malp}} 
		((\pid, \gamma,\ \sigma{_2},\ {\DMap},\ \Acc,\ (\loc, 0)) \Mid  \Config_1)
	\end{array} }
\\ \\
% ALLOC / DEALLOC
Multiparty Private Free	\\
\inferrule{ \begin{array}{l}
		\{\gamma^\pid(\x) = (\loc^\pid,\ {\Priv\ \btype*})\}^{\pidZ}_{\pid = \pidA}  
		\qq \Acc = \AccZ  \qq
		(\btype = \Int) \lor (\btype = \Float) \crcr
		\{\sigma^\pid(\loc^\pid) = (\byte^\pid, \Priv\ \btype*, \nl, \PtrPermL(\PermF, \Priv\ \btype*, \Priv, \nl))\}^{\pidZ}_{\pid = \pidA} \crcr
		\{[\nl,\ \locL^\pid\ \tagbL^\pid,\ \indir] = \DecodePtr({\Priv\ \btype*},\ \nl,\ \byte^\pid)\}^{\pidZ}_{\pid = \pidA}
		\qq \{\nl > 1\}^{\pidZ}_{\pid = \pidA}  \crcr
		\If(\indir>1) \{\Type = \Priv\ \btype* \}\ \Else\ \{ \Type = \Priv\ \btype \}
		\crcr
		\{\SelectFreeable(\gamma^\pid, \locL^\pid, \tagbL^\pid, \sigma^\pid) = 1\}^{\pidZ}_{\pid = \pidA} 
		\crcr
		\{\forall (\loc^\pid_m, 0) \in \locL.\quad \sigma^\pid(\loc^\pid_m) = (\byte^\pid_m, \Type, n, \PermL(\PermF, \Type, \Priv, n))\}^{\pidZ}_{\pid = \pidA}
		\crcr
		\PFree([[\byte^\pidA_0, ..., \byte^\pidA_{\nl-1}], ..., [\byte^\pidZ_0, ..., \byte^\pidZ_{\nl-1}]], [\tagbL^\pidA, ...\tagbL^\pidZ]) 
			\crcr\qq= ([[\byte'^\pidA_0, ..., \byte'^\pidA_{\nl-1}], ..., [\byte'^\pidZ_0, ..., \byte'^\pidZ_{\nl-1}]], [\tagbL'^\pidA, ..., \tagbL'^\pidZ])
		\crcr
		\{\UpdateBytesFree(\sigma^\pid, \locL^\pid, [\byte'^\pid_0, ..., \byte'^\pid_{\nl-1}]) = \sigma^\pid_1\}^{\pidZ}_{\pid = \pidA}
		\crcr
		\{\sigma^\pid_2 = \UpdatePtrLocs(\sigma^\pid_1, \locL^\pid[1:\nl-1], \tagbL^\pid[1:\nl-1], \locL^\pid[0], \tagbL^\pid[0])\}^{\pidZ}_{\pid = \pidA}
	\end{array}}
	{\begin{array}{l}((\pidA, \gamma^{\pidA}, \sigma^{\pidA}, \DMap^\pidA, \Acc, {\pfree (\x)})\Mid ...\Mid (\pidZ, \gamma^{\pidZ}, \sigma^{\pidZ}, \DMap^\pidZ, \Acc, {\pfree (\x)})) 
		\Deval{(\pidA, \locL^\pidA) \Mid ... \Mid (\pidZ, \locL^\pidZ)}{\codeMP{mpfre}} 
		\crcr((\pidA, \gamma^{\pidA}, \sigma^{\pidA}_2, \DMap^\pidA, \Acc, \Skip)\quad\-\ \ \Mid ...\Mid 
		(\pidZ, \gamma^{\pidZ}, \sigma^{\pidZ}_2, \DMap^\pidZ, \Acc, \Skip)) \end{array}}
\\ \\ 
% PTR  EVAL
Private Pointer Declaration \\ 
\inferrule{\begin{array}{l}\begin{array}{l l}
		\getIndirection(*) = \indir\qq
		&((\Type = {\btype *}) \lor (\Type = {\Priv\ \btype*})) \land ((\btype = \Int) \lor (\btype = \Float))
		\end{array}\crcr \begin{array}{l l l} 
		\loc = \phi() \qq
		& \gamma{_1} = \gamma[\x \to (\loc, \Priv\ \btype*)] 	
		\end{array}\crcr\begin{array}{l}
		\byte = \EncodePtr(\Priv\ \btype*, [1, [(\locDefault, 0)], [1], \indir])	\crcr
		\sigma{_1} = \sigma[\loc \to (\byte, \Priv\ \btype*, 1, \PtrPermL(\PermF, \Priv\ \btype*, \Priv, 1))]
	\end{array}\end{array}}					
	{((\pid, \gamma,\ \sigma,\ {\DMap},\ \Acc,\ {\Type\ \x}) \Mid  \Config)\ 
		\Deval{(\pid, [(\loc, 0)])}{\codeSP{dp1}}  ((\pid, \gamma{_1},\ \sigma{_1},\ {\DMap},\ \Acc,\ \Skip) \Mid  \Config)}
% ARR  EVAL
\\ \\
Multiparty Array Read Private Index \\
	\inferrule{\begin{array}{l}
		\qquad((\pidA, \gamma^{\pidA}, \sigma^{\pidA}, \DMap^\pidA, \Acc, \Expr)\ \Mid ...\Mid (\pidZ, \gamma^{\pidZ}, \sigma^{\pidZ}, \DMap^\pidZ, \Acc, \Expr)) 
		\qquad\ \
		\{(\Expr) \isPriv \gamma^\pid\}^{\pidZ}_{\pid = \pidA} 
		\crcr		
			\crcr\Deval{\locLL_1}{\codeLL_1} ((\pidA, \gamma^{\pidA}_{}, \sigma^{\pidA}_{1}, \DMap^{\pidA}_{1}, \Acc, \ind^{\pidA})\Mid ...\Mid (\pidZ, \gamma^{\pidZ}_{}, \sigma^{\pidZ}_{1}, \DMap^{\pidZ}_{1}, \Acc, \ind^{\pidZ}))
		\qquad\ \{\gamma^\pid(\x) = (\loc^\pid, \Const\ \llabel\ \btype*)\}^{\pidZ}_{\pid = \pidA}
		\crcr
		\{\sigma^\pid_1(\loc^\pid) = (\byte^\pid,\ {\llabel\ \Const\ \btype*}, 1, 
			\PtrPermL(\PermF, {\llabel\ \Const\ \btype*}, \llabel, 1))\}^{\pidZ}_{\pid = \pidA} 
		\crcr \{\DecodePtr({\llabel\ \Const\ \btype*},\ 1,\ \byte^\pid) 
			= [1,\ [({\loc^\pid_1}, 0)],\ [1],\ 1]\}^{\pidZ}_{\pid = \pidA}  
		\crcr \{\sigma^\pid_1({\loc^\pid_1}) = ({\byte^\pid_1}, {\llabel\ \btype}, {\nl}, 
									\ArrPermL(\PermF, \llabel\ \btype, \llabel, {\nl}))\}^{\pidZ}_{\pid = \pidA}
		\crcr 
			\{\forall \ind \in \{0...\nl-1\} \quad \DecodeArr({\llabel\ \btype}, \ind, {\byte^\pid_1}) =  \n^\pid_\ind\}^{\pidZ}_{\pid = \pidA}
		\crcr
		\MPC{ar}((\ind^\pidA, [\n^{\pidA}_{0}, ..., \n^{\pidA}_{\nl-1}]), ..., (\ind^\pidZ, [\n^{\pidZ}_{0}, ..., \n^{\pidZ}_{\nl-1}])) = (\n^{\pidA}, ..., \n^{\pidZ})
		\qq \{(\n^\pid) \isPriv \gamma^\pid\}^{\pidZ}_{\pid = \pidA} 
		\crcr \locLL_2 = (\pidA, [(\loc^\pidA, 0), (\loc^\pidA_1, 0), ..., (\loc^\pidA_1, \nl-1)])\Mid ... \Mid (\pidZ, [(\loc^\pidZ, 0), (\loc^\pidZ_1, 0), ..., (\loc^\pidZ_1, \nl-1)])
	\end{array}}
	{\begin{array}{l}
		((\pidA, \gamma^{\pidA}, \sigma^{\pidA}, \DMap^\pidA, \Acc, \x[\Expr])\Mid ...\Mid (\pidZ, \gamma^{\pidZ}, \sigma^{\pidZ}, \DMap^\pidZ, \Acc, \x[\Expr])) 
			\Deval{\locLL_1 \addL \locLL_2}{\codeLL_1\addC \codeMP{mpra}} 
			\crcr((\pidA, \gamma^{\pidA}_{}, \sigma^{\pidA}_{1}, \DMap^{\pidA}_{1}, \Acc, \n^{\pidA})\-\ \Mid ...\Mid (\pidZ, \gamma^{\pidZ}_{}, \sigma^{\pidZ}_{1}, \DMap^{\pidZ}_{1}, \Acc, \n^{\pidZ})) \end{array}}
\end{tabular}
\caption{\piccoC\ semantic rules.}
\label{Fig: \piccoC sem rules}
\end{figure*}


























\subsection{\piccoC\ Semantics} \label{smc C descr}


The \piccoC\ semantics are defined over multiple {\em interacting} parties.
The \piccoC\ semantics used to define the behavior of 
parties are mostly standard, with \emph{non-interactive} semantic rules identical to those of \vanillaC\ semantics aside from additional assertions over the privacy labels of data and properly managing the private data. 
A few notable exceptions are \emph{interactive} SMC operations (and in general operations over private values)
and the private-conditioned \Code{if else} statement, discussed in later in this section.    
To prevent leakage from within private-conditioned branches, we restrict all public side effects (i.e., the use of functions with public side effects, allocation and deallocation of memory, and any modifications to public variables). 
Additionally, in the case of pointer dereference write and array write statements, we have an additional check for when this occurs within a private-conditioned branch, as we need to perform additional analysis to ensure the location being written to is tracked properly due to the potential for the pointer's location being modified or an out-of-bounds array write. 
To enforce these restrictions, we use the assertion $\Acc = \AccZ$ within each restricted rule -- as the accumulator $\Acc$ is incremented at each level of nesting of a private-conditioned branch, this will result in a runtime failure. 
We annotate each evaluation with party-wise lists of the evaluation codes $\codeLL$ of all rules that were used during the execution of the rule (i.e., $\Deval{\locLL}{\codeLL}$) in order to keep an accurate evaluation tree, and party-wise lists of locations accessed $\locLL$ in order to show data-obliviousness (i.e., that given the same program and public data, we will always access the same set of locations). 


%%%%%%%%%%%%%%%%%%%%%%
%
% 		Basic Evaluations
%
%%%%%%%%%%%%%%%%%%%%%%
\subsubsection{Basic Evaluation} \label{subsec: picco basic eval}
% Less Than True
To better illustrate the correspondence between \piccoC\  and \vanillaC\  let us consider the Multiparty Binary Operation rule.
\piccoC\ rule Multiparty Binary Operation asserts that one of the given binary operators ($\cdot, +,-,\div$) is used and additionally that either expression contains private data with relation to the environment. 
We use the notation $(\Expr_{1}, \Expr_{2}) \isPriv \gamma$ to show this relation, and  
notation $\{...\}^{\pidZ}_{\pid = \pidA}$ to show that all parties will ensure that property holds locally.
We then use the multiparty protocol $\MPC{b}$, passing the given binary operator and the current values of $\n^\pid_1$ and $\n^\pid_2$ for each party $\pid$. 
This protocol will dictate how communication occurs and what data is exchanged between parties. 
We receive $\n^\pid_3$ as the result for each party, which we then return appropriately.   We assume that the protocol is implemented correctly (i.e. provided by the underlying SMC cryptographic library) and define this assumption formally, its impact on our noninterferences proof, and how to reason if a library adheres to our assumption later in Section~\ref{sec: noninterference}. 
Within the multiparty rules, each party maintains control of their own data, only sharing it with other parties in the ways dictated by the multiparty protocols. We choose to show the execution of the entire computational process here in order to emphasize what data is involved, and that each of the parties will take part in this computation. 

% Write Priv x = Pub
In rule Write Private Variable Public Value, we assert that $\Expr$ contains only public data,  
then evaluate $\Expr$ to $n$. We look up $\x$ in $\gamma$, asserting that it is a $\Priv\ \btype$ at location $\loc$. 
When we call $\Update$ to store this value to memory, we must pass $\Encrypt(n)$ as $\x$ is private and we are assigning it the public value $n$. 


%%%%%%%%%%%%%%%%%%%%%%
%
% 		Pointer Evaluations
%
%%%%%%%%%%%%%%%%%%%%%%
\subsubsection{Pointer Evaluations} \label{subsec: picco ptr eval}
In order to maintain data-oblivious execution, we need to allow storing multiple locations for pointers when they are modified within a private-conditioned branching statement. To achieve this, 
% Pointer structure
the structure of the data stored by pointers is as follows: the number $\nl$ of locations being pointed to; a list of $\nl$ locations being pointed to; a list of $\nl$ tags; and the level of indirection of the pointer. 
% 
The privacy labels of the byte-wise permissions corresponding to the number $\nl$ of locations to which the pointer refers, the list of $\nl$ locations, and the level of indirection will always be public, as it is visible to an observer of memory the number and which locations are touched by a pointer, and the level of indirection is visible in the source program.
%
The privacy labels of the permissions corresponding to the tags of public pointers will always be public; for private pointers, they will be private, as these protect an observer of memory from being able to tell which location is the true location. 
%
%%
% Priv Ptr Decl
In rule Private Pointer Declaration, we assert that the type of the variable being declared is an int or float pointer that is either declared as private or missing a privacy label (i.e., $\btype*$) and therefore is assumed to be private. 
As with the \vanillaC\ pointer declaration, we check the level of indirection and add the appropriate mappings to the environment and memory. 


%%%%%%%%%%%%%%%%%%%%%%
%
% 		Mem Alloc / Dealloc
%
%%%%%%%%%%%%%%%%%%%%%%
\subsubsection{Memory Allocation and Deallocation} \label{subsec: picco mem alloc/dealloc}
% PMalloc
When allocating private memory, we provide the \TT{pmalloc} builtin function to internally handle obtaining the size of the private type; the programmer to only needs to know how many elements of the given type they desire to allocate.
In rule Private Malloc, we assert that the given type is either private int or private float, as this function only handles those types, and that the accumulator $\Acc$ is $\AccZ$ (i.e., we are not inside an \Code{if else} statement branching on private data, as this function causes public side effects). 
Then we evaluate $\Expr$ to $n$ and obtain the next open memory location $\loc$ from $\phi$. We add to $\sigma{_1}$ the new mapping from $\loc$ to the tuple of a $\Null$ set of bytes; the type $\Type$; the size $n$; and a list of $\Priv$, $\PermF$ permissions. 
As with public \Code{malloc}, we return the new location, $(\loc, 0)$. 

% PFree
When deallocating private memory, we provide the \TT{pfree} builtin function to handle private pointers potentially having multiple locations. 
In the case of a single location, it behaves identically to Public Free; however, with multiple locations, we need to deterministically free a single location (which may or may not be the true location that was intended to be freed) to maintain data-obliviousness. We describe this case in more detail here.   
In rule Multiparty Private Free, we assert that $\x$ is a private pointer of type int or float, we are not inside a private-conditioned branch ($\Acc$ is $\AccZ$, as this rule causes public side effects), and that the number of locations the pointer refers to ($\nl$) is greater than 1 for all parties. 
We then assert that \emph{all} locations referred to by $\x$ are freeable (i.e., they are all memory blocks that were allocated via malloc) and proceed to retrieve the data that is stored for each of these locations.
This data and the tag lists are then passed to $\PFree$, as this is what we will need in order to privately free a location without revealing if it was the true location. 

To accomplish this, we must free one location based on publicly available information, regardless of the true location of the pointer. For that reason, and without loss of generality, we free the first location, $\loc_0$. 
Since $\loc_0$ may not be the true location and may be in use by other pointers, we need to do additional computation to maintain correctness without disclosing whether or not this was the true location. 
In particular, if $\loc_0$ is not the true location, we preserve the content of $\loc_0$ by obliviously copying it to the pointer's true location prior to freeing. 
This behavior is defined in function $\PFree$, and follows the strategy suggested in~\cite{Zhang18}.
$\PFree$ returns the modified bytes and tag lists. $\UpdateBytesFree$ then updates these in their corresponding locations in memory and marks the permissions of $\loc_0$ as $\PermN$ (i.e., this block has been freed). 
The remaining step is to update other pointers that stored $\loc_0$ on their lists to point to the updated location instead of $\loc_0$, which is accomplished by $\UpdatePtrLocs$. 



%%%%%%%%%%%%%%%%%%%%%%
%
% 		Array Evaluations
%
%%%%%%%%%%%%%%%%%%%%%%
\subsubsection{Array Evaluations} \label{subsec: picco arr eval}
With array evaluations, all evaluations that use a public index behave nearly identically to those over public data. 
The difference is that we have an additional check within array writes at a public index to see if we are in a private-conditioned branch; if so, we must ensure we properly track the modification made (this is because a public index that is not hard-coded could have lead to an out-of-bounds array write). 
We will discuss this further in the following section. 
When we have a private index, it is necessary to hide which location we a reading from or writing to to maintain data-obliviousness. One such semantic rule is Multiparty Array Read Private Index.
% 
% PUB 1D ARR READ PRIV INDEX
Here, we assert that the privacy label of $\Expr$ is private as this rule handles private indexing into a public array. 
We assert that $\x$ is a public array of type int or float, as we must return a private value when using a private index. 
Then we evaluate $\Expr$ to the private index $\ind$, and perform lookups to obtain the data of the array. 
Now, because we have a private index, we must obtain the value without revealing which location we are taking the value from. 
To do this, we use multiparty protocol $\MPC{ar}$, which returns a private number containing the value from the desired location.  
It is important to note here that even if the private index is beyond the bounds of the array, we do not access beyond the elements within the array, as that would reveal information about the private index. 
An example of how this protocol can be implemented is to iterate over all values stored in the array; at each value, we encrypt the current index number $m$, privately compare it to $\ind$, and perform a bitwise \Code{and} operation over this and the encrypted value $\n_m$ stored at index $m$. We perform a bitwise \Code{or} operation over each such value obtained from the array to attain our final encrypted value $\n$, which is returned. 
%









































%%%%%%%%%%%%%%%%%%%%%%%
%
%   picco C If Else description 
%
%%%%%%%%%%%%%%%%%%%%%%%

\subsubsection{\piccoC\ If Else} \label{subsec: \piccoC if else}

The public \Code{if else} rules 
are nearly identical to the \vanillaC\ rules, (e.g., \vanillaC\ rule If Else True shown in Figure~\ref{Fig: if else \vanillaC true}), 
with the added assertion that the guard of the conditional is public (i.e., does not contain private data): $(\Expr) \isPub \gamma$.
%
The private \TT{if else} rules, shown in Figures~\ref{Fig: iep vt} and~\ref{Fig: iep lt}, are more interesting. Our strategy for dealing with private-conditioned branches involves executing both branches as a sequence of statements (with some additional helper algorithms to aid in storing changes, restoration between branches, and resolution of true values). 
We chose to use big-step semantics to facilitate the comparison of the \piccoC\ semantics with the \vanillaC\ semantics, and for its proof of correctness that we will discuss in the next Section. 
% general description
We give also an example of \piccoC\ code in Figure~\ref{Fig: if else \piccoC code} and~\ref{Fig: if else \piccoC code dp}, and the corresponding execution in Figure~\ref{Fig: if else \piccoC expanded} and~\ref{Fig: if else piccoC expanded dp}. We use coloring throughout Figure~\ref{Fig: if else color} and~\ref{Fig: if else color dp} to highlight the corresponding sections of code and rule execution. 

% SPECIFICS  --  IF ELSE
The starting and ending states of the \piccoC\ Private If Else rules are essentially the same as the starting and ending states of the corresponding \vanillaC\ If Else rule; however, there are several additional assertions that guarantee that both of the private-conditioned branches are executed. 
The assertions of these semantic rules are listed sequentially, from top to bottom.
We have two different styles of tracking modifications within conditional code blocks that are used within these rules: variable tracking and location tracking. 
Variable tracking is used when there are only single-level changes within the private-conditioned branches, whereas location tracking is used when we have multi-level changes (i.e., a branch contains a pointer dereference write) or potential out-of-bounds changes (i.e., array write at a public index). 

The main idea of both styles is to first store the original value of each variable that is modified within either branch; execute the \TT{then} branch; save the resulting values from the \TT{then} branch and restore all modified variables to their original values; execute the \TT{else} branch; and finally, to securely resolve which values should be kept -- those from the \TT{then} branch or those from the \TT{else} branch. 
In the variable style of tracking, we utilize temporary variables to keep track of all modifications made during either branch -- initializing the \TT{else} temporary with the original value, storing the result of the \TT{then} branch in the \TT{then} temporary and using the \TT{else} temporary to restore the original value, and finally using the result of the private-conditional and what is stored in each variable at the end of the \TT{else} branch as well as it's corresponding \TT{then} temporary to securely resolve what values to continue evaluating the program with. 

This style of tracking is robust enough for many uses, however, there are two notable exceptions where we run into issues, both involving the potential of the location we track not being the location that is actually modified. 
The first exception involves pointer dereference writes -- these alone are not an issue, but when location the pointer refers to is modified and we also perform pointer dereference writes, it becomes clear that variable tracking cannot easily find and handle these cases. 
The second exception involves array writes at public indices -- these become problematic due to the potential for writing out-of-bounds. As most array indices are not hard-coded, it isn't obvious that the write will be within bounds until execution, and to ensure we catch all of these cases we must use a more robust style of tracking to catch out-of-bounds writes. We stress here that array writes at private indices do not fall within this exception, as this operation will securely update the array within its bounds (as updating beyond the bounds of the array would leak that this private value is larger than the size of the array), and as such we can simply track the entire array properly using variable tracking.  
It is possible to ensure that we find all of the locations that are modified in both of these cases by dynamically adding these types of modifications as they are evaluated, which is the goal of the location tracking. 
In the location style of tracking, we still follow a similar evaluation pattern as with variable tracking, storing the original values for locations we know will be modified first, then restoring between branches, and resolving at the end. As we evaluate each branch and come upon one of these special cases, we will check to see if we have already marked that location for tracking, and if not we add that location and its original value before the modification occurs. 
It is worthwhile to stress again the role of the accumulator here with respect to other statements. We increment it when we evaluate the \TT{then} and \TT{else} statements, so that if we attempt to evaluate a (sub)statement with public side effects or restricted operations, we have an (oblivious) runtime failure. It also facilitates scoping of temporary variables within nested private-conditioned \TT{if else} statements.
We proceed to further describe the different assertions and specifics of both styles next. 


% insert IF ELSE figure


\begin{figure*} \footnotesize
\begin{tabular}{l}
\hspace{0.3cm}
\begin{tabular}{l l}
\begin{subfigure}{.36\textwidth}
\begin{lstlisting}
private int a=3,b=7,c=0;		
if ($\Code{\ExprC{a<b}}$) $\Code{\sC{c=a;}}$
else $\Code{\ssC{c=b;}}$
\end{lstlisting}
	\caption{\piccoC\ code.}
	\label{Fig: if else \piccoC code}	
\end{subfigure}		
&
\begin{subfigure}{.57\textwidth}
\begin{lstlisting}
private int a=3,b=7,c=0,$\Code{\initC{res=}\ExprC{a<b},\initC{c\_t=c,c\_e=c};}$
$\Code{\sC{c=a;}}$ $\Code{\restC{c\_t=c; c=c\_e;}}$
$\Code{\ssC{c=b;}}$ $\resoC{\Code{c=(res}\cdot\Code{c\_t)+((1-res)}\cdot\Code{c);}}$ 	
\end{lstlisting}
	\caption{Variable-tracking execution.}	
	\label{Fig: if else \piccoC expanded}
\end{subfigure} 
\end{tabular}
\\ \\
\begin{subfigure}{\textwidth}
	\inferrule{\begin{array}{l} 
		\qquad
			\ExprC{((\pidA, \gamma^\pidA, \sigma^\pidA, \DMap^\pidA, \Acc, \Expr) \ \ \Mid ... \Mid
					 (\pidZ, \gamma^\pidZ, \sigma^\pidZ, \DMap^\pidZ, \Acc, \Expr))} 
			\crcr\ExprC{\Deval{\locLL_1}{\codeLL_1} 
			((\pidA, \gamma^\pidA, \sigma^\pidA_1, \DMap^\pidA_1, \Acc, n^\pidA) \Mid ... \Mid
			 (\pidZ, \gamma^\pidZ, \sigma^\pidZ_1, \DMap^\pidZ_1, \Acc, n^\pidZ))}
		\qq
			\{(\ExprC\Expr) \isPriv \gamma^\pid\}^\pidZ_{\pid = \pidA}
		\crcr 
			\extC{\{\DynExtract(}\sC{\stmt_1}\extC{,}\ \ssC{\stmt_2} \extC{, \gamma^\pid) = (\x_\vl, 0)\}^\pidZ_{\pid = \pidA}}
		\crcr
			\initC{\{\Initialize(}\extC{\x_\vl}\initC{,\ \gamma^\pid, \sigma^\pid_1, \ExprC{\n^\pid},\ \AccPP) = (\gamma^\pid_1, \sigma^\pid_2, \locL^\pid_2)\}^\pidZ_{\pid = \pidA}}
		\crcr\qquad
			\sC{((\pidA, \gamma^\pidA_1, \sigma^\pidA_2, \DMap^\pidA_1, \AccPP, \stmt_1) \ \ \ \Mid ... \Mid
			 	  (\pidZ, \gamma^\pidZ_1, \sigma^\pidZ_2, \DMap^\pidZ_1, \AccPP, \stmt_1))}
			\crcr\sC{\Deval{\locLL_3}{\codeLL_2} 
				((\pidA, \gamma^\pidA_2, \sigma^\pidA_3, \DMap^\pidA_2, \AccPP, \Skip) \Mid ... \Mid
				 (\pidZ, \gamma^\pidZ_2, \sigma^\pidZ_3, \DMap^\pidZ_2, \AccPP, \Skip))}
		\crcr
			\restC{\{\Restore(}\extC{\x_\vl}\restC{,\ \gamma^\pid_1,\ \sigma^\pid_3,\ \AccPP) = (\sigma^\pid_4, \locL^\pid_4)\}^\pidZ_{\pid = \pidA}}
		\crcr\qquad
			\ssC{((\pidA, \gamma^\pidA_1, \sigma^\pidA_4, \DMap^\pidA_2, \AccPP, \stmt_2) \ \ \ \Mid ... \Mid
				   (\pidZ, \gamma^\pidZ_1, \sigma^\pidZ_4, \DMap^\pidZ_2, \AccPP, \stmt_2))}
				\crcr\ssC{\Deval{\locLL_5}{\codeLL_3} 
				((\pidA, \gamma^\pidA_3, \sigma^\pidA_5, \DMap^\pidA_3, \AccPP, \Skip) \Mid  ... \Mid
				 (\pidZ, \gamma^\pidZ_3, \sigma^\pidZ_5, \DMap^\pidZ_3, \AccPP, \Skip))}
		\crcr
			\resoC{\{\Resolve\_\mathrm{Retrieve}(\extC{\x_\vl},\ \AccPP, \gamma^\pid_1, \sigma^\pid_5) 
				= ([(\val^\pid_{t1}, \val^\pid_{e1}), ..., (\val^\pid_{tm}, \val^\pid_{em})], 
					\ExprC{\n^\pid}, \locL^\pid_6)\}^\pidZ_{\pid = \pidA}}
		\crcr
			\resoC{\MPC{resolve}([\ExprC{\n^\pidA}, ..., \ExprC{\n^\pidZ}], 
				[[(\val^\pidA_{t1}, \val^\pidA_{e1}), ..., (\val^\pidA_{tm}, \val^\pidA_{em})]], ..., 
				 [(\val^\pidZ_{t1}, \val^\pidZ_{e1}), ..., (\val^\pidZ_{tm}, \val^\pidZ_{em})]])} 
				\crcr\qquad\resoC{= [[\val^\pidA_1, ..., \val^\pidA_m], ... [\val^\pidZ_1, ..., \val^\pidZ_m]]}
		\crcr
			\resoC{\{\Resolve\_\mathrm{Store}(\extC{\x_\vl},\ \AccPP, \gamma^\pid_1, \sigma^\pid_5, 
				[\val^\pid_1, ..., \val^\pid_m]) = (\sigma^\pid_6, \locL^\pid_7)\}^\pidZ_{\pid = \pidA}}
		\crcr
			\locLL_6 = \locLL_1 \addL (\pidA, \locL^\pidA_2) \Mid ... \Mid (\pidZ, \locL^\pidZ_2) \addL \locLL_3 
						\addL (\pidA, \locL^\pidA_4) \Mid ... \Mid (\pidZ, \locL^\pidZ_4) \addL \locLL_5
						\addL (\pidA, \locL^\pidA_6) \Mid ... \Mid (\pidZ, \locL^\pidZ_6) 
		\crcr 
			\locLL_7 = \locLL_6 \addL (\pidA, \locL^\pidA_7) \Mid ... \Mid (\pidZ, \locL^\pidZ_7)
		\qq 
			\codeLL_4 = \codeLL_1 \addC \codeLL_2 \addC \codeLL_3
	\end{array} }
	{\begin{array}{l} 
	((\pidA, \gamma^\pidA, \sigma^\pidA, \DMap^\pidA, \Acc, \If\ (\ExprC\Expr)\ \sC{\stmt_1}\ \Else\ \ssC{\stmt_2}) 
	 \Mid ... \Mid 
	 (\pidZ, \gamma^\pidZ, \sigma^\pidZ, \DMap^\pidZ, \Acc, \If\ (\ExprC\Expr)\ \sC{\stmt_1}\ \Else\ \ssC{\stmt_2})) 
		\Deval{\locLL_7}{\codeLL_4 \addC \codeSP{iep}} 
		\crcr ((\pidA, \gamma^\pidA, \resoC{\sigma^\pidA_{6}}, \resoC{\DMap^\pidA_{3}}, \Acc, \Skip) 
		\qq \-\ \-\ \Mid ... \Mid 
		 (\pidZ, \gamma^\pidZ, \resoC{\sigma^\pidZ_{6}}, \resoC{\DMap^\pidZ_{3}}, \Acc, \Skip) )
		\end{array}}
	\caption{\piccoC\ rule Private If Else - Variable Tracking.}
	\label{Fig: iep vt}
\end{subfigure}
\\ \\
\begin{subfigure}{\textwidth}
	\inferrule{\begin{array}{l}
		\ExprC{((\pid, \hgamma, \hsigma,\ \bsq, \bsq, \hExpr)\ \ \Mid \hConfig)\ \ }
		\ExprC{\Veval_{\codeVLL_1}\ }
			\ExprC{((\pid, \hgamma,\ \hsigma_1, \bsq, \bsq, \hat{n})\ \ \ \Mid \hConfig_1)}
		\qq \ExprC{\hat{n}} \neq \ExprC{0}
		\crcr\sC{((\pid, \hgamma, \hsigma_1, \bsq, \bsq, \hstmt_1) \Mid \hConfig_1)\ }  
			\sC{\Veval_{\codeVLL_2}((\pid, \hgamma_1, \hsigma_2, \bsq, \bsq, \Skip)\Mid \hConfig_2)}
	\end{array}}
	{\begin{array}{l}
	((\pid, \hgamma, \hsigma, \bsq, \bsq, \If (\ExprC{\hExpr})\ \sC{\hstmt_1}\ \Else\ \ssC{\hstmt_2}) \Mid \hConfig)
		\Veval_{\codeVLL_1\addC\codeVLL_2\addC[\codeVS{iet}]} 
		((\pid, \hgamma, \hsigma_2, \bsq, \bsq, \Skip) \Mid \hConfig_2)\end{array}}
	\caption{\vanillaC\ rule If Else True.}	
	\label{Fig: if else \vanillaC true}
\end{subfigure}	
\end{tabular}
\caption{\Code{if else} branching on private data example (\ref{Fig: if else \piccoC code}, \ref{Fig: if else \piccoC expanded}) matching to the \piccoC\ variable-tracking (\ref{Fig: iep vt}) and \vanillaC\ (\ref{Fig: if else \vanillaC true}) semantic rules. Coloring in the rules highlight the corresponding code and rule execution.}
\label{Fig: if else color}
\end{figure*}







































%


\paragraph{Conditional Code Block Variable Tracking}
For this style of tracking, we first evaluate expression $\Expr$ over environment $\gamma$, memory $\sigma$ and accumulator $\Acc$ 
to obtain some number $n$; the same environment, 
and a potentially updated memory (e.g. in the case $\Expr = x++$). 
% green / extract _vars
We then extract the non-local variables that are modified within either branch, and check whether multi-level modifications or array writes at a public index occur. This is achieved with Algorithm $\DynExtract$ by iterating through both statement $\stmt_1$ and $\stmt_2$ and storing the variable names in list $\vl_{\Acc+1}$, as well as updating and returning a tag to indicate whether we have found multi-level modifications (0 for false, 1 for true).
%
% red / initialize_vars
Next we call Algorithm $\Initialize$, which 
stores $n$ as the value of a temporary variable $\res_{\Acc+1}$, using $\Acc +1$ to denote the current level of nesting in the upcoming \Code{then} and \Code{else} statements. The variable $\res_{\Acc+1}$  is later used in the resolution phase, to select the result according to the branching condition. 
It then iterates through the list of variables, creating two temporary versions of each variable, named $\x\_{then\_\Acc}$ and $\x\_{else\_\Acc}$, and storing each in memory with the initial value of what $\x$ has in the memory $\sigma_1$. 
%
% light blue / stmt 1
Next is the evaluation of the \TT{then} statement, and afterwards 
%
%
% yellow / restore_vars
we must restore the original memory. %, such that $\sigma_1\subset\sigma_5$. 
To do this, we call $\Restore$, which iterates through each of the variables $\x$ contained within $\vl_{\Acc+1}$, storing their current value into their \Code{then} temporary (i.e., $\x\_\mathit{then}_{\Acc+1} = \x$) and restoring their original value from their \Code{else} temporary (i.e., $\x = \x\_\mathit{else}_{\Acc+1}$). 
Once we have completed this, 
%
% light purple / stmt 2
 the evaluation of the \TT{else} statement can occur. 
 
 
% orange / resolve_vars
Finally, we need to perform the resolution of all changes made to variables in either branch. 
To enable this, we call Algorithm $\ResolveR$ to iterate through each of the variables within $\vl_{\Acc+1}$ and grab their values accordingly, as well as retrieving the result of the private condition (whose value we stored in $\res_{\Acc+1}$). 
We then use multiparty protocol $\MPC{resolve}$ to facilitate the resolution of the true values, as these computations require communication between parties. 
For variables that are not array or pointer variables (e.g., those in~\ref{Fig: if else \piccoC code}), we perform a series of binary operations over the byte values of the private variables as shown in~\ref{Fig: if else \piccoC expanded} (e.g., \Code{c=(res$\cdot$c\_t)+((1-res)$\cdot$c\_e)}). 
The process is similar for arrays, with some addition bookkeeping due to their structure as a const pointer referring to the location with the array data. 
For pointers, we must handle the different locations referred to from each branch, merging the two location lists and finding what the true location is. 
The resolved values are then returned, and Algorithm $\ResolveS$ stores all each back into memory for its respective variable. 
Notice that, in the conclusion, we revert to the original environment $\gamma$. In this way, all the temporary variables we used become out of scope and are discarded - in particular, this prevents reusing the same temporary variable mapping if we have multiple (not nested) private if else statements.





\begin{figure*} \footnotesize
\begin{tabular}{l}
\begin{tabular}{l l}
\hspace{0.3cm}
\begin{subfigure}{.27\textwidth}
\begin{lstlisting}
private int a=3,
		b=7,c=5,*p=&a;
if ($\Code{\ExprC{a>b}}$)
	$\Code{\sC{*p=c;}}$ 
else 
	$\Code{\ssC{p=\&b;}}$
\end{lstlisting}
	\caption{\piccoC\ code.}
	\label{Fig: if else \piccoC code dp}	
\end{subfigure}		
&
\begin{subfigure}{0.69\textwidth}
\begin{lstlisting}[emph={[2]res, resolve}, emphstyle={[2]\color{blue}}]
private int a=3,b=7,c=5,*p=&a,$\Code{\initC{res=}\ExprC{a>b}}$;
$\initC{\DMap[\Acc][\loc_\TT{p}]=([1,[(\loc_\TT{a}, 0)],[1],1],[],0, \Code{private int*});}$  
$\sC{\DMap[\Acc][\loc_\TT{a}]=(3,0,0,\Code{private int});}$   $\sC{\Code{*p=c;}}$ 
$\restC{\DMap[\Acc][\loc_\TT{p}]=([1,[(\loc_\TT{a}, 0)],[1],1],[1,[(\loc_\TT{a}, 0)],[1],1],1, \Code{private int*});}$
$\restC{\DMap[\Acc][\loc_\TT{a}]=(3,5,1, \Code{private int});}$ $\restC{\loc_\TT{p}=\DMap[\Acc][\loc_\TT{p}][0];}$  $\restC{\loc_\TT{a}=\DMap[\Acc][\loc_\TT{a}][0];}$
$\Code{\ssC{p=\&b;}}$
$\resoC{\loc_\TT{p}=\Code{resolve(res,}\DMap[\Acc][\loc_\TT{p}],\loc_\TT{p});}$   $\resoC{\loc_\TT{a}=\Code{resolve(res,}\DMap[\Acc][\loc_\TT{a}],\loc_\TT{a});}$
\end{lstlisting}
	\caption{Location-tracking execution.}	
	\label{Fig: if else piccoC expanded dp}
\end{subfigure} 
\end{tabular}
\\ \\
\begin{subfigure}{\textwidth}
\inferrule{\begin{array}{l} 
		\qquad
			\ExprC{((\pidA, \gamma^\pidA, \sigma^\pidA, \DMap^\pidA, \Acc, \Expr) \ \ \Mid ... \Mid
				 	 (\pidZ, \gamma^\pidZ, \sigma^\pidZ, \DMap^\pidZ, \Acc, \Expr))} 
			\crcr\ExprC{\Deval{\locLL_1}{\codeLL_1} 
			((\pidA, \gamma^\pidA, \sigma^\pidA_1, \DMap^\pidA_1, \Acc, n^\pidA) \Mid ... \Mid
			  (\pidZ, \gamma^\pidZ, \sigma^\pidZ_1, \DMap^\pidZ_1, \Acc, n^\pidZ))}
		\qq 
			\{(\ExprC\Expr) \isPriv \gamma^\pid\}^\pidZ_{\pid = \pidA}
		\crcr 
			\extC{\{\DynExtract(}\sC{\stmt_1}\extC{,}\ \ssC{\stmt_2} \extC{, \gamma^\pid) = (\x_\vl, 1)\}^\pidZ_{\pid = \pidA}}
		\crcr
			\initC{\{\DynInit(\DMap^\pid_1,}\ \extC{\x_\vl}\initC{,\ \gamma^\pid,\ \sigma^\pid_1,\ \ExprC{\n^\pid}, \AccPP) = (\gamma^\pid_1, \sigma^\pid_2, \DMap^\pid_2, \locL^\pid_2)\}^\pidZ_{\pid = \pidA}}
		\crcr\qquad 
			\sC{((\pidA, \gamma^\pidA_1, \sigma^\pidA_2, \DMap^\pidA_2, \AccPP, \stmt_1) \ \ \ \Mid  ... \Mid
				  (\pidZ, \gamma^\pidZ_1, \sigma^\pidZ_2, \DMap^\pidZ_2, \AccPP, \stmt_1))}
				\crcr\sC{\Deval{\locLL_3}{\codeLL_2} 
				((\pidA, \gamma^\pidA_2, \sigma^\pidA_3, \DMap^\pidA_3, \AccPP, \Skip) \Mid  ... \Mid
				 (\pidZ, \gamma^\pidZ_2, \sigma^\pidZ_3, \DMap^\pidZ_3, \AccPP, \Skip))}
		\crcr
			\restC{\{\DynRestore(\sigma^\pid_3, \DMap^\pid_3, \AccPP) = (\sigma^\pid_4, \DMap^\pid_4, \locL^\pid_4)\}^\pidZ_{\pid = \pidA}}
		\crcr\qquad
			\ssC{((\pidA, \gamma^\pidA_1, \sigma^\pidA_4, \DMap^\pidA_4, \AccPP, \stmt_2) \ \ \ \Mid  ... \Mid
				    (\pidZ, \gamma^\pidZ_1, \sigma^\pidZ_4, \DMap^\pidZ_4, \AccPP, \stmt_2))}
				\crcr\ssC{\Deval{\locLL_5}{\codeLL_3} 
				((\pidA, \gamma^\pidA_3, \sigma^\pidA_5, \DMap^\pidA_5, \AccPP, \Skip) \Mid  ... \Mid
				  (\pidZ, \gamma^\pidZ_3, \sigma^\pidZ_5, \DMap^\pidZ_5, \AccPP, \Skip))}
		\crcr
			\resoC{\{\DynResolve\_\mathrm{Retrieve}(\gamma^\pid_1, \sigma^\pid_5, \DMap^\pid_5, \AccPP) 
				= ([(\val^\pid_{t1}, \val^\pid_{e1}), ..., (\val^\pid_{tm}, \val^\pid_{em})], 
					\ExprC{\n^\pid}, \locL^\pid_6)\}^\pidZ_{\pid = \pidA}}
		\crcr
			\resoC{\MPC{resolve}([\ExprC{\n^\pidA}, ..., \ExprC{\n^\pidZ}], 
				[[(\val^\pidA_{t1}, \val^\pidA_{e1}), ..., (\val^\pidA_{tm}, \val^\pidA_{em})]], ..., 
				 [(\val^\pidZ_{t1}, \val^\pidZ_{e1}), ..., (\val^\pidZ_{tm}, \val^\pidZ_{em})]])} 
				\crcr\qquad\resoC{= [[\val^\pidA_1, ..., \val^\pidA_m], ... [\val^\pidZ_1, ..., \val^\pidZ_m]]}
		\crcr
			\resoC{\{\DynResolve\_\mathrm{Store}(\DMap^\pid_5, \sigma^\pid_5, \AccPP, [\val^\pid_1, ..., \val^\pid_m]) 
				= (\sigma^\pid_6, \DMap^\pid_6, \locL^\pid_7)\}^\pidZ_{\pid = \pidA}}
		\crcr
			\locLL_6 = \locLL_1 \addL (\pidA, \locL^\pidA_2) \Mid ... \Mid (\pidZ, \locL^\pidZ_2) \addL \locLL_3 
						\addL (\pidA, \locL^\pidA_4) \Mid ... \Mid (\pidZ, \locL^\pidZ_4) \addL \locLL_5
						\addL (\pidA, \locL^\pidA_6) \Mid ... \Mid (\pidZ, \locL^\pidZ_6) 
		\crcr 
			\locLL_7 = \locLL_6 \addL (\pidA, \locL^\pidA_7) \Mid ... \Mid (\pidZ, \locL^\pidZ_7)
		\qq 
			\codeLL_4 = \codeLL_1 \addC \codeLL_2 \addC \codeLL_3
	\end{array} }
	{\begin{array}{l} 
	((\pidA, \gamma^\pidA, \sigma^\pidA, \DMap^\pidA, \Acc, \If\ (\ExprC\Expr)\ \sC{\stmt_1}\ \Else\ \ssC{\stmt_2}) 
	 \Mid ... \Mid 
	 (\pidZ, \gamma^\pidZ, \sigma^\pidZ, \DMap^\pidZ, \Acc, \If\ (\ExprC\Expr)\ \sC{\stmt_1}\ \Else\ \ssC{\stmt_2})) 
		 \Deval{\locLL_7}{\codeLL_4 \addC \codeMP{iepd}} 
		 \crcr ((\pidA, \gamma^\pidA, \resoC{\sigma^\pidA_{6}}, \resoC{\DMap^\pidA_{6}}, \Acc, \Skip) 
		 	 \qq \-\ \-\ \Mid ... \Mid
		  	(\pidZ, \gamma^\pidZ, \resoC{\sigma^\pidZ_{6}}, \resoC{\DMap^\pidZ_{6}}, \Acc, \Skip))
		\end{array}}
	\caption{\piccoC\ rule Private If Else - Location Tracking.}
	\label{Fig: iep lt}
\end{subfigure}
\\ \\
\begin{subfigure}{\textwidth}
Multiparty If Else False  		\\
\inferrule{\begin{array}{l}
	\qquad 
		\ExprC{((\pidA, \hgamma,\ \hsigma,\ \bsq, \bsq, \hExpr)\ \ \ \Mid ... \Mid
		  (\pidZ, \hgamma,\ \hsigma,\ \bsq, \bsq, \hExpr))} 
	 \crcr \ExprC{\Veval_{\codeVLL_1} 
			((\pidA, \hgamma,\ \hsigma_1, \bsq, \bsq, \hat{n})\ \ \ \Mid ... \Mid
			  (\pidZ, \hgamma,\ \hsigma_1, \bsq, \bsq, \hat{n}))}
	\qq \ExprC\hn = 0
	\crcr\qquad \sC{((\pidA, \hgamma,\ \hsigma_1, \bsq, \bsq, \hstmt_1)\ \ \Mid ... \Mid
			   (\pidZ, \hgamma,\ \hsigma_1, \bsq, \bsq, \hstmt_1))} 
	\crcr \sC{\Veval_{\codeVLL_2} 
			((\pidA, \hgamma_1, \hsigma_2, \bsq, \bsq, \Skip) \Mid ... \Mid
			 (\pidZ, \hgamma_1, \hsigma_2, \bsq, \bsq, \Skip))}
	\crcr\qquad \ssC{((\pidA, \hgamma,\ \hsigma_1, \bsq, \bsq, \hstmt_2)\ \ \Mid ... \Mid
			   (\pidZ, \hgamma,\ \hsigma_1, \bsq, \bsq, \hstmt_2))} 
	\crcr \ssC{\Veval_{\codeVLL_3} 
			((\pidA, \hgamma_2, \hsigma_3, \bsq, \bsq, \Skip) \Mid ... \Mid
			 (\pidZ, \hgamma_2, \hsigma_3, \bsq, \bsq, \Skip))}
	\end{array}}
	{\begin{array}{l}
	((\pidA, \hgamma, \hsigma,\ \bsq, \bsq, \If (\hExpr)\ \hstmt_1\ \Else\ \hstmt_2)\Mid ... \Mid
	  (\pidZ, \hgamma, \hsigma,\ \bsq, \bsq, \If (\hExpr)\ \hstmt_1\ \Else\ \hstmt_2)) 
		\Veval_{\codeVLL_1\addC\codeVLL_2\addC\codeVLL_3\addC[\codeVS{mpief}]} 
		\crcr
		((\pidA, \hgamma, \ssC{\hsigma_3}, \bsq, \bsq, \Skip)
		\qq\ \Mid ... \Mid
		 (\pidZ, \hgamma, \ssC{\hsigma_3}, \bsq, \bsq, \Skip))
		 \end{array}}
	\caption{\vanillaC\ rule Multiparty If Else False.}
	\label{Fig: van mpief}
\end{subfigure}
\end{tabular}
\caption{\Code{if else} branching on private data example (\ref{Fig: if else \piccoC code dp}, \ref{Fig: if else piccoC expanded dp}), \piccoC\ location-tracking (\ref{Fig: iep lt}), and \vanillaC\ Multiparty If Else False (\ref{Fig: van mpief}) rules. Coloring in the rules highlight the corresponding code and rule execution.}
%
\label{Fig: if else color dp}
\end{figure*}

































\paragraph{Conditional Code Block Location Tracking}
\label{sec: priv if lt desc}
Here we track modifications during private-conditioned branches at the level of memory blocks and offsets, which ensures that we do not update any data in memory inaccurately, as is shown in Figure~\ref{Fig: simple pointer challenge ex} using variable tracking SMC techniques.  
To facilitate this, we use the mapping structure $\DMap$ to track changes to each location at each level of nesting. This structure maps locations to a four-tuple of the original value, the \TT{then} branch value, a tag to notate whether the \TT{then} branch value was updated during the restoration phase, and the type of value stored (i.e., $(\loc, \offset) \to$ $(\val_1, \val_2$, $\tagb$, $\Type$)). The tag is used to allow us to add to $\DMap$ as we encounter pointer dereference writes and array writes at public indices without needing to track which branch we are in. It is always initialized as 0, and updated to 1 when we enter the restoration phase and store a value into the \TT{then} position. This way, if a location was added in the \TT{else} branch (i.e., was not modified in the \TT{then} branch), we know to use the original value as the \TT{then} value when we resolve the true value of that location at the end. 

The overall structure of the location tracking rule is similar to the variable tracking rule. 
We first evaluate $\Expr$ to $\n$, then call $\DynExtract$ to find variables that are modified during the execution of either branch and that there are multi-level modifications within at least one branch. 
We then call $\DynInit$, which stores the result of the private conditional and uses the variables we found to create the initial mapping $\DMap$.  
Next, we proceed to evaluate the \TT{then} branch, and call $\DynRestore$ to update $\DMap$ with the ending \TT{then} values for all locations that are tracked and restore the original values back into memory. 
After, we evaluate the \TT{else} branch and, once complete, call $\DynResolveR$ to retrieve the result of the conditional and the \TT{then} and \TT{else} values for each location. 
As with variable tracking, we use multiparty protocol $\MPC{resolve}$ to obtain the true values, and then store them back into their respective locations using Algorithm $\DynResolveS$. 
%
It is important to note that when we evaluate a pointer dereference write or array write at a public index inside a branch, we check to see if the given location is in $\DMap[\Acc]$.  If it is not, we add a mapping to store the original data (i.e., $(\loc, \offset) \to$ (\TT{orig}, $\Null$, $0$, $\Type$)). Notice that the data can either be a regular value (i.e., for a memory block storing a private int) or a pointer data structure representing a private pointer (i.e., for a memory block storing a private int*). 

In Figure~\ref{Fig: if else piccoC expanded dp}, we show an approximation of the execution of the pointer challenge example shown in Figure~\ref{Fig: simple pointer challenge ex}. 
When we reach the private-conditioned branching statement, we first store the result of the condition \TT{a < b}. As we execute the \TT{then} branch, we add the entry for $\loc_\TT{a}$ to $\DMap$, as \TT{p} refers to \TT{a}. We restore between branches by resetting $\loc_\TT{a}$ to its original value stored in $\DMap[\loc_\TT{a}][0]$. As we execute the \TT{else} branch, we add the entry for $\loc_\TT{p}$ to $\DMap$, as we are modifying which location \TT{p} points to. Finally, we resolve the true values for each modified location in $\DMap$. This approach eliminates the issues shown in Figure~\ref{Fig: simple pointer challenge ex}, as we do not rely on the pointer's current location to appropriately resolve the true values. 










\subsection{Overshooting Memory Bounds} \label{Sec: Overshooting}
%
It is possible to overshoot memory bounds in both \vanillaC\ and \piccoC. 
When overshooting occurs, we read the bytes of data as the type we expected it to be (i.e., bytes containing private data accessed by a public variable would be decoded as though they are public - no encryption or decryption occurs, but computations using the variable beyond that point will be garbage). 
This ensures that no information about private data can be leaked when overshooting. 
This is discussed further in Appendix~\ref{app: array oob} for the interested reader. 
We can only prove correctness over well-aligned accesses (i.e., those that iterate only over aligned elements of the same type, as with one array spilling into a subsequent array), as these would still produce readable data that is not garbage. 
When proving noninterference, we must prove that these cases (particularly those involving private data) cannot leak any information about the private data that is affected. We discuss this in more detail in the following section.










\section{Metatheory}\label{Sec: Metatheory}


In this section we present the main methatheoretic results, 
with proof sketches and important definitions given in Appendix~\ref{app: metatheory}.
We will begin by discussing how we leverage multiparty protocols, then proceed to discuss 
the most challenging result, which is correctness. 
Once correctness is proven, noninterference follows from a standard argument, with some adaptations needed to deal with the fact that private data is encrypted and 
that we want to show indistinguishability of evaluation traces. 



\subsection{Multiparty Protocols}
In our semantics, we leverage multiparty protocols to compartmentalize
the complexity of handling private data. In the formal treatment this
corresponds to using Axioms in our proofs to reason about
protocols. These Axioms allow us to guarantee the desired properties
of correctness and noninterference for the overall model, to provide
easy integration with new, more efficient protocols as they become
available, and to avoid re-proving the formal guarantees for the
entire model when new protocols are added.  Proving that these Axioms
hold is a responsibility of the library implementor in order to have
the system fully encompassed by our formal model.  Secure
multiparty computation protocols that already come with guarantees
of correctness and security are the only ones worth considering, so the implementor would only need to
ensure that these guarantees match our definitions of correctness and
noninterference.


For example, if private values are represented using Shamir secret sharing~\cite{Shamir79}, Algorithm~\ref{algo: mpc mult}, $\MPC{mult}$, represents
a simple multiparty protocol for multiplying
private values from~\cite{Gennaro98}.
In Algorithm~\ref{algo: mpc mult}, lines 2 and
3 define the protocol, while lines 1, 4, and 5 relate the
protocol to our semantic representation.

\begin{algorithm*}\footnotesize
\caption{$\n^\pid_3 \gets \MPC{mult}(\n^\pid_1, \n^\pid_2)$}
\label{algo: mpc mult}
\begin{algorithmic}[1]
	\STATE Let $f_a(\pid) = \n^\pid_1$ and $f_b(\pid) = \n^\pid_1$.
	\STATE Party $\pid$ computes the value $f_a(\pid) \cdot f_b(\pid)$ and creates its shares by choosing a random polynomial $h_\pid(x)$ of degree $t$, such that $h_\pid(0)=f_a(\pid) \cdot f_b(\pid)$. Party $\pid$ sends to each party $i$ the value $h_\pid(i)$. 
	\STATE After receiving shares from all other parties, party $\pid$ computes their share of $a \cdot b$ as the linear combination $H(\pid) = \sum^{\pidZ}_{i=1} \lambda_i h_i(\pid)$.
	\STATE Let $n^\pid_3 = H(\pid)$
	\RETURN $n^\pid_3$
\end{algorithmic}
\end{algorithm*}

When computation is performed by $q$ parties, at most $t$ of whom may collude ($t < q/2$), Shamir secret sharing encodes a private integer $a$ by choosing a polynomial $f(x)$ of degree $t$ with random coefficients such that $f(0) = a$ (all computation takes place over a finite field). Each participant obtains evaluation of $f$ on a unique non-zero point as their representation of private $a$; for example, party $\pid$ obtains $f(\pid)$. This representation has the property that combining $t$ or fewer shares reveals no information about $a$ as all values of $a$ are equally likely; however, possession of $t+1$ or more shares permits recovering of $f(x)$ via polynomial interpolation and thus learning $f(0) = a$. 

Multiplication in Algorithm~\ref{algo: mpc mult} corresponds to each party locally multiplying shares of inputs $a$ and $b$, which computes the product, but raises the polynomial degree to $2t$. The parties consequently re-share their private intermediate results to lower the polynomial degree to $t$ and re-randomize the shares. Values $\lambda_\pid$ refer to interpolation coefficients which are derived from the computation setup and party $\pid$ index.

In order to preserve the correctness and noninterference guarantees of our
model when such an algorithm is added, a library developer will need
to guarantee that the implementation of this algorithm is correct, meaning that it has the expected input output behavior, and it guarantees noninterference on what is observable. 


\subsection{Correctness} \label{sec: erasure} 
We first show the correctness of the \piccoC\ semantics with respect
to the \vanillaC\ semantics. As usual we will do this by establishing
a simulation relation between a \piccoC\ program and a corresponding
\vanillaC\ program. To do so we face two main challenges.

First, we need to guarantee that
the private operations in a \piccoC\ program are reflected in the
corresponding \vanillaC\ program and that the evaluation steps between the two programs correspond. 
To address the former issue, we define an \emph{erasure function} $\bm{\erasure}$ which
translates a \piccoC\ program into a \vanillaC\ program by erasing all
labels and replacing all functions specific to \piccoC\ with their public equivalents. This function also translates memory.
As an example, let us consider
pmalloc; in this case, we have
$\bm{\erasure}({\PMalloc(\Expr,\ \Type)} 
= {(\Malloc(\bm{\erasure}(\Expr) \cdot \sizeof(\bm{\erasure}(\Type))))})$.
That is, pmalloc is rewritten to use malloc, and since the given private type is now public we can use the sizeof function to find the size we will need to allocate. 
To address the latter issue, we have defined our operational semantics in terms of big-step evaluation judgments which allow the evaluation trees of the two programs to have a corresponding structure. In particular, notice how we designed
the Private If Else rule to perform multiple operations in one step, guaranteeing that we have similar ``synchronization points'' in the two evaluation trees. 

Second, we need to guarantee that at each evaluation step the memory
used by a \piccoC\ program corresponds to the one used by the
\vanillaC\ program. 
Given that we simulate multiparty execution over $\pidZ$ parties in \piccoC, we will also use $\pidZ$ parties in \vanillaC. This allows us to easily reason about both local and global semantic rules, as each \piccoC\ party has a corresponding \vanillaC\ party at an identical position in the evaluation trace.
Unfortunately, just applying the function $\bm{\erasure}$ to the \piccoC\ memories in the evaluation trace is not enough. 
%
In our setting, with
explicit memory management, manipulations of pointers, and array overshooting, guaranteeing a correspondence between the memories becomes particularly
challenging. To better understand the issue here, let us consider
the rule Private Free, discussed in
Section~\ref{subsec: picco mem alloc/dealloc}. 
Remember that our semantic model associates a pointer
with a list of locations, and the Private Free rule frees the
first location in the list, and relocates the content of that location
if it is not the true location. 
Essentially, this rule may swap the content of two locations if the first location
in the list is not the location intended to be freed and 
make the \piccoC\ memory and the \vanillaC\ memory look quite
different. To address this challenge in the proof of correctness, we use a \emph{map}, denoted $\psi$,
to track the swaps that happen when the rule
Private Free is used. The simulation uses and modifies this map to
guarantee that the two memories correspond.
%
Another related challenge comes from array overshooting. If, by
overshooting an array, a program goes over or into memory blocks of
different types, we may end up in a situation where the locations in
the \piccoC\ memory are significantly different from the ones in the
\vanillaC\ memory. This is mostly due to the size of private types
being larger than their public counterpart. One option to address this
problem would be to keep a more complex map between the two
memories. However, this can result in a much more complex proof, for
capturing a behavior that is faulty, in principle. Instead, we prefer
to focus on situations where overshooting arrays are \emph{well-aligned}, in the sense that they access only memory locations and blocks of the right type and size. 
An illustration of this is given in the Appendix, Figure~\ref{fig: overshooting alignment}. 

Before stating our correctness, we need to introduce some notation.  We
use party-wise lists of codes $\codeLL = (\pidA, [\code_1,\ldots,\code_n])\Mid ... \Mid(\pidZ, [\code_1,\ldots,\code_n]),$ $\codeVLL = (\pidA, [\codeV_1,\ldots,\codeV_m])\Mid$ $...$ $\Mid(\pidZ, [\codeV_1,\ldots,\codeV_m])$ in evaluations (i.e.,
$\eval_{\codeLL})$ to describe the rules of the semantics that
are applied in order to derive the result.  We write
$\codeLL\cong \codeVLL$ to state that the \piccoC\
codes are in correspondence with the \vanillaC\
codes, $\codeLL^\pid$ to denote the list of codes for a specific party $\pid$, and $\codeLL_1::\codeLL_2$ to denote concatenation of the party-wise evaluation code lists. 
We write $\{...\}^\pidZ_{\pid =1}$ to show that an assertion holds for all parties. 
Almost every \piccoC\ rule is in one-to-one
correspondence with a single \vanillaC\ rule within an execution trace (exceptions being private-conditioned branches, \TT{pmalloc}, and multiparty comparison operations).

 We write ${\stmt}\cong \hstmt$ to state that the \vanillaC\ configuration statement $\hstmt$ can be obtained by applying the erasure function to the \piccoC\ statement ${\stmt}$. Similarly, we can extend this notation to configuration by also using the map $\psi$. That is, we write
$(\pid,$ $\gamma{},$ $\sigma{},$ $\DMap$, $\Acc,$ ${\stmt})$ $\cong_\psi$ $(\pid,$ $\hgamma{},$ $\hsigma{},$ $\bsq,$ $\bsq,$ $\hstmt)$ to state that the \vanillaC\ configuration  $(\pid,$ $\hgamma{},$ $\hsigma{},$ $\bsq,$ $\bsq,$ $\hstmt)$ can be obtained by applying the erasure function to the \piccoC\ configuration  $(\pid,$ $\gamma{},$ $\sigma{},$ $\DMap$, $\Acc,$ ${\stmt})$, and memory $\hsigma$ can be obtained from $\sigma{}$ by using the map $\psi$.


We state correctness in terms of  evaluation trees, since we will use evaluation trees to prove a strong form of noninterference  in the next subsection. We use capital Greek letters $\Pi, \Sigma$ to denote evaluation trees.  In the \piccoC\ semantics, we write $\Pi \deriv ((\pidA, \gamma^{{\pidA}}_{},$ $\sigma^{{\pidA}}_{},$ $\DMap^{{\pidA}}_{}$, $\Acc^{{\pidA}}_{},$ $\stmt^{{\pidA}})\ \Mid ...\Mid$ 
	$(\pidZ, \gamma^{{\pidZ}}_{},$ $\sigma^{{\pidZ}}_{},$ $\DMap^{{\pidZ}}_{}$, $\Acc^{{\pidZ}}_{},$ $\stmt^{{\pidZ}}))$ 
	$\Deval{\locLL}{\codeLL}$ 
	$((\pidA, {\gamma^{{\pidA}}_{1}},$ ${\sigma^{{\pidA}}_{1}},$ $\DMap^{{\pidA}}_{1}$, $\Acc^{{\pidA}}_{1},$ ${\val^{{\pidA}}_{}}) \Mid ... \Mid$ 
	$(\pidZ, {\gamma^{{\pidZ}}_{1}},$ ${\sigma^{{\pidZ}}_{1}},$ $\DMap^{{\pidZ}}_{1}$, $\Acc^{{\pidZ}}_{1},$ $\val^{{\pidZ}}))$, to stress that the evaluation tree $\Pi$ proves as conclusion that, for each party $\pid$, configuration  $(\pid,$ $\gamma^\pid,$ $\sigma^\pid,$ $\DMap^\pid$, $\Acc^\pid,$ $\stmt^\pid)$ evaluates to configuration
$(\pid,$ $\gamma^\pid_1,$ $\sigma^\pid_1,$ $\DMap^\pid_1$, $\Acc^\pid_1,$ $\val^\pid)$ by means of the codes in $\codeLL^\pid$. Similarly, for the \vanillaC\ semantics. We then write $\Pi\cong_\psi \Sigma$ for the extension to evaluation trees of the congruence relation with map $\psi$.


In order to properly reason about global multiparty rules, we must assert that all parties are executing from the same original program with corresponding start states and input. 
To do this, we first show that the non-determinism of the semantics will always bring all parties to the same outcome: given $\pidZ$ parties with corresponding start states, if we reach intermediate states that are not corresponding for one or more parties, then there exists a set of steps that will bring all parties to corresponding states again.


\begin{theorem}[Confluence]
\label{Thm: confluence}
Given ${\Config^\pidA} \Mid ... \Mid {\Config^\pidZ}$ such that $\{{\Config^\pidA} \sim {\Config^\pid}\}^{\pidZ}_{\pid = \pidA}$ 
\\
if $({\Config^\pidA} \Mid ... \Mid {\Config^\pidZ})$ $\Deval{{\locLL_1}}{{\codeLL_1}}$ $({\Config^\pidA_1} \Mid ... \Mid {\Config^\pidZ_1})$ such that $\exists \pid\in\{\pidA...\pidZ\} {\Config^\pidA_1} \not\sim {\Config^\pid_1}$, 
\\
then $\exists$ $({\Config^\pidA_1} \Mid ... \Mid {\Config^\pidZ_1})$ $\Deval{{\locLL_2}}{{\codeLL_2}}$ $({\Config^\pidA_2} \Mid ... \Mid {\Config^\pidZ_2})$
\\ 
such that $\{{\Config^\pidA_2} \sim {\Config^\pid_2}\}^{\pidZ}_{\pid = \pidA}$, 
$\{({\locLL^\pidA_1}\addL{\locLL^\pidA_2}) = ({\locLL^\pid_1}\addL{\locLL^\pid_2})\}^{\pidZ}_{\pid = \pidA}$, 
and $\{({\codeLL^\pidA_1}\addC{\codeLL^\pidA_2}) = ({\codeLL^\pid_1}\addC{\codeLL^\pid_2})\}^{\pidZ}_{\pid = \pidA}$.
\end{theorem}


We can now state our correctness result showing that if an \piccoC\ program $\stmt$ can be evaluated  to a value $\val$, and the evaluation is well-aligned (it is an evaluation where all the overshooting of arrays are well-aligned), then the \vanillaC\  program $\hat{\stmt}$ obtained by applying the erasure function to $\stmt$, i.e., $\stmt\cong\hat{\stmt}$, can be evaluated to $\hat{\val}$ where $\val\cong\hat{\val}$. This property can be formalized in terms of congruence: 
%
%
\begin{theorem}[Correctness]
\label{Thm: erasure}
For every configuration $\{(\pid,\ \gamma^{{\pid}}_{},$ $\sigma^{{\pid}}_{},$ $\DMap^{{\pid}}_{}$, $\Acc^{{\pid}}_{},$ $\stmt^{\pid})\}^{\pidZ}_{\pid = \pidA}$, 
\\ $\{(\pid,$ $\hgamma^\pid,$ $\hsigma^\pid,$ $\bsq,$ $\bsq,$ $\hstmt^\pid)\}^{\pidZ}_{\pid = \pidA}$ and \LocMap\ $\psi$ 
\\ such that $\{(\pid, \gamma^{{\pid}}_{},$ $\sigma^{{\pid}}_{},$ $\DMap^{{\pid}}_{}$, $\Acc^{{\pid}}_{},$ $\stmt^{{\pid}})$ $\Pcong$ 
$(\pid,$ $\hgamma^\pid,$ $\hsigma^\pid,$ $\bsq,$ $\bsq,$ $\hstmt^\pid)\}^{\pidZ}_{\pid = \pidA}$, 
\\ % a 
if $\Pi \deriv ((\pidA, \gamma^{{\pidA}}_{},$ $\sigma^{{\pidA}}_{},$ $\DMap^{{\pidA}}_{}$, $\Acc^{{\pidA}}_{},$ $\stmt^{{\pidA}})\ \Mid ...\Mid$ 
	$(\pidZ, \gamma^{{\pidZ}}_{},$ $\sigma^{{\pidZ}}_{},$ $\DMap^{{\pidZ}}_{}$, $\Acc^{{\pidZ}}_{},$ $\stmt^{{\pidZ}}))$ 
	\\ \-\ \-\ \-\ $\Deval{\locLL}{\codeLL}$ 
	$((\pidA, {\gamma^{{\pidA}}_{1}},$ ${\sigma^{{\pidA}}_{1}},$ $\DMap^{{\pidA}}_{1}$, $\Acc^{{\pidA}}_{1},$ ${\val^{{\pidA}}_{}}) \Mid ... \Mid$ 
	$(\pidZ, {\gamma^{{\pidZ}}_{1}},$ ${\sigma^{{\pidZ}}_{1}},$ $\DMap^{{\pidZ}}_{1}$, $\Acc^{{\pidZ}}_{1},$ $\val^{{\pidZ}}))$ 
\\ for codes $\codeLL \in {\piccoCodes}$,
then there exists a derivation 
\\ % b
$\Sigma \deriv ((\pidA,$ $\hgamma^\pidA,$ $\hsigma^\pidA,$ $\bsq,$ $\bsq,$ $\hstmt^\pidA)\Mid ...\Mid $
	$(\pidZ,$ $\hgamma^\pidZ,$ $\hsigma^\pidZ,$ $\bsq,$ $\bsq,$ $\hstmt^\pidZ))$ 
	\\ $\Deval{}{\codeVLL}$ 
	$((\pidA,$ $\hgamma^\pidA_1,$ $\hsigma^\pidA_1,$ $\bsq,$ $\bsq,$ $\hval^\pidA)\Mid...\Mid$
	$(\pidZ,$ $\hgamma^\pidZ_1,$ $\hsigma^\pidZ_1,$ $\bsq,$ $\bsq,$ $\hval^\pidZ))$ 
\\ for codes $\codeVLL \in \vanillaCodes$ 
and 
% c
a \LocMap\ $\psi_1$ 
% d
such that 
\\ % f
$\codeLL \cong \codeVLL$, 
% g
$\{(\pid, {\gamma^{{\pid}}_{1}},$ ${\sigma^{{\pid}}_{1}},$ $\DMap^{{\pid}}_{1}$, $\Acc^{{\pid}}_{1},$ $\val^{{\pid}}_{})$ $\cong_{\psi_1}$ 
$(\pid,$ $\hgamma^\pid_1,$ $\hsigma^\pid_1,$ $\bsq,$ $\bsq,$ $\hval^\pid)\}^{\pidZ}_{\pid = \pidA}$, 
% h
and $\Pi \cong_{\psi_1} \Sigma$.
\end{theorem}
The proof 
proceeds by induction on the evaluation tree $\Pi$ and the challenges are the ones 
discussed above related to memory management, and proving that the control flow of the erased program (and the corresponding memory) is correct with respect to the one of the \piccoC\ program.
%
\subsection{Noninterference} \label{sec: noninterference}
\piccoC\ satisfies a strong form of noninterference guaranteeing that two execution traces are indistinguishable up to differences in private values. This stronger version entails data-obliviousness. Instead of using execution traces, we will work directly with evaluation trees in the \piccoC\ semantics -- equivalence of evaluation trees up to private values implies equivalence of execution traces based on the \piccoC\ semantics. This guarantee is provided at the semantics level, we do not consider here compiler optimizations.   

For noninterference, it is convenient to introduce a notion of equivalence requiring that the two memories agree on publicly observable values. Because we assume that private data in memories are encrypted, and so their encrypted value is publicly observable, it is sufficient to consider syntactic equality of memories. Notice that if $\sigma_1=\sigma_2$ we can still have $\sigma_1\ell \neq \sigma_2\ell$, i.e., two executions starting from the same configuration can actually differ with respect to private data. 


We can now state our main noninterference result. 
\begin{theorem}[Noninterference over evaluation trees]
\label{Thm: strong noninterference}
For every environment $\{\gamma^{\pid}_{},$ $\gamma^{\pid}_{1},$ $\gamma'^{\pid}_{1}\}^{\pidZ}_{\pid = \pidA}$; 
memory $\{\sigma^{\pid}_{}$, $\sigma^{\pid}_{1}$, $\sigma'^{\pid}_{1} \}^{\pidZ}_{\pid = \pidA}\in\Mem$; 
\changeMap $\{\DMap^{\pid}_{}$, $\DMap^{\pid}_{1}$, $\DMap'^{\pid}_{1}\}^{\pidZ}_{\pid = \pidA}$;
accumulator $\{\Acc^{\pid}_{}$, $\Acc^{\pid}_{1}$, $\Acc'^{\pid}_{1}\}^{\pidZ}_{\pid = \pidA}\in\N$; 
statement $\stmt$, values $\{\val^{\pid}_{}$, $\val'^{\pid}_{}\}^{\pidZ}_{\pid = \pidA}$; 
step evaluation code lists $\codeLL,\codeLL'$ and their corresponding lists of locations accessed $\locLL,\locLL'$; 
\\
if 
$\Pi \deriv\ ((\pidA, \gamma^{\pidA}_{},$ $\sigma^{\pidA}_{},$ $\DMap^{\pidA}_{}$, $\Acc^{\pidA}_{},$ $\stmt)\ \ \Mid ...\Mid (\pidZ, \gamma^{\pidZ}_{},$ $\sigma^{\pidZ}_{},$ $\DMap^{\pidZ}_{}$, $\Acc^{\pidZ}_{},$ $\stmt))$ 
\\ \-\ \quad $\Deval{\locLL}{\codeLL}$ $((\pidA, \gamma^{\pidA}_{1},$ $\sigma^{\pidA}_{1},$ $\DMap^{\pidA}_{1}$, $\Acc^{\pidA}_{1},$ $\val^{\pidA}_{})\Mid ...\Mid (\pidZ, \gamma^{\pidZ}_{1},$ $\sigma^{\pidZ}_{1},$ $\DMap^{\pidZ}_{1}$, $\Acc^{\pidZ}_{1},$ $\val^{\pidZ}_{}))$ 
\\ and   
$\Sigma \deriv ((\pidA, \gamma^{\pidA}_{},$ $\sigma^{\pidA}_{},\ $ $\DMap^{\pidA}_{},\ $ $\Acc^{\pidA}_{},$ $\stmt)\ \ \Mid ...\Mid (\pidZ, \gamma^{\pidZ}_{},\ $ $\sigma^{\pidZ}_{},\ $ $\DMap^{\pidZ}_{}$, $\Acc^{\pidZ}_{},$ $\stmt))$ 
\\ $\-\ \quad \Deval{\locLL'}{\codeLL'}$ $((\pidA, \gamma'^{\pidA}_{1},$ $\sigma'^{\pidA}_{1},$ $\DMap'^{\pidA}_{1}$, $\Acc'^{\pidA}_{1},$ $\val'^{\pidA}_{})\Mid ...\Mid (\pidZ, \gamma'^{\pidZ}_{1},$ $\sigma'^{\pidZ}_{1},$ $\DMap'^{\pidZ}_{1}$, $\Acc'^{\pidZ}_{1},$ $\val'^{\pidZ}_{}))$
\\ then $\{\gamma^{\pid}_{1}=\gamma'^{\pid}_{1}\}^{\pidZ}_{\pid = \pidA}$, 
$\{\sigma^{\pid}_{1}=\sigma'^{\pid}_{1}\}^{\pidZ}_{\pid = \pidA}$, 
$\{\DMap^{\pid}_{1} =\DMap'^{\pid}_{1}\}^{\pidZ}_{\pid = \pidA}$, 
$\{\Acc^{\pid}_{1}=\Acc'^{\pid}_{1}\}^{\pidZ}_{\pid = \pidA}$, 
$\{\val^{\pid}_{}=\val'^{\pid}_{}\}^{\pidZ}_{\pid = \pidA}$, 
$\codeLL=\codeLL'$, 
$\locLL = \locLL'$, 
$\Pi \loweq \Sigma$.
\end{theorem} 
%
Notice that low-equivalence of evaluation trees already implies the equivalence of the resulting configurations. We repeated them to make the meaning of the theorem clearer. 
This also proves data-obliviousness over memory accesses: in any two executions of the same program with the same public data, which locations in memory are accessed will be based on public data, and therefore identical between the two executions. 
As we proceed to prove Theorem~\ref{Thm: strong noninterference}, we leverage our Axioms reasoning about noninterference of the multiparty protocols.












\section{Implementation}
\label{Sec:implementation}
\section{Implementation: Ring Abstraction}
\label{sec:implement}
\subsection{Distributed \mbox{$G_t$} in QMC Solver}
\label{distributedG4}
Before introducing the communication phase of the ring abstraction layer,
it is important to understand how the authors distributed the large device array $G_t$ across MPI ranks.
%
Original $G_t$ was compared, and $G^d_t$ versions were distributed
(Figure~\ref{fig:compare_original_distributed_g4}). 


In the original $G_t$ implementation, the measurements---which were computed by matrix-matrix multiplication---are distributed statically and independently over the MPI ranks to avoid
inter-node communications. Each MPI rank keeps its partial copy of $G_{t,i}$ to accumulate 
measurements within a rank, where $i$ is the rank index. 
After all the measurements are finished, a reduction step is 
taken to accumulate $G_{t,i}$ across all MPI ranks into a final and complete
$G_t$ in the root MPI rank. The size of the $G_{t,i}$ in each rank is 
the same size as the final and complete $G_t$. 

With the distributed $G^d_t$ implementation, this large device array 
$G_t$ was evenly partitioned across all MPI ranks; each portion of it is local to each MPI rank.
%
Instead of keeping its partial copy of $G_t$, 
each rank now keeps an instance of $G^d_{t,i}$ to accumulate measurements 
of a portion or sub-slice of the final and complete $G_t$, where the notation
$d$ in $G^d_t$  refers to the distributed version, and $i$ means the $i$-th rank.
%
The $G^d_{t,i}$ size in each rank is 
reduced to $1/p$ of the size of the final and complete $G_t$, comparing the same configuration 
in original $G_t$ implementation, where $p$ is the number of MPI ranks used. 
%
For example, in Figure~\ref{fig:distributed_g4}, there are four ranks, and rank $i$
now only keeps $G^d_{t,i}$, which is one-fourth the size of the original $G_t$ array size.
% and contains values indexing from range of $[0, ..., N/4)$ in original $G_t$ array where $N$ is the 
% number of entries in  $G_t$  when viewed as a one-dimensional array.

To compute the final and complete $G^d_{t,i}$ for the distributed $G^d_t$ implementation, 
each rank must see every $G_{\sigma,i}$ from all ranks. 
%
In other words, each rank must broadcast the
locally generated $G_{\sigma,i}$ to the remainder of the other ranks at every measurement step. 
%
To efficiently perform this ``all-to-all'' broadcast, a ring abstraction layer was built (Section. \ref{section:ring_algorithm}), which circulates
all $G_{\sigma,i}$ across all ranks.

% over high-speed GPUs interconnect (GPUDirect RDMA) to facilitate the communication phase.

% \begin{figure}
% \centering
% \subfloat[Original $G_t$ implementation.]
%     {\includegraphics[width=\columnwidth]{original_g4.pdf}}\label{fig:original_g4}

% \subfloat[Distributed $G_t$ implementation.]
%     {\includegraphics[width=0.99\columnwidth]{distributed_g4.pdf} \label{fig:distributed_g4}}

% \caption{Comparison of the original $G_t$ vs. the distributed $G^d_t$ implementation. Each rank contains one GPU resource.}
% \label{fig:compare_original_distributed_g4} 
% \end{figure} 

\begin{figure}
\centering
     \begin{subfigure}[b]{\columnwidth}
         \centering
         \includegraphics[width=\textwidth]{images/original_g4.pdf}
         \caption{Original $G_t$ implementation.}
         \label{fig:original_g4}
     \end{subfigure}
     
    \begin{subfigure}[b]{\columnwidth}
         \centering
         \includegraphics[width=\textwidth]{images/distributed_g4.pdf}
         \caption{Distributed $G_t$ implementation.}
         \label{fig:distributed_g4}
     \end{subfigure}
     
\caption{Comparison of the original $G_t$ vs. the distributed $G^d_t$ implementation. Each rank contains one GPU resource.}
\label{fig:compare_original_distributed_g4}
\end{figure}

\subsection{Pipeline Ring Algorithm}
\label{section:ring_algorithm}
A pipeline ring algorithm was implemented that broadcasts the $G_{\sigma}$ 
array circularly during every measurement. 
%
The algorithm (Algorithm \ref{alg:ring_algorithm_code}) is 
visualized in Figure~\ref{fig:ring_algorithm_figure}.

\begin{algorithm}
\SetAlgoLined
    generateGSigma(gSigmaBuf)\; \label{lst:line:generateG2}
    updateG4(gSigmaBuf)\;       \label{lst:line:updateG4}
    %\texttt{\\}
    {$i\leftarrow 0$}\;         \label{lst:line:initStart}
    {$myRank \leftarrow worldRank$}\;          \label{lst:line:initRankId}
    {$ringSize \leftarrow mpiWorldSize$}\;      \label{lst:line:initRingSize}
    {$leftRank \leftarrow (myRank - 1 + ringSize) \: \% \: ringSize $}\;
    {$rightRank \leftarrow (myRank + 1 + ringSize) \: \% \: ringSize $}\;
    sendBuf.swap(gSigmaBuf)\;           \label{lst:line:initEnd}
    \While{$i< ringSize$}{
        MPI\_Irecv(recvBuf, source=leftRank, tag = recvTag, recvRequest)\; \label{lst:line:Irecv}
        MPI\_Isend(sendBuf, source=rightRank, tag = sendTag, sendRequest)\; \label{lst:line:Isend}
        MPI\_Wait(recvRequest)\;        \label{lst:line:recevBuffWait}
        
        updateG4(recvBuf)\;             \label{lst:line:updateG4_loop}
        
        MPI\_Wait(sendRequest)\;        \label{lst:line:sendBuffWait}
        
        sendBuf.swap(recvBuf)\;         \label{lst:line:swapBuff}
        i++\;
    }
\caption{Pipeline ring algorithm}
\label{alg:ring_algorithm_code}
\end{algorithm}

\begin{figure}
	\centering
	\includegraphics[width=\columnwidth, trim=0 5cm 0 0, clip]{images/ring_algorithm.pdf}
	\caption{Workflow of ring algorithm per iteration. }
	\label{fig:ring_algorithm_figure}
\end{figure}

At the start of every new measurement, a single-particle Green's function $G_{\sigma}$
 (Line~\ref{lst:line:generateG2}) is generated 
and then used to update $G^d_{t,i}$ (Line~\ref{lst:line:updateG4})
via the formula in Eq.~(\ref{eq:G4}).
%
% Different from original method that performs 
% full matrix-matrix multiplication (Equation~(\ref{eq:G4})), the current ring algorithm only performs partial
% matrix-matrix multiplication that contributes to $G^d_{t,i}$, a subslice of the final and complete $G_t$.
%
Between Lines \ref{lst:line:initStart} to \ref{lst:line:initEnd}, the algorithm 
initializes the indices
of left and right neighbors and prepares the sending message buffer from the
previously generated $G_{\sigma}$ buffer. 
%
The processes are organized as a ring so that the first and last rank are considered to be neighbors to each other. 
%
A \textit{swap} operation is used to avoid unnecessary memory copies for \textit{sendBuf} preparation.
%
A walker-accumulator thread allocates an additional \textit{recvBuf} buffer of the same size 
as \textit{gSigmaBuf} to hold incoming 
\textit{gSigmaBuf} buffer from \textit{leftRank}. 

The \textit{while} loop is the core part of the pipeline ring algorithm. 
%
For every iteration, each thread in a rank 
receives a $G_{\sigma}$ buffer from its left neighbor rank and sends a $G_{\sigma}$ buffer to its right neighbor rank. 
A synchronization step (Line~\ref{lst:line:recevBuffWait}) is performed
afterward to ensure that each rank receives a new buffer to update the 
local $G^d_{t,i}$ (Line~\ref{lst:line:updateG4_loop}). 
%
Another synchronization step
follows to ensure that all send requests are finalized 
(Line~\ref{lst:line:sendBuffWait}). Lastly, another \textit{swap} operation is used to exchange
content pointers between \textit{sendBuf} and \textit{recvBuf} to avoid unnecessary memory copy and prepare
for the next iteration of communication.
%
In the multi-threaded version (Section~\ref{subsec:multi-thread}), the thread of index, \textit{i}, only communicates with
	threads of index, \textit{i}, in neighbor ranks, and each thread allocates two buffers: \textit{sendBuff} and \textit{recvBuff}.

The \textit{while} loop will be terminated after $\mbox{\textit{ringSize}} - 1$ steps. By that time, 
each locally generated $G_{\sigma,i}$ will have traveled across all MPI ranks and
updated $G^d_{t,i}$ in all ranks. Eventually, each $G_{\sigma,i}$ reaches
to the left neighbor of its birth rank. For example, $G_{\sigma,0}$ generated from rank $0$ will end 
in last rank in the ring communicator.

Additionally, if the $G_t$ is too large to be stored in one node, 
it is optional to accumulate all $G^d_{t,i}$
at the end of all measurements. 
%
Instead, a parallel write into the file system could be taken.

\subsubsection{Sub-Ring Optimization.}

A sub-ring optimization strategy is further proposed to reduce message communication
times if the large device array $G_t$ can fit in fewer devices. 
%
The sub-ring algorithm is visualized in Figure~\ref{fig:subring_algorithm_figure}.

For the ring algorithm (Section~\ref{section:ring_algorithm}), the size of the ring communicator
(\textit{mpiWorldSize}) is set to the same size of the global \mbox{\texttt{MPI\_COMM\_WORLD}}, and thus the size of $G_t$ is equally 
distributed across all MPI ranks.

However, to complete the update to $G^d_{t,i}$ in one measurement, 
one $G_{\sigma,i}$
must travel \textit{mpiWorldSize} ranks. In total, 
there are \textit{mpiWorldSize} numbers of $G_{\sigma,i}$
being sent and received concurrently in one measurement 
in the global
\mbox{\texttt{MPI\_COMM\_WORLD}} 
communicator. If the size of $G^d_{t,i}$ is relatively small per rank, then this will cause high communication overhead.

If $G_t$ can be distributed and fitted in fewer devices, then a shorter travel distance is required 
for $G_{\sigma,i}$, thus reducing the communication overhead. One reduction
step was performed at the end of all measurements to accumulate $G^d_{t,s_i}$, 
where $s_i$ means $i$-th rank on the $s$-th sub-ring.

At the beginning of MPI initialization, the global \mbox{\texttt{MPI\_COMM\_WORLD}} was partitioned  into several new sub-ring communicators by using \mbox{\texttt{MPI\_Comm\_split}}. 
% where each new communicator represents conceptually a subring. 
The new
communicator information was passed to the DCA++ concurrency class by substituting the original global 
\mbox{\texttt{MPI\_COMM\_WORLD}} with this new communicator. Now, only a few minor modifications
are needed to transform the ring algorithm (Algorithm~\ref{alg:ring_algorithm_code})
to sub-ring Algorithm~\ref{alg:sub_ring_algorithm}. In Line~\ref{lst:line:initRankId}, \textit{myRank} is 
initialized to \textit{subRingRank} instead of \textit{worldRank}, where 
\textit{subRingRank} is the rank index in the local sub-ring communicator. 
%
In Line~\ref{lst:line:initRingSize}, \textit{ringSize} is initialized to \textit{subRingSize}
instead of \textit{mpiWorldSize}, where \textit{subRingSize} is the
size of the new communicator.
%
The general ring algorithm is a special case for the sub-ring algorithm because the
\textit{subRingSize} of the general ring algorithm is equal to \textit{mpiWorldSize}, and
there is only one sub-ring group throughout all MPI ranks.


\LinesNumberedHidden
\begin{algorithm}
    {$\mbox{\textit{myRank}} \leftarrow \mbox{\textit{subRingRank}}$}\;         
    {$\mbox{\textit{ringSize}} \leftarrow \mbox{\textit{subRingSize}}$}\;      
\caption{Modified ring algorithm to support sub-ring communication}
\label{alg:sub_ring_algorithm}
\end{algorithm}


\begin{figure}
	\centering
	\includegraphics[width=\columnwidth, trim=0 5cm 0 0, clip]{images/subring_alg.pdf}
	\caption{Workflow of sub-ring algorithm per iteration. Every consecutive $S$ rank forms a sub-ring communicator, 
	and no communication occurs between sub-ring communicators until all measurements are finished. Here, $S$ is the number of ranks in a sub-ring.}
	\label{fig:subring_algorithm_figure}
\end{figure}

\subsubsection{Multi-Threaded Ring Communication.}
\label{subsec:multi-thread}
To take advantage of the multi-threaded QMC model already in DCA++, 
multi-threaded ring communication support was further implemented in the ring algorithm.
%
Figure~\ref{fig:dca_overview} shows that in the original DCA++ method,
each walker-accumulator
thread in a rank is independent of each other, and all the threads in a 
rank synchronize only after all rank-local measurements are finished.
%
Moreover, during every measurement, each walker-accumulator thread
generates its own thread-private $G_{\sigma, i}$ to update $G_t$. 
%

The multi-threaded ring algorithm now allows concurrent message exchange so that threads of same rank-local thread index exchange their thread-private $G_{\sigma, i}$. 
%
Conceptually, there are $k$ parallel and independent rings, where $k$ 
is number of threads per rank, because threads of the same local thread ID
form a closed ring. 
%
For example, a thread of index $0$ in rank $0$ will send its $G_\sigma$ to 
the thread of index $0$ in rank $1$ and receive another $G_\sigma$ from thread index of $0$ 
from last rank in the ring algorithm.
%

The only changes in the ring algorithm are offsetting the tag values 
(\texttt{recvTag} and \texttt{sendTag}) by the thread index value. For example,
Lines~\ref{lst:line:Irecv} and ~\ref{lst:line:Isend} from 
Algorithm~\ref{alg:ring_algorithm_code} are modified to Algorithm~\ref{alg:multi_threaded_ring}.

\LinesNumberedHidden
\begin{algorithm}
        MPI\_Irecv(recvBuf, source=leftRank, tag = recvTag + threadId, recvRequest)\; 
        MPI\_Isend(sendBuf, source=rightRank, tag = sendTag + threadId, sendRequest)\;
\caption{Modified ring algorithm to support multi-threaded ring}
\label{alg:multi_threaded_ring}
\end{algorithm}

To efficiently send and receive $G_\sigma$, each thread
will allocate one additional \textit{recvBuff} to hold incoming 
\textit{gSigmaBuf} buffer from \textit{leftRank} and perform send/receive efficiently.
%
In the original DCA++ method, there are $k$ numbers of buffers of $G_\sigma$ 
size per rank, and in the multi-threaded ring method, there are $2k$
numbers of buffers of $G_\sigma$ size per rank, where $k$ is number of 
threads per rank.


\section{Evaluation}
\label{Sec:evaluation}
\section{Evaluation}
\label{sec:evaluation}
\begin{table*}[!t]
\begin{center}
%\small
\caption {Benchmarks and applications for the study of the application-level resilience}
\vspace{-5pt}
\label{tab:benchmark}
\tiny
\begin{tabular}{|p{1.7cm}|p{7.5cm}|p{4cm}|p{2.5cm}|}
\hline
\textbf{Name} 	& \textbf{Benchmark description} 		& \textbf{Execution phase for evaluation}  			& \textbf{Target data objects}             \\ \hline \hline
CG (NPB)             & Conjugate Gradient, irregular memory access (input class S)   & The routine conj\_grad in the main computation loop  & The arrays $r$ and $colidx$     \\\hline
MG (NPB)    	       & Multi-Grid on a sequence of meshes (input class S)             & The routine mg3P in the main computation loop & The arrays $u$ and $r$ 	\\ \hline
FT (NPB)             & Discrete 3D fast Fourier Transform (input class S)            & The routine fftXYZ in the main computation loop  & The arrays $plane$ and $exp1$    \\ \hline
BT (NPB)             & Block Tri-diagonal solver (input class S)         		& The routine x\_solve in the main computation loop & The arrays $grid\_points$ and $u$	\\ \hline
SP (NPB)             & Scalar Penta-diagonal solver (input class S)         		& The routine x\_solve in the main computation loop & The arrays $rhoi$ and $grid\_points$  \\ \hline
LU (NPB)            & Lower-Upper Gauss-Seidel solver (input class S)        	& The routine ssor 	& The arrays $u$ and $rsd$ \\ \hline \hline
LULESH~\cite{IPDPS13:LULESH} & Unstructured Lagrangian explicit shock hydrodynamics (input 5x5x5) & 
The routine CalcMonotonicQRegionForElems 
& The arrays $m\_elemBC$ and $m\_delv\_zeta$ \\ \hline
AMG2013~\cite{anm02:amg} & An algebraic multigrid solver for linear systems arising from problems on unstructured grids (we use  GMRES(10) with AMG preconditioner). We use a compact version from LLNL with input matrix $aniso$. & The routine hypre\_GMRESSolve & The arrays $ipiv$ and $A$   \\ \hline
%$hierarchy.levels[0].R.V$ \\ \hline
\end{tabular}
\end{center}
\vspace{-5pt}
\end{table*}

%We evaluate the effectiveness of ARAT, and 
%We use ARAT to study the application-level resilience.
%The goal is to demonstrate 
%that aDVF can be a very useful metric to quantify the resilience of data objects
%at the application level. 
We study 12 data objects from six benchmarks of the NAS parallel benchmark (NPB) suite (we use SNU\_NPB-1.0.3) and 4 data objects from two scientific applications. 
%which is a c version of NPB 3.3, but ARAT can work for Fortran.
Those data objects are chosen to be representative: they have various data access patterns and participate in various execution phases.  
%For the benchmarks, we use CLASS S as the input problems and use the default compiler options of NPB.
For those benchmarks and applications, we use their default compiler options, and use gcc 4.7.3 and LLVM 3.4.2 for trace generation.
To count the algorithm-level fault masking, we use the default convergence thresholds (or the fault tolerance levels) for those benchmarks.
Table~\ref{tab:benchmark} gives 
%for->on by anzheng
detailed information on the benchmarks and applications.
The maximum fault propagation path for aDVF analysis is set to 10 by default.
%the value shadowing threshold is set as 0.01 (except for BT, we use $1 \times 10^{-6}$).
%These value shadowing thresholds are chosen such that any error corruption
%that results in the operand's value variance less than 1\% (for the threshold 0.01) or 0.0001\% (for the threshold $1 \times 10^{-6}$) during the 
%trace analysis does not impact the outcome correctness of six benchmarks.
%LU: check the newton-iteration residuals against the tolerance levels
%SP: check the newton-iteration residuals against the tolerance levels
%BT: check the newton-iteration residuals against the tolerance levels

\subsection{Resilience Modeling Results}
%We use ARAT to calculate aDVF values of 16 data objects. 
Figure~\ref{fig:aDVF_3tiers_profiling}
shows the aDVF results and breaks them down into the three levels 
(i.e., the operation-level, fault propagation level, and algorithm-level).
Figure~\ref{fig:aDVF_3classes_profiling} shows the 
%for->of by anzheng
results for the analyses at the levels of the operation and fault propagation,
and further breaks down the results into 
the three classes (i.e., the value overwriting, logical and comparison operations,
and value shadowing). %based on the reasons of the fault masking.
We have multiple interesting findings from the results.

\begin{figure*}
	\centering
        \includegraphics[width=0.8\textwidth]{three_tiers_gray.pdf}
% * <azguolu@gmail.com> 2017-03-23T03:20:28.808Z:
%
% ^.
        \vspace{-5pt}
        \caption{The breakdown of aDVF results based on the three level analysis. The $x$ axis is the data object name.}
        \vspace{-8pt}
        \label{fig:aDVF_3tiers_profiling}
\end{figure*}


\begin{figure*}
	\centering
	\includegraphics[width=0.8\textwidth]{three_types_gray.pdf}
	\vspace{-5pt}
	\caption{The breakdown of aDVF results based on the three classes of fault masking. The $x$ axis is the data object name. \textit{zeta} and \textit{elemBC} in LULESH are \textit{m\_delv\_zeta} and \textit{m\_elemBC} respectively.} % Anzheng
	\vspace{-5pt}
	\label{fig:aDVF_3classes_profiling}
    %\vspace{-5pt}
\end{figure*}

(1) Fault masking is common across benchmarks and applications.
Several data objects (e.g., $r$ in CG, and $exp1$ and $plane$ in FT)
have aDVF values close to 1 in Figure~\ref{fig:aDVF_3tiers_profiling}, 
which indicates that most of operations working on these data objects
have fault masking.
However, a couple of data objects have much less intensive fault masking.
For example, the aDVF value of $colidx$ in CG is 0.28 (Figure~\ref{fig:aDVF_3tiers_profiling}). 
Further study reveals that $colidx$ is an array to store column indexes of sparse matrices, and there is few operation-level or fault propagation-level fault masking  (Figure~\ref{fig:aDVF_3classes_profiling}).
The corruption of it can easily cause segmentation fault caught by the
algorithm-level analysis. 
$grid\_points$ in SP and BT also have a relatively small aDVF value (0.14 and 0.38 for SP and BT respectively in Figure~\ref{fig:aDVF_3tiers_profiling}).
Further study reveals that $grid\_points$ defines input problems for SP and BT. 
A small corruption of $grid\_points$ 
%change->changes by anzheng
can easily cause major changes in computation
caught by the fault propagation analysis. 

The data object $u$ in BT also has a relatively small aDVF value (0.82 in Figure~\ref{fig:aDVF_3tiers_profiling}).
Further study reveals that $u$ is read-only in our target code region
for matrix factorization and Jacobian, neither of which is friendly
for fault masking.
Furthermore, the major fault masking for $u$ comes from value shadowing,
and value shadowing only happens in a couple of the least significant bits 
of the operands that reference $u$, which further reduces the value of aDVF.
%also reduces fault masking.

(2) The data type is strongly correlated with fault masking.
Figure~\ref{fig:aDVF_3tiers_profiling} reveals that the integer data objects ($colidx$ in CG, $grid\_points$ in BT and SP, $m\_elemBC$ in LULESH) appear to be 
more sensitive to faults than the floating point data objects 
($u$ and $r$ in MG, $exp1$ and $plane$ in FT, $u$ and $rsd$ in LU, $m\_delv\_zeta$ in LULESH, and $rhoi$ in SP).
In HPC applications, the integer data objects are commonly employed to
define input problems and bound computation boundaries (e.g., $colidx$ in CG and $grid\_points$ in BT), 
or track computation status (e.g., $m\_elemBC$ in LULESH). Their corruption 
%these integer data objects
is very detrimental to the application correctness. 

(3) Operation-level fault masking is very common.
For many data objects, the operation-level fault masking contributes 
more than 70\% of the aDVF values. For $r$ in CG, $exp1$ in FT, and $rhoi$ in SP,
the contribution of the operation-level fault masking is close to 99\% (Figure~\ref{fig:aDVF_3tiers_profiling}).

Furthermore, the value shadowing is a very common operation level fault masking,
especially for floating point data objects (e.g., $u$ and $r$ in BT, $m\_delv\_zeta$ in LULESH, and $rhoi$ in SP in Figure~\ref{fig:aDVF_3classes_profiling}).
This finding has a very important indication for studying the application resilience.
In particular, the values of a data object can be different across different input problems. If the values of the data object are different, 
then the number of fault masking events due to the value shadowing will be different. 
Hence, we deduce that the application resilience
can be correlated with the input problems,
because of the correlation between the value shadowing and input problems. 
We must consider the input problems when studying the application resilience.
This conclusion is consistent with a very recent work~\cite{sc16:guo}.

(4) The contribution of the algorithm-level fault masking to the application resilience can be nontrivial.
For example, the algorithm-level fault masking contributes 19\% of the aDVF value for $u$ in MG and 27\% for $plane$ in FT (Figure~\ref{fig:aDVF_3tiers_profiling}).
The large contribution of algorithm-level fault masking in MG is consistent with
the results of existing work~\cite{mg_ics12}. 
For FT (particularly 3D FFT), the large contribution of algorithm-level fault masking in $plane$ (Figure~\ref{fig:aDVF_3tiers_profiling})
comes from frequent transpose and 1D FFT computations that average out 
or overwrite the data corruption.
CG, as an iterative solver, is known to have the algorithm-level fault masking
because of the iterative nature~\cite{2-shantharam2011characterizing}.
Interestingly, the algorithm-level fault masking in CG contributes most to the resilience of $colidx$ which is a vulnerable integer data object (Figure~\ref{fig:aDVF_3tiers_profiling}).

%Our study reveals the algorithm-level fault masking of CG from
%two perspectives. First, $a$ in CG, which is an array for intermediate results,
%has few algorithm-level fault masking (0.008\%);
%Second, $x$ in CG, which is a result vector, has 5.4\% of the aDVF value coming from the algorithm-level fault masking.
%This result indicates that the effects of the algorithm-level fault masking
%are not uniform across data objects. 

(5) Fault masking at the fault propagation level is small.
For all data objects, the contribution of the fault masking at the level of fault propagation is less than 5\% (Figure~\ref{fig:aDVF_3tiers_profiling}).
For 6 data objects ($r$ and $colidx$ in CG, $grid\_points$ and $u$ in BT, and 
$grid\_points$ and $rhoi$ in SP),  there is no fault masking at the level of fault propagation.
In combination with the finding 4, we conclude that once the fault
is propagated, it is difficult to mask it because of the contamination of
more data objects after fault propagation, and only the algorithm semantics can tolerate  propagated faults well. 
%This finding is consistent with our sensitivity analysis. 

(6) Fault masking by logical and comparison operations is small,
%For all data objects, the fault masking contributions due to logical and comparison operations are very small, 
comparing with the contributions of value shadowing and overwriting (Figure~\ref{fig:aDVF_3classes_profiling}). 
Among all data objects, 
the logical and comparison operations in $grid\_points$ in BT contribute the most (25\% contribution in Figure~\ref{fig:aDVF_fine_profiling}), 
because of intensive ICmp operations (integer comparison). %logical OR and SHL (left shifting).


(7) The resilience varies across data objects. %within the same application.
This fact is especially pronounced in two data objects $colidx$ and $r$ in CG (Figure~\ref{fig:aDVF_3tiers_profiling}).
 $colidx$ has aDVF much smaller than $r$, which means $colidx$ is much less resilient than $r$ (see finding 1 for a detailed analysis on $colidx$). 
Furthermore, $colidx$ and $r$ have different algorithm-level
fault masking (see finding 4 for a detailed analysis).

\begin{comment}
\textbf{Finding 7: The resilience of the same data objects varies across different applications.}
This fact is especially pronounced in BT and SP.
BT and SP address the same numerical problem but with different algorithms.
BT and SP have the same data objects, $qs$ and $rhoi$, but
$qs$ manifests different resilience in BT and SP.
This result is interesting, because it indicates that by using
different algorithms, we have opportunities to
improve the resilience of data objects.
\end{comment}

To further investigate the reasons for fault masking, 
we break down the aDVF results at the granularity of LLVM instructions,
based on the analyses at the levels of operation and fault propagation.
The results are shown in Figure~\ref{fig:aDVF_fine_profiling}.
%Because of the space limitation, 
%we only show one data object per benchmark, but each selected data object has the most diverse fault masking events within the corresponding benchmark.
%Based on Figure~\ref{fig:aDVF_fine_profiling}, we have another interesting finding.

(8) Arithmetic operations make a lot of contributions to fault masking.
%For $r$ in CG, $r$ in MG, $exp1$ in FT, $u$ in BT, $qs$ in SP, and $u$ in LU,
%the arithmetic operations, FMul (100\%), Add (16\%), FMul (85\%), 
%FMul (94\%), FMul (28\%), and FAdd (50\%)
For $r$ in CG, $u$ in BT, $plane$ and $exp1$ in FT, $m\_elemBC$ in LULESH, 
arithmetic operations (addition, multiplication, and division) contribute to almost 100\% of the fault masking (Figure~\ref{fig:aDVF_fine_profiling}).  
%(at the operation level and the fault propagation level).
%For $qs$ in SP and $u$ in LU, the store operation also makes
%important contributions as the arithmetic operations because of value overwriting.

\begin{figure*}
	\centering
	\includegraphics[width=0.77\textheight, height=0.23\textheight]{pie_chart.pdf}
	\vspace{-10pt}
	\caption{Breakdown of the aDVF results based on the analyses at the levels of operation and fault propagation}
    \vspace{-10pt}
	\label{fig:aDVF_fine_profiling}
\end{figure*}


\subsection{Sensitivity Study}
\label{sec:eval_sen}
%\textbf{change the fault propagation threshold and study the sensitivity of analysis to the threshold}
ARAT uses 10 as the default fault propagation analysis threshold. 
The fault propagation analysis will not go beyond 10 operations. Instead,
we will use deterministic fault injection after 10 operations. 
In this section, we study the impact of this threshold on the modeling accuracy. We use a range of threshold values and examine how the aDVF value varies and whether
the identification of fault masking varies. 
Figure~\ref{fig:sensitivity_error_propagation} shows the results for 
%add , after BT by anzheng
multiple data objects in CG, BT, and SP.
We perform the sensitivity study for all 16 data objects.
%in six benchmarks and two applications.
Due to the page space limitation, we only show the results for three data objects,
but we summarize the sensitivity study results for all data objects in this section.
%but other data objects in all benchmarks have the same trend.

Our results reveal that the identification of fault masking by tracking fault propagation is not significantly 
affected by the fault propagation analysis threshold. Even if we use a rather large threshold (50), 
the variation of aDVF values is 4.48\% on average among all data objects,
and the variation at each of the three levels of analysis (the operation level, fault propagation level,  and algorithm level) is less than 5.2\% on average. 
In fact, using a threshold value of 5 is sufficiently accurate in most of the cases (14 out of 16 data objects).
This result is consistent with our finding 5 (i.e., fault masking at the fault propagation level is small). %in most benchmarks).
However, we do find a data object ($m\_elementBC$ in LULESH) %and $exp1$ in FT) 
showing relatively high-sensitive (up to 15\% variation) to the threshold. For this uncommon data object, using 50 as the fault propagation path is sufficient. 

%In other words, even though using a larger threshold value can identify more error masking by tracking error 
%propagation, the implicit error masking induced by the error propagation is very limited.

\begin{figure}
		\begin{center}
		\includegraphics[width=0.48\textwidth,height=0.11\textheight]{sensi_study_gray.pdf}
		\vspace{-15pt}
		\caption{Sensitivity study for fault propagation threshold}
		\label{fig:sensitivity_error_propagation}
		\end{center}
\vspace{-15pt}
\end{figure}


\begin{comment}
\subsection{Comparison with the Traditional Random Fault Injection}
%\textbf{compare with the traditional fault injection to verify accuracy}
To show the effectiveness of our resilience modeling, we compare traditional random fault injection
and our analytical modeling. Figure~\ref{fig:comparison_fi} and Table~\ref{tab:comparison} show the results.
The figure shows the success rate of all random fault injection. The ``success'' means the application
outcome is verified successfully by the benchmarks and the execution does not have any segfault. The success rate is used as a metric
to evaluate the application resilience.

We use a data-oriented approach to perform random fault injection.
In particular, given a data object, for each fault injection test we trigger a bit flip
in an operand of a random instruction, and this operand must be a reference to the
target data object. We develop a tool based on PIN~\cite{pintool} to implement the above fault injection functionality.
For each data object, we conduct five sets of random fault injection tests, 
and each set has 200 tests (in total 1000 tests per data object). 
We show the results for CG and FT in this section, but we find that
the conclusions we draw from CG and FT are also valid for the other four benchmarks.


%\begin{table*}
%\label{tab:success_rate}
%\begin{centering}
%\renewcommand\arraystretch{1.1}
%\begin{tabular}{|c|c|c|c|c|c|c|}
%\hline 
%Success Rate (Difference) & Test set 1 & Test set 2 & Test set 3 & Test set 4 & Test set 5 & Average\tabularnewline
%\hline 
%\hline 
%CG-a & 66.1\% (11.7\%) & 68.5\% (15.7\%) & 56.7\% (4.21\%) & 61.3\% (3.57\%) & 43.3\% (26.8\%) & 59.2\%\tabularnewline
%\hline 
%CG-x & 99.2\% (2.2\%) & 98.6\% (1.5\%) & 96.5\% (0.63\%) & 97.8\% (0.64\%) & 93.6\% (3.7\%) & 97.1\%\tabularnewline
%\hline 
%CG-colidx & 36.8\% (12.7\%) & 49.6\% (17.8\%) & 40.2\% (4.6\%) & 52.6\% (24.9\%) & 31.4\% (25.4\%) & 42.1\%\tabularnewline
%\hline 
%FT-exp1 & 52.7\% (1.4\%) & 22.6\% (56.5\%) & 78.5\% (51.0\%) & 60.7\% (16.7\%) & 45.4\% (12.7\%) & 51.9\%\tabularnewline
%\hline 
%FT-plane & 82.1\% (2.5\%) & 79.3\% (5.6\%) & 99.5\% (18.2\%) & 93.2\% (10.7\%) & 66.8\% (20.6\%) & 84.2\%\tabularnewline
%\hline 
%\end{tabular}
%\par\end{centering}
%\caption{XXXXX}
%\end{table*}


\begin{table*}
\begin{centering}
\caption{\small The results for random fault injection. The numbers in parentheses for each set of tests (200 tests per set) are the success rate difference from the average success rate of 1000 fault injection tests.}
\label{tab:comparison}
\renewcommand\arraystretch{1.1}
\begin{tabular}{|c|p{2.2cm}|p{2.2cm}|p{2.2cm}|p{2.2cm}|p{2.2cm}|p{1.8cm}|}
\hline 
       %& Test set 1 & Test set 2 & Test set 3 & Test set 4 & Test set 5 & Average\tabularnewline
       & \hspace{13pt} Test set 1 \hspace{1pt}/  & \hspace{13pt} Test set 2 \hspace{1pt}/ & \hspace{13pt} Test set 3 \hspace{1pt}/ & \hspace{13pt} Test set 4 \hspace{1pt}/ & \hspace{13pt} Test set 5 \hspace{1pt}/ & Ave. of all test / \\
       & success rate (diff.) & success rate (diff.) & success rate (diff.) & success rate (diff.) & success rate (diff.) & \hspace{5pt} success rate \\
\hline 
\hline 
CG-a & 66.1\% (6.9\%) & 68.5\% (9.3\%) & 56.7\% (-2.5\%) & 61.3\% (2.1\%) & 43.3\% (-15.9\%) & 59.2\%\tabularnewline
\hline 
CG-x & 99.2\% (2.1\%) & 98.6\% (1.5\%) & 96.5\% (-0.6\%) & 97.8\% (0.7\%) & 93.6\% (-3.5\%) & 97.1\%\tabularnewline
\hline 
CG-colidx & 36.8\% (-5.3\%) & 49.6\% (7.5\%) & 40.2\% (-2.0\%) & 52.6\% (10.5\%) & 31.4\% (-10.7\%) & 42.1\%\tabularnewline
\hline 
FT-exp1 & 52.7\% (0.8\%) & 22.6\% (-29.3\%) & 78.5\% (26.6\%) & 60.7\% (8.8\%) & 45.4\% (-6.5\%) & 51.9\%\tabularnewline
\hline 
FT-plane & 82.1\% (-2.1\%) & 79.3\% (-4.9\%) & 99.5\% (15.3\%) & 93.2\% (9.0\%) & 66.8\% (-17.4\%) & 84.2\%\tabularnewline
\hline 
\end{tabular}
\par\end{centering}
\vspace{-0.4cm}
\end{table*}

\begin{figure}
	\begin{center}
		\includegraphics[width=0.48\textwidth,keepaspectratio]{verifi-study.png}
		\caption{The traditional random fault injection vs. ARAT}
		\label{fig:comparison_fi}
	\end{center}
\vspace{-0.7cm}
\end{figure}


We first notice from Table~\ref{tab:comparison} that 
%across 5 sets of random fault injection tests, there are big variances (up to 55.9\% in $exp1$ of FT) in terms of the success rate. 
the results of 5 test sets can be quite different from each other and from 1000 random fault inject tests (up to 29.3\%).
1000 fault injection tests provide better statistical significance than 200 fault injection tests.
We expect 1000 fault injection tests potentially provide higher accuracy to quantify the application resilience.
The above result difference is clearly an indication to the randomness of fault injection, and there
is no guarantee on the random fault injection accuracy.

%In Figure~\ref{fig:comparison_fi}, 
We compare the success rate of 1000 fault inject tests with the aDVF value (Fig.~\ref{fig:comparison_fi}). 
We find that the order of the success rate of the three data objects in CG (i.e., $colidx < a < x$) and the two data objects in FT 
(i.e., $exp1 < plane$) is the same as the order of the aDVF values of these data objects. 
%In fact, 1000 fault injection tests
%account for \textcolor{blue}{\textbf{xxx\%}} of total memory references to the data object,
%and provide better resilience quantification than 200 fault injection tests.
The same order (or the same resilience trend)
%between our approach and the random fault injection based on a large number of tests 
is a demonstration of the effectiveness of our approach.
Note that the values of the aDVF and success rate %for a data object
cannot be exactly the same (even if we have sufficiently large numbers of random fault injection), 
because aDVF and random fault injection quantify
the resilience based on different metrics.
Also, the random fault injection can miss some fault masking events that can be captured by our approach.

\end{comment}

\section{Conclusions}
\label{sec:conclusions}

\begin{figure*}  \footnotesize
\begin{lstlisting}[firstnumber = 22]
avgFemaleSalPub=smcopen(avgFemaleSalary); 
femaleCountPub=smcopen(femaleCount);
avgMaleSalPub=smcopen(avgMaleSalary); maleCountPub=smcopen(maleCount);
avgFemaleSalPub=(avgFemaleSalPub/femaleCountPub)/2+historicFemaleSalAvg/2; 
avgMaleSalPub=(avgMaleSalPub/maleCountPub)/2+historicMaleSalAvg/2; 

for (i=1; i<numParticipants+1; i++) 
	smcoutput(avgFemaleSalPub, i);  smcoutput(avgMaleSalPub, i); 
\end{lstlisting}
\caption{Securely calculating the gender pay gap for 100 organizations with additional information released.}
\label{Fig: salary vs gender smcopen}
\end{figure*}

In this paper we have presented a formal model for a general SMC compiler, supporting both safe and unsafe features of C.  
Our model does not artificially restrict what C features can be present in private branches -- restrictions are instead guided by which operations
our model has shown to be unsafe. 
Our extension supports additional tracking meta-data to provide support for features unsafe in
current SMC techniques.  The intuition, shown in our motivation, is that state-of-the-art SMC techniques cannot track complex memory indirections that can occur when using pointers.  By providing this tracking, these operations can be made safe.
As future work we plan on extending our model to support explicit declassification, through a primitive PICCO calls \texttt{smcopen}.   Consider Figure~\ref{Fig: salary vs gender smcopen} which highlights a modification to our original
gender based salary computation (lines 16-17) from Figure~\ref{Fig: salary vs gender}.  Explicitly declassifying the sum and count earlier in the program, allows us to change the average computation to a public computation.  This reduces the number of high cost communications and cryptographic computations in the program.  To support explicit declassification in our model we would need to extend our semantics with gradual 
release~\cite{GR}.

\section*{Acknowledgements}

This work was supported in part by a Google Faculty Research Award and US National Science Foundation grants 1749539, 1845803, 2040249, and 2213057.
Any opinions, findings, and conclusions or recommendations expressed in this publication are those of the authors and do not necessarily reflect the views of the funding sources.

%%
%% Bibliography
%%
\bibliography{main}

\appendix
\chapter{Supplementary Material}
\label{appendix}

In this appendix, we present supplementary material for the techniques and
experiments presented in the main text.

\section{Baseline Results and Analysis for Informed Sampler}
\label{appendix:chap3}

Here, we give an in-depth
performance analysis of the various samplers and the effect of their
hyperparameters. We choose hyperparameters with the lowest PSRF value
after $10k$ iterations, for each sampler individually. If the
differences between PSRF are not significantly different among
multiple values, we choose the one that has the highest acceptance
rate.

\subsection{Experiment: Estimating Camera Extrinsics}
\label{appendix:chap3:room}

\subsubsection{Parameter Selection}
\paragraph{Metropolis Hastings (\MH)}

Figure~\ref{fig:exp1_MH} shows the median acceptance rates and PSRF
values corresponding to various proposal standard deviations of plain
\MH~sampling. Mixing gets better and the acceptance rate gets worse as
the standard deviation increases. The value $0.3$ is selected standard
deviation for this sampler.

\paragraph{Metropolis Hastings Within Gibbs (\MHWG)}

As mentioned in Section~\ref{sec:room}, the \MHWG~sampler with one-dimensional
updates did not converge for any value of proposal standard deviation.
This problem has high correlation of the camera parameters and is of
multi-modal nature, which this sampler has problems with.

\paragraph{Parallel Tempering (\PT)}

For \PT~sampling, we took the best performing \MH~sampler and used
different temperature chains to improve the mixing of the
sampler. Figure~\ref{fig:exp1_PT} shows the results corresponding to
different combination of temperature levels. The sampler with
temperature levels of $[1,3,27]$ performed best in terms of both
mixing and acceptance rate.

\paragraph{Effect of Mixture Coefficient in Informed Sampling (\MIXLMH)}

Figure~\ref{fig:exp1_alpha} shows the effect of mixture
coefficient ($\alpha$) on the informed sampling
\MIXLMH. Since there is no significant different in PSRF values for
$0 \le \alpha \le 0.7$, we chose $0.7$ due to its high acceptance
rate.


% \end{multicols}

\begin{figure}[h]
\centering
  \subfigure[MH]{%
    \includegraphics[width=.48\textwidth]{figures/supplementary/camPose_MH.pdf} \label{fig:exp1_MH}
  }
  \subfigure[PT]{%
    \includegraphics[width=.48\textwidth]{figures/supplementary/camPose_PT.pdf} \label{fig:exp1_PT}
  }
\\
  \subfigure[INF-MH]{%
    \includegraphics[width=.48\textwidth]{figures/supplementary/camPose_alpha.pdf} \label{fig:exp1_alpha}
  }
  \mycaption{Results of the `Estimating Camera Extrinsics' experiment}{PRSFs and Acceptance rates corresponding to (a) various standard deviations of \MH, (b) various temperature level combinations of \PT sampling and (c) various mixture coefficients of \MIXLMH sampling.}
\end{figure}



\begin{figure}[!t]
\centering
  \subfigure[\MH]{%
    \includegraphics[width=.48\textwidth]{figures/supplementary/occlusionExp_MH.pdf} \label{fig:exp2_MH}
  }
  \subfigure[\BMHWG]{%
    \includegraphics[width=.48\textwidth]{figures/supplementary/occlusionExp_BMHWG.pdf} \label{fig:exp2_BMHWG}
  }
\\
  \subfigure[\MHWG]{%
    \includegraphics[width=.48\textwidth]{figures/supplementary/occlusionExp_MHWG.pdf} \label{fig:exp2_MHWG}
  }
  \subfigure[\PT]{%
    \includegraphics[width=.48\textwidth]{figures/supplementary/occlusionExp_PT.pdf} \label{fig:exp2_PT}
  }
\\
  \subfigure[\INFBMHWG]{%
    \includegraphics[width=.5\textwidth]{figures/supplementary/occlusionExp_alpha.pdf} \label{fig:exp2_alpha}
  }
  \mycaption{Results of the `Occluding Tiles' experiment}{PRSF and
    Acceptance rates corresponding to various standard deviations of
    (a) \MH, (b) \BMHWG, (c) \MHWG, (d) various temperature level
    combinations of \PT~sampling and; (e) various mixture coefficients
    of our informed \INFBMHWG sampling.}
\end{figure}

%\onecolumn\newpage\twocolumn
\subsection{Experiment: Occluding Tiles}
\label{appendix:chap3:tiles}

\subsubsection{Parameter Selection}

\paragraph{Metropolis Hastings (\MH)}

Figure~\ref{fig:exp2_MH} shows the results of
\MH~sampling. Results show the poor convergence for all proposal
standard deviations and rapid decrease of AR with increasing standard
deviation. This is due to the high-dimensional nature of
the problem. We selected a standard deviation of $1.1$.

\paragraph{Blocked Metropolis Hastings Within Gibbs (\BMHWG)}

The results of \BMHWG are shown in Figure~\ref{fig:exp2_BMHWG}. In
this sampler we update only one block of tile variables (of dimension
four) in each sampling step. Results show much better performance
compared to plain \MH. The optimal proposal standard deviation for
this sampler is $0.7$.

\paragraph{Metropolis Hastings Within Gibbs (\MHWG)}

Figure~\ref{fig:exp2_MHWG} shows the result of \MHWG sampling. This
sampler is better than \BMHWG and converges much more quickly. Here
a standard deviation of $0.9$ is found to be best.

\paragraph{Parallel Tempering (\PT)}

Figure~\ref{fig:exp2_PT} shows the results of \PT sampling with various
temperature combinations. Results show no improvement in AR from plain
\MH sampling and again $[1,3,27]$ temperature levels are found to be optimal.

\paragraph{Effect of Mixture Coefficient in Informed Sampling (\INFBMHWG)}

Figure~\ref{fig:exp2_alpha} shows the effect of mixture
coefficient ($\alpha$) on the blocked informed sampling
\INFBMHWG. Since there is no significant different in PSRF values for
$0 \le \alpha \le 0.8$, we chose $0.8$ due to its high acceptance
rate.



\subsection{Experiment: Estimating Body Shape}
\label{appendix:chap3:body}

\subsubsection{Parameter Selection}
\paragraph{Metropolis Hastings (\MH)}

Figure~\ref{fig:exp3_MH} shows the result of \MH~sampling with various
proposal standard deviations. The value of $0.1$ is found to be
best.

\paragraph{Metropolis Hastings Within Gibbs (\MHWG)}

For \MHWG sampling we select $0.3$ proposal standard
deviation. Results are shown in Fig.~\ref{fig:exp3_MHWG}.


\paragraph{Parallel Tempering (\PT)}

As before, results in Fig.~\ref{fig:exp3_PT}, the temperature levels
were selected to be $[1,3,27]$ due its slightly higher AR.

\paragraph{Effect of Mixture Coefficient in Informed Sampling (\MIXLMH)}

Figure~\ref{fig:exp3_alpha} shows the effect of $\alpha$ on PSRF and
AR. Since there is no significant differences in PSRF values for $0 \le
\alpha \le 0.8$, we choose $0.8$.


\begin{figure}[t]
\centering
  \subfigure[\MH]{%
    \includegraphics[width=.48\textwidth]{figures/supplementary/bodyShape_MH.pdf} \label{fig:exp3_MH}
  }
  \subfigure[\MHWG]{%
    \includegraphics[width=.48\textwidth]{figures/supplementary/bodyShape_MHWG.pdf} \label{fig:exp3_MHWG}
  }
\\
  \subfigure[\PT]{%
    \includegraphics[width=.48\textwidth]{figures/supplementary/bodyShape_PT.pdf} \label{fig:exp3_PT}
  }
  \subfigure[\MIXLMH]{%
    \includegraphics[width=.48\textwidth]{figures/supplementary/bodyShape_alpha.pdf} \label{fig:exp3_alpha}
  }
\\
  \mycaption{Results of the `Body Shape Estimation' experiment}{PRSFs and
    Acceptance rates corresponding to various standard deviations of
    (a) \MH, (b) \MHWG; (c) various temperature level combinations
    of \PT sampling and; (d) various mixture coefficients of the
    informed \MIXLMH sampling.}
\end{figure}


\subsection{Results Overview}
Figure~\ref{fig:exp_summary} shows the summary results of the all the three
experimental studies related to informed sampler.
\begin{figure*}[h!]
\centering
  \subfigure[Results for: Estimating Camera Extrinsics]{%
    \includegraphics[width=0.9\textwidth]{figures/supplementary/camPose_ALL.pdf} \label{fig:exp1_all}
  }
  \subfigure[Results for: Occluding Tiles]{%
    \includegraphics[width=0.9\textwidth]{figures/supplementary/occlusionExp_ALL.pdf} \label{fig:exp2_all}
  }
  \subfigure[Results for: Estimating Body Shape]{%
    \includegraphics[width=0.9\textwidth]{figures/supplementary/bodyShape_ALL.pdf} \label{fig:exp3_all}
  }
  \label{fig:exp_summary}
  \mycaption{Summary of the statistics for the three experiments}{Shown are
    for several baseline methods and the informed samplers the
    acceptance rates (left), PSRFs (middle), and RMSE values
    (right). All results are median results over multiple test
    examples.}
\end{figure*}

\subsection{Additional Qualitative Results}

\subsubsection{Occluding Tiles}
In Figure~\ref{fig:exp2_visual_more} more qualitative results of the
occluding tiles experiment are shown. The informed sampling approach
(\INFBMHWG) is better than the best baseline (\MHWG). This still is a
very challenging problem since the parameters for occluded tiles are
flat over a large region. Some of the posterior variance of the
occluded tiles is already captured by the informed sampler.

\begin{figure*}[h!]
\begin{center}
\centerline{\includegraphics[width=0.95\textwidth]{figures/supplementary/occlusionExp_Visual.pdf}}
\mycaption{Additional qualitative results of the occluding tiles experiment}
  {From left to right: (a)
  Given image, (b) Ground truth tiles, (c) OpenCV heuristic and most probable estimates
  from 5000 samples obtained by (d) MHWG sampler (best baseline) and
  (e) our INF-BMHWG sampler. (f) Posterior expectation of the tiles
  boundaries obtained by INF-BMHWG sampling (First 2000 samples are
  discarded as burn-in).}
\label{fig:exp2_visual_more}
\end{center}
\end{figure*}

\subsubsection{Body Shape}
Figure~\ref{fig:exp3_bodyMeshes} shows some more results of 3D mesh
reconstruction using posterior samples obtained by our informed
sampling \MIXLMH.

\begin{figure*}[t]
\begin{center}
\centerline{\includegraphics[width=0.75\textwidth]{figures/supplementary/bodyMeshResults.pdf}}
\mycaption{Qualitative results for the body shape experiment}
  {Shown is the 3D mesh reconstruction results with first 1000 samples obtained
  using the \MIXLMH informed sampling method. (blue indicates small
  values and red indicates high values)}
\label{fig:exp3_bodyMeshes}
\end{center}
\end{figure*}

\clearpage



\section{Additional Results on the Face Problem with CMP}

Figure~\ref{fig:shading-qualitative-multiple-subjects-supp} shows inference results for reflectance maps, normal maps and lights for randomly chosen test images, and Fig.~\ref{fig:shading-qualitative-same-subject-supp} shows reflectance estimation results on multiple images of the same subject produced under different illumination conditions. CMP is able to produce estimates that are closer to the groundtruth across different subjects and illumination conditions.

\begin{figure*}[h]
  \begin{center}
  \centerline{\includegraphics[width=1.0\columnwidth]{figures/face_cmp_visual_results_supp.pdf}}
  \vspace{-1.2cm}
  \end{center}
	\mycaption{A visual comparison of inference results}{(a)~Observed images. (b)~Inferred reflectance maps. \textit{GT} is the photometric stereo groundtruth, \textit{BU} is the Biswas \etal (2009) reflectance estimate and \textit{Forest} is the consensus prediction. (c)~The variance of the inferred reflectance estimate produced by \MTD (normalized across rows).(d)~Visualization of inferred light directions. (e)~Inferred normal maps.}
	\label{fig:shading-qualitative-multiple-subjects-supp}
\end{figure*}


\begin{figure*}[h]
	\centering
	\setlength\fboxsep{0.2mm}
	\setlength\fboxrule{0pt}
	\begin{tikzpicture}

		\matrix at (0, 0) [matrix of nodes, nodes={anchor=east}, column sep=-0.05cm, row sep=-0.2cm]
		{
			\fbox{\includegraphics[width=1cm]{figures/sample_3_4_X.png}} &
			\fbox{\includegraphics[width=1cm]{figures/sample_3_4_GT.png}} &
			\fbox{\includegraphics[width=1cm]{figures/sample_3_4_BISWAS.png}}  &
			\fbox{\includegraphics[width=1cm]{figures/sample_3_4_VMP.png}}  &
			\fbox{\includegraphics[width=1cm]{figures/sample_3_4_FOREST.png}}  &
			\fbox{\includegraphics[width=1cm]{figures/sample_3_4_CMP.png}}  &
			\fbox{\includegraphics[width=1cm]{figures/sample_3_4_CMPVAR.png}}
			 \\

			\fbox{\includegraphics[width=1cm]{figures/sample_3_5_X.png}} &
			\fbox{\includegraphics[width=1cm]{figures/sample_3_5_GT.png}} &
			\fbox{\includegraphics[width=1cm]{figures/sample_3_5_BISWAS.png}}  &
			\fbox{\includegraphics[width=1cm]{figures/sample_3_5_VMP.png}}  &
			\fbox{\includegraphics[width=1cm]{figures/sample_3_5_FOREST.png}}  &
			\fbox{\includegraphics[width=1cm]{figures/sample_3_5_CMP.png}}  &
			\fbox{\includegraphics[width=1cm]{figures/sample_3_5_CMPVAR.png}}
			 \\

			\fbox{\includegraphics[width=1cm]{figures/sample_3_6_X.png}} &
			\fbox{\includegraphics[width=1cm]{figures/sample_3_6_GT.png}} &
			\fbox{\includegraphics[width=1cm]{figures/sample_3_6_BISWAS.png}}  &
			\fbox{\includegraphics[width=1cm]{figures/sample_3_6_VMP.png}}  &
			\fbox{\includegraphics[width=1cm]{figures/sample_3_6_FOREST.png}}  &
			\fbox{\includegraphics[width=1cm]{figures/sample_3_6_CMP.png}}  &
			\fbox{\includegraphics[width=1cm]{figures/sample_3_6_CMPVAR.png}}
			 \\
	     };

       \node at (-3.85, -2.0) {\small Observed};
       \node at (-2.55, -2.0) {\small `GT'};
       \node at (-1.27, -2.0) {\small BU};
       \node at (0.0, -2.0) {\small MP};
       \node at (1.27, -2.0) {\small Forest};
       \node at (2.55, -2.0) {\small \textbf{CMP}};
       \node at (3.85, -2.0) {\small Variance};

	\end{tikzpicture}
	\mycaption{Robustness to varying illumination}{Reflectance estimation on a subject images with varying illumination. Left to right: observed image, photometric stereo estimate (GT)
  which is used as a proxy for groundtruth, bottom-up estimate of \cite{Biswas2009}, VMP result, consensus forest estimate, CMP mean, and CMP variance.}
	\label{fig:shading-qualitative-same-subject-supp}
\end{figure*}

\clearpage

\section{Additional Material for Learning Sparse High Dimensional Filters}
\label{sec:appendix-bnn}

This part of supplementary material contains a more detailed overview of the permutohedral
lattice convolution in Section~\ref{sec:permconv}, more experiments in
Section~\ref{sec:addexps} and additional results with protocols for
the experiments presented in Chapter~\ref{chap:bnn} in Section~\ref{sec:addresults}.

\vspace{-0.2cm}
\subsection{General Permutohedral Convolutions}
\label{sec:permconv}

A core technical contribution of this work is the generalization of the Gaussian permutohedral lattice
convolution proposed in~\cite{adams2010fast} to the full non-separable case with the
ability to perform back-propagation. Although, conceptually, there are minor
differences between Gaussian and general parameterized filters, there are non-trivial practical
differences in terms of the algorithmic implementation. The Gauss filters belong to
the separable class and can thus be decomposed into multiple
sequential one dimensional convolutions. We are interested in the general filter
convolutions, which can not be decomposed. Thus, performing a general permutohedral
convolution at a lattice point requires the computation of the inner product with the
neighboring elements in all the directions in the high-dimensional space.

Here, we give more details of the implementation differences of separable
and non-separable filters. In the following, we will explain the scalar case first.
Recall, that the forward pass of general permutohedral convolution
involves 3 steps: \textit{splatting}, \textit{convolving} and \textit{slicing}.
We follow the same splatting and slicing strategies as in~\cite{adams2010fast}
since these operations do not depend on the filter kernel. The main difference
between our work and the existing implementation of~\cite{adams2010fast} is
the way that the convolution operation is executed. This proceeds by constructing
a \emph{blur neighbor} matrix $K$ that stores for every lattice point all
values of the lattice neighbors that are needed to compute the filter output.

\begin{figure}[t!]
  \centering
    \includegraphics[width=0.6\columnwidth]{figures/supplementary/lattice_construction}
  \mycaption{Illustration of 1D permutohedral lattice construction}
  {A $4\times 4$ $(x,y)$ grid lattice is projected onto the plane defined by the normal
  vector $(1,1)^{\top}$. This grid has $s+1=4$ and $d=2$ $(s+1)^{d}=4^2=16$ elements.
  In the projection, all points of the same color are projected onto the same points in the plane.
  The number of elements of the projected lattice is $t=(s+1)^d-s^d=4^2-3^2=7$, that is
  the $(4\times 4)$ grid minus the size of lattice that is $1$ smaller at each size, in this
  case a $(3\times 3)$ lattice (the upper right $(3\times 3)$ elements).
  }
\label{fig:latticeconstruction}
\end{figure}

The blur neighbor matrix is constructed by traversing through all the populated
lattice points and their neighboring elements.
% For efficiency, we do this matrix construction recursively with shared computations
% since $n^{th}$ neighbourhood elements are $1^{st}$ neighborhood elements of $n-1^{th}$ neighbourhood elements. \pg{do not understand}
This is done recursively to share computations. For any lattice point, the neighbors that are
$n$ hops away are the direct neighbors of the points that are $n-1$ hops away.
The size of a $d$ dimensional spatial filter with width $s+1$ is $(s+1)^{d}$ (\eg, a
$3\times 3$ filter, $s=2$ in $d=2$ has $3^2=9$ elements) and this size grows
exponentially in the number of dimensions $d$. The permutohedral lattice is constructed by
projecting a regular grid onto the plane spanned by the $d$ dimensional normal vector ${(1,\ldots,1)}^{\top}$. See
Fig.~\ref{fig:latticeconstruction} for an illustration of the 1D lattice construction.
Many corners of a grid filter are projected onto the same point, in total $t = {(s+1)}^{d} -
s^{d}$ elements remain in the permutohedral filter with $s$ neighborhood in $d-1$ dimensions.
If the lattice has $m$ populated elements, the
matrix $K$ has size $t\times m$. Note that, since the input signal is typically
sparse, only a few lattice corners are being populated in the \textit{slicing} step.
We use a hash-table to keep track of these points and traverse only through
the populated lattice points for this neighborhood matrix construction.

Once the blur neighbor matrix $K$ is constructed, we can perform the convolution
by the matrix vector multiplication
\begin{equation}
\ell' = BK,
\label{eq:conv}
\end{equation}
where $B$ is the $1 \times t$ filter kernel (whose values we will learn) and $\ell'\in\mathbb{R}^{1\times m}$
is the result of the filtering at the $m$ lattice points. In practice, we found that the
matrix $K$ is sometimes too large to fit into GPU memory and we divided the matrix $K$
into smaller pieces to compute Eq.~\ref{eq:conv} sequentially.

In the general multi-dimensional case, the signal $\ell$ is of $c$ dimensions. Then
the kernel $B$ is of size $c \times t$ and $K$ stores the $c$ dimensional vectors
accordingly. When the input and output points are different, we slice only the
input points and splat only at the output points.


\subsection{Additional Experiments}
\label{sec:addexps}
In this section, we discuss more use-cases for the learned bilateral filters, one
use-case of BNNs and two single filter applications for image and 3D mesh denoising.

\subsubsection{Recognition of subsampled MNIST}\label{sec:app_mnist}

One of the strengths of the proposed filter convolution is that it does not
require the input to lie on a regular grid. The only requirement is to define a distance
between features of the input signal.
We highlight this feature with the following experiment using the
classical MNIST ten class classification problem~\cite{lecun1998mnist}. We sample a
sparse set of $N$ points $(x,y)\in [0,1]\times [0,1]$
uniformly at random in the input image, use their interpolated values
as signal and the \emph{continuous} $(x,y)$ positions as features. This mimics
sub-sampling of a high-dimensional signal. To compare against a spatial convolution,
we interpolate the sparse set of values at the grid positions.

We take a reference implementation of LeNet~\cite{lecun1998gradient} that
is part of the Caffe project~\cite{jia2014caffe} and compare it
against the same architecture but replacing the first convolutional
layer with a bilateral convolution layer (BCL). The filter size
and numbers are adjusted to get a comparable number of parameters
($5\times 5$ for LeNet, $2$-neighborhood for BCL).

The results are shown in Table~\ref{tab:all-results}. We see that training
on the original MNIST data (column Original, LeNet vs. BNN) leads to a slight
decrease in performance of the BNN (99.03\%) compared to LeNet
(99.19\%). The BNN can be trained and evaluated on sparse
signals, and we resample the image as described above for $N=$ 100\%, 60\% and
20\% of the total number of pixels. The methods are also evaluated
on test images that are subsampled in the same way. Note that we can
train and test with different subsampling rates. We introduce an additional
bilinear interpolation layer for the LeNet architecture to train on the same
data. In essence, both models perform a spatial interpolation and thus we
expect them to yield a similar classification accuracy. Once the data is of
higher dimensions, the permutohedral convolution will be faster due to hashing
the sparse input points, as well as less memory demanding in comparison to
naive application of a spatial convolution with interpolated values.

\begin{table}[t]
  \begin{center}
    \footnotesize
    \centering
    \begin{tabular}[t]{lllll}
      \toprule
              &     & \multicolumn{3}{c}{Test Subsampling} \\
       Method  & Original & 100\% & 60\% & 20\%\\
      \midrule
       LeNet &  \textbf{0.9919} & 0.9660 & 0.9348 & \textbf{0.6434} \\
       BNN &  0.9903 & \textbf{0.9844} & \textbf{0.9534} & 0.5767 \\
      \hline
       LeNet 100\% & 0.9856 & 0.9809 & 0.9678 & \textbf{0.7386} \\
       BNN 100\% & \textbf{0.9900} & \textbf{0.9863} & \textbf{0.9699} & 0.6910 \\
      \hline
       LeNet 60\% & 0.9848 & 0.9821 & 0.9740 & 0.8151 \\
       BNN 60\% & \textbf{0.9885} & \textbf{0.9864} & \textbf{0.9771} & \textbf{0.8214}\\
      \hline
       LeNet 20\% & \textbf{0.9763} & \textbf{0.9754} & 0.9695 & 0.8928 \\
       BNN 20\% & 0.9728 & 0.9735 & \textbf{0.9701} & \textbf{0.9042}\\
      \bottomrule
    \end{tabular}
  \end{center}
\vspace{-.2cm}
\caption{Classification accuracy on MNIST. We compare the
    LeNet~\cite{lecun1998gradient} implementation that is part of
    Caffe~\cite{jia2014caffe} to the network with the first layer
    replaced by a bilateral convolution layer (BCL). Both are trained
    on the original image resolution (first two rows). Three more BNN
    and CNN models are trained with randomly subsampled images (100\%,
    60\% and 20\% of the pixels). An additional bilinear interpolation
    layer samples the input signal on a spatial grid for the CNN model.
  }
  \label{tab:all-results}
\vspace{-.5cm}
\end{table}

\subsubsection{Image Denoising}

The main application that inspired the development of the bilateral
filtering operation is image denoising~\cite{aurich1995non}, there
using a single Gaussian kernel. Our development allows to learn this
kernel function from data and we explore how to improve using a \emph{single}
but more general bilateral filter.

We use the Berkeley segmentation dataset
(BSDS500)~\cite{arbelaezi2011bsds500} as a test bed. The color
images in the dataset are converted to gray-scale,
and corrupted with Gaussian noise with a standard deviation of
$\frac {25} {255}$.

We compare the performance of four different filter models on a
denoising task.
The first baseline model (`Spatial' in Table \ref{tab:denoising}, $25$
weights) uses a single spatial filter with a kernel size of
$5$ and predicts the scalar gray-scale value at the center pixel. The next model
(`Gauss Bilateral') applies a bilateral \emph{Gaussian}
filter to the noisy input, using position and intensity features $\f=(x,y,v)^\top$.
The third setup (`Learned Bilateral', $65$ weights)
takes a Gauss kernel as initialization and
fits all filter weights on the train set to minimize the
mean squared error with respect to the clean images.
We run a combination
of spatial and permutohedral convolutions on spatial and bilateral
features (`Spatial + Bilateral (Learned)') to check for a complementary
performance of the two convolutions.

\label{sec:exp:denoising}
\begin{table}[!h]
\begin{center}
  \footnotesize
  \begin{tabular}[t]{lr}
    \toprule
    Method & PSNR \\
    \midrule
    Noisy Input & $20.17$ \\
    Spatial & $26.27$ \\
    Gauss Bilateral & $26.51$ \\
    Learned Bilateral & $26.58$ \\
    Spatial + Bilateral (Learned) & \textbf{$26.65$} \\
    \bottomrule
  \end{tabular}
\end{center}
\vspace{-0.5em}
\caption{PSNR results of a denoising task using the BSDS500
  dataset~\cite{arbelaezi2011bsds500}}
\vspace{-0.5em}
\label{tab:denoising}
\end{table}
\vspace{-0.2em}

The PSNR scores evaluated on full images of the test set are
shown in Table \ref{tab:denoising}. We find that an untrained bilateral
filter already performs better than a trained spatial convolution
($26.27$ to $26.51$). A learned convolution further improve the
performance slightly. We chose this simple one-kernel setup to
validate an advantage of the generalized bilateral filter. A competitive
denoising system would employ RGB color information and also
needs to be properly adjusted in network size. Multi-layer perceptrons
have obtained state-of-the-art denoising results~\cite{burger12cvpr}
and the permutohedral lattice layer can readily be used in such an
architecture, which is intended future work.

\subsection{Additional results}
\label{sec:addresults}

This section contains more qualitative results for the experiments presented in Chapter~\ref{chap:bnn}.

\begin{figure*}[th!]
  \centering
    \includegraphics[width=\columnwidth,trim={5cm 2.5cm 5cm 4.5cm},clip]{figures/supplementary/lattice_viz.pdf}
    \vspace{-0.7cm}
  \mycaption{Visualization of the Permutohedral Lattice}
  {Sample lattice visualizations for different feature spaces. All pixels falling in the same simplex cell are shown with
  the same color. $(x,y)$ features correspond to image pixel positions, and $(r,g,b) \in [0,255]$ correspond
  to the red, green and blue color values.}
\label{fig:latticeviz}
\end{figure*}

\subsubsection{Lattice Visualization}

Figure~\ref{fig:latticeviz} shows sample lattice visualizations for different feature spaces.

\newcolumntype{L}[1]{>{\raggedright\let\newline\\\arraybackslash\hspace{0pt}}b{#1}}
\newcolumntype{C}[1]{>{\centering\let\newline\\\arraybackslash\hspace{0pt}}b{#1}}
\newcolumntype{R}[1]{>{\raggedleft\let\newline\\\arraybackslash\hspace{0pt}}b{#1}}

\subsubsection{Color Upsampling}\label{sec:color_upsampling}
\label{sec:col_upsample_extra}

Some images of the upsampling for the Pascal
VOC12 dataset are shown in Fig.~\ref{fig:Colour_upsample_visuals}. It is
especially the low level image details that are better preserved with
a learned bilateral filter compared to the Gaussian case.

\begin{figure*}[t!]
  \centering
    \subfigure{%
   \raisebox{2.0em}{
    \includegraphics[width=.06\columnwidth]{figures/supplementary/2007_004969.jpg}
   }
  }
  \subfigure{%
    \includegraphics[width=.17\columnwidth]{figures/supplementary/2007_004969_gray.pdf}
  }
  \subfigure{%
    \includegraphics[width=.17\columnwidth]{figures/supplementary/2007_004969_gt.pdf}
  }
  \subfigure{%
    \includegraphics[width=.17\columnwidth]{figures/supplementary/2007_004969_bicubic.pdf}
  }
  \subfigure{%
    \includegraphics[width=.17\columnwidth]{figures/supplementary/2007_004969_gauss.pdf}
  }
  \subfigure{%
    \includegraphics[width=.17\columnwidth]{figures/supplementary/2007_004969_learnt.pdf}
  }\\
    \subfigure{%
   \raisebox{2.0em}{
    \includegraphics[width=.06\columnwidth]{figures/supplementary/2007_003106.jpg}
   }
  }
  \subfigure{%
    \includegraphics[width=.17\columnwidth]{figures/supplementary/2007_003106_gray.pdf}
  }
  \subfigure{%
    \includegraphics[width=.17\columnwidth]{figures/supplementary/2007_003106_gt.pdf}
  }
  \subfigure{%
    \includegraphics[width=.17\columnwidth]{figures/supplementary/2007_003106_bicubic.pdf}
  }
  \subfigure{%
    \includegraphics[width=.17\columnwidth]{figures/supplementary/2007_003106_gauss.pdf}
  }
  \subfigure{%
    \includegraphics[width=.17\columnwidth]{figures/supplementary/2007_003106_learnt.pdf}
  }\\
  \setcounter{subfigure}{0}
  \small{
  \subfigure[Inp.]{%
  \raisebox{2.0em}{
    \includegraphics[width=.06\columnwidth]{figures/supplementary/2007_006837.jpg}
   }
  }
  \subfigure[Guidance]{%
    \includegraphics[width=.17\columnwidth]{figures/supplementary/2007_006837_gray.pdf}
  }
   \subfigure[GT]{%
    \includegraphics[width=.17\columnwidth]{figures/supplementary/2007_006837_gt.pdf}
  }
  \subfigure[Bicubic]{%
    \includegraphics[width=.17\columnwidth]{figures/supplementary/2007_006837_bicubic.pdf}
  }
  \subfigure[Gauss-BF]{%
    \includegraphics[width=.17\columnwidth]{figures/supplementary/2007_006837_gauss.pdf}
  }
  \subfigure[Learned-BF]{%
    \includegraphics[width=.17\columnwidth]{figures/supplementary/2007_006837_learnt.pdf}
  }
  }
  \vspace{-0.5cm}
  \mycaption{Color Upsampling}{Color $8\times$ upsampling results
  using different methods, from left to right, (a)~Low-resolution input color image (Inp.),
  (b)~Gray scale guidance image, (c)~Ground-truth color image; Upsampled color images with
  (d)~Bicubic interpolation, (e) Gauss bilateral upsampling and, (f)~Learned bilateral
  updampgling (best viewed on screen).}

\label{fig:Colour_upsample_visuals}
\end{figure*}

\subsubsection{Depth Upsampling}
\label{sec:depth_upsample_extra}

Figure~\ref{fig:depth_upsample_visuals} presents some more qualitative results comparing bicubic interpolation, Gauss
bilateral and learned bilateral upsampling on NYU depth dataset image~\cite{silberman2012indoor}.

\subsubsection{Character Recognition}\label{sec:app_character}

 Figure~\ref{fig:nnrecognition} shows the schematic of different layers
 of the network architecture for LeNet-7~\cite{lecun1998mnist}
 and DeepCNet(5, 50)~\cite{ciresan2012multi,graham2014spatially}. For the BNN variants, the first layer filters are replaced
 with learned bilateral filters and are learned end-to-end.

\subsubsection{Semantic Segmentation}\label{sec:app_semantic_segmentation}
\label{sec:semantic_bnn_extra}

Some more visual results for semantic segmentation are shown in Figure~\ref{fig:semantic_visuals}.
These include the underlying DeepLab CNN\cite{chen2014semantic} result (DeepLab),
the 2 step mean-field result with Gaussian edge potentials (+2stepMF-GaussCRF)
and also corresponding results with learned edge potentials (+2stepMF-LearnedCRF).
In general, we observe that mean-field in learned CRF leads to slightly dilated
classification regions in comparison to using Gaussian CRF thereby filling-in the
false negative pixels and also correcting some mis-classified regions.

\begin{figure*}[t!]
  \centering
    \subfigure{%
   \raisebox{2.0em}{
    \includegraphics[width=.06\columnwidth]{figures/supplementary/2bicubic}
   }
  }
  \subfigure{%
    \includegraphics[width=.17\columnwidth]{figures/supplementary/2given_image}
  }
  \subfigure{%
    \includegraphics[width=.17\columnwidth]{figures/supplementary/2ground_truth}
  }
  \subfigure{%
    \includegraphics[width=.17\columnwidth]{figures/supplementary/2bicubic}
  }
  \subfigure{%
    \includegraphics[width=.17\columnwidth]{figures/supplementary/2gauss}
  }
  \subfigure{%
    \includegraphics[width=.17\columnwidth]{figures/supplementary/2learnt}
  }\\
    \subfigure{%
   \raisebox{2.0em}{
    \includegraphics[width=.06\columnwidth]{figures/supplementary/32bicubic}
   }
  }
  \subfigure{%
    \includegraphics[width=.17\columnwidth]{figures/supplementary/32given_image}
  }
  \subfigure{%
    \includegraphics[width=.17\columnwidth]{figures/supplementary/32ground_truth}
  }
  \subfigure{%
    \includegraphics[width=.17\columnwidth]{figures/supplementary/32bicubic}
  }
  \subfigure{%
    \includegraphics[width=.17\columnwidth]{figures/supplementary/32gauss}
  }
  \subfigure{%
    \includegraphics[width=.17\columnwidth]{figures/supplementary/32learnt}
  }\\
  \setcounter{subfigure}{0}
  \small{
  \subfigure[Inp.]{%
  \raisebox{2.0em}{
    \includegraphics[width=.06\columnwidth]{figures/supplementary/41bicubic}
   }
  }
  \subfigure[Guidance]{%
    \includegraphics[width=.17\columnwidth]{figures/supplementary/41given_image}
  }
   \subfigure[GT]{%
    \includegraphics[width=.17\columnwidth]{figures/supplementary/41ground_truth}
  }
  \subfigure[Bicubic]{%
    \includegraphics[width=.17\columnwidth]{figures/supplementary/41bicubic}
  }
  \subfigure[Gauss-BF]{%
    \includegraphics[width=.17\columnwidth]{figures/supplementary/41gauss}
  }
  \subfigure[Learned-BF]{%
    \includegraphics[width=.17\columnwidth]{figures/supplementary/41learnt}
  }
  }
  \mycaption{Depth Upsampling}{Depth $8\times$ upsampling results
  using different upsampling strategies, from left to right,
  (a)~Low-resolution input depth image (Inp.),
  (b)~High-resolution guidance image, (c)~Ground-truth depth; Upsampled depth images with
  (d)~Bicubic interpolation, (e) Gauss bilateral upsampling and, (f)~Learned bilateral
  updampgling (best viewed on screen).}

\label{fig:depth_upsample_visuals}
\end{figure*}

\subsubsection{Material Segmentation}\label{sec:app_material_segmentation}
\label{sec:material_bnn_extra}

In Fig.~\ref{fig:material_visuals-app2}, we present visual results comparing 2 step
mean-field inference with Gaussian and learned pairwise CRF potentials. In
general, we observe that the pixels belonging to dominant classes in the
training data are being more accurately classified with learned CRF. This leads to
a significant improvements in overall pixel accuracy. This also results
in a slight decrease of the accuracy from less frequent class pixels thereby
slightly reducing the average class accuracy with learning. We attribute this
to the type of annotation that is available for this dataset, which is not
for the entire image but for some segments in the image. We have very few
images of the infrequent classes to combat this behaviour during training.

\subsubsection{Experiment Protocols}
\label{sec:protocols}

Table~\ref{tbl:parameters} shows experiment protocols of different experiments.

 \begin{figure*}[t!]
  \centering
  \subfigure[LeNet-7]{
    \includegraphics[width=0.7\columnwidth]{figures/supplementary/lenet_cnn_network}
    }\\
    \subfigure[DeepCNet]{
    \includegraphics[width=\columnwidth]{figures/supplementary/deepcnet_cnn_network}
    }
  \mycaption{CNNs for Character Recognition}
  {Schematic of (top) LeNet-7~\cite{lecun1998mnist} and (bottom) DeepCNet(5,50)~\cite{ciresan2012multi,graham2014spatially} architectures used in Assamese
  character recognition experiments.}
\label{fig:nnrecognition}
\end{figure*}

\definecolor{voc_1}{RGB}{0, 0, 0}
\definecolor{voc_2}{RGB}{128, 0, 0}
\definecolor{voc_3}{RGB}{0, 128, 0}
\definecolor{voc_4}{RGB}{128, 128, 0}
\definecolor{voc_5}{RGB}{0, 0, 128}
\definecolor{voc_6}{RGB}{128, 0, 128}
\definecolor{voc_7}{RGB}{0, 128, 128}
\definecolor{voc_8}{RGB}{128, 128, 128}
\definecolor{voc_9}{RGB}{64, 0, 0}
\definecolor{voc_10}{RGB}{192, 0, 0}
\definecolor{voc_11}{RGB}{64, 128, 0}
\definecolor{voc_12}{RGB}{192, 128, 0}
\definecolor{voc_13}{RGB}{64, 0, 128}
\definecolor{voc_14}{RGB}{192, 0, 128}
\definecolor{voc_15}{RGB}{64, 128, 128}
\definecolor{voc_16}{RGB}{192, 128, 128}
\definecolor{voc_17}{RGB}{0, 64, 0}
\definecolor{voc_18}{RGB}{128, 64, 0}
\definecolor{voc_19}{RGB}{0, 192, 0}
\definecolor{voc_20}{RGB}{128, 192, 0}
\definecolor{voc_21}{RGB}{0, 64, 128}
\definecolor{voc_22}{RGB}{128, 64, 128}

\begin{figure*}[t]
  \centering
  \small{
  \fcolorbox{white}{voc_1}{\rule{0pt}{6pt}\rule{6pt}{0pt}} Background~~
  \fcolorbox{white}{voc_2}{\rule{0pt}{6pt}\rule{6pt}{0pt}} Aeroplane~~
  \fcolorbox{white}{voc_3}{\rule{0pt}{6pt}\rule{6pt}{0pt}} Bicycle~~
  \fcolorbox{white}{voc_4}{\rule{0pt}{6pt}\rule{6pt}{0pt}} Bird~~
  \fcolorbox{white}{voc_5}{\rule{0pt}{6pt}\rule{6pt}{0pt}} Boat~~
  \fcolorbox{white}{voc_6}{\rule{0pt}{6pt}\rule{6pt}{0pt}} Bottle~~
  \fcolorbox{white}{voc_7}{\rule{0pt}{6pt}\rule{6pt}{0pt}} Bus~~
  \fcolorbox{white}{voc_8}{\rule{0pt}{6pt}\rule{6pt}{0pt}} Car~~ \\
  \fcolorbox{white}{voc_9}{\rule{0pt}{6pt}\rule{6pt}{0pt}} Cat~~
  \fcolorbox{white}{voc_10}{\rule{0pt}{6pt}\rule{6pt}{0pt}} Chair~~
  \fcolorbox{white}{voc_11}{\rule{0pt}{6pt}\rule{6pt}{0pt}} Cow~~
  \fcolorbox{white}{voc_12}{\rule{0pt}{6pt}\rule{6pt}{0pt}} Dining Table~~
  \fcolorbox{white}{voc_13}{\rule{0pt}{6pt}\rule{6pt}{0pt}} Dog~~
  \fcolorbox{white}{voc_14}{\rule{0pt}{6pt}\rule{6pt}{0pt}} Horse~~
  \fcolorbox{white}{voc_15}{\rule{0pt}{6pt}\rule{6pt}{0pt}} Motorbike~~
  \fcolorbox{white}{voc_16}{\rule{0pt}{6pt}\rule{6pt}{0pt}} Person~~ \\
  \fcolorbox{white}{voc_17}{\rule{0pt}{6pt}\rule{6pt}{0pt}} Potted Plant~~
  \fcolorbox{white}{voc_18}{\rule{0pt}{6pt}\rule{6pt}{0pt}} Sheep~~
  \fcolorbox{white}{voc_19}{\rule{0pt}{6pt}\rule{6pt}{0pt}} Sofa~~
  \fcolorbox{white}{voc_20}{\rule{0pt}{6pt}\rule{6pt}{0pt}} Train~~
  \fcolorbox{white}{voc_21}{\rule{0pt}{6pt}\rule{6pt}{0pt}} TV monitor~~ \\
  }
  \subfigure{%
    \includegraphics[width=.18\columnwidth]{figures/supplementary/2007_001423_given.jpg}
  }
  \subfigure{%
    \includegraphics[width=.18\columnwidth]{figures/supplementary/2007_001423_gt.png}
  }
  \subfigure{%
    \includegraphics[width=.18\columnwidth]{figures/supplementary/2007_001423_cnn.png}
  }
  \subfigure{%
    \includegraphics[width=.18\columnwidth]{figures/supplementary/2007_001423_gauss.png}
  }
  \subfigure{%
    \includegraphics[width=.18\columnwidth]{figures/supplementary/2007_001423_learnt.png}
  }\\
  \subfigure{%
    \includegraphics[width=.18\columnwidth]{figures/supplementary/2007_001430_given.jpg}
  }
  \subfigure{%
    \includegraphics[width=.18\columnwidth]{figures/supplementary/2007_001430_gt.png}
  }
  \subfigure{%
    \includegraphics[width=.18\columnwidth]{figures/supplementary/2007_001430_cnn.png}
  }
  \subfigure{%
    \includegraphics[width=.18\columnwidth]{figures/supplementary/2007_001430_gauss.png}
  }
  \subfigure{%
    \includegraphics[width=.18\columnwidth]{figures/supplementary/2007_001430_learnt.png}
  }\\
    \subfigure{%
    \includegraphics[width=.18\columnwidth]{figures/supplementary/2007_007996_given.jpg}
  }
  \subfigure{%
    \includegraphics[width=.18\columnwidth]{figures/supplementary/2007_007996_gt.png}
  }
  \subfigure{%
    \includegraphics[width=.18\columnwidth]{figures/supplementary/2007_007996_cnn.png}
  }
  \subfigure{%
    \includegraphics[width=.18\columnwidth]{figures/supplementary/2007_007996_gauss.png}
  }
  \subfigure{%
    \includegraphics[width=.18\columnwidth]{figures/supplementary/2007_007996_learnt.png}
  }\\
   \subfigure{%
    \includegraphics[width=.18\columnwidth]{figures/supplementary/2010_002682_given.jpg}
  }
  \subfigure{%
    \includegraphics[width=.18\columnwidth]{figures/supplementary/2010_002682_gt.png}
  }
  \subfigure{%
    \includegraphics[width=.18\columnwidth]{figures/supplementary/2010_002682_cnn.png}
  }
  \subfigure{%
    \includegraphics[width=.18\columnwidth]{figures/supplementary/2010_002682_gauss.png}
  }
  \subfigure{%
    \includegraphics[width=.18\columnwidth]{figures/supplementary/2010_002682_learnt.png}
  }\\
     \subfigure{%
    \includegraphics[width=.18\columnwidth]{figures/supplementary/2010_004789_given.jpg}
  }
  \subfigure{%
    \includegraphics[width=.18\columnwidth]{figures/supplementary/2010_004789_gt.png}
  }
  \subfigure{%
    \includegraphics[width=.18\columnwidth]{figures/supplementary/2010_004789_cnn.png}
  }
  \subfigure{%
    \includegraphics[width=.18\columnwidth]{figures/supplementary/2010_004789_gauss.png}
  }
  \subfigure{%
    \includegraphics[width=.18\columnwidth]{figures/supplementary/2010_004789_learnt.png}
  }\\
       \subfigure{%
    \includegraphics[width=.18\columnwidth]{figures/supplementary/2007_001311_given.jpg}
  }
  \subfigure{%
    \includegraphics[width=.18\columnwidth]{figures/supplementary/2007_001311_gt.png}
  }
  \subfigure{%
    \includegraphics[width=.18\columnwidth]{figures/supplementary/2007_001311_cnn.png}
  }
  \subfigure{%
    \includegraphics[width=.18\columnwidth]{figures/supplementary/2007_001311_gauss.png}
  }
  \subfigure{%
    \includegraphics[width=.18\columnwidth]{figures/supplementary/2007_001311_learnt.png}
  }\\
  \setcounter{subfigure}{0}
  \subfigure[Input]{%
    \includegraphics[width=.18\columnwidth]{figures/supplementary/2010_003531_given.jpg}
  }
  \subfigure[Ground Truth]{%
    \includegraphics[width=.18\columnwidth]{figures/supplementary/2010_003531_gt.png}
  }
  \subfigure[DeepLab]{%
    \includegraphics[width=.18\columnwidth]{figures/supplementary/2010_003531_cnn.png}
  }
  \subfigure[+GaussCRF]{%
    \includegraphics[width=.18\columnwidth]{figures/supplementary/2010_003531_gauss.png}
  }
  \subfigure[+LearnedCRF]{%
    \includegraphics[width=.18\columnwidth]{figures/supplementary/2010_003531_learnt.png}
  }
  \vspace{-0.3cm}
  \mycaption{Semantic Segmentation}{Example results of semantic segmentation.
  (c)~depicts the unary results before application of MF, (d)~after two steps of MF with Gaussian edge CRF potentials, (e)~after
  two steps of MF with learned edge CRF potentials.}
    \label{fig:semantic_visuals}
\end{figure*}


\definecolor{minc_1}{HTML}{771111}
\definecolor{minc_2}{HTML}{CAC690}
\definecolor{minc_3}{HTML}{EEEEEE}
\definecolor{minc_4}{HTML}{7C8FA6}
\definecolor{minc_5}{HTML}{597D31}
\definecolor{minc_6}{HTML}{104410}
\definecolor{minc_7}{HTML}{BB819C}
\definecolor{minc_8}{HTML}{D0CE48}
\definecolor{minc_9}{HTML}{622745}
\definecolor{minc_10}{HTML}{666666}
\definecolor{minc_11}{HTML}{D54A31}
\definecolor{minc_12}{HTML}{101044}
\definecolor{minc_13}{HTML}{444126}
\definecolor{minc_14}{HTML}{75D646}
\definecolor{minc_15}{HTML}{DD4348}
\definecolor{minc_16}{HTML}{5C8577}
\definecolor{minc_17}{HTML}{C78472}
\definecolor{minc_18}{HTML}{75D6D0}
\definecolor{minc_19}{HTML}{5B4586}
\definecolor{minc_20}{HTML}{C04393}
\definecolor{minc_21}{HTML}{D69948}
\definecolor{minc_22}{HTML}{7370D8}
\definecolor{minc_23}{HTML}{7A3622}
\definecolor{minc_24}{HTML}{000000}

\begin{figure*}[t]
  \centering
  \small{
  \fcolorbox{white}{minc_1}{\rule{0pt}{6pt}\rule{6pt}{0pt}} Brick~~
  \fcolorbox{white}{minc_2}{\rule{0pt}{6pt}\rule{6pt}{0pt}} Carpet~~
  \fcolorbox{white}{minc_3}{\rule{0pt}{6pt}\rule{6pt}{0pt}} Ceramic~~
  \fcolorbox{white}{minc_4}{\rule{0pt}{6pt}\rule{6pt}{0pt}} Fabric~~
  \fcolorbox{white}{minc_5}{\rule{0pt}{6pt}\rule{6pt}{0pt}} Foliage~~
  \fcolorbox{white}{minc_6}{\rule{0pt}{6pt}\rule{6pt}{0pt}} Food~~
  \fcolorbox{white}{minc_7}{\rule{0pt}{6pt}\rule{6pt}{0pt}} Glass~~
  \fcolorbox{white}{minc_8}{\rule{0pt}{6pt}\rule{6pt}{0pt}} Hair~~ \\
  \fcolorbox{white}{minc_9}{\rule{0pt}{6pt}\rule{6pt}{0pt}} Leather~~
  \fcolorbox{white}{minc_10}{\rule{0pt}{6pt}\rule{6pt}{0pt}} Metal~~
  \fcolorbox{white}{minc_11}{\rule{0pt}{6pt}\rule{6pt}{0pt}} Mirror~~
  \fcolorbox{white}{minc_12}{\rule{0pt}{6pt}\rule{6pt}{0pt}} Other~~
  \fcolorbox{white}{minc_13}{\rule{0pt}{6pt}\rule{6pt}{0pt}} Painted~~
  \fcolorbox{white}{minc_14}{\rule{0pt}{6pt}\rule{6pt}{0pt}} Paper~~
  \fcolorbox{white}{minc_15}{\rule{0pt}{6pt}\rule{6pt}{0pt}} Plastic~~\\
  \fcolorbox{white}{minc_16}{\rule{0pt}{6pt}\rule{6pt}{0pt}} Polished Stone~~
  \fcolorbox{white}{minc_17}{\rule{0pt}{6pt}\rule{6pt}{0pt}} Skin~~
  \fcolorbox{white}{minc_18}{\rule{0pt}{6pt}\rule{6pt}{0pt}} Sky~~
  \fcolorbox{white}{minc_19}{\rule{0pt}{6pt}\rule{6pt}{0pt}} Stone~~
  \fcolorbox{white}{minc_20}{\rule{0pt}{6pt}\rule{6pt}{0pt}} Tile~~
  \fcolorbox{white}{minc_21}{\rule{0pt}{6pt}\rule{6pt}{0pt}} Wallpaper~~
  \fcolorbox{white}{minc_22}{\rule{0pt}{6pt}\rule{6pt}{0pt}} Water~~
  \fcolorbox{white}{minc_23}{\rule{0pt}{6pt}\rule{6pt}{0pt}} Wood~~ \\
  }
  \subfigure{%
    \includegraphics[width=.18\columnwidth]{figures/supplementary/000010868_given.jpg}
  }
  \subfigure{%
    \includegraphics[width=.18\columnwidth]{figures/supplementary/000010868_gt.png}
  }
  \subfigure{%
    \includegraphics[width=.18\columnwidth]{figures/supplementary/000010868_cnn.png}
  }
  \subfigure{%
    \includegraphics[width=.18\columnwidth]{figures/supplementary/000010868_gauss.png}
  }
  \subfigure{%
    \includegraphics[width=.18\columnwidth]{figures/supplementary/000010868_learnt.png}
  }\\[-2ex]
  \subfigure{%
    \includegraphics[width=.18\columnwidth]{figures/supplementary/000006011_given.jpg}
  }
  \subfigure{%
    \includegraphics[width=.18\columnwidth]{figures/supplementary/000006011_gt.png}
  }
  \subfigure{%
    \includegraphics[width=.18\columnwidth]{figures/supplementary/000006011_cnn.png}
  }
  \subfigure{%
    \includegraphics[width=.18\columnwidth]{figures/supplementary/000006011_gauss.png}
  }
  \subfigure{%
    \includegraphics[width=.18\columnwidth]{figures/supplementary/000006011_learnt.png}
  }\\[-2ex]
    \subfigure{%
    \includegraphics[width=.18\columnwidth]{figures/supplementary/000008553_given.jpg}
  }
  \subfigure{%
    \includegraphics[width=.18\columnwidth]{figures/supplementary/000008553_gt.png}
  }
  \subfigure{%
    \includegraphics[width=.18\columnwidth]{figures/supplementary/000008553_cnn.png}
  }
  \subfigure{%
    \includegraphics[width=.18\columnwidth]{figures/supplementary/000008553_gauss.png}
  }
  \subfigure{%
    \includegraphics[width=.18\columnwidth]{figures/supplementary/000008553_learnt.png}
  }\\[-2ex]
   \subfigure{%
    \includegraphics[width=.18\columnwidth]{figures/supplementary/000009188_given.jpg}
  }
  \subfigure{%
    \includegraphics[width=.18\columnwidth]{figures/supplementary/000009188_gt.png}
  }
  \subfigure{%
    \includegraphics[width=.18\columnwidth]{figures/supplementary/000009188_cnn.png}
  }
  \subfigure{%
    \includegraphics[width=.18\columnwidth]{figures/supplementary/000009188_gauss.png}
  }
  \subfigure{%
    \includegraphics[width=.18\columnwidth]{figures/supplementary/000009188_learnt.png}
  }\\[-2ex]
  \setcounter{subfigure}{0}
  \subfigure[Input]{%
    \includegraphics[width=.18\columnwidth]{figures/supplementary/000023570_given.jpg}
  }
  \subfigure[Ground Truth]{%
    \includegraphics[width=.18\columnwidth]{figures/supplementary/000023570_gt.png}
  }
  \subfigure[DeepLab]{%
    \includegraphics[width=.18\columnwidth]{figures/supplementary/000023570_cnn.png}
  }
  \subfigure[+GaussCRF]{%
    \includegraphics[width=.18\columnwidth]{figures/supplementary/000023570_gauss.png}
  }
  \subfigure[+LearnedCRF]{%
    \includegraphics[width=.18\columnwidth]{figures/supplementary/000023570_learnt.png}
  }
  \mycaption{Material Segmentation}{Example results of material segmentation.
  (c)~depicts the unary results before application of MF, (d)~after two steps of MF with Gaussian edge CRF potentials, (e)~after two steps of MF with learned edge CRF potentials.}
    \label{fig:material_visuals-app2}
\end{figure*}


\begin{table*}[h]
\tiny
  \centering
    \begin{tabular}{L{2.3cm} L{2.25cm} C{1.5cm} C{0.7cm} C{0.6cm} C{0.7cm} C{0.7cm} C{0.7cm} C{1.6cm} C{0.6cm} C{0.6cm} C{0.6cm}}
      \toprule
& & & & & \multicolumn{3}{c}{\textbf{Data Statistics}} & \multicolumn{4}{c}{\textbf{Training Protocol}} \\

\textbf{Experiment} & \textbf{Feature Types} & \textbf{Feature Scales} & \textbf{Filter Size} & \textbf{Filter Nbr.} & \textbf{Train}  & \textbf{Val.} & \textbf{Test} & \textbf{Loss Type} & \textbf{LR} & \textbf{Batch} & \textbf{Epochs} \\
      \midrule
      \multicolumn{2}{c}{\textbf{Single Bilateral Filter Applications}} & & & & & & & & & \\
      \textbf{2$\times$ Color Upsampling} & Position$_{1}$, Intensity (3D) & 0.13, 0.17 & 65 & 2 & 10581 & 1449 & 1456 & MSE & 1e-06 & 200 & 94.5\\
      \textbf{4$\times$ Color Upsampling} & Position$_{1}$, Intensity (3D) & 0.06, 0.17 & 65 & 2 & 10581 & 1449 & 1456 & MSE & 1e-06 & 200 & 94.5\\
      \textbf{8$\times$ Color Upsampling} & Position$_{1}$, Intensity (3D) & 0.03, 0.17 & 65 & 2 & 10581 & 1449 & 1456 & MSE & 1e-06 & 200 & 94.5\\
      \textbf{16$\times$ Color Upsampling} & Position$_{1}$, Intensity (3D) & 0.02, 0.17 & 65 & 2 & 10581 & 1449 & 1456 & MSE & 1e-06 & 200 & 94.5\\
      \textbf{Depth Upsampling} & Position$_{1}$, Color (5D) & 0.05, 0.02 & 665 & 2 & 795 & 100 & 654 & MSE & 1e-07 & 50 & 251.6\\
      \textbf{Mesh Denoising} & Isomap (4D) & 46.00 & 63 & 2 & 1000 & 200 & 500 & MSE & 100 & 10 & 100.0 \\
      \midrule
      \multicolumn{2}{c}{\textbf{DenseCRF Applications}} & & & & & & & & &\\
      \multicolumn{2}{l}{\textbf{Semantic Segmentation}} & & & & & & & & &\\
      \textbf{- 1step MF} & Position$_{1}$, Color (5D); Position$_{1}$ (2D) & 0.01, 0.34; 0.34  & 665; 19  & 2; 2 & 10581 & 1449 & 1456 & Logistic & 0.1 & 5 & 1.4 \\
      \textbf{- 2step MF} & Position$_{1}$, Color (5D); Position$_{1}$ (2D) & 0.01, 0.34; 0.34 & 665; 19 & 2; 2 & 10581 & 1449 & 1456 & Logistic & 0.1 & 5 & 1.4 \\
      \textbf{- \textit{loose} 2step MF} & Position$_{1}$, Color (5D); Position$_{1}$ (2D) & 0.01, 0.34; 0.34 & 665; 19 & 2; 2 &10581 & 1449 & 1456 & Logistic & 0.1 & 5 & +1.9  \\ \\
      \multicolumn{2}{l}{\textbf{Material Segmentation}} & & & & & & & & &\\
      \textbf{- 1step MF} & Position$_{2}$, Lab-Color (5D) & 5.00, 0.05, 0.30  & 665 & 2 & 928 & 150 & 1798 & Weighted Logistic & 1e-04 & 24 & 2.6 \\
      \textbf{- 2step MF} & Position$_{2}$, Lab-Color (5D) & 5.00, 0.05, 0.30 & 665 & 2 & 928 & 150 & 1798 & Weighted Logistic & 1e-04 & 12 & +0.7 \\
      \textbf{- \textit{loose} 2step MF} & Position$_{2}$, Lab-Color (5D) & 5.00, 0.05, 0.30 & 665 & 2 & 928 & 150 & 1798 & Weighted Logistic & 1e-04 & 12 & +0.2\\
      \midrule
      \multicolumn{2}{c}{\textbf{Neural Network Applications}} & & & & & & & & &\\
      \textbf{Tiles: CNN-9$\times$9} & - & - & 81 & 4 & 10000 & 1000 & 1000 & Logistic & 0.01 & 100 & 500.0 \\
      \textbf{Tiles: CNN-13$\times$13} & - & - & 169 & 6 & 10000 & 1000 & 1000 & Logistic & 0.01 & 100 & 500.0 \\
      \textbf{Tiles: CNN-17$\times$17} & - & - & 289 & 8 & 10000 & 1000 & 1000 & Logistic & 0.01 & 100 & 500.0 \\
      \textbf{Tiles: CNN-21$\times$21} & - & - & 441 & 10 & 10000 & 1000 & 1000 & Logistic & 0.01 & 100 & 500.0 \\
      \textbf{Tiles: BNN} & Position$_{1}$, Color (5D) & 0.05, 0.04 & 63 & 1 & 10000 & 1000 & 1000 & Logistic & 0.01 & 100 & 30.0 \\
      \textbf{LeNet} & - & - & 25 & 2 & 5490 & 1098 & 1647 & Logistic & 0.1 & 100 & 182.2 \\
      \textbf{Crop-LeNet} & - & - & 25 & 2 & 5490 & 1098 & 1647 & Logistic & 0.1 & 100 & 182.2 \\
      \textbf{BNN-LeNet} & Position$_{2}$ (2D) & 20.00 & 7 & 1 & 5490 & 1098 & 1647 & Logistic & 0.1 & 100 & 182.2 \\
      \textbf{DeepCNet} & - & - & 9 & 1 & 5490 & 1098 & 1647 & Logistic & 0.1 & 100 & 182.2 \\
      \textbf{Crop-DeepCNet} & - & - & 9 & 1 & 5490 & 1098 & 1647 & Logistic & 0.1 & 100 & 182.2 \\
      \textbf{BNN-DeepCNet} & Position$_{2}$ (2D) & 40.00  & 7 & 1 & 5490 & 1098 & 1647 & Logistic & 0.1 & 100 & 182.2 \\
      \bottomrule
      \\
    \end{tabular}
    \mycaption{Experiment Protocols} {Experiment protocols for the different experiments presented in this work. \textbf{Feature Types}:
    Feature spaces used for the bilateral convolutions. Position$_1$ corresponds to un-normalized pixel positions whereas Position$_2$ corresponds
    to pixel positions normalized to $[0,1]$ with respect to the given image. \textbf{Feature Scales}: Cross-validated scales for the features used.
     \textbf{Filter Size}: Number of elements in the filter that is being learned. \textbf{Filter Nbr.}: Half-width of the filter. \textbf{Train},
     \textbf{Val.} and \textbf{Test} corresponds to the number of train, validation and test images used in the experiment. \textbf{Loss Type}: Type
     of loss used for back-propagation. ``MSE'' corresponds to Euclidean mean squared error loss and ``Logistic'' corresponds to multinomial logistic
     loss. ``Weighted Logistic'' is the class-weighted multinomial logistic loss. We weighted the loss with inverse class probability for material
     segmentation task due to the small availability of training data with class imbalance. \textbf{LR}: Fixed learning rate used in stochastic gradient
     descent. \textbf{Batch}: Number of images used in one parameter update step. \textbf{Epochs}: Number of training epochs. In all the experiments,
     we used fixed momentum of 0.9 and weight decay of 0.0005 for stochastic gradient descent. ```Color Upsampling'' experiments in this Table corresponds
     to those performed on Pascal VOC12 dataset images. For all experiments using Pascal VOC12 images, we use extended
     training segmentation dataset available from~\cite{hariharan2011moredata}, and used standard validation and test splits
     from the main dataset~\cite{voc2012segmentation}.}
  \label{tbl:parameters}
\end{table*}

\clearpage

\section{Parameters and Additional Results for Video Propagation Networks}

In this Section, we present experiment protocols and additional qualitative results for experiments
on video object segmentation, semantic video segmentation and video color
propagation. Table~\ref{tbl:parameters_supp} shows the feature scales and other parameters used in different experiments.
Figures~\ref{fig:video_seg_pos_supp} show some qualitative results on video object segmentation
with some failure cases in Fig.~\ref{fig:video_seg_neg_supp}.
Figure~\ref{fig:semantic_visuals_supp} shows some qualitative results on semantic video segmentation and
Fig.~\ref{fig:color_visuals_supp} shows results on video color propagation.

\newcolumntype{L}[1]{>{\raggedright\let\newline\\\arraybackslash\hspace{0pt}}b{#1}}
\newcolumntype{C}[1]{>{\centering\let\newline\\\arraybackslash\hspace{0pt}}b{#1}}
\newcolumntype{R}[1]{>{\raggedleft\let\newline\\\arraybackslash\hspace{0pt}}b{#1}}

\begin{table*}[h]
\tiny
  \centering
    \begin{tabular}{L{3.0cm} L{2.4cm} L{2.8cm} L{2.8cm} C{0.5cm} C{1.0cm} L{1.2cm}}
      \toprule
\textbf{Experiment} & \textbf{Feature Type} & \textbf{Feature Scale-1, $\Lambda_a$} & \textbf{Feature Scale-2, $\Lambda_b$} & \textbf{$\alpha$} & \textbf{Input Frames} & \textbf{Loss Type} \\
      \midrule
      \textbf{Video Object Segmentation} & ($x,y,Y,Cb,Cr,t$) & (0.02,0.02,0.07,0.4,0.4,0.01) & (0.03,0.03,0.09,0.5,0.5,0.2) & 0.5 & 9 & Logistic\\
      \midrule
      \textbf{Semantic Video Segmentation} & & & & & \\
      \textbf{with CNN1~\cite{yu2015multi}-NoFlow} & ($x,y,R,G,B,t$) & (0.08,0.08,0.2,0.2,0.2,0.04) & (0.11,0.11,0.2,0.2,0.2,0.04) & 0.5 & 3 & Logistic \\
      \textbf{with CNN1~\cite{yu2015multi}-Flow} & ($x+u_x,y+u_y,R,G,B,t$) & (0.11,0.11,0.14,0.14,0.14,0.03) & (0.08,0.08,0.12,0.12,0.12,0.01) & 0.65 & 3 & Logistic\\
      \textbf{with CNN2~\cite{richter2016playing}-Flow} & ($x+u_x,y+u_y,R,G,B,t$) & (0.08,0.08,0.2,0.2,0.2,0.04) & (0.09,0.09,0.25,0.25,0.25,0.03) & 0.5 & 4 & Logistic\\
      \midrule
      \textbf{Video Color Propagation} & ($x,y,I,t$)  & (0.04,0.04,0.2,0.04) & No second kernel & 1 & 4 & MSE\\
      \bottomrule
      \\
    \end{tabular}
    \mycaption{Experiment Protocols} {Experiment protocols for the different experiments presented in this work. \textbf{Feature Types}:
    Feature spaces used for the bilateral convolutions, with position ($x,y$) and color
    ($R,G,B$ or $Y,Cb,Cr$) features $\in [0,255]$. $u_x$, $u_y$ denotes optical flow with respect
    to the present frame and $I$ denotes grayscale intensity.
    \textbf{Feature Scales ($\Lambda_a, \Lambda_b$)}: Cross-validated scales for the features used.
    \textbf{$\alpha$}: Exponential time decay for the input frames.
    \textbf{Input Frames}: Number of input frames for VPN.
    \textbf{Loss Type}: Type
     of loss used for back-propagation. ``MSE'' corresponds to Euclidean mean squared error loss and ``Logistic'' corresponds to multinomial logistic loss.}
  \label{tbl:parameters_supp}
\end{table*}

% \begin{figure}[th!]
% \begin{center}
%   \centerline{\includegraphics[width=\textwidth]{figures/video_seg_visuals_supp_small.pdf}}
%     \mycaption{Video Object Segmentation}
%     {Shown are the different frames in example videos with the corresponding
%     ground truth (GT) masks, predictions from BVS~\cite{marki2016bilateral},
%     OFL~\cite{tsaivideo}, VPN (VPN-Stage2) and VPN-DLab (VPN-DeepLab) models.}
%     \label{fig:video_seg_small_supp}
% \end{center}
% \vspace{-1.0cm}
% \end{figure}

\begin{figure}[th!]
\begin{center}
  \centerline{\includegraphics[width=0.7\textwidth]{figures/video_seg_visuals_supp_positive.pdf}}
    \mycaption{Video Object Segmentation}
    {Shown are the different frames in example videos with the corresponding
    ground truth (GT) masks, predictions from BVS~\cite{marki2016bilateral},
    OFL~\cite{tsaivideo}, VPN (VPN-Stage2) and VPN-DLab (VPN-DeepLab) models.}
    \label{fig:video_seg_pos_supp}
\end{center}
\vspace{-1.0cm}
\end{figure}

\begin{figure}[th!]
\begin{center}
  \centerline{\includegraphics[width=0.7\textwidth]{figures/video_seg_visuals_supp_negative.pdf}}
    \mycaption{Failure Cases for Video Object Segmentation}
    {Shown are the different frames in example videos with the corresponding
    ground truth (GT) masks, predictions from BVS~\cite{marki2016bilateral},
    OFL~\cite{tsaivideo}, VPN (VPN-Stage2) and VPN-DLab (VPN-DeepLab) models.}
    \label{fig:video_seg_neg_supp}
\end{center}
\vspace{-1.0cm}
\end{figure}

\begin{figure}[th!]
\begin{center}
  \centerline{\includegraphics[width=0.9\textwidth]{figures/supp_semantic_visual.pdf}}
    \mycaption{Semantic Video Segmentation}
    {Input video frames and the corresponding ground truth (GT)
    segmentation together with the predictions of CNN~\cite{yu2015multi} and with
    VPN-Flow.}
    \label{fig:semantic_visuals_supp}
\end{center}
\vspace{-0.7cm}
\end{figure}

\begin{figure}[th!]
\begin{center}
  \centerline{\includegraphics[width=\textwidth]{figures/colorization_visuals_supp.pdf}}
  \mycaption{Video Color Propagation}
  {Input grayscale video frames and corresponding ground-truth (GT) color images
  together with color predictions of Levin et al.~\cite{levin2004colorization} and VPN-Stage1 models.}
  \label{fig:color_visuals_supp}
\end{center}
\vspace{-0.7cm}
\end{figure}

\clearpage

\section{Additional Material for Bilateral Inception Networks}
\label{sec:binception-app}

In this section of the Appendix, we first discuss the use of approximate bilateral
filtering in BI modules (Sec.~\ref{sec:lattice}).
Later, we present some qualitative results using different models for the approach presented in
Chapter~\ref{chap:binception} (Sec.~\ref{sec:qualitative-app}).

\subsection{Approximate Bilateral Filtering}
\label{sec:lattice}

The bilateral inception module presented in Chapter~\ref{chap:binception} computes a matrix-vector
product between a Gaussian filter $K$ and a vector of activations $\bz_c$.
Bilateral filtering is an important operation and many algorithmic techniques have been
proposed to speed-up this operation~\cite{paris2006fast,adams2010fast,gastal2011domain}.
In the main paper we opted to implement what can be considered the
brute-force variant of explicitly constructing $K$ and then using BLAS to compute the
matrix-vector product. This resulted in a few millisecond operation.
The explicit way to compute is possible due to the
reduction to super-pixels, e.g., it would not work for DenseCRF variants
that operate on the full image resolution.

Here, we present experiments where we use the fast approximate bilateral filtering
algorithm of~\cite{adams2010fast}, which is also used in Chapter~\ref{chap:bnn}
for learning sparse high dimensional filters. This
choice allows for larger dimensions of matrix-vector multiplication. The reason for choosing
the explicit multiplication in Chapter~\ref{chap:binception} was that it was computationally faster.
For the small sizes of the involved matrices and vectors, the explicit computation is sufficient and we had no
GPU implementation of an approximate technique that matched this runtime. Also it
is conceptually easier and the gradient to the feature transformations ($\Lambda \mathbf{f}$) is
obtained using standard matrix calculus.

\subsubsection{Experiments}

We modified the existing segmentation architectures analogous to those in Chapter~\ref{chap:binception}.
The main difference is that, here, the inception modules use the lattice
approximation~\cite{adams2010fast} to compute the bilateral filtering.
Using the lattice approximation did not allow us to back-propagate through feature transformations ($\Lambda$)
and thus we used hand-specified feature scales as will be explained later.
Specifically, we take CNN architectures from the works
of~\cite{chen2014semantic,zheng2015conditional,bell2015minc} and insert the BI modules between
the spatial FC layers.
We use superpixels from~\cite{DollarICCV13edges}
for all the experiments with the lattice approximation. Experiments are
performed using Caffe neural network framework~\cite{jia2014caffe}.

\begin{table}
  \small
  \centering
  \begin{tabular}{p{5.5cm}>{\raggedright\arraybackslash}p{1.4cm}>{\centering\arraybackslash}p{2.2cm}}
    \toprule
		\textbf{Model} & \emph{IoU} & \emph{Runtime}(ms) \\
    \midrule

    %%%%%%%%%%%% Scores computed by us)%%%%%%%%%%%%
		\deeplablargefov & 68.9 & 145ms\\
    \midrule
    \bi{7}{2}-\bi{8}{10}& \textbf{73.8} & +600 \\
    \midrule
    \deeplablargefovcrf~\cite{chen2014semantic} & 72.7 & +830\\
    \deeplabmsclargefovcrf~\cite{chen2014semantic} & \textbf{73.6} & +880\\
    DeepLab-EdgeNet~\cite{chen2015semantic} & 71.7 & +30\\
    DeepLab-EdgeNet-CRF~\cite{chen2015semantic} & \textbf{73.6} & +860\\
  \bottomrule \\
  \end{tabular}
  \mycaption{Semantic Segmentation using the DeepLab model}
  {IoU scores on the Pascal VOC12 segmentation test dataset
  with different models and our modified inception model.
  Also shown are the corresponding runtimes in milliseconds. Runtimes
  also include superpixel computations (300 ms with Dollar superpixels~\cite{DollarICCV13edges})}
  \label{tab:largefovresults}
\end{table}

\paragraph{Semantic Segmentation}
The experiments in this section use the Pascal VOC12 segmentation dataset~\cite{voc2012segmentation} with 21 object classes and the images have a maximum resolution of 0.25 megapixels.
For all experiments on VOC12, we train using the extended training set of
10581 images collected by~\cite{hariharan2011moredata}.
We modified the \deeplab~network architecture of~\cite{chen2014semantic} and
the CRFasRNN architecture from~\cite{zheng2015conditional} which uses a CNN with
deconvolution layers followed by DenseCRF trained end-to-end.

\paragraph{DeepLab Model}\label{sec:deeplabmodel}
We experimented with the \bi{7}{2}-\bi{8}{10} inception model.
Results using the~\deeplab~model are summarized in Tab.~\ref{tab:largefovresults}.
Although we get similar improvements with inception modules as with the
explicit kernel computation, using lattice approximation is slower.

\begin{table}
  \small
  \centering
  \begin{tabular}{p{6.4cm}>{\raggedright\arraybackslash}p{1.8cm}>{\raggedright\arraybackslash}p{1.8cm}}
    \toprule
    \textbf{Model} & \emph{IoU (Val)} & \emph{IoU (Test)}\\
    \midrule
    %%%%%%%%%%%% Scores computed by us)%%%%%%%%%%%%
    CNN &  67.5 & - \\
    \deconv (CNN+Deconvolutions) & 69.8 & 72.0 \\
    \midrule
    \bi{3}{6}-\bi{4}{6}-\bi{7}{2}-\bi{8}{6}& 71.9 & - \\
    \bi{3}{6}-\bi{4}{6}-\bi{7}{2}-\bi{8}{6}-\gi{6}& 73.6 &  \href{http://host.robots.ox.ac.uk:8080/anonymous/VOTV5E.html}{\textbf{75.2}}\\
    \midrule
    \deconvcrf (CRF-RNN)~\cite{zheng2015conditional} & 73.0 & 74.7\\
    Context-CRF-RNN~\cite{yu2015multi} & ~~ - ~ & \textbf{75.3} \\
    \bottomrule \\
  \end{tabular}
  \mycaption{Semantic Segmentation using the CRFasRNN model}{IoU score corresponding to different models
  on Pascal VOC12 reduced validation / test segmentation dataset. The reduced validation set consists of 346 images
  as used in~\cite{zheng2015conditional} where we adapted the model from.}
  \label{tab:deconvresults-app}
\end{table}

\paragraph{CRFasRNN Model}\label{sec:deepinception}
We add BI modules after score-pool3, score-pool4, \fc{7} and \fc{8} $1\times1$ convolution layers
resulting in the \bi{3}{6}-\bi{4}{6}-\bi{7}{2}-\bi{8}{6}
model and also experimented with another variant where $BI_8$ is followed by another inception
module, G$(6)$, with 6 Gaussian kernels.
Note that here also we discarded both deconvolution and DenseCRF parts of the original model~\cite{zheng2015conditional}
and inserted the BI modules in the base CNN and found similar improvements compared to the inception modules with explicit
kernel computaion. See Tab.~\ref{tab:deconvresults-app} for results on the CRFasRNN model.

\paragraph{Material Segmentation}
Table~\ref{tab:mincresults-app} shows the results on the MINC dataset~\cite{bell2015minc}
obtained by modifying the AlexNet architecture with our inception modules. We observe
similar improvements as with explicit kernel construction.
For this model, we do not provide any learned setup due to very limited segment training
data. The weights to combine outputs in the bilateral inception layer are
found by validation on the validation set.

\begin{table}[t]
  \small
  \centering
  \begin{tabular}{p{3.5cm}>{\centering\arraybackslash}p{4.0cm}}
    \toprule
    \textbf{Model} & Class / Total accuracy\\
    \midrule

    %%%%%%%%%%%% Scores computed by us)%%%%%%%%%%%%
    AlexNet CNN & 55.3 / 58.9 \\
    \midrule
    \bi{7}{2}-\bi{8}{6}& 68.5 / 71.8 \\
    \bi{7}{2}-\bi{8}{6}-G$(6)$& 67.6 / 73.1 \\
    \midrule
    AlexNet-CRF & 65.5 / 71.0 \\
    \bottomrule \\
  \end{tabular}
  \mycaption{Material Segmentation using AlexNet}{Pixel accuracy of different models on
  the MINC material segmentation test dataset~\cite{bell2015minc}.}
  \label{tab:mincresults-app}
\end{table}

\paragraph{Scales of Bilateral Inception Modules}
\label{sec:scales}

Unlike the explicit kernel technique presented in the main text (Chapter~\ref{chap:binception}),
we didn't back-propagate through feature transformation ($\Lambda$)
using the approximate bilateral filter technique.
So, the feature scales are hand-specified and validated, which are as follows.
The optimal scale values for the \bi{7}{2}-\bi{8}{2} model are found by validation for the best performance which are
$\sigma_{xy}$ = (0.1, 0.1) for the spatial (XY) kernel and $\sigma_{rgbxy}$ = (0.1, 0.1, 0.1, 0.01, 0.01) for color and position (RGBXY)  kernel.
Next, as more kernels are added to \bi{8}{2}, we set scales to be $\alpha$*($\sigma_{xy}$, $\sigma_{rgbxy}$).
The value of $\alpha$ is chosen as  1, 0.5, 0.1, 0.05, 0.1, at uniform interval, for the \bi{8}{10} bilateral inception module.


\subsection{Qualitative Results}
\label{sec:qualitative-app}

In this section, we present more qualitative results obtained using the BI module with explicit
kernel computation technique presented in Chapter~\ref{chap:binception}. Results on the Pascal VOC12
dataset~\cite{voc2012segmentation} using the DeepLab-LargeFOV model are shown in Fig.~\ref{fig:semantic_visuals-app},
followed by the results on MINC dataset~\cite{bell2015minc}
in Fig.~\ref{fig:material_visuals-app} and on
Cityscapes dataset~\cite{Cordts2015Cvprw} in Fig.~\ref{fig:street_visuals-app}.


\definecolor{voc_1}{RGB}{0, 0, 0}
\definecolor{voc_2}{RGB}{128, 0, 0}
\definecolor{voc_3}{RGB}{0, 128, 0}
\definecolor{voc_4}{RGB}{128, 128, 0}
\definecolor{voc_5}{RGB}{0, 0, 128}
\definecolor{voc_6}{RGB}{128, 0, 128}
\definecolor{voc_7}{RGB}{0, 128, 128}
\definecolor{voc_8}{RGB}{128, 128, 128}
\definecolor{voc_9}{RGB}{64, 0, 0}
\definecolor{voc_10}{RGB}{192, 0, 0}
\definecolor{voc_11}{RGB}{64, 128, 0}
\definecolor{voc_12}{RGB}{192, 128, 0}
\definecolor{voc_13}{RGB}{64, 0, 128}
\definecolor{voc_14}{RGB}{192, 0, 128}
\definecolor{voc_15}{RGB}{64, 128, 128}
\definecolor{voc_16}{RGB}{192, 128, 128}
\definecolor{voc_17}{RGB}{0, 64, 0}
\definecolor{voc_18}{RGB}{128, 64, 0}
\definecolor{voc_19}{RGB}{0, 192, 0}
\definecolor{voc_20}{RGB}{128, 192, 0}
\definecolor{voc_21}{RGB}{0, 64, 128}
\definecolor{voc_22}{RGB}{128, 64, 128}

\begin{figure*}[!ht]
  \small
  \centering
  \fcolorbox{white}{voc_1}{\rule{0pt}{4pt}\rule{4pt}{0pt}} Background~~
  \fcolorbox{white}{voc_2}{\rule{0pt}{4pt}\rule{4pt}{0pt}} Aeroplane~~
  \fcolorbox{white}{voc_3}{\rule{0pt}{4pt}\rule{4pt}{0pt}} Bicycle~~
  \fcolorbox{white}{voc_4}{\rule{0pt}{4pt}\rule{4pt}{0pt}} Bird~~
  \fcolorbox{white}{voc_5}{\rule{0pt}{4pt}\rule{4pt}{0pt}} Boat~~
  \fcolorbox{white}{voc_6}{\rule{0pt}{4pt}\rule{4pt}{0pt}} Bottle~~
  \fcolorbox{white}{voc_7}{\rule{0pt}{4pt}\rule{4pt}{0pt}} Bus~~
  \fcolorbox{white}{voc_8}{\rule{0pt}{4pt}\rule{4pt}{0pt}} Car~~\\
  \fcolorbox{white}{voc_9}{\rule{0pt}{4pt}\rule{4pt}{0pt}} Cat~~
  \fcolorbox{white}{voc_10}{\rule{0pt}{4pt}\rule{4pt}{0pt}} Chair~~
  \fcolorbox{white}{voc_11}{\rule{0pt}{4pt}\rule{4pt}{0pt}} Cow~~
  \fcolorbox{white}{voc_12}{\rule{0pt}{4pt}\rule{4pt}{0pt}} Dining Table~~
  \fcolorbox{white}{voc_13}{\rule{0pt}{4pt}\rule{4pt}{0pt}} Dog~~
  \fcolorbox{white}{voc_14}{\rule{0pt}{4pt}\rule{4pt}{0pt}} Horse~~
  \fcolorbox{white}{voc_15}{\rule{0pt}{4pt}\rule{4pt}{0pt}} Motorbike~~
  \fcolorbox{white}{voc_16}{\rule{0pt}{4pt}\rule{4pt}{0pt}} Person~~\\
  \fcolorbox{white}{voc_17}{\rule{0pt}{4pt}\rule{4pt}{0pt}} Potted Plant~~
  \fcolorbox{white}{voc_18}{\rule{0pt}{4pt}\rule{4pt}{0pt}} Sheep~~
  \fcolorbox{white}{voc_19}{\rule{0pt}{4pt}\rule{4pt}{0pt}} Sofa~~
  \fcolorbox{white}{voc_20}{\rule{0pt}{4pt}\rule{4pt}{0pt}} Train~~
  \fcolorbox{white}{voc_21}{\rule{0pt}{4pt}\rule{4pt}{0pt}} TV monitor~~\\


  \subfigure{%
    \includegraphics[width=.15\columnwidth]{figures/supplementary/2008_001308_given.png}
  }
  \subfigure{%
    \includegraphics[width=.15\columnwidth]{figures/supplementary/2008_001308_sp.png}
  }
  \subfigure{%
    \includegraphics[width=.15\columnwidth]{figures/supplementary/2008_001308_gt.png}
  }
  \subfigure{%
    \includegraphics[width=.15\columnwidth]{figures/supplementary/2008_001308_cnn.png}
  }
  \subfigure{%
    \includegraphics[width=.15\columnwidth]{figures/supplementary/2008_001308_crf.png}
  }
  \subfigure{%
    \includegraphics[width=.15\columnwidth]{figures/supplementary/2008_001308_ours.png}
  }\\[-2ex]


  \subfigure{%
    \includegraphics[width=.15\columnwidth]{figures/supplementary/2008_001821_given.png}
  }
  \subfigure{%
    \includegraphics[width=.15\columnwidth]{figures/supplementary/2008_001821_sp.png}
  }
  \subfigure{%
    \includegraphics[width=.15\columnwidth]{figures/supplementary/2008_001821_gt.png}
  }
  \subfigure{%
    \includegraphics[width=.15\columnwidth]{figures/supplementary/2008_001821_cnn.png}
  }
  \subfigure{%
    \includegraphics[width=.15\columnwidth]{figures/supplementary/2008_001821_crf.png}
  }
  \subfigure{%
    \includegraphics[width=.15\columnwidth]{figures/supplementary/2008_001821_ours.png}
  }\\[-2ex]



  \subfigure{%
    \includegraphics[width=.15\columnwidth]{figures/supplementary/2008_004612_given.png}
  }
  \subfigure{%
    \includegraphics[width=.15\columnwidth]{figures/supplementary/2008_004612_sp.png}
  }
  \subfigure{%
    \includegraphics[width=.15\columnwidth]{figures/supplementary/2008_004612_gt.png}
  }
  \subfigure{%
    \includegraphics[width=.15\columnwidth]{figures/supplementary/2008_004612_cnn.png}
  }
  \subfigure{%
    \includegraphics[width=.15\columnwidth]{figures/supplementary/2008_004612_crf.png}
  }
  \subfigure{%
    \includegraphics[width=.15\columnwidth]{figures/supplementary/2008_004612_ours.png}
  }\\[-2ex]


  \subfigure{%
    \includegraphics[width=.15\columnwidth]{figures/supplementary/2009_001008_given.png}
  }
  \subfigure{%
    \includegraphics[width=.15\columnwidth]{figures/supplementary/2009_001008_sp.png}
  }
  \subfigure{%
    \includegraphics[width=.15\columnwidth]{figures/supplementary/2009_001008_gt.png}
  }
  \subfigure{%
    \includegraphics[width=.15\columnwidth]{figures/supplementary/2009_001008_cnn.png}
  }
  \subfigure{%
    \includegraphics[width=.15\columnwidth]{figures/supplementary/2009_001008_crf.png}
  }
  \subfigure{%
    \includegraphics[width=.15\columnwidth]{figures/supplementary/2009_001008_ours.png}
  }\\[-2ex]




  \subfigure{%
    \includegraphics[width=.15\columnwidth]{figures/supplementary/2009_004497_given.png}
  }
  \subfigure{%
    \includegraphics[width=.15\columnwidth]{figures/supplementary/2009_004497_sp.png}
  }
  \subfigure{%
    \includegraphics[width=.15\columnwidth]{figures/supplementary/2009_004497_gt.png}
  }
  \subfigure{%
    \includegraphics[width=.15\columnwidth]{figures/supplementary/2009_004497_cnn.png}
  }
  \subfigure{%
    \includegraphics[width=.15\columnwidth]{figures/supplementary/2009_004497_crf.png}
  }
  \subfigure{%
    \includegraphics[width=.15\columnwidth]{figures/supplementary/2009_004497_ours.png}
  }\\[-2ex]



  \setcounter{subfigure}{0}
  \subfigure[\scriptsize Input]{%
    \includegraphics[width=.15\columnwidth]{figures/supplementary/2010_001327_given.png}
  }
  \subfigure[\scriptsize Superpixels]{%
    \includegraphics[width=.15\columnwidth]{figures/supplementary/2010_001327_sp.png}
  }
  \subfigure[\scriptsize GT]{%
    \includegraphics[width=.15\columnwidth]{figures/supplementary/2010_001327_gt.png}
  }
  \subfigure[\scriptsize Deeplab]{%
    \includegraphics[width=.15\columnwidth]{figures/supplementary/2010_001327_cnn.png}
  }
  \subfigure[\scriptsize +DenseCRF]{%
    \includegraphics[width=.15\columnwidth]{figures/supplementary/2010_001327_crf.png}
  }
  \subfigure[\scriptsize Using BI]{%
    \includegraphics[width=.15\columnwidth]{figures/supplementary/2010_001327_ours.png}
  }
  \mycaption{Semantic Segmentation}{Example results of semantic segmentation
  on the Pascal VOC12 dataset.
  (d)~depicts the DeepLab CNN result, (e)~CNN + 10 steps of mean-field inference,
  (f~result obtained with bilateral inception (BI) modules (\bi{6}{2}+\bi{7}{6}) between \fc~layers.}
  \label{fig:semantic_visuals-app}
\end{figure*}


\definecolor{minc_1}{HTML}{771111}
\definecolor{minc_2}{HTML}{CAC690}
\definecolor{minc_3}{HTML}{EEEEEE}
\definecolor{minc_4}{HTML}{7C8FA6}
\definecolor{minc_5}{HTML}{597D31}
\definecolor{minc_6}{HTML}{104410}
\definecolor{minc_7}{HTML}{BB819C}
\definecolor{minc_8}{HTML}{D0CE48}
\definecolor{minc_9}{HTML}{622745}
\definecolor{minc_10}{HTML}{666666}
\definecolor{minc_11}{HTML}{D54A31}
\definecolor{minc_12}{HTML}{101044}
\definecolor{minc_13}{HTML}{444126}
\definecolor{minc_14}{HTML}{75D646}
\definecolor{minc_15}{HTML}{DD4348}
\definecolor{minc_16}{HTML}{5C8577}
\definecolor{minc_17}{HTML}{C78472}
\definecolor{minc_18}{HTML}{75D6D0}
\definecolor{minc_19}{HTML}{5B4586}
\definecolor{minc_20}{HTML}{C04393}
\definecolor{minc_21}{HTML}{D69948}
\definecolor{minc_22}{HTML}{7370D8}
\definecolor{minc_23}{HTML}{7A3622}
\definecolor{minc_24}{HTML}{000000}

\begin{figure*}[!ht]
  \small % scriptsize
  \centering
  \fcolorbox{white}{minc_1}{\rule{0pt}{4pt}\rule{4pt}{0pt}} Brick~~
  \fcolorbox{white}{minc_2}{\rule{0pt}{4pt}\rule{4pt}{0pt}} Carpet~~
  \fcolorbox{white}{minc_3}{\rule{0pt}{4pt}\rule{4pt}{0pt}} Ceramic~~
  \fcolorbox{white}{minc_4}{\rule{0pt}{4pt}\rule{4pt}{0pt}} Fabric~~
  \fcolorbox{white}{minc_5}{\rule{0pt}{4pt}\rule{4pt}{0pt}} Foliage~~
  \fcolorbox{white}{minc_6}{\rule{0pt}{4pt}\rule{4pt}{0pt}} Food~~
  \fcolorbox{white}{minc_7}{\rule{0pt}{4pt}\rule{4pt}{0pt}} Glass~~
  \fcolorbox{white}{minc_8}{\rule{0pt}{4pt}\rule{4pt}{0pt}} Hair~~\\
  \fcolorbox{white}{minc_9}{\rule{0pt}{4pt}\rule{4pt}{0pt}} Leather~~
  \fcolorbox{white}{minc_10}{\rule{0pt}{4pt}\rule{4pt}{0pt}} Metal~~
  \fcolorbox{white}{minc_11}{\rule{0pt}{4pt}\rule{4pt}{0pt}} Mirror~~
  \fcolorbox{white}{minc_12}{\rule{0pt}{4pt}\rule{4pt}{0pt}} Other~~
  \fcolorbox{white}{minc_13}{\rule{0pt}{4pt}\rule{4pt}{0pt}} Painted~~
  \fcolorbox{white}{minc_14}{\rule{0pt}{4pt}\rule{4pt}{0pt}} Paper~~
  \fcolorbox{white}{minc_15}{\rule{0pt}{4pt}\rule{4pt}{0pt}} Plastic~~\\
  \fcolorbox{white}{minc_16}{\rule{0pt}{4pt}\rule{4pt}{0pt}} Polished Stone~~
  \fcolorbox{white}{minc_17}{\rule{0pt}{4pt}\rule{4pt}{0pt}} Skin~~
  \fcolorbox{white}{minc_18}{\rule{0pt}{4pt}\rule{4pt}{0pt}} Sky~~
  \fcolorbox{white}{minc_19}{\rule{0pt}{4pt}\rule{4pt}{0pt}} Stone~~
  \fcolorbox{white}{minc_20}{\rule{0pt}{4pt}\rule{4pt}{0pt}} Tile~~
  \fcolorbox{white}{minc_21}{\rule{0pt}{4pt}\rule{4pt}{0pt}} Wallpaper~~
  \fcolorbox{white}{minc_22}{\rule{0pt}{4pt}\rule{4pt}{0pt}} Water~~
  \fcolorbox{white}{minc_23}{\rule{0pt}{4pt}\rule{4pt}{0pt}} Wood~~\\
  \subfigure{%
    \includegraphics[width=.15\columnwidth]{figures/supplementary/000008468_given.png}
  }
  \subfigure{%
    \includegraphics[width=.15\columnwidth]{figures/supplementary/000008468_sp.png}
  }
  \subfigure{%
    \includegraphics[width=.15\columnwidth]{figures/supplementary/000008468_gt.png}
  }
  \subfigure{%
    \includegraphics[width=.15\columnwidth]{figures/supplementary/000008468_cnn.png}
  }
  \subfigure{%
    \includegraphics[width=.15\columnwidth]{figures/supplementary/000008468_crf.png}
  }
  \subfigure{%
    \includegraphics[width=.15\columnwidth]{figures/supplementary/000008468_ours.png}
  }\\[-2ex]

  \subfigure{%
    \includegraphics[width=.15\columnwidth]{figures/supplementary/000009053_given.png}
  }
  \subfigure{%
    \includegraphics[width=.15\columnwidth]{figures/supplementary/000009053_sp.png}
  }
  \subfigure{%
    \includegraphics[width=.15\columnwidth]{figures/supplementary/000009053_gt.png}
  }
  \subfigure{%
    \includegraphics[width=.15\columnwidth]{figures/supplementary/000009053_cnn.png}
  }
  \subfigure{%
    \includegraphics[width=.15\columnwidth]{figures/supplementary/000009053_crf.png}
  }
  \subfigure{%
    \includegraphics[width=.15\columnwidth]{figures/supplementary/000009053_ours.png}
  }\\[-2ex]




  \subfigure{%
    \includegraphics[width=.15\columnwidth]{figures/supplementary/000014977_given.png}
  }
  \subfigure{%
    \includegraphics[width=.15\columnwidth]{figures/supplementary/000014977_sp.png}
  }
  \subfigure{%
    \includegraphics[width=.15\columnwidth]{figures/supplementary/000014977_gt.png}
  }
  \subfigure{%
    \includegraphics[width=.15\columnwidth]{figures/supplementary/000014977_cnn.png}
  }
  \subfigure{%
    \includegraphics[width=.15\columnwidth]{figures/supplementary/000014977_crf.png}
  }
  \subfigure{%
    \includegraphics[width=.15\columnwidth]{figures/supplementary/000014977_ours.png}
  }\\[-2ex]


  \subfigure{%
    \includegraphics[width=.15\columnwidth]{figures/supplementary/000022922_given.png}
  }
  \subfigure{%
    \includegraphics[width=.15\columnwidth]{figures/supplementary/000022922_sp.png}
  }
  \subfigure{%
    \includegraphics[width=.15\columnwidth]{figures/supplementary/000022922_gt.png}
  }
  \subfigure{%
    \includegraphics[width=.15\columnwidth]{figures/supplementary/000022922_cnn.png}
  }
  \subfigure{%
    \includegraphics[width=.15\columnwidth]{figures/supplementary/000022922_crf.png}
  }
  \subfigure{%
    \includegraphics[width=.15\columnwidth]{figures/supplementary/000022922_ours.png}
  }\\[-2ex]


  \subfigure{%
    \includegraphics[width=.15\columnwidth]{figures/supplementary/000025711_given.png}
  }
  \subfigure{%
    \includegraphics[width=.15\columnwidth]{figures/supplementary/000025711_sp.png}
  }
  \subfigure{%
    \includegraphics[width=.15\columnwidth]{figures/supplementary/000025711_gt.png}
  }
  \subfigure{%
    \includegraphics[width=.15\columnwidth]{figures/supplementary/000025711_cnn.png}
  }
  \subfigure{%
    \includegraphics[width=.15\columnwidth]{figures/supplementary/000025711_crf.png}
  }
  \subfigure{%
    \includegraphics[width=.15\columnwidth]{figures/supplementary/000025711_ours.png}
  }\\[-2ex]


  \subfigure{%
    \includegraphics[width=.15\columnwidth]{figures/supplementary/000034473_given.png}
  }
  \subfigure{%
    \includegraphics[width=.15\columnwidth]{figures/supplementary/000034473_sp.png}
  }
  \subfigure{%
    \includegraphics[width=.15\columnwidth]{figures/supplementary/000034473_gt.png}
  }
  \subfigure{%
    \includegraphics[width=.15\columnwidth]{figures/supplementary/000034473_cnn.png}
  }
  \subfigure{%
    \includegraphics[width=.15\columnwidth]{figures/supplementary/000034473_crf.png}
  }
  \subfigure{%
    \includegraphics[width=.15\columnwidth]{figures/supplementary/000034473_ours.png}
  }\\[-2ex]


  \subfigure{%
    \includegraphics[width=.15\columnwidth]{figures/supplementary/000035463_given.png}
  }
  \subfigure{%
    \includegraphics[width=.15\columnwidth]{figures/supplementary/000035463_sp.png}
  }
  \subfigure{%
    \includegraphics[width=.15\columnwidth]{figures/supplementary/000035463_gt.png}
  }
  \subfigure{%
    \includegraphics[width=.15\columnwidth]{figures/supplementary/000035463_cnn.png}
  }
  \subfigure{%
    \includegraphics[width=.15\columnwidth]{figures/supplementary/000035463_crf.png}
  }
  \subfigure{%
    \includegraphics[width=.15\columnwidth]{figures/supplementary/000035463_ours.png}
  }\\[-2ex]


  \setcounter{subfigure}{0}
  \subfigure[\scriptsize Input]{%
    \includegraphics[width=.15\columnwidth]{figures/supplementary/000035993_given.png}
  }
  \subfigure[\scriptsize Superpixels]{%
    \includegraphics[width=.15\columnwidth]{figures/supplementary/000035993_sp.png}
  }
  \subfigure[\scriptsize GT]{%
    \includegraphics[width=.15\columnwidth]{figures/supplementary/000035993_gt.png}
  }
  \subfigure[\scriptsize AlexNet]{%
    \includegraphics[width=.15\columnwidth]{figures/supplementary/000035993_cnn.png}
  }
  \subfigure[\scriptsize +DenseCRF]{%
    \includegraphics[width=.15\columnwidth]{figures/supplementary/000035993_crf.png}
  }
  \subfigure[\scriptsize Using BI]{%
    \includegraphics[width=.15\columnwidth]{figures/supplementary/000035993_ours.png}
  }
  \mycaption{Material Segmentation}{Example results of material segmentation.
  (d)~depicts the AlexNet CNN result, (e)~CNN + 10 steps of mean-field inference,
  (f)~result obtained with bilateral inception (BI) modules (\bi{7}{2}+\bi{8}{6}) between
  \fc~layers.}
\label{fig:material_visuals-app}
\end{figure*}


\definecolor{city_1}{RGB}{128, 64, 128}
\definecolor{city_2}{RGB}{244, 35, 232}
\definecolor{city_3}{RGB}{70, 70, 70}
\definecolor{city_4}{RGB}{102, 102, 156}
\definecolor{city_5}{RGB}{190, 153, 153}
\definecolor{city_6}{RGB}{153, 153, 153}
\definecolor{city_7}{RGB}{250, 170, 30}
\definecolor{city_8}{RGB}{220, 220, 0}
\definecolor{city_9}{RGB}{107, 142, 35}
\definecolor{city_10}{RGB}{152, 251, 152}
\definecolor{city_11}{RGB}{70, 130, 180}
\definecolor{city_12}{RGB}{220, 20, 60}
\definecolor{city_13}{RGB}{255, 0, 0}
\definecolor{city_14}{RGB}{0, 0, 142}
\definecolor{city_15}{RGB}{0, 0, 70}
\definecolor{city_16}{RGB}{0, 60, 100}
\definecolor{city_17}{RGB}{0, 80, 100}
\definecolor{city_18}{RGB}{0, 0, 230}
\definecolor{city_19}{RGB}{119, 11, 32}
\begin{figure*}[!ht]
  \small % scriptsize
  \centering


  \subfigure{%
    \includegraphics[width=.18\columnwidth]{figures/supplementary/frankfurt00000_016005_given.png}
  }
  \subfigure{%
    \includegraphics[width=.18\columnwidth]{figures/supplementary/frankfurt00000_016005_sp.png}
  }
  \subfigure{%
    \includegraphics[width=.18\columnwidth]{figures/supplementary/frankfurt00000_016005_gt.png}
  }
  \subfigure{%
    \includegraphics[width=.18\columnwidth]{figures/supplementary/frankfurt00000_016005_cnn.png}
  }
  \subfigure{%
    \includegraphics[width=.18\columnwidth]{figures/supplementary/frankfurt00000_016005_ours.png}
  }\\[-2ex]

  \subfigure{%
    \includegraphics[width=.18\columnwidth]{figures/supplementary/frankfurt00000_004617_given.png}
  }
  \subfigure{%
    \includegraphics[width=.18\columnwidth]{figures/supplementary/frankfurt00000_004617_sp.png}
  }
  \subfigure{%
    \includegraphics[width=.18\columnwidth]{figures/supplementary/frankfurt00000_004617_gt.png}
  }
  \subfigure{%
    \includegraphics[width=.18\columnwidth]{figures/supplementary/frankfurt00000_004617_cnn.png}
  }
  \subfigure{%
    \includegraphics[width=.18\columnwidth]{figures/supplementary/frankfurt00000_004617_ours.png}
  }\\[-2ex]

  \subfigure{%
    \includegraphics[width=.18\columnwidth]{figures/supplementary/frankfurt00000_020880_given.png}
  }
  \subfigure{%
    \includegraphics[width=.18\columnwidth]{figures/supplementary/frankfurt00000_020880_sp.png}
  }
  \subfigure{%
    \includegraphics[width=.18\columnwidth]{figures/supplementary/frankfurt00000_020880_gt.png}
  }
  \subfigure{%
    \includegraphics[width=.18\columnwidth]{figures/supplementary/frankfurt00000_020880_cnn.png}
  }
  \subfigure{%
    \includegraphics[width=.18\columnwidth]{figures/supplementary/frankfurt00000_020880_ours.png}
  }\\[-2ex]



  \subfigure{%
    \includegraphics[width=.18\columnwidth]{figures/supplementary/frankfurt00001_007285_given.png}
  }
  \subfigure{%
    \includegraphics[width=.18\columnwidth]{figures/supplementary/frankfurt00001_007285_sp.png}
  }
  \subfigure{%
    \includegraphics[width=.18\columnwidth]{figures/supplementary/frankfurt00001_007285_gt.png}
  }
  \subfigure{%
    \includegraphics[width=.18\columnwidth]{figures/supplementary/frankfurt00001_007285_cnn.png}
  }
  \subfigure{%
    \includegraphics[width=.18\columnwidth]{figures/supplementary/frankfurt00001_007285_ours.png}
  }\\[-2ex]


  \subfigure{%
    \includegraphics[width=.18\columnwidth]{figures/supplementary/frankfurt00001_059789_given.png}
  }
  \subfigure{%
    \includegraphics[width=.18\columnwidth]{figures/supplementary/frankfurt00001_059789_sp.png}
  }
  \subfigure{%
    \includegraphics[width=.18\columnwidth]{figures/supplementary/frankfurt00001_059789_gt.png}
  }
  \subfigure{%
    \includegraphics[width=.18\columnwidth]{figures/supplementary/frankfurt00001_059789_cnn.png}
  }
  \subfigure{%
    \includegraphics[width=.18\columnwidth]{figures/supplementary/frankfurt00001_059789_ours.png}
  }\\[-2ex]


  \subfigure{%
    \includegraphics[width=.18\columnwidth]{figures/supplementary/frankfurt00001_068208_given.png}
  }
  \subfigure{%
    \includegraphics[width=.18\columnwidth]{figures/supplementary/frankfurt00001_068208_sp.png}
  }
  \subfigure{%
    \includegraphics[width=.18\columnwidth]{figures/supplementary/frankfurt00001_068208_gt.png}
  }
  \subfigure{%
    \includegraphics[width=.18\columnwidth]{figures/supplementary/frankfurt00001_068208_cnn.png}
  }
  \subfigure{%
    \includegraphics[width=.18\columnwidth]{figures/supplementary/frankfurt00001_068208_ours.png}
  }\\[-2ex]

  \subfigure{%
    \includegraphics[width=.18\columnwidth]{figures/supplementary/frankfurt00001_082466_given.png}
  }
  \subfigure{%
    \includegraphics[width=.18\columnwidth]{figures/supplementary/frankfurt00001_082466_sp.png}
  }
  \subfigure{%
    \includegraphics[width=.18\columnwidth]{figures/supplementary/frankfurt00001_082466_gt.png}
  }
  \subfigure{%
    \includegraphics[width=.18\columnwidth]{figures/supplementary/frankfurt00001_082466_cnn.png}
  }
  \subfigure{%
    \includegraphics[width=.18\columnwidth]{figures/supplementary/frankfurt00001_082466_ours.png}
  }\\[-2ex]

  \subfigure{%
    \includegraphics[width=.18\columnwidth]{figures/supplementary/lindau00033_000019_given.png}
  }
  \subfigure{%
    \includegraphics[width=.18\columnwidth]{figures/supplementary/lindau00033_000019_sp.png}
  }
  \subfigure{%
    \includegraphics[width=.18\columnwidth]{figures/supplementary/lindau00033_000019_gt.png}
  }
  \subfigure{%
    \includegraphics[width=.18\columnwidth]{figures/supplementary/lindau00033_000019_cnn.png}
  }
  \subfigure{%
    \includegraphics[width=.18\columnwidth]{figures/supplementary/lindau00033_000019_ours.png}
  }\\[-2ex]

  \subfigure{%
    \includegraphics[width=.18\columnwidth]{figures/supplementary/lindau00052_000019_given.png}
  }
  \subfigure{%
    \includegraphics[width=.18\columnwidth]{figures/supplementary/lindau00052_000019_sp.png}
  }
  \subfigure{%
    \includegraphics[width=.18\columnwidth]{figures/supplementary/lindau00052_000019_gt.png}
  }
  \subfigure{%
    \includegraphics[width=.18\columnwidth]{figures/supplementary/lindau00052_000019_cnn.png}
  }
  \subfigure{%
    \includegraphics[width=.18\columnwidth]{figures/supplementary/lindau00052_000019_ours.png}
  }\\[-2ex]




  \subfigure{%
    \includegraphics[width=.18\columnwidth]{figures/supplementary/lindau00027_000019_given.png}
  }
  \subfigure{%
    \includegraphics[width=.18\columnwidth]{figures/supplementary/lindau00027_000019_sp.png}
  }
  \subfigure{%
    \includegraphics[width=.18\columnwidth]{figures/supplementary/lindau00027_000019_gt.png}
  }
  \subfigure{%
    \includegraphics[width=.18\columnwidth]{figures/supplementary/lindau00027_000019_cnn.png}
  }
  \subfigure{%
    \includegraphics[width=.18\columnwidth]{figures/supplementary/lindau00027_000019_ours.png}
  }\\[-2ex]



  \setcounter{subfigure}{0}
  \subfigure[\scriptsize Input]{%
    \includegraphics[width=.18\columnwidth]{figures/supplementary/lindau00029_000019_given.png}
  }
  \subfigure[\scriptsize Superpixels]{%
    \includegraphics[width=.18\columnwidth]{figures/supplementary/lindau00029_000019_sp.png}
  }
  \subfigure[\scriptsize GT]{%
    \includegraphics[width=.18\columnwidth]{figures/supplementary/lindau00029_000019_gt.png}
  }
  \subfigure[\scriptsize Deeplab]{%
    \includegraphics[width=.18\columnwidth]{figures/supplementary/lindau00029_000019_cnn.png}
  }
  \subfigure[\scriptsize Using BI]{%
    \includegraphics[width=.18\columnwidth]{figures/supplementary/lindau00029_000019_ours.png}
  }%\\[-2ex]

  \mycaption{Street Scene Segmentation}{Example results of street scene segmentation.
  (d)~depicts the DeepLab results, (e)~result obtained by adding bilateral inception (BI) modules (\bi{6}{2}+\bi{7}{6}) between \fc~layers.}
\label{fig:street_visuals-app}
\end{figure*}



\end{document}
