
We implement our semantics in the PICCO compiler.
Our implementation is available at \url{https://github.com/applied-crypto-lab/formal-picco} and we plan to merge it into the main PICCO branch as a submitted patch. 
The main modifications to PICCO revolve around a {\em conditional code block tracking} scheme and resolution mechanism for private variables, support for pointer operations and computation within private-conditioned branches, and the addition of tracking structures for pointers based on our semantics in Section~\ref{Sec: Semantics}.
PICCO uses single-statement resolution to manage modifications to private variables made within a private-conditioned branch, which can prove to be more costly when a single variable is modified multiple times within a private-conditioned branch, as shown in the example in Figure~\ref{Fig: resolution cost ex}. 
Here, we show how PICCO conditionally updates the true value of a variable after each statement, resulting in a more costly operation for each statement, as obtaining the true value requires communication between the various computational parties (i.e. this is a distributed secure computation). 
In \piccoC, we provide a conditional code block tracking scheme, where we store the original value for each modified variable before the execution of the \TT{then} branch, execute the \TT{then} branch as normal, store the updated values, restore the original values, execute the \TT{else} branch as normal, and perform resolution of all modified variables once at the end of both branches.
The cost of executing each statement remains the same, however, in \piccoC, the communication overhead is greatly reduced due to only needing to communicate between computational parties to resolve a single variable once at the end of both branches. 
In Figures~\ref{Fig: resolution cost ex PICCO} and~\ref{Fig: resolution cost ex \piccoC}, this amounts to six fewer times the program needs to communicate between computational parties to resolve. 


\begin{figure*}\footnotesize
\hspace{0.3cm} \begin{tabular}{l l l}
\begin{subfigure}{0.21\textwidth}
\begin{lstlisting}
private int c, 
		a=1,b=2;
if(a < b) {		
	c = a; 
	a = a + b;
	c = c * b; 
	a = c + a; 
} else {						
	c = b; 
	a = a - b;
	c = c * a; 
	a = c - a; 
}
\end{lstlisting}
\caption{Example code.}
\label{Fig: resolution cost ex code}
\end{subfigure} &
\-\ 
\begin{subfigure}{0.36\textwidth}
\begin{lstlisting}[emph={[2]res,c_e,c_t},emphstyle={[2]\color{blue}}]
private int c,a=1,b=2;
private int res=a<b;
{
c=(res$\cdot$a)$+$((1-res)$\cdot$c); 
a=(res$\cdot$(a+b))$+$((1-res)$\cdot$a);
c=(res$\cdot$(c*b))$+$((1-res)$\cdot$c); 
a=(res$\cdot$(c+a))$+$((1-res)$\cdot$a);	
}	
{c=((1-res)$\cdot$b)$+$(res$\cdot$c);
a=((1-res)$\cdot$(a-b))$+$(res$\cdot$a);
c=((1-res)$\cdot$(c*a))$+$(res$\cdot$c);
a=((1-res)$\cdot$(c-a))$+$(res$\cdot$a);
}
\end{lstlisting}
\caption{How PICCO executes part a.}
\label{Fig: resolution cost ex PICCO}
\end{subfigure} &
\-\ 
\begin{subfigure}{0.37\textwidth}
\begin{lstlisting}[emph={[2]res,c_e,c_t,a_e,a_t},emphstyle={[2]\color{blue}}]
private int c,a=1,b=2,c_t,
	res=a<b,c_e=c,a_t,a_e=a; 
{c = a; 
 a = a + b;
 c = c * b; 
 a = c + a;}
c_t=c;c=c_e;a_t=a;a=a_e;
{c = b; 
 a = a - b;
 c = c * a; 
 a = c - a;}
c=(res$\cdot$c_t)$+$((1-res)$\cdot$c); 
a=(res$\cdot$a_t)$+$((1-res)$\cdot$a); 
\end{lstlisting}
\caption{How \piccoC\ executes part a.}
\label{Fig: resolution cost ex \piccoC}
\end{subfigure}
\end{tabular}
\caption{Example illustrating why single-statement resolution is more costly when modifying variables multiple times in both branches.}
\label{Fig: resolution cost ex}
\end{figure*}


\subsection{Conditional Code Block Tracking}


To implement data-oblivious execution of branches on private data, SMC implementations typically execute both branches, but privately apply the effects of only the relevant branch. Figure~\ref{Fig: simple correct ex} shows the standard handling of private-conditioned branches. 
The top left shows the original C-code block, with annotations for private data;  the top right shows how compilers would flatten the branch to hide the private data used in the condition. 
Here, the only variable modified within either branch is \texttt{c}, so we insert temporary variables to assist in tracking the results of both branches.
Notation $l_\texttt{x}$ in the table indicates the (private) value stored at the location of variable \texttt{x}. For every variable we list the value in the \texttt{initial} state, before executing the conditional; the one after executing the \texttt{then} branch; the one after the initial state has been restored; the one after the execution of the \texttt{else} branch; and the one after the true value of \texttt{c} has been resolved.


To guarantee data-oblivious executions we also need to guarantee that when we branch on a private condition, the two branches do not have different public side effects. A way to address this is to reject programs that contain public side effects in the body of private-conditioned branching statements. This has an impact on other language constructs such as functions. That is, to be able to call a function from the body of such a branching statement, it must have no public side effects. To address this challenge, in our formalism we evaluate each declared function for public side effects and mark it with a flag that indicates whether it is allowed to be called from within a private-conditioned branch.


PICCO is the only compiler we are aware of that treats pointers to private values, or private pointers.  For private pointers, the location being pointed to might be private as well. That is, if a pointer is assigned a new value inside a private-conditioned branch, we cannot reveal whether the original or new location is to be used. For that reason, a private pointer is associated with multiple locations and a private tag that indicates which location is the true location. In particular, a private pointer is represented as a data structure which tracks the number of locations $\nl$ being pointed to; a list of these $\nl$ (known) locations; a list of $\nl$ (private) tags, one of which is set to 1 and all others set to 0; and the level of pointer indirection. Because locations associated with pointers can now be private, there might be additional limitations on what programs can contain within private-conditioned branches to guarantee data-oblivious execution. 
To understand why multiple locations may potentially be stored for a pointer consider Figure~\ref{Fig: simple correct pointer ex}, whose code will result in the pointer \texttt{p} storing two locations, $l_\texttt{a}$ and $l_\texttt{b}$, with $l_\texttt{a}$ being the true location.



\begin{figure*}  \small
\begin{subfigure}{\textwidth} \small
\begin{minipage}{.35\textwidth}
\begin{lstlisting}[emph={[2]if,else}, emphstyle={[2]\color{blue}}]
private int a=3,b=7,c=0;
if(a<b) {c=a;}
else {c=b;}
\end{lstlisting}
\end{minipage}
\hfill
\begin{minipage}{.6\textwidth}
\begin{lstlisting}[emph={[2]res,c_e,c_t},emphstyle={[2]\color{blue}}]
private int a=3,b=7,c=0,res=a<b,c_e=c,c_t;				
{c=a;} c_t=c; c=c_e; 						
{c=b;} c=(res$\cdot$c_t)$+$((1-res)$\cdot$c);
\end{lstlisting}
\end{minipage}
    \\
    \begin{tabular}[t]{| c | c | c | c | c | c |} \hline
        	\TT{location}		&\TT{init} 	& {\TT{then}} &\TT{restore} 	& {\TT{else}}	&\TT{resolve}
        \\ \hline
        $l_\texttt{c}$ 		& {0} 		&	\teal{3}	&	\teal{0}		&	\teal{7}		&\teal{3}
        \\ \hline 
        $l_\texttt{c\_t}$ 	& {} 		&	{}	&	\teal{3}		&	{3}		&{3}
        \\ \hline
    \end{tabular}
    \hfill
   \begin{tabular}[t]{| c | c |} \hline
   		\TT{location} 	& \TT{value} 	\\ \hline
    		$l_\TT{res}$ 	& 1 			\\ \hline
		$l_\TT{c\_e}$ 	& 0				\\ \hline
    \end{tabular}
    \caption{Regular variable handling within private-conditioned branches.}
    \label{Fig: simple correct ex}
\end{subfigure}
\\ \\ \\
{
\begin{subfigure}{\textwidth}
\begin{minipage}{0.35\textwidth}
\begin{lstlisting}[emph={[2]if,else}, emphstyle={[2]\color{blue}}]
private int a=3,b=7,*p;		
if(a$<$b) {p=&a;}	
else {p=&b;}
\end{lstlisting}	
\end{minipage}
\hfill
\begin{minipage}{0.6\textwidth}
\begin{lstlisting}[emph={[2]res,p_t,p_e}, emphstyle={[2]\color{blue}}]
private int a=3,b=7,*p,res=a$<$b,*p_e=p,*p_t;	
{p=&a;} p_t=p; p=p_e;		
{p=&b;} p=$\purple{\TT{resolve}}$(res,p_t,p);	
\end{lstlisting}	
\end{minipage}\small
    \\
    \begin{tabular}[t]{| c | c | c | c | c | c |} \hline
    	\TT{location}		&\TT{init} 	& {\TT{then}} 	& 	\TT{restore} 	& {\TT{else}} 	&\TT{resolve}
    \\ \hline
    $l_\texttt{p}$ 		& {(\ ),\ (\ )}	& {($l_\texttt{a}$),\ (1)}	& 	\teal{(\ ),\ (\ )}	& \teal{($l_\texttt{b}$),\ (1)}	&\teal{($l_\texttt{a}, l_\texttt{b}$),}
    \teal{(1, 0)}
    \\ \hline 
    $l_\texttt{p\_t}$ 	& {}	& {}	& \teal{($l_\texttt{a}$),\ (1)}	& {($l_\texttt{a}$),\ (1)}	&{($l_\texttt{a}$),\ (1)}
    \\ \hline
    \end{tabular}
    \hfill
   \begin{tabular}[t]{| c | c |} \hline
   		\TT{location} 	& \TT{value} 	\\ \hline
		$l_\TT{res}$ 	& 1				\\ \hline
		$l_\TT{p\_e}$ 	& (\ ),\ (\ )		\\ \hline
    \end{tabular}
    \caption{Pointer handling within private-conditioned branches.}
    \label{Fig: simple correct pointer ex}
\end{subfigure}}
\\ \\ \\
%
\begin{subfigure}{\textwidth}
\begin{minipage}{.32\textwidth}
\begin{lstlisting}[emph={[2]if,else}, emphstyle={[2]\color{blue}}]
private int a=3,
	b=7,c=5,*p=&a;
if (a$<$b) {
	*p=c; }
else {
	p=&b; }
\end{lstlisting}
\end{minipage}
\hfill
\begin{minipage}{.66\textwidth}
\begin{lstlisting}[emph={[2]res,dp_t,p_e,p_t,dp_e}, emphstyle={[2]\color{blue}}]
private int a=3,b=7,c=5,*p=&a,res=a$<$b,
	dp_e=*p,dp_t,*p_e=p,*p_t; 
{*p=c;} 
dp_t=*p; p_t=p; *p=dp_e; p=p_e; 
{p=&b;} 
*p=(res$\cdot$dp_t)$+$((1-res)$\cdot$*p); p=$\purple{\TT{resolve}}$(res,p_t,p);	
\end{lstlisting}
\end{minipage}\small
    \\
    \begin{tabular}[t]{| c | c | c | c | c | c |} \hline
   \TT{location}	& \TT{initial} 		&  \TT{then} 	& \TT{restore} 		& \TT{else} 	& \TT{resolve} 				\\ \hline
    $l_\TT{a}$ 	& {3}       		& \teal{5}   		& \teal{3}			& {3}			& \red{3}					\\ \hline
    $l_\TT{b}$ 	& {7} 	     		& {7}			& {7}				& {7}			& \red{5}					\\ \hline
    $l_\TT{p}$ 	& {($l_\TT{a}$), (1)}	& {($l_\TT{a}$), (1)}	& \teal{($l_\TT{a}$), (1)}	& {($l_\TT{b}$), (1)}	& \teal{($l_\TT{a}$, $l_\TT{b}$),\ (1, 0)}\\ \hline
    $l_\TT{dp\_t}$& {}			&  				& \teal{5} 			& {5}			& {5}					\\ \hline
    $l_\TT{p\_t}$	& {}			& {} & \teal{($l_\TT{a}$), (1)} 	& {($l_\TT{a}$), (1)} & {($l_\TT{a}$), (1)}		\\ \hline
    \end{tabular}
    \hfill
   \begin{tabular}[t]{| c | c |} \hline
   		\TT{location} 	& \TT{value} 		\\ \hline
		$l_\TT{res}$ 	& 1					\\ \hline
		$l_\TT{dp\_e}$ 	& 3					\\ \hline
		$l_\TT{p\_e}$ 	& ($l_\TT{a}$), (1)	\\ \hline
    \end{tabular}
    \caption{Challenges of pointer manipulations within private-conditioned branches.}
    \label{Fig: simple pointer challenge ex}
\end{subfigure}
\\ \\ \\
\begin{subfigure}{\textwidth}
\begin{minipage}[t]{0.32\textwidth}
\begin{lstlisting}[emph={[2]if,else}, emphstyle={[2]\color{blue}}]
public int i=1,j=2;
private int a[j]={0,0},
	b=7,c=3,d=4;
if (c$<$d) { 
	a[i]=c; }
else {
	a[j]=d; }
\end{lstlisting}
\end{minipage}
\hfill
\begin{minipage}[t]{0.66\textwidth}
\begin{lstlisting}[emph={[2]a_t,a_e,b_t,b_e,res}, emphstyle={[2]\color{blue}}]
public int i=1,j=2;
private int a[j]={0,0},b=7,c=3,d=4;
private int res=c$<$d,a_t,a_e=a;
a[i]=c; 
a_t=a; a=a_e; 
a[j]=d;
a=(res$\cdot$a_t)$+$((1-res)$\cdot$a);
\end{lstlisting}
\end{minipage}
\\
    \begin{tabular}[t]{| c | c | c | c | c | c |} \hline
    \TT{location}	& \TT{initial} &  \TT{then} 	& \TT{restore} 		& \TT{else} 	& \TT{resolve} 		\\ \hline
    $l_\TT{a}$ 	& {0, 0}     	& \teal{0, 3}   	& \teal{0, 0}			& {0, 0}			& \teal{0, 3}		\\ \hline
    $l_\TT{b}$ 	& {7} 	     	& {7}			& {7}				& \red{4}		& \red{4}		\\ \hline
    $l_\TT{a\_t}$& {}			&  				& \teal{0, 3} 		& {0, 3}			& {0, 3}			\\ \hline
    \end{tabular}
    \hfill
   \begin{tabular}[t]{| c | c |} \hline
   		\TT{location} 	& \TT{value} 		\\ \hline
		$l_\TT{res}$ 	& 1					\\ \hline
		$l_\TT{a\_e}$ 	& 0, 0				\\ \hline
    \end{tabular}
\caption{Challenges of writing at a public index in a private array within private-conditioned branches.}
\label{Fig: array challenge}
\end{subfigure}
\caption{Private-conditioned branching examples. Simple examples are shown in~\ref{Fig: simple correct ex} and~\ref{Fig: simple correct pointer ex}, and challenging examples are shown in~\ref{Fig: simple pointer challenge ex} and ~\ref{Fig: array challenge}. We show values in memory that change in the table to the left, and values for temporary variables that do not change in the table to the right. We indicate correct updates in memory in \teal{green}, and problematic values in memory in \red{red}.}
\label{Fig: challenge ex}
\end{figure*}


Unfortunately, directly adopting this approach exhibits problematic behavior when more complex operations are considered.
The first level of locations that the pointer refers to is managed, as shown in Figure~\ref{Fig: simple correct pointer ex}, but dereferencing the pointer and modifying the value stored at the location that is pointed to can result in incorrect program behavior and ultimately information leakage. 
Any SMC system that supports both pointers and branches, using PICCO style pointers and classic branch resolution cannot handle these cases.
This occurs because the approach we discussed above relies on assignment statements and single-level, constant location changes to properly restore and resolve the changes made inside the private-conditioned branch. 
One can try to modify this approach to support tracking changes made using pointers at a higher level of indirection  (i.e., tracking \texttt{*p=c} using temporary variables \TT{dp\_t} and \TT{dp\_e}, as shown in Figure~\ref{Fig: simple pointer challenge ex}). 
However, this modification can lead to the incorrect resolution of data when multiple levels of indirection of a pointer are modified within the private-conditioned branches. 
An example of this is shown in Figure~\ref{Fig: simple pointer challenge ex}, where we modify \texttt{*p} in the \TT{then} branch, and then change the location that is pointed to by \texttt{p} in the \TT{else} branch. 
The \TT{then} branch, restoration, and the \TT{else} branch will execute correctly, however, resolving the variables after the completion of the \TT{else} branch will not. 
Given that we have modified the location pointed to by \TT{p}, when we attempt to resolve the modification we made using \texttt{*p}, we will read from and modify the value in $l_\texttt{b}$ (where \TT{p} currently points) instead of the value in $l_\texttt{a}$ (where \TT{p} pointed and wrote to in the \TT{then} branch).


We encounter a similar issue if we write to a public index within a private array during the execution of a private-conditioned branch, and that index happens to be out of bounds. 
We give an example of this in Figure~\ref{Fig: array challenge}. Here, we assume that \TT{b} is assigned the location directly after the array data's location, thus giving us a \emph{well-aligned} out-of-bounds write to illustrate why simple variable tracking is not enough here. 
In any given implementation, an out-of-bounds access is not guaranteed to be \emph{well-aligned}, and therefore unpredictable behavior can occur. 
In this example, we have a zero-indexed array \TT{a} of two elements. In the \TT{then} branch, we modify index 1 (or the second element), then store this updated array and return the array back to it's initial state. 
In the \TT{else} branch, we modify index 2 (out-of-bounds of the array), which updates the value stored for \TT{b}. Given that we were not tracking \TT{b}, this value does not get resolved, and any further uses of \TT{b} will result in incorrect results. 
By using location tracking, we would catch that the location $\loc_\TT{b}$ was modified, and in turn properly restore it to it's original value (as long as the out-of-bounds access is \emph{well-aligned}).


To be able to handle pointers and arrays correctly, 
we must use a location based tracking instead of a variable based tracking.  
However, as our semantics illustrated, this requires additional tracking structures and dynamically checking to ensure that the locations modified during pointer dereference write statements are tracked.  
We, thus, propose a small optimization to full location tracking, which analyzes each top-level private-conditioned branching statement to see if it contains a pointer dereference writes or array writes at public indices in the \TT{then} or \TT{else} clauses, as well as in any nested branching statements present in those clauses.
If no such writes occur, we are able to use simple variable tracking, as shown in rule Private If Else Variable Tracking in Figure~\ref{Fig: iep vt}. 


When any such write statement occurs in either branch, we switch to tracking by location, as shown in rule Private If Else Location Tracking in Figure~\ref{Fig: iep lt}. 
For example, consider a program using a pointer to iterate through and modify elements of an array. Allowing pointer dereference writes enables us to perform a different operation on the array depending on whether a private condition holds. 
For location based conditional code block tracking, we create a map to store the \TT{original} and \TT{then} values for each location that is modified within each branch, as well as a tag to indicate whether this location has a value stored for the \TT{then} branch. 
This tag ensures that even when a location is modified for the first time in the \TT{else} branch, we are still able to properly resolve the value for that location by using the original value stored at that location.  This corresponds to the rules and explanation given in subsection~\ref{sec: priv if lt desc}.

As with the {\em conditional code block} variable  tracking scheme, first we find a list of all modified variables, excluding those only used for pointer dereference writes or array writes at a public index. 
We exclude pointer dereference writes as we will grab the location that is pointed to dynamically to ensure we are tracking the modification at the correct location.  
We use this list to store the original values at the location referred to by each of these variables before the execution of the \TT{then} branch. 
Between branches, our restoration is similar to that formalized for \piccoC, just by location. 
We iterate through our map, storing the current value of each location as the \TT{then} value, and restoring the value at the location as the original value. 
We set the tag associated with each location to be 1, as we have added \TT{then} values for each of these locations. 
When executing the \TT{then} and \TT{else} branches, we check before the execution of a pointer dereference write to see if the location we will modify is already being tracked. 
If it is not, we store the current value as the original value for that location, and then continue to execute as normal; if it is already tracked, we proceed as we do not need to store anything additional. 
We set the tag for each new location to 0, as we do not currently have a \TT{then} value stored for those locations.
After the execution of the \TT{else} branch, we proceed to resolve similarly to \piccoC, just by location. 
For each location in the map, we securely compute whether to keep the \TT{then} value (or \TT{original} value, if the tag is 0) stored in the map or the current value at that location based on the private condition. 
















