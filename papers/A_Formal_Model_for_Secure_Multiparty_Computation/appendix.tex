

%%%%%%%%%%%%%%%%%
%								 %
% 		Multiparty Protocols	 %
%								 %
%%%%%%%%%%%%%%%%%
\section{Multiparty Protocols}
\label{app: mpc protocols}

Algorithm~\ref{algo: mpc binop}, $\MPC{b}$, is a selection control algorithm that directs the evaluation to the relevant multiparty computation algorithm based on the given binary operation $\binop\in\{\cdot,\div,+,-\}$. 

\begin{algorithm}[H]\footnotesize
\caption{$(\n^\pidA_3, ..., \n^\pidZ_3) \gets \MPC{b}(\binop, [\n^\pidA_1, ..., \n^\pidZ_1], [\n^\pidA_2, ..., \n^\pidZ_2])$}
\label{algo: mpc binop}
\begin{algorithmic}
 	\IF{$(\binop = \cdot)$}
		\FORALL{$\pid\in\{\pidA...\pidZ\}$} 
			\STATE $\n^\pid_3 = \MPC{mult}(\n^\pid_1, \n^\pid_2)$
		\ENDFOR
		\RETURN $(\n^\pidA_3, ..., \n^\pidZ_3)$
	\ELSIF{$(\binop = \div)$}
		\FORALL{$\pid\in\{\pidA...\pidZ\}$} 
			\STATE $\n^\pid_3 = \MPC{div}(\n^\pid_1, \n^\pid_2)$
		\ENDFOR
		\RETURN $(\n^\pidA_3, ..., \n^\pidZ_3)$
	\ELSIF{$(\binop = -)$}
		\FORALL{$\pid\in\{\pidA...\pidZ\}$} 
			\STATE $\n^\pid_3 = \MPC{sub}(\n^\pid_1, \n^\pid_2)$
		\ENDFOR
		\RETURN $(\n^\pidA_3, ..., \n^\pidZ_3)$
	\ELSIF{$(\binop = +)$}
		\FORALL{$\pid\in\{\pidA...\pidZ\}$} 
			\STATE $\n^\pid_3 = \MPC{add}(\n^\pid_1, \n^\pid_2)$
		\ENDFOR
		\RETURN $(\n^\pidA_3, ..., \n^\pidZ_3)$
	\ENDIF
\end{algorithmic}
\end{algorithm}


Each of the given multiparty protocols in Algorithm~\ref{algo: mpc binop} (i.e., $\MPC{mult},$ $\MPC{sub},$ $\MPC{add},$ $\MPC{div}$) must be defined using protocols that have been proven to uphold the desired properties within our proofs (i.e., correctness and noninterference). 
We give an example definition for $\MPC{mult}$ in Algorithm~\ref{algo: mpc mult}, but this definition can be swapped out with any protocol for the secure multiparty computation of multiplication that maintains the properties of correctness and noninterference. 
We defer the definition of all other SMC binary operations, rely on assertions that the protocols chosen to be used with this model will maintain both correctness and noninterference in our proofs. 
We chose this strategy as SMC implementations of such protocols will be proven to hold our desired properties on their own, and this allows us to not only leverage those proofs, but to also improve the versatility of our model by allowing such algorithms to be easily swapped out as newer, improved versions become available. 



%%%%%%%%%%%%%%%%%
%								 %
% 		Semantics				 %
%								 %
%%%%%%%%%%%%%%%%%
\section{Semantics}
\label{app: semantics}


\begin{figure*}[h]\footnotesize
\begin{tabular}{l}
\phantomsection\label{rule: da1}
Private Array Declaration \\
	\inferrule{\begin{array}{l}\begin{array}{l l}
		({\Expr}) \isPub \rrgamma \qq\qquad&
		((\Type = {\Priv\ \btype}) \lor (\Type = {\btype})) \land
			((\btype = \Int) \lor (\btype = \Float))
		\crcr
		&((\pid, \rrgamma,\ \sigma,\ {\DMap},\ \Acc,\ \Expr) \Mid  \Config) 
			\Deval{\locLL_1}{\codeLL_1} 
			((\pid, \rrgamma,\ \sigma{_1},\ {\DMap},\ \Acc,\ \nl) \Mid  \Config_1)
		\crcr \nl > 0
		& \byte = \EncodePtr({\Priv\ \Const\ \btype *},\ [1,\ [({\loc_1}, 0)],\ [1],\ 1]) 
		\crcr {\loc} = \phi() 
		& \byte_1 = \EncodeArr(\Priv\ \btype, \nl, \Null)   
		\crcr  {\loc_1} = \phi()
		&\RT{\gamma{_1} = \gamma[\x\ \to\ (\loc,\ {\Priv\ \Const\ \btype*})]} 
		\end{array}\crcr \begin{array}{l}
		\sigma{_2} = \sigma{_1}[{\loc} \to (\byte, {\Priv\ \Const\ \btype *}, 1, 
			\PtrPermL(\PermF, {\Priv\ \Const\ \btype*}, \Priv, 1))]
		\crcr 
		\sigma{_3} = \sigma{_2}[{\loc_1} \to (\byte_1,\ {\Priv\ \btype},\ \nl,\ 
			\ArrPermL(\PermF, \Priv\ \btype, \Priv, \nl))]
	\end{array}\end{array}}					
	{((\pid, \rrgamma,\ \sigma,\ {\DMap},\ \Acc,\ {\Type\ \x[\Expr]}) \Mid  \Config)\ 
		\Deval{\locLL_1 \addL (\pid, [(\loc, 0), ({\loc_1}, 0)])}{\codeLL_1 \addC \codeSP{da1}}  
		((\pid, \rgamma{_1},\ \sigma{_3},\ {\DMap},\ \Acc,\ \Skip) \Mid  \Config_1)}
\\ \\ 
\phantomsection\label{rule: ss}
Statement Sequencing \\
  	\inferrule{\begin{array}{l}
		((\pid,\ \rrgamma,\ \sigma,\ {\DMap},\ \Acc, {\stmt_1}) \Mid  \Config)\ 
			\Deval{\locLL_1}{\codeLL_1}  ((\pid, \rgamma{_1}, \sigma{_1}, {\DMap_1}, \Acc, \val_1) \Mid  \Config_1)
		\crcr
		((\pid, \rgamma{_1}, \sigma{_1}, {\DMap_1}, \Acc, {\stmt_2}) \Mid  \Config_1) 
			\Deval{\locLL_2}{\codeLL_2} ((\pid, \rgamma{_2}, \sigma{_2}, {\DMap_2}, \Acc, \val_2) \Mid  \Config_2)
	\end{array}}
	{((\pid, \rrgamma,\ \sigma,\ {\DMap},\ \Acc,\ {\stmt_1;\ \stmt_2}) \Mid  \Config)
		\Deval{\locLL_1 \addL \locLL_2}{\codeLL_1 \addC \codeLL_2 \addC \codeSP{ss}} 
		((\pid, \RT{\gamma{_2}},\ \sigma{_2},\ {\DMap_2},\ \Acc,\ {\val}) \Mid  \Config_2)}
\\ \\
\phantomsection\label{rule: iet}
Public If Else True \\
  	\inferrule{\begin{array}{l l}
		(\Expr) \isPub \rrgamma \qq &
		((\pid, \rrgamma,\ \sigma,\ {\DMap},\ \Acc, \Expr)\ \ \Mid  \Config)\ 
			\Deval{\locLL_1}{\codeLL_1}  ((\pid, \rrgamma,\ \sigma{_1}, {\DMap_1}, \Acc, \n)\quad\ \Mid  \Config_1) 
		\crcr \n\ \neq\ 0 
		& ((\pid, \rrgamma, \sigma{_1}, {\DMap_1}, \Acc, {\stmt_1}) \Mid  \Config_1) 
			\Deval{\locLL_2}{\codeLL_2} ((\pid, \rgamma{_1}, \sigma{_2}, {\DMap_2}, \Acc, \Skip) \Mid  \Config_2)
	\end{array}}
	{((\pid, \rrgamma,\ \sigma,\ {\DMap},\ \Acc,\ {\If\ (\Expr)\ \stmt_1\ \Else\ \stmt_2}) \Mid  \Config)\ 
		\Deval{\locLL_1\addL\locLL_2}{\codeLL_1 \addC \codeLL_2 \addC \codeSP{iet}}  ((\pid, \rrgamma,\ \sigma{_2},\ {\DMap_2},\ \Acc,\ \Skip) \Mid  \Config_2)}
\end{tabular}
\caption{An illustration of scoping within \DynamicPicco\ rules. We highlight the environment $\RT\gamma$ and its modifications in \red{red}.}
\label{Fig: scoping}
\end{figure*}






\begin{figure*}\footnotesize
\begin{tabular}{l}
\phantomsection\label{rule: bm}
Public Multiplication \\
\inferrule{\begin{array}{l}
		({\Expr_1}, {\Expr_2}) \isPub \gamma \qq\quad
		((\pid, \gamma,\ \sigma,\ {\DMap},\ \Acc,\ {\Expr_1}) \Mid  \Config) \ 
			\Deval{\locLL_1}{\codeLL_1}  ((\pid, \gamma,\ \sigma{_1},\ {\DMap_1},\ \Acc,\ {n_1}) \Mid  \Config_1) 
		\crcr 
		((\pid, \gamma,\ \sigma{_1},\ {\DMap_1},\ \Acc,\ {\Expr_2}) \Mid  \Config_1) \ 
			\Deval{\locLL_2}{\codeLL_2}  ((\pid, \gamma,\ \sigma{_2},\ {\DMap_2},\ \Acc,\ {n_2}) \Mid  \Config_2) 
		\qq {n_1} \cdot {n_2} = {n_3}
	\end{array}}
	{((\pid, \gamma,\ \sigma,\ {\DMap},\ \Acc,\ {\Expr_1 \cdot \Expr_2}) \Mid  \Config)\ 
		\Deval{\locLL_1\addL\locLL_2}{\codeLL_1\addC \codeLL_2 \addC \codeSP{bm}} ((\pid, \gamma,\ \sigma{_2},\ {\DMap_2},\ \Acc,\ {n_3}) \Mid  \Config_2) }
\\ \\
\phantomsection\label{rule: r1}
Read Private Variable \\
  	\inferrule{\begin{array}{l l} 
		\gamma(\x) = (\loc,\ \Priv\ \btype)\ \qquad 
		&\sigma(\loc) = (\byte,\ \Priv\ \btype,\ 1,\ \VarPermL(\PermF, \Priv\ \btype, \Priv, 1)) 
		\crcr &\Decode(\Priv\ \btype,\ \byte) = \n
	\end{array}}
	{((\pid, \gamma,\ \sigma,\ {\DMap},\ \Acc,\ \x) \Mid  \Config)\ 
		\Deval{(\pid, [(\loc, 0)])}{\codeSP{r1}}  
		((\pid, \gamma,\ \sigma,\ {\DMap},\ \Acc,\ \n) \Mid  \Config)}		
\\ \\
\phantomsection\label{rule: mal}
Public Malloc \\
	\inferrule{\begin{array}{l l l}
		\Acc = \AccZ\qq
		& (\Expr) \isPub \gamma\qq
		& ((\pid, \gamma,\ \sigma,\ {\DMap},\ \Acc,\ \Expr) \Mid  \Config)\ 
			\Deval{\locLL_1}{\codeLL_1}  ((\pid, \gamma,\ \sigma{_1},\ {\DMap},\ \Acc,\ \n) \Mid  \Config_1)
		\crcr \loc = \phi()
		&& \sigma_2 = \sigma_1\big[\loc \to \big(\Null, \Void*, \n, \PermL(\PermF, \Void*, \Pub, \n)\big)\big]
	\end{array}}
	{\begin{array}{l}
		((\pid, \gamma,\ \sigma,\ {\DMap},\ \Acc,\ \Malloc (\Expr)) \Mid  \Config) 
			\Deval{\locLL_1 \addL (\pid, [(\loc, 0)])}{\codeLL_1 \addC \codeSP{mal}} 
			((\pid, \gamma,\ \sigma{_2},\ {\DMap},\ \Acc,\ (\loc, 0)) \Mid  \Config_1) 
	\end{array}}
\\ \\
\phantomsection\label{rule: fre}
Public Free \\ 
  	\inferrule{\begin{array}{l}\begin{array}{l l}	
		\gamma(\x) = (\loc,\ {\Pub\ \btype*})\quad\
		&	\sigma(\loc) = (\byte, \Pub\ \btype*, 1, \PermL(\PermF, \Pub\ \btype*, \Pub, 1)) 
		\crcr \Acc = \AccZ \qq 
		& 	\DecodePtr({\Pub\ \btype*},\ 1,\ \byte) = [1,\ [({\loc_1}, 0)],\ [1],\ 1]
		\end{array}\crcr \begin{array}{l l}
		\SelectFreeable(\gamma, [({\loc_1}, 0)], [1], \sigma) = 1\qq
		&	 \Free(\sigma,\ \loc_1) = (\sigma{_1}, (\loc_1, 0))
	\end{array}\end{array}}
	{((\pid, \gamma,\ \sigma,\ {\DMap},\ \Acc,\ {\free (\x)}) \Mid  \Config)\ 
		\Deval{(\pid, [(\loc, 0), (\loc_1, 0)])}{\codeSP{fre}}  
		((\pid, \gamma,\ \sigma{_1},\ {\DMap},\ \Acc,\ \Skip) \Mid  \Config)}
\\ \\
\phantomsection\label{rule: inp}
SMC Input Public Value \\
	\inferrule{ \begin{array}{l l}
		({\Expr}) \isPub \gamma \qq
		&((\pid, \gamma, \sigma, {\DMap}, \Acc, {\Expr} ) \Mid  \Config)\ 
			\Deval{\locLL_1}{\codeLL_1} ((\pid, \gamma, \sigma{_1}, {\DMap_1}, \Acc, {n})\quad \Mid  \Config_1) 	\qquad
		\crcr
		\gamma(\x) = (\loc, \Pub\ \btype)
		& \InputVal(\x, {n}) = {n_1}	\qq
		\crcr 
		\Acc =\AccZ \qq
		&((\pid, \gamma, \sigma{_1}, {\DMap_1}, \Acc, {\x = n_1} ) \Mid  \Config_1) 
			\Deval{\locLL_2}{\codeLL_2} ((\pid, \gamma, \sigma{_2}, {\DMap_2}, \Acc, \Skip) \Mid  \Config_2)
	\end{array}}
	{((\pid, \gamma,\ \sigma,\ {\DMap},\ \Acc,\ {\smcinput(\x,\ \Expr)}) \Mid  \Config) 
		\Deval{\locLL_1 \addL \locLL_2}{\codeLL_1 \addC \codeLL_2 \addC \codeSP{inp}}  
		((\pid, \gamma,\ \sigma{_2},\ {\DMap_2},\ \Acc,\ \Skip) \Mid  \Config_2)}
\\ \\ 
\phantomsection\label{rule: out3}
SMC Output Private Array \\
	\inferrule{ \begin{array}{l}\begin{array}{l l}
		({\Expr_1}, {\Expr_2}) \isPub \gamma \qquad
		&	((\pid, \gamma,\ \sigma,\ {\DMap},\ \Acc, {\Expr_1} ) \Mid  \Config)\ 
				\Deval{\locLL_1}{\codeLL_1} ((\pid, \gamma, \sigma{_1}, {\DMap_1}, \Acc, {n}) \Mid  \Config_1) 
		\crcr \gamma(\x) = ({\loc}, {\Priv\ \Const\ \btype*}) \qquad
		&	((\pid, \gamma, \sigma{_1}, {\DMap_1}, \Acc, {\Expr_2} ) \Mid  \Config_1) 
				\Deval{\locLL_2}{\codeLL_2} ((\pid, \gamma, \sigma{_2}, {\DMap_2}, \Acc, {\nl}) \Mid  \Config_2)
		\end{array}\crcr\begin{array}{l}
			\sigma{_2}({\loc}) = ({\byte}, {\Priv\ \Const\ \btype*}, 1, 
					\PtrPermL(\PermF, {\Priv\ \Const\ \btype*}, \Priv, 1))
		\crcr \DecodePtr({\Priv\ \Const\ \btype*},\ 1,\ {\byte}) = [1,\ [({\loc_1}, 0)],\ [1],\ \Priv\ \btype,\ 1]
		\crcr \sigma{_2}({\loc_1}) = ({\byte_1},\ {\Priv\ \btype},\ {\nl},\ \ArrPermL(\PermF, \Priv\ \btype, \Priv, {\nl}))
		\crcr \forall \ind \in \{0, ..., \nl-1\} \qquad \DecodeArr({\Priv\ \btype},\ {\ind},\ {\byte_1})  = {m_\ind}
		\crcr \OutputArr(\x,\ {n},\ [{m_0},\ {...},\ {m_{\nl-1}}])
	\end{array}\end{array}}
	{\begin{array}{l}
	((\pid, \gamma,\ \sigma,\ {\DMap},\ \Acc,\ {\smcoutput(\x,\ \Expr_1,\ \Expr_2)}) \Mid  \Config) 
		\crcr\Deval{\locLL_1 \addL \locLL_2 \addL (\pid, [(\loc, 0), (\loc_1, 0), ..., (\loc_1, n_1-1)])}{\codeLL_1 \addC \codeLL_2 \addC \codeSP{out3}}  
		((\pid, \gamma,\ \sigma{_2},\ {\DMap_2},\ \Acc,\ \Skip) \Mid  \Config_2)
		\end{array}}
\\ \\ 
\phantomsection\label{rule: pin}
Pre-Increment Public Variable \\
  	\inferrule{\begin{array}{l l}
		\gamma (\x) = (\loc,\ \Pub\ \btype) \qquad
		&	\sigma(\loc) = (\byte,\ \Pub\ \btype,\ 1,\ \VarPermL(\PermF, \Pub\ \btype, \Pub, 1))
		\crcr \Acc =\AccZ \qquad
		&	\Decode(\Pub\ \btype,\ \byte) = {\n}	
		\crcr {\n_1} = {\n} + 1 
		&	\Update(\sigma,\ \loc,\ {\n_1},\ {\DMap},\ \Acc, \Pub\ \btype) = (\sigma{_1},\ {\DMap})
	\end{array}} 
	{((\pid, \gamma,\ \sigma,\ {\DMap},\ \Acc,\ {\plpl\x}) \Mid  \Config)\ 
		\Deval{(\pid, [(\loc, 0)])}{\codeSP{pin}}  
		((\pid, \gamma,\ \sigma{_1},\ {\DMap},\ \Acc,\ {\n_1}) \Mid  \Config)}
\\ \\
\phantomsection\label{rule: mprdp}
Multiparty Private Pointer Dereference Single Level Indirection\\ 
	\inferrule{\begin{array}{l}
		\{(\x) \isPriv \gamma^\pid\}^{\pidZ}_{\pid = \pidA} 
		\qq \{ \gamma^\pid(\x) = (\loc^\pid, \Priv\ \btype*)\}^{\pidZ}_{\pid = \pidA}
		\qq  \nl > 1  \crcr
		\{\sigma^\pid(\loc^\pid) = (\byte^\pid,\ {\Priv\ \btype*},\ \nl,\ \PtrPermL(\PermF, {\Priv\ \btype*}, \Priv, \nl))\}^{\pidZ}_{\pid = \pidA}
		\crcr 
		\{\DecodePtr({\Priv\ \btype*},\ \nl,\ \byte^\pid) = [\nl,\ \locL^\pid,\ \tagbL^\pid, 1] \}^{\pidZ}_{\pid = \pidA}
		\crcr 
		\{\Retrieve(\nl, \locL^\pid, \Priv\ \btype, \sigma^\pid) = ([\n^\pid_0, ...\n^\pid_{\nl-1}], 1)\}^{\pidZ}_{\pid = \pidA}
		\crcr \MPC{dv}([[\n^{\pidA}_{0}, ..., \n^{\pidA}_{\nl-1}], ..., [\n^{\pidZ}_{0}, ..., \n^{\pidZ}_{\nl-1}]], [\tagbL^\pidA, ..., \tagbL^\pidZ]) = (\n^{\pidA}, ..., \n^{\pidZ})
	\end{array}}
	{\begin{array}{l}
	((\pidA, \gamma^{\pidA}, \sigma^{\pidA}, \DMap^\pidA, \Acc, {* \x})\Mid ...\Mid 
	(\pidZ, \gamma^{\pidZ}, \sigma^{\pidZ}, \DMap^\pidZ, \Acc, {* \x}))
		\Deval{(\pidA, (\loc^\pidA, 0)\addL\locL^{\pidA}) \Mid ... \Mid (\pidZ, (\loc^\pidZ, 0)\addL\locL^{\pidZ})}{\codeMP{mprdp}}  
		\crcr((\pidA, \gamma^{\pidA}, \sigma^{\pidA}, \DMap^\pidA, \Acc, \n^{\pidA}_{})\Mid ...\Mid 
		(\pidZ, \gamma^{\pidZ}, \sigma^{\pidZ}, \DMap^\pidZ, \Acc, \n^{\pidZ}_{}))
		\end{array}}
\end{tabular}
\caption{Additional \DynamicPicco\ semantic rules.}
\label{Fig: sem app}
\end{figure*}







\begin{figure*}\footnotesize
\begin{tabular}{l}
\phantomsection\label{rule: ra1}
Private Array Read Public Index \\
  	\inferrule{
	\begin{array}{l}
	(\Expr) \isPub \gamma \qq
	((\pid, \gamma,\ \sigma,\ {\DMap},\ \Acc,\ \Expr) \Mid  \Config)\ 
		\Deval{\locLL_1}{\codeLL_1}  ((\pid, \gamma,\ \sigma{_1},\ {\DMap_1},\ \Acc,\ \ind) \Mid  \Config_1) \crcr
	\gamma(\x) = (\loc,\ {\Priv\ \Const\ \btype*}) \crcr
	\sigma{_1}(\loc) = (\byte,\ {\Priv\ \Const\ \btype*}, 1, 
		\PtrPermL(\PermF, {\Priv\ \Const\ \btype*}, \Priv, 1))
	\crcr \DecodePtr({\Priv\ \Const\ \btype*},\ 1,\ \byte) = [1,\ [({\loc_1}, 0)],\ [1],\ 1] 
	\qquad 0 \leq \ind \leq {\nl} -1 
	\crcr \sigma{_1}({\loc_1}) = ({\byte_1}, {\Priv\ \btype}, {\nl}, 
		\ArrPermL(\PermF, \Priv\ \btype, \Priv, {\nl})) 
	\crcr \DecodeArr({\Priv\ \btype},\ \ind,\ {\byte_1}) = \n_\ind 
	\end{array}}
	{ ((\pid, \gamma,\ \sigma,\ {\DMap},\ \Acc,\ {\x[\Expr]}) \Mid  \Config)\ 
		\Deval{\locLL_1 \addL (\pid, [(\loc, 0), ({\loc_1}, \ind)])}{\codeLL_1 \addC \codeSP{ra1}}  ((\pid, \gamma,\ \sigma{_1},\ {\DMap_1},\ \Acc,\ {\n_\ind}) \Mid  \Config_1)}
\\ \\ 
\phantomsection\label{rule: wa}
Public Array Write Public Value Public Index \\
  	\inferrule{\begin{array}{l}
		({\Expr_1}, {\Expr_2}) \isPub \gamma \qquad\quad
		((\pid, \gamma,\ \sigma,\ {\DMap},\ \Acc, {\Expr_1}) \Mid  \Config)\ 
			\Deval{\locLL_1}{\codeLL_1}  ((\pid, \gamma,\ \sigma{_1},\ {\DMap_1},\ \Acc,\ \ind) \Mid  \Config_1) \crcr
		\Acc = \AccZ \qq\ \
		((\pid, \gamma, \sigma{_1}, {\DMap_1}, \Acc, {\Expr_2}) \Mid  \Config_1) 
			\Deval{\locLL_2}{\codeLL_2}  ((\pid, \gamma,\ \sigma{_2},\ {\DMap_2},\ \Acc,\ \n) \Mid  \Config_2) \crcr
		\gamma(\x) = (\loc,\ {\Pub\ \Const\ \btype*})	\crcr
%		\val \neq \Skip \qquad
		\sigma{_2}(\loc) = (\byte, {\Pub\ \Const\ \btype*}, 1, \PtrPermL(\PermF, {\Pub\ \Const\ \btype*}, \Pub, 1)) 
		\crcr \DecodePtr({\Pub\ \Const\ \btype*}, 1, \byte) = [1,\ [({\loc_1, 0})],\ [1],\ 1]
		\crcr \sigma{_2}({\loc_1}) = ({\byte_1}, \Pub\ \btype, \nl, \ArrPermL(\PermF, \Pub\ \btype, \Pub, \nl))
		\qquad 0 \leq {\ind} \leq {\nl-1} 
		\crcr 
		 \UpdateArr(\sigma{_2},\ ({\loc_1}, \ind),\ \n,\ \Pub\ \btype) = \sigma{_3}
	\end{array}}
	{((\pid, \gamma,\ \sigma,\ {\DMap},\ \Acc,\ {\x[\Expr_1]\ = \Expr_2}) \Mid  \Config)\ 
		\Deval{\locLL_1 \addL \locLL_2 \addL (\pid, [(\loc, 0), (\loc_1, \ind)])}{\codeLL_1 \addC \codeLL_2 \addC \codeSP{wa}}  
		((\pid, \gamma,\ \sigma{_3},\ {\DMap_2},\ \Acc,\ \Skip) \Mid  \Config_2)}
\\ \\ 
\phantomsection\label{rule: rp}
Pointer Read Single Location \\
	\inferrule{\begin{array}{l l}
		\gamma(\x) = (\loc,\ {\llabel\ \btype*}) \qquad 
		& \sigma(\loc) = (\byte,\ {\llabel\ \btype*},\ 1,\ 
			\PtrPermL(\PermF, {\llabel\ \btype*}, \llabel, 1))
		\crcr &\DecodePtr({\llabel\ \btype*},\ 1,\ \byte) = [1,\ [({\loc_1}, \offset_1)],\ [1],\ \indir] 
	\end{array}}
	{((\pid, \gamma,\ \sigma,\ {\DMap},\ \Acc,\ \x) \Mid  \Config)\ 
		\Deval{(\pid, [(\loc, 0)])}{\codeSP{rp}}  ((\pid, \gamma,\ \sigma,\ {\DMap},\ \Acc,\ {(\loc_1, \offset_1)}) \Mid  \Config)}
\\ \\
\phantomsection\label{rule: wp1}
Private Pointer Write \\
  	\inferrule{\begin{array}{l}
		\gamma(\x) = ({\loc}, {\Priv\ \btype*}) \qquad\
		((\pid, \gamma, \sigma, {\DMap}, \Acc, \Expr) \Mid  \Config) 
			\Deval{\locLL_1}{\codeLL_1} ((\pid, \gamma, \sigma{_1}, {\DMap_1}, \Acc, ({\loc_e}, \offset_e)) \Mid  \Config_1) \crcr
		(\Expr) \isPub \gamma \qq
		\sigma{_1}({\loc}) = (\byte,\ {\Priv\ \btype*},\ \nl, \PtrPermL(\PermF, {\Priv\ \btype*}, \Priv, \nl))
		\crcr \DecodePtr({\Priv\ \btype*},\ \nl,\ \byte) = [\nl,\ \locL,\ \tagbL,\ \indir]
		\crcr \UpdatePtr(\sigma{_1},\ {(\loc, 0)},\ [1,\ [({\loc_e}, \offset_e)],\ [1],\ \indir],\ 
			{\DMap_1},\ \Acc, \Priv\ \btype*) = (\sigma{_2},\ {\DMap_2}, 1) 
	\end{array}} 
	{((\pid, \gamma,\ \sigma,\ {\DMap},\ \Acc,\ {\x = \Expr}) \Mid  \Config)\ 
		\Deval{\locLL_1 \addL (\pid, [(\loc, 0)])}{\codeLL_1\addC\codeSP{wp1}}  ((\pid, \gamma,\ \sigma{_2},\ {\DMap_2},\ \Acc,\ \Skip) \Mid  \Config_1)}
\\ \\ 
\phantomsection\label{rule: wdp3}
Private Pointer Dereference Write Single Location Private Value \\ 
	\inferrule{
	\begin{array}{l}
		(\Expr) \isPriv \gamma	\qq
		((\pid, \gamma,\ \sigma,\ {\DMap},\ \Acc,\ \Expr) \Mid  \Config) 
			\Deval{\locLL_1}{\codeLL_1} ((\pid, \gamma,\ \sigma{_1},\ {\DMap_1},\ \Acc,\ \n) \Mid  \Config_1) 
		\crcr \gamma(\x) = (\loc,\ {\Priv\ \btype*}) 
		\qq (\btype = \Int) \lor (\btype = \Float) 
		\crcr \sigma{_1}(\loc) = (\byte,\ {\Priv\ \btype*},\ 1,\ \PtrPermL(\PermF, {\Priv\ \btype*}, \Priv, 1)) 
		\crcr \DecodePtr({\Priv\ \btype*},\ 1,\ \byte) = [1,\ [(\loc_1, \offset_1)],\ [1],\ 1]
		\crcr \DynUpdate(\DMap_1, \sigma, [(\loc_1, \offset_1)],\ \Acc, \Priv\ \btype) = \DMap_2
		\crcr \UpdateOffset(\sigma{_1},\ ({\loc_1}, \offset_1),\ \n, \Priv\ \btype) = (\sigma{_2}, 1)
	\end{array}}
	{((\pid, \gamma,\ \sigma,\ {\DMap},\ \Acc,\ {* \x = \Expr}) \Mid  \Config) 
		\Deval{\locLL_1 \addL (\pid, [(\loc, 0), ({\loc_1}, \offset_1)])}{\codeLL_1\addC\codeSP{wdp3}}  
		((\pid, \gamma,\ \sigma{_2},\ {\DMap_2},\ \Acc,\ \Skip) \Mid  \Config_1) }
\\ \\
\phantomsection\label{rule: pin1}
Pre-Increment Public Pointer Single Location \\
  	\inferrule{\begin{array}{l}
		\gamma (\x) = (\loc,\ {\Pub\ \btype*}) \crcr
		\sigma(\loc) = (\byte,\ {\Pub\ \btype*},\ 1,\ \PtrPermL(\PermF, {\Pub\ \btype*}, \Pub, 1))
		\crcr \DecodePtr({\Pub\ \btype*},\ 1,\ \byte) = [1,\ [({\loc_1}, \offset_1)],\ [1],\ 1]
		\crcr (({\loc_2}, \offset_2), 1) = \GetLoc(({\loc_1}, \offset_1), \tau({\Pub\ \btype}), \sigma)
		\crcr \UpdatePtr(\sigma,\ (\loc, 0),\ [1,\ [({\loc_2}, \offset_2)],\ [1],\ 1],\ {\DMap},\ \Acc, 
			\Pub\ \btype*) = (\sigma{_1},\ {\DMap_1}, 1)
	\end{array}} 
	{((\pid, \gamma,\ \sigma,\ {\DMap},\ \Acc,\ {\plpl\x}) \Mid  \Config)\ 
		\Deval{(\pid, [(\loc, 0)])}{\codeSP{pin1}}  
		((\pid, \gamma,\ \sigma{_1},\ {\DMap_1},\ \Acc,\ ({\loc_2}, \offset_2)) \Mid  \Config)}
\\ \\
\phantomsection\label{rule: rdp}
Pointer Dereference Single Location  \\ 
	\inferrule{\begin{array}{l}
		\gamma(\x) = (\loc,\ {\llabel\ \btype*}) 
		\qq  \sigma({\loc}) = ({\byte},\ {\llabel\ \btype*},\ 1,\ \PtrPermL(\PermF, {\llabel\ \btype*}, \llabel, 1)) 
		\crcr \DecodePtr({\llabel\ \btype*},\ 1,\ {\byte}) = [1,\ [({\loc_1}, \offset_1)],\ [1],\ 1] 
		\qquad\quad \DerefPtrPub(\sigma, \llabel\ \btype, ({\loc_1}, \offset_1)) = (\n, 1)
	\end{array}}
	{((\pid, \gamma,\ \sigma,\ {\DMap},\ \Acc,\  {* \x}) \Mid  \Config)\ 
		\Deval{(\pid, [(\loc, 0), ({\loc_1}, \offset_1)])}{\codeSP{rdp}}  ((\pid, \gamma,\ \sigma,\ {\DMap},\ \Acc,\ \n) \Mid  \Config)}
\end{tabular}
\caption{Additional \DynamicPicco\ semantic rules for arrays and pointers.}
\label{Fig: sem app arr}
\end{figure*}























We show selected additional semantic rules in Figures~\ref{Fig: sem app} and~\ref{Fig: sem app arr} to give a more encompassing view of our semantic model.
In particular, we show a larger subset of array and pointer rules, as well as input/output and public allocation and deallocation. 
We use several algorithms within the rules in order to increase the readability of the rules and compartmentalize common functionalities between rules; we will discuss a few further here. 

The variations of $\PermL$ are all used to map the information given as arguments into the appropriate byte-wise permission tuples. This allows us to show the important information about the byte-wise permissions without the repetition of showing the list of permission tuples and getting into the more intricate details, particularly for pointer data structures which have a mix of public and private data and therefore a mix of permissions. 
Algorithms such as $\InputVal$ and $\OutputArr$ are, respectively, handlers for reading data in and writing data to a file. 
Algorithm $\SelectFreeable$ is used to evaluate whether the locations a pointer refers to are indeed freeable (i.e., all locations were allocated through the use of \TT{malloc} or \TT{pmalloc} and not the default location or a location allocated during a variable declaration. 
Algorithm $\Free$ is used to modify the permissions of the location that is being freed to be $\PermN$. 
All algorithms such as $\Update$, $\UpdateArr$, and $\UpdatePtr$ contain the intricacies of updating the specific type of data within memory. 

Algorithm $\Retrieve$ will go through memory and pull all values that are referred to at the $\nl$ locations of that pointer. This is a helper function to allow up to show the exact information that will be used within the multiparty protocol. 
Algorithm $\DynUpdate$ ensures we are tracking every location that is modified within a private-conditioned branch by checking if the current location we are modifying with the pointer dereference write is already tracked in $\DMap$, and if it is not, adding it and the original value for that location to $\DMap$. 
Algorithm $\GetLoc$ handles incrementing the given location and offset by the appropriate size in bytes. It returns the new location and offset that the pointer will refer to, without complicating the rule with all the redundancies of handling locations of different sizes and finding the appropriate position. 
Algorithm $\DerefPtrPub$ will take the location and offset that the pointer refers to as well as the expected type, and give back the value that is obtained from reading from that position. This handles all the intricacies of pointers with non-zero offsets, and ensures that we will always read data from memory as the expected type. 

We also handle pointers of higher levels of indirection; we chose not to show those rules here as the main concepts behind the rules are fairly similar to those shown. 
The full semantic model for \piccoC\ is available at \cite{amys-dissertation} (Chapter 5), including the algorithms used within the semantics and the full \vanillaC\ semantics. 



\subsection{Scoping}
In our semantics, we implement standard C scoping through our use of the environment $\gamma$. 
To illustrate this, Figure~\ref{Fig: scoping} contains a few rules with $\gamma$ and its additions highlighted in \red{red}. 
First, we show the Private Array Declaration rule, where we add a new mapping for the array variable and its newly allotted location to the environment and return the updated environment. 
Next, we have the standard Statement Sequencing rule, where we pass along all additions to the environment that were made within each statement. 
Finally we show the Public If Else True rule. Here, we have an updated environment $\RT{\gamma_1}$ returned from the evaluation of the \TT{then} branch $\stmt_1$; any new mappings introduced within $\stmt_1$ become out of scope once we exit this rule, so we return the original environment $\RT\gamma$. This way, any further references to local variables declared within $\stmt_1$ will no longer be found in the environment and cannot execute, as is expected. 



\begin{algorithm*}\footnotesize
\caption{$(\x_\mathit{mod}, \tagb) \gets \DynExtract(\stmt_1,\ \stmt_2, \gamma)$}
\label{algo: dyn extract}
\begin{algorithmic}
	\STATE $\tagb = 0$
 	\STATE $\x_\mathit{local} = [\ ] $
	\STATE $\x_\mathit{mod} = [\ ] $		
       \FORALL {$\stmt \in \{\stmt_1;\ \stmt_2\}$}												
		 \IF{$((\stmt ==\ \Type\ \x) \lor (\stmt ==\ \Type\ \x[\Expr]))$}
            			\STATE $\x_\mathit{local}.\mathit{append}(\x)$ 								
		\ELSIF{$((\stmt ==\ \x = \Expr) \land (\lnot\x_\mathit{local}.\mathit{contains}(\x)))$}			
            			\STATE $\x_\mathit{mod} = \x_\mathit{mod} \cup [\x]$
				\FORALL{$\Expr_1 \in \Expr$}
					\IF{$((\Expr_1 ==\ \plpl\x_1) \land (\lnot\x_\mathit{local}.\mathit{contains}(\x_1)))$}
						\STATE $\x_\mathit{mod} = \x_\mathit{mod} \cup [\x_1]$
					\ENDIF
				\ENDFOR	
		\ELSIF{$((\stmt ==\ \x[\Expr_1] = \Expr_2) \land (\lnot\x_\mathit{local}.\mathit{contains}(\x)))$}
				\IF{$(\Expr_1)\isPriv \gamma$}
					\STATE $\x_\mathit{mod} = \x_\mathit{mod} \cup [\x]$
				\ELSE
					\STATE $\tagb = 1$
				\ENDIF
				\FORALL{$\Expr \in \{\Expr_1, \Expr_2\}$}
					\IF{$((\Expr ==\ \plpl\x_1) \land (\lnot\x_\mathit{local}.\mathit{contains}(\x_1)))$}
						\STATE $\x_\mathit{mod} = \x_\mathit{mod} \cup [\x_1]$
					\ENDIF
				\ENDFOR
		\ELSIF{$((\stmt ==\ \plpl\x) \land (\lnot\x_\mathit{local}.\mathit{contains}(\x)))$}
				\STATE $\x_\mathit{mod} = \x_\mathit{mod} \cup [\x]$
		\ELSIF{$(\stmt == *\x = \Expr)$}
				\STATE $\tagb = 1$
				\FORALL{$\Expr_1 \in \Expr$}
					\IF{$((\Expr_1 ==\ \plpl\x_1) \land (\lnot\x_\mathit{local}.\mathit{contains}(\x_1)))$}
						\STATE $\x_\mathit{mod} = \x_\mathit{mod} \cup [\x_1]$
					\ENDIF
				\ENDFOR
		\ENDIF
	\ENDFOR
         \RETURN $(\x_\mathit{mod}, \tagb)$
\end{algorithmic}
\end{algorithm*}



\begin{algorithm*}\footnotesize
\caption{$(\gamma_1, \sigma_1, \locL) \gets \Initialize(\x_{\vl},\ \gamma, \sigma, n, \Acc)$}
\label{algo: initialize}
\begin{algorithmic}
	\STATE $\loc_\res = \phi(\mathit{temp})$
	\STATE $\gamma_1 = \gamma[\res\_\Acc \to (\loc_\res, \Priv\ \Int)]$
	\STATE $\byte_\res = \Encode(\Priv\ \Int, n)$
	\STATE $\sigma_1 = \sigma[\loc_\res \to (\byte_\res, \Priv\ \Int, 1, \VarPermL(\PermF, \Priv\ \Int, \Priv, 1))]$
	\STATE $\locL = [(\loc_\res, 0)]$
	\FORALL{$\x \in \x_{\vl}$}
		\STATE $(\loc_\x, \Type) = \gamma(\x)$	
		\STATE $\loc_t = \phi(\mathit{temp})$
		\STATE $\loc_e = \phi(\mathit{temp})$
		% record touched locs
		\STATE $\locL = \locL \addL[(\loc_\x, 0), (\loc_t, 0), (\loc_e, 0)]$
		\STATE $\gamma_1 = \gamma_1[\x\_t\_\Acc \to (\loc_t, \Type)][\x\_e\_\Acc \to (\loc_e, \Type)]$
		\STATE $(\byte_\x, \Type, \nl, \VarPermL(\PermF, \Type, \Priv, \nl)) = \sigma_1(\loc_\x)$
		% Special case for if entire array is modified
		\IF{$(\Type = \Priv\ \Const\ \btype*)$} 
			% temp locs for array data
			\STATE $\loc_{ta} = \phi(\mathit{temp})$
			\STATE $\loc_{ea} = \phi(\mathit{temp})$
			% Decode x
			\STATE $[1, [(\loc_{xa}, 0)], [1], 1] = \DecodePtr(\Type, 1, \byte_{x})$
			% Look up x array data 
			\STATE $(\byte_{xa}, \Priv\ \btype, \nl, \PtrPermL(\PermF, \Priv\ \btype, \Priv, \nl)) = \sigma_1(\loc_{xa})$
			% add mappings for array data
			\STATE $\sigma_1 = \sigma_1[\loc_{ta} \to (\byte_{xa}, \Priv\ \btype, \nl, \VarPermL(\PermF, \Priv\ \btype, \Priv, \nl))]$
			\STATE $\sigma_1 = \sigma_1[\loc_{ea} \to (\byte_{xa}, \Priv\ \btype, \nl, \VarPermL(\PermF, \Priv\ \btype, \Priv, \nl))]$
			% Encode ptr repr.
			\STATE $\byte_{t} = \EncodePtr(\Type, [1, [(\loc_{t}, 0)], [1], 1])$
			\STATE $\byte_{e} = \EncodePtr(\Type, [1, [(\loc_{e}, 0)], [1], 1])$
			% add mappings for const
			\STATE $\sigma_1 = \sigma_1[\loc_t \to (\byte_{t}, \Type, 1, \PtrPermL(\PermF, \Type, \Priv, 1))]$
			\STATE $\sigma_1 = \sigma_1[\loc_e \to (\byte_{e}, \Type, 1, \PtrPermL(\PermF, \Type, \Priv, 1))]$
			% add locations touched
			\FORALL{$i \in \{0...\nl-1\}$}
				\STATE $\locL = \locL \addL [(\loc_{xa}, i), (\loc_{ta}, i), (\loc_{ea}, i)]$
			\ENDFOR
		\ELSE
			\STATE $\sigma_1 = \sigma_1[\loc_t \to (\byte_\x, \Type, \nl, \VarPermL(\PermF, \Type, \Priv, \nl))]$
			\STATE $\sigma_1 = \sigma_1[\loc_e \to (\byte_\x, \Type, \nl, \VarPermL(\PermF, \Type, \Priv, \nl))]$
		\ENDIF
	\ENDFOR									
	\RETURN $(\gamma_1, \sigma_1, \locL)$
\end{algorithmic}
\end{algorithm*}





\begin{algorithm*}\footnotesize
\caption{$(\sigma_4, \locL) \gets \Restore(\x_{\vl}, \gamma, \sigma, \Acc)$}
\label{algo: restore}
\begin{algorithmic}
	\STATE $\locL = [\ ]$
	\FORALL{$\x \in \x_{\vl}$}	 
		\STATE $(\loc_\x, \Type) = \gamma(\x)$
		\STATE $(\loc_t, \Type) = \gamma(\x\_t\_\Acc)$
		\STATE $(\loc_e, \Type) = \gamma(\x\_e\_\Acc)$
		% record touched locs
		\STATE $\locL = \locL \addL [(\loc_\x, 0), (\loc_t, 0), (\loc_e, 0)]$
		% Special case for if entire array is modified
		\IF{$(\Type = \Priv\ \Const\ \btype*)$}
			% look up x/t/e
			\STATE $(\byte_{xa}, \Type, 1, \VarPermL(\PermF, \Type, \Priv, 1)) = \sigma(\loc_\x)$
			\STATE $(\byte_{ta}, \Type, 1, \VarPermL(\PermF, \Type, \Priv, 1)) = \sigma(\loc_t)$
			\STATE $(\byte_{ea}, \Type, 1, \VarPermL(\PermF, \Type, \Priv, 1)) = \sigma(\loc_e)$
			% Decode x/t/e
			\STATE $[1, [(\loc_{xa}, 0)], [1], 1] = \DecodePtr(\Type, 1, \byte_{xa})$
			\STATE $[1, [(\loc_{ta}, 0)], [1], 1] = \DecodePtr(\Type, 1, \byte_{ta})$
			\STATE $[1, [(\loc_{ea}, 0)], [1], 1] = \DecodePtr(\Type, 1, \byte_{ea})$
			% Look up x/t array data 
			\STATE $\sigma_1[\loc_{xa} \to (\byte_t, \Type, \nl, \VarPermL(\PermF, \Type, \Priv, \nl))] = \sigma$
			\STATE $\sigma_2[\loc_{ta} \to (\byte_\x, \Type, \nl, \VarPermL(\PermF, \Type, \Priv, \nl))] = \sigma_1$
			% store in then
			\STATE $\sigma_3 = \sigma_2[\loc_{ta} \to (\byte_t, \Type, \nl, \VarPermL(\PermF, \Type, \Priv, \nl)]$
			% look up else
			\STATE $(\byte_\x, \Type, \nl, \VarPermL(\PermF, \Type, \Priv, \nl)) = \sigma_3(\loc_{ea})$
			% store in x
			\STATE $\sigma_4 = \sigma_3[\loc_{xa} \to (\byte_\x, \Type, \nl, \VarPermL(\PermF, \Type, \Priv, \nl)]$
			% record touched locs
			\FORALL{$i \in \{0...\nl-1\}$}
				\STATE $\locL = \locL \addL [(\loc_{xa}, i), (\loc_{ta}, i), (\loc_{ea}, i)]$
			\ENDFOR
		\ELSE
			% look up x/t
			\STATE $\sigma_1[\loc_\x \to (\byte_t, \Type, \nl, \VarPermL(\PermF, \Type, \Priv, \nl))] = \sigma$
			\STATE $\sigma_2[\loc_t \to (\byte_\x, \Type, \nl, \VarPermL(\PermF, \Type, \Priv, \nl)] = \sigma_1$
			% store in then
			\STATE $\sigma_3 = \sigma_2[\loc_t \to (\byte_t, \Type, \nl, \VarPermL(\PermF, \Type, \Priv, \nl)]$
			% look up else
			\STATE $(\byte_\x, \Type, \nl, \VarPermL(\PermF, \Type, \Priv, \nl)) = \sigma_3(\loc_e)$
			% store in x
			\STATE $\sigma_4 = \sigma_3[\loc_\x \to (\byte_\x, \Type, \nl, \VarPermL(\PermF, \Type, \Priv, \nl)]$
		\ENDIF
		\STATE $\sigma = \sigma_4$
	\ENDFOR							
	\RETURN $(\sigma_4, \locL)$
\end{algorithmic}
\end{algorithm*}




\begin{algorithm*}\footnotesize
\caption{$(\valL, \n_{\res}, \locL) \gets \ResolveR(\x_{\vl}, \Acc, \gamma, \sigma)$}
\label{algo: resolve R}
\begin{algorithmic}
	\STATE $\valL = [\ ]$
	% look up res
	\STATE $(\loc_{\res}, \Priv\ \Int) = \gamma(\res\_\Acc)$
	\STATE $(\byte_{\res}, \Priv\ \Int, 1, \VarPermL(\PermF, \Priv\ \Int, \Priv, 1)) = \sigma(\loc_{\res})$
	\STATE $\n_{\res} = \Decode(\Priv\ \Int, \byte_{\res})$
	\STATE $\locL = [(\loc_{\res}, 0)]$
	\FORALL{ $\x \in \x_{\vl}$ }	
		% look up x	 		(== current else value)	
		% look up x_then	(== current then value)
		\STATE $(\loc_\x, \Type) = \gamma(\x)$
		\STATE $(\loc_t, \Type) = \gamma(\x_t)$	
		% Look up x/t	
		\STATE $(\byte_{x}, \Type, \nl, \VarPermL(\PermF, \Type, \Priv, \nl)) = \sigma(\loc_\x)$
		\STATE $(\byte_{t}, \Type, \nl, \VarPermL(\PermF, \Type, \Priv, \nl)) = \sigma(\loc_t)$
		% record touched locs
		\STATE $\locL = \locL \addL [(\loc_\x, 0), (\loc_t, 0)]$
		% based on type look up in sigma, add 2-tuple V = V::[(v_t, v_e)] 
		% Maintain this constant ordering to ensure correct storage later on -> X[i] == V[i]
		\IF{ $(\Type = \Priv\ \btype)$ }	
			% Decode x/t
			\STATE $\val_{x} = \Decode(\Priv\ \btype, \byte_{x})$
			\STATE $\val_{t} = \Decode(\Priv\ \btype, \byte_{t})$
			% Add to V
			\STATE $\valL = \valL.\mathit{append}((\val_{t}, \val_{x}))$
		\ELSIF{ $(\Type = \Priv\ \Const\ \btype*)$ }	
			% Decode x/t
			\STATE $[1, [(\loc_{xa}, 0)], [1], 1] = \DecodePtr(\Type, 1, \byte_{x})$
			\STATE $[1, [(\loc_{ta}, 0)], [1], 1] = \DecodePtr(\Type, 1, \byte_{t})$
			% Look up xa/ta	
			\STATE $(\byte_{xa}, \Priv\ \btype, \nl, \VarPermL(\PermF, \Priv\ \btype, \Priv, \nl)) = \sigma(\loc_{xa})$
			\STATE $(\byte_{ta}, \Priv\ \btype, \nl, \VarPermL(\PermF, \Priv\ \btype, \Priv, \nl)) = \sigma(\loc_{ta})$
			\FORALL{$i \in \{0 ... \nl-1\}$}
				% Decode x/t
				\STATE $\val_{xi} = \DecodeArr(\Priv\ \btype, i, \byte_{xa})$
				\STATE $\val_{ti} = \DecodeArr(\Priv\ \btype, i, \byte_{ta})$
				% Add to V
				\STATE $\valL = \valL.\mathit{append}((\val_{ti}, \val_{xi}))$
				% record touched locs
				\STATE $\locL = \locL \addL [(\loc_{xa}, i), (\loc_{ta}, i)]$
			\ENDFOR
		\ELSIF{$(\Type = \Priv\ \btype*)$}	
			% Decode x/t
			\STATE $[\nl, \locL_\x, \tagbL_\x, \indir] = \DecodePtr(\Type, \nl, \byte_{x})$
			\STATE $[\nl, \locL_t, \tagbL_t, \indir] = \DecodePtr(\Type, \nl, \byte_{t})$
			% Add to V	
			\STATE $\valL = \valL.\mathit{append}(([\nl, \locL_t, \tagbL_t, \indir], [\nl, \locL_\x, \tagbL_\x, \indir]))$
		\ENDIF	
	\ENDFOR
	\RETURN $(\valL, \n_\res, \locL)$	
\end{algorithmic}
\end{algorithm*}


\begin{algorithm*}\footnotesize
\caption{$(\sigma_1, \locL) \gets \ResolveS(\x_{\vl}, \Acc, \gamma, \sigma, \valL)$}
\label{algo: resolve S}
\begin{algorithmic}
	\STATE $\locL = [\ ]$
	\STATE $\sigma_1 = \sigma$
	\FORALL{ $i \in \{0...|\valL| -1\}$ }			
		% look up x	 		(== current else value)	
		% look up x_then	(== current then value)
		\STATE $\x = \x_{\vl}[i]$
		\STATE $\val_\x = \valL[i]$
		\STATE $(\loc_\x, \Type) = \gamma(\x)$
		% record touched locs
		\STATE $\locL = \locL.\mathit{append}((\loc_\x, 0))$
		% based on type look up in sigma, add 2-tuple V = V::[(v_t, v_e)] 
		% Maintain this constant ordering to ensure correct storage later on -> X[i] == V[i]
		\IF{ $(\Type = \Priv\ \btype)$ }	
			% Update x	
			\STATE $\sigma_2 = \Update(\sigma_1,\ \loc_\x,\ \val_\x,\ \Type)$
			\STATE $\sigma_1 = \sigma_2$
		\ELSIF{ $(\Type = \Priv\ \Const\ \btype*)$ }	
			% Decode x/t
			\STATE $[1, [(\loc_{xa}, 0)], [1], 1] = \DecodePtr(\Type, 1, \byte_{x})$
			\FORALL{$\offset \in \{0 ... \nl-1\}$}
				\STATE $\val_\offset = \val_\x[\offset]$
				% Update offset
				\STATE $\sigma_{2+\offset} = \UpdateArr(\sigma_{1+\offset},\ (\loc_{xa},\offset),\ \val_\offset,\ \Type)$
				% record touched locs
				\STATE $\locL = \locL.\mathit{append}((\loc_{xa}, \offset))$
			\ENDFOR
			\STATE $\sigma_1 = \sigma_{2+\offset}$
		\ELSIF{$(\Type = \Priv\ \btype*)$}	
			\STATE $\sigma_2 = \UpdatePtr(\sigma_1,\ (\loc_\x, 0),\ \val_\x,\ \Type)$ 
			\STATE $\sigma_1 = \sigma_2$
		\ENDIF	
	\ENDFOR
	\RETURN $(\sigma_1, \locL)$	
\end{algorithmic}
\end{algorithm*}





\subsection{Array Overshooting}
\label{app: array oob}

\begin{minipage}{\textwidth}
\centering{\includegraphics[width=0.75\textwidth]{MemoryOvershootingVanC.pdf}}
\captionof{figure}{Types of overshooting array accesses.}
\label{Fig: arr oob read vanC}
\end{minipage}

It is possible to overshoot memory bounds in both \vanillaC\ and \piccoC. 
Figure~\ref{Fig: arr oob read vanC} shows an example of an array read that overshoots the bounds of the array \TT{x} (for simplicity, \TT{x} is of size 1).  
The first access shown  is an in-bounds access -- this is the default behavior of a  correct
program.  The second access is an out-of-bounds access of a memory block of a different type, but the same size. This data would be read as if it was the type of the array, and may not be meaningful.  This corresponds to an access where implicit conversions between types is possible, but not always correct.
The third out-of-bounds access corresponds to reading out of a memory block of the same type. This data would be meaningful from a type perspective, but the specific value read may not be semantically meaningful to the program. 
 The fourth out-of-bounds access is of a memory block of a different type of a smaller size. This read would grab the data from the smaller memory block, then grab data from the next memory block(s) to obtain the correct amount for the expected type.  In this situation a value, which may not be meaningful, is constructed from two, or more, values in memory.
 The last out-of-bounds access is of a larger memory block, not aligned. This read would obtain a portion of the data of the larger memory block, and read it as the type of the array, thereby reading a partial value from memory.

With \piccoC, when dealing with array overshooting, we have the added complexity of private data, which has a different representation and is of a larger size than the corresponding C representation of the type. 
Additionally, we need to ensure that no leakage can occur, so we must consider all possible combinations of bytes from public and private data with either public or private variables. 
Consider reading a value from an overshot array and storing it into a variable. If both the data read and variable are private or both are public, no leakage can occur as these are the default cases.
Next, consider reading public data and storing into a private variable. The public data will be grabbed at the byte-level, and interpreted as though it is private (no encryption will occur), so no leakage occurs.
Third, we consider reading private data and storing in a public variable. The private data will be grabbed at the byte-level, and interpreted as though it is public. No decryption will occur, so no leakage can occur. This is similar in nature to reading a partial value in Figure~\ref{Fig: arr oob read vanC}. 
Fourth, consider if the data read is a mix of public and private data and stored in a public variable. Given that the private data will not be decrypted, this read will not result in any leakage, but a value is constructed from a mix
of private (encrypted) data and public data. 
Lastly, consider reading a mix of public and private data and storing into a private variable. Like before, the byte-level data will be merged and read as the expected type. 


Writes that occur out-of-bounds of an array have situations similar to out-of-bound read accesses (and can be illustrated as with the reads shown in Figure~\ref{Fig: arr oob read vanC}).
Writing private data out-of-bounds to a private location results in the data still residing in a private memory block, so no leakage will occur. 
Writing public data out-of-bounds to a public location is safe, as the data is already public. 
When writing private data out-of-bounds to a public location, the data will be written as-is -- no decryption will occur when the data is written to or later read back from that location -- therefore, there is no leakage.
Writing public data out-of-bounds to a private location or a mix of public and private locations is safe, as the data was already public; no encryption will occur. 
Lastly, writing private data out-of-bounds to a mix of locations will result in the data being written to the locations as-is. No decryption will occur when the data is written to any location or later read back, therefore, there is no leakage.

In \piccoC, we ensure this behavior, 
using algorithms $\ReadOOB$ and $\WriteOOB$. 
In particular, $\ReadOOB$ ensures that no matter what mix of byte-wise data we grab from memory, we will decode it as a value of the type of data in the array, ignoring it's true type. 
Similarly, $\WriteOOB$ ensures that we will write to memory the byte-wise encoding of the given value as the type for the array, without taking into consideration the type of the memory block(s) and without modifying any of the metadata within the memory block(s) we write to. 
In proving the correctness of \piccoC\ with relation to \vanillaC, the various possible alignments for reading and writing out-of-bounds poses complications due to the different sizes of private and public data (an example of this is shown in the Appendix, Figure~\ref{fig: overshooting alignment}). Therefore, we can only prove correctness over well-aligned accesses (i.e., those that iterate only over aligned elements of the same type). 
When proving noninterference, we must prove that these cases (particularly those involving private data) cannot leak any information about the private data that is affected. We discuss this in more detail in
the following section.





















%%%%%%%%%%%%%%%%%
%								 %
% 		Metatheory				 %
%								 %
%%%%%%%%%%%%%%%%%
\section{Metatheory}
\label{app: metatheory}
In this section, we will provide our main definitions and metatheory, including proof sketches of the most crucial Theorems and Lemmas, to give the reader a more complete understanding and some intuition behind how our proofs work. 

\subsection{Correctness}
In our semantics, we give each evaluation an identifying code as a shorthand way to refer to that specific evaluation, as well as to allow us to quickly reason about the \vanillaC\ and \piccoC\ evaluations that are congruent to each other (i.e., a \vanillaC\ rule and an identical one handling only public data in \piccoC). 




The list of \vanillaC\ codes are as follows: 
$\vanillaCodes$ = 
	[$\mathit{mpb}$, $\mathit{mpcmpt}$, $\mathit{mpcmpf}$, $\mathit{mppin}$, 
	$\mathit{mpra}$, $\mathit{mpwe}$, $\mathit{mpfre}$, $\mathit{mpiet}$, $\mathit{mpief}$,
	$\mathit{mprdp}$, $\mathit{mprdp1}$, $\mathit{mpwdp}$, $\mathit{mpwdp1}$, 
	$\mathit{fls}$, $\mathit{ss}$, $\mathit{sb}$, $\mathit{ep}$, $\mathit{cv}$, $\mathit{cl}$, 
	$\mathit{r}$, $\mathit{w}$, $\mathit{ds}$, $\mathit{dv}$, $\mathit{dp}$, $\mathit{da}$, 
	$\mathit{wle}$, $\mathit{wlc}$, $\mathit{bp}$, $\mathit{bs}$, $\mathit{bm}$, $\mathit{bd}$, 
	$\mathit{ltf}$, $\mathit{ltt}$, $\mathit{eqf}$, $\mathit{eqt}$, $\mathit{nef}$, $\mathit{net}$, 
	$\mathit{mal}$, $\mathit{fre}$, $\mathit{wp}$, $\mathit{wdp}$, $\mathit{wdp1}$, $\mathit{rp}$, 
	$\mathit{rdp}$, $\mathit{rdp1}$, $\mathit{ra}$, $\mathit{wa}$, $\mathit{rao}$, $\mathit{wao}$, 
	$\mathit{rae}$, $\mathit{wae}$, $\mathit{loc}$, $\mathit{iet}$, $\mathit{ief}$,  
	$\mathit{inp}$, $\mathit{inp1}$, $\mathit{out}$, $\mathit{out1}$, $\mathit{df}$, $\mathit{ty}$, 
	$\mathit{fd}$, $\mathit{fpd}$, $\mathit{fc}$, $\mathit{pin}$, $\mathit{pin1}$, $\mathit{pin2}$]. 


The list of \piccoC\ codes are as follows: 
$\piccoCodes$ = 
	[$\mathit{mpb}$, $\mathit{mpcmp}$, 
	$\mathit{mpra}$, $\mathit{mpwa}$, $\mathit{mppin}$, $\mathit{mpdp}$, $\mathit{mpdph}$, 
	$\mathit{mpfre}$, 
	$\mathit{mprdp}$, $\mathit{mprdp1}$, 
	$\mathit{mpwdp}$, $\mathit{mpwdp1}$, $\mathit{mpwdp2}$, $\mathit{mpwdp3}$, 
	$\mathit{iet}$, $\mathit{ief}$, $\mathit{iep}$, $\mathit{iepd}$, $\mathit{wle}$, $\mathit{wlc}$, 
	$\mathit{dp}$, $\mathit{dp1}$, 
	$\mathit{rp}$, $\mathit{rp1}$, $\mathit{rdp}$, $\mathit{rdp1}$, $\mathit{rdp2}$, 
	$\mathit{wp}$, $\mathit{wp1}$, $\mathit{wp2}$, 
	$\mathit{wdp}$, $\mathit{wdp1}$, $\mathit{wdp2}$, $\mathit{wdp3}$, $\mathit{wdp4}$, $\mathit{wdp5}$, 
	$\mathit{da}$, $\mathit{da1}$, $\mathit{das}$, 
	$\mathit{ra}$, $\mathit{ra1}$, $\mathit{rea}$, 
	$\mathit{wa}$, $\mathit{wa1}$, $\mathit{wa2}$, 
	$\mathit{wea}$, $\mathit{wea1}$, $\mathit{wea2}$, 
	$\mathit{rao}$, $\mathit{rao1}$, 
	$\mathit{wao}$, $\mathit{wao1}$, $\mathit{wao2}$, 
	$\mathit{pin}$, $\mathit{pin1}$, $\mathit{pin2}$, $\mathit{pin3}$, $\mathit{pin4}$, 
	$\mathit{pin5}$, $\mathit{pin6}$, $\mathit{pin7}$, 
	$\mathit{mal}$, $\mathit{malp}$, $\mathit{fre}$, $\mathit{pfre}$, 
	$\mathit{cv}$, $\mathit{cv1}$, $\mathit{cl}$, $\mathit{cl1}$, 
	$\mathit{loc}$, $\mathit{ty}$, 
	$\mathit{df}$, $\mathit{fd}$, $\mathit{fpd}$, $\mathit{fc}$, $\mathit{fc1}$, 
	$\mathit{bp}$, $\mathit{bs}$, $\mathit{bm}$, $\mathit{bd}$, 
	$\mathit{ltf}$, $\mathit{ltt}$, $\mathit{eqf}$, $\mathit{eqt}$, $\mathit{nef}$, $\mathit{net}$, 
	$\mathit{dv}$, $\mathit{d1}$, $\mathit{r}$, $\mathit{r1}$, $\mathit{w}$, $\mathit{w1}$, $\mathit{w2}$, 
	$\mathit{ds}$, $\mathit{ss}$, $\mathit{sb}$, $\mathit{ep}$, 
	$\mathit{inp}$, $\mathit{inp1}$, $\mathit{inp2}$, $\mathit{inp3}$, 
	$\mathit{out}$, $\mathit{out1}$, $\mathit{out2}$, $\mathit{out}$]. 







\begin{figure*}[h]
\begin{subfigure}{0.55\textwidth}
\includegraphics[width=\textwidth]{WellAlignedEx.pdf}
\caption{Well-aligned accesses}
\label{fig: well-aligned}
\end{subfigure}
\quad
\begin{subfigure}{0.35\textwidth}
\includegraphics[width=\textwidth]{NotWellAlignedEx.pdf}
\caption{Not well-aligned accesses}
\label{fig: not well-aligned}
\end{subfigure}
\caption{Examples of alignment between \piccoC\ and \vanillaC\ in overshooting accesses by incrementing pointer \TT{p} three times.}
\label{fig: overshooting alignment}
\end{figure*}

\subsubsection{Erasure Function}




Here, we show the full erasure function in Figure~\ref{Fig: app erasure}. 
This function is intended to take a \piccoC\ program or configuration and remove all private privacy labels, decrypt any private data, and clear any additional tracking features that are specific to \piccoC; this process will result in a \vanillaC\ program or configuration. 




\begin{figure*}
\footnotesize
\begin{minipage}{0.5\textwidth}
% STMT
\begin{subfigure}{\textwidth}
$\begin{array}{l}
\bm{\erasure(\rstmt)} =  \\ 
	\mid\ \RT{\x[\Expr]}\ =>\ \bm{\x[\erasure(\rExpr)]}  \\ 
	\mid\ \RT{\x(\Elist)}\ =>\ \bm{\x(\erasure(\RT{\Elist}))} \\ 
	\mid\ \RT{\Expr_1\ \binop\ \Expr_2}\ =>\ \bm{\erasure(\RT{\Expr_1})\ \binop\ \erasure(\RT{\Expr_2})}  \\ 
	\mid\ \RT{\unop\ \x}\ =>\ \bm{\unop\ \x}  \\ 
	\mid\ \RT{( \Expr )}\ =>\ \bm{(\erasure(\rExpr))}  \\ 
	\mid\ \RT{(\Type)\ \Expr}\ =>\ \bm{\erasure(\rType))\ \erasure(\rExpr)} \\ 
	\mid\ \RT{\var = \Expr}\ =>\ \bm{\erasure(\RT\var) = \erasure(\rExpr)}  \\ 
	\mid\ \RT{*\x = \Expr}\ =>\ \bm{*\x = \erasure(\rExpr)}  \\ 
	\mid\ \RT{\stmt_1;\ \stmt_2}\ =>\ \bm{\erasure(\RT{\stmt_1});\ \erasure(\RT{\stmt_2})}  \\ 
	\mid\ \RT{\{ \stmt \}}\ =>\ \bm{\{ \erasure(\rstmt) \}}  \\ 
	\mid\ \RT{\free(\Expr)}\ =>\ \bm{\free(\erasure(\rExpr))}  \\ 
	\mid\ \RT{\pfree(\Expr)}\ =>\ \bm{\free(\erasure(\rExpr))}  \\ 
	\mid\ \RT{\sizeof(\Type)}\ =>\ \bm{\sizeof(\erasure(\rType))}  \\ 
	\mid\ \RT{\Malloc(\Expr)}\ =>\ \bm{\Malloc(\erasure(\rExpr))}  \\ 
	\mid\ \RT{\PMalloc(\Expr,\ \Type)}\ =>\ \\ \ \quad \bm{\Malloc(\sizeof(\erasure(\rType)) \cdot \erasure(\rExpr))}  \\ 
	\mid\ \RT{\smcinput(\Elist)}\ =>\ \\ \ \quad \bm{\inputFun(\erasure(\RT\Elist))}  \\ 
	\mid\ \RT{\smcoutput(\Elist)}\ =>\ \\ \ \quad \bm{\outputFun(\erasure(\RT\Elist))} \\ 
	\mid\ \RT{[\val_0, ..., \val_n]} =>\ \\ \ \quad  \bm{[\erasure(\RT{\val_0}),\ \erasure(\RT{...}),\ \erasure(\RT{\val_n})]} \\ 
	\mid\ \RT{\Type\ \var}\ =>\ \\ \ \quad  \bm{\erasure(\rType)\ \erasure(\RT{\var})}  \\ 
	\mid\ \RT{\Type\ \var = \Expr}\ =>\ \\ \ \quad  \bm{\erasure(\rType)\ \erasure(\RT{\var}) = \erasure(\rExpr)}  \\ 
	\mid\ \RT{\Type\ \x(\plist)}\ =>\ \\ \ \quad  \bm{\erasure(\rType)\ \x(\erasure(\RT{\plist}))}  \\ 
	\mid\ \RT{\Type\ \x (\plist)\ \{ \stmt\}}\ =>\ \\ \ \quad  \bm{\erasure(\RT{\Type\ \x (\plist)})\ \{ \erasure(\rstmt) \} }  \\ 
	\mid\ \RT{\If (\Expr)\ \stmt_1\ \Else\ \stmt_2}\ =>\ \\ \ \quad \bm{\If (\erasure(\rExpr))\ \erasure(\RT{\stmt_1})\ \Else\ \erasure(\RT{\stmt_2})}  \\ 
	\mid\ \RT{\While\ (\Expr)\ \stmt}\ \\ \ \ \ =>\ \bm{\While\ (\erasure(\rExpr))\ \erasure(\rstmt)}  \\ 
	\mid\ \RT\_\ =>\ \bm{\stmt}
\end{array}$
\caption{Erasure function over statements} 	\label{Fig: erasure stmt}
\end{subfigure}
\end{minipage}
%
\begin{minipage}{0.4\textwidth}
% CONFIG
\begin{subfigure}{\textwidth}
$\begin{array}{l}
\bm{\erasure(\RT\Config)} = 
	\\ \mid \RT{\Config_1} \Mid \RT{\Config_2}\ =>\ \bm{\erasure(\RT{\Config_1})} \Mid \bm{\erasure(\RT{\Config_2})}
	\\ \mid (\RT\pid, \rrgamma,\ \rrsigma,\ \RT{\DMap}, \rAcc,\ \rstmt) \ =>\
		\\ \ \ \ \bm{(\pid, \erasure(\rrgamma, \rrsigma, [\ ], [\ ]), \bsq, \bsq, \erasure(\rstmt))}
\end{array}$
\caption{Erasure function over configurations} 	\label{Fig: erasure config}
\end{subfigure}
\\ \\ \\
% TYPES  /  LISTS
\begin{subfigure}{\textwidth}
$\begin{array}{l}
\bm{\erasure(\rType)} =  \\  	
	\mid\ \rlabel\ \rbtype\ =>\ \bm{\btype}  \\ 
	\mid\ \RT{\rlabel\ \rbtype\ *}\ =>\ \bm{\btype*}  \\ 
	\mid\ \RT{\Tlist \to \rType}\ =>\ \\ \ \quad \bm{\erasure(\RT{\Tlist}) \to \erasure(\rType))}  \\ 
	\mid\ \RT\_\ => \bm{\Type} \\ \\
\bm{\erasure(\rTlist)} =  \\ 
	\mid\ \RT{[\ ]}\ =>\ \bm{[\ ]}  \\ 
	\mid\ \RT{\Type::\Tlist}\ =>\ \bm{\erasure(\RT{\Type})::\erasure(\RT{\Tlist})} 
\end{array}$
\caption{Erasure function over types and type lists} 	\label{Fig: erasure ty}
\end{subfigure}
\\ \\ \\
% LISTS
\begin{subfigure}{\textwidth} 
$\begin{array}{l}
\bm{\erasure(\rElist)} = \\ 
	\mid\ \RT{\Elist,\ \Expr}\ =>\ \bm{\erasure(\RT{\Elist}),\ \erasure(\rExpr)}  \\ 
	\mid\ \rExpr\ =>\ \bm{\erasure(\rExpr)}  \\ 
	\mid\ \rVoid\ =>\ \bm{\Void} \\  \\
\bm{\erasure(\rplist)} =  \\ 
	\mid\ \RT{\plist,\ \Type\ \var}\ =>\ \\ \ \quad \bm{\erasure(\RT{\plist}),\ \erasure(\RT{\Type\ \var})}  \\ 
	\mid\ \RT{\Type\ \var}\ =>\ \bm{\erasure(\rType)\ \erasure(\RT\var)}  \\ 
	\mid\ \rVoid\ =>\ \bm{\Void}
\end{array}$
\caption{Erasure function over lists} 	\label{Fig: erasure list}
\end{subfigure}
\end{minipage}
\\ \\ \\
% BYTES
\centering
\begin{subfigure}{0.8\textwidth}
$\begin{array}{l}
\bm{\erasure(\rbyte,\ \rType,\ \rn)} = \\ 
% public variable / array => no modifications necessary
	\mid\ (\rbyte,\ \rPub\ \rbtype,\ \rn) =>\ \bm{\byte} 	\\ 
% private variable => need to decrypt stored values
	\mid\ (\rbyte,\ \rPriv\ \rbtype,\ \RT1) =>\  \\ \-\ \quad
		\RT{\val_1 = \Decode(\Type,\ 1,\ \byte)};\ 
		\val_2 = \RT{\Decrypt(\val_1)};\ 
		\byte_1 = \Encode(\btype,\ \val_2);\ 
		\bm{\byte_1}	\\ 
% private array	=> need to decrypt stored values
	\mid\ (\rbyte,\ \rPriv\ \rbtype,\ \rn) =>\  
		\RT{\val_1 = \Decode(\Type,\ n,\ \byte)};\ \\ \ \quad
		[\val_1', ..., \val_n'] = \RT{[\Decrypt(\val_1), \Decrypt(...), \Decrypt(\val_n)]};\ 
		\byte_1 = \Encode(\btype,\ [\val_1',\ ...,\ \val_n']);\ 
		\bm{\byte_1} \\ 	
% public pointer, single loc => need to remove labels
	\mid\ (\rbyte,\ \RT{\Pub\ \btype\ *},\ \RT1) =>\ 
		[\RT1,\ [(\rloc, \RT\offset)],\ [\RT1],\ \RT\indir] = \RT{\DecodePtr(\Pub\ \btype\ *,\ 1,\ \byte)};\ \\ \-\ \quad
		\byte_1 = \EncodePtr(\btype\ *,\ [1,\ [(\loc, \offset)],\ [1],\ \erasure(\RT{\Type'}),\ \indir]);\ 
		\bm{\byte_1} \\ 
% private pointer, single loc => need to remove labels, reduce size of offset
	\mid\ (\rbyte,\ \RT{\Priv\ \btype\ *},\ \RT1) =>\ 
		[\RT1,\ [(\rloc, \RT\offset)],\ [\RT1],\ \RT\indir] = \RT{\DecodePtr(\Priv\ \btype\ *,\ 1,\ \byte)};\ \\ \-\ \quad
		\If (\RT\indir = 1) \Then \{ \RT{\Type_1} = \rPub\ \rbtype; \RT{\Type_2} = \rPriv\ \rbtype \} \Else \{ \RT{\Type_1} = \rPub\ \rbtype\RT*; \RT{\Type_2} = \rPriv\ \rbtype\RT* \}; \ \\ \-\ \quad 
		\offset_1 = \frac{\RT\offset \cdot \tau(\RT{\Type_1})}{\tau(\RT{\Type_2})};\  
		\byte_1 = \EncodePtr(\btype\ *,\ [1,\ [(\loc, \offset_1)],\ [1],\ \erasure(\RT{\Type'}),\ \indir]);\ 
		\bm{\byte_1} \\ 
% private pointer, multiple locs => need to remove labels, find true loc if multiple, reduce size of offset
	\mid\ (\rbyte,\ \RT{\Priv\ \btype\ *},\ \rn) =>\ 
		[\RT\nl,\ \RT\locL,\ \RT\tagbL,\ \RT\indir] = \RT{\DecodePtr(\Priv\ \btype\ *,\ n,\ \byte)};\  \\ \-\ \quad
		(\loc, \RT\offset) = \rrDeclassifyPtr([\RT\nl,\ \RT\locL,\ \RT\tagbL,\ \RT\indir],\ \RT{\rPriv\ \btype*});\ \\ \-\ \quad
		\If (\RT\indir = 1) \Then \{ \RT{\Type_1} = \rPub\ \rbtype; \RT{\Type_2} = \rPriv\ \rbtype \} \Else \{ \RT{\Type_1} = \rPub\ \rbtype\RT*; \RT{\Type_2} = \rPriv\ \rbtype\RT* \}; \ \\ \-\ \quad
		\offset_1 = \frac{\RT\offset \cdot \tau(\RT{\Type_1})}{\tau(\RT{\Type_2})};\  
		\byte_1 = \EncodePtr(\btype\ *,\ [1,\ [(\loc, \offset_1)],\ [1],\ \ \indir]);\ 
		\bm{\byte_1} \\ 
% function  => need to call erasure on stored param types and statement
	\mid\ (\rbyte,\ \RT{\Tlist \to \Type},\ \RT1) =>\  \\ \-\ \quad
		\RT{(\rstmt,\ \rn,\ \rplist) = \DecodeFun(\byte)};\ % \\ \-\ \qq
		\byte_1 = \EncodeFun(\erasure(\rstmt),\ \bsq,\ \erasure(\rplist));\ 
		\bm{\byte_1}
\end{array}$
\caption{Erasure function over bytes} 	\label{Fig: erasure bytes}
\end{subfigure}
\caption{The Erasure function, broken down into various functionalities. } 	\label{Fig: app erasure}
\end{figure*}




Figure~\ref{Fig: erasure config} shows erasure over an entire configuration, calling $\erasure$ on the four-tuple of the environment, memory, and two empty maps needed as the base for the \vanillaC\ environment and memory; removing the accumulator (i.e., replacing it with $\Box$); and calling $\erasure$ on the statement. 
%
%
Figure~\ref{Fig: erasure ty} shows erasure over types and type lists (i.e., for function types).
Here, we remove any privacy labels given to the types, with unlabeled types being returned as is. 
For function types, we must iterate over the entire list of types as well as the return type. 
%
%
Figure~\ref{Fig: erasure list} shows erasure over expression lists (i.e., from function calls) and parameter lists (i.e., from function definitions). 

Figure~\ref{Fig: erasure stmt} shows erasure over statements. 
For statements, we case over the various possible statements. 
When we reach a private value (i.e., $\Encrypt(n)$), we decrypt and then return the decrypted value. 
For function $\PMalloc$, we replace the function name with $\Malloc$, modifying the argument to appropriately evaluate the expected size of the type. 
For functions $\pfree$, $\smcinput$, and $\smcoutput$, we simply replace the function name with its \vanillaC\ equivalent. 
All other cases recursively call the erasure function as needed, with the last case $(\_)$ handling all cases that are already identical to the \vanillaC equivalent (i.e., $\Null$, locations). 

Figure~\ref{Fig: erasure bytes} shows erasure over bytes stored in memory, which is used from within the erasure on the environment and memory. 
This function takes the byte-wise data representation, the type that it should be interpreted as, and the size expected for the data. 
For regular public types, we do not need to modify the byte-wise data. 
For regular private types (i.e., single values and array data), we get back the value(s) from the representation, decrypt, and obtain the byte-wise data for the decrypted value(s). 
For pointers with a single location, we must get back the pointer data structure, then simply remove the privacy label from the type stored there. 
For private pointers with multiple locations, we must declassify the pointer, retrieving it's true location and returning the byte-wise data for the pointer data structure with only that location. 
For functions, we get back the function data, then call $\erasure$ on the function body, remove the tag for whether the function has public side effects (i.e., replace with $\Box$), and call $\erasure$ on the function parameter list. 





% ENV  /  MEM
\begin{figure*}\footnotesize
$\begin{array}{l}
\bm{\erasure(\rrgamma,\ \rrsigma, \hgamma, \hsigma)} = \\ 
	\mathrm{match}\ (\rrgamma, \rrsigma)\ \mathrm{with} \\ 
% empty, empty
	\mid\ (\RT{[\ ]}, \RT{[\ ]})\ =>\ \bm{(\hgamma,\ \hsigma)} \\ 
% empty, allocated pub
	\mid\ (\RT{[\ ]}, \RT{\sigma_1[\loc \to (\Null,\ \Void*,\ n,\ \PermL(\PermF,\ \Void*,\ \Pub,\ n))])}\ =>\ \\ \-\ \quad
		 \bm{(\erasure(\RT{[\ ]},\ \rsigma{_1},\ \hgamma,\ \hsigma[\loc \to (\Null,\ \Void*,\ \hn,} \
			\bm{\PermL(\perm,\ \Void*,\ \Pub,\ \hn))]))} 
	\\ 
% empty, allocated priv
	\mid\ (\RT{[\ ]}, \RT{\sigma_1[\loc \to (\Null,\ \RT{\Void*},\ n,\ \PermL(\PermF,\ \RT{\Type},\ \Priv,\ n))])}\ =>\ \\ \-\ \quad
		\hn = \Big(\frac{\rn}{\tau(\rType)} \Big) \cdot \tau(\erasure(\rType)) \\ \-\ \quad
		\bm{(\erasure(\RT{[\ ]},\ \rsigma{_1},\ \hgamma,\ \hsigma[\loc \to (\Null,\ \Void*,\ \hn,} \
			\bm{\PermL(\perm,\ \Void*,\ \Pub,\ \hn))]))} 
	\\ 
% empty, var/array
	\mid\ (\RT{[\ ]}, \RT{\sigma_1[\loc \to (\byte,\ \Type,\ n,\ \PermL(\perm,\ \Type,\ \llabel,\ n))])}\ =>\ \\ \-\ \quad
		\bm{(\erasure(\RT{[\ ]}, \rsigma{_1}, \hgamma, \hsigma[\loc \to (\erasure(\rbyte, \rType, \rn), \erasure(\rType), n,}
			\bm{\PermL(\perm, \erasure(\rType), \Pub, n))]))} 
	\\ 
% empty, ptr
	\mid\ (\RT{[\ ]}, \RT{\sigma_1[\loc \to (\byte,\ \Type,\ n,\ \PtrPermL(\perm,\ \Type,\ \llabel,\ n))])}\ =>\ \\ \ \ \ 
		\bm{(\erasure(\RT{[\ ]}, \rsigma{_1}, \hgamma, \hsigma[\loc \to (\erasure(\rbyte, \rType, \rn), \erasure(\rType), n,}
			\bm{\PtrPermL(\perm, \erasure(\rType), \Pub, n))]))}
	\\ 
% empty, func
	\mid\ (\RT{[\ ]},\RT{\sigma_1[\loc \to (\byte,\ \Type,\ 1,\ \FunPermL(\Pub))])}\ =>\ \\ \-\ \quad
		\bm{(\erasure(\RT{[\ ]},\ \rsigma{_1},\ \hgamma,\ \hsigma[\loc \to (\erasure(\rbyte, \rType, \RT1),\ \erasure(\rType),\ 1,}\
			\bm{\FunPermL(\Pub))]))}
	\\ 
% var, var
	\mid\ (\RT{\gamma_1[\x \to (\loc,\ \llabel\ \btype)]},\ \rsigma{_1}\RT{[\loc \to (\byte,\ \llabel\ \btype,\ 1,\ \PermL(\perm,\ \llabel\ \btype,\ \llabel,\ 1))]})\ =>\ \\ \-\ \quad
	 	\bm{(\erasure(\rgamma{_1}, \rsigma{_1}, \hgamma[\x \to (\loc, \btype)], \hsigma[\loc \to}
			\bm{(\erasure(\rbyte, \rlabel\ \rbtype, \RT1), \btype, 1,} % \\ \-\ \qq\qquad
			\bm{\PermL(\perm, \btype, \Pub, 1))]))}
	\\ 
% RES var, var
	\mid\ (\RT{\gamma_1[\res\_n \to (\loc,\ \Priv\ \btype)]},\ \rsigma{_1}\RT{[\loc \to (\byte,\ \Priv\ \btype,\ 1,}\ \RT{\PermL(\perm,\ \Priv\ \btype,\ \Priv,\ 1))]})\ =>\ 
		\\ \-\ \quad
		\bm{(\erasure(\rgamma{_1},\ \rsigma{_1},\ \hgamma,\ \hsigma))}
	\\ 
% THEN var, var
	\mid\ (\RT{\gamma_1[\x\_then\_n \to (\loc, \llabel\ \btype)]}, \rsigma{_1}\RT{[\loc \to (\byte, \llabel\ \btype, 1,} \RT{\PermL(\perm, \llabel\ \btype, \llabel, 1))]})\ =>\ \\ \-\ \quad
		%\\ \-\ \qquad
		\bm{(\erasure(\rgamma{_1}, \rsigma{_1}, \hgamma, \hsigma))}
	\\ 
% ELSE var, var
	\mid\ (\RT{\gamma_1[\x\_else\_n \to (\loc, \llabel\ \btype)]},\ \rsigma{_1}\RT{[\loc \to (\byte, \llabel\ \btype, 1,} \RT{\PermL(\perm, \llabel\ \btype, \llabel, 1))]})\ =>\ \\ \-\ \quad
	%\\ \-\ \qquad
		\bm{(\erasure(\rgamma{_1}, \rsigma{_1}, \hgamma, \hsigma))}
	\\ 
% array, array
	\mid\ (\RT{\gamma_1[\x \to (\loc,\ \llabel\ \Const\ \btype*)]},\ \rsigma{_1}\RT{[\loc \to (\byte,\ \llabel\ \Const\ \btype*,\ 1,}\ \RT{\PermL(\perm,\ \llabel\ \Const\ \btype*,\ \llabel,\ 1))]})\ =>\ 
	\\ \-\ \quad
		\RT{\DecodePtr(\llabel\ \Const\ \btype*, 1, \byte)} = \RT{[1,\ [(\loc_1, 0)],\ [1],\ 1]}; \\ \-\ \quad
			\rsigma{_1} = \rsigma{_2}\RT{[\loc_1 \to (\byte_1,\ \llabel\ \btype,\ n,\ \PermL(\perm,\ \llabel\ \btype,\ \llabel,\ n))]}; \\ \-\ \quad
		\bm{(\erasure(\rgamma{_1},\ \rsigma{_2},\ \hgamma[\x \to (\loc,\ \erasure(\llabel\ \Const\ \btype*))],}\ \\ \-\ \quad
			\bm{\hsigma[\loc \to (\erasure(\rbyte, \RT{\llabel\ \Const\ \btype*}, \RT1), \Const\ \btype*), 1,} 
				\bm{\PtrPermL(\perm,\ \Const\ \btype*,\ \Pub,\ 1))]} \
			\\ \-\ \quad
			\bm{[\loc_1 \to (\erasure(\RT{\byte_1}, \rlabel\ \rbtype, \rn), \btype, n,} %\\ \-\ \qq
				\bm{\PermL(\perm,\ \btype,\ \Pub,\ n))]))}
	\\ 
% THEN array, array
	\mid\ (\RT{\gamma_1[\x\_then\_n \to (\loc,\ \llabel\ \Const\ \btype*)]}, \rsigma{_1}\RT{[\loc \to (\byte, \llabel\ \Const\ \btype*, 1,} 
		\RT{\PtrPermL(\perm, \llabel\ \Const\ \btype*, \llabel, 1))]}) =>\ \\ \-\ \quad
		\RT{\DecodePtr(\llabel\ \Const\ \btype*, 1, \byte)} = \RT{[1,\ [(\loc_1, 0)],\ [1],\ 1]}; \\ \-\ \quad
			\rsigma{_1} = \rsigma{_2}\RT{[\loc_1 \to (\byte_1,\ \llabel\ \btype,\ n,\ \PermL(\perm,\ \llabel\ \btype,\ \llabel,\ n))]}; %\\ \-\ \qq
		\bm{(\erasure(\rgamma{_1},\ \rsigma{_2},\ \hgamma,\ \hsigma))}
	\\ 
% ELSE array, array
	\mid\ (\RT{\gamma_1[\x\_else\_n \to (\loc,\ \llabel\ \Const\ \btype*)]},\ \rsigma{_1}\RT{[\loc \to (\byte,\ \llabel\ \Const\ \btype*,\ 1,}\ 
		\RT{\PermL(\perm,\ \llabel\ \Const\ \btype*,\ \llabel,\ 1))]}) =>\ \\ \-\ \quad
		\RT{\DecodePtr(\llabel\ \Const\ \btype*, 1, \byte)} = \RT{[1,\ [(\loc_1, 0)],\ [1],\ 1]}; \\ \-\ \quad
			\rsigma{_1} = \rsigma{_2}\RT{[\loc_1 \to (\byte_1,\ \llabel\ \btype,\ n,\ \PermL(\perm,\ \llabel\ \btype,\ \llabel,\ n))]}; %\\ \-\ \qq
		\bm{(\erasure(\rgamma{_1},\ \rsigma{_2},\ \hgamma,\ \hsigma))}
	\\ 
% pointer, pointer
	\mid\ (\RT{\gamma_1[\x \to (\loc,\ \llabel\ \btype*)]},\ \rsigma{_1}\RT{[\loc \to (\byte,\ \llabel\ \btype*,\ n,}\ 
			\RT{\PtrPermL(\perm,\ \llabel\ \btype*,\ \llabel,\ n))]})\ =>\ \\ \-\ \quad
		 \bm{(\erasure(\rgamma{_1},\ \rsigma{_1},\ \hgamma[\x \to (\loc,\ \erasure(\llabel\ \btype*))],\ } \\ \-\ \quad
			\bm{\hsigma[\loc \to (\erasure(\rbyte, \rType, \rn),\ \erasure(\rType),\ n,}\ 
				\bm{\PtrPermL(\perm,\ \erasure(\rType),\ \Pub,\ n))]))}
	\\ 
% THEN pointer, pointer
	\mid\ (\RT{\gamma_1[\x\_then\_n \to (\loc,\ \llabel\ \btype*)]},\ \rsigma{_1}\RT{[\loc \to (\byte,\ \llabel\ \btype*,\ n,}\ 
			\RT{\PtrPermL(\perm,\ \llabel\ \btype*,\ \llabel,\ n))]})\ %
			=>\ \\ \-\ \quad
		\bm{(\erasure(\rgamma{_1},\ \rsigma{_1},\ \hgamma,\ \hsigma))}
	\\ 
% ELSE pointer, pointer
	\mid\ (\RT{\gamma_1[\x\_else\_n \to (\loc,\ \llabel\ \btype*)]},\ \rsigma{_1}\RT{[\loc \to (\byte,\ \llabel\ \btype*,\ n,}\ 
			\RT{\PtrPermL(\perm,\ \llabel\ \btype*,\ \llabel,\ n))]})\ %
			=>\ \\ \-\ \quad
		 \bm{(\erasure(\rgamma{_1},\ \rsigma{_1},\ \hgamma,\ \hsigma))}
	\\ 
% function, function
	\mid\ (\RT{\gamma_1[\x \to (\loc,\ \Tlist \to \Type)]}, \RT{\sigma_1[\loc \to (\byte,\ \Tlist \to \Type,\ 1,\ \FunPermL(\Pub))]}\ =>\ \\ \-\ \quad
		\bm{(\erasure(\rgamma{_1},\ \rsigma{_1},\ \hgamma[\x \to (\loc,\ \erasure(\RT{\Tlist \to \Type}))],}\ \\ \-\ \quad
			\bm{\hsigma[\loc \to (\erasure(\rbyte, \RT{\Tlist \to \Type}, \RT1),\ \erasure(\rTlist \to \rType),\ 1,\ \FunPermL(\Pub))]))}
\end{array}$
\caption{Erasure function over the environment and memory} 	\label{Fig: erasure env mem}
\end{figure*}







Figure~\ref{Fig: erasure env mem} shows erasure over the environment and memory. 
In order to properly handle all types of variables and data stored, we must iterate over both the \piccoC\ environment and memory maps, and pass along the \vanillaC\ environment and memory maps as we remove elements from the \piccoC\ maps and either add to them to the \vanillaC\ maps or discard them. 
The first case is the base case, when the \piccoC\ environment and memory are both empty, and we return the \vanillaC\ environment and memory. 
Next, we have three cases which continue to iterate through the \piccoC\ memory after the environment has been emptied. These cases are possible due to the fact that in \piccoC\ we remove mappings from the environment once they are out of scope, but we never remove mappings from memory. 

Then we have three cases to handle regular variables. The first adds mappings to the \vanillaC\ environment and memory without the privacy annotations on the types, and calls $\erasure$ on the byte-wise data stored at that location (the behavior of this is shown in Figure~\ref{Fig: erasure bytes} and described later in this section). The other two remove temporary variables (an their corresponding data) inserted by an \TT{if else} statement branching on private data. 
The cases for arrays, pointers, and functions behave similarly; however, when we have an array we handle the array pointer as well as the array data within those cases. 














\subsubsection{Selected Metatheory}

\begin{definition}[$\psi$] 
\label{Def: psi}
A \LocMap\ $\psi$ is defined as a list of lists of locations, in symbols $\psi = [\ ]\ |\ \psi[\locL]$,
that is formed by tracking which locations are privately switched during the execution of the statement $\RT{\pfree(\x)}$ in a \piccoC\ program $\rstmt$ to enable comparison with the \emph{congruent} \vanillaC\ program $\hstmt$. 
\end{definition}



\begin{definition}[$\rval \Pcong \hval$]
\label{Def: val psi cong}
A \piccoC\ value and \vanillaC\ value are \emph{$\psi$-congruent}, 
in symbols $\rval \Pcong \hval$, \\
if and only if either 
$\rval \neq (\rloc, \RT\offset)$, $\hval \neq (\hloc, \Hoffset)$ and $\rval \cong \hval$, \\
or 
$\rval = (\rloc, \RT\offset)$, $\hval = (\hloc, \Hoffset)$ and $(\rloc, \RT\offset) \Pcong (\hloc, \Hoffset)$. 
\end{definition}



\begin{definition}[$\RT\code \cong \codeV$]
\label{Def: code cong}
We define \emph{congruence} over \piccoC\ codes $\RT\code \in \piccoCodes$ and $\codeV \in \vanillaCodes$, in symbols $\RT\code \cong \codeV$, by cases as follows: 
\\ if $\RT\code = \codeV$, then $\RT\code \cong \codeV$, 
\\ if $\RT\code = \RT{\mathit{iep}} \lxor \RT{\mathit{iepd}}$, then $\codeV = \codeVV{mpiet} \lxor \codeVV{mpief}$ and $\RT\code \cong \codeV$, 
\\ if $\RT\code = \RT{\mathit{mpcmp}}$, then $\codeV = \codeVV{mpcmpt} \lxor \codeVV{mpcmpf}$ and $\code \cong \codeV$, 
\\ otherwise we have 
$[\RT{\mathit{malp}}] \cong [\codeVV{ty}, \codeVV{bm}, \codeVV{mal}]$, 
$\RT{\mathit{fc1}} \cong \codeVV{fc}$, $\RT{\mathit{pin3}} \cong \codeVV{pin}$, 
$\RT{\mathit{cl1}} \cong \codeVV{cl}$, $\RT{\mathit{mpwdp2}} \cong \codeVV{mpwdp1}$, 
$\RT{\mathit{cv1}} \cong \codeVV{cv}$, $\RT{\mathit{mpwdp}} \cong \codeVV{mpwdp}$, 
$\RT{\mathit{pin4}} \cong \codeVV{pin1}$, $\RT{\mathit{pin5}} \cong \codeVV{pin2}$, 
$\RT{\mathit{mpwdp3}} \cong \codeVV{mpwdp}$, $\RT{\mathit{pin6}} \cong \codeVV{pin1}$, 
$\RT{\mathit{pin7}} \cong \codeVV{pin2}$, $\RT{\mathit{r1}} \cong \codeVV{r}$, 
$\RT{\mathit{w1}} \cong \codeVV{w}$, $\RT{\mathit{w2}} \cong \codeVV{w}$, 
$\RT{\mathit{d1}} \cong \codeVV{d}$, $\RT{\mathit{wdp2}} \cong \codeVV{wdp1}$, 
$\RT{\mathit{dp1}} \cong \codeVV{dp}$, $\RT{\mathit{wdp3}} \cong \codeVV{wdp}$, 
$\RT{\mathit{rp1}} \cong \codeVV{rp}$, $\RT{\mathit{wdp4}} \cong \codeVV{wdp}$, 
$\RT{\mathit{wp1}} \cong \codeVV{wp}$, $\RT{\mathit{rdp1}} \cong \codeVV{rdp1}$, 
$\RT{\mathit{wp2}} \cong \codeVV{wp}$, $\RT{\mathit{da1}} \cong \codeVV{da}$, 
$\RT{\mathit{ra1}} \cong \codeVV{ra}$, $\RT{\mathit{wea2}} \cong \codeVV{wea}$, 
$\RT{\mathit{wea1}} \cong \codeVV{wea}$, $\RT{\mathit{rao1}} \cong \codeVV{rao}$, 
$\RT{\mathit{wa1}} \cong \codeVV{wa}$, $\RT{\mathit{wa2}} \cong \codeVV{wa}$, 
$\RT{\mathit{wa1p}} \cong \codeVV{wa}$, $\RT{\mathit{wa2p}} \cong \codeVV{wa}$, 
$\RT{\mathit{wao2}} \cong \codeVV{wao}$, $\RT{\mathit{wao1}} \cong \codeVV{wao}$, 
$\RT{\mathit{inp3}} \cong \codeVV{inp1}$, $\RT{\mathit{inp2}} \cong \codeVV{inp}$, 
$\RT{\mathit{out3}} \cong \codeVV{out1}$, and $\RT{\mathit{out2}} \cong \codeVV{out}$. 
\end{definition}








\begin{definition}[$\RT\Pi\Pcong\Sigma$]%\small
\label{Def: deriv cong}
Two derivations and \emph{$\psi$-congruent}, in symbols $\RT\Pi\cong_{\psi}\Sigma$, 
if and only if 
\\ $\RT\Pi \deriv ((\RT\pidA, \rrgamma^{\RT{\pidA}}_{},$ $\rrsigma^{\RT{\pidA}}_{},$ $\RT\DMap^{\RT{\pidA}}_{}$, $\rAcc^{\RT{\pidA}}_{},$ $\rstmt^{\RT{\pidA}})\ \Mid ...\Mid$ 
	$(\RT\pidZ, \rrgamma^{\RT{\pidZ}}_{},$ $\rrsigma^{\RT{\pidZ}}_{},$ $\RT\DMap^{\RT{\pidZ}}_{}$, $\rAcc^{\RT{\pidZ}}_{},$ $\rstmt^{\RT{\pidZ}}))$ 
	 \\$\Deval{\RT\locLL}{\RT\codeLL}$ 
	$((\RT\pidA, {\rrgamma^{\RT{\pidA}}_{\RT1}},$ ${\rrsigma^{\RT{\pidA}}_{\RT1}},$ $\RT\DMap^{\RT{\pidA}}_{\RT1}$, $\rAcc^{\RT{\pidA}}_{\RT1},$ ${\RT\val^{\RT{\pidA}}_{}}) \Mid ... \Mid$ 
	$(\RT\pidZ, {\rrgamma^{\RT{\pidZ}}_{\RT1}},$ ${\rrsigma^{\RT{\pidZ}}_{\RT1}},$ $\RT\DMap^{\RT{\pidZ}}_{\RT1}$, $\rAcc^{\RT{\pidZ}}_{\RT1},$ $\RT\val^{\RT{\pidZ}}))$
and 
\\$\Sigma \deriv ((\pidA,$ $\hgamma^\pidA,$ $\hsigma^\pidA,$ $\bsq,$ $\bsq,$ $\hstmt^\pidA)\Mid ...\Mid $
	$(\pidZ,$ $\hgamma^\pidZ,$ $\hsigma^\pidZ,$ $\bsq,$ $\bsq,$ $\hstmt^\pidZ))$ 
\\	$\Deval{}{\codeVLL}$ 
	$((\pidA,$ $\hgamma^\pidA_1,$ $\hsigma^\pidA_1,$ $\bsq,$ $\bsq,$ $\hval^\pidA)\Mid...\Mid$
	$(\pidZ,$ $\hgamma^\pidZ_1,$ $\hsigma^\pidZ_1,$ $\bsq,$ $\bsq,$ $\hval^\pidZ))$
such that 
\\$\{(\RT\pid, \rrgamma^{\RT{\pid}}_{},$ $\rrsigma^{\RT{\pid}}_{},$ $\RT\DMap^{\RT{\pid}}_{}$, $\rAcc^{\RT{\pid}}_{},$ $\rstmt^{\RT{\pid}})$ $\cong_{\psi_1}$ 
$(\pid,$ $\hgamma^\pid,$ $\hsigma^\pid,$ $\bsq,$ $\bsq,$ $\hstmt^\pid)\}^{\pidZ}_{\pid = \pidA}$, 
$\RT\codeLL \cong \codeVLL$, and 
\\$\{(\RT\pid, {\rrgamma^{\RT{\pid}}_{\RT1}},$ ${\rrsigma^{\RT{\pid}}_{\RT1}},$ $\RT\DMap^{\RT{\pid}}_{\RT1}$, $\rAcc^{\RT{\pid}}_{\RT1},$ $\RT\val^{\RT{\pid}}_{})$ $\cong_{\psi}$ 
$(\pid,$ $\hgamma^\pid_1,$ $\hsigma^\pid_1,$ $\bsq,$ $\bsq,$ $\hval^\pid)\}^{\pidZ}_{\pid = \pidA}$ such that $\psi$ was derived from $\psi_1$ and the derivation $\RT\Pi$. 
\end{definition}



\begin{definition}[$\RT{\val^1}\sim\RT{\val^2}$]%\small
\label{def: val sim}
Two values are \emph{corresponding}, in symbols $\RT{\val^1}\sim\RT{\val^2}$, 
if and only if either both $\RT{\val^1},\RT{\val^2}$ are public (including locations) and $\RT{\val^1}=\RT{\val^2}$, 
or $\RT{\val^1},\RT{\val^2}$ are private and $\erasure(\RT{\val^1}) = \erasure(\RT{\val^2})$.
\end{definition}



\begin{definition}[$\RT{\Config^1} \sim \RT{\Config^2}$]%\small
\label{def: config sim}
Two configurations are \emph{corresponding}, in symbols $\RT{\Config^1} \sim \RT{\Config^2}$ or 
$(\RT1, \RT{\gamma^1}, \RT{\sigma^1}, \RT{\DMap^1}, \RT{\Acc^1}, \RT{\stmt^1})$ $\sim$ $(\RT2, \RT{\gamma^2}, \RT{\sigma^2}, \RT{\DMap^2}, \RT{\Acc^2}, \RT{\stmt^2})$, 
if and only if 
$\RT{\gamma^1}=\RT{\gamma^2}$, $\RT{\sigma^1} \sim \RT{\sigma^2}$, $\RT{\DMap^1} \sim \RT{\DMap^2}$, $\RT{\Acc^1}=\RT{\Acc^2}$, and $\RT{\stmt^1}=\RT{\stmt^2}$. 
\end{definition}



\begin{axiom}[$MPC_b$]%\small
\label{axiom: mpc bop}
Given $\binop\in\{+,-,\cdot, \div\}$, values $\{\RT{\n^{\pid}_{1}},$ $\RT{\n^{\pid}_{2}}, \hn_{1},$ $\hn_{2}\}^{\pidZ}_{\pid=\pidA}\in\N$, 
\\
if $MPC_b(\binop,$ $[\RT{\n^{\pidA}_{1}}, ...,$ $\RT{\n^{\pidZ}_{1}}],$ $[\RT{\n^{\pidA}_{2}}, ..., \RT{\n^{\pidZ}_{2}}])$ $= (\RT{\n^{\pidA}_{3}}, ..., \RT{\n^{\pidZ}_{3}})$, 
$\{\RT{\n^{\pid}_{1}} \cong \hn_1\}^{\pidZ}_{\pid=\pidA}$, 
	and
$\{\RT{\n^{\pid}_{2}} \cong \hn_2\}^{\pidZ}_{\pid=\pidA}$,  
\\
then $\{\RT{\n^{\pid}_{3}} \cong \hn^\pid_3\}^{\pidZ}_{\pid=\pidA}$ 
such that $\hn_1\ \binop\ \hn_2 = \hn_3$. 
\end{axiom}


\begin{lemma}[$\RT{\Config^1} \sim \RT{\Config^2} \implies \RT{\Config^1}\Pcong\hConfig \land \RT{\Config^2}\Pcong\hConfig$]%\small
\label{lem: sim implies cong same}
Given two configurations $\RT{\Config^1},\RT{\Config^2}$ such that 
$\RT{\Config^1} = (\RT1, \RT{\gamma^1}, \RT{\sigma^1}, \RT{\DMap^1}, \RT{\Acc^1}, \RT{\stmt^1})$ and $\RT{\Config^2} = (\RT2, \RT{\gamma^2}, \RT{\sigma^2}, \RT{\DMap^2}, \RT{\Acc^2}, \RT{\stmt^2})$ and $\psi$, 
if $\RT{\Config^1} \sim \RT{\Config^2}$ 
then $\{\RT{\Config^\pid}\Pcong(\pid, \hgamma, \hsigma, \bsq, \bsq, \stmt)\}^2_{\pid=1}$. 
\end{lemma}
\begin{proof}[Proof Sketch]%\small
Using the definition of $\erasure$ and Definition~\ref{def: config sim}, there is only one possible \vanillaC\ configuration $\hConfig$ (modulo party ID) that can be obtained from both 
$\erasure(\RT{\Config^1})$ and $\erasure(\RT{\Config^2})$. 
\end{proof}




\begin{lemma}[Unique party-wise transitions]%\small
\label{lem: smc unique rules}
Given $((\RT\pid, \RT\gamma, \RT\sigma, \RT\DMap, \RT\Acc, \RT\stmt) \Mid \RT\Config)$
if $((\RT\pid, \RT\gamma, \RT\sigma, \RT\DMap, \RT\Acc, \RT\stmt) \Mid \RT\Config)$ $\Deval{\RT\locLL}{\RT\codeLL}$
   $((\RT\pid, \RT{\gamma_1}, \RT{\sigma_1}, \RT{\DMap_1}, \RT\Acc, \RT\val) \Mid \RT{\Config_1})$
then there exists no other rule by which $(\RT\pid, \RT\gamma, \RT\sigma, \RT\DMap, \RT\Acc, \RT\stmt)$ can step. 
\end{lemma}
\begin{proof}[Proof Sketch]%\small
By induction on $(\RT\pid, \RT\gamma, \RT\sigma, \RT\DMap, \RT\Acc, \RT\stmt)$. 
We verify that for every configuration, given $\RT\stmt$, $\RT\Acc$, and stored type information, there is only one corresponding semantic rule.
\end{proof}







\begin{theorem}[Confluence]%\small
\label{thm: smc confluence}
Given $\RT{\Config^\pidA} \Mid ... \Mid \RT{\Config^\pidZ}$ such that $\{\RT{\Config^\pidA} \sim \RT{\Config^\pid}\}^{\RT\pidZ}_{\RT\pid = \RT\pidA}$ 
\\
if $(\RT{\Config^\pidA} \Mid ... \Mid \RT{\Config^\pidZ})$ $\Deval{\RT{\locLL_1}}{\RT{\codeLL_1}}$ $(\RT{\Config^\pidA_1} \Mid ... \Mid \RT{\Config^\pidZ_1})$ such that $\exists \RT\pid\in\{\RT\pidA...\RT\pidZ\} \RT{\Config^\pidA_1} \not\sim \RT{\Config^\pid_1}$, 
\\
then $\exists$ $(\RT{\Config^\pidA_1} \Mid ... \Mid \RT{\Config^\pidZ_1})$ $\Deval{\RT{\locLL_2}}{\RT{\codeLL_2}}$ $(\RT{\Config^\pidA_2} \Mid ... \Mid \RT{\Config^\pidZ_2})$
\\ 
such that $\{\RT{\Config^\pidA_2} \sim \RT{\Config^\pid_2}\}^{\RT\pidZ}_{\RT\pid = \RT\pidA}$, 
$\{(\RT{\locLL^\pidA_1}\addL\RT{\locLL^\pidA_2}) = (\RT{\locLL^\pid_1}\addL\RT{\locLL^\pid_2})\}^{\RT\pidZ}_{\RT\pid = \RT\pidA}$, 
and $\{(\RT{\codeLL^\pidA_1}\addC\RT{\codeLL^\pidA_2}) = (\RT{\codeLL^\pid_1}\addC\RT{\codeLL^\pid_2})\}^{\RT\pidZ}_{\RT\pid = \RT\pidA}$.
\end{theorem}

\begin{proof}[Proof Sketch]%\small
By Lemma~\ref{lem: smc unique rules}, we have that there is only one possible execution trace for any given party based on the starting configuration. 
~\\ 
By definition of $\{\RT{\Config^\pidA} \sim \RT{\Config^\pid}\}^{\RT\pidZ}_{\RT\pid = \RT\pidA}$, 
we have that the starting states of all parties are corresponding, with identical statements. 
~\\ 
Therefore, all parties must follow the same execution trace and will eventually reach another set of corresponding states.
\end{proof}




\begin{axiom}
\label{correctness assumption}%\small
For purposes of correctness, 
we assume all parties are executing program $\RT\stmt$ 
from initial state $(\RT\pid, \RT{[\ ]}, \RT{[\ ]}, \RT{[\ ]}, \RT\AccZ, \RT\stmt)$
and corresponding input data. 
\end{axiom}



\begin{theorem}[Semantic Correctness]%\small
\label{Thm: app correctness}
~\\
For every configuration $\{(\RT\pid,\ \rrgamma^{\RT{\pid}}_{},$ $\rrsigma^{\RT{\pid}}_{},$ $\RT\DMap^{\RT{\pid}}_{}$, $\rAcc^{\RT{\pid}}_{},$ $\rstmt^{\RT\pid})\}^{\RT\pidZ}_{\RT\pid = \RT\pidA}$, 
$\{(\pid,$ $\hgamma^\pid,$ $\hsigma^\pid,$ $\bsq,$ $\bsq,$ $\hstmt^\pid)\}^{\pidZ}_{\pid = \pidA}$ and \LocMap\ $\psi$ 
\\ such that $\{(\RT\pid, \rrgamma^{\RT{\pid}}_{},$ $\rrsigma^{\RT{\pid}}_{},$ $\RT\DMap^{\RT{\pid}}_{}$, $\rAcc^{\RT{\pid}}_{},$ $\rstmt^{\RT{\pid}})$ $\Pcong$ 
$(\pid,$ $\hgamma^\pid,$ $\hsigma^\pid,$ $\bsq,$ $\bsq,$ $\hstmt^\pid)\}^{\pidZ}_{\pid = \pidA}$, 
\\ % a 
if $\RT\Pi \deriv ((\RT\pidA, \rrgamma^{\RT{\pidA}}_{},$ $\rrsigma^{\RT{\pidA}}_{},$ $\RT\DMap^{\RT{\pidA}}_{}$, $\rAcc^{\RT{\pidA}}_{},$ $\rstmt^{\RT{\pidA}})\ \Mid ...\Mid$ 
	$(\RT\pidZ, \rrgamma^{\RT{\pidZ}}_{},$ $\rrsigma^{\RT{\pidZ}}_{},$ $\RT\DMap^{\RT{\pidZ}}_{}$, $\rAcc^{\RT{\pidZ}}_{},$ $\rstmt^{\RT{\pidZ}}))$ 
	\\ \-\ \-\ \-\ $\Deval{\RT\locLL}{\RT\codeLL}$ 
	$((\RT\pidA, {\rrgamma^{\RT{\pidA}}_{\RT1}},$ ${\rrsigma^{\RT{\pidA}}_{\RT1}},$ $\RT\DMap^{\RT{\pidA}}_{\RT1}$, $\rAcc^{\RT{\pidA}}_{\RT1},$ ${\RT\val^{\RT{\pidA}}_{}}) \Mid ... \Mid$ 
	$(\RT\pidZ, {\rrgamma^{\RT{\pidZ}}_{\RT1}},$ ${\rrsigma^{\RT{\pidZ}}_{\RT1}},$ $\RT\DMap^{\RT{\pidZ}}_{\RT1}$, $\rAcc^{\RT{\pidZ}}_{\RT1},$ $\RT\val^{\RT{\pidZ}}))$ 
\\ for codes $\RT\codeLL \in \RT{\piccoCodes}$,
then there exists a derivation 
\\ % b
$\Sigma \deriv\ ((\pidA,$ $\hgamma^\pidA,$ $\hsigma^\pidA,$ $\bsq,$ $\bsq,$ $\hstmt^\pidA)\Mid ...\Mid $
	$(\pidZ,$ $\hgamma^\pidZ,$ $\hsigma^\pidZ,$ $\bsq,$ $\bsq,$ $\hstmt^\pidZ))$ 
	\\ $\Deval{}{\codeVLL}$ 
	$((\pidA,$ $\hgamma^\pidA_1,$ $\hsigma^\pidA_1,$ $\bsq,$ $\bsq,$ $\hval^\pidA)\Mid...\Mid$
	$(\pidZ,$ $\hgamma^\pidZ_1,$ $\hsigma^\pidZ_1,$ $\bsq,$ $\bsq,$ $\hval^\pidZ))$ 
\\ for codes $\codeVLL \in \vanillaCodes$ 
and 
% c
a \LocMap\ $\psi_1$ 
% d
such that 
\\ % f
$\RT\codeLL \cong \codeVLL$, 
% g
$\{(\RT\pid, {\rrgamma^{\RT{\pid}}_{\RT1}},$ ${\rrsigma^{\RT{\pid}}_{\RT1}},$ $\RT\DMap^{\RT{\pid}}_{\RT1}$, $\rAcc^{\RT{\pid}}_{\RT1},$ $\RT\val^{\RT{\pid}}_{})$ $\cong_{\psi_1}$ 
$(\pid,$ $\hgamma^\pid_1,$ $\hsigma^\pid_1,$ $\bsq,$ $\bsq,$ $\hval^\pid)\}^{\pidZ}_{\pid = \pidA}$, 
% h
and $\RT\Pi \cong_{\psi_1} \Sigma$.
\end{theorem}



\begin{proof}[Proof Sketch]
By induction over all \piccoC\ semantic rules. 

The bulk of the complexity of this proof lies with rules pertaining to Private If Else, handling of pointers, and freeing of memory.  We first provide a brief overview of the intuition for the simpler cases and then dive deeper
into the details for the more complex cases.  Full proofs are available in our artifact submission.

For the rules evaluating over public data, correctness follows simply as the \vanillaC\ and \piccoC\ rules for public data are nearly identical. For all the semantic rules that use general helper algorithms (i.e., algorithms in common to both \vanillaC\ and \piccoC), we also reason about the correctness of the helper algorithms, comparing the \vanillaC\ version and the \piccoC version. Correctness over such algorithms is easily proven, as these algorithms are nearly identical, differing on privacy labels as we do not have private data in \vanillaC. 

For all \piccoC\ multiparty semantic rules, we relate them to the multiparty versions of the \vanillaC\ rules. To reason about the multiparty protocols, we leverage Axioms, such as Axiom~\ref{axiom: mpc bop}, to prove these rules correct. These Axioms should be proven correct by a library developer to ensure the completeness of the formal model. The correctness of most multiparty semantic rules follows easily, with Multiparty Private Free being an exception. For this rule, we also must reason about our helper algorithms that are specific to the \piccoC\ semantics (e.g., $\UpdateBytesFree$, $\UpdatePtrLocs$). We leverage the correctness of the behavior of the multiparty protocol $\PFree$, to show that correctness of these algorithms follows due to the deterministic definitions of the algorithms. In this case, we must also show that the locations that are swapped within this rule (which is done to hide the true location) are deterministic based on our memory model definition. We use $\psi$ to map the swapped locations, enabling us to show that, if these swaps were reversed, we would once again have memories that are directly congruent. This concept of locations being $\psi$-congruent is particularly necessary when reasoning about pointers in other rule cases. 
For all the rules using private pointers, we will rely upon the pointer data structure containing a set of locations and their associated tags, only one of which being the true location. With this proven to be the case, it is then clear that the true location indicated within the private pointer's data structure in \piccoC\ will be $\psi$-congruent  with the location given by the pointer data structure in \vanillaC.
In our proof, we make the assumption that locations are not hard-coded, as hard-coded locations would lead to potentially differing results between \vanillaC\ and \piccoC\ execution due to the behavior of \TT{pfree}. Additionally, given the distributed nature of the \piccoC, it would not make sense to allow hard-coded locations, as a single program will be executed on several different machines.

For rule Private Malloc, we must relate this rule to the sequence of \vanillaC\ rules for Malloc, Multiplication, and Size Of Type. This is due to the definition of \TT{pmalloc} as a helper that allows the user to write programs without knowing the size of private types. This case follows from the definition of translating the \piccoC\ program to a \vanillaC\ program, $\bm{\erasure}({\PMalloc(\Expr,\ \Type)} 
= {(\Malloc(\sizeof(\bm{\erasure}(\Type)) \cdot \bm{\erasure}(\Expr)))})$.

For the Private If Else rules, we must reason that our end results in memory after executing both branches and resolving correctly match the end result of having only executed the intended branch. 
The cases for both of these rules will have two subcases - one for the conditional being true, and the other for false. 
To obtain correctness, we use multiparty versions of the if else true and false rules that execute both branches - this allows us to reason that both branches will evaluate properly, and that we will obtain the correct ending state once completed. 
For both rules, we must first show that $\DynExtract$ will correctly find all non-local variables that are modified within both branches, including non-assignment modifications such as use of the pre-increment operator $++\x$, and that all such modified variables will be added to the list (excluding pointers modified exclusively by pointer dereference write statements). We must also show that it will correctly find and tag if a pointer dereference write statement was found. These properties follow deterministically from the definition of the algorithm. 

For rule Private If Else Variable Tracking, we will leverage the correctness of $\DynExtract$, and that if $\DynExtract$ returns the tag 0, no pointer dereference writes were found. We then reason that $\Initialize$ will correctly create the assignment statements for our temporary variables, and that the original values for each of the modified variables will be stored into the \TT{else} temporary variables. The temporaries being stored into memory correctly through the evaluation of these statements follows by induction. Next we have the evaluation of the \TT{then} branch, which will result in the values that are correct for if the condition had been true - this holds by induction. 
We then proceed to reason that $\Restore$ will properly create the statements to store the ending results of the \TT{then} branch into the \TT{then} temporary variables, and restore all of the original values from the \TT{else} variables (the original values being correctly stored follows from $\Initialize$ and the evaluation of it's statements). The correct evaluation of the this set of statements follows by induction. 
Next we have the evaluation of the \TT{else} branch, which will result in the values that are correct for if the condition had been false - this holds by induction and the values having been restored to the original values properly.
We will then reason about the correctness of the statements created by $\Resolve$. These statements must be set up to correctly take the information from the \TT{then} temporary variable, the temporary variable for the condition for the branch, and the ending result for all variables from the \TT{else} branch. For the resolution of pointers, we insert a call for a resolution function ($\resolve$), because the resolution of pointer data is more involved. The evaluation of this function is shown in rule Multiparty Resolve Pointer Locations. By proving that this rule will correctly resolve the true locations for pointers, we will then have that the statements created by $\Resolve$ will appropriately resolve all 


For rule Private If Else Location Tracking, the structure of the case is similar to the rule for variable tracking, but with a few differences we will discuss here. 
For this rule, we will need to reason about $\DynUpdate$, and that we will catch all modifications by pointer dereference writes and properly add them to $\DMap$ if the location being modified is not already tracked. If a new mapping is added, we store the current value in $\val_\mathit{orig}$ (as this location has not yet been modified) and the tag has to be set to 0. This behavior will be used to ensure the correctness during resolution.
For $\DynInit$, we must reason that we correctly initialize the map $\DMap$ with all of the locations we found within $\DynExtract$ to be modified by means other than pointer dereference writes and store their original values in $\val_\mathit{orig}$. 
Then we can evaluate the \TT{then} branch, which will result in the values that are correct for if the condition had been true - this holds by induction. 
For $\DynRestore$, we reason that we properly store the results of the \TT{then} branch, and update the tag for the location to signify that we should use $\val_\mathit{then}$ instead of $\val_\mathit{orig}$. We will then restore the original values, leveraging the correctness of $\DynInit$ to prove this will happen correctly. 
Then we can evaluate the \TT{else} branch, which will result in the values that are correct for if the condition had been false - this holds by induction. 
For $\DynResolve$, we reason that we will create the appropriate resolution statements to be executed. For the \TT{then} result, these statements must use the value stored in $\val_\mathit{orig}$ if the tag is set to 0 (this occurs if the first modification to the location was a pointer dereference write within the \TT{else} branch), and the value stored in $\val_\mathit{then}$ if the tag is set to 1. We prove this to be the correct \TT{then} result through the correctness of $\DynUpdate$ and $\DynRestore$. The \TT{else} result must use the current value for that location in memory, which is proven to be the correct \TT{else} result through the correctness of $\DynInit$ and $\DynResolve$. In this way, we can prove the correctness the contents of the statements created by $\DynResolve$, and then the correctness of the evaluation of the statements created by $\DynRestore$ will hold as we discussed for with those created by $\Resolve$ for Private If Else Variable tracking. 
\end{proof}






\subsection{Noninterference}
\label{app: noninterference}


\begin{definition}%\small
\label{def: loweq tree}
Two \piccoC\ evaluation trees $\Pi$ and $\Sigma$ are  \emph{low-equivalent}, in symbols $\Pi \loweq \Sigma$, if and only if $\Pi$ and $\Sigma$ have the same structure as trees, and for each node in $\Pi$ proving 
\\ \-\ \quad \-\ $((\pidA, \gamma^{\pidA}_{},$ $\sigma^{\pidA}_{},$ $\DMap^{\pidA}_{}$, $\Acc^{\pidA}_{},$ $\stmt)\Mid ...\Mid (\pidZ, \gamma^{\pidZ}_{},$ $\sigma^{\pidZ}_{},$ $\DMap^{\pidZ}_{}$, $\Acc^{\pidZ}_{},$ $\stmt))$ 
\\ $\Deval{\locLL}{\codeLL}$ $((\pidA, \gamma^{\pidA}_{1},$ $\sigma^{\pidA}_{1},$ $\DMap^{\pidA}_{1}$, $\Acc^{\pidA}_{1},$ $\val^{\pidA}_{})\Mid ...\Mid (\pidZ, \gamma^{\pidZ}_{1},$ $\sigma^{\pidZ}_{1},$ $\DMap^{\pidZ}_{1}$, $\Acc^{\pidZ}_{1},$ $\val^{\pidZ}_{}))$, the corresponding node in $\Sigma$ proves 
\\ \-\ \quad \-\ \-\ $((\pidA, \gamma^{\pidA}_{},$ $\sigma^{\pidA}_{},$ $\DMap^{\pidA}_{}$, $\Acc^{\pidA}_{},$ $\stmt)\Mid ...\Mid (\pidZ, \gamma^{\pidZ}_{},$ $\sigma^{\pidZ}_{},$ $\DMap^{\pidZ}_{}$, $\Acc^{\pidZ}_{},$ $\stmt))$ 
\\ $\Deval{\locLL'}{\codeLL'}$ $((\pidA, \gamma^{\pidA}_{1},$ $\sigma^{\pidA}_{1},$ $\DMap^{\pidA}_{1}$, $\Acc^{\pidA}_{1},$ $\val^{\pidA}_{})\Mid ...\Mid (\pidZ, \gamma^{\pidZ}_{1},$ $\sigma^{\pidZ}_{1},$ $\DMap^{\pidZ}_{1}$, $\Acc^{\pidZ}_{1},$ $\val^{\pidZ}_{}))$, $\codeLL = \codeLL'$ and $\locLL = \locLL'$.
\end{definition}



\begin{axiom}[$MPC_{ar}$]%\small
\label{axiom: mpc ar ni}
Given indices $\{\ind^{\pid}_{},$ $\ind'^{\pid}_{}\}^{\pidZ}_{\pid=\pidA}$, arrays $\{[\val^{\pid}_{1},$ $...,$ $\val^{\pid}_{n}]$, $[\val'^{\pid}_{1},$ $...,$ $\val'^{\pid}_{n}]\}^{\pidZ}_{\pid=\pidA}$, 
\\
if $MPC_{ar}((\ind^{\pidA}_{},$ $[\val^{\pidA}_{1},$ $...,$ $\val^{\pidA}_{n}]),$ $...,$ $(\ind^{\pidZ}_{},$ $[\val^{\pidZ}_{1},$ $...,$ $\val^{\pidZ}_{n}]))$ $= (\val^{\pidA}_{},$ $...,$ $\val^{\pidZ}_{})$, 
\\ \-\ \-\ 
$MPC_{ar}((\ind'^{\pidA}_{},$ $[\val'^{\pidA}_{1},$ $...,$ $\val'^{\pidA}_{n}]),$ $...,$ $(\ind'^{\pidZ}_{},$ $[\val'^{\pidZ}_{1},$ $...,$ $\val'^{\pidZ}_{n}]))$ $= (\val'^{\pidA}_{},$ $...,$ $\val'^{\pidZ}_{})$,  
\\ \-\ \-\ 
$\{\ind^{\pid}_{} = \ind'^{\pid}_{}\}^{\pidZ}_{\pid=\pidA}$, 
and $\{[\val^{\pid}_{1},$ $...,$ $\val^{\pid}_{n}] = [\val'^{\pid}_{1},$ $...,$ $\val'^{\pid}_{n}]\}^{\pidZ}_{\pid=\pidA}$
\\
then $\{\val^{\pid}_{} = \val'^{\pid}_{}\}^{\pidZ}_{\pid=\pidA}$. 
\end{axiom}



\begin{axiom}[$\MPC{b}$]%\small
\label{axiom: mpc b ni}
Given values $\{\val^{\pid}_{1},$ $\val^{\pid}_{2},$ $\val^{\pid}_{3},$ $\val'^{\pid}_{1},$ $\val'^{\pid}_{2},$ $\val'^{\pid}_{3}\}^{\pidZ}_{\pid=\pidA}$ and binary operation $\binop\in\{\cdot, +, -, \div\}$, 
\\
if $MPC_{b}(\binop,$ $\val^{\pidA}_{1},$ $\val^{\pidA}_{2}, ...,$ $\val^{\pidZ}_{1}, \val^{\pidZ}_{2})$ $= (\val^{\pidA}_{3}, ..., \val^{\pidZ}_{3})$, 
\\ \-\ \-\ 
$MPC_{b}(\binop,$ $\val'^{\pidA}_{1},$ $\val'^{\pidA}_{2}, ...,$ $\val'^{\pidZ}_{1}, \val'^{\pidZ}_{2})$ $= (\val'^{\pidA}_{3}, ..., \val'^{\pidZ}_{3})$,  
$\{\val^{\pid}_{1} = \val'^{\pid}_{1}\}^{\pidZ}_{\pid=\pidA}$, 
and $\{\val^{\pid}_{2} = \val'^{\pid}_{2}\}^{\pidZ}_{\pid=\pidA}$
\\
then $\{\val^{\pid}_{3} = \val'^{\pid}_{3}\}^{\pidZ}_{\pid=\pidA}$. 
\end{axiom}





\begin{theorem}[Multiparty Noninterference]%\small
\label{thm: NI app}
For every environment $\{\gamma^{\pid}_{},$ $\gamma^{\pid}_{1},$ $\gamma'^{\pid}_{1}\}^{\pidZ}_{\pid = \pidA}$; 
memory $\{\sigma^{\pid}_{}$, $\sigma^{\pid}_{1}$, $\sigma'^{\pid}_{1} \}^{\pidZ}_{\pid = \pidA}\in\Mem$; 
\changeMap $\{\DMap^{\pid}_{}$, $\DMap^{\pid}_{1}$, $\DMap'^{\pid}_{1}\}^{\pidZ}_{\pid = \pidA}$;
accumulator $\{\Acc^{\pid}_{}$, $\Acc^{\pid}_{1}$, $\Acc'^{\pid}_{1}\}^{\pidZ}_{\pid = \pidA}\in\N$; 
statement $\stmt$, values $\{\val^{\pid}_{}$, $\val'^{\pid}_{}\}^{\pidZ}_{\pid = \pidA}$; 
step evaluation code lists $\codeLL,\codeLL'$ and their corresponding lists of locations accessed $\locLL,\locLL'$, 
party $\pid \in \{\pidA...\pidZ\}$; 
\\
if 
$\Pi \deriv ((\pidA, \gamma^{\pidA}_{},$ $\sigma^{\pidA}_{},$ $\DMap^{\pidA}_{}$, $\Acc^{\pidA}_{},$ $\stmt)\Mid ...\Mid (\pidZ, \gamma^{\pidZ}_{},$ $\sigma^{\pidZ}_{},$ $\DMap^{\pidZ}_{}$, $\Acc^{\pidZ}_{},$ $\stmt))$ 
\\ $\-\ \quad\Deval{\locLL}{\codeLL}$ $((\pidA, \gamma^{\pidA}_{1},$ $\sigma^{\pidA}_{1},$ $\DMap^{\pidA}_{1}$, $\Acc^{\pidA}_{1},$ $\val^{\pidA}_{})\Mid ...\Mid (\pidZ, \gamma^{\pidZ}_{1},$ $\sigma^{\pidZ}_{1},$ $\DMap^{\pidZ}_{1}$, $\Acc^{\pidZ}_{1},$ $\val^{\pidZ}_{}))$ 
\\ and   
$\Sigma \deriv ((\pidA, \gamma^{\pidA}_{},$ $\sigma^{\pidA}_{},$ $\DMap^{\pidA}_{}$, $\Acc^{\pidA}_{},$ $\stmt)\Mid ...\Mid (\pidZ, \gamma^{\pidZ}_{},$ $\sigma^{\pidZ}_{},$ $\DMap^{\pidZ}_{}$, $\Acc^{\pidZ}_{},$ $\stmt))$ 
\\ $\-\ \quad \-\ \Deval{\locLL'}{\codeLL'}$ $((\pidA, \gamma'^{\pidA}_{1},$ $\sigma'^{\pidA}_{1},$ $\DMap'^{\pidA}_{1}$, $\Acc'^{\pidA}_{1},$ $\val'^{\pidA}_{})\Mid ...\Mid (\pidZ, \gamma'^{\pidZ}_{1},$ $\sigma'^{\pidZ}_{1},$ $\DMap'^{\pidZ}_{1}$, $\Acc'^{\pidZ}_{1},$ $\val'^{\pidZ}_{}))$
\\ then $\{\gamma^{\pid}_{1}=\gamma'^{\pid}_{1}\}^{\pidZ}_{\pid = \pidA}$, 
$\{\sigma^{\pid}_{1}=\sigma'^{\pid}_{1}\}^{\pidZ}_{\pid = \pidA}$, 
$\{\DMap^{\pid}_{1} =\DMap'^{\pid}_{1}\}^{\pidZ}_{\pid = \pidA}$, 
$\{\Acc^{\pid}_{1}=\Acc'^{\pid}_{1}\}^{\pidZ}_{\pid = \pidA}$, 
$\{\val^{\pid}_{}=\val'^{\pid}_{}\}^{\pidZ}_{\pid = \pidA}$, 
$\codeLL=\codeLL'$, 
$\locLL = \locLL'$, 
$\Pi \loweq \Sigma$.
~\\
\end{theorem} 


\begin{proof}[Proof Sketch]
By induction over all \piccoC\ semantic rules. We make the assumption that both evaluation traces are over the same program (this is given by having the same $\stmt$ in the starting states) and all public data will remain the same, including data read as input during the evaluation of the program. A portion of the complexity of this proof is within ensuring that memory accesses within our semantics remain data oblivious. 
Several rules follow fairly simply and leverage similar ideas, which we will discuss first, and then we will provide further intuition behind the more complex cases. 

For all rules leveraging helper algorithms, we must reason about the helper algorithms, and that they behave deterministically by definition and have data-oblivious memory accesses. Given this and that these helper algorithms do no modify the private data, we maintain the properties of noninterference of this theorem. First we reason that our helper algorithms to translate values into their byte representation will do so deterministically, and therefore maintain indistinguishability between the value and byte representation. We can then reason that our helper algorithms that take these byte values and store them into memory will also do so deterministically, so that when we later access the data in memory we will obtain the same indistinguishable values we had stored. 

It is also important to take note here our functions to help us retrieve data from memory, particularly in cases such as when reading out of bounds of an array. When proving these cases to maintain noninterference, we leverage our definition of how memory blocks are assigned in a monotonically increasing fashion, and how the algorithms for choosing which memory block to read into after the current one are deterministic. This, as well as our original assumptions of having identical public input, allows us to reason that if we access out of bounds (including accessing data at a non-aligned position, such as a chunk of bytes in the middle of a memory block), we will be pulling from the same set of bytes each time, and therefore we will end up with the same interpretation of the data as we continue to evaluate the remainder of the program. It is important to note again here that by definition, our semantics will always interpret bytes of data as the type it is expected to be, not the type it actually is (i.e., reading bytes of data that marked private in memory by overshooting a public array will not decrypt the bytes of data, but instead give you back a garbage public value). To reiterate this point, even when reading out of bounds, we will not reveal anything about private data, as the results of these helper algorithms will be indistinguishable.

To reason about the multiparty protocols, we leverage Axioms, such as Axiom~\ref{axiom: mpc b ni}, to reason that the protocols will maintain our definition of noninterference. With each of these Axioms, we ensure that over two different evaluations, if the values of the first run ($\val^\pid_1, \val^\pid_2$) are not distinguishable from those of the second ($\val'^\pid_1, \val'^\pid_2$), then the resulting values are also not distinguishable ($\val^\pid_3 = \val'^\pid_3$). These Axioms should be proven by a library developer to ensure the completeness of the formal model. 

For private pointers, it is important to note that the obtaining multiple locations is deterministic based upon the program that is being evaluated. A pointer can initially gain multiple locations through the evaluation of a private if else. Once there exists a pointer that has obtained multiple locations in such a way, it can be assigned to another pointer to give that pointer multiple locations. The other case for a pointer to gain multiple location is through the use of \TT{pfree} on a pointer with multiple locations (i.e., the case where a pointer has locations $\loc_1$, $\loc_2$, $\loc_3$ and we free $\loc_1$) - when this occurs, if another pointer had referred to only $\loc_1$, it will now gain locations in order to mask whether we had to move the true location or not. 
When reasoning about pointers with multiple locations, we maintain that given the tags for which location is the true location are indistinguishable, then it is not possible to distinguish between them by their usage as defined in the rules or helper algorithms using them. 
Additionally, to reason about \TT{pfree}, we leverage that the definitions of the helper algorithms are deterministic, and that (wlog), we will be freeing the same location. We will then leverage our Axiom about the multiparty protocol $\PFree$. After the evaluation of $\PFree$, it will deterministically update memory and all other pointers as we mentioned in the brief example above.

For both Private If Else rules, the most important element we must leverage is how values are resolved, showing that given our resolution style, we are not able to distinguish between the ending values. 
In order to do this, we also must reason about the entirety of the rule, including all of if else helper algorithms. 
First, we note that the evaluation of the \TT{then} branches follows by induction, as does the evaluation of the \TT{else} branch once we have reasoned through the restoration phase. 
For variable tracking, it is clear from the definitions of $\DynExtract$, $\Initialize$, and $\Restore$ that the behavior of these algorithms is deterministic and given the same program, we will be extracting, initializing, and restoring the same set variables every time we evaluate the program. For location tracking, $\DynInit$ is also immediately clear that it will be initializing the same locations each time. We must then reason about $\DynUpdate$, and how given a program, we will deterministically find the pointer dereference writes and array writes at public indices at corresponding positions in memory and add them to our tracking structure $\DMap$. Then we can reason that the behavior of $\DynRestore$ will deterministically perform the same updates, because $\DMap$ will contain the same information in every evaluation. 
Now, we are able to move on to reasoning about resolution, and show that given all of this and the definitions of the resolution helper algorithms and rule, we are not able to distinguish between the ending values.  

One of the main complexities of this proof revolves around ensuring \emph{data-oblivious memory accesses} (i.e. that we always access locations deliberately based on public information), particularly when handling arrays and pointers. Within the proof, we must consider all helper algorithms, and what locations are accessed within the algorithms as well as within the rules. What locations are accessed within the algorithms follows deterministically from the definition of the algorithms, and we return from the algorithms which locations were accessed in order to properly reason about the entire evaluation trace of the program. Our semantics are designed in such a way that we give the multiparty protocols all of the information they need, with all memory accesses being completed within the rule itself or our helper algorithms. This also helps show that memory accesses are purely local, not distributed operations.  
%
Within the array rules, the main concern is in reading from and writing at a private index. We currently handle this complexity within our rules by accessing all locations within the array in rules Multiparty Array Read Private Index and Multiparty Array Write Private Index. 
In Multiparty Array Read Private Index, we clearly read data from every index of the array ($\{\forall \ind \in \{0...\nl-1\} \quad \DecodeArr({\llabel\ \btype}, \ind, {\byte^\pid_1}) =  \n^\pid_\ind\}^{\pidZ}_{\pid = \pidA}$), then that data is passed to the multiparty protocol. 
Similarly, in Multiparty Array Write Private Index, we read data from every index of the array, pass it to the multiparty protocol, then proceed to update every index of the array with what was returned from the protocol. 
Within the multiparty protocols used in these two rules, we will ensure the usage of the data is data-oblivious within the main noninterference proof in the following subsection. 
All other array rules use public indices, and in turn only access that publicly known location.
Within the pointer rules, our main concern is that we access all locations that are referred to by a private pointer when we have multiple locations. For this, we will reason about the contents of the rules and the helper algorithms used by the pointer rules, which can be shown to deterministically do so.
\end{proof}






















