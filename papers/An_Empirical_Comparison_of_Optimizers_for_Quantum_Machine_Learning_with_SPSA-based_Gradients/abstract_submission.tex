
% Abstract 

VQA have attracted a lot of attention from the quantum computing community for the last few years. Their hybrid quantum-classical nature with relatively shallow quantum circuits makes them a promising platform for demonstrating the capabilities of NISQ devices. Although the classical machine learning community focuses on gradient-based parameter optimization, finding near-exact gradients for VQC with the parameter-shift rule introduces a large sampling overhead. Therefore, gradient-free optimizers have gained popularity in quantum machine learning circles. Among the most promising candidates is the SPSA algorithm, due to its low computational cost and inherent noise resilience. We introduce a novel approach that uses the approximated gradient from SPSA in combination with state-of-the-art gradient-based classical optimizers. We demonstrate numerically that this outperforms both standard SPSA and the parameter-shift rule in terms of convergence rate and absolute error in simple regression tasks. The improvement of our novel approach over SPSA with stochastic gradient decent is even amplified when shot- and hardware-noise are taken into account. We also demonstrate that error mitigation does not significantly affect our results.


% key words:
variational quantum computing
 quantum error mitigation 
 SPSA
 gradient free optimization 
 classical optimizers 
 quantum regression

% Authors:

Author 1: Marco Wiedmann
email: wiedmamo@iis.fraunhofer.de
institute: Fraunhofer-IIS, Nuremberg

Author 2: Marc Hölle
email: marc.hoelle@iis.fraunhofer.de
institute: Fraunhofer-IIS, Nuremberg

Author 3: Maniraman Periyasamy
email: maniraman.periyasamy@iis.fraunhofer.de
institute: Fraunhofer-IIS, Nuremberg

Author 4: Nico Meyer
email: nico.meyer@iis.fraunhofer.de
institute: Fraunhofer-IIS, Nuremberg

Author 5: Christian Ufrecht
email: christian.ufrecht@iis.fraunhofer.de
institute: Fraunhofer-IIS, Nuremberg

Author 6: Daniel D. Scherer
email: daniel.scherer2@iis.fraunhofer.de
institute: Fraunhofer-IIS, Nuremberg

Author 7: Axel Plinge
email: axel.plinge@iis.fraunhofer.de
institute: Fraunhofer-IIS, Nuremberg

Author 8: Christopher Mutschler
email: christopher.mutschler@iis.fraunhofer.de
institute: Fraunhofer-IIS, Nuremberg