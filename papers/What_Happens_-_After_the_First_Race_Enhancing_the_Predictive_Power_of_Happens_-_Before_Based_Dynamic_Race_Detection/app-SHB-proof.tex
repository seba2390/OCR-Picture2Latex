%!TEX root = main.tex

In this section, we prove Theorem~\ref{thm:SHBSoundness}. We begin
with a couple of technical lemmas.

\begin{lemma}\label{lem:cr_hb}
% For any trace $\tr$, consider a correct reordering $\tr'$ of $\tr$
% that also respects $\hb{\tr}$. Then $\tr'$ respects $\shb{\tr}$, i.e.,
% for any $e,e'$ such that $e \shb{\tr} e'$ and $e' \in \events{\tr'}$,
% we have $e \in \events{\tr'}$ and $e \trord{\tr'} e'$.
Let $\tr$ be a trace and $\tr'$ be a correct reordering of $\tr'$ that
respects $\hb{\tr}$.  For any $e,e'$ such that $e \shb{\tr} e'$, if
$e' \in \events{\tr'}$ and $e'$ is not the last read event of its
thread in $\tr'$, then $e \in \events{\tr'}$ and $e \trord{\tr'} e'$.
\end{lemma}

\begin{proof}
Consider any $e,e'$ such that $e \shb{\tr} e'$, $e' \in \events{\tr'}$
and $e' = \ev{t, op}$ is not the last read event of the thread $t$ in
the trace $\tr'$.  Then it follows from Definition~\ref{def:shb} that
there is a sequence $e = e_0, e_1, \ldots e_n = e'$ such that for
every $i \leq n-1$, $e_i \trord{\tr} e_{i+1}$ and either (a)
$e_i \tho{\tr} e_{i+1}$ or (b) $e_i = \ev{t_i,\rel(\lk)}$, $e_{i+1}
= \ev{t_{i+1},\acq(\lk)}$, or (c) $e_{i+1} \in \reads{\tr}$ and $e_i
= \lw{\tr}(e_{i+1})$.

We will prove by induction on $i$, starting from $i = n$, that
$e_i \in \events{\tr'}$ and $e_i$ is not the last read event of its
thread in $\tr'$. Observe that these properties hold for $e' = e_n$
--- $e_n \in \events{\tr'}$ and $e_n$ is not the last read event of
its thread in $\tr'$. Assume we have established the claim for
$e_{i+1}$. Now there are three cases to consider for $e_i$. If
$e_i \tho{\tr'} e_{i+1}$ then clearly $e_i \in \events{\tr'}$ because
$e_{i+1} \in \events{\tr'}$. Further, if $e_i$ is a read event, then
it is not the last event of its thread because $e_{i+1}$ is after
it. If $e_i = \ev{t_i,\rel(\lk)}$ and $e_{i+1}
= \ev{t_{i+1},\acq(\lk)}$ then $e_i \in \events{\tr'}$ because $\tr'$
respects $\hb{\tr}$. Further $e_i$ is not the last read event because
it is not a read event! The last case to consider is where $e_i
= \lw{\tr}(e_{i+1})$. In this case, by induction hypothesis, we know
that $e_{i+1}$ is not the last read event of its thread, and therefore
by properties of a correct reordering, we have
$e_i \in \events{\tr'}$. Notice that in this case $e_i$ is not a read
event, and so the claim holds. Thus, we have established that $e =
e_0 \in \events{\tr'}$.

Next, we show that for every $i \leq n-1$, $e_i \trord{\tr'}
e_{i+1}$. If $e_i \tho{\tr} e_{i+1}$ or $e_i = \ev{t_i,\rel(\lk)}$ and
$e_{i+1} = \ev{t_{i+1},\acq(\lk)}$ with $e_i \trord{\tr} e_{i+1}$ then
$e_i \trord{\tr'} e_{i+1}$ because $\tr'$ respects $\hb{\tr}$. On the
other hand, if $e_i = \lw{\tr}(e_{i+1})$ then because $\tr'$ is a
correct reordering of $\tr$ and $e_{i+1}$ is not the last read event
of its thread (established in the previous paragraph), we have $e_i
= \lw{\tr}(e_{i+1}) = \lw{\tr'}(e_{i+1})$.  This establishes the fact
that $e = e_0 \trord{\tr'} e_n = e'$, which completes the proof of the
lemma.
\end{proof}

A slightly weaker form of the converse of Lemma~\ref{lem:cr_hb} also
holds.
%
\begin{lemma}\label{lem:shb_cr_hb}
For a trace $\tr$, let $\tr'$ be a trace with
$\events{\tr'} \subseteq \events{\tr}$ such that (a) $\tr'$ is
$\shb{\tr}$ downward closed, i.e., for any $e,e' \in \events{\tr}$ if
$e \shb{\tr} e'$ and $e' \in \events{\tr'}$ then
$e \in \events{\tr'}$, and (b) $\trord{\tr'} = \trord{\tr} \cap
(\events{\tr'} \times \events{\tr'})$. Then $\tr'$ is a correct
reordering of $\tr$ that respects $\hb{\tr}$.  Further,
for \textbf{every} read event $e \in \reads{\tr'}$, we have
$\lw{\tr'}(e) \simeq \lw{\tr}(e)$, i.e., either both $\lw{\tr'}(e)$
and $\lw{\tr}(e)$ are undefined, or they are both defined and equal.
\end{lemma}

\begin{proof}
The trace $\tr'$ in the lemma is such that the events in $\tr'$ are
downward closed with respect to $\shb{\tr}$ and in $\tr'$ they are
ordered in exactly the same way as in $\tr$. The fact that $\tr'$
respects $\hb{\tr}$ simply follows from the fact that
$\hb{\tr} \subseteq \shb{\tr}$ and
$\hb{\tr} \subseteq \trord{\tr}$. So the main goal is to establish
that $\tr'$ is a correct reordering of $\tr$ that preserves the last
writes of \emph{all} read events.

First we show that $\tr'$ respects lock semantics. Suppose $e_1
= \ev{t_1, \acq(\lk)}$ and $e_2 = \ev{t_2, \acq(\lk)}$ are two lock
acquire events for some lock $\lk$ such that $e_1 \trord{\tr} e_2$ and
$\{e_1,e_2\} \subseteq \events{\tr'}$. Let $e_1'$ be the matching
release event for $e_1$ in $\tr$; such an $e_1'$ exists because $\tr$
is a valid trace. Then we have $e_1 \hb{\tr} e_1' \hb{\tr} e_2$, and
so $e_1' \in \events{\tr'}$ and $e_1' \trord{\tr'} e_2$ because $\tr'$
respects $\hb{\tr}$.

Next observe that since $\tho{\tr} \subseteq \hb{\tr}$ and $\tr'$
respects $\hb{\tr}$, we can conclude that $\proj{\tr'}{t}$ is a prefix
of $\proj{\tr}{t}$ for any thread $t$.

Finally, consider any $e' \in \reads{\tr'}$. Suppose $\lw{\tr}(e')$ is
defined. Let $e = \lw{\tr}(e')$. Since $e \shb{\tr} e'$ and $\tr'$ is
downward closed with respect to $\shb{\tr}$, we have
$e \in \events{\tr'}$. Let $e_1 = \lw{\tr'}(e')$. We need to argue
that $e_1 = e$. Suppose (for contradiction) it is not, i.e., $e \neq
e_1$. Then either $e_1 \trord{\tr} e$ or $e' \trord{\tr} e_1$, because
$e = \lw{\tr}(e')$. However, the fact that $e_1 = \lw{\tr'}(e')$
contradicts the fact that $\trord{\tr'} = \trord{\tr} \cap
(\events{\tr'} \times \events{\tr'})$. Conversely, if $\lw{\tr'}(e')$
is defined then let $e = \lw{\tr'}(e')$. Since $\trord{\tr'}
= \trord{\tr} \cap (\events{\tr'} \times \events{\tr'})$, we have
$e \trord{\tr} e'$. Thus, $\lw{\tr}(e')$ is defined. Let $e_1
= \lw{\tr}(e')$. Once again, since $e_1 \shb{\tr} e'$, and $\tr'$ is
downward closed with respect to $\shb{\tr}$, we have
$e_1 \in \tr'$. Just like in the previous direction, we can conclude
that $e = e_1$ because otherwise we violate the fact that
$\trord{\tr'}$ is identical to $\trord{\tr}$ over $\events{\tr'}$.
\end{proof}

We now prove Theorem~\ref{thm:SHBSoundness} below
\begin{reptheorem}{thm:SHBSoundness}
Let $\tr$ be a trace and $e_1 \trord{\tr} e_2$ be conflicting events in $\tr$.
$(e_1, e_2)$ is an $\hb{\tr}$-schedulable race iff 
either $\ltho{\tr}(e_2)$ is undefined, or $e_1 \not\leq^\tr_{\mathsf{SHB}} \ltho{\tr}(e_2)$.
\end{reptheorem}

\begin{proof}
Let us first prove the forward direction.
That is, let $(e_1,e_2)$ be an HB-race such that the
event $e = \ltho{\tr}(e_2)$ is defined and $e_1 \shb{\tr} e$. 
Consider any correct reordering $\tr'$ that contains both
$e_1$ and $e_2$ and respects $\hb{\tr}$. 
First, since $\tr'$ is a correct reordering of $\tr$, we must have
$e \in \events{\tr'}$ and $e \trord{\tr'} e_2$.
Further, since $e_1 \shb{\tr} e$, from Lemma~\ref{lem:cr_hb}, 
$e_1 \trord{\tr'} e$.
Thus, we have that $e_1 \trord{\tr'} e \trord{\tr'} e_1$
for any correct reordering $\tr'$ of $\tr$ that respects $\hb{\tr}$.
This means, $(e_1, e_2)$ cannot be a $\hb{\tr}$-schedulable race.
% Also, since $\tr'$ is a correct reordering
% also respects $\shb{\tr}$. But that would mean that
% $e \in \events{\tr'}$ and $e_1 \trord{\tr'} e \trord{\tr'} e_2$, and
% so $e_1,e_2$ are not consecutive in $\tr'$. Since this holds for any
% correct reordering $\tr'$ of $\tr$ that also respects $\hb{\tr}$, we
% can conclude that $(e_1,e_2)$ is not an $\hb{\tr}$-schedulable race.

We now prove the backward direction.
Consider an HB-race $(e_1,e_2)$ such that 
either $\ltho{\tr}(e_2)$ is undefined, or if it exists, then it satisfies 
$e_1 \not\leq^\tr_{\mathsf{SHB}} \ltho{\tr}(e_2)$.
% satisfying either of the two
% conditions. 
% Notice that in either case this means that there is no
% event $e \in \events{\tr} \setminus \{e_1,e_2\}$ such that
% $e_1 \shb{\tr} e \shb{\tr} e_2$. 
Consider the set $\setreq$ defined as
\[
\setreq = \{e \in \events{\tr} \setminus\{e_1,e_2\}\: |\: 
        e \shb{\tr} e_1 \mbox{ or } 
        e \shb{\tr} \ltho{\tr}(e_2)\}
\]
where we assume that if $\ltho{\tr}(e_2)$ is undefined then no event
$e$ satisfies the condition $e \shb{\tr} \ltho{\tr}(e_2)$.

First we will show that $\setreq$ is downward closed with respect to
$\shb{\tr}$. Consider $e,e'$ such that $e \shb{\tr} e'$ and
$e' \in \setreq$. By definition of $\setreq$, we have
$e' \not\in \{e_1,e_2\}$ and either $e' \shb{\tr} e_1$ or
$e' \shb{\tr} \ltho{\tr}(e_2)$. Observe that if
$e \not\in \{e_1,e_2\}$, then it is clear that $e \in \setreq$ by
definition since $\shb{\tr}$ is transitive. It is easy to see that
$e \neq e_2$ --- this is because since $e' \neq e_2$, and
$\shb{\tr} \subseteq \trord{\tr}$, $e' \stricttrord{\tr} e_2$ and so
$e \stricttrord{\tr} e_2$. So, all we have left to establish is that
$e \neq e_1$. Suppose for contradiction $e = e_1$. Then it must be the
case that $e' \shb{\tr} \ltho{\tr}(e_2)$. Since $e_1 = e \shb{\tr}
e' \shb{\tr} \ltho{\tr}(e_2)$, we have
$e_1 \shb{\tr} \ltho{\tr}(e_2)$, which contradicts our assumption
about $(e_1,e_2)$.

Let us now consider a trace $\tr''$ which consists of the events in
$\setreq$ ordered according to $\trord{\tr}$. That is, $\trord{\tr''}
= \trord{\tr} \cap (\setreq \times \setreq)$. Since $\tr''$ satisfies the
conditions of Lemma~\ref{lem:shb_cr_hb}, we can conclude that $\tr''$
is a correct reordering of $\tr$ that respects $\hb{\tr}$ and preserves the
last-writes of \emph{every} read event present.

Consider the trace $\tr' = \tr''e_1e_2$. First we prove that $\tr'$
respects $\hb{\tr}$. To do that, we first show that for any event
$e\in \events{\tr}$ such that $e \hb{\tr} e_1$ and $e \neq e_1$, or
$e \hb{\tr} e_2$ and $e \neq e_2$, then $e \in \setreq$.  If
$e \hb{\tr} e_1$ then $e \shb{\tr} e_1$ and so $e \in \setreq$. On the
other hand, if $e \hb{\tr} e_2$ (and $e \neq e_2$), since $(e_1,e_2)$
is an HB-race, we must have $e \neq e_1$ and
$e \hb{\tr} \ltho{\tr}(e_2)$. So $e \in \setreq$. Now the fact $\tr'$
respects $\hb{\tr}$ follows from the fact that $\tr''$ respects
$\hb{\tr}$ and the claim just proved.

We now prove that $\tr'$ is a correct reordering. Observe that since
$\tr'$ respects $\hb{\tr}$, $\tr'$ is well formed (lock semantics is
not violated) and preserves thread-wise prefixes ($\forall
t, \proj{\tr'}{t}$ is a prefix of $\proj{\tr}{t}$).  Further, $\tr''$
is such that every read event in $\tr''$ reads the same last write as
in $\tr$.  Also, since $e_1$ and $e_2$ are the last events in their
threads in $\tr'$, we conclude that $\tr'$ is a correct reordering of
$\tr$ that respects $\hb{\tr}$.
%
% Recall that in this case, we
% have $e_1 \in \reads{\tr}(x)$ (for some $x$),
% $e_2 \in \writes{\tr}(x)$, and $\writes{\tr}(x) \cap \setreq \cap
% (\pretwo\setminus\preone) = \emptyset$. We claim that $\tr'$ is a
% correct reordering that respects $\hb{\tr}$. Since $\tr''$ respects
% $\hb{\tr}$, we also have $\tr'$ respects $\hb{\tr}$. Similar to the
% proof of Lemma~\ref{lem:cr_hb}, we can conclude that since $\tr'$
% respects $\hb{\tr}$, $\tr'$ satisfies lock semantics and thread
% order. All we are left to prove is that $\tr'$ preserves the last
% writes. For any event $e \in \setreq \cap \reads{\tr}$, we know that
% $\lw{\tr}(e)
% \simeq \lw{\tr''}(e) \simeq \lw{\tr'}(e)$ since $\tr''$ is a correct 
% reordering; here we use $a \simeq b$ to mean either both $a,b$ are
% undefined or they are both defined and equal. Finally, we have to
% prove $\lw{\tr}(e_1) \simeq \lw{\tr'}(e_1)$. Observe that, if
% $\lw{\tr}(e_1)$ is undefined then we have
% $\writes{\tr}(x) \cap \preone = \emptyset$. Combining this with the
% fact that $\writes{\tr}(x) \cap \setreq (\pretwo\setminus\preone)
% = \emptyset$, we get $\writes{\tr}(x) \cap \setreq = \emptyset$ and so
% $\lw{\tr'}(e_1)$ is also undefined. On the other hand, suppose
% $\lw{\tr}(e_1)$ is defined. Take $e = \lw{\tr}(e_1)$. Since
% $e \shb{\tr} e_1$, $e \in \setreq$. Now, using the observations that
% $\writes{\tr}(x) \cap \setreq \cap (\pretwo\setminus\preone)
% = \emptyset$, and $\trord{\tr''} = \trord{\tr} \cap
% (\setreq \times \setreq)$, we get $\lw{\tr'}(e_1) = e$.
%
% \paragraph{Proof of part~\ref{lbl:sufficient-write}.}
% In this case we have $e_1 \in \writes{\tr}(x)$, for some $x$. There
% are two cases to consider. Let us consider first the case when either
% $e_2 \in \writes{\tr}(x)$, or $e_2 \in \reads{\tr}(x)$ with
% $\lw{\tr}(e_2) = e_1$. This is the easy case when we prove that $\tr'
% = \tr''e_1e_2$ is a correct reordering that respects $\hb{\tr}$. Like
% in the previous paragraph, we can argue that $\tr'$ respects
% $\hb{\tr}$, satisfies lock semantics and thread order. Further because
% $\tr''$ is a correct reordering, the last write of every
% $e \in \setreq \cap \reads{\tr}$ is preserved. Finally, if
% $e_2 \in \reads{\tr}(x)$ then $\lw{\tr}(e_2) = e_1 = \lw{\tr'}(e_2)$
% by assumption and construction of $\tr'$. The second case to consider
% is when $e_2 \in \reads{\tr}(x)$ but $e_1 \neq \lw{\tr}(e_2)$. In this
% case we will prove that $\tr' = \tr''e_2e_1$ is a correct reordering
% that respects $\hb{\tr}$. The proof that $\tr'$ respects $\hb{\tr}$,
% lock semantics, thread order, and preserves the last write of all
% events in $\setreq$, is the same as before. The only thing we need to
% argue is that the last write of $e_2$ is preserved.  If
% $\lw{\tr}(e_2)$ is undefined then since $\setreq \subseteq \pretwo$,
% $\lw{\tr'}(e_2)$ is also undefined. Suppose $e = \lw{\tr}(e_2)$. Then
% since $e \shb{\tr} e_2$, we have $e \in \setreq$. Further since
% $\trord{\tr''} = \trord{\tr} \cap (\setreq \times \setreq)$,
% $\lw{\tr'}(e_2) = e$.
\end{proof}

