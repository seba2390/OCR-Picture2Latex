\documentclass[acmsmall,10pt,table,xcdraw]{acmart}
%% For double-blind review submission, w/o CCS and ACM Reference (max submission space)
% \documentclass[sigplan,10pt,review,anonymous]{acmart}
\settopmatter{printfolios=true,printccs=false,printacmref=false}
%% For double-blind review submission, w/ CCS and ACM Reference
%\documentclass[sigplan,10pt,review,anonymous]{acmart}\settopmatter{printfolios=true}
%% For single-blind review submission, w/o CCS and ACM Reference (max submission space)
%\documentclass[sigplan,10pt,review]{acmart}\settopmatter{printfolios=true,printccs=false,printacmref=false}
%% For single-blind review submission, w/ CCS and ACM Reference
%\documentclass[sigplan,10pt,review]{acmart}\settopmatter{printfolios=true}
%% For final camera-ready submission, w/ required CCS and ACM Reference
%\documentclass[sigplan,10pt]{acmart}\settopmatter{}


%% Conference information
%% Supplied to authors by publisher for camera-ready submission;
%% use defaults for review submission.
\acmConference[OOPSLA'18]{ACM SIGPLAN Conference on Object Oriented Programming, Systems, Languages, and Applications}{November 04--09, 2018}{Boston, MA, USA}
\acmYear{2018}
\acmISBN{} % \acmISBN{978-x-xxxx-xxxx-x/YY/MM}
\acmDOI{} % \acmDOI{10.1145/nnnnnnn.nnnnnnn}
\startPage{1}

%% Copyright information
%% Supplied to authors (based on authors' rights management selection;
%% see authors.acm.org) by publisher for camera-ready submission;
%% use 'none' for review submission.
\setcopyright{none}
%\setcopyright{acmcopyright}
%\setcopyright{acmlicensed}
%\setcopyright{rightsretained}
%\copyrightyear{2017}           %% If different from \acmYear

%% Bibliography style
\bibliographystyle{ACM-Reference-Format}
%% Citation style
\citestyle{acmauthoryear}  %% For author/year citations
%\citestyle{acmnumeric}     %% For numeric citations
%\setcitestyle{nosort}      %% With 'acmnumeric', to disable automatic
                            %% sorting of references within a single citation;
                            %% e.g., \cite{Smith99,Carpenter05,Baker12}
                            %% rendered as [14,5,2] rather than [2,5,14].
%\setcitesyle{nocompress}   %% With 'acmnumeric', to disable automatic
                            %% compression of sequential references within a
                            %% single citation;
                            %% e.g., \cite{Baker12,Baker14,Baker16}
                            %% rendered as [2,3,4] rather than [2-4].


%%%%%%%%%%%%%%%%%%%%%%%%%%%%%%%%%%%%%%%%%%%%%%%%%%%%%%%%%%%%%%%%%%%%%%
%% Note: Authors migrating a paper from traditional SIGPLAN
%% proceedings format to PACMPL format must update the
%% '\documentclass' and topmatter commands above; see
%% 'acmart-pacmpl-template.tex'.
%%%%%%%%%%%%%%%%%%%%%%%%%%%%%%%%%%%%%%%%%%%%%%%%%%%%%%%%%%%%%%%%%%%%%%


%% Some recommended packages.
\usepackage{booktabs}   %% For formal tables:
                        %% http://ctan.org/pkg/booktabs
\usepackage{subcaption} %% For complex figures with subfigures/subcaptions
                        %% http://ctan.org/pkg/subcaption
\usepackage{xcolor}
\usepackage{tikz}
\usepackage[inline]{enumitem}
\usepackage[ruled,vlined,resetcount,linesnumbered]{algorithm2e}
\usepackage{multicol}
\setlength{\aboverulesep}{-0pt}
\setlength{\belowrulesep}{0pt}{}
\usepackage{amsthm}
% \usepackage{minted}
\usepackage{fancyvrb}



%TODO
\newcommand{\todo}[1]{{\color{red}{\bf [TODO]:~{#1}}}}

%THEOREMS
\newtheorem{theorem}{Theorem}
\newtheorem{corollary}{Corollary}
\newtheorem{lemma}{Lemma}
\newtheorem{proposition}{Proposition}
\newtheorem{problem}{Problem}
\newtheorem{definition}{Definition}
\newtheorem{remark}{Remark}
\newtheorem{example}{Example}
\newtheorem{assumption}{Assumption}

%HANS' CONVENIENCES
\newcommand{\define}[1]{\textit{#1}}
\newcommand{\join}{\vee}
\newcommand{\meet}{\wedge}
\newcommand{\bigjoin}{\bigvee}
\newcommand{\bigmeet}{\bigwedge}
\newcommand{\jointimes}{\boxplus}
\newcommand{\meettimes}{\boxplus'}
\newcommand{\bigjoinplus}{\bigjoin}
\newcommand{\bigmeetplus}{\bigmeet}
\newcommand{\joinplus}{\join}
\newcommand{\meetplus}{\meet}
\newcommand{\lattice}[1]{\mathbf{#1}}
\newcommand{\semimod}{\mathcal{S}}
\newcommand{\graph}{\mathcal{G}}
\newcommand{\nodes}{\mathcal{V}}
\newcommand{\agents}{\{1,2,\dots,N\}}
\newcommand{\edges}{\mathcal{E}}
\newcommand{\neighbors}{\mathcal{N}}
\newcommand{\Weights}{\mathcal{A}}
\renewcommand{\leq}{\leqslant}
\renewcommand{\geq}{\geqslant}
\renewcommand{\preceq}{\preccurlyeq}
\renewcommand{\succeq}{\succcurlyeq}
\newcommand{\Rmax}{\mathbb{R}_{\mathrm{max}}}
\newcommand{\Rmin}{\mathbb{R}_{\mathrm{min}}}
\newcommand{\Rext}{\overline{\mathbb{R}}}
\newcommand{\R}{\mathbb{R}}
\newcommand{\N}{\mathbb{N}}
\newcommand{\A}{\mathbf{A}}
\newcommand{\B}{\mathbf{B}}
\newcommand{\x}{\mathbf{x}}
\newcommand{\e}{\mathbf{e}}
\newcommand{\X}{\mathbf{X}}
\newcommand{\W}{\mathbf{W}}
\newcommand{\weights}{\mathcal{W}}
\newcommand{\alternatives}{\mathcal{X}}
\newcommand{\xsol}{\bar{\mathbf{x}}}
\newcommand{\y}{\mathbf{y}}
\newcommand{\Y}{\mathbf{Y}}
\newcommand{\z}{\mathbf{z}}
\newcommand{\Z}{\mathbf{Z}}
\renewcommand{\a}{\mathbf{a}}
\renewcommand{\b}{\mathbf{b}}
\newcommand{\I}{\mathbf{I}}
\DeclareMathOperator{\supp}{supp}
\newcommand{\Par}[2]{\mathcal{P}_{{#1} \to {#2}}}
\newcommand{\Laplacian}{\mathcal{L}}
\newcommand{\F}{\mathcal{F}}
\newcommand{\inv}[1]{{#1}^{\sharp}}
\newcommand{\energy}{Q}
\newcommand{\err}{\mathrm{err}}
\newcommand{\argmin}{\mathrm{argmin}}
\newcommand{\argmax}{\mathrm{argmax}}

\makeatletter
\newtheorem*{rep@theorem}{\rep@title}
\newcommand{\newreptheorem}[2]{%
\newenvironment{rep#1}[1]{%
 \def\rep@title{#2 \ref{##1}}%
 \begin{rep@theorem}}%
 {\end{rep@theorem}}}
\makeatother

\newreptheorem{theorem}{Theorem}
\newreptheorem{lemma}{Lemma}

\theoremstyle{remark}
\newtheorem*{remark}{Remark}

\begin{document}

%% Title information
\title[What Happens-After the First Race?]{What Happens-After the First Race?}         %% [Short Title] is optional;
                                        %% when present, will be used in
                                        %% header instead of Full Title.
% \titlenote{with title note}             %% \titlenote is optional;
                                        %% can be repeated if necessary;
                                        %% contents suppressed with 'anonymous'
\subtitle{Enhancing the Predictive Power of Happens-Before Based Dynamic Race Detection}                     %% \subtitle is optional
% \subtitlenote{with subtitle note}       %% \subtitlenote is optional;
                                        %% can be repeated if necessary;
                                        %% contents suppressed with 'anonymous'


%% Author information
%% Contents and number of authors suppressed with 'anonymous'.
%% Each author should be introduced by \author, followed by
%% \authornote (optional), \orcid (optional), \affiliation, and
%% \email.
%% An author may have multiple affiliations and/or emails; repeat the
%% appropriate command.
%% Many elements are not rendered, but should be provided for metadata
%% extraction tools.

%% Author with single affiliation.
\author{Umang Mathur}
% \authornote{with author1 note}          %% \authornote is optional;
                                        %% can be repeated if necessary
\orcid{0000-0002-7610-0660}             %% \orcid is optional
\affiliation{
%   \position{Position2a}
  \department{Department of Computer Science}              %% \department is recommended
  \institution{University of Illinois, Urbana Champaign}            %% \institution is required
  % \streetaddress{Street1 Address1}
  % \city{City1}
  % \state{State1}
  % \postcode{Post-Code1}
  \country{USA}                    %% \country is recommended
}
\email{umathur3@illinois.edu}          %% \email is recommended

%% Author with two affiliations and emails.
\author{Dileep Kini}
\affiliation{
%   \position{Position2a}
  % \department{Department of Computer Science}              %% \department is recommended
  \institution{Akuna Capital LLC}            %% \institution is required
  % \streetaddress{Street1 Address1}
  % \city{City1}
  % \state{State1}
  % \postcode{Post-Code1}
  \country{USA}                    %% \country is recommended
}
\email{dileeprkini@gmail.com}          %% \email is recommended

\author{Mahesh Viswanathan}
\affiliation{
%   \position{Position2a}
  \department{Department of Computer Science}              %% \department is recommended
  \institution{University of Illinois, Urbana Champaign}            %% \institution is required
  % \streetaddress{Street1 Address1}
  % \city{City1}
  % \state{State1}
  % \postcode{Post-Code1}
  \country{USA}                    %% \country is recommended
}
\email{vmahesh@illinois.edu} 
% \authornote{with author2 note}          %% \authornote is optional;
                                        %% can be repeated if necessary
% \orcid{nnnn-nnnn-nnnn-nnnn}             %% \orcid is optional
% \affiliation{
%   \position{Position2a}
%   \department{Department2a}             %% \department is recommended
%   \institution{Institution2a}           %% \institution is required
%   \streetaddress{Street2a Address2a}
%   \city{City2a}
%   \state{State2a}
%   \postcode{Post-Code2a}
%   \country{Country2a}                   %% \country is recommended
% }
% \email{first2.last2@inst2a.com}         %% \email is recommended
% \affiliation{
%   \position{Position2b}
%   \department{Department2b}             %% \department is recommended
%   \institution{Institution2b}           %% \institution is required
%   \streetaddress{Street3b Address2b}
%   \city{City2b}
%   \state{State2b}
%   \postcode{Post-Code2b}
%   \country{Country2b}                   %% \country is recommended
% }
% \email{first2.last2@inst2b.org}         %% \email is recommended


%% Abstract
%% Note: \begin{abstract}...\end{abstract} environment must come
%% before \maketitle command
\begin{abstract}
  In this paper, we explore the connection between secret key agreement and secure omniscience within the setting of the multiterminal source model with a wiretapper who has side information. While the secret key agreement problem considers the generation of a maximum-rate secret key through public discussion, the secure omniscience problem is concerned with communication protocols for omniscience that minimize the rate of information leakage to the wiretapper. The starting point of our work is a lower bound on the minimum leakage rate for omniscience, $\rl$, in terms of the wiretap secret key capacity, $\wskc$. Our interest is in identifying broad classes of sources for which this lower bound is met with equality, in which case we say that there is a duality between secure omniscience and secret key agreement. We show that this duality holds in the case of certain finite linear source (FLS) models, such as two-terminal FLS models and pairwise independent network models on trees with a linear wiretapper. Duality also holds for any FLS model in which $\wskc$ is achieved by a perfect linear secret key agreement scheme. We conjecture that the duality in fact holds unconditionally for any FLS model. On the negative side, we give an example of a (non-FLS) source model for which duality does not hold if we limit ourselves to communication-for-omniscience protocols with at most two (interactive) communications.  We also address the secure function computation problem and explore the connection between the minimum leakage rate for computing a function and the wiretap secret key capacity.
  
%   Finally, we demonstrate the usefulness of our lower bound on $\rl$ by using it to derive equivalent conditions for the positivity of $\wskc$ in the multiterminal model. This extends a recent result of Gohari, G\"{u}nl\"{u} and Kramer (2020) obtained for the two-user setting.
  
   
%   In this paper, we study the problem of secret key generation through an omniscience achieving communication that minimizes the 
%   leakage rate to a wiretapper who has side information in the setting of multiterminal source model.  We explore this problem by deriving a lower bound on the wiretap secret key capacity $\wskc$ in terms of the minimum leakage rate for omniscience, $\rl$. 
%   %The former quantity is defined to be the maximum secret key rate achievable, and the latter one is defined as the minimum possible leakage rate about the source through an omniscience scheme to a wiretapper. 
%   The main focus of our work is the characterization of the sources for which the lower bound holds with equality \textemdash it is referred to as a duality between secure omniscience and wiretap secret key agreement. For general source models, we show that duality need not hold if we limit to the communication protocols with at most two (interactive) communications. In the case when there is no restriction on the number of communications, whether the duality holds or not is still unknown. However, we resolve this question affirmatively for two-user finite linear sources (FLS) and pairwise independent networks (PIN) defined on trees, a subclass of FLS. Moreover, for these sources, we give a single-letter expression for $\wskc$. Furthermore, in the direction of proving the conjecture that duality holds for all FLS, we show that if $\wskc$ is achieved by a \emph{perfect} secret key agreement scheme for FLS then the duality must hold. All these results mount up the evidence in favor of the conjecture on FLS. Moreover, we demonstrate the usefulness of our lower bound on $\wskc$ in terms of $\rl$ by deriving some equivalent conditions on the positivity of secret key capacity for multiterminal source model. Our result indeed extends the work of Gohari, G\"{u}nl\"{u} and Kramer in two-user case.
\end{abstract}


%% 2012 ACM Computing Classification System (CSS) concepts
%% Generate at 'http://dl.acm.org/ccs/ccs.cfm'.
\begin{CCSXML}
<ccs2012>
<concept>
<concept_id>10011007.10011006.10011008</concept_id>
<concept_desc>Software and its engineering~General programming languages</concept_desc>
<concept_significance>500</concept_significance>
</concept>
<concept>
<concept_id>10003456.10003457.10003521.10003525</concept_id>
<concept_desc>Social and professional topics~History of programming languages</concept_desc>
<concept_significance>300</concept_significance>
</concept>
</ccs2012>
\end{CCSXML}

\ccsdesc[500]{Software and its engineering~General programming languages}
\ccsdesc[300]{Social and professional topics~History of programming languages}
%% End of generated code


%% Keywords
%% comma separated list
\keywords{Concurrency, Race Detection, Dynamic Program Analysis, Soundness, Happens-Before}  %% \keywords are mandatory in final camera-ready submission


%% \maketitle
%% Note: \maketitle command must come after title commands, author
%% commands, abstract environment, Computing Classification System
%% environment and commands, and keywords command.
\maketitle


\section{Introduction}
\label{sec:intro}
% !TEX root = ../arxiv.tex

Unsupervised domain adaptation (UDA) is a variant of semi-supervised learning \cite{blum1998combining}, where the available unlabelled data comes from a different distribution than the annotated dataset \cite{Ben-DavidBCP06}.
A case in point is to exploit synthetic data, where annotation is more accessible compared to the costly labelling of real-world images \cite{RichterVRK16,RosSMVL16}.
Along with some success in addressing UDA for semantic segmentation \cite{TsaiHSS0C18,VuJBCP19,0001S20,ZouYKW18}, the developed methods are growing increasingly sophisticated and often combine style transfer networks, adversarial training or network ensembles \cite{KimB20a,LiYV19,TsaiSSC19,Yang_2020_ECCV}.
This increase in model complexity impedes reproducibility, potentially slowing further progress.

In this work, we propose a UDA framework reaching state-of-the-art segmentation accuracy (measured by the Intersection-over-Union, IoU) without incurring substantial training efforts.
Toward this goal, we adopt a simple semi-supervised approach, \emph{self-training} \cite{ChenWB11,lee2013pseudo,ZouYKW18}, used in recent works only in conjunction with adversarial training or network ensembles \cite{ChoiKK19,KimB20a,Mei_2020_ECCV,Wang_2020_ECCV,0001S20,Zheng_2020_IJCV,ZhengY20}.
By contrast, we use self-training \emph{standalone}.
Compared to previous self-training methods \cite{ChenLCCCZAS20,Li_2020_ECCV,subhani2020learning,ZouYKW18,ZouYLKW19}, our approach also sidesteps the inconvenience of multiple training rounds, as they often require expert intervention between consecutive rounds.
We train our model using co-evolving pseudo labels end-to-end without such need.

\begin{figure}[t]%
    \centering
    \def\svgwidth{\linewidth}
    \input{figures/preview/bars.pdf_tex}
    \caption{\textbf{Results preview.} Unlike much recent work that combines multiple training paradigms, such as adversarial training and style transfer, our approach retains the modest single-round training complexity of self-training, yet improves the state of the art for adapting semantic segmentation by a significant margin.}
    \label{fig:preview}
\end{figure}

Our method leverages the ubiquitous \emph{data augmentation} techniques from fully supervised learning \cite{deeplabv3plus2018,ZhaoSQWJ17}: photometric jitter, flipping and multi-scale cropping.
We enforce \emph{consistency} of the semantic maps produced by the model across these image perturbations.
The following assumption formalises the key premise:

\myparagraph{Assumption 1.}
Let $f: \mathcal{I} \rightarrow \mathcal{M}$ represent a pixelwise mapping from images $\mathcal{I}$ to semantic output $\mathcal{M}$.
Denote $\rho_{\bm{\epsilon}}: \mathcal{I} \rightarrow \mathcal{I}$ a photometric image transform and, similarly, $\tau_{\bm{\epsilon}'}: \mathcal{I} \rightarrow \mathcal{I}$ a spatial similarity transformation, where $\bm{\epsilon},\bm{\epsilon}'\sim p(\cdot)$ are control variables following some pre-defined density (\eg, $p \equiv \mathcal{N}(0, 1)$).
Then, for any image $I \in \mathcal{I}$, $f$ is \emph{invariant} under $\rho_{\bm{\epsilon}}$ and \emph{equivariant} under $\tau_{\bm{\epsilon}'}$, \ie~$f(\rho_{\bm{\epsilon}}(I)) = f(I)$ and $f(\tau_{\bm{\epsilon}'}(I)) = \tau_{\bm{\epsilon}'}(f(I))$.

\smallskip
\noindent Next, we introduce a training framework using a \emph{momentum network} -- a slowly advancing copy of the original model.
The momentum network provides stable, yet recent targets for model updates, as opposed to the fixed supervision in model distillation \cite{Chen0G18,Zheng_2020_IJCV,ZhengY20}.
We also re-visit the problem of long-tail recognition in the context of generating pseudo labels for self-supervision.
In particular, we maintain an \emph{exponentially moving class prior} used to discount the confidence thresholds for those classes with few samples and increase their relative contribution to the training loss.
Our framework is simple to train, adds moderate computational overhead compared to a fully supervised setup, yet sets a new state of the art on established benchmarks (\cf \cref{fig:preview}).


\section{Preliminaries}
\label{sec:prelim}
\section{Preliminaries}
Given a graph $G=(V,E)$, and vertex $u \in V$, let $\deg(u,G)$ be the degree of $u$ in $G$. 
Given a tree $T$ and $u, v \in T$, denote the $u$-$v$ path in $T$ by $\pi(u,v,T)$. When the tree $T$ is clear from the context, we may omit it and write $\pi(u,v)$. For a (possibly weighted) subgraph $G' \subseteq G$ and a vertex pair $s,t \in V$, let $\dist_{G'}(s,t)$ denote the length of the $s$-$t$ shortest path in $G'$. 

\paragraph{Fault-Tolerant Labeling Schemes.}
For a given graph $G$, let $\Pi: V\times V \times \mathcal{G} \to \mathbb{R}_{\geq 0}$ %\mtodo{a reviewer pointed out that the last element in the domain should probably be subgraphs of $G$ and not $G$, see which notation we want here.} \mertodo{I am not sure that it is needed as we consider $G$ in the fault-free setting and $G \setminus F$ in the FT setting. We can of course write $\Pi: V\times V \times \mathcal{G} \to \mathbb{R}_{\geq 0}$ where $\mathcal{G}$ is the family of all $G$-subgraphs, but I cannot see why we need it.} 
be a function defined on pairs of vertices and a subgraph $G' \subset G$, where $\mathcal{G}$ is the family of all subgraphs of $G$. For an integer parameter $f\geq 1$, an $f$-\emph{fault-tolerant labeling scheme} for a function $\Pi$ and a graph family $\mathcal{F}$ is a pair of functions $(L_{\Pi},D_{\Pi})$. The function $L_{\Pi}$ is called the \emph{labeling function}, and $D_{\Pi}$ is called the \emph{decoding function}. For every graph $G$ in the family $\mathcal{F}$, the labeling function $L_{\Pi}$ associates with each vertex $u \in V(G)$ and every edge $e \in E(G)$, a label $L_{\Pi}(u,G)$ (resp., $L_{\Pi}(e,G)$). It is then required that given the labels of any triplets $s,t, F \in V \times V \times E^f$, the decoding function $D_{\Pi}$ computes $\Pi(s,t, G \setminus F)$.  The primary complexity measure of a labeling scheme is the \emph{label length}, measured by the length (in bits) of the largest label it assigns to some vertices (or edges) in $G$ over all graphs $G \in \mathcal{F}$. An $f$-FT connectivity labeling scheme is required to output YES iff $s$ and $t$ are connected in $G \setminus F$.  In $f$-FT \emph{approximate distance labeling scheme} it is required to output an estimate for the $s$-$t$ distance in the graph $G \setminus F$. Formally, an $f$-FT labeling scheme is $q$\emph{-approximate} if the value $\delta(s,t,F)$ returned by the decoder algorithm satisfies that $\dist_{G \setminus F}(s,t)\leq \delta(s,t,F) \leq q \cdot \dist_{G \setminus F}(s,t)$.  Throughout the paper we provide randomized labeling schemes which provide a high probability guarantee of correctness for any fixed triplet $\langle s,t, F \rangle$. 


\paragraph{Fault-Tolerant Routing Schemes.} In the setting of FT routing scheme, one is given a pair of source $s$ and destination $t$ as well as $F$ edge faults, which are initially unknown to $s$. The routing scheme consists of \emph{preprocessing} and \emph{routing} algorithms. The preprocessing algorithm defines labels $L(u)$ to each of the vertices $u$, and a header $H(M)$ to the designated message $M$. In addition, it defines for every vertex $u$ a routing table $R(u)$. The labels and headers are usually required to be short, i.e., of poly-logarithmic bits. 
The routing procedure determines at each vertex $u$ the port-number on which $u$ should send the messages it receives. The computation of the next-hop is done by considering the header of the message $H(M)$, the label of the source and destination $L(s)$ and $L(t)$ and the routing table $R(u)$. The routing procedure at vertex $u$ might also edit the header of the message $H(M)$. The failing edges are not known in advance and can only be revealed by reaching (throughout the message routing) one of their endpoints. The \emph{space} of the scheme is determined based on maximal length of message headers, labels and the individual routing tables. The stretch of the scheme is measured by the ratio between the length of the path traversed until the message arrived its destination and the length of the shortest $s$-$t$ path in $G \setminus F$. In the more relaxed setting of \emph{forbidden-set routing schemes} the failing edges are given as input to the routing algorithm.

%\paragraph{Forbidden-Set Routing Schemes.} One of the key applications of labeling schemes is routing.
%In the setting of forbidden-set routing schemes, given the labels of $s$, $t$ and a set of forbidden edges $
%F$, it is required to determine the next-hop neighbor of $s$ on some short $s$-$t$ path in $G \setminus F$. The main two complexity measures are the stretch induced by the $s$-$t$ path encoded by the labels.  \mtodo{Maybe the definition should be more similar to the next one? (consider labels, tables, headers)}
%That is, given the labels of $u$, $v$ and the faults $F$, the decoder function returns the port-number of $u$'s neighbor lying on a $u$-$v$ path $P$ in $G \setminus F$ such that $P \subseteq G$ and $len(P)\leq s \cdot \dist(u,v, G \setminus F)$ for some approximation factor $s$. 


\section{Characterizing Schedulable Races}
\label{sec:shb}
%!TEX root = main.tex

The example in Fig.~\ref{fig:example1} shows that not every HB-race
corresponds to an actual data race in the program. The goal of this
section is to characterize those HB-races which correspond to actual
data races. We do this by introducing a new partial order, called
schedulable happens-before, and using it to identify the actual data
races amongst the HB-races of a trace. We begin by characterizing the
HB-races that correspond to actual data races.
%
\begin{definition}[$\hb{\tr}$-schedulable race]
\label{def:hb-sched-race}
Let $\tr$ be a trace and let $e \trord{\tr} e'$ be conflicting events in $\tr$.
We say that $(e,e')$ is a $\hb{\tr}$-schedulable race if there is a correct
reordering $\tr'$ of $\tr$ that respects $\hb{\tr}$ and $\tr' = \tr''
ee'$ or $\tr' = \tr'' e'e$ for some trace $\tr''$. 
%
% We should keep this definition more general than needed.
% \ucomment{Only $\tr''ee'$}
\end{definition}

Note that any $\hb{\tr}$-schedulable race is a valid data race in
$\tr$. Our aim is to characterize $\hb{\tr}$-schedulable races by means
of a new partial order. 
The new partial order, given below, is a strengthening of $\hb{}$.
%
\begin{definition}[Schedulable Happens-Before]
\label{def:shb}
Let $\tr$ be a trace. Schedulable happens-before, denoted by
$\shb{\tr}$, is the smallest partial order on $\events{\tr}$ such that
\begin{enumerate}[label=({\alph*})]
\item $\hb{\tr} \subseteq \shb{\tr}$
\item $\forall e,e' \in \events{\tr}, e' \in \reads{\tr} \land 
  e = \lw{\sigma}(e') \implies e \shb{\tr} e'$
\end{enumerate}
\end{definition}
%
The partial order $\shb{\tr}$ can be used to characterize
$\hb{\tr}$-schedulable races. We state this result, before giving
examples illustrating the definition of $\shb{\tr}$.
%
% \begin{theorem}%[SHB Soundness]
% \label{thm:SHBSoundness}
% Let $\tr$ be a trace and $(e_1, e_2)$ be an HB-race. 
% \begin{enumerate}
% \item\label{lbl:necessary} If there is an event $e \in \events{\tr} 
%   \setminus \{e_1, e_2\}$ such that $e_1 \shb{\tr} e \shb{\tr} e_2$
%   then $(e_1,e_2)$ is not a $\hb{\tr}$-schedulable race.
% \item\label{lbl:sufficient-write} Suppose $e_1 \in \writes{\tr}$ and 
%   there is no event $e \in \events{\tr} \setminus \{e_1, e_2\}$ such
%   that $e_1 \shb{\tr} e \shb{\tr} e_2$. Then $(e_1,e_2)$ is a
%   $\hb{\tr}$-schedulable race.
% \item\label{lbl:sufficient-read} Let $e_1 \in \reads{\tr}(x)$, for some 
%   variable $x$, such that there is no event
%   $e \in \writes{\tr}(x) \setminus \{e_2\}$ such that $e_1 \trord{\tr}
%   e \shb{\tr} e_2$. Further, suppose that there is no event
%   $e \in \events{\tr} \setminus \{e_1, e_2\}$ such that $e_1 \shb{\tr}
%   e \shb{\tr} e_2$.  Then $(e_1,e_2)$ is a $\hb{\tr}$-schedulable
%   race.
% \end{enumerate}
% \end{theorem} 
% \begin{proof}(Sketch)
% The full proof is presented in Appendix~\ref{app:shb-proof}; here we
% sketch the main ideas. Part~(\ref{lbl:necessary}) follows from the
% observation that any correct reordering of $\tr'$ of $\tr$ that also
% respects $\hb{\tr}$, also respects $\shb{\tr}$. That is, for any
% $e,e'$ such that $e \shb{\tr} e'$ and $e' \in \events{\tr'}$, we have
% $e \in \events{\tr'}$ and $e \trord{\tr'}
% e'$. Parts~(\ref{lbl:sufficient-write})
% and~(\ref{lbl:sufficient-read}) are established as follows. Let
% $\tr''$ be the trace consisting of events that are $\shb{\tr}$-below
% $e_1$ or $e_2$, ordered as in $\tr$. Define $\tr' = \tr''e_2e_1$, when
% $e_2 \in \reads{\tr}$ and $e_1 \neq \lw{\tr}(e_2)$, and $\tr'
% = \tr''e_1e_2$, otherwise. We prove that when the conditions in
% parts~(\ref{lbl:sufficient-write}) and~(\ref{lbl:sufficient-read})
% hold, $\tr'$ as defined here, is a correct reordering.
% \end{proof}
\begin{theorem}%[SHB Soundness]
\label{thm:SHBSoundness}
Let $\tr$ be a trace and $e_1 \trord{\tr} e_2$ be conflicting events in $\tr$.
$(e_1, e_2)$ is an $\hb{\tr}$-schedulable race iff 
either $\ltho{\tr}(e_2)$ is undefined, or $e_1 \not\leq^\tr_{\mathsf{SHB}} \ltho{\tr}(e_2)$.
  % $\neg (e_1 \shb{\tr} \ltho{\tr}(e_2))$.
% there is no event $e \in \events{\tr} \setminus \{e_1, e_2\}$ such
%   that $e_1 \shb{\tr} e \shb{\tr} e_2$.
\end{theorem}

\begin{proof}(Sketch)
The full proof is presented in Appendix~\ref{app:shb-proof}; here we
sketch the main ideas.  We observe that if $\tr'$ is a correct
reordering of $\tr$ that also respects $\hb{\tr}$, then $\tr'$ also
respects $\shb{\tr}$ except possibly for the last events of every
thread in $\tr'$. That is, for any $e,e'$ such that $e \shb{\tr} e'$,
$e' \in \events{\tr'}$, and $e'$ is not the last event of some thread
in $\tr'$, we have $e \in \events{\tr'}$ and $e \trord{\tr'}
e'$. Therefore, if $e \shb{\tr} \ltho{\tr}(e_2)$, then any correct
reordering $\tr'$ respecting $\hb{\tr}$ that contains both $e_1$ and
$e_2$ will also have $e = \ltho{\tr}(e_2)$. Further since $e$ is not
the last event of its thread (since $e_2$ is present in $\tr'$) and
$e_1 \shb{\tr} e$, $e$ must occur between $e_1$ and $e_2$ in
$\tr'$. Therefore $(e_1,e_2)$ is not a $\hb{\tr}$-schedulable race.
The other direction can be established as follows. Let $\tr''$ be the
trace consisting of events that are $\shb{\tr}$-before $e_1$ or
$\ltho{\tr}(e_2)$ (if defined), ordered as in $\tr$. Define
$\tr'= \tr''e_1e_2$.  We prove that when $e_1$ and $e_2$ satisfy the
condition in the theorem, $\tr'$ as defined here, is a correct
reordering and also respects $\hb{\tr}$.
\end{proof}

\begin{figure}
\centering
\execution{4}{
  \figev{1}{\acq(\lk)}
  \figev{1}{\wt(x)}
  \figev{2}{\rd(x)}
  \figev{2}{\wt(y)}
  \figev{2}{\wt(x)}
  \figev{1}{\rd(x)}
  \figev{1}{\rel(\lk)}
  \figev{4}{\acq(\lk)}
  \figev{4}{\wt(z)}
  \figev{3}{\rd(z)}
  \figev{3}{\wt(y)}
  \figev{3}{\wt(z)}
  \figev{4}{\rd(z)}
  \figev{4}{\rel(\lk)}
}
\caption{Trace $\tr_4$.}
\label{fig:example4}
\end{figure}


We now illustrate the use of $\shb{}$ through some examples.
%
\begin{example}
In this example, we will look at different traces, and see how
$\shb{}$ reasons. Like in the introduction, we will use $e_i$ to refer
to the $i$th event of a given trace (which will be clear from
context). Let us begin by considering the example program and trace
$\tr_1$ from Fig.~\ref{fig:example1}. Notice that $\hb{\tr_1}
= \tho{\tr_1}$, and so $(e_1,e_4)$ and $(e_2,e_3)$ are
HB-races. Because $e_2 = \lw{\tr_1}(e_3)$, we have $e_1 \shb{\tr_1}
e_2 \shb{\tr_1} e_3 \shb{\tr_1} e_4$. Using
Theorem~\ref{thm:SHBSoundness}, we can conclude correctly that (a)
$(e_2,e_3)$ is $\hb{\tr_1}$-schedulable as $\ltho{\tr_1}(e_3)$ is
undefined, but (b) $(e_1,e_4)$ is not, as
$e_1 \shb{\tr} \ltho{\tr}(e_4) = e_3$.

Let us now consider trace $\tr_2$ from
Fig.~\ref{fig:example2}. Observe that $\hb{\tr_2} = \shb{\tr_2}
= \tho{\tr_2}$, and so both $(e_1,e_4)$ and $(e_2,e_3)$ are
$\hb{\tr_2}$-schedulable races by Theorem~\ref{thm:SHBSoundness}. Note
that, unlike force ordering, $\shb{\tr_2}$ correctly identifies all
real data races.

Finally, let us consider two trace examples that highlight the kind of
subtle reasoning $\shb{}$ is capable of. Let us begin with $\tr_3$
from Fig.~\ref{fig:example3}. As observed in Example~\ref{ex:hb-race},
the only HB-races in this trace are $(e_2,e_7)$, $(e_5,e_7)$,
$(e_2,e_9)$, $(e_5,e_9)$, $(e_2,e_{10})$, $(e_5,e_{10})$,
$(e_2,e_{12})$, and $(e_5,e_{12})$. Both $(e_2,e_7)$ and $(e_5,e_7)$
are $\hb{\tr_3}$-schedulable as demonstrated by the reorderings
$\rho_1$ and $\rho_3$ from Example~\ref{ex:reorderings}. However, the
remaining are not real data races. Let us consider the pairs
$(e_2,e_9)$ and $(e_5,e_9)$ for
example. Theorem~\ref{thm:SHBSoundness}'s justification for it is as
follows: $e_2 \hb{\tr_3} e_5 = \lw{\tr_3}(e_7) \tho{\tr_3} e_8
= \ltho{\tr_3}(e_9)$. But, let us unravel the reasoning behind why
neither $(e_2,e_9)$ nor $(e_5,e_9)$ are data races. Consider an
arbitary correct reordering $\tr'$ of $\tr_3$ that respects
$\hb{\tr_3}$ and contains $e_9$. Since $e_8$ is also an event of
$t_4$, $e_8 \in \events{\tr'}$. In addition, $e_7 \in \events{\tr'}$
as $e_7 \tho{\tr_3} e_8$. Now, since $e_5 = \lw{\tr_3}(e_7)$, $e_5$ is
before $e_7$ in $\tr'$ and since $e_2 \hb{\tr_3} e_5$, $e_2$ must also
be before $e_7$. Therefore, $e_7$ and $e_8$ will be between $e_2$ and
$e_9$ and between $e_5$ and $e_9$. Similar reasoning can be used to
conclude that the other pairs are not $\hb{\tr_3}$-schedulable as
well. 

Lastly, consider trace $\tr_4$ shown in Fig.~\ref{fig:example4}. In
this case, $\shb{\tr_4} = \trord{\tr_4}$. All conflicting memory
accesses are in HB-race.  While HB correctly identifies the first race
$(e_2, e_3)$ as valid, there are 3 HB-races that are not real data
races --- $(e_2,e_5)$, $(e_9,e_{12})$, and $(e_4,e_{11})$. $(e_2,e_5)$
is not valid because any correct reordering of $\tr_4$ must have $e_2$
before $e_3$ and $e_3$ before $e_5$. This is also captured by SHB
reasoning because $e_2 \shb{\tr_4} e_3 \tho{\tr_4} e_4
= \ltho{\tr_4}(e_5)$. A similar reasoning shows that $(e_9,e_{12})$ is
not valid. The interesting case is that of $(e_4,e_{11})$. Here, in
any correct reordering $\tr'$ of $\tr_4$, the following must be true:
(a) if $e_4 \in \events{\tr'}$ then $e_1 \in \events{\tr'}$; (b) if
$e_{11} \in \events{\tr'}$ then $e_8 \in \events{\tr'}$; (c) if
$\{e_1,e_4,e_7\} \subseteq \events{\tr'}$ then $e_1 \trord{\tr'}
e_4 \trord{\tr'} e_7$; and (d) if
$\{e_8,e_{11},e_{14}\} \subseteq \events{\tr'}$ then $e_8 \trord{\tr'}
e_{11} \trord{\tr'} e_{14}$. Therefore, any correct reordering $\tr'$
of $\tr_4$ containing both $e_4$ and $e_{11}$ contains $e_1$ and $e_8$
(because of (a) and (b)) and must contain at least one of $e_7$ or
$e_{14}$ to ensure that critical sections of $\lk$ don't overlap. Then
in $\tr'$, $e_4$ and $e_{11}$ cannot be consecutive because either
$e_7$ or $e_{14}$ will appear between them (properties (c) and
(d)). This is captured using SHB and Theorem~\ref{thm:SHBSoundness} by
the fact that $e_4 \shb{\tr_3} e_7 \shb{\tr_3} e_{10}
= \ltho{\tr_4}(e_{11})$.
\end{example}

We conclude this section by observing that the soundness guarantees of
HB (Theorem~\ref{thm:hb_sound}) follows from
Theorem~\ref{thm:SHBSoundness}. Consider a trace $\tr$ whose first
HB-race is $(e_1,e_2)$. We claim that $(e_1,e_2)$ is a
$\hb{\tr}$-schedulable race. Suppose (for contradiction) it is
not. Then by Theorem~\ref{thm:SHBSoundness}, $e = \ltho{\tr}(e_2)$ is
defined and $e_1 \shb{\tr} e$.  Now observe that we must have
$\neg(e_1 \hb{\tr} e)$ (or otherwise $e_1 \hb{\tr} e_2$, contradicting
our assumption that $(e_1, e_2)$ is an HB-race).  Then, by the
definition of $\shb{\tr}$ (Definition~\ref{def:shb}), there are two
events $e_3$ and $e_4$ (possibly same as $e_1$ and $e$) such that
$e_1 \shb{\tr} e_3$, $e_3 = \lw{\tr}(e_4)$, $e_4 \shb{\tr} e$, and
$\neg (e_3 \hb{\tr} e_4)$.  Then $(e_3,e_4)$ is an HB-race, and it
contradicts the assumption that $(e_1,e_2)$ is the first HB-race.

The above argument that Theorem~\ref{thm:hb_sound} follows from
Theorem~\ref{thm:SHBSoundness}, establishes that our SHB-based
analysis using Theorem~\ref{thm:SHBSoundness} does not miss the race
detected by a \emph{sound} HB-based race detection algorithm.

% \subsection{Discussion}
% \label{sec:shb-discussion}

% Consider an HB-race $(e_1,e_2)$ of trace $\tr$ such that
% $e_1 \in \writes{\tr}$. Conditions~\ref{lbl:necessary}
% and~\ref{lbl:sufficient-write} of Theorem~\ref{thm:SHBSoundness} are
% converses of each other. Thus, if $(e_1,e_2)$ is an HB-race with
% $e_1 \in \writes{\tr}$ then $(e_1,e_2)$ is $\hb{\tr}$-schedulable if
% and only if there is no event $e \not\in \{e_1,e_2\}$, with
% $e_1 \shb{\tr} e \shb{\tr} e_2$. If $\shb{\tr}$ can be computed
% efficiently, we can use it, together with Theorem~\ref{thm:SHBSoundness}, to
% detect all $\hb{\tr}$-schedulable races $(e_1,e_2)$, where
% $e_1 \in \writes{\tr}$.

% However, when $e_1 \in \reads{\tr}$, Theorem~\ref{thm:SHBSoundness}
% does not provide an exact characterization. Getting an exact
% characterization in this case seems difficult. In
% Appendix~\ref{app:shb-examples}, we present examples that highlight
% some of the subtle challenges in this enterprise.

% Even if one obtains an exact characterization of all read-write
% $\hb{\tr}$-schedulable races, computing this set is likely to be much
% more difficult than detecting races $(e_1,e_2)$, where
% $e_1 \in \writes{\tr}$. In Section~\ref{sec:algo} we present a vector
% clock based algorithm to compute $\shb{}$ which has the same
% characteristics as the vector clock algorithm for $\hb{}$ --- it is a
% single pass algorithm that processes every event of the trace exactly
% once, and if the trace has constantly many threads, locks, and
% variables, uses only $O(\log n)$ memory, where $n$ is the length of the
% trace (Theorem~\ref{thm:complexityVC}). On the other hand, we prove
% that any streaming algorithm detecting all $\hb{}$-schedulable
% read-write races requres at least $\Omega(n)$ memory.

% \begin{theorem}
% \label{thm:lowerbound}
% Let $\tr$ be a trace with $n$ events and constantly many threads,
% variables, and locks. Let $(e_1,e_2)$ be a HB-race with
% $e_1 \in \reads{\tr}$. Any single pass, streaming algorithm that
% determines if $(e_1,e_2)$ is a $\hb{\tr}$-schedulable race, must use
% $\Omega(n)$ space.
% \end{theorem}

% The proof of this theorem is postponed to
% Appendix~\ref{app:lowerbound}.


\section{Algorithm for Detecting \texorpdfstring{$\hb{}$}{HB}-schedulable Races}
\label{sec:algo}
\section{DROP: Workload Optimization}
\label{sec:algo}

In this section, we introduce DROP, a system that performs workload-aware DR via progressive sampling and online progress estimation.
DROP takes as input a target dataset, metric to preserve (default, target $TLB$), and an optional downstream runtime model.
DROP then uses sample-based PCA to identify and return a low-dimensional representation of the input that preserves the specified property while minimizing estimated workload runtime (Figure 2, Alg.~\ref{alg:DROP}).

%DROP answers a crucial question that stochastic PCA techniques have traditionally ignored: how long should these methods run? 

\begin{comment}
Notation used is in Table~\ref{table:inputs}.

\begin{table}
\centering
\small
\caption{\label{table:inputs} 
 DROP algorithm notation and defaults}
{\renewcommand{\arraystretch}{1.2}
\begin{tabular}{|c|l l|}
\hline 
Symbol & Description (\emph{Default}) & Type\tabularnewline
\hline
$X$  & Input dataset                          & $\mathbb{R}^{\mvar \times \dvar}$ \tabularnewline
$\mvar$  & Number of input data points            & $\mathbb{Z}_{+}$\tabularnewline
$\dvar$  & Input data dimension                   & $\mathbb{Z}_{+}$ \tabularnewline
$B$  & Target $TLB$ preservation      		 & $0 < \mathbb{R} \leq 1 $ \tabularnewline
$\mathcal{C}_\mvar(\dvar)$  & Downstream runtime function (\textit{k-NN runtime})       & $\mathbb{Z}_{+} \to \mathbb{R}_{+}$\tabularnewline
$R$  & Total DROP runtime       & $\mathbb{R}_{+}$ \tabularnewline
$c$ & Confidence level for $TLB$ preservation (\textit{$95 \%$})          & $\mathbb{R}$  \tabularnewline
$T_k$  & DROP output $k$-dimensional transformation &$\mathbb{R}^{\dvar \times k}$ \tabularnewline
$i $ & Current DROP iteration        & $\mathbb{Z}_+$  \tabularnewline

\hline 
\end{tabular}
}
\end{table}
\end{comment}

\begin{figure}
\includegraphics[width=\linewidth]{figs/progressive.pdf}
\caption[]{ Reduction in dimensionality for  $TLB = 0.80$ with progressive sampling. Dimensionality decreases until reaching a state equivalent to running PCA over the full dataset ("convergence").}
\label{fig:progressive}
\end{figure}

\subsection{DROP Algorithm}
\label{subsec:arch}
%DROP is a system that performs workload-aware dimensionality reduction, optimizing the combined runtime of downstream tasks and DR as defined in Problem~\ref{def:opt}.
DROP operates over a series of data samples, and determines when to terminate via \red{a} four-step procedure at each iteration: %progressive sampling, transformation evaluation, progress estimation, and cost-based optimization:

%To power this pipeline, DROP combines database and machine learning techniques spanning online aggregation (\S\ref{subsec:teval}), progress estimation (\S\ref{subsec:pest}), progressive sampling (\S\ref{subsec:psample}), and PCA approximation (\S\S\ref{subsec:pcaroutine},\ref{subsec:reuse}).

%We now provide a brief overview of DROP's sample-based iterative architecture before detailing each.

\begin{comment}
\item Progressive Sampling (\S\ref{subsec:psample}): DROP draws a data sample, performs PCA over it, and uses of a novel reuse mechanism across iterations (\S\ref{subsec:reuse}).

\item Transform Evaluation (\S\ref{subsec:teval}): DROP evaluates the above by identifying the size of the smallest metric-preserving transformation that can be extracted. 

\item Progress Estimation (\S\ref{subsec:pest}): Given the size of the smallest metric-preserving transform and the time required to obtain this transform, DROP estimates the size and computation time of continued iteration.

\item Cost-Based Optimization (\S\ref{subsec:opt}): DROP optimizes over DR and downstream task runtime to determine if it should terminate.
\end{comment}

\minihead{Step 1: Progressive Sampling (\S\ref{subsec:psample})}

\noindent DROP draws a data sample, performs PCA over it, and uses a novel reuse mechanism across iterations (\S\ref{subsec:reuse}).

\minihead{Step 2: Transform Evaluation (\S\ref{subsec:teval})} 

\noindent DROP evaluates the above by identifying the size of the smallest metric-preserving transformation that can be extracted. 

\minihead{Step 3: Progress Estimation (\S\ref{subsec:pest})} 

\noindent Given the size of the smallest metric-preserving transform and the time required to obtain this transform, DROP estimates the size and computation time of continued iteration.

\minihead{Step 4: Cost-Based Optimization (\S\ref{subsec:opt})} 

\noindent DROP optimizes over DR and downstream task runtime to determine if it should terminate.

\subsection{Progressive Sampling}
\label{subsec:psample}

Inspired by stochastic PCA methods (\S\ref{sec:relatedwork}), DROP uses sampling to tackle workload-aware DR. 
Many real-world \red{datasets} are intrinsically low-dimensional; a small data sample is sufficient to characterize dataset behavior. 
To verify, we extend our case study (\S\ref{sec:RQW}) by computing how many uniformly selected data samples are required to obtain a $TLB$-preserving transform with $k$ equal to input dimension $\dvar$.
On average, a sample of under $0.64\%$ $(\text{up to } 5.5\%)$ of the input is sufficient for $TLB = 0.75$, and under $4.2\%$ $(\text{up to } 38.6\%)$ is sufficient for $TLB=0.99$.  
If this sample rate is known a priori, we obtain up to \red{$91\times$ speedup} over PCA via SVD.%---with no algorithmic improvement. 

However, this benefit is dataset-dependent, and unknown a priori.
We thus turn to progressive sampling (gradually increasing the sample size) to identify how large a sample suffices.
Figure~\ref{fig:progressive} shows how the dimensionality required to attain a given $TLB$ changes when we vary dataset and proportion of data sampled.
Increasing the number of samples (which increases PCA runtime) provides lower $k$ for the same $TLB$.
However, this decrease in dimension plateaus as the number of samples increases.
Thus, while progressive sampling allows DROP to tune the amount of time spent on DR, DROP must determine when the downstream value of decreased dimension is overpowered by the cost of DR---that is, whether to sample to convergence or terminate early (e.g., at $0.3$ proportion of data sampled for SmallKitchenAppliances). 


Concretely, DROP first repeatedly chooses a subset of data and computes a $\dvar$-dimensional transformation via PCA on the subsample, and then proceeds to determine if continued sampling is beneficial to end-to-end runtime.
We consider a simple uniform sampling strategy: each iteration, DROP samples a fixed percentage of the data.
 
 
 
 
 
%Exploring data-dependent and weighted sampling schemes that are dependent on the current basis is an exciting area for future work. 
%While we considered a range of alternative sampling strategies, uniform sampling strikes a balance between computational and statistical efficiency. 
%Data-dependent and weighted sampling schemes that are dependent on the current basis may decrease the total number of iterations required by DROP, but may require expensive reshuffling of data at each iteration~\cite{coresets}. 

%DROP provides configurable strategies for both base number of samples and the per-iteration increment, in our experimental evaluation in \S\ref{sec:experiments}, we consider a sampling rate of $1\%$ per iteration.
%We discuss more sophisticated additions to this base sampling schedule in the extended manuscript.

\begin{algorithm}[t!]
\begin{algorithmic}[1]
\small
\Statex \textbf{Input:}  $X$: data; $B$: target metric preservation level; $\mathcal{C}_\mvar$: cost of downstream operations
\Statex \textbf{Output:} $T_k$: $k$-dimensional transformation matrix
\Statex
\Statex \hrule
\Function{drop}{$X,  B, \mathcal{C}_\mvar$}:
	\State Initialize: $i = 0; k_0 = \infty$ 
		\Comment{iteration and current basis size}
	\Do
		\State i$\texttt{++}$, \textsc{clock.restart}
		\State $X_i$ = \textsc{sample}($X, \textsc{sample-schedule}(i)$) \label{eq:sample}
			\Comment{\S~\ref{subsec:psample}}
		\State $T_{k_i}$ = \textsc{compute-transform}($X, X_i,  B$) \label{eq:evaluate}
			\Comment{\S~\ref{subsec:teval}}
		\State $r_i = \textsc{clock.elapsed}$	
			\Comment{$R = \sum_i r_i$}
		\State $\hat{k}_{i+1}, \hat{r}_{i+1} $ = \textsc{estimate}($k_i, r_i$) \label{eq:estimate}
			\Comment{\S~\ref{subsec:pest}}
	\doWhile{\textsc{optimize}($\mathcal{C}_\mvar,k_i,r_i,\hat{k}_{i+1}, \hat{r}_{i+1}$)} \label{eq:optimize}
		\Comment{\S~\ref{subsec:opt}}
	\\\Return{$T_{k_i}$}
\EndFunction
\end{algorithmic}
\caption{DROP Algorithm}
\label{alg:DROP}
\end{algorithm}



\subsection{Transform Evaluation}
\label{subsec:teval}
DROP must accurately and efficiently evaluate this iteration's performance with respect to the metric of interest \red{over the entire dataset}. 
%To do so, DROP adapts an approach for deterministic queries in online aggregation: treating quality metrics as aggregation functions and using confidence intervals for fast estimation. 
%We first discuss this approach in the context of $TLB$, then discuss how to extend this approach to alternative metrics at the end of this section.
We define this iteration's performance as the size of the lowest dimensional $TLB$-preserving transform ($k_i$) that it can return. 
There are two challenges in performance evaluation.
First, the lowest $TLB$-achieving $k_i$ is unknown a priori. 
Second, brute-force $TLB$ computation would dominate the runtime of computing PCA over a sample. 
We now describe how to solve these challenges.

\subsubsection{Computing the Lowest Dimensional Transformation}

Given the $\dvar$-dimensional transformation from step 1, to reduce dimensionality, DROP must determine if a smaller dimensional $TLB$-preserving transformation can be obtained and return the smallest such transform. 
Ideally, the smallest $k_i$ would be known a priori, but in practice, this is not true---thus, DROP uses the $TLB$ constraint and two properties of PCA to automatically identify it.
%A na\"ive strategy would evaluate the $TLB$ for every combination of the $\dvar$ basis vectors for every transformation size, requiring $O(2^\dvar)$ evaluations. 
%Instead, DROP exploits two key properties of PCA to avoid this.

First, PCA via SVD produces an orthogonal linear transformation where the principal components  are returned in order of decreasing dataset variance explained.
As a result, once DROP has computed the transformation matrix for dimension $\dvar$, DROP obtains the transformations for all dimensions $k$ less than $\dvar$ by truncating the matrix to $\dvar \times k$ .
%PCA via SVD produces an orthogonal linear transformation where the first principal component explains the most variance in the dataset, the second explains the second most---subject to being orthogonal to the first---and so on.  

Second, with respect to $TLB$ preservation, the more principal components that are retained, the better the lower-dimensional representation in terms of $TLB$.  
This is because orthogonal transformations such as PCA preserve inner products. 
Therefore, an $\dvar$-dimensional PCA perfectly preserves $\ell_2$-distance between data points. 
As $\ell_2$-distance is a sum of squared (positive) terms, the more principal components retained, the better the representation preserves $\ell_2$-distance.

Using the first property, DROP obtains all low-dimensional transformations for the sample from the $\dvar$-dimensional basis.  
Using the second property, DROP runs binary search over these transformations to return the lowest-dimensional basis that attains $B$ (Alg.~\ref{alg:candidate}, l\ref{eq:basis}).
If $B$ cannot be realized with this sample, DROP omits further optimization steps and continues the next iteration by drawing a larger sample.

Additionally, computing the full $\dvar$-dimensional basis at every iteration may be wasteful. 
Thus, if DROP has found a candidate $TLB$-preserving basis of size $\dvar' < \dvar$ in prior iterations, then DROP only computes $\dvar'$ components at the start of the next iteration.
This allows for more efficient PCA computation for future iterations, as advanced PCA routines can exploit the $\dvar'$-th eigengap to converge faster (\S\ref{sec:relatedwork}).
% \red{This is because similar to a hold-out or validation set, $TLB$ evaluation is representative of the entire dataset, not just the current sample (see Alg.~\ref{alg:candidate} L5). 
%Thus, sampling additional training datapoints enables DROP to better learn global data structure and perform at least as well as over a smaller sample.}


% stop here!

\subsubsection{Efficient $TLB$ Computation}

Given a transformation, DROP must determine if it preserves the desired $TLB$.
Computing pairwise $TLB$ for all data points requires $O(\mvar^2\dvar)$ time, which dominates the runtime of computing PCA on a sample.
However, as the $TLB$ is an average of random variables bounded from 0 to 1, DROP can use sampling and confidence intervals to compute the $TLB$ to arbitrary confidences.

Given a transformation, DROP iteratively refines an estimate of its $TLB$ (Alg.~\ref{alg:candidate}, l\ref{eq:eval}) by \red{incrementally sampling an increasing number of} pairs from the input data (Alg.~\ref{alg:candidate}, l\ref{eq:paircheck}), transforming each pair into the new basis, then measuring the distortion of $\ell_2$-distance between the pairs, providing a $TLB$ estimate to confidence level $c$ (Alg.~\ref{alg:candidate}, l\ref{eq:tlbeval}). 
If the confidence interval's lower bound is greater than the target $TLB$, the basis is a sufficiently good fit; if its upper bound is less than the target $TLB$, the basis is not a sufficiently good fit. 
If the confidence interval contains the target $TLB$,  \red{ DROP cannot determine if the target $TLB$ is achieved. 
Thus, DROP automatically samples additional pairs to refine its estimate.
%in practice, and especially for our initial target time series datasets, DROP rarely uses more than 500 pairs on average in its $TLB$ estimates (often using far fewer)
}

To estimate the $TLB$ to confidence $c$, DROP uses the Central Limit Theorem: computing the standard deviation of a set of sampled pairs' $TLB$ measures and applying a confidence interval to the sample according to the $c$.
%For low variance data, DROP evaluates a candidate basis with few samples from the dataset \red{as the confidence intervals shrink rapidly}. 

The techniques in this section are presented in the context of $TLB$, but can be applied to any downstream task and metric for which we can compute confidence intervals and are monotonic in number of principal components retained.

\begin{comment}
\red{For instance, DROP can operate while using all of its optimizations when using any $L^p$-norm.}
\red{Euclidean similarity search} is simply one such domain that is a good fit for PCA: when performing DR via PCA, as we increase the number of principal components, a clear positive correlation exists between the percent of variance explained and the $TLB$ regardless of data spectrum.
We demonstrate this correlation in the experiment below, where we generate three synthetic datasets with predefined spectrum (right), representing varying levels of structure present in real-world datasets. 
The positive correlation is evident (left) despite the fact that the two do not directly correspond ($x=y$ provided as reference). 
This holds true for all of the evaluated real world datasets.

\vspace{.2cm}
\includegraphics[width= .9\linewidth]{figs/tlb-pca.pdf}

For alternative preservation metrics, we can utilize closed-form confidence intervals~\cite{stats-book,ci1,onlineagg}, or bootstrap-based methods~\cite{bootstrap1,bootstrap2}, which incur higher overhead but can be more generally applied.
\end{comment}

\begin{algorithm}
\begin{algorithmic}[1]
\small
\Statex \textbf{Input:}  
\Statex $X$: sampled data matrix
\Statex $B$: target metric preservation level; default $TLB = 0.98$
\Statex  \hrule 
\Function{compute-transform}{$X, X_i B$}: \label{eq:basis}
	\State \textsc{pca.fit}$(X_i)$
			\Comment{fit PCA on the sample}
	\State Initialize: high $= k_{i-1}$; low $=0$; $k_i= \frac{1}{2}$(low + high); $B_i = 0$
	\While{(low $!=$ high)}
		\State $T_{k_i}, B_i  = \textsc{evaluate-tlb}( X, B, k_i)$
		\If{$B_i \leq B$}  low $= k_i + 1$ 
		\Else  \hspace{0pt} high $= k_i $
		\EndIf
		\State $k_i = \frac{1}{2}$(low + high)
	\EndWhile
	\State $T_{k_i} = $ cached $k_i$-dimensional PCA transform\\
	\Return $T_{k_i}$
\EndFunction
\Statex 
\Function{evaluate-tlb}{$X, B, k$}: \label{eq:eval}
	\State numPairs $= \frac{1}{2}\mvar(\mvar-1)$
	\State $p = 100$
		\Comment{number of pairs to check metric preservation}
	\While{($p < $ numPairs)}
		\State $B_i, B_{lo}, B_{hi} = $ \textsc{tlb}($ X, p, k$)
			 \label{eq:paircheck}
		\If{($B_{lo} > B$ or $B_{hi} < B$)}   \textbf{break}
		\Else \hspace{0pt} pairs $\times$= $ 2$
		\EndIf
	\EndWhile
	\\\Return $B_i$	
\EndFunction
\Statex 
\Function{tlb}{$X, p, k$}: \label{eq:tlbeval}
	\State \textbf{return } mean and 95\%-CI of the $TLB$ after transforming $p$ $d$-dimensional pairs of points from $X$ to dimension $k$. The highest transformation computed thus far is cached to avoid recomputation of the transformation matrix.
\EndFunction

\end{algorithmic}
\caption{Basis Evaluation and Search}
\label{alg:candidate}
\end{algorithm}


\subsection{Progress Estimation}
\label{subsec:pest}
%Given a low dimensional $TLB$-achieving transformation from the evaluation step, DROP must identify the dimensionality $k_i$ and runtime ($r_i$) of the transformation that would be obtained from an additional DROP iteration.
%We refer to this as the $progress estimation$ step.

Recall that the goal of workload-aware DR is to minimize $R + \mathcal{C}_\mvar(k)$ such that $TLB(XT_k) \geq B$, with $R$ denoting total DR (i.e., DROP's) runtime, $T_k$ the $k$-dimensional $TLB$-preserving transformation of data $X$ returned by DROP, and $\mathcal{C}_\mvar(k)$ the workload cost function. 
Therefore, given a $k_i$-dimensional transformation $T_{k_i}$ returned by the evaluation step of DROP's $i^{\text{th}}$ iteration, DROP can compute the value of this objective function by substituting its elapsed runtime for $R$ and $T_{k_i}$ for $T_k$.  
We denote the value of the objective at the end of iteration $i$ as $obj_i$. 

To decide whether to continue iterating to find a lower dimensional transform, we show in  \S\ref{subsec:opt} that DROP must estimate $obj_{i+1}$. To do so, DROP must estimate the runtime required for iteration $i+1$ (which we denote as $r_{i+1}$, where $R=\sum_i r_i$ after $i$ iterations) and the dimensionality of the $TLB$-preserving transformation produced by iteration $i+1$, $k_{i+1}$. 
DROP cannot directly measure $r_{i+1}$ or $k_{i+1}$ without performing iteration $i+1$, thus performs online progress estimation. Specifically, DROP performs online parametric fitting to compute future values based on prior values for $r_{i}$ and $k_i$ (Alg.~\ref{alg:DROP}, l\ref{eq:estimate}). 
By default, given a sample of size $m_i$ in iteration $i$, DROP performs linear extrapolation to estimate $k_{i+1}$ and $r_{i+1}$. The estimate of $r_{i+1}$, for instance, is:

\vspace{-.4cm}
\begin{equation*}
\hat{r}_{i+1} = r_i + \frac{r_i - r_{i-1}}{m_i - m_{i-1}} (m_{i+1} -  m_i).
\end{equation*}

\begin{comment}
\red{
DROP's use of a basic first-order approximation is motivated by the fact that when adding a small number of data samples each iteration, both runtime and resulting lower dimension do not change drastically (i.e., see Fig.~\ref{fig:progressive} after a feasible point is achieved). 
While linear extrapolation acts as a proof-of-concept for progress estimation, the architecture can incorporate more sophisticated functions as needed (\S\ref{sec:relwork}).
}
\end{comment}

\subsection{Cost-Based Optimization}
\label{subsec:opt}

DROP must determine if continued PCA on additional samples will improve overall runtime. 
%We refer to this as the $cost-based optimization$ step. 
Given predictions of the next iteration's runtime ($\hat{r}_{i+1}$) and dimensionality ($\hat{k}_{i+1}$), DROP uses a greedy heuristic to estimate the optimal stopping point.
If the estimated objective value is greater than its current value ($obj_i < \widehat{obj}_{i+1}$), DROP will terminate. 
If DROP's runtime is convex in the number of iterations, we can prove that this condition is the optimal stopping criterion via convexity of composition of convex functions. 
This stopping criterion leads to the following check at each iteration (Alg.\ref{alg:DROP}, l\ref{eq:optimize}): 

\vspace{-.4cm}
\begin{align}
  obj_i &< \widehat{obj}_{i+1} \nonumber \\
  \mathcal{C}_\mvar(k_i) + \sum_{j=0}^i r_j &< \mathcal{C}_\mvar(\hat{k}_{i+1}) + \sum_{j=0}^{i} r_j + \hat{r}_{i+1} \nonumber \\
  % \mathcal{C}_\mvar(k_i)  &< \mathcal{C}_\mvar(\hat{k}_{i+1}) + \hat{r}_{i+1}  \nonumber \\
  \mathcal{C}_\mvar(k_i) - \mathcal{C}_\mvar(\hat{k}_{i+1}) &< \hat{r}_{i+1}  \label{eq:check}
\end{align}

DROP terminates when the projected time of the next iteration exceeds the estimated downstream runtime benefit. 
%Absent $\mathcal{C}_d$, we default to execution until convergence (i.e, $k$ plateaus), and show the cost of doing so in \S\ref{sec:experiments}.


\begin{comment}
\red{In the general case as the rate of decrease in dimension ($k_i$) is data dependent, thus convexity is not guaranteed. 
Should $k_i$ plateau before continued decrease, DROP will terminate prematurely. 
This occurs during DROP's first iterations if sufficient data to meet the $TLB$ threshold at a dimension lower than $\dvar$ has not been sampled (SmallKitchenAppliances in Fig.~\ref{fig:progressive}).
Thus, optimization is only enabled once a feasible point is attained, as we prioritize accuracy over runtime (i.e., $0.3$ for SmallKitchenAppliances).
We show the implications of this decision in DROP in \S\ref{subsec:arch}.%, and in the streaming setting in the extended manuscript.
}
\end{comment}

\subsection{Choice of PCA Subroutine}
\label{subsec:pcaroutine}

The most straightforward means of implementing PCA via SVD in DROP is computationally inefficient compared to DR alternatives (\S\ref{sec:background}).  
DROP computes PCA via a randomized SVD algorithm from~\cite{tropp} (SVD-Halko).
Alternative efficient methods for PCA exist (i.e., PPCA, which we also provide), but we found that SVD-Halko is asymptotically of the same running time as techniques used in practice, is straightforward to implement, is $2.5-28\times$ faster than our baseline implementations of SVD-based PCA, PPCA, and Oja's method, and does not require hyperparameter tuning for batch size, learning rate, or convergence criteria.  
%While SVD-Halko is not as efficient as other techniques with respect to communication complexity as in~\cite{ppca-sigmod}, or convergence rate as in~\cite{re-new}, these techniques can be easily substituted for SVD-Halko in DROP's architecture.
%%%%We demonstrate this by implementing multiple alternatives in \S\ref{subsec:pcaexp}.
%%%%\red{Further, we also demonstrate that this implementation is competitive with the widely used SciPy Python library~\cite{scipy}}.

\begin{comment}
\begin{algorithm}[t]
\begin{algorithmic}
\State \textbf{Input:}  \\
$H$: concatenation of previous transformation matrices \\
$T$: new sample's transformation \\
 points to sample per iteration; default 5\% \\
 
\\ \hrule

\Function{distill}{$H, T$}:
	\State $H \gets [H | T]$
		\Comment{Horizontal concatenation to update history}
	\State $U, \Sigma, V^\intercal \gets \textsc{SVD}(H)$ 
				\Comment{$U$ is a basis for the range of $T$}
	\State $T \gets U[:,\textsc{num-columns(T)}]$
	\\\Return{$T$}
\EndFunction
\end{algorithmic}
\caption{Work Reuse}
\label{alg:reuse}
\end{algorithm} 
\end{comment}

\subsection{Work Reuse}
\label{subsec:reuse}

A natural question arises due to DROP's iterative architecture: can we combine information across each sample's transformations without computing PCA over the union of the data samples? 
Stochastic PCA methods enable work reuse across samples as they iteratively refine a single transformation matrix, but other methods do not.
%We propose an algorithm that allows reuse of previous work when utilizing arbitrary PCA routines with DROP.
DROP uses two insights to enable work reuse over any PCA routine.

First, given PCA transformation matrices $T_1$ and $T_2$, their horizontal concatenation $H = [T_1 | T_2]$ is a transformation into the union of their range spaces.
Second, principal components returned from running PCA on repeated data samples generally concentrate to the true top principal components for datasets with rapid spectrum drop off.
Work reuse thus proceeds as follows:
DROP maintains a transformation history consisting of the horizontal concatenation of all transformations to this point, computes the SVD of this matrix, and returns the first $k$ columns as the transformation matrix. 

Although this requires an SVD computation, computational overhead is dependent on the size of the history matrix, not the dataset size.
This size is proportional to the original dimensionality $\dvar$ and size of lower dimensional transformations, which are in turn proportional to the data's intrinsic dimensionality and the $TLB$ constraint.
As preserving \emph{all history} can be expensive in practice, 
DROP periodically shrinks the history matrix using DR via PCA. 
We validate the benefit of using work reuse---up to \red{15\%} on real-world data---in \S\ref{sec:experiments}.



\section{Experiments}
\label{sec:exp}
In this section we conduct comprehensive experiments to emphasise the effectiveness of DIAL, including evaluations under white-box and black-box settings, robustness to unforeseen adversaries, robustness to unforeseen corruptions, transfer learning, and ablation studies. Finally, we present a new measurement to test the balance between robustness and natural accuracy, which we named $F_1$-robust score. 


\subsection{A case study on SVHN and CIFAR-100}
In the first part of our analysis, we conduct a case study experiment on two benchmark datasets: SVHN \citep{netzer2011reading} and CIFAR-100 \cite{krizhevsky2009learning}. We follow common experiment settings as in \cite{rice2020overfitting, wu2020adversarial}. We used the PreAct ResNet-18 \citep{he2016identity} architecture on which we integrate a domain classification layer. The adversarial training is done using 10-step PGD adversary with perturbation size of 0.031 and a step size of 0.003 for SVHN and 0.007 for CIFAR-100. The batch size is 128, weight decay is $7e^{-4}$ and the model is trained for 100 epochs. For SVHN, the initial learinnig rate is set to 0.01 and decays by a factor of 10 after 55, 75 and 90 iteration. For CIFAR-100, the initial learning rate is set to 0.1 and decays by a factor of 10 after 75 and 90 iterations. 
%We compared DIAL to \cite{madry2017towards} and TRADES \citep{zhang2019theoretically}. 
%The evaluation is done using Auto-Attack~\citep{croce2020reliable}, which is an ensemble of three white-box and one black-box parameter-free attacks, and various $\ell_{\infty}$ adversaries: PGD$^{20}$, PGD$^{100}$, PGD$^{1000}$ and CW$_{\infty}$ with step size of 0.003. 
Results are averaged over 3 restarts while omitting one standard deviation (which is smaller than 0.2\% in all experiments). As can be seen by the results in Tables~\ref{black-and_white-svhn} and \ref{black-and_white-cifar100}, DIAL presents consistent improvement in robustness (e.g., 5.75\% improved robustness on SVHN against AA) compared to the standard AT 
%under variety of attacks 
while also improving the natural accuracy. More results are presented in Appendix \ref{cifar100-svhn-appendix}.


\begin{table}[!ht]
  \caption{Robustness against white-box, black-box attacks and Auto-Attack (AA) on SVHN. Black-box attacks are generated using naturally trained surrogate model. Natural represents the naturally trained (non-adversarial) model.
  %and applied to the best performing robust models.
  }
  \vskip 0.1in
  \label{black-and_white-svhn}
  \centering
  \small
  \begin{tabular}{l@{\hspace{1\tabcolsep}}c@{\hspace{1\tabcolsep}}c@{\hspace{1\tabcolsep}}c@{\hspace{1\tabcolsep}}c@{\hspace{1\tabcolsep}}c@{\hspace{1\tabcolsep}}c@{\hspace{1\tabcolsep}}c@{\hspace{1\tabcolsep}}c@{\hspace{1\tabcolsep}}c@{\hspace{1\tabcolsep}}c}
    \toprule
    & & \multicolumn{4}{c}{White-box} & \multicolumn{4}{c}{Black-Box}  \\
    \cmidrule(r){3-6} 
    \cmidrule(r){7-10}
    Defense Model & Natural & PGD$^{20}$ & PGD$^{100}$  & PGD$^{1000}$  & CW$^{\infty}$ & PGD$^{20}$ & PGD$^{100}$ & PGD$^{1000}$  & CW$^{\infty}$ & AA \\
    \midrule
    NATURAL & 96.85 & 0 & 0 & 0 & 0 & 0 & 0 & 0 & 0 & 0 \\
    \midrule
    AT & 89.90 & 53.23 & 49.45 & 49.23 & 48.25 & 86.44 & 86.28 & 86.18 & 86.42 & 45.25 \\
    % TRADES & 90.35 & 57.10 & 54.13 & 54.08 & 52.19 & 86.89 & 86.73 & 86.57 & 86.70 &  49.50 \\
    $\DIAL_{\kl}$ (Ours) & 90.66 & \textbf{58.91} & \textbf{55.30} & \textbf{55.11} & \textbf{53.67} & 87.62 & 87.52 & 87.41 & 87.63 & \textbf{51.00} \\
    $\DIAL_{\ce}$ (Ours) & \textbf{92.88} & 55.26  & 50.82 & 50.54 & 49.66 & \textbf{89.12} & \textbf{89.01} & \textbf{88.74} & \textbf{89.10} &  46.52  \\
    \bottomrule
  \end{tabular}
\end{table}


\begin{table}[!ht]
  \caption{Robustness against white-box, black-box attacks and Auto-Attack (AA) on CIFAR100. Black-box attacks are generated using naturally trained surrogate model. Natural represents the naturally trained (non-adversarial) model.
  %and applied to the best performing robust models.
  }
  \vskip 0.1in
  \label{black-and_white-cifar100}
  \centering
  \small
  \begin{tabular}{l@{\hspace{1\tabcolsep}}c@{\hspace{1\tabcolsep}}c@{\hspace{1\tabcolsep}}c@{\hspace{1\tabcolsep}}c@{\hspace{1\tabcolsep}}c@{\hspace{1\tabcolsep}}c@{\hspace{1\tabcolsep}}c@{\hspace{1\tabcolsep}}c@{\hspace{1\tabcolsep}}c@{\hspace{1\tabcolsep}}c}
    \toprule
    & & \multicolumn{4}{c}{White-box} & \multicolumn{4}{c}{Black-Box}  \\
    \cmidrule(r){3-6} 
    \cmidrule(r){7-10}
    Defense Model & Natural & PGD$^{20}$ & PGD$^{100}$  & PGD$^{1000}$  & CW$^{\infty}$ & PGD$^{20}$ & PGD$^{100}$ & PGD$^{1000}$  & CW$^{\infty}$ & AA \\
    \midrule
    NATURAL & 79.30 & 0 & 0 & 0 & 0 & 0 & 0 & 0 & 0 & 0 \\
    \midrule
    AT & 56.73 & 29.57 & 28.45 & 28.39 & 26.6 & 55.52 & 55.29 & 55.26 & 55.40 & 24.12 \\
    % TRADES & 58.24 & 30.10 & 29.66 & 29.64 & 25.97 & 57.05 & 56.71 & 56.67 & 56.77 & 24.92 \\
    $\DIAL_{\kl}$ (Ours) & 58.47 & \textbf{31.19} & \textbf{30.50} & \textbf{30.42} & \textbf{26.91} & 57.16 & 56.81 & 56.80 & 57.00 & \textbf{25.87} \\
    $\DIAL_{\ce}$ (Ours) & \textbf{60.77} & 27.87 & 26.66 & 26.61 & 25.98 & \textbf{59.48} & \textbf{59.06} & \textbf{58.96} & \textbf{59.20} & 23.51  \\
    \bottomrule
  \end{tabular}
\end{table}


% \begin{table}[!ht]
%   \caption{Robustness comparison of DIAL to Madry et al. and TRADES defense models on the SVHN dataset under different PGD white-box attacks and the ensemble Auto-Attack (AA).}
%   \label{svhn}
%   \centering
%   \begin{tabular}{llllll|l}
%     \toprule
%     \cmidrule(r){1-5}
%     Defense Model & Natural & PGD$^{20}$ & PGD$^{100}$ & PGD$^{1000}$ & CW$_{\infty}$ & AA\\
%     \midrule
%     $\DIAL_{\kl}$ (Ours) & $\mathbf{90.66}$ & $\mathbf{58.91}$ & $\mathbf{55.30}$ & $\mathbf{55.12}$ & $\mathbf{53.67}$  & $\mathbf{51.00}$  \\
%     Madry et al. & 89.90 & 53.23 & 49.45 & 49.23 & 48.25 & 45.25  \\
%     TRADES & 90.35 & 57.10 & 54.13 & 54.08 & 52.19 & 49.50 \\
%     \bottomrule
%   \end{tabular}
% \end{table}


\subsection{Performance comparison on CIFAR-10} \label{defence-settings}
In this part, we evaluate the performance of DIAL compared to other well-known methods on CIFAR-10. 
%To be consistent with other methods, 
We follow the same experiment setups as in~\cite{madry2017towards, wang2019improving, zhang2019theoretically}. When experiment settings are not identical between tested methods, we choose the most commonly used settings, and apply it to all experiments. This way, we keep the comparison as fair as possible and avoid reporting changes in results which are caused by inconsistent experiment settings \citep{pang2020bag}. To show that our results are not caused because of what is referred to as \textit{obfuscated gradients}~\citep{athalye2018obfuscated}, we evaluate our method with same setup as in our defense model, under strong attacks (e.g., PGD$^{1000}$) in both white-box, black-box settings, Auto-Attack ~\citep{croce2020reliable}, unforeseen "natural" corruptions~\citep{hendrycks2018benchmarking}, and unforeseen adversaries. To make sure that the reported improvements are not caused by \textit{adversarial overfitting}~\citep{rice2020overfitting}, we report best robust results for each method on average of 3 restarts, while omitting one standard deviation (which is smaller than 0.2\% in all experiments). Additional results for CIFAR-10 as well as comprehensive evaluation on MNIST can be found in Appendix \ref{mnist-results} and \ref{additional_res}.
%To further keep the comparison consistent, we followed the same attack settings for all methods.


\begin{table}[ht]
  \caption{Robustness against white-box, black-box attacks and Auto-Attack (AA) on CIFAR-10. Black-box attacks are generated using naturally trained surrogate model. Natural represents the naturally trained (non-adversarial) model.
  %and applied to the best performing robust models.
  }
  \vskip 0.1in
  \label{black-and_white-cifar}
  \centering
  \small
  \begin{tabular}{cccccccc@{\hspace{1\tabcolsep}}c}
    \toprule
    & & \multicolumn{3}{c}{White-box} & \multicolumn{3}{c}{Black-Box} \\
    \cmidrule(r){3-5} 
    \cmidrule(r){6-8}
    Defense Model & Natural & PGD$^{20}$ & PGD$^{100}$ & CW$^{\infty}$ & PGD$^{20}$ & PGD$^{100}$ & CW$^{\infty}$ & AA \\
    \midrule
    NATURAL & 95.43 & 0 & 0 & 0 & 0 & 0 & 0 &  0 \\
    \midrule
    TRADES & 84.92 & 56.60 & 55.56 & 54.20 & 84.08 & 83.89 & 83.91 &  53.08 \\
    MART & 83.62 & 58.12 & 56.48 & 53.09 & 82.82 & 82.52 & 82.80 & 51.10 \\
    AT & 85.10 & 56.28 & 54.46 & 53.99 & 84.22 & 84.14 & 83.92 & 51.52 \\
    ATDA & 76.91 & 43.27 & 41.13 & 41.01 & 75.59 & 75.37 & 75.35 & 40.08\\
    $\DIAL_{\kl}$ (Ours) & 85.25 & $\mathbf{58.43}$ & $\mathbf{56.80}$ & $\mathbf{55.00}$ & 84.30 & 84.18 & 84.05 & \textbf{53.75} \\
    $\DIAL_{\ce}$ (Ours)  & $\mathbf{89.59}$ & 54.31 & 51.67 & 52.04 &$ \mathbf{88.60}$ & $\mathbf{88.39}$ & $\mathbf{88.44}$ & 49.85 \\
    \midrule
    $\DIAL_{\awp}$ (Ours) & $\mathbf{85.91}$ & $\mathbf{61.10}$ & $\mathbf{59.86}$ & $\mathbf{57.67}$ & $\mathbf{85.13}$ & $\mathbf{84.93}$ & $\mathbf{85.03}$  & \textbf{56.78} \\
    $\TRADES_{\awp}$ & 85.36 & 59.27 & 59.12 & 57.07 & 84.58 & 84.58 & 84.59 & 56.17 \\
    \bottomrule
  \end{tabular}
\end{table}



\paragraph{CIFAR-10 setup.} We use the wide residual network (WRN-34-10)~\citep{zagoruyko2016wide} architecture. %used in the experiments of~\cite{madry2017towards, wang2019improving, zhang2019theoretically}. 
Sidelong this architecture, we integrate a domain classification layer. To generate the adversarial domain dataset, we use a perturbation size of $\epsilon=0.031$. We apply 10 of inner maximization iterations with perturbation step size of 0.007. Batch size is set to 128, weight decay is set to $7e^{-4}$, and the model is trained for 100 epochs. Similar to the other methods, the initial learning rate was set to 0.1, and decays by a factor of 10 at iterations 75 and 90. 
%For being consistent with other methods, the natural images are padded with 4-pixel padding with 32-random crop and random horizontal flip. Furthermore, all methods are trained using SGD with momentum 0.9. For $\DIAL_{\kl}$, we balance the robust loss with $\lambda=6$ and the domains loss with $r=4$. For $\DIAL_{\ce}$, we balance the robust loss with $\lambda=1$ and the domains loss with $r=2$. 
%We also introduce a version of our method that incorporates the AWP double-perturbation mechanism, named DIAL-AWP.
%which is trained using the same learning rate schedule used in ~\cite{wu2020adversarial}, where the initial 0.1 learning rate decays by a factor of 10 after 100 and 150 iterations. 
See Appendix \ref{cifar10-additional-setup} for additional details.

\begin{table}[ht]
  \caption{Black-box attack using the adversarially trained surrogate models on CIFAR-10.}
  \vskip 0.1in
  \label{black-box-cifar-adv}
  \centering
  \small
  \begin{tabular}{ll|c}
    \toprule
    \cmidrule(r){1-2}
    Surrogate (source) model & Target model & robustness \% \\
    % \midrule
    \midrule
    TRADES & $\DIAL_{\ce}$ & $\mathbf{67.77}$ \\
    $\DIAL_{\ce}$ & TRADES & 65.75 \\
    \midrule
    MART & $\DIAL_{\ce}$ & $\mathbf{70.30}$ \\
    $\DIAL_{\ce}$ & MART & 64.91 \\
    \midrule
    AT & $\DIAL_{\ce}$ & $\mathbf{65.32}$ \\
    $\DIAL_{\ce}$ & AT  & 63.54 \\
    \midrule
    ATDA & $\DIAL_{\ce}$ & $\mathbf{66.77}$ \\
    $\DIAL_{\ce}$ & ATDA & 52.56 \\
    \bottomrule
  \end{tabular}
\end{table}

\paragraph{White-box/Black-box robustness.} 
%We evaluate all defense models using Auto-Attack, PGD$^{20}$, PGD$^{100}$, PGD$^{1000}$ and CW$_{\infty}$ with step size 0.003. We constrain all attacks by the same perturbation $\epsilon=0.031$. 
As reported in Table~\ref{black-and_white-cifar} and Appendix~\ref{additional_res}, our method achieves better robustness compared to the other methods. Specifically, in the white-box settings, our method improves robustness over~\citet{madry2017towards} and TRADES by 2\% 
%using the common PGD$^{20}$ attack 
while keeping higher natural accuracy. We also observe better natural accuracy of 1.65\% over MART while also achieving better robustness over all attacks. Moreover, our method presents significant improvement of up to 15\% compared to the the domain invariant method suggested by~\citet{song2018improving} (ATDA).
%in both natural and robust accuracy. 
When incorporating 
%the double-perturbation mechanism of 
AWP, our method improves the results of $\TRADES_{\awp}$ by almost 2\%.
%and reaches state-of-the-art results for robust models with no additional data. 
% Additional results are available in Appendix~\ref{additional_res}.
When tested on black-box settings, $\DIAL_{\ce}$ presents a significant improvement of more than 4.4\% over the second-best performing method, and up to 13\%. In Table~\ref{black-box-cifar-adv}, we also present the black-box results when the source model is taken from one of the adversarially trained models. %Then, we compare our model to each one of them both as the source model and target model. 
In addition to the improvement in black-box robustness, $\DIAL_{\ce}$ also manages to achieve better clean accuracy of more than 4.5\% over the second-best performing method.
% Moreover, based on the auto-attack leader-board \footnote{\url{https://github.com/fra31/auto-attack}}, our method achieves the 1st place among models without additional data using the WRN-34-10 architecture.

% \begin{table}
%   \caption{White-box robustness on CIFAR-10 using WRN-34-10}
%   \label{white-box-cifar-10}
%   \centering
%   \begin{tabular}{lllll}
%     \toprule
%     \cmidrule(r){1-2}
%     Defense Model & Natural & PGD$^{20}$ & PGD$^{100}$ & PGD$^{1000}$ \\
%     \midrule
%     TRADES ~\cite{zhang2019theoretically} & 84.92  & 56.6 & 55.56 & 56.43  \\
%     MART ~\cite{wang2019improving} & 83.62  & 58.12 & 56.48 & 56.55  \\
%     Madry et al. ~\cite{madry2017towards} & 85.1  & 56.28 & 54.46 & 54.4  \\
%     Song et al. ~\cite{song2018improving} & 76.91 & 43.27 & 41.13 & 41.02  \\
%     $\DIAL_{\ce}$ (Ours) & $ \mathbf{90}$  & 52.12 & 48.88 & 48.78  \\
%     $\DIAL_{\kl}$ (Ours) & 85.25 & $\mathbf{58.43}$ & $\mathbf{56.8}$ & $\mathbf{56.73}$ \\
%     \midrule
%     $\DIAL_{\kl}$+AWP (Ours) & $\mathbf{85.91}$ & $\mathbf{61.1}$ & - & -  \\
%     TRADES+AWP \cite{wu2020adversarial} & 85.36 & 59.27 & 59.12 & -  \\
%     % MART + AWP & 84.43 & 60.68 & 59.32 & -  \\
%     \bottomrule
%   \end{tabular}
% \end{table}


% \begin{table}
%   \caption{White-box robustness on MNIST}
%   \label{white-box-mnist}
%   \centering
%   \begin{tabular}{llllll}
%     \toprule
%     \cmidrule(r){1-2}
%     Defense Model & Natural & PGD$^{40}$ & PGD$^{100}$ & PGD$^{1000}$ \\
%     \midrule
%     TRADES ~\cite{zhang2019theoretically} & 99.48 & 96.07 & 95.52 & 95.22 \\
%     MART ~\cite{wang2019improving} & 99.38  & 96.99 & 96.11 & 95.74  \\
%     Madry et al. ~\cite{madry2017towards} & 99.41  & 96.01 & 95.49 & 95.36 \\
%     Song et al. ~\cite{song2018improving}  & 98.72 & 96.82 & 96.26 & 96.2  \\
%     $\DIAL_{\kl}$ (Ours) & 99.46 & 97.05 & 96.06 & 95.99  \\
%     $\DIAL_{\ce}$ (Ours) & $\mathbf{99.49}$  & $\mathbf{97.38}$ & $\mathbf{96.45}$ & $\mathbf{96.33}$ \\
%     \bottomrule
%   \end{tabular}
% \end{table}


% \paragraph{Attacking MNIST.} For consistency, we use the same perturbation and step sizes. For MNIST, we use $\epsilon=0.3$ and step size of $0.01$. The natural accuracy of our surrogate (source) model is 99.51\%. Attacks results are reported in Table~\ref{black-and_white-mnist}. It is worth noting that the improvement margin is not conclusive on MNIST as it is on CIFAR-10, which is a more complex task.

% \begin{table}
%   \caption{Black-box robustness on MNIST and CIFAR-10 using naturally trained surrogate model and best performing robust models}
%   \label{black-box-mnist-cifar}
%   \centering
%   \begin{tabular}{lllllll}
%     \toprule
%     & \multicolumn{3}{c}{MNIST} & \multicolumn{3}{c}{CIFAR-10} \\
%     \cmidrule(r){2-4} 
%     \cmidrule(r){5-7}  
%     Defense Model & PGD$^{40}$ & PGD$^{100}$ & PGD$^{1000}$ & PGD$^{20}$ & PGD$^{100}$ & PGD$^{1000}$ \\
%     \midrule
%     TRADES ~\cite{zhang2019theoretically} & 98.12 & 97.86 & 97.81 & 84.08 & 83.89 & 83.8 \\
%     MART ~\cite{wang2019improving} & 98.16 & 97.96 & 97.89  & 82.82 & 82.52 & 82.47 \\
%     Madry et al. ~\cite{madry2017towards}  & 98.05 & 97.73 & 97.78 & 84.22 & 84.14 & 83.96 \\
%     Song et al. ~\cite{song2018improving} & 97.74 & 97.28 & 97.34 & 75.59 & 75.37 & 75.11 \\
%     $\DIAL_{\kl}$ (Ours) & 98.14 & 97.83 & 97.87  & 84.3 & 84.18 & 84.0 \\
%     $\DIAL_{\ce}$ (Ours)  & $\mathbf{98.37}$ & $\mathbf{98.12}$ & $\mathbf{98.05}$  & $\mathbf{89.13}$ & $\mathbf{88.89}$ & $\mathbf{88.78}$ \\
%     \bottomrule
%   \end{tabular}
% \end{table}



% \subsubsection{Ensemble attack} In addition to the white-box and black-box settings, we evaluate our method on the Auto-Attack ~\citep{croce2020reliable} using $\ell_{\infty}$ threat model with perturbation $\epsilon=0.031$. Auto-Attack is an ensemble of parameter-free attacks. It consists of three white-box attacks: APGD-CE which is a step size-free version of PGD on the cross-entropy ~\citep{croce2020reliable}. APGD-DLR which is a step size-free version of PGD on the DLR loss ~\citep{croce2020reliable} and FAB which  minimizes the norm of the adversarial perturbations, and one black-box attack: square attack which is a query-efficient black-box attack ~\citep{andriushchenko2020square}. Results are presented in Table~\ref{auto-attack}. Based on the auto-attack leader-board \footnote{\url{https://github.com/fra31/auto-attack}}, our method achieves the 1st place among models without additional data using the WRN-34-10 architecture.

%Additional results can be found in Appendix ~\ref{additional_res}.

% \begin{table}
%   \caption{Auto-Attack (AA) on CIFAR-10 with perturbation size $\epsilon=0.031$ with $\ell_{\infty}$ threat model}
%   \label{auto-attack}
%   \centering
%   \begin{tabular}{lll}
%     \toprule
%     \cmidrule(r){1-2}
%     Defense Model & AA \\
%     \midrule
%     TRADES ~\cite{zhang2019theoretically} & 53.08  \\
%     MART ~\cite{wang2019improving} & 51.1  \\
%     Madry et al. ~\cite{madry2017towards} & 51.52    \\
%     Song et al.   ~\cite{song2018improving} & 40.18 \\
%     $\DIAL_{\ce}$ (Ours) & 47.33  \\
%     $\DIAL_{\kl}$ (Ours) & $\mathbf{53.75}$ \\
%     \midrule
%     DIAL-AWP (Ours) & $\mathbf{56.78}$ \\
%     TRADES-AWP \cite{wu2020adversarial} & 56.17 \\
%     \bottomrule
%   \end{tabular}
% \end{table}


% \begin{table}[!ht]
%   \caption{Auto-Attack (AA) Robustness (\%) on CIFAR-10 with $\epsilon=0.031$ using an $\ell_{\infty}$ threat model}
%   \label{auto-attack}
%   \centering
%   \begin{tabular}{cccccc|cc}
%     \toprule
%     % \multicolumn{8}{c}{Defence Model}  \\
%     % \cmidrule(r){1-8} 
%     TRADES & MART & Madry & Song & $\DIAL_{\ce}$ & $\DIAL_{\kl}$ & DIAL-AWP  & TRADES-AWP\\
%     \midrule
%     53.08 & 51.10 & 51.52 &  40.08 & 47.33  & $\mathbf{53.75}$ & $\mathbf{56.78}$ & 56.17 \\

%     \bottomrule
%   \end{tabular}
% \end{table}

% \begin{table}[!ht]
% \caption{$F_1$-robust measurement using PGD$^{20}$ attack in white-box and black-box settings on CIFAR-10}
%   \label{f1-robust}
%   \centering
%   \begin{tabular}{ccccccc|cc}
%     \toprule
%     % \multicolumn{8}{c}{Defence Model}  \\
%     % \cmidrule(r){1-8} 
%     Defense Model & TRADES & MART & Madry & Song & $\DIAL_{\kl}$ & $\DIAL_{\ce}$ & DIAL-AWP  & TRADES-AWP\\
%     \midrule
%     White-box & 0.659 & 0.666 & 0.657 & 0.518 & $\mathbf{0.675}$  & 0.643 & $\mathbf{0.698}$ & 0.682 \\
%     Black-box & 0.844 & 0.831 & 0.846 & 0.761 & 0.847 & $\mathbf{0.895}$ & $\mathbf{0.854}$ &  0.849 \\
%     \bottomrule
%   \end{tabular}
% \end{table}

\subsubsection{Robustness to Unforeseen Attacks and Corruptions}
\paragraph{Unforeseen Adversaries.} To further demonstrate the effectiveness of our approach, we test our method against various adversaries that were not used during the training process. We attack the model under the white-box settings with $\ell_{2}$-PGD, $\ell_{1}$-PGD, $\ell_{\infty}$-DeepFool and $\ell_{2}$-DeepFool \citep{moosavi2016deepfool} adversaries using Foolbox \citep{rauber2017foolbox}. We applied commonly used attack budget 
%(perturbation for PGD adversaries and overshot for DeepFool adversaries) 
with 20 and 50 iterations for PGD and DeepFool, respectively.
Results are presented in Table \ref{unseen-attacks}. As can be seen, our approach  gains an improvement of up to 4.73\% over the second best method under the various attack types and an average improvement of 3.7\% over all threat models.


\begin{table}[ht]
  \caption{Robustness on CIFAR-10 against unseen adversaries under white-box settings.}
  \vskip 0.1in
  \label{unseen-attacks}
  \centering
%   \small
  \begin{tabular}{c@{\hspace{1.5\tabcolsep}}c@{\hspace{1.5\tabcolsep}}c@{\hspace{1.5\tabcolsep}}c@{\hspace{1.5\tabcolsep}}c@{\hspace{1.5\tabcolsep}}c@{\hspace{1.5\tabcolsep}}c@{\hspace{1.5\tabcolsep}}c}
    \toprule
    Threat Model & Attack Constraints & $\DIAL_{\kl}$ & $\DIAL_{\ce}$ & AT & TRADES & MART & ATDA \\
    \midrule
    \multirow{2}{*}{$\ell_{2}$-PGD} & $\epsilon=0.5$ & 76.05 & \textbf{80.51} & 76.82 & 76.57 & 75.07 & 66.25 \\
    & $\epsilon=0.25$ & 80.98 & \textbf{85.38} & 81.41 & 81.10 & 80.04 & 71.87 \\\midrule
    \multirow{2}{*}{$\ell_{1}$-PGD} & $\epsilon=12$ & 74.84 & \textbf{80.00} & 76.17 & 75.52 & 75.95 & 65.76 \\
    & $\epsilon=7.84$ & 78.69 & \textbf{83.62} & 79.86 & 79.16 & 78.55 & 69.97 \\
    \midrule
    $\ell_{2}$-DeepFool & overshoot=0.02 & 84.53 & \textbf{88.88} & 84.15 & 84.23 & 82.96 & 76.08 \\\midrule
    $\ell_{\infty}$-DeepFool & overshoot=0.02 & 68.43 & \textbf{69.50} & 67.29 & 67.60 & 66.40 & 57.35 \\
    \bottomrule
  \end{tabular}
\end{table}


%%%%%%%%%%%%%%%%%%%%%%%%% conference version %%%%%%%%%%%%%%%%%%%%%%%%%%%%%%%%%%%%%
\paragraph{Unforeseen Corruptions.}
We further demonstrate that our method consistently holds against unforeseen ``natural'' corruptions, consists of 18 unforeseen diverse corruption types proposed by \citet{hendrycks2018benchmarking} on CIFAR-10, which we refer to as CIFAR10-C. The CIFAR10-C benchmark covers noise, blur, weather, and digital categories. As can be shown in Figure \ref{corruption}, our method gains a significant and consistent improvement over all the other methods. Our method leads to an average improvement of 4.7\% with minimum improvement of 3.5\% and maximum improvement of 5.9\% compared to the second best method over all unforeseen attacks. See Appendix \ref{corruptions-apendix} for the full experiment results.


\begin{figure}[h]
 \centering
  \includegraphics[width=0.4\textwidth]{figures/spider_full.png}
%   \caption{Summary of accuracy over all unforeseen corruptions compared to the second and third best performing methods.}
  \caption{Accuracy comparison over all unforeseen corruptions.}
  \label{corruption}
\end{figure}


%%%%%%%%%%%%%%%%%%%%%%%%% conference version %%%%%%%%%%%%%%%%%%%%%%%%%%%%%%%%%%%%%

%%%%%%%%%%%%%%%%%%%%%%%%% Arxiv version %%%%%%%%%%%%%%%%%%%%%%%%%%%%%%%%%%%%%
% \newpage
% \paragraph{Unforeseen Corruptions.}
% We further demonstrate that our method consistently holds against unforeseen "natural" corruptions, consists of 18 unforeseen diverse corruption types proposed by \cite{hendrycks2018benchmarking} on CIFAR-10, which we refer to as CIFAR10-C. The CIFAR10-C benchmark covers noise, blur, weather, and digital categories. As can be shown in Figure  \ref{spider-full-graph}, our method gains a significant and consistent improvement over all the other methods. Our approach leads to an average improvement of 4.7\% with minimum improvement of 3.5\% and maximum improvement of 5.9\% compared to the second best method over all unforeseen attacks. Full accuracy results against unforeseen corruptions are presented in Tables \ref{corruption-table1} and \ref{corruption-table2}. 

% \begin{table}[!ht]
%   \caption{Accuracy (\%) against unforeseen corruptions.}
%   \label{corruption-table1}
%   \centering
%   \tiny
%   \begin{tabular}{lcccccccccccccccccc}
%     \toprule
%     Defense Model & brightness & defocus blur & fog & glass blur & jpeg compression & motion blur & saturate & snow & speckle noise  \\
%     \midrule
%     TRADES & 82.63 & 80.04 & 60.19 & 78.00 & 82.81 & 76.49 & 81.53 & 80.68 & 80.14 \\
%     MART & 80.76 & 78.62 & 56.78 & 76.60 & 81.26 & 74.58 & 80.74 & 78.22 & 79.42 \\
%     AT &  83.30 & 80.42 & 60.22 & 77.90 & 82.73 & 76.64 & 82.31 & 80.37 & 80.74 \\
%     ATDA & 72.67 & 69.36 & 45.52 & 64.88 & 73.22 & 63.47 & 72.07 & 68.76 & 72.27 \\
%     DIAL (Ours)  & \textbf{87.14} & \textbf{84.84} & \textbf{66.08} & \textbf{81.82} & \textbf{87.07} & \textbf{81.20} & \textbf{86.45} & \textbf{84.18} & \textbf{84.94} \\
%     \bottomrule
%   \end{tabular}
% \end{table}


% \begin{table}[!ht]
%   \caption{Accuracy (\%) against unforeseen corruptions.}
%   \label{corruption-table2}
%   \centering
%   \tiny
%   \begin{tabular}{lcccccccccccccccccc}
%     \toprule
%     Defense Model & contrast & elastic transform & frost & gaussian noise & impulse noise & pixelate & shot noise & spatter & zoom blur \\
%     \midrule
%     TRADES & 43.11 & 79.11 & 76.45 & 79.21 & 73.72 & 82.73 & 80.42 & 80.72 & 78.97 \\
%     MART & 41.22 & 77.77 & 73.07 & 78.30 & 74.97 & 81.31 & 79.53 & 79.28 & 77.8 \\
%     AT & 43.30 & 79.58 & 77.53 & 79.47 & 73.76 & 82.78 & 80.86 & 80.49 & 79.58 \\
%     ATDA & 36.06 & 67.06 & 62.56 & 70.33 & 64.63 & 73.46 & 72.28 & 70.50 & 67.31 \\
%     DIAL (Ours) & \textbf{48.84} & \textbf{84.13} & \textbf{81.76} & \textbf{83.76} & \textbf{78.26} & \textbf{87.24} & \textbf{85.13} & \textbf{84.84} & \textbf{83.93}  \\
%     \bottomrule
%   \end{tabular}
% \end{table}


% \begin{figure}[!ht]
%   \centering
%   \includegraphics[width=9cm]{figures/spider_full.png}
%   \caption{Accuracy comparison with all tested methods over unforeseen corruptions.}
%   \label{spider-full-graph}
% \end{figure}
% %%%%%%%%%%%%%%%%%%%%%%%%% Arxiv version %%%%%%%%%%%%%%%%%%%%%%%%%%%%%%%%%%%%%
%%%%%%%%%%%%%%%%%%%%%%%%% Arxiv version %%%%%%%%%%%%%%%%%%%%%%%%%%%%%%%%%%%%%

\subsubsection{Transfer Learning}
Recent works \citep{salman2020adversarially,utrera2020adversarially} suggested that robust models transfer better on standard downstream classification tasks. In Table \ref{transfer-res} we demonstrate the advantage of our method when applied for transfer learning across CIFAR10 and CIFAR100 using the common linear evaluation protocol. see Appendix \ref{transfer-learning-settings} for detailed settings.

\begin{table}[H]
  \caption{Transfer learning results comparison.}
  \vskip 0.1in
  \label{transfer-res}
  \centering
  \small
\begin{tabular}{c|c|c|c}
\toprule

\multicolumn{2}{l}{} & \multicolumn{2}{c}{Target} \\
\cmidrule(r){3-4}
Source & Defence Model & CIFAR10 & CIFAR100 \\
\midrule
\multirow{3}{*}{CIFAR10} & DIAL & \multirow{3}{*}{-} & \textbf{28.57} \\
 & AT &  & 26.95  \\
 & TRADES &  & 25.40  \\
 \midrule
\multirow{3}{*}{CIFAR100} & DIAL & \textbf{73.68} & \multirow{3}{*}{-} \\
 & AT & 71.41 & \\
 & TRADES & 71.42 &  \\
%  \midrule
% \multirow{3}{}{SVHN} & DIAL &  &  & \multirow{3}{}{-} \\
%  & Madry et al. &  &  &  \\
%  & TRADES &  &  &  \\ 
\bottomrule
\end{tabular}
\end{table}


\subsubsection{Modularity and Ablation Studies}

We note that the domain classifier is a modular component that can be integrated into existing models for further improvements. Removing the domain head and related loss components from the different DIAL formulations results in some common adversarial training techniques. For $\DIAL_{\kl}$, removing the domain and related loss components results in the formulation of TRADES. For $\DIAL_{\ce}$, removing the domain and related loss components results in the original formulation of the standard adversarial training, and for $\DIAL_{\awp}$ the removal results in $\TRADES_{\awp}$. Therefore, the ablation studies will demonstrate the effectiveness of combining DIAL on top of different adversarial training methods. 

We investigate the contribution of the additional domain head component introduced in our method. Experiment configuration are as in \ref{defence-settings}, and robust accuracy is based on white-box PGD$^{20}$ on CIFAR-10 dataset. We remove the domain head from both $\DIAL_{\kl}$, $\DIAL_{\awp}$, and $\DIAL_{\ce}$ (equivalent to $r=0$) and report the natural and robust accuracy. We perform 3 random restarts and omit one standard deviation from the results. Results are presented in Figure \ref{ablation}. All DIAL variants exhibits stable improvements on both natural accuracy and robust accuracy. $\DIAL_{\ce}$, $\DIAL_{\kl}$, and $\DIAL_{\awp}$ present an improvement of 1.82\%, 0.33\%, and 0.55\% on natural accuracy and an improvement of 2.5\%, 1.87\%, and 0.83\% on robust accuracy, respectively. This evaluation empirically demonstrates the benefits of incorporating DIAL on top of different adversarial training techniques.
% \paragraph{semi-supervised extensions.} Since the domain classifier does not require the class labels, we argue that additional unlabeled data can be leveraged in future work.
%for improved results. 

\begin{figure}[ht]
  \centering
  \includegraphics[width=0.35\textwidth]{figures/ablation_graphs3.png}
  \caption{Ablation studies for $\DIAL_{\kl}$, $\DIAL_{\ce}$, and $\DIAL_{\awp}$ on CIFAR-10. Circle represent the robust-natural accuracy without using DIAL, and square represent the robust-natural accuracy when incorporating DIAL.
  %to further investigate the impact of the domain head and loss on natural and robust accuracy.
  }
  \label{ablation}
\end{figure}

\subsubsection{Visualizing DIAL}
To further illustrate the superiority of our method, we visualize the model outputs from the different methods on both natural and adversarial test data.
% adversarial test data generated using PGD$^{20}$ white-box attack with step size 0.003 and $\epsilon=0.031$ on CIFAR-10. 
Figure~\ref{tsne1} shows the embedding received after applying t-SNE ~\citep{van2008visualizing} with two components on the model output for our method and for TRADES. DIAL seems to preserve strong separation between classes on both natural test data and adversarial test data. Additional illustrations for the other methods are attached in Appendix~\ref{additional_viz}. 

\begin{figure}[h]
\centering
  \subfigure[\textbf{DIAL} on natural logits]{\includegraphics[width=0.21\textwidth]{figures/domain_ce_test.png}}
  \hspace{0.03\textwidth}
  \subfigure[\textbf{DIAL} on adversarial logits]{\includegraphics[width=0.21\textwidth]{figures/domain_ce_adversarial.png}}
  \hspace{0.03\textwidth}
    \subfigure[\textbf{TRADES} on natural logits]{\includegraphics[width=0.21\textwidth]{figures/trades_test.png}}
    \hspace{0.03\textwidth}
    \subfigure[\textbf{TRADES} on adversarial logits]{\includegraphics[width=0.21\textwidth]{figures/trades_adversarial.png}}
  \caption{t-SNE embedding of model output (logits) into two-dimensional space for DIAL and TRADES using the CIFAR-10 natural test data and the corresponding PGD$^{20}$ generated adversarial examples.}
  \label{tsne1}
\end{figure}


% \begin{figure}[ht]
% \centering
%   \begin{subfigure}{4cm}
%     \centering\includegraphics[width=3.3cm]{figures/domain_ce_test.png}
%     \caption{\textbf{DIAL} on nat. examples}
%   \end{subfigure}
%   \begin{subfigure}{4cm}
%     \centering\includegraphics[width=3.3cm]{figures/domain_ce_adversarial.png}
%     \caption{\textbf{DIAL} on adv. examples}
%   \end{subfigure}
  
%   \begin{subfigure}{4cm}
%     \centering\includegraphics[width=3.3cm]{figures/trades_test.png}
%     \caption{\textbf{TRADES} on nat. examples}
%   \end{subfigure}
%   \begin{subfigure}{4cm}
%     \centering\includegraphics[width=3.3cm]{figures/trades_adversarial.png}
%     \caption{\textbf{TRADES} on adv. examples}
%   \end{subfigure}
%   \caption{t-SNE embedding of model output (logits) into two-dimensional space for DIAL and TRADES using the CIFAR-10 natural test data and the corresponding adversarial examples.}
%   \label{tsne1}
% \end{figure}



% \begin{figure}[ht]
% \centering
%   \begin{subfigure}{6cm}
%     \centering\includegraphics[width=5cm]{figures/domain_ce_test.png}
%     \caption{\textbf{DIAL} on nat. examples}
%   \end{subfigure}
%   \begin{subfigure}{6cm}
%     \centering\includegraphics[width=5cm]{figures/domain_ce_adversarial.png}
%     \caption{\textbf{DIAL} on adv. examples}
%   \end{subfigure}
  
%   \begin{subfigure}{6cm}
%     \centering\includegraphics[width=5cm]{figures/trades_test.png}
%     \caption{\textbf{TRADES} on nat. examples}
%   \end{subfigure}
%   \begin{subfigure}{6cm}
%     \centering\includegraphics[width=5cm]{figures/trades_adversarial.png}
%     \caption{\textbf{TRADES} on adv. examples}
%   \end{subfigure}
%   \caption{t-SNE embedding of model output (logits) into two-dimensional space for DIAL and TRADES using the CIFAR-10 natural test data and the corresponding adversarial examples.}
%   \label{tsne1}
% \end{figure}



\subsection{Balanced measurement for robust-natural accuracy}
One of the goals of our method is to better balance between robust and natural accuracy under a given model. For a balanced metric, we adopt the idea of $F_1$-score, which is the harmonic mean between the precision and recall. However, rather than using precision and recall, we measure the $F_1$-score between robustness and natural accuracy,
using a measure we call
%We named it
the
\textbf{$\mathbf{F_1}$-robust} score.
\begin{equation}
F_1\text{-robust} = \dfrac{\text{true\_robust}}
{\text{true\_robust}+\frac{1}{2}
%\cdot
(\text{false\_{robust}}+
\text{false\_natural})},
\end{equation}
where $\text{true\_robust}$ are the adversarial examples that were correctly classified, $\text{false\_{robust}}$ are the adversarial examples that were miss-classified, and $\text{false\_natural}$ are the natural examples that were miss-classified.
%We tested the proposed $F_1$-robust score using PGD$^{20}$ on CIFAR-10 dataset in white-box and black-box settings. 
Results are presented in Table~\ref{f1-robust} and demonstrate that our method achieves the best $F_1$-robust score in both settings, which supports our findings from previous sections.

% \begin{table}[!ht]
%   \caption{$F_1$-robust measurement using PGD$^{20}$ attack in white and black box settings on CIFAR-10}
%   \label{f1-robust}
%   \centering
%   \begin{tabular}{lll}
%     \toprule
%     \cmidrule(r){1-2}
%     Defense Model & White-box & Black-box \\
%     \midrule
%     TRADES & 0.65937  & 0.84435 \\
%     MART & 0.66613  & 0.83153  \\
%     Madry et al. & 0.65755 & 0.84574   \\
%     Song et al. & 0.51823 & 0.76092  \\
%     $\DIAL_{\ce}$ (Ours) & 0.65318   & $\mathbf{0.88806}$  \\
%     $\DIAL_{\kl}$ (Ours) & $\mathbf{0.67479}$ & 0.84702 \\
%     \midrule
%     \midrule
%     DIAL-AWP (Ours) & $\mathbf{0.69753}$  & $\mathbf{0.85406}$  \\
%     TRADES-AWP & 0.68162 & 0.84917 \\
%     \bottomrule
%   \end{tabular}
% \end{table}

\begin{table}[ht]
\small
  \caption{$F_1$-robust measurement using PGD$^{20}$ attack in white and black box settings on CIFAR-10.}
  \vskip 0.1in
  \label{f1-robust}
  \centering
%   \small
  \begin{tabular}{c
  @{\hspace{1.5\tabcolsep}}c @{\hspace{1.5\tabcolsep}}c @{\hspace{1.5\tabcolsep}}c @{\hspace{1.5\tabcolsep}}c
  @{\hspace{1.5\tabcolsep}}c @{\hspace{1.5\tabcolsep}}c @{\hspace{1.5\tabcolsep}}|
  @{\hspace{1.5\tabcolsep}}c
  @{\hspace{1.5\tabcolsep}}c}
    \toprule
    % \cmidrule(r){8-9}
     & TRADES & MART & AT & ATDA & $\DIAL_{\ce}$ & $\DIAL_{\kl}$ & $\DIAL_{\awp}$ & $\TRADES_{\awp}$ \\
    \midrule
    White-box & 0.659 & 0.666 & 0.657 & 0.518 & 0.660 & \textbf{0.675} & \textbf{0.698} & 0.682 \\
    Black-box & 0.844 & 0.831 & 0.845 & 0.761 & \textbf{0.890} & 0.847 & \textbf{0.854} & 0.849 \\ 
    \bottomrule
  \end{tabular}
\end{table}


\section{Related Work}
\label{sec:related}
\section{Related Work}\label{sec:related}
 
The authors in \cite{humphreys2007noncontact} showed that it is possible to extract the PPG signal from the video using a complementary metal-oxide semiconductor camera by illuminating a region of tissue using through external light-emitting diodes at dual-wavelength (760nm and 880nm).  Further, the authors of  \cite{verkruysse2008remote} demonstrated that the PPG signal can be estimated by just using ambient light as a source of illumination along with a simple digital camera.  Further in \cite{poh2011advancements}, the PPG waveform was estimated from the videos recorded using a low-cost webcam. The red, green, and blue channels of the images were decomposed into independent sources using independent component analysis. One of the independent sources was selected to estimate PPG and further calculate HR, and HRV. All these works showed the possibility of extracting PPG signals from the videos and proved the similarity of this signal with the one obtained using a contact device. Further, the authors in \cite{10.1109/CVPR.2013.440} showed that heart rate can be extracted from features from the head as well by capturing the subtle head movements that happen due to blood flow.

%
The authors of \cite{kumar2015distanceppg} proposed a methodology that overcomes a challenge in extracting PPG for people with darker skin tones. The challenge due to slight movement and low lighting conditions during recording a video was also addressed. They implemented the method where PPG signal is extracted from different regions of the face and signal from each region is combined using their weighted average making weights different for different people depending on their skin color. 
%

There are other attempts where authors of \cite{6523142,6909939, 7410772, 7412627} have introduced different methodologies to make algorithms for estimating pulse rate robust to illumination variation and motion of the subjects. The paper \cite{6523142} introduces a chrominance-based method to reduce the effect of motion in estimating pulse rate. The authors of \cite{6909939} used a technique in which face tracking and normalized least square adaptive filtering is used to counter the effects of variations due to illumination and subject movement. 
The paper \cite{7410772} resolves the issue of subject movement by choosing the rectangular ROI's on the face relative to the facial landmarks and facial landmarks are tracked in the video using pose-free facial landmark fitting tracker discussed in \cite{yu2016face} followed by the removal of noise due to illumination to extract noise-free PPG signal for estimating pulse rate. 

Recently, the use of machine learning in the prediction of health parameters have gained attention. The paper \cite{osman2015supervised} used a supervised learning methodology to predict the pulse rate from the videos taken from any off-the-shelf camera. Their model showed the possibility of using machine learning methods to estimate the pulse rate. However, our method outperforms their results when the root mean squared error of the predicted pulse rate is compared. The authors in \cite{hsu2017deep} proposed a deep learning methodology to predict the pulse rate from the facial videos. The researchers trained a convolutional neural network (CNN) on the images generated using Short-Time Fourier Transform (STFT) applied on the R, G, \& B channels from the facial region of interests.
The authors of \cite{osman2015supervised, hsu2017deep} only predicted pulse rate, and we extended our work in predicting variance in the pulse rate measurements as well.

All the related work discussed above utilizes filtering and digital signal processing to extract PPG signals from the video which is further used to estimate the PR and PRV.  %
The method proposed in \cite{kumar2015distanceppg} is person dependent since the weights will be different for people with different skin tone. In contrast, we propose a deep learning model to predict the PR which is independent of the person who is being trained. Thus, the model would work even if there is no prior training model built for that individual and hence, making our model robust. 

%

\section{Conclusion}
\label{sec:conclusion}
% \vspace{-0.5em}
\section{Conclusion}
% \vspace{-0.5em}
Recent advances in multimodal single-cell technology have enabled the simultaneous profiling of the transcriptome alongside other cellular modalities, leading to an increase in the availability of multimodal single-cell data. In this paper, we present \method{}, a multimodal transformer model for single-cell surface protein abundance from gene expression measurements. We combined the data with prior biological interaction knowledge from the STRING database into a richly connected heterogeneous graph and leveraged the transformer architectures to learn an accurate mapping between gene expression and surface protein abundance. Remarkably, \method{} achieves superior and more stable performance than other baselines on both 2021 and 2022 NeurIPS single-cell datasets.

\noindent\textbf{Future Work.}
% Our work is an extension of the model we implemented in the NeurIPS 2022 competition. 
Our framework of multimodal transformers with the cross-modality heterogeneous graph goes far beyond the specific downstream task of modality prediction, and there are lots of potentials to be further explored. Our graph contains three types of nodes. While the cell embeddings are used for predictions, the remaining protein embeddings and gene embeddings may be further interpreted for other tasks. The similarities between proteins may show data-specific protein-protein relationships, while the attention matrix of the gene transformer may help to identify marker genes of each cell type. Additionally, we may achieve gene interaction prediction using the attention mechanism.
% under adequate regulations. 
% We expect \method{} to be capable of much more than just modality prediction. Note that currently, we fuse information from different transformers with message-passing GNNs. 
To extend more on transformers, a potential next step is implementing cross-attention cross-modalities. Ideally, all three types of nodes, namely genes, proteins, and cells, would be jointly modeled using a large transformer that includes specific regulations for each modality. 

% insight of protein and gene embedding (diff task)

% all in one transformer

% \noindent\textbf{Limitations and future work}
% Despite the noticeable performance improvement by utilizing transformers with the cross-modality heterogeneous graph, there are still bottlenecks in the current settings. To begin with, we noticed that the performance variations of all methods are consistently higher in the ``CITE'' dataset compared to the ``GEX2ADT'' dataset. We hypothesized that the increased variability in ``CITE'' was due to both less number of training samples (43k vs. 66k cells) and a significantly more number of testing samples used (28k vs. 1k cells). One straightforward solution to alleviate the high variation issue is to include more training samples, which is not always possible given the training data availability. Nevertheless, publicly available single-cell datasets have been accumulated over the past decades and are still being collected on an ever-increasing scale. Taking advantage of these large-scale atlases is the key to a more stable and well-performing model, as some of the intra-cell variations could be common across different datasets. For example, reference-based methods are commonly used to identify the cell identity of a single cell, or cell-type compositions of a mixture of cells. (other examples for pretrained, e.g., scbert)


%\noindent\textbf{Future work.}
% Our work is an extension of the model we implemented in the NeurIPS 2022 competition. Now our framework of multimodal transformers with the cross-modality heterogeneous graph goes far beyond the specific downstream task of modality prediction, and there are lots of potentials to be further explored. Our graph contains three types of nodes. while the cell embeddings are used for predictions, the remaining protein embeddings and gene embeddings may be further interpreted for other tasks. The similarities between proteins may show data-specific protein-protein relationships, while the attention matrix of the gene transformer may help to identify marker genes of each cell type. Additionally, we may achieve gene interaction prediction using the attention mechanism under adequate regulations. We expect \method{} to be capable of much more than just modality prediction. Note that currently, we fuse information from different transformers with message-passing GNNs. To extend more on transformers, a potential next step is implementing cross-attention cross-modalities. Ideally, all three types of nodes, namely genes, proteins, and cells, would be jointly modeled using a large transformer that includes specific regulations for each modality. The self-attention within each modality would reconstruct the prior interaction network, while the cross-attention between modalities would be supervised by the data observations. Then, The attention matrix will provide insights into all the internal interactions and cross-relationships. With the linearized transformer, this idea would be both practical and versatile.

% \begin{acks}
% This research is supported by the National Science Foundation (NSF) and Johnson \& Johnson.
% \end{acks}


%% Acknowledgments
\begin{acks}                            %% acks environment is optional
                                        %% contents suppressed with 'anonymous'
  %% Commands \grantsponsor{<sponsorID>}{<name>}{<url>} and
  %% \grantnum[<url>]{<sponsorID>}{<number>} should be used to
  %% acknowledge financial support and will be used by metadata
  %% extraction tools.
  % This material is based upon work supported by the
  % \grantsponsor{GS100000001}{National Science
  %   Foundation}{http://dx.doi.org/10.13039/100000001} under Grant
  % No.~\grantnum{GS100000001}{nnnnnnn} and Grant
  % No.~\grantnum{GS100000001}{mmmmmmm}.  Any opinions, findings, and
  % conclusions or recommendations expressed in this material are those
  % of the author and do not necessarily reflect the views of the
  % National Science Foundation.
  We gratefully acknowledge National Science Foundation for supporting
 Umang Mathur (grant NSF CSR 1422798) and Mahesh Viswanathan (NSF CPS 1329991).
\end{acks}

\clearpage

\bibliography{references}

%% Bibliography

\clearpage


%% Appendix
\appendix
\section{Proof of Theorem~\ref{thm:SHBSoundness}}
\label{app:shb-proof}
%!TEX root = main.tex

In this section, we prove Theorem~\ref{thm:SHBSoundness}. We begin
with a couple of technical lemmas.

\begin{lemma}\label{lem:cr_hb}
% For any trace $\tr$, consider a correct reordering $\tr'$ of $\tr$
% that also respects $\hb{\tr}$. Then $\tr'$ respects $\shb{\tr}$, i.e.,
% for any $e,e'$ such that $e \shb{\tr} e'$ and $e' \in \events{\tr'}$,
% we have $e \in \events{\tr'}$ and $e \trord{\tr'} e'$.
Let $\tr$ be a trace and $\tr'$ be a correct reordering of $\tr'$ that
respects $\hb{\tr}$.  For any $e,e'$ such that $e \shb{\tr} e'$, if
$e' \in \events{\tr'}$ and $e'$ is not the last read event of its
thread in $\tr'$, then $e \in \events{\tr'}$ and $e \trord{\tr'} e'$.
\end{lemma}

\begin{proof}
Consider any $e,e'$ such that $e \shb{\tr} e'$, $e' \in \events{\tr'}$
and $e' = \ev{t, op}$ is not the last read event of the thread $t$ in
the trace $\tr'$.  Then it follows from Definition~\ref{def:shb} that
there is a sequence $e = e_0, e_1, \ldots e_n = e'$ such that for
every $i \leq n-1$, $e_i \trord{\tr} e_{i+1}$ and either (a)
$e_i \tho{\tr} e_{i+1}$ or (b) $e_i = \ev{t_i,\rel(\lk)}$, $e_{i+1}
= \ev{t_{i+1},\acq(\lk)}$, or (c) $e_{i+1} \in \reads{\tr}$ and $e_i
= \lw{\tr}(e_{i+1})$.

We will prove by induction on $i$, starting from $i = n$, that
$e_i \in \events{\tr'}$ and $e_i$ is not the last read event of its
thread in $\tr'$. Observe that these properties hold for $e' = e_n$
--- $e_n \in \events{\tr'}$ and $e_n$ is not the last read event of
its thread in $\tr'$. Assume we have established the claim for
$e_{i+1}$. Now there are three cases to consider for $e_i$. If
$e_i \tho{\tr'} e_{i+1}$ then clearly $e_i \in \events{\tr'}$ because
$e_{i+1} \in \events{\tr'}$. Further, if $e_i$ is a read event, then
it is not the last event of its thread because $e_{i+1}$ is after
it. If $e_i = \ev{t_i,\rel(\lk)}$ and $e_{i+1}
= \ev{t_{i+1},\acq(\lk)}$ then $e_i \in \events{\tr'}$ because $\tr'$
respects $\hb{\tr}$. Further $e_i$ is not the last read event because
it is not a read event! The last case to consider is where $e_i
= \lw{\tr}(e_{i+1})$. In this case, by induction hypothesis, we know
that $e_{i+1}$ is not the last read event of its thread, and therefore
by properties of a correct reordering, we have
$e_i \in \events{\tr'}$. Notice that in this case $e_i$ is not a read
event, and so the claim holds. Thus, we have established that $e =
e_0 \in \events{\tr'}$.

Next, we show that for every $i \leq n-1$, $e_i \trord{\tr'}
e_{i+1}$. If $e_i \tho{\tr} e_{i+1}$ or $e_i = \ev{t_i,\rel(\lk)}$ and
$e_{i+1} = \ev{t_{i+1},\acq(\lk)}$ with $e_i \trord{\tr} e_{i+1}$ then
$e_i \trord{\tr'} e_{i+1}$ because $\tr'$ respects $\hb{\tr}$. On the
other hand, if $e_i = \lw{\tr}(e_{i+1})$ then because $\tr'$ is a
correct reordering of $\tr$ and $e_{i+1}$ is not the last read event
of its thread (established in the previous paragraph), we have $e_i
= \lw{\tr}(e_{i+1}) = \lw{\tr'}(e_{i+1})$.  This establishes the fact
that $e = e_0 \trord{\tr'} e_n = e'$, which completes the proof of the
lemma.
\end{proof}

A slightly weaker form of the converse of Lemma~\ref{lem:cr_hb} also
holds.
%
\begin{lemma}\label{lem:shb_cr_hb}
For a trace $\tr$, let $\tr'$ be a trace with
$\events{\tr'} \subseteq \events{\tr}$ such that (a) $\tr'$ is
$\shb{\tr}$ downward closed, i.e., for any $e,e' \in \events{\tr}$ if
$e \shb{\tr} e'$ and $e' \in \events{\tr'}$ then
$e \in \events{\tr'}$, and (b) $\trord{\tr'} = \trord{\tr} \cap
(\events{\tr'} \times \events{\tr'})$. Then $\tr'$ is a correct
reordering of $\tr$ that respects $\hb{\tr}$.  Further,
for \textbf{every} read event $e \in \reads{\tr'}$, we have
$\lw{\tr'}(e) \simeq \lw{\tr}(e)$, i.e., either both $\lw{\tr'}(e)$
and $\lw{\tr}(e)$ are undefined, or they are both defined and equal.
\end{lemma}

\begin{proof}
The trace $\tr'$ in the lemma is such that the events in $\tr'$ are
downward closed with respect to $\shb{\tr}$ and in $\tr'$ they are
ordered in exactly the same way as in $\tr$. The fact that $\tr'$
respects $\hb{\tr}$ simply follows from the fact that
$\hb{\tr} \subseteq \shb{\tr}$ and
$\hb{\tr} \subseteq \trord{\tr}$. So the main goal is to establish
that $\tr'$ is a correct reordering of $\tr$ that preserves the last
writes of \emph{all} read events.

First we show that $\tr'$ respects lock semantics. Suppose $e_1
= \ev{t_1, \acq(\lk)}$ and $e_2 = \ev{t_2, \acq(\lk)}$ are two lock
acquire events for some lock $\lk$ such that $e_1 \trord{\tr} e_2$ and
$\{e_1,e_2\} \subseteq \events{\tr'}$. Let $e_1'$ be the matching
release event for $e_1$ in $\tr$; such an $e_1'$ exists because $\tr$
is a valid trace. Then we have $e_1 \hb{\tr} e_1' \hb{\tr} e_2$, and
so $e_1' \in \events{\tr'}$ and $e_1' \trord{\tr'} e_2$ because $\tr'$
respects $\hb{\tr}$.

Next observe that since $\tho{\tr} \subseteq \hb{\tr}$ and $\tr'$
respects $\hb{\tr}$, we can conclude that $\proj{\tr'}{t}$ is a prefix
of $\proj{\tr}{t}$ for any thread $t$.

Finally, consider any $e' \in \reads{\tr'}$. Suppose $\lw{\tr}(e')$ is
defined. Let $e = \lw{\tr}(e')$. Since $e \shb{\tr} e'$ and $\tr'$ is
downward closed with respect to $\shb{\tr}$, we have
$e \in \events{\tr'}$. Let $e_1 = \lw{\tr'}(e')$. We need to argue
that $e_1 = e$. Suppose (for contradiction) it is not, i.e., $e \neq
e_1$. Then either $e_1 \trord{\tr} e$ or $e' \trord{\tr} e_1$, because
$e = \lw{\tr}(e')$. However, the fact that $e_1 = \lw{\tr'}(e')$
contradicts the fact that $\trord{\tr'} = \trord{\tr} \cap
(\events{\tr'} \times \events{\tr'})$. Conversely, if $\lw{\tr'}(e')$
is defined then let $e = \lw{\tr'}(e')$. Since $\trord{\tr'}
= \trord{\tr} \cap (\events{\tr'} \times \events{\tr'})$, we have
$e \trord{\tr} e'$. Thus, $\lw{\tr}(e')$ is defined. Let $e_1
= \lw{\tr}(e')$. Once again, since $e_1 \shb{\tr} e'$, and $\tr'$ is
downward closed with respect to $\shb{\tr}$, we have
$e_1 \in \tr'$. Just like in the previous direction, we can conclude
that $e = e_1$ because otherwise we violate the fact that
$\trord{\tr'}$ is identical to $\trord{\tr}$ over $\events{\tr'}$.
\end{proof}

We now prove Theorem~\ref{thm:SHBSoundness} below
\begin{reptheorem}{thm:SHBSoundness}
Let $\tr$ be a trace and $e_1 \trord{\tr} e_2$ be conflicting events in $\tr$.
$(e_1, e_2)$ is an $\hb{\tr}$-schedulable race iff 
either $\ltho{\tr}(e_2)$ is undefined, or $e_1 \not\leq^\tr_{\mathsf{SHB}} \ltho{\tr}(e_2)$.
\end{reptheorem}

\begin{proof}
Let us first prove the forward direction.
That is, let $(e_1,e_2)$ be an HB-race such that the
event $e = \ltho{\tr}(e_2)$ is defined and $e_1 \shb{\tr} e$. 
Consider any correct reordering $\tr'$ that contains both
$e_1$ and $e_2$ and respects $\hb{\tr}$. 
First, since $\tr'$ is a correct reordering of $\tr$, we must have
$e \in \events{\tr'}$ and $e \trord{\tr'} e_2$.
Further, since $e_1 \shb{\tr} e$, from Lemma~\ref{lem:cr_hb}, 
$e_1 \trord{\tr'} e$.
Thus, we have that $e_1 \trord{\tr'} e \trord{\tr'} e_1$
for any correct reordering $\tr'$ of $\tr$ that respects $\hb{\tr}$.
This means, $(e_1, e_2)$ cannot be a $\hb{\tr}$-schedulable race.
% Also, since $\tr'$ is a correct reordering
% also respects $\shb{\tr}$. But that would mean that
% $e \in \events{\tr'}$ and $e_1 \trord{\tr'} e \trord{\tr'} e_2$, and
% so $e_1,e_2$ are not consecutive in $\tr'$. Since this holds for any
% correct reordering $\tr'$ of $\tr$ that also respects $\hb{\tr}$, we
% can conclude that $(e_1,e_2)$ is not an $\hb{\tr}$-schedulable race.

We now prove the backward direction.
Consider an HB-race $(e_1,e_2)$ such that 
either $\ltho{\tr}(e_2)$ is undefined, or if it exists, then it satisfies 
$e_1 \not\leq^\tr_{\mathsf{SHB}} \ltho{\tr}(e_2)$.
% satisfying either of the two
% conditions. 
% Notice that in either case this means that there is no
% event $e \in \events{\tr} \setminus \{e_1,e_2\}$ such that
% $e_1 \shb{\tr} e \shb{\tr} e_2$. 
Consider the set $\setreq$ defined as
\[
\setreq = \{e \in \events{\tr} \setminus\{e_1,e_2\}\: |\: 
        e \shb{\tr} e_1 \mbox{ or } 
        e \shb{\tr} \ltho{\tr}(e_2)\}
\]
where we assume that if $\ltho{\tr}(e_2)$ is undefined then no event
$e$ satisfies the condition $e \shb{\tr} \ltho{\tr}(e_2)$.

First we will show that $\setreq$ is downward closed with respect to
$\shb{\tr}$. Consider $e,e'$ such that $e \shb{\tr} e'$ and
$e' \in \setreq$. By definition of $\setreq$, we have
$e' \not\in \{e_1,e_2\}$ and either $e' \shb{\tr} e_1$ or
$e' \shb{\tr} \ltho{\tr}(e_2)$. Observe that if
$e \not\in \{e_1,e_2\}$, then it is clear that $e \in \setreq$ by
definition since $\shb{\tr}$ is transitive. It is easy to see that
$e \neq e_2$ --- this is because since $e' \neq e_2$, and
$\shb{\tr} \subseteq \trord{\tr}$, $e' \stricttrord{\tr} e_2$ and so
$e \stricttrord{\tr} e_2$. So, all we have left to establish is that
$e \neq e_1$. Suppose for contradiction $e = e_1$. Then it must be the
case that $e' \shb{\tr} \ltho{\tr}(e_2)$. Since $e_1 = e \shb{\tr}
e' \shb{\tr} \ltho{\tr}(e_2)$, we have
$e_1 \shb{\tr} \ltho{\tr}(e_2)$, which contradicts our assumption
about $(e_1,e_2)$.

Let us now consider a trace $\tr''$ which consists of the events in
$\setreq$ ordered according to $\trord{\tr}$. That is, $\trord{\tr''}
= \trord{\tr} \cap (\setreq \times \setreq)$. Since $\tr''$ satisfies the
conditions of Lemma~\ref{lem:shb_cr_hb}, we can conclude that $\tr''$
is a correct reordering of $\tr$ that respects $\hb{\tr}$ and preserves the
last-writes of \emph{every} read event present.

Consider the trace $\tr' = \tr''e_1e_2$. First we prove that $\tr'$
respects $\hb{\tr}$. To do that, we first show that for any event
$e\in \events{\tr}$ such that $e \hb{\tr} e_1$ and $e \neq e_1$, or
$e \hb{\tr} e_2$ and $e \neq e_2$, then $e \in \setreq$.  If
$e \hb{\tr} e_1$ then $e \shb{\tr} e_1$ and so $e \in \setreq$. On the
other hand, if $e \hb{\tr} e_2$ (and $e \neq e_2$), since $(e_1,e_2)$
is an HB-race, we must have $e \neq e_1$ and
$e \hb{\tr} \ltho{\tr}(e_2)$. So $e \in \setreq$. Now the fact $\tr'$
respects $\hb{\tr}$ follows from the fact that $\tr''$ respects
$\hb{\tr}$ and the claim just proved.

We now prove that $\tr'$ is a correct reordering. Observe that since
$\tr'$ respects $\hb{\tr}$, $\tr'$ is well formed (lock semantics is
not violated) and preserves thread-wise prefixes ($\forall
t, \proj{\tr'}{t}$ is a prefix of $\proj{\tr}{t}$).  Further, $\tr''$
is such that every read event in $\tr''$ reads the same last write as
in $\tr$.  Also, since $e_1$ and $e_2$ are the last events in their
threads in $\tr'$, we conclude that $\tr'$ is a correct reordering of
$\tr$ that respects $\hb{\tr}$.
%
% Recall that in this case, we
% have $e_1 \in \reads{\tr}(x)$ (for some $x$),
% $e_2 \in \writes{\tr}(x)$, and $\writes{\tr}(x) \cap \setreq \cap
% (\pretwo\setminus\preone) = \emptyset$. We claim that $\tr'$ is a
% correct reordering that respects $\hb{\tr}$. Since $\tr''$ respects
% $\hb{\tr}$, we also have $\tr'$ respects $\hb{\tr}$. Similar to the
% proof of Lemma~\ref{lem:cr_hb}, we can conclude that since $\tr'$
% respects $\hb{\tr}$, $\tr'$ satisfies lock semantics and thread
% order. All we are left to prove is that $\tr'$ preserves the last
% writes. For any event $e \in \setreq \cap \reads{\tr}$, we know that
% $\lw{\tr}(e)
% \simeq \lw{\tr''}(e) \simeq \lw{\tr'}(e)$ since $\tr''$ is a correct 
% reordering; here we use $a \simeq b$ to mean either both $a,b$ are
% undefined or they are both defined and equal. Finally, we have to
% prove $\lw{\tr}(e_1) \simeq \lw{\tr'}(e_1)$. Observe that, if
% $\lw{\tr}(e_1)$ is undefined then we have
% $\writes{\tr}(x) \cap \preone = \emptyset$. Combining this with the
% fact that $\writes{\tr}(x) \cap \setreq (\pretwo\setminus\preone)
% = \emptyset$, we get $\writes{\tr}(x) \cap \setreq = \emptyset$ and so
% $\lw{\tr'}(e_1)$ is also undefined. On the other hand, suppose
% $\lw{\tr}(e_1)$ is defined. Take $e = \lw{\tr}(e_1)$. Since
% $e \shb{\tr} e_1$, $e \in \setreq$. Now, using the observations that
% $\writes{\tr}(x) \cap \setreq \cap (\pretwo\setminus\preone)
% = \emptyset$, and $\trord{\tr''} = \trord{\tr} \cap
% (\setreq \times \setreq)$, we get $\lw{\tr'}(e_1) = e$.
%
% \paragraph{Proof of part~\ref{lbl:sufficient-write}.}
% In this case we have $e_1 \in \writes{\tr}(x)$, for some $x$. There
% are two cases to consider. Let us consider first the case when either
% $e_2 \in \writes{\tr}(x)$, or $e_2 \in \reads{\tr}(x)$ with
% $\lw{\tr}(e_2) = e_1$. This is the easy case when we prove that $\tr'
% = \tr''e_1e_2$ is a correct reordering that respects $\hb{\tr}$. Like
% in the previous paragraph, we can argue that $\tr'$ respects
% $\hb{\tr}$, satisfies lock semantics and thread order. Further because
% $\tr''$ is a correct reordering, the last write of every
% $e \in \setreq \cap \reads{\tr}$ is preserved. Finally, if
% $e_2 \in \reads{\tr}(x)$ then $\lw{\tr}(e_2) = e_1 = \lw{\tr'}(e_2)$
% by assumption and construction of $\tr'$. The second case to consider
% is when $e_2 \in \reads{\tr}(x)$ but $e_1 \neq \lw{\tr}(e_2)$. In this
% case we will prove that $\tr' = \tr''e_2e_1$ is a correct reordering
% that respects $\hb{\tr}$. The proof that $\tr'$ respects $\hb{\tr}$,
% lock semantics, thread order, and preserves the last write of all
% events in $\setreq$, is the same as before. The only thing we need to
% argue is that the last write of $e_2$ is preserved.  If
% $\lw{\tr}(e_2)$ is undefined then since $\setreq \subseteq \pretwo$,
% $\lw{\tr'}(e_2)$ is also undefined. Suppose $e = \lw{\tr}(e_2)$. Then
% since $e \shb{\tr} e_2$, we have $e \in \setreq$. Further since
% $\trord{\tr''} = \trord{\tr} \cap (\setreq \times \setreq)$,
% $\lw{\tr'}(e_2) = e$.
\end{proof}



\section{Proofs for Algorithm~\ref{algo:vc}}
\label{app:algo}
%!TEX root = main.tex

% \subsection{Proofs for Algorithm~\ref{algo:vc}}
% \label{app:algo_update}

We now prove Theorem~\ref{thm:isomorphicVC}, which states the
correctness of Algorithm~\ref{algo:vc}. Before establishing this claim
we would like to introduce some notation and prove some auxiliary
claims. 

Let us fix a trace $\tr$. Recall that for any event $e$, $C_e$ is the
(vector) timestamp assigned by Algorithm~\ref{algo:vc}. Let us denote
by $L_\lk^e$ the value of clock $\Ll_\lk$ just before the event $e$ is
processed. Similarly, let $LW_x^e$ denote the value of clock $\LW_x$
just before $e$ is processed. It is easy to see that the following
invariant is maintained by Algorithm~\ref{algo:vc}.
%
\begin{proposition}
\label{prop:vc-invariant1}
Let $e$ be an arbitrary event of trace $\tr$. Let $e_\lk$ be the last
$\rel(\lk)$-event in $\tr$ before $e$, and let $e_x$ be the last
$\wt(x)$-event in $\tr$ before $e$ (with respect to
$\trord{\tr}$). Note that $e_\lk$ and $e_x$ maybe undefined. Then,
$L_\lk^e = C_{e_\lk}$ and $LW_x^e = C_{e_x}$, where if an event $f$ is
undefined, we take $C_f = \bot$.
\end{proposition}
%
\begin{proof}
The observation follows from the way vector clocks $\Ll_\lk$ and
$\LW_x$ are updated.
\end{proof}

Another invariant that follows from the update rules of
Algorithm~\ref{algo:vc} is the following.
%
\begin{proposition}
\label{prop:vc-invariant2}
Let $e_1$ and $e_2$ be events of thread $t$ such that $e_1 \trord{\tr}
e_2$, i.e., $e_1 \tho{\tr} e_2$. Let $t'$ be any thread such that
$t \neq t'$. Then the following observations hold.
\begin{enumerate}
\item\label{lbl:monotonic} $C_{e_1} \cle C_{e_2}$.
\item\label{lbl:remote-clk-upd} $C_{e_1}(t') = C_{e_2}(t')$ unless
  there is an event $e$ of thread $t$ that is either an $\acq$-event, or
  a $\rd$-event, or a $\join$-event such that $e \neq e_1$ and
  $e_1 \trord{\tr} e \trord{\tr} e_2$.
\item\label{lbl:local-clk-upd} $C_{e_1}(t) = C_{e_2}(t)$ unless there
  is an event $e$ of thread $t$ that is either a $\rel$-event, or a
  $\wt$-event, or a $\fork$-event such that $e \neq e_2$ and
  $e_1 \trord{\tr} e \trord{\tr} e_2$; in this case $C_{e_1}(t) <
  C_{e_2}(t)$.
\end{enumerate}
\end{proposition}
%
\begin{proof}
Follows from the way $\Cc_t$ is updated by Algorithm~\ref{algo:vc}.
\end{proof}

We now prove the main lemma crucial to the correctness of
Algorithm~\ref{algo:vc}, that relates $\shb{}$ to the ordering on
vector clocks.
%
\begin{lemma}
\label{lem:vc-shb}
Let $e = \ev{t,op}$ be an event such that $C_e(t') = k$ for some $t'
\neq t$. Let $e' = \ev{t',op'}$ be the last event such that
$C_{e'}(t') = k$. Then $e' \shb{\tr} e$. 
\end{lemma}
%
\begin{proof}
The result will be proved by induction on the position of $e$ in the
trace $\tr$. Observe that if $e$ is the first event of $\tr$, then
$C_e(t') = 0$ for all $t' \neq t$, no matter what event $e$ is. And
there is no event $e' = \ev{t',op'}$ such that $C_{e'}(t') = 0$. Thus,
the lemma holds vaccuously in the base case.

Let us now consider the inductive step. Define $e_1 = \ev{t,op_1}$ be
the last event in $\tr$ before $e$ (possibly same as $e$) such that
$op_1$ is either $\acq$, $\rd$, or $\join$; if no such $e_1$ exists
then take $e_1$ to be the first event performed by $t$.  Notice, by
our choice of $e_1$ and
Proposition~\ref{prop:vc-invariant2}(\ref{lbl:remote-clk-upd}), for
every $t'' \neq t$, $C_{e_1}(t'') = C_e(t'')$. If $e_1 \neq e$, the
result follows by induction hypothesis on $e_1$. 

Let us assume $e_1 = e$. We need to consider different cases based on
what $e_1$ is.
\begin{itemize}
\item {\bf Case $e = e_1 = \ev{t,\acq(\lk)}$:} Let $f_1$ be the event 
  immediately before $e$ in $\proj{\tr}{t}$ and $f_2$ be the event
  such that $C_{f_2} = L_\lk^e$ (given by
  Proposition~\ref{prop:vc-invariant1}). Note that both $f_1$ and
  $f_2$ may be undefined. Also notice that, for any $t'$, either
  $C_e(t') = 0$, or $C_e(t') = C_{f_1}(t') \neq 0$ (and $f_1$ is
  defined), or $C_e(t') = C_{f_2}(t') \neq 0$ (and $f_2$ is
  defined). If $C_e(t') = 0$ then the lemma follows vaccuously as in
  the base case because there is no event $e' = \ev{t',op'}$ with
  $C_{e'}(t') = 0$. Let us now consider the remaining cases. Let $t_2$
  denote the thread performing $f_2$, if $f_2$ is defined. Consider
  the case when either $C_e(t') = C_{f_1}(t') \neq 0$ or $C_e(t') =
  C_{f_2}(t')$ with $t' \neq t_2$. In this situation, the lemma
  follows using the induction hypothesis on either $f_1$ or $f_2$
  since both $f_1$ and $f_2$ (when defined) are $\shb{\tr} e$. The
  last case to consider is when $t' = t_2$ and $C_e(t') =
  C_{f_2}(t')$. By
  Proposition~\ref{prop:vc-invariant2}(\ref{lbl:local-clk-upd}), $f_2$
  is the last event of $t' = t_2$ whose $t'$th component is
  $k$. Further, by definition $f_2 \shb{\tr} e$, and so the lemma
  holds.
\item {\bf Case $e = e_1 = \ev{t,\rd(x)}$:} Let $f_1$ be the event
  immediately before $e$ in $\proj{\tr}{t}$ and $f_2$ be the event
  such that $C_{f_2} = LW_x^e$ (given by
  Proposition~\ref{prop:vc-invariant1}). Again, both $f_1$ and $f_2$
  may be undefined. Also notice that, for any $t'$, either $C_e(t') =
  0$, or $C_e(t') = C_{f_1}(t') \neq 0$ (and $f_1$ is defined), or
  $C_e(t') = C_{f_2}(t') \neq 0$ (and $f_2$ is defined). If $C_e(t') =
  0$ then the lemma follows vaccuously as in the base case because
  there is no event $e' = \ev{t',op'}$ with $C_{e'}(t') = 0$. Let us
  now consider the remaining cases. Let $t_2$ denote the thread
  performing $f_2$, if $f_2$ is defined. Consider the case when either
  $C_e(t') = C_{f_1}(t') \neq 0$ or $C_e(t') = C_{f_2}(t')$ with
  $t' \neq t_2$. In this situation, the lemma follows using the
  induction hypothesis on either $f_1$ or $f_2$ since both $f_1$ and
  $f_2$ (when defined) are $\shb{\tr} e$. The last case to consider is
  when $t' = t_2$ and $C_e(t') = C_{f_2}(t')$. By
  Proposition~\ref{prop:vc-invariant2}(\ref{lbl:local-clk-upd}), $f_2$
  is the last event of $t' = t_2$ whose $t'$th component is
  $k$. Further, by definition $f_2 \shb{\tr} e$, and so the lemma
  holds.
\item {\bf Case $e = e_1 = \ev{t,\join(t_1)}$:} Let $f_1$ be the event
  immediately before $e$ in $\proj{\tr}{t}$ and $f_2$ be the last
  event of the form $\ev{t_1,op}$. Again, both $f_1$ and $f_2$ may be
  undefined. Also notice that, for any $t'$, either (a) $C_e(t') = 0$,
  or (b) $t_1 = t'$, $C_e(t') = 1$, and $f_2$ is undefined, or (c)
  $C_e(t') = C_{f_1}(t') \neq 0$ and $f_1$ is defined, or (d) $C_e(t')
  = C_{f_2}(t') \neq 0$ and $f_2$ is defined. In cases (a) or (b)
  above, the lemma follows vaccuously as in the base case because
  there is no event $e' = \ev{t',op'}$ with $C_{e'}(t') = k$ (where
  $k$ is either $0$ or $1$ depending on which we case we
  consider). Let us now consider the remaining cases. Let $t_2$ denote
  the thread performing $f_2$, if $f_2$ is defined. Consider the case
  when either $C_e(t') = C_{f_1}(t') \neq 0$ or $C_e(t') =
  C_{f_2}(t')$ with $t' \neq t_2$ (and $f_2$ defined). In this
  situation, the lemma follows using the induction hypothesis on
  either $f_1$ or $f_2$ since both $f_1$ and $f_2$ (when defined) are
  $\shb{\tr} e$. The last case to consider is when $t' = t_2$ and
  $C_e(t') = C_{f_2}(t')$. By definition, $f_2$ is the last event of
  $t' = t_2$ whose $t'$th component is $k$. Further, by definition
  $f_2 \shb{\tr} e$, and so the lemma holds.
\item {\bf Case $e = e_1$ is the first event:} This is the case when 
  the above 3 cases don't hold. So $e = e_1$ is not an $\acq$-event,
  nor a $\rd$-event, nor a $\join$-event. Moreover, since $e$ is the
  first event of thread $t$ and is of the form $\ev{t,op}$, it must be
  the the thread $t$ has not been forked by any thread in $\tr$. Thus,
  for any $t' \neq t$, $C_e(t') = 0$. The lemma, therefore, follows
  vaccuously as in the base case. \qedhere
\end{itemize}
\end{proof}

We are ready to present the proof of Theorem~\ref{thm:isomorphicVC}.

\begin{reptheorem}{thm:isomorphicVC}
For events $e, e' \in \events{\tr}$ such that $e \trord{\tr} e'$,
$C_e \cle C_{e'}$ iff $e \shb{\tr} e'$
\end{reptheorem}

\begin{proof}
Let us first prove the implication from left to right. Consider $e,e'$
such that $e \trord{\tr} e'$. If $e \tho{\tr} e'$ then $e \shb{\tr}
e'$ since $\tho{\tr} \subseteq \shb{\tr}$. On the other hand, if $e$
and $e'$ are not events of the same thread, then this direction of the
theorem follows from Lemma~\ref{lem:vc-shb}.

Let us now prove the implication from right to left. Consider events
such that $e \shb{\tr} e'$. Then, by definition, we have a sequence of
events $e = f_1,f_2,\ldots f_k = e'$ such that for every $i$,
$f_i \trord{\tr} f_{i+1}$ and either (i) $f_i$ and $f_{i+1}$ are both
events of the form $\ev{t,op}$, or (ii) $f_i$ is a $\rel(\lk)$-event
and $f_{i+1}$ is a $\acq(\lk)$-event, or (iii) $f_i$ is a
$\fork(t)$-event and $f_{i+1}$ is an event of the form $\ev{t,op}$, or
(iv) $f_i$ is an event of the form $\ev{t,op}$ and $f_{i+1}$ is a
$\join(t)$-event, or (v) $f_i = \lw{\tr}(f_{i+1})$. In each of these
cases, Algorithm~\ref{algo:vc} ensures that $C_{f_i} \cle
C_{f_{i+1}}$. Thus, we have $C_e \cle C_{e'}$.
\end{proof}

We now prove Theorem~\ref{thm:correct-races}.  We first note some
auxiliary propositions.  Let us denote by $R_x^e$ the value of clock
$\Rr_x$ just before the event $e$ is processed.  Similarly, let
$W_x^e$ denote the value of clock $\Ww_x$ just before $e$ is
processed. It is easy to see that the following invariant is
maintained by Algorithm~\ref{algo:vc}.

\begin{proposition}
\label{prop:vc-invariantrw}
Let $e$ be an arbitrary event of trace $\tr$. Let $e_t^{\rd(x)}$ 
be the last $\ev{t, \rd(x)}$-event in $\tr$ before $e$, 
and let $e_t^{\wt(x)}$ be the last
$\ev{t, \wt(x)}$-event in $\tr$ before $e$ (with respect to
$\trord{\tr}$). 
Note that $e_t^{\rd(x)}$  and $e_t^{\wt(x)}$  maybe undefined. 
Then, $\forall t, R_x(t) = C_{e_t^{\rd(x)}}(t)$ and 
$\forall t, W_x(t) = C_{e_t^{\wt(x)}}(t)$ 
where if an event $f$ is undefined, we take $C_f = \bot$.
\end{proposition}

\begin{proof}
The observation follows from the way vector clocks $\Rr_x$ and
$\Ww_x$ are updated.
\end{proof}

\begin{lemma}
\label{lem:prev_clock}
Let $e_1, e_2 \in \events{\tr}$ performed by threads $t_1$ and $t_2$,
respectively, such that $t_1 \neq t_2$.  Then, $e_1 \shb{\tr} e_2$ iff
$C_{e_1} \cle C_{e_2}[( C_{e_2}(t_2) + 1)/t_2]$.
\end{lemma}

\begin{proof}
Let $c_2 = C_{e_2}(t_2)$.
First suppose that $e_1 \shb{\tr} e_2$.
Then, from Theorem~\ref{thm:isomorphicVC}, we have $C_{e_1} \cle C_{e_2}$
and thus $C_{e_1} \cle C_{e_2}[(c_2+1)/t_2]$.
Next, assume that $C_{e_1} \cle C_{e_2}[(c_2 + 1)/t_2]$.
In particular, $C_{e_1}(t_1) \leq C_{e_2}(t_1)$.
Then by Lemma~\ref{lem:vc-shb}, we have $e_1 \shb{\tr} e_2$
\end{proof}

\begin{reptheorem}{thm:correct-races}
Let $e$ be a read/write event $e \in \events{\tr}$.
Algorithm~\ref{algo:vc} reports a race at $e$
iff there is an event $e' \in \events{\tr}$ such that $(e', e)$
is an $\hb{\tr}$-schedulable race.
\end{reptheorem}

\begin{proof}
% \ucomment{Prove this. You will have to prove that $\Rr_x$ and $\Ww_x$ are maintained correctly.}
% Let $e'$ and $e$ be conflicting events in $\tr$.
% Then by Theorem~\ref{thm:SHBSoundness}, $(e, e')$ is an $\hb{\tr}$-schedulable race
% iff either $\ltho{\tr}(e)$ does not exist or if it exists, then 
% $e' \not\leq_\textsf{SHB}^\tr \ltho{\tr}(e)$.
% 
Let us first consider the case when $\ltho{\tr}(e)$ is not defined.
Then, the value of the clock $\Cc_t = \bot[1/t]$ at line 19, 24 or 26 (depending
upon whether $e$ is a read or a write event).
If the check $\neg(\Ww_x \cle \Cc_t)$ passes, then
there is a $t'$ such that $\Ww_x(t') > \Cc_t$ and thus
the there is an event $e'$ (namely the last write event of $x$ in thread $t'$)
that conflicts with $e$. Thus, $(e', e)$ is a $\hb{\tr}$-schedulable race
by Theorem~\ref{thm:SHBSoundness}.
On the other hand, if the check fails, then
$\Ww_x = \bot[1/t]$ or $\Ww_x = \bot$ and in either case
there is no event that conflicts with $e$.
One can similarly argue that the checks on Lines 24 and 26
are both necessary and sufficient for the case when $e$ is a write event
and $\ltho{\tr}(e)$ is undefined.
% Similarly, for a write event $e$, if one of the checks
% $\neg(\Rr_x \cle \Cc_t)$ or $\neg(\Ww_x \cle \Cc_t)$ passes,
% it means that there is an event $e'$ that conflicts with $e$
% and thus $(e', e)$ is an $\hb{\tr}$-schedulable race.

Next we consider the case when $f = \ltho{\tr}(e)$ is defined.  Now
let $e$ be a read event.  If the check $\neg(\Ww_x \cle \Cc_t)$
passes, then there is a $t'$ such that $\Ww_x(t') > \Cc_t(t')$ and
thus there is an event $e' = \ev{t', \wt(x)}$ such that $C_{e'}(t')
> \Cc_t(t')$ and thus $C_{e'} \not\cle \Cc_t$; note that it must be
the case that $t' \neq t$.  Depending upon whether $f$ is a
read/join/acquire event or a write/fork/release event, the value of
the clock $\Cc_t$ at Line 19 is $\Cc_t = C_f$ or $\Cc_t =
C_f[(C_f(t)+1)/t]$.  In either case, by Lemma~\ref{lem:prev_clock}, we
have that $e' \not\leq_\textsf{SHB}^\tr f$.  On the other hand if,
$\Ww_x \cle \Cc_t$, then $\forall t' \neq t,
C_{e_{t'}^{\wt(x)}}(t') \leq \Cc_t(t')$, where $e_{u}^{\wt(x)}$ is the
last write event of $x$ performed by thread $u$.  This means that for
every event $e'$ such that $e'$ conflicts with, by
Lemma~\ref{lem:prev_clock}, we have $e' \shb{\tr} f$ and thus $(e',
e)$ is not an $\hb{\tr}$-schedulable race.  The argument for the case
when $e$ is a write event is similar.
\end{proof}

We now establish the asymptotic space and time bounds for
Algorithm~\ref{algo:vc}.

\begin{reptheorem}{thm:complexityVC}
For a trace $\tr$ with $n$ events, $T$ threads, $V$ variables, and $L$
locks, Algorithm~\ref{algo:vc} runs in time $O(nT\log n)$ and uses
$O((V+L+T)T\log n)$ space.
\end{reptheorem}

\begin{proof}
Observe that for a trace of length $n$, every component of each of the
vector clocks is bounded by $n$. Thus, each vector clock takes space
$O(T\log n)$, where $T$ is the number of threads. We have a vector
clock for each thread, lock, and variable, which gives us a space
bound of $O((V+T+L)T\log n)$. Notice that to process any event we need
to update constantly many vector clocks. The time to update any single
vector clock can be bounded by its size $O(T\log n)$. Thus, the total
running time is $O(nT\log n)$.
\end{proof}





\section{False Positives Reported by Existing Practical Dynamic Race Detection Tools}
\label{app:false_races}
%!TEX root = main.tex

We evaluate existing tools that
use happens-before based race detection to
check if they report false (unschedulable) races.

We use the following program in Figure~\ref{fig:example1} to assess if these
tools guarantee soundness.
The Java source code for this program is in Figure~\ref{simple_java}
and the C source code is in Figure~\ref{simple_c}

%!TEX root = main.tex

\begin{figure}[t]
\begin{subfigure}{.4\textwidth}
\begin{Verbatim}[commandchars=\\\{\}, fontsize=\footnotesize, baselinestretch=0.5]
\PYG{k+kd}{public} \PYG{k+kd}{class} \PYG{n+nc}{Test} \PYG{k+kd}{extends} \PYG{n}{Thread}\PYG{o}{\PYGZob{}}

    \PYG{k+kd}{static} \PYG{k+kt}{int} \PYG{n}{x}\PYG{o}{,} \PYG{n}{y}\PYG{o}{;}
    \PYG{k+kd}{public} \PYG{k+kt}{int} \PYG{n}{id}\PYG{o}{;}

    \PYG{n+nd}{@Override}
    \PYG{k+kd}{public} \PYG{k+kt}{void} \PYG{n+nf}{run}\PYG{o}{()} \PYG{o}{\PYGZob{}}
        \PYG{k}{if}\PYG{o}{(}\PYG{n}{id} \PYG{o}{==} \PYG{l+m+mi}{1}\PYG{o}{)\PYGZob{}}
            \PYG{n}{y} \PYG{o}{=} \PYG{n}{x} \PYG{o}{+} \PYG{l+m+mi}{5}\PYG{o}{;}
        \PYG{o}{\PYGZcb{}}
        \PYG{k}{if}\PYG{o}{(}\PYG{n}{id} \PYG{o}{==} \PYG{l+m+mi}{2}\PYG{o}{)\PYGZob{}}
            \PYG{k}{if} \PYG{o}{(} \PYG{n}{y}\PYG{o}{==}\PYG{l+m+mi}{5} \PYG{o}{)\PYGZob{}}
                \PYG{n}{x} \PYG{o}{=} \PYG{l+m+mi}{10}\PYG{o}{;}
            \PYG{o}{\PYGZcb{}}
            \PYG{k}{else}\PYG{o}{\PYGZob{}}
                \PYG{k}{while}\PYG{o}{(}\PYG{k+kc}{true}\PYG{o}{);}
            \PYG{o}{\PYGZcb{}}
        \PYG{o}{\PYGZcb{}}
    \PYG{o}{\PYGZcb{}}

    \PYG{k+kd}{public} \PYG{k+kd}{static} \PYG{k+kt}{void} \PYG{n+nf}{main}\PYG{o}{(}\PYG{n}{String} \PYG{n}{args}\PYG{o}{[])}
            \PYG{k+kd}{throws} \PYG{n}{Exception} \PYG{o}{\PYGZob{}}
        \PYG{k+kd}{final} \PYG{n}{Test} \PYG{n}{t1} \PYG{o}{=} \PYG{k}{new} \PYG{n}{Test}\PYG{o}{();}
        \PYG{k+kd}{final} \PYG{n}{Test} \PYG{n}{t2} \PYG{o}{=} \PYG{k}{new} \PYG{n}{Test}\PYG{o}{();}
        \PYG{n}{t1}\PYG{o}{.}\PYG{n+na}{id} \PYG{o}{=} \PYG{l+m+mi}{1}\PYG{o}{;}
        \PYG{n}{t2}\PYG{o}{.}\PYG{n+na}{id} \PYG{o}{=} \PYG{l+m+mi}{2}\PYG{o}{;}
        \PYG{n}{t1}\PYG{o}{.}\PYG{n+na}{start}\PYG{o}{();}
        \PYG{n}{t2}\PYG{o}{.}\PYG{n+na}{start}\PYG{o}{();}
        \PYG{n}{t1}\PYG{o}{.}\PYG{n+na}{join}\PYG{o}{();}
        \PYG{n}{t2}\PYG{o}{.}\PYG{n+na}{join}\PYG{o}{();}
    \PYG{o}{\PYGZcb{}}
\PYG{o}{\PYGZcb{}}
\end{Verbatim}
\vspace{0.5cm}
\caption{Multi-threaded Java program}
\label{simple_java}
\end{subfigure}
\hfill
\begin{subfigure}{.4\textwidth}
\begin{Verbatim}[commandchars=\\\{\}, fontsize=\footnotesize, baselinestretch=0.5]
\PYG{c+cp}{\PYGZsh{}include} \PYG{c+cpf}{\PYGZlt{}pthread.h\PYGZgt{}}
\PYG{c+cp}{\PYGZsh{}include} \PYG{c+cpf}{\PYGZlt{}stdio.h\PYGZgt{}}

\PYG{k+kt}{int} \PYG{n}{x}\PYG{p}{;}
\PYG{k+kt}{int} \PYG{n}{y}\PYG{p}{;}

\PYG{k+kt}{void} \PYG{o}{*}\PYG{n+nf}{Thread1}\PYG{p}{(}\PYG{k+kt}{void} \PYG{o}{*}\PYG{n}{a}\PYG{p}{)} \PYG{p}{\PYGZob{}}
  \PYG{n}{y} \PYG{o}{=} \PYG{n}{x} \PYG{o}{+} \PYG{l+m+mi}{5}\PYG{p}{;}
  \PYG{k}{return} \PYG{n+nb}{NULL}\PYG{p}{;}
\PYG{p}{\PYGZcb{}}

\PYG{k+kt}{void} \PYG{o}{*}\PYG{n+nf}{Thread2}\PYG{p}{(}\PYG{k+kt}{void} \PYG{o}{*}\PYG{n}{a}\PYG{p}{)} \PYG{p}{\PYGZob{}}
  \PYG{k}{if} \PYG{p}{(}\PYG{n}{y} \PYG{o}{==} \PYG{l+m+mi}{5}\PYG{p}{)\PYGZob{}}
    \PYG{n}{x} \PYG{o}{=} \PYG{l+m+mi}{10}\PYG{p}{;}
  \PYG{p}{\PYGZcb{}}
  \PYG{k}{else}\PYG{p}{\PYGZob{}}
    \PYG{k}{while}\PYG{p}{(}\PYG{n+nb}{true}\PYG{p}{)\PYGZob{}\PYGZcb{}}
  \PYG{p}{\PYGZcb{}}
  \PYG{k}{return} \PYG{n+nb}{NULL}\PYG{p}{;}
\PYG{p}{\PYGZcb{}}

\PYG{k+kt}{int} \PYG{n+nf}{main}\PYG{p}{()} \PYG{p}{\PYGZob{}}
  \PYG{n}{x} \PYG{o}{=} \PYG{l+m+mi}{0}\PYG{p}{;}
  \PYG{n}{y} \PYG{o}{=} \PYG{l+m+mi}{0}\PYG{p}{;}
  \PYG{n}{pthread\PYGZus{}t} \PYG{n}{t}\PYG{p}{[}\PYG{l+m+mi}{2}\PYG{p}{];}
  \PYG{n}{pthread\PYGZus{}create}\PYG{p}{(}\PYG{o}{\PYGZam{}}\PYG{n}{t}\PYG{p}{[}\PYG{l+m+mi}{0}\PYG{p}{],} \PYG{n+nb}{NULL}\PYG{p}{,} \PYG{n}{Thread1}\PYG{p}{,} \PYG{n+nb}{NULL}\PYG{p}{);}
  \PYG{n}{pthread\PYGZus{}create}\PYG{p}{(}\PYG{o}{\PYGZam{}}\PYG{n}{t}\PYG{p}{[}\PYG{l+m+mi}{1}\PYG{p}{],} \PYG{n+nb}{NULL}\PYG{p}{,} \PYG{n}{Thread2}\PYG{p}{,} \PYG{n+nb}{NULL}\PYG{p}{);}
  \PYG{n}{pthread\PYGZus{}join}\PYG{p}{(}\PYG{n}{t}\PYG{p}{[}\PYG{l+m+mi}{0}\PYG{p}{],} \PYG{n+nb}{NULL}\PYG{p}{);}
  \PYG{n}{pthread\PYGZus{}join}\PYG{p}{(}\PYG{n}{t}\PYG{p}{[}\PYG{l+m+mi}{1}\PYG{p}{],} \PYG{n+nb}{NULL}\PYG{p}{);}
\PYG{p}{\PYGZcb{}}
\end{Verbatim}
\vspace{0.5cm}
\caption{Multi-threaded C program}
\label{simple_c}
\end{subfigure}
\caption{Concurrent program from Figure~\ref{fig:program1}}
\label{fig:programs}
\end{figure}

\subsection{\fasttrack}

When run on \fasttrack, the Java program in Fig~\ref{simple_java}, \fasttrack~produces
the following output:

\vspace{0.5cm}
\hrule
{\small
\begin{verbatim}
## 
## =====================================================================
## HappensBefore Error
## 
##          Thread: 2    
##           Blame: Test.y_I
##           Count: 1    (max: 100)
##     Guard State: [(0:0) (1:1)]
##           Class: Test
##           Field: null.Test.y_I
##           Locks: []
##         Prev Op: write-by-thread-1
##      Prev Op CV: [(0:0) (1:1)]
##          Cur Op: read
##       Cur Op CV: [(0:1) (1:0) (2:1)]
##           Stack: Use -stacks to show stacks...
## =====================================================================
## 
## 
## =====================================================================
## HappensBefore Error
## 
##          Thread: 2    
##           Blame: Test.x_I
##           Count: 1    (max: 100)
##     Guard State: [(0:0) (1:1)]
##           Class: Test
##           Field: null.Test.x_I
##           Locks: []
##         Prev Op: read-by-thread-1
##      Prev Op CV: [(0:0) (1:1)]
##          Cur Op: write
##       Cur Op CV: [(0:1) (1:0) (2:1)]
##           Stack: Use -stacks to show stacks...
## =====================================================================
## 
\end{verbatim}
}
\hrule
\vspace{0.5cm}

That is, the flags both the fields \texttt{Test.y} and \texttt{Test/x}.
However, in any execution in \texttt{Test.x} both written and read by different threads,
the read always occurs before the write (with a read on \texttt{y} separating them).

\subsection{ThreadSanitizer}

We run the C program in Fig~\ref{simple_c} on ThreadSanitizer (shipped with LLVM).

\vspace{0.5cm}
\hrule
{\small
\begin{verbatim}
==================
WARNING: ThreadSanitizer: data race (pid=76224)
  Read of size 4 at 0x0001030ba074 by thread T2:
    #0 Thread2(void*) simple_race.cc:13 (a.out:x86_64+0x100000d1e)

  Previous write of size 4 at 0x0001030ba074 by thread T1:
    #0 Thread1(void*) simple_race.cc:8 (a.out:x86_64+0x100000cc9)

  Location is global 'y' at 0x0001030ba074 (a.out+0x000100001074)

  Thread T2 (tid=747232, running) created by main thread at:
    #0 pthread_create <null>:1600736 (libclang_rt.tsan_osx_dynamic.dylib:x86_64h+0x283ed)
    #1 main simple_race.cc:27 (a.out:x86_64+0x100000df3)

  Thread T1 (tid=747231, finished) created by main thread at:
    #0 pthread_create <null>:1600736 (libclang_rt.tsan_osx_dynamic.dylib:x86_64h+0x283ed)
    #1 main simple_race.cc:26 (a.out:x86_64+0x100000dd4)

SUMMARY: ThreadSanitizer: data race simple_race.cc:13 in Thread2(void*)
==================
==================
WARNING: ThreadSanitizer: data race (pid=76224)
  Write of size 4 at 0x0001030ba070 by thread T2:
    #0 Thread2(void*) simple_race.cc:14 (a.out:x86_64+0x100000d3a)

  Previous read of size 4 at 0x0001030ba070 by thread T1:
    #0 Thread1(void*) simple_race.cc:8 (a.out:x86_64+0x100000cae)

  Location is global 'x' at 0x0001030ba070 (a.out+0x000100001070)

  Thread T2 (tid=747232, running) created by main thread at:
    #0 pthread_create <null>:1600736 (libclang_rt.tsan_osx_dynamic.dylib:x86_64h+0x283ed)
    #1 main simple_race.cc:27 (a.out:x86_64+0x100000df3)

  Thread T1 (tid=747231, finished) created by main thread at:
    #0 pthread_create <null>:1600736 (libclang_rt.tsan_osx_dynamic.dylib:x86_64h+0x283ed)
    #1 main simple_race.cc:26 (a.out:x86_64+0x100000dd4)

SUMMARY: ThreadSanitizer: data race simple_race.cc:14 in Thread2(void*)
==================
ThreadSanitizer: reported 2 warnings
\end{verbatim}
}
\hrule
\vspace{0.5cm}

ThreadSanitizer also reports a race on both global locations \texttt{x} and \texttt{y}.

\subsection{Helgrind}

Helgrind is a data race detector integrated with Valgrind.
We analyzed the C program in Fig~\ref{simple_c}.
\vspace{0.5cm}
\hrule
% \vspace{0.5cm}
{\small
\begin{verbatim}
==2403== Helgrind, a thread error detector
==2403== Copyright (C) 2007-2015, and GNU GPL'd, by OpenWorks LLP et al.
==2403== Using Valgrind-3.12.0 and LibVEX; rerun with -h for copyright info
==2403== Command: ./a.out
==2403== 
==2403== ---Thread-Announcement------------------------------------------
==2403== 
==2403== Thread #3 was created
==2403==    at 0x596F30E: clone (in /usr/lib64/libc-2.17.so)
==2403==    by 0x4E41FD9: do_clone.constprop.4 (in /usr/lib64/libpthread-2.17.so)
==2403==    by 0x4E434C8: pthread_create@@GLIBC_2.2.5 (in /usr/lib64/libpthread-2.17.so)
==2403==    by 0x4C3064A: pthread_create_WRK (hg_intercepts.c:427)
==2403==    by 0x4C31728: pthread_create@* (hg_intercepts.c:460)
==2403==    by 0x400718: main (simple_race.cc:27)
==2403== 
==2403== ---Thread-Announcement------------------------------------------
==2403== 
==2403== Thread #2 was created
==2403==    at 0x596F30E: clone (in /usr/lib64/libc-2.17.so)
==2403==    by 0x4E41FD9: do_clone.constprop.4 (in /usr/lib64/libpthread-2.17.so)
==2403==    by 0x4E434C8: pthread_create@@GLIBC_2.2.5 (in /usr/lib64/libpthread-2.17.so)
==2403==    by 0x4C3064A: pthread_create_WRK (hg_intercepts.c:427)
==2403==    by 0x4C31728: pthread_create@* (hg_intercepts.c:460)
==2403==    by 0x4006F9: main (simple_race.cc:26)
==2403== 
==2403== ----------------------------------------------------------------
==2403== 
==2403== Possible data race during read of size 4 at 0x601044 by thread #3
==2403== Locks held: none
==2403==    at 0x4006A3: Thread2(void*) (simple_race.cc:13)
==2403==    by 0x4C3083E: mythread_wrapper (hg_intercepts.c:389)
==2403==    by 0x4E42E24: start_thread (in /usr/lib64/libpthread-2.17.so)
==2403==    by 0x596F34C: clone (in /usr/lib64/libc-2.17.so)
==2403== 
==2403== This conflicts with a previous write of size 4 by thread #2
==2403== Locks held: none
==2403==    at 0x40068E: Thread1(void*) (simple_race.cc:8)
==2403==    by 0x4C3083E: mythread_wrapper (hg_intercepts.c:389)
==2403==    by 0x4E42E24: start_thread (in /usr/lib64/libpthread-2.17.so)
==2403==    by 0x596F34C: clone (in /usr/lib64/libc-2.17.so)
==2403==  Address 0x601044 is 0 bytes inside data symbol "y"
==2403== 
==2403== ----------------------------------------------------------------
==2403== 
==2403== Possible data race during write of size 4 at 0x601040 by thread #3
==2403== Locks held: none
==2403==    at 0x4006AE: Thread2(void*) (simple_race.cc:14)
==2403==    by 0x4C3083E: mythread_wrapper (hg_intercepts.c:389)
==2403==    by 0x4E42E24: start_thread (in /usr/lib64/libpthread-2.17.so)
==2403==    by 0x596F34C: clone (in /usr/lib64/libc-2.17.so)
==2403== 
==2403== This conflicts with a previous read of size 4 by thread #2
==2403== Locks held: none
==2403==    at 0x400685: Thread1(void*) (simple_race.cc:8)
==2403==    by 0x4C3083E: mythread_wrapper (hg_intercepts.c:389)
==2403==    by 0x4E42E24: start_thread (in /usr/lib64/libpthread-2.17.so)
==2403==    by 0x596F34C: clone (in /usr/lib64/libc-2.17.so)
==2403==  Address 0x601040 is 0 bytes inside data symbol "x"
==2403== 
==2403== 
==2403== For counts of detected and suppressed errors, rerun with: -v
==2403== Use --history-level=approx or =none to gain increased speed, at
==2403== the cost of reduced accuracy of conflicting-access information
==2403== ERROR SUMMARY: 2 errors from 2 contexts (suppressed: 0 from 0)
\end{verbatim}
}

\hrule
\vspace{0.5cm}

Helgrind also incorrectly reports a race on both \texttt{x} and \texttt{y}.

\subsection{DRD}

Valgrind also provides another race detector DRD.
We analyze program in Fig~\ref{simple_c} in DRD:

% \hrule
\vspace{0.5cm}
\hrule
{\small
\begin{verbatim}
==2624== drd, a thread error detector
==2624== Copyright (C) 2006-2015, and GNU GPL'd, by Bart Van Assche.
==2624== Using Valgrind-3.12.0 and LibVEX; rerun with -h for copyright info
==2624== Command: ./a.out
==2624== 
==2624== Thread 3:
==2624== Conflicting load by thread 3 at 0x00601044 size 4
==2624==    at 0x4006A3: Thread2(void*) (simple_race.cc:13)
==2624==    by 0x4C30193: vgDrd_thread_wrapper (drd_pthread_intercepts.c:444)
==2624==    by 0x4E4FE24: start_thread (in /usr/lib64/libpthread-2.17.so)
==2624==    by 0x597C34C: clone (in /usr/lib64/libc-2.17.so)
==2624== Allocation context: BSS section of /home/umathur3/a.out
==2624== Other segment start (thread 2)
==2624==    (thread finished, call stack no longer available)
==2624== Other segment end (thread 2)
==2624==    (thread finished, call stack no longer available)
==2624== 
==2624== Conflicting store by thread 3 at 0x00601040 size 4
==2624==    at 0x4006AE: Thread2(void*) (simple_race.cc:14)
==2624==    by 0x4C30193: vgDrd_thread_wrapper (drd_pthread_intercepts.c:444)
==2624==    by 0x4E4FE24: start_thread (in /usr/lib64/libpthread-2.17.so)
==2624==    by 0x597C34C: clone (in /usr/lib64/libc-2.17.so)
==2624== Allocation context: BSS section of /home/umathur3/a.out
==2624== Other segment start (thread 2)
==2624==    (thread finished, call stack no longer available)
==2624== Other segment end (thread 2)
==2624==    (thread finished, call stack no longer available)
==2624== 
==2624== 
==2624== For counts of detected and suppressed errors, rerun with: -v
==2624== ERROR SUMMARY: 2 errors from 2 contexts (suppressed: 18 from 12)
\end{verbatim}
}

\hrule
\vspace{0.5cm}
Again, DRD reports two races, one of which is false.

\end{document}
