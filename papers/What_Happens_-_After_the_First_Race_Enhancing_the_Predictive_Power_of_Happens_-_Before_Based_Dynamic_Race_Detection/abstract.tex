%!TEX root = main.tex

Dynamic race detection is the problem of determining if an observed
program execution reveals the presence of a data race in a
program. The classical approach to solving this problem is to detect
if there is a pair of conflicting memory accesses that are unordered
by Lamport's happens-before (HB) relation. 
HB based race detection is known to not report false positives, i.e.,
it is sound.
However, the soundness guarantee of HB only promises 
that the first pair of unordered, conflicting events is a 
\emph{schedulable} data race. That is, there
can be pairs of HB-unordered conflicting data accesses that are
not schedulable races because there is no reordering of
the events of the execution, where the events in race can be executed
immediately after each other. We introduce a new partial order, called
schedulable happens-before (SHB) that exactly characterizes the pairs of
schedulable data races --- every pair of conflicting data accesses
that are identified by SHB can be scheduled, and every HB-race that can be
scheduled is identified by SHB. 
Thus, the SHB partial order is truly sound.
We present
a linear time, vector clock algorithm to detect schedulable races
using SHB. 
Our experiments demonstrate the value of our algorithm for dynamic race
detection --- SHB incurs only little performance overhead and can scale to
executions from real-world software applications without
compromising soundness.