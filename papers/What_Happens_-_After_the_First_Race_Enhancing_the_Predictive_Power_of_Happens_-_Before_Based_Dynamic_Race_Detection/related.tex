%!TEX root = main.tex

The notion of correct reorderings to characterize causality in
executions has been derived from~\cite{cp2012,wcp2017}.
In~\cite{rv2014} a similar notion, called \emph{feasible traces}
encompasses a more general causality model based on control flow information.
Weak happens-before~\cite{sen2005detecting},  
Mazurkiewicz equivalence~\cite{mazurkiewicz1987,Abdulla14} 
and observation equivalence~\cite{Chalupa2017} are other models
that attempt to characterize causality.
Many of these models also incorporate the notion of last-write causality, similar to SHB.
However, these algorithms use expensive search algorithms like SAT solving
to explore the space of correct reorderings, unlike
a linear time vector clock algorithm like that for SHB.
Our experimental evaluation concurs with this observation.
Similar dependency relation called \emph{reads-from} is 
also used to characterize weak memory consistency 
semantics~\cite{Alglave:2014:HCM:2633904.2627752,HuangTSO16}. 

Race detection techniques can be broadly
classified as either being
static or dynamic.
Static race detection~\cite{Naik:2006:ESR:1133255.1134018,pratikakis11locksmith,Radoi:2013:PSR:2483760.2483765,racerx,voung2007relay,heisenbugs,Yahav:2001:VSP:373243.360206,echo}
is the problem of detecting if a program has an execution that exhibits
a data race, by analyzing its source code.
This problem, in its full generality, is undecidable
and practical tools employing static analysis techniques often face
a trade-off between scalability and precision.
Further, the use of such techniques often require the
programmer to add annotations to help guide static race detectors.

Dynamic race detection techniques, on the other hand,
examine a single execution of the
program to discover a data race in the program.
A large number of tools employing dynamic analysis are 
based on lockset-like analysis proposed by Eraser~\cite{savage1997eraser}.
Here, one tracks, for each memory location accessed, the set of
locks that protect the memory location on each access.
If this lockset becomes empty during the program execution,
a warning is issued.
Lockset-based analysis suffers from false positives.
Other dynamic race detectors  employ happens-before~\cite{lamport1978time} based
analysis. 
These include the use of 
vector clock~\cite{Mattern1988,Fidge:1991:LTD:112827.112860} 
algorithms such as \textsc{Djit}$^+$~\cite{Pozniansky:2003:EOD:966049.781529} and \fasttrack~\cite{fasttrack}
and the use of sets of threads and locks, as in, GoldiLocks~\cite{elmas2007goldilocks}.
As demonstrated in this paper, happens-before based analysis is 
sound only if limited to detecting the first race.
Other techniques can be categorized as \emph{predictive}
and can detect races missed by HB by exploring more correct reorderings of an observed trace.
These include use of SMT solvers~\cite{Said2011,rv2014,ipa2016,Huang2016}
and other techniques based on weakening the HB partial order
including CP~\cite{cp2012} and WCP~\cite{wcp2017}.
Amongst these, WCP is the only technique that has a linear running
time and is known to scale to large traces.
The soundness guarantee of partial order based techniques, like WCP and CP,
is again, limited to the first race.
Nevertheless, they do detect subtle races that HB can miss.
Our approach complements this line of research.
Other dynamic techniques such as random 
testing~\cite{Sen:2008:RDR:1375581.1375584},
sampling~\cite{marino2009literace,Erickson:2010:EDD:1924943.1924954},
and hybrid race detection~\cite{choi:2003:HDD:781498.781528}
are based on both locksets and happens-before relation.
