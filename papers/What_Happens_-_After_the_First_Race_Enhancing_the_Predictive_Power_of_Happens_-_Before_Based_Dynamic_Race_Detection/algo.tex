%!TEX root = main.tex

We will discuss two algorithms for detecting races identified by the
$\shb{}$ partial order. 
The algorithm is based on efficient, vector clock based
computation of the $\shb{}$-partial order. It is similar to the
standard \textsc{Djit}$^+$
algorithm~\cite{Pozniansky:2003:EOD:966049.781529} to detect
HB-races. We will first briefly discuss vector clocks and
associated notations.  Then, we will discuss a one-pass streaming
vector clock algorithm to compute $\shb{}$ for detecting races.  
Finally, we will discuss how epoch optimizations, similar to \fasttrack~\cite{fasttrack} 
can be readily applied in our setting to enhance performance
of the proposed vector clock algorithm.

% \vspace{-0.1in}
\subsection{Vector Clocks and Times}
% \vspace{-0.1in}
% \paragraph{Vector Times and clocks} 
A vector \emph{time} \textit{VT : $\threads{\tr}$ $\to$ Nat} maps each
thread in a trace $\tr$ to a natural number.  Vector times support
comparison operation $\cle$ for point-wise comparison, join operation
($\mx$) for point-wise maximum, and update operation $V[n/t]$ which
assigns the time $n \in $\textit{Nat} to the component $t
\in \threads{\tr}$ in the vector time $V$.  Vector time $\bot$ maps all
threads to 0. Formally, 
% \vspace*{.2\baselineskip} 
% \begin{tabular}{rclcrcl}
% $V_1 \cle V_2$ & \! {iff} \! & $\forall t: V_1(t) \le V_2(t)$ &  & $V_1 \mx V_2$ & = & $\lambda t: \mathit{max}(V_1(t), V_2(t))$\\
% $V[n/u]$ & = & $\lambda t: \mathtt{if}\; (t = u)\; \mathtt{then}\; n\; \mathtt{else}\; V(t)$ & & $\bot$ & = & $\lambda t: 0$\\
% \end{tabular}
%
% \vspace*{.2\baselineskip} 
\begin{align*}
\begin{array}{rclr}
V_1 \cle V_2 & \text{iff} & \forall t: V_1(t) \le V_2(t) & \text{(Point-wise Comparison)}\\
V_1 \mx V_2 & = & \lambda t: \mathit{max}(V_1(t), V_2(t))  & \text{(VC-Join)}\\
V[n/u] & = & \lambda t: \mathtt{if}\; (t = u)\; \mathtt{then}\; n\; \mathtt{else}\; V(t) & \text{(VC-Update)}\\
\bot & = & \lambda t: 0 & \text{(VC-Bottom)}
\end{array}
\end{align*}
% \noindent 
Vector \emph{clocks} are place holders for vector timestamps, or
variables whose domain is the space of vector times.
All the above operations, therefore, also apply to vector clocks.
The algorithms described next maintain a state comprising of
several vector  clocks, whose values, at specific instants, 
will be used to assign timestamps to events.
We will use double struck font ($\Cc$, $\Ll$, $\Rr$, etc.,) for vector clocks 
and normal font ($C$, $R$, etc.,) for vector times .

%!TEX root = main.tex

\subsection{Vector Clock Algorithm for Detecting Schedulable Races}


Algorithm~\ref{algo:vc} depicts the vector clock algorithm for
detecting $\hb{}$-schedulable races using the $\shb{}$ partial order.
Similar to the vector clock algorithm for detecting HB races,
Algorithm~\ref{algo:vc} maintains a state comprising of several vector
clocks.  The idea behind Algorithm~\ref{algo:vc} is to use these
vector clocks to \emph{assign a vector timestamp} to each event $e$
(denoted by $C_e$) such that the ordering relation on the assigned
timestamps ($\cle$) enables determining the partial order $\shb{}$ on
events.  This is formalized in Theorem~\ref{thm:isomorphicVC}.  The
algorithm runs in a streaming fashion and processes each event in the
order in which it occurs in the trace.  Depending upon the type of the
observed event, an appropriate handler is invoked.  The formal
parameter $t$ in each of the handlers refers to the thread performing
the event, and the parameters $\lk$, $x$ and $u$ represent the lock
being acquired or released, the memory location being accessed and the
thread being forked or joined, respectively.  The procedure
\texttt{Initialization} assigns the initial values to the vector
clocks in the state.  We next present details of different parts of
the algorithm.

%!TEX root = main.tex
% \begin{Centering}
\SetKwProg{myalg}{procedure}{}{}

\begin{algorithm*}
\begin{multicols}{2}
\SetKwFunction{facq}{acquire}
\SetKwFunction{frel}{release}
\SetKwFunction{fork}{fork}
\SetKwFunction{join}{join}
\SetKwFunction{read}{read}
\SetKwFunction{wr}{write}
\SetKwFunction{init}{Initialization}

\myalg{\init}{
    \lForEach{$t$}{
    $\Cc_t$ := $\bot[1/t]$
    } 
    \lForEach{$\ell$}{
    $\Ll_\ell$ := $\bot$ % $\lambda t: 0$
    } 
    \For{$x \in \vars{}$}{ 
    $\LW_x$ := $\bot$; \\ % $\lambda t: 0$ 
    $\Rr_x$ := $\bot$; \\% $\lambda t: 0$ ; 
    $\Ww_x$ := $\bot$;  % $\lambda t: 0$ ;
    }
}

\myalg{\facq{$t$, $\ell$}}{ %
    % \If{previous event in $t$ was $\rel$, $\wt$ or $\fork$}{
    %     $\Cc_t(t)$ := $\Cc_t(t) + 1$ ;
    % }
    $\Cc_t$ := $\Cc_t \mx \Ll_\ell$ ;
}

\myalg{\frel{$t$, $\ell$}}{
    $\Ll_\ell$ := $\Cc_t$ ; \\
    $\Cc_t(t)$ := $\Cc_t(t) + 1$ ; {\scriptsize \texttt{(* next event *)}}
}

\myalg{\fork{$t$, $u$}}{
    $\Cc_u$ := $\Cc_t[1/u]$ ; \\
    $\Cc_t(t)$ := $\Cc_t(t) + 1$ ; {\scriptsize \texttt{(* next event *)}}
}

\myalg{\join{$t$, $u$}}{
    $\Cc_t$ := $\Cc_t \mx \Cc_u$ ;
}

\myalg{\read{$t$, $x$}}{
    \If{$\neg (\Ww_x \cle \Cc_t)$}{
      declare `\textbf{race with write}';
    }
    $\Cc_t$ := $\Cc_t \mx \LW_x$; \\
    $\Rr_x(t)$ := $\Cc_t(t)$; \\
}

\myalg{\wr{$t$, $x$}}{
    \If{$\neg (\Rr_x \cle \Cc_t)$}{
      declare `\textbf{race with read}'; \\
    }
    \If{$\neg (\Ww_x \cle \Cc_t)$}{
      declare `\textbf{race with write}'; \\
    } 
    $\LW_x$ := $\Cc_t$ ; \\
    $\Ww_x(t)$ := $\Cc_t(t)$; \\
    $\Cc_t(t)$ := $\Cc_t(t) + 1$ ; {\scriptsize \texttt{(* next event *)}}
}
\end{multicols}
\vspace*{0.25cm}
\caption{\textit{Vector Clock for Checking $\shb{}$-schedulable races}}
\label{algo:vc}
\end{algorithm*}



\subsubsection{Vector clocks in the state}
\label{subsec:vc}
The description of each of the vector clocks that are
maintained in the state of Algorithm~\ref{algo:vc} is
as follows:

\begin{enumerate}
\item \textbf{Clocks} $\Cc_t$: For every thread $t$ in the trace
being analyzed, the algorithm maintains a vector clock $\Cc_t$.
At any point during the algorithm, 
let us denote by $C_{e_t}$ the last event performed by thread
$t$ in the trace so far.
Then, the \emph{timestamp} $C_{e_t}$ 
of the event $e_t$ can be obtained from the value of the clock $\Cc_t$ as follows.
If $e_t$ is a read, acquire or a join event, then $C_{e_t} = \Cc_t$, 
otherwise $C_{e_t} = \Cc_t[(c-1)/t]$, where $c = \Cc_t(t)$.
% Loosely speaking, at any point during the algorithm, the clock $\Cc_t$ stores the
% timestamp $C_{e_t}$ of the last event $e_t$ seen in thread $t$. 
% \ucomment{These lines are marked with a $\dagger$ sign in the algorithm.}

\item \textbf{Clocks} $\Ll_\lk$: The algorithm maintains
a vector clock $\Ll_\lk$ for every lock $\lk$ in the trace.
At any point during the algorithm, the clock $\Ll_\lk$ stores the
timestamp $C_{e_\lk}$, where $e_\lk$ is the last event of the form
$e_\lk = \ev{\cdot, \rel(\lk)}$, in the trace seen so far. 
% Intuitively, the clock $\Ll_\lk$ is used as a \emph{channel} 
% to communicate the timestamps of the release events of lock $\lk$.

\item \textbf{Clocks} $\LW_x$: For every memory location $x$
accessed in the trace, the algorithm maintains a clock $\LW_x$ ($\mathbb{L}$ast $\mathbb{W}$rite to $x$)
to store the timestamp $C_{e_x}$, of the last event $e_x$ 
of the form $\ev{\cdot, \wt(x)}$.

\item \textbf{Clocks} $\Rr_x$ and $\Ww_x$: The clocks
$\Rr_x$ and $\Ww_x$ store the read and write \emph{access histories} of
each memory location $x$.
% Formally, for a trace $\tr$ processed by the algorithm,
At any point in the algorithm,
the vector time $R_x$ stored in the 
the $\mathbb{R}$ead access history clock $\Rr_x$ is
such that $\forall t, R_x(t) = C_{e_t^{\rd(x)}}(t)$ where 
$e_t^{\rd(x)}$ is the last event of thread $t$ that reads $x$ in 
the trace seen so far.
Similarly, the vector time $W_x$ stored in the $\mathbb{W}$rite access history
clock $\Ww_x$ is such that $\forall t, W_x(t) = C_{e_t^{\wt(x)}}(t)$ where $e_t^{\wt(x)}$ 
is the last event of thread $t$ that writes to $x$ in 
the trace seen so far.
\end{enumerate}

The clocks $\Cc_t$, $\Ll_\lk$, $\LW_x$ are used to correctly compute the
timestamps of the events, while the access history clocks $\Rr_x$ and $\Ww_x$
are used to detect races.

\subsubsection{Initialization and Clock Updates}

For every thread $t$, the clock $\Cc_t$ 
is initialized to the vector time $\bot[1/t]$.
Each of the clocks $\Ll_\lk$, $\LW_x$, $\Rr_x$ and $\Ww_x$ are initialized to $\bot$. 
This is in accordance with the semantics of these clocks presented 
in Section~\ref{subsec:vc}.

When processing an acquire event $e = \ev{t, \acq(\lk)}$, the algorithm
reads the clock $\Ll_\lk$ and updates the clock $\Cc_t$ 
with $\Cc_t \sqcup \Ll_\lk$ (see Line 9).
This ensures that the timestamp $C_e$ (
which is the value of the clock $\Cc_t$ after executing Line 9) 
is such that $C_{e'} \cle C_e$ for every $\lk$-release event $e' = \ev{t', \rel(\lk)}$
observed in the trace so far.

At a release event $e = \ev{t, \rel(\lk)}$,
the algorithm writes the timestamp
$C_e$ of the current event $e$
to the clock $\Ll_\lk$ (see Line 11).
Notice that $e$ is also
the last release event of lock $\lk$ in the trace seen so far,
and thus, this update correctly maintains the invariant
stated in Section~\ref{subsec:vc}.
This update ensures that any future events that 
acquire the lock $\lk$ can update their timestamps correctly.
The algorithm then increments the local clock $\Cc_t(t)$ (Line 12).
This ensures that if the next event $e'$ in the thread $t$ and the 
next acquire event $f$ of lock $\lk$ satisfy $e' \not\leq_\textsf{SHB} f$,
then the timestamps of these events satisfy $C_{e'} \not\cle C_f$.
This is crucial for the correctness of the algorithm (Theorem~\ref{thm:isomorphicVC}).

The updates performed by the algorithm at a fork (resp. join)
event are similar to the updates performed when observing a release (resp. acquire) event.
The update at Line 14 is equivalent to the update 
$\Cc_u$ := $\Cc_t \sqcup \Cc_u$ and ensures that the
timestamp of each event $e' = \ev{u, \cdot}$ performed by the forked thread $u$
satisfy $C_e \cle C_{e'}$, where $e$ is the current event forking the new thread $u$.
Similarly, the update performed at Line 17 when processing the join
event $e = \ev{t, \join(u)}$ ensures that the timestamp of
each event $e' = \ev{u, \cdot}$ of the joined thread $u$
is such that $C_{e'} \cle C_e$.

At a read event $e = \ev{t, \rd(x)}$, the clock $\Cc_t$ is updated with 
the join $\Cc_t \sqcup \LW_x$ (Line 21).
Recall that $\LW_x$ stores the timestamp of 
the last event that writes to $x$ (or in other words, the event $\lw{}(e)$)
in the trace seen so far.
This ensures that the timestamps $C_e$ and $C_{\lw{}(e)}$
satisfy $C_{\lw{}(e)} \cle C_e$.
In addition, the algorithm also updates the component $\Rr_x(t)$ with the 
local component of the clock
$\Cc_x$ (Line 22) in order to maintain the invariant described in Section~\ref{subsec:vc}.

At a write event $e = \ev{t, \wt(x)}$, the algorithm updates the value of the
last-write clock $\LW_x$ (Line 28) with the timestamp $C_e$ stored in $\Cc_t$.
The component $\Ww_x(t)$ is updated with the value of the local component $\Cc_t(t)$
to ensure the invariant described in Section~\ref{subsec:vc} is maintained correctly.
Finally, similar to the increment after a release event, the local clock is incremented
in Line 30.

\subsubsection{Checking for races}

At a read/write event $e$, the algorithm determines if
there is a conflicting event $e'$ in the trace seen so far such that
$(e', e)$ is an $\hb{}$-schedulable race.
From Theorem~\ref{thm:SHBSoundness} and Theorem~\ref{thm:isomorphicVC}, 
it follows that it is sufficient to check if $C_{e'} \not\cle C_{\ltho{}(e)}$.
However, since the algorithm does not explicitly store the
timestamps of events, we use the access histories $\Rr_x$
and $\Ww_x$ to check for races.
Below we briefly describe these checks.
The formal statement of correctness is presented in Theorem~\ref{thm:correct-races}
and its proof is presented in Appendix~\ref{app:algo}.
We briefly outline the ideas here.

Recall that, for an event $e = \ev{t, \cdot}$ 
if $\ltho{}(e)$ is undefined, the \texttt{Initialization}
procedure ensures that  $C_e = \bot[1/t]$.
In this case, we have $V \not\cle C_e$, for any
vector-timestamp $V$ with non-negative entries
such that $V(t) = 0$, $\bot \cle V$ and $V \neq \bot$. 
Algorithm~\ref{algo:vc} correctly reports a race in 
this case (see Lines 19-20, 24-27).

On the other hand, if $\ltho{}(e)$ is defined,
then the clock $\Cc_t$, at Line 19, 24 or 26, 
is either the timestamp $C_{\ltho{}(e)}$ (if $\ltho{}(e)$
was a read, join or an acquire event) or the timestamp 
$C_{\ltho{}(e)}[(c+1)/t]$, where $c = C_{\ltho{}(e)}(t)$
(if $\ltho{}(e)$
was a write, fork or a release event).
In either case, if the check $\Ww_x \cle \Cc_t$ at Line 19 fails,
then the read event $e$ being processed is correctly declared to
be in race with an earlier conflicting write event.
Similarly, Algorithm~\ref{algo:vc} reports that
a write event $e$ is in race with an earlier read (resp. write)
event based on whether the check on Line 24 (resp. Line 26) fails or not. 

\subsubsection{Correctness and Complexity}

Here, we fix a trace $\tr$.  Recall that, for an event $e$, we say
that $C_e$ is the timestamp assigned by Algorithm~\ref{algo:vc} to
event $e$.  Theorem~\ref{thm:isomorphicVC} asserts that the time
stamps computed by Algorithm~\ref{algo:vc} can be used to determine
the partial order $\shb{\tr}$.

\begin{theorem}
\label{thm:isomorphicVC}
For events $e, e' \in \events{\tr}$ such that $e \trord{\tr} e'$,
$C_e \cle C_{e'}$ iff $e \shb{\tr} e'$
\end{theorem}

Next, we state the correctness of the algorithm.
We say that Algorithm~\ref{algo:vc} reports a race
at an event $e$, if it executes lines 20, 25 or 27
while processing the handler corresponding to $e$.

\begin{theorem}
\label{thm:correct-races}
Let $e$ be a read/write event $e \in \events{\tr}$.
Algorithm~\ref{algo:vc} reports a race at $e$
iff there is an event $e' \in \events{\tr}$ such that $(e', e)$
is an $\hb{\tr}$-schedulable race.
\end{theorem}

The following theorem states that the asymptotic 
time and space requirements for Algorithm~\ref{algo:vc}
are the same as that of the standard HB algorithm.

\begin{theorem}
\label{thm:complexityVC}
For a trace $\tr$ with $n$ events, $T$ threads, $V$ variables, and $L$
locks, Algorithm~\ref{algo:vc} runs in time $O(nT\log n)$ and uses
$O((V+L+T)T\log n)$ space.
\end{theorem}

The proofs of Theorem~\ref{thm:isomorphicVC}, Theorem~\ref{thm:correct-races} and
Theorem~\ref{thm:complexityVC} are presented in Appendix~\ref{app:algo}.

\subsubsection{Differences from the HB algorithm}
While the spirit of Algorithm~\ref{algo:vc} is similar to standard HB
vector clock algorithms (such as \textsc{Djit}$^+$
\cite{Pozniansky:2003:EOD:966049.781529}), it differs from them in the
following ways.  First, we maintain an additional vector clock $\LW_x$
to track the timestamp of the last event that writes to memory
location $x$ (line 28), and use this clock to correctly update $\Cc_t$
(line 21).  This difference is a direct consequence of the additional
ordering edges in the $\shb{}$ partial order---every read event $e$ is
ordered after the event $\lw{}(e)$, unlike $\hb{}$.  Second, the
`local' component of the clock $\Cc_t$ is also incremented after every
write event (line 19), in addition to after a release or a fork event
(in contrast with \textsc{Djit}$^+$).  This is to ensure correctness
in the following scenario.  Let $e, e'$ and $e''$ be events such that
$e = \ev{t, \rd(x)}\in \reads{}$, $e' = \ev{t', \wt(x)} = \lw{}(e)$
($t'$ may be different from $t$), and $e''$ is the next event after
$e'$ in the thread $t'$.  Incrementing the local component of the
clock $\Cc_{t'}$ ensures that the vector timestamps of $e$ and $e''$
are ordered only when $e'' \shb{} e$.  Third, our algorithm remains
sound even beyond the first race, in contrast to \textsc{Djit}$^+$,
which can lead to false positives beyond the first race.% reported.



\subsection{Epoch optimization}
\label{sec:algo_epoch}

The epoch optimization, popularized
by \textsc{FastTrack}~\cite{fasttrack} exploits the insight that
`\textit{the full generality of vector clocks is unnecessary in most
cases}', and can result in significant performance enhancement,
especially when the traces are predominated by read and write events.

An epoch is a pair of an integer $c$ and a thread $t$, denoted by $c@t$.
Intuitively, epoch $c@t$ can be treated as the vector time $\bot[c/t]$.
Thus, in order to compare an epoch $c@t$ with vector time $V$, 
it suffices to compare the $t$-th component of $V$ with $c$.
That is,
% \begin{tabular}{rcl}
% $c@t \cle V$ & iff & $c \leq V(t)$.
% \end{tabular}
\begin{align*}
\begin{array}{rcl}
c@t \cle V & \text{iff} & c \leq V(t).
\end{array}
\end{align*}
%
Therefore, comparison between epochs is less expensive than that
between vector times --- $O(1)$ as opposed to $O((|\threads{\tr}|)$
for full vector times. To exploit this speedup, some vector clocks in
the new algorithm will adaptively store either epochs or vector times.

%!TEX root = main.tex
% \begin{Centering}
\SetKwProg{myalg}{procedure}{}{}

\begin{algorithm*}
\begin{multicols}{2}
\SetKwFunction{facq}{acquire}
\SetKwFunction{frel}{release}
\SetKwFunction{fork}{fork}
\SetKwFunction{join}{join}
\SetKwFunction{read}{read}
\SetKwFunction{wr}{write}
\SetKwFunction{init}{Initialization}

% \myalg{\init}{
%     \lForEach{$t$}{
%     $\Cc_t$ := $\bot[1/t]$
%     } 
%     \lForEach{$\ell$}{
%     $\Ll_\ell$ := $\bot$ % $\lambda t: 0$
%     } 
%     \For{$x \in \vars{}$}{ 
%     $\LW_x$ := $\bot$; \\ % $\lambda t: 0$ 
%     $\Rr_x$ := $0@0$; \\% $\lambda t: 0$ ; 
%     $\Ww_x$ := $0@0$;  % $\lambda t: 0$ ;
%     }
% }

\myalg{\read{$t$, $x$}}{
    \If{$\neg (\Ww_x \cle \Cc_t)$}{
      declare `\textbf{race with write}';
    }
    $\Cc_t$ := $\Cc_t \mx \LW_x$; \\
   \eIf{$\Rr_x$ is an epoch $c@u$}{
		\eIf{$c \leq \Cc_t(u)$}{
			$\Rr_x$ := $\Cc_t(t)@t$;
		}{
			$\Rr_x$ := $\bot[\Cc_t(t)/t][c/u]$;
		}
	}{
		$\Rr_x(t)$ := $\Cc_t(t)$;
	}
}

\myalg{\wr{$t$, $x$}}{
    \If{$\neg (\Rr_x \cle \Cc_t)$}{
      declare `\textbf{race with read}'; \\
    }
    % \If{$\neg (\Ww_x \cle \Cc_t)$}{
    %   declare `\textbf{race with write}'; \\
    % } 
    \eIf{$(\Ww_x \cle \Cc_t)$}{
		$\Ww_x$ := $\Cc_t(t)@t$
   	}{
   		declare `\textbf{race with write}'; \\	
   		\eIf{$\Ww_x$ is an epoch $c@u$}{
			$\Ww_x$ := $\bot[\Cc_t(t)/t][c/u]$;
   		}{
   			$\Ww_x(t)$ := $\Cc_t(t)$;
   		}

   	}
    $\LW_x$ := $\Cc_t$ ; \\
    $\Ww_x(t)$ := $\Cc_t(t)$; \\
    $\Cc_t(t)$ := $\Cc_t(t) + 1$ ; {\scriptsize \texttt{(* next event *)}}
}
\end{multicols}
\vspace*{0.25cm}
\caption{\textit{Epoch Optimization for Algorithm~\ref{algo:vc}}}
\label{algo:epoch}
\end{algorithm*}

Algorithm~\ref{algo:epoch} applies the epoch optimization to
Algorithm~\ref{algo:vc}.  Here, similar to the \fasttrack~algorithm,
we allow clocks $\Rr_x$ and $\Ww_x$ to be adaptive, while other clocks
($\Cc_t$, $\Ll_\lk$ and $\LW_x$) always store vector times.  The
optimization only applies to the $\read$ and $\texttt{write}$ handlers
and thus we omit the other handlers from the description as they are
same as those described in Algorithm~\ref{algo:vc}.  We also omit
the \texttt{Initialization} procedure which only differs in that the
$\Rr_x$ and $\Ww_x$ are initialized to the epoch $0@0$.

Depending upon how these clocks compare with the thread's clock $\Cc_t$,
the clocks switch back and forth between epoch and vector time values:
\begin{itemize}
\item Initially, both $\Rr_x$ and $\Ww_x$ are assigned the epoch $0@0$.
The element $0@0$ can be thought of as the analogue of $\bot$.

\item
The clock $\Ww_x$ is \emph{fully adaptive} --- it can switch back
and forth between vector and epoch times depending upon how it compares with $\Cc_t$. 
Notice that, in the \fasttrack~algorithm proposed in~\cite{fasttrack},
the clock $\Ww_x$ is always an epoch.
The underlying assumption for such a simplification is that all 
the events that write to a given memory location are 
totally ordered with respect to $\hb{}$.
This assumption, however, need not hold beyond the first HB race.
After the first race is encountered, two $\wt(x)$ events 
$e$ and $e'$ may be unordered by both $\hb{}$ and $\shb{}$.
In Algorithm~\ref{algo:epoch}, $\Ww_x$ has an epoch representation
if and only if the last write event $e$ on $x$
is such that $e' \shb{} e$ for every event $e'$
of the form $e' = \ev{\cdot, \wt(x)}$ in the trace seen so far.
When performing a write event $e = \ev{t, \wt(x)}$,
if $\Ww_x$ satisfies $\Ww_x \cle \Cc_t$ (Line 15), 
then the event $e$ is ordered after
all previous $\wt(x)$ events, and thus,
in this case, $\Ww_x$ is converted to an epoch representation 
independent of its original representation (see Line 16).
Otherwise, there are at least two $\wt(x)$ events that are
not ordered by $\shb{}$ and thus $\Ww_x$ becomes a full-fledged vector clock
(Lines 20 and 22).

\item
The clock $\Rr_x$ is only \emph{semi-adaptive} --- we do not switch back to
epoch representation once the clock $\Rr_x$ takes up a vector-time value.
The clock $\Rr_x$ is initialized to be an epoch.
When processing a read event $e = \ev{t, \rd(x)}$, if the algorithm
determines that there is a read event $e' = \ev{t', \rd(x)}$ observed earlier
such that $e' \not\leq_\textsf{SHB} e$, then the clock $\Rr_x$
takes a vector-time representation.
After this point, $\Rr_x$ stays in the vector clock representation forever.
The $\Rr_x$ clock is an epoch only if 
all the reads of $x$ observed are ordered totally by $\shb{}$.
Thus, in order to determine if $\Rr_x$ can be converted back to an epoch representation,
one needs to check if $(\Rr_x \cle \Cc_t)$ every time a read event is processed.
Since this is an expensive additional comparison and because most traces
from real-world examples are dominated by read events, we avoid such a check
and force $\Rr_x$ to be only semi-adaptive. 
This is similar to the \fasttrack~algorithm.
\end{itemize}
% Complete algorithm implementing this optimization can be found in Appendix~\ref{app:epoch}.
% \input{pseudocode2col_epoch}

As with \textsc{FastTrack}, the epoch optimization for $\shb{}$
is sound and does not lead to any loss of precision --- the optimized
algorithm (Algorithm~\ref{algo:epoch}) declares a race at an event $e$ 
iff the corresponding unoptimized algorithm (Algorithm~\ref{algo:vc}) declares a race at $e$.

One must however note that the new clock $\LW_x$
does not have an adaptive representation, and is always
required to be a vector clock. 
One can think of $\LW_x$ to be similar to, say, the clocks $\Ll_\lk$.
These clocks are used to maintain the partial order,
unlike the clocks $\Rr_x$ or $\Ww_x$ which are only
used to check for races.
Thus, one needs the full generality of vector times for $\LW_x$.
