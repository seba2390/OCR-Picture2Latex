%________________________________________________________________________
%\begin{comment}
\begin{figure*}[t]
%
 \begin{subfigure}{.5\textwidth}
   \centering
   \includegraphics[width=0.95\textwidth]{FIGURES/SED_and_limits_gmin1e3_nuLnu}
   \end{subfigure}%
 %
 \begin{subfigure}{.5\textwidth}
   \centering
   \includegraphics[width=0.95\textwidth]{FIGURES/SED_and_limits_gmin1e5_nuLnu}
   \end{subfigure}
 %  
  \caption{
  \textbf{Left}: Spectral energy distribution of the emission produced by electrons injected in the ISM (in red) and JET regions (in blue). A magnetic field $B_{\rm ISM} = 5\mu$G is assumed for the ISM, whereas a range in the value of $B_{\rm JET} = 5, 10, 15, 20$ and $30\mu$G is probed for the JET region. Particle injection is considered up to $t_{\rm inj} = 10^3$~yr for the JET region and from $t_{\rm inj} = 10^3$~yr until $t_{\rm inj} = 20$~kyr for the ISM region (red-dotted lines). Particles are assumed to follow a power-law distribution with a low and a high energy cutoff $\gamma_{\rm min}^{cut} = 10^{3}$ and $\gamma_{\rm max}^{cut} = 10^{8}$, respectively. Green and orange vertical bands denote \textit{Chandra} and MeerKAT frequency coverage. \textbf{Right}: Same as in the left panel but for a particle spectrum with a low-energy cutoff $\gamma_{\rm min}^{cut} = 10^{5}$.
  }
 \label{figure:SEDs}
\end{figure*}
%\end{comment}
%_____________________________________________________________________