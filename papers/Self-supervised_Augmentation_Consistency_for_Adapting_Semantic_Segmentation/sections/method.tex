% !TEX root = ../arxiv.tex

\subsection{Framework overview}

Shown in \cref{fig:model_overview}, our framework comprises a segmentation network, which we intend to adapt to a target domain, and its slowly changing copy updated with a momentum, a \emph{momentum network}.
To perform self-supervised scene adaptation, we first supply a batch of random crops and horizontal flips from a sample image of the target domain to both networks.
For each pixel we average the predictions (\ie~semantic masks) from the momentum network after the appropriate inverse spatial transformation.
We then create a pseudo ground truth by selecting confident pixels from the averaged map using thresholds based on running statistics, which are capable of adapting to individual samples.
Finally, the segmentation network uses stochastic gradient descent to update its parameters \wrt these pseudo labels.

Our approach closely resembles the mean teacher framework \cite{FrenchMF18,tarvainen2017mean} and temporal ensembling \cite{IzmailovPGVW18,LaineA17}.
However, as we will show empirically, the ensembling property itself plays only an auxiliary role.
More importantly, akin to the critic network in reinforcement learning \cite{LillicrapHPHETS15} and the momentum encoder in unsupervised learning \cite{He0WXG20}, our momentum network provides stable targets for self-supervised training of the segmentation network.
This view allows us to focus on the target-generating process, detailed next.

\subsection{Batch construction}
\label{sec:batch}

For each sampled target image, we generate $N$ crops with random scales, flips and locations, but preserving the aspect ratio.
We re-scale the crops as well as the original image to a fixed input resolution $h \times w$
and pass them as the input to the networks.
\cref{fig:batch} demonstrates this process.
Following the noisy student model in image classification \cite{XieLHL20}, the input to the segmentation network additionally
undergoes a photometric augmentation:
we add random colour jitter and smooth the images with a Gaussian filter at random.
The momentum network, on the other hand, receives a `clean' input, \ie~without such augmentations.
This is to encourage model invariance to photometric perturbations.

\subsection{Self-supervision}
\label{sec:selftrain}

\paragraph{Multi-scale fusion.}
We re-project the output masks from the momentum network back to the original image canvas of size $h \times w$, as illustrated in \cref{fig:fusion}.
For each pixel, the overlapping areas average their predictions.
Note that some pixels may lie outside the crops, hence contain the result of a single forward pass with the original image.
We keep these predictions intact.
The merged maps are then used to extract the pseudo masks for self-supervision.

\myparagraph{A short long-tail interlude.}
Handling rare classes (\ie\ classes with only a few training samples) is notoriously difficult in recognition \cite{GuptaDG19}.
For semantic segmentation, we here distinguish between the classes with low image-level (\eg, ``truck'', ``bus'') and pixel-level (\eg, ``traffic light'', ``pole'') frequency.
While generating self-supervision, we take special care of these cases and encourage
\begin{enumerate*}[label=(\roman*), font=\itshape]
  \item lower thresholds for selecting their pseudo labels,
  \item increased contributions to the gradient with a focal loss, and
  \item employ importance sampling.
\end{enumerate*}
We describe these in detail next.

\myparagraph{Sample-based moving threshold.}
Most previous work with self-training employs multi-round training that requires interrupting the training process and re-generating the pseudo labels \cite{Li_2020_ECCV,LiYV19,Mei_2020_ECCV,subhani2020learning,ZouYKW18}.
One of the reasons is the need to re-compute the thresholds for filtering the pseudo labels for supervision, which requires traversing the predictions for the complete target dataset with the model parameters fixed.
In pursuit of our goal of enabling end-to-end training without expert intervention, we take a different approach and compute the thresholds on-the-go.
As the main ingredient, we maintain an \emph{exponentially moving class prior}.
In detail, for each softmax prediction of the momentum network, we first compute a prior estimate of the probability that a pixel in sample $n$ belongs to class $c$ as
\begin{equation}
\chi_{c,n} = \frac{1}{h w} \sum_{i,j} m_{c,n,i,j},
\label{eq:prior}
\end{equation}
where $m_{c,n,:,:}$ is the mask prediction for class $c$ (with resolution $h \times w$).
We keep an exponentially moving average after each training iteration $t$ with a momentum $\gamma_\chi \in [0, 1]$:
\begin{equation}
\chi_c^{t+1} = \gamma_\chi \chi^t_c + (1 - \gamma_\chi) \chi_{c,n}.
\end{equation}
Our sample-based moving threshold $\theta_{c,n}$ takes lower values when the moving prior $\chi_c \approx 0$ (\ie~for long-tail classes), but is bounded from above as $\chi_c \rightarrow 1$.
We define it as
\begin{equation}
\theta_{c,n} = \zeta \big(1 - e^{-\chi_c / \beta}\big) m^\ast_{c,n},
\label{eq:sbrt}
\end{equation}
where $\beta$ and $\zeta$ are hyperparameters and $m^\ast_{c,n}$ is the predicted peak confidence score for class $c$, \ie
\begin{equation}
m^\ast_{c,n} = \max_{i,j} m_{c,n,i,j}.
\label{eq:peak-class-confidence}
\end{equation}
\cref{fig:threshold} plots \cref{eq:sbrt} as a function of the moving class prior $\chi_c$ for a selection of $\beta$.
For predominant classes (\eg, ``road''), the exponential term has nearly no effect; the threshold is static \wrt the peak class confidence, \ie $\theta_{c,n} \approx \zeta m^\ast_{c,n}$.
However, for long-tail classes such that $\chi_c \approx \beta$, the threshold is lower than this upper bound, hence more pixels for these classes are selected for supervision.
To obtain the pseudo labels, we apply the threshold $\theta_{c,n}$ to the peak predictions of the merged output from the momentum network:
\begin{equation}
\hat{m}_{n,i,j} =
\begin{cases}
      c^\ast & m_{c^\ast,n,i,j} > \theta_{c,n} \\
      \text{ignore} & \text{otherwise},
\end{cases}
\label{eq:pseudo}
\end{equation}
where $c^\ast = \argmax_{c} m_{c,n,i,j}$ is the dominant class for that pixel.
Note that the pixels with confidence values lower than the threshold, as well as non-dominant predictions, will be ignored in the self-supervised loss.

\begin{figure}[t]
% GNUPLOT: LaTeX picture with Postscript
\begingroup
  \fontfamily{Helvetica}%
  \selectfont
  \makeatletter
  \providecommand\color[2][]{%
    \GenericError{(gnuplot) \space\space\space\@spaces}{%
      Package color not loaded in conjunction with
      terminal option `colourtext'%
    }{See the gnuplot documentation for explanation.%
    }{Either use 'blacktext' in gnuplot or load the package
      color.sty in LaTeX.}%
    \renewcommand\color[2][]{}%
  }%
  \providecommand\includegraphics[2][]{%
    \GenericError{(gnuplot) \space\space\space\@spaces}{%
      Package graphicx or graphics not loaded%
    }{See the gnuplot documentation for explanation.%
    }{The gnuplot epslatex terminal needs graphicx.sty or graphics.sty.}%
    \renewcommand\includegraphics[2][]{}%
  }%
  \providecommand\rotatebox[2]{#2}%
  \@ifundefined{ifGPcolor}{%
    \newif\ifGPcolor
    \GPcolortrue
  }{}%
  \@ifundefined{ifGPblacktext}{%
    \newif\ifGPblacktext
    \GPblacktexttrue
  }{}%
  % define a \g@addto@macro without @ in the name:
  \let\gplgaddtomacro\g@addto@macro
  % define empty templates for all commands taking text:
  \gdef\gplbacktext{}%
  \gdef\gplfronttext{}%
  \makeatother
  \ifGPblacktext
    % no textcolor at all
    \def\colorrgb#1{}%
    \def\colorgray#1{}%
  \else
    % gray or color?
    \ifGPcolor
      \def\colorrgb#1{\color[rgb]{#1}}%
      \def\colorgray#1{\color[gray]{#1}}%
      \expandafter\def\csname LTw\endcsname{\color{white}}%
      \expandafter\def\csname LTb\endcsname{\color{black}}%
      \expandafter\def\csname LTa\endcsname{\color{black}}%
      \expandafter\def\csname LT0\endcsname{\color[rgb]{1,0,0}}%
      \expandafter\def\csname LT1\endcsname{\color[rgb]{0,1,0}}%
      \expandafter\def\csname LT2\endcsname{\color[rgb]{0,0,1}}%
      \expandafter\def\csname LT3\endcsname{\color[rgb]{1,0,1}}%
      \expandafter\def\csname LT4\endcsname{\color[rgb]{0,1,1}}%
      \expandafter\def\csname LT5\endcsname{\color[rgb]{1,1,0}}%
      \expandafter\def\csname LT6\endcsname{\color[rgb]{0,0,0}}%
      \expandafter\def\csname LT7\endcsname{\color[rgb]{1,0.3,0}}%
      \expandafter\def\csname LT8\endcsname{\color[rgb]{0.5,0.5,0.5}}%
    \else
      % gray
      \def\colorrgb#1{\color{black}}%
      \def\colorgray#1{\color[gray]{#1}}%
      \expandafter\def\csname LTw\endcsname{\color{white}}%
      \expandafter\def\csname LTb\endcsname{\color{black}}%
      \expandafter\def\csname LTa\endcsname{\color{black}}%
      \expandafter\def\csname LT0\endcsname{\color{black}}%
      \expandafter\def\csname LT1\endcsname{\color{black}}%
      \expandafter\def\csname LT2\endcsname{\color{black}}%
      \expandafter\def\csname LT3\endcsname{\color{black}}%
      \expandafter\def\csname LT4\endcsname{\color{black}}%
      \expandafter\def\csname LT5\endcsname{\color{black}}%
      \expandafter\def\csname LT6\endcsname{\color{black}}%
      \expandafter\def\csname LT7\endcsname{\color{black}}%
      \expandafter\def\csname LT8\endcsname{\color{black}}%
    \fi
  \fi
    \setlength{\unitlength}{0.0500bp}%
    \ifx\gptboxheight\undefined%
      \newlength{\gptboxheight}%
      \newlength{\gptboxwidth}%
      \newsavebox{\gptboxtext}%
    \fi%
    \setlength{\fboxrule}{0.5pt}%
    \setlength{\fboxsep}{1pt}%
\begin{picture}(4608.00,2160.00)%
  \scriptsize
    \gplgaddtomacro\gplbacktext{%
      \csname LTb\endcsname%%
      \put(378,280){\makebox(0,0)[r]{\strut{}$0$}}%
      \put(378,596){\makebox(0,0)[r]{\strut{}$0.2$}}%
      \put(378,912){\makebox(0,0)[r]{\strut{}$0.4$}}%
      \put(378,1229){\makebox(0,0)[r]{\strut{}$0.6$}}%
      \put(378,1550){\makebox(0,0)[r]{\strut{}$0.8$}}%
      \put(378,1861){\makebox(0,0)[r]{\strut{}$1$}}%
      \put(378,2100){\makebox(0,0)[r]{\strut{}$\theta_{c,n}$}}%
      \put(462,140){\makebox(0,0){\strut{}$0$}}%
      \put(1241,140){\makebox(0,0){\strut{}$0.01$}}%
      \put(2019,140){\makebox(0,0){\strut{}$0.02$}}%
      \put(2798,140){\makebox(0,0){\strut{}$0.03$}}%
      \put(3576,140){\makebox(0,0){\strut{}$0.04$}}%
      \put(4355,140){\makebox(0,0){\strut{}$\chi_c$}}%
    }%
    \gplgaddtomacro\gplfronttext{%
      \put(700,1450){\makebox(0,0){\colorbox{white}{$\zeta$}}}%
      \put(800,1861){\makebox(0,0){\colorbox{white}{$m^\ast_{c,n}$}}}%
      \put(1400,1100){\makebox(0,0){\colorbox{white}{$\beta=0.01$}}}%
      \put(1200,1300){\makebox(0,0){\colorbox{white}{$\beta=0.005$}}}%
      \put(1600,900){\makebox(0,0){\colorbox{white}{$\beta=0.02$}}}%
    }%
    \gplbacktext
    \put(0,0){\includegraphics{figures/threshold/threshold_figure}}%
    \gplfronttext
  \end{picture}%
\endgroup

\caption{\textbf{Sample-based moving threshold.} Our thresholding scheme has two hyperparameters, $\zeta$ and $\beta$. In this example, $m^\ast_{c,n} = 1$ and $\zeta=0.75$. Predominant classes (\eg, ``road'') have $\chi_c \gg 0$, hence their threshold approximates $\zeta m^\ast_{c,n}$. Long-tail classes (\eg, ``traffic light'') have $\chi_c \approx 0$ and their thresholds are further reduced with a steepness controlled by $\beta$ (see Eq.~\ref{eq:sbrt}).}
\label{fig:threshold}
%\vspace{-0.25em}
\end{figure}

\myparagraph{Focal loss with confidence regularisation.}
Our loss function incorporates a focal multiplier \cite{LinGGHD20} to further increase the contribution of the long-tail classes in the gradient signal.
Unlike previous work \cite{LinGGHD20,subhani2020learning}, however, our moving class prior $\chi_c$ regulates the focal term:
\begin{equation}
\mathcal{L}^t_{n}(\bar{m}, m \mid \phi) = -m_{c^\ast\!,n} (1 - \chi_{c^\ast})^\lambda \log(\bar{m}_{c^\ast\!,n}),
\label{eq:loss}
\end{equation}
where $\bar{m}$ is the prediction of the segmentation network with parameters $\phi$, the pseudo label $c^\ast$ derives from $\hat{m}$ in \cref{eq:pseudo} and $\lambda$ is a hyperparameter of the focal term.
Recall that low values of $\chi_c$ signify a long-tail category, hence should have a higher weight.
High values of $\lambda$ (\ie~$> 1$) increase the relative weighting on the long-tail classes, while setting $\lambda = 0$ disables the focal term.
Note that we also regularise our loss with the confidence value of the momentum network, $m_{c^\ast,n}$ (\Eq~\ref{eq:peak-class-confidence}).
In case of an incorrect pseudo label, we expect this confidence to be low and to regularise the training owing to its calibration with the multi-scale fusion.
We minimise the loss in \cref{eq:loss}, applied for each pixel, \wrt $\phi$.



\begin{figure*}[t]
\begin{subfigure}{.2\linewidth}
  \centering
    \includegraphics[width=0.98\linewidth]{figures/example/input/001.png}\\
    \includegraphics[width=0.98\linewidth]{figures/example/input/002.png}\\
    \includegraphics[width=0.98\linewidth]{figures/example/input/003.png}\\
    \includegraphics[width=0.98\linewidth]{figures/example/input/004.png}
  \caption{\scriptsize Input}
  \label{fig:ex_fusion_input}
\end{subfigure}%
\begin{subfigure}{.2\linewidth}
  \centering
    \includegraphics[width=0.98\linewidth]{figures/example/pred_seg/001.png}\\
    \includegraphics[width=0.98\linewidth]{figures/example/pred_seg/002.png}\\
    \includegraphics[width=0.98\linewidth]{figures/example/pred_seg/003.png}\\
    \includegraphics[width=0.98\linewidth]{figures/example/pred_seg/004.png}
  \caption{\scriptsize Segmentation net output}
  \label{fig:ex_fusion_seg}
\end{subfigure}%
\begin{subfigure}{.2\linewidth}
  \centering
    \includegraphics[width=0.98\linewidth]{figures/example/pred_mom/001.png}\\
    \includegraphics[width=0.98\linewidth]{figures/example/pred_mom/002.png}\\
    \includegraphics[width=0.98\linewidth]{figures/example/pred_mom/003.png}\\
    \includegraphics[width=0.98\linewidth]{figures/example/pred_mom/004.png}
  \caption{\scriptsize Momentum net output}
  \label{fig:ex_fusion_mom}
\end{subfigure}%
\begin{subfigure}{.2\linewidth}
  \centering
    \includegraphics[width=0.98\linewidth]{figures/example/merged/001.png}\\
    \includegraphics[width=0.98\linewidth]{figures/example/merged/002.png}\\
    \includegraphics[width=0.98\linewidth]{figures/example/merged/003.png}\\
    \includegraphics[width=0.98\linewidth]{figures/example/merged/004.png}
  \caption{\scriptsize Fused prediction}
  \label{fig:ex_fusion_merged}
\end{subfigure}%
\begin{subfigure}{.2\linewidth}
  \centering
    \includegraphics[width=0.98\linewidth]{figures/example/pseudo/001.png}\\
    \includegraphics[width=0.98\linewidth]{figures/example/pseudo/002.png}\\
    \includegraphics[width=0.98\linewidth]{figures/example/pseudo/003.png}\\
    \includegraphics[width=0.98\linewidth]{figures/example/pseudo/004.png}
  \caption{\scriptsize Pseudo labels}
  \label{fig:ex_fusion_pseudo}
\end{subfigure}
\caption{\textbf{Self-supervision example.} In this image sample \emph{\subref{fig:ex_fusion_input}} and its crops, the segmentation network \emph{\subref{fig:ex_fusion_seg}} tends to mistake the ``motorcycle'' for a ``bicycle''.
The momentum network \emph{\subref{fig:ex_fusion_mom}} improves on this prediction, but may still produce an inconsistent labelling.
Averaging the predictions over multiple scales \emph{\subref{fig:ex_fusion_merged}} corrects this inconsistency, allowing to produce high-precision pseudo labels \emph{\subref{fig:ex_fusion_pseudo}} for self-supervision.}
\label{fig:example_fusion}
\vspace{-0.6em}
\end{figure*}


\subsection{Training}
\label{sec:training}

\paragraph{Pre-training with source-only loss.}
Following \cite{LianDLG19,ZhangQYNL020}, we use Adaptive Batch Normalisation (ABN) \cite{LiWSHL18} to jump-start our model on the segmentation task by minimising the cross-entropy loss on the source data only.
In our experiments, we found it unnecessary to re-compute the mean and the standard deviation only at the end of the training.
Instead, in pre-training we alternate batches of source and target images, but ignore the loss for the latter.
For a target batch, this implies updating the running mean and the standard deviation in the Batch Normalisation (BN) \cite{IoffeS15} layers and leaving the remaining model parameters untouched.

\myparagraph{Importance sampling.}
Our loss function in \cref{eq:loss} accounts for long-tail classes with a high image frequency (\eg, ``traffic light'', ``pole''), and may not be effective for the classes appearing in only few samples (\eg, ``bus'', ``train'').
To alleviate this imbalance, we use importance sampling \cite{DoucetFG01} and increase the sample frequency of these long-tail classes.
We minimise the \emph{expected} target loss by re-sampling the target images using the density $p_t$:
\begin{equation}
\underset{\phi}{\min} \: \mathbb{E}_{n \sim p_t} \big[\mathcal{L}^t_n (\phi) \big].
\label{eq:is_obj}
\end{equation}
To obtain $p_t$, we use our pre-trained segmentation network and pre-compute $\chi_{c,n}$, the class prior estimate, for each image $n$ using \cref{eq:prior}.
At training time, we \textit{(i)} sample a semantic class $c$ uniformly, and then \textit{(ii)} obtain a target sample $l$ with probability 
\begin{equation}
\hat{\chi}_{c,l} = \frac{\chi_{c,l}}{\sum_n \chi_{c,n}}.
\end{equation}
This two-step sampling process ensures that all images have non-zero sample probability owing to the prevalent classes for which $\hat{\chi}_{c,l} > 0$ for all $l$ (\eg, ``road'' in urban scenes).

\myparagraph{Joint target-source training.}
We train the segmentation network with stochastic gradient descent using the cross-entropy loss for the source and our focal loss for the target data sampled from $p_t$, as defined by \cref{eq:loss,eq:is_obj}.
\cref{fig:example_fusion} illustrates the synthesis of pseudo labels.
We periodically update the parameters $\psi$ of the momentum network as
\begin{equation}
\psi_{t+1} = \gamma_\psi \psi_t + (1 - \gamma_\psi) \phi,
\end{equation}
where $\phi$ are the parameters of the segmentation network.
$\gamma_\psi$ regulates the pace of the updates: low values result in faster, but unstable training,
while high $\gamma_\psi$ leads to a premature and suboptimal convergence.
We keep $\gamma_\psi$ moderate, but update the momentum network only every $T$ iterations.
