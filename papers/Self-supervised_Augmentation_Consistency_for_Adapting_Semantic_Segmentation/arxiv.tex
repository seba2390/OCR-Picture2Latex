\documentclass[final]{cvpr}

\usepackage{titling}
\usepackage{cvprtitle} % redefine cvpr title style
\usepackage{times}
\usepackage{epsfig}
\usepackage{graphicx}
\usepackage{amsmath}
\usepackage{amssymb}
\usepackage{bm}
\usepackage{multibib}


% Include other packages here, before hyperref.
\usepackage{bbding}
\usepackage{tabularx}
\usepackage{booktabs}
\usepackage{multirow}
\usepackage[keeplastbox]{flushend}
\usepackage{newtxtt}
\usepackage{etoolbox,siunitx}
\robustify\bfseries
\sisetup{detect-all = true}

% drawing boxes
% make it less epileptic
\usepackage{color}
\definecolor{mybar}{rgb}{1.0, 0.4, 0.0}
\newcommand\cbar[3][mybar]{\colorbox{#1}{\color{black}\framebox(#2,#3){}}}

\definecolor{mygreen}{rgb}{0.2, 0.7, 0.1}
\definecolor{turquoise}{rgb}{0.173, 0.627, 0.537}


% If you comment hyperref and then uncomment it, you should delete
% egpaper.aux before re-running latex.  (Or just hit 'q' on the first latex
% run, let it finish, and you should be clear).
\usepackage[pagebackref=true,breaklinks=true,colorlinks,citecolor={mygreen},bookmarks=false]{hyperref}


% figure and equations with cref.
\usepackage[capitalize]{cleveref}
\crefname{section}{Sec.}{Section}

\usepackage[format=plain,labelformat=simple,labelsep=period,font=small,skip=4pt,compatibility=false]{caption}
\usepackage[font=footnotesize,skip=2pt,subrefformat=parens]{subcaption}

% \cmark & \xmark
\usepackage{pifont} % http://ctan.org/pkg/pifont
\newcommand{\cmark}{\ding{51}}%
\newcommand{\xmark}{\ding{55}}%

\usepackage[inline]{enumitem}
\usepackage{listings}

\definecolor{codegreen}{rgb}{0,0.6,0}
\definecolor{codegray}{rgb}{0.5,0.5,0.5}
\definecolor{codepurple}{rgb}{0.58,0,0.82}
\definecolor{backcolour}{rgb}{0.95,0.95,0.92}

\lstdefinestyle{mystyle}{
    backgroundcolor=\color{backcolour},
    commentstyle=\color{codegreen},
    keywordstyle=\color{magenta},
    numberstyle=\tiny\color{codegray},
    stringstyle=\color{codepurple},
    basicstyle=\ttfamily\footnotesize,
    breakatwhitespace=false,
    breaklines=true,
    captionpos=b,
    keepspaces=true,
    numbers=left,
    numbersep=5pt,
    showspaces=false,
    showstringspaces=false,
    showtabs=false,
    tabsize=2
}

\lstset{style=mystyle}
%
% Commands
%

% abbreviations
\usepackage{xspace}
\newcommand{\iid}{\emph{i.\thinspace{}i.\thinspace{}d.}\@\xspace}
\renewcommand{\cf}{\emph{cf.}\@\xspace}
%\newcommand{\etc}{\emph{etc}}
\newcommand{\Eq}{Eq.\@\xspace}
\newcommand{\Eqs}{Eqs.\@\xspace}
\newcommand{\Fig}{Fig.\@\xspace}
\newcommand{\Figs}{Figs.\@\xspace}
\newcommand{\Tab}{Tab.\@\xspace}
\newcommand{\Tabs}{Tabs.\@\xspace}
\newcommand{\Sec}{Sec.\@\xspace}
\newcommand{\Secs}{Secs.\@\xspace}
\newcommand{\Def}{Def.\@\xspace}
\newcommand{\resp}{resp.\@\xspace}

\newcommand{\norm}[1]{\left\lVert#1\right\rVert}
\newcommand*{\myparagraph}[1]{\smallskip\noindent\textbf{#1}\hspace{0.5em}}

\DeclareMathOperator*{\argmin}{arg\,min}
\DeclareMathOperator*{\argmax}{arg\,max}

% footnote w/o a marker
\newcommand\blfootnote[1]{%
  \begingroup
  \renewcommand\thefootnote{}\footnote{#1}%
  \addtocounter{footnote}{-1}%
  \endgroup
}

% Custom footer for title page
\usepackage{fancyhdr}
\usepackage{setspace}
\renewcommand{\headrulewidth}{0pt}
\renewcommand{\footrulewidth}{0pt}
\fancyhf{}
\lfoot{{\footnotesize
\begin{spacing}{.5}
\parbox{\linewidth}{\vspace{2.5em}%
To appear in Proceedings of the \emph{IEEE/CVF Conference on Computer Vision and Pattern Recognition (CVPR)}, virtual, 2021. \\ \hrule \vspace {\baselineskip}
\copyright~2021 IEEE. Personal use of this material is permitted. Permission from IEEE must be obtained for all other uses, in any current or future media, including reprinting/republishing this material for advertising or promotional purposes, creating new collective works, for resale or redistribution to servers or lists, or reuse of any copyrighted component of this work in other works.
}\end{spacing}}}

\newcites{supp}{References}

\begin{document}

%%%%%%%%% TITLE
\title{Self-supervised Augmentation Consistency \\for Adapting Semantic Segmentation}

\author{Nikita Araslanov$^1$ \hspace{1cm} Stefan Roth$^{1,2}$\\
$\ ^1$Department of Computer Science, TU Darmstadt \hspace{1cm} $\ ^2$ hessian.AI}

\maketitle
\thispagestyle{fancy}

%%%%%%%%% ABSTRACT
\begin{abstract}
  In this paper, we explore the connection between secret key agreement and secure omniscience within the setting of the multiterminal source model with a wiretapper who has side information. While the secret key agreement problem considers the generation of a maximum-rate secret key through public discussion, the secure omniscience problem is concerned with communication protocols for omniscience that minimize the rate of information leakage to the wiretapper. The starting point of our work is a lower bound on the minimum leakage rate for omniscience, $\rl$, in terms of the wiretap secret key capacity, $\wskc$. Our interest is in identifying broad classes of sources for which this lower bound is met with equality, in which case we say that there is a duality between secure omniscience and secret key agreement. We show that this duality holds in the case of certain finite linear source (FLS) models, such as two-terminal FLS models and pairwise independent network models on trees with a linear wiretapper. Duality also holds for any FLS model in which $\wskc$ is achieved by a perfect linear secret key agreement scheme. We conjecture that the duality in fact holds unconditionally for any FLS model. On the negative side, we give an example of a (non-FLS) source model for which duality does not hold if we limit ourselves to communication-for-omniscience protocols with at most two (interactive) communications.  We also address the secure function computation problem and explore the connection between the minimum leakage rate for computing a function and the wiretap secret key capacity.
  
%   Finally, we demonstrate the usefulness of our lower bound on $\rl$ by using it to derive equivalent conditions for the positivity of $\wskc$ in the multiterminal model. This extends a recent result of Gohari, G\"{u}nl\"{u} and Kramer (2020) obtained for the two-user setting.
  
   
%   In this paper, we study the problem of secret key generation through an omniscience achieving communication that minimizes the 
%   leakage rate to a wiretapper who has side information in the setting of multiterminal source model.  We explore this problem by deriving a lower bound on the wiretap secret key capacity $\wskc$ in terms of the minimum leakage rate for omniscience, $\rl$. 
%   %The former quantity is defined to be the maximum secret key rate achievable, and the latter one is defined as the minimum possible leakage rate about the source through an omniscience scheme to a wiretapper. 
%   The main focus of our work is the characterization of the sources for which the lower bound holds with equality \textemdash it is referred to as a duality between secure omniscience and wiretap secret key agreement. For general source models, we show that duality need not hold if we limit to the communication protocols with at most two (interactive) communications. In the case when there is no restriction on the number of communications, whether the duality holds or not is still unknown. However, we resolve this question affirmatively for two-user finite linear sources (FLS) and pairwise independent networks (PIN) defined on trees, a subclass of FLS. Moreover, for these sources, we give a single-letter expression for $\wskc$. Furthermore, in the direction of proving the conjecture that duality holds for all FLS, we show that if $\wskc$ is achieved by a \emph{perfect} secret key agreement scheme for FLS then the duality must hold. All these results mount up the evidence in favor of the conjecture on FLS. Moreover, we demonstrate the usefulness of our lower bound on $\wskc$ in terms of $\rl$ by deriving some equivalent conditions on the positivity of secret key capacity for multiterminal source model. Our result indeed extends the work of Gohari, G\"{u}nl\"{u} and Kramer in two-user case.
\end{abstract}

%%%%%%%%% BODY TEXT
\section{Introduction}
\label{sec:intro}
% !TEX root = ../arxiv.tex

Unsupervised domain adaptation (UDA) is a variant of semi-supervised learning \cite{blum1998combining}, where the available unlabelled data comes from a different distribution than the annotated dataset \cite{Ben-DavidBCP06}.
A case in point is to exploit synthetic data, where annotation is more accessible compared to the costly labelling of real-world images \cite{RichterVRK16,RosSMVL16}.
Along with some success in addressing UDA for semantic segmentation \cite{TsaiHSS0C18,VuJBCP19,0001S20,ZouYKW18}, the developed methods are growing increasingly sophisticated and often combine style transfer networks, adversarial training or network ensembles \cite{KimB20a,LiYV19,TsaiSSC19,Yang_2020_ECCV}.
This increase in model complexity impedes reproducibility, potentially slowing further progress.

In this work, we propose a UDA framework reaching state-of-the-art segmentation accuracy (measured by the Intersection-over-Union, IoU) without incurring substantial training efforts.
Toward this goal, we adopt a simple semi-supervised approach, \emph{self-training} \cite{ChenWB11,lee2013pseudo,ZouYKW18}, used in recent works only in conjunction with adversarial training or network ensembles \cite{ChoiKK19,KimB20a,Mei_2020_ECCV,Wang_2020_ECCV,0001S20,Zheng_2020_IJCV,ZhengY20}.
By contrast, we use self-training \emph{standalone}.
Compared to previous self-training methods \cite{ChenLCCCZAS20,Li_2020_ECCV,subhani2020learning,ZouYKW18,ZouYLKW19}, our approach also sidesteps the inconvenience of multiple training rounds, as they often require expert intervention between consecutive rounds.
We train our model using co-evolving pseudo labels end-to-end without such need.

\begin{figure}[t]%
    \centering
    \def\svgwidth{\linewidth}
    \input{figures/preview/bars.pdf_tex}
    \caption{\textbf{Results preview.} Unlike much recent work that combines multiple training paradigms, such as adversarial training and style transfer, our approach retains the modest single-round training complexity of self-training, yet improves the state of the art for adapting semantic segmentation by a significant margin.}
    \label{fig:preview}
\end{figure}

Our method leverages the ubiquitous \emph{data augmentation} techniques from fully supervised learning \cite{deeplabv3plus2018,ZhaoSQWJ17}: photometric jitter, flipping and multi-scale cropping.
We enforce \emph{consistency} of the semantic maps produced by the model across these image perturbations.
The following assumption formalises the key premise:

\myparagraph{Assumption 1.}
Let $f: \mathcal{I} \rightarrow \mathcal{M}$ represent a pixelwise mapping from images $\mathcal{I}$ to semantic output $\mathcal{M}$.
Denote $\rho_{\bm{\epsilon}}: \mathcal{I} \rightarrow \mathcal{I}$ a photometric image transform and, similarly, $\tau_{\bm{\epsilon}'}: \mathcal{I} \rightarrow \mathcal{I}$ a spatial similarity transformation, where $\bm{\epsilon},\bm{\epsilon}'\sim p(\cdot)$ are control variables following some pre-defined density (\eg, $p \equiv \mathcal{N}(0, 1)$).
Then, for any image $I \in \mathcal{I}$, $f$ is \emph{invariant} under $\rho_{\bm{\epsilon}}$ and \emph{equivariant} under $\tau_{\bm{\epsilon}'}$, \ie~$f(\rho_{\bm{\epsilon}}(I)) = f(I)$ and $f(\tau_{\bm{\epsilon}'}(I)) = \tau_{\bm{\epsilon}'}(f(I))$.

\smallskip
\noindent Next, we introduce a training framework using a \emph{momentum network} -- a slowly advancing copy of the original model.
The momentum network provides stable, yet recent targets for model updates, as opposed to the fixed supervision in model distillation \cite{Chen0G18,Zheng_2020_IJCV,ZhengY20}.
We also re-visit the problem of long-tail recognition in the context of generating pseudo labels for self-supervision.
In particular, we maintain an \emph{exponentially moving class prior} used to discount the confidence thresholds for those classes with few samples and increase their relative contribution to the training loss.
Our framework is simple to train, adds moderate computational overhead compared to a fully supervised setup, yet sets a new state of the art on established benchmarks (\cf \cref{fig:preview}).


\pagestyle{plain}
\section{Related Work}
\section{Related Work}
%\mz{We lack a comparison to this paper: https://arxiv.org/abs/2305.14877}
%\anirudh{refine to be more on-topic?}
\iffalse
\paragraph{In-Context Learning} As language models have scaled, the ability to learn in-context, without any weight updates, has emerged. \cite{brown2020language}. While other families of large language models have emerged, in-context learning remains ubiquitous \cite{llama, bloom, gptneo, opt}. Although such as HELM \cite{helm} have arisen for systematic evaluation of \emph{models}, there is no systematic framework to our knowledge for evaluating \emph{prompting methods}, and validating prompt engineering heuristics. The test-suite we propose will ensure that progress in the field of prompt-engineering is structured and objectively evaluated. 

\paragraph{Prompt Engineering Methods} Researchers are interested in the automatic design of high performing instructions for downstream tasks. Some focus on simple heuristics, such as selecting instructions that have the lowest perplexity \cite{lowperplexityprompts}. Other methods try to use large language models to induce an instruction when provided with a few input-output pairs \cite{ape}. Researchers have also used RL objectives to create discrete token sequences that can serve as instructions \cite{rlprompt}. Since the datasets and models used in these works have very little intersection, it is impossible to compare these methods objectively and glean insights. In our work, we evaluate these three methods on a diverse set of tasks and models, and analyze their relative performance. Additionally, we recognize that there are many other interesting angles of prompting that are not covered by instruction engineering \cite{weichain, react, selfconsistency}, but we leave these to future work.

\paragraph{Analysis of Prompting Methods} While most prompt engineering methods focus on accuracy, there are many other interesting dimensions of performance as well. For instance, researchers have found that for most tasks, the selection of demonstrations plays a large role in few-shot accuracy \cite{whatmakesgoodicexamples, selectionmachinetranslation, knnprompting}. Additionally, many researchers have found that even permuting the ordering of a fixed set of demonstrations has a significant effect on downstream accuracy \cite{fantasticallyorderedprompts}. Prompts that are sensitive to the permutation of demonstrations have been shown to also have lower accuracies \cite{relationsensitivityaccuracy}. Especially in low-resource domains, which includes the large public usage of in-context learning, these large swings in accuracy make prompting less dependable. In our test-suite we include sensitivity metrics that go beyond accuracy and allow us to find methods that are not only performant but reliable.

\paragraph{Existing Benchmarks} We recognize that other holistic in-context learning benchmarks exist. BigBench is a large benchmark of 204 tasks that are beyond the capabilities of current LLMs. BigBench seeks to evaluate the few-shot abilities of state of the art large language models, focusing on performance metrics such as accuracy \cite{bigbench}. Similarly, HELM is another benchmark for language model in-context learning ability. Rather than only focusing on performance, HELM branches out and considers many other metrics such as robustness and bias \cite{helm}. Both BigBench and HELM focus on ranking different language model, while fix a generic instruction and prompt format. We instead choose to evaluate instruction induction / selection methods over a fixed set of models. We are the first ever evaluation script that compares different prompt-engineering methods head to head. 
\fi

\paragraph{In-Context Learning and Existing Benchmarks} As language models have scaled, in-context learning has emerged as a popular paradigm and remains ubiquitous among several autoregressive LLM families \cite{brown2020language, llama, bloom, gptneo, opt}. Benchmarks like BigBench \cite{bigbench} and HELM \cite{helm} have been created for the holistic evaluation of these models. BigBench focuses on few-shot abilities of state-of-the-art large language models, while HELM extends to consider metrics like robustness and bias. However, these benchmarks focus on evaluating and ranking \emph{language models}, and do not address the systematic evaluation of \emph{prompting methods}. Although contemporary work by \citet{yang2023improving} also aims to perform a similar systematic analysis of prompting methods, they focus on simple probability-based prompt selection while we evaluate a broader range of methods including trivial instruction baselines, curated manually selected instructions, and sophisticated automated instruction selection.

\paragraph{Automated Prompt Engineering Methods} There has been interest in performing automated prompt-engineering for target downstream tasks within ICL. This has led to the exploration of various prompting methods, ranging from simple heuristics such as selecting instructions with the lowest perplexity \cite{lowperplexityprompts}, inducing instructions from large language models using a few annotated input-output pairs \cite{ape}, to utilizing RL objectives to create discrete token sequences as prompts \cite{rlprompt}. However, these works restrict their evaluation to small sets of models and tasks with little intersection, hindering their objective comparison. %\mz{For paragraphs that only have one work in the last line, try to shorten the paragraph to squeeze in context.}

\paragraph{Understanding in-context learning} There has been much recent work attempting to understand the mechanisms that drive in-context learning. Studies have found that the selection of demonstrations included in prompts significantly impacts few-shot accuracy across most tasks \cite{whatmakesgoodicexamples, selectionmachinetranslation, knnprompting}. Works like \cite{fantasticallyorderedprompts} also show that altering the ordering of a fixed set of demonstrations can affect downstream accuracy. Prompts sensitive to demonstration permutation often exhibit lower accuracies \cite{relationsensitivityaccuracy}, making them less reliable, particularly in low-resource domains.

Our work aims to bridge these gaps by systematically evaluating the efficacy of popular instruction selection approaches over a diverse set of tasks and models, facilitating objective comparison. We evaluate these methods not only on accuracy metrics, but also on sensitivity metrics to glean additional insights. We recognize that other facets of prompting not covered by instruction engineering exist \cite{weichain, react, selfconsistency}, and defer these explorations to future work. 

\section{Self-Supervised Augmentation Consistency}









\section{Proposed Approach} \label{sec:method}

Our goal is to create a unified model that maps task representations (e.g., obtained using task2vec~\cite{achille2019task2vec}) to simulation parameters, which are in turn used to render synthetic pre-training datasets for not only tasks that are seen during training, but also novel tasks.
This is a challenging problem, as the number of possible simulation parameter configurations is combinatorially large, making a brute-force approach infeasible when the number of parameters grows. 

\subsection{Overview} 

\cref{fig:controller-approach} shows an overview of our approach. During training, a batch of ``seen'' tasks is provided as input. Their task2vec vector representations are fed as input to \ours, which is a parametric model (shared across all tasks) mapping these downstream task2vecs to simulation parameters, such as lighting direction, amount of blur, background variability, etc.  These parameters are then used by a data generator (in our implementation, built using the Three-D-World platform~\cite{gan2020threedworld}) to generate a dataset of synthetic images. A classifier model then gets pre-trained on these synthetic images, and the backbone is subsequently used for evaluation on specific downstream task. The classifier's accuracy on this task is used as a reward to update \ours's parameters. 
Once trained, \ours can also be used to efficiently predict simulation parameters in {\em one-shot} for ``unseen'' tasks that it has not encountered during training. 


\subsection{\ours Model} 


Let us denote \ours's parameters with $\theta$. Given the task2vec representation of a downstream task $\bs{x} \in \mc{X}$ as input, \ours outputs simulation parameters $a \in \Omega$. The model consists of $M$ output heads, one for each simulation parameter. In the following discussion, just as in our experiments, each simulation parameter is discretized to a few levels to limit the space of possible outputs. Each head outputs a categorical distribution $\pi_i(\bs{x}, \theta) \in \Delta^{k_i}$, where $k_i$ is the number of discrete values for parameter $i \in [M]$, and $\Delta^{k_i}$, a standard $k_i$-simplex. The set of argmax outputs $\nu(\bs{x}, \theta) = \{\nu_i | \nu_i = \argmax_{j \in [k_i]} \pi_{i, j} ~\forall i \in [M]\}$ is the set of simulation parameter values used for synthetic data generation. Subsequently, we drop annotating the dependence of $\pi$ and $\nu$ on $\theta$ and $\bs{x}$ when clear.

\subsection{\ours Training} 


Since Task2Sim aims to maximize downstream accuracy after pre-training, we use this accuracy as the reward in our training optimization\footnote{Note that our rewards depend only on the task2vec input and the output action and do not involve any states, and thus our problem can be considered similar to a stateless-RL or contextual bandits problem \cite{langford2007epoch}.}.
Note that this downstream accuracy is a non-differentiable function of the output simulation parameters (assuming any simulation engine can be used as a black box) and hence direct gradient-based optimization cannot be used to train \ours. Instead, we use REINFORCE~\cite{williams1992simple}, to approximate gradients of downstream task performance with respect to model parameters $\theta$. 

\ours's outputs represent a distribution over ``actions'' corresponding to different values of the set of $M$ simulation parameters. $P(a) = \prod_{i \in [M]} \pi_i(a_i)$ is the probability of picking action $a = [a_i]_{i \in [M]}$, under policy $\pi = [\pi_i]_{i \in [M]}$. Remember that the output $\pi$ is a function of the parameters $\theta$ and the task representation $\bs{x}$. To train the model, we maximize the expected reward under its policy, defined as
\begin{align}
    R = \E_{a \in \Omega}[R(a)] = \sum_{a \in \Omega} P(a) R(a)
\end{align}
where $\Omega$ is the space of all outputs $a$ and $R(a)$ is the reward when parameter values corresponding to action $a$ are chosen. Since reward is the downstream accuracy, $R(a) \in [0, 100]$.  
Using the REINFORCE rule, we have
\begin{align}
    \nabla_{\theta} R 
    &= \E_{a \in \Omega} \left[ (\nabla_{\theta} \log P(a)) R(a) \right] \\
    &= \E_{a \in \Omega} \left[ \left(\sum_{i \in [M]} \nabla_{\theta} \log \pi_i(a_i) \right) R(a) \right]
\end{align}
where the 2nd step comes from linearity of the derivative. In practice, we use a point estimate of the above expectation at a sample $a \sim (\pi + \epsilon)$ ($\epsilon$ being some exploration noise added to the Task2Sim output distribution) with a self-critical baseline following \cite{rennie2017self}:
\begin{align} \label{eq:grad-pt-est}
    \nabla_{\theta} R \approx \left(\sum_{i \in [M]} \nabla_{\theta} \log \pi_i(a_i) \right) \left( R(a) - R(\nu) \right) 
\end{align}
where, as a reminder $\nu$ is the set of the distribution argmax parameter values from the \name{} model heads.

A pseudo-code of our approach is shown in \cref{alg:train}.  Specifically, we update the model parameters $\theta$ using minibatches of tasks sampled from a set of ``seen'' tasks. Similar to \cite{oh2018self}, we also employ self-imitation learning biased towards actions found to have better rewards. This is done by keeping track of the best action encountered in the learning process and using it for additional updates to the model, besides the ones in \cref{ln:update} of \cref{alg:train}. 
Furthermore, we use the test accuracy of a 5-nearest neighbors classifier operating on features generated by the pretrained backbone as a proxy for downstream task performance since it is computationally much faster than other common evaluation criteria used in transfer learning, e.g., linear probing or full-network finetuning. Our experiments demonstrate that this proxy evaluation measure indeed correlates with, and thus, helps in final downstream performance with linear probing or full-network finetuning. 






\begin{algorithm}
\DontPrintSemicolon
 \textbf{Input:} Set of $N$ ``seen'' downstream tasks represented by task2vecs $\mc{T} = \{\bs{x}_i | i \in [N]\}$. \\
 Given initial Task2Sim parameters $\theta_0$ and initial noise level $\epsilon_0$\\
 Initialize $a_{max}^{(i)} | i \in [N]$ the maximum reward action for each seen task \\
 \For{$t \in [T]$}{
 Set noise level $\epsilon = \frac{\epsilon_0}{t} $ \\
 Sample minibatch $\tau$ of size $n$ from $\mc{T}$  \\
 Get \ours output distributions $\pi^{(i)} | i \in [n]$ \\
 Sample outputs $a^{(i)} \sim \pi^{(i)} + \epsilon$ \\
 Get Rewards $R(a^{(i)})$ by generating a synthetic dataset with parameters $a^{(i)}$, pre-training a backbone on it, and getting the 5-NN downstream accuracy using this backbone \\
 Update $a_{max}^{(i)}$ if $R(a^{(i)}) > R(a_{max}^{(i)})$ \\
 Get point estimates of reward gradients $dr^{(i)}$ for each task in minibatch using \cref{eq:grad-pt-est} \\
 $\theta_{t,0} \leftarrow \theta_{t-1} + \frac{\sum_{i \in [n]} dr^{(i)}}{n}$ \label{ln:update} \\
 \For{$j \in [T_{si}]$}{ 
    \tcp{Self Imitation}
    Get reward gradient estimates $dr_{si}^{(i)}$ from \cref{eq:grad-pt-est} for $a \leftarrow a_{max}^{(i)}$ \\
    $\theta_{t, j}  \leftarrow \theta_{t, j-1} + \frac{\sum_{i \in [n]} dr_{si}^{(i)}}{n}$
 }
 $\theta_{t} \leftarrow \theta_{t, T_{si}}$
 }
 \textbf{Output}: Trained model with parameters $\theta_T$. 
 \caption{Training Task2Sim}
 \label{alg:train}  
\end{algorithm}


\section{Experiments}
\label{sec:exp}
In this section we conduct comprehensive experiments to emphasise the effectiveness of DIAL, including evaluations under white-box and black-box settings, robustness to unforeseen adversaries, robustness to unforeseen corruptions, transfer learning, and ablation studies. Finally, we present a new measurement to test the balance between robustness and natural accuracy, which we named $F_1$-robust score. 


\subsection{A case study on SVHN and CIFAR-100}
In the first part of our analysis, we conduct a case study experiment on two benchmark datasets: SVHN \citep{netzer2011reading} and CIFAR-100 \cite{krizhevsky2009learning}. We follow common experiment settings as in \cite{rice2020overfitting, wu2020adversarial}. We used the PreAct ResNet-18 \citep{he2016identity} architecture on which we integrate a domain classification layer. The adversarial training is done using 10-step PGD adversary with perturbation size of 0.031 and a step size of 0.003 for SVHN and 0.007 for CIFAR-100. The batch size is 128, weight decay is $7e^{-4}$ and the model is trained for 100 epochs. For SVHN, the initial learinnig rate is set to 0.01 and decays by a factor of 10 after 55, 75 and 90 iteration. For CIFAR-100, the initial learning rate is set to 0.1 and decays by a factor of 10 after 75 and 90 iterations. 
%We compared DIAL to \cite{madry2017towards} and TRADES \citep{zhang2019theoretically}. 
%The evaluation is done using Auto-Attack~\citep{croce2020reliable}, which is an ensemble of three white-box and one black-box parameter-free attacks, and various $\ell_{\infty}$ adversaries: PGD$^{20}$, PGD$^{100}$, PGD$^{1000}$ and CW$_{\infty}$ with step size of 0.003. 
Results are averaged over 3 restarts while omitting one standard deviation (which is smaller than 0.2\% in all experiments). As can be seen by the results in Tables~\ref{black-and_white-svhn} and \ref{black-and_white-cifar100}, DIAL presents consistent improvement in robustness (e.g., 5.75\% improved robustness on SVHN against AA) compared to the standard AT 
%under variety of attacks 
while also improving the natural accuracy. More results are presented in Appendix \ref{cifar100-svhn-appendix}.


\begin{table}[!ht]
  \caption{Robustness against white-box, black-box attacks and Auto-Attack (AA) on SVHN. Black-box attacks are generated using naturally trained surrogate model. Natural represents the naturally trained (non-adversarial) model.
  %and applied to the best performing robust models.
  }
  \vskip 0.1in
  \label{black-and_white-svhn}
  \centering
  \small
  \begin{tabular}{l@{\hspace{1\tabcolsep}}c@{\hspace{1\tabcolsep}}c@{\hspace{1\tabcolsep}}c@{\hspace{1\tabcolsep}}c@{\hspace{1\tabcolsep}}c@{\hspace{1\tabcolsep}}c@{\hspace{1\tabcolsep}}c@{\hspace{1\tabcolsep}}c@{\hspace{1\tabcolsep}}c@{\hspace{1\tabcolsep}}c}
    \toprule
    & & \multicolumn{4}{c}{White-box} & \multicolumn{4}{c}{Black-Box}  \\
    \cmidrule(r){3-6} 
    \cmidrule(r){7-10}
    Defense Model & Natural & PGD$^{20}$ & PGD$^{100}$  & PGD$^{1000}$  & CW$^{\infty}$ & PGD$^{20}$ & PGD$^{100}$ & PGD$^{1000}$  & CW$^{\infty}$ & AA \\
    \midrule
    NATURAL & 96.85 & 0 & 0 & 0 & 0 & 0 & 0 & 0 & 0 & 0 \\
    \midrule
    AT & 89.90 & 53.23 & 49.45 & 49.23 & 48.25 & 86.44 & 86.28 & 86.18 & 86.42 & 45.25 \\
    % TRADES & 90.35 & 57.10 & 54.13 & 54.08 & 52.19 & 86.89 & 86.73 & 86.57 & 86.70 &  49.50 \\
    $\DIAL_{\kl}$ (Ours) & 90.66 & \textbf{58.91} & \textbf{55.30} & \textbf{55.11} & \textbf{53.67} & 87.62 & 87.52 & 87.41 & 87.63 & \textbf{51.00} \\
    $\DIAL_{\ce}$ (Ours) & \textbf{92.88} & 55.26  & 50.82 & 50.54 & 49.66 & \textbf{89.12} & \textbf{89.01} & \textbf{88.74} & \textbf{89.10} &  46.52  \\
    \bottomrule
  \end{tabular}
\end{table}


\begin{table}[!ht]
  \caption{Robustness against white-box, black-box attacks and Auto-Attack (AA) on CIFAR100. Black-box attacks are generated using naturally trained surrogate model. Natural represents the naturally trained (non-adversarial) model.
  %and applied to the best performing robust models.
  }
  \vskip 0.1in
  \label{black-and_white-cifar100}
  \centering
  \small
  \begin{tabular}{l@{\hspace{1\tabcolsep}}c@{\hspace{1\tabcolsep}}c@{\hspace{1\tabcolsep}}c@{\hspace{1\tabcolsep}}c@{\hspace{1\tabcolsep}}c@{\hspace{1\tabcolsep}}c@{\hspace{1\tabcolsep}}c@{\hspace{1\tabcolsep}}c@{\hspace{1\tabcolsep}}c@{\hspace{1\tabcolsep}}c}
    \toprule
    & & \multicolumn{4}{c}{White-box} & \multicolumn{4}{c}{Black-Box}  \\
    \cmidrule(r){3-6} 
    \cmidrule(r){7-10}
    Defense Model & Natural & PGD$^{20}$ & PGD$^{100}$  & PGD$^{1000}$  & CW$^{\infty}$ & PGD$^{20}$ & PGD$^{100}$ & PGD$^{1000}$  & CW$^{\infty}$ & AA \\
    \midrule
    NATURAL & 79.30 & 0 & 0 & 0 & 0 & 0 & 0 & 0 & 0 & 0 \\
    \midrule
    AT & 56.73 & 29.57 & 28.45 & 28.39 & 26.6 & 55.52 & 55.29 & 55.26 & 55.40 & 24.12 \\
    % TRADES & 58.24 & 30.10 & 29.66 & 29.64 & 25.97 & 57.05 & 56.71 & 56.67 & 56.77 & 24.92 \\
    $\DIAL_{\kl}$ (Ours) & 58.47 & \textbf{31.19} & \textbf{30.50} & \textbf{30.42} & \textbf{26.91} & 57.16 & 56.81 & 56.80 & 57.00 & \textbf{25.87} \\
    $\DIAL_{\ce}$ (Ours) & \textbf{60.77} & 27.87 & 26.66 & 26.61 & 25.98 & \textbf{59.48} & \textbf{59.06} & \textbf{58.96} & \textbf{59.20} & 23.51  \\
    \bottomrule
  \end{tabular}
\end{table}


% \begin{table}[!ht]
%   \caption{Robustness comparison of DIAL to Madry et al. and TRADES defense models on the SVHN dataset under different PGD white-box attacks and the ensemble Auto-Attack (AA).}
%   \label{svhn}
%   \centering
%   \begin{tabular}{llllll|l}
%     \toprule
%     \cmidrule(r){1-5}
%     Defense Model & Natural & PGD$^{20}$ & PGD$^{100}$ & PGD$^{1000}$ & CW$_{\infty}$ & AA\\
%     \midrule
%     $\DIAL_{\kl}$ (Ours) & $\mathbf{90.66}$ & $\mathbf{58.91}$ & $\mathbf{55.30}$ & $\mathbf{55.12}$ & $\mathbf{53.67}$  & $\mathbf{51.00}$  \\
%     Madry et al. & 89.90 & 53.23 & 49.45 & 49.23 & 48.25 & 45.25  \\
%     TRADES & 90.35 & 57.10 & 54.13 & 54.08 & 52.19 & 49.50 \\
%     \bottomrule
%   \end{tabular}
% \end{table}


\subsection{Performance comparison on CIFAR-10} \label{defence-settings}
In this part, we evaluate the performance of DIAL compared to other well-known methods on CIFAR-10. 
%To be consistent with other methods, 
We follow the same experiment setups as in~\cite{madry2017towards, wang2019improving, zhang2019theoretically}. When experiment settings are not identical between tested methods, we choose the most commonly used settings, and apply it to all experiments. This way, we keep the comparison as fair as possible and avoid reporting changes in results which are caused by inconsistent experiment settings \citep{pang2020bag}. To show that our results are not caused because of what is referred to as \textit{obfuscated gradients}~\citep{athalye2018obfuscated}, we evaluate our method with same setup as in our defense model, under strong attacks (e.g., PGD$^{1000}$) in both white-box, black-box settings, Auto-Attack ~\citep{croce2020reliable}, unforeseen "natural" corruptions~\citep{hendrycks2018benchmarking}, and unforeseen adversaries. To make sure that the reported improvements are not caused by \textit{adversarial overfitting}~\citep{rice2020overfitting}, we report best robust results for each method on average of 3 restarts, while omitting one standard deviation (which is smaller than 0.2\% in all experiments). Additional results for CIFAR-10 as well as comprehensive evaluation on MNIST can be found in Appendix \ref{mnist-results} and \ref{additional_res}.
%To further keep the comparison consistent, we followed the same attack settings for all methods.


\begin{table}[ht]
  \caption{Robustness against white-box, black-box attacks and Auto-Attack (AA) on CIFAR-10. Black-box attacks are generated using naturally trained surrogate model. Natural represents the naturally trained (non-adversarial) model.
  %and applied to the best performing robust models.
  }
  \vskip 0.1in
  \label{black-and_white-cifar}
  \centering
  \small
  \begin{tabular}{cccccccc@{\hspace{1\tabcolsep}}c}
    \toprule
    & & \multicolumn{3}{c}{White-box} & \multicolumn{3}{c}{Black-Box} \\
    \cmidrule(r){3-5} 
    \cmidrule(r){6-8}
    Defense Model & Natural & PGD$^{20}$ & PGD$^{100}$ & CW$^{\infty}$ & PGD$^{20}$ & PGD$^{100}$ & CW$^{\infty}$ & AA \\
    \midrule
    NATURAL & 95.43 & 0 & 0 & 0 & 0 & 0 & 0 &  0 \\
    \midrule
    TRADES & 84.92 & 56.60 & 55.56 & 54.20 & 84.08 & 83.89 & 83.91 &  53.08 \\
    MART & 83.62 & 58.12 & 56.48 & 53.09 & 82.82 & 82.52 & 82.80 & 51.10 \\
    AT & 85.10 & 56.28 & 54.46 & 53.99 & 84.22 & 84.14 & 83.92 & 51.52 \\
    ATDA & 76.91 & 43.27 & 41.13 & 41.01 & 75.59 & 75.37 & 75.35 & 40.08\\
    $\DIAL_{\kl}$ (Ours) & 85.25 & $\mathbf{58.43}$ & $\mathbf{56.80}$ & $\mathbf{55.00}$ & 84.30 & 84.18 & 84.05 & \textbf{53.75} \\
    $\DIAL_{\ce}$ (Ours)  & $\mathbf{89.59}$ & 54.31 & 51.67 & 52.04 &$ \mathbf{88.60}$ & $\mathbf{88.39}$ & $\mathbf{88.44}$ & 49.85 \\
    \midrule
    $\DIAL_{\awp}$ (Ours) & $\mathbf{85.91}$ & $\mathbf{61.10}$ & $\mathbf{59.86}$ & $\mathbf{57.67}$ & $\mathbf{85.13}$ & $\mathbf{84.93}$ & $\mathbf{85.03}$  & \textbf{56.78} \\
    $\TRADES_{\awp}$ & 85.36 & 59.27 & 59.12 & 57.07 & 84.58 & 84.58 & 84.59 & 56.17 \\
    \bottomrule
  \end{tabular}
\end{table}



\paragraph{CIFAR-10 setup.} We use the wide residual network (WRN-34-10)~\citep{zagoruyko2016wide} architecture. %used in the experiments of~\cite{madry2017towards, wang2019improving, zhang2019theoretically}. 
Sidelong this architecture, we integrate a domain classification layer. To generate the adversarial domain dataset, we use a perturbation size of $\epsilon=0.031$. We apply 10 of inner maximization iterations with perturbation step size of 0.007. Batch size is set to 128, weight decay is set to $7e^{-4}$, and the model is trained for 100 epochs. Similar to the other methods, the initial learning rate was set to 0.1, and decays by a factor of 10 at iterations 75 and 90. 
%For being consistent with other methods, the natural images are padded with 4-pixel padding with 32-random crop and random horizontal flip. Furthermore, all methods are trained using SGD with momentum 0.9. For $\DIAL_{\kl}$, we balance the robust loss with $\lambda=6$ and the domains loss with $r=4$. For $\DIAL_{\ce}$, we balance the robust loss with $\lambda=1$ and the domains loss with $r=2$. 
%We also introduce a version of our method that incorporates the AWP double-perturbation mechanism, named DIAL-AWP.
%which is trained using the same learning rate schedule used in ~\cite{wu2020adversarial}, where the initial 0.1 learning rate decays by a factor of 10 after 100 and 150 iterations. 
See Appendix \ref{cifar10-additional-setup} for additional details.

\begin{table}[ht]
  \caption{Black-box attack using the adversarially trained surrogate models on CIFAR-10.}
  \vskip 0.1in
  \label{black-box-cifar-adv}
  \centering
  \small
  \begin{tabular}{ll|c}
    \toprule
    \cmidrule(r){1-2}
    Surrogate (source) model & Target model & robustness \% \\
    % \midrule
    \midrule
    TRADES & $\DIAL_{\ce}$ & $\mathbf{67.77}$ \\
    $\DIAL_{\ce}$ & TRADES & 65.75 \\
    \midrule
    MART & $\DIAL_{\ce}$ & $\mathbf{70.30}$ \\
    $\DIAL_{\ce}$ & MART & 64.91 \\
    \midrule
    AT & $\DIAL_{\ce}$ & $\mathbf{65.32}$ \\
    $\DIAL_{\ce}$ & AT  & 63.54 \\
    \midrule
    ATDA & $\DIAL_{\ce}$ & $\mathbf{66.77}$ \\
    $\DIAL_{\ce}$ & ATDA & 52.56 \\
    \bottomrule
  \end{tabular}
\end{table}

\paragraph{White-box/Black-box robustness.} 
%We evaluate all defense models using Auto-Attack, PGD$^{20}$, PGD$^{100}$, PGD$^{1000}$ and CW$_{\infty}$ with step size 0.003. We constrain all attacks by the same perturbation $\epsilon=0.031$. 
As reported in Table~\ref{black-and_white-cifar} and Appendix~\ref{additional_res}, our method achieves better robustness compared to the other methods. Specifically, in the white-box settings, our method improves robustness over~\citet{madry2017towards} and TRADES by 2\% 
%using the common PGD$^{20}$ attack 
while keeping higher natural accuracy. We also observe better natural accuracy of 1.65\% over MART while also achieving better robustness over all attacks. Moreover, our method presents significant improvement of up to 15\% compared to the the domain invariant method suggested by~\citet{song2018improving} (ATDA).
%in both natural and robust accuracy. 
When incorporating 
%the double-perturbation mechanism of 
AWP, our method improves the results of $\TRADES_{\awp}$ by almost 2\%.
%and reaches state-of-the-art results for robust models with no additional data. 
% Additional results are available in Appendix~\ref{additional_res}.
When tested on black-box settings, $\DIAL_{\ce}$ presents a significant improvement of more than 4.4\% over the second-best performing method, and up to 13\%. In Table~\ref{black-box-cifar-adv}, we also present the black-box results when the source model is taken from one of the adversarially trained models. %Then, we compare our model to each one of them both as the source model and target model. 
In addition to the improvement in black-box robustness, $\DIAL_{\ce}$ also manages to achieve better clean accuracy of more than 4.5\% over the second-best performing method.
% Moreover, based on the auto-attack leader-board \footnote{\url{https://github.com/fra31/auto-attack}}, our method achieves the 1st place among models without additional data using the WRN-34-10 architecture.

% \begin{table}
%   \caption{White-box robustness on CIFAR-10 using WRN-34-10}
%   \label{white-box-cifar-10}
%   \centering
%   \begin{tabular}{lllll}
%     \toprule
%     \cmidrule(r){1-2}
%     Defense Model & Natural & PGD$^{20}$ & PGD$^{100}$ & PGD$^{1000}$ \\
%     \midrule
%     TRADES ~\cite{zhang2019theoretically} & 84.92  & 56.6 & 55.56 & 56.43  \\
%     MART ~\cite{wang2019improving} & 83.62  & 58.12 & 56.48 & 56.55  \\
%     Madry et al. ~\cite{madry2017towards} & 85.1  & 56.28 & 54.46 & 54.4  \\
%     Song et al. ~\cite{song2018improving} & 76.91 & 43.27 & 41.13 & 41.02  \\
%     $\DIAL_{\ce}$ (Ours) & $ \mathbf{90}$  & 52.12 & 48.88 & 48.78  \\
%     $\DIAL_{\kl}$ (Ours) & 85.25 & $\mathbf{58.43}$ & $\mathbf{56.8}$ & $\mathbf{56.73}$ \\
%     \midrule
%     $\DIAL_{\kl}$+AWP (Ours) & $\mathbf{85.91}$ & $\mathbf{61.1}$ & - & -  \\
%     TRADES+AWP \cite{wu2020adversarial} & 85.36 & 59.27 & 59.12 & -  \\
%     % MART + AWP & 84.43 & 60.68 & 59.32 & -  \\
%     \bottomrule
%   \end{tabular}
% \end{table}


% \begin{table}
%   \caption{White-box robustness on MNIST}
%   \label{white-box-mnist}
%   \centering
%   \begin{tabular}{llllll}
%     \toprule
%     \cmidrule(r){1-2}
%     Defense Model & Natural & PGD$^{40}$ & PGD$^{100}$ & PGD$^{1000}$ \\
%     \midrule
%     TRADES ~\cite{zhang2019theoretically} & 99.48 & 96.07 & 95.52 & 95.22 \\
%     MART ~\cite{wang2019improving} & 99.38  & 96.99 & 96.11 & 95.74  \\
%     Madry et al. ~\cite{madry2017towards} & 99.41  & 96.01 & 95.49 & 95.36 \\
%     Song et al. ~\cite{song2018improving}  & 98.72 & 96.82 & 96.26 & 96.2  \\
%     $\DIAL_{\kl}$ (Ours) & 99.46 & 97.05 & 96.06 & 95.99  \\
%     $\DIAL_{\ce}$ (Ours) & $\mathbf{99.49}$  & $\mathbf{97.38}$ & $\mathbf{96.45}$ & $\mathbf{96.33}$ \\
%     \bottomrule
%   \end{tabular}
% \end{table}


% \paragraph{Attacking MNIST.} For consistency, we use the same perturbation and step sizes. For MNIST, we use $\epsilon=0.3$ and step size of $0.01$. The natural accuracy of our surrogate (source) model is 99.51\%. Attacks results are reported in Table~\ref{black-and_white-mnist}. It is worth noting that the improvement margin is not conclusive on MNIST as it is on CIFAR-10, which is a more complex task.

% \begin{table}
%   \caption{Black-box robustness on MNIST and CIFAR-10 using naturally trained surrogate model and best performing robust models}
%   \label{black-box-mnist-cifar}
%   \centering
%   \begin{tabular}{lllllll}
%     \toprule
%     & \multicolumn{3}{c}{MNIST} & \multicolumn{3}{c}{CIFAR-10} \\
%     \cmidrule(r){2-4} 
%     \cmidrule(r){5-7}  
%     Defense Model & PGD$^{40}$ & PGD$^{100}$ & PGD$^{1000}$ & PGD$^{20}$ & PGD$^{100}$ & PGD$^{1000}$ \\
%     \midrule
%     TRADES ~\cite{zhang2019theoretically} & 98.12 & 97.86 & 97.81 & 84.08 & 83.89 & 83.8 \\
%     MART ~\cite{wang2019improving} & 98.16 & 97.96 & 97.89  & 82.82 & 82.52 & 82.47 \\
%     Madry et al. ~\cite{madry2017towards}  & 98.05 & 97.73 & 97.78 & 84.22 & 84.14 & 83.96 \\
%     Song et al. ~\cite{song2018improving} & 97.74 & 97.28 & 97.34 & 75.59 & 75.37 & 75.11 \\
%     $\DIAL_{\kl}$ (Ours) & 98.14 & 97.83 & 97.87  & 84.3 & 84.18 & 84.0 \\
%     $\DIAL_{\ce}$ (Ours)  & $\mathbf{98.37}$ & $\mathbf{98.12}$ & $\mathbf{98.05}$  & $\mathbf{89.13}$ & $\mathbf{88.89}$ & $\mathbf{88.78}$ \\
%     \bottomrule
%   \end{tabular}
% \end{table}



% \subsubsection{Ensemble attack} In addition to the white-box and black-box settings, we evaluate our method on the Auto-Attack ~\citep{croce2020reliable} using $\ell_{\infty}$ threat model with perturbation $\epsilon=0.031$. Auto-Attack is an ensemble of parameter-free attacks. It consists of three white-box attacks: APGD-CE which is a step size-free version of PGD on the cross-entropy ~\citep{croce2020reliable}. APGD-DLR which is a step size-free version of PGD on the DLR loss ~\citep{croce2020reliable} and FAB which  minimizes the norm of the adversarial perturbations, and one black-box attack: square attack which is a query-efficient black-box attack ~\citep{andriushchenko2020square}. Results are presented in Table~\ref{auto-attack}. Based on the auto-attack leader-board \footnote{\url{https://github.com/fra31/auto-attack}}, our method achieves the 1st place among models without additional data using the WRN-34-10 architecture.

%Additional results can be found in Appendix ~\ref{additional_res}.

% \begin{table}
%   \caption{Auto-Attack (AA) on CIFAR-10 with perturbation size $\epsilon=0.031$ with $\ell_{\infty}$ threat model}
%   \label{auto-attack}
%   \centering
%   \begin{tabular}{lll}
%     \toprule
%     \cmidrule(r){1-2}
%     Defense Model & AA \\
%     \midrule
%     TRADES ~\cite{zhang2019theoretically} & 53.08  \\
%     MART ~\cite{wang2019improving} & 51.1  \\
%     Madry et al. ~\cite{madry2017towards} & 51.52    \\
%     Song et al.   ~\cite{song2018improving} & 40.18 \\
%     $\DIAL_{\ce}$ (Ours) & 47.33  \\
%     $\DIAL_{\kl}$ (Ours) & $\mathbf{53.75}$ \\
%     \midrule
%     DIAL-AWP (Ours) & $\mathbf{56.78}$ \\
%     TRADES-AWP \cite{wu2020adversarial} & 56.17 \\
%     \bottomrule
%   \end{tabular}
% \end{table}


% \begin{table}[!ht]
%   \caption{Auto-Attack (AA) Robustness (\%) on CIFAR-10 with $\epsilon=0.031$ using an $\ell_{\infty}$ threat model}
%   \label{auto-attack}
%   \centering
%   \begin{tabular}{cccccc|cc}
%     \toprule
%     % \multicolumn{8}{c}{Defence Model}  \\
%     % \cmidrule(r){1-8} 
%     TRADES & MART & Madry & Song & $\DIAL_{\ce}$ & $\DIAL_{\kl}$ & DIAL-AWP  & TRADES-AWP\\
%     \midrule
%     53.08 & 51.10 & 51.52 &  40.08 & 47.33  & $\mathbf{53.75}$ & $\mathbf{56.78}$ & 56.17 \\

%     \bottomrule
%   \end{tabular}
% \end{table}

% \begin{table}[!ht]
% \caption{$F_1$-robust measurement using PGD$^{20}$ attack in white-box and black-box settings on CIFAR-10}
%   \label{f1-robust}
%   \centering
%   \begin{tabular}{ccccccc|cc}
%     \toprule
%     % \multicolumn{8}{c}{Defence Model}  \\
%     % \cmidrule(r){1-8} 
%     Defense Model & TRADES & MART & Madry & Song & $\DIAL_{\kl}$ & $\DIAL_{\ce}$ & DIAL-AWP  & TRADES-AWP\\
%     \midrule
%     White-box & 0.659 & 0.666 & 0.657 & 0.518 & $\mathbf{0.675}$  & 0.643 & $\mathbf{0.698}$ & 0.682 \\
%     Black-box & 0.844 & 0.831 & 0.846 & 0.761 & 0.847 & $\mathbf{0.895}$ & $\mathbf{0.854}$ &  0.849 \\
%     \bottomrule
%   \end{tabular}
% \end{table}

\subsubsection{Robustness to Unforeseen Attacks and Corruptions}
\paragraph{Unforeseen Adversaries.} To further demonstrate the effectiveness of our approach, we test our method against various adversaries that were not used during the training process. We attack the model under the white-box settings with $\ell_{2}$-PGD, $\ell_{1}$-PGD, $\ell_{\infty}$-DeepFool and $\ell_{2}$-DeepFool \citep{moosavi2016deepfool} adversaries using Foolbox \citep{rauber2017foolbox}. We applied commonly used attack budget 
%(perturbation for PGD adversaries and overshot for DeepFool adversaries) 
with 20 and 50 iterations for PGD and DeepFool, respectively.
Results are presented in Table \ref{unseen-attacks}. As can be seen, our approach  gains an improvement of up to 4.73\% over the second best method under the various attack types and an average improvement of 3.7\% over all threat models.


\begin{table}[ht]
  \caption{Robustness on CIFAR-10 against unseen adversaries under white-box settings.}
  \vskip 0.1in
  \label{unseen-attacks}
  \centering
%   \small
  \begin{tabular}{c@{\hspace{1.5\tabcolsep}}c@{\hspace{1.5\tabcolsep}}c@{\hspace{1.5\tabcolsep}}c@{\hspace{1.5\tabcolsep}}c@{\hspace{1.5\tabcolsep}}c@{\hspace{1.5\tabcolsep}}c@{\hspace{1.5\tabcolsep}}c}
    \toprule
    Threat Model & Attack Constraints & $\DIAL_{\kl}$ & $\DIAL_{\ce}$ & AT & TRADES & MART & ATDA \\
    \midrule
    \multirow{2}{*}{$\ell_{2}$-PGD} & $\epsilon=0.5$ & 76.05 & \textbf{80.51} & 76.82 & 76.57 & 75.07 & 66.25 \\
    & $\epsilon=0.25$ & 80.98 & \textbf{85.38} & 81.41 & 81.10 & 80.04 & 71.87 \\\midrule
    \multirow{2}{*}{$\ell_{1}$-PGD} & $\epsilon=12$ & 74.84 & \textbf{80.00} & 76.17 & 75.52 & 75.95 & 65.76 \\
    & $\epsilon=7.84$ & 78.69 & \textbf{83.62} & 79.86 & 79.16 & 78.55 & 69.97 \\
    \midrule
    $\ell_{2}$-DeepFool & overshoot=0.02 & 84.53 & \textbf{88.88} & 84.15 & 84.23 & 82.96 & 76.08 \\\midrule
    $\ell_{\infty}$-DeepFool & overshoot=0.02 & 68.43 & \textbf{69.50} & 67.29 & 67.60 & 66.40 & 57.35 \\
    \bottomrule
  \end{tabular}
\end{table}


%%%%%%%%%%%%%%%%%%%%%%%%% conference version %%%%%%%%%%%%%%%%%%%%%%%%%%%%%%%%%%%%%
\paragraph{Unforeseen Corruptions.}
We further demonstrate that our method consistently holds against unforeseen ``natural'' corruptions, consists of 18 unforeseen diverse corruption types proposed by \citet{hendrycks2018benchmarking} on CIFAR-10, which we refer to as CIFAR10-C. The CIFAR10-C benchmark covers noise, blur, weather, and digital categories. As can be shown in Figure \ref{corruption}, our method gains a significant and consistent improvement over all the other methods. Our method leads to an average improvement of 4.7\% with minimum improvement of 3.5\% and maximum improvement of 5.9\% compared to the second best method over all unforeseen attacks. See Appendix \ref{corruptions-apendix} for the full experiment results.


\begin{figure}[h]
 \centering
  \includegraphics[width=0.4\textwidth]{figures/spider_full.png}
%   \caption{Summary of accuracy over all unforeseen corruptions compared to the second and third best performing methods.}
  \caption{Accuracy comparison over all unforeseen corruptions.}
  \label{corruption}
\end{figure}


%%%%%%%%%%%%%%%%%%%%%%%%% conference version %%%%%%%%%%%%%%%%%%%%%%%%%%%%%%%%%%%%%

%%%%%%%%%%%%%%%%%%%%%%%%% Arxiv version %%%%%%%%%%%%%%%%%%%%%%%%%%%%%%%%%%%%%
% \newpage
% \paragraph{Unforeseen Corruptions.}
% We further demonstrate that our method consistently holds against unforeseen "natural" corruptions, consists of 18 unforeseen diverse corruption types proposed by \cite{hendrycks2018benchmarking} on CIFAR-10, which we refer to as CIFAR10-C. The CIFAR10-C benchmark covers noise, blur, weather, and digital categories. As can be shown in Figure  \ref{spider-full-graph}, our method gains a significant and consistent improvement over all the other methods. Our approach leads to an average improvement of 4.7\% with minimum improvement of 3.5\% and maximum improvement of 5.9\% compared to the second best method over all unforeseen attacks. Full accuracy results against unforeseen corruptions are presented in Tables \ref{corruption-table1} and \ref{corruption-table2}. 

% \begin{table}[!ht]
%   \caption{Accuracy (\%) against unforeseen corruptions.}
%   \label{corruption-table1}
%   \centering
%   \tiny
%   \begin{tabular}{lcccccccccccccccccc}
%     \toprule
%     Defense Model & brightness & defocus blur & fog & glass blur & jpeg compression & motion blur & saturate & snow & speckle noise  \\
%     \midrule
%     TRADES & 82.63 & 80.04 & 60.19 & 78.00 & 82.81 & 76.49 & 81.53 & 80.68 & 80.14 \\
%     MART & 80.76 & 78.62 & 56.78 & 76.60 & 81.26 & 74.58 & 80.74 & 78.22 & 79.42 \\
%     AT &  83.30 & 80.42 & 60.22 & 77.90 & 82.73 & 76.64 & 82.31 & 80.37 & 80.74 \\
%     ATDA & 72.67 & 69.36 & 45.52 & 64.88 & 73.22 & 63.47 & 72.07 & 68.76 & 72.27 \\
%     DIAL (Ours)  & \textbf{87.14} & \textbf{84.84} & \textbf{66.08} & \textbf{81.82} & \textbf{87.07} & \textbf{81.20} & \textbf{86.45} & \textbf{84.18} & \textbf{84.94} \\
%     \bottomrule
%   \end{tabular}
% \end{table}


% \begin{table}[!ht]
%   \caption{Accuracy (\%) against unforeseen corruptions.}
%   \label{corruption-table2}
%   \centering
%   \tiny
%   \begin{tabular}{lcccccccccccccccccc}
%     \toprule
%     Defense Model & contrast & elastic transform & frost & gaussian noise & impulse noise & pixelate & shot noise & spatter & zoom blur \\
%     \midrule
%     TRADES & 43.11 & 79.11 & 76.45 & 79.21 & 73.72 & 82.73 & 80.42 & 80.72 & 78.97 \\
%     MART & 41.22 & 77.77 & 73.07 & 78.30 & 74.97 & 81.31 & 79.53 & 79.28 & 77.8 \\
%     AT & 43.30 & 79.58 & 77.53 & 79.47 & 73.76 & 82.78 & 80.86 & 80.49 & 79.58 \\
%     ATDA & 36.06 & 67.06 & 62.56 & 70.33 & 64.63 & 73.46 & 72.28 & 70.50 & 67.31 \\
%     DIAL (Ours) & \textbf{48.84} & \textbf{84.13} & \textbf{81.76} & \textbf{83.76} & \textbf{78.26} & \textbf{87.24} & \textbf{85.13} & \textbf{84.84} & \textbf{83.93}  \\
%     \bottomrule
%   \end{tabular}
% \end{table}


% \begin{figure}[!ht]
%   \centering
%   \includegraphics[width=9cm]{figures/spider_full.png}
%   \caption{Accuracy comparison with all tested methods over unforeseen corruptions.}
%   \label{spider-full-graph}
% \end{figure}
% %%%%%%%%%%%%%%%%%%%%%%%%% Arxiv version %%%%%%%%%%%%%%%%%%%%%%%%%%%%%%%%%%%%%
%%%%%%%%%%%%%%%%%%%%%%%%% Arxiv version %%%%%%%%%%%%%%%%%%%%%%%%%%%%%%%%%%%%%

\subsubsection{Transfer Learning}
Recent works \citep{salman2020adversarially,utrera2020adversarially} suggested that robust models transfer better on standard downstream classification tasks. In Table \ref{transfer-res} we demonstrate the advantage of our method when applied for transfer learning across CIFAR10 and CIFAR100 using the common linear evaluation protocol. see Appendix \ref{transfer-learning-settings} for detailed settings.

\begin{table}[H]
  \caption{Transfer learning results comparison.}
  \vskip 0.1in
  \label{transfer-res}
  \centering
  \small
\begin{tabular}{c|c|c|c}
\toprule

\multicolumn{2}{l}{} & \multicolumn{2}{c}{Target} \\
\cmidrule(r){3-4}
Source & Defence Model & CIFAR10 & CIFAR100 \\
\midrule
\multirow{3}{*}{CIFAR10} & DIAL & \multirow{3}{*}{-} & \textbf{28.57} \\
 & AT &  & 26.95  \\
 & TRADES &  & 25.40  \\
 \midrule
\multirow{3}{*}{CIFAR100} & DIAL & \textbf{73.68} & \multirow{3}{*}{-} \\
 & AT & 71.41 & \\
 & TRADES & 71.42 &  \\
%  \midrule
% \multirow{3}{}{SVHN} & DIAL &  &  & \multirow{3}{}{-} \\
%  & Madry et al. &  &  &  \\
%  & TRADES &  &  &  \\ 
\bottomrule
\end{tabular}
\end{table}


\subsubsection{Modularity and Ablation Studies}

We note that the domain classifier is a modular component that can be integrated into existing models for further improvements. Removing the domain head and related loss components from the different DIAL formulations results in some common adversarial training techniques. For $\DIAL_{\kl}$, removing the domain and related loss components results in the formulation of TRADES. For $\DIAL_{\ce}$, removing the domain and related loss components results in the original formulation of the standard adversarial training, and for $\DIAL_{\awp}$ the removal results in $\TRADES_{\awp}$. Therefore, the ablation studies will demonstrate the effectiveness of combining DIAL on top of different adversarial training methods. 

We investigate the contribution of the additional domain head component introduced in our method. Experiment configuration are as in \ref{defence-settings}, and robust accuracy is based on white-box PGD$^{20}$ on CIFAR-10 dataset. We remove the domain head from both $\DIAL_{\kl}$, $\DIAL_{\awp}$, and $\DIAL_{\ce}$ (equivalent to $r=0$) and report the natural and robust accuracy. We perform 3 random restarts and omit one standard deviation from the results. Results are presented in Figure \ref{ablation}. All DIAL variants exhibits stable improvements on both natural accuracy and robust accuracy. $\DIAL_{\ce}$, $\DIAL_{\kl}$, and $\DIAL_{\awp}$ present an improvement of 1.82\%, 0.33\%, and 0.55\% on natural accuracy and an improvement of 2.5\%, 1.87\%, and 0.83\% on robust accuracy, respectively. This evaluation empirically demonstrates the benefits of incorporating DIAL on top of different adversarial training techniques.
% \paragraph{semi-supervised extensions.} Since the domain classifier does not require the class labels, we argue that additional unlabeled data can be leveraged in future work.
%for improved results. 

\begin{figure}[ht]
  \centering
  \includegraphics[width=0.35\textwidth]{figures/ablation_graphs3.png}
  \caption{Ablation studies for $\DIAL_{\kl}$, $\DIAL_{\ce}$, and $\DIAL_{\awp}$ on CIFAR-10. Circle represent the robust-natural accuracy without using DIAL, and square represent the robust-natural accuracy when incorporating DIAL.
  %to further investigate the impact of the domain head and loss on natural and robust accuracy.
  }
  \label{ablation}
\end{figure}

\subsubsection{Visualizing DIAL}
To further illustrate the superiority of our method, we visualize the model outputs from the different methods on both natural and adversarial test data.
% adversarial test data generated using PGD$^{20}$ white-box attack with step size 0.003 and $\epsilon=0.031$ on CIFAR-10. 
Figure~\ref{tsne1} shows the embedding received after applying t-SNE ~\citep{van2008visualizing} with two components on the model output for our method and for TRADES. DIAL seems to preserve strong separation between classes on both natural test data and adversarial test data. Additional illustrations for the other methods are attached in Appendix~\ref{additional_viz}. 

\begin{figure}[h]
\centering
  \subfigure[\textbf{DIAL} on natural logits]{\includegraphics[width=0.21\textwidth]{figures/domain_ce_test.png}}
  \hspace{0.03\textwidth}
  \subfigure[\textbf{DIAL} on adversarial logits]{\includegraphics[width=0.21\textwidth]{figures/domain_ce_adversarial.png}}
  \hspace{0.03\textwidth}
    \subfigure[\textbf{TRADES} on natural logits]{\includegraphics[width=0.21\textwidth]{figures/trades_test.png}}
    \hspace{0.03\textwidth}
    \subfigure[\textbf{TRADES} on adversarial logits]{\includegraphics[width=0.21\textwidth]{figures/trades_adversarial.png}}
  \caption{t-SNE embedding of model output (logits) into two-dimensional space for DIAL and TRADES using the CIFAR-10 natural test data and the corresponding PGD$^{20}$ generated adversarial examples.}
  \label{tsne1}
\end{figure}


% \begin{figure}[ht]
% \centering
%   \begin{subfigure}{4cm}
%     \centering\includegraphics[width=3.3cm]{figures/domain_ce_test.png}
%     \caption{\textbf{DIAL} on nat. examples}
%   \end{subfigure}
%   \begin{subfigure}{4cm}
%     \centering\includegraphics[width=3.3cm]{figures/domain_ce_adversarial.png}
%     \caption{\textbf{DIAL} on adv. examples}
%   \end{subfigure}
  
%   \begin{subfigure}{4cm}
%     \centering\includegraphics[width=3.3cm]{figures/trades_test.png}
%     \caption{\textbf{TRADES} on nat. examples}
%   \end{subfigure}
%   \begin{subfigure}{4cm}
%     \centering\includegraphics[width=3.3cm]{figures/trades_adversarial.png}
%     \caption{\textbf{TRADES} on adv. examples}
%   \end{subfigure}
%   \caption{t-SNE embedding of model output (logits) into two-dimensional space for DIAL and TRADES using the CIFAR-10 natural test data and the corresponding adversarial examples.}
%   \label{tsne1}
% \end{figure}



% \begin{figure}[ht]
% \centering
%   \begin{subfigure}{6cm}
%     \centering\includegraphics[width=5cm]{figures/domain_ce_test.png}
%     \caption{\textbf{DIAL} on nat. examples}
%   \end{subfigure}
%   \begin{subfigure}{6cm}
%     \centering\includegraphics[width=5cm]{figures/domain_ce_adversarial.png}
%     \caption{\textbf{DIAL} on adv. examples}
%   \end{subfigure}
  
%   \begin{subfigure}{6cm}
%     \centering\includegraphics[width=5cm]{figures/trades_test.png}
%     \caption{\textbf{TRADES} on nat. examples}
%   \end{subfigure}
%   \begin{subfigure}{6cm}
%     \centering\includegraphics[width=5cm]{figures/trades_adversarial.png}
%     \caption{\textbf{TRADES} on adv. examples}
%   \end{subfigure}
%   \caption{t-SNE embedding of model output (logits) into two-dimensional space for DIAL and TRADES using the CIFAR-10 natural test data and the corresponding adversarial examples.}
%   \label{tsne1}
% \end{figure}



\subsection{Balanced measurement for robust-natural accuracy}
One of the goals of our method is to better balance between robust and natural accuracy under a given model. For a balanced metric, we adopt the idea of $F_1$-score, which is the harmonic mean between the precision and recall. However, rather than using precision and recall, we measure the $F_1$-score between robustness and natural accuracy,
using a measure we call
%We named it
the
\textbf{$\mathbf{F_1}$-robust} score.
\begin{equation}
F_1\text{-robust} = \dfrac{\text{true\_robust}}
{\text{true\_robust}+\frac{1}{2}
%\cdot
(\text{false\_{robust}}+
\text{false\_natural})},
\end{equation}
where $\text{true\_robust}$ are the adversarial examples that were correctly classified, $\text{false\_{robust}}$ are the adversarial examples that were miss-classified, and $\text{false\_natural}$ are the natural examples that were miss-classified.
%We tested the proposed $F_1$-robust score using PGD$^{20}$ on CIFAR-10 dataset in white-box and black-box settings. 
Results are presented in Table~\ref{f1-robust} and demonstrate that our method achieves the best $F_1$-robust score in both settings, which supports our findings from previous sections.

% \begin{table}[!ht]
%   \caption{$F_1$-robust measurement using PGD$^{20}$ attack in white and black box settings on CIFAR-10}
%   \label{f1-robust}
%   \centering
%   \begin{tabular}{lll}
%     \toprule
%     \cmidrule(r){1-2}
%     Defense Model & White-box & Black-box \\
%     \midrule
%     TRADES & 0.65937  & 0.84435 \\
%     MART & 0.66613  & 0.83153  \\
%     Madry et al. & 0.65755 & 0.84574   \\
%     Song et al. & 0.51823 & 0.76092  \\
%     $\DIAL_{\ce}$ (Ours) & 0.65318   & $\mathbf{0.88806}$  \\
%     $\DIAL_{\kl}$ (Ours) & $\mathbf{0.67479}$ & 0.84702 \\
%     \midrule
%     \midrule
%     DIAL-AWP (Ours) & $\mathbf{0.69753}$  & $\mathbf{0.85406}$  \\
%     TRADES-AWP & 0.68162 & 0.84917 \\
%     \bottomrule
%   \end{tabular}
% \end{table}

\begin{table}[ht]
\small
  \caption{$F_1$-robust measurement using PGD$^{20}$ attack in white and black box settings on CIFAR-10.}
  \vskip 0.1in
  \label{f1-robust}
  \centering
%   \small
  \begin{tabular}{c
  @{\hspace{1.5\tabcolsep}}c @{\hspace{1.5\tabcolsep}}c @{\hspace{1.5\tabcolsep}}c @{\hspace{1.5\tabcolsep}}c
  @{\hspace{1.5\tabcolsep}}c @{\hspace{1.5\tabcolsep}}c @{\hspace{1.5\tabcolsep}}|
  @{\hspace{1.5\tabcolsep}}c
  @{\hspace{1.5\tabcolsep}}c}
    \toprule
    % \cmidrule(r){8-9}
     & TRADES & MART & AT & ATDA & $\DIAL_{\ce}$ & $\DIAL_{\kl}$ & $\DIAL_{\awp}$ & $\TRADES_{\awp}$ \\
    \midrule
    White-box & 0.659 & 0.666 & 0.657 & 0.518 & 0.660 & \textbf{0.675} & \textbf{0.698} & 0.682 \\
    Black-box & 0.844 & 0.831 & 0.845 & 0.761 & \textbf{0.890} & 0.847 & \textbf{0.854} & 0.849 \\ 
    \bottomrule
  \end{tabular}
\end{table}


\section{Conclusion}
% \vspace{-0.5em}
\section{Conclusion}
% \vspace{-0.5em}
Recent advances in multimodal single-cell technology have enabled the simultaneous profiling of the transcriptome alongside other cellular modalities, leading to an increase in the availability of multimodal single-cell data. In this paper, we present \method{}, a multimodal transformer model for single-cell surface protein abundance from gene expression measurements. We combined the data with prior biological interaction knowledge from the STRING database into a richly connected heterogeneous graph and leveraged the transformer architectures to learn an accurate mapping between gene expression and surface protein abundance. Remarkably, \method{} achieves superior and more stable performance than other baselines on both 2021 and 2022 NeurIPS single-cell datasets.

\noindent\textbf{Future Work.}
% Our work is an extension of the model we implemented in the NeurIPS 2022 competition. 
Our framework of multimodal transformers with the cross-modality heterogeneous graph goes far beyond the specific downstream task of modality prediction, and there are lots of potentials to be further explored. Our graph contains three types of nodes. While the cell embeddings are used for predictions, the remaining protein embeddings and gene embeddings may be further interpreted for other tasks. The similarities between proteins may show data-specific protein-protein relationships, while the attention matrix of the gene transformer may help to identify marker genes of each cell type. Additionally, we may achieve gene interaction prediction using the attention mechanism.
% under adequate regulations. 
% We expect \method{} to be capable of much more than just modality prediction. Note that currently, we fuse information from different transformers with message-passing GNNs. 
To extend more on transformers, a potential next step is implementing cross-attention cross-modalities. Ideally, all three types of nodes, namely genes, proteins, and cells, would be jointly modeled using a large transformer that includes specific regulations for each modality. 

% insight of protein and gene embedding (diff task)

% all in one transformer

% \noindent\textbf{Limitations and future work}
% Despite the noticeable performance improvement by utilizing transformers with the cross-modality heterogeneous graph, there are still bottlenecks in the current settings. To begin with, we noticed that the performance variations of all methods are consistently higher in the ``CITE'' dataset compared to the ``GEX2ADT'' dataset. We hypothesized that the increased variability in ``CITE'' was due to both less number of training samples (43k vs. 66k cells) and a significantly more number of testing samples used (28k vs. 1k cells). One straightforward solution to alleviate the high variation issue is to include more training samples, which is not always possible given the training data availability. Nevertheless, publicly available single-cell datasets have been accumulated over the past decades and are still being collected on an ever-increasing scale. Taking advantage of these large-scale atlases is the key to a more stable and well-performing model, as some of the intra-cell variations could be common across different datasets. For example, reference-based methods are commonly used to identify the cell identity of a single cell, or cell-type compositions of a mixture of cells. (other examples for pretrained, e.g., scbert)


%\noindent\textbf{Future work.}
% Our work is an extension of the model we implemented in the NeurIPS 2022 competition. Now our framework of multimodal transformers with the cross-modality heterogeneous graph goes far beyond the specific downstream task of modality prediction, and there are lots of potentials to be further explored. Our graph contains three types of nodes. while the cell embeddings are used for predictions, the remaining protein embeddings and gene embeddings may be further interpreted for other tasks. The similarities between proteins may show data-specific protein-protein relationships, while the attention matrix of the gene transformer may help to identify marker genes of each cell type. Additionally, we may achieve gene interaction prediction using the attention mechanism under adequate regulations. We expect \method{} to be capable of much more than just modality prediction. Note that currently, we fuse information from different transformers with message-passing GNNs. To extend more on transformers, a potential next step is implementing cross-attention cross-modalities. Ideally, all three types of nodes, namely genes, proteins, and cells, would be jointly modeled using a large transformer that includes specific regulations for each modality. The self-attention within each modality would reconstruct the prior interaction network, while the cross-attention between modalities would be supervised by the data observations. Then, The attention matrix will provide insights into all the internal interactions and cross-relationships. With the linearized transformer, this idea would be both practical and versatile.

% \begin{acks}
% This research is supported by the National Science Foundation (NSF) and Johnson \& Johnson.
% \end{acks}

{\small
\bibliographystyle{ieee_fullname}
\bibliography{egbib}
}

\clearpage
\pagenumbering{roman}
\appendix

%%%%%%%%% TITLE
\title{Self-supervised Augmentation Consistency \\for Adapting Semantic Segmentation\\[1mm]\large -- Supplemental Material --}
\author{Nikita Araslanov$^1$ \hspace{1cm} Stefan Roth$^{1,2}$\\
$\ ^1$Department of Computer Science, TU Darmstadt \hspace{1cm} $\ ^2$ hessian.AI}

\maketitle

%%%%%%%%% BODY TEXT
\section{Overview}
In this appendix, we first provide further training and implementation details of our framework.
We then take a closer look at the accuracy of long-tail classes, before and after adaptation.
Next, we discuss our strategy for hyperparameter selection and perform a sensitivity analysis.
We also evaluate our framework using another segmentation architecture, FCN8s \citesupp{ShelhamerLD17}.
Finally, we discuss the limitations of the current evaluation protocol and propose a revision based on the best practices in the field at large.

\section{Further Technical Details}
\label{sec:supp_impl}
% !TEX root = ../supp.tex

\paragraph{Photometric noise.}
Recall that our framework uses random Gaussian smoothing, greyscaling and colour jittering to implement the photometric noise.
We re-use the parameters for these operations from the MoCo-v2 framework \citesupp{chen2020mocov2}.
In detail, the kernel radius for the Gaussian blur is sampled uniformly from the range $[0.1, 2.0]$.
Note that this does not correspond to the actual filter size.\footnote{The Pillow Library \citesupp{clark2015pillow} internally converts the radius $r$ to the box length as $L = \sqrt{3 * r^2 + 1}$.}
The colour jitter, applied with probability $0.5$, implements a perturbation of the image brightness, contrast and saturation with a factor sampled uniformly from $[0.6, 1.4]$, while the hue factor is sampled uniformly at random in the range of $[0.9, 1.1]$.
We convert a target image to its greyscale version with probability \num{0.2}.
\cref{fig:photometric} demonstrates an example implementation of this procedure in Python.

\begin{figure}[t]
\lstinputlisting[language=Python]{supp_sections/code/photometric.py}
\vspace{-0.5em}
\caption{\textbf{Python implementation of the photometric noise.}}
\label{fig:photometric}
\vspace{-0.5em}
\end{figure}

\myparagraph{Constraint-free data augmentation.}
Similarly to the multi-scale cropping of the target images, we scale the source images randomly with a factor sampled uniformly from $[0.5, 1.0]$ prior to cropping.
However, we do not enforce the semantic consistency for the source data, since the ground truth of the source images is available.
For both the target and source images we also use random horizontal flipping.
%Including the random flipping to the consistency loss for the target data may lead to further accuracy gains, although is not part of the current implementation yet.
We additionally experimented with moderate rotation (both with and without semantic consistency), but did not observe a significant effect on the mean accuracy.

\begin{table*}[t!]
\footnotesize
\begin{tabularx}{\linewidth}{@{}>{\centering\arraybackslash}p{1.5em}>{\centering\arraybackslash}p{1.5em}>{\centering\arraybackslash}p{1.5em}|S[table-format=2.1]@{\hspace{0.74em}}S[table-format=2.1]@{\hspace{0.74em}}S[table-format=2.1]@{\hspace{0.74em}}S[table-format=2.1]@{\hspace{0.74em}}S[table-format=2.1]@{\hspace{0.74em}}S[table-format=2.1]@{\hspace{0.74em}}S[table-format=2.1]@{\hspace{0.74em}}S[table-format=2.1]@{\hspace{0.74em}}S[table-format=2.1]@{\hspace{0.74em}}S[table-format=2.1]@{\hspace{0.74em}}S[table-format=2.1]@{\hspace{0.74em}}S[table-format=2.1]@{\hspace{0.74em}}S[table-format=2.1]@{\hspace{0.74em}}S[table-format=2.1]@{\hspace{0.74em}}S[table-format=2.1]@{\hspace{0.74em}}S[table-format=2.1]@{\hspace{0.74em}}S[table-format=2.1]@{\hspace{0.74em}}S[table-format=2.1]@{\hspace{0.74em}}S[table-format=2.1]@{\hspace{0.74em}}|S[table-format=2.1]@{}}
\toprule
CBT & IS & FL & {road} & {sidew} & {build} & {wall} & {fence} & {pole} & {light} & {sign} & {veg} & {terr} & {sky} & {pers} & {ride} & {car} & {truck} & {bus} & {train} & {moto} & {bicy} & {mIoU} \\
\midrule
 & & & 88.1 & 41.0 & 85.7 & 30.8 & 30.6 & 33.1 & 37.0 & 22.9 & 86.6 & 36.8 & 90.7 & 67.1 & 27.1 & 86.8 & 34.4 & 30.4 & 8.5 & 7.5 & 0.0 & 44.5 \\
\midrule
& & \cmark & 89.4 & 52.3 & 86.0 & \bfseries 34.0 & 32.6 & \bfseries 38.5 & 43.3 & 30.6 & 85.2 & 30.9 & 88.5 & 66.7 & 28.0 & 85.7 & 35.6 & 39.6 & 0.0 & 6.6 & 0.0 & 46.0 \\
& \cmark & & \bfseries 90.0 & 47.1 & 85.6 & 31.3 & 24.9 & 32.3 & 38.9 & 28.2 & \bfseries 87.3 & \bfseries 39.8 & 89.4 & \bfseries 67.7 & 28.6 & \bfseries 88.1 & 40.1 & 50.0 & 7.3 & 9.9 & 2.2 & 46.8 \\
\cmark & & & 89.3 & 39.0 & 85.1 & 33.2 & 26.1 & 32.4 & 41.8 & 25.2 & 86.3 & 27.4 & \bfseries 90.4 & 66.4 & 28.2 & 87.5 & 32.9 & 45.4 & 11.0 & 7.6 & 0.0 & 45.0 \\
\midrule
& \cmark & \cmark & 89.3 & 52.6 & 86.0 & 33.4 & 30.0 & 38.0 & 44.9 & 34.3 & 86.9 & 35.3 & 88.0 & 65.4 & 27.3 & 86.2 & 37.6 & 44.0 & 20.9 & 9.6 & 6.5 & 48.2 \\
\cmark & & \cmark & 89.3 & 52.2 & 86.1 & 34.2 & 31.5 & 37.0 & 43.4 & 36.3 & 85.2 & 30.7 & 86.6 & 66.2 & \bfseries 30.3 & 85.3 & 36.2 & 43.9 & \bfseries 29.2 & 6.8 & 8.6 & 48.4 \\
\cmark & \cmark & & 89.7 & 45.1 & 85.6 & 29.6 & 28.3 & 31.7 & 41.9 & 27.5 & 87.2 & 37.4 & 89.8 & 66.9 & 29.2 & 87.5 & 37.3 & 31.6 & 24.7 & 11.9 & 20.2 & 47.5 \\
\midrule
\cmark & \cmark & \cmark & \bfseries 90.0 & \bfseries 53.1 & \bfseries 86.2 & 33.8 & \bfseries 32.7 & 38.2 & \bfseries 46.0 & \bfseries 40.3 & 84.2 & 26.4 & 88.4 & 65.8 & 28.0 & 85.6 & \bfseries 40.6 & \bfseries 52.9 & 17.3 & \bfseries 13.7 & \bfseries 23.8 & \bfseries 49.9 \\
\bottomrule
\end{tabularx}
\caption{\textbf{Per-class IoU (\%)} on Cityscapes \emph{val} using a VGG-16 backbone in the GTA5 $\rightarrow$ Cityscapes setting. We study three components in more detail: class-based thresholding (CBT), importance sampling (IS) and the focal loss (FL). The mIoU of the settings in the last four rows are reproduced from the main text. Here, we elaborate on the per-class accuracy in a broader context of the supplementary experiments in the first four rows.}
\label{table:result_longtail}
%\vspace{-0.5em}
\end{table*}

\myparagraph{Training schedule.}
Our framework typically needs $150-200$K iterations in total (\ie~including the source-only pre-training) until convergence, as determined on a random subset of \num{500} images from the training set (see our discussion in \cref{sec:supp_eval} below).
This varies slightly depending on the backbone and the source data used.
This schedule translates to approximately \num{3} days of training with standard GPUs (\eg, Titan X Pascal with 12 GB memory) for both VGG-16 and ResNet-101 backbones.
Recall that we used \num{4} GPUs for our ResNet version of the framework, hence its training time is comparable to the VGG variant, which uses only \num{2} GPUs.
All our experiments use a constant learning rate for simplicity, but more advanced schedules, such as cyclical learning rates \cite{IzmailovPGVW18}, the cosine schedule \citesupp{chen2020mocov2,LoshchilovH17} or ramp-ups \cite{LaineA17}, may further improve the accuracy of our framework.


\section{Additional Experiments}
\label{sec:supp_class}
% !TEX root = ../supp.tex

\subsection{A closer look at long-tail adaptation}
Recall that our framework features three components to attune the adaptation process to the long-tail classes: class-based thresholding (CBT), importance sampling (IS) and the focal loss (FL), which we summarily refer to as the \emph{long-tail components} in the following.
Disabling the long-tail components individually is equivalent to setting $\beta \rightarrow 0$ for CBT, using uniform sampling of the target images instead of IS or assigning $\lambda$ to \num{0} for the FL.
Here, we extend our ablation study of the GTA5 $\rightarrow$ Cityscapes setup with VGG-16 (\cf \cref{table:ablation} from the main text) and experiment with different combinations of the long-tail components.
\cref{table:result_longtail} details the per-class accuracy of the possible compositions.

We observe that the ubiquitous classes -- ``road'', ``building'', ``vegetation'', ``sky'', ``person'' and ``car'' -- are hardly affected;
it is primarily the long-tail categories that change in accuracy.
Furthermore, our long-tail components are mutually complementary.
The mean IoU improves when one of the components is active, from $44.5\%$ to up to $46.8\%$.
It is boosted further with two of the components enabled to $48.4\%$, and reaches its maximum for our model, $49.9\%$, when all three components are in place.

We further identify the following tentative patterns.
FL tends to improve classes ``wall'', ``fence'' and ``pole''.
CBT increases the accuracy of the ``traffic light'' category (which has high image frequency and occupies only a few pixels), but also rare classes, such as ``rider'', ``bus'' and ``train'' benefit from CBT, especially in conjunction with IS.
IS also enhances the mask quality of the classes ``bicycle'' and ``motorcycle''.
Nevertheless, we urge caution against interpreting the results for each class in isolation, despite such widespread practice in the literature.
Today's semantic segmentation models do not possess the notion of an `ambiguous' class prediction and each pixel receives a meaningful label.
By the pigeon's hole principle, this implies that the changes in the IoU of one class have an immediate effect on the IoU of the other classes.
Therefore, the benefits of individual framework components have to be understood in the context of their aggregated effect on multiple classes, \eg~using the mean IoU.
For instance, consider the class ``train'' for which IS appears to also decrease the IoU: while CBT together with FL achieve $29.2\%$ IoU, adding IS decreases the IoU to $17.3\%$.
However, the IoU of other classes increases (\eg, ``motorcycle'', ``bicycle''), as does the mean IoU.
Furthermore, only few classes reach their maximum accuracy when we enable all three long-tail components.
Yet, it is the setting with the best \emph{accuracy trade-off} between the individual classes, \ie~with the highest mean IoU.
Overall, the long-tail components improve our framework by $5.4\%$ mean IoU combined, a substantial margin.

\begin{table*}[t!]
\footnotesize
\begin{tabularx}{\linewidth}{@{}l|S[table-format=2.1]@{\hspace{0.7em}}S[table-format=2.1]@{\hspace{0.7em}}S[table-format=2.1]@{\hspace{0.7em}}S[table-format=2.1]@{\hspace{0.7em}}S[table-format=2.1]@{\hspace{0.7em}}S[table-format=2.1]@{\hspace{0.7em}}S[table-format=2.1]@{\hspace{0.7em}}S[table-format=2.1]@{\hspace{0.7em}}S[table-format=2.1]@{\hspace{0.7em}}S[table-format=2.1]@{\hspace{0.7em}}S[table-format=2.1]@{\hspace{0.7em}}S[table-format=2.1]@{\hspace{0.7em}}S[table-format=2.1]@{\hspace{0.7em}}S[table-format=2.1]@{\hspace{0.7em}}S[table-format=2.1]@{\hspace{0.7em}}S[table-format=2.1]@{\hspace{0.7em}}S[table-format=2.1]@{\hspace{0.7em}}S[table-format=2.1]@{\hspace{0.7em}}S[table-format=2.1]@{\hspace{0.7em}}|c@{}}
\toprule
Method & {road} & {sidew} & {build} & {wall} & {fence} & {pole} & {light} & {sign} & {veg} & {terr} & {sky} & {pers} & {ride} & {car} & {truck} & {bus} & {train} & {moto} & {bicy} & {mIoU} \\
\midrule
\multicolumn{21}{@{}l}{\scriptsize \textit{GTA5 $\rightarrow$ Cityscapes}} \\
\midrule
Baseline (ours) & 76.7 & 28.2 & 74.4 & 12.7 & 19.0 & 27.2 & 28.7 & 12.2 & 77.0 & 18.0 & 70.6 & 54.8 & 20.6 & 79.6 & 19.0 & 19.2 & 20.6 & 27.9 & 11.2 & 36.7 { \scriptsize{(37.1)}} \\
%SAC (ours) & 87.3 & 47.1 & 84.1 & 29.5 & 26.5 & 23.9 & 42.7 & 30.8 & 86.8 & 42.5 & 87.5 & 60.2 & 30.1 & 83.0 & 28.3 & 38.2 & 28.2 & 33.4 & 44.6 & 49.2 { \scriptsize{(49.9)}} \\
SAC-FCN (ours) & 86.3 & 45.6 & 84.4 & 30.3 & 27.1 & 24.8 & 42.8 & 35.2 & 86.9 & 39.7 & 88.0 & 62.3 & 32.1 & 84.1 & 28.4 & 43.7 & 31.9 & 29.4 & 45.8 & 49.9 { \scriptsize{(49.9)}} \\
\midrule
\multicolumn{21}{@{}l}{\scriptsize \textit{SYNTHIA $\rightarrow$ Cityscapes}} \\
\midrule
Baseline (ours) & 50.7 & 23.8 & 60.9 & 1.8 & 0.1 & 27.7 & 10.5 & 15.7 & 60.1 & \textemdash & 72.4 & 50.1 & 16.0 & 66.5 & \textemdash & 13.7 & \textemdash & 8.5 & 26.8 & 31.6 { \scriptsize{(34.4)}} \\
%SAC (ours) & 80.8 & 39.6 & 81.9 & 18.0 & 1.1 & 27.8 & 35.2 & 28.0 & 79.0 & \textemdash & 80.6 & 61.5 & 23.1 & 81.8 & \textemdash & 36.6 & \textemdash & 32.4 & 55.7 & 47.7 { \scriptsize{(49.1)}} \\
SAC-FCN (ours) & 74.7 & 34.2 & 81.4 & 19.8 & 1.9 & 27.2 & 34.8 & 27.2 & 80.0 & \textemdash & 86.3 & 61.5 & 20.8 & 82.5 & \textemdash & 31.2 & \textemdash & 32.0 & 53.9 & 46.8 { \scriptsize{(49.1)}} \\
\bottomrule
\end{tabularx}
\caption{\textbf{Per-class IoU (\%)} on Cityscapes \emph{val} using VGG-16 with FCN8s. For reference, the numbers in parentheses in the last column report the mean IoU of the DeepLabv2 architecture (\cf \cref{table:result_gta_to_city,table:synthia_gta_to_city} from the main text).}
\label{table:result_fcn}
%\vspace{-0.5em}
\end{table*}

\begin{table}[t]
\centering
\setlength{\tabcolsep}{0.8em}%
\begin{tabularx}{0.7\linewidth}{@{}X@{}S[table-format=2.1]S[table-format=2.1]S[table-format=2.1]@{}}
\toprule
$\downarrow \zeta \hfill/\hfill \beta \rightarrow\;$ & {$0.0001$} & {$0.001$} & {$0.01$} \\
\midrule
$0.7$ & 47.9 & 48.8 & 46.7 \\
$0.75$ & 48.6 & 49.9 & 46.3 \\
$0.8$ & 48.2 & 49.8 & 45.6 \\
\bottomrule
\end{tabularx}
\caption{\textbf{Mean IoU (\%) on GTA5 $\rightarrow$ Cityscapes (val) with varying $\zeta$ and $\beta$.} Our framework maintains strong accuracy under different settings of $\zeta$ and $\beta$. Even with a poor choice (\eg, $\zeta = 0.8$, $\beta = 0.01$), it fares well \wrt the state of the art and outperforms many previous works (\cf \cref{table:result_gta_to_city} from the main text).}
\label{table:sensitivity}
\end{table}

\subsection{Hyperparameter search and sensitivity}
\label{sec:hyper_sensitivity}

To select $\zeta$ and $\beta$, we first experimented with a few reasonable choices ($\zeta \in (0.7, 0.8)$, $\beta \in (0.0001, 0.01)$)\footnote{While $\zeta$ may seem more interpretable (the maximum confidence threshold), a reasonable range for $\beta$ can be derived from $\chi_c$ for the long-tail classes, which is simply the fraction of pixels these classes tend to occupy in the image (see Eq.~3).} using a more lightweight backbone (MobileNetV2 \citesupp{Sandler2018:MIR}).
To measure performance, we use the mean IoU on the validation set (500 images from Cityscapes \textit{train}, as in the main text).

Here, we study our framework's sensitivity to the particular choice of $\zeta$ and $\beta$.
To make the results comparable to our previous experiments, we use VGG-16 and report the mean IoU on Cityscapes \textit{val} in \cref{table:sensitivity}.
We observe moderate deviation of the IoU \wrt $\zeta$.
A more tangible drop in accuracy with $\beta = 0.01$ is expected, as it leads to low-confidence predictions, which are likely to be inaccurate, to be included into the pseudo label.
We note that while a suboptimal choice of these hyperparameters leads to inferior results (with a standard deviation of $\pm 1.4$\% mIoU), even the weakest model with $\zeta = 0.8$ and $\beta = 0.01$ did not fail to considerably improve over the baseline (by $8.5$\% IoU, \cf \cref{table:result_gta_to_city} in the main text).


\begin{table*}[t!]
\footnotesize
\begin{tabularx}{\linewidth}{@{}l|S[table-format=2.1]@{\hspace{0.6em}}S[table-format=2.1]@{\hspace{0.6em}}S[table-format=2.1]@{\hspace{0.6em}}S[table-format=2.1]@{\hspace{0.6em}}S[table-format=2.1]@{\hspace{0.6em}}S[table-format=2.1]@{\hspace{0.6em}}S[table-format=2.1]@{\hspace{0.6em}}S[table-format=2.1]@{\hspace{0.6em}}S[table-format=2.1]@{\hspace{0.6em}}S[table-format=2.1]@{\hspace{0.6em}}S[table-format=2.1]@{\hspace{0.6em}}S[table-format=2.1]@{\hspace{0.6em}}S[table-format=2.1]@{\hspace{0.6em}}S[table-format=2.1]@{\hspace{0.6em}}S[table-format=2.1]@{\hspace{0.6em}}S[table-format=2.1]@{\hspace{0.6em}}S[table-format=2.1]@{\hspace{0.6em}}S[table-format=2.1]@{\hspace{0.6em}}S[table-format=2.1]@{\hspace{0.6em}}|c@{}}
\toprule
Method & {road} & {sidew} & {build} & {wall} & {fence} & {pole} & {light} & {sign} & {veg} & {terr} & {sky} & {pers} & {ride} & {car} & {truck} & {bus} & {train} & {moto} & {bicy} & {mIoU} \\
\midrule
\multicolumn{21}{@{}l}{\scriptsize \textit{GTA5 $\rightarrow$ Cityscapes}} \\
\midrule
SAC-FCN (ours) & 87.5 & 45.2 & 85.0 & 29.2 & 26.4 & 23.3 & 44.2 & 32.0 & 88.3 & 52.6 & 91.2 & 65.2 & 35.0 & 86.0 & 24.4 & 32.8 & 31.4 & 36.9 & 40.5 & 50.4 { \scriptsize{(49.9)}} \\
SAC-VGG (ours) & 91.5 & 53.9 & 86.6 & 34.1 & 31.5 & 36.8 & 47.2 & 36.9 & 85.1 & 38.0 & 91.1 & 68.7 & 31.9 & 87.4 & 31.0 & 46.7 & 22.6 & 24.2 & 24.0 & 51.0 { \scriptsize{(49.9)}} \\
SAC-ResNet (ours) & 91.8 & 54.3 & 87.4 & 36.2 & 30.2 & 43.7 & 49.7 & 42.1 & 89.3 & 54.3 & 90.5 & 71.8 & 34.9 & 89.8 & 38.8 & 47.3 & 24.9 & 38.3 & 43.8 & 55.7 { \scriptsize{(53.8)}} \\
\midrule
\multicolumn{21}{@{}l}{\scriptsize \textit{SYNTHIA $\rightarrow$ Cityscapes}} \\
\midrule
SAC-FCN (ours) & 66.9 & 25.9 & 80.8 & 12.1 & 2.0 & 24.4 & 37.1 & 27.5 & 78.8 & \textemdash & 88.9 & 63.9 & 25.0 & 84.7 & \textemdash & 27.4 & \textemdash & 36.9 & 50.2 & 45.8 { \scriptsize{(46.8)}} \\
SAC-VGG (ours) & 70.4 & 29.7 & 83.6 & 11.6 & 1.8 & 34.2 & 41.2 & 29.2 & 81.0 & \textemdash & 87.1 & 67.9 & 25.4 & 75.9 & \textemdash & 34.3 & \textemdash & 42.5 & 57.5 & 48.3 { \scriptsize{(49.1)}} \\
SAC-ResNet (ours) & 87.4 & 41.0 & 85.5 & 17.5 & 2.6 & 40.5 & 44.7 & 34.4 & 87.9 & \textemdash & 91.2 & 68.0 & 31.0 & 89.3 & \textemdash & 33.2 & \textemdash & 38.6 & 49.9 & 52.7 { \scriptsize{(52.6)}} \\
\midrule
\multicolumn{21}{@{}l}{\scriptsize \textit{Fully supervised (Cityscapes)}} \\
\midrule
DeepLab-ResNet \cite{ChenPKMY18} & 97.9 & 81.3 & 90.4 & 48.8 & 47.4 & 49.6 & 57.9 & 67.3 & 91.9 & 69.4 & 94.2 & 79.8 & 59.8 & 93.7 & 56.5 & 67.5 & 57.5 & 57.7 & 68.8 & 70.4 \\
FCN-VGG \citesupp{ShelhamerLD17} & 97.4 & 78.4 & 89.2 & 34.9 & 44.2 & 47.4 & 60.1 & 65.0 & 91.4 & 69.3 & 93.9 & 77.1 & 51.4 & 92.6 & 35.3 & 48.6 & 46.5 & 51.6 & 66.8 & 65.3 \\
\bottomrule
\end{tabularx}
\caption[\textbf{Per-class IoU (\%)} on Cityscapes \emph{test}]{\textbf{Per-class IoU (\%)} on Cityscapes \emph{test}. In the last column, the numbers in parentheses report the mean IoU on Cityscapes \emph{val} from the previous evaluation scheme (\cf \cref{table:result_gta_to_city,table:synthia_gta_to_city} from the main text) for reference. SAC-FCN denotes our VGG-based model with FCN8s \citesupp{ShelhamerLD17} from \cref{sec:fcn}.}
\label{table:result_city_test}
%\vspace{-0.5em}
\end{table*}

\subsection{VGG-16 with FCN8s}
\label{sec:fcn}
A number of previous works (\eg, \cite{MustoZ20,Yang_2020_ECCV,0001S20}) used the FCN8s \citesupp{ShelhamerLD17} architecture with VGG-16, as opposed to DeepLabv2 \cite{ChenPKMY18}, adopted in other works (\eg, \cite{KimB20a,Wang_2020_ECCV}) and ours.
Such architecture exchange appears to have been dismissed in previous work as minor, which used only one of the architectures in the experiments.
However, the segmentation architecture alone may contribute to the observed differences in accuracy of the methods and, more critically, to the improvements otherwise attributed to other aspects of the approach.
To facilitate such transparency in our work, we replace DeepLabv2 with its FCN8s counterpart in our framework (with the VGG-16 backbone) and repeat the adaptation experiments from \cref{sec:exp}, \ie~using two source domains, GTA5 and SYNTHIA, and Cityscapes as the target domain.
We keep the values of the hyperparameters the same, with an exception of the learning rate, which we increase by a factor of \num{2} to $5\times 10^4$.
\cref{table:result_fcn} reports the results of the adaptation, which clearly show that our framework generalises well to other segmentation architectures.
Despite the FCN8s baseline model (source-only loss with ABN) achieving a slightly inferior accuracy compared to DeepLabv2 (\eg, $31.6\%$ \vs~$34.4\%$ IoU for SYNTHIA $\rightarrow$ Cityscapes), our self-supervised training still attains a remarkably high accuracy, $46.8\%$ IoU (\vs~$49.1\%$ with DeepLabv2).
This is substantially higher than the previous best method using FCN8s with the VGG-16 backbone, SA-I2I \cite{MustoZ20}: $+3.4\%$ on GTA5 $\rightarrow$ Cityscapes and $+5.3\%$ on SYNTHIA $\rightarrow$ Cityscapes.


\section{Towards Best-practice Evaluation}
\label{sec:supp_eval}
% !TEX root = ../supp.tex

The current strategy to evaluate domain adaptation (DA) methods for semantic segmentation is to use the ground truth of \num{500} randomly selected images from the Cityscapes \textit{train} split for model selection and to report the final model accuracy on the \num{500} Cityscapes \textit{val} images \cite{LianDLG19}.
In this work, we adhered to this procedure to enable a fair comparison to previous work.
However, this evaluation approach is evidently in discord with the established best practice in machine learning and with the benchmarking practice on Cityscapes \cite{CordtsORREBFRS16}, in particular.

The test set is holdout data to be used only for an unbiased performance assessment (\eg, segmentation accuracy) of the final model \citesupp{0082591}.
While it is conceivable to consult the test set for verifying a number of model variants, such access cannot be unrestrained.
This is infeasible to ensure when the test set annotation is public, as is the case with Cityscapes \textit{val}, however.
Benchmark websites traditionally enable a restricted access to the test annotation through impartial submission policies (\eg, limited number of submissions per time window and user), and Cityscapes officially provides one.\footnote{\url{https://www.cityscapes-dataset.com}}

We, therefore, suggest a simple revision of the evaluation protocol for evaluating future DA methods.
As before, we use Cityscapes \textit{train} as the training data for the target domain, naturally without the ground truth.
For model selection, however, we use Cityscapes \textit{val} images with the ground-truth labels.
The holdout test set for reporting the final segmentation accuracy after adaptation becomes Cityscapes \textit{test}, with the results obtained via submitting the predicted segmentation masks to the official Cityscapes benchmark server.

An additional advantage of this strategy is a clear interpretation of the final accuracy in the context of fully supervised methods that routinely use the same evaluation setup.
Also note that Cityscapes \textit{val} contains images from different cities than Cityscapes \textit{train} (which are also different from Cityscapes \textit{test}).
Therefore, it is more suitable for detecting cases of model overfitting on particularities of the city, since the validation set was previously a subset of the training images.

For future reference, we evaluate our framework (both the DeepLabv2 and FCN8s variants) in the proposed setup and report the results in \cref{table:result_city_test}.
To ease the comparison, we juxtapose our validation results reported in the main text (from \cref{table:result_fcn} for FCN8s).\footnote{To our best knowledge, no previous work published their results in this evaluation setting before.}
As we did not finetune our method to Cityscapes \emph{val} following the previous evaluation protocol, we expect the test accuracy on Cityscapes \emph{test} to be on a par with our previously reported accuracy on Cityscapes \emph{val}.
The results in \cref{table:result_city_test} clearly confirm this expectation: the segmentation accuracy on Cityscapes \emph{test} is comparable to the accuracy on Cityscapes \emph{val} (SYNTHIA $\rightarrow$ Cityscapes) or even tangibly higher (GTA5 $\rightarrow$ Cityscapes).
We remark that the remaining accuracy gap to the fully supervised model is still considerable (70.4\% \vs 55.7\% IoU achieved by our best DeepLabv2 model and 65.3\% \vs 51.0\% IoU compared to our best FCN8s variant), which invites further effort from the research community.

We hope that future UDA methods for semantic segmentation will follow suit in reporting the results on Cityscapes \emph{test}.
Owing to the regulated access to the test set, we believe this setting to offer more transparency and fairness to the benchmarking process, and will successfully drive the progress of UDA for semantic segmentation, as it has done in the past for the fully supervised methods.


{\small
\bibliographystylesupp{ieee_fullname}
\bibliographysupp{supp_egbib}
}

\end{document}
