\documentclass[final]{cvpr}

\usepackage{titling}
\usepackage{cvprtitle} % redefine cvpr title style
\usepackage{times}
\usepackage{epsfig}
\usepackage{graphicx}
\usepackage{amsmath}
\usepackage{amssymb}
\usepackage{bm}
\usepackage{multibib}


% Include other packages here, before hyperref.
\usepackage{bbding}
\usepackage{tabularx}
\usepackage{booktabs}
\usepackage{multirow}
\usepackage[keeplastbox]{flushend}
\usepackage{newtxtt}
\usepackage{etoolbox,siunitx}
\robustify\bfseries
\sisetup{detect-all = true}

% drawing boxes
% make it less epileptic
\usepackage{color}
\definecolor{mybar}{rgb}{1.0, 0.4, 0.0}
\newcommand\cbar[3][mybar]{\colorbox{#1}{\color{black}\framebox(#2,#3){}}}

\definecolor{mygreen}{rgb}{0.2, 0.7, 0.1}
\definecolor{turquoise}{rgb}{0.173, 0.627, 0.537}


% If you comment hyperref and then uncomment it, you should delete
% egpaper.aux before re-running latex.  (Or just hit 'q' on the first latex
% run, let it finish, and you should be clear).
\usepackage[pagebackref=true,breaklinks=true,colorlinks,citecolor={mygreen},bookmarks=false]{hyperref}


% figure and equations with cref.
\usepackage[capitalize]{cleveref}
\crefname{section}{Sec.}{Section}

\usepackage[format=plain,labelformat=simple,labelsep=period,font=small,skip=4pt,compatibility=false]{caption}
\usepackage[font=footnotesize,skip=2pt,subrefformat=parens]{subcaption}

% \cmark & \xmark
\usepackage{pifont} % http://ctan.org/pkg/pifont
\newcommand{\cmark}{\ding{51}}%
\newcommand{\xmark}{\ding{55}}%

\usepackage[inline]{enumitem}
\usepackage{listings}

\definecolor{codegreen}{rgb}{0,0.6,0}
\definecolor{codegray}{rgb}{0.5,0.5,0.5}
\definecolor{codepurple}{rgb}{0.58,0,0.82}
\definecolor{backcolour}{rgb}{0.95,0.95,0.92}

\lstdefinestyle{mystyle}{
    backgroundcolor=\color{backcolour},
    commentstyle=\color{codegreen},
    keywordstyle=\color{magenta},
    numberstyle=\tiny\color{codegray},
    stringstyle=\color{codepurple},
    basicstyle=\ttfamily\footnotesize,
    breakatwhitespace=false,
    breaklines=true,
    captionpos=b,
    keepspaces=true,
    numbers=left,
    numbersep=5pt,
    showspaces=false,
    showstringspaces=false,
    showtabs=false,
    tabsize=2
}

\lstset{style=mystyle}
%
% Commands
%

% abbreviations
\usepackage{xspace}
\newcommand{\iid}{\emph{i.\thinspace{}i.\thinspace{}d.}\@\xspace}
\renewcommand{\cf}{\emph{cf.}\@\xspace}
%\newcommand{\etc}{\emph{etc}}
\newcommand{\Eq}{Eq.\@\xspace}
\newcommand{\Eqs}{Eqs.\@\xspace}
\newcommand{\Fig}{Fig.\@\xspace}
\newcommand{\Figs}{Figs.\@\xspace}
\newcommand{\Tab}{Tab.\@\xspace}
\newcommand{\Tabs}{Tabs.\@\xspace}
\newcommand{\Sec}{Sec.\@\xspace}
\newcommand{\Secs}{Secs.\@\xspace}
\newcommand{\Def}{Def.\@\xspace}
\newcommand{\resp}{resp.\@\xspace}

\newcommand{\norm}[1]{\left\lVert#1\right\rVert}
\newcommand*{\myparagraph}[1]{\smallskip\noindent\textbf{#1}\hspace{0.5em}}

\DeclareMathOperator*{\argmin}{arg\,min}
\DeclareMathOperator*{\argmax}{arg\,max}

% footnote w/o a marker
\newcommand\blfootnote[1]{%
  \begingroup
  \renewcommand\thefootnote{}\footnote{#1}%
  \addtocounter{footnote}{-1}%
  \endgroup
}

% Custom footer for title page
\usepackage{fancyhdr}
\usepackage{setspace}
\renewcommand{\headrulewidth}{0pt}
\renewcommand{\footrulewidth}{0pt}
\fancyhf{}
\lfoot{{\footnotesize
\begin{spacing}{.5}
\parbox{\linewidth}{\vspace{2.5em}%
To appear in Proceedings of the \emph{IEEE/CVF Conference on Computer Vision and Pattern Recognition (CVPR)}, virtual, 2021. \\ \hrule \vspace {\baselineskip}
\copyright~2021 IEEE. Personal use of this material is permitted. Permission from IEEE must be obtained for all other uses, in any current or future media, including reprinting/republishing this material for advertising or promotional purposes, creating new collective works, for resale or redistribution to servers or lists, or reuse of any copyrighted component of this work in other works.
}\end{spacing}}}

\newcites{supp}{References}

\begin{document}

%%%%%%%%% TITLE
\title{Self-supervised Augmentation Consistency \\for Adapting Semantic Segmentation}

\author{Nikita Araslanov$^1$ \hspace{1cm} Stefan Roth$^{1,2}$\\
$\ ^1$Department of Computer Science, TU Darmstadt \hspace{1cm} $\ ^2$ hessian.AI}

\maketitle
\thispagestyle{fancy}

%%%%%%%%% ABSTRACT
\begin{abstract}
\begin{abstract}
\label{sec:abstract}

%% 1. what is the problem 
Scientific applications that run on leadership computing facilities often face the challenge 
of being unable to fit leading science cases onto accelerator devices due to memory constraints 
(memory-bound applications).
%
% 2. what is your solution 
In this work, the authors studied one such US Department of Energy mission-critical condensed matter 
physics application, Dynamical Cluster Approximation (DCA++), and this paper discusses how device memory-bound challenges were successfully reduced  by proposing an effective 
``all-to-all'' communication method---a ring communication algorithm. 
%
This implementation takes advantage of acceleration on GPUs and remote direct memory access (RDMA) for fast data exchange between GPUs. 
%
\\Additionally, the ring algorithm was optimized with sub-ring communicators
and multi-threaded support to further reduce communication overhead and 
expose more concurrency, respectively.
%
% 3. What's the cherry-picked evaluation result you want to mention
The computation and communication were also analyzed 
by using the Autonomic Performance Environment for Exascale 
(APEX) profiling tool,  and this paper further discusses the 
performance trade-off for the ring algorithm implementation. 
%
The memory analysis on the ring algorithm shows that the allocation size for the authors' most 
memory-intensive data structure per GPU is now reduced to $1/p$ of the original size, where $p$ is the number of GPUs in the ring communicator.
%
The communication analysis suggests that 
the distributed Quantum Monte Carlo execution time grows linearly as sub-ring size increases, and the cost of messages passing through the network interface connector could be a limiting factor.


%
% \todoRed{Ronnie: Next sentence needs rewrite, too much information about Green's function that no one knows in the abstract; recommend generalizing.} \emph {However, DCA++ is currently facing memory-bound challenge as 
% a larger device array $G_t$ is limited by device memory size, where
% $G_t$ is a two-particle Green's function that allows condensed matter
% scientists to explore larger and more complex (higher fidelity)
% physics cases.}

\end{abstract}

\keywords{DCA++, Quantum Monte Carlo, GPU Remote Direct Memory Access, memory-bound issue, exascale machines}

\end{abstract}

%%%%%%%%% BODY TEXT
\section{Introduction}
\label{sec:intro}
Reinforcement learning has achieved great success in areas such as Game-playing \citep{silver2018general,vinyals2019grandmaster}, robotics \cite{kober2013reinforcement}, large language models \citep{ouyang2022training}, etc.
However, due to safety concerns or physical limitations, in some real-world reinforcement learning problems, we must consider additional constraints that may influence the optimal policy and the learning process \citep{garcia2015comprehensive}.
% For example, a robotic arm must not take actions that may cause harm to itself or the environments.
A standard framework to handle such cases is the constrained Markov Decision Process (CMDP) \citep{altman1999constrained}.
Within the CMDP framework, the agent has to maximize
the expected cumulative reward while
obeying a finite number of constraints, which are usually in the form of expected cumulative cost criteria.

However, we are sometimes concerned with the problem with a continuum of constraints.
For example,
the constraints we meet might be time-evolving or subject to uncertain parameters, which
cannot be formulated as an ordinary CMDP
(see Examples \ref{Example_Time_Evolving} and  \ref{Example_Uncertain}).
In this paper we would study a generalized CMDP  
to address the above problem.  Because the constraints are not only infinite-number but also lie
in a continuous set,
the generalization is not trivial. Fortunately, we find that we can borrow the idea behind semi-infinite programming (SIP) \citep{remez1934determination, hettich1993semi} to deal with the semi-infinite constraints.
Accordingly, we propose \emph{semi-infinitely constrained Markov decision processes} (SICMDPs)
as a novel complement to the ordinary CMDP framework.
%More specifically,  an SICMDP model %, we consider 
%contains a continuum of constraints whereas an ordinary CMDP contains a finite number of constraints. 

%This generalization is natural but not trivial. However, we can brows the idea  
%The idea is quite natural and can be backtracked
%to the practice of extending linear programming to linear semi-infinite programming (LSIP) %\cite{remez1934determination, GobernaLSIO1998}.
%In addition, 
%As a complementary approach to the ordinary CMDP framework, 
%SICMDP can be used to model these problems  which cannot be described by a finite number of constraints
%that are not covered by .
%For example,
%the restrictions we consider can be time-evolving or subject to uncertain parameters
%, thus
%cannot be described by a finite number of constraints but a continuum of constraints 
%(see Examples \ref{Example_Time_Evolving} and  \ref{Example_Uncertain}).

We also present two reinforcement learning algorithms to solve SICMDPs called SI-CRL and SI-CPO, respectively.
SI-CRL is a model-based reinforcement learning algorithm designed for tabular cases, and SI-CPO is a policy optimization algorithm for non-tabular cases.
% and analyze its performance both theoretically and empirically.
The main challenge is that we need to deal with a continuum of constraints, thus reinforcement learning algorithms for ordinary CMDPs do not work anymore.
In SI-CRL, we tackle this difficulty by first transforming the reinforcement learning problem to an equivalent LSIP problem, which can then be solved using methods in the LSIP literature like the dual exchange methods \citep{Hu1990,reemtsen1998numerical}.
In SI-CPO, we resort to the idea of cooperative stochastic approximation developed in \cite{lan2020algorithms, wei2020comirror}.
As far as we know, we are the first to introduce tools from semi-infinitely programming (SIP) into the reinforcement learning community for solving constrained reinforcement learning problems.

% To the best of our knowledge, we are the first to apply tools from semi-infinitely programming (SIP) to solve reinforcement learning problems.
Furthermore, we give theoretical analysis for both SI-CRL and SI-CPO.
We decompose the error of SI-CRL into two parts: the statistical error from approximating the true SICMDP with an offline dataset and the optimization error due to the fact that the solution of the LSIP problem obtained by the dual exchange method is inexact.
On the optimization side, we show that the iteration complexity of SI-CRL is $O\left(\left\{\mathrm{diam}(Y)L\sqrt{|\gS|^2|\gA|m}/\left[(1-\gamma)\epsilon\right]\right\}^m\right)$.
On the statistical side, we show that the sample complexity of SI-CRL is $\widetilde O\left(\frac{|S|^2|A|^2}{\epsilon^2(1-\gamma)^3}\right)$ if the offline dataset is generated by a generative model, and $\widetilde O\left(\frac{|S||A|}{\nu_{\min} \epsilon^2(1-\gamma)^3}\right)$ if the dataset is generated by a probability measure $\nu$ as considered in \cite{chen2019information}.
Here $\widetilde O$ means that all logarithm terms are discarded.
For SI-CPO, things become a little more complicated because other than the statistical error and the optimization error, we also need to consider the function approximation error, which comes from imperfect policy parametrizations.
It is shown if the function approximation error can be controlled to $O(\epsilon)$ order, the iteration complexity of SI-CPO is $\widetilde{O}\left(\frac{1}{\epsilon^2(1-\gamma)^6}\right)$ and the sample complexity of SI-CPO is $\widetilde{O}(\frac{1}{\epsilon^4(1-\gamma)^{10}})$.
Here our iteration complexity bound is equivalent to a typical $\widetilde O(1/\sqrt{T})$ global convergence rate.

We perform a set of numerical experiments to illustrate the SICMDP model and validate our proposed algorithms.
Specifically, we examine two numerical examples, namely the discharge of sewage and ship route planning.
Through the discharge of sewage example, we show the advantage of the SICMDP framework over the CMDP baseline obtained by naive discretization in modeling realistic sequential decision-making problems.
Moreover, we demonstrate the effectiveness of the SI-CRL and SI-CPO algorithms in such tabular environments. 
In the ship route planning example, we illustrate the benefits of the SICMDP framework and the ability of the SI-CPO algorithm to address complex continuous control tasks involving continuous state spaces with modern deep reinforcement learning techniques.

% In summary, our contributions are listed as follows.
% First, we present the SICMDP model, which can be viewed as a generalization of the ordinary CMDP model.
% Second, we propose an algorithm to perform reinforcement learning for SICMDPs, which is called SI-CRL, and we believe that we are the first to apply tools from SIP
% to solve reinforcement learning problems.
% Third, we give a theoretical analysis of SI-CRL and identify both its sample complexity and iteration complexity.
% In addition, we perform numerical experiments to illustrate the SICMDP model and validate the SI-CRL algorithm.
% \{This paragraph can be removed!!! \}






\pagestyle{plain}
\section{Related Work}
The industry standard for pose edition is to create rigs, a collection of pieces of software designed to manipulate a character's skeleton. The rig describes the skeleton's bones, how they relate to each other, are constrained in their possible motion and are deformed. These rules are loosely specified and creating a good rig requires a detailed understanding of physics and anatomy, as well as technical and artistic skills. Rigging is thus a time consuming task even for experienced animators, and even more so in large scale productions which often require a different in-depth rig for each character in the cast.
Previous work has helped alleviate this difficulty by providing efficient tools to speed up/and or ease the rigging process, relying on inverse kinematics or data-driven methods.
\subsection{Character pose design}
\subsubsection{Inverse Kinematics (IK)}
IK solvers are a family of methods commonly used in robotics, engineering and computer graphics, in which the parameterization of a kinematic chain is determined from the position of its end effector.
They are a staple tool in pose design software, ensuring the respect of elementary constraints during pose edition. Their de-facto role is to guarantee the length of the limbs, and in some cases to enforce the orientation angle range of a joint.
Many IK solutions have been studied over the years \cite{aristidou_inverse_2018}; usually revolving around approximated linearizations or heuristics. 

Numerical methods require a set of iterations to achieve a satisfactory solution formulated by a cost function to be minimized.
IK solutions can generally be divided into three sub-categories: Jacobian \cite{Siciliano_Handbook_Robot_2007}, Newtonians \cite{cohen_ik_1996} and Heuristics. Most software implement heuristic methods such as Cyclic Coordinate Descent (CCD) \cite{wang_ccd_1991} or 
Forward-Backward Reaching IK (FABRIK) \cite{aristidou_fabrik:_2011} due to their simplicity and extensibility. 

The main drawback of 
these solvers is that they manipulate kinematic chains without taking into account many morphological aspects that make a pose more or less plausible. They offer a first level of help to users but are not sufficient to guarantee a realistic pose. Many joints constraints are dependent on each other and require subjective, human-made approximations.

\subsubsection{Data-driven pose edition}
Data-driven methods offer promising opportunities to solve these approximations. Using real-life data can help in modelling the complex inter-dependencies of skeletons and providing users with smarter edition tools.
While it is still an early field of research, some solutions have been studied. Wu \etal \cite{wu_posing_2009} propose a method for natural character posing from a large motion database. It employs adaptive KD-clustering to select a representative frame from a database and sparse approximations to accelerate training and posing. 
Huang \etal in \cite{Huang_IK_MGDM_2017} present a method based on the formulation of multi-variate Gaussian distribution models (MGDMs), which learn the joint constraints of a kinematic skeleton from motion capture data. 

Some work has also been dedicated to finding new editing interfaces. \modify{}{Instead of the usual setup manipulating joints directly, Guay \etal \cite{guay_line_2013} articulate a framework based on the conceptual "line of action" which describes the overall pose dynamics. They provide a mathematical definition of the line of action, and a interface in which the software modifies the pose to follow a user-provided line. In the same line of though} Garcia \etal \cite{garcia_sketching_2019} propose \modify{a method transforming doodle of trajectories (position and orientation over time) }{a virtual reality-based interface where the user's hands motion (position and orientation over time) are transformed} into sequences of actions and then into detailed character animations using a dataset of parametrized motion clips automatically fitted to the trajectory. 

% ==> DL et Latent Space. 
\subsection{Neural modelling of human motion}
Neural networks have received a great amount of attention over the last decade and shown impressive result in modelling complex data. Human motion has not been spared and deep learning methods have proven their capability of generating realistic motion in a number of difficult cases. 

The literature in neural-based animation include example in user-controlled character navigation \cite{Holden2017} and interactions with the environment \cite{starke_neural_2019}. 
Holden \etal \cite{Holden2020} also show that neural networks can be used to replace parts of existing data-driven methods, improving their scalability potential.
More recently, some work has also focused on improving smaller parts of the animation pipeline rather than replacing it completely. Berson et al. \cite{berson_intuitive_2020} leverage neural networks to provide an interactive system to edit facial animation. 

% Wrap up
Data-driven IK and pose editing can relieve animators from time-consuming, back-and-forth pose adjustments by applying constraints extracted from real-world data. Recently, neural-network-based approaches have demonstrated their ability to model the intricacies of human motion while scaling to large amount of data and retaining a fast inference time. In this paper we seek to take advantage of these properties to create an efficient posing tool, intuitively usable even by a inexperienced user.

\section{Self-Supervised Augmentation Consistency}
The proposed segmentation-by-detection framework, as depicted in Figure \ref{fig:framework}, consists of a detection module and a segmentation module.
In detection stage, 2D slices (layered box) from the input volume are fed to the RPN. Based on the region proposals obtained from RPN, an attention model (block in orange) is formed. The input volume as well as the attention model are further processed in segmentation stage to get the refined anatomical segmentation. 
\vspace{1em} 

\begin{figure}[t]
\centering
\includegraphics[width=0.95\linewidth]{fig/framework.pdf}
\caption{Schematic representation of the segmentation-by-detection framework. The left part is the detection module while the segmentation module is followed on the right. The blue block denotes the input volume which is 3D ultrasound scan of femoral head. The output segmentation is in red.}
\label{fig:framework}
\end{figure}
% dana could you improve the figure. we can try to think together of better ways 

\noindent\textbf{Detection Module:} 
% dana : here you have to make the clarification that you have ground truth on the boxes (in implementation part)
The detection module follows an RPN architecture, a fully convolutional network which takes image slice as input and outputs object region candidates. 
We use the VGG-16 model as the backbone \cite{simonyan2014very} to learn convolutional features and an $3 \times 3$ spatial window to generate region proposals. At each sliding-window location, 9 anchors are predicted associated with different scales and aspect ratios. The last layer consists of a box-regression (reg) layer and a box-classification (cls) layer in parallel. The reg layer outputs 4 regression offsets, $ t = (t_x,t_y,t_w,t_h)$, denoting a scale-invariant translation as well as log-space height and width shift, where $x,y,w$ and $h$ specify two coordinates of the box center, width and height. The cls layer outputs two scores by softmax, related to probabilities of object and background for each proposal. We assign a positive label (of being object) to candidate which has an Intersection-over-Union (IoU) ratio higher than 0.7 with ground truth box. Note that an image slice may contain multiple object regions or none. 

The loss function of RPN follows the multi-task loss \cite{ren2015faster} which is defined as $L = L_{reg} + L_{cls}$. The regression loss, $L_{reg} = -\log p_{obj}$ is log loss and the classification loss,
\begin{equation} \label{eq:loss}
L_{cls} = \sum_{i \in \{x,y,w,h\}} smooth_{L_1} (t_i - t_i^*)
\end{equation}
is smooth $L_1$ loss where $t_i^*$ denotes the ground truth box for the target object. 
\vspace{1em}

\noindent\textbf{Segmentation Module:}
3D U-Net \cite{cciccek20163d} is utilized in the segmentation module as its outstanding performance in medical image segmentation. The u-shaped architecture consists of two paths: a contracting path, where each layer contains two $3\times3\times3$ convolutions followed by a rectified linear unit (ReLU) and then a max pooling, provides high resolution features. While, the symmetric expanding path for semantically richer features replaces max pooling with a upconvolution $2\times2\times2$ with stride of 2 in each dimension, and then two $3\times3\times3$ convolutions each followed by a ReLU. Skip connections between layers of equal resolution in the contracting path and the expanding path enables context information as well as precise localization.

Different from 3D U-Net, to incorporate the attention model detected by the RPN, our architecture takes as input both the volumetric image data and the candidate RoIs proposed by the RPN, concatenated as 3D volume. 
% dana not sure what you like to say below
% densely annotated
The attention model makes the network to focus on the potential RoIs and can reduce the interference of the surrounding noise.
The anatomical segmentation is then generated from a $1\times1\times1$ convolution which reduces the number of feature maps to the number of labels.  The energy function is computed by a pixel-wise softmax combined with the cross entropy loss.
% dana equation ??

\subsection{System and implementation Details}
The segmentation-by-detection approach adopts a cascade structure with two stages: detection and segmentation. The two networks are trained separately in an end-to-end manner. All the new layers are randomly initialized from zero-mean Gaussian distribution with standard deviations 0.01. Biases are initialized to 0. We use Caffe \cite{jia2014caffe} for the implementation and an NVIDIA Titan X GPU for training.

In the detection stage, we initialize the VGG-16 model by the pre-trained model for ImageNet classification \cite{russakovsky2015imagenet} and further fine-tune the model for our detection task. The input fed to the network are image slices with a fixed size of $184\times96$ and the corresponding ground truth boxes are generated from the annotation in the format of tight bounding boxes surrounding the segmentation contour (as illustrated in Figure \ref{fig:hip} (b), the boundary of white area). To optimize the energy function, stochastic gradient descent (SGD) is used. The global learning rate is set to 0.001, while a momentum of 0.9 and a weight decay of 0.0005 are used. The batch size is set to 256 and each mini-batch only contains the positive anchors for training. The region proposals are obtained from the reg path for each image slice. The attention model is then formed by concatenating all the detected regions, as binary masks, into a volume.

In the segmentation stage, we use the Adam optimizer \cite{kingma2014adam} to learn the network parameters. A global learning rate is set to 0.001 while the two momentum coefficients are set to 0.9 and 0.999 respectively. A batch size of 1 is used due to the memory constraints of the GPU. The network takes the volume data as well as the attention model as input. We train the network for a maximum of 30K iterations and reserve the learned weights with the best performance from every 1K iterations. 
\vspace{1em}

\noindent\textbf{Inference:}
At test time, the 2D slices from an input volume are first fed to the detection module. The attention model is obtained based on the output. Then the volume data as well as the attention model are fed to the segmentation module to get the pixel-wise prediction.




\section{Experiments}
\label{sec:exp}

\section{Experiments}\label{sec:experiments}
We validate our approach using multiple datasets containing real-life data from the fields of criminal risk assessment, credit, lending, and college admissions. In each of the datasets we select a binary feature and treat it as the protected attribute (e.g., race or gender), which is the feature we require our trained classifier to behave fairly upon. Our proposed method performs well on all of these datasets, succeeding in removing unfairness almost entirely, at a very modest price in terms of accuracy.


\begin{table*}[h]
\centering
\resizebox{\textwidth}{!}{
\def\arraystretch{1.2}

\begin{tabular}{c c c | c | c | c || c | c | c || c | c | c |}

\cline{4-12}
&&&
\multicolumn{9}{ c| }{\textbf{COMPAS Dataset}}
\\ \cline{4-12}
&&&
\multicolumn{3}{ c|| }{\textbf{FPR Considerations}}&
\multicolumn{3}{ c|| }{\textbf{FNR Considerations}}&
\multicolumn{3}{ c| }{\textbf{Both Considerations}}
\\ \cline{4-12}
&&&
 $\mathbf{Acc.}$ &  $\mathbf{D_{FPR}}$ &  $\mathbf{D_{FNR}}$ &  $\mathbf{Acc.}$ &  $\mathbf{D_{FPR}}$ &  $\mathbf{D_{FNR}}$ &  $\mathbf{Acc.}$ &  $\mathbf{D_{FPR}}$ &  $\mathbf{D_{FNR}}$
\\  \cline{4-12}
\vspace*{-0.5ex}
\\ \cline{1-2} \cline{4-12}
\multicolumn{1}{ |c  }{} &
\multicolumn{1}{ c|  }{  \textbf{Our Method (AVD Penalizers)}}  &&
$\mathbf{0.660}$    &  $\mathbf{0.01}$  &  $0.04$ &
$\mathbf{0.653}$    &  $0.02$   &  $\mathbf{0.04}$ &
$\mathbf{0.654}$    &  $\mathbf{0.02}$  &  $\mathbf{0.04}$
\\ \cline{1-2} \cline{4-12}
\multicolumn{1}{ |c  }{} &
\multicolumn{1}{ c|  }{  \textbf{Our Method (SD Penalizers)}}  &&
$\mathbf{0.664}$    &  $\mathbf{0.02}$  &  $0.09$ &
$\mathbf{0.661}$    &  $0.05$   &  $\mathbf{0.03}$ &
$\mathbf{0.661}$    &  $\mathbf{0.02}$  &  $\mathbf{0.03}$
\\ \cline{1-2} \cline{4-12}
\multicolumn{1}{ |c  }{} &
\multicolumn{1}{ c|  }{  Zafar et al.~(\citeyear{disparatemistreatment})}  &&
$0.660$    &   $0.06$    &   $0.14$  &
$0.662$    &   $0.03$    &   $0.10$  &
$0.661$    &   $0.03$    &   $0.11$
\\ \cline{1-2} \cline{4-12}
\multicolumn{1}{ |c  }{} &
\multicolumn{1}{ c|  }{  Zafar et al. Baseline~(\citeyear{disparatemistreatment})}  &&
$0.643$    &   $0.03$    &   $0.11$  &
$0.660$    &   $0.00$    &   $0.07$  &
$0.660$    &   $0.01$    &   $0.09$
\\ \cline{1-2} \cline{4-12}
\multicolumn{1}{ |c  }{} &
\multicolumn{1}{ c|  }{  Hardt et al.~(\citeyear{hardt})}  &&
$0.659$    &  $0.02$    &   $0.08$  &
$0.653$    &  $0.06$   &    $0.01$  &
$0.645$    &  $0.01$   &    $0.01$
\\ \cline{1-2} \cline{4-12}
\multicolumn{1}{ |c  }{} &
\multicolumn{1}{ c|  }{  \textbf{Vanilla Regularized Logistic Regression}}  &&
$\mathbf{0.672}$    &   $\mathbf{0.20}$    &   $\mathbf{0.30}$  &
$\mathbf{0.672}$    &   $\mathbf{0.20}$    &   $\mathbf{0.30}$  &
$\mathbf{0.672}$    &   $\mathbf{0.20}$    &   $\mathbf{0.30}$
\\ \cline{1-2} \cline{4-12}
\end{tabular}
}
\vspace{3mm}
\caption{Performance comparison on the COMPAS dataset. For the approaches in bold -- Accuracy, FPR difference and FNR difference are evaluated on the test set, averaging over five runs and using a 70-30 training/test split. The performance of the remaining three approaches is stated as reported in Zafar et al.~(\citeyear{disparatemistreatment}).} \label{table:comparison_results}
\end{table*}



\begin{figure*}[b]
  \includegraphics[scale=0.6]{compas0-400.png}
  \caption{COMPAS Dataset. Accuracy, FPR difference ($\mathbf{D_{FPR}}$), and FNR difference ($\mathbf{D_{FNR}}$) (all evaluated on the test set) of the learned classifier, as a function of the weight $c=c_1 = c_2 \geq 0$ placed on the fairness penalizer terms. On the left we use the Absolute Value Difference (AVD) penalizer, and the Squared Difference (SD) penalizer on the right, both as presented in Section~\ref{regularization}. ``Relaxed FPR/FNR Diff.'' plots the value of the relevant penalization term.} %In this particular run, parameters chosen for the absolute value relaxation were: $c=80, q_c=60$, and for the squared relaxation: $c=220, q_c=30$.}
  \label{fig:compas}
\end{figure*}


\subsection{Implementation}
\textbf{Our method} 
%We instantiate our method in the following way: Given dataset $Q$, we split it randomly into a training set $S$ (which we will use for learning) and a test set $T$ (which we will only use for reporting performance). 
For the purpose of comparison with  Zafar et al.~(\citeyear{disparatemistreatment}) and Hardt et al.~\cite{hardt} on the COMPAS data, we use a parameter $c$ to induce three possible combinations of weights on the FPR and FNR penalization terms: $c = c_1$ and $c_2 = 0$; $c_1 = 0$ and $c = c_2$; and $c = c_1 = c_2$. For the other three datasets, we consider only $c = c_1 = c_2$.\footnote{The reason for varying the values of $c$ in the training phase is since we shifted to a proxy problem, in which we rely on the distance from the decision boundary rather the actual classifications. 
%Our hope is that there is no need for a worst-case cross validation between all of the combinations of $c_1, c_2, c_3$, and that the training scheme we propose is sufficient. 
It is possible, of course, that even better results are attainable using our scheme with other combinations of $c_1, c_2$, and $q$.} To explore the accuracy/fairness trade-off curve for the relaxed optimization problem~(\ref{eq:2}), we train for different values of $c$, starting at $c=0$ (which is just standard logistic regression), and growing gradually.



Given a dataset $Q$ and fixing a $d_1, d_2 \in \{0, 1\}$ of interest, we use the following training scheme:
\begin{enumerate}
\item Split $Q$ at random into training set $S$ and test set $T$.
\item For each $c$, perform cross-validation on $S$ to select the corresponding best value $q_c$ for the regularization parameter.
\item For each $(c,q_c)$, let $\theta_c = \argmin\limits_{\theta} \text{Proxy}(\theta;S,c,c,q_c)$.
\item Select $\theta^* \in \argmin\limits_{\theta_c} \text{Objective}(\theta_c;S,d_1,d_2)$.
\item Evaluate performance using $\theta^*$ on test set $T$.
\end{enumerate}
We report the average of five such runs, each with a fresh training-test split.




%We instantiate our method by solving the relaxed optimization problem~(\ref{eq:2}), in place of the original, non-convex problem~(\ref{eq:1}).  
%We test our approach with three different combinations of weights on the penalization terms:
%\katrina{What are the $d$, and how are they related to the $c$s?}
%\begin{enumerate}
%\item FPR considerations only: $d_1 = 1, d_2 = 0$.
%\item FNR considerations only: $d_1 = 0, d_2 = 1$.
%\item Both FPR, FNR considerations, assigned similar significance: $d_1 = 1, d_2 = 1$.
%\end{enumerate}
%One could, of course, pick any other combination of the FPR and FNR penalty weights.

%\katrina{I don't understand how the below is distinct from the list above}
%Learning is done by training the parameters of a logistic regressor to solve~\ref{eq:2}, while picking the value of $c_1, %c_2$ as the following:
%\begin{enumerate}
%\item FPR considerations only: $c_1 = c \geq 0$, $c_2 = 0$.
%\item FNR considerations only: $c_1 = 0$, $c_2 = c \geq 0$.
%\item Both FPR, FNR considerations, assigned similar significance: $c_1 = c_2 = c \geq 0$
%\end{enumerate}



% We then cross-validate to pick the best $c_3$ (the weight on the standard $\ell_2$-regularization term) given $c$.\footnote{The reason for varying the values of $c$ in the training phase is since we shifted to a proxy problem, in which we rely on the distance from the decision boundary rather the actual classifications. 
%Our hope is that there is no need for a worst-case cross validation between all of the combinations of $c_1, c_2, c_3$, and that the training scheme we propose is sufficient. 
%It is possible, of course, that even better results are attainable using our scheme with other combinations of $c_1, c_2, c_3$.} For each such combination, we report results as the averages of multiple \katrina{how many?} different runs, each time splitting data randomly into training and test sets.
%\yahav{We need to shorten this description.}

We solve the relaxed convex optimization problem using the CVXPY solver. Due to stability issues with large training sets, we use a train/test split of 30-70 on the larger datasets, rather than 70-30 as on the COMPAS dataset\footnote{The code implementing our method can be found at https://github.com/jjgold012/lab-project-fairness}.

%
%
%We then report the results (as evaluated on the test set) attained by a regressor $\theta \in \mathbb{R}^d$ that minimizes (on the training set $S$) a weighted combination of the $0$-$1$ loss and the differences in FPR and FNR across populations:
%\begin{equation*}
%\begin{aligned}
%&\underset{\theta}{\text{argmin}}
%& & L_{S}^{0\text{-}1}(\theta) \\
%&&& + d_1|FPR_{A=0}(\theta;S)-FPR_{A=1}(\theta;S)| \\
%&&& + d_2|FNR_{A=0}(\theta;S)-FNR_{A=1}(\theta;S)|
%\end{aligned}
%\end{equation*}
%
%\katrina{What is $d_1$ vs. $c_1$ etc.?}



%For classification, we decided use a standard cut-off threshold of $c=0.5$. There are of course, further possible interactions between the FPR, FNR considerations, and picking a certain cut-off level. These are not straightforward, since  these interactions are data-specific. 



%allows for flexibility in picking the values of $c_1, c_2$, which reflect the significance we wish to place on the objectives of achieving accuracy, equal FPR, and equal FNR. As for $c_3$, we will want to find the value of it that achieves the best results, for any combined objective of accuracy and fairness defined by a specific selection of $c_1,c_2$. Therefore, given a specific selection of $c_1, c_2$, we apply cross-validation to select the value of $c_3$. 




We briefly describe the other algorithmic approaches to which we compare:\\
\textbf{Zafar et al.}~(\citeyear{disparatemistreatment}) performs optimization by considering a proxy for the bias: the covariance between the samples' sensitive attributes and the signed distance between the feature vectors of misclassified users and the classifier decision boundary.\\
\textbf{Zafar et al. Baseline}~(\citeyear{disparatemistreatment}) tries to enforce equal FP/FN rates on the different groups by introducing different penalties for misclassified data points with different sensitive attribute values during the training phase.\\
\textbf{Hardt et al.}~(\citeyear{hardt}) performs post-processing on a standard trained (unfair) logistic regressor, picking different decision thresholds for different groups, and possibly adding randomization.


\subsection{Experimental Results}

In what follows, we use the following notation, given a trained classifier $\hat{Y}$:
\begin{align*}
\mathbf{D_{FPR}}&=\left|FPR_{A=0}(\hat{Y})-FPR_{A=1}(\hat{Y})\right| \\ 
\mathbf{D_{FNR}}&=\left|FNR_{A=0}(\hat{Y})-FNR_{A=1}(\hat{Y})\right|
\end{align*}
The values $FPR_{A=0}(\hat{Y})$, $FPR_{A=1}(\hat{Y})$, $FNR_{A=0}(\hat{Y})$, $FNR_{A=1}(\hat{Y})$ are reported as evaluated on the test set.

\paragraph{The COMPAS Dataset\footnote{https://github.com/propublica/compas-analysis}} The Correctional Offender Management Profiling for Alternative Sanctions (COMPAS) records from Broward County, Florida 2013-2014, made available online by ProPublica, are perhaps the best-studied data in the context of fairness.  The goal in this scenario is to successfully predict recidivism within two years, based on features such as age, gender, race, number of prior offenses, and charge degree. The dataset contains 5,278 samples. The protected attribute in this scenario is race, where $A$ indicates black or white. We filtered the dataset using the same features as Zafar et al.~(\citeyear{disparatemistreatment}), to allow for comparison.

%\begin{table}[h]
%\centering
%\begin{tabularx}{\columnwidth}{c|c|c|c}
%\hline
%  &  Recid. ($y = 1$)        & No Recid.  ($y = 0$)       & Total \\ \hline
%Black &  $ 1661   $ & $ 1514 $ &  $ 3175 $ \\ \hline
%White &  $ 822   $  & $1281  $ &  $ 2103 $ \\ \hline
%Total &  $ 2483  $  & $2795 $ &  $ 5278 $ \\\hline
%\end{tabularx}
%\caption{Statistics of the ProPublica COMPAS data.} \label{table:compas-stats}
%\label{tab:stats}
%\end{table}
%\vspace{-1em}

%\begin{table}[h]
%\centering
%\begin{tabularx}{\columnwidth}{c|c}
%\hline
%Feature  &  Description \\ \hline
%Age Category &  $<25$, between $25$ and $45$, $>45$ \\
%Gender &  Male or Female \\
%Race &  White or Black \\
%Priors Count &  0--37 \\
%Charge Degree &  Misconduct or Felony \\
%\hline
%2-year-recid. & Whether or not the  \\
%(target feature)  & defendant recidivated within two years
%\end{tabularx}
%\caption{Description of features used from ProPublica COMPAS data.} \label{table:compas-features}
%\label{tab:features}
%\end{table}




\begin{table*}[t]
\centering
\caption{A description of the datasets used, along with parameters of the training procedure used for each.}
\label{table:datasets_description}
\begin{adjustbox}{max width=\textwidth}
\begin{tabular}{|l|l|l|l|l|l|l|l|}
\hline
\textbf{Dataset} & \textbf{No. Samples} & \textbf{No. Features} & \textbf{Train/Test Split} & \textbf{No. Repetitions} & \textbf{No. Folds in CV} & \textbf{Protected Feature} & \textbf{Target Variable} \\ \hline
COMPAS           & 5,278                     & 5                          & 70-30                     & 5                        & 5                                 & Race                       & 2-Year-Recidivism        \\ \hline
Adult            & 30,162                    & 10                         & 30-70                     & 5                        & 5                                 & Gender                     & Income Over/Under 50K    \\ \hline
Default          & 30,000                    & 23                         & 30-70                     & 5                        & 3                                 & Gender                     & Defaulting On Payments   \\ \hline
Admissions       & 20,839                    & 17                         & 30-70                     & 5                        & 3                                 & Race                       & Passing Bar Exam         \\ \hline
\end{tabular}
\end{adjustbox}
\end{table*}


\begin{table*}[t]
\centering
\resizebox{\textwidth}{!}{
\def\arraystretch{1.2}

\begin{tabular}{c c c | c | c | c || c | c | c || c | c | c |}

\cline{4-12}
&&&
\multicolumn{3}{ c|| }{\textbf{Adult Dataset}}&
\multicolumn{3}{ c|| }{\textbf{Default Dataset}}&
\multicolumn{3}{ c| }{\textbf{Admissions Dataset}}
\\ \cline{4-12}
%&&&
%\multicolumn{3}{ c|| }{\textbf{Both Considerations}}&
%\multicolumn{3}{ c|| }{\textbf{Both Considerations}}&
%\multicolumn{3}{ c| }{\textbf{Both Considerations}}
%\\ \cline{4-12}
&&&
 $\mathbf{Acc.}$ &  $\mathbf{D_{FPR}}$ &  $\mathbf{D_{FNR}}$ &  $\mathbf{Acc.}$ &  $\mathbf{D_{FPR}}$ &  $\mathbf{D_{FNR}}$ &  $\mathbf{Acc.}$ &  $\mathbf{D_{FPR}}$ &  $\mathbf{D_{FNR}}$
\\  \cline{4-12}
\vspace*{-0.5ex}
\\ \cline{1-2} \cline{4-12}
\multicolumn{1}{ |c  }{} &
\multicolumn{1}{ c|  }{  \textbf{Our Method (AVD Penalizers)}}  &&
$\mathbf{0.776}$    &  $\mathbf{0.00}$  &  $\mathbf{0.04}$ &
$\mathbf{0.807}$    &  $\mathbf{0.00}$   &  $\mathbf{0.01}$ &
$\mathbf{0.950}$    &  $\mathbf{0.01}$  &  $\mathbf{0.00}$
\\ \cline{1-2} \cline{4-12}
\multicolumn{1}{ |c  }{} &
\multicolumn{1}{ c|  }{  \textbf{Our Method (SD Penalizers)}}  &&
$\mathbf{0.783}$    &  $\mathbf{0.00}$  &  $\mathbf{0.09}$ &
$\mathbf{0.806}$    &  $\mathbf{0.01}$   &  $\mathbf{0.02}$ &
$\mathbf{0.950}$    &  $\mathbf{0.00}$  &  $\mathbf{0.00}$
\\ \cline{1-2} \cline{4-12}
\multicolumn{1}{ |c  }{} &
\multicolumn{1}{ c|  }{  \textbf{Vanilla Regularized Logistic Regression}}  &&
$\mathbf{0.800}$    &   $\mathbf{0.08}$    &   $\mathbf{0.39}$  &
$\mathbf{0.807}$    &   $\mathbf{0.01}$    &   $\mathbf{0.05}$  &
$\mathbf{0.951}$    &   $\mathbf{0.16}$    &   $\mathbf{0.02}$
\\ \cline{1-2} \cline{4-12}
\end{tabular}
}
\vspace{3mm}
\caption{Performance on the Adult, Loan Default, and Admissions datasets, penalizing for both FPR and FNR difference. Accuracy, FPR difference and FNR difference are evaluated on the test set, averaging over five runs and using a 30-70 training/test split.} \label{table:comparison_results_rest}
\end{table*}


In Table~\ref{table:comparison_results}, we compare the performance of our approach with that of three other techniques from the literature. Each method was trained based on logistic regression.  As a basis for comparison, we also present the performance of vanilla logistic regression, absent fairness considerations, with the regularization parameter selected via cross-validation.\footnote{Zafar et al.~(\citeyear{disparatemistreatment}) do not incorporate regularization in any of the approaches they report.}
%Results are reported as the averages of 5 different runs \katrina{Is that still correct?}, each time splitting data evenly and randomly into training and test sets. 
Results for Zafar et al., Zafar et al. baseline, and Hardt et al. appear here as reported in Zafar et al.~(\citeyear{disparatemistreatment}).\footnote{Our method selects the classifier based on the training set only and reports its performance over the test set. Results for the three other approaches, reported by Zafar et al.~(\citeyear{disparatemistreatment}), are based on tuning parameters after seeing the trade-off curve over the test set, and reporting according to the best selection of these parameters.}
%\katrina{Perhaps here is the right place for a footnote about the discrepancy with the Zafar baseline}

We find that the vanilla logistic regressor (absent fairness considerations) results in significant unfairness, as $\mathbf{D_{FPR}}=0.20$, and $\mathbf{D_{FNR}}=0.30$. The overall accuracy of this classifier measured on the test set was $0.672$.\footnote{Zafar et al.~(\citeyear{disparatemistreatment}) report a slightly different baseline of: Accuracy = 0.668, $\mathbf{D_{FPR}}=0.18$, $\mathbf{D_{FNR}}=0.30$.} Our SD penalization approach empirically achieves approximately the same accuracy as the Zafar et al.~(\citeyear{disparatemistreatment}) approach, with significantly better fairness. It is difficult to compare fairness-accuracy tradeoffs with the Hardt et al.~(\citeyear{hardt}) approach, since their accuracy is significantly lower than ours. A more direct comparison is possible by noting that our learned classifier can be post-processed to improve its fairness at a direct cost to accuracy. Hence, we can achieve accuracy of $0.659$ with $\mathbf{D_{FPR}} = \mathbf{D_{FNR}} = 0.01$, which compares very favorably with the Hardt et al. accuracy rate of 0.645 given the same FPR and FNR rates.\footnote{For completeness, we note that using a 50-50 training-test split (again not using the test set for parameter selection), our method (SD, both considerations) produces a classifier that provides: Accuracy = 0.659, $\mathbf{D_{FPR}} = 0.01, \mathbf{D_{FNR}} = 0.05$. This classifier can be post-processed to achieve rates of: Accuracy = 0.655, $\mathbf{D_{FPR}} = \mathbf{D_{FNR}} = 0.01$.}

Figure \ref{fig:compas} illustrates the accuracy/fairness trade-offs achievable using our scheme. Increasing the weight $c$ on the proxy fairness penalizers results in reducing their magnitude. The figure also illustrates how our relaxed penalizers succeed in tracking the real FPR and FNR differences. 
%
%
%\katrina{Must rewrite the following paragraph}
%We observe that our method succeeds in eliminating unfairness almost completely on the COMPAS dataset, while retaining most of the accuracy, when compared to the vanilla logistic regression. We achieve very low difference rates when penalizing for achieving each of the FPR and FNR criteria individually, and also for both. We achieve preferable results comparing to Zafar et al. and Zafar et al. baseline in all 3 scenarios, and also comparing to Hardt et al. in the settings of false positive/false negative considerations only. In the setting of both considerations - The Hardt et al. method removes a larger portion of the unfairness, however it results in major accuracy loss as it achieves accuracy rate of 0.645 in comparison to our method which results in accuracy of 0.665, retaining most of the original accuracy rate while removing most of the unfairness.




%The Hardt et al.~\cite{hardt} approach as reported removes a smaller portion of the bias in the different scenarios, however for FP/FN constraints alone, it provides higher accuracy rates. The Zafar et al.~(\citeyear{disparatemistreatment}) approach as reported retains significant bias (in most cases), but in some cases  achieves slightly superior accuracy rates to the methods above. 

%These performance comparisons are incomplete in the sense that each of the compared techniques has the potential to trade off between accuracy and fairness, using some degree of parameter tuning; what we report here is only one point on the achievable trade-off frontier for each algorithm. The ``correct'' trade-off, and, in particular, the best manner in which to weigh unfairness in the FPR against unfairness in the FNR, are matters of opinion. We have chosen to report our method's performance under parameters designed to very aggressively mitigate unfairness, at some cost to the accuracy.

%It would certainly be desirable to evaluate these and other approaches to fair learning on other datasets and on different tasks, particularly on larger datasets, which might afford both greater accuracy and better bias-reduction. The present empirical evaluations, however, suggest that our regularization-based approach provides a new tool worthy of consideration---we succeed in almost entirely eliminating bias on the hold-out set, at a modest price in terms of accuracy.

%Due to the fact that our true objective includes the original non-convex penalization terms, our approach does not carry any formal guarantees. However, the ease of implementation, generality, and empirical results are encouraging. Figure~\ref{fig:test1} illustrates the rate of convergence to a fair, accurate classifier on this dataset.
%In terms of computation costs, given that at each iteration we must calculate the gradient according to the FPR and FNR regularizers, we are required to predict the labels for the entire training set at each step. 
%However, this does not pose a computational burden, as it is already required by the (classic) gradient descent algorithm in our logistic regressor fitting scheme. Furthermore, when given a sufficiently large dataset (one or two orders of magnitude larger than the one currently available for the COMPAS scores data), this could be relaxed to sampling only a mini-batch of samples from the training data set at each iteration (much as is done in stochastic gradient descent).






\subsection{Additional Datasets}


Table~\ref{table:datasets_description} provides summary statistics on each of the datasets on which we tested our approach. We also briefly describe the datasets below. 


{\bf The Adult Dataset}\footnote{http://archive.ics.uci.edu/ml/datasets/Adult} is based on 1994 US Census data. The task we consider is to predict whether the income of each individual is over or under 50K dollars per year, based on features such as occupation, marital status, and education. The protected attribute selected in this task is gender. 

{\bf The Loan Default Dataset}\footnote{{\scriptsize https://archive.ics.uci.edu/ml/datasets/default+of+credit+card+clients}}
contains data regrading Taiwanese credit card users. The task we consider is to predict whether an individual will default on payments, based on features such as history of past payments, age, and the amount of given credit. The protected attribute is gender.

{\bf The Admissions Dataset}\footnote{http://www2.law.ucla.edu/sander/Systemic/Data.htm}
contains records of law school students who went on to take the bar exam. The task we consider is to predict whether a student will pass the exam based on features such as LSAT score, undergraduate GPA, and family income. The protected attribute is set to race.

Table~\ref{table:comparison_results_rest} describes the performance of our approach on these datasets, and Figures~\ref{fig:adult},~\ref{fig:default}, and~\ref{fig:lawschool} illustrate the fairness-accuracy trade-offs we achieve in each context. Overall, we see that unfairness is nearly eliminated while accuracy remains quite high. The dataset on which accuracy suffers most under our approach is the Adult dataset, which is also the dataset on which the vanilla regression is the most unfair.


\begin{figure*}[]
  \includegraphics[scale=0.6]{adult0-800.png}
  \caption{Adult Dataset. Fairness-Accuracy tradeoffs, as in Figure~\ref{fig:compas}.}
  \label{fig:adult}  
\end{figure*}



\begin{figure*}[]
  \includegraphics[scale=0.6]{default0-50.png}
  \caption{Loan Default Dataset. Fairness-Accuracy tradeoffs, as in Figure~\ref{fig:compas}.}
  \label{fig:default}
\end{figure*}



\begin{figure*}[]
  \includegraphics[scale=0.6]{admissions0-400.png}
  \caption{Admissions Dataset. Fairness-Accuracy tradeoffs, as in Figure~\ref{fig:compas}.}
  \label{fig:lawschool}
\end{figure*}




\section{Conclusion}

\begin{comment}
\begin{figure}
\includegraphics[width=\linewidth]{figs/beyond_tss_lesion.pdf}
\caption[]{End-to-End runtime lesion study of the entire MNIST dataset and the FMA featurized music dataset. Each of DROP's contributions provides a runtime improvement.}
\label{fig:beyond_lesion}
\end{figure}
\end{comment}



\section{Conclusion}
\label{sec:conclusion}

Advanced data analytics techniques must scale to rising data volumes. 
DR techniques offer a powerful toolkit when processing these datasets, with PCA frequently outperforming popular techniques in exchange for high computational cost. 
In response, we propose DROP, a new dimensionality reduction optimizer. 
DROP combines progressive sampling, progress estimation, and online aggregation to identify high quality low dimensional bases via PCA without processing the entire dataset by balancing the runtime of downstream tasks and achieved dimensionality. 
Thus, DROP provides a first step in bridging the gap between quality and efficiency in end-to-end DR for downstream \red{analytics}. 

%We revisit canonical operators for time series dimensionality reduction and the measurement study of~\cite{keogh-study}, and show that PCA is more effective than popular alternatives in the data mining literature often by a margin of over $2\times$ on average on gold-standard time series benchmark data sets with respect to output data dimension. More surprisingly, we empirically demonstrate that a small number of samples are sufficient to accurately characterize directions of maximum variance and obtain a high-quality low-dimensional transformation.




{\small
\bibliographystyle{ieee_fullname}
\bibliography{egbib}
}

\clearpage
\pagenumbering{roman}
\appendix

%%%%%%%%% TITLE
\title{Self-supervised Augmentation Consistency \\for Adapting Semantic Segmentation\\[1mm]\large -- Supplemental Material --}
\author{Nikita Araslanov$^1$ \hspace{1cm} Stefan Roth$^{1,2}$\\
$\ ^1$Department of Computer Science, TU Darmstadt \hspace{1cm} $\ ^2$ hessian.AI}

\maketitle

%%%%%%%%% BODY TEXT
\section{Overview}
In this appendix, we first provide further training and implementation details of our framework.
We then take a closer look at the accuracy of long-tail classes, before and after adaptation.
Next, we discuss our strategy for hyperparameter selection and perform a sensitivity analysis.
We also evaluate our framework using another segmentation architecture, FCN8s \citesupp{ShelhamerLD17}.
Finally, we discuss the limitations of the current evaluation protocol and propose a revision based on the best practices in the field at large.

\section{Further Technical Details}
\label{sec:supp_impl}
% !TEX root = ../supp.tex

\paragraph{Photometric noise.}
Recall that our framework uses random Gaussian smoothing, greyscaling and colour jittering to implement the photometric noise.
We re-use the parameters for these operations from the MoCo-v2 framework \citesupp{chen2020mocov2}.
In detail, the kernel radius for the Gaussian blur is sampled uniformly from the range $[0.1, 2.0]$.
Note that this does not correspond to the actual filter size.\footnote{The Pillow Library \citesupp{clark2015pillow} internally converts the radius $r$ to the box length as $L = \sqrt{3 * r^2 + 1}$.}
The colour jitter, applied with probability $0.5$, implements a perturbation of the image brightness, contrast and saturation with a factor sampled uniformly from $[0.6, 1.4]$, while the hue factor is sampled uniformly at random in the range of $[0.9, 1.1]$.
We convert a target image to its greyscale version with probability \num{0.2}.
\cref{fig:photometric} demonstrates an example implementation of this procedure in Python.

\begin{figure}[t]
\lstinputlisting[language=Python]{supp_sections/code/photometric.py}
\vspace{-0.5em}
\caption{\textbf{Python implementation of the photometric noise.}}
\label{fig:photometric}
\vspace{-0.5em}
\end{figure}

\myparagraph{Constraint-free data augmentation.}
Similarly to the multi-scale cropping of the target images, we scale the source images randomly with a factor sampled uniformly from $[0.5, 1.0]$ prior to cropping.
However, we do not enforce the semantic consistency for the source data, since the ground truth of the source images is available.
For both the target and source images we also use random horizontal flipping.
%Including the random flipping to the consistency loss for the target data may lead to further accuracy gains, although is not part of the current implementation yet.
We additionally experimented with moderate rotation (both with and without semantic consistency), but did not observe a significant effect on the mean accuracy.

\begin{table*}[t!]
\footnotesize
\begin{tabularx}{\linewidth}{@{}>{\centering\arraybackslash}p{1.5em}>{\centering\arraybackslash}p{1.5em}>{\centering\arraybackslash}p{1.5em}|S[table-format=2.1]@{\hspace{0.74em}}S[table-format=2.1]@{\hspace{0.74em}}S[table-format=2.1]@{\hspace{0.74em}}S[table-format=2.1]@{\hspace{0.74em}}S[table-format=2.1]@{\hspace{0.74em}}S[table-format=2.1]@{\hspace{0.74em}}S[table-format=2.1]@{\hspace{0.74em}}S[table-format=2.1]@{\hspace{0.74em}}S[table-format=2.1]@{\hspace{0.74em}}S[table-format=2.1]@{\hspace{0.74em}}S[table-format=2.1]@{\hspace{0.74em}}S[table-format=2.1]@{\hspace{0.74em}}S[table-format=2.1]@{\hspace{0.74em}}S[table-format=2.1]@{\hspace{0.74em}}S[table-format=2.1]@{\hspace{0.74em}}S[table-format=2.1]@{\hspace{0.74em}}S[table-format=2.1]@{\hspace{0.74em}}S[table-format=2.1]@{\hspace{0.74em}}S[table-format=2.1]@{\hspace{0.74em}}|S[table-format=2.1]@{}}
\toprule
CBT & IS & FL & {road} & {sidew} & {build} & {wall} & {fence} & {pole} & {light} & {sign} & {veg} & {terr} & {sky} & {pers} & {ride} & {car} & {truck} & {bus} & {train} & {moto} & {bicy} & {mIoU} \\
\midrule
 & & & 88.1 & 41.0 & 85.7 & 30.8 & 30.6 & 33.1 & 37.0 & 22.9 & 86.6 & 36.8 & 90.7 & 67.1 & 27.1 & 86.8 & 34.4 & 30.4 & 8.5 & 7.5 & 0.0 & 44.5 \\
\midrule
& & \cmark & 89.4 & 52.3 & 86.0 & \bfseries 34.0 & 32.6 & \bfseries 38.5 & 43.3 & 30.6 & 85.2 & 30.9 & 88.5 & 66.7 & 28.0 & 85.7 & 35.6 & 39.6 & 0.0 & 6.6 & 0.0 & 46.0 \\
& \cmark & & \bfseries 90.0 & 47.1 & 85.6 & 31.3 & 24.9 & 32.3 & 38.9 & 28.2 & \bfseries 87.3 & \bfseries 39.8 & 89.4 & \bfseries 67.7 & 28.6 & \bfseries 88.1 & 40.1 & 50.0 & 7.3 & 9.9 & 2.2 & 46.8 \\
\cmark & & & 89.3 & 39.0 & 85.1 & 33.2 & 26.1 & 32.4 & 41.8 & 25.2 & 86.3 & 27.4 & \bfseries 90.4 & 66.4 & 28.2 & 87.5 & 32.9 & 45.4 & 11.0 & 7.6 & 0.0 & 45.0 \\
\midrule
& \cmark & \cmark & 89.3 & 52.6 & 86.0 & 33.4 & 30.0 & 38.0 & 44.9 & 34.3 & 86.9 & 35.3 & 88.0 & 65.4 & 27.3 & 86.2 & 37.6 & 44.0 & 20.9 & 9.6 & 6.5 & 48.2 \\
\cmark & & \cmark & 89.3 & 52.2 & 86.1 & 34.2 & 31.5 & 37.0 & 43.4 & 36.3 & 85.2 & 30.7 & 86.6 & 66.2 & \bfseries 30.3 & 85.3 & 36.2 & 43.9 & \bfseries 29.2 & 6.8 & 8.6 & 48.4 \\
\cmark & \cmark & & 89.7 & 45.1 & 85.6 & 29.6 & 28.3 & 31.7 & 41.9 & 27.5 & 87.2 & 37.4 & 89.8 & 66.9 & 29.2 & 87.5 & 37.3 & 31.6 & 24.7 & 11.9 & 20.2 & 47.5 \\
\midrule
\cmark & \cmark & \cmark & \bfseries 90.0 & \bfseries 53.1 & \bfseries 86.2 & 33.8 & \bfseries 32.7 & 38.2 & \bfseries 46.0 & \bfseries 40.3 & 84.2 & 26.4 & 88.4 & 65.8 & 28.0 & 85.6 & \bfseries 40.6 & \bfseries 52.9 & 17.3 & \bfseries 13.7 & \bfseries 23.8 & \bfseries 49.9 \\
\bottomrule
\end{tabularx}
\caption{\textbf{Per-class IoU (\%)} on Cityscapes \emph{val} using a VGG-16 backbone in the GTA5 $\rightarrow$ Cityscapes setting. We study three components in more detail: class-based thresholding (CBT), importance sampling (IS) and the focal loss (FL). The mIoU of the settings in the last four rows are reproduced from the main text. Here, we elaborate on the per-class accuracy in a broader context of the supplementary experiments in the first four rows.}
\label{table:result_longtail}
%\vspace{-0.5em}
\end{table*}

\myparagraph{Training schedule.}
Our framework typically needs $150-200$K iterations in total (\ie~including the source-only pre-training) until convergence, as determined on a random subset of \num{500} images from the training set (see our discussion in \cref{sec:supp_eval} below).
This varies slightly depending on the backbone and the source data used.
This schedule translates to approximately \num{3} days of training with standard GPUs (\eg, Titan X Pascal with 12 GB memory) for both VGG-16 and ResNet-101 backbones.
Recall that we used \num{4} GPUs for our ResNet version of the framework, hence its training time is comparable to the VGG variant, which uses only \num{2} GPUs.
All our experiments use a constant learning rate for simplicity, but more advanced schedules, such as cyclical learning rates \cite{IzmailovPGVW18}, the cosine schedule \citesupp{chen2020mocov2,LoshchilovH17} or ramp-ups \cite{LaineA17}, may further improve the accuracy of our framework.


\section{Additional Experiments}
\label{sec:supp_class}
% !TEX root = ../supp.tex

\subsection{A closer look at long-tail adaptation}
Recall that our framework features three components to attune the adaptation process to the long-tail classes: class-based thresholding (CBT), importance sampling (IS) and the focal loss (FL), which we summarily refer to as the \emph{long-tail components} in the following.
Disabling the long-tail components individually is equivalent to setting $\beta \rightarrow 0$ for CBT, using uniform sampling of the target images instead of IS or assigning $\lambda$ to \num{0} for the FL.
Here, we extend our ablation study of the GTA5 $\rightarrow$ Cityscapes setup with VGG-16 (\cf \cref{table:ablation} from the main text) and experiment with different combinations of the long-tail components.
\cref{table:result_longtail} details the per-class accuracy of the possible compositions.

We observe that the ubiquitous classes -- ``road'', ``building'', ``vegetation'', ``sky'', ``person'' and ``car'' -- are hardly affected;
it is primarily the long-tail categories that change in accuracy.
Furthermore, our long-tail components are mutually complementary.
The mean IoU improves when one of the components is active, from $44.5\%$ to up to $46.8\%$.
It is boosted further with two of the components enabled to $48.4\%$, and reaches its maximum for our model, $49.9\%$, when all three components are in place.

We further identify the following tentative patterns.
FL tends to improve classes ``wall'', ``fence'' and ``pole''.
CBT increases the accuracy of the ``traffic light'' category (which has high image frequency and occupies only a few pixels), but also rare classes, such as ``rider'', ``bus'' and ``train'' benefit from CBT, especially in conjunction with IS.
IS also enhances the mask quality of the classes ``bicycle'' and ``motorcycle''.
Nevertheless, we urge caution against interpreting the results for each class in isolation, despite such widespread practice in the literature.
Today's semantic segmentation models do not possess the notion of an `ambiguous' class prediction and each pixel receives a meaningful label.
By the pigeon's hole principle, this implies that the changes in the IoU of one class have an immediate effect on the IoU of the other classes.
Therefore, the benefits of individual framework components have to be understood in the context of their aggregated effect on multiple classes, \eg~using the mean IoU.
For instance, consider the class ``train'' for which IS appears to also decrease the IoU: while CBT together with FL achieve $29.2\%$ IoU, adding IS decreases the IoU to $17.3\%$.
However, the IoU of other classes increases (\eg, ``motorcycle'', ``bicycle''), as does the mean IoU.
Furthermore, only few classes reach their maximum accuracy when we enable all three long-tail components.
Yet, it is the setting with the best \emph{accuracy trade-off} between the individual classes, \ie~with the highest mean IoU.
Overall, the long-tail components improve our framework by $5.4\%$ mean IoU combined, a substantial margin.

\begin{table*}[t!]
\footnotesize
\begin{tabularx}{\linewidth}{@{}l|S[table-format=2.1]@{\hspace{0.7em}}S[table-format=2.1]@{\hspace{0.7em}}S[table-format=2.1]@{\hspace{0.7em}}S[table-format=2.1]@{\hspace{0.7em}}S[table-format=2.1]@{\hspace{0.7em}}S[table-format=2.1]@{\hspace{0.7em}}S[table-format=2.1]@{\hspace{0.7em}}S[table-format=2.1]@{\hspace{0.7em}}S[table-format=2.1]@{\hspace{0.7em}}S[table-format=2.1]@{\hspace{0.7em}}S[table-format=2.1]@{\hspace{0.7em}}S[table-format=2.1]@{\hspace{0.7em}}S[table-format=2.1]@{\hspace{0.7em}}S[table-format=2.1]@{\hspace{0.7em}}S[table-format=2.1]@{\hspace{0.7em}}S[table-format=2.1]@{\hspace{0.7em}}S[table-format=2.1]@{\hspace{0.7em}}S[table-format=2.1]@{\hspace{0.7em}}S[table-format=2.1]@{\hspace{0.7em}}|c@{}}
\toprule
Method & {road} & {sidew} & {build} & {wall} & {fence} & {pole} & {light} & {sign} & {veg} & {terr} & {sky} & {pers} & {ride} & {car} & {truck} & {bus} & {train} & {moto} & {bicy} & {mIoU} \\
\midrule
\multicolumn{21}{@{}l}{\scriptsize \textit{GTA5 $\rightarrow$ Cityscapes}} \\
\midrule
Baseline (ours) & 76.7 & 28.2 & 74.4 & 12.7 & 19.0 & 27.2 & 28.7 & 12.2 & 77.0 & 18.0 & 70.6 & 54.8 & 20.6 & 79.6 & 19.0 & 19.2 & 20.6 & 27.9 & 11.2 & 36.7 { \scriptsize{(37.1)}} \\
%SAC (ours) & 87.3 & 47.1 & 84.1 & 29.5 & 26.5 & 23.9 & 42.7 & 30.8 & 86.8 & 42.5 & 87.5 & 60.2 & 30.1 & 83.0 & 28.3 & 38.2 & 28.2 & 33.4 & 44.6 & 49.2 { \scriptsize{(49.9)}} \\
SAC-FCN (ours) & 86.3 & 45.6 & 84.4 & 30.3 & 27.1 & 24.8 & 42.8 & 35.2 & 86.9 & 39.7 & 88.0 & 62.3 & 32.1 & 84.1 & 28.4 & 43.7 & 31.9 & 29.4 & 45.8 & 49.9 { \scriptsize{(49.9)}} \\
\midrule
\multicolumn{21}{@{}l}{\scriptsize \textit{SYNTHIA $\rightarrow$ Cityscapes}} \\
\midrule
Baseline (ours) & 50.7 & 23.8 & 60.9 & 1.8 & 0.1 & 27.7 & 10.5 & 15.7 & 60.1 & \textemdash & 72.4 & 50.1 & 16.0 & 66.5 & \textemdash & 13.7 & \textemdash & 8.5 & 26.8 & 31.6 { \scriptsize{(34.4)}} \\
%SAC (ours) & 80.8 & 39.6 & 81.9 & 18.0 & 1.1 & 27.8 & 35.2 & 28.0 & 79.0 & \textemdash & 80.6 & 61.5 & 23.1 & 81.8 & \textemdash & 36.6 & \textemdash & 32.4 & 55.7 & 47.7 { \scriptsize{(49.1)}} \\
SAC-FCN (ours) & 74.7 & 34.2 & 81.4 & 19.8 & 1.9 & 27.2 & 34.8 & 27.2 & 80.0 & \textemdash & 86.3 & 61.5 & 20.8 & 82.5 & \textemdash & 31.2 & \textemdash & 32.0 & 53.9 & 46.8 { \scriptsize{(49.1)}} \\
\bottomrule
\end{tabularx}
\caption{\textbf{Per-class IoU (\%)} on Cityscapes \emph{val} using VGG-16 with FCN8s. For reference, the numbers in parentheses in the last column report the mean IoU of the DeepLabv2 architecture (\cf \cref{table:result_gta_to_city,table:synthia_gta_to_city} from the main text).}
\label{table:result_fcn}
%\vspace{-0.5em}
\end{table*}

\begin{table}[t]
\centering
\setlength{\tabcolsep}{0.8em}%
\begin{tabularx}{0.7\linewidth}{@{}X@{}S[table-format=2.1]S[table-format=2.1]S[table-format=2.1]@{}}
\toprule
$\downarrow \zeta \hfill/\hfill \beta \rightarrow\;$ & {$0.0001$} & {$0.001$} & {$0.01$} \\
\midrule
$0.7$ & 47.9 & 48.8 & 46.7 \\
$0.75$ & 48.6 & 49.9 & 46.3 \\
$0.8$ & 48.2 & 49.8 & 45.6 \\
\bottomrule
\end{tabularx}
\caption{\textbf{Mean IoU (\%) on GTA5 $\rightarrow$ Cityscapes (val) with varying $\zeta$ and $\beta$.} Our framework maintains strong accuracy under different settings of $\zeta$ and $\beta$. Even with a poor choice (\eg, $\zeta = 0.8$, $\beta = 0.01$), it fares well \wrt the state of the art and outperforms many previous works (\cf \cref{table:result_gta_to_city} from the main text).}
\label{table:sensitivity}
\end{table}

\subsection{Hyperparameter search and sensitivity}
\label{sec:hyper_sensitivity}

To select $\zeta$ and $\beta$, we first experimented with a few reasonable choices ($\zeta \in (0.7, 0.8)$, $\beta \in (0.0001, 0.01)$)\footnote{While $\zeta$ may seem more interpretable (the maximum confidence threshold), a reasonable range for $\beta$ can be derived from $\chi_c$ for the long-tail classes, which is simply the fraction of pixels these classes tend to occupy in the image (see Eq.~3).} using a more lightweight backbone (MobileNetV2 \citesupp{Sandler2018:MIR}).
To measure performance, we use the mean IoU on the validation set (500 images from Cityscapes \textit{train}, as in the main text).

Here, we study our framework's sensitivity to the particular choice of $\zeta$ and $\beta$.
To make the results comparable to our previous experiments, we use VGG-16 and report the mean IoU on Cityscapes \textit{val} in \cref{table:sensitivity}.
We observe moderate deviation of the IoU \wrt $\zeta$.
A more tangible drop in accuracy with $\beta = 0.01$ is expected, as it leads to low-confidence predictions, which are likely to be inaccurate, to be included into the pseudo label.
We note that while a suboptimal choice of these hyperparameters leads to inferior results (with a standard deviation of $\pm 1.4$\% mIoU), even the weakest model with $\zeta = 0.8$ and $\beta = 0.01$ did not fail to considerably improve over the baseline (by $8.5$\% IoU, \cf \cref{table:result_gta_to_city} in the main text).


\begin{table*}[t!]
\footnotesize
\begin{tabularx}{\linewidth}{@{}l|S[table-format=2.1]@{\hspace{0.6em}}S[table-format=2.1]@{\hspace{0.6em}}S[table-format=2.1]@{\hspace{0.6em}}S[table-format=2.1]@{\hspace{0.6em}}S[table-format=2.1]@{\hspace{0.6em}}S[table-format=2.1]@{\hspace{0.6em}}S[table-format=2.1]@{\hspace{0.6em}}S[table-format=2.1]@{\hspace{0.6em}}S[table-format=2.1]@{\hspace{0.6em}}S[table-format=2.1]@{\hspace{0.6em}}S[table-format=2.1]@{\hspace{0.6em}}S[table-format=2.1]@{\hspace{0.6em}}S[table-format=2.1]@{\hspace{0.6em}}S[table-format=2.1]@{\hspace{0.6em}}S[table-format=2.1]@{\hspace{0.6em}}S[table-format=2.1]@{\hspace{0.6em}}S[table-format=2.1]@{\hspace{0.6em}}S[table-format=2.1]@{\hspace{0.6em}}S[table-format=2.1]@{\hspace{0.6em}}|c@{}}
\toprule
Method & {road} & {sidew} & {build} & {wall} & {fence} & {pole} & {light} & {sign} & {veg} & {terr} & {sky} & {pers} & {ride} & {car} & {truck} & {bus} & {train} & {moto} & {bicy} & {mIoU} \\
\midrule
\multicolumn{21}{@{}l}{\scriptsize \textit{GTA5 $\rightarrow$ Cityscapes}} \\
\midrule
SAC-FCN (ours) & 87.5 & 45.2 & 85.0 & 29.2 & 26.4 & 23.3 & 44.2 & 32.0 & 88.3 & 52.6 & 91.2 & 65.2 & 35.0 & 86.0 & 24.4 & 32.8 & 31.4 & 36.9 & 40.5 & 50.4 { \scriptsize{(49.9)}} \\
SAC-VGG (ours) & 91.5 & 53.9 & 86.6 & 34.1 & 31.5 & 36.8 & 47.2 & 36.9 & 85.1 & 38.0 & 91.1 & 68.7 & 31.9 & 87.4 & 31.0 & 46.7 & 22.6 & 24.2 & 24.0 & 51.0 { \scriptsize{(49.9)}} \\
SAC-ResNet (ours) & 91.8 & 54.3 & 87.4 & 36.2 & 30.2 & 43.7 & 49.7 & 42.1 & 89.3 & 54.3 & 90.5 & 71.8 & 34.9 & 89.8 & 38.8 & 47.3 & 24.9 & 38.3 & 43.8 & 55.7 { \scriptsize{(53.8)}} \\
\midrule
\multicolumn{21}{@{}l}{\scriptsize \textit{SYNTHIA $\rightarrow$ Cityscapes}} \\
\midrule
SAC-FCN (ours) & 66.9 & 25.9 & 80.8 & 12.1 & 2.0 & 24.4 & 37.1 & 27.5 & 78.8 & \textemdash & 88.9 & 63.9 & 25.0 & 84.7 & \textemdash & 27.4 & \textemdash & 36.9 & 50.2 & 45.8 { \scriptsize{(46.8)}} \\
SAC-VGG (ours) & 70.4 & 29.7 & 83.6 & 11.6 & 1.8 & 34.2 & 41.2 & 29.2 & 81.0 & \textemdash & 87.1 & 67.9 & 25.4 & 75.9 & \textemdash & 34.3 & \textemdash & 42.5 & 57.5 & 48.3 { \scriptsize{(49.1)}} \\
SAC-ResNet (ours) & 87.4 & 41.0 & 85.5 & 17.5 & 2.6 & 40.5 & 44.7 & 34.4 & 87.9 & \textemdash & 91.2 & 68.0 & 31.0 & 89.3 & \textemdash & 33.2 & \textemdash & 38.6 & 49.9 & 52.7 { \scriptsize{(52.6)}} \\
\midrule
\multicolumn{21}{@{}l}{\scriptsize \textit{Fully supervised (Cityscapes)}} \\
\midrule
DeepLab-ResNet \cite{ChenPKMY18} & 97.9 & 81.3 & 90.4 & 48.8 & 47.4 & 49.6 & 57.9 & 67.3 & 91.9 & 69.4 & 94.2 & 79.8 & 59.8 & 93.7 & 56.5 & 67.5 & 57.5 & 57.7 & 68.8 & 70.4 \\
FCN-VGG \citesupp{ShelhamerLD17} & 97.4 & 78.4 & 89.2 & 34.9 & 44.2 & 47.4 & 60.1 & 65.0 & 91.4 & 69.3 & 93.9 & 77.1 & 51.4 & 92.6 & 35.3 & 48.6 & 46.5 & 51.6 & 66.8 & 65.3 \\
\bottomrule
\end{tabularx}
\caption[\textbf{Per-class IoU (\%)} on Cityscapes \emph{test}]{\textbf{Per-class IoU (\%)} on Cityscapes \emph{test}. In the last column, the numbers in parentheses report the mean IoU on Cityscapes \emph{val} from the previous evaluation scheme (\cf \cref{table:result_gta_to_city,table:synthia_gta_to_city} from the main text) for reference. SAC-FCN denotes our VGG-based model with FCN8s \citesupp{ShelhamerLD17} from \cref{sec:fcn}.}
\label{table:result_city_test}
%\vspace{-0.5em}
\end{table*}

\subsection{VGG-16 with FCN8s}
\label{sec:fcn}
A number of previous works (\eg, \cite{MustoZ20,Yang_2020_ECCV,0001S20}) used the FCN8s \citesupp{ShelhamerLD17} architecture with VGG-16, as opposed to DeepLabv2 \cite{ChenPKMY18}, adopted in other works (\eg, \cite{KimB20a,Wang_2020_ECCV}) and ours.
Such architecture exchange appears to have been dismissed in previous work as minor, which used only one of the architectures in the experiments.
However, the segmentation architecture alone may contribute to the observed differences in accuracy of the methods and, more critically, to the improvements otherwise attributed to other aspects of the approach.
To facilitate such transparency in our work, we replace DeepLabv2 with its FCN8s counterpart in our framework (with the VGG-16 backbone) and repeat the adaptation experiments from \cref{sec:exp}, \ie~using two source domains, GTA5 and SYNTHIA, and Cityscapes as the target domain.
We keep the values of the hyperparameters the same, with an exception of the learning rate, which we increase by a factor of \num{2} to $5\times 10^4$.
\cref{table:result_fcn} reports the results of the adaptation, which clearly show that our framework generalises well to other segmentation architectures.
Despite the FCN8s baseline model (source-only loss with ABN) achieving a slightly inferior accuracy compared to DeepLabv2 (\eg, $31.6\%$ \vs~$34.4\%$ IoU for SYNTHIA $\rightarrow$ Cityscapes), our self-supervised training still attains a remarkably high accuracy, $46.8\%$ IoU (\vs~$49.1\%$ with DeepLabv2).
This is substantially higher than the previous best method using FCN8s with the VGG-16 backbone, SA-I2I \cite{MustoZ20}: $+3.4\%$ on GTA5 $\rightarrow$ Cityscapes and $+5.3\%$ on SYNTHIA $\rightarrow$ Cityscapes.


\section{Towards Best-practice Evaluation}
\label{sec:supp_eval}
% !TEX root = ../supp.tex

The current strategy to evaluate domain adaptation (DA) methods for semantic segmentation is to use the ground truth of \num{500} randomly selected images from the Cityscapes \textit{train} split for model selection and to report the final model accuracy on the \num{500} Cityscapes \textit{val} images \cite{LianDLG19}.
In this work, we adhered to this procedure to enable a fair comparison to previous work.
However, this evaluation approach is evidently in discord with the established best practice in machine learning and with the benchmarking practice on Cityscapes \cite{CordtsORREBFRS16}, in particular.

The test set is holdout data to be used only for an unbiased performance assessment (\eg, segmentation accuracy) of the final model \citesupp{0082591}.
While it is conceivable to consult the test set for verifying a number of model variants, such access cannot be unrestrained.
This is infeasible to ensure when the test set annotation is public, as is the case with Cityscapes \textit{val}, however.
Benchmark websites traditionally enable a restricted access to the test annotation through impartial submission policies (\eg, limited number of submissions per time window and user), and Cityscapes officially provides one.\footnote{\url{https://www.cityscapes-dataset.com}}

We, therefore, suggest a simple revision of the evaluation protocol for evaluating future DA methods.
As before, we use Cityscapes \textit{train} as the training data for the target domain, naturally without the ground truth.
For model selection, however, we use Cityscapes \textit{val} images with the ground-truth labels.
The holdout test set for reporting the final segmentation accuracy after adaptation becomes Cityscapes \textit{test}, with the results obtained via submitting the predicted segmentation masks to the official Cityscapes benchmark server.

An additional advantage of this strategy is a clear interpretation of the final accuracy in the context of fully supervised methods that routinely use the same evaluation setup.
Also note that Cityscapes \textit{val} contains images from different cities than Cityscapes \textit{train} (which are also different from Cityscapes \textit{test}).
Therefore, it is more suitable for detecting cases of model overfitting on particularities of the city, since the validation set was previously a subset of the training images.

For future reference, we evaluate our framework (both the DeepLabv2 and FCN8s variants) in the proposed setup and report the results in \cref{table:result_city_test}.
To ease the comparison, we juxtapose our validation results reported in the main text (from \cref{table:result_fcn} for FCN8s).\footnote{To our best knowledge, no previous work published their results in this evaluation setting before.}
As we did not finetune our method to Cityscapes \emph{val} following the previous evaluation protocol, we expect the test accuracy on Cityscapes \emph{test} to be on a par with our previously reported accuracy on Cityscapes \emph{val}.
The results in \cref{table:result_city_test} clearly confirm this expectation: the segmentation accuracy on Cityscapes \emph{test} is comparable to the accuracy on Cityscapes \emph{val} (SYNTHIA $\rightarrow$ Cityscapes) or even tangibly higher (GTA5 $\rightarrow$ Cityscapes).
We remark that the remaining accuracy gap to the fully supervised model is still considerable (70.4\% \vs 55.7\% IoU achieved by our best DeepLabv2 model and 65.3\% \vs 51.0\% IoU compared to our best FCN8s variant), which invites further effort from the research community.

We hope that future UDA methods for semantic segmentation will follow suit in reporting the results on Cityscapes \emph{test}.
Owing to the regulated access to the test set, we believe this setting to offer more transparency and fairness to the benchmarking process, and will successfully drive the progress of UDA for semantic segmentation, as it has done in the past for the fully supervised methods.


{\small
\bibliographystylesupp{ieee_fullname}
\bibliographysupp{supp_egbib}
}

\end{document}
