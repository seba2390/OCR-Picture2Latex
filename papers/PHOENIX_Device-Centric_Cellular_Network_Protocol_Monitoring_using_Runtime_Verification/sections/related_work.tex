\section{Related Work}

\paragraph{Runtime Monitors.} Extensive work has been done in
developing efficient runtime monitors using different types of
logic~\cite{basin2010monitoring, basin2010policy, basin2008runtime,
basin2015monitoring, monpoly, bauer2011runtime, du2015trace,
rosu2001synthesizing,soewito2009self, wpse}. However, all but
~\cite{soewito2009self, wpse} attempt to create a deployable
system which tries to apply runtime monitoring to web protocols.
In contrast, \system aims to be a deployable system, similar
to ~\cite{soewito2009self, wpse}, however, we apply runtime
monitoring to 4G LTE cellular networks. In addition, we apply three different
runtime monitor approaches while~\cite{soewito2009self, wpse}
only rely on automata based approaches. \system not only serves
as the runtime monitor but also provide the learning component
to generate signatures, including \pltl formulas.

\paragraph{Anomaly Detection in Cellular Devices.} Some work has been done to detect anomalies in cellular networks within the cellular device, precisely to discover the presence of fake base stations proposed by Dabrowski et al.~\cite{imsi_catcher_catchers}. In addition,  multiple apps have attempted to enable the detection of fake base stations using an application, but unfortunately do not generalize well~\cite{white_stingray}. In contrast to these attempts at anomaly detection, \system looks for specific patterns of message flow to detect specific attacks and provide a possible remedy.


\paragraph{Modification of Protocol.} Another approach researchers have leveraged to provide a defense mechanism is the modification of the protocol, such as in \cite{concealing_imsi_in_5g, privacy_5g, PMSI_desynchronization, imsicatcher_ccs15, wisec_root}. Out of these works, only \cite{wisec_root} provides a wide array of coverage while the others mainly focus on the IMSI catching attack. In contrast to other work, \system is the first warning system for cellular networks that provides the device more intelligence about other components of the network by only relying on message flows.
