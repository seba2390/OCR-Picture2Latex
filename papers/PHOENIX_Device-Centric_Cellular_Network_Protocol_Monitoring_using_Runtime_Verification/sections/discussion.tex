\section{Discussion}
We now discuss different salient aspects of \system.
%


\paragraph{Impacts of the attacks identified by \system.}
One of the questions that may come up is to ask how prevalent
are the different attacks and unsafe practices identified by \system.
Unfortunately, there are no public quantitative data on the attack
prevalence. We have seen anecdotal evidence
of some attacks and unsafe practices
in the literature. For instance, until two years ago, one of the major
US network operator did not use  encryption to protect their control-plane traffic.
Another major US operator used a persistent identifier in paging messages even though
this is discouraged. In addition, bidding down attacks are pretty commonplace
as warned by many media outlets and even Senators. We envision that \system can enable
more awareness on the security issues of cellular network and protect users from
such attacks. A possible instantiation of PHOENIX is to be deployed as cellular
network probes or honey pots that can log protocol sessions with undesired
behavior in order to perform statistical measurements, and in fact, its
application based instantiation has helped uncover unsafe practices on three
major U.S. cellular network providers.



\paragraph{Android and Qualcomm chipsets.}
Our current implementation of \system supports Qualcomm
baseband processors running on Android. We focus on Android
not only because it is the most popular mobile OS but also
it allows one to expose the cellular interface in the debug
mode with root access. We envision that OSes can expose the modem
information by requesting the permission from the user similar to how
other high privilege permissions can be granted to user
level applications. Additionally, in the future
we aim to extend this for other OSes
and baseband processors \cite{SCAT}.

\paragraph{\system deployed as an Android Application.}
\system{}'s application instantiation
is developed on top of MobileInsight which is written in Python.
This requires a python interpreter to be installed in Android.
Even then \system has shown its effectiveness without incurring a substantial
overhead. Going forward, however, writing the core functionality of \system in
C/C++ will not only positively
impact the performance but also the battery life of the device.




%have the desired effect on the user.

\paragraph{Supporting other protocol versions and device-specific attacks.}
\system is currently instantiated only for 4G LTE. Support of prior protocol
versions  (i.e., 2G/3G) and upcoming versions (i.e., 5G) can be effortlessly
added in the current instantiation by enhancing the current parsing
functionality of \system to include 2G, 3G, and 5G protocol packets. In addition,
signatures for device-specific attacks, due to implementation bugs in a device,
can be supported by the current version of
\system without any change.

\paragraph{User Study.} To be effective as a warning system,
it is paramount to ensure that warning messages \system generates
convey the right information regarding the warning. Performing
a user study with different warning designs is one of the effective
ways of obtaining some assurances on the efficacy of the warning messages.
To limit the scope of the paper to only the technical design of \system,
we leave the user study as future work.

\paragraph{False positives and the quality of signatures.}
The quality of synthesized signatures used in the empirical evaluation
of this paper heavily relies on the diversity of malicious traces used during training.
Having very similar malicious traces in training will likely induce the synthesizer
to come up with a non-generalized behavioral signature that will only be
effective in detecting undesired behavior similar to the ones observed in the malicious training traces and possibly,
a few of its variants. Such a non-generalized signature, however, is
unlikely to capture other diverse variants of the undesired behavior in question that are not present in the
malicious training traces.
This is because the synthesized signatures only guarantee \emph{observationally consistency},
that is, the synthesized signatures will not make any mistakes in correctly
classifying the sample traces given to it during training. Not having diverse
malicious traces during synthesis can thus induce a signature that incurs
false positives. In addition to this situation, false positives can also be incurred due to
having only coarse-grained information on the training traces (i.e., lacking fine-grained information).

% Recall that we use the temporal ordering of events as a proxy for detecting undesired behavior. For some
% attacks, however, \emph{only} the temporal ordering of events may not be a reliable proxy.
Let us consider the downgrade attack through fabricated $\mathsf{attachReject}$ message  as an example \cite{privacy_ndss16}.
In this attack, the adversary establishes a malicious base
station which emits a higher signal strength and then lures the victim device into connecting to the malicious base station.
Then, the adversary during the mutual authentication of the attach procedure injects a fabricated $\mathsf{attachReject}$ message,
resulting in the device to downgrade to an insecure protocol version (e.g., moving from 4G LTE to 3G).
Just by observing the temporal ordering of events, without taking into consideration the migration of the device from 4G LTE to 3G, the best possible
signature a synthesizer can come up with is the existence of the $\mathsf{attachReject}$ message. Such a signature
may induce false positives because $\mathsf{attachReject}$ messages can be sent by the network in benign situations.

In our experiments, due to random selection of benign sessions from real-life traces and $\mathsf{attachReject}$
messages being rare, none of the benign traces during
training had any $\mathsf{attachReject}$ messages, whereas all the negative traces at least had one.
As a result, the synthesizer rightfully came up with a signature saying an existence of
the $\mathsf{attachReject}$ message is an attack, especially because this is the smallest signature that is
observationally consistent. Due to our session-level trace generation approach discussed before,
we did not have faithful information about the downgrade of the protocol version.
Even if we had that precise information as part of the training traces, the synthesizer will give us a precise
signature according to which the monitor will identify the attack only
when the connection downgrade \emph{has already happened} instead of before protocol version downgrade happens.
We argue that a proactive monitor, which notifies the user as soon as an $\mathsf{attachReject}$ message has occurred instead
of waiting for a downgrade, is more effective in protecting the user even  at the cost of a
few false positives.  Since $\mathsf{attachReject}$ messages are indeed rare in practice,
corroborated by the MobileInsight traffic, such a trade-off is a reasonable
choice.

\paragraph{Alternate approaches.}
One alternate method for in-device monitoring of cellular traffic for
attack detection is to construct a state machine representing the desired
cellular protocol behavior from the 3GPP standard  specification \cite{3gpp, 3gpp12}.
At runtime, one could then flag any state transitions that are not allowed by this
reference state machine as potential attacks.
Such an approach, however, suffers from the following limitations:
(1) Manually constructing the protocol state machine by reading natural languages specification
is an error-prone and time-consuming ordeal;
(2) The \emph{abstract} protocol state machine given by the standard specification is not always prescriptive, leaving
aspects to be decided by implementors;
(3) Not all deviant state transitions necessarily signify an attack
and thus can induce a high number of false warnings.

In another approach, one may consider manually writing one small program
for each attack using simple if-then-else constructs that checks for a certain
attack in the trace. The attacks we consider here, however, would require
storing memory and would require intensive manual effort to identify the portion
of the traces that need to be stored. In contrast, our approach is completely
automated in figuring out what portion of the history to store for careful attack
identification without requiring human intervention. Such automation comes in handy
when new attacks are discovered and their signatures need to be deployed promptly.
