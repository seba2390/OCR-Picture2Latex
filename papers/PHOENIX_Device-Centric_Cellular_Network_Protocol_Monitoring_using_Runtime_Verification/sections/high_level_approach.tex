\section{Overview of \systemtitle}
In this section, we discuss the scope, threat model, challenges, and requirements of
a \system like system. We conclude by presenting two concrete instantiations of \system, namely,
as a warning system and a full-fledged defense.

% In this section, before we present the architecture of our
% in-device, real-time warning system \system, we first touch
% upon the scope, threat model, challenges, and requirements of
% our approach.
\subsection{Undesired Behavior and Scope}

In our presentation, we define an
\emph{undesired behavior/vulnerability}
broadly to include
inherent protocol flaws at the design-level,
an exploitable implementation vulnerability
of the baseband processor, an exploitable misconfiguration or
deployment choice of a network operator, and unsafe security
practices by a baseband manufacturer and network operator.
For instance, not using encryption for protecting traffic
is considered a vulnerability in our presentation. Even though null encryption
is permitted by the specification on the NAS layer \cite{3gppNAS},
we argue that this is an unsafe practice since subsequent NAS traffic (e.g., SMS over NAS~\cite{kim_ltefuzz_sp19, lteinspector})
would be exposed in plaintext. %Such a broad definition
%of attacks\, albeit undesired, is adopted just for the ease of presentation.

In this paper, we focus on the undesired behavior
% exploitable vulnerabilities
of the 4G LTE control-plane protocols, i.e., protocols
running in the NAS and RRC layers \cite{lteinspector, TORPEDO, privacy_ndss16, kim_ltefuzz_sp19, 5Gformal_authentication_basin,
5g_reasoner, lte_redirection, how_not_to_break_crypto}.
Among these attacks, we focus on attacks that are detectable from the
device's perspective and can be viewed as undesired outcomes of protocols'
state-machines. \emph{One distinct advantage of a device-centric attack detection mechanism
is that certain attacks necessarily cannot be observed by the network operators, which is
observable only from the device vantage point}. Examples of such attacks include ones that require
an adversary setting up a fake base station that lures the victim device and then launch an attack \cite{lteinspector,kim_ltefuzz_sp19, TORPEDO}.
%
Attacks that target other network components
or employ adversary's passive sniffing capabilities are out of scope as they are not detectable
through in-device traffic monitoring \cite{TORPEDO, alter,guti_reallocation_demystified_ndss18}.
In addition, the current instantiations of \system do not support attacks that require reasoning about quantitative
aspects (e.g., the number of certain messages received in a time window) of the protocol (e.g., ToRPEDO attack \cite{TORPEDO}).
Please consult Table \ref{app:cellular_network_extensive_list} in the Appendix for an
exhaustive list of \system supported and unsupported attacks.






\subsection{Threat Model}
We consider an adversary with the following capabilities:
(1) He has access to malicious cellular devices with legitimate
credentials;
(2) He can setup a rouge base station, cloning parameters of a legitimate one,
provides a higher signal strength than legitimate base stations within the vicinity.
(3) He can setup a base station which acts as a relay between the device
and legitimate base station, enabling him to drop, replay, and inject messages
at will while respecting cryptographic assumptions;
(4) For targeted attacks, we assume the attacker has access to the victim's
soft identity such as phone number and social network profile.
We assume that the device in which \system  runs is not compromised.

\subsection{Example: A Privacy Attack on Radio Link Failure (RLF) Report}
\label{sec:running_example}
% We now present a running example below.

In cellular networks, there is essentially
no authentication mechanism between a device and the base station during the connection initiation with the core network.
The device trusts the base station emitting the highest signal strength and
establishes an unsafe connection with it using unprotected RRC layer messages.
The base station acts as the trusted intermediary to facilitate communication between
the device and core network. Once the device and core network mutually authenticate each other,
they setup a security context making all the following control-plane messages to be encrypted and integrity protected.
One  such control-plane message is the \rlfReport which contains neighboring base stations'
signal strengths (and, optionally the device's GPS coordinates). This is used to identify
potential failures and aids when identifying coverage problems.

A privacy attack against
this RLF report message \cite{privacy_ndss16} proceeds by luring a cellular device to connect to a rogue base station, which exploits
the lack of authentication of initial broadcast messages as well as the unprotected RRC connection setup in the bootstrapping phase.
Before setting up the security context (with protected \securityModeCommand and \securityModeComplete messages)
at the RRC layer, the rogue base station sends an unprotected \ueInformationRequest
message to the device. This triggers the device to respond with a \rlfReport message (if it posses one) in the clear.
Since the RLF report includes signal strength measurements of neighboring cells (and optionally GPS coordinates),
the attacker can use that information to triangulate  the victim's location.

\subsection{Challenges}
Realizing the vision of \system has the following challenges.
(C-1) An attack detection mechanism like \system has to be lightweight,
otherwise substantial overhead can impede adoption due to negatively impacting the user's Quality-of-service (QoS).
(C-2) The system must be able to operate in a standalone fashion without
requiring assistance from network operators.
(C-3) The system must be attack- and protocol-agnostic, and amenable to extension
to new attacks discovered after its deployment and future protocol versions (e.g., 5G).
(C-4) The detection accuracy of the system must be high (i.e., low false positives and negatives).
If the system incurs a large number of false positives, then in its instantiation
as part of the baseband processor, can create interoperability issue. In the same vein, false positives in
\system{}'s instantiation as a warning system can overwhelm the user, making her ignore the raised warnings.
A large number of false negatives, on the other hand, makes the system prone to vulnerabilities.
(C-5) The attack detection system should detect the attack as soon as it is feasible when
the malicious session is underway. As an example, let us consider the above attack on RLF report.
If a detection system identifies the attack only after the device has already sent the \rlfReport message in the clear
to the adversary then the attack has happened and this reduces the impact of a detection system like \system. An
effective detection mechanism will identify the attack as soon as the device receives the
unprotected  \ueInformationRequest before security context establishment in which case
it can thwart the attack.




\subsection{\systemtitle Architecture}
We now discuss the architecture of \system in two settings:
(1) when it is deployed inside a baseband processor as a full-fledged
defense (see Figure \ref{fig:baseband_implementation});
(2) when it is deployed as an Android application and serves as a warning system (see Figure \ref{fig:overview}).


% We now discuss the architecture and workflow of \system.

%In this subsection, we will discuss the components in \system and then briefly describe its work-flow. See Figure~\ref{fig:overview} to see an overview of \system.

\begin{figure}[t]
	\centering
		\includegraphics[width=\columnwidth]{figures/phoenix_baseband.pdf}
		\caption{The envisioned architecture of \system inside a baseband processor.}
		\label{fig:baseband_implementation}
	\end{figure}


\begin{figure}[t]
\centering
	\includegraphics[width=\columnwidth]{figures/vigilant.pdf}
	\caption{The envisioned architecture of \system as an Android app.}
	\label{fig:overview}
\end{figure}



\paragraph{\systemtitle Components.}
In its purest form (Figure~\ref{fig:baseband_implementation}), \system has two main components, namely,
\textit{Attack Signature Database} and  \textit{Monitor}.


\textbf{Attack Signature Database.} \system expects a pre-populated attack signature database containing
the signatures of attacks it is tasked to detect. An example attack signature for the privacy attack on
RLF report above is: \emph{receiving the unprotected  \ueInformationRequest message
before security context establishment in a session.} Note that, a signature that requires the device to send a \rlfReport
message before security context establishment is ineffective as it detects the attack only after it has occurred. Signatures
can be generated by cellular network security experts, possibly in collaboration with an optional \system component that
can automatically generate candidate signatures from benign and attack traces.

\textbf{Monitor.} The \monitor  component analyzes the decoded messages and payloads (potentially,
received from the message extractor component discussed below in case of Android app deployment),
and matches them with its pre-populated undesired behavioral signature database.
In case a behavioral signature is identified, the action of \monitor component depends on the deployment scenario.
For its baseband processor deployment, the \monitor communicates the violation information to a corrective action
module who can either terminate the session or drop the particular message depending on the signature. In its
Android app deployment, it identifies which vulnerabilities have occurred and returns this information to the user along
with possible remedies, if any exists.


For its instantiation as an Android app, \system requires an additional component called \textit{message extractor}.
It gathers information about incoming/outgoing traffic (e.g., decoding a protocol message)
between the baseband processor and network. This collected information
(e.g., message type, payload) is then fed into the \textit{\monitor} component
for vulnerability detection. Note that, in the baseband deployment, \system does
not require this component as the baseband processor inherently decodes and
interprets the messages.




\paragraph{Workflow of \systemtitle.}
The workflow of \system deployed as an Android app is given below.
The baseband deployment does not require step (1) of the workflow.


(1) The \textit{message extractor} intercepts
an incoming/outgoing protocol message and decodes it.
(2) Pre-defined predicates over this message
(and, its payload) are then calculated and sent to the \monitor.
(3) The \monitor then classifies the ongoing trace as either benign or
vulnerable (with label). %along with the vulnerability label. %and returns this information.
(4) If \system identifies a vulnerability, it either drops the message/terminates
the connection when implemented inside a baseband processor, or alerts the user
of the undesired behavior with possible remedies when deployed as an Android
app (see Figure \ref{fig:phoenix_app_screenshots} for an example)
