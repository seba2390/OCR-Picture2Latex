

\begin{abstract}
	End-user-devices in the current cellular ecosystem are prone to many different
	vulnerabilities across different generations and protocol layers.
	Fixing these vulnerabilities retrospectively can be
	expensive, challenging, or just infeasible. A pragmatic approach for dealing with
	such a diverse set of vulnerabilities would be to identify attack attempts at runtime on the device side,
	and thwart them with mitigating and corrective actions. Towards this goal, in the paper we propose a general
	and extendable approach called \system for identifying n-day cellular network
	control-plane vulnerabilities as well as dangerous practices of network operators from
	the device vantage point. \system monitors the device-side cellular network traffic
	for performing signature-based unexpected behavior detection through lightweight
	runtime verification techniques. Signatures in \system can be manually-crafted by a
	cellular network security expert or can be automatically \emph{synthesized} using
	an optional component of \system, which reduces the signature synthesis problem to
	the \emph{language learning from the informant} problem. Based on the corrective
	actions that are available to \system when an undesired behavior is detected,
	different instantiations of \system are possible: a full-fledged defense
	when deployed inside a baseband processor; a user warning system when deployed
	as a mobile application; a probe for identifying attacks in the wild. One such
	instantiation of \system was able to identify all 15 representative
	n-day vulnerabilities and unsafe practices of 4G LTE networks considered in our
	evaluation with a high packet processing speed ($\sim$68000 packets/second) while
	inducing only a moderate amount of energy overhead ($\sim$4mW). %overhead on the device.
\end{abstract}
