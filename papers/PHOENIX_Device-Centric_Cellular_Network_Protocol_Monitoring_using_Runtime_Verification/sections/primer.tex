\section{Preliminaries}
\label{sec:preliminary}

In this section, we briefly overview the background material necessary to understand
our technical discussions.


\textbf{LTE Architecture.}
The LTE network ecosystem can be broken down into 3 main components
(See Figure~\ref{fig:lte_architecture}): User Equipment (\textbf{UE}),
Evolved Packet Core (\textbf{EPC}) and the Radio Access Network (\textbf{E-UTRAN}).
%\textbf{UE.}
The UE is a cellular device equipped with a SIM card. Each SIM card contains a
unique and permanent identifier known as the International Mobile Subscriber Identity (IMSI).
Also, each  device comes with a unique and device-specific identifier called
International Mobile Equipment Entity (IMEI). As both the IMSI and IMEI are unique
and permanent, their exposure can be detrimental to a user's privacy and security.
In LTE, the coverage area of a network can be broken down into hexagon cells where
each cell is powered by a base station (\textbf{eNodeB}). The network created by
the base stations powering up the coverage area and the UE is referred to as E-UTRAN.
The Evolved Packet Core (EPC) is the core network providing service to users.
The EPC can be seen as an amalgamation of services running together and continuously communicating with one another.

\begin{figure}[t]
	\centering
	\includegraphics[width=.8\columnwidth]{figures/System_Architecture.pdf}
	\caption{4G LTE Network Architecture.}
	\label{fig:lte_architecture}
\end{figure}

%\subsection{LTE Protocol Stack}

\textbf{LTE Protocols.}
The LTE network protocol consists of multiple layers, however,
this paper focuses only on the \emph{Network Layer}. This layer
consists of 3 protocols: NAS (Non-access Stratum), RRC (Radio Resource Control),
and IP (Internal Protocol). In this paper, we only explore NAS and RRC.
The NAS protocol is the logical channel between the UE and the EPC.
This protocol is in charge of highly critical procedures such as the
attach procedure which provides mutual authentication between the EPC and the UE.
The RRC protocol can be seen as the backbone of multiple protocols, including NAS.
In addition, RRC is the main channel between the UE and the eNodeB.

\textbf{Past-Time Propositional Linear Temporal Logic (\pltl).}
\label{sec:pltl_syntax_and_semantics}
%
\pltl extends propositional logic with past temporal operators
% \pltl is expressive enough in representing
% any cellular network control-plane attack that can be specified as a safety property~\cite{sistla1994safety}.
%This class captures a range of interesting problems as shown in model checking community.
and allows a succinct representation of the temporal ordering
of events.
%
Therefore, we use it as one of our vulnerability signature representation.
%
Here, we only provide a brief overview of \pltl but the detailed presentation can be found elsewhere \cite{pastpltl1985}.
% review the syntax and semantics of \pltl formulas.
% \begin{mydef}[Syntax]
The syntax of \pltl is defined inductively below where $\Phi, \Psi$ (possibly, with subscripts) are meta-variables
denoting well-formed \pltl formulas.
\begin{equation*}
  \Phi,\Psi  \quad::=\quad \top\quad|\quad \bot\quad|\quad p \quad|\quad  \circ^{1} \Phi_1 \quad|\quad \Phi_1 \circ^{2} \Psi_1
\end{equation*}
%
In the above presentation, $\top$ and $\bot$ refer to Boolean constants \true\; and \false, respectively.
The propositional variable $p$ is drawn from the set of a fixed alphabet
$\mathcal{A}$ (i.e., a set of propositions). \pltl supports unary operators
$\circ^{1}\in\{\neg, \yesterday, \once, \historically\}$,
 as well as binary operators  $\circ^{2}\in\{\wedge,\vee, \since\}$.
The Boolean logical operators include $\neg$ (not), $\vee$ (disjunction), and
$\wedge$ (conjunction) and the temporal operators include
$\yesterday$ (yesterday), $\once$ (once), $\historically$ (historically), and $\since$ (since).
We will now discuss the semantics of \pltl.

The Boolean logic operators in \pltl have their usual definition as in propositional logic.
We \emph{fix an alphabet $\mathcal{A}$} (i.e., a set of propositions)
for the PLTL formulas and consider it in the rest of the paper.
The semantics of \pltl is given with respect to a Kripke structure.
In a Kripke structure~\cite{Kri63}, a trace $\sigma$ is a finite sequence of states $(\sigma_0, \dots, \sigma_{n-1})$
that maps  propositions $p$ in $\mathcal{A}$ to  Boolean values
at each step $i\in[0,n-1]$\footnote{We write $i\in[0,n-1]$ to denote $0\leq i\leq n-1$.}
(i.e., $\sigma_i(p)\in\mathbb{B}$).
Although, the standard \pltl semantics are defined over (infinite) traces,
we are only required to reason about \emph{finite} traces.

\begin{mydef}[Semantics]\label{p_sem}
Given a \pltl formula $\Phi$ and a finite trace $\sigma = (\sigma_0, \dots, \sigma_{n-1})$ of length $n\in\mathbb{N}$, the satisfiability relation $(\sigma,i)\models$ $\Phi$ ($\Phi$
holds at position $i\in\mathbb{N}$) is inductively defined as follows:
  \begin{itemize}\setlength{\itemsep}{0em}
  \item $(\sigma,i)\models\top$ iff $\models\true$
 \item  $(\sigma,i)\models\bot$ iff  $\models\false$
  \item $(\sigma,i)\models p$ iff $\sigma_i(p) = \true$
  \item $(\sigma,i)\models \neg \Phi$ iff  $(\sigma,i)\not\models\Phi$
  \item $(\sigma,i)\models\Phi\wedge \Psi$ iff  $(\sigma,i)\models\Phi$  and $(\sigma, i)\models\Psi$
 \item  $(\sigma,i)\models\Phi\vee \Psi$ iff  $(\sigma,i)\models\Phi$  or $(\sigma, i)\models\Psi$
  \item $(\sigma,i)\models \yesterday \Phi$ iff  $i>0$ and $(\sigma,{i-1})\models\Phi$
 \item $(\sigma,i)\models \once \Phi$ iff  $\exists j\in[0, i].(\sigma, {j}) \models\Psi$
 \item $(\sigma,i)\models \historically \Phi$  iff  $\forall j\in[0, i].(\sigma, {j}) \models\Psi$.
  \item $(\sigma,i)\models \Phi \since \Psi$  iff $\exists j\in[0, i].(\sigma, {j})\models\Psi$ and $\forall k\in[j+1, i].(\sigma, k)\models\Phi$
  \end{itemize}
\end{mydef}

Intuitively, $\yesterday \Phi$ (read, Yesterday $\Phi$) holds in the current state if and only if
the current state is not the initial state and $\Phi$ held in the previous state. $\Phi \since \Psi$
holds  true currently if and only if
$\Psi$ held in any previous state (inclusive) and $\Phi$ held in all successive
states including the current one.
%\fa{suggestion:
%$\Psi$ held in the past and $\Phi$ holds true from then up now (current state).}
%
The rest of temporal operators \once (read, true once in the past) and \historically (read,
always true in the past)  can be defined through the following
equivalences: $\once\Phi\equiv (\top\since\Phi);
\historically\Phi\equiv \neg(\once (\neg\Phi))$.
