\section{Introduction}
Along with global-scale communication, cellular networks facilitate
a wide range of critical applications and services including
earthquake and tsunami warning system (ETWS), telemedicine, and smart-grid electricity distribution.
Unfortunately, cellular networks, including the most recent generation, have been often plagued with debilitating
attacks due to design weaknesses~\cite{lteinspector, TORPEDO, 5g_reasoner, 5Gformal_authentication_basin}
and deployment slip-ups \cite{privacy_ndss16, kim_ltefuzz_sp19, lte_redirection,how_not_to_break_crypto}.
Implications of these attacks range from intercepting and eavesdropping messages, tracking users' locations,
and disrupting cellular services, which in turn may severely affect the security and privacy of both
individual users and primary operations of a nation's critical infrastructures. To make matters worse,
vulnerabilities discovered in this ecosystem take a long time to generate and distribute patches as they not only
require collaboration between different stakeholders (e.g., standards body, network operator,
baseband processor manufacturer) but also
incur high operational costs. To make matters worse, different patches could potentially lead to
unforeseen errors if their integration is not accounted for.

In addition to it, although a majority of the existing work focus on discovering new attacks through analysis of the \emph{control-plane} protocol
specification or deployment \cite{lteinspector, TORPEDO, privacy_ndss16, kim_ltefuzz_sp19, 5Gformal_authentication_basin,
5g_reasoner, lte_redirection, how_not_to_break_crypto}, only a handful of efforts have focused on proposing defense mechanisms or
any apparatus to detect attack occurrences~\cite{imsi_catcher_catchers, FBSRADAR, mobile_self_defense, FBSleuth, wisec_root}.
Unfortunately, these proposed mechanisms are far from being widely adopted since they suffer from one of the following
limitations: \textbf{(i)} Requires modifications to an already deployed cellular network protocol \cite{wisec_root} which require
network operator cooperation;
\textbf{(ii)} Focuses on identifying particular attacks and hence are not easily extensible \cite{imsi_catcher_catchers, FBSRADAR,
mobile_self_defense, FBSleuth}; and
\textbf{(iii)} Fails to handle realistic scenarios (e.g., roaming) \cite{wisec_root}.

A pragmatic approach for protecting users and their devices from such a wide-variety
of vulnerabilities and dubious practices of the operators
(referred to as \textbf{\emph{undesired behavior}}\footnote{
	In our context, \emph{not} all undesired behavior are necessarily exploitable
	attacks. We also call some not-necessarily-malicious behavior (e.g., the use of null encryption by real network operators)
	undesired behavior if they can be detrimental to a user's privacy and security. In our exposition, we use
	attack, vulnerability, and undesired behavior, interchangeably.
	}
at the abstract in this paper) is to deploy a device-centric defense.
Such a defense, similar to an intrusion prevention system in principle, will monitor the network traffic at
runtime to identify undesired behavior and then take different corrective actions to possibly thwart it
(e.g., dropping a packet). In this paper, we focus on the core problem of
developing a general, lightweight, and extendable mechanism \system that can empower cellular devices
to detect various undesired behavior. %(e.g., vulnerabilities, unsafe practices employed by the stakeholders).
To limit the scope of the paper, we focus on monitoring the control-plane traffic for undesired behavior,
although \system is generalizable to data-plane traffic. Monitoring control-plane traffic is vital as
flaws in control-plane procedures, such as registration and mutual authentication, are
entry points for most attacks in both control- and data-plane procedures.

\system{}'s undesired behavior detection approach can induce different instantiations depending on the corrective
actions that are available to it. When deployed inside a baseband processor, \system can be used as a full-fledged
device-centric defense, akin to the pragmatic approach discussed above, that intercepts each message before getting
processed by the message handler and take corrective actions (e.g., drop the message, terminate the session) when it
identifies the message as part of an attack sequence. Alternatively, if \system is deployed as a mobile application that
can obtain a copy of the protocol message from the baseband processor, then one can envision building a warning system, which
notifies device owners when it detects that a protocol packet is part of an undesired behavior. Finally, \system can be
deployed and distributed as part of cellular network probes or honeypots that log protocol sessions with undesired behavior.
% that demonstrate an undesired behavior.




\paragraph{Approach.} In this paper, we follow a \emph{behavioral signature-based}
attack (or, generally undesired behavior) detection approach. It is enabled by the observation that a
substantial number of cellular network undesired behavior, which is detectable from the device's point-of-view,
often can be viewed as protocol state-machine bugs. Signatures of such undesired behavior can be constructed by
considering the relative temporal ordering of events (e.g., receiving an unprotected message
after mutual authentication).

Based on this above insight, we design a lightweight, generic, and in-device runtime
undesired behavior detection system
dubbed \system for cellular devices. In its core, \system{}'s detection has
two main components: (1) a pre-populated signature database for undesired behavior;
(2) a monitoring component that efficiently \emph{monitors} the device's cellular network
traffic for those behavioral signatures and takes corresponding corrective measures based
on its deployment (e.g., drop a message, log a message, warn the user).
%
%
Such a detection system is highly efficient and deployable as it
neither induces any extra communication overhead nor calls for any
changes in the cellular protocol. \system works with only a local view of the network,
yet is effective without provider-side support in identifying a wide array of undesired behavioral signatures.

For capturing behavioral signatures, we consider the following three different signature representations
that induce different tradeoffs in terms of space and runtime overhead, explainability, and detection accuracy:
(1) Deterministic Finite Automata (DFA);
(2) Mealy machine (MM) \cite{mealy1955method};
(3) propositional, past linear temporal (\pltl) \cite{ltl} formulas.
Cellular network security experts can add behavioral signatures in these representations
to \system{}'s  database. In case an expert is not
familiar with one of the above signature representations, they can get help/confirmation
from an \textbf{\emph{optional}} automatic signature synthesis component we propose.
We show that for all the above representations the automatic signature synthesis problem
can be viewed as an instance of the \emph{language learning from the informant} problem.
For DFA and MM representations, we rely on existing automata learning algorithms, whereas
for PLTL, we propose a new algorithm, an extension of
prior work \cite{learning_ltl}. For runtime monitoring of these signature representations
in \system, we use standard algorithms \cite{havelund2004}.
%\cite{rosu2001synthesizing}.



We consider two different instantiations for \system.
First, we implemented \system as an Android application and instantiated with the following monitors:
DFA-based, %$\system_\mathrm{DFA}$ ,
MM-based, %$\system_\mathrm{Mealy}$,
and PLTL-based. %$\system_\mathrm{PLTL}$.
In \system app, for capturing in-device cellular traffic,
%which requires root privilege in the device,
we enhanced the MobileInsight Android \cite{mobile_insight} application
to efficiently parse messages and invoke the relevant monitors.
Second, we implemented \system inside srsUE, distributed as part of
the open-source protocol stack
srsLTE \cite{gomez2016srslte}, powered by
the PLTL-based monitor---the most efficient in our evaluation,
to mimic \system{}'s deployment inside the baseband processor.

% allow \system to have baseband implementation corrective measures.

% \says{Omar}{Fix the following two paragraphs based on Mitziu's feedback.}
We evaluated \system{}'s Android app instantiation based on both testbed generated
and real-world network traffic in 3 COTS devices. In our evaluation with 15
existing cellular network attacks for 4G LTE,
we observed that in general all of the approaches were able to identify the existing
attacks with a high degree of success. Among the different monitors, however,
DFA on average produced a higher number of false positives
(21.5\%) and false negatives (17.1\%) whereas MM and PLTL
turn out to be more reliable; producing a significantly less number of false positives
($\sim\!\!$ 0.03\%) and false negatives ($\sim\!\!$ 0.01\%).
In addition, we observed that all monitors can handle a high number of
control-plane packets (i.e., 3.5K-369K packets/second).
We measured the power consumption induced by different monitors
and observed that on average, they all consume a moderate amount of energy ($\sim\!\!\!$ 2-6 mW).
Interestingly, we discover that \system, when powered by the PLTL-based monitor,
	produces no false warnings on real networks and in fact, it helped us discover
	unsafe network operator practices in three major U.S. cellular network providers.
% Note that, when evaluating \system in the wild with 3 major US network operators, we identified misconfigured
% base stations belonging to two operators whose misbehavior triggered an undesired behavior.
Finally, we evaluated \system instantiation as part of  srsUE \cite{gomez2016srslte} with testbed
generated traffic and observed that it only incurs a small memory overhead (i.e., 159.25 KB).


\paragraph*{Contributions.}
In summary, the paper makes the following  contributions:
\begin{itemize}\setlength{\itemsep}{0em}
	\item %We investigate the plausibility of having a general in-device
	%monitor for detecting cellular network control-plane attacks.
	We design an in-device, behavioral-signature based
	cellular network control-plane undesired behavior detection system called \system.
	We explore the design space of developing such a vulnerability detection system and consider
	different trade-offs.
	\item We implement \system as an Android app, which during our evaluation with
	3 COTS cellular devices in our testbed has been found to be effective
	in identifying 15 existing 4G LTE attacks while incurring a small overhead.

	\item We implement \system by extending srsUE \cite{gomez2016srslte}---mimicking  a full-fledged
	defense, and show its effectiveness at preventing
	attacks. %when capable of dropping messages.
	\item We finally show how one could automatically synthesize behavioral signatures \system expects by
	posing it as a learning from an informant problem \cite{informant_learning} and solve it with different
	techniques from automata learning and syntax-guided synthesis.
	% In particular, we present a
	% novel algorithm for synthesizing a \pltl vulnerability signature from
	% benign and vulnerable traces.
\end{itemize}
