\section{Conclusion}
In this paper, we develop \system, a general approach which can efficiently monitor
a device's cellular network traffic and identify the presence of attacks. We achieve
this by instantiating two different implementations
of PHOENIX: a runtime monitor within an Android application, allowing the
cellular device to reason about malicious message flow and alert the user;
A modified version of srsUE \cite{gomez2016srslte} powered by a runtime monitor
allowing it to detect vulnerabilities and prevent potential undesired behavior.

Overall we observe that our best approach with PLTL can correctly identify all the
15 n-day 4G LTE attacks and unsafe practices used in the evaluation section
with a high packet processing speed ($\sim$68000 packets/second), while inducing
a moderate energy ($\sim$4mW) and negligible memory overhead ($0.04\%$) on the device.


\section*{Acknowledgment}
We thank the anonymous reviewers for their valuable suggestions. This
work was funded by DARPA YFA contract no. D19AP00039. The views and conclusions
contained herein are those of the authors and should not be interpreted as
necessarily representing the official policies or endorsements of DARPA.
