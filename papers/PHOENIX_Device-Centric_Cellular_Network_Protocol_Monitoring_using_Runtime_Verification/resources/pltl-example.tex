\begin{example}
For a better understanding of how \pltl represents a vulnerability signature, let us consider 
the following  measurement report attack signature $\Phi$:
%
\begin{equation*}
\resizebox{.99\columnwidth}{!}{$\measurementReport \Rightarrow \yesterday (\neg \rrcConnectionRequest \since \securityModeComplete)$}
\end{equation*}
%
This vulnerability signature states that if there is a \textsf{measurement report} message sent by the UE then it is not the case
that UE has started a new session (i.e., sent an \rrcConnectionRequest message)
 since the last time a security context was established (i.e., sent a \securityModeComplete message).
%\[
%\Phi = 
%\measurementReport \Rightarrow \yesterday (\neg \rrcConnectionRequest \since \securityModeComplete)
%\]
%
%
In Table~\ref{table:attk-eg}, we show evaluation 
of this vulnerability signature $\Phi$ using few example positive (benign) and 
negative (malicious) traces.
%
We can see that the vulnerability signature $\Phi$ is only falsified for an attack trace and 
remains true for all positive traces that exhibit benign traffic. 
%
Thus, a monitor built using $\Phi$ correctly alarms the user about the occurrence of measurement report attack behavior. 
%
\begin{table}[]
    \centering
\resizebox{0.99\columnwidth}{!}{
    \begin{tabular}{|c|c|l|l|l|}
        \hline
        & &\measurementReport & \rrcConnectionRequest &  \securityModeComplete \\
        \hline
        $\sigma$ & $\Phi$ & $\sigma_{0}\sigma_{1}\sigma_{2}\sigma_{3}$ & $\sigma_{0}\sigma_{1}\sigma_{2}\sigma_{3}$ & $\sigma_{0}\sigma_{1}\sigma_{2}\sigma_{3}$ \\       
        \hline
        P1&\true & \texttt{0000} & \texttt{0100} & \texttt{1000} \\
        P2&\true & \texttt{0110} & \texttt{0100} & \texttt{1101}\\
        P3&\true & \texttt{0011} & \texttt{0110} & \texttt{0110} \\ \hline
        N4&\false & \texttt{1011} & \texttt{0110} & \texttt{1001}\\
        N5&\false & \texttt{1001} & \texttt{0100} & \texttt{1001}\\
        N6&\false & \texttt{0001} & \texttt{0110} & \texttt{0001}\\
        \hline      
        \end{tabular}    
}
\caption{$\Phi$ over positive traces [P1-P3] and negative traces [N4-N6].}
\label{table:attk-eg}
\end{table}

\end{example} 
