% \documentclass{article}
\documentclass[conference,compsoc,10pt]{IEEEtran}

% if you need to pass options to natbib, use, e.g.:
%     \PassOptionsToPackage{numbers, compress}{natbib}
% before loading neurips_2023


% ready for submission
% \usepackage{iclr2024_conference,times}


% to compile a preprint version, e.g., for submission to arXiv, add the
% [preprint] option:
%     \usepackage[preprint]{neurips_2023}


% to compile a camera-ready version, add the [final] option, e.g.:
%     \usepackage[final]{neurips_2023}


% to avoid loading the natbib package, add option nonatbib:
%    \usepackage[nonatbib]{neurips_2023}

\ifCLASSOPTIONcompsoc
  % IEEE Computer Society needs nocompress option
  % requires cite.sty v4.0 or later (November 2003)
  % \usepackage[nocompress]{cite}
  \usepackage[numbers]{natbib}
\else
  % normal IEEE
  % \usepackage{cite}
  \usepackage{natbib}
\fi


\usepackage[utf8]{inputenc} % allow utf-8 input
\usepackage[T1]{fontenc}    % use 8-bit T1 fonts
\usepackage{hyperref}       % hyperlinks
\hypersetup{
    colorlinks,
    linkcolor={red!70!black},
    citecolor={green!70!black},
    urlcolor={blue!80!black}
}
\usepackage{url}            % simple URL typesetting
\usepackage{booktabs}       % professional-quality tables
\usepackage{amsfonts}       % blackboard math symbols
\usepackage{nicefrac}       % compact symbols for 1/2, etc.
\usepackage{microtype}      % microtypography
\usepackage{xcolor}         % colors
\usepackage{enumitem}
\usepackage{amsmath}
\usepackage{amssymb}
\usepackage{mathtools}
\usepackage{tablefootnote}
\usepackage{amsthm}
\usepackage{bbm}
\usepackage{bm}
\usepackage{todonotes}
\usepackage{dblfloatfix}

\usepackage[capitalize]{cleveref}
\usepackage{multirow}
\usepackage{makecell}
\usepackage{wrapfig}
\usepackage{tikz}
\usepackage{xspace}
\usepackage{caption}

\renewcommand{\cellalign}{vh}
\renewcommand{\theadalign}{vh}

\DeclareMathOperator*{\argmax}{arg\,max}
\DeclareMathOperator*{\argmin}{arg\,min}

\newcommand{\hullvolume}[0]{$V_{\text{hull}}$}
\newcommand{\hullarea}[0]{$A_{\text{hull}}$}
\newcommand{\quality}[0]{$Q$}
\newcommand{\efficiency}[0]{$E$}
\newcommand{\robustness}[0]{$R$}

\title{Mark My Words: Analyzing and Evaluating Language Model Watermarks}


% The \author macro works with any number of authors. There are two commands
% used to separate the names and addresses of multiple authors: \And and \AND.
%
% Using \And between authors leaves it to LaTeX to determine where to break the
% lines. Using \AND forces a line break at that point. So, if LaTeX puts 3 of 4
% authors names on the first line, and the last on the second line, try using
% \AND instead of \And before the third author name.

\author{\IEEEauthorblockN{Julien Piet}
\IEEEauthorblockA{\textit{UC Berkeley} }
\and
\IEEEauthorblockN{Chawin Sitawarin}
\IEEEauthorblockA{\textit{UC Berkeley} }
\and
\IEEEauthorblockN{Vivian Fang}
\IEEEauthorblockA{\textit{UC Berkeley} }
\and

\IEEEauthorblockN{Norman Mu}
\IEEEauthorblockA{\textit{UC Berkeley} }
\and
\IEEEauthorblockN{David Wagner}
\IEEEauthorblockA{\textit{UC Berkeley} }
}


\newcommand{\jp}[1]{{\color{green!15!blue}[#1] --Julien}}
\newcommand{\vf}[1]{{\color{violet}[#1] --Vivian}}
\newcommand{\norm}[1]{{\color{green} N: #1}}
\newcommand{\chawin}[1]{{\color{orange} C: #1}}
\newcommand{\dave}[1]{{\color{purple} D: #1}}
\newcommand{\itemlm}{{15pt}}
% \newcommand{\benchmarkname}{{MarkMyWords}}
\newcommand{\benchmarkname}{\textsc{MarkMyWords}}
\newcommand{\scott}{\citet{aaronson_watermarking_2022}}
\newcommand{\mylink}[1]{\url{#1}}
% \newcommand{\mylink}[1]{\texttt{this anonymized link\xspace}}

\begin{document}


\maketitle
\thispagestyle{plain}
\pagestyle{plain}


\begin{abstract}
  The capabilities of large language models have grown significantly in recent years and so too have concerns about their misuse. 
  In this context, the ability to distinguish machine-generated text from human-authored content becomes important. 
  Prior works have proposed numerous schemes to watermark text, which would benefit from a systematic evaluation framework.
  This work focuses on text watermarking techniques --- as opposed to image watermarks --- and proposes \benchmarkname{}, a 
  comprehensive benchmark for them under different tasks 
  as well as practical attacks. We focus on three main metrics: quality, size (e.g. the number of tokens needed to detect a watermark), and tamper-resistance. 
  Current watermarking techniques are good enough to be deployed:~\citet{kirchenbauer_watermark_2023} can 
  watermark Llama2-7B-chat with no perceivable loss in quality, the watermark can be detected with fewer than 100 tokens, and the scheme offers good tamper-resistance to simple attacks. 
  We argue that watermark indistinguishability, a criteria emphasized in some prior works, is too strong a requirement: schemes that slightly modify logit 
  distributions outperform their indistinguishable counterparts with no noticeable loss in generation quality. 
  We publicly release our benchmark (\mylink{https://github.com/wagner-group/MarkMyWords})
\end{abstract}

Reinforcement learning has achieved great success in areas such as Game-playing \citep{silver2018general,vinyals2019grandmaster}, robotics \cite{kober2013reinforcement}, large language models \citep{ouyang2022training}, etc.
However, due to safety concerns or physical limitations, in some real-world reinforcement learning problems, we must consider additional constraints that may influence the optimal policy and the learning process \citep{garcia2015comprehensive}.
% For example, a robotic arm must not take actions that may cause harm to itself or the environments.
A standard framework to handle such cases is the constrained Markov Decision Process (CMDP) \citep{altman1999constrained}.
Within the CMDP framework, the agent has to maximize
the expected cumulative reward while
obeying a finite number of constraints, which are usually in the form of expected cumulative cost criteria.

However, we are sometimes concerned with the problem with a continuum of constraints.
For example,
the constraints we meet might be time-evolving or subject to uncertain parameters, which
cannot be formulated as an ordinary CMDP
(see Examples \ref{Example_Time_Evolving} and  \ref{Example_Uncertain}).
In this paper we would study a generalized CMDP  
to address the above problem.  Because the constraints are not only infinite-number but also lie
in a continuous set,
the generalization is not trivial. Fortunately, we find that we can borrow the idea behind semi-infinite programming (SIP) \citep{remez1934determination, hettich1993semi} to deal with the semi-infinite constraints.
Accordingly, we propose \emph{semi-infinitely constrained Markov decision processes} (SICMDPs)
as a novel complement to the ordinary CMDP framework.
%More specifically,  an SICMDP model %, we consider 
%contains a continuum of constraints whereas an ordinary CMDP contains a finite number of constraints. 

%This generalization is natural but not trivial. However, we can brows the idea  
%The idea is quite natural and can be backtracked
%to the practice of extending linear programming to linear semi-infinite programming (LSIP) %\cite{remez1934determination, GobernaLSIO1998}.
%In addition, 
%As a complementary approach to the ordinary CMDP framework, 
%SICMDP can be used to model these problems  which cannot be described by a finite number of constraints
%that are not covered by .
%For example,
%the restrictions we consider can be time-evolving or subject to uncertain parameters
%, thus
%cannot be described by a finite number of constraints but a continuum of constraints 
%(see Examples \ref{Example_Time_Evolving} and  \ref{Example_Uncertain}).

We also present two reinforcement learning algorithms to solve SICMDPs called SI-CRL and SI-CPO, respectively.
SI-CRL is a model-based reinforcement learning algorithm designed for tabular cases, and SI-CPO is a policy optimization algorithm for non-tabular cases.
% and analyze its performance both theoretically and empirically.
The main challenge is that we need to deal with a continuum of constraints, thus reinforcement learning algorithms for ordinary CMDPs do not work anymore.
In SI-CRL, we tackle this difficulty by first transforming the reinforcement learning problem to an equivalent LSIP problem, which can then be solved using methods in the LSIP literature like the dual exchange methods \citep{Hu1990,reemtsen1998numerical}.
In SI-CPO, we resort to the idea of cooperative stochastic approximation developed in \cite{lan2020algorithms, wei2020comirror}.
As far as we know, we are the first to introduce tools from semi-infinitely programming (SIP) into the reinforcement learning community for solving constrained reinforcement learning problems.

% To the best of our knowledge, we are the first to apply tools from semi-infinitely programming (SIP) to solve reinforcement learning problems.
Furthermore, we give theoretical analysis for both SI-CRL and SI-CPO.
We decompose the error of SI-CRL into two parts: the statistical error from approximating the true SICMDP with an offline dataset and the optimization error due to the fact that the solution of the LSIP problem obtained by the dual exchange method is inexact.
On the optimization side, we show that the iteration complexity of SI-CRL is $O\left(\left\{\mathrm{diam}(Y)L\sqrt{|\gS|^2|\gA|m}/\left[(1-\gamma)\epsilon\right]\right\}^m\right)$.
On the statistical side, we show that the sample complexity of SI-CRL is $\widetilde O\left(\frac{|S|^2|A|^2}{\epsilon^2(1-\gamma)^3}\right)$ if the offline dataset is generated by a generative model, and $\widetilde O\left(\frac{|S||A|}{\nu_{\min} \epsilon^2(1-\gamma)^3}\right)$ if the dataset is generated by a probability measure $\nu$ as considered in \cite{chen2019information}.
Here $\widetilde O$ means that all logarithm terms are discarded.
For SI-CPO, things become a little more complicated because other than the statistical error and the optimization error, we also need to consider the function approximation error, which comes from imperfect policy parametrizations.
It is shown if the function approximation error can be controlled to $O(\epsilon)$ order, the iteration complexity of SI-CPO is $\widetilde{O}\left(\frac{1}{\epsilon^2(1-\gamma)^6}\right)$ and the sample complexity of SI-CPO is $\widetilde{O}(\frac{1}{\epsilon^4(1-\gamma)^{10}})$.
Here our iteration complexity bound is equivalent to a typical $\widetilde O(1/\sqrt{T})$ global convergence rate.

We perform a set of numerical experiments to illustrate the SICMDP model and validate our proposed algorithms.
Specifically, we examine two numerical examples, namely the discharge of sewage and ship route planning.
Through the discharge of sewage example, we show the advantage of the SICMDP framework over the CMDP baseline obtained by naive discretization in modeling realistic sequential decision-making problems.
Moreover, we demonstrate the effectiveness of the SI-CRL and SI-CPO algorithms in such tabular environments. 
In the ship route planning example, we illustrate the benefits of the SICMDP framework and the ability of the SI-CPO algorithm to address complex continuous control tasks involving continuous state spaces with modern deep reinforcement learning techniques.

% In summary, our contributions are listed as follows.
% First, we present the SICMDP model, which can be viewed as a generalization of the ordinary CMDP model.
% Second, we propose an algorithm to perform reinforcement learning for SICMDPs, which is called SI-CRL, and we believe that we are the first to apply tools from SIP
% to solve reinforcement learning problems.
% Third, we give a theoretical analysis of SI-CRL and identify both its sample complexity and iteration complexity.
% In addition, we perform numerical experiments to illustrate the SICMDP model and validate the SI-CRL algorithm.
% \{This paragraph can be removed!!! \}






% Panoptic segmentation

% 3D segmentation

% Multi-object tracking

% Online 3D panoptic:

% PanopticFusion: (IROS 2019)
% https://arxiv.org/pdf/1903.01177.pdf
%
% - most similar to ours
% - PSPNet + M-RCNN + 2D fusion
% - volumetric mapping, 
% - greedy matching with IoU -> optimal only with 0.5 threshold
% - voxel & class weighting
% - CRF regularisation
%
% - good:
%
% - bad:
%  - CRF post-processing step
%  - greedy data-association
%    - can't be tuned for lower overlap ratios -> has to have high framerate, large changes in viewpoint could break this
%    - IoU: sensitive to 2D labels projecting over object borders (CRF and voxel weighting seem to alleviate this)

% Voxblox++: (Robotics & automation letters 2019)
% https://arxiv.org/pdf/1903.00268.pdf
% https://github.com/ethz-asl/voxblox-plusplus
%
% - M-RCNN + geometric segmentation + fusion 
% - data association of geometric segments with 3D overlap (no. points inside volume), fixed threshold for min number of points
% - instance label is assigned to a segment based on highest overlap
% - only one detected segment per reference label, as in PanopticFusion and Ours
% - TSDF Integration 
%
% good: 
% - because of geometric segmentation objects with no associated semantic class can also be segmented
% bad:
% - two different object segment types -> confusing, overly complicated ?
% - quite inaccurate (fixed below)

% Reconstructing Interactive 3D Scenes by Panoptic Mapping and CAD Model Alignments (ICRA 2021)
% https://arxiv.org/pdf/2103.16095.pdf
% https://github.com/hmz-15/Interactive-Scene-Reconstruction
%
% - based heavily on Voxblox++, much more accurate
% - Scene-graph ("contact graph") for mapping object relations
% - Search & replace voxels with CAD models, with geometrical and physical constraints
% - Object 6D pose
% - Format for robot interaction
%
% - Segmentation: bilateral fusion of geomatric and semantic segments -> reduce segmentation noise compared to Voxblox++
% - Fusion: triplet count improves consistency over Voxblox++ pairwise count strategy (take semantic label into account in addition to instance and geometry)
% - Fusion: instance labels are also combined if there is enough overlap with common geometric label for long enough time
%   - this means multiple detections can match the same reference unlike ours, voxblox++ and PanopticFusion ?
%

% Panoptic-MOPE: (ROBOTICS AND AUTOMATION LETTERS 2020)
% https://ieeexplore.ieee.org/stamp/stamp.jsp?tp=&arnumber=8977356
% https://github.com/hoangcuongbk80/Object-RPE/tree/panoptic-mope
%
% - novel RGB-D semantic segmentation model + M-RCNN
% - camera tracking based on "addaptively weighted optimization of geometric, appearance, and semantic cues"
% - surfel map: 
%   - how does it scale ? authors satate they tested on room-sized environments, but could be applied in larger scale as well ...
%     - could maybe be applied as VO in a SLAM algorithm ...
%   - demo only on a small pallet + surroundings, might not be applicable in large-scale SLAM

% US VS THEM:
%
% - based heavily on PanopticFusion, with modifications:
%   - instead of greedy data-association (which seems to be the case in others as well), we solve LAP (JPDA?)
%     - overlap threshold can be tuned, which renders the algorithm more flexible
%     - could be extended to dynamic tracking ?
%   - multiple options for association likelihood
%   - outlier rejection (either clustering or probabilistic)
%   - test different options for decreasing processing time
%   - no post-processing
%
% - model-agnostic:
%   - completely separated from segmentation
%   - does not care how point clouds are obtained -> applicable for LIDAR segmentation (e.g. EfficientLPS) as well
%
% - also agnostic to localisation method
%   - could, however, be utilised to find landmark locations / poses

% More compact version of this paragraph to introduction to save space?
%Panoptic segmentation -- proposed in \cite{panoptic_segmentation} -- aims to solve the unified task of semantic- and instance segmentation. Semantic classes are separated to \textit{stuff} -- amorphous, unquantifiable regions like sky, road or floor -- and \textit{things} -- quantifiable objects. The distinction between the two can vary depending on the application, but a semantic class can only belong to one or another. The article also proposes a unified panoptic evaluation metric, coined \textbf{Panoptic Quality} (PQ). Many 2D approaches to panoptic segmentation -- \textit{e.g.} \cite{panopticfpn,seamless,panoptic_deeplab,efficientps} -- have since been proposed. Deep neural networks for performing semantic- or instance segmentation directly on the 3D reconstruction -- \textit{e.g.} on \cite{scannet,s3dis,paris_lille_3d} -- have also been proposed, but since they require the reconstructed 3D scene, they are mostly offline approaches and therefore out of scope for this work. Some recent works also apply panoptic segmentation to point clouds -- \textit{e.g.} methods in the SemanticKITTI panoptic segmentation competition \cite{semantic_kitti} -- mostly aimed at segmenting LiDAR output. They are suitable for online processing, but similar to RGB-D images require a method for tracking object instances persistent in both time and space. In fact, our proposed method, as well as some others mentioned in this work, could use segmented LiDAR point clouds as an input similarly to RGB-D images.

PanopticFusion \cite{panopticfusion} is the first work to propose online integration of panoptic image segmentations to a 3D reconstruction. They integrate point clouds generated from segmented images to a TSDF voxel volume \cite{tsdf,voxblox} by greedily matching detected segments with the reconstruction and regulating each voxel's corresponding instance with a weighting function. Semantic labels are inferred in a bayesian manner based on confidence scores provided by the segmentation model. They also apply a Conditional Random Field (CRF) to regularise the reconstruction, improving results significantly. Voxblox++ \cite{voxblox++} -- introduced later the same year -- is a similar approach that also integrates segmented RGB-D images into a TSDF volume. It leverages geometric segmentation of depth images to improve instance segmentation accuracy. Both geometric and semantic segments are used to compute a pair-wise weight, which is used to greedily match them with segments in the reconstruction. Because of the geometric segmentation, the method allows segmentation of objects with no known semantic class in addition to objects recognised by the instance segmentation model. 

Recently, \cite{interactive_3d_scenes} built upon the idea of Voxblox++. They apply Voxblox++ for 3D instance integration, with two small but effective modifications: the pair-wise weight is replaced by a triplet weight that also takes semantic labels into account in the fusion, and -- in addition to geometric segments -- instance segments are fused if they overlap by a significant amount. The article introduces a method for searching and aligning CAD models to reconstructed objects based on geometry and semantic class, as well as geometrical and physical rules. With the CAD models, a contact graph and interactive virtual scene are reconstructed to allow a robot to simulate its interaction with the environment. SceneGraphFusion \cite{scenegraphfusion} is another approach that forms a scene graph online from a stream of RGB-D images, but unlike the above-mentioned approach, it generates the graph with a deep neural network, after which the panoptic labels for geometrically segmented portions of the 3D reconstruction are produced a side product.

Panoptic-MOPE \cite{panoptic_mope} is another recent approach, which integrates sequences of RGB-D images into a surfel reconstruction. Unlike other mentioned approaches -- which assume the camera pose either known or estimated elsewhere -- it also tracks camera movements based on geometric-, appearance- and semantic cues. The method also applies a novel RGB-D panoptic segmentation model. Although it is only tested on room-sized environments, the authors claim it could be scaled to larger environments as well.
\section{Assumptions for security}
\label{sec:attacks}

The security of a quantum cryptographic protocol relies on assumptions about the physics of the devices that are employed to implement the protocol.  In this section, we discuss these assumptions. For concreteness, we focus on the case of QKD, for which we describe the full set of assumptions in \secref{sec:attacks:assumptionlist}.  We then explain why these assumptions are needed and to what extent they are justified in \secref{sec:attacks:necessity}. Experimental work in QKD has shown however that the assumptions are often very difficult to meet, and are actually not met in many cases. This fact can be exploited by quantum hacking attacks, which are described in \ref{sec:attacks:hacking}. Finally, in Section~\ref{sec:attacks:countermeasures}, we discuss countermeasures against these attacks. 

\subsection{Standard assumptions for QKD} \label{sec:attacks:assumptionlist}

The security of QKD protocols usually relies on the following assumptions.

\begin{enumerate}
  \item \label{item_qm} All devices used by Alice and Bob, as well as the communication channels connecting them, are correctly and completely\footnote{The completeness of quantum theory can be derived from their correctness;  see \secref{sec:completeness}.} described by quantum theory.
    \item \label{item_res} The channel that Alice and Bob use to exchange classical messages is authentic, i.e., it is impossible for an adversary to modify messages or insert new ones. 
  \item \label{item_conv} The devices that Alice and Bob use locally to execute the steps of the protocol, e.g., for preparing and measuring quantum systems, do exactly what they are instructed to do.
\end{enumerate}

As already indicated earlier, due to the lack of proof techniques, additional assumptions had been introduced in the past. A prominent example is the \emph{i.i.d.}\ assumption, which demands that the quantum channel connecting Alice and Bob be described by a sequence of identical and independently distributed maps. Physically, this means that an adversary's interception strategy is such that each signal sent from Alice to Bob is modified in the same manner and independently of the other signals. Security under the i.i.d.\ assumption is called security against \emph{collective attacks}~\cite[see also \secref{sec:qkd.other.models}]{BM97b,BBBvdGM02}. Another assumption, which  usually comes on top of the i.i.d.\ assumption, is that Eve only stores classical data, which she obtains by individually measuring the pieces of information she gained from each signal sent from Alice to Bob. Since it is difficult to argue why an adversary should be restricted in that particular way, the corresponding security guarantee is rather weak. It is usually referred to as security against \emph{individual attacks}~\cite[see \secref{sec:qkd.other.models}]{Fuchsetal1997,Lutkenhaus2000}. 

Most modern security proofs do however not require such additional assumptions, i.e., they are based entirely on Assumptions~\ref{item_qm}--\ref{item_conv} above. This means, in particular, that the quantum channel connecting Alice and Bob can be arbitrary, and may even be entirely controlled by Eve. In this case, one talks about security against \emph{general attacks}, \emph{coherent attacks}, or \emph{joint attacks}. Sometimes the term  \emph{unconditional security} appeared in the literature~\cite{SBCDLP09}, but it is important to keep in mind that the assumptions listed above are still necessary.

\subsection{Necessity and justification of assumptions} \label{sec:attacks:necessity}

Assumption~\ref{item_qm} is often implicit, for it is a prerequisite to even describe the cryptographic scheme. It justifies the use of the formalism of quantum theory to model the different systems, such as the communication channel, including any possible attacks on them. The assumption thus captures the main idea behind quantum cryptography, namely that an adversary is limited by the laws of quantum theory.  The other two assumptions ensure that the experimental implementation follows the theoretical prescription that enters the security definition (Definition~\ref{def:security}), namely the description of the protocol $\pi_{AB}$ and the used resources. In particular, Assumption~\ref{item_res} guarantees that the resources shared between Alice and Bob fulfil the theoretical specifications~$\aR$, which in the case of QKD includes the classical authentic communication channel. Assumption~\ref{item_conv} guarantees that the steps prescribed by the protocol~$\pi_{A B}$ are correctly executed.

Assumption~\ref{item_qm} is widely accepted \--- and proving it wrong would represent a major breakthrough in physics. Nevertheless, it has been shown that there exist QKD protocols that only rely on the weaker assumption of \emph{no-signalling}~\cite{BHK05}.  

Assumption~\ref{item_res} demands that an authentic communication channel is set up between Alice and Bob. There exist information-theoretically secure protocols that achieve this, provided that Alice and Bob share a weak secret key~\cite[see also \secref{sec:smt.auth}]{RW03,DW09,ACLV19}.  Assumption~\ref{item_res} can thus be met by the use of such authentication protocols (see also~\secref{sec:intro} as well as standard textbooks on classical cryptography)

Although Assumption~\ref{item_conv} sounds rather natural, and is in fact required for almost any cryptographic scheme, including any classical one, it is rather challenging to meet.  Numerous quantum hacking experiments, which have been conducted over the past few years, have shown that many implementations of QKD failed to satisfy this assumption. To illustrate this problem, we describe selected examples of such attacks in the following subsection.

\subsection{Quantum hacking attacks} \label{sec:attacks:hacking}          

We start with the \emph{photon number splitting attack}~\cite{Brassardetal2000}, which targets optical implementations of QKD that use individual photons as quantum information carriers. Suppose, for concreteness, that Alice and Bob implement the BB84 protocol~\cite{BB84} by encoding the qubits into the polarisation degree of freedom of individual photons. Specifically, Alice may use a single-photon source that emits photons with a polarisation that she can choose. The BB84 protocol\footnote{This protocol is explained in more detail in \secref{sec:securityproofs}, where a security proof is also sketched.} requires her to send in each round at random a state from one orthonormal basis, say $\{\ket{h}, \ket{v}\}$, where $\ket{h}$ may be realised by a horizontally polarised photon and $\ket{v}$ by a vertically polarised one, or from a complementary basis $\{\ket{d^+}, \ket{d^-}\}$, where $\ket{d^+} = \smash{\frac{1}{\sqrt{2}}} (\ket{h} + \ket{v})$ and $\ket{d^-} = \smash{\frac{1}{\sqrt{2}}} (\ket{h} - \ket{v})$. It may now happen that, in an experimental implementation, the source sometimes accidentally emits two photons at once, which then carry the same polarisation. The states emitted in the four cases are thus $\ket{h} \otimes \ket{h}$, $\ket{v} \otimes \ket{v}$, $\ket{d^+} \otimes \ket{d^+}$, and $\ket{d^-} \otimes \ket{d^-}$.

Before describing the actual attack, we first give a simple information-theoretic argument for why this is problematic. Note first that one single photon carries no information about the choice of the basis made by Alice. Indeed, for either of the basis choices, the density operator describing the photon is maximally mixed, i.e., $\frac{1}{2} \proj{h} + \frac{1}{2} \proj{v} = \frac{1}{2} \proj{d^+} + \frac{1}{2} \proj{d^-} =  \frac{1}{2} \mathbf{1}$. This is however no longer the case for a pulse consisting of two photons, i.e., 
\begin{align}
  \frac{1}{2} \proj{h}^{\otimes 2} + \frac{1}{2} \proj{v}^{\otimes 2} \neq \frac{1}{2} \proj{d^+}^{\otimes 2} + \frac{1}{2} \proj{d^-}^{\otimes 2} \ .
\end{align}
Hence, if the source accidentally emits two equally polarised photons instead of one, it reveals information about Alice's basis choice, which it shouldn't. 

It is therefore not surprising that such two-photon pulses can be exploited by an adversary to attack the system. Eve, who intercepts the channel, may split the two-photon pulse into two, keep one of the photons and send the other one to Bob. The latter thus receives photons in exactly the way prescribed by the protocol, and hence does not notice the interception. Eve, meanwhile, may measure the photons she captured. In principle, if Eve had quantum memory, she could even wait with the measurement until Alice announces the basis choice to Bob, and hence always gain full information about the polarisation state that Alice prepared. 

While the photon number splitting attack exploits an imperfection of the sender (namely that it sometimes emits two identically polarised photons instead of one), many quantum attacks are targeted towards the receiver. An example is the \emph{time-shift attack}~\cite{Makarovetal2006,qi2007time,Zhaoetal2008}, which exploits inaccuracies of the photon detectors. In order to avoid dark counts, the photon detectors are often set up such that they only count photons that arrive within a small time window around the time when a signal is expected to arrive. Furthermore, Bob's receiver device may consist of more than one detector, e.g., one for each possible polarisation state. The time windows of the different detectors are then never perfectly synchronised. This means that there are times at which the receiver is more sensitive to signals with respect to one polarisation than another. Eve may therefore, by appropriately delaying the signals sent from Alice and Bob, bias the detected signals towards one or the other polarisation, and thus gain information about what Bob measures. While this information may be partial, it can, together with the error correction information that is available to Eve, be sufficient to infer the final key. 

Another attack that is targeted towards the receiver is the \emph{detector blinding attack} ~\cite{Makarov2009,WKRFNW11,LWWESM10,GLLSKM11}, where the adversary tries to control the detectors by illuminating them with bright laser light.  In a QKD implementation that uses the encoding of information into the polarisation of individual photons, the detectors are usually configured such they can optimally detect single photon pulses. That is, they should click whenever the incoming pulse contains a photon, and not click if the pulse is empty.  However, the behaviour of such detectors may be rather different in a regime where the incoming pulses contain many photons. For example, it could be that they always click when they are exposed to bright light with a particular intensity, and they may never click for another intensity. Hence, by sending in light with appropriately chosen polarisation and intensity, Eve may gain immediate control over the clicks of Bob's detector. To exploit this for an attack, Eve may mimic Bob's receiver, i.e., intercept the photons sent from Alice and measure them in a randomly chosen basis, as Bob would do. She then sends bright light to Bob to ensure that he obtains the same detector clicks as if he had directly obtained Alice's photons. This works particularly well for implementations that use a \emph{passive basis choice}, i.e., where Bob's measurement basis is not  provided as an input, but rather made by the detection device itself. In this case, an adversary can essentially remote-control Bob and thus get hold of the entire key. 

Yet another hacking strategy are \emph{Trojan-horse attacks}~\cite{Vakhitov2001,GisinFaselKraus2006}. Here the idea is to send a bright laser pulse via the optical fibre into Alice or Bob's component to extract information about its internal settings. Depending on the sender and receiver hardware which is used, measuring the reflection of the pulse can allow Eve, for instance, to determine the basis choices made by Alice and Bob.

In some optical implementations of QKD, e.g., in the \emph{plug-and-play}~\cite{Muller97} or the \emph{circular-type}~\cite{Nishioka2002} system, Alice does not have a photon source but instead encodes information by modulating an incoming signal from Bob before sending it back to him. The signal thus travels twice in opposite directions through the same optical links, which helps reducing fluctuations due to birefringence  and environmental noise. The two-fold use of the (insecure) channel however opens additional possibilities of attacks~\cite{GisinFaselKraus2006}. A prominent example is the \emph{phase-remapping attack}~\cite{FQTL07,XQL10}. It exploits the fact that the modulator used by Alice to encode information into the signal coming from Bob acts on that signal during a particular time interval. In the attack, the adversary slightly advances or delays the signal on its way from Bob to Alice, so that it no longer lies fully within that time interval. The modulation by Alice will then be incomplete, which means that the encoding of the information in the signal differs from what is foreseen by the protocol. This can in turn be exploited by Eve in an intercept-and-resend attack on the signal returned from Alice to Bob. 


\subsection{Countermeasures against quantum hacking} \label{sec:attacks:countermeasures}

The attacks described here have in common that they all exploit a breakdown of Assumption~\ref{item_conv}. Specifically, in the case of the photon-splitting attack, the device used by Alice sends out more information than it is supposed to. In the case of the time-shift attack, it is Bob's measurement device  whose measurement operators are not constant over time and can even be partially controlled by Eve. Finally, in the case of the detector blinding attack on systems with passive basis choice, Eve even takes over control of the randomness used to choose the basis.

A seemingly obvious countermeasure to prevent such attacks is to manufacture sources and detectors that meet the theoretical specifications. That is, one would need a perfect single-photon source, as well as detectors that are perfectly efficient and only measure photon pulses in a specified parameter regime. Such requirements are however unrealistic --- the devices used in experiments will always, at least slightly, deviate from these specifications. 

The other possibility is to develop cryptographic protocols and  security proofs that tolerate imperfections of the devices~\cite{GLLP04}.  This has been done in particular for the attacks described above. To prevent photon number splitting attacks, an efficient countermeasure is the \emph{decoy-state} method~\cite{Hwang2003,Wang2005,Loetal2005}. The idea here is that Alice sometimes deliberately sends multi-photon pulses. Alice and Bob can  then check statistically whether an adversary captured them. Another possibility is to use protocols where Alice's encoding of information has the property that, even when one photon is extracted from a pulse, the information about what Alice sent is still partial~\cite{SARG,TamakiLo,SYK14}. In the case of time-shift attacks, it is sufficient to characterise the maximum bias in the detector efficiencies that can be introduced and account for it in the security proofs. Finally, for the detector blinding attacks, a possible countermeasure is to add tests to the protocol, such as a monitoring of the photocurrent, in order to detect those~\cite{Yuanetal2010}.  

The main problem with such countermeasures is however that the space of possible imperfections is hard to characterise. The above are just a few examples of attacks, and many others have been proposed, and sometimes even demonstrated to work successfully in experiments. For example, an adversary may exploit imperfections in the randomness that Alice and Bob use for choosing their measurement basis. To prevent such attacks, one may again extend the protocols such that they can tolerate imperfect randomness (see \secref{sec:alternative.randomness}). 



The last decade has thus seen an arms race between designers and attackers of quantum cryptographic schemes. A possible way out of this unsatisfactory situation is \emph{device-independent cryptography}. Here the idea is to replace Assumption~\ref{item_conv} by something much weaker. Namely, one requires that the devices used by Alice and Bob do not unintentionally send information out to an adversary, and that the classical processing of information done by Alice and Bob is correct. Crucially, however,  one does no longer demand that the sources and detectors used by Alice and Bob work according to their specifications. The way this can work is explained in \secref{sec:alternative.di}. 

%%% Local Variables:
%%% TeX-master: "main.tex"
%%% End:

\section{Metrics}\label{sec:metrics}
\dataset{} can be used to measure performance on three related tasks. % \textit{Factual change detection}:
%: in other words, whether or not a question should be generated. 
% \textit{Discriminating Question Generation} metrics measure how similar generated questions are to annotated questions. \textit{Full System} metrics measure a system's performance on the overall task. 

\paragraph{Factual Change Detection} \label{Metric:Diff Detection}  Given an example consisting of a base passage, target passage, and answer span, the goal is to determine whether there exists a valid differentiating question. In other words, whether there is new information about this answer span that is present in the target passage when compared to the base passage. To measure this, we report accuracy, precision, recall and F1 score over the existence of a differentiating question in our annotations. Note that always predicting no change achieves 60.3\% accuracy but 0\% F1, but random guessing corresponds to 44.1\% F1. 

\paragraph{Discriminating Question Generation} Given a target passage and answer span, write a specific, unambiguous and information-seeking query that can be answered  with the target passage. To measure this, we compare machine generated questions to those that humans verified, edited, or hand wrote. We use two model-free metrics Rouge-1 and Rouge-L \citep{lin2004rouge} which measure the token-level overlap and longest subsequence overlap of the questions, respectively. We also consider two model-based metrics, BLEURT \citep{sellam-etal-2020-learning}, which is a learned evaluation for text similarity based on BERT \citep{devlin-etal-2019-bert}, and a query similarity model \citep{reimers-2019-sentence-bert} trained on Quora Question Pairs \footnote{\href{https://huggingface.co/cross-encoder/quora-roberta-large}{huggingface.co/cross-encoder/quora-roberta-large}}.

Note that we evaluate discriminating question generation despite using a question generation model in our annotation procedure. Note that all of these questions are reviewed by humans and only the very fluent ones are kept. As question generation models vary in which of their productions are very fluent, this set is less trivial than it would initially appear. Nonetheless, we also separate human-written or edited questions and evaluate that set independently. 
% The query similarity metric is based on a T5-XXL model \citep{raffel2020exploring} trained on Quora Question Pairs\footnote{\href{https://www.kaggle.com/c/quora-question-pairs}{https://www.kaggle.com/c/quora-question-pairs}} to predict either ``duplicate'' or ``not duplicate'', where a model gets 1.0 if the question it produces is considered a ''duplicate'' and 0.0 otherwise. 
%Note that these metrics are only computed over the true positive examples, 
%\pj{This is slightly confusing in the Metrics section. There already seems to be some explanation in Results. Maybe we can omit it here.} 
%so question generation models can only be compared easily based on the same set of data; i.e., producing questions based on the same factual change detection model. 

\paragraph{Full System} This is the overall measure of performance on \dataset{}. We reuse the metrics from discriminating question generation, using 0.0 for BLEURT, ROUGE-1, ROUGE-L, and Query Similarity if the factual change detection is incorrect.  
%and is scored in a standard way otherwise. Note that certain systems produce a question or ``None'' in the same output space, while others compose a factual change detection method and question generation method.
\section{\benchmarkname{}}
\label{sec:experiments}

% DONE: I think it might help to have a name for the benchmark,
% and to use that throughout the paper.  Then others who use the
% benchmark can refer to it by that name.

% We can call it MarkMyWords

% Our benchmark measures a watermark's quality, efficiency, and tamper-resistance in various contexts.
%
We now present \benchmarkname{}, our benchmark for evaluating watermarking schemes.

\subsection{Text Generation Tasks}

%Our benchmark relies on three tasks, all intended to generate long-form text, while also representing real scenarios in which one would wish to watermark the model outputs.
%

\benchmarkname{} uses three text generation tasks.
There were chosen to generate long text, in order to ensure enough tokens to watermark most outputs and get a good size estimate,
and to represent realistic scenarios in which watermarking would be useful.
We watermark the outputs of each task and measure quality, watermark size, and tamper-resistance on these outputs.
\cref{tab:bench-prompts} shows the prompt for each task.

\smallskip\noindent\textbf{(1) Book reports.}
%
Our first task instructs the model to generate book reports for a predefined list of 100 well-known books.
%
This represents a realistic use case of a watermark in an education context where instructors might wish to verify if students have used a language 
model to complete their homework assignment.
%
% Our prompt format is ``Write a book report about X, written by Y.'',
% where X is a book title and Y is the book's author. 

\smallskip\noindent\textbf{(2) Story generation.}
%
We ask the model to generate a short story, with a specific tone (e.g., funny, sad, suspenseful), on a particular topic (e.g., ``three strangers that win a getaway vacation together'').
%
We use 100 combinations of topics and tones.
%
In this case, detecting watermarks in creative content is a way to trace the authorship of a 
piece of text.
% Our prompt is ``Write a X story about Y.'', where X is the tone, and Y is the topic.

\smallskip\noindent\textbf{(3) Fake news.}
%
Our last task asks the model to generate a news story about two prominent political figures meeting at an event.
%
In total, we have a list of 100 unique combinations of political figures and locations.
%
Detecting watermarks in short news articles or headlines could be useful for fighting against automated fake news bots on social media.
%
% Our prompt is ``Write a news article about X's visit to Y in Z.'', where X and Y are two political figures, and Z is a location.

\subsection{Benchmark Implementation}

We implemented the \benchmarkname{} benchmark in Python using the \texttt{transformers} library~\citep{wolf_transformers_2020} to implement models and watermarks.
%
In general, the run times of the watermarked models are comparable to those of base models.
%
We rely on the 7B-parameter version of Llama-2 chat both for generation and rating.
%
We chose this model for its high quality and because its output distribution, like other modern models, has low entropy.
We've noticed newer language models are more certain of their next-token predections, i.e., have lower entropy in their output distribution,
which makes watermarking more difficult. Thus, using a low-entropy model makes for a more realistic 
evaluation of a watermark scheme's performance. 
%
This version of the Llama-2 model is known for often refusing to generate content---we 
altered the system prompt to avoid this issue, especially for the fake news task.

Our benchmark generates a total of 300 outputs from the language model.
%
We only attack a third of these (the first third of each task),
as running robustness tests on all 300 would make the benchmark too slow.
%
The maximum number of generated tokens per prompt is 1024.
%
On a A5000 GPU, the benchmark completes in 30 minutes to an hour. 

%Users can choose whether to sample a new secret watermarking key for each generation or not.
%
%We enable this in our experiments, but practitioners can deactivate it in order to evaluate a model's performance for a specific secret key, for instance when evaluating a public scheme.

Our code for the \benchmarkname{} benchmark has been made public.\footnote{\mylink{https://github.com/wagner-group/MarkMyWords}}
%
It is easily updated to support new watermarking schemes or new language models.
%Anyone can run it, either to compare new watermarking schemes to the ones presented in the paper, or to find the best hyper-parameter 
%selection for another language model. 
%
We hope that future proposals of watermark schemes will include an evaluation with this benchmark.

\smallskip\noindent\textbf{Efficiency.}
%
We have developed custom implementations directly in CUDA of some of the watermarking schemes, to speed up computation.
\begin{itemize}[leftmargin=\itemlm,nosep]
    \item \textbf{Hash function.} The exponential sampling scheme relies on computing the hash of many elements. Our CUDA implementation allows this to be done in parallel.
    \item \textbf{Edit distance.} The edit distance is computed with every possible key offset; our code implements this in parallel.
\end{itemize}
% Additionally, for empirical verification of edit scores, since we use a $t$-test~\cite{kuditipudi_robust_2023}, we need to compute test statistic 100 times under the null hypothesis to derive an empirical $p$-value.
% %
% This is done by replacing the randomness used at generation time by fresh random 
% values.
% Doing so for each detection is inefficient.
% %
% Instead, for edit scores, we pre-compute the test statistic under the null hypothesis 200 times and then sample 100 random values of this statistic every time we want to verify a watermark.
% \chawin{I'm a bit confused by this part.}
% %
% This means we only need to run the expensive computation once.
% %
% Empirically, we did not see a change in detection accuracy when doing so. 

\begin{table}[t!]
\centering
\caption{Voice conversion \& F0 manipulation results. MOS results are reported with 95\% confidence interval. VDE, and FFE are reported for F0 manipulation while PER, WER, EER, and MOS are reported for voice conversion. Notice, for VDE, and FFE higher is the better since F0 was flattened.}
\label{tab:conv}

\resizebox{1\columnwidth}{!}{
\begin{tabular}{c@{~} | c@{~} | c@{~}c@{~} | c@{~} | c@{~} ||  c@{~}c@{~} }
\toprule
\multirow{2}{*}{Dataset} & \multirow{2}{*}{Method} & \multicolumn{4}{c||}{Voice Conversion} & \multicolumn{2}{c}{F0 Manipulation} \\
\cmidrule{3-8}
& & PER~$\downarrow$ & WER~$\downarrow$ & EER~$\downarrow$ & MOS~$\uparrow$ & VDE~$\uparrow$ & FFE~$\uparrow$ \\
\midrule
VCTK & GT  & 17.16 & 4.32 & 3.25 & 4.11$\pm$0.29 & -- & -- \\
\midrule 
\multirow{3}{*}{LJ}
% & ASR-TTS   & 50.74  & --     & 66.08 & 32.96 & 1.46 \\
& CPC       & 22.22 	& 16.11 		& 0.46 		& 3.57$\pm$0.15 		& \bf 46.68 & \bf 48.71\\
& HuBERT    & \bf 19.09 & \bf 12.23 & \bf 0.31  & \bf 3.71$\pm$0.24 & 39.20 		& 48.42\\
& VQ-VAE    & 40.88 	& 36.96 		& 9.65 		& 2.90$\pm$0.17 		& 10.54 	& 12.08 \\
\midrule 
\multirow{3}{*}{VCTK} 
% & ASR-TTS   & 68.88  & --    & 41.77 & 13.55 & 6.48 \\
& CPC       &  23.58 		& 15.98 		& \bf 4.83  &  3.42 $\pm$ 0.24 		& \bf 25.29 & \bf 26.97 \\
& HuBERT    &  \bf 20.85 	& \bf 12.72 & 6.01  		& \bf  3.58 $\pm$ 0.28 	& 23.46 	& 26.67 \\
& VQ-VAE    & 36.88  		& 29.44 		& 11.56 		& 3.08 $\pm$ 0.34 		& 7.03  	& 7.80  \\
\bottomrule
\end{tabular}}
\vspace{-0.4cm}
\end{table}

\vspace{-0.1cm}
\section{Results}
\vspace{-0.1cm}
Our results cover
% We report results for 
three different settings: (i) speech reconstruction experiments; (ii) speaker conversion and F0 manipulation; (iii) bitrate analysis with subjective tests for speech codec evaluation. We employ two datasets: LJ~\cite{ljspeech17} single speaker dataset and VCTK~\cite{vctk} multi-speaker dataset. All datasets were resampled to a 16kHz sample rate.

% \paragraph*{Implementation Details.}
% \smallskip
\noindent{\bf Implementation Details\quad} 
\label{sec:impl}
We follow the same setup as in~\cite{lakhotia2021generative}. For CPC, we used the model from~\cite{Riviere2020}, which was trained on a ``clean'' 6k hour sub-sample of the LibriLight dataset~\cite{Kahn2020,Riviere2020}. We extract a downsampled representation from an intermediate layer with a 256-dimensional embedding and a hop size of 160 audio samples. For HuBERT we used a \textsc{Base} 12 transformer-layer model trained for two iterations~\cite{hsu2020hubert} on 960 hours of LibriSpeech corpus~\cite{Panayotov2015}. 
% This model encodes every 320 raw audio samples into a 768-dimensional vector. 
This model downsamples the raw audio $\times320$ into a sequence of 768-dimensional vectors. Similarly to~\cite{lakhotia2021generative}, activations were extracted from the sixth layer.

%CPC: We use a dictionary of 100 units, leading to a bitrate of 700bps.
%HuBERT: A dictionary of 100 units is used, leading to a bitrate of 350bps. 
%VQVE: The VQ-VAE discrete code operates at a bitrate of 800bps.
% For both CPC and HuBERT, the k-means algorithm is applied to convert continuous frames to discrete codes, using the LibriSpeech clean-100h~\cite{Panayotov2015} dataset. 
For CPC and HuBERT, the k-means algorithm is trained on LibriSpeech clean-100h~\cite{Panayotov2015} dataset to convert continuous frames to discrete codes. We quantize both learned representations with $K=100$ centroids. Leading to a bitrate of 700bps for CPC and 350bps for HuBERT.

% VQ-VAE
Similarly to CPC models, we trained the VQ-VAE content encoder model on the ``clean'' 6K hours subset from the LibriLight dataset. We use an encoder operating on the raw signal to extract discrete units, similar to~\cite{jukebox}. In addition, ``random restarts'' were performed when the mean usage of a codebook vector fell below a predetermined threshold. Finally, we used HiFiGAN (architecture and objective) as the decoder instead of a simple convolutional decoder, as it improved the overall audio quality. This model encodes the raw audio into a sequence of discrete tokens from 256 possible tokens~\cite{garbacea2019low} with a hop size of 160 raw audio samples. The VQ-VAE discrete code operates at a bitrate of 800bps. We additionally experimented with 100 discrete units for VQ-VAE, however results were the best for 256. This finding is consistent with~\cite{garbacea2019low}.

% verification model
The speaker verification network uses the architecture proposed in~\cite{heigold2016end}. It was trained on the VoxCeleb2~\cite{voxceleb2} dataset, achieving a 7.4\% Equal Error Rate (EER) for speaker verification on the test split of the VoxCeleb1~\cite{Nagrani17} dataset.

% pitch
Only a single F0 representation is considered across all evaluated models, trained on the VCTK dataset.
% The F0 is extracted from the raw audio using YAAPT~\cite{yaapt} algorithm, using a window size of 20ms and a 5ms hop. 
The F0 is extracted from the raw audio using a window size of 20ms and a 5ms hop. 
As a result, the F0 sequence is sampled at 200Hz. 
% We apply the quantization described at Sec.~\ref{sec:method}, using a pitch codebook of $K'=20$ tokens and an encoder that downsamples the pitch by $\times16$. 
The quantization described at Sec.~\ref{sec:method}, is applied using an F0 codebook of $K'=20$ tokens and an encoder that downsamples the signal by $\times16$. Hence, the discrete F0 representation is sampled at 12.5Hz, leading to a bitrate of 65bps. The final bitrate of the evaluated codecs is the sum of the pitch code bitrate with the content code bitrate.

% \paragraph*{Evaluation Metrics}
% \smallskip
\noindent{\bf Evaluation Metrics\quad} 
We consider both subjective and objective evaluation metrics. For subjective tests, we report the Mean Opinion Scores (MOS). In which human evaluators rate the naturalness of audio samples on a scale of 1--5. Each experiment, included 50 randomly selected samples rated by 30 raters. For objective evaluation, we consider: (i) Equal Error Rate~(EER) as an automatic speaker verification metric obtained using a pre-trained speaker verification network. We report EER between test utterances and enrolled speakers; (ii) Voicing Decision Error (VDE)~\cite{nakatani2008method}, which measures the portion of frames with voicing decision error; (iii) F0 Frame Error (FFE)~\cite{chu2009reducing}, measures the percentage of frames that contain a deviation of more than 20\% in pitch value or have a voicing decision error; (iv) Word Error Rate (WER) and Phoneme Error Rate (PER), proxy metrics to the intelligibility of the generated audio. We used a pre-trained ASR network~\cite{baevski2020wav2vec} on both reconstructed and converted samples to calculate both metrics. %To generate target phonemes, the g2p-en~\cite{g2pE2019} Grapheme2Phoneme module was used.

% \vspace{-0.1cm}
% \smallskip
\noindent{\bf Reconstruction \& Conversion}
% \vspace{-0.1cm}
We start by reporting the reconstruction performance. Results are summarized in Table~\ref{tab:recon}. When considering the intelligibility of the reconstructed signal HuBERT reaches the lowest PER and WER scores across all models, where both CPC and HuBERT are superior to VQ-VAE. However, when considering F0 reconstruction VQ-VAE outperforms both HuBERT and CPC by a significant margin. This results are somewhat intuitive, bearing in mind VQ-VAE objective is to fully reconstruct the input signal. In terms of subjective evaluation, all models reach similar MOS scores, with one exception of CPC on LJ. 

%Notice, since the same F0 units are used for each method, this result implies the VQ-VAE units contain some information about the F0 of the signal, enabling better reconstruction. Regarding speaker information, the CPC gets the lowest EER. 

To better evaluate the disentanglement properties of each method with respect to speaker identity and F0, we conducted an additional set of experiments aiming at speaker conversion and F0 manipulation. For voice conversion, we converted each test utterance into five random target speakers. Next, we employed a speaker verification network, which extracts \emph{d-vector} representation to evaluate speaker-converted utterances' similarity to real speaker utterances (low error-rate indicates good conversion), providing measurement to the speaker identity's disentanglement from the evaluated coding method. The error-rate is reported between converted test utterances and enrolled speakers. For the LJ speech single speaker dataset, we converted samples from the VCTK dataset to the single speaker and enrolled all VCTK speakers together with the single speaker. Results are summarized in Table~\ref{tab:conv} (left). Unlike resynthesis results, on voice conversion CPC and HuBERT outperform VQ-VAE on both LJ and VCTK datasets, indicating VQ-VAE contains more information about the speaker in the encoded units, hence producing more artifacts. Notice, this also affects WER, PER, and the overall subjective quality (MOS). 

Next, to evaluate the presence of F0 in the discrete units, we flattened the F0 units before synthesizing the signal and calculated VDE and FFE with respect to the original F0 values. F0 flattening was done by setting the speakers' mean F0 value across all voiced frames. In this experiment, we expected units that contain F0 information to be better at F0 reconstruction over disentangled units. Results are summarized in Table~\ref{tab:conv} (right). Notice VQ-VAE can still reconstruct the F0 almost at the same level as when using the original F0 as conditioning (5.2 vs 7.03, and 5.59 vs 7.8), in contrast to CPC and HuBERT.

\begin{figure}[t!]
\centering
\includegraphics[width=0.65\columnwidth, trim={50 20 70 20}]{figures/codec_2.pdf}
% \caption{MUSHRA subjective listening test results as a function of bitrate per second for various methods. Purple dots denote the baseline methods, and green dots the proposed SSL based method.} 
\caption{MUSHRA subjective quality results as a function of bitrate per second. Purple dots denote the baseline methods, and green dots the proposed SSL based method.} 
\label{fig:codec}
\vspace{-0.5cm}
\end{figure}

% \vspace{-0.1cm}
% \smallskip
\noindent{\bf Speech Codec}
Our final experiment evaluates the obtained speech units as a low bitrate speech codec. 
% Therefore, we evaluate how the performance varies as a function of the number of discrete units. Changing the number of units is equivalent to varying the bitrate of the encoded signal. 
We use a subjective MUSHRA-type listening test~\cite{series2014method} to measure the perceived quality of the proposed speech codec with regard to its bitrate constraints. In MUSHRA evaluations, listeners are presented with a labeled uncompressed signal for reference, a set of test samples to rate, a copy of the uncompressed reference, and a low-quality anchor. Listeners are asked to rate each test utterance and the copy of the uncompressed reference with respect to the labeled reference in a scale of 1-100.

The experiment is performed on the VCTK dataset~\cite{vctk}. For evaluation, we used 20 utterances from 5 speakers. The set of speakers in the test data is disjoint with those in the training data. For this experiment, HuBERT models with 50, 100, and 200 units were trained as described in Sec.~\ref{sec:impl}. For comparison, we included other speech codecs in our evaluation: Opus~\cite{valin2012definition} wideband at 9 kbps VBR, Codec2~\cite{rowe2011codec} at 2.4 kbps and LPCNet~\cite{valin2019real} operating at 1.6 kbps. The LPCNet model was trained from scratch on the VCTK dataset following the experimental setup in~\cite{valin2019real}. The VQ-VAE model employs the HiFiGAN decoder trained on the LibriLight dataset to match the amount of data reported in~\cite{garbacea2019low}. We compressed the anchor sample with Speex~\cite{valin2016speex} at 4 kbps as a low anchor. Fig.~\ref{fig:codec} depicts the results. HuBERT with 50 units reaches the best MUSHRA score while its bitrate is only 365bps, which is significantly lower than the baseline methods.
\section{Scalable Representations for Communication Patterns}
\label{sec:design}

Using the lessons learned in our preliminary studies, along with existing case studies~\cite{isaacs2014combing, Isaacs2016} using idealized unit time, we design a set of strategies for representing communication patterns when there are too many PEs to draw distinct communication lines in Gantt charts. We first describe our design goals. Then, we present our designs. Finally, we discuss initial feedback from experts familiar trace analysis in HPC.


\subsection{Design Goals}

Our goal is to design a representation of communication in execution traces that (1) aids users in recognizing and understanding what communication is occurring in that temporal and logical position in the Gantt chart and (2) is agnostic to the number of processing elements, thereby scaling to larger traces. These goals are derived from usage and scalability limitations noted in prior work~\cite{isaacs2014combing}. 

We limit our focus to scaling in PEs (y-axis) rather than time. Traces are typically explored using a time window, so we focus on that case. Adapting a design or creating a new one for compressed time settings we leave for future work.

Based on our preliminary study (\autoref{sec:prelim}), we chose to focus on offset, ring, and exchange pattern types as stencils require more design consideration even at small scales. 

\subsection{Visualization Design}

Our design process began with open brainstorming on paper, which we include in the supplemental material. We tried a variety of strategies, including linked views and added channels to the traditional Gantt chart encoding rules. However, most of these retained scaling problems, leading us to focus on designs centering on glyphs.

In designing the representation, we considered the saliency of what was to be encoded (e.g., temporal range, pattern type, grouping, stride) and efficacy of available channels, taking into account that the design needs to be incorporated in a Gantt chart. For example, temporal range is set to a horizontal position matching where a pattern would be drawn in a full chart. See supplemental materials for a table containing discussion of channel considerations.

We prioritize the type of pattern before the grouping factor or stride. The rationale is that the pattern type is fixed by the source code while the grouping and stride are often computed from the problem size and number of resources. Therefore, a user will recognize pattern type first before considering other factors. \autoref{fig:abstract_designs} shows the resulting designs.


\begin{figure*}
    \centering
    \begin{subfigure}{0.18\textwidth}
         \centering
         \includegraphics[width=\textwidth]{figures/new-basic-offset.png}
         \caption{Continuous offset pattern}
         \label{fig:noc}
    \end{subfigure}
    \begin{subfigure}{0.18\textwidth}
         \centering
         \includegraphics[width=\textwidth]{figures/new-basic-offset-grouped.png}
         \caption{Grouped offset pattern}
         \label{fig:nog}
    \end{subfigure}
    \begin{subfigure}{0.18\textwidth}
         \centering
         \includegraphics[width=\textwidth]{figures/new-basic-ring.png}
         \caption{Continuous ring pattern}
         \label{fig:nrc}
    \end{subfigure}
    \begin{subfigure}{0.18\textwidth}
         \centering
         \includegraphics[width=\textwidth]{figures/new-basic-ring-grouped.png}
         \caption{Grouped ring pattern}
         \label{fig:nrg}
    \end{subfigure}
    \begin{subfigure}{0.18\textwidth}
         \centering
         \includegraphics[width=\textwidth]{figures/new-basic-exchange.png}
         \caption{Exchange pattern}
         \label{fig:neg}
    \end{subfigure}
    \caption{Examples of our designs for five communication patterns. They are reminiscent of the underlying communication pattern encoding, but not aligned to the underlying chart and agnostic to the number of rows the underlying pattern repeats over. 
    %Note that the angle for our ring pattern is slightly shallower than the angle for offset, this reflects the difference in stride between the two patterns. 
    Grouped representations fill the vertical space to indicate that the repetition continues from the top of row to the bottom.}
    \label{fig:abstract_designs}
\end{figure*}

\vspace{1ex}

\textbf{Encoding Pattern Type.} To encode the pattern type, we started with the overall shape of of the pattern when drawn at small scale with a small stride. Offsets are drawn with angled repeating lines forming a rhombus-like shape. We use a fixed distance between lines and draw as many will fit in the relevant area.

Rings add indicators of the ``wrap-around'' communication. However, unlike fully drawn rings, we only render the protruding segments at the ends of the shape. There are two main rationales for only drawing protruding segments: (1) we want to indicate this is an abstraction and (2) participants in our interviews found the crossing lines difficult to disambiguate. The number of protruding segments is proportional to the stride of a ring. 
%For a ring with a small stride, one segment is added. This increases to a max of four for a very large stride.

Exchange patterns are drawn as a series of symmetrical ``x'' shapes and avoid direct crossings for the same reason as rings. The number of lines in each cross is proportional to stride of the exchange. Short stride exchanges will exchange between only a few PEs, a long stride exchange spans many PEs. Our glyphs approximate this by increasing the number of crossing lines as stride increases.

\vspace{1ex}

\textbf{Encoding Grouping Factor.} To represent grouping, we partition the available area vertically and repeat the pattern type drawing in those partitions. More formally, the encoding rule to show ``grouping" is repetition and alignment on a non-common scale. The number of partitions is determined by the available vertical space in a chart.


\vspace{1ex}

\textbf{Encoding Stride.} We express the notion of stride through the angle of lines used in our pattern types. As people had difficulty with steeply angled lines in our preliminary study, we limit the angles to a range of 15 degrees to 60 degrees. Therefore, these do not match the encoding of a full view. Instead, they hint at the magnitude of distance over which communication is occurring. This allows users to see that there are differences in stride between glyphs, but not necessarily calculate the exact stride visually.

\vspace{1ex}

\textbf{Temporal range.} Rather than show the exact range, we place the glyphs on the x-axis so they are centered in their range. If two structures overlap, they are placed alongside one another. 

\vspace{1ex}

\textbf{Incorporation in Gantt Charts.} These are designed to be used in Gantt charts when exact lines would be too dense to be interpreted. The underlying interval rectangles will still be drawn. The color encoding of these intervals was shown to be a secondary indicator in our preliminary study, so we preserve them. We add a slight blur effect to the background as another signifier that the glyphs are an abstraction and should not be confused for exact lines.



\subsection{Expert Feedback}
\label{sec:expertfeedback}

We sent our designs to two HPC experts for feedback regarding both the designs themselves and the overall approach. Both experts were familiar with idealized unit time representations of traces. The first expert, E1, had previously collaborated on this strategy with the authors but was not involved in any of the work presented here. The second expert, E2, had managed an integration of the strategy into an HPC center's performance tools, referencing the open-source research code~\cite{isaacs2014combing} but using an alternate calculation method and front-end technology.

We sent both experts a short email with a PDF describing the visualizations with comparisons to fully drawn traces and showing how they might be applied in practice, including a few complicated examples such as zoomed-out time and idle processes. (See supplemental materials.) We asked if and how the strategy would be useful and if there were any suggestions or concerns. E1 responded the designs ``definitely look helpful,'' noted the trade off in exactness, and then pointed out figures which led to ambiguities in his view. He also identified a error where the mock-up did not match the underlying trace. E2 noted that stride is less important and wondered how the translation from data to glyph would be calculated. He suggested the strategy might also be helpful for collective communications (e.g., broadcasts, all-to-all, reductions), a set of patterns we did not consider in this work.

We interpreted these responses to suggest the designs were worth further study, particularly E1's ability to interpret well enough to detect an error and E2's interest in further patterns. However, there are design decisions in applying these glyphs in some scenarios, particularly in zoomed-out time, that require refinement. We leave these cases for future iterations and instead focus on how the base designs could be interpreted by a wider range of users in a controlled study.


\section{Summary}

We performed a series of galactic disk $N$-body simulations
to investigate the formation and dynamical evolution of spiral arm 
and bar structures in stellar disks which are embedded in live 
dark matter halos.
We adopted a range of initial conditions where the models have similar halo 
rotation curves, but different masses for the disk and bulge components, 
scale lengths, initial $Q$ values, and halo spin parameters.
The results indicate that the bar formation epoch increases exponentially 
as a function of the disk mass fraction with respect to the total mass at the 
reference radius (2.2 times the disk scale length), $f_{\rm d}$.
This relation is a consequence of swing amplification~\citep{1981seng.proc..111T},
which describes the amplification rate of the spiral arm when it transitions from 
leading arm to trailing arm because of the disk's differential rotation.
Swing amplification depends on the properties that characterize the disk, 
Toomre's $Q$, $X$, and $\Gamma$. The growth rate reaches its maximum
for $1<X<2$,  although the position of the peak slightly depends on $Q$ as well as on
$\Gamma$. We computed $X$ for 
$m=2$ ($X_2$), which corresponds to a bar or two-armed spiral, 
for each of our models and found that this value is related to the bar's
formation epoch.

The bar amplitude grows most efficiently when $1<X_2<2$. For models 
with $1<X_2<2$ the bar develops immediately after the start 
of the simulation. As $X_2$ increases beyond $X_2=2$, the growth rate
decreases exponentially. We find that the bar formation epoch increases
exponentially as $X_2$ increases beyond $X_2=2$, in other words
$f_{\rm d}$ decreases. The bar formation epoch exceeds a Hubble time
for $f_{\rm d}\lesssim 0.35$.

Apart from $X$, the growth rate is also influenced by $Q$ where
a larger $Q$ results in a slower growth. This indicates that the bar formation
occurs later for larger values of $Q$. 
Our simulations confirmed this and showed that for the bar ($m=2$) the growth rate
is predicted by swing amplification and becomes visible when it grows beyond a certain amplitude.

Toomre's swing amplification theory further predicts that
the number of spiral arms is related to the mass of the disk, with
massive disks having fewer spiral arms. In addition, larger $\Gamma$
predicts a smaller number of spiral arms.
We confirmed these relations in our simulations. 
The shear rate ($\Gamma$) also affects the pitch angle of spiral
  arms. We further confirmed that our result is consistent with previous
studies.

We found that the disk-to-total mass fraction ($f_{\rm d}$)
and the shear rate ($\Gamma$) are the most important parameters that determine the
morphology of disk galaxies. 
When juxtaposing our models with the Hubble sequence,
the fundamental subdivisions of (barred-)spiral galaxies with 
massive bulges and tightly wound-up spiral arms from S(B)a to S(B)c is 
also be observed as a sequence in our simulations. Where the models 
with either massive bulges or massive disks have more tightly
wound spiral arms. This is because having both a massive disk and bulge results in 
a larger $\Gamma$, i.e., more tightly wound spiral arms. 


Once the
bar is formed it starts to heat the outer parts of the disk.
From this point onwards, 
the self-gravitating spiral arms disappear.
This may be in part caused by the 
lack of gas in our simulations. 
After the bar grows, we no longer discern  
spiral arms in the outer regions of the disk. This could imply
that gas cooling and star formation are required in order to 
maintain spiral structures in barred spiral galaxies for over 
a Hubble time~\citep{1981ApJ...247...77S,1984ApJ...282...61S}.


Our simulations further indicate that non-barred grand-design spirals are
transient structures which immediately evolve into barred
galaxies. Swing amplification teaches us that a massive disk is
required to form two-armed spiral galaxies. This condition, at the
same time, satisfies the short formation time of the bar structure.
Non-barred grand-design spiral galaxies therefore must evolve into barred
galaxies.  We consider that isolated non-barred grand-design spiral galaxies 
are in the process of developing a bar.




\bibliography{refs}
% \bibliographystyle{abbrvnat}
\bibliographystyle{IEEEtranN}
\newpage

\appendix
% \section{Detailed Description of Watermark Design Space}\label{app:sec:watermark}

\subsection{Techniques to Allow Diverse Generation}\label{app:ssec:diverse}

In order to add some diversity to the generations, there are different strategies.
\begin{enumerate}[leftmargin=\itemlm,itemsep=2pt]
    \item For a fixed randomness source, \citet{kuditipudi_robust_2023} proposes to randomly shift the sequence of random values $\{r^k_i\}$ by an offset $s$, such that $r_n = r^k_{(s+n) \text{ (mod L)}}$. This means there are a total of $L$ possible unique values for $r_n$ depending on $s$. For a text-dependent randomness source, this trick does not work. 
    \item Instead, one natural strategy is to randomly skip the watermarking selection procedure for some tokens, and instead sample the next token from the original multinomial distribution. We denote \textbf{S} this skip probability. 
    \item Another strategy, discussed by \citet{christ_undetectable_2023}, is to only start watermarking text after enough empirical entropy has been generated: the first tokens are selected without a watermark. This accomplishes the same effect, and guarantees undetectability. However, as discussed in the paper's appendix, a user not wanting to generate watermarked text can simply run the model, keep the first few tokens, add them to the prompt, and start again: After repeating this step enough time, they can generate arbitrarily long text without a watermark. This seems like a larger practical drawback than loosing the guarantee of undetectability, thus we use the skip probability instead for promoting diversity. 
\end{enumerate}

\subsection{Exact Distribution of the Score Function}\label{app:ssec:exact_dist}

The null hypothesis distribution for the exponential scheme with the regular test variable is an Irwin-Hall distribution centered with parameter $N$ (whose average quickly converges to a normal distribution centered in 0.5 with variance $\frac{1}{12N}$).
%
When using the log test variable, the null distribution is the Erlang distribution with parameter $N$. The binary scheme also follows an Erlang distribution, but with many more tokens since each token is broken down into a binary vector. The distribution-shift scheme has a null distribution a binomial with parameters $\gamma, N$. We derive these distributions in the appendix.

However, for both other scores, and for the inverse transform, the null hypothesis distribution is too complex to compute. In these cases, verification uses a permutation test, as described in Kuditipudi et al.. Instead of comparing the score to a known distribution, we sample independent random sequences $\tilde{r}_i$ and compute the score of the text for that randomness: these trials are distributed like non watermarked text, so we can use them to compute an empirical p-value. 
\subsection{Additional analysis}
\label{app:ssec:additional_figures}

Here we present additional figures to support the main text.

\begin{figure}
    \centering
    \includegraphics[width=1\linewidth]{figures/fig1b.png}
    \caption{Same plot as~\cref{fig:aggregate}, but for a quality threshold of 10\%, and a minimum robustness of 0.2. The same conclusions from the main figure still hold.}
    \label{fig:aggregate2}
\end{figure}

\begin{figure}[t]
    \includegraphics[width=\linewidth]{figures/robustness_to_attack.png}
    \centering
    \caption{Correlation between robustness metric and attack success. On the right, for the Russian translation attack. 
    On the left, for the GPT paraphrasing attack. Each dot is a unique watermark parameter setting.}
    \label{fig:robustness-to-attacks}
\end{figure}

\begin{figure}[t]
\includegraphics[width=\linewidth]{figures/delta.png}
\centering
\caption{Size and quality for varying biases, at T=0.3 and T=1. The quality is relative to the quality of the non-watermarked model at the given temperature.}
\label{bias-fig}
\end{figure}

\begin{figure}[t]
\includegraphics[width=\linewidth]{figures/keylen.png}
\centering
\caption{Size and tamper resistance as a function of key lengths (only using distribution-shift schemes with $\delta \leq 2$).}
\label{keylen-fig}
\end{figure}
\section{Proofs}
\subsection{Proof of Theorem \ref{th:inexactLS1}}
We start from a similar argument as in \cite[proof of Therorem~2]{Blumen}. 
%proof of \cite[Theorem~2]{Blumen}. 
Set $g := 2\nabla f(x^{k-1})=2A^T(Ax^{k-1}-y)$ and $\g:=2\nablaa^{\nug} f(x^{k-1})= g+2\eg^k$ for some vector $\eg^k$ which by definition~\eqref{eq:grad} is bounded $\norm{\eg^k}\leq \nug^k$. It follows that
\ifCLASSOPTIONtwocolumn
\begin{align*} 
&\norm{y-Ax^k}^2-\norm{y-Ax^{k-1}}^2	\\
&= \langle x^k-x^{k-1},g \rangle +\norm{A(x^k-x^{k-1})}^2 \\
&\leq \langle x^k-x^{k-1},g \rangle + \MM \norm{x^k-x^{k-1}}^2, 
\end{align*}
\else
\begin{align*} 
\norm{y-Ax^k}^2-\norm{y-Ax^{k-1}}^2	&= \langle x^k-x^{k-1},g \rangle +\norm{A(x^k-x^{k-1})}^2 \\
&\leq \langle x^k-x^{k-1},g \rangle + \MM \norm{x^k-x^{k-1}}^2, 
\end{align*}
\fi
where the last inequality follows from the ULE property in Definition \ref{def:Lip}. Assuming $\MM \leq 1/\mu$, we have
\ifCLASSOPTIONtwocolumn
\begin{align*}
&\langle x^k-x^{k-1},g \rangle + \MM \norm{x^k-x^{k-1}}^2 \\
& \leq \langle x^k-x^{k-1},g \rangle + \frac{1}{\mu} \norm{x^k-x^{k-1}}^2\\
&= \langle x^k-x^{k-1},\g \rangle + \frac{1}{\mu} \norm{x^k-x^{k-1}}^2 - \langle x^k-x^{k-1},2\eg^k \rangle\\
& = \frac{1}{\mu} \norm{x^k-x^{k-1}+\frac{\mu}{2} \g }^2 - \frac{\mu}{4} \norm{\g}^2 - \langle x^k-x^{k-1},2\eg^k \rangle.
\end{align*}
\else
\begin{align*}
\langle x^k-x^{k-1},g \rangle + \MM \norm{x^k-x^{k-1}}^2 
& \leq \langle x^k-x^{k-1},g \rangle + \frac{1}{\mu} \norm{x^k-x^{k-1}}^2\\
&= \langle x^k-x^{k-1},\g \rangle + \frac{1}{\mu} \norm{x^k-x^{k-1}}^2 - \langle x^k-x^{k-1},2\eg^k \rangle\\
& = \frac{1}{\mu} \norm{x^k-x^{k-1}+\frac{\mu}{2} \g }^2 - \frac{\mu}{4} \norm{\g}^2 - \langle x^k-x^{k-1},2\eg^k \rangle.
\end{align*}
\fi
Due to the update rule of Algorithm \eqref{eq:inIP} and the inexact (fixed-precision) projection step, we have
\ifCLASSOPTIONtwocolumn
\begin{align*}
	&\norm{x^k-x^{k-1}+\frac{\mu}{2} \g }^2 \\
	&\leq  \norm{\pp_{\Cc}(x^{k-1}-\frac{\mu}{2} \g)-x^{k-1}+\frac{\mu}{2} \g }^2 +(\nup^k)^2\\
	&\leq \norm{x^\gt-x^{k-1}+\frac{\mu}{2} \g }^2 +(\nup^k)^2.
\end{align*}
\else
\begin{align*}
\norm{x^k-x^{k-1}+\frac{\mu}{2} \g }^2 
&\leq  \norm{\pp_{\Cc}(x^{k-1}-\frac{\mu}{2} \g)-x^{k-1}+\frac{\mu}{2} \g }^2 +(\nup^k)^2\\
&\leq \norm{x^\gt-x^{k-1}+\frac{\mu}{2} \g }^2 +(\nup^k)^2.
\end{align*}
\fi
The last inequality holds for any member of $\Cc$ and thus here for $x^\gt$. Therefore we can write
\ifCLASSOPTIONtwocolumn
\begin{align}
&\norm{y-Ax^k}^2-\norm{y-Ax^{k-1}}^2 \nonumber	\\
%&=\langle x^{t+1}-x^t,g \rangle + \frac{1}{\mu} \norm{x^{t+1}-x^t}^2 \nonumber\\
&\leq \frac{1}{\mu} \norm{x^\gt-x^{k-1}+\frac{\mu}{2} \g }^2 - \frac{\mu}{4} \norm{\g}^2 \nonumber\\
&\qquad - \langle x^k-x^{k-1},2\eg^k \rangle +(\frac{\nup^k}{\sqrt\mu})^2 \nonumber \\
&= \langle x^\gt-x^{k-1},\g \rangle + \frac{1}{\mu} \norm{x^\gt-x^{k-1}}^2 \nonumber\\
&\qquad - \langle x^k-x^{k-1},2\eg^k \rangle 
 +(\frac{\nup^k}{\sqrt\mu})^2\nonumber\\
%&= \langle x^\gt-x^{k-1},g \rangle + \frac{1}{\mu} \norm{x^\gt-x^{k-1}}^2 - \langle x^k-x^*,2\eg^k \rangle +\frac{\nup^k}{\mu}\nonumber\\
&\leq \langle x^\gt-x^{k-1},g \rangle + \frac{1}{\mu} \norm{x^\gt-x^{k-1}}^2 \nonumber\\
&\qquad+2\nug^k\norm{x^k-x^*} +(\frac{\nup^k}{\sqrt\mu})^2. \label{eq:p1b2}
\end{align}
\else
\begin{align}
\norm{y-Ax^k}^2-\norm{y-Ax^{k-1}}^2 \nonumber	
%&=\langle x^{t+1}-x^t,g \rangle + \frac{1}{\mu} \norm{x^{t+1}-x^t}^2 \nonumber\\
&\leq \frac{1}{\mu} \norm{x^\gt-x^{k-1}+\frac{\mu}{2} \g }^2 - \frac{\mu}{4} \norm{\g}^2 
 - \langle x^k-x^{k-1},2\eg^k \rangle +(\frac{\nup^k}{\sqrt\mu})^2 \nonumber \\
&= \langle x^\gt-x^{k-1},\g \rangle + \frac{1}{\mu} \norm{x^\gt-x^{k-1}}^2 
 - \langle x^k-x^{k-1},2\eg^k \rangle 
+(\frac{\nup^k}{\sqrt\mu})^2\nonumber\\
%&= \langle x^\gt-x^{k-1},g \rangle + \frac{1}{\mu} \norm{x^\gt-x^{k-1}}^2 - \langle x^k-x^*,2\eg^k \rangle +\frac{\nup^k}{\mu}\nonumber\\
&\leq \langle x^\gt-x^{k-1},g \rangle + \frac{1}{\mu} \norm{x^\gt-x^{k-1}}^2 
+2\nug^k\norm{x^k-x^*} +(\frac{\nup^k}{\sqrt\mu})^2. \label{eq:p1b2}
\end{align}
\fi
The last line replaces $\g= g+2\eg^k$ and uses the Cauchy-Schwartz inequality. 


Similarly we use the LLE property in Definition \ref{def:Lip} to obtain an upper bound on $ \langle x^\gt-x^{k-1},g \rangle$:
\begin{align*} 
\langle x^\gt-x^{k-1},g \rangle 	&= w^2-\norm{y-Ax^{k-1}}^2 +\norm{A(x_0-x^{k-1})}^2 \\
&\leq w^2 -\norm{y-Ax^{k-1}}^2 +\mmx\norm{x^\gt-x^{k-1}}^2,
\end{align*}
where $w=\norm{ y-Ax^\gt}$. Replacing this bound in \eqref{eq:p1b2} and cancelling $-\norm{y-Ax^{k-1}}^2$ from both sides of the inequality yields
\ifCLASSOPTIONtwocolumn
\begin{align}
&\norm{y-Ax^k}^2- 2\nug^k\norm{x^k-x^\gt}\nonumber \\ 
&\leq \left(\frac{1}{\mu}-\mmx \right)\norm{x^{k-1}-x^\gt}^2 + (\frac{\nup^k}{\sqrt\mu})^2+w^2. \label{eq:p1b3}
\end{align}
\else
\begin{align}
\norm{y-Ax^k}^2- 2\nug^k\norm{x^k-x^\gt}
\leq \left(\frac{1}{\mu}-\mmx \right)\norm{x^{k-1}-x^\gt}^2 + (\frac{\nup^k}{\sqrt\mu})^2+w^2. \label{eq:p1b3}
\end{align}
\fi
We continue to lower bound the left-hand side of this inequality:
\ifCLASSOPTIONtwocolumn
\begin{align*}
&\norm{y-Ax^k}^2- 2\nug^k\norm{x^k-x^\gt}\\
&= \norm{A(x^k-x^\gt)}^2+w^2-2\langle y-Ax^\gt, A(x^k-x^\gt)\rangle\\
&- 2\nug^k\norm{x^k-x^\gt} \\
&\geq \norm{A(x^k-x^\gt)}^2+w^2-2w \norm{A(x^k-x^\gt)}\\
&- 2\nug^k\norm{x^k-x^\gt} \\
& \geq \mmx\norm{x^k-x^\gt}^2+w^2-2(w \sqrt{\MM}+\nug^k)\norm{x^k-x^\gt}\\
&= \left(\sqrt{\mmx}\norm{x^k-x^\gt}-\frac{\nug^k}{\sqrt{\mmx}}- \sqrt{\frac{\MM}{\mmx}}w\right)^2 \\
&- (\frac{\nug^k}{\sqrt{\mmx}})^2 -(\frac{\MM}{\mmx}-1)w^2.
\end{align*}
\else
\begin{align*}
\norm{y-Ax^k}^2- 2\nug^k\norm{x^k-x^\gt}
&= \norm{A(x^k-x^\gt)}^2+w^2-2\langle y-Ax^\gt, A(x^k-x^\gt)\rangle- 2\nug^k\norm{x^k-x^\gt} \\
&\geq \norm{A(x^k-x^\gt)}^2+w^2-2w \norm{A(x^k-x^\gt)}
- 2\nug^k\norm{x^k-x^\gt} \\
& \geq \mmx\norm{x^k-x^\gt}^2+w^2-2(w \sqrt{\MM}+\nug^k)\norm{x^k-x^\gt}\\
&= \left(\sqrt{\mmx}\norm{x^k-x^\gt}-\frac{\nug^k}{\sqrt{\mmx}}- \sqrt{\frac{\MM}{\mmx}}w\right)^2 - (\frac{\nug^k}{\sqrt{\mmx}})^2 -(\frac{\MM}{\mmx}-1)w^2.
\end{align*}
\fi
The first inequality uses the Cauchy-Schwartz's and the second inequality follows from the ULE and LLE properties. Using this bound together with \eqref{eq:p1b3} we get
\ifCLASSOPTIONtwocolumn
\begin{align*}
&\left(\sqrt{\mmx}\norm{x^k-x^\gt}-\frac{\nug^k}{\sqrt{\mmx}}- \sqrt{\frac{\MM}{\mmx}}w\right)^2\\
&\leq \left(\frac{1}{\mu}-\mmx \right)\norm{x^{k-1}-x^\gt}^2 + (\frac{\nug^k}{\sqrt{\mmx}})^2+ (\frac{\nup^k}{\sqrt\mu})^2+\frac{\MM}{\mmx}w^2 \\
&\leq \left(\sqrt{\frac{1}{\mu}-\mmx} \norm{x^{k-1}-x^\gt} + \frac{\nug^k}{\sqrt{\mmx}}+ \frac{\nup^k}{\sqrt\mu}+\sqrt{\frac{\MM}{\mmx}}w \right)^2.
\end{align*}
\else
\begin{align*}
\left(\sqrt{\mmx}\norm{x^k-x^\gt}-\frac{\nug^k}{\sqrt{\mmx}}- \sqrt{\frac{\MM}{\mmx}}w\right)^2
&\leq \left(\frac{1}{\mu}-\mmx \right)\norm{x^{k-1}-x^\gt}^2 + (\frac{\nug^k}{\sqrt{\mmx}})^2+ (\frac{\nup^k}{\sqrt\mu})^2+\frac{\MM}{\mmx}w^2 \\
&\leq \left(\sqrt{\frac{1}{\mu}-\mmx} \norm{x^{k-1}-x^\gt} + \frac{\nug^k}{\sqrt{\mmx}}+ \frac{\nup^k}{\sqrt\mu}+\sqrt{\frac{\MM}{\mmx}}w \right)^2.
\end{align*}
\fi
The last inequality assumes $\mu\leq \mmx^{-1}$ which holds since we previously assumed $\mu\leq \MM^{-1}$. As a result we deduce that
\begin{align}
\norm{x^k-x^\gt}\leq \rho \norm{x^{k-1}-x^\gt} + \nut^k + 2\frac{\sqrt{\MM}}{\mmx}w \label{eq:p1b4}
\end{align}
for $\rho$ and $\nut^k$ defined in Theorem \ref{th:inexactLS1}. Applying this bound recursively (and setting $x^0=0$) completes the proof:
\begin{align*}
\norm{x^k-x^\gt}\leq \rho^k \norm{x^\gt} + \sum_{i=1}^k \rho^{k-i} \nut^i + \frac{2\sqrt{\MM}}{\mmx(1-\rho)}w.
\end{align*} 
Note that for convergence we require $\rho<1$ and therefore, a lower bound on the step size which is $\mu> (2\mmx)^{-1}$. 

\subsection{Proof of Corollary~\ref{cor:decay}}
Following the error bound \eqref{eq:errbound} derived in   Theorem~\ref{th:inexactLS1} and by setting $\nut^k\leq C r^k$ we obtain:
		\eq{
			\norm{x^{k}-x^\gt}\leq  \rho^k \left(\norm{x^\gt}+C\sum_{i=1}^k (r/\rho)^{i}  \right)+ \frac{2\sqrt{\MM}}{\mmx(1-\rho)}w,			
		}
which for $r<\rho$ it implies 		
		\eq{
			\norm{x^{k}-x^\gt}\leq 
			 \rho^k \left(\norm{x^\gt}+\frac{C}{1-r/\rho}\right)+ \frac{2\sqrt{\MM}}{\mmx(1-\rho)}w,
		}
and for $r>\rho$ implies %and following \eqref{eq:errbound} we get		
\begin{align*}
\norm{x^{k}-x^\gt}&\leq  \rho^k \norm{x^\gt}+C r^k \sum_{i=1}^k (\rho/r)^{k-i}  + \frac{2\sqrt{\MM}}{\mmx(1-\rho)}w\\
&\leq r^k \left(\norm{x^\gt}+\frac{C}{1-\rho/r}\right)+ \frac{2\sqrt{\MM}}{\mmx(1-\rho)}w,	
\end{align*}
and for $r=\rho$ we immediately get
\eq{
\norm{x^{k}-x^\gt}\leq  \rho^k \norm{x^\gt}+C k \rho^k + \frac{2\sqrt{\MM}}{\mmx(1-\rho)}w.	
}
Note that there exists a constant $c$ such that for an arbitrary small $\xi>0$ it holds $k\rho^k\leq c(\rho+\xi)^k$. Therefore we also achieve a linear convergence for the case $r=\rho$.
\subsection{Proof of Theorem \ref{th:inexactLS2}}
As before set $g= 2A^T(Ax^{k-1}-y)$ and $\g= g+2\eg^k$ for some bounded gradient error vector $\eg^k$ i.e. $\norm{\eg^k}\leq \nug^k$. Note that 
here the update rule of Algorithm \eqref{eq:inIP2} uses the  $(1+\epsilon)$-approximate projection  which by definition \eqref{eq:eproj} implies
\ifCLASSOPTIONtwocolumn
\begin{align*}
&\norm{x^k-x^{k-1}+\frac{\mu}{2} \g }^2 =  \norm{\pp^{\epsilon}_{\Cc}(x^{k-1}-\frac{\mu}{2} \g)-x^{k-1}+\frac{\mu}{2} \g }^2\\
&\leq  (1+\epsilon)^2\norm{\pp_{\Cc}(x^{k-1}-\frac{\mu}{2} \g)-x^{k-1}+\frac{\mu}{2} \g }^2\\
&\leq \norm{x^\gt-x^{k-1}+\frac{\mu}{2} \g }^2 + \phi(\epsilon)^2\frac{\mu^2}{4}\norm{\g}^2
\end{align*}
\else
\begin{align*}
\norm{x^k-x^{k-1}+\frac{\mu}{2} \g }^2 &=  \norm{\pp^{\epsilon}_{\Cc}(x^{k-1}-\frac{\mu}{2} \g)-x^{k-1}+\frac{\mu}{2} \g }^2\\
&\leq  (1+\epsilon)^2\norm{\pp_{\Cc}(x^{k-1}-\frac{\mu}{2} \g)-x^{k-1}+\frac{\mu}{2} \g }^2\\
&\leq \norm{x^\gt-x^{k-1}+\frac{\mu}{2} \g }^2 + \phi(\epsilon)^2\frac{\mu^2}{4}\norm{\g}^2
\end{align*}
\fi
where $\phi(\epsilon):=\sqrt{2\epsilon+\epsilon^2}$. For the last inequality we replace $\pp_{\Cc}(x^{k-1}-\frac{\mu}{2} \g)$ with two feasible points $x^\gt,x^{k-1}\in \Cc$. 

As a result by only replacing $\nug^k$ with $\mu\phi(\epsilon)\norm{\g}/2$, we can follow identical steps as for the proof of Theorem \ref{th:inexactLS1} up to \eqref{eq:p1b4}, revise the definition of $\nut^k:={2\nug^k}/{\mmx} + {\sqrt{\mu}\phi(\epsilon)\norm{\g}}/(2\sqrt{{\mmx}})$ and write
\ifCLASSOPTIONtwocolumn
\begin{align*}
\norm{x^k-x^\gt}\leq& \sqrt{\frac{1}{\mu\mmx}-1} \norm{x^{k-1}-x^\gt} \\
&+ \frac{2\nug^k}{\mmx} +\frac{\phi(\epsilon)}{2} \sqrt{\frac{\mu}{\mmx}}\norm{\g} + 2\frac{\sqrt{\MM}}{\mmx}w. 
\end{align*}
\else
\begin{align*}
\norm{x^k-x^\gt}\leq \sqrt{\frac{1}{\mu\mmx}-1} \norm{x^{k-1}-x^\gt} 
+ \frac{2\nug^k}{\mmx} +\frac{\phi(\epsilon)}{2} \sqrt{\frac{\mu}{\mmx}}\norm{\g} + 2\frac{\sqrt{\MM}}{\mmx}w. 
\end{align*}
\fi
Note that so far we only assumed $\mu\leq \MM^{-1}$. 

On the other hand by triangle inequality we have
\begin{align*}
	\norm{\g}&\leq \norm{g}+2\nug^k\\
	&\leq 2\norm{A^TA(x^{k-1}-x^\gt)}+2\norm{A^T(y-Ax^\gt)}+2\nug^k \\
	&\leq 2\sqrt \MM\vertiii{A}\norm{(x^{k-1}-x^\gt)}+2\vertiii{A}w+2\nug^k\\
	&\leq 2\sqrt{ 1/\mu}\vertiii{A}\norm{(x^{k-1}-x^\gt)}+2\vertiii{A}w+2\nug^k.
\end{align*}
The third inequality uses the ULE property and the last one holds since $\mu\leq \MM^{-1}$.
Therefore, we get
\ifCLASSOPTIONtwocolumn
\begin{align*}
&\norm{x^k-x^\gt}\leq
\left(\sqrt{\frac{1}{\mu\mmx}-1}+ \phi(\epsilon)\frac{\vertiii{A}}{\sqrt{\mmx}}\right) \norm{x^{k-1}-x^\gt} \\
&+ \left( \frac{2}{\mmx} +\phi(\epsilon){\sqrt{\frac{\mu}{\mmx}}}\right) \nug^k 
+ \left( 2\frac{\sqrt{\MM}}{\mmx}+ \phi(\epsilon)\sqrt{\frac{\mu}{\mmx}} \vertiii{A} \right)w. 
\end{align*}
\else
\begin{align*}
\norm{x^k-x^\gt}\leq&
\left(\sqrt{\frac{1}{\mu\mmx}-1}+ \phi(\epsilon)\frac{\vertiii{A}}{\sqrt{\mmx}}\right) \norm{x^{k-1}-x^\gt} \\
&+ \left( \frac{2}{\mmx} +\phi(\epsilon){\sqrt{\frac{\mu}{\mmx}}}\right) \nug^k 
+ \left( 2\frac{\sqrt{\MM}}{\mmx}+ \phi(\epsilon)\sqrt{\frac{\mu}{\mmx}} \vertiii{A} \right)w. 
\end{align*}
\fi
Based on assumption $\phi(\epsilon)\frac{\vertiii{A}}{\sqrt{\mmx}}\leq \delta$ of the theorem  we can deduce
\ifCLASSOPTIONtwocolumn
\begin{align*}
\norm{x^k-x^\gt}\leq&
\rho \norm{x^{k-1}-x^\gt} + \left( \frac{2}{\mmx} +\frac{\sqrt \mu}{\vertiii{A}} \delta\right) \nug^k \\
&
+ \left( 2\frac{\sqrt{\MM}}{\mmx}+\sqrt{\mu}\delta \right)w 
\end{align*}
\else
\begin{align*}
\norm{x^k-x^\gt}\leq
\rho \norm{x^{k-1}-x^\gt} + \left( \frac{2}{\mmx} +\frac{\sqrt \mu}{\vertiii{A}} \delta\right) \nug^k 
+ \left( 2\frac{\sqrt{\MM}}{\mmx}+\sqrt{\mu}\delta \right)w 
\end{align*}
\fi
where $\rho=\sqrt{\frac{1}{\mu\mmx}-1}+\delta$.

Applying this bound recursively (and setting $x^0=0$) completes the proof:
\eq{
\norm{x^{k}-x^\gt}\leq  \rho^k \norm{x^\gt}+\kappa_g \sum_{i=1}^k \rho^{k-i} \nug^i+ \frac{\kappa_w}{1-\rho}w
}
for $\kappa_g, \kappa_w$ defined in Theorem \ref{th:inexactLS2}. The condition for convergence is $\rho<1$ which implies $\delta<1$ and a lower bound on the step size which is $\mu> (\mmx+(1-\delta)^2\mmx)^{-1}$. 
		

\end{document}
