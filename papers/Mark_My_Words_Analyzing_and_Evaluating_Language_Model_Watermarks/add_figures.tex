\subsection{Additional analysis}
\label{app:ssec:additional_figures}

Here we present additional figures to support the main text.

\begin{figure}
    \centering
    \includegraphics[width=1\linewidth]{figures/fig1b.png}
    \caption{Same plot as~\cref{fig:aggregate}, but for a quality threshold of 10\%, and a minimum robustness of 0.2. The same conclusions from the main figure still hold.}
    \label{fig:aggregate2}
\end{figure}

\begin{figure}[t]
    \includegraphics[width=\linewidth]{figures/robustness_to_attack.png}
    \centering
    \caption{Correlation between robustness metric and attack success. On the right, for the Russian translation attack. 
    On the left, for the GPT paraphrasing attack. Each dot is a unique watermark parameter setting.}
    \label{fig:robustness-to-attacks}
\end{figure}

\begin{figure}[t]
\includegraphics[width=\linewidth]{figures/delta.png}
\centering
\caption{Size and quality for varying biases, at T=0.3 and T=1. The quality is relative to the quality of the non-watermarked model at the given temperature.}
\label{bias-fig}
\end{figure}

\begin{figure}[t]
\includegraphics[width=\linewidth]{figures/keylen.png}
\centering
\caption{Size and tamper resistance as a function of key lengths (only using distribution-shift schemes with $\delta \leq 2$).}
\label{keylen-fig}
\end{figure}