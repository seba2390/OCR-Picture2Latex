\subsection{Approach Overview}

\begin{figure}[t]
	\centering
    \includegraphics[width=0.6\textwidth]{images/overview.pdf}
    \caption{Approach Overview} 
    \label{fig:approach_overview}
\end{figure}

Figure~\ref{fig:approach_overview} shows the overview of \nickName{}. Given an Android app, \nickName{} automatically decomposes the app into decoupled bundles, and enables users to install these bundles on demand. The decomposition of \nickName{} performs at the activity level, i.e., each activity is regarded as an atomic feature that may be desired by users~\cite{Ubicomp17Lu}. We choose this decomposition level because Android advocates component-based development and developers tend to design different features in different activities, which are Android's basic components that provide user interfaces for users to interact with. After decomposition, \nickName{} generates two types of bundles:

\begin{itemize}
	\item {A \textbf{base bundle}, which contains the code and resources related to app launching and the frequently used
activities. In general, launching an app is to load its main activity. But before loading the main activity, there may be some other \emph{``welcome''} activities, such as splash activities to show advertisements. Therefore, \nickName{} always keeps the code and resources related to the main activity as well as those \emph{welcome} activities into the base bundle. In addition, base bundle also includes code and resources that cannot be decomposed and should be mandatorily included in the installation package such as Services, BroadcastReceivers, and assets, which are important to ensure the correctness of the app behaviors.}
    \item {A set of \textbf{feature bundles}, each of which contains the code and resources related to a specific activity (called feature activity) that
        is not in the base bundle. A feature bundle is requested when users navigate to a new activity that is not included in the base bundle and has not been visited yet. By default, all feature bundles are not installed on the device. When users' actions trigger the navigation to an unvisited activity, AppRover downloads the code and resources from the corresponding feature bundle, merges resources, and loads the code dynamically.}
\end{itemize}


In order to support on-demand installation of bundles, \nickName{} has an app-level virtualization space to hook system services and take over the whole life-cycle management of an app, including installation, running, and uninstallation\footnote{In theory, it can be integrated into the Android system as system services to achieve better performance, but it would require modifications of the Android system.}. The app-level virtualization space is hosted as an Android app called \emph{\nickName{} client}. Users just need to deploy the \nickName{} client on their devices, and choose to install and launch the decomposed apps (actually are the corresponding base bundles). To some extent, we can regard the \nickName{} client as a Web browser: the Web browser downloads Web-page related resources and renders Web pages on demand; the \nickName{} client downloads and installs feature bundles on demand. The difference is that users should manually download the base bundle of an app to launch it in the \nickName{} client, unlike Web browser where all the Web pages are visited in the same way.
In the remaining part of this section, we describe the details of app decomposition, app virutalization, and on-demand installation, respectively.
