\section{Requirements and Key Challenges}
Motivated by the empirical study, we propose an approach that aims to alleviate the software-bloat problem in Android apps.
Android app development advocates the component-based development paradigm, and thus app developers tend to design their apps to be modularized.
Such modularity of Android apps present opportunities for \emph{decomposing an app into separated bundles}, each of which contains parts of the app's code and resources.
For each app, users just need to download and install a base bundle that can correctly launch the app and include the frequently used features.
In this way, the users keep only those code and resources related to their desired features on the devices so that the device's storage could be saved and the app's performance such as memory usage could be improved.
When users need to use the features not in the base bundle (i.e., those infrequently used features), we could allow the related bundles to be dynamically downloaded and installed at runtime,
achieving \emph{on-demand installation of bundles}.
Considering the current Android ecosystem, the proposed approach should satisfy the following requirements:

\noindent\textbf{Decomposable}. The approach should decompose an app into several bundles. Each bundle should be correctly launched and used just as the original app does.

\noindent\textbf{Deployable}. Users can easily and regularly deploy the decomposed apps on their devices without additional requirements compared to the current app-delivery mechanism. If any modifications of the Andoird system were requested, it is impractical to deploy and distribute our approach.

\noindent\textbf{User-friendly}. The approach should not compromise the user experiences or change the way of user interactions. The decomposed app should behave as expected as if the original app is being used.

\noindent\textbf{Developer-friendly}. The approach should work for legacy and on-the-shelf apps so that developers do not have to manually re-develop their apps to adopt the approach.



However, there are two key challenges for Android app decomposition into bundles and on-demand installation of the bundles.
First, it is difficult to obtain the complete dependencies among code and resources via only static analysis. For example, developers may use Java reflection to invoke methods or access resources indirectly. If we cannot decompose an app correctly, it would suffer crashes due to missing code or resources.
Second, it is difficult to enable the on-demand installation without modifying the Android system. We need to dynamically load the newly-downloaded code and resources without changing Android's runtime. Meanwhile, we need to identify the exact entry point to launch the feature activity and dynamically load feature bundles beforehand. However, in some cases, the target activity can be decided only by the OS at runtime if developers use an implicit intent to launch the target activity.
For example, an app may use string concatenation based on variables to construct activity names. Without correctly resolving the activity name in runtime, it is impossible to dynamically load feature bundles beforehand.




