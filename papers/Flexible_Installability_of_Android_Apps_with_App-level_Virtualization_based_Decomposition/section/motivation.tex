\section{A Motivating Empirical Study}
\label{sec:motivation}

First, we aim to know how and how much an app is used at a fine granularity inside an app. To this end, we conduct an empirical study to understand the evolving app complexity and app usage patterns in Android apps by answering three research questions:

\begin{itemize}
	\item {\textbf{RQ 1:} \textit{How does the functionality complexity of apps evolve as the apps upgrade?} As apps keep upgrading, they tend to become more and more complex with more functionalities to attract users. By answering this question, we can understand the software bloat problem in the Android ecosystem.}
	\item {\textbf{RQ 2:} \textit{How many features in an app are used by different users?} Although mobile apps tend to provide more features to retain their users, users usually use only certain features of an app rather than all of them, just as the Pareto principle~\cite{Pareto_principle} points out. By answering this question, we can understand whether all provided features of an app are really necessary for users.}
	\item {\textbf{RQ 3:} \textit{Do different users focus on different features in an app?} By answering this question, we can understand the diversity of app usage patterns.}
\end{itemize}


\begin{figure}
   \centering
  \subfigure[The APK volume of the first version and the latest version]{
    \label{fig:topapp:a}
    \includegraphics[width=0.4\textwidth]{images/top_apk_size.pdf}}
  \hspace{0in}
  \subfigure[The activity number of the first version and latest version]{
    \label{fig:topapp:b}
    \includegraphics[width=0.44\textwidth]{images/top_apk_activity_num.pdf}}
  \caption{The change of APK size and activity number of top 50 apps in Wandoujia market}
  \label{fig:topapp}
\end{figure}

\subsection{Ever-Increasing App Complexity}
To study RQ 1, i.e., \emph{how does the functionality complexity of apps evolve as the apps upgrade}, we choose the top-50 most downloaded apps in a leading Android AppStore in China, namely  Wandoujia\footnote{Till 2017, Wandoujia has more than 4 millions of users and 1.8 millions of apps~\cite{wandoujia}}.
For simplicity, we download each app's APK file (which is the installation package of Android apps)  for the first version that can be publicly found and the latest version in October, 2017.
We use the size of APK files and the number of activities as two metrics to measure an app's functionality complexity because (1) larger installation packages usually imply more features, and (2)  activities, also referred to pages, are the basic components in Android apps to provide a unique interactive functionality for users.

Figure~\ref{fig:topapp} shows the distribution of APK size and the number of activities of the two versions from the 50 chosen apps.
We find that both the APK size and the number of activities have increased dramatically, by comparing the first version and the latest version.
With the version evolving, apps have increased by 3.16$\sim$60.76 times in size and 1.53$\sim$143 times in the number of activities.
The median values for the size increase and the number of activities are 13.85 times and 8.44, respectively.
Such results indicate that Android apps are becoming more and more complex with the apps evolve.

Take WeChat\footnote{WeChat is a popular IM app with more than 900 millions of users.} as an example. When WeChat was first released in 2011, it just provided the basic feature of instant messaging, and the size of installation package was just 4.74MB. However, after 6 years, the latest version of WeChat provides a large number of features, such as online shopping, online payment, and news reading. Meanwhile, the size of WeChat's installation package has increased to 40.42MB, about 8.5 times larger compared to the first version.


\subsection{Sparse App Usage}

To study RQ 2, i.e., \emph{how many features in an app are used by different users}, we conducted a field study to understand how users interact with mobile apps. We developed an Android client, which periodically records the activity name and the package name of the foreground app. Then, we hired 38 in-college student volunteers and deployed the client on their Android smartphones. The data collection lasted for three months (Feb to April, 2016), and we finally collected 894,542 records, containing 3,527 activities from 389 apps\footnote{The data collection and analysis process was conducted with strict IRB approval from the Research Ethic Committee of the authors' institute that is anonymous for review.}. We filter out the records related to system apps and self-developed apps that cannot be downloaded in Android markets, and finally acquire a dataset consisting of 240 apps.

For each app, we count the number of all the unique activities that are visited by the app's users, and divide the number by the total number of activities, to compute the feature-usage ratio of the app. Figure~\ref{fig:used_activity_rate} depicts the distribution of the feature-usage ratio among all the 240 apps. We can see that the users visit only less than 20\% of the activities in most of the apps (about 87\%). Such a result indicates that the users need only a small set of features in mobile apps but they have to download the full-size installation package. Meanwhile, Li~\cite{WWW2016Li} finds that about 60\% of abandoned apps can survive for only less than 1.5 days, and about 80\% of abandoned apps can survive for less than a week. From the users' perspective, it is annoying to make unnecessary features in their devices, resulting in more occupation of local storage, memory, and other resources. As such, an infrequently used app of huge size could be removed by the user in a short time.

\begin{figure}
   \centering
  \subfigure[Users only visit part of features in mobile application]{
    \label{fig:used_activity_rate}
    \includegraphics[width=0.4\textwidth]{images/used_activity_rate.pdf}}
  \hspace{0in}
  \subfigure[The entropy of visited pages for each application]{
    \label{fig:visited_pages_entropy}
    \includegraphics[width=0.4\textwidth]{images/activity_sn.pdf}}
  \caption{An empirical study of how end users interact with mobile applications}
  \label{fig:empirical_study}
\end{figure}

\subsection{Diverse Usage among Users}
To study RQ 3, i.e., \emph{do different users focus on different features in an app}, we still use the dataset collected for RQ 2, choose the apps that have more than 5 users, and obtain 43 apps in total. We simply use the entropy~\cite{entropy} to measure the diversity of activity usage among users. We count the number of users for each visited activity of an app, and each visited activity has a probability of being visited by a user. For every single app, we can calculate the entropy as shown in Equation~\ref{entropy},  where $n$ is the number of activities that have been visited by users, and $p_{i}$ is the probability of being visited for an activity.
\setlength{\belowdisplayskip}{0pt} \setlength{\belowdisplayshortskip}{0pt}
\setlength{\abovedisplayskip}{0pt} \setlength{\abovedisplayshortskip}{0pt}
\begin{equation}
	\label{entropy}
	E=-\sum _{i=1}^{n}p_{i}\ln p_{i}
\end{equation}

If all activities are visited equally, all $p_{i}$ values equal $1 / n$, and the entropy hence takes the value $\ln(n)$ (upper bound). The more unequal probabilities of being visited, the larger the weighted geometric mean of the $p_{i}$ values, and the smaller the corresponding entropy. If users' visits focus on one activity, and other activities are rarely visited (even if there are many of them), entropy approaches zero (lower bound). Figure~\ref{fig:visited_pages_entropy} shows the distribution of the entropy among the 43 apps. We can find that each application has an entropy between its upper bound and lower bound, indicating that its activities have quite diverse probability of being visited.


\subsection{Findings and Implications}

Our preceding empirical study shows that Android apps become more and more complex with increasing size and activities. However, only a limited set of activities are frequently visited, and some activities may never be visited. This finding motivates us to \emph{remove those infrequently visited activities} to mitigate the software bloat problem in Android ecosystem.

However, we also observe that different users focus on different features in an app.  This observation implies that these users may need to use an infrequently visited activity occasionally. Therefore, we cannot just simply remove those infrequently visited activities, which would cause crashes when the users visit the removed activities.
We need to devise a mechanism that allows users to not only keep \emph{the basic and frequently used features} when the apps are downloaded,
but also can visit \emph{the infrequently used features whenever necessary} (i.e., on demand).



