\section{Background}
\label{sec:background}



\subsection{Inter-Component dependencies}

\begin{figure}
    \includegraphics[width=0.45\textwidth]{images/iccg.pdf}
    \caption{Inter-Component dependencies in Android applications.} \label{fig:atg}
\end{figure}


\textbf{Android Component}. Android applications consist of a set of reusable components, including Activity, Service, Content Provider, and BroadcastReceiver. An Activity is the front page that displays content and deals with users’ interaction. 
\textbf{Intent}. An intent is an abstract description of an operation to be performed[2]. Intent is one of the most heavily used Inter-Component Communication (ICC) mechanisms in Android. It can be used with \textit{startActivity} to launch an Activity.


\textbf{Activity Transition Graph (ATG)}. An Activity can transit to another Activity through implicit intent or explicit intent, and all Activitys and their transitions can generate an ATG. It is a directed graph, where an edge from parent to child means there is a transition from the parent to child. Developers can select a subset of pages to publish as a tiny-version application, and end users can download a succinct application without unwanted features. However, we need to address that an Activity may only be reached by another Activity. Therefore, we have to keep the prerequisite page so that the target page can be reached. As shown in Figure~\ref{fig:atg}, Activity C can only be arrived by path Launcher Activity $\rightarrow$ Activity A $\rightarrow$ Activity C. If developers or end users need to keep Activity A, we also need keep both of Launcher Activity and Activity A. To keep Activity B, we just need keep Launcher Activity. Based on our observation, we can divide unwanted components into three types.
\begin{itemize}
	\item{\textit{Isolated Activity}. An isolated Activity is a Activity with no predecessor node or successor node on the ATG.  Since an isolated Activity has no successor node on the ATG, it is guaranteed that deleting the isolated Activity will not break any dependencies in other pages of the app. If developers or end users want to delete Activity A, Activity C becomes an isolated component. Since there are no entries to start Activity C, we need to delete Activity C.}
	
	\item{\textit{Receiving-Only Activity}. A receiving-only Activity is a Activity with no successor node but with predecessor node on the ATG. Receiving-Only Activitys receive inter-component communication from other Activitys but do not initiate communications to other Activitys. We can delete such Activitys, and would not affect the paths to other Activitys. However, we cannot directly delete Activity D, since Activity B has an entry to start Activity D. Therefore, we also need to disable or edit the entry to Activity D so that the application would not crash when user’s operations trigger the entry to Activity D in Activity B.}
	\item{\textit{Mediate Activity}. A mediate Activity is a Activity with both predecessor nodes and successor  nodes on the ATG. Although developers or end users want to eliminate a mediate Activity, its successor node, which is needed, may only has single incoming ICC from the Activity. In other world, the needed Activity can only be invoked by the mediate Activity, and we have to keep the mediate Activity.}
\end{itemize}

Although developers and end users put forward a set of wanted pages and unwanted pages, we cannot blindly delete unwanted pages. In order to make a wanted page reachable, we have to keep some unwanted pages. Meanwhile, we need to guarantee the applications’ robustness when delete unwanted pages. A wanted page may have outcoming edge to the deleted page (For example, clicking on an button to open the deleted page), and we need to disable or edit the entry (For example, remove the callback of click listener or edit the method to open a streaming page with remote server).


\subsection{Resources dependency and Code dependency}
	Android developers develop user interface by assigning layout files to target pages, and customize the format and look with style resources, image resources, customized view, and so on. A layout file may depend on many drawable files (icons and background images), other layout files, style files. Meanwhile, classes are interdependent. Therefore, we need to consider the resources dependency and code dependency when we remove unwanted features.
	
	As shown in Listing~\ref{code_res}, MainActivity defines its layout file with an identifier \textit{R.layout.activity\_main}. The content of this layout file is shown is Listing~\ref{layout_xml}. This layout file identifies its background with the \textit{android:background} property, whose value declares the background image. This layout file has a customized layout component \textit{cn.edu.pku.CustomizedView} and a button component, which has a property that declares the background with a drawable file. The drawable file is shown in Listing~\ref{drawable_xml}, and declares the real background images for different states. Actually, Android provides many kinds of resources, and we can use these identifies to infer such dependencies.
	
	\definecolor{pblue}{rgb}{0.13,0.13,1}
\definecolor{pgreen}{rgb}{0,0.5,0}
\definecolor{pred}{rgb}{0.9,0,0}
\definecolor{pgrey}{rgb}{0.46,0.45,0.48}
\begin{lstlisting}[frame=single,
   basicstyle=\small\ttfamily,
   breaklines=true,
   style = customerStyle,
   commentstyle=\color{pgreen},
  keywordstyle=\color{pblue},
  stringstyle=\color{pred},
  caption = Each Activity identifies its layout file ,
 	label = code_res,
	 captionpos=b
]

public class MainActivity extends Activity {
  @Override
  protected void onCreate(Bundle savedInstanceState){
    super.onCreate(savedInstanceState);
    setContentView(R.layout.activity_main);
    ...
  }
}
\end{lstlisting}


\begin{lstlisting}[frame=single,
   basicstyle=\small\ttfamily,
   breaklines=true,
   language=xml,
   commentstyle=\color{pgreen},
  keywordstyle=\color{pblue},
  stringstyle=\color{pgreen},
  caption = Layout file,
 label = layout_xml,
 captionpos=b
]
<LinearLayout android:background="@drawable/background">
    <cn.edu.pku.CustomizedView>
        <Button android:background="@drawable/button"/>
    </cn.edu.pku.CustomizedView>
</LinearLayout>
\end{lstlisting}


\begin{lstlisting}[frame=single,
   basicstyle=\small\ttfamily,
   breaklines=true,
   language=xml,
   commentstyle=\color{pgreen},
  keywordstyle=\color{pblue},
  stringstyle=\color{pgreen},
  caption = Drawable file,
 label = drawable_xml,
 captionpos=b
]
<selector>
    <item android:state_pressed="true"
          android:drawable="@drawable/button_pressed" />
</selector>
\end{lstlisting}