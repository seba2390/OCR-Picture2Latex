\section{Discussion}
\label{sec:discussion}
In this section, we discuss some issues that could affect the effectiveness of our approach.

\noindent \textbf{Missing code and resources}. Due to the limitation of static program  analysis, the decomposed bundles may miss some code or resources. Therefore, \nickName{} uses the record-and-replay technique to iteratively launch each bundle, performs actions in the corresponding activity to explore the  behaviors, and collects the information of missing code and resources that are later included in the bundle. However, it is difficult to exhaustively explore all the behaviors in an activity. As a result, there may still exist some missing code and resources when the unexplored behaviors are triggered. To address this issue, we plan to  develop a runtime complementary mechanism in the \nickName{} client to retrieve the missing code and resources.

\noindent \textbf{App-level virtualization}. Although \nickName{} indeed brings improvement compared to directly running an app, there are some more space of optimization. Our current implementation is based on the \textit{VirtualApp} container. VirtualApp works at the user space, and requires a lot of computation resources to virtualize various system services. As a result, the overhead of the VirtualApp may potentially compromise the performance improvement gained by \nickName{}. Theoretically, \nickName{} could be implemented at the kernel space of Android OS  to reduce the overhead. Indeed, modifying the  kernel may not be as deployable and practical as the current \nickName{} client implemented upon VirtualApp, unless the modification can be adopted and deployed into the Android OS.


\noindent \textbf{Consistency with app-version  update}. \nickName{} decomposes an app into multiple bundles, and may break the existing app-updating mechanism where users download and re-install a new APK file to update an app. To address this issue, \nickName{} can download the latest bundles that contain the changed code and resources when a new version of the app is released. Actually, such an approach could bring benefits to the existing app-updating mechanism by enabling to update apps incrementally at runtime.


\noindent \textbf{Security concerns}. \nickName{} loads feature bundles in the way of dynamic code loading based on the ~\textit{DexClassLoader} provided by the Android system. However Poeplau et al.~\cite{NDSS14Poeplau} identified severe vulnerabilities (e.g., remote code injection) related to incorrect usage of dynamic code loading. To address this issue, we can check the integrity of each bundle before it is dynamically loaded~\cite{ACSAC15Falsina}.

\noindent \textbf{Potential additional cost}. We have demonstrated that  \nickName{} can save both local storage and memory usage compared to directly launching the original app. Ideally, we can suppose that the energy drain can be also reduced. Indeed, other potential additional cost should also be discussed. Requesting the feature bundles requires the data transfer, and the cost can vary according to the network connections and bandwidth. Hence, we cannot simply guarantee that \nickName{} can always gain benefits. However, given that the feature bundles are usually non-frequently visited pages, \nickName{} still takes values in most cases.
