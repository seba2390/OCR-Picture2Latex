\section{Introduction}
\label{sec:introduction}


In the past decade, we have witnessed the burst of mobile applications (a.k.a., apps). Apps have played an indispensable role in our daily life and work, and even changed the way that we interact with the external world. People become to download the \textit{installation packages} from app stores such as Apple AppStore and Google Play, and perform various tasks such as web browsing, social networking, watching online videos, playing games, and so on.

Millions of software developers have gained great success on the appstore-centric ecosystem, e.g., rewarded by thousands of downloads, high ranking, and even revenues. However, it is also true that a large number of developers are not lucky enough and their apps never win the expected success. One recent study over millions of Android users indicates that the popularity of apps typically complies with the power law, i.e., only a small portion of apps account for substantial downloads and user-interaction time~\cite{Liu:TSE17}\cite{WWW15Li}. It is also reported that more than 90\% of apps are launched for only once after they are downloaded~\cite{WWW15Li}. Although there could be various reasons why users tend to dislike and uninstall some apps, it should be arguable whether \textit{these ``abandoned'' apps are ``completely'' useless to users}. It is possible that some of the features of these apps are still useful to some ``\textit{opportunistic and situational}'' requirement. For example, a tourist may need to search for local news on ``\texttt{Lyon Daily}'', and compare the expense of renting the same car model via various e-car-rental platforms. Apparently, she needs only some relevant features of these apps but the current app-distribution model requires her to download and install apps. Therefore, she is likely to give up downloading the apps and the developers' opportunity of winning this user is compromised.

By contrast, it is also true that we may not access every single feature of even those frequently used apps. One fact is that current apps become very complex and account for large-volume installation packages. For example, the most popular app in China, \texttt{WeChat}, now contains more than 600 ``activities''\footnote{The basic program unit on  Android OS. Suppose that an app can be analogous to a website, then an activity refers to a webpage of this site. We exchangeably use ``page'' and ``activity'' in the rest of this paper.} along with thousands of features, and stands for more than 40 MB package size. In fact, such a problem is known as the ``\textit{software bloat problem}''~\cite{software_bloat} of Android, and many users complain that the increasing package size not only requires local storage, but also leads to the sluggish experience due to more local computing-resource consumption and background collusion~\cite{WWW17Xu}.

An ideal way to mitigate the software bloat shall be that every single page of an app can be installed in a ``service-oriented'' fashion at the page level rather than downloading the whole package as Web apps~\cite{ICWS15Liu}. Such a desire is reasonable due to many successful Software-as-a-Service applications such as Salesforce and Google Docs. In practice, we notice that some solutions are stepping toward this way. For example, the recent Google Instant apps~\cite{instantapp} provide an ``installation-free'' style to Android users by clicking through a hyperlink to access a desired page of an app, rather than downloading this app. However, Google Instant apps require the system-level support: they can be deployed on only versions after Android 5.0 and require the developers' manual efforts to grant the API level higher than level-21~\cite{instantapp}. Tencent WeChat announces its Mini Program platform~\cite{miniprogram}, which works in the similar way. These Mini Programs are essentially customized Web applications implementing some features that are thus accessed via hyperlinks from the app. Unfortunately, Mini Programs suffer from poor user experience compared to native apps.

However, given that current apps are usually developed in an iterative style and incrementally updated over the code base from their previous versions, it is challenging  to re-design and re-develop millions of existing on-the-shelf apps to be service-oriented. Therefore, we need to rethink how to make apps better deployed and delivered.  We broadly require that apps should be refactored to equip the following abilities: (1) \textit{\textbf{decomposable}}, i.e., the ability that an app can be decoupled into relatively independent pages; (2) \textit{\textbf{deployable}}, i.e., the ability that end-users can easily and regularly deploy the apps on devices without additional requirements compared to the current app-installation mechanism; (3) \textit{\textbf{user-friendly}}, i.e., the ability that the apps should operate as expected and smoothly without compromising the user experience; (4) \textit{\textbf{developer-friendly}}, i.e., the ability that developers can transform apps to be service-oriented without re-development.


In this paper, we propose a novel approach, called \nickName{}, to help developers decompose and re-deploy existing Android apps for on-demand and flexible installability. 
\nickName{}  aims to meet all the four aforementioned requirements. 
\emph{Decomposable}: \nickName \emph{decomposes} an Android app into a core ``\emph{base bundle}'' and the remaining pages as installable features that can be loaded in an on-demand way. A base bundle contains frequently used ``pages'' and the dependencies promising the reachability of every single page and the correct behavior of the whole app, including code and resources. In this way, developers can then choose to release the app with only the base bundle and a list of pages that can be selected by users according to their personal preferences. 
\emph{Deployable}: \nickName{} provides an app-layer virtualized controller that resides between the app and the underlying OS, which is a normal Android app that users can download and install. 
\emph{User-friendly}: When users access pages that have not been installed, \nickName{}'s runtime execution environment dynamically requests these pages from a remote cloud where a full version is deployed, and guarantees original functionalities and user interface.
\emph{Developer-friendly}: \nickName provides a fully automated approach that leverages static analysis to identify what pages and resources form the base bundle, and includes an iterative and back-complementary recovery mechanism with recording and replay to supplement the code and resources missed by the static analysis. 
As such, apps can be automatically transformed without involving re-development from developers.

This paper makes the following main contributions:

\begin{itemize}
	\item We motivate our research based on an empirical study and reveal that the page-level app installation and access are non-trivially  required.
	\item We design a virtualized app-layer controller to properly decompose an Android app for more flexibly personalized and customized, while  preserving the correct functionality and smooth user experience.
	\item We demonstrate that our approach is practical by decomposing 1,000 real-world Android apps and evaluate the performance after decomposition along with on-demand installation. Our approach can save \textbf{44.17\%} initial download size, and reduce \textbf{10.9\%} and \textbf{20\%} time on launching base bundles and feature bundles in median case, respectively. Meanwhile, our approach can reduce \textbf{6.5\%} and \textbf{10.7\%} memory usage when launching base bundles and feature bundles in the median case, respectively.
\end{itemize}
