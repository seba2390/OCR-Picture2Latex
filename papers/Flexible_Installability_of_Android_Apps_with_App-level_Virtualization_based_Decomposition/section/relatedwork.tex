\section{Related Work}\
\label{sec:related_work}

\textbf{Software bloat}. McGrenere~\cite{McGrenereCHI2000} conducted an user study to unveil the software bloat problem of complex PC software. On average, users only used 27\% of functions, and they differed in which functions were unused. A. Akiki et al.~\cite{Akiki13aEICS}\cite{Akiki13bEICS} provided tools to simply and customize user interfaces of complex enterprise softwares so that end users were aware of only those UI feature-set. Tencent officially provides Mini Programs that are essentially Web applications to provide native app-like experiences for low frequency interactions~\cite{miniprogram}. Google provides Instant App~\cite{instantapp} with extra development efforts to reconstruct existing Android apps, but users have to download full APKs if their need other features. We offer an approach to decompose existing Android apps automatically, and users can visit those temporarily removed activities on demand.


\textbf{Decompostion}. Gui et al.~\cite{ICSE15Gui} and Liu et al.~\cite{mobisys2015Liu} statically analyzed apps to find the usage of ad libraries and rewrote bytecode for privilege de-escalation or better user experience. Carzaniga et al.~\cite{Carzaniga14ICSE} and Jelschen et al.~\cite{Jelschen12CSMR} detected and removed enery-wasting code in Android apps with the knowledge of energy-inefficiency patterns. Rubinov et al.~\cite{ICSE2016Rubinov}, Zhang et al.~\cite{OOPSLA2012Zhang}\cite{FCS12Zhang}, Wang et al.~\cite{Splash12Wang} and Liu et al.~\cite{TOIT17Liu} focused on extracting and offloading code to remote servers or trusted environments. Huang et al.~\cite{huangASE2017} proposed a static technique to remove code elements that were relevant to user specified unwanted UI elements in Android apps. Our approach considers to remove the infrequently visited activities and provides a mechanism to visit these removed activities on demand.


\textbf{Virtualization}. Chen et al.~\cite{Chen15TC} and Sun et al.~\cite{Sun2013ICCT} enforced isolation of apps and data to mitigate security risks by lightweight Android virtualization based on container technology. LBE Tech~\cite{multidroid} developed an virtualization engine called MutliDroid that empowered a user to run apps in the virtual operating system the same way the app ran in the original Android operating environment. Ki et al.~\cite{Mobisys17Ki} proposed a system called Reptor that enabled developers to modify and instrument Android platform API call behavior with app-layer API virtualization. Our approach offers an app-level virtualization space to take over the management of apps' lifecycle on non-rooted devices.
