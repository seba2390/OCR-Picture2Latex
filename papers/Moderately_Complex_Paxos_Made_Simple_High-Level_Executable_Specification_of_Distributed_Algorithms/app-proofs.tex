% !TeX root = paxos.tex

\notes{
\documentclass[11pt]{article}
\usepackage{fullpage}
%\documentclass[fleqn]{llncs}
%\usepackage{llncsdoc}
\usepackage{url}
\usepackage{pdfpages}
\usepackage{subcaption}
\usepackage{comment}
\newcommand{\mypar}[1]{\vspace{-1ex}\paragraph{\bf #1.}}
\newenvironment{myeqn}{\vspace{-1ex}\begin{equation}\small\setlength{\lineskip}{-1ex}}{\end{equation}}

%\usepackage{mathptmx}       % selects Times Roman as basic font
%\usepackage{helvet}         % selects Helvetica as sans-serif font
%\usepackage{courier}        % selects Courier as typewriter font
%\usepackage{type1cm}        % activate if the above 3 fonts are
                             % not available on your system
%
%\usepackage{graphicx}       % standard LaTeX graphics tool
                             % when including figure files

\usepackage{amsmath}
\usepackage{amssymb}
%\usepackage{mathtools}
%\usepackage{amsfonts}
%\usepackage{dsfont}
%\usepackage{framed}
\usepackage{multirow}

\begin{document}
\begin{comment}

%TODO - replace fig-vrapaxos-da with Figure 3 ref which presents the distalgo vRA MultiPaxos
%annie: done
%TODO - replace Cha+16PaxosTLAPS-FM with the FM paper ref
%annie: done
We manually translated the specification of vRA Multi-Paxos in Figure~\ref{fig-vrapaxos-da} and the two extensions in Section~\ref{sec-optimize} to specifications in TLA+. We then specified and proved what it means to be safe for the three versions---first vRA Multi-Paxos, second with state reduction, and third with state reduction and failure detection. The high-level nature of our specifications in DistAlgo makes the translation process simple.

Specifying acceptors is straightforward as they are made of two independent atomic receive-send blocks - lines \texttt{43-48}, and \texttt{49-53} in Figure~\ref{fig-vrapaxos-da}. We say a block of code is atomic if between entry and exit, the process executing the code block does not change system state. This means that in the block, the process does not send messages and does not see new arriving messages. The actions performed in the atomic block are solely based on the messages received before entering the block. When the process finishes the block, it can send messages. Usually, the receipt of a new message makes a process perform an atomic block.

The implementation of leaders (called proposers in the specification), is not a simple atomic block. It is rather a sequence of atomic blocks. The atomic blocks in Figure~\ref{fig-vrapaxos-da} are lines \texttt{26}, \texttt{27-29}, \texttt{31-32}, \texttt{33-35}, and \texttt{36-38}. After identifying these blocks, the remaining task is to ensure the sequence is supported in the specification. This is done by adding a state variable in the specification called pState. This variable changes between two sequential atomic blocks. For example, since lines \texttt{26} and \texttt{27-29} are successive, the variable changes. But it wouldn't change between \texttt{31-32}, \texttt{33-35}, and \texttt{36-38} as they are not sequential.

The only statements that DistAlgo had and TLAPS did not, are the \texttt{while} loop and \texttt{await}. They are specified in TLAPS as shown in Figure~\ref{while-await-spec}.

%annie: how about initloop, exitloop, and await be put together with the rest of the spec?
\begin{figure}[h]
  \hline
    \begin{subfigure}[h]{0.4\textwidth}
        \centering
        \begin{tabular}{@{}l@{}}
              \texttt{while f(X):}\\
                $\quad$\texttt{P}\\
              \texttt{Q}\\
              \texttt{await(g(Y))}\\
              R
        \end{tabular}
    \end{subfigure}
    \hfill
    \begin{subfigure}[h]{0.4\textwidth}
        \centering
        \begin{tabular}{@{}l@{}}
              \texttt{insideLoop $\triangleq$ f(X) $\wedge$ P}\\
              \texttt{exitLoop $\triangleq$ $\neg$ f(X) $\wedge$ Q}\\
              \texttt{ }\\
              \texttt{await $\triangleq$ g(Y) $\wedge$ R}
        \end{tabular}
    \end{subfigure}%
  \hline
  \caption{A \texttt{while} loop and \texttt{await} condition in DistAlgo and their specification in TLAPS}
  \label{while-await-spec}
\end{figure}

\appendix
\end{comment}

\section{TLA+ Specifications and Proofs}
} %end \notes from beginning

%scott: I included here part of the first paragraph above, since annie said we might omit the above section about translation.

\notes{
We manually translated the specification of vRA Multi-Paxos in Figure~\ref{fig-vrapaxos-da} and the two extensions in Section~\ref{sec-optimize} to TLA+. We specified and proved safety for three versions: vRA Multi-Paxos, vRA Multi-Paxos with state reduction, and vRA Multi-Paxos with state reduction and failure detection. The high-level nature of our DistAlgo specifications makes the translation relatively simple.

} %end notes

We developed inductive proofs of safety for all three specifications. %
%Our proofs are manually developed and automatically machine-checked using
%TLAPS~\cite{tlaps15}, a proof system for TLA+.  
%
Like the proof for Multi-Paxos from \cite{Cha+16PaxosTLAPS-FM}, they are
inductive proofs based on several invariants that together imply safety.
The proofs involve three types of invariants: (1) type invariants, stating
that as the system progresses, all data in the system have the expected
types, (2) invariants about local data of processes, for example, about the
values of \texttt{ballot}, \texttt{accepted}, and \texttt{maxb}, and (3)
invariants about global data of the system, in particular, about the
messages sent in the system.

Our proofs for vRA Multi-Paxos differ from the proof for Multi-Paxos from
\cite{Cha+16PaxosTLAPS-FM} for several reasons, including differences
between the algorithms themselves.  For example, the \texttt{accepted} set
in vRA Multi-Paxos contains all triples for which a \co{2a} message was
sent and received and may contain a triple for which a \co{2b} message was not sent, whereas in Multi-Paxos in~\cite{Cha+16PaxosTLAPS-FM}, the \texttt{accepted} set would only keep a triple if a \co{2b} message was sent containing that triple.
%annie: unclear: for a ballot?  this is refering to when the 2b was sent?
%sc: done
Also, to keep our specification in TLA+ close to the specification in
DistAlgo, we model ballots as tuples containing a natural number and a
process ID, not as natural numbers in~\cite{Cha+16PaxosTLAPS-FM}. This
modeling difference has huge impact on the proof, because comparison
operators like $>$ and $\geq$ on natural numbers are built-ins in TLAPS,
and are reasoned about automatically, but comparison operators on tuples need to be defined using predicates, and all of their properties, including fundamental properties like transitivity and non-commutativity, need to be explicitly stated in lemmas and proved.
%annie: quantify "huge".  number of lines added in proof?
In addition, we specify and prove safety of three versions of Multi-Paxos, all of which are variations not considered in~\cite{Cha+16PaxosTLAPS-FM}.

\begin{figure*}[htbp]
  \arxiv{\small}
  \centering
\begin{tabular}{@{~}l@{\ppdp{\hfill}~}|@{~}l@{~}|l@{~\,}l|l@{~\,}l@{~}l@{}}
  & Basic 
  & \multicolumn{2}{@{~}c@{~}}{\mbox{~}\hfill Multi-Paxos\hfill\mbox{~}} 
  & \multicolumn{3}{|l}{vRA Multi-Paxos \& extensions}\\
  \cline{3-7}
%  & \multirow{1}{0pt}{Paxos} & Multi- & Multi- w/ 
%  & \multicolumn{1}{|l}{w/ details} & w/ also & w/ also\\
  Metric & Paxos & Multi- & Multi- w/ 
  & \multicolumn{1}{|l}{w/details} & w/also & w/also\\
  &  & Paxos & Preempt. & \&reconfig. & st. reduct. & fail. detect.\\
  \hline
  Spec size (lines excl.\ comments) & 56 & 81 & 97 & 154 & 157 & 217\\
  Spec size incl.\ comments (lines) & 115 & 133 & 158 & 249 & 254 & 347\\
  \hline
  Proof size (lines excl.\ comments) & 306 & 1003 & 1033 & 4959 & 5005 & 7006\\
  Proof size incl.\ comments (lines) & 423 & 1106 & 1136 & 5256 & 5301 & 7384\\
  \hline
  Max level of proof tree nodes & 7 & 11 & 11 & 12 & 12 & 12\\
  Max degree of proof tree nodes & 3 & 17 & 17 & 28 & 28 & 48\\
  \hline
  \# lemmas & 4 & 11 & 12 & 24 & 23 & 23\\
  \# stability lemmas & 1 & 5 & 6 & 8 & 8 & 8\\
  \# uses of stability lemmas & 8 & 27 & 29 & 76 & 76 & 76\\
  \hline
  \# proofs by induction on set increment & 0 & 4 & 4 & 30 & 30 & 30\\
%  ~~~~set increment\hfill\mbox{} &   &   &   &    &    &   \\
  \# proofs by contradiction & 1 & 1 & 1 & 14 & 16 & 17\\\hline
  \# obligations in TLAPS & 239 & 918 & 959 & 4364 & 4517 & 5580\\\hline
  TLAPS check time (seconds) & 24 & 128 & 94 & 590 & 569 & 781\\
  \hline
\end{tabular}\vspace{-1ex}

\caption{Comparison of results for safety proofs of Basic Paxos
  from~\cite{lam12basicproof}, Multi-Paxos from~\cite{Cha+16PaxosTLAPS-FM},
  and vRA Multi-Paxos.  %Spec size and proof size are measured in lines.
  Spec and proof sizes including comments are also compared because
  they are used in~\cite{Cha+16PaxosTLAPS-FM} as opposed to sizes
  excluding comments.
  %
  Stability lemmas are called continuity lemmas in~\cite{Cha+16PaxosTLAPS-FM}.
  %Width of proof is the product of \textit{Inv} conjuncts and \textit{Next} disjuncts explicitly combined as a goal for the prover to prove; %todo: sum? sc -- product, cartesian product: Inv x Next
  % what is a proof step? sc -- reworded a bit
  % should be $Inv$ and $Next$? - mathmode doesn't compile in this caption - hence, I'm using \textit Haha!!
  % depth is the maximum level of nested steps.
  An obligation is a condition that TLAPS checks.
  %
  The time to check is on an Intel i7-4720HQ 2.6 GHz CPU with 16 GB of
  memory, running Ubuntu 16.04 LTS and TLAPS 1.5.2.}
 \label{fig-sum}
\end{figure*}

\mypar{Results and comparisons}

Figure \ref{fig-sum} presents the results about our specifications and proofs of vRA Multi-Paxos and its extensions, and the specifications and proofs of Multi-Paxos from \cite{Cha+16PaxosTLAPS-FM} and Basic Paxos from~\cite{lam12basicproof}.  First, we compare the specifications and proofs of vRA Multi-Paxos and its extensions with each other:
\begin{itemize}
  \setlength{\itemsep}{1ex}

\item The specification size grows by only 3 lines (1.9\%) from 154 when we
  add state reduction, but by 60 more lines (38\%), for the new actions
  added, when we add failure detection.
  % in lines \texttt{37.1-37.3} and \texttt{38.1-38.2} in
  % Section~\ref{sec-optimize}.

\item The proof size grows by only 46 lines (0.9\%) from 4959 when we add
  state reduction, but by 2001 more lines (40\%) when we add failure
  detection, roughly proportional to the increase in specification size.

\item The maximum level and degree of proof tree nodes remain unchanged
  when state reduction is added.  When failure detection is added, the
  maximum level of proof tree nodes remains unchanged, but the maximum
  degree of proof tree nodes increases by 20 (71\%), from 28 to 48, due to
  more complex proofs for the new actions added for failure detection.
    %annie: but degree goes from 28 to 48 on last column
    %sc- done

\item An interesting decrease of one lemma is seen after state reduction is
  added. The lemma states that the maximum of a set is one of the maximums
  of its two partitions. This lemma was needed in the case when all triples
  in \co{2a} messages are kept by the acceptors. However, owing to state
  reduction, only triples with the maximum ballots are kept, making the
  proofs simpler.

  The number of stability lemmas and their uses remain unchanged when we
  add extensions.  A {\em stability lemma} is a lemma asserting that a
  predicate continues to hold (or not hold) as the system goes from one
  state to the next in a single step.
    
\item The number of proofs by induction on set increment remains unchanged
  when we add extensions. The number of proofs by contradiction
  increases; in those cases, constructive proofs were more challenging.
    
\item The number of obligations, i.e., conditions that TLAPS proves,
  increases by 153 (3.5\%) from 4364 when state reduction is added, and by
  1063 (24\%) more when failure detection is added, contributing to the
  increase in proof size.
    
\item The proof check time decreases by 21 seconds (3.6\%) from 590 to 569
  when state reduction is added. This was expected because, with state
  reduction, for each slot, only the triple with the maximum ballot is
  kept. Upon receiving a triple with a larger ballot,
  %the old one is discarded. Thus, for each slot, at most one triple is kept, 
  only the new triple is kept, 
  %
  and the maximum of a singleton set is the item itself, %. This makes
  making the proof time decrease.
    
  The proof check time increases by 212 seconds (37\%) when failure
  detection is added.  This is expected, because there are more proof
  obligations (24\%) and the proof is larger (40\%).
\end{itemize}

Next, we compare our TLA+ specification and proof of vRA Multi-Paxos
(without state reduction or failure detection) with those of Multi-Paxos
with preemption from \cite{Cha+16PaxosTLAPS-FM}.
\begin{itemize}
  \setlength{\itemsep}{1ex}

\item The specification of vRA Multi-Paxos, excluding comments, is 154
  lines, which is 59\% more.  This increase is
  because~\cite{Cha+16PaxosTLAPS-FM} omits many algorithm details, while
  our specification models the many more details in
  Figure~\ref{fig-vrapaxos-da}. %and reconfiguration.
    
\item The proof of vRA Multi-Paxos, excluding comments, is 4959 lines,
  which is 380\% more. This increase is due to many factors, including more
  actions (for sending \co{2a} messages in two cases and for sending
  decisions), more invariants (about the looser \co{accepted} set and about
  program points for the additional actions), and representing ballots as
  tuples instead of natural numbers, as mentioned above.
    
\item The proof tree for vRA Multi-Paxos is more complex, as shown by the
  65\% increase, from 17 to 28, in the maximum degree of proof tree nodes
  and 9\% increase, from 11 to 12, in the maximum level of proof tree
  nodes.
    
\item Twice as many lemmas are needed for vRA Multi-Paxos, 24 vs.\ 12,
  because properties of operations on tuples need to be explicitly stated
  in lemmas and proved, as mentioned above.
    
  We prove 2 more stability lemmas for vRA Multi-Paxos for the additional
  actions.  The number of uses of stability lemmas increases by 47
  (162\%), from 29 to 76, because of the additional actions and the larger
  number of invariants.  
    
\item The number of proofs by induction on set increment increases by 26
  (650\%), from 4 to 30, the number of proofs by contradiction increases
  from 1 to 14 (1300\%), the number of obligations increases by 3405
  (355\%), from 959 to 4364, and the proof check time increases by 496
  seconds (527\%), from 94 seconds to 590 seconds, all due to increased
  complexity in the specification, more actions, and more invariants.
\end{itemize}

\notes{
\end{document}
} %end notes

%%% Local Variables: 
%%% mode: latex
%%% TeX-master: "paxos.tex"
%%% TeX-PDF-mode: t
%%% End: 
