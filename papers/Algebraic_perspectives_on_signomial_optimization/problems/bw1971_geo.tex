\begin{align}
    \min_{\vct{t} \in \R^8_{++}} ~~& 2.0425\, t_{1}^{0.782} + 52.25\, t_{2} + 192.85\, t_{2}^{0.9} + 5.25\, t_{2}^3 + 61.465\, t_{6}^{0.467} \label{prob:ex2} \\[-0.75em]
    & \qquad + 0.01748\,  t_{3}^{1.33} / t_{4}^{0.8} + 100.7\, t_{4}^{0.546} + 3.66 {\cdot 10}^{-10} \, t_{3}^{2.85}/ t_{4}^{1.7} \nonumber \\
    & \qquad + 0.00945\, t_{5} + 1.06 {\cdot 10}^{-10}\, t_{5}^{2.8}/t_{4}^{1.8} + 116\, t_{6} - 205\, t_{6} t_{7} - 278\, t_{2}^3 t_{7} \nonumber \\[0.25em]
    \text{s.t.}~~ & 1 - 129.4/t_{2}^3 - 105/t_{6} = 0 \nonumber \\
    & 1 - 1.03\cdot 10^5 \, t_{2}^3 t_{7}/(t_{3} t_{8}) - 1.2\cdot 10^6 /(t_{3} t_{8}) = 0 \nonumber \\
    & 1 - 4.68\, t_{2}^3/t_{1} - 61.3\, t_{2}^2/t_{1} - 160.5\, t_{2}/t_{1} = 0 \nonumber \\
    & 1 - 1.79\, t_{7} - 3.02\, t_{2}^3 t_{7}/t_{6} - 35.7\, /t_{6} - 1 = 0 \nonumber \\
    %& 1000 - 1.22\, t_{3} t_{8}/(t_{4}^{0.2} t_{5}^{0.8}) - 1.67\, t_{8} t_{3}^{0.4}/ t_{4}^{0.43} \nonumber \\[-0.0em] & \qquad - 3.6 {\cdot 10}^{-2}\, t_{3} t_{8}/t_{4} - 2\, t_{3} t_{8}/t_{5} - 4\, t_{8} = 0 \nonumber
    & 1 - 1.22\cdot 10^{-3}\, t_{3} t_{8}/(t_{4}^{0.2} t_{5}^{0.8}) - 1.67\cdot 10^{-3}\, t_{8} t_{3}^{0.4}/ t_{4}^{0.43} \nonumber \\[-0.0em] & \qquad - 3.6 {\cdot 10}^{-5}\, t_{3} t_{8}/t_{4} - 2\cdot 10^{-3}\, t_{3} t_{8}/t_{5} - 4\cdot 10^{-3}\, t_{8} = 0 \nonumber
\end{align}

\iffalse
\begin{remark}
    This model first appeared in \cite{BW1969-sig-ChemE}, which was written very much for practicing engineers.
    Later, the problem was considered as an example for a proposed algorithm for equality-constrained signomial programming \cite{BW1971}.
    We used the formulation \cite{BW1971} since it was easier to read than that in \cite{BW1969-sig-ChemE}.
    However, there is a clear typo in \cite[Equation 4]{BW1971}: the term ``$d_V^2/W$'' appears with two different coefficients.
    We consulted the original paper \cite{BW1969-sig-ChemE} and believe the correct version of \cite[Equation 4]{BW1971} is $4.68 d_V^3/W + 6.13 d_V^2/ W + 160.5 d_V/ W = 1$.
    Our comparisons to alternative SAGE-based techniques and global solvers are equally informative regardless of whether our correction was valid.
\end{remark}
\fi