\section{Introduction}  \label{sec_introduction}
\noindent
The title of this article was by inspired by a 2-sentence abstract from a   92-page 
publication~\cite{OPUS-matching-1998-Fueredi-bigraphs}: 
%{\it ``{\sf Abstract}. 
{\it ``Almost all combinatorial questions can be reformulated as either a matching or a covering problem of a hypergraph. In this paper we survey some of the important results.''.}

Rather than  theorems and proofs, 
this article is about {\it asymptotic performance  experiments}
on matching and covering problems with data structures that represent the hypergraph as a bipartite graph: a matrix with $m$ rows and $n$ columns. For an illustration of matching and covering problems
addressed in this article, see the example of the 11-row, 9-column bigraph in Section~\ref{sec_matching}, Figure~\ref{fg_bgmc_matching_cover}.

Companion articles~\cite{OPUS2-2022-coupon-arxiv-Brglez, OPUS2-2022-mclass-arxiv-Brglez} 
provide the background and the motivation
for a series of experiments we report in four sections of this article:
\begin{description}

\item[\sf{Beyond CSC316 and Java}]~\\\
Data structures impact the 
asymptotic runtime performance when creating lists
such as {\it the top 10 most frequently watched movies}.
The key finding is that the {\tt data.table} structure in \R~\cite{OPUS-R-manual} 
significantly outperforms all of the best-known and widely
available ADTs in Java. The largest movie list has $2^{20}$ titles.

\item[\sf{Maximum Matching in Bipartite Graphs}]~\\\
We compare runtime performance of two 
public-domain implementations of the Ford-Fulkerson algorithm: Java and \R{}.
The hardest instance has 88452 rows and 729 columns.
Again, \R{} significantly outperforms the implementation in Java.

\item[\sf{Greedy Heuristic Distributions for Set Cover}]~\\\
The importance of greedy heuristics is increasing as the
instance sizes increase for problems such as {\em the minimum set cover}.
We demonstrate that a stochastic version of a greedy algorithm 
in \R{} can significantly outperform a state-of-the-art stochastic
solver in C++ on instances with 
$num\_rows \ge 300$ and $num\_columns \ge 3000$. 

\item[\sf{Future Work}]~\\\
The work in progress includes extensions of new heuristics,
outlined in the  companion article~\cite{OPUS2-2022-mclass-arxiv-Brglez},
to a number of {\it hard} combinatorial optimization problems.

\end{description}
%Each section includes relevant citations.
%\vspace*{9ex}
%\newpage % does not work as I want HERE
