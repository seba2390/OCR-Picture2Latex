\vspace*{2ex}
\section{Summary and Future Work}  \label{sec_summary}
\noindent
The roots for this article have been  provided by the rather unexpected 
empirical result in December 2020, as a follow-up on the just completed 
CSC316 Java project in a junior-level course in data
structures and algorithms~\cite{OPUS-csc316-fall-2020}.
This result is summarized with two asymptotic plots, 
Figure~\ref{fg_bgmc_movieLib_runtime}a and Figure~\ref{fg_bgmc_movieLib_runtime}b.
Elements of surprise include not only the significant runtime improvements
with R-code versus the Java-code but also that these results were produced in a time frame of two weeks by the first author who was completely new to R.
However, as it frequently happens, the time required to {\it explain} a new result can be much longer than the time required 
%for the initial result itself.
to produce the result itself.

\begin{description}
\item[\sf{Note}]~\
\\
For all datasets, programs and asymptotic experiments with data structures
in this article, see~\cite{OPUS-github-rBedPlus-bgmc}.

\item[\sf{Summary}]~\
%Nunc eleifend consequat lorem. Sed lacinia nulla vitae enim. Pellentesque tin- cidunt purus vel magna. Integer non enim. Praesent euismod nunc eu purus. 
%provide the background and the motivation
%for a series of experiments we report in four sections of this article:
\begin{itemize}
\item
Our model of {\tt movieLib} in 
Figure~\ref{fg_bgmc_movieLib_data}b
is a simplified case of an {\it affiliation} bipartite graph.
For example, the first sentence
from~\cite{OPUS-affiliation-2021-Springer-Stankova-bg_classification} 
begins with
{\small\it `Many real-world large datasets correspond to bipartite graph 
data settings—think for example of users rating movies or people visiting locations'}.
\\
Data structures in R may well provide an
advantage over Java for the class of affiliation bipartite graphs.

\item
The maximum bipartite matching experiments in
Figure~\ref{fg_bgmc_matching_experiment}
consistently demonstrate the runtime advantages of R
data structures in comparison with Java.

\item
The introduction of two stochastic algorithms demonstrates
advantages of greedy heuristic for the {\it set cover problem}.
The merits of rapid prototyping these algorithms in R is apparent
by the simplicity and the readability of the code 
in Figure~\ref{fg_bgmc_unate_greedy_chvatal_stoc}.

\item
The results on lines 6--8 in Table~\ref{tb_bgmc_data_BRKGA} 
provide the currently {\it best-known-values} (BKVs) for the
three largest OR-instances listed on lines 6--8. 
By significantly outperforming
a state-of-the-art algorithm designed to search for optimum solutions -- 
not only in runtime but more importantly, in delivering 
significantly better solutions -- these greedy solutions
provide a strong basis and motivation for future work.


\end{itemize}

\item[\sf{Future Work}]~\
%Praesent euismod nunc eu purus. Praesent euismod nunc eu purus. 
\begin{itemize}
\item
The current work focuses on completing two 
companion 
articles,~\cite{OPUS2-2022-coupon-arxiv-Brglez} 
and~\cite{OPUS2-2022-mclass-arxiv-Brglez}.
Both provide support and components for 
for the work in this article as well as for the
articles to follow.
\item
In the immediate future, methods that produced the results 
on lines 6--8 in Table~\ref{tb_bgmc_data_BRKGA} of this paper
will provide the basis for
new methods in stochastic combinatorial optimization. 

\end{itemize}

\end{description}

%\par\vspace*{2.0ex}\noindent
%{\sf Acknowledgements}
%\\[2.0ex]
\section*{Acknowledgements}
\vspace*{-0.5ex}\noindent
A number of individuals and teams have
contributed to the evolution of this article.
We gratefully acknowledge them all.

\begin{itemize}
\item
Dr. Jason King gave permission to use and post course-related {\tt movieLib} and {\tt javaLib} 
after the completion his CSC316 Data Structures Course in Fall 2020.
\item
Dr. Barbara Adams advised Eason Li to enroll for a 3-hour credit course 
CSC499 in the Fall 2021 as a part of this project.
\item
The R-project team created and supports the \R-platform and environment~\cite{OPUS-R-manual}.
\item
Numerous volunteers continue to 
post valuable snippets of \R-code and advice on the Web, in particular
r-bloggers.com, stackoverflow.com, and geeksforgeeks.org.
\item
The team of the BRKGA algorithm project posted results of their research and experiments
\cite{OPUS-setc-2014-BRKGA-Resende-code,OPUS-setc-2014-BRKGA-Resende}.
\item
The team of the  ABC algorithm project for posted results of their research and experiments
\cite{OPUS-setc-2014-SWJ-Broderick-Bee_Colony}.
\end{itemize}
