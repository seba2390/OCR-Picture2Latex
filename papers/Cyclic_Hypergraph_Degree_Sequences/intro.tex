\section{Introduction}

For a given list of positive integers $D=(d_1,\dots,d_n)\in \mathbb{Z}_+^n$, the {\em graph realization} problem asks if there exists a simple graph\footnote{A loopless graph without repeated edges.} $G_D$ on $n$ vertices whose vertex degrees are given by the list $D$.
If such a graph exists then the degree sequence $D$ is said to be {\em realizable}. 
A hypergraph $H$ is said to be a simple $k$-hypergraph if every edge has $k$ vertices. A $k$-hypergraph is called as a simple hypergraph if none of its edges are repeated and no edge contains a repeated vertex.
The case of a simple graph can be seen to correspond to the case of  $2$-hypergraphs. Generally, given a degree sequence and a positive integer $k$, one wants to find a $k$-hypergraph  {\em realizing} it.
When $k=2$, Erd\"os-Gallai Theorem \cite{gallai1960graphs} gives necessary and sufficient conditions that must be satisfied by $D$ to be {\em realizable}. 
Other characterizations for the {\em realizability} of graphs have been given by Havel \cite{havel1955remark} and Hakimi \cite{hakami1962realizability}. Sierksma and Hoogeveen  \cite{sierksma1991seven} collected seven different characterizations for graph realizability and proved their equivalence.

An input degree sequence is assumed to be given in binary encoding, and an efficient characterization requires conditions which can be checked in time which is a polynomial in input size.
For $k\geq3$, the problem becomes difficult in the sense that there is no known efficient characterization of {\em realizable} degree sequences. A characterization was given by Dewdeny \cite{dewdney1975degree} for all $k\geq 3$, which does not yield an efficient algorithm. Recently, some sharp sufficient conditions for {\em realizability} of a degree sequence based on a sequence's length and degree sum were given \cite{behrens2013new}. For a fixed value of $k\geq 3$, the problem is easily seen to be in {\sf NP}, but neither a polynomial time algorithm nor a hardness proof is known for this class of problems.
However, Colbourn, Kocay and Stinson \cite{colbourn1986some} proved that several other problems related to $3$-graphic sequences are \NPC.
Achuthan et al \cite{achuthan19933}, Billington \cite{billington1988conditions} and Choudum \cite{choudum1991graphic} gave several necessary conditions for $3$-hypergraphs, however Achuthan et al \cite{achuthan19933} also showed that none of these conditions are sufficient.
There are many surveys available for this problem \cite{hakimi1978graphs,rao1981survey,tyshkevich1987graphs1,tyshkevich1988graphs2,tyshkevich1988graphs3},
for a recent survey on related problems see \cite{ferrara2013some}.

If we give up the restriction on sizes of edges to be $k$, we get the \hds problem which was first stated in~\cite{berge1984hypergraphs} for a restricted class of hypergraphs. In particular, the \hds problem, given a list of degrees $D$, asks if there exists a simple hypergraph which is a {\em realization} of $D$. This problem appears to be harder than the class of $k$-hypergraph problems in the sense that it not even known to be in {\sf NP}.
It can be easily seen to be in \cc{PSPACE}.
Several restricted versions of this problem have been studied in the past.
Bhave, Bam, Deshpande~\cite{bhave200813}
gave an Erd\"os-Gallai-type characterization of degree sequences of loopless linear hypergraphs, where a linear hypergraph is one in which any two edges have at most one common vertex.
In another direction,
characterization for a partial Steiner triple system (PSTS), which is a linear 3-hypergraph, were given by Keranen et al~\cite{keranen2009degree}. The results in \cite{bhave200813} and~\cite{keranen2009degree} were recently generalized by Khan~\cite{} using partial $(n,k,\lambda)$-systems.

This paper provides a sufficient condition for a degree sequence to {\em realizable} by a hypergraph. We do not place any restriction on hypergraph being a linear hypergraph or a have bounded edge intersections like the cases studied earlier.

\begin{table}
\caption{$n$-Bit-Table for $n=3$}
\centering
\begin{tabular}{|c||c|c|c|}
\hline
$\times$ & $c_{3,3}$ & $c_{2,3}$ & $c_{1,3}$\\
\hline
\noalign{\smallskip}
\hline
0 & 0 & 0 & 0\\
1& 0 & 0 & 1\\
2& 0 & 1 & 0\\
3& 0  & 1 & 1\\
4& 1 & 0 & 0\\
5& 1 & 0 & 1\\
6& 1 & 1 & 0\\
7& 1 & 1 & 1\\
\hline
\end{tabular}
\label{table:bitlist3}
\end{table}


\paragraph*{Our results and Contribution}

We define a notion of cyclic permutations of a binary table (for example see Table~\ref{table:bitlist3})
and use it to find an efficiently computable sufficient condition for a given degree sequence to be {\em realizable} by a simple hypergraph.

\begin{enumerate}
\item
Firstly, we prove that for any set of cyclic permutations acting on the columns of a binary table, the resulting table has all of its $2^n$ rows distinct. 
\item
 Next, we define a notion of {\em cyclic hyper degrees} viz. the degree sequences which are the sum of contiguous rows in a  binary table in which each column has been permuted by different cyclic permutations. As cyclic permutations of binary tables have every row distinct, a {\em cyclic hyper degree} sequence is {\em realizable} for some simple hypergraph. Then we give an efficient algorithm which checks if a given degree sequence is a {\em cyclic hyper degree}. An equivalent definition of {\em cyclic hyper degrees} in Theorem \ref{thm:chd} shows their structure, the {\em cyclic hyper degrees} are contained in the union of some $n$-dimensional rectangles.
\item
Finally, we provide a lower bound of $2^{\frac{(n-1)(n-2)}{2}}$ for the number of {\em cyclic hyper degree} sequences, by proving the existence of a large $n$-dimensional rectangle such that every integral tuple contained in it is a realizable degree sequence. This also gives us a lower bound on the number of {\em realizable} sequences. For comparision, there is an upper bound of $\mathcal{O}(2^{n\cdot(n-1)})$ for the number of {\em realizable} hypergraph degree sequences. 
\end{enumerate}