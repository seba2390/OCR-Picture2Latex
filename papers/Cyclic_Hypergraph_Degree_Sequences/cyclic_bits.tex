\section{Cyclic permutations and Binary Tables}
In this section, we will be working with binary tables of size $2^n \times n$ and study the action of cyclic permutations on the columns of the table. For a given number $n \in \mathbb{Z}_+$, list out the binary expansion of numbers in increasing order from $\{0,\dots,2^n-1\}$ as rows in a table. We pad the binary expansion with sufficient numbers of zeros on the left to make the length of each row exactly equal to $n$.
For example, when $n=3$, the table is as given in Table~\ref{table:bitlist3}.

To state it formally we need the following definitions. Given a number $m$ we denote the $i^{th}$  bit in its binary representation by $\rm{bin}(m,i)$. If the most significant bit in the binary expansion occurs at the $s^{th}$-position in the binary expansion of $m$, then for all values of $i>s$ the value of $\rm{bin}(m,i)$ is zero. For example, $\rm{bin}(4,2)=1$ and $\rm{bin}(4,i)=0$ for every $i\geq 3$.

For a given $n$, we construct $n$ lists $c_{1,n}, \dots, c_{n,n}$, with each list $c_{i,n}$ having length equal to $2^n$. 
For $n=2$, we have $c_{1,2} = (0,1,0,1)$ and $c_{2,2} = (0,0,1,1)$.
For $n=3$, the lists $c_{1,3},c_{2,3},c_{3,3}$ correspond to the columns of the Table~\ref{table:bitlist3}. %For each $j\in[n]$, the $i^{th}$ value in the list of $c_{j,n}$ is $\rm{bin}(i,j)$ for $i \in \{0, \ldots, 2^n-1\}$.
Formally, the lists are defined as follows.

\begin{definition}[$n$-Bit-Lists]
 For a given $n$, we define $n$-Bit-List to consist of $n$ lists $c_{1,n}, \dots, c_{n,n}$. For $j\in [2^n]$, the value of $c_{i,n}(j)$ is equal to $\rm{bin}(j-1,i)$.
\end{definition}

\begin{definition}[$n$-Bit-Table]
 For a positive integer $n$, the $n$-Bit-Table $T_n$ is defined to be a size $2^n\times n$ table with $T_n=[c_{n,n},c_{n-1,n},\dots,c_{1,n}]$. 
\end{definition}

The lists $c_{1,n}, \dots, c_{n,n}$ have a nice recursive structure and can be generated in an alternative way by concatenation.
Given a positive integer $n$, the base case of $n=1$ is one list $c_{1,1}=(0,1)$.
For $n\geq 2$, the tuple $c_{n,n}=0_{\times 2^{n-1}}\cdot 1_{\times 2^{n-1}}$ and for $j\in [n-1]$, the list $c_{j,n}$ is equal to the concatenated list $c_{j,n-1}\cdot c_{j,n-1}$. Thus, we get the following observation about the lists.

\begin{observation}
 \label{obs:zero-one-power}
 For $n\in\mathbb{Z}_+$ and $i\in [n]$, we have
  $c_{i,n}= (0_{\times 2^{i-1}} \cdot 1_{\times 2^{i-1}})_{\times 2^{n-i}}$.
\end{observation}

Let $\mathcal{C}_n$ be the set of all cyclic permutations on an $2^n$ length list.
Let $\Pi \subseteq \mathcal{C}_n$ be a multi-set consisting of $n$ arbitrary cyclic permutations $\pi_1, \ldots, \pi_n$.
Let $\Pi(T_n)=[\pi_n(c_{n,n}),\dots,\pi_1(c_{1,n})]$ be the new table obtained from $T_n$. 
For clarity we will change the notation slightly, let $\Pi(T_n,i)$ denote row $i$ of $\Pi(T_n)$, it is a length $n$ binary tuple
$(c_{n,n}(\pi_n(i)),  c_{n-1,n}(\pi_{n-1}(i)), \ldots, c_{1,n}(\pi_1(i)))$,
for $i \in [2^n]$.

We next state a general lemma whose proof follows by induction over the order of a cyclic permutation.

\begin{lemma}
\label{lem:halfcycle}
Let $L$ be a list and $\pi$ be any cyclic permutation, then $\pi(L\cdot L)= \pi(L)\cdot \pi(L).$
\end{lemma}

Recall that for $j\in [n-1]$, the tuple $c_{j,n}$ consists of two copies of $c_{j,n-1}$ concatenated together. Combining this fact with Lemma~\ref{lem:halfcycle}, we obtain the following as a special case.

\begin{corollary}
\label{cor:halfcycle}
 Given a positive integer $n$ and a cyclic permutation $\pi \in \mathcal{C}_n$.
 For $j \in [n-1]$, the tuple $\pi(c_{j,n})$ is equal to $\pi(c_{j,n-1}) \cdot \pi(c_{j,n-1})$.
\end{corollary}

We next prove that for any set of cyclic permutations $\Pi$, any two rows in $\Pi(T_n)$ will never become equal.

\begin{theorem}\label{thm:rotatebits}
For any list of $n$ cyclic permutations $\Pi$, the rows of $\Pi(T_n)$ are pair-wise distinct.
\end{theorem}
\begin{proof}
 We prove this by induction on $n$. For the base case $n=1$, the statement is  trivially true. For the rest of the proof, assume that the list of cyclic permutations is $\Pi=(\pi_1,\ldots,\pi_n)$ and $\Pi_{\overline{n}}$ is used to denote the list $(\pi_1,\dots,\pi_{n-1})$.
 %Even though a permutation $pi_i$ is a cyclic permutation acting on a $2^{n}$ length list, it is easy to see how the 
 
% By the induction hypothesis the table $\Pi(T_{n-1})$ does not contain any repeated row. 
The table $T_n$ can be constructed recursively by taking two copies of $T_{n-1}$ and appending the rows of one below the other, after that we add $0_{\times 2^{n-1}}\cdot 1_{\times 2^{n-1}}$ as the first column. Consider the table 
\begin{align*}
T_{\overline n} &= [c_{n-1,n},\dots,c_{1,n}] \\
             &= [c_{n-1,n-1}\cdot c_{n-1,n-1},\dots, c_{1,n-1}\cdot c_{1,n-1}].
\end{align*}
 Apply the list of permutations $\Pi_{\overline{n}}$ on $T_{\overline n}$ to get
\begin{align*}
 \Pi_{\overline{n}}(T_{\overline n}) &= [\pi_{n-1}(c_{n-1,n}), \ldots, \pi_1(c_{1,n})]\\
 			&= [\pi_{n-1}(c_{n-1,n-1})\cdot \pi_{n-1}(c_{n-1,n-1}),\dots, \pi_1(c_{1,n-1})\cdot \pi_1(c_{1,n-1})],
\end{align*}
 where the second equality follows from Corollary~\ref{cor:halfcycle}.
% $\Pi_{\overline{n}}$ are.
 By the induction hypothesis, the first row-wise half of $\Pi_{\overline{n}}(T_{\overline n})$, which is the same as $\Pi_{\overline{n}}(T_{n-1})$, consists of distinct rows. 
 Therefore, the table $\Pi_{\overline{n}}(T_{\overline n})$ consists of rows which are repeated exactly twice. For $i < j$, the rows $\Pi_{\overline{n}}(T_{\overline n},i)$ and
 $\Pi_{\overline{n}}(T_{\overline n},j)$ are equal when $j=i+2^{n-1}$.
Therefore, it suffices to prove that the rows  $\Pi(T_n,i)$ and $\Pi(T_n,i+2^{n-1})$ are distinct.
 Observe that to obtain the table $\Pi(T_n)$, we need to append $\pi_n(c_{n,n})$ as the first column in $\Pi_{\overline{n}}(T_{\overline n})$. As $c_{n,n}$ is equal to $0_{\times 2^{n-1}}\cdot 1_{\times 2^{n-1}}$, we have $c_{n,n}(i)\neq c_{n,n}(i+2^{n-1})$, this implies that $c_{n,n}(\pi_n(i))\neq c_{n,n}(\pi_n(i+2^{n-1}))$.
\end{proof}

Next, we define a notion of {\em cyclic hyper degrees} as follows.

\begin{definition}[Cyclic Hyper Degree]\label{def:chd}
 Given $\Pi$, a list of $n$ cyclic permutations, a tuple $d \in \mathbb{Z}_+^n$ is said to be a {\em cyclic hyper degree} if there exist $i, N \in [2^n]$ such that $d=\sum_{k=i}^N \Pi(T_n, k).$
\end{definition}

As the rows of $\Pi(T_n)$ are distinct, their contiguous sum  is in $H_n$ by definition. This gives us the main theorem of this section.

\begin{theorem}
 If $w\in \mathbb{Z}_+^n$ is a {\em cyclic hyper degree}, then $w\in H_n$.
\end{theorem}

We note that $(4,1,1,1)$ is a {\em realizable} hypergraph degree sequence but it is not a {\em cyclic hyper degree} sequence.
In the next section, we will show how to efficiently check if a given sequence $d$ is a {\em cyclic hyper degree}. 
