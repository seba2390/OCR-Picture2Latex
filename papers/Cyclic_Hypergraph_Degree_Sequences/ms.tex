\documentclass[a4paper,UKenglish]{lipics-v2016}
\usepackage{xspace}
\usepackage{amssymb}
\usepackage{amsfonts}
\usepackage{amsmath}

\newtheorem{observation}{Observation}
\renewcommand{\mod}{{\rm ~mod~}}

\usepackage{mathtools}
\DeclarePairedDelimiter{\floor}{\lfloor}{\rfloor}
%\newtheorem{lemma}[theorem]{Lemma}
%\newcommand{\min}[1]{${\rm min}$[#1]$}
%\newcommand{\max}[1]{${\rm max}[#1]$}
%complexity classes
\newcommand{\cc}[1]	{\textsf{#1}\xspace}
\newcommand{\YES}	{\textsc{Yes}\xspace}
\newcommand{\NO}		{\textsc{No}\xspace}
\newcommand{\OO}		{\mathcal{O}}

\newcommand{\problemdef}[3]{
    \begin{center}
    \noindent
    \framebox{
        \begin{minipage}{4.6in}
            \textbf{\sc #1} \\
            \emph{Input}: #2 \\
            \emph{Question}: #3
        \end{minipage}
    }
    \end{center}
}

\newcommand{\hds}{{\sc Hypergraph Degree Sequence}\xspace}
\newcommand{\NPC}{{\sf NP}-Complete\xspace}

\title{Cyclic Hypergraph Degree Sequences}
%\titlerunning{}
%\toctitle{<Your changed title for the table of contents>}
%\subtitle{<subtitle of your contribution>}
%\authorrunning{<abbreviated author list>}

\author{S M Meesum}
\affil{The Institute of Mathematical Sciences,\\ HBNI, Chennai, India.\\
\texttt{meesum@imsc.res.in}} 

%\date{}                                           % Activate to display a given date or no date

\subjclass{G.2.1 Combinatorics, G.2.2 Hypergraphs}% mandatory: Please choose ACM 1998 classifications from http://www.acm.org/about/class/ccs98-html . E.g., cite as "F.1.1 Models of Computation". 
\keywords{Hypergraph, Realizable Degree Sequence}% mandatory: Please provide 1-5 keywords
% Author macros::end %%%%%%%%%%%%%%%%%%%%%%%%%%%%%%%%%%%%%%%%%%%%%%%%%

%Editor-only macros:: begin (do not touch as author)%%%%%%%%%%%%%%%%%%%%%%%%%%%%%%%%%%
\EventEditors{John Q. Open and Joan R. Acces}
\EventNoEds{2}
\EventLongTitle{42nd Conference on Very Important Topics (CVIT 2016)}
\EventShortTitle{CVIT 2016}
\EventAcronym{CVIT}
\EventYear{2016}
\EventDate{December 24--27, 2016}
\EventLocation{Little Whinging, United Kingdom}
\EventLogo{}
\SeriesVolume{42}
\ArticleNo{23}
% Editor-only macros::end %%%%%%%%%%%%%%%%%%%%%%%%%%%%%%%%%%%%%%%%%%%%%%%


\begin{document}
\maketitle
\begin{abstract}
The problem of efficiently characterizing degree sequences of simple hypergraphs is a fundamental long-standing open problem in Graph Theory.
Several results are known for restricted versions of this problem. This paper adds to the list of sufficient conditions for a degree sequence to be {\em hypergraphic}.
This paper proves a combinatorial lemma about cyclically permuting the columns of a binary table with length $n$ binary sequences as rows. We prove that for any set of cyclic permutations acting on its columns, the resulting table has all of its $2^n$ rows distinct. Using this property, we first define a subset {\em cyclic hyper degrees} of hypergraphic sequences and show that they admit a polynomial time recognition algorithm. Next, we prove that there are at least $2^{\frac{(n-1)(n-2)}{2}}$ {\em cyclic hyper degrees}, which also serves as a lower bound on the number of {\em hypergraphic} sequences. The {\em cyclic hyper degrees} also enjoy a structural characterization, they are the integral points contained in the union of some $n$-dimensional rectangles.
\end{abstract}

% !TEX root = ../arxiv.tex

Unsupervised domain adaptation (UDA) is a variant of semi-supervised learning \cite{blum1998combining}, where the available unlabelled data comes from a different distribution than the annotated dataset \cite{Ben-DavidBCP06}.
A case in point is to exploit synthetic data, where annotation is more accessible compared to the costly labelling of real-world images \cite{RichterVRK16,RosSMVL16}.
Along with some success in addressing UDA for semantic segmentation \cite{TsaiHSS0C18,VuJBCP19,0001S20,ZouYKW18}, the developed methods are growing increasingly sophisticated and often combine style transfer networks, adversarial training or network ensembles \cite{KimB20a,LiYV19,TsaiSSC19,Yang_2020_ECCV}.
This increase in model complexity impedes reproducibility, potentially slowing further progress.

In this work, we propose a UDA framework reaching state-of-the-art segmentation accuracy (measured by the Intersection-over-Union, IoU) without incurring substantial training efforts.
Toward this goal, we adopt a simple semi-supervised approach, \emph{self-training} \cite{ChenWB11,lee2013pseudo,ZouYKW18}, used in recent works only in conjunction with adversarial training or network ensembles \cite{ChoiKK19,KimB20a,Mei_2020_ECCV,Wang_2020_ECCV,0001S20,Zheng_2020_IJCV,ZhengY20}.
By contrast, we use self-training \emph{standalone}.
Compared to previous self-training methods \cite{ChenLCCCZAS20,Li_2020_ECCV,subhani2020learning,ZouYKW18,ZouYLKW19}, our approach also sidesteps the inconvenience of multiple training rounds, as they often require expert intervention between consecutive rounds.
We train our model using co-evolving pseudo labels end-to-end without such need.

\begin{figure}[t]%
    \centering
    \def\svgwidth{\linewidth}
    \input{figures/preview/bars.pdf_tex}
    \caption{\textbf{Results preview.} Unlike much recent work that combines multiple training paradigms, such as adversarial training and style transfer, our approach retains the modest single-round training complexity of self-training, yet improves the state of the art for adapting semantic segmentation by a significant margin.}
    \label{fig:preview}
\end{figure}

Our method leverages the ubiquitous \emph{data augmentation} techniques from fully supervised learning \cite{deeplabv3plus2018,ZhaoSQWJ17}: photometric jitter, flipping and multi-scale cropping.
We enforce \emph{consistency} of the semantic maps produced by the model across these image perturbations.
The following assumption formalises the key premise:

\myparagraph{Assumption 1.}
Let $f: \mathcal{I} \rightarrow \mathcal{M}$ represent a pixelwise mapping from images $\mathcal{I}$ to semantic output $\mathcal{M}$.
Denote $\rho_{\bm{\epsilon}}: \mathcal{I} \rightarrow \mathcal{I}$ a photometric image transform and, similarly, $\tau_{\bm{\epsilon}'}: \mathcal{I} \rightarrow \mathcal{I}$ a spatial similarity transformation, where $\bm{\epsilon},\bm{\epsilon}'\sim p(\cdot)$ are control variables following some pre-defined density (\eg, $p \equiv \mathcal{N}(0, 1)$).
Then, for any image $I \in \mathcal{I}$, $f$ is \emph{invariant} under $\rho_{\bm{\epsilon}}$ and \emph{equivariant} under $\tau_{\bm{\epsilon}'}$, \ie~$f(\rho_{\bm{\epsilon}}(I)) = f(I)$ and $f(\tau_{\bm{\epsilon}'}(I)) = \tau_{\bm{\epsilon}'}(f(I))$.

\smallskip
\noindent Next, we introduce a training framework using a \emph{momentum network} -- a slowly advancing copy of the original model.
The momentum network provides stable, yet recent targets for model updates, as opposed to the fixed supervision in model distillation \cite{Chen0G18,Zheng_2020_IJCV,ZhengY20}.
We also re-visit the problem of long-tail recognition in the context of generating pseudo labels for self-supervision.
In particular, we maintain an \emph{exponentially moving class prior} used to discount the confidence thresholds for those classes with few samples and increase their relative contribution to the training loss.
Our framework is simple to train, adds moderate computational overhead compared to a fully supervised setup, yet sets a new state of the art on established benchmarks (\cf \cref{fig:preview}).

%\documentclass[main]{subfiles}

\begin{document}

\section{Preliminaries}
\label{sec:preliminaries}
%\paragraph{Notation} 
\noindent
For $n \in \N$, we denote $[n] := \{1,\ldots,n\}$ and the vector with all ones as $\1_n \in \R^n$.
%\todo{Any vector $x \in \R^n$ is a column vector and its transpose is denoted by $x^T$. The entries of $x \in \R^n$ will be $x_1, \ldots, x_n$. Inequalities like $x \geq 0$ abbreviate the statement $\forall i \in [n] \, : \, x_i \geq 0$. For $i \in [n]$, we will use $e_i \in \R^n$ to denote the $i$-th standard basis vector (with a $1$ in its $i$-th entry and $0$'s anywhere else).} 
 
%\\
%\todo{When working with sets $\{S^i\}_{i=1}^N$, we denote $\displaystyle \bigtimes_{i=1}^N S^i := S^1 \times \ldots \times S^N$.} \\


\subsection{Multiplayer Games} 
A multiplayer game $G$ specifies (a) the number of players $N \in \N, N \geq 2,$ (b) a set of pure strategies $S^i = [m_i]$ for each player~$i$ where $m_i \in \N, m_i \geq 2,$ and (c) the utility payoffs for each player~$i$ given as a function $u_i: S = S^1 \times \ldots \times S^N \longrightarrow \R$. Throughout this paper, all multiplayer games considered shall have the same number of players $N$ and the same strategy sets $S^1, \ldots, S^N$. Hence, any game $G$ will be determined by its utility functions $\{u_i\}_{i \in [N]}$. The players choose their strategies simultaneously and they cannot communicate with each other. A utility function $u_i$ can be summarized by its pure strategy outcomes for player~$i$, captured as an $N$-dimensional array $\big\{ u_i(\ks) \big\}_{\ks \in S}$.

\begin{ex}
$2$-player games are better known as bimatrix games because their $2$-dimensional payoff arrays in become matrices $A,B \in \R^{m \times n}$.
\end{ex}

As usual, we allow the players to randomize over their pure strategies. Then, player~$i$'s strategy space extends to the set of probability distributions over $S^i$. We identify this set with $\Delta(S^i) := \, \Big\{ s^i = (s_k^i)_k \, \in \R^{m_i} \, \Big| \, s_k^i \geq 0 \, \, \forall k \in [m_i] \, \, \text{and} \, \sum_{k \in [m_i]} s_k^i = 1 \Big\}$ and refer to the probability distributions as mixed strategies. A tuple $\strats = (s^1, \ldots, s^N) \in \Delta(S^1) \times \ldots \times \Delta(S^N) =: \Delta(S)$ of mixed strategies is called a strategy profile in $G$\footnote{Note that in our notation, $\Delta(S)$ is not a simplex of higher dimensions itself but only the product of $N$ simplices.}. The utility payoff of player~$i$ for the strategy profile $\strats$ is defined as the player's utility payoff in expectation 
\[ u_i(\strats) := \sum_{\ks \in S} s_{k_1}^1 \cdot \ldots \cdot s_{k_N}^N \cdot u_i(\ks) \, .\]
The goal of each player is to maximize her utility.


We will abbreviate with $S^{-i}$ the set that consists of all possible pure strategy choices $\ks_{-i} = (k_1, \ldots, k_{i-1},k_{i+1}, \ldots, k_N)$ of the opponent players (resp. $\Delta(S^{-i})$ for the set of mixed strategy choices $\strats^{-i} = (s^1, \ldots, s^{i-1},s^{i+1}, \ldots, s^N)$). We will also often use $u_i(k_i,\ks_{-i})$ instead of $u_i(\ks)$ to stress how player~$i$ can only influence her own strategy when it comes to her payoff (resp. $u_i(s^i,\strats^{-i})$ instead of $u_i(\strats)$).
\begin{defn}
The best response set of player~$i$ to the opponents' strategy choices $\strats^{-i}$ is defined as $\BR_{u_i}(\strats^{-i}) :=  \argmax_{t^i \in \Delta(S^i)} \big\{ \, u_i(t^i,\strats^{-i}) \, \big\}$. 
\end{defn} 
Best response strategies capture the idea of optimal play against the other player's strategy choices. The most popular equilibrium concept in non-cooperative games is based on best responses.
\begin{defn}
A strategy profile $\strats \in \Delta(S)$ to a game $G = \{u_i\}_{i \in [N]}$ is called a \NE{} if for all player~$i \in [N]$ we have $s^i \in \BR_{u_i}(\strats^{-i})$.
\end{defn}
\noindent
By \cite{Nash48}, any multiplayer game $G$ admits at least one \NE{}.

\subsection{Positive Affine Transformations} 

The following lemma is a well-known result for $2$-player games\footnote{ See \cite[Lemma 2.1]{heyman}, \cite[Theorem 5.35]{maschler_solan_zamir_2013}, \cite[Chapter 3]{harsanyi1988general} or \cite[Proposition 3.1]{DynGT}.}:
\begin{lemma}
\label{PAT preserves lemma}
Let $(A,B)$ be a $m \times n$ bimatrix game and take arbitrary $\alpha, \beta >0$ and $c \in \R^n, d \in \R^m$. Define $A' = \alpha A + \1_m c^T$ and $B' = \beta B + d \1_n^T$.

Then the game $(A', B')$ has the same best response sets as the game $(A,B)$. Consequently, both games have the same \NE{} set.
\end{lemma}
The intuition behind this lemma is as follows:
player~$1$ wants to maximize her utility given the strategy that player~$2$ chose. A positive rescaling of $u_1$ will change the utility payoffs but will not change the sets of best response strategies. The same holds true if we add utility payoffs to $u_1$ that are only dependent on the strategy choice $s^2$ of her opponent. In the notation of bimatrix games, this intuition yields that the transformation $A \mapsto \alpha A + \1_m c^T$ does not affect the best response sets of player~$1$. The analogous result holds for player~$2$ and the transformation $B \mapsto \beta B + d \1_n^T$.

Let us generalize PATs to multiplayer games.
\begin{defn}
\label{multiplayer PAT defn}
A positive affine transformation (PAT) specifies for each player~$i$ a scaling parameter $\alpha^i \in \R, \alpha^i >0,$ and translation constants $C^i := ( c_{\ks_{-i}})_{\ks_{-i} \in S^{-i}}$ for each choice of pure strategies from the opponents. 
The PAT $H_{\textnormal{PAT}} = \big\{ \alpha^i, C^i \big\}_{i \in [N]}$ applied to an input game $G = \{u_i\}_{i \in [N]}$ returns the transformed game $H_{\textnormal{PAT}}(G) = \{u_i'\}_{i \in [N]}$ in which (only) the utility functions changed to
\begin{align}
\label{PAT transformed utilities}
\begin{aligned}
u_i' : S &\longrightarrow \R \\
\ks &\longmapsto \alpha_i \cdot u_i(\ks) + c_{\ks_{-i}}^i \, .
\end{aligned}
\end{align}
\end{defn}
We could not find multiplayer PATs defined in the literature, so we came up with the natural generalization above. As shown in Section \ref{sec:bimatrix games}, they indeed generalize the 2-player PATs from Lemma~\ref{PAT preserves lemma} to multiplayer settings. Moreover, multiplayer PATs also preserve the best response sets and \NE{} set.
\begin{lemma}
\label{multiplayer PAT preserves}
Take a PAT $H_{\textnormal{PAT}} = \big\{ \alpha^i, C^i \big\}_{i \in [N]}$ and any game $G = \{u_i\}_{i \in [N]}$. Then, the transformed game $H_{\textnormal{PAT}}(G) = \{u_i'\}_{i \in [N]}$ has the same best response sets as the input game $G$. Consequently, $H_{\textnormal{PAT}}(G)$ also has the same \NE{} set as $G$.
\end{lemma}
\begin{proof}
See \ref{sec:helpinglemmas}.
\end{proof}
PATs have found much success as a tool for simplifying an input game precisely because of this property. We want to investigate which other game transformations also preserve the best response sets or the \NE{} set. If we found more of these transformations, we could use them to, e.g., further increase the class of efficiently solvable games.

\subsection{Game Transformations}

There are various ways in which we could define the concept of a game transformation. Section~\ref{literature review} gives an overview of some definitions from the literature that are useful for different purposes. A key component of PATs are that they operate player-wise and strategy-wise, that is, they do not change the player set nor the players' strategy sets. This allows for a direct comparison of the strategic structure between a game and its PAT-transform. We argue that this is a natural desideratum for a definition of more general game transformation.

\begin{defn}
\label{def game trafo}
A game transformation $H = \{H^i\}_{i \in [N]}$ specifies for each player~$i$ a collection of functions $H^i := \Big\{ h_{\ks}^i : \R \longrightarrow \R \Big\}_{\ks \in S}$, indexed by the different pure strategy profiles $\ks$. \\
The transformation $H$ can then be applied to any $N$-player game $G = \{u_i\}_{i \in [N]}$ to construct the transformed game $H(G) = \{H^i(u_i)\}_{i \in [N]}$ where 
\begin{align}
\label{transformed pure utilities evaluation}
    H^i(u_i) : S \to \R, \quad \ks \mapsto h_{\ks}^i \big( u_i(\ks) \big) \, .
\end{align}
\end{defn}
Observe that the utility payoff of player~$i$ in the transformed game $H(G)$ from the pure strategy outcome $\ks$ is only a function of the utility payoff from \textit{that same} player in \textit{that same} pure strategy outcome of the input game~$G$.

We extend the utility functions $H^i(u_i)$ to mixed strategy profiles $\strats \in \Delta(S)$ as usual through $H^i(u_i)(\strats) := \sum_{\ks \in S} s_{k_1}^1 \cdot \ldots \cdot s_{k_N}^N \cdot h_{\ks}^i \big( u_i(\ks) \big)$. To simplify future notation, we will often use $h_{k_i,\ks_{-i}}^i$ to refer to $h_{\ks}^i$.

\begin{rem}
A multiplayer positive affine transformation $H_{\textnormal{PAT}} = \big\{ \alpha^i, C^i \big\}_{i \in [N]}$ makes a game transformation $H = \{H^i\}_{i \in [N]}$ in the above sense by setting $h_{\ks}^i : \, \, \R \to \R$, $z \mapsto \alpha^i  \cdot z + c_{\ks_{-i}}^i$.
\end{rem}
\begin{defn}
\label{defn NE preserving}
Let $H = \{H^i\}_{i \in [N]}$ be a game transformation. Then we say that $H$ universally preserves \NE{} sets if for all input games $G = \{u_i\}_{i \in [N]}$, the transformed game $H(G) = \{H^i(u_i)\}_{i \in [N]}$ has the same \NE{} set as the input game $G$.
\end{defn}

\begin{defn}
\label{defn BR preserving}
Let map $H^i$ come from a game transformation $H$. Then we say that $H^i$ universally preserves best responses if for all utility functions $u_i : S \longrightarrow \R$ and for all opponents' strategy choices $\strats^{-i} \in \Delta(S^{-i})$:
\begin{equation*}
\BR_{H^i(u_i)}(\strats^{-i}) = \argmax_{t^i \in \Delta(S^i)} \big\{ H^i(u_i)(t^i,\strats^{-i}) \big\} = \argmax_{t^i \in \Delta(S^i)} \big\{ u_i(t^i,\strats^{-i}) \big\} = \BR_{u_i}(\strats^{-i}) \, .
\end{equation*}
\end{defn}

\begin{defn}
\label{defn opponent dependence}
Let map $H^i$ come from a game transformation $H$. Then we say that $H^i$ only depends on the strategy choice of the opponents if for all pure strategy choices $\ks_{-i} \in S^{-i}$ of the opponents, we have the map identities
    \[h_{1, \ks_{-i}}^i = \ldots = h_{m_i, \ks_{-i}}^i\,: \R \to \R \, .\]
\end{defn}

\end{document}
\section{Cyclic permutations and Binary Tables}
In this section, we will be working with binary tables of size $2^n \times n$ and study the action of cyclic permutations on the columns of the table. For a given number $n \in \mathbb{Z}_+$, list out the binary expansion of numbers in increasing order from $\{0,\dots,2^n-1\}$ as rows in a table. We pad the binary expansion with sufficient numbers of zeros on the left to make the length of each row exactly equal to $n$.
For example, when $n=3$, the table is as given in Table~\ref{table:bitlist3}.

To state it formally we need the following definitions. Given a number $m$ we denote the $i^{th}$  bit in its binary representation by $\rm{bin}(m,i)$. If the most significant bit in the binary expansion occurs at the $s^{th}$-position in the binary expansion of $m$, then for all values of $i>s$ the value of $\rm{bin}(m,i)$ is zero. For example, $\rm{bin}(4,2)=1$ and $\rm{bin}(4,i)=0$ for every $i\geq 3$.

For a given $n$, we construct $n$ lists $c_{1,n}, \dots, c_{n,n}$, with each list $c_{i,n}$ having length equal to $2^n$. 
For $n=2$, we have $c_{1,2} = (0,1,0,1)$ and $c_{2,2} = (0,0,1,1)$.
For $n=3$, the lists $c_{1,3},c_{2,3},c_{3,3}$ correspond to the columns of the Table~\ref{table:bitlist3}. %For each $j\in[n]$, the $i^{th}$ value in the list of $c_{j,n}$ is $\rm{bin}(i,j)$ for $i \in \{0, \ldots, 2^n-1\}$.
Formally, the lists are defined as follows.

\begin{definition}[$n$-Bit-Lists]
 For a given $n$, we define $n$-Bit-List to consist of $n$ lists $c_{1,n}, \dots, c_{n,n}$. For $j\in [2^n]$, the value of $c_{i,n}(j)$ is equal to $\rm{bin}(j-1,i)$.
\end{definition}

\begin{definition}[$n$-Bit-Table]
 For a positive integer $n$, the $n$-Bit-Table $T_n$ is defined to be a size $2^n\times n$ table with $T_n=[c_{n,n},c_{n-1,n},\dots,c_{1,n}]$. 
\end{definition}

The lists $c_{1,n}, \dots, c_{n,n}$ have a nice recursive structure and can be generated in an alternative way by concatenation.
Given a positive integer $n$, the base case of $n=1$ is one list $c_{1,1}=(0,1)$.
For $n\geq 2$, the tuple $c_{n,n}=0_{\times 2^{n-1}}\cdot 1_{\times 2^{n-1}}$ and for $j\in [n-1]$, the list $c_{j,n}$ is equal to the concatenated list $c_{j,n-1}\cdot c_{j,n-1}$. Thus, we get the following observation about the lists.

\begin{observation}
 \label{obs:zero-one-power}
 For $n\in\mathbb{Z}_+$ and $i\in [n]$, we have
  $c_{i,n}= (0_{\times 2^{i-1}} \cdot 1_{\times 2^{i-1}})_{\times 2^{n-i}}$.
\end{observation}

Let $\mathcal{C}_n$ be the set of all cyclic permutations on an $2^n$ length list.
Let $\Pi \subseteq \mathcal{C}_n$ be a multi-set consisting of $n$ arbitrary cyclic permutations $\pi_1, \ldots, \pi_n$.
Let $\Pi(T_n)=[\pi_n(c_{n,n}),\dots,\pi_1(c_{1,n})]$ be the new table obtained from $T_n$. 
For clarity we will change the notation slightly, let $\Pi(T_n,i)$ denote row $i$ of $\Pi(T_n)$, it is a length $n$ binary tuple
$(c_{n,n}(\pi_n(i)),  c_{n-1,n}(\pi_{n-1}(i)), \ldots, c_{1,n}(\pi_1(i)))$,
for $i \in [2^n]$.

We next state a general lemma whose proof follows by induction over the order of a cyclic permutation.

\begin{lemma}
\label{lem:halfcycle}
Let $L$ be a list and $\pi$ be any cyclic permutation, then $\pi(L\cdot L)= \pi(L)\cdot \pi(L).$
\end{lemma}

Recall that for $j\in [n-1]$, the tuple $c_{j,n}$ consists of two copies of $c_{j,n-1}$ concatenated together. Combining this fact with Lemma~\ref{lem:halfcycle}, we obtain the following as a special case.

\begin{corollary}
\label{cor:halfcycle}
 Given a positive integer $n$ and a cyclic permutation $\pi \in \mathcal{C}_n$.
 For $j \in [n-1]$, the tuple $\pi(c_{j,n})$ is equal to $\pi(c_{j,n-1}) \cdot \pi(c_{j,n-1})$.
\end{corollary}

We next prove that for any set of cyclic permutations $\Pi$, any two rows in $\Pi(T_n)$ will never become equal.

\begin{theorem}\label{thm:rotatebits}
For any list of $n$ cyclic permutations $\Pi$, the rows of $\Pi(T_n)$ are pair-wise distinct.
\end{theorem}
\begin{proof}
 We prove this by induction on $n$. For the base case $n=1$, the statement is  trivially true. For the rest of the proof, assume that the list of cyclic permutations is $\Pi=(\pi_1,\ldots,\pi_n)$ and $\Pi_{\overline{n}}$ is used to denote the list $(\pi_1,\dots,\pi_{n-1})$.
 %Even though a permutation $pi_i$ is a cyclic permutation acting on a $2^{n}$ length list, it is easy to see how the 
 
% By the induction hypothesis the table $\Pi(T_{n-1})$ does not contain any repeated row. 
The table $T_n$ can be constructed recursively by taking two copies of $T_{n-1}$ and appending the rows of one below the other, after that we add $0_{\times 2^{n-1}}\cdot 1_{\times 2^{n-1}}$ as the first column. Consider the table 
\begin{align*}
T_{\overline n} &= [c_{n-1,n},\dots,c_{1,n}] \\
             &= [c_{n-1,n-1}\cdot c_{n-1,n-1},\dots, c_{1,n-1}\cdot c_{1,n-1}].
\end{align*}
 Apply the list of permutations $\Pi_{\overline{n}}$ on $T_{\overline n}$ to get
\begin{align*}
 \Pi_{\overline{n}}(T_{\overline n}) &= [\pi_{n-1}(c_{n-1,n}), \ldots, \pi_1(c_{1,n})]\\
 			&= [\pi_{n-1}(c_{n-1,n-1})\cdot \pi_{n-1}(c_{n-1,n-1}),\dots, \pi_1(c_{1,n-1})\cdot \pi_1(c_{1,n-1})],
\end{align*}
 where the second equality follows from Corollary~\ref{cor:halfcycle}.
% $\Pi_{\overline{n}}$ are.
 By the induction hypothesis, the first row-wise half of $\Pi_{\overline{n}}(T_{\overline n})$, which is the same as $\Pi_{\overline{n}}(T_{n-1})$, consists of distinct rows. 
 Therefore, the table $\Pi_{\overline{n}}(T_{\overline n})$ consists of rows which are repeated exactly twice. For $i < j$, the rows $\Pi_{\overline{n}}(T_{\overline n},i)$ and
 $\Pi_{\overline{n}}(T_{\overline n},j)$ are equal when $j=i+2^{n-1}$.
Therefore, it suffices to prove that the rows  $\Pi(T_n,i)$ and $\Pi(T_n,i+2^{n-1})$ are distinct.
 Observe that to obtain the table $\Pi(T_n)$, we need to append $\pi_n(c_{n,n})$ as the first column in $\Pi_{\overline{n}}(T_{\overline n})$. As $c_{n,n}$ is equal to $0_{\times 2^{n-1}}\cdot 1_{\times 2^{n-1}}$, we have $c_{n,n}(i)\neq c_{n,n}(i+2^{n-1})$, this implies that $c_{n,n}(\pi_n(i))\neq c_{n,n}(\pi_n(i+2^{n-1}))$.
\end{proof}

Next, we define a notion of {\em cyclic hyper degrees} as follows.

\begin{definition}[Cyclic Hyper Degree]\label{def:chd}
 Given $\Pi$, a list of $n$ cyclic permutations, a tuple $d \in \mathbb{Z}_+^n$ is said to be a {\em cyclic hyper degree} if there exist $i, N \in [2^n]$ such that $d=\sum_{k=i}^N \Pi(T_n, k).$
\end{definition}

As the rows of $\Pi(T_n)$ are distinct, their contiguous sum  is in $H_n$ by definition. This gives us the main theorem of this section.

\begin{theorem}
 If $w\in \mathbb{Z}_+^n$ is a {\em cyclic hyper degree}, then $w\in H_n$.
\end{theorem}

We note that $(4,1,1,1)$ is a {\em realizable} hypergraph degree sequence but it is not a {\em cyclic hyper degree} sequence.
In the next section, we will show how to efficiently check if a given sequence $d$ is a {\em cyclic hyper degree}. 

\section{DROP: Workload Optimization}
\label{sec:algo}

In this section, we introduce DROP, a system that performs workload-aware DR via progressive sampling and online progress estimation.
DROP takes as input a target dataset, metric to preserve (default, target $TLB$), and an optional downstream runtime model.
DROP then uses sample-based PCA to identify and return a low-dimensional representation of the input that preserves the specified property while minimizing estimated workload runtime (Figure 2, Alg.~\ref{alg:DROP}).

%DROP answers a crucial question that stochastic PCA techniques have traditionally ignored: how long should these methods run? 

\begin{comment}
Notation used is in Table~\ref{table:inputs}.

\begin{table}
\centering
\small
\caption{\label{table:inputs} 
 DROP algorithm notation and defaults}
{\renewcommand{\arraystretch}{1.2}
\begin{tabular}{|c|l l|}
\hline 
Symbol & Description (\emph{Default}) & Type\tabularnewline
\hline
$X$  & Input dataset                          & $\mathbb{R}^{\mvar \times \dvar}$ \tabularnewline
$\mvar$  & Number of input data points            & $\mathbb{Z}_{+}$\tabularnewline
$\dvar$  & Input data dimension                   & $\mathbb{Z}_{+}$ \tabularnewline
$B$  & Target $TLB$ preservation      		 & $0 < \mathbb{R} \leq 1 $ \tabularnewline
$\mathcal{C}_\mvar(\dvar)$  & Downstream runtime function (\textit{k-NN runtime})       & $\mathbb{Z}_{+} \to \mathbb{R}_{+}$\tabularnewline
$R$  & Total DROP runtime       & $\mathbb{R}_{+}$ \tabularnewline
$c$ & Confidence level for $TLB$ preservation (\textit{$95 \%$})          & $\mathbb{R}$  \tabularnewline
$T_k$  & DROP output $k$-dimensional transformation &$\mathbb{R}^{\dvar \times k}$ \tabularnewline
$i $ & Current DROP iteration        & $\mathbb{Z}_+$  \tabularnewline

\hline 
\end{tabular}
}
\end{table}
\end{comment}

\begin{figure}
\includegraphics[width=\linewidth]{figs/progressive.pdf}
\caption[]{ Reduction in dimensionality for  $TLB = 0.80$ with progressive sampling. Dimensionality decreases until reaching a state equivalent to running PCA over the full dataset ("convergence").}
\label{fig:progressive}
\end{figure}

\subsection{DROP Algorithm}
\label{subsec:arch}
%DROP is a system that performs workload-aware dimensionality reduction, optimizing the combined runtime of downstream tasks and DR as defined in Problem~\ref{def:opt}.
DROP operates over a series of data samples, and determines when to terminate via \red{a} four-step procedure at each iteration: %progressive sampling, transformation evaluation, progress estimation, and cost-based optimization:

%To power this pipeline, DROP combines database and machine learning techniques spanning online aggregation (\S\ref{subsec:teval}), progress estimation (\S\ref{subsec:pest}), progressive sampling (\S\ref{subsec:psample}), and PCA approximation (\S\S\ref{subsec:pcaroutine},\ref{subsec:reuse}).

%We now provide a brief overview of DROP's sample-based iterative architecture before detailing each.

\begin{comment}
\item Progressive Sampling (\S\ref{subsec:psample}): DROP draws a data sample, performs PCA over it, and uses of a novel reuse mechanism across iterations (\S\ref{subsec:reuse}).

\item Transform Evaluation (\S\ref{subsec:teval}): DROP evaluates the above by identifying the size of the smallest metric-preserving transformation that can be extracted. 

\item Progress Estimation (\S\ref{subsec:pest}): Given the size of the smallest metric-preserving transform and the time required to obtain this transform, DROP estimates the size and computation time of continued iteration.

\item Cost-Based Optimization (\S\ref{subsec:opt}): DROP optimizes over DR and downstream task runtime to determine if it should terminate.
\end{comment}

\minihead{Step 1: Progressive Sampling (\S\ref{subsec:psample})}

\noindent DROP draws a data sample, performs PCA over it, and uses a novel reuse mechanism across iterations (\S\ref{subsec:reuse}).

\minihead{Step 2: Transform Evaluation (\S\ref{subsec:teval})} 

\noindent DROP evaluates the above by identifying the size of the smallest metric-preserving transformation that can be extracted. 

\minihead{Step 3: Progress Estimation (\S\ref{subsec:pest})} 

\noindent Given the size of the smallest metric-preserving transform and the time required to obtain this transform, DROP estimates the size and computation time of continued iteration.

\minihead{Step 4: Cost-Based Optimization (\S\ref{subsec:opt})} 

\noindent DROP optimizes over DR and downstream task runtime to determine if it should terminate.

\subsection{Progressive Sampling}
\label{subsec:psample}

Inspired by stochastic PCA methods (\S\ref{sec:relatedwork}), DROP uses sampling to tackle workload-aware DR. 
Many real-world \red{datasets} are intrinsically low-dimensional; a small data sample is sufficient to characterize dataset behavior. 
To verify, we extend our case study (\S\ref{sec:RQW}) by computing how many uniformly selected data samples are required to obtain a $TLB$-preserving transform with $k$ equal to input dimension $\dvar$.
On average, a sample of under $0.64\%$ $(\text{up to } 5.5\%)$ of the input is sufficient for $TLB = 0.75$, and under $4.2\%$ $(\text{up to } 38.6\%)$ is sufficient for $TLB=0.99$.  
If this sample rate is known a priori, we obtain up to \red{$91\times$ speedup} over PCA via SVD.%---with no algorithmic improvement. 

However, this benefit is dataset-dependent, and unknown a priori.
We thus turn to progressive sampling (gradually increasing the sample size) to identify how large a sample suffices.
Figure~\ref{fig:progressive} shows how the dimensionality required to attain a given $TLB$ changes when we vary dataset and proportion of data sampled.
Increasing the number of samples (which increases PCA runtime) provides lower $k$ for the same $TLB$.
However, this decrease in dimension plateaus as the number of samples increases.
Thus, while progressive sampling allows DROP to tune the amount of time spent on DR, DROP must determine when the downstream value of decreased dimension is overpowered by the cost of DR---that is, whether to sample to convergence or terminate early (e.g., at $0.3$ proportion of data sampled for SmallKitchenAppliances). 


Concretely, DROP first repeatedly chooses a subset of data and computes a $\dvar$-dimensional transformation via PCA on the subsample, and then proceeds to determine if continued sampling is beneficial to end-to-end runtime.
We consider a simple uniform sampling strategy: each iteration, DROP samples a fixed percentage of the data.
 
 
 
 
 
%Exploring data-dependent and weighted sampling schemes that are dependent on the current basis is an exciting area for future work. 
%While we considered a range of alternative sampling strategies, uniform sampling strikes a balance between computational and statistical efficiency. 
%Data-dependent and weighted sampling schemes that are dependent on the current basis may decrease the total number of iterations required by DROP, but may require expensive reshuffling of data at each iteration~\cite{coresets}. 

%DROP provides configurable strategies for both base number of samples and the per-iteration increment, in our experimental evaluation in \S\ref{sec:experiments}, we consider a sampling rate of $1\%$ per iteration.
%We discuss more sophisticated additions to this base sampling schedule in the extended manuscript.

\begin{algorithm}[t!]
\begin{algorithmic}[1]
\small
\Statex \textbf{Input:}  $X$: data; $B$: target metric preservation level; $\mathcal{C}_\mvar$: cost of downstream operations
\Statex \textbf{Output:} $T_k$: $k$-dimensional transformation matrix
\Statex
\Statex \hrule
\Function{drop}{$X,  B, \mathcal{C}_\mvar$}:
	\State Initialize: $i = 0; k_0 = \infty$ 
		\Comment{iteration and current basis size}
	\Do
		\State i$\texttt{++}$, \textsc{clock.restart}
		\State $X_i$ = \textsc{sample}($X, \textsc{sample-schedule}(i)$) \label{eq:sample}
			\Comment{\S~\ref{subsec:psample}}
		\State $T_{k_i}$ = \textsc{compute-transform}($X, X_i,  B$) \label{eq:evaluate}
			\Comment{\S~\ref{subsec:teval}}
		\State $r_i = \textsc{clock.elapsed}$	
			\Comment{$R = \sum_i r_i$}
		\State $\hat{k}_{i+1}, \hat{r}_{i+1} $ = \textsc{estimate}($k_i, r_i$) \label{eq:estimate}
			\Comment{\S~\ref{subsec:pest}}
	\doWhile{\textsc{optimize}($\mathcal{C}_\mvar,k_i,r_i,\hat{k}_{i+1}, \hat{r}_{i+1}$)} \label{eq:optimize}
		\Comment{\S~\ref{subsec:opt}}
	\\\Return{$T_{k_i}$}
\EndFunction
\end{algorithmic}
\caption{DROP Algorithm}
\label{alg:DROP}
\end{algorithm}



\subsection{Transform Evaluation}
\label{subsec:teval}
DROP must accurately and efficiently evaluate this iteration's performance with respect to the metric of interest \red{over the entire dataset}. 
%To do so, DROP adapts an approach for deterministic queries in online aggregation: treating quality metrics as aggregation functions and using confidence intervals for fast estimation. 
%We first discuss this approach in the context of $TLB$, then discuss how to extend this approach to alternative metrics at the end of this section.
We define this iteration's performance as the size of the lowest dimensional $TLB$-preserving transform ($k_i$) that it can return. 
There are two challenges in performance evaluation.
First, the lowest $TLB$-achieving $k_i$ is unknown a priori. 
Second, brute-force $TLB$ computation would dominate the runtime of computing PCA over a sample. 
We now describe how to solve these challenges.

\subsubsection{Computing the Lowest Dimensional Transformation}

Given the $\dvar$-dimensional transformation from step 1, to reduce dimensionality, DROP must determine if a smaller dimensional $TLB$-preserving transformation can be obtained and return the smallest such transform. 
Ideally, the smallest $k_i$ would be known a priori, but in practice, this is not true---thus, DROP uses the $TLB$ constraint and two properties of PCA to automatically identify it.
%A na\"ive strategy would evaluate the $TLB$ for every combination of the $\dvar$ basis vectors for every transformation size, requiring $O(2^\dvar)$ evaluations. 
%Instead, DROP exploits two key properties of PCA to avoid this.

First, PCA via SVD produces an orthogonal linear transformation where the principal components  are returned in order of decreasing dataset variance explained.
As a result, once DROP has computed the transformation matrix for dimension $\dvar$, DROP obtains the transformations for all dimensions $k$ less than $\dvar$ by truncating the matrix to $\dvar \times k$ .
%PCA via SVD produces an orthogonal linear transformation where the first principal component explains the most variance in the dataset, the second explains the second most---subject to being orthogonal to the first---and so on.  

Second, with respect to $TLB$ preservation, the more principal components that are retained, the better the lower-dimensional representation in terms of $TLB$.  
This is because orthogonal transformations such as PCA preserve inner products. 
Therefore, an $\dvar$-dimensional PCA perfectly preserves $\ell_2$-distance between data points. 
As $\ell_2$-distance is a sum of squared (positive) terms, the more principal components retained, the better the representation preserves $\ell_2$-distance.

Using the first property, DROP obtains all low-dimensional transformations for the sample from the $\dvar$-dimensional basis.  
Using the second property, DROP runs binary search over these transformations to return the lowest-dimensional basis that attains $B$ (Alg.~\ref{alg:candidate}, l\ref{eq:basis}).
If $B$ cannot be realized with this sample, DROP omits further optimization steps and continues the next iteration by drawing a larger sample.

Additionally, computing the full $\dvar$-dimensional basis at every iteration may be wasteful. 
Thus, if DROP has found a candidate $TLB$-preserving basis of size $\dvar' < \dvar$ in prior iterations, then DROP only computes $\dvar'$ components at the start of the next iteration.
This allows for more efficient PCA computation for future iterations, as advanced PCA routines can exploit the $\dvar'$-th eigengap to converge faster (\S\ref{sec:relatedwork}).
% \red{This is because similar to a hold-out or validation set, $TLB$ evaluation is representative of the entire dataset, not just the current sample (see Alg.~\ref{alg:candidate} L5). 
%Thus, sampling additional training datapoints enables DROP to better learn global data structure and perform at least as well as over a smaller sample.}


% stop here!

\subsubsection{Efficient $TLB$ Computation}

Given a transformation, DROP must determine if it preserves the desired $TLB$.
Computing pairwise $TLB$ for all data points requires $O(\mvar^2\dvar)$ time, which dominates the runtime of computing PCA on a sample.
However, as the $TLB$ is an average of random variables bounded from 0 to 1, DROP can use sampling and confidence intervals to compute the $TLB$ to arbitrary confidences.

Given a transformation, DROP iteratively refines an estimate of its $TLB$ (Alg.~\ref{alg:candidate}, l\ref{eq:eval}) by \red{incrementally sampling an increasing number of} pairs from the input data (Alg.~\ref{alg:candidate}, l\ref{eq:paircheck}), transforming each pair into the new basis, then measuring the distortion of $\ell_2$-distance between the pairs, providing a $TLB$ estimate to confidence level $c$ (Alg.~\ref{alg:candidate}, l\ref{eq:tlbeval}). 
If the confidence interval's lower bound is greater than the target $TLB$, the basis is a sufficiently good fit; if its upper bound is less than the target $TLB$, the basis is not a sufficiently good fit. 
If the confidence interval contains the target $TLB$,  \red{ DROP cannot determine if the target $TLB$ is achieved. 
Thus, DROP automatically samples additional pairs to refine its estimate.
%in practice, and especially for our initial target time series datasets, DROP rarely uses more than 500 pairs on average in its $TLB$ estimates (often using far fewer)
}

To estimate the $TLB$ to confidence $c$, DROP uses the Central Limit Theorem: computing the standard deviation of a set of sampled pairs' $TLB$ measures and applying a confidence interval to the sample according to the $c$.
%For low variance data, DROP evaluates a candidate basis with few samples from the dataset \red{as the confidence intervals shrink rapidly}. 

The techniques in this section are presented in the context of $TLB$, but can be applied to any downstream task and metric for which we can compute confidence intervals and are monotonic in number of principal components retained.

\begin{comment}
\red{For instance, DROP can operate while using all of its optimizations when using any $L^p$-norm.}
\red{Euclidean similarity search} is simply one such domain that is a good fit for PCA: when performing DR via PCA, as we increase the number of principal components, a clear positive correlation exists between the percent of variance explained and the $TLB$ regardless of data spectrum.
We demonstrate this correlation in the experiment below, where we generate three synthetic datasets with predefined spectrum (right), representing varying levels of structure present in real-world datasets. 
The positive correlation is evident (left) despite the fact that the two do not directly correspond ($x=y$ provided as reference). 
This holds true for all of the evaluated real world datasets.

\vspace{.2cm}
\includegraphics[width= .9\linewidth]{figs/tlb-pca.pdf}

For alternative preservation metrics, we can utilize closed-form confidence intervals~\cite{stats-book,ci1,onlineagg}, or bootstrap-based methods~\cite{bootstrap1,bootstrap2}, which incur higher overhead but can be more generally applied.
\end{comment}

\begin{algorithm}
\begin{algorithmic}[1]
\small
\Statex \textbf{Input:}  
\Statex $X$: sampled data matrix
\Statex $B$: target metric preservation level; default $TLB = 0.98$
\Statex  \hrule 
\Function{compute-transform}{$X, X_i B$}: \label{eq:basis}
	\State \textsc{pca.fit}$(X_i)$
			\Comment{fit PCA on the sample}
	\State Initialize: high $= k_{i-1}$; low $=0$; $k_i= \frac{1}{2}$(low + high); $B_i = 0$
	\While{(low $!=$ high)}
		\State $T_{k_i}, B_i  = \textsc{evaluate-tlb}( X, B, k_i)$
		\If{$B_i \leq B$}  low $= k_i + 1$ 
		\Else  \hspace{0pt} high $= k_i $
		\EndIf
		\State $k_i = \frac{1}{2}$(low + high)
	\EndWhile
	\State $T_{k_i} = $ cached $k_i$-dimensional PCA transform\\
	\Return $T_{k_i}$
\EndFunction
\Statex 
\Function{evaluate-tlb}{$X, B, k$}: \label{eq:eval}
	\State numPairs $= \frac{1}{2}\mvar(\mvar-1)$
	\State $p = 100$
		\Comment{number of pairs to check metric preservation}
	\While{($p < $ numPairs)}
		\State $B_i, B_{lo}, B_{hi} = $ \textsc{tlb}($ X, p, k$)
			 \label{eq:paircheck}
		\If{($B_{lo} > B$ or $B_{hi} < B$)}   \textbf{break}
		\Else \hspace{0pt} pairs $\times$= $ 2$
		\EndIf
	\EndWhile
	\\\Return $B_i$	
\EndFunction
\Statex 
\Function{tlb}{$X, p, k$}: \label{eq:tlbeval}
	\State \textbf{return } mean and 95\%-CI of the $TLB$ after transforming $p$ $d$-dimensional pairs of points from $X$ to dimension $k$. The highest transformation computed thus far is cached to avoid recomputation of the transformation matrix.
\EndFunction

\end{algorithmic}
\caption{Basis Evaluation and Search}
\label{alg:candidate}
\end{algorithm}


\subsection{Progress Estimation}
\label{subsec:pest}
%Given a low dimensional $TLB$-achieving transformation from the evaluation step, DROP must identify the dimensionality $k_i$ and runtime ($r_i$) of the transformation that would be obtained from an additional DROP iteration.
%We refer to this as the $progress estimation$ step.

Recall that the goal of workload-aware DR is to minimize $R + \mathcal{C}_\mvar(k)$ such that $TLB(XT_k) \geq B$, with $R$ denoting total DR (i.e., DROP's) runtime, $T_k$ the $k$-dimensional $TLB$-preserving transformation of data $X$ returned by DROP, and $\mathcal{C}_\mvar(k)$ the workload cost function. 
Therefore, given a $k_i$-dimensional transformation $T_{k_i}$ returned by the evaluation step of DROP's $i^{\text{th}}$ iteration, DROP can compute the value of this objective function by substituting its elapsed runtime for $R$ and $T_{k_i}$ for $T_k$.  
We denote the value of the objective at the end of iteration $i$ as $obj_i$. 

To decide whether to continue iterating to find a lower dimensional transform, we show in  \S\ref{subsec:opt} that DROP must estimate $obj_{i+1}$. To do so, DROP must estimate the runtime required for iteration $i+1$ (which we denote as $r_{i+1}$, where $R=\sum_i r_i$ after $i$ iterations) and the dimensionality of the $TLB$-preserving transformation produced by iteration $i+1$, $k_{i+1}$. 
DROP cannot directly measure $r_{i+1}$ or $k_{i+1}$ without performing iteration $i+1$, thus performs online progress estimation. Specifically, DROP performs online parametric fitting to compute future values based on prior values for $r_{i}$ and $k_i$ (Alg.~\ref{alg:DROP}, l\ref{eq:estimate}). 
By default, given a sample of size $m_i$ in iteration $i$, DROP performs linear extrapolation to estimate $k_{i+1}$ and $r_{i+1}$. The estimate of $r_{i+1}$, for instance, is:

\vspace{-.4cm}
\begin{equation*}
\hat{r}_{i+1} = r_i + \frac{r_i - r_{i-1}}{m_i - m_{i-1}} (m_{i+1} -  m_i).
\end{equation*}

\begin{comment}
\red{
DROP's use of a basic first-order approximation is motivated by the fact that when adding a small number of data samples each iteration, both runtime and resulting lower dimension do not change drastically (i.e., see Fig.~\ref{fig:progressive} after a feasible point is achieved). 
While linear extrapolation acts as a proof-of-concept for progress estimation, the architecture can incorporate more sophisticated functions as needed (\S\ref{sec:relwork}).
}
\end{comment}

\subsection{Cost-Based Optimization}
\label{subsec:opt}

DROP must determine if continued PCA on additional samples will improve overall runtime. 
%We refer to this as the $cost-based optimization$ step. 
Given predictions of the next iteration's runtime ($\hat{r}_{i+1}$) and dimensionality ($\hat{k}_{i+1}$), DROP uses a greedy heuristic to estimate the optimal stopping point.
If the estimated objective value is greater than its current value ($obj_i < \widehat{obj}_{i+1}$), DROP will terminate. 
If DROP's runtime is convex in the number of iterations, we can prove that this condition is the optimal stopping criterion via convexity of composition of convex functions. 
This stopping criterion leads to the following check at each iteration (Alg.\ref{alg:DROP}, l\ref{eq:optimize}): 

\vspace{-.4cm}
\begin{align}
  obj_i &< \widehat{obj}_{i+1} \nonumber \\
  \mathcal{C}_\mvar(k_i) + \sum_{j=0}^i r_j &< \mathcal{C}_\mvar(\hat{k}_{i+1}) + \sum_{j=0}^{i} r_j + \hat{r}_{i+1} \nonumber \\
  % \mathcal{C}_\mvar(k_i)  &< \mathcal{C}_\mvar(\hat{k}_{i+1}) + \hat{r}_{i+1}  \nonumber \\
  \mathcal{C}_\mvar(k_i) - \mathcal{C}_\mvar(\hat{k}_{i+1}) &< \hat{r}_{i+1}  \label{eq:check}
\end{align}

DROP terminates when the projected time of the next iteration exceeds the estimated downstream runtime benefit. 
%Absent $\mathcal{C}_d$, we default to execution until convergence (i.e, $k$ plateaus), and show the cost of doing so in \S\ref{sec:experiments}.


\begin{comment}
\red{In the general case as the rate of decrease in dimension ($k_i$) is data dependent, thus convexity is not guaranteed. 
Should $k_i$ plateau before continued decrease, DROP will terminate prematurely. 
This occurs during DROP's first iterations if sufficient data to meet the $TLB$ threshold at a dimension lower than $\dvar$ has not been sampled (SmallKitchenAppliances in Fig.~\ref{fig:progressive}).
Thus, optimization is only enabled once a feasible point is attained, as we prioritize accuracy over runtime (i.e., $0.3$ for SmallKitchenAppliances).
We show the implications of this decision in DROP in \S\ref{subsec:arch}.%, and in the streaming setting in the extended manuscript.
}
\end{comment}

\subsection{Choice of PCA Subroutine}
\label{subsec:pcaroutine}

The most straightforward means of implementing PCA via SVD in DROP is computationally inefficient compared to DR alternatives (\S\ref{sec:background}).  
DROP computes PCA via a randomized SVD algorithm from~\cite{tropp} (SVD-Halko).
Alternative efficient methods for PCA exist (i.e., PPCA, which we also provide), but we found that SVD-Halko is asymptotically of the same running time as techniques used in practice, is straightforward to implement, is $2.5-28\times$ faster than our baseline implementations of SVD-based PCA, PPCA, and Oja's method, and does not require hyperparameter tuning for batch size, learning rate, or convergence criteria.  
%While SVD-Halko is not as efficient as other techniques with respect to communication complexity as in~\cite{ppca-sigmod}, or convergence rate as in~\cite{re-new}, these techniques can be easily substituted for SVD-Halko in DROP's architecture.
%%%%We demonstrate this by implementing multiple alternatives in \S\ref{subsec:pcaexp}.
%%%%\red{Further, we also demonstrate that this implementation is competitive with the widely used SciPy Python library~\cite{scipy}}.

\begin{comment}
\begin{algorithm}[t]
\begin{algorithmic}
\State \textbf{Input:}  \\
$H$: concatenation of previous transformation matrices \\
$T$: new sample's transformation \\
 points to sample per iteration; default 5\% \\
 
\\ \hrule

\Function{distill}{$H, T$}:
	\State $H \gets [H | T]$
		\Comment{Horizontal concatenation to update history}
	\State $U, \Sigma, V^\intercal \gets \textsc{SVD}(H)$ 
				\Comment{$U$ is a basis for the range of $T$}
	\State $T \gets U[:,\textsc{num-columns(T)}]$
	\\\Return{$T$}
\EndFunction
\end{algorithmic}
\caption{Work Reuse}
\label{alg:reuse}
\end{algorithm} 
\end{comment}

\subsection{Work Reuse}
\label{subsec:reuse}

A natural question arises due to DROP's iterative architecture: can we combine information across each sample's transformations without computing PCA over the union of the data samples? 
Stochastic PCA methods enable work reuse across samples as they iteratively refine a single transformation matrix, but other methods do not.
%We propose an algorithm that allows reuse of previous work when utilizing arbitrary PCA routines with DROP.
DROP uses two insights to enable work reuse over any PCA routine.

First, given PCA transformation matrices $T_1$ and $T_2$, their horizontal concatenation $H = [T_1 | T_2]$ is a transformation into the union of their range spaces.
Second, principal components returned from running PCA on repeated data samples generally concentrate to the true top principal components for datasets with rapid spectrum drop off.
Work reuse thus proceeds as follows:
DROP maintains a transformation history consisting of the horizontal concatenation of all transformations to this point, computes the SVD of this matrix, and returns the first $k$ columns as the transformation matrix. 

Although this requires an SVD computation, computational overhead is dependent on the size of the history matrix, not the dataset size.
This size is proportional to the original dimensionality $\dvar$ and size of lower dimensional transformations, which are in turn proportional to the data's intrinsic dimensionality and the $TLB$ constraint.
As preserving \emph{all history} can be expensive in practice, 
DROP periodically shrinks the history matrix using DR via PCA. 
We validate the benefit of using work reuse---up to \red{15\%} on real-world data---in \S\ref{sec:experiments}.



\section{Case Studies}
\label{sec:case_studies}
In this section, we present a case study of Facebook posts from an Australian public page.
The page shifts between early 2020 (\emph{2019-2020 Australian bushfire season}) and late 2020 (\emph{COVID-19 crises}) from being a moderate-right group for discussion around climate change to a far-right extremist group for conspiracy theories.


\begin{figure*}[!tbp]
	\begin{subfigure}{0.21\textwidth}
		\includegraphics[width=\textwidth]{images/facebook1.png}
		\caption{}
		\label{subfig:first-posting}
		\includegraphics[width=0.9\textwidth]{images/facebook3.jpg}
		\caption{}
		\label{subfig:comment-post-1}
	\end{subfigure}
    \begin{subfigure}{0.28\textwidth}
		\includegraphics[width=\textwidth]{images/facebook2.jpg}
		\caption{}
		\label{subfig:second-posting}
	\end{subfigure}
    \begin{subfigure}{0.23\textwidth}
		\includegraphics[width=\textwidth]{images/facebook4.jpg}
		\caption{}
		\label{subfig:comment-post-2a}
	\end{subfigure}
    \begin{subfigure}{0.23\textwidth}
		\includegraphics[width=\textwidth]{images/facebook5.jpg}
		\caption{}
		\label{subfig:comment-post-2b}
	\end{subfigure}
	\caption{
		Examples of postings and comment threads from a public Facebook page from two periods of time early 2020 (a) and late 2020 (b)-(e), which show a shift from climate change debates to extremist and far-right messaging.
	}
	\label{fig:facebook}
\end{figure*}

We focus on a sample of 2 postings and commenting threads from one Australian Facebook page we classified as ``far-right'' based on the content on the page. 
We have anonymized the users in \Cref{fig:facebook} to avoid re-identification.
The first posting and comment thread (see \Cref{subfig:first-posting}) was collected on Jan 10, 2020, and responded to the Australian bushfire crisis that began in late 2019 and was still ongoing in January 2020. It contains an ambivalent text-based provocation that references disputes in the community regarding the validity of climate change and climate science. 

The second posting and comment thread (see \Cref{subfig:second-posting}) was collected from the same page in September 2020, months after the bushfire crisis had abated.
At that time, a new crisis was energizing and connecting the far-right groups in our dataset --- i.e., the COVID-19 pandemic and the government interventions to curb the spread of the virus. 
The post is different in style compared to the first.
It is image-based instead of text-based and highly emotive, with a photo collage bringing together images of prison inmates with iron masks on their faces (top row) juxtaposed to people wearing face masks during COVID-19 (bottom row). 
The image references the public health orders issued during Melbourne's second lockdown and suggests that being ordered to wear masks is an infringement of citizen rights and freedoms, similar to dehumanizing restraints used on prisoners.

To analyze reactions to the posts, two researchers used a deductive analytical approach to separately code and to analyze the commenting threads --- see \Cref{subfig:comment-post-1} for comments of the first posting, and \Cref{subfig:comment-post-2a,subfig:comment-post-2b} for comments on the second posting. 
Conversations were also inductively coded for emerging themes. 
During the analysis, we observed qualitative differences in the types of content users posted, interactions between commenters, tone and language of debate, linked media shared in the commenting thread, and the opinions expressed.
The rest of this section further details these differences.
To ensure this was not a random occurrence, we tested the exemplar threads against field notes collected on the group during the entire study.
We also used Facebook's search function within pages to find a sample of posts from the same period and which dealt with similar topics. 
After this analysis, we can confidently say that key changes occurred in the group between the bushfire crisis and COVID-19, that we detail next.

\subsubsection*{Exemplar 1 --- climate change skepticism.}
To explore this transformation in more depth, we analyzed comments scraped on the first posting --- \cref{sub@subfig:comment-post-1} shows a small sample of these comments.
The language used was similar to comments that we observed on numerous far-right nationalist pages at the time of the bushfires.
These comments are usually text-based, employing emojis to denote emotions, and sometimes being mocking or provocative in tone. 
Noteworthy for this commenting thread is the 50/50 split in the number of members posting in favor of action on climate change (on one side) and those who posted anti-Greens and anti-climate change science posts and memes (on the other side).
The two sides aligned strongly with political partisanship --- either with Liberal/National coalition (climate change deniers) or Labor/Green (climate change believers) parties. 
This is rather unusual for pages classified as far-right. 

We observed trolling practices between the climate change deniers and believers, which often descend into \emph{flame wars} --- i.e., online ``firefights that take place between disembodied combatants on electronic bulletin boards''~\citep{bukatman1994flame}.
The result is a boosted engagement on the post but also the frustration and confusion of community members and lurkers who came to the discussions to become informed or debate rationally on key differences between the two positions.
They often even become targeted, victimized, and baited by trolls on both sides of the partisan divide. 
The opinions expressed by deniers in commenting sections range from skepticism regarding climate change science to plain denial.
Deniers also regard a range of targets as embroiled in a climate change conspiracy to deceive the public, such as The Greens and their environmental policy, in some cases the government, the United Nations, and climate change celebrities like David Attenborough and Greta Thunberg. 
These figures are blamed for either exaggerating risks of climate change or creating a climate change hoax to increase the influence of the UN on domestic governments or to increase domestic governments' social control over citizens. 

Both coders noted that flame wars between these opposing personas contained very few links to external media. 
Where links were added, they often seemed disconnected from the rest of the conversation and were from users whose profiles suggested they believed in more radical conspiracy theories.
One such example is ``geo-engineering'' (see \cref{sub@subfig:comment-post-1}).
Its adherents believe that solar geo-engineering programs designed to combat climate change are secretly used by a global elite to depopulate the world through sterilization or to control and weaponize the weather.

Nonetheless, apart from the random comments that hijack the thread, redirecting users to external ``alternative'' news sites and Twitter, and the trolls who seem to delight in victimizing unsuspecting victims, the discussion was pretty healthy.
There are many questions, rational inquiries, and debates between users of different political persuasion and views on climate change.
This, however, changes in the span of only a couple of months.

\subsubsection*{Exemplar 2 --- posting and commenting thread.}
We observe a shift in the comment section of the post collected during the second wave of the COVID pandemic (\Cref{sub@subfig:second-posting}) --- which coincided with government laws mandating the public to wear masks and stay at home in Victoria, Australia.
There emerges much more extreme far-right content that converges with anti-vaccination opinions and content.
We also note a much higher prevalence of conspiracy theories often implicating racialized targets.
This is exemplified in the comments on the second post (\Cref{sub@subfig:comment-post-2a,sub@subfig:comment-post-2b}) where Islamophobia and antisemitism are confidently asserted alongside anti-mask rhetoric.
These comments consider face masks similar to the religious head coverings worn by some Muslim women, which users describe as ``oppressive'' and ``silencing''. 
In this way, anti-maskers cast women as a distinct, sympathetic marginalized demographic.
However, this is enacted alongside the racialization and demonization of Islam as an oppressive religion. 

Given the extreme racialization of anti-mask rhetoric, some commenters contest these positions, arguing that the page is becoming less an anti-Scott Morrison page (Australia's Prime Minister at the time) and changing into a page that harbors ``far-right dickheads''.
This questioning is actively challenged by far-right commenters and conspiracy theorists on the page, who regarded pro-mask users and the Scott Morrison government as ``puppets'' being manipulated by higher forces (see \Cref{sub@subfig:comment-post-2b}). 

This indicates a significant change on the page's membership towards the extreme-right, who employs more extreme forms of racialized imagery, with more extreme opinion being shared.
Conspiracy theorists become more active and vocal, and they consistently challenge the opinions of both center conservative and left-leaning users. 
This is evident in the final two comments in \Cref{subfig:comment-post-2b}, which reflect QAnon style conspiracy theories and language.
Public health orders to wear masks are being connected to a conspiracy that all of these decisions are directed by a secret network of global Jewish elites, who manipulate the pandemic to increase their power and control. 
This rhetoric intersects with the contemporary ``QAnon'' conspiracy theory, which evolved from the ``Pizzagate'' conspiracy theory.
They also heavily draw on well-established antisemitic blood libel conspiracy theories, which foster beliefs that a powerful global elite is controlling the decisions of organizations such as WHO and are responsible for the vaccine rollout and public health orders related to the pandemic.
The QAnon conspiracy is also influenced by Bill Gates' Microchips conspiracy theory, i.e., the theory that the WHO and the Bill Gates Foundation global vaccine programs are used to inject tracking microchips into people.

These conspiracy theories have, since COVID-19, connected formerly separate communities and discourses, uniting existing anti-vaxxer communities, older demographics who are mistrustful of technology, far-right communities suspicious of global and national left-wing agendas, communities protesting against 5G mobile networks (for fear that they will brainwash, control, or harm people), as well as generating its own followers out of those anxious during the 2020 onset of the COVID-19 pandemic.
We detect and describe some of these opinion dynamics in the next section.

% \vspace{-0.5em}
\section{Conclusion}
% \vspace{-0.5em}
Recent advances in multimodal single-cell technology have enabled the simultaneous profiling of the transcriptome alongside other cellular modalities, leading to an increase in the availability of multimodal single-cell data. In this paper, we present \method{}, a multimodal transformer model for single-cell surface protein abundance from gene expression measurements. We combined the data with prior biological interaction knowledge from the STRING database into a richly connected heterogeneous graph and leveraged the transformer architectures to learn an accurate mapping between gene expression and surface protein abundance. Remarkably, \method{} achieves superior and more stable performance than other baselines on both 2021 and 2022 NeurIPS single-cell datasets.

\noindent\textbf{Future Work.}
% Our work is an extension of the model we implemented in the NeurIPS 2022 competition. 
Our framework of multimodal transformers with the cross-modality heterogeneous graph goes far beyond the specific downstream task of modality prediction, and there are lots of potentials to be further explored. Our graph contains three types of nodes. While the cell embeddings are used for predictions, the remaining protein embeddings and gene embeddings may be further interpreted for other tasks. The similarities between proteins may show data-specific protein-protein relationships, while the attention matrix of the gene transformer may help to identify marker genes of each cell type. Additionally, we may achieve gene interaction prediction using the attention mechanism.
% under adequate regulations. 
% We expect \method{} to be capable of much more than just modality prediction. Note that currently, we fuse information from different transformers with message-passing GNNs. 
To extend more on transformers, a potential next step is implementing cross-attention cross-modalities. Ideally, all three types of nodes, namely genes, proteins, and cells, would be jointly modeled using a large transformer that includes specific regulations for each modality. 

% insight of protein and gene embedding (diff task)

% all in one transformer

% \noindent\textbf{Limitations and future work}
% Despite the noticeable performance improvement by utilizing transformers with the cross-modality heterogeneous graph, there are still bottlenecks in the current settings. To begin with, we noticed that the performance variations of all methods are consistently higher in the ``CITE'' dataset compared to the ``GEX2ADT'' dataset. We hypothesized that the increased variability in ``CITE'' was due to both less number of training samples (43k vs. 66k cells) and a significantly more number of testing samples used (28k vs. 1k cells). One straightforward solution to alleviate the high variation issue is to include more training samples, which is not always possible given the training data availability. Nevertheless, publicly available single-cell datasets have been accumulated over the past decades and are still being collected on an ever-increasing scale. Taking advantage of these large-scale atlases is the key to a more stable and well-performing model, as some of the intra-cell variations could be common across different datasets. For example, reference-based methods are commonly used to identify the cell identity of a single cell, or cell-type compositions of a mixture of cells. (other examples for pretrained, e.g., scbert)


%\noindent\textbf{Future work.}
% Our work is an extension of the model we implemented in the NeurIPS 2022 competition. Now our framework of multimodal transformers with the cross-modality heterogeneous graph goes far beyond the specific downstream task of modality prediction, and there are lots of potentials to be further explored. Our graph contains three types of nodes. while the cell embeddings are used for predictions, the remaining protein embeddings and gene embeddings may be further interpreted for other tasks. The similarities between proteins may show data-specific protein-protein relationships, while the attention matrix of the gene transformer may help to identify marker genes of each cell type. Additionally, we may achieve gene interaction prediction using the attention mechanism under adequate regulations. We expect \method{} to be capable of much more than just modality prediction. Note that currently, we fuse information from different transformers with message-passing GNNs. To extend more on transformers, a potential next step is implementing cross-attention cross-modalities. Ideally, all three types of nodes, namely genes, proteins, and cells, would be jointly modeled using a large transformer that includes specific regulations for each modality. The self-attention within each modality would reconstruct the prior interaction network, while the cross-attention between modalities would be supervised by the data observations. Then, The attention matrix will provide insights into all the internal interactions and cross-relationships. With the linearized transformer, this idea would be both practical and versatile.

% \begin{acks}
% This research is supported by the National Science Foundation (NSF) and Johnson \& Johnson.
% \end{acks}

%\bibliographystyle{splncs03}
\bibliographystyle{plainurl}% the recommended bibstyle
\bibliography{ref}

\end{document}  