\documentclass[a4paper,UKenglish]{lipics-v2016}
\usepackage{xspace}
\usepackage{amssymb}
\usepackage{amsfonts}
\usepackage{amsmath}

\newtheorem{observation}{Observation}
\renewcommand{\mod}{{\rm ~mod~}}

\usepackage{mathtools}
\DeclarePairedDelimiter{\floor}{\lfloor}{\rfloor}
%\newtheorem{lemma}[theorem]{Lemma}
%\newcommand{\min}[1]{${\rm min}$[#1]$}
%\newcommand{\max}[1]{${\rm max}[#1]$}
%complexity classes
\newcommand{\cc}[1]	{\textsf{#1}\xspace}
\newcommand{\YES}	{\textsc{Yes}\xspace}
\newcommand{\NO}		{\textsc{No}\xspace}
\newcommand{\OO}		{\mathcal{O}}

\newcommand{\problemdef}[3]{
    \begin{center}
    \noindent
    \framebox{
        \begin{minipage}{4.6in}
            \textbf{\sc #1} \\
            \emph{Input}: #2 \\
            \emph{Question}: #3
        \end{minipage}
    }
    \end{center}
}

\newcommand{\hds}{{\sc Hypergraph Degree Sequence}\xspace}
\newcommand{\NPC}{{\sf NP}-Complete\xspace}

\title{Cyclic Hypergraph Degree Sequences}
%\titlerunning{}
%\toctitle{<Your changed title for the table of contents>}
%\subtitle{<subtitle of your contribution>}
%\authorrunning{<abbreviated author list>}

\author{S M Meesum}
\affil{The Institute of Mathematical Sciences,\\ HBNI, Chennai, India.\\
\texttt{meesum@imsc.res.in}} 

%\date{}                                           % Activate to display a given date or no date

\subjclass{G.2.1 Combinatorics, G.2.2 Hypergraphs}% mandatory: Please choose ACM 1998 classifications from http://www.acm.org/about/class/ccs98-html . E.g., cite as "F.1.1 Models of Computation". 
\keywords{Hypergraph, Realizable Degree Sequence}% mandatory: Please provide 1-5 keywords
% Author macros::end %%%%%%%%%%%%%%%%%%%%%%%%%%%%%%%%%%%%%%%%%%%%%%%%%

%Editor-only macros:: begin (do not touch as author)%%%%%%%%%%%%%%%%%%%%%%%%%%%%%%%%%%
\EventEditors{John Q. Open and Joan R. Acces}
\EventNoEds{2}
\EventLongTitle{42nd Conference on Very Important Topics (CVIT 2016)}
\EventShortTitle{CVIT 2016}
\EventAcronym{CVIT}
\EventYear{2016}
\EventDate{December 24--27, 2016}
\EventLocation{Little Whinging, United Kingdom}
\EventLogo{}
\SeriesVolume{42}
\ArticleNo{23}
% Editor-only macros::end %%%%%%%%%%%%%%%%%%%%%%%%%%%%%%%%%%%%%%%%%%%%%%%


\begin{document}
\maketitle
\begin{abstract}
The problem of efficiently characterizing degree sequences of simple hypergraphs is a fundamental long-standing open problem in Graph Theory.
Several results are known for restricted versions of this problem. This paper adds to the list of sufficient conditions for a degree sequence to be {\em hypergraphic}.
This paper proves a combinatorial lemma about cyclically permuting the columns of a binary table with length $n$ binary sequences as rows. We prove that for any set of cyclic permutations acting on its columns, the resulting table has all of its $2^n$ rows distinct. Using this property, we first define a subset {\em cyclic hyper degrees} of hypergraphic sequences and show that they admit a polynomial time recognition algorithm. Next, we prove that there are at least $2^{\frac{(n-1)(n-2)}{2}}$ {\em cyclic hyper degrees}, which also serves as a lower bound on the number of {\em hypergraphic} sequences. The {\em cyclic hyper degrees} also enjoy a structural characterization, they are the integral points contained in the union of some $n$-dimensional rectangles.
\end{abstract}

Reinforcement learning has achieved great success in areas such as Game-playing \citep{silver2018general,vinyals2019grandmaster}, robotics \cite{kober2013reinforcement}, large language models \citep{ouyang2022training}, etc.
However, due to safety concerns or physical limitations, in some real-world reinforcement learning problems, we must consider additional constraints that may influence the optimal policy and the learning process \citep{garcia2015comprehensive}.
% For example, a robotic arm must not take actions that may cause harm to itself or the environments.
A standard framework to handle such cases is the constrained Markov Decision Process (CMDP) \citep{altman1999constrained}.
Within the CMDP framework, the agent has to maximize
the expected cumulative reward while
obeying a finite number of constraints, which are usually in the form of expected cumulative cost criteria.

However, we are sometimes concerned with the problem with a continuum of constraints.
For example,
the constraints we meet might be time-evolving or subject to uncertain parameters, which
cannot be formulated as an ordinary CMDP
(see Examples \ref{Example_Time_Evolving} and  \ref{Example_Uncertain}).
In this paper we would study a generalized CMDP  
to address the above problem.  Because the constraints are not only infinite-number but also lie
in a continuous set,
the generalization is not trivial. Fortunately, we find that we can borrow the idea behind semi-infinite programming (SIP) \citep{remez1934determination, hettich1993semi} to deal with the semi-infinite constraints.
Accordingly, we propose \emph{semi-infinitely constrained Markov decision processes} (SICMDPs)
as a novel complement to the ordinary CMDP framework.
%More specifically,  an SICMDP model %, we consider 
%contains a continuum of constraints whereas an ordinary CMDP contains a finite number of constraints. 

%This generalization is natural but not trivial. However, we can brows the idea  
%The idea is quite natural and can be backtracked
%to the practice of extending linear programming to linear semi-infinite programming (LSIP) %\cite{remez1934determination, GobernaLSIO1998}.
%In addition, 
%As a complementary approach to the ordinary CMDP framework, 
%SICMDP can be used to model these problems  which cannot be described by a finite number of constraints
%that are not covered by .
%For example,
%the restrictions we consider can be time-evolving or subject to uncertain parameters
%, thus
%cannot be described by a finite number of constraints but a continuum of constraints 
%(see Examples \ref{Example_Time_Evolving} and  \ref{Example_Uncertain}).

We also present two reinforcement learning algorithms to solve SICMDPs called SI-CRL and SI-CPO, respectively.
SI-CRL is a model-based reinforcement learning algorithm designed for tabular cases, and SI-CPO is a policy optimization algorithm for non-tabular cases.
% and analyze its performance both theoretically and empirically.
The main challenge is that we need to deal with a continuum of constraints, thus reinforcement learning algorithms for ordinary CMDPs do not work anymore.
In SI-CRL, we tackle this difficulty by first transforming the reinforcement learning problem to an equivalent LSIP problem, which can then be solved using methods in the LSIP literature like the dual exchange methods \citep{Hu1990,reemtsen1998numerical}.
In SI-CPO, we resort to the idea of cooperative stochastic approximation developed in \cite{lan2020algorithms, wei2020comirror}.
As far as we know, we are the first to introduce tools from semi-infinitely programming (SIP) into the reinforcement learning community for solving constrained reinforcement learning problems.

% To the best of our knowledge, we are the first to apply tools from semi-infinitely programming (SIP) to solve reinforcement learning problems.
Furthermore, we give theoretical analysis for both SI-CRL and SI-CPO.
We decompose the error of SI-CRL into two parts: the statistical error from approximating the true SICMDP with an offline dataset and the optimization error due to the fact that the solution of the LSIP problem obtained by the dual exchange method is inexact.
On the optimization side, we show that the iteration complexity of SI-CRL is $O\left(\left\{\mathrm{diam}(Y)L\sqrt{|\gS|^2|\gA|m}/\left[(1-\gamma)\epsilon\right]\right\}^m\right)$.
On the statistical side, we show that the sample complexity of SI-CRL is $\widetilde O\left(\frac{|S|^2|A|^2}{\epsilon^2(1-\gamma)^3}\right)$ if the offline dataset is generated by a generative model, and $\widetilde O\left(\frac{|S||A|}{\nu_{\min} \epsilon^2(1-\gamma)^3}\right)$ if the dataset is generated by a probability measure $\nu$ as considered in \cite{chen2019information}.
Here $\widetilde O$ means that all logarithm terms are discarded.
For SI-CPO, things become a little more complicated because other than the statistical error and the optimization error, we also need to consider the function approximation error, which comes from imperfect policy parametrizations.
It is shown if the function approximation error can be controlled to $O(\epsilon)$ order, the iteration complexity of SI-CPO is $\widetilde{O}\left(\frac{1}{\epsilon^2(1-\gamma)^6}\right)$ and the sample complexity of SI-CPO is $\widetilde{O}(\frac{1}{\epsilon^4(1-\gamma)^{10}})$.
Here our iteration complexity bound is equivalent to a typical $\widetilde O(1/\sqrt{T})$ global convergence rate.

We perform a set of numerical experiments to illustrate the SICMDP model and validate our proposed algorithms.
Specifically, we examine two numerical examples, namely the discharge of sewage and ship route planning.
Through the discharge of sewage example, we show the advantage of the SICMDP framework over the CMDP baseline obtained by naive discretization in modeling realistic sequential decision-making problems.
Moreover, we demonstrate the effectiveness of the SI-CRL and SI-CPO algorithms in such tabular environments. 
In the ship route planning example, we illustrate the benefits of the SICMDP framework and the ability of the SI-CPO algorithm to address complex continuous control tasks involving continuous state spaces with modern deep reinforcement learning techniques.

% In summary, our contributions are listed as follows.
% First, we present the SICMDP model, which can be viewed as a generalization of the ordinary CMDP model.
% Second, we propose an algorithm to perform reinforcement learning for SICMDPs, which is called SI-CRL, and we believe that we are the first to apply tools from SIP
% to solve reinforcement learning problems.
% Third, we give a theoretical analysis of SI-CRL and identify both its sample complexity and iteration complexity.
% In addition, we perform numerical experiments to illustrate the SICMDP model and validate the SI-CRL algorithm.
% \{This paragraph can be removed!!! \}






\section{Preliminaries}\label{sec:prelims}

\subparagraph{Notations.}
For a given positive integer $k \in \mathbb{N}$, the set of integers $\{1,2,\ldots,k\}$ is denoted for short as $[k]$. Given a graph $G$, the vertex set is denoted as $V(G)$ and the edge set as $E(G)$. Given two graphs $G_1$ and $G_2$, $G_1 \cup G_2$ denotes the graph $G$ where $V(G) = V(G_1) \cup V(G_2)$ and $E(G) = E(G_1) \cup E(G_2)$. 

In this paper, a regular $n$-gon is denoted by $A_1A_2A_3...A_n$ or $B_1B_2B_3...B_n$. For convenience, we define $A_{n + 1} := A_1$, $B_{n + 1} := B_1$, $A_0 := A_n$ and $B_0 := B_n$. We use the notation $\{A_i\}$ to denote the polygon $A_1A_2A_3 \ldots A_n$ and $\{B_i\}$ to denote the polygon $B_1B_2B_3 \ldots B_n$. For any regular polygon $A_1A_2A_3...A_n$, the circumcircle of the polygon is denoted as $(A_1A_2A_3...A_n)$. Given any $n$-vertex polygon in the Euclidean plane with vertices $\mathcal P = P_1P_2P_3\ldots P_n$, and interval in $\mathcal{K}$ is a subset of consecutive vertices $P_iP_{i+1\ldots P_j}$, $i,j\in [n]$, also denoted as $[P_i,P_j]$. Here $P_i$ is considered the starting vertex of the interval and $P_j$ the ending vertex. For any $P_k$, $i \leq k\leq j$ in the interval we will also use the notation $P_i \leq P_k \leq P_j$.

Given two points $P$, $Q$ in the Euclidean plane, we denote by ${\sf dist}(P,Q)$ the Euclidean distance between $P$ and $Q$. Given a line segment $AB$ in the Euclidean plane, $\overline{AB} = {\sf dist}(A,B)$. For two distinct points $A$ and $B$, $L_{AB}$ denotes the line containing $A$ and $B$; and $\overrightarrow{AB}$ denotes the ray originating from $A$ and containing $B$. 

When we refer to a graph $\mathcal{G}$ in the Euclidean plane then $V(\mathcal{G})$ is a set of points in the Euclidean plane, and $E(\mathcal{G})$ is a subset of the family of line segments $\{P_1P_2 | P_1,P_2 \in V(\mathcal{G})\}$. For any tree $\mathcal T$ in the Euclidean plane, we denote by the notation $|\mathcal T|$ the value of $\Sigma_{e \in E(\mathcal T)} \overline{e}$. A path in a tree $\mathcal T$ is uniquely specified by the sequence of vertices on the path; therefore, $P_1$, $P_2$, $P_3$, \ldots, $P_k$ (where $P_i \in V(\mathcal T), \forall i \in [k]$ and $P_iP_{i+1} \in E(\mathcal T), \forall i \in [k-1]$) denotes the path starting from the vertex $P_1$, going through the vertices $P_2$, $P_3$, \ldots, $P_{k-1}$ and finally ending at $P_k$. Equivalently, we can specify the same path as \emph{the path from $P_1$ to $P_k$}, since $\mathcal T$ is a tree. Consider the graph $T$ such that $V(T) = \{v_P| P \in V(\mathcal{T})\}$, $E(T) = \{v_{P_1}v_{P_2}| P_1P_2 \mbox{ is a line segment in } E(\mathcal{T})\}$. Then $T$ is said to be the topology of $\mathcal{T}$ while $\mathcal{T}$ is said to realize the topology $T$. Given two trees $\mathcal{T}_1$, $\mathcal{T}_2$ in the Euclidean plane, $\mathcal{T}' = \mathcal{T}_1\cup \mathcal{T}_2$ is the graph where $V(\mathcal{T}')= V(\mathcal{T}_1) \cup V(\mathcal{T}_2)$ and $E(\mathcal{T}')= E(\mathcal{T}_1) \cup E(\mathcal{T}_2)$. 

Given any graph $G$, a Steiner minimal tree or SMT for a terminal set $\mathcal{P} \subseteq V(G)$ is the minimum length connected subgraph $G'$ of $G$ such that $\mathcal{P} \subseteq V(G')$. The {\sc Steiner Minimal Tree} problem on graphs takes as input a set $\mathcal{P}$ of terminals and aims to find a minimum length SMT for $\mathcal{P}$. For the rest of the paper, we also refer to a Euclidean Steiner minimal tree as an SMT. Given a set of points $\mathcal{P}$ in the Euclidean plane, the convex hull of $\mathcal{P}$ is denoted as $\mathrm{CH(\mathcal P)}$.

\subparagraph{Euclidean Minimum Spanning Tree (MST).}
Given a set $\mathcal P$ of $n$ points in the Euclidean plane, let $G$ be a graph where $V(G) = \{v_P| P \in \mathcal P\}$ and $E(G) = \{v_{P_i}v_{P_j} | P_i,P_j \in \mathcal P\}$. Also, a weight function $w_{G}: E(T) \rightarrow \mathbb{R}$ is defined such that for each edge $v_{P_1}v_{P_2} \in E(T)$, $w_{G}(v_{P_1}v_{P_2}) = \overline{P_1P_2}$. The Euclidean minimum spanning tree of a set $\mathcal P$ is the minimum spanning tree of the graph $G$ with edge weights $w_G$. Note that a Steiner tree may have shorter length than a minimum spanning tree of the point set $\mathcal P$. 

In the plane, the Euclidean minimum spanning tree is a subgraph of the Delaunay triangulation. Using this fact, the Euclidean minimum spanning tree for a given set of points in the Euclidean plane can be found in $\OO(n\log n)$ time as discussed in \cite{Shamos1975ClosestpointP}. 

\subparagraph{Properties of a Euclidean Steiner minimal tree.}
A Euclidean Steiner minimal tree (SMT) has certain structural properties as given in~\cite{cockayne1967steiner}. We state them in the following Proposition.

\begin{proposition}\label{smt-prop}
Consider an SMT on $n$ terminals.
 \begin{enumerate}
   \item No two edges of the SMT intersect with each other.
 
   \item Each Steiner point has degree exactly $3$ and the incident edges meet at $120^\circ$ angles. The terminals have degree at most $3$ and the incident edges form angles that are at least $120^\circ$.
  
   \item The number of Steiner points is at most $n-2$, where $n$ is the number of terminals.

\end{enumerate}
\end{proposition}

 A full Steiner tree (FST) is a Steiner tree (need not be minimal, but may include Steiner points) having exactly $n-2$ Steiner points, where $n$ is the number of terminals. In an FST, all terminals are leaves and Steiner points are interior nodes. When the length of an FST is minimized, it is called a minimum FST.

All SMTs can be decomposed into FST components such that, in each component a terminal is always a  leaf. This decomposition is unique for a given SMT~\cite{hwang1992steiner}. A topology for an FST is called a full Steiner topology and that of a Steiner tree is called a Steiner topology.


%For a tree $\mathcal T$, we would denote the set of vertices (the terminal vertices and the Steiner points) as $V(\mathcal T)$ and the set of edges as $E(\mathcal T)$. Similarly for a topology $T$, $V(T)$ and $E(T)$ denote vertex set (the terminal vertices and the Steiner points) and the edge set respectively. \todo{\color{white}Anubhav: Added this defition}

\subparagraph{Steiner Hulls.}
A Steiner hull for a given set of points is defined to be a region which is known to contain an SMT. We get the following propositions from~\cite{hwang1992steiner}.

\begin{proposition}\label{convex-steiner}
    For a given set of terminals, every SMT is always contained inside the convex hull of those points. Thus, the convex hull is also a Steiner hull.
\end{proposition}

The next two propositions are useful in restricting the structure of SMTs and the location of Steiner points.

\begin{proposition} [The Lune property]\label{lune}
    Let $\rm UV$ be any edge of an SMT. Let $L(\rm{U},\rm{V})$ be the lune-shaped intersection of circles of radius $|\rm UV|$ centered on $\rm U$ and $\rm V$. No other vertex of the SMT can lie in $L(\rm{U},\rm{V})$, except $U$ and $V$ themselves.
\end{proposition}

\begin{proposition} [The Wedge property]\label{wedge}
    Let $W$ be any open wedge-shaped region having angle $120^\circ$ or more and containing none of the points from the input terminal set $\mathcal P$. Then $W$ contains no Steiner points from an SMT of $\mathcal P$.
\end{proposition}

\subparagraph{Approximation Algorithms.}
We define all the necessary terminology required in terms of a minimization problem, as ESMT is a minimization problem.
%\begin{definition} [Approximation Factor for a Minimization Problem]
%    Let $\mathcal{P}$ be a minimization problem. An algorithm $\mathcal{A}$ for the problem $\mathcal{P}$ is called an $\alpha$ factor approximation algorithm if, for every instance $\Pi$ of $\mathcal{P}$, we have $\rm{ALG}(\Pi) \leq \alpha \rm{OPT}(\Pi)$ where $\rm{ALG}(\Pi)$ and $\rm{OPT}(\Pi)$ are the values of the output of the algorithm and optimal solution for the instance $\Pi$ respectively. $\alpha$ can be a constant or a function of the input size $n$, and is always at least $1$.
%\end{definition}

%\begin{definition} [Polynomial Time Approximation Scheme (PTAS)]
%    An algorithm is called a polynomial time approximation scheme (PTAS) for a problem if it takes an input instance and a parameter $\epsilon > 0$, and outputs a solution with approximation factor $(1+\epsilon)$ for a minimization problem in time $\OO(n^{f(1/\epsilon)})$ where $n$ is the input size and $f(1/\epsilon)$ is any computable function.
%\end{definition}

\begin{definition} [Efficient Polynomial Time Approximation Scheme (EPTAS)]
    An algorithm is called an efficient polynomial time approximation scheme (EPTAS) for a problem if it takes an input instance and a parameter $\epsilon > 0$, and outputs a solution with approximation factor $(1+\epsilon)$ for a minimization problem in time $f(1/\epsilon)n^{\OO(1)}$ where $n$ is the input size and $f(1/\epsilon)$ is any computable function.
\end{definition}

\begin{definition} [Fully Polynomial Time Approximation Scheme (FPTAS)]
    An algorithm is called a fully polynomial time approximation scheme (FPTAS) for a problem if it takes an input instance and a parameter $\epsilon > 0$, and outputs a solution with approximation factor $(1+\epsilon)$ for a minimization problem in time $(1/\epsilon)^{\OO(1)}n^{\OO(1)}$ where $n$ is the input size.
\end{definition}

% appending preliminaries.tex --- Anubhav 

\section{Cyclic permutations and Binary Tables}
In this section, we will be working with binary tables of size $2^n \times n$ and study the action of cyclic permutations on the columns of the table. For a given number $n \in \mathbb{Z}_+$, list out the binary expansion of numbers in increasing order from $\{0,\dots,2^n-1\}$ as rows in a table. We pad the binary expansion with sufficient numbers of zeros on the left to make the length of each row exactly equal to $n$.
For example, when $n=3$, the table is as given in Table~\ref{table:bitlist3}.

To state it formally we need the following definitions. Given a number $m$ we denote the $i^{th}$  bit in its binary representation by $\rm{bin}(m,i)$. If the most significant bit in the binary expansion occurs at the $s^{th}$-position in the binary expansion of $m$, then for all values of $i>s$ the value of $\rm{bin}(m,i)$ is zero. For example, $\rm{bin}(4,2)=1$ and $\rm{bin}(4,i)=0$ for every $i\geq 3$.

For a given $n$, we construct $n$ lists $c_{1,n}, \dots, c_{n,n}$, with each list $c_{i,n}$ having length equal to $2^n$. 
For $n=2$, we have $c_{1,2} = (0,1,0,1)$ and $c_{2,2} = (0,0,1,1)$.
For $n=3$, the lists $c_{1,3},c_{2,3},c_{3,3}$ correspond to the columns of the Table~\ref{table:bitlist3}. %For each $j\in[n]$, the $i^{th}$ value in the list of $c_{j,n}$ is $\rm{bin}(i,j)$ for $i \in \{0, \ldots, 2^n-1\}$.
Formally, the lists are defined as follows.

\begin{definition}[$n$-Bit-Lists]
 For a given $n$, we define $n$-Bit-List to consist of $n$ lists $c_{1,n}, \dots, c_{n,n}$. For $j\in [2^n]$, the value of $c_{i,n}(j)$ is equal to $\rm{bin}(j-1,i)$.
\end{definition}

\begin{definition}[$n$-Bit-Table]
 For a positive integer $n$, the $n$-Bit-Table $T_n$ is defined to be a size $2^n\times n$ table with $T_n=[c_{n,n},c_{n-1,n},\dots,c_{1,n}]$. 
\end{definition}

The lists $c_{1,n}, \dots, c_{n,n}$ have a nice recursive structure and can be generated in an alternative way by concatenation.
Given a positive integer $n$, the base case of $n=1$ is one list $c_{1,1}=(0,1)$.
For $n\geq 2$, the tuple $c_{n,n}=0_{\times 2^{n-1}}\cdot 1_{\times 2^{n-1}}$ and for $j\in [n-1]$, the list $c_{j,n}$ is equal to the concatenated list $c_{j,n-1}\cdot c_{j,n-1}$. Thus, we get the following observation about the lists.

\begin{observation}
 \label{obs:zero-one-power}
 For $n\in\mathbb{Z}_+$ and $i\in [n]$, we have
  $c_{i,n}= (0_{\times 2^{i-1}} \cdot 1_{\times 2^{i-1}})_{\times 2^{n-i}}$.
\end{observation}

Let $\mathcal{C}_n$ be the set of all cyclic permutations on an $2^n$ length list.
Let $\Pi \subseteq \mathcal{C}_n$ be a multi-set consisting of $n$ arbitrary cyclic permutations $\pi_1, \ldots, \pi_n$.
Let $\Pi(T_n)=[\pi_n(c_{n,n}),\dots,\pi_1(c_{1,n})]$ be the new table obtained from $T_n$. 
For clarity we will change the notation slightly, let $\Pi(T_n,i)$ denote row $i$ of $\Pi(T_n)$, it is a length $n$ binary tuple
$(c_{n,n}(\pi_n(i)),  c_{n-1,n}(\pi_{n-1}(i)), \ldots, c_{1,n}(\pi_1(i)))$,
for $i \in [2^n]$.

We next state a general lemma whose proof follows by induction over the order of a cyclic permutation.

\begin{lemma}
\label{lem:halfcycle}
Let $L$ be a list and $\pi$ be any cyclic permutation, then $\pi(L\cdot L)= \pi(L)\cdot \pi(L).$
\end{lemma}

Recall that for $j\in [n-1]$, the tuple $c_{j,n}$ consists of two copies of $c_{j,n-1}$ concatenated together. Combining this fact with Lemma~\ref{lem:halfcycle}, we obtain the following as a special case.

\begin{corollary}
\label{cor:halfcycle}
 Given a positive integer $n$ and a cyclic permutation $\pi \in \mathcal{C}_n$.
 For $j \in [n-1]$, the tuple $\pi(c_{j,n})$ is equal to $\pi(c_{j,n-1}) \cdot \pi(c_{j,n-1})$.
\end{corollary}

We next prove that for any set of cyclic permutations $\Pi$, any two rows in $\Pi(T_n)$ will never become equal.

\begin{theorem}\label{thm:rotatebits}
For any list of $n$ cyclic permutations $\Pi$, the rows of $\Pi(T_n)$ are pair-wise distinct.
\end{theorem}
\begin{proof}
 We prove this by induction on $n$. For the base case $n=1$, the statement is  trivially true. For the rest of the proof, assume that the list of cyclic permutations is $\Pi=(\pi_1,\ldots,\pi_n)$ and $\Pi_{\overline{n}}$ is used to denote the list $(\pi_1,\dots,\pi_{n-1})$.
 %Even though a permutation $pi_i$ is a cyclic permutation acting on a $2^{n}$ length list, it is easy to see how the 
 
% By the induction hypothesis the table $\Pi(T_{n-1})$ does not contain any repeated row. 
The table $T_n$ can be constructed recursively by taking two copies of $T_{n-1}$ and appending the rows of one below the other, after that we add $0_{\times 2^{n-1}}\cdot 1_{\times 2^{n-1}}$ as the first column. Consider the table 
\begin{align*}
T_{\overline n} &= [c_{n-1,n},\dots,c_{1,n}] \\
             &= [c_{n-1,n-1}\cdot c_{n-1,n-1},\dots, c_{1,n-1}\cdot c_{1,n-1}].
\end{align*}
 Apply the list of permutations $\Pi_{\overline{n}}$ on $T_{\overline n}$ to get
\begin{align*}
 \Pi_{\overline{n}}(T_{\overline n}) &= [\pi_{n-1}(c_{n-1,n}), \ldots, \pi_1(c_{1,n})]\\
 			&= [\pi_{n-1}(c_{n-1,n-1})\cdot \pi_{n-1}(c_{n-1,n-1}),\dots, \pi_1(c_{1,n-1})\cdot \pi_1(c_{1,n-1})],
\end{align*}
 where the second equality follows from Corollary~\ref{cor:halfcycle}.
% $\Pi_{\overline{n}}$ are.
 By the induction hypothesis, the first row-wise half of $\Pi_{\overline{n}}(T_{\overline n})$, which is the same as $\Pi_{\overline{n}}(T_{n-1})$, consists of distinct rows. 
 Therefore, the table $\Pi_{\overline{n}}(T_{\overline n})$ consists of rows which are repeated exactly twice. For $i < j$, the rows $\Pi_{\overline{n}}(T_{\overline n},i)$ and
 $\Pi_{\overline{n}}(T_{\overline n},j)$ are equal when $j=i+2^{n-1}$.
Therefore, it suffices to prove that the rows  $\Pi(T_n,i)$ and $\Pi(T_n,i+2^{n-1})$ are distinct.
 Observe that to obtain the table $\Pi(T_n)$, we need to append $\pi_n(c_{n,n})$ as the first column in $\Pi_{\overline{n}}(T_{\overline n})$. As $c_{n,n}$ is equal to $0_{\times 2^{n-1}}\cdot 1_{\times 2^{n-1}}$, we have $c_{n,n}(i)\neq c_{n,n}(i+2^{n-1})$, this implies that $c_{n,n}(\pi_n(i))\neq c_{n,n}(\pi_n(i+2^{n-1}))$.
\end{proof}

Next, we define a notion of {\em cyclic hyper degrees} as follows.

\begin{definition}[Cyclic Hyper Degree]\label{def:chd}
 Given $\Pi$, a list of $n$ cyclic permutations, a tuple $d \in \mathbb{Z}_+^n$ is said to be a {\em cyclic hyper degree} if there exist $i, N \in [2^n]$ such that $d=\sum_{k=i}^N \Pi(T_n, k).$
\end{definition}

As the rows of $\Pi(T_n)$ are distinct, their contiguous sum  is in $H_n$ by definition. This gives us the main theorem of this section.

\begin{theorem}
 If $w\in \mathbb{Z}_+^n$ is a {\em cyclic hyper degree}, then $w\in H_n$.
\end{theorem}

We note that $(4,1,1,1)$ is a {\em realizable} hypergraph degree sequence but it is not a {\em cyclic hyper degree} sequence.
In the next section, we will show how to efficiently check if a given sequence $d$ is a {\em cyclic hyper degree}. 

\section{Efficiently Recognizing {\em Cyclic Hyper Degrees}}
This section consists of two parts. In the first part will culminate with Theorem~\ref{thm:shiftrange} which provides an efficiently computable closed form formula for the range of values taken by contiguous sum of $N$ elements in a list $c_{i,n}$, for any $i$. In the second part we will show how to use Theorem~\ref{thm:shiftrange} to decide if a given degree sequence is a {\em cyclic hyper degree}.

The elements in the columns of $T_n$ do not change their relative position after application of a cyclic permutation when seen as a cyclic list. We shall use this property to efficiently search for possible bit subsets which may sum up to a given input degree sequence.

\subsection{Contiguous Sum of Bit Lists}

\begin{definition}[Contiguous Sum]
\label{def:csum}
 Given a list $L$ of length $m$, the contiguous sum of $N$ elements in $L$ starting at the index $i\in [m]$ is defined to be
 $$\mathcal{S}(L,i,N):=\sum_{j=0}^{N-1} L(1+ ((i+j-1)\mod m)).$$
\end{definition}

The summation above treats the list $L$ as a cyclic list. Next, we prove that the contiguous sum function is a `continuous' function, this property will allow us to specify the range of sum by stating the minimum and the maximum value taken by it.
 Note that if $L$ is a $0$-$1$ list, for any index $\ell \in [m]$, 
 we have $\vert \mathcal{S}(L,\ell,N)-\mathcal{S}(L,\ell+1,N) \vert \in \{0,1\}$. This fact gives us the following property.

\begin{observation}
\label{obs:continuity}
 Let $L$ be a size $m$ list having $0$-$1$ entries and $N\in \mathbb{Z}_+$.
 If $v_i=\mathcal{S}(L,i,N)$ and $v_j=\mathcal{S}(L,j,N)$, for some $i,j\in[m]$, then 
 for every $v\in \mathbb{Z}_+$ contained between  $v_i$ and $v_j$
 there exists a $k\in [m]$  such that $\mathcal{S}(L,k,N)=v$.
\end{observation}

As the lists $c_{i,n}$ are over $0$-$1$ we get an easy relation between the maximum and minimum values taken by the contiguous sum as follows.

\begin{lemma}
\label{lem:sumcomplement}
  Let  $j\in\{0,\dots,n\}$, $i \in [n]$ and $N\in[2^n]$. The minimum of the sum of $N$ contiguous bits in a bit list $c_{i,n}$ is $m$ if and only if its maximum is $N-m$.
\end{lemma}
\begin{proof}
 Let $\overline{c}_{i,n}$ be the bit list obtained from the list $c_{i,n}$ by flipping each zero to one and vice versa. Let $\sigma_{2^{i-1}}$ be an order $2^{i-1}$ cyclic permutation, observe that $\overline{c}_{i,n}$ is equal to $\sigma_{2^{i-1}}(c_{i,n})$.
 If the minimum value is obtained at the contiguous segment which starts at the index $j$ in $c_{i,n}$, then the value $N-m$ can be obtained by the contiguous sum starting at index $\sigma_{2^{i-1}}(j)$. Finally, note that $m$ is the minimum value if and only if $N-m$ is the maximum value.
\end{proof}

Combining Observation~\ref{obs:continuity} and Lemma~\ref{lem:sumcomplement} we get the following.

\begin{lemma}
\label{lem:minmax}
 Let $N\in[2^n]$ and $m=\min_{j \in [2^n]} \mathcal{S}(c_{i,n}, j, N)$. For every value $v$ in the range $\{m,\dots, N-m\}$ there exists a $j\in [2^n]$ such that $\mathcal{S}(c_{i,n}, j, N)=v$.
\end{lemma}

The lemma above allows us to find the range of values taken by the contiguous sum by just finding the minimum value taken by it.
Next we prove a simpler lemma about the range of values taken. Using that, in Theorem~\ref{thm:shiftrange}, we will find the range of values taken by the contiguous sum of $N$ elements in any list $c_{i,n}$.

\begin{lemma}
\label{lem:shiftpower}
 For  $j\in\{0,\dots,n\}$ and $i \in [n]$, the sum of $2^j$ contiguous bits in a bit list $c_{i,n}$ takes the following values.
\begin{enumerate}
\item
\label{enum:shiftpower-one}
If $j \leq (i-1)$, then the range is $\{0, \dots,2^j \}$, and
\item
\label{enum:shiftpower-two}
If $j\geq i$, then the sum is exactly $2^{j-1}$.
\end{enumerate}
\end{lemma}
\begin{proof}
 By Lemma~\ref{lem:minmax}, it suffices to find the minimum value of contiguous sum function. Notice that we have, $c_{i,n}= (0_{\times 2^{i-1}} \cdot 1_{\times 2^{i-1}})_{\times 2^{n-i}}$, by Observation~\ref{obs:zero-one-power}.
 \begin{enumerate}
 \item
 When $j\leq (i-1)$, we can pick a block of $2^j$ zeros giving a total of zero, which is the minimum possible value.
 \item
 When $j\geq i$, let $L_k$ be a list of $2^j$ contiguous bits of $c_{i,n}$ starting at the index $k$ in $c_{i,n}$. To prove that $\sum L_k = \sum L_{k+1} $, it suffices to show that $c_{i,n}(k)=c_{i,n}(k+2^j)$. Rewriting
 $c_{i,n}=((0_{\times 2^{i-1}}\cdot 1_{\times 2^{i-1}})_{\times 2^{j-i}})_{\times 2^{n-j}}$ shows that any two indices with difference equal to $2^j$ store the same value. As the choice of $k$ was arbitrary, the contiguous sum is equal to $2^{j-1}$.
 \end{enumerate} 
\end{proof}

\begin{theorem}
\label{thm:shiftrange}
 For $i \in [n]$, $N\in[2^{n}]$ and $p=2^i$, the sum of $N$ contiguous bits in a bit list $c_{i,n}$ takes values in the range, ${\rm range}(i,N) \triangleq$
 $$ 
 \left\{
	\floor[\Big]{\frac{N}{p}} \frac{p}{2} + \max \left( (N\mod p) - \frac{p}{2}, 0 \right),
  	\cdots,
   	\floor[\Big]{\frac{N}{p}} \frac{p}{2} + \min \left( N\mod p, \frac{p}{2} \right)
 \right\}.
 $$
\end{theorem}
\begin{proof}
 For a fixed $i\in[n]$ consider the list $c_{i,n}$. Assuming that the minimum value of the range is as claimed, by Lemma~\ref{lem:minmax}, the maximum value is
\begin{align*}
   \max_{j \in [2^n]} \mathcal{S}(c_{i,n}, j, N)
   &= N-\min_{j \in [2^n]} \mathcal{S}(c_{i,n}, j, N)\\
   &=N- \left( \floor[\Big]{\frac{N}{p}} \frac{p}{2} + \max \left( (N\mod p) - \frac{p}{2}, 0 \right)\right)\\
 &= \floor[\Big]{\frac{N}{p}}p + (N\mod p) - 
 \left( 
 \floor [\Big] {\frac{N}{p}} \frac{p}{2} + \max \left( (N\mod p) - \frac{p}{2}, 0 \right)
 \right)	\\
 &= \floor[\Big]{\frac{N}{p}} \frac{p}{2} +(N\mod p) - 
 \max	\left(		(N \mod p) - \frac{p}{2}, 	0 	\right)	\\
 &= \floor[\Big]{\frac{N}{p}} \frac{p}{2} + \min \left( (N\mod p)-(N \mod p) + \frac{p}{2}, N\mod p\right)\\
 &=  \floor[\Big]{\frac{N}{p}} \frac{p}{2} + \min \left(\frac{p}{2}, N\mod p\right).
\end{align*}
 
 As proved in case~\ref{enum:shiftpower-two} of Lemma~\ref{lem:shiftpower}, the sum of $\floor{\frac{N}{p}}p$ contiguous bits is equal to $\floor{\frac{N}{p}}\frac{p}{2}$  irrespective of the starting index.
 Therefore, it suffices to find the minimum sum of $R=(N\mod 2^i)$ contiguous bits. Let $L_k$ be a list of $R$ bits occurring contiguously in $c_{i,n}$ starting at index $k$. If the first bit of $L_k$ is $1$, then $\sum L_{k+1}\leq \sum L_k$. Therefore, we can keep on increasing the value of $k$ until the first bit is zero,  without increasing the value of the contiguous sum. On the other hand, if $c_{i,n}(k-1)=0$, then $\sum L_{k-1}\leq \sum L_k$. Therefore, we can keep on decreasing the value of $k$ one at a time until $c_{i,n}(k-1)=1$, without increasing the value of the contiguous sum. Thus the minimum value of the contiguous sum is achieved when the index $k$ points to the start of any block $0_{\times 2^{i-1}}$ contained in $c_{i,n}$. The value of minimum is  $\max(R-\frac{p}{2},0)$ as the ones start appearing after $\frac{p}{2}$ indices from the start of a list $0_{\times 2^{i-1}}\cdot 1_{\times 2^{i-1}}$.  Adding it to $\floor{\frac{N}{p}}\frac{p}{2}$ gives the required minimum value.
\end{proof}

\subsection{Algorithm}
We next state a theorem which gives an equivalent definition of {\em cyclic hyper degrees}.
\begin{theorem}
\label{thm:chd}
A list $w=\{w_1,\dots,w_n\}\in \mathbb{Z}_+^n$ is a {\em cyclic hyper degree} if and only if there exist $N\in[2^n]$ and a permutation $\pi$, such that for each $i\in [n]$, $w_{\pi(i)} \in {\rm range}(i,N)$.% (see Theorem~\ref{thm:shiftrange}).
\end{theorem}

\begin{proof}
 Forward direction is a direct consequence of the definition.
 
 Using Definitions~\ref{def:chd} and~\ref{def:csum}, we get that there exist numbers $s_1,\dots,s_n \in [2^n]$ such that for each $i\in [n]$, we have $w_{\pi(i)}=\mathcal{S}(c_{i,n},s_i,N)$. Let $\Pi^{-1}=(\sigma_{s_1}^{-1},\dots,\sigma_{s_n}^{-1})$ be the list of cyclic permutations, where for each $i\in [n]$, $\sigma_{s_i}^{-1}$ is the inverse of the cyclic permutation of order $s_i$. Consider the table $\Pi^{-1}(T_n)$, by
 Theorem~\ref{thm:rotatebits}, all its rows are distinct. In particular, the first $N$ rows are distinct and their sum is $\pi(w)$. Finally, $w\in H_n$ if and only if $\pi(w)\in H_n$.
\end{proof}

Theorem~\ref{thm:shiftrange} gives us a way to efficiently find the number of bits in a contiguous sum of $N$ bits. If we know the number of distinct bit sequences that can sum up to a given vector $w\in \mathbb{Z}_+^n$, then using Theorem~\ref{thm:shiftrange} we can generate all the possible ranges of values which can be taken by each coordinate of the sum. Finally, we need to check if each coordinate of $w$ is contained in different ranges, this corresponds to finding the permutation $\pi$ in Theorem~\ref{thm:chd}. In the next lemma, we will find the number of possible  distinct bit-sequences which can sum up to a given $w$ using cyclic shifts, this corresponds to finding $N$ in Theorem~\ref{thm:chd}.
\begin{lemma}
\label{lem:set-size}
If $w=\{w_1,\dots,w_n\}\in \mathbb{Z}_+^n$ is a {\em cyclic hyper degree}, then the number of bit sequences which sum up to $w$ is an element of the set
$$\mathcal{N}_w\triangleq \{2w_i+j~:~i\in[n], j\in \{-1,0,1\}\}.$$
\end{lemma}
\begin{proof}
 As one of the coordinates of $w$, say $w_k$, is the contiguous sum of $c_{1,n}$, we need to find the number of bits which sum up to $w_k$. From the structure of $c_{1,n}$, it is easily seen that there are just three values viz. $2w_k-1,2w_k,2w_k+1$ which contain $w_k$ in their range of sums. Conversely, for any number $x$ not contained in $\{2w_i+j~:~i\in[n], j\in \{-1,0,1\}\}$, the sum of $x$ contiguous bits $c_{1,n}$ will not contain any of $w_i$, for $i\in [n]$.
\end{proof}

\begin{lemma}
\label{lem:embed}
 Given $w\in\mathbb{Z}_+^n$ and a list of integer intervals $R_1,\dots,R_n \subset \mathbb{Z}_+^2$. There exists an algorithm running in time polynomial in $n$ which correctly answers if there exists a permutation $\pi$ such that  for each $i\in [n]$, $w_{\pi(i)}\in R_i$. 
\end{lemma}
\begin{proof}
 Construct a bipartite graph $G=(A,B,E)$ on $2n$ vertices. Let $A=B=[n]$ and $(i,j)\in E$ if and only if $w_i \in R_j$. Using a polynomial time algorithm one can find if there exists a perfect matching in $G$. If there is a perfect matching then the answer is \YES, otherwise it is \NO.
\end{proof}

\begin{theorem}
 \label{thm:chd-poly}
 There is a polynomial time algorithm in $n$ which decides if a given $w\in\mathbb{Z}_+^n$ is a {\em cyclic hyper degree}.
\end{theorem}

\begin{proof}
 For each $N\in\mathcal{N}_w$, given by Lemma~\ref{lem:set-size}, and $i\in[n]$ compute ${\rm range}(i,N)$ as given by Theorem~\ref{thm:shiftrange}. Now, use Lemma~\ref{lem:embed} on these ranges of numbers and decide if $w$ is a {\em cyclic hyper degree}, if it is not then try the next number from the set $\mathcal{N}_w$. If it succeeds for at least one element of $\mathcal{N}_w$, we answer \YES, otherwise we answer \NO. Finally, note that $\vert N_w \vert\leq 3n$ and all the other steps can be performed in time which is a polynomial function of $n$. 
\end{proof}
% \vspace{-0.20in}
%\subsection{Analysis:}


\textbf{Use of Multiple Projection Heads:} The use of different projection heads for each view on OpenImages classification gives us a boost of $1.1$ mAP on Obj-Obj+Dilate crop. Pre-training on COCO and finetuning on VOC dataset for object-detection task gives a boost of $0.4$ mAP. Hence using multiple projection heads results in a consistent improvement. 

\textbf{Varying Dilation Parameter:} Table 3 (appendix) shows the effect of varying the dilation parameter. A sweet spot exists at a moderate dilation value of $\delta=0.1$ for COCO object detection. 

% \textbf{Computational Cost:} BING adds negligible time to the pre-training. Generating object proposals takes ~29 mins for the full OHMS dataset (one-time cost) and ~16 mins for COCO. Instead of pre-generating, adding the BING operator to the data loader pipeline has a trivial overhead (+$0.1\%$). %As an example, the wall-clock time taken for 1 epoch of training is 1'46'' for the Dense-CL baseline and 1'45'' for our method.
% \textbf{}



%Between two views, we measure the number of common pixels; and then measure the fraction of these common pixels that overlap with a ground truth bounding box (object). We find that this fraction for COCO is $99\%$ for object-scene crops and $92.1\%$ for the scene-scene crop. In the case of OpenImages-Subset, the numbers are, respectively, $99.1\%$ and $87.3\%$. This is another way of seeing that OpenImages-Subset can benefit more from object-scene crops, borne out by the numbers in Tables \ref{tab:ssl_comparison_classification} and \ref{tab:coco_detection}. 


% \as{Shlok: could you please make this description a little better and clear?}
% We find the overlapping pixels between two crops ($C_{int} = C_1 \cap C_2$). Next we calculate intersection of $C_{int}$  with the most overlapping ground truth object ($O$) and calculate the score $\frac{C_{int} \cap O}{C_{int}}$ for each image and average it. 
% To do this, we calculate the \% intersection of the most overlapping ground truth object with the inter
% Next we try to find the probability of an actual ground truth object co-occuring in between two crops. We find  object-overlap between both Scene-Object crops and Scene-Scene crops. To do this we firstly calculate the overlapping region between two crops. Overlapping region is the area of overlap between two crops before the resize operation. Then for all the ground truth objects present in the original image  we find the object with maximum overlap in the overlapping region. Intuitively for a object to have high overlap, the object should be present in both the crops. 

% Similarly instead of taking an crop with maximum overlap we calculate average of all the crops that are present in the image. We find this average probability to be 65.12 \% for Object-Scene crop and 73.47 \% for Scene-Scene crop. 
% This is consistent with the findings of the InfoMin \cite{tian2020contrastive} that there is a tradeoff between how much information views can share.  

% Similarly in the case of OpenImages we can see from Fig \ref{fig:radius_openimages} that as we increase the radius of the object-object crops the performance firstly increases and then decreases, suggesting there is a sweet point on mutual information on OpenImages dataset as well.
% \\

% \textbf{Performance on 5 classes per image images?}










\begin{comment}
\begin{figure}
\includegraphics[width=\linewidth]{figs/beyond_tss_lesion.pdf}
\caption[]{End-to-End runtime lesion study of the entire MNIST dataset and the FMA featurized music dataset. Each of DROP's contributions provides a runtime improvement.}
\label{fig:beyond_lesion}
\end{figure}
\end{comment}



\section{Conclusion}
\label{sec:conclusion}

Advanced data analytics techniques must scale to rising data volumes. 
DR techniques offer a powerful toolkit when processing these datasets, with PCA frequently outperforming popular techniques in exchange for high computational cost. 
In response, we propose DROP, a new dimensionality reduction optimizer. 
DROP combines progressive sampling, progress estimation, and online aggregation to identify high quality low dimensional bases via PCA without processing the entire dataset by balancing the runtime of downstream tasks and achieved dimensionality. 
Thus, DROP provides a first step in bridging the gap between quality and efficiency in end-to-end DR for downstream \red{analytics}. 

%We revisit canonical operators for time series dimensionality reduction and the measurement study of~\cite{keogh-study}, and show that PCA is more effective than popular alternatives in the data mining literature often by a margin of over $2\times$ on average on gold-standard time series benchmark data sets with respect to output data dimension. More surprisingly, we empirically demonstrate that a small number of samples are sufficient to accurately characterize directions of maximum variance and obtain a high-quality low-dimensional transformation.




%\bibliographystyle{splncs03}
\bibliographystyle{plainurl}% the recommended bibstyle
\bibliography{ref}

\end{document}  