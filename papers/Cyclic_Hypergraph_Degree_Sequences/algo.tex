\section{Efficiently Recognizing {\em Cyclic Hyper Degrees}}
This section consists of two parts. In the first part will culminate with Theorem~\ref{thm:shiftrange} which provides an efficiently computable closed form formula for the range of values taken by contiguous sum of $N$ elements in a list $c_{i,n}$, for any $i$. In the second part we will show how to use Theorem~\ref{thm:shiftrange} to decide if a given degree sequence is a {\em cyclic hyper degree}.

The elements in the columns of $T_n$ do not change their relative position after application of a cyclic permutation when seen as a cyclic list. We shall use this property to efficiently search for possible bit subsets which may sum up to a given input degree sequence.

\subsection{Contiguous Sum of Bit Lists}

\begin{definition}[Contiguous Sum]
\label{def:csum}
 Given a list $L$ of length $m$, the contiguous sum of $N$ elements in $L$ starting at the index $i\in [m]$ is defined to be
 $$\mathcal{S}(L,i,N):=\sum_{j=0}^{N-1} L(1+ ((i+j-1)\mod m)).$$
\end{definition}

The summation above treats the list $L$ as a cyclic list. Next, we prove that the contiguous sum function is a `continuous' function, this property will allow us to specify the range of sum by stating the minimum and the maximum value taken by it.
 Note that if $L$ is a $0$-$1$ list, for any index $\ell \in [m]$, 
 we have $\vert \mathcal{S}(L,\ell,N)-\mathcal{S}(L,\ell+1,N) \vert \in \{0,1\}$. This fact gives us the following property.

\begin{observation}
\label{obs:continuity}
 Let $L$ be a size $m$ list having $0$-$1$ entries and $N\in \mathbb{Z}_+$.
 If $v_i=\mathcal{S}(L,i,N)$ and $v_j=\mathcal{S}(L,j,N)$, for some $i,j\in[m]$, then 
 for every $v\in \mathbb{Z}_+$ contained between  $v_i$ and $v_j$
 there exists a $k\in [m]$  such that $\mathcal{S}(L,k,N)=v$.
\end{observation}

As the lists $c_{i,n}$ are over $0$-$1$ we get an easy relation between the maximum and minimum values taken by the contiguous sum as follows.

\begin{lemma}
\label{lem:sumcomplement}
  Let  $j\in\{0,\dots,n\}$, $i \in [n]$ and $N\in[2^n]$. The minimum of the sum of $N$ contiguous bits in a bit list $c_{i,n}$ is $m$ if and only if its maximum is $N-m$.
\end{lemma}
\begin{proof}
 Let $\overline{c}_{i,n}$ be the bit list obtained from the list $c_{i,n}$ by flipping each zero to one and vice versa. Let $\sigma_{2^{i-1}}$ be an order $2^{i-1}$ cyclic permutation, observe that $\overline{c}_{i,n}$ is equal to $\sigma_{2^{i-1}}(c_{i,n})$.
 If the minimum value is obtained at the contiguous segment which starts at the index $j$ in $c_{i,n}$, then the value $N-m$ can be obtained by the contiguous sum starting at index $\sigma_{2^{i-1}}(j)$. Finally, note that $m$ is the minimum value if and only if $N-m$ is the maximum value.
\end{proof}

Combining Observation~\ref{obs:continuity} and Lemma~\ref{lem:sumcomplement} we get the following.

\begin{lemma}
\label{lem:minmax}
 Let $N\in[2^n]$ and $m=\min_{j \in [2^n]} \mathcal{S}(c_{i,n}, j, N)$. For every value $v$ in the range $\{m,\dots, N-m\}$ there exists a $j\in [2^n]$ such that $\mathcal{S}(c_{i,n}, j, N)=v$.
\end{lemma}

The lemma above allows us to find the range of values taken by the contiguous sum by just finding the minimum value taken by it.
Next we prove a simpler lemma about the range of values taken. Using that, in Theorem~\ref{thm:shiftrange}, we will find the range of values taken by the contiguous sum of $N$ elements in any list $c_{i,n}$.

\begin{lemma}
\label{lem:shiftpower}
 For  $j\in\{0,\dots,n\}$ and $i \in [n]$, the sum of $2^j$ contiguous bits in a bit list $c_{i,n}$ takes the following values.
\begin{enumerate}
\item
\label{enum:shiftpower-one}
If $j \leq (i-1)$, then the range is $\{0, \dots,2^j \}$, and
\item
\label{enum:shiftpower-two}
If $j\geq i$, then the sum is exactly $2^{j-1}$.
\end{enumerate}
\end{lemma}
\begin{proof}
 By Lemma~\ref{lem:minmax}, it suffices to find the minimum value of contiguous sum function. Notice that we have, $c_{i,n}= (0_{\times 2^{i-1}} \cdot 1_{\times 2^{i-1}})_{\times 2^{n-i}}$, by Observation~\ref{obs:zero-one-power}.
 \begin{enumerate}
 \item
 When $j\leq (i-1)$, we can pick a block of $2^j$ zeros giving a total of zero, which is the minimum possible value.
 \item
 When $j\geq i$, let $L_k$ be a list of $2^j$ contiguous bits of $c_{i,n}$ starting at the index $k$ in $c_{i,n}$. To prove that $\sum L_k = \sum L_{k+1} $, it suffices to show that $c_{i,n}(k)=c_{i,n}(k+2^j)$. Rewriting
 $c_{i,n}=((0_{\times 2^{i-1}}\cdot 1_{\times 2^{i-1}})_{\times 2^{j-i}})_{\times 2^{n-j}}$ shows that any two indices with difference equal to $2^j$ store the same value. As the choice of $k$ was arbitrary, the contiguous sum is equal to $2^{j-1}$.
 \end{enumerate} 
\end{proof}

\begin{theorem}
\label{thm:shiftrange}
 For $i \in [n]$, $N\in[2^{n}]$ and $p=2^i$, the sum of $N$ contiguous bits in a bit list $c_{i,n}$ takes values in the range, ${\rm range}(i,N) \triangleq$
 $$ 
 \left\{
	\floor[\Big]{\frac{N}{p}} \frac{p}{2} + \max \left( (N\mod p) - \frac{p}{2}, 0 \right),
  	\cdots,
   	\floor[\Big]{\frac{N}{p}} \frac{p}{2} + \min \left( N\mod p, \frac{p}{2} \right)
 \right\}.
 $$
\end{theorem}
\begin{proof}
 For a fixed $i\in[n]$ consider the list $c_{i,n}$. Assuming that the minimum value of the range is as claimed, by Lemma~\ref{lem:minmax}, the maximum value is
\begin{align*}
   \max_{j \in [2^n]} \mathcal{S}(c_{i,n}, j, N)
   &= N-\min_{j \in [2^n]} \mathcal{S}(c_{i,n}, j, N)\\
   &=N- \left( \floor[\Big]{\frac{N}{p}} \frac{p}{2} + \max \left( (N\mod p) - \frac{p}{2}, 0 \right)\right)\\
 &= \floor[\Big]{\frac{N}{p}}p + (N\mod p) - 
 \left( 
 \floor [\Big] {\frac{N}{p}} \frac{p}{2} + \max \left( (N\mod p) - \frac{p}{2}, 0 \right)
 \right)	\\
 &= \floor[\Big]{\frac{N}{p}} \frac{p}{2} +(N\mod p) - 
 \max	\left(		(N \mod p) - \frac{p}{2}, 	0 	\right)	\\
 &= \floor[\Big]{\frac{N}{p}} \frac{p}{2} + \min \left( (N\mod p)-(N \mod p) + \frac{p}{2}, N\mod p\right)\\
 &=  \floor[\Big]{\frac{N}{p}} \frac{p}{2} + \min \left(\frac{p}{2}, N\mod p\right).
\end{align*}
 
 As proved in case~\ref{enum:shiftpower-two} of Lemma~\ref{lem:shiftpower}, the sum of $\floor{\frac{N}{p}}p$ contiguous bits is equal to $\floor{\frac{N}{p}}\frac{p}{2}$  irrespective of the starting index.
 Therefore, it suffices to find the minimum sum of $R=(N\mod 2^i)$ contiguous bits. Let $L_k$ be a list of $R$ bits occurring contiguously in $c_{i,n}$ starting at index $k$. If the first bit of $L_k$ is $1$, then $\sum L_{k+1}\leq \sum L_k$. Therefore, we can keep on increasing the value of $k$ until the first bit is zero,  without increasing the value of the contiguous sum. On the other hand, if $c_{i,n}(k-1)=0$, then $\sum L_{k-1}\leq \sum L_k$. Therefore, we can keep on decreasing the value of $k$ one at a time until $c_{i,n}(k-1)=1$, without increasing the value of the contiguous sum. Thus the minimum value of the contiguous sum is achieved when the index $k$ points to the start of any block $0_{\times 2^{i-1}}$ contained in $c_{i,n}$. The value of minimum is  $\max(R-\frac{p}{2},0)$ as the ones start appearing after $\frac{p}{2}$ indices from the start of a list $0_{\times 2^{i-1}}\cdot 1_{\times 2^{i-1}}$.  Adding it to $\floor{\frac{N}{p}}\frac{p}{2}$ gives the required minimum value.
\end{proof}

\subsection{Algorithm}
We next state a theorem which gives an equivalent definition of {\em cyclic hyper degrees}.
\begin{theorem}
\label{thm:chd}
A list $w=\{w_1,\dots,w_n\}\in \mathbb{Z}_+^n$ is a {\em cyclic hyper degree} if and only if there exist $N\in[2^n]$ and a permutation $\pi$, such that for each $i\in [n]$, $w_{\pi(i)} \in {\rm range}(i,N)$.% (see Theorem~\ref{thm:shiftrange}).
\end{theorem}

\begin{proof}
 Forward direction is a direct consequence of the definition.
 
 Using Definitions~\ref{def:chd} and~\ref{def:csum}, we get that there exist numbers $s_1,\dots,s_n \in [2^n]$ such that for each $i\in [n]$, we have $w_{\pi(i)}=\mathcal{S}(c_{i,n},s_i,N)$. Let $\Pi^{-1}=(\sigma_{s_1}^{-1},\dots,\sigma_{s_n}^{-1})$ be the list of cyclic permutations, where for each $i\in [n]$, $\sigma_{s_i}^{-1}$ is the inverse of the cyclic permutation of order $s_i$. Consider the table $\Pi^{-1}(T_n)$, by
 Theorem~\ref{thm:rotatebits}, all its rows are distinct. In particular, the first $N$ rows are distinct and their sum is $\pi(w)$. Finally, $w\in H_n$ if and only if $\pi(w)\in H_n$.
\end{proof}

Theorem~\ref{thm:shiftrange} gives us a way to efficiently find the number of bits in a contiguous sum of $N$ bits. If we know the number of distinct bit sequences that can sum up to a given vector $w\in \mathbb{Z}_+^n$, then using Theorem~\ref{thm:shiftrange} we can generate all the possible ranges of values which can be taken by each coordinate of the sum. Finally, we need to check if each coordinate of $w$ is contained in different ranges, this corresponds to finding the permutation $\pi$ in Theorem~\ref{thm:chd}. In the next lemma, we will find the number of possible  distinct bit-sequences which can sum up to a given $w$ using cyclic shifts, this corresponds to finding $N$ in Theorem~\ref{thm:chd}.
\begin{lemma}
\label{lem:set-size}
If $w=\{w_1,\dots,w_n\}\in \mathbb{Z}_+^n$ is a {\em cyclic hyper degree}, then the number of bit sequences which sum up to $w$ is an element of the set
$$\mathcal{N}_w\triangleq \{2w_i+j~:~i\in[n], j\in \{-1,0,1\}\}.$$
\end{lemma}
\begin{proof}
 As one of the coordinates of $w$, say $w_k$, is the contiguous sum of $c_{1,n}$, we need to find the number of bits which sum up to $w_k$. From the structure of $c_{1,n}$, it is easily seen that there are just three values viz. $2w_k-1,2w_k,2w_k+1$ which contain $w_k$ in their range of sums. Conversely, for any number $x$ not contained in $\{2w_i+j~:~i\in[n], j\in \{-1,0,1\}\}$, the sum of $x$ contiguous bits $c_{1,n}$ will not contain any of $w_i$, for $i\in [n]$.
\end{proof}

\begin{lemma}
\label{lem:embed}
 Given $w\in\mathbb{Z}_+^n$ and a list of integer intervals $R_1,\dots,R_n \subset \mathbb{Z}_+^2$. There exists an algorithm running in time polynomial in $n$ which correctly answers if there exists a permutation $\pi$ such that  for each $i\in [n]$, $w_{\pi(i)}\in R_i$. 
\end{lemma}
\begin{proof}
 Construct a bipartite graph $G=(A,B,E)$ on $2n$ vertices. Let $A=B=[n]$ and $(i,j)\in E$ if and only if $w_i \in R_j$. Using a polynomial time algorithm one can find if there exists a perfect matching in $G$. If there is a perfect matching then the answer is \YES, otherwise it is \NO.
\end{proof}

\begin{theorem}
 \label{thm:chd-poly}
 There is a polynomial time algorithm in $n$ which decides if a given $w\in\mathbb{Z}_+^n$ is a {\em cyclic hyper degree}.
\end{theorem}

\begin{proof}
 For each $N\in\mathcal{N}_w$, given by Lemma~\ref{lem:set-size}, and $i\in[n]$ compute ${\rm range}(i,N)$ as given by Theorem~\ref{thm:shiftrange}. Now, use Lemma~\ref{lem:embed} on these ranges of numbers and decide if $w$ is a {\em cyclic hyper degree}, if it is not then try the next number from the set $\mathcal{N}_w$. If it succeeds for at least one element of $\mathcal{N}_w$, we answer \YES, otherwise we answer \NO. Finally, note that $\vert N_w \vert\leq 3n$ and all the other steps can be performed in time which is a polynomial function of $n$. 
\end{proof}