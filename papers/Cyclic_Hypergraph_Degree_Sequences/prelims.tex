\section{Preliminaries}
The set of non-negative integers is denoted using the symbol $\mathbb{Z}_+$.
The set of integers $\{1,\dots,n\}$ is denoted by $[n]$. For $m, M\in \mathbb{Z}_+$, we refer to an integral interval $\{i : m\leq i\leq M\}$ as a range and specify it by providing its minimum and maximum element.
An $n$-tuple, also simply referred to as a list, is $L=(\ell_1,\ell_2,\dots,\ell_n)$, of size $n$, is an ordered collection of elements. We refer to its $i^{th}$ element $\ell_i$ as $L(i)$. We index any list or tuple with natural numbers starting with $1$.
The notation $a_{\times m}$ denotes the tuple
$(a,\dots,a)$ consisting of $a$ repeated $m$ times.
We use $L_1 \cdot L_2$ to denote the list obtained by the operation of concatenating two lists $L_1$ and $L_2$.
A table $T=[L_1,L_2,\dots,L_m]$ is a collection of lists of equal size. Pictorially, the lists $L_1,\dots,L_m$ are arranged as columns in the table $T$. If each list $L_i$ is of size $n$, then the table $T$ is said to be of size $n\times m$. We refer to the $i^{th}$ row of a table $T$ using the notation $T(i)$ and to an $(i,j)^{th}$ entry using the symbol $T(i,j)$.

For ease of notation, the sum of the entries in a list $L$ will be denoted by $\sum L = \sum_{i\in [\vert L\vert]} L(i)$.
%We define the sum of two $n$-tuples to be the usual coordinate-wise addition over $\mathbb{Z}$.
The sum of $L=(\ell_1,\dots,\ell_n)$ and $L'=(\ell'_1,\dots,\ell'_n)$, denoted by $L+L'$, is the $n$-tuple $(\ell_1+\ell'_1,\dots,\ell_n+\ell'_n)$. If $S$ is a list of lists then the sum of its elements will be denoted using $\sum S = \sum_{x\in S} x$.

%%%%%%%                 cyclic permutation definitions, to include or not ?

%Give a number $n$, a permutation $\pi$ is any bijective function $\pi:[n]\rightarrow [n]$. 
For $k\geq 0$, a permutation $\pi$ is called a cyclic permutation of order $k$, for each $i\in [n]$ it maps $i \mapsto 1+((i+k-1)\mbox{ mod }n)$.
%A permutation $\pi$ can be applied to a list $L=(\ell_1,\dots,\ell_n)$ to produce a list $\pi(L)=(\ell_{\pi(1)},\dots,\ell_{\pi(n)})$. In our notation $\pi(L)$ can also be written as $(L(\pi(1)),\dots,L(\pi(n)))$.

A hypergraph $H$ is a pair $([n],\mathcal{F})$, where $\mathcal{F}$ is a family of subsets of $[n]$. A hypergraph is simple if no set is repeated in $\mathcal{F}$. The degree of a vertex $v\in[n]$ is equal to $\vert\{F\in\mathcal{F}~:~v\in F\}\vert$.
We will be working with an equivalent version, which can be stated in terms of co-ordinate wise sum of binary sequences of length $n$.

For a given positive integer $n$, consider the set $S_n = \{0,1\}^n$ consisting of all binary tuples of length $n$. The elements of $S_n$ will also be referred to as binary sequences.
We construct a set $H_n = \{ \sum_{x \in S} x : \varnothing \subseteq S\subseteq S_n\}$. Note that for the empty set $\varnothing$ the corresponding sum $\sum_{x\in\varnothing} x$ is equal to $0_{\times n}$. By construction, each element of $H_n$ is {\em realized} by some simple hypergraph and the degree sequence of every simple hypergraph is contained in $H_n$. Each element of $H_n$ is said to be {\em representable} or is said to admit a {\em representation}.
Given this setting the {\em realizability} problem for simple hypergraphs can be restated as follows.

\problemdef
{\hds}
{A tuple $w \in \mathbb{Z}_+^n$ which is provided as a binary input.}
{Is $w \in H_n$ ?}

If $w=(w_1,\dots,w_n) \in H_n$ and $w_n=0$, then $(w_1,\dots, w_{n-1})\in H_{n-1}$. As $\sum S_n = 2^{n-1}_{\times n}$, the maximum possible value of any entry in $w$ is $2^{n-1}$. Thus, if any entry of $w$ is outside the range $\{0, \ldots, 2^{n-1}\}$, then it is not a member of $H_n$.
Even though the number of subsets of $S_n$ is $2^{2^n}$, we get that the cardinality of $H_n$ is at most $2^{n\cdot (n-1)}$.
