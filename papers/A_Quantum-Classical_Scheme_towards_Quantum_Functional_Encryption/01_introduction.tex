\section{Introduction}

In a pioneering work, Boneh, Sahai and Waters formalized the notion of functional encryption in 2010 \cite{fe}. This generalization of an encryption scheme enabled users possessing different keys to learn different functions over the plaintext from the ciphertext. There have been many (classical) schemes proposed to realize different variants of functional encryption since the inception of the notion. However, despite it's power, there have been no equivalent formalizations of functional encryption for quantum information. \\
Over the last few years, formal definitions of quantum entropic security \cite{entsec} and quantum computational security \cite{compsec} have been introduced and accepted. Moreover, for classical functional encryption schemes, refined security definitions \cite{fesec} have been recently presented, which extend the notion of security given in the original paper \cite{fe}. \\
In this work, we first present a one-qubit secret-key quantum encryption scheme for classical information. The scheme is proven secure under quantum semantic security (Definition 8, \cite{compsec}) and quantum entropic indistinguishability (Definition 3, \cite{entsec}). The classical, functional extension of this scheme is then proven to be full-message private, full-function private (Definitions 2.4,3.2 \cite{fesec}) and weakly simulation-secure (Definition 5, \cite{fe}). Intuitively, the security of the quantum encryption scheme and it's extension are based upon distinguishing computationally between two different but uniformly distributed bits, which is a hard problem. Also the functional extension allows the user to learn different length subsequences of the message, with a different subsequence per instantiation of the scheme. \\
This paper is organized as follows. First, we present the syntax - the notations and definitions used to present our scheme, in the preliminaries section. The third section of the paper gives the construction and correctness arguments for our scheme. The proofs of security are given in the following section. We then give the operational aspects of the scheme in the discussion section. To catalogue other works in quantum encryption, we give next a section on related work. In the final section, we present our conclusions and future extensions possible for this work.