\section{Discussion}

We see that the scheme $(\Pi,\Xi)$ permits an arbitrary, polynomial-sized stretch in the length of the message, given the security parameter. This is realizable due to the infinitely many cipher positions permissible on the Bloch-sphere equator. Note that this is not possible classically. \\
Also, there needs to be consideration on efficient representations of arbitrary injective functions from $\lambda$-bits to $Q$-bits, which in general have exponential size tables. Note that for the purpose of our scheme (where $[[x]]$ denotes an efficient representation of $x$), \\ 
$[[a]] := (\sigma(\kappa_Q), \kappa_1, \kappa_2, ..., \kappa_Q)$ should suffice (see Definition \ref{def:key-f}). \\
Finally, it's important to observe that the keys $k$ should be distributed only after instantiating $a$, that is, running \texttt{Setup}. This is true because under different $\sigma$'s, the same key $k$ does not necessarily give the same decryption.