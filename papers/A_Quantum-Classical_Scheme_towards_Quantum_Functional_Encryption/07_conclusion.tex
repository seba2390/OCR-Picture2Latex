\section{Conclusions and Future Work}

In this work, we have introduced a novel one-qubit secret-key quantum encryption scheme for classical information. We have proved this scheme to have quantum semantic security, and quantum entropic indistinguishability as a function of the min-entropy of the message distribution. We have extended this scheme to permit recovery of different length subsequences of the message using different keys, under a new notion of positional secrecy. The resulting (hybrid) functional encryption scheme is proven to be full-message private, full-function private and weakly simulation-secure. \\
We hope to see the following improvements in the future, given the current status of our quantum-classical scheme: \\
\begin{enumerate}
\vspace*{-10pt}
\item Given that, under $\Xi$, encryptions of both $0$ and $1$ are statistically indistinguishable, perhaps there exists a proof of entropic security (Definition 2, \cite{entsec}) for the quantum scheme.
\item We recognize that the biggest drawback of the classical extension is that \kgen and \enc algorithms are only available via oracle calls, although that does not affect the security proofs. There could be a modification which permits making these algorithms public. 
\item We hope that this work motivates a new general definition for quantum functional encryption - fully quantum schemes which permit learning meaningful functions from encryptions of (general) quantum states.
\end{enumerate}