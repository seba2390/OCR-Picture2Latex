\section{Scheme Security}
\label{sec:security}

We will use the following corollary for our security proofs in this section.
\begin{corollary}
\label{cor:h-zero}
If $\mathcal{H}^\theta_u$ is as defined in Theorem \ref{thm:xi-correct}, then $\forall u,b,\theta \hspace*{5pt} \mathcal{H}^\theta_u\ket{b} = \mathcal{H}^\theta_{u \oplus b}\ket{0}$.
\end{corollary}

\subsection{Security of the Quantum Encryption Scheme}
\label{sec:quan-sec}

We first note that (before prepending with the randomness) $|\mathcal{QE}_{s,\theta}(b)| = 2|b|$ and so the core-function based impossibility result (Definition 6.1,6.2 Theorem 6.3, \cite{semsec}) does not apply to $\Xi$.

\begin{theorem}[Semantic Security]
\label{thm:xi-semsecure}
The one-qubit secret-key encryption scheme $\Xi$ is quantum semantically secure.
\end{theorem}
\begin{proof}
First, we prove that $\Xi = (\mathcal{QE},\mathcal{QD})$ is IND-secure under Definition \ref{def:ind-sec}. Let's say there exists a QPT distinguisher $q\mathcal{D}$ that can distinguish between $\mathcal{QE}_{s,\theta}(\ket{b})$ and $\mathcal{QE}_{s,\theta}(\ket{0})$ in time \poly \text{ } with a probability \nonegl. Consider the following attack where \texttt{cD} is a distinguisher between $r \oplus 0$ and $r \oplus b$, given $s$, $\theta$ and a uniformly random $r$:
\begin{enumerate}
\item The message adversary chooses $b$.
\item Sample $r \Leftarrow \udist(\{0,1\})$.
\item Prepare $\mathcal{H}^\theta_{r \oplus s}\ket{0}$ and send it to $q\mathcal{D}$ as it's first argument.
\item Choose a challenge $c^*$ uniformly between $r \oplus 0$ and $r \oplus b$ and send it to \texttt{cD}.
\item \texttt{cD} prepares $\mathcal{H}^\theta_{c^*}\ket{0}$ and sends it to $q\mathcal{D}$ as it's second argument.
\item $q\mathcal{D}$ now has one of the encryptions between $\mathcal{QE}_{s,\theta}(\ket{0})$ and $\mathcal{QE}_{s,\theta}(\ket{b})$.
\item $q\mathcal{D}$ distinguishes between the two possible encryptions and sends it's decision, $0$ or $b$, to \texttt{cD} (in time \poly \text{ } with a probability \nonegl).
\item \texttt{cD}, for the challenge $c^*$ outputs it's decision ($0$ or $b$), the same as the result given by $q\mathcal{D}$.
\end{enumerate}
Thus we have constructed a computational distinguisher \texttt{cD} between two statistically identical distributions $r \oplus 0$ and $r \oplus b$, even for a $b$ different than $0$, a contradiction. So $q\mathcal{D}$ cannot exist (step 7 cannot happen) and $\Xi$ is IND-secure. \\
We next use Theorem 9 from \cite{compsec} to conclude that $\Xi$ is quantum semantic secure. 
\end{proof}

\begin{theorem}[Entropic Indistinguishability]
\label{thm:xi-entindist}
The one-qubit secret-key encryption scheme $\Xi$ is entropically $(t,\frac{1}{2}(2^{1-t}-1))$-indistinguishable for min-entropy $t \in [0,1]$.
\end{theorem}
\begin{proof}
Here we prove entropic-indistinguishability (Definition \ref{def:ent-ind}) of the message qubit under $\Xi$. Since the secret qubit comes from the uniform distribution, it is perfectly indistinguishable (under the same superoperator) and it's security proof is thus implied. \\
Let $\rho = \sum_{j \in \{0,1\}} \gamma_j \ket{j}\bra{j}$ be the operator corresponding to the message qubit. The only associated interpretations of $\rho$ are classical, according to our scheme. Let $\mathcal{E}$ be the superoperator corresponding to our unitary map. So 
$\mathcal{E}(\rho) := \mathcal{H}^\theta_r \rho (\mathcal{H}^\theta_r)^\dagger$ for uniformly random $r$ and the message space distribution has min-entropy $t = - \log(\max\{\gamma_0,\gamma_1\}) \in [0,1]$. \\
Firstly, it's an easy verification that 
\[ \mathcal{E}(\rho) := \mathcal{H}^\theta_r \rho (\mathcal{H}^\theta_r)^\dagger = \frac{1}{2}
\begin{bmatrix}
    1 & (\gamma_0 - \gamma_1)(-1)^re^{i\theta} \\
    (\gamma_0 - \gamma_1)(-1)^re^{i\theta} & 1 \\
\end{bmatrix} \]
\[ \text{Now, } \lvert\lvert \mathcal{E}(\rho) - \frac{1}{2}\mathbb{I}\rvert\rvert_{tr} = 
\frac{1}{2} \texttt{Tr} \sqrt{(\mathcal{E}(\rho) - \frac{1}{2}\mathbb{I})^\dagger(\mathcal{E}(\rho) - \frac{1}{2}\mathbb{I})} 
= \frac{1}{2} \texttt{Tr}
\begin{bmatrix}
    \frac{\lvert \gamma_0 - \gamma_1\rvert}{2} & 0 \\
    0 & \frac{\lvert \gamma_0 - \gamma_1\rvert}{2}  \\
\end{bmatrix}
 \]
 
 \[ = \frac{\lvert \gamma_0 - \gamma_1\rvert}{2} =  \frac{\lvert 2 \times \max\{\gamma_0,\gamma_1\} - 1 \rvert}{2}  = \frac{\lvert 2 \times 2^{-t} - 1 \rvert}{2} =  \frac{1}{2}(2^{1-t}-1) \]
Thus $\Xi$ is $(t,\frac{1}{2}(2^{1-t}-1))$-indistinguishable.
\end{proof}


\subsection{Security of the hFE scheme}
\label{sec:hFE-sec}

Now we prove security of the classical extension of our quantum encryption using the definitions of privacy/security from the domain of classical functional encryption.

\begin{theorem}[Fully Message Private]
\label{thm:pixi-msgpriv}
The hFE scheme $(\Pi,\Xi)$ is fully message private (under Definition \ref{def:fmp}).
\end{theorem}
\begin{proof}
We start by observing that since the QPT adversary $\mathcal{A}$ is a valid message-privacy adversary (Definition \ref{def:vmpa}), the messages $m^0$ and $m^1$ agree on all bits $[q]$ such that $\mathcal{A}$ queries $\kappa_q$. Let $q^* = max \{q : \mathcal{A} \text{ queries } \kappa_q \}$. Then $\exists j^* \in \{ q^*+1, ..., Q \}$ such that $m^0_{j^*} = 1 - m^1_{j^*}$, otherwise the two encryption oracle calls are identical. \\
Now, let's say $\mathcal{A}$ can distinguish between $\texttt{Enc}_{a,0}(m^0,m^1)$ and $\texttt{Enc}_{a,1}(m^0,m^1)$ in time \poly with a probability \nonegl. Let $\texttt{cD}_j, \forall j \in [Q]$ be  distinguishers on a bit. Let $s \in \{ 0,1 \}^Q$ be the given secret and $\theta_j = \frac{2 \pi j}{Q}, \forall j \in [Q]$ be the given encryption angles. Now consider the following attack for a uniformly random $r \in \{ 0,1 \}^Q$: \\
\vspace*{-8pt}
\begin{enumerate}
\item The adversary chooses messages $m^0,m^1$.
\item Sample $r \Leftarrow \udist(\{0,1\}^Q)$.
\item Prepare $\forall j, \mathcal{H}^{\theta_j}_{r_j \oplus s_j}\ket{0}$ and send them to $\mathcal{A}$ in order as it's odd-position arguments.
\item Choose a challenge bit $b$ uniformly between $0$ and $1$.
\item Send challenge $\forall j, u_j = r_j \oplus m^b_j$ to $\texttt{cD}_j$.
\item Each $\texttt{cD}_j$ prepares $\mathcal{H}^{\theta_j}_{u_j}\ket{0}$ and sends it (in order) to $\mathcal{A}$ as it's even-position arguments.
\item $\mathcal{A}$ now has the encryption $\texttt{Enc}_a(m^b)$.
\item $\mathcal{A}$ outputs $b' = b$ sends it to $\texttt{cD}_j ,\forall j$ (in time \poly \text{ } with a probability \nonegl).
\item $\texttt{cD}_j$, for the challenge $u_j$ outputs it's decision $b'$.
\end{enumerate}
Thus we have constructed a computational distinguisher $\texttt{cD}_{j^*}$ between two statistically identical distributions $r \oplus m^b_{j^*}$ and $r \oplus m^{1-b}_{j^*}$ where 
$m^b_{j^*} = 1 - m^{1-b}_{j^*}$. This is a contradiction. So, $\mathcal{A}$ cannot exist (step 8 cannot happen with a non-negligible probability) and $(\Pi,\Xi)$ is fully message private.
\end{proof}

\begin{corollary}[Fully Function Private]
\label{cor:pixi-funcpriv}
The hFE scheme $(\Pi,\Xi)$ is fully function private (under Definition \ref{def:ffp}).
\end{corollary}
\begin{proof}
Again, on grounds that the QPT adversary $\mathcal{A}$ is a valid function-privacy adversary (Definition \ref{def:vfpa}), 
$f^0_{\kappa_{q^0}}(m^0) = f^1_{\kappa_{q^1}}(m^1) \Rightarrow f^0_{\kappa_{q^0}} = f^1_{\kappa_{q^1}} \Rightarrow \kappa_{q^0} = \kappa_{q^1}$. \\
This means that $\forall q^0,q^1 \in [Q],  \texttt{KeyGen}_{a,0}(\kappa_{q^0},\kappa_{q^1}) = \texttt{KeyGen}_{a,1}(\kappa_{q^0},\kappa_{q^1})$ and the problem reduces to proving full message privacy, which has been proven (Theorem \ref{thm:pixi-msgpriv}). Also, for completeness, every key in $\{ 0,1 \}^\lambda \setminus \{ \kappa_q: q \in [Q] \}$ gives a secret $\vec \bot$, rendering the two $\kgen_{a,u}, u \in \{ 0,1 \}$ oracles same (in output this time). 
\end{proof}

\begin{theorem}[Weakly Simulation Secure]
\label{thm:pixi-weaksim}
The hFE scheme $(\Pi,\Xi)$ is weakly simulation-secure (under Definition \ref{def:wss}).
\end{theorem}
\begin{proof}[Sketch]
Intuitively, we will show that the output distributions of the adversary and simulator are statistically identical (i.e., having zero statistical distance). By virtue of the game, other distributions are same and proving the $\alpha$'s identical is sufficient to prove the real and ideal distribution tuples are statistically identical. \\
First, let us define the following (deterministic) function: \\
$\texttt{S}_a(k) := $ If $k = \kappa_q$ for some $q \in [Q]$, return the first $q$-bits of $\sigma(\kappa_Q)$. Otherwise return $\vec \bot$. \\
And let $\alpha = (\alpha_1,\alpha_2,...,\alpha_l)$ corresponding to key query distributions $( y_j )_{j \in [l]}$. The first thing to note is that the $\alpha_j$'s are distributions on $m_1m_2...m_q$ for $q \in \{ 0 \} \cup [Q]$, and these distributions are a deterministic map, say $\phi$, from $\texttt{S}_a(y_j)$. To see this in the real world game, first observe that since the quantum cipher-text is IND-secure, it does not leak any information about the message to \texttt{Adv}, and so $\alpha^{\textsc{real}}_j = \phi(\texttt{KeyGen}_a(y_j)) =  \phi(\texttt{S}_a(y_j))$. In the ideal world game, firstly, $F_a(\aleph, \vec m) = zQ$ for some integer $z$, leaking nothing additional about the chosen message vector. Also, from the view of comparing the distribution vectors, and since the $\vec m$ distributions are identical by definition, the queries of \texttt{Sim} to $F_a(y_j,\vec m)$ can be reduced to a function of $a$ and $y_j$ alone. That function is exactly $\alpha^{\textsc{ideal}}_j = \phi(\texttt{S}_a(y_j))$ since only one key $\kappa_q$ maps to one $m_1m_2...m_q$, $q \in [Q]$ and all other keys map to the null message. Now let's say there is a non-zero statistical distance $\Delta$ between $\alpha^{\textsc{real}}_j$ and $\alpha^{\textsc{ideal}}_j$. Then 
$0 < \Delta(\alpha^{\textsc{real}}_j,\alpha^{\textsc{ideal}}_j) = \Delta(\texttt{S}^{\textsc{real}}_a(y_j),\texttt{S}^{\textsc{ideal}}_a(y_j)) $ \\
$ \le \Delta(y_j, y_j)$ (as $\texttt{S}^{\textsc{real}}_a = \texttt{S}^{\textsc{ideal}}_a = \texttt{S}_a$, applying Theorem 7.6, \cite{statdist}) \\
$\Rightarrow 0 < \Delta(y_j, y_j)$. \\
Thus we have a contradiction. So $\forall j \in [l], \alpha^{\textsc{real}}_j = \alpha^{\textsc{ideal}}_j$, and the real and ideal distribution ensembles are statistically indistinguishable.
\end{proof}

