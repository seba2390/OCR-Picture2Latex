%%%%%%%%%%%%%%%%%%%%%%%%%%%%%%%%%%%%%%%%%%%%%%%%%%%%%%%%%%%%%%%%%%%%%%%%%%%%%%%%
%2345678901234567890123456789012345678901234567890123456789012345678901234567890
%        1         2         3         4         5         6         7         8

\documentclass[letterpaper, 10 pt, conference]{ieeeconf}  % Comment this line out if you need a4paper

%\documentclass[a4paper, 10pt, conference]{ieeeconf}      % Use this line for a4 paper

\IEEEoverridecommandlockouts                              % This command is only needed if 
                                                          % you want to use the \thanks command

\overrideIEEEmargins                                      % Needed to meet printer requirements.

%In case you encounter the following error:
%Error 1010 The PDF file may be corrupt (unable to open PDF file) OR
%Error 1000 An error occurred while parsing a contents stream. Unable to analyze the PDF file.
%This is a known problem with pdfLaTeX conversion filter. The file cannot be opened with acrobat reader
%Please use one of the alternatives below to circumvent this error by uncommenting one or the other
%\pdfobjcompresslevel=0
%\pdfminorversion=4

% See the \addtolength command later in the file to balance the column lengths
% on the last page of the document

% The following packages can be found on http:\\www.ctan.org
%\usepackage{graphics} % for pdf, bitmapped graphics files
%\usepackage{epsfig} % for postscript graphics files
%\usepackage{mathptmx} % assumes new font selection scheme installed
%\usepackage{times} % assumes new font selection scheme installed
%\usepackage{amsmath} % assumes amsmath package installed
%\usepackage{amssymb}  % assumes amsmath package installed
\let\proof\relax
\let\endproof\relax
\usepackage{amsthm}

\usepackage{microtype}
\usepackage{graphicx}
\usepackage{subfigure}
\usepackage{booktabs} % for professional tables
\usepackage{algorithm}
\usepackage{algorithmic}
\usepackage[hidelinks]{hyperref}
\usepackage{url}
\usepackage{dsfont} %added for indicator function symbol
\usepackage{siunitx}
\usepackage{amsmath,amssymb,amsfonts}
\usepackage{cleveref}
\usepackage{mathrsfs}
\usepackage{hyperref}
\usepackage{cite}



\newtheorem{proposition}{Proposition}
\newtheorem{problem}{Problem}
\newtheorem{assumption}{Assumption}
\newtheorem{lemma}{Lemma}
\newtheorem{remark}{Remark}
\newtheorem{theorem}{Theorem}
\newtheorem{corollary}{Corollary}

\title{\LARGE \bf
Convergence Analysis of the Best Response Algorithm \\ for Time-Varying Games
}


\author{Zifan Wang, Yi Shen, Michael M. Zavlanos, and Karl H. Johansson
% <-this % stops a space
\thanks{* This work was supported in part by Swedish Research Council Distinguished Professor Grant 2017-01078, Knut and Alice Wallenberg Foundation, Wallenberg Scholar Grant, the Swedish Strategic Research Foundation CLAS Grant RIT17-0046, AFOSR under award \#FA9550-19-1-0169, and  NSF under award CNS-1932011.
}% <-this % stops a space
\thanks{Zifan Wang and Karl H. Johansson are with Division of Decision and Control Systems, School of Electrical Enginnering and Computer Science, KTH Royal Institute of Technology, and also with Digital Futures, SE-10044 Stockholm, Sweden. Email: \{zifanw,kallej\}@kth.se.}
\thanks{Yi Shen and Michael M. Zavlanos are with the Department of Mechanical Engineering and Materials Science, Duke University, Durham, NC, USA. Email: \{yi.shen478, michael.zavlanos\}@duke.edu}%
}


\begin{document}



\maketitle
\thispagestyle{empty}
\pagestyle{empty}


%%%%%%%%%%%%%%%%%%%%%%%%%%%%%%%%%%%%%%%%%%%%%%%%%%%%%%%%%%%%%%%%%%%%%%%%%%%%%%%%
\begin{abstract}
This paper studies a class of strongly monotone games involving non-cooperative agents that optimize their own time-varying cost functions.
%
We assume that the agents can observe other agents' historical actions and choose actions that best respond to other agents' previous actions; we call this a best response scheme.
%
We start by analyzing the convergence rate of this best response scheme for standard time-invariant games.
%
Specifically, we provide a sufficient condition on the strong monotonicity parameter of the time-invariant games under which the proposed best response algorithm achieves exponential convergence to the static Nash equilibrium. 
%
We further illustrate that this best response algorithm may oscillate when the proposed sufficient condition fails to hold, which indicates that this condition is tight.
%
Next, we analyze this best response algorithm for time-varying games where the cost functions of each agent change over time.
%
Under similar conditions as for time-invariant games, we show that the proposed best response algorithm stays asymptotically close to the evolving equilibrium. We do so by analyzing both the equilibrium tracking  error and the dynamic regret.
%
Numerical experiments on economic market problems are presented to validate our analysis. 




\end{abstract}


%%%%%%%%%%%%%%%%%%%%%%%%%%%%%%%%%%%%%%%%%%%%%%%%%%%%%%%%%%%%%%%%%%%%%%%%%%%%%%%%
\section{Introduction}
Online convex games study the interplay between game theory and online learning, and find many applications ranging from traffic routing \cite{sessa2019no} to economic market optimization \cite{wang2022risk,lin2020finite}. 
%
In these games, agents simultaneously take actions to minimize their loss functions, which depend on the other agents' actions.
%


Generally, every agent in an online convex game adapts its actions to the actions of other agents in a dynamic manner with the objective to minimize its regret, defined as the cumulative difference in performance between the agent's online actions and the best single action in hindsight.
%
An algorithm is said to achieve no-regret learning if every agent's regret generated by this algorithm is sub-linear in the total number of episodes.
%
If the agents in an online game reach a stationary point from which no agent has an incentive to deviate, then we say the game has reached a Nash equilibrium.
%
There is a growing literature \cite{tatarenko2018learning,bravo2018bandit,drusvyatskiy2021improved,mertikopoulos2019learning,wang2022zeroth} that analyzes the Nash equilibrium convergence in strongly monotone games which admit a unique Nash equilibrium as shown in~\cite{rosen1965existence}.


In non-cooperative games, a common strategy used by competitive agents that selfishly minimize their own cost functions is the best response algorithm since it produces the most favorable outcome given the other agents plays.
%
The best response algorithm has been shown to converge under a spectral condition associated with the best-response map \cite{shanbhag2016inexact,facchinei201012}.
%
In general, best response algorithms have been studied for several classes of games, including supermodular games \cite{milgrom1990rationalizability}, potential games \cite{pass2019course,swenson2018best,lei2017randomized} and zero-sum games \cite{leslie2020best}. 
%
For example, \cite{swenson2018best} shows that in almost every potential game with finite actions, the best response dynamics converges to the unique Nash equilibrium with linear rate. 
%
Similarly, \cite{leslie2020best} shows the convergence of several best response dynamics in two-player zero-sum games.




In this paper, we study the regret and equilibrium tracking error of the best response algorithm for time-varying games. Specifically, we consider a class of strongly monotone games \cite{rosen1965existence,bravo2018bandit}, which guarantee the uniqueness of the well-defined Nash equilibrium. To the best of our knowledge, the best response algorithm has not been explored in the literature for time-varying games. 
%
Instead, time-varying games have been analyzed using  gradient-based algorithms for, e.g., strongly monotone games \cite{duvocelle2022multiagent} and zero-sum games \cite{zhang2022no}. Specifically, \cite{duvocelle2022multiagent} analyzes the Nash equilibrium convergence and the equilibrium tracking properties of the mirror descent algorithm for games that converge and diverge, respectively. In \cite{zhang2022no}, a gradient-type algorithm is proposed that achieves performance guarantees under three different measures.
As gradient-based algorithms are fundamentally different compared to the best response method, the techniques developed in these works cannot be applied here to analyze the best response algorithm.



To address this challenge, we first start with time-invariant games. Specifically, we assume games that satisfy the so-called strong monotonicity condition with parameter $m>0$, which guarantees the uniqueness of the Nash equilibrium \cite{rosen1965existence}. We provide a sufficient condition $m>L\sqrt{N-1}$ under which the best response algorithm achieves linear convergence to the static Nash equilibrium, where $L$ is the Lipschitz constant related to the gradient of the individual loss functions and $N$ is the number of agents. 
Moreover, we show numerically that when this condition fails to hold, the best response algorithm may oscillate. 
Compared to \cite{facchinei201012}, here we characterize the convergence in terms of the strong monotonicity parameter. For simple problems, we can show that our proposed condition is equivalent to the spectral condition proposed in \cite{facchinei201012}. 
%
Then, we analyze the best response algorithm for time-varying games where the Nash equilibrium evolves over time. Specifically, under similar conditions as for time-invariant games, we show that the average distance from the evolving equilibrium is bounded by the equilibrium variation. 
We also show that the dynamic regret is bounded by the cumulative variations of the loss functions.





%
The rest of the paper is organized as follows. In Section~\ref{sec:problem}, we provide some preliminaries and formally define the problem. In Section~\ref{sec:BR}, we present the regret and equilibrium convergence of the best response algorithm for time-invariant games. In Section~\ref{sec:TV_BR}, we extend our result to time-varying games and analyze the equilibrium tracking error and the dynamic regret. In Section~\ref{sec:simulation}, numerical experiments on  a Cournot game are presented to verify our method. Finally, in Section~\ref{sec:conclusion}, we conclude the paper.

\section{Preliminaries and Problem Definition}\label{sec:problem}
\subsection{Online Convex Games}
Consider an online convex game $\mathcal{G}$ with $N$ agents, whose goal is to learn their best individual actions that minimize their local loss functions.
%
For each agent  $i\in \mathcal{N}=\{1,\ldots,N\}$, denote by $\mathcal{C}_i(x_i,x_{-i}) : \mathcal{X} \rightarrow \mathbb{R}$ its individual loss function, where $x_i \in \mathcal{X}_i$ is the action of agent $i$, $x_{-i}$ are the actions of all agents excluding agent $i$, and we define by $\mathcal{X} =\Pi_{i=1}^N\mathcal{X}_i$ the joint action space since each agent takes actions independently. 
%
For ease of notation, we collect all agents' actions in a vector $x:=(x_1,\ldots,x_N)$. 
%
We assume that $\mathcal{C}_i(x)$ is convex in $x_i$ for all $x_{-i} \in \mathcal{X}_{-i}$, where $\mathcal{X}_{-i}$ is the joint action space excluding agent $i$.
%
% In addition, we assume that the diameter of the convex set $\mathcal{X}_i$ is bounded by $D$, for all $i=1,\ldots,N$. 
The goal of every agent~$i$ is to determine the action $x_i$ that minimizes its individual cost function, i.e., 
\begin{align}\label{eq:def:game}
    \mathop{{\rm{min}}}_{x_i \in \mathcal{X}_i} \mathcal{C}_{i}(x_i,x_{-i}).
\end{align}
As shown in \cite{rosen1965existence}, convex games always have at least one Nash equilibrium. In what follows, we denote by $x^{*}$ a Nash equilibrium of the game \eqref{eq:def:game}. Then, for each agent $i$, we have $\mathcal{C}_i(x^{*})\leq \mathcal{C}_i(x_i,x_{-i}^{*}),$ $\forall x_i \in \mathcal{X}_i$, $i\in\mathcal{N}$. At this Nash equilibrium point, agents are strategically stable in the sense that each agent lacks incentive to change its action.
%
Since the agents' loss functions are convex, the Nash equilibrium can also be characterized by the first-order optimality condition, i.e., $\langle \nabla_{x_i} \mathcal{C}_i(x^{*}), x_i - x_i^{*} \rangle \geq 0, \; \forall x_i \in \mathcal{X}_i, i\in\mathcal{N},$ where $\nabla_{x_i} \mathcal{C}_i(x)$ is the partial derivative of the loss function with respect to each agent's action.  We write $\nabla_{i} \mathcal{C}_i(x)$ instead of $\nabla_{x_i} \mathcal{C}_i(x)$ whenever it is clear from the context. 

% Throughout the paper, we make the following assumptions on the convex loss functions. 


In general, it is not easy to show convergence to a Nash equilibrium for games with multiple Nash Equilibria. 
%
For this reason, recent studies often focus on games that are so-called strongly monotone and are well-known to have a unique Nash equilibrium \cite{rosen1965existence}. 
%
The game \eqref{eq:def:game} is said to be $m$-strongly monotone if for $\forall x,x'\in \mathcal{X}$ we have that 
\begin{align}\label{eq:strong_monotone}
    \sum_{i=1}^N \langle \nabla_i \mathcal{C}_i(x) -\nabla_i \mathcal{C}_i(x'),x_i-x_i' \rangle \geq m \left\|x -x' \right\|^2.
\end{align}
%
The ability of the agents to efficiently learn their optimal actions can be quantified using the notion of (static) regret that captures the cumulative loss of the learned online actions compared to the best actions in hindsight, and can be formally defined as
\begin{align}\label{eq:def:regret:game}
    {\rm{SR}}_i(T)= \sum_{t=1}^T \mathcal{C}_i(x_t) - \mathop{\rm{min}}_{x_i} \sum_{t=1}^T\mathcal{C}_i(x_i,x_{-i,t}),
\end{align}
for sequences of actions $\{x_{i,t} \}_{t=1}^T, i=1,\ldots,N$.
An algorithm is said to be no-regret if the regret of each agent is sub-linear in the total number of episodes $T$, i.e., ${\rm{SR}}_i(T)=\mathcal{O}(T^a), a\in[0,1)$, $\forall i \in \mathcal{N}$.

\subsection{Problem Definition}
In this work, we consider the time-varying game $\mathcal{G}_t$ where at episode $t$ each agent aims to minimize its time-varying cost function, i.e.,
\begin{align}\label{eq:def:TV:game}
    \mathop{{\rm{min}}}_{x_i \in \mathcal{X}_i} \mathcal{C}_{i,t}(x_i,x_{-i}).
\end{align}
Then, we can define the best response algorithm for time-varying games as
\begin{align}\label{eq:TVBR:update}
    x_{i,t+1} = \mathop{\rm{arg min}}_{x_i \in \mathcal{X}_i} \mathcal{C}_{i,t} (x_i, x_{-i,t}).
\end{align}
%
To attain the best response action $x_{i,t+1}$, for each agent $i$, we assume the cost function $\mathcal{C}_{i,t}$ is known and all other agents' previous actions are provided. This is not a very strong assumption. For example, in supply chain problems \cite{cachon2006game}, $\mathcal{C}_{i,t}$ can represent an agent's local revenue model that depends on all competitors' actions and unknown market demands. At the beginning of episode $t+1$, the agents may not be able to observe the other agents' actions and precisely predict the market demands. However, previous actions and demands can be obtained from public revenue reports. 
%
Thus, it is reasonable to implement a strategy where the agents take actions that best respond to the other agents' actions from the previous episode. 
%
In addition, we assume that at every episode $t$, the time-varying game with the cost function $\mathcal{C}_{i,t}$ is strongly monotone and thus has a unique Nash equilibrium, which we denote by $x_t^{*}$.
To analyze the performance of the best response algorithm \eqref{eq:TVBR:update} for time-varying games, we define the equilibrium tracking error 
\begin{align}\label{eq:BRTV:trackingerror}
    {\rm{Err}}(T):=\sum_{t=1}^T\left\| x_t - x_t^{*}\right\|^2,
\end{align}
and the dynamic regret
\begin{align}\label{eq:BRTV:dynamic:regret}
    {\rm{DR}}_i(T) := \sum_{t=1}^T \Big( \mathcal{C}_{i,t}(x_t) - \mathop{\rm{min}}_{y_i} \mathcal{C}_{i,t}(y_i,x_{-i,t})\Big),
\end{align}
where $T$ is the total number of episodes.
If the game $\mathcal{G}_t$ changes significantly over time, it is reasonable to expect that it may become impossible to track the evolving equilibrium. 
The time-varying  problem becomes meaningful only when the variation of the game $\mathcal{G}_t$ is reasonably small.
To capture the effect of the variation of the game $\mathcal{G}_t$ on the performance of the best response algorithm, we first define the equilibrium variation
%
\begin{align}\label{eq:def:VT}
    V_T:=\sum_{t=1}^T\left\|x_{t}^{*}- x_{t+1}^{*}  \right\|^2,
\end{align}
%
which tracks the changes of Nash equilibria.
%
It is possible that the cost function $ \mathcal{C}_{i,t}$ changes over time but the equilibrium stays constant, i.e., $V_T=0$. 
To further capture the variations of the cost functions, we define the function variation 
\begin{align}\label{eq:def:WT}
    W_{i,T} = \sum_{t=1}^T \sup_{x\in\mathcal{X}}|C_{i,t}(x) - C_{i,t+1}(x)|.
\end{align}
%
% Clearly, if both the equilibrium variation and function variation equal to zero, the game becomes time-invariant. 
Our goal in this paper is to analyze the equilibrium tracking error and the dynamic regret of the best response algorithm \eqref{eq:TVBR:update} for time-varying games. To do so, we start with the analysis of time-invariant games and then extend our results to the time-varying case.




\section{Time-Invariant Games}\label{sec:BR}
In this section, we provide sufficient conditions for Nash equilibrium convergence of the best response algorithm for time-invariant games. The best response algorithm in this case becomes 
\begin{align}\label{eq:BR:update}
    x_{i,t+1} = \mathop{\rm{arg min}}_{x_i \in \mathcal{X}_i} \mathcal{C}_{i} (x_i, x_{-i,t}).
\end{align} 
%
% We first show a useful lemma that lays the foundation for the subsequent analysis.
% %
%
%
\begin{proposition}\label{prop:BR}
Suppose that the game $\mathcal{G}$ is $m$-strongly monotone, and $\nabla_i \mathcal{C}_i(x_i,x_{-i})$ is $L$-Lipschitz continuous in $x_{-i}$ 
for every $x_i \in \mathcal{X}_i$, with parameter $m>L \sqrt{N-1}$. 
Then, the best response algorithm \eqref{eq:BR:update} satisfies that
\begin{align}\label{eq:BR:convergence}
    \left\| x_T - x^{*}\right\| \leq \rho^{T-1} \left\| x_1 - x^{*}\right\|,
\end{align}
where $\rho:= \frac{L\sqrt{N-1}}{m}$.
\end{proposition}
%
\begin{proof}
Applying the first order optimality condition to the cost function $\mathcal{C}_i$  at the optimal point $x_{i,t+1}$ and using the update rule \eqref{eq:BR:update}, we have that
%
\begin{align}\label{eq:BR_temp1}
    \langle \nabla_i \mathcal{C}_i (x_{i,t+1},x_{-i,t}),x_i - x_{i,t+1} \rangle \geq 0, \; \; \forall x_i \in \mathcal{X}_i.
\end{align}
%
Since the game is strongly monotone, we have that for all $x_i \in \mathcal{X}_i$,
\begin{align}\label{eq:BR_temp2}
    \langle \nabla_i \mathcal{C}_i (x_{i},x_{-i,t})- \nabla_i \mathcal{C}_i (x_{i,t+1},x_{-i,t}), x_i - x_{i,t+1} \rangle  \nonumber \\
    \geq m \left\| x_i -x_{i,t+1} \right\|^2,
\end{align}
%
which follows from the definition \eqref{eq:strong_monotone} by setting $x=(x_{i},x_{-i,t})$ and $x'=(x_{i,t+1},x_{-i,t})$.
Combining \eqref{eq:BR_temp2} with \eqref{eq:BR_temp1} and replacing $x_i$ with $x_i^{*}$, we get
\begin{align}\label{eq:BR_temp3}
    &m \left\| x_i^{*} -x_{i,t+1} \right\|^2  
    % \leq & \langle \nabla_i \mathcal{C}_i (x_{i}^{*},x_{-i,t})- \nabla_i \mathcal{C}_i (x_{i,t+1},x_{-i,t}), x_i^{*} - x_{i,t+1} \rangle   \nonumber \\
    \leq  \langle \nabla_i \mathcal{C}_i (x_{i}^{*},x_{-i,t}) , x_i^{*} - x_{i,t+1} \rangle .
\end{align}
Summing the both sides of inequality \eqref{eq:BR_temp3} over $i=1,\ldots,N$, we have that
%
\begin{align}\label{eq:BR_temp4}
    & \left\| x_{t+1} - x^{*} \right\|^2 
    \leq  \frac{1}{m} \sum_i \langle \nabla_i \mathcal{C}_i (x_{i}^{*},x_{-i,t}) , x_i^{*} - x_{i,t+1} \rangle \nonumber \\
    & \leq  \frac{1}{m} \sum_i \langle \nabla_i \mathcal{C}_i (x_{i}^{*},x_{-i,t}) - \nabla_i \mathcal{C}_i (x^{*}) , x_i^{*} - x_{i,t+1} \rangle \nonumber \\
    & \leq   \frac{1}{m} \sum_i L \left\| x_{-i,t} - x_{-i}^{*}\right\| \left\| x_{i}^{*} - x_{i,t+1}\right\| \nonumber \\
   & \leq   \frac{L\sqrt{N-1}}{m}   \left\| x_{t} - x^{*}\right\| \left\| x^{*} - x_{t+1}\right\|,
\end{align}
%
where the second inequality follows from the Nash equilibrium condition $\langle \nabla_i \mathcal{C}_i (x^{*}),x_i - x_{i}^{*} \rangle \geq 0 $, $\forall x_i \in \mathcal{X}_i$ and the third inequality is due to the Lipschitz continuous property of the function $\mathcal{C}_i$ in $x_{-i}$.
The last inequality follows from the Cauchy-Schwarz inequality.
Dividing the inequality \eqref{eq:BR_temp4} by $\left\| x_{t+1} - x^{*} \right\|$ yields 
\begin{align}\label{eq:BR_temp5}
    \left\| x_{t+1} - x^{*} \right\| \leq \frac{L \sqrt{N-1}}{m}\left\| x_{t} - x^{*} \right\|.
\end{align}
Note, if $\left\| x_{t+1} - x^{*} \right\| = 0$, then \eqref{eq:BR_temp5} holds trivially. Applying inequality \eqref{eq:BR_temp5} iteratively over $t=1,\ldots,T-1$ completes the proof.
\end{proof}
%
In what follows, we provide some intuition and explain the condition $m>L \sqrt{N-1}$.
First, suppose that $L_1$ is the Lipschitz constant of the function $ \nabla_i \mathcal{C}_i(x)$ with respect to $x$. From its definitions we conclude that $L\leq L_1$. Therefore, the Lipschitz constant $L_1$ provides an upper bound on the variation of the gradients and is always greater than the strongly monotone parameter $m$ which provides a lower bound, i.e., $m\leq L_1$. However, it is still possible to have $m>L \sqrt{N-1} $. For example, if $\mathcal{C}_i$ only depends on $x_i$, we have that $L = 0$ and thus the condition naturally holds as long as $m>0$.

On the other hand, consider the condition $m>L\sqrt{N-1}$ and rearrange the terms to get $L<\frac{m}{ \sqrt{N-1}}$. Recall that $L$ is the Lipschitz constant of the function $\nabla_i \mathcal{C}_i(x_i,x_{-i})$ with respect to $x_{-i}$, which can be interpreted as the maximum influence of the other agents' actions on agent $i$. The condition $L<\frac{m}{ \sqrt{N-1}}$ requires that this influence is small enough for the game to converge.  The presence of multiple agents ($N$ is large) reduces the upper bound on the influence of other agents' actions which , effectively, increases the difficulty of the game.



Note that \cite{facchinei201012} also provides a sufficient condition for convergence of the best response algorithm, that involves the spectral norm of a matrix composed of parameters related to the second-order partial derivative of the cost function.
%
In this work, we analyze the best response algorithm from a different perspective that relies on strong monotonicity to characterize  convergence.
In simple cases such as two-player potential games, it is easy to show that our condition is equivalent to the condition in \cite{facchinei201012}.
However, in general, strong monotonicity  provides a more intuitive condition for convergence. Finally, we experimentally show that when the condition $m>L\sqrt{N-1}$ does not hold, the best-response algorithm may lead to cycles. This result further validates the utility of the proposed condition. 

Proposition \ref{prop:BR} shows that the best response algorithm converges to the Nash equilibrium at an exponential rate. Indeed, it is a no-regret learning algorithm for each agent as well, as shown in the following proposition.

\begin{proposition}\label{prop:BR:no_regret}
Suppose that the game $\mathcal{G}$ is $m$-strongly monotone with parameter $m>L \sqrt{N-1}$, the cost $C_i(x_i,x_{-i})$ is $L_0$-Lipschitz continuous in $x_{-i}$ for every $x_i \in \mathcal{X}_i$, and the diameter of the convex set $\mathcal{X}_i$ is bounded by $D$, for all $i=1,\ldots,N$ Then, the static regret of the best response algorithm satisfies
\begin{align*}
    {\rm{SR}}_i(T) \leq  \sum_{t=1}^T \mathcal{C}_i(x_t) -  \sum_{t=1}^T \min_{x_i}\mathcal{C}_i(x_i,x_{-i,t}) = \mathcal{O}(1).
\end{align*}
\end{proposition}

\begin{proof}
The first inequality holds due to the fact that $\sum_{t=1}^T \min_{x_i}\mathcal{C}_i(x_i,x_{-i,t})\leq \min_{x_i}\sum_{t=1}^T \mathcal{C}_i(x_i,x_{-i,t})$.
% Let $y_i^{*} = \mathop{\rm{argmin}}_{x_i} \sum_{t=1}^T \mathcal{C}_i(x_i,x_{-i,t})$.
% %
% Recalling that $x_{i,t+1} = \mathop{\rm{arg min}}_{x_i} \mathcal{C}_i (x_i, x_{-i,t})$, we have that $\langle \nabla_i \mathcal{C}_i (x_{i,t+1},x_{-i,t}),y_i^{*}  - x_{i,t+1} \rangle \geq 0$. From the definition of regret in \eqref{eq:def:regret:game}, we have that 
% \begin{align*}
%     {\rm{SR}}_i(T) =& \sum_{t=1}^T \Big( \mathcal{C}_i(x_t) - \mathcal{C}_i(x_{i,t+1},x_{-i,t}) \nonumber \\
%     &+ \mathcal{C}_i(x_{i,t+1},x_{-i,t})  -  \mathcal{C}_i(y_i^{*},x_{-i,t})\Big) \nonumber \\
%     \leq & \sum_{t=1}^T \Big( \mathcal{C}_i(x_t) - \mathcal{C}_i(x_{i,t+1},x_{-i,t}) \Big) \nonumber \\
%     &+ \sum_{t=1}^T \langle \nabla_i \mathcal{C}_i(x_{i,t+1},x_{-i,t}),x_{i,t+1} - y_i^{*} \rangle \nonumber \\
%     \leq & \sum_{t=1}^T \Big( \mathcal{C}_i(x_t) - \mathcal{C}_i(x_{i,t+1},x_{-i,t}) \Big) ,
% \end{align*}
% where the first inequality is due to the convexity of the loss function $\mathcal{C}_i(x)$ with respect to $x_i$ and the second inequality follows from the necessary condition of optimality. 
Observe that $\mathcal{C}_i(x_{i,t+1},x_{-i,t})=\min_{x_i}\mathcal{C}_i(x_i,x_{-i,t})$ since $x_{i,t+1}=\mathop{\rm{arg min}}_{x_i \in \mathcal{X}_i} \mathcal{C}_{i} (x_i, x_{-i,t})$. Then, it follows that
%
\begin{align}\label{eq:BR_no_temp1}
    &{\rm{SR}}_i(T) \leq \sum_{t=1}^T \mathcal{C}_i(x_t) -  \sum_{t=1}^T \min_{x_i}\mathcal{C}_i(x_i,x_{-i,t}) \nonumber\\
    & = \sum_{t=1}^T \Big( \mathcal{C}_i(x_t)- \mathcal{C}_i(x_{t+1})+ \mathcal{C}_i(x_{t+1}) - \mathcal{C}_i(x_{i,t+1},x_{-i,t}) \Big) \nonumber \\
    & \leq \mathcal{C}_i(x_1) + \sum_{t=1}^T \Big(\mathcal{C}_i(x_{t+1}) - \mathcal{C}_i(x_{i,t+1},x_{-i,t}) \Big) \nonumber \\
    & \leq  \mathcal{C}_i(x_1) + L_0 \sum_{t=1}^T \left\|x_{-i,t+1} - x_{-i,t}\right\| \nonumber \\
    & \leq  \mathcal{C}_i(x_1) + L_0 \sum_{t=1}^T \left\|x_{t+1} - x_{t}\right\|,
\end{align}
where the second to the last inequality follows from the Lipschitz continuous property of the function $\mathcal{C}_i$ in $x$.
By virtue of \eqref{eq:BR:convergence} in Proposition 1, we have
\begin{align}\label{eq:BR_no_temp2}
    &\left\|x_{t+1} - x_{t}\right\|^2 = \left\|x_{t+1} -x^{*}+x^{*} -x_{t}\right\|^2 \nonumber \\
    &\leq  2 \left\|x_{t+1} -x^{*}\right\|^2 + 2 \left\|x^{*} -x_{t}\right\|^2 
    \leq  2(\rho^2+1)\left\|x_{t}-x^{*} \right\|^2.
\end{align}
Substituting the inequality \eqref{eq:BR_no_temp2} into \eqref{eq:BR_no_temp1}, we have 
\begin{align}\label{eq:BR_no_temp3}
    {\rm{SR}}_i(T) 
    \leq & \mathcal{C}_i(x_1) + L_0 \sum_{t=1}^T \sqrt{2(\rho^2+1)} \left\|x_{t} - x^{*}\right\|  \nonumber \\
    \leq &  \mathcal{C}_i(x_1) + L_0 \sqrt{2(\rho^2+1)} \sum_{t=1}^T \rho^t D \nonumber \\
    \leq & \mathcal{C}_i(x_1)+\frac{ DL_0 \sqrt{2(\rho^2+1)}} {1-\rho},
\end{align}
which completes the proof.
\end{proof}

Proposition \ref{prop:BR:no_regret} indeed provides a stronger bound than the static regret defined in \eqref{eq:def:regret:game}. Instead of comparing to a single best action in hindsight, it compares with a sequence of episode-wise best actions, which is equivalent to the dynamic regret with time-invariant cost functions. This strong result own itself to the best response algorithm.


\section{Time-varying games}\label{sec:TV_BR}
In this section, we analyze time-varying games $\mathcal{G}_t$ where the cost functions of the agents change over time. Since the equilibrium of these games also varies, in what follows we analyze the ability of the best response algorithm \eqref{eq:TVBR:update} to generate actions that track the evolving equilibrium. 

If the game $\mathcal{G}_t$ changes significantly, it is reasonable to expect that it will be hard to track the evolving equilibrium. Therefore, as in related literature \cite{duvocelle2022multiagent,zhang2022no}, we assume that both the equilibrium variation $V_T$ in \eqref{eq:def:VT} and the function variation $W_{i,T}$ in \eqref{eq:def:WT} are sub-linear in $T$, for $i=1,\ldots,N$.
% i.e., $V_T = \mathcal{O}(T)$ and $\mathcal{W}_{i,T}=\mathcal{O}(T)$.

In what follows, we analyze the equilibrium tracking error of the best response algorithm \eqref{eq:TVBR:update} in terms of the equilibrium variation.
\begin{theorem}\label{theorem:BRTV}
Suppose that the time-varying game $\mathcal{G}_t$ is $m_t$-strongly monotone and $\nabla_i \mathcal{C}_{i,t}(x_i,x_{-i})$ is $L_t$-Lipschitz continuous in $x_{-i}$ for every $x_i \in \mathcal{X}_i$ with parameter $m_t>L_t \sqrt{N-1}$, for $\forall t$. Then, the best response algorithm \eqref{eq:TVBR:update} satisfies that
\begin{align}\label{eq:BRTV:convergence}
    {\rm{Err}}(T) \leq \frac{\left\| x_{1}- x_{1}^{*}\right\|^2}{1-\rho_m} + \frac{V_T}{(1-\rho_m)^2}=\mathcal{O}\left( 1+ V_T\right),
\end{align}
where $\rho_m:= \mathop{\rm{max}}_t \left\{ \frac{L_t\sqrt{N-1}}{m_t} \right\}$.
\end{theorem}

\begin{proof}
Applying the same arguments as in Proposition \ref{prop:BR} to the cost function $\mathcal{C}_{i,t}$, we can obtain an inequality similar to \eqref{eq:BR_temp5} as
\begin{align}\label{eq:BRTV_temp4}
    \left\| x_{t+1}-x_{t}^{*}  \right\| 
    \leq  \rho_t  \left\| x_{t} - x_{t}^{*}\right\|,
\end{align}
where $\rho_t:=  \frac{L_t\sqrt{N-1}}{m_t} $.
Observe that 
\begin{align*}
    &\left\| x_{t+1}-x_{t+1}^{*}  \right\|^2 = \left\| x_{t+1}- x_{t}^{*} + x_{t}^{*}- x_{t+1}^{*}  \right\|^2 \nonumber \\
    \leq &(1+\lambda) \left\| x_{t+1}- x_{t}^{*}\right\|^2 + (1+\frac{1}{\lambda}) \left\|x_{t}^{*}- x_{t+1}^{*}  \right\|^2,
\end{align*}
for $\forall \lambda >0$. Setting $\lambda = \frac{1}{\rho_t}-1>0$ yields 
\begin{align}\label{eq:BRTV_temp5}
    &\left\| x_{t+1}-x_{t+1}^{*}  \right\|^2 
    \leq \frac{1}{\rho_t } \left\| x_{t+1}- x_{t}^{*}\right\|^2 + \frac{1}{1-\rho_t } \left\|x_{t}^{*}- x_{t+1}^{*}  \right\|^2 \nonumber \\
    & \leq  \rho_t \left\| x_{t}- x_{t}^{*}\right\|^2 + \frac{1}{1-\rho_t } \left\|x_{t}^{*}- x_{t+1}^{*}  \right\|^2 \nonumber \\
    & \leq  \rho_m \left\| x_{t}- x_{t}^{*}\right\|^2 + \frac{1}{1-\rho_m } \left\|x_{t}^{*}- x_{t+1}^{*}  \right\|^2,
\end{align}
where the second inequality follows from \eqref{eq:BRTV_temp4} and the last inequality is due to the fact that $\rho_t \leq \rho_m <1$. Rearranging and summing \eqref{eq:BRTV_temp5} over $t=1,\ldots,T$, we have that
\begin{align*}
    &(1-\rho_m) \sum_{t=1}^T \left\| x_{t}- x_{t}^{*}\right\|^2 \nonumber \\
    \leq & \sum_{t=1}^T \left(\left\| x_{t}- x_{t}^{*}\right\|^2 - \left\| x_{t+1}-x_{t+1}^{*}  \right\|^2 + \frac{\left\|x_{t}^{*}- x_{t+1}^{*}  \right\|^2}{1-\rho_m } \right) \nonumber \\
    \leq & \left\| x_{1}- x_{1}^{*}\right\|^2 +\frac{1}{1-\rho_m } \sum_{t=1}^T\left\|x_{t}^{*}- x_{t+1}^{*}  \right\|^2 \nonumber \\
    \leq & \left\| x_{1}- x_{1}^{*}\right\|^2 +  \frac{1}{1-\rho_m } V_T.
\end{align*}
Dividing both sides of the above inequality by $(1-\rho_m)$ completes the proof.
\end{proof}

Theorem~\ref{theorem:BRTV} shows that $V_T$ dominates the equilibrium tracking error. If $V_T$ is sub-linear in $T$, so is the equilibrium tracking error.
%
In what follows, we analyze the dynamic regret of each agent in terms of the equilibrium variation and the function variation.
\begin{theorem}\label{theorem:BRTV:no_regret}
Suppose that the time-varying game $\mathcal{G}_t$ is $m_t$-strongly monotone, $\nabla_i \mathcal{C}_{i,t}(x_i,x_{-i})$ is $L_t$-Lipschitz continuous in $x_{-i}$ for every $x_i \in \mathcal{X}_i$ with parameter $m_t>L_t \sqrt{N-1}$, and the cost $\mathcal{C}_{i,t}(x)$ is $L_0$-Lipschitz continuous in $x_{-i}$ for every $x_i \in \mathcal{X}_i$ for $\forall t$. Then, the dynamic regret of the best response algorithm \eqref{eq:TVBR:update} satisfies
\begin{align}
    {\rm{DR}}_i(T)  = \mathcal{O}\left( W_{i,t} +\sqrt{TV_T} \right), \; i=1,\ldots,N.
\end{align}
\end{theorem}

\begin{proof}
Using the update rule of the best response algorithm \eqref{eq:TVBR:update}, we have 
\begin{align}
    &{\rm{DR}}_i(T) = \sum_{t=1}^T \Big( \mathcal{C}_{i,t}(x_t) -\mathcal{C}_{i,t}(x_{i,t+1},x_{-i,t})\Big) \nonumber \\
    =& \sum_{t=1}^T \Big( \mathcal{C}_{i,t}(x_t) - \mathcal{C}_{i,t+1}(x_{t+1}) + \mathcal{C}_{i,t+1}(x_{t+1}) \nonumber \\ 
    &- \mathcal{C}_{i,t}(x_{t+1}) + \mathcal{C}_{i,t}(x_{t+1}) - \mathcal{C}_{i,t}(x_{i,t+1},x_{-i,t}) \Big) \nonumber \\ 
    \leq & \mathcal{C}_{i,1}(x_1) +  W_{i,T} + \sum_{t=1}^T \Big(\mathcal{C}_{i,t}(x_{t+1}) - \mathcal{C}_{i,t}(x_{i,t+1},x_{-i,t}) \Big) \nonumber \\ 
    \leq & \mathcal{C}_{i,1}(x_1)+ W_{i,T} + L_0 \sum_{t=1}^T\left\| x_{-i,t+1}-x_{-i,t}\right\| \nonumber \\ 
    \leq & \mathcal{C}_{i,1}(x_1)+ W_{i,T} + L_0 \sum_{t=1}^T \left\| x_{t+1}-x_{t}\right\|.
\end{align}
%
Using the inequality \eqref{eq:BRTV_temp4} and the fact that $\rho_t \leq \rho_m <1$, we have
\begin{align*}
    &\sum_{t=1}^T\left\| x_{t+1}-x_{t}\right\|^2 =\sum_{t=1}^T\left\| x_{t+1}- x_t^{*} +x_t^{*} - x_{t}\right\|^2 \nonumber \\
    & \leq \sum_{t=1}^T \big( (1+\frac{1}{\rho_m}) \left\| x_{t+1}- x_t^{*}\right\|^2 + (1+\rho_m)  \left\|  x_t^{*} - x_{t}\right\|^2 \big) \nonumber \\
    & \leq (\rho_m+1)^2 \sum_{t=1}^T \left\|  x_{t} - x_t^{*} \right\|^2,
\end{align*}
which further yields
\begin{align}
    &{\rm{DR}}_i(T)  \nonumber \\
    \leq & \mathcal{C}_{i,1}(x_1)+ W_{i,T} + L_0 \sqrt{T} \sqrt{\sum_{t=1}^T \left\| x_{t+1}-x_{t}\right\|^2} \nonumber \\
    \leq &  \mathcal{C}_{i,1}(x_1)+ W_{i,T} + L_0 \sqrt{T} \sqrt{(\rho_m+1)^2 \sum_{t=1}^T \left\|  x_{t} - x_t^{*} \right\|^2} \nonumber \\
    =& \mathcal{O}\left( W_{i,t} +\sqrt{TV_T} \right),
\end{align}
where in the last inequality we use the results from Theorem~\ref{theorem:BRTV}.
The proof is complete.
\end{proof}

Theorem~\ref{theorem:BRTV:no_regret} shows that the dynamic regret is sublinear in $T$ if the variation of the game satisfies $W_{i,T}=\mathcal{O}(T^a)$ and $V_{T}=\mathcal{O}(T^b)$ with $a,b\in[0,1)$. 
%
\begin{remark}
(Connection between dynamic regret and equilibrium tracking error). In the single agent case, equilibrium tracking error is equivalent to the dynamic regret. However, this is not true for games involving multiple agents. This is due to the fact that the function $\mathcal{C}_{i,t}(\cdot,x_{-i,t})$ is time-varying due to changes in the function $\mathcal{C}_{i,t}$ itself and changes in other agents' actions $x_{-i,t}$.
To see this, consider the class of time-varying games with time-varying cost functions but constant equilibrium, i.e., $V_T=0$, $W_{i,T}=\mathcal{O}(T^a)$ for some $a>0$. In this case, we have $ {\rm{Err}}(T) = \mathcal{O}(1)$ but  ${\rm{DR}}_i(T) = \mathcal{O}(T^a)$.
\end{remark}


\section{Numerical Experiments}\label{sec:simulation}
In this section, we validate our analysis on a Cournot game for both time-invariant and time-varying losses.
\subsection{Time-invariant game}
We first focus on the time-invariant case.
We consider a Cournot game with two agents whose goal is to minimize their local losses by appropriately setting the production quantity $x_i$, $i=1,2$.
The loss function of each agent is given by $\mathcal{C}_i(x) = x_i(\frac{a_i x_i}{2} + b_i x_{-i} - e_i)+ 1$, where $a_i>0$ , $b_i$, $e_i$ are constant parameters, and $x_{-i}$ denotes the production quantity of the opponent of agent~$i$. 
It is easy to show that $\nabla_i \mathcal{C}_i(x) = a_i x_i + b_i x_{-i} -e_i$. Recalling that $L$ is the Lipschitz constant of the function $\nabla_i \mathcal{C}_i(x)$ with respect to $x_{-i}$, we have $L={\rm{max}}\{ |b_1|,|b_2|\}$.
%
Define $g(x) = (\nabla_1 \mathcal{C}_1(x), \nabla_2 \mathcal{C}_2(x))$ and let $G(x)$ denote the Jacobian of $g(x)$, i.e., $G(x)=[a_1,b_1;b_2,a_2]$. 
According to \cite{rosen1965existence}, the strong monotonicity parameter $m$  coincides with the smallest eigenvalue of the matrix $\frac{G(x)+G'(x)}{2}$. 

We validate our methods for three different selections of parameters $\theta^k:=(a_1^k,a_2^k,b_1^k,b_2^k,e_1^k,e_2^k)$ for $k=1,2,3$. Specifically, We select $\theta^1 = (1,1,0.6,-0.5,1.2,0.8)$, $\theta^2 = (1,1,1,-1,1.2,0.8)$ and $\theta^3 = (1,1,2,-1,1.2,0.8)$. It is easy to verify that $\theta^1$, $\theta^2$ and $\theta^3$ correspond to the cases $m>L\sqrt{N-1}$, $m=L\sqrt{N-1}$, and $m<L\sqrt{N-1}$, respectively. The convergence results are shown in Figure~\ref{fig_TI}. We observe that when $m>L\sqrt{N-1}$, the best response converges with exponential rate. When $m\leq L\sqrt{N-1}$, the best response algorithm fails to converge, which indicates the tightness of our theoretical results.

\begin{figure}[t] 
\begin{center}
\centerline{\includegraphics[width=0.9\columnwidth]{Time-invariant.pdf}}
\caption{Convergence of the best response algorithm for time-invariant games.}
\label{fig_TI}
\end{center}
\vskip -0.2in
\end{figure}



\subsection{Time-varying games}
For the time-varying case, the loss function of agent $i$ is defined as $\mathcal{C}_{i,t}(x) = x_i(\frac{a_i x_i}{2} + b_{i,t} x_{-i} - e_{i,t})+ 1$, where $a_i=2$, $i=1,2$, and $b_{i,t}$, $e_{i,t}$ are time-varying parameters. The time-varying parameters are selected as 
\begin{align*}
    b_{i,t}=\left\{ \begin{array}{cc} 0.3+ 0.1\times (-1)^t  & t\in[1,T^{0.6}] \\
    0.3 & t\in(T^{0.6},T] \end{array},\right.
\end{align*}
\begin{align*}
    e_{i,t}=\left\{ \begin{array}{cc} 0.4 & t\in[1,T^{0.6}] \\
    0.4+ 0.1\times (-1)^t t^{-1/4}  & t\in(T^{0.6},T] \end{array}.\right.
\end{align*}
%
We select $T=1000$ and thus $T^{0.6}\approx63$.
It can be verified that the selection of parameters yields $m_t\geq L_t \sqrt{N-1}$ for $\forall t$, and $V_T = \mathcal{O}(T^{3/4})$, $W_{i,T} = \mathcal{O}(T^{3/4})$, $i=1,2$. Figures \ref{fig_tracking}--\ref{fig_DR}  illustrate the equilibrium tracking  error and the dynamic regret of the best response algorithm, respectively. We observe that, when $t\in[1,T^{0.6}]$, both the equilibrium tracking error and the dynamic regret grow rapidly due to the oscillations of $b_{i,t}$; when $t\in(T^{0.6},T]$, they grow slowly since $b_{i,t}$ is a constant and the variation of $e_{i,t}$ is decreasing over time.
Moreover, both the equilibrium tracking error and the dynamic regret are sub-linear in the total number of episodes, which supports our theoretical results.


\begin{figure}[t]
\begin{center}
\centerline{\includegraphics[width=0.9\columnwidth]{Tracking.pdf}}
\caption{Equilibrium tracking error of the best response algorithm for time-varying games. }
\label{fig_tracking}
\end{center}
\vskip -0.3in
\end{figure}

\begin{figure}[t]
\begin{center}
\centerline{\includegraphics[width=0.9\columnwidth]{DR.pdf}}
\caption{Dynamic regret of the best response algorithm for time-varying games.}
\label{fig_DR}
\end{center}
\vskip -0.3in
\end{figure}

\section{Conclusion}\label{sec:conclusion}
In this work, we analyzed the best response algorithm for the class of strongly monotone games. We first considered standard time-invariant games and obtained a sufficient condition under which the best response algorithm converges at an exponential rate.
%
We provided numerical experiments that showed the best response algorithm can diverge if this condition fails to hold, which indicates that the condition is tight. Subsequently, we analyzed the best response algorithm for time-varying games with evolving equilibria. We showed that the equilibrium tracking error and the dynamic regret can be bounded in terms of the variations of evolving equilibria and loss functions. 
Moreover, we provided additional numerical simulations to verify our results.


% \section{INTRODUCTION}

% This template provides authors with most of the formatting specifications needed for preparing electronic versions of their papers. All standard paper components have been specified for three reasons: (1) ease of use when formatting individual papers, (2) automatic compliance to electronic requirements that facilitate the concurrent or later production of electronic products, and (3) conformity of style throughout a conference proceedings. Margins, column widths, line spacing, and type styles are built-in; examples of the type styles are provided throughout this document and are identified in italic type, within parentheses, following the example. Some components, such as multi-leveled equations, graphics, and tables are not prescribed, although the various table text styles are provided. The formatter will need to create these components, incorporating the applicable criteria that follow.

% \section{PROCEDURE FOR PAPER SUBMISSION}

% \subsection{Selecting a Template (Heading 2)}

% First, confirm that you have the correct template for your paper size. This template has been tailored for output on the US-letter paper size. 
% It may be used for A4 paper size if the paper size setting is suitably modified.

% \subsection{Maintaining the Integrity of the Specifications}

% The template is used to format your paper and style the text. All margins, column widths, line spaces, and text fonts are prescribed; please do not alter them. You may note peculiarities. For example, the head margin in this template measures proportionately more than is customary. This measurement and others are deliberate, using specifications that anticipate your paper as one part of the entire proceedings, and not as an independent document. Please do not revise any of the current designations

% \section{MATH}

% Before you begin to format your paper, first write and save the content as a separate text file. Keep your text and graphic files separate until after the text has been formatted and styled. Do not use hard tabs, and limit use of hard returns to only one return at the end of a paragraph. Do not add any kind of pagination anywhere in the paper. Do not number text heads-the template will do that for you.

% Finally, complete content and organizational editing before formatting. Please take note of the following items when proofreading spelling and grammar:

% \subsection{Abbreviations and Acronyms} Define abbreviations and acronyms the first time they are used in the text, even after they have been defined in the abstract. Abbreviations such as IEEE, SI, MKS, CGS, sc, dc, and rms do not have to be defined. Do not use abbreviations in the title or heads unless they are unavoidable.

% \subsection{Units}

% \begin{itemize}

% \item Use either SI (MKS) or CGS as primary units. (SI units are encouraged.) English units may be used as secondary units (in parentheses). An exception would be the use of English units as identifiers in trade, such as Ò3.5-inch disk driveÓ.
% \item Avoid combining SI and CGS units, such as current in amperes and magnetic field in oersteds. This often leads to confusion because equations do not balance dimensionally. If you must use mixed units, clearly state the units for each quantity that you use in an equation.
% \item Do not mix complete spellings and abbreviations of units: ÒWb/m2Ó or Òwebers per square meterÓ, not Òwebers/m2Ó.  Spell out units when they appear in text: Ò. . . a few henriesÓ, not Ò. . . a few HÓ.
% \item Use a zero before decimal points: Ò0.25Ó, not Ò.25Ó. Use Òcm3Ó, not ÒccÓ. (bullet list)

% \end{itemize}


% \subsection{Equations}

% The equations are an exception to the prescribed specifications of this template. You will need to determine whether or not your equation should be typed using either the Times New Roman or the Symbol font (please no other font). To create multileveled equations, it may be necessary to treat the equation as a graphic and insert it into the text after your paper is styled. Number equations consecutively. Equation numbers, within parentheses, are to position flush right, as in (1), using a right tab stop. To make your equations more compact, you may use the solidus ( / ), the exp function, or appropriate exponents. Italicize Roman symbols for quantities and variables, but not Greek symbols. Use a long dash rather than a hyphen for a minus sign. Punctuate equations with commas or periods when they are part of a sentence, as in

% $$
% \alpha + \beta = \chi \eqno{(1)}
% $$

% Note that the equation is centered using a center tab stop. Be sure that the symbols in your equation have been defined before or immediately following the equation. Use Ò(1)Ó, not ÒEq. (1)Ó or Òequation (1)Ó, except at the beginning of a sentence: ÒEquation (1) is . . .Ó

% \subsection{Some Common Mistakes}
% \begin{itemize}


% \item The word ÒdataÓ is plural, not singular.
% \item The subscript for the permeability of vacuum ?0, and other common scientific constants, is zero with subscript formatting, not a lowercase letter ÒoÓ.
% \item In American English, commas, semi-/colons, periods, question and exclamation marks are located within quotation marks only when a complete thought or name is cited, such as a title or full quotation. When quotation marks are used, instead of a bold or italic typeface, to highlight a word or phrase, punctuation should appear outside of the quotation marks. A parenthetical phrase or statement at the end of a sentence is punctuated outside of the closing parenthesis (like this). (A parenthetical sentence is punctuated within the parentheses.)
% \item A graph within a graph is an ÒinsetÓ, not an ÒinsertÓ. The word alternatively is preferred to the word ÒalternatelyÓ (unless you really mean something that alternates).
% \item Do not use the word ÒessentiallyÓ to mean ÒapproximatelyÓ or ÒeffectivelyÓ.
% \item In your paper title, if the words Òthat usesÓ can accurately replace the word ÒusingÓ, capitalize the ÒuÓ; if not, keep using lower-cased.
% \item Be aware of the different meanings of the homophones ÒaffectÓ and ÒeffectÓ, ÒcomplementÓ and ÒcomplimentÓ, ÒdiscreetÓ and ÒdiscreteÓ, ÒprincipalÓ and ÒprincipleÓ.
% \item Do not confuse ÒimplyÓ and ÒinferÓ.
% \item The prefix ÒnonÓ is not a word; it should be joined to the word it modifies, usually without a hyphen.
% \item There is no period after the ÒetÓ in the Latin abbreviation Òet al.Ó.
% \item The abbreviation Òi.e.Ó means Òthat isÓ, and the abbreviation Òe.g.Ó means Òfor exampleÓ.

% \end{itemize}


% \section{USING THE TEMPLATE}

% Use this sample document as your LaTeX source file to create your document. Save this file as {\bf root.tex}. You have to make sure to use the cls file that came with this distribution. If you use a different style file, you cannot expect to get required margins. Note also that when you are creating your out PDF file, the source file is only part of the equation. {\it Your \TeX\ $\rightarrow$ PDF filter determines the output file size. Even if you make all the specifications to output a letter file in the source - if your filter is set to produce A4, you will only get A4 output. }

% It is impossible to account for all possible situation, one would encounter using \TeX. If you are using multiple \TeX\ files you must make sure that the ``MAIN`` source file is called root.tex - this is particularly important if your conference is using PaperPlaza's built in \TeX\ to PDF conversion tool.

% \subsection{Headings, etc}

% Text heads organize the topics on a relational, hierarchical basis. For example, the paper title is the primary text head because all subsequent material relates and elaborates on this one topic. If there are two or more sub-topics, the next level head (uppercase Roman numerals) should be used and, conversely, if there are not at least two sub-topics, then no subheads should be introduced. Styles named ÒHeading 1Ó, ÒHeading 2Ó, ÒHeading 3Ó, and ÒHeading 4Ó are prescribed.

% \subsection{Figures and Tables}

% Positioning Figures and Tables: Place figures and tables at the top and bottom of columns. Avoid placing them in the middle of columns. Large figures and tables may span across both columns. Figure captions should be below the figures; table heads should appear above the tables. Insert figures and tables after they are cited in the text. Use the abbreviation ÒFig. 1Ó, even at the beginning of a sentence.

% \begin{table}[h]
% \caption{An Example of a Table}
% \label{table_example}
% \begin{center}
% \begin{tabular}{|c||c|}
% \hline
% One & Two\\
% \hline
% Three & Four\\
% \hline
% \end{tabular}
% \end{center}
% \end{table}


%   \begin{figure}[thpb]
%       \centering
%       \framebox{\parbox{3in}{We suggest that you use a text box to insert a graphic (which is ideally a 300 dpi TIFF or EPS file, with all fonts embedded) because, in an document, this method is somewhat more stable than directly inserting a picture.
% }}
%       %\includegraphics[scale=1.0]{figurefile}
%       \caption{Inductance of oscillation winding on amorphous
%       magnetic core versus DC bias magnetic field}
%       \label{figurelabel}
%   \end{figure}
   

% Figure Labels: Use 8 point Times New Roman for Figure labels. Use words rather than symbols or abbreviations when writing Figure axis labels to avoid confusing the reader. As an example, write the quantity ÒMagnetizationÓ, or ÒMagnetization, MÓ, not just ÒMÓ. If including units in the label, present them within parentheses. Do not label axes only with units. In the example, write ÒMagnetization (A/m)Ó or ÒMagnetization {A[m(1)]}Ó, not just ÒA/mÓ. Do not label axes with a ratio of quantities and units. For example, write ÒTemperature (K)Ó, not ÒTemperature/K.Ó

% \section{CONCLUSIONS}

% A conclusion section is not required. Although a conclusion may review the main points of the paper, do not replicate the abstract as the conclusion. A conclusion might elaborate on the importance of the work or suggest applications and extensions. 

% \addtolength{\textheight}{-12cm}   % This command serves to balance the column lengths
%                                   % on the last page of the document manually. It shortens
%                                   % the textheight of the last page by a suitable amount.
%                                   % This command does not take effect until the next page
%                                   % so it should come on the page before the last. Make
%                                   % sure that you do not shorten the textheight too much.

% %%%%%%%%%%%%%%%%%%%%%%%%%%%%%%%%%%%%%%%%%%%%%%%%%%%%%%%%%%%%%%%%%%%%%%%%%%%%%%%%



% %%%%%%%%%%%%%%%%%%%%%%%%%%%%%%%%%%%%%%%%%%%%%%%%%%%%%%%%%%%%%%%%%%%%%%%%%%%%%%%%



% %%%%%%%%%%%%%%%%%%%%%%%%%%%%%%%%%%%%%%%%%%%%%%%%%%%%%%%%%%%%%%%%%%%%%%%%%%%%%%%%
% \section*{APPENDIX}

% Appendixes should appear before the acknowledgment.

% \section*{ACKNOWLEDGMENT}

% The preferred spelling of the word ÒacknowledgmentÓ in America is without an ÒeÓ after the ÒgÓ. Avoid the stilted expression, ÒOne of us (R. B. G.) thanks . . .Ó  Instead, try ÒR. B. G. thanksÓ. Put sponsor acknowledgments in the unnumbered footnote on the first page.



% %%%%%%%%%%%%%%%%%%%%%%%%%%%%%%%%%%%%%%%%%%%%%%%%%%%%%%%%%%%%%%%%%%%%%%%%%%%%%%%%

% References are important to the reader; therefore, each citation must be complete and correct. If at all possible, references should be commonly available publications.



% \begin{thebibliography}{99}

% \bibitem{c1} G. O. Young, ÒSynthetic structure of industrial plastics (Book style with paper title and editor),Ó 	in Plastics, 2nd ed. vol. 3, J. Peters, Ed.  New York: McGraw-Hill, 1964, pp. 15Ð64.
% \bibitem{c2} W.-K. Chen, Linear Networks and Systems (Book style).	Belmont, CA: Wadsworth, 1993, pp. 123Ð135.
% \bibitem{c3} H. Poor, An Introduction to Signal Detection and Estimation.   New York: Springer-Verlag, 1985, ch. 4.
% \bibitem{c4} B. Smith, ÒAn approach to graphs of linear forms (Unpublished work style),Ó unpublished.
% \bibitem{c5} E. H. Miller, ÒA note on reflector arrays (Periodical styleÑAccepted for publication),Ó IEEE Trans. Antennas Propagat., to be publised.
% \bibitem{c6} J. Wang, ÒFundamentals of erbium-doped fiber amplifiers arrays (Periodical styleÑSubmitted for publication),Ó IEEE J. Quantum Electron., submitted for publication.
% \bibitem{c7} C. J. Kaufman, Rocky Mountain Research Lab., Boulder, CO, private communication, May 1995.
% \bibitem{c8} Y. Yorozu, M. Hirano, K. Oka, and Y. Tagawa, ÒElectron spectroscopy studies on magneto-optical media and plastic substrate interfaces(Translation Journals style),Ó IEEE Transl. J. Magn.Jpn., vol. 2, Aug. 1987, pp. 740Ð741 [Dig. 9th Annu. Conf. Magnetics Japan, 1982, p. 301].
% \bibitem{c9} M. Young, The Techincal Writers Handbook.  Mill Valley, CA: University Science, 1989.
% \bibitem{c10} J. U. Duncombe, ÒInfrared navigationÑPart I: An assessment of feasibility (Periodical style),Ó IEEE Trans. Electron Devices, vol. ED-11, pp. 34Ð39, Jan. 1959.
% \bibitem{c11} S. Chen, B. Mulgrew, and P. M. Grant, ÒA clustering technique for digital communications channel equalization using radial basis function networks,Ó IEEE Trans. Neural Networks, vol. 4, pp. 570Ð578, July 1993.
% \bibitem{c12} R. W. Lucky, ÒAutomatic equalization for digital communication,Ó Bell Syst. Tech. J., vol. 44, no. 4, pp. 547Ð588, Apr. 1965.
% \bibitem{c13} S. P. Bingulac, ÒOn the compatibility of adaptive controllers (Published Conference Proceedings style),Ó in Proc. 4th Annu. Allerton Conf. Circuits and Systems Theory, New York, 1994, pp. 8Ð16.
% \bibitem{c14} G. R. Faulhaber, ÒDesign of service systems with priority reservation,Ó in Conf. Rec. 1995 IEEE Int. Conf. Communications, pp. 3Ð8.
% \bibitem{c15} W. D. Doyle, ÒMagnetization reversal in films with biaxial anisotropy,Ó in 1987 Proc. INTERMAG Conf., pp. 2.2-1Ð2.2-6.
% \bibitem{c16} G. W. Juette and L. E. Zeffanella, ÒRadio noise currents n short sections on bundle conductors (Presented Conference Paper style),Ó presented at the IEEE Summer power Meeting, Dallas, TX, June 22Ð27, 1990, Paper 90 SM 690-0 PWRS.
% \bibitem{c17} J. G. Kreifeldt, ÒAn analysis of surface-detected EMG as an amplitude-modulated noise,Ó presented at the 1989 Int. Conf. Medicine and Biological Engineering, Chicago, IL.
% \bibitem{c18} J. Williams, ÒNarrow-band analyzer (Thesis or Dissertation style),Ó Ph.D. dissertation, Dept. Elect. Eng., Harvard Univ., Cambridge, MA, 1993. 
% \bibitem{c19} N. Kawasaki, ÒParametric study of thermal and chemical nonequilibrium nozzle flow,Ó M.S. thesis, Dept. Electron. Eng., Osaka Univ., Osaka, Japan, 1993.
% \bibitem{c20} J. P. Wilkinson, ÒNonlinear resonant circuit devices (Patent style),Ó U.S. Patent 3 624 12, July 16, 1990. 






% \end{thebibliography}

\bibliography{ref}
\bibliographystyle{IEEEtran}


\end{document}
