% ar-sample-1col.tex, dated 30th Mar. 2013
% This is a sample file for AR journals
%
% Compilation using 'ar.cls' - version 1.0, Aptara Inc.
% (c) 2013 AR
%
% Steps to compile: latex latex latex
%
% For tracking purposes => this is v1.0 - Mar. 2013

%\documentclass{ar-1col}
%
%\usepackage{rotating}%
%\usepackage{subfigure}
%\usepackage{natbib}
%\usepackage{aas_macros}
%\bibliographystyle{ar-style2}
%
%
%\usepackage[numbers]{natbib}%
%
% \usepackage[T1]{fontenc}
% \usepackage[latin1]{inputenc}
% \usepackage{babel}
% \usepackage[font=small,labelfont=bf,tableposition=top]{caption}
% \usepackage[font=footnotesize]{subfig}
% \usepackage{blindtext}
% 
% Metadata Information
%\jname{Xxxx. Xxx. Xxx. Xxx.}
%\jvol{00}
%\jyear{YYYY}
%\doi{10.1146/((please add article doi))}

%%%%%%%%%%%%%%%%%%%%%%%%%%%%%%%%%%%%%%%%
% \def\simlt{\lower.5ex\hbox{$\; \buildrel < \over \sim \;$}}
% \def\simgt{\lower.5ex\hbox{$\; \buildrel > \over \sim \;$}}
\def\XL{\textcolor{red}}
\def\AB{\textcolor{black}}

\def\sT{\sigma_{\rm T}}
\def\dd{{\rm d}}

\def\be{\begin{equation}}
\def\ee{\end{equation}}
\def\beq{\begin{eqnarray}}
\def\eeq{\end{eqnarray}}
\def\bA{{\,\mathbf A}}
\def\bB{{\,\mathbf B}}
\def\bE{{\,\mathbf E}}
\def\bj{{\,\mathbf j}}
\def\bv{{\,\mathbf v}}
\def\jB{j_B}
\def\RNS{R}
\def\Rmax{R_{\rm max}}
\def\Rlc{R_{\rm lc}}

\def\fm{f}
\def\fR{f_R}
\def\ta{\psi}
\def\tamax{\psi_{\rm max}}
\def\Etw{E_{\rm tw}}
  \def\Etor{E_{\rm tw}}
\def\tf{t_{\rm twist}}
\def\tend{t_{\rm end}}
  \def\Iopen{I_{\rm lc}}
  \def\Lopen{L_{\rm lc}}
\def\uf{u_{\star}}
\def\vf{v_{\star}}
\def\uuu{\tilde{u}_{\rm twist}}
\def\BQ{B_Q}
\def\js{j_\star}
\def\V{{\cal V}}
\def\LV{L_{\cal V}}
\def\tV{t_{\cal V}}
\def\Vei{{\cal V}_{ei}}
\def\Vpm{{\cal V}_{\pm}}
\def\gres{\gamma_{\rm res}}
\def\bOm{{\bf\Omega}}
\def\tev{t_{\rm ev}}
\def\tg{t_{\rm delay}}
\def\VGeV{V_9}
\def\Lradio{L_{\rm radio}}
\def\Lobs{L_{\rm obs}}
\def\Aobs{A_{\rm obs}}
\def\epsradio{\epsilon_{\rm radio}}
\def\XTE{XTE~J1810-197}
\def\AXP{1E~1547.0-5408}
\def\sigcr{\sigma_{\rm cr}}
\def\scr{s_{\rm cr}}

\def\rhoGJ{\rho_{\rm GJ}}
\def\rpc{r_{\rm pc}}
\def\Epar{E_\parallel}
\def\br{{\mathbf r}}
\def\M{{\cal M}}
\def\nmin{n_{\rm min}}
\def\vH{v_{\rm H}}
\def\T{{\cal T}}
\def\Tc{T_{\rm core}}
\def\thB{\Theta_B}
\def\Tcrit{T_{\rm crit}}
\def\Lum{L}
\def\ff{f_\star}
\def\Phie{\Phi_\parallel}
\def\vA{v_{\rm A}}
\def\Emag{E_{\rm mag}}
\def\sSB{\sigma_{\rm SB}}
\def\Eq{Equation}
\def\RLC{R_{\rm LC}}
\def\Phipc{\Phi_0}
\def\fopen{f_{\rm open}}
\def\tohm{t_{\rm ohm}}
\def\kB{k_{\rm B}}
\def\bk{{\mathbf k}}

%%%%
\newbox\grsign \setbox\grsign=\hbox{$>$} \newdimen\grdimen \grdimen=\ht\grsign
\newbox\simlessbox \newbox\simgreatbox \newbox\simpropbox
\setbox\simgreatbox=\hbox{\raise.5ex\hbox{$>$}\llap
     {\lower.5ex\hbox{$\sim$}}}\ht1=\grdimen\dp1=0pt
\setbox\simlessbox=\hbox{\raise.5ex\hbox{$<$}\llap
     {\lower.5ex\hbox{$\sim$}}}\ht2=\grdimen\dp2=0pt
\setbox\simpropbox=\hbox{\raise.5ex\hbox{$\propto$}\llap
     {\lower.5ex\hbox{$\sim$}}}\ht2=\grdimen\dp2=0pt
\def\simgt{\mathrel{\copy\simgreatbox}}
\def\simlt{\mathrel{\copy\simlessbox}}
\def\simprop{\mathrel{\copy\simpropbox}}
%%%%

%%%%%%%%%%%%%%%%%%%%%%%%%%%%%%%%%%%%%%%%
% Document starts
%\begin{document}

% Page heads
%\markboth{Kaspi \& Beloborodov}{Magnetars}

% Title
%\title{Magnetars}

% Author/affiliation
%\author{Victoria M. Kaspi,$^1$ Andrei M. Beloborodov$^2$%
%\affil{$^1$Department of Physics, McGill University,
% Rutherford Physics Building, 3600 University Street, 
%Montreal, Quebec, H3A 2T8; 
%email: vkaspi@physicsl.mcgill.ca}
%\affil{$^2$Department of Physics, 
% and Columbia Astrophysics Laboratory,
%Columbia University, New York, New York 10027;
%email: amb@phys.columbia.edu}
% \affil{$^3$Department of Chemistry, Purdue University, West Lafayette, Indiana 47907; email: lslipchenko@purdue.edu}}

% First page note
%\firstpagenote{This is an example of dummy text used to illustrate an example of first page note.}

% Abstract
%\begin{abstract}
%\end{abstract}

% Keywords
%\begin{keywords}
% ultrafast, core electron correlation, coherence, stimulated Raman
%\end{keywords}


%\maketitle
%
% to generate article TOC
%\tableofcontents

%\section{MECHANISM OF MAGNETAR ACTIVITY}


\subsection{Internal Dynamics}

 
A nascent magnetar experiences fast evolution in the first minutes of its life.
Magnetohydrodynamic (MHD) relaxation leads to a magnetic configuration that has a 
strong toroidal component \citep{bra09}, 
whose stability is assisted by compositional stratification \citep{arm+13}.
Neutrino cooling leads to crystallization of the crust. The resulting object 
has a liquid core of radius $\sim 10$~km surrounded by a 1 km-thick solid crust. 

The 
% dynamic 
\AB{subsequent }
behavior of magnetars 
on the timescales of 1-10~kyr
is associated with slow evolution of the magnetic field inside the star, which is 
capable of breaking the solid crust.
The interior of a neutron star is an excellent conductor,
and hence the magnetic field is practically ``frozen'' in the stellar material. 
More precisely, the magnetic field is frozen in the electron fluid,
and field evolution is possible due to the multi-fluid composition of the star.
The electrons may slowly drift with respect to neutrons and also with respect to ions. 
This gives two processes capable of  moving the magnetic field lines: 
ambipolar diffusion and Hall drift \citep{gr92}.


\subsubsection{Ambipolar diffusion}


Ambipolar diffusion is the motion of the electron-proton plasma 
(coupled with the magnetic field) through the static neutron liquid in the core. 
The magnetar is born with  electric currents $\bj=(c/4\pi)\nabla\times\bB$, and
hence  Lorentz forces $\bj\times\bB/c$ are applied to the plasma.
The plasma does not move in a young hot magnetar -- it is 
stuck in the heavy neutron liquid due to frequent proton-neutron collisions.
As the star ages and cools down, the p-n collision rate decreases 
\citep{ys90}, and ambipolar drift develops on the timescale 
$t_{\rm amb}\sim 10^3 (T_9/k_{-5}B_{16})^2$~yr,\footnote{
   Hereafter we use the standard notation $X_m$ for a quantity $X$ normalized to
   $10^m$ in CGS units.} 
where the wavenumber
$k\sim 10^{-5}$~cm$^{-1}$ describes the gradient of the magnetic field in the core
and corresponds to a characteristic scale $\pi/k\simlt R$.
The timescale of ambipolar diffusion becomes comparable to the 
star's age when the core temperature 
decreases to $\sim 10^9$~K. Then a significant drift occurs
and a large fraction of magnetic energy can be dissipated by the p-n friction.
As the drift develops, it generates plasma pressure gradients, which 
tend to be erased by weak interactions $e+p\leftrightarrow n$ \citep{gr92,td96a}.
\AB{Core superfluidity is capable of suppressing these processes and quickly ending
ambipolar diffusion \citep{gjs11}. However, for a plausible critical temperature of superfluidity,
the suppression effect is moderate, and ambipolar diffusion can still be the main cause of 
magnetar activity \citep{bl16}.}
The ambipolar drift tends to relax the magnetic stresses that drive it and 
eventually stalls while the core temperature drops.


\subsubsection{Hall drift}


Hall drift is the transport of magnetic field lines by the electric current $\bj$,
which implies a flow of the electron fluid relative
to the ions with velocity ${\mathbf v}_{\rm H}=-\mathbf j/en_e$,
where $n_e$ is the electron density. 
 Hall drift is normally very slow in the core, because of its high density
$n_e\sim 10^{37}$~cm$^{-3}$, but can be significant in the crust. 

Hall drift conserves the total magnetic energy, however it can generate new electric 
currents. When ohmic dissipation is taken into account, the evolution 
may come to a steady state \citep{gc14}. The field evolution in the crust 
was simulated numerically for axisymmetric configurations (e.g. \cite{pmg09}) 
and in more general 3D configurations \citep{gwh16}.
It was seen to build up strong elastic stresses in the crust \citep{pp11}.
Hall waves can also be excited near the crust-core boundary \citep{td96a}. 
% These waves propagate outward and cause significant crustal deformations \citep{llb16}.


\subsubsection{Mechanical failures of the crust}


The ambipolar and Hall drift of the magnetic field lines results in the gradual accumulation 
of crustal stresses which can trigger surface motions in magnetars.
The crust is nearly incompressible, however its Coulomb lattice can yield to shear stresses.
\citet{td95} proposed the picture of ``starquakes'' --- sudden 
fractures and displacements of the crust, which shake the magnetosphere and trigger 
bursts. Cracks involving void formation are impossible in a neutron star crust 
because of the huge hydrostatic pressure \citep{jon03},
and slips are forbidden by the strong magnetic fields unless the slip plane is aligned 
with the magnetic flux surfaces \citep{ll12}. 
A plausible yielding mechanism is a plastic flow.
It is triggered when the elastic strain exceeds a critical value $\scr\simlt 0.1$,
where the maximum value of $\scr\sim 0.1$ describes the strength of an ideal crystal.
The lattice failure was  demonstrated on the microscopic level using 
molecular dynamic simulations \citep{hk09,ch10}.

A growing magnetic stress may be applied from the evolving core and 
reach the maximum elastic stress $\scr\mu$
where $\mu\sim 10^{30}$~erg~cm$^{-3}$ is the shear modulus of the lower crust;
then the crust will experience a shear flow.
Magnetic stresses can also be generated in the crust itself, due to Hall drift \citep{td96a,pp11}.
This leads to a thermoplastic instability \citep{bl14}, which 
launches thermoplastic waves, relieving the stress.
The propagating wave burns magnetic energy, resembling the deflagration front in combustion physics. Its speed is 
$v\sim (\chi B^2/4\pi \eta)^{1/2}$ where $\chi\sim 10$~cm$^2$~s$^{-1}$ is the heat 
diffusion coefficient and $\eta$ is the viscosity coefficient of the plastic flow.
\AB{Crustal flows are capable of releasing significant magnetic energy 
(e.g. \citealt{laa+15}) with a complicated temporal pattern.}
% They are likely to occur in avalanches sustained due to the excitation of short Hall waves in the avalanche \citep{llb16}.


\subsubsection{Observational signatures of internal instabilities}


Internal dynamics of magnetars shape their observational properties in three basic ways:
(1) glitches in the rotation rate, (2) internal heating and increased surface luminosity,
and (3) twisting of the external magnetosphere. This leads to 
rich phenomenology of magnetar activity described in \S2-4,
from giant flares to timing anomalies to persistent hard X-ray 
emission and hot spots on the magnetar surface.
Here we briefly discuss internal glitches; mechanisms of internal heating and 
magnetospheric phenomena are described in \S5.2-5.4.

A common internal mechanism for glitches in pulsars is related to neutron superfluidity
\citep{ai75}. The magnetospheric spin-down torque is applied to the crustal lattice 
and not directly to the free neutrons which dominate the star's moment of inertia. 
Neutrons in the lower crust are expected to form superfluid and their spin down
can lag behind, i.e. the neutrons rotate slightly faster than the crustal lattice. 
Vorticity of the neutron superfluid is quantized into vortex lines, and the lag 
is the consequence of the vortices being pinned to the lattice nuclei.
When the vortices become unpinned, they are allowed 
to move away from the rotation axis and the superfluid can spin down;
its angular momentum is passed to the lattice in this event, producing a spin-up glitch.
\citet{td96a} argued that sudden starquakes can cause such unpinning.
Alternatively, the superfluid vortices can be unpinned due to a 
heating episode \citep{le95}, and this is expected to occur in a 
thermoplastic wave. 


%#####################################################

\subsection{Internal Heating and Surface Emission}
\label{sec:heating}


Heat that is initially stored in a nascent neutron star is eventually lost to neutrino emission 
and surface radiation. As a result, a kyr-old neutron star is normally expected to have an 
internal temperature $\Tc\sim 10^8$~K and a surface temperature $T_s\sim 10^6$~K 
\citep{yp04}, which corresponds to a surface luminosity 
$L_s\sim 10^{33}$~erg~s$^{-1}$.
In contrast, the surface luminosities of classical active magnetars are around 
$L_s\approx 10^{35}$~erg~s$^{-1}$.
For a neutron star of radius $R\approx 10-13$~km, this luminosity
corresponds to effective surface temperature 
$T_s=(L_s/4\pi R^2 \sigma_{\rm SB})^{1/4}\approx 4\times 10^6$~K, 
where $\sigma_{\rm SB}$ is the Stefan-Boltzmann constant. 
This indicates that magnetars are strongly heated by some mechanism.


\subsubsection{Heating of the core}


The high surface luminosity may be associated with a strong heat flux 
from the core, which implies an unusually high $\Tc$ \citep{td96a}. 
A steady heat flux to the surface is established on the conduction timescale 
of $1-10$~yr, which is much shorter than the magnetar age.
Then the steady heat diffusion determines the relation between $\Tc$ and $T_s$.
The main drop of temperature from $\Tc$ to $T_s$ occurs in the blanketing envelope
in the upper crust, in particular where $\rho<10^{9}$~g~cm$^{-3}$
\citep{yp04}. The $\Tc$-$T_s$ relation depends on the magnetic field $\bB$ in the blanket
and its chemical composition \citep{pyc+03}.
% Detailed calculations show that sustaining $L_s\sim 10^{35}$~erg~s$^{-1}$ 
% requires $\Tc\simgt 6\times 10^8$~K under most favorable assumptions \citep{bl16},
% in particular assuming a light-element blanket, which conducts heat better than iron does.
Increasing the core temperature above $\sim 10^9$~K does not increase $L_s$ 
because the heat flux is lost to neutrino emission on its way through the crust
\citep{ppp15}. 
Thus $L_s\sim 10^{35}$~erg~s$^{-1}$ is naturally obtained for any $\Tc\simgt 10^9$~K.

However, the ability of magnetars to sustain $\Tc\sim 10^9$~K for 1-10~kyr is 
questionable. At such high temperatures, huge neutrino losses are expected in the core itself
\citep{yp04}.
A minimum neutrino cooling rate is found in non-superfluid cores due to 
modified URCA reactions, $\dot{q}_\nu\sim 10^{21} T_9^8$~erg~cm$^{-3}$~s$^{-1}$.
% \citep{fm79}.
A possible transition to superfluidity at a critical temperature 
$\Tcrit\simgt 10^9$~K would only increase the cooling rate, as a 
result of continual breaking and formation of Cooper pairs \citep{frs76}.
% \citep{flo76,kgy+06,plps09}.
% \citep{flo76}.
In general, only a very powerful heat source can compete with neutrino losses at 
$\Tc\sim 10^9$~K.
A recent analysis by \citet{bl16} shows that heating by ambipolar diffusion may 
sustain the observed surface luminosity $L_s\sim 10^{35}$~erg~s$^{-1}$ if the internal 
magnetic field is ultrastrong, $B\simgt 10^{16}$~G, but only for a time shorter than 1~kyr.


\subsubsection{Heating of the crust}


An alternative scenario of internal heating assumes a cool core and a 
heat source in the crust. Then the radial temperature profile $T(r)$ peaks in the crust. 
Most of the heat is conducted to the core and lost to neutrino emission,
however a fraction reaches the stellar surface and could power the observed $L_s$.
Requirements of this scenario were investigated by 
% \citet{kyp+06,kpyc09,kkp+14}.
\citet{kkp+14}.
They placed a phenomenological heat source at various depths 
without specifying its mechanism, and 
found that sustaining the surface luminosity $L_s\approx 10^{35}$~erg~s$^{-1}$ 
requires a heating rate $\dot{q}_h>3\times 10^{19}$~erg~s$^{-1}$~cm$^{-3}$
at shallow depths $z\simlt 300$~m. 

Two possible mechanisms for converting magnetic energy to heat
are mechanical dissipation in the failing crust and ohmic heating.
Both obey strong upper limits \citep{bl16}. In particular, 
mechanical heating can only occur in the solid phase below the melted ocean,
and its rate cannot exceed $\dot{q}_{\max}\sim 0.1\mu\dot{s}$, where 
$\mu\sim 10^{28}\rho_{12}$~erg~cm$^{-3}$ is the shear modulus of the lattice and 
$\dot{s}$ is the strain rate of the deformation. This leads to an upper limit on
the persistent surface luminosity, which falls short of $10^{35}$~erg~s$^{-1}$.

The rate of ohmic heating $\dot{q}_h=j^2/\sigma$ is also limited, due to the 
high electric conductivity of the crustal material: $\sigma\sim 10^{22}$~s$^{-1}$ 
in the relevant range of temperatures and densities in the upper crust \citep{ppp15}. 
The timescale for dissipating electric currents that sustain variations $\delta B$
on a scale $\ell$ is 
$\tohm\sim 4\pi\sigma \ell^2/c^2\sim 4\times 10^4\, \sigma_{22}\,\ell_{\rm km}^2$~yr.
Ohmic dissipation could provide the required heating if two demanding conditions 
are satisfied:
(1) $\tohm$ is sufficiently short, comparable to the magnetar age of 1-10~kyr,
which requires a small scale of the field variations $\ell\sim 3\times 10^4$~cm, and
(2) the field variations on this scale must be huge, $\delta B \sim 10^{16}$~G, to provide 
$\dot{q}_h\sim (\delta B)^2/8\pi \tohm>3\times 10^{19}$~erg~s$^{-1}$~cm$^{-3}$.

It was proposed that ohmic and mechanical heating are boosted by Hall drift, which can 
transport magnetic energy to the shallow layers \citep{jon88} and develop 
large current densities \citep{gr92}.
%,vco00}.
Heating of the crust reduces its conductivity, leading to the coupled evolution of 
temperature and magnetic field. This magneto-thermal evolution was studied numerically
(e.g. \citealt{pmg09,vrp+13}), showing inhomogeneous 
heating of the stellar surface on timescales comparable to the magnetar ages.
The results suggest that the magneto-thermal evolution of the crust may explain 
the observed properties of a broader class of neutron stars, not only magnetars.

For simplicity, the simulations assumed that the crust is static, and this assumption 
needs to be relaxed in more realistic models.
Hall drift in the upper crust induces magnetic stresses exceeding 
the maximum elastic stress $\sigma_{\max}\sim 0.1\mu$, and the crust must flow, 
releiveing the stress. This leads to an upper limit on Hall-driven dissipation,
rendering it hardly capable of sustaining the surface luminosity of classical 
persistent magnetars \citep{bl16}. 
The full magneto-thermo-plastic evolution has been simulated in a one-dimensional 
model by \citet{llb16}. It showed intermittent heating 
through avalanches developing due to the excitation of short Hall waves in the avalanche,
which may explain the activity of transient magnetars.
% Future models will need to study this evolution in two and three dimensions.


\subsubsection{Transient heating and flare afterglow}


\citet{let02}
explored how a sudden deposition of heat in the crust could power the 
afterglow of the 1998 giant flare from SGR~1900+14. They found that heating 
would need to be approximately uniform throughout the upper crust, which implies 
enormous heat per unit mass in the surface layers $z< 100$~m.
It is, however, unclear how magnetic energy could be 
suddenly dissipated in the shallow layers,
which should behave as a liquid ideal conductor during the flare.
An additional complication is that the afterglow spectrum was nonthermal \citep{wkg+01}, 
suggesting magnetospheric origin. 
Nevertheless, the phenomenological picture of sudden crustal heating was 
applied to several transient magnetars, using detailed time-dependent 
simulations of heat conduction \citep[e.g.][]{rip+13,skc14}. 
The model was found capable of reproducing the observed light curves of 
some transient magnetars, except the cases where the emission area was 
observed to shrink with time. A mechanism explaining this shrinking will
be discussed in \S\ref{sec:untwist}.

A concrete mechanism for sudden crustal heating by a magnetospheric flare
was recently described by \citet{lb15}. 
They showed that Alfv\'en waves generated by the flare are quickly absorbed by the star 
and damped into plastic heat in the solid crust immediately below the melted ocean. 
A fraction of the deposited heat is eventually conducted to the stellar surface, 
contributing to the surface afterglow months to years after the flare. 


\subsubsection{Surface spectrum}


The spectrum of thermal radiation emerging from the hot magnetar 
deviates from a Planck spectrum due to radiative transfer effects,
which are sensitive to the magnetic field \citep[see][for a review]{pdp15}.
An interesting potential spectral feature is the ion cyclotron line at energy
$\hbar ZeB/Am_pc\approx 0.63\,(Z/A)B_{14}$~keV.

The first detailed simulations of magnetar spectra assumed a fully ionized
hydrogen atmosphere \citep{hl01b,oze01,zts+01};
calculations have also been performed for a broader class of models.
Instead of forming a low-density atmosphere, the surface may
condense into a Coulomb liquid \citep{rud71,ml06} with a high density
$\rho\sim 10^5 AZ^{-3/5}B_{14}^{6/5}$~g~cm$^{-3}$, where $A$ and $Z$ are the 
ion mass and charge numbers. In this case, the emerging spectrum depends on the 
reflectivity of the liquid surface \citep{tzd04}.
It is also possible that the liquid is covered by a thin low-density atmosphere
\citep{spw09}.
% ,psv+12}.

Magnetar fields exceed the characteristic field 
$\BQ=m_e^2c^3/\hbar e\approx 4.4\times 10^{13}$~G 
defined by $\hbar \omega_B=m_ec^2$, where $\omega_B=eB/m_ec$. 
As a result, the photon normal modes are significantly 
impacted by the QED effect of vacuum polarization as well as plasma polarization.
Vacuum polarization dominates at densities 
$\rho\ll\rho_V\sim 1\, B_{14}^2 (\hbar\omega/1{\rm ~keV})^2$~g~cm$^{-3}$
and defines two linearly polarized normal modes:
the ordinary or O-mode (polarized in the ${\mathbf k}$-$\bB$ plane, 
where ${\mathbf k}$ is the photon wavevector) and the extraordinary or X-mode
(polarized perpendicular to the ${\mathbf k}$-$\bB$ plane).
At densities $\rho\gg \rho_V$ the plasma polarization dominates, 
which also gives O- and X-modes with highly elliptical polarizations. 
As photons pass through the layer with $\rho=\rho_V$
the polarization ellipse rotates by $90^{\rm o}$, effectively switching the modes
$O\leftrightarrow X$ \citep{lh03}, and the photon begins to see a different opacity. 
This effect tends to deplete the spectrum at high energies 
and also weakens the cyclotron absorption line.

% The surface is not homogeneous and 
\AB{In principle, the surface spectrum contains information on the surface magnetic field,
 however in practice extracting this information is extremely difficult.}
The cyclotron line is hard to detect when its position is smeared 
out due to variations in the surface magnetic field.
\AB{Besides $\bB$, the surface spectrum depends on the chemical composition 
and possible condensation, and disentangling all the effects is a challenging task.} 
% disentangling all the effects influencing the surface spectrum is difficult.
% In view of the uncertainties in the chemical composition and possible condensation, 
% disentangling all the effects influencing the surface spectrum is difficult.
\AB{Attempts to infer the surface $\bB$ from the observed 1-10~keV spectrum 
(e.g. \citealt{ggo11}) are further complicated by the superimposed 
magnetospheric emission.}
% The existing models suggest that the overall shape of the surface spectrum does 
% not dramatically deviate from a simple Planck spectrum. 

The inhomogeneity of temperature and magnetic field on the stellar surface
leads to flux pulsations with the magnetar spin period, 
which are influenced by the anisotropy of surface radiation and gravitational light 
bending. The surface radiation has a component beamed 
along the local $\bB$ and a broader fan component peaking perpendicular to $\bB$ \citep{zpsv95}. 
\AB{The observed pulse profiles are also strongly influenced by photon scattering in 
the magnetosphere.}


\subsubsection{Polarization}


Thermal radiation diffusing toward the magnetar surface is dominated by the X-mode 
photons, which have longer free paths and hence escape from deeper and hotter 
atmospheric layers. This radiation is linearly polarized with the electric field perpendicular
to the $\bk$-$\bB$ plane. Radiation emerging from a condensed surface is also 
linearly polarized; polarization of the radiation emitted by gaseous and condensed 
surfaces is discussed in detail by \citet{ttg+15}.

Polarization measured by a distant observer is influenced by two factors. 
(1) The observer receives surface radiation from regions with different magnetic fields,
which implies some averaging of the polarization.
(2) As the photons propagate through the magnetosphere, they ``adiabatically track'' 
the local normal mode defined by the local $\bB$: the polarization vector of the 
X-mode shifts so that it stays perpendicular to the $\bk$-$\bB$ plane.
This tracking ends and the polarization angle freezes when the photon reaches the
``adiabatic'' radius $r_{\rm ad}\sim 80 R\, B_{s,14}^{2/5}(\hbar\omega/1{\rm~keV})^{1/5}$,
where $B_s$ is the surface magnetic field \citep{hsl03}.
Therefore, effectively the observed polarization angle is controlled by the direction of 
$\bB$ at $r_{\rm ad}$ rather than at the stellar surface. 


\subsection{Flares}


Magnetar flares and bursts emit hard X-rays and hence must be produced outside 
the neutron star. An ``internal trigger'' scenario assumes that the flares are
 triggered by an instability that leads to sudden ejection of magnetic 
energy from the core to the magnetosphere. The trigger 
could be an MHD instability in the liquid core or a sudden failure of the solid crust 
in response to a slow buildup of magnetic stresses at the crust-core boundary
\citep{td95,td01}. Excitation of core motions with displacement $\xi$  carries energy 
up to $\sim \Emag (\xi/R)^2$, where $\Emag\sim 10^{48}$~erg is the putative 
magnetic energy of the core. A displacement $\xi\sim R$ would strongly deform the 
magnetosphere and trigger a giant flare, however it implies a huge 
energy budget for the event, comparable to the entire $\Emag$.
A smaller $\xi$ could provide sufficient energy for a giant flare,
however  $\xi\ll R$ makes energy transfer to the magnetosphere inefficient \citep{lin14}.

An alternative ``external trigger'' scenario assumes a gradual deformation of the 
magnetosphere and the build up of the ``free'' magnetic energy followed by its 
sudden release, similar to solar flares \citep{lyu03,gh10,pbh13}.
This scenario requires only that the crustal motions pump an external twist 
faster than the magnetosphere damps it. This condition 
can be satisfied by thermoplastic crustal motions.
% Multiple  flares are expected to occur over the lifetime of a magnetar,
% as long as magnetic energy continues to be gradually 
% transferred from the star to the magnetosphere.


\subsubsection{Magnetospheric instability}


The magnetosphere twisted by surface shear motions becomes non-potential, 
$\nabla\times\bB\neq 0$, and threaded by electric currents $\bj=(c/4\pi)\,\nabla\times\bB$.
The magnetic energy still dominates over the plasma energy (including its rest mass),
and in the absence of strong ohmic dissipation the magnetosphere remains nearly 
force-free, $\bj\times\bB\approx 0$.
On a microscopic level, this condition corresponds to 
particles being kept in the ground Landau state in the strong magnetic field,
so that charges can only flow along $\bB$.

Strongly deformed magnetospheres are prone to global instabilities, 
as shown by the analysis of twisted magnetic equilibria (e.g. \citet{uzd02}).
An equilibrium magnetosphere obeys the 
equation $\nabla\times(\nabla\times\bB)=0$. The same equation describes the 
force-free solar corona, and
% (e.g. \citet{low78,aly84,wol95a}). 
the first models of static magnetar 
magnetospheres  \citep{tlk02} followed the self-similar solutions of \citet{wol95a}.
% ptzn09}.
More realistic configurations are obtained by numerical simulations
of the magnetosphere responding to surface shear \citep{pbh13}.
As the surface shearing continues, the twist angle $\ta$ grows until it reaches 
a critical value at which the magnetosphere becomes unstable.
Then it suddenly releases a significant fraction of the stored twist energy and 
produces a powerful flare. 
The flare involves a current sheet formation, its tearing instability and ejection of 
plasmoids, resembling coronal mass ejections from the sun \citep{ml94a}.
The flare development in axisymmetric geometry is shown in Figure~\ref{recon}.
Future simulations can significantly advance this picture
by relaxing axisymmetry and demonstrating the 
development of current sheets near the null points of more general magnetic configurations.

%%%%%%%%%%%%%%%%%%%%%%%%%%%%%
\begin{figure}
\includegraphics[width=5in,height=3in]{polar-recon}
\caption{Formation of the current sheet in an ``over-twisted'' magnetosphere.
Reconnection begins at $t_{\rm rec}$, panel (d). Color shows toroidal current density, 
lines are the poloidal magnetic field lines. One field line is highlighted in heavy black; 
it first opens and then closes again. Time is indicated in units of the light crossing 
time of the star. From \citet{pbh13}.
}
\label{recon}
\end{figure}
%%%%%%%%%%%%%%%%%%%%%%%%%%%%%


Current sheets spontaneously forming in over-twisted magnetospheres are fated 
to fast reconnection and energy dissipation. This process may naturally explain 
the fast onset of giant flares, as the dynamical timescale of the magnetosphere 
is $\delta t\sim R/c <1$~ms.
However, tension exists between the observed duration of the main peak,
$t_{\rm peak}\sim 0.3-0.5$~s, and the theoretical duration of the reconnection 
event $\sim 10^2R/c\sim 10^{-2}$~s \citep{uzd11}.


\subsubsection{Fast Dissipation and Radiation}


The main spike of a giant flare releases $\sim 10^{44}$-$10^{46}$~erg, 
which implies a huge energy density near the star. It is consistent with
dissipation of ultrastrong fields, releasing energy density  up to $B^2/8\pi\approx 4\times 10^{26}B_{14}^2$~erg~cm$^{-3}$.
The energy is immediately thermalized at a temperature $kT\simgt m_ec^2$,
creating a dense population of $e^\pm$ pairs and making the region 
highly collisional and opaque. In addition to sudden heating, reconnection 
generates strong Alfv\'en waves, which are ducted along the 
magnetic field lines and trapped in the closed magnetosphere. 
Part of the wave energy may cascade to small scales and dissipate
\citep{tb98}, and part is damped in the star.
% \citep{lb15}. 

The initial short spike of the giant flare must be emitted by a relativistic outflow
which traps radiation near the star, advects it and releases at a large radius 
$r>10^{10}$~cm \citep{td95}. This process resembles what happens in a 
cosmological gamma-ray burst \citep{mes02}.
A high Lorentz factor of the outflow $\Gamma$ helps the radiation to escape,
because the scattering 
optical depth is reduced as $\Gamma^{-3}$, and the 
photon-photon reaction $\gamma+\gamma\rightarrow e^++e^-$ is suppressed 
due to beaming of radiation within angle $\sim\Gamma^{-1}$.

After the initial spike, the temporarily opened field lines must close 
back to the star and trap a cloud of hot plasma (``fireball''), which produces the 
pulsating tail of the flare \citep{td95}. The fireball is confined if its energy $E_0$ is a 
fraction of $\RNS^3B^2/8\pi\sim 10^{45}B_{14}^2$~erg, so the field 
$B\simgt 10^{14}$~G is required by the observed energy radiated in the 
pulsating tail $E_0\sim 10^{44}$~erg. The trapped heat is 
gradually radiated away and the fireball gradually ``evaporates,'' layer by layer.
In the simplest model, the remaining heat $E(t)$ is proportional to the fireball volume 
while the luminosity $L=-dE/dt$ is proportional to the fireball surface area $A$. It implies 
$L\propto E^a$, where $a$ may be between $2/3$ (sphere) and 0 (slab), and 
gives  $L(t)=L_0(1-t/t_{\rm evap})^\chi$ with $0<\chi=a/(1-a)<2$ \citep{td01}.
This relation fits the observed envelope of the pulsating tail after $\sim 40$~s 
with  $t_{\rm evap}\approx 6$~min and large $\chi=3$, 
possibly indicating inhomogeneous structure of  the fireball  \citep{fhd+01}.
Further constraints on the model are provided by detailed calculations of neutrino 
losses \citep{goo11}.

The emission observed during the first 40~s of the flare 
(and after its main peak) requires an additional source, which
is brighter,  less modulated by rotation, and also harder. 
It was interpreted as emission from a heated corona surrounding the fireball \citep{td01}.
The emission observed after 40~s does not require additional heating and appears 
to be dominated by the evaporating fireball.
Its effective surface temperature 
$kT_s\sim 15\,L_{42}^{1/4}A_{13}^{-1/4}$~keV is smaller than 
the temperature inside the fireball by the factor of $\sim 10^{-2}$. 
The mean photon energy in a blackbody spectrum, $\hbar\omega\sim 3kT_s$, 
would be roughly consistent 
with the characteristic energy of observed hard X-rays, 
however the radiation spectrum is not Planckian. 
The spectrum must be shaped by radiative transfer in the fireball, and
a key feature of this transfer problem is the presence of two polarization 
states with drastically different free paths: the O- and X-mode.
Scattering of the X-mode is suppressed by the factor of 
$\sim (\omega/\omega_B)^2\sim 10^{-4}(\hbar\omega/10{\rm ~keV})^2 B_{14}^{-2}$.
Therefore, the X-mode photons dominate the energy transport and the emerging
luminosity; they escape from large Thomson optical depths where the O-mode 
photons are still trapped.
Since the E-mode cross section scales as $\omega^2$, photons of different
energies $\hbar\omega$ escape from different depths, leading to a flat 
spectrum at $\hbar\omega\simlt kT$ \citep{lub02}.
At high energies $\hbar\omega>kT$, the E-mode photons can  
split into O-mode photons in the ultrastrong magnetic field \citep{adl71}. 
The resulting theoretical spectrum of escaping radiation is far from Planckian
and resembles the observed broad spectrum.
\AB{The amplitude of  observed pulsations is likely affected by the scattering of 
fireball radiation in a continuing outflow from the star \citep{td01,vwb+16}.}

Similar fast dissipation of magnetic energy must occur in less powerful bursts,
which are much more frequent than the giant flares. Ordinary bursts with 
luminosity $L\ll 10^{42}$~erg~s$^{-1}$ may not be capable of producing 
fireballs with thermalized radiation, which explains their different spectra. 
The burst spectra are harder to model as they are more sensitive to the uncertainties 
in the dissipation mechanism, and detailed models have yet to be developed.


\subsubsection{Quasi-Periodic Oscillations}


The QPOs observed in giant flares (\S\ref{sec:qpos})
were interpreted as shear oscillations of the magnetar crust, which were studied by 
\citet{dun98}.
Simple QPO models assume that the crustal oscillations are decoupled
from the liquid core \citep{pir05,sa07}.
% \citealt{pir05,wr07,lee07,sks07}.
More realistic models allow for the crust coupling by the magnetic field
to the continuum of Alfv\'en waves in the liquid core
% and the core are coupled by the magnetic field lines
%  \citep{lev06,lev07,gsa06,csf09,ck11,vl11,vl12,gcf+13,gcs+14,pl14}.
\citep{lev07}.
%,gsa06}.
This creates significant uncertainties in the frequency spectrum of the global modes, 
as it depends on the magnetic configuration in the star. However some general 
features have been identified. 
The strongest oscillations are expected near the edges of the spectrum \citep{vl11}. 
The oscillations experience phase mixing and QPOs can be transient and delayed.
The models may be reconciled with the observed range of QPO frequencies
if neutrons are decoupled from the oscillations \citep{vl11,gcs+13,pl14},
which is possible if the neutrons are superfluid. 
High-frequency oscillations, in particular the 625-Hz QPO in the 1998 giant flare,
are most challenging to model. 

The magnetosphere attached to the oscillating crust is periodically
deformed \citep{gcs+14}, which could lead to periodic changes in its luminosity. 
The models predict small amplitudes of surface oscillations, much smaller 
than the stellar radius, and it remains unclear how they generate the 
observed 10-20 percent modulation of the X-ray luminosity.

% \citep{tel08,dw12,gcs+14}. 


\subsection{Gradual Energy Release in the Twisted Magnetosphere}
\label{sec:twist}


The magnetosphere can remain twisted for years between the flares,
explaining the magnetar activity described in \S3.
The presence of a magnetic twist $\nabla\times\bB\neq 0$ implies long-lived electric 
currents, which are accompanied by some ohmic dissipation. The 
magnetosphere always tends to slowly untwist,
and the released magnetic energy feeds its long-lived emission 
\citep{tlk02,bel09}.


\subsubsection{Electric discharge}


Magnetospheric electric currents can only flow along the magnetic field lines and their
ohmic dissipation occurs due to a small electric field 
$\Epar$ parallel to $\bB$. The electric field has three functions: 
(1) it maintains the electric current $\bj\parallel \bB$ demanded by $\nabla\times\bB\neq 0$, 
(2) it regulates the dissipation rate $\Epar j$ and the observed nonthermal 
luminosity, and 
(3) it determines the evolution of $\bB$ in the untwisting magnetosphere
according to the Maxwell-Faraday equation $\partial\bB/\partial t=-c\,\nabla\times\bE$.

The longitudinal voltage between the two footprints `1' and `2' of a magnetospheric 
field line, $\Phi_\parallel=-\int_1^2 E_\parallel dl$, controls 
the ohmically released power $L \approx I \Phie$, where 
$I<I_{\max}\sim c\mu/R^2$ is the net  electric current circulating through the 
% twisted 
magnetosphere and $\mu$ is the magnetic dipole moment of the star.
The observed nonthermal luminosities of magnetars 
typically require $\Phie\sim 10^{10}$~V. The voltage 
was proposed to be regulated 
% through continual
by 
$e^\pm$ discharge: 
when $\Phie$ exceeds a threshold value, an exponential runaway of $e^\pm$ 
creation occurs until the pair plasma screens $E_\parallel$  \citep{bt07}. 
As the plasma leaves the discharge region, $E_\parallel$ grows again and the discharge 
repeats, resembling continual lightning. 

In a magnetar magnetosphere, the discharge is triggered when an electron accelerated 
to Lorentz factor $\gamma$ begins to resonantly scatter X-rays streaming from the star.
The X-ray photon is blueshifted  in the electron rest frame by the factor of $\sim\gamma$,
and resonant scattering occurs when the blueshifted photon frequency 
matches the cyclotron frequency $\omega_B=eB/m_ec$.
In the ultrastrong field near the magnetar this typically requires $\gamma\simgt 10^3$. 
The scattered photon has a high energy 
$\sim \gamma^2$~keV and quickly converts to $e^\pm$ in the strong 
magnetic field. This process gives the threshold voltage $\Phie\sim 10^9-10^{10}$~V.
The discharge can be modeled ab initio using "particle in cell" (PIC) method that 
follows plasma particles individually in their collective electromagnetic field. First 
axisymmetric PIC simulations of twisted magnetospheres have been performed 
recently \citep{cb16}. 


\subsubsection{Resistive untwisting and shrinking hot spots}
\label{sec:untwist}


The resistive evolution of twisted magnetospheres has only been studied thus far 
in axisymmetric geometry.
% \citep{bel09}. 
In this case the magnetosphere may be 
described as a foliation of axisymmetric flux surfaces labeled by magnetic flux 
function $f$, and the twist of a magnetospheric field line $\psi(f)$ 
is the difference between the azimuthal angles of its footpoints on the star,
$\ta=\phi_2-\phi_1$.
The twist evolution is governed by the electrodynamic equation obtained from 
$\partial\bB/\partial t =-c\,\nabla\times \bE$,
\be
\label{eq:evol}
   \frac{\partial\psi}{\partial t}=2\pi c\frac{\partial\Phie}{\partial f}
     +\omega(f,t),
\ee
where $\omega=\dot{\phi}_2-\dot{\phi}_1$ is the applied shear rate at the stellar surface
\citep{bel09}.
When the fast crustal motions stop, so that $\omega\ll 2\pi c\, \partial\Phie/\partial f$, 
the untwisting phase begins.

Two distinct regions evolve in an untwisting magnetosphere: 
a ``cavity'' with $j=0$ and a ``j-bundle'' where the currents flow.
% \citep{bel09}.
The cavity is comprised of field lines that close near the star,
and has a sharp boundary along a flux surface $\ff$.
The process of resistive untwisting is the slow expansion of the boundary $f_\star$, 
which erases the electric currents in the j-bundle on a timescale
$t_{\rm ohm}\simlt\mu/cR\Phie$. 
As a result, the magnetospheric currents have the longest lifetime on magnetic 
field lines with large apex radii $R_{\max}\gg R$, i.e. the 
magnetar activity tends to be confined to field lines extending far from the star.

This electrodynamics implies a special observational feature: 
a  slowly shrinking hot spot on the magnetar surface.
The footprint of the j-bundle is expected to be hotter than the rest of the stellar surface,
because it is bombarded by relativistic particles from the $e^\pm$ discharge.  
As the j-bundle slowly shrinks so does its footprint. 
The theoretically expected relation between the spot area $A$ and luminosity $\Lum$
is given by $\Lum\sim 1.3 \times 10^{33} K\,A_{11}^2$~erg~s$^{-1}$ where 
$K=B_{14} \Phi_{\parallel 9} \psi$.
% \citep{bel09}. 
Such shrinking hot spots have been 
observed in seven transient magnetars (see \cite{bl16} for a compilation of data). 
The spot area and luminosity
decrease with time, and the observed slope of the $A$-$\Lum$ relation 
(controlled by the behavior of $\Phi_\parallel$) varies between 1 and 2. 
The typical timescale of this evolution, months to years, 
is also consistent with theoretical expectations, however there are outliers that 
require a more detailed modeling.

Transient magnetars also show different behavior, which is inconsistent with 
the simple model of untwisting  magnetosphere attached to a static crust.
For instance, 1E~1547$-$5407 displays complicated activity \citep{khd+12},
possibly due to repeated twist injection, which prevents
a clean, long twist decay. In addition, a different emission component
may  result from cooling of a suddenly heated crust, as discussed in 
Section~\ref{sec:heating}.


\subsubsection{Nonthermal X-rays}
\label{sec:nonth}


The dissipated twist energy is given to the $e^\pm$ plasma in the j-bundle, 
which can radiate it away in two ways: the particles can hit the stellar surface or 
pass their energy to the ambient X-rays streaming from the star, mainly through  
resonant Compton scattering \citep{tlk02,bh07}. 

Scattering of a keV photon by a particle with Lorentz factor $\gamma$ 
boosts the photon energy by a factor of $\sim\gamma^2$. Early work \citep{tlk02}
proposed that the magnetosphere is filled with mildly relativistic electrons 
which scatter the thermal surface radiation and produce 
a soft power-law tail in the X-ray spectrum.
The resonance scattering condition implies that mildly relativistic Comptonization 
can occur where $B\sim 10^{11}$~G which corresponds to $\hbar eB/m_ec\sim 1$~keV;
this region is typically at radii $r\sim 10\RNS$. Detailed Comptonization models 
with ad hoc particle distributions were developed 
\citep{lg06,ft07,ntz08}
and found capable of reproducing the observed 1-10~keV spectra.
A simplified resonant Comptonization model was implemented under XSPEC and 
used to fit the X-ray spectra of magnetars below 10~keV 
\citep{rzt+08,zrtn09}.

%%%%%%%%%%%%%%%%%%%%%%%%%%%%%
\begin{figure}
\parbox{6cm}{
\includegraphics[width=6cm]{illustr}
% \caption{First.}
% \label{fig:2figsA}
}
\qquad
\begin{minipage}{6cm}
%\includegraphics[width=6cm]{spectra.jpg}
\includegraphics[width=6cm]{spectra}
% \caption{Second.}
% \label{fig:2figsB}
\end{minipage}
\caption{Left: A magnetic loop in the j-bundle. Relativistic particles are assumed to be
injected near the star (black sphere), and a large $e^\pm$ multiplicity $\M\sim 100$ 
develops in the `adiabatic' zone $B > 10^{13}$~G (shaded in blue). The outer part of 
the loop is in the radiative zone; here the resonantly scattered photons of energy 
$h\nu_{\rm sc}\sim 1\,B_{13}^2$~MeV escape and form the observed hard X-ray spectrum.
% shown in 
% Figure~\ref{fits}.
The outflow decelerates and eventually annihilates at the top of the loop (shaded in pink); 
here it becomes opaque to the thermal keV photons flowing from the star. 
Photons reflected from the pink region have the best chance to be scattered by 
the relativistic outflow in the lower parts of the loop, and control its deceleration.
The footprint of the j-bundle is heated by the relativistic backflow from the discharge region; 
in an axisymmetric model it forms a hot ring on the stellar surface.
Right: Radiation spectrum from the j-bundle viewed at four different 
angles with respect to the magnetic dipole axis. From \citet{bel13b}.
}
\label{fig:illustr}
\end{figure}
%%%%%%%%%%%%%%%%%%%%%%%%%%%%%


Recent work attempted a self-consistent calculation of the particle motion in the 
j-bundle \citep{bel13a,bel13b}, which led to the picture shown in 
Figure~\ref{fig:illustr}.
Pairs must be created with high Lorentz factors $\gamma$ and their subsequent motion 
is influenced by radiative losses due to resonant scattering. The upscattered
photons have high energies and convert to $e^\pm$ pairs in the region of $B>10^{13}$~G.
As a result, energy radiated near the star is processed into an outflow (``fountain'') of 
copious pairs with a standard profile of Lorentz factor $\gamma\approx 100(B/\BQ)$.
The outflow radiates away its kinetic energy in the outer zone $B\simlt 10^{13}$~G and 
comes to the top of the magnetic loop with $\gamma\sim 1$. 
Mildly relativistic Comptonization may occur in this outer region and 
influence the X-ray spectrum below 10~keV. However, most of the nonthermal 
luminosity is radiated above 10~keV in the lower parts of the loop, where the outflow 
has $\gamma\gg 1$. 

Regardless of the details of the electric discharge near the star, the  
$e^\pm$ fountain emits a power-law spectrum
$dL/d\ln E\propto E^{1/2}$.
It naturally produces a distinct hard component in magnetar spectra, which 
peaks and cuts off in the MeV band. However, the $E^{1/2}$ power law is predicted
only for spectra averaged over viewing angles.
The emission is beamed along the magnetic field lines within angle 
$\sim \gamma^{-1}\sim 0.1 B_{13}^{-1}$, 
and the overall angular distribution is determined by the field-line curvature.
The field may be approximated as dipolar
in the radiative zone $B\simlt 10^{13}$~G, which is relatively far from star.
In this approximation, the predicted spectrum varies with the viewing angle 
as shown in Figure~\ref{fig:illustr}.

This model provided good fits to the phase-resolved 
hard X-ray spectra of magnetars 
% \citep{ahk+13,aah+15,hbd14,vhk+14,thy+15}.
\citep{hbd14,aah+15,vhk+14}.
The main unknown geometric parameters are 
the angle between the magnetic dipole axis and the rotation axis, 
$\alpha_{\rm mag}$, and the angle between the line of sight and the rotation axis, 
$\beta_{\rm obs}$.
Remarkably, only a small region in the parameter space provided a good fit, 
which allows one to estimate $\alpha_{\rm mag}$ and $\beta_{\rm obs}$. The results suggest
that the magnetic dipole axis in magnetars is slightly misaligned with the rotation axis.  

The polarization of radiation upscattered in the magnetosphere was discussed
by \cite{fd11} and \citet{bel13b}. 
After resonant scattering the photon ``forgets'' its initial polarization; it becomes X-mode
with 75\% probability and O-mode 25\% probability. The escaping radiation
should be dominated by the X-mode, except at energies approaching 1~MeV, where 
photon splitting occurs $X\rightarrow O+O$  \citep{adl71}.
One may observe the O-mode polarization if the daughter photons from splitting do 
not scatter on the way out from the magnetosphere.
Future measurements of X-ray polarization can provide powerful diagnostic 
tools, taking the advantage of magnetar rotation (which gives a periodically changing viewing angle) and combining with the phase-resolved spectra.

A few other ideas were proposed for the origin of the hot plasma capable of emitting hard 
X-rays. \citet{hh05} discussed the possibility of shock formation by 
waves traveling in the magnetosphere. \citet{tb05} 
% and \citet{bt07}
proposed that a dense transition layer with $kT\sim 100$~keV  forms between 
the magnetosphere and the stellar surface and radiates bremsstrahlung photons.
This model does not, however, explain the variations of the hard X-ray spectrum with the 
rotational phase and why the emission at different energies peaks at different phases.


\subsubsection{Low-frequency Emission}


In ordinary radio pulsars, radio emission is believed to come from the open field lines 
that connect the star to its light cylinder --- this part of the magnetosphere is 
persistently active (twisted) and carries electric current 
$I_{\rm open}\sim \mu\Omega^2/2c$.
The maximum voltage induced by stellar rotation on the open field 
lines is $\Phipc\approx \mu \Omega^2/c^2$ \citep{rs75}. 
For typical magnetar parameters, $\mu\simgt 10^{32}$~G~cm$^3$ and 
$\Omega\sim 1$~rad$^{-1}$, one finds $e\Phipc/m_ec^2\sim 10^8$, more than
sufficient to sustain pair creation.

However, magnetar radio emission from the pair plasma in the open bundle may be 
undetectable for two reasons. (1) Pair discharge can limit the voltage to a much lower
value $\Phie\ll\Phipc$;  then the power dissipated in the open bundle is 
small: $\Phie I_{\rm open}\sim 10^{29} (\Phie/10^{10}{\rm V})$~erg~s$^{-1}$. 
With a reasonable radiative efficiency this implies a low radio luminosity,
well below $L_{\rm radio}\sim 10^{30}$~erg~s$^{-1}$ observed in \XTE~\citep{crh+06}.
(2) The radio beam from the open bundle
may be narrow and miss our line of sight.

The fact that radio pulsations are associated with outbursts (\S\ref{sec:multitemporal}) 
suggests a connection with a magnetospheric twist.
A scenario with strong dissipation on open field lines, $\Phie I_{\rm open}$, enhanced 
by the twist in the closed magnetosphere \citep{tho08b}, is problematic
--- this corresponds to $\partial\Phie/\partial f<0$ in \Eq~(\ref{eq:evol}) and the twist 
around the open bundle should be quickly erased. 
Radio emission can be produced by the closed j-bundle itself. It 
is much thicker and more energetic than the open bundle, and therefore capable of 
producing much brighter radio emission with a broad pulse. 
The untwisting magnetosphere in XTE J1810-197 had a j-bundle with 
magnetic flux $f_j\sim 3\times 10^2 \fopen$, and carried electric current 
$I\sim 10^5I_{\rm open}$. Then a reasonable efficiency of radio emission 
$L_{\rm radio}/\Phie I\sim 10^{-3}$ is sufficient to explain the observed 
radio luminosity \citep{bel09}.
Given the high plasma density in the j-bundle, 
$n\sim 10^{18}-10^{20}$~cm$^{-3}$, the plasma frequency in the $e^\pm$ outflow
may approach the infrared or even optical band. 
Thus, plasma processes in the j-bundle might contribute to emission from radio to optical
\citep{egl02}.


\subsection{Spin-down Torque}


The twisted magnetosphere of an active magnetar is somewhat inflated due to the 
additional pressure of the toroidal field $B_\phi^2/8\pi$. 
This inflation increases the open magnetic 
flux $f_{\rm open}$ that connects the star to its light cylinder and therefore increases 
the spindown torque applied to the star \citep{tlk02}. 
% Electric 
% currents flowing in an axisymmetric j-bundle with magnetic flux $f_j$ and a small 
% twist angle $\psi<1$  increase the effective magnetic moment of the star by 
% $\Delta\mu/\mu\sim (\psi^2/4\pi)\ln (f_j/f_{\rm open})$ \citep{bel09}. 
A strong increase in spindown is expected for strong twists 
$\psi>1$, which also produce a higher magnetospheric luminosity.

However, no strict general relation between the X-ray emission and spindown 
is expected, in particular in non-axisymmetric magnetospheres. 
The torque is sensitive to the 
behavior of the small flux bundle $f_{\rm open}$, a tiny fraction of the total magnetic flux 
of the star, $f_{\rm open}/f_{\rm total}\sim \RNS/\RLC\sim 10^{-4}$. 
The open flux is much smaller than $f_j$ and may be outside the j-bundle. 
Therefore, the spindown torque may react to changes 
in the magnetospheric twist with a delay or sometimes even anti-correlate with 
the X-ray emission, depending on the details of the magnetospheric configuration.

It was also proposed that the persistent high spindown rate is due to strong plasma 
loading of the magnetar wind \citep{hck99,txs+13}. This proposal posits that the true 
dipole component is much smaller than inferred from the standard spindown formula 
$B\approx 3\times 10^{19}(P\dot{P})^{1/2}$~G, and the magnetic energy required to 
feed the magnetar activity is stored in much stronger multipoles. A challenge for 
this scenario is that it needs a dense plasma outflow that would energetically 
dominate at the light cylinder and ``comb out'' the magnetic field lines,  increasing 
the open magnetic flux and boosting the spindown rate. 
It is unclear if e.g. seismic activity of the star would be able to drive such an outflow 
 \citep{tdw+00}.

Magnetospheric flares are expected to impact the magnetar spindown.
The existing flare simulations show that the magnetic flux connecting 
the star to its light cylinder $f_{\rm open}$ is dramatically increased during the flare, 
a result of strong inflation of the twisted field lines \citep{pbh13}. 
The spindown torque exerted on the star $\dot{J}\sim-f_{\rm open}^2/2\pi c P$ 
becomes enormous for a short time $\Delta t$ comparable to $\Omega^{-1}=P/2\pi$ in 
the simulation, and produces a sudden increase of the 
rotation period $\Delta P>0$ --- an ``anti-glitch''.
When applied to the 1998 August giant flare in SGR~1900+14, the model gives
$\Delta P/P\sim 10^{-4}$, consistent with the observed anti-glitch \citep{wkv+99}.
The predicted $\Delta P$ may vary in more realistic models 
where the flare is not axisymmetric and the light cylinder is far outside of the main 
flare region. This might explain the non-detection of an anti-glitch 
($\Delta P/P<5\times 10^{-6}$) in the exceptionally powerful giant flare of SGR~1806-20 
on December~24, 2004, despite 
the observed ejection of a powerful outflow during the initial spike of the flare. 
Note also that the above picture does not work for anti-glitches that are not associated 
with giant flares, such as the one reported by \citet{akn+13}.


%\bibliography{review.bib}


%\end{document}
%
%##################################################################
%##################################################################
%##################################################################

