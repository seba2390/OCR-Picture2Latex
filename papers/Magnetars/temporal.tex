
%\section{Temporal Behavior}

\subsection{X-ray Pulsations, Spin-Down and Timing}
\label{sec:pulsations}
% MONDAY

%A basic observational property of magnetars is their X-ray pulsations.
%In the known magnetar population, pulsation periods lie roughly uniformly in the range 2--12 s. 
%The narrowness of this range itself is interesting and is discussed below.  
%Note however that
%magnetar-like behavior has been seen from two apparently rotation-powered pulsars
%having much shorter spin periods, 0.327 s \citep{ggg+08} and 0.407 s \citep{akts16}.
%Also, very recently, strong evidence for a 6.67-hr rotation period in a magnetar has been
%presented.  This is discussed further below.

Magnetar X-ray pulse profiles are generally very broad, with one or two components, and
with duty cycles that approach 100\% (see Fig.~\ref{fig:xrayprofiles} left).  X-ray 
pulsed fractions (the fraction of point-source emission that is pulsed)
are typically $\sim$30\% but range from $\sim$10\% \citep{ioa+99} to as high as $\sim$80\% \citep{kkp+12}.
The profile morphologies and pulsed fractions can vary strongly with energy.
Pulse profiles are generally stable apart from during outbursts
where large profile variations are common.
However there is at least one example of long-term, low-level pulse
profile evolution, in magnetar 4U~0142+61 \citep{gdk+10}.


\begin{figure}
\begin{minipage}{2.9in}
\hspace{-0.2in}
\includegraphics[scale=0.3]{xrayprofiles}
\end{minipage}
\hfill
\hspace{-0.55in}
\begin{minipage}{2.9in}
\vspace{0.4in}
\includegraphics[scale=0.13]{bursts}
\end{minipage}
\caption{(Left) Several X-ray pulse profiles of magnetars in the 1---10-keV band (courtesy R. F. Archibald).
(Right) Examples of single bursts from SGRs 1806$-$20 and 1900+14 shown with 7-ms time resolution in
the 2--60-keV band from {\it RXTE}/PCA data.  From \citet{gkw+01}.
}
\label{fig:xrayprofiles}
\end{figure}

Because the X-ray pulsations are generally the most practically observable,
long-term timing of magnetars has been done almost exclusively therein, with
some inclusions of radio data where available (see \S\ref{sec:multitemporal}).  
In the past, when X-ray telescope time was very difficult to acquire due to required
long exposures, timing was
done by measuring the pulse period at multiple epochs, typically spaced by months
to years \citep[e.g.][]{bs96,bss+00}.  This clearly demonstrated regular spin-down in
the first known sources, which distinguished them from the accreting
X-ray pulsars as these often spin up.

Today, the timing method used most commonly
is ``phase-coherent timing,'' borrowed from the radio pulsar world 
and brought first to the magnetar world thanks to the great sensitivity and ease of scheduling
of {\it RXTE} \citep{kcs99}.  In this technique, 
{\it every} rotation of the pulsar is accounted, sometimes over years, enabling precise measurements
of period and eventually spin-down rate.  Phase-coherent timing has been accomplished
long-term for the 5 brightest known magnetars using {\it RXTE} \citep[see][and references therein]{dk14}
and now {\it Swift} \citep[e.g.][]{akn+13}.  Phase-coherent timing has also sometimes been
done following magnetar outbursts, however pulse profile changes,
common in outbursts, make this difficult.  Moreover, as the source fades, particularly for
the `transient' magnetars that are faint in quiescence,
ever-longer integration times are required, often rendering long-term timing impractical.


%\begin{textbox}
%In phase-coherent timing, carefully spaced snapshots of the source are obtained, with integration
%times just long enough to determine the phase $\phi$ of the pulse, typically to within at least $\phi \simeq 0.1$.
%A stable pulse profile is crucial to permit a reliable fiducial point -- presumably
%corresponding to a fixed location on the neutron star -- for phase measurement.
%Given even a rough initial period (say from a periodogram in a single observation), 
%the initial spacing of the first two snapshots, typically separated by several hours,
%determines an improved period, thanks to the long time baseline.  With the next observed typically
%the next day, the improvement on the pulse period is improved further, thanks to the extended
%baseline.  With each improvement, the time baseline until the next snapshot can be extended.
%This works well until a period derivative begins to matter, which introduces a quadratic term
%into the phases, since
%\begin{equation}
%\phi(t) = \phi(t_0) + \nu (t-t_0) + \frac{1}{2}\dot{\nu} (t-t_0)^2,
%\end{equation}
%where the frequency $\nu \equiv 1/P$ with $P$ the pulse period.
%This introduction of the quadratic enables an initial meaurement of the spin-down rate $\dot{nu}$
%typically after only a few days of observations for magnetar-strength fields.  
%Once determined in this way, barring a glitch or large rotational instabilities,
%the phase can be monitored and modelled with only monthly snapshots, leading to
%precise values for $\nu$ and $\dot{\nu}$.
%\end{textbox}

Phase-coherent timing has enabled very precise measurements of $P$ and $\dot{P}$ 
now for many magnetars; by the
same token, however, deviations from simple spin-down become easily detectable with this technique.
In this regard, magnetars are quite prolific.  Ubiquitous in their rotational evolution is what is termed
``timing noise:'' apparently random wandering of phases particularly
on months to years time scales.  Such wandering is phenomenologically modelled by higher-order derivatives
of $P$ (or, equivalently, $\nu$) and in some cases, as many as 10 such derivatives are required \citep[e.g.][]{dk14}.
%Nevertheless, even given all the observed spin-down noise, in no case has any extended episode of spin-up been observed.
However, apart from the smoothly varying wander of timing noise,
and thanks to the available timing precision inherent to phase-coherent timing, 
sudden spin-ups (``glitches'') (and, more rarely, anti-glitches) are now observed frequently in magnetars
(\S\ref{sec:glitches}).

%Mention low-Pdot magnetars here?
% no in overall section

%One interesting question is regarding the origin of the observed narrow magnetar period range.  
Whereas most radio pulsars are thought to be born with periods
of at most a few hundred milliseconds, that the shortest known {\it bona fide} magnetar
has a relatively long 2-s period in spite of a young age 
is surely a result of rapid magnetic braking. 
%inevitable in so highly magnetized a neutron star.
%Faster rotators may exist, but they slow down so rapidly (in under $\sim$1000 years), they are rare to observe.
The long period cutoff of 12~s (but see below) has been more of a puzzle, which is related to the life-time
of magnetar activity \citep[e.g.][]{cgp00,vrp+13}.  
%The known magnetars are a very young population, with all ages almost certainly below $\sim$100 kyr.  
%If that is true, one puzzle is why the older -- and presumably slower --
%magnetars are not seen; this has been argued as independent evidence for magnetic field decay \citep[e.g.][]{cgp00,vrp+13}.  
%One possibility is that the decay of the high field proceeds very rapidly, on a $\sim 10^4$-yr time scale,
%so that older sources are less X-ray bright \citep{cgp00}.  However, there is no evidence for reduced X-ray luminosities
%for slower magnetars compared with faster ones.  On the other hand, detailed modelling involving magnetothermal
%evolution \citep[e.g.]{vrp+13} bears out the argument that there is rapid field decay on the aforementioned timescale,
%resulting in predicted rotational evolutionary tracks that stagnate near 10--20 s due to spin-down torque reduction.
%Hence many longer-period magnetars are not necessarily expected to be observed.

%In this light, 
%the central source of the supernova remnant RCW~103, 1E~161348$-$5055, is notable.  It
%shows a strange 6.67-hr X-ray periodicity with a variable pulse profile, as well as repeated large X-ray outbursts
%\citep{dcm+06b}.  Recently, \citet{aeb+16} and \citet{rbe+16} reported on the discovery of a bright magnetar-like burst from the
%source, coupled with another large X-ray flux outburst.  The source thus bears all the hallmarks of a magnetar,
%except for the anomalously long spin period. 
%%which is 2000 times longer than the previous record holder!  
%Simple magnetic braking cannot accomplish this for any reasonable initial set of parameters given the young age of the supernova remnant.
%However, the long period may be explainable by invoking a fall-back disk of material
%that slows down, via the propeller mechanism, the initially faster-rotating neutron star very early in its history
%\citep{li07,ha16}.  Such a scenario is plausible but leaves open the question of why such a mechanism occured only in this source
%and not in the any of the other two-dozen known magnetars.  
%In this magnetar picture, the source should at least today
%be spinning down due to conventional magnetic braking.  However, such slow down will be very difficult to measure:
%for a $10^{15}$-G magnetar spinning once every 6.67 hr, the expected $\dot{P}$ is predicted to result in a
%phase deviation of unity only after 5000 years!  This will be impossible to measure, even if there were not the observed extreme pulse
%profile variations (which themselves render phase-coherent analyses challenging).
%Regardless, this disk-slow-down scenario does not appear to solve
%the previously mentioned long-period boundary problem as there is clearly an enormous gap from 12 s to 6.67 hr.

Finally, one source shows unusual timing behavior that is worthy of note.
1E~1048.1$-$5937 has shown episodes of large (factor of 5--10) spin-down rate variations which appear to be,
curiously, quasi-periodic on a time scale of $\sim$1800 days \citep{akn+15}.  These episodes have, in all cases
observed thus far, followed major flux outbursts, with a delay between the radiative outburst and spin-down fluctuations
of $\sim$100 days.  
%The origin of the quasi-periodicity is not known, but with only 3 such events observed, one may question
%whether indeed it is real.  
A fourth such flux outburst has very recently begun, at the approximate epoch
predicted by the apparent quasi-periodicity \citep{atsk16}. 
%The interested reader is encouraged to watch the literature
%to see if the trend indeed persists.  A fourth such event would be difficult to dismiss and would require some physical explanation.
%One possible origin is suggested by \citep{akn+15} to lie in a magnetospheric instability;  the rationale
%for this is the apparent similarity to other curious long-time scale variations in the spin-down rates and radio pulse profiles
%of some radio pulsars \citep{lhk+10}.




\subsubsection{Glitches}
\label{sec:glitches}

Phase-coherent timing of magnetars by {\it RXTE} enabled the discovery that magnetars are among
the most frequently glitching neutron stars known \citep{klc00,dkg08}.  A ``glitch,''  a phenomenon common to
young radio pulsars \citep[e.g.][]{ymh+13}, consists of a sudden spin-up, typically involving $\Delta\nu/\nu$ in the range $10^{-9} - 10^{-5}$
in both magnetars and radio pulsars.  
%Note however for the same fractional change, since magnetars rotate so much slower, the effective
%change in angular momentum during a magnetar glitch is $\sim 100\times$ smaller.
Also typically associated with glitches in both radio pulsars and magnetars are long-term changes in spin-down rate $\dot{\nu}$, with
typical $\Delta\dot{\nu}/\dot{\nu}$, almost always positive, of at most a few percent and usually far smaller in radio pulsars.
A common phenomenon is glitch recovery, in which a sizable fraction of the glitch (in radio pulsars, 
typically 0--0.5) recovers quasi-exponentially within a week or two following the glitch.
A remarkable behavior seen practically exclusively in magnetars is extremely strong glitch recovery, such that the full
initial spin-up is recovered, and in some cases {\it over}-recovery is seen, such that the overall effect is a spin-{\it down}
\citep[e.g.][]{gdk11}.
These strong recoveries involve initially very large values of $\dot{\nu}$, sometimes upwards of 10$\times$ the 
pre-glitch long-term $\dot{\nu}$ \citep[e.g.][]{kgw+03,kg03,dis+03,wkt+04}. 
Additionally, in one magnetar-like glitch case, the level of timing noise was observed to be greatly
enhanced for several years following the over-recovered glitch \citep{lnk+11}.

%What is the origin of the remarkable glitch recoveries seen in magnetars?  The answer to this question is presently unknown.
%However, if, during the glitch recovery, the external torque on the neutron star remains constant, then
%a very large fraction of the moment of inertia of the star (which of necessity involves the stellar core)
%must decouple from the crust at the glitch epoch, then slowly re-couples.  
%This is not formally ruled out, however 
The coincidence of many such glitches
and recoveries with large radiative outbursts and their relaxations is suggestive of magnetospheric phenomenon. 
Indeed many -- though not all -- magnetar glitches occur at epochs of large X-ray flux outbursts, which themselves
are often accompanied by many outward radiative changes such as short bursts and X-ray pulse profile changes (see \S\ref{sec:outbursts} below).
Curiously some magnetars have only shown radiatively silent glitches \citep[e.g. 1RXS J170849.0$-$400910; ][]{sak+14} and some have shown both
silent and loud glitches \citep[e.g. 1E~2259+586; ][]{dk14}.  
%It could be that glitches are an internal phenomenon that can trigger magnetospheric
%twists, depending on, for example, depth.  This is discussed further in \S\ref{sec:andrei}.

Several apparent ``anti-glitches'' have also been reported in magnetars
\citep{wkv+99,sag14}.
These events appear consistent to within the available time resolution with being sudden spin-{\it downs}  of magnetars
and have not been seen in radio pulsars at all.  The most convincing of these 
is the anti-glitch reported in 
1E~2259+586 \citep{akn+13} in which an apparent sudden spin-down of amplitude $\Delta\nu/\nu \sim 10^{-7}$ accompanied
a bright, short X-ray burst and a long-lived flux outburst.  Although this event could in principle have resulted from
an over-recovered spin-up glitch, the recovery time scale of the initial event would have had to be at most a few days, much shorter than
any previously observed glitch recovery.  
%Actually, the 1E~2259+586 anti-glitch was not the first reported in a magnetar.
%Previously a possible very large ($\Delta\nu/\nu = 1 \times 10^{-4}$) anti-glitch was reported in magnetar
%SGR 1900+14 \citep{wkv+99} near its 1998 giant flare.  However the spin-down occured in an 80-day observing gap, 
%so could not be demonstrated to have been a truly sudden event.  Also,
%\citet{sag14} reported an anti-glitch in magnetar 1E 1841$-$045 however there was no evidence for any radiative
%enhancement at the relevant epoch and an analysis of the same data by \citet{dk14}
%does not find an such anti-glitch.  
The origin of anti-glitches is still debated; see \S\ref{sec:andrei} for further discussion.



\subsection{Transient Radiative Behavior: Bursts, Outbursts, Giant Flares \& QPOs}
\label{sec:transient}

%The original hallmark observational characteristic of a magnetar is
%its dramatic X-ray and soft-gamma-ray bursting and flaring.  Magnetars,
%when active, produce sudden radiative outbursts that span orders of
%magnitude in temporal and luminosity space.  
%Some nomenclature introduction is in order.  
The term ``bursts'' here is used to mean the short, few millisecond to
second events, some of which are followed by longer-lived ``tails,'' an afterglow
of sorts.  The term ``outburst'' is used to describe a sudden but much longer-lived
(weeks to months) flux enhancement, which typically involves many of the shorter
bursts, and involves a long (many months) ``tail'' or afterglow.
The term ``giant flare'' is reserved exclusively for what appear to be catastrophic
events involving the sudden release of over $10^{44}$~erg of energy.
%These events are the ones that saturate detectors and invoke superlatives like
%`greatly outshone the entire cosmic hard X-ray sky for a few moments' or
%`released as much energy in a few seconds as does the Sun in 250,000 years.'
``Quasi-periodic oscillations'' have been seen in the tails of some giant flares.
It is fair to say that nearly all magnetar radiative variability unrelated
to their pulsations can be placed into one of these categories.

\subsubsection{Bursts}
\label{sec:bursts}

Short bursts are by far the most common type of magnetar radiative event.
There are magnetars which have emitted thousands
of bursts, usually very much clustered in time, and there are magnetars that, in spite
of intensive monitoring programs, have shown at most a handful of bursts.
Indeed the former were long thought to be the `SGRs' of the magnetar population,
with the latter being the `AXPs.'  However with further study it appears that 
there is a full spectrum in burst rates, and some sources that might originally
have been thought to be extremely active (e.g. SGR 0526$-$66) have lain dormant
for decades subsequently \citep{kkm+03,tem+09}.  Similarly, sources that for years were not known to burst
(e.g. 1E~2259+586) suddenly entered an active burst phase, emitting several dozen
bursts in a few days \citep{kgw+03}.  This is one key reason the AXP/SGR classification scheme
seems obsolete.  Note that bursts are more common during outbursts (see \S\ref{sec:outbursts})
however there are examples of bursts occuring when the source appears otherwise in
quiescence \citep[e.g.][]{gkw02}.

There have been multiple detailed statistical studies of the properties of
short magnetar bursts \citep{gwk+99,gwk+00,vkg+12,lgkk13}.
Here we summarize the findings of these studies.
Burst peak luminosities can be hyper-Eddington but span a broad spectrum, typically
ranging from $10^{36}$ to $10^{43}$~erg~s$^{-1}$, with bursts detected
right down to the sensitivity limit of current X-ray detectors.  
Burst durations span over two orders of magnitude, ranging between a few ms and a few sec,
with distributions typically peaking near $\sim$100 ms.
Burst fluence distributions are generally well described with power-law functions
of indexes in the range  $-$1.6 to $-$1.8 \citep{gwk+99,gwk+00}.
Bursts are usually but not always
single-peaked, with the rise typically faster than the decay.
Interestingly, although some studies have shown that short bursts arrive
randomly in pulse phase \citep[e.g.][]{gwk+99,gwk+00}, others have found a preference
for bursts near the pulse maximum \citep[e.g.][]{gkw04}.
Figure~\ref{fig:xrayprofiles} (right) shows examples of short magnetar bursts.


Some bursts show long, sometimes several-minute tails \citep[e.g.][]{lwg+03,gwk+11,mgk+15}, during which the pulsed
flux is sometimes greatly enhanced \citep[e.g.][]{wkg+05,mgw+09,akb+14}.  In this way they are almost like miniature giant flares
(see \S\ref{sec:giantflares}).  Tails fade slowly, with decays well described by relatively flat power laws
of indexes well under unity.  Though much lower in flux, the long-duration tails
can sometimes contain significantly more energy than the burst itself.  Ratios of
burst to tail energies can vary by over an order of magnitude in different sources and even in
different bursts in the same source \citep[e.g.][]{gwk+11}.



%Observationally, it has been suggested \citep{wkg+05} that there are two classes
%of bursts: those with long tails in which tail energy greatly exceeds burst energy, in which 
%This dichotomy has been suggested to align with the predicted picture of
%\citet{lyu0?} 


\subsubsection{Outbursts}
\label{sec:outbursts}

A magnetar `outburst' is an event consisting of a large (factor of 10--1000) and usually sudden increase in
the source X-ray flux, sometimes as high as $10^{36}$~erg~s$^{-1}$ \citep[see][for a compilation]{re11}.  These events are generally 
accompanied by a bevy of other radiative anomalies such as spectral hardening, change in pulsed fraction, pulse profile changes 
(often from a simpler to a more complex profile), multi-wavelength changes, and multiple short X-ray bursts.
Most outbursts for which there are available data also are accompanied by some form of timing
anomaly, ususally a spin-up glitch or occasionally an anti-glitch.  The flux following an outburst
usually decays on multiple time scales, with a very rapid initial decay within minutes to hours
(and hence which is often missed by observatories) followed by a slower decay \citep[e.g.][]{wkt+04},
sometimes termed an `afterglow' which can last months to years.  These slowly fading afterglows are often quasi-exponential
\citep[e.g.][]{rit+09} but sometimes have an interesting
time evolution, with periods of power-law decays interrupted by few-month periods of flux stability \citep[e.g.][]{akt+12}.
In general, magnetar outbursts (even those from the same source) show a variety of time scales for their 
relaxations \citep[e.g.][]{etr+13}.
As in the tails of the more common bursts, there is a diversity in the ratios of burst to afterglow energies, ranging from a
few percent to one to two orders of magnitude \citep{wkt+04}.  
An example of an X-ray light curve from a magnetar in outburst
is shown in Figure~\ref{fig:outburst}.
A description of the spectral evolution of magnetars during and following outbursts is
deferred to \S\ref{sec:spectral} below.
%The aforementioned strange quasi-periodic outbursts of 1E~1048.1$-$5937 are also notable for their slow rises;
%most magnetar outbursts begin suddenly.

Some sources have shown no outbursts over decades \citep[e.g. 1RXS J1708$-$4009; ][]{dk14}
while others
have had multiple \citep[e.g. SGRs 1806$-$20, 1900+14, 1E 1547.0$-$5408;][]{wkf+07,ggo+11,nkd+11}. 
In the past, AXPs have tended to be associated with sources
that have few if any outbursts, whereas SGRs are sources that are much more outburst-active.  However in terms of
reasonable outburst rate estimates, there is no clear evidence for bi-modality, suggesting a full continuum of
activity.  
%Indeed a continuum of behaviors is predicted by current models of magnetothermal evolution and
%depends on source age and magnetic field \citep[][see \S\ref{sec:andrei}]{pp11a}.

\begin{figure}
\begin{minipage}{2.9in}
\includegraphics[scale=0.28]{fluxevol2259}
\end{minipage}
\hfill
\hspace{-0.75in}
\begin{minipage}{2.9in}
\hspace{.35in}
\includegraphics[scale=0.12]{2259}
\end{minipage}
\caption{(Left) Flux evolution during and after the outburst of 1E 2259+586.
Note the initial rapid decrease in flux on the first day, when the vast majority of associated short bursts were detected, followed by the slower
subsequent fading. 
(Right) Spectral evolution of 1E 2259+586 through and following its 2002 outburst (discussed in \S\ref{sec:specevol}).
Top to bottom: unabsorbed flux (2–10 keV), blackbody temperature (kT),
photon index, blackbody radius, and ratio of power-law (2–10 keV) to bolometric blackbody flux.
Horizontal dashed lines denote the values of each parameter fortuitously measured
one week prior to the outburst.  From \citet{wkt+04}.
}
%(Right) Compilation of post-outburst light curves of magnetars showing
%the variety of flux relaxation behaviors.
%From \citet{re11}.}
\label{fig:outburst}
\end{figure}

%Note that the known magnetar population is likely heavily biased toward
%sources that are more outburst-active, since this is very frequently how they are discovered.  
%This is thanks to sensitive all-sky monitors built to study gamma-ray bursts, such as
%the Burst Alert Telescope aboard {\it Swift} and the Gamma-Ray Burst Monitor aboard {\it Fermi}.
%Indeed \citet{ok14} showed a clear increase in the rates of magnetar discoveries at the times
%of the launches of these two telescopes.   
The term `transient' magnetar was introduced to describe those sources which have very low
$< 10^{33}$~erg~s$^{-1}$ quiescent luminosities, which generally go unnoticed until they produce
outbursts involving flux increases of factors of 100-1000, accompanied by bright bursts that 
trigger monitors \citep[e.g.][]{mgz+13}.
The first discovered transient magnetar was XTE J1810$-$197.  It was caught in an outburst in 2003
\citep{ims+04} and observed to decay on a year time scale
\citep{gh07}. 

Transient magnetars may in fact be the norm among the magnetar population, with the
well studied bright sources like 4U~0142+61 or 1E~2259+586 unusual for relatively high
quiescent luminosities, upwards of $10^{35}$~erg~s$^{-1}$.  Why some magnetars in quiescence
are much brighter than most is an interesting question.
Some transient magnetars, e.g. SGR 0418+5729 \citep{ret+10} or Swift J1822.3$-$1606
\citep{rie+12,skc14}, have low spin-inferred dipole magnetic fields and are thought to possess
strong internal toroidal fields.

%This also is a plausible explanation for the apparently low-magnetic-field magnetars like
%SGR 0418+5729 \citep{ret+10} or Swift J1822.3$-$1606 \citep{rie+12,skc14}. 
%(see \S\ref{sec:pulsations} above).
%This is discussed in more depth in \S\ref{sec:andrei}.


%Further,
%some magnetars ever never reach a truly `quiescent' flux state given that sometimes
%new outbursts occur before a full relaxation from the previous one \citep[e.g.][]{nkd+11}.  On the other hand, at least the five
%brights sources monitored over 16 years with {\it RXTE}, and now with {\it Swift}, have generally stable
%fluxes for years after any outburst.  




\subsubsection{Giant Flares}
\label{sec:giantflares}

The queen of magnetar radiative outbursts is the giant flare.
Thus far, only three giant flares (GFs) have been recorded, all from
different sources.  These events occured on March 5, 1979 \citep[SGR
0526$-$66; ][]{ekl+80}, August 27, 1998 \citep[SGR 1900+14][]{hcm+99},
and December 27, 2004 \citep[SGR 1806$-$20; ][]{hbs+05,mgv+05,bzh+07}.  These had peak
X-ray luminosities in the range $10^{44} - 10^{47}$~erg~s$^{-1}$, and each
was characterized by total energy release of over $10^{44}$~erg~s$^{-1}$
in the X-ray and soft-gamma-ray band.  The third flare was roughly 100
times more energetic than the first two and was by far the most luminous
transient yet observed in the Galaxy; it briefly outshone all the stars
in our Galaxy by a factor of 1000.  All three GFs have come from sources
traditionally called ``SGRs,'' and this one phenomenon appears to be the
{\it only} behavior that could in principle distinguish SGRs from AXPs;
on the other hand, SGR 0526$-$66 has now been inactive for several decades
since its GF \citep{kkm+03,tem+09}, and if it had been discovered during
that interval, probably would have been classified as an AXP.

\begin{figure}
\begin{minipage}{2.9in}
\hspace{-0.3in}
\vspace{0.2in}
\includegraphics[height=3in]{giantflare}
\end{minipage}
\hfill
\begin{minipage}{2.9in}
\vspace{0.4in}
\hspace{-0.35in}
\includegraphics[scale=0.15]{israel}
\end{minipage}
\caption{(Left) The 2004 giant flare from SGR 1806$-$20.  a. 20--100-keV time history with 0.5-s resolution
from {\it RHESSI}, showing the initial spike (which saturated the detector) at 26 s.  The inset shows
a pre-cursor burst that occured just prior (with 8-ms resolution).  The oscillations in the decaying tail are at the neutron-star rotation period.  b. Blackbody temperature of the emission; see original
reference for details.  From \citet{hbs+05}.
(Right) Power spectrum of the SGR 1806$-$20 giant flare 2--80-keV light curve in the interval 200--300 s. 
Two low-frequency peaks at $\sim$18 and $\sim$30 Hz are visible, together with a small excess at $\sim$95 Hz.
From \citet{ibs+05}.
}
\label{fig:1806gf}
\end{figure}

The light curve and effective temperature evolution of the SGR
1806$-$20 giant flare is shown in Figure~\ref{fig:1806gf} (left) as an example
\citep[see][for details]{hbs+05}.
In the initial hard spike, lasting only $\sim$0.2~s,
over $10^{46}$~erg were released (assuming
a distance of 15 kpc).  Its peak luminosity was $10^{47}$~erg~s$^{-1}$.
This spike was followed by a several-minute long decay superimposed on
which are pulsations at the 7.56-s period of this source.  The total
fluence in this six-minute tail was $10^{44}$~erg.

%Giant flares are
%thought to be a result of a catastrophic instability in the star resulting in the
%eruption of a hot fireball consisting of a pair plasma, with little baryon
%contamination \citep{td95}.  The decaying portion of the giant flare is thought
%to represent the abatement of a trapped fireball, which is contained magnetically
%following the sudden initial outflow.  The pulsations are due to the rotation
%of the neutron star, which modulates the emission due to the magnetospheric
%emission anisotropy.  

The enormous peak luminosities observed in the giant flares together with their spectral
peak in the soft gamma-ray band makes them interesting possible counterparts of short, hard 
gamma-ray burst (GRB).  \citet{hbs+05} estimated that $\sim$40\% of all short, hard GRBs
detected by the BATSE instrument aboard the {\it Compton Gamma-Ray Observatory} could have
been GFs from distant extragalactic magnetars.  \citet{omq+08} suggested GRB 070201 may
have been a GF from a magnetar in M31, and \citet{hrb+10} suggested another such example,
though this interpretation is difficult to confirm in either case.  Moreover, young magnetars
cannot dominate the short GRB population as the latter are known to be commonly found at
large offsets from late-type galaxies \citep{ber14}.

\subsubsection{Quasi-Periodic Oscillations}
\label{sec:qpos}

A remarkable phenomenon detected during the pulsating tails of giant flares is the high-frequency
quasi-periodic oscillations (QPOs).
These are believed to be seismic vibrations
of the neuton star and may inform us on properties of the stellar interior.  
%Specifically, they may provide
%information on the internal magnetic field strength, the coupling between the crust and core, and on
%the dense matter equation of state.
These oscillations were first reported in phase-resolved portions of the tail emission of SGR~1806$-$20
following its 2004 GF \citep{ibs+05} at 92.5~Hz (see Fig.~\ref{fig:1806gf} right), 
and in phase-resolved emission also immediately post-GF in
the 1998 event from SGR~1900+14 at 84~Hz \citep{sw05}.  Several other QPO frequencies were also reported in these data,
which were obtained with {\it RXTE}.  Similar signals from SGR 1900+14 were also seen in {\it RHESSI} data
and new frequencies including a strong 626.5-Hz QPO, again only in specific rotational phase intervals but at a different time
\citep{ws06}.
%Some of these QPOs had extremely high amplitudes, in the range XXXX.
%LIFETIMES.
More recently, possible QPOs have also been reported from much fainter bursts from 1E 1547.0$-$5048
\citep{hdw+14}, however these oscillations have proven elusive in other sources
\citep[e.g.][]{hwu+13}.  
%The precise frequencies observed and their durations in some cases are consistent with expectations
%for stellar interior models \citep[e.g.][]{lev07}, but not always \citep[e.g.][ and see \S\ref{sec:andrei}]{gcs+12}.  
%Why certain frequencies appear but
%not others is also challenging to explain.  
%Sorely needed are more data containing QPOs.  However,
%as QPOs are clearly best detected in GFs, and GFs are very rare, the opportunities
%for understanding the origin of magnetar QPOs are few and far between and must await the next GF.


\subsection{Temporal Properties of Low-Frequency Emission}
\label{sec:multitemporal}

While every known magnetar has been detected as an X-ray pulsar, small handfuls have had
pulsations detected in the optical and radio bands.

Optical pulsations have been detected in just three magnetars \citep{km02,dmh+05,dml+09,dml+11}. 
%1E~1048.1$-$5937 \citep{dml+09} and SGR 0501+4516 \citep{dml+11}. 
The optical pulses seen thus far are comparably broad and similar to the corresponding X-ray pulse
profile, at least within the limited optical statistics. 
Importantly, all are detected with high pulsed
fractions ranging from 20\% to 50\%, in one case {\it higher} than in X-rays \citep{dml+11}. 
%which is hard to reconcile with the optical emission
%coming from reprocessing of the X-rays in a surrounding disk.  
This is strongly suggestive of a magnetospheric origin (see \S\ref{sec:andrei}).
Only three have had optical pulsations detected, and six more have shown optical and/or infrared emission not
yet seen to pulse (Table~\ref{ta:srcs}).  
%and Table 2 of the online catalog.  
%Near-infrared emission in particular (K band) is a favorite magnetar observing window, because of the high
%interstellar extinctions associated with magnetars' Galactic plane locations.
The overall picture of the relationship of the infrared emission to the X-rays remains unclear.
In several cases, clear infrared enhancements have been noted at the time of outbursts, often
with the infrared correlated with the X-rays \citep{tkvd04,icm+05,rtv+04}.
However in some cases, no correlation has been seen \citep{dk06,trm+08,wbk+08,tgd+08}.
%A clear correlation was seen between the infrared and X-ray fluxes after the 2002 outburst
%of 1E~2259+586, with identical flux decay indexes \citep{kgw+03,tkvd04}.
%\citet{icm+05} also reported similarity in the flux variability in X-rays and infrared emission.
%\citet{rtv+04} reported a correlation between infrared and X-ray fluxes following the outburst
%of XTE J1810$-$197, however \citet{crp+07} disputed this, and \citet{trm+08} later confirmed uncorrelated
%infrared variability.
%\citet{dk06} found a lack of correlation between variability in the infrared and the X-rays for 4U~0142+61.
%Near-infrared observations of 1E~1048.1$-$5937 suggest a possible correlation with X-ray flux,
%although the correlation does not seem exact \citep{wbk+08,tgd+08}.
%A discussion of the overall magnetar spectrum is deferred to \S\ref{sec:spectral}.

%\subsubsection{Radio}
%\label{sec:temporal_radio}

Radio pulsations have now been
detected in four magnetars \citep{crh+06,crhr07,lbb+10,sj13,efk+13},
%, 1E~1547.0$-$5408 \citep{crhr07},
%PSR J1622$-$4950 \citep{lbb+10}, and SGR J1745$-$2900 \citep{sj13,efk+13} 
five if counting the recent magnetar-like outburst from the 
radio pulsar PSR J1119$-$6127 \citep{akts16}.  
All four are `transient' magnetars and their radio emission
is transient too, associated with an X-ray outburst.
In XTE J1810$-$197 (the first detected radio-emitting magnetar), pulsations were
observed after, but not before, its 2003 outburst.  The radio pulsations
disappeared in late 2008, with no prior secular decrease in radio flux \citep{crh+16}.
Behavior in the other sources is similar \citep{crhr07,lbb+12}.
The persistent (i.e. non-transient) magnetars and several other transient magnetars 
have been searched for radio pulsations but with no detections 
\citep[e.g.][]{lkc+12,tyl13}.  
%Note that tentative claims of low-radio-frequency
%detections of some magnetars \citep[e.g.][]{mmt+05} have not to our knowledge been confirmed.
In the 4 magnetars with confirmed radio pulsations, the radio emission in all
cases is very bright, shows large pulse-to-pulse variability, with pulse morphologies,
both single and average, that can be broad and which generally change enormously on time scales of minutes.
They can be punctuated by spiky peaks that can be much shorter than the
pulse period.  This radio emission is also
highly linearly polarized, with polarization fractions of 60\%--100\%.
%The radio spectrum is discussed in \S\ref{sec:spectra_radio}.
No evidence for a radio burst at the epoch of a giant flare has been
seen \citep{tkp16}.


%Also, the radio emission in all cases has a surprisingly flat spectrum,
%with spectral indexes near zero and in some cases inverted (see \S\ref{sec:spectra_radio}).
%Indeed for an extended period after its 2003 outburst, XTE J1810$-$197 had
%greater flux above 20 GHz than any known radio pulsar \citep{crh+06}.
%Such flat spectra are remarkable as they are very different from
%those of radio pulsars, for which the average spectral index is $-1.8$
%\citep{mkkw00a}.

%\begin{figure}
%\includegraphics[height=3.5in]{radiopulses}
%\caption{
%Single radio pulses from XTE J1810$-$197 at 2 GHz and 42 GHz.
%a.  Typical set of 40 consecutive pulses seen with the Green Bank Telescope  (GBT)
%on MJD 53857 at 2 GHz.  The average pulse is shown at the top and was seen
%to be highly variable from day to day.
%b.  A train of $\sim$115 consecutive pulses at 42 GHz also from the GBT.
%The inset shows the single brightest pulse in the train (with 81.92~$\mu$s resolution).
%From Camilo et al. 2006.}
%\label{fig:radiopulses}
%\end{figure}

%Magnetar radio emission is transient, appearing near the epochs of X-ray outbursts and fading on time scales of months to years.
%The first identified radio emitting magnetar was also the first identified transient source,
%XTE~J1810$-$197, which had an outburst in 2003 \citep{ims+04,hgb+05}.  
%The bright, pulsed radio emission as observed beginning over a year
%post-outburst was clearly not present pre-outburst \citep{crh+06}.  The source's radio flux decayed 
%in 2006 but remained stable until its disappearance in 2008, unlike the secular decay
%at X-ray energies \citep{crh+16}.
%No secular decrease in flux preceded the disappearance.
%The discovery and properties of the radio emission from the second radio-detected
%magnetar, 1E~1547.0$-$5408 \citep{crhr07} were similar to those of XTE J1810$-$197.  However, the
%third detection was different, in that the source, PSR J1622$-$4950, 
%was first found in the radio band as a radio pulsar \citep{lbb+10}.  \citet{ags+12} then showed
%from X-ray observations that the source was likely relaxing following a transient-magnetar-type
%outburst.  Further monitoring revealed
%its radio properties to be similar to the first two \citep{lbb+12}.

One radio-detected magnetar, 
SGR 1745$-$2900, is located in the Galactic Center \citep{mgz+13,kbk+13,sj13,efk+13}.
The radio properties of the magnetar are similar to those of the first three, 
with evidence for independent X-ray/radio flux evolution \citep{laks15,tek+15}. 
Notable is the far lower than expected interstellar scattering
\citep{sle+14,bdd+14}.  
%Although not relevant to our understanding of magnetars, 
This is exciting as it suggests renewed hope for finding
radio pulsars for dynamical studies near Sgr A* and for constraining properties
of the Sgr A* accretion disk \citep{efk+13}.
%Also, the highly linearly radio polarized pulse profile of SGR J1745$-$2900 also enabled
%\citet{efk+13} to constrain for the first time the strength of the magnetic field in the 
%accretion disk surrounding Sgr A*.




