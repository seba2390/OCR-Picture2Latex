


\subsection{Burst Spectra}
\label{sec:burstspectra}

\subsubsection{Short Bursts}

Bursts are generally spectrally much harder than the persistent X-ray
emission from magnetars and though easily detectable below 10~keV,
peak above that energy.  Hence, they are best studied by broadband
X-ray instruments or by combining simultaneous data from multiple
instruments, if possible.  Multiple different models have been used
to describe burst spectra, including a simple blackbody, double
blackbodies, optically thin thermal bremsstrahlung (OTTB) models, or
Componization models (a power law with an expontial cutoff).  Even with
very broadband spectra (e.g. 8--200 keV) it is hard to distinguish among
these models \citep[e.g.][]{vkg+12}; moreover
sometimes the overall spectra of bursts in a cluster changes with epoch
\citep[e.g.][]{vgk+12}.  Regardless of model, in all studies, spectral
hardness is found to be related to burst fluence.  In most cases the
two appear anti-correlated \citep{gwk+99,gwk+00,gkw+01,vkg+12} and in
some cases correlated \citep{gkw04}.  Tail spectra are typically well
modelled by blackbodies that show decreasing $kT$ at constant radius
\citep[e.g.][]{gwk+11,akb+14}.


%\subsubsection{Spectral Features in Short Bursts}

Curiously, an apparent emission feature near
$\sim$13~keV has been noted in the spectra of bright magnetar bursts detected with {\it
RXTE}/PCA in the sources 1E~1048.1$-$5937, 4U~0142+61 and XTE~J1810$-$197
\citep{gkw02,wkg+05,gdk11,cgsk16}.  These features are variable,
occuring transiently during the burst evolution, but are not subtle:
they are easily visible by eye, with equivalent widths of $\sim$1~keV.
Although it is tempting to argue these features are of some unknown
instrumental origin (even though they are not always seen in {\it RXTE}
magnetar burst data), \citet{akb+14} found evidence of it in {\it NuSTAR}
data from 1E~1048.1$-$5937, which happened to burst during an observation.
The origin of these burst features is unknown; if some form of cyclotron
emission, it is unclear why all the sources in which it has been observed
show it near the same energy, since presumably they have a variety of
field strengths in the emission region. There is no spectral line known
with that energy.

\subsubsection{Giant Flares}

Described in \S\ref{sec:giantflares}, the initial brief spike seen in magnetar
giant flares is extremely hard, peaking in the soft gamma-ray band and extending
at least to MeV energies. It is
followed by a softening on time scales of seconds to minutes. In
Figure~\ref{fig:1806gf}, where the 2004 giant flare of SGR 1806$-$20 is shown, a blackbody
describes the data reasonably well, and
the initial $kT$ is $\sim$175~keV \citep{hbs+05,bzh+07}.  
Following the spike was a few-second decay whose spectrum was non-thermal, described
by a power law of index $\Gamma \sim 1.4$ and then a series of pulsations at the 7.5-s spin period, with spectra consisting
of a combination of blackbody and power-law emission.  Both softened over the next few minutes,
from $kT \simeq 11$~keV to 3.5~keV and $\Gamma \simeq 1.7$ to 2, for the two components,
respectively.
Similar initial and subsequent spectra were seen in the
1998 giant flare from SGR 1900+14 \citep{hcm+99}.  Although at the peak of the
1979 giant flare from SGR 0526$-$66 the spectrum may have been slightly softer than in the latter
two cases (the best estimate is $\sim$30~keV), 
the overall trend of a very hard spike and subsequent softening were also seen \citep{fek+81}. 
In some ways, giant flare spectral evolution mirrors that of the X-ray emission in less-energetic
magnetar outbursts:  sudden hardening and subsequent softening; see \S\ref{sec:specevol}.  
%However, the emission in giant flares is thought to originate from an expanding e$^+$/e$^-$ fireball
%triggered by a large-scale reconnection/interchange instability of the stellar magnetic field \citep{td95},
%rather than by a simple magnetospheric twist.  This is discussed in more detail in \S\ref{sec:andrei}.

%However in giant flares,
%due to instrumental saturation, the true spectrum of the initial spike is not
%known with certainty; nevertheless the total flux at Earth has been estimated to be
%1 erg~cm$^{-2}$!

\subsection{Persistent Emission}
\label{sec:persistentspectra}

The X-ray spectra of magnetars in quiescence fall into two broad classes:
those in high-quiescent-luminosity sources (the `persistent' magnetars)
and those in low-quiescent-luminosity sources (the `transient' magnetars).
The observational distinction is based both on quiescent luminosity ($\gapp 10^{33}$~erg~s$^{-1}$
for persistent sources like 1E 2259+586 or 4U 0142+61 versus
$\lapp 10^{33}$~erg~s$^{-1}$ for transient sources like XTE J1810$-$197 or
SGR J1745$-$2900) and on flux dynamic range in outbursts (factor of $\lapp$100 in persistent
sources versus $\gapp$100 in transient sources).

\noindent
{\it Classic Magnetars in quiescence}
show multiple-component X-ray spectra that are usually well parameterized in the 0.5--10-keV band
by an absorbed blackbody of $kT \simeq 0.3-0.5$~keV plus a power-law component of photon
index in the range $-2$ to $-4$ (see OK14 for a compilation).  An example of such
a spectrum is shown in Figure~\ref{fig:spectrum} (left).   Typically
the non-thermal component begins to dominate the spectrum above $\sim$3--4 keV.
At least qualitatively, the thermal component is thought to arise from the hot neutron-star
surface, while the power-law tail likely arises from a combination of atmospheric 
and magnetospheric effects.
%A correlation between power-law index and spin-down rate was reported by \citep{mw01} and
%later confirmed by \citet{kb10} and is a cornerstone of twisted magnetosphere models
%(see \S\ref{sec:andrei} below).
%In these classic spectra, a diversity in the relative normalizations 
%of the two components is seen.  
Typically the power law dominates energetically by
at least a factor of two, although in the soft X-ray band this quantity depends
strongly on the absorption, since $N_H$ and $kT$ are generally highly covariant. 
Double blackbodies can often fit magnetar spectra as well, and in rare cases double power laws.
Note, however, as described in \S\ref{sec:andrei}, the soft-band X-ray spectrum is thought
to arise physically from a complicated blending of surface thermal emission distorted
by the presence of a highly magnetized atmosphere, then Comptonized by currents
in the magnetosphere which can themselves result in surface heating via return currents.
Hence the simple e.g. blackbody plus power-law or double-blackbody parameterizations should
be seen as merely convenient and readily available (i.e. in XSPEC) descriptions of the data 
rather than actual measurements of physical properties.  
%The latter requires detailed fitting
%of custom models that include all the above effects, as described in \S\ref{sec:andrei}. 

Observations using {\it INTEGRAL} and the {\it Rossi X-ray Timing Explorer}
led to the surprising discovery just over a decade ago
that for persistent magnetars, the spectrum turns up above 10 keV, such that the bulk of
their energy comes out {\it above} the traditional 0.5--10-keV band \citep{khm04,khdc06}.
This prominent hard spectral component is shown for magnetar 1E 2259+586 in Figure~\ref{fig:spectrum} (left).
Such hard components have been seen now in six sources in quiescence.
\citet{kb10} and \citet{enk+10} 
reported a possible anti-correlation between degree of spectral up-turn and spin-down rate 
and/or spin-inferred magnetic field strength,
such that higher spin-inferred-B sources show little to no
spectral up-turn \citep{mgmh05,gmte06}.
%1048 an exception \cite{turkey,wang}

Another remarkable feature of magnetar spectra is that they are highly
rotational-phase dependent \citep{dkh08,dkh+08}.  This is diagnosed in two different ways:  by
strong energy-dependence of the pulse profile as shown for 
magnetar 1RXS J170849$-$400910 in Figure~\ref{fig:spectrum} (right),
or, equivalently, as variations in fitted spectral parameters with rotational phase.
The latter is detected generically.
%, at least in sources for which there
%is sufficient flux to allow a meaningful fit in many phase bins.  
The strong phase variation is expected for magnetospheric emission beamed along magnetic field lines
and has been used to deduce constraints on the geometry of the hard X-ray source
\citep{hbd14,vhk+14,aah+15,thy+15}.
This is discussed in detail in \S\ref{sec:andrei}. 

\begin{figure}
\begin{minipage}{2.9in}
\includegraphics[scale=0.3]{2259spectrum}
\end{minipage}
\hfill
\hspace{-0.3in}
\begin{minipage}{2.9in}
\includegraphics[scale=0.22]{phaseres}
\end{minipage}
\caption{
(Left) 
Broadband phase-averaged X-ray spectrum from combined {\it Swift}/XRT (green) and {\it NuSTAR} observations of 1E~2259+586
from \citet{vhk+14}.  The best-fit model of an absorbed blackbody plus two power laws is shown.  The spectral
turn up in this source near 15 keV is obvious.
(Right) 
Pulse profiles in different X-ray energy bands for 1RXS J170849-400910,
from \citet{dkh08}.}
\label{fig:spectrum}
\end{figure}



\noindent
{\it Transient Magnetars in quiescence}
show X-ray spectra that are consistent with being
pure absorbed blackbodies, with $kT \simeq 0.15-0.3$~keV (see OK14 for a compilation).
As these sources are generally
discovered in outburst, measuring quiescent spectra and temperatures requires waiting
months to years until the source returns to a quiescent state \citep[e.g.][]{ah16}.
Phase-resolved spectroscopy for transient magnetars in quiescence has yet to be done owing to the faintness
of the sources.  In these objects, the non-thermal magnetosperic processes responsible
for the power laws seen in persistent magnetars seem absent, in spite of the spin properties of
the two classes being similar, though sensitivity may play a role.  
Interestingly, quiescent transient magnetar spectra are similar to the
X-ray spectra of some high-B radio pulsars (see \S\ref{sec:highB}).
%notably those with sufficiently low spin-down luminosity
%that non-thermal
%emission is not a contaminant.  In the latter sources, the X-ray spectra
%appear to be purely thermal, with blackbody $kT \simeq 0.15-0.2$~keV.
%Interestingly, there is evidence that their temperatures are higher than
%in radio pulsars of similar age but lower field \citep{km05,zkm+11,ozv+13}
%suggesting possible internal heating due to the high field.  Also,
%\citet{akt+12} showed evidence for a correlation between quiescent X-ray
%luminosity and spin-inferred magnetic field strength in high-B radio
%pulsars and magnetars, with the two source classes lying on the same
%correlation curve.  These observations suggest that high-B radio pulsars
%could be quiescent magnetars, a prediction validated by magnetar-like
%outbursts from two rotation-powered pulsars (see \S\ref{sec:highB}).
%Conversely, they also suggest that transient magnetars might show
%conventional radio pulsations in quiescence, a prediction that has not
%thus far been verified, though additional sensitive searches during
%quiescence are needed, and radio beams are narrow so that the chance of
%detecting any one source is likely limited.


\subsection{X-ray Spectral Evolution in Outburst}
\label{sec:specevol}

The spectra of magnetars change dramatically at times of outburst,
generically hardening initially, then slowly softening as the flux
relaxes back to quiescence over typically months to years.  The flux
evolution in outbursts was discussed in \S\ref{sec:outbursts}, with an
example shown in Figure~\ref{fig:outburst} (left).
The hardening at outburst, for a spectrum parameterized by an absorbed
blackbody plus power law, can generally be described by an initial increase
in $kT$ by a factor of $\sim$2--3 (often, but not always, with a decrease in effective blackbody
radius by a factor of a few), together with a decrease in photon index
by a factor of $\sim$2.  These quantities then relax back to their
quiescent values on the same time scale as the flux relaxation.  
An example of the spectral evolution seen in one magnetar (1E 2259+586)
outburst is shown in Figure~\ref{fig:specfeature}.
The hardness/flux correlation seen in magnetar outbursts is thought to
be closely related to the correlation between hardness and spin-down rate
noted by \citet{mw01}, in that all these quantities
are related to the degree of magnetspheric twist, with larger twists corresponding
to higher luminosities, spin-down rates, and hardness.  

There is, however, considerable diversity in the spectral changes and evolution
post-outburst.  Just as relaxation light curves for different sources can
look very different, the degree of hardening
and the manifestation of that hardening (be it a greater increase in $kT$ or decrease
in photon index) varies from outburst to outburst \citep{re11}.  
In the first-discovered transient magnetar, XTE J1810$-$197, for example,
\citet{gh07} found that the spectrum was well described by two blackbodies each
of which had its luminosity decay on exponential time scales, albeit different ones (870 and 280 days),
behavior not reproduced in most other sources.
Even for the same source, outbursts can show a variety
of behaviors \citep{ier+10,nkd+11,khdu12}.
\citet{sk11} show that there does not appear to exist a universal law linking the degree
of flux increase over the quiescent level with the degree of flux
hardening.   On the other hand, \citet{bl16} showed a relationship
between X-ray luminosity and inferred blackbody emitting area during outburst relaxations
of 7 different magnetars, consistent with theoretical predictions based on j-bundle
untwisting (see \S\ref{sec:andrei}).

%\begin{figure}
%\includegraphics[scale=0.15]{2259}
%\caption{Spectral evolution of 1E 2259+586 through and following its 2002 outburst. 
%Top to bottom: unabsorbed flux (2–10 keV), blackbody temperature (kT), 
%photon index, blackbody radius, and ratio of power-law (2–10 keV) to bolometric blackbody flux. 
%Horizontal dashed lines denote the values of each parameter fortuitously measured 
%one week prior to the outburst.  From \citet{wkt+04}.}
%\label{fig:specevol}
%\end{figure}

\begin{figure}
%\begin{minipage}{1.2in}
%\hspace{-2.0in}
%\includegraphics[scale=0.13]{2259}
%\end{minipage}
%\hfill
%\hspace{0.45in}
%\begin{minipage}{3.0in}
\includegraphics[scale=0.37]{specfeature_grey}
%\end{minipage}
\caption{
Phase-resolved spectroscopy of SGR 0418+5729.
The spectral flux is shown in the energy versus phase plane for {\it XMM-Newton} EPIC data, with
100 phase bins and 100-eV energy channels.
The red line shows (for only one of the two displayed cycles) 
a simple proton cyclotron model. 
See \citet{tem+13} for details.}
\label{fig:specfeature}
\end{figure}

\subsubsection{Spectral Features}

\citet{tem+13} reported the presence of a feature -- an absorption line --
in the outburst X-ray spectrum of SGR 0418+5729, the source
with low spin-inferred magnetic field \citep{ret+10}.  The
energy of the line apparently varies strongly with pulse phase;
see Figure~\ref{fig:specfeature}.  The variation in energy is roughly a factor of 5
over just 10\% of the pulse phase.
Those authors interpret the line as a proton cyclotron feature; its energy implies a magnetic field
ranging from $2 \times 10^{14}$~G to $>10^{15}$~G.  If interpreted as an electron cyclotron line, however,
the implied field is 2000$\times$ lower.  If the proton cyclotron interpretation is correct,
this observation strongly supports the hypothesis that SGR 0418+5729 has a far stronger field than
is inferred from the dipolar component. 
A similar phase-dependent absorption line was
recently reported by \citet{rit+16} for the magnetar Swift J1822.3$-$1606.  This is particularly
interesting as this source has the second lowest spin-inferred B field of the known magnetars 
\citep{rie+12,skc14,rit+16}.  Why the two lowest-inferred B sources
should be the only ones with such phase-dependent features is unclear.  The emission may come
from a magnetic loop near the surface of the star, wherein the field
energy is appropriate for the line to be in the observed spectral window.
%In higher-inferred-field magnetars, perhaps the loops have
%much higher B, so that the lines are in the fainter higher-energy portion of the X-ray spectrum
%where the source is fainter.  In this case, more sensitive hard X-ray telescopes may be able
%to see these features in more magnetars.



\subsection{Low-Frequency Emission}
\label{sec:spectra_oir}

%The origin and nature of optical and infrared (IR) emission in magnetars is not well understood.
As discussed in \S\ref{sec:multitemporal}, six magnetars have
optical or IR emission detected.
One is bright enough to have had its optical/IR spectrum studied
in detail:  4U~0142+61 \citep{wck06}.  
%Its broadband optical/IR spectrum from
%\citet{wck06} is shown in Figure~\ref{fig:oir_spectrum}.  
The optical/IR 
emission is well described by power law of index 0.3 and is presumed to be magnetospheric,
in line with the detection of strong optical pulsations \citep{km02,dmh+05}.
However, the IR {\it Spitzer}-measured 4.5 and 8.0 $\mu$m emission deviates from this function
and can be well described by blackbody emission for a temperature of 920 K.  
\citet{wck06} interpreted this near-IR emission as arising from an X-ray heated dust disk that is a remnant
of material that fell back toward the newly born neutron star following the supernova.
%Here `passive' means that the disk does not provide the source's X-ray luminosity
%as in accreting neutron stars, but rather is heated by the magnetar's X-ray emission so that
%it shines in the infrared range.  They estimate a disk of mass $\sim$10 Earth masses.  
However, this interpretation is not unique; 
it may also be some form of non-thermal magnetospheric emission.
The detection of pulsations in the near-IR would be key as 
these are not expected at more than the few-percent level for a disk.  
%However, such an observation has yet to be done.  
Note that \citet{wbk+08} did deep {\it Spitzer} observations of magnetar 1E~1048.1$-$5937 that
appear to rule out any IR emission similar to that in 4U~0142+61, and hence challenge
the disk interpretation.

%\begin{figure}
%\includegraphics[height=2.5in]{oir_spectrum}
%\caption{ Optical/infrared spectral energy distribution
%of 4U 0142+61 from \citet{wck06}.
%The triangles indicate the observed optical/IR
%flux while the squares indicate the dereddened flux assuming
%AV = 3.5. The optical emission is assumed here to be magnetospheric,
%and is well described by a power law of index 0.3.  The 4.5 and
%8.0 $\mu$m emission, however, does not follow this power law, but
%is well described by a blackbody of temperature 920 K, as shown.
%}
%\label{fig:oir_spectrum}
%\end{figure}



%\subsubsection{Radio Emission}
%\label{sec:spectra_radio}

The spectrum of the pulsating radio emission seen at least transiently from
four magnetars is remarkably flat \citep{crp+07,crhr07,ljk+08,lbb+10}. 
This is true even over a very wide range of radio frequencies
\citep[e.g. 1.4--45 GHz for 1E~1547.0$-$5408][]{crj+08}.  
This flatness is in contrast to the spectra of rotation-powered pulsars which are typically
steep, with negative spectral indexes of $\sim -1.8$ \citep[e.g.][]{mkkw00a}.
%Indeed for an extended period after its 2003 outburst, XTE J1810$-$197 had
%greater flux above 20 GHz than any known radio pulsar \citep{crh+06}.
Magnetar SGR J1745$-$2900
has been detected at frequencies as high as 225 GHz, the highest
yet for any pulsar \citep{tek+15}.  On the other hand, for this same source,
\citet{ppe+15} reported a steep spectral index ($-$1.4) between 2 and 9 GHz.
Although overall approximately flat, magnetar radio spectra may not be well described by a single spectral index.
%That said, the actual functional form of the spectrum is not well known, as 
Measurement of the broadband radio spectra is 
challenging because of the high variability and the common presence of
terrestrial interference at relevant time scales.

%The relationship of the radio spectrum to that of the X-rays is
%not well understood.   While it is clear that radio emission is broadly correlated with the X-rays
%in that the former appears near X-ray outbursts, the fading time scales are not obviously the same,
%with the X-rays showing typically less short time scale variability during the post-outburst phase 
%but more secular, long-term decay.  For example, \citet{laks15}
%found that the 2–10-keV X-ray flux decayed steadily in 
%SGR J1745$-$2900 following the source's initial X-ray outburst, 
%while the average radio flux remained stable to within ∼20% during a 5.5-month-long stable state.
%However, even while the X-rays continued their slow fading, the radio emission susbsequently
%became more erratic.



