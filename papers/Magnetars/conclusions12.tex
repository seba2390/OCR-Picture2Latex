
The magnetar model has now been used to predict, naturally and uniquely, 
a wide variety of remarkable phenomena and behaviors in sources that 
once seemed highly `anomalous.'
The now seemless chain of phenomenology from otherwise conventional radio pulsars through
sources previously known for radically different behavior makes clear that these objects are
one continuous family, with activity correlated with spin-inferred
magnetic field strength.  Recent advances in
the physics of these objects, from the core through the crust and
to the outer magnetosphere, hold significant promise.
Below are issues we believe hold potential for important progress
in the field in the near future, as well as remaining unsolved problems that are
worthy of more thought.

\begin{issues}[FUTURE ISSUES]
\begin{enumerate}

\item Why the `transient' magnetars are orders-of-magnitude fainter and significantly softer in X-rays
than the persistent sources remains an important puzzle.  
Continued X-ray and multi-wavelength follow-up of newly discovered magnetars, found using
all-sky X-ray monitors sensitive to outbursts, will
flesh out spin-property distributions, better constrain the Galactic
population of transient magnetars, and their outburst rates.

\item Future X-ray polarimetric observations of magnetars will
test basic predictions of quantum electrodynamics, and
illuminate the geometry of the magnetic field for comparison with that
inferred from modeling of phase-resolved hard X-ray spectra.

\item Monitoring of high-magnetic-field radio pulsars, particularly their behavior near
glitch epochs and at rare times of magnetar-type activity, will help clarify the 
onset of instabilities with increasing spin-inferred magnetic field.

\item Detailed studies of magnetar wind nebulae and associations with TeV
emission may be useful for calorimetric
determinations of past magnetar activity.

\item Numerical simulations 
of ambipolar diffusion in the core and advanced magneto-thermoplastic models of crust evolution may shed light on how magnetars become active, and permit quantitative predictions for their transient and persistent activity.

\item First-principle simulations of twisted magnetospheres have now become possible using the plasma particle-in-cell
method. This technique has recently been successfully applied to ordinary radio pulsars and can be applied to magnetars.

\item Three-dimensional simulations of relativistic reconnection in the magnetosphere can give more realistic models of bursts and giant flares.

\item The results of recent modeling of internal heating and post-flare QPOs provide strong constraints on magnetar interior and can be used to infer basic properties such as superfluidity in the core and the strength of the internal magnetic field.

\item 
Modelling of the remarkable glitches and anti-glitches observed in magnetars and their 
radiative signatures may become possible in the near future as part of detailed 
simulations of magnetar interiors that include superfluid neutrons.


\end{enumerate}
\end{issues}

