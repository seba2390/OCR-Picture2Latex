
\subsection{Basic Properties}
\label{sec:basic}

Recently, the first magnetar catalog was published \citep[][hereafter OK14]{ok14}, and includes a detailed compendium of the properties
of the known magnetars.   Here we summarize these properties and refer the reader for details to that
paper or, for the most recent updates, to the online catalog\footnote{www.physics.mcgill.ca/$\sim$pulsar/magnetar/main.html}.

The vast majority of known magnetars were discovered via their short X-ray bursts, thanks to sensitive all-sky
monitors like the Burst Alert Telescope (BAT) aboard {\it Swift} and the Gamma-ray Burst Monitor (GBM) aboard {\it Fermi}.  These instruments
were designed to study gamma-ray bursts so are fine-tuned to finding brief, bright bursts over the full sky.  
Thus there is strong
bias in the known magnetar population toward sources most likely to burst.  
That nearly all known magnetars share common
spin properties and high spin-inferred surface dipolar magnetic fields is a powerful statement regarding which objects
in the neutron-star population are burst-prone.

In short, apart from the hallmark X-ray activity that defines the class (see \S\ref{sec:bursts}),
magnetars are observed to produce X-ray pulsations in the period range 2--12 s (ignoring two
faster-rotating sources that only briefly exhibited magnetar-like properties; 
see Table~\ref{ta:srcs} and \S\ref{sec:highB}).  Magnetars
are, without exception, spinning down, with spin-down rates that imply spin-down time scales ($\sim P/\dot{P}$)
of a few thousand years, suggesting great youth.  The spin-down luminosity $\dot{E} \equiv 4 \pi^2 I \dot{P}/P^3$, where
$I \simeq 10^{45}$~g~cm$^2$ is the stellar moment of inertia, is usually far smaller than the persistent quiescent
X-ray luminosity of the sources (see Table~\ref{ta:srcs}).
Moreover, these spin-down rates imply, for 20/23 of the sources
for which it has been measured, $B>5\times 10^{13}$~G, with the vast majority over $10^{14}$~G.  
Whereas most radio pulsars are thought to be born with periods
of at most a few hundred milliseconds, that the shortest known {\it bona fide} magnetar
has a relatively long 2-s period in spite of a young age
is surely a result of rapid magnetic braking. 
%inevitable in so highly magnetized a neutron star.
%Faster rotators may exist, but they slow down so rapidly (in under $\sim$1000 years), they are rare to observe.
The long period cutoff of 12~s has been more of a puzzle, which is related to the life-time
of magnetar activity \citep[e.g.][]{cgp00,vrp+13}.
The small observed
ranges in $P$ and $B$ are contrasted by a far larger range in quiescent X-ray luminosity, spanning 
$\sim 10^{30}$~erg~s$^{-1}$ up to $2 \times 10^{35}$~erg~s$^{-1}$ in the 2-10-keV band.  
In fact, the distribution of quiescent luminosities appears somewhat bi-modal, with the
brighter group the `persistent' magnetars and the fainter ones the `transient' magnetars.  The latter 
show greater dynamic range in their outbursts.
This large range in luminosity is presently an interesting puzzle, as is
the small range in period (see \S\ref{sec:pulsations}).
%The X-ray pulse profiles in all cases are broad, with duty cycles of well over 50\%, and significant energy dependence of the pulse morphology.
%Pulse fractions range from a few percent to over 70\%.
In the soft X-ray band, magnetar spectra are fairly well described by a blackbody and in some
cases an additional power-law component, while in the hard X-ray band significant spectral hardening occurs such that
in some cases the energy spectrum {\it rises}, at least as far as has been detected (typically until $\sim$60~keV).   
%Magnetar spectra are described in more detail in \S\ref{sec:spectral} and discussed in \S\ref{sec:andrei}.
Magnetar emission has been seen in some cases at radio, infrared and optical wavelengths.
%The multi-wavelength properties of magnetars are discussed further
%in \S\ref{sec:multitemporal}, \S\ref{sec:spectra_oir} and \S\ref{sec:spectra_radio}.


One source is notable and not included in Table~\ref{ta:srcs} --
the central source of the supernova remnant RCW~103, 1E~161348$-$5055.  It
shows a strange 6.67-hr X-ray periodicity with a variable pulse profile, as well as repeated large X-ray outbursts
\citep{dcm+06b}.  \citet{aeb+16} and \citet{rbe+16} reported on the discovery of a bright magnetar-like burst from the
source, coupled with another large X-ray flux outburst.  The source thus bears all the hallmarks of a magnetar,
except for the bizarrely long spin period, which cannot be from simple magnetic braking.  
%which is 2000 times longer than the previous record holder!  
%Simple magnetic braking cannot accomplish this for any reasonable initial set of parameters given the young age of the supernova remnant.
The long period may be explainable with a fall-back disk 
that slows down the initially faster-rotating neutron star 
\citep{li07,ha17}.  
%but leaves open the question of why such a mechanism occured only in this source
%and not in the any of the other known magnetars.  
%In this magnetar picture, the source should at least today
%be spinning down due to conventional magnetic braking.  However, such slow down will be very difficult to measure:
%for a $10^{15}$-G magnetar spinning once every 6.67 hr, the expected $\dot{P}$ is predicted to result in a


% Example of a Table
\begin{table}
\tabcolsep3.5pt
\caption{Known Magnetars and Magnetar Candidates$^{\rm a}$}
\label{tab1}
\begin{center}
\begin{tabular}{@{}l|c|c|c|c|c|c|c@{}}
\hline
Name$^{\rm b}$ & $P$ & B$^{\rm c}$ & Age$^{\rm d}$ & $\dot{E}$$^{\rm e}$ & D$^{\rm f}$ & $L_X$$^{\rm g}$ & Band$^{\rm h}$ \\
     & (s) & ($10^{14}$ G) & (kyr) & $10^{33}$ erg~s$^{-1}$ & (kpc) & $10^{33}$ erg~s$^{-1}$ &  \\
\hline
CXOU J010043.1−-721134 & 8.02 & 3.9 & 6.8 & 1.4 & 62.4 & 65 & ... \\
4U 0142+61 & 8.69 & 1.3 & 68 & 0.12&  3.6 & 105 &  OIR/H \\
SGR 0418+5729 & 9.08 & 0.06 & 36000 & 0.00021 & $\sim$2 & 0.00096 &  ... \\
SGR 0501+4516 & 5.76 & 1.9 & 15 & 1.2& $\sim$2 & 0.81 & OIR/H  \\
{\bf SGR 0526$-$66} & 8.05 & 5.6 & 3.4 & 2.9 & 53.6 & 189 &  ... \\
1E 1048.1−-5937 & 6.46 & 3.9 & 4.5 & 3.3 & 9.0 & 49 &  OIR \\
(PSR J1119$-$6127) & 0.41 & 4.1 & 1.6 & 2300 & 8.4 & 0.2 & R/H \\
1E 1547.0−-5408 & 2.07 & 3.2 & 0.69 & 210 & 4.5 & 1.3 & O?/R/H \\
PSR J1622--4950 & 4.33 & 2.7 & 4.0 & 8.3 & $\sim$9 & 0.4 & R \\
SGR 1627$-$41 & 2.59 & 2.2 & 2.2 & 43 & 11 & 3.6 & ...\\
CXOU J164710.2--455216 & 10.6 & $<$0.66 & $>$420& $<$0.013 & 3.9 & 0.45 &  ...  \\
1RXS J170849.0--400910 & 11.01 & 4.7 & 9.0 & 0.58 & 3.8 & 42 & O?/H \\
CXOU J171405.7--381031 & 3.82 & 5.0 & 0.95 & 45 & $\sim$13 & 56 & ...\\
SGR J1745--2900 & 3.76 & 2.3 & 4.3 & 10& 8.3 & $<$0.11 &  R/H \\
{\bf SGR 1806$-$20} & 7.55 & 20 & 0.24 &45 & 8.7 & 163 &  OIR/H \\
XTE J1810--197 & 5.54 & 2.1 & 11 & 1.8& 3.5 & 0.043 & OIR/R \\
Swift J1822.3--1606 & 8.44 & 0.14 & 6300 & 0.0014 & 1.6 & $>$0.0004 &  ... \\
SGR 1833--0832 & 7.56 & 1.6 & 34 & 0.32 & ... & ... & ... \\
Swift J1834.9--0846 & 2.48 & 1.4 & 4.9 & 21 & 4.2 & $<$0.0084 & ... \\
1E 1841--045 & 11.79 & 7.0 & 4.6 & 0.99& 8.5 & 184 & ...  \\
(PSR J1846$-$0258) & 0.327 & 0.49 & 0.73 & 8100 & 6.0 & 19 & ... \\
3XMM J185246.6+003317 & 11.56 & $<0.41$ & $>1300$ & $<0.0036$& $\sim$7 & $<0.006$ & ... \\
{\bf SGR 1900+14} & 5.20 & 7.0 & 0.9 & 26& 12.5 & 90 & H \\
SGR 1935+2154 & 3.24 & 2.2 & 3.6 & 17 & ... & ... & ...\\
1E 2259+586 & 6.98 & 0.59 & 230 & 0.056 & 3.2 & 17 &  OIR/H\\
{\it SGR 0755$-$2933} & ... & ... & ...& ... & ... & ... & ... \\
{\it SGR 1801$-$23}   & ... & ... & ... & ... & ... & ... & ...\\
{\it SGR 1808$-$20}    & ... & ... & ... & ... & ... & ... & ...\\
{\it AX J1818.8$-$1559}     & ... & ... & ... & ... & ... & ... & ...\\
{\it AX J1845.0$-$0258} & 6.97 & ... & ... & ...& ... & 2.9 &  ... \\
{\it SGR 2013+34 } & ... & ... & ... & ... & ... & ... & ...\\
\hline
\end{tabular}
\end{center}
\begin{tabnote}
$^{\rm a)}$All tabulated values from OK14;
$^{\rm b)}$Sources in {\bf bold} have had giant flares. Sources in {\it italics} are candidates only. Sources in parentheses are
normally rotation-powered pulsars;
$^{\rm c)}$Spin-inferred magnetic field strength; 
$^{\rm d)}$Characteristic age from $P/2\dot{P}$; 
$^{\rm e)}$Spin-down luminosity;
$^{\rm f)}$Distance; 
$^{\rm g)}$Unabsorbed quiescent luminosity in the 2--10-keV band for the distance provided; 
$^{\rm h)}$OIR=Optical/Infrared counterpart. R=Radio counterpart. H=Hard X-rays detected.
\end{tabnote}
\label{ta:srcs}
\end{table}

% Name, P, B, tau, dist, Lx, Association, OIR?, radio?


\subsection{Spatial Distribution}
\label{sec:spatial}

One of the best-determined aspects of magnetars'  spatial distribution in the Galaxy is their strict confinement
to the Galactic Plane.  As shown by OK14, the scale height of known magnetars is only 20--30~pc, in spite
of the vast majority of known objects having been discovered through their X-ray bursts by 
spatially unbiased all-sky X-ray monitors.  
%This scale height reflects uncertainties in distance (see below) and
%accounts for small biases in discovery.  
This scale height is far smaller than that of the radio pulsar
population, clearly indicating great youth in magnetars.  A
200 km~s$^{-1}$ spatial velocity will have moved an object by $\sim$20~pc in $10^5$~yr.  
Direct proper motion
measurements for magnetars have found a
weighted average 200~km~s$^{-1}$ with standard deviation 100~km~s$^{-1}$
\citep{tck13}, somewhat lower than that of the radio pulsar population \citep[e.g.][]{acc02,bfg+03,fk06}.
Thus magnetars typically cannot be much older than $10^5$~yr, and are generally much younger.  
%Clearly models involving white dwarf progenitors are strongly at odds with this basic property.

%Conventional radio pulsars generally have relatively large spatial velocities \citep[e.g.]{cc98,acc02,bfg+03,fk06}
%likely owing to asymmetric natal kicks.
%, but 
%early papers on the subject suggested magnetars should have even higher space velocities.  
%Motivated by the significant
%offset of  SGR 0526$-$66 from the center of its putatively associated
%supernova remnant N49 \citep{rkl94} -- though this same offset has also been used to argue against
%an association; \citep{gsgv01} -- \citet{dt92a} proposed multiple mechanisms for the production
%of kick velocities as high as 1000~km~s$^{-1}$ in magnetars.   
%Measurements of magnetar proper motions
%combined with best available distance estimates yield the following 6 tangential spatial velocities for magnetars:
%212$\pm$35~km~s$^{-1}$ for 1E 1810$-$197 \citep{hcb+07}, 
%280$\pm$120~km~s$^{-1}$ for 1E 1547.0$-$5408 \citep{dcrh12}, 
%130$\pm$30~km~s$^{-1}$ for SGR 1900+14 and 350$\pm$100~km~s$^{-1}$ for SGR 1806$-$20
%\citep{tck12},
%157$\pm$17~km~s$^{-1}$ for 1E~2259+586 and 102$\pm$26~km~s$^{-1}$ for 4U~0142+61
%\citep{tck13}.  The weighted average is 


The inferred spatial locations within the Galaxy (see Fig.~\ref{fig:galdist}) strongly suggest we are biased
against finding more distant magnetars. 
%presumably because bursts (the phenomenon most likely
%to result in detection given existing technology) from more distant objects go undetected.  
Nevertheless, some inferred
distances that are comparable to that of the Galactic Center indicate that
we have been sensitive to a large fraction of the volume of the Milky Way; future all-sky monitors
with only modest increases in sensitivity could fully flesh-out the active magnetar population in the Galaxy.

\begin{figure}
\begin{minipage}{2.9in}
\includegraphics[scale=0.45]{galdist}
\end{minipage}
%\hfill
\hspace{-1.55in}
\begin{minipage}{2.9in}
\includegraphics[scale=0.4]{scaleheight}
\end{minipage}
\caption{Left:  Top-down view of the Galaxy, with the Galactic Center at  (0,0) and the Sun marked by a cyan arrow. 
The grayscale shows the distribution of free electrons \citep{cl01}. Known magnetars are shown as red circles with 
distance uncertainties indicated, known X-ray Isolated Neutron Stars (XINSs) are shown by yellow circles, and 
all other known pulsars are blue dots.  Right:  Top panel: Cumulative distribution function of the height z above the Galactic plane for the 
19 known magnetars in the Milky Way. Data are fit to an exponential model (solid line) and a self-gravitating, isothermal disc 
model (dashed line).  Bottom panel: Histogram of the distribution in z of the known Galactic magnetars. Lines are as above.
Note the offset of the peak from zero; this is the offset of the Sun from the Galactic Plane.  From OK14.}
\label{fig:galdist}
\end{figure}

\subsection{Associations with Supernova Remnants and Wind Nebulae}
\label{sec:snrs}

Of the 23 confirmed magnetars, 8 are reliably associated with supernova remnant shells and an additional
2 have possible associations.  The large number of remnant associations is fully consistent with the great youth implied
both by magnetar spin-down ages, as well as by their proximity to the Galactic Plane.  
The associated remnants lack unusual properties when compared
with shell remnants that harbor neutron stars with lower magnetic fields \citep{vk06,mrtp14}.   
This appears to be in conflict with the proposal of \citet{dt92a} that
magnetars form from neutron stars rotating with period $\sim$1~ms at birth, which assists a fast dynamo.
%Indeed detailed simulations have shown that a millisecond magnetar could power observed hyper-luminous supernovae and gamma-ray bursts
%\citep{mgt+11,bmtq12}.
The difficulty with this picture is that a neutron star with magnetic field $> 10^{14}$~G spinning at 1 ms 
quickly loses most of its rotational energy, releasing energy
in excess of $10^{52}$ erg, greater than the supernova explosion energy itself.  It is therefore likely to be associated with
either anomalously large shell remnants, or else no remnant at all, if it expanded sufficiently rapidly to dissipate on a few hundred year time scale.
%As magnetar remnants show no difference from regular remnants, the dynamo origin of magnetar fields put forth by \citet{dt92a} has been 
%called into question, and 
The normality of magnetar supernova remnants challenged the dynamo model and led to discussion of strong fossil fields from the progenitor star \citep{fw06},
however the latter is not without difficulties \citep[see, e.g.][]{spr08}.


%\subsubsection{Evidence for `Magnetar Wind Nebulae'}
%\label{sec:mwn}

Extended, nebular emission near magnetars, `magnetar
wind nebulae (MWN),' may exist, in analogy with pulsar wind nebulae (PWNe).
PWNe are extended synchrotron
nebulae surrounding some radio pulsars, a result of the interaction of relativistic,
magnetized pulsar particle winds interacting with their environments, and
ultimately powered by the pulsar's rotation \citep[see][for
a review]{gs06}.  Observations
of a MWN could, in principle, provide important information on the
composition and energetics of continuous particle outflows from magnetars.
Clear evidence for temporary magnetar outflows has been seen in the form of transient extended radio emission following two giant flares \citep{fkb99,gkg+05,gle+05}.
%\citet{fkb99} observed a fading, albeit unresolved, radio source following the 1998 giant flare from SGR 1900+14 which the authors argued was due
%to a synchrotron nebula produced by particle outflow in the giant flare.  \citep{gkg+05} and \citep{gle+05}
%reported on a relativistically expanding linearly polarized radio nebula observed using the Very Large Array and the Australia Telescope Compact
%Array after the 2004 giant flare from SGR 1806$-$20.  Those authors
%argued the nebula resulted from shocked ambient medium following the release of $\sim 10^{24.5}$~g of material having initial kinetic
%energy $\sim 10^{44.5}$~erg.  \citet{tgg+05} detected expansion directly in Very Large Array data, inferring an expansion velocity of 0.26c followed by deceleration
%simultaneous with a radio re-brightening, which together suggest the onset of a Sedov-Taylor phase.
However interesting, these do not constitute MWNe since the latter by definition are long-lived and result
from continuous particle outflow even when the magnetar is in quiescence.

There are multiple reports of stable MWNe in the literature \citep[e.g.][]{rmg+09,crb+13},
%The first possibility was surrounding not a magnetar but a Rotating Radio Transient (RRAT), J1819$-$1458.   This object is a sporadic
%radio emitter whose potential relevance to magnetars is its long spin period (4.3 s), its relative youth (characteristic age 117 kyr), and
%its high spin-inferred magnetic field, $5 \times 10^{13}$~G \citep{mll+06}.
%\citet{rmg+09} and \citet{crb+13} reported possibly related extended X-ray emission around J1819$-$1458, but noted a large ratio of X-ray to spin-down
%luminosity for the nebula ($\eta \equiv Lx/\dot{E} \simeq 0.2$), much higher than seen for PWNe, and suggested it may be powered somehow
%by magnetic energy.  They could not, however, rule out the extended emission being a dust-scattering halo.
%Indeed, \citet{vb09} reported an X-ray MWN around the magnetar 1E~1547.0$-$5408, but \citet{okn+11} showed the flux of the extended emission
%was strongly correlated with the magnetars' flux, concluding it must be a dust scattering halo.
however extended emission can also be due to dust scattering along the line of sight \citep[e.g.][]{okn+11}.
Currently the most compelling case of a MWN is an asymmetrical X-ray structure around Swift J1834.9$-$0846 \citep{ykk+16}.
One thing is certain:  
%even if one or more of the above putative MWNe were real, 
the phenomenon is not generic to the source class.  
Deep X-ray imaging observations of many different
magnetars have revealed no evidence for any MWN down to constraining (though, unfortunately, not generally well quantified) 
limits \citep[e.g][]{ghbb04,akac13}.  
%Why some magnetar have nebulae and others do not is unclear and will require more MWN detections to be answered.
However, \citet{hg10} report a potential association between a magnetar, CXOU J171405.7$-$381031, in 
the supernova remnant CTB 37B, and 
TeV emission, which they speculate may be a relic MWN. 



%Mention Wind braking  Tong et al. 2013?


%\citet{ykk+12} reported on two {\it XMM-Newton} observations of magnetar Swift J1834.9$-$0846, one in 2005, well prior to this source's 
%2011 X-ray outburst, and one month afterward.
%They find both a symmetric X-ray ring surrounding the point source, argued to be a dust-scattering halo, as well as asymmetrical X-ray emission 
%present at approximately the same level in both observations.
%The efficiency $\eta \simeq 0.7$ for the asymmetrical emission's X-ray luminosity, again higher than for PWNe, suggesting an alternative energy 
%source likely related to the high magnetic field inferred
%for the central source from its spin-down, $1.4 \times 10^{14}$~G.  \citet{etr+13} suggested the asymmetrical feature
%is due to dust scattering by the same cloud
%producing the symmetrical structure, because of some variability seen between epochs.  However very recently, \citet{ykk+16} 
%reported on two more recent {\it XMM} observations long post-outburst, showing the putative wind nebula remains essentially
%unchanged.  This is strong support for the reality of the nebula.


\subsection{Relationship to High-Magnetic-Field Radio Pulsars}
\label{sec:highB}

%{\bf LEAVE THIS SECTION FOR CLOSER TO THE END OF THE PAPER?}
If high magnetic fields in neutron stars are responsible for the dramatic X-ray and soft-gamma-ray activity in magnetars, and given
that some magnetar behavior has been seen in apparently low-B sources \citep[such as SGR 0418+5729;][]{ret+10}, then it stands
to reason that high-B radio pulsars may occasionally exhibit magnetar-like activity \citep{km05,nk11}.  
This possibility was also hinted at by higher blackbody temperatures in high-B radio pulsars compared
with lower-B sources of the same age \citep[e.g.][]{zkm+11,ozv+13}.
The idea of high-B radio pulsars as quiescent magnetars has proven
to be correct.

The first example came from the 
young (age $<$1 kyr), high-B ($5 \times 10^{13}$~G) but curiously
radio undetected \citep{aklm08} rotation-powered pulsar 
PSR J1846$-$0258 in the supernova remnant Kes 75 \citep{gvbt00}.   In long-term monitoring designed to 
measure the source's braking index, \citet{ggg+08} detected a sudden X-ray outburst lasting 
$\sim$6~wks and
of total energy $\sim 3 \times 10^{41}$~erg in the 2--10-keV band \citep[see also][]{ks08b}.  The outburst also included
several short magnetar-like bursts and a large glitch with unusual recovery \citep{kh09,lkg10}.
%This was the first example of a temporary metamorphosis of an otherwise \citep[though curiously radio-quiet][]{aklm08}
%ordinary pulsar into a magnetar-like object.   
Post-outburst, the source has returned to its quiescent, apparently rotation-powered
state, albeit with enhanced timing noise and a significant change in braking index, from 
2.65$\pm$0.01 pre-outburst to 2.19$\pm$0.03 post-outburst \citep{lnk+11,akb+15}.  
%So large a change in braking index
%is unprecedented for a rotation-powered pulsar and poses a challenge to models of pulsar spin-down.
A second such metamorphosis was seen very recently in PSR J1119$-$6127, a {\it bona fide} radio pulsar, also
very young (age $<$ 2 kyr) and apparently high-B ($4 \times 10^{13}$~G).  In this case the outburst was
heralded by bright, magnetar-like X-ray bursts \citep{ykr16,klm+16,glk+16}.  Follow-up X-ray observations \citep{akts16}
showed an increase in X-ray flux of nearly a factor of 200 with a dramatic spectral hardening.
%, similar to what is observed in magnetar outbursts (see \S\ref{sec:outbursts}).  
This outburst was also accompanied
by a large glitch \citep{akts16} and, remarkably, by a temporary cessation of radio emission \citep{bpk+16}.
%It seems likely a change in braking index will be seen post-outburst.  
%Interestingly, prior to this tell-tale event,
%PSR J1119$-$6127 had shown unusual radio variability and glitch recoveries \citep{wje11,awe+15} which
%were distinctly magnetar-like.

The similarity of the PSR J1846$-$0258 and PSR J1119$-$6127 events both to each other and to magnetar
outbursts, together with the fact that they are both relatively rare high-B rotation-powered pulsars normally,
confirms the close relationship between radio pulsars and magnetars, and
the correlation between high spin-inferred magnetic field and magnetar-like activity.   
%That magnetar-type emission
%comes nearly exclusively from high-spin-inferred magnetic field sources is clearly no coincidence.

The discovery of radio pulsations from magnetars \citep[see \S\ref{sec:multitemporal}][]{crh+06,crhr07} also 
provides an observational link between radio
pulsars and magnetars.  However, the radio properties of magnetars
(see \S\ref{sec:multitemporal} and \S\ref{sec:spectra_oir}) are somewhat different from
those of conventional radio pulsars.

%For completeness, we mention yet another link between radio pulsars and magnetars, observed in deep X-ray observations of thermal emission
%from high-B radio pulsars \citep[e.g.][]{pkc00,km05,zkm+11,ozv+13}.   Comparing blackbody temperatures of radio pulsars
%of the same characteristic age but having different spin-inferred magnetic field strengths reveals higher-B sources generally
%have higher blackbody temperatures and luminosities.  This suggests some active interior heating due to magnetic field
%decay even in sources with $\sim 10^{13}$~G sources.  It has been suggested that such sources could be descendants
%of higher-B pulsars, even {\it bona fide} magnetars whose radiative properties are well described by magnetothermal evolution
%\citep{pp11a,pp11b}.  This picture could further help explain the properties of yet another class of object, the so-called
%`Isolated Neutron Stars' \citep[see][for reviews]{krh06,tur09,kas10,ppvr14}.  Magnetothermal evolution and its relevance to both
%magnetars and high-B radio pulsars is discussed further in \S\ref{sec:andrei}.

%\subsection{Evidence for Binary Magnetars}
%\label{sec:binary}
%
%REMOVE ENTIRE SECTION?
%Finally, we briefly review evidence for the existence of magnetars in binary systems.
%The binary fraction of massive stars -- magnetar progenitors -- is $\sim$50-70\% \citep{dk13}, suggesting
%some magnetars -- like some radio pulsars -- could be found in binary systems. 
%From observations of the radio pulsar population, however, it is known that the
%binary fraction of these objects is very small -- $<$10\%.  This is
%thought to be due largely to the high kick velocities imparted at the time of the supernova explosion,
%which disrupt progenitor binaries.  Given the number of known magnetars, therefore, if the
%binary fraction of magnetars is comparable to that of radio pulsars, the probability of us having
%detected a binary magnetar is very small, all else being equal.  However, as discussed above
%(\S\ref{sec:spatial}), magnetar space velocities show evidence of being lower on average
%than those of radio pulsars, suggesting a smaller average supernova-induced kick velocity (although higher-mass progenitor
%companions could have a similar effect), and hence potentially a higher binary fraction.  Moreover, magnetar creation
%could be enhanced in some binaries due to pre-supernova angular momentum transfer that could enable
%magnetar-strength field creation \citep{pp06}.  Therefore, it is interesting 
%to consider the evidence for magnetars in binary systems.
%
%Thus far, all suggestions of binary magnetars have been for accreting systems.  Recent reviews
%are presented by \citet{pskh14} and \citet{pop15}.  One example is the unusually young Be/X-ray binary SXP 1062 in the Small Magellanic Cloud.
%The source is unusual for its long spin period (1062 s) and its likely association
%with a supernova remnant which provides an independent age estimate ($\sim 10^4$ yr). 
%\citet{pt12} suggest that the long period argues in favor of an initially highly magnetic neutron star which experienced field decay;
%this enables accretion to commence much earlier in the life of the neutron star than is usual.   Similarly, 4U 2206+54 is a very slowly rotating
%accreting pulsar ($P = 5560$~s) in a high-mass X-ray binary containing a main-sequence O star. 
%The spin properties and relative youth again suggest a high ($>10^{14}$~G) field \citep{rtn+09},
%and a measurement of spin-down implying such a field supports this argument \citep{fiwp10}.   Other similar (long spin-period, accreting)
%though less certain (less evidence of great youth) systems have also been argued to be progeny of accreting magnetars 
%\citep[2S 0114+650][]{lv99},  \citep[GX 301$-$2][]{dss+10},  \citep[Swift J045106.8$-$694803][]{kbb+13}.  See
%\cite5{ly06} and \citet{bfs08} for a more general discussion of the possibility of magnetars in certain
%high-mass X-ray binaries.
%
%However the evidence for either a presently high magnetic field or a previously high one in any of the above
%cases is far from certain.   \citet{pskh14}
%find standard strength field scenarios that can reproduce the system's current properties by considering recent models for
%subsonic quasi-spherical accretion onto slowly rotating neutron stars \citep{spkh12}.
%
%A couple of other suggestions of magnetars in binary systems are worthy of note.
%The famous gamma-ray binary LSI+61 303 has been suggested to harbor
%a magnetar because of a single magnetar-like burst observed from its direction \citep{tre+12}, however the
%association of the burst with the binary is not certain.
%Additionally, the ultra-luminous X-ray source M82 X-2 has shown 1.37-s pulsations modulated by a
%2.5-day binary orbit \citep{bhw+14}.  One possibility to explain the source's hyper-Eddington luminosity
%is via a magnetar-strength field, as the latter has been shown to significantly increase the Eddington luminosity due
%to suppression of the electron scattering cross section of one polarization \citep{pac92,td95}.    Indeed several
%authors have proposed magnetar-strength fields in this source \citep{eac+15,ton15} although a high field is
%not required \citet{bhw+14}.

