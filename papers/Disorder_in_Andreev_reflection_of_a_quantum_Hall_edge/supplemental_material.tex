\documentclass[pra,aps,a4paper,twocolumn,superscriptaddress,longbibliography]{revtex4-2}

\pdfoutput=1

%%%%%%%%%%%%%%%%%%%%%%%%%%%%%%%%%%%%%%%%%%%%%%%%%%%%%%%%%%%%%%%%%%%%%%%%%%%%%%%%%%%%%%%%%%%%%%%%%%%%%%%%%%%%%%%%%%%%%%%%%%%%
\usepackage{amssymb}
\usepackage{amsmath}
\usepackage{amsfonts}
\usepackage{graphicx}
\usepackage{bm}
\usepackage{color}
\usepackage{multirow}
\usepackage{natbib}
\usepackage[normalem]{ulem}
\usepackage{relsize}
\usepackage{xspace}

\newcommand{\rngl}{\rangle\hspace{-0.05cm}\rangle}
\newcommand{\lngl}{\langle\hspace{-0.05cm}\langle}
\newcommand{\Rngl}{\big\rangle\hspace{-0.085cm}\big\rangle}
\newcommand{\Lngl}{\big\langle\hspace{-0.085cm}\big\langle}
\renewcommand{\thesection}{S.\Roman{section}}
\renewcommand{\theequation}{S\arabic{equation}}


% Notations
\newcommand{\difz}{{\cal D}_{\rm D}}
\newcommand{\coop}{{\cal D}_{\rm C}}
\newcommand{\tdeg}{{\rm 2DEG}}
\newcommand{\SC}{{\rm SC}}
\newcommand{\T}{{\rm T}}
\newcommand{\prox}{{\rm prox}}
\newcommand{\M}{{\rm M}}
\newcommand{\QH}{{\rm QH}}
\newcommand{\lprox}{L}
\newcommand{\kf}{k_\mu}
\newcommand{\pf}{p_F}
\newcommand{\lA}{l_{\rm A}}
\newcommand{\dv}{d}
\newcommand{\Cjump}{{\cal C}_{\rm jump}}
\newcommand{\Ah}{a_{\rm h}}
\newcommand{\Ae}{a_{\rm e}}
\newcommand{\he}{A_{\rm h}}
\newcommand{\ee}{A_{\rm e}}
\newcommand{\psie}{\eta}
\newcommand{\edg}{{\rm edge}}
\newcommand{\xic}{\xi}
\newcommand{\GR}{{\cal G}^{\rm R}_{\rm N}}
\newcommand{\GA}{{\cal G}^{\rm A}_{\rm N}}

\usepackage{hyperref}

\begin{document}

\title{Supplemental Material for ``Disorder in Andreev reflection of a quantum Hall edge''}

\author{Vladislav D.~Kurilovich}
\affiliation{Department of Physics, Yale University, New Haven, CT 06520, USA}
\author{Zachary M.~Raines}
\affiliation{Department of Physics, Yale University, New Haven, CT 06520, USA}
\author{Leonid I.~Glazman}
\affiliation{Department of Physics, Yale University, New Haven, CT 06520, USA}

\maketitle

\onecolumngrid
\vspace*{-5mm}

\section{Derivation of $1/\lA$, Eq.~(8)\label{sec:lA}}

In this section, we present details of the derivation of Eq.~(8).
We start with the obtained in the main text expression for the Andreev amplitude, see Eq.~(7). For convenience, we reproduce it here:
\begin{align}\label{eq:amplitudes}
    \he &= -\frac{(\partial_y \Phi)^2 t^2}{v}\int_0^\lprox dx_1 dx_2 e^{i\kf (x_1 + x_2)}\partial^2_{y_1 y_2}{\cal G}_{\rm he} (x_1, x_2),
\end{align}
(we remind that ${\cal G}_{\rm he}(x_1, x_2) \equiv {\cal G}_{\rm he}(\bm r_1, \bm r_2 | E = 0)|_{y_{1,2}, z_{1,2} = 0}$ is the anomalous Green's function of the superconductor).
For calculations, it is convenient to choose a gauge in which the vector potential vanishes at the interface between the superconductor and the 2DEG.
In this gauge, the wave vector $k_\mu$ in Eq.~\eqref{eq:amplitudes} is related to the distance $y_{\rm c}$ between the cyclotron orbit center and the interface: $k_\mu = y_{\rm c} / l_B^2$, where $l_B = \sqrt{c / e B}$ is the magnetic length (throughout the supplement, we work in units with $\hbar = 1$). At $\nu = 2$, we can thus estimate 
\begin{equation}
    k_\mu \lesssim 1 / l_B.
\end{equation}

For simplicity, we first consider a type I superconductor. In this case, the vector potential vanishes not only at the interface but everywhere in the superconductor. We describe the superconductor with the standard BCS Hamiltonian:
\begin{equation}\label{eq:H_SC}
    H_{\rm SC} = \sum_\sigma\int d^3r\,\chi^\dagger_\sigma({\bm r})\Big[-\frac{\partial_{\bm r}^2}{2m} - \mu + U({\bm r})\Big] \chi_\sigma({\bm r}) + \int d^3 r\,\Delta \big(\chi_\uparrow^\dagger ({\bm r})\chi_\downarrow^\dagger({\bm r}) + \chi_\downarrow ({\bm r})\chi_\uparrow ({\bm r})\big).
\end{equation}
Here $\chi_\sigma(\bm r)$ is an annihilation operator for an electron with spin $\sigma$, $m$ is the effective mass, $\mu$ is the chemical potential, and $\Delta$ is the superconducting order parameter. 
$U({\bm r})$ is the disorder potential, which we assume to be a Gaussian random variable with a short-ranged correlation function,
\begin{equation}\label{eq:disorder}
    \langle U({\bm r}) U({\bm r}^\prime)\rangle = \frac{1}{2\pi \nu_{\rm M} \tau_{\rm mfp}} \delta({\bm r} - {\bm r}^\prime).
\end{equation}
We parameterized the correlation function by the normal-state density of states in the metal $\nu_{\rm M}$ and the electron mean free time $\tau_{\rm mfp}$. We assume that the superconductor is ``dirty'',  $\Delta\cdot \tau_{\rm mfp} \ll 1$. 

Let us now compute the average probability of the Andreev reflection (our approach is similar in spirit to that in Ref.~\onlinecite{hek94}).
Using Eq.~\eqref{eq:amplitudes}, we first represent $\langle |\he|^2 \rangle$ as
\begin{align}\label{eq:var}
    \langle |\he |^2 \rangle = \frac{(\partial_y \Phi)^4 t^4}{v^2}\int^{\lprox}_{0} dx_{1}dx_{2}dx_{3}dx_{4}e^{ik_\mu(x_1 +x_2)}e^{-ik_\mu(x_3 + x_4)}\partial^2_{y_1,y_2}\partial^2_{y_3, y_4}\Lngl {\cal G}_{\rm he}(\bm r_1, \bm r_2|0)\cdot {\cal G}_{\rm eh}(\bm r_4, \bm r_3|0)\Rngl\big|_{y_\alpha, z_\alpha = 0}.
\end{align}
On the right hand side, we replaced the average by its irreducible component. This is possible because $\langle \he \rangle = 0$ (see discussion in the main text). 
The superconductor Green's functions in Eq.~\eqref{eq:var} can be expressed in terms of the retarded Green's function $\GR$ of the metal in the normal state:
\begin{equation}\label{eq:greeen}
    {\cal G} ({\bm r}_1, {\bm r}_2| E) = \int \frac{d\epsilon}{\Delta^2 - E^2 + \epsilon^2} \begin{pmatrix}
    E+\epsilon & \Delta\\
    \Delta & E - \epsilon
    \end{pmatrix} \frac{1}{\pi}\,{\rm Im}\,\GR ({\bm r}_1, {\bm r}_2| \epsilon).
\end{equation}
Substituting this relation with $E = 0$ into Eq.~\eqref{eq:var} we obtain
\begin{align}\label{eq:var2}
   \langle |\he|^2 \rangle  =&  \frac{(\partial_y \Phi)^4 t^4}{\pi^2 v^2}\int^{\lprox}_{0} dx_{1}dx_{2}dx_{3}dx_{4}e^{ik_\mu(x_1 +x_2)}e^{-ik_\mu(x_3 + x_4)}\int \frac{\Delta d\epsilon}{\Delta^2 +\epsilon^2} \frac{\Delta d\epsilon^\prime}{\Delta^2+\epsilon^{\prime 2}}\notag\\
    &\times\partial^2_{y_1,y_2}\partial^2_{y_3, y_4}\Lngl {\rm Im}\, \GR(\bm r_1, \bm r_2|\epsilon)\cdot{\rm Im}\, \GR(\bm r_4, \bm r_3|\epsilon^\prime)\Rngl\big|_{y_\alpha, z_\alpha = 0}.
\end{align}
Let us focus on the averaged-over-disorder product of the Green's functions in the above expression. We can represent this product as
\begin{align}\label{eq:two_conts}
    \Lngl {\rm Im}\, \GR({\bm r}_1,{\bm r}_2|\epsilon)&\cdot {\rm Im}\, \GR({\bm r}_4, {\bm r}_3|\epsilon^\prime)\Rngl = \frac{1}{2}{\rm Re}\big[\Lngl \GR({\bm r}_1,{\bm r}_2|\epsilon)\cdot \GA({\bm r}_4, {\bm r}_3|\epsilon^\prime)\Rngl - \Lngl \GR({\bm r}_1,{\bm r}_2|\epsilon)\cdot \GR({\bm r}_4, {\bm r}_3|\epsilon^\prime)\Rngl\big],
\end{align}
where $\GA$ is the advanced normal state Green's function. We will see below that the contribution of the first term to $\langle |\he|^2\rangle$ is determined by long diffusive electron trajectories of size $\sim \xic$ ($\xic$ is the superconducting coherence length). On the other hand, the contribution of the second term is determined by trajectories of length $\lesssim \lambda_F$ only ($\lambda_F$ is the Fermi wave length in the superconductor). This means that the latter contribution is small compared to the one produced by the first term in Eq.~\eqref{eq:two_conts}. In what follows we neglect the second term.

The average $\lngl \GR \cdot \GA\rngl$ can be expressed in terms of the normal-state diffuson and Cooperon \cite{aleiner96}. 
Using Eq.~\eqref{eq:disorder} and neglecting small corrections that have a relative magnitude $\sim \lambda_F / l_{\rm mfp} \ll 1$ (with $l_{\rm mfp} = v_F \tau_{\rm mfp}$ being the mean free path), we represent $\lngl \GR \cdot \GA\rngl$ as
\begin{align}
    \Lngl \GR({\bm r}_1,{\bm r}_2|\epsilon)&\cdot \GA({\bm r}_4, {\bm r}_3|\epsilon^\prime)\Rngl\notag\\
    &= \frac{1}{2\pi\nu_{\rm M}\tau_{\rm mfp}^{2}}\int d^3rd^3r^\prime\,\langle \GR({\bm r}_{1},{\bm r}|\epsilon)\rangle\langle \GA({\bm r},{\bm r}_{3}|\epsilon^{\prime})\rangle \difz ({\bm r},{\bm r}^\prime|\epsilon-\epsilon^{\prime})\langle \GA({\bm r}_4,{\bm r}^\prime|\epsilon^{\prime})\rangle\langle \GR({\bm r}^\prime,{\bm r}_{2}|\epsilon)\rangle\label{eq:diffuson}\\
    &+ \frac{1}{2\pi\nu_{\rm M}\tau_{\rm mfp}^{2}}\int d^3rd^3r^\prime\,\langle \GR({\bm r}_{1},{\bm r}|\epsilon)\rangle\langle \GA({\bm r},{\bm r}_{4}|\epsilon^{\prime})\rangle \coop ({\bm r},{\bm r}^\prime|\epsilon-\epsilon^{\prime})\langle \GA({\bm r}_{3},{\bm r}^\prime|\epsilon^{\prime})\rangle\langle \GR({\bm r}^\prime,{\bm r}_{2} |\epsilon)\rangle.\label{eq:cooperon}
\end{align}
Here functions ${\difz}({\bm r},{\bm r}^\prime|\epsilon-\epsilon^{\prime})$ and ${\coop}({\bm r},{\bm r}^\prime|\epsilon-\epsilon^{\prime})$ are the diffuson and the Cooperon, respectively. 
The magnetic field does not penetrate a type I superconductor, so $\difz({\bm r},{\bm r}^\prime|\epsilon-\epsilon^{\prime}) = \coop({\bm r},{\bm r}^\prime|\epsilon-\epsilon^{\prime})$ in this case. In the time domain, ${\difz}({\bm r},{\bm r}^\prime|t)$ satisfies the diffusion equation \cite{aleiner96},
\begin{equation}\label{eq:diffusion}
    (\partial_t - D \partial_{\bm r}^2) {\difz}({\bm r},{\bm r}^\prime|t) = \delta(t)\delta({\bm r} - {\bm r}^\prime),
\end{equation}
with the boundary condition corresponding to the vanishing of the probability current at the metal's surface. Here $D = v_F l_{\rm mfp} / 3$ is the diffusion constant.

At relevant energies $\epsilon - \epsilon^\prime \sim \Delta$, the diffuson ${\difz}({\bm r},{\bm r}^\prime|\epsilon - \epsilon^\prime)$ varies at a length scale of the order of $\xic$. The latter satisfies $\xic \gg l_{\rm mfp}$ for a dirty superconductor. At the same time, the average Green's functions decay at a distance $\sim l_{\rm mfp}$. This means that in Eqs.~\eqref{eq:diffuson} and \eqref{eq:cooperon} the argument ${\bm r}$ of $\difz$ and $\coop$ is close to ${\bm r}_1$ and the argument ${\bm r}^\prime$ is close to ${\bm r}_2$. Consequently, we can approximate $\lngl \GR \cdot \GA \rngl$ as
\begin{align}\label{eq:RA}
    \Lngl \GR ({\bm r}_1,{\bm r}_2|\epsilon)\cdot \GA({\bm r}_4, {\bm r}_3|\epsilon^\prime)\Rngl
    = 2\pi \nu_{\rm M}\,{\difz}({\bm r}_1, {\bm r}_2|\epsilon - \epsilon^\prime) [V({\bm r}_1, {\bm r}_3)V({\bm r}_2, {\bm r}_4) + V({\bm r}_1, {\bm r}_4)V({\bm r}_2, {\bm r}_3)], 
\end{align}
where we abbreviated
\begin{equation}\label{eq:vertex}
    V({\bm r}_1, {\bm r}_3) = \frac{1}{2\pi\nu_{\rm M} \tau_{\rm mfp}}\int d^3r\,\langle \GR({\bm r}_{1},{\bm r}|\epsilon)\rangle\langle \GA({\bm r},{\bm r}_3|\epsilon^{\prime})\rangle.
\end{equation}
Combining Eqs.~\eqref{eq:var2}, \eqref{eq:two_conts}, and \eqref{eq:RA}, we obtain the following expression for $\langle |\he|^2 \rangle$:
\begin{align}\label{eq:var3}
   \langle |\he|^2 \rangle = & \frac{\nu_{\rm M}(\partial_y \Phi)^4 t^4}{\pi v^2}\int^{\lprox}_{0} dx_{1}dx_{2}dx_{3}dx_{4}e^{ik_\mu(x_1 +x_2)}e^{-ik_\mu(x_3 + x_4)}\int \frac{\Delta d\epsilon}{\Delta^2 +\epsilon^2} \frac{\Delta d\epsilon^\prime}{\Delta^2+\epsilon^{\prime 2}} \notag\\
    &\times{\rm Re}\,{\difz}(x_1,x_2|\epsilon - \epsilon^\prime)\,\partial^2_{y_1,y_2}\partial^2_{y_3, y_4}[V({\bm r}_1, {\bm r}_3)V({\bm r}_2, {\bm r}_4) + V({\bm r}_1, {\bm r}_4)V({\bm r}_2, {\bm r}_3)]\big|_{y_\alpha, z_\alpha = 0},
\end{align}
where ${\difz}(x_1,x_2|\epsilon - \epsilon^\prime) \equiv {\difz}(\bm r_1,\bm r_2|\epsilon - \epsilon^\prime)|_{y_{1,2},z_{1,2} = 0}$.

So far, we have been focusing on the case of a type I superconductor. 
Type II superconductor is different in that it admits magnetic field $B$.
The field affects functions ${\difz}$ and ${\coop}$ leading to additional phase factors in them. At relevant distances $\sim \xic$ the corresponding phases can be estimated as $\sim B \xic^2 / \phi_0 \sim B / H_{\rm c2}$ ($\phi_0$ is the flux quantum and $H_{\rm c2}$ is the upper critical field).
We see that for fields $B \ll H_{\rm c2}$ the phases are small and can be disregarded.
This means the derived at $B = 0$ Eq.~\eqref{eq:var3} is also applicable for the case of a type II superconductor in the regime $B \ll H_{\rm c2}$.
The same holds for all of the results presented in the remainder of the section.

Let us proceed with the derivation of $1/\lA$. Functions $V$ in Eq.~\eqref{eq:var3} stipulate ${\bm r}_1 \approx {\bm r}_3$, ${\bm r}_2 \approx {\bm r}_4$ in the diffuson's contribution and ${\bm r}_1 \approx {\bm r}_4$, ${\bm r}_2 \approx {\bm r}_3$ in the Cooperon's contribution. By making a direct calculation of the integral in Eq.~\eqref{eq:vertex}, we find for the combination of functions $V$ in Eq.~\eqref{eq:var3}:
\begin{align}\label{eq:vertex_expr}
    \partial^2_{y_1,y_2}\partial^2_{y_3, y_4}[V({\bm r}_1, {\bm r}_3)V({\bm r}_2, {\bm r}_4) + & V({\bm r}_1, {\bm r}_4)V({\bm r}_2, {\bm r}_3)]\big|_{y_\alpha, z_\alpha = 0}\notag\\
    &= (\pi p_F)^2 \big[\delta(x_1 -x_3) \delta(x_2-x_4) + \delta(x_1 -x_4)\delta(x_2-x_3)\big],
\end{align}
where $p_F$ is the Fermi momentum of the superconductor. The delta-functions in this expression should be interpreted as peaks of width $\sim \lambda_F$. With the help of Eq.~\eqref{eq:vertex_expr}, we can rewrite Eq.~\eqref{eq:var3} as
\begin{align}\label{eq:var4}
   \langle |\he|^2 \rangle = \frac{2\pi \nu_{\rm M}(\partial_y \Phi)^4 t^4 p_F^2}{v^2} \int_0^\lprox dx_{1}dx_{2}\int \frac{\Delta d\epsilon}{\Delta^2 +\epsilon^2} \frac{\Delta d\epsilon^\prime}{\Delta^2+\epsilon^{\prime 2}}
   {\rm Re}\,{\difz}(x_1,x_2|\epsilon - \epsilon^\prime).
\end{align}
The expression for ${\difz}(x_1,x_2|\epsilon - \epsilon^\prime)$ is sensitive to a particular geometry of the considered device. We will assume that the width of the superconducting film exceeds $\xic$. In this case, the film can be regarded as being three-dimensional for diffusion. We then find:
\begin{equation}\label{eq:diffuson_expl}
    {\difz}(x_1,x_2|\epsilon - \epsilon^\prime) = 2 \int_0^{+\infty}\frac{dt }{(4\pi D t)^{3/2}}e^{-i(\epsilon - \epsilon^\prime)t - \frac{(x_1 - x_2)^2}{4Dt}}
\end{equation}
(the factor of $2$ results from the boundary condition for Eq.~\eqref{eq:diffusion}).
Using this expression, one can easily show that
\begin{equation}\label{eq:diff_int}
    \int \frac{\Delta d\epsilon}{\Delta^2 +\epsilon^2} \frac{\Delta d\epsilon^\prime}{\Delta^2+\epsilon^{\prime 2}}
   {\rm Re}\,{\difz}(x_1,x_2|\epsilon - \epsilon^\prime) = \frac{\pi}{2D|x_{1}-x_{2}|}e^{-|x_{1}-x_{2}|/\xic},\quad\quad \xic = \sqrt{\frac{D}{2\Delta}}.
\end{equation}
We will assume that the length of the proximitized segment exceeds the coherence length, $\lprox \gg \xic$. Then, using Eq.~\eqref{eq:diff_int} in Eq.~\eqref{eq:var4} we obtain
\begin{equation}\label{eq:finalish}
   \langle |\he|^2 \rangle = \frac{2\pi^2 \nu_{\rm M}(\partial_y \Phi)^4 t^4 p_F^2}{v^2 D}
   \int_0^{+\infty} \frac{dx}{x} e^{-x/\xic} \cdot \lprox = \frac{2\pi^2 \nu_{\rm M}(\partial_y \Phi)^4 t^4 p_F^2}{v^2 D}\ln{\frac{\xic}{l_{\rm mfp}}}\cdot \lprox. 
\end{equation}
In the latter equality, we regularized the logarithmic divergence at small distances by the mean free path $l_{\rm mfp}$, i.e., by the length scale at which the diffusive behavior ceases. 

Finally, it is convenient to express the factor in front of the logarithm in Eq.~\eqref{eq:finalish} in terms of the normal-state conductivity of the metal $\sigma = 2e^2\nu_{\rm M}D$, and of the conductance per unit length of the interface $g = 2\pi^2 G_Q t^2 (\partial_y \Phi)^2 \nu_{\rm QH} \nu_{\rm M} p_F$ (here $\nu_{\rm QH} = (2\pi v)^{-1}$ is the density of edge states per spin projection and $G_Q = e^2 / \pi$). In this way we obtain Eq.~(8) of the main text.

\section{Derivation of $\langle \Theta^2 \rangle$ for the forward scattering phase $\Theta$}

Here we present the derivation of $\langle\Theta^2\rangle$ for the forward scattering phase $\Theta$ accumulated by an electron across a short proximitized segment.
We can obtain an expression for $\Theta$ similarly to how we found the amplitude $\he$, see Eq.~(7) of the main text. By treating $H_{\rm prox}$ in Eq.~(3) as a perturbation, we find:
\begin{equation}
    \Theta = \frac{(\partial_y \Phi)^2 t^2}{v}\hspace{-0.1cm}\int_0^\lprox \hspace{-0.1cm} dx_1 dx_2 e^{i\kf (x_1 - x_2)}\partial^2_{y_1 y_2}{\cal G}_{\rm ee} (x_1, x_2).\label{eq:amplitudes2}
\end{equation}
Here ${\cal G}_{\rm ee}(x_1, x_2) = {\cal G}_{\rm ee}(\bm r_1, \bm r_2 | E = 0)|_{y_{1,2}, z_{1,2} = 0}$ is the normal component of the superconductor Green's function. 
It is easy to verify using Eq.~\eqref{eq:greeen} at $E = 0$ that $\langle \Theta \rangle = 0$.
An expression for $\langle \Theta^2 \rangle$ can be obtained similarly to how we found $\langle |\he|^2\rangle$. A counterpart of Eq.~\eqref{eq:var3} is 
\begin{align}
     \langle \Theta^2 \rangle = &\frac{2\pi \nu_{\rm M}(\partial_y \Phi)^4 t^4 p_F^2}{v^2}\int^{\lprox}_{0} dx_{1}dx_{2}dx_{3}dx_{4}e^{ik_\mu(x_1 -x_2)}e^{-ik_\mu(x_3 - x_4)}\int \frac{\epsilon d\epsilon}{\Delta^2 +\epsilon^2} \frac{\epsilon^\prime d\epsilon^\prime}{\Delta^2+\epsilon^{\prime 2}} \notag\\
    &\times{\rm Re}\,{\difz}(x_1,x_2|\epsilon - \epsilon^\prime)\,\big[\delta(x_1 -x_3) \delta(x_2-x_4) + \delta(x_1 -x_4)\delta(x_2-x_3)\big],
\end{align}
where we also used Eq.~\eqref{eq:vertex_expr} for functions $V$. 
Using Eq.~\eqref{eq:diffuson_expl} for ${\difz}(x_1,x_2|\epsilon - \epsilon^\prime)$, we can rewrite the above expression as
\begin{align}
    \langle \Theta^2 \rangle &= \frac{\pi^2 \nu_{\rm M}(\partial_y \Phi)^4 t^4 p_F^2}{2v^2 D} \int^{\lprox}_{0} \frac{dx_{1}dx_{2}}{|x_1 - x_2|}e^{-|x_1-x_2|/\xic} \big[1 + \cos[2k_{\mu} (x_1 - x_2)]\big].\label{eq:fwdfinal}
\end{align}
The distance between points $x_1$ and $x_2$ here does not exceed the coherence length $\xic$. The latter satisfies $\xic \ll l_B \lesssim |k_\mu|^{-1}$ for a type II superconductor in field $B \ll H_{\rm c2}$. These estimates mean that the argument of cosine in Eq.~\eqref{eq:fwdfinal} is small, allowing one to approximate $\cos[2k_\mu (x_1 - x_2)] = 1$. Then, the right hand side of Eq.~\eqref{eq:fwdfinal} becomes identical to that of Eq.~\eqref{eq:finalish} for $\langle |\he|^2\rangle$. As a result, we obtain
\begin{equation}\label{eq:Theta}
    \langle \Theta^2 \rangle = \frac{\lprox}{\lA},
\end{equation}
where $\lA$ is given by Eq.~(8) of the main text. 

\section{Derivation of the conductance correlation function\label{sec:condcorrf}}

Here we present the derivation of the conductance correlation function ${\cal C}(\delta n, \delta B) = \lngl G(n, B) \cdot G(n + \delta n, B + \delta B) \rngl$,
which we use to obtain Eqs.~(16) and (17) of the main text.

To start with, we briefly discuss the main mechanism leading to the loss of correlation between the values of $G$ at parameters $(n, B)$ and $(n + \delta n, B + \delta B)$, respectively. Firstly, the variation $(\delta n, \delta B$) shifts the Fermi momentum of chiral electrons by $\delta k_\mu(\delta n, \delta B)$. As discussed after Eq.~(15) of the main text, this affects the phases of the Andreev amplitudes $\alpha(x)$. The phases are also affected by the change in the diamagnetic current flowing along the superconductor's surface.
The two effects can be accounted for by adding the phase factor to the Andreev amplitude, $\alpha (x) \rightarrow \alpha(x) e^{2i\delta k^{\rm (tot)}_{\mu} x}$, where $\delta k^{\rm (tot)}_{\mu} = \delta k_\mu  - \frac{1}{2} \delta (\partial_x \varphi)$ and $\partial_x \varphi$ is the gradient of the order parameter phase associated with the diamagnetic current. The variation $(\delta n, \delta B)$ also affects the magnitudes $|\alpha(x)|$ and $|\vartheta(x)|$. 
The reason is the dependence of $\partial_y \Phi$ and $v$ in Eqs.~\eqref{eq:amplitudes} and \eqref{eq:amplitudes2} on $n$ and $B$.
The magnitudes change as $|\alpha(x)| \rightarrow (1 + \delta g / g) |\alpha(x)|$ and $|\vartheta(x)| \rightarrow (1 + \delta g / g) |\vartheta(x)|$, where we used the relation for $g$ presented at the end of Sec.~\ref{sec:lA}.

To find ${\cal C}(\delta n, \delta B)$, we use Eq.~(10) of the main text to compare the results of the wave function evolution across the proximitized segment at parameters $(n, B)$ and $(n+\delta n, B + \delta B)$. We denote the components of the wave function by $a_{\rm e}(x)$, $a_{\rm h}(x)$ and $b_{\rm e}(x)$, $b_{\rm h}(x)$ for the respective sets of parameters.
The corresponding evolution equations read
\begin{align}\label{eq:evol}
    i \frac{\partial}{\partial x}
    \begin{pmatrix}
    a_{\rm e}(x)\\
    a_{\rm h}(x)
    \end{pmatrix}&=
    \begin{pmatrix}
    -\vartheta(x) & \alpha^\star(x)\\
    \alpha(x) & \vartheta(x)
    \end{pmatrix}
    \begin{pmatrix}
    a_{\rm e}(x)\\
    a_{\rm h}(x)
    \end{pmatrix},\\
    i \frac{\partial}{\partial x}
    \begin{pmatrix}
    b_{\rm e}(x)\\
    b_{\rm h}(x)
    \end{pmatrix}&=
    \left(1 + \frac{\delta g}{g}\right)
    \begin{pmatrix}
    -\vartheta(x) & \alpha^\star(x) e^{-2i\delta k^{\rm (tot)}_{\mu} x}\\
    \alpha(x) e^{2i\delta k^{\rm (tot)}_{\mu} x} & \vartheta(x)
    \end{pmatrix}
    \begin{pmatrix}
    b_{\rm e}(x)\\
    b_{\rm h}(x)
    \end{pmatrix}.\label{eq:evol2}
\end{align}
We can represent ${\cal C}(\delta n, \delta B)$ in terms of the wave function components as
\begin{equation}\label{eq:Cintermsofwf}
{\cal C}(\delta n, \delta B) = \lngl G^2 \rngl \frac{\lngl |a_{\rm h}(\lprox)|^2 \cdot |b_{\rm h}(\lprox)|^2\rngl}{\lngl |a_{\rm h}(\lprox)|^2 \cdot |a_{\rm h}(\lprox)|^2\rngl}.
\end{equation}
To determine $\lngl |a_{\rm h}(\lprox)|^2 \cdot |b_{\rm h}(\lprox)|^2\rngl$, we derive
a system of differential equations describing the evolution with $x$ of the correlators  $\lngl a_i^\star(x) a_j(x) \cdot b_k^\star(x) b_l(x) \rngl$, where $i,j,k,l = {\rm e, h}$. In fact, a closed system of equations can be obtained using Eq.~(9) of the main text and following the approach described in Ref.~\cite{ovchinnikov1980}.
The system has a particularly simple form in terms of the following variables:
\begin{align}
c_0(x) &= \lngl |a_{\rm h}(x)|^2 \cdot |b_{\rm h}(x)|^2 \rngl + e^{-2\big(1+(1+\frac{\delta g}{g})^2\big) \frac{x}{\lA}} / 4,\label{eq:c0}\\
c_+(x) &= {\rm Re}\,\lngl a_{\rm e}^\star(x)a_{\rm h}(x) \cdot b_{\rm h}^\star (x)b_{\rm e}(x)\rngl,\label{eq:cp}\\
c_-(x) &= {\rm Im}\,\lngl a_{\rm e}^\star(x)a_{\rm h}(x) \cdot b_{\rm h}^\star (x)b_{\rm e}(x)\rngl.\label{eq:cm}
\end{align}
We obtain
\begin{equation}\label{eq:system}
\frac{\partial}{\partial x}\left(\begin{array}{c}
c_0(x)\\
c_+(x)\\
c_-(x)
\end{array}\right)=
\frac{1}{\lA}\left(\begin{array}{ccc}
-2\big( 1+\big(1+\frac{\delta g}{g}\big)^{2}\big)  & 2\big(1+\frac{\delta g}{g}\big) & 0\\
4\big(1+\frac{\delta g}{g}\big) & -\big( 1+\big(1 + \frac{\delta g}{g}\big)^{2}\big) -2\big(\frac{\delta g}{g}\big)^{2} & 2\delta k_\mu^{\rm (tot)}\lA\\
0 & -2\delta k_\mu^{\rm (tot)}\lA & -\big( 1+\big(1 + \frac{\delta g}{g}\big)^{2}\big) -2\big(\frac{\delta g}{g}\big)^{2}
\end{array}\right)\left(\begin{array}{c}
c_0(x)\\
c_+(x)\\
c_-(x)
\end{array}\right)
\end{equation}
(we also made a gauge transformation $b_{\rm e/h}(x) \rightarrow e^{\mp i\delta k_{\mu}^{\rm (tot)} x} b_{\rm e/h}(x)$ when deriving the system). The initial conditions are $c_0(0) = 1/4$, $c_\pm (0) = 0$. 

Let us assume that $\delta k_\mu^{(\rm tot)}\lA\ll 1$ and $\delta g / g \ll 1$. Under these conditions, system \eqref{eq:system} can be analyzed with the help of the perturbation theory. At $\delta g = 0$ and $\delta k_\mu^{(\rm tot)} = 0$, the $3\times 3$ matrix on the right hand side of  Eq.~\eqref{eq:system} has an eigenvalue $\omega = 0$. The zero eigenvalue corresponds to the steady state solution of the Fokker-Planck equation, see Eq.~(11) of the main text. The respective eigenvector is $(1, 2, 0)^T$.
The correction to $\omega = 0$ due to finite $\delta k_\mu^{\rm (tot)}$ and $\delta g$ is of the second order in these parameters:
\begin{equation}
    \omega = -\frac{4}{3} (\delta k_{\mu}^{(\rm tot)})^2 \lA - \frac{8}{3} \left(\frac{\delta g}{g}\right)^2 \frac{1}{\lA}.
\end{equation}
Using this expression, we find the solution of system \eqref{eq:system} at $x \gg \lA$:
\begin{equation}\label{eq:solforcs}
\left(\begin{array}{c}
c_0(x)\\
c_+(x)\\
c_-(x)
\end{array}\right)
=
\frac{1}{12} 
\begin{pmatrix}
1\\
2\\
0
\end{pmatrix}
\exp\Big[-\frac{4}{3} (\delta k_{\mu}^{(\rm tot)})^2 \lA x - \frac{8}{3} \Big(\frac{\delta g}{g}\Big)^2 \frac{x}{\lA}\Big].
\end{equation}
Setting $x = \lprox$ and using Eqs.~\eqref{eq:c0} and \eqref{eq:Cintermsofwf}, we obtain
\begin{equation}\label{eq:corrgen}
    {\cal C}(\delta n, \delta B) = \lngl G^2 \rngl \exp \Big[-\frac{4}{3} (\delta k_{\mu}^{(\rm tot)})^2 \lA \lprox -  \frac{8}{3} \left(\frac{\delta g}{g}\right)^2 \frac{\lprox}{\lA}\Big].
\end{equation}

We now apply the general result \eqref{eq:corrgen} to find the conductance correlation function with density ${\cal C}(\delta n)$. The change of the wave vector $k_\mu$ upon the variation $\delta n$ can be expressed as (we recall that $\hbar = 1$)
\begin{equation}\label{eq:kmudn}
    \delta k_\mu = \frac{\partial \mu}{\partial n} \frac{\delta n}{v},
\end{equation}
where $\partial \mu / \partial n$ is the inverse compressibility of the quantum Hall state.
The influence of $\delta n$ on $g$ can be disregarded provided $\lA \gg l_B = \sqrt{c / eB}$. Assuming the latter condition to be satisfied, we disregard the second term in the square brackets of Eq.~\eqref{eq:corrgen}. Then, substituting Eq.~\eqref{eq:kmudn} in Eq.~\eqref{eq:corrgen} we arrive to Eqs.~(16) and (17) of the main text.

\section{Derivation of Eq.~(18) for ${\cal C}_{\rm jump}(d)$}

In this section, we present details of the derivation of Eq.~(18) for the variance of the conductance jumps ${\cal C}_{\rm jump}(d) = \langle (\delta G)^2 \rangle$. We assume the proximitized segment to be long throughout the section, $\lprox \gg \lA$.


To find ${\cal C}_{\rm jump}(d)$, we compare the results of the wave function evolution across the proximitized segment before and after a vortex has entered the superconductor.
We denote the wave function components as $a_{\rm e}(x)$ and $a_{\rm h}(x)$ before the vortex entrance, and as $b_{\rm e}(x)$ and $b_{\rm h}(x)$ after it. The corresponding evolution equations are given by
\begin{equation}\label{eq:evol_v}
    i \frac{\partial}{\partial x}
    \begin{pmatrix}
    a_{\rm e}(x)\\
    a_{\rm h}(x)
    \end{pmatrix}
    = 
    \begin{pmatrix}
    -\vartheta(x) & \alpha^\star(x)\\
    \alpha(x) & \vartheta(x)
    \end{pmatrix}
    \begin{pmatrix}
    a_{\rm e}(x)\\
    a_{\rm h}(x)
    \end{pmatrix}, \quad \quad
    i \frac{\partial}{\partial x}\begin{pmatrix}
    b_{\rm e}(x)\\
    b_{\rm h}(x)
    \end{pmatrix} = 
    \begin{pmatrix}
    -\vartheta(x) & \alpha^\star(x) e^{i\delta \varphi(x)}\\
    \alpha(x)e^{-i\delta\varphi(x)} & \vartheta(x)
    \end{pmatrix} 
    \begin{pmatrix}
    b_{\rm e}(x)\\
    b_{\rm h}(x)
    \end{pmatrix}.
\end{equation}
Here $\delta \varphi(x) = \pi + \arctan ([x - x_{\rm v}] / d)$ is the phase induced by the entered vortex (we assume that pinning in the superconductor is strong enough so that the entrance of the vortex does not affect the preexisting vortex distribution). As mentioned in the main text, $d$ is the distance between the vortex core and the interface, and $x_{\rm v}$ is the core's coordinate along the $x$-direction.

The variance of the conductance jumps can be expressed in terms of the wave functions components as:
\begin{equation}\label{eq:jump_def}
    {\cal C}_{\rm jump}(d) = 2 \lngl G^2 \rngl \left[1 - \frac{\lngl |a_{\rm h}(\lprox)|^2 \cdot |b_{\rm h}(\lprox)|^2\rngl}{\lngl |a_{\rm h}(\lprox)|^2 \cdot |a_{\rm h}(\lprox)|^2\rngl}
\right].
\end{equation}
To find $\lngl |a_{\rm h}(\lprox)|^2 \cdot |b_{\rm h}(\lprox)|^2\rngl$, we derive a system of equations for correlators $\lngl a_i^\star(x) a_j(x) \cdot b_k^\star(x) b_l(x) \rngl$ similarly to how we did it in Sec.~\ref{sec:condcorrf}. The system reads
\begin{equation}\label{eq:sys_2}
    \frac{\partial}{\partial x}\left(\begin{array}{c}
c_0(x)\\
c_+(x)\\
c_-(x)
\end{array}\right)
=
\frac{1}{l_{\rm A}}\left(\begin{array}{ccc}
-4  & 2 & 0\\
4 & -2  & -\lA \partial_x \delta \varphi(x) \\
0 & \lA \partial_x \delta \varphi(x) & -2 
\end{array}\right)\left(\begin{array}{c}
c_0(x)\\
c_+(x)\\
c_-(x)
\end{array}\right).
\end{equation}
Here variable $c_{0}(x) = \lngl |a_{\rm h}(x)|^2 \cdot |b_{\rm h}(x)|^2 \rngl + e^{-4 x / \lA} / 4$, whereas variables $c_\pm (x)$ are defined in the same way as in Eqs.~\eqref{eq:cp} and \eqref{eq:cm}. 

System of equations \eqref{eq:sys_2} can be solved analytically in the two limiting cases, $d \ll \lA$ and $d \gg \lA$. Let us start with the former case. The condition $d \ll \lA$ means that the kink in the superconducting phase $\delta \varphi(x)$ is narrow. This suggests one to approximate
\begin{equation}\label{eq:deltaappr}
    \partial_x \delta \varphi(x) = 2\pi\cdot  \frac{1}{\pi}\frac{d}{d^2 + (x - x_{\rm v})^2}\approx 2\pi \delta (x - x_{\rm v})
\end{equation}
in Eq.~\eqref{eq:sys_2}.
However, such an approximation is too crude. Indeed, it can be easily verified that the vector $(c_0(x),c_+(x),c_-(x))^T$ does not change across $x = x_{\rm v}$ if we replace $\partial_x \delta \varphi(x) \rightarrow 2\pi \delta (x - x_{\rm v})$. 
Thus, ${\cal C}_{\rm jump}(d)=0$ to the zeroth order in $d / \lA$.

The leading in $d / \lA$ result for ${\cal C}_{\rm jump}(d)$ can be obtained in the following way. First of all, we go to a rotating frame in Eq.~\eqref{eq:sys_2}:
\begin{equation}
    \left(\begin{array}{c}
c_0(x)\\
c_+(x)\\
c_-(x)
\end{array}\right) = \exp\Big[ \begin{pmatrix}
0 & 0 & 0\\
0 & 0 & -\delta\varphi(x)\\
0 & \delta \varphi(x) & 0
\end{pmatrix}\Big]\left(\begin{array}{c}
\tilde{c}_0(x)\\
\tilde{c}_+(x)\\
\tilde{c}_-(x)
\end{array}\right)
\end{equation}
(we choose the frame in such a way that terms $\propto \partial_{x}\delta \varphi(x)$ cancel on two sides of the equation after the transformation).
In this frame, the ``scatterer'' associated with the vortex is described by a local perturbation of magnitude $\sim 1 / \lA$ and width $\sim d$. It can be treated using an analog of Born approximation. A straightforward calculation leads to
\begin{equation}
\lngl |a_{\rm h}(\lprox)|^2 \cdot |b_{\rm h}(\lprox)|^2\rngl = \lngl |a_{\rm h}(\lprox)|^2 \cdot |a_{\rm h}(\lprox)|^2\rngl \left(1 - \frac{16\pi}{3}\frac{d}{\lA}\right).
\end{equation}
Using this expression in Eq.~\eqref{eq:jump_def}, we obtain the result presented in the first line of Eq.~(18) of the main text.

Now we consider the limit of $d \gg \lA$. In this limit, we can account for $\partial_x \delta \varphi(x) \lA$ in system \eqref{eq:sys_2} with the help of the adiabatic approximation. Using the similarity of system \eqref{eq:sys_2} to system \eqref{eq:system} (taken at $\delta g = 0$), we find by generalizing Eq.~\eqref{eq:solforcs}:
\begin{equation}
\left(\begin{array}{c}
c_0(x)\\
c_+(x)\\
c_-(x)
\end{array}\right) \approx \frac{1}{12} \begin{pmatrix}
1 \\ 2\\ 0
\end{pmatrix} e^{-\frac{1}{3}\int_0^x [\partial_x \delta \varphi(x^\prime) \lA]^2 dx^\prime}.
\end{equation}
Taking $x = \lprox \gg \lA$, computing the integral in the exponent, and using the definition of a variable $c_0(x)$, we find
\begin{equation}
    \lngl |a_{\rm h}(\lprox)|^2 \cdot |b_{\rm h}(\lprox)|^2\rngl   = \lngl |a_{\rm h}(\lprox)|^2 \cdot |a_{\rm h}(\lprox)|^2\rngl\,\exp\Big[-\frac{2\pi \lA}{d}\Big] \approx \lngl |a_{\rm h}(\lprox)|^2 \cdot |a_{\rm h}(\lprox)|^2\rngl  \left(1 -\frac{2\pi \lA}{d}\right).
\end{equation}
Substituting this expression in Eq.~\eqref{eq:jump_def}, we obtain the result presented in the second line of Eq.~(18) of the main text.

\begin{thebibliography}{100}
\newcounter{Sbib}

\stepcounter{Sbib}
\bibitem[S\theSbib]{hek94}
F.~W.~J.~Hekking  and Yu.~V.~Nazarov, {\em Subgap conductivity of a superconductor--normal-metal tunnel interface}, Phys.~Rev.~B {\bf 49}, 10, 6847 (1994).

\stepcounter{Sbib}
\bibitem[S\theSbib]{aleiner96}
I.~L.~Aleiner and A.~I.~Larkin, {\em Divergence of classical trajectories and weak localization}, Phys.~Rev.~B {\bf 54}, 20, 14423 (1996).

\stepcounter{Sbib}
\bibitem[S\theSbib]{ovchinnikov1980}
A.~A.~Ovchinnikov and N.~S.~Erikhman, {\em Temperature and frequency dependence of the electron
conductivity in a two-band model with impurities}, Sov. Phys. JETP {\bf 51}, 4, 728 (1980).

\end{thebibliography}
\end{document}