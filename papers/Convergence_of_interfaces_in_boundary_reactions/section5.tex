\section{The limiting varifold}

\subsection{Rectifiable varifolds } We define rectifiable varifolds and  some notions related to them that we require in this paper. For a thorough treatment of rectifiable varifolds we refer the reader to chapter 4 in \cite{si}.
\begin{defi}
A set $\Sigma \subset \RR^{n+1}$ is $k$-rectifiable if and only if $\Sigma \subset \cup_{i=0}^{\infty} \Sigma_i$, where $\mathcal{H}^k(\Sigma_0)=0$ and for $i\geq 1$, each $\Sigma_i$ is an embedded $k$-dimensional  $C^1$ submanifold in $\RR^{n+1}$ \end{defi}

\begin{rem}
    An important property of $k$-rectifiable sets that we will be using is the existence of (approximate) tangent space $T_x \Sigma$ for $\mathcal{H}^k$ a.e point $x \in \Sigma$.
\end{rem}

We denote by $Gr_{k,n+1}$ the $k$- grassmann manifold that contains unoriented $k$ planes in $\RR^{n+1}$, and identify each subspace will the (symmetric) matrix of orthogonal projection on it. Then a $k$- varifold in $U \subset \RR^{n+1}$ is a Radon measure on $U \times Gr_{k,n+1}$. For our purpose we are interested in the smaller class of rectifiable varifolds, namely

\begin{defi}
Let $\Sigma$ be a $k$-rectifiable set in an open set $U \subset \RR^{n+1}$ and a positive Borel function $\theta: \Sigma \to \RR$. Then $V=V(\Sigma, \theta)$ is a $k$-rectifiable varifold, with multiplicity function $\theta$, given by 
\begin{equation*}
    \langle V(\Sigma,\theta),f \rangle = \int_{\Sigma} \theta(x) f(T_x \Sigma) \,d\mathcal{H}^k_x \quad \text{for any } f\in C^0(Gr_{k,n+1})
\end{equation*}
Further, if the multiplicity function $\theta \in \NN \backslash \{0\}$,   $\mathcal{H}^k$ a.e. on $\Sigma$, then $V$ is called \textit{integral} or \textit{integer rectifiable}. 
\end{defi} 
\begin{defi}
For any $k$-rectifiable varifold $V(\Sigma,\theta)$, the associated mass measure $\mu_V$ is a Radon measure given by,
$$ 
\mu_V(A) = \int_{A \cap \Sigma} \theta \,d\mathcal{H}^k \quad \text{ for any }\mathcal{H}^k \text{ measurable set } A 
$$
and the mass of $V$ is 
$$
\textbf{M}(V) = \mu_V(\RR^{n+1})= \int_{\Sigma} \theta \,d\mathcal{H}^k
$$
Note that this is just the $k$-area functional. We denote by $\Phi_{\#}V$ the pushforward of $V$ by a diffeomorphism $\Phi$.
\end{defi}
Given any vector field $X \in C_c^1(U, \RR^{n+1})$, denote by $\{\Phi_t\}$ the one-parameter family of diffeomorphisms generated by it. We can now define the first variation of $V$.
\begin{defi}
    Let $V(\Sigma,\theta)$ be a rectifiable $k$- varifold and $X \in C_c^1(U, \RR^{n+1})$, then the first variation of $V$ is 
    \begin{equation}\label{fvari}
      \langle \delta V , X \rangle = \frac{d}{dt}\bigg|_{t=0}\textbf{M}((\Phi_{t})_{\#}V) = \int_{\Sigma} \theta \D_{\Sigma} X \,d\mathcal{H}^k  
    \end{equation}
     
    The varifold is said to be \textit{stationary} if $\langle \delta V, X \rangle =0$ for all $X$.
\end{defi}
\begin{rem}
     Note that the stationarity condition just states that $V(\Sigma,\theta)$ is a critical point of the $k$-area functional under pushforwards by diffeomorphisms. Therefore, it is reasonable to think of stationary rectifiable varifolds as a weak notion of minimal surfaces. 
\end{rem}


\subsection{Generalized Varifolds}
In several geometric problems involving energy concentration, the stress energy tensors associated to the problem are not varifolds and therefore, it is necessary to consider a larger space than the space of varifolds to deal with such problems.  For this purpose, Ambrosio and Soner introduced the notion of \textit{generalized varifold} in \cite{as} in their study of parabolic ginzburg-landau equations. Precisely, instead of looking at Radon measures in $U \times Gr_{k,n+1}$, they take Radon measures in $U \times A_{k,n+1}$, where $A_{k,n+1}$ compared to $Gr_{k,n+1}$ is a slightly larger subset of symmetric matrices, namely 
$$A_{k,n+1}= \{A \in Sym_{n+1} \, | \, tr(A)=k, -(n+1) I_{n+1} \leq A \leq I_{n+1}\}$$

Generalized varifolds enjoy similar compactness properties as varifolds and therefore one obtains that a limiting generalized varifold on energy concentration. Since introduction, they have been used in several works like \cite{lw,ms} and recently in the work of Pigati and Stern \cite{ps} on minmax construction of codimension-$2$ integer rectifiable stationary varifolds. 


\begin{defi}
  A generalized $k$-varifold in $U \subset \RR^{n+1}$ is a nonnegative Radon measure $V$ in $U \times A_{k,n+1}$.
\end{defi}

The notions of mass measure, first variation, stationarity and density extend in a straightforward manner from varifolds  to generalized varifolds. We refer to section 3 in \cite{as} for details. 

\subsection{The associated generalized varifolds}
We now see that the stress energy tensors associated to any $\ue$, satisfying (\ref{sec4eqn}) are $(n-1)$-generalized varifolds. Note that we will just write $A_{n-1}$ to denote $A_{n-1,n+1}$ The stress energy tensor corresponding to $\ue$ is
\begin{equation} \label{varifold}
V_\ve \coloneqq \frac{1}{2}|\nabla \ue|^2 \, T_x  
\end{equation}
where $T_x$ is given by
\begin{equation} \label{defT}
    T_x =
    \begin{cases}
        [I_{n+1} - 2 \nu_\ve \otimes \nu_\ve] \text{ if }|\nabla \ue (x)| \neq 0, \text{ here } \nu_\ve(x) = \frac{\nabla \ue (x)}{|\nabla \ue (x)|} \quad  \\
        0  \quad \text{if }|\nabla \ue (x)|=0
    \end{cases}
\end{equation}
Observe that 
$$
tr(T_x)=n-1 \text{ and }  -|v|^2 \leq \langle T_x(v),v \rangle \leq |v|^2 \quad \text{for any } v
$$
therefore $T_x \in A_{n-1}$ whenever $|\nabla \ue| \neq 0$ and we have the following definition. 
\begin{defi}
    Let $u_{\ve}$ satisfy (\ref{sec4eqn}). Then, there is an associated $(n-1)$ generalized varifold given by
    \begin{equation} \label{eqn:gvarifold}
    \langle V_\ve,f \rangle = \int_{|\nabla \ue|\neq 0} \frac{1}{2}|\nabla \ue|^2 f(T_x) \,dx \quad \text{for all } f\in C^0(A_{n-1}) 
\end{equation}
\end{defi}

There is also the related notion of convergece. 
\begin{defi}
    We say that a sequence of generalized varifolds $\{V_i\}$ converges to a generalized varifold $V$ if $\{V_i\}$ converge as Radon measures in $U \times A_{n-1}$ to $V$.
\end{defi}


\subsection{Rectifiablity of the limiting varifold} We will prove that there is a (classical) varifold with mass measure $\mu_{\Sigma}$ supported on $\Sigma$, and that it is stationary and rectifiable. Let $V_\ve$ be the generalized varifold associated to $\ue$ by (\ref{eqn:gvarifold}). We will show that by Theorem \ref{thm:sing},  $V_\ve \rightharpoonup V = V_* + V_{\Sigma}$, and the mass measure of $V_\Sigma$ is $\mu_{\Sigma}$. Further as a consequence of Proposition \ref{vanishing} we will show that $V$ is stationary. We are also able to show that $V_*$ and $V_{\Sigma}$ are stationary. As we already established a lower and upper bounds on density $\theta(x)$ for $\mathcal{H}^{n-1}$ a.e. $x \in \Sigma$ (Theorem \ref{thm:sing}(3)),  we will invoke a result in \cite{as} to conclud that $V_{\Sigma}$ is actually a stationary, rectifiable, $(n-1)$-varifold and in particular $\Sigma$ is a $(n-1)$-rectifiable set. 

\begin{theo}
Let $u_i$ satisfy (\ref{sec4eqn}) and $V_i$ be the generalized varifolds associated to it. Then as $i \to \infty$, we have the following,
\begin{enumerate}
    \item After passing to a subsequence, $V_i \rightharpoonup V= V_* + V_{\Sigma}$, here $V_*$ is the generalized varifold associated to $u_*$, and $V_{\Sigma}$ is a generalized varifold supported on $\Sigma \times A_{n-1}$ with mass measure $\mu_{\Sigma}$.
    \item $V$, $V_*$ and $V_{\Sigma}$ are stationary generalized varifolds. 
    \item And further $V_{\Sigma}$ can also be associated to a stationary, rectifiable varifold, with density $\theta(x)$ and supported on a $(n-1)$ rectifiable set $\Sigma$, i.e $V_{\Sigma} = V(\Sigma, \theta)$.
\end{enumerate}
\end{theo}

\begin{proof}
We first compute the first variation of $\delta V_i$. For any vector field $X\in C^1(U\cup \partial^0U)$ with $X_{n+1}=0$, the first variation is
\begin{align*}
    \langle \delta V_i, X \rangle
    &= \int_{U} \langle V_i , DX \rangle  dx  \\
    &= \int_U \frac{1}{2}|\nabla u_i|^2\langle T_x, DX \rangle \,dx \\
    &= \int_U \frac{1}{2}|\nabla u_i|^2\D X - 2DX(\nabla u_i, \nabla u_i)  \,dx \text{ by equation (\ref{defT})} \\
    &= \int_{\partial^0 U} \frac{-W(u_i)}{\ve_i} \D_{\RR^n} X \mh \quad \text{by equation (\ref{eqn:fv})} \\
    &\leq  \sup_{\partial^0 U}|\D_{\RR^n} X| \cdot \int_{\partial^0 U} \frac{W(u_i)}{\ve_i}  \mh  \quad \text{as } W \geq 0 \\
    &\leq C_{X} \ve_i \quad \text{by Theorem \ref{vanishing}}
\end{align*}
As $\mu_{V_i}(U) < E_0$, after passing to a subsequence there is a generalized varifold $V$ such that $V_i \rightharpoonup V$. In particular we have
\begin{align}
    \langle \delta V, X \rangle 
    &= \lim_{i \to \infty} \langle \delta V_i, X \rangle \\
    &=0   \label{Tsta}
\end{align}
Hence, $V$ is stationary. Further note that,
\begin{align*}
    \langle \delta V, X \rangle 
    &= \lim_{i\to \infty}\int_{U} \langle V_i , DX \rangle  dx \\
    &=\lim_{i \to \infty} \frac{1}{2} \int_{U}|\nabla \ui|^2 \langle I_{n+1} - 2 \nu_i \otimes \nu_i , DX \rangle \,dx  \\
    &=\lim_{i \to \infty} \frac{1}{2} \int_{U}|\nabla \ui|^2 \D X - 2DX\langle\nabla \ui,\nabla \ui\rangle \,dx  \\
\end{align*}
Note that by (\ref{measdecom}) we get
$$
\lim_{i \to \infty} \int_{U}|\nabla \ui|^2 \D X \,dx = \int_U |\nabla u_*|^2 \D X \,dx + \int_{\Sigma} \D_{\Sigma} X \,d\mu_{\Sigma}
$$
and because $\ui \rightharpoonup u_*$ weakly in $H^1(U)$ we get
$$
\lim_{i \to \infty}  \int_{U} DX\langle\nabla \ui,\nabla \ui\rangle \,dx = \int_U DX\langle\nabla u_*, \nabla u_*\rangle \,dx
$$
Combining these we resume the calculation
\begin{align*} 
    \langle \delta V, X \rangle  
    &= \frac{1}{2} \int_{U}|\nabla u_*|^2 \D X \,dx - 2DX\langle\nabla u_*,\nabla u_*\rangle \,dx+ \int_{\Sigma} \D_{\Sigma} X \,d\mu_{\Sigma}  \\
    &= \frac{1}{2} \int_{U}|\nabla u_*|^2 \langle I_{n+1} - 2 \nu_* \otimes \nu_* , DX \rangle \,dx + \int_{\Sigma} \D_{\Sigma} X \,d\mu_{\Sigma}  \\
    &=  \int_{U} \frac{1}{2}|\nabla u_*|^2 \langle T_x , DX \rangle \,dx + \int_{\Sigma} \D_{\Sigma} X \,d\mu_{\Sigma}                  \\
    &= \int_{U} \frac{1}{2}|\nabla u_*|^2 \langle T_x , DX \rangle \,dx + \int_{\Sigma} \theta(x) \D_{\Sigma} X \,d\mathcal{H}^{n-1} \text{ by Theorem \ref{thm:sing}}(3) \\
    &= \langle \delta V_*,X \rangle + \langle \delta V_{\Sigma},X \rangle 
\end{align*}
 Combined with equation (\ref{Tsta}), we get
$$\langle \delta V_{\Sigma},X \rangle  = -\langle \delta V_*,X \rangle$$
 We now show that $V_{\Sigma}$ is stationary. By the above, it is equivalent to showing that $\langle \delta V_*, X \rangle = 0$.
\begin{align*}
    \langle \delta V_{\Sigma},X \rangle 
    &= -\langle \delta V_*, X \rangle \\
    &= -\int_{U} \bigg( \frac{|\nabla u_*|^2}{2} \D X - DX\langle\nabla u_*,\nabla u_*\rangle \bigg)\,dx  \\
    &=-\int_{\partial^0U} (\nabla u_* \cdot X)\frac{\partial u_*}{\partial \nu} \mh \quad \text{by 2nd term in equation (\ref{ibp})} \\
    &= -\int_{\partial^0U \backslash \Sigma}\nabla(\pm 1)\cdot X \pdv{u_*}{\nu} \mh \text{ Proposition \ref{onu*} and } \mathcal{H}^n(\Sigma)=0   \\
    &= 0 
\end{align*}

 Hence, $V_{\Sigma}$ is stationary. We already have lower density bound $\mathcal{H}^{n-1}$ a.e. on $\Sigma$ for the mass measure $\mu_{\Sigma}$,  then by [Theorem 3.8(c)] in \cite{as}, $V_{\Sigma}$ is a stationary, $(n-1)$ rectifiable varifold with mass measure $\mu_{\Sigma}$, supported on $\Sigma$ . In particular, the set $\Sigma$ is $(n-1)$ rectifiable.
\end{proof}
