\section{Monotonicity formula and convergence lemma}
In this section and the next we will be studying the $\ve$-perturbed version of the following problem for $f=-W'$.
\begin{equation} \label{eqn:main1}
        \begin{cases}
            \Delta u = 0 \qquad &\text{in } B_R^+ \\
            \pdv{u}{\nu} = f(u) \qquad  &\text{on  } D_R
        \end{cases}
\end{equation}
A simple proof for the regularity of solutions of this equation can be found in \cite{csm}(lemma 2.3). We state it here for convenience as we will be invoking it a few times. 
\begin{theo}(from \cite{csm})
    Let $R>0$, $\alpha \in (0,1)$ and $u \in L^{\infty}(B^+_{R} \cap H^1(B^+_R)$ be a weak solution of (\ref{eqn:main1}). If $f \in C^{1,\alpha}$, then $u \in C^{2,\alpha}(\overline{B^+_{R/4}})$ and 
    \begin{equation} \label{csmest}
        \norm{u}_{C^{2,\alpha}(\overline{B^+_{R/4}})} \leq C(n,\alpha,R,\norm{f}_{C^{1,\alpha}},\norm{u}_{L^{\infty}_{B_R^+}})
    \end{equation}
\end{theo}
\begin{coro}
    When $f(u)= \frac{1}{\ve}(u-u^3)$, then $u \in C^{\infty}(\ball {R/4} \cup D_{R/4})$.
\end{coro}





\subsection{Monotonicity formula}We now prove a monotonicity formula for the associated energy and a convergence lemma that will be required in the proof of the epsilon regularity result. A monotonicity formula in this setting was proved in \cite{ms}. Here we give a proof following the allen-cahn case \cite{ht}.

 For any $x \in D_R$ and for $0<r\leq R-|x|$, we denote the scaled energy by 
 $$
    I_{\ve}(r,x) =  \frac{1}{r^{n-1}}E_{\ve}(u, \ball {r}(x))
 $$
\begin{theo}
Let $u_\ve \in C^2(\ball R \cup D_R)$, satisfying $|u_\ve|\leq 1$, with $\ve <R$ be a solution of 

    \begin{equation} \label{eqn:main}
        \begin{cases}
            \Delta \ue = 0 \qquad &\text{in } B_R^+ \\
            \pdv{\ue}{\nu} = -\frac{1}{\ve}W'(u_\ve) \qquad  &\text{on  } D_R
        \end{cases}
    \end{equation}
Then for any $x \in D_R$, and for $0< r_1 \leq r_2 \leq R-|x|$, the scaled energy $I_\ve(r,x)$ is monotonically increasing in $r$ and satisfies the following relation
\begin{multline} \label{eqn:mf1}
   I_{\ve}(r_2,x) - I_{\ve}(r_1,x)
    = \int_{r_1}^{r_2} \frac{1}{r^{n-1}}\bigg(\int_{\partial ^+\ball r(x)} |(y-x)\cdot \nabla u|^2 \mh_y \bigg) \,dr \\
    + \int_{r_1}^{r_2} \frac{1}{r^{n}} \bigg( \int_{D_r(x)} \frac{W(u)}{\ve} \mh \bigg) \,dr
    \geq 0
\end{multline}
\end{theo}
\begin{proof}
    
Given any vector field $X \in C_c^1(\ball R \cup D_R)$ such that $X^{n+1}|_{D_R}=0$, we claim that 
\begin{equation} \label{eqn:fv}
    \int_{\ball R} \bigg( \frac{|\nabla u|^2}{2} \D X - DX \langle \nabla u, \nabla u \rangle \bigg)\,dx =- \frac{1}{\ve}\int_{D_R} W(u) \D_{\RR^n} X \mh 
\end{equation}

we multiply the equation (\ref{eqn:main}) by $\nabla u \cdot X$ and integrate by parts. 

\begin{align}
    0 &= \int_{\ball R} \Delta u \nabla u 
    \cdot X \,dx \\
    &= -\int_{\ball R} \nabla[\nabla u \cdot X] \cdot \nabla u \,dx + \int_{D_R} (\nabla u \cdot X)\frac{\partial u}{\partial \nu} \mh \label{ibp}  \\
    &= -\int_{\ball R} \nabla[\nabla u \cdot X] \cdot \nabla u \,dx + \int_{D_R} (\nabla u \cdot X)\frac{-W'(u)}{\ve} \mh \\
    &= \Bigg[-\int_{\ball R} \nabla\frac{|\nabla u|^2}{2} \cdot X \,dx -\int_{\ball R} DX \langle \nabla u, \nabla u \rangle\,dx \Bigg] + \Bigg[\int_{D_R} -\frac{\nabla W(u)}{\ve} \cdot X \mh \Bigg] \\
    &= \int_{\ball R} \bigg( \frac{|\nabla u|^2}{2} \D X - DX \langle \nabla u, \nabla u \rangle\bigg)\,dx + \frac{1}{\ve}\int_{D_R} W(u) \D_{\RR^n} X \mh  
\end{align}

Let $\rho$ be a smooth mollification of $\chi_{\overline{\ball r}}$, the characteristic function of the half ball. Then we take $X(y) = y\rho(|y|)=(y_1\rho(r), \cdots,y_{n+1}\rho(r))$. Then we have 
$$ X^i_j = \delta^i_j \rho + \frac{y_i y_j}{r}\rho '(r)$$ Plugging this into the equation we get 

\begin{equation}
    \int_{\ball R} \bigg( \frac{|\nabla u|^2}{2}[(n-1)\rho+r\rho ')] - \frac{\rho '}{r}(y\cdot \nabla u)^2\bigg)\,dx + \frac{1}{\ve}\int_{D_R} W(u)(n\rho + r \rho ') \mh = 0
\end{equation}

As $\rho \to \chi_{\overline{\ball r}}$, and then dividing by $r^{n-1}$ we get
\begin{equation} \label{eqn:mf}
    \frac{d}{dr} \frac{1}{r^{n-1}} E_{\ve}(u, \ball r) = \frac{1}{r^{n-1}} \int_{\partial ^+ \ball r} (y\cdot \nabla u)^2 \mh + \frac{1}{t^n} \int_{D_r}\frac{W(u)}{\ve} \mh 
\end{equation}

Integrating from $r_1$ to $r_2$ we get (\ref{eqn:mf1})
\end{proof}

The following corollary is an easy consequence of the monotonicity formula (\ref{eqn:mf1}) that will be used frequently.

\begin{coro}
If $I_{\ve}(R,0) \leq \eta$, then for any $x \in D_{R/2}$ and for $0\leq r \leq R-|x|$,  $I_\ve (r,x) \leq 2^{n-1} \eta$
\end{coro}
\begin{proof}
  We have
\begin{equation}\label{eqn:mflemma}
    \begin{split}
                   I_\ve (r,x) &\leq I_\ve (R-|x|,x) \\
                    &\leq \bigg(\frac{1}{R-|x|}\bigg)^{n-1} E_{\ve}(u,\ball R) \\
                    &\leq \bigg(\frac{R}{R-|x|}\bigg)^{n-1} I_\ve (R,0) \leq 2^{n-1} \eta \quad \because |x|<R/2
    \end{split}
\end{equation}  
\end{proof}






\subsection{Convergence lemma}We now prove a convergence lemma that is central to the proof of our epsilon regularity result. It also provides context for the discussion in section 4.

\begin{lemm}\label{lem:conv}
    Let $\{\ui\}_{i \in \NN } \in C^2(B_1\cup D_1)$ be solutions of 
    \begin{equation} \label{eqn:conveqn}
        \begin{cases}
            \Delta \ui = 0 \qquad &\text{in } B_1^+ \\
            \pdv{\ui}{\nu} = \frac{1}{\ve_i}(\ui-\ui^3) \qquad  &\text{on  } D_1
        \end{cases}
    \end{equation}
satisfying
    \begin{equation} \label{eqn:bdry}
       |\ui| \leq 1, |\nabla \ui| \leq 2 \quad \text{in } B_1^+ \cup D_1 \quad \text{and }|\ui|\geq 1/2 \quad \text{on } D_1 
    \end{equation}
Then there is a function $u_{*}$ such that as $\ve_i \to 0$, $\ui \to u_{*}$ in $C^{1,\alpha}_{loc}(\ball 1 \cup D_1)$, and
\begin{equation}
        \begin{cases}
            \Delta u_{*} = 0 \qquad &\text{in } B_1^+ \\
            u_{*} = \pm 1 \qquad  &\text{on  } D_1
        \end{cases}
    \end{equation}
\end{lemm}

\begin{proof}
Note that due to the uniform bound on gradient and $\ui$, there is a weak limit $u_*$ of $\ui$ in $H^1(\ball 1)$. Since, $\ui$ are harmonic, so is $u_*$, and thus $\ui \to u$ in $C^{\infty}_{loc}(\ball 1)$. Further, note that using the uniform $C^0$ and $C^1$ estimates (\ref{eqn:bdry}), interpolation of Holder norms gives 
\begin{equation}\label{alphaconv}
    u_i \to u_* \quad \text{in $C^{0,\alpha}_{loc}(\ball 1 \cup D_1)$ for every $0<\alpha<1$} 
\end{equation} 
Combining the gradient bound with the boundary condition of equation (\ref{eqn:conveqn}), we get either $u_*=0$ or $|u_*|=  1$ on $D_1$, but since $|u_i| \geq 1/2$ on $D_1$ so $u_* \neq 0$. Further we just saw that $u_*$ is continuous up to the boundary, so $u_*=1$ or $u_*=-1$. Without loss of generality we will assume that $u_*=1$ on $D_1$. Now it just remains to upgrade the convergence from $C^{0,\alpha}_{loc}$ to $C^{1,\alpha}_{loc}$ for points on $D_1$. If we show that 
$$
\norm{\frac{1}{\ve_i}(\ui-\ui^3)}_{C^{0,\alpha}(D_{1/4})} \leq C_{\alpha}
$$
then by (\ref{csmest}), $\norm{u_k}_{C^{1,\alpha}({\ball {1/8} \cup D_{1/8}})} \leq C_{\alpha}$ and we will have the desired convergence. To prove the above, due to (\ref{eqn:bdry}) and $u_*=1$ we have $u_i\geq 1/2$ for $i$ large enough, therefore it is enough to show that
\begin{equation} \label{eqn:convlemmaest}
\norm{\frac{1}{\ve_i}(1-\ui)}_{C^{0,\alpha}(D_{1/4})} \leq C_{\alpha}
\end{equation}
That is, we need to show that
\begin{equation} \label{goal}
     |1-\ui(x)| + \frac{|u_i(x+y)-u_i(x)|}{|y|^\alpha} \leq C_\alpha \ve_i \quad \text{for every } x,y \in D_{1/4}
\end{equation}
In the rest of the proof we establish this. First note that the estimates in (\ref{eqn:bdry}) combined with the boundary condition (\ref{eqn:conveqn}) gives us the first term in (\ref{goal}) 
\begin{equation}\label{eqn:hest}
    0 \leq 1-\ui(x) \leq \frac{8}{3}\ve_i \quad \text{for all } x \in D_1
\end{equation}

Now fix a $y \in \{D_{1/4} \backslash \{0\} \}$, then for $x\in \overline{\ball {1/2}}$ we write $u_{i,y}(x)=u_i(x+y)$ and $g_{i,y}(x) = u_{i,y}(x)-u_i(x)$. To get the desired second term in (\ref{goal}) we need to show that  $$|g_{i,y}| \leq \tilde{C}_\alpha  |y|^{\alpha}\ve_i \quad \text{on } D_{1/4} $$
For this, we will construct a function $v_i$ such that on 
$$
    |v_i|<C\ve_i \text{ and } v_i \pm g_{i,y}/|y|^\alpha \geq 0 \quad \text{on }D_{1/4}
$$ 
First note that $g_{i,y}$ solves,
\begin{equation} \label{geqn}
    \begin{cases}
        \Delta g_{i,y} = 0 \qquad \text{in } \ball {1/2} \\
        \frac{\ve_i}{u_i(u_i+1)} \pdv{g_{i,y}}{\nu}+ g_{i,y} = \ve_i  f_{i,y} \quad \text{on } D_{1/2}
    \end{cases}
\end{equation}
Here 
\begin{align*}
    f_{i,y} &= \bigg [ \frac{1}{u_{i,y} (1+u_{i,y})}  -\frac{1}{u_i(1+u_i)} \bigg] \pdv{u_{i,y}}{\nu} \\
    &= (u_i-u_{i,y})\bigg  [ \frac{1+\ui+u_{i,y}}{\ui u_{i,y}(1+\ui)(1+u_{i,y})} \bigg] \pdv{u_{i,y}}{\nu}
\end{align*}  
Using (\ref{eqn:bdry}), we estimate $f_{i,y}$ on $D_{1/2}$. 
\begin{align*}
    |f_{i,y}| &\leq |u_{i,y}-u_i| \cdot \frac{16}{3} \bigg|\pdv{u_{i,y}}{\nu} \bigg| \quad \because 1/2 \leq u_i \leq 1 \text{ on } D_1  \\
    &\leq |u_{i,y}-u_i|  \frac{32}{3} \qquad \because |\nabla u_i| \leq 2 \\
    &\leq C_\alpha |y|^{\alpha} \qquad \text{by } (\ref{alphaconv})
\end{align*}
Therefore,
\begin{equation}\label{eqn:gfest}
    \norm{f_{i,y}}_{L^{\infty}(D_{1/2})} < C_{\alpha}|y|^\alpha  
\end{equation}
Further, again by (\ref{eqn:bdry}), we have the following,
\begin{align} \label{supest}
    \norm{g_{i,y}}_{L^{\infty}(\overline{\ball {1/2}})} =C_y &\leq 2|y| \quad \because |\nabla u_i| \leq 2  \\
    &\leq C'_{\alpha}|y|^{\alpha} \text{ for some }C'_\alpha \because |y|<1/4 \label{cprimeeqn}
\end{align}
Now, we construct the function $\vi$ mentioned above. Consider the solution of the following mixed boundary value problem on $\ball {1/2}$
\begin{equation}\label{veqn}
    \begin{cases}
        \Delta \vi =0 \quad \text{in } \ball {1/2} \\
        \vi = 1  \quad \text{on } \partial^{+}\ball {1/2} \\
        \frac{4\ve_i}{3} \pdv{\vi}{\nu} + \vi = 0 \quad \text{in } D_{1/2} 
    \end{cases}   
\end{equation}
Clearly $\vi \leq 1$. However, $\vi \geq 0$ as well because if not then the point of negative minimum $x \in D_{1/2}$ but then Hopf boundary point lemma at $x$ gives $\ve_i \pdv{\vi}{\nu} + \vi <0$, which contradicts the boundary condition on $D_{1/2}$. So $|\vi|\leq 1$ and by lemma 6.26 in \cite{gt}, we have $$|\nabla \vi| \leq C \quad \text{in } \overline{\ball {1/4}}  $$
Here $C$ is a dimensional constant. Combining this with the boundary condition on $D_{1/2}$ for $\vi$ we get 
\begin{equation}\label{veboundary}
    |\vi| \leq C\ve_i \quad \text{on } D_{1/4}
\end{equation}
Now we have all the required estimates. Fix $\alpha \in (0,1)$, and recall that $y\in D_{1/4} \backslash \{0\}$ is fixed. Recall the definitions of $C_\alpha$, $C'_\alpha$ and $C_y$ from (\ref{eqn:gfest}) and (\ref{supest}). Now, consider the functions $$w_{i,\alpha,y}^{\pm} = C_y \vi   \pm g_{i,y} + \ve_i C_\alpha|y|^{\alpha}$$ 
We claim that $w_{i,\alpha,y}^{\pm}$ satisfy 
\begin{equation} \label{weqn}
    \begin{cases}
        \Delta w_{i,\alpha,y}^{\pm}=0\quad  \text{in } \ball {1/2} \\
        w_{i,\alpha,y}^{\pm} \geq \ve_i C_\alpha|y|^{\alpha} > 0 \quad \text{on } \partial^{+}\ball {1/2} \\
        \frac{\ve_i}{u_i(1+u_i)} \pdv{w_{i,\alpha,y}^{\pm}}{\nu} + w_{i,\alpha,y}^{\pm}  \geq 0 \quad \text{in } D_{1/2}
    \end{cases}
\end{equation}
First one is clear from definition as both $v_i$ and $g_{i,y}$ are harmonic and $\ve_i C_\alpha|y|^\alpha$ is constant as $y$ is fixed. For the second one we have 
\begin{align*}
    w_{i,\alpha,y}^{\pm} &= C_y v_i \pm g_{i,y} +\ve_i C_\alpha|y|^{\alpha} \\
    &\geq (C_y - |g_{i,y}|\,) +\ve_i C_\alpha|y|^{\alpha}  \quad \text{by (\ref{veqn}) on }\partial^+ \ball {1/2} \\
    &\geq \ve_i C_\alpha|y|^{\alpha} > 0 \quad \text{ by }(\ref{supest})
\end{align*}
For third  one we first note that on $D_{1/2}$ we have
\begin{align}
    \frac{\ve_i}{\ui(\ui+1)}\pdv{v_i}{\nu} +\vi
    &=\frac{4\ve_i}{3} \pdv{\vi}{\nu} + \bigg(\frac{1}{\ui(\ui+1)} - \frac{4}{3} \bigg)\ve_i \pdv{\vi}{\nu}  +\vi   \\
    &\geq \frac{4\ve_i}{3}  \pdv{v_i}{\nu} +\vi =0 \text{ by (\ref{veqn})} \label{vis0}
\end{align}
The inequality holds because the term in bracket is non-positive as $1/2\leq u_i \leq 1$ (\ref{eqn:bdry}), and because $\pdv{\vi}{\nu}\leq 0$ by (\ref{veqn}) as $\vi \geq0$ on $D_{1/2}.$ Hence, we have
\begin{align*}
    \frac{\ve_i}{\ui(1+\ui)} \pdv{w_{i,\alpha,y}^{\pm}}{\nu} + w_{i,\alpha,y}^{\pm} &\geq \pm \bigg(\frac{\ve_i}{\ui(\ui+1)} \pdv{g_{i,y}}{\nu}+ g_{i,y} \bigg)  + \ve_i C_\alpha|y|^{\alpha} \text{ by (\ref{vis0})}\\
    &=  \pm f_{i,y} + \ve_i C_\alpha|y|^{\alpha} \text{ by (\ref{geqn}) on } D_{1/2} \\ 
    &=\ve_i \big[C_\alpha|y|^{\alpha} \pm f_{i,y} \big] \quad  \\
    &\geq \ve_i \big[C_\alpha|y|^{\alpha}-C_\alpha |y|^\alpha \big] \quad \text{ by (\ref{eqn:gfest})} \\
    &\geq 0    
\end{align*}
    
Therefore $w_{i,\alpha,y}^{\pm}$ satisfies (\ref{weqn}). Observe that if $w_{i,\alpha,y}^{\pm} < 0$, then there is a point of negative minimum $x \in D_{1/2}$ as $w^{\pm}_{i,\alpha,y}\geq 0$ on $\partial^+ \ball {1/2}$. By Hopf boundary point lemma at $x$, this contradicts the boundary condition on $D_{1/2}$. So, $w_{i,\alpha,y}^{\pm} \geq 0$ in $\overline{\ball {1/2}}$ and therefore we have
\begin{align*}
    |g_{i,y}(x)| &\leq C_y \vi(x) + \ve_iC_{\alpha}|y|^{\alpha}  \quad \text{for } x\in \overline{\ball {1/2}} \\
    &\leq \ve_i|y|^{\alpha} \big[ C'_\alpha C + C_\alpha \big] \quad \text{for } x\in D_{1/4} \text{ by (\ref{cprimeeqn}) and (\ref{veboundary}) }
\end{align*}
Combining with estimates  (\ref{eqn:hest}),(\ref{supest}), and (\ref{veboundary}), we get
\begin{equation*}
    \frac{1}{\ve_i}\bigg|1-u_i(x)\bigg| + \frac{1}{\ve_i}\bigg|\frac{u_i(x+y)-u_i(x)}{|y|^\alpha} \bigg| \leq \frac{8}{3} + CC'_\alpha + C_\alpha \quad \text{for every } x,y \in D_{1/4}
\end{equation*}
This is the required estimate (\ref{eqn:convlemmaest}).
\end{proof}


