\section{Introduction}


\subsection{Minimal surfaces and phase transitions} Minimal hypersurfaces are the critical points of the area functional. The relationship between minimal hypersurfaces and interfaces formed by double well phase transitions has been a subject of many works over the past few decades \cite{g,gg,ht,m,st,tw}. The latter are the critical points of the Allen-Cahn energy functional
\begin{equation} \label{ace}
    \frac{1}{2}\int_U |\nabla u|^2 + \frac{1}{\ve ^2} W(u) \,dx
\end{equation}
whose Euler-Lagrange equation is the elliptic Allen-Cahn equation 
\begin{equation} \label{ac}
    -\Delta u_{\ve} = -\frac{1}{\ve ^2} W'(u_{\ve})  \quad \text{in } U 
\end{equation}
here $u$ is the density, $U$ is a bounded domain, $W: \RR \to [0,\infty)$ is a double well potential with two minima $1$ and $-1$, which are the densities of the stable phases. In this paper $W(t)=\frac{(1-t^2)^2}{4}$. The potential term $W$ penalizes densities away from $\pm 1$, and the Sobolev energy term, $|\nabla u|^2$ penalizes oscillations, so a critical point usually has two regions where $u$ is $1$ and $-1$ respectively with a diffuse phase transition interface which can be seen as an $\ve$-neighbourhood of $u_{\ve}^{-1}(0)$-the zero level set. As $\ve \to 0$, it was expected that the phase transition interface will converge to a minimal hypersurface \cite{i}. We describe some of the results in this direction.


If $u_\ve$ are minimizers then as $\ve \to 0$, after passing to a subsequence, $u_\ve \to \chi_A - \chi_{A^c}$. Here the boundary $\partial A = \Sigma$ is an area minimizing hypersurface. This was shown in \cite{mm} by Modica and Mortola. Further, in \cite{cc}, Caffarelli and Cordoba proved uniform convergence of the level sets 
    $\{|\ue| \leq \delta\}$ to the minimal surface.

For critical points with bounded energy, but without any stability or minimizing condition, Hutchinson and Tonegawa \cite{ht} proved that as $\ve \to 0$ the energy measures converge to a stationary integer rectifiable $(n-1)$-varifold, i.e a generalized minimal hypersurface. Later in \cite{tw}, Tonegawa and Wickramesekara proved that if $\{\ue\}$ are stable critical points, then this limiting varifold is a stable minimal hypersurface. 


This relationship between minimal surfaces and the Allen-Cahn equation has been used in the last decade in several advances in the theory minimal surfaces. We mention two: in \cite{g}, building on the works \cite{ht,tw} Guaraco gave a min-max construction of minimal surfaces based on Allen-Cahn equation as an alternative to the Almgren-Pitts theory \cite{p}. Chodosh and Mantoulidis proved the multiplicity one conjecture for minimal surfaces in $3$-manifolds in \cite{cm}.  

\subsection{Boundary reactions and nonlocal phase transitions} A natural variant of energy (\ref{ac}) analysed by Alberti, Bouchitté and Seppecher in \cite{abs}, and then extensively studied by Cabré and Solà-Morales in \cite{csm}, has the potential $W$ placed on the boundary of the euclidean half space, i.e. they consider the energy
\begin{equation} \label{hace}
   E_{\ve}(u)= \frac{1}{2}\int_{\RR^{n+1}_+} |\nabla u|^2 \,dx + \frac{1}{\ve}\int_{\partial \RR^{n+1}_+}  W(u) \mh
\end{equation}
The critical points of (\ref{hace}) satisfy the nonlinear Neumann problem 
 \begin{equation} \label{ext}
        \begin{cases}
            \Delta \ue = 0 \qquad &\text{in } \RR^{n+1}_+ \\
            \pdv{\ue}{\nu} = -\frac{1}{\ve}W'(u_\ve) \qquad  &\text{on  } \partial \RR^{n+1}_+ 
        \end{cases}
    \end{equation}
This equation appears in crystal dislocation \cite{gm,to}. It also occurs in analysis of vortices for soft thin films \cite{k}. This problem also has a nonlocal formulation on $\RR^n \cong \partial \RR^{n+1}_+$ as a fractional Allen-Cahn equation \cite{cs} 
\begin{equation}\label{hac}
    (-\Delta)^{1/2} u_{\ve} = -\frac{1}{\ve} W'(u_{\ve})  \quad \text{in } \RR^n
\end{equation}
where 
$$(-\Delta)^{1/2}u(x) = 2 {\, \rm p.v.} \int_{\RR^n} \frac{u(x)-u(y)}{|x-y|^{n+1}} \,dy$$

Seen this way, the problem appears when the effects of long-range interactions in phase transitions are studied through a nonlocal energy \cite{msw, sv,sv1,svp}. In energy functional (\ref{ace}) the $H^1$ Sobolev energy term $\int |\nabla u|^2 \,dx=\norm{u}_{H^1}^2$ is replaced by contribution from $U$ in the nonlocal energy $\norm{u}_{H^s}^2$ for $s\in(0,1)$, giving the energy functional 
\begin{equation*}
    \norm{u}_{H^s(U)}^2  + \frac{1}{\ve ^{2s}}\int_U  W(u) \,dx
\end{equation*}
The $H^s$ energy of $u$ is 
$$\norm{u}_{H^s}^2 =\int \int _{\RR^d \times \RR^d} \frac{|u(x)-u(y)|^2}{|x-y|^{n+2s}} \,dx\,dy  $$
and therefore the appropriate energy functional with $H^s$, instead of $H^1$ energy in (\ref{ace}),  is 
\begin{equation} \label{nace}
    \frac{1}{2}\int \int _{\RR^d \times \RR^d \backslash U^c \times U^c} \frac{|u(x)-u(y)|^2}{|x-y|^{n+2s}} \,dx\,dy  + \frac{1}{\ve ^{2s}}\int_U  W(u) \,dx
\end{equation}
 Note that here the boundary value of $u$ will be fixed in complement of $U$ and $U^c=\RR^n \backslash U$. The Euler-Lagrange equation is the fractional Allen-Cahn equation, i.e, equation (\ref{ac}) with the laplacian $-\Delta$ replaced by the fractional laplacian $(-\Delta)^s$, for $s\in (0,1)$
\begin{equation}\label{nacs}
    (-\Delta)^{s} u_{\ve} = -\frac{1}{\ve ^{2s}} W'(u_{\ve})  \quad \text{in } U, \quad s\in(0,1)
\end{equation}
with some boundary data in $U^c$. Thus one sees that the equation (\ref{hac}) is just the $s=1/2$ case in above, where 
$$(-\Delta)^{s}u(x)= c_{n,s}\,{\rm p.v.} \int_{\RR^n} \frac{u(x)-u(y)}{|x-y|^{n+2s}} \quad $$

\subsection{Minimal surfaces and boundary reactions} Given the geometric properties of the interface formed by the Allen-Cahn equation as highlighted above, it is of interest to understand the limit interface formed for nonlocal phase transitions as $\ve \to 0$. Here, there are two different outcomes depending on whether $s\in(0,1/2)$ or $s\in [1/2,1).$ The dichotomy is primarily there because the characteristic function of smooth sets are not in $H^s$ when $s\in[1/2,1)$. Whereas for $s\in (0,1/2)$ this is not the case, so as $\ve \to 0$, the $H^s$ energy remains uniformly bounded. 


In case of minimizers, Savin and Valdinoci \cite{sv} proved that as $\ve \to 0$, after passing to a subsequence, $u_\ve \to \chi_A - \chi_{A^c}$. If $s\in(0,1/2)$, then the boundary of the set $A$, $\partial A = \Sigma$ is a minimizing  $H^s$-nonlocal minimal hypersurface, i.e. boundaries of sets minimizing the $H^s$ energy. They were first studied by Caffarelli, Roquejoffre and Savin in \cite{crs}.
    
Interestingly if $s\in[1/2,1)$, the limit object is same as in the classical case, i.e, the boundary of the set $A$, $\partial A = \Sigma$ is an area minimizing hypersurface. Analogous to \cite{cc} in the local case, Savin and Valdinoci also prove in \cite{sv1} the uniform convergence of level sets of $\ue$ to the non-local $(s<1/2)$  or local minimal surface $(s\geq1/2)$.
    
For critical points with bounded energy, but without the minimizing condition, only the case with $s\in (0,1/2)$ is known. Millot, Sire and Wang \cite{msw} proved that that the energies converge to a  stationary $H^s$ nonlocal minimal surface. 

One might hope that for $s \in [1/2,1)$, the energy measures of critical points with bounded energy will converge as in the classical case \cite{ht} to a stationary integer rectifiable $(n-1)$-varifold. However a monotonicity formula for the nonlocal energy (\ref{nace}) is known only for $s\in(0,1/2)$. See also the discussion in \cite{csv}. Still for $s=1/2$ we are able to show the convergence to a stationary rectifiable $(n-1)$ varifold. For this we view $u_{\ve}$ as a solution of the non-linear Neumann problem (\ref{ext}) instead of viewing $u_\ve$ as solutions to (\ref{hac}). We state the result below precisely for the half ball $\ball 1 = B_1(0) \cap \RR^{n+1}_+$, and will later specify the exact definition of admissible open sets in $\RR^{n+1}_+$.
\begin{theo} \label{maintheorem}
Let $\{u_\ve \}$ be critical points for the energy functional $E_\ve$ (\ref{hace}), in $\ball 1$ satisfying a uniform energy bound $E_{\ve}(u_\ve) \leq E_o < \infty$.  Then as $\ve \to 0$, there is a naturally associated stationary rectifiable $(n-1)$-varifold $V$, supported on the energy concentration set $\Sigma$ that forms on the boundary $\ball 1 \cap \{x_{n+1}=0 \}$.
\end{theo}

We also remark here that a major motivation for our work came from the very interesting work of Figalli and Serra \cite{fs}, in which they proved that every bounded stable solution of the equation  (\ref{hac}) in $\RR^3$ is one dimensional. Theorem \ref{maintheorem} is the first step in the direction of applying the results of \cite{fs} to the study of minimal hypersurfaces. 

We plan to address the integrality of the limiting varifold and the counterpart of \cite{tw} in this setting in future works.


\subsection{Organization} 
  
  In section $2$, we establish two preliminary results. First, the monotonicity formula for the energy functional (\ref{hace}); to the best of our knowledge it was first given in \cite{ms}. Here we provide a proof based on the proof of \cite{ht} in the classical case. The other result we prove is a technical convergence lemma for solutions $\{ \ue \}$ under a uniform gradient bound.

  In section $3$ we study the behavior of equation (\ref{ext}) in a domain $U$, in the small energy regime. Here we first prove a clearing-out type result on the boundary. Further, we establish an epsilon regularity result for our equation. The epsilon regularity used in conjunction with the convergence lemma is a key tool for many of our proofs. 

  In section $4$, we use the results proved in the previous two sections to construct the energy concentration set $\Sigma$, and prove that it has Hausdorff dimension  $(n-1)$. Further we show that $\Sigma$ is the obstruction to lack of compactness of $\{ \ue\}$ in $H^1$. This is characterised by a defect measure $\mu_{\Sigma}$ supported in the concentration set. We also show that $\mu_{\Sigma}$ is absolutely continuous with respect to $\mathcal{H}^{n-1}\measurer \Sigma$.

  Finally in section $5$, we briefly recall the formalism of varifolds and generalized varifolds. Then we show that the stress-energy tensors associated to $\ue$ are naturally seen as generalized varifolds $V_{\ve}$ that converge as $\ve \to 0$ to a limiting generalized varifold $V$. Further, there is a stationary rectifiable $(n-1)$-varifold $V_{\Sigma}$ that is supported on $\Sigma$ that is naturally associated to $V$. 
  

\subsection*{Notation of sets and boundaries}
\begin{itemize}
    \item  $\RR^n$ is identified with $\partial \RR^{n+1}_+ = \RR^n \times \{0\}$.
    \item  $B_r(x)$ is the open ball in  $\RR^{n+1}$ centered at $x$. If the center is not specified then it is the origin.
    \item $\ball r (x) = B_r(x) \cap \RR^{n+1}_+ $ is a half ball with center $x \in \RR^n$.
    \item $D_r(x)= B_r(x) \cap \RR^{n}$ is a disc on the boundary.
    \item For a set $U$, $\quad U^+=U \cap \RR^{n+1}_+$, $\quad \partial^+ U=\partial  U \cap \RR^{n+1}_+$.
    \item $\partial^0U = \{ x\in U \cap \RR^n \, : \ball r (x) \subset U \text{ for some } r>0 \}$. For example $\partial^0 \ball r (x) = D_r(x)$.
    \item An admissible open set $U$ is a bounded open set in $\RR^{n+1}_+$ such that $\partial U$ is Lipschitz, $\partial^0 U$ is non empty and has a Lipschitz boundary, and $\partial U= \partial^+U \cup \overline{\partial^0U}$
\end{itemize}

 

\subsection*{Acknowledgement} The author is immensely grateful to his advisor Professor Yannick Sire for his constant encouragement and invaluable guidance. He also wishes to thank Junfu Yao, Yifu Zhou and Jonah Duncan for several helpful discussions.























