\section{Small energy regime}
In this section we prove two results under a small energy assumption. The first one is a clearing out result on the boundary. The second is the epsilon regularity result. We emphasize that these are proved for $\ue$, with estimates that are uniform in $\ve$. This is crucial for their application in the study of $\ue$ as $\ve \to 0$.

\subsection{Clearing out on the boundary} This is an important consequence of the monotonicity formula. It captures the intuition that if the energy is small enough then $\ue$ stays close to the potential wells uniformly with respect to $\ve$.
\begin{lemm} \label{lemmaclearing}
Let $\ue$ be a solution of (\ref{eqn:main}) for $R=1$ such that $|\ue| \leq 1$. There is a constant $\eta$ independent of $\ve$ such that such that $E_\ve(\ue,\ball 1)\leq \eta$ implies $|\ue| \geq \frac{1}{2}$ on $D_{1/2}$.
\end{lemm}
\begin{proof}


We first consider solution $u$ for $\ve =1$ and prove the claim by contradiction. Then we will prove it for any $\ve<1$ by rescaling. As $|u| \leq 1$, we have by (\ref{csmest}) that $\norm{u}_{C^{2,\alpha}(\ball {1/2})} \leq C_{\alpha}$. Let $\eta_1=2^{n-1}\eta$ such that $E_\ve(\ue,\ball 1)\leq \eta_1$. If the result is not true then we can find a sequence of solutions $\ui = u_{\ve_i}$ of decreasing energy and points $x_i \in D_{1/2}$ such that $|\ui(x_i)|\leq 1/2$ and $E_1(u_i,\ball 1) \to 0$. Due to the uniform estimate we have uniform convergence of $u_i$ in $\overline{\ball {1/2}}$. As $E_1(u_i,\ball 1) \to 0$, $u_i \to 1$ on $D_{1/2}$ which contradicts $|\ui(x_i)|\leq 1/2$. This proves the case $\ve=1$. 

Now for any $\epsilon<1$, for any solution $\ue$ and any point $x_0 \in D_{1/2}$,  consider the map $x \to x_0 + \ve x$ sending $\ball 1 \to \ball \ve (x_0)$. Then $v_{\Tilde{\ve}}(x)=\ue(x_0+\ve x)$ satisfies (\ref{eqn:main}) for $\ve=1$ and $R=1$. Further note that $E_1(v_{\Tilde{\ve}}, \ball 1)= I_{\ve}(\ve,x_0) \leq 2^{n-1} \eta=\eta_1$, the inequality is by (\ref{eqn:mflemma}). So we can apply the $\ve=1$ result which gives $|v_{\Tilde{\ve}}| \geq 1/2$ on $D_{1/2}$, but this is same as $|\ue| \geq 1/2$ on $D_{\frac{\ve}{2}}(x_0)$ for any $x_0 \in D_{1/2}$.
\end{proof}


\subsection{Epsilon regularity} We now prove the epsilon regularity result. It will become clear soon that it should be thought of as stating that if the energy bound is small enough, then the gradient of $\ue$ is bounded uniformly independent of $\ve$ \textit{up to the boundary.} 


\begin{theo}  There exist constants $\eta_0$ and $C_0$ independent of $\ve$ such that for $\ve <R$, and $u_\ve \in C^2(\ball R \cup D_R)$ satisfying $|u_\ve| \leq 1$ solving 

    \begin{equation} \label{eqn:Rscale}
        \begin{cases}
            \Delta \ue = 0 \qquad &\text{in } B_R^+ \\
            \pdv{\ue}{\nu} = -\frac{1}{\ve}W'(u_\ve) \qquad  &\text{on  } D_R
        \end{cases}
    \end{equation}
If we have $I_{\ve}(R,0) \leq \eta_0$, then 
    \begin{equation} \label{epsreg}
        \sup_{\ball {\frac{R}{4}}} |\nabla \ue|^2 + \sup_{D_{\frac{R}{4}}} \frac{W(\ue)}{\ve ^2} \leq \frac{C_0}{R^2}\eta_0
    \end{equation}
\end{theo}
\begin{rem}
We use an idea due to Schoen \cite{s} in harmonic maps setting, also used in similar geometric problems by several authors \cite{cs,cw,ms,t,w}. To obtain an estimate independent of $\ve$ we need a scale $r_{\ve}$ for which the gradients are uniformly bounded. Then the problem reduces to having a uniform bound on $r_{\ve}$. In \cite{s} this is done by using the mean value property to get a contradiction to the smallness of energy. However, due to the boundary we may only use the mean value property for $|\nabla \ue|^2$ only for points sufficiently away from the boundary. In the other case, we will rescale the problem to $r_{\ve}$-scale and then use the convergence lemma \ref{lem:conv} to get a contradiction to smallness of energy
\end{rem}

\begin{proof}
It is sufficient to prove the result for $R=1$. First observe that we have 
$$\frac{W(u)}{\ve^2}=\frac{1}{4u^2}\frac{W'(u)^2}{\ve^2} = \frac{1}{4u^2}\bigg|\pdv{\ue}{\nu}\bigg|^2 $$
We may assume that $I_\ve \leq \eta$, then by clearing out result (\ref{lemmaclearing}), $1/2 \leq|u_\ve|\leq 1$ on $D_{1/2}$ and therefore
\begin{equation*}
    \frac{W(u)}{\ve^2}= \frac{1}{4u^2}\bigg|\pdv{\ue}{\nu}\bigg|^2  \leq \bigg|\pdv{\ue}{\nu}\bigg|^2
\end{equation*}

Therefore to establish (\ref{epsreg}), it is enough estimate the gradient $|\nabla \ue|$ upto the boundary, i.e. we need to show
\begin{equation} \label{gradest}
    \sup_{\overline{\ball {\frac{1}{4}}}} |\nabla \ue| \leq C\sqrt{\eta_0} 
\end{equation}

Consider the distance weighted gradient on $\overline{\ball {\frac{1}{2}}}$,  $F(s)=(\frac{1}{2}-s)|\nabla \ue(x)|$. It attains its maximum for some $s_{\ve} \in [0,1/2]$ and we have 
$$\max _{s} F(s) = \max_{s} \bigg(\frac{1}{2}-s \bigg) \sup_{\overline{\ball s}}|\nabla \ue| =\bigg(\frac{1}{2}-s_{\ve}\bigg) \sup_{\overline{\ball {s_{\ve}}}} |\nabla \ue| $$

Let $x_\ve$ be such that $\sup_{\overline{\ball {s_{\ve}}}} |\nabla \ue|=|\nabla \ue(x_\ve)| =e_\ve$. Note that due to the definition of $F(s)$, $|x_{\ve}|=s_{\ve}$, so $\dist{(x_\ve, \partial^+ \ball \frac{1}{2})}= \frac{1}{2}-s_\ve$. We write $$ \frac{1}{2}-s_\ve = 2\rho_\ve \quad \text{and} \quad r_\ve = \rho_\ve e_\ve$$

Then we have $\max_{s} F(s) =2\rho_\ve  e_\ve = 2r_\ve $. Taking $s=\frac{1}{4}$ gives
\begin{equation} \label{eqn:grad1}
    \sup_{\overline{\ball {\frac{1}{4}}}} |\nabla \ue| \leq 8 r_\ve
\end{equation}
Therefore the gradient estimate would follow from a uniform bound on $r_\ve$. We collect another consequence of definition of $F$ that will be used later.
\begin{equation} \label{eqn:gs}
    |\nabla \ue|(x) \leq \frac{(\frac{1}{2}-s_\ve)}{(\frac{1}{2}+s_\ve)}2 e_\ve < 2e_\ve \quad \text{for all } x \in B_{\rho_\varepsilon}(x_\ve) \cap \overline{\ball {\frac{1}{2}}} 
\end{equation}

%Now we first deal with the case when $x_\ve$ is sufficiently away from $D_1$. 
Let $\overline{x_\ve}$ be projection of $x_\ve$ on $D_1$. Then denote by $z_\ve$ the height of $x_\ve$, i.e $z_\ve = x_\ve - \overline{x}_{\ve}$. Depending on how the height of $x_\ve$ compares to it is distance from the spherical boundary we get the above described two cases.


%\begin{enumerate}
    %\item 
      \textit{Case 1:}   $ \frac{z_\ve}{2\rho_\ve} > \frac{1}{4}$ i.e., away from  $D_1$\\
           The ball $B_{\frac{z_\ve}{2}}(x_\ve) \subset \ball {2z_\ve}(\overline{x}_\ve) \subset \ball 1 $, then by mean value property for $|\nabla u_\ve|^2$ we have 
        \begin{equation} \label{eqn:case1}
            e_\ve ^2 \leq \frac{1}{|B_{\frac{z_\ve}{2}}(x_\ve)|}\int_{B_{\frac{z_\ve}{2}}(x_\ve)} |\nabla u_\ve|^2 \,dx 
                    \leq \frac{1}{4z_{\ve}^2}\cdot C\eta_0  
        \end{equation}
        The second inequality follows from (\ref{eqn:mflemma}). Combining this with (\ref{eqn:grad1}) gives the desired estimate
        $$ \sup_{\overline{\ball {\frac{1}{4}}}} |\nabla \ue| \leq 8 r_\ve < 8\cdot  2z_\ve e_\ve \leq C\sqrt{\eta_0} $$   
   % \item 


\textit{Case 2: }$ \frac{z_\ve}{2\rho_\ve} \leq \frac{1}{4}$ i.e., close to $D_1$\\
            Given (\ref{eqn:grad1}), if $r_\ve \leq 1$, then we are done. So we assume otherwise, i.e $r_\ve > 1$ and arrive at a contradiction. For this, we will consider the problem at the $r_\ve$-scale. Consider $\ue$ solving (\ref{eqn:Rscale}) in the ball $\ball {\rho_{\ve}}(\overline{x}_\ve) \subset \ball {\frac{1}{2}} $. Then for $x \in \ball {r_\ve} \cup D_{r_\ve}$, with $\Tilde{\ve} = \ve e_\ve$, take
            $v_{\Tilde{\ve}}(x) = \ue (\overline{x}_\ve + x/e_\ve) $. With this rescaling $\ball {\rho_\ve}(\overline{x}_\ve)$ goes to $\ball {r_\ve}$. As $r_\ve > 1$, all $v_{\Tilde{\ve}}$ solve
            \begin{equation} \label{eqn:rscale}
        \begin{cases}
            \Delta v_{\Tilde{\ve}} = 0 \qquad &\text{in } \ball {1} \\
            \pdv{v_{\Tilde{\ve}}}{\nu} = -\frac{1}{\Tilde{\ve}}W'(v_{\Tilde{\ve}}) \qquad  &\text{on  } D_{1}
        \end{cases}
    \end{equation}
           Further due to our assumptions, and (\ref{eqn:gs}) we have
        \begin{equation} \label{eqn:convestimates}
            |\nabla v_{\Tilde{\ve}} (y_\ve)|=1 ,|\nabla v_{\Tilde{\ve}}| \leq 2,|v_{\Tilde{\ve}}| \leq 1 \quad \text{in } B_1^+ \cup D_1 \text{and } |v_{\Tilde{\ve}}|\geq 1/2 \quad \text{on } D_1 
        \end{equation}
        further using (\ref{eqn:mflemma}) we also get
        \begin{equation}\label{eqn:energyconv}
            E_{\Tilde{\ve}}(v_{\Tilde{\ve}}, B_1)<E_{\Tilde{\ve}}(v_{\Tilde{\ve}}, B_{r_\ve})=I_\ve(\rho_\ve,\overline{x}_\ve) \leq 2^{n-1} \eta_0
        \end{equation}
            Here $y_\ve = e_\ve(x_\ve-\overline{x}_\ve)$, so $|y_\ve|= z_\ve e_\ve$. Note that $z_\ve \leq \frac{1}{2}\rho_\ve$ is equivalent to $z_\ve e_\ve \leq r_\ve/2$ but we need $y_\ve \in \ball {1/2}$, indeed this is the case i.e.   $z_\ve e_\ve \leq 1/2$ as if we have $z_\ve > \frac{1}{2e_\ve}$ then exactly like (\ref{eqn:case1}) we have
            $$ e_\ve^2 \leq \frac{1}{4z_\ve^2}\cdot C\eta_0 \leq e_\ve^2\cdot C\eta_0$$ this gives $1 \leq C\eta_0 $
            which is false for $\eta_0$ small enough. Therefore, $y_\ve \in \ball {1/2}$. To simplify notation we write, $v_i=v_{\Tilde{\ve_i}}$ and $y_i=y_{\ve_{i}}$. 
            
            We will show that as $i \to \infty$, we have $v_i \to v_*$ $C^{1,\alpha}_{loc}(\ball 1 \cup D_1)$, then it will give us
            
            \begin{equation}\label{assump}
                \begin{cases}
                 \nabla v_i(y_i) \to \nabla v_*(y_*) \text{ thus } |\nabla v_*(y_*)|=1 \text{ as } |\nabla v_i(y_i)|=1  \\
               \text{There is a }\sigma<\frac{1}{10} \text{ such that }|\nabla v_*|  \geq 1/2 \text{ on }B_{\frac{1}{10}}(y_*)\cap \ball 1 
            \end{cases}
            \end{equation}
    This gives us the following. The last inequality is due to the estimate (\ref{eqn:energyconv}).
        \begin{equation}
            \frac{\sigma^{n+1}|\ball 1|}{2} \leq \int_{\ball 1}|\nabla v_*|^2\,dx \leq \liminf_{i\to \infty} \int_{\ball 1} |\nabla v_i|^2 \,dx \leq 2^{n-1} \eta_0
        \end{equation}    

This leads to a contradiction for small enough $\eta_0$. Therefore $r_\ve <1$ as desired. 



We now just need to show that $v_i \to v_*$ in $C^{1,\alpha}_{loc}(\ball 1 \cup D_1)$.  Recall that $\Tilde{\ve_i}=\ve_i e_{\ve_i}$. We have two cases: As $\ve_i \to 0$ we also have $\Tilde{\ve}_i \to 0$ Then because of the estimates (\ref{eqn:convestimates}), we can apply lemma (\ref{lem:conv}) to $\{v_i\}$.This gives uniform $C^{1,\alpha}_{loc}$ convergence $v_i$ to $v_*$ up to the boundary as required. However, if as $\ve_i \to 0$, $\Tilde{\ve}_i \nrightarrow 0$, that is $\Tilde{\ve}_i= e_{\ve_i} \ve_i \geq \beta >0$. Then by (\ref{csmest}) we have uniform $C^{2,\gamma}$ estimate up to the boundary, $\norm{v_i}_{C^{2,\gamma}_{loc}(\overline{\ball {3/4}}) } \leq C_{\gamma}$. By Holder interpolation, this gives uniform $C^{1,\alpha}_{loc}$ convergence $v_i$ to $v_*$ up to the boundary in this case as well. This finishes the proof.
\end{proof}

With the monotonicity formula, convergence lemma and the epsilon regularity result we can now study the behavior of $\ue$ and the associated energy as $\ve \to 0$. 