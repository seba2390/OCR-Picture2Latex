\section{The energy concentration set and its properties}
In this section we introduce the concentration set of the energy and prove several results about it using the tools developed in the last two sections. Throughout this section, we will have the following assumptions unless otherwise stated. $\{u_i\}_{i\in \NN} \subset H^1(U)$ are the critical points of $E_{\ve_i}$, satisfying

 \begin{equation} \label{sec4eqn}
        \begin{cases}
            \Delta u_i = 0 \qquad &\text{in } U \\
            \pdv{u_i}{\nu} = -\frac{1}{\ve_i}W'(\ue) \qquad  &\text{on  } \partial^0 U \\
            |u_i|\leq 1$, $E_{\ve_i}(u_i) \leq E_0
        \end{cases}
\end{equation}


\subsection{Limiting energy measure and its density}Consider the energy measures on $U \cup \partial^0U$,
$$ \mu_i = \frac{1}{2}|\nabla u_i|^2 \,dx + \frac{1}{\ve_i}W(u_i) \,\mh $$
Because $\mu_i(U \cup \partial^0U) \leq E_0$, after passing to a subsequence there exists a Radon measure $\mu$ on $U\cup \partial^0U$, such that $ \mu_i \rightharpoonup \mu $. For any $x \in \partial^0U$, and $0<s<r<\dist(x,\partial^+U)$, using the monotonicity formula (\ref{eqn:mf1}), with $i \uparrow \infty$ gives 
$$ \frac{1}{s^{n-1}} \mu(\ball s (x)) \leq \frac{1}{r^{n-1}} \mu(\ball r (x)) < \frac{E_0}{r^{n-1}} $$
Hence, the $(n-1)$ density of the measure $\mu$ is well-defined and finite 
\begin{equation}
    \Theta^{n-1}(\mu,x) = \lim_{r \downarrow 0} \frac{\mu(\ball r (x))}{\omega_{n-1}r^{n-1}}
\end{equation}
Here, $\omega_{n-1}$ is the volume of $(n-1)$ dimensional unit ball. We will write $\omega_{n-1}$ is $1$ to simplify notation. We first observe that the density is zero for points in the interior. 

\begin{prop} \label{lem0}
 If $x \in U$ then $\Theta^{n-1}(\mu,x)=0$.
\end{prop} 
\begin{proof}
    First note that the uniform energy bound and $|u_i| \leq 1$, gives us a weak limit $u_* \in H^1(U)$ for $\{u_i\}$. After passing to a subsequence $|\nabla u_i|^2 \to |\nabla u_*|^2$ pointwise almost everywhere so by Fatou's lemma we have,
    $$\int_U |\nabla u_*|^2 \,dx \leq \lim_{i \to \infty} \int_U |\nabla u_i|^2 \,dx$$
    We now show that this is actually an inequality for proper subsets of $U$. First we see that $u_*$ is not just a weak $H^1$ and pointwise limit, but in fact, $u_*$ is harmonic in $U$ and $u_i \to u_*$ in $C^{\infty}_{loc}(U)$ as $u_i$ are harmonic in $U$ and satisfy $|u_i| \leq 1$. Therefore, for a $x \in U$, let $r<\dist(x,\RR^n)$. Then on the ball $B_r(x)$, the $C^1_{loc}$ convergence gives equality in Fatou's lemma
    \begin{equation} \label{intconvu}
       \int_{B_r(x)} |\nabla u_*|^2 \,dx = \lim_{i \to \infty} \int_{B_r(x)} |\nabla u_i|^2 \,dx 
    \end{equation}
    and since $\mu_i(B_{r}(x)) = \int_{B_r(x)} |\nabla u_i|^2 \,dx$ we have 
    \begin{equation} \label{intmeasure}
        \mu(B_r(x)) = \int _{B_r(x)}|\nabla u_*|^2 \,dx
    \end{equation} 
    Therefore, we have
    \begin{align*}
        \Theta^{n-1}(\mu,x)&= \lim_{r \to 0}\frac{1}{r^{n-1}} \int_{B_r(x)} |\nabla u_*|^2 \,dx \\
        &\leq C_{n+1} \lim_{r\to0} r^{2} \quad \text{since }u_* \text{ is harmonic and }|u_*|\leq 1 \\
        &=0
    \end{align*}
\end{proof}

Now we consider $\Theta^{n-1}(\mu,x)$ for points on the boundary, $x \in \partial^0U$. As density is upper semicontinous we have for each $i$ and small enough $r>0$ a closed subset 
\begin{equation} \label{rel:concset0}
    \Sigma_{i,r} = \{x \in \partial^0U : r^{1-n} \mu_i(\ball r (x)) \geq \eta_0\}
\end{equation}

Here $\eta_0$ is the epsilon regularity threshold from Theorem \ref{epsreg}. From monotonicity we have $\Sigma_{i,s}\subset \Sigma_{i,r}$ for $s<r$. After a diagonal argument we can pass to a subsequence such that(without relabeling) $\Sigma_{i,2^{-k}}$ converges in a hausdorff distance sense to $\Sigma_{2^{-k}}$. Note that for $k<l$, $\Sigma_{2^{-l}} \subset \Sigma_{2^{-k}}$, and we set $\Sigma = \cap_{k} \Sigma_{2^{-k}}$. By construction $\Sigma$ is the concentration set of the energy, i.e set of all points where the density is not zero, hence above the epsilon regularity threshold $\eta_0$, namely 
\begin{equation} \label{rel:concset}
    \Sigma = \{x \in \partial^0 U : \Theta^{n-1}(\mu,x) \geq \eta_0 \}
\end{equation}  


We now show that the concentration set $\Sigma$ is a $(n-1)$ dimensional object in $\RR^n \cong \partial \RR^{n+1}$
\begin{lemm}
  For $\Sigma$ defined in (\ref{rel:concset})  $\mathcal{H}^{n-1}(\Sigma) < \infty$. 
\end{lemm}
\begin{proof}
    Let $K$ be any compact subset of $U \cup \partial^0 U$. Let $d_K= \dist(K, \partial^+U)$. Then, for any $0<r< d_K$, we can take a finite subcover of $\Sigma\cap K$, $\{\ball {r_k}(x_k)\}$ such that $\{\ball {r_k/2}(x_k) \}$ are disjoint, $r_k < r$, and $x_k \in \Sigma$.  Then by (\ref{rel:concset}) and monotonicity formula we have for large enough $i$
    \begin{equation}
        \eta_0 \leq \frac{1}{(r_k/2)^{n-1}} \mu_{i}(\ball {r_k}(x_k)) \leq  \frac{C_n}{r_k^{n-1}} E_0   
    \end{equation}
    Summing over $k$ gives us          
   \begin{equation*}
        \sum _{k} r_k^{n-1} \leq \frac{1}{\eta_0} C_n E_0 
    \end{equation*}
Therefore, for all compact sets $K$ we have $\mathcal{H}^{n-1}(\Sigma\cap K) \leq \frac{1}{\eta_0} C_n E_0 $, hence
$$\mathcal{H}^{n-1}(\Sigma) \leq \frac{C_n}{\eta_0}E_0 < \infty$$
\end{proof}




\subsection{Structure of limiting measure}
We now prove that the potential vanishes in the limit. This will allow us to clearly describe the relationship between limit measure $\mu$ and the limiting function $u_*$.
\begin{prop} \label{vanishing}
Let $\{u_i\}_{i \in \NN}$ satisfy the conditions in (\ref{sec4eqn}), then as Radon measures on $\partial^0 U$,
$$\frac{W(\ui)}{\ve_i}\,d\mathcal{H}^n \rightharpoonup 0 $$
\end{prop}
\begin{proof}
    Recall that the set $\Sigma$ is closed. So for any $x\notin \Sigma$ and $r < \dist(x,\Sigma \cup \partial^+ U)$ we have by $(\ref{rel:concset0})$ and $(\ref{rel:concset})$, for all $i$ large enough , $r^{1-n} \mu_i(\ball r (x) ) \leq \eta_0$. Hence, epsilon regularity (\ref{epsreg}) gives 
$$\frac{1}{\ve_i} W(u_i)  \leq \frac{C}{r^2} \ve_i \quad \text{in } D_{r/4}(x) $$
and therefore
$$\int_{D_{r/4}(x)}\frac{W(\ui)}{\ve_i}\,d\mathcal{H}^n \leq Cr^{n-2}\ve_i \to 0  \quad \text{as }\ve_i \downarrow 0 $$
In view of this we consider a countable covering of $\partial^0U \backslash \Sigma$: for every $k\in \NN$, there are only finitely many points $\{x_{j,k}\}$, such that $D_{j,k}=D_{2^{-k}}(x_{j,k})$ is disjoint with all balls until the $(k-1)$ step, and $4\cdot 2^{-k} < \dist(x,\Sigma \cup \partial^+ U)$. Take $D_k = \cup_{j} D_{j,k}$(note that this is a finite union). By construction we have
$$ \coprod_{k \in \NN} D_k = \partial^0U \backslash \Sigma $$
and also that
$$ \lim_{i \to \infty} \int_{D_k} \frac{1}{\ve_i} W(u_i)  \mh = 0 \quad \text{for all } D_k   $$
Note that $H^{n-1}(\Sigma)< \infty$  therefore $H^n(\Sigma)=0$ and we have
\begin{align*}
    \lim_{i \to \infty} \int_{\partial^0 U} \frac{1}{\ve_i} W(u_i) \mh  &= \lim_{i \to \infty} \int_{\partial^0 U \backslash \Sigma} \frac{1}{\ve_i} W(u_i) \mh  \\
     &= \lim_{i \to \infty} \sum_{k} \int_{D_{k}} \frac{1}{\ve_i} W(u_i)  \mh \\
     &= 0 \quad \text{by Dominated Convergence Theorem}
\end{align*}

\end{proof}


We can now give a complete description of the limiting measure $\mu$ and it's relationship with $\Sigma$ and $u_*$. As Radon measures on $U \cup \partial^0U$, we have $\mu_i \rightharpoonup \mu $. However as the potential vanishes in the limit, we get
$$\frac{1}{2}|\nabla u_i|^2 \,dx \rightharpoonup \mu$$
Now, as we saw in the proof of lemma \ref{lem0}, by Fatou's lemma 
\begin{equation} \label{fatou}
\int_U \frac{1}{2}|\nabla u_*|^2 \,dx \leq \lim_{i \to \infty} \int_U \frac{1}{2}|\nabla u_i|^2 \,dx     
\end{equation}

For sets in the interior the equality holds, but in general it does not hold. We first prove a lemma describing the limit function $u_*$ on the boundary. 

\begin{prop} \label{onu*}
    As $\ve_i \to 0$, we have $u_i \to u_*$ in $C^{1,\alpha}_{loc}(U\cup \partial^0U \backslash \Sigma)$, for $\alpha \in (0,1)$. Further, $u_*=1$ or $-1$ on each connected component of $\partial^0U \backslash \Sigma$.
\end{prop}
\begin{proof}
    The proof in the interior was subsumed in proof for proposition \ref{lem0}. Therefore let $x \in \partial^0U \backslash \Sigma$. Let $r = \frac{1}{2} \min(\dist(x,\Sigma),\dist(x,\partial^+U))$. Then on $\ball r (x)$, for $i$ large enough all $\{\ui\}$ satisfy $I_{\ve_i}(r,x) < \eta_0$. Therefore by epsilon regularity (\ref{epsreg}) and clearing out (\ref{lemmaclearing}) applied in $\ball r (x)$, the functions $\{\ui\}$ satisfy all hypothesis of the convergence lemma \ref{lem:conv} in $\ball {r/4}(x) $, and we obtain $\ui \to u_*$ in $C^{1,\alpha}_{loc}(U\cup \partial^0U \backslash \Sigma)$ and $u_*=\pm 1$.  
\end{proof}

The following corollary says that the equality in (\ref{fatou}) holds for a subset $V$ of
$U$ if it does not touch $\Sigma$. 

\begin{coro} \label{propfatou}
    Let $V$ be an admissible open set in $U$ such that $\overline{V} \cap \Sigma = \phi$, then 
    $$\int_V \frac{1}{2}|\nabla u_*|^2 \,dx = \lim_{i \to \infty} \int_V \frac{1}{2}|\nabla u_i|^2 \,dx  $$
\end{coro}
\begin{proof}
   We first assume that $V$ is a ball centered at $x$ and split the proof into two cases.
   
   Case 1: Let $V = B_r(x) \subset U$. In this case the conclusion was already established in (\ref{intconvu}) while proving lemma \ref{lem0}.$$
    \int_{B_r(x)} |\nabla u_*|^2 \,dx = \lim_{i \to \infty} \int_{B_r(x)} |\nabla u_i|^2 \,dx $$
   
   Case 2: When $V = \ball r (x)$. Due to the assumption $\overline{V} \cap \Sigma = \phi $, $x \in   \partial^0U \backslash \Sigma $ and as $\Sigma$ is closed, there is a half ball $\ball r (x)$ disjoint from $\Sigma$, such that we have uniform $C^{1,\alpha}(\overline{\ball {r/4}(x))}$ convergence $u_i$ to $u_*$ by (\ref{onu*}) hence $$\int_{\ball {r/4}(x)} |\nabla u_*|^2\,dx = \lim_{i \to \infty} \int_{\ball {r/4}(x)} |\nabla u_i|^2\,dx $$
   as desired, in this case as well. 
   Together, these two cases imply the result for every admissible open set $V$ by a covering argument similar to the one used in proof of Theorem \ref{vanishing}
\end{proof}

The above two results show us that the obstruction to equality in (\ref{fatou}) is the set $\Sigma$. That is the lack of compactness in $H^1(U)$, i.e inequality in (\ref{fatou}) is due to the fact that when the energy of $\{u_i\}$ is high, then the tendency of $|u_i|$ to converge to $1$, as $\ve_i \downarrow 0$, leads to loss of energy in the singular limit. This energy loss however is captured in the energy concentration set $\Sigma$. Indeed we may rewrite (\ref{fatou}) as,
\begin{equation} \label{measdecom}
    \frac{1}{2}|\nabla u_i|^2 \,dx \rightharpoonup \mu= \frac{1}{2}|\nabla u_i|^2 \,dx + \mu_{\Sigma}
\end{equation} 
where $\mu_{\Sigma}$ is the defect measure that arises if there is a failure of strong convergence in $H^1(U)$ as explained above. The following result gives a complete description of $\mu_{\Sigma}$.

\begin{theo}\label{thm:sing}
    The measure $\mu_{\Sigma}$ has the following properties. 
    \begin{enumerate}
        \item It is supported in the energy concentration set $\Sigma$.
        \item $\Theta^{n-1}(\mu_{\Sigma},x)=\Theta^{n-1}(\mu, x)$ for $\mathcal{H}^{n-1}$ a.e. on $\Sigma$. 
        \item Further writing $\theta(x) = \Theta^{n-1}(\mu_{\Sigma},x)$, we have  $$\mu_{\Sigma}=\theta \, \mathcal{H}^{n-1}\measurer \Sigma$$ with $\eta_0 \leq \theta(x) \leq C < \infty$, for $\mathcal{H}^{n-1}$ a.e. $x \in \Sigma$.
    \end{enumerate}
    
     
\end{theo}
\begin{proof}
Let $x \in U \cup \partial^0U \backslash \Sigma$. Then there is a ball(or half ball for boundary point) $B_x$ containing $x$, and disjoint from $\Sigma$, by corollary \ref{propfatou}, $$\mu(B_x)= \int_{B_x} |\nabla u_*|^2 \,dx$$
and therefore by (\ref{measdecom}), $\mu_{\Sigma}(B_x)=0$, for all such $x$, therefore $spt(\mu_{\Sigma}) \subset \Sigma$. This proves part (1). 

Next, as $u_* \in H^1(U)$ and note that $U \subset \RR^{n+1}$ therefore by equation (3.3.38) in \cite{z}, the $(n-1)$-density of $u_*$ is $\mathcal{H}^{n-1}$ a.e. $0$ i.e we have $$ \lim _{r \to 0}\frac{1}{r^{n-2}}\int_{B_r(x)} |\nabla u_*|^2\,dx = 0 \quad \mathcal{H}^{n-1} a.e. $$
In particular this implies part (2) i.e 
$$\Theta^{n-1}(\mu_{\Sigma},x)=\Theta^{n-1}(\mu, x)  \quad \mathcal{H}^{n-1} a.e. \text{ } x \in \Sigma $$
For part (3), recall that by definition $\eta_0 \leq \Theta^{n-1}(\mu, x)$ for all $x$ on $\Sigma$. The upper bound follows from monotoncity formula. Combined with part (2), this gives 
$$\eta_0 \leq \Theta_{\Sigma}^{n-1}(\mu, x) < C \quad \mathcal{H}^{n-1} a.e. \text{ } x\in \Sigma  $$
Therefore is absolute continous with respect to $\mathcal{H}^{n-1}\measurer \Sigma$ and by Radon-Nikodyn Theorem we get part (3), i.e  $\mu_{\Sigma} = \theta \, \mathcal{H}^{n-1}\measurer \Sigma$.
\end{proof}










