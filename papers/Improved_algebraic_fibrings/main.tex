\documentclass[11pt, letterpaper]{amsart}
% CVPR 2022 Paper Template
% based on the CVPR template provided by Ming-Ming Cheng (https://github.com/MCG-NKU/CVPR_Template)
% modified and extended by Stefan Roth (stefan.roth@NOSPAMtu-darmstadt.de)

\documentclass[10pt,twocolumn,letterpaper]{article}

%%%%%%%%% PAPER TYPE  - PLEASE UPDATE FOR FINAL VERSION
%\usepackage[review]{cvpr}      % To produce the REVIEW version
%\usepackage{cvpr}              % To produce the CAMERA-READY version
\usepackage[pagenumbers]{cvpr} % To force page numbers, e.g. for an arXiv version

% Include other packages here, before hyperref.
\usepackage{graphicx}
\usepackage{amsmath}
\usepackage{amssymb}
\usepackage{booktabs}
\usepackage{paralist}




% It is strongly recommended to use hyperref, especially for the review version.
% hyperref with option pagebackref eases the reviewers' job.
% Please disable hyperref *only* if you encounter grave issues, e.g. with the
% file validation for the camera-ready version.
%
% If you comment hyperref and then uncomment it, you should delete
% ReviewTempalte.aux before re-running LaTeX.
% (Or just hit 'q' on the first LaTeX run, let it finish, and you
%  should be clear).
\usepackage[pagebackref,breaklinks,colorlinks]{hyperref}


% Support for easy cross-referencing
\usepackage[capitalize]{cleveref}
\crefname{section}{Sec.}{Secs.}
\Crefname{section}{Section}{Sections}
\Crefname{table}{Table}{Tables}
\crefname{table}{Tab.}{Tabs.}


%%%%%%%%% PAPER ID  - PLEASE UPDATE
\def\cvprPaperID{5910} % *** Enter the CVPR Paper ID here
\def\confName{CVPR}
\def\confYear{2022}


\begin{document}

%%%%%%%%% TITLE - PLEASE UPDATE
\title{Dyadic Human Motion Prediction}

\author{
	Isinsu Katircioglu$^1$
	\and
	Costa Georgantas$^2$
	\and
	Mathieu Salzmann$^{1,3}$
	\and
	Pascal Fua$^1$   
	\and
	$^1$CVLab, EPFL, Switzerland\quad
	$^2$CHUV, Switzerland\\
	$^3$ClearSpace SA, Switzerland
}

\maketitle
\makeatletter
\let\tmp=\phi \let\phi=\varphi \let\varphi=\tmp
\let\tmp=\epsilon \let\epsilon=\varepsilon \let\varepsilon=\tmp

\let\emptyset=\varnothing
\def\O{\Ocal}
\def\o{\scalebox{0.65}{\Ocal}}
\def\aseq#1{\overset{#1}{=}}
\def\i{{\bm\imath}}
\def\j{{\bm\jmath}}
\def\stab{\operatorname{\mathrm{Stab}}}
\def\emp{\operatorname{\qbf}}
\def\convconend{\mathrm{\Kbf}_{k,l}}
\def\dist{\operatorname{{\bm\delta}}}
\def\dista{\operatorname{\hat{\bm\delta}}}
\def\distaa{\operatorname{\hatt{\bm\delta}}}
\def\d{\operatorname{\drm}}
\def\kd{\operatorname{\mathrm{kd}}}
\def\ad{\operatorname{\mathbf{ad}}}
\def\ikd{\operatorname{\mathbf{k}}}
\def\aikd{\operatorname{{\bm\kappa}}}\let\akd=\aikd
\def\aep{\operatorname{\mathbf{aep}}}
\def\Ent{\operatorname{\mathcal{E}}\!\!{nt}}\let\ent=\Ent
\def\MI{\operatorname{\mathcal{I}}}
\def\Aut{\operatorname{\mathrm{Aut}}}
\def\IO{\operatorname{\mathrm{IO}}}\let\io=\IO
\def\Ext{\operatorname{\mathrm{Ext}}}
\def\Hom{\operatorname{\mathrm{Hom}}}
\def\Interior{\operatorname{\mathrm{Interior}}}
\def\closure{\operatorname{\mathrm{Closure}}}
\def\supp{\operatorname{\mathrm{supp}}}
\def\defect{\operatorname{\mathrm{Defect}}}
\def\aeq{\sim\raisebox{-0.4mm}{$\mkern-8mu{\scriptscriptstyle{}a}\mkern3mu$}}
\def\prob{\operatorname{\mathbf{Prob}}}\let\Prob=\prob
\def\probhom{\prob_{\mathbf{h}}}\let\Probhom=\probhom
\def\probas{\prob^{\infty}}
\def\probhomas{\probhom^{\infty}}
\def\Set{\operatorname{\mathbf{Set}}}
\def\Id{\operatorname{\mathrm{Id}}}
\def\lin{\operatorname{\Lsf}}
\def\qlin{\operatorname{\Qsf\Lsf}}
\def\ac{\operatorname{\Afrak\Cfrak}}
\def\dia{{\bm\Diamond}}
\def\emptyconfig{\mkern3mu\raisebox{0.2ex}{\ensuremath{\bm/}}\mkern-13mu\Ocal}
\def\emptycat{\bm{\emptyset}}
\def\res{\Gammabf^{\circ}}
\def\sres{\bar\Gammabf}
\def\c#1{(\!#1\!)}
\def\<#1>{\left\langle#1\right\rangle}
\def\size#1{[\mkern-5mu[#1]\mkern-5mu]}
\def\sep{\,||\,}
\def\mix{\operatorname{\Mcal\mkern-2mu\mathit{ix}}}
%\def\rel{|\mkern-5.1mu\llcorner}
\def\rel{|\mkern-2.5mu\raisebox{-0.57ex}{\rule{0.3em}{0.15ex}}}
\def\linseq#1{\overset{\to}{#1}}
\let\trop=\top
\let\bar=\overline
\newlength{\hattheight}
\def\Xcalbf{\bm\Xcal}
\def\hatt#1{\hat{\hat{#1}}}%
%      \settoheight{\hattheight}{\ensuremath{\hat{#1}}}%
%      \addtolength{\hattheight}{-0.3ex}%
%       \hat{\vphantom{\rule{1pt}{\hattheight}}%
%      \hat{\vphantom{\rule{1pt}{\hattheight}}%
%      \smash{\hat{#1}}}%
% }
\newcommand{\into}[1][]{\stackrel{#1}{\hookrightarrow}}
\renewcommand{\to}[1][]{\stackrel{#1}{\rightarrow}}
\newcommand{\too}[1][]{\stackrel{#1}{\longrightarrow}}
\newcommand{\ot}[1][]{\stackrel{#1}{\leftarrow}}
\newcommand{\oot}[1][]{\stackrel{#1}{\longleftarrow}}
%\newcommand{\into}[1][]{\stackrel{#1}{\hookrightarrow}}
\def\indep{\,\makebox[0cm][l]{$\bot$}\mkern2mu\bot\,}
\newcommand{\un}[2][5mu]{\underline{#2\mkern-#1}\mkern#1} 

\newcommand{\Deltan}[1][n]{\Delta^{\!\!\!^{(\!#1\!)}}\!}

\def\bluepar#1\par{%
    \par\textcolor{blue}{#1}\par}
\def\redpar#1\par{%
    \par\textcolor{red}{\hspace{-1.87\parindent}$\bullet$ #1}\par}
\def\greenpar#1\par{%
    \par\textcolor{green}{#1}\par}
\def\jimpar#1\par{%
	\par\textcolor{purple}{\hspace{-1.87\parindent}$\blacksquare$  #1}\par}

\def\skippar#1\par{\par\relax}

\def\blue#1{{\color{blue}#1}}
\def\red#1{{\color{red}#1}}
\def\green#1{{\color{green}#1}}

\def\theenumi{\roman{enumi}}


\makeatother

\begin{abstract}
\label{sec:abstract}

%% 1. what is the problem 
Scientific applications that run on leadership computing facilities often face the challenge 
of being unable to fit leading science cases onto accelerator devices due to memory constraints 
(memory-bound applications).
%
% 2. what is your solution 
In this work, the authors studied one such US Department of Energy mission-critical condensed matter 
physics application, Dynamical Cluster Approximation (DCA++), and this paper discusses how device memory-bound challenges were successfully reduced  by proposing an effective 
``all-to-all'' communication method---a ring communication algorithm. 
%
This implementation takes advantage of acceleration on GPUs and remote direct memory access (RDMA) for fast data exchange between GPUs. 
%
\\Additionally, the ring algorithm was optimized with sub-ring communicators
and multi-threaded support to further reduce communication overhead and 
expose more concurrency, respectively.
%
% 3. What's the cherry-picked evaluation result you want to mention
The computation and communication were also analyzed 
by using the Autonomic Performance Environment for Exascale 
(APEX) profiling tool,  and this paper further discusses the 
performance trade-off for the ring algorithm implementation. 
%
The memory analysis on the ring algorithm shows that the allocation size for the authors' most 
memory-intensive data structure per GPU is now reduced to $1/p$ of the original size, where $p$ is the number of GPUs in the ring communicator.
%
The communication analysis suggests that 
the distributed Quantum Monte Carlo execution time grows linearly as sub-ring size increases, and the cost of messages passing through the network interface connector could be a limiting factor.


%
% \todoRed{Ronnie: Next sentence needs rewrite, too much information about Green's function that no one knows in the abstract; recommend generalizing.} \emph {However, DCA++ is currently facing memory-bound challenge as 
% a larger device array $G_t$ is limited by device memory size, where
% $G_t$ is a two-particle Green's function that allows condensed matter
% scientists to explore larger and more complex (higher fidelity)
% physics cases.}

\end{abstract}

\keywords{DCA++, Quantum Monte Carlo, GPU Remote Direct Memory Access, memory-bound issue, exascale machines}

\section{Introduction}  \label{sec:introduction}

\newcommand\inexpIntro[3]{#1?(#2,#3).}
\newcommand\rinexpIntro[3]{*#1?(#2,#3).}
\newcommand\outexpIntro[3]{#1!(#2,#3).}
\newcommand\outatomIntro[3]{#1!(#2,#3)}

We propose a fully automated method for proving termination of \(\pi\)-calculus processes.
Although there have been a lot of studies on termination analysis for the \(\pi\)-calculus
and related calculi~\cite{Deng06IC,Demangeon07,SangiorgiTermination,KobayashiHybrid,Yoshida04IC,DBLP:journals/jlp/DemangeonHS10,Venet98SAS}, most of them have been rather theoretical,
and there have been surprisingly little efforts in developing  fully automated termination
verification methods and tools based on them. To our knowledge,
Kobayashi's \typical{}~\cite{TyPiCal,KobayashiHybrid} is the only exception that
can prove termination of \(\pi\)-calculus processes (extended with natural numbers)
fully automatically, but its termination analysis is quite limited (see Section~\ref{sec:relatedwork}).

Our method is based on a reduction to termination analysis for sequential programs:
we translate a \(\pi\)-calculus process \(P\) to a sequential program \(S_P\), so that
if \(S_P\) is terminating, so is \(P\). The reduction allows us to use
powerful, mature methods and tools
for termination analysis of sequential programs~\cite{heizmann2016ultimate,freqterm,DBLP:conf/lics/PodelskiR04,Kuwahara2014Termination,DBLP:journals/cacm/CookPR11}.

The idea of the translation is to convert a chain of communications on replicated input
channels to a chain of recursive function calls of the target sequential program.
Let us consider the following Fibonacci process:
\begin{align*}
    & \rinexpIntro{\fib}{n}{r}
        \ifexp{n<2}{ \soutatom{r}{1} \\ &\quad}
                   { \nuexp{s_1} \nuexp{s_2} (\outatomIntro{\fib}{n-1}{s_1} \PAR \outatomIntro{\fib}{n-2}{s_2} \PAR \sinexp{s_1}{x}\sinexp{s_2}{y}\soutatom{r}{x+y}) \\}
    & \PAR \outatomIntro{\fib}{m}{r}
\end{align*}
Here, the process
$\rinexpIntro{\fib}{n}{r} \ldots$ is a function server that computes the \(n\)-th Fibonacci number
in parallel and returns the result to \(r\),
and $\outatom{\fib}{m}{r}$ sends a request for computing the \(m\)-th Fibonacci number;
those who are not familiar with the syntax of the \(\pi\)-calculus may wish to consult
Section~\ref{sec:targetlanguage} first.
To prove that the process above is terminating for any integer \(m\),
it suffices to show that there is no infinite chain of communications on $\fib$:
\[
    \fib(m,r) \to \fib(m_1,r_1) \to \fib(m_2,r_2) \to \cdots.
\]
We convert the process above to the following program:\footnote{The actual translation
  given later is a little more complex.}
\begin{verbatim}
 let rec fib(n) = if n<2 then () else (fib(n-1) [] fib(n-2)) in
 fib(m)
\end{verbatim}
Here, \texttt{[]} represents the non-deterministic choice.
Note that, although the calculation of Fibonacci numbers is not preserved,
for each chain of communications on \texttt{fib}, there is a corresponding
sequence of recursive calls:
\[
\mathtt{fib}(m) \to \mathtt{fib}(m_1) \to \mathtt{fib}(m_2) \to \cdots.
\]
Thus, the termination of the sequential program above implies the termination of
the original process.
As shown in the example above, (i) each communication on a replicated input channel
is converted to a function call, (ii) each communication on a non-replicated input
channel is just removed (or, in the actual translation, replaced by a call of
a trivial function defined by \(f(\seq{x})=(\,)\)), and (iii) parallel composition
is replaced by a non-deterministic choice.
We formalize the translation outlined above and prove its correctness.

The basic translation sketched above sometimes loses too much information.
For example, consider the following process:
\begin{align*}
    & \rinexpIntro{\pre}{n}{r} \soutatom{r}{n-1} \\
    & \PAR \rinexpIntro{f}{n}{r} \ifexp{n<0}{ \soutatom{r}{1} }
                                       { \nuexp{s} (\outatomIntro{\pre}{n}{s} \PAR \sinexp{s}{x}\outatomIntro{f}{x}{r}) } \\
    & \PAR \outatomIntro{f}{m}{r}
\end{align*}
The translation sketched above would yield:
\begin{verbatim}
  let pred(n) = n-1 in
  let rec f(n) = if n<0 then () else (pred(n) [] f(*)) in
  f(m)
\end{verbatim}
Here, \texttt{*} represents a non-deterministic integer: since we have removed
the input $\sinatom{s}{x}$, we do not have information about the value of \( x \).
As a result, the sequential program above is non-terminating, although the original
process is terminating.
To remedy this problem, we also refine the basic translation above by using a refinement
type system for the \(\pi\)-calculus. Using the refinement type system,
we can infer that the value of \(x\) in the original process is less than \(n\),
so that we can refine the definition of \texttt{f} to:
\begin{verbatim}
 let rec f(n) = ... else (pred(n) [] let x=* in assume(x<n);f(x))
\end{verbatim}
The target program is now terminating, from which
we can deduce that the original process is also terminating.
We have implemented an automated tool based on the refined translation above.

The contributions of this paper are summarized as follows.
\begin{itemize}
\item The formalization of the basic translation from the \(\pi\)-calculus
  (extended with integers) to sequential programs, and a proof of its correctness.
\item The formalization of a refined translation based on a refinement type system.
\item An implementation of the refined translation, including automated refinement type
  inference based on CHC solving, and experiments to evaluate the effectiveness of
  our method.
\end{itemize}

The rest of this paper is structured as follows.
Section~\ref{sec:targetlanguage} introduces the source and target languages
of our translation.
Section~\ref{sec:approach} 
formalizes the basic translation, and proves its correctness.
Section~\ref{sec:refinement} refines the basic translation by using a refinement type system.
Section~\ref{sec:implementation} reports an implementation and experiments.
Section~\ref{sec:relatedwork} discusses related work,
and Section~\ref{sec:conclusion} concludes the paper.

\textbf{Related work}:
% Object detection related datasets/algo in non-medical domain
% Locally labeled CXR dataset
A few CXR datasets have localized abnormality annotations \cite{shih2019augmenting,filice2020crowdsourcing,jaeger2014two} that are curated manually. These are high quality gold standard ground truth datasets but tend to be smaller in scale (< 30,000 images) and have a narrow coverage, with typically only 1-2 labels. In addition, since most labeling efforts only have abnormality semantics attached, no direct relationships with the affected anatomical locations are available. 

%MEHDI: repeated concepts from above. I am removing the following: 

%The lack of anatomic semantics in the annotation is a limitation for complex multi-modal clinical reasoning work, e.g., differential diagnosis, since clinicians often integrate information along anatomical lines, and for downstream report generation tasks, which often requires describing not only the abnormality but also correctly communicate the location of the abnormalities (and medical devices) to the receiving clinicians. 

Two recent CXR datasets have labels for anatomies described in the reports. In \cite{datta2020dataset}, a small manually annotated dataset (2000 reports) included 10 abnormalities that are individually associated with 29 unique spatial locations (anatomies) at the report level. Another CXR dataset has automatically extracted abnormality and anatomy labels as disconnected concepts that are only correlated at the study level from  160,000 reports using a supervised NLP algorithm \cite{bustos2020padchest}. This was trained on a smaller set of manually annotated data. Neither datasets contain localized annotations for the associated CXR images, nor any comparison relation annotations between sequential exams, both of which are available in the Chest ImaGenome dataset. In Table \ref{tab:related}, we present a comparison of our Chest ImagGenome dataset with other datasets available in the literature.

% Table -- Kashyap

% MEdical imaging datasets to go here: Discussed that we will only focus on cxr datasets that are available for this paper. 
% \caption{\color{red} Kashyap, feel free to continue with the table. We should remove the questionmarks and add a line for our dataset (since all others are not graph). For longer text, using abbreviations and explaining them in the caption often works better. If fill in the values is not possible, it is better to remove the table altogether.}


\begin{table}[t!]
\caption{Summary of existing chest X-ray datasets}
\resizebox{\textwidth}{!}{%
\begin{tabular}{@{}lllllllll@{}}
\toprule
\textbf{Dataset} & \textbf{Annotation Level} & \textbf{Annotation Method} & \textbf{Num Labels} & \textbf{Anatomy Labeled} & \textbf{Graph} & \textbf{Dataset Size} & \textbf{Temporal Labels} & \textbf{Reports} \\ \midrule
SIIM-ACR Pneumothorax Segmentation \cite{filice2020crowdsourcing} & Segmentation & Manual + augmented & 1 & No & No & 12,047 & No & No \\
RSNA Pneumonia Detection Challenge   \cite{shih2019augmenting} & Bounding Boxes & Manual & 1 & No & No & 30,000 & No & No \\
Indiana University Chest X-ray collection \cite{demner2016preparing} & Global & Automated & 10 & No & No & 3,813 & No & Yes \\
NIH CXR dataset \cite{wang2017chestx} & Global & Automated & 14 & No & No & 112,120 & No & No \\
PLCO \cite{team2000prostate} & Global & Automated & 24 & Yes & No & 236,000 & Yes & No \\
Stanford CheXpert \cite{irvin2019chexpert} & Global & Automated & 14 & No & No & 224,316 & No & No \\
MIMIC-CXR \cite{johnson2019mimic} & Global & Automated & 14 & No & No & 377,110 & No & Yes \\
Dutta \cite{datta2020dataset} & Global & Manual & 10 & Yes & Yes & 2,000 & No & Yes \\
PadChest \cite{bustos2020padchest} & Global & Manual + automated & 297 & Yes & No & 160,868 & No & Yes \\
Montgomery County Chest X-ray   \cite{jaeger2014two} & Segmentation & Manual & 1 & Yes & No & 138 & No & No \\
Shenzen Hospital Chest X-ray   \cite{jaeger2014two} & Segmentation & Manual & 1 & Yes & No & 662 & No & No \\  \hline \hline
\textbf{Chest ImaGenome} & Bounding Boxes & Automated & 131 & Yes & Yes & 242,072 & Yes & Yes \\
\bottomrule
\end{tabular}%
}
\label{tab:related}
\vspace{-0.4cm}
\end{table}
% removed (Derived from MIMIC-CXR \cite{johnson2019mimic}) % makes table really small

The proposed segmentation-by-detection framework, as depicted in Figure \ref{fig:framework}, consists of a detection module and a segmentation module.
In detection stage, 2D slices (layered box) from the input volume are fed to the RPN. Based on the region proposals obtained from RPN, an attention model (block in orange) is formed. The input volume as well as the attention model are further processed in segmentation stage to get the refined anatomical segmentation. 
\vspace{1em} 

\begin{figure}[t]
\centering
\includegraphics[width=0.95\linewidth]{fig/framework.pdf}
\caption{Schematic representation of the segmentation-by-detection framework. The left part is the detection module while the segmentation module is followed on the right. The blue block denotes the input volume which is 3D ultrasound scan of femoral head. The output segmentation is in red.}
\label{fig:framework}
\end{figure}
% dana could you improve the figure. we can try to think together of better ways 

\noindent\textbf{Detection Module:} 
% dana : here you have to make the clarification that you have ground truth on the boxes (in implementation part)
The detection module follows an RPN architecture, a fully convolutional network which takes image slice as input and outputs object region candidates. 
We use the VGG-16 model as the backbone \cite{simonyan2014very} to learn convolutional features and an $3 \times 3$ spatial window to generate region proposals. At each sliding-window location, 9 anchors are predicted associated with different scales and aspect ratios. The last layer consists of a box-regression (reg) layer and a box-classification (cls) layer in parallel. The reg layer outputs 4 regression offsets, $ t = (t_x,t_y,t_w,t_h)$, denoting a scale-invariant translation as well as log-space height and width shift, where $x,y,w$ and $h$ specify two coordinates of the box center, width and height. The cls layer outputs two scores by softmax, related to probabilities of object and background for each proposal. We assign a positive label (of being object) to candidate which has an Intersection-over-Union (IoU) ratio higher than 0.7 with ground truth box. Note that an image slice may contain multiple object regions or none. 

The loss function of RPN follows the multi-task loss \cite{ren2015faster} which is defined as $L = L_{reg} + L_{cls}$. The regression loss, $L_{reg} = -\log p_{obj}$ is log loss and the classification loss,
\begin{equation} \label{eq:loss}
L_{cls} = \sum_{i \in \{x,y,w,h\}} smooth_{L_1} (t_i - t_i^*)
\end{equation}
is smooth $L_1$ loss where $t_i^*$ denotes the ground truth box for the target object. 
\vspace{1em}

\noindent\textbf{Segmentation Module:}
3D U-Net \cite{cciccek20163d} is utilized in the segmentation module as its outstanding performance in medical image segmentation. The u-shaped architecture consists of two paths: a contracting path, where each layer contains two $3\times3\times3$ convolutions followed by a rectified linear unit (ReLU) and then a max pooling, provides high resolution features. While, the symmetric expanding path for semantically richer features replaces max pooling with a upconvolution $2\times2\times2$ with stride of 2 in each dimension, and then two $3\times3\times3$ convolutions each followed by a ReLU. Skip connections between layers of equal resolution in the contracting path and the expanding path enables context information as well as precise localization.

Different from 3D U-Net, to incorporate the attention model detected by the RPN, our architecture takes as input both the volumetric image data and the candidate RoIs proposed by the RPN, concatenated as 3D volume. 
% dana not sure what you like to say below
% densely annotated
The attention model makes the network to focus on the potential RoIs and can reduce the interference of the surrounding noise.
The anatomical segmentation is then generated from a $1\times1\times1$ convolution which reduces the number of feature maps to the number of labels.  The energy function is computed by a pixel-wise softmax combined with the cross entropy loss.
% dana equation ??

\subsection{System and implementation Details}
The segmentation-by-detection approach adopts a cascade structure with two stages: detection and segmentation. The two networks are trained separately in an end-to-end manner. All the new layers are randomly initialized from zero-mean Gaussian distribution with standard deviations 0.01. Biases are initialized to 0. We use Caffe \cite{jia2014caffe} for the implementation and an NVIDIA Titan X GPU for training.

In the detection stage, we initialize the VGG-16 model by the pre-trained model for ImageNet classification \cite{russakovsky2015imagenet} and further fine-tune the model for our detection task. The input fed to the network are image slices with a fixed size of $184\times96$ and the corresponding ground truth boxes are generated from the annotation in the format of tight bounding boxes surrounding the segmentation contour (as illustrated in Figure \ref{fig:hip} (b), the boundary of white area). To optimize the energy function, stochastic gradient descent (SGD) is used. The global learning rate is set to 0.001, while a momentum of 0.9 and a weight decay of 0.0005 are used. The batch size is set to 256 and each mini-batch only contains the positive anchors for training. The region proposals are obtained from the reg path for each image slice. The attention model is then formed by concatenating all the detected regions, as binary masks, into a volume.

In the segmentation stage, we use the Adam optimizer \cite{kingma2014adam} to learn the network parameters. A global learning rate is set to 0.001 while the two momentum coefficients are set to 0.9 and 0.999 respectively. A batch size of 1 is used due to the memory constraints of the GPU. The network takes the volume data as well as the attention model as input. We train the network for a maximum of 30K iterations and reserve the learned weights with the best performance from every 1K iterations. 
\vspace{1em}

\noindent\textbf{Inference:}
At test time, the 2D slices from an input volume are first fed to the detection module. The attention model is obtained based on the output. Then the volume data as well as the attention model are fed to the segmentation module to get the pixel-wise prediction.



\section{Experiments}

In this section, we demonstrate the effectiveness of our approach at exploiting dyadic interactions. To this end, we first introduce our \lindyhop{} dataset depicting couples that perform lindy hop dance movements.

\subsection{LindyHop600K}
Lindy hop is a type of swing dance with fast-paced steps synchronized with the music. It constitutes a good example of motions with strong mutual dependencies between the subjects, who are engaged in close interactions. To build this dataset, we filmed three men and four women dancers paired up in different combinations. Overall, \lindyhop{} contains nine dance sequences, each two to three minutes long, with a maximum of eight cameras at 60 fps. We use the shortest two sequences as validation and test sets. Table~\ref{table:seq_lhop} shows the details of the dataset organization. Our dataset displays standard lindy hop dancer positions and steps, such as the so-called open, closed, side and behind positions. In the open and closed positions, the dancers are facing each other with a varying distance between them. In the side position, both are facing the same direction, and in the behind position, the leader stands directly behind the follower, both facing the same direction. In each position, the dancers communicate through hand and shoulder grips. To the best of our knowledge, \lindyhop{} is the first large dance dataset involving the videos and 3D ground-truth poses of dancers.

\begin{table}[t] 
	\centering 
	\scalebox{0.9}{
		
		\begin{tabular}{ c|c|c|c|c } 
			\hline
			Sequence & Couple & Frames & Cameras & Split \\
			\hline
			{1} & A1 & 10152 & 5 & Train  \\ 
			{2} & B2 & 8819 & 8 & Train  \\ 
			{3} & C3 & 6519 & 8 & Validation  \\ 
			{4} & A4 & 7687 & 8 & Test \\ 
			{5} & B1 & 9977 & 8 & Train \\ 
			{6} & C2 & 9636 & 8 & Train\\ 
			{7} & A3 & 8930 & 7 & Train \\ 
			{8} & B4 & 9027 & 8 & Train \\ 
			{9} & C1 & 9635 & 8 & Train \\ 
			\hline
		\end{tabular}
	}
	\caption[\lindyhop{} dataset structure]{ \textbf{\lindyhop{} dataset structure.}}
	\label{table:seq_lhop}
	\vspace{-4mm}
\end{table}

To obtain the 3D poses of the dancers, we first extract 2D pixel locations of the visible joints using OpenPose~\cite{Cao17}. Because our dataset was captured with multiple cameras, this lets us obtain the  3D joint coordinates by performing a bundle adjustment using the 2D joint locations in all the views. However, this process comes with several problems because it requires annotating the poses of both subjects together. The major issues encompass body part confusions, missing 2D annotations and tracking errors in the OpenPose predictions, which occur when two people are very close to each other or wear similar garments. An example of this is shown in Fig.~\ref{fig:optimizing_3dposes}. To remedy this, we adopt a solution based on temporal smoothness. Specifically, we assign manually the 2D joint locations to each dancer in the first frame of each sequence. For the subsequent frames, the low confidence joint detections are replaced with ones interpolated using the high confidence joints from the neighboring frames. Despite these 2D joint corrections, the 3D locations extracted from the bundle adjustment procedure can still be very noisy. Thus, we employ a third degree spline interpolation across 30 frames coupled with an optimization scheme to generate the final 3D poses. Since the spline interpolation is done separately for each dimension of each joint, the length of each limb varies from one frame to another. To tackle this problem, we implement an optimization scheme which minimizes the squared difference between the length of a limb $c$ in the current frame and the average length of  limb $c$. We combine this loss function with additional regularizers penalizing feet from sliding on the floor, constraining the shape of the hips and shoulders, and preventing the optimization to the initial 3D pose estimates. For more detail, we refer the reader to the supplementary material.


\begin{figure}
	\centering
	\begin{tabular}{c}
		
		\includegraphics[width=0.67\linewidth]{figures/lindyhop_failure.pdf} \\
		(a) \footnotesize OpenPose 2D detection failure and the optimized 3D poses \\ \\
		\includegraphics[width=0.67\linewidth]{figures/lindyhop_success.pdf} \\
		(b) \footnotesize Correct OpenPose detections and the optimized 3D poses\\
	\end{tabular}
	\caption[Optimizing 3D poses in the \lindyhop{} dataset]{\textbf{Optimizing 3D poses in the \lindyhop{} dataset.} (a) Example of OpenPose 2D detection failure. The left leg of the woman is mapped to the left leg of the man. Our multi-view footage and refinement strategy allow us to obtain accurate 3D poses of the dancers despite the mismatch in the 2D detections. (b) Example of correct OpenPose detections and the optimized 3D ground truth poses.}
	\label{fig:optimizing_3dposes}
	\vspace{-4mm}
\end{figure}

\subsection{Data Pre-processing}
Each video sequence is first downsampled to 30 fps. The human body skeleton in the \lindyhop{} dataset originally comprises of $25$ body joints. We remove some of the facial, hand and foot joints and train our models with a skeleton of $19$ joints. The 3D joint locations are represented in the world coordinates. Since the position and orientation of the dancers change from one frame to another, we apply a rigid transformation to the poses.  We first subtract the global position of the hip center joint from every joint coordinate in every frame. Then, for each sequence, we take the first pose as  reference and rotate it such that the unit vector from the left to right shoulder is aligned with the $x$-axis and the unit vector from the center hip joint to the neck is aligned with the $z$-axis. We apply the same rotation to all the other poses in the sequence. 

\subsection{Results}

In this section, we evaluate our approach depicted by Fig.~\ref{fig:overview_3dmotion_forecasting} on our new \lindyhop{} dataset. We compare our method with the state-of-the-art single person approaches. They include HRI~\cite{Mao20}, which relies on an attention mechanism and a GCN decoder~\cite{Mao19} to predict the future poses of the individuals in isolation; HRI-Itr, which uses the output of the predictor as input and predicts the future motion recursively; TIM~\cite{Lebailly20}, which extends~\cite{Mao19} by combining it with a temporal inception layer to process the input at different subsequence lengths; and MSR-GCN~\cite{Lingwei21}, the most recent method, which extracts features from the human body at different scales by grouping the joints in close proximity. All the baselines rely on a GCN architecture that is trained and tested according to the data split shown in Table~\ref{table:seq_lhop}. They take as input a sequence of $60$ poses as  past motion. Except for HRI-Itr that recursively predicts $10$ poses at a time, all the baselines predict $30$ poses in the future. 

In Table~\ref{table:sota_lhop}, we report the MPJPE for short-term ($<$ 500ms) and long-term ($>$ 500ms) motion prediction in mm. Our method outperforms the baselines by a large margin. Fig.~\ref{fig:qualitative_lhop_sota} depicts qualitative results of our approach and the best performing three baselines for the \lindyhop{} test subjects with the corresponding follower and leader roles in the top two and bottom two portions, respectively. In contrast to the baselines, our method accurately predicts moves that are hard to anticipate in the long term, such as fast changing feet movements and less frequent arm openings. Although the observed motion of the primary subject does not include sufficient clues for such moves, the second person provides a useful prior so that our model can learn to predict the motion complementary or symmetric to that of the auxiliary subject. Therefore, we attribute this performance to our modeling of the motion dependencies via our pairwise attention mechanism. We provide additional qualitative results and further analysis on the learned pairwise attention scores in the supplementary material.

\begin{figure*}
	\vspace{-4mm}
	\centering
	\begin{tabular}{c}
		\includegraphics[width=0.93\linewidth]{figures/sota_qual_lhop_two_people.pdf} \\
	\end{tabular}
	\vspace{-4mm}
	\caption[Qualitative 3D motion prediction results on the \lindyhop{} test subjects]{\textbf{Qualitative evaluation of our results on the LindyHop600K test subjects compared to the state-of-the-art methods.} Black: Ground truth, green: TIM~\cite{Lebailly20}, blue: MSR-GCN~\cite{Lingwei21}, violet: HRI~\cite{Mao20}, red: Ours-Dyadic. Top two portions show the predictions for dancer with the follower role. Bottom two portions show the predictions for the dancer with the leader role. The left side of the vertical bar in the black row depicts the sampled input to our model and the right side shows the ground truth future poses. The colored rows correspond to the predictions of the state-of-the-art single person approaches. The red row depicts the output of our model shown in Fig.~\ref{fig:overview_3dmotion_forecasting}. The numbers at the top indicate the timestamp in milliseconds and the green region highlights the long-term predictions.}
	\label{fig:qualitative_lhop_sota}
\end{figure*}




\begin{table*}[t]
	%\vspace{0.2cm}
	\centering
	\scalebox{1.0}{
		\begin{tabular}{lccccccccccc}
			\toprule
			milliseconds											&100	&200	&300	&400	&500	&600 &700 &800 &900 &1000 &Average  \\ 
			\midrule
			
			
			{TIM~\cite{Lebailly20}}				   &6.06 &12.39 &19.83 &29.35 &41.80 &56.91 &73.17 &89.23 &104.31 &118.20    &51.13 \\
			{MSR-GCN~\cite{Lingwei21}}		&9.02 &17.02 &24.79 &33.26 &43.69 &56.34 &70.49 &85.00  &98.37   &109.73 &51.11  \\	
			{HRI-Itr~\cite{Mao20}}				   &2.21 &4.94 &9.51 &17.71 &30.93 &49.66 &72.95 &98.39 &122.93 &144.24  &50.41\\
			{HRI~\cite{Mao20}}						&5.34 &9.95 &15.08 &22.19 &32.45 &45.82 &61.29 &77.40 &92.47 &105.15    &43.17 \\
			{Ours}		&\textbf{1.31} &\textbf{4.31} &\textbf{9.49} &\textbf{17.33} &\textbf{27.42} &\textbf{39.85} &\textbf{54.22} &\textbf{70.20} &\textbf{86.23} &\textbf{100.09} &\textbf{37.57}\\	
			\bottomrule 
		\end{tabular}
		
	}  \\
	\caption[Comparison of our dyadic motion prediction approach with the state-of-the-art methods on the \lindyhop{} dataset]{\textbf{Comparison of our dyadic motion prediction approach with the state-of-the-art single person methods on the \lindyhop{} dataset.} We present the MPJPE for short-term ($<$ 500ms) and long-term ($>$ 500ms) motion prediction in mm. Despite the fast-paced and nonrepetitive nature of the dance moves, our method outperforms all the baselines for both short-term and long-term prediction. The best results in each column are shown in bold.}
	\label{table:sota_lhop}
\end{table*}


\subsection{Ablation Study}

\begin{table*}
	%\vspace{0.2cm}
	\centering
	\scalebox{1.0}{
		\renewcommand{\tabcolsep}{1.5mm}
		\begin{tabular}{lccccccccccc}
			\toprule
			milliseconds											&100	&200	&300	&400	&500	&600 &700 &800 &900 &1000 &Average  \\ 
			\midrule
			
			
			{HRI-Concat}	   &17.13 &33.99 &51.32 &69.89 &90.67 &113.41 &136.00 &156.10 &172.06 &183.40 &96.34\\	
			{Ours-SumPooling} &5.77&10.78&16.07&22.86&32.41&45.17&60.63&77.40&93.45&106.94&43.54\\
			{Ours-AvgPooling} &5.66&10.47&15.90&23.53&34.46&48.68&65.13&82.19&97.99&111.02&45.77\\
			{Ours-MaxPooling} &5.07&9.50&14.57&21.65&31.79&44.89&60.13&76.26&91.61&104.72&42.48\\
			{Ours-w/oPairwiseAtt} &3.60 &11.48 &25.08 &43.00 &62.22  &81.41 &100.25 &118.70 &135.48 &149.39 &68.04 \\	
			{Ours-w/o$\Delta$Pose}		&3.28 &8.36 &16.84 &23.87 &36.77 &52.22 &68.67 &85.02 &100.02 &112.07 &46.33\\	
			{Ours-EarlyMerge}		 &4.25 &8.11 &12.78 &19.25 &28.45 &40.84 &56.05 &73.11 &90.27 &105.40 &40.27\\		
			{Ours-w/SelfAttAux} &1.30 &5.04 &10.47 &18.12 &28.95 &42.41 &57.89 &74.52 &90.47 &104.09 &39.76\\
			{Ours-PairwiseAtt$\textbf{U}^{12}$ } &\textbf{1.17}&4.48&9.74&17.82&28.35&41.27&56.25&72.32&88.09&101.77&38.66\\
			{Ours}	&1.31 &\textbf{4.31} &\textbf{9.49} &\textbf{17.33} &\textbf{27.42} &\textbf{39.85} &\textbf{54.22} &\textbf{70.20} &\textbf{86.23} &\textbf{100.09} &\textbf{37.57}\\	\\	
			\bottomrule 
		\end{tabular}
		
	}  \\
	\caption[Ablation study for incorporating interactions]{\textbf{Ablation study for incorporating interactions.} We present the MPJPE for short-term ($<$ 500ms) and long-term ($>$ 500ms) motion prediction in mm. Here, we analyze different ways of incorporating interactions. HRI-Concat concatenates the motion history of the primary and auxiliary subject to treat them as one person. Ours-SumPooling, Ours-AvgPooling and Ours-MaxPooling use the social pooling layers from~\cite{Adeli20}. The remaining baselines show the benefits of the different components in our approach. Ours, depicted in Fig.~\ref{fig:overview_3dmotion_forecasting}, outperforms all other baselines and poses an effective way of handling coupled motion. The best results in each column are shown in bold.}
	\label{table:ablation_study_lhop}
	\vspace{-3mm}
\end{table*}

We evaluate the effect of modeling interactions via different strategies: \\
\textit{HRI-Concat} concatenates the motion history of the primary and auxiliary subject to treat them as one person. \\
\textit{Ours-SumPooling}, \textit{Ours-AvgPooling} and \textit{Ours-MaxPooling} discard the pairwise attention module, apply self-attention on the sequences of both subjects independently and combines the individual embeddings using the different pooling strategies proposed by~\cite{Adeli20}. The resulting vector is fed to the GCN decoder to predict the future poses of the primary subject. \\
\textit{Ours-w/oPairwiseAtt} excludes the pairwise attention module, applies self-attention and the GCN decoder on the sequences of both subjects independently and merges the GCN outputs from the two people to predict the future poses of the primary subject. \\
 \textit{Ours-w/o$\Delta$Pose} is our model which takes as input the past motion of the auxiliary subject directly instead of their relative motion to the primary subject.\\
 \textit{Ours-EarlyMerge} merges the pairwise embeddings $\textbf{U}^{12}$ and $\textbf{U}^{21}$ with the self-attention embedding of the primary subject $\textbf{U}^{1}$ before feeding them to the GCN module. \\
\textit{Ours-w/SelfAttAux} applies self-attention also on the sequence of the auxiliary subject and merges the result with the pairwise embeddings $\textbf{U}^{12}$ and $\textbf{U}^{21}$. \\
\textit{Ours-PairwiseAtt$\textbf{U}^{12}$ } excludes the pairwise attention that takes the keys and values from the auxiliary and the query from the primary subject. 
 

As can be seen in Table~\ref{table:ablation_study_lhop}, our method achieves the highest MPJPE in all timestamps. The comparison with \textit{HRI-Concat} shows that the naive way of combining the motion of the subjects is not an effective strategy to model their dependencies. The results of \textit{Ours-SumPooling}, \textit{Ours-AvgPooling} and \textit{Ours-MaxPooling} show that the social pooling layers proposed by~\cite{Adeli20} are suboptimal in the presence of strong interactions. The comparison to the remaining baselines evidence the benefits of the different components in our approach, which all contribute to the final results. 

\subsection{Limitations}
In Fig.~\ref{fig:qualitative_lhop_sota} and in the additional qualitative results, we observe that the lower arms and feet joints are usually difficult to predict and deviate the most from the ground-truth positions. Although Lindy Hop is a structured dance with highly correlated coupled motion, the dancers have their own styles. Therefore, predicting a single future is likely not to accurately match the body extremities which undergo the largest motion. This, however, can be overcome performing multiple diverse motion prediction, following a similar strategy to that used in~\cite{Yuan20,Aliakbarian21,Mao21b} for single-person motion prediction.

Another limitation of our model and many other motion prediction works in general is its use of complete sequences of ground-truth 3D poses as input. This may make our model sensitive to missing or faulty observations. To remedy this, as future work, we aim to incorporate the 3D poses obtained from the input images into our forecasting network and handle incomplete or noisy sequences to predict realistic future 3D poses for the interacting people.



\begin{comment}
\begin{figure}
\includegraphics[width=\linewidth]{figs/beyond_tss_lesion.pdf}
\caption[]{End-to-End runtime lesion study of the entire MNIST dataset and the FMA featurized music dataset. Each of DROP's contributions provides a runtime improvement.}
\label{fig:beyond_lesion}
\end{figure}
\end{comment}



\section{Conclusion}
\label{sec:conclusion}

Advanced data analytics techniques must scale to rising data volumes. 
DR techniques offer a powerful toolkit when processing these datasets, with PCA frequently outperforming popular techniques in exchange for high computational cost. 
In response, we propose DROP, a new dimensionality reduction optimizer. 
DROP combines progressive sampling, progress estimation, and online aggregation to identify high quality low dimensional bases via PCA without processing the entire dataset by balancing the runtime of downstream tasks and achieved dimensionality. 
Thus, DROP provides a first step in bridging the gap between quality and efficiency in end-to-end DR for downstream \red{analytics}. 

%We revisit canonical operators for time series dimensionality reduction and the measurement study of~\cite{keogh-study}, and show that PCA is more effective than popular alternatives in the data mining literature often by a margin of over $2\times$ on average on gold-standard time series benchmark data sets with respect to output data dimension. More surprisingly, we empirically demonstrate that a small number of samples are sufficient to accurately characterize directions of maximum variance and obtain a high-quality low-dimensional transformation.





%%%%%%%%% REFERENCES
{\small
\bibliographystyle{ieee_fullname}
\bibliography{vision}
}

\end{document}


\title{Improved algebraic fibrings}
\author{Sam P.~Fisher}
\email{sam.fisher@maths.ox.ac.uk}
\address{University of Oxford, United Kingdom}

\begin{document}
\maketitle

\begin{abstract}
We show that a virtually RFRS group $G$ of type $\mathtt{FP}_n(\Q)$ virtually algebraically fibres with kernel of type $\mathtt{FP}_n(\Q)$ if and only if the first $n$ $\ell^2$-Betti  numbers of $G$ vanish, that is, $b_p^{(2)}(G) = 0$ for $0 \leqslant p \leqslant n$. This confirms a conjecture of Kielak. We also offer a variant of this result over other fields, in particular in positive characteristic.

As an application of the main result, we show that virtually amenable RFRS groups of type $\mathtt{FP}(\Q)$ are polycyclic-by-finite. It then follows that if $G$ is a virtually RFRS group of type $\mathtt{FP}(\Q)$ such that $\Z G$ is Noetherian, then $G$ is polycyclic-by-finite. This answers a longstanding conjecture of Baer for virtually RFRS groups of type $\mathtt{FP}(\Q)$.
\end{abstract}

%%%                %%%
%%% Introduction 2 %%%
%%%                %%%

\section{Introduction}

A group $G$ is \textit{algebraically fibred} (or simply \textit{fibred}) if it admits a homomorphism onto $\Z$ with a finitely generated kernel. The interest in algebraic fibrings arose from the study of $3$-manifolds fibring over the circle. Using the long exact sequence of homotopy groups associated to a fibration, one sees that a surface bundle $M \longrightarrow S^1$, where $M$ is a compact $3$-manifold, induces an algebraic fibration $\pi_1(M) \longrightarrow \Z$. A celebrated theorem of Stallings \cite{Stallings3mflds} establishes the converse: if $G$ is isomorphic to the fundamental group of a closed, compact $3$-manifold $M$, then an algebraic fibration $G \longrightarrow \Z$ is induced by a surface bundle $M \longrightarrow S^1$.

In 2020, Kielak characterised the virtual algebraic fibring of residually finite rationally solvable (RFRS) groups in terms of the vanishing of the first $\ell^2$-Betti number. More precisely, he showed the following.

\begin{thm}[Kielak \cite{KielakRFRS}] \label{thm:kielak}
Let $G$ be an infinite, finitely generated virtually RFRS group. Then $G$ virtually algebraically fibres if and only if $b_1^{(2)}(G) = 0$.
\end{thm}

It should be noted that virtually RFRS groups are not hard to find ``in the wild," for example subgroups of right-angled Artin groups and right-angled Coxeter groups are virtually RFRS and, in particular, special groups (in the sense of Wise) are RFRS. Moreover, the RFRS property passes to subgroups and is preserved by taking products and free products of RFRS groups. Kielak's theorem is an algebraic analogue of the following theorem of Agol, which was a key step in confirming Thurston's Virtually Fibred Conjecture. 

\begin{thm}[Agol \cite{AgolCritVirtFib}]
Every compact, irreducible, orientable $3$-manifold $M$ with $\chi(M) = 0$ and nontrivial RFRS fundamental group admits a finite covering that fibres over the circle.
\end{thm}

Since algebraic fibrings induce topological fibrings of $3$-manifolds, Kielak's theorem generalises the above theorem of Agol by removing the assumption that $G$ is the fundamental group of a $3$-manifold. Note that \cite[Theorem 4.1]{Luck02} states that if $M$ is a compact, irreducible $3$-manifold with no $S^2$ boundary components, then $b_1^{(2)}(\pi_1(M)) = -\chi(M)$. Thus, we interpret the condition $b_1^{(2)}(G) = 0$ in Kielak's theorem as the algebraic analogue of the condition $\chi(M) = 0$ in Agol's theorem.

In light of Kielak's theorem, it is natural to ask whether the vanishing of higher $\ell^2$-Betti numbers of a group $G$ controls the higher finiteness properties of the kernel of the virtual fibration. Indeed, Kielak conjectured that a virtually RFRS group of type $\mathtt{FP}_n(\mathbb{Q})$ virtually algebraically fibres with kernel of type $\mathtt{FP}_n(\Q)$ if and only if $b_p^{(2)}(G)$ vanishes for all $p \leqslant n$ \cite[Conjecture 8]{KielakOber20}. The main result of this paper confirms Kielak's conjecture, and gives another characterisation of RFRS groups virtually fibring with kernel of type $\FP_n(\Q)$. We also note that the equivalence of (2) and (3) in the following theorem generalises \cite[Corollary 1.5]{JaikinZapirain2020THEUO}, where Jaikin-Zapirain proves the $n = 1$ case.

\begin{manualtheorem}{A}\label{thm:A}
Let $G$ be a virtually RFRS group of type $\mathtt{FP}_n(\Q)$. Then the following are equivalent:
    \begin{enumerate}[label = (\arabic*)]
        \item\label{item:b2vanish} $b_p^{(2)}(G) = 0$ for all $p \leqslant n$;
        \item\label{item:FPn} there is a finite-index subgroup $H \leqslant G$ and an surjection $\varphi \colon H \longrightarrow \Z$ such that $\ker \varphi$ is of type $\FP_n(\Q)$;
        \item\label{item:finiteBetti} there is a finite-index subgroup $H' \leqslant G$ and an surjection $\varphi' \colon H' \longrightarrow \Z$ such that $b_p(\ker \varphi') < \infty$ for all $p \leqslant n$.
    \end{enumerate}
\end{manualtheorem}

It should be emphasized that if $\ker \varphi'$ has finite Betti numbers in dimensions $\leqslant n$, it is not necessarily the case that $\ker \varphi'$ is of type $\FP_n(\Q)$. We prove a more general theorem that treats algebraic fibring with kernels of type $\FP_n(\mathbb F)$ for any skew-field $\mathbb F$, from which we obtain \cref{thm:A} as a special case.  Before stating the result, we give some background. If $\mathbb{F}$ is a skew-field and $G$ is a locally indicable group, then under certain conditions the group ring $\mathbb{F}G$ embeds into a skew-field $\mathcal{D}_{\mathbb{F}G}$ called the \textit{Hughes-free division ring} of $\mathbb{F}G$ (see \cref{def:HfreeDiv}). If it exists, $\mathcal{D}_{\mathbb{F}G}$ is unique up to $\mathbb{F}G$-algebra isomorphism \cite{HughesDivRings1970} and the \textit{$\mathcal{D}_{\mathbb{F}G}$-homology} of $G$ in dimension $p$ is defined to be $H_p^{\DF{G}}(G) := H_p(G; \DF{G})$. The $p$th \textit{$\mathcal{D}_{\mathbb{F}G}$-Betti number} is defined to be
\[
    b_p^{\mathcal{D}_{\mathbb{F}G}}(G) := \dim_{\mathcal{D}_{\mathbb{F}G}} H_p^{\mathcal{D}_{\mathbb{F}G}} (G).
\]
In \cite[Corollary 1.3]{JaikinZapirain2020THEUO}, Jaikin-Zapirain proves that if $G$ is finitely generated and RFRS, then $\mathcal{D}_{\mathbb{F}G}$ exists for any skew-field $\mathbb{F}$ and, if $\mathbb{F} = \mathbb{Q}$, it is isomorphic to the Linnell-skew field of $G$. For the purposes of this paper, it will not be necessary to know how the Linnell skew-field is defined, but that it can be used to define the $\ell^2$-homology and $\ell^2$-Betti numbers of a group $G$ (see \cref{def:l2b}). Importantly for us, when $G$ is finitely generated and RFRS, we have $b_p^{\mathcal{D}_{\Q G}} (G) = b_p^{(2)}(G)$ for all $p$. We prove the following theorem, which reduces to \cref{thm:A} in the case $\mathbb F = \Q$.

\begin{manualtheorem}{B}\label{thm:B}
    Let $\mathbb{F}$ be a skew-field and let $G$ be a virtually RFRS group of type $\FP_n(\mathbb F)$. Then the following are equivalent:
    \begin{enumerate}
        \item $b_p^{\DF{G}} = 0$ for all $p \leqslant n$;
        \item there is a finite-index subgroup $H \leqslant G$ and an surjection $\varphi \colon H \longrightarrow \Z$ such that $\ker \varphi$ is of type $\FP_n(\mathbb F)$;
        \item there is a finite index subgroup $H' \leqslant G$ and an surjection $\varphi' \colon H' \longrightarrow \Z$ such that $b_p(\ker \varphi'; \mathbb F) < \infty$ for all $p \leqslant n$.
    \end{enumerate}
\end{manualtheorem}

We highlight the following corollary, which implies, in particular that if $\mathbb F$ and $\mathbb F'$ are skew-fields of the same characteristic, then a RFRS group $G$ is $\DF{G}$-acyclic in dimension $\leqslant n$ if only if it is $\mathcal D_{\mathbb F'G}$-acyclic in dimensions $\leqslant n$. Moreover, it also implies that if $G$ is $\mathcal D_{\mathbb F_p G}$ acyclic in dimensions $\leqslant n$ for some prime $p$, then it is also $\ell^2$-acyclic in dimensions $\leqslant n$.

\begin{manualcor}{C}[\cref{cor:charac}]
    Let $G$ be a virtually RFRS group and let $n \in \N$.
    \begin{enumerate}
        \item If $\mathbb F$ and $\mathbb F'$ are skew-fields of the same characteristic, then $G$ virtually algebraically fibres with kernel of type $\FP_n(\mathbb F)$ if and only if it virtually algebraically fibres with kernel of type $\FP_n(\mathbb F')$.
        \item If $p$ is a prime such that $G$ virtually algebraically fibres with kernel of type $\FP_n(\mathbb F_p)$, then it virtually fibres with kernel of type $\FP_n(\Q)$.
    \end{enumerate}
\end{manualcor}

The final section of the paper is devoted to some applications of \cref{thm:A}. An interesting question is to determine general conditions under which amenable groups are elementary amenable. There are many examples of amenable groups that are not elementary amenable, for instance, Grigorchuk's group of intermediate growth, but it is not known whether there are examples of amenable groups of finite cohomological dimension that are not elementary amenable. Moreover, elementary amenable groups of finite cohomological dimension are virtually solvable by \cite[Lemma 2]{Hillman91} and \cite[Corollary 1]{HillmanLinnell}. This leads us to the following question, which first appeared in the work of Degrijse.

\begin{q}[Degrijse \cite{degrijse2016amenable}]
Are amenable groups of finite cohomological dimension over $\Z$ virtually solvable?
\end{q}

We obtain the following as an application of \cref{thm:A}, which provides evidence of a positive answer for virtually RFRS groups.

\begin{manualtheorem}{D}[\cref{thm:amRFRSelemAm}]\label{thm:C}
\sloppy If $G$ is a virtually amenable RFRS group of type $\mathtt{FP}(\Q)$, then $G$ is polycyclic-by-finite, and in particular virtually solvable.
\end{manualtheorem}

We can also confirm the following longstanding conjecture of Baer for virtually RFRS groups of type $\mathtt{FP}(\Q)$.

\begin{conj}
If $G$ is a group such that the group ring $\Z G$ is Noetherian, then $G$ is polycyclic-by-finite.
\end{conj}

Hall showed that polycyclic-by-finite groups have Noetherian group rings \cite[Theorem 4]{Hall59}, but it is still unknown whether the converse holds. Some progress was made recently by P.~Kropholler and Lorensen, who showed that if $RG$ is right Noetherian and $R$ is a domain, then $G$ is amenable and all of its subgroups are finitely generated \cite[Corollary B]{KrophollerLorensen19}. This result provides evidence for the conjecture, as the only known amenable groups in which every subgroup is finitely generated are polycyclic-by-finite. We obtain the following as a consequence of \cref{thm:C} and Kielak's appendix to \cite{BartholdiKielakApp}.

\begin{manualcor}{E}[\cref{cor:baer}]
    Let $G$ be a virtually RFRS group of type $\mathtt{FP}(\Q)$ such that $\Z G$ is Noetherian. Then $G$ is polycyclic-by-finite.
\end{manualcor}

Finally, we mention that Llosa-Isenrich, Martelli, and Py remarked in  \cite{isenrich2021hyperbolic} that an easy consequence of \cref{thm:A}, Agol's Theorem \cite{AgolHaken}, and work of Agol and Bergeron--Haglund--Wise \cite{BHW_2011} is the existence of hyperbolic groups containing type $\FP_{n-1}(\Q)$ subgroups that are not of type $\FP_n(\Q)$ for all $n \in \N$.

\begin{prop}[{\cite[Proposition 19]{isenrich2021hyperbolic}}]\label{prop:LMP}
    Let $\Gamma \in \PO(2n,1)$ be a hyperbolic arithmetic lattice of the simplest type. Then $\Gamma$ virtually fibres with kernel of type $\FP_{n-1}(\Q)$ but not of type $\FP_n(\Q)$.
\end{prop}

Note that if $G$ is virtually RFRS, of type $\FP(\Q)$, and $\ell^2$-acyclic, then \cref{thm:A} implies that $G$ fibres with kernel of type $\FP(\Q)$. Together with this easy observation, the same argument given by Llosa-Isenrich--Martelli--Py also shows that all such simplest type arithmetic lattices in $\PO(2n+1,1)$ virtually fibre with kernel of type $\FP(\Q)$. In a subsequent paper \cite{IsenrichPy2022}, Llosa-Isenrich and Py showed that there are hyperbolic arithmetic lattices of the simplest type in $\PU(n,1)$ that virtually fibre with kernel of type $\F_{n-1}$ but not of type $\FP_n(\Q)$, answering a question of Brady about the existence of subgroups of hyperbolic groups with exotic finiteness properties.





\subsection*{Structure of the paper} 

In \cref{sec:prelims} we introduce some of the main tools and objects that will be used throughout the paper. In particular, we define finiteness properties of groups, Hughes-free division rings, RFRS groups, and Ore localisation.

\cref{sec:valuations} recalls what we will need from the theory of valuations on free resolutions developed by Bieri and Renz in \cite{BieriRenzValutations}. The results in \cite{BieriRenzValutations} are stated for free resolutions over group rings with coefficients in $\Z$, however we will need them for coefficients in an arbitrary associative, unital ring $R$. There is no essential dependence on the ring, so our proofs are similar to those of Bieri and Renz after replacing $\Z$ with $R$.

In \cref{sec:horo}, we introduce the complex of horochains associated to a free resolution equipped with a valuation.

\cref{sec:SigInv} begins with the definition of the higher $\Sigma$-invariants $\Sigma_R^n(G; M)$ for a group $G$, a unital, associative ring $R$, and an $RG$-module $M$. Again, these were introduced in the case $R = \Z$ in \cite{BieriRenzValutations}, and reduce to the usual Bieri--Neumann--Strebel invariant when $n = 1$, $R = \Z$, and $M = \Z$ is the trivial $\Z G$-module. The rest of the section is devoted to the proof of \cref{thm:Main}, which gives equivalent characterisations of the invariants $\Sigma_R^n(G; M)$. The most important characterisation for our purposes is the following: $[\chi] \in \Sigma_R^n(G;M)$ if and only if $\Tor_i^{RG}(M, \widehat{RG}^\chi) = 0$ for all $i \leqslant n$, where $\widehat{RG}^\chi$ is the Novikov ring. When $n = 1$, this result is Sikorav's theorem \cite{SikoravThese} and is a key ingredient in Kielak's proof of \cref{thm:kielak}. The proof follows arguments given in \cite{BieriRenzValutations} and Schweitzer's appendix to \cite{BieriDeficiency}. \cref{thm:Main} is the main technical result that will be used in the proof of \cref{thm:agrarianMain}. 

In \cref{sec:homology} we introduce $\mathcal{D}_{\mathbb{F}G}$-homology and prove properties of $\mathcal{D}_{\mathbb{F}G}$-Betti numbers analogous to those of $\ell^2$-Betti numbers. Namely, we prove that 
\[
    [G : H] \cdot b_p^{\mathcal{D}_{\mathbb{F}G}}(G) = b_p^{\mathcal{D}_{\mathbb{F}H}}(H)
\]
(\cref{lem:JBscales}) whenever $\mathcal{D}_{\mathbb{F}G}$ exists and $H$ is a finite index subgroup of $G$. In \cref{thm:JBSES}, we show that if $b_p^{\mathcal{D}_{\mathbb{F}K}}(K) < \infty$ and $G$ fits into a short exact sequence $1 \longrightarrow K \longrightarrow G \longrightarrow \Z \longrightarrow 1$, then $b_p^{\mathcal{D}_{\mathbb{F}G}}(G) = 0$. This should be thought of as an analogue of \cite[Theorem 7.2]{Luck02} for $\mathcal{D}_{\mathbb{F}G}$-Betti numbers. We then prove the main result, \cref{thm:agrarianMain}, and obtain \cref{thm:b2rfrs} as a special case. In \cref{thm:finiteBetti}, we show that virtually fibring with kernel of type $\FP_n(\Q)$ is equivalent to having a virtual map to $\Z$ whose kernel has finite Betti numbers in dimensions $\leqslant n$.

We conclude with the proofs of \cref{thm:amRFRSelemAm} and \cref{cor:baer} in \cref{sec:app}, and mention related work of Llosa Isenrich, Martelli, and Py.

\subsection*{Acknowledgements.} The author would like to thank Dawid Kielak for numerous helpful conversations and comments on this paper, Peter Kropholler for a helpful correspondence, and Sam Hughes for pointing out the application of \cref{thm:amRFRSelemAm} to \cref{cor:baer}.

This work has received funding from the European Research Council (ERC) under the European Union's Horizon 2020 research and innovation programme (Grant agreement No. 850930).









%%%               %%%
%%% Preliminaries %%%
%%%               %%%
\section{Preliminaries} \label{sec:prelims}

\begin{rem*}
In the sequel, all rings will be associative and unital with $1 \neq 0$.
\end{rem*}

%%% Finiteness properties
\subsection{Finiteness properties}

\begin{defn}[type $\mathtt{FP}_n$]
Let $R$ be a ring and $M$ be a left $R$-module. We say that $M$ is of \textit{type $\mathtt{FP}_n$}, and write $M \in \mathtt{FP}_n$, if $M$ has a projective resolution
\[
\cdots \longrightarrow P_{n+1} \longrightarrow P_n \longrightarrow \cdots \longrightarrow P_1 \longrightarrow P_0 \longrightarrow M \longrightarrow 0
\]
by left $R$-modules, where $P_j$ is finitely generated for $j \leqslant n$. If we want to specify the ring, we say that $M$ is of \textit{type $\mathtt{FP}_n$ over $R$} and write $M \in \mathtt{FP}_n(R)$.  If $P_j = 0$ for $j > n$, then we write $M \in \mathtt{FP}(R)$.

A group $G$ is of \textit{type $\mathtt{FP}_n$ over $R$} if the trivial $RG$-module $R$ is of type $\mathtt{FP}_n(RG)$; in this case we write $G \in \mathtt{FP}_n(R)$. Similarly, if the trivial $RG$-module $R$ is of type $\mathtt{FP}(RG)$, then we write $G \in \mathtt{FP}(R)$.
\end{defn}

We will often use the fact that an $R$-module $M$ is of type $\mathtt{FP}_n$ if and only if there is a free resolution $\cdots \longrightarrow F_{n+1} \longrightarrow F_n \longrightarrow \cdots \longrightarrow F_0 \longrightarrow M \longrightarrow 0$ with $F_j$ finitely generated for $j \leqslant n$ \cite[VIII Proposition 4.3]{BrownGroupCohomology}. Note that the analogous fact does not hold for type $\mathtt{FP}$.

The following definition will not be needed until \cref{sec:app}.

\begin{defn}[cohomological dimension]
A resolution $\cdots \longrightarrow P_1 \longrightarrow P_0 \longrightarrow M \longrightarrow 0$ of an $R$-module $M$ has \textit{length} $n$ if $P_n \neq 0$ and $P_m = 0$ for $m > n$. A group $G$ has \textit{cohomological dimension $n$ over $R$} if $n$ is the shortest length of any projective resolution of the trivial $RG$-module $R$. In this case we will write $\cd_R (G) = n$. If $R$ has no finite-length projective resolution, then $\cd_R(G) = \infty$.
\end{defn}

Note that $G \in \mathtt{FP}(R)$ implies that $G$ has finite cohomological dimension over $R$, but not conversely.



%%% Hughes free division rings
\subsection{Hughes-free division rings}

We are following Jaikin-Zapirain's exposition of this material; in particular, \cref{def:HfreeDiv,def:crossedprod} are taken from \cite{JaikinZapirain2020THEUO}.

\begin{defn}[Hughes-free division rings] \label{def:HfreeDiv}
Let $\mathbb{F}$ and $\mathcal{D}$ be skew-fields, let $G$ be a locally indicable group, and let $\varphi \colon \mathbb{F} G \longrightarrow \mathcal{D}$ be a ring homomorphism. Then the pair $(\mathcal{D}, \varphi)$ is \textit{Hughes-free} if
\begin{enumerate}[label=(\arabic*)]
    \item\label{item:epic} $\mathcal{D}$ is the skew-field generated by $\varphi(\mathbb{F} G)$, i.e., there are no intermediate skew-fields between $\varphi(\mathbb{F}G)$ and $\mathcal{D}$;
    \item\label{item:direct} for every nontrivial finitely generated subgroup $H \leqslant G$, every normal subgroup $N \triangleleft H$ such that $H/N \cong \Z$, and every set of elements $h_1, \dots, h_n \in H$  lying in pairwise distinct cosets of $N$, the sum
    \[
    \langle\varphi(\mathbb{F}N)\rangle \cdot \varphi(h_1) + \cdots + \langle\varphi(\mathbb{F}N)\rangle \cdot \varphi(h_n)
    \]
    is direct, where $\langle\varphi(\mathbb{F}N)\rangle$ is the sub-skew-field of $\mathcal{D}$ generated by $\varphi(\mathbb{F}N)$.
\end{enumerate}
\end{defn}

Hughes showed that if such a pair $(\mathcal{D},\varphi)$ exists, then $\mathcal{D}$ is unique up to $\mathbb{F}G$-algebra isomorphism \cite{HughesDivRings1970}. Thus, we denote $\mathcal{D}$ by $\mathcal{D}_{\mathbb{F}G}$. Let $H \leqslant G$ be a subgroup and suppose that $\mathcal{D}_{\mathbb{F}G}$ exists. Then $\mathcal{D}_{\mathbb{F}H}$ exists as well and is equal to $\langle \varphi(\mathbb{F}H) \rangle \subseteq \mathcal{D}_{\mathbb{F}G}$. Hence, we will view $\mathcal{D}_{\mathbb{F}H}$ as a subset of $\mathcal{D}_{\mathbb{F}G}$ whenever $H$ is a subgroup of $G$. Moreover, Gr\"ater showed that $\mathcal{D}_{\mathbb{F}G}$ is \textit{strongly Hughes-free} whenever it exists \cite[Corollary 8.3]{Grater20}, which is to say that condition (2) above can be replaced with the following:
\begin{enumerate}[label = {($2^\prime$)}]
    \item\label{item:2prime} for every nontrivial subgroup $H \leqslant G$, every normal subgroup $N \triangleleft H$, and every set of elements $h_1, \dots, h_n \in H$  lying in pairwise distinct cosets of $N$, the sum
    \[
    \langle\varphi(\mathbb{F}N)\rangle \cdot \varphi(h_1) + \cdots + \langle\varphi(\mathbb{F}N)\rangle \cdot \varphi(h_n)
    \]
    is direct.
\end{enumerate}


\begin{defn}[Crossed products] \label{def:crossedprod}
A ring $R$ is \textit{$G$-graded} if its underlying Abelian group is isomorphic to a direct sum $\bigoplus_{g \in G} R_g$ of Abelian groups $R_g$ and $R_g R_h \subseteq R_{gh}$ for all $g,h \in G$. If $R_g$ contains a unit for each $g \in G$, then we say that $R$ is a \textit{crossed product} of $R_e$ and $G$, and we write $R = R_e * G$.
\end{defn}


Strong Hughes-freeness implies the following useful properties, which are stated in \cite{JaikinZapirain2020THEUO}.

\begin{prop}\label{prop:twistedNormalSkew}
Let $G$ be a locally indicable group such that $\mathcal{D}_{\mathbb{F}G}$ exists, let $\varphi$ be as in \cref{def:HfreeDiv}, and $N \triangleleft G$ be a normal subgroup. If $R$ is the subring of $\mathcal{D}_{\mathbb{F}G}$ generated by $\mathcal{D}_{\mathbb{F}N}$ and $\varphi(\mathbb{F}G)$, then
\begin{enumerate}[label = (\arabic*)]
    \item\label{item:subring} $R \cong \mathcal{D}_{\mathbb{F}N} * (G/N)$;
    \item\label{item:twistedFiniteIndex} if $[G:N] < \infty$, then $\mathcal{D}_{\mathbb{F}G} = R$.
\end{enumerate}
\end{prop}

\begin{proof}

Starting with \ref{item:subring}, let $\{t_i\}_{i \in I}$ be a transversal for $N$ in $G$. We claim that every element of $R$ can be written as a finite sum $\sum_i \alpha_i \varphi(t_i)$, where $\alpha_i \in \mathcal{D}_{\mathbb{F}N}$ for each $i \in I$. Since every element of $G$ is of the  form $n t_i$ for some $n \in N$ and $i \in I$, it is enough to show that $\varphi(t_i) \alpha$ can be written in the desired form for any $\alpha \in \mathcal{D}_{\mathbb{F}N}$. Moreover, $\mathcal{D}_{\mathbb{F}N}$ is the division closure of $\varphi(\mathbb{F}N)$, so it is enough to prove the cases where $\alpha = \varphi(a)$ or $\alpha = \varphi(a)\inv$ for some $a \in \mathbb{F}N$.

Let $a = \sum_j f_j n_j$, where $f_j \in \mathbb{F}$ and $n_j \in N$ for each $j$. Then
\[
    t_i \cdot a = t_i \cdot \sum_j f_j n_j = \sum_j f_j t_i n_j = \sum_j f_j n_j^{t_i} \cdot t_i,
\]
for each $i$, where $g^t = t g t\inv$ for any group elements $g, t \in G$. Since $N$ is normal in $G$, this shows that $\varphi(t_i) \varphi(a)$ can be written in the desired form. 

For the other case, we have $\varphi(t_i) \varphi(a)\inv = \varphi(a t_i\inv)\inv$. Then
\[
    a \cdot t_i\inv = \sum_j f_j n_j t_i\inv = t_i\inv \cdot \sum_j f_j n_j^{t_i}
\]
and therefore we obtain $\varphi(t_i) \varphi(a)\inv = \varphi(a \cdot t_i\inv)\inv = \varphi(\sum_j f_j n_j^{t_i})\inv \varphi(t_i)$. This proves the claim.

By strong Hughes-freeness, every element of $R$ has a unique decomposition of the form $\sum_i \alpha_i \varphi(t_i)$, with $\alpha_i \in \mathcal{D}_{\mathbb{F}N}$ for each $i$, and therefore we obtain an isomorphism $R \cong \bigoplus_{i = 1}^n \mathcal{D}_{\mathbb{F}N} \varphi(t_i)$ of Abelian groups. Moreover, $R$ is $G/N$-graded with respect to this decomposition, so $R$ is a crossed product $\mathcal{D}_{\mathbb{F}N} * (G/N)$. 

\smallskip

To prove \ref{item:twistedFiniteIndex}, we recall the fact that a finite algebra over a skew-field with no zero-divisors is a skew-field. Hence, if $[G:N] < \infty$, then $R \cong \mathcal{D}_{\mathbb{F}N} * (G/N)$ is finitely finite over $\mathcal{D}_{\mathbb{F}N}$ and is therefore a skew-field. But $\mathcal{D}_{\mathbb{F}G}$ is the smallest skew-field containing $\varphi(\mathbb{F}G)$, so $\mathcal{D}_{\mathbb{F}G} = R$. \qedhere
\end{proof}






%%% RFRS groups

\subsection{RFRS groups}

Residually finite rationally solvable (RFRS) groups were defined by Agol in \cite{AgolCritVirtFib} in order to show that certain hyperbolic $3$-manifolds virtually fibre over the circle. Let $G$ be a group and let $G^\mathsf{ab} := G/[G,G]$ be its Abelianisation. Since $G^\mathsf{ab}$ is Abelian, it is canonically a $\Z$-module so we can form the tensor product $\Q \otimes_\Z G^\mathsf{ab}$, and there is a group homomorphism $G \longrightarrow \Q \otimes_\Z G^{\mathsf{ab}}$ sending $g \in G$ to $1 \otimes g[G,G]$.
\begin{defn} \label{def:RFRS}
A group $G$ is \textit{RFRS} if 
\begin{enumerate}[label = (\arabic*)]
    \item there is a chain $G = G_0 \geqslant G_1 \geqslant G_2 \geqslant \cdots$ of finite index normal subgroups of $G$ such that $\bigcap_{i=0}^\infty G_i = \{1\}$
    \item $\ker(G_i \longrightarrow \Q \otimes_\Z G_i^{\mathsf{ab}}) \leqslant G_{i+1}$ for every $i \geqslant 0$.
\end{enumerate}
\end{defn}

The following fact that will be used in the proof of \cref{thm:agrarianMain}.

\begin{prop}\label{prop:RFRStoQ}
If $G$ is a nontrivial RFRS group, then $H^1(G;\R) \neq 0$.
\end{prop}
\begin{proof}
Let $G = G_0 \geqslant G_1 \geqslant G_2 \geqslant \cdots$ be a residual chain of finite index subgroups as in \cref{def:RFRS} and let $V = \Q \otimes_\Z G^\mathsf{ab}$. Moreover, assume that $G \neq G_1$ so that the map $G \longrightarrow V$ cannot be trivial. Let $x \in V$ be a nontrivial element in the image of $G \longrightarrow V$; since $V$ is a $\Q$-vector space, there is a linear map $\varphi \colon V \longrightarrow \Q$ such that $\varphi(x) = 1$. Then the composition 
\[
    G \longrightarrow V \xrightarrow{\ \varphi \ } \Q \longhookrightarrow \R
\]
is a nontrivial element of $H^1(G;\R)$.
\end{proof}



%%% Ore localisation

\subsection{Ore localisation}

Ore localisation is an analogue of usual localisation for noncommutative rings. Let $R$ be a ring and let $S$ be its set of non-zerodivisors. Then $R$ satisfies the \textit{Ore condition} if for every $r \in R$ and $s \in S$ there  are elements $p,p' \in R$ and $q,q' \in S$ such that 
\[
    qr = ps \ \textnormal{and} \ rq' = sq'.
\]
If  $S = R \setminus \{0\}$ and $R$ satisfies the Ore condition, then it is called an \textit{Ore domain}, and we can form its \textit{Ore localisation} $\Ore(R)$ as follows. Define an equivalence relation $\sim_R$ on $R \times S$ by declaring that $(r,s) \sim_R (r',s')$ if and only if there are elements $a,b \in S$ such that
\[
    ra = r'b \ \textnormal{and} \ sa = s'b.
\]
The equivalence class of $(r,s)$ under $\sim_R$ is denoted $r/s$ and called a \textit{right fraction}. Then $\Ore(R)$ is defined to be the set of right fractions. We can similarly define an equivalence relation $\sim_L$ and define $\Ore(R)$ as a set of left fractions. The Ore condition ensures that these two constructions are isomorphic and indicates how to convert right fractions into left fractions and vice versa. For a detailed construction of $\Ore(R)$ and the definition of addition and multiplication making this set into a ring, we refer the reader to Section 4.4 of Passman's book \cite{PassmanGrpRng}.

The facts about Ore localisation that we will use are summarised in the following proposition.

\begin{prop}\label{prop:OreLoc}
Let $R$ be an Ore domain. Then
\begin{enumerate}[label = (\arabic*)]
    \item $\Ore(R)$ is a skew-field;
    \item for every $r \in R$, we have $r / 1 = 1 \backslash r$ and the map $R \longrightarrow \Ore(R), r \longmapsto r/1 = 1 \backslash r$ is an injective ring homomorphism;
    \item \cite[Proposition 2.2(2)]{JaikinZapirain2020THEUO} if $G$ is a group and $\mathbb{F}$ is a skew-field such that $\mathcal{D}_{\mathbb{F}G}$ exists and $K$ is a normal subgroup such that $G / K \cong \Z$, then  $\mathcal{D}_{\mathbb{F}G} \cong \Ore(\mathcal{D}_{\mathbb{F}K} * (G/K))$.
\end{enumerate}
\end{prop}













%%%            %%%
%%% VALUATIONS %%%
%%%            %%%
\section{Valuations on free resolutions} \label{sec:valuations}

In this section, we introduce valuations on free resolutions over a group ring. We will be very closely following Bieri and Renz \cite{BieriRenzValutations} where the theory is developed in the case where the ring is $\Z$. Their proofs go through without change when $
\Z$ is replaced by an arbitrary ring $R$.

Let $R$ be a ring, $G$ a group, and $M$ a left $RG$-module. A \textit{free resolution} of $M$ is an exact sequence
\[
    \cdots \xrightarrow{\partial_{n+1}} F_n \xrightarrow{\partial_n} F_{n-1} \xrightarrow{\partial_{n-1}} \cdots \xrightarrow{\partial_1} F_0 \xrightarrow{\partial_0}M \longrightarrow 0
\]
of left $RG$-modules, where $F_i$ is free for all $i \geqslant 0$. We will usually omit the subscripts on the boundary maps $\partial_n$ and denote the free resolution by $F_\bullet \longrightarrow M \longrightarrow 0$. Let $F$ be the free $RG$-module $\bigoplus_{i = 0}^\infty F_i$, and define the \textit{$n$-skeleton} of $F$ to be $F^{(n)} := \bigoplus_{i = 0}^n F_i$. The elements of $F$ are called \textit{chains}, so a chain is not necessarily an element of $F_i$ for any $i$ in our context. Fixing a basis $X_i$ for each $F_i$, we note that $X := \bigcup_{i=0}^\infty X_i$ is a basis for $F$ and $X^{(n)} := \bigcup_{i = 0}^n X_i$ is a basis for $F^{(n)}$. The resolution $F_\bullet \longrightarrow M \longrightarrow 0$ is \textit{admissible with respect to $X$} if $\partial x \neq 0$ for every $x \in X$. We will always assume that our free resolutions are admissible with respect to the basis we are working with. This is not a strong requirement, since if all boundary maps are nonzero, then $F$ has a  basis with respect to which $F_\bullet \longrightarrow M \longrightarrow 0$ is admissible; otherwise we can truncate the resolution and choose a basis to obtain an admissible resolution of finite length. We also define the \textit{support} of a chain $c \in F$ (with respect to $X$), denoted $\supp_X(c)$, as follows: every chain $c \in F$ can be written uniquely as $\sum_{g\in G, x \in X} r_{g,x} gx$, where $r_{g,x} \in R$. Then $\supp_X (c) := \{ gx : r_{g,x} \neq 0 \}$; we will usually drop the subscript $X$ when the basis is understood.

Let $\chi \colon G \longrightarrow \R$ be a nontrivial \textit{character}, that is, a nonzero group homomorphism from $G$ to the additive group $\R$. This provides the elements of $G$ with a notion of height, which we now extend to the chains of $F$. Let $\R_\infty = \R \cup \{\infty\}$, where $\infty$ is an element such that $t < \infty$ for every $t \in \R$.  We construct a function $v_X \colon F \longrightarrow \R_{\infty}$ via the following inductive procedure. For an element $c \in F_0$, define $v_X(c) = \inf\{ \chi(g) : gx \in \supp(y) \}$. Let $n > 0$ and assume that we have defined $v_X$ on $F_{n-1}$. For $x \in X_n$, let $v_X(x) := v_X(\partial x)$. For $c \in F_n$, set $v_X(c) = \inf\{ \chi(g) + v_X(x) : gx \in \supp(y) \}$. For an arbitrary $c \in F$, write $c = \sum_i c_i$, where $c_i \in F_i$, and define $c = \inf_i\{ v_X(c_i) \}$. The function $v_X$ is called the \textit{valuation extending $\chi$ with respect to $X$}. It is clear from the definition that $v_X(c) = \inf\{ \chi(g) + v_X(x) : gx \in \supp(c) \}$ for any chain $c \in F$. Again, we will usually drop the $X$ in the subscript when the basis is understood.


%%%     PROPERTIES OF v     %%%
\begin{prop} \label{PropPropsOfVal}
For the valuation $v_X = v \colon F \longrightarrow \R_\infty$ defined above and for any $c, c' \in F$ and $g \in G$, we have
\begin{enumerate}[label=(\arabic*)]
    \item\label{item:sum} $v(c + c') \geqslant \min\{ v(c), v(c') \}$;
    \item\label{item:scalar} $v(c) \leqslant v(rc)$ for all $r \in R$, and $v(c) = v(rc)$ if $r$ is not a zerodivisor;
    \item\label{item:sumequal} if $v(c) \neq v(c')$, then $v(c + c') = \min\{ v(c), v(c') \}$;
    \item\label{item:groupelement} $v(g c) = \chi(g) + v(c)$;
    \item\label{item:infheight} $v(c) = \infty$ if and only if $c = 0$;
    \item\label{item:bdyincrease} if $c \in \bigoplus_{i \geqslant 1} F_i$, then  $v(\partial c) \geqslant v(c)$.
\end{enumerate}
\end{prop}

\begin{proof}
\ref{item:sum} follows from the fact that $\supp(c+c') \subseteq \supp(c) \cup \supp(c')$. The first part of \ref{item:scalar} follows from the fact that $\supp(c) \supseteq \supp(rc)$. If $r$ is not a zerodivisor then $\supp(c) = \supp(rc)$, which yields the second statement of \ref{item:scalar}.

To prove \ref{item:sumequal}, assume without loss of generality that $v(c) < v(c')$. Then,
\[
    v(c) = v((c+c') - c') \geqslant \min\{ v(c+c'), v(-c') \} = \min\{ v(c+c'), v(c') \}.
\]
Since we assumed that $v(c) < v(c')$, the previous line implies that $\min\{ v(c+c'), v(c') \} = v(c + c')$; hence, $v(c) = \min\{ v(c), v(c') \} \geqslant v(c + c')$. But $v(c + c') \geqslant \min\{v(c),v(c')\}$ by \ref{item:sum}, so we obtain \ref{item:sumequal}. 

For \ref{item:groupelement}, we have
\begin{align*}
    v(gc) &= \inf\{ \chi(h) + v(x) : hx \in \supp(gc) \} \\
    &= \inf\{ \chi(g (g\inv h)) + v(x) : hx \in g \cdot \supp c \} \\
    &= \chi(g) + \inf\{ \chi(g\inv h) + v(x) : (g\inv h)x \in \supp c \} \\
    &= \chi(g) + v(c).
\end{align*}

For \ref{item:infheight}, we first show that if $c \in F_n \setminus \{0 \}$, then $v(c) < \infty$ by induction on $n$. This is true for $n = 0$ since $\chi(G) \subseteq \R$. Now let $n > 0$. Since $c \neq 0$, there is some element $gx$ in its support, where $g \in G$ and $x \in X$. Then
\[
    v(c) \leqslant v(gx) = \chi(g) + v(x) = \chi(g) + v(\partial x) < \infty
\]
by the inductive hypothesis and by admissibility of $F_\bullet \longrightarrow M \longrightarrow 0$ with respect to $X$. For a general nonzero element $c \in F$, write $c = \sum_i c_i$ with $c_i \in F_i$. Then $v(c) = \inf_i\{ v(c_i) \} < \infty$ since at least one of the chains $c_i$ is nonzero. Conversely, if $c = 0$, then $v(c) = \infty$, since the infimum of the empty set is $\infty$.

For \ref{item:bdyincrease}, let $c = \sum_{g \in G, x \in X} r_{g,x} gx \in \bigoplus_{i \geqslant 1} F_i$. Then
\begin{align*}
    v(\partial c) &= v\left(\partial\left(\sum_{g \in G, x \in X} r_{g,x} gx\right)\right) \\
    &= v\left( \sum_{g \in G, x \in X} r_{g,x} g \partial x \right) \\
    &\geqslant \inf\left\{ v(r_{g,x} g\partial x) : gx \in \supp(c) \right\} \tag{by \ref{item:sum}} \\
    &\geqslant \inf\left\{ v( g\partial x) : gx \in \supp(c) \right\}  \tag{by \ref{item:scalar}} \\
    &= \inf\{ \chi(g) + v(\partial x) : gx \in \supp(c) \} \\
    &= \inf\{ \chi(g) + v(x) : gx \in \supp(c) \} \\
    &= v(c). \qedhere
\end{align*}
\end{proof}








\begin{defn}[valuation subcomplex and essential acyclicity]
Given an admissible free resolution $F_\bullet \longrightarrow M \longrightarrow 0$ over $RG$ (with respect to some fixed basis $X$), a non-trivial character $\chi \colon G \longrightarrow \R$, and the valuation $v \colon F \longrightarrow \R_\infty$ extending $\chi$, define the \textit{valuation subcomplex} of $F$ with respect to $v$ to be the chain complex $\cdots \longrightarrow F_n^v \longrightarrow \cdots \longrightarrow F_0^v \longrightarrow M \longrightarrow 0$, where $F_n^v = \{c \in F_n : v(c) \geqslant 0\}$. We denote the valuation subcomplex by $F_\bullet^v \longrightarrow M \longrightarrow 0$ and let $F^v := \bigoplus_{i = 0}^\infty F_i^v$. \cref{PropPropsOfVal}\ref{item:bdyincrease} ensures that $F_\bullet^v \longrightarrow M \longrightarrow 0$ is a chain complex of left $RG_\chi$-modules, where $G_\chi$ is the monoid $\{g \in G : \chi(g) \geqslant 0\}$. It is not hard to show that each $F_i^v$ is a free $RG_\chi$-module and has an $RG_\chi$-basis of cardinality $\verti{X_i}$, where $X_i$ is an $RG$-basis for $F_i$.

The chain  complex $F_\bullet^v \longrightarrow M \longrightarrow 0$ is \textit{essentially acyclic in dimension $n$} if there is a real number $D \geqslant 0$ such that for every cycle $z \in F_n^v$ there is a $c \in F_{n+1}$ with $\partial c = z$ and $D \geqslant v(z) - v(c)$. We extend the definition of essential acyclicity to dimension $-1$ by declaring that $v(m) = 0$ for all $m \in M \setminus \{0\}$.
\end{defn}

The definition of essential acyclicity in dimension $n$ is equivalent to the following seemingly weaker condition: for every cycle $z \in F_n^v$, there is a $c \in F_{n+1}$ such that $\partial c = z$ and $v(c) \geqslant -D$. To see this, let $z \in F_n^v$ be a cycle. It is easily shown that $v(F) \subseteq \chi(G) \cup \{\infty\}$, so there is a $g \in G$ such that $\chi(g) = v(z)$. Since $g\inv z$ is also a in $F_n^v$ with $v(g\inv z) = 0$, there is some $c \in F_{n+1}^v$ such that $\partial c = g\inv z$ and $v(c) \geqslant -D$. Thus, $\partial (gc) = z$, and $D \geqslant v(z) - v(gc)$.



%%%            %%%
%%% HOROCHAINS %%%
%%%            %%%
\section{Horochains} \label{sec:horo}



\begin{defn}[complex of horochains and horo-acyclicity]
Let $F_\bullet \longrightarrow M \longrightarrow 0$ be an admissible free resolution with respect to some basis $X$, let $\chi \colon G \longrightarrow \R$ be a nontrivial character, and let $v \colon F \longrightarrow \R_\infty$ be the valuation extending $\chi$. Define $\widehat{F}$ to be the left $RG$-module of chains that are finitely supported below every height. More precisely, $\widehat{F}$ is the $RG$-module of formal sums $\sum_{g \in G, x \in X} r_{g,x} g x$ such that $\{ gx : v(gx) \leqslant t, r_{g,x} \neq 0 \}$ is finite for every $t \in \R$. The elements of $\widehat{F}$ are called \textit{horochains}. If $\hat{c} \in \widehat{F}$, then its \textit{support} is $\supp_X(\hat{c}) := \{ gx : r_{g,x} \neq 0 \}$. Let $\widehat{F}_i \subseteq \widehat{F}$ be the subset of chains with support in $F_i$ and let $\widehat{F}^{(n)} := \bigoplus_{i = 0}^n \widehat{F}_i$. \cref{PropPropsOfVal}\ref{item:bdyincrease} guarantees that $\partial \colon F_n \longrightarrow F_{n-1}$ extends to a map $\partial \colon \widehat{F}_n \longrightarrow \widehat{F}_{n-1}$ in the obvious way, so we get a complex $\cdots \longrightarrow \widehat{F}_n \longrightarrow \cdots \longrightarrow \widehat{F}_0 \longrightarrow 0$. Note that $\widehat{F}$ is not equal to $\bigoplus_{i = 0}^\infty \widehat{F}_i$ since the support of a horochain might intersect infinitely many of the modules $\widehat{F}_i$. We say that $F_\bullet \longrightarrow M \longrightarrow 0$ is \textit{horo-acyclic} in dimensions $n \geqslant 0$ with respect to $v$ if the chain complex $\cdots \longrightarrow \widehat{F}_1 \longrightarrow \widehat{F}_0 \longrightarrow 0$ is acyclic in dimension $n$.
\end{defn}

We can extend the definition of $v$ to $\widehat{F}$ by defining $v(\hat{c}) := \inf\{ v(gx) : \supp(\hat{c})\}$ for any horochain $\hat{c}$. If $\hat{c} \neq 0$, then $\{ v(gx) : \supp(\hat{c})\}$ is nonempty and attains a minimum because chains are finitely supported below any given height.  Properties \ref{item:sum} through \ref{item:infheight} of \cref{PropPropsOfVal} hold in this setting with the same proofs. 

A version of \cref{PropPropsOfVal}\ref{item:bdyincrease} holds for horochains, namely we have $v(\partial \hat{c}) \geqslant v(\hat{c})$ for all horochains $\hat{c}$, but we need to modify the proof: If $\hat{c} = 0$, then the claim is clear. Otherwise, let $\hat{c} \neq 0$ be a horochain, and let $gx \in \supp(\hat{c})$ be such that $v(gx) = v(\hat{c})$. By the finite version of \ref{item:bdyincrease}, we have that $v(\partial g'x') \geqslant v(g'x') \geqslant v(gx)$ for every $g'x' \in \supp(\hat{c})$. Since every $g''x'' \in \supp(\partial \hat{c})$ is contained in $\supp(\partial g'x')$ for some $g'x' \in \supp(\hat{c})$, we have that $v(g''x'') \geqslant v(\partial g'x') \geqslant v(gx)$ for every $g''x'' \in \supp(\partial \hat{c})$. Thus, $v(\partial \hat{c}) \geqslant v(\hat{c})$. 


\smallskip

The following lemma will be used in the proof of \cref{thm:Main}.

\begin{lem} \label{lem:ExtendHom}
Let $F_\bullet \longrightarrow M \longrightarrow 0$ (resp.~$F_\bullet' \longrightarrow M \longrightarrow 0$) be a free resolution over $RG$ admissible with respect to a basis $X$ (resp.~$X'$), and let $v$ (resp.~$v'$) be the valuation extending a nontrivial character $\chi\colon G \longrightarrow \R$. Suppose that $F^{(n)}$ is finitely generated, and that $\varphi \colon F \longrightarrow F'$ is a homomorphism of $RG$-modules. Then
\begin{enumerate}[label=(\arabic*)]
    \item \label{item:extend} $\varphi$ induces a homomorphism of left $RG$-modules given by
    \[
        \widehat{\varphi} \colon \widehat{F}^{(n)} \longrightarrow \widehat{F}', \ \sum r_{g,x} gx \longmapsto \sum r_{g,x} g \varphi(x) \ \ ;
    \]
    \item \label{item:valIneq} $v'(\widehat{\varphi}(\hat c)) \geqslant v(\hat{c}) + \min_{x \in X^{(n)}} \{ v'(\varphi(x)) - v(x) \}$ for every $\hat{c} \in F^{(n)}$.
\end{enumerate}
\end{lem}

\begin{proof}
For \ref{item:extend}, we need to show that $\widehat{\varphi}(\hat{c})$ is a horochain  for any horochain $\hat{c} \in \widehat{F}^{(n)}$. To this end, let $\hat{c} = \sum r_{g,x} gx$, and note that there are only finitely many elements $x \in X$ such that $gx \in \supp_X(\hat{c})$. If $\widehat{\varphi}(\hat{c})$ is not a horochain, then the set $\{ gx \in \supp_X(\hat{c}) : v'(g\varphi(x)) \leqslant t \}$ is infinite for some $t \in \R$. Since $F^{(n)}$ is finitely generated, there is some fixed $y \in X$ such that $v'(g\varphi(y)) \leqslant t$ and $gy \in \supp_X(\hat{c})$ for infinitely many values of $g \in G$. But then
\begin{align*}
    v(gy) &= \chi(g) + v(y) \\
        &= \chi(g) + v'(\varphi(y)) + v(y) - v'(\varphi(y))\\
        &= v'(g\varphi(y)) + v(y) - v'(\varphi(y)) \\
        &\leqslant t + v(y) - v'(\varphi(y))
\end{align*}
for infinitely many $gy \in \supp_X(\hat{c})$, but $\hat{c}$ is a horochain. 

For \ref{item:valIneq}, write $\hat{c} = \sum_{x\in X^{(n)}} \hat{c}_x$, where $\hat{c}_x = \sum_{g \in G} r_{g,x} g x$. Then
\begin{align*}
    v'(\widehat{\varphi}(\hat{c})) &\geqslant \min_{x \in X^{(n)}}\{ v'(\widehat{\varphi}(\hat{c}_x))\} \\
    &\geqslant \min_{x \in X^{(n)}} \{ \inf\{ v'(g\varphi(x)) : gx \in \supp \hat{c}_x \}   \} \\
    &= \min_{x \in X^{(n)}}\{ \inf\{ v(gx) : gx \in \supp \hat{c}_x \} + v'(\varphi(x)) - v(x)  \} \\
    &= \min_{x \in X^{(n)}} \{ v(\hat{c}_x) + v'(\varphi(x)) - v(x)  \} \\
    &\geqslant \min_{x \in X^{(n)}} \{ v(\hat{c}_x) \} +  \min_{x \in X^{(n)}}\{ v'(\varphi(x)) - v(x) \} \\
    &= v(\hat{c}) + \min_{x \in X^{(n)}} \{ v'(\varphi(x)) - v(x) \}. \qedhere
\end{align*}  \qedhere
\end{proof}

Note that \cref{lem:ExtendHom}\ref{item:valIneq} applies to chains in $F$ since these are just finite horo-chains. We will use this in the proof \cref{thm:Main}.









%%%                                                 %%%
%%%     CHARACTERISATIONS OF THE SIGMA INVARIANT    %%%
%%%                                                 %%%
\section{Characterisations of the \texorpdfstring{$\Sigma$}{Sigma}-invariant} \label{sec:SigInv}

We introduce the invariants $\Sigma^n_R(G;M)$, which are generalisations of the classical Bieri--Neumann--Strebel invariant \cite{BNSinv87} and its higher dimensional analogues \cite{BieriRenzValutations}. The only difference is that we work over a general ring $R$, while the higher BNS invariants are defined over $\Z$.

Let $G$ be a group. We declare two characters $\chi,\chi' \colon G \longrightarrow \R$ to be \textit{equivalent} if $\chi = \alpha \cdot \chi'$ for some $\alpha > 0$ and let $S(G)$ denote the set of equivalence classes of nonzero characters. We call $S(G)$ the \textit{character sphere} of $G$, because it can be given the topology of a sphere when $G$ is finitely generated. 

\begin{defn}[$\Sigma$-invariants]
Let $M$ be an $RG$-module. Then define
\[
\Sigma^n_R(G;M) = \{ [\chi] \in S(G) : M \in \mathtt{FP}_n(RG_\chi)  \},
\]
where $G_\chi = \{ g \in G : \chi(g) \geqslant 0 \}$. Note that $G_\chi = G_{\chi'}$ if $[\chi] = [\chi']$, so $\Sigma^n_R(G;M)$ is well-defined.
\end{defn}



\begin{defn}[Novikov ring]
Let $G$ be a group, let $R$ be a ring, and let $\chi \colon G \longrightarrow \R$ be a character. Then the \textit{Novikov ring} $\widehat{RG}^\chi$ is the set of formal sums
\[
\sum_{g \in G} r_g g
\]
such that $\{ g \in G : r_g \neq 0 \ \text{and} \ \varphi(g) \leqslant t \}$ is finite for every $t \in \R$. We give $\widehat{RG}^\chi$ a ring structure by defining $rg + r'g := (r + r')g$ and $rg \cdot r'g' := rr' gg'$ for $r,r' \in R$, $g,g' \in G$, and extending multiplication to all of $\widehat{RG}^\chi$ in the obvious way.
\end{defn}

\cref{thm:Main} gives several characterisations of the $\Sigma$-invariants and is the main technical tool we will need to prove \cref{thm:agrarianMain}. More specifically, we will need the characterisation of $\Sigma_R^n(G;M)$ in terms of the vanishing of Novikov homology; this it the equivalence of \ref{item:SigmaInv} and \ref{item:TorCond} in the following theorem, which should be thought of as a higher dimensional version of Sikorav's theorem \cite{SikoravThese}.

\begin{thm} \label{thm:Main}
Let $R$ be a ring, let $M$ be a left $RG$-module of type $\mathtt{FP}_n$, and let $\chi \colon G \longrightarrow \R$ be a non-trivial character. Let $F_\bullet \longrightarrow M \longrightarrow 0$ be a free resolution admissible with respect to a basis $X = \bigcup_{i = 0}^\infty X_i$ and with finitely generated $n$-skeleton $F^{(n)}$. Let $v \colon F \longrightarrow \R_\infty$ be the valuation extending $\chi$ with respect to $X$. The following are equivalent:
\begin{enumerate}[label=(\arabic*)]
    \item\label{item:SigmaInv} $[\chi] \in \Sigma_R^n(G;M)$;
    \item\label{item:EssAcyc} $F_\bullet^v \longrightarrow M \longrightarrow 0$ is essentially acyclic in dimensions $-1, \dots, n-1$;
    \item\label{item:LiftId} there is a chain map $\varphi \colon F \longrightarrow F$ lifting the identity $\id_M$ such that $v(\varphi(c)) > v(c)$ for every $c \in F^{(n)}$;
    \item\label{item:HoroAcyc} $F_\bullet \longrightarrow M \longrightarrow 0$ is horo-acyclic in dimensions $0, \dots, n$ with respect to $v$;
    \item\label{item:TorCond} $\Tor_i^{RG}(M, \widehat{RG}^\chi) = 0$ for all $0 \leqslant i \leqslant n$.
\end{enumerate}
\end{thm}



The strategy of the proof will be as follows: we begin by proving \ref{item:EssAcyc} $\Longrightarrow$ \ref{item:LiftId} $\Longrightarrow$ \ref{item:HoroAcyc} $\Longrightarrow$ \ref{item:EssAcyc}. This is done by  Schweitzer in the appendix of \cite{BieriDeficiency} in the case $R = \Z$. Once this is done, we prove the equivalence of \ref{item:HoroAcyc} and \ref{item:TorCond}, again following Schweitzer. Finally, we prove the equivalence of \ref{item:SigmaInv} and \ref{item:EssAcyc} following the appendix to Theorem 3.2 in \cite{BieriRenzValutations}, where again this is done in the case $R = \Z$. The proofs below and are essentially the same as those given in the references just cited; there is no crucial dependence on the coefficient ring $R$.

\begin{proof}[Proof of \ref{item:EssAcyc} $\Longrightarrow$ \ref{item:LiftId}]
Assume that $F_\bullet^v \longrightarrow M \longrightarrow 0$ is essentially acyclic in dimensions $\leqslant n-1$ and let $D > 0$ be a constant such that for each $k < n$ and every cycle $z \in F_k$, there is a chain $c \in F_{k+1}$ with $\partial c = z$ and $D \geqslant v(z) - v(c)$. 
We will construct a chain map $\varphi \colon F \longrightarrow F$ lifting $\id_M$ such that $v(\varphi(c)) > v(c) + (n - k)D$ for every $c \in F^{(k)}$, which implies \ref{item:LiftId}.

We define $\varphi$ on $F^{(k)}$ by induction on $k$. For the base case, let $x \in X_0$ be arbitrary, and fix some $g \in G$ such that $\chi(g) > (n+1)D$. The element $g\inv\partial x \in M$ is a cycle, so there is some $c_x \in F_0$ such that $\partial c_x = g\inv \partial x$ and $D \geqslant v(g\inv \partial x) - v(c_x) = -v(c_x)$, since $v|_{M \setminus\{0\}} = 0$. Define $\varphi$ on $F^{(0)}$ by setting $\varphi(x) = gc_x$ for each $x \in X_0$. It is  clear that $\mathrm{id}_M  \partial = \partial  \varphi$ on $F^{(0)}$. By \cref{lem:ExtendHom}\ref{item:valIneq},
\[
v(\varphi(c)) \geqslant v(c) + \min_{x \in X_0} \{v(\varphi(x)) - v(x)\} > v(c) + nD
\]
for every $c \in F^{(0)}$.

Let $k > 0$ and suppose $\varphi$ is defined on $F^{(k-1)}$ such that it lifts $\id_M$ and $v(\varphi(c)) > v(c) + (n - k + 1)D$ for all $c \in F^{(k-1)}$. Let $x \in X_k$ and note that $\varphi(\partial x)$ is a cycle. By essential acyclicity, there is a chain $d_x \in F_k$ such that $\partial d_x = \varphi(\partial x)$ and $D \geqslant v(\varphi(\partial x)) - v(d_x)$. Define $\varphi$ on $F^{(k)}$ by setting $\varphi(x) = d_x$. Then $\varphi \partial = \partial \varphi$ by construction, and for every $x \in X_k$ we have
\begin{align*}
v(\varphi(x)) - v(x) &= v(d_x) - v(x) \\
    &\geqslant v(\varphi(\partial x)) - v(x) - D \\
    &= v(\varphi(\partial x)) - v(\partial x) - D \\
    &> (n-k)D
\end{align*}
by induction. By \cref{lem:ExtendHom}\ref{item:valIneq}, we have
\[
v(\varphi(c)) \geqslant v(c) + \min_{x \in X_k} \{ v(\varphi(x)) - v(x) \} > v(c) + (n-k)D. \qedhere
\]
\end{proof}


We pause here to prove a lemma that will immediately imply \ref{item:LiftId} $\Longrightarrow$ \ref{item:HoroAcyc} and will be useful in the proofs of \ref{item:HoroAcyc} $\Longrightarrow$ \ref{item:EssAcyc} and \ref{item:HoroAcyc} $\Longrightarrow$ \ref{item:TorCond}. 

\begin{lem}\label{lem:Useful}
With the assumptions of \cref{thm:Main}, let $\varphi \colon F \longrightarrow F$ be a chain map lifting $\id_M$ such that $v(\varphi(c)) > v(c)$ for all $c \in F^{(n)}$ and let $H \colon F \longrightarrow F$ be a chain homotopy such that $\partial H + H \partial = \id_{F} - \varphi$. Let $\hat{z} \in \widehat{F}^{(n)}$ be a horocycle and define $\hat{c}_{\hat{z}} := \sum_{i = 0}^\infty \widehat{H} \widehat{\varphi}^i(\hat{z})$. Then $\hat{c}_{\hat{z}}$ is a horochain and $\partial \hat{c}_{\hat{z}} = \hat{z}$.
\end{lem}

\begin{proof}
By \cref{lem:ExtendHom}\ref{item:valIneq} there are constants $\alpha$ and $\beta$ such that
\[
    v(\widehat{\varphi}(\hat{c})) \geqslant v(\hat{c}) + \alpha \ \ \text{and} \ \ v(\widehat{H}(\hat{c})) \geqslant v(\hat{c}) + \beta
\]
for every horochain $\hat{c} \in \widehat{F}^{(n)}$. Moreover, $\alpha > 0$ since $v(\varphi(f)) > v(f)$ for every $f \in F^{(n)}$. To see that $\hat{c}$ is a horochain, by induction we have $v(\widehat{H}\widehat{\varphi}^i(\hat{z})) \geqslant v(\hat{z}) + i\alpha + \beta$, so for all $t \in \R$ there are only finitely many integers $i \geqslant 0$ such that $v(\widehat{H} \widehat{\varphi}^i(\hat{z})) \leqslant t$. Since $\supp(\hat{c}_{\hat{z}}) \subseteq \bigcup_{i=0}^\infty \supp(\widehat{H}\widehat{\varphi}^i(\hat{z}))$ and each $\widehat{H} \widehat{\varphi}^i(\hat{z})$ is a horochain, it follows that there are only finitely many $gx \in \supp \hat{c}$ such that $v(gx) \leqslant t$, so $\hat{c}$ is a horochain.

Finally, we have
\begin{equation*}
    \partial\hat{c} = \sum_{i=0}^\infty \partial \widehat{H} \widehat{\varphi}^i(\hat{z}) = \sum_{i=0}^\infty (\id_{\widehat{F}^{(n)}} - \widehat{\varphi} - \widehat{H}\partial)\widehat{\varphi}^i(\hat{z}) = \sum_{i=0}^\infty (\widehat{\varphi}^i - \widehat{\varphi}^{i+1})(\hat{z}) = \hat{z}. \qedhere
\end{equation*}
\end{proof}




\begin{proof}[Proof of \ref{item:LiftId} $\Longrightarrow$ \ref{item:HoroAcyc}]
If $\hat{z} \in \widehat{F}^{(n)}$ is a horocycle, then $\partial \hat{c}_{\hat{z}} = \hat z$ by \cref{lem:Useful}. 
\end{proof}



\begin{proof}[Proof of \ref{item:HoroAcyc} $\Longrightarrow$ \ref{item:EssAcyc}] We will prove that $F_\bullet^v \longrightarrow M \longrightarrow 0$ is essentially acyclic in dimension $k$ for all $k < n$ by induction on $k$. For the base case, we show that $F_\bullet^v \longrightarrow M \longrightarrow 0$ is exact at $M$, which implies essential acyclicity in dimension $-1$. Let $m \in M$. By exactness of $F_\bullet \longrightarrow M \longrightarrow 0$, there is a chain $c \in F_0$ such that $\partial c = m$. By horo-acyclicity in dimension $0$, there is some horochain $\hat{c} \in \widehat{F}_1$ such that $\partial \hat{c} = c$. There are $c_{-} \in F_1$ and $\hat{c}_+ \in \widehat{F}_1$ such that  $\hat{c} = c_{-} + \hat{c}_{+}$, where $v(c_{-}) < 0$ and $v(\hat{c}_{+}) \geqslant 0$. Then $\partial(c - \partial c_-) = m$ and 
\[
    v(c - \partial c_-) = v(c - \partial(\hat{c} - \hat{c}_+)) = v(\partial \hat{c}_+) \geqslant v(\hat{c}_+) \geqslant 0.
\]
This shows that $c - \partial c_0 \in F_0^v$, which proves that $F_\bullet^v \longrightarrow M \longrightarrow 0$ is exact at $M$.


Let $k > -1$ and suppose that $F_\bullet^v \longrightarrow M \longrightarrow 0$ is essentially acyclic in dimensions $< k$. By \ref{item:EssAcyc} $ \longrightarrow$ \ref{item:LiftId} applied at $k-1$, there is a chain map $\varphi \colon F \longrightarrow F$ lifting $\id_M$ such that $v(\varphi(c)) > v(c)$ for all $c \in F^{(k)}$. Since $\id_F$ and $\varphi$ both lift $\id_M$ and $F_\bullet \longrightarrow M \longrightarrow 0$ is acyclic, there is a chain homotopy $H \colon F \longrightarrow F$ such that $\partial H + H \partial = \id_{F} - \varphi$. As in the proof of \cref{lem:Useful}, there are constants $\alpha > 0$ and $\beta < 0$ such that 
\[
    v(\varphi(c)) \geqslant v(c) + \alpha \ \ \text{and} \ \ v(H(c)) \geqslant v(c) + \beta
\]
for every $c \in F^{(k)}$.

Let $z \in F^v_{k}$ be a cycle. Since $F_\bullet \longrightarrow M \longrightarrow 0$ is acyclic, there is some $d \in F_{k+1}$ such that $\partial d = z$. Consider the horocycle $\hat{z} := d - \hat{d}_z$, where $\hat{d}_z = \sum_{i = 0}^\infty H\varphi^i(z)$ is defined as in \cref{lem:Useful}. Note that
\[
    v(H\varphi^i(z)) \geqslant v(z) + i\alpha + \beta \geqslant \beta
\]
for every $i \geqslant 0$, and therefore that $v(\hat{d}_z) \geqslant \beta$. By horo-acyclicity in dimension $k+1$, there is a $(k+2)$-horochain $\hat{d}$ such that $\partial \hat{d} = \hat{z}$. As in the base case, there are $d_- \in F_{k+2}$ and $\hat{d}_+ \in \widehat{F}_{k+2}$ such that $\hat{d} = d_- + \hat{d}_+$, where $v(d_-) < 0$ and $v(\hat{d}_+) \geqslant 0$. Then $\partial(d - \partial d_-) = \partial d = z$, and
\begin{align*}
    v(d - \partial d_-) &= v(\hat{d}_z + \hat{z} - \partial(\hat{d} - \hat{d}_+)) \\
        &= v(\hat{d}_z + \partial \hat{d}_+) \\
        &\geqslant \min\{ v(\hat{d}_z), v(\partial \hat{d}_+) \} \\
        &\geqslant \beta
\end{align*}
since $v(\hat{d}_z) \geqslant \beta$ and $v(\partial d_\infty) \geqslant v(d_\infty) \geqslant 0 > \beta$. Letting $D = -\beta$ in the definition of essential acyclicity, we see that $F_\bullet^v \longrightarrow M \longrightarrow 0$ is essentially acyclic in dimension $k$. \qedhere
\end{proof}

\begin{proof}[Proof of \ref{item:TorCond} $\Longrightarrow$ \ref{item:HoroAcyc}] Suppose that $\Tor_i^{RG}(M,\widehat{RG}^\chi) = 0$ for $0 \leqslant i \leqslant n$. Consider the chain map
\[
\psi \colon \widehat{RG}^\chi \otimes_{RG} F \longrightarrow \widehat{F}, \ \ \alpha \otimes c \longmapsto \alpha c
\]
of left $RG$-modules. It is clear that $\psi$ is injective. We claim that $\psi$ induces an isomorphism $\widehat{RG}^\chi \otimes_{RG} F^{(n)} \longrightarrow \widehat{F}^{(n)}$. To see this, simply note that for an arbitrary horochain
\[
    \hat{c} = \sum_{g \in G, x \in X^{(n)}} r_{g,x} gx
\]
in $\widehat{F}^{(n)}$, we have
\[
    \sum_{x \in X^{(n)}}  \left( \sum_{g\in G} r_{g,x} g \right) \otimes x \xmapsto{\varphi} \hat{c}.
\]
The horochain condition implies that the sums $\sum_{g\in G} r_{g,x} g$ are elements of $\widehat{RG}^\chi$. Thus, $\psi$ is surjective on the $n$-skeleta and is therefore an isomorphism. Note that this only works because $X^{(n)}$ is finite; in general, we cannot expect $\psi$ to be surjective since the support of a horochain might intersect infinitely many of the modules $F_n$. Since $H_i(\widehat{RG}^\chi \otimes_{RG} F ; RG) = 0$ for all $0 \leqslant i \leqslant n$, we conclude that $H_i(\widehat{F}; RG) = 0$ for $0 \leqslant i \leqslant n$ as well.
\end{proof}


\begin{proof}[Proof of \ref{item:HoroAcyc} $\Longrightarrow$ \ref{item:TorCond}] The map $\psi$ defined above is an isomorphism of the $n$-skeleta, so we immediately have that $\Tor_i^{RG}(M, \widehat{RG}^\chi) = 0$ for $0 \leqslant i \leqslant n-1$. Since $\psi$ is not necessarily surjective as a map of the $(n+1)$-skeleta, we must work harder to show that $\Tor_n^{RG}(M, \widehat{RG}^\chi) = 0$. Let $z \in \widehat{RG}^\chi \otimes_{RG} F_n$ be an $n$-cycle, and let $\hat{z} = \psi(z)$. Since we are assuming that \ref{item:HoroAcyc} holds, we may also assume that \ref{item:LiftId} holds and use the horochain $\hat{c}_{\hat{z}}$ from \cref{lem:Useful}. Since $\hat{c}_{\hat{z}} \in \widehat{H}(\widehat{F}_n)$, we have that $\hat{c}_{\hat{z}}$ is in the $\widehat{RG}^\chi$-submodule of $\widehat{F}_{n+1}$ generated by $\widehat{H}(X_n)$, and thus $\hat{c}_{\hat{z}} \in \im \psi$ since this is a finite set. Let $c \in \widehat{RG}^\chi \otimes_{RG} F_{n+1}$ such that $\psi(c) = \hat{c}_{\hat{z}}$. Then $\psi \partial (c) = \partial \psi(c) = \partial \hat{c}_{\hat{z}} = \hat{z}$. But $\psi$ is injective, so $\partial c = z$, proving that $\Tor_n^{RG}(M, \widehat{RG}^\chi) = 0$.
\end{proof}


We pause again before proving the equivalence of \ref{item:SigmaInv} and \ref{item:EssAcyc} to prove another lemma. 

\begin{lem}\label{lem:flat}
Free $RG$-modules are flat over $RG_\chi$.
\end{lem}
\begin{proof}
It suffices to prove that $RG$ is flat as an $RG_\chi$-module, since the direct sum of flat modules is flat. To this end, let $\iota \colon M \longhookrightarrow N$ be an injection of right $RG_\chi$-modules; our goal is to show that $\iota \otimes \id \colon M \otimes_{RG_\chi} RG \longrightarrow N \otimes_{RG_\chi} RG$ is injective. Let $g \in G$ be such that $\chi(g) < 0$ and consider the left $RG_\chi$-module $RG_\chi g^k = \{ \alpha g^k : \alpha \in RG_\chi, k \in \Z \}$. The modules $RG_\chi g^k$ form a directed system with respect to the inclusion maps $RG_\chi g^k \longhookrightarrow RG_\chi g^l$ for $k \leqslant l$ and the direct limit is $\varinjlim RG_\chi g^k \cong RG$.

There are left $RG_\chi$-module isomorphisms $RG_\chi g^k \longrightarrow RG_\chi$ given by right multiplication by $g^{-k}$. Then $RG_\chi g^k$ is flat over $RG_\chi$, so $M \otimes_{RG_\chi} RG_\chi g^k \longrightarrow N \otimes_{RG_\chi} RG_\chi g^k$ is injective for all $k \in \Z$. By exactness of the direct limit,
\[
    \varinjlim(M \otimes_{RG_\chi} RG_\chi g^k) \longrightarrow \varinjlim(N \otimes_{RG_\chi} RG_\chi g^k)
\]
is injective. Since the direct limit commutes with the tensor product, the previous line implies $\iota \otimes \id_M$ is injective. \qedhere
\end{proof}

We now return to the proof of \cref{thm:Main}.

\begin{proof}[Proof of \ref{item:SigmaInv} $\Longleftrightarrow$ \ref{item:EssAcyc}] Let $g \in G$ be such that $\chi(g) < 0$ and let $E_k$ be the left $RG_\chi$-module $g^k F^v$. We denote the chain complexes $F^v_\bullet \longrightarrow M \longrightarrow 0$ and $(E_k)_\bullet \longrightarrow M \longrightarrow 0$ by $\widetilde{F}^v$ and $\widetilde{E}_k$, respectively.

Essential acyclicity in dimension $j$ is equivalent to the the existence of an integer $D \geqslant 0$ such that the inclusion-induced homomorphism $H_j(\widetilde{E}_k) \longrightarrow H_j(\widetilde{E}_{k+D})$ is the zero map for all $k \in \N$. This in turn is equivalent to $\varinjlim \prod_{I} H_j(\widetilde{E}_k) = 0$ for any index set $I$. Here, for fixed $I$ and $j$, the powers $\prod_{I} H_j(\widetilde{E}_k)$ form a directed system with respect to the inclusion-induced maps $\prod_I H_j(\widetilde{E}_k) \longrightarrow \prod_I H_j(\widetilde{E}_l)$ for $k \leqslant l$. Indeed, if $D \geqslant 0$ is such that $H_j(\widetilde{E}_k) \longrightarrow H_j(\widetilde{E}_{k+D})$ is the zero map, it is clear that the direct limit will be zero. Conversely, let $I = Z_j(\widetilde{E}_0) = Z_j(\widetilde{F}^v)$ be the set of $j$-cycles of $\widetilde{F}^v$ and consider the element $([x])_{x \in I} \in \prod_I H_j(\widetilde{E}_0)$. Since the direct limit is zero, there is some $D \geqslant 0$ such that $([x])_{x \in I} = 0$ in $\prod_I H_j(\widetilde{E}_D)$, which means that $\widetilde{F}^v$ is essentially acyclic in dimension $j$.

There is a short exact sequence of chain complexes $0 \longrightarrow M \longrightarrow \widetilde{E}_k \longrightarrow E_k \longrightarrow 0$, where by abuse of notation $M$ is a chain complex concentrated in dimension $-1$ and $E_k$ is the chain complex $(E_k)_\bullet \longrightarrow 0$ with $(E_k)_0$ in dimension $0$. The long exact sequence in homology associated to the short exact sequence gives $H_j(\widetilde{E}_k) \cong H_j(E_k)$ for $j \geqslant 1$. The interesting part of the long exact sequence is
\[
0 \longrightarrow H_0(\widetilde{E}_k) \longrightarrow H_0(E_k) \xrightarrow{\delta} M \longrightarrow H_{-1}(\widetilde{E}_k) \longrightarrow 0,
\]
where $\delta$ is the connecting homomorphism. By exactness of the direct power and direct limit functors, the sequence
\[
0 \rightarrow \varinjlim \prod_I H_0(\widetilde{E}_k) \rightarrow \varinjlim \prod_I H_0(E_k) \xrightarrow{\prod_I \delta} \prod_I M \rightarrow \varinjlim \prod_I H_{-1}(\widetilde{E}_k) \rightarrow 0
\]
is exact. Then $\widetilde{F}^v$ is essentially exact in dimension $0$ if and only if $\delta$ induces an injection $\varinjlim \prod_I H_0(E_k) \longrightarrow \prod_I M$ for every $I$. Moreover, $\widetilde{F}^v$ is essentially exact in dimension $-1$ if and only if $\delta$ induces a surjection $\varinjlim \prod_I H_0(E_k) \longrightarrow \prod_I M$ for every $I$ .

By \cref{lem:flat},  $F_\bullet \longrightarrow M \longrightarrow 0$ is a flat resolution of $M$ by left $RG_\chi$-modules, so 
\[
    \Tor^{RG_\chi}_j \left(M, \prod_I RG_\chi \right) = H_j\left( \left(\prod_I RG_\chi \right) \otimes_{RG_\chi} F \right).
\]
and therefore
\[
    \Tor^{RG_\chi}_j \left(M, \prod_I RG_\chi \right) = \varinjlim H_j \left( \left(\prod_I RG_\chi \right) \otimes_{RG_\chi} E_k \right),
\]
as $F = \varinjlim E_k$ and direct limits commute with tensor products and homology. Since $(E_k)_j$ is a finitely generated free $RG_\chi$-module for $j \leqslant n$, we have $\left(\prod_I RG_\chi \right) \otimes_{RG_\chi} (E_k)_j \cong \prod_I (E_k)_j$. Hence, $\Tor_j^{RG_\chi}(M, \prod_I RG_\chi) = \varinjlim H_j(\prod_I E_k)$ for $j < n$.

To summarize the work done above, we have $\widetilde{F}^v$ is essentially exact in dimensions $-1 \leqslant j < n$ if and only if
\begin{enumerate}[label=(\alph*)]
    \item $(\prod_I RG_\chi) \otimes_{RG_\chi} M \longrightarrow \prod_I M$ is surjective if $n = 0$ and
    \item $(\prod_I RG_\chi) \otimes_{RG_\chi} M \longrightarrow \prod_I M$ is an isomorphism and 
    \[
        \Tor_j^{RG_\chi}(M, \prod_I RG_\chi)
    \]
    vanishes for $1 \leqslant j < n$ otherwise.
\end{enumerate}
Here we have used the general fact that $\Tor_0^R(A,B) \cong A \otimes_R B$. Together with Lemma 1.1 and Proposition 1.2 of \cite{BieriEckmannFinProps}, (a) and (b) are equivalent to $M$ being of type $\mathtt{FP}_n(RG_\chi)$. Thus, we conclude that $[\chi] \in \Sigma^m_R(G;M)$ if and only $\widetilde{F}^v$ is essentially exact in dimensions $j = -1, 0, 1, \dots, n-1$. \qedhere
\end{proof}




%%%
%%% Hughes homology
%%%
\section{Agrarian homology and main result} \label{sec:homology}

\begin{defn}[agrarian groups and $\mathcal{D}$-homology]
Let $R$ be a ring. A group $G$ is \textit{agrarian over $R$} if there is a skew-field $\mathcal{D}$ and an injective ring homomorphism $R G \longhookrightarrow \mathcal{D}$. In this case, we will say that $G$ is $\mathcal{D}$-agrarian over $R$ if we wish to specify the skew-field.

If $G$ is $\mathcal{D}$-agrarian over $R$, we define its $p$-dimensional \textit{$\mathcal{D}$-homology} to be
\[
    H_p^{\mathcal{D}} (G) := \Tor_p^{RG} (R, \mathcal{D}),
\]
where $R$ is the trivial $RG$-module and $\mathcal{D}$ is viewed as a $\mathcal{D}$-$RG$-bimodule via the embedding $RG \longhookrightarrow \mathcal{D}$. The $p$th $\mathcal{D}$-Betti number of $G$ is then
\[
    b_p^{\mathcal{D}}(G) := \dim_\mathcal{D} H_p^{\mathcal{D}} (G).
\]
Note that $b_p^{\mathcal{D}}(G)$ is well-defined and integral or infinite, since a module over a skew-field has a well-defined dimension.
\end{defn}

\begin{rem}
The term ``agrarian" was introduced by Kielak in \cite{KielakBNSviaNewton} in the case $R = \mathbb{Z}$. Using strong Hughes-freeness, i.e.~condition \ref{item:2prime} after \cref{def:HfreeDiv}, it follows that if $G$ is a locally-indicable group and $\mathbb{F}$ is a skew-field such that $\mathcal{D}_{\mathbb{F}G}$ exists, then $G$ is $\mathcal{D}_{\mathbb{F}G}$-agrarian over $\mathbb{F}$. If $R$ is a skew-field or an integral domain, then there are no known examples of torsion-free groups that are not agrarian over $R$. For the remainder of the section, we will be interested in the $\mathcal{D}_{\mathbb{F}G}$-homology of $G$.
\end{rem}


In what follows we will need that $\mathcal{D}_{\mathbb{F}G}$-Betti numbers have good scaling properties when passing to finite index subgroups. This is analogous to the fact that $\ell^2$-Betti numbers also scale under taking finite index subgroups.


\begin{lem}\label{lem:JBscales}
Let $H$ be a finite index subgroup of $G$ and let $\mathbb{F}$ be a skew-field such that $\mathcal{D}_{\mathbb{F}G}$ exists. Then 
\[
    b_p^{\mathcal{D}_{\mathbb{F}G}}(G) = \frac{b_p^{\mathcal{D}_{\mathbb{F}H}}(H)}{[G:H]}.
\]
\end{lem}
\begin{proof} It suffices to prove the claim when $H$ is normal in $G$. To see this, if $H \leqslant G$ is any subgroup of finite index, then there is normal subgroup $N \triangleleft G$ of finite index such that $N \leqslant H \leqslant G$ (we can take $N$ to be the \textit{normal core} of $H$, that is, the intersection of all the conjugates of $H$). Then
\[
b_p^{\mathcal{D}_{\mathbb{F}G}}(G) = \frac{b_p^{\mathcal{D}_{\mathbb{F}N}}(N)}{[G:N]} = \frac{[H:N] b_p^{\mathcal{D}_{\mathbb{F}H}}(H)}{[G:N]} = \frac{b_p^{\mathcal{D}_{\mathbb{F}H}}(H)}{[G:H]},
\]
by the claim for normal subgroups.

Assume that $H$ is a finite index normal subgroup of $G$ and let $\{t_1, \dots, t_n \}$ be a transversal for $H$ in $G$. By \cref{prop:twistedNormalSkew}\ref{item:twistedFiniteIndex}, we have a $\mathcal{D}_{\mathbb{F}H}$-$\mathbb{F}G$-bimodule isomorphism $\mathcal{D}_{\mathbb{F}G} \cong  \mathcal{D}_{\mathbb{F}H} \otimes_{\mathbb{F}H} \mathbb{F}G$. Take a free resolution $F_\bullet \longrightarrow \mathbb{F} \longrightarrow 0$ of the trivial left $\mathbb{F}G$-module $\mathbb{F}$; there are chain-isomorphisms
\[
    \mathcal{D}_{\mathbb{F}G} \otimes_{\mathbb{F}G} F_\bullet 
    \cong (\mathcal{D}_{\mathbb{F}H} \otimes_{\mathbb{F}H} \mathbb{F}G) \otimes_{\mathbb{F}G} F_\bullet  
    \cong \mathcal{D}_{\mathbb{F}H} \otimes_{\mathbb{F}H} (\mathbb{F}G \otimes_{\mathbb{F}G} F_\bullet) 
    \cong \mathcal{D}_{\mathbb{F}H} \otimes_{\mathbb{F}H} F_\bullet
\]
of left $\mathcal{D}_{\mathbb{F}H}$-modules, so $H_p^{\mathcal{D}_{\mathbb{F}G}}(G) \cong H_p^{\mathcal{D}_{\mathbb{F}H}}(H)$ as $\mathcal{D}_{\mathbb{F}H}$-modules. Therefore,
\begin{align*}
    b_p^{\mathcal{D}_{\mathbb{F}H}}(H) &= \dim_{\mathcal{D}_{\mathbb{F}H}} H_p^{\mathcal{D}_{\mathbb{F}H}}(H) \\
        &= \dim_{\mathcal{D}_{\mathbb{F}H}} H_p^{\mathcal{D}_{\mathbb{F}G}}(G) \\
        &= [G : H] \cdot \dim_{\mathcal{D}_{\mathbb{F}G}} H_p^{\mathcal{D}_{\mathbb{F}G}}(G) \\ 
        &= [G : H] \cdot b_p^{\mathcal{D}_{\mathbb{F}G}}(G). \qedhere
\end{align*}
\end{proof}

In view of \cref{lem:JBscales}, if $G$ is a group with a finite index subgroup $H$ such that $\mathcal{D}_{\mathbb{F}H}$ exists, we can define $b_p^{\mathcal{D}_{\mathbb{F}G}}(G) = b_p^{\mathcal{D}_{\mathbb{F}H}}(H)/[G:H]$. This is an abuse of notation  since $\mathcal{D}_{\mathbb{F}G}$ might not exist.

The following theorem is an analogue of a theorem of L\"uck which holds for $\ell^2$-Betti numbers \cite[Theorem 7.2]{Luck02}.


\begin{thm}\label{thm:JBSES}
Let $1 \longrightarrow K \longrightarrow G \longrightarrow \Z \longrightarrow 1$ be a short exact sequence of groups and suppose that $\mathcal{D}_{\mathbb{F}G}$ exists for some skew-field $\mathbb{F}$. If $b_p^{\mathcal{D}_{\mathbb{F}K}}(K) < \infty$ for some $p \geqslant 0$, then $b_p^{\mathcal{D}_{\mathbb{F}G}}(G) = 0$. 
\end{thm}
\begin{proof}
Let $F_\bullet \longrightarrow \mathbb{F} \longrightarrow 0$ be a free resolution of $\mathbb{F}$ by left $\mathbb{F} G$-modules. Note that the modules $F_j$ are also free left $\mathbb{F}K$-modules and there are chain maps $\iota_j \colon \mathcal{D}_{\mathbb{F}K} \otimes_{\mathbb{F}K} F_j \longrightarrow \mathcal{D}_{\mathbb{F}G} \otimes_{\mathbb{F}G} F_j$, induced by the inclusion $\mathcal{D}_{\mathbb{F}K} \longhookrightarrow \mathcal{D}_{\mathbb{F}G}$. We claim that the maps $\iota_j$ are injective. To see this, it is enough to consider the case where $F_j = \mathbb{F}G$. Choose $t \in G$ such that $\{ t^n : n \in \Z \}$ is a transversal for $K \leqslant G$. By \cref{prop:twistedNormalSkew}, there is an embedding $\bigoplus_{n \in \Z} \mathcal{D}_{\mathbb{F}K} \cdot \varphi(t^n) \longhookrightarrow \mathcal{D}_{\mathbb{F}G}$. Since $\mathbb{F}G$ is free over $\mathbb{F}K$, there is also an isomorphism $\mathcal{D}_{\mathbb{F}K} \otimes_{\mathbb{F}K} \mathbb{F}G \cong \bigoplus_{n \in \Z} \mathcal{D}_{\mathbb{F}K} \cdot \varphi(t^n)$ determined by $\alpha \otimes t^n \longmapsto \alpha \varphi(t^n)$. Then the diagram
\[
\begin{tikzcd}
\mathcal{D}_{\mathbb{F}K} \otimes_{\mathbb{F}K} \mathbb{F}G \arrow[d, "\iota_j"'] \arrow[r, "\cong", no head] & {\bigoplus_{n \in \mathbb{Z}} \mathcal{D}_{\mathbb{F}K} \cdot \varphi(t^n)}  \arrow[d, hook] \\
\mathcal{D}_{\mathbb{F}G} \otimes_{\mathbb{F}G} \mathbb{F}G \arrow[r, "\cong", no head]                       &  \mathcal{D}_{\mathbb{F}G}
\end{tikzcd}
\]
of left $\mathcal{D}_{\mathbb{F}K}$-modules commutes, proving that $\iota_j$ is an injection. From now on, we will treat the maps $\iota_j$ as inclusions.

Consider the following portions of the chain complexes computing $b_p^{\mathcal{D}_{\mathbb{F}G}}(G)$ and $b_p^{\mathcal{D}_{\mathbb{F}K}}(K)$
\[
\begin{tikzcd}[column sep = small]
\cdots \arrow[r] & \mathcal{D}_{\mathbb{F}K} \otimes_{\mathbb{F}K} F_{p+1} \arrow[d, hook, "\iota_{p+1}"] \arrow[r] & \mathcal{D}_{\mathbb{F}K} \otimes_{\mathbb{F}K} F_p \arrow[d, hook, "\iota_p"] \arrow[r] & \mathcal{D}_{\mathbb{F}K} \otimes_{\mathbb{F}K} F_{p-1} \arrow[d, hook, "\iota_{p-1}"] \arrow[r] & \cdots \\
\cdots \arrow[r] & \mathcal{D}_{\mathbb{F}G} \otimes_{\mathbb{F}G} F_{p+1} \arrow[r]                               & \mathcal{D}_{\mathbb{F}G} \otimes_{\mathbb{F}G} F_p \arrow[r]                           & \mathcal{D}_{\mathbb{F}G} \otimes_{\mathbb{F}G} F_{p-1} \arrow[r]                                 & \cdots \nospacepunct{.}
\end{tikzcd}
\]
Let $x$ be a cycle in $\mathcal{D}_{\mathbb{F}G} \otimes_{\mathbb{F}G} F_p$. By \cite[Proposition 2.2(2)]{JaikinZapirain2020THEUO}, $\mathcal{D}_{\mathbb{F}G} \cong \mathrm{Ore}(\mathcal{D}_{\mathbb{F}K} * \Z)$, where we are making the identifications $\mathcal{D}_{\mathbb{F}K} \otimes_{\mathbb{F}K} \mathbb{F}G \cong \bigoplus_{n \in \Z} \mathcal{D}_{\mathbb{F}K} \cdot \varphi(t^n) \cong \mathcal{D}_{\mathbb{F}K} * \Z$. Hence, there is some nonzero $a \in \mathcal{D}_{\mathbb{F}K} * \Z$ such that $ax \in \mathcal{D}_{\mathbb{F}K} \otimes_{\mathbb{F}K} F_p$. Since $(\mathcal{D}_{\mathbb{F}K} * \Z) \cdot ax \subseteq Z_p(\mathcal{D}_{\mathbb{F}G} \otimes_{\mathbb{F}G} F_\bullet)$ and $\iota_{p-1}$ is injective, $(\mathcal{D}_{\mathbb{F}K} * \Z) \cdot ax$ is an infinite-dimensional $\mathcal{D}_{\mathbb{F}K}$-subspace of $Z_p(\mathcal{D}_{\mathbb{F}K} \otimes_{\mathbb{F}K} F_\bullet)$. Since $b_p^{\mathcal{D}_{\mathbb{F}K}}(K) < \infty$, there is a nonzero $b \in \mathcal{D}_{\mathbb{F}K} * \Z$ such that $bax = \partial y$ for some $y \in \mathcal{D}_{\mathbb{F}K} \otimes_{\mathbb{F}K} F_{p+1}$. But then $x = \partial((ba)\inv y)$, so we conclude that $H_p(\mathcal{D}_{\mathbb{F}G} \otimes_{\mathbb{F}G} F_\bullet) = 0$. \qedhere
\end{proof}




For the proof of the main theorem, we will need the following version of a theorem of Bieri and Renz. The details of the proof are given in \cite[Theorem 5.1]{BieriRenzValutations} in the case $R = \Z$, though the proof goes through in exactly the same way after replacing the ring $\Z$ by an arbitrary ring $R$.

\begin{thm}[Bieri-Renz]\label{thm:genBR}
Let $G$ be a finitely generated group and let $N \triangleleft G$ be a normal subgroup containing the commutator subgroup $[G,G]$. Let $R$ be a unital ring and let $M$ be an $RG$-module of type $\mathtt{FP}_n(RG)$. Then $M \in \mathtt{FP}_n(RN)$ if and only if $\Sigma_R^n(G;M) \supseteq S(G,N) := \{ [\chi] \in S(G) : \chi(N) = 0 \}$.
\end{thm}

We will also need the following result due to Kielak and Jaikin-Zapirain. Kielak first proved the result in \cite[Theorem 5.2]{KielakRFRS} by giving an explicit construction of the Linnell skew-field $\mathcal D(G)$ when $G$ is RFRS. In the appendix to \cite{JaikinZapirain2020THEUO}, Jaikin-Zapirain showed that when $G$ is RFRS and $\mathbb F$ is any skew-field, then $\mathcal D_{\mathbb F G}$ exists and admits a completely analogous construction to $\mathcal D(G)$ (in fact $\mathcal D(G) = \mathcal D_{\Q G}$ for a RFRS group $G$). As a consequence, Kielak's proof of \cref{thm:KJZ} still holds after making the replacements $\Q \rightsquigarrow \mathbb F$ and $\mathcal D(G) \rightsquigarrow \mathcal D_{\mathbb FG}$.

\begin{thm}[Kielak, Jaikin-Zapirain] \label{thm:KJZ}
Let $\mathbb F$ be a skew-field, $G$ a finitely generated RFRS group, and $n \in \N$. Let $F_\bullet$ be a chain complex of free $\mathbb F G$-modules with $F_p$ is finitely generated and $H_p(\mathcal D_{\mathbb F G} \otimes_{\mathbb F G} F_\bullet) = 0$ for all $p \leqslant N$. Then, there exist a finite index subgroup $H \leqslant G$ and an open subset $U \subseteq S(H)$ such that
\begin{enumerate}
    \item the closure of $U$ contains $S(G)$;
    \item $U$ is invariant under the antipodal map;
    \item $H_p(\widehat{\mathbb F H}^\chi \otimes_{\mathbb F H} F_\bullet) = 0$ for every $p \leqslant n$ and every $[\chi] \in U$.
\end{enumerate}
\end{thm}

We are now ready to prove the main theorem.


\begin{thm}\label{thm:agrarianMain}
Let $\mathbb{F}$ be a skew-field and let $G$ be a virtually RFRS group of type $\mathtt{FP}_n(\mathbb{F})$. Then there is a finite index subgroup $H \leqslant G$ admitting a homomorphism onto $\Z$ with kernel of type $\mathtt{FP}_n(\mathbb{F})$ if and only if $b_p^{\mathcal{D}_{\mathbb{F}G}}(G) = 0$ for $p = 0, \dots, n$.
\end{thm}
\begin{proof}
($\Longrightarrow$) Let $\varphi \colon H \longrightarrow \Z$ be an epimorphism with kernel $K \in \mathtt{FP}_n(\mathbb{F})$. Then there is a free resolution $F_\bullet \longrightarrow \mathbb{F} \longrightarrow 0$ of the trivial $\mathbb{F}K$-module $\mathbb{F}$ with finitely generated $n$-skeleton. Therefore, 
\[
    b_p^{\mathcal{D}_{\mathbb{F}K}}(K) = \dim_{\mathcal{D}_{\mathbb{F}K}} H_p (\mathcal{D}_{\mathbb{F}K} \otimes_{\mathbb{F}K} F_\bullet) < \infty
\]
for $p \leqslant n$ and we have a short exact sequence $1 \longrightarrow K \longrightarrow H \longrightarrow \Z \longrightarrow 1$, so $b_p^{\mathcal{D}_{\mathbb{F}H}}(H) = 0$ for $p \leqslant n$ by \cref{thm:JBSES}. Then $b_p^{\mathcal{D}_{\mathbb{F}G}}(G) = 0$ for $p \leqslant n$ by \cref{lem:JBscales}.

\smallskip

($\Longleftarrow$) The properties of being of type $\mathtt{FP}_n(\mathbb{F})$ and of having vanishing $p$th $\mathcal{D}_{\mathbb{F}G}$-Betti number pass to finite index subgroups. Moreover, the property of being virtually fibred passes to finite index overgroups (the same is of course true of any virtual property). Hence, we may assume that $G$ is RFRS and of type $\mathtt{FP}_n(\mathbb{F})$. Then $H^1(G; \R) \neq 0$ by \cref{prop:RFRStoQ}.

Since $G \in \mathtt{FP}_n(\mathbb{F})$, there is a free resolution $F_\bullet \longrightarrow \mathbb{F} \longrightarrow 0$ of the trivial $\mathbb{F}G$-module $\mathbb{F}$ with $F_p$ finitely generated for $p \leqslant n$. By assumption, $H_p(\mathcal{D}_{\mathbb{F}G} \otimes_{\mathbb{F}G} F_\bullet) = 0$ for $p \leqslant n$. By \cref{thm:KJZ}, there is a finite index subgroup $H \leqslant G$ and an open subset $U \subseteq H^1(H;\R)$ such that the closure of $U$ contains $H^1(G;\R)$, is invariant under nonzero scalar multiplication, and $H_p(\widehat{\mathbb{F}H}^\varphi \otimes_{\mathbb{F}H} F_\bullet) = 0$ for $p \leqslant n$ and all $\phi \in U$. Since $U$ is nonempty, we can find a surjective character $\varphi \colon H \longrightarrow \Z$ in $U$. To see this, let $\varphi \in H^1(G;\R)$ be a nontrivial character. Since $U$ is open, $\varphi$ can be perturbed so that its image is in $\Q$. Finally, since $H$ is finitely generated, we can rescale the character so that it maps $H$ onto $\Z$.  Then $[\pm \varphi] \in \Sigma_\mathbb{F}^n(H;\mathbb{F})$ by \cref{thm:Main}. It is not hard to show that $[\chi] = [\pm \varphi]$ for any character $\chi \colon H \longrightarrow \R$ with $\ker \varphi \subseteq \ker \chi$. Thus, $\ker \varphi \in \mathtt{FP}_n(\mathbb{F})$ by \cref{thm:genBR}. \qedhere

\end{proof}

\begin{cor}\label{cor:typeFP}
    Let $G$ be virtually RFRS and of type $\FP(\mathbb{F})$. Then $G$ virtually algebraically fibres with kernel of type $\mathtt{FP}(\mathbb F)$ if and only if $G$ is $\DF{G}$-acyclic.
\end{cor}

\begin{proof}
    One direction is clear. If $G$ is $\DF{G}$-acyclic, then $G$ virtually algebraically fibres with kernel $K$ of type $\FP_n(\mathbb F)$ for $n > \cd_\mathbb F(G)$. But then $n > \cd_\mathbb F (K)$, so $K$ is of type $\FP(\mathbb F)$. \qedhere
\end{proof}

We now apply \cref{thm:agrarianMain} to the case $\mathbb{F} = \mathbb{Q}$.

\begin{defn}[$\ell^2$-Betti numbers] \label{def:l2b}
Let $G$ be a torsion-free group satisfying the Atiyah conjecture and let $\mathcal{D}(G)$ be the Linnell skew-field of $G$ (see the following remark). Define
\[
    b_p^{(2)}(G) = \dim_{\mathcal{D}(G)} \Tor_p^{\Q G} (\Q, \mathcal{D}(G))
\]
to be the \textit{$p$th $\ell^2$-Betti number of $G$}.

If a group $G$ has a torsion-free subgroup $H$ of finite index satisfying the Atiyah conjecture, we extend the definition of $\ell^2$-Betti numbers by declaring that
\[
    b_n^{(2)}(G) = \frac{b_n^{(2)}(H)}{[G:H]}.
\]
\end{defn}

\begin{rem}
This definition of $\ell^2$-Betti numbers for torsion-free groups satisfying the Atiyah conjecture agrees with the usual definition by \cite[Lemma 10.28(3)]{Luck02}. Moreover, $\ell^2$-Betti numbers for virtually torsion-free groups are well-defined and coincide with the usual definition by \cite[Theorem 6.54(6)]{Luck02}. We will not give the definition of the Linnell ring $\mathcal{D}(G)$ since that would take us too far afield. We take the Atiyah conjecture to be the statement that $\mathcal{D}(G)$ is a skew-field, which allows us to make \cref{def:l2b}. Linnell showed that this formulation implies the strong Atiyah conjecture over $\Q$ for torsion-free groups \cite{LinnellDivRings93}. The details of the reverse implication can be found in L\"uck's book \cite[Section 10]{Luck02}.
\end{rem}

In the case where $G$ is finitely generated and RFRS, $\mathcal{D}_{\Q G}$ exists and is isomorphic to the Linnell skew-field $\mathcal{D}(G)$ by Jaikin-Zapirain's appendix to \cite{JaikinZapirain2020THEUO}. Then, the $\ell^2$-Betti numbers of $G$ are defined and $b_p^{(2)}(G) = b_p^{\mathcal{D}_{\Q G}}(G)$. Hence, applying \cref{thm:agrarianMain} to the case $\mathbb{F} = \Q$ yields the following result stated in the introduction.


\begin{thm}\label{thm:b2rfrs}
Let $G$ be a virtually RFRS group of type $\mathtt{FP}_n(\Q)$. Then there is a finite index subgroup $H \leqslant G$ admitting a homomorphism onto $\Z$ with kernel of type $\mathtt{FP}_n(\Q)$ if and only if $b_p^{(2)}(G) = 0$ for $p = 0, \dots, n$.
\end{thm}


Thanks to Jaikin-Zapirain's work on rank functions in \cite{JaikinZapirain2020THEUO}, we can give a third characterisation of algebraic fibring. First, we set up some notation and terminology. If $R$ is a ring and $\varphi \colon R \longrightarrow \mathcal D$ is a division $R$-ring, then there is a natural rank function on matrices over $R$, which we denote $\rk_{\mathcal D, \varphi}$ or simply $\rk_{\mathcal D}$ when the map $\varphi$ is understood. If $\varphi$ is \textit{epic}, meaning that $\varphi(R)$ generates $\mathcal D$ as a division ring, then we call $\mathcal D$ \textit{universal} if the $\rk_\mathcal D \geqslant \rk_\mathcal E$ for every division $R$-ring $\psi \colon R \longrightarrow \mathcal E$. Note that if a universal division $R$-ring exists, then it is unique up to $R$-isomorphism by a result of Cohn \cite[Theorem 4.4.1]{cohn1995skew}. If, additionally, $\varphi$ is an injection, then we call $\mathcal D$ the \textit{universal division ring of fractions} for $R$.

\begin{thm}[Jaikin, {\cite[Corollary 1.3]{JaikinZapirain2020THEUO}}]  \label{thm:jaikinUniv}
    If $G$ is a residually-(locally indicable and amenable) group and $\mathbb F$ is a skew-field, then the Hughes-free division ring $\DF{G}$ exists and is the universal division ring of fractions for $\mathbb FG$.
\end{thm}

Following Jaikin-Zapirain, whenever $G$ has a Hughes-free division ring $\DF{G}$, we denote the rank function $\rk_{\DF{G}}$ by $\rk_{\mathbb FG}$. Note, in particular, that \cref{thm:jaikinUniv} holds for RFRS groups since they are residually poly-$\Z$ (see, e.g., \cite[Proposition 4.4]{JaikinZapirain2020THEUO}). Therefore, we have the following corollary, which will be useful to us below.

\begin{cor}\label{cor:rkIneq}
    Let $G$ be a RFRS group and let $\varphi \colon G \longrightarrow \Z$ be a homomorphism. Let $A$ be a matrix over $\mathbb FG$ and let $A^\Z$ be the matrix over $\mathbb F[\Z]$ obtained by applying $\varphi$ to $A$. Then $\rk_{\mathbb FG} A \geqslant \rk_{\mathbb F[\Z]} A^\Z$.
\end{cor}

\begin{proof}
    There is a map $\overline{\varphi} \colon \mathbb F G \longrightarrow \mathbb F[\Z] \longhookrightarrow \DF{[\Z]}$ induced by $\varphi$ and therefore $\rk_{\mathbb FG} A \geqslant \rk_{\DF{[\Z]}, \overline{\varphi}} A = \rk_{\mathbb F[\Z]} A^\Z$ by universality of $\DF{G}$. \qedhere
\end{proof}

The following theorem generalizes Corollary 1.5 of \cite{JaikinZapirain2020THEUO}, where the result is proven for $n = 1$. In the proof, if $M$ is a finitely generated $\mathbb F[\Z]$-module, then we define $\dim M := \dim_{\DF{[\Z]}} (M \otimes_{\mathbb F[\Z]} \DF{[\Z]})$. We will also write $\rk_G$ instead of $\rk_{\mathbb FG}$ to lighten the notation.

\begin{thm}\label{thm:finiteBetti}
    Let $\mathbb F$ be a skew-field and let $G$ be a virtually RFRS group of type $\FP_n(\mathbb F)$. The following are equivalent:
    \begin{enumerate}[label=(\arabic*)]
        \item\label{item:FP} there is a finite index subgroup $H_0 \leqslant G$ and an surjection $\varphi_0 \colon H_0 \longrightarrow \Z$ with $\ker \varphi_0$ of type $\FP_n(\mathbb F)$;
        \item\label{item:Betti} there is a finite index subgroup $H_1 \leqslant G$ and an surjection $\varphi_1 \colon H_1 \longrightarrow \Z$ with $b_p(\ker \varphi_1; \mathbb F) < \infty$ for $p = 0, \dots, n$.
    \end{enumerate}
\end{thm}

\begin{proof}
    If $G$ algebraically fibres with kernel $K$ of type $\FP_n(\mathbb F)$, then there is a free resolution $F_\bullet \longrightarrow \mathbb F \longrightarrow 0$ of the trivial $\mathbb F K$ module $\mathbb F$ such that $F_p$ is finitely generated for all $p \leqslant n$. This resolution can be used to compute the homology of $K$, and therefore $b_p(K; \mathbb F) < \infty$ for all $p \leqslant n$.
    
    In view of \cref{thm:agrarianMain}, to prove the converse it suffices to show that $b_p^{\DF{G}}(G) = 0$ for all $p \leqslant n$. By multiplicativity of $\DF{G}$-Betti numbers (\cref{lem:JBscales}), we may assume that $H_1 = G$. Let $K = \ker \varphi_1$ and write $\mathbb F[\Z]$ for the group algebra $\mathbb F[G/K]$. Moreover, note that $H_p(K;\mathbb F) \cong H_p(G; \mathbb F[\Z])$ for all $p$. Let
    \[
        \cdots \longrightarrow \mathbb F G^{d_p} \longrightarrow \cdots \longrightarrow \mathbb F G^{d_0} \longrightarrow \mathbb F \longrightarrow 0
    \]
    be a free resolution of the trivial $\mathbb F G$-module $\mathbb F$, where $d_p$ is some cardinal for each $p$ and $d_p$ is finite for each $p \leqslant n$, and we use the (non-standard) notation $\mathbb F G^{d_p}$ to denote the $d_p$-fold direct sum of $\mathbb F G$'s, as opposed to the $d_p$-fold direct product. The quotient map $G \longrightarrow \Z$ induces a chain map
    \[
        \begin{tikzcd}
        \cdots \arrow[r, "\partial_{n+2}"]    & \mathbb F G^{d_{n+1}} \arrow[r, "\partial_{n+1}"] \arrow[d] & \mathbb F G^{d_n} \arrow[r, "\partial_n"] \arrow[d] & \cdots \arrow[r, "\partial_1"]    & \mathbb F G^{d_0} \arrow[r, "\partial_0"] \arrow[d] & 0 \\
        \cdots \arrow[r, "\partial_{n+2}^\Z"] & {\mathbb F[\Z]^{d_{n+1}}} \arrow[r, "\partial_{n+1}^\Z"]        & \mathbb F[\Z]^{d_n} \arrow[r, "\partial_n^\Z"]    & \cdots \arrow[r, "\partial_1^\Z"] & {\mathbb F[\Z]^{d_0}} \arrow[r, "\partial_0^\Z"]    & 0\nospacepunct{,}
    \end{tikzcd}
    \]
    where the boundary maps are viewed as matrices and $\partial_p^\Z$ is obtained by applying the map $G \longrightarrow \Z$ to each entry of the matrix $\partial_p$. Note that the homology of the bottom chain complex is $H_\bullet(G;\Q[\Z])$.
    
    To apply Jaikin-Zapirain's results on rank functions, we will need the boundary maps to be between finitely generated free modules. However, $d_{n+1}$ is not finite in general, so we must modify the chain complexes as follows. Since $\mathbb F[\Z]$ is Noetherian and $\mathbb F[\Z]^{d_n}$ is finitely generated, $\im \partial_{n+1}^\Z$ is a finitely generated submodule of $\mathbb F[\Z]^{d_n}$. The preimage of a finite generating set of $\im \partial_{n+1}^\Z$ is contained in a finitely generated free summand $F$ of $\mathbb F[\Z]^{d_{n+1}}$. Notice that the homology of 
    \[    
        F \longrightarrow \mathbb F[\Z]^{d_n} \longrightarrow \cdots \longrightarrow \mathbb F[\Z]^{d_0} \longrightarrow 0
    \]
    is still $H_p(G; \mathbb F[\Z])$ for $p \leqslant n$. The preimage of $F$ in $\mathbb F G^{d_{n+1}}$ is again a finitely generated free summand $\widehat{F}$ of $\mathbb F G^{d_{n+1}}$. Note that it suffices to show that the homology of
    \[
        \DF{G} \otimes_{\Q G} \widehat{F} \longrightarrow \DF{G} \otimes_{\mathbb F G} \mathbb F G^{d_n} \longrightarrow \cdots \longrightarrow \DF{G} \otimes_{\mathbb F G} \mathbb F G^{d_0} \longrightarrow 0
    \]
    vanishes in degrees $\leqslant n$ to show that $b_p^{\DF{G}}(G) = 0$ for all $p \leqslant n$.

    We assume that $d_{n+1}$ is finite and that $F = \mathbb F[\Z]^{d_{n+1}}$ and $\widehat{F} = \mathbb F G^{d_{n+1}}$. Since, for every $p \leqslant n$, the homology $H_p(G;\mathbb F[\Z])$ is finite-dimensional as a $\mathbb F$-vector space, it must be torsion as a $\mathbb F[\Z]$-module. Therefore $\rk_\Z \partial_{p+1}^\Z = \dim \ker \partial_p^\Z$ for every $p \leqslant n$. Now, for each $p \leqslant n$, we have short exact sequences
    \[
        0 \longrightarrow \ker \partial_p^\Z \longrightarrow \mathbb F G^{d_p} \longrightarrow \im \partial_p^\Z \longrightarrow 0
    \]
    which implies that $d_p = \dim \ker \partial_p^\Z + \rk_\Z \partial_p^\Z = \rk_\Z \partial_{p+1}^\Z + \rk_\Z \partial_p^\Z$. Hence,
    \begin{align*}
        d_p - \rk_G \partial_p &= \dim_{\DF{G}} \ker(\DF{G}^{d_p} \xrightarrow{\partial_p} \DF{G}^{d_{p-1}} ) \\
        &\geqslant \rk_G \partial_{p+1} \\
        &\geqslant \rk_\Z \partial_{p+1}^\Z \\
        &= d_p - \rk_\Z \partial_p^\Z \\
        &\geqslant d_p - \rk_G \partial_p,
    \end{align*}
    where we have \cref{cor:rkIneq}. Thus, 
    \[
        \rk_G \partial_{p+1} = \dim_{\DF{G}} \ker(\DF{G}^{d_p} \xrightarrow{\partial_p} \DF{G}^{d_{p-1}} ),
    \]
    and therefore $b_p^{\DF{G}}(G) = 0$ for all $p \leqslant n$. \qedhere
\end{proof}


\begin{cor}\label{cor:charac}
    Let $G$ be a virtually RFRS group and let $n \in \N$.
    \begin{enumerate}
        \item If $\mathbb F$ and $\mathbb F'$ are skew-fields of the same characteristic, then $G$ virtually algebraically fibres with kernel of type $\FP_n(\mathbb F)$ if and only if it virtually algebraically fibres with kernel of type $\FP_n(\mathbb F')$.
        \item If $p$ is a prime such that $G$ algebraically fibres with kernel of type $\FP_n(\mathbb F_p)$, then it fibres with kernel of type $\FP_n(\Q)$.
    \end{enumerate}
\end{cor}

\begin{proof}
    (1) follows from the fact that the Betti numbers of a group with trivial skew-field coefficients depend only on the characteristic of the skew-field. (2) follows from the fact that $b_k(G; \mathbb F_p) \geqslant b_k(G; \Q)$ for any group $G$ and any prime $p$ (this is a consequence of the universal coefficient theorem). \qedhere
\end{proof}



%%%                     %%%
%%%     Application     %%%
%%%                     %%%
\section{Applications} \label{sec:app}

\subsection{Amenable RFRS groups}

\begin{defn}
A group $G$ is \textit{amenable} if for every continuous $G$-action on a compact, Hausdorff space $X$, there is a $G$-invariant probability measure on $X$.
\end{defn}


\begin{defn}[elementary amenable groups] \label{def:elemAm}
The class $\mathcal{E}$ of \textit{elementary amenable} groups is the smallest class such that 
\begin{enumerate}[label = {$\bullet$}]
    \item $\mathcal{E}$ contains all finite groups and all Abelian groups;
    \item if $G \in \mathcal{E}$ then the entire isomorphism class of $G$ is contained in $\mathcal{E}$;
    \item $\mathcal{E}$ is closed under taking subgroups, quotients, extensions, and directed unions.
\end{enumerate}
\end{defn}

All elementary amenable groups are amenable, however there are many examples of amenable groups that are not elementary amenable, the earliest being Grigorchuk's group of intermediate growth \cite{GrigorchukGroup}. For a discussion of more examples, we refer the reader to the introduction of Juschenko's paper \cite{JuschenckoNEA}. The known examples of non-elementary amenable groups all have infinite cohomological dimension over any field.  Moreover, elementary amenable groups of finite cohomological dimension over $\Z$ are virtually solvable by \cite[Lemma 2]{Hillman91} and \cite[Corollary 1]{HillmanLinnell}. We are led to the following question, which was stated in the introduction. 

\begin{q}
Are amenable groups of finite cohomological dimension over $\Z$ virtually solvable?
\end{q}

We obtain a partial answer in the positive direction as an application of \cref{thm:b2rfrs}.

\begin{thm}\label{thm:amRFRSelemAm}
\sloppy If $G$ is a virtually amenable RFRS group of type $\mathtt{FP}(\Q)$, then $G$ is polycyclic-by-finite.
\end{thm}
\begin{proof}
We assume $G$ is amenable, RFRS, and of type $\mathtt{FP}(\Q)$; the virtual claim then follows immediately. Since $G$ is amenable, $b_p^{(2)}(G) = 0$ for all $p$ by \cite[Theorem 7.2(1)]{Luck02}. By \cref{thm:b2rfrs} we obtain a finite index subgroup $H \leqslant G$ and an epimorphism $\varphi \colon H \longrightarrow \Z$ such that $N = \ker \varphi \in \mathtt{FP}_n(\Q)$. It will be necessary to require that $H$ be normal in $G$, which is not an issue since we can replace $H$ with its normal core. Since $\cd_{\Q} N \leqslant \cd_\Q G \leqslant n$, we also have $N \in \mathtt{FP}(\Q)$ \cite[VIII Proposition 6.1]{BrownGroupCohomology}. Because we have a short exact sequence
\[
    1 \longrightarrow N \longrightarrow H \longrightarrow \Z \longrightarrow 1
\]
with $N \in \mathtt{FP}(\Q)$, a theorem of Fel'dman \cite[Theorem 2.4]{Feldman71} (see also \cite[Proposition 2.5]{Bieri76}) gives $\cd_\Q N = \cd_\Q H - \cd_\Q \Z = n - 1$.

Since subgroups of amenable RFRS groups are amenable and RFRS, we can repeat the argument above with $N$ instead of $G$. Iterating this process as many times as necessary, we obtain a subnormal series
\[
    G_0 \leqslant G_1 \leqslant \cdots \leqslant G_{n-1} \leqslant G_n = G
\]
of $G$ such that $\cd_\Q G_j = j$ for each $j$ (note that $N = G_{n-1}$ here). But the only torsion-free group of cohomological dimension $0$ over $\Q$ is the trivial group, so we conclude that $G$ is polycyclic-by-finite.
\end{proof}


\begin{rem}
The assumption that $G$ be RFRS is necessary. For example, the Baumslag-Solitar group $\BS(1,n)$, with $n > 1$, is amenable and of finite type, but it is not polycyclic-by-finite.
\end{rem}

Baer's conjecture states that if $G$ is a group with a Noetherian group ring $\Z G$, then $G$ is a polycyclic-by-finite group. The author is grateful to Sam Hughes for pointing out the following consequence of \cref{thm:amRFRSelemAm}.

\begin{cor}\label{cor:baer}
Let $G$ be a virtually RFRS group of type $\mathtt{FP}(\Q)$. If $\Z G$ is Noetherian, then $G$ is polycyclic-by-finite.
\end{cor}
\begin{proof}
It is enough to prove the claim in the case that $G$ is RFRS. In this case, $\mathbb{Z} G$ embeds into the Linnell-skew field $\mathcal{D}(G)$, and therefore $\Z G$ is a domain. Since Noetherian domains are Ore domains \cite[p.~47]{McConnellRobsonNNR}, we have that $\Z G$ is an Ore domain. It then follows that $\Q G$ is an Ore domain, which implies that $G$ is amenable by Kielak's appendix to \cite{BartholdiKielakApp}. Then $G$ is polycyclic-by-finite by \cref{thm:amRFRSelemAm}.
\end{proof}







\subsection{Arithmetic lattices and subgroups of hyperbolic groups}

Recall that hyperbolic groups are always of type $\F_\infty$ (and of type $\F$ if they are torsion-free). However, their subgroups can exhibit many interesting finiteness properties. In \cite{RipsF1notF2_1982}, Rips gave the first example of an incoherent hyperbolic group; phrased in terms of finiteness properties, this gives an example of a hyperbolic group with a subgroup that is of type $\F_1$ but not $\F_2$. In \cite{BradyF2notF3_1999}, Brady constructed a type $\F_2$ subgroup of a hyperbolic group that is not of type $\F_3$; this provided the first example of a finitely presented non-hyperbolic subgroup of a hyperbolic group. In the same paper, Brady asked whether there are subgroups of hyperbolic groups that are of type $\F_n$ but not $\F_{n+1}$ for all $n$. More examples of $\F_2$-not-$\F_3$ subgroups of hyperbolic groups were provided by Kropholler \cite{KrophollerF2notF3} and Lodha \cite{LodhaF_2notF_3}, and in \cite{isenrich2021hyperbolic}, Llosa Isenrich, Martelli, and Py constructed the first examples of $\F_3$-not-$\F_4$ subgroups of hyperbolic groups. This result was extended in a subsequent paper \cite{IsenrichPy2022}, where Llosa Isenrich and Py completely answer Brady's question by exhibiting cocompact hyperbolic arithmetic lattices in $\PU(n,1)$ with $\F_{n-1}$-not-$\F_n$ subgroups for all $n$. We also mention the related work of Italiano, Martelli, and Migliorini, who constructed the first example of a non-hyperbolic type $\F$ subgroup of a hyperbolic group \cite{IMM_5mfld}.

In \cite{isenrich2021hyperbolic}, Llosa Isenrich, Martelli, and Py remark that one can use \cref{thm:b2rfrs} to show that there are subgroups of hyperbolic lattices in $\PO(2n,1)$ that are of type $\FP_{n-1}(\Q)$ but not $\FP_n(\Q)$. We record their argument here, and also mention that the same line of reasoning shows that there are many hyperbolic lattices in $\PO(2n+1,1)$ that virtually fibre with kernel of type $\FP(\Q)$.

\begin{prop}[{\cite[Proposition 19]{isenrich2021hyperbolic}}]\label{prop:lattices}
    Let $\Gamma < \PO(n,1)$ be a cocompact cubulable lattice. If $n = 2k$ is even then $\Gamma$ virtually fibres with kernel of type $\FP_{k-1}(\Q)$ but not $\FP_k(\Q)$. If $n$ is odd, then $\Gamma$ virtually fibres with kernel of type $\FP(\Q)$.
\end{prop}

\begin{rem}
    In \cite{BHW_2011}, Bergeron, Haglund, and Wise show that any standard arithmetic subgroup of $\PO(n,1)$ is cubulated, so \cref{prop:lattices} applies to a nonempty class of groups. We refer the reader to their paper for the definition of standard.
\end{rem}

\begin{proof}
    By Agol's theorem \cite[Theorem 1.1]{AgolHaken}, $\Gamma$ is virtually special and in particular virtually RFRS. By, for instance \cite[Theorem 3.3]{KammeyerLattices}, the $\ell^2$-Betti numbers of lattices in semi-simple Lie groups vanish except in the middle dimension, where they are nonzero. If $n = 2k$, then $\Gamma$ virtually fibres with kernel of type $\FP_{k-1}(\Q)$, but $b_k^{(2)}(\Gamma) \neq 0$ so the kernel cannot be of type $\FP_n(\Q)$ by \cref{thm:agrarianMain}. If $n$ is odd, then $\Gamma$ is $\ell^2$-acyclic and therefore virtually fibres with kernel of type $\FP(\Q)$ in this case, where we have also used \cref{cor:typeFP}. \qedhere
\end{proof}

\bibliographystyle{alpha}
\bibliography{Fisher}

\end{document}