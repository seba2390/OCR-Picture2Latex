\documentclass[11pt, letterpaper]{amsart}
% CVPR 2022 Paper Template
% based on the CVPR template provided by Ming-Ming Cheng (https://github.com/MCG-NKU/CVPR_Template)
% modified and extended by Stefan Roth (stefan.roth@NOSPAMtu-darmstadt.de)

\documentclass[10pt,twocolumn,letterpaper]{article}

%%%%%%%%% PAPER TYPE  - PLEASE UPDATE FOR FINAL VERSION
%\usepackage[review]{cvpr}      % To produce the REVIEW version
%\usepackage{cvpr}              % To produce the CAMERA-READY version
\usepackage[pagenumbers]{cvpr} % To force page numbers, e.g. for an arXiv version

% Include other packages here, before hyperref.
\usepackage{graphicx}
\usepackage{amsmath}
\usepackage{amssymb}
\usepackage{booktabs}
\usepackage{paralist}




% It is strongly recommended to use hyperref, especially for the review version.
% hyperref with option pagebackref eases the reviewers' job.
% Please disable hyperref *only* if you encounter grave issues, e.g. with the
% file validation for the camera-ready version.
%
% If you comment hyperref and then uncomment it, you should delete
% ReviewTempalte.aux before re-running LaTeX.
% (Or just hit 'q' on the first LaTeX run, let it finish, and you
%  should be clear).
\usepackage[pagebackref,breaklinks,colorlinks]{hyperref}


% Support for easy cross-referencing
\usepackage[capitalize]{cleveref}
\crefname{section}{Sec.}{Secs.}
\Crefname{section}{Section}{Sections}
\Crefname{table}{Table}{Tables}
\crefname{table}{Tab.}{Tabs.}


%%%%%%%%% PAPER ID  - PLEASE UPDATE
\def\cvprPaperID{5910} % *** Enter the CVPR Paper ID here
\def\confName{CVPR}
\def\confYear{2022}


\begin{document}

%%%%%%%%% TITLE - PLEASE UPDATE
\title{Dyadic Human Motion Prediction}

\author{
	Isinsu Katircioglu$^1$
	\and
	Costa Georgantas$^2$
	\and
	Mathieu Salzmann$^{1,3}$
	\and
	Pascal Fua$^1$   
	\and
	$^1$CVLab, EPFL, Switzerland\quad
	$^2$CHUV, Switzerland\\
	$^3$ClearSpace SA, Switzerland
}

\maketitle

\newcommand{\marginnote}[1]{\marginpar{\framebox{\framebox{#1}}}}
% Comment macro: Usage \comment{Author}{Comment}
\newcommand{\comment}[2]{[\marginnote{#1}\textit{#1}:\ \textit{#2}]}

% \newcommand{\bra}[1]{#1^H}
% \newcommand{\ket}[1]{#1}
% \newcommand{\innerp}[2]{#1^H#2}
% \newcommand{\outerp}[2]{#1#2^H}
% \newcommand{\matrixel}[3]{#1^H#2#3}
% \newcommand{\proj}[1]{\outerp{#1}{#1}}
% \newcommand{\abs}[1]{\vert#1\vert}
% 
% 
% %norms
% \newcommand{\norm}[2]{\left\Vert#1\right\Vert_{#2}}
% 
% %a matrix or vector
% \newcommand{\mat}[2]{\lefto[\begin{array}{#1}#2\end{array}\right]}
% 
% %mathbf etc.
% \newcommand{\mbf}{\mathbf}
% \newcommand{\mbb}{\mathbb}
% \newcommand{\mcl}{\mathcal}
 \newcommand{\trm}{\textrm}
% 
% 
% %Theorems, etc.
% %\theoremstyle{definition}
% \theoremstyle{plain}
%
% 
% %Expectation
% \newcommand{\expect}[1]{\langle#1\rangle}
% 
% %Trace
% \DeclareMathOperator{\tr}{tr}
% \DeclareMathOperator{\Tr}{Tr}
% 
% %Miscellaneous stuff
% \DeclareMathOperator{\supp}{supp}
% \DeclareMathOperator{\Span}{span}
% 
% %identity
% \DeclareMathOperator{\id}{\mathbf{1}}
% 
% %BS addition
% \DeclareMathOperator{\bst}{\stackrel{\land}{+}}
% \DeclareMathOperator{\bs}{\widehat{+}}
% 
% %ladder operators
% \newcommand{\up}{a^\dagger}
% \newcommand{\down}{a}
% 
% %vec operator
% %\DeclareMathOperator{\vect}{vec}
% 
% %equations
% \def\ba#1\ea{\begin{align*}#1\end{align*}}
% %with numeration
% \def\ban#1\ean{\begin{align}#1\end{align}}
% % the same, centered
% \def\bac#1\eac{\vspace{\abovedisplayskip}{\par\centering$\begin{aligned}#1\end{aligned}$\par}\addvspace{\belowdisplayskip}}

%%%%% parentheses denoting arguments, we want an opening object. Use \lefto in
%%%%% these cases.
% \newcommand{\lefto}{\mathopen{}\left}
% 
% %real and imaginary part
% \renewcommand{\Re}{\mathfrak{Re}}
% \renewcommand{\Im}{\mathfrak{Im}}
% 
% % in case you need to break equations over several lines
 \newcommand{\dummyrel}[1]{\mathrel{\hphantom{#1}}\strut\mskip-\medmuskip}
%%%%%%%%%%%%%%%%%%%%%
%:- Theorems
 \newtheorem{thm}{Theorem}
 \newtheorem{cor}[thm]{Corollary}   % Turned off theorem numbering
 \newtheorem{prop}{Proposition}
 \newtheorem{defi}{Definition}
 \newtheorem{lem}[thm]{Lemma}
 \newtheorem{exmpl}{Example}
% \newtheorem{st}{Statement}
% \newtheorem{conj}{Conjecture}

%%%%%
%% robust recovery from sparse noise
\safemath{\dictab}{[\,\dicta\,\,\dictb\,]}


\safemath{\ysig}{\bmy}
\safemath{\ysighat}{\hat{\ysig}}
\safemath{\ysigdim}{M}
%
\safemath{\xsig}{\bmx}
\safemath{\xsigdim}{N}
\safemath{\nx}{n_x}
%
\safemath{\zsig}{\bmz}
\safemath{\zsigdim}{\ysigdim}
%
\safemath{\rsig}{\bmr}
%
\safemath{\Adict}{\bA}
\safemath{\Adicttilde}{\widetilde{\Adict}}
\safemath{\Adictdim}{\outputdim\times\xsigdim}
\safemath{\avec}{\bma}
\safemath{\avectilde}{\tilde{\avec}}
%
\safemath{\Bdict}{\bB}
\safemath{\Bdicttilde}{\widetilde{\Bdict}}
%
\safemath{\Cdict}{\bC}
\safemath{\cvec}{\bmc}
%
\safemath{\Ddict}{\bD}
\safemath{\Ddictdim}{\ysigdim\times\xsigdim}
\safemath{\dvec}{\bmd}
\safemath{\Ddicttilde}{\widetilde{\bD}}
%
\safemath{\Bonb}{\bB}
\safemath{\bvec}{\bmb}
\safemath{\Bonbdim}{\ysigdim\times\ysigdim}
%
\safemath{\noise}{\bmn}
\safemath{\noisedim}{\ysigim}
%
\safemath{\err}{\bme}
\safemath{\errdim}{\ysigdim}
\safemath{\errset}{\setE}
\safemath{\nerr}{n_e}
%
\safemath{\delop}{\bP_\errset}
\safemath{\delopc}{\bP_{{\errset}^c}}

%




%%
% Complex i and j 
\safemath{\cplxi}{\imath}
\safemath{\cplxj}{\jmath}
% Comb signal
%\safemath{\comb}{\matI\matI\matI}
\newcommand{\comb}[1]{\vecdelta_{#1}}
%:- Definition dictionary
\safemath{\dict}{\matD}
\safemath{\inputdim}{N}		% number of columns of dictionary D
\safemath{\outputdim}{M}		%number of rows of dictionary D
\safemath{\sparsity}{S}	%sparsity
\safemath{\inputdimA}{{N_a}}	%total number of elements in dictionary A
\safemath{\inputdimB}{{N_b}}	%total number of elements in dictionary B
\safemath{\elemA}{{n_a}}	%number of elements chosen from dictionary A
\safemath{\elemB}{{n_b}}	%number of elements chosen from dictionary B
\safemath{\resA}{\matR_a}	%restriction map to elements of dictionary A
\safemath{\resB}{\matR_b}	%restriction map to elements of dictionary B
\safemath{\subD}{\matS} %subdictionary
\safemath{\subA}{\matS_a} %subdictionary part of A
\safemath{\subB}{\matS_b} %subdictionary part of B
\safemath{\dicta}{\matA} 	% first subdictionary
\safemath{\dictb}{\matB} 	% second subdictionary
\safemath{\hollowS}{H}
\safemath{\hollowA}{H_a}
\safemath{\hollowB}{H_b}
\safemath{\cross}{Z}
\safemath{\coh}{\mu_d}			% coherence dictionary
\safemath{\coha}{\mu_a}			% coherence first subdictionary
\safemath{\cohb}{\mu_b}			% coherence second subdictionary
\safemath{\mubs}{\nu}	%block sub-coherence
\safemath{\cohm}{\mu_m} %mutual coherence
\safemath{\dictset}{\setD}	% set of dictionaries
\safemath{\dictsetp}{\dictset(\coh,\coha,\cohb)}	% set of dictionaries parametrized
\safemath{\dictsetgen}{\dictset_\text{gen}}
\safemath{\dictsetgenp}{\dictsetgen(\coh)}
\safemath{\dictsetonb}{\dictset_\text{onb}}
\safemath{\dictsetonbp}{\dictsetonb(\coh)}

\safemath{\leftside}{U}
\safemath{\rightsideA}{R_a}
\safemath{\rightsideB}{R_b}

\safemath{\indexS}{\setI_S} %set of indices participating in sub-dictionary S


\safemath{\na}{n_a}			% cardinality of set of linearly independent columns of first dictionary
\safemath{\nb}{n_b}			% cardinality of set of linearly independent columns of second dictionary
\safemath{\coeffa}{p_i}	%coefficients for columns of A
\safemath{\coeffb}{q_j}	%coefficients for columns of B
\safemath{\seta}{\setP}		% set of linearly independent columns of A
\safemath{\setb}{\setQ}     % set of linearly independent columns of B
\safemath{\setw}{\setW}	%set of n largest elements of w
\safemath{\setz}{\setZ}	%set of L-n largest elements of z
\safemath{\cola}{\veca}		% generic element of the dictionary A
\safemath{\colb}{\vecb}		% generic element of the dictionary B
\safemath{\cold}{\vecd}		% generic element of the dictionary D
\safemath{\inputvec}{\vecx} 	%coefficient vector (input)
\safemath{\error}{\vece}	%error vector
\safemath{\noiseout}{\vecz} 	%noisy output vector
\safemath{\inputvecel}{x}
\safemath{\inputveca}{\vecx_a}
\safemath{\inputvecb}{\vecx_b}
\safemath{\outputvec}{\vecy}	%output of Dictionary
\safemath{\lambdamin}{\lambda_{\mathrm{min}}}
%:- Math operators
\DeclareMathOperator{\spark}{spark}
\newcommand{\pos}[1]{\lefto[#1\right]^+}
\newcommand{\normtwo}[1]{\vecnorm{#1}_2}
\newcommand{\normone}[1]{\vecnorm{#1}_1}
\newcommand{\normzero}[1]{\vecnorm{#1}_0}
\newcommand{\norminf}[1]{\vecnorm{#1}_\infty}
\newcommand{\norminftilde}[1]{\vecnorm{#1}_{\widetilde\infty}}
\newcommand{\normfro}[1]{\vecnorm{#1}_F}
%\newcommand{\spectralnorm}[1]{\vecnorm{#1}_{2,2}}
\newcommand{\spectralnorm}[1]{\vecnorm{#1}}
\safemath{\elltwo}{\ell_2}
\safemath{\ellone}{\ell_1}
\safemath{\ellzero}{\ell_0}
\safemath{\ellinf}{\ell_\infty}
\safemath{\ellinftilde}{\ell_{\widetilde\infty}}
\safemath{\licard}{Z(\coh,\coha,\cohb)}
\safemath{\xsol}{\hat{x}}
\safemath{\xbord}{x_b}		%Solution at the border
\safemath{\xstat}{x_s}		%Solution stationary in l0 prob
\safemath{\xstatLone}{\tilde{x}_s}
\safemath{\order}{\mathcal{O}} %order notation (big O)
\safemath{\scales}{\Theta} %scales as
%
\safemath{\ones}{\mathbf{1}} %all ones matrix
\safemath{\zeroes}{\mathbf{0}} %all zeroes matrix
%
\safemath{\thlone}{\kappa(\coh,\cohb)} %treshold l1 problem
\safemath{\constoneA}{\delta} %constant in l1 theorem to save space
\safemath{\constoneB}{\epsilon} %constant in l1 theorem to save space
\safemath{\nlarge}{L}				   %num large elements
\safemath{\sumlarge}{S_\nlarge}
\DeclareMathOperator{\kernel}{kern}	   % kernel of a matrix
\safemath{\maxlarger}{P_\nlarge}	   % maximum in Gribonval and Nielsen
\safemath{\Pzero}{\textrm{P0}}	
\safemath{\Pone}{\textrm{P1}}
\safemath{\vecfir}{\vecw}			 % \vecv element of the kernel of the dictionary, \vecv=[\vecfir \vecsec]
\safemath{\vecsec}{\vecz}
\safemath{\elvecfir}{w}              % element of vecfir
\safemath{\elvecsec}{z}				 % element of vecsec
\safemath{\nlargefir}{n}
\safemath{\normout}{\gamma}
\safemath{\auxfun}{h}
\safemath{\supp}{\textrm{supp}}%support

\safemath{\indexa}{\ell}
\safemath{\indexb}{r}
\safemath{\indexc}{i}
\safemath{\indexd}{j}


\safemath{\project}{P}%projector
  In this paper, we explore the connection between secret key agreement and secure omniscience within the setting of the multiterminal source model with a wiretapper who has side information. While the secret key agreement problem considers the generation of a maximum-rate secret key through public discussion, the secure omniscience problem is concerned with communication protocols for omniscience that minimize the rate of information leakage to the wiretapper. The starting point of our work is a lower bound on the minimum leakage rate for omniscience, $\rl$, in terms of the wiretap secret key capacity, $\wskc$. Our interest is in identifying broad classes of sources for which this lower bound is met with equality, in which case we say that there is a duality between secure omniscience and secret key agreement. We show that this duality holds in the case of certain finite linear source (FLS) models, such as two-terminal FLS models and pairwise independent network models on trees with a linear wiretapper. Duality also holds for any FLS model in which $\wskc$ is achieved by a perfect linear secret key agreement scheme. We conjecture that the duality in fact holds unconditionally for any FLS model. On the negative side, we give an example of a (non-FLS) source model for which duality does not hold if we limit ourselves to communication-for-omniscience protocols with at most two (interactive) communications.  We also address the secure function computation problem and explore the connection between the minimum leakage rate for computing a function and the wiretap secret key capacity.
  
%   Finally, we demonstrate the usefulness of our lower bound on $\rl$ by using it to derive equivalent conditions for the positivity of $\wskc$ in the multiterminal model. This extends a recent result of Gohari, G\"{u}nl\"{u} and Kramer (2020) obtained for the two-user setting.
  
   
%   In this paper, we study the problem of secret key generation through an omniscience achieving communication that minimizes the 
%   leakage rate to a wiretapper who has side information in the setting of multiterminal source model.  We explore this problem by deriving a lower bound on the wiretap secret key capacity $\wskc$ in terms of the minimum leakage rate for omniscience, $\rl$. 
%   %The former quantity is defined to be the maximum secret key rate achievable, and the latter one is defined as the minimum possible leakage rate about the source through an omniscience scheme to a wiretapper. 
%   The main focus of our work is the characterization of the sources for which the lower bound holds with equality \textemdash it is referred to as a duality between secure omniscience and wiretap secret key agreement. For general source models, we show that duality need not hold if we limit to the communication protocols with at most two (interactive) communications. In the case when there is no restriction on the number of communications, whether the duality holds or not is still unknown. However, we resolve this question affirmatively for two-user finite linear sources (FLS) and pairwise independent networks (PIN) defined on trees, a subclass of FLS. Moreover, for these sources, we give a single-letter expression for $\wskc$. Furthermore, in the direction of proving the conjecture that duality holds for all FLS, we show that if $\wskc$ is achieved by a \emph{perfect} secret key agreement scheme for FLS then the duality must hold. All these results mount up the evidence in favor of the conjecture on FLS. Moreover, we demonstrate the usefulness of our lower bound on $\wskc$ in terms of $\rl$ by deriving some equivalent conditions on the positivity of secret key capacity for multiterminal source model. Our result indeed extends the work of Gohari, G\"{u}nl\"{u} and Kramer in two-user case.
% \leavevmode
% \\
% \\
% \\
% \\
% \\
\section{Introduction}
\label{introduction}

AutoML is the process by which machine learning models are built automatically for a new dataset. Given a dataset, AutoML systems perform a search over valid data transformations and learners, along with hyper-parameter optimization for each learner~\cite{VolcanoML}. Choosing the transformations and learners over which to search is our focus.
A significant number of systems mine from prior runs of pipelines over a set of datasets to choose transformers and learners that are effective with different types of datasets (e.g. \cite{NEURIPS2018_b59a51a3}, \cite{10.14778/3415478.3415542}, \cite{autosklearn}). Thus, they build a database by actually running different pipelines with a diverse set of datasets to estimate the accuracy of potential pipelines. Hence, they can be used to effectively reduce the search space. A new dataset, based on a set of features (meta-features) is then matched to this database to find the most plausible candidates for both learner selection and hyper-parameter tuning. This process of choosing starting points in the search space is called meta-learning for the cold start problem.  

Other meta-learning approaches include mining existing data science code and their associated datasets to learn from human expertise. The AL~\cite{al} system mined existing Kaggle notebooks using dynamic analysis, i.e., actually running the scripts, and showed that such a system has promise.  However, this meta-learning approach does not scale because it is onerous to execute a large number of pipeline scripts on datasets, preprocessing datasets is never trivial, and older scripts cease to run at all as software evolves. It is not surprising that AL therefore performed dynamic analysis on just nine datasets.

Our system, {\sysname}, provides a scalable meta-learning approach to leverage human expertise, using static analysis to mine pipelines from large repositories of scripts. Static analysis has the advantage of scaling to thousands or millions of scripts \cite{graph4code} easily, but lacks the performance data gathered by dynamic analysis. The {\sysname} meta-learning approach guides the learning process by a scalable dataset similarity search, based on dataset embeddings, to find the most similar datasets and the semantics of ML pipelines applied on them.  Many existing systems, such as Auto-Sklearn \cite{autosklearn} and AL \cite{al}, compute a set of meta-features for each dataset. We developed a deep neural network model to generate embeddings at the granularity of a dataset, e.g., a table or CSV file, to capture similarity at the level of an entire dataset rather than relying on a set of meta-features.
 
Because we use static analysis to capture the semantics of the meta-learning process, we have no mechanism to choose the \textbf{best} pipeline from many seen pipelines, unlike the dynamic execution case where one can rely on runtime to choose the best performing pipeline.  Observing that pipelines are basically workflow graphs, we use graph generator neural models to succinctly capture the statically-observed pipelines for a single dataset. In {\sysname}, we formulate learner selection as a graph generation problem to predict optimized pipelines based on pipelines seen in actual notebooks.

%. This formulation enables {\sysname} for effective pruning of the AutoML search space to predict optimized pipelines based on pipelines seen in actual notebooks.}
%We note that increasingly, state-of-the-art performance in AutoML systems is being generated by more complex pipelines such as Directed Acyclic Graphs (DAGs) \cite{piper} rather than the linear pipelines used in earlier systems.  
 
{\sysname} does learner and transformation selection, and hence is a component of an AutoML systems. To evaluate this component, we integrated it into two existing AutoML systems, FLAML \cite{flaml} and Auto-Sklearn \cite{autosklearn}.  
% We evaluate each system with and without {\sysname}.  
We chose FLAML because it does not yet have any meta-learning component for the cold start problem and instead allows user selection of learners and transformers. The authors of FLAML explicitly pointed to the fact that FLAML might benefit from a meta-learning component and pointed to it as a possibility for future work. For FLAML, if mining historical pipelines provides an advantage, we should improve its performance. We also picked Auto-Sklearn as it does have a learner selection component based on meta-features, as described earlier~\cite{autosklearn2}. For Auto-Sklearn, we should at least match performance if our static mining of pipelines can match their extensive database. For context, we also compared {\sysname} with the recent VolcanoML~\cite{VolcanoML}, which provides an efficient decomposition and execution strategy for the AutoML search space. In contrast, {\sysname} prunes the search space using our meta-learning model to perform hyperparameter optimization only for the most promising candidates. 

The contributions of this paper are the following:
\begin{itemize}
    \item Section ~\ref{sec:mining} defines a scalable meta-learning approach based on representation learning of mined ML pipeline semantics and datasets for over 100 datasets and ~11K Python scripts.  
    \newline
    \item Sections~\ref{sec:kgpipGen} formulates AutoML pipeline generation as a graph generation problem. {\sysname} predicts efficiently an optimized ML pipeline for an unseen dataset based on our meta-learning model.  To the best of our knowledge, {\sysname} is the first approach to formulate  AutoML pipeline generation in such a way.
    \newline
    \item Section~\ref{sec:eval} presents a comprehensive evaluation using a large collection of 121 datasets from major AutoML benchmarks and Kaggle. Our experimental results show that {\sysname} outperforms all existing AutoML systems and achieves state-of-the-art results on the majority of these datasets. {\sysname} significantly improves the performance of both FLAML and Auto-Sklearn in classification and regression tasks. We also outperformed AL in 75 out of 77 datasets and VolcanoML in 75  out of 121 datasets, including 44 datasets used only by VolcanoML~\cite{VolcanoML}.  On average, {\sysname} achieves scores that are statistically better than the means of all other systems. 
\end{itemize}


%This approach does not need to apply cleaning or transformation methods to handle different variances among datasets. Moreover, we do not need to deal with complex analysis, such as dynamic code analysis. Thus, our approach proved to be scalable, as discussed in Sections~\ref{sec:mining}.
\section{Related Work}\label{sec:related}
 
The authors in \cite{humphreys2007noncontact} showed that it is possible to extract the PPG signal from the video using a complementary metal-oxide semiconductor camera by illuminating a region of tissue using through external light-emitting diodes at dual-wavelength (760nm and 880nm).  Further, the authors of  \cite{verkruysse2008remote} demonstrated that the PPG signal can be estimated by just using ambient light as a source of illumination along with a simple digital camera.  Further in \cite{poh2011advancements}, the PPG waveform was estimated from the videos recorded using a low-cost webcam. The red, green, and blue channels of the images were decomposed into independent sources using independent component analysis. One of the independent sources was selected to estimate PPG and further calculate HR, and HRV. All these works showed the possibility of extracting PPG signals from the videos and proved the similarity of this signal with the one obtained using a contact device. Further, the authors in \cite{10.1109/CVPR.2013.440} showed that heart rate can be extracted from features from the head as well by capturing the subtle head movements that happen due to blood flow.

%
The authors of \cite{kumar2015distanceppg} proposed a methodology that overcomes a challenge in extracting PPG for people with darker skin tones. The challenge due to slight movement and low lighting conditions during recording a video was also addressed. They implemented the method where PPG signal is extracted from different regions of the face and signal from each region is combined using their weighted average making weights different for different people depending on their skin color. 
%

There are other attempts where authors of \cite{6523142,6909939, 7410772, 7412627} have introduced different methodologies to make algorithms for estimating pulse rate robust to illumination variation and motion of the subjects. The paper \cite{6523142} introduces a chrominance-based method to reduce the effect of motion in estimating pulse rate. The authors of \cite{6909939} used a technique in which face tracking and normalized least square adaptive filtering is used to counter the effects of variations due to illumination and subject movement. 
The paper \cite{7410772} resolves the issue of subject movement by choosing the rectangular ROI's on the face relative to the facial landmarks and facial landmarks are tracked in the video using pose-free facial landmark fitting tracker discussed in \cite{yu2016face} followed by the removal of noise due to illumination to extract noise-free PPG signal for estimating pulse rate. 

Recently, the use of machine learning in the prediction of health parameters have gained attention. The paper \cite{osman2015supervised} used a supervised learning methodology to predict the pulse rate from the videos taken from any off-the-shelf camera. Their model showed the possibility of using machine learning methods to estimate the pulse rate. However, our method outperforms their results when the root mean squared error of the predicted pulse rate is compared. The authors in \cite{hsu2017deep} proposed a deep learning methodology to predict the pulse rate from the facial videos. The researchers trained a convolutional neural network (CNN) on the images generated using Short-Time Fourier Transform (STFT) applied on the R, G, \& B channels from the facial region of interests.
The authors of \cite{osman2015supervised, hsu2017deep} only predicted pulse rate, and we extended our work in predicting variance in the pulse rate measurements as well.

All the related work discussed above utilizes filtering and digital signal processing to extract PPG signals from the video which is further used to estimate the PR and PRV.  %
The method proposed in \cite{kumar2015distanceppg} is person dependent since the weights will be different for people with different skin tone. In contrast, we propose a deep learning model to predict the PR which is independent of the person who is being trained. Thus, the model would work even if there is no prior training model built for that individual and hence, making our model robust. 

%









\section{Proposed Approach} \label{sec:method}

Our goal is to create a unified model that maps task representations (e.g., obtained using task2vec~\cite{achille2019task2vec}) to simulation parameters, which are in turn used to render synthetic pre-training datasets for not only tasks that are seen during training, but also novel tasks.
This is a challenging problem, as the number of possible simulation parameter configurations is combinatorially large, making a brute-force approach infeasible when the number of parameters grows. 

\subsection{Overview} 

\cref{fig:controller-approach} shows an overview of our approach. During training, a batch of ``seen'' tasks is provided as input. Their task2vec vector representations are fed as input to \ours, which is a parametric model (shared across all tasks) mapping these downstream task2vecs to simulation parameters, such as lighting direction, amount of blur, background variability, etc.  These parameters are then used by a data generator (in our implementation, built using the Three-D-World platform~\cite{gan2020threedworld}) to generate a dataset of synthetic images. A classifier model then gets pre-trained on these synthetic images, and the backbone is subsequently used for evaluation on specific downstream task. The classifier's accuracy on this task is used as a reward to update \ours's parameters. 
Once trained, \ours can also be used to efficiently predict simulation parameters in {\em one-shot} for ``unseen'' tasks that it has not encountered during training. 


\subsection{\ours Model} 


Let us denote \ours's parameters with $\theta$. Given the task2vec representation of a downstream task $\bs{x} \in \mc{X}$ as input, \ours outputs simulation parameters $a \in \Omega$. The model consists of $M$ output heads, one for each simulation parameter. In the following discussion, just as in our experiments, each simulation parameter is discretized to a few levels to limit the space of possible outputs. Each head outputs a categorical distribution $\pi_i(\bs{x}, \theta) \in \Delta^{k_i}$, where $k_i$ is the number of discrete values for parameter $i \in [M]$, and $\Delta^{k_i}$, a standard $k_i$-simplex. The set of argmax outputs $\nu(\bs{x}, \theta) = \{\nu_i | \nu_i = \argmax_{j \in [k_i]} \pi_{i, j} ~\forall i \in [M]\}$ is the set of simulation parameter values used for synthetic data generation. Subsequently, we drop annotating the dependence of $\pi$ and $\nu$ on $\theta$ and $\bs{x}$ when clear.

\subsection{\ours Training} 


Since Task2Sim aims to maximize downstream accuracy after pre-training, we use this accuracy as the reward in our training optimization\footnote{Note that our rewards depend only on the task2vec input and the output action and do not involve any states, and thus our problem can be considered similar to a stateless-RL or contextual bandits problem \cite{langford2007epoch}.}.
Note that this downstream accuracy is a non-differentiable function of the output simulation parameters (assuming any simulation engine can be used as a black box) and hence direct gradient-based optimization cannot be used to train \ours. Instead, we use REINFORCE~\cite{williams1992simple}, to approximate gradients of downstream task performance with respect to model parameters $\theta$. 

\ours's outputs represent a distribution over ``actions'' corresponding to different values of the set of $M$ simulation parameters. $P(a) = \prod_{i \in [M]} \pi_i(a_i)$ is the probability of picking action $a = [a_i]_{i \in [M]}$, under policy $\pi = [\pi_i]_{i \in [M]}$. Remember that the output $\pi$ is a function of the parameters $\theta$ and the task representation $\bs{x}$. To train the model, we maximize the expected reward under its policy, defined as
\begin{align}
    R = \E_{a \in \Omega}[R(a)] = \sum_{a \in \Omega} P(a) R(a)
\end{align}
where $\Omega$ is the space of all outputs $a$ and $R(a)$ is the reward when parameter values corresponding to action $a$ are chosen. Since reward is the downstream accuracy, $R(a) \in [0, 100]$.  
Using the REINFORCE rule, we have
\begin{align}
    \nabla_{\theta} R 
    &= \E_{a \in \Omega} \left[ (\nabla_{\theta} \log P(a)) R(a) \right] \\
    &= \E_{a \in \Omega} \left[ \left(\sum_{i \in [M]} \nabla_{\theta} \log \pi_i(a_i) \right) R(a) \right]
\end{align}
where the 2nd step comes from linearity of the derivative. In practice, we use a point estimate of the above expectation at a sample $a \sim (\pi + \epsilon)$ ($\epsilon$ being some exploration noise added to the Task2Sim output distribution) with a self-critical baseline following \cite{rennie2017self}:
\begin{align} \label{eq:grad-pt-est}
    \nabla_{\theta} R \approx \left(\sum_{i \in [M]} \nabla_{\theta} \log \pi_i(a_i) \right) \left( R(a) - R(\nu) \right) 
\end{align}
where, as a reminder $\nu$ is the set of the distribution argmax parameter values from the \name{} model heads.

A pseudo-code of our approach is shown in \cref{alg:train}.  Specifically, we update the model parameters $\theta$ using minibatches of tasks sampled from a set of ``seen'' tasks. Similar to \cite{oh2018self}, we also employ self-imitation learning biased towards actions found to have better rewards. This is done by keeping track of the best action encountered in the learning process and using it for additional updates to the model, besides the ones in \cref{ln:update} of \cref{alg:train}. 
Furthermore, we use the test accuracy of a 5-nearest neighbors classifier operating on features generated by the pretrained backbone as a proxy for downstream task performance since it is computationally much faster than other common evaluation criteria used in transfer learning, e.g., linear probing or full-network finetuning. Our experiments demonstrate that this proxy evaluation measure indeed correlates with, and thus, helps in final downstream performance with linear probing or full-network finetuning. 






\begin{algorithm}
\DontPrintSemicolon
 \textbf{Input:} Set of $N$ ``seen'' downstream tasks represented by task2vecs $\mc{T} = \{\bs{x}_i | i \in [N]\}$. \\
 Given initial Task2Sim parameters $\theta_0$ and initial noise level $\epsilon_0$\\
 Initialize $a_{max}^{(i)} | i \in [N]$ the maximum reward action for each seen task \\
 \For{$t \in [T]$}{
 Set noise level $\epsilon = \frac{\epsilon_0}{t} $ \\
 Sample minibatch $\tau$ of size $n$ from $\mc{T}$  \\
 Get \ours output distributions $\pi^{(i)} | i \in [n]$ \\
 Sample outputs $a^{(i)} \sim \pi^{(i)} + \epsilon$ \\
 Get Rewards $R(a^{(i)})$ by generating a synthetic dataset with parameters $a^{(i)}$, pre-training a backbone on it, and getting the 5-NN downstream accuracy using this backbone \\
 Update $a_{max}^{(i)}$ if $R(a^{(i)}) > R(a_{max}^{(i)})$ \\
 Get point estimates of reward gradients $dr^{(i)}$ for each task in minibatch using \cref{eq:grad-pt-est} \\
 $\theta_{t,0} \leftarrow \theta_{t-1} + \frac{\sum_{i \in [n]} dr^{(i)}}{n}$ \label{ln:update} \\
 \For{$j \in [T_{si}]$}{ 
    \tcp{Self Imitation}
    Get reward gradient estimates $dr_{si}^{(i)}$ from \cref{eq:grad-pt-est} for $a \leftarrow a_{max}^{(i)}$ \\
    $\theta_{t, j}  \leftarrow \theta_{t, j-1} + \frac{\sum_{i \in [n]} dr_{si}^{(i)}}{n}$
 }
 $\theta_{t} \leftarrow \theta_{t, T_{si}}$
 }
 \textbf{Output}: Trained model with parameters $\theta_T$. 
 \caption{Training Task2Sim}
 \label{alg:train}  
\end{algorithm}

% obs-noise = 0.05, derivative-obs-noise = 0.2
\begin{tabular}{llll}
\toprule
            & HIP-GP & SVGP   & Exact GP \\
\midrule
RMSE        & 0.0192 & 0.0192 & 0.0192 \\
Uncertainty & 0.0198 & 0.0206 & 0.0198   \\
\bottomrule
\end{tabular}


\iffalse
% obs-noise = 0.05, derivative-obs-noise = 0.03
\begin{tabular}{llll}
\toprule
            & HIP-GP & SVGP   & Exact GP \\
\midrule
RMSE        & 0.0165 & 0.0165 & 0.0165   \\
Uncertainty & 0.0167 & 0.0175 & 0.0167   \\
\bottomrule
\end{tabular}


% obs-noise = 0.05, derivative-obs-noise = 0.1
\begin{tabular}{llll}
\toprule
            & HIP-GP & SVGP   & Exact GP \\
\midrule
RMSE        & 0.0173 & 0.0172 & 0.0173  \\
Uncertainty & 0.0181 & 0.0189 & 0.0181   \\
\bottomrule
\end{tabular}
\fi

% \vspace{-0.5em}
\section{Conclusion}
% \vspace{-0.5em}
Recent advances in multimodal single-cell technology have enabled the simultaneous profiling of the transcriptome alongside other cellular modalities, leading to an increase in the availability of multimodal single-cell data. In this paper, we present \method{}, a multimodal transformer model for single-cell surface protein abundance from gene expression measurements. We combined the data with prior biological interaction knowledge from the STRING database into a richly connected heterogeneous graph and leveraged the transformer architectures to learn an accurate mapping between gene expression and surface protein abundance. Remarkably, \method{} achieves superior and more stable performance than other baselines on both 2021 and 2022 NeurIPS single-cell datasets.

\noindent\textbf{Future Work.}
% Our work is an extension of the model we implemented in the NeurIPS 2022 competition. 
Our framework of multimodal transformers with the cross-modality heterogeneous graph goes far beyond the specific downstream task of modality prediction, and there are lots of potentials to be further explored. Our graph contains three types of nodes. While the cell embeddings are used for predictions, the remaining protein embeddings and gene embeddings may be further interpreted for other tasks. The similarities between proteins may show data-specific protein-protein relationships, while the attention matrix of the gene transformer may help to identify marker genes of each cell type. Additionally, we may achieve gene interaction prediction using the attention mechanism.
% under adequate regulations. 
% We expect \method{} to be capable of much more than just modality prediction. Note that currently, we fuse information from different transformers with message-passing GNNs. 
To extend more on transformers, a potential next step is implementing cross-attention cross-modalities. Ideally, all three types of nodes, namely genes, proteins, and cells, would be jointly modeled using a large transformer that includes specific regulations for each modality. 

% insight of protein and gene embedding (diff task)

% all in one transformer

% \noindent\textbf{Limitations and future work}
% Despite the noticeable performance improvement by utilizing transformers with the cross-modality heterogeneous graph, there are still bottlenecks in the current settings. To begin with, we noticed that the performance variations of all methods are consistently higher in the ``CITE'' dataset compared to the ``GEX2ADT'' dataset. We hypothesized that the increased variability in ``CITE'' was due to both less number of training samples (43k vs. 66k cells) and a significantly more number of testing samples used (28k vs. 1k cells). One straightforward solution to alleviate the high variation issue is to include more training samples, which is not always possible given the training data availability. Nevertheless, publicly available single-cell datasets have been accumulated over the past decades and are still being collected on an ever-increasing scale. Taking advantage of these large-scale atlases is the key to a more stable and well-performing model, as some of the intra-cell variations could be common across different datasets. For example, reference-based methods are commonly used to identify the cell identity of a single cell, or cell-type compositions of a mixture of cells. (other examples for pretrained, e.g., scbert)


%\noindent\textbf{Future work.}
% Our work is an extension of the model we implemented in the NeurIPS 2022 competition. Now our framework of multimodal transformers with the cross-modality heterogeneous graph goes far beyond the specific downstream task of modality prediction, and there are lots of potentials to be further explored. Our graph contains three types of nodes. while the cell embeddings are used for predictions, the remaining protein embeddings and gene embeddings may be further interpreted for other tasks. The similarities between proteins may show data-specific protein-protein relationships, while the attention matrix of the gene transformer may help to identify marker genes of each cell type. Additionally, we may achieve gene interaction prediction using the attention mechanism under adequate regulations. We expect \method{} to be capable of much more than just modality prediction. Note that currently, we fuse information from different transformers with message-passing GNNs. To extend more on transformers, a potential next step is implementing cross-attention cross-modalities. Ideally, all three types of nodes, namely genes, proteins, and cells, would be jointly modeled using a large transformer that includes specific regulations for each modality. The self-attention within each modality would reconstruct the prior interaction network, while the cross-attention between modalities would be supervised by the data observations. Then, The attention matrix will provide insights into all the internal interactions and cross-relationships. With the linearized transformer, this idea would be both practical and versatile.

% \begin{acks}
% This research is supported by the National Science Foundation (NSF) and Johnson \& Johnson.
% \end{acks}


%%%%%%%%% REFERENCES
{\small
\bibliographystyle{ieee_fullname}
\bibliography{vision}
}

\end{document}


\title{Improved algebraic fibrings}
\author{Sam P.~Fisher}
\email{sam.fisher@maths.ox.ac.uk}
\address{University of Oxford, United Kingdom}

\begin{document}
\maketitle

\begin{abstract}
We show that a virtually RFRS group $G$ of type $\mathtt{FP}_n(\Q)$ virtually algebraically fibres with kernel of type $\mathtt{FP}_n(\Q)$ if and only if the first $n$ $\ell^2$-Betti  numbers of $G$ vanish, that is, $b_p^{(2)}(G) = 0$ for $0 \leqslant p \leqslant n$. This confirms a conjecture of Kielak. We also offer a variant of this result over other fields, in particular in positive characteristic.

As an application of the main result, we show that virtually amenable RFRS groups of type $\mathtt{FP}(\Q)$ are polycyclic-by-finite. It then follows that if $G$ is a virtually RFRS group of type $\mathtt{FP}(\Q)$ such that $\Z G$ is Noetherian, then $G$ is polycyclic-by-finite. This answers a longstanding conjecture of Baer for virtually RFRS groups of type $\mathtt{FP}(\Q)$.
\end{abstract}

%%%                %%%
%%% Introduction 2 %%%
%%%                %%%

\section{Introduction}

A group $G$ is \textit{algebraically fibred} (or simply \textit{fibred}) if it admits a homomorphism onto $\Z$ with a finitely generated kernel. The interest in algebraic fibrings arose from the study of $3$-manifolds fibring over the circle. Using the long exact sequence of homotopy groups associated to a fibration, one sees that a surface bundle $M \longrightarrow S^1$, where $M$ is a compact $3$-manifold, induces an algebraic fibration $\pi_1(M) \longrightarrow \Z$. A celebrated theorem of Stallings \cite{Stallings3mflds} establishes the converse: if $G$ is isomorphic to the fundamental group of a closed, compact $3$-manifold $M$, then an algebraic fibration $G \longrightarrow \Z$ is induced by a surface bundle $M \longrightarrow S^1$.

In 2020, Kielak characterised the virtual algebraic fibring of residually finite rationally solvable (RFRS) groups in terms of the vanishing of the first $\ell^2$-Betti number. More precisely, he showed the following.

\begin{thm}[Kielak \cite{KielakRFRS}] \label{thm:kielak}
Let $G$ be an infinite, finitely generated virtually RFRS group. Then $G$ virtually algebraically fibres if and only if $b_1^{(2)}(G) = 0$.
\end{thm}

It should be noted that virtually RFRS groups are not hard to find ``in the wild," for example subgroups of right-angled Artin groups and right-angled Coxeter groups are virtually RFRS and, in particular, special groups (in the sense of Wise) are RFRS. Moreover, the RFRS property passes to subgroups and is preserved by taking products and free products of RFRS groups. Kielak's theorem is an algebraic analogue of the following theorem of Agol, which was a key step in confirming Thurston's Virtually Fibred Conjecture. 

\begin{thm}[Agol \cite{AgolCritVirtFib}]
Every compact, irreducible, orientable $3$-manifold $M$ with $\chi(M) = 0$ and nontrivial RFRS fundamental group admits a finite covering that fibres over the circle.
\end{thm}

Since algebraic fibrings induce topological fibrings of $3$-manifolds, Kielak's theorem generalises the above theorem of Agol by removing the assumption that $G$ is the fundamental group of a $3$-manifold. Note that \cite[Theorem 4.1]{Luck02} states that if $M$ is a compact, irreducible $3$-manifold with no $S^2$ boundary components, then $b_1^{(2)}(\pi_1(M)) = -\chi(M)$. Thus, we interpret the condition $b_1^{(2)}(G) = 0$ in Kielak's theorem as the algebraic analogue of the condition $\chi(M) = 0$ in Agol's theorem.

In light of Kielak's theorem, it is natural to ask whether the vanishing of higher $\ell^2$-Betti numbers of a group $G$ controls the higher finiteness properties of the kernel of the virtual fibration. Indeed, Kielak conjectured that a virtually RFRS group of type $\mathtt{FP}_n(\mathbb{Q})$ virtually algebraically fibres with kernel of type $\mathtt{FP}_n(\Q)$ if and only if $b_p^{(2)}(G)$ vanishes for all $p \leqslant n$ \cite[Conjecture 8]{KielakOber20}. The main result of this paper confirms Kielak's conjecture, and gives another characterisation of RFRS groups virtually fibring with kernel of type $\FP_n(\Q)$. We also note that the equivalence of (2) and (3) in the following theorem generalises \cite[Corollary 1.5]{JaikinZapirain2020THEUO}, where Jaikin-Zapirain proves the $n = 1$ case.

\begin{manualtheorem}{A}\label{thm:A}
Let $G$ be a virtually RFRS group of type $\mathtt{FP}_n(\Q)$. Then the following are equivalent:
    \begin{enumerate}[label = (\arabic*)]
        \item\label{item:b2vanish} $b_p^{(2)}(G) = 0$ for all $p \leqslant n$;
        \item\label{item:FPn} there is a finite-index subgroup $H \leqslant G$ and an surjection $\varphi \colon H \longrightarrow \Z$ such that $\ker \varphi$ is of type $\FP_n(\Q)$;
        \item\label{item:finiteBetti} there is a finite-index subgroup $H' \leqslant G$ and an surjection $\varphi' \colon H' \longrightarrow \Z$ such that $b_p(\ker \varphi') < \infty$ for all $p \leqslant n$.
    \end{enumerate}
\end{manualtheorem}

It should be emphasized that if $\ker \varphi'$ has finite Betti numbers in dimensions $\leqslant n$, it is not necessarily the case that $\ker \varphi'$ is of type $\FP_n(\Q)$. We prove a more general theorem that treats algebraic fibring with kernels of type $\FP_n(\mathbb F)$ for any skew-field $\mathbb F$, from which we obtain \cref{thm:A} as a special case.  Before stating the result, we give some background. If $\mathbb{F}$ is a skew-field and $G$ is a locally indicable group, then under certain conditions the group ring $\mathbb{F}G$ embeds into a skew-field $\mathcal{D}_{\mathbb{F}G}$ called the \textit{Hughes-free division ring} of $\mathbb{F}G$ (see \cref{def:HfreeDiv}). If it exists, $\mathcal{D}_{\mathbb{F}G}$ is unique up to $\mathbb{F}G$-algebra isomorphism \cite{HughesDivRings1970} and the \textit{$\mathcal{D}_{\mathbb{F}G}$-homology} of $G$ in dimension $p$ is defined to be $H_p^{\DF{G}}(G) := H_p(G; \DF{G})$. The $p$th \textit{$\mathcal{D}_{\mathbb{F}G}$-Betti number} is defined to be
\[
    b_p^{\mathcal{D}_{\mathbb{F}G}}(G) := \dim_{\mathcal{D}_{\mathbb{F}G}} H_p^{\mathcal{D}_{\mathbb{F}G}} (G).
\]
In \cite[Corollary 1.3]{JaikinZapirain2020THEUO}, Jaikin-Zapirain proves that if $G$ is finitely generated and RFRS, then $\mathcal{D}_{\mathbb{F}G}$ exists for any skew-field $\mathbb{F}$ and, if $\mathbb{F} = \mathbb{Q}$, it is isomorphic to the Linnell-skew field of $G$. For the purposes of this paper, it will not be necessary to know how the Linnell skew-field is defined, but that it can be used to define the $\ell^2$-homology and $\ell^2$-Betti numbers of a group $G$ (see \cref{def:l2b}). Importantly for us, when $G$ is finitely generated and RFRS, we have $b_p^{\mathcal{D}_{\Q G}} (G) = b_p^{(2)}(G)$ for all $p$. We prove the following theorem, which reduces to \cref{thm:A} in the case $\mathbb F = \Q$.

\begin{manualtheorem}{B}\label{thm:B}
    Let $\mathbb{F}$ be a skew-field and let $G$ be a virtually RFRS group of type $\FP_n(\mathbb F)$. Then the following are equivalent:
    \begin{enumerate}
        \item $b_p^{\DF{G}} = 0$ for all $p \leqslant n$;
        \item there is a finite-index subgroup $H \leqslant G$ and an surjection $\varphi \colon H \longrightarrow \Z$ such that $\ker \varphi$ is of type $\FP_n(\mathbb F)$;
        \item there is a finite index subgroup $H' \leqslant G$ and an surjection $\varphi' \colon H' \longrightarrow \Z$ such that $b_p(\ker \varphi'; \mathbb F) < \infty$ for all $p \leqslant n$.
    \end{enumerate}
\end{manualtheorem}

We highlight the following corollary, which implies, in particular that if $\mathbb F$ and $\mathbb F'$ are skew-fields of the same characteristic, then a RFRS group $G$ is $\DF{G}$-acyclic in dimension $\leqslant n$ if only if it is $\mathcal D_{\mathbb F'G}$-acyclic in dimensions $\leqslant n$. Moreover, it also implies that if $G$ is $\mathcal D_{\mathbb F_p G}$ acyclic in dimensions $\leqslant n$ for some prime $p$, then it is also $\ell^2$-acyclic in dimensions $\leqslant n$.

\begin{manualcor}{C}[\cref{cor:charac}]
    Let $G$ be a virtually RFRS group and let $n \in \N$.
    \begin{enumerate}
        \item If $\mathbb F$ and $\mathbb F'$ are skew-fields of the same characteristic, then $G$ virtually algebraically fibres with kernel of type $\FP_n(\mathbb F)$ if and only if it virtually algebraically fibres with kernel of type $\FP_n(\mathbb F')$.
        \item If $p$ is a prime such that $G$ virtually algebraically fibres with kernel of type $\FP_n(\mathbb F_p)$, then it virtually fibres with kernel of type $\FP_n(\Q)$.
    \end{enumerate}
\end{manualcor}

The final section of the paper is devoted to some applications of \cref{thm:A}. An interesting question is to determine general conditions under which amenable groups are elementary amenable. There are many examples of amenable groups that are not elementary amenable, for instance, Grigorchuk's group of intermediate growth, but it is not known whether there are examples of amenable groups of finite cohomological dimension that are not elementary amenable. Moreover, elementary amenable groups of finite cohomological dimension are virtually solvable by \cite[Lemma 2]{Hillman91} and \cite[Corollary 1]{HillmanLinnell}. This leads us to the following question, which first appeared in the work of Degrijse.

\begin{q}[Degrijse \cite{degrijse2016amenable}]
Are amenable groups of finite cohomological dimension over $\Z$ virtually solvable?
\end{q}

We obtain the following as an application of \cref{thm:A}, which provides evidence of a positive answer for virtually RFRS groups.

\begin{manualtheorem}{D}[\cref{thm:amRFRSelemAm}]\label{thm:C}
\sloppy If $G$ is a virtually amenable RFRS group of type $\mathtt{FP}(\Q)$, then $G$ is polycyclic-by-finite, and in particular virtually solvable.
\end{manualtheorem}

We can also confirm the following longstanding conjecture of Baer for virtually RFRS groups of type $\mathtt{FP}(\Q)$.

\begin{conj}
If $G$ is a group such that the group ring $\Z G$ is Noetherian, then $G$ is polycyclic-by-finite.
\end{conj}

Hall showed that polycyclic-by-finite groups have Noetherian group rings \cite[Theorem 4]{Hall59}, but it is still unknown whether the converse holds. Some progress was made recently by P.~Kropholler and Lorensen, who showed that if $RG$ is right Noetherian and $R$ is a domain, then $G$ is amenable and all of its subgroups are finitely generated \cite[Corollary B]{KrophollerLorensen19}. This result provides evidence for the conjecture, as the only known amenable groups in which every subgroup is finitely generated are polycyclic-by-finite. We obtain the following as a consequence of \cref{thm:C} and Kielak's appendix to \cite{BartholdiKielakApp}.

\begin{manualcor}{E}[\cref{cor:baer}]
    Let $G$ be a virtually RFRS group of type $\mathtt{FP}(\Q)$ such that $\Z G$ is Noetherian. Then $G$ is polycyclic-by-finite.
\end{manualcor}

Finally, we mention that Llosa-Isenrich, Martelli, and Py remarked in  \cite{isenrich2021hyperbolic} that an easy consequence of \cref{thm:A}, Agol's Theorem \cite{AgolHaken}, and work of Agol and Bergeron--Haglund--Wise \cite{BHW_2011} is the existence of hyperbolic groups containing type $\FP_{n-1}(\Q)$ subgroups that are not of type $\FP_n(\Q)$ for all $n \in \N$.

\begin{prop}[{\cite[Proposition 19]{isenrich2021hyperbolic}}]\label{prop:LMP}
    Let $\Gamma \in \PO(2n,1)$ be a hyperbolic arithmetic lattice of the simplest type. Then $\Gamma$ virtually fibres with kernel of type $\FP_{n-1}(\Q)$ but not of type $\FP_n(\Q)$.
\end{prop}

Note that if $G$ is virtually RFRS, of type $\FP(\Q)$, and $\ell^2$-acyclic, then \cref{thm:A} implies that $G$ fibres with kernel of type $\FP(\Q)$. Together with this easy observation, the same argument given by Llosa-Isenrich--Martelli--Py also shows that all such simplest type arithmetic lattices in $\PO(2n+1,1)$ virtually fibre with kernel of type $\FP(\Q)$. In a subsequent paper \cite{IsenrichPy2022}, Llosa-Isenrich and Py showed that there are hyperbolic arithmetic lattices of the simplest type in $\PU(n,1)$ that virtually fibre with kernel of type $\F_{n-1}$ but not of type $\FP_n(\Q)$, answering a question of Brady about the existence of subgroups of hyperbolic groups with exotic finiteness properties.





\subsection*{Structure of the paper} 

In \cref{sec:prelims} we introduce some of the main tools and objects that will be used throughout the paper. In particular, we define finiteness properties of groups, Hughes-free division rings, RFRS groups, and Ore localisation.

\cref{sec:valuations} recalls what we will need from the theory of valuations on free resolutions developed by Bieri and Renz in \cite{BieriRenzValutations}. The results in \cite{BieriRenzValutations} are stated for free resolutions over group rings with coefficients in $\Z$, however we will need them for coefficients in an arbitrary associative, unital ring $R$. There is no essential dependence on the ring, so our proofs are similar to those of Bieri and Renz after replacing $\Z$ with $R$.

In \cref{sec:horo}, we introduce the complex of horochains associated to a free resolution equipped with a valuation.

\cref{sec:SigInv} begins with the definition of the higher $\Sigma$-invariants $\Sigma_R^n(G; M)$ for a group $G$, a unital, associative ring $R$, and an $RG$-module $M$. Again, these were introduced in the case $R = \Z$ in \cite{BieriRenzValutations}, and reduce to the usual Bieri--Neumann--Strebel invariant when $n = 1$, $R = \Z$, and $M = \Z$ is the trivial $\Z G$-module. The rest of the section is devoted to the proof of \cref{thm:Main}, which gives equivalent characterisations of the invariants $\Sigma_R^n(G; M)$. The most important characterisation for our purposes is the following: $[\chi] \in \Sigma_R^n(G;M)$ if and only if $\Tor_i^{RG}(M, \widehat{RG}^\chi) = 0$ for all $i \leqslant n$, where $\widehat{RG}^\chi$ is the Novikov ring. When $n = 1$, this result is Sikorav's theorem \cite{SikoravThese} and is a key ingredient in Kielak's proof of \cref{thm:kielak}. The proof follows arguments given in \cite{BieriRenzValutations} and Schweitzer's appendix to \cite{BieriDeficiency}. \cref{thm:Main} is the main technical result that will be used in the proof of \cref{thm:agrarianMain}. 

In \cref{sec:homology} we introduce $\mathcal{D}_{\mathbb{F}G}$-homology and prove properties of $\mathcal{D}_{\mathbb{F}G}$-Betti numbers analogous to those of $\ell^2$-Betti numbers. Namely, we prove that 
\[
    [G : H] \cdot b_p^{\mathcal{D}_{\mathbb{F}G}}(G) = b_p^{\mathcal{D}_{\mathbb{F}H}}(H)
\]
(\cref{lem:JBscales}) whenever $\mathcal{D}_{\mathbb{F}G}$ exists and $H$ is a finite index subgroup of $G$. In \cref{thm:JBSES}, we show that if $b_p^{\mathcal{D}_{\mathbb{F}K}}(K) < \infty$ and $G$ fits into a short exact sequence $1 \longrightarrow K \longrightarrow G \longrightarrow \Z \longrightarrow 1$, then $b_p^{\mathcal{D}_{\mathbb{F}G}}(G) = 0$. This should be thought of as an analogue of \cite[Theorem 7.2]{Luck02} for $\mathcal{D}_{\mathbb{F}G}$-Betti numbers. We then prove the main result, \cref{thm:agrarianMain}, and obtain \cref{thm:b2rfrs} as a special case. In \cref{thm:finiteBetti}, we show that virtually fibring with kernel of type $\FP_n(\Q)$ is equivalent to having a virtual map to $\Z$ whose kernel has finite Betti numbers in dimensions $\leqslant n$.

We conclude with the proofs of \cref{thm:amRFRSelemAm} and \cref{cor:baer} in \cref{sec:app}, and mention related work of Llosa Isenrich, Martelli, and Py.

\subsection*{Acknowledgements.} The author would like to thank Dawid Kielak for numerous helpful conversations and comments on this paper, Peter Kropholler for a helpful correspondence, and Sam Hughes for pointing out the application of \cref{thm:amRFRSelemAm} to \cref{cor:baer}.

This work has received funding from the European Research Council (ERC) under the European Union's Horizon 2020 research and innovation programme (Grant agreement No. 850930).









%%%               %%%
%%% Preliminaries %%%
%%%               %%%
\section{Preliminaries} \label{sec:prelims}

\begin{rem*}
In the sequel, all rings will be associative and unital with $1 \neq 0$.
\end{rem*}

%%% Finiteness properties
\subsection{Finiteness properties}

\begin{defn}[type $\mathtt{FP}_n$]
Let $R$ be a ring and $M$ be a left $R$-module. We say that $M$ is of \textit{type $\mathtt{FP}_n$}, and write $M \in \mathtt{FP}_n$, if $M$ has a projective resolution
\[
\cdots \longrightarrow P_{n+1} \longrightarrow P_n \longrightarrow \cdots \longrightarrow P_1 \longrightarrow P_0 \longrightarrow M \longrightarrow 0
\]
by left $R$-modules, where $P_j$ is finitely generated for $j \leqslant n$. If we want to specify the ring, we say that $M$ is of \textit{type $\mathtt{FP}_n$ over $R$} and write $M \in \mathtt{FP}_n(R)$.  If $P_j = 0$ for $j > n$, then we write $M \in \mathtt{FP}(R)$.

A group $G$ is of \textit{type $\mathtt{FP}_n$ over $R$} if the trivial $RG$-module $R$ is of type $\mathtt{FP}_n(RG)$; in this case we write $G \in \mathtt{FP}_n(R)$. Similarly, if the trivial $RG$-module $R$ is of type $\mathtt{FP}(RG)$, then we write $G \in \mathtt{FP}(R)$.
\end{defn}

We will often use the fact that an $R$-module $M$ is of type $\mathtt{FP}_n$ if and only if there is a free resolution $\cdots \longrightarrow F_{n+1} \longrightarrow F_n \longrightarrow \cdots \longrightarrow F_0 \longrightarrow M \longrightarrow 0$ with $F_j$ finitely generated for $j \leqslant n$ \cite[VIII Proposition 4.3]{BrownGroupCohomology}. Note that the analogous fact does not hold for type $\mathtt{FP}$.

The following definition will not be needed until \cref{sec:app}.

\begin{defn}[cohomological dimension]
A resolution $\cdots \longrightarrow P_1 \longrightarrow P_0 \longrightarrow M \longrightarrow 0$ of an $R$-module $M$ has \textit{length} $n$ if $P_n \neq 0$ and $P_m = 0$ for $m > n$. A group $G$ has \textit{cohomological dimension $n$ over $R$} if $n$ is the shortest length of any projective resolution of the trivial $RG$-module $R$. In this case we will write $\cd_R (G) = n$. If $R$ has no finite-length projective resolution, then $\cd_R(G) = \infty$.
\end{defn}

Note that $G \in \mathtt{FP}(R)$ implies that $G$ has finite cohomological dimension over $R$, but not conversely.



%%% Hughes free division rings
\subsection{Hughes-free division rings}

We are following Jaikin-Zapirain's exposition of this material; in particular, \cref{def:HfreeDiv,def:crossedprod} are taken from \cite{JaikinZapirain2020THEUO}.

\begin{defn}[Hughes-free division rings] \label{def:HfreeDiv}
Let $\mathbb{F}$ and $\mathcal{D}$ be skew-fields, let $G$ be a locally indicable group, and let $\varphi \colon \mathbb{F} G \longrightarrow \mathcal{D}$ be a ring homomorphism. Then the pair $(\mathcal{D}, \varphi)$ is \textit{Hughes-free} if
\begin{enumerate}[label=(\arabic*)]
    \item\label{item:epic} $\mathcal{D}$ is the skew-field generated by $\varphi(\mathbb{F} G)$, i.e., there are no intermediate skew-fields between $\varphi(\mathbb{F}G)$ and $\mathcal{D}$;
    \item\label{item:direct} for every nontrivial finitely generated subgroup $H \leqslant G$, every normal subgroup $N \triangleleft H$ such that $H/N \cong \Z$, and every set of elements $h_1, \dots, h_n \in H$  lying in pairwise distinct cosets of $N$, the sum
    \[
    \langle\varphi(\mathbb{F}N)\rangle \cdot \varphi(h_1) + \cdots + \langle\varphi(\mathbb{F}N)\rangle \cdot \varphi(h_n)
    \]
    is direct, where $\langle\varphi(\mathbb{F}N)\rangle$ is the sub-skew-field of $\mathcal{D}$ generated by $\varphi(\mathbb{F}N)$.
\end{enumerate}
\end{defn}

Hughes showed that if such a pair $(\mathcal{D},\varphi)$ exists, then $\mathcal{D}$ is unique up to $\mathbb{F}G$-algebra isomorphism \cite{HughesDivRings1970}. Thus, we denote $\mathcal{D}$ by $\mathcal{D}_{\mathbb{F}G}$. Let $H \leqslant G$ be a subgroup and suppose that $\mathcal{D}_{\mathbb{F}G}$ exists. Then $\mathcal{D}_{\mathbb{F}H}$ exists as well and is equal to $\langle \varphi(\mathbb{F}H) \rangle \subseteq \mathcal{D}_{\mathbb{F}G}$. Hence, we will view $\mathcal{D}_{\mathbb{F}H}$ as a subset of $\mathcal{D}_{\mathbb{F}G}$ whenever $H$ is a subgroup of $G$. Moreover, Gr\"ater showed that $\mathcal{D}_{\mathbb{F}G}$ is \textit{strongly Hughes-free} whenever it exists \cite[Corollary 8.3]{Grater20}, which is to say that condition (2) above can be replaced with the following:
\begin{enumerate}[label = {($2^\prime$)}]
    \item\label{item:2prime} for every nontrivial subgroup $H \leqslant G$, every normal subgroup $N \triangleleft H$, and every set of elements $h_1, \dots, h_n \in H$  lying in pairwise distinct cosets of $N$, the sum
    \[
    \langle\varphi(\mathbb{F}N)\rangle \cdot \varphi(h_1) + \cdots + \langle\varphi(\mathbb{F}N)\rangle \cdot \varphi(h_n)
    \]
    is direct.
\end{enumerate}


\begin{defn}[Crossed products] \label{def:crossedprod}
A ring $R$ is \textit{$G$-graded} if its underlying Abelian group is isomorphic to a direct sum $\bigoplus_{g \in G} R_g$ of Abelian groups $R_g$ and $R_g R_h \subseteq R_{gh}$ for all $g,h \in G$. If $R_g$ contains a unit for each $g \in G$, then we say that $R$ is a \textit{crossed product} of $R_e$ and $G$, and we write $R = R_e * G$.
\end{defn}


Strong Hughes-freeness implies the following useful properties, which are stated in \cite{JaikinZapirain2020THEUO}.

\begin{prop}\label{prop:twistedNormalSkew}
Let $G$ be a locally indicable group such that $\mathcal{D}_{\mathbb{F}G}$ exists, let $\varphi$ be as in \cref{def:HfreeDiv}, and $N \triangleleft G$ be a normal subgroup. If $R$ is the subring of $\mathcal{D}_{\mathbb{F}G}$ generated by $\mathcal{D}_{\mathbb{F}N}$ and $\varphi(\mathbb{F}G)$, then
\begin{enumerate}[label = (\arabic*)]
    \item\label{item:subring} $R \cong \mathcal{D}_{\mathbb{F}N} * (G/N)$;
    \item\label{item:twistedFiniteIndex} if $[G:N] < \infty$, then $\mathcal{D}_{\mathbb{F}G} = R$.
\end{enumerate}
\end{prop}

\begin{proof}

Starting with \ref{item:subring}, let $\{t_i\}_{i \in I}$ be a transversal for $N$ in $G$. We claim that every element of $R$ can be written as a finite sum $\sum_i \alpha_i \varphi(t_i)$, where $\alpha_i \in \mathcal{D}_{\mathbb{F}N}$ for each $i \in I$. Since every element of $G$ is of the  form $n t_i$ for some $n \in N$ and $i \in I$, it is enough to show that $\varphi(t_i) \alpha$ can be written in the desired form for any $\alpha \in \mathcal{D}_{\mathbb{F}N}$. Moreover, $\mathcal{D}_{\mathbb{F}N}$ is the division closure of $\varphi(\mathbb{F}N)$, so it is enough to prove the cases where $\alpha = \varphi(a)$ or $\alpha = \varphi(a)\inv$ for some $a \in \mathbb{F}N$.

Let $a = \sum_j f_j n_j$, where $f_j \in \mathbb{F}$ and $n_j \in N$ for each $j$. Then
\[
    t_i \cdot a = t_i \cdot \sum_j f_j n_j = \sum_j f_j t_i n_j = \sum_j f_j n_j^{t_i} \cdot t_i,
\]
for each $i$, where $g^t = t g t\inv$ for any group elements $g, t \in G$. Since $N$ is normal in $G$, this shows that $\varphi(t_i) \varphi(a)$ can be written in the desired form. 

For the other case, we have $\varphi(t_i) \varphi(a)\inv = \varphi(a t_i\inv)\inv$. Then
\[
    a \cdot t_i\inv = \sum_j f_j n_j t_i\inv = t_i\inv \cdot \sum_j f_j n_j^{t_i}
\]
and therefore we obtain $\varphi(t_i) \varphi(a)\inv = \varphi(a \cdot t_i\inv)\inv = \varphi(\sum_j f_j n_j^{t_i})\inv \varphi(t_i)$. This proves the claim.

By strong Hughes-freeness, every element of $R$ has a unique decomposition of the form $\sum_i \alpha_i \varphi(t_i)$, with $\alpha_i \in \mathcal{D}_{\mathbb{F}N}$ for each $i$, and therefore we obtain an isomorphism $R \cong \bigoplus_{i = 1}^n \mathcal{D}_{\mathbb{F}N} \varphi(t_i)$ of Abelian groups. Moreover, $R$ is $G/N$-graded with respect to this decomposition, so $R$ is a crossed product $\mathcal{D}_{\mathbb{F}N} * (G/N)$. 

\smallskip

To prove \ref{item:twistedFiniteIndex}, we recall the fact that a finite algebra over a skew-field with no zero-divisors is a skew-field. Hence, if $[G:N] < \infty$, then $R \cong \mathcal{D}_{\mathbb{F}N} * (G/N)$ is finitely finite over $\mathcal{D}_{\mathbb{F}N}$ and is therefore a skew-field. But $\mathcal{D}_{\mathbb{F}G}$ is the smallest skew-field containing $\varphi(\mathbb{F}G)$, so $\mathcal{D}_{\mathbb{F}G} = R$. \qedhere
\end{proof}






%%% RFRS groups

\subsection{RFRS groups}

Residually finite rationally solvable (RFRS) groups were defined by Agol in \cite{AgolCritVirtFib} in order to show that certain hyperbolic $3$-manifolds virtually fibre over the circle. Let $G$ be a group and let $G^\mathsf{ab} := G/[G,G]$ be its Abelianisation. Since $G^\mathsf{ab}$ is Abelian, it is canonically a $\Z$-module so we can form the tensor product $\Q \otimes_\Z G^\mathsf{ab}$, and there is a group homomorphism $G \longrightarrow \Q \otimes_\Z G^{\mathsf{ab}}$ sending $g \in G$ to $1 \otimes g[G,G]$.
\begin{defn} \label{def:RFRS}
A group $G$ is \textit{RFRS} if 
\begin{enumerate}[label = (\arabic*)]
    \item there is a chain $G = G_0 \geqslant G_1 \geqslant G_2 \geqslant \cdots$ of finite index normal subgroups of $G$ such that $\bigcap_{i=0}^\infty G_i = \{1\}$
    \item $\ker(G_i \longrightarrow \Q \otimes_\Z G_i^{\mathsf{ab}}) \leqslant G_{i+1}$ for every $i \geqslant 0$.
\end{enumerate}
\end{defn}

The following fact that will be used in the proof of \cref{thm:agrarianMain}.

\begin{prop}\label{prop:RFRStoQ}
If $G$ is a nontrivial RFRS group, then $H^1(G;\R) \neq 0$.
\end{prop}
\begin{proof}
Let $G = G_0 \geqslant G_1 \geqslant G_2 \geqslant \cdots$ be a residual chain of finite index subgroups as in \cref{def:RFRS} and let $V = \Q \otimes_\Z G^\mathsf{ab}$. Moreover, assume that $G \neq G_1$ so that the map $G \longrightarrow V$ cannot be trivial. Let $x \in V$ be a nontrivial element in the image of $G \longrightarrow V$; since $V$ is a $\Q$-vector space, there is a linear map $\varphi \colon V \longrightarrow \Q$ such that $\varphi(x) = 1$. Then the composition 
\[
    G \longrightarrow V \xrightarrow{\ \varphi \ } \Q \longhookrightarrow \R
\]
is a nontrivial element of $H^1(G;\R)$.
\end{proof}



%%% Ore localisation

\subsection{Ore localisation}

Ore localisation is an analogue of usual localisation for noncommutative rings. Let $R$ be a ring and let $S$ be its set of non-zerodivisors. Then $R$ satisfies the \textit{Ore condition} if for every $r \in R$ and $s \in S$ there  are elements $p,p' \in R$ and $q,q' \in S$ such that 
\[
    qr = ps \ \textnormal{and} \ rq' = sq'.
\]
If  $S = R \setminus \{0\}$ and $R$ satisfies the Ore condition, then it is called an \textit{Ore domain}, and we can form its \textit{Ore localisation} $\Ore(R)$ as follows. Define an equivalence relation $\sim_R$ on $R \times S$ by declaring that $(r,s) \sim_R (r',s')$ if and only if there are elements $a,b \in S$ such that
\[
    ra = r'b \ \textnormal{and} \ sa = s'b.
\]
The equivalence class of $(r,s)$ under $\sim_R$ is denoted $r/s$ and called a \textit{right fraction}. Then $\Ore(R)$ is defined to be the set of right fractions. We can similarly define an equivalence relation $\sim_L$ and define $\Ore(R)$ as a set of left fractions. The Ore condition ensures that these two constructions are isomorphic and indicates how to convert right fractions into left fractions and vice versa. For a detailed construction of $\Ore(R)$ and the definition of addition and multiplication making this set into a ring, we refer the reader to Section 4.4 of Passman's book \cite{PassmanGrpRng}.

The facts about Ore localisation that we will use are summarised in the following proposition.

\begin{prop}\label{prop:OreLoc}
Let $R$ be an Ore domain. Then
\begin{enumerate}[label = (\arabic*)]
    \item $\Ore(R)$ is a skew-field;
    \item for every $r \in R$, we have $r / 1 = 1 \backslash r$ and the map $R \longrightarrow \Ore(R), r \longmapsto r/1 = 1 \backslash r$ is an injective ring homomorphism;
    \item \cite[Proposition 2.2(2)]{JaikinZapirain2020THEUO} if $G$ is a group and $\mathbb{F}$ is a skew-field such that $\mathcal{D}_{\mathbb{F}G}$ exists and $K$ is a normal subgroup such that $G / K \cong \Z$, then  $\mathcal{D}_{\mathbb{F}G} \cong \Ore(\mathcal{D}_{\mathbb{F}K} * (G/K))$.
\end{enumerate}
\end{prop}













%%%            %%%
%%% VALUATIONS %%%
%%%            %%%
\section{Valuations on free resolutions} \label{sec:valuations}

In this section, we introduce valuations on free resolutions over a group ring. We will be very closely following Bieri and Renz \cite{BieriRenzValutations} where the theory is developed in the case where the ring is $\Z$. Their proofs go through without change when $
\Z$ is replaced by an arbitrary ring $R$.

Let $R$ be a ring, $G$ a group, and $M$ a left $RG$-module. A \textit{free resolution} of $M$ is an exact sequence
\[
    \cdots \xrightarrow{\partial_{n+1}} F_n \xrightarrow{\partial_n} F_{n-1} \xrightarrow{\partial_{n-1}} \cdots \xrightarrow{\partial_1} F_0 \xrightarrow{\partial_0}M \longrightarrow 0
\]
of left $RG$-modules, where $F_i$ is free for all $i \geqslant 0$. We will usually omit the subscripts on the boundary maps $\partial_n$ and denote the free resolution by $F_\bullet \longrightarrow M \longrightarrow 0$. Let $F$ be the free $RG$-module $\bigoplus_{i = 0}^\infty F_i$, and define the \textit{$n$-skeleton} of $F$ to be $F^{(n)} := \bigoplus_{i = 0}^n F_i$. The elements of $F$ are called \textit{chains}, so a chain is not necessarily an element of $F_i$ for any $i$ in our context. Fixing a basis $X_i$ for each $F_i$, we note that $X := \bigcup_{i=0}^\infty X_i$ is a basis for $F$ and $X^{(n)} := \bigcup_{i = 0}^n X_i$ is a basis for $F^{(n)}$. The resolution $F_\bullet \longrightarrow M \longrightarrow 0$ is \textit{admissible with respect to $X$} if $\partial x \neq 0$ for every $x \in X$. We will always assume that our free resolutions are admissible with respect to the basis we are working with. This is not a strong requirement, since if all boundary maps are nonzero, then $F$ has a  basis with respect to which $F_\bullet \longrightarrow M \longrightarrow 0$ is admissible; otherwise we can truncate the resolution and choose a basis to obtain an admissible resolution of finite length. We also define the \textit{support} of a chain $c \in F$ (with respect to $X$), denoted $\supp_X(c)$, as follows: every chain $c \in F$ can be written uniquely as $\sum_{g\in G, x \in X} r_{g,x} gx$, where $r_{g,x} \in R$. Then $\supp_X (c) := \{ gx : r_{g,x} \neq 0 \}$; we will usually drop the subscript $X$ when the basis is understood.

Let $\chi \colon G \longrightarrow \R$ be a nontrivial \textit{character}, that is, a nonzero group homomorphism from $G$ to the additive group $\R$. This provides the elements of $G$ with a notion of height, which we now extend to the chains of $F$. Let $\R_\infty = \R \cup \{\infty\}$, where $\infty$ is an element such that $t < \infty$ for every $t \in \R$.  We construct a function $v_X \colon F \longrightarrow \R_{\infty}$ via the following inductive procedure. For an element $c \in F_0$, define $v_X(c) = \inf\{ \chi(g) : gx \in \supp(y) \}$. Let $n > 0$ and assume that we have defined $v_X$ on $F_{n-1}$. For $x \in X_n$, let $v_X(x) := v_X(\partial x)$. For $c \in F_n$, set $v_X(c) = \inf\{ \chi(g) + v_X(x) : gx \in \supp(y) \}$. For an arbitrary $c \in F$, write $c = \sum_i c_i$, where $c_i \in F_i$, and define $c = \inf_i\{ v_X(c_i) \}$. The function $v_X$ is called the \textit{valuation extending $\chi$ with respect to $X$}. It is clear from the definition that $v_X(c) = \inf\{ \chi(g) + v_X(x) : gx \in \supp(c) \}$ for any chain $c \in F$. Again, we will usually drop the $X$ in the subscript when the basis is understood.


%%%     PROPERTIES OF v     %%%
\begin{prop} \label{PropPropsOfVal}
For the valuation $v_X = v \colon F \longrightarrow \R_\infty$ defined above and for any $c, c' \in F$ and $g \in G$, we have
\begin{enumerate}[label=(\arabic*)]
    \item\label{item:sum} $v(c + c') \geqslant \min\{ v(c), v(c') \}$;
    \item\label{item:scalar} $v(c) \leqslant v(rc)$ for all $r \in R$, and $v(c) = v(rc)$ if $r$ is not a zerodivisor;
    \item\label{item:sumequal} if $v(c) \neq v(c')$, then $v(c + c') = \min\{ v(c), v(c') \}$;
    \item\label{item:groupelement} $v(g c) = \chi(g) + v(c)$;
    \item\label{item:infheight} $v(c) = \infty$ if and only if $c = 0$;
    \item\label{item:bdyincrease} if $c \in \bigoplus_{i \geqslant 1} F_i$, then  $v(\partial c) \geqslant v(c)$.
\end{enumerate}
\end{prop}

\begin{proof}
\ref{item:sum} follows from the fact that $\supp(c+c') \subseteq \supp(c) \cup \supp(c')$. The first part of \ref{item:scalar} follows from the fact that $\supp(c) \supseteq \supp(rc)$. If $r$ is not a zerodivisor then $\supp(c) = \supp(rc)$, which yields the second statement of \ref{item:scalar}.

To prove \ref{item:sumequal}, assume without loss of generality that $v(c) < v(c')$. Then,
\[
    v(c) = v((c+c') - c') \geqslant \min\{ v(c+c'), v(-c') \} = \min\{ v(c+c'), v(c') \}.
\]
Since we assumed that $v(c) < v(c')$, the previous line implies that $\min\{ v(c+c'), v(c') \} = v(c + c')$; hence, $v(c) = \min\{ v(c), v(c') \} \geqslant v(c + c')$. But $v(c + c') \geqslant \min\{v(c),v(c')\}$ by \ref{item:sum}, so we obtain \ref{item:sumequal}. 

For \ref{item:groupelement}, we have
\begin{align*}
    v(gc) &= \inf\{ \chi(h) + v(x) : hx \in \supp(gc) \} \\
    &= \inf\{ \chi(g (g\inv h)) + v(x) : hx \in g \cdot \supp c \} \\
    &= \chi(g) + \inf\{ \chi(g\inv h) + v(x) : (g\inv h)x \in \supp c \} \\
    &= \chi(g) + v(c).
\end{align*}

For \ref{item:infheight}, we first show that if $c \in F_n \setminus \{0 \}$, then $v(c) < \infty$ by induction on $n$. This is true for $n = 0$ since $\chi(G) \subseteq \R$. Now let $n > 0$. Since $c \neq 0$, there is some element $gx$ in its support, where $g \in G$ and $x \in X$. Then
\[
    v(c) \leqslant v(gx) = \chi(g) + v(x) = \chi(g) + v(\partial x) < \infty
\]
by the inductive hypothesis and by admissibility of $F_\bullet \longrightarrow M \longrightarrow 0$ with respect to $X$. For a general nonzero element $c \in F$, write $c = \sum_i c_i$ with $c_i \in F_i$. Then $v(c) = \inf_i\{ v(c_i) \} < \infty$ since at least one of the chains $c_i$ is nonzero. Conversely, if $c = 0$, then $v(c) = \infty$, since the infimum of the empty set is $\infty$.

For \ref{item:bdyincrease}, let $c = \sum_{g \in G, x \in X} r_{g,x} gx \in \bigoplus_{i \geqslant 1} F_i$. Then
\begin{align*}
    v(\partial c) &= v\left(\partial\left(\sum_{g \in G, x \in X} r_{g,x} gx\right)\right) \\
    &= v\left( \sum_{g \in G, x \in X} r_{g,x} g \partial x \right) \\
    &\geqslant \inf\left\{ v(r_{g,x} g\partial x) : gx \in \supp(c) \right\} \tag{by \ref{item:sum}} \\
    &\geqslant \inf\left\{ v( g\partial x) : gx \in \supp(c) \right\}  \tag{by \ref{item:scalar}} \\
    &= \inf\{ \chi(g) + v(\partial x) : gx \in \supp(c) \} \\
    &= \inf\{ \chi(g) + v(x) : gx \in \supp(c) \} \\
    &= v(c). \qedhere
\end{align*}
\end{proof}








\begin{defn}[valuation subcomplex and essential acyclicity]
Given an admissible free resolution $F_\bullet \longrightarrow M \longrightarrow 0$ over $RG$ (with respect to some fixed basis $X$), a non-trivial character $\chi \colon G \longrightarrow \R$, and the valuation $v \colon F \longrightarrow \R_\infty$ extending $\chi$, define the \textit{valuation subcomplex} of $F$ with respect to $v$ to be the chain complex $\cdots \longrightarrow F_n^v \longrightarrow \cdots \longrightarrow F_0^v \longrightarrow M \longrightarrow 0$, where $F_n^v = \{c \in F_n : v(c) \geqslant 0\}$. We denote the valuation subcomplex by $F_\bullet^v \longrightarrow M \longrightarrow 0$ and let $F^v := \bigoplus_{i = 0}^\infty F_i^v$. \cref{PropPropsOfVal}\ref{item:bdyincrease} ensures that $F_\bullet^v \longrightarrow M \longrightarrow 0$ is a chain complex of left $RG_\chi$-modules, where $G_\chi$ is the monoid $\{g \in G : \chi(g) \geqslant 0\}$. It is not hard to show that each $F_i^v$ is a free $RG_\chi$-module and has an $RG_\chi$-basis of cardinality $\verti{X_i}$, where $X_i$ is an $RG$-basis for $F_i$.

The chain  complex $F_\bullet^v \longrightarrow M \longrightarrow 0$ is \textit{essentially acyclic in dimension $n$} if there is a real number $D \geqslant 0$ such that for every cycle $z \in F_n^v$ there is a $c \in F_{n+1}$ with $\partial c = z$ and $D \geqslant v(z) - v(c)$. We extend the definition of essential acyclicity to dimension $-1$ by declaring that $v(m) = 0$ for all $m \in M \setminus \{0\}$.
\end{defn}

The definition of essential acyclicity in dimension $n$ is equivalent to the following seemingly weaker condition: for every cycle $z \in F_n^v$, there is a $c \in F_{n+1}$ such that $\partial c = z$ and $v(c) \geqslant -D$. To see this, let $z \in F_n^v$ be a cycle. It is easily shown that $v(F) \subseteq \chi(G) \cup \{\infty\}$, so there is a $g \in G$ such that $\chi(g) = v(z)$. Since $g\inv z$ is also a in $F_n^v$ with $v(g\inv z) = 0$, there is some $c \in F_{n+1}^v$ such that $\partial c = g\inv z$ and $v(c) \geqslant -D$. Thus, $\partial (gc) = z$, and $D \geqslant v(z) - v(gc)$.



%%%            %%%
%%% HOROCHAINS %%%
%%%            %%%
\section{Horochains} \label{sec:horo}



\begin{defn}[complex of horochains and horo-acyclicity]
Let $F_\bullet \longrightarrow M \longrightarrow 0$ be an admissible free resolution with respect to some basis $X$, let $\chi \colon G \longrightarrow \R$ be a nontrivial character, and let $v \colon F \longrightarrow \R_\infty$ be the valuation extending $\chi$. Define $\widehat{F}$ to be the left $RG$-module of chains that are finitely supported below every height. More precisely, $\widehat{F}$ is the $RG$-module of formal sums $\sum_{g \in G, x \in X} r_{g,x} g x$ such that $\{ gx : v(gx) \leqslant t, r_{g,x} \neq 0 \}$ is finite for every $t \in \R$. The elements of $\widehat{F}$ are called \textit{horochains}. If $\hat{c} \in \widehat{F}$, then its \textit{support} is $\supp_X(\hat{c}) := \{ gx : r_{g,x} \neq 0 \}$. Let $\widehat{F}_i \subseteq \widehat{F}$ be the subset of chains with support in $F_i$ and let $\widehat{F}^{(n)} := \bigoplus_{i = 0}^n \widehat{F}_i$. \cref{PropPropsOfVal}\ref{item:bdyincrease} guarantees that $\partial \colon F_n \longrightarrow F_{n-1}$ extends to a map $\partial \colon \widehat{F}_n \longrightarrow \widehat{F}_{n-1}$ in the obvious way, so we get a complex $\cdots \longrightarrow \widehat{F}_n \longrightarrow \cdots \longrightarrow \widehat{F}_0 \longrightarrow 0$. Note that $\widehat{F}$ is not equal to $\bigoplus_{i = 0}^\infty \widehat{F}_i$ since the support of a horochain might intersect infinitely many of the modules $\widehat{F}_i$. We say that $F_\bullet \longrightarrow M \longrightarrow 0$ is \textit{horo-acyclic} in dimensions $n \geqslant 0$ with respect to $v$ if the chain complex $\cdots \longrightarrow \widehat{F}_1 \longrightarrow \widehat{F}_0 \longrightarrow 0$ is acyclic in dimension $n$.
\end{defn}

We can extend the definition of $v$ to $\widehat{F}$ by defining $v(\hat{c}) := \inf\{ v(gx) : \supp(\hat{c})\}$ for any horochain $\hat{c}$. If $\hat{c} \neq 0$, then $\{ v(gx) : \supp(\hat{c})\}$ is nonempty and attains a minimum because chains are finitely supported below any given height.  Properties \ref{item:sum} through \ref{item:infheight} of \cref{PropPropsOfVal} hold in this setting with the same proofs. 

A version of \cref{PropPropsOfVal}\ref{item:bdyincrease} holds for horochains, namely we have $v(\partial \hat{c}) \geqslant v(\hat{c})$ for all horochains $\hat{c}$, but we need to modify the proof: If $\hat{c} = 0$, then the claim is clear. Otherwise, let $\hat{c} \neq 0$ be a horochain, and let $gx \in \supp(\hat{c})$ be such that $v(gx) = v(\hat{c})$. By the finite version of \ref{item:bdyincrease}, we have that $v(\partial g'x') \geqslant v(g'x') \geqslant v(gx)$ for every $g'x' \in \supp(\hat{c})$. Since every $g''x'' \in \supp(\partial \hat{c})$ is contained in $\supp(\partial g'x')$ for some $g'x' \in \supp(\hat{c})$, we have that $v(g''x'') \geqslant v(\partial g'x') \geqslant v(gx)$ for every $g''x'' \in \supp(\partial \hat{c})$. Thus, $v(\partial \hat{c}) \geqslant v(\hat{c})$. 


\smallskip

The following lemma will be used in the proof of \cref{thm:Main}.

\begin{lem} \label{lem:ExtendHom}
Let $F_\bullet \longrightarrow M \longrightarrow 0$ (resp.~$F_\bullet' \longrightarrow M \longrightarrow 0$) be a free resolution over $RG$ admissible with respect to a basis $X$ (resp.~$X'$), and let $v$ (resp.~$v'$) be the valuation extending a nontrivial character $\chi\colon G \longrightarrow \R$. Suppose that $F^{(n)}$ is finitely generated, and that $\varphi \colon F \longrightarrow F'$ is a homomorphism of $RG$-modules. Then
\begin{enumerate}[label=(\arabic*)]
    \item \label{item:extend} $\varphi$ induces a homomorphism of left $RG$-modules given by
    \[
        \widehat{\varphi} \colon \widehat{F}^{(n)} \longrightarrow \widehat{F}', \ \sum r_{g,x} gx \longmapsto \sum r_{g,x} g \varphi(x) \ \ ;
    \]
    \item \label{item:valIneq} $v'(\widehat{\varphi}(\hat c)) \geqslant v(\hat{c}) + \min_{x \in X^{(n)}} \{ v'(\varphi(x)) - v(x) \}$ for every $\hat{c} \in F^{(n)}$.
\end{enumerate}
\end{lem}

\begin{proof}
For \ref{item:extend}, we need to show that $\widehat{\varphi}(\hat{c})$ is a horochain  for any horochain $\hat{c} \in \widehat{F}^{(n)}$. To this end, let $\hat{c} = \sum r_{g,x} gx$, and note that there are only finitely many elements $x \in X$ such that $gx \in \supp_X(\hat{c})$. If $\widehat{\varphi}(\hat{c})$ is not a horochain, then the set $\{ gx \in \supp_X(\hat{c}) : v'(g\varphi(x)) \leqslant t \}$ is infinite for some $t \in \R$. Since $F^{(n)}$ is finitely generated, there is some fixed $y \in X$ such that $v'(g\varphi(y)) \leqslant t$ and $gy \in \supp_X(\hat{c})$ for infinitely many values of $g \in G$. But then
\begin{align*}
    v(gy) &= \chi(g) + v(y) \\
        &= \chi(g) + v'(\varphi(y)) + v(y) - v'(\varphi(y))\\
        &= v'(g\varphi(y)) + v(y) - v'(\varphi(y)) \\
        &\leqslant t + v(y) - v'(\varphi(y))
\end{align*}
for infinitely many $gy \in \supp_X(\hat{c})$, but $\hat{c}$ is a horochain. 

For \ref{item:valIneq}, write $\hat{c} = \sum_{x\in X^{(n)}} \hat{c}_x$, where $\hat{c}_x = \sum_{g \in G} r_{g,x} g x$. Then
\begin{align*}
    v'(\widehat{\varphi}(\hat{c})) &\geqslant \min_{x \in X^{(n)}}\{ v'(\widehat{\varphi}(\hat{c}_x))\} \\
    &\geqslant \min_{x \in X^{(n)}} \{ \inf\{ v'(g\varphi(x)) : gx \in \supp \hat{c}_x \}   \} \\
    &= \min_{x \in X^{(n)}}\{ \inf\{ v(gx) : gx \in \supp \hat{c}_x \} + v'(\varphi(x)) - v(x)  \} \\
    &= \min_{x \in X^{(n)}} \{ v(\hat{c}_x) + v'(\varphi(x)) - v(x)  \} \\
    &\geqslant \min_{x \in X^{(n)}} \{ v(\hat{c}_x) \} +  \min_{x \in X^{(n)}}\{ v'(\varphi(x)) - v(x) \} \\
    &= v(\hat{c}) + \min_{x \in X^{(n)}} \{ v'(\varphi(x)) - v(x) \}. \qedhere
\end{align*}  \qedhere
\end{proof}

Note that \cref{lem:ExtendHom}\ref{item:valIneq} applies to chains in $F$ since these are just finite horo-chains. We will use this in the proof \cref{thm:Main}.









%%%                                                 %%%
%%%     CHARACTERISATIONS OF THE SIGMA INVARIANT    %%%
%%%                                                 %%%
\section{Characterisations of the \texorpdfstring{$\Sigma$}{Sigma}-invariant} \label{sec:SigInv}

We introduce the invariants $\Sigma^n_R(G;M)$, which are generalisations of the classical Bieri--Neumann--Strebel invariant \cite{BNSinv87} and its higher dimensional analogues \cite{BieriRenzValutations}. The only difference is that we work over a general ring $R$, while the higher BNS invariants are defined over $\Z$.

Let $G$ be a group. We declare two characters $\chi,\chi' \colon G \longrightarrow \R$ to be \textit{equivalent} if $\chi = \alpha \cdot \chi'$ for some $\alpha > 0$ and let $S(G)$ denote the set of equivalence classes of nonzero characters. We call $S(G)$ the \textit{character sphere} of $G$, because it can be given the topology of a sphere when $G$ is finitely generated. 

\begin{defn}[$\Sigma$-invariants]
Let $M$ be an $RG$-module. Then define
\[
\Sigma^n_R(G;M) = \{ [\chi] \in S(G) : M \in \mathtt{FP}_n(RG_\chi)  \},
\]
where $G_\chi = \{ g \in G : \chi(g) \geqslant 0 \}$. Note that $G_\chi = G_{\chi'}$ if $[\chi] = [\chi']$, so $\Sigma^n_R(G;M)$ is well-defined.
\end{defn}



\begin{defn}[Novikov ring]
Let $G$ be a group, let $R$ be a ring, and let $\chi \colon G \longrightarrow \R$ be a character. Then the \textit{Novikov ring} $\widehat{RG}^\chi$ is the set of formal sums
\[
\sum_{g \in G} r_g g
\]
such that $\{ g \in G : r_g \neq 0 \ \text{and} \ \varphi(g) \leqslant t \}$ is finite for every $t \in \R$. We give $\widehat{RG}^\chi$ a ring structure by defining $rg + r'g := (r + r')g$ and $rg \cdot r'g' := rr' gg'$ for $r,r' \in R$, $g,g' \in G$, and extending multiplication to all of $\widehat{RG}^\chi$ in the obvious way.
\end{defn}

\cref{thm:Main} gives several characterisations of the $\Sigma$-invariants and is the main technical tool we will need to prove \cref{thm:agrarianMain}. More specifically, we will need the characterisation of $\Sigma_R^n(G;M)$ in terms of the vanishing of Novikov homology; this it the equivalence of \ref{item:SigmaInv} and \ref{item:TorCond} in the following theorem, which should be thought of as a higher dimensional version of Sikorav's theorem \cite{SikoravThese}.

\begin{thm} \label{thm:Main}
Let $R$ be a ring, let $M$ be a left $RG$-module of type $\mathtt{FP}_n$, and let $\chi \colon G \longrightarrow \R$ be a non-trivial character. Let $F_\bullet \longrightarrow M \longrightarrow 0$ be a free resolution admissible with respect to a basis $X = \bigcup_{i = 0}^\infty X_i$ and with finitely generated $n$-skeleton $F^{(n)}$. Let $v \colon F \longrightarrow \R_\infty$ be the valuation extending $\chi$ with respect to $X$. The following are equivalent:
\begin{enumerate}[label=(\arabic*)]
    \item\label{item:SigmaInv} $[\chi] \in \Sigma_R^n(G;M)$;
    \item\label{item:EssAcyc} $F_\bullet^v \longrightarrow M \longrightarrow 0$ is essentially acyclic in dimensions $-1, \dots, n-1$;
    \item\label{item:LiftId} there is a chain map $\varphi \colon F \longrightarrow F$ lifting the identity $\id_M$ such that $v(\varphi(c)) > v(c)$ for every $c \in F^{(n)}$;
    \item\label{item:HoroAcyc} $F_\bullet \longrightarrow M \longrightarrow 0$ is horo-acyclic in dimensions $0, \dots, n$ with respect to $v$;
    \item\label{item:TorCond} $\Tor_i^{RG}(M, \widehat{RG}^\chi) = 0$ for all $0 \leqslant i \leqslant n$.
\end{enumerate}
\end{thm}



The strategy of the proof will be as follows: we begin by proving \ref{item:EssAcyc} $\Longrightarrow$ \ref{item:LiftId} $\Longrightarrow$ \ref{item:HoroAcyc} $\Longrightarrow$ \ref{item:EssAcyc}. This is done by  Schweitzer in the appendix of \cite{BieriDeficiency} in the case $R = \Z$. Once this is done, we prove the equivalence of \ref{item:HoroAcyc} and \ref{item:TorCond}, again following Schweitzer. Finally, we prove the equivalence of \ref{item:SigmaInv} and \ref{item:EssAcyc} following the appendix to Theorem 3.2 in \cite{BieriRenzValutations}, where again this is done in the case $R = \Z$. The proofs below and are essentially the same as those given in the references just cited; there is no crucial dependence on the coefficient ring $R$.

\begin{proof}[Proof of \ref{item:EssAcyc} $\Longrightarrow$ \ref{item:LiftId}]
Assume that $F_\bullet^v \longrightarrow M \longrightarrow 0$ is essentially acyclic in dimensions $\leqslant n-1$ and let $D > 0$ be a constant such that for each $k < n$ and every cycle $z \in F_k$, there is a chain $c \in F_{k+1}$ with $\partial c = z$ and $D \geqslant v(z) - v(c)$. 
We will construct a chain map $\varphi \colon F \longrightarrow F$ lifting $\id_M$ such that $v(\varphi(c)) > v(c) + (n - k)D$ for every $c \in F^{(k)}$, which implies \ref{item:LiftId}.

We define $\varphi$ on $F^{(k)}$ by induction on $k$. For the base case, let $x \in X_0$ be arbitrary, and fix some $g \in G$ such that $\chi(g) > (n+1)D$. The element $g\inv\partial x \in M$ is a cycle, so there is some $c_x \in F_0$ such that $\partial c_x = g\inv \partial x$ and $D \geqslant v(g\inv \partial x) - v(c_x) = -v(c_x)$, since $v|_{M \setminus\{0\}} = 0$. Define $\varphi$ on $F^{(0)}$ by setting $\varphi(x) = gc_x$ for each $x \in X_0$. It is  clear that $\mathrm{id}_M  \partial = \partial  \varphi$ on $F^{(0)}$. By \cref{lem:ExtendHom}\ref{item:valIneq},
\[
v(\varphi(c)) \geqslant v(c) + \min_{x \in X_0} \{v(\varphi(x)) - v(x)\} > v(c) + nD
\]
for every $c \in F^{(0)}$.

Let $k > 0$ and suppose $\varphi$ is defined on $F^{(k-1)}$ such that it lifts $\id_M$ and $v(\varphi(c)) > v(c) + (n - k + 1)D$ for all $c \in F^{(k-1)}$. Let $x \in X_k$ and note that $\varphi(\partial x)$ is a cycle. By essential acyclicity, there is a chain $d_x \in F_k$ such that $\partial d_x = \varphi(\partial x)$ and $D \geqslant v(\varphi(\partial x)) - v(d_x)$. Define $\varphi$ on $F^{(k)}$ by setting $\varphi(x) = d_x$. Then $\varphi \partial = \partial \varphi$ by construction, and for every $x \in X_k$ we have
\begin{align*}
v(\varphi(x)) - v(x) &= v(d_x) - v(x) \\
    &\geqslant v(\varphi(\partial x)) - v(x) - D \\
    &= v(\varphi(\partial x)) - v(\partial x) - D \\
    &> (n-k)D
\end{align*}
by induction. By \cref{lem:ExtendHom}\ref{item:valIneq}, we have
\[
v(\varphi(c)) \geqslant v(c) + \min_{x \in X_k} \{ v(\varphi(x)) - v(x) \} > v(c) + (n-k)D. \qedhere
\]
\end{proof}


We pause here to prove a lemma that will immediately imply \ref{item:LiftId} $\Longrightarrow$ \ref{item:HoroAcyc} and will be useful in the proofs of \ref{item:HoroAcyc} $\Longrightarrow$ \ref{item:EssAcyc} and \ref{item:HoroAcyc} $\Longrightarrow$ \ref{item:TorCond}. 

\begin{lem}\label{lem:Useful}
With the assumptions of \cref{thm:Main}, let $\varphi \colon F \longrightarrow F$ be a chain map lifting $\id_M$ such that $v(\varphi(c)) > v(c)$ for all $c \in F^{(n)}$ and let $H \colon F \longrightarrow F$ be a chain homotopy such that $\partial H + H \partial = \id_{F} - \varphi$. Let $\hat{z} \in \widehat{F}^{(n)}$ be a horocycle and define $\hat{c}_{\hat{z}} := \sum_{i = 0}^\infty \widehat{H} \widehat{\varphi}^i(\hat{z})$. Then $\hat{c}_{\hat{z}}$ is a horochain and $\partial \hat{c}_{\hat{z}} = \hat{z}$.
\end{lem}

\begin{proof}
By \cref{lem:ExtendHom}\ref{item:valIneq} there are constants $\alpha$ and $\beta$ such that
\[
    v(\widehat{\varphi}(\hat{c})) \geqslant v(\hat{c}) + \alpha \ \ \text{and} \ \ v(\widehat{H}(\hat{c})) \geqslant v(\hat{c}) + \beta
\]
for every horochain $\hat{c} \in \widehat{F}^{(n)}$. Moreover, $\alpha > 0$ since $v(\varphi(f)) > v(f)$ for every $f \in F^{(n)}$. To see that $\hat{c}$ is a horochain, by induction we have $v(\widehat{H}\widehat{\varphi}^i(\hat{z})) \geqslant v(\hat{z}) + i\alpha + \beta$, so for all $t \in \R$ there are only finitely many integers $i \geqslant 0$ such that $v(\widehat{H} \widehat{\varphi}^i(\hat{z})) \leqslant t$. Since $\supp(\hat{c}_{\hat{z}}) \subseteq \bigcup_{i=0}^\infty \supp(\widehat{H}\widehat{\varphi}^i(\hat{z}))$ and each $\widehat{H} \widehat{\varphi}^i(\hat{z})$ is a horochain, it follows that there are only finitely many $gx \in \supp \hat{c}$ such that $v(gx) \leqslant t$, so $\hat{c}$ is a horochain.

Finally, we have
\begin{equation*}
    \partial\hat{c} = \sum_{i=0}^\infty \partial \widehat{H} \widehat{\varphi}^i(\hat{z}) = \sum_{i=0}^\infty (\id_{\widehat{F}^{(n)}} - \widehat{\varphi} - \widehat{H}\partial)\widehat{\varphi}^i(\hat{z}) = \sum_{i=0}^\infty (\widehat{\varphi}^i - \widehat{\varphi}^{i+1})(\hat{z}) = \hat{z}. \qedhere
\end{equation*}
\end{proof}




\begin{proof}[Proof of \ref{item:LiftId} $\Longrightarrow$ \ref{item:HoroAcyc}]
If $\hat{z} \in \widehat{F}^{(n)}$ is a horocycle, then $\partial \hat{c}_{\hat{z}} = \hat z$ by \cref{lem:Useful}. 
\end{proof}



\begin{proof}[Proof of \ref{item:HoroAcyc} $\Longrightarrow$ \ref{item:EssAcyc}] We will prove that $F_\bullet^v \longrightarrow M \longrightarrow 0$ is essentially acyclic in dimension $k$ for all $k < n$ by induction on $k$. For the base case, we show that $F_\bullet^v \longrightarrow M \longrightarrow 0$ is exact at $M$, which implies essential acyclicity in dimension $-1$. Let $m \in M$. By exactness of $F_\bullet \longrightarrow M \longrightarrow 0$, there is a chain $c \in F_0$ such that $\partial c = m$. By horo-acyclicity in dimension $0$, there is some horochain $\hat{c} \in \widehat{F}_1$ such that $\partial \hat{c} = c$. There are $c_{-} \in F_1$ and $\hat{c}_+ \in \widehat{F}_1$ such that  $\hat{c} = c_{-} + \hat{c}_{+}$, where $v(c_{-}) < 0$ and $v(\hat{c}_{+}) \geqslant 0$. Then $\partial(c - \partial c_-) = m$ and 
\[
    v(c - \partial c_-) = v(c - \partial(\hat{c} - \hat{c}_+)) = v(\partial \hat{c}_+) \geqslant v(\hat{c}_+) \geqslant 0.
\]
This shows that $c - \partial c_0 \in F_0^v$, which proves that $F_\bullet^v \longrightarrow M \longrightarrow 0$ is exact at $M$.


Let $k > -1$ and suppose that $F_\bullet^v \longrightarrow M \longrightarrow 0$ is essentially acyclic in dimensions $< k$. By \ref{item:EssAcyc} $ \longrightarrow$ \ref{item:LiftId} applied at $k-1$, there is a chain map $\varphi \colon F \longrightarrow F$ lifting $\id_M$ such that $v(\varphi(c)) > v(c)$ for all $c \in F^{(k)}$. Since $\id_F$ and $\varphi$ both lift $\id_M$ and $F_\bullet \longrightarrow M \longrightarrow 0$ is acyclic, there is a chain homotopy $H \colon F \longrightarrow F$ such that $\partial H + H \partial = \id_{F} - \varphi$. As in the proof of \cref{lem:Useful}, there are constants $\alpha > 0$ and $\beta < 0$ such that 
\[
    v(\varphi(c)) \geqslant v(c) + \alpha \ \ \text{and} \ \ v(H(c)) \geqslant v(c) + \beta
\]
for every $c \in F^{(k)}$.

Let $z \in F^v_{k}$ be a cycle. Since $F_\bullet \longrightarrow M \longrightarrow 0$ is acyclic, there is some $d \in F_{k+1}$ such that $\partial d = z$. Consider the horocycle $\hat{z} := d - \hat{d}_z$, where $\hat{d}_z = \sum_{i = 0}^\infty H\varphi^i(z)$ is defined as in \cref{lem:Useful}. Note that
\[
    v(H\varphi^i(z)) \geqslant v(z) + i\alpha + \beta \geqslant \beta
\]
for every $i \geqslant 0$, and therefore that $v(\hat{d}_z) \geqslant \beta$. By horo-acyclicity in dimension $k+1$, there is a $(k+2)$-horochain $\hat{d}$ such that $\partial \hat{d} = \hat{z}$. As in the base case, there are $d_- \in F_{k+2}$ and $\hat{d}_+ \in \widehat{F}_{k+2}$ such that $\hat{d} = d_- + \hat{d}_+$, where $v(d_-) < 0$ and $v(\hat{d}_+) \geqslant 0$. Then $\partial(d - \partial d_-) = \partial d = z$, and
\begin{align*}
    v(d - \partial d_-) &= v(\hat{d}_z + \hat{z} - \partial(\hat{d} - \hat{d}_+)) \\
        &= v(\hat{d}_z + \partial \hat{d}_+) \\
        &\geqslant \min\{ v(\hat{d}_z), v(\partial \hat{d}_+) \} \\
        &\geqslant \beta
\end{align*}
since $v(\hat{d}_z) \geqslant \beta$ and $v(\partial d_\infty) \geqslant v(d_\infty) \geqslant 0 > \beta$. Letting $D = -\beta$ in the definition of essential acyclicity, we see that $F_\bullet^v \longrightarrow M \longrightarrow 0$ is essentially acyclic in dimension $k$. \qedhere
\end{proof}

\begin{proof}[Proof of \ref{item:TorCond} $\Longrightarrow$ \ref{item:HoroAcyc}] Suppose that $\Tor_i^{RG}(M,\widehat{RG}^\chi) = 0$ for $0 \leqslant i \leqslant n$. Consider the chain map
\[
\psi \colon \widehat{RG}^\chi \otimes_{RG} F \longrightarrow \widehat{F}, \ \ \alpha \otimes c \longmapsto \alpha c
\]
of left $RG$-modules. It is clear that $\psi$ is injective. We claim that $\psi$ induces an isomorphism $\widehat{RG}^\chi \otimes_{RG} F^{(n)} \longrightarrow \widehat{F}^{(n)}$. To see this, simply note that for an arbitrary horochain
\[
    \hat{c} = \sum_{g \in G, x \in X^{(n)}} r_{g,x} gx
\]
in $\widehat{F}^{(n)}$, we have
\[
    \sum_{x \in X^{(n)}}  \left( \sum_{g\in G} r_{g,x} g \right) \otimes x \xmapsto{\varphi} \hat{c}.
\]
The horochain condition implies that the sums $\sum_{g\in G} r_{g,x} g$ are elements of $\widehat{RG}^\chi$. Thus, $\psi$ is surjective on the $n$-skeleta and is therefore an isomorphism. Note that this only works because $X^{(n)}$ is finite; in general, we cannot expect $\psi$ to be surjective since the support of a horochain might intersect infinitely many of the modules $F_n$. Since $H_i(\widehat{RG}^\chi \otimes_{RG} F ; RG) = 0$ for all $0 \leqslant i \leqslant n$, we conclude that $H_i(\widehat{F}; RG) = 0$ for $0 \leqslant i \leqslant n$ as well.
\end{proof}


\begin{proof}[Proof of \ref{item:HoroAcyc} $\Longrightarrow$ \ref{item:TorCond}] The map $\psi$ defined above is an isomorphism of the $n$-skeleta, so we immediately have that $\Tor_i^{RG}(M, \widehat{RG}^\chi) = 0$ for $0 \leqslant i \leqslant n-1$. Since $\psi$ is not necessarily surjective as a map of the $(n+1)$-skeleta, we must work harder to show that $\Tor_n^{RG}(M, \widehat{RG}^\chi) = 0$. Let $z \in \widehat{RG}^\chi \otimes_{RG} F_n$ be an $n$-cycle, and let $\hat{z} = \psi(z)$. Since we are assuming that \ref{item:HoroAcyc} holds, we may also assume that \ref{item:LiftId} holds and use the horochain $\hat{c}_{\hat{z}}$ from \cref{lem:Useful}. Since $\hat{c}_{\hat{z}} \in \widehat{H}(\widehat{F}_n)$, we have that $\hat{c}_{\hat{z}}$ is in the $\widehat{RG}^\chi$-submodule of $\widehat{F}_{n+1}$ generated by $\widehat{H}(X_n)$, and thus $\hat{c}_{\hat{z}} \in \im \psi$ since this is a finite set. Let $c \in \widehat{RG}^\chi \otimes_{RG} F_{n+1}$ such that $\psi(c) = \hat{c}_{\hat{z}}$. Then $\psi \partial (c) = \partial \psi(c) = \partial \hat{c}_{\hat{z}} = \hat{z}$. But $\psi$ is injective, so $\partial c = z$, proving that $\Tor_n^{RG}(M, \widehat{RG}^\chi) = 0$.
\end{proof}


We pause again before proving the equivalence of \ref{item:SigmaInv} and \ref{item:EssAcyc} to prove another lemma. 

\begin{lem}\label{lem:flat}
Free $RG$-modules are flat over $RG_\chi$.
\end{lem}
\begin{proof}
It suffices to prove that $RG$ is flat as an $RG_\chi$-module, since the direct sum of flat modules is flat. To this end, let $\iota \colon M \longhookrightarrow N$ be an injection of right $RG_\chi$-modules; our goal is to show that $\iota \otimes \id \colon M \otimes_{RG_\chi} RG \longrightarrow N \otimes_{RG_\chi} RG$ is injective. Let $g \in G$ be such that $\chi(g) < 0$ and consider the left $RG_\chi$-module $RG_\chi g^k = \{ \alpha g^k : \alpha \in RG_\chi, k \in \Z \}$. The modules $RG_\chi g^k$ form a directed system with respect to the inclusion maps $RG_\chi g^k \longhookrightarrow RG_\chi g^l$ for $k \leqslant l$ and the direct limit is $\varinjlim RG_\chi g^k \cong RG$.

There are left $RG_\chi$-module isomorphisms $RG_\chi g^k \longrightarrow RG_\chi$ given by right multiplication by $g^{-k}$. Then $RG_\chi g^k$ is flat over $RG_\chi$, so $M \otimes_{RG_\chi} RG_\chi g^k \longrightarrow N \otimes_{RG_\chi} RG_\chi g^k$ is injective for all $k \in \Z$. By exactness of the direct limit,
\[
    \varinjlim(M \otimes_{RG_\chi} RG_\chi g^k) \longrightarrow \varinjlim(N \otimes_{RG_\chi} RG_\chi g^k)
\]
is injective. Since the direct limit commutes with the tensor product, the previous line implies $\iota \otimes \id_M$ is injective. \qedhere
\end{proof}

We now return to the proof of \cref{thm:Main}.

\begin{proof}[Proof of \ref{item:SigmaInv} $\Longleftrightarrow$ \ref{item:EssAcyc}] Let $g \in G$ be such that $\chi(g) < 0$ and let $E_k$ be the left $RG_\chi$-module $g^k F^v$. We denote the chain complexes $F^v_\bullet \longrightarrow M \longrightarrow 0$ and $(E_k)_\bullet \longrightarrow M \longrightarrow 0$ by $\widetilde{F}^v$ and $\widetilde{E}_k$, respectively.

Essential acyclicity in dimension $j$ is equivalent to the the existence of an integer $D \geqslant 0$ such that the inclusion-induced homomorphism $H_j(\widetilde{E}_k) \longrightarrow H_j(\widetilde{E}_{k+D})$ is the zero map for all $k \in \N$. This in turn is equivalent to $\varinjlim \prod_{I} H_j(\widetilde{E}_k) = 0$ for any index set $I$. Here, for fixed $I$ and $j$, the powers $\prod_{I} H_j(\widetilde{E}_k)$ form a directed system with respect to the inclusion-induced maps $\prod_I H_j(\widetilde{E}_k) \longrightarrow \prod_I H_j(\widetilde{E}_l)$ for $k \leqslant l$. Indeed, if $D \geqslant 0$ is such that $H_j(\widetilde{E}_k) \longrightarrow H_j(\widetilde{E}_{k+D})$ is the zero map, it is clear that the direct limit will be zero. Conversely, let $I = Z_j(\widetilde{E}_0) = Z_j(\widetilde{F}^v)$ be the set of $j$-cycles of $\widetilde{F}^v$ and consider the element $([x])_{x \in I} \in \prod_I H_j(\widetilde{E}_0)$. Since the direct limit is zero, there is some $D \geqslant 0$ such that $([x])_{x \in I} = 0$ in $\prod_I H_j(\widetilde{E}_D)$, which means that $\widetilde{F}^v$ is essentially acyclic in dimension $j$.

There is a short exact sequence of chain complexes $0 \longrightarrow M \longrightarrow \widetilde{E}_k \longrightarrow E_k \longrightarrow 0$, where by abuse of notation $M$ is a chain complex concentrated in dimension $-1$ and $E_k$ is the chain complex $(E_k)_\bullet \longrightarrow 0$ with $(E_k)_0$ in dimension $0$. The long exact sequence in homology associated to the short exact sequence gives $H_j(\widetilde{E}_k) \cong H_j(E_k)$ for $j \geqslant 1$. The interesting part of the long exact sequence is
\[
0 \longrightarrow H_0(\widetilde{E}_k) \longrightarrow H_0(E_k) \xrightarrow{\delta} M \longrightarrow H_{-1}(\widetilde{E}_k) \longrightarrow 0,
\]
where $\delta$ is the connecting homomorphism. By exactness of the direct power and direct limit functors, the sequence
\[
0 \rightarrow \varinjlim \prod_I H_0(\widetilde{E}_k) \rightarrow \varinjlim \prod_I H_0(E_k) \xrightarrow{\prod_I \delta} \prod_I M \rightarrow \varinjlim \prod_I H_{-1}(\widetilde{E}_k) \rightarrow 0
\]
is exact. Then $\widetilde{F}^v$ is essentially exact in dimension $0$ if and only if $\delta$ induces an injection $\varinjlim \prod_I H_0(E_k) \longrightarrow \prod_I M$ for every $I$. Moreover, $\widetilde{F}^v$ is essentially exact in dimension $-1$ if and only if $\delta$ induces a surjection $\varinjlim \prod_I H_0(E_k) \longrightarrow \prod_I M$ for every $I$ .

By \cref{lem:flat},  $F_\bullet \longrightarrow M \longrightarrow 0$ is a flat resolution of $M$ by left $RG_\chi$-modules, so 
\[
    \Tor^{RG_\chi}_j \left(M, \prod_I RG_\chi \right) = H_j\left( \left(\prod_I RG_\chi \right) \otimes_{RG_\chi} F \right).
\]
and therefore
\[
    \Tor^{RG_\chi}_j \left(M, \prod_I RG_\chi \right) = \varinjlim H_j \left( \left(\prod_I RG_\chi \right) \otimes_{RG_\chi} E_k \right),
\]
as $F = \varinjlim E_k$ and direct limits commute with tensor products and homology. Since $(E_k)_j$ is a finitely generated free $RG_\chi$-module for $j \leqslant n$, we have $\left(\prod_I RG_\chi \right) \otimes_{RG_\chi} (E_k)_j \cong \prod_I (E_k)_j$. Hence, $\Tor_j^{RG_\chi}(M, \prod_I RG_\chi) = \varinjlim H_j(\prod_I E_k)$ for $j < n$.

To summarize the work done above, we have $\widetilde{F}^v$ is essentially exact in dimensions $-1 \leqslant j < n$ if and only if
\begin{enumerate}[label=(\alph*)]
    \item $(\prod_I RG_\chi) \otimes_{RG_\chi} M \longrightarrow \prod_I M$ is surjective if $n = 0$ and
    \item $(\prod_I RG_\chi) \otimes_{RG_\chi} M \longrightarrow \prod_I M$ is an isomorphism and 
    \[
        \Tor_j^{RG_\chi}(M, \prod_I RG_\chi)
    \]
    vanishes for $1 \leqslant j < n$ otherwise.
\end{enumerate}
Here we have used the general fact that $\Tor_0^R(A,B) \cong A \otimes_R B$. Together with Lemma 1.1 and Proposition 1.2 of \cite{BieriEckmannFinProps}, (a) and (b) are equivalent to $M$ being of type $\mathtt{FP}_n(RG_\chi)$. Thus, we conclude that $[\chi] \in \Sigma^m_R(G;M)$ if and only $\widetilde{F}^v$ is essentially exact in dimensions $j = -1, 0, 1, \dots, n-1$. \qedhere
\end{proof}




%%%
%%% Hughes homology
%%%
\section{Agrarian homology and main result} \label{sec:homology}

\begin{defn}[agrarian groups and $\mathcal{D}$-homology]
Let $R$ be a ring. A group $G$ is \textit{agrarian over $R$} if there is a skew-field $\mathcal{D}$ and an injective ring homomorphism $R G \longhookrightarrow \mathcal{D}$. In this case, we will say that $G$ is $\mathcal{D}$-agrarian over $R$ if we wish to specify the skew-field.

If $G$ is $\mathcal{D}$-agrarian over $R$, we define its $p$-dimensional \textit{$\mathcal{D}$-homology} to be
\[
    H_p^{\mathcal{D}} (G) := \Tor_p^{RG} (R, \mathcal{D}),
\]
where $R$ is the trivial $RG$-module and $\mathcal{D}$ is viewed as a $\mathcal{D}$-$RG$-bimodule via the embedding $RG \longhookrightarrow \mathcal{D}$. The $p$th $\mathcal{D}$-Betti number of $G$ is then
\[
    b_p^{\mathcal{D}}(G) := \dim_\mathcal{D} H_p^{\mathcal{D}} (G).
\]
Note that $b_p^{\mathcal{D}}(G)$ is well-defined and integral or infinite, since a module over a skew-field has a well-defined dimension.
\end{defn}

\begin{rem}
The term ``agrarian" was introduced by Kielak in \cite{KielakBNSviaNewton} in the case $R = \mathbb{Z}$. Using strong Hughes-freeness, i.e.~condition \ref{item:2prime} after \cref{def:HfreeDiv}, it follows that if $G$ is a locally-indicable group and $\mathbb{F}$ is a skew-field such that $\mathcal{D}_{\mathbb{F}G}$ exists, then $G$ is $\mathcal{D}_{\mathbb{F}G}$-agrarian over $\mathbb{F}$. If $R$ is a skew-field or an integral domain, then there are no known examples of torsion-free groups that are not agrarian over $R$. For the remainder of the section, we will be interested in the $\mathcal{D}_{\mathbb{F}G}$-homology of $G$.
\end{rem}


In what follows we will need that $\mathcal{D}_{\mathbb{F}G}$-Betti numbers have good scaling properties when passing to finite index subgroups. This is analogous to the fact that $\ell^2$-Betti numbers also scale under taking finite index subgroups.


\begin{lem}\label{lem:JBscales}
Let $H$ be a finite index subgroup of $G$ and let $\mathbb{F}$ be a skew-field such that $\mathcal{D}_{\mathbb{F}G}$ exists. Then 
\[
    b_p^{\mathcal{D}_{\mathbb{F}G}}(G) = \frac{b_p^{\mathcal{D}_{\mathbb{F}H}}(H)}{[G:H]}.
\]
\end{lem}
\begin{proof} It suffices to prove the claim when $H$ is normal in $G$. To see this, if $H \leqslant G$ is any subgroup of finite index, then there is normal subgroup $N \triangleleft G$ of finite index such that $N \leqslant H \leqslant G$ (we can take $N$ to be the \textit{normal core} of $H$, that is, the intersection of all the conjugates of $H$). Then
\[
b_p^{\mathcal{D}_{\mathbb{F}G}}(G) = \frac{b_p^{\mathcal{D}_{\mathbb{F}N}}(N)}{[G:N]} = \frac{[H:N] b_p^{\mathcal{D}_{\mathbb{F}H}}(H)}{[G:N]} = \frac{b_p^{\mathcal{D}_{\mathbb{F}H}}(H)}{[G:H]},
\]
by the claim for normal subgroups.

Assume that $H$ is a finite index normal subgroup of $G$ and let $\{t_1, \dots, t_n \}$ be a transversal for $H$ in $G$. By \cref{prop:twistedNormalSkew}\ref{item:twistedFiniteIndex}, we have a $\mathcal{D}_{\mathbb{F}H}$-$\mathbb{F}G$-bimodule isomorphism $\mathcal{D}_{\mathbb{F}G} \cong  \mathcal{D}_{\mathbb{F}H} \otimes_{\mathbb{F}H} \mathbb{F}G$. Take a free resolution $F_\bullet \longrightarrow \mathbb{F} \longrightarrow 0$ of the trivial left $\mathbb{F}G$-module $\mathbb{F}$; there are chain-isomorphisms
\[
    \mathcal{D}_{\mathbb{F}G} \otimes_{\mathbb{F}G} F_\bullet 
    \cong (\mathcal{D}_{\mathbb{F}H} \otimes_{\mathbb{F}H} \mathbb{F}G) \otimes_{\mathbb{F}G} F_\bullet  
    \cong \mathcal{D}_{\mathbb{F}H} \otimes_{\mathbb{F}H} (\mathbb{F}G \otimes_{\mathbb{F}G} F_\bullet) 
    \cong \mathcal{D}_{\mathbb{F}H} \otimes_{\mathbb{F}H} F_\bullet
\]
of left $\mathcal{D}_{\mathbb{F}H}$-modules, so $H_p^{\mathcal{D}_{\mathbb{F}G}}(G) \cong H_p^{\mathcal{D}_{\mathbb{F}H}}(H)$ as $\mathcal{D}_{\mathbb{F}H}$-modules. Therefore,
\begin{align*}
    b_p^{\mathcal{D}_{\mathbb{F}H}}(H) &= \dim_{\mathcal{D}_{\mathbb{F}H}} H_p^{\mathcal{D}_{\mathbb{F}H}}(H) \\
        &= \dim_{\mathcal{D}_{\mathbb{F}H}} H_p^{\mathcal{D}_{\mathbb{F}G}}(G) \\
        &= [G : H] \cdot \dim_{\mathcal{D}_{\mathbb{F}G}} H_p^{\mathcal{D}_{\mathbb{F}G}}(G) \\ 
        &= [G : H] \cdot b_p^{\mathcal{D}_{\mathbb{F}G}}(G). \qedhere
\end{align*}
\end{proof}

In view of \cref{lem:JBscales}, if $G$ is a group with a finite index subgroup $H$ such that $\mathcal{D}_{\mathbb{F}H}$ exists, we can define $b_p^{\mathcal{D}_{\mathbb{F}G}}(G) = b_p^{\mathcal{D}_{\mathbb{F}H}}(H)/[G:H]$. This is an abuse of notation  since $\mathcal{D}_{\mathbb{F}G}$ might not exist.

The following theorem is an analogue of a theorem of L\"uck which holds for $\ell^2$-Betti numbers \cite[Theorem 7.2]{Luck02}.


\begin{thm}\label{thm:JBSES}
Let $1 \longrightarrow K \longrightarrow G \longrightarrow \Z \longrightarrow 1$ be a short exact sequence of groups and suppose that $\mathcal{D}_{\mathbb{F}G}$ exists for some skew-field $\mathbb{F}$. If $b_p^{\mathcal{D}_{\mathbb{F}K}}(K) < \infty$ for some $p \geqslant 0$, then $b_p^{\mathcal{D}_{\mathbb{F}G}}(G) = 0$. 
\end{thm}
\begin{proof}
Let $F_\bullet \longrightarrow \mathbb{F} \longrightarrow 0$ be a free resolution of $\mathbb{F}$ by left $\mathbb{F} G$-modules. Note that the modules $F_j$ are also free left $\mathbb{F}K$-modules and there are chain maps $\iota_j \colon \mathcal{D}_{\mathbb{F}K} \otimes_{\mathbb{F}K} F_j \longrightarrow \mathcal{D}_{\mathbb{F}G} \otimes_{\mathbb{F}G} F_j$, induced by the inclusion $\mathcal{D}_{\mathbb{F}K} \longhookrightarrow \mathcal{D}_{\mathbb{F}G}$. We claim that the maps $\iota_j$ are injective. To see this, it is enough to consider the case where $F_j = \mathbb{F}G$. Choose $t \in G$ such that $\{ t^n : n \in \Z \}$ is a transversal for $K \leqslant G$. By \cref{prop:twistedNormalSkew}, there is an embedding $\bigoplus_{n \in \Z} \mathcal{D}_{\mathbb{F}K} \cdot \varphi(t^n) \longhookrightarrow \mathcal{D}_{\mathbb{F}G}$. Since $\mathbb{F}G$ is free over $\mathbb{F}K$, there is also an isomorphism $\mathcal{D}_{\mathbb{F}K} \otimes_{\mathbb{F}K} \mathbb{F}G \cong \bigoplus_{n \in \Z} \mathcal{D}_{\mathbb{F}K} \cdot \varphi(t^n)$ determined by $\alpha \otimes t^n \longmapsto \alpha \varphi(t^n)$. Then the diagram
\[
\begin{tikzcd}
\mathcal{D}_{\mathbb{F}K} \otimes_{\mathbb{F}K} \mathbb{F}G \arrow[d, "\iota_j"'] \arrow[r, "\cong", no head] & {\bigoplus_{n \in \mathbb{Z}} \mathcal{D}_{\mathbb{F}K} \cdot \varphi(t^n)}  \arrow[d, hook] \\
\mathcal{D}_{\mathbb{F}G} \otimes_{\mathbb{F}G} \mathbb{F}G \arrow[r, "\cong", no head]                       &  \mathcal{D}_{\mathbb{F}G}
\end{tikzcd}
\]
of left $\mathcal{D}_{\mathbb{F}K}$-modules commutes, proving that $\iota_j$ is an injection. From now on, we will treat the maps $\iota_j$ as inclusions.

Consider the following portions of the chain complexes computing $b_p^{\mathcal{D}_{\mathbb{F}G}}(G)$ and $b_p^{\mathcal{D}_{\mathbb{F}K}}(K)$
\[
\begin{tikzcd}[column sep = small]
\cdots \arrow[r] & \mathcal{D}_{\mathbb{F}K} \otimes_{\mathbb{F}K} F_{p+1} \arrow[d, hook, "\iota_{p+1}"] \arrow[r] & \mathcal{D}_{\mathbb{F}K} \otimes_{\mathbb{F}K} F_p \arrow[d, hook, "\iota_p"] \arrow[r] & \mathcal{D}_{\mathbb{F}K} \otimes_{\mathbb{F}K} F_{p-1} \arrow[d, hook, "\iota_{p-1}"] \arrow[r] & \cdots \\
\cdots \arrow[r] & \mathcal{D}_{\mathbb{F}G} \otimes_{\mathbb{F}G} F_{p+1} \arrow[r]                               & \mathcal{D}_{\mathbb{F}G} \otimes_{\mathbb{F}G} F_p \arrow[r]                           & \mathcal{D}_{\mathbb{F}G} \otimes_{\mathbb{F}G} F_{p-1} \arrow[r]                                 & \cdots \nospacepunct{.}
\end{tikzcd}
\]
Let $x$ be a cycle in $\mathcal{D}_{\mathbb{F}G} \otimes_{\mathbb{F}G} F_p$. By \cite[Proposition 2.2(2)]{JaikinZapirain2020THEUO}, $\mathcal{D}_{\mathbb{F}G} \cong \mathrm{Ore}(\mathcal{D}_{\mathbb{F}K} * \Z)$, where we are making the identifications $\mathcal{D}_{\mathbb{F}K} \otimes_{\mathbb{F}K} \mathbb{F}G \cong \bigoplus_{n \in \Z} \mathcal{D}_{\mathbb{F}K} \cdot \varphi(t^n) \cong \mathcal{D}_{\mathbb{F}K} * \Z$. Hence, there is some nonzero $a \in \mathcal{D}_{\mathbb{F}K} * \Z$ such that $ax \in \mathcal{D}_{\mathbb{F}K} \otimes_{\mathbb{F}K} F_p$. Since $(\mathcal{D}_{\mathbb{F}K} * \Z) \cdot ax \subseteq Z_p(\mathcal{D}_{\mathbb{F}G} \otimes_{\mathbb{F}G} F_\bullet)$ and $\iota_{p-1}$ is injective, $(\mathcal{D}_{\mathbb{F}K} * \Z) \cdot ax$ is an infinite-dimensional $\mathcal{D}_{\mathbb{F}K}$-subspace of $Z_p(\mathcal{D}_{\mathbb{F}K} \otimes_{\mathbb{F}K} F_\bullet)$. Since $b_p^{\mathcal{D}_{\mathbb{F}K}}(K) < \infty$, there is a nonzero $b \in \mathcal{D}_{\mathbb{F}K} * \Z$ such that $bax = \partial y$ for some $y \in \mathcal{D}_{\mathbb{F}K} \otimes_{\mathbb{F}K} F_{p+1}$. But then $x = \partial((ba)\inv y)$, so we conclude that $H_p(\mathcal{D}_{\mathbb{F}G} \otimes_{\mathbb{F}G} F_\bullet) = 0$. \qedhere
\end{proof}




For the proof of the main theorem, we will need the following version of a theorem of Bieri and Renz. The details of the proof are given in \cite[Theorem 5.1]{BieriRenzValutations} in the case $R = \Z$, though the proof goes through in exactly the same way after replacing the ring $\Z$ by an arbitrary ring $R$.

\begin{thm}[Bieri-Renz]\label{thm:genBR}
Let $G$ be a finitely generated group and let $N \triangleleft G$ be a normal subgroup containing the commutator subgroup $[G,G]$. Let $R$ be a unital ring and let $M$ be an $RG$-module of type $\mathtt{FP}_n(RG)$. Then $M \in \mathtt{FP}_n(RN)$ if and only if $\Sigma_R^n(G;M) \supseteq S(G,N) := \{ [\chi] \in S(G) : \chi(N) = 0 \}$.
\end{thm}

We will also need the following result due to Kielak and Jaikin-Zapirain. Kielak first proved the result in \cite[Theorem 5.2]{KielakRFRS} by giving an explicit construction of the Linnell skew-field $\mathcal D(G)$ when $G$ is RFRS. In the appendix to \cite{JaikinZapirain2020THEUO}, Jaikin-Zapirain showed that when $G$ is RFRS and $\mathbb F$ is any skew-field, then $\mathcal D_{\mathbb F G}$ exists and admits a completely analogous construction to $\mathcal D(G)$ (in fact $\mathcal D(G) = \mathcal D_{\Q G}$ for a RFRS group $G$). As a consequence, Kielak's proof of \cref{thm:KJZ} still holds after making the replacements $\Q \rightsquigarrow \mathbb F$ and $\mathcal D(G) \rightsquigarrow \mathcal D_{\mathbb FG}$.

\begin{thm}[Kielak, Jaikin-Zapirain] \label{thm:KJZ}
Let $\mathbb F$ be a skew-field, $G$ a finitely generated RFRS group, and $n \in \N$. Let $F_\bullet$ be a chain complex of free $\mathbb F G$-modules with $F_p$ is finitely generated and $H_p(\mathcal D_{\mathbb F G} \otimes_{\mathbb F G} F_\bullet) = 0$ for all $p \leqslant N$. Then, there exist a finite index subgroup $H \leqslant G$ and an open subset $U \subseteq S(H)$ such that
\begin{enumerate}
    \item the closure of $U$ contains $S(G)$;
    \item $U$ is invariant under the antipodal map;
    \item $H_p(\widehat{\mathbb F H}^\chi \otimes_{\mathbb F H} F_\bullet) = 0$ for every $p \leqslant n$ and every $[\chi] \in U$.
\end{enumerate}
\end{thm}

We are now ready to prove the main theorem.


\begin{thm}\label{thm:agrarianMain}
Let $\mathbb{F}$ be a skew-field and let $G$ be a virtually RFRS group of type $\mathtt{FP}_n(\mathbb{F})$. Then there is a finite index subgroup $H \leqslant G$ admitting a homomorphism onto $\Z$ with kernel of type $\mathtt{FP}_n(\mathbb{F})$ if and only if $b_p^{\mathcal{D}_{\mathbb{F}G}}(G) = 0$ for $p = 0, \dots, n$.
\end{thm}
\begin{proof}
($\Longrightarrow$) Let $\varphi \colon H \longrightarrow \Z$ be an epimorphism with kernel $K \in \mathtt{FP}_n(\mathbb{F})$. Then there is a free resolution $F_\bullet \longrightarrow \mathbb{F} \longrightarrow 0$ of the trivial $\mathbb{F}K$-module $\mathbb{F}$ with finitely generated $n$-skeleton. Therefore, 
\[
    b_p^{\mathcal{D}_{\mathbb{F}K}}(K) = \dim_{\mathcal{D}_{\mathbb{F}K}} H_p (\mathcal{D}_{\mathbb{F}K} \otimes_{\mathbb{F}K} F_\bullet) < \infty
\]
for $p \leqslant n$ and we have a short exact sequence $1 \longrightarrow K \longrightarrow H \longrightarrow \Z \longrightarrow 1$, so $b_p^{\mathcal{D}_{\mathbb{F}H}}(H) = 0$ for $p \leqslant n$ by \cref{thm:JBSES}. Then $b_p^{\mathcal{D}_{\mathbb{F}G}}(G) = 0$ for $p \leqslant n$ by \cref{lem:JBscales}.

\smallskip

($\Longleftarrow$) The properties of being of type $\mathtt{FP}_n(\mathbb{F})$ and of having vanishing $p$th $\mathcal{D}_{\mathbb{F}G}$-Betti number pass to finite index subgroups. Moreover, the property of being virtually fibred passes to finite index overgroups (the same is of course true of any virtual property). Hence, we may assume that $G$ is RFRS and of type $\mathtt{FP}_n(\mathbb{F})$. Then $H^1(G; \R) \neq 0$ by \cref{prop:RFRStoQ}.

Since $G \in \mathtt{FP}_n(\mathbb{F})$, there is a free resolution $F_\bullet \longrightarrow \mathbb{F} \longrightarrow 0$ of the trivial $\mathbb{F}G$-module $\mathbb{F}$ with $F_p$ finitely generated for $p \leqslant n$. By assumption, $H_p(\mathcal{D}_{\mathbb{F}G} \otimes_{\mathbb{F}G} F_\bullet) = 0$ for $p \leqslant n$. By \cref{thm:KJZ}, there is a finite index subgroup $H \leqslant G$ and an open subset $U \subseteq H^1(H;\R)$ such that the closure of $U$ contains $H^1(G;\R)$, is invariant under nonzero scalar multiplication, and $H_p(\widehat{\mathbb{F}H}^\varphi \otimes_{\mathbb{F}H} F_\bullet) = 0$ for $p \leqslant n$ and all $\phi \in U$. Since $U$ is nonempty, we can find a surjective character $\varphi \colon H \longrightarrow \Z$ in $U$. To see this, let $\varphi \in H^1(G;\R)$ be a nontrivial character. Since $U$ is open, $\varphi$ can be perturbed so that its image is in $\Q$. Finally, since $H$ is finitely generated, we can rescale the character so that it maps $H$ onto $\Z$.  Then $[\pm \varphi] \in \Sigma_\mathbb{F}^n(H;\mathbb{F})$ by \cref{thm:Main}. It is not hard to show that $[\chi] = [\pm \varphi]$ for any character $\chi \colon H \longrightarrow \R$ with $\ker \varphi \subseteq \ker \chi$. Thus, $\ker \varphi \in \mathtt{FP}_n(\mathbb{F})$ by \cref{thm:genBR}. \qedhere

\end{proof}

\begin{cor}\label{cor:typeFP}
    Let $G$ be virtually RFRS and of type $\FP(\mathbb{F})$. Then $G$ virtually algebraically fibres with kernel of type $\mathtt{FP}(\mathbb F)$ if and only if $G$ is $\DF{G}$-acyclic.
\end{cor}

\begin{proof}
    One direction is clear. If $G$ is $\DF{G}$-acyclic, then $G$ virtually algebraically fibres with kernel $K$ of type $\FP_n(\mathbb F)$ for $n > \cd_\mathbb F(G)$. But then $n > \cd_\mathbb F (K)$, so $K$ is of type $\FP(\mathbb F)$. \qedhere
\end{proof}

We now apply \cref{thm:agrarianMain} to the case $\mathbb{F} = \mathbb{Q}$.

\begin{defn}[$\ell^2$-Betti numbers] \label{def:l2b}
Let $G$ be a torsion-free group satisfying the Atiyah conjecture and let $\mathcal{D}(G)$ be the Linnell skew-field of $G$ (see the following remark). Define
\[
    b_p^{(2)}(G) = \dim_{\mathcal{D}(G)} \Tor_p^{\Q G} (\Q, \mathcal{D}(G))
\]
to be the \textit{$p$th $\ell^2$-Betti number of $G$}.

If a group $G$ has a torsion-free subgroup $H$ of finite index satisfying the Atiyah conjecture, we extend the definition of $\ell^2$-Betti numbers by declaring that
\[
    b_n^{(2)}(G) = \frac{b_n^{(2)}(H)}{[G:H]}.
\]
\end{defn}

\begin{rem}
This definition of $\ell^2$-Betti numbers for torsion-free groups satisfying the Atiyah conjecture agrees with the usual definition by \cite[Lemma 10.28(3)]{Luck02}. Moreover, $\ell^2$-Betti numbers for virtually torsion-free groups are well-defined and coincide with the usual definition by \cite[Theorem 6.54(6)]{Luck02}. We will not give the definition of the Linnell ring $\mathcal{D}(G)$ since that would take us too far afield. We take the Atiyah conjecture to be the statement that $\mathcal{D}(G)$ is a skew-field, which allows us to make \cref{def:l2b}. Linnell showed that this formulation implies the strong Atiyah conjecture over $\Q$ for torsion-free groups \cite{LinnellDivRings93}. The details of the reverse implication can be found in L\"uck's book \cite[Section 10]{Luck02}.
\end{rem}

In the case where $G$ is finitely generated and RFRS, $\mathcal{D}_{\Q G}$ exists and is isomorphic to the Linnell skew-field $\mathcal{D}(G)$ by Jaikin-Zapirain's appendix to \cite{JaikinZapirain2020THEUO}. Then, the $\ell^2$-Betti numbers of $G$ are defined and $b_p^{(2)}(G) = b_p^{\mathcal{D}_{\Q G}}(G)$. Hence, applying \cref{thm:agrarianMain} to the case $\mathbb{F} = \Q$ yields the following result stated in the introduction.


\begin{thm}\label{thm:b2rfrs}
Let $G$ be a virtually RFRS group of type $\mathtt{FP}_n(\Q)$. Then there is a finite index subgroup $H \leqslant G$ admitting a homomorphism onto $\Z$ with kernel of type $\mathtt{FP}_n(\Q)$ if and only if $b_p^{(2)}(G) = 0$ for $p = 0, \dots, n$.
\end{thm}


Thanks to Jaikin-Zapirain's work on rank functions in \cite{JaikinZapirain2020THEUO}, we can give a third characterisation of algebraic fibring. First, we set up some notation and terminology. If $R$ is a ring and $\varphi \colon R \longrightarrow \mathcal D$ is a division $R$-ring, then there is a natural rank function on matrices over $R$, which we denote $\rk_{\mathcal D, \varphi}$ or simply $\rk_{\mathcal D}$ when the map $\varphi$ is understood. If $\varphi$ is \textit{epic}, meaning that $\varphi(R)$ generates $\mathcal D$ as a division ring, then we call $\mathcal D$ \textit{universal} if the $\rk_\mathcal D \geqslant \rk_\mathcal E$ for every division $R$-ring $\psi \colon R \longrightarrow \mathcal E$. Note that if a universal division $R$-ring exists, then it is unique up to $R$-isomorphism by a result of Cohn \cite[Theorem 4.4.1]{cohn1995skew}. If, additionally, $\varphi$ is an injection, then we call $\mathcal D$ the \textit{universal division ring of fractions} for $R$.

\begin{thm}[Jaikin, {\cite[Corollary 1.3]{JaikinZapirain2020THEUO}}]  \label{thm:jaikinUniv}
    If $G$ is a residually-(locally indicable and amenable) group and $\mathbb F$ is a skew-field, then the Hughes-free division ring $\DF{G}$ exists and is the universal division ring of fractions for $\mathbb FG$.
\end{thm}

Following Jaikin-Zapirain, whenever $G$ has a Hughes-free division ring $\DF{G}$, we denote the rank function $\rk_{\DF{G}}$ by $\rk_{\mathbb FG}$. Note, in particular, that \cref{thm:jaikinUniv} holds for RFRS groups since they are residually poly-$\Z$ (see, e.g., \cite[Proposition 4.4]{JaikinZapirain2020THEUO}). Therefore, we have the following corollary, which will be useful to us below.

\begin{cor}\label{cor:rkIneq}
    Let $G$ be a RFRS group and let $\varphi \colon G \longrightarrow \Z$ be a homomorphism. Let $A$ be a matrix over $\mathbb FG$ and let $A^\Z$ be the matrix over $\mathbb F[\Z]$ obtained by applying $\varphi$ to $A$. Then $\rk_{\mathbb FG} A \geqslant \rk_{\mathbb F[\Z]} A^\Z$.
\end{cor}

\begin{proof}
    There is a map $\overline{\varphi} \colon \mathbb F G \longrightarrow \mathbb F[\Z] \longhookrightarrow \DF{[\Z]}$ induced by $\varphi$ and therefore $\rk_{\mathbb FG} A \geqslant \rk_{\DF{[\Z]}, \overline{\varphi}} A = \rk_{\mathbb F[\Z]} A^\Z$ by universality of $\DF{G}$. \qedhere
\end{proof}

The following theorem generalizes Corollary 1.5 of \cite{JaikinZapirain2020THEUO}, where the result is proven for $n = 1$. In the proof, if $M$ is a finitely generated $\mathbb F[\Z]$-module, then we define $\dim M := \dim_{\DF{[\Z]}} (M \otimes_{\mathbb F[\Z]} \DF{[\Z]})$. We will also write $\rk_G$ instead of $\rk_{\mathbb FG}$ to lighten the notation.

\begin{thm}\label{thm:finiteBetti}
    Let $\mathbb F$ be a skew-field and let $G$ be a virtually RFRS group of type $\FP_n(\mathbb F)$. The following are equivalent:
    \begin{enumerate}[label=(\arabic*)]
        \item\label{item:FP} there is a finite index subgroup $H_0 \leqslant G$ and an surjection $\varphi_0 \colon H_0 \longrightarrow \Z$ with $\ker \varphi_0$ of type $\FP_n(\mathbb F)$;
        \item\label{item:Betti} there is a finite index subgroup $H_1 \leqslant G$ and an surjection $\varphi_1 \colon H_1 \longrightarrow \Z$ with $b_p(\ker \varphi_1; \mathbb F) < \infty$ for $p = 0, \dots, n$.
    \end{enumerate}
\end{thm}

\begin{proof}
    If $G$ algebraically fibres with kernel $K$ of type $\FP_n(\mathbb F)$, then there is a free resolution $F_\bullet \longrightarrow \mathbb F \longrightarrow 0$ of the trivial $\mathbb F K$ module $\mathbb F$ such that $F_p$ is finitely generated for all $p \leqslant n$. This resolution can be used to compute the homology of $K$, and therefore $b_p(K; \mathbb F) < \infty$ for all $p \leqslant n$.
    
    In view of \cref{thm:agrarianMain}, to prove the converse it suffices to show that $b_p^{\DF{G}}(G) = 0$ for all $p \leqslant n$. By multiplicativity of $\DF{G}$-Betti numbers (\cref{lem:JBscales}), we may assume that $H_1 = G$. Let $K = \ker \varphi_1$ and write $\mathbb F[\Z]$ for the group algebra $\mathbb F[G/K]$. Moreover, note that $H_p(K;\mathbb F) \cong H_p(G; \mathbb F[\Z])$ for all $p$. Let
    \[
        \cdots \longrightarrow \mathbb F G^{d_p} \longrightarrow \cdots \longrightarrow \mathbb F G^{d_0} \longrightarrow \mathbb F \longrightarrow 0
    \]
    be a free resolution of the trivial $\mathbb F G$-module $\mathbb F$, where $d_p$ is some cardinal for each $p$ and $d_p$ is finite for each $p \leqslant n$, and we use the (non-standard) notation $\mathbb F G^{d_p}$ to denote the $d_p$-fold direct sum of $\mathbb F G$'s, as opposed to the $d_p$-fold direct product. The quotient map $G \longrightarrow \Z$ induces a chain map
    \[
        \begin{tikzcd}
        \cdots \arrow[r, "\partial_{n+2}"]    & \mathbb F G^{d_{n+1}} \arrow[r, "\partial_{n+1}"] \arrow[d] & \mathbb F G^{d_n} \arrow[r, "\partial_n"] \arrow[d] & \cdots \arrow[r, "\partial_1"]    & \mathbb F G^{d_0} \arrow[r, "\partial_0"] \arrow[d] & 0 \\
        \cdots \arrow[r, "\partial_{n+2}^\Z"] & {\mathbb F[\Z]^{d_{n+1}}} \arrow[r, "\partial_{n+1}^\Z"]        & \mathbb F[\Z]^{d_n} \arrow[r, "\partial_n^\Z"]    & \cdots \arrow[r, "\partial_1^\Z"] & {\mathbb F[\Z]^{d_0}} \arrow[r, "\partial_0^\Z"]    & 0\nospacepunct{,}
    \end{tikzcd}
    \]
    where the boundary maps are viewed as matrices and $\partial_p^\Z$ is obtained by applying the map $G \longrightarrow \Z$ to each entry of the matrix $\partial_p$. Note that the homology of the bottom chain complex is $H_\bullet(G;\Q[\Z])$.
    
    To apply Jaikin-Zapirain's results on rank functions, we will need the boundary maps to be between finitely generated free modules. However, $d_{n+1}$ is not finite in general, so we must modify the chain complexes as follows. Since $\mathbb F[\Z]$ is Noetherian and $\mathbb F[\Z]^{d_n}$ is finitely generated, $\im \partial_{n+1}^\Z$ is a finitely generated submodule of $\mathbb F[\Z]^{d_n}$. The preimage of a finite generating set of $\im \partial_{n+1}^\Z$ is contained in a finitely generated free summand $F$ of $\mathbb F[\Z]^{d_{n+1}}$. Notice that the homology of 
    \[    
        F \longrightarrow \mathbb F[\Z]^{d_n} \longrightarrow \cdots \longrightarrow \mathbb F[\Z]^{d_0} \longrightarrow 0
    \]
    is still $H_p(G; \mathbb F[\Z])$ for $p \leqslant n$. The preimage of $F$ in $\mathbb F G^{d_{n+1}}$ is again a finitely generated free summand $\widehat{F}$ of $\mathbb F G^{d_{n+1}}$. Note that it suffices to show that the homology of
    \[
        \DF{G} \otimes_{\Q G} \widehat{F} \longrightarrow \DF{G} \otimes_{\mathbb F G} \mathbb F G^{d_n} \longrightarrow \cdots \longrightarrow \DF{G} \otimes_{\mathbb F G} \mathbb F G^{d_0} \longrightarrow 0
    \]
    vanishes in degrees $\leqslant n$ to show that $b_p^{\DF{G}}(G) = 0$ for all $p \leqslant n$.

    We assume that $d_{n+1}$ is finite and that $F = \mathbb F[\Z]^{d_{n+1}}$ and $\widehat{F} = \mathbb F G^{d_{n+1}}$. Since, for every $p \leqslant n$, the homology $H_p(G;\mathbb F[\Z])$ is finite-dimensional as a $\mathbb F$-vector space, it must be torsion as a $\mathbb F[\Z]$-module. Therefore $\rk_\Z \partial_{p+1}^\Z = \dim \ker \partial_p^\Z$ for every $p \leqslant n$. Now, for each $p \leqslant n$, we have short exact sequences
    \[
        0 \longrightarrow \ker \partial_p^\Z \longrightarrow \mathbb F G^{d_p} \longrightarrow \im \partial_p^\Z \longrightarrow 0
    \]
    which implies that $d_p = \dim \ker \partial_p^\Z + \rk_\Z \partial_p^\Z = \rk_\Z \partial_{p+1}^\Z + \rk_\Z \partial_p^\Z$. Hence,
    \begin{align*}
        d_p - \rk_G \partial_p &= \dim_{\DF{G}} \ker(\DF{G}^{d_p} \xrightarrow{\partial_p} \DF{G}^{d_{p-1}} ) \\
        &\geqslant \rk_G \partial_{p+1} \\
        &\geqslant \rk_\Z \partial_{p+1}^\Z \\
        &= d_p - \rk_\Z \partial_p^\Z \\
        &\geqslant d_p - \rk_G \partial_p,
    \end{align*}
    where we have \cref{cor:rkIneq}. Thus, 
    \[
        \rk_G \partial_{p+1} = \dim_{\DF{G}} \ker(\DF{G}^{d_p} \xrightarrow{\partial_p} \DF{G}^{d_{p-1}} ),
    \]
    and therefore $b_p^{\DF{G}}(G) = 0$ for all $p \leqslant n$. \qedhere
\end{proof}


\begin{cor}\label{cor:charac}
    Let $G$ be a virtually RFRS group and let $n \in \N$.
    \begin{enumerate}
        \item If $\mathbb F$ and $\mathbb F'$ are skew-fields of the same characteristic, then $G$ virtually algebraically fibres with kernel of type $\FP_n(\mathbb F)$ if and only if it virtually algebraically fibres with kernel of type $\FP_n(\mathbb F')$.
        \item If $p$ is a prime such that $G$ algebraically fibres with kernel of type $\FP_n(\mathbb F_p)$, then it fibres with kernel of type $\FP_n(\Q)$.
    \end{enumerate}
\end{cor}

\begin{proof}
    (1) follows from the fact that the Betti numbers of a group with trivial skew-field coefficients depend only on the characteristic of the skew-field. (2) follows from the fact that $b_k(G; \mathbb F_p) \geqslant b_k(G; \Q)$ for any group $G$ and any prime $p$ (this is a consequence of the universal coefficient theorem). \qedhere
\end{proof}



%%%                     %%%
%%%     Application     %%%
%%%                     %%%
\section{Applications} \label{sec:app}

\subsection{Amenable RFRS groups}

\begin{defn}
A group $G$ is \textit{amenable} if for every continuous $G$-action on a compact, Hausdorff space $X$, there is a $G$-invariant probability measure on $X$.
\end{defn}


\begin{defn}[elementary amenable groups] \label{def:elemAm}
The class $\mathcal{E}$ of \textit{elementary amenable} groups is the smallest class such that 
\begin{enumerate}[label = {$\bullet$}]
    \item $\mathcal{E}$ contains all finite groups and all Abelian groups;
    \item if $G \in \mathcal{E}$ then the entire isomorphism class of $G$ is contained in $\mathcal{E}$;
    \item $\mathcal{E}$ is closed under taking subgroups, quotients, extensions, and directed unions.
\end{enumerate}
\end{defn}

All elementary amenable groups are amenable, however there are many examples of amenable groups that are not elementary amenable, the earliest being Grigorchuk's group of intermediate growth \cite{GrigorchukGroup}. For a discussion of more examples, we refer the reader to the introduction of Juschenko's paper \cite{JuschenckoNEA}. The known examples of non-elementary amenable groups all have infinite cohomological dimension over any field.  Moreover, elementary amenable groups of finite cohomological dimension over $\Z$ are virtually solvable by \cite[Lemma 2]{Hillman91} and \cite[Corollary 1]{HillmanLinnell}. We are led to the following question, which was stated in the introduction. 

\begin{q}
Are amenable groups of finite cohomological dimension over $\Z$ virtually solvable?
\end{q}

We obtain a partial answer in the positive direction as an application of \cref{thm:b2rfrs}.

\begin{thm}\label{thm:amRFRSelemAm}
\sloppy If $G$ is a virtually amenable RFRS group of type $\mathtt{FP}(\Q)$, then $G$ is polycyclic-by-finite.
\end{thm}
\begin{proof}
We assume $G$ is amenable, RFRS, and of type $\mathtt{FP}(\Q)$; the virtual claim then follows immediately. Since $G$ is amenable, $b_p^{(2)}(G) = 0$ for all $p$ by \cite[Theorem 7.2(1)]{Luck02}. By \cref{thm:b2rfrs} we obtain a finite index subgroup $H \leqslant G$ and an epimorphism $\varphi \colon H \longrightarrow \Z$ such that $N = \ker \varphi \in \mathtt{FP}_n(\Q)$. It will be necessary to require that $H$ be normal in $G$, which is not an issue since we can replace $H$ with its normal core. Since $\cd_{\Q} N \leqslant \cd_\Q G \leqslant n$, we also have $N \in \mathtt{FP}(\Q)$ \cite[VIII Proposition 6.1]{BrownGroupCohomology}. Because we have a short exact sequence
\[
    1 \longrightarrow N \longrightarrow H \longrightarrow \Z \longrightarrow 1
\]
with $N \in \mathtt{FP}(\Q)$, a theorem of Fel'dman \cite[Theorem 2.4]{Feldman71} (see also \cite[Proposition 2.5]{Bieri76}) gives $\cd_\Q N = \cd_\Q H - \cd_\Q \Z = n - 1$.

Since subgroups of amenable RFRS groups are amenable and RFRS, we can repeat the argument above with $N$ instead of $G$. Iterating this process as many times as necessary, we obtain a subnormal series
\[
    G_0 \leqslant G_1 \leqslant \cdots \leqslant G_{n-1} \leqslant G_n = G
\]
of $G$ such that $\cd_\Q G_j = j$ for each $j$ (note that $N = G_{n-1}$ here). But the only torsion-free group of cohomological dimension $0$ over $\Q$ is the trivial group, so we conclude that $G$ is polycyclic-by-finite.
\end{proof}


\begin{rem}
The assumption that $G$ be RFRS is necessary. For example, the Baumslag-Solitar group $\BS(1,n)$, with $n > 1$, is amenable and of finite type, but it is not polycyclic-by-finite.
\end{rem}

Baer's conjecture states that if $G$ is a group with a Noetherian group ring $\Z G$, then $G$ is a polycyclic-by-finite group. The author is grateful to Sam Hughes for pointing out the following consequence of \cref{thm:amRFRSelemAm}.

\begin{cor}\label{cor:baer}
Let $G$ be a virtually RFRS group of type $\mathtt{FP}(\Q)$. If $\Z G$ is Noetherian, then $G$ is polycyclic-by-finite.
\end{cor}
\begin{proof}
It is enough to prove the claim in the case that $G$ is RFRS. In this case, $\mathbb{Z} G$ embeds into the Linnell-skew field $\mathcal{D}(G)$, and therefore $\Z G$ is a domain. Since Noetherian domains are Ore domains \cite[p.~47]{McConnellRobsonNNR}, we have that $\Z G$ is an Ore domain. It then follows that $\Q G$ is an Ore domain, which implies that $G$ is amenable by Kielak's appendix to \cite{BartholdiKielakApp}. Then $G$ is polycyclic-by-finite by \cref{thm:amRFRSelemAm}.
\end{proof}







\subsection{Arithmetic lattices and subgroups of hyperbolic groups}

Recall that hyperbolic groups are always of type $\F_\infty$ (and of type $\F$ if they are torsion-free). However, their subgroups can exhibit many interesting finiteness properties. In \cite{RipsF1notF2_1982}, Rips gave the first example of an incoherent hyperbolic group; phrased in terms of finiteness properties, this gives an example of a hyperbolic group with a subgroup that is of type $\F_1$ but not $\F_2$. In \cite{BradyF2notF3_1999}, Brady constructed a type $\F_2$ subgroup of a hyperbolic group that is not of type $\F_3$; this provided the first example of a finitely presented non-hyperbolic subgroup of a hyperbolic group. In the same paper, Brady asked whether there are subgroups of hyperbolic groups that are of type $\F_n$ but not $\F_{n+1}$ for all $n$. More examples of $\F_2$-not-$\F_3$ subgroups of hyperbolic groups were provided by Kropholler \cite{KrophollerF2notF3} and Lodha \cite{LodhaF_2notF_3}, and in \cite{isenrich2021hyperbolic}, Llosa Isenrich, Martelli, and Py constructed the first examples of $\F_3$-not-$\F_4$ subgroups of hyperbolic groups. This result was extended in a subsequent paper \cite{IsenrichPy2022}, where Llosa Isenrich and Py completely answer Brady's question by exhibiting cocompact hyperbolic arithmetic lattices in $\PU(n,1)$ with $\F_{n-1}$-not-$\F_n$ subgroups for all $n$. We also mention the related work of Italiano, Martelli, and Migliorini, who constructed the first example of a non-hyperbolic type $\F$ subgroup of a hyperbolic group \cite{IMM_5mfld}.

In \cite{isenrich2021hyperbolic}, Llosa Isenrich, Martelli, and Py remark that one can use \cref{thm:b2rfrs} to show that there are subgroups of hyperbolic lattices in $\PO(2n,1)$ that are of type $\FP_{n-1}(\Q)$ but not $\FP_n(\Q)$. We record their argument here, and also mention that the same line of reasoning shows that there are many hyperbolic lattices in $\PO(2n+1,1)$ that virtually fibre with kernel of type $\FP(\Q)$.

\begin{prop}[{\cite[Proposition 19]{isenrich2021hyperbolic}}]\label{prop:lattices}
    Let $\Gamma < \PO(n,1)$ be a cocompact cubulable lattice. If $n = 2k$ is even then $\Gamma$ virtually fibres with kernel of type $\FP_{k-1}(\Q)$ but not $\FP_k(\Q)$. If $n$ is odd, then $\Gamma$ virtually fibres with kernel of type $\FP(\Q)$.
\end{prop}

\begin{rem}
    In \cite{BHW_2011}, Bergeron, Haglund, and Wise show that any standard arithmetic subgroup of $\PO(n,1)$ is cubulated, so \cref{prop:lattices} applies to a nonempty class of groups. We refer the reader to their paper for the definition of standard.
\end{rem}

\begin{proof}
    By Agol's theorem \cite[Theorem 1.1]{AgolHaken}, $\Gamma$ is virtually special and in particular virtually RFRS. By, for instance \cite[Theorem 3.3]{KammeyerLattices}, the $\ell^2$-Betti numbers of lattices in semi-simple Lie groups vanish except in the middle dimension, where they are nonzero. If $n = 2k$, then $\Gamma$ virtually fibres with kernel of type $\FP_{k-1}(\Q)$, but $b_k^{(2)}(\Gamma) \neq 0$ so the kernel cannot be of type $\FP_n(\Q)$ by \cref{thm:agrarianMain}. If $n$ is odd, then $\Gamma$ is $\ell^2$-acyclic and therefore virtually fibres with kernel of type $\FP(\Q)$ in this case, where we have also used \cref{cor:typeFP}. \qedhere
\end{proof}

\bibliographystyle{alpha}
\bibliography{Fisher}

\end{document}