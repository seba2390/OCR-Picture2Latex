%!TEX root = ../JpsiVsChMult.tex
 
Quarkonium states, such as the \jpsi meson, are prominent probes of the deconfined state of matter, the Quark-Gluon Plasma (QGP), formed in high-energy heavy-ion collisions \cite{Matsui:1986aa}. A suppression of \jpsi production in nucleus-nucleus (AA) collisions with respect to that in proton-proton (\pp) collisions has been observed by several experiments \cite{Alessandro:2005aa,Arnaldi:2007aa,Adare:2011aa,Abelev:2012ab,Adamczyk:2013tvk,Abelev:2014ad,Chatrchyan:2012aa,Aad:2011aa,Adam:2016rdg}. A remarkable feature is that, for \jpsi production at low transverse momentum (\pt) at the Large Hadron Collider (LHC), the suppression is significantly smaller than that at lower energies \cite{Abelev:2012ab,Abelev:2014ad,Adam:2016rdg}. The measurements of \jpsi production in proton (deuteron)-nucleus collisions, where the formation of the QGP is not expected, are essential to quantify effects (often denoted ``cold nuclear matter, CNM, effects''), present also in AA collisions but not associated to the QGP formation. At LHC energies, gluon shadowing/saturation is the most relevant effect which was expected to be quantified with measurements in \ppb collisions~\cite{Salgado:2012aa,Vogt:2013aa}. Furthermore, a novel effect, coherent energy loss in CNM (medium-induced gluon radiation), was proposed \cite{Arleo:2013aa}.

The measurements in d--Au collisions at the Relativistic Heavy Ion Collider (RHIC) have underlined the role of CNM effects in \jpsi production at $\snn=200$ GeV \cite{Adare:2008aa,Adare:2011ab,Adare:2013aa}. At the LHC, the first measurements of \jpsi production in minimum-bias \ppb collisions at $\snn=5.02$ TeV \cite{Aaij:2013zxa,Abelev:2014aa} showed that \jpsi production in \ppb collisions is suppressed at forward rapidity with respect to the expectation from a superposition of nucleon-nucleon collisions. The data have been further analysed to provide more differential measurements and discussed in comparison with several theoretical models \cite{Adam:2015ac}. A fair agreement is observed between data and models including nuclear shadowing \cite{Vogt:2013aa} or saturation \cite{Ma:2015sia,Ducloue:2015gfa}; also models including a contribution from coherent energy loss in CNM \cite{Arleo:2013aa} describe the data. These measurements are also relevant with respect to \jpsi production in \pb collisions at the LHC \cite{Abelev:2012ab,Abelev:2014ad}, currently understood to be strongly influenced by the presence of a deconfined medium. The measurements of $\Upsilon$ production in minimum-bias \ppb collisions at the LHC \cite{Aaij:2014ab,Abelev:2015ab} are also consistent with predictions based on CNM effects. Recent measurements of the $\psi$(2S) state in \ppb collisions have revealed a larger suppression than that measured for \jpsi production \cite{Abelev:2014ac,Aaij:2016eyl}. Such an observation was not expected from the available predictions based on CNM effects.

Concurrently, measurements of two-particle angular correlations in \ppb collisions at the LHC \cite{Chatrchyan:2013aa,Abelev:2013ad,Aad:2013aa,Abelev:2013ae,Chatrchyan:2013ab,Abelev:2014ae,Abelev:2015aa} revealed for high-multiplicity events features that, in \pb collisions, have been interpreted as a result of the collective expansion of a hot and dense medium. 
Furthermore, the identified particle \pt ~spectra \cite{Abelev:2013haa} show features akin to those in \pb collisions, where models  including collective flow, assuming local thermal equilibrium, agree with the data.

The measurement of \jpsi production as a function of centrality in \ppb collisions at the LHC \cite{Adam:2015jsa} showed that the nuclear effects depend on centrality.
$\Upsilon$ production has been studied as a function of charged-particle multiplicity in \pp ~and \ppb collisions by the CMS collaboration \cite{Chatrchyan:2014aa}. The yields of $\Upsilon$ mesons increase with multiplicity, while a decrease of the relative production of $\Upsilon(\mathrm{2S})$ and $\Upsilon(\mathrm{3S})$ with respect to $\Upsilon(\mathrm{1S})$ is observed.
The measurement of D-meson production as a function of event multiplicity in \ppb collisions \cite{Adam:2016mkz} exhibits features similar to those observed earlier in pp collisions, both for \jpsi \cite{Abelev:2012aa} and D-meson \cite{Adam:2015ota} production.

In this Letter measurements of the inclusive \jpsi yield and average transverse momentum as a function of charged-particle pseudorapidity density in \ppb collisions at $\snn=5.02$ TeV are presented. Performed in three ranges of rapidity for \pt ~$>0$ with the ALICE detector at the LHC, these measurements complement the studies of \jpsi and $\psi(2S)$ production as a function of the event centrality estimated from the energy deposited in the Zero Degree Calorimeters (ZDC) \cite{Adam:2015jsa,Abelev:ab}.
A measurement as a function of the charged-particle multiplicity does not require an interpretation of the event classes in terms of the collision geometry. 
Importantly, it enables the possibility to study rare events where collective-like effects may arise. 
The present data allow the investigation of events with very high multiplicities of charged particles, corresponding to less than 1\% of the hadronic cross section and establish as well a connection to the recent measurements of D-meson production as a function of event multiplicity \cite{Adam:2016mkz}. A measurement of the forward-to-backward \jpsi nuclear modification factor ratio is also presented.

