%!TEX root = ../JpsiVsChMult.tex
 For the \jpsi analysis at forward and backward rapidities, muon candidates are selected by requiring the reconstructed track in the muon chambers to match a track segment in the trigger chambers. Furthermore, the radial distance of the muon tracks with respect to the beam axis at the end of the front absorber is required to be between 17.6 and 89.5 cm. This criterion rejects tracks crossing the high-density part of the absorber, where the scattering and energy-loss effects are large. A selection on the muon pseudorapidity $-4  < \eta <  -2.5 $ is also applied to reject muons at the edges of the spectrometer's acceptance. 

When building the invariant mass distributions, each dimuon pair (of a given \pt ~and \y) is corrected by the detector acceptance times efficiency factor 1/(\acef(\pt,\y)). The  \acef(\pt,\y) map is obtained from a particle-gun Monte Carlo (MC) simulation based on GEANT~3~\cite{GEANT3} and simulating the detector response as in \cite{Abelev:2014aa}. Since  the \acef-factor does not depend on the multiplicity for the event multiplicities relevant for the \ppb analyses, the simulated events only contain a dimuon pair at the generator level.  The simulations assume an unpolarised \jpsi production. The same reconstruction procedure and selection cuts are applied to MC events and to real data.
The extraction of the \jpsi signal in the dimuon channel is performed via a fit to the \acef-corrected opposite-sign (OS) dimuon invariant mass distributions obtained for \pt ~$<$ 15 GeV/$c$. The fitting procedure is similar to that used in a previous \jpsi analysis in \ppb collisions \cite{Abelev:2014aa}. The distributions are fitted using a superposition of \jpsi and \psip~signals and a background shape. 
The resonances are parameterized using a Crystal Ball function with asymmetric tails while for the background a gaussian with its width linearly varying with mass is used. In the present analysis, the parameters of the non-gaussian tails of the resonance shape are determined from fits of the MC \jpsi signal, and fixed in the data fitting procedure. Examples of fits of the \acef-corrected dimuon invariant mass distributions for two selected bins, low and high multiplicities, are given in the left panel of Fig.~\ref{fig:minv}.

In the dielectron decay channel, electrons and positrons are reconstructed in the central barrel detectors by requiring a minimum of 70 out of maximally 159 track points in the TPC and a maximum value of 4 for the track fit $\chi^2$ over the number of track points. Furthermore, only tracks with at least two associated hits in the ITS, one of them in the innermost layer, are accepted.  This selection reduces the amount of electrons and positrons from photon conversions in the material of the detector beyond the first ITS layer. In addition a veto cut on topologically identified tracks from photon conversions is applied. The electron identification is achieved by the measurement of the energy deposition of the track in the TPC, which is required to be compatible with that expected for electrons within 3 standard deviations.  Tracks with specific energy loss being consistent with that of the pion or proton hypothesis within 3.5 standard deviations are rejected. These selection criteria are identical to those used in~\cite{Adam:2015jsa}. Electrons and positrons are selected in the pseudorapidity range $|\eta|< $ 0.9 and in the transverse momentum range \pt ~$>$ 1 GeV/$c$. 

The background in the OS invariant mass distribution is estimated with dielectron pairs formed with tracks from different events (mixed-event background). The background shape is normalised such that its integral over ranges of the invariant mass in the sidebands of the \jpsi~mass peak equals the number of measured OS dielectron pairs in the same ranges (typical ranges used are $[3.2,3.7]$ GeV/$c^2$ and $[2.0,2.5]$ GeV/$c^2$). The signal itself is extracted by counting the entries in the background-subtracted invariant mass distribution (the standard range used is $[2.92,3.16]$ GeV/$c^2$). Due to bremsstrahlung of the electron and positron in the detector material and radiative corrections of the decay vertex, the J/$\psi$ signal shape has a tail towards lower invariant masses.
 The standard range for the signal extraction contains, according to MC simulations, about 69\% of the J/$\psi$ signal. The number of reconstructed J/$\psi$ mesons and its statistical uncertainty are derived from the mean obtained when varying the counting window for the signal extraction and the invariant mass ranges used for the normalisation of the background. The variations that are taken into account are the same as in~\cite{Adam:2015jsa}. 
Examples of the dielectron invariant mass distributions in data, for two selected analysis bins at low and high multiplicities, are given in the right panel of Fig.~\ref{fig:minv}.

The correction for the acceptance and efficiency of the raw yields is based on simulated \ppb collisions with the HIJING event generator~\cite{Wang:1991hta} with an injected J/$\psi$ signal.  The dielectron decay is simulated with the EVTGEN package~\cite{Lange:2001uf} using PHOTOS~\cite{BARBERIO1991115,BARBERIO1994291} to describe the final state radiation. The production is assumed to be unpolarised as in the muon decay channel analysis. The propagation of the simulated particles is done by GEANT~3~\cite{GEANT3} and a full simulation of the detector response is performed. The same reconstruction procedure and selection cuts are applied to MC events and to real data.

 \begin{figure}[htb]
  {\centering 
\includegraphics[width=0.48\textwidth]{MinvPanel_mm}
\includegraphics[width=0.48\textwidth]{MinvPanel_ee}
\par}
\caption{\label{fig:minv}  Opposite-sign invariant mass distributions of selected muon (left panel, for the forward rapidity) and electron (right panel) pairs, for selected multiplicity bins. In the left panel, the distributions are corrected for \acef. The curves show the fit functions for signal, background and combined signal with background (see text for details). In the right panel, the background is evaluated with the event-mixing technique, and the overlaid signal is obtained from Monte Carlo (see text for details).}
\end{figure}

The inclusive \jpsi yield per event is obtained in each multiplicity bin as
$N_{\mathrm{J/\psi}} = {N_{\mathrm{J/\psi}}^{\mathrm{corr}}}/{N_{\mathrm{MB}}}$,
where $N_{\mathrm{J/\psi}}^{\mathrm{corr}}$ is the number of reconstructed \jpsi mesons corrected for the acceptance times efficiency factor. In the dimuon decay channel analysis, the number of MB events equivalent to the analysed dimuon sample ($N_{\mathrm{MB}}$) in each multiplicity bin is obtained from the number of dimuon triggers ($N_{\mathrm{DIMU}}$), through the normalisation factor of dimuon-triggered to MB-triggered events $F_{2\mu / \mathrm{MB}}$, as $N_{\mathrm{MB}} = F_{2\mu / \mathrm{MB}} \cdot N_{\mathrm{DIMU}}$. This factor is computed using two different methods, as discussed in \cite{Adam:2015jsa}. The \jpsi %integrated 
cross section values for minimum-bias events obtained in the dimuon channel at forward and backward rapidities, and in the dielectron channel at mid-rapidity are compatible with those presented in \cite{Abelev:2014aa} and \cite{Adam:2015ac}, respectively.
The results presented here are provided relative to the yield in NSD events, $\langle {\rm d}N_{J/\psi}/{\rm d}y\rangle$.
%This allows a comparison with other collision systems at different energies. 
The event-averaged yield is normalised to the NSD event class; the normalisation uncertainty is 3.1\% \cite{Abelev:2013ab}. 

 \begin{figure}[htb]
  {\centering 
\includegraphics[width=0.5\textwidth]{MPtPanel_mm}
\par}
\caption{\label{fig:mpt} Average transverse momentum of opposite-sign muon pairs as a function of the invariant mass at forward rapidity, for two multiplicity bins. The curves are fits of the background and combined signal and background (see text).}
\end{figure}

In previous analyses, e.g. \cite{Adam:2015ac}, the \jpsi yield was extracted in \pt ~bins, and the resulting distribution was fitted to extract the \mpt value. The present analysis aims at studying effects that may arise at high charged-particle multiplicities, where the usual method is no longer suitable due to statistical limitations. The method presented here does not require to sample the data in \pt ~bins, hence allowing the analysis in finer multiplicity bins. 
The extraction of the average transverse momentum of \jpsi mesons is done via a fit to the dimuon mean transverse momentum as a function of the invariant mass, $\average{p_{\rm{T}}^{\mu^{+}\mu^{-}}} (m_{\mu^{+}\mu^{-}})$.  A correction for the acceptance times efficiency has to be applied when building these distributions. Hence, the contribution of each dimuon pair with a certain \pt ~and $y$ in a given invariant mass bin is weighted with the two-dimensional \acef(\pt,\y). 
In order to extract the \jpsi \mpt, the \acef-corrected $\average{\pt^{\mu^{+}\mu^{-}}} (m_{\mu^{+} \mu^{-}})$ distributions, which are shown in Fig.~\ref{fig:mpt}, are fitted using the following functional shape:

\begin{eqnarray} \label{mptfitfcn}
\average{\pt^{\mu^{+} \mu^{-}}} (m_{\mu^{+} \mu^{-}}) &=& \alpha^{ J/\psi }(m_{\mu^{+} \mu^{-}}) \times \average{\pt^{J/\psi}}
\nonumber  \\
&+& \alpha^{ \psi(2S) }(m_{\mu^{+} \mu^{-}}) \times \average{\pt^{\psi(2S)}}
\nonumber  \\
&+& \left( 1 - \alpha^{J/\psi}(m_{\mu^{+} \mu^{-}}) -  \alpha^{\psi(2S)}(m_{\mu^{+} \mu^{-}}) \right) \times \average{\pt^{\mathrm{bkg}}}
\end{eqnarray}

where $\alpha(m_{\mu^{+} \mu^{-}}) = S(m_{\mu^{+} \mu^{-}}) / (S(m_{\mu^{+} \mu^{-}}) +B(m_{\mu^{+} \mu^{-}}))$; the signal ($S$) and background ($B$) dependence on the dimuon invariant mass is extracted from the corrected invariant mass spectrum fits mentioned above. The \jpsi and \psip ~average transverse momenta, $\average{\pt^{J/\psi}}$ and $\average{\pt^{\psi(2S)}}$, respectively, are fit parameters assumed to be independent of the invariant mass, while the background one, $\average{\pt^{\mathrm{bkg}}}$, is parameterized with a second order polynomial function. Note that, as for the yield extraction, the quantity $\average{\pt^{\psi(2S)}}$ is not a measurement of the $\psi(2\mathrm{S})$ mean transverse momentum, since the \acef ~is obtained only from \jpsi signals in the simulation. 
The \mpt ~results presented here for bins in multiplicity are relative to the value obtained for inclusive events, $\mpt_{\mathrm{MB}}$ \cite{Adam:2015ac}.
