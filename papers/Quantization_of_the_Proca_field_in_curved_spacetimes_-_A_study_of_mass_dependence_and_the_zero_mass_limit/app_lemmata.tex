\section{Additional Lemmata}\label{app:lemmata}
%
%
%
\begin{lemma}\label{lem:epsilon_contraction}
	Let $\N$ be a $N$ dimensional manifold with metric $k$. It holds for any $j \in {1,2,\dots,N-1}$
	\begin{align}
		\detk ^2 \; \epsilon\indices{^{\alpha_1\dots\alpha_j \alpha_{j+1} \dots \alpha_N }} \epsilon\indices{_{\alpha_1\dots\alpha_j \beta_{j+1} \dots \beta_N} }
		= (-1) ^s \, (N-j)! j! \; \delta\indices{^{[\alpha_{j+1}}_{\beta_{j+1}}}\,\delta\indices{^{\alpha_{j+2}}_{\beta_{j+2}}}\, \dots \, \delta\indices{^{\alpha_N ] }_{\beta_N}} \formspace \notag
	\end{align}
	where $s$ is the number of negative eigenvalues of the metric $k$.
\end{lemma}
\begin{proof}
The above formula is shown to hold in \cite[Equation (B.2.13)]{wald_GR}
\end{proof}
%
%
\begin{lemma} \label{lem:injective_hom_trivial_kernel}
	Let $\mathscr{A},\mathscr{B}$ be unital $^*$-algebras and let $\alpha : \mathscr{A} \to \mathscr{B}$ be a unit preserving $^*$-algebra homomorphism. Then:\\
	\begin{center}
		$\alpha$ is injective if and only if the kernel of $\alpha$ is trivial, that is, $\Ker{\alpha} = \{0\}$.
	\end{center}
\end{lemma}
%
\begin{proof}
	1.) Assume that $\alpha$ is injective, that is, for $a,\tilde{a} \in \mathscr{A}, \alpha(a) = \alpha(\tilde{a}) \implies a = \tilde{a}$.
	Then, because $\alpha$ is a homomorphism, it follows that $\alpha(0) = 0$.
	Let $g \in \Ker{\alpha}$. Then
	\begin{align}
		\alpha(g) &= 0 = \alpha(0) \notag \\
		\implies g &= 0 \quad \textrm{by injectivity} \notag \\
		\implies \Ker{\alpha} &= \{ 0 \} \formspace,
	\end{align}
	which completes the proof of the first direction. \par
2.) Assume that $\Ker{\alpha}=\{ 0 \}$.
	Let $a,\tilde{a} \in \mathscr{A},$ such that $\alpha(a) = \alpha(\tilde{a})$. Then
	\begin{align}
		\alpha(a) - \alpha(\tilde{a}) &= 0 \notag\\
		\implies \alpha(a - \tilde{a} ) &= 0 	\quad\textrm{because }\alpha \textrm{ is homomorphism} \notag \\
		\implies a - \tilde{a} &= 0 					\quad\textrm{because the kernel is trivial} \formspace,
	\end{align}
	therefore, $\alpha$ is injective. This completes the proof.
\end{proof}
%
\begin{lemma}\label{lem:injective_mor_simple_algebra}
	Let $\mathscr{A},\mathscr{B}$ be unital $^*$-algebras and let $\alpha : \mathscr{A} \to \mathscr{B}$ be a unit preserving $^*$-algebra homomorphism.
	Then: \\
	\begin{center}
		$\alpha$ is injective if $\mathscr{A}$ is simple.
	\end{center}
\end{lemma}
\begin{proof}
	The proof follows straight forward from Lemma \ref{lem:injective_hom_trivial_kernel}:
	First, note that for every homomorphism $\alpha : \mathscr{A} \to \mathscr{B}$, the kernel of $\alpha$ is an ideal in $\mathscr{A}$:\\
	Let $a \in \mathscr{A}$ arbitrary and $g \in \Ker{\alpha}$. Then
	\begin{align}
		\alpha(ag)
		&= \alpha(a) \alpha(g) \notag\\
		&= \alpha(a)\,0 \notag\\
		&= 0 \notag\\
		\implies (ag) &\in \Ker{\alpha}\formspace.
	\end{align}
	And an analogous result follows for $(ga)$ which shows that $\Ker{\alpha}$ is a two sided ideal in $\mathscr{A}$.\\
	Now, since $\mathscr{A}$ is simple, $\alpha$ has either full or trivial kernel, that is, $\Ker{\alpha} = \{0 \}$ or $\Ker{\alpha} = \mathscr{A}$. But since $\alpha(\mathbbm{1}) = \mathbbm{1} \neq 0$ it follows that $\Ker{\alpha} = \{0 \}$ and by Lemma \ref{lem:injective_hom_trivial_kernel} that $\alpha$ is injective.
\end{proof}
%
\begin{lemma}\label{lem:BU-algebra-barreled}
	Let $\mathfrak{X}$ a complex vector bundle over a smooth differential manifold. Then, the Borchers-Uhlmann algebra $\BU(\Gamma_0(\mathfrak{X}))$ is barreled.
\end{lemma}
\begin{proof}
	Let $\mathfrak{X}$ a complex vector bundle over a smooth differential manifold $\N$. Recall from chapter \ref{sec:BU-algebra} that we have defined the BU-algebra over $\Gamma_0(\mathfrak{X})$ as the tensor algebra $\BU(\Gamma_0(\mathfrak{X})) = \IC \oplus \bigoplus\limits_{n= 1}^\infty \Gamma_0(\mathfrak{X})^{\otimes n}$ and endowed it with the inductive limit topology of the spaces
\begin{align}
\BU_N = \IC \oplus \bigoplus\limits_{n= 1}^N \Gamma_0(\mathfrak{X})^{\otimes n} \formspace.
\end{align}
	 Equivalently, one can densely embed the tensor products $\Gamma_0(\mathfrak{X})^{\otimes n} \subset \Gamma_0( \boxtimes^n \mathfrak{X} )$, where $\boxtimes$ denotes the outer tensor product of vector bundles (see \cite[Chapter 3.3]{verch_sahlman}). The space of compactly supported section of a complex vector bundle is a LF-space, as it is defined as the inductive limit of the Frech\'et spaces of sections with support in some compact $K_l$ where $\left\{ K_l \right\}_l$ is a fundamental sequence of compact $K_l \subset \N$ (see \cite[17.2.2 and 17.3.1]{dieudonne_3}). Since LF-spaces are barreled \cite[Chapter 33, Corollary 3]{treves} and the direct sum of barreled spaces is again barreled \cite[18.11]{kelly-namioka}, we find for any $N \in \IN$ that $\BU_N$ is barreled. Additionally, the inductive limit of barreled spaces is barreled \cite[Chapter V, Proposition 6]{robertson}, hence the BU-algebra over smooth compactly supported sections $\Gamma_0(\mathfrak{X})$ over a complex vector bundle $\mathfrak{X}$ is barreled.
\end{proof}
%
%
%
\begin{lemma}[Symmetrization of fields] \label{lem:symmetrization-of-fields}
	Let $\left\{f_m\right\}_m \subset \BU\left(\Dzs \right)$ be a continuous family in the Borchers-Uhlmann algebra algebra of initial data $\Dzs$.
	Then
	\begin{align}
		f_m = f_{m,\text{sym}} + g_m \formspace,
	\end{align}
	where $\left\{f_{m,\text{sym}}\right\}_m$ is a continuous family of symmetric algebra elements in $\BU\left(\Dzs \right)$, and $\left\{ g_m\right\}_m$ is a continuous family in the ideal $\ICCR$ as defined in section \ref{sec:field-algebra-topology}.
\end{lemma}
\emph{Note:} An element $f_{\text{sym}} \in \BU(\Dzs)$ is symmetric, if it is symmetric in every degree.
\begin{proof}
First, one notes that it suffices to show the statement for any continuous family of \emph{homogeneous} elements of degree $N$, that is, $f_m = (0,0,\dots,0,f^{(N)}_m,0,0,\dots)$, for $f_m^{(N)} \in \left(\Dzs\right)^{\otimes N}$, since an arbitrary continuous family of elements in $\BU\left(\Dzs \right)$ can be written as a finite sum of continuous families of homogeneous elements. The statement then follows, because the sum of symmetric elements remains symmetric, and the ideal $\ICCR$ is a closed sub-*-algebra of $\BU\left(\Dzs \right)$ and hence by definition closed under addition. In the context of this proof, continuous families of elements in the BU-algebra are always meant to be continuous in $m$. 	\par
The main idea for this proof is then to write a (homogeneous) element as the sum of something symmetrized plus something containing commutations of indices. One can then use the commutator to identify an element of degree $N$ with an element of degree $N-2$ plus some terms that lie in the ideal $\ICCR$. Furthermore we need to show that all these operations are continuous in order to get the result for continuous families. To formulate this precisely, we need to introduce some notation.\par
Recall that we have defined test differential one-forms on the Cauchy surface $\Sigma$ as smooth sections of the cotangent bundle $\TsS$ with compact support, that is, $\Omega^1_0(\Sigma) = \Gamma_0(\TsS)$. We can therefore identify the space of initial data $\Dzs$ as
\begin{align}
	\Dzs &= \Omega^1_0(\Sigma) \oplus \Omega^1_0(\Sigma) \notag\\
	&= \Gamma_0(\TsS) \oplus \Gamma_0(\TsS) \notag\\
	&= \Gamma_0(\TsS \oplus \TsS) \formspace,
\end{align}
where $\TsS \oplus \TsS$ denotes the Whitney sum of vector bundles as defined in section \ref{sec:spacetime_geometry}. Furthermore, we can identify tensor products of initial data as
\begin{align}
	\Dzs \otimes \Dzs &= \Gamma_0\left( \TsS \oplus \TsS \right) \otimes \Gamma_0\left( \TsS \oplus \TsS \right) \notag\\
	&\subset \Gamma_0\big( (\TsS \oplus \TsS) \boxtimes (\TsS \oplus \TsS) \big) \formspace,
\end{align}
where the canonical embedding is dense (c.f. \cite[Chapter 3.3]{verch_sahlman}). The outer tensor product $\boxtimes$ of vector bundles is also defined in section \ref{sec:spacetime_geometry}. We may therefore identify an element in $\Dzs \otimes \Dzs$ with an element in $\Gamma_0\big( (\TsS \oplus \TsS) \boxtimes (\TsS \oplus \TsS) \big) $. We can furthermore re-write this in terms of components as
\begin{align}
\Gamma_0\big( &(\TsS \oplus \TsS) \boxtimes (\TsS \oplus \TsS) \big)\notag \\
  &= \underbrace{\Gamma_0\big( (\TsS \boxtimes \TsS) \big) }_{\eqqcolon \Omega_{11}} \oplus \underbrace{\Gamma_0\big( (\TsS \boxtimes \TsS) \big) }_{\eqqcolon \Omega_{12}}  \oplus \underbrace{\Gamma_0\big( (\TsS \boxtimes \TsS) \big)}_{\eqqcolon \Omega_{21}}    \oplus \underbrace{\Gamma_0\big( (\TsS \boxtimes \TsS) \big)   }_{\eqqcolon \Omega_{22}}  \formspace.
\end{align}
We define the corresponding continuous projectors
\begin{align}
	\text{pr}_{ij} : \Gamma_0\big( &(\TsS \oplus \TsS) \boxtimes (\TsS \oplus \TsS) \big) \to \Omega_{ij} \;,\quad i,j=1,2 \formspace.
\end{align}
As an example, let $\psi = (\varphi, \pi) \otimes (\varphi',\pi') \in \Gamma_0\big( (\TsS \oplus \TsS) \boxtimes (\TsS \oplus \TsS) \big)$, then $\text{pr}_{11}(\psi) = \varphi \otimes \varphi'$, $\text{pr}_{12}(\psi) = \varphi \otimes \pi'$, etcetera.\\
We generalize this to higher order tensor products of the space of initial data to obtain for any $N \in \IN$
\begin{align}
	\left(\Dzs\right)^{\otimes N}
	&\subset \Gamma_0\big( (\TsS \oplus \TsS)^{\boxtimes N} \big) \notag\\
	 &= \underbrace{\Gamma_0\big( (\TsS)^{\boxtimes N} \big) }_{\eqqcolon \Omega_{11\dots11}} \oplus \underbrace{\Gamma_0\big( (\TsS)^{\boxtimes N} \big) }_{\eqqcolon \Omega_{11\dots12}}  \oplus \cdots \oplus \underbrace{\Gamma_0\big( (\TsS)^{\boxtimes N} \big)   }_{\eqqcolon \Omega_{22\dots22}} \formspace,
\end{align}
where there are $2^N$ summands in the direct sum. By construction, $ (\TsS \oplus \TsS)^{\boxtimes N}$ is a vector bundle over $\Sigma^N$. Again, we define the continuous projectors $\text{pr}_{a_1 a_2 \dots a_N} : \Gamma_0\big( (\TsS \oplus \TsS)^{\boxtimes N} \big) \to \Omega_{a_1 a_2 \dots a_N}$, $a_i = 1, 2$ for $i=1,2,\dots,N$. To simplify notation for later use, we also define for any $k=1,\dots,N-1$, $i,j=1,2$,
\begin{align}
	\text{pr}_{k,ij} &:  \Gamma_0\big( (\TsS \oplus \TsS)^{\boxtimes N} \big) \to  \Gamma_0(\mathfrak{X}) \notag \\
	\text{pr}_{k,ij} &\coloneqq \bigoplus\limits_{\substack{a_i = 1,2 \\ i \neq k,k+1}} \text{pr}_{a_1 a_2 \dots a_{k-1} i j a_{k+2} \dots a_N}  \formspace,
\end{align}
where naturally, $\text{pr}_{k,ij}$ is a map into $\bigoplus^{2^{N-2}} \Gamma_0 (\TsS^{\boxtimes N})$ that we embed into $\Gamma_0(\mathfrak{X}) = \Gamma_0\big( (\TsS \oplus \TsS)^{\boxtimes (k-1)} \boxtimes \TsS \boxtimes \TsS \boxtimes (\TsS \oplus \TsS)^{\boxtimes(N-k-1)} \big)$. This notation makes clear that basically $\text{pr}_{k,ij}$ corresponds to a projection of the $k$-th and $(k+1)$-th $i$- respectively $j$-component of a tensor of degree $N$.\par
%
%
%
%
With this notation at our disposal we may continue the actual proof. We want to prove that, for any $N \in \IN$, a continuous family $\big\{ (0,0,\dots,f^{(N)}_m,0,0,\dots) \big\}_m$  of homogeneous elements can be decomposed into a sum of a continuous family $f_{m,\text{sym}}$ of symmetric elements and a continuous family in the ideal $\ICCR$. We will do this by induction in steps of $2$.\par
Base case:\par
1.) $N=0$ \\
Let $\big\{(f_m^{(0)} , 0 , 0 ,\dots)\big\}_m$ be a family of homogeneous elements of degree zero, $f_m^{(0)} \in \IC$. Trivially, the element is symmetric and by assumption continuous in $m$, and since $0 \in \ICCR$ by definition, the statement follows directly. \par
2.) $N=1$\\
Let $\big\{(0,f_m^{(1)} , 0 , 0 ,\dots)\big\}_m$ be a family of homogeneous elements of degree one, $f_m^{(1)} \in \Dzs$. Again, this is already symmetric and continuous in $m$, and the statement follows trivially. \par
We now proceed with the induction step.\\
The induction assumption is that for $N\in \IN$, every continuous family of homogeneous elements $(0,0,\dots,0,f_m^{(N)},0,0,\dots )$, where $f_m^{(N)} \in (\Dzs)^{\otimes N}$, can be written as a sum of a continuous family of fully symmetric elements in the field algebra and a continuous family of elements in the ideal. \par
%
%
Now let $f_m^{(N+2)} \in (\Dzs)^{\otimes(N+2)}$, which we identify with an element
\begin{align}
f_m^{(N+2)} (p_1,p_2,\dots,p_{N+2}) \in \Gamma_0\big( (\TsS \oplus \TsS)^{\boxtimes (N+2)} \big) \formspace,
\end{align}specify a continuous family of homogeneous elements $(0,0,\dots,0,f^{(N+2)}_m ,0,0,\dots)$ in the BU-algebra of initial data. In order to re-write this into the desired form, we start by looking at the symmetrized element
\begin{align}
	f_{m,\text{sym}}^{(N+2)}(p_1,p_2,\dots,p_{N+2})
	&\coloneqq \frac{1}{(N+2)!} \sum\limits_\sigma f_m^{(N+2)} (p_{\sigma(1)},p_{\sigma(2)},\dots,p_{\sigma(N+2)}) \formspace,
\end{align}
where the sum is taken all permutations $\sigma$ of $\left\{ 1,2,\dots,N+2\right\}$. Introducing the permutation operator $P_\sigma : \Gamma_0\big( (\TsS \oplus \TsS)^{\boxtimes (N+2)} \big)  \to \Gamma_0\big( (\TsS \oplus \TsS)^{\boxtimes (N+2)} \big) $, such that $(P_\sigma \psi)(p_1,p_2,\dots,p_{N+2}) = \psi (p_{\sigma(1)},p_{\sigma(2)},\dots,p_{\sigma(N+2)}) $, we may write this in the short hand notation as
\begin{align}
f_{m,\text{sym}}^{(N+2)}
&=  \frac{1}{(N+2)!} \sum\limits_\sigma P_\sigma f_m^{(N+2)} \notag\\
&= f_{m}^{(N+2)} +\frac{1}{(N+2)!} \sum\limits_\sigma  (P_\sigma - 1) f_m^{(N+2)} \formspace,
\end{align}
and therefore
\begin{align}\label{eqn:symmetrization_of_field}
f_{m}^{(N+2)} = f_{m,\text{sym}}^{(N+2)} - \frac{1}{(N+2)!} \sum\limits_\sigma  (P_\sigma - 1) f_m^{(N+2)}  \formspace.
\end{align}
Since the topology of $\Gamma_0\big( (\TsS \oplus \TsS)^{\boxtimes (N+2)} \big) $ is invariant under the swapping of variables, we find that if $f_m^{(N+2)} (p_1,p_2,\dots,p_{N+2})$ is continuous in $m$, then also the family associated with $f_m^{(N+2)} (p_{\sigma(1)},p_{\sigma(2)},\dots,p_{\sigma(N+2)})$ is continuous in $m$ for any permutation $\sigma$. Since taking sums is continuous, we conclude that $f_{m,\text{sym}}^{(N+2)}$ as defined above is continuous in $m$ as well as $\sum_\sigma  (P_\sigma - 1) f_m^{(N+2)}$. \par
So, we have successfully decomposed $(0,0,\dots,0,f_m^{(N+2)},0,0,\dots)$ into a sum of a continuous family of symmetric elements, and a continuous family of permutations. We have now left to show that this residual second term of permutations
\begin{align}\label{eqn:permuted_homogenous_element}
	\big(0,0,\dots,0,\sum_\sigma  (P_\sigma - 1) f_m^{(N+2)},0,0,\dots \big)
\end{align}
can be decomposed into the desired form.\par
We note that every permutation $\sigma$ can be written as the composition of transpositions of neighbor indices $\tau_i$, $i=1,2,\dots,l$, for some $l \in \IN$. For example. $\tau(1,2,3,4,5) = (1,2,4,3,5)$. Decomposing $\sigma = \tau_1 \comp \tau_2 \comp \cdots \comp \tau_l$, we naturally find $P_\sigma = P_{\tau_1}\cdot P_{\tau_2}\cdots P_{\tau_l}$. We can therefore write, by expanding the expression into a telescoping series,
\begin{align}\label{eqn:commuted-elements}
(P_\sigma - 1) f_m^{(N+2)} = \sum_{i=1}^l (P_{\tau_i} - 1)\,P_{\tau_{i+1}}\cdots P_{\tau_{l}}\,   f_m^{(N+2)} \formspace.
\end{align}
This is now a sum over differences of elements where only two indices are swapped, for example a difference like
\begin{align}
\psi_m^{(N+2)}(p_1,p_2,\dots,p_i,p_{i+1},\dots,p_{N+2}) - \psi_m^{(N+2)}(p_1,p_2,\dots,p_{i+1},p_{i},\dots,p_{N+2}) \formspace,
\end{align}
where $\psi_m^{(N+2)} = P_{\tau_{i+1}}\cdots P_{\tau_{l}}\,   f_m^{(N+2)}$ for some transpositions $\tau_i$.
We now want to use a generalized notion of the commutator to reduce this to something of degree $N$.\par
We define the map $G^{(N)}_k : \Gamma_0\big( (\TsS \oplus \TsS)^{\boxtimes (N+2)} \big) \to \Gamma_0\big( (\TsS \oplus \TsS)^{\boxtimes N} \big)$, for all $k=1,2,\dots,N-1$, by
\begin{align}
G^{(N)}_k (\psi) =& \int\limits_\Sigma h^{ij}(p) \left( \text{pr}_{k,21} \psi - \text{pr}_{k,12} \psi  \right)_{b_1 b_2 \dots b_{k-1} i j b_{k+2}\dotsb_{N+2}}(\dots,p,p,\dots)\,\dvolh(p) \,\notag \\
&\cdot \; dx^{b_1} \otimes \cdots \otimes  dx^{b_{k-1}}\otimes  dx^{b_{k+2}} \otimes \cdots \otimes  dx^{b_{N+2}}   \formspace,
\end{align}
where for the point $(\dots,p,p,\dots) \in \Sigma^{(N+2)}$ the $p$'s are put in the $k$-th and $(k+1)$-th entry. By definition of the projectors the map $G^{(N)}_k (\psi)$ applied gives an element in $\Gamma_0\big( (\TsS \oplus \TsS)^{\boxtimes N} \big)$. By construction, if $f_m^{(N+2)}$ specifies a continuous family of algebra elements, so does $G_k^{(N)}\left( f_m^{(N+2)} \right)$: Continuity in the space of smooth test one-forms means continuity in all orders of derivatives, using a partition of unity of the compact support and local charts. In particular this yields continuity in the components itself. Since the projectors are continuous, $G^{(N)}_k$ acts as a distribution and is therefore continuous .Using this generalized propagator, we find for an arbitrary $\tilde{f}_m^{(N+2)} \in (\Dzs)^{\otimes(N+2)}$ and an arbitrary $\tau$ that swaps the $k$-th and $(k+1)$-th component of $\psi^{(N+2)}$
\begin{align}\label{eqn:inserting-commutator-zero}
	\big( 0,0,\dots,0, (P_\tau - 1)& \tilde{f}_m^{(N+2)},0,0,\dots \big) \notag \\
&= \left(0,0,\dots,0,-\i G^{(N)}_k\big( \tilde{f}_m^{(N+2)}\big)  , 0, (P_\tau - 1)\tilde{f}_m^{(N+2)} ,0,0,\dots\right) \notag \\
&+ \left( 0,0,\dots,0,\i G^{(N)}_k\big( \tilde{f}_m^{(N+2)}\big)  , 0, 0,\dots \right) \formspace,
\end{align}
where the second term is naturally a homogeneous element of degree $N$ that is continuous in $m$.\\
The first term, also continuous in $m$, may be explicitly worked out by choosing a pure $\psi^{(N+2)} = \psi_{(1)} \otimes \psi_{(2)} \otimes \cdots \otimes \psi_{(N+2)}$, where $\psi_{(i)} = (\varphi_{(i)}, \pi_{(i)} )\in \Dzs$ for $i=1,\dots,N+2$.  By construction, we find
\begin{align}
	G^{(N)}_k &\big( \psi^{(N+2)} \big)  \\
	&= \Big( {\langle \pi_{(k)} , \varphi_{(k+1)} \rangle_\Sigma - \langle \varphi_{(k)} , \pi_{(k+1)} \rangle_\Sigma}\Big) \cdot  \psi_{(1)} \otimes \cdots \otimes \psi_{(k-1)} \otimes \psi_{(k+2)} \otimes \cdots \otimes \psi_{(N+2)} \formspace, \notag
\end{align}
and hence we obtain the product
\begin{align} \label{eqn:big_ideal_element}
\big(0,0,\dots,0,-\i  G^{(N)}_k&\big( \psi^{(N+2)}\big)  , 0, (P_\tau - 1) \psi^{(N+2)} ,0,0,\dots\big) \notag \\
	&=\left(0,0,\dots,0,  \psi_{(1)} \otimes \cdots \otimes \psi_{(k-1)} ,0,0,\dots \right) \notag \\
	&\quad \cdot  \big( -\i G(\psi_{(k)}  , \psi_{(k+1)})  ,  0, \psi_{(k)} \otimes  \psi_{(k+1)} - \psi_{(k+1)} \otimes \psi_{(k)},0,0,\dots   \big)   \notag \\
	&\quad\cdot \left(0,0,\dots,0, \psi_{(k+2)} \otimes \cdots \otimes \psi_{(N+2)} ,0,0,\dots \right) \formspace,
\end{align}
where we have introduced $G(\psi_{(k)}  , \psi_{(k+1)}) = \langle \pi_{(k)} , \varphi_{(k+1)} \rangle_\Sigma - \langle \varphi_{(k)} , \pi_{(k+1)} \rangle_\Sigma$ as a shorthand notation in analogy to the classical propagator $\mathcal G (\cdot,\cdot)$.
Now the construction ``pays off'' and we find that element in the middle of the above product,
\begin{align}
&\big( -\i G(\psi_{(k)}  , \psi_{(k+1)})   ,  0, \psi_{(k)} \otimes  \psi_{(k+1)} - \psi_{(k+1)} \otimes \psi_{(k)},0,0,\dots   \big)   \notag \\
&=  \big( -\i ( \langle \pi_{(k)} , \varphi_{(k+1)} \rangle_\Sigma - \langle \varphi_{(k)} , \pi_{(k+1)} \rangle_\Sigma )  ,  0, \notag \\
&\quad\quad (\varphi_{(k)},\pi_{(k)}) \otimes  (\varphi_{(k+1)},\pi_{(k+1)}) - (\varphi_{(k+1)},\pi_{(k+1)}) \otimes (\varphi_{(k)},\pi_{(k)}),0,0,\dots   \big)  \formspace,
\end{align}
is clearly an element of $\ICCR$ by definition. Hence, using equation (\ref{eqn:big_ideal_element}), we find that $\big(0,0,\dots,0,-\i  G^{(N)}_k\big( \psi^{(N+2)}\big)  , 0, (P_\tau - 1)\psi^{(N+2)} ,0,0,\dots\big)$ is an element of the ideal $\ICCR$. Since an arbitrary element of $(\Dzs)^{\otimes(N+2)}$ can be written as a sum of simple tensor product elements $\psi^{(N+2)}$ and using again that $\ICCR$ is closed under addition, we find that
\begin{align}
\left(0,0,\dots,0,-\i G^{(N)}_k\big( \tilde{f}_m^{(N+2)}\big)  , 0, (P_\tau - 1) \tilde{f}_m^{(N+2)} ,0,0,\dots\right)
\end{align}
is an element of the ideal $\ICCR$. \\
By equation (\ref{eqn:inserting-commutator-zero}) we find that the continuous family of homogeneous elements associated with $(P_\tau -1) \tilde{f}_m^{(N+2)}$ can be written as a sum of a continuous family of elements in the ideal $\ICCR$, and a continuous family of homogeneous elements of degree $N$. By equation (\ref{eqn:commuted-elements}) and again using that the ideal is closed under addition and that the sum of homogeneous elements of degree $N$ is again a homogeneous element of degree $N$, we find that the homogeneous element associated with $\sum_{\sigma} (P_\sigma -1) f^{(N+2)}_m$ can be written as the sum of an element in the ideal $\ICCR$ plus a homogeneous element of degree $N$ and overall using equation (\ref{eqn:symmetrization_of_field}), we find the result that
\begin{align}
	(0,0,\dots,0,f_m^{(N+2)},0,0,\dots) &= (0,0,\dots,0,f_{m,\text{sym}}^{(N+2)},0,0,\dots) \notag \\
																&\phantom{M}+ g_m + (0,0,\dots,0,f_m^{(N)},0,0,\dots) \formspace,
\end{align}
where all the summands give rise to families that are continuous in $m$, $g_m \in \ICCR$ and $f_m^{(N)} \in (\Dzs)^{\otimes N}$. We can now apply the induction hypothesis to the homogeneous element of degree $N$ and write it as a sum of a continuous family of symmetric elements, and a continuous family of elements in the ideal $\ICCR$. Finally, once again using that the sum of symmetric elements is still symmetric, that the ideal $\ICCR$ is closed under addition, and that the sum of continuous families is continuous, we obtain the desired result.
\end{proof}
%%
%
%
\begin{lemma}\label{lem:states_estimate}
Let $\W (E,\sigma)$ be the Weyl algebra generated over some real, pre-symplectic space $(E, \sigma)$. For all states $C \in \C(E,\sigma)$ it holds
\begin{align}
	\abs{C(F)} \leq 1 \formspace.
\end{align}
\end{lemma}
\begin{proof}
		Let $z \in \IC$ and $F \in E$. Define $A = \big( \mathbbm{1} + z W(F) \big) \in \W(E,\sigma)$. Then, for any state $C \in \C(E,\sigma)$ we find
		\begin{align}
			\omega_C (A^* A) = \Big(1+ z\bar{z} + 2 \Real{z\,C(F)}   \Big)\formspace.
		\end{align}
		We can now choose $z$ with $\abs{z} =1$, such that $z\,C(F) = - \abs{C(F)}$ and find the result
		\begin{align}
			0 & \leq 	\omega_C (A^* A)  = 2 -2\, \abs{C(F)} \notag\\
			\implies   \abs{C(F)}  & \leq 1 \formspace.
		\end{align}
		This completes the proof.
\end{proof}
