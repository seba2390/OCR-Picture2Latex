\section{Conclusion and Outlook} \label{chpt::conclusion}
In this thesis we have studied the Proca field in curved spacetimes, including external sources and without restrictive assumptions on the topology of the spacetime, in the classical and the quantum case. We have rigorously constructed classical solutions to the Proca equation by decomposing the equation into a wave equation and a Lorenz constraint. After solving the wave equation we implemented the constraint by restricting the initial data. Investigating the classical zero mass limit, we found that the limit exists if we implement a gauge invariance of the distributional solutions to Proca's equation by exact distributional one-forms. This gauge is independently argued to be the correct physical gauge by Sanders, Dappiaggi and Hack \cite{Sanders} as it accounts for phenomena such as the Aharonov-Bohm effect, as opposed to a gauge invariance by closed distributional one-forms. In the zero mass limit, we find the correct Maxwell dynamics if we make sure that the Lorenz constraint is well behaved in the limit. This naturally leads to conservation of current and a restriction of the initial data as found in the Maxwell case by Pfenning \cite{pfenning}.\par
Studying the quantum problem, we first constructed the generally covariant QFTCS in the framework of Brunetti, Fredenhagen and Verch and proved that the theory is local. Choosing the Borchers-Uhlmann algebra as the algebra of observables we rigorously constructed an initial data formulation of the quantum Proca theory in curved spacetimes. With this initial data formulation we were able to define a precise notion of continuity of the Proca fields with respect to the mass: Using specifically constructed BU-algebra homeomorphism, we were able to map a family of Proca fields at different masses, initially elements in different BU-algebras, into one topological space, the BU-algebra of initial data. In the BU-algebra of initial data there is a natural notion of continuity provided by the topology. Using this notion of continuity, we studied the zero mass limit and showed that the limit exists if and only if we restrict the class of test one-forms to those that are co-closed. Analogously to the classical case, this effectively implements a gauge invariance by exact distributional fields. In the zero mass limit, we find that the obtained fields fulfill the basic properties of linearity, the real field property and the correct CCR, that is the same as in the Maxwell case, but we do not find the expected Maxwell dynamics. Unlike in the classical case, this is not caused by ill behaved constraints, since in the quantum case the dynamics are implemented directly by dividing out appropriate dynamical ideals instead of first defining solutions to a wave equation and implement a Lorenz constraint. It is not clear how to obtain the Maxwell dynamics naturally in the zero mass limit. Within the presented framework, it might be possible to find natural additional assumptions, for example demanding continuity of the defined BU-algebra homeomorphisms with respect to the mass, to restore the dynamics. Other approaches, for example an investigation of states and the zero mass limit, are worth considering in future research projects and might lead to a full description of Maxwell's theory as a limit of Proca's theory. As we argue in Appendix \ref{app:weyl-algebra}, a C*-Weyl algebra approach is ill-suited for the investigation of the zero mass limit. A recent argument by Belokogne and Folacci \cite{stueckelberg_curvedST} states that the Proca theory is indeed unfit to study the zero mass limit in the quantum case and should be replaced by Stueckelberg electromagnetism. It is our hope that the presented construction can be adapted to Stueckelberg's theory, but as Stueckelberg's theory includes interaction with a scalar field, it is not clear whether this is possible. \par
Further possible application of the presented construction is the investigation of locality in the zero mass limit. Since Proca's theory is local, as opposed the Maxwell's \cite{Dappiaggi2012,Sanders}, one might gain insight into this issue with the presented initial data formulation and the zero mass limit.\par
Moreover, it is of crucial importance to this thesis that Assumption \ref{ass:propagator_continuity} holds. While this seems reasonable, it is suggested to investigate a proof of the assumption, for example by a use of energy estimates. Within the scope of this thesis a deeper investigation of the assumption was not possible and is left open for future projects.
