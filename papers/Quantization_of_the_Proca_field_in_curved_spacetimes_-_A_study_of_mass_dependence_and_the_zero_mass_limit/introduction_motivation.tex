\section{Introduction and Motivation}\label{chpt:introduction}
%
%
%
%
%
%
As of now there exist two very well tested theories describing two highly diverse realms of the vast landscape of physical phenomena: That is, on the one hand the theory of gravitation, called \emph{General Relativity} (GR), and on the other hand the \emph{Standard Model of Particle Physics}, describing the remaining three of the four known fundamental interactions, namely the electromagnetic, weak and strong interaction. \par
%
GR is a \emph{classical} field theory and describes gravitational large scale phenomena, as for example observed in astronomy, and provides our current understanding of the universe as an increasingly expanding one originating from a Big Bang. It was introduced by \name{Einstein} in the early twentieth century and has since been intensively tested, in its scope of application, and confirmed to be valid up to astonishing accuracy\footnote{Just this year, 2016, one of the last predictions of GR lacking experimental confirmation, gravitational waves, have been confirmed by the LIGO Scientific and Virgo Collaboration \cite{grav_waves_detection}.}. GR is a generalization of Einstein's theory of special relativity, which itself generalizes the principles of Newton's classical mechanics and was needed to account for the experimentally confirmed principle that the speed of light has the same constant value for all observers, even when moving relatively to each other. This counter intuitive fact changed the physical perception of space and time. In GR, gravitation is indirectly described via the curvature of spacetime, a four-dimensional space consisting of the observed three spatial dimensions together with one dimension describing time. According to GR, mass and energy, which are considered equivalent, curve the initially flat spacetime, similar to a rubber surface being deformed when putting masses on it. The connection between the curvature of spacetime and mass, or, more precisely, between the Einstein tensor and the stress-energy tensor, is described by Einstein's field equations.\par
%
The Standard Model on the other hand is a \emph{quantum} field theory (QFT) and unifies the description of electromagnetic interaction (quantum electrodynamics), weak interaction (quantum flavourdynamics) and strong interaction (quantum chromodynamics). It describes short scale and subatomic physical phenomena and has also been confirmed to very high accuracy. At very short scales, matter behaves very differently to what we are used to from our own perception of our surrounding world. In particular, experiments regarding the spectra of excited gases, the photoelectric effect and the so called Rutherford scattering led to a quantum description of matter, which includes a probabilistic behavior of observables.  In QFT, matter is described by quantum \emph{fields}, fulfilling non-trivial commutation relations that were abstracted from the earlier theory of Quantum Mechanics, which was also introduced in the early twentieth century. Under certain circumstances\footnote{For example in the case of free fields or the asymptotic ``in" and ``out" states of scattering processes.}, one can think of these fields as particles, called elementary particles. Even though the Standard Model really is a theory of fields rather than particles, one often uses the two words equivalently. In that sense, there are two classes of particles, the \emph{fermions} (for example electrons or neutrinos) with half integer spin, making up most of the matter around us, and the \emph{bosons} (for example photons or gluons) with integer spin. In the Standard Model, the gauge bosons are the transmitter of the field interactions, for example the photon (the ``light particle") is the transmitter of the electromagnetic interaction. The corresponding quantum field is a quantized version of Maxwell's electromagnetic field describing classical electromagnetism.\par
%
%
Even though both theories by themselves have been tremendously successful, it is a priori clear that neither of them describes all of the physical phenomena. While in most (terrestrial) microscopic scenarios gravity, being almost {32} orders of magnitude weaker then the weak interaction, can be neglected, it should play a role in extreme astronomical situations, for example near black holes or at the very early times in the beginning of the universe. Moreover, since matter is responsible for gravitation and is itself made up of elementary particles, there should exist a quantum description of gravitation. Also, there are observed phenomena that both of the theories cannot explain: investigating the rotation speeds of galaxies one finds that the observable mass in the universe cannot account for the measured speeds alone. It turns out that actually about one third of the gravitational matter is \emph{not observable}, that is, only interacting via gravitation and none of the other known interactions. This so called \emph{dark matter} is not described by the Standard Model.
Physicists have therefore tried to find a \emph{Theory of Everything} (ToE) for example by \emph{unifying} the Standard Model and the theory of gravitation into one theory describing all known interactions. So far, all attempts on formulating a quantum description of gravity and unifying it with the Standard Model have failed, ranging from early work by Kaluza \cite{kaluza1921}, Klein \cite{Klein1926} and Bronstein \cite{bronstein1936} to work in the 1960's and 1970's where it became clear that GR, as a QFT, is non-renormalizable \cite{tHooft1973,tHooft1974,Deser1974}, that is, simplifying, it yields unphysical infinite measurement results which cannot be brought under control.  There are also some alternative approaches, not based on QFT, to find a ToE, for example theoretical frameworks collected under the name \emph{string theories}, that seemed promising at first but failed to provide a consistent description of physical phenomena. Even though some important physicists, most prominently \name{Edward Witten}, claim that string theory, or the different parts of the underlying M-theory, is the correct theory to describe all observable physical phenomena, there is a lot of criticism against it. Many critics, prominent figures being \name{Lee Smolin} and \name{Peter Woit}, claim that, while string theory provides elegant and beautiful ideas about physics and mathematics, it lacks a clear description as a theory. It is said to provide only some fragmental descriptions and ideas and, most severely, lacks to be a physical theory a priori as it cannot be falsified: As there is an infinite number of possibilities to compactify the excrescent dimensions\footnote{The mathematical formalism of M-theory only works in {10} rather then the four dimensions that we observe.} and there is no preferred principle, string theory provides a description of \emph{all} possible physical theories and can always be adapted when in conflict with experiments and therefore cannot provide any insight or predictions at all.\par
%
%
Instead of constructing a ToE, one might therefore take a step back and try to approximately describe scenarios in which quantum matter is under the influence of gravitation, or find a quantum description of gravitation without unifying it with the other fundamental interactions. In doing so, one hopes to find and understand basic underlying principles that a ToE ought to have. In this thesis we will investigate quantum fields in curved spacetimes, that is, quantum fields under the influence of gravitation, and neglect the influence that the quantum fields themselves have on gravitation. Early investigations of quantum fields in curved spacetimes include the investigation of the influence of an expanding universe on quantum fields and its connection to particle creation by Parker \cite{Parker1969}, the study of radiating black holes, most successfully by Hawking in the 1970's \cite{Hawking1975}, and the description of what is now called the Unruh effect \cite{Unruh1976}. Investigating quantum fields on curved rather than flat spacetime as one does in the Standard Model, one is forced to rethink the underlying principles of QFT. In particular, QFT usually relies heavily on symmetries of the underlying spacetime, such as time translation and Lorentz invariance, implementing the special relativistic effects of quantum mechanics. Searching for Hilbert space representations of the canonical commutation relations (CCR) together with a unitary representation of the Lorentz group, one finds many (unitary) equivalent possibilities and picks out a convenient one specified by a vacuum state - the unique state that is Lorentz invariant.  In a general spacetime, such a global symmetry is of course not present. Hence, the different possibilities of the Hilbert space representation are not equivalent anymore and there is no preferred vacuum state. One therefore takes a different approach and formulates the theory purely algebraically, independent of any Hilbert space representation. This algebraic description of quantum field theory (originally on flat spacetime), AQFT, was studied and axiomatized by \name{Haag} and \name{Kastler} \cite{HaagKastler1964}. From the algebraic description one can construct the corresponding Hilbert space formulation via the so called GNS construction, named after Gelfand, Naimark and Segal. Dyson \cite{dyson1972} realized that the algebraic approach to QFT is suitable for a generalization to account for general covariance. Together with a generalization of the spectrum condition, known as the microlocal spectrum condition \cite{Brunetti1996}, the framework has then been further refined, leading to a categorical formulation of Quantum Field Theory on Curved Spacetimes (QFTCS) by \name{Brunetti, Fredenhagen} and \name{Verch} \cite{Brunetti_Fredenhagen_Verch}. Details on the principles and development of QFTCS can be found in the literature \cite{wald_QFT,baer_ginoux_pfaeffle,wald_hollands_review}.\par
%
In this thesis we investigate the Proca field in curved spacetimes. The Proca field is a massive vector\footnote{That is, it has spin one.} field first studied by \name{Proca} \cite{proca_original} as the most straightforward massive generalization of the electromagnetic field. Since the photon associated with the electromagnetic field has mass zero, Proca's theory is also called \emph{massive electrodynamics}. On a classical level, it can be used experimentally to find a lower bound of the photon mass. Assuming Proca's equation to describe electromagnetism, one finds that the corresponding electric potential is of Yukawa rather then Coulomb type as it is for a massless photon. Experimentally, one finds at very high accuracy at many orders of length-scales that the electric potential is indeed of Coulomb and not of Yukawa type. With these and other sophisticated methods the photon mass has been determined to be smaller then $4 \times 10^{-51}$\,kg (see \cite[Section I.2]{jackson})\footnote{More recent studies even suggest the bound to be lowered to $1.5 \times 10^{-54}$\,kg \cite{photon_mass}.}. Besides the photon there are several other elementary particles that are described by vector fields. In fact, all gauge bosons in the Standard Model are vector bosons. While the photon and the gluons\footnote{The gluons are the transmitters of the strong interaction.} are massless, the gauge bosons of the weak interaction, the W- and Z-bosons, are massive vector fields and may be described using Proca's equation\footnote{Of course, this is not the case in the Standard Model, as the gauge bosons are by construction massless and only appear massive by their interaction with the Higgs field.}. Further examples of massive vector fields include certain mesons, for example the $\omega$- or the $\varphi$-meson. It is thus desirable to study Proca's equation in a curved spacetime. This was first done by Furlani \cite{FURLANI} in the case of vanishing external sources and under a restrictive assumption on the topology of the spacetime\footnote{In particular, Furlani assumes the Cauchy surface of the spacetime to be compact.}. We are going to formulate the theory as general as possible, including external sources and without topological restrictions. More importantly, we are interested in the zero mass limit of the theory. As we shall see at many points, the massive (Proca-) and the massless (Maxwell-) theory differ enormously in detail. Most severely, the massless theory possesses a gauge invariance while the massive theory does not. While there have been several studies regarding the Maxwell field in curved spacetimes \cite{Sanders,pfenning,dimock1992quantizedEM,Dappiaggi2012}, there are questions regarding locality and the choice of gauge that remain open for discussion. In flat spacetimes, these questions do not arise as the topology is trivial, therefore, in particular, all closed $p$-forms are exact, as we will discuss later in more detail. In curved spacetimes, choosing the vector potential as the fundamental physical entity rather than the field strength tensor, it is a priori not clear if the gauge invariance by closed distributional one-forms is too general to account for all physical phenomena. As argued in \cite{Sanders}, implementing a gauge invariance by closed distributional one-forms rather than exact ones, one cannot capture experimentally established phenomena like the Aharonov-Bohm effect. Furthermore, one finds that the quantum Maxwell theory is not local, as opposed to the quantum Proca theory, and one might look for alternative implementations of locality in the theory. Recent proposals, with emphasis on the question of formulating the same physics in all spacetimes, are discussed in \cite{Fewster2012, Fewster2016}.
One reason to look at the massless limit of the Proca theory is to find answers to these questions naturally arising in the limiting procedure. In the zero mass limit, we indeed find a natural gauge invariance by exact rather than closed distributional one-forms. Questions concerning locality in the limit remain open for further investigations as they are not discussed in detail in this thesis. As a first step, our investigation will be purely based on observables, states are not included in the description. It should in principle be possible to extend the presented framework to include states. Throughout this thesis we work in natural units, that is, in particular we set $\text c = 1 = \hbar$.\par
%
%
%
The structure of this thesis is as follows:
In Chapter \ref{chpt:preliminiaries} we will recap some basic mathematical notations and definitions. Most of the discussion is kept rather brief as it is expected that the reader is familiar with the basics of differential geometry as it is the mathematical framework of GR. We will recap some notions regarding the spacetime geometry and vector bundles. In a bit more detail, we discuss differential forms as it is usually not part of the curriculum for physicists and the used formulation relies heavily on it. Furthermore, we give a brief overview of hyperbolic partial differential operators and their connection to global hyperbolicity. We conclude the first chapter by introducing some basics of category theory and a summary of the chosen sign conventions. \par
In Chapter \ref{chpt:classical} we will investigate the classical problem. We will find solutions to the classical Proca equation including external sources as a generalization of the work by Furlani \cite{FURLANI} by decomposing Proca's equation into a hyperbolic differential equation and a Lorenz constraint. We will then solve the hyperbolic equation and implement the constraint by restricting the initial data. As a foundation of understanding the quantum problem, we will investigate the classical zero mass limit. As it turns out, the existence of the zero mass limit in the quantum case is deeply connected to the classical one. \par
In Chapter \ref{chpt:quantum} we study the quantum problem. First, we will construct the generally covariant quantum Proca field theory in curved spacetimes in the categorical framework of Brunetti, Fredenhagen and Verch and show that the obtained theory is local, as opposed to the quantization of Maxwell's field (see \cite{Dappiaggi2012,Sanders}). Using the Borchers-Uhlmann algebra, we will define a appropriate notion of continuity of quantum fields with respect to the mass and will ultimately investigate the zero mass limit of the quantum Proca theory. It turns out that in the zero mass limit, the quantum fields do not solve Maxwell's equation in a distributional sense. We will discuss the reason from several perspectives and possible solutions. \par
A conclusion and outlook is presented in Chapter \ref{chpt::conclusion}. Our previous attempts that we formulated using a C*-algebraic approach to find a notion of continuity of the quantum Proca theory are presented in Appendix \ref{app:weyl-algebra}. We will discuss that this approach is not suited for the investigation of the zero mass limit but nevertheless present the results obtained along the way as they contain some mathematical results on continuous families of pre-symplectic spaces that have to our knowledge not been discussed in the literature. For clarity, some of the mathematical work needed along the investigation is put in Appendix \ref{app:lemmata} - despite their crucial importance for the results. Finally, a list of used symbols and references can be found at the very end of this thesis.
%
%
%
%
%
%
%
%
%
%
%
