\usepackage[utf8]{inputenc}
\usepackage[english]{babel}
\usepackage{amsmath}
\usepackage{amsfonts}
\usepackage{amssymb}
\usepackage{graphicx}
\usepackage[left=3.5cm,right=2cm,top=2.5cm,bottom=2.5cm]{geometry}

\usepackage[scr=rsfso, cal=cm]{mathalfa} %%use different standard font for mathscr that is not that much slanted.

%%%TEST
%\usepackage[complete*]{mtpro2}
%%%END TEST


\title{Quantization of the Proca field on curved spacetimes}
\author{Maximilian Schambach}
\date{\today}

\pdfinfo{%
  /Title    (Quantization of the Proca field on curved spacetimes)
  /Author   (Maximilian Schambach)
  /Creator  (Maximilian Schambach)
  /Producer (Maximilian Schambach)
  /Subject  ()
  /Keywords ()
}


%erweiterungen
\usepackage{titlesec,titletoc}
\usepackage{scrtime}
\usepackage{textcomp}
\usepackage{upgreek}
\usepackage{framed}
\usepackage{tabularx}
\usepackage{xcolor}
\substitutecolormodel{rgb}{cmyk} %%wandelt intern alles in cmyk
\usepackage{cases}

%\usepackage{siunitx}
\usepackage{enumitem} %%%for enumeration
%%set standard enumeration to roman i), ii) etc. and vertical separation of item list to 1ex:
\setlist[enumerate]{noitemsep,label=(\roman*),itemsep=1ex}


\usepackage{float} 	%to specify H in figure environment


\definecolor{linkblue}{cmyk}{1,.7,0,0}
\definecolor{linkred}{cmyk}{0,.92,.92,.27}
\definecolor{ownwhite}{cmyk}{1,1,1,1}



\usepackage{subfigure}
\usepackage{setspace} %für zeilenabstand
\usepackage{bbm} 	%für doppelstrich eins operator
\usepackage[normalem]{ulem}	%wellenuntersteicehen mit \uwave
\usepackage{marvosym} %for lightning symbol

\usepackage[all]{xy} %for drawing diagrams
\usepackage{tikz} %for tikz
%\setlength\parindent{0pt} %noindent for new paragraph
%\setlength{\parskip}{1em} %space after new paragraph
\usepackage{parskip}
\bigskipamount=10pt




%\usepackage[labeled,resetlabels]{multibib}	%%meherere literaturverzeichnisse

\usepackage{mathtools} %%mathtools

\usepackage{braket}  %%QM Braket schreibweise mit \braket{} oder \bra ...

\usepackage{bm}		%%fette zeichen im mathemodus

\usepackage{slashed}  %%feynman slash notation mit \slashed{...}

\usepackage{tensor}

%layout einstellungen

%equation nummerierungsformat
\renewcommand{\theequation}{\arabic{section}.\arabic{equation}}
\numberwithin{equation}{section}


%caption-formatierung

\numberwithin{figure}{section}	%abbildungszähler
\numberwithin{table}{section}	%tabellenzähler


\usepackage[margin=0pt,font=small,labelformat=simple,labelfont=bf,labelsep=space, justification=justified, hypcap]{caption}
\captionsetup[figure]{name=Figure}
\captionsetup[table]{name=Diagram}
%\captionsetup[table]{position=above}



%sidecap formatierung (ergänzung zur figure und table umgebung)
\usepackage[outercaption]{sidecap}


%header einstellungen
\usepackage{fancyhdr}
\pagestyle{fancy}
\fancyhead{}
\fancyhead[RO]{\slshape \nouppercase{\rightmark}}
\fancyhead[LE]{\slshape \nouppercase{\leftmark}}

\fancyfoot{}
\fancyfoot[CE,CO]{\thepage}




%Einstellung serifenlose überschriften
\titleformat*{\section}{\sffamily \bfseries \Large}
\titleformat*{\subsection}{\sffamily \bfseries \large}
\titleformat*{\subsubsection}{\sffamily \bfseries \normalsize}


%%%THEOREM ENVIREMENT
\usepackage{mdframed}
\usepackage[amsmath,thmmarks, framed]{ntheorem}
\usepackage{lipsum}
\usepackage{amssymb}


%%%tweak ntheorem to inlclude line breaks in title
\usepackage{etoolbox}% for command patching

\usepackage{lipsum}% for dummy text
%
%\usepackage[xindy,toc]{glossaries}

\usepackage[hidelinks,
linkcolor=linkblue,
citecolor=linkblue,
urlcolor=black,
pdftitle={Quantization of the Proca field on curved spacetimes},
pdfauthor={Maximilian Schambach}]{hyperref}


\usepackage{ellipsis} %%fixes bug for correct ellipsis typesetting with \dots










\makeatletter
\let\nobreakitem\item
\let\@nobreakitem\@item
\patchcmd{\nobreakitem}{\@item}{\@nobreakitem}{}{}
\patchcmd{\nobreakitem}{\@item}{\@nobreakitem}{}{}
\patchcmd{\@nobreakitem}{\@itempenalty}{\@M}{}{}
\patchcmd{\@xthm}{\ignorespaces}{\nobreak\ignorespaces}{}{}
\patchcmd{\@ythm}{\ignorespaces}{\nobreak\ignorespaces}{}{}


%%enable line break in theorem names
\renewtheoremstyle{break}%
{\item{\theorem@headerfont
		##1\ ##2\theorem@separator}\hskip\labelsep\relax\nobreakitem}%
{\item{\theorem@headerfont
		##1\ ##2\ (##3)\theorem@separator}\hskip\labelsep\relax\nobreakitem}
\makeatother
\theoremheaderfont{\kern-0cm\normalfont\bfseries}


%%%theorem styles for all theorems



\theoremframepreskip{0.6cm}
\theoremframepostskip{0.6cm}
\theorempreskip{0.6cm}
\theorempostskip{0.6cm}

%%define Theorem, Proposition, Corollary environoment, all those that have a box around it
\theoremstyle{break}
\theoremheaderfont{\normalfont \bfseries}
\theorembodyfont{\itshape}


\theoreminframepreskip{0.3cm}
\theoreminframepostskip{0.3cm}
\theoremindent 10pt      %somehow theorem has double indent, so to make it the same as for def and lem we need half
\theoremrightindent 8pt
\newframedtheorem{theorem}{Theorem}[section]
\newtheorem{proposition}[theorem]{Proposition}
\newtheorem{corollary}[theorem]{Corollary}
\AtBeginEnvironment {theorem}{
	\renewcommand*\FrameCommand{
		\fboxrule=\FrameRule \fboxsep=\FrameSep \fboxrule0.5mm \fcolorbox{linkblue}{white}
	}
}

\theoremindent 20pt
\theoremrightindent 16pt
%%%%DEFINITION
\theoreminframepreskip{0.4cm}
\theoreminframepostskip{0.4cm}
\usepackage{etoolbox}
\theorembodyfont{\normalfont}
\theoremheaderfont{\normalfont \bfseries}
\newtheorem{definition}[theorem]{Definition}
\newframedtheorem{lemma}[theorem]{Lemma}
\AtBeginEnvironment {lemma}{\renewcommand*\FrameCommand{{\color{linkblue}\vrule width 0.5mm }}}
%\AtBeginEnvironment {definition}{\hspace{50pt}}

\theorembodyfont{\normalfont}
\theoremheaderfont{\normalfont \bfseries}
\newframedtheorem{assumption}[theorem]{Assumption}
\AtBeginEnvironment {assumption}{\renewcommand*\FrameCommand{{\color{white}\vrule width 0.5mm \hspace{10pt}}}}


\theorembodyfont{\normalfont}
\theoremheaderfont{\normalfont \bfseries}
\newframedtheorem{conjecture}[theorem]{Conjecture}
\AtBeginEnvironment {conjecture}{\renewcommand*\FrameCommand{{\color{white}\vrule width 0.5mm \hspace{10pt}}}}


%%%%%PROOOF
\theoremindent 0pt
\theoremrightindent 0pt
\theorempreskip{0cm}
\theorempostskip{0.1cm}
\theoremstyle{nonumberplain}
\theoremheaderfont{\itshape}
\theorembodyfont{\normalfont}
\theoremsymbol{$\square$}
\theorempostwork{\hrule\bigskip}
\newtheorem{proof}{Proof:}[section]
\newtheorem{proof-idea}{Proof idea:}[section]


%\renewenvironment{proof}[1][Proof]{\begin{trivlist}
%\item[\hskip \labelsep {\textit{#1:}}]}{\end{trivlist}}
%\newenvironment{definition}[1][Definition]{\begin{trivlist}
%\item[\hskip \labelsep {\bfseries #1}]}{\end{trivlist}}
%\newenvironment{example}[1][Example]{\begin{trivlist}
%\item[\hskip \labelsep {\bfseries #1}]}{\end{trivlist}}
%\newenvironment{remark}[1][Remark]{\begin{trivlist}
%\item[\hskip \labelsep {\bfseries #1}]}{\end{trivlist}}

%\renewcommand{\proofSymbol}{\ensuremath{\square}}








%%EIGENDEFINITIONEN


\definecolor{grey}{rgb}{0.925,0.925,0.925}
\newcommand{\ueberschrift}[1]{{\sffamily\normalsize\bfseries {#1}}\\[2mm]}

\newcommand{\ergebnis}[2][0mm]{
\fcolorbox{black}{grey}{\parbox{\columnwidth}{\vspace{#1}
\begin{equation}
{#2}
\end{equation}
}}}


\DeclarePairedDelimiter\abs{\lvert}{\rvert}%
\DeclarePairedDelimiter\norm{\lVert}{\rVert}%

% Swap the definition of \abs* and \norm*, so that \abs
% and \norm resizes the size of the brackets, and the
% starred version does not.
\makeatletter
\let\oldabs\abs
\def\abs{\@ifstar{\oldabs}{\oldabs*}}
%
\let\oldnorm\norm
\def\norm{\@ifstar{\oldnorm}{\oldnorm*}}
\makeatother

\newcommand{\largewedge}{\mbox{\Large $\wedge$}}


\newcommand{\e}{\ensuremath{\mathrm{e}}}


%MENGEN
\newcommand{\IC}{\mathbb{C}} % komplexe
\newcommand{\IR}{\mathbb{R}} % reelle
\newcommand{\IQ}{\mathbb{Q}} % rationale
\newcommand{\IZ}{\mathbb{Z}} % ganze
\newcommand{\IN}{\mathbb{N}} % natuerliche

%OFT BENUTZTE GRößEN
\newcommand{\Z}{\ensuremath{\mathcal{Z}}}    %ZUSTANDSSUMME
\newcommand{\M}{\ensuremath{\mathcal{M}}}    %MANIFOLD M
\newcommand{\N}{\ensuremath{\mathcal{N}}}    %MANIFOLD N
\newcommand{\OO}{\ensuremath{\mathcal{O}}}    %MANIFOLD O
\newcommand{\F}{\ensuremath{\mathcal{F}}}    %Testfunction
\newcommand{\A}{\ensuremath{\mathcal{A}}}    %Algebra element A
\renewcommand{\AA}{\ensuremath{\mathscr{A}}}    %Algebra element A  (\AA was previously definad to be Angstrom symbol by default)
\newcommand{\W}{\ensuremath{\mathcal{W}}}    %Weyl algebra bundle
\newcommand{\Wm}{\ensuremath{\mathcal{W}_m}}    %Weyl algebra at m

\newcommand{\Wdyn}{\ensuremath{\mathcal{W}^\text{dyn}}}    %Weyl algebra bundle
\newcommand{\Wmdyn}{\ensuremath{\mathcal{W}^\text{dyn}_m}}    %Weyl algebra at m

\newcommand{\BUOmega}{\ensuremath{{\mathcal{BU}\big(\Omega^1_0(\M)\big)}}}    %Borchers Uhlmann algebra
\newcommand{\BUm}{\ensuremath{\mathcal{BU}_m}}    %Borchers Uhlmann algebra
\newcommand{\BUmz}{\ensuremath{\mathcal{BU}_{m,0}}}    %Borchers Uhlmann algebra
\newcommand{\BUmj}{\ensuremath{{\mathcal{BU}_{m,j}}}}    %Borchers Uhlmann algebra
\newcommand{\BUmjdyn}{\ensuremath{\mathcal{BU}_{m,j}^\text{dyn}}}    %Borchers Uhlmann algebra
\newcommand{\BUmzdyn}{\ensuremath{\mathcal{BU}_{m,0}^\text{dyn}}}    %Borchers Uhlmann algebra
\newcommand{\BUmzdynMaxw}{\ensuremath{\mathcal{BU}_{m,0}^\text{dyn, Maxw.}}}    %Borchers Uhlmann algebra
\newcommand{\BUmdyn}{\ensuremath{\mathcal{BU}_m^\text{dyn}}}    %Borchers Uhlmann algebra
\newcommand{\BU}{\ensuremath{\mathcal{BU}}}    %Borchers Uhlmann algebra

\newcommand{\K}{\ensuremath{\mathcal{K}}}    %field
\newcommand{\D}{\ensuremath{\mathcal{D}}}    %field
\newcommand{\Dzs}{\ensuremath{\mathcal{D}_{0}(\Sigma)}}    %space of initial data m noeq 0
\newcommand{\Dzzs}{\ensuremath{\mathcal{D}^\text{Maxw.}_{0}(\Sigma)}}    %space of initial data m=0
\newcommand{\Dzsp}{\ensuremath{\mathcal{D}_{0}(\Sigma')}}    %space of initial data prime

\newcommand{\PPM}{\ensuremath{\mathscr{P}_{\mathcal{M}}}}    %free algebra on M
\newcommand{\PPN}{\ensuremath{\mathscr{P}_{\mathcal{N}}}}    %free algebra on N


\newcommand{\Green}[2]{\ensuremath{\mathcal{G}_m(#1,#2)}}    %greens function
%\newcommand{\Gm}[2]{\ensuremath{\mathcal{G}_m(#1,#2)}}    %greens function
\newcommand{\GreenM}[2]{\ensuremath{\mathcal{G}_{m,\mathcal{M}}(#1,#2)}}    %greens function
\newcommand{\GreenN}[2]{\ensuremath{\mathcal{G}_{m,\mathcal{N}}(#1,#2)}}    %greens function
\newcommand{\Ker}[1]{\ensuremath{\mathrm{ker}{}(#1)}}    %Kernel of a function
\newcommand{\IMG}[1]{\ensuremath{\mathrm{img}{}(#1)}}    %Kernel of a function
\newcommand{\Span}[1]{\ensuremath{\textrm{span}\big\{#1\big\}}}   %Span of a set
\newcommand{\Real}[1]{\ensuremath{\textrm{Re}\big(#1\big)}}   %real part of a omplex number
\newcommand{\Imag}[1]{\ensuremath{\textrm{Im}\big(#1\big)}}   %real part of a omplex number

\newcommand{\C}{\ensuremath{\mathcal{C}}}    %greens function
\newcommand{\I}{\ensuremath{\mathcal{I}}}    %ideal
\newcommand{\J}{\ensuremath{\mathcal{J}}}    %ideal

\newcommand{\IMDYN}{\ensuremath{\mathcal{I}_m^\text{\,dyn}}}    %ideal
\newcommand{\IMJDYN}{\ensuremath{\mathcal{I}_{m,j}^\text{\,dyn}}}    %ideal
\newcommand{\IMZDYN}{\ensuremath{\mathcal{I}_{m,0}^\text{\,dyn}}}    %ideal
\newcommand{\IZJDYN}{\ensuremath{\mathcal{I}_{j}^\text{\,dyn, Maxwell}}}    %ideal
\newcommand{\ILORENZ}{\ensuremath{\mathcal{I}_{j}^\text{\,Lorenz}}}    %ideal
\newcommand{\IMCCR}{\ensuremath{\mathcal{I}_m^\text{\,CCR}}}    %ideal
\newcommand{\IMJCCR}{\ensuremath{\mathcal{I}_{m,j}^\text{\,CCR}}}    %ideal
\newcommand{\IMJZCCR}{\ensuremath{\mathcal{I}_{m,0}^\text{\,CCR}}}    %ideal
\newcommand{\ICCR}{\ensuremath{{\mathcal{I}}_{\sim}^\text{\,CCR}}}    %ideal
\newcommand{\ICCRS}{\ensuremath{{\mathcal{I}}_{\sim}^{\text{\,CCR},\Sigma}}}    %ideal
\newcommand{\ICCRSP}{\ensuremath{{\mathcal{I}}_{\sim}^{\text{\,CCR},\Sigma'}}}    %ideal
\newcommand{\IM}{\ensuremath{\mathcal{I}_m}}    %ideal
\newcommand{\IMJ}{\ensuremath{\mathcal{I}_{m,j}}}    %ideal
\newcommand{\IMZ}{\ensuremath{\mathcal{I}_{m,0}}}    %ideal
\newcommand{\JMDYN}{\ensuremath{\mathcal{J}_m^\text{dyn}}}    %ideal
\newcommand{\JM}{\ensuremath{\mathcal{J}_m}}    %ideal

\newcommand{\KERN}[1]{\ensuremath{\textrm{ker}{\left(#1\right)}}}   %real part of a omplex number
\newcommand{\SPAN}[1]{\ensuremath{\textrm{span}{\left\{#1\right\} }}}   %real part of a omplex number


%some short notation for cotangent bundles
\newcommand{\TsS}{\ensuremath{T^*\Sigma}}    %cotangent bundle of Sigma
\newcommand{\TsM}{\ensuremath{T^*\M}}    %cotangent bundle of manifold M
\newcommand{\TS}{\ensuremath{T\Sigma}}    %tangent bundle of Sigma
\newcommand{\TM}{\ensuremath{T\M}}    %tangent bundle of manifold M



\DeclareDocumentCommand\Gm{ m g }{%
	{\ensuremath{\mathcal{G}_m %
		\IfNoValueF {#2} {(#1, #2)}%
	}%
}
}

\DeclareDocumentCommand\Gmz{ m g }{%
	{\ensuremath{\mathcal{G}_{m_0} %
			\IfNoValueF {#2} {(#1, #2)}%
		}%
	}
}


\DeclareDocumentCommand\Em{ m g }{%
	{\ensuremath{\mathcal{E}_m %
			\IfNoValueF {#2} {(#1, #2)}%
		}%
	}
}

\DeclareDocumentCommand\Ez{ m g }{%
	{\ensuremath{\mathcal{E}_{0} %
			\IfNoValueF {#2} {(#1, #2)}%
		}%
	}
}




\usepackage{xfrac}
%\newcommand{\Quotient}[2]{\scalebox{1.3}{\ensuremath{\sfrac{#1}{#2}}}}    %greens function
\newcommand{\Quotient}[2]{\ensuremath{\sfrac{#1}{#2}}}    %greens function

\newcommand{\Quotientscale}[2]{\ensuremath{{\scalebox{1.2}{\Quotient{#1}{#2}}}}}    %greens function

%%%%%composition of functions:
\newcommand{\comp}{\mathbin{\mathchoice
  {\xcirc\scriptstyle}
  {\xcirc\scriptstyle}
  {\xcirc\scriptscriptstyle}
  {\xcirc\scriptscriptstyle}
}}
\newcommand{\xcirc}[1]{\vcenter{\hbox{$#1\circ$}}}




%%INITIAL DATA MAPPING OPERATORS
\newcommand{\rhoz}{\ensuremath{\rho_{(0)}}}    %rho 0
\newcommand{\rhon}{\ensuremath{\rho_{(n)}}}    %rho n
\newcommand{\rhod}{\ensuremath{\rho_{(d)}}}    %rho d
\newcommand{\rhodelta}{\ensuremath{\rho_{(\delta)}}}    %rho delta


\newcommand{\Az}{\ensuremath{{A_{(0)}}}}    %rho 0
\newcommand{\Azp}{\ensuremath{{A'_{(0)}}}}    %rho 0
\newcommand{\An}{\ensuremath{{A_{(n)}}}}    %rho n
\newcommand{\Ad}{\ensuremath{{A_{(d)}}}}    %rho d
\newcommand{\Adelta}{\ensuremath{{A_{(\delta)}}}}    %rho delta

\newcommand{\supp}[1]{\ensuremath{\text{\upshape{supp}}\left(#1\right)}}    %support
\newcommand{\dvolg}{d\textrm{vol}_g} %Volume element on M
\newcommand{\dvolh}{d\textrm{vol}_h} %Volume element on Sigma

\newcommand{\detg}{\sqrt{\abs{g_{\mu\nu}}}}
\newcommand{\deth}{\sqrt{\abs{h_{\mu\nu}}}}
\newcommand{\detk}{\sqrt{\abs{k_{\mu\nu}}}}
\newcommand{\dvolk}{d\textrm{vol}_k} %Volume element

%%NAMEN
\newcommand{\name}[1]{\text{#1}}    %%fließtext namen

\newcommand{\namek}[1]{{#1}}    %%kontext namen


%ABSTAND PUNKTUATION NACH FORMEL
\newcommand{\formspace}{\;}


%KONSTANTEN
\renewcommand{\i}{\ensuremath{\text{\upshape{i}}}}


%DIFFERENTIALE
%\newcommand{\D}[1]{\ensuremath{\mathcal{D}#1\;}}  %PFADINTEGRAL
\newcommand{\dd}[1]{\ensuremath{\mathrm{d}#1\;}}  %upright d


%KATEGORIES
\newcommand{\Spac}{\ensuremath{\mathsf{Spac}}}  %Spac
\newcommand{\SpacCurr}{\ensuremath{\mathsf{SpacCurr}}}  %SpacCurr
\newcommand{\Alg}{\ensuremath{\mathsf{Alg}}}  %ALG



%%FUNKTIONEN
\newcommand{\lagrangian}{\ensuremath{L}}    %%%LAGRANGEFUNKTION
\newcommand{\Ldens}{\ensuremath{\mathcal{L}}}    %Lagrangedichte

\newcommand{\hamiltonian}{\ensuremath{H}}   %%%HAMILTONFUNKTION
\newcommand{\diracdelta}{\ensuremath{\updelta}}
%\newcommand{\limes}{\ensuremath{\text{lim}}}

%%VEKTOREN UND MULTIINDIZES
\renewcommand{\vector}[1]{\ensuremath{\bm{#1}}}  %%vekotren kursiv
\newcommand{\multiindex}[1]{\ensuremath{\bm{#1}}}  %%multiindex


%%%RESTRICTION TO A SUBSET

\newcommand\restr[2]{{% we make the whole thing an ordinary symbol
  \left.\kern-\nulldelimiterspace % automatically resize the bar with \right
  #1 % the function
  \vphantom{\big|} % pretend it's a little taller at normal size
  \right|_{#2} % this is the delimiter
  }}



%\renewcommand{\restriction}{\mathord{\upharpoonright}}
%\newcommand\restr[2]{{% we make the whole thing an ordinary symbol
%  \left.\kern-\nulldelimiterspace % automatically resize the bar with \right
%  #1 % the function
% \vphantom{\big|} % pretend it's a little taller at normal size
%  \restriction_{#2} % this is the delimiter
%  \right.\kern-\nulldelimiterspace
%  }}




%% OVERLINE FOR ITALIC FONT
\newcommand{\olit}[2][3]{{}\mkern#1mu\overline{\mkern-#1mu#2}}
%\newcommand{\olit}{\bar}

%%besser für italic \psi mit: \skoverline{\psi}^{\hspace{0.75pt}i}

\newbox\usefulbox

\makeatletter
\def\getslant #1{\strip@pt\fontdimen1 #1}

\def\skoverline #1{\mathchoice
 {{\setbox\usefulbox=\hbox{$\m@th\displaystyle #1$}%
    \dimen@ \getslant\the\textfont\symletters \ht\usefulbox
    \divide\dimen@ \tw@
    \kern\dimen@
 \hspace{2pt}   \overline{\mkern -3mu \kern-\dimen@ \box\usefulbox\kern\dimen@ \mkern -1mu}\kern-\dimen@ }}
 {{\setbox\usefulbox=\hbox{$\m@th\textstyle #1$}%
    \dimen@ \getslant\the\textfont\symletters \ht\usefulbox
    \divide\dimen@ \tw@
    \kern\dimen@
 \hspace{2pt}  \overline{\mkern -3mu \kern-\dimen@ \box\usefulbox\kern\dimen@ \mkern -1mu}\kern-\dimen@ }}
 {{\setbox\usefulbox=\hbox{$\m@th\scriptstyle #1$}%
    \dimen@ \getslant\the\scriptfont\symletters \ht\usefulbox
    \divide\dimen@ \tw@
    \kern\dimen@
 \hspace{2pt} \overline{\mkern -3mu \kern-\dimen@ \box\usefulbox\kern\dimen@ \mkern -1mu}\kern-\dimen@ }}
 {{\setbox\usefulbox=\hbox{$\m@th\scriptscriptstyle #1$}%
    \dimen@ \getslant\the\scriptscriptfont\symletters \ht\usefulbox
    \divide\dimen@ \tw@
    \kern\dimen@
 \hspace{2pt}  \overline{\mkern -3mu \kern-\dimen@ \box\usefulbox\kern\dimen@ \mkern -1mu}\kern-\dimen@ }}%
 {}}
\makeatother
