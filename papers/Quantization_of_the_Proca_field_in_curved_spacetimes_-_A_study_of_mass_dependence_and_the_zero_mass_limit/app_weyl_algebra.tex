\section{The C*-Weyl Algebra as the Field Algebra}\label{app:weyl-algebra}
In this chapter we investigate the C*-Weyl algebra as the field algebra for the quantum Proca field theory.  The Weyl algebra is often used as the field algebra, both in algebraic quantum field theory in Minkowski space and in quantum field theory in curved spacetimes: Field operators obtained by GNS-construction (for details see e.g. \cite[Chapter III.]{haag} or \cite[Chapter III.14]{fragoulopoulou}) from the Weyl algebra are bounded, unlike the ones obtained from the field-algebra constructed from the free algebra. Since the product of such unbounded operators, which appear for example when calculating the commutator of field operators, are in general not well defined, it is useful to go over to bounded ones. Furthermore, the Weyl algebra is a normed algebra which makes it, at first glance, suitable for our purposes of finding a notion of continuity of the Proca field theory with respect to the mass. It turns out that similar investigations have been made in the literature:  In \cite{rieckers_honegger_deformation}, \name{Binz}, \name{Honegger} and \name{Rieckers} investigate the limit $\hbar \to 0$ for a family of Weyl algebras generated over an arbitrary pre-symplectic space, for which they introduce the notion of continuous fields of C*-algebras, which we will adapt to formulate a mass dependence of the Proca field theory.\par
%
In Section \ref{sec:continuous-families-of-symplectic-forms} we will generalize some mathematical results from \cite{rieckers_honegger_deformation} regarding continuous families of pre-symplectic forms and the corresponding C*-Weyl algebras. At this point, the dynamics of the theory are not implemented. This is done in a second step in Section \ref{sec:weyl-algebra-dynamics}. Unfortunately, it turns out that a notion of continuity with respect to the mass for the Weyl algebra formulation does not work in the wanted generality. This is why the use of the Weyl algebra is unsuited for the investigation of the zero mass limit and was discarded. Nevertheless, we present the results in this appendix to on the one hand illustrate why the ansatz is not suited for our problem and on the other hand to present the mathematical results in Section \ref{sec:continuous-families-of-symplectic-forms} that to our knowledge have not been discussed in the literature.
%
Most of the notation in this chapter is adapted from \cite{rieckers_honegger_deformation}.
%
%
%
%
%
\subsection{On C*-Weyl algebras and continuous family of pre-symplectic forms}\label{sec:continuous-families-of-symplectic-forms}
%
In this section we will generalize results from in \cite{rieckers_honegger_deformation} and \cite{rieckers_honegger_construction} regarding C*-Weyl algebras generated over a \emph{continuous family} of pre-symplectic spaces.
First, we briefly review the general construction of the C*-Weyl algebra over a pre-symplectic space. This field algebra will depend on the mass, even though we will only implement the commutation relations and not the dynamics yet. We would then like to find a notion of comparing these Weyl algebras at different masses with each other, using the notion of continuous fields of C*-algebras, similar to \cite{rieckers_honegger_deformation}. In our case, these algebras are generated over a \emph{continuous family} of pre-symplectic spaces, which makes it necessary to generalize the results obtained in \cite{rieckers_honegger_deformation} and \cite{rieckers_honegger_construction}. Just as in the discussion of the quantum problem using the BU-algebra, we need the Proca propagator to be continuous in the mass, that is, we need Assumption \ref{ass:propagator_continuity} to hold.
To start, we investigate the Weyl algebra generated over the real \emph{pre}-symplectic space\footnote{Unlike in the previous chapters, here we will view $\Omega^1_0(\M)$ as a real vector space.} $\big(\Omega^1_0(\M),\mathcal{G}_m\big)$ and, for simplicity, we write in this chapter $\gls{E} \equiv \Omega^1_0(\M)$ and keep in mind that, of course, there is an underlying manifold structure. 
Generalizing to a pre-symplectic space, rather than a symplectic one, does not have any effects on the construction (and uniqueness) of the C*-Weyl algebra of observables (see \cite{rieckers_honegger_construction}), but the constructed algebra will not be simple.
%
%
%
\begin{definition}[Weyl elements]
\label{def:weyl_elements}
For any $m >0$ and $F \in E$ we define the linearly independent \emph{Weyl-elements} \gls{WmF}, such that for all $F,F' \in E$ it holds
\begin{align}
\textrm{(i)} \quad&W_m(F)W_m(F') = \e^{-\i \Gm{F}{F'}/2 } W_m(F + F') \formspace,\\
\textrm{(ii)} \quad&W_m(F)^* = W_m(-F) \formspace.
\end{align}
\end{definition}
Condition (i) of the above definition is also known as the \emph{Weyl-form} of the CCR.
From these properties, it immediately follows that
\begin{align} 
W_m(0) W_m(F) 
&= W_m(F) W_m(0)\notag\\
&= \e^{-\i \Gm{F}{0}/2} W_m(F)\notag\\
&= W_m(F) \\  
\implies W_m(0) &= \mathbbm{1} \formspace,
\end{align}
from which it follows that
\begin{align}
\mathbbm{1}
&= W_m(0) \notag\\
&= W_m(F-F)\notag\\
&= W_m(F)^* W_m(F)\notag\\
&= W_m(F) W_m(F)^* \formspace,
\end{align}
that is, the Weyl elements are unitary.
Using these linearly independent Weyl elements, one can define the span of the Weyl elements:
\begin{align}
\gls{Wmtilde} \coloneqq \Span{W_m(F) : F \in E} \formspace.
\end{align}
Together with the above defined (twisted) product and $^*$-operation, $\widetilde{\mathcal{W}}_m$ becomes a unital $^*$-algebra, where the unit is given by $\mathbbm{1} = W_m(0)$.
Now, we would like to endow $\widetilde{\mathcal{W}}_m$ with a (unique) norm to define a C*-algebra as the norm closure of $\widetilde{\mathcal{W}}_m$. For this, we need the notion of a \emph{state} on $\widetilde{\mathcal{W}}_m$:
%
%fasf
%asfasfasf
\begin{definition}[States]\label{def:states}
	Define \gls{CEGm} as the convex set of normalized, projectively positive functions $C : E \to \IC$. That is, for $C \in \mathcal{C}(E,\Gm{})$ it holds by definition:
%
	 \begin{subequations} 
	 	\begin{align}
	 		& C(0) = 1 																															&\textrm{(normalization)} \\
	 		& \sum\limits_{i,j=1}^N \bar{z}_i z_j \,\e^{\i \Gm{F_i}{F_j}/2}\, C(F_j - F_i) \geq 0 		&\textrm{(positiveness)}	 		
	 	\end{align}
	 	for every $N \in \IN$, $z_i \in \IC$  and $F_i \in E$.
	 \end{subequations}
	 %
	 Spanning the convex set of these function means that for every linear combination of $C$'s the corresponding coefficients add up to one. This is to ensure that linear combinations of states are also normalized.
	 To each $C \in \mathcal{C}(E,\Gm{})$ we associate a unique positive linear functional $\gls{omegaC} : \widetilde{\mathcal{W}}_m \to \IC $ via
	 \begin{align}
	 \omega_C \big(W_m(F)\big) = C(F)
	 \end{align}
	 for all $F \in E$. By the properties of $C$ it then holds for every $A \in \widetilde{\mathcal{W}}_m$ :
	 	\begin{align}
	 	\quad & \omega_C(\mathbbm{1}) = 1 			\\																												
	 	\quad & \omega_C (A^* A) \geq 0 		\formspace.	 	
	 	\end{align}	 
\end{definition}
In fact, every state $\omega_C$ on $\widetilde{\mathcal{W}}_m $ corresponds to a \emph{unique} $C \in \mathcal{C}(E,\Gm{})$	(see for example \cite[Chapter 3]{rieckers_honegger_deformation}). With this notion of a state we are able to define a C*-norm on the algebra $\widetilde{\mathcal{W}}_m$:
%
%
\begin{definition}[Weyl Algebra]
	Let $A \in \widetilde{\mathcal{W}}_m$ and $\omega_C$ be a state corresponding to a $C \in \mathcal{C}(E,\Gm{})$. On the *-algebra $\widetilde{\mathcal{W}}_m$ we define the (unique) C*-norm $\| . \|_m : \widetilde{\mathcal{W}}_m \to \IR$ by
	\begin{align}
		\| A \|_m \coloneqq \sup\left\{\sqrt{\omega_C(A^* A)} : C \in \mathcal{C}(E,\Gm{})\right\} \formspace.
	\end{align}
	The $\|.\|_m$-closure of $\widetilde{\mathcal{W}}_m$ is called the \emph{Weyl algebra} and will be denoted by \gls{Wm}.
\end{definition}
\noindent Since we are interested in a change of the mass parameter $m$ we need to build a mathematical structure in which we are able to compare the Weyl algebras for different masses in a continuous way. This can be done in the bundle $\W = \bigcup\limits_m \Wm$ with the notion of a \emph{continuous field of C*-algebras}. 
\begin{definition}[Sections and continous fields of C*-algebras]\label{def:continous_field_algebra}
	The set of sections $K$ of the bundle $\W$ is defined as
	\begin{align}
		\prod_{m} \Wm = \left\{ K : \IR^+ \ni m \mapsto K(m) \in \Wm \right\} \formspace.
	\end{align}	
	A \emph{continuous field of C*-algebras} is a tuple $(\left\{ \Wm \right\}_m , \K)$ consisting of a family $\left\{ \Wm \right\}_m $ of C*-algebras and a sub *-algebra $\K$ of $\prod_{m} \Wm$ such that
	\begin{enumerate}
		\item The map $\IR^+ \ni m \mapsto \|K(m)\|_m$ is continuous for all $K \in \K$, that is, $K$ is a continuous section.
		\item For every $m \in \IR^+$ the set $\left\{ K(m) : K \in \K \right\}$ is dense in $\Wm$.
		\item Let $K \in \prod_{m} \Wm $. If for every $m_0 \in \IR^+$ and every $\epsilon > 0$ there exists a section $H \in \K$ and a neighborhood $U_0$ of $m_0$ such that for all $m \in U_0$ it holds $\| K(m) - H(m)\|_m < \epsilon$, then $K \in \K$.
	\end{enumerate}
\end{definition}
We will in the following sometimes denote the sections explicitly by $[m \mapsto K(m)] \in \prod_{m} \Wm$.
In order to construct the desired field of C*-algebras, the following lemma is essential. It allows us to construct a specific sub *-algebra of $\prod_{m} \Wm$ containing only continuous sections that are point-wise dense in $\Wm$, which then guarantees the existence of a continuous field of C*-algebras:
\begin{lemma}\label{lem:cont_field_existance}
	Let $\D$ be a sub *-algebra of $\prod_m \Wm$ such that the conditions (i) and (ii) of the above Definition \ref{def:continous_field_algebra} are fulfilled with $\K$ replaced by $\D$. 
	Then 
	\begin{center}
there exists a unique continuous field of C*-algebras $(\left\{ \Wm \right\}_m , \K)$\\ such that $\D \subseteq \K$. 
	\end{center}
	In fact, $\K$ contains only those $K \in \prod_m \Wm$ such that for every $m_0 \in \IR^+$ and every $\epsilon > 0$ there exists a section $H \in \D$ and a neighborhood $U_0$ of $m_0$ such that for all $m_0 \in U_0$ it holds $\| K(m) - H(m)\|_m < \epsilon$.
\end{lemma}
\begin{proof}
	The above definition and lemma in this form are due to \cite[Chapter 2]{rieckers_honegger_deformation}. The lemma follows from \cite[Proposition 10.2.3]{dixmier} (which gives the same statement for continuous fields of Banach spaces) and applying \cite[Proposition 10.3.2]{dixmier} with use of the subset $\D$.
\end{proof}
Now, the procedure is as follows: We will define a specific sub *-algebra of $\prod_m \Wm$ that contains the sections $[m \mapsto W_m(F)]$, that is, it contains all physically interesting observables, and we will then show that this subset fulfills the necessary conditions, such that the above Lemma \ref{lem:cont_field_existance} is applicable. The non trivial part is to show that all the contained sections are continuous. For that, we need a notion on how to compare the norm of a section at different masses $m$ with each other. More specifically, we need to know how to relate states $C_m \in \mathcal{C}(E, \Gm{} )$ to states $C_{m_0} \in \mathcal{C} (E, \Gmz{})$. This is done in the following lemma, which provides some new insight on the Weyl algebra bundle over vector spaces endowed with a \emph{continuous family} of pre-symplectic forms that has, to our knowledge, not yet been discussed in the literature:
\begin{lemma}[On continuous families of pre-symplectic forms]\label{lem:on_contiuous_families}
	Let $E$ be a $\IR$-vector space of arbitrary dimension (including infinite dimensional). Let $I \in \IR$ open be an index set. Let $\left\{ \sigma_i \right\}_{i \in I}$ be a continuous family of pre-symplectic forms (that is, for all $F,F' \in E$ the map $I \ni i \mapsto \sigma_i(F,F')$ is continuous). 
	Then the following holds:
	\begin{enumerate}
		\item $\rho \coloneqq \sigma_i + \sigma_j$, for some $i,j \in I$, defines a pre-symplectic form on $E$. \\If $C_i \in \mathcal{C}(E,\sigma_i)$ and $C_j \in \mathcal{C}(E,\sigma_j)$, then
		\begin{align}
			C = C_i C_j \in \mathcal{C}(E,\rho) \formspace.
		\end{align}
		\item If $\left\{s_i\right\}_{i \in I}$ is a family of symmetric, positive, $\IR$-bilinear forms on $E$, such that for all $i \in I$ and all $F,F' \in E$ it holds $\sigma_i(F,F')^2 \leq s_i(F,F) s_i(F',F')$, then for all $F \in E$ it holds
		\begin{align}
			[F \mapsto \e^{-s_i(F,F)/2}] \in \mathcal{C}(E,\sigma_i)\formspace.
		\end{align}
		Furthermore, if $E$ is finite dimensional, there exists a family $\left\{s_i\right\}_{i \in I}$ with the above properties and such that $I \ni i \mapsto s_i (F,F')$ is continuous for all $F,F' \in E$.
	\end{enumerate}
\end{lemma}
\begin{proof}
(i) The proof is given for fixed pairs of pre-symplectic forms $\sigma_i , \sigma_j$ in \cite[Theorem 2.3]{rieckers_honegger_partially_classical_states}. \par 
(ii) For a fixed $i \in \IN$, the statement is proven in \cite[Theorem 3.4]{petz}\footnote{Actually, the proof in this reference is given to hold for symplectic forms rather than pre-symplectic forms, but generalizes directly to the pre-symplectic case since the non-degeneracy property is not used neither within the proof directly nor in the lemma that is needed for the proof.}. Here, we want to show that, in the finite dimensional case, the family $\left\{ s_i \right\}$ exists and can be chosen to be continuous:
Let $E$ be of finite dimension $N \in \IN$, let $I \in \IR$, open, be an index set and let $\left\{\sigma_i \right\}_{i \in I}$ be a continuous family of pre-symplectic forms on $E$. In the following we will assume to have chosen a basis $\left\{ e_i \right\}_{i=1}^{N}$ of $E$, $\norm{e_i} = 1$,  which induces a scalar product by $\langle e_i , e_j \rangle = \delta_{ij}$.
To clarify the structure of the proof, we start off with a claim: \\ 
\emph{Claim: There is an operator $\Lambda_i : E \to E$ for every $i \in I$ such that $\sigma_i(F,F') = \langle \Lambda_i F , F' \rangle$.} \\
We will use the operator $\Lambda_i$ explicitly to construct the symmetric form $s_i$. To see that the claim holds, define for a fixed $i \in I$ the operator
\begin{align}
\tilde{\sigma}_i :E &\to E^*  \\ 
F &\mapsto \tilde{\sigma}_i (F) \coloneqq \sigma_i (F , \cdot) \formspace, \notag
\end{align}
where $E^*$ denote the dual space of $E$. By Riesz' representation theorem (see for example \cite[Chapter II.16]{akhiezer-linear_ops_on_HS}), there exists a unique dual vector of $\tilde{\sigma}_i(F) \equiv \tilde{\sigma}_{i,F} $ given by $\tilde{\sigma}_i(F) ^*$ such that
\begin{align}
	\tilde{\sigma}_{i,F}(F') = \langle \tilde{\sigma}_i(F)^* , F' \rangle \formspace. 
\end{align}
Now, define the desired operator
\begin{align}
	\Lambda_i : E &\to E  \\
	F &\mapsto \Lambda_i(F) \coloneqq \tilde{\sigma}_i (F)^* \formspace. \notag
\end{align}
The claim then follows by construction: $\sigma_i(F,F') = \tilde{\sigma}_{i,F}(F') = \langle \Lambda_i F , F' \rangle $. \\
Now we choose, using the operator norm $\| \Lambda_i \| = \sup\left\{ \frac{\| \Lambda_i F\|}{\|F \|}: {F \in E}   \right\}$, 
\begin{align}
	s_i (F , F') \coloneqq \norm{\Lambda_i} {\cdot }\langle F , F' \rangle \formspace.
\end{align}
Clearly, this fulfills, using the Cauchy-Schwarz inequality:
\begin{align}
	\sigma_i (F , F')^2 
	&= \langle \Lambda_i F , F' \rangle ^2 \notag \\
	&\leq \langle \Lambda_i F, \Lambda_i F \rangle \, \langle F', F' \rangle\notag \\
	&\leq \| \Lambda_i \|^2 {\cdot} \langle F, F \rangle \, \langle F', F' \rangle \notag\\
	&= s_i(F,F) \, s_i(F' , F') \formspace.
\end{align}
This proves existence. We are left to show that this particular choice of $\left\{ s_i \right\}$ is continuous:
Let $ \left\{ x_i \right\}_{i \in \IN} \subset \IR $ be a continuous sequence in $\IR$ with $\lim\limits_{i \to \infty} \left( x_i -x_0\right) = 0$. It then holds by definition for all fixed $F,F' \in E$ that 
\begin{align}
0&=\lim\limits_{i \to \infty} \Big( \sigma_{x_i}(F,F') -\sigma_{x_0}(F,F') \Big) \notag	\\
\iff 0&= \lim\limits_{i \to \infty} \Big(  \langle \Lambda_{x_i} F, F' \rangle -  \langle \Lambda_{x_0} F, F' \rangle    \Big) \notag \\
\iff 0&= \lim\limits_{i \to \infty}      \langle (\Lambda_{x_i} - \Lambda_{x_0} ) F, F' \rangle \notag \\
\iff 0&=    \langle ( \lim\limits_{i \to \infty}  \Lambda_{x_i} - \Lambda_{x_0} ) F, F' \rangle \formspace.
\end{align}
Specifying $F' = \lim\limits_{i \to \infty}  (\Lambda_{x_i} - \Lambda_{x_0} ) F  )$,  this implies
\begin{align}
\implies 0&= \norm{\lim\limits_{i \to \infty}     (\Lambda_{x_i} - \Lambda_{x_0} ) F }\notag \\
\implies 0&= \lim\limits_{i \to \infty}     \| \left( \Lambda_{x_i} - \Lambda_{x_0} \right) F\|  
\end{align}
holds for all $F \in E$. We have used the continuity of the scalar product. Consequently, in the chosen basis, the matrix elements of $(\Lambda_{x_i} - \Lambda_{x_0})$ converge to zero, that is, for each pair $(i,j) \in I \times I$ it holds: For all $\epsilon > 0$ there exists a $i_0 \in \IN_0$, such that if $i \geq i_0$, then 
\begin{align}
	\abs{(\Lambda_{x_i} - \Lambda_{x_0})_{ij}} < \epsilon / N^2 \formspace.
\end{align}
We now choose $F \in E$ such that $\| F \| = 1$, that is, in the given basis we find $F = \sum_{n=1}^N f_n \, e_n$, where $f_n = \langle e_n , F \rangle$ and $\sum_{n=1}^N f_n ^2 = 1$. Therefore, it holds for all $n= 1,2, \dots, N$ that $\abs{f_n} \leq 1$. We find for $i \geq i_0$ that:
\begin{align}
	\| \Lambda_{x_i} - \Lambda_{x_0} \|
	&= \sup \left\{ \norm{\big( \Lambda_{x_i} - \Lambda_{x_0} \big) F} : F \in E, \norm{F} = 1 \right\} \notag\\
	&= \sup \left\{\norm{  \sum_{m,n=1}^N \big( \Lambda_{x_i} - \Lambda_{x_0} \big)_{mn}\;  f_n \, e_n  } : F \in E, \norm{F} = 1 \right\} \notag\\
	&\leq \sup \left\{ \sum_{m,n=1}^N \abs{\big( \Lambda_{x_i} - \Lambda_{x_0} \big)_{mn}}\cdot  \underbrace{\abs{f_n}}_{\leq 1} \cdot \underbrace{\norm{e_n}}_{=1}   : F \in E, \|F\| = 1 \right\} \notag\\
	&\leq  \sum_{m,n=1}^N \abs{\big( \Lambda_{x_i} - \Lambda_{x_0} \big)_{mn}}\notag\\
	&< N^2 \cdot \epsilon / N^2 \notag\\
	&= \epsilon \formspace.
\end{align}
Hence,  $i \mapsto \| \Lambda_i \| $ is continuous. 
Therefore, by construction, $i \mapsto s_i(F,F') = \norm{ \Lambda_i}{\cdot} \langle F , F' \rangle $ is continuous for all fixed $F,F' \in E$.
\end{proof}
With this lemma at our disposal we are able to compare states at different masses $m$ with each other. This is crucial for the proof of the following theorem:
%
%
\begin{theorem}[Continuous Weyl C*-field]\label{thm:cont_field_W}
	There exists a unique continuous field $\left( \left\{ \Wm \right\}_m , \K \right)$ of C*-algebras, such that for every $F \in E$ it holds $\left[ m \mapsto W_m(F) \right] \in \K $.
\end{theorem}
\begin{proof}
	The proof presented here is a generalization of the proof given in \cite[Theorem 5.2]{rieckers_honegger_deformation}. It was even possible, even though we are looking at a more general setup,  to simplify the given proof quite a bit. The main idea presented in \cite{rieckers_honegger_deformation}, that is, to use Lemma \ref{lem:cont_field_existance} and \ref{lem:on_contiuous_families}, was nevertheless essential for the proof.
	As mentioned, we will proof the statement by applying Lemma \ref{lem:cont_field_existance} to a specific sub *-algebra $\D$ of $\W$. For this we define:
	\begin{align}
		\D \coloneqq \Span{ \left[  m \mapsto \e^{-\i \alpha(m)} \, W_m(F)  \right]: F\in E, \alpha : \IR \to \IR_+ \text{ continuous} }  \formspace.
	\end{align}
	Clearly, as desired, we find for every $F \in E$ that $\left[ m \mapsto W_m(F) \right] \in \D $. 
	First, we need to check that $\D$, equipped with point-wise defined algebra relations, really constitutes a sub *-algebra of $\W$: 
	An arbitrary element of $\D$ is of the form
	\begin{align}
		K = \sum\limits_{i=1}^{N} \sum\limits_{j=1}^{M} z_{ij} \left[ m \mapsto \e^{-\i \alpha_{ij}(m) }\,W_m(F_i) \right] \label{def:K_arb_element_D}
	\end{align}
	for some $z_{ij} \in \IC$, $F_i \in E$ and some continuous maps $\alpha_{ij} : \IR_+ \to \IR$. 
	Clearly, for every $m \in \IR_+$ it holds that $K(m) \in \Wm$, and hence $\D \subset \W$. The nontrivial part is to show that $\D$ is closed under the point-wise defined algebra relations.
	Let $K, \tilde{K} \in \D$ be of the form specified above. We find
	\begin{align}
		(K + \tilde{K})(m)
		&\coloneqq K(m) + \tilde{K}(m) \notag \\
		&= \sum\limits_{i,j} z_{ij} \e^{-\i \alpha_{ij}(m) }\,W_m(F_i)   + \sum\limits_{k,l} \tilde{z}_{kl} \e^{-\i \tilde{\alpha}_{kl}(m) }\,W_m(\tilde{F}_k)  \notag \\
		&= \sum\limits_{pq} w_{pq} \e^{-\i \alpha'_{pq}(m) }\,W_m(F'_{p}) 
	\end{align} 
	for some $z'_{pq} \in \IC$, $F'_{p} \in E$ and some continuous $\alpha'_{pq} : \IR_+ \to \IR$ (because the composition of continuous maps is continuous)\footnote{In particular: if for some $i, k$ we find $F_i$ = $\tilde{F_k}$, then we certainly find $w_{pq}$ and continuous maps $\alpha'_{pq} : \IR_+ \to \IR$, such that $\sum_q w_{pq} \e^{-\i \alpha'_{pq}(m) } = \sum_j z_{ij} \e^{-\i \alpha_{ij}(m) } + \sum_l \tilde{z}_{kl} \e^{-\i \tilde{\alpha}_{kl}(m) }$.}.	Therefore, $\D$ is closed with respect to addition. \\
	Next, we find for the multiplication, using the Weyl-relations :
	\begin{align}
		(K \cdot \tilde{K})(m) 
		&\coloneqq K(m) \cdot \tilde{K}(m) \notag \\
		%&= \sum\limits_{i,j} z_{ij} \e^{-\i \alpha_{ij}(m) }\,W_m(F_i) \cdot \sum\limits_{k,l} \tilde{z_{kl}} \e^{-\i \tilde{\alpha}_{kl}(m) }\,W_m(\tilde{F}_k)   \\
		&= \sum\limits_{i,j,k,l}z_{ij}\,\tilde{z}_{kl} \,\e^{-\i \left( \alpha_{ij}(m) + \tilde{\alpha}_{kl}(m)  + \Gm{F_i}{\tilde{F}_k}/2   \right) }\,W_m(F_i + \tilde{F}_k) \formspace.
	\end{align}
	This also is clearly an element in $\D$, since the combination of continuous maps is again continuous\footnote{Here, of course, we need $\Gm{}$ to be continuous.}.
	In the same fashion, it follows trivially that $\D$ is also closed under involution, hence $\D$ does indeed constitute a sub *-algebra of $\W$.\par
	Now, the main effort is to show that $\D$ fulfills the assumptions (i) and (ii) of Lemma \ref{def:continous_field_algebra}: 
	First, it is trivial to see that,  since $\left[ m \mapsto W_m(F) \right] \in \D$ for all $F\in E $ and since for every $m \in \IR_+$ the $W_m(F)$ are dense in $\Wm$ by construction, 
	we find that for every $m \in \IR^+$ the set $\left\{ K(m) : K \in \D \right\}$ is dense in $\Wm$.
	The non trivial part is to show that every section $K \in \D$ is continuous: 
	Again, let $K \in \D$ denote an arbitrary element of the form specified in equation (\ref{def:K_arb_element_D}).
	Further more, since only finitely many $F_i$'s contribute to the construction of $K(m)$, we can equivalently view $K(m)$ as an element of a subspace of $\Wm$ that is generated by only finitely many $W_m(F_i)$'s. In particular, define $E_N = \Span{F_i, i=1,2,\dots,N }$, where the $F_i$'s correspond to the ones contributing to $K(m)$. Then $\Wm(E_N,\Gm{})$ is a sub-C*-algebra of $\Wm(E,\Gm{})$  and we can evaluate the norm of $K(m)$ in terms of states $C \in \C (E_N, \Gm{})$ (see e.g. \cite[Section III.B]{rieckers_honegger_construction}), and the norms of $K(m)$ on $\Wm(E_N,\Gm{})$ and $\Wm(E,\Gm{})$ agree.
		Next, we define the pre-symplectic form $\sigma_m = \Gm{} - \Gmz{}$ on $E$. Since for fixed $F,F' \in E$, $\Gmz{F,F'}$ is constant and $m \mapsto \Gm{F}{F'}$ is continuous by assumption, it is clear that $\left\{\sigma_m \right\}_{m \in \IR_+}$ constitutes a continuous family of pre-symplectic forms on $E$, hence it also constitutes a continuous family of pre-symplectic forms on $E_N$. 
		Moreover, we find
		\begin{align}
			\lim\limits_{m \to m_0} \sigma_m (F,F') = 0
		\end{align}
		for all fixed $F,F' \in E_N$. 
		Now, since by construction $E_N$ is finite dimensional, we can make use of the additional feature of Lemma \ref{lem:on_contiuous_families}, that is, there exists a continuous family $\left\{ s_m \right\}_{m \in \IR_+}$ of symmetric, positive, bilinear forms on $E_N$ such that
			\begin{align}
				E_N \ni F \mapsto \e^{- s_m(F,F) /2} \in \C (E_N,\sigma_m) \formspace.
			\end{align}
	Knowing the existence of such a family $\left\{ s_m \right\}_{m \in \IR_+}$ and the corresponding exponential states is the reason why we go over to a finite dimensional subspace $E_N$ of $E$.
	By construction we moreover know that $\Gm{} = \sigma_m + \Gmz{}$ is a sum of pre-symplectic forms on $E_N$. Making use of the first part of Lemma \ref{lem:on_contiuous_families}, we conclude that for an arbitrary state $C_0 \in \C (E_N , \Gmz{})$ there exists a state $C \in \C (E_N , \Gm{})$ of the form
	\begin{align}
		C (F) = C_0(F) \, \e^{- s_m(F,F) /2} \formspace.
	\end{align}
	With this notion we are able to compare states, and hence the norm of a section of the Weyl algebra bundle, at different masses $m$. But it is worth mentioning that not necessarily \emph{all} states $C \in \C (E_N , \Gm{})$ are of the above product form. This subtlety has to be kept in mind. 
	Let us now look at the expectation value of the section $K$ at point $m$ and at point $m_0$, inserting the product form of the state $C \in \C (E_N , \Gm{})$:
	\begin{align}
		\omega_C\big( K^*(m) K(m) \big)
		&= \sum_{ijkl} \bar{z}_{ij}z_{kl} \, \e^{-\i \big(  \alpha_{kl}(m)  - \alpha_{ij}(m) + \Gm{F_i}{F_k}/2\big)} \notag \\ &\phantom{M}\cdot \e^{-s_m(F_k - F_i , F_k - F_i) /2} \, C_0(F_k - F_i)  \formspace,\\
		\omega_{C_0}\big( K^*(m_0) K(m_0) \big)
		&= \sum_{ijkl} \bar{z}_{ij}z_{kl} \, \e^{-\i \big(  \alpha_{kl}(m_0)  - \alpha_{ij}(m_0) + \Gmz{F_i}{F_k}/2\big)} \, C_0(F_k - F_i) \formspace.
		\end{align}	
Since for all $i,j$ and all $F_i, F_k$, the maps $m \mapsto \Gm{F_k}{F_i}$ and $m \mapsto S_m(F_k,F_i)$ as well as the maps $\alpha_{ij}, \alpha_{kl}$ are continuous, we conclude that
\begin{align}
	\Lambda_{ijkl}(m) \coloneqq \e^{-\i \big(  \alpha_{kl}(m)  - \alpha_{ij}(m) + \Gm{F_i}{F_k}/2\big)} \, \e^{-s_m(F_k - F_i , F_k - F_i) /2}
\end{align}
is continuous. In particular it holds that
\begin{align}
	\lim\limits_{m \to m_0} 	\Lambda_{ijkl}(m)  = \e^{-\i \big(  \alpha_{kl}(m_0)  - \alpha_{ij}(m_0) + \Gmz{F_i}{F_k}/2\big)} \eqqcolon \Lambda_{ijkl}(m_0)  \formspace.
\end{align}
This means, for all $i,k= 1,\dots,N$ and $j,l=1,\cdots,M$ and for every $\epsilon >0$ we find a $\delta > 0 $, such that for every $m \in \IR_+$ with $\abs{m - m_0} < \delta$ it holds
\begin{align}
\abs{ \Lambda_{ijkl}(m) - \Lambda_{ijkl}(m_0) }< \epsilon/\left( M^2 N^2 \,\max\limits_{i,j,k,l}\abs{\bar{z}_{ij} z_{kl}} \right)  \formspace.
\end{align}
With this we can write (here $C$ is a state of product form)
\begin{align}
\omega_{C_0}&\big( K^*(m_0) K(m_0) \big) \notag \\
&= \sum_{ijkl} \bar{z}_{ij}z_{kl} \, \Lambda_{ijkl}(m_0)\, C_0(F_k - F_i) \notag \\
&= \omega_{C}\big( K^*(m) K(m) \big) -  \sum_{ijkl} \bar{z}_{ij}z_{kl} \, \big( \Lambda_{ijkl}(m) - \Lambda_{ijkl}(m_0)\big) \, C_0(F_k - F_i) \formspace,
\end{align}
where in the second step we have just inserted a zero term.  Since by definition $\omega_{C_0}\big( K^*(m_0) K(m_0) \big)  \geq 0$ we conclude that
\begin{align}
\omega_{C_0}&\big( K^*(m_0) K(m_0) \big)\notag  \\
&=\abs{\omega_{C}\big( K^*(m) K(m) \big) -  \sum_{ijkl} \bar{z}_{ij}z_{kl} \, \big( \Lambda_{ijkl}(m) - \Lambda_{ijkl}(m_0)\big) \, C_0(F_k - F_i)}  \notag  \\
&\leq \omega_{C}\big( K^*(m) K(m) \big) + \sum_{ijkl} \abs{\bar{z}_{ij}z_{kl}} \, \abs{\big( \Lambda_{ijkl}(m) - \Lambda_{ijkl}(m_0)\big)} \, \abs{C_0(F_k - F_i)} \notag  \\
&\leq \omega_{C}\big( K^*(m) K(m) \big) + \epsilon \formspace,
\end{align}
where the last step holds for $\abs{m - m_0} < \delta$. We have made use of the estimate 
\begin{align}
\abs{C(F)} \leq 1 \formspace,
\end{align}
as stated in Lemma \ref{lem:states_estimate}. Having found an estimate of $\omega_{C_0}\big( K^*(m_0) K(m_0) \big)$, we can estimate the norm of $K$ at $m_0$, where here $C$ is of the above specified product form:
\begin{align}
	\norm{K(m_0)}^2 
	&= \sup\Big\{ \omega_{C_0}\big( K^*(m_0) K(m_0) \big) : C_0 \in \C(E_N , \Gmz{})\Big\} \notag \\
	&= \sup\Big\{ \omega_{C}\big( K^*(m) K(m) \big) - \omega_{C}\big( K^*(m) K(m) \big) + \omega_{C_0}\big( K^*(m_0) K(m_0) \big)   :\notag \notag \\
	&\phantom{M} C=C_0\, e^{s_m(\cdot, \cdot)/2}, C_0 \in \C(E_N , \Gmz{})\Big\}\notag \\
	&\leq \sup\Big\{ \omega_{C}\big( K^*(m) K(m) \big)+\epsilon   : C=C_0\, e^{s_m(\cdot, \cdot)/2}, C_0 \in \C(E_N , \Gmz{})\Big\} \notag\\	
	& \leq  \sup\Big\{ \omega_{C}\big( K^*(m) K(m) \big) + \epsilon : C \in \C(E_N , \Gm{})\Big\} \notag\\
	&= 	\norm{K(m)}^2 + \epsilon.
\end{align}
The calculation might seem a bit too detailed at first, but one has to be a careful in which order one makes the estimate for the proof to work, due to the mentioned subtlety that not \emph{all} states at mass $m$ are of product form.
%
%
In the exact same fashion, interchanging the roles of $C_0$ and $C$, we define (using the same notation as before) the pre-symplectic form
\begin{align}
	\sigma_m = \Gmz{} -\Gm{} \formspace.
\end{align}
In the same line of arguments as before, we can express states at $m_0$ by $C_0 = C\, \e^{s_m(\cdot, \cdot)/2}$, where here actually the symmetric forms $s_m$ are the same as before, since, by construction, they only depend on the norm of $\sigma_m$, and, reversing the roles of $m$ and $m_0$, we only have flipped the sign of $\sigma_m$. 
We find
\begin{align}
	\omega_{C}\big( K^*(m) K(m) \big) 
	&= \Big\lvert\omega_{C_0}\big( K^*(m_0) K(m_0) \big)  \notag \\
	&\phantom{M}-  \sum_{ijkl} \bar{z}_{ij}z_{kl} \, \Big( \Lambda_{ijkl}(m)\,\e^{s_m(F_k - F_i, F_k - F_i)/2} \notag \\
		&\phantom{M}- \Lambda_{ijkl}(m_0) \, \e^{-s_m(F_k - F_i, F_k - F_i)/2}\Big) \, C(F_k - F_i) \Big\rvert\notag\\
	&\leq \omega_{C_0}\big( K^*(m_0) K(m_0) \big)  + \epsilon
\end{align}
for $\abs{m-m_0} < \delta$.\\
By an estimate of the same fashion as before we find for $\abs{m-m_0}<\delta$:
\begin{align}
\norm{K(m)}^2  \leq  \norm{K(m_0)}^2 + \epsilon \formspace.
\end{align}
Combining the two estimates, we find the wanted result, namely that for all $\epsilon >0$ there exists a $\delta > 0$ such that for all $m \in \IR_+$ with $\abs{m-m_0}<\delta$ it holds:
\begin{align}
	\abs{ \norm{K(m)}^2 - \norm{K(m_0)}^2} < \epsilon \formspace,
\end{align}
hence, $m \mapsto \norm{K(m)}^2$ is continuous. Since $0 \leq \norm{K(m)}^2$ it follows that $m \mapsto \norm{K(m)}$ is also continuous. This concludes the proof.
%(see \cite[chapter III.B]{rieckers_honegger_construction}).
\end{proof}
%
%
%
%
\subsection{Dynamics and the C*-Weyl algebra} \label{sec:weyl-algebra-dynamics}
%
%
%
%
%
%
Having found a notion of continuity of the fields with respect to the mass, the task is now to implement the dynamics of the system, that is, the Weyl operators should solve the Proca equation in a suitable sense. Motivated by Definition \ref{def:algebra-A(M)} of the field operators $\A(F)$ we want to implement the relation $W_m\big( (\delta d + m^2) F\big) = \e^{\langle j , F' \rangle} \mathbbm{1}$ (the relation follows from Definition \ref{def:algebra-A(M)} by heuristically viewing the Weyl operators as $\e^{\A(F)}$). For now, we specify to the source free case $j = 0$. To formulate the implementation of the above relation mathematically precisely, we define the set 
\begin{align}
	\JM \coloneqq \left\{ F \in E : \exists F' \in E : F = (\delta d + m^2)F'  \right\} \equiv \left\{ F \in E : G_mF=0  \right\} \formspace.
\end{align} 
	The dynamical theory is then implemented in the Weyl algebra $\Wm\left({\Quotientscale{E}{\JM}}, \Gm{}\right)$, where the elements $\left[ F \right]_m \in {\Quotientscale{E}{\JM}}$ are equivalence classes with respect to the equivalence relation $F \sim_m F' :\iff (F-F') \in \JM$. Identifying $[0]_m = \JM$, we have implemented the wanted field equation $W_m([0]_m) = \mathbbm{1}$. Note, that we use the same symbol $W_m$ for the Weyl operators in $\Wm$ and $\Wmdyn$ since it is clear from the context, that is, whether they act on functions $F$ or equivalence classes $[F]_m$, in which algebra the corresponding Weyl elements lie.\par 
	The question is how to apply the results we have gained on the algebra $\Wm$ and $\W$ to the dynamical algebras $\Wmdyn$ and $\Wdyn$. To accomplish this we define the following *-algebra homomorphism:
	\begin{align}
		\tilde{\alpha}_m : \;&\Wm \to \Wmdyn  \\
										& W_m(F) \mapsto W_m([F]_m) \notag
	\end{align}
	and analogously for linear combinations of Weyl elements. It is easily checked, that indeed $\tilde{\alpha}_m$ is a *-algebra homomorphism:
	Let $F, F' \in E$, then
	\begin{align}
		\tilde{\alpha}_m\big( W_m(F) W_m(F') \big) 
		&= \e^{-\i \Gm{F}{F'}} \, \tilde{\alpha}_m\big( W_m(F + F')\big) \notag\\
		&= \e^{-\i \Gm{F}{F'}} \,  W_m\big([F]_m + [F']_m\big) \notag\\
		&= \e^{-\i \Gm{F}{F'}}\e^{\i \Gm{[F]_m}{[F']_m}} \,  W_m\big([F]_m\big) W_m\big([F']_m\big) \notag\\
		&= W_m\big([F]_m\big) W_m\big([F']_m\big) \formspace.
	\end{align}
	Also, it holds that
	\begin{align}
	\tilde{\alpha}_m\big( W_m(F) ^* \big) 
	&=\tilde{\alpha}_m\big( W_m(-F) \big) \notag\\
	&= W_m([-F]_m)  \notag\\
	&=  W_m(-[F]_m)  \notag\\
	 &= W_m([F]_m)^* \formspace.
	\end{align}
	Now, from the first isomorphism theorem, since clearly the image of $\tilde{\alpha}_m$ is the whole algebra $\Wmdyn$, it holds that $\Wmdyn = \tilde{\alpha}_m\left( \Wm \right) \simeq {\Quotientscale{\Wm}{\Ker{\tilde{\alpha}_m}}}$ .
	In particular, $\Ker{\tilde{\alpha}_m}$ is a two sided ideal in $\Wm$ and one can show that this ideal corresponds to the ideal that is generated by the relation $\big( W_m(F) -\mathbbm{1} \big)$ for $F \in \JM$. So the morphism $\tilde{\alpha}$ implements the wanted dynamics. \\[4mm]  
	%
	To see this, let us explicitly compare the two ideals:
	\begin{align}
		\Ker{\tilde{\alpha}_m} &= \Big\{A = \sum z_i W_m(F_i) : 0 = \sum z_i W_m\big([F_i]_m\big)\Big\}  \formspace,\\
		\IM &=  \Big\{A\, \big(W_m(F) - \mathbbm{1}\big)\,  B  : A, B  \in \Wm, F \in E \Big\} \notag  \\
									& = \Big\{A\, \big(W_m(F) - \mathbbm{1}\big)  : A  \in \Wm, F \in E \Big\} \notag\\
									& =  \Big\{\sum z_i  \big( W_m(F_i + H ) - W_m(F_i)\big) :  z_I \in \IC, F_i \in E , H \in \JM\Big\} \formspace.
	\end{align}
	\begin{lemma}
		Let $\tilde{\alpha}_m$ and $\IM$ be defined as above. Then
		\begin{align}
		 \Ker{\tilde{\alpha}_m} = \IM \formspace.
		\end{align}
	\end{lemma}
	\begin{proof}
1.) For the first direction of the proof let $A \in \IM$. Then
\begin{align}
	\tilde{\alpha}_m(A) = \sum z_i  \big( W_m([F_i + H]_m ) - W_m([F_i]_m)\big)
	=0
\end{align}
since $[F_i + H]_m = [F_i]_m + [H]_m = [F_i]_m$, because $H \in \JM$ by assumption. Hence, $\IM \subset \Ker{\tilde{\alpha}_m}$. \par 
2.) For the other direction, let $A \in \Ker{\tilde{\alpha}_m}$. \\
An arbitrary $A \in \Wm$ is of the form $A = \sum z_i \, W_m(F_i)$. We can reorder this finite sum, such that we group together the $F_i$'s that are in the same equivalence class, thus writing $A = \sum_j A_j$, where $A_j = \sum_i z_{ij} W_m(F_{ij})$ with $[F_{ij}] = [F_{i'j'}] \iff j=j'$. With this notation it is easier to classify elements of the kernel of $\tilde{\alpha}$. By construction $\tilde{\alpha}(A) = 0 \iff \tilde{\alpha}(A_j) =0$ for all $j \iff \sum_i z_{ij} = 0$ for all $j$. It therefore suffices to show that $A_j \in \IM$ for all $j$:
\begin{align}
	A_j 
	&= \sum_i z_{ij} W(F_{ij})  \notag\\		
	&= \sum_i z_{ij} \big( \underbrace{W(F_{ij}) - W(F_{i'j})}_{\in \IM} +W(F_{i'j}) \big) , \text{ for some }i' \notag \\
	&= \underbrace{\sum_i z_{ij}}_{=0}  W(F_{i'j}) + \underbrace{X}_{\in \IM}
\end{align}
hence, $A_j \in \IM$ for all $j$, from which we conclude $\Ker{\tilde{\alpha}_m} \subset \IM$ which completes the proof. 
	\end{proof}
	%
%
%
%
%
We now want to construct a homomorphism from the Weyl algebra bundle $\W$ to $\Wdyn$. We do this by a point-wise definition using the homomorphism $\tilde{\alpha}$. 
Define:
\begin{align}
	\alpha : \;&\W \to \Wdyn  \\
	& K \mapsto \alpha(K), \text{ such that }\big(  \alpha(K) \big) (m) = \tilde{\alpha}_m \big( K(m)\big) \formspace. \notag
\end{align}
Since the algebra relations in $\W$ and $\Wdyn$ are defined point-wise and we have furthermore already seen that $\tilde{\alpha}_m$ is a *-algebra-homomorphism for each $m$ we conclude that $\alpha$ is a *-algebra homomorphism. By the same argument as before we also conclude that $\Wmdyn \simeq {\Quotientscale{\W}{\Ker{\alpha}}}$.\par 
Now we are interested if we can find continuous fields of C*-algebras on $\Wdyn$ such that the physically interesting observables are contained. In particular we would like to have a theorem of the kind of Theorem \ref{thm:cont_field_W} for the algebra $\Wdyn$. Unfortunately it turns out that a theorem of the same generality as for the non-dynamical case is not possible. This is stated in form of the following theorem:
%
\begin{theorem}
	There cannot exist a continuous field $\big( \left\{\Wmdyn \right\} , \K^\textrm{dyn} \big)$ of C*-algebras such that for all $[F]_m \in {\Quotientscale{E}{\JM}}$ it holds $\big[ m \mapsto W_m([F]_m)\big] \in \K^\textrm{dyn}$.
\end{theorem}
\begin{proof}
	Let $\K^\text{dyn}$ be a sub *-algebra of $\prod_m \Wmdyn$ such that for all $[F]_m \in {\Quotientscale{E}{\JM}}$ it holds $\big[ m \mapsto W_m([F]_m)\big] \in \K^\textrm{dyn}$. In particular all sections $K \in \K$ are continuous in the sense that $m \mapsto \norm{K(m)}_m$ is continuous. It is clear that for all $[F]_m \in {\Quotientscale{E}{\JM}}$ the sections $\left[ m \mapsto W_m([F]_m)\right]$ are continuous. But, since $\K$ is an algebra, also linear combinations of $\left[ m \mapsto W_m([F]_m)\right]$ are elements of $\K$ and are ought to be continuous. This does not hold as we shall see:\\
	Let $K = \Big( \big[ m \mapsto W_m([F]_m) \big] - \big[ m \mapsto W_m([H]_m) \big]  \Big) \in \K$ where we choose $F, H \in E$ such that $[F]_{m_0} = [H]_{m_0}$ for some $m_0 \in \IR$ but $[F]_{m_0} \neq [H]_{m_0}$ for $m \neq m_0$.\\
	Then by construction we find $\norm{K(m_0)}_{m_0} = 0$ but, because for Weyl elements it holds $\norm{z W_m([F]_m) + w W_m([H]_m)}_m = \abs{z} + \abs{w}$ (see e.g. \cite[Proposition 3.10]{rieckers_honegger_construction}), for $m \neq m_0$ it holds $\norm{K(m)}_m = 2$, hence $K(m)$ is not continuous. \Lightning\\
\end{proof}
Note, that this does not mean that there cannot be a continuous field $\K^\textrm{dyn}$ of C*-algebras such that at least for \emph{some} $[F]_m\in {\Quotientscale{E}{\JM}}$ the sections $\big[ m \mapsto W_m([F]_m)\big] \in \K^\textrm{dyn}$. By the argument in the above proof, we certainly need to leave out all the elements that can for some $m_0$ and some $F' \in E$ be written as $F = (\delta d + m_0^2)F'$. In particular this includes all closed test forms $F$: Choosing $m_0 =1$ and $F' = F$ one finds $(\delta d + 1)F' = F$. But there are certainly a lot more $F$ of the above form that we would have to discard. Also, when removing some $\big[ m \mapsto W_m([F]_m)\big]$ from the field of C*-algebras it is not clear, whether we are still able to fulfill the property that the set $\big\{  K(m) : K \in \K^\text{dyn}\big\}$ is dense in $\Wmdyn$. 
We would have to include some sections that take at every $m$ a value in $\Wmdyn$ that cannot be represented by a linear combination of $W_m([F]_m)$'s. These elements exist due to $\Wmdyn$ being defined as the $\norm{\cdot}_m$-closure of the span of the Weyl elements $W_m([F]_m)$. Such elements cannot be written down in a fashion that we have used to characterize sections so far.\par
Due to these reasons, the C*-Weyl algebra does not seem to be suitable for our problem and the ansatz of finding continuous fields of C*-algebras to investigate the zero mass limit is hereby discarded.
%
%