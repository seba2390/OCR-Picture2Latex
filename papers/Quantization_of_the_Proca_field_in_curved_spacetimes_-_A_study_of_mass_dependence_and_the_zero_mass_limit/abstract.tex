\section*{Abstract}
In this thesis we investigate the Proca field in arbitrary globally hyperbolic curved spacetimes. We rigorously construct solutions to the classical Proca equation, including external sources and without restrictive assumptions on the topology of the spacetime, and investigate the classical zero mass limit. We formulate necessary and sufficient conditions for the limit to exist in terms of initial data. We find that the limit exists if we restrict the class of test one-forms, that we smear the distributional solutions to Proca's equations with, to those that are co-closed, effectively implementing a gauge invariance by exact distributional one-forms of the vector potential. In order to obtain also the Maxwell dynamics in the limit, one has to restrict the initial data such that the Lorenz constraint is well behaved. With this, we naturally find conservation of current and the same constraints on the initial data that are independently found in the investigation of the Maxwell field by Pfenning.\par
For the quantum problem we first construct the generally covariant quantum Proca field theory in curved spacetimes in the framework of Brunetti, Fredenhagen and Verch and show that the theory is local. Using the Borchers-Uhlmann algebra and an initial value formulation, we define a precise notion of continuity of the quantum Proca field with respect to the mass. With this notion at our disposal we investigate the zero mass limit in the quantum case and find that, like in the classical case, the limit exists if and only if the class of test one-forms is restricted to co-closed ones, again implementing a gauge equivalence relation by exact distributional one-forms. It turns out that in the limit the fields do not solve Maxwell's equation in a distributional sense. We will discuss the reason from different perspectives and suggest possible solutions to find the correct Maxwell dynamics in the zero mass limit.
