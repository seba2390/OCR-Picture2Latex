\section{The Classical Problem}\label{chpt:classical}
In this chapter we will examine Proca's equation at a classical level in an arbitrary globally hyperbolic spacetime. Using differential forms, the formulation will be mostly coordinate independent. The goal is to find a solution to Proca's equation, including external classical sources and without restrictive assumptions on the topology of the spacetime, in terms of fundamental solutions of the Proca operator. Already at this classical stage we will emphasize on similarities and, partially crucial, differences of Proca's equation and Maxwell's equations.\par
We will start by finding the equations of motion from the Lagrangian\footnote{We could equivalently start by imposing the equations of motion directly, but the Lagrangian yields a more familiar comparison to the Maxwell field.}. In order to solve the equations of motion, it is then crucial to decompose the equations in a set of a hyperbolic differential equation and a constraint. After discussing the initial value problem in detail, we will determine the solution of the equations of motion of the Proca field in terms of fundamental solutions of the Proca operator.
%
Having found solutions to the classical Proca equation, we will investigate the classical zero mass limit as it will be the basis of understanding the according limit in the quantum case.\par
%
In the following, let $(\M,g)$ denote a globally hyperbolic four dimensional spacetime, consisting of a smooth manifold $\M$, assumed to be Hausdorff, connected, oriented, time-oriented and para-compact,  and a Lorentzian metric $g$, whose signature is chosen to be $(-,+,+,+)$. The Cauchy surface of the spacetime is denoted by $\Sigma$, with an induced Riemannian metric $h$. The Levi-Civita connection on $(\M,g)$ will as usual be denoted by $\nabla$, the one on $\Sigma$ by $\nabla_{(\Sigma)}$.
%
%
\subsection{Deriving the equations of motion of the Proca field from the Lagrangian}
%
Let $A, j \in \Omega ^1 (\M)$ be smooth one-forms on $\M$, $j \in \Omega^1(\M)$ a external source and $m >0$ a positive constant. We will call $A$ the \emph{vector potential}, \gls{mass} the \emph{mass} and \gls{current} denotes an \emph{external current}. The Lagrangian of the Proca field reads:
%
\begin{align}
\Ldens = -\frac{1}{2} \, dA  \wedge  *dA + A \wedge *j - \frac{1}{2} \, m^2 A \wedge *A \formspace.
\end{align}
In local coordinates this can equivalently be expressed, defining the field-strength tensor $F = dA$, as:
\begin{align}
\Ldens = \left( -\frac{1}{4}\, F\indices{_{\alpha\beta}} F\indices{^{\alpha\beta}} + A\indices{_\mu} j^\mu - \frac{1}{2} \, m^2 A\indices{_\nu} A\indices{^\nu} \right) \dvolg \formspace. \label{eqn:Proca_Lagrangian_coordinates}
\end{align}
At this stage, the similarity to the Lagrangian of the Maxwell field is obvious. Setting $m=0$ in the Proca Lagrangian\footnote{Even though we defined the Proca Lagrangian for non-zero masses only, at this stage setting $m=0$ is not a problem. The restriction to strictly positive masses becomes important later.} yields the Maxwell Lagrangian, that is, the Maxwell field is a massless Proca field.
%
%
The Euler-Lagrange-equations for a Lagrangian depending only on the field and the field's first derivative, $\Ldens = \Ldens(A\indices{_\mu}, \partial\indices{_\nu}A\indices{_\mu})$, are:
\begin{align}
0 = \frac{\partial \Ldens}{\partial A\indices{_\mu}} - \partial\indices{_\nu} \frac{\partial \Ldens}{\partial(\partial\indices{_\nu} A\indices{_\mu})} \formspace.
\end{align}
In local coordinates the first summand of the Euler-Lagrange-equations is easily obtained from equation (\ref{eqn:Proca_Lagrangian_coordinates}):
\begin{align}
\frac{\partial \Ldens}{\partial A\indices{_\mu}} = j^\mu - m^2 A^\mu \formspace.
\end{align}
The second term is most easily calculated using the coordinate representation of $F\indices{_{\alpha\beta}} = 2 \nabla\indices{_{[\alpha}} A\indices{_{\beta ]}} = \partial_\alpha A_\beta - \partial_\beta A_\alpha$ (all curvature dependent terms drop out due to the symmetry of the Christoffel symbol in it's lower two indices):
\begin{align}
\frac{\partial \Ldens}{\partial(\partial\indices{_\nu} A\indices{_\mu})}
=& -\frac{\partial }{\partial(\partial\indices{_\nu} A\indices{_\mu})} \left( \frac{1}{4} \left(  \partial_\alpha A_\beta - \partial_\beta A_\alpha \right) \left(  \partial^\alpha A^\beta - \partial^\beta A^\alpha \right)   \right) \notag \\
=&  -\frac{1}{4} \Big( \left( \delta\indices{^\nu_\alpha}  \delta\indices{^\mu_\beta} -  \delta\indices{^\nu_\beta}  \delta\indices{^\mu_\alpha} \right ) F\indices{^{\alpha\beta}} + F\indices{_{\alpha\beta}} \left( g\indices{^{\nu \alpha}} g\indices{^{\mu \beta}} - g\indices{^{\nu \beta}} g\indices{^{\mu \alpha}}  \right ) \Big) \notag  \\
 =& -\frac{1}{4} \,4 \, F\indices{^{\nu\mu}} =- F\indices{^{\nu\mu}}  \formspace
\end{align}
In that last step we have used the anti-symmetry of the two-form $F$, that is, in local coordinates $F\indices{_{\alpha\beta}}  = - F\indices{_{\beta\alpha}}$.
%
Combining the two results we obtain the equations of motion of the Proca field:
\begin{align}
0 &=  j^\mu - m^2 A^\mu  + \partial_{\nu} F\indices{^{\nu\mu}}\notag \\
\iff 0 &=  j_\mu - m^2 A_\mu  + \partial^{\nu} F\indices{_{\nu\mu}} \formspace.
\end{align}
Going back to an coordinate independent description, identifying $\partial^{\nu} F\indices{_{\nu\mu}} = -(\delta F)_\mu = -(\delta d A)_\mu$, we have found Proca's equation for a smooth one-form $A$ on a curved spacetime:
%
\begin{definition}[Proca's equation]\label{def:proca_operator}
For smooth one-forms $A,j \in \Omega^1(\M)$ and constant $m>0$, the Proca equation reads:
\begin{align}
\left( \delta d + m^2 \right) A = j \formspace.\label{eqn:Proca}
\end{align}
\end{definition}
Accordingly, the Proca operator is defined by $(\delta d + m^2)$. Again, setting $m=0$ in Proca's equation  yields the (homogeneous) Maxwell equations. It is worth noting that, unlike Maxwell's equation, the Proca equation does not posses a gauge symmetry. In the Maxwell case, two one-forms $A, A'$ that differ by an exact form, that is, $A' = A + d \chi$ for some zero-form $\chi$, yield the same equation of motion; if $A$ solves the Maxwell equation, so does $A'$:
\begin{align}
	\delta d A' = \delta d(A + d \chi) = \delta d A =j \formspace,
\end{align}
using the defining property $d^2 = 0$ of the exterior derivative. In the Proca case no such symmetry has to be accommodated when finding a solution to the equations of motion.
%
%
%
%
%
%
\subsection{Solving Proca's equation}\label{sec:solving_procas_equation}
The goal of this section is to find a solution to Proca's equation (\ref{eqn:Proca}) on an arbitrary globally hyperbolic curved spacetime. The procedure presented in this section is a generalization of \cite{FURLANI}, where \name{Furlani} tackles the same problem but with some topological restrictions on the manifold (he supposes the Cauchy surface to be compact) and excluding external sources. Most of the notation used in this chapter is adopted from \name{Furlani} based on previous work by Dimock \cite{dimock1992quantizedEM}.\par
The problem that one encounters when trying to solve Proca's equation in curved spacetimes is that the Proca operator is not normally hyperbolic\footnote{For the definition of normally hyperbolic operators see Section \ref{sec:global_hyperbolicity}.}. For non normally hyperbolic operators, we do not know a priori of the existence fundamental solutions or if the initial value problem is even well posed.  Fortunately, we are able to decompose Proca's equation into a normally hyperbolic second order partial differential equation and a constraint. For the normally hyperbolic equation on a globally hyperbolic spacetime, we have a well understood theory at our disposal that ensures in particular the well-posedness of the initial value problem, the existence and uniqueness of a solution to the differential equation and the existence and uniqueness of fundamental solutions to the differential operator. Finally, it turns out that the Proca operator is Green-hyperbolic, that is, it possesses unique advanced and retarded fundamental solutions.\par
We will start with decomposing Proca's equation, which is stated in the following lemma:
\begin{lemma}
Let $A, j \in \Omega^1(\M)$ be one-forms on the manifold $\M$ and let $m>0$ be a positive constant. Then Proca's equation is equivalent to a set of equations consisting of a wave equation and a constraint. That is:
%
\begin{subnumcases}
{(\delta d  +m^2)A = j \iff}
\left(\square +m^2 \right) A = j + \frac{1}{m^2} \, d \delta j \label{eqn:classical_wave_eqation} \\
\delta A = \frac{1}{m^2} \delta j \label{eqn:classical_constraint}
\end{subnumcases}
\label{lem:Proca_wave_equiv}
\end{lemma}
%
\begin{proof}
1.) $\Rightarrow$ - direction: Let $A, j \in \Omega^1(\M)$ and let $A$ satisfy Proca's equation with $m >0$. We obtain the constraint (\ref{eqn:classical_constraint}) of the lemma by applying $\delta$ to Proca's equation and using $\delta^2 = 0$:
\begin{align}
(\delta d +m^2) A 		&= j \notag \\
\implies 			\delta(m^2 A) 				&= \delta j  \notag \\
\iff 		\delta A 							&= \frac{1}{m^2} \, \delta j \formspace.
\end{align}
By adding $0=d(0) = d(\delta A - \frac{1}{m^2}\delta j) = d\delta A - \frac{1}{m^2}d\delta j$ to Proca's equation one obtains the wave equation (\ref{eqn:classical_wave_eqation}) of the lemma:
\begin{align}
j =& \, (\delta d +m^2) A \notag \\
=& \, (\delta d + d \delta +m^2) A -\frac{1}{m^2} d \delta j \notag \\
\iff (\square +m^2)A =& \, j+ \frac{1}{m^2}d\delta j \formspace,
\end{align}
which completes the first direction of the proof.  \par
%
%
2.) $\Leftarrow$ - direction:
Let $A, j \in \Omega^1(\M)$ and let $A$ satisfy the wave equation (\ref{eqn:classical_wave_eqation}) of the lemma:
\begin{align}
(\square +m^2)A =& \, j+ \frac{1}{m^2}d\delta j \formspace.
\end{align}
Inserting the constraint (\ref{eqn:classical_constraint}) into the wave equation we obtain
\begin{align}
 (\delta d + d \delta +m^2) A &= \, j + \frac{1}{m^2} d \delta j \notag  \\
 \implies (\delta d +m^2) A + \frac{1}{m^2} d \delta j  &= j + \frac{1}{m^2} d \delta j \notag \\
 \iff (\delta d +m^2) A  &= j  \formspace,
\end{align}
which completes the proof.
\end{proof}
\noindent As argued in Section \ref{sec:sign_conventions}, the positive sign of the mass term in the wave equation (\ref{eqn:classical_wave_eqation}) is consistent with our conventions. Before we proceed with solving Proca's equation, the above lemma allows for a another brief comparison of Maxwell's and Proca's theory. It is well known that the Maxwell field possesses two independent degrees of freedom, known as the two independent polarization modes of the electromagnetic field. Starting with four independent components of the vector potential $A$ one reduces the degrees of freedom to three by implementing the Lorenz constraint $\delta A = 0$. One can show that this does not completely fix the gauge of the theory: As presented in the previous section, Maxwell's theory is independent of a gauge by exact forms\footnote{We will briefly discuss the different possibilities of choices of gauge of Maxwell's theory in manifolds with non-trivial topology at the end of Section \ref{sec:zero-mass-limit-existence-classical-vanishing-source}.}, that is, it is independent of the transformation $A \to A + d\chi$ for some zero-form $\chi$. Therefore, the exterior derivative transforms as $\delta A \to \delta A + \delta d \chi = \delta A + \square \chi$ as for every zero form it holds $\delta \chi = 0$. So the Lorenz constraint does not fix the gauge completely, as the theory is gauge invariant under addition of $\chi$ fulfilling the wave equation $\square \chi=0$. One can therefore further reduce the degrees of freedom to two. In an appropriate reference frame, the remaining polarization modes will be \emph{transversal}. As shown in the above lemma, in the Proca case one also has a Lorenz constraint (\ref{eqn:classical_constraint}) to solve, quite similar to the Lorenz constraint in Maxwell's theory, but no gauge equivalence relation to accommodate. Therefore, the Proca field has three independent polarization modes, which, in an appropriate reference frame, consist of two transversal and one longitudinal mode.\par
%
%
%
The procedure to solve Proca's equation is as follows: we have already successfully decomposed the equation into a wave equation with external sources and a Lorenz constraint. Next we will solve the wave equation in terms of fundamental solutions and then restrict the initial data in such a way that the constraint (\ref{eqn:classical_constraint}) of Lemma \ref{lem:Proca_wave_equiv} is also fulfilled. This will work as follows:
Assume $A\in \Omega^1(\M)$ solves the wave equation, $(\square +m^2)A =  j + \frac{1}{m^2} d \delta j$. Now, we observe
\begin{align}
(\square +m^2) \delta A
 &= \delta (\square + m^2) A \notag\\
&= \delta \left( j + \frac{1}{m^2} d \delta j \right) \notag\\
&= \delta  j + \frac{1}{m^2} \delta d \delta j + \frac{1}{m^2} d \delta \delta j \notag\\
&= (\square + m^2 ) \left (\frac{1}{m^2} \delta j \right)  \notag \\
\iff 0 &= (\square + m^2 )\left (\delta A - \frac{1}{m^2} \delta j \right)   \label{eqn:waveeqn_constraint}
\formspace,
\end{align}
where again we have used $\delta ^2 = 0$. The solution to the wave equation for $A$ therefore yields a Klein-Gordon equation for $(\delta A - \frac{1}{m^2} \delta j)$. Imposing initial data of $(\delta A - \frac{1}{m^2} \delta j)$ with respect to the Klein-Gordon equation that \emph{vanish} on the Cauchy surface $\Sigma$ implies a globally vanishing solution to the Klein-Gordon equation\footnote{Note that for any normally hyperbolic operator acting on sections of a vector bundle over a globally hyperbolic Lorentzian manifold, specifying vanishing initial data on a Cauchy surface yields a globally vanishing solution to the corresponding homogeneous differential equation (see \cite[Corollary 3.2.4]{baer_ginoux_pfaeffle}).} (\ref{eqn:waveeqn_constraint}) which is equivalent to $A$ globally fulfilling the Lorenz constraint (\ref{eqn:classical_constraint}).
To obtain a more self contained result, we then will re-express the vanishing initial data of $(\delta A -\frac{1}{m^2} \delta j)$ with respect to the Klein-Gordon equation in terms of initial data of $A$ with respect to the wave equation.\par
In conclusion: instead of \emph{globally} constraining $\delta A = \frac{1}{m^2} \delta j$ for a solution $A$ of the wave equation (\ref{eqn:classical_wave_eqation}) one can \emph{equivalently} state vanishing initial data of $(\delta A -\frac{1}{m^2} \delta j)$ on the Cauchy surface $\Sigma$ (with respect to a Klein-Gordan equation (\ref{eqn:waveeqn_constraint})) and restrict the initial data of $A$ (with respect to the wave equation (\ref{eqn:classical_wave_eqation})), such that they are compatible with the vanishing initial data of $\left(\delta A - \frac{1}{m^2} \delta j\right)$.
%
%
%
%
\newpage
\subsubsection{Solving the wave equation}
%The following is a generalization of the work by \name{Furlani} presented in \cite{FURLANI}, whose notation and procedure are also adopted.
Before solving the wave equation it is useful to define some operators, which will map a one-form $A \in \Omega^1(\M)$ to its initial data on $\Sigma$ with respect to the wave equation. In this form, these operators were first introduced by Dimock \cite{dimock1992quantizedEM} but are also used by Furlani \cite{FURLANI} and Pfenning \cite{pfenning}, allowing for a direct comparison of the obtained results, in particular when taking the zero mass limit.
%
\begin{definition}\label{def:cauchy_mapping_operators}
 Let $\gls{inclusionmap} : \Sigma \hookrightarrow \M$ the inclusion map of the Cauchy surface $\Sigma$ with pullback $i^*$.\\
 The operators $\rhoz, \rhod : \Omega^1(\M) \to \Omega^1(\Sigma)$ and $\rhon, \rhodelta : \Omega^1(\M) \to \Omega^0(\Sigma)$ are defined as:
 \begin{subequations}
  \begin{align}
  \gls{rhoz} 		&= i^* 										&\textrm{pullback,} \\
  \gls{rhod} 		&= -*_{(\Sigma)} i^* * d 		&\textrm{forward normal derivative,} \\
  \gls{rhodelta} 	&= i^*\delta 						&\textrm{pullback of the divergence,} \\
  \gls{rhon} 		&= -*_{(\Sigma)} i^* *  		&\textrm{forward normal.}
 \end{align}
 \end{subequations}
%
%
\end{definition}
These operators can be extended to act on arbitrary $p$-forms, as will be important when we will deal with the constraint (\ref{eqn:classical_constraint}). As we will see, the operators $\rho_{(\cdot)}$ not only map a one-form to its initial data with respect to the wave equation, but also map a zero-form to its initial data with respect to the Klein-Gordon equation. This will allow us to elegantly re-express the vanishing initial data of the constraint as mentioned in the previous section.
But first, we define a set of differential forms on $\Sigma$, which will turn out to be equivalent to the initial data of a solution $A$ to the wave equation.
%
\begin{definition}\label{def:cauchy_data_wave_eq}
 Let $A \in \Omega^1(\M)$ be a one-form on $\M$. The differential forms $\Az, \Ad \in \Omega^1(\Sigma)$ and $\An, \Adelta \in \Omega^0(\Sigma)$ are defined as:
 \begin{subequations}
  \begin{align}
  \Az &= \rhoz A \formspace,\\
  \Ad &= \rhod A \formspace, \\
  \An &= \rhon A \formspace, \\
  \Adelta &= \rhodelta A \formspace.
  \end{align}
 \end{subequations}
%
%
\end{definition}
Specifying these $p$-forms is equivalent to imposing the initial data $A_\mu$ and $n^\alpha \nabla_\alpha A_\mu$ on the Cauchy surface $\Sigma$ with future pointing unit normal vector field $n \in \Gamma(T\M)$ (see \cite[Chapter III]{FURLANI}). Therefore, in the following we will view $\Az,\Ad, \An, \Adelta$ as the initial data of $A$ with respect to the wave equation (\ref{eqn:classical_wave_eqation}).\par
The main key to finding a solution to the wave equation is the following lemma.
%
%
\begin{lemma}
 Let $A \in \Omega^1(\M)$ be a one-form, $F \in \Omega^1_0(\M)$ a test one-form and $m \geq 0$.
 Let $\Sigma$ denote a Cauchy surface of $\M$ with causal future/past  $\gls{sigmapm} \coloneqq J^\pm (\Sigma)$.
 Then it holds
 \begin{align}
  \int\limits_{\Sigma^\pm} &\Big[ A \wedge * \left( \square + m^2 \right) F - F \wedge *  \left( \square + m^2 \right) A \Big]  \label{eqn:greens_identity} \\
  &= \pm \Big\{ \langle \Az , \rhod F\rangle_\Sigma
  +\langle \Adelta , \rhon F\rangle_\Sigma
  -\langle \An , \rhodelta F\rangle_\Sigma
  -\langle \Ad , \rhoz F\rangle_\Sigma \Big\}   \formspace. \notag
 \end{align}
\label{lem:greens_identity}
\end{lemma}
%
%
\begin{proof}
 The proof is based on a proof outline from \cite[Appendix A]{FURLANI}. We start with Stoke's theorem for a $(p+1)$-dimensional sub-manifold $\mathcal{O} \subset \M$ with boundary $\partial \mathcal O$. Let $i : \partial\mathcal{O} \hookrightarrow \mathcal{O}$ be the inclusion operator. Then for any compactly supported $p$-form $\omega \in \Omega^p_0 (\M)$ it holds that (see Theorem \ref{thm:stokes})
 \begin{align}
  \int \limits_\mathcal O d\omega = \int \limits_{\partial \mathcal O} i^* \omega \formspace.
  \end{align}
Let $A \in \Omega^1(\M)$ and $F \in \Omega_0^1 (\M)$. We  define the compactly supported three-forms
\begin{subequations}
\begin{align}
 H' &= \delta F \wedge *A - \delta A \wedge *F \\
 H'' &= A \wedge * dF - F \wedge *dA \\
 H &=  H' -  H'' \formspace.
\end{align}
\end{subequations}
We will apply Stoke's theorem to the three-form $H$. For that we first need to calculate $dH$:
\begin{align}
 dH &=\; d\big( \delta F \wedge *A - \delta A \wedge *F \big) - d \big(A \wedge * dF - F \wedge *dA \big)  \notag \\
  &= d\delta F \wedge *A + (-1)^0 \delta F \wedge d{*A} - d\delta A \wedge *F - (-1)^0 \delta A \wedge d{*F} \notag \notag \\
   &\phantom{M}- dA \wedge *dF - (-1)^1 A \wedge (d{*{dF}}) + dF \wedge *{dA} + (-1)^1 F \wedge d{*{d A}} \notag \\
  &= A \wedge *d\delta F + \delta F \wedge d {* A} - F \wedge *{d\delta A} - \delta A \wedge d {* F }\notag \notag \\
   &\phantom{M}- {dA \wedge *dF} + A \wedge d{*{d F}}  + {dF \wedge * {dA}} -F \wedge d{*{dA }}
\end{align}
where we have used the linearity and graded Leibniz rule for the exterior derivative $d$, and the property $\alpha \wedge * \beta = \beta \wedge * \alpha$ of the wedge product and the Hodge star operator. With this, the terms $-{dA \wedge *dF}$ and ${dF \wedge * {dA}}$ cancel. Next, we use that for the test one-form $F$ it holds that $d{*dF} = (-1)^4 {*{*{d*}}} dF =  *\delta d F$ and analogously for $A$. We therefore get
\begin{align}
 dH &= A \wedge * ( d \delta F + \delta d F) - F \wedge * (d\delta A + \delta d A ) \notag \\
  &\phantom{M}+ \delta F \wedge d{*A} - \delta A \wedge d{* F} \label{eqn:greens_identity_zwischenergebnis}  \\
  &= A \wedge * \square F - F \wedge * \square A \formspace.
\end{align}
We have used in the last step that
\begin{align}
 \delta F \wedge d{* A}\; &=  \delta F \wedge {* *^{-1}}d{*A} \notag\\
 &=  - \delta F \wedge * \delta A \notag\\
 &=  - \delta A \wedge * \delta F \notag\\
 &=  \delta A \wedge {**^{-1}}{d*}F \notag\\
 &= \delta A \wedge d{*F} \formspace,
\end{align}
therefore the last two terms in equation (\ref{eqn:greens_identity_zwischenergebnis}) cancel. Furthermore we can add a vanishing term $0 = A \wedge * m^2 F - F \wedge * m^2 A$ to $dH$ to find the wanted relation by Stoke's theorem:
\begin{align}
 \int\limits_\mathcal O A \wedge * (\square + m^2) F - F \wedge * (\square + m^2) A
 =& \int\limits_{\partial O} i^*\left( \delta F \wedge *A + F \wedge *dA\right) \notag \\
 &- \int\limits_{\partial O} i^*\left( A \wedge *dF + \delta A \wedge *F\right)
\end{align}
Now, we specify $\mathcal O = \Sigma^\pm - \Sigma \Rightarrow \partial \mathcal O = \Sigma$ and note that with respect to integration on $\M$, the Cauchy surfaces denotes a set of measure zero, $\int_{\Sigma^\pm - \Sigma} = \int_{\Sigma^\pm}$. Furthermore, for Stoke's theorem to hold, we need to choose a consistent orientation of the boundary $\Sigma$ with respect to $\Sigma^\pm$. Following \cite[Appendix B.2]{wald_GR}, the orientation of $\Sigma^\pm$ induces a natural orientation on the boundary $\Sigma$. In the case of a Cauchy surface, coordinates in the neighborhood can be parametrized by one parameter $t$ such that $x_\mu=(t, x_1,x_2,x_3)$ where $x_i$ are (right handed) coordinates on $\Sigma$. For $\Sigma^+$ we get a right handed coordinate system in the natural way. For $\Sigma^-$ we get a right handed system if we flip the orientation of $\Sigma$, therefore we will get a relative sign for Stoke's theorem applied to $\mathcal O = \Sigma^-$, where we choose the natural standard orientation (induced by $\M$) on $\Sigma$. Therefore,  Stoke's Theorem reads $\int_{\Sigma^{\pm}} d\omega = \pm \int_{\Sigma} i^* \omega$. Specifying $\omega = H$ we find:
\begin{align}
 \int\limits_{\Sigma^\pm} A \wedge * (\square + m^2) F &- F \wedge * (\square + m^2) A   \\
 = \pm  \Bigg\{ &\int\limits_{\Sigma} i^* \delta F \wedge i^* *A + \int\limits_{\Sigma}  i^* F \wedge i^* *dA \notag \\
 &- \int\limits_{\Sigma} i^* A \wedge i^* *dF - \int\limits_{\Sigma}  i^* \delta A \wedge i^* *F  \Bigg\} \notag \\
 = \pm  \Bigg\{ &\int\limits_{\Sigma} \rhodelta F \wedge *_{(\Sigma)} *_{(\Sigma)} i^* *A + \rhoz F \wedge *_{(\Sigma)} *_{(\Sigma)} i^* *dA \notag \\
 &- \int\limits_{\Sigma} \Az \wedge *_{(\Sigma)} *_{(\Sigma)} i^* *dF -\int\limits_{\Sigma} \Adelta \wedge *_{(\Sigma)} *_{(\Sigma)} i^* *F  \Bigg\} \notag \\
 = \pm  \Bigg\{- &\int\limits_{\Sigma} \rhodelta F \wedge *_{(\Sigma)} \An -  \int\limits_{\Sigma} \rhoz F \wedge *_{(\Sigma)} \Ad \notag \\
 &+ \int\limits_{\Sigma} \Az \wedge *_{(\Sigma)} \rhod F +\int\limits_{\Sigma}  \Adelta \wedge *_{(\Sigma)} \rhon F  \Bigg\} \notag \\
 = \pm  \Big\{ &\langle \Az , \rhod F \rangle_\Sigma +\langle  \Adelta , \rhon F \rangle_\Sigma
 - \langle \rhodelta F , \An \rangle_\Sigma -  \langle \rhoz F , \Ad \rangle_\Sigma \Big\}  \notag
\end{align}
In the last steps we have made use of the Definitions \ref{def:cauchy_mapping_operators} and  \ref{def:cauchy_data_wave_eq}.
\end{proof}
%
To write down a solution to the wave equation (\ref{eqn:classical_wave_eqation}) we need to introduce the notion of fundamental solution of partial differential operators, in particular of the wave operator and, later, the Proca operator.
\begin{lemma}[Fundamental solutions of the wave operator]\label{lem:fundamental_solution_wave_operator}
Let $m \geq 0$. The wave operator  $(\square + m^2) : \Omega^p(\M) \to \Omega^p(\M)$ has unique advanced ($-$) and retarded ($+$) fundamental solutions $\gls{Epm} : \Omega^p_0(\M) \to \Omega^p(\M)$, which fulfill
\begin{subequations}
\begin{align}
 E_m^\pm (\square + m^2) &= \mathbbm 1 = (\square + m^2) E_m^\pm \formspace \text{\upshape{\quad\quad and}} \label{def:fundamental_solution} \\
 \supp{E_m^\pm F } &\subset J^\pm \big( \supp{F } \big) \formspace, \label{def:fundamental_solution_support}
\end{align}
\end{subequations}
where $F \in \Omega^p_0(\M)$ is a test $p$-form.
%
Furthermore, the fundamental solutions commute (or intertwine their action) with the exterior and interior derivative:
\begin{align}
E^\pm_m \delta &= \delta E^\pm_m \\
E^\pm_m d &= d E^\pm_m \formspace.
\end{align}
%
%
The advanced minus retarded fundamental solution is denoted by
\begin{align}
 \gls{Em} = E^-_m - E^{+}_{m} \formspace.
\end{align}
\end{lemma}
%%
%%
%%
\begin{proof}
The properties (\ref{def:fundamental_solution_support}) and (\ref{def:fundamental_solution}) for the fundamental solutions for any normally hyperbolic operator acting on smooth sections in a vector bundle over a Lorentzian manifold are proven in \cite[Corollary 3.4.3]{baer_ginoux_pfaeffle}.
We will here give a proof of the commutation with the exterior and interior derivative:
	\begin{figure}[h]
		\begin{center}
			\scalebox{0.9}{
%				\begin{tikzpicture}
%				\node at (0,0) {\includegraphics[scale=0.9]{./img/img2.pdf}};
%				\node at (1,3.5)  {$J^+\big( \supp{F} \big)$};
%				\node at (1,-3.5)  {$J^-\big( \supp{F} \big)$};
%				\node at (4.2,2.25)  {$\Sigma_+$};
%				\node at (4.2,-2.25)  {$\Sigma_-$};
%				\node at (2.9,-0.9)  {$\supp{F}$};
%				\end{tikzpicture}
				\begin{tikzpicture}
				\node at (0,0) {\includegraphics[scale=0.9]{./img/img2_b.pdf}};
				\node at (4,3.5)  {$J^+\big( \supp{F} \big)$};
				\node at (-4,-3.5)  {$J^-\big( \supp{F} \big)$};
				\node at (-8.3,2.2)  {$\Sigma_+$};
				\node at (8.3,-2.3)  {$\Sigma_-$};
				\node at (7,-0.9)  {$\supp{F}$};
				\node at (-7,-0.9)  {$\supp{F}$};
				\end{tikzpicture}
			}
		\end{center}
		\caption{Illustrating the setup of the proof of Lemma \ref{lem:fundamental_solution_wave_operator}: The Cauchy surfaces $\Sigma_\mp$ are chosen such that they have vanishing intersection with $J^\pm\big( \supp{F}\big)$ and thus $(\delta E^\pm_m - E^\pm_m \delta) F $ specifies vanishing initial data on the corresponding Cauchy surface.}
		\label{fig:fund_solutions}
	\end{figure}
Let $F \in \Omega^p_0(\M)$ be a test $p$-form and $E^\pm_m$ the fundamental solutions to the wave operator $(\square + m^2)$.
Then, from $\delta (\square + m^2) =(\square + m^2) \delta $ it follows:
\begin{align}
(\square + m^2)  E^\pm_m \delta F
&= \delta  F \notag \\
&= \delta (\square +m^2) E^\pm_m F \notag\\
&= (\square + m^2) \delta E^\pm_m F \notag \\
\implies (\square + m^2)(\delta E^\pm_m - E^\pm_m \delta) F &= 0 \formspace.
\end{align}
Since derivatives do not enlarge the support of a function they are acting on, we know
\begin{align}
\supp{\delta E^\pm_m F } &\subset \supp{E^\pm_m F } \subset J^\pm(\supp{F }) \formspace \textrm{and} \\
\supp{E^\pm_m \delta F } & \subset J^\pm(\supp{\delta F }) \subset J^\pm(\supp{F })\formspace.
\end{align}
Now, on a Cauchy surface $\Sigma_\mp$ in the past/future of $\supp{F }$ we specify initial data of $(\delta E^\pm_m - E^\pm_m \delta) F $ with respect to the wave operator $(\square + m^2)$  for the plus and minus sign  respectively. Because of the support property mentioned above, we know by construction that these initial data vanish. This is also illustrated in Figure \ref{fig:fund_solutions}. With respect to the homogeneous differential equation, specifying vanishing initial data on a Cauchy surface yields a globally vanishing solution (c.f. \cite[Corollary 3.2.4]{baer_ginoux_pfaeffle}), therefore for all $F \in \Omega^p_0(\M)$ it holds
\begin{align}
E^\pm_m \delta F= \delta E^\pm_m F\formspace.
\end{align}
Since also the exterior derivative $d$ does not extend the support of a function and commutes with the wave operator $(\square + m^2)$, the proof for the commutativity follows in complete analogy.
\end{proof}
%
%
With the notion of the fundamental solutions we can state a solution to the wave equation (\ref{eqn:classical_wave_eqation}) in form of the following theorem:
%
%
\begin{theorem}[Solution to the wave equation]
 Let $\Az,\Ad \in \Omega^1(\Sigma)$ and $\An,\Adelta \in \Omega^0(\Sigma)$ specify initial data of the one-form $A\in \Omega^1(\M)$ on the Cauchy surface $\Sigma$. Let $F \in \Omega^1_0(\M)$ be a test one-form and $\kappa \in \Omega^1(\M)$ be an external source.\\
 Then
 %
 \begin{align}
  \langle A, F \rangle_\M = \sum\limits_\pm \langle E_m^\mp F , \kappa \rangle_{\Sigma^\pm}
  &+ \langle \Az , \rhod E_m F \rangle_\Sigma
  + \langle \Adelta , \rhon E_m F \rangle_\Sigma \notag \\
  &- \langle \An , \rhodelta E_m F \rangle_\Sigma
  - \langle \Ad , \rhoz E_m F \rangle_\Sigma
 \end{align}
specifies the unique smooth solution of the wave equation $(\square + m^2)A= \kappa$, where $m \geq 0$, with the given initial data. Furthermore, the solution depends continuously on the initial data. \label{thm:solution_wave_equation}
 %
\end{theorem}
%
%
\begin{proof}
The proof follows directly from Lemma \ref{lem:greens_identity} by adding the equations for the plus and minus sign and specifying $F ' = E_m^\mp F $ (even though, for a compactly supported one form $F $,  $F '$ does not have compact support, we will see below that the integrals are still well defined and Lemma \ref{lem:greens_identity} is applicable): \\
1.) Let $A\in \Omega^1(\M)$ be a smooth solution to the wave equation $(\square + m^2)A = \kappa$ and $F ' \in \Omega^1(\M)$ a test one-form. Then the LHS of equation (\ref{eqn:greens_identity}) reads
\begin{align}
 \int\limits_{\Sigma^\pm} \Big[ A \wedge * \left( \square + m^2 \right) F ' - F ' \wedge * \left( \square + m^2 \right) A \Big]
&=  \int\limits_{\Sigma^\pm} \Big[ A \wedge * \left( \square + m^2 \right) F ' - F ' \wedge * \kappa \Big] \notag \\
&= \int\limits_{\Sigma^\pm} \Big[ A \wedge * \left( \square + m^2 \right) E_m^\mp F - E_m^\mp F \wedge * \kappa \Big]  \notag \\
&= \int\limits_{\Sigma^\pm} \Big[ A \wedge *F - E_m^\mp F \wedge * \kappa \Big]  \formspace,
\end{align}
where we have  substituted $F ' = E_m^\mp F $ for the regions $\Sigma^\pm$ respectively, such that, because $\supp{E_m^\mp F } \subset J^\mp\big(  \supp{F }\big)$, $\supp{E_m^\mp F } \cap \Sigma^\pm$ is compact  and therefore the integrals are well defined and Lemma \ref{lem:greens_identity} is indeed applicable\footnote{ We have also commented on this in the proof of Stoke's Theorem \ref{thm:stokes}.}. If we now add the left hand side equations for the plus and the minus sign respectively, we obtain
\begin{align}
\textrm{LHS}^+ + \textrm{LHS}^- &= \int\limits_{\Sigma^+}  A \wedge *F
- \int\limits_{\Sigma^+}  E_m^- F \wedge * \kappa
+\int\limits_{\Sigma^-}  A \wedge *F
- \int\limits_{\Sigma^-}  E_m^+ F \wedge * \kappa \notag \\
&= \int\limits_{\M}  A \wedge *F - \sum\limits_\pm \int\limits_{\Sigma^\pm}  E_m^\mp F \wedge * \kappa \formspace.
\end{align}
We have used that, with respect to an integration over the manifold $\M$, the Cauchy surface $\Sigma$ denotes a set of zero measure and therefore it holds for any compactly supported $4$-form $\alpha$
\begin{align}
\int \limits_{\M} \alpha = \int \limits_{\Sigma^+} \alpha  + \int\limits_{\Sigma^-} \alpha \formspace.
\end{align}
2.) For the right hand side we identify $E_m = E_m^- - E_m^+$ and directly obtain, after summing the equations for the two signs:
\begin{align}
\textrm{RHS}^+ + \textrm{RHS}^-
=   &\,\langle \Az , \rhod E_m F \rangle_\Sigma
  + \langle \Adelta , \rhon E_m F \rangle_\Sigma \notag \\
  & -\langle \An , \rhodelta E_m F \rangle_\Sigma
  - \langle \Ad , \rhoz E_m F \rangle_\Sigma \formspace.
\end{align}
Bringing the factor containing $\kappa$ from the left to the right hand side one finds the wanted relation. Uniqueness of the solution, in a distributional sense, and continuous dependence on the initial data follows from \cite[Theorem, 3.2.12]{baer_ginoux_pfaeffle} which is generalized for non-compactly supported initial data in \cite[Theorem 2.3]{Sanders}.
\end{proof}
%
%
%
%
%
%
%
%
%
%
\subsubsection{Implementing the Lorenz constraint}\vspace{-5mm}
Now that we have found a solution to the wave equation (\ref{eqn:classical_wave_eqation}) we would like to restrict the initial data, such that the Lorenz constraint (\ref{eqn:classical_constraint}) is fulfilled and we therefore have a solution to Proca's equation. Again, we do this by restricting the initial data of $A$ such that $(\delta A - \frac{1}{m^2} \delta j)$ has vanishing initial data on a Cauchy surface $\Sigma$. It is useful to express the initial data of the zero-form $(\delta A - \frac{1}{m^2} \delta j)$ in terms of the operators $\rho_{(\cdot)}$. For that we need to extend these operators to act on zero-forms. By using the same Definition \ref{def:cauchy_mapping_operators} but letting the operators act on zero-forms we find that:
\begin{lemma}\label{lem:klein_gordon_cauchy_data}
Let $\Sigma$ be a Cauchy surface with unit normal vector field $n$. For any smooth zero-form $f\in \Omega^0(\M)$ it holds that
\begin{subequations}
\begin{align}
\rhon f &= 0 \formspace, \\
\rhodelta f &= 0 \formspace, \\
\rhoz f &= \restr{f}{\Sigma} \formspace, \\
\rhod f &= \restr{(df)(n)}{\Sigma} = \restr{\left(n^\alpha \nabla_\alpha f\right)}{\Sigma} \formspace.
\end{align}
\end{subequations}
Therefore, with respect to the Klein Gordon equation, $\rhoz f$ and $\rhod f$ specify initial data on $\Sigma$.
\end{lemma}
%
\begin{proof}
1.) We begin with the first equation and chose local coordinates such that  the natural inclusion $i: \Sigma \hookrightarrow \M$ maps $i: (\vector x) \mapsto (0, \vector x)$. In those coordinates the pullback of a $p$-form $T$ is given by $i^*\left( T\indices{_{\alpha_1 \dots \alpha_p}}\right) = T\indices{_{a_1 \dots a_p}}$, where as usual Greek letters take values in $\{0,1,2,3\}$ and Latin letters in $\{1,2,3\}$. From this we find
\begin{align}
\rhon f
&= - *_{(\Sigma)} i^* * f \notag \\
&= - *_{(\Sigma)} i^* \left( \sqrt{\abs{g\indices{_{\mu \nu}}}}\, f\, \epsilon_{\alpha \beta \gamma \delta} \, dx^\alpha \otimes dx^\beta \otimes dx^\gamma \otimes dx^\delta \right) \notag \\
&=  *_{(\Sigma)}  \sqrt{\abs{g\indices{_{\mu \nu}}}}\, f\, \underbrace{\epsilon_{ijkl} \, dx^i \otimes dx^j \otimes dx^k \otimes dx^l}_{=0} \notag \\
&= 0 \formspace .
\end{align}
%
2.) The second equation follows from, using that for any zero-form $f$ it holds ${\delta f} = 0$,
\begin{align}
\rhodelta f
&= i^* \delta f \notag \\
&= 0 \formspace.
\end{align}
%
3.) It follows directly from the definition of $\rhoz$ and the inclusion operator $i$ that
\begin{align}
\rhoz f = i^* f = \restr{f}{\Sigma} \formspace.
\end{align}
%
4.) The action of $\rhod$ on a zero form $f$ is equivalent to the action of $\rhon$ on the one form $df$. For any one-form $A$, the action of $\rhon$ is shown by Furlani \cite[Appendix A]{FURLANI} to be $\rhon A = n^\alpha A_\alpha$. Therefore, we find:
\begin{align}
\rhod f
&= - *_{(\Sigma)} {i^* * }(d f) \notag  \\
&= \rhon df \notag \\
&= n^\alpha (df)_\alpha \notag \\
&= n^\alpha \nabla_\alpha f \formspace.
\end{align}
\end{proof}
%
%
%
%%
%
%
%
%
%
In this section we need to introduce some additional properties regarding the properties of the normal vector field $n$ of the Cauchy surface $\Sigma$.
\begin{lemma}[Gaussian Coordinates] \label{lem:normal_vectors}
Let $\Sigma$ be a Cauchy surface of $\M$ with future pointing unit normal vector field $n$. We can extend $n$ to a neighborhood of $\Sigma$ such that the following holds:
%
\begin{align}
n^\alpha \nabla_\alpha n^\beta = 0 \formspace, \label{eqn:gauss_coordinates_geodesic}\\
dn = 2 \nabla_{[\mu} n_{\nu]} = 0 \formspace. \label{eqn:gauss_coordinates_symmetric}
\end{align}
\end{lemma}
\begin{proof}
	An introduction to Gaussian (normal) coordinates is for example given in \cite[pp. 42,43]{wald_GR} or \cite[pp. 445,446]{carroll_spacetime-and-gr} where equation (\ref{eqn:gauss_coordinates_geodesic}) is shown to hold by construction. Equation (\ref{eqn:gauss_coordinates_symmetric}) can be derived by using Frobenius' theorem (see for example \cite[Theorem B.3.1 and B.3.2]{wald_GR}) as argued in \cite[Section 2.3.2 Equation 5]{Sanders}.
\end{proof}
With this normal vector field we can write the metric $g$ of the spacetime $\M$ in a neighborhood of the Cauchy surface as (see \cite[Equation 10.2.10]{wald_GR})
\begin{align}
g\indices{_{\mu\nu}} = - n_\mu n_\nu + h\indices{_{\mu\nu}} \formspace,
\end{align}
where $h$ denotes the induced metric on $\Sigma$.
Having introduced this notion, we can state the final result of this chapter with the following theorem:
%
%
%
\begin{theorem}[Solution of Proca's equation - constrained version] \label{thm:solution_proca_constrained}
 Let $\Az,\Ad \in \Omega^1(\Sigma)$ and $\An, \Adelta \in \Omega^0(\Sigma)$ specify initial data on the Cauchy surface $\Sigma$. Let $F \in \Omega^1_0(\M)$ be a test one-form, $j \in \Omega^1(\M)$ an external source and $m > 0$ a positive constant.\\
 Restrict the initial data by specifying
 \begin{subequations}\label{eqn:proca_constraints}
  \begin{align}
 \Adelta &= \frac{1}{m^2} \rhodelta j \\
 m^2 \, \An - \delta_{(\Sigma)}\Ad &= \rhon j  \formspace.
 \end{align}
 \end{subequations}
 Then
 %
 \begin{align}
  \langle A, F \rangle_\M = \sum\limits_\pm \langle j + \frac{1}{m^2}d\delta j, E_m^\mp F   \rangle_{\Sigma^\pm}
  &+ \langle \Az , \rhod E_m F \rangle_\Sigma
  + \langle \Adelta , \rhon E_m F \rangle_\Sigma \notag \\
  &- \langle \An , \rhodelta E_m F \rangle_\Sigma
  - \langle \Ad , \rhoz E_m F \rangle_\Sigma \label{eqn:solution_proca_constrained}
 \end{align}
specifies the unique smooth solution of Proca's equation $\left( \delta d + m^2 \right) A = j$ with the given initial data. Furthermore,  the solution depends continuously on the initial data.
 %
\end{theorem}
\begin{proof}
From Theorem \ref{thm:solution_wave_equation} we know that equation (\ref{eqn:solution_proca_constrained}) specifies the unique smooth solution to the wave equation (\ref{eqn:classical_wave_eqation}). We are left with showing that the specified constraints on the initial data are equivalent to the vanishing initial data of $(\delta A - \frac{1}{m^2} \delta j)$ on the Cauchy surface which, as we have seen, is equivalent to the constraint (\ref{eqn:classical_constraint}).
We have to look at the initial data with respect to the Klein-Gordon equation as stated in Lemma \ref{lem:klein_gordon_cauchy_data}. In the following, $i : \Sigma \hookrightarrow \M$ denotes again the inclusion operator.\\
1.) The vanishing of initial value yields
\begin{align}
0 &= \rhoz \left( \delta A - \frac{1}{m^2}\, \delta j\right) \notag\\
&= i^* \delta A - \frac{1}{m^2}\, i^* \delta j \notag \\
&= \rhodelta A - \frac{1}{m^2} \rhodelta j \notag \\
\Leftrightarrow \Adelta &= \frac{1}{m^2} \rhodelta j \formspace.
\end{align}
We have used the linearity of the pullback and Definition \ref{def:cauchy_mapping_operators} of the initial data mapping operator $\rhodelta$. \par
%
%
%
2.) We will calculate the vanishing of the normal derivative in Gaussian normal coordinates and in the end turn back to a coordinate independent notation:
\begin{align}
0 &= \rhod \left( \delta A - \frac{1}{m^2}\, \delta j\right) \notag \\
 &= \rhod \delta A  - \frac{1}{m^2} \rhod  \delta j \notag \\
 &= \restr{\Big[n^\alpha \nabla_\alpha  \delta A \Big]}{\Sigma} - \frac{1}{m^2} \rhod  \delta j  \formspace. \label{eqn:cauchy_normal_derivative_tmp}
\end{align}
We will take a separate look at the first summand:
\begin{align}
n^\alpha \nabla_\alpha  \delta A
&= n^\alpha \left( d\delta A \right)_{\alpha} \notag\\
&= n^\alpha \square A_\alpha - n^\beta \left( \delta  d A \right)_\beta \notag\\
&= n^\alpha \kappa_\alpha -{m^2} \, n^\mu A_\mu + n^\beta \nabla^\nu \nabla_{[\nu} A_{\beta]} \notag\\
&= n^\alpha \kappa_\alpha - {m^2} \, n^\mu A_\mu +  \nabla^\nu \left( n^\beta  \nabla_{[\nu} A_{\beta]} \right) \formspace ,
\end{align}
where we have used that $\left( \nabla^\nu n^\beta \right) \left(  \nabla_{[\nu} A_{\beta]} \right) = 0 $, since from Lemma  \ref{lem:normal_vectors} we know that $\nabla^\nu n^\beta $ is symmetric, and every contraction of a fully symmetric with a fully antisymmetric tensor vanishes. Next we express the metric $g$ as $g\indices{_{\mu\nu}} = -n_\mu n_\nu + h\indices{_{\mu\nu}}$ and use the expansion of the normal vectors as geodesics as stated in Lemma \ref{lem:normal_vectors}. Therefore we further find:
\begin{align}
n^\alpha \nabla_\alpha  \delta A
&= n^\alpha \kappa_\alpha - {m^2} \, n^\mu A_\mu +  g\indices{^{\sigma\nu}} \nabla_\sigma \left( n^\beta  \nabla_{[\nu} A_{\beta]} \right) \notag\\
&= n^\alpha \kappa_\alpha - {m^2} \, n^\mu A_\mu +  \left(- n^\sigma n^\nu +  h\indices{^{\sigma\nu}} \right) \nabla_\sigma \left( n^\beta  \nabla_{[\nu} A_{\beta]} \right) \notag \\
&= n^\alpha \kappa_\alpha - {m^2} \, n^\mu A_\mu  + \nabla_{(\Sigma)}^{\nu} \left( n^\beta  \nabla_{[\nu} A_{\beta]} \right) \formspace.
\end{align}
Here we have made use of the identification $ h\indices{^{\sigma\nu}} \nabla_\sigma B_\mu= \nabla_{(\Sigma)}^\nu  B_\mu$ for any one-form $B$ (see \cite[Lemma 10.2.1]{wald_GR}).
Now, we can identify $n^\beta  \nabla_{[\nu} A_{\beta]} = - n^\beta  \nabla_{[\beta} A_{\nu]} =- A_{(d)\nu}$ and use the local coordinate representation of the exterior derivative of a one-form $B$ on the Cauchy surface $\Sigma$, $\delta_{(\Sigma)} B =  -\nabla_{(\Sigma)} ^\alpha B_\alpha$, and finally obtain
\begin{align}
n^\alpha \nabla_\alpha  \delta A
&=n^\alpha \kappa_\alpha - m^2 \, n^\mu A_\mu  + \nabla_{(\Sigma)}^\nu \left( n^\beta  \nabla_{[\nu} A_{\beta]} \right) \notag \\
&=\rhon \kappa - m^2 \An  + \delta_{(\Sigma)} \Ad \formspace.
\end{align}
In the last step we have taken the restriction to the Cauchy surface, as we are interested in the initial data on the Cauchy surface.
Now, looking back at the normal derivative (\ref{eqn:cauchy_normal_derivative_tmp}) and inserting the definition of the source term $\kappa = j + \frac{1}{m^2}d \delta j$, we find
\begin{align}
 m^2 \An  - \delta_{(\Sigma)} \Ad
 &= \rhon \kappa - \frac{1}{m^2} \rhod \delta j \notag\\
&= \rhon j + \frac{1}{m^2} \rhon d \delta j - \frac{1}{m^2} \rhod \delta j \notag\\
&=  \rhon j \formspace,
\end{align}
since for any $p$-form $B$ it holds by definition $\rhon dB = \rhod B$.\\
By Lemma \ref{lem:Proca_wave_equiv} we have thus found the solution of Proca's equation.
\end{proof}
For further calculations it is useful to develop an unconstrained solution to Proca's equation from the previous theorem.
Basically, instead of considering the full space of initial data for the solution and constraining them, we just take initial data living in a subspace of all initial data and change the dependency of the solution on the data in such a way that this subspace automatically solves the constraints.
Before we state that theorem, we need to introduce fundamental solutions for the Proca operator $\delta d + m^2$ and relate them to the fundamental solutions of the wave operator that we have encountered so far.
%
%
\begin{lemma}[Fundamental solutions of the Proca operator]\label{lem:fundamental_solution_proca_operator}
The Proca operator  $(\delta d + m^2) : \Omega^p(\M) \to \Omega^p(\M)$, for $m > 0$,  has unique advanced ($-$) and retarded ($+$) fundamental solutions $\gls{Gpm} : \Omega^p_0(\M) \to \Omega^p(\M)$ that are given by
\begin{align}
G_m^\pm = \left( \frac{d \delta}{m^2} + 1\right) E_m^\pm \formspace,
\end{align}
where $E^\pm_m$ are the advanced and retarded fundamental solutions to the wave operator $(\square + m^2)$. \\
They fulfill the properties
\begin{subequations}
\begin{align}
 G_m^\pm (\delta d + m^2) &= \mathbbm 1 = (\delta d + m^2) G_m^\pm \formspace \text{\upshape{\quad\quad and}} \label{def:fundamental_solution_proca_operator} \\
 \supp{G_m^\pm F } &\subset J^\pm \supp{F } \label{def:fundamental_solution_proca_support} \formspace.
\end{align}
\end{subequations}
The advanced minus retarded fundamental solution is denoted by
\begin{align}
\gls{Gm} = G_m^- - G_m^+ \formspace.
\end{align}
\end{lemma}
%
\begin{proof}
	First note that existence of $G_m^\pm$ follows from existence of $E_m^\pm$ as stated in Lemma \ref{lem:fundamental_solution_wave_operator}. The quasi inverse property is easily proven by using $\delta ^2 = 0 = d^2$ and the properties for the fundamental solutions $E_m^\pm$ as stated in Lemma \ref{lem:fundamental_solution_wave_operator}. Let $F \in \Omega^p_0(\M)$, then
\begin{align}
(\delta d +m^2) G^\pm_m F
&= (\delta d +m^2)\left( \frac{d \delta}{m^2} + 1\right) E_m^\pm F \notag\\
&= (\delta d +d \delta +m^2) E_m^\pm F \notag\\
&= F
\end{align}
and
\begin{align}
G^\pm_m (\delta d +m^2)  F
&=\left( \frac{d \delta}{m^2} + 1\right) E_m^\pm (\delta d +m^2) F \notag \\
&=\left( \frac{d \delta}{m^2} + 1\right) (\delta d +m^2)E_m^\pm F\notag \\
&= (\delta d +d \delta +m^2) E_m^\pm F \notag\\
&= F \formspace.
\end{align}
The support property follows directly from the support property of the fundamental solutions to $(\square + m^2)$ as stated in Lemma \ref{lem:fundamental_solution_wave_operator}. Again, we use that derivatives do not increase the support of a function. Therefore:
\begin{align}
\supp{G^\pm_m  F }
&= \supp{\left( \frac{d \delta}{m^2} + 1\right) E_m^\pm  F } \notag\\
%&= \supp{E_m^\pm \left( \frac{d \delta}{m^2} + 1\right)  F } \\
&\subset \supp{E_m^\pm F } \notag\\
&\subset J^\pm(\supp{F }) \formspace.
\end{align}
For the proof of uniqueness, let $G^\pm_m$ and $G'^\pm_m$ denote two fundamental solutions to the Proca operator. Let $F \in \Omega^p_0(\M)$ and define $A^\pm = (G^\pm_m - G'^\pm_m)F $.
Then:
\begin{align}
(\delta d + m^2) A^\pm = 0 \notag\\
\implies \delta A^\pm = 0 \notag\\
\implies (\square + m^2)A^\pm = 0.
\end{align}
Since $\supp{A^\pm} \subset J^\pm(\supp{F })$, we know that initial data of $A^\pm$ vanishes on every Cauchy surface in the past/future of $\supp{F }$ for the $+/-$ sign respectively. By the same argument as used in the proof of Lemma \ref{lem:fundamental_solution_wave_operator}, we find a globally vanishing solution (again, see \cite[Corollary 3.2.4]{baer_ginoux_pfaeffle}), that is,
\begin{align}
A^\pm &= 0 \notag\\
\implies G^\pm_m &= G'^\pm_m \formspace,
\end{align}
since $F $ is arbitrary. This completes the proof.
\end{proof}
%
%
%
%
Now we are ready to state the final, unconstrained version of the solution of Proca's equation.This following theorem will by the cornerstone of the calculations in the next chapter:
%
%
%
\begin{theorem}[Solution of Proca's equation - unconstrained version]\label{thm:solution_proca_unconstrained}
 Let $\Az,\Ad \in \Omega^1(\Sigma)$ specify a subset of initial data on the Cauchy surface $\Sigma$. Let $F \in \Omega^1_0(\M)$ be a test one-form, $j \in \Omega^1(\M)$ an external source and $m >0$ a positive constant.
 Then
 %
 \begin{align}
  \langle A, F \rangle_\M = \sum\limits_\pm \langle j , G_m^\mp F   \rangle_{\Sigma^\pm}
  + \langle \Az , \rhod G_m F \rangle_\Sigma
 - \langle \Ad , \rhoz G_m F \rangle_\Sigma \label{eqn:solution_proca_unconstrained}
 \end{align}
specifies the unique smooth solution of Proca's equation $\left( \delta d + m^2 \right) A = j$ with the given subset of initial data. Furthermore,  the solution depends continuously on the initial data.
 %
\end{theorem}
%
%
%
\begin{proof}
The theorem follows from Theorem \ref{thm:solution_proca_constrained} by inserting the constraints (\ref{eqn:proca_constraints}) into the expression (\ref{eqn:solution_proca_constrained}).
We find
%
 \begin{align} \label{eqn:tmp_proca_solution_constraint_inserted}
  \langle A, F \rangle_\M =
  & \sum\limits_\pm \langle j + \frac{1}{m^2}d\delta j, E_m^\mp F   \rangle_{\Sigma^\pm}
  + \langle \Az , \rhod E_m F \rangle_\Sigma
  + \frac{1}{m^2} \langle  \rhodelta j, \rhon E_m F \rangle_\Sigma \notag \\
  & - \langle \frac{1}{m^2} \delta_{(\Sigma)} \Ad , \rhodelta E_m F \rangle_\Sigma
  - \frac{1}{m^2} \langle \rhon j, \rhodelta E_m F \rangle_\Sigma
  - \langle \Ad, \rhoz E_m F \rangle_\Sigma \formspace.
 \end{align}
%
Now, for clarity's sake, we take a look at the appearing terms separately:\\
1.) To get rid of the appearing divergence of $\Ad$, we use some basic identities, that is, in particular the formal adjointness of $\delta$ and $d$ and the commutativity of $d$ with the pullback $i^*$:
\begin{align}
\langle \delta_{(\Sigma)} \Ad, \rhodelta E_m F \rangle_\Sigma
=& \langle \Ad, d_{(\Sigma)} \rhodelta E_m F  \rangle_\Sigma \notag\\
=& \langle \Ad, d_{(\Sigma)} i^* \delta  E_m F \rangle_\Sigma \notag\\
=& \langle \Ad,  i^* d \delta  E_m F \rangle_\Sigma \notag \\
=& \langle \Ad, \rhoz d \delta  E_m F \rangle_\Sigma \formspace,
\end{align}
which, together with $\langle \Ad, \rhoz E_m F \rangle_\Sigma$,  combines to $\langle \Ad , \rhoz \left( \frac{d \delta}{m^2} +1 \right) E_m F \rangle_\Sigma$. \par
%
2.) Next, we have a look at a part of the sum term and use Stoke's theorem (again, we get a sign due to the orientation of $\Sigma$ with respect to $\Sigma^\pm$) for a formal partial integration, at the cost of some boundary terms:
\begin{align}
\sum\limits_\pm \langle d \delta j ,  E_m^\mp F   \rangle_{\Sigma^\pm}
&= \sum\limits_\pm  \int\limits_{\Sigma^\pm}  \left( d \delta j \wedge {* E_m^\mp} F \right) \notag \\
&= \sum\limits_\pm  \int\limits_{\Sigma^\pm} \left\{ d \left(  \delta j \wedge {* E_m^\mp} F \right) - \delta j \wedge d {* E_m^\mp} F \right\} \notag\\
&= \sum\limits_\pm \left\{ \phantom{\pm} \int\limits_{\Sigma^\pm}  d \left(  \delta j \wedge * E_m^\mp F \right) + \int\limits_{\Sigma^\pm} \delta j \wedge {**} {d {* E_m^\mp}} F \right\} \notag\\
&= \sum\limits_\pm \left\{  \pm \int\limits_{\Sigma}  i^*{\left(  \delta j \wedge * E_m^\mp F \right) }+ \int\limits_{\Sigma^\pm} \delta j \wedge * \delta E_m^\mp F \right\} \notag\\
&= \sum\limits_\pm \left\{  \pm \int\limits_{\Sigma}  i^*{\left(  \delta j \wedge * E_m^\mp F \right) }+ \int\limits_{\Sigma^\pm} \delta E_m^\mp F \wedge {* *}{d*}j \right\} \notag\\
&= \sum\limits_\pm \left\{  \pm \int\limits_{\Sigma}  i^*{\left(  \delta j \wedge * E_m^\mp F \right)} - \int\limits_{\Sigma^\pm} \delta E_m^\mp F \wedge {d*}j \right\} \notag \\
&= \sum\limits_\pm \left\{  \pm \int\limits_{\Sigma}  i^*{\left(  \delta j \wedge * E_m^\mp F \right) }- \int\limits_{\Sigma^\pm} \Big( d \left( \delta E_m^\mp F \wedge *j \right) - d \delta E_m^\mp F \wedge *j \Big) \right\} \notag \\
&= \sum\limits_\pm \left\{  \pm \int\limits_{\Sigma}  i^*{\left(  \delta j \wedge * E_m^\mp F \right)} \mp \int\limits_{\Sigma} i^*{\left( \delta E_m^\mp F \wedge *j \right) }+ \int\limits_{\Sigma^\pm} j \wedge *d \delta E_m^\mp F  \right\} \notag \\
&= \sum\limits_\pm \langle  j , d \delta E_m^\mp F \rangle_{\Sigma^\pm}  + \int\limits_{\Sigma}  i^{*}{\left(  \delta j \wedge * E_m F \right)} - \int\limits_{\Sigma} i^*{\left(j \wedge * \delta E_m F \right)} \label{eqn:tmp_proca_boundary_terms}
\end{align}
Lastly, we will see that the remaining source dependent terms of (\ref{eqn:tmp_proca_solution_constraint_inserted}) will cancel with the boundary terms obtained from partial integration above:
\begin{align}
 \langle  \rhodelta j, \rhon E_m F \rangle_\Sigma \notag  -  \langle \rhon &j, \rhodelta E_m F \rangle_\Sigma \notag\\
 &= - \langle  i^* \delta  j, *_{(\Sigma)}{ i^* *}E_m F \rangle_\Sigma
 +  \langle *_{(\Sigma)} {i^* *}  j, i^* \delta  E_m F \rangle_\Sigma  \notag\\
 &= -\int\limits_{\Sigma}  i^* \delta  j \wedge *_{(\Sigma)} {*_{(\Sigma)}{ i^* {*E_m F }}}
 +  \int\limits_{\Sigma}  i^* \delta  E_m F \wedge  {*_{(\Sigma)}  {*_{(\Sigma)}{ i^* {*  j} }}}    \notag\\
  &=-\int\limits_{\Sigma}  i^* {\left( \delta  j \wedge {*E_m F } \right) }  +  \int\limits_{\Sigma}  i^* {\left( \delta  E_m F \wedge  {*  j}   \right) }\formspace.
\end{align}
These terms cancel out the boundary terms in (\ref{eqn:tmp_proca_boundary_terms}) (note that all of them have a prefactor $\frac{1}{m^2}$ that was not carried along the calculation for simplicity). Therefore, we obtain the result
 \begin{align}
  \langle A, F \rangle_\M = \sum\limits_\pm \langle j , \left( \frac{d \delta}{m^2} +1 \right) E_m^\mp F   \rangle_{\Sigma^\pm}
  &+ \langle \Az , \rhod E_m F \rangle_\Sigma \notag \\
  &- \langle \Ad , \rhoz \left( \frac{d \delta}{m^2} +1 \right) E_m F \rangle_\Sigma \formspace.
 \end{align}
 Now, to complete the proof, we make use of the identity $G_m = \left( \frac{d \delta}{m^2} +1 \right) E_m$, see Lemma \ref{lem:fundamental_solution_proca_operator}, and a simple calculation additionally gives, using $d^2 =0$:
 \begin{align}
 \rhod G_m = - {*_{(\Sigma)} {i^* *}} d\left( \frac{d \delta}{m^2} +1 \right) E_m = - {*_{(\Sigma)}{i^* *}} E_m = \rhod E_m \formspace,
 \end{align}
 which completes the proof.
\end{proof}
At this stage, one might wonder how this result compares to the discussed fact that the Proca field only possesses three independent degrees of freedom as discussed at the beginning in Section \ref{sec:solving_procas_equation}. In the formalism that we work with, the counting of degrees of freedom is a bit subtle. We have started with a solution to a wave equation, clearly possessing four independent degrees of freedom expressed in the initial data formulation by $\Az, \Ad, \Adelta$ and $\An$. To obtain a solution to Proca's equation, we have implemented a Lorenz constraint by restricting the initial data. In the last step, concluding in Theorem \ref{thm:solution_proca_unconstrained}, we have effectively eliminated the initial zero-forms $\Adelta$ and $\An$ on the Cauchy surface $\Sigma$. But those two zero forms can be viewed as initial data to a scalar Klein-Gordon field! In that sense, we have eliminated one scalar degree of freedom, and are left with the ``correct'' three independent degrees of freedom in Proca's theory.
%
%
\subsection{The zero mass limit}\label{sec:zero-mass-limit-classical}
As a basis for understanding the zero mass limit in the quantum case, we will now investigate the corresponding classical limit. The question is for which test one-forms the zero mass limit of distributional solutions to the Proca equation exists, or more precisely:
\begin{center}\textit{
		Let $A_m$ be a solution to the Proca equation with mass $m$,\\ and $\Az,\Ad \in \Omega^1(\Sigma)$ its initial data.\\ For which $F \in \Omega^1_0(\M)$, if any, does the limit $\lim\limits_{m\to 0^+} \langle A_m , F \rangle_\M = \lim\limits_{m \to 0^+}\Big( \sum\limits_\pm \langle j , G_m^\mp F   \rangle_{\Sigma^\pm} +\langle \Az , \rhod G_m F \rangle_\Sigma
		- \langle \Ad , \rhoz G_m F \rangle_\Sigma \Big)$ exist?}
\end{center}
We have used the explicit form of distributional solutions to the Proca equation as presented in Theorem \ref{thm:solution_proca_unconstrained}.
Before we can answer this question, we need to make an assumption regarding the continuity of the propagators of the Proca operator with respect to the mass. As we have specified the propagator for the Proca operator in terms of the propagator of the Klein-Gordon operator, we will state it in the following way:
\begin{assumption}\label{ass:propagator_continuity}
	Let $m\geq 0$ and $E_m^\pm$ the fundamental solutions to the Klein-Gordon operator $(\delta d + d \delta + m^2)$. Then, for a fixed $F\in \Omega^1_0(\M)$,
	\begin{align}
		m \mapsto E^\pm_m F
	\end{align}
	is continuous. Therefore
		\begin{align}
		m \mapsto E_m F
		\end{align}
	is continuous and
	\begin{align}
	\lim\limits_{m \to 0^+} E^\pm_m F &= E^\pm_0 F \quad \text{and} \\
		\lim\limits_{m \to 0^+} E_m F &= E_0 F \formspace.
	\end{align}
\end{assumption}
This assumption remains unproven in the context of this thesis. With this, continuity of $G_m F$ for a fixed test one-form $F$ follows directly for $m>0$. Using the assumption, we can now investigate the zero mass limit.
We will split this up into the case of vanishing external sources, $j=0$, and the general case with sources for clarity.
%
%
%
%
%
%
%
%
\subsubsection{Existence of the limit in the current-free case}\label{sec:zero-mass-limit-existence-classical-vanishing-source}
Let $A_m$ specify a solution to Proca's equation with mass $m$ and vanishing external sources $j=0$. Recall that by Theorem \ref{thm:solution_proca_unconstrained} a solution to Proca's equation with mass $m$ is uniquely determined by initial data $\Az, \Ad \in \Omega^1(\Sigma)$ by
 \begin{align}
\langle A_m, F \rangle_\M = \langle \Az , \rhod G_m F \rangle_\Sigma
 - \langle \Ad , \rhoz G_m F \rangle_\Sigma \formspace,
 \end{align}
where $G_m = \frac{1}{m^2} \left( d \delta + m^2 \right) E_m$. From this expression, we want to find necessary and sufficient conditions for the limit $\lim\limits_{m\to 0^+} \langle A_m , F \rangle_\M$ to exist. This is a rather tricky task, because it is not clear how to link the continuity in this distributional sense to test one-forms. We therefore have to tighten the request to the existence of the corresponding limits of initial data of $G_m F$, that is:
\begin{center}\textit{
		For which $F \in \Omega^1_0(\M)$, if any, do the limits\\ $\lim\limits_{m \to 0^+} \rhoz G_m F$ and  $\lim\limits_{m \to 0^+} \rhod G_m F$ exist?}
\end{center}
Clearly, the existence of these limits is sufficient for the existence of the limit in the distributional sense as stated above. To answer this tightened question, we make use of the following lemma:
\begin{lemma}\label{lem:limit_existence_classical_equivalence}
	Let $j=0$, $F \in \Omega^1_0(\M)$ fixed and assume Assumption \ref{ass:propagator_continuity} holds. The following statements are equivalent:
	\begin{enumerate}
		\item {The limits $\lim\limits_{m \to 0^+} \rhoz G_m F$ and  $\lim\limits_{m \to 0^+} \rhod G_m F$ exist. \\}
		\item {The limit $\lim\limits_{m \to 0^+} G_m F$ exists. \\}
		\item {The limit $\lim\limits_{m \to 0^+} \frac{1}{m^2}E_m d \delta  F $ exists. }
	\end{enumerate}
\end{lemma}
\begin{proof}
1.) We show (i) being equivalent to (ii): \\
a) (ii) $\implies$ (i) is trivial since, if $\lim\limits_{m \to 0^+} G_m F$ exists, clearly the limits $\lim\limits_{m \to 0^+} \rhoz G_m F$ and  $\lim\limits_{m \to 0^+} \rhod G_m F$ exist, as the operators $\rho_{(\cdot)}$ are continuous and do not depend on the mass $m$.  \par
b) (i) $\implies$ (ii):\\
Assume  $\lim\limits_{m \to 0^+} \rhoz G_m F$ and  $\lim\limits_{m \to 0^+} \rhod G_m F$ exist. We know that $\rho_{(\cdot)} G_m F$ specify initial data to the solution $G_m F \equiv B_m$ of the source free Proca equation: Specifying $\rhoz B_m$, $\rhod B_m$, $\rhon B_m$, $\rhodelta B_m$ is equivalent to specifying $B_0 = \restr{B_m}{\Sigma}$, $B_1 = \restr{\nabla_n B_m}{\Sigma}$, for some future pointing timelike unit normal field $n$ of the Cauchy surface $\Sigma$, as shown in \cite[pp. 2613]{FURLANI}. Furthermore, we know that the solution depends continuously on the initial data $B_0$ and $B_1$: Since $ B_m= G_m F$ has compact spacelike support, $B_0$ and $B_1$ will be compactly supported on $\Sigma$. For the case of compactly supported initial data, continuous dependence of the solution on the data is shown in \cite[Theorem 3.2.12]{baer_ginoux_pfaeffle}, which generalizes to arbitrarily supported initial data \cite[Theorem 2.3]{Sanders}.
We conclude that the solution $B_m$ depends continuously on $\rhoz B_m$, $\rhod B_m$, $\rhon B_m$, $\rhodelta B_m$ with respect to the topology of $\Omega^1(\M)$ and restricting $\rhon B_m$ and $\rhodelta B_m$ in terms of $\rhoz B_m$ and $\rhod B_m$ will not change the continuous dependence of the solution on $\rhoz B_m$ and $\rhod B_m$. A more direct approach to this statement is shown in \cite[Proposition 2.5]{pfenning}.
Therefore, $G_m F$ is continuous in $m$ and the corresponding limit exists.\par
%
%
2.) It remains to show the equivalence of (ii) and (iii):\\
	a.) (iii) $\implies$ (ii):\\ Assume that $\lim\limits_{m \to 0^+} \frac{1}{m^2}d \delta E_m F $ exists. Then
	\begin{align}
		\lim\limits_{m \to 0^+} G_m F
		&= \lim\limits_{m \to 0^+} \left( \frac{d \delta}{m^2} + 1\right) E_m F \notag\\
		&= \lim\limits_{m \to 0^+} \frac{1}{m^2}E_m d \delta  F  + \lim\limits_{m \to 0^+} E_m F
	\end{align}
	exists, using Assumption $\ref{ass:propagator_continuity}$ and that $d$, respectively $\delta$, commutes with $E_m$.\\
	b.) (ii) $\implies$ (iii):\\ Assume that $\lim\limits_{m \to 0^+} G_m F  $ exists. Then
	\begin{align}
		\lim\limits_{m \to 0^+} \frac{1}{m^2}E_m d \delta  F
		&= \lim\limits_{m \to 0^+} \frac{1}{m^2}d \delta E_m   F \notag \\
		&= \lim\limits_{m \to 0^+} \left( G_m F - E_m F \right) \notag \\
		&= \lim\limits_{m \to 0^+}  G_m F - \lim\limits_{m \to 0^+}  E_m F
	\end{align}
	exists, again using Assumption $\ref{ass:propagator_continuity}$.\par This completes the proof.
\end{proof}
%
%
%
With this result, the existence of the desired limit is purely determined by the existence of the zero mass limit of the propagator of the Proca operator. This can be quite easily formulated in terms of conditions on the test one-forms that the propagator acts on:
\begin{lemma}\label{lem:mass-zero-limit-existence-classical_weak}
	Let $F\in \Omega^1_0(\M)$ and $A_m \in \Omega^1(\M)$ be a solution to Proca's equation with vanishing external sources. Then,
	\begin{center}
		the limits $\lim\limits_{m \to 0^+} \rhoz G_m F$ and  $\lim\limits_{m \to 0^+} \rhod G_m F$ exist\\ if and only if $F = F' + F''$,\\
	\end{center}
	where $F', F'' \in \Omega^1_0(\M)$ such that $dF' = 0 = \delta F''$.\\
	Then also the limit $\lim\limits_{m \to 0^+} \langle A_m , F \rangle_\M$ exists.
\end{lemma}
\begin{proof}
	Let $F \in \Omega^1_0(\M)$. Using Lemma \ref{lem:limit_existence_classical_equivalence}, the existence of the desired limit is equivalent to the existence of the limit $\lim\limits_{m \to 0^+} \frac{1}{m^2}d \delta E_m F $.\par
	1.) We begin by finding the necessary conditions that this limit exists:\\
	 Assume $\lim\limits_{m \to 0^+} \frac{1}{m^2}d \delta E_m F $ exists. Rewriting $E_m d \delta F = m^2 \left( \frac{1}{m^2} d \delta E_m F\right)$ and using the assumption of the existence of the limit of the terms in brackets, we directly find
	\begin{align}
		E_0 d \delta F
		&= \lim\limits_{m \to 0^+} m^2 \left( \frac{1}{m^2}d \delta E_m F \right) \notag\\
		&= 0 \formspace.
	\end{align}
	For a compactly supported $F$ this yields the existence of a compactly supported $F' = E^+_0 d \delta F = E^-_0 d \delta F \in \Omega^1_0(\M)$, such that $d \delta F = (\delta d + d \delta)F'$ (see e.g. \cite[Proposition 2.6]{Sanders}). From the definition of $F'$ we immediately find $dF' = 0$.\\
	%
	%%%
	%
	%	, from which we obtain
	%	\begin{align}
	%				d \delta F &= (\delta d + d \delta)F' \\
	%\implies 0 &= \delta d \delta (F - F') = (\delta d+ d \delta ) \delta (F - F') \\
	%\implies 0 &= \delta(F -F') \formspace,
	%	\end{align}
	%	since both $F$ and $F'$ are compactly supported, $(F-F')$ define vanishing initial data in the past/future of their support, which yields a globally vanishing solution to the source free wave equation (c.f. \cite[Corollary 3.2.4]{baer_ginoux_pfaeffle}).\\
	Moreover, we obtain an additional condition:
	\begin{align}
		0
		&= E_0 d \delta F\notag  \\
		&= E_0 (d \delta + \delta d)F - E_0 \delta d F \notag \\
		&= - E_0 \delta d F \formspace.
	\end{align}
	By the same argument as before, this yields the existence of a one-form $F'' \in \Omega^1_0(\M)$, $F'' = E^+_0 \delta d F = E^-_0 \delta dF$ which yields $\delta F'' = 0$. \\
	%%
	%Also, we obtain in the same fashion as above
	%\begin{align}
	%	0 &= \delta d (F - F'') \\
	%	\implies 0 &= (d \delta + \delta d) d(F-F'')\\
	%	\implies dF &= dF'' \formspace.
	%\end{align}
	Finally, by definition we find
	\begin{align}
		F' + F '' = E_0^+ (d \delta + \delta d) F = F \formspace.
	\end{align}
	Therefore, as a necessary condition for the limit to exist, $F$ has to be the sum of a closed and a co-closed compactly supported one-form: $F = F' + F''$, where $F',F'' \in \Omega^1_0(\M)$  such that $dF' = 0 = \delta F''$. \par
	2.) In the next step, we show that the condition is also sufficient:\\
	Let $F = F' + F''$, where $F', F'' \in \Omega^1_0(\M)$ and $dF' = 0 = \delta F''$. Then
	\begin{align}
		\lim\limits_{m \to 0^+} \frac{1}{m^2} E_m d \delta F
		&=	\lim\limits_{m \to 0^+} \frac{1}{m^2} E_m d \delta (F' + F '') \notag\\
		&=  \lim\limits_{m \to 0^+} \frac{1}{m^2} E_m d \delta F'  \notag\\
		&= \lim\limits_{m \to 0^+} \Big( \frac{1}{m^2} E_m (d \delta + \delta d + m^2) F' - E_m F'\Big)\notag\\
		& = - \lim\limits_{m \to 0^+}  E_m F' \formspace,
	\end{align}
	which exists by assumption \ref{ass:propagator_continuity}. \\
	This completes the proof.
\end{proof}
From a formal point of view, we have completely classified those test one-forms, for which the zero mass limit exists. But it turns out that we can tighten the result even more, by observing that closed one-forms $F \in \Omega^1_0(\M)$, such that $dF=0$, do not contribute to the observable $\langle A_m , F \rangle_\M$ in the source free case. That is, for those $F$ it holds:
\begin{align}
	G_m F
	&= E_m  \left( \frac{d \delta}{m^2} + 1\right) F \notag \\
	&= \frac{1}{m^2}E_m (d \delta + \delta d + m^2) F \notag\\
	&= 0 \formspace,
\end{align}
which yields $\langle \Az, \rhod G_m F \rangle_\M = 0 = \langle \Ad, \rhoz G_m F \rangle_\M$ and hence $\langle A_m , F \rangle_\M= 0$. Due to the linearity of the fields, two test one-forms that differ by a closed compactly supported one-form give rise to the same physical observable. We may therefore restrict the class of test one-forms that we smear the fields $A_m$ with to the test one-forms modulo closed test one-forms. This yields the final result of this section:
%
%
%%
\begin{theorem}[Existence of the zero mass limit in the source free case]\label{thm:limit_existence_sourcefree_classical}
	Let $F, F' \in \Omega^1_0(\M)$ such that $[F] = [F']$, that is, there is a $\chi \in \Omega^1_{0,d}(\M)$ such that $F = F' + \chi$. Let $A_m$ be a solution to Proca's equation with vanishing external source $j=0$. Then,
	\begin{align}
		G_m F &= G_m F' \eqqcolon G_m [F] \quad \text{and} \\
		\langle A_m, F \rangle_\M &=\langle A_m, F' \rangle_\M \eqqcolon \langle A_m, [F] \rangle_\M \formspace,
	\end{align}
	and
	\begin{center}
		the limits $\lim\limits_{m \to 0^+} \rhoz G_m [F]$ and  $\lim\limits_{m \to 0^+} \rhod G_m [F]$ exist\\[2mm] 	 if and only if there exists a representative $\tilde{F}$ of $[F]$ with ${\delta \tilde{F}} = 0 $.
	\end{center}
	Then, also the limit $\lim\limits_{m \to 0^+} \langle A_m, [F] \rangle_\M$ exists.
\end{theorem}
\begin{proof}
	Let $F, F' \in \Omega^1_0(\M)$ such that $[F] = [F']$. Let $A_m$ be a solution to the source free Proca equation.
	We have already seen that for a closed test one-form $\chi \in \Omega^1_{0,d}(\M)$ it holds $G_m \chi =0$. It follows directly that $G_m F = G_m F'$ and hence $G_m [F]$ is well defined using a representative of the equivalence class $[F]$. Using Theorem \ref{thm:solution_proca_unconstrained} and the linearity of $\langle \cdot, \cdot \rangle_\M$, it directly follows $\langle A_m, F \rangle_\M =\langle A_m, F' \rangle_\M$ and hence $\langle A_m, [F] \rangle_\M$ is well defined. Therefore, we can, without losing any observables, divide out the test one-forms that are closed.
	By Lemma \ref{lem:mass-zero-limit-existence-classical_weak} we know that the limit exists if and only if $F$ is a sum of a closed and a co-closed test one-form. Hence, the limits $\lim\limits_{m \to 0^+} \rhoz G_m [F]$ and  $\lim\limits_{m \to 0^+} \rhod G_m [F]$ exists if and only if
	\begin{align}
		[F] \in \frac{\Omega^1_{0,d}(\M) + \Omega^1_{0,\delta}(\M)}{\Omega^1_{0,d}(\M)}\formspace.
	\end{align}
	Here, $\Omega^1_{0,d}(\M)$, $\Omega^1_{0,\delta}(\M)$ denotes the set of closed and co-closed test one-forms respectively.\\
	We will now show that in fact it holds
	\begin{align}
		\frac{\Omega^1_{0,d}(\M) + \Omega^1_{0,\delta}(\M)}{\Omega^1_{0,d}(\M)} \cong \Omega^1_{0,\delta}(\M) \formspace.
	\end{align}
	Let $F \in \Omega^1_{0,d}(\M) + \Omega^1_{0,\delta}(\M)$, that is, $F = F' + F''$ such that $dF' = 0 = \delta F''$. It directly follows $[F] = [F' + F''] = [F'']$. Indeed, $F'' \in \Omega^1_{0,\delta}(\M)$ is the \emph{unique} co-closed representative of the equivalence class $[F]$: Assume there exists a $\tilde{F} \in \Omega^1_{0,\delta}(\M)$ such that $[\tilde{F}] = [F] = [F'']$. From this it follows $[\tilde{F} - F''] = 0$, that is, $\tilde{F}$ and $F''$ differ by a closed test one-form, therefore we conclude $d(\tilde{F} - F'') = 0$.	By construction, it additionally holds $\delta(\tilde{F} - F'') =0$. Therefore $(\tilde{F} - F'')$ solves a source free massless wave equation:
	\begin{align}
		(\delta d + d \delta)(\tilde{F} - F'') = 0 \formspace.
	\end{align}
	Since $(\tilde{F} - F'')$ is compactly supported, it follows from \cite[Corollary 3.2.4]{baer_ginoux_pfaeffle} that $(\tilde{F} - F'') =0$. Hence $F''$ is the unique co-closed representative of $[F]$. \\
	The map
	\begin{align}
		\gamma : \frac{\Omega^1_{0,d}(\M) + \Omega^1_{0,\delta}(\M)}{\Omega^1_{0,d}(\M)} &\to \Omega^1_{0,\delta}(\M)\\
		[F] = [F' + F''] &\mapsto F'' \notag
	\end{align}
	is therefore well defined. Clearly, $\gamma$ is linearly bijective.
\end{proof}
%
Note, that this ``gauge'' by closed test one-forms is only present in the source free theory and not a real gauge freedom of the theory. We will therefore drop the explicit notation of the equivalence classes in the source free case and keep in mind that closed test one-forms do not contribute to the observables in the source free theory.\\
We find that it is sufficient as well as necessary for the mass zero limit to exist in the source free case to restrict to co-closed test one-forms. What is the interpretation of this?
In fact, this can be quite easily understood under the duality $\langle \cdot , \cdot \rangle_\M$. One finds that $\Quotientscale{\mathcal{D}^1(\M)}{d\mathcal{D}^{0}(\M)}$ is dual to $\Omega^1_{0,\delta}(\M)$ (see \cite[Section 3.1]{Sanders}). Here, $\mathcal{D}^1(\M)$ denotes the set of distributional one-forms (in a physical sense, these are classical vector fields) and $\Omega^1_{0,\delta}(\M)$ denotes the set of all co-closed test one-forms. Therefore, restricting to co-closed test one-forms is equivalent to implementing the gauge equivalence $A \to A + d\chi$, for $A \in \mathcal{D}^1(\M)$ and $ \chi \in \mathcal{D}^0(\M)$, in the theory! This dual relation is easily checked for $A' = A + d\chi$ dual to $F \in \Omega^1_{0,\delta}(\M)$
		\begin{align}
			\langle A', F \rangle
			&= \langle A, F \rangle + \langle d\chi, F \rangle \notag\\
			&= \langle A, F \rangle + \langle \chi, \delta F \rangle \notag\\
			&= \langle A, F \rangle \formspace.
		\end{align}
This is a nice result, since the gauge equivalence is naturally present in the Maxwell theory. And due to the non trivial topology on a general spacetime, it is a priori not clear how to implement the gauge equivalence in Maxwell's theory on curved spacetime: Maxwell's equation $\delta d A = 0$ yields that two solutions that differ by a closed one-form give rise to the same observable. For Minkowski spacetime this yields the familiar gauge equivalence $A \to A + d\chi$ since all closed one-forms are exact due to the trivial topology\footnote{For Minkowski spacetime, this follows from the Lemma of Poincar\'e, see e.g. \cite[Corollary 4.3.11]{rudolph_schmidt}.}. This does not hold for arbitrary spacetimes $\M$. One can argue that the gauge equivalence class given by the gauge equivalence of closed rather then exact one-forms is too large as it does not capture all physical phenomena of the theory: As presented in \cite[p. 626]{Sanders}, the Aharonov-Bohm effect \emph{does} distinguish between forms that differ by a form that is closed but not exact, so the gauge equivalence by closed one-forms cannot be the true physical gauge equivalence class. Hence, arguing with physical properties is needed to find the ``right'' gauge equivalence class for the Maxwell theory in curved spacetimes. With our result, this gauge equivalence by exact forms comes naturally in the limit process! \\
 Hence, we have already captured one important feature of the Maxwell theory also in the massless limit of the Proca theory! It remains to check whether also the dynamics are ``well behaved'' in the massless limit. But first, we investigate the zero mass limit for the general theory with sources.
%
%
%
%
\subsubsection{Existence of the limit in the general case with current}\label{sec:zero-mass-limit-existence-classical-general-source}
The question of interest is analogous to the one presented in the previous section but now including external sources $j \neq 0$. Again, a solution to Proca's equation with initial data $\Az, \Ad \in \Omega^1(\Sigma)$ is uniquely determined  by
\begin{align}
	\langle A_m , F \rangle_\M = \Big( \sum\limits_\pm \langle j , G_m^\mp F   \rangle_{\Sigma^\pm} +\langle \Az , \rhod G_m F \rangle_\Sigma
	- \langle \Ad , \rhoz G_m F \rangle_\Sigma \Big) \formspace.
\end{align}
In order to classify those test one-forms $F$ for which the zero mass limit exists, we have to tighten the main question as posed in the beginning of Section \ref{sec:zero-mass-limit-classical} which was formulated in a distributional sense. Just as in the source free case we again demand that $\lim\limits_{m \to 0^+} \rhoz G_m F$ and  $\lim\limits_{m \to 0^+} \rhod G_m F$ exist. Furthermore, we need a tightened  condition for the limit $\lim\limits_{m \to 0^+} \sum_\pm \langle j , G_m^\mp F   \rangle_{\Sigma^\pm} $ to exist. First, we note that there are mainly three situations that can occur regarding this sum of integrals. Either, the support of $F$ lies in the future of the Cauchy surface $\Sigma$ in which case $\supp{G_m^+ F} \cap \Sigma^- = \emptyset$ and $\langle j , G_m^+ F   \rangle_{\Sigma^-} = 0$. Similarly, if the support of $F$ lies in the past of $\Sigma$, then $\langle j , G_m^- F   \rangle_{\Sigma^+} = 0$. Or, the intersection of the support of $F$ and the Cauchy surface $\Sigma$ is non-empty in which case both terms appear. But since we want the existence of the limit to be independent of the choice of the Cauchy surface $\Sigma$ we conclude that the limit $\lim\limits_{m \to 0^+} \sum_\pm \langle j , G_m^\mp F   \rangle_{\Sigma^\pm} $ exists if and only if $\lim\limits_{m \to 0^+} \langle j , G_m^+ F   \rangle_{\Sigma^-}$ and $\lim\limits_{m \to 0^+} \langle j , G_m^- F   \rangle_{\Sigma^+}$ exist separately. In the same fashion as for the initial data terms, we therefore want the limits $\lim\limits_{m \to 0^+} G_m^\pm F$ to exist. In this sense of the existence of the limit, the question of interest is now a slightly generalized version of what was stated in the previous section:
\begin{center}\textit{
		For which $F \in \Omega^1_0(\M)$, if any, do the limits \\ $\lim\limits_{m \to 0^+} G_m^\pm F$, $\lim\limits_{m \to 0^+} \rhoz G_m F$ and  $\lim\limits_{m \to 0^+} \rhod G_m F$ exist?}
\end{center}
We have already classified the existence of the latter two limits. And with similar calculations, also the first term is quite easy to handle. We find the following result:
\begin{theorem}[Existence of the zero mass limit in the general case]\label{thm:limit_existence_general_classical}
	Let $F \in \Omega^1_0(\M)$, $A_m$ a solution to Proca's equation with external source $j\neq0$. Then,
	\begin{center}
		the limits $\lim\limits_{m \to 0^+} G_m^\pm F$, $\lim\limits_{m \to 0^+} \rhoz G_m F$ and  $\lim\limits_{m \to 0^+} \rhod G_m F$ exist\\[2.5mm]
		if and only if ${\delta F} = 0 $.
	\end{center}
	Then, also the limit $\lim\limits_{m \to 0^+} \langle A_m, F \rangle_\M$ exists.
\end{theorem}
\begin{proof}
Let $F \in \Omega^1_0(\M)$, $A_m$ a solution to the source free Proca equation. \\
1.)  For the limits $\lim\limits_{m \to 0^+} \rhoz G_m F$ and  $\lim\limits_{m \to 0^+} \rhod G_m F$ to exist we have found in Lemma \ref{lem:mass-zero-limit-existence-classical_weak} that it is sufficient and necessary for $F$ to be the sum of a closed and a co-closed test one-form, $F = F' + F''$, $F',F'' \in \Omega^1_0(\M)$ such that $dF' = 0 = \delta F''$. \par
%
2.) For the limits $\lim\limits_{m \to 0^+} G_m^\pm F$ we find existence if and only if $\lim\limits_{m \to 0^+} \frac{1}{m^2}E_m^\pm d \delta F0$ exists by a calculation analogous to the one presented in the proof of Lemma \ref{lem:limit_existence_classical_equivalence}.\par
a) Assume $\lim\limits_{m \to 0^+} \frac{1}{m^2}E_m^\pm d \delta F$ exists. We conclude $E_0^\pm d \delta F = 0$, following the calculations in the proof of Lemma \ref{lem:mass-zero-limit-existence-classical_weak} again replacing $G_m$, respectively $E_m$, with $G_m^\pm$, respectively $E_m^\pm$. Using $E_m^\pm (\delta d + d \delta )F = F$ we find
\begin{align}
	0
	&=  E_0^\pm d \delta F  \notag\\
	&= F - E_0^\pm \delta d F
\end{align}
and hence $F = E_0^\pm \delta d F$ which is compactly supported. It clearly follows that $F$ is co-closed using that $\delta$ commutes with $E_m^\pm$:
\begin{align}
	\delta F = \delta E_0^\pm \delta d F = E_0 \delta \delta d F = 0 \formspace.
\end{align}
b) Assuming $F \in \Omega^1_0(\M)$ being co-closed, $\delta F = 0$, we easily conclude that $G^\pm_m F = E^\pm_m F$ and hence the limits $\lim\limits_{m \to 0^+} G_m^\pm F$ exist. \par
%
We have therefore shown that the limits $\lim\limits_{m \to 0^+} G_m^\pm F$ exist if and only if $F$ is co-closed. Combining this with the existence of the remaining limits $\lim\limits_{m \to 0^+} \rhoz G_m F$ and  $\lim\limits_{m \to 0^+} \rhod G_m F$, we find the desired result, as it is necessary and sufficient for the limits $\lim\limits_{m \to 0^+} G_m^\pm F$, $\lim\limits_{m \to 0^+} \rhoz G_m F$ and  $\lim\limits_{m \to 0^+} \rhod G_m F$ to exist that $F$ is co-closed. This completes the proof.
\end{proof}
Therefore, also in the general case with currents we find existence of the zero mass limit of the Proca field if and only if we implement the gauge equivalence, as a restriction on the dual space of test one-forms, before taking the limit. This was already discussed in the previous Section \ref{sec:zero-mass-limit-existence-classical-vanishing-source}. We can now discuss the dynamics of the fields in the zero mass limit.
%%
%
%%%
%
%
%
%
%
%
%
%
%
%
%
%
\subsubsection{Dynamics and the zero mass limit}\label{sec:limit_dynamics_classical}
We have found that both in the source free and the general case, the zero mass limit of the classical Proca theory exists if we restrict the test one-forms that we smear the classical fields with to the ones that are co-closed. So the question regarding the dynamics of the theory in the limit, that is, the behavior of $\langle A_0 , \delta d F \rangle_\M= \lim\limits_{m \to 0^+} \langle A_m, \delta d F \rangle_\M$, is well posed for any $F \in \Omega^1_0(\M)$ since naturally $\delta d F$ is co-closed using $\delta$ being nilpotent. For the Maxwell theory, we expect $\langle A_0 , \delta d F \rangle_\M= \langle j, F \rangle_\M$ as the field $A_0$ should solve Maxwell's equation in that distributional sense. But defining the field as a zero mass limit of the Proca theory, we find
\begin{align}
\langle A_0 , \delta d F \rangle_\M
&\coloneqq  \lim\limits_{m \to 0^+} \langle A_m, \delta d F \rangle_\M \\
&=  \lim\limits_{m \to 0^+}\Big( \sum\limits_\pm \langle j , G_m^\mp \delta d F   \rangle_{\Sigma^\pm} +\langle \Az , \rhod G_m \delta d F \rangle_\Sigma
- \langle \Ad , \rhoz G_m \delta d F \rangle_\Sigma \Big) \formspace.\notag
\end{align}
Recalling $G^\pm_m = \frac{1}{m^2}(d \delta + m^2) E^\pm_m$, we find $G^\pm_m \delta d F = E^\pm_m \delta d F$, and using $E_m^\pm (\delta d F) = F - E_m^\pm (d\delta + m^2) F$ we obtain
\begin{align}
	\langle &A_0 , \delta d F \rangle_\M  \\
&=  \lim\limits_{m \to 0^+}\Big( \sum\limits_\pm \big(  \langle j , F\rangle_{\Sigma^\pm} - \langle j ,  E_m^\mp d\delta F   \rangle_{\Sigma^\pm} \big)
		 -\langle \Az , \rhod E_m d\delta  F \rangle_\Sigma
		+ \langle \Ad , \rhoz E_m d\delta  F \rangle_\Sigma  \notag \\
		 &\phantom{=I} - m^2 \big(
		 \sum\limits_\pm \langle j , E_m^\mp F\rangle_{\Sigma^\pm}
				 + \langle \Az , \rhod E_m  F \rangle_\Sigma
				  -\langle \Ad , \rhoz E_m  F \rangle_\Sigma
		 \big)
		\Big)	\notag \\
&=	\langle j , F \rangle_\M - \lim\limits_{m \to 0^+}\Big(\sum\limits_\pm \langle j ,  E_m^\mp d\delta F   \rangle_{\Sigma^\pm}
+\langle \Az , \rhod E_m d\delta  F \rangle_\Sigma
- \langle \Ad , \rhoz E_m d\delta  F \rangle_\Sigma \Big) \formspace.		\notag
\end{align}
We have used $\sum\limits_\pm \langle j , F\rangle_{\Sigma^\pm} = \langle j , F\rangle_\M$ and that the terms proportional to $m^2$ are continuous by Assumption \ref{ass:propagator_continuity} and bounded and hence vanish in the limit. Furthermore, we find by definition that $\rhod E_m d\delta  F = - *_{(\Sigma)}i^* * d E_m d \delta F = 0$ since $d$ and $E_m$ commute. Concluding, we have calculated
\begin{align} \label{eqn:dynamics_limit_classical_unconstraint}
\langle A_0 , &\delta d F \rangle_\M =  \langle j , F \rangle_\M - \lim\limits_{m \to 0^+}\Big(\sum\limits_\pm \langle j ,  E_m^\mp d\delta F   \rangle_{\Sigma^\pm}
- \langle \Ad , \rhoz E_m d\delta  F \rangle_\Sigma \Big) \formspace.
\end{align}
The second term though will not vanish in general. Ergo, the fields $A_0$ defined as the zero mass limit of the Proca field $A_m$ will not fulfill Maxwell's equation in a distributional sense. While this might seem surprising at first, it is quite easy to understand when we recall how we have found solutions to Proca's equation: instead of finding solutions directly, we have specified solutions to the massive wave equation (\ref{eqn:classical_wave_eqation}) and then restricted the initial data such that the Lorenz constraint (\ref{eqn:classical_constraint}) is fulfilled. Only then we also have found a solution to Proca's equation. And similarly, one solves Maxwell's equation by specifying a solution to the massless wave equation $(\delta d + d \delta )A_0 = j$ and restricting the initial data such that the Lorenz constraint $\delta A_0 = 0$ is fulfilled. Only then, the solution also solves Maxwell's equation. And it is with the constraint where the problem in the limit lies. Recall from Theorem \ref{thm:solution_proca_constrained} that, in order to implement the Lorenz constraint, we have restricted the initial data by
\begin{align}
	\Adelta &= \frac{1}{m^2}\rhodelta j \; , \quad \text{and} \\
	\An &= \frac{1}{m^2}\left( \rhon j  + \delta_{(\Sigma)} \Ad \right) \formspace.
\end{align}
It is obvious that, in general, this is not well defined in the zero mass limit. So in order to keep the dynamics in the zero mass limit, we need to make sure that the constraints are well behaved in the limit. Since we do not want the external source or the initial data to be dependent of the mass, we have to specify
\begin{align}
	\delta j &= 0 \quad \; , \quad \text{and} \\
	\delta_{(\Sigma)} \Ad &= -\rhon j \quad \implies \An =0 \formspace.
\end{align}
This corresponds exactly to the constraints on the initial data in the Maxwell case to implement the Lorenz constraint (see \cite[Theorem 2.11]{pfenning})! With these constraints, we can now look at the remaining term of $\langle A_0 , \delta d F\rangle_\M$ in equation (\ref{eqn:dynamics_limit_classical_unconstraint}). We do this separately for the two summands. Using that $d$ commutes with pullbacks and inserting the constraints on the initial data, we find
\begin{align}
\langle \Ad , \rhoz E_m d\delta  F \rangle_\Sigma
	&= \langle \Ad , d_{(\Sigma)} \rhoz E_m \delta  F \rangle_\Sigma \notag\\
    &=	\langle \delta_{(\Sigma)}\Ad ,  \rhoz E_m \delta  F \rangle_\Sigma \notag\\
    &= -\langle \rhon j ,  \rhoz E_m \delta  F \rangle_\Sigma \notag\\
    &= -\int\limits_{\Sigma} i^* E_m \delta F \wedge *_{(\Sigma)} (- *_{(\Sigma)}  i^* *)j \notag\\
    &=  \int\limits_{\Sigma} i^* E_m \delta F \wedge i^* * j \notag\\
    &=  \int\limits_{\Sigma} i^*\left(  E_m \delta F \wedge * j \right) \formspace.
\end{align}
For the first summand $\sum_\pm \langle j ,  E_m^\mp d\delta F   \rangle_{\Sigma^\pm}$ we use the partial integration that we have already calculated in the proof of Theorem \ref{thm:solution_proca_unconstrained} and find, using the constraint $\delta j = 0$ found above,
\begin{align}
\sum_\pm \langle j ,  E_m^\mp d\delta F   \rangle_{\Sigma^\pm}
&= \sum_\pm \langle d \delta j ,  E_m^\mp  F   \rangle_{\Sigma^\pm} + \int\limits_{\Sigma} i^*\left(  j \wedge * E_m \delta F \right) -  \int\limits_\Sigma i^*(\delta j \wedge *EF ) \notag\\
&= \int\limits_{\Sigma} i^*\left(  j \wedge * E_m \delta F \right)   \formspace.
\end{align}
Using $j \wedge * E_m \delta F = E_m \delta F \wedge * j$ we find that the remaining terms of equation (\ref{eqn:dynamics_limit_classical_unconstraint}) vanish when restricting the initial data such that they are well defined in the zero mass limit. We therefore obtain the correct dynamics
\begin{align}
\langle A_0 , \delta d F \rangle_\M
&=  \langle j , F \rangle_\M - \lim\limits_{m \to 0^+}\Big(\sum\limits_\pm \langle j ,  E_m^\mp d\delta F   \rangle_{\Sigma^\pm}
		- \langle \Ad , \rhoz E_m d\delta  F \rangle_\Sigma \Big) \notag\\
&= \langle j , F \rangle_\M \formspace.
\end{align}
Concluding, when keeping the constraints that implement the Lorenz constraint in the limit well behaved, we indeed end up with the correct dynamics of the Maxwell theory. Furthermore, we also obtain conservation of the external current $\delta j = 0$ as a necessity to get the correct dynamics. This is not surprising as, opposed to Proca's theory, the current in Maxwell's theory is always conserved by the equations of motion! We conclude this in the final theorem of this chapter:
%
%
%
\begin{theorem}[The zero mass limit of the Proca field]
	Let $F\in \Omega^1_0(\M)$ be a test one-form and $j \in \Omega^1(\M)$ an external current. \\
	Let $A_m$ be a solution to Proca's equation specified by initial data $\Az, \Ad \in \Omega^1_0(\Sigma)$ via Theorem \ref{thm:solution_proca_unconstrained}.	\\
	Defining the zero mass limit $\langle A_0 , F \rangle_\M = \lim\limits_{m \to 0^+} \langle A_m, F \rangle_\M$ of the Proca field, the following holds:
	\begin{enumerate}
		\item The limit exists if and only if $\delta F = 0$, effectively implementing the gauge equivalence of the Maxwell theory.
		\item The field $A_0$ is a Maxwell field, that is, it solves Maxwell's equation in a distributional sense if and only if $\delta j = 0$, implementing the conservation of current, and $\rhon j = - \delta_{(\Sigma)} \Ad$ , implementing the Lorenz condition.
	\end{enumerate}
\end{theorem}
\begin{proof}
	The proof follows directly from Theorem \ref{thm:limit_existence_general_classical} and the calculations presented in the above Section \ref{sec:limit_dynamics_classical}
\end{proof}
%
%
%
%
%
%
