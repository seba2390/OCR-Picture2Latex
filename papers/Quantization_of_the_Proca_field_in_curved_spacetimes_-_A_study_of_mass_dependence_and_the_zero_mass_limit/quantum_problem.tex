\section{The Quantum Problem}\label{chpt:quantum}
Having established a good understanding of the classical theory, we will now investigate the quantum Proca field in curved spacetimes. In particular, we are going to construct a generally covariant quantum field theory of the Proca field in the framework of Brunetti Fredenhagen Verch \cite{Brunetti_Fredenhagen_Verch} in Section \ref{sec:generally_covariant_QFTCS} and show that the theory is local. In Section \ref{sec:BU-algebra} we study the Borchers-Uhlmann algebra as the field algebra and rigorously construct an initial data formulation of the quantum Proca field theory. This will allow us to define a notion of continuity of the Proca field with respect to the mass and, finally, to study the mass dependence and the zero mass limit of the theory in Section \ref{sec:mass_depenence_and_limit}.
%
%
%
%
%
\subsection{Construction of the generally covariant quantum Proca field theory in curved spacetimes}\label{sec:generally_covariant_QFTCS}
The quantization of the Proca field in a generally covariant way will follow the framework of Brunetti, Fredenhagen and Verch \cite{Brunetti_Fredenhagen_Verch} as well as some natural modifications needed for background source dependent theories which are made analogous to \cite{Sanders}. In the framework of \cite{Brunetti_Fredenhagen_Verch}, a \emph{generally covariant quantum field theory} is mathematically described as a functor between the category $\Spac$, consisting of globally hyperbolic spacetimes as objects and orientation preserving isometric hyperbolic embeddings as morphisms, and the category $\Alg$, consisting of unital $^*$-algebras as objects and unit preserving $^*$-algebra-homomorphisms as morphisms. If these $^*$-algebra-homomorphisms are injective, the theory is said to be \emph{local} (rigorous definitions are given below). To accommodate the given background source $j$ in the theory, we will generalize $\Spac$ to a category whose objects also contain the given background source. \par
In this section, we want to give the necessary definitions and construct this functor explicitly, that is, we give a detailed definition on how to map globally hyperbolic spacetimes with a given background current to an algebra of observables, and how to map the morphisms onto each other. The main work is then to show that these maps are well defined. It is then rather trivial to show that we have obtained a functor. Throughout this section, the mass $m$ as well as the external current $j$ are assumed to be fixed.
\newpage
%
%
We begin by defining the necessary objects and morphisms of the two categories.
\begin{definition}[Orientation preserving isometric hyperbolic embedding]
Let $(\M,g)$ and $(\N,g_\N)$ be two globally hyperbolic spacetimes. \\
A  map $\psi : (\M,g) \to (\N,g_N)$ is called a \emph{orientation preserving isometric hyperbolic embedding} if
\begin{enumerate}
\item $\psi$ is a diffeomorphism, that is it is smoothly bijective,
\item $\psi$ preserves orientation and time orientation,
	\item $\psi$ is an isometry, that is $\psi^* g_\N = g$, and
	\item $\psi(\M)$ is causally convex, that is for $p \in \M$ it holds
\begin{align*}
J_\M^\pm(p) = \psi^{-1} \Big( J_\N^\pm \big( \psi(p)\big) \Big) \formspace .
\end{align*}
\end{enumerate}
\end{definition}
\begin{definition}[The categories $\SpacCurr$, $\Alg$ and $\Alg'$]\label{def:categories_alg_spaccurr}
The category \gls{spaccurr} consists of triples $(\M,g,j_\M)$ as objects, where $(\M,g)$ is a globally hyperbolic spacetime  and $j_\M \in \Omega^1(\M)$ corresponds to the background current of the theory,
and morphisms $\psi$, where $\psi : (\M,g) \to (\N,g_\N)$ is an orientation preserving isometric hyperbolic embedding such that $\psi^* j_\N = j_\M$. \par
The category \gls{alg} consists of unital $^*$-algebras as objects and unit preserving  $^*$-algebra-homomorphisms as morphisms. \par
The category \gls{algprime} is a subcategory of $\Alg$ consisting of the same objects but only injective morphisms.
\end{definition}
With the notion of these two categories we are able to define:
%
\begin{definition}[Generally (locally) covariant quantum field theory with background source]\label{def:generally-coveriant-qftcs}
A \emph{generally covariant quantum field theory with background source} is a covariant functor between the categories $\SpacCurr$ and $\Alg$. \\
The theory is called \emph{local} or \emph{locally covariant} if and only if the range of the functor is contained in $\Alg'$.
\end{definition}
%
%
%
%
%
%
To construct this functor for the Proca field, we will first define how to map a globally hyperbolic spacetime to a unital $^*$-algebra and how to map the morphisms onto each other. Most of the work is to show that those maps are well defined and injective. Then, it is not hard to show that we have obtained a functor and thus the desired generally locally covariant quantum field theory.
%
\begin{definition}\label{def:algebra-A(M)}
Let $M=(\M,g,j_\M) \in \mathsf{Obj}_\SpacCurr$ be an object of $\SpacCurr$, $F,F' \in \Omega^1_0(\M)$ be test one-forms and $c_1, c_2 \in \IC$ be constants. \\
Let \gls{Gmcurly} be the propagator of the Proca operator with integral kernel $G_m$, that is, $\Green{F}{F'} = \langle F, G_mF' \rangle_\M$.\\
Define  $\AA : \mathsf{Obj}_\SpacCurr \to \mathsf{Obj}_\Alg$,  where $\AA(M)$ is the unital $^*$-algebra obtained from the free algebra, generated by $\mathbbm{1}$ and the objects $\A(F)$, factoring by the relations
 \begin{subequations}  \label{def:ideal_generators}
  \begin{align}
\text{(i)}\; &\A(c_1 F + c_2 F') = c_1 \A(F) + c_2 \A(F') 														&\textrm{linearity,} \\
\text{(ii)}\; &\A(F)^* = \A(\skoverline{F}\,) 																															&\textrm{real field,} \\
\text{(iii)}\; &\A\big( (\delta d + m^2) F \big) = \langle j_\M , F \rangle_\M \cdot \mathbbm{1} 	&\textrm{equation of motion,} \\
\text{(iv)}\; &[\A(F) , \A(F') ] = \i \Green{F}{F'} \cdot \mathbbm{1}															&\textrm{commutation relations}.
 \end{align}
 \end{subequations}
\end{definition}
To be mathematically more precise, the algebra is obtained as the quotient algebra from the free algebra $\PPM$ dividing out the (two-sided) ideal $\mathcal{J}_\M$ that is generated by the relations  (\ref{def:ideal_generators}). As an example, a sub-ideal of $\mathcal{J}_\M$ implementing (\ref{def:ideal_generators}b) is defined as $\widetilde{\mathcal{J}}_\M = \big\{ a\big(\A(F)^* - \A(\skoverline{F}\,) \big)b \;\;\vert\;\; a,b \in \PPM, F \in \Omega^1_0(\M) \big\}$. One obtains an algebra of equivalence classes $\AA(M) = {\Quotientscale{\PPM}{ \mathcal{J}_\M}}$.
For this to be well defined, it suffices to show that the obtained algebra $\AA(M)$ is not trivial, that is, not the zero algebra. Therefore, we need to show that the ideal $\mathcal{J}_\M$ is not the full free algebra $\PPM$.
Clearly, for a suitable test one-form $F$, that is, in particular a one-form that is not of the form $F = (\delta d + m^2)H$ for some test one-form $H$,  $\A(F)$ will not be an element of $\mathcal{J}_\M$, and therefore $\AA(M)$ is not trivial.\par
%
%
%
%
%
%
Next, we define the action of the map $\AA$ on morphisms of $\SpacCurr$.
%
\begin{definition}\label{def:morphism_alpha_psi}
Let $M,N\in \mathsf{Obj}_\SpacCurr$, where $M=(\M,g,j_\M)$ and $N=(\N,g_\N,j_\N)$, be objects and $\psi \in \textsf{Mor}_\SpacCurr(M,N), \psi: (\M,g,j_\M) \to (\N,g_\N,j_\N)$ be a morphism of the category $\SpacCurr$.
Define $\AA(\psi) \equiv \alpha_\psi : \AA(M) \to \AA(N)$ as a unit preserving $^*$-algebra-homomorphism whose action on elements of $\AA(M)$ is then fully determined by the action on the generators $\A_\M(F)$ :
\begin{align}
\alpha_\psi \big(\A_\M(F)\big) = \A_\N(\psi_*(F)) \formspace.
\end{align}
\end{definition}
%
%
We need to show that this is well defined\footnote{Note that in Definition \ref{def:pullback} we have only defined the \emph{pullback} $\psi^* F$ of a one-form $F$. Since here $\psi$ is assumed to be a diffeomorphism, the pushforward of one-forms on $\M$ to one-forms on $\N$ can be defined as the pullback with respect to $\psi^{-1}$.}, in particular that it is compatible with the algebra relations in $\AA(\N)$. To be more precise, the proceeding is as follows:\\
Let $\PPM,\PPN$ be the free unital *-algebras as defined above. We define a morphism $\pi : \PPM \to \PPN$ as a unit preserving $^*$-algebra homomorphism such that $\A_\M(F) \mapsto \A_\N(\psi_* F)$. We need to show that $\pi(\mathcal{J}_\M) \subset \mathcal{J}_\N$, so that if we divide out the ideal $\mathcal{J}_\M$, $\pi$ descends to the wanted unit preserving $^*$-algebra homomorphism $\AA(\psi) \equiv \alpha_\psi : \AA(M) \to \AA(N)$. We do this step by step, showing that each generator of $\mathcal{J}_\M$ maps to a corresponding generator of $\mathcal{J_\N}$. In the following let  $F,F' \in \Omega^1_0(\M)$ and $c_1,c_2 \in \IC$ be constants.\par
%
%
%
\textit{1.) Linearity:}\\
The generator of the corresponding ideal\footnote{Note, that actually we are interested in $^*$-ideals, so we would need to add (or subtract) the hermitian adjoint to that expression, but since $\pi$ is defined as a $^*$-algebra homomorphism this would not change anything in the calculations and is therefore neglected for clarity.} is $\big( \A_\M (c_1\,F+ c_2\,F') -   c_1\, \A_\M (F) - c_2\, \A_\M (F')  \big)$. We calculate:
\begin{align}
\pi \big( \A_\M (c_1\,F+ c_2\,F') &-   c_1\, \A_\M (F) - c_2\, \A_\M (F')  \big) \notag\\
&=  \pi \big( \A_\M (c_1\,F+ c_2\,F') \big)  -  c_1\, \pi \big(  \A_\M (F) \big) - c_2\, \pi \big( \A_\M (F')  \big) \notag\\
&= \A_\N (c_1\, \psi_* F+ c_2\,\psi_*F') -  c_1\,  \A_\N (\psi_*F) - c_2\,  \A_\N (\psi_*F') \formspace.
\end{align}
We have used  the homomorphism property of $\pi$ and that $\psi_*$ is linear and naturally commutes with scalars.
The result clearly is an element of the corresponding ideal in $\PPN$ specifying $\tilde{F}, \tilde{F}' \in \Omega^1_0(\N) $ by $\tilde{F} = \psi_* F$ and $\tilde{F}' = \psi_* F'$. \par
%
%
%
\textit{2.) Real field:} \\
The corresponding generator is $\big( \A_\M( F)  - \A_\M( \skoverline{F}\,)^* \big)$. We obtain:
\begin{align}
\pi \big( \A_\M( \skoverline{F}\,)  - \A_\M( F)^* \big)
&= \pi \big( \A_\M( \skoverline{F}\,) \big)   - \pi \big( \A_\M( F)^* \big) \notag\\
&= \pi \big( \A_\M( \skoverline{F}\,) \big)   - \pi \big( \A_\M( F) \big)^* \notag\\
&=  \A_\N( \psi_* \skoverline{F}\,)    -  \A_\N(\psi_*  F)^* \formspace.
\end{align}
Again, the result clearly is an element of the corresponding ideal in $\PPN$.\newpage
%
%
\textit{3.) Equations of motion:}\\
The generator of interest is $\big( \A_\M\big( (\delta d + m^2) F\big )  -  \langle j_\M , F \rangle_\M \, \mathbbm{1}_{\PPM} \big)$.
First, note that
\begin{align}
(\delta d + m^2)\psi_* = \psi_* (\delta d + m^2)
\end{align}
since $\psi_*$ is linear and commutes with $d$. Also, because $\psi$ is an orientation preserving isometry and therefore preserves the volume form, $\psi_*$ commutes with the Hodge star  and thus it also commutes  with the interior derivative $\delta$.
It then follows that
\begin{align}
\pi  \big( \A_\M\big( (\delta d + m^2) F\big )  & -  \langle j_\M , F \rangle_\M \, \mathbbm{1}_{\PPM} \big)  \notag\\
&= \A_\N\big(\psi_*  (\delta d + m^2) F\big )  -  \langle j_\M , F \rangle_\M \, \mathbbm{1}_{\PPN} \notag\\
&= \A_\N\big(  (\delta d + m^2) \psi_* F\big )  -  \langle j_\N , \psi_* F \rangle_\N \, \mathbbm{1}_{\PPN} \formspace.
\end{align}
In the last step it was used that, since $\psi$ is an isometry and $\psi^* j_\N = j_\M$:
\begin{align}
\langle j_\M , F \rangle_\M
&= \langle \psi^* j_\N , F \rangle_\M \notag\\
&= \langle  j_\N , \psi_* F \rangle_\N \formspace,
\end{align}
which yields the wanted generator in $\PPN$.\par
%
%
%
\textit{4.) Commutation relation:}\\
The generator is $\big ( \big[ \A_\M(F) , \A_\M( F') \big] - \i \GreenM{F}{F'}\,\mathbbm{1}_{\PPM} \big)$. We calculate:
\begin{align}
\pi   \big ( \big[ \A_\M(F) , \A_\M( F') \big] & - \i \GreenM{F}{F'}\,\mathbbm{1}_{\PPM} \big) \notag\\
&= \big[ \A_\N(\psi_* F) , \A_\N(\psi_* F') \big] - \i \GreenM{F}{F'}\,\mathbbm{1}_{\PPN}  \notag\\
&= \big[ \A_\N(\psi_* F) , \A_\N(\psi_* F') \big] - \i \GreenN{\psi_* F}{\psi_* F'}\,\mathbbm{1}_{\PPN}
\end{align}
In the last step we have used the uniqueness of the fundamental solutions and the properties that $\psi$ is an isometry and that $\psi(\M)$ is causally convex.
Together, this implies $G_{m,\M} F = \psi^* G_{m,\N} \psi_* F $ (see \cite[Chapter 4.3]{Sanders}) and therefore
\begin{align}
\GreenM{F}{F'}
&= \langle F, G_{m,\M} F' \rangle_\M \notag\\
&= \langle F, \psi^* G_{m,\N} \psi_* F' \rangle_\M\notag \\
&= \langle \psi_* F ,  G_{m,\N} \psi_* F' \rangle_\N \notag\\
&= \GreenN{\psi_*F}{\psi_* F'} \formspace.
\end{align}
%
%
Altogether, we have shown that $\pi(\mathcal{J}_\M)  \subset \mathcal{J}_\N$ and therefore $\pi$ descends to the wanted unit preserving $^*$-algebra homomorphism $\alpha_\psi$, having divided out the ideal $\mathcal{J}_\M$.\par
%
%
%
%
%
\newpage
Now, to obtain a \emph{locally covariant} QFT from these definitions, we need the defined homomorphism to be injective. We do this the following way:
To show that $\AA(\psi) \equiv \alpha_\psi$ is injective, we can equivalently show that the algebra $\AA(M)$ is simple, that is, there is no non-trivial two-sided ideal\footnote{For a general algebra to be simple one also needs that the multiplication operation is not uniformly zero. Since we deal with unital algebras this is trivially fulfilled.} in $\AA(M)$. It turns out that for $\AA(M)$ to be simple, it is sufficient that $\Green{\cdot}{\cdot}$ is non-degenerate. The basic algebraic work necessary for these arguments is put in Appendix \ref{app:lemmata}.
\begin{lemma}\label{lem:propagator-non-degenerate}
Let $F,F' \in \Omega^1_0(\M)$ be two test one-forms and let $\gls{DSz} = \Omega^1_0(\Sigma) \oplus  \Omega^1_0(\Sigma)$ be the space of initial data on some Cauchy surface $\Sigma$ with respect to Proca's equation.
Then,
\begin{center}
$\Green{F}{F'}$, viewed as a map $\mathcal{G}_m : \Dzs \oplus \Dzs \to \IC$\\ on the space of initial data, is a symplectic form,
\end{center}
 that is it is bilinear, anti-symmetric and non-degenerate.
\end{lemma}
\begin{proof}
First, we want to see how to view $\mathcal{G}_m$ as a map on initial data. Let $F,F' \in \Omega^1_0(\M)$, then by definition
\begin{align}
\Green{F}{F'}
&= \langle F , G_m F' \rangle_\M \notag\\
&= \langle G_m F' , F \rangle_\M  \formspace.
\end{align}
Note that $G_m F'$ is a solution to the source free Proca equation, that is,
\begin{align}
(\delta d + m^2)G_m F' = 0 \formspace.
\end{align}
By Theorem \ref{thm:solution_proca_unconstrained}, setting $j=0$, we find for a test one-form $F$,
\begin{align}
 \langle G_m F' , F \rangle_\M =  \langle (G_m F')_{0} , \rhod GF \rangle_\Sigma - \langle (G_m F')_{d} , \rhoz G_m F \rangle_\Sigma  \formspace.
\end{align}
In the same way, $G_m F$ is a solution to Proca's equation and therefore, since $\rhoz$ and $\rhod$ map a solution to its initial data, we can view $\rhoz G_m F$ and  $\rhod G_m F$ as initial data of $G_m F$.
For short hand we will write for the initial data $\rhoz G_m F = (G_m F)_0 = \varphi$ and  $\rhod G_m F = (G_m F)_0 = \pi$ and analogously for $G_m F'$.
Therefore we obtain
\begin{align}
\Green{F}{F'} &= \langle \varphi' , \pi \rangle_\Sigma  - \langle \pi' , \varphi \rangle_\Sigma \notag \\
&= \langle \pi , \varphi' \rangle_\Sigma  - \langle \varphi , \pi' \rangle_\Sigma   \formspace.
\end{align}
So instead of viewing $\mathcal{G}_m$ as a map on one forms, we view it as map on the space of initial data
\begin{align}
\mathcal{G}_m : \Dzs \oplus \Dzs &\to \IC\formspace \\
\big( (\varphi , \pi) , (\varphi' , \pi')  \big) &\mapsto \langle \pi' , \varphi \rangle_\Sigma - \langle \varphi' , \pi \rangle_\Sigma \formspace. \notag
\end{align}
Now, it is straightforward to show that $\mathcal{G}_m$ is a symplectic form: \par
%
\textit{1.) Bilinearity: \\}
Bilinearity follows trivially from the bilinearity of $\langle \cdot , \cdot \rangle$. \par
%
\textit{2.) Alternating: \\}
Let $\psi = (\varphi , \pi) \in \Dzs$. Then
\begin{align}
\Gm{\psi}{\psi}
&=  \langle \pi , \varphi \rangle_\Sigma - \langle \varphi , \pi \rangle_\Sigma \notag\\
&= 0 \formspace.
\end{align}
Therefore, $\mathcal{G}_m$ is anti-symmetric: Specifying $\psi = \psi' + \tilde{\psi}$ and using $0=\Gm{\psi}{\psi}$ together with bilinearity  yields anti-symmetry.\par
%
\textit{3.) Non-degeneracy: \\}
Let $\psi'= (\varphi' , \pi') \in \Dzs$ specify initial data. Assume, for all $\psi = (\varphi, \pi) \in \Dzs$ it holds that
\begin{align}
0
&= \Gm{\psi}{\psi'} \notag\\
&= \langle \pi , \varphi' \rangle_\Sigma - \langle \varphi , \pi' \rangle_\Sigma \formspace \notag\\
\implies \varphi' &= 0 = \pi' \notag\\
\iff (\varphi' , \pi') &= 0 \in \Dzs\formspace.
\end{align}
Hence, $\Gm{\cdot}{\cdot}$ is non-degenerate.
\end{proof}
%
%
%
%
%
%
\newpage
We are now ready to show that the defined morphism $\alpha_\psi$ is injective:
\begin{lemma}\label{lem:alpha_is_injective}
Let $\AA(M)$ be the unital $^*$-algebra as defined in Definition \ref{def:algebra-A(M)} and $\alpha_\psi$ the unit preserving $^*$-algebra homomorphism as defined in Definition \ref{def:morphism_alpha_psi}. It then holds that
\begin{center}
$\alpha_\psi$ is injective.
\end{center}
\end{lemma}
\begin{proof}
Since by Lemma \ref{lem:propagator-non-degenerate} the propagator $\Gm{\cdot}{\cdot}$ is non-degenerate when viewed as a map on initial data\footnote{We will show in Section \ref{sec:BU-algebra} that the dynamical field algebra is homeomorphic to the field algebra of initial data. We can therefore safely view the propagator as a map on initial data.}, the algebra $\AA(M)$ is simple (c.f. \cite[Scholium 7.1]{baez_segal_zhou}). Using Lemma \ref{lem:injective_mor_simple_algebra} this implies that all unit preserving $^*$-algebra homomorphisms are injective.
%Let $\mathcal{J}$ be an ideal in $\AA(M)$ and let $\mathcal{P} \in \mathcal{J}$ be an element of that ideal, that is, in general $\mathcal{P}$ is a polynomial expression in the generators $\A(F)$ of degree $N$.\\
%For any element $\mathcal{P} \in \mathcal{J}$, it follows by definition that $[\mathcal{P}, \A(F)] \in \mathcal{J}$ for an arbitrary $\A(F) \in \AA(M)$. Note, by applying the CCR relation in $\AA(M)$, that $[\mathcal{P}, \A(F)] \in \mathcal{J}$ is a polynomial of degree $(N-1)$. Reapplying this procedure, one finds for $F_1, F_2, \dots , F_N \in \Omega^1_0(\M)$ that
%\begin{align}
%\Big[ \dots \big[ [ \mathcal{P}, \A(F_1) ] , \A(F_2) \big] , \dots \, , \A(F_N) \Big] = c\, \mathbbm{1}
%\end{align}
%for some constant $c \in \IC$. \\
%To see how this constant can be expressed using the propagator kernel $\Green{\cdot}{\cdot}$ it is instructive to look at the case $\mathcal{P} = \A(F_1')  \A(F_2')$:
%\begin{align}
%[\A(F_1)'  \A(F_2)' , \A(F_1)]
%&= \A(F'_1) \, [\A(F'_2) , \A(F_1)] =  [\A(F'_1) , \A(F_1)]\,  \A(F'_2) \\
%&= \i \Green{F'_2}{F_1} \, \A(F'_1) +  \i \Green{F'_1}{F_1} \, \A(F'_2) \formspace.
%\end{align}
%If we commute yet with another generator we find
%\begin{align}
%\big[ [\A(F_1)'  \A(F_2)' , \A(F_1)] , \A(F_2) \big]
%&= \i \Green{F'_2}{F_1} \, [ \A(F'_1), \A(F_2) ] \,\mathbbm{1} +  \i \Green{F'_1}{F_1} \, [  \A(F'_2) , \A(F_2] \,\mathbbm{1}\\
%&= \i^2 \, \Green{F'_2}{F_1} \, \Green{F'_1}{F_2}  \,\mathbbm{1} +  \i^2\,  \Green{F'_1}{F_1} \,\Green{F'_2}{F_2} \,\mathbbm{1}\\
%&= \i^2 \,\mathcal{G}\Big( \Green{F'_2}{F_1} \, F'_1+  \Green{F'_1}{F_1} \, F'_2\, , F_2 \Big)\,\mathbbm{1}
%\end{align}
%where it was used that $\Green{\cdot}{\cdot}$ is bilinear.  \\
%Now, since $\Green{\cdot}{\cdot}$ is non-degenerate, we can choose $F_1$ (or rather the corresponding initial data of $GF_1$) such that at least one of the terms  $\Green{F'_2}{F_1}, \Green{F'_1}{F_1}$ is nonzero and then choose $F_2$ such that the whole expression is non zero. \\
%Applying this procedure for general polynomials one finds that the constant $c$ can be expressed using the propagator kernel $\Green{\cdot}{\cdot}$, that is,
%\begin{align}
%c = \i^N \, \sum \prod \Green{F'_i}{F_j}
%\end{align}
%where the sum and product run over suitable index sets. Again, using the bilinearity one can reexpress this as
%\begin{align}
%c = \mathcal{G} \Big( \Green{ \dots }{ \dots } + \dots , F_N  \Big)
%\end{align}
%Just as in the case of $N=2$ we can choose $F_1$ to $F_N$ from "inside out" such that the constant $c$ is non vanishing.\\
%Therefore we have found that $\mathbbm{1} \in \mathcal{J}$.\\
%By definition it follows for an arbitrary element $a \in \AA(M)$ and $g \in \mathcal{J}$ that $(ag) \in \mathcal{J}$. Specifying $g = \mathbbm{1}$ we find
%\begin{align}
%\AA(M) \subset \mathcal{J}
%\end{align}
%and, since by definition $\mathcal{J} \subset \AA(M)$, that
%\begin{align}
%\mathcal{J} = \AA(M) \formspace.
%\end{align}
%Therefore there is no non trivial ideal in $\AA(M)$, hence $\AA(M)$ is simple.
\end{proof}
%
%
%
%
%
%
%
%
%
%
%
With this, we are ready to state the first major result of this chapter, that is, we have constructed a generally locally covariant quantum theory of the Proca field in curved spacetimes including external sources,  in form of the following theorem:
\begin{theorem}
Let $\SpacCurr$ and $\Alg'$ be the categories as defined in Definition \ref{def:categories_alg_spaccurr}.\\
Let $\AA : \SpacCurr \to \Alg'$ be as defined in Definition \ref{def:algebra-A(M)} (action on objects) and Definition \ref{def:morphism_alpha_psi} (action on morphisms). Then
\begin{center}
$\AA$ is a covariant functor,
\end{center}
that is it describes the generally covariant quantum theory of the Proca field in (globally hyperbolic) spacetimes.
Furthermore, the image of the functor is contained in $\Alg'$. Therefore, the theory is local.
\end{theorem}
\begin{proof}
We have already proven most of the statement by showing that $\AA$ is well defined and that $\alpha_\psi \equiv \AA(\psi)$ is injective (see Lemma \ref{lem:alpha_is_injective}).\\
%
%
%
For $\AA$ to be a (covariant) functor, we need to show that it behaves well under composition of morphisms, that is for any objects $M,N,K \in \mathsf{Obj}_\SpacCurr$, where $M=(\M,g,j_\M)$, $N=(\N,g_\N,j_\N)$ and $K=(\K,g_\K,j_\K)$, and morphisms $\psi \in \textsf{Mor}_\SpacCurr(M,N)$ and $\phi \in \textsf{Mor}_\SpacCurr(N,K)$, it holds that
\begin{align}
\alpha_{\phi \,\comp\, \psi} = \alpha_\phi \comp \alpha_\psi \formspace,
\end{align}
and that it maps the identity $\text{id}_M \in \mathsf{Hom}_\SpacCurr(M,M)$ to the identity $\text{id}_{\AA(M)} \in  \mathsf{Hom}_\Alg\big(\mathscr{A}(M),\mathscr{A}(M)\big)$, which it trivially does by definition.
The behavior under composition follows directly from the definition of $\alpha_\psi$. Let $\psi : (\M,g,jJ_\M) \to (\N, g_\N ,j_\N)$ and $\varphi : (\N, g_\N ,j_\N) \to (\mathcal{K}, g_\K, j_\mathcal{K})$ two orientation preserving hyperbolic isometric embeddings. Then
\begin{align}
\alpha_{\phi\, \comp \, \psi} \big( \A_\M(F) \big)
&= \A_\mathcal{K} \big( (\phi \comp \psi )_* F \big) \notag \\
&= \A_\mathcal{K} \big(\phi_*(\psi_* F) \big) \notag\\
&= \alpha_\phi \big(\A_\N(\psi_* F) \big) \notag\\
&= \alpha_\phi \Big( \alpha_\psi \big(\A_\M (F) \big) \Big) \notag\\
&= (\alpha_\phi \comp \alpha_\psi)\big( \A_\M (F) \big)  \formspace.
\end{align}
Therefore, $\AA$ is a covariant functor.
\end{proof}
%
%
%
%
%
%
%
%
%
%
\subsection{The Borchers-Uhlmann algebra as the field algebra}\label{sec:BU-algebra}
One algebra that can be used to describe quantum fields in curved spacetimes is the \emph{Borchers-Uhlmann algebra} (BU-algebra), which was studied by \name{Borchers} \cite{borchers} and
\name{Uhlmann} \cite{uhlmann} in 1962. First applied to the case of quantum field theory on Minkowski spacetime, in particular in connection to the Wightman $n$-point functions, the BU-algebra is well suited to generalize to the curved spacetime case. The BU-algebra can be constructed over any vector space which in our case will be the space $\Omega^1_0(\M)$ of compactly supported test one-forms. The construction presented here follows \cite[Chapter 4.1]{verch_sahlman} and \cite[Chapter 8.3.2]{haag}.\par
The BU-algebra $\BUOmega$ is defined as the tensor algebra of the vector space $\Omega^1_0(\M)$. That means elements $f \in \BUOmega$ are tuples $f= (f^{(0)}, f^{(1)}, f^{(2)}, \dots)$, where $f^{(0)} \in \IC$ and for $i>0, f^{(i)} \in \big(\Omega^1_0(\M)\big)^{\otimes i}$, such that only finitely many $f^{(i)}$'s are non-vanishing\footnote{As noted in \cite[Chapter 3.3]{verch_sahlman}, we can alternatively view the $f^{(i)}$'s as smooth sections of the $i$-fold outer product bundle $T^*\M \boxtimes T^*\M \boxtimes \cdots \boxtimes T^*\M$ over $\M ^i$.}. Here, $\otimes$ denotes the algebraic tensor product, without taking any topological completion. We will call the component $f^{(i)}$ the \emph{degree-i-part} of $f$.\newpage
Formally, we summarize in the following definition:
\begin{definition}[Borchers-Uhlmann algebra]
Let $V$ be a vector space. The Borchers-Uhlmann algebra over $V$ is defined as
\begin{align}
	\gls{bualgebra} = \IC \oplus \bigoplus\limits_{n= 1}^\infty V^{\otimes n} \formspace.
\end{align}
\end{definition}
%
\noindent
Again, $V \otimes V \otimes \cdots \otimes V$ denotes the algebraic tensor product without any topological completion.
This definition makes $\BUOmega$ a vector space (addition and scalar multiplication is defined component wise), which can be endowed with a $*$-algebraic structure: \\
For two elements $f, g$ we define the product $(f\cdot g)$ by defining the degree-$n$-part to be
\begin{align}
	(f\cdot g)^{(n)} = \sum_{i+j = n} f^{(i)} \otimes g^{(j)} \formspace,
\end{align}
which is equivalent to specifying
\begin{align}
(f\cdot g)^{(n)} (p_1,p_2, \dots , p_n) = \sum_{i+j = n} f^{(i)} (p_1, p_2, \dots , p_i) g^{(j)} (p_{i+1} , \dots , p_n) \formspace,
\end{align}
where $p_i \in \M$. The involution is defined by complex conjugation and reversing the order of the arguments in every degree:
\begin{align}
	(f^*)^{(n)} (p_1, \dots , p_n) = \bar{f}^{(n)}(p_n, p_{n-1}, \dots , p_1) \formspace.
\end{align}
With these two additional operations, $\BUOmega$ is a *-algebra. Furthermore, defining a unit element $\mathbbm{1}_{\BU(\Omega^1_0(\M))} = (1, 0, 0, \dots)$, $\BUOmega$ becomes a unital *-algebra. The reason why the BU-algebra is well suited for our problem is that it can be endowed with a topology which will later allow us to define a notion of continuously varying the mass $m$ and compare the corresponding quantum fields with each other.
%
%
%
%
%
\subsubsection{The topology of the Borchers-Uhlmann algebra}
In the following we present the basic ideas to defining a (locally convex) topology on the BU-algebra over the space of compactly supported one-forms without going into too much of the necessary details since this procedure is well understood. At every step we make explicit references to the missing details or proofs.
As a first step, we need to find a topology on the space $\Omega^1_0(\M)$ of compactly supported one-forms which will be used for the construction of the topology on the BU-algebra.  The construction follows \cite[Chapter 17.1 to 17.3]{dieudonne_3} where all presented statements are made rigorous.
First we define a locally convex topology on the space of smooth functions $C^\infty(U)$ for an open $U \subset \IR^n$. A locally convex topology is induced by a family of semi-norms\footnote{See \cite[Theorem 12.14.3]{dieudonne_2} or for a general introduction to locally convex topological vector spaces \cite[Chapter 7, in particular page 63 ]{treves} and \cite[Chapter 12.14]{dieudonne_2}.}. Let $\left\{ K_l \right\}_l$ be an increasing family of compact sets of $U$, such that $\left\{ K_l \right\}_l$ is a covering of $U$. We will call such a family a \emph{fundamental sequence}. We define the semi-norm for a class $C^\infty$ function $f: U \to \IC$
\begin{align}
	p_{s,l}(f) = \sup\Big\{ \abs{D^\nu f(x)} : x \in K_l, \abs{\nu} \leq s\Big\} \formspace,
\end{align}
where $\nu \in \IN^n$ is a multi-index, $\abs{\nu} = \sum_{i=1}^n \nu_i$ and the derivative operator is defined as
\begin{align}
D^\nu f (x_1, \dots , x_n) = \frac{\partial^{\nu_1} \, \partial^{\nu_2} \dots\, \partial^{\nu_n}  }{\partial_{x_1}^{\nu_1} \,\partial_{x_2}^{\nu_2} \dots \, \partial_{x_n}^{\nu_n}} \formspace.
\end{align}
Actually, the above definition also endows the space of $n$-times differentiable functions $C^n(U)$ with a locally convex topology, which we will use for the generalization in a few steps.
This procedure generalizes to compactly supported differential forms on a smooth mani\-fold. Even more general, let $\mathfrak{X}=(E, \M,\pi)$ be a vector bundle over $\M$ of rank $n$. We will define a topology on the space $\Gamma(E,U)$ of smooth sections of $E$ over $U$. The basic idea is to use local charts of the manifold and a local trivialization of the vector bundle to define a family of semi-norms on $\Gamma(E,U)$ using the semi-norms we have specified above for functions $f\in C^\infty(U)$. Let $\left\{U_\alpha\right\}_\alpha$ be a locally finite covering of $U \subset \M$ such that there are local charts $(\psi_\alpha, U_\alpha)$ of the manifold $\M$. For each $\alpha$, the map $z \mapsto \Big( \psi_\alpha\big(\pi(z)\big) , v_{1\alpha} , v_{2 \alpha} , \dots , v_{n \alpha} \Big)$ from the fibers $\pi^{-1} (U_\alpha)$ to $\psi_\alpha (U_\alpha) \times \IC^n$ is a diffeomorphism. Such linear diffeomorphisms $v_{i\alpha}$ exist for any open neighborhood of $\M$ as they can be defined as the components of a local trivialization of the bundle $E$ over the neighborhood $U_\alpha$.  Let $u_\alpha$ be the restriction of a section $u \in \Gamma(E,U)$ to $U_\alpha$. Finally, define for every $\alpha$ the fundamental sequence  $\left\{ K_{l\alpha} \right\}_l$ of compact sets in $\psi_\alpha(U_\alpha)$ and denote the semi-norms on $C^n(\psi_\alpha(U_\alpha))$ as specified above by $\tilde{p}_{s, l, \alpha}$. Then
\begin{align}
	p_{s,l,\alpha}(u) = \sum_{j=1}^{n} \tilde{p}_{s, l, \alpha} \left( v_{j\alpha} \comp u_\alpha \comp \psi^{-1}_\alpha \right)
\end{align}
defines a family of semi-norms on $\Gamma(E,U)$ (see \cite[Equation 17.2.1]{dieudonne_2}). Now we specify to the vector bundles $\wedge^p T^* \M$ whose smooth sections are smooth $p$-forms. The space $\Gamma(\wedge^p T^* \M , \M)$, together with the family of semi-norms specified above, is a locally convex topological vector space. For every compact $K \subset \M$, the space $\Gamma(\wedge^p T^* \M,K)$ denotes the set of $p$-forms that are compactly supported in $K$. This space is a closed subspace of $\Gamma(\wedge^p T^* \M,\M)$. The union of $\Gamma(\wedge^p T^* \M,K)$ over all compact subspaces $K \subset \M$ is a locally convex topological vector space, the space of compactly supported $p$-forms\footnote{To be precise, for compact $K \subset \M$, the spaces $\Gamma(\wedge^p T^* \M, K)$ are Fr\'echet spaces \cite[Theorem 17.2.2]{dieudonne_3} and the space of compactly supported $p$-forms is defined as the inductive limit of the spaces $\Gamma(\wedge^p T^* \M,K)$, making it a LF-space which in particular is locally convex.} which we have already denoted by $\Omega^p_0 (\M)$. This in particular yields a locally convex topology on the space $\Omega^1_0(\M)$ that we are interested in.\par
Having found a topology on $\Omega^1_0(\M)$ it is a straightforward procedure to endow $\big(\Omega^1_0(\M)\big)^{\otimes n}$ with a locally convex topology for every $n$: We equip $\big(\Omega^1_0(\M)\big)^{\otimes n}$ with the \emph{projective topology} which is induced by formal tensor products of the semi-norms on $\Omega^1_0(\M)$ (see \cite[Definition 43.2 and Chapter 43]{treves} for details).  For every $n \in \IN$ we define
\begin{align}
	\BU_n = \IC \oplus \bigoplus_{i=1}^n \big(\Omega^1_0(\M)\big)^{\otimes i} \formspace,
\end{align}
which yields a family $\left\{\BU_n\right\}_n$ of locally convex topological vector spaces, where the topology on each $\BU_n$ is given as the direct sum topology\footnote{For a definition, see for example \cite[515]{treves}.}.
The BU-algebra is then endowed with the so called inductive limit topology\footnote{Again, for a definition see \cite[514]{treves}.} of the family $\left\{\BU_n\right\}_n$ (see \cite[Appendix B]{verch_sahlman}). With this construction, the BU-algebra is a locally convex topological *-algebra with unit (c.f. \cite[Lemma 4.1]{verch_sahlman}).
%
%
%
%
%
\subsubsection{Dynamics, commutation relations and the field algebra}
So far, the constructed BU-algebra, which we would like to use as a field algebra, does neither incorporate any dynamics, in our case are given by the Proca equation, nor the desired quantum commutation relations. We want to identify quantum fields $\phi$ as elements $\phi(F) = (0, F, 0, 0, \dots)$ of an appropriate field algebra. To endow the algebra with dynamics,  the fields $\phi$ have to solve the Proca equation in a distributional sense. Furthermore, we incorporate the canonical commutation relations (CCR) in the field algebra. We will do this, for reasons that will become clear in the next section, in a two step procedure. Throughout this section, the mass dependence is again made explicit in the notation, but the mass $m$, as well as the external source $j$, are assumed to be fixed.\par
First we will divide out the two-sided ideal $\IMJDYN$ in $\BUOmega$ that is generated by elements
\begin{align}
\big(-\langle j, F \rangle_\M, (\delta d + m^2)F,0,0,\dots\big) \in \BUOmega\formspace,
\end{align}
for $F \in \Omega^1_0(\M)$, to implement the dynamics. That means, by definition, an element $f \in \IMJDYN$ can be written as a finite sum
\begin{align}
f = \sum_i g_i \cdot \left(-\langle j, F_i \rangle_\M, (\delta d + m^2)F_i,0,0,\dots\right) \cdot h_i \formspace,
\end{align}
for some $F_i \in \Omega^1_0(\M)$ and $g_i, h_i \in \BUOmega$. We define
\begin{align}
	\BUmjdyn \coloneqq {\Quotientscale{\BUOmega}{\IMJDYN}} \formspace.
\end{align}
Elements $f \in \BUmjdyn$ are then equivalence classes $ f = \left[ g \right]_m^\text{dyn}$, $g \in \BUOmega$, where the equivalence relation is given for any $g,h \in \BUOmega$ by
\begin{align}
g \sim_m h :\iff g-h \in \IMJDYN \formspace.
\end{align}
Now, in the second step, we incorporate the CCR by dividing out the two-sided ideal $\IMCCR$ that is generated by elements
\begin{align}
\Big[ \big(-\i \Gm{F}{F'}, 0 , F \otimes F' - F' \otimes F, 0 , 0 , \dots\big) \Big]_m^\text{dyn} \in \BUmjdyn
\end{align}
to obtain the final field algebra $\BUmj$ as specified by the following definition:
\begin{definition}[Field algebra and quantum Proca fields]
The \emph{Borchers-Uhlmann field algebra} $\BUmj$, for some fixed $m>0$, is defined by
	\begin{align}
     \gls{bumj} \coloneqq {\Quotientscale{\BUmjdyn}{\IMCCR}} \formspace.
	\end{align}
We will sometimes equivalently write $\BUmj = {\Quotientscale{\BUOmega}{\IMJ}}$, where $\IMJ$ is generated by both of the wanted relations for short hand.
A \emph{quantum Proca field} is then an element
\begin{align}
	\gls{phimf} \coloneqq [(0,F,0,0,\dots)]_{m,j} \in \BUmj \formspace,
\end{align}
where the equivalence class $[\cdot]_{m,j}$ is taken w.r.t. $\IMJ$.
\end{definition}
By construction, the quantum Proca fields fulfill the wanted dynamical and commutation relations
\begin{align}
	\phi_{m,j}( (\delta d +m^2) F ) 				&= \langle j, F \rangle_\M \cdot \mathbbm{1}_{\BUmj} \formspace,\\
	\big[ \phi_{m,j}(F) , \phi_{m,j}(F') \big] 	   &= \i \Gm{F}{F'} \cdot  \mathbbm{1}_{\BUmj} \formspace.
\end{align}
We still have to endow the field algebra $\BUmj$ with a topology.
%
%
%
%
%
\subsubsection{Topology, initial data and the field algebra}\label{sec:field-algebra-topology}
In this chapter, as a preparation for the investigation of the mass dependence and the zero mass limit of the theory, we investigate the topology of the field algebra and its connection to a field algebra of initial data. The mass $m$ as well as the external source $j$ again remain fixed.\par
The straightforward way to obtain a topology on $\BUmj$ is from the topology on $\BUOmega$ as the quotient topology, assuming the ideal $\IMJ$ is closed\footnote{We will argue shortly that, at least in the case of $j=0$, it is.}:
If $\IMJ$ is a closed subspace of $\BUOmega$, and $\BUOmega$ is endowed with a locally convex topology, also $\BUmj$ is locally convex \cite[Theorem 12.14.8]{dieudonne_2}. Indeed, the quotient topology on $\BUmj$ coincides with the topology induced by the semi-norms
\begin{align}
	q_{m, \alpha}( [f]_m ) = \inf\big\{ p_\alpha(g) : g \in [f]_m \big\}
\end{align}
where $\left\{ p_\alpha\right\}_\alpha $ is a family of semi-norms on $\BUOmega$ that induces its topology (see \cite[Lemma 12.14.8]{dieudonne_2}).
While this topology allows us to define continuous families $\left\{ \phi_{m,j,n}(F_n)\right\}_n$ of fields at a fixed mass, it is not suitable, as we will discuss in the next Section \ref{sec:mass_depenence_and_limit}, to define a notion of continuity with respect to the mass $m$.
In principal, the problem is that at different masses the fields live in different algebras that we are a priori unable to compare with each other. The idea to solve this is simple: we will for every fixed mass find a topological algebra that is \emph{homeomorphic to the field algebra} $\BUmj$ which does not depend on the mass (the homeomorphism of course does). We are then able to map a family $\{ \phi_{m,j} (F_m)\}_m$ of fields into only \emph{one} topological algebra where we have a natural sense of continuity given by the topology. This mass independent algebra will be the Borchers-Uhlmann algebra of \emph{initial data}.\par
To start the construction, we first look at the  topological vector space $\Omega^1_0(\M)$. Let $\Sigma$ be a Cauchy surface. As done at the classical level, we introduce the short notation
$\Dzs = \Omega^1_0(\Sigma) \otimes \Omega^1_0(\Sigma)$ for the space of initial data with respect to a fixed Cauchy surface $\Sigma$. We define the map
\begin{align}
	\kappa_m: \Omega^1_0(\M) &\to \Dzs  \\
					F &\mapsto (\rhoz G_m F , \rhod G_m F) \formspace, \notag
\end{align}
where the operators $\rho_{(\cdot)}$ were introduced in Definition \ref{def:cauchy_mapping_operators}. The map $\kappa_m$ maps a test one-form $F$ to the initial data of $G_m F$ on the Cauchy surface $\Sigma$. In the notation we omit the dependence of the map on the Cauchy surface. By construction, the map $\kappa_m$ is continuous for a fixed value of $m$ with respect to the topology on $\Dzs$ that is induced by the topology of $\Omega^1_0(\Sigma)$.
Since $\kappa_m$ is continuous, we know that $\KERN{\kappa_m}$ is closed (see \cite[34-36 ]{treves}). By construction the kernel of the map $\kappa_m$ is the set
\begin{align}
	\KERN{\kappa_m} = \JMDYN \coloneqq \big\{ (\delta d + m^2) F , F \in \Omega^1_0(\M) \big\} \formspace.
\end{align}
This is useful since at the field algebra level we want to divide out fields where the degree-one-part is of the above form to incorporate the dynamics of the theory.
We would like to find a homeomorphism between the quotient space ${\Quotientscale{\Omega^1_0(\M)}{\JMDYN}}$ and $\Dzs$.
From a standard construction (see \cite[ibid.]{treves}) we find the map $\xi_m : {\Quotientscale{\Omega^1_0(\M)}{\JMDYN}} \to \IMG{\kappa_m}$ which is the unique bijective map such that $\xi_m([F]_m) = \kappa_m(F)$ (see \cite[16]{treves}). The construction is illustrated in Diagram \ref{dia:homeomorphism_one_particle_level}.
\begin{table}
\begin{displaymath}
\xymatrix @R=20mm @C=30mm
{
	\Omega^1_0 (\M)  \ar[r]^{\kappa_m}   \ar[dr]_{[\cdot]_m}  			&     \IMG{\kappa_m} 													\ar@<-.5ex>[d]_{\xi_m^{-1}} \ar@{^{(}->}[r]^i &   \Dzs \\
	&      {{\Quotientscale{\Omega^1_0(\M)}{\JMDYN}}}          \ar@<-.5ex>[u]_{\xi_m}
}
\end{displaymath}
\caption{Illustrating the construction of the homeomorphism $\xi_m$ of the space of dynamical test one-forms and the space of initial data.}
\label{dia:homeomorphism_one_particle_level}
\end{table}
%
Now we would like to show that $\IMG{\kappa_m} = \Dzs$ and that $\Dzs$ and ${\Quotientscale{\Omega^1_0(\M)}{\IMDYN}}$ are homeomorphic, that is, we need to show that $\xi_m$ and $\xi_m^{-1}$ are continuous. We will state this in form of the following lemma:
\begin{lemma}\label{lem:one-particle-homeomorphism}
	Let $\Sigma$ be a Cauchy surface and $\Dzs$ be the space of initial data on $\Sigma$.
	\begin{center}
	The spaces ${\Quotientscale{\Omega^1_0(\M)}{\JMDYN}}$ and $\Dzs$ are homeomorphic.
	\end{center}
\end{lemma}
\begin{proof}
1.) First we will show that $\kappa_m$ is surjective, that is, for every initial data $(\varphi, \pi)$ we find a corresponding $F \in \Omega_0^1(\M)$ such that $(\varphi , \pi) = (\rhoz G_m F , \rhod G_m F)$.
For this, we explicitly construct a map $\vartheta : \Dzs \to \Omega^1_0(\M)$ that maps any pair $(\varphi , \pi)$ to a corresponding $F$.\par
Let $(\varphi,\pi) \in \Dzs$ specify initial data on a Cauchy surface $\Sigma$. Then by Theorem \ref{thm:solution_proca_unconstrained} there exists a unique solution $A \in \Omega^1(\M)$ to the source free Proca equation $(\delta d + m^2) A = 0$ with the given data. Furthermore, $A$ depends continuously on $(\varphi,\pi)$.\\
We need to construct a \emph{compactly supported} one-form $F$ from $A$, such that $A$ and $G_mF$ determine the same initial data. First, we note that (see \cite[Theorem 3.2.11]{baer_ginoux_pfaeffle})
\begin{align}
	\supp{A} \subset J\big( \supp{\varphi} \cup \supp{\pi} \big) \formspace.
\end{align}
We choose a $\chi \in \Omega^1(\M)$ such that
\begin{align}
	\chi =
	\begin{cases}
	1 , &\text{in the future of some Cauchy surface } \Sigma_+ \\
	0 , &\text{in the past of some Cauchy surface } \Sigma_-
	\end{cases}
\end{align}
where $\Sigma_\pm$ are Cauchy surfaces in the future/past of the Cauchy surface $\Sigma$.\\
Then, by construction,
\begin{align}
	F \coloneqq -(\delta d + m^2) \chi A \eqqcolon \vartheta(\varphi,\pi)
\end{align}
is a compactly supported one-form, since $F = 0$ on $J^+(\Sigma_+)$ and $J^-(\Sigma_-)$, hence $\supp{F} \subset J\big( \supp{\varphi} \cup \supp{\pi}\big) \cap  J^-(\Sigma_+) \cap J^+(\Sigma_-)$ which is compact. Furthermore we observe that, since $A$ depends continuously on the initial data, $\vartheta$ is continuous. The setup is illustrated in Figure \ref{fig:init_data_chi}.
\begin{figure}[]
	\begin{center}
		\scalebox{0.9}{
		\begin{tikzpicture}
		\node at (0,0) {\includegraphics[scale=0.9]{./img/img1.pdf}};
		\node at (1,3.5)  {$\supp{A}$};
		\node at (-3.7,3.5)  {$\chi$};
		\node at (-5.2,-4.7)  {$0$};
		\node at (-3.3,-4.7)  {$1$};
		\node at (5.65,2.25)  {$\Sigma_+$};
		\node at (5.65,0.01)  {$\Sigma$};
		\node at (5.65,-2.25)  {$\Sigma_-$};
		\node at (5.3,-0.9)  {$\supp{\varphi} \cup \supp{\pi}$};
		\end{tikzpicture}
		}
	\end{center}
	\caption{Illustrating the setup of the proof of Lemma \ref{lem:one-particle-homeomorphism}: On the Cauchy surface $\Sigma$, $\varphi$ and $\pi$ denote initial data of the solution $A$ to Proca's equation. The area marked grey contains the support of the one-form $F = -(\delta d + m^2) \chi A$.}
	\label{fig:init_data_chi}
\end{figure}
We finally want to show that $G_m F = A$, since then they in particular have the same initial data $(\varphi, \pi)$ which would complete the proof of surjectivity.
A priori, the domain of $G_m^\pm$ is $\Omega^1_0(\M)$ but we can extend its action to one-forms that are supported in the future (or past) of some Cauchy surface, as it is the case for $(\chi A)$ which is supported in the future of $\Sigma_-$. We extend the action by defining
\begin{align}
	G_m^+(\chi A) = \sum\limits_i  G_m^+ \alpha_i \chi A
\end{align}
where $\alpha_i$ is a partition of unity and at any point $x \in \M$ only for finitely many $i$'s $\left( G_m^+ \alpha_i \chi A\right)(x)$ is non-zero since $J^- (x) \cap J^+(\Sigma_-)$ is compact (and for the retarded propagator it holds $\supp{G_m^+ F} \subset J^+\big( \supp{F}\big)$). In the same fashion we extend the action of $G_m^-$ to act on one-forms that are supported to the past of some Cauchy surface as it is the case for $(1-\chi)A$:
\begin{align}
	G_m^-(1-\chi)A = \sum\limits_i  G_m^+ \alpha_i (1-\chi) A \formspace.
\end{align}
With this notion we find
\begin{align}
	G_m^+ F
	&= - G_m^+ (\delta d + m^2) \chi A \notag\\
	&= -  \sum_i G_m^+ \alpha_i (\delta d + m^2) \chi A\notag \\
	&= -  \sum_i G_m^+  (\delta d + m^2) \alpha_i \chi A \notag\\
	&= -  \sum_i \alpha_i \chi A\notag \\
	&= - \chi A \formspace.
\end{align}
Also, we observe that
\begin{align}
	G_m^- (\delta d + m^2) (1-\chi) A
	&=  G_m^- \underbrace{(\delta d + m^2)  A}_{= 0} - G_m^- (\delta d + m^2)\chi A \notag\\
	&= - G_m^- (\delta d + m^2)\chi A \notag\\
	&= G_m^- F \formspace,
\end{align}
and therefore we find in the same fashion as above
\begin{align}
G_m^- F
&=  G_m^- (\delta d + m^2) (1-\chi) A \notag\\
&=  \sum_i G_m^- \alpha_i (\delta d + m^2) (1-\chi) A \notag\\
&=  (1-\chi) A \formspace.
\end{align}
We therefore find the result
\begin{align}
G_m F
&= (G_m^- - G_m^+)F \notag\\
&= (1- \chi) A + \chi A \notag\\
&= A
\end{align}
which completes the proof of surjectivity. That is, we have found $\IMG{\kappa_m} = \Dzs$.\par
%%
%%
2.) Now we are left to show that both $\xi_m$ and $\xi_m^{-1}$ are continuous so that indeed $\xi_m$ is a homeomorphism.\par
\emph{i)} By construction, $\xi_m$ is continuous if and only if $\kappa_m$ is continuous \cite[Proposition 4.6]{treves}, which, as we have argued, is the case.\par
\emph{ii)} The inverse is by construction given by $\xi_m^{-1} = [\cdot]_m \comp \,\vartheta$. As we have argued, the map $\vartheta$ is continuous. Also by construction, the map $[\cdot]_m$ is continuous which yields that indeed $\xi^{-1}_m$ is continuous.
This completes the proof.
\end{proof}
\par We will now generalize these ideas, by explicit use of the constructed maps, to the field algebra $\BUmj$.
First, we set the external source to vanish, $j = 0$. We make this explicit in the notation by indexing the effected elements with $0$ instead of $j$. In a second step we will then generalize to non vanishing external sources.\par
On the level of $\BUOmega$, to implement the dynamics, we divide out the ideal generated by $\big(0,(\delta d + m^2)F',0,0,\dots\big)$  where $F'\in \Omega^1_0(\M)$. This is equivalent to the ideal generated by $= \big(0,F,0,0,\dots\big)$, $F \in \JMDYN$. As we did on the degree-one level, we would like to find a map $K_m : \BUOmega \to \BU(\Dzs)$ such that $\KERN{K_m} = \IMZDYN$ and then show that $\BUmzdyn$ is homeomorphic to $\BU(\Dzs)$. We do this by \emph{lifting} the map $\kappa_m$ to the BU-algebra:
We define the map $\gls{Km} : \BUOmega \to \BU(\Dzs) $ as a BU-algebra-homomorphism which is then completely determined by its degree wise action:
\begin{align}
	K_m : \BUOmega &\to \BU(\Dzs) \\
	(f^{(0)},0,0,\dots) &\mapsto 	(f^{(0)},0,0,\dots) \notag \\
	\big(0,F,0,0,\dots\big) &\mapsto 	\big(0, \kappa_m (F),0,0,\dots\big)\notag \\
		\left(0,0, \sum\nolimits_i F_i \otimes F_i',0,0,\dots\right) &\mapsto 	\left(0,0, \sum\nolimits_i \kappa_m (F_i) \otimes \kappa_m (F'_i) ,0,0,\dots\right) \notag\\
		&\dots\notag
\end{align}
We will call this definition the \emph{lifting} of a map from the space of test one-forms to the BU-algebra.
Before we can proceed to show homeomorphy we need to investigate some properties of the map $K_m$.
\begin{lemma}
	Let $\Sigma$ be a Cauchy surface, $\Dzs$ be the space of initial data on $\Sigma$ and the map $K_m : \BUOmega \to  \BU\big(\Dzs\big) $ be as defined above. Then:
	\begin{align}
	\IMG{K_m} &=  \BU\big(\Dzs\big)  \formspace \text{and}\\
		\KERN{K_m} &= \IMZDYN \formspace.
	\end{align}
\end{lemma}
\begin{proof}
1.) As we have already shown in the proof of Lemma \ref{lem:one-particle-homeomorphism}, $\kappa_m$ is surjective. Therefore, it directly follows $\IMG{K_m} = \BU\big(\Dzs\big) $.\par
2.) We will show the equivalence of the kernel and the ideal by a two way inclusion.\par
\emph{i)} It is obvious by construction that $\IMZDYN \subset \KERN{K_m} $:
Let $f \in \IMZDYN$ be arbitrary. By definition, we can write $f$ as a finite sum
\begin{align}
f = \sum\limits_i  g_{i} \cdot \left(0,F_i,0,0,\dots \right) \cdot h_{i} \formspace,
\end{align}
where $F_i \in \JMDYN$ and $g_i, h_i \in \BUOmega$. Since $K_m$ was constructed as a homomorphism and $\KERN{\kappa_m}  = \JMDYN$, we find
\begin{align}
K_m(f)
= \sum\limits_i  K_m\left(g_{i}\right) \cdot \underbrace{\left(0,\kappa_m(F_i),0,0,\dots \right)}_{=0} \cdot K_m\left(h_{i}\right)
= 0 \formspace.
\end{align}
Hence we find $f \in \KERN{K_m}$. \par
%
%
%
\emph{ii)} The non-trivial part is to show $\KERN{K_m} \subset \IMZDYN$:\\
Let $f = (f^{(0)}, f^{(1)}, f^{(2)}, \dots , f^{(N)} , 0 , 0 ,\dots) \in \KERN{K_m}$, $f^{(k)} \in \left(\Omega^1_0(\M)\right)^{\otimes k}$, be arbitrary.
With a slight abuse of notation\footnote{We abuse the notation by applying $\kappa_m$ to a tensor product of differential forms, that is, write for shorthand $\kappa_m(\sum_i F_i \otimes G_i) =\sum_i \kappa_m(F_i) \otimes \kappa_m(G_i)$.} we formulate the condition on $f$ being an element of $\KERN{K_m}$ by stating $\kappa_m\left(f^{(k)}\right)  = 0$ for all $k=1,\dots,N$.
We will show by induction in $n$ that an arbitrary homogeneous element $(0,0, \dots , f^{(n)} , 0 ,0, \dots)$ of degree $n$ with $\kappa_m\left(f^{(n)}\right)  = 0$ can be written in the form $\sum_i  g_{i} \cdot \left(0,F_i,0,0,\dots \right) \cdot h_{i}$ for some $g_{i}, h_{i} \in \BUOmega$, that is, it is an element of $\IMZDYN$. Since $f$ can be written as a sum of those elements, it then follows that $f \in \IMZDYN$.
%
%
%
We begin the induction with the base case:\par
\emph{a)} A homogeneous element $f'$ of degree zero in the kernel of $K_m$ can be trivially shown to be an element of $\IMZDYN.$ Since it holds $\kappa_m(f'^{(0)}) = 0$ for all $f'^{(0)} \in \IC$ it follows by definition $f'^{(0)} = 0$ and, trivially, $(0,0,\dots) \in \IMZDYN$.\par
\emph{b)} For a homogeneous element $f'$ of degree one in the kernel of $K_m$ we find the condition $\kappa_m(f'^{(1)}) = 0$, or equivalently $f'^{(1)}  \in \JMDYN$. Therefore, also $(0, f'^{(1)} , 0 ,0 ,\dots) \in \IMDYN$. \par
We can now make the induction step.
The assumption is that for some $n$ it holds for an arbitrary $f^{(n)} \in \left(\Omega^1_0(\M)\right)^{\otimes n}$ with $\kappa_m(f^{(n)}) = 0$ that
$(0,0,\dots,f^{(n)},0,0,\dots) \in \IMZDYN$.
Now we look at an element of the form $(0,0,\dots,0,f^{(n+1)},0,0\dots)$ where $f^{(n+1)} \in \left(\Omega^1_0(\M)\right)^{\otimes (n+1)}$ such that $\kappa_m(f^{(n+1)}) = 0$.
We can write this more explicitly for some $F_i \in \Omega^1_0(\M)$ and some $\F_i^{(n)}\in \left(\Omega^1_0(\M)\right)^{\otimes n}$ as
\begin{align}
(0,0,\dots,0,f^{(n+1)},0,0, \dots)  = (0,0,\dots,\sum\limits_{i=1}^{M} F_i \otimes \F_i^{(n)},0, 0 , \dots) \formspace.
\end{align}
%
Let $V \coloneqq \SPAN{F_1,F_2, \dots , F_M}$ and $V_{\J} \coloneqq V \cap \JMDYN$ define finite dimensional subspaces of $\Omega^1_0(\M)$. We find a basis $\{ \widetilde{F}_1, \dots ,\widetilde{F}_\mu \}$, $\mu \leq M$, of $V_{\J}$ which we can extend to a basis $\{ \widetilde{F}_1, \dots ,\widetilde{F}_M \}$ of $V$.\\
With the use of this basis we can re-write
\begin{align}
f^{(n+1)} = \sum\limits_{i=1}^{M} F_i \otimes \F_i^{(n)}
&= \sum\limits_{i=1}^{M} \widetilde{F}_i \otimes \widetilde{\F}_i^{(n)} \notag\\
&= \sum\limits_{i=1}^{\mu} \widetilde{F}_i \otimes \widetilde{\F}_i^{(n)}  +  \sum\limits_{i=\mu + 1}^{M} \widetilde{F}_i \otimes \widetilde{\F}_i^{(n)} \notag\\
& \eqqcolon X_1^{(n+1)} + X_2^{(n+1)} \formspace.
\end{align}
Here, each $\widetilde{\F}_i^{(n)}$ can be constructed as linear combinations of the ${\F}_i^{(n)}$'s.\\
Now we first have a look at $X_1^{(n+1)}$. We know by construction for $i=1,\dots,\mu$ that  $\kappa_m ( \widetilde{F_i} ) = 0$. Therefore, $\kappa_m(X_1^{(n+1)}) = 0$. Hence, $X_1^{(n+1)}$ is already of the wanted form, that is, we can choose $g_{i}, h_{i} \in \BUOmega$, such that for $\alpha = 1, \dots , \mu$
\begin{align}
	g^{(0)}_{\alpha} &= 1 \\
	F_\alpha &= \widetilde{F}_\alpha \\
	h^{(n)}_{\alpha} &= \widetilde{\F}_\alpha^{(n)}
\end{align}
and all remaining components of $g_{\alpha}, h_{\alpha}$ vanish. With this we have brought the first part into the wanted form
\begin{align}
(0,0,\dots,X_1^{(n+1)},0,0, \dots)
&= \sum\limits_{\alpha=1}^\mu  \left(0,0,\dots,\widetilde{F}_\alpha \otimes \widetilde{\F}_\alpha^{(n)},0,0,\dots \right) \notag\\
&= \sum\limits_{\alpha=1}^\mu  g_{\alpha} \cdot \left(0, F_\alpha, 0 ,0,\dots \right) \cdot h_{\alpha} \in \IMZDYN \formspace.
\end{align}
%
%
Now we have a closer look at the remaining part $X_2^{(n+1)}$.
First, by construction, we observe $\SPAN{ \widetilde{F}_{\mu+1}, \dots ,\widetilde{F}_M } \cap \JMDYN = \{ 0 \}$. This implies that the $\kappa_m(\widetilde{F_i})$'s are linearly independent\footnote{For them to be linearly dependent we would need to find some constants $z_i \in \IC$, for $i=\mu+1,\dots,M$, such that $\sum z_i \kappa_m(\widetilde{F_i}) = 0$. Using the linearity of $\kappa_m$, this yields $\kappa_m(\sum z_i \widetilde{F_i}) =0$. This contradicts the fact that $\SPAN{ \widetilde{F}_{\mu+1}, \dots ,\widetilde{F}_M } \cap \JMDYN = \{ 0 \}$.} for $i=\mu+1,\dots,M$.
Furthermore, since $\kappa_m(X_1^{(n+1)}) = 0$,  $\kappa_m(f^{(n+1)}) = 0$ and due to the linearity of $\kappa_m$, also $\kappa_m(X_2^{(n+1)}) = 0$. Using the linear independence we conclude that $\kappa_m( \widetilde{\F}_i^{(n)}) = 0$. Since $\widetilde{\F}_i^{(n)}$ is of degree $n$ and lies in the kernel of $\kappa_m$, we can apply the induction assumption and find
\begin{align}
\Big(0,0,\dots, \widetilde{\F}_i^{(n)} , 0 , 0 , \dots \Big) \in \IMZDYN \formspace.
\end{align}
Therefore, also
\begin{align}
\Big(0,0,\dots, \widetilde{F}_i &\otimes \widetilde{\F}_i^{(n)} , 0 , 0 , \dots \Big) \notag \\
&= \Big(0, \widetilde{F}_i , 0 , 0 , \dots \Big) \cdot \Big(0,0,\dots, \widetilde{\F}_i^{(n)} , 0 , 0 , \dots \Big) \in \IMDYN \formspace.
\end{align}
Now, bringing the two parts together we successfully have completed the induction:
\begin{align}
\Big(0,0,\dots, f^{(n+1)} , 0 , 0 , \dots \Big) = 	\Big(0,0,\dots, X_1^{(n+1)} + X_2^{(n+1)} , 0 , 0 , \dots \Big) \in \IMZDYN
\end{align}
and, as we have argued before, therefore $f \in \IMZDYN$, which completes the proof.
\end{proof}
With these results we can, analogously to the degree-one-part, construct a homeomorphism.
By the same argument as for the degree-one level, we find a unique map $\gls{Xim} : \BUmzdyn \to \BU\big( \Dzs \big)$ where $\Xi_m ([f]_m) = K_m(f)$, $f \in \BUOmega$. This is again best illustrated in form of a diagram, shown in Diagram \ref{dia:source-free-dynamical-homeo}.
%
\begin{table}
\begin{displaymath}
\xymatrix @R=20mm @C=30mm
{
	\BUOmega  \ar[r]^{K_m} \ar[dr]_{[\cdot]_m^{\text{dyn}}} 	& \BU\big(\Dzs\big) \ar@<-.5ex>[d]_{\Xi^{-1}_m}  		\\
	%
	&  \BUmzdyn  	       \ar@<-.5ex>[u]_{\Xi_m}
}
\end{displaymath}
\caption{Illustrating the construction of the homeomorphism $\Xi_m$ between the source-free dynamical field algebra and the field algebra of initial data.}
\label{dia:source-free-dynamical-homeo}
\end{table}
%
We formulate one important result of this thesis in form of the following lemma:
%
\begin{lemma}\label{lem:field-algebra-homeomorphism}
	Let $m>0$ and $j=0$.
	\begin{center}
The spaces $\BUmzdyn$ and $\BU\big( \Dzs \big)$ are homeomorphic.
	\end{center}
\end{lemma}
\begin{proof}
	1.) To show that $\Xi_m$ is continuous is trivial. As we have argued when we first introduced the map, $\kappa_m$ is continuous. Therefore $K_m$ is continuous and hence $\Xi_m$ is continuous (again, see \cite[Proposition 4.6]{treves}).\par
	2.) As we did on the test one-form level, we directly construct the inverse $\Xi_m^{-1}$ as follows:
	On the degree-one level we have constructed a continuous map $\vartheta_m : \Dzs \to \Omega^1_0 (M)$ to construct $\xi_m^{-1} = [\cdot]_m \comp \vartheta_m$.
	We lift the map $\vartheta_m$ to the map $\Theta_m : \BU\big( \Dzs \big) \to \BUOmega$ in the same fashion as we have lifted the map $\kappa_m$ to $K_m$. Then, by construction, $\Theta_m$ is continuous and $\Theta_m(K_m(f)) = f$.
	We can thus conclude $\Xi^{-1} = [\cdot]_m^\text{dyn} \comp \Theta_m$ which is continuous by construction.
	This completes the proof.
\end{proof}
%%%%
%%%%
%%%%
%%%%
%%%%
With this lemma, we have successfully incorporated the dynamics of the Proca field into our field algebra. We are left to also include the quantum nature of the fields, that is we have to divide out the relation that implements the CCR. In $\BUmzdyn$, we need to divide out the two-sided ideal $\IMCCR$ that is generated by elements $\big(-\i \Gm{F}{F'}, 0 , F \otimes F' - F' \otimes F , 0 , 0 , \dots\big)$. As we have already calculated in the proof of Lemma \ref{lem:propagator-non-degenerate}, we can write the action of $\Gm{\cdot}{\cdot}$ in form of initial data $(\varphi, \pi), (\varphi', \pi') \in \Dzs$ as follows: \vspace{-.6cm}
\begin{align}
\Gm{F}{F'} = \langle \pi , \varphi' \rangle_\Sigma - \langle \varphi, \pi'\rangle_\Sigma \formspace,
\end{align}
where $(\varphi, \pi), (\varphi', \pi') \in \Dzs$ are the initial data of $G_m F$, $G_m F'$ respectively.
Therefore, the ideal $\IMCCR$ maps under $\Xi_m$ to the two-sided ideal $\ICCR \subset \BU\big( \Dzs \big)$ that is generated by elements \vspace{-.3cm}
\begin{align}
\big(-\i ( \langle \pi , \varphi' \rangle_\Sigma - \langle \varphi, \pi'\rangle_\Sigma), 0 , (\varphi, \pi) \otimes (\varphi', \pi') - (\varphi', \pi') \otimes (\varphi, \pi) , 0 , 0 , \dots\big)\formspace.
\end{align}
With this, the following theorem follows easily.
\begin{theorem}\label{thm:field_algebra_homeomorphy_source_free}
	Let $m>0$ and $j=0$.
	\begin{center}
The field algebra $\BUmz$ is homeomorphic to ${\Quotientscale{\BU\big( \Dzs \big)}{\ICCR}}$.
	\end{center}
\end{theorem}
\begin{proof}
	We have already argued most of what is necessary for the proof:\\
	$\BUmz$ is obtained from $\BUmzdyn$ by dividing out the ideal $\IMCCR$. This ideal maps under $\Xi_m$ one-to-one to the ideal $\ICCR$ in $\BU\big( \Dzs\big)$. 	Intuitively, it is quite clear that if we have two homeomorphic algebras, where in each algebra we divide out an ideal and the two ideals are homeomorphic, we end up with two algebras that are homeomorphic.
	The construction of the homeomorphism formalizes that idea. Due the the homeomorphy of the two ideals $\IMCCR$ and $\ICCR$, the map $[\cdot]_\sim^\text{CCR} \comp \Xi_m : \BUmzdyn \to {\Quotientscale{\BU\big( \Dzs \big)}{\ICCR}}$ has kernel $\IMCCR$. This yields the unique existence of the continuous map $\gls{Lambdam} : \BUmz \to {\Quotientscale{\BU\big( \Dzs \big)}{\ICCR}}$. The construction of the inverse $\Lambda^{-1}$ follows analogously using the map $[\cdot]_m^\text{CCR} \comp \Xi_m^{-1} : \BUmzdyn \to {\Quotientscale{\BU\big( \Dzs \big)}{\ICCR}}$.\\ This completes the proof.
\end{proof}
To conclude the construction made so far, the situation is illustrated in Diagram \ref{dia:source-free-homeomorphism-final}.
%
\begin{table}[]
\begin{displaymath}
\xymatrix @R=20mm @C=30mm
{
	\BUOmega  \ar[r]^{K_m} \ar[dr]_{[\cdot]_m^{\text{dyn}}} 	& \BU\big(\Dzs \big) \ar[r]^{[\cdot]^\text{CCR}_{\sim}}\ar@[linkred][dr]^(.35){  \color{linkred}{\;\;\;[\cdot]^\text{CCR}_m \comp \, \Xi_m^{-1}}  } \ar@<-.5ex>[d]_{\Xi^{-1}_m}  		&       {{\Quotientscale{\BU\big( \Dzs\big) }{\ICCR}}} \ar@<-.5ex>[d]_{\Lambda_m^{-1}} \\
	%
	&  \BUmzdyn  		\ar[r]_{[\cdot]^\text{CCR}_{m}} \ar@[linkblue][ur]_(.35){  \color{linkblue}{\;\;\;[\cdot]^\text{CCR}_\sim \comp \, \Xi_m} } 	                                     \ar@<-.5ex>[u]_{\Xi_m}		& 		 \BUmz	\ar@<-.5ex>[u]_{\Lambda_m}\\
}
\end{displaymath}
\caption{Illustrating the construction of the homeomorphy of the source-free field algebra and the field algebra of initial data. Bi-directional arrows represent homeomorphisms.}
\label{dia:source-free-homeomorphism-final}
\end{table}
%
So, for the case with vanishing external sources, we have found a way to compare the field algebra at different masses with each other. This will be the starting point for the investigation of mass dependence for $j=0$ in section \ref{sec:mass-dependence-j-zero}. Before we turn to this, we first study the case where $m$ is still fixed but we allow for external currents $j \neq 0$.\par
Assume we have given a non vanishing external current $j$. It is clear that the previous construction will not generalize in a trivial fashion, since the ideal that implements the dynamics is generated by elements $\big(-\langle j, F \rangle_\M, (\delta d + m^2)F,0,0,\dots\big)$ and it is not obvious to find a map, similar to $K_m$, such that the dynamical ideal is the kernel of that map. Instead, we will show that actually the field algebra with source dependent dynamics is homeomorphic to the field algebra with vanishing source dynamics, $\BUmzdyn \cong \BUmjdyn$.
\begin{theorem}[Field algebra homeomorphy]\label{thm:field-algebra-homeomorphy}
	Let $m > 0$ and $j \in \Omega^1(\M)$.
	\begin{center}
	The field algebras $\BUmj$ and $\BUmz$ are homeomorphic.
	\end{center}
	In particular, this yields that also $\BUmj$ and ${\Quotientscale{\BU\big(\Dzs\big)}{\ICCR}}$ are homeomorphic.
\end{theorem}
\begin{proof}
	The proof works in two steps:\\
	1.) First we construct a non-trivial homeomorphism
	\begin{align}
	\gls{Gammamj} : \BUOmega &\to \BUOmega \formspace,
	\end{align}
	where $\varphi_{m,j}$ is a fixed solution to the classical source dependent Proca equation $(\delta d +m^2) \varphi_{m,j}= j$,
	by defining $\Gamma_{m,j,\varphi}$ as a *-algebra-homomorphism which is then uniquely determined by its action on homogeneous elements of degree zero and one in $\BUOmega$. That is, we define for any $c \in \IC$ and $F \in \Omega^1_0(\M)$
			\begin{align}
			\Gamma_{m,j,\varphi} :
			(c,0,0,\dots) &\mapsto (c,0,0,\dots) \\
			(0,F,0,0,\dots) &\mapsto (-\langle \varphi , F \rangle_\M , F , 0,0,\dots) \notag
			\end{align}
	which is extended to act on arbitrary elements of $\BUOmega$ by linearity and homomorphy with respect to multiplication.
	The inverse is obviously determined by
	\begin{align}
	\Gamma_{m,j,\varphi}^{-1} : \BUOmega &\to \BUOmega \\
	(c,0,0,\dots) &\mapsto (c,0,0,\dots) \notag\\
	(0,F,0,0,\dots) &\mapsto (+\langle \varphi , F \rangle_\M , F , 0,0,\dots) \formspace.\notag
	\end{align}
	Both $\Gamma_{m,j,\varphi}$ and $\Gamma_{m,j,\varphi}^{-1}$ are continuous:\\
	Trivially, the identity map $c \mapsto c$ and $(0,F,0,0,\dots) \mapsto (0,F,0,0,\dots)$ is continuous. Furthermore $(0,F,0,0,\dots) \mapsto (\pm \langle \varphi_{m,j}, F \rangle_\M , 0,0,\dots )$ is continuous since $\varphi_{m,j}$ is assumed fixed. Since the sum of continuous functions in continuous, $(0,F,0,0,\dots) \mapsto (\pm \langle \varphi_{m,j},F \rangle_\M, F, 0,0,\dots)$ is continuous. Hence, $\Gamma_{m,j,\varphi}$ defines a non-trivial homeomorphism of $\BUOmega$.\par
	2.) In this second step we will show that, with respect to $\Gamma_{m,j,\varphi}$, the ideals $\IMDYN$ and $\IMCCR$ are homeomorphic to $\IMJDYN$ and $\IMCCR$ respectively.
	It suffices to show that the generators of the source-free ideals map under $\Gamma_{m,j,\varphi}$ to the corresponding generators of the source dependent ideals and vice versa. Let $F\in \Omega^1_0(\M)$, then
	\begin{align}
	\Gamma_{m,j,\varphi}&\Big( \big( 0,(\delta d +m^2)F , 0 , 0 , \dots   \big)  \Big) \notag\\
	&= \big( 0,(\delta d +m^2)F , 0 , 0 , \dots   \big) - \langle \varphi_{m,j}, (\delta d + m^2)F \rangle_\M \cdot \mathbbm{1}_\BU \notag\\
	&= \big( 0,(\delta d +m^2)F , 0 , 0 , \dots   \big) - \langle (\delta d + m^2) \varphi_{m,j}, F \rangle_\M \cdot \mathbbm{1}_\BU \notag\\
	&= \big( - \langle j, F \rangle_\M ,(\delta d +m^2)F , 0 , 0 , \dots   \big)  \formspace.
	\end{align}
	So the generators for the dynamics transform in the desired way. For the commutation relations we first decompose:
	\begin{align}
	\big( -\i \Gm{F}{F'},0 , F \otimes F' - F' \otimes F   , 0 ,0, \dots   \big)  &=\big( -\i \Gm{F}{F'}, 0 ,0, \dots   \big) \\
	&\phantom{M}+ \big( 0,F,0,0,\dots  \big)\cdot \big( 0,F',0,0,\dots  \big)  \notag \\
	&\phantom{M}-\big( 0,F',0,0,\dots  \big)\cdot \big( 0,F,0,0,\dots  \big) \notag
	\end{align}
	and therefore obtain
	\begin{align}
	\Gamma_{m,j,\varphi}\Big( \big( -\i \Gm{F}{F'},&0 , F \otimes F' - F' \otimes F   , 0 ,0, \dots   \big)  \Big) \notag\\
	&=\big( -\i \Gm{F}{F'}, 0 ,0, \dots   \big)  \notag \\
	&\phantom{M}+ \big( - \langle \varphi_{m,j},F \rangle_\M,F,0,0,\dots  \big)\cdot \big( - \langle \varphi_{m,j},F' \rangle_\M,F',0,0,\dots  \big)  \notag \\
	&\phantom{M}-\big( - \langle \varphi_{m,j},F' \rangle_\M,F',0,0,\dots  \big)\cdot \big( - \langle \varphi_{m,j},F \rangle_\M,F,0,0,\dots  \big) \notag\\
	&= \big( -\i \Gm{F}{F'},0 , F \otimes F' - F' \otimes F   , 0 ,0, \dots   \big)   \formspace.
	\end{align}
	It is straightforward to check in a completely analogous fashion that the generators of the source-dependent ideal map under $\Gamma_{m,j,\varphi}^{-1}$ to the generators of the source-free ideal.
	In conclusion, we find that, with respect to $\Gamma_{m,j,\varphi}$, the ideals $\IMJ$ and $\IMZ$ are indeed homeomorphic. By the same argument as used in Theorem \ref{thm:field_algebra_homeomorphy_source_free}, that is if we divide out two ideals that are homeomorphic the resulting algebras are homeomorphic, we find that $\BUmj$ and $\BUmz$ are homeomorphic. The implication that then $\BUmj$ and ${\Quotientscale{\BU\big(\Dzs\big)}{\ICCR}}$ are homeomorphic follows trivially using Theorem \ref{thm:field_algebra_homeomorphy_source_free}.
\end{proof}
To conclude all of the construction made so far, the final algebraic structure is illustrated in Diagram \ref{dia:final_structure}. Here, $\gls{Psimj}$ is the homeomorphism constructed implicitly in the proof of Theorem \ref{thm:field-algebra-homeomorphy}.
\begin{table}[]
\begin{displaymath}
\xymatrix @R=20mm @C=30mm
{ 																																	& \BU{\left( \Dzs \right)} 		\ar[r]^{[\cdot]_{\sim}^\text{CCR}}		\ar@<.5ex>[d]^{\Xi^{-1}_{m}}				& {{\Quotientscale{\BU{\left( \Dzs \right)}}{\ICCR}}} \ar@<.5ex>[d]^{\Lambda^{-1}_m}\\
	%
	%
	\BUOmega \ar[r]^{[\cdot]_{m,0}^\text{dyn}}  	\ar@<.5ex>[d]^{\Gamma_{m,j,\varphi}} \ar[ur]^{K_m}	&  \BUmzdyn  \ar[r]^{[\cdot]_{m,0}^\text{CCR}} \ar@<.5ex>[u]^{\Xi_{m}}  \ar@<.5ex>[d] &   \BUmz \ar@<.5ex>[d]^{\Psi_{m,j,\varphi}} \ar@<.5ex>[u]^{\Lambda_m}  \\
	%
	%
	\BUOmega  \ar[r]_{[\cdot]_{m,j}^\text{dyn}}      \ar@<.5ex>[u]^{\Gamma^{-1}_{m,j,\varphi}}    & \BUmjdyn \ar[r]_{[\cdot]_{m,j}^\text{CCR}}	\ar@<.5ex>[u]& 	\BUmj 	\ar@<.5ex>[u]^{\Psi_{m,j,\varphi}^{-1}}     \\
}
\end{displaymath}
\caption{Overview of the final algebraic structure and connections between the field algebras. Bi-directional arrows represent homeomorphisms.}
\label{dia:final_structure}
\end{table}
%
Given an observable of the source free theory $\phi_{m,0}(F)$, we obtain
\begin{align}
\Psi_{m,j,\varphi}(\phi_{m,0}(F))
&= \big[ \Gamma_{m,j,\varphi}(0,F,0,0,\dots)\big]_{m,j} \notag\\
&= \big[ (-\langle \varphi_{m,j}, F \rangle_\M,F,0,0,\dots)\big]_{m,j} \notag\\
&=  -\langle \varphi_{m,j}, F \rangle_\M \cdot \mathbbm{1}_\BUmj + \phi_{m,j}(F) \formspace.
\end{align}
Hence we can easily check that, given an observable $\phi_{m,0}(F)$ that solves the source-free field equations $\phi_{m,0}\big((\delta d + m^2)F\big)=0$ , and fulfills the commutation relations, we obtain by
\begin{align}
\phi_{m,j}(F) = \langle \varphi_{m,j}, F \rangle_\M \cdot \mathbbm{1}_\BUmj + \Psi_{m,j,\varphi}\big(\phi_{m,0}(F)\big)
\end{align}
an observable that clearly solves the source-dependent field equations and fulfills the commutation relations:
\begin{align}
\phi_{m,j}((\delta d+m^2)F)
&=  \langle \varphi, (\delta d +m^2)F \rangle_\M \cdot \mathbbm{1}_\BUmj + \underbrace{\Psi_{m,j,\varphi}\big(\phi_{m,0}\big((\delta d+m^2)F\big)\big)}_{=0 }\notag\\
&= \langle j, F \rangle_\M \cdot \mathbbm{1}_\BUmj
\end{align}
and
\begin{align}
[\phi_{m,j}(F), \phi_{m,j}(F')]
&= \Psi_{m,j,\varphi} \big([\phi_{m,0}(F), \phi_{m,0}(F')] \big) \notag\\
&= \i \Gm{F}{F'} \cdot \mathbbm{1}_\BUmj \formspace.
\end{align}
This concludes our investigation of the field algebra for a fixed mass. With the results of this section, we are able to define a notion of continuity with respect to the mass of the theory and investigate the zero mass limit.
%
%%%
%%
%
%
%
%
%
\subsection{Mass dependence and the zero mass limit}\label{sec:mass_depenence_and_limit}
In this section we will present one of the main results of this thesis, that is, we will formulate a notion of continuity of the field theory with respect to a change of the mass and study the zero mass limit.
We will study \emph{existence} of the limit, split into two parts, one for the case of vanishing external sources in Section \ref{sec:mass-dependence-j-zero} and one for the general case with sources in Section \ref{sec:mass-dependence-j-general}. Then, we study the algebraic relations and the dynamics of the fields in the zero mass limit in Section \ref{sec:zero-mass-limit-quantum-algebra-relations}. At given points, we compare our results with the theory of the quantum Maxwell field in curved spacetimes as studied in \cite{Sanders, pfenning}.\par
In order to investigate the zero mass limit we need Assumption \ref{ass:propagator_continuity} to hold. As in the classical theory, this ensures the continuity of the propagators with respect to the mass. This assumption is also essential for the quantum case.
We observe that for every $m$ we obtain a different field algebra $\BUmj$, since both the dynamics and the commutation relations that we implement depend on the mass. The problem is first to find a notion of comparing the Proca fields at different masses with each other.
%
%
%%
As we have hinted in section \ref{sec:field-algebra-topology}, we could use the semi-norms
\begin{align}
q_{m, \alpha}( [f]_m ) = \inf\big\{ p_\alpha(g) : g \in [f]_m \big\}
\end{align}
to define a notion of continuity of the theory with respect to the mass $m$. We could define that a function $\eta: \IR_+ \to \bigcup_m \BUmj$, such that $\eta(m_0) \in \BU_{m_0,j}$, is called continuous if and only if the map $m \mapsto q_{\alpha,m}\big(\eta(m)\big)$ is continuous for all $\alpha$ with respect to the standard topology in $\IR$. While this definition seems appropriate, we could not, despite much time and effort spent, prove that for a fixed $F \in \BUOmega$ the map $m \mapsto \phi_m([F]_m)$ is continuous in the above sense, not even in the simpler case $j=0$. But this is certainly a map that we would like to be continuous: If we fix a test function $F$ that we smear the field $\phi_m$ with and vary the mass continuously, we certainly want the observable $\phi_m([F]_m)$ to vary continuously as well. Even at the more simple one-particle level of $\Omega^1_0(\M)$, where we can formulate an equivalent notion of continuity for compactly supported one-forms that incorporate the dynamics in the source free case, that is, one-forms that are of the form $F= (\delta d +m^2)F'$ for some $F' \in \Omega^1_0(\M)$, we were not able to prove that the equivalence classes $[F]_m$ vary continuously with respect to the continuity formulated using semi-norms on the quotient space. It might be a problem worth tackling with more sophisticated mathematical methods. Another approach of formulating a notion of continuity is to use the C*-Weyl algebra rather then the BU-algebra. Being a normed algebra, the Weyl algebra seems suited for the investigation of the zero mass limit at first but, as it turns out, this is not the case as discussed in Appendix \ref{app:weyl-algebra}.
We will now turn to a formulation of continuity that we were able to show to have the desired behavior for fixed test functions $F$.
%
%
%
%
%
%
%
\subsubsection{Existence of the limit in the current-free case}\label{sec:mass-dependence-j-zero}
Of course we would like to make use of the constructed homeomorphisms to formulate a notion of continuity. And since we have found for every mass $m$ that $\BUmz$ is homeomorphic to $\Quotientscale{\BU\big(\Dzs\big)}{\ICCR}$, we can map a family of elements $\left\{ f_m \right\}_m$, such that for every $m$ it holds $f_m \in \BUmz$, to a family in the algebra ${\Quotientscale{\BU\big(\Dzs\big)}{\ICCR}}$. Here, we have a natural sense of continuity given by the topology.
We state this important result in form of the following definition:
\begin{definition}[Continuity of a family of observables with respect to the mass]\label{def:field_continuity}
	Let $K_m: \BUOmega \to \BU{\left( \Dzs \right)}$ and $\Lambda_m : \BUmz \to {\Quotientscale{\BU{\left( \Dzs \right)}}{\ICCR}}$ be as defined in section \ref{sec:field-algebra-topology}. We call a function
	\begin{align}
	\eta : \IR_+ &\to \bigcup_m \BUmz\\
	m &\mapsto \eta_m  \in \BUmz \notag
	\end{align}
	continuous if the map
	\begin{align}
	\widetilde{\eta} : \IR_+ &\to {\Quotientscale{\BU\big( \Dzs \big) }{\ICCR}} \\
	m &\mapsto \Lambda_m (\eta_m)\notag
	\end{align}
	is continuous.\\
	Equivalently, one can identify $\eta_m = [f_m]_{m,0}$ for some family $\left\{ f_m \right\}_m \subset \BUOmega$ and define the map $\eta$ to be continuous if the map
	\begin{align}
	\hat{\eta} : \IR_+ &\to {\Quotientscale{\BU\big( \Dzs \big) }{\ICCR}} \\
	m &\mapsto [K_m(f_m)]_\sim^\text{CCR}\notag
	\end{align}
	 is continuous. The latter will actually be the more practical definition.
\end{definition}
As an example, and also to check that this notion of continuity has the desired properties, we check that the map $\eta: m \mapsto \phi_m(F)$ is continuous for a fixed $F \in \Omega^1_0(\M)$:\\
According to the above definition, $\eta$ is continuous if
\begin{align}
\widetilde{\eta}: m \mapsto [(0, \kappa_m (F), 0, 0, \dots)]^\text{CCR}_\sim
\end{align}
is continuous in ${\Quotientscale{\BU\big(\Dzs\big)}{\ICCR}}$. Recall that
\begin{align}
\kappa_m : \Omega^1_0(\M) &\to \Dzs \\
F&\mapsto (\rhoz G_m F , \rhod G_m F)  \notag
\end{align} and
\begin{align}
	G_m = \left( \frac{d \delta}{m^2} +1\right) E_m \formspace.
\end{align}
By Assumption \ref{ass:propagator_continuity} and using that the operators $\rho_{(\cdot)}$ are continuous, we find that $m \mapsto \kappa_m F$ is continuous in $m >0$ for any fixed test one-form $F$. By construction, the map $[\cdot]^\text{CCR}_\sim$ is continuous and does not depend on $m$. Therefore, $\widetilde{\eta}$ is continuous and thus $\eta$ is continuous in the above sense.
We indeed find the desired property of continuously varying quantum fields with respect to the mass! Additionally, we find for a fixed $F \in \Omega^1_0(\M)$ and the corresponding field $\phi_m(F)$ that the notion of continuity is independent of the choice of the Cauchy surface $\Sigma$, since $\kappa_m(F)$ is continuous in $m$ for any Cauchy surface.\par
The notion of continuity defined in Definition \ref{def:field_continuity} therefore seems appropriate, and, at least for a field $\phi_m(F)$, is independent of the choice of the Cauchy surface. We therefore will from now on identify a field $\phi_m$ with its initial data formulation $\phi_m(F) = \big[\big (0, \kappa_m(F),0,0,\dots\big)\big]_\sim^\text{CCR}$ and also view the propagator $\Gm{\cdot}{\cdot}$ as a map on initial data as explained in Lemma \ref{lem:propagator-non-degenerate}. But is the notion in general independent of the choice of the Cauchy surface $\Sigma$? To some extent, we can answer this positively:
\begin{lemma}\label{lem:continuity-independence-source-free}
	Let $\Sigma$, $\Sigma'$ be Cauchy surfaces, $a,b \in \IR^+$ arbitrary. Let $\left\{ f_m\right\}_m$ be a family in $\BUOmega$. Then
	\begin{align*}
		\eta^{(\Sigma)} : [ a , b] &\to {\Quotientscale{\BU(\Dzs  )}{\ICCRS}}\hspace{1mm}, \quad m \mapsto \big[ K_m^{(\Sigma)}(f_m) \big]_\sim^{\text{CCR},\Sigma}
	\end{align*}
		is continuous if and only if
		\begin{align*}
		\eta^{(\Sigma')} : [ a , b] &\to {\Quotientscale{\BU(\Dzsp)}{\ICCRSP}}, \quad m \mapsto \big[ K_m^{(\Sigma')}(f_m) \big]_\sim^{\text{CCR},\Sigma'}
	\end{align*}
	is continuous.
	That means, the notion of continuity defined in Definition \ref{def:field_continuity} is independent of the choice of the Cauchy surface $\Sigma$ for $m$ being an element of a compact set.
\end{lemma}
\begin{proof}
	Let $\Sigma, \Sigma'$ be Cauchy surfaces and $\left\{ f_m \right\}_m$ be a family in $\BUOmega$.\\
In this proof we will make the dependence on the Cauchy surfaces $\Sigma, \Sigma'$ of the maps introduced so far explicit, that is, we will write $\kappa_m^{(\Sigma)}, \kappa_m^{(\Sigma')}$ for the map that maps to initial data on $\Sigma$ and $\Sigma'$ respectively.\par
1.) The $\Rightarrow$-direction: First, we want to show that if the map $m \mapsto \big[  K_m^{(\Sigma)}(f_m) \big]_\sim^{\text{CCR},\Sigma}$ is continuous, then also $m \mapsto \big[  K_m^{(\Sigma')}(f_m) \big]_\sim^{\text{CCR},\Sigma'}$ is continuous. Using the homeomorphisms defined in Section \ref{sec:field-algebra-topology}, recall for example the diagrammatic overview in Diagram \ref{dia:final_structure}, we identify
\begin{align}
	m \mapsto \big[  K_m^{(\Sigma')}(f_m) \big]_\sim^{\text{CCR},\Sigma'} = \Big( \Lambda^{(\Sigma')}_m \comp \left( \Lambda^{(\Sigma)}_m \right)^{-1} \Big) \Big(  \big[ K_m^{(\Sigma)}  (f_m)\big]_\sim^{\text{CCR}, \Sigma}\Big) \formspace.
\end{align}
We split this up into the map
\begin{align}
(m,m') \mapsto \Big( \Lambda^{(\Sigma')}_m \comp \left( \Lambda^{(\Sigma)}_m \right)^{-1} \Big) \Big(  \big[ K_{m'}^{(\Sigma)}  (f_{m'})\big]_\sim^{\text{CCR}, \Sigma}\Big)
\end{align}
and show that it is continuous both in $m$ and $m'$ using the assumption that $m \mapsto  \big[ K_m^{(\Sigma)}(f_m) \big]_\sim^{\text{CCR},\Sigma} $ is continuous. We will then use the Banach-Steinhaus theorem to show joint continuity.\par
a) Let $m$ be fixed.\\
By assumption, $m' \mapsto  \big[ K_{m'}^{(\Sigma)}(f_{m'}) \big]_\sim^{\text{CCR},\Sigma} $ is continuous. Furthermore, both $\Lambda^{(\Sigma')}_{m}$ and $\left( \Lambda^{(\Sigma)}_{m} \right)^{-1}$ are continuous for a fixed $m$ since they were shown to be homeomorphisms, see Lemma \ref{lem:field-algebra-homeomorphism}.\\
Consequently, the map
\begin{align}
	m' \mapsto  \Big( \Lambda^{(\Sigma')}_m \comp \left( \Lambda^{(\Sigma)}_m \right)^{-1} \Big) \Big(  \big[ K_{m'}^{(\Sigma)}  (f_{m'})\big]_\sim^{\text{CCR}, \Sigma}\Big)
\end{align}
is continuous for fixed $m$.\par
%
%
%
b) Now assume $m'$ to be fixed. \\
We identify $\big[ K_{m'}^{(\Sigma)}(f_{m'}) \big]_\sim^{\text{CCR},\Sigma}= \big[ \psi \big]_\sim^{\text{CCR},\Sigma}$ for some $\psi \in \BU{\left( \Dzs \right)}$. We will let that initial data propagate with mass $m$ to the Cauchy surface $\Sigma'$ by
\begin{align}
	[\psi]_\sim^{\text{CCR},\Sigma} \mapsto \Big( \Lambda^{(\Sigma')}_{m} \comp \big( \Lambda^{(\Sigma)}_{m} \big)^{-1} \Big) \big([\psi]_\sim^{\text{CCR},\Sigma}\big) \formspace.
	\end{align}
Recall that we have explicitly constructed the homeomorphisms and their inverses in Section \ref{sec:field-algebra-topology} using the map $\vartheta_m^{(\Sigma)} : \Dzs  \to \Omega^1_0(\M) $ which mapped initial data to a compactly supported test one-form $F$ such that $G_m F$ is a solution to Proca's equation with these data. This map was shown to be continuous. Furthermore, $\vartheta_m^{(\Sigma)}$ is continuous in $m$:\\
For fixed initial data $\psi= (\varphi, \pi) \in \Dzs$, a solution $A_{m} \in \Omega^1(\M)$ to $(\delta d + {m}^2)A_{m} =0$ with the given data was uniquely specified by, see Theorem \ref{thm:solution_proca_unconstrained}:
\begin{align}
	\langle A_{m}, F \rangle_\M =  \langle \varphi, \rhod G_{m} F \rangle_\Sigma - \langle \pi , \rhoz G_{m} F \rangle_\Sigma
\end{align}
which by Assumption \ref{ass:propagator_continuity} depends continuously on $m$. Furthermore, we defined $\vartheta_m^{(\Sigma)} (\varphi,\pi) = - (\delta d + {m}^2) \chi A_{m}$ which is hence continuous in $m$. Thus, the map $\vartheta_m^{(\Sigma)}$ is continuous in $m$. We lifted $\vartheta_m^{(\Sigma)}$ to the map $\Theta_m^{(\Sigma)} : \BU\big( \Dzs\big) \to \BUOmega$ which is then by construction also continuous in $m$ and we find
\begin{align}
\Big( \Lambda^{(\Sigma')}_{m} \comp \big( \Lambda^{(\Sigma)}_{m} \big)^{-1} \Big) \big( [\psi]_\sim^{\text{CCR},\Sigma} \big) = \big[ \big( K_{m}^{(\Sigma')} \comp  \Theta_{m}^{(\Sigma)}\big) \big( \psi \big) \big]_\sim^{\text{CCR},\Sigma'}
\end{align}
is continuous in $m$.\par
Therefore the map $	(m,m') \mapsto \Big( \Lambda^{(\Sigma')}_m \comp \left( \Lambda^{(\Sigma)}_m \right)^{-1} \Big) \Big(  \big[ K_{m'}^{(\Sigma)}  (f_{m'})\big]_\sim^{\text{CCR}, \Sigma}\Big)$ is \emph{separately} continuous. In order to obtain the wanted result, we need to show \emph{joint} continuity. \par
We have shown that $\left\{T_{m} \right\}_{m}$, where
\begin{align}
T_{m} : {\Quotientscale{\BU(\Dzs)}{\ICCRS}} & \to {\Quotientscale{\BU(\Dzsp)}{\ICCRSP}}\\
T_{m} &=  \Lambda^{(\Sigma')}_{m} \comp \big( \Lambda^{(\Sigma)}_{m} \big)^{-1} \notag
\end{align}
 is a family of continuous linear mappings, with continuous linear inverse, which is point-wise continuous in $m$.
 Therefore, for any $a,b \in \IR^+$ the family $\left\{T_{m} \right\}_{m \in [a,b]}$ is \emph{point-wise bounded}, that is for any $\psi \in \BU(\Dzs)$ the set $\left\{ T_m([\psi]_\sim^{\text{CCR},\Sigma}) : m \in [a,b] \right\}$ is bounded.\par
Since $\BU(\Dzs)$ is barreled (see Lemma \ref{lem:BU-algebra-barreled}) and the ideal $\ICCR$ was shown to be a closed subspace in Section \ref{sec:field-algebra-topology}, the quotient ${\Quotientscale{\BU(\Dzs)}{\ICCRS}}$ is barreled (c.f. \cite[Proposition 33.1]{treves}). Using the Banach-Steinhaus theorem (see for example \cite[Theorem 33.1]{treves}), we find that $\left\{T_{m} \right\}_{m \in [a,b]}$ is equicontinuous, that is, for all $m \in [a,b]$ and all open $W \subset {\Quotientscale{\BU(\Dzsp)}{\ICCRSP}}$, there is a open $U \subset {\Quotientscale{\BU(\Dzs)}{\ICCRS}}$ such that $T_m (U) \subset W$.\\ From this, joint continuity follows:
Since we have shown that $m \mapsto \left[ K_m^{(\Sigma)} (f_m)\right] \eqqcolon \tau(m)$ is continuous, we find a open $V \subset [a,b]$ such that $\tau (V) \subset U$. Ergo, we find that for all open $W \subset {\Quotientscale{\BU(\Dzs)}{\ICCRS}}$, there is a open $V \subset [a,b]$ such that $(T \comp \tau)(V) \subset W$. Therefore, the map $m \mapsto (T \comp \tau)(m) = \Big( \Lambda^{(\Sigma')}_m \comp \left( \Lambda^{(\Sigma)}_m \right)^{-1} \Big) \Big(  \big[ K_m^{(\Sigma)}  (f_m)\big]_\sim^{\text{CCR}, \Sigma}\Big)$ is continuous for $m \in [a,b]$. \par
2.) The $\Leftarrow$-direction: Using that $\Lambda^{(\Sigma)}_m$ and $\Lambda^{(\Sigma)}_m$ are homeomorphisms, the map $T_m$ introduced above possesses a continuous linear inverse which depends continuously on $m$, hence, the above proof works analogously in the other direction, interchanging $\Sigma$ and $\Sigma'$.
\end{proof}
%
%
%
%
%
With this notion of continuity at our disposal, we are ready to investigate one of the main questions of this thesis: Does the zero mass limit of the quantum Proca field theory in curved spacetimes exist for the case $j=0$? To phrase this more specifically:
\begin{center}\textit{
		Let $j=0$. For which $F \in \Omega^1_0(\M)$, if any, does the limit\\[2mm] $\lim\limits_{m\to 0^+} \big( \phi_{m,0}(F) \big)$\\[2mm] exist with the notion of continuity defined in Definition \ref{def:field_continuity}?}
\end{center}
Using the above notion of continuity, the question of interest is if the limit
\begin{align}
	\lim\limits_{m \to 0^+} \phi_{m,0}(F) = \lim\limits_{m \to 0^+} \big[\big( 0 , \kappa_m(F) , 0 , 0, \dots\big)\big]^\text{CCR}_\sim
	\end{align}
exists. We recall the definitions
\begin{align}
\kappa_m(F) = (\rhoz G_m F, \rhod G_m F)
\end{align} and
\begin{align}
	G_m F = \left( \frac{d \delta}{m^2} + 1\right) E_m F \formspace.
\end{align}
In order to precisely answer this question, we find that the existence of the desired limit is equivalent to the existence of the corresponding classical limit as stated in the following lemma:
\begin{lemma}\label{lem:limit_existence_quantum_equivalence}
	Let $j=0$, $F \in \Omega^1_0(\M)$ be fixed and assume Assumption \ref{ass:propagator_continuity} holds. The following statements are equivalent:
	\begin{enumerate}
		\item {The limit $\lim\limits_{m \to 0^+} \phi_{m,0}(F) = \lim\limits_{m \to 0^+} \big[\big( 0 , \kappa_m(F) , 0 , 0, \dots\big)\big]^\text{CCR}_\sim $ exists. \\}
        \item {The limit $\lim\limits_{m \to 0^+} \big( 0 , \kappa_m(F) , 0 , 0, \dots\big)$ exists. \\}
		\item {The limit $\lim\limits_{m \to 0^+} G_m F $ exists.  \\}
		\item {The limit $\lim\limits_{m \to 0^+} \frac{1}{m^2}d \delta E_m F $ exists. }
	\end{enumerate}
\end{lemma}
\begin{proof}
	The nontrivial part is to show the equivalence of the first two statements:\par
	1.) a) (ii) $\implies$ (i) is obvious, since $[\cdot]_\sim^\text{CCR}$ is continuous and does not depend on the mass. So if $\big( 0 , \kappa_m(F) , 0 , 0, \dots\big)$ is continuous, so is $\big[\big( 0 , \kappa_m(F) , 0 , 0, \dots\big)\big]^\text{CCR}_\sim$. This yields the existence of the corresponding limit.\par
	b) (i) $\implies$ (ii) is highly non-trivial and most of the work needed to show this is put in the appendix in form of Lemma \ref{lem:symmetrization-of-fields}. Assume that $\big[\big( 0 , \kappa_m(F) , 0 , 0, \dots\big)\big]^\text{CCR}_\sim$ specifies a continuous family of algebra elements and that the corresponding zero mass limit exists. This implies that there is a family $\left\{\tilde{g}_m\right\}_m \subset \ICCR$ such that
	\begin{align}
     \big( 0 , \kappa_m(F) , 0 , 0, \dots\big) + \tilde{g}_m
	\end{align}
	specifies a continuous family. Note that this does not imply that neither $\tilde{g}_m$ nor $\big( 0 , \kappa_m(F) , 0 , 0, \dots\big) $ specify continuous families by themselves. In this context, continuity is as always assumed to be continuity in $m$. By Lemma \ref{lem:symmetrization-of-fields}, we can decompose this continuous family into a sum of a continuous family of symmetric elements and a continuous family of elements in the ideal $\ICCR$, that is,
	\begin{align}
     \big( 0 , \kappa_m(F) , 0 , 0, \dots\big) + \tilde{g}_m = f_{m,\text{sym}} + g_m
	\end{align}
	where $g_m \in \ICCR$ and both $g_m$ and  $f_{m,\text{sym}}$ specify continuous families of algebra elements. An element in the BU-algebra of initial data is called symmetric if it is symmetric in every degree. Therefore,
	\begin{align}
\big( 0 , \kappa_m(F) , 0 , 0, \dots\big) - f_{m,\text{sym}} =  ( g_m -\tilde{g}_m)
	\end{align}
	specifies a symmetric element, since the sum, respectively the difference, of two symmetric elements is again symmetric. We conclude that $(g_m - \tilde{g}_m) \in \ICCR$ is symmetric.\\
	By definition, we can write an element in the ideal $\ICCR$ as a finite sum
	\begin{align}
		g_m - \tilde{g_m} = \sum_{i=1}^{k} h_i \cdot \big( -\i \mathcal{G}_m(\psi_i, \psi'_i) , 0 , \psi_i \otimes \psi'_i - \psi'_i \otimes \psi_i , 0 , 0 ,\dots  \big) \cdot \tilde{h}_i \formspace
	\end{align}
	for some $k \in \IN$, $h_i, \tilde{h}_i \in \BU(\Dzs)$ and $\psi_i, \psi'_i \in \Dzs$. Writing $\psi_i = (\varphi_i , \pi_i)$ we have used for shorthand notation $\Gm{\psi_i}{ \psi'_i} = \langle \pi_i , \varphi'_i \rangle_\Sigma - \langle \varphi_i , \pi'_i \rangle_\Sigma$. Furthermore, we have dropped the mass dependence of the summands for clarity, as we are not using it in the following. Since $g_m - \tilde{g}_m$ is symmetric, in particular its highest degree $(g_m - \tilde{g}_m)^{(N)}$, for some $N \in \IN$, is symmetric. We can write the highest degree explicitly, using the above representation,
	\begin{align}
(g_m - \tilde{g}_m)^{(N)} = \sum_{i=1}^{k} h_i^{(N_i)}\, \phi_i^{(2)}\, \tilde{h}_i^{(\tilde{N}_i)} \formspace,
	\end{align}
	such that $N_i + \tilde{N}_i = N-2$ for every $i = 1,2,\dots,k$. Additionally, we have introduced $\phi_i^{(2)} =  \psi_i \otimes \psi'_i - \psi'_i \otimes \psi_i $ as yet another shorthand notation. As usual, $h_i^{(N_i)} \in (\Dzs)^{\otimes N_i}$ denotes the degree-$N_i$ part of $h_i \in \BU(\Dzs)$. By definition, the $\phi^{(2)}_i$'s are fully anti-symmetric, that is, $\phi^{(2)}_i(p,q) = - \phi^{(2)}_i(q,p)$ for any $p,q \in \Sigma$. Since $(g_m - \tilde{g}_m)^{(N)}$ is symmetric, it is invariant under arbitrary permutations of its arguments, that is, for any permutation $\sigma$ of $\{ 1,2, \dots , N\}$ it holds
	\begin{align}
		(g_m - \tilde{g}_m)^{(N)}(p_1,p_2,\dots,p_N) = (g_m - \tilde{g}_m)^{(N)}(p_{\sigma(1)},p_{\sigma(2)},\dots,p_{\sigma(N)}) \formspace.
	\end{align}
	Introducing the permutation operator $P_\sigma : (\Dzs)^{\otimes N} \to (\Dzs)^{\otimes N}$ by defining $(P_\sigma f^{(N)})(p_1,p_2,\dots,p_N) = f^{(N)}(p_{\sigma(1)},p_{\sigma(2)},\dots,p_{\sigma(N)}) $ we conclude in a short hand notation
	\begin{align}
(g_m - \tilde{g}_m)^{(N)}
&= \frac{1}{N!}\sum\limits_\sigma P_\sigma (g_m - \tilde{g}_m)^{(N)} \notag\\
&= \frac{1}{N!}\sum\limits_\sigma P_\sigma  \sum_{i=1}^{k} h_i^{(N_i)}\, \phi_i^{(2)}\, \tilde{h}_i^{(\tilde{N}_i)}\notag \\
&= \frac{1}{N!}  \sum_{i=1}^{k}  \sum\limits_\sigma P_\sigma \big(  h_i^{(N_i)}\, \phi_i^{(2)}\, \tilde{h}_i^{(\tilde{N}_i)} \big) \formspace.
	\end{align}
	In the last step we used that $P_\sigma$ is by construction linear, and that, since both sums are finite, we can rearrange the summands in the desired way. We find for any (fixed) $i=1,2,\dots,k$
	\begin{align}
\sum\limits_\sigma P_\sigma \big(  h_i^{(N_i)}\, \phi_i^{(2)}\, \tilde{h}_i^{(\tilde{N}_i)} \big) = 0
	\end{align}
	since to every permutation $\sigma$ there exist another permutation that differs from $\sigma$ by a swap of the $N_{i+1}$ and $N_{i+2}$ index. Because $\phi^{(2)}_i$ is fully anti-symmetric, the corresponding terms will cancel. Hence, all the summands in the sum over all permutations will cancel pair-wise, using the anti-symmetry of $\phi^{(2)}$. We can thus conclude
	\begin{align}
(g_m - \tilde{g}_m)^{(N)} = 0 \formspace,
	\end{align}
	which is a contradiction, because we have assumed the degree $N$ to be the highest degree of $(g_m - \tilde{g}_m)$. Therefore, it must hold $(g_m - \tilde{g}_m) = 0$. Concluding, we have shown that the only fully symmetric element in the ideal $\ICCR$ is the zero element. From this we can straightforwardly conclude
	\begin{align}
		\tilde{g}_m &= g_m &\text{specifies continuous family,} \\
		\implies  \big( 0, \kappa_m(F),0,0,\dots\big) &= f_{m,\text{sym}} &\text{specifies continuous family,}
	\end{align}
	which completes the proof of 1 b).\par
	%
	%
	%%
	%
	%
	%
	2.) The equivalence of (ii) and (iii) was mostly shown in Lemma \ref{lem:limit_existence_classical_equivalence}.\\
	%
	a) (iii) $\implies$ (ii):\\
	Assume $\lim\limits_{m \to 0^+} G_m F $ exists. Then, the limit $\lim\limits_{m \to 0^+} \kappa_m(F) = \lim\limits_{m \to 0^+} (\rhod G_m F, \rhoz G_m F)$ exists, since convergence in the direct sum topology on $\Omega^1_0(\Sigma) \oplus \Omega^1_0(\Sigma)$ means convergence in every component and the operators $\rho_{(\cdot)}$ are continuous. Then, since convergence in the BU-algebra means convergence in every degree, clearly, the limit $\lim\limits_{m \to 0^+} \big( 0 , \kappa_m(F),0,0,\dots \big)$ exists.\par
	b) (ii) $\implies$ (iii):\\ Assume that $\lim\limits_{m \to 0^+} \big( 0 , \kappa_m(F),0,0,\dots \big)$ exists.
	Since convergence in the BU-algebra means convergence in every degree, we conclude that $\lim\limits_{m \to 0^+} \kappa_m(F)$ exists. Recall, $\kappa_m(F) = (\rhoz G_m F , \rhod G_m F) \in \Dzs$, and also here convergence means convergence in every component, hence $\lim\limits_{m \to 0^+}  \rhoz G_m F $ and $\lim\limits_{m \to 0^+}  \rhod G_m F$ exist. According to Lemma \ref{lem:limit_existence_classical_equivalence} this yields that $\lim\limits_{m \to 0^+} G_m F $ exists.\par
	%
%
	3.) The equivalence of (iii) and (iv) was already proven in the classical chapter, see Lemma \ref{lem:limit_existence_classical_equivalence}.
	\end{proof}
With this lemma, we have found quite a remarkable result: The existence of the zero mass limit of the quantum Proca field theory in curved spacetimes is purely determined by its classical properties!\\
With this result, we can now quite easily find an answer to our original question, that is, explicitly find test functions $F \in \Omega^1_0(\M)$ such that the limit $\lim\limits_{m \to 0^+} \phi_{m,0}(F)$, or equivalently, $\lim\limits_{m \to 0^+} \frac{1}{m^2}d \delta E_m F $ exists. We have already investigated this in the classical section, therefore, we directly find the following lemma which we will later tighten to the main result as stated in Theorem \ref{thm:limit_existence_sourcefree}.
\begin{lemma}\label{lem:mass-zero-limit-existence_weak}
Let $F\in \Omega^1_0(\M)$ and $j=0$. Then,
\begin{center}
the limit $\lim\limits_{m \to 0^+}  \phi_{m,0}(F)$ exists if and only if $F = F' + F''$,\\
\end{center}
where $F', F'' \in \Omega^1_0(\M)$ such that $dF' = 0 = \delta F''$.
\end{lemma}
\begin{proof}
	According to Lemma \ref{lem:limit_existence_quantum_equivalence}, the desired limit $\lim\limits_{m \to 0^+}  \phi_{m,0}(F)$ exists if and only if $\lim\limits_{m \to 0^+} \frac{1}{m^2}d \delta E_m F $ exists. Using Lemma \ref{lem:limit_existence_classical_equivalence} together with Lemma \ref{lem:mass-zero-limit-existence-classical_weak} we find that this is equivalent to $F$ being a sum of a closed and a co-closed test one-form.
\end{proof}
From a formal point of view we have completely classified those test one-forms for which the zero mass limit exists. But it turns out that, just like in the classical case, we can tighten the result even more, by observing that closed test one-forms $F \in \Omega^1_0(\M)$, such that $dF=0$, do not contribute to the observables $\phi_{m,0}(F)$ for the source free case. We therefore may restrict the class of test-functions that we smear the fields $\phi_{m,0}$ with to the test one-forms modulo closed test one-forms, analogously to the classical case. This yields the final result of this section:
%
%
%%
\begin{theorem}[Existence of the zero mass limit in the source free case]\label{thm:limit_existence_sourcefree}
Let $j=0$ and $F, F' \in \Omega^1_0(\M)$ such that $[F] = [F']$, that is, there is a $\chi \in \Omega^1_{0,d}(\M)$ such that $F = F' + \chi$.  Then,
	\begin{align*}
	\kappa_m(F) &= \kappa_m (F') \eqqcolon \kappa_m ([F]) \quad\text{and hence}\\
	\phi_{m,0}(F) &= \phi_{m,0}(F') \eqqcolon \phi_{m,0}([F]) \formspace,
	\end{align*}
and
\begin{center}
	the limit $\lim\limits_{m \to 0^+}  \phi_{m,0}([F]) \coloneqq \lim\limits_{m \to 0^+} \big[\big( 0 , \kappa_m([F]) , 0 , 0, \dots\big)\big]^\text{CCR}_\sim $ exists\\[2mm]  if and only if there exists a representative $\tilde{F}$ of $[F]$ with ${\delta{\tilde{F}}} = 0 $.
\end{center}
Furthermore, the existence of the limit is independent of the choice of the Cauchy surface $\Sigma$.
\end{theorem}
\begin{proof}
	Recall from Theorem \ref{thm:limit_existence_sourcefree_classical} that for closed one-forms $F \in \Omega^1_{0,d}(M)$ it holds $G_m F = 0$, which directly yields $\kappa_m(F) = 0$ and hence $\phi_{m,0}(F) = 0$. Ergo, due to the linearity of the fields, two test one-forms that differ by a closed test one-form give rise to the same physical observable.
	Therefore, we can, without losing any observables, divide out the test one-forms that are closed. \\
	By Lemma \ref{lem:mass-zero-limit-existence_weak}  we know that the limit exists if and only if $F$ is a sum of a closed and a co-closed test one-form.
	Hence, the proof of the existence of the limit $\lim\limits_{m \to 0^+}  \phi_{m,0}([F])$ follows in complete analogy to the proof of Theorem \ref{thm:limit_existence_sourcefree_classical}, that is, basically showing that 	\begin{align}
	\frac{\Omega^1_{0,d}(\M) + \Omega^1_{0,\delta}(\M)}{\Omega^1_{0,d}(\M)} \cong \Omega^1_{0,\delta}(\M) \formspace.
	\end{align}
For the second part, let $F \in \Omega^1_{0,\delta}(\M)$ be a co-closed test one-form. Then it is clear the corresponding limit $\lim\limits_{m \to 0^+} \big[\big( 0 , \kappa_m(F) , 0 , 0, \dots\big)\big]^\text{CCR}_\sim $ exists regardless of the choice of the Cauchy surface, since $G_m F = E_m F$ and the $\rho^{(\Sigma)}_{(\cdot)}$ are continuous for any Cauchy surface $\Sigma$. Furthermore, as already argued, for a fixed $F \in \Omega^1_0(\M)$, continuity of the map $m \mapsto \phi_m(F) = \big[\big( 0 , \kappa_m(F) , 0 , 0, \dots\big)\big]^\text{CCR}_\sim$ is independent of the choice of the Cauchy surface since for two Cauchy surfaces $\Sigma, \Sigma'$ both $m \mapsto \kappa_m^{(\Sigma)}(F)$ and $m \mapsto \kappa_m^{(\Sigma')}(F)$ are continuous. Therefore, the statements in Lemma \ref{lem:limit_existence_quantum_equivalence} are independent of the choice of the Cauchy surface $\Sigma$.
\end{proof}
%
%
We find that it is sufficient as well as necessary for the mass zero limit to exist in the source free case to restrict to co-closed test one-forms. And just as in the classical case, as discussed in Section \ref{sec:zero-mass-limit-existence-classical-vanishing-source}, this implements the gauge equivalence of the Maxwell theory at the classical level. But also in the quantization of the vector potential of the Maxwell theory, restricting to co-closed test one-forms $F$ is a way to implement the gauge equivalence is the theory as presented in \cite{Sanders} or \cite{fewster_pfenning_quantum_weak}. Hence also in the quantum case, the limit exists only if we implement the gauge equivalence! Before we investigate the algebraic structures of the zero mass limit fields, we discuss the existence of the limit in the general case, including external sources.
%
%
%
%
%
%
%
%
%
%
%
%
%
%
%
%
%
%
%
%
%
%
%
%
%
%
\subsubsection{Existence of the limit in the general case with current}\label{sec:mass-dependence-j-general}
In this next step we would like to include non vanishing external currents $j \neq 0$. So far, most of the quantum investigation, in particular the construction of a notion of continuity of the theory with respect to the mass, has been done for the source-free case $j = 0$. So before we can investigate a zero mass limit for the general theory, we need to again find a notion of continuity for the general field algebra with respect to the mass $m$. Fortunately, we can make use of the existent notion, since we have already constructed a homeomorphism of the general field algebra $\BUmj$ and the BU-algebra of initial data $\Quotientscale{\BU\big(\Dzs\big)}{\ICCR}$ in Section \ref{sec:field-algebra-topology}. This homeomorphism was constructed using a classical solution $\varphi_{m,j}$ to the inhomogeneous Proca equation $(\delta d + m^2) \varphi_{m,j} = j$. Recalling the algebraic structure (see Diagram \ref{dia:final_structure}) we can formulate a natural notion of continuity analogously to the source free case.
\begin{definition}[Continuity of a family of observables with respect to the mass]\label{def:field_continuity_general}
	Let $j \in \Omega^1(\M)$ be fixed. Let $K_m$, $\Gamma_{m,j,\varphi}$, $\Psi_{m,j,\varphi}$ and $\Lambda_{m}$ be defined as  in Section \ref{sec:field-algebra-topology}, and let $\left\{ \varphi_{m,j} \right\}_m$ specify a continuous family of classical solutions to the inhomogeneous Proca equation $(\delta d + m^2) \varphi_{m,j} = j$.\\
	We call a function
	\begin{align}
	\eta : \IR_+ &\to \bigcup_m \BUmj\\
	m &\mapsto \eta_m  \in \BUmj\notag
	\end{align}
	continuous if the map
	\begin{align}
	\widetilde{\eta} : \IR_+ &\to {\Quotientscale{\BU\big( \Dzs \big) }{\ICCR}} \\
	m &\mapsto \big( \Lambda_m \comp  \Psi_{m,j,\varphi}^{-1} \big) (\eta_m)\notag
	\end{align}
	is continuous.\\
	Equivalently, one can identify $\eta_m = [f_m]_{m,j}$ for some family $\left\{ f_m \right\}_m \subset \BUOmega$ and define the map $\eta$ to be continuous if the map
	\begin{align}
	\hat{\eta} : \IR_+ &\to {\Quotientscale{\BU\big( \Dzs \big) }{\ICCR}} \\
	m &\mapsto \big[  \big( K_m \comp  \Gamma_{m,j,\varphi}^{-1} \big) (f_m) \big]_\sim^\text{CCR}\notag
	\end{align}
	is continuous. The latter will again be the more practical definition.
\end{definition}
First, we again would like to check if, with that notion of continuity, the map $m \mapsto \phi_{m,j}(F)$ is continuous for a fixed $F \in \Omega^1_0(\M).$
Hence, we need to check whether $m \mapsto \big[ \big( K_m \comp \Gamma^{-1}_{m,j,\varphi}\big)\big( (0,F,0,0,\dots) \big) \big]_\sim^\text{CCR}$ is continuous. Assume we have fixed a classical solution $\varphi_{m,j} \in \Omega^1(\M)$ by specifying vanishing initial data on the Cauchy surface $\Sigma$. Shortly, we will show that the notion of continuity does not depend on the choice of the classical solution $\varphi_{m,j}$, so we can indeed chose arbitrary initial data without loss of generality. According to Theorem \ref{thm:solution_proca_unconstrained}, the solution is then specified by
\begin{align}
	\langle \varphi_{m,j} , F \rangle = \sum\limits_\pm \langle j , G_m^\mp F \rangle_{\Sigma^\pm} \formspace.
\end{align}
For $m > 0$, this clearly depends continuously on $m$. We therefore find
\begin{align}
\left( K_m \comp \Gamma^{-1}_{m,j,\varphi}\right)\big( (0,F,0,0,\dots) \big)
&= K_m \Big( \big(   \sum\limits_\pm \langle j , G_m^\mp F \rangle_{\Sigma^\pm} , F,0,0,\dots\big)  \Big) \notag\\
&=  \big(   \sum\limits_\pm \langle j , G_m^\mp F \rangle_{\Sigma^\pm} , \kappa_m(F),0,0,\dots\big)  \formspace,
\end{align}
which again is continuous in $m$. Since $[\cdot]_\sim^\text{CCR}$ does not depend on the mass, we find that $m \mapsto \phi_{m,j}(F)$ is continuous. Therefore, the notion of continuity seems appropriate!\\
We have yet to check that the notion is independent of the choice of the Cauchy surface $\Sigma$ and the classical solution $\varphi_{m,j}$.
\begin{lemma}
	Let $\Sigma$, $\Sigma'$ be Cauchy surfaces, $\left\{\varphi_{m,j} \right\}_m$ specify a continuous family of classical solutions to the inhomogeneous Proca equation and $a,b \in \IR^+$ be arbitrary. Let $\left\{ f_m\right\}_m$ be a family in $\BUOmega$. Then
	\begin{align*}
	\eta^{(\Sigma)} : [ a , b] \to {\Quotientscale{\BU(\Dzs  )}{\ICCR}}\hspace{1mm}, \quad m \mapsto \big[ \big( K^{(\Sigma)}_m \comp \Gamma^{-1}_{m,j,\varphi} \big) \big(f_m\big) \big]_\sim^{\text{CCR}, \Sigma}
	\end{align*}
	is continuous if and only if
	\begin{align*}
	\eta^{(\Sigma')} : [ a , b] \to {\Quotientscale{\BU(\Dzsp)}{\ICCR}}, \quad m \mapsto\big[ \big( K^{(\Sigma')}_m \comp \Gamma^{-1}_{m,j,\varphi} \big) \big(f_m\big) \big]_\sim^{\text{CCR}, \Sigma'}
	\end{align*}
	is continuous.
	That means, the notion of continuity defined in Definition \ref{def:field_continuity_general} is independent of the choice of the Cauchy surface $\Sigma$ for $m$ being an element of a compact set.
	Furthermore, the notion of continuity is independent of the choice of the classical solutions $\left\{\varphi_{m,j} \right\}_m$.
\end{lemma}
\begin{proof}
	Note that both $K_m$ and $\Gamma_{m,j,\varphi}$ are defined as BU-algebra-homomorphisms. Furthermore, since they act on homogeneous elements of degree zero as the identity map, the statement clearly holds for those elements. Since every BU-algebra element can be decomposed into a sum of products of homogeneous degree zero and degree one elements, it suffices to prove the statement for homogeneous elements of degree one.
	Let $\left\{ F_m\right\}_m$ be a family in $\Omega^1_0(\M)$ specifying a family $\left\{ f_m\right\}_m= \left\{(0, F_m,0,0,\dots)\right\}_m$ of BU-algebra elements. Let $a,b \in \IR^+$ and $\Sigma, \Sigma'$ be Cauchy surfaces. The map $\eta: [a,b] \to \BUmj$, $\quad \eta : m\to \eta_m = [f_m]_{m,j}$ is assumed to be continuous in the sense that
\begin{align}
	m \mapsto  \left[ \left( K^{(\Sigma)}_m \comp \Gamma^{-1}_{m,j,\varphi} \right) (f_m) \right]_\sim^{\text{CCR}, \Sigma}
\end{align}
is continuous. We calculate
\begin{align}
\big[ \big( K^{(\Sigma)}_m \comp &\Gamma^{-1}_{m,j,\varphi} \big) \big( (0,F_m,0,0,\dots) \big) \big]_\sim^{\text{CCR}, \Sigma} \notag\\
&=  \left[  K^{(\Sigma)}_m  \big( (-\langle \varphi_{m,j},F_m\rangle_\M,F_m,0,0,\dots) \big) \right]_\sim^{\text{CCR}, \Sigma} \notag\\
&=  \langle \varphi_{m,j} , F_m \rangle_\M \cdot \big[  \mathbbm{1}_{\BU(\Dzs)} \big]_\sim^{\text{CCR}, \Sigma}   + \left[ \big(0,\kappa_m(F_m),0,0,\dots\big) \right]_\sim^{\text{CCR}, \Sigma} \formspace.
\end{align}
We note that, using Lemma \ref{lem:continuity-independence-source-free}, the continuity of $m \mapsto  \left[ \big(0,\kappa_m(F_m),0,0,\dots\big) \right]_\sim^{\text{CCR}, \Sigma}$ is independent of the choice of the Cauchy surface $\Sigma$ for $m \in [a,b]$. Moreover, it is obvious that the continuity of $m \mapsto \langle \varphi_{m,j} , F_m \rangle_\M \cdot \big[  \mathbbm{1}_{\BU(\Dzs)} \big]_\sim^{\text{CCR}, \Sigma} $ is also independent of the choice of $\Sigma$ since it is determined by the continuity of $m \mapsto \langle \varphi_{m,j} , F_m \rangle_\M$ which is independent of the Cauchy surface $\Sigma$. Ergo, the defined notion of continuity is independent of the choice of the Cauchy surface $\Sigma$ as long as $m \in [a,b]$.\par
%
Now, we can check if the notion depends on the choice of the classical solution $\varphi_{m,j}$. \\
Let $\left\{\varphi_{m,j} \right\}_m$ and $\left\{\varphi'_{m,j} \right\}_m$ specify continuous families of classical solutions to the inhomogeneous Proca equation and assume that
\begin{align}
	m &\mapsto \left[ \left( K^{(\Sigma)}_m \comp \Gamma^{-1}_{m,j,\varphi} \right) (f_m) \right]_\sim^{\text{CCR}, \Sigma}
	=\left[   \big( \langle \varphi_{m,j} , F_m \rangle_\M ,\kappa_m(F_m),0,0,\dots\big) \right]_\sim^{\text{CCR}, \Sigma}
\end{align}
is continuous. Since trivially$\big( \langle \varphi_{m,j} , F_m \rangle_\M ,\kappa_m(F_m),0,0,\dots\big)$ is a symmetric element, this is equivalent to
\begin{align}
	m \mapsto   \big( \langle \varphi_{m,j} , F_m \rangle_\M ,\kappa_m(F_m),0,0,\dots\big)
\end{align}
being continuous, following the proof of Lemma \ref{lem:limit_existence_quantum_equivalence} part 1 b) in complete analogy. Since in the topology of the BU-algebra continuity is equivalent to continuity in every degree and recalling $\kappa_m(F_m) = (\rhoz G_m F_m, \rhod G_m F_m)$, we find that
\begin{align}
	m &\mapsto \langle \varphi_{m,j} , F_m \rangle_\M \\
	m &\mapsto \rhoz G_m F_m \quad\text{and}\\
	m &\mapsto \rhod G_m F_m
\end{align}
are continuous. Let $\phi_m,\phi_m'$, $\pi_m,\pi_m'$ denote the initial data of $\varphi_{m,j}$ and $\varphi'_{m,j}$ respectively. By assumption these are continuous in $m$. We calculate
\begin{align}
	\langle \varphi'_{m,j} , F_m \rangle_\M
	&= \langle \varphi'_{m,j} - \varphi_{m,j} , F_m \rangle_\M + \langle \varphi_{m,j} , F_m \rangle_\M \\
	&= \langle \phi'_m - \phi_m, \rhod G_m F_m\rangle_\Sigma - \langle \pi'_m - \pi_m, \rhoz G_m F_m\rangle_\Sigma + \langle \varphi_{m,j} , F_m \rangle_\M \notag
\end{align}
using Theorem \ref{thm:solution_proca_unconstrained}. Using the assumptions on continuity stated above we directly find that $m \mapsto \langle \varphi'_{m,j} , F_m \rangle_\M $ is continuous and hence
\begin{align}
	m \mapsto \left[   \big( \langle \varphi'_{m,j} , F_m \rangle_\M ,\kappa_m(F_m),0,0,\dots\big) \right]_\sim^{\text{CCR}, \Sigma}
\end{align}
is continuous. Therefore, the notion of continuity is independent of the choice of the classical solution $\varphi_{m,j}$.
This completes the proof.
\end{proof}
Choosing vanishing initial data of the classical solution $\varphi_{m,j}$ we will therefore, as in the source free case, from now on identify a field $\phi_{m,j}$ with its initial data formulation $\phi_{m,j}(F) =  \big[\big( \sum_{\pm}\langle j , G_m^\mp F \rangle_{\Sigma^\pm} , \kappa_m(F) , 0 , 0, \dots\big)\big]^\text{CCR}_\sim$. Now, with the work done for the quantum source free case and the general classical case, the result for the general quantum case follows easily:
%
\begin{theorem}[Existence of the zero mass limit in the general case with sources]\label{thm:limit_existence_general}
Let $F \in \Omega^1_0(\M)$ and $j \in \Omega^1(\M)$. Then,
\begin{center}
	the limit $\lim\limits_{m \to 0^+}  \phi_{m,j}(F) \coloneqq \lim\limits_{m \to 0^+} \big[\big( \sum_{\pm}\langle j , G_m^\mp F \rangle_{\Sigma^\pm} , \kappa_m(F) , 0 , 0, \dots\big)\big]^\text{CCR}_\sim $ exists\\[2.5mm] if and only if ${\delta F} = 0 $.
\end{center}
Furthermore, the existence of the limit is independent of the choice of the Cauchy surface $\Sigma$.
\end{theorem}
\begin{proof}
	First, we note that since the notion of continuity is independent of the choice of the classical solution $\varphi_{m,j}$ of Proca's equation, we can choose $\varphi_{m,j}$ such that it has vanishing initial data on the Cauchy surface $\Sigma$. It then holds for all test one-forms $F$, using Theorem \ref{thm:solution_proca_unconstrained}, that $\langle \varphi_{m,j} , F \rangle_\M = \sum_{\pm}\langle j , G_m^\mp F \rangle_{\Sigma^\pm}$. Therefore, by the notion of continuity defined above, the zero mass limit is identified with
	\begin{align}
	\lim\limits_{m \to 0^+}  \phi_{m,j}(F)
	&= \lim\limits_{m \to 0^+} \big[\big( K_m \comp \Gamma^{-1}_{m,j,\varphi} \big)\big(  (0,F,0,0,\dots)   \big)\big]^\text{CCR}_\sim \notag \\
	&= \lim\limits_{m \to 0^+} \big[K_m \big(  (\sum_{\pm} \langle \varphi_{m,j} , F \rangle_\M,F,0,0,\dots)   \big)\big]^\text{CCR}_\sim 	    \notag\\
	&= \lim\limits_{m \to 0^+} \big[\big( \sum_{\pm} \langle j , G_m^\mp F \rangle_{\Sigma^\pm} , \kappa_m(F) , 0 , 0, \dots\big)\big]^\text{CCR}_\sim  \formspace.
	\end{align}
Now, using the same argument as presented in the proof of Lemma \ref{lem:limit_existence_quantum_equivalence}, because $\big( \sum_{\pm} \langle j , G_m^\mp F \rangle_{\Sigma^\pm} , \kappa_m(F) , 0 , 0, \dots\big)$ is symmetric, the limit
\begin{align}
\lim\limits_{m \to 0^+} \big[\big( \sum_{\pm} \langle j , G_m^\mp F \rangle_{\Sigma^\pm} , \kappa_m(F) , 0 , 0, \dots\big)\big]^\text{CCR}_\sim
\end{align} exists if and only the limit
\begin{align}
\lim\limits_{m \to 0^+} \big( \sum_{\pm} \langle j , G_m^\mp F \rangle_{\Sigma^\pm} , \kappa_m(F) , 0 , 0, \dots\big)
\end{align}
exists.
	Since convergence in the BU-algebra means convergence in every degree, we find that the desired limit exists if and only if the limits $\lim\limits_{m \to 0^+}\sum_{\pm} \langle j , G_m^\mp F \rangle_{\Sigma^\pm}$ and $\lim\limits_{m \to 0^+} \kappa_m(F) $ exist. Recalling Lemma \ref{lem:limit_existence_quantum_equivalence} the latter is equivalent to the existence of the limit $\lim\limits_{m \to 0^+} \frac{1}{m^2}E_m d \delta F$. Together, this corresponds exactly to the classical situation and using Theorem \ref{thm:limit_existence_general_classical} completes the proof.
\end{proof}
Just as in the classical case, also with external sources present, the limit exists if and only if we implement the gauge equivalence of Maxwell's theory as discussed in Section \ref{sec:zero-mass-limit-existence-classical-vanishing-source}.
%
%
%
%
%
%
%
%%
%
%
\subsubsection{Algebra relations, dynamics and the zero mass limit}\label{sec:zero-mass-limit-quantum-algebra-relations}
Now that we have classified the existence of the zero mass limit of the quantum Proca field in curved spacetimes we want study the algebra relations of the fields in the zero mass limit. We would like to compare these with the ones obtained from the quantization of the Maxwell field.\par
Identifying the fields $\phi_{0,j}(F)$ with the zero mass limit of
\begin{align*}
\phi_{m,j}(F) = \big[\big( \sum_{\pm}\langle j , G_m^\mp F \rangle_{\Sigma^\pm} , \kappa_m(F) , 0 , 0, \dots\big)\big]^\text{CCR}_\sim \formspace,
\end{align*}
we define the algebra $\mathscr{A_0}$ as the algebra generated by $\mathbbm{1}$ and the symbols $\phi_{0,j}(F)$ for any co-closed test one-form $F$. Using the algebra relations of the fields $\phi_{m,j}$, it is clear that in the zero mass limit the fields are subject to the relations
\begin{align}
	1.)\; &\phi_{0,j}(\alpha F + \beta F') = \alpha \phi_{0,j}(F) + \beta \phi_{0,j}(F') 														 \\
	2.)\; &\phi_{0,j}(F)^* = \phi_{0,j}(\skoverline{F}\,)
\end{align}
for all $F \in \Omega^1_{0,\delta}(\M)$ and $\alpha, \beta \in \IC$, corresponding to the linearity and the real field property.
For the canonical commutation relations we calculate, $F,F' \in \Omega^1_{0,\delta}(\M)$,
\begin{align}
\gls{Ezcurly}(F,F')
&=\langle F, E_0 F' \rangle_\M \notag\\
&= \langle E_0 F' , F \rangle_\M \notag\\
&= \langle \rhoz E_0 F' , \rhod E_0 F  \rangle_\Sigma + \langle \rhodelta E_0 F' , \rhon E_0 F \rangle_\Sigma \notag \notag\\
&\phantom{=I}- \langle  \rhod E_0 F' , \rhoz E_0 F \rangle_\Sigma - \langle \rhon E_0 F', \rhodelta E_0 F \rangle_\Sigma \formspace.
\end{align}
We have used that $E_0 F'$ solves a homogeneous massless wave equation to which the solution is determined by initial data using Theorem \ref{thm:solution_wave_equation}. Using $\rhodelta E_0 F' = i^* \delta E_0 F' = i^* E_0 \delta F' = 0$ and the analogue expression for $F$, we find
\begin{align}
\Ez{F}{F'}
&= \langle \rhoz E_0 F' , \rhod E_0 F  \rangle_\Sigma - \langle  \rhod E_0 F' , \rhoz E_0 F \rangle_\Sigma \notag \\
&= \lim\limits_{m \to 0^+} \Big( \langle \rhoz E_m F' , \rhod E_m F  \rangle_\Sigma - \langle  \rhod E_m F' , \rhoz E_m F \rangle_\Sigma  \Big) \notag \\
&= \lim\limits_{m \to 0^+} \Big( \langle \rhoz G_m F' , \rhod G_m F  \rangle_\Sigma - \langle  \rhod G_m F' , \rhoz G_m F \rangle_\Sigma  \Big) \notag \\
&= \lim\limits_{m \to 0^+}  \Gm{F}{F'} \formspace.
\end{align}
Again we have used that for co-closed test one-forms $F$ it holds $G_m F = E_m F$.
Since for co-closed test one-forms $F \in \Omega^1_{0,\delta}$ the fundamental solutions $E^\pm_0$ of the massless Klein-Gordon operator are actually also fundamental solutions to Maxwell's equation, $E_0^\pm \delta d F = E_0^\pm (\delta d + d \delta) F = F$, we find that the fields in the zero mass limit are subject to the correct canonical commutation relations
\begin{align}
	\big[ \phi_{0,j}(F) ,\phi_{0,j}(F')  \big]
	&= \lim\limits_{m \to 0^+} \big[ \phi_{m,j}(F) ,\phi_{m,j}(F')  \big]  \notag\\
	&=\i \cdot \lim\limits_{m \to 0^+} \Gm{F}{F'}\cdot \mathbbm{1} \notag\\
	&= \i \, \Ez{F}{F'}\cdot \mathbbm{1}
\end{align}
for all $F,F' \in \Omega^1_{0,\delta}(\M)$. So far, this also corresponds perfectly to the relations presented in \cite[Definition 4.5]{Sanders}.
Note that now $\Ez{F}{F'}$, for $F,F' \in \Omega^1_{0,\delta}(\M)$, is in general degenerate, hence the quantum field theory associated with $\phi_{0,j}$ will in general fail to be local in the sense defined in Definition \ref{def:generally-coveriant-qftcs}. This is in detail discussed in \cite{Sanders}. \par
%
It remains to check the dynamics of the theory. We want to check if the fields solve Maxwell's equation in a distributional sense, that is if $\phi_{0,j}(\delta d F) = \langle j , F\rangle_\M$ holds for all $F\in \Omega^1_0(\M)$. Since it holds for all $F \in \Omega^1_0(\M)$ that $\delta d F$ is co-closed, the limit
\begin{align}
	\phi_{0,j} (\delta d F) =  \lim\limits_{m \to 0^+} \big[\big( \sum_{\pm}\langle j , G_m^\mp \delta d F \rangle_{\Sigma^\pm} , \kappa_m(\delta d F) , 0 , 0, \dots\big)\big]^\text{CCR}_\sim
\end{align}
exists for all test one-forms $F$. We use $G_m \delta d F = E_m \delta d F$ and find
\begin{align}
\phi_{0,j} (\delta d F) = \lim\limits_{m \to 0^+} \big[\big( \sum_{\pm}\langle j , E_m^\mp \delta d F \rangle_{\Sigma^\pm} , (\rhoz E_m \delta d F, \rhod E_m \delta d F) , 0 , 0, \dots\big)\big]^\text{CCR}_\sim \formspace.
\end{align}
Using $E^\pm_0 \delta d F = F - (d \delta + m^2) E^\pm_0 F$, we find
\begin{align}
\phi_{0,j} &(\delta d F) \notag \\
&= \lim\limits_{m \to 0^+}\Big(  \big[\big( \sum_{\pm}\langle j ,  F \rangle_{\Sigma^\pm} - \sum_{\pm}\langle j , E_m^\mp d\delta  F \rangle_{\Sigma^\pm} ,- (\rhoz E_m d \delta F, \rhod E_m d \delta F) , 0 , 0, \dots\big)\big]^\text{CCR}_\sim \notag \\
&\phantom{=I} - m^2 \big[\big( \sum_{\pm}\langle j ,  E_m^\mp F \rangle_{\Sigma^\pm}  , (\rhoz E_m  F, \rhod E_m  F) , 0 , 0, \dots\big)\big]^\text{CCR}_\sim\Big) \formspace.
\end{align}
Since the term proportional to $m^2$ is continuous and bounded in every degree and hence vanishes in the limit and using that $\rhod E_m d \delta F = \rhon d E_m d \delta F =0$, we find
\begin{align}\label{eqn:zero-mass-limit-field-temp}
\phi_{0,j} (\delta d F)
&= \langle j ,  F \rangle_{\M} - \lim\limits_{m \to 0^+}\big[ \big(  \sum_{\pm}\langle j , E_m^\mp d\delta  F \rangle_{\Sigma^\pm} , (d_{(\Sigma)} \rhoz E_m  \delta F, 0) , 0 , 0, \dots\big)\big]^\text{CCR}_\sim \notag\\
&= \langle j ,  F \rangle_{\M} - \big[ \big(  \sum_{\pm}\langle j , E_0^\mp d\delta  F \rangle_{\Sigma^\pm} , (d_{(\Sigma)} \rhoz E_0  \delta F, 0) , 0 , 0, \dots\big)\big]^\text{CCR}_\sim  \formspace.
\end{align}
Note that there appears a remainder $\big[ \big(  \sum_{\pm}\langle j , E_0^\mp d\delta  F \rangle_{\Sigma^\pm} , (d_{(\Sigma)} \rhoz E_0  \delta F, 0) , 0 , 0, \dots\big)\big]^\text{CCR}_\sim$ which, in order for the quantum fields to solve the correct dynamics, should vanish. It turns out that this is in general not the case. We have encountered a similar situation in the investigation of the classical theory in Section \ref{sec:limit_dynamics_classical}. There we could get rid of similar remaining terms by restricting the initial data such that the Lorenz constraint is well behaved in the limit.
In the quantum scenario this will also partly solve the problem, but note that the construction of the quantum theory is in that point quite different from the classical construction as we directly impose the dynamics by dividing out corresponding ideals, rather than first solving a wave equation and restrict to those solutions that fulfill the Lorenz constraint. Therefore, at the quantum level, the Lorenz constraint does not appear directly. It does, however, appear indirectly as we have fixed a classical solution $\varphi_{m,j}$ to map the source dependent theory to the source free theory. And it is with this homeomorphism where one part of the problem lies: \\
Recall that we have mapped the source dependent theory to the source free theory via the homeomorphism $\Gamma_{m,j,\varphi}$, choosing a classical solution to Proca's equation $(\delta d + m^2)\varphi_{m,j} = j$. We have used that the notion of continuity does not depend on the initial data that we choose for $\varphi_{m,j}$ and, for simplicity, we chose vanishing initial data. Using Theorem \ref{thm:solution_proca_unconstrained} for the classical solution $\varphi_{m,j}$ with vanishing initial data, this gave for some $F \in \Omega^1_0(\M)$
\begin{align}
	\Gamma^{-1}_{m,j,\varphi}\big( (0,F,0,0,\dots) \big)
	&= \langle \varphi_{m,j} , F \rangle_\M \cdot \mathbbm{1} + (0,F,0,0,\dots)   \notag \\
	&= \sum_{\pm}\langle j ,G_m^\mp F \rangle_{\Sigma^\pm} \cdot \mathbbm{1} + (0,F,0,0,\dots) \formspace.
\end{align}
For non-zero $m$, this is not a problem. But in the zero mass limit, $\lim\limits_{m \to 0^+} \langle \varphi_{m,j} , F \rangle_\M = \lim\limits_{m \to 0^+}  \sum_{\pm}\langle j ,G_m^\mp F \rangle_{\Sigma^\pm}$ will not specify a solution to Maxwell's equation as the Lorenz constraint expressed by the initial data,
\begin{align}
	\rhodelta \varphi_{m,j} &= \frac{1}{m^2}\rhodelta j \; , \quad \text{and} \\
	\rhon \varphi_{m,j} &= \frac{1}{m^2}\left( \rhon j  + \delta_{(\Sigma)} \rhod \varphi_{m,j} \right) \formspace,
\end{align}
is not well behaved (as we have chosen $\rhod \varphi_{m,j}=0$). As discussed in the classical Section \ref{sec:limit_dynamics_classical}, we need to impose $\delta j=0$ and $\rhon j= -\delta_{(\Sigma)} \pi$, where $\pi = \rhod \varphi_{m,j}$. Choosing vanishing initial data, we have violated the latter for non-zero external sources! Ergo, the homeomorphism between the source free and source dependent theory will only be well behaved in the limit if we set
\begin{align}
	\Gamma^{-1}_{m,j,\varphi} \big( (0,F,0,0,\dots) \big)
	&= \langle \varphi_{m,j} , F \rangle_\M \cdot \mathbbm{1} +(0,F,0,0,\dots)  \\
	&= \big(  \sum_{\pm}\langle j ,G_m^\mp F \rangle_{\Sigma^\pm} - \langle \pi , \rhoz G_m F\rangle_\Sigma \big) \cdot \mathbbm{1} +(0,F,0,0,\dots) \formspace,\notag
\end{align}
such that $\delta j=0$ and $\rhon j= -\delta_{(\Sigma)} \pi$.\\
Concluding, we have to define the zero mass limit field as
\begin{align}
	\phi_{0,j}(F) = \lim_{m \to 0^+} \big[\big( \sum_{\pm}\langle j , G_m^\mp F \rangle_{\Sigma^\pm} - \langle \pi , \rhoz G_m F \rangle_\Sigma, \kappa_0(F) , 0 , 0, \dots\big)\big]^\text{CCR}_\sim \formspace,
\end{align}
such that $\delta j=0$ and $\rhon j= -\delta_{(\Sigma)} \pi$. The limit exists if and only if $F$ is co-closed. With this we obtain
\begin{align}
	\phi_{0,j} (\delta d F)
	&= \langle j ,  F \rangle_{\M} \notag \\
	&\phantom{=I}- \big[ \big(  \sum_{\pm}\langle j , E_0^\mp d\delta  F \rangle_{\Sigma^\pm} - \langle \pi , \rhoz E_0 F \rangle_\Sigma , (d_{(\Sigma)} \rhoz E_0  \delta F, 0) , 0 , 0, \dots\big)\big]^\text{CCR}_\sim \notag \\
	&= \langle j ,  F \rangle_{\M}
	- \big[ \big( 0 , (d_{(\Sigma)} \rhoz E_0  \delta F, 0) , 0 , 0, \dots\big)\big]^\text{CCR}_\sim
\end{align}
since $\sum_{\pm}\langle j , E_0^\mp d\delta  F \rangle_{\Sigma^\pm} = \langle \pi , \rhoz E_0 F \rangle_\Sigma$ as we have shown in Section \ref{sec:limit_dynamics_classical}.\par
At this point, the remainder $\big[ \big( 0 , (d_{(\Sigma)} \rhoz E_0  \delta F, 0) , 0 , 0, \dots\big)\big]^\text{CCR}_\sim $ does not seem to vanish naturally. Its appearance might be explained by gauge equivalence:
Note that $-E_0 \delta dF$ solves the source free Maxwell equation, $-\delta d E_0 \delta d F = \delta d  E_0 d \delta F = 0$, and has initial data $-(\rhoz E_0 \delta d F, \rhod E_0 \delta d F) =  (d_{(\Sigma)} \rhoz E_0  \delta F, 0)$ as we have calculated before. From the classical investigation, see for example \cite{Sanders} or \cite{pfenning} which works in the same initial data formalism as we do here, we know that two solutions $A, A'$ to Maxwell's equation are gauge equivalent if and only if their initial values are gauge equivalent, that is, if  $\Az = \Azp + d\chi$ for some $\chi \in \Omega^0_0(\Sigma)$ (see \cite[Proposition 2.13]{pfenning}). With this we find that the solution $-E_0 \delta dF$ is gauge equivalent to zero and $(d_{(\Sigma)} \rhoz E_0  \delta F, 0) \sim_\text{gauge} 0$. This would give rise to a quantum field with the correct dynamics implemented. But this gauge equivalence on the level of the observables rather than the fields does not seem to come out of the limiting procedure naturally! It might be possible to find natural conditions that hold in the limiting procedure (as we have demanded that the homeomorphism $\Gamma_{m,j,\varphi}$ should be well behaved in the zero mass limit) to obtain the correct dynamics in the limit. Another possibility is to include states in the investigation and formulate a similar limiting process which might give rise to conditions identifying the remaining observables with zero. Alternatively, one could even go as far as concluding that the Proca field is unsuitable to describe massive electrodynamics with a well defined zero mass limit. A recent study \cite{stueckelberg_curvedST} suggests to abandon the investigation of Proca's theory as it is unsuited for the quantum investigation and rather examine Stueckelberg massive electromagnetism. As it is argued in \cite{stueckelberg_curvedST}, Proca's theory is nothing more than Stueckelberg's in a certain gauge that seems unfit for the investigation of the zero mass limit. It might therefore be of interest to apply the construction presented in this thesis to Stueckelberg electromagnetism, but since Stueckelberg's theory involves interaction with a scalar field, it is a priori not clear if that is possible.\par
Unfortunately within the (time) scope of this thesis, these ideas were not further studied but they are worthwhile investigating in future research projects.
%
%Following the construction for the Proca theory presented in Section \ref{sec:field-algebra-topology}, we found the initial data formulation by starting at the one particle level $\Omega^1_0(\M)$:\\
%We defined the map $\kappa_m : \Omega^1_0(\M) \to \Dzs$ and found that the zero mass limit exists if and only if we restrict the domain to co-closed test one-forms and found
%\begin{align}
%	\kappa_0 : \Omega^1_{0,\delta}(\M) & \to \IMG{\kappa_0} \subset \Dzs \\
%	F &\mapsto (\rhoz E_0 F, \rhod E_0 F)
%\end{align}
%where naturally, the image of $\kappa_0$ will not be the full space $\Dzs$ anymore. For co-closed test one-forms $F$, we know that $E_0 F$ solves the source free Maxwell equation. As a result from the classical investigation, see for example \cite{Sanders} or \cite{pfenning} which works in the same initial data formalism as we do here, a solution $A$ to Maxwell's equation is specified by initial data $\Az, \Ad \in \Omega^1_0(\Sigma)$ such that $\delta_{(\Sigma)} \Ad =0$. As we have seen in the classical Section \ref{sec:limit_dynamics_classical}, this implements the Lorenz constraint in the source free case. Furthermore, one finds that two solutions $A, A'$ are gauge equivalent if and only if their initial values are gauge equivalent, that is, if  $\Az = \Azp + d\chi$ for some $\chi \in \Omega^0_0(\Sigma)$ (see \cite[Proposition 2.13]{pfenning}). Therefore, the map $ \kappa_0$  has range contained in
%\begin{align}
%\Dzzs \coloneqq \frac{\Omega^1_0(\M)}{d_{(\Sigma)}\Omega^0_0(\Sigma)} \oplus \Omega^1_{0,\delta} \formspace.
%\end{align}
%Lifting the map $\kappa_0$ to the field algebra map $K_0$, as we have done for the massive case, we find that the image of $K_0$ has to be contained in $\BU(\Dzzs)$.\\
%Ergo, the remainder $\big[ \big(  \sum_{\pm}\langle j , E_0^\mp d\delta  F \rangle_{\Sigma^\pm} , (d_{(\Sigma)} \rhoz E_0  \delta F, 0) , 0 , 0, \dots\big)\big]^\text{CCR}_\sim$ is actually gauge equivalent to
%$\big[ \big(  \sum_{\pm}\langle j , E_0^\mp d\delta  F \rangle_{\Sigma^\pm} , 0 , 0 , 0, \dots\big)\big]^\text{CCR}_\sim$.\par
%So, for the source free theory, this yields the correct dynamics for the zero mass limit following equation (\ref{eqn:zero-mass-limit-field-temp}):
%\begin{align}
%\phi_{0,j} (\delta d F)
%&=  - \big[ \big(  0 , (d_{(\Sigma)} \rhoz E_0  \delta F, 0) , 0 , 0, \dots\big)\big]^\text{CCR}_\sim \\
%&\sim_\text{gauge} 0 \formspace.
%\end{align}
%
%
%
%
%
%
%Ergo, the zero mass limit fields $\phi_{0,j}$ do not fulfill Maxwell's equation! Recall, that we have encountered the analogous result in the classical case in Section \ref{sec:zero-mass-limit-classical}. In the classical case, the problem lay with the restriction on the initial data that implement the Lorenz constraint such that the solution to the wave equation becomes a solution of Proca's equation. This seems also be the case in the quantum case as we have indirectly implemented these constraints by identifying the fields with their initial data formulation on $\BU(\Dzs)$. With the help of the classical work, we were able to show $\BUmjdyn$ being homeomorphic to $\BU(\Dzs)$. Without making sure that the constraints are well behaved in the zero mass limit, this homeomorphy will not hold in the limit, as we will shortly see. But how do we make sure, that the relation holds in the limit, that is, do we implement the constraints corresponding to the Lorenz condition in the quantum case?\\
%Recall that in the classical case we had to restrict to
%\begin{align}
%	\delta j &= 0 \quad \text{and} \\
%	\rhon j &= -\delta_{(\Sigma)} \Ad \formspace.
%\end{align}
%It seems natural to also demand conservation of current, $\delta j = 0$ in the quantum case in order to obtain the correct dynamics. In the classical case, we were able to show that the remainder of the dynamics in the limit,
%\begin{align}
%\sum_{\pm} \langle j, E_m^\mp d\delta F \rangle_{\Sigma^\pm} - \langle \Ad , d_{(\Sigma)} \rhoz E_m \delta F \rangle_\Sigma
%\end{align}
%vanishes when constraining $\Ad$ such that the Lorenz constraint is well behaved in the limit. This seems very closely related to the remainder
%\begin{align}
%	\lim\limits_{m \to 0^+}\big[ \big(  \sum_{\pm}\langle j , E_m^\mp d\delta  F \rangle_{\Sigma^\pm} , (d_{(\Sigma)} \rhoz E_m  \delta F, 0) , 0 , 0, \dots\big)\big]^\text{CCR}_\sim
%\end{align}
%in the quantum case and we wonder how to find a deeper understanding of the quantum analogue.\\
%In the quantum case we have put in our knowledge of the classical theory in the definitions of the homeomorphisms of the field algebras, in particular the homeomorphism $\Xi_m$ between the source-free dynamical field algebra $\BUmdyn$ and the BU-algebra of initial data $\BU(\Dzs)$, and the homeomorphism $\Gamma_{m,j,\varphi}$ between the source-free and the general theory. We have to make sure, that both of these homeomorphisms are well behaved in the zero mass limit. Without further control of the initial data, homeomorphy will not hold, as we will see,  which explains why the obtained zero mass limit fields do not obey the desired dynamics!\\
%To illustrate this and solve the problem, we first look at the source free theory again and investigate the homeomorphism $\Xi_m$.\par
%Recall, that $\Xi_m$ was defined as the lift of the map $\kappa_m : \Omega^1_0(\M) \to \Dzs$ that mapped test one-forms to initial data, $F \mapsto (\rhoz G_m F, \rhod G_m F)$. On $\Dzs$, the map was surjective and gave rise to a homeomorphism between the space of dynamical one-forms ${\Quotientscale{\Omega^1_{0,\delta}(\M)}{\delta d \Omega^1_0(\M)}}$ and the space of initial data $\Dzzs$. In the zero mass limit we need to put further constraints on the initial data in order for surjectivity to hold:\\
%For $F \in \Omega^1_0$, $E_0 F$ is a solution to the source free Maxwell equation.
%
%
%We have shown, that the zero mass limit of this map exists, if $F$ is co-closed due to $G_m F = E_m F$ for $F \in \Omega^1_{0,\delta}$.
%
%But in the limit, this map will not behave in the desired way, that is, it will not map a co-closed test one-form to its initial data with respect to Maxwell's equation! This is, because we have to accommodate for the Lorenz constraint and the gauge equivalence of the theory. For Maxwell's theory, a solution $A \in \Omega^1_0(\M)$ to $\delta d A = 0$ is specified by initial data $A_0, A_d \in \Omega^1_0(\Sigma)$ such that $\delta A_d = 0$. This restriction corresponds to implementing the Lorenz constraint, as explained in the classical chapter. Furthermore, due to the gauge equivalence, to solutions $A,A'$ specified by initial data $A_0,A_d, A'_0, A'_d$ will be gauge equivalent if $A_0 \sim A'_0$ where the gauge equivalence is given by $A_0 \sim A'_0 \iff A_0 - A'_0 = d\chi$ for some $\chi \in \Omega^0_0(\Sigma)$, see for example \cite[Theorem 2.22]{Sanders} or \cite[Proposition 2.13]{pfenning} which uses the same initial data description adapted form \name{Dimock} as we do in this thesis. Hence, the space of initial data on which $\kappa_0$ will be surjective on is
%\begin{align}
%	\Dzzs = \frac{\Omega^1_0(\M)}{d_{(\Sigma)}\Omega^0_0(\Sigma)} \oplus \Omega^1_{0,\delta}
%\end{align}
%rather then $\Dzs = \Omega^1_0(\Sigma) \oplus \Omega^1_0(\Sigma)$ . Defining $\kappa_0 : \Omega^1_{0,\delta} (\M) \to \Dzzs$ such that $F \mapsto (\rhoz E_0 F, \rhod E_0 F)$ will give, complete analogy to Lemma \ref{lem:one-particle-homeomorphism}, a homeomorphism between the space of dynamical one-forms ${\Quotientscale{\Omega^1_{0,\delta}(\M)}{\delta d \Omega^1_0(\M)}}$ and the space of initial data $\Dzzs$. Having adapted the initial data to the Maxwell case, the construction presented in Section \ref{sec:field-algebra-topology} is directly applicable for the re-defined $\kappa_0$. That is, we lift $\kappa_0$ to the *-algebra-homomorphism $K_0 : \BU(\Omega^1_{0,\delta}) \to \BU(\Dzzs)$ which gives rise to the homeomorphism $\Xi_0$ between the source-free dynamical field algebra $\BUmzdynMaxw$ and the BU-algebra of Maxwell initial data $\BU(\Dzzs)$. In order to get the correct dynamics in the limit, we have to restrict the initial data such that the homeomorphism $\Xi_m$ converges to $\Xi_0$. Ergo, the zero mass limit field in equation (\ref{eqn:zero-mass-limit-field-temp}) , setting $j=0$,
%\begin{align}
%\phi_{0,j} (\delta d F)
%&= - \big[ \big( 0 , (d_{(\Sigma)} \rhoz E_m  \delta F, 0) , 0 , 0, \dots\big)
%\end{align}
%is equivalent, w.r.t to the equivalence class obtained from the one-particle level equivalence of $A_0 \sim A'_0$ explained above, to the desired solution
%\begin{align}
%\phi_{0,j} (\delta d F) = 0 \formspace.
%\end{align}
%Concluding, if we keep the homeomorphism $\Xi_m$ well defined in the limit, by restricting to the Maxwell initial data space $\Dzzs$ on the one particle level, the dynamics of the zero mass limit fields turn out correctly!\par
%In the next step, we include external sources $j$. Since we have constructed the source dependent initial data theory via a homeomorphism $\Gamma_{m,j,\varphi}$, we also need to make sure that this is well behaved in the zero mass limit. Recall, we have constructed the homeomorphism $\Gamma_{m,j,\varphi}$ using a classical solution $\varphi_{m,j}$ to the inhomogeneous Proca equation $(\delta d + m^2) \varphi_{m,j} = j$ to which the solution is uniquely determined by Theorem \ref{thm:solution_proca_unconstrained}. We have shown that the notion of continuity of the quantum Proca fields with respect to the mass is independent of the choice of initial data of $\varphi_{m,j}$ and have hence chosen $\varphi_{m,j}$ to have vanishing initial data on the Cauchy surface. The solution is then uniquely specified by $\langle \varphi_{m,j} , F \rangle_\M = \sum_{\pm} \langle j, G_m^\mp F \rangle_{\Sigma^\pm}$. But it is immediately clear, that in the zero mass limit this does not specify a solution to Maxwell's equation since $\langle \varphi_{m,j} , \delta dF \rangle_\M = \sum_{\pm} \langle j, E_0^\mp F \rangle_{\Sigma^\pm} \neq 0$ as we have discussed in the classical chapter. Making sure that the Lorenz constraint is well behaved in the limit, we have to constrain $\rhod \varphi_{m,j} = - \rhon j$ and cannot specify $\varphi_{m,j}$ to have vanishing initial data if we want the homeomorphism to be well defined in the limit! So specifying $\varphi_{m,j}$ such that it solves the Maxwell equation in the limit, we need to set $\langle \varphi_{m,j} , F \rangle_\M = \sum_{\pm} \langle j, G_m^\mp F \rangle_{\Sigma^\pm} - \langle \pi , \rhoz G_m F \rangle_\Sigma$ such that $\delta_{(\Sigma \pi)} = -\rhon j$, as worked out in Section \ref{sec:limit_dynamics_classical}.\par
%Bringing the two steps together, we identify the zero mass limit field with, $F \in \Omega^1_{0,\delta}(\M)$,
%\begin{align}
%\phi_{0,j}(F) = \big[\big( \sum_{\pm}\langle j , E_0^\mp F \rangle_{\Sigma^\pm} - \langle \pi , \rhoz E_0 F \rangle_\Sigma, \kappa_0(F) , 0 , 0, \dots\big)\big]^\text{CCR}_\sim
%\end{align}
%and obtain the dynamics, for any $F \in \Omega^1_0(\M)$
%\begin{align}
%\phi_{0,j} (\delta d F)
%&= \langle j ,  F \rangle_{\M} \notag \\
%&- \lim\limits_{m \to 0^+}\big[ \big(  \sum_{\pm}\langle j , E_m^\mp d\delta  F \rangle_{\Sigma^\pm} - \langle \pi , \rhoz E_m F\rangle_\Sigma, ([d_{(\Sigma)} \rhoz E_m  \delta F], 0) , 0 , 0, \dots\big)\big]^\text{CCR}_\sim
%\end{align}
%such that $\delta_{(\Sigma)}\pi = -\rhon j $ and $\delta j =0$. Then, $\sum_{\pm}\langle j , E_m^\mp d\delta  F \rangle_{\Sigma^\pm} = \langle \pi , \rhoz E_m F\rangle_\Sigma$ as shown in Section $\ref{sec:limit_dynamics_classical}$, and by construction $([d_{(\Sigma)} \rhoz E_m  \delta F] = 0$ and hence
%\begin{align}
%\phi_{0,j} (\delta d F)
%&= \langle j ,  F \rangle_{\M}
%\end{align}
%gives the correct dynamics!\par
%In the end, the result of the zero mass limit investigation is not surprising:\\
%We obtain the quantum Maxwell field theory if we identify gauge equivalent solutions and make sure that the Lorenz constraint is well behaved, by restricting to co-closed test one-forms $F \in \Omega^1_{0,\delta}(\M)$ and the initial data space to $\Dzzs$. Furthermore, we need to implement conservation of current $\delta j = 0$.
%
%
%
%
%
%
