
\section{Conclusions and Future Work}
\label{sec:conclusion}
In conclusion, our proposed multiscale architecture, coupled with the genetic algorithm tuning procedure, provides a step forward in making bio-inspired systems work with both large scale simulated and real-world data and capable of handling large velocity ranges. This is a key step towards making bio-inspired networks competitive with conventional robotic navigation systems.

Through the development of a multiscale cognitive architecture, we were able to significantly enhance the performance of the continuous attractor model. Our approach results in a system capable of handling a wide range of velocities and complex environments. The proposed genetic algorithm tuning procedure allowed for efficient and effective optimization of network parameters, reducing the need for manual tuning and increasing the scalability of the system. Our results show that the genetic algorithm can successfully optimize both the heading direction and multiscale parameters, leading to improved performance in navigation tasks.

While our proposed system has shown promise, there is potential for future work to address some existing limitations. One limitation is the absence of landmarks or memorized locations of previously visited areas, which means errors could eventually accumulate without a corrective mechanism. Integrating a loop closure component will move this from a dead reckoning system to a full mapping and localization system. Although we conducted extensive simulation experiments drawing upon real world city networks as well as real-world data from Kitti, further live experimentation on deployed robot platforms will provide insights that can drive and inform future model development.

