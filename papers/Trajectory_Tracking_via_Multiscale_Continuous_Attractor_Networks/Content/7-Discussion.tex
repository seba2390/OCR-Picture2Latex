
\subsection{Comparison of Multiscale versus Single-Scale}
\label{subsec:Performance}

Our experiments demonstrate that incorporating a multiscale system can lead to significant performance enhancements on the standard CAN model. Table~\ref{tab:atecomparison} presents the average translational error (ATE) for the single-scale CAN and our proposed MCAN model on five different datasets. We observed that the ATE for MCAN was orders of magnitude lower than that of the single-scale baseline, with improvements evident across all tasks. The City Scale Navigation simulations showed the most significant difference between MCAN and CAN, likely because of their more extensive range of input velocities. On the Kitti dataset, MCAN showed performance improvements specifically in areas of the path with higher velocities and minor improvements in the other regions, as depicted in Figure~\ref{fig:5.6}. 

Furthermore, we evaluated the impact of the input velocity range on the performance of the single-scale and multiscale systems. Figure~\ref{fig:5.2} shows the ATE for different velocity ranges for the OSM simulation dataset. We can observe that the ATE of the single-scale network increases linearly with the velocity range, whereas the ATE of the MCAN increases with a smaller gradient. This demonstrates that MCAN can handle varying velocity ranges and maintain high accuracy, even when faced with challenging trajectories, generated from the tracks seen in Figure~\ref{fig:5.1}.

%

%

Figure~\ref{fig:5.5} provides further insights into the behaviour of MCAN. It shows track segments from the Berlin CSN simulator with increasing tracking errors. Trajectories with multiple 90-degree turns have increased errors in comparison to trajectories with smaller turns. 

%

However, these errors are orders of magnitude less than the single-scale model. An example of this is in Figure~\ref{fig:5.7}, where the error accumulates up to 10000 meters within the single-scale run and MCAN has a maximum error of 450 meters with less accumulation of error over time. This is further supported by Figure~\ref{fig:5.8} where the MCAN error is consistently lower in all 18 tracks through Berlin. 



\subsection{Tuning Performance}
\label{subsec:GA}

The genetic algorithm tuning procedure was successful in achieving good performance despite the added complexity of multiple interacting models. Both heading direction (1D) and multiscale (2D) tuning converged after about 3 generations, as shown in Figure~\ref{fig:5.3}. The multiscale tuning had limited fitness improvements after generation 2, suggesting that either there is an upper limit on what fitness is realistically possible or that the algorithm is prone to getting stuck in local optima.

%

%

%
%
%



%
%

%





    %
    %
    %
    %

    %
    %
    %
    %
%

