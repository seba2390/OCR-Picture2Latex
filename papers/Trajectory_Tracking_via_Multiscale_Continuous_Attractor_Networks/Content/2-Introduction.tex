\section{Introduction}

Robotic navigation systems encounter several challenges: creating efficient maps, adapting to significant environmental changes, and long-term tracking. In contrast, biological systems have evolved efficient strategies for lifelong navigation while performing tasks such as foraging, migration, and homing. Neuroscience research on spatial representation has identified some of these mechanisms that encode and integrate sensory information to build spatial maps using models such as Continuous Attractor Networks (CAN)~\cite{seung1997learning,arnold1991learning,burak2009accurate}. Robotics has drawn inspiration from this field and developed algorithms such as RatSLAM~\cite{milford2010persistent} and NeuroSLAM~\cite{yu2019neuroslam} that attempt to emulate some of the biological mechanisms of spatial navigation. These systems still face significant challenges in terms of one or more of robustness, scalability, ease of deployment, and adaptability.

%

In particular, robust navigation systems should be able to operate over a wide range of spatial scales without incurring excessive memory usage. While most CAN navigation systems use single-scale networks for 2D and 3D spaces, a multiscale network can switch modes of operation according to the input scale, making them suitable for long-term navigation. 

%

\begin{figure}[t]
     %
     \includegraphics[width=0.9\linewidth]
      {Figures/0_Fig1_new.pdf}
      \vspace*{-0.2cm}
      \caption{Overview of the Multiscale Continuous Attractor Network (MCAN) architecture: The head direction network on the left processes angular velocity to estimate the robot's heading direction, which is then fed into MCAN along with linear velocity. The MCAN integrates these inputs at multiple scales and generates trajectory estimates -- an example of a generated trajectory is shown on the right. Note that MCAN only uses odometry information.}
    \label{fig:front}
    \vspace*{-0.2cm}
\end{figure}

This paper proposes a new multiscale biologically inspired network, with the following contributions: 
%
%
%
%
%
\begin{enumerate}
    \item \textbf{Multiscale Continuous Attractor Networks (MCAN):} Our proposed Multiscale Continuous Attractor Networks (MCAN) is a bio-inspired neural network architecture with multi-scale parallel spatial neural networks building on previous theoretical models to accurately encode a wide range of velocity inputs and enable large scale trajectory tracking (\Cref{fig:front}).   
    \item \textbf{A tuning method using genetic algorithms:} The reliance of previous systems on hand-tuned parameters is overcome by presenting a genetic algorithm-based approach for automated tuning of the MCAN and a head direction network. This optimization method automatically identifies high performance parameter spaces in large CAN networks. 
    \item \textbf{A city-scale navigation simulator:} To provide challenging navigational scale ranges, we contribute a flexible city-scale navigation simulator that adapts to any street network, enabling high throughput experimentation for evaluating path integration performance. %
\end{enumerate}

%

The paper is structured as follows: In Section~\ref{sec:relatedworks}, we will provide an overview of the brain's navigational mechanisms, attractor networks, and related research. Section~\ref{sec:methodology} presents the methodology, which explains the process of simulating robot trajectories within city road networks, optimizing network performance using genetic algorithms, and utilizing the multiscale network architecture with attractor dynamics. The results of extensive experiments using both the new simulator and Kitti are presented in Sections~\ref{sec:experimentalsetup} and~\ref{sec:results}, where we quantitatively compare a single-scale baseline to the proposed multiscale system. Finally, Section~\ref{sec:conclusion} concludes the paper with recommendations for future work.
