
\section{Related Works}
\label{sec:relatedworks}

%
Section~\ref{subsec:navigationmechanisms} introduces spatial cells in the hippocampus, including place cells, head direction cells, grid cells, and border cells, that play a critical role in mammalian navigation.Section~\ref{subsec:theoreticmodels} discusses the development of several computational models that integrate linear and angular velocity cues for path integration while mimicking the characteristics of spatial cells. Section~\ref{subsec:bioinspiredrobotnav} reviews bio-inspired robotic systems that emulate the behaviour of spatial cells within real environments using robotic hardware. Section~\ref{subsec:tuningcans} then presents methods for tuning CANs using optimization techniques, and finally, Section~\ref{subsec:simsfornavigation} discusses simulators for robot navigation.

\subsection{Navigation Mechanisms in the Brain}
\label{subsec:navigationmechanisms}
Mammalian navigation is a complex process that involves various spatial cells in the hippocampus, such as place cells, head direction cells, and grid cells. Place cells \cite{o1976place} are neurons that fire at unique spatial locations, representing places within an environment. Head direction cells \cite{taube1998head} encode an animal's orientation and provide information about its heading. Grid cells \cite{hafting2005microstructure,moser2014grid,banino2018vector} use a tessellated grid pattern to integrate direction and speed, enabling efficient encoding of large environments \cite{grieves2017representation}. Border cells and object vector cells are additional spatial cells that provide information about the boundaries within an environment along with the distance and direction of objects within it \cite{hoydal2019object, bordercell}. These biological mechanisms offer an alternative to classical robotic Simultaneous Localization And Mapping (SLAM) algorithms by integrating sensory visual and motion cues to update estimates of a location while building a cognitive map of the environment. 

\subsection{Theoretic Computational Neuroscience Models}
\label{subsec:theoreticmodels}
The discovery of spatial cells led to the development of several computational models that can integrate linear and angular velocity cues for path integration while mimicking the characteristics of spatial cells. Attractor networks were proposed to model head direction~\cite{skaggs1994model} using neurons in a ring structure with a stable bump of activity, which shifts based on angular velocity inputs. They were also used to model grid cells with 2D CAN models that combine head direction and speed for shifting activity bumps with toroidal connections at the boundary~\cite{fyhn2004spatial,mcnaughton2006path,burak2009accurate}. %
%
%

Early works \cite{zhang1996representation,samsonovich1997path,widloski2014model} implemented attractor models with unsupervised learning, while supervised models \cite{hahnloser2000permitted,arnold1991learning} incorporated learning rules or error signal feedback. Other models \cite{banino2018vector,kanitscheider2017emergence,seung1997learning} have used backpropagation and architecture constraints to form computational navigation models. 

Although these theoretical models cannot fully replicate the complexity and nuance of how the brain solves navigational problems, they offer valuable insights into the computations that the brain employs for integration, error correction, and learning \cite{khona2022attractor}. 

\subsection{Bio-inspired Robot Navigation}
\label{subsec:bioinspiredrobotnav}
The discovery of navigational mechanisms in the brain also resulted in algorithms that emulate the behaviour of spatial cells in physical robotic hardware, demonstrating navigational capabilities through bio-inspiration. Early works include \cite{gaussier2002view,cuperlier2007neurobiologically,milford2010persistent} which rely on place cells along with curated mechanisms like ``view cells'', ``transition cells'' and ``pose cells'' based on standard robotics principles. %
Extensions of these works include NeuroSLAM~\cite{yu2019neuroslam} which extends RatSLAM~\cite{milford2010persistent} to a 3D space and \cite{li2022brain} which corrects absolute heading using polarization of light as seen in Desert Ants. 

Grid cell mechanisms have also been the basis for numerous multiscale systems in robotics such as the large-scale aerial mapping system developed by Hausler et al.~\cite{hausler2020bio}, a grid-cell inspired place recognition system that utilizes homogeneous maps at varying scales \cite{chen2015bio}, and a system for multiscale path integration in 3D for unmanned aerial vehicles (UAVs) \cite{yang2021path}. These systems demonstrate the potential of grid cell-based models for solving complex tasks in different domains. 


%
%


%
%
\subsection{Tuning CANs}
\label{subsec:tuningcans}
Existing methods for tuning CANs have relied on either optimization techniques or hand-crafted network parameters to achieve stable activity that integrates input signals accurately. In early works, DeGris et al.~\cite{degris2004spiking} used a Genetic Algorithm to fine-tune 1D CAN networks in a spiking neuron model. Dall'Osto et al.~\cite{dall2018automatic} proposed an automatic calibration method utilizing global optimization methods, while Menezes et al.~\cite{menezes2020automatic} presented an iterative closest point algorithm for automatic tuning of RatSLAM. More recently, Fox et al.~\cite{fox2022new} proposed a new evolutionary dynamic optimization method for shifting attractor peaks, along with a benchmark suite. However, with the growth of network complexity in scale and dimension, optimizing additional parameters becomes increasingly challenging.

%
%
\subsection{Simulators for Robot Navigation}
\label{subsec:simsfornavigation}
Simulations play a crucial role in enabling rapid experimentation and development of any system. OpenStreetMap (OSM) is a crowd-sourced map that includes road network information obtained from portable GPS devices, and it has been used in several systems. For example, Brubaker et al.~\cite{brubaker2013lost} performed self-localization using OSM maps and visual odometry. Fleischmann et al.~\cite{fleischmann2017using} used OSM paths to assess the quality of GNSS signals during navigation. Li et al.~\cite{li2021openstreetmap} combined OSM road network paths with onboard sensors for path tracking. By integrating OSM maps with other data sources, these systems showcase the versatility and potential of OSM in various applications and motivated the simulation developed within this work.

%
%



