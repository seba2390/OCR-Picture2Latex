\section{Experimental Setup}
\label{sec:experimentalsetup}
\subsection{Implementation Details}
The MCAN trajectory tracking system was implemented in Python3 using standard libraries. The head direction network consists of 360 neurons that were tuned using a genetic algorithm to accurately integrate angular velocity inputs to produce an estimate of agent orientation. The 2D networks in the multiscale system was generated with 100$\times$100 neurons and 4 scales with scale ratios incremented by a factor of 4, i.e., $(0.25, 1, 4, 16)$. This was suitable for the desired operating range of 0-20 m/s (0-72 km/h). For a fair comparison, the single-scale CAN was implemented with 200x200 neurons, so both systems had a total of 40000 neurons.  

The CANs were tuned using 24 genomes mutated and evaluated for 20 generations with 14 parallel processes for fitness evaluation. Based on the network size, the parameter ranges were set to $A \in [1, 10]$, $E \in [1, 10]$, $\gamma \in [0, 1]$, and $\phi \in [0.00001, 0.005]$.

\subsection{Datasets}
The performance of our system was evaluated on the City scale navigation simulator (Section~\ref{subsec:simulator}) and the Kitti Odometry dataset~\cite{geiger2012we}, which is commonly used to benchmark robotic systems. The navigation simulator was used to generate trajectories based on road networks within a 10 km $\times$ 10 km region from Tokyo, Berlin, Brisbane, and New York. These road maps were converted into occupancy grids, and a path-finding distance transform algorithm from the Robotics Toolbox for Python \cite{corke2021not} was used to find an optimal route between two points on the road map. Our city scale simulator was used to generate paths through Tokyo,  Brisbane, Berlin and New York, covering distances of 38 km, 70 km, 167 km, and 63 km, respectively. Using both the simulator and the Kitti Odometry dataset, we evaluated the proposed system in a variety of realistic scenarios.



\subsection{Evaluation Metrics}
Our networks were evaluated using the Absolute Trajectory Error (ATE), which is a common metric used in trajectory estimation systems~\cite{zhang2018tutorial}. ATE is defined as the Root Mean Square Error (RMSE) between the estimated and desired trajectory after alignment. In order to ensure a fair comparison between different datasets, ATE was averaged over the total distance, resulting in ATE/meter. While we used ATE to evaluate the performance of our networks, we utilized the Sum of Absolute Differences (SAD) method~\cite{milford2012seqslam} during the tuning process to measure the fitness of the networks.

%
%
%

%
%
%




