\begin{abstract}

Animals and insects showcase remarkably robust and adept navigational abilities, up to literally circumnavigating the globe. Primary progress in robotics inspired by these natural systems has occurred in two areas: highly theoretical computational neuroscience models, and handcrafted systems like RatSLAM and NeuroSLAM. In this research, we present work bridging the gap between the two, in the form of Multiscale Continuous Attractor Networks (MCAN), that combine the multiscale parallel spatial neural networks of the previous theoretical models with the real-world robustness of the robot-targeted systems, to enable trajectory tracking over large velocity ranges. To overcome the limitations of the reliance of previous systems on hand-tuned parameters, we present a genetic algorithm-based approach for automated tuning of these networks, substantially improving their usability. To provide challenging navigational scale ranges, we open source a flexible city-scale navigation simulator that adapts to any street network, enabling high throughput experimentation\footnote{\label{GitHubRepo} \url{https://github.com/theresejoseph/Trajectory_Tracking_via_MCAN/}}. In extensive experiments using the city-scale navigation environment and Kitti, we show that the system is capable of stable dead reckoning over a wide range of velocities and environmental scales, where a single-scale approach fails.



%


\end{abstract}

%
%
%
%
%
%
%