
\section{Results and Discussion}
\label{sec:results}
This section provides a comparison of single-scale CAN versus the proposed multiscale system.  We evaluate the performance of our system on multiple trajectories generated from simulation and existing datasets in Section~\ref{subsec:Performance}. This is followed by an analysis of the tuning mechanism in Section~\ref{subsec:GA}.

\begin{table}[t]
\vspace{2mm}
\caption{Comparison of ATE per meter}
\label{tab:atecomparison}
\renewcommand{\arraystretch}{1.2}
\setlength{\tabcolsep}{2.5pt}
\label{T_results}
\centering
\begin{tabular}{l|cc}

%
%
%
%
%

%
%
%
%
%


Dataset & Single-scale & Multiscale\\\hline
%
 Tokyo (CSN simulation)& 1.093 $\pm$ 0.110 & \textbf{0.068 $\pm$ 0.010}\\

 New York (CSN simulation)& 0.893 $\pm$ 0.137& \textbf{0.070 $\pm$ 0.028}\\

 Brisbane (CSN simulation)& 0.934 $\pm$ 0.102 & \textbf{0.070 $\pm$ 0.019}\\

 Berlin (CSN simulation)& 0.896 $\pm$ 0.166& \textbf{0.046 $\pm$ 0.021}\\

 Kitti (odometry)& 0.136 $\pm$ 0.138 & \textbf{0.041 $\pm$ 0.026}\\
\end{tabular}
\vspace{-0.2cm}
\end{table}

\begin{figure}[t]
\vspace{2mm}
     \centering
     \includegraphics[width=0.99\linewidth]{NewFigures/2_KittiSinglevsMulti_0.pdf}  
    \caption{Comparison of trajectory tracking (in meters) between single-scale and multiscale CAN on the Kitti dataset plotted against ground truth, where the single-scale CAN performance degrades as velocity increases.}
    \label{fig:5.6}
\end{figure}

\begin{figure}[t]
     \centering
     \includegraphics[width=0.8\linewidth]{NewFigures/3_MultivsSingleErrors_Path0.pdf} 
    \caption{A Comparison between multiscale and single-scale attractor networks models on Berlin City Scale Navigation Simulation dataset. The two networks are tested on 20 trajectories with increasing velocity ranges to evaluate the invariance to velocity within the multiscale network.}
    \label{fig:5.2}
\end{figure}

\begin{figure}[t]
\vspace{3.7mm}
    \centering
    \includegraphics[width=0.99\linewidth]
    {NewFigures/4_MapAllCities.pdf}  
    \caption{Maps generated from the City Scale Navigation Simulation used to generate test trajectories: Brisbane, Berlin, Tokyo and New York, with colours representing the road speeds in each city. For example, the highways in Brisbane are shown in orange while the residential roads are in purple.}
    \label{fig:5.1}
\end{figure}



\begin{figure}[t]
\vspace{2mm}
     \centering
     \includegraphics[width=0.99\linewidth]{NewFigures/5_BerlinTestingTrackswithSpeeds0to20_Multiscale.pdf} 
    \caption{Trajectory tracking with multiscale networks across 18 tracks from the Berlin City Scale Navigation Simulation Dataset with a total distance of 167km. The tracks, sorted based on tracking error, show varied and complex trajectories that emulate Berlin road networks.}
    \label{fig:5.5}
\end{figure}




\begin{figure}[t]
     \centering     \includegraphics[width=0.85\linewidth]{NewFigures/6_CumalitiveError_Path1_SinglevsMulti.pdf}
    \caption{ATE error over time for single-scale and multiscale networks integrating velocities ranging from 0-20m/s for a single trajectory from the Berlin City Scale Navigation Simulation dataset. Note the scale difference between plots.}
    \label{fig:5.7}
\end{figure}


\begin{figure}[t]   
\vspace{2mm}
    \centering
    \includegraphics[width=0.8\linewidth]{NewFigures/7_LocalSegmentError_AllPaths_SinglevsMulti.pdf}  
    \caption{Average Trajectory Error (ATE) within local segments of simulated trajectories through Berlin. The trajectories are realigned after each segment so the error doesn't accumulate across trajectories. The multiscale error ranges from 0-500m; whereas, the single-scale ranges from 0-11000m.}
    \label{fig:5.8}
\end{figure}


\begin{figure}[t]
     \centering
     \includegraphics[width=0.85\linewidth]{NewFigures/8_1D_2D_NetworkTuning.pdf}  
    \caption{Fitness evolution of the head direction network [left] and Multiscale Network [right] over 20 generations. Here the fitness is displayed as -SAD which approaches 0 over generations, as the algorithm converges.}
    \label{fig:5.3}
\end{figure}

%
%
%
%
%
%











