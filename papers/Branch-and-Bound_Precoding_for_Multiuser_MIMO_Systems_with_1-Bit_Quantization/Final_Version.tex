

 

%% bare_conf.tex
%% V1.3
%% 2007/01/11
%% by Michael Shell
%% See:
%% http://www.michaelshell.org/
%% for current contact information.
%%
%% This is a skeleton file demonstrating the use of IEEEtran.cls
%% (requires IEEEtran.cls version 1.7 or later) with an IEEE conference paper.
%%
%% Support sites:
%% http://www.michaelshell.org/tex/ieeetran/
%% http://www.ctan.org/tex-archive/macros/latex/contrib/IEEEtran/
%% and
%% http://www.ieee.org/

%%*************************************************************************
%% Legal Notice:
%% This code is offered as-is without any warranty either expressed or
%% implied; without even the implied warranty of MERCHANTABILITY or
%% FITNESS FOR A PARTICULAR PURPOSE! 
%% User assumes all risk.
%% In no event shall IEEE or any contributor to this code be liable for
%% any damages or losses, including, but not limited to, incidental,
%% consequential, or any other damages, resulting from the use or misuse
%% of any information contained here.
%%
%% All comments are the opinions of their respective authors and are not
%% necessarily endorsed by the IEEE.
%%
%% This work is distributed under the LaTeX Project Public License (LPPL)
%% ( http://www.latex-project.org/ ) version 1.3, and may be freely used,
%% distributed and modified. A copy of the LPPL, version 1.3, is included
%% in the base LaTeX documentation of all distributions of LaTeX released
%% 2003/12/01 or later.
%% Retain all contribution notices and credits.
%% ** Modified files should be clearly indicated as such, including  **
%% ** renaming them and changing author support contact information. **
%%
%% File list of work: IEEEtran.cls, IEEEtran_HOWTO.pdf, bare_adv.tex,
%%                    bare_conf.tex, bare_jrnl.tex, bare_jrnl_compsoc.tex
%%*************************************************************************

% *** Authors should verify (and, if needed, correct) their LaTeX system  ***
% *** with the testflow diagnostic prior to trusting their LaTeX platform ***
% *** with production work. IEEE's font choices can trigger bugs that do  ***
% *** not appear when using other class files.                            ***
% The testflow support page is at:
% http://www.michaelshell.org/tex/testflow/



% Note that the a4paper option is mainly intended so that authors in
% countries using A4 can easily print to A4 and see how their papers will
% look in print - the typesetting of the document will not typically be
% affected with changes in paper size (but the bottom and side margins will).
% Use the testflow package mentioned above to verify correct handling of
% both paper sizes by the user's LaTeX system.
%
% Also note that the "draftcls" or "draftclsnofoot", not "draft", option
% should be used if it is desired that the figures are to be displayed in
% draft mode.
%
%\documentclass[journal,comsoc,onecolumn,draftcls,twoside,12pt]{IEEEtran}
\documentclass[journal,comsoc]{IEEEtran}

%\documentclass[journal,comsoc,twocolumn,twoside,10pt]{IEEEtran}

%\documentclass[conference]{IEEEtran}
% Add the compsoc option for Computer Society conferences.
%
% If IEEEtran.cls has not been installed into the LaTeX system files,
% manually specify the path to it like:
% \documentclass[conference]{../sty/IEEEtran}





% Some very useful LaTeX packages include:
% (uncomment the ones you want to load)


% *** MISC UTILITY PACKAGES ***
%
%\usepackage{ifpdf}
% Heiko Oberdiek's ifpdf.sty is very useful if you need conditional
% compilation based on whether the output is pdf or dvi.
% usage:
% \ifpdf
%   % pdf code
% \else
%   % dvi code
% \fi
% The latest version of ifpdf.sty can be obtained from:
% http://www.ctan.org/tex-archive/macros/latex/contrib/oberdiek/
% Also, note that IEEEtran.cls V1.7 and later provides a builtin
% \ifCLASSINFOpdf conditional that works the same way.
% When switching from latex to pdflatex and vice-versa, the compiler may
% have to be run twice to clear warning/error messages.






% *** CITATION PACKAGES ***
%
%\usepackage{cite}
% cite.sty was written by Donald Arseneau
% V1.6 and later of IEEEtran pre-defines the format of the cite.sty package
% \cite{} output to follow that of IEEE. Loading the cite package will
% result in citation numbers being automatically sorted and properly
% "compressed/ranged". e.g., [1], [9], [2], [7], [5], [6] without using
% cite.sty will become [1], [2], [5]--[7], [9] using cite.sty. cite.sty's
% \cite will automatically add leading space, if needed. Use cite.sty's
% noadjust option (cite.sty V3.8 and later) if you want to turn this off.
% cite.sty is already installed on most LaTeX systems. Be sure and use
% version 4.0 (2003-05-27) and later if using hyperref.sty. cite.sty does
% not currently provide for hyperlinked citations.
% The latest version can be obtained at:
% http://www.ctan.org/tex-archive/macros/latex/contrib/cite/
% The documentation is contained in the cite.sty file itself.






% *** GRAPHICS RELATED PACKAGES ***
%
\ifCLASSINFOpdf
  % \usepackage[pdftex]{graphicx}
  % declare the path(s) where your graphic files are
  % \graphicspath{{../pdf/}{../jpeg/}}
  % and their extensions so you won't have to specify these with
  % every instance of \includegraphics
  % \DeclareGraphicsExtensions{.pdf,.jpeg,.png}
\else
  % or other class option (dvipsone, dvipdf, if not using dvips). graphicx
  % will default to the driver specified in the system graphics.cfg if no
  % driver is specified.
  % \usepackage[dvips]{graphicx}
  % declare the path(s) where your graphic files are
  % \graphicspath{{../eps/}}
  % and their extensions so you won't have to specify these with
  % every instance of \includegraphics
  % \DeclareGraphicsExtensions{.eps}
\fi
% graphicx was written by David Carlisle and Sebastian Rahtz. It is
% required if you want graphics, photos, etc. graphicx.sty is already
% installed on most LaTeX systems. The latest version and documentation can
% be obtained at: 
% http://www.ctan.org/tex-archive/macros/latex/required/graphics/
% Another good source of documentation is "Using Imported Graphics in
% LaTeX2e" by Keith Reckdahl which can be found as epslatex.ps or
% epslatex.pdf at: http://www.ctan.org/tex-archive/info/
%
% latex, and pdflatex in dvi mode, support graphics in encapsulated
% postscript (.eps) format. pdflatex in pdf mode supports graphics
% in .pdf, .jpeg, .png and .mps (metapost) formats. Users should ensure
% that all non-photo figures use a vector format (.eps, .pdf, .mps) and
% not a bitmapped formats (.jpeg, .png). IEEE frowns on bitmapped formats
% which can result in "jaggedy"/blurry rendering of lines and letters as
% well as large increases in file sizes.
%
% You can find documentation about the pdfTeX application at:
% http://www.tug.org/applications/pdftex





% *** MATH PACKAGES ***
%
%\usepackage[cmex10]{amsmath}
% A popular package from the American Mathematical Society that provides
% many useful and powerful commands for dealing with mathematics. If using
% it, be sure to load this package with the cmex10 option to ensure that
% only type 1 fonts will utilized at all point sizes. Without this option,
% it is possible that some math symbols, particularly those within
% footnotes, will be rendered in bitmap form which will result in a
% document that can not be IEEE Xplore compliant!
%
% Also, note that the amsmath package sets \interdisplaylinepenalty to 10000
% thus preventing page breaks from occurring within multiline equations. Use:
%\interdisplaylinepenalty=2500
% after loading amsmath to restore such page breaks as IEEEtran.cls normally
% does. amsmath.sty is already installed on most LaTeX systems. The latest
% version and documentation can be obtained at:
% http://www.ctan.org/tex-archive/macros/latex/required/amslatex/math/





% *** SPECIALIZED LIST PACKAGES ***
%
%\usepackage{algorithmic}
% algorithmic.sty was written by Peter Williams and Rogerio Brito.
% This package provides an algorithmic environment fo describing algorithms.
% You can use the algorithmic environment in-text or within a figure
% environment to provide for a floating algorithm. Do NOT use the algorithm
% floating environment provided by algorithm.sty (by the same authors) or
% algorithm2e.sty (by Christophe Fiorio) as IEEE does not use dedicated
% algorithm float types and packages that provide these will not provide
% correct IEEE style captions. The latest version and documentation of
% algorithmic.sty can be obtained at:
% http://www.ctan.org/tex-archive/macros/latex/contrib/algorithms/
% There is also a support site at:
% http://algorithms.berlios.de/index.html
% Also of interest may be the (relatively newer and more customizable)
% algorithmicx.sty package by Szasz Janos:
% http://www.ctan.org/tex-archive/macros/latex/contrib/algorithmicx/




% *** ALIGNMENT PACKAGES ***
%
%\usepackage{array}
% Frank Mittelbach's and David Carlisle's array.sty patches and improves
% the standard LaTeX2e array and tabular environments to provide better
% appearance and additional user controls. As the default LaTeX2e table
% generation code is lacking to the point of almost being broken with
% respect to the quality of the end results, all users are strongly
% advised to use an enhanced (at the very least that provided by array.sty)
% set of table tools. array.sty is already installed on most systems. The
% latest version and documentation can be obtained at:
% http://www.ctan.org/tex-archive/macros/latex/required/tools/


%\usepackage{mdwmath}
%\usepackage{mdwtab}
% Also highly recommended is Mark Wooding's extremely powerful MDW tools,
% especially mdwmath.sty and mdwtab.sty which are used to format equations
% and tables, respectively. The MDWtools set is already installed on most
% LaTeX systems. The lastest version and documentation is available at:
% http://www.ctan.org/tex-archive/macros/latex/contrib/mdwtools/


% IEEEtran contains the IEEEeqnarray family of commands that can be used to
% generate multiline equations as well as matrices, tables, etc., of high
% quality.


%\usepackage{eqparbox}
% Also of notable interest is Scott Pakin's eqparbox package for creating
% (automatically sized) equal width boxes - aka "natural width parboxes".
% Available at:
% http://www.ctan.org/tex-archive/macros/latex/contrib/eqparbox/


\usepackage[cmex10]{amsmath}
\usepackage{amssymb}
\usepackage{siunitx}
\usepackage{caption}
\captionsetup{font=footnotesize}
\usepackage{algpseudocode}
\usepackage{algorithm}
\algnewcommand{\Initialization}[1]{%
  \State \textbf{initialization:}
  \Statex \hspace*{\algorithmicindent}\parbox[t]{.8\linewidth}{\raggedright #1}
}

\algnewcommand{\Rep}[1]{%
  \State \textbf{repeat:}
  \Statex \hspace*{\algorithmicindent}\parbox[t]{.8\linewidth}{\raggedright #1}
}




\usepackage{tikz}
\usetikzlibrary{shapes,arrows,fit,positioning,calc}
\usetikzlibrary{plotmarks}
\usetikzlibrary{decorations.pathreplacing}
\usetikzlibrary{patterns}
\usetikzlibrary{arrows,automata}

\usepackage{pgfplots}
\pgfplotsset{compat=newest}


% *** SUBFIGURE PACKAGES ***
%\usepackage[tight,footnotesize]{subfigure}
% subfigure.sty was written by Steven Douglas Cochran. This package makes it
% easy to put subfigures in your figures. e.g., "Figure 1a and 1b". For IEEE
% work, it is a good idea to load it with the tight package option to reduce
% the amount of white space around the subfigures. subfigure.sty is already
% installed on most LaTeX systems. The latest version and documentation can
% be obtained at:
% http://www.ctan.org/tex-archive/obsolete/macros/latex/contrib/subfigure/
% subfigure.sty has been superceeded by subfig.sty.



%\usepackage[caption=false]{caption}
%\usepackage[font=footnotesize]{subfig}
% subfig.sty, also written by Steven Douglas Cochran, is the modern
% replacement for subfigure.sty. However, subfig.sty requires and
% automatically loads Axel Sommerfeldt's caption.sty which will override
% IEEEtran.cls handling of captions and this will result in nonIEEE style
% figure/table captions. To prevent this problem, be sure and preload
% caption.sty with its "caption=false" package option. This is will preserve
% IEEEtran.cls handing of captions. Version 1.3 (2005/06/28) and later 
% (recommended due to many improvements over 1.2) of subfig.sty supports
% the caption=false option directly:
%\usepackage[caption=false,font=footnotesize]{subfig}
%
% The latest version and documentation can be obtained at:
% http://www.ctan.org/tex-archive/macros/latex/contrib/subfig/
% The latest version and documentation of caption.sty can be obtained at:
% http://www.ctan.org/tex-archive/macros/latex/contrib/caption/




% *** FLOAT PACKAGES ***
%
%\usepackage{fixltx2e}
% fixltx2e, the successor to the earlier fix2col.sty, was written by
% Frank Mittelbach and David Carlisle. This package corrects a few problems
% in the LaTeX2e kernel, the most notable of which is that in current
% LaTeX2e releases, the ordering of single and double column floats is not
% guaranteed to be preserved. Thus, an unpatched LaTeX2e can allow a
% single column figure to be placed prior to an earlier double column
% figure. The latest version and documentation can be found at:
% http://www.ctan.org/tex-archive/macros/latex/base/



%\usepackage{stfloats}
% stfloats.sty was written by Sigitas Tolusis. This package gives LaTeX2e
% the ability to do double column floats at the bottom of the page as well
% as the top. (e.g., "\begin{figure*}[!b]" is not normally possible in
% LaTeX2e). It also provides a command:
%\fnbelowfloat
% to enable the placement of footnotes below bottom floats (the standard
% LaTeX2e kernel puts them above bottom floats). This is an invasive package
% which rewrites many portions of the LaTeX2e float routines. It may not work
% with other packages that modify the LaTeX2e float routines. The latest
% version and documentation can be obtained at:
% http://www.ctan.org/tex-archive/macros/latex/contrib/sttools/
% Documentation is contained in the stfloats.sty comments as well as in the
% presfull.pdf file. Do not use the stfloats baselinefloat ability as IEEE
% does not allow \baselineskip to stretch. Authors submitting work to the
% IEEE should note that IEEE rarely uses double column equations and
% that authors should try to avoid such use. Do not be tempted to use the
% cuted.sty or midfloat.sty packages (also by Sigitas Tolusis) as IEEE does
% not format its papers in such ways.





% *** PDF, URL AND HYPERLINK PACKAGES ***
%
%\usepackage{url}
% url.sty was written by Donald Arseneau. It provides better support for
% handling and breaking URLs. url.sty is already installed on most LaTeX
% systems. The latest version can be obtained at:
% http://www.ctan.org/tex-archive/macros/latex/contrib/misc/
% Read the url.sty source comments for usage information. Basically,
% \url{my_url_here}.





% *** Do not adjust lengths that control margins, column widths, etc. ***
% *** Do not use packages that alter fonts (such as pslatex).         ***
% There should be no need to do such things with IEEEtran.cls V1.6 and later.
% (Unless specifically asked to do so by the journal or conference you plan
% to submit to, of course. )



\newcommand\blfootnote[1]{%
  \begingroup
  \renewcommand\thefootnote{}\footnote{#1}%
  \addtocounter{footnote}{-1}%
  \endgroup
}




% correct bad hyphenation here
\hyphenation{op-tical net-works semi-conduc-tor}


\begin{document}


%
% paper title
% can use linebreaks \\ within to get better formatting as desired
\title{Branch-and-Bound Precoding for Multiuser MIMO Systems with 1-Bit Quantization}

\author{Lukas~T.~N.~Landau,~\IEEEmembership{Member,~IEEE,}
        and~Rodrigo C.\ de Lamare,~\IEEEmembership{Senior~Member,~IEEE}\vspace{-1em}				
% <-this % stops a space
\thanks{The authors would like to thank Johannes Israel from the Institute of Numerical Mathematics, TU Dresden for the introduction of the branch-and-bound method and for verifying the notation.}
\thanks{The authors are with Centro de Estudos em Telecomunica\c{c}\~{o}es Pontif\'{i}cia Universidade Cat\'{o}lica do Rio de Janeiro, Rio de Janeiro CEP 22453-900, Brazil, (email: \{lukas.landau, delamare\}@cetuc.puc-rio.br).} }
%\thanks{L.\ Landau was with the Technische Universit\"at Dresden, Vodafone Chair Mobile Communications Systems and SFB 912 - HAEC, Dresden 01062, Germany.
%He is now with the Pontif\'{i}cia Universidade Cat\'{o}lica do Rio de Janeiro, Centro de Estudos em Telecomunica\c{c}\~{o}es, Rio de Janeiro CEP 22453-900, Brazil, (e-mail: lukas.landau@cetuc.puc-rio.br). }
%\thanks{M. D\"{o}rpinghaus and G. Fettweis are with the Technische Universit\"at Dresden, Vodafone Chair Mobile Communications Systems and SFB 912 - HAEC,
%Dresden 01062, Germany (e-mail: meik.doerpinghaus@tu-dresden.de; gerhard.fettweis@tu-dresden.de).}% <-this % stops a space
%\thanks{Manuscript received May 30, 2016; revised -, -.}



% author names and affiliations
% use a multiple column layout for up to three different
% affiliations


% conference papers do not typically use \thanks and this command
% is locked out in conference mode. If really needed, such as for
% the acknowledgment of grants, issue a \IEEEoverridecommandlockouts
% after \documentclass

% for over three affiliations, or if they all won't fit within the width
% of the page, use this alternative format:
% 
%\author{\IEEEauthorblockN{Michael Shell\IEEEauthorrefmark{1},
%Homer Simpson\IEEEauthorrefmark{2},
%James Kirk\IEEEauthorrefmark{3}, 
%Montgomery Scott\IEEEauthorrefmark{3} and
%Eldon Tyrell\IEEEauthorrefmark{4}}
%\IEEEauthorblockA{\IEEEauthorrefmark{1}School of Electrical and Computer Engineering\\
%Georgia Institute of Technology,
%Atlanta, Georgia 30332--0250\\ Email: see http://www.michaelshell.org/contact.html}
%\IEEEauthorblockA{\IEEEauthorrefmark{2}Twentieth Century Fox, Springfield, USA\\
%Email: homer@thesimpsons.com}
%\IEEEauthorblockA{\IEEEauthorrefmark{3}Starfleet Academy, San Francisco, California 96678-2391\\
%Telephone: (800) 555--1212, Fax: (888) 555--1212}
%\IEEEauthorblockA{\IEEEauthorrefmark{4}Tyrell Inc., 123 Replicant Street, Los Angeles, California 90210--4321}}




% use for special paper notices
%\IEEEspecialpapernotice{(Invited Paper)}




% make the title area
\maketitle

%\vspace{-0.5em}


\begin{abstract}
Multiple-antenna systems is a key technique to serve multiple users in future wireless systems.
For low energy consumption and hardware complexity we first consider transmit symbols with constant magnitude and then 1-bit digital-to-analog converters. We propose precoding designs which maximize the minimum distance to the decision threshold at the receiver.
The precoding design with 1-bit DAC corresponds to a discrete optimization problem, which we solve exactly with a branch-and-bound strategy. We alternatively present an approximation based on relaxation. Our results show that the proposed branch-and-bound approach has polynomial complexity. The proposed methods outperform existing precoding methods with 1-bit DAC in terms of uncoded bit error rate and sum-rate. The performance loss in comparison to infinite DAC resolution is small.
\end{abstract}
%\vspace{-1em}
\begin{IEEEkeywords}
Precoding, 1-bit quantization, MIMO systems, branch-and-bound methods.
\end{IEEEkeywords}


% IEEEtran.cls defaults to using nonbold math in the Abstract.
% This preserves the distinction between vectors and scalars. However,
% if the conference you are submitting to favors bold math in the abstract,
% then you can use LaTeX's standard command \boldmath at the very start
% of the abstract to achieve this. Many IEEE journals/conferences frown on
% math in the abstract anyway.

% no keywords




% For peer review papers, you can put extra information on the cover
% page as needed:
% \ifCLASSOPTIONpeerreview
% \begin{center} \bfseries EDICS Category: 3-BBND \end{center}
% \fi
%
% For peerreview papers, this IEEEtran command inserts a page break and
% creates the second title. It will be ignored for other modes.
%\IEEEpeerreviewmaketitle


\vspace{-1em}
\section{Introduction}
Low peak-to-average ratio is essential for the use of cheap and efficient power amplifiers, which are key in
multiuser multiple-input multiple-output (MIMO) communication systems \cite{Spencer_2004}.
Especially in short range wireless communications with a low path loss, also the converters become an important factor of system cost.
It is known that digital-to-analog converters (DACs) have a lower energy consumption in comparison to analog-to-digital converters (ADCs) with the same clock speed and resolution. For example, when considering the converter pair presented in \cite{Naber_1989} and assuming that the energy consumption of a DAC or ADC doubles with every additional bit of resolution, the DAC consumes only $30\%$ of the energy consumption of the ADC with the same parameters. 
For this reason, the DAC is often neglected in the optimization of communications systems.
However, when considering the recent trends in multiuser MIMO communication, the number of antennas at the base station (BS) is commonly larger than the total number of receive antennas, which guides our attention to the DACs at the BS.
In this regard, DACs with 1-bit resolution are promising, where we assume that the pulse shaping can be efficiently performed in the analog domain.    
For some applications, e.g., internet of things, where the receivers should have a low-complexity and limited energy budget it is reasonable to also consider 1-bit quantization at the receivers.
In this context, MIMO communication systems with 1-bit quantization have received increased attention, where the design of the precoder is a particular problem.
In this regard, linear precoding strategies \cite{Saxena_2016} and \cite{JacobssonDCGS16a}, such as maximal-ratio transmission and zero-forcing (ZF) methods, followed by quantization have been studied. At the same time a nonlinear approach has been reported in \cite{Jedda_2016}, where the precoding vector is obtained based on an optimization method. Another nonlinear precoder has been presented in \cite{Tirkonnen_2017} whose computational complexity is only linear with the number of antennas.

In this work, we develop a precoding design that maximizes the minimum distance to the decision thresholds at the receivers which is promising in terms of bit error rate (BER). First we propose a precoding design with constant magnitude transmit symbols, termed phase-only precoding (PoP), and then we consider 1-bit DACs, which then corresponds to a scaled version of an integer linear program. We optimally solve this non convex problem by a branch-and-bound method, which has been lately considered for discrete receive beamforming \cite{Israel_2015_Letter}.
The bounding step relies on a relaxed problem which is a linear program (LP). This relaxation is also used to approximate the optimal precoder as an alternative solution.

The paper is organized as follows: Section~\ref{sec:system_model} describes the system model, whereas Section~\ref{sec:PoP} describes the proposed precoding design with constant magnitude transmit symbols. In Section~\ref{sec:BB_precoding}, we describe the proposed precoding algorithms with 1-bit DACs. Section~\ref{sec:numerical_results} presents and discusses numerical results, while Section VI gives the conclusions.

We use the notation $P(\boldsymbol{y}_k \vert \boldsymbol{s}_k)=P(\mathbf{y}_k=\boldsymbol{y}_k \vert \mathbf{s}_k=\boldsymbol{s}_k)$, where the random quantities are upright letter and the realizations italic.

%
%Regarding the notation we want to highlight that the real and imaginary part operator are also applied to vectors and matrices, e.g., $\mathrm{Re}\left\{ \boldsymbol{x} \right\} = \left[ \mathrm{Re}\left\{ \left[\boldsymbol{x}\right]_1 \right\},\ldots,    \mathrm{Re}\left\{ \left[\boldsymbol{x}\right]_M \right\}  \right]^T$ and equivalently for $\mathrm{Im}\left\{\cdot\right\}$. 


\section{System Model}
\label{sec:system_model}
\begin{figure*}[h]
\begin{center}
\captionsetup{justification=centering}
\includegraphics{system_precoding.eps}
\caption{MIMO system model}
\label{fig:system_model}       % Give a unique label
\vspace{-0.75em}
\end{center}
\end{figure*}
A multiuser MIMO downlink is considered where the BS has $M$ transmit antennas which communicates with $K$ users, each having $L$ antennas.
The transmit vector is denoted by $\mathbf{x}=[\mathrm{x}_1,\ldots,\mathrm{x}_M]^T$.
The vector of transmit symbols of length $K L$ is denoted by $\mathbf{s}=[\mathbf{s}_1^T,\ldots,\mathbf{s}_K^T]^T$ with $\mathbf{s}_k=[\mathrm{s}_{k,1},\ldots,\mathrm{s}_{k,L}]^T$ and $\mathrm{s}_{k,l} \in \left\{  1+j,1-j,-1+j,-1-j  \right\}$ for $k=1,\ldots,K$ and $l=1,\ldots,L$. It is assumed a frequency flat fading channel described with zero-mean independent and identically distributed (i.i.d.) complex Gaussian random variables $\mathrm{h}_{k,l,m}$, where $k$, $l$ and $m$ denote the index of the user, the receive antenna and the transmit antenna, respectively.
After matched filtering at the transmitter and the receiver the received sample at the $l$th antenna of the $k$th user is described by\looseness-1 
\begin{align}
\label{eq:received_sample}
\mathrm{z}_{k,l} & =  \mathrm{r}_{k,l} +  \mathrm{n}_{k,l}  = \sum_{m=1}^{M}     \mathrm{h}_{k,l,m } \ \mathrm{x}_m + \mathrm{n}_{k,l} \textrm{,}
\end{align}
where $\mathrm{n}_{k,l}$ is a zero-mean i.i.d.\ complex Gaussian random variable with variance $\sigma_n^2$ representing thermal noise.
The received signals are applied to hard decision detectors or equivalently 1-bit ADCs described by $\mathrm{y}_{k,l}=Q(\mathrm{z}_{k,l})= \mathrm{sgn}(\mathrm{Re}\left\{\mathrm{z}_{k,l}\right\}) +j \ \mathrm{sgn}(\mathrm{Im}\left\{\mathrm{z}_{k,l}\right\})$, such that ${\mathrm{y}}_{k,l} \in \left\{ 1+j,1-j,-1+j,-1-j \right\}$, where $Q(\cdot)$ denotes the 1-bit quantization. 
By using vector notation the received $K L$ samples can be expressed as
\begin{align}
\mathbf{y}=Q(\mathbf{z})= Q(\mathbf{r} + \mathbf{n}) =Q(  \mathbf{H} \mathbf{x} + \mathbf{n}) \textrm{,}
\end{align}
where $\mathbf{H}=[\mathbf{H}_1^T,\ldots,\mathbf{H}_K^T]$ is the channel matrix with dimensions $K L \times M$ which consists of the matrices $\mathbf{H}_k$ each having dimensions $L \times M$. The compact system model is illustrated in Fig.~\ref{fig:system_model}.       
The receive vector has the structure $\mathbf{y}=[\mathbf{y}_1^T,\ldots,\mathbf{y}_K^T]^T$ accordingly. 
The following describes, how to choose the transmit vector $\mathbf{x}$ such that transmit symbols $\mathbf{s}$ can be appropriately detected.
 
  



\section{Proposed Phase-only Precoding (PoP)}
\label{sec:PoP}
Instead of considering the BER or the achievable rate as the objective function, we consider a design criterion which has been proposed before in \cite{Landau_SCC2013} in the context of intersymbol interference, and which has been used later also in \cite{Mo_2015}, \cite{Gokceoglu_2016b}.
The design criterion implies that $\mathbf{x}$ is chosen such that the minimum distance to the decision threshold, denoted by $\epsilon$, is maximized, which provides robustness against perturbations of the received signals. In the proposed Phase-only Precoder (PoP) it is considered that each $x_m$ has constant magnitude.
By using instead the corresponding inequality, the optimization problem can be cast as: \looseness-1
\begin{align}
\label{eq:PoP}
\mathbf{x}_{\textrm{opt}} = &  \arg\min_{\mathbf{x}, \mathrm{\epsilon}}  -\mathrm{\epsilon} \\
& \textrm{s.t. }     \mathrm{Re} \left\{   \mathrm{diag} \left\{ \mathbf{s} \right\}  \right\}    \mathrm{Re} \left\{ \mathbf{H} \mathbf{x}  \right\}   \geq  \epsilon \boldsymbol{1}_{K L} \textrm{,} \notag \\
& \ \ \ \            \mathrm{Im} \left\{   \mathrm{diag} \left\{ \mathbf{s} \right\}  \right\}    \mathrm{Im} \left\{ \mathbf{H} \mathbf{x}  \right\}   \geq  \epsilon \boldsymbol{1}_{K L} \textrm{,} \notag \\
& \ \ \ \      \left|  \mathrm{x}_m \right| \leq   1/\sqrt{M}, \ \ \textrm{for } m=1,\ldots,M   \textrm{,}   \notag 
\end{align}
where the negation is applied to obtain a minimization problem.
If needed, the equality $\left|  \mathrm{x}_m \right| =   1/\sqrt{M}$ is subsequently introduced by scaling the entries of $\mathbf{x}_{\textrm{opt}}$.


\section{Proposed Precoding with 1-Bit DAC}
\label{sec:BB_precoding}
In this section, we propose a precoder designs using the objective from Section~\ref{sec:PoP} but with DACs having 1-bit resolution where we implicitly assume analog pulse shaping. With this, the optimization problem changes to
\begin{align}
\label{eq:optimization_problem_precoder}
\mathbf{x}_{\textrm{opt}} = &  \arg\min_{\mathbf{x}, \epsilon}  -\epsilon \\
& \textrm{s.t. }     \mathrm{Re} \left\{   \mathrm{diag} \left\{ \mathbf{s} \right\}  \right\}    \mathrm{Re} \left\{ \mathbf{H} \mathbf{x}  \right\}   \geq  \epsilon \boldsymbol{1}_{K L} \textrm{,} \notag \\
& \ \ \ \          \mathrm{Im} \left\{   \mathrm{diag} \left\{ \mathbf{s} \right\}  \right\}    \mathrm{Im} \left\{ \mathbf{H} \mathbf{x}  \right\}   \geq  \epsilon \boldsymbol{1}_{K L} \textrm{,} \notag \\
& \ \ \ \      \mathbf{x} \in \mathcal{P}^M \textrm{,}   \notag 
\end{align}
such that the elements of $\mathbf{x}$, termed $\mathrm{x}_m$ for $m=1,\ldots,M$ are taken from the set \newline $\mathcal{P}:= \left\{   e^{j \frac{\pi}{4}}/ \sqrt{M}, e^{j \frac{ 3\pi}{ 4 }}/ \sqrt{M}, e^{j \frac{ 5\pi}{ 4 }}/ \sqrt{M}, e^{j \frac{ 7\pi}{ 4 }}/ \sqrt{M}     \right\}$. Due to the constraint the optimization problem is non-convex.


\subsection{Proposed Approximate 1-Bit Precoder (1-Bit Approx.)}
\label{sec:lower-bounding}
The first step in the algorithm corresponds to a lower-bounding of the objective by relaxation.
Relaxing the input constraint in \eqref{eq:optimization_problem_precoder} yields the LP: 
\begin{align}
\label{eq:lower_bound}
\mathbf{x}_{\textrm{lb}}=&  \arg\min_{\mathbf{x}, \epsilon}  -\epsilon \\
& \textrm{s.t. }     \mathrm{Re} \left\{   \mathrm{diag} \left\{ \mathbf{s} \right\}  \right\}    \mathrm{Re} \left\{ \mathbf{H} \mathbf{x}  \right\}   \geq  \epsilon \boldsymbol{1}_{K L} \textrm{,} \notag \\
& \ \ \ \            \mathrm{Im} \left\{   \mathrm{diag} \left\{ \mathbf{s} \right\}  \right\}    \mathrm{Im} \left\{ \mathbf{H} \mathbf{x}  \right\}   \geq  \epsilon \boldsymbol{1}_{K L} \textrm{,} \notag \\
&\ \ \ \   \left| \mathrm{Re} \left\{{\mathrm{x}}_m \right\}\right| \leq  1/\sqrt{2M}  ,     \notag \\
& \ \ \ \   \left| \mathrm{Im} \left\{{\mathrm{x}}_m \right\}\right| \leq   1/\sqrt{2M} , \ \ \textrm{for } m=1,\ldots,M  \textrm{.}   \notag
\end{align}
The optimal value of \eqref{eq:lower_bound} is always smaller than or equal to the optimal value of \eqref{eq:optimization_problem_precoder}. A valid solution in terms of $\mathbf{x} \in \mathcal{P}^M$ can be obtained by mapping the solution of \eqref{eq:lower_bound} to the discrete input vector with the smallest Euclidean distance. This corresponding design is the proposed approximate 1-bit precoder. The resulting $-\epsilon$ is an upperbound of the optimal value of \eqref{eq:optimization_problem_precoder}.


\subsection{Concept of Branch-and-Bound Precoding}
We consider a precoder design in terms of a constrained minimization of an objective function $f(\mathbf{x}, \mathbf{s})$, given by
\begin{align}
\label{eq:original_problem}
\mathbf{x}_{\textrm{opt}} =& \arg\min_{\mathbf{x}} f(\mathbf{x}, \mathbf{s} ) \ \ \textrm{    s.t. } \mathbf{x} \in   \mathcal{P}^{M} \textrm{.}    
\end{align}
A lower bound on the optimal value for the objective function in \eqref{eq:original_problem} can be obtained by relaxing the problem, e.g., as done in \eqref{eq:lower_bound}.  
The solution of the relaxed problem is termed $\mathbf{x}_{\textrm{lb}}$.
 
An upper bound on the problem \eqref{eq:original_problem} is given by any valid solution which fulfills the restriction that the entries of the vector $\mathbf{x}$ are taken from the discrete set $\mathcal{P}^{M}$. The vector of the discrete set which has a minimum Euclidean distance to the optimal vector of the continuous problem is often suitable for computing an upper bound. In this regard, we denote the smallest known upper bound by $\check{ f } \geq  f(\mathbf{x}_{\textrm{opt}})$, where $\mathbf{x}_{\textrm{opt}}$ is the solution of \eqref{eq:original_problem}.
Now we consider that $d$ entries of $\mathbf{x}$ are fixed and taken out of the discrete set.
The precoding vector is then given by $\mathbf{x}=[\mathbf{x}_1^T, \mathbf{x}_2^T ]^T$, with $\mathbf{x}_1 \in \mathcal{P}^d $.
Based on that, a subproblem can be formulated by
\begin{align}
\label{eq:lb_subproblem}
\mathbf{x}_{2,\textrm{lb}} =& \arg\min_{\mathbf{x}_2} f(\mathbf{x}_2, \mathbf{x}_1, \mathbf{s} ) \\
&\textrm{s.t. }   \left| \mathrm{Re} \left\{   [\mathbf{x}_2]_m  \right\}\right| \leq  1/\sqrt{2M},     \notag \\
& \ \ \ \   \left| \mathrm{Im} \left\{   [\mathbf{x}_2]_m  \right\}\right| \leq  1/\sqrt{2M}, , \ \   \textrm{for }m=1\ldots{M-d}\textrm{.}  \notag  
 \end{align}
If the optimal value of \eqref{eq:lb_subproblem} is larger than a known upper bound $\check{ f }$ on the solution of \eqref{eq:original_problem} the fixed vector $\mathbf{x}_{1}$ and all its evolutions can be excluded from the possible candidates. The branch-and-bound method is efficient when there are many exclusions and the bounds can be computed with relatively low complexity. 


\subsection{Proposed 1-Bit Branch-and-Bound Algorithm (1-Bit B\&B)}
\label{sec:bb_algorithm_design}
In this section a branch-and-bound algorithm is proposed which solves \eqref{eq:optimization_problem_precoder} by employing \eqref{eq:lower_bound}, where a real-valued description is used.
The real-valued representations of the precoding vector and the transmit symbol vector are given by\looseness-1
\begin{align}								
\mathbf{x}_{\textrm{r}}					=			\begin{bmatrix} \mathrm{Re} \left\{\mathbf{x}\right\} \\  \mathrm{Im} \left\{\mathbf{x}\right\} \end{bmatrix} \textrm{,} \qquad
\mathbf{s}_{\textrm{r}}					=			\begin{bmatrix} \mathrm{Re} \left\{\mathbf{s}\right\} \\  \mathrm{Im} \left\{\mathbf{s}\right\} \end{bmatrix}  \textrm{,} 
\end{align}
and the real-valued notation of the channel matrix is given by
\begin{align}
\mathbf{H}_{\textrm{r}}    &=       \begin{bmatrix}    \mathrm{Re} \left\{  \mathbf{H}  \right\}  &  -\mathrm{Im} \left\{  \mathbf{H}  \right\} \\
								\mathrm{Im} \left\{  \mathbf{H}  \right\} &  \ \mathrm{Re} \left\{  \mathbf{H}  \right\} 
								\end{bmatrix} \textrm{,}
\end{align}
such that the real-valued noiseless received vector is $\mathbf{r}_{\textrm{r}} =\mathbf{H}_{\textrm{r}} \mathbf{x}_{\textrm{r}}$ and the received vector is $\mathbf{y}_{\textrm{r}}=[\mathrm{Re}\left\{\mathbf{y}\right\}^T \mathrm{Im}\left\{\mathbf{y}\right\}^T ]^T$ and equivalently for the $k$th user the specific notation $\mathbf{H}_{\textrm{r},k}$, $\mathbf{y}_{\textrm{r},k}$ and $\mathbf{s}_{\textrm{r},k}$.
With the vector of variables of the optimization problem described by $\mathbf{v}=[\mathbf{x}_{\textrm{r}}^T, \epsilon]^T$ the optimization problem can be written as:
\begin{align}
\label{eq:v_real}
\mathbf{v}_{\text{opt}}=& \arg\min_{\mathbf{v}}  \boldsymbol{a}^T \mathbf{v} \\
& \textrm{s.t. } \mathbf{A} \mathbf{v} \geq  \boldsymbol{0}_{2 K L},  \notag \\
&     \left|\left[\mathbf{v}\right]_{m}\right|   \in \mathcal{P}_{\mathrm{r}},  \ \  \textrm{for } m=1,\ldots,2M     \textrm{,} \notag
\end{align}
where $\boldsymbol{a}=[\boldsymbol{0}_{2M}^T,-1]^T$, 
$\mathbf{A}= \begin{bmatrix}    \mathrm{diag} \left\{ \mathbf{s}_{\textrm{r}} \right\} \mathbf{H}_{\textrm{r}}, -\boldsymbol{1}_{K L}     \end{bmatrix}$.
In the branch-and-bound method subproblems are solved due to $\mathbf{v}=\left[\mathbf{v}_1^T,\mathbf{v}_2^T   \right]^T$, where $\mathbf{v}_1$ is a fixed vector of length $d$, taken from the discrete set $\mathcal{P}_{\mathrm{r}}^d$ with $\mathcal{P}_{\mathrm{r}}:=\left\{ 1/\sqrt{2M}, -1/\sqrt{2M}  \right\}$.
Accordingly the matrix of the first inequality in \eqref{eq:v_real} is expressed as $\mathbf{A}=\left[\mathbf{A}_1,  \mathbf{A}_2   \right]$, where $\mathbf{A}_1$ contains the first $d$ columns of $\mathbf{A}$.
The lower bound on the subproblem associated with the vector $\mathbf{v}_2$ is given by \looseness-1
\begin{align}
\label{eq:bb_optimzation_problem}
{\mathbf{v}_{2,\textrm{lb}}}=& \arg\min_{\mathbf{v}_2}  \boldsymbol{a}_2^T \mathbf{v}_2 \\
& \textrm{s.t. } \mathbf{A}_2 \mathbf{v}_2 \geq  \mathbf{b}    \notag \\
&     \left|\left[\mathbf{v}\right]_{m}\right|   \leq  c  \ \  \textrm{for all } m=1,\ldots,2M-d     \textrm{,} \notag
\end{align}   
where $\boldsymbol{a}_2=\left[ \boldsymbol{0}_{2M-d}^T,-1 \right]^T$ and $\mathbf{b}=-\mathbf{A}_1 \mathbf{v}_1$. 
The aim is to reduce the number of candidates by excluding individual vectors $\mathbf{v}_1$ based on the lower bounds obtained by \eqref{eq:bb_optimzation_problem} and upper-bounding of \eqref{eq:optimization_problem_precoder}, such that the solution finally can be obtained by an exhaustive search.
For the considered problem \eqref{eq:optimization_problem_precoder} a breadth-first strategy is proposed, where the individual steps are described in Algorithm~\ref{alg:BB_Precoding}. The subproblem \eqref{eq:bb_optimzation_problem} is an LP that can be solved by active set methods, which can take advantage of initialization vectors near the optimum. This can be practically exploited by the branch-and-bound strategy where a series of similar problems have to be solved.
\begin{algorithm}
  \caption{Proposed 1-Bit B\&B Precoding for solving \eqref{eq:optimization_problem_precoder}}
	\label{alg:BB_Precoding}
  \begin{algorithmic}    %[1] %for numbers
	\Initialization{}
		\vspace{-1.25em}
	\State{Given the channel $\mathbf{H}$ and transmit symbols $\mathbf{s}$ compute a valid upper bound $\check{f}$ on the problem in \eqref{eq:optimization_problem_precoder}, e.g., by solving \eqref{eq:lower_bound} followed by a mapping to the closest precoding vector $\mathbf{x}_{\textrm{r}} \in \mathcal{P}_{\textrm{r}}^{2M}$}
	\vspace{2mm}
	\State{Define the first level ($d=1$) of the tree by $\mathcal{G}_{d}:=\mathcal{P}_{\textrm{r}}$}
	\vspace{2mm}	
	\For{$d=1:2M-1$}
	\State{ Partition  $\mathcal{G}_{d}$ in $\mathbf{x}_{1,1},\ldots,\mathbf{x}_{1,\left|\mathcal{G}_{d}\right|}$ }  
	
	  \For{$i=1:\left| \mathcal{G}_{d} \right|$}
		\vspace{2mm}
		\State{Given $\mathbf{x}_{1,i}$ and $\mathbf{s}_{\textrm{r}}$ solve $\mathbf{v}_{2,\textrm{lb}}$ from \eqref{eq:bb_optimzation_problem}  }
		\State{Determine $\epsilon=\left[\mathbf{v}_{2,\textrm{lb}}\right]_{2M-d+1}$}
		\State{Compute the lower bound:  $\mathrm{lb}(\mathbf{x}_{1,i}):=  -\epsilon $;}
		\vspace{2mm}
		\State{Map $\mathbf{x}_{2,\mathrm{lb}}$ to the discrete solution with the closest} 
	  \State{Euclidean distance:}
		\State{$\check{\mathbf{x}}_2(\mathbf{x}_{2,\mathrm{lb}}) \in \mathcal{P}_{\textrm{r}}^{2M-d} $}
		\State{Using $\check{\mathbf{x}}_2$ find the smallest (inverted) distance to the}
		\State{decision threshold:}	
		\begin{align*}
		\mathrm{ub}(\mathbf{x}_{1,i}) := \max_{k}  \left[ -\mathrm{diag} \left\{ \mathbf{s}_{\textrm{r}} \right\} \mathbf{H}_{\textrm{r}}  
					        \begin{bmatrix} \mathbf{x}_{1,i} \\   \check{\mathbf{x}}_2   \end{bmatrix}      \right]_k
									\end{align*}  
									\State{Update the best upper bound with:}
		\State{$\check{f} =\min\left( \check{f}, \mathrm{ub}(\mathbf{x}_{1,i})  \right)    $}
	\EndFor
	\State{Build a reduced set by comparing lower bounds with}
	\State{the upper bound}
	\State{$\mathcal{G}_{d}^{\prime}:=\left\{  \mathbf{x}_{1,i} \vert \mathrm{lb}(\mathbf{x}_{1,i})  \leq  \check{f}       , i=1,\ldots,  \left|\mathcal{G}_{d}\right|  \right\} $}
	\vspace{2mm}
	\State{Define the set for the next level in the tree}
	\State{$\mathcal{G}_{d+1}:=\mathcal{G}_{d}^{\prime} \times \mathcal{P}_{\textrm{r}}$}
	\EndFor
	\State{ Partition  $\mathcal{G}_{d+1}$ in $\mathbf{x}_{1,1},\ldots,\mathbf{x}_{1,\left|\mathcal{G}_{d+1}\right|}$ }	
	  \State{  \begin{align*}
		\epsilon(\mathbf{x}_{1,i}) := \min_{k}   \left[  \mathrm{diag} \left\{ \mathbf{s}_{\textrm{r}} \right\} \mathbf{H}_{\textrm{r}}  
					         \mathbf{x}_{1,i}    \right]_k    \end{align*}}	
\State{The global solution is 
\begin{align*}
 \mathbf{x}_{\textrm{opt}} = \mathrm{arg} \max_{\mathbf{x}_{1,i} \in \mathcal{G}_{d+1}} \epsilon(\mathbf{x}_{1,i})  		
\end{align*}}
\end{algorithmic}
\end{algorithm}
Computing the precoding vector in each time instance exceeds the computational capacities in most applications. However, for small arrays and a large coherence time of the channel, a look-up-table can store the set of precoding vectors as suggested in \cite{Jedda_2016}. Considering the symmetries of the constellation $4^{K L-1}$ precoding vectors need to be computed and stored.


\section{Numerical Results}
\label{sec:numerical_results}
For the numerical evaluation of the uncoded BER and the sum-rate the signal-to-noise ratio is defined by
\begin{align}
\mathrm{SNR}=\frac{\mathrm{E}\left\{ E_{\textrm{Tx}}  \right\}   }{N_0} =\frac{ \left\|\mathbf{x}\right\|^2_2 }{\sigma_n^2} \textrm{,}
\end{align}
where $N_0$ is the noise power density.
We have compared our algorithms with the state-of-the-art precoders with 1-Bit DAC \cite{Saxena_2016} and \cite{Jedda_2016} and the high resolution precoder \cite{Mo_2015} whose performance is slightly better as only its total power is constrained.
Fig.~\ref{fig:BER_1} shows the BER performance of a scenario with two users with a single antenna and different sizes of the array at the transmitter.
As known from the literature the ZF approach \cite{Saxena_2016} shows an error floor which decreases with an increasing number of BS antennas. The proposed 1-Bit methods show a superior performance in terms of BER in comparison to the existing methods using 1-Bit DACs for $M=10$ (and below) and $K L=2$.
For $M=16$ (and above) the approximation, by mapping the solution of \eqref{eq:lower_bound} to the discrete vector with smallest Euclidean distance, and the exact solution of \eqref{eq:optimization_problem_precoder} have almost identical BER.
The optimal utilization of the 1-bit DACs shows a loss of not more than \si{2}{dB} in comparison to the proposed PoP (high resolution), which shows only marginal performance loss when considering an 8-PSK like phase quantization at the transmitter, e.g., by using 2-Bit DACs. 
The LHS of Fig.~\ref{fig:BER_2} shows the performance for larger arrays and more users with $K L=5$.
With the quantization at the receivers the channel output is discrete, such that the sum-rate corresponds to the sum over the mutual information of $K$ discrete memoryless channels $\sum_{k=1}^{K} I_k $, where 
\begin{align}
I_k =  \sum_{\boldsymbol{s}_k, \boldsymbol{y}_k }  P(\boldsymbol{y}_k \vert \boldsymbol{s}_k)P(\boldsymbol{s}_k)  \log_2  \frac{ P(\boldsymbol{y}_k \vert \boldsymbol{s}_k)P(\boldsymbol{s}_k)  }{ P(\boldsymbol{s}_k) P(\boldsymbol{y}_k) }~(\textrm{bpcu}) \textrm{,}
\label{eq:sumRate}
\end{align} 
where $P(\boldsymbol{s}_k)=1/4^{L}$, $P(\boldsymbol{y}_k \vert \boldsymbol{s}_k)= P(\boldsymbol{s}^{\prime}) \sum_{\boldsymbol{s}^{\prime}}  \prod_{i=1}^{2L}  1/2~ \mathrm{erfc}( - [\boldsymbol{y}_{\textrm{r},k}]_i [\mathbf{H}_{\textrm{r},k}\boldsymbol{x}_{\textrm{r}}(\boldsymbol{s}_k,\boldsymbol{s}^{\prime})]_i / \sigma_n )$ with $\boldsymbol{s}^{\prime}=[\boldsymbol{s}_1^T,\ldots,\boldsymbol{s}_{k-1}^T,\boldsymbol{s}_{k+1}^T,\ldots,\boldsymbol{s}_{K}^T]^T$ and $P(\boldsymbol{y}_k)=\sum_{\boldsymbol{s}_k} P(\boldsymbol{y}_k \vert \boldsymbol{s}_k) P(\boldsymbol{s}_k)$.
The RHS of Fig.~\ref{fig:BER_2} shows the sum-rate averaged over varied $\mathbf{H}$, where the proposed methods are gainful in medium $\mathrm{SNR}$.

In what follows, we show the benefit over the exhaustive search, which yields the same solution, by a pessimistic complexity estimation.
Assuming that the subproblems \eqref{eq:bb_optimzation_problem}, which are LPs, are solved by interior point methods (IPMs), the complexity of a subproblem is according to a widely accepted estimate in the community \cite{boyd_2004} in the order of $\mathcal{O}(n^{3.5})$, with $n\leq2M+1$.
Moreover, the average number of visited branches for the proposed branch-and-bound approach and the setting in Fig.~\ref{fig:BER_1} is shown in Table~\ref{tab:efficiency}, which is based on these numbers in the order of $\frac{3}{5}(2 M)^{2.5} $.
With this, the proposed method has polynomial complexity whereas the exhaustive search has exponential complexity and ZF precoders have a cost of $\mathcal{O}(n^{3})$, which also serves as a tight lower bound for the complexity of \cite{Jedda_2016} in which ZF is used for initialization. 
In our case active set methods (ASMs) are practically more efficient in comparison to IPMs because warm starting is possible, which can be exploited in branch-and-bound schemes with a series of similar problems.
The drawback of ASMs, namely its worst-case number of iterations is only theoretical because of the discrepancy between practice and worst-case behavior \cite{Borgwardt_1987}.
\begin{figure}
%\vspace{-2em}
\begin{center}
%\sidecaption[t]
\includegraphics{results_BER.eps}
%\vspace{-0.5em}
\captionsetup{justification=centering}
\caption{Uncoded BER versus $\mathrm{SNR}$, $K L=2$, colored curves with 1-Bit DACs } 
\label{fig:BER_1}       % Give a unique label
\vspace{-1em}
\end{center}
\end{figure}
\begin{figure}
%\vspace{-0.5em}
\begin{center}
%\sidecaption[t]
\includegraphics{results_BER_massive_and_SumRate.eps}%\vspace{-0.5em}
\captionsetup{justification=centering}
\caption{Uncoded BER (left), $K L=5$; Sum Rate (right), $M=10$, $K L=2$} 
\label{fig:BER_2}       % Give a unique label
\vspace{-1em}
\end{center}
\end{figure}




\section{Conclusions}
We have proposed an approach for optimal precoding for DACs with 1-bit resolution.
The design criterion describes the maximization of the minimum distance to the decision threshold at the receivers. Both the exact solution and the approximation based on the continuous relaxation followed by a mapping to the discrete set yield outstanding performance in terms of uncoded bit error rate, which can serve as a benchmark for designers working on 1-bit precoding techniques. The simulation results show that the loss brought by 1-Bit DACs is less than \si{2}{dB}.
\begin{table}
%\caption{ Source entropy rates of reconstructable sequences and independent uniformly distributed symbols} 
 \vspace{0pt}
\begin{center}
\captionsetup{justification=centering,font=scriptsize}
\caption{Efficiency of the proposed branch-and-bound approach}
\vspace{-0.5em}
\includegraphics{table.eps}
			\label{tab:efficiency}
		\end{center}
		\vspace{-2em}
		\end{table}


% An example of a floating figure using the graphicx package.
% Note that \label must occur AFTER (or within) \caption.
% For figures, \caption should occur after the \includegraphics. 
% Note that IEEEtran v1.7 and later has special internal code that
% is designed to preserve the operation of \label within \caption
% even when the captionsoff option is in effect. However, because
% of issues like this, it may be the safest practice to put all your
% \label just after \caption rather than within \caption{}.
%
% Reminder: the "draftcls" or "draftclsnofoot", not "draft", class
% option should be used if it is desired that the figures are to be
% displayed while in draft mode.
%
%\begin{figure}[!t]
%\centering
%\includegraphics[width=2.5in]{myfigure}
% where an .eps filename suffix will be assumed under latex, 
% and a .pdf suffix will be assumed for pdflatex; or what has been declared
% via \DeclareGraphicsExtensions.
%\caption{Simulation Results}
%\label{fig_sim}
%\end{figure}

% Note that IEEE typically puts floats only at the top, even when this
% results in a large percentage of a column being occupied by floats.


% An example of a double column floating figure using two subfigures.
% (The subfig.sty package must be loaded for this to work.)
% The subfigure \label commands are set within each subfloat command, the
% \label for the overall figure must come after \caption.
% \hfil must be used as a separator to get equal spacing.
% The subfigure.sty package works much the same way, except \subfigure is
% used instead of \subfloat.
%
%\begin{figure*}[!t]
%\centerline{\subfloat[Case I]\includegraphics[width=2.5in]{subfigcase1}%
%\label{fig_first_case}}
%\hfil
%\subfloat[Case II]{\includegraphics[width=2.5in]{subfigcase2}%
%\label{fig_second_case}}}
%\caption{Simulation results}
%\label{fig_sim}
%\end{figure*}
%
% Note that often IEEE papers with subfigures do not employ subfigure
% captions (using the optional argument to \subfloat), but instead will
% reference/describe all of them (a), (b), etc., within the main caption.


% An example of a floating table. Note that, for IEEE style tables, the 
% \caption command should come BEFORE the table. Table text will default to
% \footnotesize as IEEE normally uses this smaller font for tables.
% The \label must come after \caption as always.
%
%\begin{table}[!t]
%% increase table row spacing, adjust to taste
%\renewcommand{\arraystretch}{1.3}
% if using array.sty, it might be a good idea to tweak the value of
% \extrarowheight as needed to properly center the text within the cells
%\caption{An Example of a Table}
%\label{table_example}
%\centering
%% Some packages, such as MDW tools, offer better commands for making tables
%% than the plain LaTeX2e tabular which is used here.
%\begin{tabular}{|c||c|}
%\hline
%One & Two\\
%\hline
%Three & Four\\
%\hline
%\end{tabular}
%\end{table}


% Note that IEEE does not put floats in the very first column - or typically
% anywhere on the first page for that matter. Also, in-text middle ("here")
% positioning is not used. Most IEEE journals/conferences use top floats
% exclusively. Note that, LaTeX2e, unlike IEEE journals/conferences, places
% footnotes above bottom floats. This can be corrected via the \fnbelowfloat
% command of the stfloats package.




% conference papers do not normally have an appendix


% use section* for acknowledgement





% trigger a \newpage just before the given reference
% number - used to balance the columns on the last page
% adjust value as needed - may need to be readjusted if
% the document is modified later
%\IEEEtriggeratref{8}
% The "triggered" command can be changed if desired:
%\IEEEtriggercmd{\enlargethispage{-5in}}

% references section

% can use a bibliography generated by BibTeX as a .bbl file
% BibTeX documentation can be easily obtained at:
% http://www.ctan.org/tex-archive/biblio/bibtex/contrib/doc/
% The IEEEtran BibTeX style support page is at:
% http://www.michaelshell.org/tex/ieeetran/bibtex/
%\bibliographystyle{IEEEtran}
% argument is your BibTeX string definitions and bibliography database(s)
%\bibliography{IEEEabrv,../bib/paper}
%
% <OR> manually copy in the resultant .bbl file
% set second argument of \begin to the number of references
% (used to reserve space for the reference number labels box)
\bibliographystyle{IEEEtran}
\begin{thebibliography}{10}
\providecommand{\url}[1]{#1}
\csname url@samestyle\endcsname
\providecommand{\newblock}{\relax}
\providecommand{\bibinfo}[2]{#2}
\providecommand{\BIBentrySTDinterwordspacing}{\spaceskip=0pt\relax}
\providecommand{\BIBentryALTinterwordstretchfactor}{4}
\providecommand{\BIBentryALTinterwordspacing}{\spaceskip=\fontdimen2\font plus
\BIBentryALTinterwordstretchfactor\fontdimen3\font minus
  \fontdimen4\font\relax}
\providecommand{\BIBforeignlanguage}[2]{{%
\expandafter\ifx\csname l@#1\endcsname\relax
\typeout{** WARNING: IEEEtran.bst: No hyphenation pattern has been}%
\typeout{** loaded for the language `#1'. Using the pattern for}%
\typeout{** the default language instead.}%
\else
\language=\csname l@#1\endcsname
\fi
#2}}
\providecommand{\BIBdecl}{\relax}
\BIBdecl

\bibitem{Spencer_2004}
Q.~H. Spencer, C.~B. Peel, A.~L. Swindlehurst, and M.~Haardt, ``An introduction
  to the multi-user {MIMO} downlink,'' \emph{{IEEE} Commun. Mag.}, vol.~42,
  no.~10, pp. 60--67, Oct 2004.

\bibitem{Naber_1989}
J.~F. Naber, H.~P. Singh, R.~A. Sadler, and J.~M. Milan, ``A low-power,
  high-speed 4-bit gaas adc and 5-bit dac,'' in \emph{Proc. 11th Annual Gallium
  Arsenide Integrated Circuit (GaAs IC) Symp.}, Oct 1989, pp. 333--336.

\bibitem{Saxena_2016}
A.~K. Saxena, I.~Fijalkow, and A.~L. Swindlehurst, ``On one-bit quantized {ZF}
  precoding for the multiuser massive {MIMO} downlink,'' in \emph{Proc. of IEEE
  Sensor Array and Multichannel Signal Processing Workshop (SAM)}, July 2016.

\bibitem{JacobssonDCGS16a}
S.~Jacobsson, G.~Durisi, M.~Coldrey, T.~Goldstein, and C.~Studer, ``Quantized
  precoding for massive {MU-MIMO},'' \emph{{IEEE} Trans. Commun.}, 2017,
  accepted.

\bibitem{Jedda_2016}
H.~Jedda, J.~A. Nossek, and A.~Mezghani, ``Minimum {BER} precoding in 1-bit
  massive {MIMO} systems,'' in \emph{Proc. of IEEE Sensor Array and
  Multichannel Signal Processing Workshop (SAM)}, July 2016.

\bibitem{Tirkonnen_2017}
O.~Tirkkonen and C.~Studer, ``Subset-codebook precoding for 1-bit massive
  multiuser {MIMO},'' in \emph{Proc. of Conf. on Information Sciences and
  Systems}, March 2017.

\bibitem{Israel_2015_Letter}
J.~Israel, A.~Fischer, J.~Martinovic, E.~A. Jorswieck, and M.~Mesyagutov,
  ``Discrete receive beamforming,'' \emph{{IEEE} Signal Process. Lett.},
  vol.~22, no.~7, pp. 958--962, July 2015.

\bibitem{Landau_SCC2013}
L.~Landau, S.~Krone, and G.~P. Fettweis, ``Intersymbol-interference design for
  maximum information rates with 1-bit quantization and oversampling at the
  receiver,'' in \emph{Proc. of the Int. ITG Conf. on Systems, Communications
  and Coding}, Munich, Germany, Jan. 2013.

\bibitem{Mo_2015}
J.~Mo and R.~W. Heath~Jr, ``Capacity analysis of one-bit quantized {MIMO}
  systems with transmitter channel state information,'' \emph{{IEEE} Trans.
  Signal Process.}, vol.~63, no.~20, pp. 5498--5512, Oct 2015.

\bibitem{Gokceoglu_2016b}
A.~Gokceoglu, E.~Bj\"{o}rnson, E.~G. Larsson, and M.~Valkama, ``Spatio-temporal
  waveform design for multi-user massive {MIMO} downlink with 1-bit
  receivers,'' \emph{{IEEE} J. Sel. Topics Signal Process.}, Nov. 2016.

\bibitem{boyd_2004}
S.~Boyd and L.~Vandenberghe, \emph{Convex Optimization}.\hskip 1em plus 0.5em
  minus 0.4em\relax New York, NY, USA: Cambridge University Press, 2004.

\bibitem{Borgwardt_1987}
K.~Borgwardt, \emph{The Simplex Method, A Probabilistic Analysis}.\hskip 1em
  plus 0.5em minus 0.4em\relax Berlin, Germany: Springer-Verlag, 1987.

\end{thebibliography}


% that's all folks
\end{document}


