%%
%% This is file `sample-sigconf.tex',
%% generated with the docstrip utility.
%%
%% The original source files were:
%%
%% samples.dtx  (with options: `sigconf')
%% 
%% IMPORTANT NOTICE:
%% 
%% For the copyright see the source file.
%% 
%% Any modified versions of this file must be renamed
%% with new filenames distinct from sample-sigconf.tex.
%% 
%% For distribution of the original source see the terms
%% for copying and modification in the file samples.dtx.
%% 
%% This generated file may be distributed as long as the
%% original source files, as listed above, are part of the
%% same distribution. (The sources need not necessarily be
%% in the same archive or directory.)
%%
%% Commands for TeXCount
%TC:macro \cite [option:text,text]
%TC:macro \citep [option:text,text]
%TC:macro \citet [option:text,text]
%TC:envir table 0 1
%TC:envir table* 0 1
%TC:envir tabular [ignore] word
%TC:envir displaymath 0 word
%TC:envir math 0 word
%TC:envir comment 0 0
%%
%%
%% The first command in your LaTeX source must be the \documentclass command.
\documentclass[sigconf]{acmart}
\settopmatter{authorsperrow=4}

\setcopyright{none}
\settopmatter{printacmref=false} % Removes citation information below abstract
\renewcommand\footnotetextcopyrightpermission[1]{} % removes footnote with conference information in first column
%% NOTE that a single column version is required for 
%% submission and peer review. This can be done by changing
%% the \doucmentclass[...]{acmart} in this template to 
%% \documentclass[manuscript,screen]{acmart}
%% 
%% To ensure 100% compatibility, please check the white list of
%% approved LaTeX packages to be used with the Master Article Template at
%% https://www.acm.org/publications/taps/whitelist-of-latex-packages 
%% before creating your document. The white list page provides 
%% information on how to submit additional LaTeX packages for 
%% review and adoption.
%% Fonts used in the template cannot be substituted; margin 
%% adjustments are not allowed.

%%
%% \BibTeX command to typeset BibTeX logo in the docs

%% Rights management information.  This information is sent to you
%% when you complete the rights form.  These commands have SAMPLE
%% values in them; it is your responsibility as an author to replace
%% the commands and values with those provided to you when you
%% complete the rights form.
\setcopyright{acmcopyright}
\copyrightyear{2018}
\acmYear{2018}
\acmDOI{XXXXXXX.XXXXXXX}

%% These commands are for a PROCEEDINGS abstract or paper.
\acmConference[]{}{}{}
%
%  Uncomment \acmBooktitle if th title of the proceedings is different
%  from ``Proceedings of ...''!
%
%\acmBooktitle{Woodstock '18: ACM Symposium on Neural Gaze Detection,
%  June 03--05, 2018, Woodstock, NY} 
\acmPrice{15.00}
\acmISBN{978-1-4503-XXXX-X/18/06}

\DeclareMathOperator*{\argmax}{arg\,max}
\DeclareMathOperator*{\argmin}{arg\,min}


%%
%% Submission ID.
%% Use this when submitting an article to a sponsored event. You'll
%% receive a unique submission ID from the organizers
%% of the event, and this ID should be used as the parameter to this command.
%%\acmSubmissionID{123-A56-BU3}

%%
%% For managing citations, it is recommended to use bibliography
%% files in BibTeX format.
%%
%% You can then either use BibTeX with the ACM-Reference-Format style,
%% or BibLaTeX with the acmnumeric or acmauthoryear sytles, that include
%% support for advanced citation of software artefact from the
%% biblatex-software package, also separately available on CTAN.
%%
%% Look at the sample-*-biblatex.tex files for templates showcasing
%% the biblatex styles.
%%

%%
%% The majority of ACM publications use numbered citations and
%% references.  The command \citestyle{authoryear} switches to the
%% "author year" style.
%%
%% If you are preparing content for an event
%% sponsored by ACM SIGGRAPH, you must use the "author year" style of
%% citations and references.
%% Uncommenting
%% the next command will enable that style.
%%\citestyle{acmauthoryear}

% Use packages.
\usepackage{multirow}
\usepackage{tikz}
\usepackage{enumitem}
\usepackage{amsmath}

\author{Tunhou Zhang}
\authornote{A majority of this work was done when the first author was an intern at Meta Platforms, Inc.}
\affiliation{%
  \institution{Duke University}
  \city{Durham}
  \country{USA}
 }
 \email{tunhou.zhang@duke.edu}

\author{Wei Wen}
\authornote{Intern Manager}
\affiliation{%
  \institution{Meta AI}
  \city{Menlo Park}
  \country{USA}
}
\email{wewen@meta.com}

\author{Igor Fedorov}
\affiliation{%
  \institution{Meta AI}
  \city{Menlo Park}
  \country{USA}
}
\email{ifedorov@meta.com}

\author{Xi Liu}
\affiliation{%
  \institution{Meta AI}
  \city{Menlo Park}
  \country{USA}
}
\email{xliu1@meta.com}

\author{Buyun Zhang}
\affiliation{%
  \institution{Meta AI}
  \city{Menlo Park}
  \country{USA}
}
\email{buyunz@meta.com}

\author{Fangqiu Han}
\affiliation{%
  \institution{Meta AI}
  \city{Menlo Park}
  \country{USA}
}
\email{fhan@meta.com}

\author{Wen-Yen Chen}
\affiliation{%
  \institution{Meta AI}
  \city{Menlo Park}
  \country{USA}
}
\email{wychen@meta.com}

\author{Yiping Han}
\affiliation{%
  \institution{Meta AI}
  \city{Menlo Park}
  \country{USA}
}
\email{yipinghan@meta.com}

\author{Feng Yan}
\affiliation{%
  \institution{University of Houston}
  \city{Houston}
  \country{USA}
}
\email{fyan5@central.uh.edu}

\author{Hai Li}
\affiliation{%
  \institution{Duke University}
  \city{Durham}
  \country{USA}
 }
 \email{hai.li@duke.edu}
 
\author{Yiran Chen}
\affiliation{%
  \institution{Duke University}
  \city{Durham}
  \country{USA}
 }
\email{yiran.chen@duke.edu}



%%
%% end of the preamble, start of the body of the document source.
\begin{document}

%%
%% The "title" command has an optional parameter,
%% allowing the author to define a "short title" to be used in page headers.
\title{DistDNAS: Search Efficient Feature Interactions within 2 Hours}

%%
%% The "author" command and its associated commands are used to define
%% the authors and their affiliations.
%% Of note is the shared affiliation of the first two authors, and the
%% "authornote" and "authornotemark" commands
%% used to denote shared contribution to the research.
%%
%% By default, the full list of authors will be used in the page
%% headers. Often, this list is too long, and will overlap
%% other information printed in the page headers. This command allows
%% the author to define a more concise list
%% of authors' names for this purpose.

%%
%% The abstract is a short summary of the work to be presented in the
%% article.
\begin{abstract}
Search efficiency and serving efficiency are two major axes in building feature interactions and expediting the model development process in recommender systems.
On large-scale benchmarks, searching for the optimal feature interaction design requires extensive cost due to the sequential workflow on the large volume of data.
In addition, fusing interactions of various sources, orders, and mathematical operations introduces potential conflicts and additional redundancy toward recommender models, leading to sub-optimal trade-offs in performance and serving cost. 
In this paper, we present DistDNAS as a neat solution to brew swift and efficient feature interaction design.
DistDNAS proposes a supernet to incorporate interaction modules of varying orders and types as a search space. 
To optimize search efficiency, DistDNAS distributes the search and aggregates the choice of optimal interaction modules on varying data dates, achieving over 25$\times$ speed-up and reducing search cost from 2 days to 2 hours.
To optimize serving efficiency, DistDNAS introduces a differentiable cost-aware loss to penalize the selection of redundant interaction modules, enhancing the efficiency of discovered feature interactions in serving.
We extensively evaluate the best models crafted by DistDNAS on a 1TB Criteo Terabyte dataset.
Experimental evaluations demonstrate 0.001 AUC improvement and 60\% FLOPs saving over current state-of-the-art CTR models.
\end{abstract}

%%
%% The code below is generated by the tool at http://dl.acm.org/ccs.cfm.
%% Please copy and paste the code instead of the example below.
%%
\begin{CCSXML}
<ccs2012>
   <concept>
       <concept_id>10002951.10003317.10003347.10003350</concept_id>
       <concept_desc>Information systems~Recommender systems</concept_desc>
       <concept_significance>500</concept_significance>
       </concept>
   <concept>
       <concept_id>10002951.10003317.10003338.10003344</concept_id>
       <concept_desc>Information systems~Combination, fusion and federated search</concept_desc>
       <concept_significance>300</concept_significance>
       </concept>
   <concept>
       <concept_id>10010147.10010178.10010205.10010207</concept_id>
       <concept_desc>Computing methodologies~Discrete space search</concept_desc>
       <concept_significance>500</concept_significance>
       </concept>
   <concept>
       <concept_id>10010147.10010257.10010293.10010294</concept_id>
       <concept_desc>Computing methodologies~Neural networks</concept_desc>
       <concept_significance>500</concept_significance>
       </concept>
 </ccs2012>
\end{CCSXML}

\ccsdesc[500]{Information systems~Recommender systems}
\ccsdesc[300]{Information systems~Combination, fusion and federated search}
\ccsdesc[500]{Computing methodologies~Discrete space search}
\ccsdesc[500]{Computing methodologies~Neural networks}

%%
%% Keywords. The author(s) should pick words that accurately describe
%% the work being presented. Separate the keywords with commas.
\keywords{Recommender Systems, Neural Architecture Search, AutoML, Click-Through Rate Prediction}


\maketitle

%
%%%%%%%%%%%%%%%%%%%%%%%%%%%
\section{Introduction}
\label{Sec: Introduction}
%%%%%%%%%%%%%%%%%%%%%%%%%%%
%

Fractional models construct a tractable mathematical framework to describe and predict the behavior of multi-scales multi-physics complex phenomena. Particularly, fractional differential equations, as a well-structured generalization of their integer order counterparts, provide a rigorous mathematical tool to develop models, describing anomalous behavior in complex physical systems \cite{zhang2017review,jaishankar2014,jha2003evidence, Castillo2004Plasma,jaishankar2013,naghibolhosseini2015estimation,naghibolhosseini2017fractional,meer01, meral2010fractional}, where the anomalies manifest in heavy tailed distribution of corresponding underlying stochastic processes, moreover, exhibiting sharp peaks, intermittency, and asymmetry in the underlying distributions. Significant approximations as inherent part of assumptions upon which the model is built, lack of information about true values of parameters (incomplete data), and random nature of quantities being modeled pervade uncertainty in the corresponding mathematical formulations \cite{cullen1999probabilistic,roy2011comprehensive}. In this work, we develop an uncertainty quantification (UQ) framework in the context of stochastic fractional partial differential equations (SFPDEs), in which we characterize different sources of uncertainties and further propagate the associated randomness to the system response quantity of interest (QoI).


\vspace{0.1 in}
%
%%%%%%%%%%%%%%%%%%%%%%%%%%%
\noindent\textbf{Types and Sources of Uncertainty}.
%%%%%%%%%%%%%%%%%%%%%%%%%%%
%
The model uncertainties are in general being classified as aleatory, epistemic, and mixed, according to their fundamental essence. They can also be broadly characterized as occurring in model inputs, numerical approximations, and model form. Model inputs encompass all model parameters coming from geometry, constitutive laws, and fields equation, while also pertaining surrounding interactions, such as boundary conditions and random forcing sources (noise). Numerical approximations, which are an essence of differential equations since they generally do not lend themselves to analytical solutions, introduce uncertainty by imposing different sources of discretization error, iterative convergence error, and round off error. The fractional derivatives introduce derivative orders, namely fractional indices, as new set of model parameters in addition to model coefficients. These new parameters are strongly tied to the distribution of underlying stochastic process and their statistics are estimated from experimental observations in practice, see e.g.  \cite{benson2000application,baeumer2001subordinated}. 




\vspace{0.1 in}
%
%%%%%%%%%%%%%%%%%%%%%%%%%%%
\noindent\textbf{Uncertainty Framework}.
%%%%%%%%%%%%%%%%%%%%%%%%%%%
%
Conventional approaches in parametric UQ of differential equations is based around Monte Carlo sampling (MCS) \cite{carlo1996concepts}, which performs ensemble of forward calculations to map the uncertain input space to the uncertain output space.
This method enjoys from being embarrassingly parallelizable and can be implement quite readily on high dimensional random spaces. However, the key issue is the slow rate of convergence $\sim 1/\sqrt{N}$ with $N$ number of realization, which consequently imposes exhaustively so many operations of forward solver, makes it not practical for expensive simulations. Other methods such as sequential MCS \cite{del2006sequential} and multilevel sequential MSC \cite{beskos2017multilevel} are also developed and recently used in \cite{jasra2016forward} to improve the parametric uncertainty assessment in elliptic nonlocal equations. An alternative to expensive MCS is to build surrogate models. An extensive comparison of two widely used ones, namely polynomial chaos and Gaussian process, are provided in a recent work \cite{owen2017comparison}. Polynomial chaos, in which the output of stochastic model is represented as a series expansion of input parameters, was initially applied in \cite{ghanem2003stochastic}, and later extended and used in \cite{xiu2002modeling,xiu2002wiener,knio2006uncertainty}. It is also generalized and used in constructing stochastic Galerkin methods \cite{babuska2004galerkin,babuvska2005solving,le2004uncertainty,le2004multi} for problems with higher-dimensional random spaces. Other non-sampling numerical methods, including but not limited to perturbation method \cite{schuss1980singular,babuvska2002solving,todor2005sparse,winter2002groundwater} and moment equation method \cite{liu1986probabilistic,liu1986random} are also developed, however their applications are restricted to stochastic systems with relatively low-dimensional random space. These so-called ``\textit{intrusive}" approaches typically do not treat the forward solver as a black-box, rather require some knowledge and reformulation of the governing equations and thus, may not be practical in many problems with complex codes.
%; therefore, a ``non-intrusive" approach may be more practical.

A wide range of ``\textit{non-intrusive}" techniques mostly stretch over sampling, quadrature, and regression, see \cite{owen2017comparison} and references therein. More recently, high-order probabilistic collocation methods (PCM), employing the idea of interpolation/collocation in the random spaces, are developed in \cite{xiu2005high,babuvska2007stochastic,nobile2008sparse}. These methods limit the sample points to an efficient subset of random space, while adequately sampling the necessary range. The excellence in use of PCM is twofold; it has the benefit of easily sampling at nodal points that naturally leads to independent realizations of the problem as in MCS, and the advantage of fast convergence rate. The challenging post processing of solution statistics, which can essentially be described as a high-dimensional integration problem, can also be resolved by adopting sparse grid generators, such as Smolyak algorithm \cite{nobile2008sparse,Smolyak63}. The use of sparse grids, as opposed to full tensor product construction from one-dimensional quadrature rules, will effectively reduce the number of sampling, while preserving a fast convergence rate to high level of accuracy. %Utilizing domain decomposition tools, PCM is also extended to multi-element PCM \cite{Foo08}. 






\vspace{0.1 in}
%
%%%%%%%%%%%%%%%%%%%%%%%%%%%
\noindent\textbf{Forward Solver}.
%%%%%%%%%%%%%%%%%%%%%%%%%%%
%
A core task in computational forward UQ is to form an efficient numerical method, which for each realizations of random variables can accurately solves and simulates the deterministic counterpart of stochastic model in the physical domain. Such numerical method is usually called ``\textit{forward solver}" or ``\textit{simulator}". In the case of FPDEs, the excessive cost of numerical approximations due to the inherent nonlocal nature of fractional differential equations additionally become more challenging when generally most of uncertainty propagation techniques instruct operations of forward solver many times. This requires implementation of more efficient numerical schemes, which can manage increasing computational costs while maintaining sufficiently low error level in mitigating the corresponding uncertainties. In addition to numerous finite difference methods for solving FPDEs \cite{Gorenflo2002, Sun2006,  Lin2007, wang2010direct, wang2011fast, Cao2013, zeng2015numerical, zayernouri2016_JCP_Frac_AB_AM}, recent works have elaborated efficient spectral schemes, for discretizing FPDEs in physical domain, see e.g., \cite{Rawashdeh2006, Lin2007, Khader2011, Khader2012, Li2009, Li2010, chen2014generalized, wang2015high, bhrawy2015spectral}. More recently, Zayernouri et al. \cite{Zayernouri2013, zayernouri2015tempered} developed two new spectral theories on fractional and tempered fractional Sturm-Liouville problems, and introduced explicit corresponding eigenfunctions, namely \textit{Jacobi poly-fractonomials} of first and second kind. These eignefunctions are comprised of smooth and fractional parts, where the latter can be tunned to capture singularities of true solution. They are successfully employed in constructing discrete solution/test function spaces and developing a series of high-order and efficient Petrov-Galerkin spectral methods, see \cite{lischke2017petrov, suzuki2016fractional,samiee2016,samiee2017fast,samiee2017unified,samiee2018petrov,kharazmi2017petrov,kharazmi2017sem,kharazmi2018fractional}.



\vspace{0.1 in}

The main focus of this work is to develop an operator-based computational forward UQ framework in the context of stochastic fractional partial differential equation. Assuming that the mathematical model under consideration is well-posed and accounts in principle for all features of underlying phenomena, we identify three main sources of uncertainty, \textit{i}) parametric uncertainty, including fractional indices as new set of random parameters appeared in the operator, \textit{ii}) additive noises, which incorporates all intrinsic/extrinsic unknown forcing sources as lumped random inputs, and \textit{iii}) numerical approximations. Computational challenges arise when the admissible space of random inputs is infinite-dimensional, e.g. problems subject to additive noise \cite{rizzi2012uncertainty}, and thus, the framework involves uncertainty parametrization via a finite number of random space basis. Unlike the classical approach in modeling random inputs, which considers idealized uncorrelated processes (white noises), we model the random inputs as more/fully correlated random processes (colored noises), and parametrize them via Karhunen-Lo\`{e}ve (KL) expansion by assuming finite-dimensional noise assumption. This yields the problem in finite dimensional random space. We then, propagate the parametric uncertainties into the system response by applying PCM. We obtain the corresponding deterministic FPDE for each realization of random variables, using the Smolyak sparse grid generators for low to moderately high dimensions. In order to formulate the forward solver, we follow \cite{samiee2016} and develop a high-order Petrov-Galerkin (PG) spectral method to solve for each realization of SFPDE in the physical domain, employing Jacobi poly-fractonomials in addition to Legendre polynomials as temporal and spatial basis/test functions, respectively. The smart choice of coefficients in construction of spatial basis/test functions yields symmetric properties in the resulting mass/stiffness matrix, which is then exploited to formulate an efficient fast solver. We also show that for each realization of random variables, the deterministic problem is mathematically well-posed and the proposed forward solver is stable. By adopting sufficient number of basis in the physical domain, we eliminate the epistemic uncertainty associated with numerical approximation and isolate the impact of parametric uncertainty on system response QoI.


%\vspace{0.1 in}

The organization of the paper is as follows. We recall some preliminaries on fractional calculus in section \ref{Sec: Fractional Calculus}. Then, we formulate the stochastic system in section \ref{Sec: Uncertainty framework}, and parametrize the random inputs. We also develop the stochastic sampling, namely PCM and MCS for our stochastic problem. We further construct the deterministic solver in section \ref{Sec: Forward Solver}, and provide the numerical results in section \ref{Sec: numerical results}. We end the paper with a conclusion and summary.




\section{Related Work}

\noindent \textbf{Feature Interactions in Recommender Systems.}
The feature interactions within recommender systems such as CTR prediction have been thoroughly investigated in various approaches, such as Logistic Regression~\cite{richardson2007predicting}, and Gradient-Boosting Decision Trees~\cite{he2014practical}.
Recent approaches apply deep learning based interaction~\cite{zhang2019deep} to enhance end-to-end modeling experience by innovating Wide \& Deep Neural Networks~\cite{cheng2016wide}, Deep Crossing~\cite{wang2017deep}, Factorization Machines~\cite{guo2017deepfm,lian2018xdeepfm}, DotProduct~\cite{naumov2019deep} and gating mechanism~\cite{wang2017deep, wang2021dcn}, ensemble of feature interactions~\cite{zhang2022dhen}, feature-wise multiplications~\cite{wang2021masknet}, and sparsifications~\cite{deng2021deeplight}.
In addition, these works do not fully consider the impact of fusing different types of feature interactions, such as the potential redundancy, conflict, and performance enhancement induced by a variety of feature interactions.
DistDNAS constructs a supernet to explore different orders and types of interaction modules and distributes differentiable search to advance search efficiency.

\noindent \textbf{Cost-aware Neural Architecture Search.}     
Neural Architecture Search (NAS) automates the design of Deep Neural Network (DNN) in various applications: the popularity of NAS is consistently growing in brewing Computer Vision~\cite{zoph2018learning,liu2018darts,wen2020neural,cai2019once}, Natural Language Processing~\cite{so2019evolved,wang2020hat}, and Recommender Systems~\cite{song2020towards,gao2021progressive,krishna2021differentiable,zhang2022nasrec}.
Tremendous efforts are made to advance the performance of discovered architectures to brew a state-of-the-art model.
Despite the improvement in search/evaluation algorithms, existing NAS algorithms overlook the opportunity to harvest performance improvements by addressing potential conflict and redundancy in feature interaction modules.
DistDNAS regularizes the cost of the searched feature interactions and prunes unnecessary interaction modules as building blocks, yielding better FLOPs-LogLoss trade-offs on CTR benchmarks.
\begin{figure*}[t]
    \begin{center}
    \includegraphics[width=0.9\linewidth]{figs/DistDNAS_Search_Space_v2.pdf}
    \vspace{-1em}
    \caption{Feature interaction search space for each choice block in DistDNAS. Here, a dashed line denotes a searchable feature interaction in DistDNAS, and $\otimes$ denotes the mixing of different feature interaction modules.}
    \label{fig:dnas_search_space}
    \end{center}
    \vspace{-1em}
\end{figure*}

\section{Differentiable Feature Interaction Search Space}
Supernet is a natural fusion that incorporates feature interaction modules. DistDNAS emphasizes search on feature interaction modules and simplifies the search space from NASRec~\cite{zhang2022nasrec}. 
A supernet in DistDNAS contains multiple choice blocks, with a fixed connection between these choice blocks, and fully enabled dense-to-sparse/sparse-to-dense merger within all choice blocks. 
Unlike NASRec~\cite{zhang2022nasrec} which solely selects the optimal interaction module within each choice block, DistDNAS can select an arbitrary number of interaction modules within each choice block and use differentiable bi-level optimization~\cite{liu2018darts} to determine the best selection.
This allows flexibility in fusing varying feature interactions and obtaining the best combination with enhanced search efficiency.
We present the details of feature interaction modules as follows.


\subsection{Feature Interaction Modules}
Feature interaction modules connect dense 2D inputs with sparse 3D inputs to learn useful representations on user modeling.
In a recommender system, a dense input is a 2D tensor from either raw dense features or generated by dense interaction modules, such as a fully connected layer.
A sparse input is a 3D tensor of sparse embeddings either generated by raw sparse/categorical features or by sparse interaction modules such as a self-attention layer.
We define a dense input as $X_{d} \in \mathbb{R}^{B \times dim_d}$ and a sparse input $X_{s} \in \mathbb{R}^{B \times N_s \times dim_s}$. 
Here, B denotes the batch size, $dim_d$/$dim_s$ denotes the dimension of the dense/sparse input, and $N_s$ denotes the number of inputs in the sparse input.

We collect a set of simple feature interaction modules from the existing literature, as demonstrated in Figure \ref{fig:dnas_search_space}.
A dense interaction module produces a dense output given input features, and a sparse interaction module produces a sparse output given input features.
These interaction modules can cover a set of state-of-the-art CTR models, such as DLRM~\cite{naumov2019deep}, DeepFM~\cite{wang2017deep}, xDeepInt~\cite{yan2023xdeepint}, DCN-v2~\cite{wang2021dcn}, and AutoInt~\cite{song2019autoint}.
\begin{itemize}[noitemsep,leftmargin=*]
    \item A \textbf{Identity/Identity3D} layer is a dense/sparse interaction module that carries an identity transformation on dense/sparse input. Through identity layers, one can bypass low-order interaction outputs towards deeper choice blocks and formulate a higher-order feature interaction.

    \item A \textbf{Linear/Linear3D} layer is a dense/sparse interaction module that applies on 2D/3D dense inputs, followed by a ReLU activation.
    Linear/Linear3D is the backbone of recommender systems since Wide \& Deep Learning~\cite{cheng2016wide}.     

    %if the dimension of $x_1$ and $x_2$ does not match.
    \item A \textbf{DotProduct}~\cite{cheng2016wide,naumov2019deep} layer is a dense interaction module that computes pairwise inner products of dense/sparse inputs.
    Given a dense input $X_{d} \in \mathbb{R}^{B \times dim_{d}}$ and a sparse input $X_{s} \in \mathbb{R}^{B \times N_c \times dim_{s}}$, a DotProduct first concatenates them as $X = Concat[X_d, X_s]$, then performs pairwise inner products and collects upper triangular matrices as output: $DP(X_d, X_s)=Triu(XX^{T})$.

    \item A \textbf{CrossNet}~\cite{wang2021dcn} layer is a dense interaction module with gate inputs from various sources. 
    Given a dense input $X_{d}\in \mathbb{R}^{B \times dim_{d}}$, we process the dense input and the raw dense input $X_0$ with $CrossNet(X_{d}) = sigmoid(Linear(X_d)) * X_{0}$.

    \item An \textbf{PIN}~\cite{yan2023xdeepint} layer is a dense interaction module with gate inputs from various sources. 
    Given a dense input $X_{d}\in \mathbb{R}^{B \times dim_{d}}$, xDeepInt interacts the dense input with the raw dense input $X_0$ as follows: $PIN(X_{d}) = sigmoid(Linear(X_{0})) * X_{d}$.
    % If the dimension of two dense inputs does not match, zero padding is applied on the input with a lower dimension. 
    In an xDeepInt layer, we switch the order of left/right input and perform $PIN(X_{d1}, X_{d2}) = X_{d1} * sigmoid(Linear(X_{d2}))$.

    \item A \textbf{Transformer}~\cite{vaswani2017attention} layer is a sparse interaction module that utilizes the multihead attention mechanism to learn the weighting of different sparse inputs. 
    The queries, keys, and values of a Transformer layer are identically the sparse input $X_{s} \in \mathbb{R}^{B\times N_s \times dim_{s}}$.
    Two Linear3D layers are then utilized to process sparse features in the embedding dimension, followed by addition and layer normalization.
    
    \item An \textbf{Embedded Fully-Connected (EmbedFC)} is a sparse interaction module that applies a linear operation along the middle dimension. Specifically, an EmbedFC with weights $W \in \mathbb{R}^{N_{in} \times N_{out}}$ transforms an input $X_s \in \mathbb{R}^{B \times N_{in} \times dim_{s}}$ to $Y_s \in \mathbb{R}^{B \times N_{out} \times dim_{s}}$.



    \item A \textbf{Pooling by Multihead Attention (PMA)}~\cite{lee2019set} layer is a sparse interaction module that forms attention between seed vectors and sparse features.
    This encourages permutation-invariant representations in aggregation. 
    Here, PMA applies multihead attention on a given sparse input $X_{s} \in \mathbb{R}^{B\times N_s \times dim_{s}}$ with seed vector $V \in \mathbb{R}^{1 \times N_{s} \times dim_{s}}$. PMA uses the seed vector $V$ as queries and uses sparse features as keys and values.
\end{itemize}

Within all dense/sparse interaction modules, a proper linear projection (e.g., Linear) will be applied if the dimension of inputs does not match. The feature interaction search space contains versatile dense/sparse interaction modules, providing 


\begin{figure*}[t]
    \begin{center}
    \includegraphics[width=0.8\linewidth]{figs/dnas_diagram.pdf}
    \vspace{-1em}
    \caption{Overview of DistDNAS methodology. Here, dashed lines denote searchable interaction modules, and the size of interaction modules indicates the cost penalty applied to each interaction module for serving efficiency.}
    \label{fig:distributed_dnas}
    \end{center}
    \vspace{-1em}
\end{figure*}
\subsection{Differentiable Supernet}
In DistDNAS, a differentiable supernet contains $N$ choice blocks with a rich collection of feature interactions.
Each choice block employs a set of dense/sparse interactions $op^{(i)}=\{op_{d}^{(i)}, op_{s}^{(i)}\}$ to take dense/sparse inputs $X^{(i)}_{d}$/$X^{(i)}_{s}$ and learn useful representations. 
Each choice block contains $|op_{d}|=5$ dense feature interactions and $|op_{s}|=5$ sparse feature interactions. 
Each choice block receives input from previous choice blocks and produces a dense output $Y_d^{(i)}$ and a sparse output $Y_{s}^{(i)}$. 
In block-wise feature aggregation, each choice block concatenates the dense/sparse output from the previous 2/1 choice blocks as dense/sparse block input. 
For dense inputs in previous blocks, the concatenation occurs in the last feature dimension. 
For sparse inputs in previous blocks, concatenation occurs in the middle dimension to aggregate different sparse inputs.
We present the mixing of different feature interaction modules in the following context.


\noindent \textbf{Continuous Relaxation of Feature Interactions.}
Within a single choice block in the differentiable supernet, we depict the mixing of candidate feature interaction modules in Figure \ref{fig:dnas_search_space}.
We parameterize the weighting of each dense/sparse/dense-sparse interaction using architecture weights. 
In choice block $i$, we use $\alpha^{(i)}=\{\alpha_{1}^{(i)}, \alpha_{2}^{(i)}, ..., \alpha_{|op_{d}|}^{(i)}\}$ to represent the weighting of a dense interaction module and parameterize the selection of dense interaction modules with architecture weight $\mathrm{A}=\{\alpha^{(1)}, \alpha^{(2)}, ..., \alpha^{(N)}\}$.
Similarly, we parameterize the selection of sparse interaction modules with architecture weight $\mathrm{B}=\{ \beta^{(1)}, \beta^{(2)}, ..., \beta^{(N)} \}$. We employ the Gumbel Softmax~\cite{jang2016categorical} trick to allow a smoother sampling from categorical distribution on dense/sparse inputs as follows: 
\begin{equation}
Y_{d}^{(i)} = \sum_{j=1}^{|op_d|}\frac{\exp(\frac{\log \alpha_{j}^{(i)} + g_{j}}{\lambda})}{\sum_{k}^{|op_d|}\exp(\frac{\log \alpha_{k}^{(i)} + g_{k}}{\lambda})} op_{j}(X_{d}^{i},X_{d}^{i-1}),
    \label{eq:dense_block_out_dnas}
\end{equation}
\vspace{-0.5em}
\begin{equation}
    Y_{s}^{(i)} = \sum_{j=1}^{|op_s|} \frac{\exp{(\frac{\log \beta_{j}^{(i)} + g_{j}}{\lambda}})}{\sum_{k}^{|op_s|} \exp(\frac{\log {\beta_k}^{(i)} + g_{k}}{\lambda})} op_{j}(X_{s}^{i}, X_{s}^{i-1})
    \label{eq:sparse_block_out_dnas}.
\end{equation}

Here, $g_{j}$ and $g_{k}$ are sampled from the Gumbel distribution, and $\lambda$ is the temperature. As a result, a candidate architecture $\mathcal{C}$ can be represented as a tuple of dense/sparse feature interaction: $\mathcal{C}=(\mathrm{A}, \mathrm{B})$. Within DistDNAS, our goal is to perform a differentiable search and obtain the optimal architecture $C^{*}$ that contains the weighting of dense/sparse interaction modules. 

\noindent \textbf{Discretization.} Discretization converts the weighting in the optimal architecture to standalone models to serve CTR applications. 
Past DNAS practices~\cite{liu2018darts} typically select the top k modules for each choice block, brewing a building cell containing a fixed number of modules in all parts of the network.
In recommender models, we discretize each choice block by a fixed threshold $\theta$ to determine useful interaction modules. For example, in the choice block $i$, we discretize the weight $\alpha^{(i)}$ to obtain the discretized dense interaction module $\hat{\alpha}^{(i)}$ as follows:
$$
\hat{\alpha}^{(i)}_{j} =
\begin{cases}
1, if    \hat{\alpha}^{(i)}_{j} \geq \theta, \\
0, otherwise.\\
\end{cases}
$$
We typically set threshold $\theta=1/|op|$ (i.e., 0.2 for dense/sparse module search) for each choice block, or use a slightly larger value (e.g., 0.25) to remove more redundancy.
There are a few advantages of adopting threshold-based discretization in recommender models. 
First, using a threshold $\theta$ is a clearer criterion to distinguish important/unimportant interaction modules within each choice block.
Second, since a recommender model contains multiple choice blocks with different hierarchies, levels, and dense/sparse input sources, there is a need for varying numbers of dense/sparse interactions to maximize the representation capacity within each module.


\section{Towards Efficiency in DNAS}
Search efficiency and serving efficiency are two major considerations in deploying DNAS algorithms in large-scale CTR datasets.
In this section, we first revisit DNAS and address the efficiency bottleneck via a distributed search mechanism. Then, we propose our solution to reduce the service cost of feature interaction via a cost-aware regularization approach. 
Figure \ref{fig:distributed_dnas} provides the core methodology of DistDNAS.

\label{sec4:dist}
\subsection{Revisiting DNAS on Recommender Systems}
A Click-Through Rate (CTR) prediction task usually contains multiple days of training data. In recommender systems, we typically use a few days of data (i.e., day 1 to day $T$) as the training source and evaluate the trained model based on its CTR prediction over subsequent days. DNAS carries bilevel optimization to find the optimal candidate architecture $C^{*}=(A^{*}, B^{*})$ as follows:
\begin{equation}
    (\mathrm{A}^{*}, \mathrm{B}^{*}) = \argmin_{\mathrm{A}, \mathrm{B}} LogLoss_{x \sim D}[x; w^{*}(\mathrm{A}, \mathrm{B}), \mathrm{A}, \mathrm{B}].
\end{equation}
Here, $BCE$ denotes binary cross-entropy, $D=(D_1, ..., D_{T})$ indicates the training data from day 1 to day $T$, $w$ indicates the weight parameters within the DNN architecture, and $t$ indicates a certain day of data. 
The previous DNAS workflow must iterate over $T$ days of data, with a significant search cost. More specifically, the large search cost originates from the following considerations in search efficiency and scalability:
\begin{itemize}[noitemsep,leftmargin=*]
    \item Sequentially iterating over $T$ days of data requires $T$ times the search cost of DNAS on a single day of data. This creates challenges for model freshness in production environments where $T$ can be extremely large. 
    \item Deploying the search over multiple devices may suffer from poor scalability due to communication. For example, the forward/backward process needs to shift from model parallelism to data parallelism when offloading tensors from embedding table shards to feature interaction modules.
    \item Within our implementation on Criteo Terabyte, the throughput on multiple NVIDIA A5000 GPUs is lower than the throughput on a single GPU during search, as demonstrated in the Queries-Per-Second (QPS) analysis in Figure \ref{fig:dnas_qps}. 
    Thus, it is difficult to realize good scalability in the growth of computing devices in DNAS.
\end{itemize}
 The above considerations point to a distributed version of DNAS, where we partition the training data, launch a DNAS procedure on each day of training data, and average the results to derive the final architecture.
 We hereby propose a DistDNAS search with the following bilevel optimization.
\begin{equation}
\small
    (\mathrm{A}^{*}, \mathrm{B}^{*})^{(t)} = \argmin_{\mathrm{A}, \mathrm{B}} LogLoss_{x \sim D_{t}}[x; w^{*}(\mathrm{A}, \mathrm{B}), \mathrm{A}, \mathrm{B}],
\end{equation}
such that
\begin{equation}
\small
    w^{*}(\mathrm{A}, \mathrm{B})^{(t)} = \argmin_{w} LogLoss_{x \sim D_{t}}[x; w(\mathrm{A}, \mathrm{B}), \mathrm{A}, \mathrm{B}].
\end{equation}


\begin{figure}[t]
    \begin{center}
    \includegraphics[width=0.8\linewidth]{figs/dnas_qps_gain.pdf}
    \vspace{-1em}
    \caption{QPS comparison between DistDNAS and DNAS.}
    \label{fig:dnas_qps}
    \end{center}
    \vspace{-1em}
\end{figure}

Here, $t \in \{1, 2, ..,, T\}$ indicates a certain day of training data. 
DistDNAS aggregates the learned weights on each day to retrieve the final architecture coefficients with a simple averaging aggregator as follows:
\begin{equation}
\small
    (\mathrm{A}^{*}, \mathrm{B}^{*})^{(*)} = \sum_{t=1}^{T} \frac{1}{T}(\mathrm{A}^{*}, \mathrm{B}^{*})^{(t)}
\end{equation}
The simple averaging scheme incorporates the statistics from each day of data to obtain the learned architecture weights. 
In addition, DistDNAS can be asynchronously paralleled on different computing devices, accelerating the scalability of search and reducing the total wall-clock run-time in recommender systems. Figure \ref{fig:dnas_qps} presents a comparison of DistDNAS versus DNAS on 1-8 NVIDIA A5000 GPUs, with 4K batch size. Due to communication savings with 1-GPU training, DistDNAS benefits from significantly lower search cost compared to vanilla DNAS.



\begin{figure}[b]
    \vspace{-2em}
    \begin{center}
    \includegraphics[width=0.9\linewidth]{figs/FLOPS_importance.pdf}
    \vspace{-1em}
    \caption{Normalized cost importance in a 7-block supernet.}
    \label{fig:flops_feature_interactions}    
    \end{center}
\end{figure}


\subsection{Cost-aware Regularization}
Serving cost of feature interactions, e.g. FLoating-point OPerations (FLOPs), is critical in recommender systems. A lower servicing cost indicates a shorter response time to process a user query request. As a result, optimizing the cost of recommender models is as important as optimizing the performance of recommender models in the production environment.

In DNAS, we measure the cost as a combination of training FLOPs and inference latency.
The cost of a feature interaction in discovery is dependent on the weights of the learned architecture $C^{*}=(A^{*}, B^{*})$. An intuition to optimize the feature interaction module is rewarding cost-efficient operators (e.g., Linear, Identity) while penalizing cost-inefficient operators (e.g., Transformer, CrossNet) during differentiable search.
Motivated by this, we introduce a differentiable cost regularizer to penalize large models in discovery.
The cost regularizer adds an additional regularization term $R$ to the loss function during DNAS to induce cost-effective feature interactions in discovery.


We use $j$ to represent an index of a feature interaction module in $\mathcal{C}$, for example, the index of a dense interaction module.
We first sample a few pairs of architecture and cost metrics from the DistDNAS search space and create a cost mapping $cost: \mathrm{C} \to \mathrm{R}$ to model the relationship between feature interactions and FLOPs.
Then, we use the permutation importance~\cite{breiman2001random} to obtain the importance of offline cost $s_{j}$ in the cost mapping $cost$, illustrating the offline FLOPs importance of an interaction module $i$. 
Finally, we formulate a cost-aware loss and incorporate it to regularize all interaction modules: $R(\mathrm{A}, \mathrm{B}) = \gamma \sum_{op^{j} \in (\mathrm{A}, \mathrm{B})}s_{j}$.
Here, $\gamma$ is an adjustable coefficient to control the strength of cost-aware regularization. With cost-aware regularization, the final architecture of DNAS with cost-aware loss searched on a single day $t$ can be formulated as follows.
\begin{equation}
    \small
    (\mathrm{A}^{*}, \mathrm{B}^{*})^{(t)} = \argmin_{\mathrm{A}, \mathrm{B}} LogLoss_{x \sim D_{t}}[x; w^{*}(\mathrm{A}, \mathrm{B}), \mathrm{A}, \mathrm{B}] + R(\mathrm{A}, \mathrm{B}),
\end{equation}
such that 
\begin{equation}
\small
    w^{*}(\mathrm{A}, \mathrm{B})^{(t)} = \argmin_{w} LogLoss_{x \sim D_{t}}[x; w(\mathrm{A}, \mathrm{B}), \mathrm{A}, \mathrm{B}] + R(\mathrm{A}, \mathrm{B}).
\end{equation}

As the feature interaction search space adopts a fixed connectivity and dimension configuration during the search, the cost importance of different interaction modules is unique in the first choice block, and identical across all other choice blocks.
We demonstrate the normalized cost importance of each interaction module in Figure \ref{fig:flops_feature_interactions}. 
Among all interaction modules, the DotProduct contributes to a significant amount of FLOPs consumption by integrating dense and/or sparse features. Except for Transformer, sparse interaction modules contribute significantly fewer serving costs compared to their dense counterparts.
Thus, despite the strong empirical performance of Transformer models, recommender models choose Transformer sparingly to build an efficient feature interaction.


\begin{table*}[t]
    \vspace{-1em}
    \begin{center}
    \caption{Performance of the best discovered DistDNAS model on 1TB Criteo Terabyte.}
    \vspace{-1.0em}
    \scalebox{1.0}{
    \begin{tabular}{|c|c|c|c|c|c|c|}
    \hline
    \textbf{Model} & \textbf{FLOPS(M)} & \textbf{Params(M)} & \textbf{NE (\%)} $\downarrow$ & \textbf{Relative NE (\%)} $\downarrow$ & \textbf{AUC} $\uparrow$  & \textbf{LogLoss} $\downarrow$   \\
    \hline \hline
    DLRM~\cite{naumov2019deep} & 1.79 & 453.60 & 0.8462 & 0.0 & 0.8017 & 0.12432  \\
    DeepFM~\cite{guo2017deepfm} & 1.81 & 453.64 & 0.845 & -0.14 & 0.8028 & 0.12413 \\
    xDeepFM~\cite{lian2018xdeepfm} & 6.03 & 454.14 & 0.846 & -0.02 & 0.8023 & 0.12429 \\
    AutoInt~\cite{song2019autoint} & 2.16 & 453.80 & 0.8455 & -0.08 & 0.8024 & 0.12421 \\
    DCN-v2~\cite{wang2021dcn} & 8.08 & 459.91 & 0.845 & -0.14 & 0.8031 & 0.12413  \\
    xDeepInt~\cite{yan2023xdeepint} & 8.08 & 459.91 & 0.8455 & -0.08 & 0.8027 & 0.12421 \\
    \hline
    NASRec-tiny~\cite{zhang2022nasrec} & 0.57 & 452.47 & 0.8463 & 0.01 &  0.8014 & 0.12437 \\
    AutoCTR-tiny~\cite{song2020towards} & 1.02 & 452.78 & 0.8460 & -0.02 & 0.8017 & 0.12429 \\
    \multirow{2}{*}{DistDNAS} & 1.97 & 453.62 & \textbf{0.8448} & \textbf{-0.17} & 0.8030 & \textbf{0.12410} \\
    & $3.11^{*}$ & 454.70 & \textbf{0.8444} & \textbf{-0.21} & \textbf{0.8032} & \textbf{0.12405} \\
    DistDNAS (M=2) & 3.94 & 455.48 & \textbf{0.8448} & \textbf{-0.17} & \textbf{0.8033} & \textbf{0.12410} \\
    DistDNAS (M=3) & 5.90 & 457.31 & \textbf{0.8440} & \textbf{-0.26} & \textbf{0.8035} & \textbf{0.12399} \\
    DistDNAS (M=4) & 7.87 & 459.14 & \textbf{0.8438} & \textbf{-0.29} & \textbf{0.8039} & \textbf{0.12395} \\
    \hline
    \end{tabular}
    \label{tab:criteo_terabyte_tab} 
    }
    \end{center}
    \vspace{-1em}
\end{table*}
\section{Experiments}
We thoroughly examine DistDNAS on Criteo Terabyte.
We first introduce the experiment settings of DistDNAS that produce efficient feature interaction in discovery.
Then, we compare the performance of models crafted by DistDNAS versus a series of metrics with strong hand-crafted baselines and AutoML baselines.

% We further evaluate the best architectures found on Criteo Terabyte on small-scale CTR benchmarks, Criteo Kaggle\footnote{\hyperlink{https://www.kaggle.com/c/criteo-display-ad-challenge}{https://www.kaggle.com/c/criteo-display-ad-challenge}}, Avazu\footnote{\hyperlink{https://www.kaggle.com/c/avazu-ctr-prediction/data}{https://www.kaggle.com/c/avazu-ctr-prediction/data}}, and KDD Cup 2012~\footnote{\hyperlink{https://www.kaggle.com/c/kddcup2012-track2/data}{https://www.kaggle.com/c/kddcup2012-track2/data}} to demonstrate the transferability of our searched models.

\subsection{Experiment Setup} 
We illustrate the key components of our experiment setup and elaborate on the detailed configuration.

\noindent \textbf{Training Dataset.}
Criteo Terabyte contains 4B training data on 24 different days. Each data item contains 13 integer features and 26 categorical features.
Each day of data on Criteo Terabyte contains $\sim$ 0.2B data.
During the DistDNAS search, we use data from day 1 to day 22 to learn architecture the optimal architecture weights: $C^{*}=(A^{*}, B^{*})$. During the evaluation, we use the data from day 1 to day 23 as a training dataset and use \textbf{day 24} as a holdout testing dataset. We perform inter-day data shuffling during training, yet iterate over data from day 1 to day 23 in a sequential order.

\noindent \textbf{Data Preprocessing.} We do not apply any special preprocessing to dense features except for normalization. For sparse embedding tables, we cap the maximum embedding table size to 5M, and use an embedding dimension of 16 to obtain each sparse feature. Thus, each model contains $\sim$450M parameters in the embedding table.

\noindent \textbf{Optimization.} 
We train all models from scratch without inheriting knowledge from other sources, such as pre-trained models or knowledge distillation.
We use different optimizers for sparse embedding parameters and dense parameters (e.g., other parameters except for sparse embedding parameters). For sparse parameters, we utilize Adagrad with a learning rate of 0.04. For dense parameters, we use Adam with a learning rate of 0.001. 
No weight decay is performed.
During training, we use a fixed batch size of 8192, with a fixed learning rate schedule after initial warm-up. We enable Auto-Mixed Precision (AMP) to speed up training.

\noindent \textbf{Architecture Search.}
Our supernet contains $N=7$ choice blocks during the search.
We choose $\gamma$=0.004 for cost-aware regularization to balance trade-offs between performance.
During the search, we linearly warm up the learning rate from 0 to maximum with 10K warm-up steps, and use a batch size of 8K to learn the architecture weights while optimizing the DNAS supernet.
Each search takes $\sim$ 2 GPU hours on an NVIDIA A5000 GPU.

\noindent \textbf{Discretization.}
During discretization, we only attempt on 0.25/0.2, and select the discovered feature interaction with a better performance/FLOPs trade-off as the product of the search. 
As most baseline models are larger, we naively stack $M$ copies of feature interactions in parallel to match the FLOPs of large models, such as DCN-v2~\cite{wang2021dcn} and xDeepInt~\cite{yan2023xdeepint}. In Table \ref{tab:criteo_terabyte_tab}, we use $\theta$=0.20 as the discretization threshold for DistDNAS marked with $*$, and use $\theta=0.25$ to other feature interactions created by DistDNAS. We use the discovered feature interaction to connect raw dense/sparse inputs and craft a recommender model as the product of search. 


\iffalse
\noindent  \textbf{Small-scale CTR tasks.} We use Criteo, Avazu, and KDD Cup 2012 as a set of small-scale CTR tasks.
We follow the data processing protocol in NASRec~\cite{zhang2022nasrec}. Specifically, we use a split of 80\% training data, 10\% validation data, and 10\% test data for each benchmark. We use Stratified K-fold to obtain the dataset splits, ensuring a balanced distribution of positive/negative labels. We use training and validation split to train the best discovered model, and report LogLoss/AUC on the test split.
\fi


\noindent  \textbf{Training.} To ensure a fair comparison and better demonstrate the strength of the discovered models, we employ the aforementioned optimization methodologies without hyperparameter tuning.
We linearly warm up the learning rate from 0 to maximum using the first 2 days of training data.
To prevent overfitting, we use single-pass training and iterate the whole training dataset only once.

\noindent \textbf{Baselines.} We select the popular hand-crafted design choice of CTR models from the existing literature to serve as baselines, i.e., DLRM~\cite{naumov2019deep}, DeepFM~\cite{guo2017deepfm}, xDeepFM~\cite{lian2018xdeepfm}, AutoInt~\cite{song2019autoint}, DCN-v2~\cite{wang2021dcn} and xDeepInt~\cite{yan2023xdeepint}. 
We also incorporate the best models from the NAS literature: AutoCTR~\cite{song2020towards} and NASRec~\cite{zhang2022nasrec} to serve as baselines and use the best model discovered for Criteo Kaggle.

Without further specification, 
all hand-crafted or AutoML baselines use $dim_{s}=16$ as the embedding dimension.
All hand-crafted or AutoML baselines use 512 or $dim_{d}=256$ units in the MLP layer, including 1 MLP layer in dense feature processing and 7 MLP layer in aggregating high-level dense/sparse features.
All AutoML models use $N_{s}=16$ for sparse interaction modules.
This ensures a fair comparison between hand-crafted and AutoML models, as the widest part in hand-crafted/AutoML models does not exceed 512.
All hand-crafted feature interactions (e.g., CrossNet) are stacked 7 times to match $N=7$ blocks in the AutoML supernet, as NASRec, AutoCTR, and proposed DistDNAS employ $N=7$ blocks for feature interaction. As a result, the FLOPs cost of AutoCTR~\cite{song2020towards} and NASRec~\cite{zhang2022nasrec} reduces significantly. We name the derived NAS baselines \textbf{NASRec (tiny)} and \textbf{AutoCTR (tiny)}. We implement all of the baseline feature interactions based on open-source code and/or paper demonstration. 

\subsection{Evaluation on Criteo Terabyte}
We use DistDNAS to represent the performance of the best models discovered by DistDNAS and compare performance against a series of cost metrics such as FLOPs and parameters. 
We use AUC, Normalized Entropy (NE)~\cite{he2014practical}, and LogLoss as evaluation metrics to measure model performance.
We also calculate the testing NE of each model relative to DLRM and demonstrate relative performance. Note that relative NE is equivalent to relative LogLoss on the same testing day of data. 
Table \ref{tab:criteo_terabyte_tab} summarizes our evaluation of DistDNAS.
Here, $M$ indicates the number of parallel stackings we apply on DistDNAS to match the FLOPs of baseline models.

Upon transferring to large datasets, previous AutoML models~\cite{song2020towards,zhang2022nasrec} searched on smaller datasets are less competitive when applied to large-scale Criteo Terabyte. This is due to sub-optimal architecture transferability from the source dataset (i.e., Criteo Kaggle) to the target dataset (i.e., Criteo Terabyte). Among all baseline models, DCN-v2 achieves state-of-the-art performance on Criteo Terabyte with the lowest LogLoss/NE and highest AUC. Despite the same parameter count and FLOPs, xDeepInt shows 0.06\% NE degradation, due to the sub-optimal interaction design compared to the cross-net interaction modules.

DistDNAS shows remarkable model efficiency by establishing a new Pareto frontier on AUC/NE versus FLOPs.
With a discretization threshold of 0.25, the 1.97M DistDNAS model outperforms tiny baseline models such as DLRM and DeepFM, unlocking at least 0.02\% AUC/NE with on-par FLOPs complexity.
With a discretization threshold of 0.2, we achieve better AUC/NE as state-of-the-art DCN-v2 models, yet with a reduction of over 60\% FLOPS.
By naively stacking more blocks in parallel, DistDNAS achieves the state-of-the-art AUC/NE and outperforms DCN-v2 by 0.001 AUC.


\begin{figure}[t]
    \begin{center}
    \includegraphics[width=0.9\linewidth]{figs/dist_vs_vanilla.pdf}
    \vspace{-1em}
    \caption{Comparison of learned architecture weights of dense interaction modules in choice block 1.}
    \vspace{-2em}
    \label{fig:dist_vs_vanilla}    
    \end{center}
\end{figure}

\section{Discussion}
In this section, we conduct ablation studies and analyze various confounding factors within DistDNAS, including the effect of distributed search, the effect of cost-aware regularization, and the analysis of performance on state-of-the-art models under a recurring training setting.

\subsection{DistDNAS Search Strategy}
DistDNAS proposes distributed search and cost-aware regularization as the main contribution. Upon applying DNAS to the recommender systems, we discuss the alternative choices to DistDNAS as follows.

\noindent \textbf{SuperNet} indicates a direct use of the DistDNAS supernet as a feature interaction module. No search is performed.

\noindent \textbf{Distributed DNAS} applies the DistDNAS search process in a distributed manner, but does not involve cost-aware regularization. Both distributed DNAS and DistDNAS take 2 hours to complete on NVIDIA A5000 GPUs.

\noindent \textbf{One-shot DNAS} kicks off DNAS and iteratively over the entire search data set (that is, 22 days in Criteo Terabyte) to obtain the best architecture. Despite the inefficiency of the search discussed in Section \ref{sec4:dist}, a single shot DNAS on multiple days of training data cannot converge to a standalone architecture, with loss divergence during training on a large-scale dataset. Running a one-shot DNAS takes $\sim$ 50 GPU hours on an NVIDIA A5000 GPU.

\noindent \textbf{Fresh DNAS} only covers the most recent data in the training data set (i.e. day 22 on Criteo Terabyte) and performs a search to learn the best architecture. Intuitively, this serves as a strong baseline, as the testing data set is more correlated with the most recent data due to model freshness.
We compare the learning dynamics of the architecture weights for fresh DNAS (Day-22) versus Distributed DNAS.
Thanks to the averaging mechanism, Distributed DNAS benefits from more robust learning dynamics and shows smoother progress toward the learned architecture weights, see Figure \ref{fig:dist_vs_vanilla}. 

\begin{figure*}[t]
    \begin{center}
    \includegraphics[width=0.95\linewidth]{figs/DistDNAS_Flops_Vis.pdf}
    \vspace{-0.5em}
    \caption{Comparison of learned architecture weights under distributed DNAS versus DistDNAS.}
    \vspace{-1em}
    \label{fig:cost_ablation}    
    \end{center}
\end{figure*}

\noindent \textbf{DistDNAS} applies all the techniques proposed in this paper, including distributed search and cost-aware regularization. Figure \ref{fig:cost_ablation} shows a comparison of the learned architecture weights in Distributed DNAS versus DistDNAS.
DistDNAS is more likely to preserve cost-efficient interaction modules, such as EmbedFC/Identity compared to distributed DNAS without cost-aware regularization.


We perform each of the aforementioned searches and evaluate different search strategies based on the following questions:
\begin{itemize}[noitemsep,leftmargin=*]
    \item \textbf{(Search Convergence) }Whether the search converges to a stable architecture weight and produces a feature interaction in discovery?
    \item \textbf{(Training Convergence)} Does the discovered feature interaction converge on a large-scale Criteo Terabyte benchmark with 24 days of training data?
    \item \textbf{(Testing Quality)} What is the quality of the interactions of the features discovered?
\end{itemize}


\begin{table}[t]
\begin{center}
    \caption{Study of different DistDNAS search strategies.}
    \vspace{-1em}
    \begin{scalebox}{0.96}{
        \begin{tabular}{|c|c|c|c|}
        \hline
        \multirow{2}{*}{\textbf{Strategy}} & \textbf{Searching} & \textbf{Training} & \textbf{Testing}  \\
        & \textbf{Converge?} & \textbf{Converge?} & \textbf{FLOPs/NE} \\
        \hline
        \textbf{SuperNet} & N/A & No & N/A \\
        \textbf{One-shot DNAS} & No & No & N/A \\
        \textbf{Freshness DNAS} & Yes & No & N/A \\
        \textbf{Distributed DNAS} & Yes & Yes & 3.56M/0.8460 \\
        \textbf{DistDNAS} & Yes & Yes & 3.11M/0.8444 \\
        \hline
    \end{tabular}
    }    
    \end{scalebox}
    \label{tab:distdnas_strategy}    
\end{center}
\vspace{-1em}
\end{table}


Table \ref{tab:distdnas_strategy} summarizes a study of different search strategies for these questions. We have a few findings regarding the use of distributed search and the use of cost-aware regularization.

\begin{figure}[b]
    \vspace{-1em}
    \begin{center}
    \includegraphics[width=1.0\linewidth]{figs/ablation_daily_ne.pdf}
    \vspace{-1em}
    \caption{Comparison of AUC and Relative NE on Criteo terabyte under recurring training setting.}
    \label{fig:ablation_ne_recurring}    
    \end{center}
\end{figure}

A standalone supernet cannot converge when trained on Criteo Terabyte dataset. This indicates that varying feature interactions may have conflicts with each other, increasing the difficulty of performing search and identify promising feature interactions.

On a large-scale dataset such as Criteo Terabyte, distributing DNAS over multiple-day splits and aggregating the learned weight architectures are critical to the convergence of search and training.
This is because in recommender systems, there might be an abrupt change in different user behaviors across/intra-days; thus, a standalone architecture learned on a single day may not be suitable to capture the knowledge and fit all user-item representations. Additionally, as NAS may overfit the target dataset, a standalone feature interaction searched on day $X$ may not be able to learn day $Y$ well and is likely to collapse due to changes in user behavior.

We also compare distributed DNAS with DistDNAS to demonstrate the importance of cost-aware regularization. 
Experimental evaluation demonstrates that FLOPS-regularization enhances the performance of searched feature interaction, removing the redundancy contained in the supernet. This observation provides another potential direction for recommender models to compress unnecessary characteristics and derive better recommender models, such as the usage of pruning. A more recent study using an instance-guided mask~\cite{wang2021masknet} supports the feasibility of applying model compression to advance performance on recommender models.


\subsection{Performance Analysis under Recurring Training Scenario}
Recurring training~\cite{he2014practical} is a common practice in recommender system applications. In recurring training, practitioners must regularly update the model on the latest data to gain fresh knowledge. Here, we simulate the scenario in recurring training to evaluate top-performing feature interactions (i.e., DistDNAS and DCN-v2) on different training/evaluation splits of Criteo Terabyte. More specifically, we use day 1 to day $t$ as training dates, and day $t+1$ as testing date to report AUC and relative NE. 
In Criteo Terabyte, $t$ can choose from \{2, 3, 4,..., 24\}.

We demonstrate the evaluation of recurring training on DistDNAS, DCN-v2 and DLRM (baseline) in Figure \ref{fig:ablation_ne_recurring}.
Although performing better on the final testing date, DistDNAS consistently outperforms the previous state-of-the-art DCN-v2 on all testing splits under recurring training. This indicates that DistDNAS successfully injects the implicit patterns contained within the large-scale dataset into the searched feature interaction, learning a better prior to gain knowledge from a large amount of training data.
As DistDNAS can be highly paralleled with only a single pass of the dataset, we envision that it can be beneficial for large-scale applications in search efficiency and serving efficiency.  





\iffalse
\subsection{Architecture Transferability Analysis}
\noindent \textcolor{red}{FixME}
We show the transferability of search feature interactions on three alternative small-scale CTR benchmarks:
Criteo Kaggle\footnote{\hyperlink{https://www.kaggle.com/c/criteo-display-ad-challenge}{https://www.kaggle.com/c/criteo-display-ad-challenge}}, Avazu\footnote{\hyperlink{https://www.kaggle.com/c/avazu-ctr-prediction/data}{https://www.kaggle.com/c/avazu-ctr-prediction/data}}, and KDD Cup 2012~\footnote{\hyperlink{https://www.kaggle.com/c/kddcup2012-track2/data}{https://www.kaggle.com/c/kddcup2012-track2/data}}. We demonstrate the evaluation of LogLoss and AUC of different models in Table \ref{tab:ctr_results}. 
We reuse the NASRec training pipeline~\citep{zhang2022nasrec} to ensure a fair comparison.

The three small CTR benchmarks contain different dense/sparse features with unique priors.
Thus, prior state-of-the-art performs well on certain CTR benchmarks (e.g., DCN-v2 on Criteo Kaggle), while performing less competitively when transferred to alternative CTR benchmarks. 
Yet, DistDNAS achieves state-of-the-art results on all three benchmarks without further search or fabrication, thanks to the distribution of search in the DistDNAS that aggregates the feature representation from different days of data.
With similar log loss (i.e. 0.02\% improvement over DCN-v2), DistDNAS saves $\sim$35\% FLOPs, thanks to the joint force of flexible discretization choice within feature interaction supernet and FLOP-aware regularization. \noindent \textcolor{red}{Tunhou: FIXME.}


\begin{table}[t]
    \begin{center}
    \caption{Performance of NASRec on General CTR Predictions Tasks. We use \textbf{bold} to mark the best result one each benchmark and use \underline{underline} to mark the second-best result.}
    \vspace{-0.5em}
    \scalebox{0.78}{
    \begin{tabular}{|c|cc|cc|cc|}
    \hline
     \multirow{2}{*}{\textbf{Model}} & \multicolumn{2}{|c|}{\textbf{Criteo}}  &  \multicolumn{2}{|c|}{\textbf{Avazu}} & \multicolumn{2}{|c|}{\textbf{KDD Cup 2012}}  \\
       &  LogLoss  $\downarrow$ & AUC $\uparrow$ & LogLoss $\downarrow$ & AUC $\uparrow$ & LogLoss $\downarrow$ & AUC $\uparrow$ \\
    \hline \hline
    DLRM~\cite{naumov2019deep} & 0.4406 & 0.8107  & 0.3756 & 0.7872 & 0.1494 & 0.8145 \\
    AutoInt~\cite{song2019autoint} & 0.4452 & 0.8055 & 0.3812 & 0.7778  & 0.1498 & 0.8121  \\
    DeepFM~\cite{guo2017deepfm} & 0.4403 & 0.8110 & 0.3797 & 0.7801  & 0.1494 & 0.8141   \\
    xDeepFM~\cite{lian2018xdeepfm} & 0.4408 & 0.8105  & 0.3781 & 0.7830 & 0.1495 & 0.8135    \\
    DCN-v2~\cite{wang2021dcn} & 0.4382 & 0.8132   & 0.3782 & 0.7825 & 0.1498 & 0.8119 \\
    xDeepInt~\cite{yan2023xdeepint} & 0.4393 & 0.8121 & 0.3778 & 0.7833 & 0.1489 & 0.8162  \\
    \hline
    NASRec~\cite{zhang2022nasrec} & 0.4395 & 0.8118 & 0.3748 & 0.7886 & 0.1493 & 0.8151\\
    AutoCTR~\cite{song2020towards} & 0.4397 & 0.8117 & \underline{0.3742} & \underline{0.7895} & \underline{0.1489} & \underline{0.8165} \\
    DistDNAS & \textbf{} & \textbf{} & \textbf{} & \textbf{} & \textbf{} & \textbf{}  \\
    \hline
    \end{tabular}
    \label{tab:ctr_results} 
    }
    \vspace{-2em}
    \end{center}
\end{table}
\fi
\section{Conclusion}
In this article, we emphasize search efficiency and serving efficiency in the design of feature interactions through a differentiable supernet. 
We propose DistDNAS to explore the differentiable supernet containing various dense and sparse interaction modules.
We distribute the search on different days of training data to advance search scalability, reducing end-to-end search cost from 2 days to 2 hours with a 25$\times$ speed-up in scalability.
In addition, DistDNAS incorporates cost-aware regularization to remove potential conflicts and redundancies within feature interaction modules, yielding higher serving efficiency in searched architectures.
Our empirical evaluation justifies the search efficiency of DistDNAS on Criteo Terabyte dataset, reducing search cost from 2 days to 2 hours.
Despite search efficiency, DistDNAS discovers an efficient feature interaction to benefit serving, advancing AUC by 0.001 and surpassing existing hand-crafted/AutoML interaction designs.

\bibliographystyle{acm_ref_format}
\bibliography{references}

\end{document}
