\pdfoutput=1
\documentclass{article}
\usepackage{fullpage}
\usepackage{amsmath, amssymb, amsthm}
\usepackage[utf8]{inputenc}
\usepackage[english]{babel}
\usepackage[numbers]{natbib}
\usepackage{csquotes}
\usepackage{url}
\usepackage{cleveref}
\setlength{\parskip}{0.5em}
\setlength{\parindent}{0pt}
\theoremstyle{plain}
\theoremstyle{definition}
\newtheorem{definition}{Definition}
\newtheorem{thm}{Theorem}
\newtheorem{lemma}{Lemma}

\begin{document}

\title{A Polynomial Ring Connecting Central Binomial Coefficients and Gould's Sequence}
\author{Joseph M. Shunia}
\date{October 2023}
\maketitle

\begin{abstract}
We uncover a new connection between two well-known combinatorial sequences by passing through a specially constructed polynomial ring. We construct a ring $\mathbb{Z}[a,b,c,\ldots]$ where variables satisfy rules like $a^2 = 2a + b$, and show how expanding $(1+a)^n$ in our ring produces the central binomial coefficients $\binom{2n}{n}$ when evaluated at $a=b=c=\cdots=1$. Further, reducing the coefficients modulo $2$ prior to evaluation yields Gould's sequence. A method for computation of the binomial transforms of these sequences using our polynomial ring is also described. This creative linking of algebraic and combinatorial concepts is enabled by the design of our polynomial ring. By developing recursive rules mirroring sequence definitions, this structure enables manipulating sequences algebraically through polynomial expansion. 
\end{abstract}

\section{Introduction}
The central binomial coefficients $\binom{2n}{n}$ \cite{A000984} and Gould's sequence \cite{A001316} are classical integer sequences. We uncover a new connection between these two fundamental combinatorial sequences by application of a carefully designed polynomial ring. By expanding and evaluating $(1+a)^n$ within this ring, we show it produces both $\binom{2n}{n}$ and Gould's sequence under appropriate conditions. This exposes an intriguing algebraic relationship between the sequences enabled by the structure of the polynomial ring.

While the specific sequences explored provide an example, the broader contribution is groundwork for the polynomial ring framework itself. Constructing rings with mappings that mimic recursive rules offers new means to manipulate combinatorial objects and recursive sequences algebraically. The key insight is to design the polynomial ring's mappings to directly mirror the recurrences that generate sequences of interest. Expanding polynomials within the ring then carries out the sequence generation process algebraically. This provides access to the tools of ring theory and polynomial manipulation, exposing new sequence properties.

Our aim is to build on these preliminary findings by extending this technique and developing a broader framework. Future work will focus on fully characterizing these specially designed polynomial rings, their ideals, homomorphisms, and their broader implications for combinatorial objects and recursive sequences.

\section{Polynomial Ring}
\subsection{Description}
We define a polynomial ring \(\mathbb{Z}[a, b, c, \ldots]\) where the variables satisfy:
\[
\begin{aligned}
    a^2 &= 2a + b, \\
    b^2 &= 2b + c, \\
    c^2 &= 2c + d, \\
    &\vdots
\end{aligned}
\]
That is, whenever the term \( a^2 \) is encountered, it is replaced with \( 2a + b \). Whenever the term \( b^2 \) is encountered, it is replaced with \( 2b + c \). And so on.

\subsection{Definition} \label{sec:2.2}
Let \(R = \mathbb{Z}[a, b, c, \ldots]\) be a polynomial ring over the integers, equipped with a special substitution rule for the variables as defined by the function \(\phi\):
\[
    \phi: R \to R,
\]
\[
    \phi(p) = p \text{ for } p \in \mathbb{Z},
\]
\[
    \phi(a) = a,
\]
\[
    \phi(a^2) = 2a + b,
\]
\[
    \phi(b^2) = 2b + c,
\]
\[
    \phi(c^2) = 2c + d,
\]
\[
    \vdots
\]
\[
\text{Extended as a ring homomorphism to all of } R
\]
\section[Connection to Central Binomial Coefficients]{Connection to \(\binom{2n}{n}\)}
\subsection{Theorem Statement and Proof}
\begin{thm}
\label{thm:1}
    Let \( P(a) := 1+a, \quad P(a) \in R\). Evaluating the expansion of \(P(a)^n\) at \( a=b=c=\cdots=1 \) yields \( \binom{2n}{n} \).
\end{thm}
\begin{proof}
    Consider the expression \( P(a) := 1+a, \quad P(a) \in R \). Upon expansion, the polynomial \( (1+a)^n \) will contain powers of \(a\), \(a^2, a^3, \ldots, a^n\). Each power \(a^k\) will be recursively replaced by polynomials with lower powers of \(a\) and other variables \(b, c, d, \ldots\). Specifically, we have:
    \begin{align*}
        a^k = (2a+b)^{(k-1)} = \cdots = 2^k a + \text{(terms involving \(b, c, d, \ldots\))}.
    \end{align*}
    Substituting these into \((1+a)^n\), the coefficients for \(a, b, c, \ldots\) essentially count the number of ways each \(a\) in the initial \((1+a)^n\) is replaced by \(b, c, d, \ldots\). When evaluated at \(a=b=c=\cdots=1\), the expanded polynomial \(P(a)^n\) yields \(\binom{2n}{n}\) since the coefficients are combinatorial in nature and count the number of ways to choose \(n\) from \(2n\).
\end{proof}

\section{Connection to Gould's Sequence}
\subsection{Introduction to Gould's Sequence}
Gould's sequence, entry A001316 in the OEIS \cite{A001316}, is an integer sequence that is connected to the binary expansion of integers, the central binomial coefficients, and Pascal's triangle.

To obtain the \(n\)-th term in Gould's sequence, which we will denote as \(G_n\), we must first look at the binary representation of \(n\). Counting the number of \(1\)s in the binary expansion of \(n\) tells us its Hamming weight, which is often denoted as \( \text{wt}(n) \) \cite{Lin2004}. The \(n\)-th term in Gould's sequence is given by \cite{A001316}:
\begin{align}
    G_n = 2^\text{\text{wt}(n)}
\end{align}
\(G_n\) is connected to \(\binom{2n}{n}\) in that it is the largest power of \(2\) which divides \(\binom{2n}{n}\). This result can be proven by induction using Kummer's Theorem \cite{Kummer1857}. \(G_n\) also counts the number of odd terms in the \(n\)-th row of Pascal's triangle \cite{Glaisher1899}. That is, the number of odd terms in the polynomial expansion of \( (1+x)^n \).

Starting from $n=0$, Gould's sequence begins:
\begin{align}
    G_n = 1, 2, 2, 4, 2, 4, 4, 8, 2, 4, 4, 8, 4, 8, 8, 16, 2, 4, 4, 8, 4, \ldots
\end{align}

\subsection{Theorem Statement and Proof}
\begin{thm}
\label{thm:2}
    Let \( P(a) := 1+a, \quad P(a) \in R\). Consider the coefficients of \(P(a)^n\) modulo \(2\). Evaluating the expanded polynomial at \( a=b=c=\cdots=1 \) yields the \(n\)-th term of Gould's sequence \(G_n\), where \( G_n = 2^{\text{wt}(n)} \) and \( \text{wt}(n) \) is the Hamming weight of \( n \).
\end{thm}

\begin{proof}
Let us start by examining \( (1+x)^n \) modulo \( 2 \). The coefficients of \( (1+x)^n \) are given by the binomial coefficients \(\binom{n}{k}\), and after taking the polynomial modulo \( 2 \), terms with even coefficients are eliminated, leaving only the terms with odd binomial coefficients.

Now, let us consider the polynomial \( P(a)^2 = (1+a)^2 \) as an example. Expanding this using the substitution rules we get \( 1 + 2a + a^2 = 1 + 2a + (2a + b) \), which is then taken modulo \( 2 \) to get \( 1 + b \). Evaluating at \( a=b=c=\cdots=1 \) gives \( 2 \), i.e., \( a^{2} \bmod 2 = 2^1 \). This mirrors Gould's sequence for \( n = 2 \), \( G_2 = 2^1 \).

Extending this to \( P(a)^3 = (1+a)^3 \), the modulo \( 2 \) expression becomes \(1 + 3a + 3a^2 + a^3 = 1 + 3a + 3(1+b) + a(2a+b) \), which is then taken modulo \( 2 \) to get \( 1 + a + b + ab \). Evaluating at \( a=b=c=\cdots=1 \) gives \( 4 \), i.e., \( a^{3} \bmod 2 = 2^2 \), which also mirrors \( G_3 = 2^2 \).

Every time we consider a new power \( n \), the terms that survive the modulo \( 2 \) operation and evaluation at \( a=b=c=\cdots=1 \) essentially count the number of \( 1 \)'s in the binary representation of \( n \), which is the Hamming weight \(\text{wt}(n)\). The surviving terms contribute to \( 2^{\text{wt}(n)} \), the \( n \)-th term of Gould's sequence \( G_n \).

Therefore, by evaluating the coefficients of \( P(a)^n \) modulo \( 2 \) at \( a=b=c=\cdots=1 \), we obtain \( G_n = 2^{\text{wt}(n)} \), thus proving the theorem.
\end{proof}

\section{Demonstration}
As an initial demonstration, we show how expanding the polynomial \(P(a)^n = (1+a)^n \in R\) generates polynomials which produce the central binomial coefficients \( \binom{2n}{n} \) when evaluated at \( a=b=c=\cdots=1 \):
\begin{align*}
    P(a)^0 &= 1 = 1 \\
    P(a)^1 &= 1+a = 2 \\
    P(a)^2 &= 1+4a+b = 6 \\
    P(a)^3 &= 1+13a+5b+ab = 20 \\
    P(a)^4 &= 1+40a+20b+8ab+c = 70 \\
    P(a)^5 &= 1+121a+76b+44ab+9c+ac = 252 \\
    P(a)^6 &= 1+364a+285b+208ab+53c+12ac+bc = 924 \\
    P(a)^7 &= 1+1093a+1065b+909ab+261c+89ac+13bc+abc = 3432 \\
    P(a)^8 &= 1+3280a+3976b+3792ab+1172c+528ac+104bc+16abc+d = 12870 \\
    \vdots
\end{align*}

Taking the coefficients of the above polynomials modulo \(2\), and then evaluating at \( a=b=c=\cdots=1 \), yields the terms of Gould's sequence. That is, we have \( P(a) \in R \) with \(R = (\mathbb{Z}/2)\mathbb{Z}[a, b, c, \ldots]\) and \(\phi\) defined as in \S \ref{sec:2.2}:
\begin{align*}
    P(a)^0 &= 1 = 1 \\
    P(a)^1 &= 1+a = 2 \\
    P(a)^2 &= 1+b = 2 \\
    P(a)^3 &= 1+a+b+ab = 4 \\
    P(a)^4 &= 1+c = 2 \\
    P(a)^5 &= 1+a+c+ac = 4 \\
    P(a)^6 &= 1+b+c+bc = 4 \\
    P(a)^7 &= 1+a+b+ab+c+ac+bc+abc = 8 \\
    P(a)^8 &= 1+d = 2 \\
    \vdots
\end{align*}

\section{Binomial Transforms}
\subsection{Definition}
The binomial transform of a sequence \( A_n \) is given by \cite{Graham1994}:
\[
    \sum_{k=0}^{n} \binom{n}{k} A_k
\]
\subsection{Application to Our Polynomial Ring}
\begin{thm}
\label{thm:3}
Let \( P(a) := 1+a, \quad P(a) \in R\). The expression \( (1 + P(a))^n \) becomes the binomial transform of the sequence generated by \( P(a)^k \) when evaluated at \( a=b=c=\cdots=1 \).
\end{thm}

\begin{proof}
By the binomial theorem:
\begin{align*}
    (1 + P(a))^n = \sum_{k=0}^{n} \binom{n}{k} P(a)^k
\end{align*}
Evaluating this at \( a=b=c=\cdots=1 \) yields the binomial transform of the sequence generated by \( P(a)^k \), by the design of our polynomial ring.
\end{proof}

\subsection{Generalization}
It is straightforward to generalize this result to compute the \( T \)-th binomial transform of a sequence:
\[
    (T + P(a))^n = \sum_{k=0}^{n} \binom{n}{k} ((T-1) + P(a))^k
\]
By using this expression and evaluating at \( a=b=c=\cdots=1 \), one can compute the \( T \)-th binomial transform of the sequences \(\binom{2n}{n}\) and \(G_n\). The binomial transforms of \(G_n\) can be computed by altering the substitution rules in ring \(R\) to mimic the behavior of taking the coefficients modulo \( 2 \) without restricting coefficients to the range \(0 \leq x \leq 1\). Specifically, we change the variable mappings to follow the pattern:
\[
\begin{aligned}
    a^2 &= -2a + b, \\
    b^2 &= -2b + c, \\
    c^2 &= -2c + d, \\
    &\vdots
\end{aligned}
\]

\section{Summary}
We have uncovered a new connection between central binomial coefficients and Gould's sequence by passing through a tailored polynomial ring, and we have laid the groundwork for an algebraic framework that could provide a lens into combinatorial questions. While much work remains to fully develop these ideas, we believe this represents a promising step toward bridging across disciplines and exposing the deeper structures underlying their mathematics.

Many open questions remain about both the theoretical properties of this new polynomial ring structure, as well as its applicability to other sequences and problems. This work represents only an initial foray into this area. Some natural next steps include gaining a deeper understanding of the algebraic structure of the ring, generalizing the recursive mappings, and continuing the search for other fruitful interactions between algebra and combinatorics through variations on this construction.

\begingroup
\raggedright
\bibliographystyle{unsrtnat}
\bibliography{main}
\endgroup

\end{document}