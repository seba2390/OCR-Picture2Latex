\begin{figure}[!t]
	\centering
	\subfloat[]{\includegraphics[width=0.42\textwidth]{fig/teaser.pdf}  \label{fig:teaser_1}} \hfill
	\subfloat[]{\includegraphics[width=0.52\textwidth]{fig/teaser2.pdf} \label{fig:teaser_2} }
% 	\vspace{-1ex}
	\caption{\label{fig:teaser}
    	Conceptual illustrations of trustworthiness prediction. 
    	(a) shows the process of predicting trustworthiness where the oracle is the trustworthiness predictor. 
    	The illustration in (b) shows that the task is challenging on ImageNet, where TCP's confidence loss \cite{Corbiere_NIPS_2019} is used in this example.
    	The confidence that is greater (lower) than the positive (negative) threshold would be classified as a trustworthy (untrustworthy) prediction.
    	Usually, both the positive threshold and the negative threshold are 0.5, but the negative threshold is $\frac{1}{\text{\# of classes}}$ in the case of TCP. 
    % 	Ideally,  we hope the trustworthiness confidences w.r.t. trustworthy predictions are on the right-hand side of positive threshold (\ie 0.5) and the trustworthiness confidences w.r.t. untrustworthy predictions are on the left-hand side of negative threshold $\frac{1}{\text{\# of classes}}$ in the case of TCP. Note that we slightly adjust the negative threshold on ImageNet for illustrative purposes. TP, FP, TN, and FN stand for true positive, false positive, true negative, and false negative, respectively.
    % 	Illustration of predicting trustworthiness
    % Illustration showing performance discrepancy between MNIST and ImageNet. 
    	}
\end{figure}