\section{Driving Use Cases for Next-Generation Networking Infrastructure} \label{sec:UseCases}
% \section{Driving Use Cases for NGN}

\noindent In-Network Computing has the potential to shape the Next-Generation Networking Infrastructure (NGNI) into an integrated computation and communication infrastructure that is  needed  to fulfill the stringent requirements of emerging applications. The enhancement of 3C integration throughout Cloud-Edge-Mist Continuum provides even further advantages to achieve this goal. Thus, to highlight the necessity of enabling INC with 3C integration, we present in this section some of the key use cases which prove the limits of current network infrastructure. We also emphasize, for each use case, the features that INC and 3C should provide to fulfill their requirements.

% Next-Generation Networking Infrastructure (NGNI) will explore new and innovative application use cases based on the convergence of different technologies (edge-cloud computing and in-network computing) and the integration of communication, computation, and caching functionalities. Some potential use cases for NGNI are multisensory extended reality (XR) applications, holographic and haptic applications, connected robots and autonomous systems, entertainment, and e-health. Although these cases have been proposed to be enabled by the current network (e.g., 4G/5G), the diversity and stringent requirements of those applications cannot yet be fully met. These high demands impose a reshaping of current network infrastructure towards an NGNI acting as a truly distributed, collaborative, and pervasive system that enables the execution of application-specific tasks and storage of data contents in the Cloud-Edge-Mist continuum with high QoS/QoE guarantees. In the remaining of this section, we present some driving use cases for NGNI and highlight their system requirements.

% \textcolor{blue} {Next-Generation Networking (NGN) will explore new and innovative application use cases based on the convergence of different technologies (edge-fog-cloud computing and in-network computing and caching) and the integration of communication, computation, and caching functionalities. Potential driven cases for NGN are multisensory extended reality (XR) applications, connected robots and autonomous systems, wireless brain-computer interfaces, holographic and haptic applications, entertainment, and e-health. In this section, some of these use cases are discussed.}

 \subsection{Multisensory Extended Reality (XR) Applications}
 
\noindent We are witnessing the evolution of AR/VR applications into a full immersive eXtended Reality (XR) experience that captures multi-modal human sensory information, including sight, hearing, touch, smell, taste, and even emotion~\cite{8869705,9369324}. The applications of XR include remote surgery, immersive education, and gaming.
%In this context, multi-modal human sensory information  should be exchanged between remote receivers in order to support real-time interactions between users in a combined real and virtual application environment (e.g., remote surgery, immersive gaming).

Multisensory XR applications require not only engineering (i.e., communication, computing, caching) requirements but also perceptual requirements (e.g., human senses, cognition, and physiology)~\cite{8869705,9369324}. In the engineering context, these applications require high data rates (1 -- 1000 Gbps), ultra-low latency (0.1 -- 1 ms), and high reliability (99.999 -- 99.99999 \%) due to the vast amount of data generated and human sensitivity~\cite{8329628,9369324}. However,  4G and 5G networks cannot support these applications because they do not incorporate the joint management of 3C functionality. 
Hence, NGNI has the potential to fulfill this void. Consequently, INC allows a certain extent of data analysis such as aggregation, de-duplication, filtering, and preprocessing to be performed \textit{inside} the network as the data traverses the network core. Performing these operations on-the-fly will help reduce the latency and decrease network load. Edge computing may complement INC to deliver content and offload computation-intensive tasks such as AR/VR video coding and decoding and content rendering, further improving the latency and throughput. Finally, collaboration among network nodes allows data content and computation tasks to be replicated and distributed within the network, increasing end-to-end reliability.

Enabling multisensory XR applications introduces several challenges to be overcome.  From a technical perspective,  cross-layer architectures and resource management schemes that seamlessly and efficiently integrate 3C functionality in the NGNI should be designed.  Accordingly, new data compression schemes, data streaming management, and synchronization are required and need further investigation.  Note that considering multisensory communications brings forth the question of how best to relay the meaning of the content rather than how to deliver symbols accurately, i.e., a subject currently investigated in the literature as \textit{Post-Shannon communications}~\cite{CALVANESESTRINATI2021107930}.

Furthermore, multisensory XR inherently involves strictly personal information and processing it inside the network requires further attention to privacy and legal aspects.  Additionally, differentiating the incoming traffic by processing them inside the network requires a revisit to open internet regulations. In particular, Internet Service Providers (ISPs) should have the necessary economic incentives to replace routers with INC devices.

%  \subsection{Wireless Brain-Computer Interfaces}
 
% Brain-Computer Interface (BCI) provides a new form of communication for humans by using their thoughts to interact and control external devices and the environment (e.g., turn on a light, control prosthetic limbs). BCI usually involves electrode sensors to record neuronal activity, communication technologies to transmit the signals, and algorithms to translate those signals into commands relayed to external devices that perform the desired actions. However, traditional BCI systems with wire connections to transmit the monitored brain’s signals are bulky and not user-friendly, thus restricting their use in experimental laboratories. Therefore, removing wire connections with wireless technologies enhances portability and wearability, enabling new and innovative applications (e.g., brain-controlled movies, fully-fledged multibrain-controlled cinemas~\cite{zioga2018enheduanna}). Like XR, wireless BCU requires high data rates, ultra-low latency, high reliability, as well as powerful computation capability~\cite{8869705,9369324}.

 \subsection{Holographic and Haptic Applications}
 
 \noindent Holographic and haptic communications are new forms of interactions expected in the future. Holographic communication consists of capturing, in real-time, information (e.g., images and sounds) of people and objects and sending them to remote locations where they will be projected on the real world using 3D holographic displays. Moreover, to fully realize an immersive experience, haptic feedback can be integrated into holographic applications. In a haptic interaction, haptic devices (e.g., sensors and actuators) capable of sensing and delivering cutaneous, haptic, and kinesthetic feedback enable users to feel, touch and manipulate virtual and remote objects. For instance, remote repair applications can allow technicians to interact with holographic images of real objects located in remote and inaccessible places~\cite{9078581}. As the amount of data necessary to stream the holographic media is significant, holographic and haptic communications require high throughput in the range of hundreds of gigabits per second or even terabits per second. Besides, these communications require ultra-low latency and high reliability because a jitter of more than microseconds immediately degrades holographic and haptic interactions~\cite{9078581,9178307}.
 
As described in XR application case, NGNI is a promising way to meet the strict requirements of holographic and haptic interactions by integrating communication, computation, and caching within the network. Therefore, in-network computing and caching enable holographic and haptic-related computation tasks (e.g., big data analysis, 3D video rendering) and contents (e.g., holographic video) to be placed close to their users to reduce latencies and increase data rates.

Effectively transmitting and processing multimodal data streams, such as audio, holographic, and haptic, is a challenging task that requires 3C integration due to the high demand of these different modalities in terms of transmission delay, jitter, rate, and reliability. This challenge is further aggravated when the users (e.g., surrounding location and current viewpoint) and network contexts (e.g., network congestion and topology) can change over time. Moreover, similar to XR, in-network computing of private personal information requires research into privacy and legal issues.  Furthermore, as haptic communication may be used for applications such as remote surgery, the safety of end-users becomes an important aspect for this type of communication.  Finally, haptic communications will enable remote control of physical systems, and as such, an integrated design of end-system control policies jointly with 3C protocols will be needed.
 
 \subsection{Distributed Machine Learning Applications}
 
 \noindent AI/ML enabling intelligent systems has emerged as a key technology benefiting many aspects of our daily life. Currently, most data-driven systems collect data from local devices and then train the data at centralized cloud data centers. However, user privacy, application latency, and network congestion are major concerns of these centralized intelligent systems. To address these issues, distributed and collaborative learning schemes with the most famous one being federated learning, have received considerable attention. In most of these schemes, local workers only share their built machine learning models instead of sending their raw data to a centralized server. Consequently, these distributed schemes have many benefits in terms of data storage, data acquisition, and privacy presentation by not exchanging raw data. 

Although communication overhead is mitigated in distributed learning schemes, the network is still a major performance bottleneck, especially for large-scale systems, due to two main reasons~\cite{sapio2020scaling,dinh2021innetwork}. First, local workers produce intense bursts of traffic to communicate their model updates as the size of these exchanged models can be very large from hundreds of megabytes to a few gigabytes~\cite{dinh2021innetwork}.  Second, as the centralized server needs to wait to receive the local models from all workers, straggling workers have high latency or slow local training can dramatically stagnate the whole training process. Therefore, large-scale distributed learning requires low latency and high data rates.

By integrating 3C functionality in the whole Cloud-Edge-Mist Continuum, NGNI can be a promising solution to support the two phases of large-scale distributed learning. In essence, distributed learning alternates compute-intensive local training phases with communication-intensive model update synchronization. In the first phase, local training can be performed in edge nodes close to the data sources, as shown in Figure~\ref{fig:FL_with_INC}. Then, in the synchronization phase, intermediate network nodes can execute aggregation operations over models with multiple local workers as these models flow towards the centralized server. In addition, in-network caching can quickly deliver the updated global model to local workers. As a result, in-network computing and caching increase throughput, diminish latency, and speed up training time~\cite{sapio2020scaling,dinh2021innetwork}. 

 \begin{figure}
    \centering
    \includegraphics[width=0.48\textwidth]{Figures/FL_with_INC.pdf}
    \caption{Distributed Machine Learning with INC.}
    \label{fig:FL_with_INC}
\end{figure}

Security and privacy are major problems of implementing distributed learning in the NGNI. As network nodes can belong to different entities, not all of them are trusted, and even some nodes may be malicious and attack the training process. Regarding privacy issues, computation, offloading and data processing at network nodes may still involve the transmission of potentially sensitive personal data, discouraging privacy-sensitive applications from taking part in model training in the NGNI. A privacy breach can still be incurred when raw data is not exchanged between local workers because private information can be extracted from the shared trained models and parameters. Hence, it is critical to develop 3C solutions in the NGNI that preserve privacy and security while maintaining efficiency. Another challenge facing the practical implementation of distributed learning is that local workers may be reluctant to participate in the learning process without receiving some incentive as the training model is a resource-consuming task.
	
 
 \subsection{Connected Robots and Autonomous Systems}
 
 \noindent Automation, robotics, and autonomous systems (e.g., drones, autonomous cars, robots) are being used to facilitate a variety of vertical domains (e.g., agriculture, automotive, smart factories, and smart cities) and transform our daily life. Leveraging the full potential of such systems introduces many challenges that today’s networks cannot fulfill. Hereafter, we discuss two domains enabled by such systems and driving the development of NGNI.
 
 \subsubsection{Automotive/Vehicular} 
 
\noindent With the rapid growth of urbanization and industrialization, Intelligent Transportation System (ITS) has received increasing interest from industry and academia. ITS plays a crucial key role in providing innovative services to assist traffic authorities efficiently managing traffic flow, making the roads safer, less congested, and reducing emissions. Connected and Autonomous Vehicles (CAVs) are expected to be one of the main building blocks of future smart cities. Indeed, CAVs are equipped with various sensors (e.g., cameras, LiDar, radar, and GPS), communication modules, and on-board units with computing and storage capabilities that enable a wide range of applications. Accordingly, ITS system is a highly dynamic and data-intensive system with stringent delay and reliability requirements. NGNI has the potential to provide high reliability (\textgreater 99.999 \%) and ultra low latency (\textless 1 ms) with high throughput (\textgreater 100 Gbps) necessary for safe functioning of ITS systems~\cite{8869705,9369324}.

With the integration of CAVs resources with those of the NGNI by using Vehicle-to-Vehicle (V2V) and Vehicle-to-Infrastructure (V2I) collaboration, NGNI should be able to provide high reliability in addition to low latency and high throughput for different ITS services. For instance, comprehensive information from the vehicle’s on-board sensor and from the infrastructure (e.g., traffic lights, cameras) can be collected and processed by networked devices such as  road-side units or mobile base stations  to quickly issue warning signals to vehicles with potential safety hazards and assist vehicle safety drive. In-vehicular infotainment (e.g., traffic and weather reports, high-definition maps, and entertainment videos) is another ITS service that can be improved by caching content and offloading related processing tasks (e.g., video transcoding) within the network~\cite{DZIYAUDDIN2021108228}.

The highly dynamic nature of ITS together with the reliability of communications affecting the safety of humans appear as the main challenges.  INC can provide significant flexibility and efficiency of ITS operations by processing vast data as it traverses the network.  However, as the data processed reflects upon the traffic safety, regulations should be put in place to allow all parties involved in ITS to take responsibility for their actions.  Furthermore, the processing of data may also open up new ethical challenges that need to be resolved.  For example, a quick hazard response may save lives but causes financial burden by delaying the traffic in certain areas.  Therefore, 3C resource management techniques to support high-speed vehicles and massive connectivity in ITS systems need further investigation. 
 
 \subsubsection{Industry 4.0} 
 
\noindent The ongoing digital transformation of manufacturing and supply chains that is characterized by a high level of automation and industrial integration through information and communication technologies (e.g., IoT, cyber-physical systems, mobile networks, and cloud computing) is aptly termed as Industry 4.0. Industry automation realizes the control of processes, devices, and systems through a network of sensors and actuators to improve productivity, flexibility, and efficiency in a cost-effective way. This automation comes with rigorous requirements across the rate-reliability-latency spectrum, a balance that is not yet available in today’s networks but which should be in NGNI~\cite{8869705,9178307}. 

NGNI has the potential to enable reliable and real-time control applications (e.g., robotic motion control~\cite{9293092}, industrial assembly~\cite{9468247}) for Industry 4.0 with the help of INC. Data-related operations can be conducted as the measured data flows over the network towards the control decision process, thereby reducing network traffic and increasing throughput.  Control decisions can be performed in network nodes closer to the factory equipment to reduce data path length and processing latencies. Moreover, by supporting collaboration among network nodes, distributed control techniques can be used to improve reliability and latency.  To support a high level of automation for industrial applications or other autonomous systems with stringent requirements, autonomous network management that exploits the interdependence between computing, caching, communication, and control is crucial and quite challenging. Furthermore, distributed control techniques and intelligent decision-making methods require more research investigations to enable full autonomous factories.
 
 \subsection{Entertainment}
 
\noindent The continued growth of video streaming and the widespread use of ultra-high-definition devices  will lead to an unprecedented traffic demand for ultra- and high-quality videos~\cite{rufino20216g}. In particular, 8K video streaming requires four times the bitrate of the 4K video format, which already requires at least the double bitrate of current HD videos, which may have video bitrates up to 15 Mbps. Moreover, the bitrate for streaming non-2D video formats (e.g., $360^\circ$ VR video) can be even higher. However, this high video traffic demand may cause heavy congestion in the current network. Note that video streaming has several important differences compared to the XR and holographic communications discussed earlier.  In video streaming, the data transmitted is not generated in real time and the initial delay is usually not that important, which allows the use of caches in the network for data delivery.  In-network caching can alleviate this traffic burden on the network core by delivering video content within a few hops to the users. Furthermore, on-demand video transcoding within the network allows a fast adaptation to the current user's network conditions, thus improving the user's QoE. Therefore, in-network caching and computing are promising features of NGNI to support the evolution of video streaming.
 
 Another entertainment application that will benefit from NGNI is cloud gaming, a type of online gaming that runs a game on a remote location (e.g., a cloud datacenter) and streams it as a video to end-user devices. Cloud gaming is a computation-intensive application that requires reliable (99.99 \%), high bandwidth (15 -- 500 Mbps), and low latency (\textless  100 ms) communication for an immersive experience~\cite{9464920}. By offloading the rendering and other tasks (e.g., game logic) to distributed computation nodes within the network, cloud gaming may leverage 3C resources to guarantee users’ QoE~\cite{8685768}. For online multi-player games, these nodes should also be able to synchronize the shared game logic.
 
 As entertainment applications are expected to generate massive data traffic in the NGNI, advanced techniques to offload such large data traffic are required. Designing these techniques can be quite challenging due to the heterogeneity and dynamicity of the  3C resources available in the Cloud-Edge-Mist continuum.  The dynamicity of user demands also necessitates these 3C resources to be rented on demand as the need arises, requiring further research into network-wide pricing policies.  The pricing policies should be designed to be both efficient and incentive-compatible.
 
 \subsection{E-Health}
 
\noindent NGNI can help expand the healthcare sector, eliminating time and physical location barriers through diverse remote healthcare services, such as remote monitoring, diagnosis, prevention, treatment, and even remote surgery. For example, doctors can diagnose patients remotely by controlling robots while receiving haptic, audio, and holographic feedback. However, healthcare applications are generally time-sensitive (sub-millisecond latency) and require high-reliable (up to 99.99999 \%) communications~\cite{9040264}. 
With the convergence and collaboration between technologies such as wireless body area networks, edge-cloud computing, and in-network computing technologies, NGNI will enable quick and reliable transportation and processing of large volume of medical data from sensor devices placed closed to, attached on, or implanted inside the human body. 
%Then, real-time health information can be provided from these collected data with the help of AI, in-network computing, and caching. 

Similar to distributed learning applications, security and privacy are major challenges for the e-Health case. As the patient’s health information should be carefully stored and considered sensitive, reliable and secure computing, caching, and communication services are essential. Besides, trust and security mechanisms are required to guarantee that the collected health information from patients is genuine and  malicious attacks on this information are prevented.
 
%  \subsection{Discussion}

% Although the aforementioned or similar application use cases have been proposed to be enabled by the current network (e.g., 4G/5G), the diversity and stringent requirements of those applications cannot be fully met in the current network. These high demands impose a reshaping of current network infrastructure towards an NGN with seamless end-to-end integration of communication, computation, and caching functionalities since individual functionalities affect each other as well as the final outcome in terms of system performance. In this regard, edge-fog-cloud computing and in-network computing are promising concepts to offer communication, caching, and computing functionalities at different parts of the NGN infrastructure. Therefore, NGN can be seen as a truly distributed, cooperative, and pervasive system that enables the execution of application-specific tasks and storage of data contents within the network with high QoS/QoE guarantees.
 

 
