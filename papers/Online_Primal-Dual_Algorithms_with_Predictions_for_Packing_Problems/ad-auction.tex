%!TEX root = main.tex

\section{Ad-Auction Revenue Maximization}		\label{sec:ad-auction}

\paragraph{Problem.} In the problem, we are given $m$ buyers, each buyer $1 \leq i \leq m$ has a budget $B_{i}$.  
Items arrive online and at the arrival of item $e$, 
each buyer $1 \leq i \leq n$ provides a bid $b_{i,e} \ll B_{i}$ for buying $e$.
In the integral variant of the problem, one needs to allocate the item to at most one of the buyers. For every buyer, the total bids of the allocated items should not exceed the budget. Note that this forms a packing constraint, similar to the ones studied in the previous section, except that we don't normalize the constraints by the factor $1/B_i$.
In the fractional variant of the problem, items can be allocated in fractions to buyers, not exceeding a total of $1$ among the fractions.  In both variants, the objective is to maximize the total revenue received from all buyers, which is the sum over all  allocations of corresponding bid, possibly multiplied with the fraction of the allocation.

In out setting, every item $e$ arrives with a prediction, suggesting a buyer to whom to allocate that item, if any.
For convenience we denote the items by the integers from $1$ to $n$.

\subsection{The algorithm}

\paragraph{Formulation.}
The fractional variant of problem can be expressed as the following linear program, where $x_{i,e}$ indicates the fraction at which item $e$ is allocated to buyer $i$.

\begin{minipage}[t]{0.45\textwidth}
\begin{align*}
\max  \sum_{e=1}^{n} & \sum_{i=1}^{m} b_{i,e} x_{i,e} & \\
\sum_{i=1}^{m} x_{i,e}  &\leq 1 & &  \forall e & (\beta_e) \\
\sum_{e=1}^{n} b_{i,e} x_{i,e} &\leq  B_{i} & & \forall i & (\alpha_i) \\
x_{i,e} &\geq 0 & & \forall i, e\\
\end{align*}
\end{minipage}
\quad
\begin{minipage}[t]{0.5\textwidth}
\begin{align*}
\min \sum_{i=1}^{n} B_{i} \alpha_{i} &+ \sum_{e=1}^{m} \beta_{e} \\ 
b_{i,e} \alpha_{i} + \beta_{e} &\geq b_{i,e}  & &  \forall i, e & (x_{i,e})\\  \\
\alpha_{i}, \beta_{e} &\geq 0 & & \forall i,e \\
\end{align*}
\end{minipage}
%
In the primal linear program, the first constraint ensures that an item can be allocated to at most one buyer and 
the second constraint ensures that a buyer $i$ does not spend more than the budget $B_{i}$.
The dual of the relaxation is presented in the right.  

\paragraph{Algorithm.} 

For the sake of simplicity in the algorithm description, we introduce a fictitious buyer, denoted as $0$, such that  
$b_{0, e} = 0$ for all items $e$. The non-assignment of an item $e$ to any buyer can be seen as being assigned 
to the fictitious buyer $0$ with revenue 0.
The purpose of buyer $0$   is to simplify the description of the algorithm.
 We define the constant $C = (1 + R_{\max})^{\frac{\eta}{R_{\max}}}$ where 
$R_{\max} = \max_{i,e} \frac{b_{i,e}}{B_{i}}$. Adapting the primal-dual algorithm presented in the previous section, we obtain the following algorithm for the fractional ad-auction problem.

\begin{algorithm}[ht]
\begin{algorithmic}[1]  
\STATE All primal and dual variables are initially set to 0.
\STATE For the analysis, we maintain for every buyer $i$ two sets $N(i), M(i)$, both initially empty. 
\FOR{each arrival of a new item $e$}
	\STATE Let $i^{*}$ be the buyer such that $x^{\pred}_{i^{*},e} = 1$. If either the prediction is not feasible or 
	 	there is no such $i^{*}$ (i.e., $x^{\pred}_{i',e} = 0 ~\forall i'$) then
		set $i^{*} = 0$.  
	\STATE Let $i$ be the buyer that maximizes $b_{i',e} (1 - \alpha_{i'})$. 
		If $b_{i,e} (1 - \alpha_{i}) \leq 0$ then 
		set $i = 0$.
	\STATE Set $\beta_{e} \gets \max \bigl \{0,  b_{i,e} (1 - \alpha_{i}) \bigr \}$	 \COMMENT{for the purpose of analysis}
	\IF{$b_{i,e} < b_{i^{*},e}$}
		\STATE Set $x_{i,e} \gets \eta$ and $x_{i^{*},e} \gets 1-\eta$.
		\STATE Define $\overline{b}_{i,e} = b_{i,e}/\eta$ 
		\STATE $N(i^{*}) \gets N(i^{*}) \cup \{e\}$. 		\COMMENT{for the purpose of analysis}
	\ELSE 
		\STATE Set $x_{i,e} \gets 1$ and define $\overline{b}_{i,e} = b_{i,e}$
		\COMMENT{includes case $\vect x^\pred$ is infeasible}
	\ENDIF
	\STATE Update $M(i) \gets M(i) \cup \{e\}$.  \COMMENT{for the purpose of analysis}
	\STATE Update $\alpha_{i} \gets \alpha_{i}\left( 1 + \frac{b_{i,e}}{B_{i}} \right) 
										+  \frac{b_{i,e}}{B_{i}} \cdot \frac{1}{C - 1}$.	
\ENDFOR
% \STATE Scale $\vect x$ by factor $1-R_\max$
\end{algorithmic}
\caption{Algorithm for Ad-Auctions Revenue Maximization.}
\label{algo:ad-auctions}
\end{algorithm}




\begin{lemma}	\label{lem:alpha}
For every $i$, it always holds that 
$$
\alpha_{i} \geq \frac{1}{C - 1} \biggl( C^{\frac{\sum_{e \in M(i)} b_{i,e}}{\eta B_{i}}} - 1 \biggr).
$$
\end{lemma}
%
\begin{proof}
We adapt the proof in \cite{BuchbinderNaor09:The-Design-of-Competitive} 
with a slight modification. 
The primal inequality $\alpha_i$ is proved by induction on the number of released items. 
Initially, when no item is released, the inequality  is trivially true. 
Assume that the inequality holds right before the arrival of an item $e$. 
The inequality remains unchanged for all but the buyer $i$ selected in line 5 of the algorithm, i.e.\ $i$ maximizes $b_{i,e} (1 - \alpha_{i})$. 
We denote by $\alpha_i$  the value before the update triggered by the arrival of $e$ and $\alpha'_i$ its value after the update.
We have
\begin{align*}
\alpha'_{i}
&= \alpha_{i} \cdot \left( 1 + \frac{b_{i,e}}{B_{i}} \right) 
										+  \frac{b_{i,e}}{B_{i}} \cdot \frac{1}{C - 1}	\\
%
&\geq \frac{1}{C - 1} \biggl( C^{\frac{\sum_{e' \in M(i) \setminus e } b_{i,e}}{\eta B_{i}}} - 1 \biggr)
			\cdot \left( 1 + \frac{b_{i,e}}{B_{i}} \right) 
										+   \frac{b_{i,e}}{B_{i}} \cdot \frac{1}{C - 1} \\
%
&=  \frac{1}{C - 1} \biggl( C^{\frac{\sum_{e' \in M(i) \setminus e } b_{i,e}}{\eta B_{i}}} 
						\cdot \left( 1 + \frac{b_{i,e}}{B_{i}} \right)  - 1 \biggr) 	\\						
%
&\geq  \frac{1}{C - 1} \biggl( C^{\frac{\sum_{e' \in M(i) \setminus e } b_{i,e}}{\eta B_{i}}} 
						\cdot  C^{\frac{b_{i,e} }{\eta B_{i}}}   - 1 \biggr) \\
%
&= \frac{1}{C - 1} \biggl( C^{\frac{\sum_{e \in M(i)} b_{i,e}}{\eta B_{i}}} - 1 \biggr).
\end{align*}
%
The first inequality is due to the induction hypothesis.
The second inequality is due to this sequence of transformations. 
For any $0 < y \leq z \leq 1$ we have
\begin{align*}
\frac{\ln(1+y)}{y} &\geq \frac{\ln(1+z)}{z} 
\\
\Leftrightarrow \quad \ln(1+y) &\geq \ln(1+z)y / z 
\\
\Leftrightarrow \quad 1+y &\geq (1+z)^{y/z},
\end{align*}
which we apply with $y=b_{i,e}/{B_i}$ and $z=R_{\max}$. By the definition of $C$ we obtain
$$
\biggl( 1+R_{\max} \biggr)^{\frac{1}{R_{\max}} \cdot \frac{b_{i,e} }{B_{i}}} 
= C^{\frac{b_{i,e}}{\eta B_{i}}}.
$$
This completes the induction step.
\end{proof}



\begin{lemma}	\label{lem:ad-auctions-primal-feasibility}
The primal solution is feasible up to a factor $(1 + R_{\max})$.
\end{lemma}
%
\begin{proof}
The first primal constraint $\sum_{e} x_{i,e} \leq 1$ follows the values of $x_{i,e}$ and $x_{i^{*},e}$ in Algorithm~\ref{algo:ad-auctions}.
For the second primal constraint, we first prove that $\sum_{e=1}^{m} b_{i,e} \leq  B_{i}$ for every $i$. 
By Lemma~\ref{lem:alpha}, for every $i$, 
$$
\alpha_{i} \geq \frac{1}{C - 1} \biggl( C^{\frac{\sum_{e \in M(i)} b_{i,e}}{\eta B_{i}}} - 1 \biggr)
$$
So whenever $\sum_{e \in M(i)} b_{i,e} \geq \eta B_{i}$, we have $\alpha_{i} \geq 1$; so the algorithm 
stops allocating items to buyer $i$. Hence, the buyer $i$ can be allocated at most one additional item once her budget is 
already saturated. Hence, $\sum_{e \in M(i)} b_{i,e}  < \eta B_{i} + \max_{e} b_{i,e}$. 
Therefore, 
\begin{align*}
\sum_{e=1}^{m} b_{i,e}x_{i,e} = 
\sum_{e \in M(i)} b_{i,e}x_{i,e} + \sum_{e \in N(i)} b_{i,e}x_{i,e}
<  B_{i} + \max_{e} b_{i,e}
\end{align*}
where the latter is due to the feasibility of $N(i)$, the set of items assigned by the prediction to buyer $i$,
i.e., $\sum_{e \in N(i)} b_{i,e}x_{i,e} \leq  (1 - \eta) \sum_{e} b_{i,e} x^{\pred}_{i,e} \leq (1 - \eta) B_{i}$.
So, $\sum_{e=1}^{m} b_{i,e}x_{i,e} \leq B_{i}(1 + R_{\max})$.
\end{proof}

\begin{theorem}
The algorithm is $(1 - \eta)$-consistent and $\frac{1 - 1/C}{1 + R_{\max}}$-robust.
The robustness tends to $1 - e^{-\eta}$ when $R_{\max}$ tends to 0.
\end{theorem}
%
\begin{proof}
First, we establish robustness. At the arrival of item $e$, the increase in the primal is 
$$
\begin{cases}
	(1-\eta) \cdot b_{i^{*},e} + \eta \cdot b_{i,e} & \text{ if } b_{i,e} < b_{i^{*},e}, \\
	b_{i,e} & \text{ if } b_{i,e} \geq b_{i^{*},e} 
\end{cases}
$$
which is always larger than $b_{i,e}$. Besides, the increase in the dual is 
%\begin{align*}
%B_{i} \Delta \alpha_{i} + B_{i^{*}} \Delta \alpha_{i^{*}} + \beta_{e}
%&= \frac{\eta}{1+\eta} \biggl( b_{i,e} \alpha_{i} + \frac{b_{i,e}}{c-1} \biggr)
%+ \frac{1}{1+\eta} \biggl( b_{i^{*},e} \alpha_{i^{*}} + \frac{b_{i^{*},e}}{c-1} \biggr) \\
%%
%& \qquad + \frac{\eta}{1+\eta} b_{i,e} (1 - \alpha_{i}) + \frac{1}{1+\eta} b_{i^{*},e} (1 - \alpha_{i^{*}}) \\
%%
%&= \biggl( \frac{1}{1+\eta} \cdot b_{i^{*},e} +  \frac{\eta}{1+\eta} \cdot b_{i,e} \biggr) \biggl( 1 + \frac{1}{c-1} \biggr)
%\end{align*}
\begin{align*}
B_{i} \Delta \alpha_{i}  + \beta_{e}
&= b_{i,e} \alpha_{i} + \frac{b_{i,e}}{C-1} 
 + b_{i,e} (1 - \alpha_{i}) 
 = \biggl( 1 + \frac{1}{C-1} \biggr) b_{i,e}
 = \frac{C}{C - 1}.
 \end{align*}
Hence, by Lemma~\ref{lem:ad-auctions-primal-feasibility}, the robustness is 
$\frac{C-1}{C} \cdot \frac{1}{1+R_{\max}}$.

Secondly we establish consistency quite easily. At every time the prediction solution gets a profit $b_{i^{*},e}$, the algorithm achieves a profit 
at least $(1 - \eta) b_{i^{*},e}$.
\end{proof}

%\paragraph{Upper bound.} The upper bound follows the construction in \cite{EsfandiariKorula18:Allocation-with}. 
%The latter considers the Robust Budgeted Allocation problem with forecast. Even though their model differs from ours, their construction can be used to prove an upper bound to our problem. 
%
%\begin{theorem}[\cite{EsfandiariKorula18:Allocation-with}]
%For any $0 < \eta < 1$,  any $(1 - \eta)$-consist algorithm for the fraction variant of the problem has a robustness at most $1 - \eta e^{-\eta}$.
%\end{theorem}
%%
%\begin{proof}
%We first construct an instance in which every buyer $i$ has unit budget ($B_{i} = 1$) and makes unit bids $b_{i,e} \in \{0,1\}$. In this setting we are facing a standard maximum cardinality bipartite matching problem in the vertex arrival model. 
%Then we extend the construction to satisfy the assumption $b_{i,e} \ll B_{i}$.
%
%Fix $0 < \eta \leq 1$ and an arbitrary $(1-\eta)$-consistent fractional algorithm. 
%Consider an instance consisting of $n$ buyers, each with a unit budget, and $n$ items. The \emph{load} of a buyer is the total item fraction
%assigned to the buyer. In the instance, the first $(1-\eta)n$ items\footnote{For notational convenience we assume $(1-\eta)n$ to be integer.}
%are connected to all buyers, i.e.,  $b_{i,e} = 1$ for all $1 \leq e \leq (1-\eta)n$, $1 \leq i \leq m$. The prediction assigns every 
%item $1 \leq e \leq (1-\eta)n$ to a different buyer $1 \leq i \leq m$. After the first $(1-\eta)n$ items, we mark $(1-\eta)n$ buyers 
%with smallest load, namely buyers $a_{1}, \ldots, a_{(1-\eta)n}$ with loads $\ell_{1}, \ldots, \ell_{(1-\eta)n}$, respectively. 
%For every subsequent item $e > (1-\eta)n$, we do the following: (i) at its arrival, the item is connected to every currently unmarked buyer, 
%and (ii) after its (fractional) assignment, mark the buyer with the smallest load, namely buyer $a_{e}$ with load $\ell_{e}$. 
%The prediction does not assign any item $e > (1-\eta)n$
%to any buyer. 
% 
%In this instance, the prediction has revenue $(1 - \eta) n$ and the offline optimal solution has revenue $n$ 
%(by assigning every item $e$ to $a_{e}$, the buyer marked after its arrival).  
%
%As the algorithm is $(1 - \eta)$-consistent, $\sum_{j=1}^{n} \ell_{j} \geq (1 - \eta) \cdot (1 - \eta) n$. 
%Moreover, $\ell_{1}, \ldots, \ell_{(1-\eta)n}$ are the $(1-\eta)n$ smallest loads, 
%so $\sum_{j=1}^{(1-\eta)n} \ell_{j} \leq (1 - \eta) \cdot (1 - \eta) n$.
%
%Besides, for each $(1 - \eta) n < j \leq n$, the first $j$ items provide the total load at most $j$, and 
%the total load on buyers $a_{1}, \ldots, a_{j-1}$ is $\sum_{k=1}^{j-1} \ell_{k}$. Therefore,
%$$
%\ell_{j} \leq \frac{j - \sum_{k=1}^{j-1} \ell_{k}}{n - j + 1}.
%$$ 
%Hence, the best robustness is the value of the following LP.
%    
%\begin{align*}
%& \max \frac{1}{n} \sum_{j=1}^{n} \ell_{j} &  \\
% \sum_{j=1}^{(1-\eta)n} \ell_{j}  &\leq (1 - \eta)^{2} n \leq \sum_{j=1}^{n} \ell_{j}  &  \\
%\ell_{j} &\leq \frac{j - \sum_{k=1}^{j-1} \ell_{k}}{n - j + 1} \qquad& \forall (1-\eta)n < j \leq n \\
%0 &\leq \ell_{j} \leq 1 \qquad& \forall 1 \leq j \leq n\\
%\end{align*}    
% %
%It is proved in \cite[Theorem 3.23]{EsfandiariKorula18:Allocation-with} that this LP has the maximum value of $1 - \eta e^{-\eta}$    
%in which 
%\begin{align*}
%\ell_{1} &= \ldots = \ell_{(1-\eta)n} = 1-\eta, \\
%\ell_{j} &= \min \biggl \{ 1, (1 - \eta) + \sum_{k = (1-\eta)n + 1}^{j} \frac{1}{n-k+1} \biggr \}
%\leq (1 - \eta) + \sum_{k = (1-\eta)n + 1}^{(1-\eta)n + \xi n} \frac{1}{n-k+1}
%\quad \forall j > (1 - \eta)n,
%\end{align*}
%where $\xi = \eta(1 - e^{-\eta})$. Hence, in the instance, no $(1-\eta)$-consistent fractional algorithm has the robustness larger 
%than $1 - \eta e^{-\eta}$. 
%
%In order to transform the above instance into one in which $b_{i,e} \ll B_{i}$ holds for all $i$ and $e$, it is sufficient to define, for every buyer $i$,  
%$B_{i} = B$ where $B$ is a large number and for each item $e$, create $B$ copies which are released in a row. 
%The same arguments and analysis hold. 
%\end{proof}

\subsection{Experiments}

We run an experimental evaluation of Algorithm~\ref{algo:ad-auctions}. Source code is publicly available\footnote{\url{http://www.lip6.fr/christoph.durr/packing}}.
For this purpose we constructed a random instance on 100 buyers and 10,000 items, adapting the model described in \cite{Lavastida20:Predictions-Matching}. Every item $e$ receives a bid from exactly 6 random buyers. This means that in average every bidder makes 600 bids. The value of the bid follows a lognormal distribution with mean $1/2$ and deviation $1/2$ as well. By choosing the budget of the bidders we can tune the hardness of the instance, under the constraint that $R_{\max}$ remains reasonably small. We choose the budget as a $1/10$ fraction of the total bids, leading to a value $R_{\max}\approx 0.19$.

The competitive ratio is determined using the fractional offline solution. The integrality gap of our instance is very close to $1$, roughly $1.0001$ in fact. For the prediction, we computed first the optimal integral solution, which is a partial mapping from items to buyers. This solution was perturbed as suggested by \cite{BamasMaggiori20:The-Primal-Dual-method}: Every item was mapped, with probability $\epsilon$,  to a uniformly chose random buyer, among the buyers who bid on the item.

The experiments are shown in Figure~\ref{fig:experiments}, and indicate clearly the benefit of the prediction when its perturbation is small.  Note that the ratio is not close to $1$ in that case for $\eta$ close to zero, because the algorithm does not follow blindly the prediction. 
Infeasibility of the prediction was detected at 82\% of the input sequence for $\epsilon=0.01$ and at 64\% for $\epsilon=0.1$.
As expected, with the performance degrades with the perturbation of the prediction.  
Note that the observed robustness is not monotone in $\eta$, unlike the bound shown in this paper.  We think that this performance degradation for $\eta$ around $1/2$ is due to the rather simplistic mixture between the primal-dual solution and the predicted solution.

\begin{figure}[ht]
\begin{center}
	\begin{tikzpicture}[scale=1]
		\begin{axis}[
xlabel={$\eta$},
ylabel={robustness},
yticklabel style={/pgf/number format/precision=5},
legend style={at={(1,1)},anchor=north east},
legend entries={$\epsilon=0$,$\epsilon=0.01$,$\epsilon=0.1$}
]
\addplot [blue,mark=*] table {ratio_0.0.dat};
\addplot [green,mark=o] table {ratio_0.01.dat};
\addplot [red,mark=x] table {ratio_0.1.dat};
\end{axis}
	\end{tikzpicture}
\end{center}
\caption{Experimental evaluation}
\label{fig:experiments}
\end{figure}
