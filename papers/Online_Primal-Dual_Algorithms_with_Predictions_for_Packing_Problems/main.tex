\documentclass[11pt,a4paper]{article}
%\usepackage[pdftex]{graphicx,color}
\usepackage{graphicx,color}
% removed fullpage for now. BTW, better use a4wide if printing in A4 instead of paper
\usepackage{amsmath,amssymb,dsfont,enumerate,epsfig,multirow,longtable,url,bm}
\usepackage{epsfig,epstopdf,hhline,a4wide}
\usepackage{cases}
\usepackage{algorithm}
%\usepackage[noend]{algorithmic}
\usepackage{algorithmic}
\renewcommand{\algorithmicrequire}{\textbf{Input:}}
\renewcommand{\algorithmicensure}{\textbf{Output:}}
\usepackage[numbers]{natbib}
% \usepackage[top=1 in, bottom=1 in, left=1 in, right=1 in, letterpaper]{geometry}
\usepackage{pgfplots}
\usetikzlibrary{arrows,positioning,decorations.pathreplacing,shapes}


\usepackage{thmtools}
\usepackage{thm-restate}

%\declaretheorem[name=Theorem]{thm}

% for affiliation
%\usepackage{authblk}

%%%%%%%%%% Start TeXmacs macrosƒ
\newenvironment{proof}{\noindent\emph{Proof\ }}{\hspace*{\fill}$\Box$\medskip}
\newenvironment{claimproof}{\noindent\emph{Proof of claim\ }}{\hspace*{\fill}$\Box$\medskip}
\newenvironment{plainproof}{\noindent\emph{Proof\ }}{}
\newtheorem{theorem}{Theorem}
\newtheorem{definition}{Definition}
\newtheorem{lemma}{Lemma}
\newtheorem{claim}{Claim}
\newtheorem{proposition}{Proposition}
\newtheorem{corollary}{Corollary}
\usepackage{amsmath}
\usepackage{paralist}
\usepackage{framed}

%%comment in algorithms
\renewcommand{\algorithmiccomment}[1]{\hfill  {\small  \tt \# #1}}

\newcommand\restr[2]{{% we make the whole thing an ordinary symbol
  \left.\kern-\nulldelimiterspace % automatically resize the bar with \right
  #1 % the function
  \vphantom{\big|} % pretend it's a little taller at normal size
  \right|_{#2} % this is the delimiter
  }}

\newcommand{\pred}{\texttt{pred}}
\newcommand{\vect}[1]{\ensuremath{\bm{#1}}}
\newcommand{\one}{\ensuremath{\mathds{1}}}
\newcommand{\PoA}{\text{PoA}}
\newcommand{\E}{\ensuremath{\mathbb{E}}}


\usepackage[utf8]{inputenc} % Required for inputting international characters
\usepackage[T5]{fontenc} % Output font encoding for international characters

\usepackage{color, colortbl}
\usepackage{hyperref}
\hypersetup{colorlinks,
            linkcolor=blue,
            citecolor=blue,
            urlcolor=magenta,
            linktocpage,
            plainpages=false}
\usepackage[capitalise,noabbrev]{cleveref}



\newcommand{\comment}[1]{\textcolor{blue}{{\footnotesize
#1}}\marginpar{\raggedright\tiny \textcolor{blue}{Comment}}}

\title{Online Primal-Dual Algorithms with Predictions for Packing Problems}

\author{
Nguyễn Kim Thắng \\
University Paris-Saclay, IBISC, France
\and Christoph D\"urr \\
Sorbonne University, CNRS, LIP6, France
}



%\date{}

\begin{document}

\maketitle

\begin{abstract}
The domain of online algorithms with predictions has been extensively studied
for different applications such as scheduling, caching (paging), clustering, ski rental, etc.
Recently, Bamas et al., aiming for an unified method, have provided a primal-dual framework 
for linear covering problems. They extended the online primal-dual method by incorporating 
predictions in order to achieve a performance beyond the worst-case case analysis.
In this paper, we consider this research line and  present a framework to design algorithms with predictions for non-linear 
packing problems. We illustrate the applicability of our framework in submodular maximization
and in particular ad-auction maximization in which the optimal bound is given and supporting experiments are provided.
\end{abstract}

%\thispagestyle{empty}

%\newpage

%\setcounter{page}{1}



% \leavevmode
% \\
% \\
% \\
% \\
% \\
\section{Introduction}
\label{introduction}

AutoML is the process by which machine learning models are built automatically for a new dataset. Given a dataset, AutoML systems perform a search over valid data transformations and learners, along with hyper-parameter optimization for each learner~\cite{VolcanoML}. Choosing the transformations and learners over which to search is our focus.
A significant number of systems mine from prior runs of pipelines over a set of datasets to choose transformers and learners that are effective with different types of datasets (e.g. \cite{NEURIPS2018_b59a51a3}, \cite{10.14778/3415478.3415542}, \cite{autosklearn}). Thus, they build a database by actually running different pipelines with a diverse set of datasets to estimate the accuracy of potential pipelines. Hence, they can be used to effectively reduce the search space. A new dataset, based on a set of features (meta-features) is then matched to this database to find the most plausible candidates for both learner selection and hyper-parameter tuning. This process of choosing starting points in the search space is called meta-learning for the cold start problem.  

Other meta-learning approaches include mining existing data science code and their associated datasets to learn from human expertise. The AL~\cite{al} system mined existing Kaggle notebooks using dynamic analysis, i.e., actually running the scripts, and showed that such a system has promise.  However, this meta-learning approach does not scale because it is onerous to execute a large number of pipeline scripts on datasets, preprocessing datasets is never trivial, and older scripts cease to run at all as software evolves. It is not surprising that AL therefore performed dynamic analysis on just nine datasets.

Our system, {\sysname}, provides a scalable meta-learning approach to leverage human expertise, using static analysis to mine pipelines from large repositories of scripts. Static analysis has the advantage of scaling to thousands or millions of scripts \cite{graph4code} easily, but lacks the performance data gathered by dynamic analysis. The {\sysname} meta-learning approach guides the learning process by a scalable dataset similarity search, based on dataset embeddings, to find the most similar datasets and the semantics of ML pipelines applied on them.  Many existing systems, such as Auto-Sklearn \cite{autosklearn} and AL \cite{al}, compute a set of meta-features for each dataset. We developed a deep neural network model to generate embeddings at the granularity of a dataset, e.g., a table or CSV file, to capture similarity at the level of an entire dataset rather than relying on a set of meta-features.
 
Because we use static analysis to capture the semantics of the meta-learning process, we have no mechanism to choose the \textbf{best} pipeline from many seen pipelines, unlike the dynamic execution case where one can rely on runtime to choose the best performing pipeline.  Observing that pipelines are basically workflow graphs, we use graph generator neural models to succinctly capture the statically-observed pipelines for a single dataset. In {\sysname}, we formulate learner selection as a graph generation problem to predict optimized pipelines based on pipelines seen in actual notebooks.

%. This formulation enables {\sysname} for effective pruning of the AutoML search space to predict optimized pipelines based on pipelines seen in actual notebooks.}
%We note that increasingly, state-of-the-art performance in AutoML systems is being generated by more complex pipelines such as Directed Acyclic Graphs (DAGs) \cite{piper} rather than the linear pipelines used in earlier systems.  
 
{\sysname} does learner and transformation selection, and hence is a component of an AutoML systems. To evaluate this component, we integrated it into two existing AutoML systems, FLAML \cite{flaml} and Auto-Sklearn \cite{autosklearn}.  
% We evaluate each system with and without {\sysname}.  
We chose FLAML because it does not yet have any meta-learning component for the cold start problem and instead allows user selection of learners and transformers. The authors of FLAML explicitly pointed to the fact that FLAML might benefit from a meta-learning component and pointed to it as a possibility for future work. For FLAML, if mining historical pipelines provides an advantage, we should improve its performance. We also picked Auto-Sklearn as it does have a learner selection component based on meta-features, as described earlier~\cite{autosklearn2}. For Auto-Sklearn, we should at least match performance if our static mining of pipelines can match their extensive database. For context, we also compared {\sysname} with the recent VolcanoML~\cite{VolcanoML}, which provides an efficient decomposition and execution strategy for the AutoML search space. In contrast, {\sysname} prunes the search space using our meta-learning model to perform hyperparameter optimization only for the most promising candidates. 

The contributions of this paper are the following:
\begin{itemize}
    \item Section ~\ref{sec:mining} defines a scalable meta-learning approach based on representation learning of mined ML pipeline semantics and datasets for over 100 datasets and ~11K Python scripts.  
    \newline
    \item Sections~\ref{sec:kgpipGen} formulates AutoML pipeline generation as a graph generation problem. {\sysname} predicts efficiently an optimized ML pipeline for an unseen dataset based on our meta-learning model.  To the best of our knowledge, {\sysname} is the first approach to formulate  AutoML pipeline generation in such a way.
    \newline
    \item Section~\ref{sec:eval} presents a comprehensive evaluation using a large collection of 121 datasets from major AutoML benchmarks and Kaggle. Our experimental results show that {\sysname} outperforms all existing AutoML systems and achieves state-of-the-art results on the majority of these datasets. {\sysname} significantly improves the performance of both FLAML and Auto-Sklearn in classification and regression tasks. We also outperformed AL in 75 out of 77 datasets and VolcanoML in 75  out of 121 datasets, including 44 datasets used only by VolcanoML~\cite{VolcanoML}.  On average, {\sysname} achieves scores that are statistically better than the means of all other systems. 
\end{itemize}


%This approach does not need to apply cleaning or transformation methods to handle different variances among datasets. Moreover, we do not need to deal with complex analysis, such as dynamic code analysis. Thus, our approach proved to be scalable, as discussed in Sections~\ref{sec:mining}.

%\section{Complete algorithm}
\label{sec:general}


In this section we gather all the facts on the Janis--Newman algorithm and we explain how to apply it to a general setting.
We write the formulas corresponding to the most general configurations that can be obtained.
We insist again on the fact that all these results can also be derived from the tetrad formalism.


\subsection{Seed configuration}
\label{sec:general:seed}


We consider a general configuration with a metric $g_{\mu\nu}$, gauge fields $A_\mu^I$, complex scalar fields $\tau^i$ and real scalar fields $q^u$.
The initial parameters of the seed configuration are the mass $m$, electric charges $q^I$, magnetic charges $p^i$ and some other parameters $\lambda^A$ (such as the scalar charges).
The electric and magnetic charges are grouped in complex parameters
\begin{equation}
	Z^I = q^I + i p^I.
\end{equation} 
All indices run over some arbitrary ranges.

The seed configuration is spherically symmetric and in particular all the fields depend only on the radial direction $r$
\begin{subequations}
\label{gen:eq:static:tr}
\begin{gather}
	\label{gen:eq:static:metric:tr}
	\dd s^2 = - f_t(r)\, \dd t^2 + f_r(r)\, \dd r^2 + f_\Omega(r)\, \dd\Omega^2, \\
	A^I = f^I(r)\, \dd t + p^I H'(\theta)\, \dd\phi, \\
	\tau^i = \tau^i(r), \qquad
	q^u = q^u(r)
\end{gather}
\end{subequations}
where
\begin{equation}
	\dd \Omega^2 = \dd\theta^2 + H(\theta)^2\, \dd \phi^2, \qquad
	H(\theta) =
	\begin{cases}
		\sin \theta & \kappa = 1 \quad (S^2), \\
		\sinh \theta & \kappa = -1 \quad (H^2).
	\end{cases}
\end{equation} 
Note that only two functions in the metric are relevant since the last one can be fixed through a diffeomorphism.
All the real functions are denoted collectively by
\begin{equation}
	f_i = \{ f_t, f_r, f_\Omega, f^I, q^u \}.
\end{equation} 

The transformation to null coordinates is
\begin{equation}
	\label{gen:eq:change:null}
	\dd t = \dd u - \sqrt{\frac{f_r}{f_t}}\, \dd r
\end{equation} 
and yields
\begin{subequations}
\label{gen:eq:static:ur}
\begin{gather}
	\label{gen:eq:static:metric:ur}
	\dd s^2 = - f_t\, \dd u^2 - 2 \sqrt{f_t f_r}\, \dd r^2 + f_\Omega\, \dd\Omega^2, \\
	A^I = f^I\, \dd u + p^I H'\, \dd\phi
\end{gather}
\end{subequations}
where the radial component of the gauge field
\begin{equation}
	A^I_r = f^I \sqrt{\frac{f_r}{f_t}}
\end{equation} 
has been set to zero through a gauge transformation.


\subsection{Janis--Newman algorithm}
\label{sec:general:jna}


\subsubsection{Complex transformation}


One performs the complex change of coordinates
\begin{equation}
	\label{gen:eq:change:jna}
	r = r' + i\, F(\theta), \qquad
	u = u' + i\, G(\theta).
\end{equation}
In the case of topological horizons the Giampieri ansatz \eqref{algo:eq:giampieri-ansatz} generalizes to
\begin{equation}
	\label{gen:eq:giampieri-ansatz}
	i\, \dd \theta = H(\theta)\, \dd \phi
\end{equation} 
leading to the differentials
\begin{equation}
	\dd r = \dd r' + F'(\theta) H(\theta)\, \dd \phi, \qquad
	\dd u = \dd u' + G'(\theta) H(\theta)\, \dd \phi.
\end{equation} 
The ansatz \eqref{gen:eq:giampieri-ansatz} is a direct consequence of the fact that the $2$-dimensional slice $(\theta, \phi)$ is given by $\dd \Omega^2 = \dd\theta^2 + H(\theta)^2\, \dd \phi^2$, such that the function in the RHS of \eqref{gen:eq:giampieri-ansatz} corresponds to $\sqrt{g^\Omega_{\phi\phi}}$ (where $g$ is the static metric), as can be seen by doing the computation with $i\, \dd \theta = \mc H(\theta) \dd\phi$ and identifying $\mc H = H$ at the end.

The most general known transformation is
\begin{subequations}
\begin{gather}
	\label{gen:eq:change:jna-functions-FG}
	F(\theta) = n - a\, H'(\theta) + c \left( 1 + H'(\theta)\, \ln \frac{H(\theta/2)}{H'(\theta/2)} \right), \\
	G(\theta) = \kappa a\, H'(\theta)
		- 2 \kappa n \ln H(\theta)
		- \kappa c\, H'(\theta)\, \ln \frac{H(\theta/2)}{H'(\theta/2)}, \\
	m = m' + i \kappa n, \\
	\kappa = \kappa' - \frac{4\Lambda}{3}\, n^2,
\end{gather}
\end{subequations}
where $a, c \neq 0$ only if $\Lambda = 0$ (see \cref{sec:derivation} for the derivation).
The mass that is transformed is the physical mass: even if it written in terms of other parameters one should identify it and transform it.

The parameters $a$ and $n$ correspond respectively to the angular momentum and to the NUT charge.
On the other hand the constant $c$ did not receive any clear interpretation (see for example~\cites{Demianski:1972:NewKerrlikeSpacetime, Adamo:2014:KerrNewmanMetricReview}[sec.~5.3]{Krasinski:2006:InhomogeneousCosmologicalModels}).
It can be noted that the solution is of type II in Petrov classification (and thus the JN algorithm \emph{can} change the Petrov type) and it corresponds to a wire singularity on the rotation axis.
Moreover the BL transformation is not well-defined.


\subsubsection{Function transformation}
\label{sec:general:jna:functions}


All the real functions $f_i = f_i(r)$ must be modified to be kept real once $r \in \C$
\begin{equation}
	\label{gen:eq:complexification-functions}
	\tilde f_i = \tilde f_i(r, \bar r)
		= \tilde f_i\big(r', F(\theta) \big) \in \R.
\end{equation} 
The last equality means that $\tilde f_i$ can depend on $\theta$ only through $\Im r = F(\theta)$.
The condition that one recovers the seed for $\bar r = r = r'$ is
\begin{equation}
	\tilde f_i(r', 0) = f_i(r').
\end{equation} 

If all magnetic charges are vanishing or in terms without electromagnetic charges the rules for finding the $\tilde f_i$ are
\begin{subequations}
\label{gen:eq:rules}
\begin{align}
	\label{gen:eq:rules:r}
	r & \longrightarrow \frac{1}{2} (r + \bar r) = \Re r, \\
	\label{gen:eq:rules:1/r}
	\frac{1}{r} & \longrightarrow \frac{1}{2} \left(\frac{1}{r} + \frac{1}{\bar r}\right) = \frac{\Re r}{\abs{r}^2}, \\
	\label{gen:eq:rules:r2}
	r^2 & \longrightarrow \abs{r}^2.
\end{align}
\end{subequations}
Up to quadratic powers of $r$ and $r^{-1}$ these rules determine almost uniquely the result.
This is not anymore the case when the configurations involve higher power.
These can be dealt with by splitting it in lower powers: generically one should try to factorize the expression into at most quadratic pieces.
Some examples of this with natural guesses are
\begin{equation}
	r^4 - b^2 = (r^2 + b) (r^2 - b), \qquad
	r^4 + b = r^2 \left( r^2 + \frac{b}{r^2} \right).
\end{equation} 
Moreover the same power of $r$ can be transformed differently, for example
\begin{equation}
	\frac{1}{r^n} \longrightarrow \frac{1}{r^{n-2}}\, \frac{1}{\abs{r}^2}.
\end{equation} 

Denoting by $Q(r)$ and $P(r)$ collectively all functions that multiply $q^I$ and $p^I$ respectively, all such terms should be rewritten as
\begin{equation}
	\Big( q^I Q(r), p^I P(r) \Big) = \Big( \Re\big(Z^I Q(r)\big), \Im\big(Z^I P(r)\big) \Big)
\end{equation} 
before performing the transformation \eqref{gen:eq:change:jna}.
Note that in this case one does not use the rules \eqref{gen:eq:rules}.

Finally the transformed complex scalars are obtained by simply plugging \eqref{gen:eq:change:jna}
\begin{equation}
	\tau'^i(r', \theta) = \tau^i\big(r + i F(\theta)\big).
\end{equation} 


\subsubsection{Null coordinates}


Plugging the transformation \eqref{gen:eq:change:jna} inside the seed metric and gauge fields \eqref{gen:eq:static:ur} leads to\footnotemark{}%
\footnotetext{%
	We stress that at this stage these formula do not satisfy Einstein equations, they are just proxies to simplify later computations.
}
\begin{subequations}
\label{gen:eq:rotating:ur}
\begin{gather}
	\dd s^2 = - \tilde f_t\, (\dd u' + \alpha\, \dd r' + \omega H\, \dd\phi )^2
		+ 2 \beta\, \dd r' \dd \phi
		+ \tilde f_\Omega\, \big(\dd\theta^2 + \sigma^2 H^2\, \dd\phi^2 \big), \\
	A^I = \tilde f^I\, (\dd u' + G' H\, \dd \phi) + p^I H'\, \dd\phi
\end{gather}
\end{subequations}
where (one should not confuse the primes to indicate derivatives from the primes on the coordinates)
\begin{equation}
	\omega = G' + \sqrt{\frac{\tilde f_r}{\tilde f_t}}\, F', \qquad
	\sigma^2 = 1 + \frac{\tilde f_r}{\tilde f_\Omega}\, F'^2, \qquad
	\alpha = \sqrt{\frac{\tilde f_r}{\tilde f_t}}, \qquad
	\beta = \tilde f_r\, F' H.
\end{equation} 


\subsubsection{Boyer--Lindquist coordinates}


The Boyer--Lindquist transformation
\begin{equation}
	\label{gen:eq:change:bl}
	\dd u' = \dd t' - g(r') \dd r', \qquad
	\dd \phi = \dd \phi' - h(r') \dd r',
\end{equation} 
can be used to remove the off-diagonal $tr$ and $r\phi$ components of the metric
\begin{equation}
	g_{t'r'} = g_{r'\phi'} = 0.
\end{equation} 
The solution to these equations is
\begin{equation}
	\label{gen:eq:change:bl:solution-gh}
	g(r') = \frac{\sqrt{\big(\tilde f_t \tilde f_r \big)^{-1}}\, \tilde f_\Omega - F' G'}{\Delta}, \qquad
	h(r') = \frac{F'}{H \Delta}
\end{equation} 
where
\begin{equation}
	\label{gen:eq:change:bl:delta}
	\Delta = \frac{\tilde f_\Omega}{\tilde f_r}\, \sigma^2
		= \frac{\tilde f_\Omega}{\tilde f_r} + F'^2.
\end{equation} 
Remember that the changes of coordinate is valid only if $g$ and $h$ are functions of $r'$ only.

Inserting \eqref{gen:eq:change:bl:solution-gh} into \eqref{gen:eq:rotating:ur} yields
\begin{subequations}
\label{gen:eq:rotating:tr}
\begin{gather}
	\dd s^2 = - \tilde f_t\, (\dd t' + \omega H\, \dd\phi' )^2
		+ \frac{\tilde f_\Omega}{\Delta}\, \dd r'^2
		+ \tilde f_\Omega\, \big(\dd\theta^2 + \sigma^2 H^2\, \dd\phi'^2 \big), \\
	A^I = \tilde f^I\, \left(\dd t' - \frac{\tilde f_\Omega}{\Delta \sqrt{\tilde f_t \tilde f_r}}\, \dd r' + G' H\, \dd \phi' \right) + p^I H'\, \dd\phi'
\end{gather}
\end{subequations}
where we recall that
\begin{equation}
	\omega = G' + \sqrt{\frac{\tilde f_r}{\tilde f_t}}\, F', \qquad
	\sigma^2 = 1 + \frac{\tilde f_r}{\tilde f_\Omega}\, F'^2.
\end{equation} 
Generically one finds $A_r = A_r(r)$ which can be set to zero thanks to a gauge transformation.

Before closing this section we simplify the above formulas for few simple cases that are often used.


\paragraph{Degenerate Schwarzschild seed}

A degenerate seed (one unknown function) in Schwarzschild coordinates has
\begin{equation}
	f_r = f_t^{-1}, \qquad
	f_\Omega = r^2.
\end{equation} 
The above formulas for this case can be found in \cref{sec:derivation:ansatz}.


\paragraph{Degenerate isotropic seed}

A degenerate seed in isotropic coordinates has
\begin{equation}
	f_t = f^{-1}, \qquad
	f_r = f, \qquad
	f_\Omega = r^2 f.
\end{equation} 
In this case the above formulas reduced to
\begin{subequations}
\label{gen:eq:rotating:tr-degenerate-isotropic}
\begin{gather}
	\dd s^2 = - \tilde f^{-1}\, (\dd t + \omega H\, \dd\phi )^2
		+ \tilde f \rho^2 \left( \frac{\dd r^2}{\Delta}
			+ \dd\theta^2 + \sigma^2 H^2\, \dd\phi^2 \right), \\
	A^I = \tilde f^I\, \left(\dd t - \frac{\tilde f \rho^2}{\Delta}\, \dd r + G' H\, \dd \phi \right) + p^I H'\, \dd\phi
\end{gather}
\end{subequations}
where we recall that
\begin{equation}
	\omega = G' + \tilde f\, F', \qquad
	\sigma^2 = 1 + \frac{F'^2}{\rho^2}, \qquad
	\Delta = \tilde f \rho^2 + F'^2.
\end{equation} 

\paragraph{Constant $F$}

The expressions simplify greatly if $F' = 0$ (for example when $\Lambda \neq 0$).
First all functions depend only on $r$ since $F(\theta) = \cst$
\begin{equation}
	\tilde f_i(r, \theta) = \tilde f_i(r, 0).
\end{equation} 
As a consequence the Boyer--Lindquist transformation \eqref{gen:eq:change:bl:solution-gh}
\begin{equation}
	g(r') = \sqrt{\frac{\tilde f_r}{\tilde f_t}}, \qquad
	h(r') = 0
\end{equation} 
is always well-defined.
For the same reason it is always possible to perform a gauge transformation.
Finally the metric and gauge fields \eqref{gen:eq:rotating:tr} becomes
\begin{subequations}
\label{gen:eq:rotating:tr-F-cst}
\begin{gather}
	\dd s^2 = - \tilde f_t \big(\dd t + G' H\, \dd\phi \big)^2
		+ \tilde f_r\, \dd r^2
		+ \tilde f_\Omega\, \dd\Omega^2, \\
	A^I = \tilde f^I\, \left(\dd t' + G' H\, \dd \phi' \right) + p^I H'\, \dd\phi'.
\end{gather}
\end{subequations}


\subsection{Open questions}


The algorithm we have described help to work with five (four if $\Lambda \neq 0$) of the six parameters of the Plebański--Demiański (PD) solution.
It is tempting to conjecture that it can be extended to the full set of parameters by generalizing the ideas described in \cref{sec:extension:nut} (shifting $\kappa$, writing $a + i \alpha$…).
Indeed we have found that these operations were quite natural in the context of the  PD solution and inspiration could be found in~\cite{Griffiths:2006:NewLookPlebanskiDemianski}.


%!TEX root = main.tex

\section{Primal-Dual Framework for Packing Problems}		\label{sec:packing}
%\paragraph{Packing Problem.}
%Let $\mathcal{E}$ be a set of $n$ resources 
%and let $f: \{0,1\}^{n} \rightarrow \mathbb{R}^{+}$ be an \emph{arbitrary} non-decreasing function.
%Let $x_{e} \in \{0,1\}$ be a variable indicating whether resource $e$ is selected. 
%The set of packing constraints $\sum_{e} b_{i,e} x_{e} \leq 1 ~\forall i$ (including $x_{e} \leq 1 ~\forall e$) are given in advance and 
%resources $e$ are revealed online one-by-one. At the arrival of resource $e$, one receives a prediction $x^{\pred}_{e} \in \{0,1\}$ 
%and needs to make a decision on $x_{e}$ while maintaining $\vect{x} = (x_{e})_{e \in \mathcal{E}}$ feasible to the set of constraints.
%The objective of the problem is to maximize $f(\vect{x})$. In this problem, we seek a fractional solution that 
%is consistent to the prediction and robust to the optimal offline solution. 
%
%
%
%
%\subsection{Algorithm for Fractional Packing}		\label{sec:packing-main}
%Recall that a differentiable function $F: [0,1]^{n} \rightarrow \mathbb{R}^{+}$ is $(\lambda,\mu)$-max-locally-smooth
%if for any set $S \subset \mathcal{E}$, and for every vector $\vect{x} \in [0,1]^{n}$, 
%the following inequality holds:
%$$
%\sum_{e \in S} \nabla_{e} F(\vect{x}) \geq \lambda F\bigl( \vect{1}_{S} \bigr) - \mu F\bigl( \vect{x} \bigr)
%$$
%%where $\vect{x} := \bigvee_{e \in S} \vect{x}^{e}$, meaning that $x_{e'}  = \max \{x^{e}_{e'}\}$ for any coordinate $e'$.
% 


\paragraph{Formulation.}
First, we model the packing problem as a configuration linear program. 
For the integral variant of the packing problem, the decision variable $x_{e}\in\{0,1\}$  indicates whether element $e$ is selected in the solution. A configuration is a set of elements $S \subseteq \mathcal{E}$ and could be feasible or not.  In addition to $\vect x$ the linear program contains variables $z_S\in\{0,1\}$ for every configuration $S$. 
The idea is that $z_S=1$ solely for the set $S$ containing all selected elements $e$, i.e.\ for which $x_e=1$.  In this case $S$ is feasible by the constraints imposed on $\vect x$.

The fractional variant of the packing problem is modeled with the same variables and constraints, but without the integrality constraints.  Now $x_e$ specifies the fraction with which $e$ is selected.  The $\vect z$ variables represent now a distribution on configurations $S$ and have to be consistent with $\vect{x}$ in the following sense. When $S$ is selected with probability $z_S$, then $e$ belongs to the selected set $S$ with probability $x_e$. Unlike for the integral linear program, $\vect z$ is not unique for given $\vect x$.  In our algorithm we chose a particular vector $\vect z$. Note that the support of $\vect z$ might contain non-feasible configurations.

We consider the following linear program and its  dual.  

\begin{minipage}[t]{0.45\textwidth}
\begin{align*}b
\max  \sum_{S} &f(\vect{1}_{S}) z_{S} \\
\sum_{e} b_{i,e}  \cdot x_{e}  &\leq 1 & &  \forall i & (\alpha_i)\\
\sum_{S: e \in S} z_{S}  &= x_{e} 	& & \forall e & (\beta_e)\\
\sum_{S} z_{S} &= 1 & & & (\gamma) \\
x_{e} , z_{S} &\geq 0 & & \forall e,S\\
\end{align*}
\end{minipage}
\quad
\begin{minipage}[t]{0.5\textwidth}
\begin{align*}
\min \sum_{i} \alpha_{i} &+ \gamma \\
\sum_{i} b_{i,e}  \cdot \alpha_{i} &\geq \beta_{e}  & &  \forall e & (x_e)\\
\gamma + \sum_{e \in S} \beta_{e} &\geq f(\vect{1}_{S})  & & \forall S & (z_S) \\
\alpha_{i} &\geq 0 & & \forall i 
\end{align*}
\end{minipage}

In the primal, $(\alpha_i)$ are the packing constraints of the given problem.  Constraints $(\beta_e)$ force the aforementioned  relation between $\vect x$ and $\vect z$.  Constraint $(\gamma)$ ensures that $\vect z$ represents a distribution.
Note that the primal constraints $(\gamma)$ and $(\beta_e)$, imply the box constraints $x_{e}  \leq 1 ~\forall e$. 
 
\paragraph{Algorithm.} 
Assume that function $F(\cdot)$ is $(\lambda, \mu)$-locally smooth.
Let $d$ be the maximal number of positive entries in a row, i.e., $d = \max_{i} |\{b_{ie}: b_{ie} > 0\}|$.  
Let $\rho$ the maximum divergence between positive row coefficients, i.e.\ $\rho = \max_{i} \max_{e,e': b_{ie' > 0}} b_{ie} / b_{ie'}$.   
The algorithm is given a prediction $\vect{x}^\pred$, i.e.\ with every arriving element $e$, it receives the values $x^\pred_{ie}$ for each resource $i$.  It uses this prediction to specify coefficients $\overline{\vect b}$, which are scaled from $\vect b$ in a specific manner.  The maximum divergence $\overline \rho$ is defined similarly as $\rho$ with $\overline{\vect b}$ replacing $\vect b$, i.e., $\overline{\rho} = \max_{i} \max_{e,e': \overline{b}_{ie' > 0}} \overline{b}_{i,e} / \overline{b}_{ie'}$.

The algorithm maintains a primal solution $\vect y\in[0,1]^{\cal E}$, computed with the primal-dual method with respect to the coefficients $\overline {\vect b}$. Its decision for element $e$ either follows the predicted solution $x^\pred_e$, or it follows $y_{e}$ in case infeasibility of the predicted solution $\vect x^\pred$ has been detected.

The value of $\overline{b}_{i,e}$ depends on coefficient $b_{i,e}$ and 
the prediction $x^\pred_e$. Specifically, $\overline{b}_{i,e} = b_{i,e}$ if
$x^{\pred}_{e} = 1$ and the predictive solution is \emph{not} feasible; $\overline{b}_{i,e} = \frac{1}{\eta} b_{i,e}$ otherwise. In both cases, packing constraints using $\overline{\vect b}$ are stronger than they would be with coefficients $\vect b$.

Intuitively, if we do not trust the prediction at all, i.e.\ $\eta=1$, then 
$\overline{b}_{i,e} = b_{i,e}$ and therefore $x_{e}, y_{e}$ would get a value proportional to the one returned by a primal-dual algorithm
in the classic setting.
Inversely, if we trust the prediction (i.e., $\eta$ is close to 0) and the prediction is feasible, then 
$\overline{b}_{i,e} = b_{i,e}$ when $x^{\pred}_{e} = 1$ and $\overline{b}_{i,e} = b_{i,e}/\eta$ when 
$x^{\pred}_{e} = 0$. Therefore, the modified constraint
$\sum_{e'} \overline{b}_{i,e'} y_{e'} \leq 1$ will likely prevent $y_{e}$, for $e$ such that $x^{\pred}_{e} = 0$, 
from getting a large value. Hence, $y_{e}$ for $e$ such that $x^{\pred}_{e} = 1$ could get a large value.
In the end of each iteration, we set the output solution $x_{e}$ roughly by scaling a factor $\frac{1}{1+\eta}$ to  $y_{e}$ or $x^{\pred}_{e}$
(depending on cases)
in order to maintain the feasibility and the consistency to the prediction.


The construction of $\vect{y}$ follows the scheme in \cite{Thang20:Online-Primal-Dual}. 
We recall the definition of the divergence factor $\overline{\rho} = \max_{i} \max_{e,e': \overline{b}_{ie' > 0}} \overline{b}_{i,e} / \overline{b}_{ie'}$.
So in particular, $\overline{\rho} \leq \rho/\eta$. 
Recall that the gradient in direction $e$ is $\nabla_{e} F(\vect{y}) = \partial F(\vect{y})/\partial y_{e}$.
By convention, when an element $e$ is not released, $\nabla_{e} F(\vect{y}) = 0$.  
While $\nabla_{e} F(\vect{y}) > 0$ --- i.e., increasing $y_{e}$ improves the objective value ---
and $\sum_{i} \overline{b}_{i,e}  \alpha_{e} \leq \frac{1}{\lambda} \nabla_{e} F(\vect{y})$, the primal variable $y_{e} $
and dual variables $\alpha_{i}$'s are increased by appropriate rates. We will argue in the analysis that 
the primal and dual solutions returned by the algorithm are feasible. 

Recall that by definition of the multilinear extension, 
%\marginpar{Better avoid game theory notation}
$$
\nabla_{e} F(\vect{y})
= F((\vect{y}_{-e}, 1)) - F((\vect{y}_{-e},0))
= \mathbb{E}_{R} \bigl[ f\bigl(\vect{1}_{R \cup \{e\}}\bigr) - f\bigl(\vect{1}_{R}\bigr) \bigr]
$$
where $(\vect{y}_{-e}, 1)$ denotes a vector which is identical to $\vect{y}$ on every coordinate different to $e$ and 
the value at coordinate $e$ is 1.  The vector $(\vect{y}_{-e}, 0)$ is defined similarly. 
The expectation is taken over random subset $R \subseteq \mathcal{E} \setminus \{e\}$ such that $e'$ is included with probability $y_{e'}$.
Therefore, during the iteration of the while loop with respect to element $e$, only $y_{e} $ is modified and $y_{e'} $ remains fixed for all $e' \neq e$, 
as a consequence $\nabla_{e} F(\vect{y})$ is constant during the iteration.
Moreover,  for every $e$, $F(\vect{y})$ and $\nabla_{e} F(\vect{y})$ 
can be efficiently approximated up to any required precision \cite{Vondrak10:Polyhedral-techniques}.

One important aspect of the primal-dual algorithm presented below, is that it works only with primal variables $\vect x$ and dual variables $\alpha$, and uses the multilinear extension $F$ instead of $f$.  However for the analysis, we will later show how to extend these solutions with variables $\vect z$ and $\vect \beta$, $\gamma$ both to show feasibility of the constructed solution and to analyze consistency and robustness of the algorithm.  

\begin{algorithm}[ht]
\begin{algorithmic}[1]  
\STATE All primal and dual variables are initially set to 0. 
\STATE Let $\vect{y}\in[0,1]^{\cal E}$ be such that $y_{e} = 0 ~\forall e$. 
\STATE At every step, always maintain $z_{S} = \prod_{e \in S} x_{e}  \prod_{e \notin S} (1 - x_{e} )$.
\FOR{each arrival of a new element $e$} 
	\IF{$x^{\pred}_{e} = 1$ \OR the predictive solution $\vect x^\pred$ is infeasible} 
	\STATE{ set $\overline{b}_{i,e} = b_{i,e}$} 
	\ELSE \STATE{ set $\overline{b}_{i,e} = b_{i,e}/\eta$}
	\ENDIF
	\WHILE{$\sum_{i} \overline{b}_{i,e}  \alpha_{i} \leq \frac{1}{\lambda} \nabla_{e} F(\vect{y})$ \AND $\nabla_{e} F(\vect{y}) > 0$}
%		\STATE During the while loop, always maintain $\beta_{e} \gets \frac{1}{\lambda} \nabla_{e} F(\vect{y})$. The value 
%			of $\beta_{e}$ will be fixed after the for loop related to resource $e$. 
		\STATE Some of the primal and dual variables are increased continuously as follows, where $\tau$ is the time during this process.
		% Let $\tau$ be the time in the execution of the algorithm. The dual variables evolve during the execution of the algorithm as follows. 
		% \marginpar{But $y_e$ is primal}
		\STATE Increase $y_{e} $ at a rate such that $ \frac{dy_{e}}{d\tau} \gets \frac{1}{\nabla_{e} F(\vect{y}) \cdot \ln(1+ d\overline{\rho}  )}$.	
				\label{algo-packing:x}
% 		\FOR{$i$ such that $\overline{b}_{i,e}  > 0$}	
% 			\STATE Increase $\alpha_{i}$ at a rate such that
% %				\begin{align*}
% 					$
% 					\frac{d \alpha_{i}}{d \tau}	\gets \frac{\overline{b}_{i,e}  \cdot \alpha_i}{\nabla_{e} F(\vect{y})}  + \frac{1}{d \lambda}
% 					$
% %				\end{align*}
% 					\label{algo-packing:alpha}
% 		\ENDFOR
	\ENDWHILE 
	\IF{$x^{\pred}_{e} = 1$ \AND the predictive solution is still feasible} 
		\STATE set $x_{e}  \gets \frac{1}{1+\eta} x^{\pred}_{e} = \frac{1}{1+\eta}$ 
	\ELSE\STATE set $x_{e}  \gets  \frac{1}{1+\eta} y_{e}$
	\ENDIF
\ENDFOR
\end{algorithmic}
\caption{Algorithm for Packing Problem.}
	\label{algo:packing}
\end{algorithm}

Note that once the prediction becomes infeasible, the algorithm works with the given $b$ coefficients and outputs the solution computed by 
$y_e$ scaled by $1/(1+\eta)$.

\paragraph{Primal variables.}
The vector $\vect x$ is completed by $\vect z$ to form a complete solution to
the primal linear program, by setting $z_{S} = \prod_{e \in S} x_{e}  \prod_
{e \notin S} (1 - x_{e} )$. 


\paragraph{Dual variables.} 
Variables $\alpha_{i}$'s  are constructed in the algorithm. 
%Let $\vect{x} = (x_{e'})_{e'}$ and 
%Let $\vect{y} = (y_{e'})_{e'}$ be the current vectors of the algorithm and 
%Let $\vect{y}^{e}$ be the vector $\vect{y}$ just after the while loop with respect to resource $e$.
Define $\gamma = \frac{\mu}{\lambda} F(\vect{y})$ and 
$\beta_{e} = \frac{1}{\lambda} \cdot \nabla_{e} F(\vect{y})$.
%During the while loop with respect to resource $e$, by the observation above (i.e., $\nabla_{e} F(\vect{y})$ is constant during the iteration), 
%we have $\beta_{e} = \frac{1}{\lambda} \cdot \nabla_{e} F(\vect{y})$. 


The following lemma provides a lower bound on the $\alpha$ variables.
%Remark that the monotonicity of the gradient is crucial in the analysis of \cite{AzarBuchbinder16:Online-Algorithms}, in particular 
%to prove the bounds on $x$-variables. However, by our approach the gradient monotonicity  
%is not needed. 


\begin{restatable}{lemma}{BoundAlpha}
\label{lem:bound-alpha}
At any moment during the iteration related to element $e$,  
for every constraint $i$
it always holds that 
%$$
%\alpha_{i}	\geq  \frac{\beta_{e}}{\max_{e'}  \overline{b}_{i,e'}  \cdot d} 
%		\left[ \exp\biggl( \ln \bigl(1+ d\overline{\rho} \bigr) 
%				\cdot \sum_{e'} \overline{b}_{i,e'}  \cdot y_{e'}  \biggr) - 1 \right].
\begin{align}
\alpha_{i}	&\geq  \frac{\nabla_{e} F(\vect{y})}{\max_{e'}  \overline{b}_{i,e'}  \cdot d \lambda} 
		\left[ \exp\biggl( \ln \bigl(1+ d\overline{\rho} \bigr) 
				\cdot \sum_{e'} \overline{b}_{i,e'}  \cdot y_{e'}  \biggr) - 1 \right]. \label{eq:inv_alpha}
\end{align}
\end{restatable}
\begin{proof}
Fix a constraint $i$. We prove the claimed inequality by induction.
In the very beginning of the algorithm, when no elements are released yet, the inequality holds since both sides are 0.  
Consider any moment $\tau$ during the loop corresponding to an arriving element $e$.
Assume that the inequality holds at the beginning of the iteration step. We will show that the inequality still holds after the step.

As $F$ is a multilinear extension, by its very definition, $F$ is linear in $y_{e}$. Hence 
$\nabla_{e} F(\vect{y})$ is independent of $y_{e} $. 
Moreover, during the loop corresponding to resource $e$, only $y_{e} $ is modified, leaving  $y_{e'} $ unchanged for all $e' \neq e$.  
As as a result, 
$d \nabla_{e} F(\vect{y})/d \tau = 0$. Therefore, the rate at time $\tau$ of the the right-hand-side of Inequality~\eqref{eq:inv_alpha} is:

\begin{align*}
&\frac{\nabla_{e} F(\vect{y})}{\max_{e'}  \overline{b}_{i,e'}  \cdot d \lambda } \cdot \ln \bigl(1+ d\overline{\rho} \bigr) \cdot \overline{b}_{i,e}  \cdot
		 \frac{d y_{e} }{d \tau} \cdot \exp \biggl( \ln \bigl(1+ d\overline{\rho} \bigr) \cdot \sum_{e'} \overline{b}_{i,e'}  \cdot y_{e'}  \biggr) \\
%
&\leq  
\frac{\nabla_{e} F(\vect{y})}{\max_{e'}  \overline{b}_{i,e'}  \cdot d \lambda} \cdot \ln \bigl(1+ d\overline{\rho} \bigr) \cdot \overline{b}_{i,e}  \cdot
		 \frac{1}{\nabla_{e} F(\vect{y}) \cdot \ln(1+ d\overline{\rho} )}
		 \cdot \biggl( \frac{\max_{e'} \overline{b}_{i,e'}  \cdot d\lambda \cdot \alpha_{i}}{ \nabla_{e} F(\vect{y}) } + 1 \biggr) \\
%
&\leq \frac{\overline{b}_{i,e}  \cdot \alpha_{i}}{\nabla_{e} F(\vect{y})}  + \frac{1}{d \lambda} = \frac{d\alpha_{i}}{d \tau},
\end{align*}
%
where in the first inequality we use the induction hypothesis and the increasing rate of $y_{e}$.
So at any time during the iteration related to $e$, the increasing rate of the left-hand side is always larger than that of the right-hand side. 
Hence, the lemma follows.
\end{proof}

%\begin{proof}
%Before the release of resource $e$, $\beta_{e} = 0$ and the inequality trivially holds. 
%It is sufficient to show the lemma inequality for any moment $\tau$ after the release of resource $e$. 
%By the algorithm, $\beta_{e}$ is set up to $\frac{1}{\lambda} \nabla_{e} F(\vect{y})$ at the very beginning 
%of the loop corresponding to $e$ and 
%then $\beta_{e}$ will never be strictly increased, i.e., $\partial \beta_{e}/\partial \tau \leq 0$. 
%%
%%We will 
%%
%%Now consider the arrival of $e$. we will prove that the follow inequality holds 
%%for any moment $\tau$ during the loop corresponding to resource $e$ (note that $\beta_{e} = \frac{1}{\lambda} \nabla_{e} F(\vect{y})$ 
%%at time $\tau$). 
%%$$
%%\alpha_{i}	\geq  \frac{\nabla_{e} F(\vect{y})}{\max_{e'}  \overline{b}_{i,e'}  \cdot d \lambda} 
%%		\left[ \exp\biggl( \ln \bigl(1+ d\overline{\rho} \bigr) 
%%				\cdot \sum_{e'} \overline{b}_{i,e'}  \cdot y_{e'}  \biggr) - 1 \right].
%%$$
%%
%%Recall that $F$ is a linear extension, so by its definition, $F$ is linear w.r.t any fixed $y_{e} $; hence 
%%$\nabla_{e} F(\vect{y})$ is independent of $y_{e} $. 
%%During the loop corresponding to resource $e$, only $y_{e} $ is modified whereas other $y_{e'} $'s for $e' \neq e$ remain unchanged.  
%%As $\nabla_{e} F(\vect{y})$ is independent of $y_{e} $ and other $y_{e'} $'s remain unchanged, 
%%it holds that $\partial \nabla_{e} F(\vect{y})/\partial \tau = 0$.
%Therefore, the derivative of the right hand side of the lemma inequality according to $\tau$ is at most
%\begin{align*}
%&\frac{\beta_{e}}{\max_{e'}  \overline{b}_{i,e'}  \cdot d } \cdot \ln \bigl(1+ d\overline{\rho} \bigr) \cdot \overline{b}_{i,e}  \cdot
%		 \frac{\partial y_{e} }{\partial \tau} \cdot \exp \biggl( \ln \bigl(1+ d\overline{\rho} \bigr) \cdot \sum_{e'} \overline{b}_{i,e'}  \cdot y_{e'}  \biggr) \\
%%
%\leq&  
%\frac{\beta_{e}}{\max_{e'}  \overline{b}_{i,e'}  \cdot d} \cdot \ln \bigl(1+ d\overline{\rho} \bigr) \cdot \overline{b}_{i,e}  \cdot
%		 \frac{1}{\lambda \beta_{e} \cdot \ln(1+ d\overline{\rho} )}
%		 \cdot \biggl( \frac{\max_{e'} \overline{b}_{i,e'}  \cdot d \cdot \alpha_{i}}{ \beta_{e}} + 1 \biggr) \\
%%
%\leq& \frac{\overline{b}_{i,e}  \cdot \alpha_{i}}{\lambda \beta_{e}}  + \frac{1}{d \lambda} = \frac{\partial \alpha_{i}}{\partial \tau},
%\end{align*}
%%\begin{align*}
%%&\frac{\nabla_{e} F(\vect{y})}{\max_{e'}  \overline{b}_{i,e'}  \cdot d \lambda } \cdot \ln \bigl(1+ d\overline{\rho} \bigr) \cdot \overline{b}_{i,e}  \cdot
%%		 \frac{\partial y_{e} }{\partial \tau} \cdot \exp \biggl( \ln \bigl(1+ d\overline{\rho} \bigr) \cdot \sum_{e'} \overline{b}_{i,e'}  \cdot y_{e'}  \biggr) \\
%%%
%%&\leq  
%%\frac{\nabla_{e} F(\vect{y})}{\max_{e'}  \overline{b}_{i,e'}  \cdot d \lambda} \cdot \ln \bigl(1+ d\overline{\rho} \bigr) \cdot \overline{b}_{i,e}  \cdot
%%		 \frac{1}{\nabla_{e} F(\vect{y}) \cdot \ln(1+ d\overline{\rho} )}
%%		 \cdot \biggl( \frac{\max_{e'} \overline{b}_{i,e'}  \cdot d\lambda \cdot \alpha_{i}}{ \nabla_{e} F(\vect{y}) } + 1 \biggr) \\
%%%
%%&\leq \frac{\overline{b}_{i,e}  \cdot \alpha_{i}}{\nabla_{e} F(\vect{y})}  + \frac{1}{d \lambda} = \frac{\partial \alpha_{i}}{\partial \tau},
%%\end{align*}
%%
%where in the first inequality we use the induction hypothesis and the increasing rate of $y_{e}$.
%So the rate in the left-hand side is always larger than that in the right-hand side. 
%Hence, the lemma follows.
%\end{proof}


\begin{lemma}		\label{lem:packing-primal-feasible}
The primal variables constructed by the algorithm are feasible. 
\end{lemma}
%
\begin{proof}
%We prove the primal feasibility. 
First observe that if during the execution of the algorithm in the iteration related to some element $e$,  we have
$\sum_{e'} \overline{b}_{i,e'}  y_{e'}  > 1$ for some constraint $i$ then by Lemma~\ref{lem:bound-alpha},
$$
\alpha_{i} 
>  \frac{\nabla_{e} F(\vect{y})}{\max_{e'}  \overline{b}_{i,e'}  \cdot d\lambda } 
		\left[ \exp\biggl( \ln \bigl(1+ d\overline{\rho} \bigr)  \biggr) - 1 \right]
= \frac{\overline{\rho} \cdot \nabla_{e} F(\vect{y})}{\lambda \max_{e'}  \overline{b}_{i,e'} }
\geq \frac{\nabla_{e} F(\vect{y})}{\lambda \overline{b}_{i,e} }
$$
Therefore, $\sum_{i} \overline{b}_{i,e}  \alpha_{i} > \frac{1}{\lambda} \nabla_{e} F(\vect{y})$ and hence the algorithm would have 
stopped increasing $y_{e} $ at some earlier point. 
Therefore, every constraint $\sum_{e'} \overline{b}_{i,e'}  y_{e'}  \leq 1$ is always maintained during the execution of the algorithm. 
%(Note that by definition $\overline{b}_{i,e'} \geq b_{i,e'} \geq 0$ thus it also holds that $\sum_{i} b_{i,e'}  y_{e'}  \leq 1$.)

Secondly, we show primal feasibility (even when the prediction oracle provides an infeasible solution).
If the prediction oracle provides a feasible solution, we set $S_1=\{e: x^\pred_e=1\}$ and $S_2=\emptyset$.
Otherwise, let $e^{*}$ be the the first element for which the prediction oracle provides an infeasible solution. 
Let $S_{1}$ be the set of all resources $e$ such that $x^{\pred}_{e} = 1$ and $e$ is released before $e^{*}$.
Further let
$S_{2} = \{e: x^{\pred}_{e} = 1\} \setminus S_{1}$.
For every constraint $i$, 
\begin{align*}
\sum_{e} b_{i,e} x_{e} 
&= \sum_{e: x^{\pred}_{e} = 1} b_{i,e} x_{e} +  \sum_{e: x^{\pred}_{e} = 0} b_{i,e} x_{e} \\
%
&= \sum_{e \in S_{1}}  b_{i,e} x_{e} +  \sum_{e \in S_{2}} b_{i,e} x_{e}  +   \sum_{e: x^{\pred}_{e} = 0} b_{i,e} x_{e} \\
%
&\leq \frac{1}{1+\eta} \cdot 1
+ \eta \cdot \sum_{e \in S_{2}} \overline{b}_{i,e} x_{e} 
+ \eta \cdot \sum_{e: x^{\pred}_{e} = 0} \overline{b}_{i,e} x_{e} \\
%
&= \frac{1}{1+\eta}  
+ \frac{\eta}{1+\eta}  \sum_{e \in S_{2}} \overline{b}_{i,e} y_{e} 
+ \frac{\eta}{1+\eta}  \sum_{e: x^{\pred}_{e} = 0} \overline{b}_{i,e} y_{e} \\
%
&\leq \frac{1}{1+\eta}
+ \frac{\eta}{1+\eta}  \sum_{e} \overline{b}_{i,e} y_{e} \\
%
&\leq \frac{1}{1+\eta}
+ \frac{\eta}{1+\eta}  \cdot 1 = 1.
\end{align*}
%
The first inequality is due to: (1) the feasibility of the prediction restricted on $S_{1}$, i.e., 
$\sum_{e \in S_{1}} b_{i,e} x^{\pred}_{e} \leq 1$; and (2) 
$x_{e} = \frac{1}{1 + \eta} = \frac{1}{1 + \eta} x^{\pred}_{e}$
for $e \in S_{1}$; and (3) the definitions of $\overline{b}_{i,e}$ in Algorithm~\ref{algo:packing}.
The third equality follows by the algorithm: 
$x_{e} = \frac{1}{1 + \eta} \cdot y_{e}$
for $e \notin S_{1}$. 
The last inequality holds since $\sum_{e} \overline{b}_{i,e} y_{e} \leq 1$ by the observation made at the beginning of the lemma.
Hence, $\sum_{e} b_{i,e} x_{e} \leq 1$ for every constraint $i$.

Besides, by definition of $z_{S} = \prod_{e \in S} x_{e} \prod_{e \notin S} (1 - x_{e})$
where $0 \leq x_{e} \leq 1$ for all $e$, the identity $\sum_{S} z_{S} = 1$ always holds. 
In fact, if one chooses an element $e$ with probability $x_{e}$ then $z_{S}$ is the 
probability that the set of selected elements is $S$. So the total probability $\sum_{S} z_{S}$ must be 1.
Similarly, 
$\sum_{S: e \in S} z_{S} = x_{e} \sum_{S' \subset E \setminus \{e\}} \prod_{e' \in S'} x_{e'} \prod_{e' \notin S'} (1 - x_{e'}) = x_{e}$
since $\sum_{S' \subset E \setminus \{e\}} \prod_{e' \in S'} x_{e'} \prod_{e' \notin S'} (1 - x_{e'}) = 1$ (by the same argument). 

Therefore, the solution $(\vect{x}, \vect{z})$ is primal feasible.
\end{proof}

\begin{lemma}
The dual variables defined as above are feasible. 
\end{lemma}
%
\begin{proof}
The first dual constraint $\sum_{i} b_{i,e} \alpha_{i} \geq \beta_{e}$ is satisfied by the while loop condition of the algorithm
and the definition of $\beta_{e}$.
The second dual constraint $\gamma + \sum_{e \in S} \beta_{e} \geq f(\vect{1}_{S})$ reads
\begin{align*}
	\frac{1}{\lambda} \sum_{e \in S} \nabla_{e} F(\vect{y}) + \frac{\mu}{\lambda} F(\vect{y}) \geq F(\vect{1}_{S}), 
\end{align*}
which is, by arranging terms, exactly the $(\lambda, \mu)$-max-local smoothness of $F$. 
(Recall that $F(\vect{1}_{S}) = f(\vect{1}_{S})$.)
Hence, the lemma follows.
\end{proof}

%We are now ready to prove the main theorem. 

%%
%\begin{theorem}	\label{thm:packing}
%Assume that the multilinear extension is $(\lambda, \mu)$-max-locally-smooth.
%Then, the algorithm is $\Omega\bigl( \frac{\lambda}{2\ln(1+ d\rho/\eta ) + \mu} \bigr)$-competitive
%where ...
%\end{theorem}

\Packing*
%
\begin{proof}
\paragraph{Robustness.}
First, we bound the increases of $F(\vect{y})$ and of the dual objective value --- which we denote by $D$ --- at any time $\tau$ in the execution of 
Algorithm~\ref{algo:packing}.
The derivative of $F(\vect{y})$ with respect to $\tau$ is:
%
\begin{align*} %	\label{eq:packing-primal}
\nabla_{e} F(\vect{y}) \cdot \frac{d y_{e} }{ d \tau}
= \nabla_{e} F(\vect{y}) \cdot\frac{1}{\nabla_{e} F(\vect{y}) \cdot \ln(1+ d\overline{\rho})}
= \frac{1}{\ln(1+ d\overline{\rho})}
\end{align*}
%
Besides, the rate of the dual at time $\tau$ is:
\begin{align*}
\frac{d D}{d \tau} 
&=  \sum_{i} \frac{d \alpha_{i}}{d \tau} + \frac{d \gamma}{d \tau}  
= \sum_{i: \overline{b}_{i,e}  > 0} \biggl( \frac{\overline{b}_{i,e}  \cdot \alpha_{i}}{\nabla_{e} F(\vect{y})}  + \frac{1}{d\lambda} \biggr) 
		+ \frac{\mu}{\lambda} \frac{d F(\vect{y})}{d \tau}  \\
%
&= \sum_{i: \overline{b}_{i,e}  > 0} \frac{\overline{b}_{i,e}  \cdot \alpha_{i}}{\nabla_{e} F(\vect{y})}  +  \sum_{i: \overline{b}_{i,e}  > 0} \frac{1}{d\lambda}
		+ \frac{\mu}{\lambda} \cdot \frac{1}{\ln(1+ d\overline{\rho})} \\
%
&\leq \frac{2}{\lambda} + \frac{\mu}{\lambda \cdot \ln(1+ d\overline{\rho})}
=  \frac{2\ln(1+ d\overline{\rho}) + \mu}{\lambda \cdot \ln(1+ d\overline{\rho})},
\end{align*}
where the inequality holds since during the algorithm
$\sum_{i} \overline{b}_{i,e}  \cdot \alpha_{i} \leq  \frac{1}{\lambda} \nabla_{e} F(\vect{y})$.
Hence, the ratio between $F(\vect{y})$ and the dual $D$ is at least $\frac{\lambda}{2\ln(1+ d\overline{\rho}) + \mu}$.

Besides, $x_{e} = \frac{1}{1 + \eta} \geq \frac{1}{1 + \eta} y_{e}$ if $x^{\pred}_{e} = 1$ and the predictive solution is still feasible; and
 $x_{e} = \frac{1}{1 + \eta} y_{e}$ otherwise. Therefore, 
 $\vect{x} \geq \frac{\vect{y}}{1 + \eta} $ and so 
 $F(\vect{x}) \geq F\bigl(\frac{\vect{y}}{1 + \eta}\bigr)$ by 
 monotonicity\footnote{Note that this is the only step in the analysis we use the monotonicity of $f$, which implies the monotonicity of $F$.} 
 of $F$. 
 Hence the robustness is at least 
$$
\frac{F(\vect{x})}{F(\vect{y})} \cdot \frac{\lambda}{2\ln(1+ d\overline{\rho}) + \mu}
\geq \min_{\vect{0} \leq \vect{u} \leq \vect{1}} \frac{F(\frac{1}{1 + \eta} \vect{u})}{F(\vect{u})} \cdot \frac{\lambda}{2\ln(1+ d\rho/\eta ) + \mu} 
$$ 
where the latter is due to $\overline{\rho} \leq \rho/\eta$.

\paragraph{Consistency.} By our algorithm, for every element $e$, if $x^{\pred}_{e} = 1$ (and the prediction 
oracle provides a feasible solution) 
then $x_{e} = \frac{1}{1+\eta}$. Hence, the consistency of the algorithm 
$F(\vect{x})/F(\vect{x}^{\pred}) \geq F(\frac{\vect{x}^{\pred}}{1 + \eta})/F(\vect{x}^{\pred}) \geq r(\eta)$.
\end{proof}

%Note that the competitive ratio is 
%the same up to a constant factor as the performance guarantee for maximizing a linear function
%under packing constraints. Specifically, if function $f$ is linear then the smooth parameters are
%$\lambda = \mu = 1$.
%
%\paragraph{Remark.} One can define $\overline{b}_{e} = b_{e}/(\frac{1}{1+\eta})$ instead of 
%$\overline{b}_{e} = b_{e} (1 + \eta)$ but be careful when $\eta = 1$. One can assume that 
%$\eta \in (0,1)$.
\subsection{Applications}

\subsubsection{Applications to linear functions}
When the objective $f$ can be expressed as a monotone linear functions, its multilinear extension $F$ 
is $(1,1)$-locally-smooth. Moreover, $r(\eta) = 1/(1+\eta)$. Consequently, 
Algorithm \ref{algo:packing} provides a $1/(1 + \eta)$-consistent
and $O\bigl(1/\ln(1+ d \rho/\eta)\bigr)$-robust fractional solution for the
online linear packing problem.


\subsubsection{Applications to online submodular maximization}	\label{sec:sub-max}
Consider the online problem of maximizing a monotone submodular function subject to packing constraints. 
A set-function $f: 2^{\mathcal{E}} \rightarrow \mathbb{R}+$ is \emph{submodular} if
$f(S \cup e) - f(S) \geq f(T \cup e) - f(T)$ for all $S \subset T \subseteq \mathcal{E}$. 
Let $F$ be the multilinear extension of a monotone submodular function $f$. Function $F$
admits several useful properties: (i) if $f$ is monotone then so is $F$; (ii) $F$ is concave in any
positive direction, i.e., $\nabla F(\vect{x}) \geq \nabla F(\vect{y})$ for all $\vect{x} \leq \vect{y}$
($\vect{x} \leq \vect{y}$ means $x_{e} \leq y_{e} ~\forall e$). 

\begin{lemma}	\label{lem:sub-max-locally-smooth}
Let $f$ be an arbitrary monotone submodular function. Then, its multilinear extension 
$F$ is (1,1)-locally-smooth.
\end{lemma}
%
\begin{proof}
As $F$ is the linear extension of a submodular function, 
$\nabla_{e} F(\vect{x}) = \mathbb{E}_{R}\bigl[ f\bigl(\vect{1}_{R \cup \{e\}}\bigr) - f\bigl(\vect{1}_{R}\bigr) \bigr]$
where $R$ is a random subset of $\mathcal{E} \setminus \{e\}$ such that $e'$ is included with probability $x_{e'} $.
For any subset $S = \{e_{1}, \ldots, e_{\ell}\}$, we have
\begin{align*}
F(\vect{x}) + \sum_{e \in S} \nabla_{e} F(\vect{x}) 
&= F(\vect{x}) + \sum_{e \in S} \mathbb{E}_{R} \bigl[ f\bigl(\vect{1}_{R \cup \{e\}}\bigr) - f\bigl(\vect{1}_{R}\bigr) \bigr] \\
%
&= \mathbb{E}_{R} \biggl[ f(\vect{1}_{R}) + \sum_{e \in S} \bigl[ f\bigl(\vect{1}_{R \cup \{e\}}\bigr) - f\bigl(\vect{1}_{R}\bigr) \bigr] \biggl] \\
%
&\geq \mathbb{E}_{R} \biggl[ f(\vect{1}_{R}) + \sum_{i=1}^{\ell} \bigl[ f(\vect{1}_{R \cup \{e_{1}, \ldots, e_{i}\}}) - 
								f(\vect{1}_{R \cup \{e_{1}, \ldots, e_{i-1}\}}) \bigr] \biggr] \\
%
&= \mathbb{E}_{R} \bigl[ f(\vect{1}_{R \cup S})  \bigl]  \geq \mathbb{E}_{R} \bigl[ f(\vect{1}_{S})  \bigl] \\
%
&= F\bigl( \vect{1}_{S} \bigr)
\end{align*}
the first inequality is due to the submodularity $f$ and the second one due to its monotonicity.
The lemma follows.
\end{proof}



%\begin{lemma}
%Algorithm \ref{algo:packing} yields a  $O(\ln r(\mathcal{M}))$-competitive fractional solution where $r$ is the rank of 
%matroid $\mathcal{M}$.
%\end{lemma}

%The previous lemma and Theorem \ref{thm:packing} lead to the following result. 

\begin{proposition}	\label{prop:max-submodular}
For any $0 < \eta \leq 1$, Algorithm \ref{algo:packing} gives a $(1 - \eta)$-consistent
and $O\bigl(1/\ln(1+ d \rho/\eta)\bigr)$-robust fractional solution to
the problem of online submodular maximization under packing constraints.
\end{proposition}
%
\begin{proof}
We first bound $r(\eta) = \min_{\vect{0} \leq \vect{u} \leq \vect{1}} F(\frac{1}{1 + \eta} \vect{u})/F(\vect{u})$. 
By the non-negativity and the concavity in positive direction of $F$, for any set $S \subseteq \mathcal{E}$, 
%
\begin{align}
F\biggl(\frac{\vect{u}}{1 + \eta}  \biggr) 
&\geq F(\vect{u}) - \biggl \langle \nabla F\biggl(\frac{\vect{u}}{1 + \eta} \biggr), \vect{u} - \frac{\vect{u}}{1 + \eta} \biggr \rangle
\label{eq:max-sub-1}
\\
F\biggl(\frac{\vect{u}}{1 + \eta} \biggr) 
 &\geq F(\vect{0}) + \biggl \langle \nabla F\biggl(\frac{\vect{u}}{1 + \eta} \biggr), \frac{\vect{u}}{1 + \eta}\biggr \rangle
 \geq  \biggl \langle \nabla F\biggl(\frac{\vect{u}}{1 + \eta} \biggr), \frac{\vect{u}}{1 + \eta} \biggr \rangle.
\label{eq:max-sub-2}
\end{align}
Therefore,
\begin{align}
\frac{F\biggl(\frac{\vect{u}}{1 + \eta}\biggr)}{F(\vect{u})}
&\geq 1 - \frac{\eta}{1 + \eta} \cdot \frac{\biggl \langle \nabla F\biggl(\frac{\vect{u}}{1 + \eta}\biggr), \vect{u} \biggr \rangle}{F(\vect{u})} 	\tag{by (\ref{eq:max-sub-1})}\\
%
&\geq 
1 - \frac{\eta}{1 + \eta} \cdot \frac{\biggl \langle \nabla F\biggl(\frac{\vect{u}}{1 + \eta}\biggr), \vect{u} \biggr \rangle}{F\biggl(\frac{\vect{u}}{1 + \eta}\biggr)}  	\tag{by monotonicity of $F$}\\
%
&\geq 
1 - \frac{\eta}{1 + \eta} \cdot \frac{\biggl \langle \nabla F\biggl( \frac{\vect{u}}{1 + \eta} \biggr), \vect{u} \biggr \rangle}{\biggl \langle \nabla F\biggl(\frac{\vect{u}}{1 + \eta}\biggr), \frac{\vect{u}}{1 + \eta}\biggr \rangle}
	\tag{by (\ref{eq:max-sub-2})} \\
&= 1 - \eta. 	\notag
\end{align}
So, $r(\eta) \geq 1 - \eta$. Therefore, the proposition holds by Theorem~\ref{thm:packing} and by the (1,1)-locally-smoothness of $F$ (Lemma~\ref{lem:sub-max-locally-smooth}).
\end{proof}

One can derive online randomized algorithms for the integral variants of these problems by rounding the fractional solutions.
For example, using the online contention resolution rounding schemes \cite{FeldmanSvensson16:Online-contention}, 
one can obtain randomized algorithms for several specific constraint polytopes, for example, knapsack polytopes, 
matching polytopes and matroid polytopes. 




%!TEX root = main.tex

\section{Ad-Auction Revenue Maximization}		\label{sec:ad-auction}

\paragraph{Problem.} In the problem, we are given $m$ buyers, each buyer $1 \leq i \leq m$ has a budget $B_{i}$.  
Items arrive online and at the arrival of item $e$, 
each buyer $1 \leq i \leq n$ provides a bid $b_{i,e} \ll B_{i}$ for buying $e$.
In the integral variant of the problem, one needs to allocate the item to at most one of the buyers. For every buyer, the total bids of the allocated items should not exceed the budget. Note that this forms a packing constraint, similar to the ones studied in the previous section, except that we don't normalize the constraints by the factor $1/B_i$.
In the fractional variant of the problem, items can be allocated in fractions to buyers, not exceeding a total of $1$ among the fractions.  In both variants, the objective is to maximize the total revenue received from all buyers, which is the sum over all  allocations of corresponding bid, possibly multiplied with the fraction of the allocation.

In out setting, every item $e$ arrives with a prediction, suggesting a buyer to whom to allocate that item, if any.
For convenience we denote the items by the integers from $1$ to $n$.

\subsection{The algorithm}

\paragraph{Formulation.}
The fractional variant of problem can be expressed as the following linear program, where $x_{i,e}$ indicates the fraction at which item $e$ is allocated to buyer $i$.

\begin{minipage}[t]{0.45\textwidth}
\begin{align*}
\max  \sum_{e=1}^{n} & \sum_{i=1}^{m} b_{i,e} x_{i,e} & \\
\sum_{i=1}^{m} x_{i,e}  &\leq 1 & &  \forall e & (\beta_e) \\
\sum_{e=1}^{n} b_{i,e} x_{i,e} &\leq  B_{i} & & \forall i & (\alpha_i) \\
x_{i,e} &\geq 0 & & \forall i, e\\
\end{align*}
\end{minipage}
\quad
\begin{minipage}[t]{0.5\textwidth}
\begin{align*}
\min \sum_{i=1}^{n} B_{i} \alpha_{i} &+ \sum_{e=1}^{m} \beta_{e} \\ 
b_{i,e} \alpha_{i} + \beta_{e} &\geq b_{i,e}  & &  \forall i, e & (x_{i,e})\\  \\
\alpha_{i}, \beta_{e} &\geq 0 & & \forall i,e \\
\end{align*}
\end{minipage}
%
In the primal linear program, the first constraint ensures that an item can be allocated to at most one buyer and 
the second constraint ensures that a buyer $i$ does not spend more than the budget $B_{i}$.
The dual of the relaxation is presented in the right.  

\paragraph{Algorithm.} 

For the sake of simplicity in the algorithm description, we introduce a fictitious buyer, denoted as $0$, such that  
$b_{0, e} = 0$ for all items $e$. The non-assignment of an item $e$ to any buyer can be seen as being assigned 
to the fictitious buyer $0$ with revenue 0.
The purpose of buyer $0$   is to simplify the description of the algorithm.
 We define the constant $C = (1 + R_{\max})^{\frac{\eta}{R_{\max}}}$ where 
$R_{\max} = \max_{i,e} \frac{b_{i,e}}{B_{i}}$. Adapting the primal-dual algorithm presented in the previous section, we obtain the following algorithm for the fractional ad-auction problem.

\begin{algorithm}[ht]
\begin{algorithmic}[1]  
\STATE All primal and dual variables are initially set to 0.
\STATE For the analysis, we maintain for every buyer $i$ two sets $N(i), M(i)$, both initially empty. 
\FOR{each arrival of a new item $e$}
	\STATE Let $i^{*}$ be the buyer such that $x^{\pred}_{i^{*},e} = 1$. If either the prediction is not feasible or 
	 	there is no such $i^{*}$ (i.e., $x^{\pred}_{i',e} = 0 ~\forall i'$) then
		set $i^{*} = 0$.  
	\STATE Let $i$ be the buyer that maximizes $b_{i',e} (1 - \alpha_{i'})$. 
		If $b_{i,e} (1 - \alpha_{i}) \leq 0$ then 
		set $i = 0$.
	\STATE Set $\beta_{e} \gets \max \bigl \{0,  b_{i,e} (1 - \alpha_{i}) \bigr \}$	 \COMMENT{for the purpose of analysis}
	\IF{$b_{i,e} < b_{i^{*},e}$}
		\STATE Set $x_{i,e} \gets \eta$ and $x_{i^{*},e} \gets 1-\eta$.
		\STATE Define $\overline{b}_{i,e} = b_{i,e}/\eta$ 
		\STATE $N(i^{*}) \gets N(i^{*}) \cup \{e\}$. 		\COMMENT{for the purpose of analysis}
	\ELSE 
		\STATE Set $x_{i,e} \gets 1$ and define $\overline{b}_{i,e} = b_{i,e}$
		\COMMENT{includes case $\vect x^\pred$ is infeasible}
	\ENDIF
	\STATE Update $M(i) \gets M(i) \cup \{e\}$.  \COMMENT{for the purpose of analysis}
	\STATE Update $\alpha_{i} \gets \alpha_{i}\left( 1 + \frac{b_{i,e}}{B_{i}} \right) 
										+  \frac{b_{i,e}}{B_{i}} \cdot \frac{1}{C - 1}$.	
\ENDFOR
% \STATE Scale $\vect x$ by factor $1-R_\max$
\end{algorithmic}
\caption{Algorithm for Ad-Auctions Revenue Maximization.}
\label{algo:ad-auctions}
\end{algorithm}




\begin{lemma}	\label{lem:alpha}
For every $i$, it always holds that 
$$
\alpha_{i} \geq \frac{1}{C - 1} \biggl( C^{\frac{\sum_{e \in M(i)} b_{i,e}}{\eta B_{i}}} - 1 \biggr).
$$
\end{lemma}
%
\begin{proof}
We adapt the proof in \cite{BuchbinderNaor09:The-Design-of-Competitive} 
with a slight modification. 
The primal inequality $\alpha_i$ is proved by induction on the number of released items. 
Initially, when no item is released, the inequality  is trivially true. 
Assume that the inequality holds right before the arrival of an item $e$. 
The inequality remains unchanged for all but the buyer $i$ selected in line 5 of the algorithm, i.e.\ $i$ maximizes $b_{i,e} (1 - \alpha_{i})$. 
We denote by $\alpha_i$  the value before the update triggered by the arrival of $e$ and $\alpha'_i$ its value after the update.
We have
\begin{align*}
\alpha'_{i}
&= \alpha_{i} \cdot \left( 1 + \frac{b_{i,e}}{B_{i}} \right) 
										+  \frac{b_{i,e}}{B_{i}} \cdot \frac{1}{C - 1}	\\
%
&\geq \frac{1}{C - 1} \biggl( C^{\frac{\sum_{e' \in M(i) \setminus e } b_{i,e}}{\eta B_{i}}} - 1 \biggr)
			\cdot \left( 1 + \frac{b_{i,e}}{B_{i}} \right) 
										+   \frac{b_{i,e}}{B_{i}} \cdot \frac{1}{C - 1} \\
%
&=  \frac{1}{C - 1} \biggl( C^{\frac{\sum_{e' \in M(i) \setminus e } b_{i,e}}{\eta B_{i}}} 
						\cdot \left( 1 + \frac{b_{i,e}}{B_{i}} \right)  - 1 \biggr) 	\\						
%
&\geq  \frac{1}{C - 1} \biggl( C^{\frac{\sum_{e' \in M(i) \setminus e } b_{i,e}}{\eta B_{i}}} 
						\cdot  C^{\frac{b_{i,e} }{\eta B_{i}}}   - 1 \biggr) \\
%
&= \frac{1}{C - 1} \biggl( C^{\frac{\sum_{e \in M(i)} b_{i,e}}{\eta B_{i}}} - 1 \biggr).
\end{align*}
%
The first inequality is due to the induction hypothesis.
The second inequality is due to this sequence of transformations. 
For any $0 < y \leq z \leq 1$ we have
\begin{align*}
\frac{\ln(1+y)}{y} &\geq \frac{\ln(1+z)}{z} 
\\
\Leftrightarrow \quad \ln(1+y) &\geq \ln(1+z)y / z 
\\
\Leftrightarrow \quad 1+y &\geq (1+z)^{y/z},
\end{align*}
which we apply with $y=b_{i,e}/{B_i}$ and $z=R_{\max}$. By the definition of $C$ we obtain
$$
\biggl( 1+R_{\max} \biggr)^{\frac{1}{R_{\max}} \cdot \frac{b_{i,e} }{B_{i}}} 
= C^{\frac{b_{i,e}}{\eta B_{i}}}.
$$
This completes the induction step.
\end{proof}



\begin{lemma}	\label{lem:ad-auctions-primal-feasibility}
The primal solution is feasible up to a factor $(1 + R_{\max})$.
\end{lemma}
%
\begin{proof}
The first primal constraint $\sum_{e} x_{i,e} \leq 1$ follows the values of $x_{i,e}$ and $x_{i^{*},e}$ in Algorithm~\ref{algo:ad-auctions}.
For the second primal constraint, we first prove that $\sum_{e=1}^{m} b_{i,e} \leq  B_{i}$ for every $i$. 
By Lemma~\ref{lem:alpha}, for every $i$, 
$$
\alpha_{i} \geq \frac{1}{C - 1} \biggl( C^{\frac{\sum_{e \in M(i)} b_{i,e}}{\eta B_{i}}} - 1 \biggr)
$$
So whenever $\sum_{e \in M(i)} b_{i,e} \geq \eta B_{i}$, we have $\alpha_{i} \geq 1$; so the algorithm 
stops allocating items to buyer $i$. Hence, the buyer $i$ can be allocated at most one additional item once her budget is 
already saturated. Hence, $\sum_{e \in M(i)} b_{i,e}  < \eta B_{i} + \max_{e} b_{i,e}$. 
Therefore, 
\begin{align*}
\sum_{e=1}^{m} b_{i,e}x_{i,e} = 
\sum_{e \in M(i)} b_{i,e}x_{i,e} + \sum_{e \in N(i)} b_{i,e}x_{i,e}
<  B_{i} + \max_{e} b_{i,e}
\end{align*}
where the latter is due to the feasibility of $N(i)$, the set of items assigned by the prediction to buyer $i$,
i.e., $\sum_{e \in N(i)} b_{i,e}x_{i,e} \leq  (1 - \eta) \sum_{e} b_{i,e} x^{\pred}_{i,e} \leq (1 - \eta) B_{i}$.
So, $\sum_{e=1}^{m} b_{i,e}x_{i,e} \leq B_{i}(1 + R_{\max})$.
\end{proof}

\begin{theorem}
The algorithm is $(1 - \eta)$-consistent and $\frac{1 - 1/C}{1 + R_{\max}}$-robust.
The robustness tends to $1 - e^{-\eta}$ when $R_{\max}$ tends to 0.
\end{theorem}
%
\begin{proof}
First, we establish robustness. At the arrival of item $e$, the increase in the primal is 
$$
\begin{cases}
	(1-\eta) \cdot b_{i^{*},e} + \eta \cdot b_{i,e} & \text{ if } b_{i,e} < b_{i^{*},e}, \\
	b_{i,e} & \text{ if } b_{i,e} \geq b_{i^{*},e} 
\end{cases}
$$
which is always larger than $b_{i,e}$. Besides, the increase in the dual is 
%\begin{align*}
%B_{i} \Delta \alpha_{i} + B_{i^{*}} \Delta \alpha_{i^{*}} + \beta_{e}
%&= \frac{\eta}{1+\eta} \biggl( b_{i,e} \alpha_{i} + \frac{b_{i,e}}{c-1} \biggr)
%+ \frac{1}{1+\eta} \biggl( b_{i^{*},e} \alpha_{i^{*}} + \frac{b_{i^{*},e}}{c-1} \biggr) \\
%%
%& \qquad + \frac{\eta}{1+\eta} b_{i,e} (1 - \alpha_{i}) + \frac{1}{1+\eta} b_{i^{*},e} (1 - \alpha_{i^{*}}) \\
%%
%&= \biggl( \frac{1}{1+\eta} \cdot b_{i^{*},e} +  \frac{\eta}{1+\eta} \cdot b_{i,e} \biggr) \biggl( 1 + \frac{1}{c-1} \biggr)
%\end{align*}
\begin{align*}
B_{i} \Delta \alpha_{i}  + \beta_{e}
&= b_{i,e} \alpha_{i} + \frac{b_{i,e}}{C-1} 
 + b_{i,e} (1 - \alpha_{i}) 
 = \biggl( 1 + \frac{1}{C-1} \biggr) b_{i,e}
 = \frac{C}{C - 1}.
 \end{align*}
Hence, by Lemma~\ref{lem:ad-auctions-primal-feasibility}, the robustness is 
$\frac{C-1}{C} \cdot \frac{1}{1+R_{\max}}$.

Secondly we establish consistency quite easily. At every time the prediction solution gets a profit $b_{i^{*},e}$, the algorithm achieves a profit 
at least $(1 - \eta) b_{i^{*},e}$.
\end{proof}

%\paragraph{Upper bound.} The upper bound follows the construction in \cite{EsfandiariKorula18:Allocation-with}. 
%The latter considers the Robust Budgeted Allocation problem with forecast. Even though their model differs from ours, their construction can be used to prove an upper bound to our problem. 
%
%\begin{theorem}[\cite{EsfandiariKorula18:Allocation-with}]
%For any $0 < \eta < 1$,  any $(1 - \eta)$-consist algorithm for the fraction variant of the problem has a robustness at most $1 - \eta e^{-\eta}$.
%\end{theorem}
%%
%\begin{proof}
%We first construct an instance in which every buyer $i$ has unit budget ($B_{i} = 1$) and makes unit bids $b_{i,e} \in \{0,1\}$. In this setting we are facing a standard maximum cardinality bipartite matching problem in the vertex arrival model. 
%Then we extend the construction to satisfy the assumption $b_{i,e} \ll B_{i}$.
%
%Fix $0 < \eta \leq 1$ and an arbitrary $(1-\eta)$-consistent fractional algorithm. 
%Consider an instance consisting of $n$ buyers, each with a unit budget, and $n$ items. The \emph{load} of a buyer is the total item fraction
%assigned to the buyer. In the instance, the first $(1-\eta)n$ items\footnote{For notational convenience we assume $(1-\eta)n$ to be integer.}
%are connected to all buyers, i.e.,  $b_{i,e} = 1$ for all $1 \leq e \leq (1-\eta)n$, $1 \leq i \leq m$. The prediction assigns every 
%item $1 \leq e \leq (1-\eta)n$ to a different buyer $1 \leq i \leq m$. After the first $(1-\eta)n$ items, we mark $(1-\eta)n$ buyers 
%with smallest load, namely buyers $a_{1}, \ldots, a_{(1-\eta)n}$ with loads $\ell_{1}, \ldots, \ell_{(1-\eta)n}$, respectively. 
%For every subsequent item $e > (1-\eta)n$, we do the following: (i) at its arrival, the item is connected to every currently unmarked buyer, 
%and (ii) after its (fractional) assignment, mark the buyer with the smallest load, namely buyer $a_{e}$ with load $\ell_{e}$. 
%The prediction does not assign any item $e > (1-\eta)n$
%to any buyer. 
% 
%In this instance, the prediction has revenue $(1 - \eta) n$ and the offline optimal solution has revenue $n$ 
%(by assigning every item $e$ to $a_{e}$, the buyer marked after its arrival).  
%
%As the algorithm is $(1 - \eta)$-consistent, $\sum_{j=1}^{n} \ell_{j} \geq (1 - \eta) \cdot (1 - \eta) n$. 
%Moreover, $\ell_{1}, \ldots, \ell_{(1-\eta)n}$ are the $(1-\eta)n$ smallest loads, 
%so $\sum_{j=1}^{(1-\eta)n} \ell_{j} \leq (1 - \eta) \cdot (1 - \eta) n$.
%
%Besides, for each $(1 - \eta) n < j \leq n$, the first $j$ items provide the total load at most $j$, and 
%the total load on buyers $a_{1}, \ldots, a_{j-1}$ is $\sum_{k=1}^{j-1} \ell_{k}$. Therefore,
%$$
%\ell_{j} \leq \frac{j - \sum_{k=1}^{j-1} \ell_{k}}{n - j + 1}.
%$$ 
%Hence, the best robustness is the value of the following LP.
%    
%\begin{align*}
%& \max \frac{1}{n} \sum_{j=1}^{n} \ell_{j} &  \\
% \sum_{j=1}^{(1-\eta)n} \ell_{j}  &\leq (1 - \eta)^{2} n \leq \sum_{j=1}^{n} \ell_{j}  &  \\
%\ell_{j} &\leq \frac{j - \sum_{k=1}^{j-1} \ell_{k}}{n - j + 1} \qquad& \forall (1-\eta)n < j \leq n \\
%0 &\leq \ell_{j} \leq 1 \qquad& \forall 1 \leq j \leq n\\
%\end{align*}    
% %
%It is proved in \cite[Theorem 3.23]{EsfandiariKorula18:Allocation-with} that this LP has the maximum value of $1 - \eta e^{-\eta}$    
%in which 
%\begin{align*}
%\ell_{1} &= \ldots = \ell_{(1-\eta)n} = 1-\eta, \\
%\ell_{j} &= \min \biggl \{ 1, (1 - \eta) + \sum_{k = (1-\eta)n + 1}^{j} \frac{1}{n-k+1} \biggr \}
%\leq (1 - \eta) + \sum_{k = (1-\eta)n + 1}^{(1-\eta)n + \xi n} \frac{1}{n-k+1}
%\quad \forall j > (1 - \eta)n,
%\end{align*}
%where $\xi = \eta(1 - e^{-\eta})$. Hence, in the instance, no $(1-\eta)$-consistent fractional algorithm has the robustness larger 
%than $1 - \eta e^{-\eta}$. 
%
%In order to transform the above instance into one in which $b_{i,e} \ll B_{i}$ holds for all $i$ and $e$, it is sufficient to define, for every buyer $i$,  
%$B_{i} = B$ where $B$ is a large number and for each item $e$, create $B$ copies which are released in a row. 
%The same arguments and analysis hold. 
%\end{proof}

\subsection{Experiments}

We run an experimental evaluation of Algorithm~\ref{algo:ad-auctions}. Source code is publicly available\footnote{\url{http://www.lip6.fr/christoph.durr/packing}}.
For this purpose we constructed a random instance on 100 buyers and 10,000 items, adapting the model described in \cite{Lavastida20:Predictions-Matching}. Every item $e$ receives a bid from exactly 6 random buyers. This means that in average every bidder makes 600 bids. The value of the bid follows a lognormal distribution with mean $1/2$ and deviation $1/2$ as well. By choosing the budget of the bidders we can tune the hardness of the instance, under the constraint that $R_{\max}$ remains reasonably small. We choose the budget as a $1/10$ fraction of the total bids, leading to a value $R_{\max}\approx 0.19$.

The competitive ratio is determined using the fractional offline solution. The integrality gap of our instance is very close to $1$, roughly $1.0001$ in fact. For the prediction, we computed first the optimal integral solution, which is a partial mapping from items to buyers. This solution was perturbed as suggested by \cite{BamasMaggiori20:The-Primal-Dual-method}: Every item was mapped, with probability $\epsilon$,  to a uniformly chose random buyer, among the buyers who bid on the item.

The experiments are shown in Figure~\ref{fig:experiments}, and indicate clearly the benefit of the prediction when its perturbation is small.  Note that the ratio is not close to $1$ in that case for $\eta$ close to zero, because the algorithm does not follow blindly the prediction. 
Infeasibility of the prediction was detected at 82\% of the input sequence for $\epsilon=0.01$ and at 64\% for $\epsilon=0.1$.
As expected, with the performance degrades with the perturbation of the prediction.  
Note that the observed robustness is not monotone in $\eta$, unlike the bound shown in this paper.  We think that this performance degradation for $\eta$ around $1/2$ is due to the rather simplistic mixture between the primal-dual solution and the predicted solution.

\begin{figure}[ht]
\begin{center}
	\begin{tikzpicture}[scale=1]
		\begin{axis}[
xlabel={$\eta$},
ylabel={robustness},
yticklabel style={/pgf/number format/precision=5},
legend style={at={(1,1)},anchor=north east},
legend entries={$\epsilon=0$,$\epsilon=0.01$,$\epsilon=0.1$}
]
\addplot [blue,mark=*] table {ratio_0.0.dat};
\addplot [green,mark=o] table {ratio_0.01.dat};
\addplot [red,mark=x] table {ratio_0.1.dat};
\end{axis}
	\end{tikzpicture}
\end{center}
\caption{Experimental evaluation}
\label{fig:experiments}
\end{figure}

%\input{covering}



\section{Conclusion}
%In this paper, we have presented primal-dual approaches based on configuration 
%linear programs to design competitive algorithms for covering problems with non-convex objectives.
%Non-convexity until now is considered as a strong barrier in optimization. 
%We hope that our approach would contribute some elements toward the study of non-linear/non-convex problems.

In the paper, we have presented primal-dual 
frameworks to design algorithms with predictions for non-linear problems with packing constraints.
Through applications, we show the potential of our approach and provide useful ideas/guarantees 
in incorporating predictions into solutions to problems with high impact such as submodular optimization and ad-auctions. 
An interesting direction is to prove lower bounds for non-linear packing problems
in terms of smoothness and confidence parameters.

\bibliographystyle{plainnat}
\bibliography{configuration} 


\end{document}

