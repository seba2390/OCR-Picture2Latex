% CURRENTLY BEING EDITED IN A COPY OF THE WHITE PAPER
% CURRENTLY BEING EDITED IN A COPY OF THE WHITE PAPER
% CURRENTLY BEING EDITED IN A COPY OF THE WHITE PAPER
% CURRENTLY BEING EDITED IN A COPY OF THE WHITE PAPER
% CURRENTLY BEING EDITED IN A COPY OF THE WHITE PAPER
% CURRENTLY BEING EDITED IN A COPY OF THE WHITE PAPER

\newpage

\textit{TEMPORARY TEXT - MAIN TEXT BEING EDITED IN SEPARATE DOCUMENT}


\section{ QCD and Nuclei [JRA-first pass]}
\label{sec:ALERTmeasurements}

High-energy electron scattering gives access to a range of unique aspects of nuclear structure, providing important data relevant to nuclear structure, nucleon modification in the nuclear medium, and quark/hadron interactions in cold QCD matter. Precision measurements of nuclear form factors, quasielastic, and inelastic scattering provide important nuclear structure information that goes into a range of measurements including neutrino-nucleus scattering and nuclear astrophysics. Studies of the high-momentum part of the nucleon distributions yield important input for a range of lepton-nucleus and heavy ion collision measurements, as well as helping to constrain calculations of neutron stars. 

Studies of the partonic structure of nuclei provide insights into the impact of the dense nuclear medium on the structure of protons and neutrons and will allow, for the first time, imaging of the nuclear gluon distribution. Measurements at higher energy allow for studies of hadron formation over a wide range of kinematics, as well as detailed studies of quark and hadron interaction with cold, dense nuclear matter, including color transparency studies which attempt to isolate the interaction of pre-hadrons.

The following sections summarize the future program of measurements in nuclei, examining first measurements which probe nuclear structure at extremes of ... \textit{(finish at the end if there's enough space)}




\subsection{Nuclear structure at the extremes [JRA+IC]}


Low-energy probes of nuclear structure are limited in their ability to isolate the high-momentum components of the nuclear momentum distribution. Measurements made as part of the 6 GeV program demonstrated the ability to probe these distributions with minimal corrections from final-state interactions, which limited electron scattering measurements at lower energy. It also demonstrated the importance of understanding the isospin structure of short-range correlations (SRCs) which generate the high-momentum part of the momentum distribution.

At 12 GeV, our improved understanding of final-state interactions and the expanded kinematic range available will allow for better constraints on the nucleon momentum distributions at high momentum, as well as providing reference data needed to further constrain meson-exchange and final-state interaction corrections. In addition, targeted measurements will provide important input for nuclear structure relevant to neutrino oscillation experiments and nuclear astrophysics.


\subsubsection{Deuteron (and few-body nuclei) measurements at high momentum [JRA]}

In the PWIA, quasielastic A(e,e'p) measurements can be used to probe the spectral function of nuclei. Accessing the high-momentum part of the spectral function has proved extremely challenging, due to large final-state interactions that shift strength from low reconstructed momenta to large values~\cite{Sargsian:2002wc}. However, JLab measurements on the deuteron~\cite{Arrington:2011xs}
%\textit{(review - need original CLAS/Hall A refs)}
have been used to constrain FSI models over a wide range of kinematics, validating the models and demonstrating that in specific kinematic regions, the corrections become small. A new set of measurements was proposed to push deuteron measurements to extremely large nucleon momenta, exceeding 1~GeV/c~\cite{proposal}. The low-momentum piece of this experiment was run and results are available~\cite{HallC:2020kdm} that will be used to further optimize the remaining measurements. Other experiments~\cite{} look to further to extend this approach to study the spectral function of light nuclei~\cite{}, as well as providing further constraints on FSI.
%\textit{check proposal/paper, add a few more details.}

\subsubsection{Short range NN correlations [JRA]}

Nucleons with momenta well above the Fermi momentum are associated with short range correlations (SRCs), which are generated by the hard, short-range components of the NN interaction~\cite{Frankfurt:1981mk, Frankfurt:1988nt, Sargsian:2002wc, Arrington:2011xs}. Because they are generated by 2-body interactions, one can study them by taking ratios of inclusive scattering from heavy nuclei to the deuteron at modest $Q^2$ and $x>1.4$, where scattering from low-momentum nucleons is kinematically forbidden, providing sensitivity to the relative contribution of SRCs as a function of A. During 6 GeV running, experiments confirmed the initial observation of SRCs~\cite{Frankfurt:1993sp} and mapped out the A dependence of SRCs in light nuclei. These data demonstrated that the contribution is sensitive to details of the nuclear structure~\cite{Seely:2009gt} rather than the previously assumed average density~\cite{Gomez:1993ri}. In addition, they showed a clear correlation between the contribution of SRCs and the size of the EMC effect, discussed further in Sec.~\ref{sec:4:emc}. Measurements of two-nucleon knockout, where both nucleons from the SRC are observed in the final state, showed dominance of np-SRC, which was later confirmed in additional nuclei for A(e,e'p) and through inclusive measurements on $48$Ca/$40$Ca and, $^3$H/$^3$He. Finally, measurements at $x>2$ tried to establish the presence of three-nucleon SRCs (3N-SRCs), but low-$Q^2$ measurements did not observed 3N-SRC dominance, and higher-$Q^2$ data was consistent with 3N-SRC dominance but had extremely limited statistics. 
% A dependence as mystery for further study??
% Tagged stuff comes in EMC section

\textit{Further 12 GeV program on SRCs: (1) Additional measurements on light nuclei to improve our understanding of what aspects of nuclear structure drive the EMC effect, (2) Measurements on medium-to-heavy nuclei over a range of neutron excess (more for EMC than SRC, really), and (3) 3N-SRCs.  Also 3H/3He (previous and future) give more details on scaling violation, 3N-SRC structure, etc....  But probably just a simple statement about understanding momentum-isospin distribution in nuclei.}




\subsection{Impact on neutron stars, nuclear astrophysics, neutrino scattering [IC/JRA]}
\label{sec:PREX}

\textit{JRA: Importance of taking what we've learned (and are learning) about SRC contributions and isospin structure in a variety of other measurements: neutrino scattering, neutron star structure, etc.... Then include dedicated measurements aimed at providing input to other physics - PREX/CREX, neutrino scattering, etc...}


\subsection{Nucleon modification/Nucleons under extreme conditions [IC/JRA]}\label{sec:4:emc}
\textit{JRA: Either expand on nucleus as "nucleons under extreme conditions" as in the brief overview, or put this in the overview and do a brief historical/context discussion: idea of modified nucleon structure has been studies for decades and data had to deal with lack of interpretability due to experimental corrections or lack of detailed theory. JLab has made important new measurements and led to range of calculations self-consistently describing multiple observables, which can/will be probed in the future.}

\subsubsection{Modified form factors [JRA/IC?]}
\textit{JRA: My take would be to make this brief: new measurements at 6 GeV, difficult to interpret due to things like FSI. Future work focused on 'tagged' measurements of highly-virtual nucleons, providing some benefits for form factors and allowing a focus on tagged DIS measurements.}

\subsubsection{EMC effect: modified parton distributions [IC/JRA/WB]}
\textit{JRA: 1) Brief summary of what we've learned from new inclusive measurements and future plans (light nuclei and varying N/Z) [JRA]. 2) New 'simple' observables (spin and flavor dependence, antiquarks) - calculations that provide unified description and planned measurements [IC]. 3) Tagged measurements (CLAS and Hall C (LAD/BAND), ALERT,etc...) [IC]. [Super-fast quarks here, or as 'extension' of SRC measurements in previous section?? - here may be better, with hidden color, $\Delta\Delta$ production...}

\subsubsection{Nuclear TMDs, GPDs [IC?]}
%\textbf{Last WP meeting - conclusion was to put the physics here, and then ask them to add appropriate sentences in the Femtography section about possibilities (e.g. fewer GPDs for spin-0 nuclei) and point here for physics we can get out of it}.


\subsubsection{Connections to other fields, future EIC??? [IC/JRA/WB}
\textit{JRA: pdfs (including polarized, flavor-dependent) as input to high-energy collider measurements on nuclei; also to polarized 3He extractions of neutron structure. Maybe this doesn't need a separate section, if they are identified with the main physics topics? Wait and see.}

\begin{itemize}
    \item GPDs for spin-0 nuclei
    \item Studies on the deuteron (Look at the EIC work and figure out what JLab12+ can do).
\end{itemize}


\subsection{Quark/hadron propagation in the nuclear medium}
\textit{Short section context/overview}

\subsubsection{Quark/hadron propagation [WB]}
The confinement principle of QCD dictates that color charge cannot be separated from the color-neutral hadrons that contain it. This statement, however, only pertains to equilibrium conditions. Color charge can be briefly liberated from a hadron through a hard scattering. In the simplest case, a valence quark carrying color charge can undergo a high-energy scattering process that propels the quark over long distances, closely accompanied by a spray of quarks and gluons of lower energies that ultimately evolve into new hadrons. This process, call hadronization or fragmentation, can take place on distance scales of 1-100 fm or more, and its characteristics at lower energies can be studied by implanting it into an atomic nucleus. The hadronization constitutents interact with the nuclear medium, modifying hadron production in ways that reveal characteristics of this fundamental process. Questions that can be answered by such experimental studies include: how long does the struck quark propagate before becoming bound into a forming hadron? with what mechanisms do propagating quarks interact with the nuclear medium, and at which spatial scales do they interact? how much energy do they lose in the medium? how much medium-induced transverse momentum do they acquire? is a semi-classical description of the process adequate? In semi-inclusive deep inelastic scattering, the picture of the initial state is particularly clear at the values of $x_{Bj}$ accessible at Jefferson Lab. A valence quark absorbs all the energy and momentum from the interaction, which can be measured directly by the scattered electron. Neglecting small contributions from intrinsic transverse momentum and Fermi momentum, the energy and momentum of the struck quark are known; thus, the initial state of the interaction between the quark and the medium is well determined. This secondary "beam" of quarks is generated throughout the nucleus with an initial position probability determined by the well-known nuclear density distribution, allowing precise modeling of the process.

In this picture, the hadron containing the struck quark will interact with the medium with a total cross section dominated by inelastic processes at the energies relevant to JLab into the future. In the kinematics where the hadron forms earliest, which is at high relative energy $z_h=E_h/\nu$, this cross section can be determined with moderate accuracy using  sufficiently precise data. The experimental signature of this region is a maximal suppression of hadron production in large nuclei relative to the proton or deuteron.

Hadron propagation in the medium can also be studied via other reaction types, often motivated by the search for color transparency. An increased nuclear transparency to hadrons in specific kinematics is a definite prediction of QCD that is linked to important properties such as factorization of initial and final states in high-energy scattering. Experiments in the 6 GeV era saw the onset of color transparency for pions and rho mesons. Future studies of color transparency (CT) at Jefferson Lab for mesons and baryons  will strongly benefit from the high luminosity and large reach in four-momentum transfer foreseen. The A(e,e'h)X channel in diffractive and non-diffractive kinematics probes CT in meson production, while it can be used to look for proton transparency in quasi-elastic kinematics. The principal signature of CT is a reduced nuclear transparency observed as four-momentum transfer $Q^2$ increases, for fixed coherence length. While the transparency seen in the 6 GeV era was small but significant, a robust signature for light mesons is expected to be very clear as beam energy and luminosity rise.

\subsubsection{Hadron formation [WB]}
The mechanisms involved in the hadronization process that dynamically enforce color confinement are poorly known. More insight into these mechanisms can be obtained by systematic study of production of different baryon and meson types in large and small nuclear systems. Questions that can be answered by such studies include: what are the differences between formation of $q\bar{q}$ (baryon number B=0) systems and $qqq$ (B=1) systems? how do the characteristics of formation change as the number of strange quarks increases, both in mesons and baryons? what can we learn about the time required for complete formation of hadrons? is there evidence of diquark structure seen for baryon formation, and if so, how does it influence our understanding of proton and neutron structure? 


\subsection{Future opportunities (at higher energy?) [JRA+IC+WB]} 

%Higher energy - multiple opportunities: hadronization (D mesons, if possible) (WB), super-fast quarks (JA), TMD/GPD (IC)......?

%Positrons - very useful for Coulomb distortion, but probably not at a highlight level. Maybe just a very brief mention of what can be done, assuming positron beams presented elsewhere.
 

%\textcolor{green}{
%\textbf{Notes from May WP meeting. 1) Does 'complex nuclear structure' (mentioned in intro) go in chapter 4?- C/PREX, Argon [yes]. 2) Aiming to stay close to 2012 length, so similar pages per chapter (~8 pages?) but some room for modest growth. }}


\newpage

