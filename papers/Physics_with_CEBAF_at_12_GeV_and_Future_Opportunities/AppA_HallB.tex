\subsection{Hall B		}
\label{sec:app-hallb}

Hall-B is the smallest of the current four experimental halls at Jefferson Lab. The Hall-B physics program covers all area of the lab science: nucleon structure (SIDIS, GPDs, TMDs), hadron spectroscopy (meson and baryon), nuclear structure (SRCs and quark hadronization in nuclear medium), and BSM physics (Dark Sector). The approved experimental program started in 2018 and will continue for the next $\sim$10 years.

The Hall is equipped with a permanent 4$\pi$ acceptance detector (CLAS12) but different experimental setups for dedicated measurements (PRAD, HPS) are also in place\footnote{CLAS12, PRad and HPS can not run in parallel.}. This paragraph describes the experimental equipment available in the Hall and plans for a future upgrade to pursue the physics program described in the first section of this document.\\ 

The CEBAF Large Acceptance Spectrometer for operations at 12 GeV beam energy (CLAS12) is used to study electro-induced nuclear and hadronic reactions. This spectrometer provides efficient detection of charged and neutral particles over a large fraction of the full solid angle. CLAS12
is based on a dual-magnet system with a superconducting torus magnet that provides a largely azimuthal
field distribution that covers the forward polar angle range up to 35$^o$, and a solenoid magnet and detector covering the polar angles from 35$^\circ$ to 125$^\circ$with full azimuthal coverage. Trajectory reconstruction in the
forward direction using drift chambers and in the central direction using a vertex tracker results in momentum
resolutions of $<1\%$ and $<3\%$, respectively. Cherenkov counters, time-of-flight scintillators, and
electromagnetic calorimeters provide identification of the scattered electron and produced hadrons. Fast triggering and high data-acquisition
rates allow operation at a luminosity of 10$^{35}$ cm$^{-2}$s$^{-1}$. These capabilities are being used in a broad program
to study the structure and interactions of nucleons, nuclei, and mesons, using polarized and unpolarized
electron beams and targets for beam energies up to 11 GeV.

The PRad (Proton Radius) experiment measured the charge radius of the proton with the highest precision ever reached in electron-scattering. The experiment required a dedicated setup consisting of an electromagnetic calorimeter and a GEM-based tracker, as well as a partial modification of the beam line. An upgraded version of the experiment (PRad-II) has been recently approved. The PRad Collaboration, in coordination with the Hall-B staff, is currently preparing the new set up that is expected to reduce by a factor of four the error bars on the proton radius.

The HPS (Heavy Photon Search) experiment uses the electron beam on a nuclear target to produce the postulated new boson A' (heavy photon) expected to mediate the interaction between the Standard Model particles and the Dark Sector. The experiment requires different beam energies (1.1. GeV to 6.0 GeV) to cover unexplored areas in the parameter space (A' coupling vs. mass). The HPS detector, installed in the Hall-B alcove downstream of the CLAS12 detector, is made by an electromagnetic calorimeter, a silicon-vertex tracker and a hodoscope.
\\

