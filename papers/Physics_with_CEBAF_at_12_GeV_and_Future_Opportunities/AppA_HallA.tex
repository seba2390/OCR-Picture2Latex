\subsection{Hall A}
\label{sec:app-halla}
Hall A has transitioned from being a focusing spectrometer Hall to a long-term experimental installation Hall, most notably facilitating the Super BigBite Spectrometer (SBS), the Measurement of a Lepton-Lepton Electroweak Reaction (MOLLER), and the Solenoidal Large Intensity Device (SoLID) equipment and programs. One existing High Resolution Spectrometer (HRS), constructed in the 1990's to access
a momentum range of 0.3 to 4 GeV/c with high momentum resolution, is also still available. To facilitate in particular parity-violating experiments, the beamline -- including a precision Unser current monitor, several beam position monitors, and Compton and Møller polarimeters -- was and continues to be upgraded to operate with unprecedented precision up to 11 GeV.

The SBS program involves two open, single bend, resistive spectrometers, and two large standalone calorimeters (ECal for electromagnetic and HCal for hadron calorimetry). This apparatus will enable a complete study of the unpolarized nucleon form factors G$_E$ and G$_M$ for both the proton and neutron at high Q$^2$, as well as approved semi-inclusive and tagged DIS studies. 

MOLLER will measure the PV asymmetry in electron-electron scattering by rapidly flipping the longitudinal polarization of electrons that have been accelerated to 11 GeV and observing the resulting fractional difference in the probability of these electrons scattering off atomic electrons in a high power liquid hydrogen target. This asymmetry is proportional to the weak charge of the electron, which in turn is a function of the electroweak mixing angle, a fundamental parameter of the electroweak theory. 

The planned SoLID apparatus will enable a broad program of experiments at the high luminosity frontier also requiring large acceptance, made possible by developments in both detector technology and simulation accuracy that were not available in the early stages of the 12 GeV program planning. The spectrometer is, moreover, designed with a unique reconfiguration capability to optimize for either PVDIS or SIDIS/threshold production of the J/$\Psi$ meson.