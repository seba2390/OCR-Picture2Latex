\documentclass[10pt,aps,prl,amsmath,amssymb,superscriptaddress,nofootinbib,showpacs,preprintnumbers,amsfonts, notitlepage]{revtex4-2}
\usepackage{xcolor}
\newcommand{\us}[1]{{\normalfont\color{black} #1}}

\usepackage{mfirstuc} % to uppercase the name
% \usepackage{todonotes} % to uppercase the name
\newcommand{\addReviewer}[2]{
  \expandafter\newcommand\csname #1\endcsname[1]{{\textbf{ \color{#2} \capitalisewords{#1}:\,##1}}}
  \expandafter\newcommand\csname #1cor\endcsname[2]{{\color{#2} \capitalisewords{#1}:\,\st{##1}{\textbf{##2}}}}
  \expandafter\newcommand\csname #1color\endcsname{#2}
  \expandafter\newcommand\csname #1todo\endcsname[1]{{\todo[inline,color=white!70!#2, caption={}]{\textbf{\capitalisewords{#1}}: ##1}}}
}

\usepackage{soul,color}
\definecolor{chromeyellow}{rgb}{1.0, 0.65, 0.0}


\begin{document}



We thank both referees for their valuable comments. The manuscript has been    modified accordingly. 

\vskip 10pt 

{\bf Response to Referee A} 

\vskip 10pt 
{\bf Referee: } 

This paper describes in detail the ongoing, extensive physics program with JLab’s Continuous Electron Beam Accelerator Facility (CEBAF) at beam energies of 12GeV. The versatility and scope of this program is impressive spanning nuclear, hadronic and electroweak measurements.
I believe that the paper serves an important purpose - to articulate the impact and potential of CEBAF-enabled physics for the broad community of theory and experiment and also perhaps for funding agencies and policy makers. The paper also describes quite well the case for upgrades to CEBAF. Much of the technical detail is appropriately in appendices and adds to conÿdence in the physics planning discussed in the main document.
The paper should be published. While I do not find fault with the physics that is discussed or planned, I do believe that the paper would benefit  from some editing to improve readability and coherence and from an improved introduction and motivation to some topics as well as a better summary section. I will highlight below some issues that I think will help.
There are quite a few typos and grammatical errors in the text that can hopefully be easily ad-dressed.
Some acronyms are defined  multiple times and some not at all. Some words and phrases are given slightly different  acronyms in 
 different sections.
I found that some of the introductory paragraphs in sections were muddled and confusing to read. I would expect, as either an expert or non-expert, to find a clear statement of why a topic is interesting, the current state-of-the-art and what the ongoing or proposed measurements can do to, for example,

\begin{itemize} 
\item uniquely contribute to our knowledge of the topic

\item improve current precision to a level needed to make a clear statement, rule out particular model etc.

\end{itemize} 

This was not always clear and varied between topics.
It would be helpful to review and simplify some sections. As an example, section 4.1 on gluonic  excitations in Hadron Spectroscopy. There is nothing incorrect, that I noticed, in the text but a reader may be left confused by the number of theoretical approaches that are mentioned, somewhat indiscriminately - quark model(s), large Nc including string (gluon) junctions, lattice QCD etc. In addition, the discussion mixes favor  - work on charmonium hybrids from a lattice calculation (which is described as recent but was published 10 years ago) is cited but light quark notation of $\pi_1$  and  $\eta_1$ is used too, also mentioning static quarks. I presume that light hybrids are the target states so why are lattice QCD results with charm quarks cited and not those with light quarks? A clearer motivation and background would be helpful.
A further example (and again this is not to single this out but to point to some pervasive structural problems in the text) is e.g. on lines 769-777 from which the point is quite hard to see. The last paragraph states 2 proposals were submitted to the JLab PAC in 2020 - since this is already 2 years ago, presumably the outcome is known? Can the reader be told?
Can the summary include mention of e.g. some of the “many important experimental results” that are “being reported” as a way to draw the previous sections together? I also noticed that the appendices are not refered to in the main text (at least not as far as I noticed). Given the technical detail and planning in the appendices is relevant these probably should be connected to the core document at least in passing?
In summary, I believe the paper should be published. I encourage the authors to review the text with a goal to improve readability and the overall cohesiveness of the different  topics that are discussed so as to sharpen the overall impact. The examples given above are not comprehensive but hopefully useful.

\vskip 5pt 
{\bf Response: } 


%\vskip 10pt
%{\bf Referee:}

%\vskip 5pt 

%{\bf Response:}

\vskip 20pt 

{\bf Response to Referee B} 
\vskip 10pt 
{\bf Referee: } 


Jefferson Laboratory has an excellent track record with many important results provided by the experiments from the CEBAF facility, The manuscript gives an adequate, and impressive, account of the contributions to very different fields of nuclear physics covering nuclear structure, the onset and detailed 0study of quark degrees of freedom both in nucleon structure and hadron spectroscopy as well as their role in nuclei, and dark matter searches as an example to contributions to (beyond the) standard model physics. The commissioning of the 12 GeV upgrade and the adaption of the experimental equipment recently expanded the physics reach of the facility once more. First results obtained employing the new equipment have been published and demonstrate the new physics reach. Therefore, a summary of what has been achieved in the past, what the current status is and which promises it holds, as well as a layout of future opportunities, is timely and well done in the manuscript. I therefore highly recommend it for publication.

I would like to add a few minor comments for the authors to consider:
Overall, the manuscript is very well written. I find it a bit  difficult to read in some instances where the style from chapter to chapter is very different. In some of the chapters, prepositions and articles are missing or sometimes chosen improperly. This is of course a minor issue but might be addressed in a revision.
Also, in some chapters some more detail on work from other laboratories might still be added. While it is understood that the manuscript mainly deals with the JLab successes over the years, some chapters contain a lot of references to the others players in the field, while others seem to only broadly take reference to the history of a field and then detail on the CEBAF results, e.g. in the discussion of the Rosenbluth vs. two-photon results on form factors, the only brief discussion of the proton radius, and the baryon spectroscopy. While the focus of the manuscript clearly is on the new and exciting opportunities of the extended energy reach, these parts of the experimental program have been working with a number of other contributing facilities.

A few very minor comments remain:
p. 12, line 473: “to both to”, delete one “to”
line 488: “With planned at” insert “experiments” or “measurements”
line 490: “studying [the] TPE … using possible up to ..” delete “possible”?
p. 13, line 513: HERMES, R and M exchanged
(This is one of the chapters with a number of issues in language)

p. 20, line 743: “lead—tungsten calorimeter” - lead tungstate, lead glass?
line 746: CLA12 -> CLAS12
line 747: spacelike and time-like: both with or without hyphens?





\vskip 5pt 
{\bf Response: }




%It provides 
%a framework 
%that once fed with the many possible line shapes which %encode
%the possible physical explanations can answer the question %on the likelihood of each explanation and on how each data %point influences that decision, overpowering the standard %analysis methods widely used in the hadron spectroscopy %literature, such as $\chi^2$ fitting. 


\end{document}\documentclass[10pt,aps,prl,amsmath,amssymb,superscriptaddress,nofootinbib,showpacs,preprintnumbers,amsfonts, notitlepage]{revtex4-2}

\vskip 10pt 
{\bf Comments to the Editor} 
\end{document} 

