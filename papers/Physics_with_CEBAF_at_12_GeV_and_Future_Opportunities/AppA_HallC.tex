\subsection{Hall C}
\label{sec:app-hallc}


In the upcoming era, Hall C at Jefferson Lab will
be the only high luminosity, flexible, large-scale installation
electron scattering facility in the world. This will enable critical benchmark measurements for nucleon and nuclear experiments at the EIC and elsewhere. The existing focusing magnet spectrometers HMS and SHMS (High Momentum and Super High Momentum Spectrometers, respectively) with well-shielded detector huts, high power cryotargets, and a 1 MWatt beam dump, allow routine operation at luminosities of $10^{38}$ to $10^{39}$.
Other equipment, such as the Neutral Particle Spectrometer (NPS), Compact Photon Source (CPS), high power cryogenic targets, and multiple
polarized targets are available to use in various combinations.  The hall itself provides floor space for a variety of
experimental configurations utilizing existing or new detectors or
equipment.

The base detectors for Hall C are the HMS and SHMS. These several msr acceptance devices can reach momenta of 11 GeV/c and 6+ GeV/c respectively  with momentum bites exceeding 10\%. 
Momentum reconstruction resolution and accuracy at the 0.1\% level are routinely achieved.
An electron beam energy measurement facility is provided by a string of upstream dipoles. These spectrometers can be placed at scattering angles as small as 5.5
and 10.5 degrees with sub-mrad pointing accuracy. They may be used singly or in
coincidence with each other or with other detectors, and have well-shielded
detector stacks.  This provides a unique ability to measure small
cross sections which demand demand high luminosity and facilitate the
careful study of systematic uncertainties.

When the magnetic spectrometers are rotated to large angles, the hall,
with its large beam height provides the flexibility to install a
large variety of additional detectors.  Historically, this flexibility
has allowed for large acceptance devices for parity violating
electron scattering, spectrometers optimized for hypernuclear physics, 
and other equipment. In the future, it can accommodate devices such as
the Super BigBite and BigBite (SBS and BB) open detector
spectrometers, the large hadronic calorimeter HCal, or new detectors such as NPS. The new NPS is a $62\times74~\textrm{cm}^2$
$\textrm{PbWO}_4$ calorimeter for photon and $\pi^0$ detection.

A variety of targets and beamline instruments are available in Hall C.
The standard target assembly allows for liquid
hydrogen/deuterium, dense ${}^3\textrm{He}$/${}^4\textrm{He}$, and other nuclear targets - all of
which may be operated at high luminosities.  Other targets have
included dynamically polarized hydrogen and deuterium targets
($\textrm{NH}_3$ and $\textrm{ND}_3$ as well as polarized
${}^3\textrm{He}$. Beam polarization can be measured to the
sub-percent level with both Moller and Compton polarimetry. As well as producing mixed photon/electron beam with a Bremsstrahlung
radiator, pure photon beams can be produced with the new CPS.  With electrons removed from the photon beam, the photon
flux on power-limited polarized targets can be maximized.


