\section{Overview}

The ability to predict and understand the properties of nucleons and atomic nuclei from the first principles of Quantum Chromodynamics (QCD) with quarks and gluons as the underlying degrees of freedom is one of the goals of modern nuclear physics. Electron scattering at multi-GeV -- with resolutions ten times or more smaller than the size of the proton -- is a powerful microscope for probing the partonic structure and QCD dynamics of the nucleons and nuclei. With recent advances in accelerator and detector technologies, high-performing computing, new algorithms from machine learning and artificial intelligence, we are entering a new era of QCD intensity frontier in electron scattering, allowing for unprecedented precision and measurements of new observables and processes which are unimaginable previously.

The Nuclear Physics community has resourcefully exploited these advanced accelerator facilities for studies of the fundamental interactions. This research has impacted the entirety of Nuclear Physics, as well as High Energy Physics and Astrophysics. In fact, the ``Fundamental Symmetries and Neutrinos'' cohort within Nuclear Physics continues to make strides in both experiment and theory.
  
The Continuous Electron Beam Accelerator Facility (CEBAF) at Jefferson Lab has been delivering the world's highest intensity and highest precision multi-GeV electron beams for more than 25 years. The scientific justification for a project to double the CEBAF beam energy to 12~GeV was developed~\cite{Dudek:2012vr}, and construction began in 2008. This project was completed in Fall~2017, beginning a new era at the Laboratory. A total of more than 2094 publications of Nuclear Physics results, including both experiment and theory, have appeared in refereed journals.  A sizable fraction of these papers appeared in prestigious journals of high impact beyond the physics community such as Nature and Science, demonstrating the broader scientific impact of the results coming out of Jefferson Lab. 

The 12-GeV era includes a suite of new experimental equipment, including the Super Bigbite Spectrometer (SBS) in Hall~A, the CEBAF Large Acceptance Spectrometer (CLAS)-12 detector complex in Hall~B, the Super High Momentum Spectrometer (SHMS) in Hall~C, and the GlueX spectrometer and polarized photon beam in the newly completed Hall~D. Data are currently being acquired, and new results are being published, from all of these laboratory assets.

Innovations that exploited existing equipment including the latest additions have also yielded ground-breaking Nuclear Physics results. A few among these include imaging the nucleon in three dimensions via the extraction of Generalized Parton Distributions (GPDs) using Deeply Virtual Compton Scattering (DVCS) and the Transverse Momentum Dependent (TMD) and Parton Distribution Functions (PDF) from Inclusive and Semi-Inclusive Deep Inelastic Scattering (SIDIS); isospin decomposition of nucleon structure functions using a tritium gas target; observation of $J/\psi$ photoproduction from the proton near the threshold, limits on the photo-coupling of the $uudc\bar{c}$ pentaquark, %and the extraction of the matter radius of the proton,
%{\bf [Revise this after Hall C results are out]};
the resolution of the ``Proton Radius Puzzle'' through the PRad experiment with a major impact on atomic physics; and new information on neutron matter from parity violating electron scattering on $^{208}$Pb (PREX-II) and $^{48}$Ca (CREX) with implications for the structure of neutron stars.

Future additions to the experimental equipment contingent include the Solenoidal Large Intensity Device (SoLID) for measurements of SIDIS for nucleon tomography in three-dimensional momentum space, Parity-Violating Deep Inelastic Scattering (PVDIS) pushing the phase space in the search of new physics and of hadronic physics, and precision measurement of $J/\psi$ production from the proton near threshold to access the quantum anomalous energy contribution to the proton energy (mass of the proton in its rest frame ); the Measurement Of Lepton-Lepton Elastic Reaction (MOLLER) experiment, a test of the standard model and search for new physics ; new large angle tagging detectors (TDIS in Hall~A and ALERT in Hall~B); the neutral particle spectrometer (NPS) and the compact photon source (CPS) for a slate of approved experiments in Hall C; and an intense $K_L$ beamline that would serve new experiments in the GlueX spectrometer. 

Lattice QCD (LQCD), which offers a first-principle numerical
approach to QCD, predicted masses for new exotic mesons and baryons, and has matured to the point that it can reliably calculate  branching ratios of exotic decays.  Recent predictions for the full decay width of exotic $\pi_1$, as well as the first three-body Dalitz plot, from LQCD calculations
provide the much needed support and new opportunities for experimental searches of exotics at Jefferson Lab.  
%Although it is difficult, if not impossible, for LQCD to calculate the GPDs and TMDs directly to predict the internal landscape of nucleons due to its Euclidean space-time formulation, new developments in theory have made it possible to extract GPDs, TMDs and other quark/gluon correlations from LQCD calculations, complementary to the on-going phenomenological extractions of 3D hadron structure based on experimental data from Jefferson Lab and other facilities.
Due to its Euclidean space-time formulation it has been more 
difficult for LQCD to calculate the GPDs and TMDs directly and to predict 
the internal landscape of nucleons. However, new developments in theory 
have now made it possible to extract GPDs, TMDs and other quark/gluon 
correlations from LQCD calculations, testing and providing complementarity to the on-going 
phenomenological extractions of 3D hadron structure based on 
experimental data from Jefferson Lab and other facilities.

Furthermore, a new round of upgrades to CEBAF are presently under technical development. One of these is a potential energy upgrade to 24~GeV using novel magnet designs in the existing recirculation arcs. Another is a potential for intense polarized positron beams, which would allow for new measurements in nucleon tomography, and provide precision extraction of contributions from higher order electromagnetic processes. 

Clearly, CEBAF combines an illustrious history with an exciting future outlook. Our purpose in this article is to document this position, and provide a basis for future discussions as to how the Nuclear Physics community can best make use of this unique high intensity and high precision facility.

\subsection{Specific Scientific Accomplishments}

CEBAF and Jefferson Lab have produced results which impact all of Nuclear Physics, as well as High Energy Physics, Atomic Physics, and Astrophysics. These results are delineated and detailed in the remaining sections of this review. The following, however, is a short description of some specific accomplishments of the experimental program which should be of particular interest to the broader community.

The PRad Experiment (Section~\ref{sec:PRad}) has effectively settled the so-called ``Proton Charge Radius Puzzle''~\cite{CARLSON201559,gao2021proton}. Up until 2010, we believed that the proton charge radius was known with an accuracy close to 1\%, but an experiment based on the Lamb Shift in muonic hydrogen gave a number 4\% smaller than previous results  with a 0.1\% uncertainty~\cite{Pohl10,Anti13}. The PRad collaboration made a very precise measurement of the proton form factors explicitly at very low momentum transfer squared ($Q^2$), facilitating a highly accurate extrapolation to $Q^2=0$ and extraction of the proton charge radius. This experiment would not have been possible without the precision beams of CEBAF %available at Jefferson Lab 
and an innovative and new method of performing electron-proton elastic scattering measurements.

%Heavy quarkonia photoproduction close to threshold can be used to determine the quantum anomalous energy terms in QCD, which can in turn be used to determine the mass radius of the proton~\cite{Kharzeev:2021qkd}. Based on a measurement of $J/\psi$ photoproduction from GlueX the mass radius was found to be close to 0.55~fm, considerably smaller than the charge radius of 0.84~fm. More precision results are expected soon from GlueX, CLAS-12, and Hall~C, see Section~\ref{sec:PMass}. Future measurements with SoLID are expected to increase the precision drastically.

Heavy quark photoproduction can potentially be related to the quantum 
anomalous energy terms in QCD, which may in turn be used to determine 
the mass radius of the proton~\cite{Kharzeev:2021qkd}.  New precise measurements of $J/\psi$ 
photoproduction from GlueX, CLAS-12, and in Hall C (Section~\ref{sec:PMass}) are expected to 
further elucidate these issues. 

Precise measurements of the nucleon (proton and neutron) valence quark distribution functions have long been known to be important for discriminating useful models including those inspired by QCD. Determining these distributions from measurements of deep inelastic scattering from protons and deuterons, however, is problematic~\cite{Melnitchouk:1995fc, Accardi:2016qay} . The MARATHON experiment (Section~\ref{sec:duRatio}) has carried out a novel measurement by comparing scattering from $^3$He and tritium targets, which also has implications for nucleon correlations and the EMC effect in $A=3$ nuclei.

Accurate determinations of the neutron radii of spherical doubly-magic nuclei have important implications both for the neutron star equation of state~\cite{Horowitz:2001ya} and for microscopic calculations of nuclear structure~\cite{Hagen:2015yea}. Parity violating electron scattering (PVES) in the elastic regime provides a model-independent way of determining these neutron radii, because the value of the weak mixing angle $\sin^2\theta_W$ fortuitously leads to a neutral-current coupling of the neutron that is much larger than that for the proton. The high-quality polarized electron beams from CEBAF were used by the PREX-II and CREX experiments to determine the neutron radii of $^{208}$Pb and $^{48}$Ca, respectively. See Section~\ref{sec:PREX} for details.

One important motivation for the 12~GeV upgrade of CEBAF was to search for mesons with exotic quantum numbers, due to gluonic degrees of freedom, and is one of the primary goals of the GlueX experiment. One such exotic meson candidate, the $\pi_1(1400)/\pi_1(1600)$, has been reported in the $\eta\pi$ $P$-wave by several experiments~\cite{Zyla:2020zbs}. Elastic photoproduction of the $\eta\pi$ system from the proton has been studied by GlueX, including a detailed partial wave decomposition. These results are discussed in Section~\ref{sec:expmesons}.

The LHCb experiment at CERN has confirmed their observations of a heavy $c\bar{c}$ meson decaying to $J/\psi p$, a candidate for a pentaquark. Observed by them in the weak decay of $\Lambda_b^0$, the production cross section in $\gamma p\to J/\psi p$ would discriminate between a true pentaquark and possible molecular states or some dynamic effect~\cite{PhysRevD.100.054033}. The non-observation of this state in the $J/\psi$-007 experiment in Hall~C (Section~\ref{sec:CharmPentaquark}) suggests a non-pentaquark interpretation, although there is still much room for a smaller pentaquark-$J/\psi$ coupling.
%\eugene{This statement seems to be too strong. Indeed, some models can be ruled out, but in general there is a large room for the pentaquark-J/psi coupling. The article quoted points out that polarized measurements would be more sensitive. BTW at this time only the GlueX results has been published}.

The possible existence of particles with the quantum numbers of the photon but with only weak coupling to charge leptons, so called ``dark photons'', is attractive for a number of reasons~\cite{fabbrichesi2021physics}. The precise CEBAF beams have made possible searches for these objects, with results already in hand from APEX and HPS, see Section~\ref{sec:APEXHPS}. More results are forthcoming.

%\underline{Hadron properties and structure} has been the focus of much of the CEBAF program, including studies from low values of Bjorken $x$ up to $x=1$, that is, elastic scattering.
%\textcolor{red}{The PRad collaboration at Jefferson Lab carried out an innovative electron-proton scattering experiment and reported a proton charge radius value $r_{p} = 0.831 \pm 0.007 ({\rm stat.}) \pm 0.012 ({\rm syst.})$ in Nature in 2019 (W. Xiong et al., Nature 575, 147-150 (2019)). This result is consistent with a smaller proton radius value from muonic hydrogen measurements (R. Pohl, et al., Nature 466, 213 (2010); A. Antognini, et al., Science 339, 417 (2013)), and also supports the shift of the Rydberg constant that the CODATA released in 2019. The PrimEx collaboration at Jefferson Lab reported the world's most precise determination of the neutral pion radiative decay width of $7.802 \pm 0.056 {\rm (stat.)} \pm 0.105 {\rm (syst.)} $ eV in Science in 2020 (I. Larin, Y. Zhang et al.,
%Science 6490, 506-509 (2020)). The result is found to be in excellent agreement with the chiral anomaly prediction, and is two standard deviations away from calculations including higher order chiral corrections.}
%{\bf Nucleon form factor results are a traditional mainstay here. Perhaps also $A_1^n$. 
%The Hall C result on $J/\psi$ with implications for the matter radius of the proton, is also a good example. We may want to mention the latest spin structure results from Hall A/B published in two Nature Physics papers recently, though both from 6 GeV data. But they connect nicely with 11-GeV results.
%}

%The behavior of \underline{parton distributions in the nuclear medium} has been a particularly fruitful area of research.
%{\bf Possible examples: EMC effect results, \ldots}

%One vexing problem in nuclear physics concerns the nature of the \underline{three body system}, in particular the nuclei $^3$H and $^3$He, and their hypernuclear counterparts.
%{\bf Possible examples: New results from MARATHON, other results from tritium targets, \ldots}

%\underline{Short range correlations} arising from the short distance nature of the nuclear force between nucleons are a correlation momentum space of $pp$, $nn$, and $np$ pairs. There are indications that this can lead to an explanation of the EMC effect.
%{\bf Possible examples: SRC results from CLAS, new results?, \ldots}

%\underline{Hadron spectroscopy} has become a pillar of the experimental and theoretical communities at Jefferson Lab.
%{\bf Possible examples: Emphasize GlueX results on exotics. Also baryon spectroscopy from CLAS, $J/\psi$ and $\psi(2S)$, \ldots}

%Experiments on the \underline{structure of complex nuclei} have not only given us new windows into systems typically studied in other Nuclear Physics facilities, but they also inform experiments with high energy neutrino beams and interpretations of astrophysical phenomena including neutron stars.
%{\bf Possible examples: $^{40}$Ar, PREX-II, CREX, and hypernuclear experiments.}

%Searching for \underline{physics beyond the Standard Model} is one of the areas of research which impacts fields outside traditional nuclear physics.
%{\bf Examples: QWeak, HPS, APEX}

%It is important to note that the \underline{Theoretical and Computational Nuclear Physics} effort at Jefferson Laboratory not only works strongly in tandem with the experimental program, but also form a strong, independent component of the intellectual effort.
%{\bf Possible examples: Probably LQCD, maybe highlight the results on meson photoproduction?}

\subsection{Currently Planned Experimental Program}

Following the 49$^{th}$ meeting of the Jefferson Lab Program Advisory Committee (PAC) in Summer 2021, there are a total of 85 approved experiments\footnote{A list of approved experiments is available at {\sf https://www.jlab.org/physics/experiments}} in the 12 GeV program, of which 30 have received the highest scientific rating of A.  There are 56 approved experiments still waiting to run, representing at least a decade of running in the future.
Furthermore, PAC meetings are expected to continue each summer, preceded by a call for new proposals for beam time. Clearly, CEBAF is a facility in high demand.

Jefferson Lab continues to invest in facilities that make optimum use of CEBAF's capabilities, in particular those that will produce high-impact science across different areas within Nuclear Physics and beyond. This section reviews some of these new facilities, and points to the various approved experiments that will make use of them.

The SBS 
%Super Bigbite Spectrometer (SBS) -- CPS, NPS were spelled out, why not SBS?
facility was installed in Hall~A at Jefferson Lab in Fall~2021. The principal goal is to provide large acceptance at high luminosity so that small cross sections can be measured with high precision. Particle tracking is accomplished using Gas Electron Multiplier (GEM) detectors, which are able to run at high rates while providing excellent ($\sim70~\mu$m) position resolution. The high acceptance SBS is generally paired with a complementary spectrometer and detector system for tagging the scattered electrons. The first experiments to run will focus on neutron and proton magnetic and electric form factors, followed by SIDIS measurements. See Section~\ref{sec:MomTomNucleon}.

%A long standing problem in QCD is the discrepancy between experiment and theory for the pion structure function~\cite{PhysRevC.63.025213}.
Experimental determinations and theoretical interpretations of 
the pion structure function have been controversial for many years. 
Recently, the pion valence quark distribution has been extracted from 
LQCD calculations of matrix elements analyzed in terms of QCD collinear 
factorization, showing good agreements with experiments~\cite{PhysRevD.102.054508}. 
A lattice study of pion valence parton distribution within the framework of Large Momentum Effective Theory has also been reported~\cite{PhysRevD.100.034516}.
Experimental results to date are derived from the Drell-Yan process with a pion beam. CEBAF will be used to make a different type of measurement, using electron deep inelastic scattering (DIS) from the pion cloud of the nucleon. This Tagged Deep Inelastic Scattering (TDIS) measurement will be performed with the SBS apparatus, augmented with a radial time projection chamber to tag recoil protons from the reaction $n(e,e^\prime p)X$. 
Details of this TDIS measurement are discussed in Section~\ref{sec:PionKaonStructure}.

As an ambitious general purpose apparatus,  %Solenoidal Large Intensity Device (SoLID) %already defined
SoLID is being prepared for eventual use in Hall~A. SoLID makes use of the superconducting CLEO-III magnet from Cornell, which has already been delivered to the laboratory along with the yoke iron. SoLID will exploit the full capabilities of CEBAF, by combining the high intensity beam of excellent polarization quality with a large acceptance that includes a full azimuthal acceptance. %with several experiments already approved for SIDIS and for PVDIS. -- already mentioned, also J/Psi
The apparatus allows for transverse and longitudinally polarized targets for SIDIS measurements, see Section~\ref{sec:PionKaonStructure}. The PVDIS experiment will provide a precision measurement of %search for anomalous -- the coupling is not anomalous
electron-quark neutral current couplings as a test of the Standard Model, as discussed in Section~\ref{sec:BSM}. Measurements of near threshold $J/\psi$ production are also planned for SoLID allowing for the probe of quantum anomalous energy inside the proton.

The lightest nuclei, $^4$He, $^3$He, and $^2$H, offer many opportunities for understanding the fundamental partonic behavior of bound nucleons. Disecting these reactions, however, require the ability to tag spectator fragments in the midst of reactions with multi-GeV electrons. A Low Energy Recoil Tracker (ALERT) has been commissioned to join with CLAS12 to carry out these studies. Measurements on these light nuclei will include % Deeply Virtual Compton Scattering, 
DVCS, Deeply Virtual Meson Production (DVMP), and a variety of inclusive and semi-inclusive reactions. The many facets of this program including the EMC effect are discussed in Section~\ref{sec:ALERTmeasurements}.

A copious flux of kaons can be produced by the CEBAF beam through $\phi$ photoproduction. This is the basis for a new $K_L$ beam that will be developed in the Hall~D beam line, with the kaon beam incident on a target in the GlueX detector. The primary aim is spectroscopy of strange baryons and their excitations. This is discussed further in Section~\ref{sec:expbaryons}. At the same time, data will be collected on strange mesons through diffraction. 

New facilities are under development to search for physics beyond the Standard Model, once again using the unique capabilities of CEBAF. %The Measurement of Lepton-Lepton Elastic Reaction (MOLLER) %already defined
Among these, MOLLER is a large scale effort to make a percent level measurement of parity violation in electron-electron elastic scattering. This experiment will be sensitive to new electroweak interactions with mass scales on the order of 39~TeV. %see Eq.(13), should br 39 TeV
More details are given in Section~\ref{sec:MOLLER}. A new Beam Dump Experiment (BDX) is planned that would run parasitically with MOLLER (or other high intensity running), which will search for dark sector particles produced in the Hall~A beam dump. See Section~\ref{sec:BDX}.

\subsection{Future Science Opportunities}

It is becoming increasingly clear that there are exciting scientific opportunities using CEBAF beyond the currently planned decade of experiments. % with the recently upgraded 12~GeV facility. -- no need, the whole paper is about 12 GeV
In fact, one can envision that CEBAF will continue to operate with a fixed target program at the ``luminosity frontier,'' up to $10^{39}~{\rm cm}^{-2}{\rm s}^{-1}$, with large acceptance detection systems. This regime is a factor of 100,000 higher in luminosity than the Electron-Ion Collider (EIC), to be built in the next decade, and so represents very complementary capabilities even in the era of EIC operations.

Three dimensional imaging of the quark structure of the nucleon through DVCS and SIDIS 
%deeply virtual Compton scattering (DVCS) and semi-inclusive deep inelastic scattering (SIDIS) %defined many times in this section already
will be major programs for the presently planned CEBAF program as well as at the EIC. However, significant additional information can be obtained by studying double DVCS (DDVCS), a process with two virtual photons with cross sections and interaction rates a factor of 100 lower than DVCS. Therefore this process is not viable at EIC and must be studied using CEBAF at Jefferson Lab. Modest future detector upgrades will facilitate DDVCS studies in experimental Halls~A and~B.

Some very important information about nucleon imaging would require measurements of DVCS with positrons as well as electrons. A very large community of nuclear physicists has been holding workshops to develop these concepts in the last few years. Recent proposals to the Jefferson Lab PAC to use positron beams at CEBAF have received conditional approval, pending further study of the technical realization of positron beams. Operations with positron beams (polarized and unpolarized) will open a new area of study for CEBAF in the future.

Recently, the CBETA facility at Cornell has demonstrated eight pass recirculation of an electron beam with energy recovery (four accelerating beam passes and four decelerating beam passes). All eight beams are recirculated by single arcs of fixed field alternating gradient (FFA) magnets. This exciting new technology would enable a cost-effective method to double the energy of CEBAF, enabling new scientific opportunities in meson spectroscopy and extending the kinematic range of nucleon imaging studies. Technical studies of the implementation of FFA technology at CEBAF are in progress.

The individual sections of this paper include descriptions of a large number of scientific opportunities in the future of CEBAF, including higher energy beams and intense beams of polarized and unpolarized positrons.

\subsection{Complementarity with Existing and Future Experimental Facilities Worldwide}

The electron beams delivered by CEBAF will continue to be unparalleled in intensity and precision for decades to come. Energies up to 24~GeV will be available using an innovative upgrade to the accelerator arcs, and %polarized
positron beams will be delivered to the experimental halls. These facilities will keep CEBAF uniquely capable of a large number of important measurements in nuclear and hadronic physics. This section points out the various ways that CEBAF complements existing and planned facilities at other laboratories worldwide.

\subsubsection{The COMPASS Experiment at CERN and Electron-Positron Collider Experiments}

Currently the only fixed-target facility in operation that utilizes a lepton beam with the capability of both polarized beam and polarized target is the COMPASS experiment at CERN, using a muon beam at 160 GeV/$c$. The kinematic coverage of the COMPASS experiment is complementary to that of CEBAF at Jefferson Lab with significantly larger $Q^2$ values reaching 100 (GeV/c)$^2$ and a parton momentum fraction $x$ range of 0.008 to 0.21. The COMPASS collaboration plans to take the remaining SIDIS data using a polarized $^6$LiD target as an effective polarized ``neutron'' target in the near future.

The BES-III is a particle physics experiment at the Beijing Electron-Positron collider II at the Institute of High Energy Physics in China. Belle II -- an upgrade from the Belle experiment -- is a particle physics experiment at SuperKEKB, an electron-positron collider located at KEK in Japan. Both BES-III and Belle II contribute to the study of the three-dimensional imaging of the nucleon in momentum space through the extraction of the Collins fragmentation function from pion pairs produced in electron-positron collisions.

\subsubsection{Experiments at other Nuclear and Particle Physics facilities}

There are a number of experiments/programs ongoing at nuclear and particle physics facilities worldwide using hadron beams. These include the spin physics program at the Relativistic Heavy Ion Collider (RHIC) at Brookhaven National Laboratory (BNL) with polarized proton beams, and the polarized Drell-Yan experiment (SpinQuest) at Fermi National Accelerator Laboratory (FNAL) using polarized targets, both in the United States. Internationally, there are experiments using the primary proton beam of 30 GeV, and kaon and pion secondary beams at the Hadron Experimental Facility of Japan Proton Accelerator Research Complex (J-PARC). 

FAIR, an international accelerator facility in Darmstadt, Germany, is being built at GSI Helmholtzzentrum für Schwerionenforschung, where existing accelerator facilities will become part of FAIR and will serve as the first acceleration stage. The  FAIR experiment most relevant to the Jefferson Lab and the EIC science is the Antiproton Annihilation at Darmstadt (PANDA) experiment, in which anti-proton beams will be used to produce new particles in order to understand how mass is created by the strong nuclear force. 
Another future facility that is under construction is the Nuclotron-based Ion Collider fAcility (NICA) at the Joint Institute for Nuclear Research in Dubna, Russia, proposed by the Spin Physics Detector (SPD) collaboration, to study the spin structure of the proton and deuteron and other spin-related phenomena with polarized proton and deuteron beams at a collision energy up to 27 GeV and luminosity up to 10$^{32}$ cm$^{-2}$s$^{-1}$. 

\subsubsection{The Electron-Ion Collider (EIC) in the U.S.}

A principal aim of the EIC program is the study of the QCD sea inside the nucleon and nuclei, by focusing on quark, antiquark, and gluon parton distributions at small Bjorken $x$. The EIC program will accomplish this with much higher center-of-mass energies and exceptional luminosity for an environment of polarized colliding beams.

Of course, this aim is directly complementary to the extremely high luminosity, high $x$ reach of fixed target experiments that will be accomplished using CEBAF.
Contrasting these two accelerator environments, along with other current and prior facilities, it is not only evident that CEBAF and the EIC will each greatly extend  what is currently available, but they have significant overlap in the region of moderate $x$. In fact, the evolution of the structure functions from the valence into the sea quark region is an important area and motivation for study. The overlapping kinematic regions between CEBAF and the EIC will make it possible to study this evolution in detail, using both inclusive and semi-inclusive deep inelastic scattering reactions with polarized beams and targets or colliding beams, including, for CEBAF, polarized positron beams. Such studies will advance our knowledge about how QCD works which in turn allows for the extraction of the three-dimensional structural information of quarks and gluons inside the nucleon and nuclei.

The kinematic landscape for deep inelastic scattering at CEBAF and the EIC is shown schematically in Fig.~\ref{fig:CEBAFvsEIC}.
\begin{figure}[t]
\begin{center}
\includegraphics[width=0.75\textwidth]{Figures/plotDIS.pdf}
%\includegraphics[width=0.49\textwidth]{Figures/CEBAFvsEIC_edit2.pdf}
\caption{Kinematic regions of Deep Inelastic Scattering and the comparative reach of EIC and CEBAF, as well as other facilities compared with parton distributions from CJ15~\cite{Accardi:2016qay}. CEBAF (at 12~GeV) will provide very high luminosity in the valence quark region, while providing overlap here with EIC measurements. The focus of the EIC will be on the sea at very low $x$, with orders of magnitude higher luminosity than other colliders. While not shown in the figure, a 24~GeV CEBAF provides significantly higher $Q^2$, low $x$ values in the valence region which is critical for several key measurements.}
\label{fig:CEBAFvsEIC} 
\end{center}
\end{figure}
Precision measurements in the valence quark region requiring high luminosity are clearly the purview of CEBAF, with the 24~GeV upgrade providing important overlap into the sea quark region where the EIC is designed to probe at low $x$.

 
%{\bf Are there good places in the rest of the document where this physics can be referred to?}

\subsubsection{The Electron-Ion Collider (EIC) in China}

The physicist community in China, %Chinese physicists -- Chinese can be overseas
together with international collaborators, proposed a polarized electron-ion collider at the  High Intensity heavy-ion Accelerator Facility (HIAF), currently under construction in southern China.
EIC@HIAF~\cite{Anderle_2021} was proposed as an extension to HIAF in a phased approach. The first phase of the China EIC will include 3 to 5 GeV polarized electrons on 12 to 23 GeV polarized protons (and ions at about 12 GeV/nucleon), with luminosities of 1 to 2 $\times 10^{33}$ cm$^{-2}$ s$^{-1}$. This facility with complementary kinematic reach to both Jefferson Lab and the US EIC will allow for the studies of one and three-dimensional nucleon structure, the QCD dynamics, and to advance the understanding of the strong nuclear force. 





%{\bf [Maybe different groupings based on the community and the science? FRIB, Mainz, J-PARC, PANDA, BES, RHIC/RHIC-Spin, COMPASS++/Amber, Belle-II (TMD related), Fermilab SpinQuest \ldots]}

%\subsection{International Context}

%{\bf [Focus on international contributions to the laboratory program, in addition to the complementarity with international facilities. There is overlap with the section above. I believe the international context is about the science, facilities and complementarities. They should be combined or move the international facilities to this section. We can also mention the international users and contributions in the overview section briefly.]}