\section{Electromagnetic Form Factors and Parton Distributions}\label{sec:section2-FF-PDF}
%The proton and neutron, collectively known as the nucleon, are fundamental particles with spin-1/2. Therefore they can have exactly two static electromagnetic moments, namely the charge and magnetic dipole moment. Although current conservation fixes the charge, the magnetic moment relies on detailed dynamics. Dirac's relativistic wave equation correctly predicted the magnetic moment of the electron, but in 1933 Otto Stern discovered that the proton's magnetic moment was anomalous. Soon afterwards, different groups showed that the neutron magnetic moment was nonzero, and therefore also anomalous. Thus began the field of hadronic dynamics, in trying to understand the underlying structure of the nucleon.

With the advent of particle accelerator technology, high energy electron scattering became an indispensable tool for understanding the internal structure of nucleons. This is because the fundamental cross section separates into an electronic part (which is accurately determined from Quantum Electrodynamics) and a hadronic part which can be formulated in terms of structure functions of various kinematic quantities, and measured in electron scattering. For elastic scattering from the nucleon, the structure functions can be written in terms of form factors $G_E(q^2)$ and $G_M(q^2)$, which represent the internal charge and magnetic moment distributions as a function of the square of the four-momentum transfer. In Deep Inelastic Scattering (DIS), where both the $q^2$  and invariant mass of the hadron fragments are large, the structure functions are decomposed into Parton Distribution Functions (PDFs) which describe the internal structure in terms of quark degrees of freedom.

These approaches have been exploited with great success. Robert Hofstadter~\cite{RevModPhys.28.214} led a program of elastic (and inelastic) electron scattering from nucleons and nuclei. The resulting distributions for charge and current elucidated some of the most fundamental properties that any dynamic theory of hadron structure must predict. Some time later, the first DIS experiments by Richard Taylor, Henry Kendall, and Jerome Friedman~\cite{RevModPhys.63.573,RevModPhys.63.597,RevModPhys.63.615} showed the partonic internal structure of the nucleon. 

However, at each new stage of experimental investigation, new challenges and inconsistencies arose. New experiments, many of which were performed at Jefferson Lab, were performed to address discrepancies, but some issues still remain.

One important example has been the ``proton radius puzzle.'' Precision measurements of the proton charge radius from muonic hydrogen disagreed with electron scattering form factor measurements extrapolated to zero momentum transfer. The PRad experiment at Jefferson Lab appears to have resolved this puzzle, with a precision measurement at very low momentum transfer $q^2$, getting a result that agrees with the muonic hydrogen experiments.

The high $q^2$ behavior of the proton elastic form factors $G_E(q^2)$ and $G_M(q^2)$ has also been a puzzle. It was tacitly assumed that the ratio of these form factors should be the proton magnetic moment over all $q^2$, based on early measurements. This appeared to be borne out by Rosenbluth separations of the elastic cross section. However, precision measurements of the ratio using recoil polarization techniques at Jefferson Lab showed that this was not the case. It appears that the discrepancy is due to higher order ``two--photon'' corrections to the cross sections, which have also been investigated at Jefferson Lab.

Experimental facilities at CEBAF will continue to tackle these questions in the years to come. A program in nucleon elastic form factors is currently underway with the Super Bigbite Spectrometer (SBS) system. Spin-- and flavor--dependent structure functions in the valence region will be probed with greater precision, with tighter control on systematic uncertainties. Higher electron beam energies and beams of polarized positrons will provide new handles to be exploited. Furthermore, elastic and inelastic scattering from pions and kaons (in tagged electron scattering from nucleons) will extend this physics into the meson sector.

%The time-honored approaches to studying the internal structure of the nucleons are elastic electron scattering to measure the electromagnetic form factors and deep-inelastic scattering (DIS) to measure the parton distribution functions (PDFs). Although much is known about these quantities, the Jefferson Lab (JLab) 12 GeV program will provide new and exciting insights into these fundamental structure measurements.
%The role of the two-photon process in elastic scattering, critical to providing precise extractions of the form factors, can be directly accessed through positron beams, which can be implemented in the future. The structure of pion and kaon could be accessed through
%processes on the nucleon target with small $t$-channel momentum transfer. 

\subsection{Elastic Form Factors at Ultra Low and High $Q^2$}
 
Probing the internal structure of the nucleon through elastic scattering began with the classical work of Hofstadter~\cite{Hofstadter56} who measured the proton radius for the first time in 1955. The precision measurement of the proton radius has again become a focus in nuclear and atomic physics experiments in recent years~\cite{gao2021proton}, motivating innovative approaches to make precise measurements of the proton form factor at ultra-low $Q^2$. Measuring the proton form factors with innovative technologies at ultra low $Q^2$ at JLab helps resolve the discrepancy among various experimental approaches. The high-precision high-$Q^2$ form factors directly probe the underlying physics mechanism of high-energy elastic scattering. 

\subsubsection*{\it Proton Radius:}
\label{sec:PRad}
The PRad experiment~\cite{PRad_nature:2019} performed in 2016, was the first high-precision $e-p$ experiment 
to utilize a magnetic-spectrometer-free method along with a windowless hydrogen gas target aiming at resolving the proton charge radius puzzle triggered by the muonic hydrogen spectroscopic measurements~\cite{Pohl10,Anti13}. These innovations overcame several limitations of previous $e-p$ ~experiments and reached unprecedentedly small scattering angles. The PRad result~\cite{PRad_nature:2019} is, within its experimental uncertainty, consistent with the $\mu$H results, and was one of the critical inputs in the updated CODATA-2018 value for $r_p$~\cite{CODATA_2018}. To reach the ultimate precision offered by this new method, the PRad collaboration proposed an enhanced version of PRad, the PRad-II experiment~\cite{PRad2} with several improvements including adding a second plane of tracking detectors, upgrading the hybrid electromagnetic calorimeter to an all-PbWO$_4$ crystal calorimeter, etc.. 
The approved PRad-II experiment will deliver the most precise measurement of $G^{p}_{E}$ ~reaching the lowest ever $Q^2$ (10$^{-5}$ GeV$^2$) in lepton scattering experiments, critical for the model-independent extraction of $r_p$. PRad-II will achieve a total uncertainty of 0.0036 fm on $r_p$,  which  is $\sim$ 4 times smaller than PRad and better than the most precise atomic hydrogen (H) spectroscopy result~\cite{Grinin20} with a total uncertainty of 0.0038 fm.
The projected $r_p$ from the PRad-II experiment is shown in Fig.~\ref{fig:PRad2_proj} along with recent electron scattering extractions~\cite{PRad_nature:2019,Bernauer10,Zhan11}, atomic physics measurements on ordinary hydrogen~\cite{CREMA_2017,hspec2018,Eric2019,Grinin20} and muonic hydrogen~\cite{Pohl10,Anti13}, and the CODATA values~\cite{CODATA_2014,CODATA_2018}.
The PRad-II precision will help address possible systematic differences between the most precise ordinary hydrogen and $\mu$H spectroscopy results and provide independent input for future CODATA recommendations for $r_p$ and the Rydberg constant. The precision of the PRad-II will also stimulate future high-precision lattice QCD predictions for the proton radius and contribute to new physics searches such as the violation of lepton universality.

\begin{figure}[ht]
%\vskip 0.5truecm
\centerline{
\includegraphics[width=0.8\textwidth]{Figures/PRad-II-figure.pdf}}
\caption{The projected $r_{p}$ result from PRad-II, shown along with the result from PRad and other measurements.}
\label{fig:PRad2_proj}
\end{figure}

\subsubsection*{Nucleon Form Factors at Large Momentum Transfer}

At a large value of $Q^2$, the form factors should reflect a transition to the perturbatively dominated mechanism and reveal the role of orbital angular momentum of the quarks and gluons in the nucleon. One of the first completed experiments in Hall A with the upgraded JLab accelerator was a precision measurement of the proton magnetic form factor up to $Q^2 = 16$~GeV$^2$~\cite{Christy:2021snt}. This experiment nearly doubled the $Q^2$ range over which direct Rosenbluth separations of $G_E$ and $G_M$ can be performed, and confirming the discrepancy with polarization measurements (believe to be the result of two-photon exchange corrections) to larger $Q^2$ values.
The new SBS and the upgraded BigBite Spectrometer are being installed in Hall~A and will be ready for experiments starting late 2021. 
A series of SBS experiments~\cite{GEP5_PAC47-short,GEN2-short,GMN-short,GENHC-short,GENRP-short,GENTPE-short} will measure the magnetic and electric form factors of the proton and neutron and allow a determination of the flavor separated form factors
to $Q^2 = 10-12$~GeV$^2$. A complementary measurement of the neutron magnetic form factor will be performed with CLAS12 in Hall B \cite{GMN-HallB-short}.
In Fig.~\ref{fig:SBSFF_summaryfig}, plots compare the projected results of the SBS form factor experiments to various theoretical models. To visualize the impact of the SBS experiments, the uncertainty bands from a fit to the existing data is compared to a fit including the SBS projected data for the ratio, $Q^2F_2/F_1$,  of the Pauli to Dirac form factors and the ratio, $F_1^d/F_1^u$, of the flavor separated down and up quark Dirac form factors are plotted in Fig.~\ref{fig:SBSFF_summaryfig}.
The SBS form factor experiments will push into a $Q^2$ regions in which theory expects new degrees of freedom to emerge in our understanding of QCD non-pertubative phenomena in nucleon structure, e.g., log scaling 
of $F_2/F_1$ predicted in Ref.~\cite{Belitsky:2002kj}.

\begin{figure}[ht]
  \begin{center}
   \includegraphics[width=0.98\columnwidth]{Figures/NewFourFFs_fig.pdf}
  \end{center}
  \caption{\label{fig:SBSFF_summaryfig} Projected results from the SBS form factor program, compared to existing data, including the preliminary results for $G_M^p$ extracted from the recent high-$Q^2$ elastic $ep$ cross section measurements from Hall A (GMP12), and selected theoretical predictions. Projected SBS results are plotted at values extrapolated from the global fit described in the appendix of Ref.~\cite{Puckett:2017flj} for the proton, the Kelly fit~\cite{KellyFit} for $G_M^n$, and the Riordan fit~\cite{Riordan:2010id} for $G_E^n$. Theoretical curves are from the GPD-based model of Ref.~\cite{Diehl:2004cx} (Diehl05), the DSE-based calculation of Ref.~\cite{Segovia:2014aza} (Segovia14), the VMD model of Ref.~\cite{Lomon:2006xb,Lomon:2012pn} (Lomon06), the covariant spectator model from Ref.~\cite{Gross:2006fg} (Gross08), the quark-diquark model calculation from Ref.~\cite{Cloet:2012cy} (Cloet12), and a relativistic constituent quark model calculation by Miller~\cite{Miller:2002ig} (Miller05). Top right: projected impact of SBS program on the flavor separated form factor ratio $F_1^d/F_1^u$, with central value and uncertainty band evaluated using the Kelly fit~\cite{KellyFit}. The improvement in this ratio is mainly due to the new $G_M^n$ data. Bottom right: projected impact of SBS program on the ratio $Q^2F_2/F_1$ of Pauli and Dirac form factors for the proton and neutron, also evaluated using the Kelly parametrization~\cite{KellyFit}. The improvement in these ratios is driven by the $G_E^p$ and $G_E^n$ data. The figure is taken from Ref.~\cite{Barabanov_2021}.}
\end{figure}

% - new data from LHC -> W/lepton asymmetries in pp collisions
% - SeaQuest data
% - new global QCD analysis, particularly by the JAM Collaboration,
% including simultaneous determination of PDFs and FFs, and including
% for the first time SIDIS data ... important constraints on sea quark PDFs.

\subsection{Quark Parton Distributions at High $x$}

JLab 12 GeV facility provides the unprecedented opportunities to access to high-$x$ quark distributions. New data from JLab and other facilities, including RHIC at BNL, FNAL, and the Large Hadron Collider (LHC), have provided more stringent constraints on PDFs in previously unmeasured regions at small and large values of $x$. %the parton momentum fraction $x$ -> moved to Section 1
At the same time, new analysis techniques have been developed, notably by the JLab Angular Momentum (JAM) collaboration~\cite{JAM}, using Monte Carlo methods and modern Bayesian analysis tools, which provide a more rigorous theoretical framework in which to analyze the new data.

\subsubsection*{Valence Quarks at High $x$}
\label{sec:duRatio}

One of the key properties of the nucleon is the structure of its valence quark distributions, $q_v = q(x)-\bar q(x)$.
% Valence quarks are the irreducible kernel of each hadron, responsible for its charge, baryon number and other macroscopic quantum numbers.
While many features of the valence quark PDFs have been mapped out in previous generations of experiments, the region of large %parton momentum fractions 
$x$ ($x \gtrsim 0.5$), where a single parton carries most of the nucleon's momentum, remains elusive. 
In particular, the traditional method of determining the $d$-quark PDF, from inclusive proton and deuteron structure function measurements, suffers from large uncertainties in describing the nuclear effects in the deuteron when extracting neutron structure information~\cite{Melnitchouk:1995fc, Accardi:2016qay}.% --- see the left panel in Fig.~\ref{f.donu}.

% $\bullet$ Updates from recent JAM QCD analyses

Recently, three JLab 12~GeV unpolarized DIS experiments, MARATHON~\cite{Abrams:2021xum} in Hall~A, BoNUS12~\cite{BONUS12} in Hall~B, and $F_{2d}/F_{2p}$~\cite{Niculescu:201002} in Hall-C completed data taking. These experiments aim to provide data to constrain PDFs in the high-$x$ region, especially the $d/u$ PDF ratio.
% To obtain the neutron structure function, deuterium or $^3$He target is usually used and nuclear effect needs to be corrected.
MARATHON measured the ratio of $^3$H to $^3$He structure functions, while BONUS12 tagged slow-recoiling protons in the deuteron, two different approaches to minimizing nuclear effects in extracting the neutron (and hence d) information. 
The MARATHON experiment also provided important input for the study of the nuclear EMC effect for isospin partners~\cite{Cocuzza:2021rfn}. The Hall-C experiment measured $H(e,e^\prime )$ and $D(e,e^\prime )$ inclusive cross sections in the resonance region and beyond. While there will be nuclear effects in the deuterium data, the experiment provides significant large $x$ range and reduced uncertainty to be combined with the large global data set of inclusive cross sections for PDF extraction. 

A planned experiment using PVDIS~\cite{PVDIS} on the proton with the proposed SoLID~\cite{SoLID} spectrometer will provide input on the $d/u$ ratio at high $x$ without contamination from nuclear corrections.
The experiment will measure the ratio of $\gamma Z$ interference to total structure functions, which at leading twist is sensitive to a new combination of PDFs that is not accessible to electromagnetic probes.

\iffalse
\begin{figure}[t]
\centering
\includegraphics[width=0.35\textwidth]{Figures/du_data.pdf}
\includegraphics[width=0.49\textwidth]{Figures/spinPDF.pdf}
\vspace*{-0.2cm}
\caption{{\color{blue} [UPDATE?]} (left) $d/u$ PDF ratio from the CJ15 analysis~\cite{Accardi:2016qay} (red band), illustrating the effects of nuclear corrections in deuteron DIS data (green band) and removing deuterium data altogether (hashed).
(right) Helicity PDFs for various quark flavors from the JAM15~\cite{Sato:2016tuz} Monte Carlo analysis of inclusive DIS data, using SU(3) symmetry constraints (blue bands), and the updated JAM17~\cite{Ethier:2017zbq} analysis of inclusive and semi-inclusive DIS without SU(3) constraints.}
\label{f.donu}
\end{figure}
\fi

\subsubsection*{Spin Structure Functions}
\label{sec:spinstrfunct}

As one of the high-impact flagship experiments, the $A_1^n$ experiment~\cite{A1n12GeV} in Hall~C at JLab completed data taking in 2020 with a polarized $^3$He target and a 10.4~GeV polarized electron beam. 
The polarized $^3$He target~\cite{polHe3} was upgraded to double the polarized luminosity to $2 \times 10^{36}$~cm$^{-2}$~s$^{-1}$ with in-beam polarization up to $\approx 55\%$ with a 30~$\mu$A beam, which is a new world record.
The new precision measurement on $A_1^{^3{\rm He}}$ expanded the Bjorken-$x$ of the extracted $g_1^n$ structure function to $x=0.75$.
Combined with the planned experiments to measure the proton and deuteron asymmetries $A_1^p$ and $A_1^d$ with CLAS12~\cite{CLAS12spin}, new global analyses will be able to extract the $\Delta u$ and $\Delta d$ quark helicity distributions in the high-$x$ region with much improved precision. 
%An example of a recent global analysis of inclusive and semi-inclusive polarized DIS data from the JAM collaboration is shown on the right panel in Fig.~\ref{f.donu}.

\iffalse
\begin{figure}[t]
\begin{center}
\includegraphics[width=0.49\textwidth]{Figures/spinPDF.pdf}
\caption{{\color{blue} [UPDATE?]} Helicity PDFs for various quark flavors from the JAM15~\cite{Sato:2016tuz} Monte Carlo analysis of inclusive DIS data, using SU(3) symmetry constraints (blue bands), and the updated JAM17~\cite{Ethier:2017zbq} analysis of inclusive and semi-inclusive DIS without SU(3) constraints.}
\label{f.spdf}
\end{center}
\end{figure}
\fi

Another experiment that completed data taking in Hall~C in 2020 was the d2n experiment~\cite{d2n12GeV}, which measured both $g_1$ and $g_2$ with a polarized $^3$He target. 
Extracted $g_1^n$ and $g_2^n$ structure functions allow the
    $d_2^n = \int_0^1 x^2 (2 g_1^n + 3 g_2^n)dx$ 
moment to be determined in a $Q^2$ range from 3.5 to 5.5~(GeV/c)$^2$, significantly extended from the previous range covered by JLab~6~GeV experiments~\cite{Posik:2014usi, Armstrong:2018xgk}.
%
A planned measurement~\cite{SoLIDd2n} of $d_2^n$ with the proposed SoLID spectrometer in Hall~A will extend the $Q^2$ range to cover $1.5 < Q^2 < 6.5$~(GeV/c)$^2$ with improved precision.
These measurements will provide a benchmark test of lattice QCD calculations of twist-3 matrix elements. 

\iffalse
\begin{figure}[ht]
\begin{center}
\includegraphics[width=0.49\textwidth]{Figures/d2.pdf}
\caption{{\color{blue} [UPDATE?]} Comparison of proton (red) and neutron (blue) $d_2$ matrix element extractions from JLab experiments and lattice QCD with the JAM15 global QCD analysis~\cite{Sato:2016tuz}.}
\label{f.d2n}
\end{center}
\end{figure}
\fi

Finally, a recently approved experiment~\cite{dalton2020measurement} in Hall~D with circularly polarized photon beam on a polarized proton target will provide the final missing piece, the high-energy part of the contributions to the GDH sum rule~\cite{Gerasimov:1965et,Drell:1966jv,HY}. 
This will complete a multi-decade-long effort to test this fundamental sum rule.

\subsubsection*{Flavor separation, SIDIS and PVDIS}
While inclusive DIS from protons and neutrons allows us to separate contributions from $u^+ = u + \bar u$ and $d^+ = d + \bar d$ PDFs at large $x$ (and similarly for the polarized $\Delta u^+ = \Delta u + \Delta \bar u$ and $\Delta d^+ = \Delta d + \Delta \bar d$ PDFs), further decomposition into $u$, $d$ and $s$ flavors, or valence and sea separation, requires additional observables sensitive to different combinations of PDFs.
Drell-Yan lepton-pair and $W$-boson production in unpolarized and polarized $pp/pd$ experiments provide evidence of asymmetries in the spin-averaged $\bar d - \bar u$~\cite{Dove:2021ejl} and spin-dependent $\Delta\bar d - \Delta\bar u$~\cite{Adam:2018bam} sea quark PDFs, respectively.
SIDIS measurements with tagged pions or kaons provide an additional powerful method for PDF flavor separation~\cite{Sato:2019yez, Ethier:2017zbq}.

Planned JLab12 SIDIS experiments with pion production~\cite{SIDIS-SeaAsymmetry} on unpolarized proton and deutron targets will provide precision data for the study of the light antiquark asymmetry in the medium-$x$ ($0.1 \lesssim x \lesssim 0.6$) region. 
SIDIS pion production with polarized proton, dueteron~\cite{CLAS12spin} and $^3$He~\cite{SIDISHe3} targets will in addition provide high-precision data for the study of the light-quark asymmetry in the polarized sea.

The strange quark sea can be constrained by SIDIS kaon production data.
However, the difficulty in the precision determination of kaon fragmentation functions at relatively low energy scales %of JLab12 and HERMES 
highlights the limitations of JLab12 in the study of the strange sea with SIDIS~\cite{Moffat:2021dji}.
%
On the other hand, PVDIS provides a clean alternative method for strange quark PDF extraction, without the complications of fragmentation, utilizing the interference between electromagnetic and weak interactions.
A proposed PVDIS measurement in Hall C~\cite{PVPDF} would provide precision data for the determination of the strange sea distribution.
%
Furthermore, PVDIS on a polarized $^3$He target with SoLID~\cite{PVDIS_polHe3} is being considered as a possible additional method for polarized strange quark extraction.

\subsection{Pion and Kaon Structure} 
\label{sec:PionKaonStructure}

In general, accessing the structure of the pion and kaon is difficult in electron scattering as these Goldstone bosons of QCD cannot easily be made into a stable target. However, chiral perturbation theory indicates that these effective degrees of freedom play an important role in nucleon structure. Through a small $t$-channel momentum transfer measurement of a tagged proton recoil (energy close to beam energy) in coincidence with a DIS event of large invariant mass $W$, HERA experiments at DESY was able to probe the pion structure of the nucleon within a theoretical assessment of the background contributions. Access to meson structure can be done at elastic scattering kinematics as well, where one can probe the meson form factors $F_{\pi, K}(Q^2)$. Here, in addition to a recoiling target baryon, the final product contains a pion or kaon. The JLab 12 GeV program has approved both this type of experiment and the DIS type to access the partonic structure of meson targets. 

The E12-06-101 experiment~\cite{CLAS12spin} will extract the pion form factor through $p(e,e'\pi^+)n$ and $d(e,e'\pi^-)pp$ with $Q^2$ extending to 6 GeV$^2$ from 2 GeV$^2$ and $-t_{\min} \sim 0.005\sim 0.2$ GeV$^2$. The proposed separation of longitudinal and transverse
structure functions is a critical check of the reaction dynamics. The charged pion electric form factor is a topic of fundamental importance to our understanding of hadronic structure. In contrast to the nucleon, the asymptotic normalization of the pion wave function is known from pion decay. There is a robust pQCD prediction in the asymptotic limit where $Q^2\rightarrow \infty$: $Q^2F_\pi(Q^2)\rightarrow 16\pi\alpha_s(Q^2)f_\pi^2$. Therefore it is an interesting question at what $Q^2$ this pQCD result will become dominant. The available data indicate that the form factor at $Q^2=2$ GeV is at least a factor of 3-4 larger. The new data will provide improved understanding of the non-perturbative contribution to this important property of the pion as well as mapping out the transition to the perturbative regime. 

The partonic structure of the pion and kaon can be accessed as at HERA through a (semi-inclusive, target-tagged) experiment leveraging the Sullivan process. The approved E12-15-006 experiment in Hall A~\cite{E12-15-006} studies the reactions $p(e,e'p)X$ and $d(e,e'pp)X$ with $M_x > 1$ GeV, using a dedicated GEM-based time projection chamber for large angular acceptance and low momentum kinematic coverage to detect the recoiling protons. To probe the soft part of the nucleon wave function, the $t$-channel momentum transfer on the nucleon is limited to 0.2 GeV$^2$ and several values of $t$ are measured for each $x$ for meson pole extrapolation. The reactions and targets allow access to both the charge and neutral pion
cloud in the nucleon. The result can be compared with parton distributions from the pion-initiated Drell-Yan process and thus offers an important check on the concept of the virtual pion target in the nucleon. A follow-up proposal
through tagging the hyperon final state studies the possibility of extracting the kaon structure function~\cite{C12-15-006A}, although theoretically the kaon pole is further away
from the accessible kinematic region and thus will have larger contamination
from background contributions.  An important prospect for these measurements is to compare with lattice QCD
calculations of the parton structure using large momentum effective theory~\cite{Ji:2020ect}. 

\subsection{Two-photon Exchange Physics with Positron Beams} 

The discrepancy between high precision measurements of the proton's elastic form factor ($\mu_pG_E/G_M$) using  Rosenbluth separations of unpolarized cross-section data and polarization transfer in elastic electron-proton scattering exposed limitations of $1\gamma$ exchange Born approximation. The discrepancy between the two methods, see Fig.~\ref{fig:tpe}
~\cite{Accardi:2020swt}, is attributed to unaccounted hard $2\gamma$-exchange (TPE) radiative corrections. The two-photon exchange (and the large class of hadronic box diagrams) is hard to calculate without the inclusion of a significant degree of model dependence. While a few $e^+/e^-$ experiments measure a small TPE effect in the region of $Q^2<2$ GeV$^2$ \cite{Arrington:2004tpe,vepp3:2015tpe,clas:2015tpe,olympus:2017tpe,clas:2017tpe}, as precise measurements show \cite{Christy:2021snt}, it is possible that the TPE contribution is significant at higher momentum transfers. With planned measurements at $Q^2$ up to $10-16$ GeV$^2$ for different elastic form factors discussed in this section before, validation of calculations of TPE contribution to the elastic scattering at large $Q^2$ is crucial.  

\begin{figure}
\begin{center}
\includegraphics[width=0.49\textwidth]{Figures/ff_ratio.pdf}
\caption{The proton form factor ratio $\mu G_E/G_M$, as determined via Rosenbluth-type (blue points, from Litt '70\cite{Lit70}, Bartel '73\cite{Bar73}, Andivahis '94\cite{And94}, Walker '94\cite{Wal94}, Christy '04\cite{Chr04}, Qattan '05\cite{Qat05}]) and polarization-type (red points, from [Gay01\cite{Gay01}, Pun05\cite{Pun05}, Jon06\cite{Jon06}, Puc10\cite{Puc10}, Pao10\cite{Pao10}, Puc12]) experiments. Curves are from a phenomenological global fit by Bernauer \cite{Ber14} to the Rosenbluth-type world data set.}
\label{fig:tpe} 
\end{center}
\end{figure}

Two strategies are proposed for studying TPE contribution using up to 11 GeV positron beams: the unpolarized $e^+p$/$e^-p$ cross-section ratio, and polarization transfer in polarized elastic scattering.  The next-order correction to the Born approximation contains terms corresponding to the interference of $1\gamma$- and $2\gamma$ exchange diagrams that are lepton charge sign dependent. This makes it possible to determine the size of the TPE effect in the ratio of positron to electron scattering cross-section.
The TPE contribution to the cross section ratio convolutes additional form-factors that become non-zero when moving beyond the one-photon exchange approximation. A cross-section ratio measurement is proposed for all three experimental halls with electron/positron beams (A, B, and C), with high precision spectrometers in Halls A and C \cite{AY:2021,CBS:2021} and the CLAS12 in Hall B~\cite{BBCSS:2021}.
%\cite{Accardi:2020swt}.

The polarization transfer measurement is less sensitive to hard TPE but can contribute its determination by providing additional information. As described in~\cite{Carlson:2007sp}, the ratio of transverse and longitudinal polarization transfer has different dependence to the additional form factors. The proposed polarization transfer measurement with positron beams in Hall-A will complement the cross section ratio measurements and help to further constrain TPE effects.

\section{Nuclear Femtography}

QCD is responsible for nearly all of the visible mass in the universe. However, our understanding of the nucleon in terms of its fundamental quark and gluon degrees of freedom is still miniscule when compared to our understanding the structure of atoms and molecules. The ultimate goal is to experimentally determine the quantum mechanical Wigner distribution in phase space. Measurements of PDFs integrate out all of the spatial and most of the momentum variables, so more sophisticated experiments are necessary to more completely map out the Wigner distribution.

Semi--inclusive measurements, including spin polarization observables, were provided by the pioneering measurements at HERMES, COMPASS, and the Jefferson Lab 6~GeV program, among others. Results on Generalized Parton Distributions (GPDs) and Transverse Momentum Distributions (TMDs) are now published, over limited ranges of the relevant kinematic variables. The upgraded detectors and CEBAF beam energy and intensity, as well as the potential for polarized positron beams, promise to provide a more detailed three-dimensional spatial mapping of the nucleon.

Indeed, this is a major thrust of the JLab 12 GeV facility. Mapping the (2+1)D mixed spatial-momentum images of the nucleon in terms of GPDs has been one of the important goals. GPDs expand greatly the scope of the physics in the traditional elastic form factors %FF is defined as fragmentation function later
and PDFs. On the other hand, the 3D images in the pure momentum space can be made with another generalized distributions: the TMDs. 
These femto-scale images (or femtography) will provide, among other insights, an intuitive understanding on how the fundamental properties of the nucleon, such as its mass and spin, arise from the underlying quark and gluon degrees of freedom. 

However, obtaining high-quality images from experimental data has remained a dream for more than two decades.  First, one needs to have a large amount of a specific type of experimental data -- Deep Exclusive Processes (DEP) and SIDIS -- from high-energy electron-nucleon collisions, which have not been systematically available despite the previous studies at JLab 6 GeV, HERMES, and COMPASS experiments. Second, as in any other imaging process, turning the data into images requires algorithms that can efficiently extract critical information. 
In the case of nucleon femtography, this represents a major challenge that can first be met with the JLab 12 GeV data. 

\subsection{Spatial Tomography of the Nucleon} 

The standard approach of imaging a microscopic object is through diffractive scattering, as in optics. To obtain the phase-space quark and gluon distribution in a hadron, a new type of diffractive scattering was suggested in which a deeply-virtual photon (Bjorken limit) diffracts on a nucleon, generating a real photon or other hadrons~\cite{Ji:1996ek}. These DEPs allow probing entirely new structural information of the nucleon through QCD factorization (see Fig.~\ref{fig:dep}). The real photon production process has been called %deeply-virtual Compton scattering or 
DVCS~\cite{Ji:1996nm}, and for meson production, deeply-virtual meson production (DVMP)~\cite{Radyushkin:1996ru}.
DEPs contain the elements to learn the origin
of the nucleon mass and spin, and its gravitational properties. The information on these fundamental physical properties are encoded in the
GPDs~\cite{mueller1994wave,Ji:1996ek}.

\begin{figure}[htb]
\centering
\begin{minipage}{0.45\textwidth}
\centering
\includegraphics[width=0.55\textwidth]{Figures/DVCS.png}
\end{minipage}
\begin{minipage}{0.45\textwidth}
\centering
\includegraphics[width=0.55\textwidth]{Figures/DVMP.png}
\end{minipage}
    \caption{Deep exclusive processes in electron scattering as new type of hard scattering allowing QCD factorization and probing of the generalized structure~\cite{Ji:1996ek}.}
    \label{fig:dep}
\end{figure}

GPDs are hybrid physical quantities reducing
to PDFs and form factors in the special kinematic limit. Thus they generally depend on Feynman $x$, momentum transfer $t= - \Delta^2= (P'-P)^2$, 
as well as the skewness parameter $\xi$. 
The GPD $E$ and $H$ defined from vector
current and its generalization to gluons 
provide the form factors $A$, $B$, $C$, and $\bar C$ of the QCD energy-momentum tensor (EMT)~\cite{Ji:1996ek}, 
\begin{equation}
\begin{split}
\label{Tmunuformfactors}
\langle{P'}|T_{q, g}^{\mu \nu}|\rangle{P}\rangle=\bar{u}\left(P^{\prime}\right)&\bigg{[}A_{q, g}(t) \gamma^{(\mu} \bar{P}^{\nu)}+B_{q, g}(t) \bar{P}^{(\mu} i \sigma^{\nu) \alpha} \Delta_{\alpha} / 2 M\\
&\;\;+C_{q, g}(t)\left(\Delta^{\mu} \Delta^{\nu}-g^{\mu \nu} \Delta^{2}\right) / M
+\bar{C}_{q, g}(t) g^{\mu \nu} M\bigg{]} u(P)\ .
\end{split}
\end{equation}
One of the combinations yields the mass form factor~\cite{Ji:2021mtz} 
\begin{equation}
    G_m(t)=\left[ MA\left(t\right) +B(t)\frac{t}{4M}
-C(t)\frac{t}{M}\right] \ . 
\end{equation}
from which one can construct the mass distribution
as well as the mass radius, $\langle r^2 \rangle_{m}
     = 6 \left|\frac{dG_{m}(t)/M}{dt}\right|_{t=0}$. 
The EMT form factors also provide the key information about 
the proton spin carried by quarks 
and gluons~\cite{Ji:1996ek}, 
\begin{equation}
     J_{q, g} = \frac{1}{2}[A_{q,g}+B_{q,g}]
\end{equation}
Finally, the form factors $C(t)$ (also called $D$-term) and $\bar C(t)$ have been related to ``pressure and shear pressure distributions''~\cite{Polyakov:2018zvc}. It was discovered that the GPDs provide phase-space images of the quarks and gluons with a fixed longitudinal momentum $x$ (momentum-dissected tomography)~\cite{Burkardt:2000za, Belitsky:2003nz}. 

Experimental observables in DVCS are parameterized by Compton Form Factors (CFFs) which as
functions of $t$, $\xi$, and $Q^2$ (which corresponds to renormalization scale of GPDs)~\cite{Belitsky:2010jw}.
At leading twist, there
are eight CFFs (four complex pairs) which are related to four relevant GPDs, $H$, $E$, $\tilde H$, $\tilde E$, which contain one additional variable $x$ integrated over in CFFs, 
\begin{equation}
     {\cal F}(\xi, t, Q^2) = \int dx F(x, \xi, t)
     \left( \frac{1}{\xi-x+i\epsilon}- \frac{1}{\xi+ x+i\epsilon}          \right)
\end{equation}
where $F$ is a generic GPD. 
From the analysis of data from HERA and HERMES at DESY, as well as the results of new dedicated experiments at JLab, and at COMPASS at CERN, the experimental constraints on CFFs have been obtained from global extraction fits~\cite{Kumericki:2016ehc,Moutarde:2019tqa}.  However, data covering a sufficiently-large kinematic range, and the many different polarization observables, have not been systematically available. Moreover, meson 
production at JLab 6 GeV has not yet shown parton dominance of scattering. The 12 GeV program at JLab will provide comprehensive information on these hard diffractive processes, entering the precision era for GPD studies.

Extracting all 8 CFFs independently at fixed kinematics require a complete set of experiments. However, given these CFFs is not sufficient to reconstruct the GPDs due to the loop integral in 
the hand-bag diagram. One either has to make some models with parameters to fit to experimental data or make combined fits with lattice QCD data.
Experimentally, one needs to explore processes that will give both $x$ and $\xi$ information, such as double DVCS or similar processes as we discussed in the next subsection. On the other hand, large-momentum effective theories developed in recent years have made possible to calculate GPDs directly on lattice~\cite{Ji:2013dva,Ji:2014gla} and some preliminary calculations can be found in Refs.\cite{Lin:2020rxa,Alexandrou:2020okk}. 
 
Experiment E12-06-114~\cite{Hyde2019} in Hall A proposed a precision measurement of the helicity dependent and helity independent {\it cross sections} for the $ep \rightarrow ep\gamma$ reaction in DVCS kinematics. The experiment considered the special kinematic range $Q^2> 2$ (GeV/c)$^2$ , $W > 2$ GeV, and $-t < 1$ GeV$^2$, with $Q^2$ extending to 9 (GeV/c)$^2$ and $x$ central region from 0.36 to 0.60.  The experiment is a follow up of the successful Hall
A DVCS run at the 5.75 GeV (E00-110). With polarized 6.6, 8.8, and 11 GeV beams incident on the liquid hydrogen target, the scattered electrons will be detected in the Hall A beam-left High Resolution Spectrometer (HRS)  %(with maximum central momentum 4.3 GeV/c) 
and the emitted photon in a slightly expanded PbF2 calorimeter. In general, the experiment will not detect the recoil proton. The $H(e, e^\prime\gamma )X$ missing mass resolution is sufficient to isolate the exclusive channel with 3\% systematic precision. The specific scientific goals are: 1) Measuring Compton cross section and comparing
with the scaling prediction of DVCS process and the dominant GDP predictions; 2) Extracting all kinematically independent observables for each $Q^2$, $x$, and $t$ point. These include five azimuthal dependencies as
$\cos(n\phi_{\gamma\gamma})$ with $n=0,1,2$, and
$\sin(n\phi_{\gamma\gamma})$ with $n=1,2$; and 3) Measuring the $t$ dependence of each angular harmonic term. 

There are two important DVCS experiments in Hall-B using CLAS12: E12-06-119\cite{E12-06-119} at 11 GeV and E12-16-010\cite{E12-16-010B} at lower energies of 6.6 and 8.8 GeV. 
These measurements will allow a large kinematical coverage and hence a more comprehensive study of GPDs. With a longitudinally polarized beam, one can extract the chiral even GPDs $H(x,\xi)$ and $E(x,\xi)$. The cross-section will be used to separate the interference of pure DVCS squared amplitude contributions to each of the Fourier moments of the cross section. The $Q^2$ dependence will allow extraction of the subtraction constant in the dispersion relation. %Finally, the $\pi^0$ cross section will also be measured to separate the longitudinal and transverse contributions. 

{\it Flavor separation:} The CFF from experimental data is a sum of contributions 
from different quark flavor weighted with charged squared. To get interesting flavor singlet information, such as in the spin sum rule, it is necessary to perform the flavor separation of GPDs. Thus it is minimally necessary to consider DVCS on the neutron. In the proposed E12-11-003\cite{E12-11-003} experiment in Hall-B, the beam spin asymmetry will be measured for incoherent DVCS scattering on the deuteron with recoiling neutron detection. In an experiment running together with E12-06-113, a similar measurement will be made with additional spectator proton measurement. It is also potentially possible to search for DVCS events on polarized neutron with a polarized deuteron target through E12-06-109 experiment. 

{\it Transversely polarized target}: 
An experiment has been proposed to measure the
target single spin asymmetry on a transversely polarized
proton~\cite{C12-12-010}. The asymmetry has
particular sensitivity on the
GPD $E(x,\xi)$ which is related to the spin flip
nucleon matrix element and hence carries important information on the up an down quark orbital angular momentum. The expected asymmetries are in the range of 20 to 40\% in the kinematics covered by C12-12-010 experiment. In addition, the double spin 
asymmetry involving longitudinally polarized electron will also be studied. 

{\it DVCS on nuclei target}: An experiment, E12-17-012\cite{E12-17-012}, has been proposed to measure coherent DVCS and DVMP with emphasis  on $\phi$ meson production. These exclusive measurements 
allow comparing the quark and gluon radii of the helium nucleus. To isolate the reaction mechanism, the low-energy
recoil nuclei are detected. Moreover, through incoherent spectator-tagged DVCS on light nuclei, the GPDs of the bound proton and neutron can be measured and compared
with the free proton and quasi-free neutron. The new apparatus used to detector low-energy nuclei recoil 
is the aforementioned ALERT, composed of a stereo drift chamber for track reconstruction and an array of scintillators for particle identification. 

\subsubsection*{Time-like Compton Scattering}

%The significant part of the physics program of JLab12 is the description of the partonic structure of hadronic matter via the GPDs. The Compton scattering is the golden reaction for mapping GPDs in the longitudinal and transverse momentum space.

While the most attention so far is on studies of GPD using spin (beam/target) observables and cross-sections in DVCS, the two other Compton-like processes (Fig.~\ref{fig:tcsddvcs}), Time-like Compton Scattering (TCS) and DDVCS are accessible with high energy electron beams and have much to offer. 

\begin{figure}
\begin{center}
\includegraphics[width=0.35\textwidth]{Figures/tcs_diag.pdf}
\includegraphics[width=0.35\textwidth]{Figures/ddvcs_diag.pdf}
\caption{Time-like Compton Scattering in which GPDs can also be probed, and Double DVCS (right) allowing probing $\xi\ne\xi$. In` $J/\psi$ production, the final-state time-like virtual photon is replaced by a $J/\psi$.
}
\label{fig:tcsddvcs} 
\end{center}
\end{figure}
%A major scientific goal for future physics program with CEBAF would be to explore Double Deeply Virtual Compton Scattering (DDVCS). 

TCS is the time-reversal symmetric process of DVCS where the incoming photon is real and the outgoing photon has large time-like virtuality, see the left diagram of Fig.~\ref{fig:tcsddvcs}. The TCS hard scale is set by the virtuality $Q^{\prime 2}\equiv M^2_{l^+l^-}$ of the outgoing photon. As in the case of DVCS, the Bethe-Heitler process, $\gamma p\to p^\prime l^+l^-$, also contributes in the same final state. The formalism developed for TCS in \cite{Berger:2002tc} is based on the similar DVCS-type factorization when the TCS amplitude can be expressed as a convolution of the hard scattering kernels with GPDs appearing in CFFs.  With an unpolarized photon beam, TCS offers straightforward access to the real part of the CFFs through the interference between the Compton and Bethe-Heitler (BH) amplitudes. In the meantime, using circular photon polarization one can access the imaginary part of CCFs. 

Studies of the TCS process at JLab have already begun. The experiment using CLAS12 in Hall-B, E12-12-001\cite{E12-12-001}, acquired part of the expected statistics. In Fig.~\ref{fig:clas12tcs}, the first experimental results on TCS, $\gamma p\to e^+e^- p^\prime$, are presented from a subset of obtained data \cite{clas12tcs}. The squared momentum transferred, $-t$, dependence of the $\sin{\phi}$ modulation of photon beam polarization asymmetry  (the left graph of the figure) is reported in the range of timelike photon virtualities $2.25 < Q^{\prime 2} < 9$ (GeV/c)$^2$, and average total center-of-mass energy squared $s = 14.5$ GeV$^2$. The measured values are in agreement with the predictions of GPD-based models.  The $-t$ dependence of decay lepton angular forward-backward asymmetry $A_{FB}$ is shown on the right graph of Fig.~\ref{fig:clas12tcs}. The $A_{FB}$ is proportional to the real part of the Compton amplitude. The experimental data is better described by the model with the D-term (taken from Ref. \cite{PASQUINI2014133}). This observation validates the application of the GPD formalism to describe TCS data and hints at the universality of GPDs \cite{Mueller:2012sma, Grocholski:2019pqj}. The analysis of the full data set is in progress.
%TCS studies at JLab have already started.  Both the cosine and sine moments of the weighted cross section have been measured over a wider range of momentum transfer $-t$, for outgoing time-like photon virtuality up to 9 GeV$^2$. The analysis to extract both the real and imaginary parts of Compton amplitude has been completed, preparing the publication underway. %{\it In addition to $\cos(\phi)$ moment analysis to access $Re[\tilde{M}^{--}]$ as proposed in \cite{BERG}, the forward-backward asymmetry, initially proposed for J/$\psi$ near-threshold photoproduction studies in \cite{MARCV} has been employed. The forward-backward transformation corresponds to inverting the vectors of leptons in the center of mass frame of the pair. }

\begin{figure}[ht]
%\vskip 0.5truecm
\begin{center}
\includegraphics[width=.490\textwidth]{Figures/clas12_tcs_bsat.pdf}
\includegraphics[width=0.480\textwidth]{Figures/clas12_tcs_afbt.pdf}
\caption{Left: Photon polarization asymmetry $A_{\odot U}$ as a function of $-t$ at the averaged kinematic point $E_\gamma = 7.29 \pm 1.55$ GeV; $M = 1.80 \pm 0.26$ GeV. The data points are represented in red with statistical vertical error bars. The horizontal bars represent the bin widths. The shaded error bars show the total systematic uncertainty. The blue triangles show the asymmetry computed for simulated BH events. Right: FB asymmetry as a function of $-t$ in the same average kinematics. The plain line shows the model prediction for the VGG model with D-term (from \cite{PASQUINI2014133}) evaluated at the average kinematic point. The dashed and dashed-dotted lines are the predictions of, respectively, the VGG \cite{vgg1,vgg2,gprv,gmv} and the GK \cite{Goloskokov2005, Goloskokov2008, Goloskokov2009} models.}
\label{fig:clas12tcs}
\end{center}
\end{figure}


Besides CLAS12, plans are in place to study TCS using SoLID in Hall-A, LOI-12-13-001. The high luminosity of SoLID will make it possible to perform a mapping of the Q$^{\prime 2}$- and $\eta$-dependence, which is essential for understanding factorization, higher-twist effects, and Next-Leading-Order (NLO) corrections \cite{Pire:2011st}. 

%{\it Time-like Compton scattering}: An experiment E12-12-001 has been proposed to measure diffractive electron-positron pair production which allows studying time-like Compton scattering and $J/\psi$ photoproduction on the proton. Both the four-fold differential cross section and the consine and sine moments of the weighted cross seciton will be measured over a wider range of momentum transfer $-t$, for outgoing time-like photon virtuality up to 9 GeV$^2$. This type of process has never been studied before except for a test run at 6 GeV. The experimental results can first be compared and contrasted with the usual DVCS processes in which the virtual photon is space-like, to test the scattering mechanism. This so-called TCS processes have extra sensitivity on the real part of the Compton form factors which are quite different in different model predictions. 
\subsubsection*{Meson Production}

At high-energy, the leading contribution to DVMP is dominated by 
the longitudinally polarized photon. However, at JLab 6 GeV, the 
experimental data does not show the dominance of the quark 
scattering process. In fact, the sub-leading contribution from
transversely polarized photon is quite significant. There has been suggestion
for the important twist-3 GPD effect with helicity-flip GPDs. 
Experimental data from 11 GeV beam will provide important
test of the deep-exclusive meson production mechanism. Experiments
have been proposed for $\pi^0$ and $\eta$ production with CLAS12 \cite{E12-06-108}
running contemporaneously with DVCS \cite{E12-06-119} experiment. Additional beam
time has also been requested at 8 GeV to make $\sigma_L$ and $\sigma_T$
separation to test the dominance of the former. 
In addition, there is an experiment E12-12-007\cite{E12-12-007} to measure $\phi$
production. Differential cross sections and beam asymmetries will be measured
as a function of the $\phi\rightarrow KK$ decay angles, $\theta$ and $\phi$, 
to extract the various structure functions. It is hoped 
that $\phi$ production can be used to extract gluon GPDs to reveal
the transverse spatial distribution of gluons in the nucleon. However, there %is also 
may be intrinsic strangeness contribution which can
only be separated if one has information through models or lattice
QCD calculations.

The Hall-A E12-06-114 experiment will also generate high-precision $\pi^0$ production data. For Hall C, the E12-09-011~\cite{E12-09-011} will perform L/T separated on Kaon Electroproduction and E12-20-007~\cite{E12-20-007} will look into backward-angle exclusive $\pi^0$ production. A summary of additional progress and opportunities for measurements in the backward-angle kinematic regime can be found in Ref.~\cite{Gayoso:2021rzj}.

\subsubsection*{Threshold $J/\psi$ production and proton mass}
\label{sec:PMass}

Near threshold $J/\psi$ production can provide unique access to the gluon GPD and the form factors of the gluon EMT, providing important information on the mass structure of the nucleon. There are approved and planned experiments for TCS in Halls A and B, and $J/\psi$ production experiments in all experimental halls, but to advance with DDVCS measurement, new high luminosity facilities will be needed.

At JLab 12 GeV, the $J/\psi$ can be produced by photon and electron beams on the proton and nuclei targets near the threshold. Besides the interest of searching for Pentaquark resonances and $J/\psi$-nucleon nuclear interactions, the experimental data can be used to study quantum anomalous energy (QAE) contribution to the proton mass and the proton mass radius~\cite{Ji:2021mtz,Kharzeev:2021qkd,Mamo:2021krl}. The most rigorous approach to analyze the threshold production is to use a heavy-quark expansion to derive a factorization theorem in terms of gluon GPDs or proton wave functions~\cite{Guo:2021ibg, Sun:2021gmi,Sun:2021pyw}. Near the threshold, the skewness parameter $\xi\to 1$, and thus the form factors of the gluon EMT could dominate over other higher-spins. Therefore data at the threshold could allow an approximate extraction of gravitational form factors $A_g(t)$, $B_g(t)$, and $C_g(t)$ at large $t$. One can also extract the QAE contribution to the 
proton mass either through Vector Dominance assumption~\cite{Kou:2021bez} or through the matrix elements of the trace part of the EMT~\cite{Ji:2021mtz}. Likewise, the mass-radius can be either extracted through vector meson or Reggeon dominance~\cite{Kharzeev:2021qkd,Mamo:2021krl,Wang:2021dis} or again through the EMT form factors~\cite{Ji:2021mtz,Guo:2021ibg}. The theoretical assumptions can be verified and the uncertainties reduced if the measurements are done closer to the threshold, and augmented by polarized photoproduction and electroproduction measurements. 

\iffalse
\begin{figure}
\begin{center}
\includegraphics[width=0.9\textwidth]{Figures/total_tdist-xsec-zm.pdf}
\caption{Projected total threshold cross section (left) and differential cross section at function of $t$ at $E_\gamma\sim$ 9.5 GeV (right) for $J/\psi$ production with SoLID detector.}
\label{fig:crosssection} 
\end{center}
\end{figure}
\fi

The GlueX collaboration in Hall-D has recently published the first $J/\psi$ photoproduction data (about 470 events) ~\cite{Ali:2019lzf}, the results from a factor of 10 more data will be available soon. In Hall-C experiment E12-16-007~\cite{Meziani:2016lhg}, a similar amount of $J/\psi$ events (about 4k) were collected with a focus on the large $t$ region in search of the LHCb pentaquark, and with results to be published soon. Hall-B has ongoing experiments E12-12-001~\cite{E12-12-001} to measure TCS+$J/\psi$ in photo-production on a hydrogen target, which can have $\sim$10K events after their luminosity upgrade, and similarly E12-11-003B~\cite{E12-11-003B} utilizing a deuterium target. Hall-A has an approved experiment E12-12-006~\cite{SoLIDjpsi:proposal} using SoLID and can obtain at least another order of magnitude more events ($\sim$800k in photoproduction and $\sim$20k in electroproduction). With this large number of  threshold events, one can fit the cross section as a function of $W$ 
and $t$ %, Fig.~\ref{fig:crosssection},
to obtain all three gluon EMT form factors, and hence could shed important light on the origin of the nucleon mass. % (see Fig.~1). 
 Fig.~\ref{fig:mass} shows the possible precision one can obtain in the determination of the fractional contribution of the QAE in the proton mass following vector dominance in Ref.~\cite{Wang:2019mza} as well as the proton mechanical radius according to Ref.~\cite{Kharzeev:2021qkd} in comparison to its charge radius. A more accurate analysis is possible with QCD factorization in terms of GPDs.

\begin{figure}
\begin{center}
\includegraphics[width=0.48\textwidth]{Figures/b_par_solid_lattice.pdf}
\includegraphics[width=0.48\textwidth]{Figures/dipole_radius.pdf}
\caption{Statistical errors on the extraction of the quantum anomalous energy contribution to the proton mass and proton mass radius from SoLID data with the vector dominance assumption. More accurate analysis is possible with QCD factorization in terms
of GPDs.}
\label{fig:mass} 
\end{center}
\end{figure}

\subsection{Momentum Tomography of the Nucleon } 
\label{sec:MomTomNucleon}

Experimental explorations of the spin dependent observables in SIDIS, Drell-Yan (DY), $e^+e^-$, and hadron-hadron scattering spurred extensive theoretical and phenomenological studies opening a new era of exploration of the 3D partonic structure of the nucleon. In SIDIS, involving non-collinear dynamics the nucleon structure is defined by TMD-PDF and TMD fragmentation function (TMD-FF) commonly called TMDs. Together with rapid advances of precision of lattice QCD calculations~\cite{Hagler:2009mb,Musch:2009ku,Musch:2010ka,Ji:2013dva,Ji:2020ect} and detailed predictions from theory and phenomenology, new precise experimental data on TMDs are needed. 

In the kinematic region where TMD description of SIDIS is appropriate, namely in the beam fragmentation region, $P_{hT}/z \ll Q$, the transverse momentum of the produced hadron $P_{hT}$ is generated by intrinsic momenta of the parton in the nucleon ${\bf k}_T$ and the transverse momentum of the produced hadron with respect to the fragmenting parton ${\bf p}_T$, such that
the structure functions become convolutions of TMD PDFs and TMD-FFs. The convolution integral ${\cal C}[w f D]$, for a given combination of TMD-PDF $f$ and TMD-FF $D$ is defined as~\cite{Bacchetta:2006tn}
\begin{eqnarray}
\propto \sum_q e_q^2 \int d^2{\bf k}_T \, d^2{\bf p}_T \delta^{(2)}({\bf p}_T + z{\bf k}_T - {\bf P}_{hT}) w({\bf k}_T, {\bf p}_T) f^{q}(x, ~k_T^2) D^{q}(z,~P_{hT}^2), 
\label{Eq:convo}
\end{eqnarray}
where $w$ is a kinematical factor, and the sum goes over all flavors of quarks and anti-quarks.
Well known SIDIS structure functions $F_{UU,T}$ and $F_{LL}$ will be, thus, described by convolutions
of $f_1$ and $g_1$ TMD-PDFs and $D_1$ the unpolarized TMD-FFs, with $F_{UU,T} =  {\cal{C}}\left[ f_1 D_1 \right], F_{LL}  =  {\cal{C}}\left[ g_1 D_1 \right]$. %The full list of TMD PDFs accessible in SIDIS is given in Table.\ref{tab:TMD-tables}. The TMDs depend on polarization state of the quark (rows) and polarization state of the nucleon (columns). 
The leading and higher twist non-perturbative functions
describe various spin-spin and spin-orbit correlations as corresponding operators include additional gluon and/or quark fields in the matrix element.
While the higher twist are expected to be suppressed by $P_{hT}/Q$ with respect to leading twist asymmetries, at large $P_T$ and relatively low $Q^2$, where the spin-orbit correlation are large and measurable by SIDIS experiments, they may be very significant, and play important role in development of consistent framework for analysis of SIDIS data.

One of the most important questions about the 3D structure of the nucleon is the transverse momentum dependence of the distribution and fragmentation TMDs and flavour and spin dependence of those distributions. %{\xiaochao{Is this sentence out of place?}}
At JLab, three experimental halls, A, B, and C are involved in 3D structure studies through azimuthal modulations in SIDIS for different hadron types, targets, and polarizations in a broad kinematic range, including the High Momentum Spectrometer (HMS) and Super HMS at Hall C~\cite{E12-06-104,E12-09-017,E12-13-007}, the BigBite spectrometer and SBS\cite{E12-09-018}, as well as, the SoLID detector at Hall A~\cite{SoLID-SIDIS-p,SoLID-SIDIS-He3-T,SoLID-SIDIS-He3-L}, and CLAS12 at Hall B \cite{E12-06-112,E12-07-107,E12-09-008,E12-09-009,C12-20-002}. 



The most celebrated SIDIS measurements on TMDs are the surprising non-zero results of the Sivers asymmetries and the Collins asymmetries~\cite{Airapetian_2005, Adolph_2014, Qian_2011}. These initial explorations established the significance of the SIDIS-TMD experiments and attracted increasingly great efforts in both experimental and theoretical studies of TMDs. The planned SoLID experiments with transversely polarized proton and neutron/$^3$He targets~\cite{SoLID-SIDIS-p, SoLID-SIDIS-He3-T} will provide the most precise measurements of Sivers and Collins asymmetries of charged pion and Kaon productions in the valance quark (large-$x$) region in 4-dimension ($x$, $Q^2$, $z$ and P$_T$).
There will be over 1000 data points in the 4-D space with high precision with a polarized $^3$He target and hundreds of data points with a polarized ${\rm NH}_{3}$ target with SoLID. These data will allow precision extractions of the Sivers functions and transversity distributions through global analyses. 
Figure~\ref{fig:SoLID-transversity} (left panel) shows the projected precision of the extracted transversity $h_{1}(x)$ from the SoLID base configuration for both the $u$ and the $d$ quark flavor compared with the current knowledge from a global analysis of the world data~\cite{YE201791}.  
In addition to providing 3-D imaging in momentum space, the Sivers functions also contain information on the quark orbital angular momentum. The transversity distributions is one of the three leading twist colinear distributions when integrated over the transverse momentum. The other two are the well-known unpolarized distributions and the helicity distributions. The integration of the transversity over $x$ is the tensor
charge. Tensor charge is a fundamental property of the nucleon which has been precisely calculated with Lattice QCD. Precision determination of the tensor charge would provide a benchmark test of Lattice QCD calculations. Figure~\ref{fig:SoLID-transversity} (right panel) shows the projection of expected precision from SoLID measurements in determining the tensor charge along with Lattice QCD calculations. Also shown are other theory/model predictions and phenomenological determinations from current world data~\cite{YE201791}.


\begin{figure}[!h]
\begin{center}
%\includegraphics[width=0.5\textwidth,trim={15mm 0mm 10mm 0mm}, height=0.35\textwidth, clip]{Figures/SoLID-Collins.pdf}
%\includegraphics[width=.490\textwidth, height=0.35\textwidth]{Figures/SoLID-TensorCharge.pdf}
\includegraphics[width=\textwidth]{Figures/trans-and-tensor.pdf}
\caption{(Left panel) The projected precision of SoLID measurements~\cite{SoLID-SIDIS-p,SoLID-SIDIS-He3-T} of transversity $h_{1}(x)$ for $u$ and $d$ quark (red bands), together with results from global analysis of the world data (grey bands). (Right panel) The extracted tensor charge for $u$ and $d$ quark together with predictions from lattice QCD, models, and phenomenological analyses of world data~\cite{YE201791}.  
\label{fig:SoLID-transversity}}
\end{center}
\end{figure}
\iffalse
\begin{figure}
\begin{center}
\includegraphics[width=.490\textwidth]{SoLID-TensorCharge.pdf}
\caption{The projected precision of extraction of the tensor charge from SoLID measurements~\cite{SoLID-SIDIS-He3-T,SoLID-SIDIS-p} along with LQCD calculations, other theory/model predictions and phenomenological determinations from current world data.}
\end{center}
\label{fig:tensorcharge}
\end{figure}
\fi 

Studies of correlations of final state hadrons are crucial for understanding of the hadronization process in general, and the TMD FFs, in particular. First publication of CLAS12, dedicated to correlations in two hadron production in SIDIS~\cite{Hayward:2021psm}, revealed significant correlations between hadrons produced in the current fragmentation region. Significant single-spin asymmetry has been measured, which can be related to higher twist PDF $e$, interpreted in terms of
the average transverse forces acting on a quark after
it absorbs the virtual photon~\cite{Burkardt:2008vd}. The difference of error bars of 6 GeV and 12 GeV measurements, see Fig.~\ref{fig:clas12dih}, demonstrates the impact of the beam energy on the phase space for production of multiple hadrons in the final state.

\begin{figure}[ht]
%\vskip 0.5truecm
\begin{center}
\includegraphics[width=.490\textwidth]{Figures/clas12_2pi.pdf}
\includegraphics[width=0.480\textwidth]{Figures/mass_clas6.pdf}
\caption{The new CLAS12 results on beam helicity asymmetry in two-pion semi-inclusive deep inelastic electroproduction \cite{Hayward:2021psm}. The asymmetry as a function of $x$ (left) and the invariant mass of pion pairs (right). The red points on the right graph are from CLAS6 measurements \cite{Mirazita:2020lik}.}
\label{fig:clas12dih}
\end{center}
\end{figure}

Measurements of flavor asymmetries in sea quark distributions performed in DY experiments, indicate very significant non-perturbative effects at large Bjorken-$x$, where the valence quarks are relevant~\cite{Alberg:2017ijg}.
 The measurements by E866 collaboration~\cite{Garvey:2001yq}, and more recently by SeaQuest~\cite{Nagai:2017dhp} suggest that $\bar{d}$ is significantly larger than $\bar{u}$ in the full accessible $x$-range, where non-perturbative effects are measurable.
The non-perturbative $q\bar{q}$ pairs, most likely responsible for those differences, are also correlated with spins and play a crucial role in spin orbit correlations, and in particular, single-spin asymmetries measured by various experiments in last few decades.

Collinear PDFs have flavour dependence, thus it is not unexpected that also
the transverse momentum dependence may be different for the different
flavours~\cite{Signori:2013mda}. Model calculations of transverse momentum dependence of TMDs ~\cite{Pasquini:2008ax,Lu:2004au,Anselmino:2006yc,Bourrely:2010ng} and lattice QCD results~\cite{Hagler:2009mb,Musch:2010ka} suggest that the dependence of widths of TMDs on the quark polarization and flavor may be
significant. It was found, in particular, that the average
transverse momentum of antiquarks is considerably larger than that of
quarks~\cite{Wakamatsu:2009fn,Schweitzer:2012hh}. 


\subsection{Opportunities with high-intensity, positron and higher energy beams}

\subsubsection*{Double DVCS}
A major scientific goal for future physics program with CEBAF would be to explore DDVCS. 
As the subject of nuclear femtography has become a central focus of the 12 GeV science program, as well as the future EIC, DDVCS could be a major new frontier in the future. 
DVCS is primarily sensitive to the region near $x=\pm\xi$ or integrals of GPDs over internal $x$ due to the loop in the ``handbag'' diagram, and thus
cannot determine the GPDs uniquely.
The study of DDVCS enables more complete coverage of the $x-\xi$ plane giving wider access to the GPDs. 
The DDVCS process, when both the incoming and outgoing photons have large virtualities, allows for mapping the GPD's along each of the three axes ($x$, $\xi$, and $t$) as the three variables can now be varied independently. 
%The $\xi$ dependence of GPDs contains unique information about the distribution of nuclear forces and also provides a constraint on GPD models in the establishment of the total spin sum rule. 

While DDVCS has a huge advantage for mapping the GPDs, experimentally it is a challenging reaction to study.  The cross section of DDVCS is significantly smaller (more than two orders of magnitude) than that of DVCS. 
 To eliminate ambiguity and anti-symmetrization issues, the outgoing time-like photon must be reconstructed through their di-muon decays. These two conditions require a large acceptance detector capable of running at very high luminosities, $10^{37} -10^{38}$  cm$^{-2}$ s$^{-1}$, with good muon detection. %for the DDVCS studies. 

\iffalse
\begin{figure}
\begin{center}
\includegraphics[width=0.49\textwidth]{xs_dvcs_ddvcs_6gev.pdf}
\includegraphics[width=0.47\textwidth]{bsa_ddvcs.pdf}
\caption{On the left differential cross sections for DVCS+BH and DDVCS+BH processes. On the right the beam spin asymmetry for the reaction $ep\to e^\prime p\mu^+\mu^-$ (DDVCS+BH) for different virtualities of the lepton pair : Q$^2$ = 0 GeV$^2$ (thick solid line), which corresponds to a DVCS calculation, 1.5 GeV$^2$ (thick dashed line), 2 GeV$^2$ (thick dash-dotted line), 2.8 GeV$^2$ (dotted line) 3.6 GeV$^2$ (thin solid line) and 4.4 GeV$^2$ (thin dashed line). Calculations and predictions from [M. Guidal and M. Vanderhaeghen. Phys. Rev. Lett. 90 (2003) 012001]. .}
\label{fig:xsddvcs} 
\end{center}
\end{figure}
\fi

The possibility of extending studies of DVCS to DDVCS with upgraded detectors in Halls A (SoLID) and B ($\mu$-CLAS12) were discussed in letter-of-intents to PAC43 (LOI12-15-005) and PAC44 (LOI12-16-004). The SoLID DDVCS studies will take advantage of the setup proposed for the already approved J/$\psi$ experiment (E12-12-006). After adding muon detectors in forward angles, a significant set of experimental data will be collected at the luminosity of few $\times 10^{37}$ cm$^{-2}$ s$^{-1}$ to study di-muon production in deeply virtual electron scattering in the limited phase space of $Q^2>$M$_{\mu\mu}^2$. With a dedicated setup, target inside the solenoid, additional iron shielding in front of the calorimeter, GEM trackers, and muon detectors, reaction $ep\to e^\prime p^\prime \mu^+ \mu^-$ can be studied with a luminosity of $10^{38}$ cm$^{-2}$ s$^{-1}$. 
%{\it At these luminosities, studies of both space-like ($Q^2>$M$_{\mu\mu}^2\equiv Q^{\prime 2}$) and time-like ($Q^2<Q^{\prime 2}$) regimes will be possible. }

The Hall-B DDVCS setup will use the CLAS12 forward detector for muon detection and a lead tungsten calorimeter that will replace the high threshold Cherenkov counter for the detection of scattered electrons. A recoil detector inside the CLAS12 solenoid magnet for proton detection and the vertex tracker in front of the calorimeter/shield is part of the conversion of CLAS12 to $\mu$-CLAS12. It is expected that the detector will run at luminosities of $> 10^{37}$ cm$^{-2}$ s$^{-1}$. With this proposed luminosity, $\mu$-CLAS12 will be able to study DDVCS in both spacelike and timelike regions, producing high-quality results with wide kinematical coverage. 

With these upgrades to the experimental equipment in Halls A and B, SoLID and CLAS12, respectively, a DDVCS program can be realized and the GPD program at JLab can be taken to a unique new level, accessible only with large acceptance, high luminosity detectors.

\subsubsection*{DVCS with positron beam} 

An impressive amount of high-precision data start to come out and will be produced in the next several years from the experimental program developed at JLab. 
%Fine details of the inner structure of matter will be explored in measurements of cross-sections and polarization observables using unpolarized and polarized beams and targets. 
Often, however, interpretation of experimental results and extracting the information on hadron structure requires model-dependent assumption and calculations. Experimentally we measure reactions that are composed of multiple interfering elementary lepton-hadron scattering mechanisms. 
While there is no lepton charge sign difference in the obtained physics information, the comparison between electron and positron scattering, for example, allows separation of contributions of different components in the scattering process and isolates them in a model-independent way. Combining measurements with polarized electrons and polarized positrons in the elastic and DIS regime allows obtaining unique experimental observables enabling a more accurate and refined interpretation. Therefore, positron beams, both polarized and unpolarized, are essential for the experimental study of the structure of hadronic matter carried out at JLab.


The perspectives of an experimental program with positron beams at JLab are discussed in the ``$e^+$@JLab White Paper"~\cite{Accardi:2020swt} with a follow-up letter-of-intent to JLab PAC46 (LOI12-18-004~\cite{LOI12-18-004}) outlining prospects of experiments using unpolarized and polarized positron beams at CEBAF. The outlined experimental program with positron beams will contribute in a major way to the high-impact 12 GeV physics program: the physics of the two-photon exchange, the determination of GPDs, and tests of the Standard Model.




%There have been significant efforts devoted to extracting GPDs from the DVCS observables. However, with the single lepton charge measurements, separation of the reaction amplitudes cannot be done in a model-independent way. A possible approach to this separation with a single change beam is to exploit different beam energy sensitivity of DVCS and interference amplitudes [E12-13-010, E12-16-010B]. Of particular interest is the real part of the DVCS amplitude that contains the D-term [Pol99] which parameterizes the Gravitational Form Factors of the nucleon [[Pol03, Bur18, Pol18,Kum19]. Without direct measurement, the dispersion relation has been employed to determine the real part in leading order approximation [Ani07, Die07, Pol08]. However, as has been shown in recent studies, there are significant power corrections that cannot be neglected in precision DVCS phenomenology.  The only way to determine the D-term without model assumptions is to measure both the real and imaginary parts of the CFF independently. 

The combination of DVCS measurements with oppositely charged incident beams is the only unambiguous way to disentangle the contribution of different terms to the cross-section. The BH amplitude is electron charge even while the DVCS amplitude is electric charge odd. The DVCS contribution to the cross-section has a different sign for electron vs. positron scattering. Using various combinations of the difference and sum of the measured helicity ($\lambda$) and the lepton charge ($\pm$)-dependent cross-sections, one can project out the real and imaginary parts of Compton amplitudes:
\begin{eqnarray}
\sigma_\lambda^{\pm}=\sigma_{BH}+\sigma_{DVCS}+\lambda\tilde{\sigma}_{DVCS}\pm(\sigma_{INT}+\lambda\tilde{\sigma}_{INT})
\end{eqnarray}
Here $\sigma_{DVCS}$ and $\sigma_{INT}$ related to the real part of the Compton form-factor, while $\tilde{\sigma}_{DVCS}$ and $\tilde{\sigma}_{INT}$ to its imaginary part. In particular, the difference of the unpolarized cross-sections will single out $\sigma_{INT}$ that contains the real part of CFF. While the helicity-dependent charge asymmetry measures the ${\tilde \sigma}_{INT}$ and last but not least, the lepton charge integrated helicity difference measures the imaginary part of the DVCS amplitude (${\tilde \sigma}_{DVCS}$). 
%
%\begin{eqnarray}
%\Delta^C_{UU}&=&(\sigma_+^+ + %\sigma_-^+)-(\sigma_+^- + \sigma_-^-)=\sigma_{INT} \\
%\Delta^C_{LU}&=&(\sigma_+^+ - \sigma_-^+)-(\sigma_+^- - \sigma_-^-)=\lambda\tilde{\sigma}_{INT} \\
%\Delta^0_{LU}&=&(\sigma_+^++\sigma_+^-)-(\sigma_-^++\sigma_-^-)=\lambda\tilde{\sigma}_{DVCS}
%\end{eqnarray}

There were two proposals submitted to JLab PAC48 in 2020, receiving conditional approval to motivate further studies. In PR12-20-009 \cite{C12-20-009} it is proposed to use CLAS12 in Hall B to measure the unpolarized and polarized Beam Charge Asymmetries (BCAs) in DVCS on unpolarized hydrogen. The second proposal, PR12-20-012 \cite{C12-20-012}, would use the HMS of Hall C together with the Neutral Particle Spectrometer (NPS) for BCA measurement. 
%In Fig.~\ref{fig:cff}, extraction of CFFs when both positron and electron data are used compared with fits when only electron data are included. 
A significant reduction of extraction errors is expected when both positron and electron data are used compared with fits when only electron data are included. %with the combined data. 

\iffalse
\begin{figure}
\begin{center}
\includegraphics[width=0.49\textwidth]{Figures/hallc_pos_el_gpdH.pdf}
\end{center}
\caption{Figure from PR12-20-012: extraction of CFFs from fits to the experimental measurements. The first column in the left shows the
results of the helicity-conserving CFFs when both positron and electron data are used in the fit
(dark symbols), and when only the electron approved data is used (open symbols). The second and third columns
show the same information for the helicity-flip CFFs. The solid horizontal lines indicate the input
values used to generate the cross-section data.}
\label{fig:cff}
\end{figure}
\fi 

\subsubsection*{Opportunities at higher energy}

JLab at 24 GeV will open up a big phase space for studying the structure of protons and neutrons, particularly for 3D nuclear femtography both
in coordinate and momentum spaces.
%
A 20-24 GeV beam energy presents several crucial advantages for the extraction of GPDs from data. First of all, the proposed upgrade of the current JLab settings features a guaranteed high luminosity which is strictly necessary to extract information from deeply virtual exclusive experiments. Furthermore, it will sensibly expand the accessible $Q^2$ range thus allowing thorough tests of both factorization and of the perturbative QCD behavior of GPDs in the regime $x \geq 0.05$. On the other hand, because of the sharp falloff of the BH cross section relative to the DVCS, shown in Fig.~\ref{fig:DVCS/BH},
one will be able to increase the precision  of the CFF extraction from experimental data in the $Q^2$ range already available in the 12 GeV program. A precise knowledge of the latter is, in turn, crucial for proton 3D spatial imaging. High electron energy is particularly important for meson production for which the asymptotic scaling needs larger $Q^2$. 

\begin{figure}
\centerline{
\includegraphics[width=0.48\textwidth]{Figures/Jlab24GeV.png}
\includegraphics[width=0.48\textwidth]{Figures/xsectionspt3.png}
}
\caption{(Left) Ratio of the DVCS plus DVCS-BH interference cross section over the BH cross section in unpolarized $ep \rightarrow e'p' \gamma$, plotted as a function of the electron beam energy, $E_e$. The kinematics is defined by the azimuthal angle, $\phi=180^\circ$, Bjorken $x=0.36$, four-momentum transfer square, $t=-0.2$ GeV$^2$; the four values of the initial photon four momentum squared, $Q^2 =2, 4, 6, 10$ (GeV/c)$^2$, respectively, are indicated on the right hand side of the curves. All cross sections were calculated using the formalism in Ref.~\cite{Kriesten:2019jep}, using the GPD model in Refs.~\cite{Goldstein:2010gu,Kriesten:2021sqc}. 
(Right) The unpolarized cross section for the kinematic bins from Ref.~\cite{Georges:2018kyi} with initial electron energy $E_1=$24 GeV, $Q^2= 4.5$ GeV$^2$, $t= -0.26$ GeV$^2$, $x = 0.37$; The curves correspond to contributions from the BH (purple), DVCS (red), DVCS-BH interference (green) and total (blue).}
\label{fig:DVCS/BH}
\end{figure}


%DVCS \& meson production: 

%\begin{figure}[ht!]
%\begin{center}
%\vspace{0.0cm}
%\includegraphics[width=0.38\textwidth]{Figures/JLAB25-q2.png}
%\includegraphics[width=0.3\textwidth]{Figures/JLab25-xdep.png}
%\includegraphics[width=0.3\textwidth]{Figures/JLab25-pt.png}
%\end{center}
%\caption{\small Kinematical coverage of JLab for beam energies 12, 24 and 50 GeV. Events for fixed kinematic are plotted in $Q^2$ (left panel), $x_B$ (middle panel), and $P_T^2$ (right panel)}
%\label{jlab12-25}
%\end{figure}

SIDIS program at Halls A, B, and C at JLab12 \cite{E12-06-104}-- \cite{C12-20-002} with high luminosity and large acceptance detectors will provide high precision data in multi-dimensions in the valance quark (high-$x$) region to study TMDs. With the energies available, the kinematical reach of JLab12 is limited to the kinematic region of high-$x$, but low-Q$^2$ and low-P$_T$. Recent theoretical developments in TMD physics show that while this region is sensitive to the 3D nucleon structure in momentum space, non-perturbative effects could complicate the interpretation in terms of pure TMD distributions ~\cite{Scimemi:2019cmh}. TMD Studies planned for the future EIC with much higher center-of-mass (CM) energies would allow us to reach lower x, higher $Q^2$, and higher P$_T$ region, and have a cleaner theoretical interpretation for the extracted TMDs. But the much higher CM energy also means the extracted quark TMDs often being smeared by significant gluon showers/gluon radiations. The effects, such as Sivers single-spin asymmetry will also reduce at large $Q^2$. There is an energy gap between JLab12 and EIC where the “sweet region” for TMD studies lies. An energy upgrade of JLab to 24 GeV (JLab24) would fill a significant part of this energy gap to reach the “sweet region”. High luminosity and large acceptance detectors of JLab24 would allow us to access significantly higher P$_T$ and Q$^2$ range for high-$x$ where the TMD effects are still large while theory interpretation is better defined. Combining with JLab12 and EIC, JLab24 would allow for validation and extension to wider kinematics of the studies of transverse momentum dependence of various TMD distributions and fragmentation functions as well as the transition from TMD regime (P$_T$/$z\ll Q$) to the collinear perturbative regime (P$_T$/$z\sim Q$). A much higher Q$^2$ and P$_T$ range accessible at JLab24 at high-$x$ would allow for studies of $Q^2$-dependence of different spin-azimuthal asymmetries. Apart from providing important information on quark-gluon correlations, it will help validate the TMD evolution and help in building a more consistent TMD theory, extending its reach to higher P$_T$ of hadrons accessible by polarized SIDIS experiments. Extension of JLab12 kinematics to a lower-$x$ region with JLab24 will open new avenues for studies of non-perturbative sea effects.

%PDFs and FFs: Yes, this will also have significant improvement. With a possible future JLab~24~GeV energy upgrade, the kinematic reach of DIS and SIDIS measurements would increase significantly.
%It will also allow us to fully utilize SIDIS processes for flavor separation, especially for the precision study of the strange quark spin distributions.
