\section{QCD and Nuclei}

It has been nearly four decades since the European Muon Collaboration (EMC) published an astonishing finding on how the nucleon PDFs are strongly modified in the medium of the iron nucleus~\cite{EuropeanMuon:1983wih}. Many experiments have been done since then which both confirm this ``EMC Effect'' and have extended it to other nuclei, establishing systematic dependencies on nuclear size and density. However, although some recent studies suggest a connection to short-range correlations (SRCs) in nuclei, a full understanding of this phenomenon is still lacking.

Indeed, there are several ways in which QCD manifests itself in complex nuclei. CEBAF has contributed to this area of study, and will continue to provide new experimental data to point the way to a complete comprehension. Electron scattering gives access to a range of unique aspects of nuclear structure, providing important data relevant to nucleon modification in the nuclear medium, and quark/hadron interactions in cold QCD matter. Precision measurements of nuclear elastic, quasielastic, and inelastic scattering, especially those associated with the high-momentum part of the nucleon distributions, provide critical nuclear structure information needed in a range of other areas of nuclear and high-energy physics. Such data are needed as inputs to measurements of neutrino-nucleus scattering, nuclear astrophysics, lepton-nucleus scattering, and heavy-ion collisions, as well as providing important constraints relevant to the modeling of neutron stars. 

Studies of the partonic structure of nuclei provide insights into the impact of the dense nuclear medium on the structure of protons and neutrons and will allow, for the first time, imaging of the nuclear gluon distribution. In addition, measurements at higher energy allow for studies of hadron formation over a wide range of kinematics, as well as detailed studies of quark and hadron interaction with cold, dense nuclear matter, including color transparency studies which attempt to isolate the interaction of small-sized ``pre-hadronic" quark configurations.

The following sections summarize recent insight and highlights key future programs of measurements in nuclei, examining first measurements which probe nuclear structure at extremes of nucleon momentum, studies of the impact of the dense nuclear environment on the structure of the nucleon, and finally the use of the nucleus to study the interaction of quarks and their formation of hadrons in cold nuclear matter.



\subsection{Nuclear structure at the extremes}

Low-energy probes of nuclear structure are limited in their ability to isolate the high-momentum components of the nuclear momentum distribution. Measurements made as part of the 6 GeV program demonstrated the ability to probe these distributions with minimal corrections from final-state interactions, which limited electron scattering measurements at lower energy~\cite{Arrington:2011xs}. It also demonstrated the importance of understanding the isospin structure of SRCs which generate the high-momentum part of the momentum distribution~\cite{Arrington:2011xs,Hen:2016kwk,Fomin:2017ydn}.

At 12 GeV, our improved understanding of final-state interactions (FSI) and the expanded kinematic range available will allow for better constraints on the nucleon momentum distributions at high momentum, as well as providing reference data needed to further constrain meson-exchange and FSI corrections. In addition, targeted measurements will provide important inputs for nuclear structure relevant to neutrino oscillation experiments and nuclear astrophysics.


\subsubsection{Deuteron (and few-body nuclei) measurements at high momentum}

In the plane-wave impulse approximation, quasielastic A(e,e'p) measurements can be used to probe the spectral function of nuclei. Accessing the high-momentum part of the spectral function has proved extremely challenging, due to large FSIs that shift strength from low reconstructed momenta to large values~\cite{Sargsian:2002wc}. However, JLab measurements on the deuteron~\cite{CLAS:2007tee, HallA:2011gjn} have been used to constrain FSI  models~\cite{Sargsian:2009hf} over a wide range of kinematics, validating the models and demonstrating that for modest values of $\theta_r$ -- the recoil angle of the spectator nucleon relative to the incoming photon -- the corrections become small~\cite{CLAS:2007tee,HallA:2011gjn,Arrington:2011xs}. A new set of measurements was proposed to push deuteron measurements to extremely large nucleon momenta, exceeding 1~GeV/c~\cite{E12-10-003}. The low-momentum piece of this experiment was completed and results from the initial data set are available~\cite{HallC:2020kdm}. They show that above 700 MeV/c missing momentum, the data deviate from all existing calculations. Future data taking will emphasize improving the statistics and high-momentum coverage to provide more precise constraints for detailed calculations to have a more reliable probe of the high-momentum part of the Nucleon-Nucleon (NN) potential. Other experiments will extend this approach to study the spectral function of light nuclei~\cite{E12-20-005}
while CLAS12 measurements of D$(e,e'p)$ at 12 GeV will allow for additional studies of FSI, significantly extending the $Q^2$ and missing momentum coverage over the full range of recoil angles.

\subsubsection{Short range NN correlations}

While mapping out the high missing momentum ($P_m$) part of the spectral function presents challenges, it is still possible to study SRCs by selecting kinematics where the reaction is dominated by SRC contributions. Nucleons with momenta well above the Fermi momentum are associated with SRCs that are generated by the hard, short-range components of the NN interaction~\cite{Frankfurt:1981mk, Frankfurt:1988nt, Sargsian:2002wc, Arrington:2011xs, Fomin:2017ydn, Arrington:2022sov}. Because they are generated by two-body interactions, they have a universal structure that comes from the NN interaction, and scattering measurements in kinematics dominated by SRCs allow for studies of the nature and size of SRCs in nuclei. Inclusive scattering at modest $Q^2$ and $x>1.4$, where scattering from low-momentum nucleons is kinematically forbidden, provides sensitivity to the relative contribution of SRCs as a function of the mass number  $A$ via measurements of the inclusive $A/^2$H cross section ratios. During 6 GeV running, experiments confirmed the initial observation of SRCs~\cite{Frankfurt:1993sp} and mapped out the $A$ dependence of SRCs in light and heavy nuclei~\cite{CLAS:2003eih, Fomin:2011ng}. These data demonstrated that the contribution is sensitive to details of the nuclear structure~\cite{Seely:2009gt, Arrington:2021vuu} rather than the previously assumed average nuclear density~\cite{Gomez:1993ri}. In addition, they showed a clear correlation between the contribution of SRCs~\cite{Fomin:2011ng} and the size of the EMC effect~\cite{Seely:2009gt}, discussed further in Sec.~\ref{sec:4:emc}. Measurements of two-nucleon knockout, where both nucleons from the SRC are observed in the final state, showed dominance of np-SRC as well as a dependence of the np/pp SRC ratio as a function of the struck nucleon's momentum (shown in Figure~\ref{fig:SRC-isospin}. The dominance of np-SRCs was confirmed in additional nuclei using the  $A(e,e'p)$~\cite{CLAS:2018yvt} reaction and later through inclusive measurements taking advantage of the \textit{target} isospin structure in measurements of the $^{48}$Ca/$^{40}$Ca cross section ratio~\cite{JeffersonLabHallA:2020wrr}. Finally, measurements at $x>2$ tried to establish the presence of three-nucleon SRCs (3N-SRCs), but low-$Q^2$ measurements did not observe 3N-SRC dominance~\cite{HallA:2017ivm}, while higher-$Q^2$ data was consistent with 3N-SRC dominance but had extremely limited statistics~\cite{Fomin:2011ng}. 

\begin{figure}[htb]
\centerline{
\includegraphics[height=0.56\textwidth, angle=90, trim={20mm 22mm 25mm 40mm}, clip]{Figures/np-to-pp.pdf}
\includegraphics[width=0.43\textwidth, trim={0mm 0mm 0mm 0mm}, clip]{Figures/fig4.png}
}
\caption{(Left) Ratio of np-SRC to pp-SRCs from triple-coincidence A(e,e'pN) measurements: Blue `x'~\cite{Subedi:2008zz}, red open circle~\cite{LabHallA:2014wqo}, black solid circles~\cite{CLAS:2018xvc}; the horizontal line is the average of all measurements (np/pp = 30.0$\pm$3.8), although the value from~\cite{Subedi:2008zz} would be 50-100\% larger using the updated FSI of Ref.~\cite{CLAS:2018xvc}. (Right) Fraction of high missing momentum (e,e'p) events with a high-momentum neutron or proton spectator, and the ratio of pp to pn SRC; figure taken from Ref.~\cite{LabHallA:2014wqo}.}
\label{fig:SRC-isospin}
\end{figure}

An upcoming measurement~\cite{E12-06-105} will measure SRCs for additional few-body nuclei (all stable nuclei up to $^{12}$C), as well as for medium-to-heavy nuclei over a range in $N/Z$. This will allow more detailed examinations of how the EMC effect is impacted by nuclear structure in well-understood nuclei, as well as allowing for separation of mass- and isospin-dependent effects.  It will also provide the first meaningful significant test of the predicted dominance of 3N-SRCs by taking high-statistics A/$^3$He ratio data at $x>2$ and $Q^2 \approx 3$~GeV$^2$. If the presence of 3N-SRCs is established, further dedicated measurements could map out their A dependence and isospin structure through expanded A/$^3$He measurements and direct $^3$H-$^3$He comparisons~\cite{LOI12-21-001}. 



\subsection{Impact on neutron stars, neutrino scattering} \label{sec:PREX}

The studies of SRCs completed so far, as well as those planned for the future, have provided significant information on details of nuclear structure that are difficult to cleanly probe in low-energy reactions, and hold the promise to provide more information on the short-range structure of the NN interaction. This information has impact on a range of other physics: neutrino-nucleus scattering measurements, including those needed to study neutrino oscillations, need reliable nuclear structure inputs including a quantitative understanding of SRCs~\cite{Kulagin:2007ju, Niewczas:2015iea, VanCuyck:2016fab}. An understanding of the high-momentum components of the nuclear wavefunction, including the isospin structure, is also important for understanding the quark distributions of nuclei~\cite{Kulagin:2004ie} and the structure of neutron stars~\cite{Frankfurt:2008zv, Higinbotham:2009zz}. In addition to the impact of SRC on these topics, there have also been dedicated measurements aimed at providing input relevant for understanding aspects of nuclear structure directly relevant to neutron stars and neutrino scattering measurements.

The PREX-II collaboration recently published a new measurement~\cite{PREX:2021umo} of the ``neutron skin thickness'' of $^{208}$Pb using PVES. Parity violation is an especially clean way to measure nuclear neutron densities~\cite{Horowitz:1999fk,Thiel:2019tkm}, owing to the fact that the $Z^0$ couples much more strongly to neutrons than to protons. This makes it particularly straightforward to relate such measurements to the neutron matter equation of state.

The PREX-II measurement required the full integration of the CEBAF accelerator datastream into the data pipeline so that a small scattering asymmetry could be measured with high precision. The final result of the parity-violating asymmetry of $e-^{208}$Pb elastic scattering, $A_{\rm PV}=[550\pm16~({\rm stat})\pm8~({\rm syst})] \times 10^{-9}$, yields a neutron skin thickness $R_n-R_p=0.283\pm0.071$~fm. Figure~\ref{fig:PREXII} shows an interpretation~\cite{Reed:2021nqk} of the PREX-II result in terms of the ``symmetry pressure'' $L$ of pure neutron matter. One finds that $L$ is significantly larger than predictions from most available calculations, suggesting important modifications to neutron matter and the equation of state for neutron stars~\cite{Horowitz:2001ya}.
%
A similar measurement aiming at an extraction of the neutron skin thickness of $^{48}$Ca, called CREX, has also been carried out. The complete analysis %, including interpretation of the neutron skin thickness, 
is currently in progress. For CREX, the extracted neutron skin can be directly compared to microscopic calculations~\cite{Hagen:2015yea}.

\begin{figure}
\centerline{
\includegraphics[width=0.85\textwidth]{Figures/ReedHorowitzPREXIIFigure.pdf}}
\caption{(a) Slope of the symmetry energy at nuclear
saturation density $\rho_0$ (blue upper line) and at $(2/3)\rho_0$ (green
lower line) as a function of the neutron skin thickness in $^{208}$Pb. The correlation coefficients are indicated. (b) Gaussian probability distribution for the slope of the symmetry energy $L=L(\rho_0)$ inferred by the PREX result. The six error bars are constraints on $L$ obtained by using different theoretical approaches.
Reprinted Figure~1 with permission from Brendan T. Reed, F.J. Fattoyev, C.J. Horowitz, and J. Piekarewicz, Physical Review Letters, 126, 172503, 2021.~\cite{Reed:2021nqk} Copyright 2021 by the American Physical Society.
\label{fig:PREXII}}
\end{figure}

The extraction of neutrino mixing parameters from neutrino oscillation experiments relies on the reconstruction of the incident neutrino energy and knowledge of the neutrino-nucleus interaction cross-section for various nuclei and incident neutrino energies. The energy reconstruction of the incident neutrino is often done using the yield and kinematics of particles produced from neutrino interactions in nuclei. However, none of these energy reconstruction techniques have been tested experimentally using beams of known energy. Understanding these issues is critical to the ongoing global effort to study neutrino oscillations with secondary neutrino beams and nuclear targets.

One of the early experiments of the 12 GeV eras was focused on constraining scattering from $^{40}$Ar through measurements of inclusive quasielastic scattering and single proton knockout~\cite{Dai:2018gch,JeffersonLabHallA:2020rcp}. These data will allow for tests of $\nu-^{40}$Ar scattering simulations needed for the DUNE experiment.

Because neutrinos and electrons are both leptons, they interact with nuclei in
similar ways. The ``Electrons for Neutrinos" experiment proposes to measure electron scattering from a variety of targets at a range of beam energies in CLAS12 in order to test neutrino event selection and energy reconstruction techniques and to benchmark neutrino event generators. Recently the collaboration has studied these aspects through a data-mining project using previous CLAS data from the 6 GeV era of CEBAF~\cite{e4nu}.



\subsection{Nucleon modification/Nucleons under extreme conditions}\label{sec:4:emc}

The SRCs discussed in the previous sections represent short-distance, high-density configurations of nucleon in nuclei, as well as extremely high-energy nucleons.  It is natural to wonder whether nucleons with significant overlap and with extremely large virtuality might have a modified internal structure which could contribute to the EMC effect -- the modification of nuclear PDFs relative to the sum of the individual nucleon PDFs and, beyond this, whether measurements can be made that might be directly sensitive to the internal structure of these short-distance, high-momentum nucleons. 


\subsubsection{EMC effect: Nuclear parton distributions}

%\paragraph{Overview} 
\label{sec:ALERTmeasurements}

Almost 40 years after the observation by the EMC collaboration~\cite{EuropeanMuon:1983wih} that the nuclear PDF has significant modifications from the sum of proton and neutron PDFs, this observation still provides the cleanest indication that the nucleus cannot be described as a simple collection of bound, moving nucleons. Most calculations suggest that quark contributions beyond those of moving nucleons are required to explain the nuclear PDFs~\cite{Geesaman:1995yd, Arrington:2011xs, Malace:2014uea}, and there is a growing consensus that modification of the internal structure of the nucleon is required to fully explain the effect~\cite{Miller:2001tg, Sargsian:2002wc, Hen:2016kwk}. However, the most precise measurements of the EMC effect focused on medium-to-heavy nuclei~\cite{Gomez:1993ri}, and many models were able to reproduce the universal $x$ dependence and the weak $A$ dependence observed in medium-to-heavy nuclei, making it difficult to evaluate the underlying physics. The 6 GeV program at JLab added precise measurements in light nuclei~\cite{Seely:2009gt}, which showed an anomalous behavior in $^9$Be, demonstrating that the EMC effect is sensitive to nuclear structure details and does not simply scale with average nuclear density or mass~\cite{Arrington:2012ax}. The behavior in light nuclei was nearly identical to that observed in measurements of SRCs in few-body nuclei~\cite{Fomin:2011ng}, demonstrating a connection between these two phenomena~\cite{Arrington:2012ax, Hen:2016kwk} that is not yet understood~\cite{CLAS:2019vsb, Arrington:2019wky}. 
Extension of EMC effect measurements to cover additional few-body and medium-to-heavy nuclei combined with SRC measurements in the same nuclei~\cite{E12-06-105} will provide significantly better quantification of the EMC-SRC connection. It will also provide data for heavier nuclei with a range of $N/Z$ ratios, that can help separate the mass and flavor dependence of the EMC effect. Understanding the flavor dependence of the EMC effect would provide completely new inputs to test models of nuclear parton distributions. It will also help elucidate the EMC-SRC connection, as pictures where the EMC effect is caused by the presence of highly-virtual nucleons would be expected to have a flavor dependence arising from the isospin structure of SRCs, while models where they arise from a common origin, e.g. short-distance (highly-overlapping) nucleon pairs may not translate into a flavor dependence of the nuclear quark modification~\cite{Arrington:2012ax, Arrington:2019wky}.

After many decades of study it has become clear that in order to make significant further progress in understanding the origin of the EMC effect, it will require new experimental measurements that can be used to disentangle the various physical explanations. As discussed below, measurement of the spin and flavor dependence of the EMC effect would provide significant new information that could go a long way in solving the puzzles associated with the partonic structure of nuclei. 


%Since t
The EMC effect and the proton spin puzzle were both discovered almost 40 years ago, and numerous experiments have been performed in the intervening decades. It is perhaps surprising that there is still no experimental data on the spin structure functions of nuclei beyond the deuteron and $^3$He. Nevertheless, there are several theoretical calculations of the polarized EMC effect in light nuclei and nuclear matter~\cite{Cloet:2005rt, Cloet:2006bq, Smith:2005ra, Tronchin:2018mvu} that predict modifications in the spin structure functions of nuclei at least as large as the usual EMC effect. The polarized EMC effect is defined by $\Delta R = g_A(x)/[P_p\,g_{1}^p(x) + P_n\,g_{1}^n(x)]$ where $g_A(x)$ is a nuclear spin structure function, $g_{1}^p(x)$ and $g_{1}^n(x)$ are %in Section 2.2 (and common literature), p and n are superscripts
the familiar nucleon results, and $P_{p/n}$ is the effective polarization of the nucleus carried by the protons/neutrons which can be obtained from detailed nuclear structure calculations. A prediction for the polarized EMC effect in $^7$Li is illustrated in the left panel of Fig.~\ref{fig:emc_effects}, and a JLab 12\,GeV experiment is planned to measure this effect~\cite{E12-14-001}. The polarized EMC effect is particularly interesting because it can distinguish between mean-field models of the EMC effect where all nucleons are modified by the nuclear medium, with those explanations based on SRCs where only those nucleons involved in these correlations are modified. The mean-field models predict a large polarized EMC effect, whereas approaches based on SRCs lead to a very small effect %EMC effect in the spin structure functions
because the nucleons in SRCs carry only a few percent of the nuclear polarization~\cite{Pudliner:1995wk}. Similarly, the nature of the SRCs is such that they depolarize the nucleons involved~\cite{Thomas:2018kcx}.

\begin{figure}[tbp]
\centerline{
\includegraphics[width=0.46\textwidth]{Figures/Li7_ratios3.png} \hfill
\includegraphics[width=0.46\textwidth]{Figures/flavour_dependent_EMC_lead.pdf}}
\caption{{\it Left panel:} Prediction for the polarized EMC effect in $^7$Li (solid line) compared to a calculation of the EMC effect (dashed line) in the same framework, figure adapted from Ref.~\cite{Cloet:2006bq}. The experimental data are from Ref.~\cite{Gomez:1993ri}. {\it Right panel:}   Predictions from Ref.~\cite{Cloet:2012td} for the flavor dependence of the EMC effect in nuclear matter with the same $Z/N$ ratio as lead. Figure adapted from Ref.~\cite{Cloet:2012td}.}
\label{fig:emc_effects}
\end{figure}

It has been known for some time that there are strong isovector forces in nuclei, which are captured for example in the asymmetry term in the semi-empirical mass formula and are largely responsible for the symmetry energy. In mean-field models of nuclear structure, such as the Quark Meson Coupling (QMC) model~\cite{Guichon:2004xg,Guichon:2006er,Tronchin:2018mvu} and those based on the Nambu--Jona-Lasinio model~\cite{Bentz:2001vc,Cloet:2005rt,Cloet:2006bq}, these isovector forces couple to the quarks in the bound nucleons, which means that for $N > Z$ nuclei, the $d$ quarks feel additional repulsion and $u$ quarks additional attraction. The opposite is the case for $N < Z$ nuclei. The sign and magnitude of these forces is constrained by the empirical symmetry energy. Because $u$ and $d$ quarks are modified differently in the nuclear medium, this produces an isovector or flavor dependent EMC effect. Predictions from Ref.~\cite{Cloet:2012td} for the size of this effect in nuclear matter with a $Z/N$ ratio similar to a lead nucleus are illustrated in the right panel of Fig.~\ref{fig:emc_effects}, where it is clear that the $u$ quarks have a much larger EMC effect than the $d$ quarks. Further suggestion for an isovector EMC effect is that calculation for iron can explain more than one-sigma~\cite{Cloet:2009qs,Bentz:2009yy} of the NuTeV anomaly~\cite{Zeller:2001hh}. In addition, evidence for a flavor dependence in the EMC effect could distinguish between mean-field explanations and those based on SRCs. Experiments have demonstrated that SRCs are predominantly $np$ pairs, which occur approximately 20 times more often that $nn$ and $pp$ pairs~\cite{Hen:2016kwk}. As such, SRCs are predominantly isoscalar and cannot produce an isovector EMC effect.

A flavor dependence in the EMC effect should manifest in a number of experiments, e.g., by contrasting structure function measurement in $^{40}$Ca and $^{48}$Ca. A novel method to measure the isovector EMC effect is via PVDIS to measure the $\gamma$-$Z$ interference structure function $F_2^{\gamma Z}(x)$ and contrast this with the usual DIS structure function to perform a flavor separation of the $u$ and $d$ quark PDFs in the same nuclear target. A proposal to perform this experiment on $^{48}$Ca is has been considered for the JLab 12\,GeV program. Interesting opportunities also exist in the comparison of SIDIS on $^3$H and $^3$He with either $\pi^+$ or $\pi^-$ detected in the final state.



\subsubsection{Nucleon structure from tagged DIS}

The discussion above around the EMC effect and its varying aspects concerns the partonic structure of the entire nucleus, and any interpretation in terms of the structure of the individual nucleons inside the nucleus is necessarily model dependent. Nevertheless, understanding how a nucleon responds to the nuclear environment is a fundamental question in nuclear physics and one pathway to this information is provided via tagged processes. These processes involve the measurement/tagging of a recoiling particle that can be used to isolate and identify a subcomponent of the original nuclear target. For example, in a experiment with a $^4$He target, tagging a recoiling $^3$He nucleus is an indication that the probe interacted with a bound neutron in the $^4$He target. Such processes were first pioneered at JLab by the BONuS Collaboration to measure the neutron $F_2(x)$ structure function by tagging the proton from a deuteron target~\cite{CLAS:2011qvj,CLAS:2014jvt}.
Recently, the BONuS12 collaboration~\cite{BONUS12,CLAS:2018unv} completed a dramatically enhanced program of tagged measurements from the deuteron to extract the structure of the free neutron by tagging low-momentum spectator nucleons.  

Two related experiments are approved that will make tagged measurements with high-momentum spectators to explore the virtuality dependence of the structure functions of protons and neutrons bound in the deuteron. These experiments will perform tagged DIS using a high-resolution spectrometer to detect the scattered electron and new large acceptance detectors (LAD and BAND) to measure recoiling protons and neutrons~\cite{Segarra:2020txy}. A similar measurement~\cite{E12-11-002} will focus on the virtuality dependence of quasielasatic e-N scattering to study the impact of the virtuality to the effective in-medium form factors. Finally, the TDIS program~\cite{C12-15-006A,E12-15-006} will make high-statistics measurements of free neutron structure using proton tagging on the deuteron~\cite{C12-15-006B}, and will also measure pion and kaon structure using spectator tagging of the proton to isolate pion structure in p(e,e'$p_s$)X and d(e,e'$p_s$)X scattering and the kaon using Lambda tagging, as discussed in Sec.~\ref{sec:section2-FF-PDF}.

Finally, ALERT~\cite{Armstrong:2017zcm,Armstrong:2017zqr,Armstrong:2017wfw} in Hall B will complement the CLAS12 spectrometer apparatus, enabling access to the low-energy nuclear remnants from electron scattering off the deuteron and $^4$He. ALERT plans a comprehensive program using tagged DIS and incoherent DVCS on these targets and can measure/tag final state protons, $^3$H, and $^3$He. Therefore, ALERT provides access to the structure of the bound protons and neutrons in the deuteron and $^4$He and will deliver data on the 3$D$ structure of these nuclei. 



\subsubsection{Super-fast quarks and hidden color}

Exotic configurations have been suggested as a possible contribution to the structure in nuclei.  Overlapping baryonic states could contain exotic configurations such as 6-quark bags~\cite{Bickerstaff:1984gut}, hidden color states~\cite{Brodsky:1995rn, Brodsky:2004tq, Brodsky:2004zw}, or multi-diquark contributions~\cite{West:2020rlk}. Such states allow for more direct sharing of momentum between quarks in different nucleons, yielding predictions that they could have a significant contribution of super-fast quarks, quarks with momentum fraction well above what is easily achievable through simple smearing of the nucleon PDFs~\cite{Sargsian:2002wc,Freese:2014zda}. 

A measurement is planned~\cite{E12-06-105} that will examine DIS from the deuteron for $x>1$. Just as these kinematics isolate contributions from SRCs in quasi-elastic scattering, they can also isolate SRCs in DIS at sufficiently large $Q^2$. At this point, we are probing the PDFs of the deuteron at $x>1$, and the types of models described above predict a significant enhancement of the super-fast quark distribution compared to a simple convolution model.  In addition to looking for these exotic configurations, this also allows the possibility to test models in which the large virtualities of the high-momentum nucleons cause the EMC effect. In this case, the kinematics isolate high-momentum nucleons and probe the PDFs of the moving proton and neutron at large $x$, where several models predict a significant suppression of the PDFs~\cite{Melnitchouk:1993nk, Melnitchouk:1996vp, Hen:2016kwk}. Thus, the super-fast quark distribution could be suppressed by these off-shell effects, or could be significantly enhanced by exotic configurations of overlapping nucleons, providing important information on the microscopic structure of SRCs and a new way to understand its connection to the EMC effect.

In the presence of such non-nucleonic configurations at short-distance scales, there would be a significant enhancement of SRC-like N-$\Delta$ or $\Delta$-$\Delta$-like pairs~\cite{Ji:1985ky}. Initial estimates  suggested that such measurements would be much more sensitive at 12~GeV or even higher energies. In addition to probing the distribution of super-fast quarks in nuclei, measurements of SRC-like N--$\Delta$ and $\Delta$-$\Delta$ configurations are expected to be significantly enhanced in the presence of hidden color configuration~\cite{Ji:1985ky}, and may also be sensitive to other exotic multi-baryonic contributions. Such measurements would significantly benefit from higher energies.



%===============================================================================
%===============================================================================
\subsubsection{Femtography for Nuclei} 

The 3D imaging of quarks and gluons in nuclei presents a tremendous opportunity to study aspects of QCD that cannot occur in nucleons and to explore phenomena that may expose the explicit role of QCD in nuclei, beyond the residual van der Waals type forces studied in traditional nuclear physics methods. Just as GPDs and TMDs provide a spatial and momentum tomography for the nucleon (see Sec. II), the same quantities provide analogous information for nuclear targets. However, nuclei provide a much richer environment with which to study quark-gluon dynamics than a single nucleon, in part, because nuclei not only provide stable targets with $J=1/2$ but also those with spin $J=0$ and $J \geq 1$. For example, the deuteron with $J=1$ has 9 time-reversal-even TMDs and 9 GPDs at leading twist, each revealing different aspects of the quark-gluon structure of the deuteron.

Perhaps the simplest nucleus to study experimentally is $^4$He, which is a tightly bound system of two protons and two neutrons having zero spin, and is therefore characterized by fewer observables than targets with spin. For example, $^4$He has one electromagnetic form factor and one DIS structure function in the Bjorken limit. This simplified structure and tight binding make $^4$He an ideal nucleus with which to study QCD effects in nuclei via nuclear femtography. DVCS on $^4$He was studied at JLab, with first results for coherent scattering presented in Ref.~\cite{CLAS:2017udk} and incoherent scattering results were reported in Ref.~\cite{CLAS:2018ddh}. Further studies at 12\,GeV with DVCS and other processes such as DVMP, should provide sufficient data to extract the underlying spin-independent quark GPD for $^4$He and perhaps offer insight on the gluons. Such studies stand to reveal several interesting aspects of nuclear structure, e.g., can the color singlet nucleons be identified in the spatial distributions and do they cause interference patterns, do these interference patterns for quarks and gluons overlap and how do these correlations change when viewed at different slices in $x$ and $b_T$ in impact parameter space. 

The TMDs of nuclei provide similar opportunities. At leading twist $^4$He has two TMDs, the familiar spin-independent naive time-reversal even TMD, $f_1(x,\boldsymbol{k}_T^2)$, and the Boer-Mulders function $h_1^\perp(x,\boldsymbol{k}_T^2)$ which is naive time-reversal odd and describes the distribution of transversely polarized quarks in an unpolarized or spin-zero target. Comparison of the $(x,\boldsymbol{k}_T^2)$ dependence of these functions in nuclei with those in the nucleon will provide significant new insights into quark-gluon dynamics in the nuclear medium. 

The deuteron also provides a unique system for nuclear femtography, because it is a finely-tuned system making it sensitive to small QCD effects. As a spin-one target, it possesses a tensor polarization in addition to the familiar vector polarization of a spin-half target such as the nucleon. This tensor polarization produces three additional time-reversal even and seven additional time-reversal odd TMDs compared to a spin-half target. Deuteron TMDs and GPDs could reveal new aspects of the nucleon-nucleon interaction and how this interaction has its origins in QCD~\cite{Ninomiya:2017ggn}. Interesting studies in the 3D imaging of nuclei are also provided via comparisons between $^3$He and $^3$H, such as clean flavor separation and the study of nuclear effects in the comparison between $^3$H and the proton, which can provide guidance on the extraction of neutron results from $^3$He and deuteron targets.

A robust program in the 3D imaging of quarks and gluons in nuclei at JLab 12\,GeV stands to provide unprecedented insight into nuclear structure and QCD effects in nuclei. These studies also complement the expected program at the EIC which will be focused at higher energies and smaller $x$, and similarly smaller skewness $\xi$ in GPDs and associated Compton form factors. 


\subsubsection{Impact on other physics programs}

An understanding of the EMC effect, in particular answering new questions about its spin and flavor dependence, is important for a wide range of other measurements~\cite{Cloet:2019mql}. High-energy scattering measurements from nuclei, whether $\nu$-A scattering at FNAL, A-A at RHIC or the LHC, or e-A at a future EIC, requires a good understanding of the nuclear PDFs. Understanding the flavor dependence of the EMC effect, in particular for neutron rich nuclei~\cite{Arrington:2012ax, Arrington:2015wja, CLAS:2019vsb, Arrington:2019wky} can best be addressed with PVES at JLab~\cite{Cloet:2019mql,PVEMC}. Not only does this provide the best access to flavor dependence now, but such PVES experiments would benefit significantly from higher beam energies. This is also an important issue for $^3$He, which is used as an effective neutron target for spin studies. An understanding of the polarized EMC effect, as well as an understanding of the flavor-dependence in this light but proton-rich nucleus, will be important for precision spin studies using polarized $^3$He beams and targets.


\subsection{Quark/hadron propagation in the nuclear medium}

\subsubsection{Quark/hadron propagation}
The confinement principle of QCD dictates that color charge cannot be separated from the color-neutral hadrons that contain it. This statement, however, only pertains to equilibrium conditions. Color charge can be briefly liberated from a hadron through a hard scattering. In the simplest case, a valence quark carrying color charge can undergo a high-energy scattering process that propels the quark over long distances, closely accompanied by a spray of quarks and gluons of lower energies that ultimately evolve into new hadrons. This process, called hadronization or fragmentation, can take place on distance scales of 1-100 fm or more, and its characteristics at lower energies can be studied by observing it inside an atomic nucleus. The hadronization constituents interact with the nuclear medium, modifying hadron production in ways that reveal characteristics of this fundamental process. 

Questions that can be answered by such experimental studies include: how long does the struck quark propagate before becoming bound into a forming hadron? with what mechanisms do propagating quarks interact with the nuclear medium, and at which spatial scales do they interact? how much energy do they lose in the medium? how much medium-induced transverse momentum do they acquire? is a semi-classical description of the process adequate? In SIDIS, the picture of the initial state is particularly clear at the values of $x$ accessible at JLab. A valence quark absorbs all the energy and momentum from the interaction, which can be measured directly by the scattered electron. Neglecting small contributions from intrinsic transverse momentum and Fermi momentum, the energy and momentum of the struck quark are known; thus, the initial state of the interaction between the quark and the medium is well determined. This secondary "beam" of quarks is generated throughout the nucleus with an initial position probability determined by the well-known nuclear density distribution, allowing precise modeling of the process.

In this picture, the hadron containing the struck quark will interact with the medium with a total cross section dominated by inelastic processes at the energies relevant to JLab. In the kinematics where the hadron forms earliest, which is at high relative energy $z_h=E_h/\nu$, this cross section can be determined with moderate accuracy using sufficiently precise data and simple models~\cite{BROOKS2021136171}. The experimental signature of this region is a maximal suppression of hadron production in large nuclei relative to the proton or deuteron.

An important consideration in these studies is the multi-variable dependence of observables such as multiplicity ratios and transverse momentum broadening. The nominal number of possible variables in the A(e,e'h)X reaction is up to five, typically chosen to be four-momentum transfer $\mathrm{Q^2}$, energy transfer $\mathrm{\nu}$ or Bjorken-x  $\mathrm{x_{Bj}}$, the relative energy $\mathrm{z_h}$, the momentum transverse to the direction of momentum transfer $\mathrm{p_T}$, and the azimuthal angle around that momentum transfer direction $\mathrm{\phi_{pq}}$. The HERMES Collaboration was the first to publish two-dimensional results for such observables with nuclear targets~\cite{Airapetian_2011}, finding unexpected and complex behavior that exposes the details of hadronization dynamics. This has been followed up by 3-fold differential studies by the CLAS Collaboration \cite{moran2021measurement}. Such studies provide a motivation to collect large datasets that allow binning in up to 5-fold differential bins for the hadrons that are produced most copiously, such as pions, kaons, and protons. The rarer particles, such as $\phi$, $\eta$, $\omega$, and anti-proton, can still be studied in at least one or two dimensions in the 11 GeV data.

Hadron propagation in the medium can also be studied via other reaction types, often motivated by the search for color transparency. An increased nuclear transparency to hadrons in specific kinematics is a definite prediction of QCD that is linked to important properties such as factorization of initial and final states in high-energy scattering. Experiments in the 6 GeV era of JLab saw the onset of color transparency for pions~\cite{PhysRevLett.99.242502, PhysRevC.81.055209} and $\rho$ mesons~\cite{El_Fassi_2012}. Future studies of color transparency (CT) at JLab for mesons and baryons  will strongly benefit from the high luminosity and large reach in four-momentum transfer foreseen. The A(e,e'h)X channel in diffractive and non-diffractive kinematics probes CT in meson production, while it can be used to look for proton transparency in quasi-elastic kinematics~\cite{PhysRevLett.126.082301, Bhetuwal:2022wbe}. The principal signature of CT is a reduced nuclear transparency observed as four-momentum transfer $Q^2$ increases, for fixed coherence length. While the transparency seen in the 6 GeV era was small but significant, a robust signature for light mesons is expected to be very clear as beam energy and luminosity rise.

\subsubsection{Hadron formation}
The mechanisms involved in the hadronization process that dynamically enforce color confinement are poorly known. More insight into these mechanisms can be obtained by systematic study of production of different baryon and meson types in large and small nuclear systems. Questions that can be answered by such studies include: what are the differences between formation of $q\bar{q}$ (baryon number B=0) systems and $qqq$ (B=1) systems? how do the characteristics of formation change as the number of strange quarks increases, both in mesons and baryons? how does the formation depend on hadron mass? what can we learn about the time required for complete formation of hadrons? is there evidence of diquark structure seen for baryon formation, and if so, how does it influence our understanding of proton and neutron structure?

\subsection{Future opportunities} 

There are specific measurements that would benefit significantly from an increase in the electron beam energy, and other cases where entirely new measurements are possible. Higher energies would benefit parity violating measurements, by allowing higher cross sections at fixed $Q^2$ or increased $Q^2$ values with a corresponding increase in the parity-violating asymmetries. This could improve the coverage of the parity-violating EMC measurement discussed earlier, which aims to confirm and quantify the flavor dependence of the EMC effect in $^{48}$Ca. With a sufficiently improved figure of merit, the kinematic range of the measurement could be expanded, or additional nuclei could be measured to look for flavor dependence  in isoscalar nuclei or other non-isoscalar nuclei such as $^9$Be. 

The improved $Q^2$ range provided by higher beam energies could also be used to extend the scaling studies at $x>1$, with the aim of extracting the PDFs in nuclei at $x>1$, where the cross section is dominated by scattering from SRCs. Increasing the energy would allow measurements at larger $x$ and $Q^2$, extending further into the DIS region. Such measurements on the deuteron would allow for better comparisons of models based on a simple convolution of two nucleons as opposed to those including exotic configurations (6-quark bags, hidden color, or $\Delta-\Delta$ contributions), or models with large off-shell corrections for high-momentum nucleons.

Other programs will benefit from higher energies through the increase in cross section for measurements which do not need expanded $Q^2$ coverage. Measurements of  A(e,e'p) in light nuclei, especially for polarization observables for scattering from $^3$He would benefit. Several measurements utilizing spectator tagging would also benefit, as they are frequently limited by luminosity and cross section.

At higher energies, completely new probes of quark hadronization dynamics will be possible. For example, it will be possible to produce more massive hadrons, beyond the present-day limit of 0.78 GeV for production of the $\omega$ meson off nuclei~\cite{borquez_2021}. An example of this is illustrated in Fig.~\ref{fig:heavymesons} which shows production rates for one month of operation as a function of beam energy. The highest rates shown are for the $\phi$ meson, a 1-GeV mass two-quark system which will first be probed in these studies with an 11 GeV beam in Hall B with much lower production rates, and thus with limited statistical accuracy. With the high statistical accuracy achievable with 24 GeV beam, it will be possible to study the multi-dimensional behavior of the formation of this meson for the first time, a key factor for isolating hadronization dynamics. A further advance is shown in the figure with the D meson production rates, which is sub-threshold at current JLab energies. As seen from that figure, the phase space for charm production opens up rapidly as 24 GeV is approached. The crossing of the charm threshold enables study of these 2-GeV mass two-quark systems for the first time. This is entirely {\em{terra incognita}} with respect to current-day models of hadron formation.   

Beyond mass reach, the increase in {\em{kinematic}} reach provided by approximately doubling the beam energy will have important benefits for studies of color propagation and color transparency. For a particle of mass $m$ and lab energy $E$, the relativistic boost in the lab frame will be $E/m$. This factor will approximately double relative to the 11~GeV case. This increase will permit studies of the expected color lifetime time dilation which is expected to occur for light mesons, and will improve the sensitivity to color transparency as well, not only from the boost but also from the increased coverage in four momentum transfer $Q^2$, the primary kinematic variable used in those studies. Since color transparency is expected to increase with $Q^2$, the substantially expanded range in that variable should reveal a larger magnitude of the transparency, thus increasing sensitivity of the measurement.
Another impact of the wider kinematic range is that slightly lower $x$ will be accessible, e.g. down to 0.07-0.08. At sufficiently low $x_{Bj}$, $q\bar{q}$ pairs are produced diffractively from the virtual photon. For nuclear targets this would result in dijets or dipions passing through the medium, sharing the energy of the virtual photon. This process has been studied theoretically~\cite{PhysRevD.46.931} and the onset of this distinct mechanism could be searched for at 24 GeV.

\begin{figure}
    \centering
    \includegraphics[width=120mm,scale=0.35]{Figures/HeavyMesons24GeV.pdf}
    \caption{Production rates of heavy mesons as a function of beam energy. The calculation is based on one month of operation in CLAS12 at the already-achieved luminosity for deuterium targets of up to 2 x $\mathrm 10^{35}~cm^{-2}s^{-1}$}
    \label{fig:heavymesons}
\end{figure}

\newpage
