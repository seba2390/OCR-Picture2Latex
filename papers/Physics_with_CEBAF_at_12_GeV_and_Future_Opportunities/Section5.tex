\section{ The Standard Model and Beyond} 
\label{sec:BSM}

The Standard Model (SM) of Particle Physics has reached a concluding milestone in 2012, brought by the experimental observation of the Higgs boson by both the ATLAS and the CMS collaborations at the LHC~\cite{Aaboud:2018zhk,Sirunyan:2018kst}. 
%
There are still many opening questions. But overall, the SM has withheld nearly half a century of experimental examination and has so far been a quite successful framework to describe three of the four interactions of Nature. 
On the other hand, it's probably a consensus within the physics community that the SM is not the ultimate theory: At what energy scale and by what symmetry can we unify strong with electroweak? Under what framework can we add gravity? And how do we explain dark matter and dark energy?  We must therefore continue high-precision measurements of SM parameters and look hard for beyond the SM (BSM) phenomenon -- hints of new particles and new interactions -- by experimenting across the whole energy scale. One such example is the recent muon $g-2$ measurement~\cite{PhysRevLett.126.141801} that raised challenges to lepton universality, adding exciting fresh information to the search for BSM physics. 
  

CEBAF at JLab has provided an essential tool in our pursuit of understanding the strong interaction and the nucleon and nuclei since the late 1990's. In the recent decade, two new directions have emerged in JLab's research program: studies of electroweak (EW) physics and ``dark sector'' searches for direct production of low-mass, weakly coupled new physics. Being at an energy scale between atomic physics and particle colliders, CEBAF holds a unique position in the landscape of SM and BSM study. 

Among EW physics measurements, JLab has a long history of carrying out parity-violating electron scattering (PVES) experiments. During the 6 GeV era, the Qweak experiment provided the best knowledge on the proton weak charge and improved our knowledge on the EW neutral-current (NC) axial-vector ($AV$) $C_{1q}$ couplings~\cite{Androic:2013rhu,Androic:2018kni}. The 6~GeV PVDIS experiment similarly improved our knowledge on the vector-axial ($VA$) $C_{2q}$ couplings~\cite{Wang:2014bba,Wang:2014guo}. In the coming years, the planned MOLLER experiment~\cite{Benesch:2014bas} will provide one of the most precise data on the weak mixing angle $\sin^2\theta_W$. 
%
With the addition of a high intensity device -- the SoLID spectrometer~\cite{Chen:2014psa} -- the PVDIS measurement will be extended over a wide $(x,Q^2)$ range and to higher precision, making it possible to simultaneously probe the $C_{2q}$ couplings and hadronic effects such as charge symmetry breaking and higher twists~\cite{PVDIS}. PVDIS on the proton will provide $d/u$ ratio at high $x$ without the need of using a nuclear model. Despite technical and theoretical challenges, the future positron beam at CEBAF opens up another direction -- possible measurements of the axial-axial ($AA$) couplings $C_{3q}$ -- that cannot be overlooked.

In the dark sector program, JLab's APEX and HPS experiments have pioneered searches for new bosons that are produced via a weak coupling to electrons and decay through the same interaction, with the massive, kinetically mixed ``dark photon'' as a canonical benchmark model.  HPS will continue to explore this parameter space in the coming years, exploring both prompt and displaced decays of dark photons and other weakly coupled bosons. In addition, the BDX proposal for a detector downstream of the Hall A beam dump is poised to explore light dark matter production. % at the dump.  
The future positron beam at JLab opens up new possibilities to search for production of weakly-coupled new physics via annihilation of beam positrons on atomic electrons, using the complementary missing-mass and missing-energy approaches. 

\subsection{ Neutral-current electroweak physics at low energies}
\label{sec:MOLLER}

At energies much below the mass of the $Z^0$ boson (the ``$Z$-pole''), $Q^2\ll M_Z^2$, the Lagrangian of the EW NC interaction relevant to electron-electron (M\o ller) scattering or electron deep inelastic scattering (DIS) off quarks inside the nucleon is given by~\cite{Zyla:2020zbs}:
\begin{eqnarray}
L_{NC} &=& \frac{G_F}{\sqrt{2}} \left[ g_{AV}^{ee}\bar e\gamma_\mu\gamma^5 e\bar e\gamma^\mu e+ 
%g_{VV}^{eq} \, \bar e\gamma^\mu e \bar q\gamma_\mu q + 
g_{AV}^{eq} \, \bar e\gamma^\mu\gamma_5 e \bar q\gamma_\mu q 
%\right. \nonumber \\ &+& \left. 
+ g_{VA}^{eq} \, \bar e\gamma^\mu e \bar q\gamma_\mu \gamma_5 q + 
g_{AA}^{eq} \, \bar e\gamma^\mu \gamma_5 e \bar q\gamma_\mu \gamma_5 q \right], \label{eq:L}
\end{eqnarray}
where $G_F=1.166\times 10^{-5}$~GeV$^{-2}$ is the Fermi constant and the $g^{ee}_{AV}$, $g^{eq}_{AV}$, $g^{eq}_{VA}$, $g^{eq}_{AA}$ are effective four-fermion couplings. 
We have omitted both $VV$ and $AA$ coupling of electron-electron interaction on the RHS of Eq.~(\ref{eq:L}) because they can only be measured at high energies~\cite{ALEPH:2005ab,Schael:2013ita}. Similarly, the $g_{VV}^{eq}$ term, being chirally identical to electromagnetic interactions of QED, is too difficult to measure at low energies and is omitted here. 
In contrary, the $AV$ and $VA$ couplings are both parity-violating and thus can be separated from QED effects by measuring parity violation observables, for example the cross section asymmetry between a right-handed and left-handed electron beam scattering, 
\begin{eqnarray}
  A_{RL} &=& \frac{\sigma_R-\sigma_L}{\sigma_R+\sigma_L}~. 
\end{eqnarray}
%
The coupling $g_{AV}^{ee}$ has been measured by SLAC E158~\cite{Anthony:2005pm} and is the primary operator for the planned MOLLER experiment~\cite{Benesch:2014bas} at JLab. 
The $g_{AV}^{eq}$ is best determined by combining atomic parity violation~\cite{Wood:1997zq,Guena:2005uj,Toh:2019iro} and elastic electron scattering experiments such as Qweak~\cite{Androic:2013rhu,Androic:2018kni}.
%
The $g_{VA}^{eq}$ requires spin-flip of quarks and can only be measured in deep inelastic scattering. 
The last term on the RHS of Eq.~(\ref{eq:L}) does not violate parity but can be measured by comparing DIS cross sections of lepton with anti-lepton DIS, which can be done once a positron beam becomes available at JLab~\cite{Zheng:2021hcf}. 

%Note: At HERA energies (at least using the high $Q^2$ bins) they could resolve the vector and axial-vector couplings, even though with large error bars.  New amplitudes are suppressed there, because they don’t resonate. At lower energies you only see the products but new amplitudes are relatively less suppressed. So there is complementarity.  The EIC may have a little bit of both.

All $AV$ and $VA$ effective couplings are related to the weak mixing angle $\sin^2\theta_W$, the single parameter that intertwines electromagnetic with weak interactions within the SM framework.  The present world knowledge on the weak mixing angle is shown in Fig.~\ref{fig:thetaW}. There are measurements across the energy scale from atomic to LHC experiments, but the precision of measurements much below the $Z$ pole can be improved. In the next decade, three high-precision experiments will come online: the MOLLER and the SoLID PVDIS experiments at JLab and the P2 experiment at the upcoming MESA accelerator at Mainz, with the precision of MOLLER and P2 comparable to the most precise data to date from LEP, SLC and LHC above the $Z$-pole. MOLLER and P2 experiments will potentially anchor the running of $\sin^2\theta_W$ from the low energy side, with SoLID PVDIS sitting in the transition $Q^2$ range. Low energy measurements also hold unique discovery potential if BSM physics (particles and amplitudes) exist in this energy regime and not higher. 

\begin{figure}[!ht]
\begin{center}
%\includegraphics[width=0.7\textwidth]{Figures/EW_figs/WeakPlotDic2020V2.pdf}
%\includegraphics[width=0.8\textwidth]{Figures/EW_figs/WeakPlotDic2021_edit.pdf}
\includegraphics[width=0.8\textwidth]{Figures/EW_figs/WeakPlotDic2021_noFCC_edit.pdf}
\end{center}
\caption{Running of the weak mixing angle $\sin^2\theta_W$, updated from Ref.~\cite{Erler:2017knj} (by courtesy of R. Ferro-Hernandez). Data points for Tevatron and LHC are shifted horizontally for clarity. The expected $Q^2$ coverage of the future electron-ion coillder (EIC) where its expected data will have an impact is shown by the arrows. The achieved precision from Qweak is $\pm 0.0011$~\cite{Androic:2018kni}, and the expected precision from P2, MOLLER and SoLID PVDIS are $\pm 0.00033$~\cite{Becker:2018ggl}, $\pm 0.00028$~\cite{MOLLERCDR} and $\pm 0.00057$~\cite{Erler:2014fqa}, respectively.
}\label{fig:thetaW}
\end{figure}


On the other hand, the weak mixing angle is one parameter of the EW theory that works across energy scale and link all observables, but only in the framework of SM itself.  %Should a particular BSM physics affects only a certain type of interactions (leptonic vs. semi-leptonic, or a certain chiral structure), one must dive into precision measurements of different couplings. 
By combining observables from different types of experiments, our ultimate goal is to map out all chiral combinations of the couplings for the purpose of both understanding and testing the SM. In M{\o}ller scattering, the parity-violating asymmetry for longitudinally polarized incoming electron is 
\begin{eqnarray}
 A_{PV}^{ee} &=& m E \frac{G_F}{\sqrt{2}\pi\alpha}\frac{2y(1-y)}{1+y^4+(1-y)^4}Q_W^e
\end{eqnarray}
where $Q_W^e=-2(1-4\sin^2\theta_W)$ is the weak charge of the electron, $\alpha$ is the fine structure constant, $E$ is the incident beam energy, $m$ is the electron mass, $\theta$ is the scattering angle in the center-of-mass frame, $y=1-E'/E$ with $E'$ the energy of either one of the scattered electrons. The SLAC E158 experiment measured %$Q_W^e$ to 
$g^{ee}_{AV} = 0.0190 \pm 0.0027$ and $\sin^2\theta_W^\mathrm{eff}=0.2397\pm 0.0010$(stat.)$\pm 0.0018$(syst.) at $Q^2=0.026$~(GeV)$^2$. The MOLLER experiment will improve the uncertainty of $g_{AV}^{ee}$ ($Q_W^e$) to %$\pm 0.0010$ or
relative $\pm 2.4\%$ and $\sin^2\theta_W$ to $\pm 0.00028$~\cite{MOLLERCDR}, comparable to 
%the $\pm 0.00027$ 
the results from SLC~\cite{SLD:2000leq} and 
%the $\pm 0.00021$ from 
LEP~\cite{ALEPH:2005ab}.
%Note: LEP publication had several results for sin2thetaW, from 0.00021 to 0.00019 and combined (on their first page) 0.00016
%0.0010 Tevatron PRD84(2011)012007
%0.00053 CMS 1806.00863 hep-ex

The current knowledge on the $AV,VA$ electron-quark couplings are shown in Fig.~\ref{fig:c1c2}. 
\begin{figure}[t]
\includegraphics[width=0.47\textwidth]{Figures/EW_figs/contour_C1_2021_P2_z2_twin_edit3.pdf}
\includegraphics[width=0.47\textwidth]{Figures/EW_figs/contour_SoLID_2018_twin_edit3.pdf}
\caption{Adapted from Ref.~\cite{Zheng:2021hcf}: Current experimental knowledge of the couplings $g_{AV}^{eq}$ and $g_{VA}^{eq}$. 
For $g_{AV}^{eq}$ (left): The latest measurement is from the 6 GeV Qweak experiment~\cite{Androic:2018kni} at JLab. The Atomic Parity Violation ("APV 2019") results shown utilized the theory calculations of Ref.~\cite{Toh:2019iro}. The "eDIS" band is a combination of the SLAC E122~\cite{Prescott:1978tm,Prescott:1979dh} and the JLab PVDIS~\cite{Wang:2014bba,Wang:2014guo} experiments. 
For $g_{VA}^{eq}$ (right), 
The latest measurement is from the PVDIS experiment~\cite{Wang:2014bba,Wang:2014guo} at JLab. 
Also indicated are the expected uncertainties from the planned P2 experiment~\cite{Becker:2018ggl} at Mainz (left) and from the planned SoLID project at JLab (right), both centered at the SM value. }
\label{fig:c1c2}
\end{figure}
%
The latest data on $g_{AV}^{eq}$ was from the parity-violating asymmetry in $ep$ elastic scattering during the Qweak experiment~\cite{Androic:2013rhu,Androic:2018kni}: 
\begin{eqnarray}
 A_{RL, elastic}^{ep} &=& - \frac{G_F Q^2}{4\pi\alpha\sqrt{2}} \left(Q_w^p+Q^2 B(Q^2,\theta)\right),
\end{eqnarray}
where $Q_w^p=-2(2g_{AV}^{eu}+g_{AV}^{ed})$ is the proton weak charge, $-Q^2$ is the four-momentum transfer squared, $\theta$ is the scattering angle, and $B(Q^2,\theta)$ represents the proton's internal structure. By measuring the asymmetry at $Q^2=0.025$~GeV$^2$ and extrapolating to the $Q^2=0$ point, the weak charge of the proton was determined to be $Q_w^p=0.0719\pm 0.0045$ and the weak mixing angle $\sin^2\theta_W=0.2383\pm 0.0011$. When combined with atomic parity violation experiment~\cite{Wood:1997zq,Guena:2005uj,Toh:2019iro}, the Qweak experiment provides the best constraint on the $g_{AV}^{eq}$ couplings to date. The P2 experiment planned at Mainz will improve the uncertainty by a factor 3 over Qweak, determine $Q_w^p$ to $\pm 1.83\%$ and $\sin^2\theta_W$ to $\pm 0.00033$~\cite{Becker:2018ggl}.

One may notice the stark contrast in Fig.~\ref{fig:c1c2} in the precision between $AV$ and $VA$ couplings. This is because access to the electron's axial coupling is directly provided by the spin flip of electron beam and thus is an observable in all PVES experiments. On the contrary, access to the quark's axial coupling requires quark spin flip and can only be achieved in the DIS regime, and is suppressed due to angular momentum conservation by the kinematic factor 
$Y=[1-(1-y)^2]/[1+(1-y)^2]$. For electron DIS off an isoscalar target such as the deuteron and if one considers only the light quark flavors $u$ and $d$: 
\begin{eqnarray}
  A^{e^-,\mathrm{PVDIS}}_{RL,d}
    &\approx& \frac{3G_F Q^2}{10\sqrt{2}\pi\alpha}\left[(2g_{AV}^{eu}-g_{AV}^{ed})+R_V Y(2g_{VA}^{eu}-g_{VA}^{ed})\right]~,
\end{eqnarray}
where $R_V(x) \equiv ({u_V+d_V})/({u^+ + d^+})$ 
with $q^+, q_V$ defined with parton distribution functions $q(x)$: $q^+\equiv q(x)+\bar q(x)$ and $q_V\equiv q(x)-\bar q(x)$.
%
Access to $g_{VA}^{eq}$ is thus best provided by ``high $y$'' settings such as the fixed-target configuration at JLab. As shown in the right panel of Fig.~\ref{fig:c1c2}, the SoLID PVDIS experiment~\cite{PVDIS} will improve over the previous 6 GeV measurement~\cite{Wang:2014bba,Wang:2014guo} by an order of magnitude. 


One general and model-independent way of characterizing BSM physics search potential of an experiment is to express BSM physics in terms of contact interactions that perturb the SM Lagrangian~(\ref{eq:L}), {\it i.e.}, by replacements of the form~\cite{Erler:2014fqa},
\begin{eqnarray}
%\frac{G_F}{\sqrt{2}} g_{ij} \rightarrow \frac{G_F}{\sqrt{2}} g_{ij} + \eta_{ij}\frac{4\pi}{(\Lambda_{ij})^2}\ ,
\frac{G_F}{\sqrt{2}} g_{ij} \rightarrow \frac{G_F}{\sqrt{2}} g_{ij} + \eta_{ij}\frac{g^2}{(\Lambda_{ij})^2}\ ,
\label{eq:ciqmodified}
\end{eqnarray}
where $ij=AV,VA,AA$ and can be for either $ee$ or $eq$ interaction, $g$ is the coupling and $\Lambda$ is the mass scale of BSM physics, {\it i.e.} the coupling and the mass of the hypothetical BSM particle being exchanged. 
If the new physics is strongly coupled, $g^2 = 4\pi$, then the 90\% C.L. mass limits reached or to be reached by MOLLER, Qweak, and SoLID PVDIS on $g_{VA}^{ee}$, $g_{AV}^{eq}$ and $g_{VA}^{eq}$ are, respectively:
\begin{eqnarray}
 \Lambda_{VA, \mathrm{MOLLER}}^{ee}&=& g \sqrt{\frac{\sqrt{2}}{G_F 1.96\Delta g_{VA}^{ee}}}=39~\mathrm{TeV}, \\
 \Lambda_{AV, \mathrm{Qweak}}^{eq}&=& g \sqrt{\frac{2\sqrt{2}\sqrt{5}}{G_F 1.96\Delta Q_w^p}}=30~\mathrm{TeV},\\
 \Lambda_{VA, \mathrm{SoLID+world}}^{eq} &=& g \sqrt{\frac{\sqrt{2}\sqrt{5}}{G_F 1.96\Delta \left(2g_{VA}^{eu}-g_{VA}^{ed}\right)}}=16~\mathrm{TeV}; 
\end{eqnarray}
where the $\sqrt{5}$ for $Q_w^p$ and PVDIS cases are to represent the ``best case scenario'' where BSM physics affects maximally the quark flavor combination being measured. Similar to Qweak, the P2 experiment will pose a mass limit on $g_{AV}^{eq}$ at  49~TeV~\cite{Erler:2014fqa}. The expected uncertainty $\Delta\left(2g_{VA}^{eu}-g_{VA}^{ed}\right)=\pm 0.007$ is obtained by combining SoLID PVDIS with existing world data. If one instead look for the mass limit expected from PVDIS in any combination of $g_{AV}^{eq}$ and $g_{VA}^{eq}$ then it would be 22~TeV from SoLID alone. 
%{\color{red}{(SoLID PVDIS is 22 TeV in  https://arxiv.org/pdf/1401.6199.pdf but not purely VA)}}

Another approach to characterize BSM physics reach is to use SM effective field theory (SMEFT)~\cite{Boughezal:2021kla}. In SMEFT approach, BSM physics is described as contact interactions similar to that EW NC interactions can be treated as  4-fermion contact interactions at energies much below the $Z$-pole. Furthermore, operators in SMEFT can be generalized to beyond 4-fermion (dimension-6) terms, for example, to include general dimension-8 operators. Low energy PVES experiments such as P2 and SoLID PVDIS in fact will help to disentangle dimension-6 from dimension-8 SMEFT couplings, as these cannot be separated by data from high energy colliders alone~\cite{Boughezal:2021kla}. On the other hand, in SMEFT analysis one must be careful to limit the use of assumptions, as over-simplification will minimize the apparent value of low energy data that provide direct access to a single combination of couplings.

Looking forward, in addition to MESA and the possible energy upgrade of JLab, a series of upgrades for the LHC are being discussed~\cite{LHeCStudyGroup:2012zhm,LHeC:2020van,FCC:2018byv,FCC:2018evy} that will venture into the unexplored energy range much beyond the $Z$-pole~\cite{Britzger:2020kgg}. %Electroweak observables at high energy colliders are combinations of all couplings and separation is done through global fitting (see {\it e.g.}~\cite{Spiesberger:2018vki}).
The EIC, coming online within the next 1-2 decades, likely will have decent sensitivity to EW couplings in between JLab and high-energy colliders as well. %, though the concept of effective low-energy couplings no longer apply due to its relatively high $Q^2$.  
 Among all existing and planned facilities, CEBAF is one of the few that can provide direct access and high precision measurements of the SM effective couplings owing to both its high luminosity fixed-target settings and the relatively low beam energies, and thus holds a unique place in the test of the SM across all energy scales.  

\subsection{ Searches for Dark Sectors}
\label{sec:APEXHPS}
\label{sec:BDX}
The last decade has seen rapidly growing interest in searches for physics beyond the Standard Model at low masses and much weaker couplings to familiar matter than the SM interactions.  These are motivated by the broad framework of ``dark sectors'', i.e. any new particles neutral under SM interactions, which form a natural framework for low-mass dark matter and are potentially connected to a wide range of astrophysical and experimental anomalies.  The dark sector framework and these motivations are summarized in a recent review article \cite{Lanfranchi:2020crw} and many community reports \cite{Alexander:2016aln,Battaglieri:2017aum,DarkMatterBRNReport,Beacham:2019nyx,Agrawal:2021dbo}.  

Standard Model symmetries considerably restrict the interactions of ordinary matter with SM-neutral matter. In the absence of high-energy modifications to the Standard Model, the allowed low-energy interactions or ``portals'' correspond to mixing of a new vector $A'$ (``dark photon'' or ''U boson'') with the Standard Model photon, of a new scalar $\phi$ with the SM Higgs boson, or of a new neutral fermion $\psi$ with neutrinos. Because each of these is associated with a marginal operator ($ \frac{1}{2}\partial_{[\mu} A'_{\nu]} F^{\mu\nu}$,  $|\phi|^2 |H|^2$, and $\psi HL$), loops of heavy particles can generate these interactions with parametric strength of order one or two loop factors, $\sim 10^{-2} - 10^{-6}$.   These interaction strengths are sufficiently weak that they would not be seen in generic collider searches for high-mass new physics.  Instead, these possibilities are best explored through either moderate-energy, high-intensity experiments or unconventional low-$p_T$ LHC searches.  The intensity and bunch structure of the CEBAF beam is especially well-suited to direct searches for the production of dark-sector particles. 

Of the portals above, the dark photon (a.k.a. vector or kinetic mixing portal) mentioned above has emerged as a canonical example. It allows for simple and predictive dark matter models and is also the portal for which electron-beam searches, such as those at CEBAF, are most powerful so we will focus on this case.  For massive dark photons, kinetic mixing of strength $\epsilon$ induces couplings $\epsilon e$ to SM matter of electric charge $e$ \cite{Holdom:1985ag, Okun:1982xi}. Dark photon exchange effects are parametrically smaller than their electromagnetic counterparts and respect the same symmetries, but can be constrained either by the \emph{difference} of QED-like effects at different energy scales (\emph{e.g.}~different values of $\alpha$ inferred from electron \emph{vs.}~muon anomalous magnetic moments) or from direct production of dark photons (or dark-sector particles that couple to them) in laboratory experiments \cite{Pospelov:2008zw}. Most such searches can be divided into three categories:
\begin{itemize}
\item\textbf{Visible dark photon} searches assume that a dark photon decays back to SM particles through the same kinetic mixing interaction that mediates its production.   This assumption implies a precise lifetime and branching ratios for a given dark photon mass and production cross-section; at masses below the muon threshold, the leading decay is to $e^+e^-$ pairs with a lab-frame lifetime that can be prompt or displaced at the mm to meter scale in light of existing constraints (see e.g. \cite{Bjorken:2009mm,Lanfranchi:2020crw}).   This signal can be distinguished from QED backgrounds in the same final state by searching for a small but narrow resonance peak in the $e^+e^-$ mass distribution and/or reconstructing a displaced decay vertex. The parametric estimate of interaction strength $\epsilon \sim 10^{-2} - 10^{-6}$ arising from one- or two-loop effects also motivates exploring this full range of allowed lifetimes.
\item\textbf{Dark matter production searches} test the possibility of new invisible species being produced through dark-photon interactions. Some \emph{beam dump searches} aim to detect a small fraction of these invisible products through their subsequent scattering in a detector downstream of a high-intensity beam dump; other \emph{missing energy} or \emph{missing momentum} searches instead aim to detect the production event via low total energy deposition and/or measurement of the recoiling electron's kinematics.  A particular focus of these searches is the \emph{thermal relic target}, a band of interaction strengths $\epsilon \gtrsim 10^{-6} (m_\chi/\rm{MeV})$ for which a dark-sector particle $\chi$ annihilating to ordinary matter through the dark photon could explain the observed abundance of dark matter through thermal freeze-out (see e.g.~\cite{Battaglieri:2017aum,DarkMatterBRNReport,Berlin:2020uwy} for discussion of this target as well as the complementary sensitivity of non-JLab fixed-target experiments using electron and proton beams). 
\item\textbf{Missing mass} searches generally search for the reaction $e^+e^- \rightarrow \gamma A^\prime$, in either high-intensity colliders or positron-beam fixed-target experiments.  These experiments use the detected photon kinematics to reconstruct the mass of an undetected dark-photon candidate, and search for a peak in this distribution.  In some cases, such experiments can probe dark photons inclusively, irrespective of the decay mode (including not only the decays to SM or dark matter particles discussed above, but also more complex cascade decays into multiple SM and dark-sector particles).  
\end{itemize}
It should be noted that, while these searches are conventionally interpreted in the context of dark-photon models, they apply more generally to any boson that couples to electrons --- including $B-L$ gauge bosons or scalar bosons --- with only O(1) changes in the CEBAF constraints. 


%%{\it This part needs to be finished. Include 
%(1) overview of why low-mass dark sectors are interesting. 
%(2) general phenomenological framework of vector/scalar with small coupling to electrons and/or other SM particles and kinetic mixing portal as a canonical example, 
%(3) possibilities for dark photon/mediator decay and light dark-matter production via virtual mediator, and 
%(4) exciting milestones in parameter space e.g.~for thermal dark matter and loop-level mixing.}

So far, three dark-sector search experiments have run at JLab, focused on different aspects of this physics. 
%
The A-Prime EXperiment (APEX) \cite{Essig:2010xa} searches for dark photons decaying promptly to $e^+e^-$ in Hall A.  Such a signal would show up as a narrow resonance in $e^+e^-$ mass over a smooth QED background, and is peaked when the combined energy of the pair is near the incident beam energy.  The APEX search uses this kinematics to optimize the signal acceptance relative to background rate, using the two High Resolution Spectrometers (HRSs) in coincidece to measure events with an electron in one arm and positron in the other.  A septa magnet is used to deflect pairs with central angle $\sim 6^\circ$ into the spectrometer acceptances.   The 2010 APEX test-run \cite{Abrahamyan:2011gv} demonstrated this technique in a narrow mass range, achieving new sensitivity in dark photon parameter space tough it has since been surpassed by resonance searches at MAMI-A1 \cite{Merkel:2014avp} and in BaBar data  \cite{Lees:2014xha}.  The full APEX run in 2019 accumulated over 100$\times$ greater true $e^+e^-$ pair statistics than the test run in a similar geometry; its ongoing analysis will explore new dark photon parameter space in the mass range 160--230 MeV.   

 The Heavy Photon Search (HPS) in Hall B also searches for  dark photons decaying promptly to $e^+e^-$, but with a detector consisting of a 7-layer Silicon Vertex Tracker (SVT), a scintillation hodoscope, and downstream ECal \cite{Battaglieri:2014hga,Balossino:2016nly}. This detector configuration allows for precision tracking and vertexing of $e^+e^-$ pairs produced at forward angles, enabling a displaced-vertex search that opens up sensitivity to weaker dark-photon couplings, in addition to a resonace search.  HPS engineering runs at 1.1 and 2.3 GeV beam energies took place in 2015 and 2016, with the first physics results from these runs appearing in \cite{Adrian:2018scb,Solt:2020zbi}.  Following a detector upgrade, a first physics run took place in 2019 at 4.56 GeV beam energy, for which analysis is ongoing. The second physics run completed in November 2021, with more beam time available to run approved by the JLab PAC for outyears (105 PAC days).  These runs will use beam energies between $\approx 2$ and $\approx 4$ GeV, with each beam energy allowing sensitivity to a different dark photon mass range.  Projected sensitivities for the full runs of APEX and HPS, as well as current constraints, are shown in Figure \ref{fig:darkPhotonProj}. 

\begin{figure}[htbp]
\begin{center}
  \includegraphics[angle=0,width=0.6\linewidth]{Figures/darkphoton_reach_projection_CEBAF.pdf}
  \caption{Current exclusion regions for a dark photon with the projected sensitivity of two fully approved and operating CEBAF experiments, APEX (2019 statistics --- magenta) and HPS (full run plan including 2019, 2021, and future running---green).  The exclusion regions come from \cite{Konaka:1986cb,Riordan:1987aw,Bjorken:1988as,
     Bross:1989mp,Davier:1989wz, Bjorken:2009mm, Abrahamyan:2011gv, Archilli:2011zc, Andreas:2012mt, Babusci:2012cr, Agakishiev:2013fwl, Adare:2014mgk, KLOE-2:2014qxg, Lees:2014xha, Merkel:2014avp, Adrian:2018scb,  Aaij:2017rft, LHCb:2019vmc}. Plots and projections courtesy of HPS collaboration and Vardan Khachatryan.}
  \label{fig:darkPhotonProj}
\end{center}
\end{figure}
 A third bump-hunt search at JLab has recently been proposed using the PRad apparatus in Hall B.  Instead of a magnetic spectrometer, this proposal uses GEM tracking and calorimetry $\sim 8$ meters downstream of a thin target to reconstruct the kinematics and invariant mass of $A^\prime$ candidate $e^+e^-$ pairs.  This approach is well-suited to reconstruction of small-angle pairs, enabling a search for $3-60$ MeV $A^\prime$'s with a $2-3$ GeV beam.  The proposal calls for 60 PAC days of running to cover most of the allowed parameter space in this mass range with $\epsilon$ large enough to allow prompt $A^\prime$ decays, and is conditionally approved.
 
A complementary ongoing effort at JLab is the Beam Dump eXperiment (BDX) proposal to search for light dark matter with a parasitic detector downstream of the Hall A dump \cite{Battaglieri:2014qoa,Battaglieri:2016ggd,Battaglieri:2017qen,Battaglieri:2019nok}.  This detection scheme relies on the fact that any light dark matter (LDM) particles that interacts with electrons (as is the case in benchmark dark photon models as well as other leptophilic scenarios) will be produced as beam electrons traverse a beam dump and initiate an electromagnetic shower.  The resulting diffuse secondary beam of LDM particles emanates from the dump, and LDM traversing the detector volume can, with low probability, deposit energy in the detector via a single scattering reaction off electrons or nuclei \cite{Izaguirre:2013uxa}.  In inelastic dark matter models, excited states of DM may also decay in the detector into a lighter DM state plus an $e^+e^-$ pair, whose energy deposition offers another modality for LDM detection \cite{Izaguirre:2014dua}.  BDX proposes to use a meter-scale segmented CsI(Tl) scintillation detector surrounded by layers of an active veto. The BDX technique has been tested at JLab exposing a small prototype detector (BDX-MINI ~\cite{Battaglieri:2020lds}) to a 2 GeV electron beam and accumulating in a $\sim$6-months-time $\sim 15\%$ of the Electron-On-Target expected in BDX (fully parasitic). The stability of the system and the results obtained provided a solid ground for the full  experiment. 
BDX has been approved by JLab PAC-46, and is seeking funding for the detector and associated cavern.  Once constructed, operation of BDX will be completely parasitic as data can be taken during \emph{any} high-current operations in Hall A.  A complementary detection strategy, using a DRIFT directional light-WIMP TPC \cite{Snowden-Ifft:2018bde} in the BDX cavern, has also been proposed.  

Finally, we note that JLab experiments are also sensitive to axion-like particles (ALPs).  Unlike the portals discussed above, ALPs have an approximate shift symmetry, as a result of which their leading couplings to the Standard Model are dimension-5 operators (for example, a coupling $\frac{4\pi \alpha_s c_g}{\Lambda}a G^{\mu\nu} \tilde G_{\mu\nu}$ to gluons, motivated by the strong CP problem, as well as analogous couplings to photons).  The coupling discussed above allows ALP photoproduction off protons \cite{Aloni:2019ruo}.  This process is also the basis for a recent ALP search at GlueX \cite{GlueX:2021myx}, which explores new ALP parameter space in the 180--480 MeV and 600--720 MeV regions.  

\subsection{ Rare Processes } 
%\section{The JEF experiment in Hall D}
\label{sec:jefexperiment}

The JLab Eta Factory (JEF) experiment~\cite{pro-JEF17, pro-JEF14} will
perform precision measurements of various $\eta^{(\prime)}$ decays
with emphasis on rare decay modes, using the GlueX
apparatus~\cite{Adhikari:2020cvz} and an upgraded Forward
electromagnetic Calorimeter (FCAL-II).  Significantly boosted
$\eta^{(\prime)}$ mesons will be produced through $\gamma
+p\rightarrow\eta^{(\prime)} +p$ with an 8-12 GeV tagged photon beam.
Non-coplanar backgrounds will be suppressed by tagging
$\eta^{(\prime)}$ with recoil proton detection. The $\eta^{(\prime)}$
decay photons (and leptons) will be measured by FCAL-II with a
high-granularity, high-resolution PbWO$_4$ crystal core in the central
region that minimizes shower overlaps and optimizes the resolutions of
energy and position. A capability of tagging highly-boosted
$\eta^{(\prime)}$ in combination with a state-of-the-art FCAL-II
offers two orders of magnitude improvement in background suppression
and better control of systematics compared to the other
$\eta^{(\prime)}$ experiments, such as A2-MAMI~\cite{Adlarson:2019nwa,
  Kashevarov:2017kqb}, WASA-at-COSY~\cite{Husken:2019dou},
KLOE-II~\cite{Berlowski:2019hst}, BESIII~\cite{Shuang-shiFang:2015vva}
and the proposed future REDTOP~\cite{Gatto:2019dhj}.  JEF is a unique
eta-factory, producing $\eta$ and $\eta^{\prime}$ simultaneously at
similar rates ($\sim 6\times 10^7$ tagged $\eta$ and $\sim 5\times
10^7$ tagged $\eta^{\prime}$ per 100 days), with no competition in
rare neutral decay modes.

The $\eta^{(\prime)}$ meson has the quantum numbers of the vacuum
(except the parity). Its strong and electromagnetic decays are either
anomalous or forbidden at the lowest order due to symmetries and
angular momentum conservation. This enhances the relative importance
of higher order contributions or new weakly-coupled interactions. A
study of the $\eta^{(\prime)}$ decays provides a rich
flavor-conserving laboratory to test the isospin-violating sector of
low-energy QCD and to search for new physics beyond the Standard Model
(SM)~\cite{Gan:2020aco}.  The JEF experiment will primarily focus on
the following areas:



\begin{itemize}
\item {\em Precision tests of low-energy QCD}. The $\eta \rightarrow
  3\pi$ decay promises an accurate determination of the quark mass
  ratio, $ {\cal Q} =(m_s^2-\hat{m}^2)/(m_d^2-m_u^2)$ with $\hat{m} =
  (m_u+m_d)/2$. This decay is caused almost exclusively by the isospin
  symmetry breaking part of the Hamiltonian $\sim (m_u-m_d)(u{\bar
    u}-d{\bar d})/2$.  Moreover, Sutherland's
  theorem~\cite{Bell:1968,Sutherland:1966} forbids electromagnetic
  contributions in the chiral limit; and contributions of order
  $\alpha$ are also suppressed by $(m_u+m_d)/\Lambda_\mathrm{QCD}$.
  These single out $\eta\rightarrow 3\pi$ to be the best path for an
  accurate determination of ${\cal Q}$~\cite{Leutwyler:1996,
    Bijnens:2001, Colangelo:2018jxw}.  A low-background measurement of
  the rare decay $\eta \rightarrow \pi^0 \gamma\gamma$ provides a
  clean, rare window into ${\cal O}(p^6)$ in chiral perturbation
  theory~\cite{Bijnens:1992}. This is the only known meson decay that
  proceeds via a polarizability-type mechanism. The Dalitz
  distribution measured by JEF will offer sufficient precision for the
  first time to explore the role of scalar meson dynamics and its
  interplay with the vector meson dominance. The measurements of the
  transition form factor of $\eta$ and $\eta^{\prime}$ via the
  $\eta^{(\prime)}\rightarrow e^+e^-\gamma$ decays will reveal the
  dynamic properties of those mesons, providing important input to
  calculate hadronic light-by-light corrections to the anomalous
  magnetic moment of the muon~\cite{Aoyama:2020ynm}.

\item {\em A search for various gauge boson candidates in the MeV--GeV
  mass range, probing three out of four highly motivated portals
  coupling the SM sector to the dark sector}. A leptophobic vector
  boson ($B^\prime$)~\cite{Tulin:2014} coupling to baryon number can
  be searched for via $\eta, \eta^\prime \rightarrow B^\prime
  \gamma\rightarrow \pi^0\gamma\gamma$ for $0.14< m_{B^\prime} < 0.62$
  GeV, and $\eta^\prime \rightarrow B^\prime \gamma\rightarrow
  \pi^+\pi^-\pi^0\gamma$ for $0.62 < m_{B^\prime}< 1$ GeV.  The
  leptophilic vector
  bosons~\cite{Fayet:2007ua,Reece:2009un,Bjorken:2009mm,Batell:2009yf}
  can be searched for in the decays of $\eta, \eta^\prime \rightarrow
  A^\prime \gamma\rightarrow e^+e^-\gamma$.  A
  hadrophilic~\cite{Batell:2018fqo,Liu:2019} scalar can be probed in
  $\eta\to \pi^0 S \to \pi^0 \gamma \gamma , \: \pi^0 e^+ e^-$ for
  $m_S < 2m_\pi$, and in $\eta, \eta^\prime \to \pi^0 S \to 3\pi , \:
  \eta^\prime \to \eta S \to \eta\pi\pi$ for $m_S >
  2m_\pi$. Axion-Like Particles (ALP)~\cite{
    Dobrescu:2000jt,Aloni:2018vki,Nomura:2008ru,Freytsis:2010ne} can
  be explored via~$ \eta, \eta^\prime \to \pi\pi a \to \pi\pi
  \gamma\gamma , \: \pi\pi e^+ e^-$.  Figure~\ref{b-reach-map} gives an
  example for the sensitivity of the JEF experiment. With 100 days of
  beam time, a study of $\eta \to \gamma + B^\prime (\to \gamma +
  \pi^0)$ will improve the existing model-independent bounds by two
  orders of magnitude, with sensitivity to the baryonic fine structure
  constant $\alpha_B$ as small as $10^{-7}$, indirectly constraining
  the existence of anomaly cancelling fermions at the TeV-scale.

\item {\em Fundamental symmetry tests}.  The $\eta^{(\prime)}$ meson
  is the eigenstates of C, P, CP, and G ($I^GJ^{PC} = 0^+0^{-+}$),
  representing a natural candidate to test discrete symmetries.
  Particularly, the $\eta^{(\prime)}$ meson is among very few
  self-conjugate particles existing in the nature for testing
  charge-conjugation symmetry. A search for C-violating
  $\eta^{(\prime)}$ decays (such as $\eta^{(\prime)} \to 3\gamma$,
  $\eta^{(\prime)} \to 2\pi^0 \gamma$, and $\eta^{(\prime)} \to \pi^0
  e^+ e^-$) and a mirror asymmetry in the Dalitz distribution of
  $\eta^{(\prime)} \to \pi^+\pi^-\pi^0$ will offer the best
  direct-constraints for new C-violating, P-conserving reactions.


\end{itemize}

\begin{figure}[!htb]
\begin{center}
  \includegraphics[viewport=21 49 504 524,clip,angle=0,width=0.6\linewidth]{Figures/EW_figs/B-boson.pdf}
  \caption{Current exclusion regions for a leptophobic gauge boson
    $B^{\prime}$~\cite{Gan:2020aco, Tulin:2014}, with the projected
    $2\sigma/5\sigma$ sensitivity reach for the JEF experiment via $\eta
    \to \gamma + B^\prime (\to \gamma + \pi^0)$. Color-shaded regions
    and curves are model-independent. These include constraints from
    rare $\eta$, $\eta^\prime$ decays (red), hadronic $\Upsilon (1S)$
    decays~\cite{Aranda:1998fr} (yellow), and low-energy $n$-Pb
    scattering~\cite{Barbieri:1975xy} (purple). The gray shaded regions
    and dashed contours are model-dependent and involve leptonic
    couplings via kinetic mixing $\varepsilon_x = x \frac{e g_B}{(4\pi
      )^2}$: these regions are excluded by dark photon searches for dilepton
    resonances~\cite{Babusci:2012cr,Konaka:1986cb,Riordan:1987aw,Bjorken:1988as,
     Bross:1989mp,Davier:1989wz,Banerjee:2019hmi,Tsai:2019mtm,Astier:2001ck,
     Bernardi:1985ny,Batley:2015lha,LHCb:2019vmc},
    $A^\prime \to \ell^+ \ell^-$, for $\varepsilon_{0.1}$; and the gray
    dashed contours are upper limits on $\alpha_B$ from FCNC $b \to s
    \ell^+ \ell^-$ and $s \to d
    \ell^+\ell^-$~\cite{Dror:2017nsg,Dror:2017ehi}, for
    $\varepsilon_{0.001}$ (upper line) and $\varepsilon_1$ (lower
    line). The blue dashed contours denote the upper bound on the mass
    scale $\Lambda$ for new electroweak fermions needed for anomaly cancellation.}
  \label{b-reach-map}
\end{center}
\end{figure}

\subsection{Future Opportunities and Positron Beam Prospects}

\subsubsection{EW NC couplings with a positron beam and at higher energies}
The addition of a positron beam to CEBAF will open up new landscape of EW physics study. On the topic of neutral-current EW physics, one possibility is to measure the axial-axial component of electron-quark coupling $g_{AA}^{eq}$. Historically, there has been only one measurement of the muonic $g_{AA}^{\mu q}$, experiment NA41 at CERN~\cite{Argento:1982tq}, that provided $(2g_{AA}^{\mu u}-g_{AA}^{\mu d})=1.58\pm 0.36$, analyzed with the latest knowledge on the $g_{AV,VA}$ of electrons. For the electron-quark $g_{AA}^{eq}$, it can be accessed by comparing the electron to positron scattering cross section on an isoscalar target. The yield ratio of $e^+p$ and $e^-p$ scattering was measured at SLAC and was found to be consistent with zero at the 0.3\% level~\cite{Fancher:1976ea}. 
The DIS cross sections of $e^+p$ and $e^-p$ scattering were measured at HERA and a global fit of the $g_{A,V}$ couplings was performed~\cite{ZEUS:2016vyd,H1:2018mkk}, but their sensitivity to BSM physics is different from JLab and the concept of low-energy effective coupling does not apply to HERA kinematics. The possibility of measuring the ratio of $e^+D$ to $e^-D$ DIS cross sections using the SoLID spectrometer is being explored~\cite{PR12-21-006}. In addition to possibly extracting $2g_{VV}^{eu}-g_{AA}^{ed}$, the measurement will provide data on the nucleon interference structure function $F_3^{\gamma Z}$, which are linked to the valence PDF in the parton model. 

The PVDIS measurement can be extended using a 22 GeV beam and SoLID. Under the same run conditions, the higher energy beam will provide a tighter constraint on the $g_{AV,VA}^{eq}$ coupling and the measurement is less susceptible to hadronic effects, which may allow us to utilize PV asymmetry measured in the full kinematic region for SM study and further reduce the uncertainty. 

\subsubsection{Dark sector search with a positron beam}
Positron beams also open up new opportunities for dark-sector searches that exploit the reaction $e^+ e^- \rightarrow A^\prime \gamma$ or the resonant reaction $e^+ e^- \rightarrow A' $ in the interaction of beam positrons with atomic electrons.  The latter reaction can occur for degraded positrons in a thick target so long as $m_{A^\prime} < \sqrt{2 m_e E_+}$, i.e. $m_{A^\prime} < 110$ (150) MeV for a 12 (24) GeV positron beam, and yields a huge enhancement to the production rate compared to bremsstrahlung. The availability of high energy, continuous, and high intensity positron beams at JLab allows exploration of substantial new regions in dark photon parameter space. Two complementary experimental techniques have been considered to exploit these production modes: 

In the first case, the dark photon $A'$ is produced by the interaction of the positron beam on a thin target via the process $e^+ e^- \rightarrow A' \gamma$. By detecting the associated photon, the $A'$ will be identified and its (missing) mass measured. This technique has been used by the PADME experiment at LNF-Italy. The proposed experiment will take advantage of the JLab high-energy and high intensity positron beam extending significantly the $A'$ mass range by a factor of four witth two order of magnitude higher sensitivity to the DM-SM coupling constant. 

The second set up  will use an active thick target and a total absorption calorimeter to detect remnants of the light dark matter production in a missing energy experiment.  $A'$ resonant production by positron annihilation on atomic electrons, with subsequent invisible decay, produces a distinctive peak in the missing energy distribution, providing a clear experimental signature for the signal. This experiment has the potential to  cover a wide parameter space including the thermal relic target milestone.  The use of polarization observables in the annihilation process could provide an additional tool to suppress backgrounds. 
