\subsection{Hall D}
\label{sec:app-halld}

Hall D uses a beam of polarized real photons and a nearly hermetic magnetic spectrometer~\cite{Adhikari:2020cvz}, optimized for the GlueX experiment (see Section~\ref{sec:expmesons})- for the meson spectroscopy studies. The beamline layout is schematically shown in Fig.~\ref{fig:halld_beamline}.
The electron beam is extracted from CEBAF at 5.5 passes (up to 12~GeV) to the Tagger Hall, where it passes through a 0.02~-~0.05~mm (2-5$\cdot{}10^{-4}~X_0$) thick diamond radiator and is deflected by a dipole magnet to a beam dump. The electrons that radiated 25-96\% of the initial energy are deflected by the dipole magnet into the tagger scintillator detectors. Tagging provides a ${\approx}0.1\%$ energy resolution for the beam photons. 
The radiator is aligned in order to provide the coherent radiation peaking at about 75\% of the beam energy. A 3.4 or 5.0~mm collimator 75~m downstream of the radiator increases the fraction of coherently produced photons in the photon beam. The collimated beam is characterized using the Pair Spectrometer consisting of a thin converter, a dipole magnet, and scintillating hodoscopes. It also includes a Triplet Polarimeter which allows to measure the linear polarization of the beam. Downstream, the beam arrives at the GlueX spectrometer (shown in Fig.~\ref{fig:halld_spectrometer}).

\begin{figure}[!htb]
\centering
  \includegraphics[angle=0,width=0.8\linewidth]{Figures/Hall_D_beamline.pdf}%
  \caption[Hall D beamline]{Hall D beamline layout~\cite{Adhikari:2020cvz}. 
    \label{fig:halld_beamline}
  }
\end{figure}

\begin{figure}[!htb]
%\centering
  \begin{minipage}[]{0.36\linewidth}
%    \includegraphics[angle=0,width=0.99\linewidth]{Figures/Hall_D_beam_spectrum_20um_norm_to_Al.pdf}\\
%    \includegraphics[angle=0,width=0.99\linewidth]{Figures/Hall_D_beam_polar_50um.pdf}
    \includegraphics[angle=0,width=0.99\linewidth]{Figures/Hall_D_beam_spectrum_polar_50um_5mm.pdf}
  \end{minipage}
  \begin{minipage}[]{0.63\linewidth}
    \includegraphics[angle=0,width=0.99\linewidth]{Figures/GlueX-II-V9.pdf}%
  \end{minipage}
  \caption{{\it Left, (a):} Spectrum of the photon beam produced with a 0.05~mm diamond radiator and the 5~mm collimator and measured with the Pair Spectrometer. compared with the spectrum obtained with an Aluminum radiator. 
  {\it Left, (b):} Linear polarization of the beam produced with a 0.05~mm diamond radiator and 5~mm collimator.
  {\it Right:} GlueX spectrometer layout~\cite{Adhikari:2020cvz}. The DIRC PID detector was installed for the GlueX-II experiment that started in 2020. GlueX-I had finished data taking in 2018.
    \label{fig:halld_spectrometer}
  }
\end{figure}

The GlueX experiment uses a 30~cm long liquid Hydrogen target. Liquid Helium and solid targets have been used for other experiments. The trajectories of charged tracks are detected with the help of the Central and Forward Drift Chambers (CDC and FDC), while photons are measured in the Barrel and Forward Calorimeters (BCAL and FCAL). The Start Counter positioned around the target, the Time-of-Flight (TOF) counter, and BCAL provide the timing measurements, used for event selection and Particle Identification (PID). At the end of 2019 the spectrometer was augmented with a Cherenkov detector DIRC~\cite{Ali:2020erv} for using in the GlueX-II experiment. The pipeline front-end electronics provide both the event selection (trigger) and Data Acquisition (DAQ). The luminosity is limited by the accidental rate in the tagger counters, as well as the DAQ performance. The GlueX experiment uses a relatively open trigger based on the calorimeter signals, which provides a high efficiency for most of photoproduction processes at the photon beam energies $>4$~GeV. GlueX-II runs at a post-collimator photon flux in the coherent peak of ${\approx}$50~MHz, and the DAQ event rate of 80~kHz. The data are recorded at a rate of about 1~GB/s. The momentum resolution of the spectrometer is 1-5\% for charged particle trajectories, depending on the momentum and the polar angle. The calorimeters' resolution is ${\approx}6\%/\sqrt{E}\oplus{}3\%$. For exclusive reactions the kinematic fit uses the measured energy of the beam photon, which allows to improve the resolution considerably.

The approved JEF experiment (see Section~\ref{sec:jefexperiment}) requires an upgrade of the lead glass based FCAL . The central area of about 80$\times$80~cm$^2$ will be equipped with crystals, providing a better energy and spatial resolutions. Preparations for the upgrade are in progress. The approved GDH experiment (see Section~\ref{sec:spinstrfunct}) will require a polarized target. The KLF experiment (see Section~\ref{sec:expbaryons}) will require considerable changes in the beam line. The changes will be reversible and it is assumed that after the KLF experiment the photon beam will be restored.
