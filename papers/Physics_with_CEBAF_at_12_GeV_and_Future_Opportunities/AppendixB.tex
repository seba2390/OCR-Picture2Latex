\section{Positron beams at CEBAF} 
\label{sec:appendixb}
To fully explore the potential of CEBAF, beams of polarized and unpolarized positrons with quality and modes of operation similar to the electron beam are highly desired. For many types of electron scattering experiments, conducting the measurement with a positron beam will expand the physics reach to include the lepton-charge difference, a domain that was not explored before at JLab. 
In 2018, the JLab Positron Working Group (https://wiki.jlab.org/pwg), with over 250 members from 75 institutions submitted a Letter of Intent titled {\it Physics with Positron Beams at Jefferson Lab 12 GeV} \cite{afanasev2019physics}, promoting a series of experiments using positron beams that could occupy CEBAF operations for more than 3 years. Recently, two proposals focusing on DVCS were conditionally approved based upon the availability of $>$100 nA polarized or $>$1 $\mu$A unpolarized beam intensities, respectively. 

In contrast to using a real-photon polarized gamma-ray source or exploiting self-polarization in a storage ring, a new technique for generating the positron polarization is considered. Referred to as PEPPo (Polarized Electrons for Polarized Positrons)~\cite{PhysRevLett.116.214801}, the polarization of an electron beam is transferred through the electromagnetic shower effect within a high-$Z$ target, initially by polarized bremsstrahlung and then by polarized $e^+$/$e^-$ pair creation~\cite{POTYLITSIN1997395}. This technique was experimentally tested at the CEBAF injector and demonstrated a very high transfer of polarization from an 8.2~MeV/$c$ electron beam to positrons with polarization approaching the electron beam polarization ($\sim$85\%). Most importantly, this technique is essentially independent of the initial electron beam energy, providing great flexibility in the choice of the electron injector to be used.

A conceptual design study is underway to develop continuous-wave positron beams for CEBAF. Four potential schemes are depicted in Fig.~\ref{AppB_Figure1}, each having a different “footprint” within the existing CEBAF configuration. Scheme “a” is attractive because 10~MeV is below the photo-neutron production threshold of most materials, however, suffers from the lowest yield and highest demand on the polarized electron source.  On the other hand, scheme “d” would provide a significantly higher $e^+$ yield, but likely has the largest radiological and construction footprint of the four. Rather, both schemes “b” or “c” are a better compromise between yield and footprint, each nominally generating 123 MeV positrons for acceleration to high energy, similar in function to how the electron beam at CEBAF is accelerated today.

%
\begin{figure}[tbh]
 \centering
 \includegraphics[width=0.8\textwidth]{Figures/AppB_Figure1.png}
 \caption{Four $e^+$ production schemes utilizing beams of different energies produced from a Polarized Electron Source (PES) are considered: (a) 10~MeV, (b) 123~MeV, (c) 123~MeV recycler, and (d) 1090~MeV options.}
 \label{AppB_Figure1}
\end{figure}
%

\subsection{Polarized Positron Production Target}

Three important metrics of a polarized positron conversion target are the total Yield (number of useful positrons per incident beam electron), the average longitudinal polarization of the beam $\bar{P_{z}}$ and the statistical polarized Figure of Merit (FoM=Yield$\times \bar{P_{z}}^{2}$). The total yield of $e^+$/$e^-$ pairs depends on both the electron beam energy and the thickness of the target; the target should be thick enough to generate a healthy electromagnetic shower, yet not behave as an absorber.  Figure~\ref{AppB_Figure2} shows that a tungsten converter of 3-5 mm is optimal when the electron beam energy is 123 MeV, yielding approximately 1-10 useful $e^+$ for every 1000 incident $e^-$. Assuming approximately 1\% of the positrons are collected into the positron bunch for acceleration, an electron beam current of 1~mA is needed (i.e. 123 kW) for the production of a 10-100~nA polarized $e^+$ beam.

%
\begin{figure}[tbh]
 \centering
 \includegraphics[width=0.8\textwidth]{Figures/AppB_Figure2.png}
 \caption{Geant4 simulation of $e^+$/$e^-$ yield (left) and FoM (right) versus target thickness.}
 \label{AppB_Figure2}
\end{figure}
%

If a single conversion target is used as both the $e^-$ radiator and $e^+$ generator, it must have the ability to dissipate tens of kW of beam power and be closely integrated with the positron collection. This requires a sophisticated vacuum volume which includes the target, a vacuum beam line,  some form of a powerful solenoid magnetic matching device, and shielding of the surrounding area.  Both solid targets composed of titanium alloys which spin and liquid metal jet targets are presently being considered for the target material.  An initial engineering study for a 2-3 Tesla dc solenoid matching device is also planned.

An alternative, although less efficient option, is a two-target split design, which may benefit from the development of the so-called “Compact Photon Source” (CPS) presently underway at JLab \cite{ali2017workshop}. A CPS-type photon generator and beam power absorber could serve as the first stage, generating an intense partially polarized photon beam from a high-$Z$ radiator, which is then directed at a separate positron production target. The current JLab CPS is designed for GeV-range electron beams with beam power up to 30-60 kW, producing the photon beam and absorbing the bulk of the electron beam power inside the heavily shielded enclosure. Lower electron beam energies would allow for a more compact design, while also providing adequate radiation shielding.

\subsection{Polarized Electron Source for Positron Beams}

Highly spin-polarized electron beams are generated at CEBAF within a high voltage dc photogun via photo-emission from a superlattice GaAs/GaAsP semiconductor device, yielding an average longitudinal polarization $>$85\%.  A 123~MeV PEPPo driven positron source will require polarized electron beams of $\sim$1 mA (86 C/day) for a week or more, implying that photocathode charge lifetimes $>$1000~C are needed. Measurements at CEBAF have shown the operating lifetime of GaAs/GaAsP superlattice could be extended to 450~C or more~\cite{Proceedings:GramesPSTP2017} by increasing the beam emission area to dilute the effects of ion back-bombardment. A new photo-gun with a larger emission area $>$20 mm$^{2}$ may reasonably achieve the additional factor of two in lifetime enhancement needed.  Notably, the polarized electron source must also provide a high degree of spin polarization while providing high average currents.  Using a precision Mott scattering polarimeter, the spin polarization from a GaAs/GaAsP photocathode has been demonstrated to remain high ($>$85\%) and independent of beam intensity up to over 1~mA. 

\subsection{Positron Collection}
Because of the large emission angles and varied energies of pair-created positrons, a positron collection system is needed to reduce the emission angle and transform the energy profile into an acceptable range that is suitable for the CEBAF accelerator. The useful “slice” of the transverse-longitudinal 6-D $(x,x',y,y',E,t)$ distribution is selected by using a combination of transverse focusing magnets, charge separating dipoles, acceptance limiting collimators and RF cavities for bunching and acceleration.
%
\begin{figure}[tbh]
 \centering
 \includegraphics[width=1.0\textwidth]{Figures/AppB_Figure3.png}
 \caption{A positron collection scheme.}
 \label{AppB_Figure3}
\end{figure}

One such collection scheme is shown in Fig.~\ref{AppB_Figure3}.  A large-aperture high-field tapered solenoid is used first to collect the large energy-angle positron distribution, then solenoid magnets and collimators are used to define the transverse momentum acceptance, and an energy chirping RF cavity is used to later improve bunch compression. The first dipole magnet separates positrons from electrons which are dumped, while the dispersion defines the positron momentum acceptance further at an aperture. A dipole and quadrupole chicane compresses long bunches and accelerates positrons to 123 MeV, which are then injected into the first linac of CEBAF. Using the particle tracking code General Particle Tracer~\cite{Proceedings:Stefani2021}, augmented for spin tracking, the settings of the beam line components were optimized using nominal values of acceptance for the first CEBAF recirculating linac.   Beginning with the positron distribution produced by a 123~MeV electron beam interacting with a 4~mm thick tungsten target  Table~\ref{AppB_Table1} lists pamrameters of the  polarized and unpolarized positron beams optimized and approaching the CEBAF acceptance values.   This study suggests that a 1~mA polarized electron beam would produce a 66 nA positron beam with a polarization larger than 66\%, which is acceptable for acceleration in CEBAF to an energy as high as 12~GeV.   The unpolarized positron distribution is unacceptably large in this initial study, however, in on-going optics design activities the re-injection chicane of CEBAF is now being optimized to have a larger momentum acceptance to the north linac.

An alternative acceleration scheme employs a compact Fixed Field Alternating gradient (FFA) accelerator. An FFA-type beamline uses fixed field, combined function magnets (which can also be permanent magnets) with large energy acceptances to control the trajectories of the beam. FFAs have been demonstrated at several facilities, including EMMA~\cite{EMMA-Pos07} and CBETA~\cite{PhysRevLett.125.044803}. Initial studies collecting unpolarized positrons are considered in the energy range of 25-35 MeV (see Table~\ref{AppB_Table1}). This suggests an energy boost ratio of about 6:1 is needed to achieve 123 MeV, which is well within the abilities of even more demanding FFA schemes (e.g. 30:1 for FFA@CEBAF). A notable benefit of an FFA is that fewer accelerating cavities are required to achieve the required beam energy, mitigating unnecessary beam loss within SRF cavities.

\begin{table}
\centering
\begin{tabular}{l|l|l|l}
\hline
 Parameter       & Unpolarized $e^+$ &  Polarized $e^+$ &  CEBAF Acceptance\\   
\hline
Efficiency  & $1.98\times10^{-4}$ & $0.66\times10^{-4}$ &  \\
Mean Energy  & 123 MeV & 123 MeV  & 123 MeV \\
$\frac{\Delta P}{\bar{P}}$ & 10\% & 4\%  & 2\% \\
$\epsilon_n$  & 105 mm-mrad & 38 mm-mrad  & $<$40 mm-mrad \\
Bunch Length  & 12 ps & 2.4 ps & $<$4 ps \\
Transverse rms  & 1.8 mm & 1.2 mm & $<$3 mm \\
Polarization & $\sim8$\% & $\sim66$\% &\\
\hline
\end{tabular}
\caption{Polarized and unpolarized positron beam properties after the scheme shown in Fig.~\ref{AppB_Figure3} prior to injection into the CEBAF recirculating linacs.}
\label{AppB_Table1}
\end{table}

\subsection{12 GeV CEBAF Acceptance of $e^-$ and $e^+$ Beams}

The injection chicane properties (aperture and dispersion) control the CEBAF beam acceptance. These are normally configured for low emittance and low momentum spread beams, but the configuration has considerable flexibility. For instance, the chicane dispersion can be configured to accept up to 2\% momentum dispersion and for the low currents anticipated for positron operation, the acceptable RMS beam radius may be as high as several millimeters matching a normalized emittance acceptance of 40 mm-mrad for a beam energy of 120 MeV. Because this principal limiting aperture is very localized, it can be readily modified to increase its acceptance. After injection, the beam momentum is increased by a factor of 9 in the first linac (the north linac). The result of this strong adiabatic damping is that the momentum acceptance of the accelerator is dominated by the injection chicane. The transverse emittance is similarly strongly damped, and the injection chicane again provides a principal limitation.

Estimated beam parameters are shown in Fig.~\ref{AppB_Figure4} comparing electron and positron beam momentum spread and geometric transverse emittance from the north linac entrance through to the experimental halls at 12~GeV. Two main regimes are affecting the beam properties: the acceleration damping within the CEBAF accelerating sections, and the synchrotron radiation in the recirculating arcs. As can be seen from Fig.~\ref{AppB_Figure4}, the dynamics of the momentum spread of electron beams is dominated by synchrotron radiation. On the other hand, positron beams essentially benefit from acceleration damping which results in the same momentum spread than electron beams, despite a much larger initial momentum spread. The large positron beam emittance at the injector entrance is also strongly reduced by acceleration effects which result in a final emittance 4-5 times larger than the electron beams, resulting in beam sizes only about twice as large.

%
\begin{figure}[tbh]
 \centering
 \includegraphics[width=1.0\textwidth]{Figures/AppB_Fig4.pdf}
 \caption{Comparison of simulated electron (left) and positron (right) beam properties \cite{Proceedings:RoblinJPOS2017}. The light blue arrow indicates the prominence of acceleration damping effects, and the light orange ones corresponds to the dominance of the effects of synchrotron radiations. The emittances are geometric and the momentum spread is RMS.}
 \label{AppB_Figure4}
\end{figure}
%

\subsection{Reversing CEBAF Magnets for Positron Beams}

The CEBAF magnetic transport system contains over 2100 magnets whose polarity must be configured separately for either electron or positron operation.  Fortunately, more than 90\% ($>$1900) are correctors and quadrupole magnets with bipolar power supplies able to drive positive or negative amperage. These can be used for electron or positron beams without any changes in hardware. Some trim system magnets are used for 30~Hz modulation, position modulation and fast feedback. It is expected that these will remain the same as well.

All of the recirculation dipole magnets in CEBAF are powered by unipolar power supplies without polarity reversal capacity, namely 21 recirculation and dogleg units, 12 extraction units and 6 units in the end-station transport and beam dump lines.  All 39 units and their respective shunt modules to distribute current to magnet strings require an engineered solution for reversing the current and a firmware upgrade of the controls; the cost of such effort has been estimated and is not unreasonable. The four experimental halls have a small number of mixed bipolar and unipolar power supplies for their beamline magnets that would similarly need to be addressed. The only permanent dipole is in Hall D to prevent beam into the hall in case of the tagger magnet failure. This magnet may need to be rotated to accommodate positron beams. 
%achieve the same outcome. 

All CEBAF magnets would also need to provide reliable magnetic fields for either polarity.  The existing bipolar magnet field maps have been examined for polarity invariance as an internal self-consistency validation. In some low-field magnets the earth’s field is apparent, but only as a measurement offset. Generally, observations are consistent with no change in magnetic field performance with bipolar current operation.

The unipolar meter-scale recirculation dipoles have not been tested for a calibration change after polarity inversion. The relative size of effects associated with remnant fields (10-15 Gauss), sudden power supply trips which introduce uncontrolled flux as the dipole field collapses, and occasional adjustment or correction of hardware during the life span of the magnets suggest the magnet iron is magnetically “soft” enough such that field strengths in use in the CEBAF accelerator result in no persistent calibration shifts after field reversal and restoration.  The calibration curves are expected to be polarity-invariant, so that inverting the current will invert the field as long as the hysteresis cycle is respected.  In order to demonstrate this, typical magnets operated at CEBAF would require testing in a magnet measurement lab, for restoration of field after polarity inversion to quantify any small systematic effects.  Local nuclear magnetic resonance core field measurements would suffice for this purpose.

\subsection{Conclusions and Future Work}

The Jefferson Lab Positron Working Group has developed a world-class selection of experiments which may be accomplished using polarized (and unpolarized) positron beams with energies up to 12~GeV.  In response, a Laboratory Directed Research and Development (LDRD) project investigated the parameter space of a continuous-wave linac-driven source of polarized positrons for CEBAF.  That effort produced a layout, based upon the PEPPo approach and integrated with the existing accelerator, which when optimized delivers polarized positrons with an energy of 123 MeV and within the transverse and longitudinal acceptance injection of the two 12~GeV recirculating linacs.  The combination of momentum damping in the accelerating sections and emittance damping due to synchrotron radiation in the recirculation arcs suggests positron beams with energy spreads and geometric emittances within a factor of a few of those achieved with 12 GeV electron beams are possible.  A  review of the transport system suggests all of the magnets at CEBAF could operate in reversed polarity for positrons, once magnet tests are completed and engineered solutions are developed.  The state-of-the art in polarized electron sources and high power targets suggest there are no "show stopping" limitations for this scheme to be realized at Jefferson Lab, although significant design and engineering research and development remains. 

In conclusion, a new Positron Beams Research and Design working group has been established in the Accelerator Division at Jefferson Lab which includes colleagues from US and international national laboratories, universities, and industry.   The group is presently exploring the following topics:

\begin{itemize}
    \item Design of a compact milliAmpere polarized electron injector
    \item Technical description of a 123 MeV positron injection beam line
    \item Reconfiguration of the CEBAF re-injection chicane for increased momentum and emittance acceptance
    \item Design, development and testing of high power 10-100~kW targets for positron production
    \item Design of a high-field adiabatic matching solenoid magnet
    \item Spin rotator designs of 100 MeV to 12 GeV polarized positron beams
    \item Start to end CEBAF positron beam optics model and simulation
    \item Evaluating positron production schemes compatible with simultaneous electron beam operations
\end{itemize}
The working group is planning to publish a technical report by the end of 2022, and would like to begin the engineering of a prototype target and positron collection beam line by 2023.