\section{CEBAF Energy Upgrade}
\label{sec:appendixc}

Extending the energy reach of the CEBAF accelerator up to 24 GeV should be possible within the existing tunnel layout~\cite{harwood:2001, leemann1}. Such an increase can be achieved either by increasing the overall energy per pass, increasing the number of recirculations in the accelerator, or some combination of both. Other considerations in the overall upgrade path are related to reuse of existing 12 GeV accelerator components. Preliminary studies have produced several options that could be considered for further development~\cite{mckeown:2020}. Here, the option of adding two, high energy acceptance Fixed Field Alternating Gradient (FFA) arcs to 12 GeV CEBAF is discussed~\cite {IPAC21:2018bde}.

Applying this idea to achieve significantly more beam recirculations in CEBAF to reach higher beam energy is attractive for several reasons~\cite{mckeown:2020}. First is that multiple beam energies can be confined and recirculated in the same, albeit more complicated, FFA beam line. A second reason this approach is attractive is that the existing CEBAF linacs, perhaps marginally enhanced in energy, will be all that is needed to achieve the higher energy. A third reason this approach may be attractive is that the existing ARC beamlines for the first four beam passes could remain part of the upgraded accelerator; the FFA lines would occupy roughly the area where the existing ARC 9 lies in the east region of CEBAF, and the other FFA arc would reside near the present ARC A.



\subsection{Modern Fixed Field Alternating Gradient (FFA) Technology }
 
 Initially deployed in cyclotrons in Japan for accelerating heavy particles, two small-scale electron accelerators have been built and operated successfully using the FFA idea. In the UK, the multipass EMMA~\cite{machida:2012} accelerator at Daresbury Laboratory successfully contained and accelerated electron beam through a factor of 1.7 in energy in a single beamline with non-varying magnetic field. More pertinent to a CEBAF-like accelerator, the Cornell-BNL ERL Test Accelerator (CBETA)~\cite{CBETA-DR:2018bde} recently demonstrated eight-pass beam recirculation with energy recovery (four accelerating beam passes and four decelerating beam passes) through a complete TESLA-style SRF
cryomodule~\cite{CBETA-PRL:2018bde}. The beam energies on the different beam passes were 42, 78, 114, and 150 MeV (energies spanning nearly a factor of 4) with a single beamline configured consisting of fixed-field permanent magnets. CBETA demonstrates experimentally simultaneous transport of multiple  beams and the wide dynamic energy range possible in the FFA transport system beyond the requirements for an upgrade to CEBAF.
 
%
%
\begin{figure}[tbh]
  \centering
  \includegraphics[width=0.6\textwidth]{Figures/FFA_CEBAF_1.png}
  \caption{Schematic layout of FFA CEBAF.}
  \label{fig:FFA_CEBAF}
\end{figure}
%
\subsection{Energy 'Doubling' Scheme}




Encouraged by this recent success, a proposal was formulated to increase CEBAF energy from the present 12~GeV to 20-24~GeV by replacing the highest-energy arcs, ARC 9 and Arc A, with a pair of Fixed Field Alternating-Gradient (FFA) arcs, with very large momentum acceptance to recirculate the beam for 6-7 additional passes. Such a scheme would nearly double the energy using the existing CEBAF SRF cavity system. The new pair of arcs configured with FFA lattice would support simultaneous transport of 6-7 passes with energies spanning a factor of two. This wide energy bandwidth was achieved using the non-scaling FFA principle~\cite{Trbojevic-NSFFA:2018bde} implemented with Halbach-derived permanent magnets~\cite{CBETAmagnets:2018bde}.  CBETA's maximum energy was 150~MeV, whereas CEBAF upgrades plan to extend this technology to higher beam energies. 
The design schematic is shown in Fig.~\ref{fig:FFA_CEBAF} In the accelerator passes 1-4 would be accomplished through the current 12~GeV CEBAF. Passes 5-10 (six passes) would be facilitated by constructing two 'CBETA-like' beam-lines indicated schematically as 'green' arcs with 'purple' Time of Flight (TOF) chicanes in Fig.~\ref{fig:FFA_CEBAF}. 
%
%
\subsection{Multi-pass Linac Optics}
One of the challenges of  the multi-pass (10+) linac optics is to provide uniform focusing in a vast range of energies, using fixed field lattice. Here, we configured a building block of of linac optics as a sequence of two triplet cells with reversed quad polarities flanking two cryomodules, as illustrated in Fig.~\ref{fig:Twin_Cell}, with a stable periodic solution covering energy ratio 1:18.
%
\begin{figure}[ht]
  \centering
  \includegraphics[width=0.6\columnwidth]{Figures/Twin_Cell.png}
  \caption{'Twin-Cell' periodic triplet lattice at the initial and final linac passes : 1.23 GeV and 20.3 GeV. Initial triplets, configured with 45 Tesla/m quads, are scaled with increasing momentum along the linac.}
  \label{fig:Twin_Cell}
\end{figure}
%
\subsection{Proof-of-principle FFA Arc Cell}

The parameters of the main arc cell for the energy doubler are given in Table~\ref{tab:22GeV-arc-cell}.  This lattice uses very high gradients and the 10-22~GeV beams are all confined to a region $-5~\mbox{mm}<x<4~\mbox{mm}$. The orbits and optics of the unit cell for the different energies are shown in Fig.~\ref{fig:22GeV-arc-cell}.
\begin{table}[!hbt]
  \centering
  \small
  \begin{tabular}{lcccc} 
  \hline
  %\toprule
  Element & Length & Angle & Dipole & Gradient \\
   & [m] & [$^\circ$] & [T] & [T/m] \\
  \hline\hline
  %\midrule
  BF & 0.625 & 0.5 & 0.681 & 250.91 \\
  O & 0.05 & 0 & & \\
  BD & 0.5382 & 0.5 & 0.941 & -233.13 \\
  O & 0.05 & 0 & & \\
  \hline
 \\
  %\bottomrule
  \end{tabular}
  \caption{Energy doubler FFA arc cell.}
  \label{tab:22GeV-arc-cell}
\end{table}
%
\begin{figure}[htb]
  \centering
  \includegraphics[width=0.7\columnwidth]{Figures/22GeVoptics.png}
  \caption{MAD-X optical functions for the 10-22~GeV energy doubler FFA arc cell.}
  \label{fig:22GeV-arc-cell}
\end{figure}
%
\subsection{Energy Doubler FFA Racetrack}
Figure~\ref{fig:22GeV-racetrack} shows how the beta functions are expanded and the dispersion matched adiabatically to zero in the straight sections of the racetrack-shaped lattice.  The larger beta functions allow longer cells compatible with the higher-beta optics in the main CEBAF linacs.
%
\begin{figure}[htb]
  \centering
  \includegraphics[width=0.8\columnwidth]{Figures/22GeVracetrack-optics.png}
  \caption{MAD-X optical functions for the entire 10-22~GeV energy doubler FFA racetrack lattice.}
  \label{fig:22GeV-racetrack}
\end{figure}
%
%
\subsection{1.2 GeV FFA Booster Injector} %\NoCaseChange{1.2 GeV} FFA Booster Injector} 
The current CEBAF facility is configured with a 123~MeV injector feeding into a racetrack recirculating linear accelerator (RLA) with a 1090~MeV linac on each side.  The 123~MeV minimum makes optical matching in the first linac virtually impossible due to extremely high energy span ratio (1:175). Thus, it is proposed to replace the first pass by a new FFA-based booster, outputting an energy of $123+1090=1213$~MeV into the South Linac.

This booster will have injector linacs of energy up to $1213/6=202$~MeV, delivering either electrons or positrons (for circulation in the opposite direction).  The booster resembles CBETA, with a linac on one side surrounded by splitter lines and an FFA return loop (Fig.~\ref{fig:booster-schem}).  The booster linac operates at the same energy as the injector, meaning five passes in the booster linac (four passes in the FFA return loop) produces 1213~MeV.  Energy tunability from 50 to 100\% is produced by reducing the number of booster linac passes to three and the energies of both linacs to $1213/8=152$~MeV.  

\begin{figure}[htb]
  \centering
  \includegraphics[width=0.8\columnwidth]{Figures/booster-schem.PNG}
  \caption{Proposed 1.2~GeV FFA booster for CEBAF.}
  \label{fig:booster-schem}
\end{figure}
%
\subsection{Synchrotron Radiation Effects on Beam Quality}
 Staying within CEBAF footprint, while transporting high energy beams 10-24 GeV) calls for increase of the bend radius at the arc dipoles (packing factor of  the FFA arcs increased to about 87.6\%), to suppress adverse effects of the synchrotron radiation on beam quality. Arc optics was designed to ease individual adjustment of momentum compaction and the horizontal emittance  dispersion, $H$, in each arc.
Tab.~\ref{tab:SREmittance} lists arc-by-arc dilution of the transverse, $\Delta \epsilon$, and longitudinal, $\Delta \sigma_{\frac{\Delta E}{E}}$, emittance due to quantum excitations calculated using analytic formulas:  
%
\begin{equation}
  \Delta E = \frac{2 \pi}{3} r_0 ~mc^2~ \frac{\gamma^4}{\rho}\,
  \label{eq:Emit_dil_1}
\end{equation}
%
\begin{equation}
  \Delta \epsilon_N = \frac{2 \pi}{3} C_q r_0 <H> \frac{\gamma^6}{\rho^2}\,,
  \label{eq:Emit_dil_2}
\end{equation}
%
\begin{equation}
  \frac{\Delta \epsilon_E^2}{E^2} = \frac{2 \pi}{3} C_q r_0~ \frac{\gamma^5}{\rho^2}\,,
  \label{eq:Emit_dil_3}
\end{equation}

Here, $\Delta \epsilon^2_E$ is an increment of energy square variance, $r_0$ is the classical electron radius, $\gamma$ is the Lorentz boost and $C_q = \frac{55}{32 \sqrt{3}} \frac{\hbar}{m c} \approx 3.832 \cdot 10^{-13}$~m for electrons (or positrons).
The horizontal emittance dispersion in Eq.~\ref{eq:Emit_dil_2},  is given by the following formula: $H = (1+\alpha^2)/\beta \cdot D^2 + 2 \alpha \ D D' + \beta \cdot D'^2$ where $D, D'$ are the bending plane dispersion and its derivative, with averaging over bends defined as: $<...>~=~\frac{1}{\pi}\int_{bends}...~d\theta$.
%
\begin{table}[!hbt]
  \centering
  \small
  \begin{tabular}{lccccc} 
  \hline
  %\toprule
  Beamline & Beam Energy & $\rho$ & $\Delta E$  & $\Delta \epsilon^x_N$ & $\Delta \sigma_{\frac{\Delta E}{E}}$  \\
    & [GeV] & [m] & [MeV]&  [mm~mrad] &  [\%]\\
  \hline\hline
  %\midrule
  Arc 1 & 1.2 & 5.1 & 0.02 & 0.003 & 0.0003\\
  ... & ... & ... & ... & ... & ...\\
  Arc 8 & 8.8 & 30.6 & 9 & 12 & 0.022\\
  FFA 9 & 9.91 & 70.6 & 6 & 13 & 0.026\\
  ... & ... & ... & ... & ... & ...\\
  FFA 19 & 20.44 & 70.6 & 109 & 37 & 0.15\\
  FFA 20 & 21.42 & 70.6 & 132 & 47 & 0.17\\
  FFA 21 & 22.38 & 70.6 & 157 & 60 & 0.020\\
  FFA 22 & 23.31 & 70.6 & 185 & 76 & 0.023\\
  \hline
  %\bottomrule
  \end{tabular}
  \caption{Energy loss and cumulative emittance dilution (horizontal and longitudinal) due to synchroton radiation at the end of selected  180$^{\circ}$ arcs (not including Spreaders, Recombiners and Doglegs). Here, $\Delta \sigma_{\frac{\Delta E}{E}} = \sqrt{\frac{\Delta \epsilon_E^2}{E^2}}$}.
  \label{tab:SREmittance}
\end{table}
An 11.5 pass, 24~GeV design would deliver a normalized emittance of  \SI{76}{mm~mrad} with a relative energy spread of $2.3 \cdot 10^{-3}$ and each electron looses 976 MeV traversing the entire accelerator. Additional recirculations become less effective due to increasingly large energy loss from synchrotron radiation.
%
\subsection{Conclusions and Future Work}
It appears possible to roughly double the energy of CEBAF by implementing an FFA in place of the highest pass of the present configuration~\cite{IPAC21:2018bde}. Our work expands upon CBETA efforts, and shows a promising possible way forward for CEBAF after the 12 GeV era. Initial studies into the beam dynamics, possible machine layouts, and magnet designs have been made and paint a positive picture. However, considerable work remains to validate this concept, including, but not limited to, full start-to-end beam dynamics simulations, detailed magnet designs, diagnostics, controls, and detailed engineering. Furthermore, positron acceleration in the CEBAF machine must be investigated in detail.     
%

A Working Group has been established including Jefferson Lab, Cornell, and BNL scientists to further develop this concept. At present the group is pursuing:
\begin{itemize}
  \item Development of  a prototype arc cell based on permanent magnet technology spearheaded by CBETA
  \item Adiabatic arc architecture with gradually increased bending to alleviate complexity of a horizontal time-of-flight switchyard to be configured between current CEBAF Spr/Rec and the new FFA arcs
  \item New design of multi-pass linac optics based on triplet focusing (140 deg. phase advance per cell, at the lowest energy pass) implemented with permanent quads.
  \item New Injector design (1.2 GeV) including three C-75 cryomodules in a 5-pass FFA re-circulator
\end{itemize}
The group is targeting the end of 2022 for the completion of this stage of the design’s development work.