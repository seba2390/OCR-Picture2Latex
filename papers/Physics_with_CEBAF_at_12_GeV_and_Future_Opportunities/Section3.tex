\section{ Hadron Spectroscopy}

The energies and other quantum numbers of the excited state of any physical system provide important clues to the underlying dynamics and relevant degrees of freedom. This is especially true in the case of hadrons, where the spectrum of meson and baryon excitations established the quark model, including elements of special relativity and QCD.~\cite{Godfrey:1985xj,Capstick:1986ter}

We know, however, that the full picture has not been experimentally verified. For example, we fully expect that excited gluonic degrees of freedom should manifest themselves, but these have not yet been conclusively isolated. It is generally argued that the best discovery path is through searching for so-called ``exotic'' meson states, which have quantum numbers that cannot be obtained with only quark--antiquark degrees of freedom. Previous searches, using hadron beams, have been inconclusive, but new results using polarized photon beams from CEBAF are on the horizon.

QCD and the quark model also predict a number of baryon excitations that have yet to be observed experimentally. A new program at Jefferson lab will focus on mapping the spectrum of baryons with strangeness. Excited states in this sector should be less numerous and more narrow than for the nonstrange baryons, which will ameliorate the difficulties associated with overlapping resonances.

There have been a number of narrow charmonium states discovered in recent years, which defy description in terms of the quark model. Their existence points to dynamics of multiquark states that should in principle be predicted by QCD.

Jefferson Lab is aggressively pursuing the current spectroscopic understanding of QCD dynamics. This includes photoproduction of meson and baryon states in GlueX, CLAS12, and other CEBAF facilities. It also includes new strides in Lattice QCD that not only are sorting out the hadron spectrum in concert with experimental measurements, but also pointing the way to quantifying the photoproduction cross sections.

\subsection{Role of gluonic excitations} 
\label{sec:gluerole}

Hadron spectroscopy is key to understanding the inner workings of QCD and it is particularly relevant for determining the role of gluons. On one hand gluons are responsible for the confining force that binds quarks together into color-singlet hadrons, but they also carry most of the hadron's mass and a large fraction of its spin. Thus, gluons are also expected to act as constituents affecting hadronic excitations. In the quark model, dominance of the valence constituent quarks, which explains, among other things the observed symmetry patterns of spin-orbit excitations, the anomalous magnetic moments of baryons, the OZI rule, etc., restricts the spectrum of hadron resonances to those having the quantum numbers of the quark-antiquark pair for mesons or three quarks for baryons. For example, non-strange mesons with natural parity and odd CP are forbidden in the quark model. In the past these were referred to as exotics of the second kind, to distinguish them from flavor exotics, \textit{e.g.}, baryons with positive strangeness. The  constituent  quark model has its rooting in the large-$N_c$ world of planar diagrams and strings, which is phenomenologically supported, for example, by the observed linearity of meson and baryon Regge trajectories. Thus, it is not surprising that exotic states do not appear in the conventional hadron spectrum. There are strong indications however, that such states ought to exist and have properties, \textit{e.g.}, masses and decay widths that are similar to those of the ordinary quark model  states~\cite{Cohen:1998jb}. In the language of large-$N_c$ these exotics could, for example, be associated with string junctions, which in QCD originate from gluon-gluon interactions~\cite{Montanet:1980te}. Further insights  into the nature of gluon fields that can result in hybrid hadrons have come from lattice studies of charmonia. States with a high density of the chromo-magnetic field were identified as forming a quadruplet with $J^{PC} = (0,1,2)^{-+},1^{--}$ quantum numbers~\cite{Liu:2012ze}. Phenomenologically  these can be interpreted as bound states of a color octet  $Q\bar Q$ pair and  an effective constituent gluon with $J^{PC} = 1^{+-}$. Such axial gluon states could also play a role in formation of light flavor 
hybrids~\cite{Szczepaniak:2001rg,Guo:2008yz}, 
  including those of exotic quantum numbers,  
  $J^{PC} = 1^{-+}$, referred to as the $\pi_1$ or $\eta_1$ in the isovector or isoscalar channel, respectively. 
%There exist  several  phenomenological and lattice %studies 
% of gluons configuration that 
%  may result in hybrid %hadrons~\cite{Guo:2008yz,Greensite:2014bua,Bicudo:201%5bra}.
 % and more recently were studied using lattice QCD %simulations ~\cite{Bicudo:2015bra}. 
%For example, gluon distributions that 
% can potently result in hybrid hadrons were %identified % gluon  junctions appear in the 
% distribution of the QCD action in the presence of %static quarks gluon chains connecting $Q\bar Q$ pairs %are a natural representation of QCD eigenstates in %physical gauges~\cite{Guo:2008yz,Greensite:2014bua}, %
%  This multiplet contains the exotic state with %$J^{PC} = 1^{-+}$, referred to as the $\pi_1$ or %$\eta_1$ in the isovector or isoscalar case, %respectively. 
Following the arguments 
 based on Regge-resonance duality one finds other intriguing connections. 
  The exotic meson with $J^{PC} = 0^{--}$ would lie on a daughter Regge trajectory together with the $J^{PC}=2^{--}$ state. The latter has non-exotic quantum numbers and in the quark model it corresponds to a $D$ wave excitations of the $\rho$ (isovector) or, $\omega/\phi$ (isoscalar) mesons. All these states appear in lattice simulations but, have not yet been observed experimentally. Similarly the unknown exotics $J^{PC}=0^{+-},2^{+-}$, referred to as $b_{0,2}$ for isovector or $h_{0,2}$ for isoscalar, are expected on a trajectory degenerate with that of the $b_1/h_1$ mesons. The latter are quark model states, with quarks in the $L=S=1$, spin-orbit configurations, but little is known about them. 
 Discovering quark-gluon hybrids, and exotic mesons in particular would give a clear, new direction for future research into strongly coupled QCD.  


\subsection{Recent developments}
\label{sec:spectdev}
So far, the exotic $\pi_1$, which is expected to be the lightest of the hybrid mesons, has attracted most of the attention. Phenomenological models predict that it is dominated by $Q\bar Q$ pairs of spin-1 and thus should be copiously produced in photon diffractive dissociation at GlueX and CLAS12. It should be noted that at high energies, $E_\gamma > 10-20$~GeV photon interactions are mainly diffractive and dominated by the Pomeron exchange, which results mainly in production of neutral vector mesons.  In the JLab12 kinematics, however,  charge exchange reactions  have sizable cross sections and this setup ought to be ideal for searches of exotic mesons.  On the other hand, it is expected that the $\pi_1$ decays dominantly to complicated final states, like the $ b_1 \pi$ resulting in a $5\pi$ final state.  A “simpler” channel for the $\pi_1$ is the $\rho\pi$, which results in the $3\pi$ final state, and the golden channel is the $\eta^{(')}\pi$ although the branching rations for these are expected to be at the $1\%$ level.  In the past, the $\pi_1$ candidates were observed in the pion-induced reactions at VES, BNL and COMPASS, $p\bar p$ annihilation and in $\chi_c$ decays (for a review see~\cite{Meyer:2015eta}).  The main challenge in identifying the $\pi_1$ resides in complexity of amplitude analysis needed in order to isolate the resonance signal. Amplitude level analysis is necessary because the signal interferes with other nearby resonances and non-resonant processes, \textit{e.g.}, Deck production. This type of analysis can involve hundreds of parameters and to control systematic uncertainties, require sophisticated amplitude models that incorporate many constraints of the $S$-matrix theory. 
 
By far the most sophisticated results on $3\pi$ partial wave analysis have been performed by COMPASS, which studied $\sim 50$~M  events from pion diffractive dissociation at $190$~ GeV~\cite{COMPASS:2018uzl,Ketzer:2019wmd}. They observed an exotic $J^{PC}=1^{-+}$ partial wave, with a mass distribution that strongly depends on the momentum transfer. At low momentum transfer $-t< 0.5$~GeV$^2$ the production appears to be consistent with the Deck process, however, for $-t> 0.5$~GeV$^2$ the Deck process diminishes and a resonant-like signal appears. When parametrized as a Breit-Wigner it yields a mass and with of $m_{\pi_1}(1600) =1600^{+110}_{-60}$~MeV and  $\Gamma_{\pi_1(1600)} = 580^{+100}_{-230}$~MeV, respectively. COMPASS has also published results on partial wave analysis of the $\eta^{(')}\pi$ final state from the same reaction~\cite{COMPASS:2014vkj}.
Their results appear to be consistent with the previous findings by E852, where a $\pi_1(1400)$ signal at a mass of approximately 1400~MeV was found decaying to $\eta\pi$ and a peak at a higher mass of $\sim 1600$~MeV was observed in the $\eta'\pi$ decay. It is difficult to reconcile with theoretical predictions existence of two exotic resonance in this mass range only a few hundred MeV apart. In Ref.~\cite{Rodas:2018owy} the JPAC collaboration performed a mass dependent analysis of the two channels using an analytical, unitary amplitude model and demonstrated that the two peaks are actually consistent with a single resonance, defined as a pole in the complex energy plane. The coupling between the two channels and the difference in thresholds for production of $\eta\pi$ and $\eta'\pi$ pairs is responsible for the resonance peaking at different positions on the real energy axis.  The resulting pole positions of the resonances determined by the JPAC analysis are shown in Fig.~\ref{fig:poles}, including the tensor mesons $a_2(1320)$ and $a_2'(1700)$ as well as the exotic $\pi_1(1600)$.

\begin{figure*}
\includegraphics[width=\textwidth]{Figures/polesbig}
\caption{\label{fig:poles} Positions of the poles identified as the $a_2(1320)$, $\pi_1$, and $a_2'(1700)$. The inset shows the position of the $a_2(1320)$. The green and yellow ellipses show the $1\sigma$ and $2\sigma$ confidence levels, respectively. The gray ellipses in the background show, 
 within $2\sigma$, variation of the pole position  upon changing the functional form and the parameters of the model, as discussed in the text
 }
\end{figure*}


The recent analysis of coupled channel scattering lattice data, by the Hadspec collaboration~\cite{Woss:2020ayi} have confirmed existence of a single mass pole in the vicinity of the nearby decay thresholds. The calculation was done using an $SU(3)$ symmetric quark mass matrix. When extrapolated to coincide with $\pi_1$ mass obtained by JPAC, it was found that the total width, $\Gamma = 139-590$~MeV is dominated by the $\pi_1 \to b_1\pi$  channel, and branching ratios are consistent with phenomenological expectations,  $\Gamma(\pi_1 \to b_1\pi) > \Gamma(\pi_1 \to \rho\pi) > \Gamma(\pi_1 \to \eta'\pi) \sim 1 \to 12$~MeV. 



\subsection{Experimental meson spectroscopy program} 
\label{sec:expmesons}

The GlueX~\cite{E12-06-102,GlueX:2014hxq,Adhikari:2020cvz} and CLAS12 MesonEx~\cite{E12-11-005} experiments provide a unique contribution to the landscape of experimental meson spectroscopy through the novel photoproduction mechanism, which has previously been relatively unexplored.  Utilizing a real, linearly-polarized photon beam in GlueX and quasi-real, low-$Q^2$ photons in CLAS12, this program covers a wide range of beam energies from $E_\gamma = 3-12$ GeV.  Unlike high-energy pion beam measurements (\textit{e.g.} the COMPASS results described above) where Pomeron exchange is dominant, for intermediate-energy photoproduction such as that explored at GlueX and CLAS12, meson production may proceed through a range of charged and neutral Reggeon exchanges.  An understanding of these exchange mechanisms is required to model the production of both conventional and exotic mesons.  

\begin{figure*}
\begin{center}
\includegraphics[width=0.45\textwidth]{Figures/gluex_eta.pdf}
\includegraphics[width=0.41\textwidth]{Figures/gluex_Delta++pi-.pdf}
\caption{Linearly-polarized $\Sigma$ beam asymmetries measured by GlueX for $\eta$~\cite{GlueX:2020fam} (left) and $\pi^-$~\cite{GlueX:2019adl} (right) meson production as a function of momentum transfer, $-t$.}
\label{fig:gluex_asym}
\end{center}
\end{figure*}

Thus, early measurements from GlueX have focused on a quantitative understanding of the meson photoproduction mechanism in this energy regime.  Utilizing the linearly-polarized photon beam, beam asymmetries $\Sigma$ have been measured for several single pseudoscalar mesons including the $\pi^0, \pi^-, K^+, \eta,$ and $\eta'$ over a broad range of momentum transfer, $-t$~\cite{GlueX:2017zoo,GlueX:2020fam,GlueX:2019adl,GlueX:2020qat}.  As demonstrated in Fig.~\ref{fig:gluex_asym}, these measurements indicate a dominance of natural exchange (\textit{i.e.} vector Reggeon exchange in the production of neutral pseudoscalars ($\Sigma\sim1$)), while the charge-exchange process requires both natural and unnatural exchanges with a significant $t$-dependence.  

The production of well-known vector meson and excited baryon, \textit{e.g.} $\Lambda(1520)$, resonances provide another opportunity to study these Regge exchange dynamics.  However, these studies take on the additional complication of modeling the angular distributions of the resonance decay described by the so-called Spin Density Matrix Elements (SDME).  Similar to amplitude or partial wave analyses, these SDME studies require a complete understanding of the detector response to account for acceptance and efficiencies in fits to multi-dimensional data.  First measurements of the polarized SDMEs for the reaction $\vec{\gamma}p \rightarrow \Lambda(1520)K^+$ find that natural parity exchange amplitudes are dominant~\cite{GlueX:2021pcl}.  Preliminary measurements of the polarized SDMEs for vector meson production~\cite{Austregesilo:2019tld} have statistical precision surpassing previous measurements by orders of magnitude~\cite{Ballam:1972eq} and show a strong preference for natural parity exchanges.  Detailed comparisons with theoretical models~\cite{Mathieu:2018xyc} will continue to refine our understanding of these meson production mechanisms.

As described in Sec.~\ref{sec:spectdev}, there have been considerable recent theoretical developments and partial wave analysis of the $\eta\pi$ and $\eta'\pi$ systems from pion beam experiments.  Independent confirmation of these observations in alternative production mechanisms, such as photoproduction, are an essential component of the global spectroscopy effort.  An initial measurement of the reaction $\gamma p \rightarrow \eta\pi^0 p$ with a photon beam by CLAS observed a significant signal in the region of the $a_2(1320)$~\cite{CLAS:2020rdz}, but did not have sufficient statistics to perform a partial wave analysis.  Recent data collected by the GlueX and MesonEx experiments at higher beam energies provide the statistical precision necessary for such measurements, as shown in Fig.~\ref{fig:gluex_etapi}, where clear signals can be seen in the region of both the $a_0(980)$ and $a_2(1320)$ for both the neutral $\eta\pi^0$ (left) and charged $\eta\pi^-$ (right) systems in the preliminary analysis of the GlueX data~\cite{gluexetapihadron}.

Similar to the beam asymmetry and SDME measurements, the angle between the meson production and linear beam polarization planes is dependent on the production mechanism for the $\eta\pi$ system.  Thus, natural and unnatural parity exchange amplitudes, referred to as positive and negative ``reflectivity'' ($\epsilon$), can be determined through an amplitude analysis~\cite{Mathieu:2019fts}.  Amplitudes describing the polarization and $\eta\pi$ decay angles for $L_m^\epsilon = S_0^\pm, D_0^\pm, D_1^\pm, D_2^+$ and $D_{-1}^-$ are utilized in the fit following the model described in Ref.~\cite{Mathieu:2020zpm}.  Figure~\ref{fig:gluex_etapi} shows preliminary results for the measured intensity of the dominant waves, with evidence for tensor meson $a_2(1320)$ production in the $D_2^+$ wave in $\eta\pi^0$ produced through natural parity (vector) exchange and the $D_1^-$ wave in $\eta\pi^-$ through unnatural parity (pion) exchange.  These studies are consistent with those described above for pseudoscalar and vector meson production, indicating that for photoproduction at JLab 12 GeV energies natural exchange tends to dominate, except when unnatural (pion) exchange is permitted and the kinematics are such that the influence of the pion pole is significant (\textit{i.e.} low $-t$). Understanding these conventional meson photoproduction amplitudes lays the foundation for searches for the exotic $P$-wave in the $\eta^{(')}\pi$ system. 

\begin{figure*}
\begin{center}
\includegraphics[width=0.49\textwidth]{Figures/gluex_etapi_neutral.pdf}
\includegraphics[width=0.49\textwidth]{Figures/gluex_etapi_charged.pdf}
\caption{Preliminary mass spectra and amplitude analysis results from GlueX for the reactions $\gamma p \rightarrow \eta\pi^0 p$ (left) and $\gamma p \rightarrow \eta\pi^- \Delta^{++}$ (right) with $0.1 < -t < 0.3$~GeV$^2$ and $8.2 < E_\gamma < 8.8$~GeV~\cite{gluexetapihadron}.  The total measured intensity is shown in black with colored points for the dominant tensor $a_2(1320)$ amplitudes, labeled $L_m^\epsilon$.}
\label{fig:gluex_etapi}
\end{center}
\end{figure*}

% future data on flavor composition
Finally, in addition to predicting the existence of gluonic excitations in mesons, Lattice QCD calculations also predict the flavor of quarks expected to be associated with these excitations and, in fact, a spectrum of hybrid mesons containing strange quarks are expected.  A significant increase in statistics is required to study these mesons containing strange quarks as they're produced at a rate roughly an order of magnitude smaller than non-strange mesons.  Thus, higher statistics data samples are currently being collected by the GlueX and CLAS12 experiments to complete program in strange meson spectroscopy, which is required to clearly identify a pattern of gluonic excitations.  Along with additional statistical precision, the recent addition of Cherenkov detectors in GlueX~\cite{Ali:2020erv} and CLAS12~\cite{Contalbrigo:2020lnd} will provide critical separation between charged pions and kaons to separate these strange and non-strange final states.

\subsection{Experimental baryon spectroscopy program}
\label{sec:expbaryons}

Another critical component of the JLab spectroscopy program carried out over the last $\sim$15 years is the study of the spectrum and structure of excited nucleon states, referred to as the $N^*$ program.  Through measurements of exclusive electroproduction of both strange and non-strange final states, detailed electrocouplings measurements over a wide kinematic range have provided critical input to global analyses to elucidate the $N^*$ spectrum (see Ref.~\cite{Carman:2019lkk,Carman:2020qmb} for recent reviews). Studies of these $N^*$ states are currently being extended with the new CLAS12 detector in the 12 GeV era of experiments, which will significantly extend the kinematic range to $Q^2 > 5$~GeV$^2$~\cite{E12-09-003,E12-06-108A}.  

The discussion of the gluon's role in the hadron spectrum as described in Sect.~\ref{sec:gluerole} implies the search for and study of hybrid baryons with constituent gluonic excitations, and Lattice QCD calculations predict a rich spectrum of such baryons with an excitation scale comparable to that expected for hybrid mesons~\cite{Dudek:2012ag}. Hybrid baryons could be identified as supernumerary states in the $N^*$ spectrum, but they do not exhibit exotic quantum numbers, making them challenging to clearly distinguish from conventional baryons. However, measurements of the electrocoupling evolution with $Q^2$ becomes critical in the search for hybrid baryons, where a distinctively different $Q^2$ evolution of the hybrid-baryon electrocouplings is expected considering the different color-multiplet assignments for the quark-core in a conventional baryon compared to a hybrid baryon~\cite{E12-16-010}.


Finally, many hyperon spectroscopy measurements are expected from the GlueX and CLAS12 experiments, where the associated production of kaons allows one to study baryons with net strangeness including the $\Xi$ and $\Omega$~\cite{GlueX:2014hxq,E12-12-008}.  However, this program will be expanded by proposal to perform hyperon spectroscopy with a neutral kaon beam in Hall D, which was recently approved by the PAC~\cite{KLF:2020gai}.  The $K_L$ Facility (KLF) will produce a secondary beam in Hall D with a flux of $\sim10^4~K_L/s$ and utilize both hydrogen and deuterium targets inside the large-acceptance GlueX experimental setup.  Differential cross sections and hyperon recoil polarizations over a broad range of kinematics will provide significant new constrains on the partial wave analyses to search for and determine the pole positions of strange $\Lambda, \Sigma, \Xi,$ and $\Omega$ hyperon resonances, where many states are predicted by quark models and lattice QCD, which have not yet been observed.  %\textit{Include a figure here with pseudodata PWA and Cascade limits?}

\subsection{Charmonium and future opportunities at higher energy} 
\label{sec:CharmPentaquark}

The discovery of multi-quark candidates, the $XYZP$ states, observed mainly in the charmonium spectrum have revolutionized the field of hadron spectroscopy. These candidates have been observed in reactions involving complicated production and/or decay processes and some signals may be difficult to interpret due to kinematic effects which could mimic a resonance signal~\cite{Esposito:2016noz,Guo:2017jvc,Olsen:2017bmm,Brambilla:2019esw}.  Therefore, observation of these states in photo- or electro-production is needed to confirm their   existence and obtain more information on their structure.

Following the observation of the narrow pentaquark candidates $P_c^+(4312)$, $P_c^+(4440)$, and $P_c^+(4457)$ by LHCb in the $J/\psi p$ channel of the $\Lambda^0_b \rightarrow J/\psi pK^-$ decay \cite{LHCb:2015yax,LHCb:2019kea}, it was proposed to search for these states in $\gamma p \to J/\psi p$ where it can be produced directly in a much simpler $2 \to 2$ body kinematics~\cite{Wang:2015jsa,Kubarovsky:2015aaa,Karliner:2015voa,HillerBlin:2016odx}.
  
  
The first measurements of this process at JLab were performed by GlueX~\cite{Ali:2019lzf} and are shown in Fig.~\ref{fig:ccbar-xsec}, with curves depicting the strength of hypothetical $P_c$ signals.  No structures are observed in the measured cross section, however model-dependent upper limits are set on the branching ratio of the possible $P_C \rightarrow J/\psi p$ decays.  Preliminary results from the $J\psi-007$ experiment in Hall C also observe no $P_c$ signal and will set more restrictive limits on the branching ratio~\cite{ghpJpsi007}.

\begin{figure}[ht]
\begin{center}
    \includegraphics[width=0.4\textwidth]{Figures/GlueXJpsi.pdf}
    \includegraphics[width=0.41\textwidth]{Figures/SOLIDpsi2s.png}
    \caption{(left) GlueX results for the $J/\psi $ total cross section vs beam energy, compared to the JPAC model~\cite{HillerBlin:2016odx} with hypothetical branching ratios provided in the legend for $P_c^+$ with $J^P=3/2^-$ as described in Ref.~\cite{Ali:2019lzf}.  (right) Projections for SOLID $\psi(2s)$ total cross section vs beam energy for CEBAF upgrade with $E_e = 17$~GeV.
}
\label{fig:ccbar-xsec}
\end{center}
\end{figure}
   
At higher energies, (quasi-real) photoproduction is especially appealing since many of the $XYZP$ states could be produced directly and observed decaying to relatively simple final states, eliminating some of the kinematical effects. 
Furthermore, one can use the polarization of beam and target to achieve a precise separation of the various production mechanisms, which is not possible, for example, at hadron colliders.
Another advantage is that one can scan different center-of-mass energies by detecting the scattered electron at different angles, while keeping the beam at the nominal energy. This cannot be done by the existing $e^+e^-$ $\tau$-charm factories, where one has to tune carefully the beam energy to do so.

Three candidates stand out in particular: the   $X(3872)$, $Z_c(3900)$ and the $Y(4260)$.  The $X$ state is by far the best known. Its most unusual feature is the strength of isospin violation observed in decays,   $B(X\to J\psi\,\omega)/B(X\to J\psi\,\pi^+\pi^-) = 1.1\pm 0.4$~\cite{Zyla:2020zbs}, which is impossible for ordinary charmonium. 
Furthermore, the  mass of the $X(3872)$ is within a fraction of an MeV from the $\overline{D} {}^0 D^{0*}$ threshold, making it a good candidate for a threshold bound state.   Since the $X$ has sizeable branching fractions to $J\psi \,\rho$ and $J\psi \,\omega$, peripheral photoproduction involving light vector meson exchange can result in  sizable yields. 
The charged $Z_c(3900)^+$ is observed as a resonance in  $J\psi \, \pi^+$,  making it a good candidate for a four-quark state. Finally, there is an overpopulation of hidden-charm vector resonances. Three ordinary $\psi$ states appear in the inclusive $R_D$ measurements, leaving no room for other vectors like the $Y(4260)$. The latter can be produced diffractively.
    
    
The photoproduction cross sections for these states have recently  been estimated to be of the order of a nanobarn for photon energies $E_\gamma \sim 20$--$25$~GeV~\cite{Albaladejo:2020tzt}. 
The yields have been computed using a hypothetical detector setup based on the existing GlueX apparatus at Jefferson Lab~\cite{Adhikari:2020cvz}. Specifically, for a luminosity of $\sim 500\,\text{pb}^{-1}/\text{year}$ and even with a conservative assumption about  efficiency, one expects hundreds of events per year of data taking.
While diffractive production of $Y$ states benefits from higher energies, energies $E_\gamma \sim 20$--$25$~GeV are much more efficient in producing $X$ and $Z$ states. Simulations from the SoLID detector proposal, utilizing a 17 GeV $e-$ beam show excellent precision in determining the photo- and electro-production cross sections of the $\psi'$, as shown in Fig.~\ref{fig:ccbar-xsec} (right).

A photoproduction facility with a 24 GeV CEBAF will provide an opportunity to study these exotic states in exclusive reactions that are complementary to the ones where they have been observed so far.  A spectroscopy program at the forthcoming Electron-Ion Collider is presently under consideration~\cite{AbdulKhalek:2021gbh}. However, it is clear that a machine able to work at lower energies and with higher luminosity would be much more efficient in studying many of the $XYZP$ states.  Such a facility could provide much needed insights into the nature of these intriguing resonances.

