\section{Data sets}
\label{sec:newdata}
\begin{figure}[h]
	\begin{center}
		\includegraphics[width=0.8\columnwidth]{out.png}
		\caption{
			Samples from the data sets contributed by this work.
			Each column belongs to a distinct data set.
			The top row includes non-challenging samples, which can be
			considered relatively similar to laboratory conditions and represent
			only a small fraction of each data set.
			The other two rows include challenging samples with artifacts caused
			by the natural environment.
		}
		\label{fig:newdatasets}
	\end{center}
\end{figure}
In this study, we used the data sets provided by
\citet{fuhl2015excuse,fuhl2015else}, complemented by five additional
new hand-labeled data sets contributed by this work. In total, over 135,000 manually labeled eye images were employed for evaluation. Our data sets introduced with this work include 41,217 images collected during driving sessions
in public roads for an experiment \citet{kasneci2013towards} that were not related to pupil
detection and were chosen due the non-satisfactory performance of the
proprietary pupil detection algorithm.
These new data sets include fast changing and adverse illumination, spectacle
reflections, and disruptive physiological eye characteristics (e.g., dark spot
on the iris); samples from these data sets are shown in
\figref{fig:newdatasets}.
