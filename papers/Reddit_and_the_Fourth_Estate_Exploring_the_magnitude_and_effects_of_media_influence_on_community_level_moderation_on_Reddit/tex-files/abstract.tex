\begin{abstract}

  Most platforms, including Reddit, face a dilemma when applying interventions
  such as subreddit bans to toxic communities --- do they risk angering their user
  base by proactively enforcing stricter controls on discourse or do they defer
  interventions at the risk of eventually triggering negative media reactions
  which might impact their advertising revenue? 
  %
  In this paper, we analyze Reddit's previous administrative interventions to
  understand one aspect of this dilemma: the relationship between the media and
  administrative interventions. More specifically, we make two primary
  contributions. First, using a mediation analysis framework, we find evidence
  that Reddit's interventions for violating their content policy for toxic
  content occur because of media pressure. Second, using interrupted time
  series analysis, we show that media attention on communities with toxic
  content only increases the problematic behavior associated with that
  community (both within the community itself and across the platform).
  However, we find no significant difference in the impact of administrative
  interventions on subreddits with and without media pressure. Taken all
  together, this study provides evidence of a media-driven moderation strategy
  at Reddit and also suggests that such a strategy may not have a significantly
  different impact than a more proactive strategy.
 \end{abstract}
