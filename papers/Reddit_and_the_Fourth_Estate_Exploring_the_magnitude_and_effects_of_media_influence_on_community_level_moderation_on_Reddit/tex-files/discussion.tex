\section{Discussion and conclusions} \label{sec:discussion}

\para{Limitations and challenges.} This work is fundamentally a best-effort
study to understand one aspect of the relationship between platform
administrators and the media, using observational data. Consequently, each of
our contributions has their own limitations. First, given the use of
observational data and our inability to experimentally manipulate media
pressure, we are unable to make strong causal claims regarding the relationship
between media pressure and administrative interventions. This resulted in our
need to frame a weaker hypothesis. We note, however, despite much debate
regarding the use of mediation analyses for making causal inferences, the
approach has been leveraged for precisely this purpose in many prior studies
and one could argue that our models satisfy all the criteria required to make
a causal inference \cite{pearl2014interpretation, Pieters-JCR2017}. Next, our
study required us to develop proxies for several parameters such as subreddit
profitability, topics, and external pressure. It is unclear if our analysis
found no impact from these variables due to the inaccuracy of our proxies or
the actual absence of effects from them. We note that in the absence of
ground-truth, however, one can only make best-effort approximations. Finally,
our study is also limited by our decision to treat the
media attention and interventions applied on each subreddit as independent
events. This might have implications in scenarios where one subreddit receives
negative media attention and this results in the closure of multiple related
communities (\eg Reddit banned five communities  associated with encouraging
self-harm on the same day). However, the alternate decision (grouping all
subreddits receiving an intervention together as a single class) is also
fraught with challenges that arise from the assumption that all
simultaneous interventions occur due to the same effect.

\para{Takeaways and implications.} At a high-level, our study provides evidence
of: (1) a reactionary (media- and internal-) pressure-driven administrative
strategy being leveraged by Reddit, (2) the harms of giving media attention to
toxic communities, and (3) the statistically similar (in)effectiveness of
media- and non-media driven administrative interventions. Each of these
findings has profound implications for platform administrators and media
outlets. 

\parait{Implications for platforms.} As online social platforms increasingly
find their communities becoming the originators and propagators of toxic and
harmful content, calls to regulate them have started emerging. Particularly
relevant is \S230 of the US Communications and Decency Act which grants
complete immunity to online platforms for publishing or censoring speech on
their platforms --- \ie \S230 guarantees no judicial consequences for
moderation and administration decisions. Changes to this regulation have been
proposed by both sides of the American political spectrum and, if enacted, are
expected to have severe implications for moderation strategies employed by
platforms such as Reddit. For example, any change which results in liability
for publishing a users' toxic content will likely render reactionary
administrative strategies, such as the one uncovered in our work, untenable.
Further, although our findings suggest no significant difference in the
effectiveness of interventions driven by media attention and otherwise, they do
provide evidence that reactionary interventions do facilitate an increase in
problematic behavior across the platform. Both these findings suggest the
benefits of investing in and adopting proactive intervention strategies.

\parait{Implications for media outlets.} Our study simultaneously highlights
the importance of and the dilemma faced by the media in platform moderation. On
the one hand, in the presence of reactionary platform administration and the
absence of regulatory demands, it is imperative that the media hold platforms
accountable for their administrative decisions. On the other hand, our findings
also show that shining the media spotlight on problematic communities 
results in the growth and spread of the problematic activity. Thus, it remains
unclear how media outlets should proceed --- must they continue to hold
platforms accountable or should they avoid publicizing problematic communities?
Journalists have faced similar dilemmas in the past while negotiating reporting
on hate crimes, suicides, and school shootings where they are faced with the
consequences of possibly inspiring ``copycat'' behavior. In each such case,
institutions of journalism such as the Society for Professional Journalists,
the Poynter Institute, Thomson Reuters, and others have sought input from
a variety of stakeholders in order to develop guidelines or ``best practices''
for these reports. Our research suggests the need for and value of such
guidelines for reporting toxic online content and communities.


% 
% 
% Our results along with previous works on the effectiveness of moderation
% highlight the flaws in the current moderation strategies of     platforms
% (specifically Reddit). In our first section we demonstrate how Reddit is
% reluctant towards closing communities unless eno    ugh internal or media
% pressure is present. Furthermore, Reddit's administrative decisions towards
% toxicity is dependent upon the pop    ularity of the community as well where
% popular communities do no experience the same treatment (intervention) as
% smaller ones. Moder    ation is one of the key commodities that a social media
% platform offers, however our results show current moderation techniques buil
% t 1) without any legal pressure (Section 230) 2) to intervene and discourage
% profitable behavior 3) by the capitalist social media s    ites where their
% value is driven by number of active users are failing to effectively create an
% open and safe platform. In our resul    ts we focus on the single case study of
% the Incel community which originated and spread from online fringe communities,
% eventually r    esulting in violent outbursts and terrorism. We observe how
% media's effort to pressure Reddit in closing these communities only aide
% d in the normalization and mainstreaming of such ideologies furthermore,
% Reddit's effort to close these communities were futile due     to user
% migration to other subreddits. Our results show the spread of dangerous
% ideologies by media mentions does not significantly     effect the period after
% intervention but we see clear growth of users in the community after media
% attention.
% 

