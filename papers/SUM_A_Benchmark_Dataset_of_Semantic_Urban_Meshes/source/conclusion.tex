\section{Conclusion}\label{sec:Discussion}
We have developed a semi-automatic mesh annotation framework to generate a large-scale semantic urban mesh benchmark dataset covering about 4 $km^2$. 
In particular, we have first used a set of handcrafted features and a random forest classifier to generate the pre-labelled dataset, which saved us around 600 hours of manual labour. 
Then we have developed a mesh labelling tool that allows the users to interactively refining the labels at both the triangle face and the segment levels.
We have further evaluated the current state-of-the-art semantic segmentation methods that can be applied to large-scale urban meshes, and as a result, we have found that our classification based on handcrafted features achieves $93.0\%$ overall accuracy and $66.2\%$ of mIoU.
This outperforms the state-of-the-art machine learning and most deep learning-based methods that use point clouds as input.
Despite this, there is still room for improvement, especially on the issues of imbalanced classes and object scalability. 
For future work, we plan to label more urban meshes of different cities and extend our Helsinki dataset to include parts of urban objects (such as roof, chimney, dormer, and facade). 
We will also investigate smart annotation operators (such as automatic boundary refinement and structure extraction), which involve more user interactivity and may help reduce further the manual labelling task.